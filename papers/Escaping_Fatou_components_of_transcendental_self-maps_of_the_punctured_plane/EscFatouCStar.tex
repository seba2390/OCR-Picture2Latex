\documentclass[a4paper, 12pt, reqno]{amsart}
\usepackage{amsmath}
\usepackage{amssymb}
\usepackage{graphics}
\usepackage{mathrsfs}
\usepackage[pdflatex]{graphicx}
%\usepackage{wrapfig}
%\usepackage{sidecap}
%\usepackage{floatflt}
\usepackage{hyperref}
\usepackage[marginratio=1:1,tmargin=117pt, height=650pt]{geometry}
%\usepackage{fullpage}
\usepackage{color}
\usepackage{amsthm}

\usepackage[all]{xy}

\usepackage{eurosym}

\usepackage[english]{babel}
\usepackage[T1]{fontenc}
\usepackage[latin1]{inputenc}
\selectlanguage{english}

\usepackage{marginnote}

\newcommand{\margin}[1]{\marginnote{\red{#1}}}

\newcommand{\tabitem}{~~\llap{\textbullet}~~}

%FROM DAVE SIXSMITH

\def\ack{\subsubsection*{Acknowledgment.}}
\usepackage{amsmath,amsfonts,amsthm,amssymb,verbatim,enumerate,quotes,graphicx}
\usepackage[flushmargin]{footmisc}
\usepackage[noadjust]{cite}
\usepackage{color}
\newcommand{\comments}[1]{}
\numberwithin{equation}{section}
\makeatletter
\def\blfootnote{\xdef\@thefnmark{}\@footnotetext}
\makeatother






%\usepackage{multienum}

%\usepackage{makeidx}

%\setlength{\textwidth}{17cm}
%\setlength{\oddsidemargin}{-1cm}
%\setlength{\textheight}{22cm}
%\setlength{\topmargin}{-1cm}

%\setlength{\oddsidemargin}{1cm}
%\setlength{\hoffset}{-2cm}
%\setlength{\voffset}{-1cm}
%\setlength{\topmargin}{-1cm}
%\addtolength\headheight{4pt}

\newcommand{\red}[1]{{\color{red} #1}}
\newcommand{\blue}[1]{{\color{blue} #1}}
\newcommand{\green}[1]{{\color{green} #1}}
\definecolor{orange}{rgb}{1,0.5,0}
\newcommand{\orange}[1]{{\color{orange} #1}}
\newcommand{\ds}{\displaystyle}
\newcommand{\interior}{\mbox{int}\,}

\theoremstyle{plain}
\newtheorem{thm}{Theorem}[section]
%\newtheorem*{thmx}[thm]{Theorem}
\newtheorem{prop}[thm]{Proposition}
\newtheorem{cor}[thm]{Corollary}
\newtheorem{cors}[thm]{Corollaries}
\newtheorem{lem}[thm]{Lemma}

\theoremstyle{definition}
\newtheorem{dfn}[thm]{Definition}
\newtheorem{ex}[thm]{Example}
\newtheorem{exs}[thm]{Examples}

\theoremstyle{remark} 
\newtheorem{nota}[thm]{Note}
\newtheorem{rmk}[thm]{Remark}

\renewcommand{\qedsymbol}{$\blacksquare$}


\newcommand{\C}{{\mathbb{C}}}
\newcommand{\CR}{{\hat{\mathbb{C}}}}
\newcommand{\CS}{{\mathbb{C}^*}}
\newcommand{\B}{\mathcal B}
\newcommand{\R}{{\mathbb{R}}}
\newcommand{\Z}{{\mathbb{Z}}}
\newcommand{\N}{{\mathbb{N}}}
\newcommand{\Q}{{\mathbb{Q}}}
\newcommand{\K}{{\mathcal{K}}}
\renewcommand{\L}{\Lambda}
\renewcommand{\Re}{\operatorname{Re}}
\renewcommand{\Im}{\operatorname{Im}}
\newcommand{\dist}{\operatorname{dist}}
\newcommand{\eps}{\varepsilon}
\newcommand{\dbar}{\bar\partial}
\newcommand{\e}{\varepsilon} 
\newcommand{\p}{\partial}
\newcommand{\al}{\alpha}
\newcommand{\z}{\zeta}

\def\mline#1{\par\hspace*{-\leftmargin}\parbox{\textwidth}{\vspace{5pt}\begin{center}\hspace{10pt}$\ds #1$\end{center}\vspace{5pt}}}

\makeatletter
\newcommand{\displaybump}{\hbox to \@totalleftmargin{\hfil}}
\makeatother

\begin{document}


%\bibliographystyle{/usr/local/texlive/2009/texmf-dist/bibtex/bst/amscls/amsalpha-abbrvsort.bst}
\bibliographystyle{amsalpha}

\title[Escaping Fatou components of transcendental self-maps of $\C^*$]{Escaping Fatou components of transcendental self-maps of the punctured plane}
\author[D. Mart\'i-Pete]{David Mart\'i-Pete}
%\address{Department of Mathematics and Statistics\\ The Open University\\ Walton Hall\\ Milton Keynes MK7 6AA\\ United Kingdom}
\address{Department of Mathematics\\ Faculty of Science\\ Kyoto University\\ Kyoto 606-8502\\ Japan}
\email{martipete@math.kyoto-u.ac.jp}
%\email{david.martipete@open.ac.uk}
%\address{Department of Mathematics\\ Faculty of Science\\ Kyoto University\\ Kyoto 606-8502\\ Japan}
%\email{martipete@kyoto-u.ac.jp}
\date{\today}
%\thanks{This research was partially supported by...}
\thanks{This research was supported by The Open University and by the Grant-in-Aid for Scientific Research JP16F16807 from the Japanese Society for the Promotion of Science.}
%\dedicatory{}

\maketitle

\begin{abstract}
We study the iteration of transcendental self-maps of $\C^*:=\C\setminus \{0\}$, that is, holomorphic functions $f:\C^*\to\C^*$ for which both zero and infinity are essential singularities. We use approximation theory to construct functions in this class with escaping Fatou components, both wandering domains and Baker domains, that accumulate to $\{0,\infty\}$ in any possible way under iteration. We also give the first explicit examples of transcendental self-maps of $\C^*$ with Baker domains and with wandering domains. In doing so, we developed a sufficient condition for a function to have a simply connected escaping wandering domain. Finally, we remark that our results also provide new examples of entire functions with escaping Fatou components. 
\end{abstract}


\section{Introduction}

Complex dynamics concerns the iteration of a holomorphic function on a Riemann surface $S$. Given a point $z\in S$, we consider the sequence given by its iterates $f^n(z)=(f\circ\ds\mathop{\cdots}^{n}\circ f)(z)$ and study the possible behaviours as $n$ tends to infinity. We partition $S$ into the \textit{Fatou set}, or stable set,
$$
F(f):=\bigl\{z\in S\ :\ (f^n)_{n\in\mathbb N} \mbox{ is a normal family in some neighbourhood of } z\bigr\}
$$
and the \textit{Julia set} $J(f):=S\setminus  F(f)$, where the chaotic behaviour takes place. We refer to a connected component of $F(f)$ as a \textit{Fatou component} of $f$. If $S\subseteq \CR$, $f:S\rightarrow S$ is holomorphic and $\CR\setminus S$ consists of essential singularities, then conjugating by a M\"obius transformation, we can reduce to one of the following three cases:
\begin{itemize}
\item $S=\CR:=\C\cup\{\infty\}$ and $f$ is a rational map;
\item $S=\C$ and $f$ is a transcendental entire function;
\item $S=\CS:=\C\setminus\{0\}$ and \textit{both} zero and infinity are essential singularities.
\end{itemize}
We study this third class of maps, which we call \textit{transcendental self-maps of} $\CS$. Such maps are all of the form
\begin{equation}
f(z)=z^n\exp\bigl(g(z)+h(1/z)\bigr),
\label{eqn:bhat}
\end{equation}
where $n\in\Z$ and $g,h$ are non-constant entire functions. We define the \textit{index}~of~$f$, denoted by $\textup{ind}(f)$, as the index (or winding number) of $f(\gamma)$ with respect to the origin for any positively oriented simple closed curve $\gamma$ around the origin; note that $\textup{ind}(f)=n$ in \eqref{eqn:bhat}. Transcendental self-maps of $\C^*$ arise in a natural way in many situations, for example, when you complexify circle maps, like the so-called Arnol'd standard family: $f_{\alpha,\beta}(z)=ze^{i\alpha}e^{\beta(z-1/z)/2}$, $0\leqslant \alpha\leqslant 2\pi,\ \beta\geqslant 0$ \cite{fagella99}. Note that if $f$ has three or more omitted points, then, by Picard's theorem, $f$ is constant and, consequently, a non-constant holomorphic function $f:\C^*\rightarrow \C^*$ has no omitted values. The book \cite{milnor06} is a basic reference on the iteration of holomorphic functions in one complex variable. See \cite{bergweiler93} for a survey on transcendental entire and meromorphic functions. Although the iteration of transcendental (entire) functions dates back to the times of Fatou \cite{fatou26}, R{\aa}dstr\"om \cite{radstrom53} was the first to consider the iteration of transcendental self-maps of $\C^*$. An extensive list of references on this topic can be found in \cite{martipete}.

We recall the definition of the \textit{escaping set} of an entire function $f$,
$$
I(f):=\{z\in\C\ :\ f^n(z)\rightarrow \infty \mbox{ as } n\rightarrow \infty\},
$$
whose investigation has provided important insight into the Julia set of entire functions. For polynomials, the escaping set consists of the basin of attraction of infinity and its boundary equals the Julia set. For transcendental entire functions, Eremenko showed that $I(f)\cap J(f)\neq \emptyset$, $J(f)=\partial I(f)$ and the components of $\overline{I(f)}$ are all unbounded \cite{eremenko89}. If $f$ is a transcendental self-map of $\C^*$, then the escaping set of $f$ is given by 
$$
I(f):=\{z\in\CS\ :\ \omega(z,f)\subseteq \{0,\infty\}\},
$$
where $\omega(z,f)$ is the classical omega-limit set $\omega(z,f):=\bigcap_{n\in\N}\overline{\{f^k(z)\ :\ k\geqslant n\}}$ with the closure being taken in $\CR$. In \cite{martipete1}, we studied the basic properties of $I(f)$ for transcendental self-maps of $\C^*$ and introduced the following notion. We define the \textit{essential itinerary} of a point $z\in I(f)$ as the symbol sequence \mbox{$e=(e_n)\in\{0,\infty\}^{\N_0}$} given by
$$
e_n:=\left\{
\begin{array}{ll}
0, & \mbox{ if } |f^n(z)|\leqslant 1,\vspace{5pt}\\
\infty, & \mbox{ if } |f^n(z)|>1,
\end{array}
\right.
$$
for all $n\in \N_0$. Then, for each sequence $e\in\{0,\infty\}^{\N_0}$, we consider the set of points whose essential itinerary is eventually a shift of $e$, that is,
$$
I_e(f):=\{z\in I(f)\ :\ \exists \ell,k\in\N_0,\ \forall n\in\mathbb{N}_0,\ |f^{n+\ell}(z)|>1\Leftrightarrow e_{n+k}=\infty\}.
$$
Observe that if $e_1,e_2\in\{0,\infty\}^{\N_0}$ satisfy $\sigma^m(e_1)=\sigma^n(e_2)$ for some $m,n\in\N_0$, where $\sigma$ is the Bernoulli shift  map (we say that $e_1$ and $e_2$ are \textit{equivalent}), then $I_{e_1}(f)=I_{e_2}(f)$ and, otherwise, the sets $I_{e_1}(f)$ and $I_{e_2}(f)$ are disjoint. Hence, the concept of essential itinerary provides a partition of $I(f)$ into uncountably many non-empty sets of the form $I_e(f)$ for some $e\in\{0,\infty\}^{\mathbb{N}_0}$. In \cite{martipete1}, we also showed that, for each $e\in\{0,\infty\}^{\N_0}$, we have $I_e(f)\cap J(f)\neq \emptyset$, $J(f)=\partial I_e(f)$~and the components of $\overline{I_e(f)}$ are all unbounded in $\C^*$, that is, their closure in $\CR$ contains zero or infinity. We say that $U$ is an \textit{escaping Fatou component} of $f$ if $U$ is a component of $F(f)\cap I(f)$.

As usual, the set of singularities of the inverse function, $\mbox{sing}(f^{-1})$, which consists of the critical values and the finite asymptotic values of $f$, plays an important role in the dynamics of $f$. In \cite{fagella-martipete} we studied the class 
$$
\B^*:=\{f \mbox{ transcendental self-map of } \CS\ :\ \mbox{sing}(f^{-1}) \mbox{ is bounded away from } 0,\infty\},
$$
which is the analogue of the Eremenko-Lyubich class~$\mathcal{B}$ considered in \cite{eremenko-lyubich92}. We proved that if $f\in \B^*$, then $I(f)\subseteq J(f)$ or, in other words, functions in the class~$\mathcal{B}^*$ have no escaping Fatou components. 

In this paper, we are concerned with transcendental self-maps of $\C^*$ that have escaping Fatou components. By normality, if $U$ is a Fatou component of a transcendental self-map $f$ of $\C^*$ and $U\cap I(f)\neq\emptyset$, then $U\subseteq I(f)$. Moreover, note that every pair of points in an escaping Fatou component $U$ have, eventually, the same essential itinerary, and hence we can associate an essential itinerary to $U$, which is unique up to equivalence. We mentioned before that $I_e(f)\cap J(f)\neq \emptyset$ for each sequence $e\in\{0,\infty\}^{\N_0}$ (see \cite[Theorem~1.1]{martipete1}). Therefore it is a natural question whether for each $e\in\{0,\infty\}^{\N_0}$, there exists a transcendental self-map of~$\C^*$ with a Fatou component in $I_e(f)$.

  %We use results from approximation theory to construct such functions. Approximation theory has been used extensively to construct examples of transcendental entire functions with wandering domains and Baker domains; see, for example, Eremenko and Lyubich \cite{eremenko-lyubich87} and Baker \cite{baker88}.\margin{cite BF01?}

For both transcendental entire functions and transcendental self-maps of~$\C^*$, escaping Fatou components can be classified in the following two kinds: let $U$ be a Fatou component of $f$ and denote by $U_n$, $n\in\mathbb{N}$, the Fatou component of $f$ that contains $f^n(U)$, then we say that\vspace{-1pt}
\begin{itemize}
\item $U$ is a \textit{wandering domain} if $U_m\cap U_n=\emptyset$ for all $m,n\in\N$ such that $m\neq n$,\vspace*{1pt}
\item $U$ is a \textit{Baker domain} (or a preimage of it) if $U\subseteq I(f)$ and $U$ is (\textit{pre})\textit{periodic}, that is, $U_{p+m}=U_m$ for some $p\in \mathbb{N}$, the \textit{period} of $U$, and $m=0$ ($m>0$).\vspace{-1pt}
\end{itemize}
%\vspace{-2pt}
Note that not all wandering domains are in $I(f)$. For instance, Bishop~\cite[Theorem~17.1]{bishop15} constructed an entire function in the class $\B$ with a wandering domain whose orbit is unbounded but it does not escape. %In \cite{mihaljevicbrandt-rempegillen13} the authors gave a condition implying that a function in the class $\B$ has no wandering domains.

The first example of a transcendental entire function with a wandering domain was given by Baker \cite{baker63, baker76} and was an infinite product that had a sequence of multiply connected Fatou components escaping to infinity; see \cite{bergweiler-rippon-stallard13} for a detailed study of the properties of such functions. For holomorphic self-maps of $\C^*$, Baker \cite{baker87} showed that all Fatou components, except possibly one, are simply connected, and hence this kind of wandering domains cannot occur. Further examples of simply connected wandering domains of entire functions are due, for example, to Herman \cite[Example~2]{baker84} or Baker \cite[Example~5.3]{baker84}.  

Baker \cite{baker87} also constructed the first holomorphic self-map of $\C^*$ (which is entire) with a wandering domain that escapes to infinity. The first examples of trans\-cendental self-maps of $\C^*$ with a wandering domain are due to Kotus \cite{kotus90}, where the wandering domain accumulates to zero, infinity or both of them. In the same paper, Kotus also constructed an example with an infinite limit set (by adapting the techniques from \cite{eremenko-lyubich92}). Mukhamedshin \cite{mukhamedshin91} used quasiconformal surgery to create a trans\-cendental self-map of~$\C^*$ with a Herman ring and two wandering domains, one escaping to zero and the other one to infinity. Finally, Baker and Dom\'inguez \cite[Theorem~6]{baker-dominguez98} gave an example of a doubly connected wandering domain that is relatively compact in~$\C^*$ and all of whose images are simply connected and escape to infinity. 

In our notation, all the previous examples of wandering domains of transcendental self-maps of $\C^*$ had essential itinerary $e\in\{\overline{\infty},\overline{0}, \overline{\infty 0}\}$, where $\overline{e_1e_2\hdots e_p}$ denotes the $p$-periodic sequence that repeats $e_1e_2\hdots e_p$. The following result provides examples of transcendental self-maps of $\C^*$ that have a wandering domain with any prescribed essential itinerary $e\in\{0,\infty\}^{\N_0}$. 

\begin{thm}
\label{thm:wandering-domains}
For each sequence $e\in\{0,\infty\}^{\N_0}$ and $n\in\Z$, there exists a trans\-cendental self-map $f$ of $\C^*$ such that $\textup{ind}(f)=n$ and the set $I_e(f)$ contains a wandering domain.
\end{thm}

Observe that, in particular, in Theorem~\ref{thm:wandering-domains} we obtain functions with wandering domains whose essential itinerary is not necessarily a periodic sequence. %This theorem is proved by modifying Baker's original construction from \cite{baker87}.

The other type of escaping Fatou component is a Baker domain. The first example of a transcendental entire function with a Baker domain was already given by Fatou \cite{fatou26}: $f(z)=z+1+e^{-z}$. See \cite{rippon08} for a survey on Baker domains. 

A result of Cowen \cite{cowen81} on holomorphic self-maps of $\mathbb D$ whose Denjoy-Woff point lies on $\partial \mathbb D$ led to the following classification of Baker domains by Fagella and Henriksen \cite{fagella-henriksen06}, where $U/f$ is the Riemann surface obtained by identifying points of $U$ that belong to the same orbit under $f$: 
\begin{itemize}
\item a Baker domain $U$ is \textit{hyperbolic} if $U/f$ is conformally equivalent to $\{z\in\C\ :$\linebreak $-s<\Im z<s\}/\Z$ for some $s>0$;
\item a Baker domain $U$ is \textit{simply parabolic} if $U/f$ is conformally equivalent to $\{z\in\C\ :\ \Im z>0\}/\Z$;
\item a Baker domain $U$ is \textit{doubly parabolic} if $U/f$ is conformally equivalent to $\C/\Z$.
\end{itemize}
Note that this classification does not require $f$ to be entire and is valid also for Baker domains of transcendental self-maps of $\C^*$. K\"onig \cite{koenig99} provided a geometric characterisation for each of these types (see Lemma~\ref{lem:bd-koenig}). It is known that if $U$ is a doubly parabolic Baker domain, then $f_{|U}$ is not univalent, but if $U$ is a hyperbolic or simply parabolic Baker domain, then $f_{|U}$ can be either univalent or multivalent. Several examples of each type had been constructed and recently Bergweiler and Zheng completed the table of examples by constructing a transcendental entire function with a simply parabolic Baker domain in which the function is not univalent \cite[Theorem~1.1]{bergweiler-zheng12}.

The only previous examples of Baker domains of transcendental self-maps of~$\C^*$ that the author is aware of are due to Kotus \cite{kotus90}. She used approximation \mbox{theory} to construct two functions with invariant hyperbolic Baker domains whose points escape to zero and to infinity respectively. The following theorem provides functions with Baker domains that have any periodic essential itinerary $e\in\{0,\infty\}^{\N_0}$ and, in particular, Baker domains whose points accumulate to both zero and infinity.

\begin{thm}
\label{thm:baker-domains}
For each periodic sequence $e\in\{0,\infty\}^{\N_0}$ and $n\in\Z$, there exists a trans\-cendental self-map $f$ of $\C^*$ such that $\textup{ind}(f)=n$ and $I_e(f)$ contains a hyperbolic Baker domain. 
\end{thm}

\begin{rmk}
We observe that our method can be modified to produce doubly parabolic Bakers domains as well. However, the cons\-truction of simply parabolic Baker domains using approximation theory seems more difficult.
\end{rmk}

%\red{Check univalent/multivalent... can we put $z^n$ in front in all cases? Note that the functions in Example 1 have multivalent hyperbolic Baker domains. So it still remains to show that...}

We also give the first explicit examples of transcendental self-maps of $\C^*$ with wandering domains and Baker domains. They all have the property that in a neighbourhood of infinity they behave like known examples of transcendental entire functions with wandering domains and Baker domains.

\begin{ex}
\label{ex:main-ex}
The following transcendental self-maps of $\C^*$ have escaping Fatou components:
\begin{enumerate}
\item[(i)] The function $f(z)=z\exp\left(\frac{\sin z}{z}+\frac{2\pi}{z}\right)$ has a bounded wandering domain that escapes to infinity (see Example~\ref{ex:wand-domain}).
%\item[(ii)] The function $f(z)=z\exp\bigl(\frac{e^{-z}-1}{z}+\frac{2\pi i}{z}\bigr)$ has an unbounded wandering domain that escapes to infinity (see Example~\ref{ex:wand-domain-unbdd}).
\item[(ii)] The function $f(z)=2z\exp\bigl(\exp(-z)+1/z\bigr)$ has an invariant hyperbolic Baker domain that contains a right half-plane and whose points escape to infinity (see Example~\ref{ex:hyp-baker-domain}).
\item[(iii)] The function $f(z)=z\exp\left((e^{-z}+1)/z\right)$ has an invariant doubly parabolic Baker domain that contains a right half-plane and whose point escape to infinity (see Example~\ref{ex:dpar-baker-domain}).
\end{enumerate}
\end{ex} 

It seems hard to find explicit examples of functions with Baker domains and wandering domains with any given essential itinerary, but it would be interesting to have a concrete example of a function with an escaping Fatou component that accumulates to both zero and infinity. It also seems difficult to find explicit examples of functions with simply parabolic Baker domains.

We remark that in order to show that the function from Example~\ref{ex:main-ex}~(i) has a simply connected escaping wandering domain we introduced a new criterion (see Lemma~\ref{lem:Julia-in-annulus}) which is of more general interest. 

Let $f$ be a transcendental self-map of $\C^*$, then there exists a transcendental entire function $\tilde{f}$ such that $\exp \circ \,\tilde{f}=f\circ \exp$; we call $\tilde{f}$ a \textit{lift} of $f$. If the function $f$ has a wandering domain, then $\tilde{f}$ has a wandering domain, while if $f$ has a Baker domain, then $\tilde{f}$ has either a Baker domain (of the same type) or a wandering domain; see Lemmas~\ref{lem:semiconj-wd}~and~\ref{lem:semiconj-bd}. 

It is important that in both Theorems~\ref{thm:wandering-domains}~and~\ref{thm:baker-domains} we can choose the index of the function since, for example, if $\textup{ind}(f)\neq 1$, then $f$ does not have Herman rings. In \cite{martipete4} the author compares the escaping set of $f$ with that of a lift $\tilde{f}$ of $f$ according to $\textup{ind}(f)$.

Finally, observe that our constructions using approximation theory can also produce holomorphic self-maps of $\C^*$ of the form $f(z)=z^n\exp(g(z))$, with $n\in\Z$ and $g$ a non-constant entire function. In particular, they can provide new examples of transcendental entire functions with no zeros in $\C^*$ that have wandering domains and Baker domains.

%Bergweiler and Hinkkanen introduced the fast escaping set of a transcendental entire function in connection with the study of permutable functions \cite{bergweiler-hinkkanen99}; see \cite{rippon-stallard12} for an introduction to the fast escaping set. Rippon and Stallard \red{[...]} showed that Baker domains are never fast escaping. \red{Same here?} 

%\margin{remove?}For entire functions, multiply connected Fatou components are always fast escaping wandering domains \red{[...]}. There are two examples of fast escaping simply-connected Fatou components by Bergweiler \cite{bergweiler11} and Sixsmith \cite{sixsmith12}. In \cite{martipete} we introduced the fast escaping set of transcendental self-maps of $\C^*$. It is an open question whether there exist fast escaping Fatou components for such maps. Note that if $U$ is a wandering domain of a transcendental self-maps of $\C^*$, then all the iterates of $U$ are simply connected with at most one exception.

%\margin{remove?}Finally, we remark that points in the boundaries of escaping Fatou components may not escape. Actually, it was an open question whether escaping Fatou components must have escaping points in their boundaries. See \cite{rippon-stallard11} and the recent works \cite{rippon-stallard} and \cite{bfjk} where the authors show that \red{explain...}

%\cite{rippon06}, \cite{rempe-rippon12}, \cite{rippon-stallard06},  \cite{fleischmann08},  \cite{bfjk15}, \cite{bergweiler95ii}\\

%\red{families of functions with Baker domains:} \cite{rippon-stallard99i}, \cite{rippon-stallard99ii}, \cite{lauber07}, \cite{fleischmann09},

\vspace{5pt}

\noindent 
\textbf{Structure of the paper.} In Sections 2 and 3 we prove that the functions from Example~\ref{ex:main-ex} have the properties that we state. In Section 4 we introduce the tools from approximation theory that we will use in Sections 5 and 6 to construct functions with escaping wandering domains and Baker domains respectively. Theorem~\ref{thm:wandering-domains} is proved in Section 5 and Theorem~\ref{thm:baker-domains} is proved in Section 6.

\vspace{5pt}

\noindent
\textbf{Notation.} In this paper $\N_0=\N\cup\{0\}=\{0,1,2,\hdots\}$ and, for $z_0\in \C$ and $r>0$, we define
%$$
%A(r_1,r_2):=\{z\in \mathbb C\ :\ r_1<|z|<r_2\}  ~\mbox{ and }~ \overline{A}(r_1,r_2):=\{z\in \mathbb C\ :\ r_1\leqslant |z|\leqslant r_2\},
%$$
$$
D(z_0,r):=\{z\in\C\ :\ |z-z_0|<r\},\quad \mathbb H_r:=\{z\in\C\ :\ \Re z>r\}. 
$$


\vspace{10pt}

\noindent
\textbf{Acknowledgments.} The author would like to thank his supervisors Phil Rippon and Gwyneth Stallard for their support and guidance in the preparation of this paper.

%%%\section{Explicit functions with wandering domains}
%%%
%%%As mentioned before, the author is not aware of any previous explicit examples of transcendental self-maps of $\C^*$ with wandering domains nor Baker domains as all such functions were constructed using approximation theory.
%%%
%%%%\begin{dfn}[Wandering domain]
%%%%Let $f$ be a transcendental self-map of $\C^*$. Let $U$ be a Fatou component of $f$ and let $U_n$, $n\in\N$, denote the Fatou component containing $f^n(U)$. We say that $U$ is a \textit{wandering domain} if, for all $m,n\in\N$, $U_m=U_n$ if and only if $m=n$.
%%%%\end{dfn}
%%%
%%%Kotus \cite{kotus90} showed that transcendental self-maps of $\C^*$ can have escaping wandering domains by constructing examples of such functions using approximation theory. Here we give an explicit example of such a function by modifying a transcendental entire function that has a wandering domain. 
%%%
%%%\begin{ex}
%%%The function $f(z)=z\exp\bigl(\frac{\sin z}{z}+\frac{2\pi}{z}\bigr)$ is a transcendental self-map of~$\C^*$ which has a bounded wandering domain that escapes to infinity (see Figure~\ref{fig:wand-domain}).
%%%\label{ex:wand-domain}
%%%\end{ex}
%%%
%%%\begin{figure}[h!]
%%%%\includegraphics[width=.49\linewidth]{wd001-9-small.png}
%%%\includegraphics[width=.49\linewidth]{002-dot.png}
%%%\includegraphics[width=.49\linewidth]{wd001-12-2.png}
%%%\caption{Phase space of the function $f(z)=z\exp\left(\frac{\sin z}{z}+\frac{2\pi}{z}\right)$ from Example~\ref{ex:wand-domain}. On the right, the wandering domain for large values of $\Re z$.}
%%%\label{fig:wand-domain}
%%%\end{figure}
%%%
%%%Baker \cite[Example~5.3]{baker84} (see also \cite[Example~2]{rippon-stallard08}) studied the dynamics of the trans\-cendental entire function $f_1(z)=z+\sin z+2\pi$ that has a wandering domain containing the origin that escapes to infinity. Observe that the function $f$ from Example~\ref{ex:wand-domain} satisfies that%Devaney \cite{devaney89}
%%%\begin{equation}
%%%f(z)=z+\sin z+2\pi+o(1)\quad \mbox{ as } \mbox{Re}\,z\rightarrow +\infty
%%%\label{eq:ex-wand-domain}
%%%\end{equation}
%%%in a horizontal band defined by $|\Im z|<K$ for some $K>0$.
%%%
%%%We first prove a result that implies that a function has a bounded wandering domain. This is a generalisation of \cite[Lemma~7(c)]{rippon-stallard08}. Given a doubly connected set $X$, we define the inner boundary, $\partial_\textup{in}X$, and the outer boundary, $\partial_\textup{out}X$, of $X$ to be the boundary of the bounded and unbounded complementary components of $X$ respectively.
%%%
%%%\begin{lem}
%%%%Let $f$ be a function that is holomorphic on a domain $G\subseteq \C$, let $M$ be an orientation-preserving isometry of $\C$ and let $A\subseteq G$ be a doubly connected closed set such that 
%%%Let $f$ be a function that is holomorphic on a domain $G\subseteq \C$, let $M$ be an affine map and let $A\subseteq G$ be a doubly connected closed set such that 
%%%$$
%%%A_n:=M^n(A)\subseteq G\quad \mbox{ for all } n\in\N
%%%$$
%%%and the closures of the bounded complementary components of the sets $\{A_n\}_{n\in\N}$ are pairwise disjoint. Suppose that, for all $n\in\N$,
%%%\begin{itemize}
%%%\item $f(\partial_\textup{in}A_n)$ lies in the bounded complementary component of $A_{n+1}$;
%%%\item $f(\partial_\textup{out}A_n)$ lies in the unbounded complementary component of $A_{n+1}$.
%%%\end{itemize}
%%%Then $f$ has wandering domains $\{U_n\}_{n\in\N}$ such that $\partial_\textup{in}A_n\subseteq U_n$ and $\partial U_n\subseteq A_n$ for all $n\in\N$.
%%%\label{lem:Julia-in-annulus}
%%%\end{lem}
%%%%\begin{lem}
%%%%Let $f$ be a function that is holomorphic on a domain $G\subseteq \C$ and let $\{A_n\}_{n\in\N}$ be a sequence of doubly connected closed sets in $G$ such that the closures of their bounded complementary components are pairwise disjoint and there exist constants $C_1,C_2>0$ such that 
%%%%$$
%%%%\textup{diam}(\partial_\textup{in}A_n)<C_1 \quad \mbox{ and } \quad \textup{dist}(\partial_\textup{in}A_n,\partial_\textup{out}A_n)>C_2\quad \mbox{ for all } n\in\N.
%%%%$$
%%%%Suppose that, for all $n\in\N$,
%%%%\begin{itemize}
%%%%\item $f(\partial_\textup{in}A_n)$ lies in the bounded complementary component of $A_{n+1}$;
%%%%\item $f(\partial_\textup{out}A_n)$ lies in the unbounded complementary component of $A_{n+1}$.
%%%%\end{itemize}
%%%%Then $f$ has wandering domains $\{U_n\}_{n\in\N}$ such that $\partial_\textup{in}A_n\subseteq U_n$ and $\partial U_n\subseteq A_n$ for all $n\in\N$.
%%%%\label{lem:Julia-in-annulus}
%%%%\end{lem}
%%%\begin{proof}
%%%Let $B_n$, $n\in\N$, denote the bounded complementary component of $A_n$. Since $f(B_n)\subseteq B_{n+1}$, the iterates of $f$ on each domain $B_n$ omit more than three points and hence, by Montel's theorem, they are contained in $F(f)$. Let $U_n$ denote the Fatou component of $f$ that contains $B_n$.
%%%
%%%We now show that the Fatou components $U_n$, $n\in\N$, are all different. Suppose to the contrary that two sets $B_m$ and $B_{m+p}$ with $m\in\N$ and $p>0$ are contained in the same Fatou component $U_m=U_{m+p}$. Then, since $f^p(B_m)\subseteq B_{m+p}$ and $B_m\to\infty$ as $m\to\infty$, $U_m$ has to be a Baker domain. Let $(V,\phi,T)$ be a conformal conjugacy on $U_m$. Observe that the sets $B_n$ are contained in $V$ for large values of $n\in\N$. By Lemma~\ref{lem:bd-koenig}, the map $T$ is either $T_1(z)=\lambda z$ with $\lambda>1$, $T_2(z)=z\pm i$ or $T_3(z)=z+1$. But the functions $T_k$ all map part of a circle outside itself and hence this contradicts the fact that $f^p(\partial B_m)\subseteq B_{m+p}$. Therefore the Fatou components $U_m$, $m\in\N$, are pairwise disjoint and hence are wandering domains.
%%%
%%%In order to show that $\partial U_n\subseteq A_n$ for all $n\in\N$, let $\mathcal F$ be the family of functions given by
%%%$$
%%%\phi_{n}(z):=M^{-n}f^n(z)\quad  \mbox{ for } n\in\N.
%%%$$
%%%The family $\mathcal F$ is normal in $U_m$ for all $m\in\N$ because the restriction of every function $\phi_{n}$ to $U_m$ avoids the points in $M^n(A_m)$ for all $n>0$. Indeed, this follows from the fact that the Fatou components $U_n$, $n\in\N$, are pairwise disjoint and $M^n(B_m)=B_{m+n}\subseteq U_{m+n}$. Suppose now that there is a point $z\in\partial_\textup{out}A_m$ that lies in $U_m$. Let $\gamma$ be a curve that joins $z$ to a point $w\in B_m$. Since $\gamma$ is compact and $\mathcal F$ is a normal family in $U_m$, there exists a subsequence of $\mathcal F$ that converges to a function on $\gamma$ with values in the interior of $B_m$. But $\phi_{n}(z)$ lies outside $A_m$ for all $n\in\N$. Therefore $\partial_\textup{out}A_m\cap U_m=\emptyset$ and $\partial U_m\subseteq A_m$. 
%%%\end{proof}
%%%
%%%We now show that, indeed, the function $f$ from Example~\ref{ex:wand-domain} has a bounded wandering domain that escapes to infinity along the positive real axis.
%%%
%%%\begin{proof}[Proof of Example~\ref{ex:wand-domain}]
%%%%%First of all, observe that since $f(\overline{z})=\overline{f(z)}$, the Fatou set $F(f)$ is symmetric with respect to the real axis. %Moreover, the function $g(z)=z+\sin z$ satis\-fies that $g(z+2\pi)=g(z)+2\pi$ and hence $F(g)$ is $2\pi$-periodic. 
%%% 
%%%%Each point $z_k$ has a bounded basin of attraction $U_k$ under $g$ that is contained in the vertical band $V_k:=\{z\in\C\ :\ z_k-\pi<\Re z<z_k+\pi\}$ and is a~$(2k+1)\pi$-translate of $U_0$. Therefore, we have $\hat{f}^n(U_0)=U_n\subseteq V_n$ and $U=U_0$ is a wandering domain of $\hat{f}$. 
%%%
%%%%%The function $g(z)=z+\sin z$ has fixed points at $z=n\pi$, $n\in\Z$, which are superattracting, if $n$ is odd, and repelling, if $n$ is even. Thus, the function $f(z)-2\pi$ has infinitely many fixed points $z_n$ in the positive real axis, which are the solutions of the equation
%%%%%$$
%%%%%\exp\left(\frac{\sin x}{x}+\frac{2\pi}{x}\right)=1+\frac{2\pi}{x} \quad \mbox{ for } x>0
%%%%%$$
%%%%%or, equivalently, $\sin x=o(1)$ as $x\to+\infty$. Hence the points $z_n$ are asymptotically close to $n\pi$ as $n\to \infty$ and are attracting, if $n$ is odd, or repelling, if $n$ is even.
%%%
%%%The function $g(z)=z+\sin z$ has superattracting fixed points at the even multiples of $\pi$. For every $n\in\N$, take $B_n:=D(2n\pi,r)$ for some $r>0$ sufficiently small that $g(B_n)\subseteq B_{n}$ and put
%%%$$
%%%R_n:= \{z\in \C\ :\ |\Re z-2n\pi|\leqslant 3\pi/2,\ |\Im z|\leqslant 3\}.
%%%$$
%%%It follows from a straightforward computation that $g(\partial R_n)\subseteq R_n^c$ (see Figure~\ref{fig:spiral}). 
%%%
%%%\begin{figure}[h!]
%%%\centering
%%%\includegraphics[width=.6\linewidth]{spiral-final.png} 
%%%\caption{Rectangle $R_0$ and its image under $g(z)=z+\sin z$.}
%%%\label{fig:spiral}
%%%\end{figure}
%%%
%%%Then, by \eqref{ex:wand-domain}, there exists $N\in\N$ such that $f(B_n)\subseteq B_{n+1}$ and $f(\partial R_n)\subseteq R_{n+1}^c$ for all $n>N$. Thus, we can apply Lemma~\ref{lem:Julia-in-annulus} to $f$ with $M(z)=z+2\pi$ and $A_n:=R_n\setminus B_n$ for $n>N$ and conclude that the function $f$ has wandering domains $U_n$ that contain $B_n$ and whose boundary is contained in $R_n$.
%%%\end{proof}
%%%%%Let $U_k$ be the immediate basin of attraction of the point $z_{2k+1}$ under~$f-2\pi$ for $k\in\Z$ sufficiently large. By \eqref{eq:ex-wand-domain}, for every $\varepsilon>0$, there exists $k_0=k_0(\varepsilon)\in\Z$ such that 
%%%%%$$
%%%%%|f'(z_{2k+1})|<\varepsilon \quad \mbox{ and } \quad  |z_{2k+3}-(z_{2k+1}+2\pi)|<\varepsilon \quad \mbox{ for all } k>k_0.
%%%%%$$ 
%%%%%Let $r>0$ be sufficiently small, then choosing $\varepsilon=\varepsilon(r):=r/(2+2r)$ we have that
%%%%%$$
%%%%%|z_{2k+3}-(z_{2k+1}+2\pi)|+|f'(z_{2k+1})|r<\varepsilon+\varepsilon r=r/2 \quad \mbox{ for all } k>k_0(\varepsilon)
%%%%%$$
%%%%%and hence
%%%%%$$
%%%%%f\bigl(D(z_{2k+1},r)\bigr)\sim D(z_{2k+1},\varepsilon r)+2\pi\subseteq D(z_{2k+3},r/2) \quad \mbox{ for all } k>k_0(\varepsilon).
%%%%%$$
%%%%%Thus, the union of all the discs $D_k:=D(z_{2k+1},r)\subseteq U_k$, for $k>k_0(\varepsilon)$, is contained in $F(f)\cap I(f)$.
%%%
%%%%%To prove that each $D_k$ is contained in a wandering domain, we must show that the Fatou components $U_k$ are pairwise disjoint for sufficiently large values of $k\in\Z$. Suppose to the contrary that $U_p=U_q$ for some $p\neq q$. Then there is a Jordan curve $\gamma\subseteq U_p$ which is symmetric with respect to the real axis and such that $\gamma$ intersects $D_p$ and~$D_q$. Such curve would surround a repelling fixed point $z_{2j}$ for some $j\in\Z$ and hence $U_k$ would be doubly connected. However, Baker \cite[Theorem~1]{baker87} showed that the only multiply connected Fatou components in $\C^*$ are doubly connected and must separate zero from infinity. Therefore each $U_n$, $n\in\Z$, is a wandering domain whose iterates escape along the positive real axis.  
%%%
%%%
%%%%Then, for all $n\in\N$, the bounded complementary component defined by the Jordan curve $f^n(\gamma)$ contains a repelling fixed point of~$f$ that must lie in the real axis by symmetry. But since repelling periodic points are in $J(f)$, this implies that $U_p$ is multiply connected, which contradicts the fact that $f$ can have at most one multiply connected Fatou component which must be doubly connected and separate zero from infinity . 
%%%
%%%
%%%
%%%%%\noindent
%%%%%Therefore, by  \eqref{eq:ex-wand-domain}, we have $f(\gamma_k)-2\pi\subseteq\C\setminus R_k$ and hence $\gamma_k\subseteq \C\setminus U_k$. Thus the sets $U_k$ are bounded for sufficiently large values of $k$.
%%%%%\end{proof}
%%%% the symmetry of $f$. Suppose to the contrary there exist some $n_0\in\N$ such that $U_{n_0}$ is unbounded. Since $f(\overline{z})=\overline{f(z)}$, the Fatou components $U_n$ are symmetric with respect to the real axis. Therefore the set $U_{n_0}$ must contain a curve $\delta$ that is also symmetric with respect to the real axis and splits the plane into two connected components, and the curve $f(\delta)$ must also separate the plane into two components. Thus, if $V$ is the vertical band such that $\partial V=\delta\cup f(\delta)$, the images $f^n(V)$ will all be vertical bands intersecting $U_{n_0+n}$ and $U_{n_0+n+1}$, but this is a contradiction with the blow up property of $J(f)$.
%%%
%%%
%%%
%%%%%%We now provide a second example of a transcendental self-map of $\C^*$ with a wandering domain that, in this case, is unbounded.
%%%%%%
%%%%%%\begin{ex}
%%%%%%The function $f(z)=z\exp\bigl(\frac{e^{-z}-1}{z}+\frac{2\pi i}{z}\bigr)$ is a transcendental self-map of $\C^*$ which has an unbounded wandering domain that escapes to infinity (see Figure~\ref{fig:unbdd-wand-domain}).
%%%%%%\label{ex:wand-domain-unbdd}
%%%%%%\end{ex}
%%%%%%
%%%%%%\begin{figure}[h!]
%%%%%%\includegraphics[width=.49\linewidth]{c05.png}
%%%%%%%\includegraphics[width=.49\linewidth]{b02.png}
%%%%%%\includegraphics[width=.49\linewidth]{c07.png}
%%%%%%\caption{Phase space of the function $f(z)=z\exp\bigl(\frac{e^{-z}-1}{z}+\frac{2\pi i}{z}\bigr)$ from Example~\ref{ex:wand-domain-unbdd}. In the right, the wandering domain for large values of $\Im z$.}
%%%%%%\label{fig:unbdd-wand-domain}
%%%%%%\end{figure}
%%%%%%
%%%%%%The trans\-cendental entire function $f(z)=z-1+e^{-z}+2\pi i$ was studied by Herman (see \cite[Examples~2~and~5.1]{baker84}) who showed that $f$ has an unbounded wandering domain containing the real axis which escapes to infinity along the positive imaginary axis. Herman constructed this function from the lift by $e^{-z}$ of the function $g(z)=eze^{-z}$ which is a holomorphic self-map of $\C^*$ that is entire. Such lift is a $2\pi i$-periodic entire function which has a sequence of unbounded basins of attraction that become a wandering domain when you add $2\pi i$ to the function. Observe that the function $f$ from Example~\ref{ex:wand-domain-unbdd} satisfies that%Devaney \cite{devaney89}
%%%%%%\begin{equation}
%%%%%%f(z)=z-1+e^{-z}+2\pi i+o(1)\quad \mbox{ as } \mbox{Re}\,z\rightarrow +\infty
%%%%%%\label{eq:ex-wand-domain}
%%%%%%\end{equation}
%%%%%%in a right half plane.
%%%%%%
%%%%%%\begin{proof}[Proof of Example~\ref{ex:wand-domain-unbdd}]
%%%%%%Similar arguments to those in the proof of Example~\ref{ex:wand-domain} show that the function $f-2\pi i$ has a sequence of attracting fixed points $z_n$, $n\in\N$, near the imaginary axis that are asymptotically close to $z=2n\pi i$ as $n\to \infty$. There exists $r>0$ such that each point $z_n$ is contained in the disk $B_n:=D(z_n,r)$ and $f(B_n)\subseteq B_{n+1}$ for all sufficiently large values of $n\in\N$. Hence the union of all $D_k$ is in $F(f)$. Moreover, each disc $D_k$ is in a different Fatou component of $f$, and are thus part of a wandering domain~$U$.
%%%%%%
%%%%%%To prove that the wandering domain $U$ is unbounded, we observe that, for sufficiently large values of $k\in\Z$, if we define the band
%%%%%%$$
%%%%%%B_k:=\{z\in\C\ :\ \Re z>0,\ |\Im z -2k\pi|<\varepsilon\}%t+2k\pi i, \quad t\in [0,+\infty),
%%%%%%$$
%%%%%%then the set $D_k\cup B_k$ is connected and it is contained in the basin of attraction of $z_k$. Indeed, if $z=x+(2k\pi \pm\varepsilon)i$ with $x>0$, $k\in\Z$ and $\varepsilon>0$ sufficiently small, then 
%%%%%%$$
%%%%%%\Im (f(z))-2\pi=(2k\pi\pm\varepsilon)+e^{-x}\sin(-2k\pi\mp\varepsilon)+o(1) \quad \mbox{ as } k\to+\infty,
%%%%%%$$
%%%%%%where $e^{-x}\sin(-2k\pi-\varepsilon)<0$ and $e^{-x}\sin(-2k\pi+\varepsilon)>0$. Hence, for a given value of $\varepsilon$, $f(D_k\cup B_k)-2\pi\subseteq D_k\cup B_k$ for sufficiently large values of $k\in\Z$, and therefore the wandering domain $U$ is unbounded.
%%%%%%\end{proof}
%%%
%%%If $f$ is a transcendental self-map of $\C^*$, then there exists an entire function $\tilde{f}$ that is semiconjugated to $f$ by the exponential function, that is, $\exp \circ \tilde{f}=f\circ \exp$. We call $\tilde{f}$ a \textit{lift} of $f$. The next lemma relates the wandering domains of $f$ and $\tilde{f}$.
%%%
%%%\begin{lem}
%%%Let $f$ be a transcendental self-map of $\C^*$ and let $\tilde{f}$ be a lift of $f$. Then,\linebreak if $U$ is a wandering domain of $f$, every component of $\exp^{-1}(U)$ is a wandering domain of $\tilde{f}$ which must be simply connected.
%%%\label{lem:semiconj-wd}
%%%\end{lem}
%%%\begin{proof}
%%%By a result of Bergweiler \cite{bergweiler95}, every component of $\exp^{-1}(U)$ is a Fatou component of $\tilde{f}$. Let $U_0$ be a component of $\exp^{-1}(U)$ and suppose to the contrary that there exist $m,n\in\N$, $m\neq n$, and a point $z_0\in\tilde{f}^m(U_0)\cap \tilde{f}^n(U_0)$. Then, there exists points $z_1,z_2\in U_0$ such that 
%%%$$
%%%f^m(e^{z_1})=\exp \tilde{f}^m(z_1)=\exp z_0=\exp \tilde{f}^n(z_2)=f^n(e^{z_2}).
%%%$$
%%%Since $e^{z_1},e^{z_2}\in U$, this constradicts the assumption that $U$ is a wandering domain of $f$. Hence $U_0$ is a wandering domain of $\tilde{f}$. 
%%%
%%%Finally, by \cite[Theorem~1]{baker87}, the Fatou component $U$ is either simply connected or doubly connected and surrounds the origin, and hence the components of $\exp^{-1}(U)$ are simply connected.
%%%% Suppose to the contrary that a component $\tilde{U}$ of $\exp^{-1}(U)$ is multiply connected. Then, by \cite[Theorem~3.1]{baker84}, all the images $\tilde{f}^n(\tilde{U}),\ n\in\N$, are multiply connected and escape to infinity. Baker also showed that for large values of $n$, the components $\tilde{f}^n(\tilde{U})$ surround the origin.
%%%%  Since each $\tilde{f}^n(\tilde{U})$ is bounded, the sets $\exp(\tilde{f}^n(\tilde{U}))$ surround zero for all $n\in\N$. By \cite[Theorem~1]{baker87}, the set $F(f)$ can only have one doubly connected component, so all the sets $\tilde{f}^n(\tilde{U})$ are mapped to the same Fatou component of $f$ by the exponential function. But this contradicts \cite{bergweiler95} because the Julia set in between the sets $\tilde{f}^n(\tilde{U})$ would be mapped to $F(f)$.
%%%\end{proof}
%%%
%%%\begin{rmk}
%%%Observe that the converse of Lemma~\ref{lem:semiconj-wd} does not hold. If $f$ is a transcendental self-map of $\C^*$ with an attracting fixed point $p$ and $A$ is the immediate basin of attraction of $p$, then there is a lift $\tilde{f}$ of $f$ such that a component of $\exp^{-1}(A)$ is a wandering domain.
%%%\end{rmk}
%%%
%%%If a transcendental self-map of $\C^*$ has an escaping wandering domain, then we can use the previous lemma to obtain automatically an example of a transcendental entire function with an escaping wandering domain.
%%%
%%%\begin{ex}
%%%The transcendental entire functions $\tilde{f}(z)=z+\frac{\sin e^z}{e^z}+\frac{2\pi}{e^z}$, which is a lift of the function $f$ from Example~\ref{ex:wand-domain}, has infinitely many grand orbits of bounded wandering domains that escape to infinity.
%%%\end{ex}
%%%
%%%%%%\begin{ex}
%%%%%%The transcendental entire function $\tilde{f}(z)=z+\frac{e^{-z}}{z}+\frac{2\pi i}{z}$, which is a lift of the function $f$ from Example~\ref{ex:wand-domain-unbdd}, has infinitely many grand orbits of unbounded wandering domains whose points escape to infinity.
%%%%%%\end{ex}
%%%
%%% 
%%%%\begin{center}
%%%%\begin{tabular}{cc}
%%%%%\includegraphics[width=130pt]{standard.png} & \includegraphics[width=130pt]{standard2.png}\vspace{2pt}\\
%%%%\includegraphics[width=140pt]{wd001-9-small.png} & \includegraphics[width=140pt]{wd001-12.png}\vspace{2pt}\\
%%%%%$\alpha=3.1,\ \beta=0.8.$ & $\alpha=3.1,\ \beta=5$.\\
%%%%\end{tabular}
%%%%\end{center}
%%%%\end{frame}
%%%
%%%\section{Explicit functions with Baker domains}
%%%
%%%%\begin{dfn}[Baker domain]
%%%%Let $f$ be a transcendental self-map of $\C^*$. We say that a Fatou component $U$ of $f$ is a \textit{Baker domain} if $U$ is \textit{periodic}, that is, there exists $p\geqslant 1$ such that $f^p(U)=U$, and $U\subseteq I(f)$.
%%%%\end{dfn}
%%%
%%%We now turn our attention to Baker domains. As we mentioned in the introduction, Baker domains can be classified into hyperbolic, simply parabolic and doubly parabolic according to the Riemann surface $U/f$ obtained by identifying the points of the Baker domain $U$ that belong to the same orbit under iteration by the function $f$. K\"onig \cite{koenig99} introduced the following notation.
%%%
%%%\begin{dfn}[Conformal conjugacy]
%%%Let $U$ be a domain and let $f:U\to U$ be analytic. Then a domain $V\subseteq U$ is \textit{absorbing} (or \textit{fundamental}) for $f$ if $V$ is simply connected, $f(V)\subseteq V$ and for each compact set $K\subseteq U$, there exists $N=N_K$ such that $f^N(K)\subseteq V$.
%%%Let $\mathbb H:=\{z\in\C\ :\ \Re z>0\}$. The triple $(V,\phi,T)$ is called a \textit{conformal conjugacy} (or \textit{eventual conjugacy}) of $f$ in $U$ if
%%%\begin{enumerate}
%%%\item[(a)] $V$ is absorbing for $f$;
%%%\item[(b)] $\phi:U\to \Omega\in\{\mathbb H, \C\}$ is analytic and univalent in $V$;
%%%\item[(c)] $T:\Omega\to\Omega$ is a bijection and $\phi(V)$ is absorbing for $T$;
%%%\item[(d)] $\phi(f(z))=T(\phi(z))$ for $z\in U$.
%%%\end{enumerate}
%%%In this situation we write $f\sim T$.
%%%\end{dfn}
%%%
%%%Observe that properties (b) and (d) imply that $f$ is univalent in $V$. K\"onig also provided the following geometrical characterization of the three types of Baker domains \cite[Theorem~3]{koenig99}.%, which correspond, respectively, to the hyperbolic, simply parabolic and doubly parabolic types.
%%%
%%%\begin{lem}
%%%Let $U$ be a $p$-periodic Baker domain of a meromorphic function $f$ in which $f^{np}\to\infty$ and on which $f^p$ has a conformal conjugacy. For $z_0\in U$, put
%%%$$
%%%c_n=c_n(z_0)=\frac{|f^{(n+1)p}(z_0)-f^{np}(z_0)|}{\textup{dist}(f^{np}(z_0),\partial U)}.
%%%$$
%%%Then exactly one of the following cases holds:
%%%\begin{enumerate}
%%%\item[(a)] $U$ is hyperbolic and $f^p\sim T_1(z):=\lambda z$ with $\lambda>1$, which is equivalent to
%%%$$
%%%c_n>c\quad \mbox{ for } z_0\in U,\ n\in \N,\quad \mbox{ where } c=c(f)>0.
%%%$$
%%%\item[(b)] $U$ is simply parabolic and $f^p\sim T_2(z):=z\pm i$, which is equivalent to 
%%%$$
%%%\liminf_{n\to\infty} c_n>0 \quad \mbox{ for } z_0\in U,\quad \mbox{ but } \inf_{z_0\in U}\limsup_{n\to\infty} c_n=0;
%%%$$
%%%\item[(c)] $U$ is doubly parabolic and $f^p\sim T_3(z):=z+1$, which is equivalent to
%%%$$
%%%\lim_{n\to\infty} c_n=0\quad \mbox{ for } z_0\in U;
%%%$$
%%%\end{enumerate}
%%%\label{lem:bd-koenig}
%%%\end{lem}
%%%
%%%\begin{figure}[h!]
%%%\centering
%%%\def\svgwidth{\linewidth}
%%%\input{bd-types1.pdf_tex}
%%%\hspace*{18pt}(a) $U$ hyperbolic\hspace*{38pt} (b) $U$ simply parabolic \hspace*{20pt} (c) $U$ doubly parabolic
%%%\caption{Classification of Baker domains with their absorbing domains.}
%%%\label{fig:bd-types}
%%%\end{figure}
%%%
%%%We now give a couple of explicit examples of transcendental self-maps of $\C^*$, with a hyperbolic and a doubly parabolic Baker domain respectively.
%%%
%%%\begin{ex}
%%%For every $\lambda>1$, the function $f_\lambda(z)=\lambda z\exp(e^{-z}+1/z)$ is a transcendental self-map of $\C^*$ which has an invariant hyperbolic Baker domain $U$ whose boundary contains both zero and infinity and the points in $U$ escape to infinity (see Figure~\ref{fig:hyp-baker-domain}).
%%%\label{ex:hyp-baker-domain}
%%%\end{ex}
%%%
%%%\begin{proof}[Proof of Example~\ref{ex:hyp-baker-domain}]
%%%Let $g_\lambda(z):=\lambda z\exp(e^{-z})$ with $\lambda>1$ and observe that
%%%$$
%%%g_\lambda(z)=\lambda z+O(ze^{-z})\quad \mbox{ as } \Re z\rightarrow +\infty.
%%%$$
%%%Since the exponential function maps a left half-plane to a disc centered at the origin, the function $\exp(e^{-z})$ maps the half-plane $H_R:=\{z\in\C\ :\ \Re z>R\}$ to a neighbourhood of $z=1$ for sufficiently large values of $R>0$ and hence $g_\lambda(H_R)\subseteq H_R$. Therefore, for every $\lambda>1$, the functions $g_\lambda$ and $f_\lambda$ have invariant Baker domains containing $H_R$. Let $U$ be the Baker domain of $f_\lambda$ and let $z_0\in U$, then
%%%$$
%%%f_\lambda^{n+1}(z_0)-f_\lambda^n(z_0)=(\lambda^n(\lambda-1)+o(1))z\quad\mbox{ and } \quad \textup{dist}(f_\lambda^n(z_0,\partial U)\leqslant \Re f_\lambda^n(z_0)+o(1)
%%%$$
%%%as $n\to\infty$, so there exists $n_0\in\N$ such that 
%%%$$
%%%c_n=\frac{|f_\lambda^{(n+1)p}(z_0)-f_\lambda^{np}(z_0)|}{\textup{dist}(f_\lambda^{np}(z_0),\partial U)}\geqslant \frac{(\lambda^n(\lambda-1)+o(1))|z|}{\Re f_\lambda^n(z_0)+o(1)}>\frac{\lambda-1}{2} \quad \mbox{ for all } n>n_0.
%%%$$
%%%Thus, there exists $0<c(f_\lambda)\leqslant (\lambda-1)/2$ such that $c_n<c(f_\lambda)$ for all $n\in\N$ and, by Lemma~\ref{lem:bd-koenig}, the Baker domain $U$ is hyperbolic.
%%%
%%%
%%%
%%%%By Lemma~\ref{lem:bd-koenig}, the Baker domain $U$ is hyperbolic. Indeed, if $z_0\in U$, then $f^n(z_0)$ and $\textup{dist}(f^{np}(z_0),\partial U)$ are asymptotically close to $\lambda^n z_0$ and $\textup{dist}(z_0,\partial U)+\lambda^nz_0-z_0$, respectively, and hence there exists $c=c(f_\lambda):=(\lambda-1)/2$ such that $c_n>c$ for sufficiently large values of $n\in\N$.
%%%
%%%Finally, observe that $U$ contains the positive real line and $f_\lambda$ is not univalent in $U$ (see Figure \ref{fig:hyp-baker-domain}). This follows from the fact that if $x>0$ then $e^{-x}+1/x>0$ and therefore $f(x)>x$ in the positive real line. Since $f_\lambda(z)\to +\infty$ as $z\to 0^+$ and as $z\to +\infty$, the positive real axis contains a critical point. 
%%%\end{proof}
%%%
%%%%Observe that, in the limit case $\lambda=1$, the function $g_1(z):=z\exp(e^{-z})$ has no longer have a Baker domain, but...\red{finish this}
%%%
%%%\begin{figure}[h!]
%%%\includegraphics[width=.49\linewidth]{hypbd-01.png}
%%%\includegraphics[width=.49\linewidth]{hypbd-02.png}
%%%\caption{Phase space of the function $f_2(z)=2z\exp(e^{-z}+1/z)$ from Example~\ref{ex:hyp-baker-domain}.}
%%%\label{fig:hyp-baker-domain}
%%%\end{figure}
%%%
%%%The function $f(z)=2 z\exp(e^{-z}+1/z)$ has a repelling fixed point in the negative real line. If we choose $h(z)=1/z^2$ instead of $1/z$, then $f(z)=2 z\exp(e^{-z}+1/z^2)$ has the positive real axis in a Baker domain while the negative real axis is in the fast escaping set.
%%%
%%%%\begin{figure}[h!]
%%%%\label{fig:baker-domains2}
%%%%\includegraphics[width=.49\linewidth]{019-2.png}
%%%%\includegraphics[width=.49\linewidth]{020.png}
%%%%\caption{Phase space of the function $f(z)=2z\exp(e^{-z}+1/z^2)$, for which the positive real line is in a Baker domain (gray) and the negative real line is in the fast escaping set (blue).}
%%%%\end{figure}
%%%
%%%
%%%%\begin{center}
%%%%\begin{tabular}{cc}
%%%%%\includegraphics[width=130pt]{standard.png} & \includegraphics[width=130pt]{standard2.png}\vspace{2pt}\\
%%%%\includegraphics[width=140pt]{011-3.png} & \includegraphics[width=140pt]{012-2.png}\vspace{2pt}\\
%%%%%$\alpha=3.1,\ \beta=0.8.$ & $\alpha=3.1,\ \beta=5$.\\
%%%%\end{tabular}
%%%%\end{center}
%%%
%%%We now give a second explicit example of transcendental self-map of $\C^*$ with a Baker domain which, in this case, is doubly parabolic.
%%%
%%%\begin{ex}
%%%The function $f(z)=z\exp\left((e^{-z}+1)/z\right)$ is a transcendental self-map of $\C^*$ which has a fixed doubly parabolic Baker domain $U$ whose boundary contains both zero and infinity and the points in $U$ escape to infinity (see Figure~\ref{fig:dpar-baker-domain}). \vspace*{-10pt}
%%%\label{ex:dpar-baker-domain}
%%%\end{ex}
%%%
%%%\begin{proof}[Proof of Example~\ref{ex:dpar-baker-domain}]
%%%Looking at the power series expansion of $f$, we have that
%%%$$
%%%f(z)=z+1+o(1)\quad \mbox{ as } \Re z\rightarrow +\infty.
%%%$$
%%%Therefore $f$ maps the right half-plane $H_R:=\{z\in\C\ :\ \Re z>R\}$ into itself for sufficiently large values of $R>0$ and $H_R$ is contained in an invariant Baker domain $U$ of $f$. Since $f(x)>x$ for all $x>0$, the positive real axis lies in $U$. Let $z_0\in U$, then
%%%$$
%%%f^{n+1}(z_0)-f^n(z_0)=1+o(1)\quad\mbox{ and } \quad \textup{dist}(f^n(z_0,\partial U)\geqslant \Re f^n(z_0)+o(1)
%%%$$
%%%as $n\to \infty$, so 
%%%$$
%%%c_n=\frac{|f^{(n+1)p}(z_0)-f^{np}(z_0)|}{\textup{dist}(f^{np}(z_0),\partial U)}\leqslant \frac{1+o(1)}{\Re f^n(z_0)+o(1)}\to 0 \quad \mbox{ as } n\to\infty
%%%$$
%%%and, by Lemma~\ref{lem:bd-koenig}, the Baker domain $U$ is doubly parabolic.
%%%% $f^n(z_0)$ and $\textup{dist}(f^{np}(z_0),\partial U)$ are asymptotically close to $z_0+n$ and $\textup{dist}(z_0,\partial U)+n$, respectively, and hence the quantity $c_n$ from Lemma~\ref{lem:bd-koenig} tends to $0$ as $n\to\infty$. 
%%%\end{proof}
%%%
%%%\begin{figure}[h!]
%%%\includegraphics[width=.49\linewidth]{dpbd-01.png}
%%%\includegraphics[width=.49\linewidth]{dpbd-02.png}%{d002.png}
%%%\caption{Phase space of the function $f(z)=z\exp\left((e^{-z}+1)/z\right)$ from Example~\ref{ex:dpar-baker-domain}.}
%%%\label{fig:dpar-baker-domain}
%%%\end{figure}
%%%
%%%%\begin{center}
%%%%\begin{tabular}{cc}
%%%%%\includegraphics[width=130pt]{standard.png} & \includegraphics[width=130pt]{standard2.png}\vspace{2pt}\\
%%%%\includegraphics[width=140pt]{d001-2.png} & \includegraphics[width=140pt]{d002.png}\vspace{2pt}\\
%%%%%$\alpha=3.1,\ \beta=0.8.$ & $\alpha=3.1,\ \beta=5$.\\
%%%%\end{tabular}
%%%%\end{center}
%%%
%%%\begin{lem}
%%%Let $f$ be a transcendental self-map of $\C^*$ and let $\tilde{f}$ be a lift of $f$. Then, if $U$ is a Baker domain of $f$, every component $U_k,\ k\in\Z,$ of $\exp^{-1}(U)$ is either a (preimage of a) Baker domain or a wandering domain of $\tilde{f}$. Moreover, if $U_k$ is a Baker domain, then $U_k$ is hyperbolic, simply parabolic or doubly parabolic if and only if $U$ is hyperbolic, simply parabolic or doubly parabolic, respectively.
%%%\label{lem:semiconj-bd}
%%%\end{lem}
%%%\begin{proof}
%%%By \cite{bergweiler95}, every component of $\exp^{-1}(U)$ is a Fatou component of $\tilde{f}$. Moreover, since $\exp^{-1}(I(f))\subseteq I(\tilde{f})$, $U_k$ is either a Baker domain, a preimage of a Baker domain or an escaping wandering domain of $\tilde{f}$.
%%%
%%%Suppose that $U$ has period $p\geqslant 1$ and $U_k$ is periodic, then the Baker domain $U_k$ has period $q$ with $p\mid q$. Let $(V,\phi,T)$ be a conformal conjugacy of $f^q$ in $U$. Note that although $U$ may be doubly connected, the absorbing domain $V$ is simply connected. Then $(\tilde{V},\tilde{\phi},T)$ is a conformal conjugacy of $\tilde{f}^q$ in $U_k$, where $\tilde{V}$ is the component of $\exp^{-1}V$ that lies in $U_k$ and $\tilde{\phi}=\phi\circ\exp$. Thus, the Baker domains $U$ and $U_k$ are of the same type.
%%%\end{proof}
%%%
%%%As before, we use Lemma~\ref{lem:semiconj-bd} to provide examples of transcendental entire functions with Baker domains and wandering domains.
%%%
%%%\begin{ex}
%%%\mbox{The\,transcendental\,entire\,function\,$\tilde{f}(z)\!=\!\ln \lambda\!+\!z\!+\!\exp(-e^z)\!+\!e^{-z}$},\linebreak which is a lift of the function $f$ from Example~\ref{ex:hyp-baker-domain}, has an invariant hyperbolic Baker domain that contains the real line.
%%%\end{ex}
%%%
%%%\begin{ex}
%%%The transcendental entire function $\tilde{f}(z)=z+\frac{\exp(-e^z)}{e^z}+e^{-z}$, which is a lift of the function $f$ from Example~\ref{ex:dpar-baker-domain}, has an invariant doubly parabolic Baker domain that contains the real line.
%%%\label{ex:dpar-baker-domain}
%%%\end{ex}
%%%
%%%%
%%%%
%%%%~\\
%%%%\red{Check what happens with $f(z)=z\exp(...)$\\ 
%%%%
%%%%\noindent
%%%%What other parameters have the Baker domain when we consider $\lambda\in\C$?\\
%%%%
%%%%\noindent
%%%%Does the fact that $f$ is univalent on the BD depend on $h$?\\
%%%%
%%%%\noindent
%%%%Is the BD of $\lambda z\exp(e^{-z})$ univalent?\\
%%%%
%%%%\noindent
%%%%Can we say for all $\lambda>1$ the BD is not univalent?}
%%%
%%%
%%%\section{Preliminaries on approximation theory}
%%%
%%%In this section we state the results from approximation theory that will be used in Sections~\ref{sec:wd} and \ref{sec:bd} to construct examples of functions with wandering domains and Baker domains, respectively. We follow the terminology from \cite[Chapter~IV]{gaier87}, and introduce Weierstrass and Carleman sets. Recall that if $F\subseteq \C$ is a closed set, then $A(F)$ denotes the set of continuous functions $f:F\to\C$ that are holomorphic in the interior of $F$.
%%%
%%%%%\red{Let $F,G\subseteq \C$ be, respectively, closed and open sets, then we define
%%%%%$$
%%%%%\begin{array}{rl}
%%%%%\mathcal C(F)\hspace*{-6pt}&:=\{f:F\to \C\ :\ f \mbox{ continuous in } F \},\vspace*{5pt}\\
%%%%%\textup{Hol}(G)\hspace*{-6pt}&:=\{f:G\to \C\ :\ f \mbox{ holomorphic in } G\},
%%%%%\end{array}
%%%%%$$
%%%%%and $A(F):=\mathcal C(F)\cap \textup{Hol}(F^\circ)$, where $F^\circ$ denotes the interior of $F$.\margin{$G^*$?}}
%%%% Note that there are more general versions of these results for approximating using functions that are only holomorphic in a domain $G\subseteq \C$.
%%%
%%%
%%%
%%%%Let $F\subseteq \C$ be closed. 
%%%%\begin{enumerate}
%%%%\item[(K$_1$)] $\CR\setminus F$ is connected;
%%%%\item[(K$_2$)] $\CR\setminus F$ is locally connected at $\infty$;
%%%%\item[(A)] for every compact subset $K\subseteq \C$ there exists a neighbourhood $V$ of $\infty$ in $\CR$ such that no component of $F^\circ$ intersects both $K$ and $V$.
%%%%\end{enumerate}
%%%
%%%\begin{dfn}[Weierstrass set]
%%%\label{dfn:weierstrass-set}
%%%We say that a closed set $F\subseteq\C $ is a \textit{Weierstrass set} in $\C$ if each $f\in A(F)$ can be approximated by entire functions \textit{uniformly} on $F$; that is, for every $\varepsilon>0$, there is an entire function $g$ for which
%%%$$
%%%|f(z)-g(z)|<\varepsilon\quad \mbox{ for all } z\in F.
%%%$$
%%%\end{dfn}
%%%
%%%The next result is due to Arakeljan and provides a characterisation of Weierstrass sets \cite{arakeljan64}. In the case that $F\subseteq \C$ is compact and $\C\setminus F$ is connected, then it follows from Mergelyan's theorem \cite[Theorem~1~on~p.~97]{gaier87} that functions in $A(F)$ can be uniformly approximated on $F$ by polynomials.
%%%
%%%\begin{lem}[Arakeljan's theorem]
%%%A closed set $F\subseteq \C$ is a Weierstrass set if and only if the following two conditions are satisfied:
%%%\begin{enumerate}
%%%\item[\emph{(K$_1$)}] $\CR\setminus F$ is connected;
%%%\item[\emph{(K$_2$)}] $\CR\setminus F$ is locally connected at infinity.
%%%\end{enumerate}
%%%\end{lem}
%%%
%%%%\begin{lem}\margin{see Gauthier}
%%%%If $E$ is a real-symmetric Weierstrass set in $G$ and $f$ is a real-symmetric function meromorphic on $E$ then, for each $\varepsilon>0$, there is a real-symmetric meromorphic function whose poles in $\C$ are the same as those of $f$ on $E$ and with same
%%%%principal parts, such that
%%%%$$
%%%%|f(z)-g(z)|<\varepsilon\exp(-|z|^{1/4}),\quad \mbox{ for all } z\in E.
%%%%$$
%%%%\end{lem}
%%%
%%%If in addition both the set $F$ and the function $f\in A(f)$ are symmetric with respect to the real line, then the approximating function $g$ can be chosen to be symmetric as well (see \cite[Section 2]{gauthier13}).\margin{other ref?}
%%%
%%%Sometimes we may want to approximate a function in $A(f)$ so that the error is bounded by a given strictly positive function $\varepsilon:\C\to\R_+$ that is not constant, and $\varepsilon(z)$ may tend to zero as $z\to\infty$. 
%%%
%%%\begin{dfn}[Carleman set]
%%%\label{dfn:carleman-set}
%%%We say that a closed set $F\subseteq \C$ is a \textit{Carleman set} in $\C$ if every function $f\in A(F)$ admits \textit{tangential approximation} on $F$ by entire functions; that is, for every strictly positive function $\varepsilon\in\mathcal C(F)$, there is an entire function $g$ for which
%%%$$
%%%|f(z)-g(z)|<\varepsilon(|z|)\quad \mbox{ for all } z\in F.
%%%$$
%%%\end{dfn}
%%%
%%%It is clear that Carleman sets are a special case of Weierstrass sets and hence conditions ($\text{K}_1$) and ($\text{K}_2$) are necessary. Nersesjan's theorem gives sufficient conditions for tangential approximation \cite{nersesjan71}.
%%%
%%%\begin{lem}[Nersesjan's theorem]
%%%A closed set $F$ is a Carleman set in $G$ if and only if conditions \emph{($K_1$)}, \emph{($K_2$)} and 
%%%\begin{enumerate}
%%%\item[\emph{(A)}] for every compact set $K\subseteq \C$ there exists a neighbourhood $V$ of infinity in $\CR$ such that no component of $\mbox{int}\, F$ intersects both $K$ and $V$;
%%%\end{enumerate}
%%%are satisfied.
%%%\label{lem:nersesjan}
%%%\end{lem}
%%%
%%%Note that there is also a symmetric version of this result: if the set $F$ and the functions $f$ and $\varepsilon$ are in addition symmetric with respect to $\R$ then the entire function $g$ can be chosen to be symmetric with respect to $\R$ \cite[Section 2]{gauthier13}.
%%%
%%%In some cases, depending on the geometry of the set $F$ and the decay of the error function $\varepsilon$, we can perform tangential approximation on Weierstrass sets without needing condition (A); the next result can be found in \cite[Corollary in p.162]{gaier87}.
%%%
%%%\begin{lem}
%%%Suppose $F\subseteq \C$ is a closed set satisfying conditions ($\text{K}_1$) and ($\text{K}_2$) and lying in a sector
%%%$$
%%%W_\alpha:=\{z\in \C\ :\ |\textup{arg}\,z|\leqslant \alpha/2\},
%%%$$
%%%for some $0<\alpha\leqslant 2\pi$. Suppose $\tilde{\varepsilon}(t)$ is a real function that is continuous and positive for $t\geqslant 0$ and satisfies
%%%$$
%%%\int_1^{+\infty} t^{-(\pi/\alpha)-1}\log\tilde{\varepsilon}(t)dt>-\infty.
%%%$$
%%%Then every function $f\in A(F)$ admits $\varepsilon$-approximation on $F$ with $\varepsilon(z)=\tilde{\varepsilon}(|z|)$ for $z\in F$. 
%%%\label{lem:approx-sectors}
%%%\end{lem}
%%%
%%%\section{Construction of functions with wandering domains}
%%%
%%%\label{sec:wd}
%%%
%%%%\red{History of wandering domains... which have zeros.... mention Kotus example of WD accumulating everywhere following EL.}
%%%
%%%To prove Theorem \ref{thm:wandering-domains} we modify Baker's construction of a holomorphic self-map of $\C^*$ with a wandering domain escaping to infinity \cite[Theorem 4]{baker87} to create instead a transcendental self-map of $\C^*$ with a wandering domain that accumulates to zero and to infinity according to a prescribed essential itinerary $e\in\{0,\infty\}^\N$.
%%%
%%%\begin{proof}[Proof of Theorem \ref{thm:wandering-domains}] 
%%%We construct two entire functions $g$ and $h$ using Nersesjan's theorem so that the function $f(z)=z^n\exp\bigl(g(z)+h(1/z)\bigr)$, which is a transcendental self-map of $\C^*$, has the following properties:
%%%\begin{itemize}
%%%\item there is a bi-infinite sequence of annuli sectors $\{A_m\}$, $m\in\Z\setminus\{0\}$, that accumulate to zero and infinity and such that $f(A_m)\subseteq A_{s(m)}$ for all $m\in\Z$;
%%%\item the discs $B_+:=\overline{D(2,1/4)}$ and $B_-:=1/B_+=\overline{D(32/63, 4/63)}$ both map strictly inside themselves under $f$, $f(B_+)\subseteq \textup{int}\,B_+$ and $f(B_-)\subseteq \textup{int}\,B_-$;
%%%\item there is a bi-infinite sequence of closed discs $\{B_m\}$, $m\in\Z\setminus\{0\}$, such that $f(B_m)\subseteq \textup{int}\,B_+$, if $m>0$, and $f(B_m)\subseteq \textup{int}\,B_-$, if $m<0$;
%%%\end{itemize}
%%%%Remember that if $X\subseteq \C$ and $\varepsilon>0$, we use the notation $X+\varepsilon:=\{z\in \C\ :\ \mbox{dist}(z,X)<\varepsilon\}$. 
%%%where $s(m):=p(p^{-1}(m)+1)$ and the map $p:\N\longrightarrow \Z\setminus \{0\}$, which is an ordering of the sets $\{A_m\}$ according to the sequence $e$, that is, $p(m)$ is the position of the $m$th disc in the orbit of the wandering domain. More formally,
%%%\begin{equation}
%%%p(m):=\left\{
%%%\begin{array}{ll}
%%%%\min\bigl\{\min_{\ell<k} p^{\ell+1}(0),\ 0\bigr\}-1=
%%%\ds-\#\{\ell<k\ :\ e_\ell=0\}-1, & \mbox{ if } e_{k}=0,\vspace{5pt}\\
%%%%\max\bigl\{\max_{\ell<k} p^{\ell+1}(0),\ 0\bigr\}+1=
%%%\ds\#\{\ell<k\ :\ e_\ell=\infty\}+1, & \mbox{ if } e_{k}=\infty,
%%%\end{array}
%%%\right.
%%%\label{eq:p-function}
%%%\end{equation}
%%%for all $m\in \N$ (see Figure \ref{fig:sketch-wd}).
%%%
%%%\begin{figure}[h!]
%%%\centering
%%%\vspace{30pt}
%%%\def\svgwidth{.8\linewidth}
%%%\input{wd3.pdf_tex}
%%%\vspace{15pt}
%%%\caption{Sketch of the construction in the proof of Theorem \ref{thm:wandering-domains}.}
%%%\label{fig:sketch-wd}
%%%\end{figure}
%%%
%%%By Montel's theorem, the domains $\{A_m\}$, $\{B_m\}$ and $B_\pm$ are all contained in the Fatou set. Since $f(B_+)\subseteq \textup{int}\,B_+$, $f$ has an attracting fixed point in $B_+$ and the sets $\{B_m\}$, $m>0$, are contained in the preimages of the immediate basin of attraction of this fixed point. Likewise, the sets $\{B_m\}$, $m<0$, belong to the basin of attraction of a fixed point in $B_-$. Observe that in order to show that $A_1$ is contained in a wandering domain that escapes following the essential itinerary $e$ we need to prove that every $A_m$ is contained in a different Fatou component. 
%%%
%%%Now let us construct the entire functions $g$ and $h$ that will ensure that $f$ has the properties stated above. Let $0<R<\pi/2$ and set, for $m>0$,
%%%$$
%%%\begin{array}{l}
%%%A_m:=\{z\in\C\ :\ -R\leqslant\mbox{arg}(z)\leqslant R,\ k_m\leqslant |z|\leqslant \exp(2R+\log k_m)\},\vspace{5pt}\\
%%%B_m:=\overline{D\bigl((k_{m+1}-k_m)/2,\ 1/8\bigr)},
%%%\end{array}
%%%$$
%%%where $k_m$ is any sequence of positive real numbers such that $k_m>5/2$ and $k_{m+1}>k_m+1/4$ for all $m\in \N$. We define $A_{-m}:=1/A_m$ and $B_{-m}:=1/B_m$ for all $m\in\N$. Note that $\log A_m$ is a square of side $2R$ centered at a point that we denote by $a_m\in \R$. Hence, $\log A_m$ contains the disc $D(a_m,R)$ for all $m\in\Z\setminus\{0\}$. The set 
%%%$$
%%%F:=\overline{D(0,1)}\cup B_+\cup \bigcup_{m>0} (A_m\cup B_m)
%%%$$
%%%which consists of a countable union of disjoint compact sets is a Carleman set.
%%%
%%%Let $\delta_+,\delta_->0$ be such that $|w-\ln 2|<\delta_+$ and $|w-\ln 32/63|<\delta_-$ imply respectively $|e^w-2|<1/8$ and $|e^w-32/63|<2/63$. Let $K:=\min\{R/4, \delta_\pm/4\}$. By Lemma \ref{lem:nersesjan}, there is an entire function $g$ that satisfies the following conditions:
%%%$$
%%%\left\{
%%%\begin{array}{ll}
%%%|g(z)-a_{s(m)}-n\log z|<R/4, & \mbox{if } z\in A_m \mbox{ with } m>0,\vspace{10pt}\\
%%%|g(z)-\ln 2-n\log z|<\delta_+/4, & \ds\mbox{if } z\in \bigcup_{m>0} B_m\cup B_+,\vspace{5pt}\\
%%%|g(z)|<K, & \mbox{if } z\in D(0,1),
%%%\end{array}
%%%\right.\vspace{5pt}
%%%$$
%%%Similarly, there is an entire function $h$ that satisfies the following conditions:
%%% $$ \left\{
%%%\begin{array}{ll}
%%%|h(z)-a_{s(-m)}-n\log (1/z)|<R/4, & \mbox{if } z\in A_m \mbox{ with } m>0,\vspace{10pt}\\
%%%|h(z)-\ln 32/63-n\log (1/z)|<\delta_-/4, & \ds\mbox{if } z\in \bigcup_{m>0} B_m\cup B_+,\vspace{5pt}\\
%%%|h(z)|<K, & \mbox{if } z\in D(0,1).
%%%\end{array}
%%%\right.\vspace{5pt}
%%%$$
%%%Therefore, the function $\log f(z)=g(z)+h(1/z)+n\log z$ satisfies that
%%%$$
%%%\left\{
%%%\begin{array}{ll}
%%%|\log f(z)-a_{s(m)}|<R/2, & \mbox{if } z\in A_m \mbox{ with } m\neq 0,\vspace{10pt}\\
%%%|\log f(z)-\ln 2|<\delta_+/2, & \ds\mbox{if } z\in  \bigcup_{m>0} B_m\cup B_+,\vspace{10pt}\\
%%%|\log f(z)-\ln 32/63|<\delta_-/2, & \ds\mbox{if } z\in  \bigcup_{m<0} B_m\cup B_-\\
%%%\end{array}
%%%\right.
%%%$$
%%%and hence $f$ has the required mapping properties.
%%%
%%%%Observe\margin{$R$ should also\\ be small in\\ the basins of attraction $B_{\pm}$,\\ should we\\ remove this?*} that if $R$ is a rational function such that $R(0)=R(\infty)=0$ then in a neighbourhood of the essential singularities $R(z)$ will be arbitrarily small and hence $|\log f(z)+R(z)|<R$ in $A_n$ for $|n|$ large. Hence, $\tilde{f}=f\cdot e^R$ has a Baker domain in $I_e(\tilde{f})$ too.
%%%
%%%Finally, note that this construction is symmetric with respect to the real line and hence all Fatou components of $f$ will be symmetric too. Thus, since transcendental self-maps of $\C^*$ cannot have doubly connected Fatou components that do not surround the origin \cite[Theorem 1]{baker87}, the Fatou components containing the sets $\{A_m\}$, $m\in\Z\setminus \{0\}$, are pairwise disjoint and $A_{p(0)}$ is an escaping wandering domain in $I_e(f)$.
%%%\end{proof}
%%%
%%%%\red{* We can add any tef that is small in $F$, but it seems nicer to state it in terms of a rational function because it is less technical. Perhaps we can get rid of the problem that the new function does not map $B_\pm$ into itself by making each $B_n$ map into itself instead of map into $B_\pm$? This can be done easily if $B_n$ are not discs but have the same shape as the $A_n$...}
%%%
%%%\section{Construction of functions with Baker domains}
%%%
%%%\label{sec:bd}
%%%
%%%
%%%
%%%In this section we construct holomorphic self-maps of $\C^*$ with Baker domains. The construction is split in two cases: first, we deal with the cases that the function $f$ is a transcendental entire or meromorphic function, that is, $f(z)=z^n\exp(g(z))$ with $n\in\Z$ and $g$ is a non-constant entire function (see Theorem~\ref{thm:baker-domains-entire}), and then we deal with the case that the function $f$ is a transcendental self-map of $\C^*$, that is, $f(z)=z^n\exp(g(z)+h(1/z))$ with $n\in \Z$ and $g,h$ are non-constant entire functions (see Theorem~\ref{thm:baker-domains}). For transcendental self-maps of $\C^*$, we are able to construct functions with Baker domains that have any given \textit{periodic} essential itinerary $e\in\{0,\infty\}^\N$. 
%%%
%%%To that end, we use Lemma~\ref{lem:approx-sectors}~to obtain entire functions $g$ and, if necessary,~$h$ so that the function $f$ has a Baker domain with the desired properties. After this approximation process, the resulting function $f$ will behave as the function~$T_j$, $j\in\{1,2,3\}$, from Lemma~\ref{lem:bd-koenig} in a domain $W_j$, $j\in\{1,2,3\}$, depending on whether we want a hyperbolic, simply parabolic or doubly parabolic Baker domain, respectively. We first require the following result that estimates~the asymptotic distance between the boundaries of $\log W_j$ and $\log T_j(W_j)\subseteq \log W_j$ for each $j\in\{1,2,3\}$.
%%%%$W$ is an absorbing domain and , if $N$ is the period of the Baker domain,  %, but before we need a result that will ensure that the function we want to construct maps the domain well inside itself so that we can apply Lemma \ref{lem:approx-sectors}.
%%%
%%%\begin{lem}
%%%Let $\delta_j (r)$, $j=1,2,3$, denote the vertical distance between the curves 
%%%$$
%%%\partial\log  W_j\quad \mbox{ and } \quad \partial\log T_j(W_j)\subseteq \log W_j
%%%$$
%%%along the line \mbox{$V_r:=\{z\in\C : \Re z=r\}$, $r>0$}, in each of the three following cases:
%%%\begin{enumerate}
%%%\item[(a)] \mbox{$W_{1}=\{z\in\C\, :\, \Re z\geqslant 2\}$ and $T_1(z)=\lambda z$ where $\lambda>1$;}
%%%\item[(b)] $W_{2}=\{z\in\C\, :\, \Re z> 2,\ \Im z\geqslant 1/(\Re z-2)\}$ and $T_2(z)=z+i$;
%%%\item[(c)] $W_{3}=\{z\in\C\, :\, \Re z\geqslant 2\}$ and $T_3(z)=z+1$.
%%%\end{enumerate}
%%%Then, $\delta_1(r)\sim 2(\lambda-1)e^{-r}$, $\delta_2(r)\sim e^{-3r}$ and $\delta_3(r)\sim e^{-r}$.
%%%\label{lem:approx-BD}
%%%\end{lem}
%%%\begin{proof}
%%%Since $\log z=\ln |z|+i\,\mbox{arg}(z)$, the quantity $\delta_j(r)$ equals the difference between the arguments of the points $z_1,z_2$ with $\Im z_k>\Re z_k$, $k\in\{1,2\}$, where the vertical lines $\partial W_{j}$ and $\partial T_j(W_{j})$ intersect the circle $\exp V_r$ of radius $e^r$ for $j\in\{1,2,3\}$ (see Figure \ref{fig:sketch-bd}). Since $\mbox{arg}\,z_1,\mbox{arg}\,z_2 \to\pi/2$ as $r\to+\infty$, for $j=1$,
%%%$$
%%%\delta_1(r)=\arccos \frac{2}{e^r}-\arccos\frac{2\lambda}{e^r}\sim\left(\frac{\pi}{2}-\frac{2}{e^r}\right)-\left(\frac{\pi}{2}-\frac{2\lambda}{e^r}\right)=\frac{2(\lambda-1)}{e^r},
%%%$$
%%%where $\lambda>1$ and, for $j=3$,
%%%$$
%%%\delta_3(r)=\arccos \frac{2}{e^r}-\arccos\frac{3}{e^r}\sim\left(\frac{\pi}{2}-\frac{2}{e^r}\right)-\left(\frac{\pi}{2}-\frac{3}{e^r}\right)= \frac{1}{e^r}.
%%%$$
%%%
%%%\begin{figure}[h!]
%%%\centering
%%%\vspace{30pt}
%%%\def\svgwidth{.8\linewidth}
%%%\input{bd.pdf_tex}
%%%\vspace{15pt}
%%%\caption{Sketch of the construction in the proof of Lemma \ref{lem:approx-BD}.}
%%%\label{fig:sketch-bd} 
%%%\end{figure}
%%%
%%%%The set $F:=\log\bigl(\{\Re z>x\}\bigr)$, with $x>0$, is a Weierstrass set. We will perfom tangential approximation with $\varepsilon(z)$ as error function. Since $\log(\{\Re z=x\})$ is contained in a horizontal band of width $2\pi$, we can use Lemma \ref{lem:approx-sectors} with a sector of any angle $\alpha>0$, and $\tilde{\varepsilon}(t)=\sqrt{2}e^{-t}$ clearly satisfies the condition
%%%%$$
%%%%\int_1^{+\infty} t^{-(\pi/\alpha)-1}\log\tilde{\varepsilon}(t)dt>-\infty
%%%%$$
%%%%if we choose $\alpha<\pi$.\\
%%%%\margin{do we need\\ this?}Observe that the points in the intersection of the vertical lines $V_k$ and $V_{k+1}$ with the circle of radius $e^r$ tend asymptotically to the points in the intersection of $V_k$ and $V_{k+1}$ with the horizontal line $H_{e^r}:=\{z\in\C\ :\ \Im z=e^r\}$ as $r\to+\infty$.
%%%
%%%%For $j=2$, the angle $\delta_2(r)$ is bounded below by the angle defined by the points $z_1',z_2'$ in the hyperbolas $\partial W_2$ and $\partial T_2(W_2)$ with $\Im z_1'=\Im z_2'=e^r$. Indeed, first observe that if $z_2''\in\partial T_2(W_2)$ with $\Im z_2''=\Im z_1$, then the angle defined by $z_1$ and $z_2''$ is smaller than the angle defined by $z_1$ and any point $w\in\partial T_2(W_2)$ with $\Im w<\Im z_2''$. Moreover, the angle defined by the points $z,w$ in $\partial W_2$ and $\partial T_2(W_2)$ with $\Im z=\Im w=e^r$ is strictly decreasing as $r\to +\infty$. Thus, since $\mbox{arg}\,z_1',\mbox{arg}\,z_2' \to\pi/2$ as $r\to+\infty$, 
%%%For $j=2$, the quantity $\delta_2(r)$ is asymptotically close to the angle defined by the points $z_1',z_2'$ in the hyperbolas $\partial W_2$ and $\partial T_2(W_2)$ with $\Im z_1'=\Im z_2'=e^r$, thus
%%%$$
%%%\begin{array}{rl}
%%%\ds\delta_2(r)\hspace*{-6pt}&\ds\sim\arccos \frac{\Re z_1'}{e^r}-\arccos \frac{\Re z_2'}{e^r}\sim\left(\frac{\pi}{2}-\frac{\Re z_1'}{e^r}\right)-\left(\frac{\pi}{2}-\frac{\Re z_2'}{e^r}\right)=\vspace{5pt}\\
%%%&\ds=-\frac{1}{e^r}\left(\frac{1}{e^r}+2\right)+\frac{1}{e^r}\left(\frac{1}{e^r-1}+2\right)=\frac{1}{e^{2r}(e^r-1)}\sim \frac{1}{e^{3r}}
%%%\end{array}
%%%$$
%%%as we wanted to show.
%%%%and hence $\delta_2(r)>e^{-3r}/2$ for sufficiently large values of $r$. The other inequality follows from the fact that...
%%%\end{proof}
%%%
%%%Given a periodic essential itinerary $e=\overline{e_0e_1\cdots e_{N-1}}\in\{0,\infty\}^\N$, let $p,q\in\N$ denote
%%%$$
%%%p=\#\{0\leqslant k<N\ :\ e_k=\infty\}\quad \mbox{ and }\quad q=\#\{0\leqslant k<N\ :\ e_k=0\}
%%%$$
%%%with $p+q=N$. We want to construct a holomorphic function $f:\C^*\to\C^*$ with an $N$-cycle of Baker domains that has components $U_i^\infty$, $0\leqslant i<p$, and $U_i^0$, $0\leqslant i<q$, in which 
%%%$$
%%%f_{|U_i^\infty}^{Nn}\to\infty\quad \mbox{ and } \quad f_{|U_i^0}^{Nn}\to 0 \quad \mbox{ locally uniformly as } n\to \infty.
%%%$$
%%%In the case that zero is not an essential singularity of $f$, then $q=0 $ and $N=p$. Note that the closure of a Baker domain in $\CR$ may contain both zero and infinity. 
%%%
%%%For $p\in\N$ and $X\subseteq \C$, we define $\sqrt[p]{X}:=\{z\in\C\ :\ z^p\in X\}$. In order to construct a function with an $N$-periodic Baker domain that has $p$ components around zero and infinity, we will semiconjugate the function $T_j$ that we want to approximate in the domain $W_j$ by the $p$th root function:
%%%$$
%%%\xymatrix{
%%%W_j \ar[r]^{T_j}  & W_j\\
%%%\sqrt[p]{W_j} \ar[u]^{z^p} \ar[r]_{T_{j,p}} & \sqrt[p]{W_j}. \ar[u]_{z^p}
%%%}
%%%$$
%%%Next we look at the effect of this semiconjugation to the distance function $\delta_j$ for $j\in\{1,2,3\}$.
%%%
%%%\begin{lem}
%%%Let $W_{j}$ and $T_j$, $j\in\{1,2,3\}$, be the sets and the functions from Lemma~\ref{lem:approx-BD}. For $p\in\N$, define the function $T_{j,p}(z)=\sqrt[p]{T_j(z^p)}$ on $\sqrt[p]{W_j}$ and, for $r>0$, let $\delta_{j,p}(r)$ denote the vertical distance between the curves 
%%%$$
%%%\partial \log \sqrt[p]{W_{j}} \quad \mbox{ and } \quad \partial \log T_{j,p}(\sqrt[p]{W_{j}})\subseteq \log \sqrt[p]{W_{j}}
%%%$$
%%%along the line \mbox{$V_r:=\{z\in\C : \Re z=r\}$}. Then, we have $\delta_{1,p}(r)\sim 2(\lambda-1)e^{-pr}/p$,  $\delta_{2,p}(r)\sim e^{-3pr}/p$ and $\delta_{3,p}(r)\sim e^{-pr}/p$.
%%%\label{lem:approx-root-BD}
%%%\end{lem}
%%%\begin{proof}
%%%The function $z\mapsto z^p$ maps the circle of radius $e^r$ to the circle of radius $e^{pr}$ while the function $z\mapsto \sqrt[p]{z}$ divides the argument of points in that circle by $p$, thus
%%%$$
%%%\delta_{j,p}(r)=\frac{\delta_j(pr)}{p}
%%%$$
%%%as required.
%%%\end{proof}
%%%
%%%In the following theorem we construct transcendental entire or meromorphic functions that are self-maps of $\C^*$ and have Baker domains in which points escape to infinity. These functions are of the form $f(z)=z^n\exp(g(z))$ where $n\in\Z$ and $g$~is a non-constant entire function. 
%%%
%%%\begin{thm}
%%%\label{thm:baker-domains-entire}
%%%For every $N\in\N$ and $n\in\Z$, there exists a trancendental entire or meromorphic function $f$ that is a self-map of $\C^*$ with $\textup{ind}(f)=n$ that has a cycle of Baker domains of period $N$. Moreover these Baker domains can be chosen to be hyper\-bolic, simply parabolic or doubly parabolic.
%%%\end{thm}
%%%\begin{proof}
%%%Fix $j\in\{1,2,3\}$ according to whether the Baker domain to be constructed is hyperbolic, simply parabolic or doubly parabolic, respectively. Let $\omega_{p}:=e^{2\pi i/p}$ where $p:=N$ and define 
%%%$$
%%%V_{m}:=\omega_{p}^m\sqrt[p]{W_{j}}\subseteq \C\setminus \overline{\mathbb D} \quad \mbox{ for } 0\leqslant m<p,
%%%$$
%%%where $W_j$, $j\in\{1,2,3\}$, are the unbounded closed sets from Lemma~\ref{lem:approx-BD}. We denote by $V$ the union of all $V_m$ for $0\leqslant m<p$, and let $R=\R_-$, if $p$ is odd, or $R=\{z\in\C^*\,:\,\textup{arg}\,z=\pi(1-1/p)\}$, if $p$ is even. Note that $R\subseteq -\mathbb H$. Then put 
%%%$$
%%%d:=\min\{(\sqrt[p]{2}-1)/3,\ \textup{dist}(V,R)\},
%%%$$
%%%and define the closed connected set
%%%$$
%%%B:=\{z\in\C\,:\,\textup{dist}\,(z,V)\geqslant d/2 \mbox{ and } \textup{dist}\,(z,R)\geqslant d/2\}
%%%$$
%%%that satisfies $B_0:=\overline{D(1,d)}\subseteq \textup{int}\,B$ (see Figure~\ref{fig:sketch-bd-entire}).
%%%
%%%\begin{figure}[h!]
%%%\centering
%%%%\vspace{30pt}
%%%\def\svgwidth{.45\linewidth}
%%%\input{bd-entire.pdf_tex}
%%%%\vspace{15pt}
%%%\caption{Sketch of the construction in the proof of Theorem~\ref{thm:baker-domains-entire}.}
%%%\label{fig:sketch-bd-entire} 
%%%\end{figure}
%%%
%%%Observe that the closed set $F:=B\cup V$ satisfies the hypothesis of Lemma \ref{lem:approx-sectors}; namely $\CR\setminus F$ is connected and $\CR \setminus F$ is locally connected at infinity. We now define a function $\hat{g}\in A(F)$. For $z\in V_m$, $0\leqslant m<p$, we put 
%%%$$
%%%\hat{g}(z):=\left\{\begin{array}{ll}
%%%\log \left(\omega_{p}^{m+1}\sqrt[p]{\lambda (z/\omega_{p}^m)^{p}}\right)-n\log z, & \mbox{ if } j=1;\vspace{5pt}\\
%%%\log \left(\omega_{p}^{m+1}\sqrt[p]{(z/\omega_{p}^m)^{p}+1}\right)-n\log z, & \mbox{ if } j=2; \vspace{5pt}\\
%%%\log \left(\omega_{p}^{m+1}\sqrt[p]{(z/\omega_{p}^m)^{p}+i}\right)-n\log z, & \mbox{ if } j=3;
%%%\end{array}\right. 
%%%$$
%%%and, for $z\in B$, we put $\hat{g}(z):=-n\log z$. Then $\hat{g}\in A(F)$ because $F\subseteq\C^*\setminus R$ and hence $\mbox{arg}\,z$ is well defined on $F$ and we can take an analytic branch of the logarithm. 
%%%
%%%For $r>0$, we define the positive continuous function $\varepsilon$ by
%%%$$
%%%\varepsilon(r):=\min\{\ln(d),\ R^{-3p-1},\ r^{-3p-1}\} 
%%%$$
%%%where the constant $R>0$ is sufficiently large that, if $\delta_{j,p}(r)$ is the function from Lemma~\ref{lem:approx-root-BD}, for $r\geqslant R$, 
%%%$$
%%%\varepsilon(r)<\left\{
%%%\begin{array}{ll}
%%%\ds\delta_{1,p}(\ln (\lambda r))\sim \frac{2(\lambda-1)}{p\lambda^pr^p}, & \mbox{ if } j=1;\vspace{5pt}\\
%%%\ds\delta_{2,p}(\ln (r+1))\sim \frac{1}{p(r+1)^{3p}}, & \mbox{ if } j=2;\vspace{5pt}\\
%%%\ds\delta_{3,p}(\ln (r+1))\sim \frac{1}{p(r+1)^p}, & \mbox{ if } j=3.
%%%\end{array}
%%%\right.
%%%$$
%%%Since $\varepsilon$ satisfies that
%%%$$
%%%\int_1^{+\infty} r^{-3/2}\ln\varepsilon(r)dt=C-(3p+1)\int_{r_0'}^{+\infty} \frac{\ln r}{r^{3/2}}dr>-\infty
%%%$$
%%%for some constants $C\in\R$ and $r_0'\geqslant r_0$, by Lemma \ref{lem:approx-sectors}, there is an entire function~$g$ such that
%%%\begin{equation}
%%%|g(z)-\hat{g}(z)|<\varepsilon(|z|)\quad\mbox{ for all } z\in F.
%%%\label{eq:bd-entire-approx}
%%%\end{equation}
%%%We put 
%%%$$
%%%f(z):=z^n\exp(g(z))=z^n\exp(\hat{g}(z))\exp(g(z)-\hat{g}(z))
%%%$$
%%%and, by Lemma \ref{lem:approx-root-BD}, $f(V_m)\subseteq V_{m+1}$ for $0\leqslant m<p-1$, $f(V_{p-1})\subseteq V_0$ and $f(B)\subseteq D(1,d)$. Hence each set $V_m$ is contained in an $N$-periodic Baker domain~$U_m$ for $0\leqslant m<p$ and $B$ is contained in the immediate basin of attraction of a fixed point that lies in $B_0$.
%%%
%%%To conclude the proof of Theorem~\ref{thm:baker-domains-entire}, it only remains to check that the Baker domains $U_m$, $0\leqslant m<p$, are hyperbolic ($j=1$), simply parabolic ($j=2$) and doubly parabolic ($j=3$). Due to symmetry, it is enough to deal with the case $m=0$. Let $z_0\in U_0$, since $V_0\subseteq U_0$ is an absorbing region, we can assume without loss of generality that $z_0\in V_0$ and $|z_0|$ is sufficiently large. For $n\in\N$, let 
%%%$$
%%%e_n:=g(f^{n-1}(z_0))-\hat{g}(f^{n-1}(z_0))
%%%$$
%%%which, by \eqref{eq:bd-entire-approx}, since $z_0\in I(f)$, satisfies that
%%%$$
%%%|e_n|<\varepsilon(|f^{n-1}(z_0)|)\searrow 0 \quad \mbox{ as } n\to \infty.
%%%$$
%%%For $n\in\N$, define 
%%%$$
%%%C_n:=\prod_{0<k\leqslant n} \exp e_k=\exp \sum_{0<k\leqslant n} e_k,
%%%$$
%%%then, using the triangular inequality,
%%%$$
%%%|C_n|\leqslant \exp \sum_{0<k\leqslant n} |e_k|<\exp\sum_{0<k\leqslant n} \varepsilon(|f^{k-1}(z_0)|).
%%%$$
%%%We now show that $|C_n|$ is bounded above for all $n\in\N$. To that end, we find a lower bound for $|f^{k}(z_0)|$ for $k\in\N$ assuming, if necessary, that $|z_0|=r_0$ is sufficiently large. We treat separately the cases $j=1$, $j=2$ and $j=3$.
%%%\begin{enumerate}
%%%\item[(a)] For $j=1$, put $K:=(\sqrt[p]{\lambda}-1)/2>1$, then $|C_1|>1/K$ for $r$ sufficiently large and
%%%\mline{
%%%|f(z_0)|=\sqrt[p]{\lambda}|z_0||C_1|\geqslant \frac{\sqrt[p]{\lambda}}{K}r_0=\mu r_0
%%%}
%%%with $\mu:=\sqrt[p]{\lambda}/K>1$ and
%%%\mline{
%%%|f^k(z_0)|\geqslant \mu^k r_0\quad \mbox{ for } k\in\N.
%%%}
%%%Thus, $|C_n|<\exp S_1$ for all $n\in\N$, where $S_1$ is the sum of the following geometric series
%%%%$$
%%%%S_1:=\sum_{k=0}^\infty \frac{1}{(\mu^kr_0)^{3p+1}}=\frac{1}{r_0^{3p+1}} \sum_{k=0}^\infty \left( \frac{1}{\mu^{3p+1}} \right)^k =\frac{\mu^{3p+1}}{r_0^{3p+1}(\mu^{3p+1}-1)}<+\infty.
%%%%$$
%%%\mline{S_1:=\sum_{k=0}^\infty \frac{1}{(\mu^kr_0)^{3p+1}}=\frac{1}{r_0^{3p+1}} \sum_{k=0}^\infty \left( \frac{1}{\mu^{3p+1}} \right)^k =\frac{\mu^{3p+1}}{r_0^{3p+1}(\mu^{3p+1}-1)}<+\infty.}
%%%%$$
%%%%\begin{array}{rl}
%%%%\ds S_1\hspace{-6pt}&\ds:=\sum_{k=0}^\infty \left(\frac{\sqrt[p]{\lambda}^{k}}{K^{k}}r_0\right)^{-3p-1}=\frac{1}{r_0^{3p+1}} \sum_{k=0}^\infty \left( \frac{K^{3p+1}}{\lambda^3\sqrt[p]{\lambda}} \right)^k \vspace{5pt}\\
%%%%&\ds=\frac{\lambda^3\sqrt[p]{\lambda}}{r_0^{3p+1}(\lambda^3\sqrt[p]{\lambda}-K^{3p+1})}<+\infty.
%%%%\end{array}
%%%%$$
%%%\item[(b)] For $j=2$, since $\Im z^p>0$, we have 
%%%\mline{
%%%|f(z_0)|=\!\sqrt[p]{|z_0^p+i|}|C_1|=\!\!\sqrt[2p]{|z_0|^{2p}+1+2\Im z^p}|C_1|>\!\!\sqrt[2p]{r_0^{2p}+1}|C_1|
%%%}
%%%and using the fact that $\sqrt[n]{r^n+a}>r+a/r^n$ for $r$ sufficiently large, $n\in\N$ and $a=o(r^n)$, 
%%%\mline{
%%%|f(z_0)|>\left(r_0+\frac{1}{r_0^{2p}}\right)|\exp e_1|>\left(r_0+\frac{1}{r_0^{2p}}\right)\left(1-\frac{1}{r_0^{3p+1}}\right)>r_0+\frac{1}{2r_0^{2p}}
%%%}
%%%for $r_0$ sufficiently large and
%%%\mline{
%%%|f^k(z_0)|\geqslant r_0+\frac{k}{2r_0^{2p}} \quad \mbox{ for } k\in\N.
%%%}
%%%Thus, $|C_n|< \exp S_2$ for all $n\in\N$, where $S_2$ is the following harmonic series
%%%\mline{
%%%S_2:=\sum_{k=0}^\infty\left( r_0+\frac{k}{2r_0^{2p}} \right)^{-3p-1}<+\infty
%%%}
%%%because $3p+1>1$.
%%%
%%%\pagebreak
%%%
%%%\item[(c)] For $j=3$, since $\Re z^p>0$, we have 
%%%\mline{
%%%|f(z_0)|=\!\sqrt[p]{|z_0^p+1|}|C_1|=\!\!\sqrt[2p]{|z_0|^{2p}+1+2\Re z^p}|C_1|>\!\!\sqrt[2p]{r_0^{2p}+1}|C_1|
%%%}
%%%and, as in the case $j=2$, 
%%%\mline{
%%%|f(z_0)|>r_0+\frac{1}{2r_0^{2p}}
%%%}
%%%for $r_0$ sufficiently large and 
%%%\mline{
%%%|f^k(z_0)|\geqslant r_0+\frac{k}{2r_0^{2p}} \quad \mbox{ for } k\in\N.
%%%}
%%%Thus, $|C_n|<\exp S_2$ for all $n\in\N$.
%%%\end{enumerate}
%%%Now we are ready to show that each Baker domain has the type that corresponds to it. We use the characterisation of Lemma~\ref{lem:bd-koenig}. For $n\in\N$, define
%%%$$
%%%c_n=c_n(z_0)=\frac{|f^{(n+1)p}(z_0)-f^{np}(z_0)|}{\textup{dist}(f^{np}(z_0),\partial U)}.
%%%$$
%%%Again, we split the discussion in the three cases $j=1$, $j=2$ and $j=3$. 
%%%\begin{enumerate}
%%%\item[(a)] For $j=1$, we have
%%%\mline{
%%%f^{np}(z_0)=C_{np}\sqrt[p]{\lambda^{np}z_0^p}=C_{np}\lambda^{n}z_0 \quad \mbox{ for } n\in\N
%%%}
%%%and therefore
%%%\mline{
%%%|f^{(n+1)p}(z_0)-f^{np}(z_0)|\sim e^{S_1}\lambda^n(\lambda-1)|z_0|
%%%}
%%%while $\textup{dist}(f^{np}(z_0),\partial U_0)\sim e^{S_1}\lambda^n|z_0|$ and hence putting $c:=(\lambda-1)/2>0$, we have $c_n(z_0)>c$ for all $n\in\N$ and, by Lemma~\ref{lem:bd-koenig}, the Baker domain $U_0$ is hyperbolic. 
%%%
%%%\item[(b)] For $j=2$, we have...\\
%%%
%%%\item[(c)] Finally, for $j=3$, we have 
%%%\mline{
%%%f^{np}(z_0)=C_{np}\sqrt[p]{z_0^{p}+1+\sum_{k=0}^{np-1}\frac{1}{C_k^{p}}}
%%%}
%%%and therefore
%%%\mline{
%%%|f^{(n+1)p}(z_0)-f^{np}(z_0)|\sim \exp S_2\cdot |\sqrt[p]{z_0^p+(n+1)p}-\sqrt[p]{z_0^p+np}|=O(1)
%%%}
%%%while $\textup{dist}(f^{np}(z_0),\partial U_0)\to +\infty$ as $n\to \infty$ and hence $c_n(z_0)\to 0$ as $n\to \infty$ and, by Lemma~\ref{lem:bd-koenig}, the Baker domain $U_0$ is doubly parabolic. 
%%%\end{enumerate}
%%%This completes the proof of Theorem~\ref{thm:baker-domains-entire}.
%%%\end{proof}
%%%
%%%\begin{rmk}
%%%\margin{check}The functions constructed in Theorem~\ref{thm:baker-domains-entire} are the only examples of entire functions with a simply parabolic Baker domain that are not a lift of a holomorphic self-map of $\C^*$ with a Siegel disc.
%%%\end{rmk}
%%%
%%%Finally we prove Theorem \ref{thm:baker-domains} in which we construct a function $f$ that is a transcendental self-map of $\C^*$ with $\textup{ind}(f)=n$ that has a hyperbolic, simply parabolic or doubly parabolic Baker domain $U$ in $I_e(f)$, where $e$ is any prescribed periodic essential itinerary $e\in\{0,\infty\}^\N$.
%%%
%%%\begin{proof}[Proof of Theorem \ref{thm:baker-domains}] 
%%%Let $N\in \N$ be the period of $e$ and let $p,q\in\N$ denote, respectively, the number of symbols $0$ and $\infty$ in the sequence $e_0e_1\hdots e_{N-1}$, where $p+q=N$. We modify the proof of Theorem~\ref{thm:baker-domains-entire} to obtain a transcendental self-map of $\C^*$ of the form
%%%$$
%%%f(z):=z^n\exp(g(z)z^{4N}+h(1/z)/z^{4N})
%%%$$
%%%that has a hyperbolic Baker domain $U$ in $I_e(f)$, where the entire functions $g, h$~will be constructed using approximation theory. 
%%%
%%%We start by defining a collection of $p$ sets whose closure in $\CR$ contains infinity. Put $\omega_{p}:=e^{2\pi i/p}$ and define
%%%$$
%%%V_{m}^\infty:=\omega_{p}^m\sqrt[p]{W}\subseteq \C\setminus \overline{D(0,r)} \quad \mbox{ for } 0\leqslant m<p,
%%%$$
%%%where $W$ is the unbounded closed set from Lemma~\ref{lem:approx-BD} and $r:=1+(\sqrt[N]{2}-1)/6$. We denote by $V_\infty$ the union of all $V_m^\infty$, $0\leqslant m<p$.
%%%
%%%As before, we define a $B_\infty$ that will be contained in immediate basins of attraction of $f$ and put $R_\infty=\R_-$, if $p$ is odd, or $R_\infty=\{z\in\C^*\,:\,\textup{arg}\,z=\pi(1-1/p)\}$, if $p$ is even. Let
%%%$$
%%%d_\infty:=\min\{(\sqrt[N]{2}-1)/3,\ \textup{dist}(V_\infty,R_\infty)\},
%%%$$
%%%and define the closed connected set
%%%$$
%%%B_\infty:=\{z\in\C\,:\,\textup{dist}\,(z,V_\infty)\geqslant d/2 \mbox{ and } \textup{dist}\,(z,R_\infty)\geqslant d/2\}\setminus D(0,r),
%%%$$
%%%that compactly contains the disc $B_\infty':=\overline{D(1+(\sqrt[n]{2}-1)/2,(\sqrt[n]{2}-1)/6)}$. Finally, we denote $D:=D(0,1/r)$. We will construct $g$ by approximating it on the closed set $F_\infty:=V_\infty\cup B_\infty\cup D$ that satisfies the hypothesis of Lemma \ref{lem:approx-sectors}; namely $\CR\setminus F_\infty$ is connected and $\CR \setminus F_\infty$ is locally connected at infinity (see Figure~\ref{fig:sketch-bd-cstar-1side}).
%%%
%%%\begin{figure}[h!]
%%%\centering
%%%%\vspace{30pt}
%%%\def\svgwidth{.45\linewidth}
%%%\input{bd-cstar-3.pdf_tex}
%%%%\vspace{15pt}
%%%\caption{Sketch of the construction of the function $\hat{f}$ in the proof of Theorem~\ref{thm:baker-domains} with $e=\overline{\infty\infty00\infty}$.}
%%%\label{fig:sketch-bd-cstar-1side} 
%%%\end{figure}
%%%
%%%Similarly, we define sets $V_0$ and $B_0$ by using the same procedure as above, replacing $p$ by $q$, and then put $F_0:=V_\infty\cup B_\infty\cup D$.
%%%
%%%In order to define the functions $\hat{g}\in A(F_\infty)$ and $\hat{h}\in A(F_0)$ that we will approximate, we first need to introduce some notation to describe how $\hat{g},\hat{h}$ map the components of $V_\infty,V_0$, respectively. Let $\pi:\{0,\hdots,N-1\}\to \{-q,\hdots,-1,1,\hdots,p\}$ denote the function given by, for $0\leqslant k<N$,
%%%$$
%%%\pi (k):=\left\{
%%%\begin{array}{ll}
%%%\#\{e_j=\infty\ :\ 0\leqslant j<k\}+1, & \mbox{ if } e_j=\infty,\vspace{5pt}\\
%%%-\#\{e_j=0\ :\ 0\leqslant j<k\}-1, & \mbox{ if } e_j=0.\\
%%%\end{array}
%%%\right.
%%%$$
%%%The function $\pi$ is an ordering of the components of $V_0$ and $V_\infty$ according to the sequence $e$. Let $V$ be the starting component; that is, $V=V_0^0$, if $e_0=0$, and $V=V_0^\infty$, if $e_0=\infty$. Then $f^k(V)\subseteq V_{\pi(k)}^\infty$, if $\pi(k)>0$, and $f^k(V)\subseteq V_{-\pi(k)}^0$, if $\pi(k)<0$. Then, for $j\in \{-q,\hdots,-1,1,\hdots ,p\}$, we define the function
%%%$$
%%%s(j):=\pi(\pi^{-1}(j)+1 \pmod{N})
%%%$$
%%%that describes the image of the component $V_j^\infty$, if $j>0$, and $V_j^0$, if $j<0$, so that the function $f$ to be constructed has a Baker domain that has essential itinerary $e$. More formally, 
%%%\begin{itemize}
%%%\item for $0\leqslant m<p$, $f(V_m^\infty)\subseteq V_{s(m)}^\infty$, if $s(m)>0$, and $f(V_m^\infty)\subseteq V_{-s(m)}^0$, if $s(m)<0$; 
%%%\item for $0\leqslant m<q$, $f(V_m^0)\subseteq V_{s(-m)}^\infty$, if $s(-m)>0$, and $f(V_m^\infty)\subseteq V_{-s(-m)}^0$, if $s(-m)<0$.
%%%\end{itemize}
%%%
%%%We give the details of the construction of the function $g$ from the function $\hat{g}\in A(F_\infty)$. For $z\in V_m^\infty$, $0\leqslant m<p$, we put 
%%%$$
%%%\hat{g}(z):=\left\{\begin{array}{ll}
%%%\left(\log \left(\omega_{p}^{s(m)}\sqrt[p]{\lambda (z/\omega_{p}^m)^{p}}\right)-n\log z\right)/z^{4N} & \mbox{ if } s(m)>0,\vspace{5pt}\\
%%%\left(\log \left(\omega_{p}^{s(m)}/\sqrt[p]{\lambda (z/\omega_{p}^m)^{p}}\right)-n\log z\right)/z^{4N} & \mbox{ if } s(m)<0,
%%%\end{array}\right.
%%%$$
%%%for $z\in B_\infty$, we put $\hat{g}(z):=(\log(1+(\sqrt[n]{2}-1)/2)-n\log z)/z^{4N}$ and, for $z\in D$, we put $\hat{g}(z):=1$ (see Figure~\ref{fig:sketch-bd-cstar-1side}). For $r>0$, we define the positive continuous function $\varepsilon_\infty$ by
%%%$$
%%%\varepsilon_\infty(r):=\min\{d_\infty,\ R_\infty^{-p-1},\ r^{-p-1}\} /r^{4N}
%%%$$
%%%where the constant $R_\infty>0$ is sufficiently large that, if $\delta_{p}(r)$ is the function from Lemma~\ref{lem:approx-root-BD}, for $r\geqslant R$, 
%%%$$
%%%\varepsilon_\infty(r)r^{4N}<\delta_{p}(\ln (\lambda r))\sim \frac{2(\lambda-1)}{p\lambda^pr^p}.
%%%$$
%%%Since $\varepsilon_\infty$ satisfies that
%%%$$
%%%\int_1^{+\infty} r^{-3/2}\ln\varepsilon_\infty(r)dt>-\infty,
%%%$$
%%%by Lemma \ref{lem:approx-sectors}, there is an entire function~$g$ such that
%%%\begin{equation}
%%%|g(z)-\hat{g}(z)|<\varepsilon_\infty(|z|)\quad\mbox{ for all } z\in F_\infty.
%%%\label{eq:bd-entire-approx}
%%%\end{equation}
%%%
%%%Similarly, we can construct an entire function $h$ that approximates a function $h\in A(F_\infty)$ so that the function
%%%$$
%%%\begin{array}{rl}
%%%f(z):=\hspace{-6pt}&z^n\exp(g(z)z^{4N}+h(1/z)/z^{4N})\vspace{5pt}\\
%%%=\hspace{-6pt}&z^n\exp(\hat{g}(z)z^{4N}+\hat{h}(1/z)/z^{4N})\exp((g(z)-\hat{g}(z))z^{4N})\exp((h(z)-\hat{h}(z))/z^{4N})
%%%\end{array}
%%%$$
%%%has the desired properties. Since $|h(1/z)|<1$ for $z\in D$, the contribution of the term $h(1/z)/z^{4N}$ is smaller than $1/r^{4N}$ which is much smaller than $\varepsilon_\infty(r)$. Therefore, by Lemma \ref{lem:approx-root-BD}, each component of the set $V_0\cup V_\infty$ is contained in an $N$-periodic Baker domain and the sets $B_\infty$ and $B_0$ are contained in the immediate basins of attraction of two fixed points that lie in $B_\infty'$ and $B_0'$, respectively (see Figure~\ref{fig:sketch-bd-cstar-2sides}).
%%%
%%%
%%%
%%%\begin{figure}[h!]
%%%\centering
%%%%\vspace{30pt}
%%%\def\svgwidth{.45\linewidth}
%%%\input{bd-cstar-2.pdf_tex}
%%%%\vspace{15pt}
%%%\caption{Sketch of the construction of the function $f$ in the proof of Theorem~\ref{thm:baker-domains} with $e=\overline{\infty\infty00\infty}$.}
%%%\label{fig:sketch-bd-cstar-2sides} 
%%%\end{figure}
%%%
%%%
%%%
%%%
%%%
%%%%Let $N$ be the period of the essential itinerary $e$ and let $p,q$ denote the number of $\infty$ and $0$ symbols in a period, $N=p+q$. Consider the following domains, which will correspond to a function with a hyperbolic, simply parabolic or doubly parabolic Baker domain respectively: for $0\leqslant l<p$, $k>0$,
%%%%\begin{enumerate}
%%%%\item[(a)] $H_{1,p}^l:=\{z\in\C\ :\ z=\omega_p^l\sqrt[p]{w} \mbox{ with } \Re w>k\}$, 
%%%%\item[(b)] $H_{2,p}^l:=\{z\in\C\ :\ z=\omega_p^l\sqrt[p]{w} \mbox{ with } \Re w>k\}$, 
%%%%\item[(c)] $H_{3,p}^l:=\{z\in\C\ :\ z=\omega_p^l\sqrt[p]{w} \mbox{ with } \Im w>1/(\Re w-k)\}$,
%%%%\end{enumerate}
%%%%where $\omega_p:=e^{2\pi i/p}$. Consider, for each of the cases, the following functions defined on $H_{i,p}:=\bigcup_k H_{i,p}^k$:
%%%%\begin{enumerate}
%%%%\item[(a)] $g_{1,p}(z):=\log\sqrt[p]{\lambda}z-n\log z$, 
%%%%\item[(b)] $g_{2,p}(z):=\log\sqrt[p]{z^p+1}-n\log z$, 
%%%%\item[(c)] $g_{3,p}(z):=\log\sqrt[p]{z^p+i}-n\log z$. 
%%%%\end{enumerate}
%%%%By Lemma \ref{lem:approx-root-BD} and Lemma \ref{lem:approx-sectors}, we can find an entire function $g$ such that, for all $0\leqslant l<p$,\margin{finish..}
%%%%$$
%%%%\left\{
%%%%\begin{array}{lll}
%%%%|g(z)-\omega_p^{s(l)}g_{j,p}(z)|<\varepsilon_k(|z|), & \mbox{ on } H_{j,p}^l,& \mbox{ if } e_n=\infty,\vspace{10pt}\\
%%%%|g(z)-1/\bigl(\omega_p^{s(l)}g_{j,p}(z)\bigr)|<\varepsilon_j(|z|), & \mbox{ on } H_{j,p}^l, & \mbox{ if } e_n=0.
%%%%\end{array}
%%%%\right.
%%%%$$
%%%%where $s(n):=p(p^{-1}(n)+1 \pmod{N})$ and the map $p:\{0,\hdots,N-1\}\longrightarrow \{-q,\hdots, p\}\setminus\{0\}$ is defined by 
%%%%$$
%%%%p(n)\margin{modify}:=\left\{
%%%%\begin{array}{ll}
%%%%\begin{array}{l}
%%%%\ds\min\bigl\{\min_{m<n} p^{m+1}(0),\ 0\bigr\}-1=\vspace{5pt}\\
%%%%\ds \hspace{20pt}=-\#\{0\leqslant m<n \pmod{N}\ :\ e_m=0\}-1,
%%%%\end{array} & \mbox{ if } e_{n}=0,\vspace{5pt}\\
%%%%\begin{array}{l}
%%%%\ds\max\bigl\{\max_{m<n} p^{m+1}(0),\ 0\bigr\}+1=\vspace{5pt}\\
%%%%\ds\hspace{20pt}=\#\{0\leqslant m<n\pmod{N}\ :\ e_m=\infty\}+1,
%%%%\end{array} & \mbox{ if } e_{n}=\infty,
%%%%\end{array}
%%%%\right.
%%%%$$
%%%%for all $n\in \N$. Note the similarity between $s$ and \eqref{eq:p-function} in the proof of Theorem \ref{thm:wandering-domains}, again $s$ denotes the subsequent domain that has not been used.
%%%%
%%%%Lemma \ref{lem:approx-root-BD}\margin{in case it\\ works...} ensures that $z^n\exp\bigl( g(H_{i,p})\bigr)\subseteq H_{i,p}$ and hence $H_{i,p}\subseteq F(f_{i,p})$ is a Baker domain.
%%%%
%%%%
%%%%
%%%%$$
%%%%\xymatrix{
%%%%\C \ar[r]^{z+1} \ar[d]_{z^p} & \C\\
%%%%\C \ar[r]_{z^ne^{g(z)}}  & \C \ar[u]_{\sqrt[p]{z}}\\
%%%%}
%%%%$$
%%%%
%%%%
%%%
%%%Finally, the same arguments from the proof of Theorem~\ref{thm:baker-domains-entire} show that the Baker domain $U$ is hyperbolic, and hence we skip the details.
%%%\end{proof}
%%%
%%%\begin{rmk}
%%%The discussion about simply parabolic and doubly parabolic Baker domains is left for the paper...
%%%\end{rmk}
%%%
%%%%\begin{rmk}
%%%%Since the construction of the Baker domains in Theorems .... takes place in a sector, it is possible to obtain a function that has all the types of Baker domains.
%%%%\end{rmk}
%%%%
%%%%- Check univalent/not univalent\\
%%%%
%%%%\begin{dfn}
%%%%We say that a Baker domain $U$ has \textit{width} $\alpha\in (0,2\pi)$ if there is a translate of $U$ that contains a sector 
%%%%$$
%%%%S_\alpha:=\{z\in \C\ :\ |\mbox{arg}\, z|<\alpha/2\}.
%%%%$$
%%%%\end{dfn}
%%%%
%%%%The Baker domains constructed in Theorem \ref{thm:hyp-baker-domains} all have width less than $\pi$, but the next result allows us to build Baker domains of any width.
%%%%
%%%%\begin{cor}
%%%%Let $\alpha\in (0,2\pi)$, for every $e\in \{0,\infty\}^\N$ and $n\in \N$, there exists a transcendental self-map of $\C^*$ $f$ with $\textup{ind}(f)=n$ such that $I_e(f)$ contains a hyperbolic, simply parabolic or doubly parabolic Baker domain of width $\alpha$.
%%%%\end{cor}
%%%%\begin{proof}
%%%%If $\alpha<\pi$ then this is covered by Theorem \ref{thm:hyp-baker-domains}. Suppose that $\pi\leqslant \alpha<2\pi$. Then taking the square root and using Lemma \ref{lem:approx-root-BD} and Theorem \ref{thm:hyp-baker-domains} we can construct an entire function $g$ such that
%%%%$$
%%%%\begin{array}{ll}
%%%%|g(z)-g_{j,2}(z)|<\varepsilon_j(|z|), & \mbox{ for all } z\in H_{j,2},\vspace{10pt}\\
%%%%|g(z)-g_{j,2}(-z)|<\varepsilon_j(|z|), & \mbox{ for all } z\in -H_{j,2},
%%%%\end{array}
%%%%$$
%%%%where $g_{j,2}(z)=\sqrt{g_j(z^2)}$ and $H_{j,2}=\sqrt{H_j}$ following the notation of Theorem \ref{thm:hyp-baker-domains}.
%%%%
%%%%Let $f(z):=\bigl(g(z)+g(-z)\bigr)/2$ and note that $f(z)=g(z)$ if $z\in H$. The function $f$ is entire and it approximates $g$ on $H$. Observe that the function $f$ has a Baker domain of width $\alpha/2$. 
%%%%
%%%%Finally let $F(z):=f(\sqrt{z})$ which is well defined and entire because $f$ is an even function and therefore only has even powers in its Taylor expansion. The function $F$ has a Baker domain of width $\alpha$.
%%%%\end{proof}
%%%
%%%%\section{Entire functions with escaping Fatou components}
%%%%
%%%%%There are two ways that we can associate an entire function to a transcendental self-map of $\C^*$: periodic function (lift) and entire function with no zeros ($\exp(g(z))$)...
%%%%
%%%%\begin{proof}[Proof of Theorem \ref{thm:entire-lift}]
%%%%.\margin{Is it worth?}
%%%%\end{proof}
%%%%
%%%%%\begin{proof}[Proof of Theorem \ref{thm:entire-nozeros}]
%%%%%.
%%%%%\end{proof}
%%%
%%%%\section{Doubly connected escaping Fatou components}
%%%%
%%%%...


\section{Explicit functions with wandering domains}

\label{sec:explicit-wd}

As mentioned in the introduction, the author is not aware of any previous explicit examples of transcendental self-maps of $\C^*$ with wandering domains or Baker domains as all such functions were constructed using approximation theory.

%\begin{dfn}[Wandering domain]
%Let $f$ be a transcendental self-map of $\C^*$. Let $U$ be a Fatou component of $f$ and let $U_n$, $n\in\N$, denote the Fatou component containing $f^n(U)$. We say that $U$ is a \textit{wandering domain} if, for all $m,n\in\N$, $U_m=U_n$ if and only if $m=n$.
%\end{dfn}

Kotus \cite{kotus90} showed that transcendental self-maps of $\C^*$ can have escaping wandering domains by constructing examples of such functions using approximation theory. Here we give an explicit example of such a function by modifying a transcendental entire function that has a wandering domain. 

\begin{ex}
The function $f(z)=z\exp\bigl(\frac{\sin z}{z}+\frac{2\pi}{z}\bigr)$ is a transcendental self-map of~$\C^*$ which has a bounded simply connected wandering domain that escapes to infi\-nity (see Figure~\ref{fig:wand-domain}).
\label{ex:wand-domain}
\end{ex}

\begin{figure}[h!]
%\includegraphics[width=.49\linewidth]{wd001-9-small.png}
\includegraphics[width=.49\linewidth]{002-dot.png}
\includegraphics[width=.49\linewidth]{wd001-12-2.png}
\caption[Phase space of a transcendental self-map of $\C^*$ which has a wandering domain]{Phase space of the function $f(z)=z\exp\left(\frac{\sin z}{z}+\frac{2\pi}{z}\right)$ from Example~\ref{ex:wand-domain} which has a wandering domain. On the right, the wandering domain for large values of $\textup{Re}\, z$.}
\label{fig:wand-domain}
\end{figure}

Baker \cite[Example~5.3]{baker84} (see also \cite[Example~2]{rippon-stallard08}) studied the dynamics of the trans\-cendental entire function $f_1(z)=z+\sin z+2\pi$ that has a wandering domain containing the point $z=\pi$ that escapes to infinity. Observe that the function $f$ from Example~\ref{ex:wand-domain} satisfies that%Devaney \cite{devaney89}
\begin{equation}
f(z)=z+\sin z+2\pi+o(1)\quad \mbox{ as } \mbox{Re}\,z\rightarrow +\infty
\label{eq:ex-wand-domain}
\end{equation}
in a horizontal band defined by $|\textup{Im}\, z|<K$ for some $K>0$.

We first prove a general result which gives a sufficient condition that implies that a function has a bounded wandering domain (see Figure~\ref{fig:annuli-lemma}) using some of the ideas from \cite[Lemma~7(c)]{rippon-stallard08}. Given a doubly connected open set $A\subseteq \C$, we define the \textit{inner boundary}, $\partial_\textup{in}A$, and the \textit{outer boundary}, $\partial_\textup{out}A$, of $A$ to be the boundary of the bounded and unbounded complementary components of $A$ respectively.%% The annulus $A$ can be mapped conformally onto an annulus of the form $A_R:=\{z\in\C\ :\ 1<|z|<R\}$, $R\in (1,+\infty]$, in a unique way. Then, the \textit{modulus}\index{modulus of an annulus} of $A$ is defined as
%$$
%\textup{mod}(A):=\left\{
%\begin{array}{rl}
%\frac{1}{2\pi}\log R & \mbox{ if } R<+\infty,\vspace{5pt}\\
%+\infty & \mbox{ if } R=+\infty.
%\end{array}\right.
%$$

%\begin{lem}
%%Let $f$ be a function that is holomorphic on a domain $G\subseteq \C$, let $M$ be an orientation-preserving isometry of $\C$ and let $A\subseteq G$ be a doubly connected closed set such that 
%Let $f$ be a function that is holomorphic on a domain $G\subseteq \C$, let $M$ be an affine map and let $A\subseteq G$ be a doubly connected closed set such that 
%$$
%A_n:=M^n(A)\subseteq G\quad \mbox{ for all } n\in\N
%$$
%and the closures of the bounded complementary components of the annuli\linebreak $\{A_n\}_{n\in\N}$ are pairwise disjoint. Suppose that there exists $K>1$ such that,\linebreak for all $n\in\N$,
%\begin{itemize}
%\item $f(\partial_\textup{in}A_n)$ lies in the bounded complementary component of $A_{n+1}$ and the modulus of the annulus defined by $f(\partial_\textup{in}A_n)$ and $\partial_\textup{in}A_{n+1}$ is bounded below by $K$;
%\item $f(\partial_\textup{out}A_n)$ lies in the unbounded complementary component of $A_{n+1}$.
%\end{itemize}
%Then $f$ has wandering domains $\{U_n\}_{n\in\N}$ such that $\partial_\textup{in}A_n\subseteq U_n$ and $\partial U_n\subseteq A_n$ for all $n\in\N$.
%\label{lem:Julia-in-annulus}
%\end{lem}
%
%\begin{lem}
%Let $f$ be a function that is holomorphic on a domain $G\subseteq \C$ and let $\{A_n\}_{n\in\N}$ be a sequence of doubly connected closed sets in $G$ such that the closures of their bounded complementary components are pairwise disjoint and there exist constants $C_1,C_2>0$ such that 
%$$
%\textup{diam}(\partial_\textup{in}A_n)<C_1 \quad \mbox{ and } \quad \textup{dist}(\partial_\textup{in}A_n,\partial_\textup{out}A_n)>C_2\quad \mbox{ for all } n\in\N.
%$$
%Suppose that, for all $n\in\N$,
%\begin{itemize}
%\item $f(\partial_\textup{in}A_n)$ lies in the bounded complementary component of $A_{n+1}$;
%\item $f(\partial_\textup{out}A_n)$ lies in the unbounded complementary component of $A_{n+1}$.
%\end{itemize}
%Then $f$ has wandering domains $\{U_n\}_{n\in\N}$ such that $\partial_\textup{in}A_n\subseteq U_n$ and $\partial U_n\subseteq A_n$ for all $n\in\N$.
%\label{lem:Julia-in-annulus}
%\end{lem}

\begin{lem}
%Let $f$ be a function that is holomorphic on a domain $G\subseteq \C$, let $M$ be an orientation-preserving isometry of $\C$ and let $A\subseteq G$ be a doubly connected closed set such that 
Let $f$ be a function that is holomorphic on $\C^*$, let $M$ be an affine map, let $A$ be a doubly connected closed set in $\C^*$ with bounded complementary component $B$, and let $C\subseteq B$ be compact. Put
$$
A_n:=M^n(A),\quad B_n:=M^n(B)\quad \mbox{ and }\quad C_n:=M^n(C)\quad \mbox{ for } n\in\N_0,
$$
and suppose that 
\begin{itemize}
\item $A_n\cup B_n\subseteq \C^*$ for $n\in\N_0$,
\item the sets $\{B_n\}_{n\in\N_0}$ are pairwise disjoint,
\item $f(\partial_\textup{in}\,A_n)\subseteq C_{n+1}$ for $n\in\N_0$,
\item $f(\partial_\textup{out}\,A_n)\subseteq \C^*\setminus (A_{n+1}\cup B_{n+1})$ for $n\in\N_0$.
\end{itemize}
Then $f$ has bounded simply connected wandering domains $\{U_n\}_{n\in\N_0}$ such that
$$
\partial_\textup{in}\,A_n\subseteq U_n \quad \mbox{ and } \quad \partial U_n \subseteq A_n \quad \mbox{ for } n\in\N_0.
$$
\label{lem:Julia-in-annulus}
\end{lem}


\begin{figure}[h!]
\centering
\vspace*{-10pt}
\def\svgwidth{\linewidth}
\input{annuli-lemma.pdf_tex}
\vspace{15pt}
\caption[Sketch of a construction which implies that a function has a wandering domain]{Sketch of the construction in Lemma \ref{lem:Julia-in-annulus}.}
\label{fig:annuli-lemma} 
\end{figure}


In order to prove this lemma, we first need the following result on limit functions of holomorphic iterated function systems by Keen and Lakic \cite[Theorem~1]{keen-lakic03}. %Let $\Omega\subseteq \C$ be a domain, then $\mbox{Hol}(\D,\Omega)$ denotes the set of all holomorphic functions from $\D$ to $\Omega$.

\begin{lem}
Let $X$ be a subdomain of the unit disc $\mathbb D$. Then all limit functions of any sequence of functions $(F_n)$ of the form
$$
F_n:=f_n\circ f_{n-1}\circ \cdots \circ f_2\circ f_1\quad \mbox{ for } n\in\N,
$$
where $f_n:\mathbb D\to X$ is a holomorphic function for all $n\in \N$, are constant functions in $\overline{X}$ if and only if $X\neq\mathbb D$.
\label{lem:keen-lakic}
\end{lem}

We now proceed to prove Lemma~\ref{lem:Julia-in-annulus}.

\begin{proof}[Proof of Lemma~\ref{lem:Julia-in-annulus}]
Since $f(\overline{B_n})\subseteq C_{n+1}\subseteq B_{n+1}$, the iterates of $f$ on each set $\overline{B_n}$ omit more than three points and hence, by Montel's theorem, the sets $\{\overline{B_n}\}_{n\in\N_0}$ are all contained in $F(f)$. For $n\in \N_0$, let $U_n$ denote the Fatou component of $f$ that contains $\overline{B_n}$. We now show that the functions
$$
\Phi_{k}(z):=M^{-k}(f^k(z))\quad \mbox{ for } k\in\N_0,
$$
form a normal family in $U_n$ for all $n\in\N_0$.

Suppose first that the Fatou components $\{U_n\}_{n\in\N_0}$ are not distinct. Then there are two sets $B_m$ and $B_{m+p}$ with $m\in\N_0$ and $p>0$ which lie in the same Fatou components $U_m=U_{m+p}$. Then, since $f^p(B_m)\subseteq B_{m+p}$ and $B_n\to \infty$ as $n\to \infty$, $U_m$ must be periodic and in $I(f)$, and hence a Baker domain.

Let $z_m\in B_m$ and let $K$ be any compact connected subset of $U_m$ such that $K\supseteq B_m$. Then by Baker's distortion lemma (see \cite[Lemma~6.2]{martipete1} or \cite[Lemma~2.22]{martipete} for a proof of the version of this result that we use here), there exist constants \mbox{$C(K)>1$} and $n_0\in\N_0$ such that 
$$
|f^k(z)|\leqslant C(K) |f^k(z_m)|\quad \mbox{ for } z\in K,\ k\geqslant n_0.
$$
Since $M$, and hence $M^{-k}$, is an affine transformation, $M^{-k}$ preserves the ratios of distances, so
$$
|\Phi_k(z)|=|M^{-k}(f^k(z))|\leqslant C(K)|M^{-k}(f^k(z_m))|=C(K)|z_m'|
$$
where $z_m'\in B_m$ satisfies $M^k(z_m')=f^k(z_m)$. Hence the family $\{\Phi_k\}_{k\in \N_0}$ is locally uniformly bounded on $U_m$, and hence is normal on $U_m$.

Suppose next that the Fatou components $\{U_n\}_{n\in\N_0}$ are disjoint. In this case we consider the sequence of functions 
$$
\varphi_k(z):=M^{-(k+1)}(f(M^k(z)))\quad \mbox{ for } k\in\N_0,
$$
which are defined on $U_n$, for $n\in\N_0$. Then
\begin{equation}
\Phi_k(z) = (\varphi_{k-1} \circ \cdots \circ  \varphi_1 \circ \varphi_0)(z) = M^{-k}(f^k(z))\quad \mbox{ for } k\in\N_0.
\label{eq:annuli-lem-1}
\end{equation}
Since the Fatou components $\{U_n\}_{n\in\N_0}$ are pairwise disjoint and
$$
f^k(U_n)\subseteq U_{n+k},
$$
we deduce that
$$
f^k(U_n)\cap B_{n+k+1}=\emptyset
$$
and hence
$$
\Phi_k(U_n)\cap B_{n+1}=\emptyset\quad \mbox{ for } k,n\in\N_0.
$$
Thus $\{\Phi_k\}_{k\in\N_0}$ is normal on each $U_n$, by Montel's theorem, as required.

Now take $n\in\N_0$, and let $\{\Phi_{k_j}\}_{j\in\N_0}$ be a locally uniformly convergent subsequence of $\{\Phi_k\}_{k\in\N_0}$ on $B_n$. Note that
$$
M^k(B_n)=B_{n+k} \quad \mbox{ so } \quad f(M^k(B_n))\subseteq C_{n+k+1}
$$
and hence, for $k\in \N_0$, 
$$
\varphi_k(B_n)=M^{-(k+1)}(f(M^k(B_n)))\subseteq M^{-(k+1)}(C_{n+k+1})=C_n.
$$
We can now apply Lemma~\ref{lem:keen-lakic}, after a Riemann mapping from $B_n$ to the open unit disc $\mathbb D$, to deduce from \eqref{eq:annuli-lem-1} that there exists $\alpha_n\in \overline{B_n}$ such that, for all $z\in U_n$,
$$
\Phi_{k_j}(z)\to\alpha_n\quad \mbox{ as } j\to\infty.
$$

%%We now show that the Fatou components $U_n$, $n\in\N_0$, are all different. Suppose to the contrary that two sets $B_m$ and $B_{m+p}$ with $m\in\N_0$ and $p>0$ are contained in the same Fatou component $U_m=U_{m+p}$. Then, since $f^p(B_m)\subseteq B_{m+p}$ and $B_n\to\infty$ as $n\to\infty$, $U_m$ has to be a Baker domain. Observe that there exists $n_0\in\N_0$ such that, for $n\geqslant n_0$, the sets $B_n$ are contained in a simply connected absorbing domain $V\subseteq U$ where $f$ is univalent. For $n\in\N$, let $G_n\subseteq B_n$ denote the annulus defined by the curves $f(\partial_\textup{in}A_{n-1})$ and $\partial_\textup{in}A_n$ which, by assumption, satisfies that $\textup{mod}(G_n)>K$. Now, for $n>n_0$, let $G_n'$ denote the preimage of $G_n$ under $f^{-(n-n_0)}$ that lies inside $A_{n_0}$. Since the function is univalent, the sets $G_n'$, $n>n_0$, are all disjoint and have modulus at least $K$. But this contradicts the fact that the modulus of $A_{n_0}$ is finite and hence the Fatou components $U_n$, $n\in\N_0$, are pairwise disjoint.



%By Lemma~\ref{lem:bd-koenig}, the map $T$ is either $T_1(z)=\lambda z$ with $\lambda>1$, $T_2(z)=z\pm i$ or $T_3(z)=z+1$. But the functions $T_k$ all map part of a circle outside itself and hence this contradicts the fact that $f^p(\partial B_m)\subseteq B_{m+p}$. Therefore the Fatou components $U_m$, $m\in\N$, are pairwise disjoint and hence are wandering domains.

%%In order to prove that $\partial U_n\subseteq A_n$ for all $n\in\N$, we consider the sequence of functions
%%$$
%%\varphi_{k}(z):=M^{-(k+1)}(f(M^k(z)))\quad  \mbox{ for } k\in\N
%%$$
%%that are defined on $U_n$, for $n\in\N$. Then, we define the composition sequence
%%$$
%%\Phi_k(z):=\varphi_{k-1}(\varphi_{k-2}(\cdots \varphi_1(\varphi_0(z))\cdots))=M^{-k}(f^k(z)) \quad \mbox{ for } k\in\N.
%%$$
%%Since the Fatou components $\{U_n\}$, $n\in\N$, are pairwise disjoint and $M^k(B_{n+1})=B_{n+k+1}\subseteq U_{n+k+1}$, we have $B_{n+1}\cap \Phi_k(U_n)=\emptyset$ for all $k\in\N$ and, by Montel's theorem, the family $\mathcal F:=\{\Phi_k\}$, $k\in\N$, is normal in $U_n$ for all $n\in\N$. Thus, there exists a sequence $(k_j)$ such that, for all $n\in\N$, $\Phi_{k_j}\to \Phi$ locally uniformly on $U_n$ as $j\to\infty$.
%%
%%Since $\textup{mod}(G_n)>K$ for all $n\in \N$, there exists a domain $\Omega_n \subsetneq B_n$ such that $\phi_k(B_n)\subseteq \Omega_n$ for all $k\in\N$ and hence, by Lemma~\ref{lem:keen-lakic}, there exists $\alpha_n\in\overline{\Omega_n}$ such that, for all $z\in B_{n}$, $\Phi_{k_j}(z)\to \alpha_n$ as $j\to\infty$. And, since $\mathcal F$ is a normal family in $U_n$, for all $z\in U_{n}$, $\phi_{k_j}(z)\to \alpha_n$ as $j\to\infty$.

To complete the proof that $U_n$ is bounded by $\partial_\textup{out}\,A_n$ for all $n\in\N$, suppose to the contrary that there is a point $z_0\in\partial_\textup{out}\,A_n$ that lies in $U_n$ for some $n\in\N$. Let $\gamma\subseteq U_n$ be a curve that joins $z_0$ to a point $z_1\in B_n$. Since $\gamma$ is compact, $\Phi_{k_j}(\gamma)\to \alpha$ as $j\to\infty$ which contradicts the fact that $f^k(\gamma)\cap \partial_\textup{out}\,A_{n+k}\neq\emptyset$ for all $k\in\N$ (this follows from the hypothesis that $f(\partial_\textup{out}\,A_n)\subseteq (A_{n+1}\cup B_{n+1})^c$ for $n\in\N_0$). Thus, $\partial  U_n\subseteq A_n$ for all $n\in\N$, and so the proof is complete.
\end{proof}

We now use Lemma~\ref{lem:Julia-in-annulus} to show that the function $f$ from Example~\ref{ex:wand-domain} has a bounded wandering domain that escapes to infinity along the positive real axis.

\begin{proof}[Proof of Example~\ref{ex:wand-domain}]
%%First of all, observe that since $f(\overline{z})=\overline{f(z)}$, the Fatou set $F(f)$ is symmetric with respect to the real axis. %Moreover, the function $g(z)=z+\sin z$ satis\-fies that $g(z+2\pi)=g(z)+2\pi$ and hence $F(g)$ is $2\pi$-periodic. 
 
%Each point $z_k$ has a bounded basin of attraction $U_k$ under $g$ that is contained in the vertical band $V_k:=\{z\in\C\ :\ z_k-\pi<\Re z<z_k+\pi\}$ and is a~$(2k+1)\pi$-translate of $U_0$. Therefore, we have $\hat{f}^n(U_0)=U_n\subseteq V_n$ and $U=U_0$ is a wandering domain of $\hat{f}$. 

%%The function $g(z)=z+\sin z$ has fixed points at $z=n\pi$, $n\in\Z$, which are superattracting, if $n$ is odd, and repelling, if $n$ is even. Thus, the function $f(z)-2\pi$ has infinitely many fixed points $z_n$ in the positive real axis, which are the solutions of the equation
%%$$
%%\exp\left(\frac{\sin x}{x}+\frac{2\pi}{x}\right)=1+\frac{2\pi}{x} \quad \mbox{ for } x>0
%%$$
%%or, equivalently, $\sin x=o(1)$ as $x\to+\infty$. Hence the points $z_n$ are asymptotically close to $n\pi$ as $n\to \infty$ and are attracting, if $n$ is odd, or repelling, if $n$ is even.

The transcendental entire function $g(z)=z+\sin z$ has\linebreak superattracting fixed points at the odd multiples of $\pi$. For $n\in\N_0$, take\linebreak \mbox{$B_n := D((2n+1)\pi,r)$} and $C_n:=D((2n+1)\pi,r/2)$ for some $r>0$ sufficiently small that \mbox{$g(B_n)\subseteq C_{n}$} and put
$$
R_n:= \{z\in \C\ :\ |\textup{Re}\, z-(2n+1)\pi|\leqslant 3\pi/2,\ |\textup{Im}\, z|\leqslant 3\}.
$$
It follows from a straightforward computation that $g(\partial R_n)\subseteq R_n^c$ for all $n\in\N_0$ (see Figure~\ref{fig:spiral}). 

%\begin{figure}[h!]
%\centering
%\includegraphics[width=.6\linewidth]{Figures/spiral-final.png} 
%\caption{Rectangle $R_0$ and its image under $g(z)=z+\sin z$.}
%\label{fig:spiral}
%\end{figure}

\begin{figure}[h!]
\centering
\def\svgwidth{.65\linewidth}
\input{spiral-final2.pdf_tex} 
\caption[Image of a rectangle by the map~\mbox{$g(z)=z+\sin z$}]{Rectangle $R_0$ and its image under $g(z)=z+\sin z$.}
\label{fig:spiral}
\end{figure}


Then, by \eqref{eq:ex-wand-domain}, there exists $N\in\N_0$ such that $f(B_n)\subseteq C_{n+1}$ and $f(\partial R_n)\subseteq R_{n+1}^c$ for all $n>N$. Thus, we can apply Lemma~\ref{lem:Julia-in-annulus} to $f$ with $M(z)=z+2\pi$ and $A_n:=R_n\setminus B_n$ for $n>N$ and conclude that the function $f$ has wandering domains $U_n$ that contain $B_n$ and whose boundary is contained in $R_n$.
\end{proof}
%%Let $U_k$ be the immediate basin of attraction of the point $z_{2k+1}$ under~$f-2\pi$ for $k\in\Z$ sufficiently large. By \eqref{eq:ex-wand-domain}, for every $\varepsilon>0$, there exists $k_0=k_0(\varepsilon)\in\Z$ such that 
%%$$
%%|f'(z_{2k+1})|<\varepsilon \quad \mbox{ and } \quad  |z_{2k+3}-(z_{2k+1}+2\pi)|<\varepsilon \quad \mbox{ for all } k>k_0.
%%$$ 
%%Let $r>0$ be sufficiently small, then choosing $\varepsilon=\varepsilon(r):=r/(2+2r)$ we have that
%%$$
%%|z_{2k+3}-(z_{2k+1}+2\pi)|+|f'(z_{2k+1})|r<\varepsilon+\varepsilon r=r/2 \quad \mbox{ for all } k>k_0(\varepsilon)
%%$$
%%and hence
%%$$
%%f\bigl(D(z_{2k+1},r)\bigr)\sim D(z_{2k+1},\varepsilon r)+2\pi\subseteq D(z_{2k+3},r/2) \quad \mbox{ for all } k>k_0(\varepsilon).
%%$$
%%Thus, the union of all the discs $D_k:=D(z_{2k+1},r)\subseteq U_k$, for $k>k_0(\varepsilon)$, is contained in $F(f)\cap I(f)$.

%%To prove that each $D_k$ is contained in a wandering domain, we must show that the Fatou components $U_k$ are pairwise disjoint for sufficiently large values of $k\in\Z$. Suppose to the contrary that $U_p=U_q$ for some $p\neq q$. Then there is a Jordan curve $\gamma\subseteq U_p$ which is symmetric with respect to the real axis and such that $\gamma$ intersects $D_p$ and~$D_q$. Such curve would surround a repelling fixed point $z_{2j}$ for some $j\in\Z$ and hence $U_k$ would be doubly connected. However, Baker \cite[Theorem~1]{baker87} showed that the only multiply connected Fatou components in $\C^*$ are doubly connected and must separate zero from infinity. Therefore each $U_n$, $n\in\Z$, is a wandering domain whose iterates escape along the positive real axis.  


%Then, for all $n\in\N$, the bounded complementary component defined by the Jordan curve $f^n(\gamma)$ contains a repelling fixed point of~$f$ that must lie in the real axis by symmetry. But since repelling periodic points are in $J(f)$, this implies that $U_p$ is multiply connected, which contradicts the fact that $f$ can have at most one multiply connected Fatou component which must be doubly connected and separate zero from infinity . 



%%\noindent
%%Therefore, by  \eqref{eq:ex-wand-domain}, we have $f(\gamma_k)-2\pi\subseteq\C\setminus R_k$ and hence $\gamma_k\subseteq \C\setminus U_k$. Thus the sets $U_k$ are bounded for sufficiently large values of $k$.
%%\end{proof}
% the symmetry of $f$. Suppose to the contrary there exist some $n_0\in\N$ such that $U_{n_0}$ is unbounded. Since $f(\overline{z})=\overline{f(z)}$, the Fatou components $U_n$ are symmetric with respect to the real axis. Therefore the set $U_{n_0}$ must contain a curve $\delta$ that is also symmetric with respect to the real axis and splits the plane into two connected components, and the curve $f(\delta)$ must also separate the plane into two components. Thus, if $V$ is the vertical band such that $\partial V=\delta\cup f(\delta)$, the images $f^n(V)$ will all be vertical bands intersecting $U_{n_0+n}$ and $U_{n_0+n+1}$, but this is a contradiction with the blow up property of $J(f)$.



%%%We now provide a second example of a transcendental self-map of $\C^*$ with a wandering domain that, in this case, is unbounded.
%%%
%%%\begin{ex}
%%%The function $f(z)=z\exp\bigl(\frac{e^{-z}-1}{z}+\frac{2\pi i}{z}\bigr)$ is a transcendental self-map of $\C^*$ which has an unbounded wandering domain that escapes to infinity (see Figure~\ref{fig:unbdd-wand-domain}).
%%%\label{ex:wand-domain-unbdd}
%%%\end{ex}
%%%
%%%\begin{figure}[h!]
%%%\includegraphics[width=.49\linewidth]{c05.png}
%%%%\includegraphics[width=.49\linewidth]{b02.png}
%%%\includegraphics[width=.49\linewidth]{c07.png}
%%%\caption{Phase space of the function $f(z)=z\exp\bigl(\frac{e^{-z}-1}{z}+\frac{2\pi i}{z}\bigr)$ from Example~\ref{ex:wand-domain-unbdd}. In the right, the wandering domain for large values of $\Im z$.}
%%%\label{fig:unbdd-wand-domain}
%%%\end{figure}
%%%
%%%The trans\-cendental entire function $f(z)=z-1+e^{-z}+2\pi i$ was studied by Herman (see \cite[Examples~2~and~5.1]{baker84}) who showed that $f$ has an unbounded wandering domain containing the real axis which escapes to infinity along the positive imaginary axis. Herman constructed this function from the lift by $e^{-z}$ of the function $g(z)=eze^{-z}$ which is a holomorphic self-map of $\C^*$ that is entire. Such lift is a $2\pi i$-periodic entire function which has a sequence of unbounded basins of attraction that become a wandering domain when you add $2\pi i$ to the function. Observe that the function $f$ from Example~\ref{ex:wand-domain-unbdd} satisfies that%Devaney \cite{devaney89}
%%%\begin{equation}
%%%f(z)=z-1+e^{-z}+2\pi i+o(1)\quad \mbox{ as } \mbox{Re}\,z\rightarrow +\infty
%%%\label{eq:ex-wand-domain}
%%%\end{equation}
%%%in a right half plane.
%%%
%%%\begin{proof}[Proof of Example~\ref{ex:wand-domain-unbdd}]
%%%Similar arguments to those in the proof of Example~\ref{ex:wand-domain} show that the function $f-2\pi i$ has a sequence of attracting fixed points $z_n$, $n\in\N$, near the imaginary axis that are asymptotically close to $z=2n\pi i$ as $n\to \infty$. There exists $r>0$ such that each point $z_n$ is contained in the disk $B_n:=D(z_n,r)$ and $f(B_n)\subseteq B_{n+1}$ for all sufficiently large values of $n\in\N$. Hence the union of all $D_k$ is in $F(f)$. Moreover, each disc $D_k$ is in a different Fatou component of $f$, and are thus part of a wandering domain~$U$.
%%%
%%%To prove that the wandering domain $U$ is unbounded, we observe that, for sufficiently large values of $k\in\Z$, if we define the band
%%%$$
%%%B_k:=\{z\in\C\ :\ \Re z>0,\ |\Im z -2k\pi|<\varepsilon\}%t+2k\pi i, \quad t\in [0,+\infty),
%%%$$
%%%then the set $D_k\cup B_k$ is connected and it is contained in the basin of attraction of $z_k$. Indeed, if $z=x+(2k\pi \pm\varepsilon)i$ with $x>0$, $k\in\Z$ and $\varepsilon>0$ sufficiently small, then 
%%%$$
%%%\Im (f(z))-2\pi=(2k\pi\pm\varepsilon)+e^{-x}\sin(-2k\pi\mp\varepsilon)+o(1) \quad \mbox{ as } k\to+\infty,
%%%$$
%%%where $e^{-x}\sin(-2k\pi-\varepsilon)<0$ and $e^{-x}\sin(-2k\pi+\varepsilon)>0$. Hence, for a given value of $\varepsilon$, $f(D_k\cup B_k)-2\pi\subseteq D_k\cup B_k$ for sufficiently large values of $k\in\Z$, and therefore the wandering domain $U$ is unbounded.
%%%\end{proof}

The next lemma relates the wandering domains of a transcendental self-map of~$\C^*$ and a lift of it.

\begin{lem}
Let $f$ be a transcendental self-map of $\C^*$ and let $\tilde{f}$ be a lift of $f$. Then, if $U$ is a wandering domain of $f$, every component of $\exp^{-1}(U)$ is a wandering domain of $\tilde{f}$ which must be simply connected.
\label{lem:semiconj-wd}
\end{lem}
\begin{proof}
By a result of Bergweiler \cite{bergweiler95}, every component of $\exp^{-1}(U)$ is a Fatou component of $\tilde{f}$. Let $U_0$ be a component of $\exp^{-1}(U)$ and suppose to the contrary that there exist $m,n\in\N_0$, $m\neq n$, and a point $z_0\in\tilde{f}^m(U_0)\cap \tilde{f}^n(U_0)$. Then, there exists points $z_1,z_2\in U_0$ such that \vspace*{-5pt}
$$
f^m(e^{z_1})=\exp \tilde{f}^m(z_1)=\exp z_0=\exp \tilde{f}^n(z_2)=f^n(e^{z_2}).\vspace*{-5pt}
$$
Since $e^{z_1},e^{z_2}\in U$, this contradicts the assumption that $U$ is a wandering domain of $f$. Hence $U_0$ is a wandering domain of $\tilde{f}$. 

Finally, by \cite[Theorem~1]{baker87}, the Fatou component $U$ is either simply connected or doubly connected and surrounds the origin. Since the exponential function is periodic, taking a suitable branch of the logarithm one can show that the components of $\exp^{-1}(U)$ are simply connected.
% Suppose to the contrary that a component $\tilde{U}$ of $\exp^{-1}(U)$ is multiply connected. Then, by \cite[Theorem~3.1]{baker84}, all the images $\tilde{f}^n(\tilde{U}),\ n\in\N$, are multiply connected and escape to infinity. Baker also showed that for large values of $n$, the components $\tilde{f}^n(\tilde{U})$ surround the origin.
%  Since each $\tilde{f}^n(\tilde{U})$ is bounded, the sets $\exp(\tilde{f}^n(\tilde{U}))$ surround zero for all $n\in\N$. By \cite[Theorem~1]{baker87}, the set $F(f)$ can only have one doubly connected component, so all the sets $\tilde{f}^n(\tilde{U})$ are mapped to the same Fatou component of $f$ by the exponential function. But this contradicts \cite{bergweiler95} because the Julia set in between the sets $\tilde{f}^n(\tilde{U})$ would be mapped to $F(f)$.
\end{proof}

\begin{rmk}
Observe that the converse of Lemma~\ref{lem:semiconj-wd} does not hold. If $f$ is a transcendental self-map of $\C^*$ with an attracting fixed point $z_0$ and $A$ is the immediate basin of attraction of $z_0$, then there is a lift $\tilde{f}$ of $f$ such that a component of $\exp^{-1}(A)$ is a wandering domain.
\end{rmk}

If a transcendental self-map of $\C^*$ has an escaping wandering domain, then we can use the previous lemma to obtain automatically an example of a transcendental entire function with an escaping wandering domain.

\begin{ex}
The transcendental entire function $\tilde{f}(z)=z+\frac{\sin e^z}{e^z}+\frac{2\pi}{e^z}$, which is a lift of the function $f$ from Example~\ref{ex:wand-domain}, has infinitely many grand orbits of bounded wandering domains that escape to infinity.
\end{ex}

%%%\begin{ex}
%%%The transcendental entire function $\tilde{f}(z)=z+\frac{e^{-z}}{z}+\frac{2\pi i}{z}$, which is a lift of the function $f$ from Example~\ref{ex:wand-domain-unbdd}, has infinitely many grand orbits of unbounded wandering domains whose points escape to infinity.
%%%\end{ex}

 
%\begin{center}
%\begin{tabular}{cc}
%%\includegraphics[width=130pt]{standard.png} & \includegraphics[width=130pt]{standard2.png}\vspace{2pt}\\
%\includegraphics[width=140pt]{wd001-9-small.png} & \includegraphics[width=140pt]{wd001-12.png}\vspace{2pt}\\
%%$\alpha=3.1,\ \beta=0.8.$ & $\alpha=3.1,\ \beta=5$.\\
%\end{tabular}
%\end{center}
%\end{frame}

\section{Explicit functions with Baker domains}

\label{sec:explicit-bd}

%\begin{dfn}[Baker domain]
%Let $f$ be a transcendental self-map of $\C^*$. We say that a Fatou component $U$ of $f$ is a \textit{Baker domain} if $U$ is \textit{periodic}, that is, there exists $p\geqslant 1$ such that $f^p(U)=U$, and $U\subseteq I(f)$.
%\end{dfn}

We now turn our attention to Baker domains. As we mentioned in the introduction, Baker domains can be classified into hyperbolic, simply parabolic and doubly parabolic according to the Riemann surface $U/f$ obtained by identifying the points of the Baker domain $U$ that belong to the same orbit under iteration by the function $f$. K\"onig \cite{koenig99} introduced the following notation.

\begin{dfn}[Conformal conjugacy]
Let $U\subseteq \C$ be a domain and let \mbox{$f:U\to U$} be analytic. Then a domain $V\subseteq U$ is \textit{absorbing} (or \textit{fundamental}) for $f$ if $V$ is simply connected, $f(V)\subseteq V$ and for each compact set $K\subseteq U$, there exists $N=N_K$ such that $f^N(K)\subseteq V$.
Let $\mathbb H:=\{z\in\C\ :\ \textup{Re}\, z>0\}$. The triple $(V,\phi,T)$ is called a \textit{conformal conjugacy} (or \textit{eventual conjugacy}) of $f$ in $U$ if
\begin{enumerate}
\item[(a)] $V$ is absorbing for $f$;
\item[(b)] $\phi:U\to \Omega\in\{\mathbb H, \C\}$ is analytic and univalent in $V$;
\item[(c)] $T:\Omega\to\Omega$ is a bijection and $\phi(V)$ is absorbing for $T$;
\item[(d)] $\phi(f(z))=T(\phi(z))$ for $z\in U$.
\end{enumerate}
In this situation we write $f\sim T$.
\end{dfn}

Observe that properties (b) and (d) imply that $f$ is univalent in $V$. K\"onig also provided the following geometrical characterization of the three types of Baker domains \cite[Theorem~3]{koenig99}. This characterisation is also valid for any simply connected Baker domain of a transcendental self-map of $\C^*$.%, which correspond, respectively, to the hyperbolic, simply parabolic and doubly parabolic types.

\begin{lem}
Let $U$ be a $p$-periodic Baker domain of a meromorphic function~$f$ in which $f^{np}\to\infty$ and on which $f^p$ has a conformal conjugacy. For $z_0\in U$, put
$$
c_n=c_n(z_0):=\frac{|f^{(n+1)p}(z_0)-f^{np}(z_0)|}{\textup{dist}(f^{np}(z_0),\partial U)}.
$$
Then exactly one of the following cases holds:
\begin{enumerate}
\item[(a)] $U$ is hyperbolic and $f^p\sim T(z)=\lambda z$ with $\lambda>1$, which is \mbox{equivalent~to}
$$
c_n>c\quad \mbox{ for } z_0\in U,\ n\in \N,\quad \mbox{ where } c=c(f)>0.
$$
\item[(b)] $U$ is simply parabolic and $f^p\sim T(z)=z\pm i$, which is equivalent to 
$$
\liminf_{n\to\infty} c_n>0 \quad \mbox{ for } z_0\in U,\quad \mbox{ but } \inf_{z_0\in U}\limsup_{n\to\infty} c_n=0;
$$
\item[(c)] $U$ is doubly parabolic and $f^p\sim T(z)=z+1$, which is equivalent to
$$
\lim_{n\to\infty} c_n=0\quad \mbox{ for } z_0\in U.
$$
\end{enumerate}
\label{lem:bd-koenig}
\end{lem}

\begin{figure}[h!]
\centering
\def\svgwidth{\linewidth}
\input{bd-types1.pdf_tex}
\hspace*{17pt}(a) $U$ hyperbolic\hspace*{14pt} (b) $U$\! simply\! parabolic \hspace*{3pt} (c) $U$\! doubly\! parabolic
\caption[Classification of Baker domains with their absorbing domains]{Classification of Baker domains with their absorbing domains.}
\label{fig:bd-types}
\end{figure}

We now give a couple of explicit examples of transcendental self-maps of $\C^*$, with a hyperbolic and a doubly parabolic Baker domain, respectively.

\begin{ex}
For every $\lambda>1$, the function $f_\lambda(z)=\lambda z\exp(e^{-z}+1/z)$ is a transcendental self-map of $\C^*$ which has an invariant, simply connected, hyperbolic Baker domain $U\subseteq \C^*\setminus \R_-$ whose boundary contains both zero and infinity, and the points in~$U$ escape to infinity (see Figure~\ref{fig:hyp-baker-domain}). 
\label{ex:hyp-baker-domain}
\end{ex}

\begin{proof}[Proof of Example~\ref{ex:hyp-baker-domain}]
First observe that 
\begin{equation}
\begin{array}{rl}
f_\lambda(z)\hspace*{-8pt} & =\lambda z\exp\left(e^{-z}+\tfrac{1}{z}\right)\vspace{5pt}\\
& = \lambda z\left(1+e^{-z}+\tfrac{1}{2!}e^{-2z}+\cdots\right)\left(1+\tfrac{1}{z}+\tfrac{1}{2!}\tfrac{1}{z^2}+\cdots   \right)\vspace{5pt}\\
& = \lambda z\left(1+O\left(\tfrac{1}{z}\right)\right) \mbox{ as } \textup{Re}\,z\to\infty.
\end{array}
\label{eq:ex-bd-1}
\end{equation}
Hence $f_\lambda$ maps $\mathbb H_R:=\{z\in\C\ :\ \textup{Re}\,z>R\}$ into itself, for $R>0$ sufficiently large, so $\mathbb H_R\subseteq U$, where $U$ is an invariant Fatou component of $f_\lambda$. Also, for real $x>0$,
$$
f_\lambda(x)=\lambda x\exp\left(e^{-x}+\tfrac{1}{x}\right)>\lambda x>x
$$
so $f_\lambda^n(x)\to\infty$ as $n\to \infty$. Thus, $U$ is an invariant Baker domain of $f$ which contains the positive real axis, so $\partial U$ contains zero and infinity.

To show that $U$ is a hyperbolic Baker domain, consider $z_0\in U$. By the contraction property of the hyperbolic metric in $U$, the orbit of $z_0$ escapes to infinity in~$\mathbb H_R$. Hence, by \eqref{eq:ex-bd-1} and since $0\in U^c$,
$$
\begin{array}{rl}
c_n\hspace*{-6pt} &\ds=\frac{|f^{n+1}(z_0)-f^n(z_0)|}{\textup{dist}\,(f^n(z_0),\partial U)}\geqslant \frac{\lambda f^n(z_0)\left(1+O\left(\frac{1}{f^n(z_0)}\right)\right)-f^n(z_0)}{|f^n(z_0)|}\vspace{10pt}\\
&\ds>\lambda -1 -\frac{O(1)}{|f^n(z_0)|} \mbox{ as } n\to \infty,
\end{array}
$$
so
$$
\liminf_{n\to\infty} c_n\geqslant \lambda-1>0,
$$
and thus the Baker domain $U$ is hyperbolic.

Finally, observe that the negative real axis is invariant under $f$, and therefore $(-\infty,0)\cap U=\emptyset$. Since doubly connected Fatou components must surround zero, $U$ is simply connected.
\end{proof}

%Let $g_\lambda(z):=\lambda z\exp(e^{-z})$ with $\lambda>1$ and observe that
%$$
%g_\lambda(z)=\lambda z+O(ze^{-z})\quad \mbox{ as } \textup{Re}\, z\rightarrow +\infty.
%$$
%Since the exponential function maps a left half-plane to a disc centered at the origin, the function $\exp(e^{-z})$ maps the half-plane $H_R:=\{z\in\C\ :\ \textup{Re}\, z>R\}$ to a neighbourhood of $z=1$ for sufficiently large values of $R>0$ and hence $g_\lambda(H_R)\subseteq H_R$. Therefore, for every $\lambda>1$, the functions $g_\lambda$ and $f_\lambda$ have invariant Baker domains containing $H_R$. Let $U$ be the Baker domain of $f_\lambda$ and let $z_0\in U$, then
%$$
%f_\lambda^{n+1}(z_0)-f_\lambda^n(z_0)=(\lambda^n(\lambda-1)+o(1))z\quad \mbox{ as } n\to\infty,
%$$
%and
%$$
%\textup{dist}(f_\lambda^n(z_0,\partial U)\leqslant \textup{Re}\, f_\lambda^n(z_0)+o(1) \quad \mbox{ as } n\to\infty,
%$$
%so there exists $n_0\in\N_0$ such that, for all $n\geqslant n_0$,
%$$
%c_n=\frac{|f_\lambda^{(n+1)p}(z_0)-f_\lambda^{np}(z_0)|}{\textup{dist}(f_\lambda^{np}(z_0),\partial U)}\geqslant \frac{(\lambda^n(\lambda-1)+o(1))|z|}{\textup{Re}\, f_\lambda^n(z_0)+o(1)}>\frac{\lambda-1}{2}.
%$$
%Thus, there exists $0<c(f_\lambda)\leqslant (\lambda-1)/2$ such that $c_n<c(f_\lambda)$ for all $n\in\N$ and, by Lemma~\ref{lem:bd-koenig}, the Baker domain $U$ is hyperbolic.



%%By Lemma~\ref{lem:bd-koenig}, the Baker domain $U$ is hyperbolic. Indeed, if $z_0\in U$, then $f^n(z_0)$ and $\textup{dist}(f^{np}(z_0),\partial U)$ are asymptotically close to $\lambda^n z_0$ and $\textup{dist}(z_0,\partial U)+\lambda^nz_0-z_0$, respectively, and hence there exists $c=c(f_\lambda):=(\lambda-1)/2$ such that $c_n>c$ for sufficiently large values of $n\in\N$.

%Finally, observe that $U$ contains the positive real line and $f_\lambda$ is not univalent in $U$ (see Figure \ref{fig:hyp-baker-domain}). This follows from the fact that if $x>0$ then $e^{-x}+1/x>0$ and therefore $f(x)>x$ in the positive real line. Since $f_\lambda(z)\to +\infty$ as $z\to 0^+$ and as $z\to +\infty$, the positive real axis contains a critical point. 


%Observe that, in the limit case $\lambda=1$, the function $g_1(z):=z\exp(e^{-z})$ has no longer have a Baker domain, but...\red{finish this}

\begin{figure}[h!]
\includegraphics[width=.49\linewidth]{hypbd-01.png}
\includegraphics[width=.49\linewidth]{hypbd-02.png}
\caption[Phase space of a transcendental self-map of $\C^*$ which has a hyperbolic Baker domain]{Phase space of the function $f_2(z)=2z\exp(e^{-z}+1/z)$ from Example~\ref{ex:hyp-baker-domain}. On the right, zoom of a neighbourhood of zero.}
\label{fig:hyp-baker-domain}
\end{figure}

The function $f(z)=2 z\exp(e^{-z}+1/z)$ has a repelling fixed point in the negative real line. If we choose $h(z)=1/z^2$ instead of $1/z$, then $f(z)=2 z\exp(e^{-z}+1/z^2)$ has the positive real axis in a Baker domain while the negative real axis is in the fast escaping set.

%\begin{figure}[h!]
%\label{fig:baker-domains2}
%\includegraphics[width=.49\linewidth]{019-2.png}
%\includegraphics[width=.49\linewidth]{020.png}
%\caption{Phase space of the function $f(z)=2z\exp(e^{-z}+1/z^2)$, for which the positive real line is in a Baker domain (gray) and the negative real line is in the fast escaping set (blue).}
%\end{figure}


%\begin{center}
%\begin{tabular}{cc}
%%\includegraphics[width=130pt]{standard.png} & \includegraphics[width=130pt]{standard2.png}\vspace{2pt}\\
%\includegraphics[width=140pt]{011-3.png} & \includegraphics[width=140pt]{012-2.png}\vspace{2pt}\\
%%$\alpha=3.1,\ \beta=0.8.$ & $\alpha=3.1,\ \beta=5$.\\
%\end{tabular}
%\end{center}

We now give a second explicit example of transcendental self-map of $\C^*$ with a Baker domain which, in this case, is doubly parabolic.

\begin{ex}
The function $f(z)=z\exp\left((e^{-z}+1)/z\right)$ is a transcendental self-map of $\C^*$ which has an invariant, simply connected, doubly parabolic Baker domain $U\subseteq \C^*\setminus \R_-$ whose boundary contains both zero and infinity, and the points in $U$ escape to infinity (see Figure~\ref{fig:dpar-baker-domain}). \vspace*{-10pt}
\label{ex:dpar-baker-domain}
\end{ex}

\begin{proof}[Proof of Example~\ref{ex:dpar-baker-domain}]
Looking at the power series expansion of $f$, we have 
$$
%f(z)=z+1+o(1)\quad \mbox{ as } \textup{Re}\, z\rightarrow +\infty.
\begin{array}{rl}
f(z)\hspace*{-8pt} & = z\exp\left(\tfrac{e^{-z}}{z}+\tfrac{1}{z}\right)\vspace{5pt}\\
& = z\left(1+\tfrac{e^{-z}}{z}+\tfrac{1}{2!}\tfrac{e^{-2z}}{z^2}+\cdots\right)\left(1+\tfrac{1}{z}+\tfrac{1}{2!}\tfrac{1}{z^2}+\cdots   \right)\vspace{5pt}\\
& =  z\left(1+\tfrac{1}{z}+O\left(\tfrac{1}{z^2}\right)\right) \mbox{ as } \textup{Re}\,z\to\infty.
\end{array}
$$
Therefore $f$ maps the right half-plane $\mathbb H_R:=\{z\in\C\ :\ \textup{Re}\, z>R\}$ into itself for sufficiently large values of $R>0$ and $\mathbb H_R$ is contained in an invariant Baker domain $U$ of $f$, in which $\textup{Re}\,f^n(z)\to+\infty$ as $n\to\infty$. Since $f(x)>x$ for all $x>0$, the positive real axis lies in $U$. Let $z_0\in U$, then
$$
f^{n+1}(z_0)-f^n(z_0)\! =\! f^n(z_0)\! \left(\! 1+O\! \left(\! \frac{1}{f^n(z_0)}\! \right)\! \right)-f^n(z_0)\! =\! O(1)\mbox{ as } n\to \infty
$$
and, if $R$ is as above,
$$
%\textup{dist}(f^n(z_0),\partial U)\geqslant |f^n(z_0)|\geqslant \textup{Re}\, f^n(z_0) \quad \mbox{ as } n\to \infty,
\textup{dist}(f^n(z_0),\partial U)\geqslant \textup{Re}\,f^n(z_0)-R \quad \mbox{ as } n\to \infty,
$$
so 
$$
c_n=\frac{|f^{n+1}(z_0)-f^{n}(z_0)|}{\textup{dist}(f^{n}(z_0),\partial U)}\leqslant \frac{O(1)}{\textup{Re}\,f^n(z_0)-R}\to 0 \quad \mbox{ as } n\to\infty.
$$
Thus, by Lemma~\ref{lem:bd-koenig}, the Baker domain $U$ is doubly parabolic.
% $f^n(z_0)$ and $\textup{dist}(f^{np}(z_0),\partial U)$ are asymptotically close to $z_0+n$ and $\textup{dist}(z_0,\partial U)+n$, respectively, and hence the quantity $c_n$ from Lemma~\ref{lem:bd-koenig} tends to $0$ as $n\to\infty$. 

Finally, observe that, for $x\in (-\infty,0)$, $f^n(x)\to \infty$ along the negative real axis as $n\to\infty$, so $(-\infty,0)\cap U=\emptyset$ and hence $U$ is simply connected.
\end{proof}

\begin{figure}[h!]
\includegraphics[width=.49\linewidth]{dpbd-01.png}
\includegraphics[width=.49\linewidth]{dpbd-02.png}%{d002.png}
\caption[\mbox{Phase space of a transcendental self-map of $\C^*$} \mbox{which has a doubly parabolic Baker domain}]{Phase space of the function $f(z)=z\exp\left((e^{-z}+1)/z\right)$ from Example~\ref{ex:dpar-baker-domain}. On the right, zoom of a neighbourhood of zero.}
\label{fig:dpar-baker-domain}
\end{figure}

%\begin{center}
%\begin{tabular}{cc}
%%\includegraphics[width=130pt]{standard.png} & \includegraphics[width=130pt]{standard2.png}\vspace{2pt}\\
%\includegraphics[width=140pt]{d001-2.png} & \includegraphics[width=140pt]{d002.png}\vspace{2pt}\\
%%$\alpha=3.1,\ \beta=0.8.$ & $\alpha=3.1,\ \beta=5$.\\
%\end{tabular}
%\end{center}

\begin{lem}
Let $f$ be a transcendental self-map of $\C^*$ and let $\tilde{f}$ be a lift of $f$. Then, if $U$ is a Baker domain of $f$, every component $U_k,\ k\in\Z,$ of $\exp^{-1}(U)$ is either a (preimage of a) Baker domain or a wandering domain \mbox{of~$\tilde{f}$}. Moreover, if $U$ is simply connected and $U_k$ is a Baker domain, then $U_k$ is hyperbolic, simply parabolic or doubly parabolic if and only if $U$ is hyperbolic, simply parabolic or doubly parabolic, respectively.
\label{lem:semiconj-bd}
\end{lem}
\begin{proof}
By \cite{bergweiler95}, every component of $\exp^{-1}(U)$ is a Fatou component of $\tilde{f}$. Moreover, since $\exp^{-1}(I(f))\subseteq I(\tilde{f})$, $U_k$ is either a Baker domain, a preimage of a Baker domain or an escaping wandering domain of $\tilde{f}$.

Suppose that $U$ has period $p\geqslant 1$ and $U_k$ is periodic. Then the Baker domain $U_k$ has period $q$ with $p\mid q$. Let $(V,\phi,T)$ be a conformal conjugacy of $f^q$ in $U$. Then $(\tilde{V},\tilde{\phi},T)$ is a conformal conjugacy of $\tilde{f}^q$ in $U_k$, where $\tilde{V}$ is the component of $\exp^{-1}V$ that lies in $U_k$ and $\tilde{\phi}=\phi\circ\exp$. Thus, the Baker domains $U$ and $U_k$ are of the same type.
\end{proof}

As before, we use Lemma~\ref{lem:semiconj-bd} to provide examples of transcendental entire functions with Baker domains and wandering domains.

\begin{ex}
The entire function $\tilde{f}(z)\!=\!\ln \lambda\!+\!z\!+\!\exp(-e^z)\!+\!e^{-z}$, which is a lift of the function $f$ from Example~\ref{ex:hyp-baker-domain}, has an invariant hyperbolic Baker domain that contains the real line.
\end{ex}

\begin{ex}
The entire function $\tilde{f}(z)=z+\frac{\exp(-e^z)}{e^z}+e^{-z}$, which is a lift of the function $f$ from Example~\ref{ex:dpar-baker-domain}, has an invariant doubly parabolic Baker domain that contains the real line.
%\label{ex:dpar-baker-domain}
\end{ex}

%
%
%~\\
%\red{Check what happens with $f(z)=z\exp(...)$\\ 
%
%\noindent
%What other parameters have the Baker domain when we consider $\lambda\in\C$?\\
%
%\noindent
%Does the fact that $f$ is univalent on the BD depend on $h$?\\
%
%\noindent
%Is the BD of $\lambda z\exp(e^{-z})$ univalent?\\
%
%\noindent
%Can we say for all $\lambda>1$ the BD is not univalent?}


\section{Preliminaries on approximation theory} \label{sec:approx-theory}

In this section we state the results from approximation theory that will be used in Sections~\ref{sec:wd} and \ref{sec:bd} to construct examples of functions with wandering domains and Baker domains, respectively. We follow the terminology from \cite[Chapter~IV]{gaier87}, and introduce Weierstrass and Carleman sets. Recall that if $F\subseteq \C$ is a closed set, then $A(F)$ denotes the set of continuous functions $f:F\to\C$ that are holomorphic in the interior of $F$.

%%\red{Let $F,G\subseteq \C$ be, respectively, closed and open sets, then we define
%%$$
%%\begin{array}{rl}
%%\mathcal C(F)\hspace*{-6pt}&:=\{f:F\to \C\ :\ f \mbox{ continuous in } F \},\vspace*{5pt}\\
%%\textup{Hol}(G)\hspace*{-6pt}&:=\{f:G\to \C\ :\ f \mbox{ holomorphic in } G\},
%%\end{array}
%%$$
%%and $A(F):=\mathcal C(F)\cap \textup{Hol}(F^\circ)$, where $F^\circ$ denotes the interior of $F$.\margin{$G^*$?}}
% Note that there are more general versions of these results for approximating using functions that are only holomorphic in a domain $G\subseteq \C$.



%Let $F\subseteq \C$ be closed. 
%\begin{enumerate}
%\item[(K$_1$)] $\CR\setminus F$ is connected;
%\item[(K$_2$)] $\CR\setminus F$ is locally connected at $\infty$;
%\item[(A)] for every compact subset $K\subseteq \C$ there exists a neighbourhood $V$ of $\infty$ in $\CR$ such that no component of $F^\circ$ intersects both $K$ and $V$.
%\end{enumerate}

\begin{dfn}[Weierstrass set]
\label{dfn:weierstrass-set}
We say that a closed set $F\subseteq\C $ is a \textit{Weierstrass set} in $\C$ if each $f\in A(F)$ can be approximated by entire functions \textit{uniformly} on $F$; that is, for every $\varepsilon>0$, there is an entire function $g$ for which
$$
|f(z)-g(z)|<\varepsilon\quad \mbox{ for all } z\in F.
$$
\end{dfn}

The next result is due to Arakelyan and provides a characterisation of Weierstrass sets \cite{arakeljan64}. In the case that $F\subseteq \C$ is compact and $\C\setminus F$ is connected, then it follows from Mergelyan's theorem \cite[Theorem~1~on~p.~97]{gaier87} that functions in $A(F)$ can be uniformly approximated on $F$ by polynomials.

\begin{lem}[Arakelyan's theorem]
A closed set $F\subseteq \C$ is a Weierstrass set if and only if the following two conditions are satisfied:
\begin{enumerate}
\item[\emph{(K$_1$)}] $\CR\setminus F$ is connected;
\item[\emph{(K$_2$)}] $\CR\setminus F$ is locally connected at infinity.
\end{enumerate}
\end{lem}

%\begin{lem}\margin{see Gauthier}
%If $E$ is a real-symmetric Weierstrass set in $G$ and $f$ is a real-symmetric function meromorphic on $E$ then, for each $\varepsilon>0$, there is a real-symmetric meromorphic function whose poles in $\C$ are the same as those of $f$ on $E$ and with same
%principal parts, such that
%$$
%|f(z)-g(z)|<\varepsilon\exp(-|z|^{1/4}),\quad \mbox{ for all } z\in E.
%$$
%\end{lem}

If in addition both the set $F$ and the function $f\in A(f)$ are symmetric with respect to the real line, then the approximating function $g$ can be chosen to be symmetric as well (see \cite[Section 2]{gauthier13}).

Sometimes we may want to approximate a function in $A(f)$ so that the error is bounded by a given strictly positive function $\varepsilon:\C\to\R_+$ that is not constant, and $\varepsilon(z)$ may tend to zero as $z\to\infty$. 

\begin{dfn}[Carleman set]
\label{dfn:carleman-set}
We say that a closed set $F\subseteq \C$ is a \textit{Carleman set} in $\C$ if every function $f\in A(F)$ admits \textit{tangential approximation} on $F$ by entire functions; that is, for every strictly positive function $\varepsilon\in\mathcal C(F)$, there is an entire function $g$ for which
$$
|f(z)-g(z)|<\varepsilon(z)\quad \mbox{ for all } z\in F.
$$
\end{dfn}

It is clear that Carleman sets are a special case of Weierstrass sets and hence conditions ($\text{K}_1$) and ($\text{K}_2$) are necessary. Nersesyan's theorem gives sufficient conditions for tangential approximation \cite{nersesjan71}.

\begin{lem}[Nersesyan's theorem]
A closed set $F$ is a Carleman set in~$\C$ if and only if conditions \emph{($K_1$)}, \emph{($K_2$)} and 
\begin{enumerate}
\item[\emph{(A)}] for every compact set $K\subseteq \C$ there exists a neighbourhood $V$ of infinity in $\CR$ such that no component of $\mbox{int}\, F$ intersects both $K$ and $V$,
\end{enumerate}
are satisfied.
\label{lem:nersesjan}
\end{lem}

Note that there is also a symmetric version of this result: if the set $F$ and the functions $f$ and $\varepsilon$ are in addition symmetric with respect to $\R$ then the entire function $g$ can be chosen to be symmetric with respect to $\R$ \cite[Section 2]{gauthier13}.

In some cases, depending on the geometry of the set $F$ and the decay of the error function $\varepsilon$, we can perform tangential approximation on Weierstrass sets without needing condition (A); the next result can be found in \cite[Corollary in p.162]{gaier87}.

\begin{lem}
Suppose $F\subseteq \C$ is a closed set satisfying conditions ($\text{K}_1$) and ($\text{K}_2$) that lies in a sector
$$
W_\alpha:=\{z\in \C\ :\ |\textup{arg}\,z|\leqslant \alpha/2\},
$$
for some $0<\alpha\leqslant 2\pi$. Suppose $\tilde{\varepsilon}(t)$ is a real function that is continuous and positive for $t\geqslant 0$ and satisfies
$$
\int_1^{+\infty} t^{-(\pi/\alpha)-1}\log\tilde{\varepsilon}(t)dt>-\infty.
$$
Then every function $f\in A(F)$ admits $\varepsilon$-approximation on the set $F$ with $\varepsilon(z)=\tilde{\varepsilon}(|z|)$ for $z\in F$. 
\label{lem:approx-sectors}
\end{lem}

\section{Construction of functions with wandering domains}

\label{sec:wd}

%\red{History of wandering domains... which have zeros.... mention Kotus example of WD accumulating everywhere following EL.}

To prove Theorem \ref{thm:wandering-domains} we modify Baker's construction of a holomorphic self-map of $\C^*$ with a wandering domain escaping to infinity \cite[Theorem 4]{baker87} to create instead a transcendental self-map of $\C^*$ with a wandering domain that accumulates to zero and to infinity according to a prescribed essential itinerary $e\in\{0,\infty\}^{\N_0}$ and with index $n\in \Z$.

\begin{proof}[Proof of Theorem \ref{thm:wandering-domains}] 
We construct two entire functions $g$ and $h$ using Nersesyan's theorem so that the function $f(z)=z^n\exp\bigl(g(z)+h(1/z)\bigr)$, which is a transcendental self-map of $\C^*$, has the following properties:
\begin{itemize}
\item there is a bi-infinite sequence of annuli sectors $\{A_m\}_{m\in\Z\setminus\{0\}}$ that accumulate at zero and infinity and integers $s(m)\in\Z\setminus\{0\}$, for $m\in\Z\setminus\{0\}$, such that $f(A_m)\subseteq A_{s(m)}$ for all $m\in\Z$;
\item the discs $B_+:=\overline{D(2,1/4)}$ and $B_-:=1/B_+=\overline{D(32/63, 4/63)}$ both map strictly inside themselves under $f$, $f(B_+)\subseteq \textup{int}\,B_+$ and $f(B_-)\subseteq \textup{int}\,B_-$;
\item there is a bi-infinite sequence of closed discs $\{B_m\}_{m\in\Z\setminus\{0\}}$ such that $f(B_m)\subseteq \textup{int}\,B_+$, if $m>0$, and $f(B_m)\subseteq \textup{int}\,B_-$, if $m<0$.
\end{itemize}
%Remember that if $X\subseteq \C$ and $\varepsilon>0$, we use the notation $X+\varepsilon:=\{z\in \C\ :\ \mbox{dist}(z,X)<\varepsilon\}$. 
Here $s(m):=\pi(\pi^{-1}(m)+1)$ and the map $\pi:\N\longrightarrow \Z\setminus \{0\}$ is an ordering of the sets $\{A_m\}_{m\in\N}$ according to the sequence $e$; that is, $\pi(k)$ is the position of the $k$th component in the orbit of the wandering domain. More formally, we define
\begin{equation}
\pi(k):=\left\{
\begin{array}{ll}
%\min\bigl\{\min_{\ell<k} p^{\ell+1}(0),\ 0\bigr\}-1=
\ds\#\{\ell\in\N_0\ :\ e_\ell=\infty \mbox{ for } \ell<k\}+1, & \mbox{ if } e_{k}=\infty,\vspace{5pt}\\
%\max\bigl\{\max_{\ell<k} p^{\ell+1}(0),\ 0\bigr\}+1=
\ds-\,\#\{\ell\in\N_0\ :\ e_\ell=0 \mbox{ for } \ell<k\}-1, & \mbox{ if } e_{k}=0,
\end{array}
\right.
\label{eq:p-function}
\end{equation}
for $k\in\N$ (see Figure \ref{fig:sketch-wd}).

\begin{figure}[h!]
\centering
\vspace*{45pt}
\def\svgwidth{.8\linewidth}
\input{wd3.pdf_tex}
\vspace*{30pt}
\caption[Sketch~of~the~construction~of~a~transcendental \mbox{self-map\,of\,$\C^*$\,with\,a\,wandering\,domain}]{Sketch of the construction in the proof of Theorem \ref{thm:wandering-domains}.}
\label{fig:sketch-wd}
\end{figure}

By Montel's theorem, the domains $\{A_m\}_{m\in \Z\setminus\{0\}}$, $\{B_m\}_{m\in\Z\setminus\{0\}}$ and $B_+,B_-$ are all contained in the Fatou set. Since $f(B_+)\subseteq \textup{int}\,B_+$, the function $f$~has an attracting fixed point in $B_+$ and the sets $\{B_m\}_{m\in\N}$ are contained in the preimages of the immediate basin of attraction of this fixed point. Likewise, the sets $\{B_{-m}\}_{m\in\N}$ belong to the basin of attraction of an attracting fixed point in $B_-$. Observe that in order to show that $A_1$ is contained in a wandering domain that escapes following the essential itinerary $e$ we need to prove that every $A_m$ is contained in a different Fatou component. 

Now let us construct the entire functions $g$ and $h$ so that the function $f(z)=z^n\exp\bigl(g(z)+h(1/z)\bigr)$ has the properties stated above. Note that in this construction $\log z$ denotes the principal branch of the logarithm with \mbox{$-\pi<\textup{arg}\,z<\pi$}. Let $0<R<\pi/2$ and set, for $m>0$, define
$$
\begin{array}{l}
A_m:=\{z\in\C\ :\ -R\leqslant\mbox{arg}(z)\leqslant R,\ k_m\leqslant |z|\leqslant k_m e^{2R}\},\vspace{5pt}\\
B_m:=\overline{D\bigl((k_{m+1}-k_m)/2,\ 1/8\bigr)},
\end{array}
$$
where $k_m$ is any sequence of positive real numbers such that \mbox{$k_m\! >5/2$} and $k_{m+1}>k_m+1/4$ for all $m\in \N$. We define $A_{-m}:=1/A_m$ and $B_{-m}:=1/B_m$ for all $m\in\N$. Note that $\log A_m$ is a square of side $2R$ centred at a point that we denote by $a_m\in \R$. Hence, $\log A_m$ contains the disc $D(a_m,R)$ for all $m\in\Z\setminus\{0\}$. The set 
$$
F:=\overline{D(0,1)}\cup B_+\cup \bigcup_{m>0} (A_m\cup B_m)
$$
which consists of a countable union of disjoint compact sets is a Carleman set.

Let $\delta_+,\delta_->0$ be such that $|w-\ln 2|\! <\! \delta_+$ and \mbox{$|w-\ln 32/63|\! <\! \delta_-$} imply, respectively, that $|e^w-2|<1/8$ and $|e^w-32/63|<2/63$. Let $K:=\min\{R/4, \delta_\pm/4\}$. By Lemma \ref{lem:nersesjan}, there is an entire function $g$ that satisfies the following conditions:
$$
\left\{
\begin{array}{ll}
|g(z)-a_{s(m)}-n\log z|<R/4, & \mbox{if } z\in A_m \mbox{ with } m>0,\vspace{10pt}\\
|g(z)-\ln 2-n\log z|<\delta_+/4, & \ds\mbox{if } z\in \bigcup_{m>0} B_m\cup B_+,\vspace{5pt}\\
|g(z)|<K, & \mbox{if } z\in D(0,1),
\end{array}
\right.\vspace{5pt}
$$
Similarly, there is an entire function $h$ that satisfies the following conditions:
 $$ \left\{
\begin{array}{ll}
|h(z)-a_{s(-m)}-n\log (1/z)|<R/4, & \mbox{if } z\in A_m \mbox{ with } m>0,\vspace{10pt}\\
|h(z)-\ln 32/63-n\log (1/z)|<\delta_-/4, & \ds\mbox{if } z\in \bigcup_{m>0} B_m\cup B_+,\vspace{5pt}\\
|h(z)|<K, & \mbox{if } z\in D(0,1).
\end{array}
\right.\vspace{5pt}
$$
Therefore, since the sets $B_-$ and $A_m$, $m<0$, are contained in $D(0,1)$ and the sets $B_+$ and $A_m$, $m>0$, are contained in $\C\setminus \overline{D(0,1)}$, the function $\log f(z)=g(z)+h(1/z)+n\log z$ satisfies
$$
\left\{
\begin{array}{ll}
|\log f(z)-a_{s(m)}|<R/2, & \mbox{if } z\in A_m \mbox{ with } m\neq 0,\vspace{10pt}\\
|\log f(z)-\ln 2|<\delta_+/2, & \ds\mbox{if } z\in  \bigcup_{m>0} B_m\cup B_+,\vspace{10pt}\\
|\log f(z)-\ln 32/63|<\delta_-/2, & \ds\mbox{if } z\in  \bigcup_{m<0} B_m\cup B_-,\\
\end{array}
\right.
$$
and hence $f$ has the required mapping properties.

%Observe\margin{$R$ should also\\ be small in\\ the basins of attraction $B_{\pm}$,\\ should we\\ remove this?*} that if $R$ is a rational function such that $R(0)=R(\infty)=0$ then in a neighbourhood of the essential singularities $R(z)$ will be arbitrarily small and hence $|\log f(z)+R(z)|<R$ in $A_n$ for $|n|$ large. Hence, $\tilde{f}=f\cdot e^R$ has a Baker domain in $I_e(\tilde{f})$ too.

Finally, note that this construction is symmetric with respect to the real line and hence all Fatou components of $f$ that intersect the real line will be symmetric too. Thus, since transcendental self-maps of~$\C^*$ cannot have doubly connected Fatou components that do not surround the origin \cite[Theorem 1]{baker87}, the Fatou components containing the sets $\{A_m\}_{m\in\Z\setminus \{0\}}$ are pairwise disjoint and $A_{\pi(0)}$ is contained in a wandering domain in $I_e(f)$.
\end{proof}

%\red{* We can add any tef that is small in $F$, but it seems nicer to state it in terms of a rational function because it is less technical. Perhaps we can get rid of the problem that the new function does not map $B_\pm$ into itself by making each $B_n$ map into itself instead of map into $B_\pm$? This can be done easily if $B_n$ are not discs but have the same shape as the $A_n$...}

\section{Construction of functions with Baker domains}

\label{sec:bd}



In this section we construct holomorphic self-maps of $\C^*$ with Baker domains. The construction is split into two cases: first, we deal with the cases that the function $f$ is a transcendental entire or meromorphic function, that is, $f(z)=z^n\exp(g(z))$ where $n\in\Z$ and $g$ is a non-constant entire function (see Theorem~\ref{thm:baker-domains-entire}), and then we deal with the case that the function $f$ is a transcendental self-map of~$\C^*$, that is, $f(z)=z^n\exp(g(z)+h(1/z))$ where $n\in \Z$ and $g,h$ are non-constant entire functions (see Theorem~\ref{thm:baker-domains}). For transcendental self-maps of~$\C^*$, we are able to construct functions with Baker domains that have any given \textit{periodic} essential itinerary $e\in\{0,\infty\}^{\N_0}$. 

To that end, we use Lemma~\ref{lem:approx-sectors}~to obtain entire functions $g$ and, if necessary, $h$ so that the function $f$ has a Baker domain. After this approximation process, the resulting function $f$ will behave as the function~$T_\lambda(z)=\lambda z$, $\lambda>1$, in a certain half-plane~$W$. We first require the following result that estimates~the asymptotic distance between the boundaries of $\log W$ and $\log T_\lambda(W)\subseteq \log W$.
%$W$ is an absorbing domain and , if $N$ is the period of the Baker domain,  %, but before we need a result that will ensure that the function we want to construct maps the domain well inside itself so that we can apply Lemma \ref{lem:approx-sectors}.

\begin{lem}
Let $W=\{z\in\C : \textup{Re}\, z\geqslant 2\}$ and, for $\lambda>1$, let \mbox{$T_\lambda(z)=\lambda z$}. For $r>0$, let $\delta (r)$ denote the vertical distance between the curves $\partial\log  W$ and $\partial\log T_\lambda(W)\subseteq \log W$ along the vertical line $V_r:=\{z\in\C : \textup{Re}\, z=r\}$. Then $\delta(r)\sim 2(\lambda-1)e^{-r}$ as $r\to+\infty$.
\label{lem:approx-BD}
\end{lem}
\begin{proof}
Since $\log z=\ln |z|+i\,\mbox{arg}(z)$, the quantity $\delta(r)$ equals the difference between the arguments of the points $z_1,z_2$ with $\textup{Im}\, z_k>0$, $k\in\{1,2\}$, where the vertical lines $\partial W$ and $\partial T(W)$ intersect the circle $\exp V_r$ of radius $e^r$ (see Figure \ref{fig:sketch-bd}).  %and, for $j=3$,
%$$
%\delta_3(r)=\arccos \frac{2}{e^r}-\arccos\frac{3}{e^r}\sim\left(\frac{\pi}{2}-\frac{2}{e^r}\right)-\left(\frac{\pi}{2}-\frac{3}{e^r}\right)= \frac{1}{e^r}.
%$$

\begin{figure}[h!]
\centering
\vspace{30pt}
\def\svgwidth{.8\linewidth}
\input{bd.pdf_tex}
\vspace{15pt}
\caption[Definition of the function $\delta(r)$]{Definition of the function $\delta(r)$.}
\label{fig:sketch-bd} 
\end{figure}


Since $\mbox{arg}\,z_1,\mbox{arg}\,z_2 \to\pi/2$ as $r\to+\infty$, we have
$$
\delta(r)=\arccos \frac{2}{e^r}-\arccos\frac{2\lambda}{e^r}\sim\left(\frac{\pi}{2}-\frac{2}{e^r}\right)-\left(\frac{\pi}{2}-\frac{2\lambda}{e^r}\right)=\frac{2(\lambda-1)}{e^r},
$$
as $r\to+\infty$, as required.
\end{proof}

%The set $F:=\log\bigl(\{\Re z>x\}\bigr)$, with $x>0$, is a Weierstrass set. We will perfom tangential approximation with $\varepsilon(z)$ as error function. Since $\log(\{\Re z=x\})$ is contained in a horizontal band of width $2\pi$, we can use Lemma \ref{lem:approx-sectors} with a sector of any angle $\alpha>0$, and $\tilde{\varepsilon}(t)=\sqrt{2}e^{-t}$ clearly satisfies the condition
%$$
%\int_1^{+\infty} t^{-(\pi/\alpha)-1}\log\tilde{\varepsilon}(t)dt>-\infty
%$$
%if we choose $\alpha<\pi$.\\
%\margin{do we need\\ this?}Observe that the points in the intersection of the vertical lines $V_k$ and $V_{k+1}$ with the circle of radius $e^r$ tend asymptotically to the points in the intersection of $V_k$ and $V_{k+1}$ with the horizontal line $H_{e^r}:=\{z\in\C\ :\ \Im z=e^r\}$ as $r\to+\infty$.

%For $j=2$, the angle $\delta_2(r)$ is bounded below by the angle defined by the points $z_1',z_2'$ in the hyperbolas $\partial W_2$ and $\partial T_2(W_2)$ with $\Im z_1'=\Im z_2'=e^r$. Indeed, first observe that if $z_2''\in\partial T_2(W_2)$ with $\Im z_2''=\Im z_1$, then the angle defined by $z_1$ and $z_2''$ is smaller than the angle defined by $z_1$ and any point $w\in\partial T_2(W_2)$ with $\Im w<\Im z_2''$. Moreover, the angle defined by the points $z,w$ in $\partial W_2$ and $\partial T_2(W_2)$ with $\Im z=\Im w=e^r$ is strictly decreasing as $r\to +\infty$. Thus, since $\mbox{arg}\,z_1',\mbox{arg}\,z_2' \to\pi/2$ as $r\to+\infty$, 
%For $j=2$, the quantity $\delta_2(r)$ is asymptotically close to the angle defined by the points $z_1',z_2'$ in the hyperbolas $\partial W_2$ and $\partial T_2(W_2)$ with $\textup{Im}\, z_1'=\textup{Im}\, z_2'=e^r$, thus
%$$
%\begin{array}{rl}
%\ds\delta_2(r)\hspace*{-6pt}&\ds\sim\arccos \frac{\textup{Re}\, z_1'}{e^r}-\arccos \frac{\textup{Re}\, z_2'}{e^r}\sim\left(\frac{\pi}{2}-\frac{\textup{Re}\, z_1'}{e^r}\right)-\left(\frac{\pi}{2}-\frac{\textup{Re}\, z_2'}{e^r}\right)=\vspace{5pt}\\
%&\ds=-\frac{1}{e^r}\left(\frac{1}{e^r}+2\right)+\frac{1}{e^r}\left(\frac{1}{e^r-1}+2\right)=\frac{1}{e^{2r}(e^r-1)}\sim \frac{1}{e^{3r}}
%\end{array}
%$$
%as we wanted to show.
%and hence $\delta_2(r)>e^{-3r}/2$ for sufficiently large values of $r$. The other inequality follows from the fact that...


Given $N\in\N$ and a periodic sequence \mbox{$e=\overline{e_0e_1\cdots e_{N-1}}\in\{0,\infty\}^{\N_0}$}, let $p,q\in\N$ denote
\begin{equation}
\begin{array}{l}
p=p(e):=\#\{k\in\N_0\ :\ e_k=\infty \mbox{ for } k<N\},\vspace{5pt}\\
q=q(e):=\#\{k\in\N_0\ :\ e_k=0 \mbox{ for } k<N\},
\end{array}
\label{eq:p-and-q}
\end{equation}
so that $p+q=N$. We want to construct a holomorphic function \mbox{$f:\C^*\to\C^*$} with an $N$-cycle of Baker domains that has components $U_i^\infty$, $0\leqslant i<p$, and $U_i^0$, $0\leqslant i<q$, in which 
$$
f_{|U_i^\infty}^{Nn}\to\infty\quad \mbox{ and } \quad f_{|U_i^0}^{Nn}\to 0 \quad \mbox{ locally uniformly as } n\to \infty.
$$
In the case that zero is \textit{not} an essential singularity of $f$, then $q=0 $ and $N=p$. Note that the closure of a Baker domain in $\CR$ may contain both zero and infinity. 

For $p\in\N$ and $X\subseteq \C^*$, we define
$$
\sqrt[p]{X}:=\{z\in\C^*\, :\, z^p\in X,\ |\textup{arg}\,z|<\pi/p\}.
$$
In order to construct a function with an $N$-periodic Baker domain that has $p$ components around zero or infinity, we will semiconjugate the function $T_\lambda$ that we want to approximate in the half-plane $W$ by the $p$th root function:
$$
\xymatrix{
W \ar[r]^{T_\lambda}  & W\\
\sqrt[p]{W} \ar[u]^{z^p} \ar[r]_{T_{\lambda,p}} & \sqrt[p]{W}. \ar[u]_{z^p}
}
$$
Next we look at the effect of this semiconjugation on the function $\delta$.

\begin{lem}
Let $W$ and $T_\lambda$, $\lambda>1$, be as in Lemma~\ref{lem:approx-BD}. For $p\in\N$ and $\lambda>1$, define the function $T_{\lambda,p}(z):=\sqrt[p]{T_\lambda(z^p)}$ on $\sqrt[p]{W}$ and, for $r>0$, let $\delta_{p}(r)$ denote the vertical distance between the curves $\partial \log \sqrt[p]{W}$ and $\partial \log T_{\lambda,p}(\sqrt[p]{W})\subseteq \log \sqrt[p]{W}$ along the vertical line \mbox{$V_r:=\{z\in\C : \textup{Re}\, z=r\}$}. Then $\delta_{p}(r)\sim 2(\lambda-1)e^{-pr}/p$ as $r\to+\infty$.
\label{lem:approx-root-BD}
\end{lem}
\begin{proof}
The function $z\mapsto z^p$ maps the circle of radius $e^r$ to the circle of radius $e^{pr}$ while the function $z\mapsto \sqrt[p]{z}$ divides the argument of points on that circle by $p$, so
$$
\delta_{p}(r)=\frac{\delta(pr)}{p}
$$
and hence, by Lemma~\ref{lem:approx-BD}, $\delta_{p}(r)\sim 2(\lambda-1)e^{-pr}/p$ as $r\to+\infty$.
\end{proof}

In the following theorem we construct transcendental entire or meromorphic functions that are self-maps of $\C^*$ and have Baker domains in which points escape to infinity. These functions are of the form $f(z)=z^n\exp(g(z))$ where $n\in\Z$ and $g$~is a non-constant entire function. 

\begin{thm}
\label{thm:baker-domains-entire}
For every $N\in\N$ and $n\in\Z$, there exists a holomorphic self-map~$f$ of $\C^*$ with $\textup{ind}(f)=n$ that is a transcendental entire function, if $n\geqslant 0$, or a transcendental meromorphic function, if $n<0$, and has a cycle of hyperbolic Baker domains of period $N$.
\end{thm}
\begin{proof}
Let $\omega_{N}:=e^{2\pi i/N}$ and define 
$$
V_{m}:=\omega_{N}^m\sqrt[N]{W}\subseteq \C\setminus \overline{\mathbb D} \quad \mbox{ for } 0\leqslant m<N,
$$
where $W$ is the closed half-plane from Lemma~\ref{lem:approx-BD}. We denote by $V$ the union of all $V_m$ for $0\leqslant m<N$, and let $R:=\R_-$, if $N$ is odd, or $R:=\{z\in\C^*\,:\,\textup{arg}\,z=\pi(1-1/N)\}$, if $N$ is even. Then put 
\begin{equation}
d:=\min\{(\sqrt[N]{2}-1)/3,\ \textup{dist}(V,R)/4\},
\label{eq:bd-1}
\end{equation}
and define the closed connected set
\begin{equation}
B:=\{z\in\C\,:\,\textup{dist}\,(z,V)\geqslant d \mbox{ and } \textup{dist}\,(z,R)\geqslant d\},
\label{eq:bd-2}
\end{equation}
which satisfies $B':=\overline{D(1,d)}\subseteq \textup{int}\,B$ (see Figure~\ref{fig:sketch-bd-entire}).

\begin{figure}[h!]
\centering
%\vspace{30pt}
\def\svgwidth{.60\linewidth}
\input{bd-entire.pdf_tex}
%\vspace{15pt}
\caption[Sketch of the construction of a transcendental entire or meromorphic function that is a self-map of $\C^*$ and has a cycle of hyperbolic Baker domains]{Sketch of the construction in the proof of Theorem~\ref{thm:baker-domains-entire} with \mbox{$N\! =\! 3$}. The sets $B$ and $V_m$, $0\leqslant m<N$, are shaded in grey.}
\label{fig:sketch-bd-entire} 
\end{figure}

Observe that the closed set $F:=B\cup V$ satisfies the hypothesis of Lemma \ref{lem:approx-sectors}; namely $\CR\setminus F$ is connected and $\CR \setminus F$ is locally connected at infinity, and $F\subseteq W_\alpha$ with $\alpha=2\pi$. We now define a function $\hat{g}$ on~$F$:
\begin{equation}
\hat{g}(z)\!:=\!\left\{\begin{array}{ll}
\!\!\!\log \left(\omega_{N}^{m+1}\sqrt[N]{\lambda (z/\omega_{N}^m)^{N}}\right)\! -n\log z, & \!\mbox{for } z\in V_m,\ 0\leqslant m<N,\vspace{5pt}\\
\!\!\!-n\log z, & \!\mbox{for } z\in B,
\end{array}\right. 
\label{eq:bd-3}
\end{equation}
where we have taken an analytic branch of the logarithm defined on $\C^*\setminus R$ and hence on $F$. Then $\hat{g}\in A(F)$.

For $r>0$, we define the positive continuous function 
\begin{equation}
\varepsilon(r):=\min\{d',\ k^{-(N+1)},\ r^{-(N+1)}\} 
\label{eq:bd-4}
\end{equation}
where the constant $d'>0$ is so small that $|e^z-1|<d$ for $|z|<d'$ and the constant $k>0$ is so large that, for all $z\in \log T_\lambda(W)$ with $\textup{Re}\,z<k$, the disc $D(z,k^{-(N+1)})$ is compactly contained in $\log W$ and, moreover, if $\delta_{N}(r)$ is the function from Lemma~\ref{lem:approx-root-BD}, then
\begin{equation}
\varepsilon(r)<\delta_{N}(\ln (\lambda r)) \quad \mbox{ for } r\geqslant k,
\label{eq:bd-5}
\end{equation}
which is possible since
$$
\delta_{N}(\ln (\lambda r))\sim \frac{2(\lambda-1)}{N\lambda^N r^N} \quad \mbox{ as } r\to+\infty.
$$
Since $\varepsilon$ satisfies
$$
\int_1^{+\infty} r^{-3/2}\ln\varepsilon(r)dt=C-(N+1)\int_{r_0'}^{+\infty} \frac{\ln r}{r^{3/2}}dr>-\infty
$$
for some constants $C\in\R$ and $r_0'\geqslant r_0$, by Lemma \ref{lem:approx-sectors} (with $\alpha=2\pi$), there is an entire function~$g$ such that
\begin{equation}
|g(z)-\hat{g}(z)|<\varepsilon(|z|)\quad\mbox{ for all } z\in F.
\label{eq:bd-entire-approx}
\end{equation}
We put 
\begin{equation}
f(z):=z^n\exp(g(z))=z^n\exp(\hat{g}(z))\exp(g(z)-\hat{g}(z)).
\label{eq:bd-6}
\end{equation}
By Lemma \ref{lem:approx-root-BD} and (\ref{eq:bd-3}-\ref{eq:bd-entire-approx}), $f(V_m)\subseteq V_{m+1}$ for \mbox{$0\leqslant m<N- 1$} and $f(V_{N-1})\subseteq V_0$ and, by (\ref{eq:bd-1}-\ref{eq:bd-entire-approx}), $f(B)\subseteq D(1,d)$. Hence each set $V_m$ is contained in an $N$-periodic Fatou component~$U_m$ for $0\leqslant m<N$ and~$B$ is contained in the immediate basin of attraction of an attracting fixed point that lies in $B'$. It follows that the Fatou components $U_m$ are all simply connected.

To conclude the proof of Theorem~\ref{thm:baker-domains-entire}, it only remains to check that the Fatou components $U_m$, $0\leqslant m<N$, are hyperbolic Baker domains. Due to symmetry, it suffices to deal with the case $m=0$. Let $z_0\in U_0$. Since $V_0\subseteq U_0$ is an absorbing region, we can assume without loss of generality that $z_0\in V_0$ and $|z_0|$ is sufficiently large. For $n\in\N$, let 
$$
\epsilon_n:=g(f^{n-1}(z_0))-\hat{g}(f^{n-1}(z_0))
$$
which, by \eqref{eq:bd-entire-approx}, satisfies 
$$
|\epsilon_n|<\varepsilon(|f^{n-1}(z_0)|) \quad \mbox{ as } n\to \infty.
$$
For $n\in\N$, define 
$$
C_n:=\prod_{0<k\leqslant n} \exp \epsilon_k=\exp \sum_{0<k\leqslant n} \epsilon_k,
$$
which represents the quotient $f^n(z_0)/\bigl(z^n\exp(\hat{g}(z_0))\bigr)$. Using the triangle inequality, we obtain
\begin{equation}
|C_n|\leqslant \exp \sum_{0<k\leqslant n} |\epsilon_k|<\exp\sum_{0<k\leqslant n} \varepsilon(|f^{k-1}(z_0)|).
\label{eq:bd-9}
\end{equation}
Next, we are going to show that $|C_n|$ is bounded above for all $n\in\N$. To that end, we find a lower bound for $|f^{k}(z_0)|$ for $k\in\N$ assuming, if necessary, that $|z_0|=r_0$ is sufficiently large. Put $K:=(\sqrt[N]{\lambda}-1)/2>0$. Then $|C_1|>1/K$ for $r_0>0$ sufficiently large and, by \eqref{eq:bd-6} and \eqref{eq:bd-3},
$$
|f(z_0)|=\sqrt[N]{\lambda}|z_0||C_1|\geqslant \frac{\sqrt[N]{\lambda}}{K}r_0=\mu r_0,
$$
with $\mu:=\sqrt[N]{\lambda}/K>1$. Hence, by induction and the symmetry properties of the sets $V_m$, $0\leqslant m<N$,
\begin{equation}
|f^k(z_0)|\geqslant \mu^k r_0\quad \mbox{ for } k\in\N.
\label{eq:bd-7}
\end{equation}
In particular, $z_0\in I(f)$ so, by normality, the periodic Fatou components $U_m$, $0\leqslant m<N$, are Baker domains. We deduce by \eqref{eq:bd-9}, \eqref{eq:bd-4} and \eqref{eq:bd-7} that $|C_n|<e^S$ for all $n\in\N$, where $S<+\infty$ is the sum of the following geometric series
%$$
%S_1:=\sum_{k=0}^\infty \frac{1}{(\mu^kr_0)^{3p+1}}=\frac{1}{r_0^{3p+1}} \sum_{k=0}^\infty \left( \frac{1}{\mu^{3p+1}} \right)^k =\frac{\mu^{3p+1}}{r_0^{3p+1}(\mu^{3p+1}-1)}<+\infty.
%$$
$$
S:=\sum_{k=0}^\infty \frac{1}{(\mu^kr_0)^{N+1}}=\frac{1}{r_0^{N+1}} \sum_{k=0}^\infty \left( \frac{1}{\mu^{N+1}} \right)^k =\frac{\mu^{N+1}}{r_0^{N+1}(\mu^{N+1}-1)}.
$$
%$$
%\begin{array}{rl}
%\ds S_1\hspace{-6pt}&\ds:=\sum_{k=0}^\infty \left(\frac{\sqrt[p]{\lambda}^{k}}{K^{k}}r_0\right)^{-3p-1}=\frac{1}{r_0^{3p+1}} \sum_{k=0}^\infty \left( \frac{K^{3p+1}}{\lambda^3\sqrt[p]{\lambda}} \right)^k \vspace{5pt}\\
%&\ds=\frac{\lambda^3\sqrt[p]{\lambda}}{r_0^{3p+1}(\lambda^3\sqrt[p]{\lambda}-K^{3p+1})}<+\infty.
%\end{array}
%$$

Next we use the characterisation of Lemma~\ref{lem:bd-koenig} to show that the Baker domains are hyperbolic. For $n\in\N$, define
$$
c_n=c_n(z_0)=\frac{|f^{(n+1)N}(z_0)-f^{nN}(z_0)|}{\textup{dist}(f^{nN}(z_0),\partial U)}.
$$
We have
$$
f^{nN}(z_0)=C_{nN}\sqrt[N]{\lambda^{nN}z_0^N}=C_{nN}\lambda^{n}z_0 \quad \mbox{ for } n\in\N
$$
and therefore
$$
|f^{(n+1)N}(z_0)-f^{nN}(z_0)|\sim C_\infty\lambda^n(\lambda-1)|z_0|\quad \mbox{ as } n\to\infty,
$$
where $C_\infty:=\lim_{n\to\infty} C_n$. Also, $\textup{dist}(f^{nN}(z_0),\partial U_0)\leqslant e^{S}\lambda^n|z_0|$ and hence if $c:=(\lambda-1)/2>0$, we have $c_n(z_0)>c$ for all $n\in\N$. Thus, by Lemma~\ref{lem:bd-koenig}, the Baker domain $U_0$ is hyperbolic. This completes the proof of Theorem~\ref{thm:baker-domains-entire}.
\end{proof}

Finally we prove Theorem \ref{thm:baker-domains} in which we construct a function~$f$ that is a transcendental self-map of $\C^*$ with $\textup{ind}(f)=n$ that has a cycle of hyperbolic Baker domains in $I_e(f)$, where $e$ is any prescribed periodic essential itinerary $e\in\{0,\infty\}^{\N_0}$.

\begin{proof}[Proof of Theorem \ref{thm:baker-domains}] 
Let $N\in \N$ be the period of $e$ and let $p,q\in\N_0$ denote, respectively, the number of symbols $0$ and $\infty$ in the sequence $e_0e_1\hdots e_{N-1}$, where $p+q=N$; see \eqref{eq:p-and-q}. We modify the proof of Theorem~\ref{thm:baker-domains-entire} to obtain a transcendental self-map of $\C^*$ of the form
$$
f(z):=z^n\exp(g(z)z^{N+1}+h(1/z)/z^{N+1})
$$
that has a hyperbolic Baker domain $U$ in $I_e(f)$, where the entire functions $g, h$~will be constructed using approximation theory. 

We start by defining a collection of $p$ sets $\{V_m^\infty\}_{0\leqslant m<p}$, whose closure in $\CR$ contains infinity. Put $\omega_{p}:=e^{2\pi i/p}$ once again and define
$$
V_{m}^\infty:=\omega_{p}^m\sqrt[p]{W}\subseteq \C\setminus \overline{D(0,\rho)} \quad \mbox{ for } 0\leqslant m<p,
$$
where $W$ is the half-plane from Lemma~\ref{lem:approx-BD} and $\rho:=1+(\sqrt[N]{2}-1)/6$. We denote by $V_\infty$ the union of all $V_m^\infty$, $0\leqslant m<p$.

As before, we define a set $B_\infty$ that will be contained in an immediate basin of attraction of $f$ and put $R_\infty=\R_-$, if $p$ is odd, or $R_\infty=\{z\in\C^*\,:\,\textup{arg}\,z=\pi(1-1/p)\}$, if $p$ is even. Then, let
$$
d_\infty:=\min\{(\sqrt[N]{2}-1)/6,\ \textup{dist}(V_\infty,R_\infty)/4\},
$$
and define the closed connected set
$$
B_\infty:=\{z\in\C\,:\,\textup{dist}\,(z,V_\infty)\geqslant d_\infty \mbox{ and } \textup{dist}\,(z,R_\infty)\geqslant d_\infty\}\setminus D(0,\rho),
$$
which compactly contains the disc $B_\infty':=\overline{D((1+\sqrt[N]{2})/2,(\sqrt[N]{2}-1)/6)}$. Finally, we define the disc $D:=D(0,1/\rho)$, which is contained in $\mathbb D$. We will construct the function $g$ by approximating it on the closed set $F_\infty:=V_\infty\cup B_\infty\cup D$, which satisfies the hypothesis of Lemma \ref{lem:approx-sectors}; namely $\CR\setminus F_\infty$ is connected and $\CR \setminus F_\infty$ is locally connected at infinity, and $F_\infty\subseteq W_\alpha$ with $\alpha=2\pi$ (see Figure~\ref{fig:sketch-bd-cstar-1side}).

\begin{figure}[h!]
\centering
%\vspace{30pt}
\def\svgwidth{.60\linewidth}
\input{bd-cstar-3.pdf_tex}
%\vspace{15pt}
\caption[Sketch of the construction of a transcendental self-map of $\C^*$ that has a cycle of hyperbolic Baker domains I]{Sketch of the construction of the entire function $g$ in the proof of Theorem~\ref{thm:baker-domains} with $e=\overline{\infty\infty00\infty}$. The sets $D$, $B_\infty$ and $V_m^\infty$, \mbox{$0\leqslant m<p$}, are shaded in grey.}
\label{fig:sketch-bd-cstar-1side} 
\end{figure}

Similarly, we define a set $B_0$ and a collection of $q$ unbounded sets $\{V_m^0\}_{0\leqslant m<q}$ by using the same procedure as above, just replacing $p$ by $q$, and then, if $V_0$ is the union of all $V_m^0$, $0\leqslant m<q$, we put $F_0:=V_0\cup B_0\cup D$. The Fatou set of the function $f$ will contain all the sets $V_m^\infty$, $0\leqslant m<p$, and all the sets $\tilde{V}_m^0:=1/V_m^0$, $0\leqslant m<q$, which are unbounded in $\C^*$.

In order to define the functions $\hat{g}\in A(F_\infty)$ and $\hat{h}\in A(F_0)$, we first introduce some notation to describe how $\hat{g}$ and $\hat{h}$ map the components of $V_\infty$ and $V_0$, respectively; we use the same notation as in Theorem~\ref{thm:wandering-domains}. Let $\pi:\{0,\hdots,N-1\}\to \{-q,\hdots,-1,1,\hdots,p\}$ denote the function given by, for $0\leqslant k<N$,
$$
\pi (k):=\left\{
\begin{array}{ll}
\#\{\ell\in\N_0\ :\ e_\ell=\infty \mbox{ for } \ell<k\}+1, & \mbox{ if } e_k=\infty,\vspace{5pt}\\
-\,\#\{\ell\in\N_0\ :\ e_\ell=0 \mbox{ for } \ell<k\}-1, & \mbox{ if } e_k=0.\\
\end{array}
\right.
$$
The function $\pi$ is an ordering of the components of $V_\infty \cup 1/V_0$ according to the sequence $e$. Suppose that $V$ is the starting component; that is, $V=\tilde{V}_0^0$, if $e_0=0$, and $V=V_0^\infty$, if $e_0=\infty$. Then
$$
f^k(V)\subseteq\left\{
\begin{array}{ll}
V_{\pi(k)}^\infty, & \mbox{ if } \pi(k)>0,\vspace{5pt}\\
\tilde{V}_{-\pi(k)}^0, & \mbox{ if } \pi(k)<0.
\end{array}\right.\vspace*{-5pt}
$$
For $m\in \{-q,\hdots,-1,1,\hdots ,p\}$, we define the function
$$
s(m):=\pi(\pi^{-1}(m)+1 \pmod{N}),\vspace*{-5pt}
$$
which describes the image of the component $V_m^\infty$, if $m>0$, and $\tilde{V}_m^0$, if $m<0$, so that the function $f$ to be constructed has a Baker domain that has essential itinerary $e$. More formally, for $0\leqslant m<p$,
$$
f(V_m^\infty)\subseteq\left\{
\begin{array}{ll}
V_{s(m)}^\infty, & \mbox{ if } s(m)>0,\vspace{5pt}\\
\tilde{V}_{-s(m)}^0, & \mbox{ if } s(m)<0;
\end{array}
\right.\vspace*{-5pt}
$$ 
and, for $0\leqslant m<q$, 
$$
f(\tilde{V}_m^0)\subseteq\left\{
\begin{array}{ll}
 V_{s(-m)}^\infty, & \mbox{ if } s(-m)>0,\vspace{5pt}\\
\tilde{V}_{-s(-m)}^0, &\mbox{ if } s(-m)<0.
 \end{array}
 \right.
 $$
We now give the details of the construction of the entire function $g$ from the function $\hat{g}\in A(F_\infty)$. For $z\in V_m^\infty$, $0\leqslant m<p$, we put 
$$
\hat{g}(z):=\left\{\begin{array}{ll}
\left(\log \left(\omega_{p}^{s(m)}\sqrt[p]{\lambda (z/\omega_{p}^m)^{p}}\right)-n\log z\right)/z^{N+1}, & \mbox{ if } s(m)>0,\vspace{5pt}\\
\left(\log \left(\omega_{p}^{s(m)}/\sqrt[p]{\lambda (z/\omega_{p}^m)^{p}}\right)-n\log z\right)/z^{N+1}, & \mbox{ if } s(m)<0,
\end{array}\right.
$$
for $z\in B_\infty$, we put $\hat{g}(z):=(\log(1+(\sqrt[n]{2}-1)/2)-n\log z)/z^{N+1}$ and, for $z\in D$, we put $\hat{g}(z):=0$, where we have taken an analytic branch of the logarithm defined on $\C^*\setminus R_\infty$ and hence on $V_\infty\cup B_\infty$ (see Figure~\ref{fig:sketch-bd-cstar-1side}). Then $\hat{g}\in A(F_\infty)$. For $r>0$, we define the positive continuous function $\varepsilon_\infty$ by
$$
\varepsilon_\infty(r):=\min\{d'_\infty,\ k_\infty^{-(N+1)},\ r^{-(N+1)}\} /(2r^{N+1})
$$
where the constant $d'_\infty>0$ is so small that $|e^z-1|<d_\infty$ for $|z|<d'_\infty$ and the constant $k_\infty>0$ is so large that, for all $z\in \log T_\lambda(W)$ with $\textup{Re}\,z<k_\infty$, the disc $D(z,k_\infty^{-(N+1)})$ is compactly contained in $\log W$ and, moreover, if $\delta_{N}(r)$ is the function from Lemma~\ref{lem:approx-root-BD}, then 
$$
\varepsilon_\infty(r)\cdot 2r^{N+1}<\delta_{N}(\ln (\lambda r)) \quad \mbox{ for } r\geqslant k_\infty, 
$$
which, as before, is possible since 
$$
\delta_{N}(\ln (\lambda r)) \sim \frac{2(\lambda-1)}{N\lambda^Nr^N} \quad \mbox{ as } r\to+\infty.
$$
Since $\varepsilon_\infty$ satisfies 
$$
\int_1^{+\infty} r^{-3/2}\ln\varepsilon_\infty(r)dt>-\infty,
$$
by Lemma \ref{lem:approx-sectors} (with $\alpha=2\pi$), there is an entire function~$g$ such that
\begin{equation}
|g(z)-\hat{g}(z)|<\left\{\begin{array}{ll}
\varepsilon_\infty(|z|) & \mbox{ for } z\in V_\infty\cup B_\infty,\vspace{5pt}\\
1/2 & \mbox{ for } z\in D.
\end{array}\right.
\label{eq:bd-entire-approx}
\end{equation}

Similarly, we can construct an entire function $h$ that approximates a function $\hat{h}\in A(F_0)$ so that the function
$$
\begin{array}{rl}
f(z):=&\hspace{-6pt}z^n\exp(g(z)z^{N+1}+h(1/z)/z^{N+1})\vspace{5pt}\\
=&\hspace{-6pt}z^n\exp(\hat{g}(z)z^{N+1})\exp(\hat{h}(1/z)/z^{N+1}) \cdot \vspace{5pt}\\
&\cdot \exp((g(z)-\hat{g}(z))z^{N+1})\exp((h(z)-\hat{h}(z))/z^{N+1})
\end{array}
$$
%\begin{array}{rl}
%f(z):=\hspace{-6pt}&z^n\exp(g(z)z^{4N}+h(1/z)/z^{4N}) \vspace{5pt}\\
%=\hspace{-6pt}&z^n\exp(\hat{g}(z)z^{4N}+\hat{h}(1/z)/z^{4N})\exp((g(z)-\hat{g}(z))z^{4N})\exp((h(z)-\hat{h}(z))/z^{4N})
%\end{array}
%$$
has the desired properties. Observe that if $z\in V_\infty\cup B_\infty$, then $1/z\in D$ and if $1/z\in V_0\cup B_0$, then $z\in D$. Thus, $\hat{h}(1/z)=0$ for $z\in V_\infty\cup B_\infty$ and
$$
\begin{array}{c}
|\hat{h}(1/z)/z^{N+1}+(g(z)-\hat{g}(z))z^{N+1}+(h(z)-\hat{h}(z))/z^{N+1}|\leqslant\vspace{5pt}\\
\leqslant 0+1/(2|z|^{N+1})+1/(2|z|^{N+1})=1/|z|^{N+1}
\end{array}
$$
for $z\in V_\infty\cup B_\infty$. %Therefore, by Lemma \ref{lem:approx-root-BD}, each component of the set $V_\infty\cup 1/V_0$ is contained in an $N$-periodic Fatou component and the sets $B_\infty$ and $1/B_0$ are contained in the immediate basins of attraction of two attracting fixed points that lie in $B_\infty'$ and $B_0'$, respectively (see Figure~\ref{fig:sketch-bd-cstar-2sides}; here $\tilde{B}_0=1/B_0$ and $\tilde{B}_0'=1/B'_0$).

\begin{figure}[ht!]
\centering
%\vspace{30pt}
\def\svgwidth{.60\linewidth}
\input{bd-cstar-2.pdf_tex}
%\vspace{15pt}
\caption[Sketch of the construction of a transcendental self-map of $\C^*$ that has a cycle of hyperbolic Baker domains II]{Sketch of the construction of the function $f$ in the proof of Theorem~\ref{thm:baker-domains} with $e=\overline{\infty\infty00\infty}$. %\red{The sets $B_\infty$, $\tilde{B}_0$ and $V_m^\infty$, $0\leqslant m<p$, and $\tilde{V}_m^0$, $0\leqslant m<q$, are shaded in grey.}
}
\label{fig:sketch-bd-cstar-2sides} 
\end{figure}


%Let $N$ be the period of the essential itinerary $e$ and let $p,q$ denote the number of $\infty$ and $0$ symbols in a period, $N=p+q$. Consider the following domains, which will correspond to a function with a hyperbolic, simply parabolic or doubly parabolic Baker domain respectively: for $0\leqslant l<p$, $k>0$,
%\begin{enumerate}
%\item[(a)] $H_{1,p}^l:=\{z\in\C\ :\ z=\omega_p^l\sqrt[p]{w} \mbox{ with } \Re w>k\}$, 
%\item[(b)] $H_{2,p}^l:=\{z\in\C\ :\ z=\omega_p^l\sqrt[p]{w} \mbox{ with } \Re w>k\}$, 
%\item[(c)] $H_{3,p}^l:=\{z\in\C\ :\ z=\omega_p^l\sqrt[p]{w} \mbox{ with } \Im w>1/(\Re w-k)\}$,
%\end{enumerate}
%where $\omega_p:=e^{2\pi i/p}$. Consider, for each of the cases, the following functions defined on $H_{i,p}:=\bigcup_k H_{i,p}^k$:
%\begin{enumerate}
%\item[(a)] $g_{1,p}(z):=\log\sqrt[p]{\lambda}z-n\log z$, 
%\item[(b)] $g_{2,p}(z):=\log\sqrt[p]{z^p+1}-n\log z$, 
%\item[(c)] $g_{3,p}(z):=\log\sqrt[p]{z^p+i}-n\log z$. 
%\end{enumerate}
%By Lemma \ref{lem:approx-root-BD} and Lemma \ref{lem:approx-sectors}, we can find an entire function $g$ such that, for all $0\leqslant l<p$,\margin{finish..}
%$$
%\left\{
%\begin{array}{lll}
%|g(z)-\omega_p^{s(l)}g_{j,p}(z)|<\varepsilon_k(|z|), & \mbox{ on } H_{j,p}^l,& \mbox{ if } e_n=\infty,\vspace{10pt}\\
%|g(z)-1/\bigl(\omega_p^{s(l)}g_{j,p}(z)\bigr)|<\varepsilon_j(|z|), & \mbox{ on } H_{j,p}^l, & \mbox{ if } e_n=0.
%\end{array}
%\right.
%$$
%where $s(n):=p(p^{-1}(n)+1 \pmod{N})$ and the map $p:\{0,\hdots,N-1\}\longrightarrow \{-q,\hdots, p\}\setminus\{0\}$ is defined by 
%$$
%p(n)\margin{modify}:=\left\{
%\begin{array}{ll}
%\begin{array}{l}
%\ds\min\bigl\{\min_{m<n} p^{m+1}(0),\ 0\bigr\}-1=\vspace{5pt}\\
%\ds \hspace{20pt}=-\#\{0\leqslant m<n \pmod{N}\ :\ e_m=0\}-1,
%\end{array} & \mbox{ if } e_{n}=0,\vspace{5pt}\\
%\begin{array}{l}
%\ds\max\bigl\{\max_{m<n} p^{m+1}(0),\ 0\bigr\}+1=\vspace{5pt}\\
%\ds\hspace{20pt}=\#\{0\leqslant m<n\pmod{N}\ :\ e_m=\infty\}+1,
%\end{array} & \mbox{ if } e_{n}=\infty,
%\end{array}
%\right.
%$$
%for all $n\in \N$. Note the similarity between $s$ and \eqref{eq:p-function} in the proof of Theorem \ref{thm:wandering-domains}, again $s$ denotes the subsequent domain that has not been used.
%
%Lemma \ref{lem:approx-root-BD}\margin{in case it\\ works...} ensures that $z^n\exp\bigl( g(H_{i,p})\bigr)\subseteq H_{i,p}$ and hence $H_{i,p}\subseteq F(f_{i,p})$ is a Baker domain.
%
%
%
%$$
%\xymatrix{
%\C \ar[r]^{z+1} \ar[d]_{z^p} & \C\\
%\C \ar[r]_{z^ne^{g(z)}}  & \C \ar[u]_{\sqrt[p]{z}}\\
%}
%$$
%
%

Finally, a similar argument to that in the proof of Theorem~\ref{thm:baker-domains-entire} shows that the Fatou components that we have constructed are hyperbolic Baker domains; we omit the details.
\end{proof}

\red{

%%%\section{Doubly connected Baker domains}
%%%
%%%One of the main differences between the iteration of transcendental entire functions and that of transcendental self-maps of $\mathbb{C}^*$ lies in their multiply connected Fatou components. On the one hand, for transcendental entire functions, Baker \cite{baker84} showed that if a Fatou component $U$ is multiply connected, then $U$ is necessarily a wandering domain and all its iterates lie in multiply connected Fatou components. On the other hand, Baker \cite{baker87} proved that transcendental self-maps of $\mathbb{C}^*$ have at most one multiply connected Fatou component, which must be doubly connected and separate zero from infinity. Moreover, Baker and Dom\'inguez \cite{baker-dominguez98} showed that if $U$ is periodic and bounded away from zero and infinity, then $U$ is a Herman ring. Baker and Dom\'inguez remarked that if $U$ is unbounded (in $\mathbb{C}^*$), then $U$ may be other kinds of periodic Fatou components, and gave the example of an unbounded basin of attraction. However, there was no example of a transcendental self-map of $\mathbb{C}^*$ with a doubly connected Baker domain. 
%%%
%%%\begin{dfn}[Doubly unbounded set]
%%%We say that a set $X\subseteq \mathbb{C}^*$ is \textit{doubly unbounded} if $\{0,\infty\}\subseteq \widehat{X}$, where $\widehat{X}$ denotes the closure of $X$ in $\hat{\mathbb{C}}$.
%%%\end{dfn}
%%%
%%%We start by proving that doubly-connected Baker domains are only possible for transcendental self-maps of $\C^*$ of index zero. We also adapt the ideas from the proof of \cite[Theorem~3.1]{baker-kotus-lu-3} to show that for transcendental self-maps of $\mathbb{C}^*$ all the components of the boundary of a Baker domain have to be unbounded in $\mathbb{C}^*$. Hence, doubly connected Baker domains are doubly unbounded in $\mathbb{C}^*$.
%%%
%%%\begin{lem}
%%%\label{lem:2-unbdd}
%%%Let $f$ be a transcendental self-map of $\C^*$ and let $U$ be a doubly connected Baker domain of $f$. Then $\mbox{ind}(f)=0$ and $U$ is doubly unbounded.
%%%\end{lem}
%%%\begin{proof}%[Proof of Lemma~\ref{lem:2-unbdd}]
%%%Since $\mbox{ind}(f^p)=\mbox{ind}(f)^p$ and $F(f^p)=F(f)$ for $p\in\mathbb N$, we can assume without loss of generality that the Baker domain $U$ is invariant by taking a suitable iterate $f^p$ of $f$. 
%%%
%%%Let $\alpha\subseteq U$ be a simple closed curve around the origin. By \cite[Theorems~1.1 and 1.4]{martipete1}, there exist two curves in the complement of $U$ whose closure in~$\hat{\mathbb{C}}$ contains zero and infinity respectively. Since $\alpha$ is a compact subset of $U$ and $U\subseteq I(f)$, for every $R>0$ there exists $N=N(R)>0$ such that $f^n(\alpha)$ is contained in the complement of the annulus $A(1/R,R)$ for all $n>N$. Thus, the curve $f^n(\alpha)$ has index 0 with respect to the origin for all sufficiently large $n\in\mathbb{N}$, and hence $\textup{ind}(f)=0$.
%%%
%%%Now we will show that $\widehat{U}$, the closure of $U$ in $\hat{\mathbb{C}}$, contains both zero and infinity. Suppose to the contrary that $\widehat{U}\cap \{0,\infty\}=\{\infty\}$. Let $\gamma_0$ denote the component of $\partial U$ that separates zero from $U$. Note that since $U$ is doubly connected and unbounded, all the other components of $\partial U$ contain infinity in their closure in $\hat{\mathbb{C}}$. 
%%%
%%%Let $\alpha\subseteq U$ be a simple closed curve around the origin as before. We will show by induction that for every $n\in\mathbb{N}_0$, there exists a simple closed curve $\beta_n\subseteq f^n(\alpha)$ that separates $\gamma_0$ from $\partial U\setminus \gamma_0$. For $n=0$, the statement is trivially satisfied by taking $\beta_0=\alpha$. Suppose that there is a curve $\beta_n\subseteq f^n(\alpha)$ that separates $\gamma_0$ from $\partial U\setminus \gamma_0$. Let $D_n$ be the domain bounded by $\gamma_0$ and $f^n(\alpha)$. Then, we have\linebreak $\partial f(D_n)\subseteq f(\partial D_n)=f(\gamma_0)\cup f(\beta_n)$. If $f(\gamma_0)\subseteq \gamma_0$, since $\beta_n\subseteq U$, there exists a simple closed curve $\beta_{n+1}\subseteq f(\beta_n)$ that separates $\gamma_0$ from the set $\partial U\setminus \gamma_0$. Otherwise, if $f(\gamma_0)\subseteq \partial U\setminus \gamma_0$, the existence of  $\beta_{n+1}\subseteq f(\beta_n)$ is due to the fact that $\gamma_0\subseteq J(f)$ and hence cannot be part of $f(D_n)$. Thus, in both cases there exists a simple closed curve $\beta_{n+1}\subseteq f(\beta_n)\subseteq f^{n+1}(\alpha)$ that separates $\gamma_0$ from $\partial U\setminus \gamma_0$ and the claim holds for all $n\in\mathbb{N}_0$ by induction. However, note that this contradicts the fact that $f^n(\alpha)\subseteq \mathbb{C}^*\setminus A(1/R,R)$ for all $n>N(R)$, a set which does not contain any loops around the origin for $R>0$ sufficiently large. Therefore all the components of $\partial U$ are unbounded in $\mathbb{C}^*$.
%%%\end{proof}
%%%
%%%We now construct a transcendental self-map of $\C^*$ with a doubly connected Baker domain.
%%%
%%%\begin{thm}
%%%There exists a transcendental self-map of $\C^*$ with a doubly connected Baker domain.
%%%\label{thm:2con-bd}
%%%\end{thm}
%%%\begin{proof}[Proof of Theorem~\ref{thm:2con-bd}]
%%%We construct a transcendental self-map of $\mathbb{C}^*$ with a doubly connected Baker domain $U$ that is invariant and in which points escape to infinity. More precisely, the Baker domain $U$ will contain the set
%%%$$
%%%V:=\{z\in \C^*\ :  \textup{Re}\, z\geqslant -2,\ |z|\geqslant 6/7\}
%%%$$
%%%which is doubly connected (see Figure~\ref{fig:sketch-bd-cstar-2sides}). Since Fatou components of transcendental self-maps of $\mathbb{C}^*$ are either simply or doubly connected by \cite[Theorem~1]{baker87}, the resulting Baker domain $U$ will be doubly connected. As before, we use approximation theory to construct a non-constant entire function $g$ so that the function 
%%%$$
%%%f(z):=z^n\exp(g(z)+1/z^2)
%%%$$
%%%has the required properties. 
%%%
%%%Define the half-planes \mbox{$W:=\{z\in \C :  \textup{Re}\, z\geqslant 2\}$ and $H:=\{z\in \C :  \textup{Re}\, z\geqslant -2\}$} and the disc $D:=\overline{D(0,7/6)}$. We will construct the entire function $g$ by approximating it on the closed half-plane $H$, which satisfies the hypothesis of Lemma \ref{lem:approx-sectors}. Note that $V=H\cap (1/D)$ (see Figure~\ref{fig:sketch-bd-cstar-2sides}).
%%%%put $\rho:=7/6$ and $d=1/6$. 
%%%
%%%\begin{figure}[ht!]
%%%\centering
%%%%\vspace{30pt}
%%%\def\svgwidth{.60\linewidth}
%%%\input{Figures/bd-cstar-2conn.pdf_tex}
%%%%\vspace{15pt}
%%%\caption{Sketch of the construction of the function $f$ with a doubly connected Baker domain in the proof of Theorem~\ref{thm:2con-bd}.}
%%%\label{fig:sketch-bd-cstar-2sides} 
%%%\end{figure}
%%%
%%%Take $\lambda>1$ and, for $z\in H$, define the functions $T(z):=\lambda(z+4)$ and
%%%$$
%%%\hat{g}(z):=\log\bigl(T(z)\bigr)=\log\bigl(\lambda(z+4)\bigr).
%%%$$
%%%Since $T(H)\subseteq W$ and we can choose an analytic branch of the logarithm on $W$, we have $\hat{g}\in A(H)$. For $r\geqslant 1$, we define the positive continuous function $\varepsilon$ by
%%%$$
%%%\varepsilon(r):=\min\{k^{-2},r^{-2}\}/(2r^2)
%%%$$
%%%where the constant $k>0$ is sufficiently large so that, for all $z\in \log T(H)$ with $\textup{Re} z<k$, the disc $D(z,k^{-2})$ is compactly contained in $\log W$ and, moreover, if $\delta_1$ is the function from Lemma~\ref{lem:approx-root-BD}, then 
%%%$$
%%%\varepsilon(r)\cdot 2r^2<\delta_1(\ln (\lambda r)) <\hdots\hdots\quad \mbox{ for } r\geqslant k, 
%%%$$
%%%which, as before, is possible since 
%%%$$
%%%\delta_1(\ln (\lambda r)) \sim \frac{2(\lambda-1)}{\lambda r} \quad \mbox{ as } r\to+\infty.
%%%$$
%%%Since $\varepsilon$ satisfies 
%%%$$
%%%\int_1^{+\infty} r^{-3/2}\ln\varepsilon_\infty(r)dt>-\infty,
%%%$$
%%%by Lemma \ref{lem:approx-sectors} (with $\alpha=2\pi$), there is an entire function~$g$ such that
%%%\begin{equation}
%%%|g(z)-\hat{g}(z)|<\left\{\begin{array}{ll}
%%%\varepsilon_\infty(|z|) & \mbox{ for } z\in V_\infty\cup B_\infty,\vspace{5pt}\\
%%%1/2 & \mbox{ for } z\in D.
%%%\end{array}\right.
%%%\label{eq:bd-entire-approx}
%%%\end{equation}
%%%
%%%$$\vdots $$
%%%
%%%
%%%
%%%
%%%
%%%
%%%Note that $W$ is as in Lemma~\ref{lem:approx-BD} and, for $z\in W$,
%%%$$
%%%\textup{Re} \exp(\hat{g}(z))=\lambda \,\textup{Re} z+4\lambda >\lambda \,\textup{Re} z
%%%$$
%%%and hence...\\ .\\
%%%
%%%
%%%
%%%For $r>0$, we define the positive continuous function $\varepsilon_\infty$ by
%%%$$
%%%\varepsilon_\infty(r):=\min\{d'_\infty,\ k_\infty^{-(N+1)},\ r^{-(N+1)}\} /(2r^{N+1})
%%%$$
%%%where the constant $d'_\infty>0$ is so small that $|e^z-1|<d_\infty$ for $|z|<d'_\infty$ and the constant $k_\infty>0$ is so large that, for all $z\in \log T_\lambda(W)$ with $\textup{Re}\,z<k_\infty$, the disc $D(z,k_\infty^{-(N+1)})$ is compactly contained in $\log W$ and, moreover, if $\delta_{N}(r)$ is the function from Lemma~\ref{lem:approx-root-BD}, then 
%%%$$
%%%\varepsilon_\infty(r)\cdot 2r^{N+1}<\delta_{N}(\ln (\lambda r)) \quad \mbox{ for } r\geqslant k_\infty, 
%%%$$
%%%which, as before, is possible since 
%%%$$
%%%\delta_{N}(\ln (\lambda r)) \sim \frac{2(\lambda-1)}{N\lambda^Nr^N} \quad \mbox{ as } r\to+\infty.
%%%$$
%%%Since $\varepsilon_\infty$ satisfies 
%%%$$
%%%\int_1^{+\infty} r^{-3/2}\ln\varepsilon_\infty(r)dt>-\infty,
%%%$$
%%%by Lemma \ref{lem:approx-sectors} (with $\alpha=2\pi$), there is an entire function~$g$ such that
%%%\begin{equation}
%%%|g(z)-\hat{g}(z)|<\left\{\begin{array}{ll}
%%%\varepsilon_\infty(|z|) & \mbox{ for } z\in V_\infty\cup B_\infty,\vspace{5pt}\\
%%%1/2 & \mbox{ for } z\in D.
%%%\end{array}\right.
%%%\label{eq:bd-entire-approx}
%%%\end{equation}
%%%
%%%Similarly, we can construct an entire function $h$ that approximates a function $\hat{h}\in A(F_0)$ so that the function
%%%$$
%%%\begin{array}{rl}
%%%f(z):=&\hspace{-6pt}z^n\exp(g(z)z^{N+1}+h(1/z)/z^{N+1})\vspace{5pt}\\
%%%=&\hspace{-6pt}z^n\exp(\hat{g}(z)z^{N+1})\exp(\hat{h}(1/z)/z^{N+1}) \cdot \vspace{5pt}\\
%%%&\cdot \exp((g(z)-\hat{g}(z))z^{N+1})\exp((h(z)-\hat{h}(z))/z^{N+1})
%%%\end{array}
%%%$$
%%%%\begin{array}{rl}
%%%%f(z):=\hspace{-6pt}&z^n\exp(g(z)z^{4N}+h(1/z)/z^{4N}) \vspace{5pt}\\
%%%%=\hspace{-6pt}&z^n\exp(\hat{g}(z)z^{4N}+\hat{h}(1/z)/z^{4N})\exp((g(z)-\hat{g}(z))z^{4N})\exp((h(z)-\hat{h}(z))/z^{4N})
%%%%\end{array}
%%%%$$
%%%has the desired properties. Observe that if $z\in V_\infty\cup B_\infty$, then $1/z\in D$ and if $1/z\in V_0\cup B_0$, then $z\in D$. Thus, $\hat{h}(1/z)=0$ for $z\in V_\infty\cup B_\infty$ and
%%%$$
%%%\begin{array}{c}
%%%|\hat{h}(1/z)/z^{N+1}+(g(z)-\hat{g}(z))z^{N+1}+(h(z)-\hat{h}(z))/z^{N+1}|\leqslant\vspace{5pt}\\
%%%\leqslant 0+1/(2|z|^{N+1})+1/(2|z|^{N+1})=1/|z|^{N+1}
%%%\end{array}
%%%$$
%%%for $z\in V_\infty\cup B_\infty$. 
%%%\end{proof}

}


\bibliography{bibliography}

\end{document}