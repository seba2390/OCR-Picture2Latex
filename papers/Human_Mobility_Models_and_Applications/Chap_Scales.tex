In this section, we will discuss some of the fundamental metrics used to characterize mobility as well as the associated spatio-temporal scales at which they are relevant. We will then move onto a discussion of some of the physics associated with mobility, including the relations between distance, time and velocity. We end the section with a discussion of energy arguments and interpolation of spatio-temporal scales through the lens of multimodality. 

\subsection{General Metrics} \label{sec:metrics}

\subsubsection{Jump Lengths}

A key factor in modeling human mobility is the distance an individual travels in a given time period. Measures of distance are often dependent on the source of data being used, and the terms flight length, jump length, displacement, and trip refer to different distance measures that may be extracted from data. For example, early measures of mobility leveraged information about the spatial trajectories of bank-notes (one can think of this as an aggregation of many individual trips as a given banknote necessarily changes many hands). Conventionally, the distance between two instances of the appearance of a banknote in the measured data is termed a \emph{jump length}~\cite{brockmann_2006_scaling}. Later, higher resolution measurements of movement were provided by Call Detail Records (CDRs). Here one could reliably measure the location (and corresponding displacement) of \emph{individuals} based on placement of mobile phone towers that are pinged when one makes a call. Associated with the displacement between two successive calls is a time interval~\cite{gonzalez_2008_understanding}, that provides an estimate of the stay of an individual in a location (waiting-times), although it is not possible to detect the position of the user between two consecutive calls. Even higher resolution spatio-temporal data available from Global Positioning System (GPS) allows one to make more stringent definitions of displacement. In addition to the standard definition of jump-lengths~\cite{shin_2008_levy}, one can now define a displacement between \emph{stops}, i.e. the displacement of individuals between two locations, given that they spent a \emph{minimum amount of time} per location~\cite{bazzani_2010_statistical}. 
Indeed, the use of the terms ``stop'', ``trip'' or ``displacement'' reflects human behavioral tendencies that motivate people to go from point A to point B. It is therefore important to verify that a ``stop'' corresponds to an actual behavior and that it is not artificially generated by oversampled or undersampled signals~\cite{turchin_1998_quantitative}. 

Regardless of how one chooses to constrain it, the jump length, typically denoted as $\Delta r$, is defined as the euclidean distance between $r(t)$ and $r(t+dt)$ corresponding to locations  recorded at intervals $t$ and $t+dt$. Of particular interest is the distribution of $\Delta r$ within a population; how likely is it that a random member of the population will travel a distance $r$ from their origin location in a time $dt$? In order to measure this, when modeling human mobility it is common to consider the probability distribution function (PDF) of jump lengths, $P(\Delta r)$. This may be defined as the probability of finding a displacement $\Delta r$ in a short time step $dt$. %({\bf Figure to be added})

Brockmann \et~\cite{brockmann_2006_scaling} determined $P(\Delta r)$ for the trajectories of dollar bills and the distribution of jump lengths was observed to follow a power law, $P(\Delta r) \sim \Delta r^{-(1 + \beta)}$, with exponent $\beta = 0.59 \pm 0.02$, independent of the size (in population) of the entry point of a bank note (see Fig.~\ref{fig:scales_brockmann_1c}).
The power-law behavior of $P(\Delta r)$ has also been observed in the trajectories of mobile phone users ~\cite{gonzalez_2008_understanding, song_2010_modelling}.  
The empirical distribution of displacements obtained from mobile phone data analyzed in \cite{gonzalez_2008_understanding} is well approximated by a truncated power-law distribution: 
%
\begin{equation}\label{eq:pdr}
P(\Delta r) = (\Delta r - \Delta r_0)^{-(1 + \beta)} \exp{(- \Delta r / \kappa)}
\end{equation}
%
where $r_0$ and $\kappa$ represent cutoffs at small and large values of $\Delta r$. 
The value of the observed scaling exponent in Eq.~(\ref{eq:pdr}) is $\beta = 0.75 \pm 0.15$, 
not far from $\beta = 0.59 \pm 0.02$ obtained from bank note dispersal, suggesting that the two distributions may capture the same fundamental mechanism driving human mobility patterns. 
Zhao \et ~\cite{zhao_2015_explaining} measured the distribution of jump lengths using two GPS data sets that, along with location and time stamps, also included transportation mode. From this, the jump length $\Delta r$ was determined as the longest straight line distance between two locations without a change of direction. Wait times, $\Delta t$, were taken to be the time spent in a particular transportation mode. A single trip may be made up of several flights, each of which have a corresponding transportation mode taken directly from the dataset. The transportation modes were grouped into four categories; Walk/Run, Car/Bus/Taxi, Subway/Train and Bike.
The distribution of jump lengths is found to be log-normal if each transportation mode is considered individually, whereas $P(\Delta r)$ for all modes combined follows a truncated power-law with $\beta = 0.55$ and $0.39$  for the two datasets. The distribution of wait times, $P(\Delta t)$, was found to be exponential, corresponding to a large number of Walk/Run flights that connect the use of other transportation modes within a single trip and therefore have only a short duration. This result provides further insight into studies using on bank note and mobile phone data ~\cite{brockmann_2006_scaling, gonzalez_2008_understanding, song_2010_modelling} in that the power-law behavior of the distribution of jump lengths is the result of combining several distinct log-normal distributions, corresponding to the different transportation modes available, with exponentially distributed wait times. 
GPS devices installed in cars transmit the location of the car to a high degree of accuracy every few seconds when the car is in motion. The analysis of such data from over 150,000 cars in Italy over a period of one month~\cite{pappalardo_2013_understanding} found that the probability density function, $P(\Delta r)$ had two different regimes; an exponential distribution up to a characteristic distance of $\sim$20 km corresponding to inter-city travels, and a power-law with $\beta=1.53$ corresponding to intra-city movement.  Here, $\Delta r$ corresponds to the true length of a single trip. Trips are distinguished by records that are greater than 20 mins apart; as the GPS device only transmits when the car is moving, using a threshold of 20 mins allows for small stops, such as waiting at traffic lights or re-fueling, to be accounted for and included within a single trip.
The exponent, $\beta$, is significantly higher than those obtained for bank note dispersal~\cite{brockmann_2006_scaling} and mobile phone data~\cite{gonzalez_2008_understanding, song_2010_modelling}. The authors suggest this is due to the limitation in the GPS data caused by restrictions on the size of the area covered (500km in length), and the physical limitation on the distance an individual will drive in a single trip.


\begin{figure}[t!]
\centering
\includegraphics[width=0.8\textwidth]{Figures_Scales/2006_brockmann_nature_1c}
\caption{The short-time dispersal kernel of bank notes. The measured probability density function $P(r)$ of traversing a distance $r$ in less than $T = 4$ days is depicted in blue symbols. It is computed from an ensemble of 20,540 short-time displacements. The dashed black line indicates a power law $P(r)\sim r^{-(1+ \beta)}$ with an exponent of $\beta \sim 0.59$. The inset shows $P(r)$ for three classes of initial entry locations (black triangles for metropolitan areas, diamonds for cities of intermediate size, circles for small towns). Their decay is consistent with the measured exponent $\beta = 0.59$ (dashed line). Figure from~\cite{brockmann_2006_scaling}.}
\label{fig:scales_brockmann_1c}
\end{figure}



\subsubsection{Mean Square Displacement (MSD)}\label{sec:msd}

Research into the distribution of jump lengths suggests that individual trajectories can be described by L\'evy flights, a family of models associated with random walks (described in detail in \sectionname~\ref{sec:rw}). In this context, a common measure of the potential exploration of area by an individual is the Mean Square Displacement (MSD):
\begin{equation}
\mathrm{MSD}(t) = \langle (\mathbf{r}(t) - \mathbf{r}_0)^2 \rangle \equiv \langle \mathbf{\Delta r}(t)^2 \rangle.
\end{equation}
Here $\mathbf{r}_0$ is a vector marking the origin of the individual relative to some reference point, or in other words, the location at which a particular trajectory starts, while $\mathbf{r}(t)$ measures the subsequent position of the individual at time $t$. 
%As evident from the definition of the metric, it serves as a measure of the spatial extent of random exploration. Of particular interest is t
The scaling of MSD with time provides a measure of the type of diffusion of individuals relative to their starting point in a trip.
The MSD(t) has the unit of an area and it corresponds to the average squared distance from the origin after a time $t$. 
In general, if the individual mobility trajectories follow a so-called Continuous Time Random Walk, (Cf. \sectionname~\ref{sec:ctrw}), the MSD follows the form $\langle \Delta \mathbf{r}(t)^2 \rangle \sim t^{\nu}$ with $\nu = 2 \alpha /\beta$ where $\alpha$ and $\beta$ are the exponents of the waiting-time, i.e. the time interval between two consecutive jumps, and jump length PDFs~\cite{brockmann_2006_scaling}. 
In particular, for a random walk (ordinary diffusion) $\nu = 1$.
Yet a CTRW type process is not entirely a realistic representation of how humans actually move. In particular, a random walker will tend to drift away from the origin of its trajectory rather rapidly, the longer the elapsed time. 
%Technically, when $\beta < 2\alpha$, the walk is said to be super-diffusive, i.e. very rapid movement away from the initial location. 
Whereas, as one would imagine, empirical measurements indicate that people have a tendency (on average) to return home on a daily basis~\cite{gonzalez_2008_understanding}. Furthermore, the means of transportation constrain the maximum jump length, therefore it has been observed that the PDF's $P(\Delta t)$ and $P(\Delta r)$ have exponential cutoffs; this implies that their variances are finite and hence in the long-time limit the scaling of the MSD will asymptotically converge to that of Brownian motion, $\mathrm{MSD(t)} \sim t^{1/2}$.

To account for this discrepancy, refinements were made in~\cite{song_2010_modelling}, where analysis was restricted to high-resolution data with a combination of mobile phone records that provided locations each time a call was made, along with location data recorded by mobile services at hourly intervals. The newer measurements revealed that human mobility apparently follows an ultra-slow diffusive process characterized by a slower than logarithmic growth of the MSD with time, $\mathrm{MSD}(t) \sim (\log \log(t))^{2/\beta}$ (see \figurename~\ref{fig:scales_song_2}). 

\begin{figure}[t!]
\centering
\includegraphics[width=0.8\textwidth]{Figures_Scales/2010_song_NaturePhys_1c}
\caption{MSD versus time for groups of individuals with different radii of gyration. The gray line represents the analytic prediction of CTRW (Cf. \sectionname~\ref{sec:ctrw}) whereas the orange line represents the analytic prediction of the asymptotic behavior $MSD(t) \sim (\log \log(t))^{2/\beta}$, which more accurately reflects the movement pattern of humans. Figure from~\cite{song_2010_modelling}.}
 \label{fig:scales_song_2}
\end{figure}

Yet, the picture is typically more complex. Indeed, a common problem arising when calculating the MSD, is that the location of an individual's origin is often ambiguous~\cite{shin_2008_levy}. One method of overcoming this issue is to take the average of MSD values measured by varying the origin among all locations that an individual visits~\cite{dimilla_1993_maximal, maruyama_2003_truncated}. Such an approach was taken in~\cite{shin_2008_levy} where they analyzed the GPS traces of individuals in a number of different location, including a university campus (KAIST), New York City, Disney World, and a US state fair. The averaging procedure applied to the MSD revealed two types of temporal behavior; up to a certain time ($\sim 30$ minutes) the participants moved super-diffusively (as expected in a CTRW process), whereas for periods greater than $30$ minutes, the individuals moved sub-diffusively. %, more in line with the observations made by~\cite{song_2010_modelling}. 
%(Cf. \figurename \ref{fig:scales_rhee}). 
It is here that the notion of scales becomes important. The authors explain this dichotomous behavior by showing that at short time scales, the distribution of jump lengths follow a power-law distribution, whereas at longer time scales, jump lengths are more homogeneous, following a Gaussian distribution. While the combination of power law waiting times and jump lengths lead to super-diffusive behavior at shorter spatio-temporal scales, the Gaussian distribution of jump lengths (while maintaining a power law behavior for the waiting times) leads to mobility that appears sub-diffusive~\cite{vazquez_1999_diffusion} at larger spatio-temporal scales. 

% \begin{figure}[htbp]
% \centering
% \includegraphics[width=0.6\textwidth]{Figures_Scales/2007_Rhee_11.pdf}
% \caption{}
% \label{fig:scales_rhee}
% \end{figure}
%
%\GG {\bf Need a good way here (sentence or two) to transition to the next section. In other words a good way to move from MSD to radius of gyration}

\subsubsection{Radius of Gyration}\label{sec:rg}
%\FS
The sub-diffusive behavior of humans at certain scales suggests that they tend to move a \emph{characteristic distance} away from their starting locations. This distance can be quantified by the so-called, radius of gyration, $r_g$, defined as the root mean square distance of a set of points from a given axis. Traditional applications of this measure are rooted in physics (where it is related to the mass moment of inertia) and engineering (distribution of cross sectional area in a column). A general formulation of $r_g$ is given by:
\begin{equation}\label{eq:rg_def}
 r_{g} = \sqrt{\frac{1}{N} \sum_{i=1}^{N}(\mathbf{r}_i - \mathbf{r}_0)^2}
\end{equation}
where $\mathbf{r}_i$ are the coordinates of the $N$ individual points (or $N$ measurements of location) and $\mathbf{r}_0$ is the position vector of the center of mass of the set of points, $\mathbf{r}_{cm} = \sum_{i=1}^{N} \mathbf{r}_i / N$. When applied to human mobility, the radius of gyration can be used to characterize the typical distance of an individual from the center of mass of their trajectory.
The radius of gyration is not equivalent to the square root of the Mean Squared Displacement, defined in Eq.~\ref{eq:rg_def}, although the two quantities are related because the former is a central second moment while the latter is a second moment about the origin. 
\begin{figure}[ht!]
\centering
\includegraphics[width=0.7\textwidth]{Figures_Scales/2008_gonzalez_Nature_1d}
\caption{The distribution $P(r_g)$ of the radius of gyration measured for two sets of mobile phone users labeled $D_1$ and $D_2$, where $r_g(T)$ was measured after 6 months of observation. The solid line represents a truncated power-law fit. The dotted, dashed, and dot-dashed curves show $P(r_g)$ obtained from Random Walk, L\'evy flight, and truncated L\'evy flight models respectively. Figure from~\cite{gonzalez_2008_understanding}.}
\label{fig:scales_gonzalez_1}
\end{figure}

Gonzalez \et\cite{gonzalez_2008_understanding} used mobile phone call records to determine the radius of gyration for two sets of users; one set with high temporal resolution (position recorded every two hours over one week), and one set with lower resolution on a longer (6 month) scale. The radius of gyration was calculated for each user, $a$, up to a time $t$  using the recorded positions, $\mathbf{r}_i^{(a)}$, $i=1...,N^{(a)}$. The value of $\mathbf{r}_0$ was taken as the center of mass of each users trajectory; $\mathbf{r}_{cm}^{a} = \sum_{i=1}^{N^{(a)}} \mathbf{r}_i^{(a)}/ N^{(a)}$.
The distribution of $r_g$, $P(r_g)$, was found to follow a truncated power law of the form:
%
\begin{equation}\label{eq:prg}
P(r_g) = (r_g - r_g^{(0)})^{-(1+\beta_r)} \exp{(- r_g / \kappa_r)}
\end{equation}
%
where $r^{(0)}_0$ represents the minimum radius of gyration cutoff due to the spatial sensitivity of the data, and $\kappa_r$ represents an upper cutoff mostly due to the finite size of the study area (see Fig. \ref{fig:scales_gonzalez_1}).

\begin{figure}[t!]
\centering
\includegraphics[width=0.8\textwidth]{Figures_GenModels/fig_gonzalez_2008_understanding_2}
\caption{(a) Radius of gyration $r_g(t)$ versus time for mobile phone users separated into three groups according to their final $r_g(T)$, where $T=6$ months. 
%The black curves
%correspond to the analytical predictions for the random walk models,
%increasing with time as Ærg(t)æ | LF,TLF,t3/21b (solid curve) and
%Ærg(t)æ |RW,t0.5 (dotted curve). 
The dashed curves correspond to a logarithmic fit of the form $A + B \ln(t)$, where $A$ and $B$ are time-independent coefficients that depend on $r_g$. 
%
(b) Probability density function of individual travel distances $P(\Delta r | r_g)$ for users with $r_g =$ 4, 10, 40, 100 and 200 km. {\it Inset} Each group displays a different $P(\Delta r | r_g)$ distribution. {\it Main} After rescaling the distance and the distribution with $r_g$, the different curves collapse to a power law (solid line).
%
(c) Return probability distribution, $F_{pt}(t)$. The prominent peaks capture the tendency of humans to return regularly to the locations they visited before, in contrast with the smooth asymptotic behavior $\sim 1/(t ln(t)^2)$ (solid line) predicted for random walks. 
%
(d) A Zipf plot showing the frequency of visiting different locations. The symbols correspond to users that have been observed to visit $n_L = 5, 10, 30,$ and $50$ different locations. Denoting with $L$ the rank of the location listed in the order of the visit frequency, the data are well approximated by $R(L) \sim L^{-1}$. The inset is the same plot in linear scale, illustrating that $40\%$ of the time individuals are found at their first two preferred locations; bars indicate the standard error. Figure from~\cite{gonzalez_2008_understanding}.}
\label{fig:scales_gonzalez_2}
\end{figure}

Indeed, the measured value of the scaling exponent $\beta_r = 0.65 \pm 0.15$ indicates a significant degree of heterogeneity in the travel habits of the observed population. Measuring the conditional jump length distribution, $P(\Delta r | r_g)$, revealed that users with small $r_g$ travel mostly over small distances, whereas those with large $r_g$ tend to display a combination of many small and a few larger jump sizes. Once one accounts for this heterogeneity in travel habits by rescaling the distribution with respect to $r_g$, it leads to a collapse of the data onto a single curve thus,
%
\begin{equation}\label{eq:rgcoll}
P(\Delta r| r_g )  \sim \Delta r^{-(1+\beta_c)} F(\Delta r / r_g) \, .
\end{equation}
%
Here $\beta_c = 0.2 \pm 0.1$ is a ``universal'' scaling exponent, and $F(x)$ is a scaling function that is constant for $x<1$ and rapidly decreasing for $x \gg 1$. 

It is important to note that the distributions described above are not independent but are related by the equality $P(\Delta r) = \int_{r_0}^{\infty} P(\Delta r | r_g) P(r_g) d r_g$, where $P(\Delta r$) is the jump-length distribution introduced previously. If $P(\Delta r)$ has a power-law scaling exponent $\beta$, then we have $\beta = \beta_c + \beta_r$, which seems to be in good agreement with empirically measured values. 
This form of scaling corresponds to a type of random walk called a L\'evy flight (\sectionname~\ref{sec:levy}), and the results suggest that this may be the behavior of individuals up to their associated characteristic distance $r_g$ (and beyond which saturation effects take over). 

A way to uncover the saturation effect is to examine the time evolution of the radius of gyration, i.e. $r_g(t)$. 
In ~\cite{gonzalez_2008_understanding}, it was found that the average $r_g$ of mobile phone users displays a logarithmic increase with time, $\langle r_g(t) \rangle \sim A + B \ln t$ (see \figurename~\ref{fig:scales_gonzalez_2}), in contrast to a pure random walk where the scaling is of the form $\langle r_g(t) \rangle \sim t^{\theta}$. The observed saturation in $r_g$ can be attributed to the regularity in travel patterns of the individuals, and specifically to the high probability for an individual to return to a few highly-frequented locations. As pointed out before, the true picture is a mixture of behaviors. Supporting the conclusions made in~\cite{shin_2008_levy} while measuring the $\mathrm{MSD}$, Zhao \et\cite{zhao_2008_empirical} argue that humans engage in a mixture of superdiffusive and subdiffusive behaviors (depending on temporal scale), and the latter specifically can be explained as a consequence of the saturation of $r_g$. This was later reproduced by Song \et\cite{song_2010_modelling} and demonstrated to be the consequence of the fat-tailed distribution of the jump lengths (\figurename~\ref{fig:scales_song_1}). 


\begin{figure}[t!]
\centering
\includegraphics[width=0.6\textwidth]{Figures_Scales/2010_song_NaturePhys_4a}
\caption{The distribution $P(r_g)$ of the radius of gyration $r_g$ for mobile-phone users at different moments of time. The straight line, shown as a guide to the eye, represents a power-law decay with the exponent $1+\alpha \approx 1.55$. Figure from~\cite{song_2010_modelling}.}
\label{fig:scales_song_1}
\end{figure}

The radius of gyration is a measure to characterize the typical distance traveled by an individual and it depends both on the mutual distance of the locations visited and on the time spent (or the total number of visits) in each location. 
However, the radius of gyration does not allow us to quantify the relevance of each location in determining an individual's characteristic mobility. Indeed, an individual who spends a majority of time in their most visited locations, e.g. home and work, will have a large $r_g$ if these two locations happen to be quite far from each other. Conversely, even if these most visited locations are close to each other, large values of  $r_g$ may be reported if the individual happens to visit a number of distant locations. 

\begin{figure}[t!]
\centering
\includegraphics[width=0.8\textwidth]{Figures_Scales/fig_pappalardo_2015_returners_5}
\caption{The mobility networks of returners and explorers for $k = 2$. Nodes (circles) indicate the geographic locations visited by the individual, and each link denotes a travel observed between two locations. When the total $r_g$ is small, the two most important locations (red and blue) are close to each other for both two-explorers and two-returners. As the total radius increases the behavior of two-returners and two-explorers starts to differ; for returners, the two most important locations move away from each other; for explorers, they stay close and other clusters of locations emerge far from the center of mass (the gray cross). Figure from~\cite{pappalardo_2015_returners}.}
\label{fig:scales_pappalardo_2015}
\end{figure}


To disentangle these effects, one can study the influence of frequency of location visits on the characteristic distance traveled, by studying the $k$-radius of gyration, $r_g^{(k)}$, defined as the radius of gyration computed over an individual's $k$ most frequently visited locations $L_1, \dots, L_k$ (see next section) ~\cite{pappalardo_2015_returners}. One can express this modified form as
\begin{equation}
\label{eq:k_radius}
r_g^{(k)} = \sqrt{{\frac{1}{N_k}} \sum_{j = 1}^{k} n_j (\mathbf{r}_j - \mathbf{r}_{cm}^{\ (k)})^2},
\end{equation}
where $N_k$ is the sum of the visits to the $k$ most frequented locations, $\vec{r}_{cm}^{\ (k)}$ is the center of mass computed on those  locations and $n_j$ is the number of visits to the $j$-th most visited location. As an example, if $r_g^{(2)} \simeq r_g$, then the characteristic traveled distance is dominated by the two most frequented locations, whereas if  $r_g^{(2)} \ll r_g$ than the two most frequented locations do not offer an accurate characterization of the individual's travel pattern, requiring us to consider more locations.

The analysis of GSM and the GPS data revealed that there exist two distinct classes of individuals, {\em returners} and {\em explorers}. 
The characteristic distance traveled by $k$-returners is dominated by their recurrent movement between a few preferred locations, and their radius of gyration is well approximated by their $k$-radius of gyration for $k \geq 2$.
In contrast, $k$-explorers have a tendency to wander between a varying number of different locations and their $k$-radius of gyration is very small compared to their overall $r_g$ (see Fig.~\ref{fig:scales_pappalardo_2015}). 




\subsubsection{Most Frequented Locations and Motifs}
%\FS
When considering patterns in human mobility, particularly movements within a single day or week, it is essential to distinguish between locations based upon their importance. As already mentioned, people have a tendency to return home on a daily basis and therefore most daily and weekly trajectories will start and finish at the same location.
One method of quantifying the importance of a location is the use of ranks; the most visited location (likely home or work) would have rank 1, a school or local shop may have rank 2 or 3, etc. 
%\COMMENT{\ML We should maybe link this part to the "identification of important places" presented in the next section}
In ~\cite{gonzalez_2008_understanding}, the rank of a location was determined for each individual mobile user by the number of times their position was recorded in the vicinity of the cell tower covering that location. It was found that visitation frequency follows a Zipf law, that is the probability of finding a user at a location of rank $L$ is approximately $P(L) \sim 1/L$. 

Another method of distinguishing between locations is to construct each individual's mobility pattern as a network. Schneider \et\cite{schneider_2013_unravelling} used data from both mobile phone users and travel survey respondents to construct weekday mobility networks for each individual. These profiles consisted of nodes to represent locations visited and directed edges to represent trips between the locations. Every daily trip network started and ended at the home location of each user, which was determined as the location they were most often recorded at between 3:00am and 3:30am. For mobile phone users, only days where calls were made, and therefore location was recorded, during 8 or more 30-min time slots were included. Daily networks were constructed for weekdays only in order to identify patterns in mobility during a typical day.

% \begin{figure}[t!]
% \centering
% \includegraphics[width=0.8\textwidth]{Figures_Scales/fig_widhalm_2015_discovering_6.pdf}
% \caption{\COMMENT{this is the caption of the old figure. needs to be updated} Daily mobility patterns for two anonymous mobile phone users over a period of 10 days. The home location of each user is highlighted and connected over the entire observation period with a gray line. While the entire mobility profiles (black circles and gray lines in the xy-plane) are rather diverse, the individual daily profiles (brown to red from bottom to top for different days) share common features.
% %
% The left user prefers commuting to one place and visits the other
% locations during a single tour, whereas the right user prefers to visit the daily locations during a single tour. On the last day, both users visit not only four locations, but also share the same daily profile consisting of two tours with one and two destinations, respectively. Figure from~\cite{widhalm_2015_discovering}.}
% \label{fig:scales_schneider}
% \end{figure}

It was found that $\sim$ 90\% of the recorded trips made by all users can be described with just 17 daily networks. These 17 trip patterns can be described as motifs; a sub-network within a complex network ~\cite{alon_2007_network}. In this case, a daily network was considered to be a motif if it occurred more than 0.5\% in the datasets. The result that the daily patterns of human mobility can be constructed by just 17 trip networks suggests that these motifs represent the underlying regularities that exist in our daily movements and are therefore useful for the accurate modeling and simulation of humans' mobility patterns.

\subsubsection{Origin-Destination Matrices}\label{sec:odmatrix}

The Origin-Destination (OD) matrix {\bf T} is the standard object in aggregated mobility studies and transport planning. It provides an estimate of the number of individuals traveling between locations in a given area, over a given period of time. More precisely, an OD matrix is a $n\times m$ matrix where $n$ is the number of different ``Origin'' zones, $m$ is the number of ``Destination'' zones, and $T_{ij}$ is the number of people traveling from zone $i$ to
zone $j$. More commonly, an area under study is partitioned to an equal number of origin and destination points and therefore, $n=m$. The size $n$ of the OD matrix then depends on the spatial scale/resolution at which the data has been collected. Traditionally, zones are administrative units, whose size may vary from census and electoral units to entire municipalities, departments or states, depending on the question that motivated the development of the OD matrix. Obviously, the maximal spatial resolution of an OD matrix depends on the data source.

An OD matrix can be empirically derived, being estimated from
household or roadside travel surveys, traffic counts, and more
recently from individual digital footprints (see below). It can
also be the output of a model, like in the classic 4-step transport
model widely used in urban planning\footnote{1-Generate travel demands and offers in each spatial unit; 2-Distribute trips in space (generally thanks to a gravity model (see \sectionname~\ref{sec:models}); 3-Evaluate the model choice; and 4-Assign trips to routes}. OD
matrices and the 4-steps framework have been used in transport
planning since at least the middle of the $20^{\text{th}}$ century, and there exists an enormous literature that covers its origins, uses, models, synthetic indicators, and so on~\cite{ortuzar_2011_modeling}. 


OD matrices are of specific importance to models of aggregate flow, that is models of human mobility at the population level, rather than individual level. Examples of such models are gravity and intervening opportunities models (\sectionname~\ref{sec:models}). 
In empirical OD matrices, $T_{ij}$ indicates the number of travelers from $i$ to $j$ measured during observation time window, whereas if the OD matrix is a model's estimate or prediction $T_{ij}$ usually indicates the expected (average) number of travelers between the two locations. The OD matrix diagonal elements are usually 0 to indicate that people only move between distinct locations. In some studies the diagonal elements are larger than zero and indicate the number of non-traveling individuals in the origin location (or those who travel within the origin location). 
The fraction of travelers from the origin $i$ to all other locations can be calculated as $p_{ij} = T_{ij} / T_i$,  where $T_i = \sum_j T_{ij}$. In this case, the entry $p_{ij}$ may represent the probability of an individual located at $i$ to select location $j$ as their destination over all other possible locations.

% A glimpse at the history of the object and its uses

%OD are used to capture travels
%associated to different purposes (shopping flows, international
%leisure mobility, migration flows, etc.), or to capture all trips
%(whatever their function) during a certain period of time. They are
%often used for capturing the average daily regular travels, in the
%first place journey-to-work commuting, the most important purpose for
%urban mobility.

%Estimating commuting OD matrices using ICT data

Until the 2000's, the traditional approaches to estimate OD matrices consisted in relying on travel surveys or counting. The drawback of such approaches are related to their cost to setup, frequency of updates, and that they cover a limited sample of individuals (a few hundreds or thousands of households)~\cite{iqbal_2014_development}. Consequently, to circumvent these difficulties, a large body of literature has dealt with extracting OD matrices from recently available, individual digital footprints \cite{white_2002_extracting, caceres_2007_deriving, isaacman_2010_tale, calabrese_2011_estimating, jiang_2013_review, iqbal_2014_development, lenormand_2014_cross, alexander_2015_origin, toole_2015_path}. 
Early work on this topic has already been summarized in a dedicated review a few years ago~\cite{caceres_2008_review}. Here we don't go deeply into the details of the numerous methods that have been proposed, and give only the general picture for the case of average daily journey-to-work commuting. 

Starting from time-sequences of successive locations, the goal is to identify important places (``anchor points'') for each individual, such as the residence and workplace. The heuristic underlying method is straightforward: due to the circadian regular daily rhythms of human activity, one can make the reasonable assumption that for most individuals, the most frequent location of mobile devices will be very near the residence during non-working hours of weekdays, and during weekends. Similarly, during working hours on weekdays, it is likely that for most individuals their devices will be located just next to their workplace or location of primary activity (which include students, unemployed, retired, and employed individuals with a non-fixed workplace)~\cite{tizzoni_2014_use, lenormand_2014_cross, alexander_2015_origin, toole_2015_path}. In some cases validation of methodology was conducted by asking a small group of volunteers to complete a survey, whose results were then compared to the key locations extracted from the recorded activity of their mobile devices \cite{isaacman_2010_tale}.

Alexander \et\cite{alexander_2015_origin} proposed a method for constructing an OD matrix from mobile phone records. Such a method allows for data at the individual level to be aggregated in a way that lends itself to analysis at the population level. Implementing the proposed method, in order to infer a user's origin and destination, clustered locations are extracted from the mobile phone record. By clustering locations in close proximity to one another into a single origin or destination, noise from the data, such as inexact triangulation, is eliminated thus allowing for a more accurate analysis. The clustered locations are then assigned a type: work, home, or other. The assignment takes into account the time of day a user is observed there, the duration of their stay, and the day of the week.
By the nature of mobile phone records, location is only recorded when a user makes or receives a call, therefore the arrival times and duration of stay at a given location may not reflect the true times. To correct for this, surveys on trips may be used to assign true arrival times and duration from a probability distribution derived from the survey data. Then, trips can be constructed for every user between each location at which they are observed. In order to ensure that the final OD matrix is representative, users that do not satisfy certain conditions, such as on average of at least one visit to the home location per day, may be removed from the data set. The daily trips are then combined for all users and locations allowing for the aggregate flow between all possible locations to be determined and an origin destination matrix to be constructed.

% Equivalence of the various sources
% Equivalence of journey-to-work commuting OD matrices extracted from different data sources 
Lenormand \et\cite{lenormand_2014_cross} compared the journey-to-work OD matrices extracted from three different data sources (census, mobile phone CDR data, and Twitter data) in Madrid and Barcelona, two of the largest European urban areas. Projecting residential and workplace locations identified in each dataset on regular square grids, in order to have a common spatial resolution for each dataset, they measured the correlation between the OD matrices estimated through each data source, as a function of the level of spatial aggregation. They found good correlation for square grids of size $2km$ ($\rho > 0.9$), and very good correlation between the various sources when aggregating the individual data sources (Twitter, CDR data) on the map of administrative units (municipalities) used by the travel surveys ($\rho > 0.99$) (here $\rho$ is the Pearson correlation coefficient). Using a slightly different method that aggregates nearby cell towers when identifying important locations, Alexander \et\cite{alexander_2015_origin} estimated OD matrices of daily average commuting trips from CDR data in the Boston area, and then performed a sensitivity analysis of the correlation between this OD and the OD issued from transport surveys at different levels of spatial aggregation. Tizzoni \et\cite{tizzoni_2014_use} also compared OD estimated from census and mobile phone data at different scales of spatial aggregation. They found lower correlation values, but applied a simpler method to estimate the OD from mobile phone data, when the aforementioned papers applied more stringent filters to remove outliers, fusion phone antennas close to one another, etc. Several algorithms in the literature are reviewed in \cite{toole_2015_path}, and further successive refinements and improvements in the methodology have been made since~\cite{jiang_2013_review,iqbal_2014_development,c_2015_analyzing,alexander_2015_origin}.


% Meso-scale indicators to summarize and compare OD matrices


%An alterna crucial part of modeling human mobility is the representation of trips individuals make between two locations. One method is to construct networks where the edges between nodes represent such trips. OD matrices are an alternative construct for this purpose. 
%An OD matrix is a directed and weighted spatial network, and it is often a large network as well. Such matrices have the added advantage that they can be used to represent flows between locations and therefore a single matrix can contain all trip data for an entire dataset. Because it contains the entire information on travel flows inside a given territory, as such it does not provide clear, synthetic and useful information about the mobility structure. Generally speaking, it is a difficult task to
%extract high-level and synthetic information from large networks, and the extraction of such meso-scale information on the graph structure is the purpose of community detection methods, which is an entire research field by itself~\cite{fortunato_2010_community}. These methods group nodes in clusters according to certain criteria, and nodes in a given cluster have similar properties (for example, in the stochastic block modeling, nodes in a given group have similar neighborhood). Such methods are very interesting when one wants to extract meso-scale information from a network, however they are unable to construct expressive categories of flows and to propose a classification of mobility networks. Several types of flows can be distinguished in a city, some constitute its backbone by connecting major residential areas to employment centers, while other flows converge from smaller residential areas to important employment centers, or diverge from major residential neighborhoods to smaller activity areas. In addition, the spatial properties of these commuting flows are fundamental in cities. Louail \et\cite{louail_2015_uncovering} proposed a method named ICDR
%to extract a simple four-number signature of the mobility structure, by distinguishing and quantifying four types of urban commuting flows: Integrated, Convergent, Divergent, and Random. The method has been applied to 31 Spanish urban areas, and has allowed to point out the relation between city size and the spatial organization of mobility in a city.



\subsection{Physics of mobility}
\subsubsection{Distance, travel time, and effective speed}

In essence, the metrics described above, deal with the concepts of distances traveled and the time elapsed within and between journeys. Another important feature, which has not been discussed, is the notion of \emph{speed}. To understand the relation between these quantities, it is important to note that mobility occurs over a large variety of distances. Framing it in the context of urban agglomerations, intra-urban mobility occurs over distances typically in the range of 1-10 kms, while inter-urban displacements occur generally over a wider range of the order of 100 kms (or more for large countries such as the US), and inter-country and intercontinental trips cover distances typically of the order of 1000 kms. While to first order, one may think of time spent on a given trip being roughly proportional to the distance traveled, one has to take into account the mode of transportation, which itself depends on the distance. For short-range travel, slow transportation modes with many stops are utilized (such as walking or public transportation), and for longer distances, one typically takes fast trains or planes with comparatively fewer stops. Thus the relation between distance, travel-times, and therefore speed, is fairly nuanced, as can be seen in \figurename~\ref{fig:scales_varga}, where we show the relation between \emph{apparent} speed (\ie the geodesic  distance divided by the travel time) and the travel distance~\cite{varga_2016_further}.

\begin{figure}[t!]
  \centering
\includegraphics[width=.8\textwidth]{./Figures_Scales/fig_vargaetal_2016.PNG}
  \caption{Apparent speed versus travel distance. The boxes represent the intervals for the different transportation modes. In the insets, the average result for cars (top left) and for plane travel (bottom right) are shown. The dashed lines represent power law fits. Figure from \cite{varga_2016_further}.}
\label{fig:scales_varga}
\end{figure}

 
In general, a trip can consist of multiple connections and may in fact be multimodal (multiple components of travel) with the corresponding waiting-time distributions associated with the modes of transportation~\cite{gallotti_2014_anatomy}. Also, it is observed that the apparent speed $\overline{v}$ increases with travel distance according to the following power-law functional form~\cite{varga_2016_further} 
\begin{equation}
\overline{v}\sim r^\beta,
\end{equation}
where $\beta\approx 0.5$. This particular dependence is primarily due to a combination of the hierarchical structure of transportation systems~\cite{gallotti_2016_stochastic} and the fact that waiting-times (parking, take-off, landing, etc) decrease in proportion to  trip distance. 

For example, it has been observed for public transportation as well as personal cars~\cite{gallotti_2016_stochastic} that a plot of the average trip velocity as a function of the trip duration reveals an effective acceleration (\figurename~\ref{fig:scales_gallotti}) of the form 
\begin{equation}
\langle\overline{v}\rangle = v_0+at.
\label{eq:at}
\end{equation}
This effective acceleration is a direct consequence of the hierarchical organization of roads: the longer the trip, the more likely that higher velocity roads such as highways are used, thus trips can be decomposed into a ascending cascade to faster roads, followed by a descending cascade when approaching the target location. \equationname~\eqref{eq:at} implies that the distance evolves with trip duration as 
\begin{equation}
r\simeq v_0t+\frac{1}{2}at^2,
\label{eq:dt2}
\end{equation}
from which the relation $\langle\overline{v}\rangle \sim \sqrt{r}$ directly follows~\cite{varga_2016_further}.
\begin{figure}[t!]
  \centering
\includegraphics[width=.8\textwidth]{./Figures_Scales/fig_gallotti-2016}
  \caption{Empirical average speed versus the duration of the trip obtained from GPS data for cars (blue dots). The red solid line represents a fit to the form seen in Eq.~\eqref{eq:at} with $v_0 = 17.9$ km h$^{-1}$ and $a = 16.7$ km h$^{-2}$.  Observe the saturation at $t > 2$ hours due to the finite number of layers in the transportation hierarchy. The orange dashed lines represents the best fit to $\langle \mid \overline{v} \mid \rangle \propto \sqrt{t}$ which corresponds to a Brownian acceleration model. Figure from \cite{gallotti_2016_stochastic}.}
\label{fig:scales_gallotti}
\end{figure}

\begin{figure}[t!]
  \centering
\includegraphics[angle=-90,width=.85\textwidth]{./Figures_Scales/2014_Louf_Scirep_2}
  \caption{Variation of the total delay due to congestion with population for 97 urbanized areas in the US. A power law fit gives an exponent  of $1.270 \pm 0.067$. Figure from \cite{louf_2014_how}.} 
\label{fig:louf_congestion}
\end{figure}


\subsubsection{Travel time budget}
In the context of intra-urban mobility, Marchetti~\cite{Marchetti_1994}, based on ideas developed by Zahavi~\cite{Zahavi_1977}, proposed the idea of a travel time budget of about one hour per day irrespective of location. This implies that with improvements in transportation technology and corresponding increase in speed, a greater amount of distance is covered within the budgeted time, thus allowing for urban sprawl. This assumption can be reformulated as the \emph{rational locator hypothesis} \cite{levinson_2005_rational} which posits that individuals maintain approximately a constant journey-to-work travel times by adjusting their home and workplace. This was revisited in \cite{levinson_2005_rational} on data for travel times in Washington DC for 1968, 1988, 1994, and Twin Cities for 1990 and 2000. The results show that there is a strong dependence on geography: for the Washington DC greater urban area, travel times are relatively stable, while the results for Twin Cities show a marked increase of the commute time.
This has also been confirmed in another study \cite{louf_2014_how} that discussed the impact of congestion on mobility patterns. In particular, it has been measured for US cities that typical travel delay due to congestion increases with the population as 
\begin{equation}
\Delta\tau\sim P^{1+\delta}
\end{equation}
where $\delta\approx 0.3$, indicating that average commuting time increases with population, due to the (nonlinear) congestion effects, in sharp contrast with the travel time budget hypothesis (See \figurename~\ref{fig:louf_congestion}). 



% \begin{figure}[ht!]
%   \centering
% \includegraphics[width=0.8\textwidth]{./Figures_Scales/fig_helbing_2008_1.pdf}
%   \caption{Average travel times for different transportation modes versus years (from 1972 to 1998 for the UK). Figure from \cite{Kolbl_2003_energy}.}. 
% \label{fig:helbing1}
% \end{figure}
%


\subsubsection{Energy arguments}

Mobility, ultimately, is about energy. Indeed, moving an object of a certain mass at a certain speed requires a given amount of energy. Consequently, one would expect to apply energy concepts to understand human travel behavior \cite{kolbl_2003_energy}. First, it has been observed that the average travel times for different modes of transportation are inversely proportional to the energy consumption rates measured for the respective physical activities. Second, when daily travel-time distributions of different transport modes such as walking, cycling, bus, or car travel are appropriately scaled, they turn out to have a universal functional relationship (see \figurename~\ref{fig:helbing2}). 

\begin{figure}[t!]
  \centering
\includegraphics[angle=-90,width=.85\textwidth]{./Figures_Scales/fig_helbing_2008_2b}
  \caption{Rescaled travel time distribution for different transport modes (linear-log). Points represent different travel modes and the solid line is a fit to the rescaled energy distribution~\eqref{eq:energyrescaled}. Figure from \cite{kolbl_2003_energy}.}
\label{fig:helbing2}
\end{figure}

This corresponds to a canonical-like energy distribution (with exceptions for short trips) hinting at a law of constant average energy consumption related to daily traveling. The argument, first proposed by Kolbl \et\cite{kolbl_2003_energy},  goes as follows.
If we define the energy $E_i$ spent per transportation mode $i$, the average energy consumption per day $\overline{E}$ is constant and independent of the mode of transportation. The corresponding 
entropy is given by
\begin{equation}
S=-\int P(E_i)\ln P(E_i)\mathrm{d}E_i,
\end{equation}
and the constraints on the energy distribution read
\begin{equation}
\int P(E_i)\mathrm{d}E_i =1; \quad \int E_iP(E_i) =\overline{E}\mathrm{d}E_i,
\end{equation}
which leads to the canonical distribution
\begin{equation}
P(E_i)\sim \mathrm{e}^{-\beta E_i}.
\end{equation}
%
However, the data indicates that the probability to expend small energy is vanishing, and this was accounted for by introducing a cutoff of the form $e^{-\alpha\overline{E}/E_i}$. This cutoff term is reminiscent of the so-called ``Simonson effect''~\cite{hettinger_1989}, where short trips are not very likely to be taken with a given mode if the energy spent is much less than $\alpha \overline{E}$. The complete energy distribution is then given by
\begin{equation}
P(E_i)\sim \mathrm{e}^{-\alpha\overline{E}/E_i-\beta E_i}.
\label{eq:energyrescaled}
\end{equation}
%
This finding, highlights the importance of physical concepts in understanding mobility and more generally social phenomena. In particular, it contains only physical variables such as travel times and energies, that are both measurable, in contrast with utilities introduced in classical choice modeling that describe preferences that are not measurable.  Yet, as the authors point out, there are several issues that need to be accounted for, including measurement errors as well as multimodal trips that combine different transportation modes. 

\subsection{Interpolation of scales: the importance of multimodality}
\label{sec:multimodal}

As discussed previously, mobility occurs over multiple spatio-temporal scales (multimodal) and thus a comprehensive picture of human mobility requires an accounting of the effects of multimodality. Indeed, the transition between different modes of transport necessarily implies a temporal cost that increases in proportion to the complexity of a trip (number of modes, or indeed level of mode). For example, in the UK, a rough estimate shows that roughly $23\%$ of travel time is accounted for by connections made between modes in trips \cite{gallotti_2014_anatomy}.

As showed in \figurename~\ref{fig:anatomy}, the average structure of travel time varies with trip length $\ell$ and between cities. With the exception of London, short trips are composed of longer waiting times than riding times (i.e more time is spent being stationary than on the move), where the waiting times are mostly intra-layer in nature, due to bus-bus interchanges. If the transportation network is particularly multi-modal, inter-layer waiting times and walking times start to play a significant role for longer values of $\ell$. 

\begin{figure}[t!]
\centering
\includegraphics[angle=0, width=0.8\textwidth]{./Figures_Scales/2014_Gallotti_Scirep_6}
\caption{The Anatomy of the transportation networks of selected cities in the UK. (a) London, (b) Manchester, (c) Edinburgh: total travel time in function of trip length separated by mode of travel. (d) Different colors represent different regimes versus the trip length. (i) Red: the trips are mostly done on the bus layer and display waiting times larger than riding times (a regime not present in London). (ii) Green: riding times exceed waiting times and most of the distance is covered in the bus layer. (iii) Blue: riding times exceed waiting times and most of the distance is covered in the metro and rail layers. Figure from~\cite{gallotti_2014_anatomy}.}
\label{fig:anatomy}
\end{figure}

The effect of multiple scales on movement can be unpacked by defining two types of trajectories. One is the (ideal) quickest path that connects two points and neglects delays due to inter-modal connections (walking and waiting times). For this ideal path to exist, one would need perfect synchronization between modes (bus schedules aligning perfectly with train schedules for example). The other (more realistic) type of trajectories are  so-called \emph{time-respecting} paths that are the quickest paths accounting for inter-modal effects such as arrival, departure and connection times. 

For OD pairs $i$ and $j$, denoting the ideal path time as $\tau_m(i,j)$ and the corresponding time-respecting path as $\tau_t(i,j)$, the delay due to the lack of synchronization can be measured by
\begin{equation}
\delta(i,j)=\frac{\tau_t(i,j)}{\tau_m(i,j)}-1.
\end{equation}
Averaging this ratio over all OD pairs in a city, one obtains a characteristic delay $\overline{\delta}$. As an example, for cities in the UK the variation of $\overline{\delta}$ with trip length $l$ is unimodal, reaching its maximum $\delta_{max}$ for short trips and then decreasing with distance $\ell$ according to 
\begin{equation}
\overline{\delta} \approx \delta_{min} + \frac{\delta_{max}-\delta_{min}}{\ell^\nu},
\label{eq:deltafun}
\end{equation}
where $\nu\approx 0.5$. This surprising collapse (see \figurename~\ref{fig:anatomyb}) suggests an underlying process describing the accumulation of waiting and walking times along time-respecting paths.

 \begin{figure}[t!]
 \centering
 \includegraphics[angle=0, width=0.7\textwidth]{./Figures_Scales/2014_Gallotti_Scirep_5}
 \caption{Dependence of the synchronization inefficiency $\delta$ on path length $l$ for cities in the UK. All cities appear to collapse onto a curve described by Eq.~\eqref{eq:deltafun}. Figure from~\cite{gallotti_2014_anatomy}.}
 \label{fig:anatomyb}
 \end{figure}

It is clear for a given trip that the number of modes and the frequency of their use play an important role in determining $\overline{\delta}$. To measure their effect a natural quantity is the average number of stop events per unit time 
\begin{equation}
\Omega = \frac{\sum_\alpha C_\alpha}{\Delta t_{\alpha}},
\end{equation}
where $C_\alpha$ is the number of stop events in mode $\alpha$ and $\Delta t_{\alpha}$ the duration spent in that mode.

The quantity $\Omega$ can be thought of as a measure of the efficiency of transportation modes in terms of synchronization. The connection between  $\bar\delta$ and $\Omega$ appears to follow a power law \cite{gallotti_2014_anatomy}
\begin{equation}
\bar\delta \sim  \Omega^{-\mu},
\label{eq:deltaomega}
\end{equation}
where $\mu\approx 0.3\pm 0.1$.  The trend is due to the fact that larger values of $\Omega$ implies larger frequency and thus a better synchronization between modes. The observed small value of $\mu$ is however bad news in terms of efficiency: Decreasing inefficiency by a factor of 2 requires a 10-fold increase in $\Omega$ meaning essentially that the trip would need to be unimodal.

% The temporal aspect of the Public Transport Networks appears to be influential for trips covering all distances. Short time-respecting paths tend to be mostly similar to the minimal ones, and waiting times are directly added to the riding times of the associated minimal paths. Waiting times then represent the largest fraction of the total travel times, and at this scale an increase of bus departure frequency, or methods like timetables offsetting~\cite{Coffey_2012_missed,Nair_2012} of the bus service may represent a good optimization strategy. Longer time-respecting paths tend instead to diverge from minimal ones, and very large waiting times can be avoided thanks to the availability of alternative routes, and when it is possible, longer trips are progressively taking advantage of the multi-modality of the system. 

Thus as the results show, for large cities with multi-modal transportation system, the temporal cost due to interchange between layers plays a key role in the statistics of trips. Indeed, a challenge is posed due to the entanglement between  the temporal and multilayer aspects of the system. 

% Waiting time (together with walking time) does not represent a simple cost to minimize, but a price to pay to access to fast transportation. 

% The value of waiting and walking times are perceived as higher than the time spent traveling~\cite{Wardman_2004}, in particular because walking demands a greater physical effort~\cite{Kolbl_2003_energy}. Waiting time has an higher perceived cost because of the frustration due to the sheer inconvenience of waiting~\cite{Wardman_2004}. All these costs have to be integrated with those related to the time needed for accessing the network~\cite{Daganzo_2010}, the stress of the transfer experience~\cite{Guo_2011}, breaking personal habits~\cite{Garling_2003}, scheduling costs and those caused by the unreliability of arrival times~\cite{Wardman_2004}. In order to optimize the travel experience and to minimize the perceived mobility cost, it is then necessary to consider the full anatomy of trips and to distinguish between transportation modes and between the nature of time spent (riding, waiting, walking). 
