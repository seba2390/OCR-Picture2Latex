

The previous section introduced general models of mobility, which normally neglect fine details of the particular transportation or mobility modes. For sake of completness, this section provides some examples of the type of problems tackled by models and data analysis related to single mobility modes.


\subsection{Pedestrian movement}

Pedestrian mobility is of relevance to architects and designers, city managers and public service stakeholders among other professions. Large concentrations of people such as those occurring during festivals, sport events or religious ceremonies can present a challenge in terms of service and transport demand as well as public safety. Avalanches in stadiums, processions and even subway stations have occurred in recent decades \cite{helbing_2013_pedestrian} causing hundreds of casualties. Consequentially, a detailed plan for ensuring prompt evacuation in case of emergency is required when considering the configuration of new buildings, public spaces and transport systems. Thus, understanding thus the empirical patterns in pedestrian displacements and being able to model them can benefit a wide range of disciplines. Given the broad range of interest, it is not surprising that this area has been subject to increased research activity in the last decades. There exist several recent reviews devoted to this topic either fully or partially \cite{helbing_2013_pedestrian,helbing_2001_traffic,zainuddin_2010_characteristics,vicsek_2012_collective,benenson_2014_ten,cao_2015_cyber}. The intention of this section, therefore, is not to provide a detailed retrospect of the field but an overview on the most recent developments. 

Modeling frameworks can be classified according to the approach considered to handle individual pedestrians. When the aim is crowd control, one can take a large-number approximation and define fields at every point in space and time to characterize the state of the system. These fields usually include the density of pedestrians $\rho(\vec{x},t)$, the local velocity $\vec{v}(\vec{x},t)$ or the flow across an exit, $J(\vec{x},t)$ \cite{hughes_2002_continuum}. Similarly to fluid dynamics, the conservation of the number of individuals leads to the fulfillment of the simple condition:
\begin{equation}
\label{eq:ped:conv} 
\frac{\partial \rho(\vec{x},t) }{\partial t} = - \nabla (\rho(\vec{x},t) \, \vec{v}(\vec{x},t)). 
\end{equation}  
The motion of pedestrians can be assumed to minimize the time spent in the system. This condition is included via the introduction of a scalar field $\phi(\vec{x},t)$ coupled with the density in such a way that the local velocity shows a relation
\begin{equation}
\label{eq:ped:speed} 
\vec{v}(\vec{x},t) = f(\rho) \frac{\nabla \phi}{||\nabla \phi||},
\end{equation}
where the gradient of $\phi$ determines the direction of motion and $f(\rho)$ is a simple function of the density ensuring that the pedestrians speed decreases  as the density increases. Possible choices are a linear relation, $f(\rho) = \rho - \rho_{max}$ \cite{hughes_2002_continuum,helbing_2006_analytical}, keeping $\rho < \rho_{max}$ or a non linear one, $f(\rho) = (\rho - \rho_{max})^2$ \cite{carrillo_2015_local}, among others. The connection between $\phi$ and $\rho$ passes through the same function $f$ with
\begin{equation}
\label{eq:ped:phi} 
||\nabla \phi|| = \frac{1}{f(\rho)}.
\end{equation}
This is the so-called Hughes model. In a scenario with individuals exiting a room, it produces two regimes in the pedestrian mobility depending on the global density: one with free flow and another with congestion \cite{hughes_2002_continuum}. Furthermore, it turns out that shock-waves and moving congestion fronts emerge due to the narrowing of the crowd close to the door, leading to intermittent exit flow, avalanches and stop-and-go effects \cite{helbing_2006_analytical}. The model has also been derived from microscopic considerations \cite{carrillo_2015_local, corbetta_2015_asymmetric}. In \cite{carrillo_2015_local} a modification was introduced to add a limited vision or knowledge to the pedestrians. 

Field models are useful in contexts where the number of individuals is very high and the focus is set on the crowd velocity and pressure. An alternative approach is to use agent based modeling where the state of every individual (agent) is simulated in detail. This framework is especially applicable to simulations of heterogeneous populations in which the agents may show different features. 
The most popular model falling into this category is the so-called social force model proposed by Helbing and Moln\'ar \cite{helbing_1995_social, helbing_2001_self}. 
This approach is analogous to Molecular Dynamics simulations in Physics, with the exception that the forces acting on each particle (individual) cannot be derived from fundamental laws or principles and are phenomenological in nature. 
The idea behind this model is to apply an effective Newton's laws to each pedestrian $i$. If the agent's velocity is $v_i$, her location $r_i$ at time $t$, then the agent's acceleration can be expressed as
\begin{equation}   
\label{eq:ped:sfm}  
\frac{d v_i}{dt} = f_i^d + f_i^a + f_i^{rb} + \sum_{j \ne i} f_{ij}^r + \eta(r,t) .
\end{equation}
The first term on the right hand side of the equation accounts for the tendency of the individuals to walk at a certain desired speed $v_i^0$. It is usually written as 
\begin{equation}   
\label{eq:ped:des}  
f_i^d = \frac{v_i^0-v_i}{\tau} ,
\end{equation}     
which ensures the recovery of the desired speed in a characteristic reaction time $\tau$. The second term $f_i^a$ is an attraction force depending on the agent's position introduced to guarantee that walkers are compelled to a certain target or direction. The third force, $f_i^{rb}$, is repulsive depending on the agent's location and takes into account the effects of the boundaries and other obstacles on the agent's trajectory. The next term stands for a repulsive force felt by the presence of other agents $j$ different from $i$. And the last force, $\eta(r,t)$, is an uncorrelated noise term that introduces low levels of random fluctuations to avoid deadlocks.

\begin{figure}
\centering
\includegraphics[width=0.7\textwidth]{Figures_Applications/fig_helbing_2000_simulating_1}
\caption{Social force model simulation of a crowd trying to leave a room through a narrow door. In a), an sketch of the situation. In b), the sequence of leaving times of the agents as afunction of the limit speed $v_0$. In c) and d), the total vacuating time and the average flow of people as a function of $v_0$. Figure from~\cite{helbing_2000_simulating}. \label{fig:helbing_2000_simulating_1}}
\end{figure}


The social force model is able to reproduce several phenomena observed in empirical circumstances. For example, when two groups of pedestrians moving in opposite directions meet in a corridor unidirectional stripes are formed, or when they meet in a narrowing or a door and the flow displays intermittence due to clogging \cite{helbing_2001_self} (see Figure \ref{fig:helbing_2000_simulating_1}). Variations in this model or similar agent-based  frameworks have been used to analyze escape patterns from a building in low visibility conditions \cite{helbing_2000_simulating, helbing_2013_pedestrian}, to assess evacuation plans in skyscrapers \cite{sikora_2012_model,parisi_2014_sequential}, ships \cite{chen_2015_modeling} and aircraft \cite{kirchner_2003_friction}. In relation to crowd control, they have also been used to analyze how to decrease the danger of avalanches in the Hajj religious gathering in Makkah, Saudi Arabia, \cite{helbing_2007_dynamics,johansson_2008_crowd}, in the Notting Hill Carnival in London \cite{batty_2003_safety} or, along with mobility models for other transport media, to study evacuation of cities and populated areas in case of natural disasters such as flooding \cite{lammel_2010_representation,kunwar_2014_large,kunwar_2015_evacuation}. These models provide the basic tool to approach questions such as the formation of mosh and circle pitts as a mark of collective motion in heavy music concerts \cite{silverberg_2013_collective} or the risk of disease propagation in massive gatherings \cite{johansson_2012_crowd}.  

In the last decade, there has been a large volume of research related to the terms of the social force model of Equation \ref{eq:ped:sfm} \cite{zainuddin_2010_characteristics}, including empirical calibration of the different coefficients \cite{johansson_2008_crowd} as well as the introduction of a cognitive basis for the different forces. For the purpose of this review we will consider three examples of how the terms of the social force model have been adapted in ways that account for pedestrian dynamics in different situations. 


\begin{figure}
\centering
\includegraphics[width=0.7\textwidth]{Figures_Applications/fig_helbing_2007_dynamics_3}
\caption{Analsyis of the an avalanche event during the Hajj in Mekkha in 2006. In a), representative trajectories of the laminar flow, the stop-and-go and the turbulent regimes for the indivifual movements. In b), the speed in the turbulent regime. In c), the "preassure" as a function of time. In d), the different distribution of speed increaments in the two regimes. In e), the disrtibution of displacements between consecutive stops. In f), the structure function in the turbulent regime. Figure from~\cite{helbing_2007_dynamics}. \label{fig:helbing_2007_dynamics_3}}
\end{figure}


In situations of very high density, such as those occurring in the videos taken of the disaster of the Jamarat Bridge of Mekkah in the Hajj of 2006, the trajectories of single pedestrians become turbulent \cite{helbing_2007_dynamics,johansson_2008_crowd} (see Figure \ref{fig:helbing_2007_dynamics_3}). The original version of the social force model had problems reproducing this turbulent regime and some modifications to the force terms in high densities were introduced to overcome these problems. One of these modifications is the centrifugal force model, so-called because of its similarity with the classical centrifugal force in rotating bodies. This model affects the term $ f_{ij}^r $ of mutual repulsion between pedestrians $i$ and $j$ \cite{yu_2005_centrifugal,yu_2007_modeling}. Using both the distance between pedestrians $i$ and $j$, $||\vec{r}_{ij}||$ and the difference in their velocities $V_{ij}$, Yu {\it et al} proposed the form:
\begin{equation}
f_{ij}^r  =  -m_i \, K_{ij} \, \frac{V_{ij}^2}{||\vec{r}_{ij}||} \, \vec{e}_{ij}. 
\end{equation}
%which contains explicitly the distance between $i$ and $j$, $||\vec{r}_{ij}||$, in the denominator. 
Here $m_i$ is the mass of pedestrian $i$, $K_{ij}$ is a coefficient to take into account that only pedestrians to the front of $i$ will influence her movements and $\vec{e}_{ij}$ is the unit vector pointing from $i$ to $j$. This modification of the social force model shows laminar as well as turbulent regimes \cite{yu_2007_modeling}. Similar functions can also be used to alter the interactions of individuals with the boundaries of the space and obstacles. A slight modification was proposed in \cite{chraibi_2010_generalized} which includes a generalization of the shape of a padestrian's personal space to elliptical instead of circular. Here the model predictions were also compared with experimental results. 

A second modification to the social force model deals with the cognitive capacities and limitations of pedestrians and how they can affect the force terms of the model \cite{moussaid_2009_experimental,moussaid_2011_how}. 
%
Examples include analysis of the time scales regulating pedestrian interactions \cite{johansson_2009_constant}, the addition of a field to take into account the (partial) global perception of individuals and the environment under consideration \cite{dietrich_2014_gradient,colombi_2015_moving} or the role of leaders \cite{degond_2015_time}. 
%
The capacity of the pedestrians to foresee the movements of others and the dangers of obstacles have been modeled within this framework in \cite{moussaid_2009_experimental,steffen_2009_modification}. 
%
The social force model has also been adapted to agents with an anisotropic perception of their environment \cite{gulikers_2013_effect}. Its instabilities in a context of stop-and-go oscillations have been studied in \cite{chraibi_2014_oscillating,chraibi_2015_jamming}, and a systematic stability analysis of its numerical solutions has been provided in \cite{koster_2013_avoiding} together with a "mollified" version of the model that notably increases the computational performance. 

A family of models can be placed between the macroscopic view of the field model and the microscopic approach of agent-based modeling. These models are characterized by the use of a space discretization in cells and rules which specify how agents move from one cell to a neighboring one. Models within this framework receive the name of cellular automata and have the advantage of being able to process a relatively high number of pedestrians in short computational times. For example, in \cite{dutta_2014_gpu} simulations of crowds containing $100,000$ have been attained with a model whose running time scales linearly with the number of individuals. The price to pay for this increased computational performance is the loss of local details within each cell. These models have been used to study clogging in exits and due to obstacles by adding a "friction term" between agents \cite{kirchner_2003_friction,yanagisawa_2009_introduction}, pedestrian flow through subsequent bottlenecks \cite{ezaki_2012_pedestrian} and the formation of stripes in junctions with agents moving in different directions \cite{cividini_2013_diagonal}.  Some modifications to these models include, for example, behavioral factors of the agents such as path selection based on the time necessary to reach the destination \cite{kirik_2009_shortest}, heterogeneous agents \cite{sarmady_2010_simulating} and long-range spatial awareness \cite{tissera_2014_simulating}.


\begin{figure}
\centering
\includegraphics[width=0.7\textwidth]{Figures_Applications/fig_kretz_2015_social_2}
\caption{Fundamental diagram with the speed as a function of the density of pedestrians for several experiments in a closed loop circuit (ovoid-like). On the left, experiments in India with different loop lenghts, plot originally taken from \cite{chattaraj_2009_comparison}.On the right, experiments in Germany with two set of people: students and soldiers, figure originally taken from \cite{protz_2011_analyzing}. The composite Figure comes from~\cite{kretz_2015_social}. \label{fig:kretz_2015_social_2}}
\end{figure}



Some of the models described so far have been inspired by empirical data obtained from crowds in different contexts: Hajj pilgrimage \cite{helbing_2007_dynamics,johansson_2008_crowd}, meetings of the German Church \cite{schultz_2014_group} or other types of evens such as the Notting Hill carnival \cite{batty_2003_safety}. The challenge in these empirical studies is the automatic detection of individuals (see \cite{benenson_2014_ten} for a recent review on the methods used for this purpose). 
%
In contrast to the direct observation of pedestrian behavior, it is also possible to run experiments on a smaller scale in which groups of people are asked to walk with a certain direction and perform a given activity. Cameras record all the trajectories and these are later analyzed in detail using computers. This controlled environment can provide insight into questions such as how pedestrians walk in social groups \cite{moussaid_2010_walking}, the quantification of pressure and density in different types of bottlenecks and junctions \cite{zhang_2014_quantification}, the perception of personal space \cite{ducourant_2005_timing}, the clogging effects and instabilities in narrow areas \cite{moussaid_2012_traffic,bukacek_2014_experimental} and the asymmetric flow produced by stairs \cite{corbetta_2015_asymmetric}. 
%
One factor that is common to experiment based research is the analysis of ``fundamental diagrams" \cite{jelic_2012_properties} (see Figure \ref{fig:kretz_2015_social_2} for several examples). These diagrams represent the relationship between the average velocity $v$ and the density $\rho$ of pedestrians, on one hand, and their flux through a surface  $J$ (an exit) and the density, on the other. 
%
In the free flow regime, where there are no obstacles and density is low, pedestrians tend to move at an optimal velocity. As the density increases, the velocity goes down until it reaches a zero value. Similarly, the flux through an exit also decreases with increasing pedestrian concentrations. The relations between these magnitudes are non linear and in general are of great relevance for the design of buildings and urban spaces. Due to this a large effort has been invested in properly measuring these diagrams. 
%
The results show that even though qualitatively the shape of $v(\rho)$ and $J(\rho)$ may have common features across settings, quantitatively they are different depending on a number of factors. Some of these factors are cultural perception of personal space (a comparison between India and Germany was performed in \cite{chattaraj_2009_comparison}), the presence of T-junctions \cite{zhang_2013_experimental}, the individual's perception of open versus closed spaces \cite{zhang_2014_effects} and the bidirectional flow of agents \cite{zhang_2014_comparison}. Recently, the social force model has been used to better characterize the fundamental diagram. An analysis with a generalized centrifugal force has been carried out in \cite{chraibi_2010_generalized}, while an analysis with high densities in Mekkah has been carried out in \cite{dridi_2015_simulation}. 

\subsection{Air Transportation}

\begin{figure}
\centering
\includegraphics[width=0.7\textwidth]{Figures_Applications/fig_guimera_2005_worldwide_2}
\caption{In a), the relation between node degree and betweenness in the World Airport Network. In b), the location of the $25$ most connected cities. In c), the top $25$ cities in betweenness. Figure from~\cite{guimer`a_2005_worldwide}. \label{fig:guimera_2005_worldwide_2}}
\end{figure}

Many results on human mobility as, for example, the early work by Brockmann \cite{brockmann_2006_scaling}, measure individual displacements in an indirect way (in this case by spatially tracking bank notes). Along with a short-range component, these results contain long displacements of hundreds or even thousands of kilometers. Air transportation is the obvious medium responsible for most of this long-range mobility. Air transport is of paramount importance for modern global connectivity, playing a central role in the world economy and in the interchange of people, ideas and, unfortunately, also the propagation of diseases. 
%
The study of the air transport system has a long tradition in the engineering community related to aeronautics and air traffic management \cite{belobaba_2009_global,cook_2007_european}. The analysis of the system from a global perspective has received, however, a boost in the last few years coinciding with the development of network science. Early research is based on a static framework in which the airports are the nodes of the network, with a link between a pair of airports if a direct flight exists between them. This formalism loses the directional information as well as fine-grain temporal resolution. The basic statistical features of these networks were analyzed, for instance, in \cite{guimera_2004_modeling,barrat_2004_architecture,guimer`a_2005_worldwide} for the World Airport Network (WAN), in \cite{li_2004_statistical} for the Chinese network and in \cite{da_2009_structural} for Brazil. Figure \ref{fig:guimera_2005_worldwide_2} shows, for instance, the location of the top degree nodes and of those with the highest betweenness in the WAN.
%
Wide distributions are a common property present in the degree (number of destinations of the airports), the traffic (number of passengers per route or per airport) and the airport betweenness. The network can be divided into modules with a certain level of self-contained traffic \cite{guimer`a_2005_worldwide}, although this division is not seasonably stable when the focus is set on smaller geographical scales such as the US network \cite{lancichinetti_2011_finding}. It was found that the average number of passengers per route $w_{ij}$ follows a non linear relation with the degree of the airports of origin $k_i$ and destination $k_j$, $w_{ij} \sim (k_i\, k_j)^\theta$ with $\theta \approx 1/2$ \cite{barrat_2004_architecture}. Simple models, inspired by the rich-get-richer effect have been introduced to explain the topological properties of the WAN \cite{guimera_2004_modeling,barrat_2004_architecture}.  
Direct connections change in time, mostly between seasons but also across different years. The variation of the number of passengers per route was studied in \cite{gautreau_2009_microdynamics,zhang_2010_evolution}. Finally, the impact of failures in the network has also been considered under a core and periphery view \cite{verma_2014_revealing} and under a multiplex-like framework in \cite{cardillo_2013_modeling}, where the network of each airline forms a layer and the airports appearing in the different layers connect the structure. A short review on the air-traffic network analysis can be found in \cite{zanin_2013_modelling}.

\begin{figure}
\centering
\includegraphics[width=0.7\textwidth]{Figures_Applications/fig_fleurquin_2014_trees_2}
\caption{Sketch with a typical configuration of a delay tree. Every node represents an airport and the links are flights between them (airports can be repeated in different hours). The link weights are the minutes of delay and the tree is composed of different levels or generations and each node has also a certain branching number ($\rho$) for the next level.  Figure from~\cite{fleurquin_2014_trees}. \label{fig:fleurquin_2014_trees_2}}
\end{figure}

Beyond the topology, these networks are the backbones through which goods and people travel. The ingredients that conform to the network are thus airports, with their internal operations, aircraft, crew and passengers/goods that must meet a predefined schedule. On longer time-scales, the displacement of people is of great relevance, for example, for characterizing global epidemic propagation \cite{hufnagel_2004_forecast} as discussed in the following section. However, on shorter time-scales logistic failures can propagate in the network. A failure in a logistic systems emerges when the schedule is not met and an event occurs later than the planned time. The delay can then pass to subsequent planned tasks relying on the first one, and cumulatively it ends up affecting a significant fraction of the system. This idea was exposed in the context of air transportation by Beatty {\it et al.} in \cite{beatty_1998_preliminary}, where the authors discussed the concept of delay multiplier. Note that in the same way that a flight delay can affect others in an airport, the affected flights can in turn delay other connections leading to the formation of avalanches or "trees of delays" \cite{fleurquin_2014_trees} (see Figure \ref{fig:fleurquin_2014_trees_2} for an illustration). Direct costs of flight delays in Europe amount to more than one billion euros \cite{cook_2011_european}, while in the US direct and indirect costs go beyond $40$ billion dollars \cite{joint_2008_your}. Delays directly affect airlines since they increase operational costs, but as an indirect factor they also bring associated reputational costs \cite{folkes_1987_field,mayer_2003_network}. Passengers, on the other hand, undergo a direct loss of time, which can be further exasperated by lost connections leading to missed business and leisure opportunities. Finally, efforts to recover delays airborne usually imply excess fuel consumption and larger CO$_2$ emissions. 

Given these challenges, it becomes important to characterize the sources of initial (primary) delays \cite{rupp_2007_further,ahmadbeygi_2008_analysis}. These are usually hard to predict, related to bad weather conditions, technical failures in the aircraft or organization issues in the airports, and the mechanisms through which these delays are transferred and amplified in subsequent operations, giving rise to the so-called reactionary delays. The Central Office for Delay Analysis of Eurocontrol (CODA) in Europe and its counterpart in the US, the Bureau of Transportation Statistics of the Department of Transportation, release monthly reports on flight performance including statistics on the major causes of delays and the airport affected. With some more detail, there have been studies concerning the delays in single hubs such as Newark \cite{allan_2001_analysis}, in the European \cite{cook_2007_european,jetzki_2009_propagation} and US \cite{mayer_2003_network,churchill_2007_examining,fleurquin_2014_trees} networks. Factors influencing the delay propagation beyond the network are airport congestion \cite{bonnefoy_2007_scalability}, problems at aircraft rotations, or at crew and passenger connections~\cite{beatty_1998_preliminary,wang_2003_flight}. Airline schedules typically include a buffer time to absorb these delays \cite{wu_2000_aircraft}, but when this is not enough the delay will extend to further flights. 


\begin{figure}
\centering
\includegraphics[width=0.7\textwidth]{Figures_Applications/fig_fleurquin_2013_systemic_4}
\caption{Comparison between the congested cluster size as a function of time measured from empirical data and from the model of ref. \cite{fleurquin_2013_systemic}. Congested airport are defined as those with average delay per departing flight over $29$ minutes in intervals of one hour. The congested cluster is obtained from the largest connected component of the network formed by congested airports connected with direct flights during the day considered. The model has been tested with all the ingredients working or only with some of them to check their importance in delay propagation. This is the full model or one with only aircraft rotations, aircraft rotations and passenger connections or aircraft rotations and airport congestion. Figure from~\cite{fleurquin_2013_systemic}. \label{fig:fleurquin_2013_systemic_4}}
\end{figure}


On the modeling side, two main approaches have been considered. Some models have taken advantage of knowledge of the network structure to search for weak points analyzing jamming and congestion phenomena, especially trying to individualize the nodes (airports) involved in case of network-wide jamming \cite{lacasa_2009_jamming,wuellner_2010_resilience,ezaki_2014_potential,lordan_2014_study}. 
%
Still at a topological level and related to the previous papers, there has been research studying the combination of airports and airlines via multiplex networks that could cause a major impact on the network in case of malfunctioning \cite{cardillo_2013_modeling}. 
%
Another family of models with further internal details coexist along with these stylized models. The objective here is to provide realistic predictions that can be implemented in the daily operations of airports and airlines, even though the involved nature of the models may conceal the possibility of gaining insights through an analytical treatment. This has been the main approach followed in the area of transport engineering related to Air Transport Management (ATM). There are hundreds of contributions on this line mainly presented in sectorial conferences such as the USA/Europe ATM Seminar, the Air Transport Research Society (ATRS) Conference, ICRAT, the SESAR Innovation Days, etc, whose proceedings are accessible online as well as publications in journals such as the Transportation Journal, Journal of Air Transport Management (JATM), etc. 
%
The advances in this area cover topics such as the optimization of runway management, the effects of bad weather conditions in traffic, sector congestion, how the advent of drone technology will affect commercial air transportation, to mention just a few. Covering all this material would require a review by itself and it is beyond the scope of the present work. However, there are some works and projects that, due to the relation that they have to the propagation of delays in networks, deserve a mention here. 

The possibility of micro modeling how operations in the air transport network can lead to the spreading of delays was explored in some early research at the beginning of the new century \cite{schaefer_2001_flight,rosenberg_2002_stochastic}. Later Jani\'c used a similar modeling framework to assess the economical impact of a major disruption in a European hub \cite{janic_2005_modeling}. Again in Europe and within the umbrella of the WPE of the Joint SESAR Undertaking, several projects as NEWO, TREE  and POEM have studied the modeling of delay propagation from different perspectives: the first two from the network and airline managers' and the latter from the passenger's point of view. 
%
Data-driven models have been developed in this context to reproduce the delay propagation patterns in the US \cite{fleurquin_2013_systemic,pyrgiotis_2013_modelling} and in Europe \cite{campanelli_2015_tree}. These models were validated against empirical performance data \cite{fleurquin_2013_systemic,campanelli_2015_modelling}, showing an acceptable level of accuracy and precision in the prediction of delayed flights, airports displaying major departure-delay problems and large network-wide congestion events. They have also been tested with operations under generalized bad weather conditions \cite{fleurquin_2013_systemic}. 
%
The models are built with an agent-based approach applied at the level of aircraft, which are followed as different flights take place performing the aircraft rotation as established in the daily schedules. Primary delays can appear at any point in the operations due to technical, airport organizational or weather-related issues and are part of the initial conditions provided to the model. Part of the delay can be then absorbed by the buffer times included in the ground stays of the aircraft in the airports or may be propagated to further flights. 
%
The main mechanisms for delay propagation besides aircraft rotation are passenger and crew connections and airport capacity limits. The airport management is simulated with a system of queues that depend on the network considered (US or Europe). However, it was shown that, more important than the airport capacity limitations, the mechanism with the strongest potential to multiply the delay was the passenger and crew connections (see Figure \ref{fig:fleurquin_2013_systemic_4}) \cite{fleurquin_2013_systemic}. These models also allow us to study the weak points of the network in terms of delays, making visible the potential trees of delays and pointing out the flights and airports with the largest potential to generate network-wide congestion \cite{fleurquin_2014_trees}.  

\subsection{Boat Networks}

\begin{figure}
\centering
\includegraphics[width=0.7\textwidth]{Figures_Applications/fig_kaluza_2010_complex_1}
\caption{In a), the network formed by ports as nodes connected when cargo boats navigate from one to the other. The links color represents the number of boats per route and their shape is obtained from the geodesics between the ports. In b), a map with the top $50$ ports in betweenness centrality and the list of the top $20$. Figure from~\cite{kaluza_2010_complex_2}. \label{fig:kaluza_2010_complex_1}}
\end{figure}


Sea transportation networks have received less attention than their air traffic counterparts. These networks are mostly devoted to cargo movements and, therefore, their relation to human mobility is less pronounced. The initial analysis defined the network formed by ports as nodes, with links connecting them whenever major cargo liners travel from one to other \cite{fremont_2007_global}. The weight of the links register the number of ships (trips) along the route. The resulting network embraces most of the planet, joining the different continents (see Figure \ref{fig:kaluza_2010_complex_1}). Its degree, weight and strength distributions show high heterogeneity with some ports absorbing most of the global traffic, while others play a more peripheral role \cite{hu_2009_empirical,kaluza_2010_complex}. The centrality of the ports was studied in \cite{hu_2009_empirical}, where it was found that Singapore, Antwerp, Bushan and Rotterdam are the global top rankers in terms of degree and betweenness (although Bushan and Rotterdam switch positions). A non linear relation between strength and degree was noted in \cite{kaluza_2010_complex}, with an exponent of $1.46 \pm 0.1$, similar to the one observed in world airport network \cite{barrat_2004_architecture}. The network can be divided by the type of ship (containers, oil tankers, etc) \cite{kaluza_2010_complex}, which is also adapted to the use of a multi-layer/multiplex network framework \cite{ducruet_2013_network}. Other more involved network analysis tools such as communities, motifs \cite{kaluza_2010_complex} and rich-club \cite{ducruet_2013_network} have also been used. The flows can be reproduced with a gravity model, although with some strong limitations  \cite{kaluza_2010_complex}. A systematic analysis on the similarities between the world airport network and the boat network was performed in \cite{woolley-meza_2011_complexity}, including a study of the resilience of both networks. An important additional application of this network, besides the transport of goods, is the study of the invasion of species brought by carriers from different parts of the world (see, for instance, \cite{kaluza_2010_complex,keller_2010_linking,seebens_2013_risk}).





























