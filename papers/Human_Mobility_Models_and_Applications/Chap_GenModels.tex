% As discussed, one must necessarily take into account scale when developing descriptive models for human mobility. Models can be aimed at reproducing individual mobility patterns, general population flows or flows discriminated by transport media and each of these aspects have been tackled with distinct modeling frameworks. 
Models of human mobility can be aimed at reproducing individual mobility patterns or general population flows. 
In both cases one must necessarily take into account the characteristic spatial and temporal scales of the mobility process, which can vary from hundreds of meters to thousands of kilometers and from hours to years.
For this reason, each of these cases have been tackled with distinct modeling frameworks. 
Individual mobility is subject to a certain level of uncertainty associated with free will and arbitrariness in the actions of individuals, leading to a degree of stochasticity in trip patterns. Consequently, minimal models borrow concepts and methods from random walks and Brownian motion. 
However, several studies highlighted that individual trajectories are far from random, possessing a high degree of regularity and predictability, which can be exploited to predict an individual's future whereabouts and to construct realistic generative models of individual mobility. 
% The level of noise decreases, of course, as the scale gets larger due to averaging effects, and at the level of population flows, modeling approaches tend to have less stochastic components, being restricted to a few relevant variables such as population levels, GDP per capita, etc. 
At the level of population flows, models describe the aggregate mobility of many individuals and aim to reproduce Origin-Destination (OD) matrices by estimating the average number of travelers between any two spatial regions (e.g. municipalities) per unit time (i.e. daily in the case of commuting flows, yearly in the case of migration flows).
Most modeling approaches derive the mobility flows as a function of a few relevant variables of the regions considered, such as mutual distances, population levels, GDP per capita, etc. 
The next sections review the state-of-the-art mobility models starting from individual-like approaches, moving on to flows at the population level and concluding by the interpolation between these scales  (inter-modality).

\subsection{Individual-Level (Random walks)}
\label{sec:rw}

A random walk is mathematically defined as a path formed by successive discrete random steps although they can be described in the continuum limit as well  (Sec.~\ref{sec:ctrw}). The simplest version, however, deals with spatial displacements $\Delta X_{i}$ that are taken at discrete times $t_i$. If $x_0 = 0$ corresponds to the initial position of the walker at time $t_i=0$ (we can arbitrarily fix the origin in the initial position of the walker), then the position after $N$ steps is given by the random variable
\begin{equation}
X(t_N) = \sum_{i=1}^{N} \Delta X_{i},
\end{equation}  
where each displacement $\Delta X_{i}$ is a random variable extracted from a probability distribution $f(\Delta x )$,   
and draws are assumed to be statistically independent. The probability distribution $f(\Delta x)$ can be used to determine the probability density function (PDF), $P(x,t)$ for the process to be at position $x$ at time $t$ which completely characterizes the nature of the walk and the related spatial and temporal measures. For example, the mean square displacement (\sectionname ~\ref{sec:msd}) corresponds to the second moment $\mathrm{MSD}(t) = \langle X(t)^2 \rangle$, 
and the $n$-th moment is obtained from
\begin{equation}
\langle X(t)^n \rangle = \int_{-\infty}^{\infty} x^n P(x,t) d x, \label{individual:moment}
\end{equation}
where brackets indicate ensemble averages over multiple realizations of walks.

Of particular interest when analyzing models of individual mobility is the scaling of the square root of the mean squared displacement (RMSD), $R(t) = \sqrt{\mathrm{MSD}(t)}$, with time $t$; a relationship that characterizes the speed of displacement from the origin with time. The scaling of the MSD can be used to categorize the type of diffusive motion of the random walker. Ordinary Brownian motion (Sec.~\ref{sec:brownian}) has a MSD that scales linearly with time (for any spatial dimension) therefore on average we expect that after a time $t$ the distance of the walker from the origin is proportional to the square root of the elapsed time; 
$R(t) \sim t^{1/2}$. Taking this definition of diffusion, random walks that have displacement growing at a slower rate than $t^{1/2}$ are said to be {\it sub-diffusive}. In contrast, if displacement grows at a rate faster than $t^{1/2}$, the random walk is classed as {\it super-diffusive}. The models of random walks that can lead to these different behaviors are outlined below. 

As an example, the most basic form of a random walk is the discrete symmetric random walk in 1D, for which $\Delta X_{i} = \pm 1$ with equal probability. For this simple case after an elapsed time time $t = N$ (where $N$ is the number of steps taken by the walker), the first two moments are $\langle X_N \rangle$ = 0 and $\langle X_{N}^{2} \rangle = N$. 

\subsubsection{Brownian Motion}
\label{sec:brownian}

Brownian motion is a class of random walk originally developed to describe the motion of a particle suspended in a fluid (liquid or gas) \cite{einstein_1905_movement}. Such a particle undergoes many rapid instantaneous collisions with much smaller particles in the medium, resulting in a trajectory characterized by a series of irregular and random displacements. Mathematically, a 1-dimensional Brownian motion is a random walk in the space of real numbers $\mathbb{R}$ with independent and normally distributed increments where the probability to observe a displacement of magnitude $X$ from the origin location after a time $t$ is Gaussian distributed with mean zero and variance proportional to $t$. Brownian motion can be defined as a limit of the discrete symmetric random walk. Let us assume that the particle can take steps of length $1/\sqrt{k}$ to its left or right with equal probability, and that after time $t$ the particle has taken $N=t\,k$ steps. For a given $k$, the displacement of the particle at time $t$ is given by
\begin{equation}
X_{k}(t) = \frac{1}{\sqrt{k}} \sum_{i=1}^{tk} \Delta X_{i},
\end{equation}
and taking the limit $k \to \infty$ leads to Brownian motion as a consequence of the Central Limit Theorem (CLT). 
In fact, in the large $k$ limit $X_{k}(t)$ tends to $X(t)$, whose PDF is 
\begin{equation}
P(x,t) = \frac{1}{\sqrt{2 \, \pi\, t \sigma^2}} e ^{\frac{-(x-\mu k t)^2}{2\, t \sigma^2}},\label{eq:3}
\end{equation}
where $\mu = \langle \Delta X \rangle$ and $\sigma^2 = \langle \Delta X^2 \rangle$ are the mean and variance of the random walk displacements.
For this case, the first moment is zero and $\sigma^2=1$, so the MSD is simply $t$ and the scaling of the RMSD is $R(t) \sim t^{1/2}$, corresponding to ordinary diffusion. 
Using a similar limit, it can be shown that a $d$-dimensional random walk converges to a $d$-dimensional Brownian motion, for which the probability that the walker is found at distance $x$ from the initial position is 
$P(x,t) = (4 \pi \, t )^{-d/2} e ^{\frac{-x^2}{4\, t}}$,
and the RMSD always scales as the square root of $t$, $R(t) \sim t^{1/2}$.


\subsubsection{L\'evy Flight}
\label{sec:levy}

Unlike Brownian motion, there is a class of random walks called L\'evy Flights, for which one cannot use the CLT. A L\'evy Flight is composed of a series of small displacements, interspersed occasionally by a very large displacement. It is formally defined as the sum of independent identically distributed random variables whose PDF for a single jump has a divergent second moment due to a long-tailed distribution of the form
\begin{equation}
f(\Delta x) \sim \frac{1}{\Delta x^{1+\beta}}, \label{eq:7}
\end{equation} 
with $0 < \beta < 2$. 
If the displacement after $k$ steps of size $1 / k^{1/\beta}$ is defined as $X_k(t)$, then the random variable $Z_N$ defined as the re-scaled sum of $N = t\, k$ independent random variables distributed as \equationname~(\ref{eq:7}), takes the form
\begin{equation} \label{eq:zy}
Z_N = \frac{1}{(t \, k)^{1/\beta}}\sum_{i=1}^{t\, k}\Delta X_i = \frac{1}{t^{1/\beta}}X_k(t).
\end{equation}
The rescaled variable satisfies a generalization of the CLT, namely the L\'evy-Khintchin theorem which states that the PDF of $Z_N$ in the limit $N \to \infty$ is a so called $\alpha$-stable (L\'evy) distribution. These distributions do not have a closed formed in real space; instead the characteristic function can be written in Fourier space, and for this particular example, the tail would show the same power law behavior as \equationname~(\ref{eq:7}). Changing variables from $Z_N$ to $X_k$ via \equationname~(\ref{eq:zy}) we obtain
\begin{equation}
Z_N \to f(z); \qquad
X_k(t) \to \frac{1}{t^{1/\beta}}f\left(\frac{x}{t^{1/\beta}}\right). 
\end{equation}
The MSD is equal to the second moment
\begin{equation}
\langle X(t)^2 \rangle \sim \int_{0}^{\infty} x^2 \frac{1}{t^{1/\beta}}f\left(\frac{x}{t^{1/\beta}}\right) d x = t^{2/\beta} \int_{0}^{\infty} y^2 f\left(y\right) d y
\label{levy:MSD}
\end{equation}
and hence RMSD for a L\'evy flight scales super-diffusively: $R(t) \sim t^{1/\beta}$.

The jumps in L\'evy flights can be considerable, but occurs within the same time step as a short one (which is unrealistic), and thus L\'evy flights can be seen as only a rough approximation to actual human trajectories.

%%RM; FROM HERE
\subsubsection{CTRW}
\label{sec:ctrw}

The random walk models discussed thus far have been discrete in time. In each time interval, an instance of a jump is dictated by the corresponding jump-length distribution. A continuous time random walk (CTRW) is a random walk in which the number of jumps made in a time interval $d t$ is also a random variable or equivalently, the time elapsed between jumps (wait-time $\Delta T$) is also a random variable. 

If the PDF of jump-lengths is $f(\Delta x)$ and that for wait-times is $\phi(\Delta t)$, and these are independent, the CTRW consists of pairwise random and independent events with $\Delta X$ and $\Delta T$ drawn from the joint PDF, $P(\Delta x, \Delta t)  = f(\Delta x)\phi(\Delta t)$, which denotes the probability that a jump of length $\Delta x$ is taken after a time $\Delta t$. According to this model, after $N$ steps, the total displacement, $X_N$, and the total elapsed time $T_N$ are given by:
\begin{equation}
X_N = \sum_{i=1}^{N} \Delta X_i; \qquad
T_N = \sum_{i=1}^{N} \Delta T_i. 
\end{equation}\\
The PDF of the process, $P(x,t)$ can be Fourier-Laplace transformed to give:
\begin{equation}
W(k,u) = \frac{1-\tilde{\phi}(u)}{u\, (1-\tilde{\phi}(u)\, \tilde{f}(k))} \label{eq:ctrw:2}
\end{equation}
where $\tilde{\phi}(u)$ and $\tilde{f}(k)$ are the Laplace and Fourier transforms of $\phi(\Delta t)$ and $f(\Delta x)$ respectively. Taking the inverse transform we get
\begin{equation}
P(x,t) = \frac{1}{2\, \pi}\frac{1}{2\, \pi \, i} \int du \int dk \, e^{u\, t-i\, k\, x}\, W(k,u). \label{eq:ctrw:3}
\end{equation}
This expression for $P(x,t)$ can be analyzed according to the asymptotic behavior of distributions $\phi(\Delta t)$ and $f(\Delta x)$. In particular four types of models are obtained, depending on whether none, either, or both distributions have heavy tails. The properties of these models and their application to human mobility are discussed below.

\begin{figure}[t!]
\centering
\includegraphics[width=0.8\textwidth]{Figures_GenModels/fig_brockmann_2006_scaling_1_2S2}
\caption{ Schematic of the different (asymptotic) classes of CTRW defined in the text, as a
function of the waiting-time and jump-length exponents $ 0 < \alpha < 1$ and $0 < \beta < 2$. L\'evy flights, fractional Brownian motion
as well as ordinary diffusion are limiting cases of the more general class of ambivalent processes. Figure from~\cite{brockmann_2006_scaling}.}
\label{fig:brockmann_2006_scaling_1_2S2}
\end{figure}



%\subsubsection{Ordinary Diffusion}
\noindent {\bf Ordinary Diffusion:} If the expectation value of $\phi(\Delta t)$ and the variance of $f(\Delta x)$ are both finite, i.e $1 - \tilde{\phi}(u) \sim \tau \, u$ and $\tilde{f}(k) \sim 1 - (\sigma \, k)^2$ then from \equationname~(\ref{eq:ctrw:2}) and \equationname~(\ref{eq:ctrw:3}), asymptotically:
\begin{equation}
P(x,t) \sim \frac{1}{\sqrt{t}}e^{-x^2/D\, t}
\end{equation}
where $D$ is a diffusion constant. Therefore a CTRW with well defined jump-length and waiting-time distributions, is asymptotically (with time) equivalent to Brownian Motion.\\ 

%\subsubsection{L\'evy Flights}
\noindent {\bf L\'evy Flights:} If $f(\Delta x) \sim \Delta x^{-(1+\beta)}$ ($0<\beta<2$), and $\phi(\Delta t)$ has a finite variance, a L\'evy flight is recovered. Following the same procedure as Sect.~\ref{sec:levy}, the PDF of the process is given by
\begin{equation}
P(x,t) \propto \frac{1}{t^{1/\beta}}G(x/t^{1/\beta}) ,
\end{equation}
where $G(x/t^{1/\beta})$ is a scaling (limiting) function. Therefore if the distribution of jump lengths is a power-law and the wait-times are well defined, $R(t) \sim t^{1/\beta}$ and a continuous time random walker will follow a super-diffusive path, equivalent to a L\'evy Flight.\\

%\subsubsection{Fractional Brownian Motion}
\noindent {\bf Fractional Brownian Motion:} Conversely, if jump-lengths with a finite variance are combined with a power-law distribution of wait-times
$\phi(\Delta t) \sim \Delta t^{-(1+ \alpha)}$, ($0 < \alpha < 2$), the PDF is
\begin{equation}
P(x,t) \sim \frac{1}{t^{\alpha/2}}H(x/t^{\alpha/2}) ,
\end{equation}
with $H$ being a non-Gaussian limiting function. In this case, the effect of the distribution of waiting times is to slow down the random walk. Here $R(t) \sim t^{\alpha/2}$, consequently the walk is sub-diffusive for $\alpha <1 $ and super-diffusive for $1 < \alpha <2$.\\

%\subsubsection{Ambivalent Processes}
\noindent {\bf Ambivalent Processes:} The fourth variant occurs when both $f(\Delta x)$ and $\phi(\Delta t)$ are heavy-tailed. In this case, $R(t) \sim t^{\alpha/\beta}$, and hence the nature of the diffusive behavior is fully specified by $\alpha$ and $\beta$. For $\beta < 2\, \alpha$, the CTRW is super-diffusive and for $\beta > 2\, \alpha$, it is sub-diffusive. If $\beta = 2\, \alpha$, the random walk converges to ordinary diffusion/Brownian motion, despite the diverging moments of the respective distributions. A schematic of these limiting cases is provided in \figurename~\ref{fig:brockmann_2006_scaling_1_2S2}. Given the sensitivity of the models to the parameters of the wait times and jump lengths, the importance of their accurate measurement from data cannot be over-stated. In any event, the ambivalent process model is most often used for describing the mobility of individuals. Analysis of various data sources (GPS, CDRs, Dollar bills) has found that both jump length distributions and the distribution of wait times display power-law behavior~\cite{brockmann_2006_scaling, gonzalez_2008_understanding, zhao_2008_empirical, song_2010_modelling}. The parameter ranges measured from data correspond to $\alpha$ estimated from empirical data range from $0.42 \leq \alpha \leq 0.8$~\cite{zhao_2008_empirical,song_2010_modelling} and values of $0.31 \leq \beta \leq 0.75$ ~\cite{zhao_2008_empirical,gonzalez_2008_understanding}.


\subsubsection{Preferential Return}
\label{sec:epr}

An important aspect of human behavior, missing from the models discussed thus far, is the tendency of individuals to return to one or more locations on a daily basis (so-called preferential return). In a CTRW process, the number of distinct sites $S$ visited by a random walker in time $t$ is given by
\begin{equation}
S(t) \sim t^{\mu} ,
\label{individual:17}
\end{equation}
with $\mu = \alpha$ for a CTRW and $\mu =1$ for a L\'evy Flight~\cite{gillis_1970_expected}. The analysis of CDR's (\sectionname~\ref{sec:cdr}), revealed a power-law distribution of jump-size with exponent $\alpha = 0.8$, while $\mu$ was independently measured to be $\mu = 0.6 \pm 0.02$, considerably less than the theoretical prediction. Indeed for random walks, the probability of an individual to visit any distinct location becomes asymptotically uniform, whereas  analysis of visitation patterns~\cite{gonzalez_2008_understanding} from data suggests that the rank-frequency visitation frequency follows a Zipf's law: 
\begin{equation}
f_{k} \sim k^{-\zeta}, 
\label{individual:18}
\end{equation}
where $k$ corresponds to the rank of location according to frequency of visit. This implies the distribution of visitation frequencies follows $P(f) \sim f^{-(1+1/\zeta)}$.

Furthermore the scaling of the MSD in the CTRW model  suggests that an individual will asymptotically drift away from the origin (home) in contrast for what is known from daily experience. To account for the human trait of returning to locations,  Song \et~\cite{song_2010_modelling} include two extensions to the CTRW model: \emph{exploration} and \emph{preferential return}. 

\begin{figure}[t!]
\centering
\includegraphics[width=0.8\textwidth]{Figures_GenModels/fig_song2010modelling_2}
\caption{Schematic description of the Exploration and Preferential Return model. Starting at time $t$ from the configuration shown in the left panel, indicating that the user visited previously $S = 4$ locations with frequency $f_i$ that is proportional to the size of circles drawn at each location, at time $t + \Delta t$ (with $\Delta t$ drawn from the $P(\Delta t)$ fat-tailed distribution) the user can either visit a new location at distance $\Delta r$ from their present location, where $\Delta r$ is chosen from the $P(\Delta r)$ fat-tailed distribution (exploration; upper panel), or return to a previously visited location with probability $P_{ret} = 1- \rho S^{-\gamma}$, where the next location will be chosen with probability $\Pi_i = f_i$ (preferential return; lower panel). Figure from~\cite{song_2010_modelling}.}
\label{fig:song_2010_2}
\end{figure}

Exploration is defined as the probability for an individual to move to a \emph{previously unvisited} location according to
\begin{equation}
\label{eq:pnew}
P_{new} = \rho \, S^{-\gamma}, 
\end{equation}
where $\rho$ and $\gamma$ are parameters of the model and $S$ increases by one for each new visit.
Preferential return on the other hand is the probability to return to a previously visited location and is thus complementary to exploration: $P_{ret} = 1-\rho \, S^{-\gamma}$ (see \figurename~\ref{fig:song_2010_2} for a schematic). 

The probability to visit a previous location $i$, denoted $\Pi_i$, is determined by the number of previous visits to $i$, i.e., $\Pi_i = f_i$ where $f_i$ is the visitation frequency. The parameters of the model are restricted to $0<\rho<1$ and $\gamma \ge 0$. Considering the total number of steps, $N$, an individual makes in time $t$, and noting that $dS/dN =P_{new}$, the number of distinct locations visited in time $t$ is given by $S \sim N^{1/(1+\gamma)}$. For a power-law waiting time distribution $\phi(\Delta t)$, we have $t\sim n^{1/\alpha}$ and hence the exponent in \equationname~(\ref{individual:17})is
\begin{equation}
\mu = \frac{\alpha}{1+ \gamma},
\end{equation}
thus leading to slower rate of exploration than predicted by CTRW processes. 

Furthermore, the number of visits to location $i$ at step $N$, $m_i(N)$, increases according to 
\begin{equation}
\frac{d m_i}{d N} = \Pi_i \, (1-P_{new}) \, ,
\end{equation}
where $\Pi_i = f_i = m_i / \sum_i m_i(N)$. As $S(t) \to \infty$, for $\gamma > 0$, $P_{new} \to 0$ and therefore $d m_i/d N = m_i/\sum_i m_i(N)$. Noting that the sum of visits over all locations is equivalent to the number of steps taken, $\sum_i m_i(N) = N$, the expression $m_i(N) = N/N_i$ is obtained, where $N_i$ denotes the first jump to location $i$. It is clear that the likelihood of visiting a location increases with number of earlier visits, and as such, the rank $k_i$ of location $i$ follows the relation $k_i = S(n_i) \sim n_i^{1/(1+ \gamma)}$. Combined with the fact that $f_i$ is proportional to $m_i(N)$ this results in the relation $f_k \sim k^{-\zeta}$ with $\zeta = 1 + \gamma$, thereby accounting for the relation in \equationname~(\ref{individual:18}).

 The MSD is related to $S$ via
\begin{equation}
\mathrm{MSD} ^{\beta/2} \sim \log \Big(\frac{1-S^{1-\zeta}}{\zeta -1}\Big) + c ,
\end{equation}
where $c$ is a constant. Given the measured range of the parameters, this leads to three asymptotic regimes: $\mathrm{MSD} \sim (\log t)^{2/\beta}$ for $\zeta<1$; $\mathrm{MSD} \sim (\log\log t)^{2/\beta}$ for $\zeta=1$;  and $\mathrm{MSD}  \to X_{max}$ for $\zeta > 1$ where $X_{max}$ denotes the saturation point of the MSD.  Measured values suggest that $\zeta = 1.2 \pm 0.1$ indicating a saturation of movement at long times~\cite{song_2010_modelling}. This regime corresponds to an individual's motion being dominated by their most visited location and is more in line with expected human behavior.

% An important thing to note about preferential return, in contrast to the random walks discussed, is that the model is dynamically quenched. After exploring a new location an individual has an increased tendency to return to it in the future. Therefore, the mobility pattern generated by this model is stable in comparison to that generated by random walks. The model itself is designed to capture the long term scaling patterns in human mobility, which are of great importance to many fields including urban planning~\cite{makse_1995_modelling, hufnagel_2004_forecast, rozenfeld_2008_laws}, economic forecasting~\cite{gabaix_2003_theory} and human driven processes such as epidemic spreading~\cite{eubank_2004_modelling, vespignani_2009_predicting, balcan_2009_seasonal}. However, at an individual level it is also important to obtain a clear picture of the short-term dynamics present in daily mobility patterns and therefore the model cannot, in its present form, provide a complete description of individual human mobility. 

\subsubsection{Recency}
\label{sec:recency}
The concept of recency was introduced to solve discrepancies that emerges under the standard preferential return assumptions. Specifically, the fact that the earlier a location is discovered, the more visits it will receive, leading to a cumulative advantage precluding people from changing preference of location (unlike what is observed in data). Barbosa \et \cite{barbosa_2015_effect} proposed a model in which the exploration phase in human movements also considers recently-visited locations and not solely frequently-visited locations. Two rank variables $K_{f}$ and $K_{s}$ are defined to characterize, respectively, the \emph{frequency} and \emph{recency} of a given location in the context of an individual's trajectories. More precisely, the rank variables can be described as:
\begin{itemize}
	\item[$K_s$] is the recency-based rank. A location with $K_s = 1$ at time $t$ means that it was the previous visited location. $K_s = 2$ means that such location was the second-most-recent location visited up to time t and so on.
	\item[$K_f$] is the frequency-based rank. A location with $K_f =1$ at time $t$ means that it was the most visited location up to that point in time. Similarly, a location with $K_f = 2$ is the second-most-visited location up to time $t$, and so on.
\end{itemize} 
% From the analysis of two mobility datasets (phone traces and check-ins from a location-based social network), both ranks were calculated in an expanding basis from accumulated sub-trajectories. To illustrate, consider a particular individual with a trajectory $T=[(l_{1},l_{2},\ldots,l_{n}),l_{i}\in[1,\ldots,N]]$ composed of $n$ steps to $S \le N$ distinct locations. For each step $j>0$, the partial trajectory $\mathcal{T} = [l_{1},l_{2},\ldots, l_{j-1}]$ composed of all the previous steps can be obtained, with $l_{j-1}$ being the immediate preceding step. From the sub-trajectory $\mathcal{T}$, the frequency-based ranks $K_{f}$ of all locations visited so far are measured. 

The model can be described as follows: first, a population of $N$ agents is initialized and scattered randomly over a discrete lattice with $L\times L$ cells, each one representing a possible location. The initial position of each agent is accounted as its first visit. At each time step agents can visit a new location with probability similar to the preferential return model, Eq.~\eqref{eq:pnew}. 

%For the datasets considered, the relevant parameters were measured to be $\gamma_{\mathtt{phone}} = 0.73\pm0.03$ and $\rho_{\mathtt{phone}} = 0.83\pm0.03$. For the check-ins data, the estimated parameters were $\gamma_{\mathtt{checkins}} = 0.50\pm0.08$ and $\rho_{\mathtt{checkins}} =  0.75\pm0.03$. 
  
The return phase happens with probability $P_{ret}$ analogous to \equationname~\eqref{eq:pnew}, with the caveat that return jumps selects location $i$ from  frequently visited ones with probability $\alpha$ and recently visited locations with probability $1 - \alpha$ thus:
\begin{equation} \label{eq:recency}
\begin{array}{ll}
P^s_{ret} = (1 - \alpha) \, P_{ret} & \Pi_i \propto k_s(i)^{-\nu} ,\\
P^f_{ret} = \alpha \, P_{ret} & \Pi_i \propto k_f(i)^{-1-\gamma} ,
\end{array}
\end{equation}
where $k_s(i)$ is the recency-based rank and $k_f(i)$ is the frequency-based rank of the location $i$. When $\alpha = 1$ the preferential return model is recovered. The empirical measurements for the two datasets considered had $\alpha = 0.1$, and $\nu = 1.6$. $\gamma = 0.6$ was kept the same as~\cite{song_2010_modelling}.

\begin{figure}[t!]
\centering
\includegraphics[width=0.9\textwidth]{Figures_GenModels/fig_barbosa_2015_recency_6}
\caption{Comparison between the Preferential Return (EPR) model and the recency-based (RM) model. (a) The analysis of the return ranks generated by the EPR model shows that it reproduces a pattern similar to the one observed from the empirical analysis.
(b) Probability of return to recently-visited locations (i.e.,low $K_s$). (c) Distribution of the frequency ranks, the preferential return mechanisms (labelled EPR) exhibits a power-law distribution. The activation of the recency mechanism does not affect the frequency rank distribution. (d) $K_s$ distribution, the EPR mechanism does not capture the power-law behavior observed on the empirical data. Figure from~\cite{barbosa_2015_effect}.}
\label{fig:barbosa_2015_6}
\end{figure}

As seen from \figurename~\ref{fig:barbosa_2015_6} the preferential return model does not capture the broader distribution of $p(K_{f},K_{s})$ for recently-visited locations, an effect captured by the recency-based refinements. The primary differences can be seen for the distribution of the recency rank $K_s$ seen in log-linear scale.

\subsubsection{Social-based models}
\label{sec:social}

It is natural to assume that two individuals who have a social interaction, such as friends, family or colleagues, do not always move independently \cite{axhausen_2005_social,carrasco_2006_exploring,dugunji_2005_discrete}. Occasionally, they will share full trips, destinations or origins. The trips can be also synchronized if the objective is to meet somewhere or go back home after a meeting. Furthermore and closing the loop, the social network of an individual typically reflects the geography of their life with tighter connections with people spatially closer or at least in clusters related to the places in which the person has previously resided. These correlations have been observed in several publications. For example, a relation between distance and ``online friendships" was described in 2005 by Liben-Nowell \et \cite{liben-nowell_2005_geographic} and was later confirmed using surveys \cite{carrasco_2008_how,vandenberg_2013_path}, social networks \cite{carrasco_2008_collecting,carrasco_2008_agency} and  mobile phone records \cite{lambiotte_2008_geographical,krings_2009_urban,phithakkitnukoon_2012_socio}. Indeed this feature formed the basis of a generative model describing the behavior of population level aggregate economic indicators across urban systems~\cite{Pan_2013_UrbanScaling}. Possible methods for exploiting this intuitive observation towards improving forecasting of individual movement has been explored in \cite{de_2013_interdependence}.
For example, nonlinear time series prediction methods based on the delay embedding theorem by Takens~\cite{Takens_1981_Delay} can be used to forecast an individual's future mobility given a detailed history of past movements and the assumption of a certain degree of determinism in mobility patterns.
Furthermore, improvements in prediction accuracy by about one or two orders of magnitude has been observed when including the time series of past movements of members of the social network (individual's acquaintances)  

Conversely, online~\cite{eagle_2009_inferring,crandall_2010_inferring,picornell_2015_exploring} social links can be inferred from co-occurrences of individuals in space and time. In particular, the probability that two users have a friendship link on the Flickr social network grows significantly with the number of distinct geographical locations that they both visited within a given time threshold, where the spatial and temporal information are extracted from the geo-tagged photos they published. 
The probability of being friends is higher if the size of the regions and the temporal range between the two observations decrease. 
This result can be explained by simple mobility models exploiting the fact that friends are more likely to spend time together in the same place, often live close to each other and that the jump-size distribution decays as a power law. Similarly, the location of a person may be predicted by those of their near contacts~\cite{backstrom_2010_find}. Two individuals with similar mobility patterns are generally in close proximity in the who-calls-whom social network of mobile phone users \cite{wang_2011_human}. 
Indeed, strong correlations exist between various classical topological measures of the proximity of two mobile phone users in the social network, such as Adamic-Adar, Jaccard and Katz, and various measures of spatio-temporal proximity, such as co-location rates and spatial cosine similarity. 
Combining co-location information with information on the relative positions of users in the social network it is possible to predict the formation of new social ties with higher precision than using only information derived from the proximity in the social network. 

The idea of interconnecting mobility and social interactions has been considered in different theoretical contexts. Detailed models have been proposed in the area of transportation to take into account the relation between social network and transport demand. This includes microsimulation of transport systems \cite{axhausen_2005_social,carrasco_2006_exploring,dugunji_2005_discrete} in which agents may have communication between them and have common objectives \cite{paez_2007_social}. Social groups and their sizes can impact transport demand  \cite{molin_2007_social}, daily schedules, social relations and face-to-face interactions \cite{arentze_2008_social,carrasco_2009_social,hackney_2011_coupled,ronald_2012_modeling,sharmeen_2014_dynamics}. A simpler framework has been introduced from a Physics perspective, as for example formulated by~\cite{gonzalez_2006_system}. The system is composed of $N$ agents within a radius $r$ moving in a 2D square of lateral size $L$ and periodic boundary conditions. The starting position of the agents and their directions of movement are randomly selected. All the agents have the same initial speed $v_0$ and they travel in straight lines until they ``crash". Every crash produces a new social tie between the pair of agents involved. After every crash, the velocities of the two agents are updated randomly selecting new directions of movement and increasing their speeds according to the expression:
\begin{equation}
|\vec{v}_i (t)| = v_o + c\, k_i(t), 
\end{equation}
where $\vec{v}_i (t)$ is the velocity of agent $i$ at time $t$ immediately after the crash, $c$ is a constant and $k_i(t)$ is the number of connections of $i$ in the social network. The number of agents $N$ is constant, but after a certain period of time old agents are removed and substituted with new agents with speed $v_o$ and no social links, allowing the system to reach a stationary state. The characteristics of the emerging social network are analyzed. The degree distribution depends on the constant $c$ and on the lifetime of the agents in the system. For small $c$, the distribution is well approximated by a Poissonian distribution, but deviates from this shape with increasing $c$ and average degree $\langle k \rangle$. Beyond the distribution, the model can reproduce other properties of real social networks such as the degree-degree correlations, the size of the large connected cluster and the abundance of cliques.

\begin{figure}[t!]
\centering
\includegraphics[width=0.6\textwidth]{Figures_GenModels/fig_grabowicz_2014_entangling_1}
\caption{Sketch showing the main ingredients of Grabowicz's model. The agent's update of position and network is marked in red, while its contacts are in blue. The model has two main steps: one in which the mobility is determined in terms of either visiting a friend with probability $p_v$ or a L\'evy-like jump otherwise. After this, a new social link may be established with  probability $p$ in the neighborhood of the new position or at random with $p_c$. Figure from~\cite{grabowicz_2014_entangling}.}
\label{fig:grabowicz_2014_entangling}
\end{figure}


More realistic mobility models in the context of generated social networks were explored in~\cite{grabowicz_2014_entangling,toole_2015_coupling}. In particular the former proposed a model inspired by the methods described in Sec.~\ref{sec:epr} with important variations in the main ingredients (See \figurename~\ref{fig:grabowicz_2014_entangling}). The initial conditions were set such that the agents are placed according to measured population density in the area considered. Then at every step, each agent decides to visit a social contact, with probability $p_v$ or, otherwise moves according to a L\'evy-like flight with probability $1-p_v$, with the destination chosen in proportion to the population density in that area. This precludes agents from moving to unphysical locations such as water-bodies, geographic barriers and the like. After movement, the agent generates a new directed social link with probability $p$ in the neighborhood of her present position and with probability $p_c$ with a random agent. The parameter $p_c$ represents social relations that emerge online, inspired by the observations made by~\cite{liben-nowell_2005_geographic}. Note that the  jump distribution can be calibrated from measurements made in~\cite{song_2010_modelling}, and the tie formation parameter $p$ can be sampled at different scales from measurements made in~\cite{crandall_2010_inferring}. 
This leaves only two free parameters in the model: the probability of visiting friends $p_v$ and that of creating new links regardless of the distance $p_c$. These parameters were adjusted to reproduce an error function containing topological and  geographical properties of a set of online social networks such as Twitter, Gowalla and Brightkite. With these parameters, the model reproduces the degree distributions found in~\cite{gonzalez_2006_system} as well as the spatial dependence of link distribution, the probability of reciprocal ties, the overlap of the social environment, the density of triangles and the disparity of the triangles in terms of distances of their vertices. The model is also amenable to a  mean-field analytical treatment that demonstrates the necessity of accounting for social effects on mobility (visiting friends for example) to obtain realistic reproductions of the behavior seen in  social networks. 

The variant proposed by~\cite{toole_2015_coupling} instead focuses on behavior seen at shorter time-scales such as intra-day mobility, for which period the social network can be approximated as static. Like the preferential return model, at each step an agent decides to return to a previously visited location with probability $1-\rho\, S^\gamma$ or explore a new one with $\rho\, S^\gamma$, but there is an additional ``social pressure'' component. This occurs with probability $\alpha$ where one of the agent's contacts are selected at random with probability proportional to the co-similarity in the location visiting profiles. Consequently, the new location is chosen from the list of the contact's previously visited locations. In the absence of social pressure, (probability $1-\alpha$) the behavior is the same as Sec.~\ref{sec:epr}. The introduction of the social pressure component does a better job at reproducing the re-visitation profiles measured in CDR data than models without this component.  

\subsection{Population-Level}
\label{sec:poplev}

Mobility information gathered at the individual level can be aggregated to study the flows of individuals traveling from one region to another at different spatio-temporal scales. These flows can be organized in the framework of Origin-Destination (OD) matrices (see \sectionname~\ref{sec:odmatrix}). Such a format, with all possible combinations of origins and destinations for trips, is easily transformed into a directed weighted network in which nodes denote locations (for example counties or municipalities) and link weights correspond to the flow of travelers between the two locations. As discussed earlier in the review, OD matrices can be empirically estimated from transportation surveys, traffic counts or individuals' geolocated ICT data. An OD matrix, or its corresponding network, provides useful information on the travel demand between the origin and destination areas, representing a valuable asset widely studied and used in Geography, Transportation research and Urban Planning. Therefore, being able to obtain accurate estimation of OD matrices is crucial for both modeling and applications, and this problem has attracted the interest of researchers and decision makers for decades.


Note that individual mobility patterns are straightforwardly aggregated into flows, however the inverse problem, i.e the disaggregation of flows, is typically not possible. Instead, one has to resort to determine dependences between mobility flows and a limited set of ``static" attributes of the locations that would allow predictions on how changes in these attributes can potentially influence future travel demand. To this end, various spatial interaction models have been proposed to predict flows of individuals based on a small number of key local attributes. Considering a region of interest divided into $n$ locations, the purpose of these models is to estimate the number of trips $T_{ij}$ from location $i$ to location $j$ from the socio-economic characteristics of the populations of $i$ and $j$, and their spatial distribution. Models of spatial flows have been traditionally developed starting from the principle of entropy maximization subjected to various constraints. In the strongest version, the constraints are the number of people leaving and entering each location, but in softer ones there can be other proxy variables to represent the demand and attraction of the trips' origins and destinations such as population levels, the finite amount of resources for travels, utility functions to describe individual choices over competing alternatives, etc \cite{wilson_1967_statistical,mcfadden_1974_measurement,benakiva_1985_discrete,sagarra_2013_statistical,sagarra_2015_role}. 


\begin{figure}[t!]
\centering
\includegraphics[width=0.8\textwidth]{Figures_GenModels/fig_Ren_2014_predicting_1}
\caption{Differences between distance-based and intervening opportunity models. (a) The radiation model uses distance as a search criterion. (b) The cost-based radiation model uses network travel cost as a search criterion, which usually has a heterogeneous distribution. (c) The flow $T_{ij}$ through edge $(i, j)$ is the sum of contributions from all those mobility fluxes $\phi_{ab}$ whose minimal cost paths $\omega_{ab}$ contain (i, j). Figure from~\cite{ren_2014_predicting}.}
\label{fig:ren_2014_predicting}
\end{figure}


In the simplest version, the objective is to infer flows from the product of two types of variables: one type that depends on an attribute of each single location (e.g. the population), and the other type that depends on a quantity relating a pair of locations (e.g. the distance or travel time). 
The differences between the various models consist of the choice of variables considered, and the specific functional forms in which these variables enter.
%
Over the years, two main schools of thought have emerged. 
The first one assumes that the number of trips between two locations is 
a decreasing function of 
%inversely proportional to the costs associated with 
their distance, giving rise to the so-called gravity models \cite{carey_1858_principles,zipf_1946_p1}; the second variant postulates that the number of intervening opportunities, defined as the number of potential destinations between two locations, determines the mobility flow between them, and models that share this assumption are called Intervening Opportunities models \cite{stouffer_1940_intervening}. 
%
An illustrative example that highlights the fundamental differences between distance and intervening opportunities is shown in \figurename~\ref{fig:ren_2014_predicting}. In addition to providing the mathematical framework to model human mobility flows \cite{barthelemy_2011_spatial,thiemann_2010_structure,jung_2008_gravity}, these models have found successful applications in estimating other spatial flows, including cargo shipping volume \cite{kaluza_2010_complex} and social interactions from inter-city phone calls \cite{krings_2009_urban,p._2011_uncovering}.

\subsubsection{Gravity models}
\label{sec:gravity}

Preceded conceptually by the work of H.C. Carey in 1858 about land use \cite{carey_1858_principles} and the retailing models of W. J. Reilly in 1931 \cite{reilly_1931_law}, George K. Zipf proposed in 1946 an equation to calculate mobility flows inspired by Newton's law of gravitation \cite{zipf_1946_p1}\footnote{Although the model is not stated as a \emph{gravity} model, Zipf draws a parallel between his model and a two dimensional ``gravitation" equation.}. In his work, Zipf highlights the importance of the distance for human migration patterns where the magnitude $T_{ij}$ of a migratory flow between two communities $i$ and $j$ can be approximated by 
\begin{equation}
T_{ij} \propto \frac{P_{i} \, P_{j}}{r_{ij}} ,
\end{equation}
where $P_{i}$ and $P_{j}$ are the respective populations and $r_{ij}$ the distance between $i$ and $j$.


The basic assumptions of this model are that the number of trips leaving $i$ is proportional to its population,  the attractivity of $j$ is also proportional to $P_j$, and finally, that there is a cost effect in terms of distance traveled. These ideas can be generalized assuming a relation of the type:
\begin{equation}
T_{ij} = K m_i m_j f(r_{ij}) ,
\label{eq:grav}
\end{equation}
where $K$ is a constant, the masses $m_i$ and $m_j$ relate to the number of trips leaving $i$ or the ones attracted by $j$, and $f(r_{ij})$, called a ``friction factor'' or ``deterrence function", is a decreasing function of distance. As in the original version of the model, the masses usually are functions of population (not necessarily linear); common functional forms used in the literature are  $m_i = {P_i}^\alpha$ or $m_j = {P_j}^\gamma$~\cite{ortuzar_2011_modeling}. Unlike the original version, however, other variables, such as gdp-per-capita, may factor in the definition of the masses~\cite{mccullogh_1989_generalized,li_2011_validation}. The distance function $f(r_{ij})$ is commonly modeled with a power-law or an exponential form, although more complicated functions can be considered, such as a combination of the two, 
\begin{equation}
f(r_{ij}) = \alpha \, r_{ij}^{-\beta} \, e^{- \gamma r_{ij}}.
\label{eq:fij}
\end{equation} 
Indeed, the optimal form of the function  may change according to the purpose of the trips, the spatial granularity of the locations, and the transportation mode \cite{barthelemy_2011_spatial}. For example, in the case of commuting flows, the value of the exponent is highly correlated with the scale~\cite{lenormand_2012_universal,lenormand_2015_systematic} as can be seen in \figurename~\ref{fig:systematic}. In other  applications, the distance may not be the appropriate variable to quantify the cost of travel between two locations, and other variables such as travel time or economic (i.e. monetary) cost of a trip may offer better characterizations.
\begin{figure}[t!]
  \centering
 \includegraphics[width=0.8\textwidth]{Figures_GenModels/fig_lenormand_2016_systematic_8}
  \caption{The distance exponent $\beta$ (\equationname~\eqref{eq:fij}) as a function of average unit surface area. (a) Normalized gravity laws with an exponential distance decay function. (b) Normalized gravity laws with a power distance decay function. (c) Schneider's intervening opportunities law. (d) Extended radiation law. Figure from \cite{lenormand_2015_systematic}.}
\label{fig:systematic}
\end{figure}

The ability to estimate, even if as a crude approximation, trip-flows, and consequently, traffic demand between two different locations as a function of their local properties, has made the gravity model widely popular in transport planning \cite{erlander_1990_gravity,ortuzar_2011_modeling}, in studies of geography \cite{wilson_1970_urban} and spatial economics \cite{karemera_2000_gravity,patuelli_2007_network}. It has been also used in situations where the knowledge of mobility flows is essential but the empirical data is not available as in the case of the characterization and modeling of epidemic spreading patterns~\cite{xia_2004_measles,viboud_2006_synchrony,balcan_2009_multiscale,balcan_2010_modeling,li_2011_validation}. 
Despite its widespread use and historical popularity, one must keep in mind that the gravity model is a gross simplification of travel flows, and in many cases, falls far short of capturing actual empirical observations~ \cite{simini_2012_universal,masucci_2013_gravity,lenormand_2015_systematic}.  Furthermore, the model requires the estimation of a number of free parameters, making it rather sensitive to fluctuations or incompleteness in data~\cite{jung_2008_gravity,simini_2012_universal}.\\ 

\noindent {\bf Constrained gravity models.} Some of the limitations apparent in the gravity model, can be resolved via certain constrained versions. For example, one may hold the number of people originating from a location $i$ to be a known quantity $O_i$, and the gravity model is then used to estimate the destination, constituting a so-called \emph{singly-constrained} gravity model of the form,
%
\begin{equation}
T_{ij} = K_i O_i m_j  f(r_{ij}) =  O_i  \frac{m_j f(r_{ij})}{\sum_k m_k \, f(r_{ik})}.
\label{eq:scgm}
\end{equation}
%
As can be seen, in this formulation, the proportionality constants $K_i$ depend on the location of the origin and its distance to the other places considered.
One can go further and fix now also the total number of travelers \emph{arriving} at a destination $j$ as $D_j = \sum_i T_{ij}$, leading to a {\it doubly-constrained gravity model}. For each Origin-Destination pair, the flow is calculated as
\begin{equation}
T_{ij}= K_i  O_i  L_j D_j \, f(r_{ij}) ,
\label{eq:dcgm}
\end{equation}
where there are now two flavors of proportionality constants,
%
\begin{equation}
K_i =  \left[ \sum_j L_j \, D_j \, f(r_{ij}) \right]^{-1}, \qquad
L_j =  \left[ \sum_i K_i \, O_i \, f(r_{ij}) \right]^{-1}, 
\end{equation}\label{eq:KL}
usually calibrated with an \textit{Iterative Proportional Fitting} procedure \cite{deming_1940_least}. 

The use of singly-, doubly- or non-constrained models depends on the amount of information available and on the pursued objective. If the aim is to approximate the mobility flows and  transport demand from indirect socio-economic variables of different geographical areas, then one employs non-constrained models. On the other hand, if out-going or in-going flows are empirically measured quantities,  and the objective is to estimate the elements of the OD matrix $T_{ij}$, then one employs constrained models.\\

\noindent {\bf Maximum entropy derivation of gravity models.} 
While the gravity model may seem rather \emph{ad hoc} in terms of the form of \equationname~\eqref{eq:grav}, an argument in favor of this functional form was provided by Alan Wilson \cite{wilson_1970_entropy} in the framework of classical transportation theory combined with entropy maximization. Essentially, the argument posits that in the absence of detailed information, out of all possible variants of OD matrices, the most probable ones are those that can be obtained with the highest number of trip configurations under opportune global and local constraint satisfaction. The objective is thus to find the set of flows $\{T_{ij}\}$ that maximizes the number of possible configurations of trips associated with it, respecting the possible constraints. 

Let $\Omega(\{T_{ij}\})$ be the number of distinct arrangements of individuals (configurations) that give rise to the set of flows $\{T_{ij}\}$, 
corresponding to the number of ways in which $T_{11}$ individuals can be selected from the total number of travelers $T = \sum_{ij} T_{ij}$; $T_{12}$ from the remaining $T - T_{11}$ and so on and so forth.   
Then we have that 
%
\begin{equation} 
		\Omega(\{T_{ij}\}) = 
\frac{T !}{T_{11}! \, (T - T_{11})!} \frac{(T - T_{11})!}{T_{12}! \, (T - T_{11} - T_{12})!} \dots = 
\frac{T !}{\prod_{ij} T_{ij}!} .
\label{eq:omega}
\end{equation}  
%
One then determines the maximum of $\Omega$ using Lagrange multipliers, subject to the constraints: 
%
\begin{equation}
 \label{eq:constraints}
  \left\{ 
    \begin{array}{l}  
        \sum_j T_{ij}=O_i\\					
				$\,$ \\	
				\sum_i T_{ij}=D_j  \\ 
        $\,$ \\	
			\sum_{ij} T_{ij}\, C_{ij}=C, \\ 
    \end{array} 
	\right.
\end{equation}
where $C_{ij}$ is the cost of travel from location $i$ to location $j$. The first two constraints ensure that trips originating and terminating in every location are equal to their observed values, whereas the third constraint fixes a total cost for all trips, $C$. 
In the limit of a large number of trips, $T$, the configuration that maximizes $\Omega$ is by far the most probable, and hence dominates over all other configurations. 
The resulting doubly-constrained gravity model takes the form
%
\begin{equation}
    T_{ij}=K_i O_i  L_j D_j \, e^{-\beta \, C_{ij}} ,
		\label{eq:doubly}
\end{equation}  
%
where the values of $K_i$ and $L_j$ are set in order to fulfill the first two constraints in \equationname~\eqref{eq:constraints}.
%Note that the total cost $C$ does not
The Lagrange multiplier $\beta$ appearing in \equationname~\eqref{eq:doubly} controls the effect of cost on flows, and its value is empirically calibrated. It is worth noting that one can introduce a power law distance decay $f(r_{ij}) = r_{ij}^{- \beta}$ by considering a cost function of the form $C_{ij} \propto ln(r_{ij})$. These arguments have been further developed in a series of references~\cite{sagarra_2013_statistical,sagarra_2015_role} that also includes the distinguishable nature of the travelers and therefore necessarily modifying the statistics.\\


\noindent {\bf Using gravity models to estimate mobility flows.} In order to calibrate and apply gravity models to estimate mobility flows within a given region the following procedure is usually adopted. 
\begin{enumerate}
%
\item First, depending on the objectives, a flavor of gravity model is selected from either the unconstrained version, \equationname~\eqref{eq:grav}, or one of the constrained versions, i.e  \equationname~\eqref{eq:scgm}, or \equationname\eqref{eq:dcgm}.%

\item Second, the set of independent variables, population size, gdp or gdp-per-capita, distance, etc, as well as their relation with the local outflow, the attractiveness and the travel cost must be established. Although the choice of functions are somewhat arbitrary, common forms are power laws for the origin and destination populations, and exponential or power laws for the distance dependence. 
These particular functional forms are chosen to enable a fast and accurate calibration of the model, as it ensures that the logarithm of the flow depends linearly on some functions of the populations and the distance, allowing researchers to apply linear regression methods to determine the parameter values. 
%
\item Third, the parameter values are selected in order to maximize the fit between the flows estimated by the gravity model and the empirical flows observed in the region of interest. 
The best fit values of the parameters are determined using an optimization algorithm that either minimizes some error function between the model's estimates and the observed data \cite{cha_2007_comprehensive}, or maximizes the likelihood function of the observed data given the model's parameters \cite{flowerdew_1982_method}. 
Generalized Linear Models (GLM) \cite{nelder_1972_generalized} are a generalization of linear regression that are usually applied to fit the parameters of globally and singly constrained gravity models. 
GLM methods are more adapt than Ordinary Linear Regression (OLM) as it allows for the use of a wider and more realistic range of  probabilistic models to capture fluctuations in flow estimates. 
\end{enumerate}

Both the model training and parameter calibration steps can be employed either on the entire dataset or on subsamples, particularly if the aim is to validate the performance of the model in relation to the remaining set. 

\subsubsection{Intervening opportunities models}
\label{sec:Intervening}

Along with the gravity model, one of the first attempts to provide a conceptual and formal model of human mobility was introduced in 1940 by Stouffer \cite{stouffer_1940_intervening}. Contradicting a long tradition in the social sciences -- that distance is the central factor in determining mobility -- Stouffer proposed a conceptual framework in which distance and mobility are not directly related. Instead, Stouffer suggested that what plays the key role in determining migration is the number of \emph{intervening opportunities} or the cumulative number of \emph{opportunities} between the origin and the destination. In the paper, Stouffer does not provide a precise definition for ``opportunities'', leaving it to be defined depending on the social phenomena under investigation.

The basic idea behind the intervening opportunities (IO) model is that the decision to make a trip is not explicitly related to the distance between origin and destination, but to the relative accessibility of opportunities for satisfying the objective of the trip. An opportunity is a destination that a trip-maker considers as a possible termination point for their journey, and an intervening opportunity is a location that is closer to the trip maker than the final destination but is rejected by the trip maker. The law of intervening opportunities as proposed by Stouffer in 1940 states ``The number of persons going a given distance is directly proportional to the number of opportunities at that distance and inversely proportional to the number of intervening opportunities''.
Stouffer used this model to estimate migration patterns between services and residences \cite{stouffer_1940_intervening}. 
The traditional form of the intervening opportunities model is usually given by Schneider's version of Stouffer's original model \cite{schneider_1959_gravity}. 
Schneider's hypothesis states that ``The probability that a trip ends in a given location is equal to the probability that this location offers an acceptable opportunity times the probability that an acceptable opportunity in another location closer to the origin of the trips has not been chosen''. 
More formally, 
%the potential destinations $j \in \{2,..,n\}$ of trips originating in $i$ are ordered by travel cost from the origin location $i$ and 
the flow $T_{ij}$ from the origin location $i$ to the $j$-th location ranked by travel cost from $i$ is given by:
\begin{equation}
  T_{ij}=O_i \frac{e^{-L \, V_{ij-1}}-e^{-L\, V_{ij}}}{1-e^{-L \,V_{in}}} .
  \label{IO}
\end{equation}
Here, $O_i$ is the total number of trips originating from $i$ and the second term represents the probability that one of these trips ends in location $j$. The denominator is a normalization factor ensuring that the probabilities sum to 1 (i.e. $\sum_j T_{ij}=O_i$). This probability depends on $V_{ij}$, the cumulative number of opportunities up to the $j$-th location ranked by travel cost from the origin location $i$ (and $n$ is the total number of locations in the region considered). 
Usually, the population, $m_j$, or the total number of arrivals, $D_j=\sum_i T_{ij}$, are assumed to be proportional to the number of ``real opportunities'' in location $j$. 
The value of the parameter $L$ can be seen as the constant probability of accepting an opportunity destination. As in the case of the gravity model, the value of $L$ is adjusted in order to obtain simulated flows as close as possible to observed data. 

Several studies and variants of intervening opportunities models have been developed on this concept~\cite{heanus_1966_comparative, ruiter_1967_toward, haynes_1973_intermetropolitan, wilson_1970_urban, fik_1990_spatial, akwawua_2001_development}. The gravity and the intervening opportunities models have been compared several times during the second half of the twentieth century, showing that generally both models perform comparably~\cite{david_1961_comparison, pyers_1966_evaluation, lawson_1967_comparison, zhao_2001_refinement}. In fact one can cast the intervening opportunities model as a special variant of the gravity model with the friction factor $f(r_{ij})$ replaced by a function of the number of opportunities between the two locations, $f(V_{ij}) = e^{-L \, V_{ij-1}}$ \cite{eash_1984_development, zhao_2001_refinement}. Consequently, the intervening opportunities model can also be derived from entropy maximization methods as discussed earlier in the context of the constrained gravity model.  Indeed, some authors  have proposed hybrid gravity-opportunities models taking into account both the effect of distance and the number of opportunities between locations~\cite{willis_1986_flexible, goncalves_1993_development}.

However, unlike the gravity approach and despite relative good performances, the intervening opportunities model has suffered a growing loss of popularity. This is mainly due to the lack of research effort into the implementation and calibration of the model, attributable to the fact that the theoretical and practical advantages of opportunities models over the gravity ones are not overwhelming \cite{ortuzar_2011_modeling}.

\begin{figure}[t!]
\centering
\includegraphics[width=0.7\textwidth]{Figures_GenModels/fig_simini_2012_universal_1}
\caption{ Schematic of the the radiation model. (a) Commuting flows in two pairs of counties,
one in Utah (UT) and the other in Alabama (AL), with similar
origin (m, blue) and destination (n, green) populations and comparable distance $r$ between
them (see bottom left table). Number of travelers in the data, as predicted by the gravity model and finally for the radiation model shown as upper right inset. The definition of
the radiation model: (b) An individual (e.g. living in Saratoga County, NY) applies for
jobs in all counties and collects potential employment offers. The number of job
opportunities in each county is chosen to be proportional to the resident
population. Each offerÕs attractiveness (benefit) is represented by a random variable
with distribution $p(z)$, the numbers placed in each county representing the best offer
among the jobs in that area. Each county is marked in green (red) if its best offer
is better (lower) than the best offer in the home county. (c) An individual
accepts the closest job that offers better benefits than his home county.  Figure from~\cite{simini_2012_universal}. 
\label{fig:fig_simini_2012_universal_1}}
\end{figure}


\subsubsection{The radiation model}
\label{sec:radiation}

The concept of intervening opportunities and Schneider's hypothesis has recently triggered a renewed interest thanks to the recently proposed radiation model \cite{simini_2012_universal}. The radiation model assumes that the choice of a traveler's destination consists of two steps. 
First, each opportunity in every location is assigned a fitness represented by a number, $z$, chosen from some distribution $p(z)$, whose value represents the quality of the opportunity for the traveler. 
Second, the traveler ranks all opportunities according to their distances from the origin location and chooses the closest opportunity with a fitness higher than the traveler's fitness threshold, which is another random number extracted from the fitness distribution $p(z)$ (see \figurename~\ref{fig:fig_simini_2012_universal_1}). 
As a result, the average number of travelers from location $i$ to location $j$, $T_{ij}$, takes the form:
\begin{equation}
   T_{ij}=O_i \, \frac{1}{1-\frac{m_i}{M}} \frac{m_i \, m_j}{(m_i+s_{ij})\, (m_i+m_j+s_{ij})} .
	\label{eq:rad}
\end{equation}
Here again, the destination of the $O_i$ trips originating in $i$ is sampled from a distribution of probabilities that a trip originating in $i$ ends in location $j$. This probability depends on the number of opportunities at the origin $m_i$, at the destination $m_j$ and the number of opportunities $s_{ij}$ in a circle of radius $r_{ij}$ centered in $i$ (excluding the source and destination). This conditional probability needs to be normalized so that the probability that a trip originating in the region of interest ends in this region is equal to $1$. In case of a finite system it is possible to show that this is equal to $1-\frac{m_i}{M}$ where $M = \sum_i m_i$ is the total number of opportunities \cite{masucci_2013_gravity}. In the original version of the radiation model, the number of opportunities is approximated by the population, but the total inflows $D_j$ to each destination can also be used \cite{lenormand_2012_universal, masucci_2013_gravity, lenormand_2015_systematic}. The great advantage of the radiation model compared with other spatial interaction models is the absence of a parameter to calibrate with observed data.  In particular, the flows defined in \equationname~\eqref{eq:rad} are independent of the fitness distribution $p(z)$.
However, this advantage represents also a limitation since the model does not seem to be very robust to changes in the spatial scale \cite{lenormand_2012_universal, masucci_2013_gravity, liang_2013_unraveling, lenormand_2015_systematic}. To overcome this drawback, a radiation model with opportunities' selection \cite{simini_2013_human} and an extended radiation model \cite{yang_2014_limits} have been proposed. In this extended version, the conditional probability to perform a trip between two locations according to the spatial distribution of opportunities is derived under the survival analysis framework introducing a parameter $\alpha$ to control the effect of the number of opportunities between the source and the destination on the location choice. The addition of this scaling parameter seems to greatly improve the performance of the model, and similar to the gravity model, its value can be inferred from the scale of the study region according to the homogeneity of the opportunities' spatial distribution \cite{yang_2014_limits,lenormand_2015_systematic}.

\begin{figure}[t!]
\centering
\includegraphics[width=0.8\textwidth]{Figures_GenModels/fig_carra_2016_modeling_2}
\caption{ The closest opportunity model (a) Sketch with the main ingredients of the model: the residence place of the agents and the opportunities with their correspondent quality until the last one at distance $r$ is selected. In (b), (c) and (d) rescaled distributions of commuting distances. In blue, the empirical data for several years in the three countries, in dark blue the averaged empirical distributions, and superimposed in red the model fits using a single parameter. Figure from~\cite{carra_2016_modeling}. 
\label{fig:carra_2016_modeling_2}}
\end{figure}


% \subsubsection{The closest opportunity model}
% \label{sec:closest}
% An extension along the lines of the intervening opportunity and radiation models has been recently proposed in the context of job search, and consequently, for the formation of commuting flow networks has been given in \cite{carra_2016_modeling}. 
% The model takes inspiration from the classic hypothesis of MacCall \cite{maccall_1970_economics}, and incorporates an extension that accounts for the spatial nature of the area in which individuals search for jobs.

An interpretation of the radiation model 
in the context of job search and, consequently, for the formation of commuting flow networks has been given in \cite{carra_2016_modeling}. 
The basic components are individuals residing within a demarcated geographical area, who are seeking employment. Job locations are uniformly distributed in space and characterized by a fitness parameter $z$, which itself is drawn from a distribution that includes aspects such as wage and worker-convenience. Individuals have a certain tolerance level, $z^{\ast}$, and will search in a progressively increasing radius from their residence, with the search terminating at the first instance of the condition $z >  z^{\ast}$. Using extreme value statistics, it is possible to determine the distance distribution, $P(r)$, between residence and job places, which is independent of the distribution of $z$ (for non-pathological cases), and was calculated to be of the form 
\begin{equation}
P(r) = \frac{2\, \rho\, \pi\, r}{(1+ \rho\, \pi \, r^2)^2} ,
\end{equation}
where $\rho$ is the density of jobs. The model satisfactorily reproduces some of the empirical features of the travel distance distributions in commuting data as can be seen in \figurename~\ref{fig:carra_2016_modeling_2}.     


\subsubsection{Comparison between models}
\label{sec:comparison}

The models described so far provide theoretical expectations for the OD matrices in terms of the flow values $T_{ij}$ between geographical regions $i$ and $j$. Before employing these for practical applications, it is essential to validate the expected (calculated) flows against empirical evidence, that may be limited in the sense of being restricted in space compared to regions of interest, or indeed to shorter time windows than the temporal period of interest. By evaluating the performance of each model, one selects %(in principle at any rate), 
the one that most closely matches the empirical data (if available). The accuracy, in general, will depend on the spatio-temporal scale as well as on available (meta) information and the extent of missing data. It is crucial that comparison between models is done on an equal footing, i.e using the same extent of available data and in equal spatio-temporal resolutions. As discussed, the model output is the value of $T_{ij}$, subject to constraints, which range from \equationname~\eqref{eq:grav}, where only the total sum of the flows is imposed via the parameter $K$, the singly constrained framework models based on intervening opportunities or the closest opportunity case, which is production constrained, in the sense that outflows $O_i$ are fixed \emph{ a priori} and must be preserved for every location, and finally to the doubly-constrained models, where both the outflows $O_i$  and the inflows $D_i$ in every region $i$ are preserved. 

\begin{figure}[t!]
\centering
\includegraphics[width=0.7\textwidth]{Figures_GenModels/fig_lenormand_2015_systematic_3}
\caption{Performance comparison of the different models described in the text using the CPC metric defined in \equationname~\eqref{eq:cpc} applied to census commuting data from England and Wales (E\&A), France (FRA), Italy (ITA), Mexico (MEX), Spain (SPA), USA. Additional census data at the city-scale was gathered from and London (LON) and Paris (PAR). The different symbols and colors represent the different flavors of the model. The short-forms exp (exponential) and pow (power-law) refer to the functional forms of distance dependence. Figure from~\cite{lenormand_2015_systematic}. 
\label{fig:lenormand_2015_systematic_3}}
\end{figure}

In general, all of the described models share some common assumptions:
\begin{enumerate}
\item The trip distribution $P_{ij}$ that generate the flows $T_{ij}$ is independent of the trip production $O_i$.
\item In both the unconstrained and singly-constrained models, choice of travel destinations are statistically independent (i.e there are no memory effects).  
\item Flows are estimated as a product of variables related to opportunities and distance (i.e. variables are "separable").
\end{enumerate}
Under these assumptions, it is possible to draw statistical laws that govern the distribution of travels $P_{ij}$, and from thereon, build models with varying levels of constraints. This is important to adapt the models to the available information and to fairly compare between them.

One such comparison was carried out in~\cite{lenormand_2015_systematic}, where commuting data was obtained from the census offices of England and Wales (E\&A), France (FRA), Italy (ITA), Mexico (MEX), Spain (SPA), USA and then in shorter spatial scales, from the cities of London (LON) and Paris (PAR). Several variants of the gravity, intervening opportunities and radiation models were considered with multiple levels of constraints. The predicted flows $T_{ij}$ were compared with the empirical data using two related metrics: The \emph{common part of commuters} ($CPC$) for all location pairs $i,j$ with positive flows both in the empirical data set and in the model prediction, defined as
\begin{equation}
CPC = \frac{\sum_{i,j} min (T^m_{ij}, T^e_{ij})}{T} = 1 - \frac{1}{2} \, \frac{\sum_{i,j} |T^m_{ij}-T^e_{ij}|}{T}.
\label{eq:cpc}
\end{equation}      
Here $T = \sum_{i,j} T^e_{ij}$ is the total number of commuters, $T^m_{ij}$ the model prediction for the flow, and $T^e_{ij}$ the empirical value. The $CPC$ is one, if the agreement is perfect and zero if there is no overlap between data and model. The ability of the models to
recover the topological structure of the original network was assessed through the second metric, termed the \emph{common part of links}
(CPL) and defined as

\begin{equation}
CPL = \frac{2 \, \sum_{i,j} 1_{T^m_{i,j} > 0} \, 1_{T^e_{i,j}}}{\sum_{i,j} 1_{T^m_{i,j} > 0} + \sum_{i,j} 1_{T^e_{i,j} > 0}} ,
\end{equation}
where $1_{T_{i,j} > 0}$ is the Heaviside function. The CPL measures the proportion of links in common between
the simulated and the observed networks, it is null if there is no link in common and one if both networks are topologically
equivalent. The results in terms of the $CPC$ are shown in Fig. \ref{fig:lenormand_2015_systematic_3}. For this choice of datasets,  the gravity model is seen to moderately outperform the other models. One must note of course, that this might change across different datasets.

\subsection{Intermodality}
\label{sec:intermodal}

A complete description of mobility must take into account the mutlimodal structure of transport and the transitions between the different modes as alluded to in \sectionname~\ref{sec:multimodal}. A good framework for carrying out such an analysis has been developed in recent years in the context of multilayer or multiplex networks~\cite{boccaletti_2014_structure,kivela_2014_multilayer}. These are networks in which the nodes can be present in one or multiple layers, and each layer by itself contains a set of links or node-node interactions; if a large fraction of nodes in the network are present in all apparent layers, then it is termed multiplex. On the other hand if one finds different flavors of nodes restricted within their own putative layers, then the network is dubbed a multilayered one. Indeed, each layer of the network constitutes a separate network in its own right depending on the context of its interactions. For example, in social networks, layers may correspond to different groupings such as coworkers, family, different cohorts of friends and so on, all of which have within-layer interactions. Information diffusion in the network, however, occurs through each layer and across the full multilayer structure. Translating this to the case of mobility and transportation networks, layers may correspond to different transportation modes (road, subway, airline), while connections between layers constitute the interchanges between these modes. Within this context, the standard way to construct multilayer networks is to associate locations to nodes and flows, and frequency of travel (or simply unweighted simple connections) to links between different transportation modes (see, for example \figurename~ \ref{fig:gallotti_2015_multilayer_2}).


\begin{figure}[t!]
\centering
\includegraphics[width=\textwidth]{Figures_GenModels/fig_gallotti_2015_multilayer_2}
\caption{An example of a multilayered network approach to mobility. Public transport networks in the London metro area, separated into multiple layers consisting of the bus, subway and rail networks. In this particular instance, one can see that the London bus network is the most ``used" layer. Figure from~\cite{gallotti_2015_multilayer}. }
\label{fig:gallotti_2015_multilayer_2}
\end{figure}

Traditionally, single-mode transport networks have been studied using mono-layers. Examples include metro networks in Beijing \cite{feng_2017_weighted}, Boston \cite{latora_2002_boston}, London \cite{angeloudis_2006_large,guo_2016_london}, Montreal \cite{derrible_2010_complexity}, Paris \cite{angeloudis_2006_large} and Seoul \cite{lee_2008_statistical}; bus networks in China \cite{xu_2007_scaling,chen_2007_study} and Poland \cite{sienkiewicz_2005_statistical}; train networks in Boston, Vienna \cite{seaton_2004_stations} and in India \cite{sen_2003_small}; and combined public transportation networks in $14$ cities around the world \cite{vonferber_2009_public}. In most of these cases, the transport network was generated using different perspectives. One such perspective is the $L$-space representation, where nodes correspond to stations in the transportation map, and are linked together if they are consecutive stops in a transportation line. Similarly, the network can be built in the $P$-space framework, if nodes are stations linked together if they correspond to stops in the \emph{same} line (irrespective of sequence). This latter representation corresponds to a  one-mode projection of a bipartite network~\cite{Newman_book} consisting of stations and lines. (For a graphic explanation of this notation see~\cite{vonferber_2009_public}.) In addition to geographical aspects, the link weights can also incorporate information on the number of passengers in the same route, among other features. 

The fact that a single transport network can be represented in multiple ways, naturally paved the way for the use of multilayer structures \cite{kurant_2006_layered,kurant_2006_extraction}. In this case, each layer contains the expression of the network in a different space, and the multilayer allows for a deeper study of the network accessibility for travelers. As mentioned before, a multilayer network can be also employed to condense the transport information coming from different modes. An example for this can be seen in \figurename~\ref{fig:gallotti_2015_multilayer_2} which shows elements of the transportation network in London, combining the bus, train and metro networks, whereas additional elements including air and ferry can also be incorporated~\cite{gallotti_2015_multilayer}. The more intricate topology of multilayer networks necessarily affects the spreading processes taking place on the system (diffusion and navigation)~\cite{dedomenico_2013_mathematical,gomez_2013_diffusion,kivela_2014_multilayer} that are modeled using the methods discussed in \sectionname~\ref{sec:rw}~and~\ref{sec:poplev}. This leads to new features such as congestion~\cite{sole_2016_congestion}, change in the navigability of the network subject to disruptions~\cite{dedomenico_2014_navigability, battiston_2016_efficient}, produces new dynamic regimes that are absent in monolayers~\cite{radicchi_2013_abrupt} and abrupt reactions to minimal topological changes~\cite{diakonova_2017_dynamical}.

A particular type of multilayer representation of transport networks is worth a separate discussion. In this case, nodes still correspond to locations, and links are assigned according to one of the  space-frameworks ($L$ or $P$), but the weight of the links now account for the temporal duration of trips between the nodes. This network configuration, which some authors refer to as a $PT$-space, appear in each of the layers, where each layer usually corresponds to a single transport mode. With such a representation, the optimal  time for a traveler traversing between a given OD pair, can be calculated using optimal path algorithms across the multilayer structure, containing the full transport network and taking into account intermodality. These types of multilayers have been studied, for example, in British \cite{gallotti_2014_anatomy} and French \cite{alessandretti_2016_user} cities. In the British case, one detects a shift in preference for different transportation modes as well as the overlap between temporally and spatially optimal paths as a function of distance. In the French case, a few particular (temporally optimal) connections were singled out, so-called \emph{efficient connections}, and their occurrence (and hence influence) in the general mobility modes were studied across a number of cities. Using a similar framework, with two layers corresponding to road and metro networks in the cities of London and New York, congestion effects as a function of varying the speed of metro lines was studied~\cite{strano_2015_multiplex}. Finally, it is worth mentioning the point raised in \cite{aleta_2016_multilayer}, where the authors discuss the differences between building the network by considering a layer as a single transport mode or as a single line. The latter description provide a deeper view of the inefficiencies due to line transfer of the passengers, while the former is well-suited to study topological aspects and geographical coverage.       













%\subsection{Urban Planning}
%IoT, LUTI models

