
Here we discuss some selected applications of the frameworks, concepts, models and datasets introduced thus far. We organize this section by scale, ranging from single scale (mostly dealing with transportation modes), multi-scale applications with a focus on movement in cities and epidemic spreading, and finally end with some new developments related to virtual scales, i.e mobility patterns seen in online activity.


\subsection{Single-Scale}

\subsubsection{Pedestrian movement}

Large concentrations of people such as those occurring during festivals, sport events or religious ceremonies can present a challenge in terms of service and transport demand as well as public safety. Avalanches in stadiums, processions and even subway stations have occurred in recent decades leading in high casualties~\cite{helbing_2013_pedestrian}. In the face of this, strategies need to be devised for ensuring prompt evacuation in cases of emergency when considering the configuration of new buildings, public spaces and transport systems. Consequently, understanding the empirical patterns governing pedestrian movement and being able to build corresponding models, has the potential to benefit a wide range of disciplines including architecture, urban planning and public service stakeholders among others. Given the broad range of interest, a large amount of research has been devoted to this sphere, over the past few decades. Indeed, several reviews exist on this topic~\cite{helbing_2013_pedestrian,helbing_2001_traffic,zainuddin_2010_characteristics,vicsek_2012_collective,benenson_2014_ten,cao_2015_cyber}, so instead of providing a detailed retrospect, we provide here an overview of the most recent developments. 

Modeling frameworks on pedestrian movement can be broadly classified according to the approach used to consider individual pedestrians. For example, when the aim is crowd control, then usually the number of individuals considered is quite large, and the metrics of interest are crowd velocities or pressures. In such a case, one can define fields at every point in space and time to characterize the state of the system. These fields usually include the density of pedestrians $\rho(\vec{x},t)$, the local velocity $\vec{v}(\vec{x},t)$ or the flow across an exit, $J(\vec{x},t)$. In analogy with fluid dynamics, one sees a continuity equation of the form,\begin{equation}
\label{eq:ped:conv} 
\frac{\partial \rho(\vec{x},t) }{\partial t} = - \nabla (\rho(\vec{x},t) \, \vec{v}(\vec{x},t)),
\end{equation}
stemming from the conservation number of people in the crowd.  
The motion of pedestrians can be assumed to minimize the time spent in the system. This condition is included via the introduction of a field $\phi(\vec{x},t)$ coupled with the density in such a way that the local velocity shows a relation
\begin{equation}
\label{eq:ped:speed} 
\vec{v}(\vec{x},t) = f(\rho) \frac{\nabla \phi}{||\nabla \phi||},
\end{equation}
where the gradient of $\phi$ determines the direction of motion and $f(\rho)$ is a simple function of the density ensuring that the pedestrians speed decreases  as the density increases. Possible choices are a linear relation, $f(\rho) = \rho - \rho_{max}$, keeping $\rho < \rho_{max}$ or a non linear one, $f(\rho) = (\rho - \rho_{max})^2$, among others~\cite{hughes_2002_continuum,helbing_2006_analytical,carrillo_2015_local}. The connection with $\phi$ is established through the function $f(\rho)$ thus,
\begin{equation}
\label{eq:ped:phi} 
||\nabla \phi|| = \frac{1}{f(\rho)},
\end{equation}
leading to the so-called \emph{Hughes model}. In a scenario with individuals exiting a room, it produces two regimes in the pedestrian mobility depending on the global density, one with free flow and another with congestion. Furthermore, it turns out that shock-waves and moving congestion fronts emerge due to the narrowing of the crowd close to the door, leading to intermittent exit flow, avalanches and stop-and-go effects \cite{helbing_2006_analytical}. 

\begin{figure}[t!]
\centering
\includegraphics[width=\textwidth]{Figures_Applications/fig_helbing_2000_simulating_1}
\caption{Social force model simulation of a crowd trying to leave a room through a narrow door. (a) a representative configuration. (b), the sequence of leaving times of the agents as a function of $v_0$. In (c) and (d), the total evacuation time and the average flow of people as a function of $v_0$. Figure from~\cite{helbing_2000_simulating}. \label{fig:helbing_2000_simulating_1}}
\end{figure}


An alternative approach is to use agent based modeling where the state of every individual (agent) $i$ is simulated in detail. This framework is especially applicable to simulations of heterogeneous populations in which the agents may show different features. The most popular model falling into this category is the so-called \emph{social force model} proposed by Helbing and Moln\'ar \cite{helbing_1995_social, helbing_2001_self}. The idea behind this model is to apply the analogues of Newton's laws of motion to each pedestrian $i$. If the agent's velocity is $v_i$, the location $r_i$ at time $t$, then the acceleration can be expressed as
\begin{equation}   
\label{eq:ped:sfm}  
\frac{d v_i}{dt} = f_i^d + f_i^a + f_i^{rb} + \sum_{j \ne i} f_{ij}^r + \eta(r,t) .
\end{equation}
The first term on the right hand side accounts for the tendency of the individuals to walk at a certain desired speed $v_i^0$. It is usually written as 
\begin{equation}   
\label{eq:ped:des}  
f_i^d = \frac{v_i^0-v_i}{\tau} ,
\end{equation}     
which ensures the recovery of the desired speed in a characteristic reaction time $\tau$. The second term $f_i^a$ is an attraction force depending on the agent's position introduced to guarantee that walkers are compelled to a certain target or direction. The third force, $f_i^{rb}$, is repulsive depending on the agent's location and takes into account the effects of the boundaries and other obstacles on the agent's trajectory. The next term represents a repulsive force felt by the presence of other agents $j$ different from $i$. Finally, $\eta(r,t)$, is an uncorrelated noise term that introduces low levels of random fluctuations to avoid deadlocks.

\begin{figure}[t!]
\centering
\includegraphics[width=0.8\textwidth]{Figures_Applications/fig_helbing_2007_dynamics_3}
\caption{Analysis of an avalanche event during the Hajj pilgrimage of 2006. (a) Representative trajectories of the laminar flow, the stop-and-go and the turbulent regimes for individual movements. (b) The velocity in the turbulent regime. (c) The "pressure" as a function of time. (d) Different distributions of speed increments in the two regimes. (e), Distribution of displacements between consecutive stops. (f) The structure function in the turbulent regime. Figure from~\cite{helbing_2007_dynamics}. \label{fig:helbing_2007_dynamics_3}}
\end{figure}


The social force model reproduces several phenomena observed in empirical circumstances. For example, when two groups of pedestrians moving in opposite directions meet in a corridor, unidirectional stripes are formed, or when they meet in a door, the flow displays intermittence due to clogging \cite{helbing_2001_self} (\figurename~\ref{fig:helbing_2000_simulating_1}). Variations of this model have been used to analyze escape patterns from a building in low visibility conditions \cite{helbing_2000_simulating, helbing_2013_pedestrian}, to assess evacuation plans in skyscrapers \cite{sikora_2012_model,parisi_2014_sequential}, ships \cite{chen_2015_modeling} and aircraft \cite{kirchner_2003_friction}. In relation to crowd control, they have also been used to analyze how to mitigate avalanches in the Hajj religious gathering in Saudi Arabia~\cite{helbing_2007_dynamics,johansson_2008_crowd}, in the Notting Hill Carnival in London~\cite{batty_2003_safety} or, along with mobility models for other transport media, to study evacuation of cities and populated areas in case of natural disasters such as flooding ~\cite{lammel_2010_representation,kunwar_2014_large,kunwar_2015_evacuation}. Additionally, these models provide the basic tool to model scenarios such as the formation of mosh and circle pits as a mark of collective motion in heavy music concerts \cite{silverberg_2013_collective}, as well as the risk of disease propagation in massive gatherings \cite{johansson_2012_crowd}.  


In the last decade, there has been a large volume of research related to the constituent terms of~\equationname\eqref{eq:ped:sfm}, including empirical calibration of the different coefficients as well as the introduction of a cognitive basis for the different forces~ \cite{zainuddin_2010_characteristics, johansson_2008_crowd}. In high density regimes,  such as those occurring in videos taken of the disaster of the Jamarat Bridge in the Hajj pilgrimage of 2006, the trajectories of single pedestrians become turbulent \cite{helbing_2007_dynamics,johansson_2008_crowd} (\figurename~\ref{fig:helbing_2007_dynamics_3}). The original version of the social force model is unable to  reproduce this turbulent regime and consequently modifications to the force terms in high density regimes, were introduced. One of these modifications is the centrifugal force model, so-called because of its similarity with the classical centrifugal force in rotating bodies, and involves a modification of the mutual repulsion term $ f_{ij}^r $ thus,\begin{equation}
f_{ij}^r  =  -m_i \, K_{ij} \, \frac{V_{ij}^2}{||\vec{r}_{ij}||} \, \vec{e}_{ij}. 
\end{equation}
%which contains explicitly the distance between $i$ and $j$, $||\vec{r}_{ij}||$, in the denominator. 
Here $m_i$ is the mass of pedestrian $i$, $K_{ij}$ is a coefficient enforcing the influence only of pedestrians in front of $i$ on its movements and $\vec{e}_{ij}$ is the unit vector pointing from $i$ to $j$~\cite{yu_2005_centrifugal,yu_2007_modeling} . This modification has the effect of introducing both laminar as well as turbulent regimes, a feature missing in earlier versions of the social force model. Further refinements also included the interactions of individuals based on specific boundaries and shapes of obstacles~\cite{chraibi_2010_generalized}. 

A second modification to the social force model deals with the cognitive capacities and limitations of pedestrians and how they can affect the force terms of the model \cite{moussaid_2009_experimental,moussaid_2011_how}. 
Examples include analysis of the time scales regulating pedestrian interactions~\cite{johansson_2009_constant}, the addition of a field to take into account the (partial) global perception of individuals and the environment under consideration \cite{dietrich_2014_gradient,colombi_2015_moving}, or the role of leaders \cite{degond_2015_time}. The capacity of the pedestrians to foresee the movements of others and the dangers of obstacles have been modeled within this framework in \cite{moussaid_2009_experimental,steffen_2009_modification}. The model has also been adapted to agents with an anisotropic perception of their environment \cite{gulikers_2013_effect} and its instabilities in a context of stop-and-go oscillations have been studied in \cite{chraibi_2014_oscillating,chraibi_2015_jamming}. A systematic stability analysis of its numerical solutions has been provided in~\cite{koster_2013_avoiding} together with a version of the model that notably increases its computational performance. 

In addition to this, there exist a family of models that lie between the macroscopic and microscopic approaches, characterized by discretization of space into cells, and the specification of rules on how agents navigate between these cells. Such \emph{cellular automata} models have the advantage of computational scaleability, as for example, in~\cite{dutta_2014_gpu} where simulations of crowds of size $10^5$, were conducted with running time scaling linearly with the number of individuals. The price to pay for this increased computational performance is the loss of local details within each cell. These models have been used to study clogging in exits, and due to obstacles by adding a ``friction term" between agents \cite{kirchner_2003_friction,yanagisawa_2009_introduction}, pedestrian flow through subsequent bottlenecks \cite{ezaki_2012_pedestrian} and the formation of stripes in junctions with agents moving in different directions \cite{cividini_2013_diagonal}.  Some modifications to these models include, for example, behavioral factors of the agents such as path selection based on the time necessary to reach the destination \cite{kirik_2009_shortest}, heterogeneous agents \cite{sarmady_2010_simulating} and long-range spatial awareness \cite{tissera_2014_simulating}.


\begin{figure}[t!]
\centering
\includegraphics[width=0.9\textwidth]{Figures_Applications/fig_kretz_2015_social_2}
\caption{Fundamental diagram with the speed as a function of the density of pedestrians for several experiments in a closed loop circuit (ovoid-like). On the left, experiments in India with different loop lenghts, plot originally taken from \cite{chattaraj_2009_comparison}.On the right, experiments in Germany with two set of people: students and soldiers, figure originally taken from \cite{protz_2011_analyzing}. The composite figure comes from~\cite{kretz_2015_social}. \label{fig:kretz_2015_social_2}}
\end{figure}



While some of the models described thus far have been inspired by empirical data obtained from crowds in different contexts, challenges remain in actually monitoring crowds to get accurate and precise measurements on individual movement (see \cite{benenson_2014_ten} for a recent review on methods used for this). 
One way to calibrate models is to take a controlled approach, i.e conduct experiments on a smaller scale, in which groups of people are directed to move under specific constraints. This controlled environment can provide insight into questions such as how pedestrians walk in social groups~\cite{moussaid_2010_walking}, the quantification of pressure and density in different types of bottlenecks and junctions~\cite{zhang_2014_quantification}, the perception of personal space~\cite{ducourant_2005_timing}, the clogging effects and instabilities in narrow areas~\cite{moussaid_2012_traffic,bukacek_2014_experimental} and the asymmetric flow produced by stairs~\cite{corbetta_2015_asymmetric}. One factor that is common to such experiment-based research is the generation and analysis of ``fundamental diagrams" \cite{jelic_2012_properties} (\figurename~\ref{fig:kretz_2015_social_2}), that represent the relationship between the average velocity $v$ and the density $\rho$ of pedestrians, as well the flux through a surface  $J$ (an exit) among other parameters. In the free flow regime, with no obstacles and low density, pedestrians tend to move at an optimal velocity. As the density increases, the velocity decreases until the crowd stops. Similarly, the flux through an exit also decreases with increasing pedestrian concentrations. The relations between these quantities are typically non-linear and are of great relevance for the design of buildings and urban spaces. Due to this, a large effort has been invested in the analysis of these diagrams. The results suggest while qualitatively the functional forms of $v(\rho)$ and $J(\rho)$ have common features across different settings, quantitative differences emerge based on extraneous factors. Some of these are cultural perception of personal space (a comparison between India and Germany was performed in \cite{chattaraj_2009_comparison}), the presence of T-junctions \cite{zhang_2013_experimental}, the individual's perception of open versus closed spaces \cite{zhang_2014_effects} and the bidirectional flow of agents \cite{zhang_2014_comparison}. 

\subsubsection{Air Transportation}
\label{sec:air}

\begin{figure}[t!]
\centering
\includegraphics[width=0.5\textwidth]{Figures_Applications/fig_guimera_2005_worldwide_2}
\caption{(a) The relation between node degree (traffic) and betweenness centrality (measure of load on a node) in the World Airport Network (WAN). (b) The location of the $25$ most connected cities. (c) The top $25$ cities in terms of the load of traffic (betweenness centrality). Figure from~\cite{guimera_2005_worldwide}. \label{fig:guimera_2005_worldwide_2}}
\end{figure}

On the opposite end of the spectrum from pedestrian mobility (which is necessarily short-range) is long-range transportation over thousands of kilometers, which is primarily represented by air transportation. Air transport is of paramount importance for modern global connectivity, playing a central role in the world economy and in the interchange of people, ideas and, unfortunately, also the propagation of diseases. 
While the study of such systems has a long tradition in the engineering community, primarily related to aeronautics and air traffic management~\cite{belobaba_2009_global,cook_2007_european}, recent years has provided further impetus to this line of research with the development of tools based on network science~\cite{Newman_book}. In this setting, initial studies were conducted from a static perspective, where airports are the nodes of the network, with a link between a pair of airports if a direct flight exists between them, discarding any directional information as well as temporal dynamics. The basic statistical features of such networks were analyzed, for instance, in~\cite{guimera_2004_modeling,barrat_2004_architecture,guimera_2005_worldwide} for the World Airport Network (WAN), in ~\cite{li_2004_statistical} for the Chinese network and in~\cite{da_2009_structural} for Brazil. Figure \ref{fig:guimera_2005_worldwide_2} shows the location of the high-degree nodes (maximum traffic) and those with the highest betweenness centrality (bottlenecks) in the WAN. Statistical distributions of the degree (number of destinations of the airports), the traffic (number of passengers per route or per airport) and betweenness were found to be typically heavy-tailed. 

Furthermore, it was found that the networks can be divided into modules with a certain level of self-contained traffic \cite{guimera_2005_worldwide}, although this division is not seasonably stable when the focus is set on smaller geographical scales such as the US network \cite{lancichinetti_2011_finding}. The average number of passengers per route $w_{ij}$ was determined to be a non-linear function of the traffic in airports of the form 
\begin{equation}
w_{ij} \sim \left(k_i\, k_j\right)^\theta,
\label{eq:wij_air}
\end{equation}
where $k_i$ is the traffic at the origin, $k_j$ at the destination and $\theta \approx 1/2$~\cite{barrat_2004_architecture}. An explanation of these measured topological properties has been proposed by simple models based on cumulative advantage~\cite{guimera_2004_modeling,barrat_2004_architecture}.  
The robustness of these networks to disruptions has been considered in terms of a unipartite framework~\cite{verma_2014_revealing} as well as multiplex-like approaches~\cite{cardillo_2013_modeling}, where the network of each airline forms a layer and the airports appearing in the different layers connect the structure. A short review of network analysis on air transportation can be found in \cite{zanin_2013_modelling}.

\begin{figure}
\centering
\includegraphics[width=0.7\textwidth]{Figures_Applications/fig_fleurquin_2014_trees_2}
\caption{A typical configuration of a delay tree. Each node represents an airport and the links represent connections between them. The link weights denote temporal delays (in minutes) and the tree is composed of different levels or generations. Each node has a certain branching number ($\rho$) for the next level.  Figure from~\cite{fleurquin_2014_trees}. \label{fig:fleurquin_2014_trees_2}}
\end{figure}

Of course the static topological framework does not take into account the dynamics occurring on the network. Indeed a key component of air transportation is that goods and people arrive in their destination in time. In particular logistical failures can propagate in the network where travel schedules are not met, with the delay propagating to effect a significant fraction of the system and paralyzing airport operations. This idea was first explored in the context of air transportation in~\cite{beatty_1998_preliminary}, where the concept of \emph{delay multipliers} was introduced. This can be seen as a combination of a delayed flight not only affecting the departure of other flights in the same airport, but also affecting connecting flights down the transportation chain, leading to the formation of avalanches or ``delay-trees"~\cite{fleurquin_2014_trees} as shown in \figurename~\ref{fig:fleurquin_2014_trees_2}. Direct costs of flight delays in Europe amount to more than one billion euros \cite{cook_2011_european}, while in the US direct and indirect costs go beyond $40$ billion dollars \cite{joint_2008_your}. Delays directly affect airlines since they increase operational costs, but as an indirect factor they also bring associated reputational costs \cite{folkes_1987_field,mayer_2003_network}. Passengers, on the other hand, undergo a direct loss of time, which can be further exasperated by lost connections leading to missed business and leisure opportunities. Finally, efforts to recover delays airborne usually imply excess fuel consumption and larger CO$_2$ emissions. 

Given these challenges, it is important to characterize the sources of initial (primary) delays~\cite{rupp_2007_further,ahmadbeygi_2008_analysis} and devise mitigating strategies. Unfortunately this is a highly complex problem with the involvement of numerous (and sometimes unrelated) factors such as weather conditions, technical failures in the aircraft, aircraft rotations, crew and passenger connections, organizational issues in airports, and the mechanisms through which these delays are transferred and amplified in subsequent operations~\cite{bonnefoy_2007_scalability, beatty_1998_preliminary, wang_2003_flight}. Airline schedules typically include a buffer time to absorb these delays \cite{wu_2000_aircraft}, but there are frequent scenarios when this is not nearly enough. The Central Office for Delay Analysis of Eurocontrol (CODA) in Europe and its counterpart in the US, the Bureau of Transportation Statistics of the Department of Transportation, release monthly reports on flight performance including statistics on the major causes of delays and the airport affected. Based on these reports, studies have been conducted in single hubs such as Newark~\cite{allan_2001_analysis} as well as parts of the European \cite{cook_2007_european, jetzki_2009_propagation} and US \cite{mayer_2003_network,churchill_2007_examining,fleurquin_2014_trees} networks. 

\begin{figure}[t!]
\centering
\includegraphics[width=0.9\textwidth]{Figures_Applications/fig_fleurquin_2013_systemic_4}
\caption{Comparison between the congested cluster size as a function of time measured from empirical data. Congested airports are defined as those with a average delay per departing flight of over $29$ minutes in intervals of one hour. The congested cluster is obtained from the largest connected component of the network formed by congested airports connected with direct flights during the day considered. The model has been tested with all the ingredients working or only with some of them to check their importance in delay propagation. (a)-(d) refer to variants of the model taking into account all or some of the factors. Figure from~\cite{fleurquin_2013_systemic}. \label{fig:fleurquin_2013_systemic_4}}
\end{figure}

There have been primarily two approaches considered to model such effects. One class of models probe the network structure to search for weak points, analyzing jamming and congestion phenomena, especially trying to isolate those airports that cause cascades of delays that span a finite-fraction of the network~\cite{lacasa_2009_jamming,wuellner_2010_resilience,ezaki_2014_potential,lordan_2014_study}. This approach has also been extended to multiplex networks of airports and airlines also attempting to identify carriers that are most responsible for delays~\cite{cardillo_2013_modeling}. 
In addition to these more-stylized approaches, there exist other families of models whose objectives are to provide realistic predictions that can be implemented in real-time. Such models are primarily agent-based and incorporate fine-grained details of air operations while sacrificing analytical tractability. Indeed, this has been the main approach adopted in the area of transport engineering related to Air Transport Management (ATM). There are hundreds of contributions on this line mainly presented in sectorial conferences such as the USA/Europe ATM Seminar, the Air Transport Research Society (ATRS) Conference, ICRAT, the SESAR Innovation Days, etc, whose proceedings are accessible online as well as publications in journals such as Journal of Air Transport Management (JATM). The advances in this area cover topics such as the optimization of runway management, the effects of bad weather, sector congestion, how the advent of drone technology affects commercial air transportation, among many other minute details. While a comprehensive listing of all the findings would require a review by itself, we mention a few that are directly related to  delay propagation in networks. Micro-modeling of the connection between air-transport logistics and delays was first explored in~\cite{schaefer_2001_flight,rosenberg_2002_stochastic}. Later a similar framework was used to assess the economic impact of a major disruption in a European hub \cite{janic_2005_modeling}. Also in Europe, within the umbrella of the WPE of the Joint SESAR Undertaking, several projects as NEWO, TREE  and POEM have studied the modeling of delay propagation from different perspectives: the first two from the network and airline managers' and the latter from the passenger's point of view. 

In the same context, data-driven models have been developed to reproduce the delay propagation patterns in the US \cite{pyrgiotis_2013_modelling}. These models were validated against empirical performance data \cite{campanelli_2015_modelling}, showing an acceptable level of accuracy and precision in the prediction of delayed flights, identification of airports displaying major departure-delays, network-wide congestion and generalized bad weather conditions~\cite{fleurquin_2013_systemic}. The models are built with an agent-based approach applied at the level of aircraft, with delays appearing at any point in the operations due to technical, airport organizational or weather-related issues incorporated as initial conditions to the model. The airport management is simulated with a system of queues that depend on the specific network considered (US or Europe). One of the primary finding of such analysis is that the mechanism with the strongest potential to effect delay is passenger and crew connections (\figurename~\ref{fig:fleurquin_2013_systemic_4}). 

\subsubsection{Sea Networks}

\begin{figure}[t!]
\centering
\includegraphics[width=\textwidth]{Figures_Applications/fig_kaluza_2010_complex_1}
\caption{(a) The global boat cargo network consisting of ports as nodes, and connections between them when boats navigate from one to the other. The links are colored according to the volume of traffic and their shape is constructed from the geodesic distance between ports. (b) A map of the top $50$ ports in terms of their betweenness centrality. Also listed are the top $20$. Figure from~\cite{kaluza_2010_complex}. \label{fig:kaluza_2010_complex_1}}
\end{figure}


Another example of long-range transportation are sea transportation networks, which have received comparatively less attention than their air traffic counterparts. While historically, sea transportation was the main mode of long-range human mobility, in recent times they mostly involve movement of cargo. Still, for sake of completeness we provide here a brief description on studies conducted on such networks. Typically such networks are studies as representing the ports as nodes, with links connecting them whenever major cargo liners travel from one to other~\cite{fremont_2007_global}. The links typically are given weights that correspond to the number of ships (trips) along the route. The resulting network spans the globe, connecting different continents as shown in~\figurename~\ref{fig:kaluza_2010_complex_1}. Similar to that seen in air transportation networks, statistical distributions of typical topological properties are heavy-tailed, with only a few number of ports accounting for a majority of global traffic, with the majority playing a peripheral role~\cite{kaluza_2010_complex}. The centrality of  ports was studied in \cite{hu_2009_empirical}, where it was found that Singapore, Antwerp, Bushan and Rotterdam top the list in terms of both traffic and load. A similar relationship between connectivity and traffic as in \equationname~\eqref{eq:wij_air} was noted with an exponent of $1.46 \pm 0.1$~\cite{kaluza_2010_complex}. Furthermore, the network can be segmented by the type of ship (containers, oil tankers, etc) in terms of a multiplex network framework \cite{ducruet_2013_network}.  A systematic analysis on the similarities between the world airport network and the boat network was performed in \cite{woolley-meza_2011_complexity}, including a study of the resilience of both networks. From a modeling perspective certain aspects of the flows can be reproduced with a gravity model (\sectionname~\ref{sec:gravity}), although there are some strong limitations. An important additional application of this network, besides the transport of goods, is the study of the invasion of species brought by carriers from different parts of the world \cite{keller_2010_linking,seebens_2013_risk}. 


\subsection{Multi-scale}

\subsubsection{Intra urban mobility}
% NB : previously named "Structure of cities"

One of the more important and much studied applications of mobility is that seen within cities. Indeed, there are good reasons for this. One is just the availability of more data. 
With rapid urbanization, an increasing part of the global population is living in urban areas.
Large cities are early adopters of new technologies have large populations and consequently more mobile phone users and high densities of 
of phone antennas.
Consequently the spatial resolution of CDR data (Sec.~\ref{sec:cdr}) collected in cities is order of magnitude higher when compared to rural/non-urban areas. 
%
The same considerations
hold for the data generated from social networking applications, such as
Twitter or Foursquare. Indeed, the usage of these applications has been shown to have socially and spatially 
biases; they are more likely to be used by urban residents, educated
individuals, young people and middle age employed
adults~\cite{adnan_2014_social}. 

The other of course is the unique set of challenges that abound from large agglomerations of people resulting in high population density. 
One of these is congestion that has an associated energy
cost as measured for example by C02 emissions. As a large part of human movement  occurs in individual
vehicles, accurate measurements of transportation flows are needed to infer how congestion affects carbon emissions. 
For example, it is known that CO2 emissions
mainly depend on the time spent traveling, and in places of high
density such as cities where congestion appears, traffic jams are
responsible for carbon emissions that
scale super-linearly with the population size \cite{louf_2013_modeling, louf_2014_how}. 
Yet another reason for studying intra urban mobility is related to that fact that cities
typically have  high levels of socioeconomic inequality among their residents, with residential segregation
that is amplified by other forms of segregation, perceived in mobility
practices and facilities that may differ depending on the neighborhood
of residence. Additionally, the multi-modal nature of transport networks
in urban areas (which include privately owned vehicles, buses, metros,
taxis, tramways, bike sharing systems, and pedestrians) gives
rise to specific questions related to the navigability of these
multi-layer networks. For example, the combination of the regularity of daily circadian
rhythms and smart transport card technology provides data
that allows for the study of intriguing phenomena, such as
\emph{familiar strangers}~\cite{sun_2013_understanding}, and can be used to uncover the
properties of ``hidden" temporal networks of the city, formed by
individuals that temporarily co-locate in the same spaces, but do not
know each other (see Fig. \ref{fig:sun_2013_understanding_2}). 
%


\begin{figure}[t!]
\centering
\includegraphics[width=0.7\textwidth]{Figures_Applications/fig_sun_2013_understanding_2}
\caption{Statistics of the co-location times for "familiar strangers". In (a), typical contact network. In (b), inter-encounter distribution times. In (c), joint probability distribution of inter-encounter times. In (d), distribution time between encounters for groups. Figure from~\cite{sun_2013_understanding}.}
\label{fig:sun_2013_understanding_2}
\end{figure}


%While, the study of movement in cities has a long tradition in urban
%geography, transport and urban economics. Despite this, the research referenced
%in this section comes from teams composed of physicists
%and computer scientists, that came to study mobility through complex
%systems and networks \cite{osullivan_2015_do}. 
%
%Here we discuss recent developments related to
%the aforementioned questions, and focus on those relying upon data
%produced by individual devices, location-based services, transport
%cards, credit cards, or GPS. 
%%
%Some previous efforts in this direction
%include \cite{jiang_2013_review}. It is remarkable that the mobility
%questions that have been studied in the recent years thanks to these
%data only partially overlap with those that have been long studied by
%transport scientists and geographers. Our impression is that the
%current situation in the field is somewhat similar to the one described
%in~\cite{hidalgo_2015_disconnected!} for the case of social networks: two different
%communities co-exist, and they produce bodies of
%literature that do not overlap much. To briefly summarize,
%physicists and computer scientists have been mainly interested in
%uncovering statistical properties and discovering hidden laws of urban
%mobility, whereas geographers, while interested by general and possibly
%universal properties, also intend to understand and explain the
%spatial determinants of local variations. Methods used in both
%communities are also different. To simplify one could say that an
%important fraction of quantitative geographers rely upon spatial
%analysis and multi-dimensional statistics, while many physicists have
%worked on mobility questions within the analytical and modeling
%framework of complex and spatial
%networks~\cite{barthelemy_2011_spatial}. 
%

There is an important body of mixed
literature, quantitative and qualitative, on intra-urban mobility
that relies upon longitudinal travel surveys, started more
than 60 years ago. Covering it is beyond the scope of this review, and
possible starting points for those interested may be found in
~\cite{chapin_1974_human, hanson_2004_geography, golledge_1997_spatial}, which
focus on North-American metropolises. Ref. \cite{ratti_2006_mobile} was one of the first papers to explicitly discuss
the possibilities offered by individual digital footprints for urban
analysis. It underlined that such data could be used to gain
understanding of a number of phenomena associated to intra-urban
mobility. Following this, a series of papers highlighted possibilities of real-time visualization and
monitoring of displacements in cities~\cite{calabrese_2006_real,
reades_2007_cellular, calabrese_2011_estimating}. 


The first studies that utilized metadata produced by handheld devices
carried by individuals focused on a single city, and essentially consisted of
visualizations and simple aggregated measures, providing information on the
density of individuals using a device in various areas of the city and at
various times of the day.  Metadata provided by mobile phone operators contains information on the nationality of the
phone number, which is also common on social networking
user profiles. Consequently one of the first uses of such databases of individual
geotagged data was to model the movements of foreign visitors in
cities. \cite{girardin_2008_digital} studied the location and mobility
patterns of tourists in Rome, and subsequent studies highlighted the
specific travel patterns of visitors, when compared to the movements
of city residents, notably in
Paris~\cite{olteanu_2012_le,fen-chong_2012_organisation}, Jakarta and
Singapore~\cite{chong_2015_not}.

\begin{figure}[t!]
\centering
\includegraphics[width=0.8\textwidth]{Figures_Applications/fig_isaacman_2010_tale_4}
\caption{Boxplot with the daily mobility ranges in Los Angeles and New York. Figure from~\cite{isaacman_2010_tale}. }
\label{fig:isaacman_2010_tale_4}
\end{figure}

Another popular case study are ones which
focus on the influence an area of residence has on intra-urban
mobility patterns. Most of the first intra-urban studies using ICT
data in these categories have focused on the organization of mobility
in one or two large cities. One of the very first in this category was
~\cite{isaacman_2010_tale}, which used mobile phone CDR data recorded
during a period $D$ of 62 consecutive days in New York and Los
Angeles. They compared aggregated mobility statistics; the
individuals' mean daily range -- defined for each individual as the mean
of the maximal distance traveled each day, $\sum_{d \in
D}{max_{ij}(d(i,j))}/|D|$ -- and the individuals' maximal daily range
-- defined as $max_D(max_{ij}(d(i,j)))$ (see Fig. \ref{fig:isaacman_2010_tale_4}). They showed that residents of
Los Angeles globally commute over further distances, but also
highlighted heterogeneities; some residents in Manhattan commute
farther than their counterparts in Los Angeles and
the neighborhood of residence has an influence on the average journey-to-work commuting
distance. 
%
Combining mobile phone data and odometer data (a distance measuring device for bikes and automobiles),
\cite{calabrese_2013_understanding} performed similar comparative
measures between the two data sources, highlighting the influence of a
number of built environmental factors on the travel distances
of Boston residents. \cite{desu_2016_effects} went further by
combining mobile phone data and census data which provided information on the
proportion of racial groups in each of the cities' census tracks. They
showed that residential segregation was amplified by a mobility
segregation in the urban areas of Boston and Los Angeles. 
%

Focusing on
a single city, some authors have investigated the relation between
morphology and mobility patterns: \cite{roth_2011_structure} used smart cards
data collected over several weeks in the London subway to provide
measures linking the polycentric spatial structure of London and the
organization of mobility patterns in the city. Relying upon mobile phone data collected during 2 months in
31 Spanish urban areas, \cite{louail_2014_mobile} showed how the spatial
organization of hotspots and the shape of cities evolve during the
course of a typical weekday, as residents move inside the cities,
highlighting different typical city forms (see Figure \ref{fig:louail_2014_mobile_7}). Thanks to a large
Foursquare dataset that records GPS positioning data,
\cite{noulas_2012_tale} compared the distribution of 
jump lengths in 34 different cities worldwide. Fitting the
distribution in each city, they conclude that at the city scale, human
displacements cannot be properly fitted by a power-law distribution, a
result in agreement with several other case studies held for
particular cities, which found that the distribution of travel
distances inside an urban area is more properly fitted by an
exponential distribution $P(\Delta r) \propto exp(-\Delta
r/r_0)$. They propose an individual model derived from Stouffer's
intervening opportunities, that states that the important parameter
for explaining destination choice in cities is not the distance by
itself, but rather the density of possible alternative destinations in
a perimeter of distance $d$. Given a set $U$ of places in a city, the
probability of moving from place $u \in U$ to a place $v \in U$ is
formally defined as
\begin{equation}
P_r[u\rightarrow w] \propto \frac{1}{rank_u(v)^a},
\end{equation}
where $rank_u(v) \propto~\mid\{w:d(u,w)<d(u,v)\}\mid $, which can be seen as an urban counterpart to the radiation model
proposed in the same year by \cite{simini_2012_universal} (see section
\ref{sec:models}).

\begin{figure}[t!]
\centering
\includegraphics[width=0.8\textwidth]{Figures_Applications/fig_louail_2014_mobile_7}
\caption{Scatter plot with the number of hotspots detected as a function of the population of the cities. The colors of the curves and the points correspond to two different ways of defining the threshold marking an activity cell as hotspot. Figure from~\cite{louail_2014_mobile}. }
\label{fig:louail_2014_mobile_7}
\end{figure}

One of the most debated questions in the recent literature on human
mobility has been the shape of the statistical distributions of 
the distances traveled by individuals $P(\Delta r)$, and their
travel times at various spatial and temporal scales and in different
environments. Providing a neat characterization of these
distributions is of extreme importance, and a prerequisite to
modeling. Indeed, the distribution of travel distances $P(\Delta r)$
is a key ingredient that any reasonable model should be able to
reproduce and explain. \cite{gallotti_2015_stochastic} contains a very clear
recap of the efforts of the last decade on this question, and a table
summarizing the parameters values of the fitting distributions that have
been proposed. These distributions have been characterized thanks to
various sources of individual data (dollar bills, mobile phones,
private car GPS, taxis, longitudinal travel surveys, and geolocated
tweets, among others). In most cases, the fit of $P(\Delta r)$ is
consistent with the general form of a truncated power law distribution
\begin{equation}
P(\Delta r) \propto (\Delta r + r_0)^{-\beta} \exp\left(-\frac{\Delta r}{\kappa}\right),
\end{equation}
with $\kappa = 0$ corresponding to the case of a non-truncated
power-law, and $\beta = 0$ corresponding to an exponential
distribution of scale $\kappa$. The majority of papers propose truncated power law fits for $P(\Delta r)$, and
argue that the distribution of displacements follows an underlying L\'evy-flight process.

%Even if we put aside the consideration that, given the shape of
%the inter-event time distribution, CDR data and dollar
%bills might in the end offer poor sampling of the effective
%underlying trajectories, one should note that these studies, by mixing
%trips of all different kinds and at various spatial and temporal
%scales, did not intend to provide a proper characterization of the
%specific properties of human mobility inside a city.

The studies compared in \cite{gallotti_2015_stochastic} provided very different
results for the parameters values $\beta$, $r_0$ and $\kappa$ of the
general form above, and consequently they proposed different
underlying processes for explaining the shape of this
distribution. To settle the discrepancies, a convincing modeling
framework, able to reproduce a number of stylized facts, and $P(\Delta
r)$ is required, and more research
and cross-checks should be performed. Data accessibility is always an issue in this type of research, so one workaround passes is repeating the same measures on many different datasets, to enable a systematic comparison of the
empirical results as done by~\cite{gallotti_2015_stochastic}.


Apart from this, some of the many aspects studied on intra-urban mobility relate to the typical distance scales of movement ~\cite{bazzani_2010_statistical,gallotti_2015_understanding}; measures of travel duration~\cite{gallotti_2015_understanding}; effects of population size on
the structure and dynamics of mobility~\cite{louf_2014_how,
louail_2015_uncovering}; the effect of city topology on the displacements of its
inhabitants (in terms of travel time, distance, and spatial
organization of flows)~\cite{kang_2012_intra} optimal `strategies' for navigating
multimodal transport networks~\cite{gallotti_2014_anatomy}, the efficiency of transport
itineraries~\cite{lima_2016_understanding}; the level of
complexity of transportation maps, which might exceed cognitive
limits, and therefore explain sub optimal navigation by city residents~\cite{gallotti_2016_lost}.


\subsubsection{Epidemic Spreading}

Perhaps one of the most important applications of studies on human mobility and transportation systems relates to  epidemic spreading. While epidemiology as a discipline has several hundred years of tradition, only recently has it been possible to model the global spreading patterns of diseases in a realistic and reasonably accurate way. Of course, every disease has its own peculiarities, and requires tailored modeling with the inclusion of the specific mechanisms associated with it. Consequently, the literature in this area is extensive, with dedicated monographs and a number of reviews ~\cite{diekmann_2000_mathematical,keeling_2008_modeling,tatem_2014_mapping,pastor-satorras_2015_epidemic}. Here we discuss how recent knowledge of human mobility has influenced the latest generation of epidemiological models.

\begin{figure}
\centering
\includegraphics[width=\textwidth]{Figures_Applications/fig_brockmann_2013_hidden_2}
\caption{In (a), Shortest (most-likely) disease propagation pathways for an epidemic starting in Hong Kong. $D_{eff}$ stands for the "radial distance from the disease origin as defined in \cite{brockmann_2013_hidden}. In (b), time evolution of a simulated pandemic starting in Hong Kong. In (c), epidemic arrival time as a function of $D_{eff}$ in the simulation. In (d) and (e), same analysis but with data for the 2009 H1N1 flu pandemic and for the 2003 SARS outbreak.  Figure from~\cite{brockmann_2013_hidden}. }
\label{fig:brockmann_2013_hidden_2}
\end{figure}


Whilst it has been known for some time that air transportation plays a major role in the global spreading of diseases \cite{rvachev_1985_mathematical,flahault_1992_method}, the implementation of air traffic data in epidemic models has only been possible in recent times with the advances in information technologies and computing resources over the past decade or so~\cite{hufnagel_2004_forecast,grais_2003_assessing,colizza_2006_role},
%The effect of the short-range mobility in the propagation of influenza in an urban context has been also explored in \cite{eubank_2004_modelling}. 
%
with the majority of studies focusing on the spread of influenza. The flu is a paradigmatic example of contact diseases that spreads annually cross the globe, and for which contagion is airborne with a patient clinical process that allows for relatively simple modeling. Indeed, improved surveillance systems have brought new insights on the temporal and spatial patterns of its propagation \cite{viboud_2006_synchrony,tizzoni_2012_real}. 

Of course, new modeling frameworks are not restricted to the flu but can be also extended to other contact diseases such as SARS \cite{bauch_2005_dynamically}, mosquito-driven infections lile malaria \cite{huang_2013_global} or yellow fever \cite{johansson_2012_crowd}, or even Ebola and other emerging health threats \cite{jones_2008_global,gomes_2014_assessing}. In many such cases, the first instance of the disease in a particular location occurs through the arrival of infected humans from the source location via air transportation. The world airport network (WAN) can thus be considered as the backbone through which new emerging diseases arrive where connections with larger flows of passengers marking the most likely pathways for propagation \cite{brockmann_2013_hidden} (see Fig. \ref{fig:brockmann_2013_hidden_2}). Indeed, within the framework of the WAN (as discussed in Sec.~\ref{sec:air}), it is possible to estimate the potential of every airport as a source for a pandemic \cite{lawyer_2016_measuring}. 

\begin{figure}[t!]
\centering
\includegraphics[width=\textwidth]{Figures_Applications/fig_ferguson_2005_strategies_2}
\caption{In (a), time evolution of a flu epidemic with $R_0 = 1.5$ in the Southeast Asia. Infectious individuals are shown in red, while green ones are those already recovered or removed. In (b), daily incidence of the simulation on average in dark blue, several realizations in different colors and in gray the 95\% confidence interval. In (c), distance from the origin of the disease. In (d), the proportion of infected people by age group. In e, distribution of number of secondary cases produced per infectious individual in the early stages of the outbreak. Figure from~\cite{ferguson_2005_strategies}. }
\label{fig:ferguson_2005_strategies_2}
\end{figure}

Concerning realistic models, two main frameworks have been used in the literature: agent-based modeling and metapopulations. The difference between the two lies in the level of information needed. Agent-based models attempt to generate synthetic population mimicking at an individual (agent) level the realistics aspects related to disease propagation. A detailed knowledge is therefore required of features such as location of residential areas, businesses and schools, leisure activities, population divisions by  age, gender and sizes of households as well as individual mobility. These are typically sourced from census-like surveys and then a synthetic population is created assigning each individual to a household, a work/school place and a characteristic mobility profile. The earliest variants of such models were developed in the 1990's within the Los Alamos project \emph{Transportation Analysis Simulation System} or TRANSIMS. The objective of TRANSIMS was to reproduce via agent-based modeling the daily life of Portland, Oregon, including population, mobility, income levels and other details sourced from the US census \cite{simon_1999_simple}. This was later adapted to study the evolution of epidemics, called \emph{Epidemic Disease Simulation System} or EPISIMS, which was used to study different scenarios to prevent the propagation of smallpox \cite{eubank_2004_modelling}. Similar models were applied to study the propagation of a strain of avian flu, H5N1, in South East Asia, where for instance, epidemic spreading was simulated with the source centered in Thailand (see Fig. \ref{fig:ferguson_2005_strategies_2}), and including neighboring countries at a range of 100 kms containing about $80$ million inhabitants (agents)~\cite{ferguson_2005_strategies,longini_2005_containing}. This was primarily done to assess the effect of several intervention measures (distribution of antivirals and reducing the social contacts) to halt the progress of the disease. Similar models have been applied to even larger populations in the US and UK~\cite{ferguson_2006_strategies,germann_2006_mitigation,halloran_2008_modeling,ajelli_2008_impact}. The level of granularity present in such models, enables minute analysis  such as the effect of closing specific schools, employing different vaccination protocols (say by age) on disease propagation~\cite{germann_2006_mitigation,halloran_2008_modeling}. The arrival of the 2009 flu pandemic provided further impetus to such modeling approaches and agent-based models have been now developed for most of Western European countries \cite{ajelli_2010_comparing,ciofi_2008_mitigation,merler_2009_role,merler_2011_determinants} where a dazzling amount of scenarios have been exhaustively simulated including the risk of spread of a newly created virus strain from a lab following the case of the creation of a new H5N1 influenza strain \cite{merler_2013_containing}. One must note that the mobility aspect in such models relates to the movement of the individuals (agents) and is explicitly implemented both at long and short ranges involving multiple models of (simulated) transportation.

While agent-based models provide the best approximation to reality, they are severely limited by the fact that they require highly detailed input information, the data for which in many cases is noisy, unreliable or just simply unavailable. Furthermore, given the sheer complexity of such models, an analytical treatment is nearly impossible. On the other hand, the metapopulation framework,  balances granularity with analytic tractability, and has easier to obtain data as input parameters~\cite{rvachev_1985_mathematical,sattenspiel_1995_structured,colizza_2007_invasion}. In this model, the population of interest is divided into subpopulations, within which contacts are modeled by a fully-mixed mean-field approach, whereas contacts between subpopulations occur through the (measured) mobility networks. One of the earliest such models is the so-called \emph{Global Epidemic and Mobility model} or GLEaM~\cite{balcan_2009_seasonal}. GLEaM is a global simulation platform that divides the globe in cells of $15\times15$ minutes of arc, which are grouped in basins of attraction around the major transportation hubs, which in most cases are airports. The  mobility is then simulated at two levels: long range air transport, which is calibrated from the world airport network, and short-range mobility obtained from either census commuting data or estimated using the gravity law~\cite{balcan_2010_modeling}. Figure \ref{fig:balcan_2009_multiscale_1} displays the different components of the model.

\begin{figure}[t!]
\centering
\includegraphics[width=\textwidth]{Figures_Applications/fig_balcan_2009_multiscale_1}
\caption{In (a), air transport network zoomed in for the US. In (b), commuting network in the same area. In both cases, the link weights are plotted using a heatmap code. In (c)-(f), comparison between flows coming from commuting data $w^D$ and those produced with a gravity model $w^M$ with the aim of extrapolating short-range mobility in zones where the data was not available. Figure from~\cite{balcan_2009_multiscale}. }
\label{fig:balcan_2009_multiscale_1}
\end{figure}

Most notably, GLEaM was used to study the propagation patterns of the 2009 H1N1 influenza pandemic in \emph{real time}. The pandemic started in a small population of Mexico early in that year \cite{fraser_2009_pandemic}, and its unfolding received attention of mass media media as well as the scientific community, which brought unprecedented details on the characteristics of the first imported cases in several countries. This information allowed for the calibration of the infectivity and the seasonality in GLEaM in early May 2009 \cite{balcan_2009_seasonal}. Once calibrated, the model predicted the disease propagation patterns: essentially, the arrival and prevalence peak times for the rest of the countries of the world, predictions that were later validated, after the full surveillance reports were available in 2012 \cite{tizzoni_2012_real}. Additionally, the model helped estimate the actual initial number of cases present in Mexico in the early stages of the disease~\cite{colizza_2009_estimate}. Ever since, the model has become an important tool for subsequent studies that takes its results as a baseline (see, for example~\cite{brockmann_2013_hidden,lawyer_2016_measuring}) and has been also used to assess the risk of spreading of new emerging health threats~\cite{goncalves_2013_human,gomes_2014_assessing,poletto_2014_assessing}.

The flexibility of the multilevel framework inherent in the model allows for the quantitative assessment of the differences between short-range and long-range traveling on the disease propagation. In particular it was determined that travel restrictions via decrease of the air traffic (or temporary quarantines) can at best delay the disease spread by a few days or weeks. Unless one totally shuts down the network, or severely restricts it (which of course carries a massive economic cost) global propagation is almost guaranteed~\cite{bajardi_2011_human,poletto_2014_assessing}. Newer versions of such models factor in facets of human behavior, such as the tendency of people to restrict contacts and travel when made aware of the existence of a dangerous disease~\cite{meloni_2011_modeling}, as well as heterogeneity in the population (age, gender, socioeconomic indicators) and its effect on travel patterns~\cite{apolloni_2014_metapopulation}. Possible future directions regarding these models include the possibility of feeding them with real-time ICT collected data on mobility \cite{jia_2012_empirical,wesolowski_2014_quantifying,tizzoni_2014_use,lenormand_2015_human} or hybrid frameworks that merge aspects of agent-based and metapopulation modeling~\cite{ajelli_2010_comparing}. 

\subsection{Virtual-Scale}

\subsubsection{Web (Online) Mobility}

Ostensibly, physical human mobility and virtual navigation are indeed activities of very distinct nature, yet there are a number of reasons why the latter can be cast in the context of the former. Indeed, many daily activities that occurred in specific physical locations, such as work, study, leisure, financial transactions, purchasing, are now increasingly migrating towards being conducted online. Moreover, the world wide web (www) has a humanitarian and social role, in the sense that it helps individuals and populations with limited mobility to have access (albeit virtually) to locations, people and information hitherto unreachable. For instance, the disabled \cite{harper_1999_towel,goble_2000_travails,ritchie_2003_promise} and  the elderly \cite{nimrod_2010_seniors,cotten_2013_impact,boxie_2007_using}, at an individual level, for whom the www represents a means of overcoming their physical limitations, or even entire populations, whose mobility is restricted for political or geographical reasons, for which the www is their only access to the outside world \cite{aouragh_2011_confined}. In this section we present an overview of the recent empirical evidence on the statistical similarities between virtual and physical navigation.



\begin{figure}[t!]
\centering
\includegraphics[width=0.9\linewidth]{Figures_Applications/szell_2012_understanding_fig5}
\caption{Mobility activities within the \emph{Pardus} MMOG. The Mean Squared Displacement (main panel) suggests a subdiffusive process with exponent $\nu=0.26$,  whereas the return probabilities in number of discrete jumps $\tau$ (inset) behave as a power law with exponent $\alpha \approx 1.3$. (adapted from Szell \et \cite{szell_2012_understanding}) }
\label{fig:szell2012understandingfig5}
\end{figure}



In the context of this parallel between virtual and physical mobility, we differentiate two classes of activities that although related, have practical and functional distinctions, that is, mobility in virtual spaces (for example, World of Warcraft, Second Life and The Sims)  and navigation  in virtual \emph{spaceless} environments (for example, browsing the WWW or navigating through the interface of a device). The difference between these two classes of activities is given by the fact that in the first there is an explicit notion of space such that  displacements between different locations come with a temporal cost, often proportional to some  distance measure between them \cite{shen_2013_human}. 


An empirical analysis of the behaviors and trajectories in the virtual space of the online game \emph{Pardus} suggests a subdiffuse behavior in online mobility \cite{szell_2012_understanding}, similar to that observed in physical mobility \cite{song_2010_modelling,gonzalez_2008_understanding} ( \figurename \ref{fig:szell2012understandingfig5}). Additionally, it appears that the return time (in number of discrete steps) of users of the Pardus game follows a power-law distribution with an exponential cut-off (inset of \figurename \ref{fig:szell2012understandingfig5}).
These similarities between virtual and physical mobility were later empirically corroborated by other similar datasets. A comparative analyses between traces of World of Warcraft (WoW) players and GPS traces~\cite{shen_2014_characterization} revealed that the distributions of  jump lengths and waiting times are both approximated by lognormal distributions (\figurename \ref{fig:shen2014characterizationfig2}). 

\begin{figure}[t!]
\centering
\includegraphics[width=0.9\linewidth]{Figures_Applications/shen_2014_characterization_fig2}\\
\includegraphics[width=0.9\linewidth]{Figures_Applications/shen_2014_characterization_fig4}

\caption{Jump lengths (top) and waiting-times (bottom) distributions within the virtual environment of World of Warcraft (left) and physical mobility from GPS traces (right) (adapted from~\cite{shen_2013_human})}
\label{fig:shen2014characterizationfig2}
\end{figure}

Furthermore, the analyses of CDRs along with the Web activities of 20,000 mobile phone users~\cite{zhao_2014_scaling} suggest that visitation frequencies in both activities have a stretched exponential distribution (see \figurename \ref{fig:zhao2014scalingfig2}). 
However, Web traces from mobile phone users tend to overrepresent a particular set of activities such as fact finding, information gathering and communication \cite{cui_2008_exploring}. More recently, the analyses of complete Web browsing activities of 582 users, has revealed that it is possible to profile Web users based on their \emph{exploratory vs. exploitative} behaviors in their browsing patterns~\cite{barbosa_2016_returners}. More surprisingly, however, is the fact that the distribution of \emph{returners} and \emph{explorers} profiles are remarkably similar to the one observed in human mobility~\cite{pappalardo_2015_returners}.

The observed similarities between physical and virtual navigation is beginning to draw increased attention from the scientific community. Of course the analogy can oly be taken so far, as compared to physical movement, virtual activity is relatively recent (on an evolutionary time scale) and much remains to be understood. It is reasonable, however, to believe that online activity is driven by the same psychological mechanism as \emph{off-line} behaviors, now adapted to new environments or contexts. In fact, such intuition has increasingly found scientific support, both from a theoretical and empirical point of view, such as evidence of foraging behaviors in online shopping~\cite{hantula_2008_online} and information consumption on the Web~\cite{pirolli_1999_information,stenstrom_2008_online}, similar to that observed in human resource acquisition behaviors~\cite{stephens_2007_foraging}.

\begin{figure}[t!]
\centering
\includegraphics[width=0.9\linewidth]{Figures_Applications/zhao_2014_scaling_fig2}
\caption{Visitation frequencies distribution in online (a) and physical (b) spaces (adapted from Zhao \et \cite{zhao_2014_scaling}). }
\label{fig:zhao2014scalingfig2}
\end{figure}


