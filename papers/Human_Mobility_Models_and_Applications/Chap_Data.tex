%\FS

%The study of human mobility is fundamentally an empirically-driven science with much of the metrics and models being inspired by observation from data. 
A natural starting point is to describe the nature of empirical data which has been used in mobility research. Indeed, empirical data has been vital to both aggregate and individual mobility investigation, by providing means of parameter calibration as well as model validation. In this section, we outline the main sources available for mobility research and the relevant information that can be extracted from them. The data sources are presented in (rough) chronological order of their historical availability and consequent use in research.

\subsection{Census Data \& Surveys}
\label{sec:census}

Census data is collected in periodical national surveys in which householders are asked questions about the socio-demographic and economic status of the household members. Of particular interest in terms of mobility are questions related to the location of the workplace, or the place of current and previous residence. Collectively, this information can then be used to estimate commuting flows or internal migration flows within a country.

Different countries introduced national censuses at different times, and the type of information collected also varies across countries and time periods. In the United Kingdom, for example, the first national census was done in 1841 and contained the names, ages, sexes, occupations, and places of birth of each individual living in a household. In 1921, the location of workplaces was added to the questions asked, and since 1961 the location of previous residence was also included. In the United States, the first population census was taken in 1790, while information on workplace and transportation activities was incorporated in 1940~\cite{potter_2010_new}. Nowadays, national censuses are held in most countries typically every ten years. Below we outline some main sources of census and survey data along with the methodology used to extract commuting and migration flows.

\subsubsection{Census Data}

The data on commuting trips between United States' counties is available from the United States Census Bureau\footnote{~\url{http://www.census.gov/hhes/commuting/data/commutingflows.html}}. These county-to-county work flow files are available for each state as well as at the aggregated level across the country. Furthermore, files are available in either county of residence format---containing work destinations of people who reside in each county---or county of work format---which contain the county of residence of people working in each county. The inclusion of a commuter in the data is preconditioned upon them working during the week leading up to the census. Due to the questionnaire design, any individual not satisfying this condition is precluded from entering a workplace location. A worker is defined as someone who has spent at least 1 hour in paid employment or 15 hours in a voluntary role. Any individual required to provide their workplace location was asked further questions about their mode of transport and travel time.

From this information, the census Bureau compiles and publishes aggregated flows corresponding to the residence and workplace location of everyone classed as a worker. In order to validate models and determine statistics such as trip distance and relationship to the origin, destination and intervening population density, further information on the spatial distribution of counties and socio-demographic data is required. This supporting data is available online\footnote{~\url{ftp://ftp2.census.gov/geo/tiger/tiger2k/}}. 
Among other things, such census data is invaluable for validating generative models of commuter flow and migration. For instance, Census data from the 2000s (consisting of county level work flow data, corresponding to 3,141 counties and 34,116,820 commuters within the United States) was used to evaluate the so-called radiation model (described in sec.~\ref{sec:Intervening}) by Simini \et\cite{simini_2012_universal}

\begin{figure}[t!]
\centering
\includegraphics[width=\textwidth]{Figures_Data/Census_example}
\caption{Commuting flows compiled from census data. Left panel: The state of Florida partitioned according to its counties. Right panel: Commuting flows between counties, where thickness of lines correspond to volume of flow. Data compiled from the United States Census Bureau.}
\label{fig:Census_example}
\end{figure}

Commuting and migration flows data for the United Kingdom is also available online\footnote{~\url{http://webarchive.nationalarchives.gov.uk/20160105160709/}\\ \url{http://www.ons.gov.uk/ons/guide-method/census/2011/census-data/2011-census-data-catalogue/origin-destination/index.html}}. The information is available at different spatial resolutions: ranging from the geographical area covered by a single postcode, to areas covered by the 7,201 electoral wards. Files contain origin and destination flows divided into subsets of age and gender allowing for a deeper socio-demographic analysis if required. There have been many different countries whose commuting data has been analyzed at the level of municipalities (or smaller administrative units such as zipcodes or electoral tracks) including France\footnote{\url{https://www.insee.fr/fr/statistiques/2022117}} and most European countries. In most cases, when not directly available on an open data portal, commuting data is available to researchers on request from national statistics institutes.

%\RM {\bf We have to standardize the use of ``data are'' or "data is". I've been using the singular version.}

\subsubsection{Tax Revenue Data}

An alternative source for estimating aggregated migration flows is the Statistics of Income Division (SOI) of the Internal Revenue Service (IRS) in the United States. The IRS maintains records of all individual income tax forms filed in each year. Using individual tax return files, it is possible to determine who has, or has not, moved residence or workplace locations in the intervening fiscal year. To do this, first, coded returns for the current filing year are matched to coded returns filed during the prior year. The mailing addresses on the two returns are then compared to one another. If the two are identical, the return is labeled a non-migrant. Any relevant information change during the fiscal year results in the return labeled as that of a migrant. US migration data from 1992-1993 to 2006-2007 is available online\footnote{~\url{http://www.irs.gov/uac/SOI-Tax-Stats-Migration-Data}}. Alongside tax returns, estimated migrations flows are also included in each Census\footnote{~\url{https://www.census.gov/hhes/migration/data/acs/county-to-county.html}}, based on yearly surveys carried out by the American Community Survey (ACS).

\subsubsection{Local Travel Surveys}

An important source of data used to construct trips, and therefore flows between two locations, is local surveys. It provides an advantage over censuses as it is possible to collect a more detailed data set that includes information such as the purpose of the trip and mode of transport used. However, the increase in accuracy, resolution, and variety of meta-data leads to a corresponding sacrifice in scale. As such, surveys typically have a much lower number of respondents compared to a census. Additionally, these typically cover smaller areas such as a single city (or indeed neighborhoods within cities), as opposed to the level of a state, or an entire country.

%\COMMENT{Here: add a paragraph giving important dates of the history of longitudinal transport surveys, first papers inspecting the variability vs. repetitive aspects of daily travels.}

A representative example is provided by the household Travel Tracker Survey for the Chicago metropolitan area, conducted between 2007 and 2008, and carried out by the Chicago Metropolitan Agency for Planning\footnote{~\url{http://www.cmap.illinois.gov/data/transportation/travel-tracker-survey}}. This survey contains information on travel activities of household members, among other details. Information was collected from a total of 10,552 households over a one or two-day period, and households were required to provide a detailed travel inventory for each member, including details such as trip purpose, transport mode, departure and arrival times, and public transport information such as boarding location, distance to final destination, and fare paid. 
%
A similar survey is available for the Los Angeles region\footnote{~\url{http://www.scag.ca.gov/travelsurvey}}
and was analyzed by Liang \et\cite{liang_2013_unraveling}, where 46,000 trips between 2,017 zones within the Los Angeles county were extracted.
The use of small scale surveys is often augmented by other forms of data (for example GPS tracks, see \sectionname~\ref{sec:gps}) in order to have accurate records of an individual's position combined with an annotated description of the purposes of each trip~\cite{liang_2013_unraveling, schneider_2013_unravelling}.

While historically most prevalent, census data provides us only with coarse-grained patterns of human migration and movement. Surveys, on the other hand, provide far more detail but on a much smaller scale, thus limiting their use in terms of statistical validation of models. While the census reveals little information about where people spend their time, or where and how they travel, at the survey level, data reliability is restricted by scale and self-reporting errors. Thus both sources lack the ability to provide a dynamic picture of human mobility~\cite{palmer_2013_new}.


\subsection{Dollar Bills}
\label{sec:dollar}

\begin{figure}[t!]
\centering
\includegraphics[width=0.9\textwidth]{Figures_Data/2006_Brockmann_Nature_1ab}
\caption{Trajectories of bank notes originating from four different locations. Tags indicate initial, symbols secondary report locations. Lines represent short time trajectories with traveling time $T < 14$ days. Lines are omitted for the long time trajectories (initial
entry: Omaha) with $T > 100$ days. The inset depicts a close-up of the New York area. Pie charts
indicate the relative number of secondary reports coarsely sorted by distance. The fractions of secondary
reports that occurred at the initial entry location (dark), at short ($0 < r < 50$km), intermediate ($50 < r < 800$km) and long ($r > 800$km) distances are ordered by increasing brightness of hue. The
total number of initial entries are $N = 2,055$ (Omaha), $N = 524$ (Seattle), $N = 231$ (New York), $N = 381$ (Jacksonville). Figure from~\cite{brockmann_2006_scaling}.}
\label{fig:2006_Brockmann_Nature_1ab}
\end{figure}

An unusual (and perhaps not immediately apparent) source of mobility data is related to the tracking of currency. An archetypal example of this are online bill-tracking websites, which are designed to monitor the dispersal of individual bank notes worldwide. The movement of a bank note between geographical locations occurs when it is carried by an individual, therefore data collected from such websites may be used to infer  trajectories and patterns of human mobility. Brockmann \emph{et al.}~\cite{brockmann_2006_scaling} obtained trajectories of 464,670 dollar bills from the website \url{www.wheresgeorge.com}, allowing for an analysis of bank note dispersal in the United States (excluding Hawaii and Alaska). The data set contains entries corresponding to time-stamped reports of the geographic location of individual bills, numbering around 250 million as of 2017.  All bills registered on the site are initially entered by a user which involves the input of their ZIP code and a serial number printed on the bill. While the majority of the bills are not entered more than once, around 11\% of the bills are reported multiple times, with 3-5 entries per bill being fairly common (Any more than 5 entries is considered rare). Once a bill has been registered, any future hits allow for time and distance between reports to be determined and recorded in the database (see \figurename~\ref{fig:2006_Brockmann_Nature_1ab}). 

While a convenient and large-scale source of mobility data, the use of currency to infer patterns in human mobility is problematic. For instance, bank note dispersals do not contain information on the number of individuals that have carried a given note during two instances of measurement (i.e., when it appears in the database). Consequently, the trajectories of bank notes are likely a convolution of the mobility of several individuals; coupled with the relatively few samples per dollar bill,  this makes it a rather inaccurate and  problematic source for inferring individual mobility patterns. Nevertheless, similar to census data, currency tracking provides a convenient, and relatively easily accessible picture of the coarse-grained patterns of human mobility. 


\subsection{Mobile Phone Records}
\label{sec:cdr}

Perhaps the most important, game-changing data of the last decade for inferring human mobility are the mobile phone Call Detail Records (CDRs). Most cell phone companies maintain detailed records of customer information related to their calls and text messages. The information contained in these CDRs include the time of the call/message as well as the company's cell tower routing the communication. Knowing the exact geographic coordinates of the cell tower makes it possible to estimate the approximate location of the user. As mobile phones are (typically) personal devices and are mostly carried by a single person, the corresponding location data can be used to infer a single individual's movements over the recorded period (an example is shown in \figurename~\ref{fig:2010_Song_Science_fig1a}). Unlike the census, surveys, or currency tracking, CDRs allow for the characterization of  \emph{individual mobility patterns}, and -- depending on the dataset -- at an unprecedented spatio-temporal resolution. Furthermore, due to the tremendous trend of global cell-phone adoption, one can conduct multi-scale studies ranging from the level of neighborhoods to the country-wide level (including international travel and movement across borders).

\begin{figure}[t!]
\centering
\includegraphics[width=0.9\textwidth]{Figures_Data/2010_Song_Science_fig1a}
\caption{Trajectories of two anonymized mobile phone users traveling in the vicinity of $N = 22$ and $76$ different cell phone towers during a 3-month-long observational period. Each dot corresponds to a mobile phone tower, and each time a user makes a call, the closest tower that routes the call is recorded, pinpointing the users approximate location. The gray lines represent the Voronoi lattice, approximating each towers area of reception. The colored lines represent the recorded movement of the user between the towers. Figure from~\cite{Song_2010_Limits}.}
\label{fig:2010_Song_Science_fig1a}
\end{figure}

CDR data concern both phone calls and SMS exchange and always include the following information: time stamp, caller ID, recipient ID, call duration and antenna (cellular tower) code. Due to privacy concerns, customer identifiers are anonymized before the data is made available to scientists for analysis. Quite often, additional clean up of the data is required to make the data more analyzable; for instance, one typically needs to restrict the dataset to users who visit a minimum threshold number of towers, as well as make calls at a high enough frequency to maintain statistical reliability for each user.

One of the first work using CDR's was performed by Gonzal\'ez \et\cite{gonzalez_2008_understanding} on two anonymized datasets provided by a major mobile operator in a large European country. The first set contained the locations of 100,000 randomly selected mobile users whose positions were recorded over a period of 6 months, each time a call or text was received or sent. The second set (collected for validation purposes) corresponded to the locations of 206 mobile users, recorded every two hours over a period of a week. From this data, traces of mobility patterns such as the distribution of displacements between the locations of consecutive calls (jump-lengths) were measured. The temporal period between consecutive calls was also measured to determine a distribution of wait times (i.e., characteristic time spent by users in a given location). The knowledge of users' consecutive locations also enabled the calculation of so-called return time distributions, measuring the probability of a user to return to a given location within a given time. Other more complex metrics, including the radius of gyration were also extracted. (For details on metrics see Sec.~\ref{sec:metrics}). 

The quality of mobility information extracted from mobile phone data depends on the frequency at which the location of the user is recorded. For example, the larger data set used in~\cite{gonzalez_2008_understanding} displayed an irregular call pattern; many calls were placed over a short time period, followed by long periods of inactivity. Consequently, the data displayed a highly inconsistent temporal frequency, which may confound the analysis of mobility patterns. The smaller-scale validation dataset was therefore collected to account for these irregularities. Other problematic issues also arise, such as the accuracy of recorded locations. In general, it is the position of the cell tower routing the call that is recorded and not the exact location of the individual carrying the phone. Since there is significant variability in the areas covered by cell towers---with coverage ranging from tens of meters in the densest neighborhoods of urban areas, to a few kilometers in rural areas---a recorded user in a rural area may transmit all communication via a single tower during their daily routine, while moving the exact same area as that of an urban user who may ping several towers. Thus, in the former case, despite possibly significant movement on part of the user, for purposes of analysis they are considered stationary, which is obviously misleading.

We must note that obtaining CDRs for research purposes poses significant challenges. Since they are not typically publicly available, one has to directly approach a mobile phone operator. As the data contains sensitive information~\cite{de_montjoye_2013_unique}, any data collected by a specific group of researchers may not be shared with other groups, making reproducibility problematic. Yet, recent initiatives have seen some phone companies release large scale CDR data within the context of "challenges" among researchers. The purpose of making this data available is to provide researchers some material allowing them to extract useful information to address major societal challenges. Such recent datasets are those that have been provided by Orange in the framework of their Data 4 Development (D4D) challenge. These contain four anonymized tables for mobile phone users in the Ivory Coast (C{\^o}te d'Ivoire) and Senegal. For example, in the case of Ivory Coast, data contained (i) hourly antenna-to-antenna traffic; (ii) individual trajectories of 50,000 users over two weeks with antenna location information; (iii) individual trajectories of 500,000 users over an entire year with sub-prefecture location information; (iv) communication graphs for 5,000 users~\cite{blondel_2012_data}. In particular (iii) contains information regarding individual trajectories for the entire observation period at the cost of reduced spatial resolution; position is recorded as a sub-prefecture rather than antenna location. The raw data consists of a user ID, date and time stamp of the communication and the sub-prefecture number (1-255) which can be linked to a file containing the latitude and longitude of the center of each sub-prefecture~\cite{lu_2013_approaching}.
%
%The data used in~\cite{lu_2013_approaching} acts as a measure of long range human mobility; due to the reduced spatial resolution, any movement within a sub-prefecture are not recorded. 
%
%However, the availability of large scale CDRs such as the D4D challenge make in-depth investigation into human mobility possible as the dataset as a whole consists of both long term CDRs with low spatial resolution and short term records with a much higher spatial and temporal frequency. Alongside this, as the data contains both caller and recipient ID, both mobility and social interactions of individuals can be observed. The combination of these two sources of information allow of the possibility to improve predictions of individual movements based on social ties. ~\cite{wang_2011_human, de_2013_interdependence}
%

Telecom Italia's ``Big Data Challenge'' is another example of a large scale CDR data set readily available to researchers. Alongside anonymized call records, additional metadata related to weather, news, Twitter activity, and electricity data from two areas in Italy (Milan and Trentino) was provided~\cite{barlacchi_2015_multi}. Data from this challenge has been used to investigate the relationship between individual daily trips and personal constraints~\cite{de_2015_personalized}, estimate urban population from call volume~\cite{douglass_2015_high} or determine the relationship between urban communication and happiness~\cite{alshamsi_2015_misery}. The increasing availability of such large-scale, anonymized datasets for research purposes is an encouraging trend for future research on human mobility. For more detailed information on CDR sources and some associated results (including other research topics than human mobility), see the review by Blondel \et~\cite{Blondel_2015_CDRreview}.

\subsection{GPS Data}
\label{sec:gps}

The greatest level of accuracy on movement trajectories is provided by Global Positioning System (GPS) data, which is capable of providing precise information on any location covered by at least four GPS Satellites. GPS trackers are units which receive signals from GPS satellites and compute the device's position at regular intervals. This technology allows researchers to trace the movement of individuals with a high degree of accuracy and temporal frequency, thus providing a rich source of data that can be analyzed and directly mapped to human mobility patterns.

\begin{figure}[t!]
\centering
\includegraphics[width=0.8\textwidth]{Figures_Data/2010_Bazzani_JStatMech_fig1}
\caption{Aggregated GPS position data (collected from vehicles) in a part of Florence, Italy,  measured during March 2008; the red dots correspond to a recorded instantaneous velocity $\leq30$ km h$^{-1}$, whereas the yellow dots correspond to velocities in the interval $30--60$ km h$^{-1}$ and the green dots to a velocity $\geq60$ km h$^{-1}$. Figure from~\cite{bazzani_2010_statistical_arxiv}.}
\label{fig:2010_Bazzani_JStatMech_fig1}
\end{figure}

Rhee \et\cite{shin_2008_levy} used GPS trackers with a position accuracy of less than three meters to record the movements of over $150$ individuals across two University campuses, a state fair (US), Disney World, and New York City. The position of the individuals carrying the trackers was recorded every 10 seconds while they were in these locations resulting in daily traces that had on average a temporal scale of 9-10 hours. Additionally, Microsoft has made GPS trajectories from 182 individuals freely available online via their GeoLife project\footnote{~\url{http://research.microsoft.com/en-us/downloads/b16d359d-d164-469e-9fd4-daa38f2b2e13}}. 
The dataset consists of 17,621 trajectories recorded by GPS trackers and GPS enabled mobile devices, and has a high spatial and temporal scale for over $90\%$ of individuals with positions recorded every 1-5 seconds or 5-10 meters for time periods that span several years. Since its release, GeoLife has been used in studies of both human mobility~\cite{zheng_2008_understanding, zheng_2009_mining} and social interactions~\cite{li_2008_mining}. For a review of potential applications see~\cite{zheng_2010_geolife}. 

Transmitters attached to vehicles~\cite{bazzani_2010_statistical, pappalardo_2013_understanding} are another source of GPS data. These are useful for studying questions related to urban traffic: monitoring, prediction, and prevention of congestion~\cite{pappalardo_2013_understanding}. In Italy, $\simeq 2\%$ of privately owned cars have a GPS system; each vehicle has assigned a unique identifying number, thus enabling the tracing of anonymous individual cars' trajectories with high precision. A GPS signal is transmitted approximately every 2km, as well as when the engine is switched on and off~\cite{bazzani_2010_statistical}. Each signal is converted to a data point containing information on position, velocity and distance covered. While the data suffers from inaccuracy due to variables such as satellite coverage and position precision, the temporal precision is of high quality, as is both instantaneous velocity and distance covered. From such GPS vehicle traces, path length distributions can be measured, which corresponds to the distribution of distances traveled in single trips, where a trip is defined as travel that occurs between instances of the engine being started and stopped. Activity distributions, which are equivalent to waiting time distributions in other studies, can also be determined from the elapsed time between consecutive trips. 
%Thus, unlike CDRs, GPS data allow the extraction of additional information on movement trajectories, such as 'flight length', pause time, direction and velocity. A flight length is defined as the geographical distance between two locations along which an individual moves, while pause times are the time intervals for which an individual is recorded at the same location. 
A further advantage of GPS traces over CDRs is that the locations are recorded to a much higher degree of accuracy and at a constant frequency. 
%
Yet, there are drawbacks: GPS transmitters do not work indoors and generally rely on batteries as their source of power. As a result, traces can contain periods during which no signal is received or may be terminated prematurely. Furthermore, GPS datasets typically feature a smaller number of individual users, up to several thousand, in comparison to mobile phone data which can provide information on the mobility of millions of users. 

\subsection{Online Data}
\label{sec:online}

Another valuable source of location information are Online Social Network (OSN) or Location-Based Social Network (LBSN) services, that attract hundreds of millions of users worldwide. Since the introduction of GPS and wifi chips in smartphones, social network providers have been able to collect valuable data on both the social connections as well as precise geographical locations of their users.
Indeed, services such as Twitter, Facebook, Foursquare and Flickr collect geotagged data every time a user enables localization for the content being posted (e.g., checking-in at a restaurant with friends); this is associated with geographic coordinates, a time stamp, and additional information relative to the location itself or the content published by the user.
The mobility profiles of the users can be obtained from the list and number of locations other than their home which they have visited over the period of the study (see \figurename~\ref{fig:2014_Hawelka_CGIS_Fig5} for an example). These profiles can range from intra-urban routine trips to worldwide travel~\cite{noulas_2012_tale, hawelka_2014_geo}. 
From the geographical locations, by inferring a user's home location, their radius of gyration can be determined using the geographical distance between the home location and reported locations.
The jump length distribution can be determined by defining a jump as the geographical distance between consecutive reports for a single user and frequency of reports can be used to assess temporal patterns in human mobility~\cite{hawelka_2014_geo}. Other measures such as visitation frequency and predictability are also obtainable~\cite{jurdak_2015_understanding}. \\

\begin{figure}[t!]
\centering
\includegraphics[width=\textwidth]{Figures_Data/2014_Hawelka_CGIS_Fig5}
\caption{Country-specific analyses of travel based on Twitter users and on the estimated total number of travelers. Panel A shows the number of Twitter users residing in a country and traveling to another while panel B shows the number of users visiting this particular country. Panels C and D represent the number of Twitter travelers normalized by the  extent of Twitter usage in their home country. Finally  panel E represents the yearly ratio between the estimated inflow and outflow of travelers, revealing which countries were the origin or destination of international travel. Figure from~\cite{hawelka_2014_geo}.}
\label{fig:2014_Hawelka_CGIS_Fig5}
\end{figure}

Geotagged data can also be used to assess social interactions between individuals. Scellato \et\cite{scellato_2011_socio} analyzed three online location-based social networks: Foursquare, Brightkite, and Gowalla, finding their exhibition of common properties found in most other real world complex networks, such as fat tailed degree distributions, high clustering coefficient, and short average path length between nodes, confirming the small-world nature of LBSN's~\cite{Newman_book}. Furthermore, the probability of a social link  between two users was found to exhibit a distance dependence, decreasing with geographical separation between the users.
Compared to CDR's, data collected from OSN's and LBSN's, have the advantage of having more contextual information associated with the geographical positions and the users, enabling the study of mobility in a broader context. For example, data collected from Twitter users has been applied to topics such as social networks~\cite{java_2007_why, huberman_2008_social, kwak_2010_what}, evolution of moods~\cite{pak_2010_twitter, bollen_2011_modeling, golder_2011_diurnal}, and crisis management~\cite{sakaki_2010_earthquake,maceachren_2011_senseplace2, thom_2012_spatiotemporal} among other features.

Yet, as in all other data sources, limitations persist. Besides requiring extensive clean up (such as removing users with suspiciously high activity, unreasonably fast movement between two consecutive check-ins or incomplete data \cite{hawelka_2014_geo}), mobility and interaction data originating from online social networks needs further validation in order to be considered truly representative of the general population. Sloan \et\cite{sloan_2015_tweets} found that Twitter data originating from the UK has significant demographic differences compared to the wider population data from UK census. Furthermore, there appear to be demographic differences between the users who enable geotagging and those who do not (only $\approx 3\%$ of the users were found to have geotagging enabled).

