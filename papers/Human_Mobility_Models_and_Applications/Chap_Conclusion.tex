In this work we discussed the state-of-the art in the field of human mobility. After a short historical description of the field, we focused on current work starting with the type of data utilized in most research projects, followed by the description of metrics and models of individual mobility and population mobility. We illustrated a list of selected applications of human mobility modeling, demonstrating the insights the field can bring to the solution of real-world problems. 

It should be clear from the survey that the study, understanding, and modeling of human mobility is of primacy to many areas of society. At a deeper level the current research raises a fundamental question regarding human cognitive limitation. Among the many features described here, one in particular stands out: Why are we likely to return to frequently-visited locations or recently-visited locations? Does it relate to our ability to remember where we have been? Is it a innate characteristic of individuals or something that can be learned (\ie nature vs. nurture)? The same can be said about regularities whereby human behavior is divided between those that are explorers or returners in their mobility pattern \cite{pappalardo_2015_returners}. Indeed, this begs the interesting question, as to whether this observed division is an inherent property of society, say acting as a stabilizing influence? Indeed, this dichotomy  is also found in other context, such as social learning, where it has been suggested that a population cooperating to solve a complex task can attain a better performance if it is composed by a mixture of explorers, individuals who try new solutions, and exploiters, individuals that have less propensity to abandon the optimally extant solutions~ \cite{rendell_2011_cognitive}. 
More generally, the exploration vs exploitation dilemma is a fundamental issue in optimal decision-making and is analyzed in economics (multi-armed bandits) and computer science (reinforcement learning) and finds natural applications in marketing and advertisement, where recommendation algorithms determine whether people are likely to adopt familiar or unfamiliar products. All these problems require an understanding of human dynamics and many of these areas can possibly be studied using approaches inspired from research in human mobility. With the data we continue to collect as a society, and given the methodology developed to explain regularities in human mobility, we now have an additional set of mathematical tools to characterize the regularities in human activities more broadly.

A different structural dimension linked to human mobility relates to the invariant that humans are linked to each other forming social networks. Studies of social networks aim at grasping the multidimensional properties of the structures formed by entities (e.g., persons and organizations) and the ties connecting them together \cite{wasserman_1994_social}. In fact, the intersection between human mobility and social networks goes beyond the obvious analogy that humans are social and their movement may be related to them being social~\cite{wellman_1979_networks,tilly_1990_transplanted,liljeros_2001_web,tassier_2008_social,barabasi_2013_network}. The field of Social Network research is a huge enterprise in its own right and elucidating the connection between social networks and human movement is a nascent effort that should be explored in more depth. 

An interesting research frontier in human mobility in the near future will likely concern the shift from traditional vehicles to autonomous, self-driving, vehicles. The diffusion of autonomous vehicles will transform private and public transportation, with sweeping consequences for society, the economy and the environment.
How exactly will these transformations happen and how will mobility habits change in response to these new technologies?
How can such vehicles be controlled and how does one plan routes in order to optimize fuel and time consumption, reducing congestion and pollution?
Answers to these important questions, may well arise from a fuller understanding of human mobility and behavior at the individual and population (collective) levels. 

Finally, it should be clear from this survey that interdisciplinarity and cross-fertilization is a {\it condicio sine qua non} to the success of the field. Computer scientists, physicists, social scientists, environmentalists, engineers, government officials and many other professions need to work together to develop and implement solutions to many of the problems society faces today that could benefit from understanding mobility (and dynamics) such as crime, urban planning, energy consumption, social integration, to name but a few.