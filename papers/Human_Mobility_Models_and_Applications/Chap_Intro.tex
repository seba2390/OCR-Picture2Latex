\subsection{Motivation and History}
While the term \emph{mobility} has multiple connotations, in the context of this review it refers to the movement of human beings (individuals as well as groups) in space and time and thus implicitly refers to \emph{human mobility}. Indeed, from the migration of {\it Homo sapiens} out of Africa around 70,000 years ago, through the European discovery of the ``New World", to the existence of expatriate populations in contemporary times, the existence of human beings has always been inextricably linked with their movement. While earlier movement patterns were primarily driven by factors such as climate change, inhospitable landscapes, conflict and food scarcity, in modern times, socio-economic factors such as wage imbalance, differences in welfare and living conditions and globalization play an increasing role. 
% \COMMENT{\TL Not sure about this: conflict/wars were already responsible for migrations of populations in very ancient times}

% \COMMENT{\TL : 
% \begin{itemize}
% \item  add the following references : \cite{ullman_1980_geography,helvig_1964_chicago,boyce_2015_forecasting,reilly_1929_methods,hanson_1980_importance,huff_1986_repetition,schaefer_1953_exceptionalism,hanson_1981_travel,hanson_1985_gender,ericksen_1977_analysis,barthelemy_2016_structure} ;
% \item introduce the term "spatial interaction" (\url{http://classes.uleth.ca/200503/geog3235a/Spatial\%20Interaction.htm})
% \end{itemize}.
% }
% Examples of short range mobility
From hunters and gatherers of prehistoric times to present-day commuters of large metropolitan areas, humans are bound to move on a daily basis in order to earn their livelihood. However, daily trips are also undertaken to perform social and leisure activities. 
The temporal and spatial scales of these trips are much shorter than those of migratory flows, and they are often characterized by the regularities and periodicities that mark human lives. 
The daily movement of a large and growing population has important impact on the lives of the individuals and the environmental conditions. Studies conducted in Europe and the United States found that the average household spending on transportation is between 15 and 25 percent of the total expenditures, making transportation the second largest expenditure category after housing. Transportation is also the second source of greenhouse gas emissions to the atmosphere. 
% Questions
From these few examples it should be clear that mobility has an enormous impact on human societies and an accurate quantitative description of human mobility is of fundamental importance to understand the processes related to human movement and their impact on the community and the environment. 
A quantitative theory of human mobility ought to be able to provide answers to relevant questions, such as, what determines the decision to start a trip? which factors determine the choice of the destination? to what extent is human movement predictable, and what is the intrinsic degree of randomness? is it possible to find general rules or laws to explain empirical patterns and regularities exhibited by travels in many diverse countries, such as the distribution of commuting times and distances, and the degree of predictability of future whereabouts?

% Regularities
% The main subject of research on human mobility pertains to several spatiotemporal and sociodynamical regularities exhibited by travels. 
% Such regularities were first measured for flows of people and goods between places. 

While the study of human mobility currently spans several disciplines, arguably, geography was the first discipline to analyze mobility data and put forward corresponding theories to describe travel patterns. Indeed, 
%Beyond the measure of flows, 
the pioneers of quantitative and theoretical geography in the 50's defined ``geography as (the scientific discipline of) \emph{spatial interaction}" \cite{ullman_1980_geography}. Early quantitative studies of the movements of people and vehicles were held in large US metropolitan areas \cite{helvig_1964_chicago} (see the first chapters of  \cite{boyce_2015_forecasting} for details), and initial studies of human travel were of scales ranging from international migrations \cite{reilly_1929_methods,stouffer_1940_intervening,zipf_1946_p1,anderson_1955_intermetropolitan} to journey-to-work commuting \cite{hanson_1980_importance,huff_1986_repetition,kitamura_2000_micro,bhat_2004_comprehensive,pendyala_2005_florida}. Indeed, the elucidation and understanding of these patterns was motivated by its relation to several real-world applications such as traffic forecasting \cite{boyce_2015_forecasting, nagel_1995_emergent, wang_2012_understanding}, urban planning \cite{hillier_2009_metric,kitamura_2000_micro}, internal security \cite{krebs_2002_mapping,clauset_2010_strategic} and epidemic modeling \cite{colizza_2007_modeling,vespignani_2012_modelling,tizzoni_2014_use} to name but a few. 

The first systematic analysis of the concept of distance as a constraint to movement was proposed in the 19$^\text{th}$ century:
in his 1965 review~\cite{olsson_1965_distance} Gunnar Olsson cites Henry C. Carey's \textit{Principles of Social Science} (1858) as the first work to explicitly make the observation about the  amount of interaction between two cities being proportional to their population size and inversely proportional to the intervening distance. Few decades later, the geographer Ernst Ravenstein further developed and popularized the idea in a seminal work where he formulated his \emph{laws of migration}~\cite{ravenstein_1885_laws}. %} 
Further refinements on this theme were made in the 1940's by the American sociologist Samuel Stouffer~\cite{stouffer_1940_intervening, bright_1941_interstate} in his \emph{law of intervening opportunities}, and by the American linguist George Kingsley Zipf~\cite{zipf_1946_p1}.
Zipf's formulation led to what is now conventionally referred to as the \emph{gravity law}. The increasing availability of datasets on population movements at various levels of granularity, coupled with the \emph{quantitative revolution} in geography \cite{schaefer_1953_exceptionalism,berry_1993_geographys,adams_2001_quantitative}, led to the introduction of more sophisticated mathematical methods such as hidden Markov models and diffusion processes. Gender and socio-economic factors behind population movement were further analyzed thanks to richer datasets resulting from surveys and interviews \cite{ericksen_1977_analysis,hanson_1981_travel,hanson_1985_gender}, and through theories of labor economics focusing on wage differential between locations~\cite{jennissen_2007_causality}. Thus through the 20$^\text{th}$ century, contributions to the theories of human movement were made chiefly in geography, %(and its modern avatar quantitative geography \COMMENT{\TL: I would remove this}), 
sociology, and economics, while the scale at which this was primarily studied was at the population level.

%\COMMENT{\TL: The following 3-step summary is clear, pleasant to read and conform to the classic story told in previous mobility papers. But attributing to Zipf the paternity of gravity models applied to social interactions and human movements, without any further discussion, maybe falls a bit short. John Q. Stewart wrote a number of papers in the 1940's and 50's in which he progressively refined the idea of gravitational human interactions and movements (including his seminal 1941 Science paper in which he writes that "the number of undergraduates or alummi of a given college who reside in a given area is directly proportional to the total population of that area and inversely proportional to the distance front the college" \cite{stewart_1941_inverse}). I can see below that Ronaldo wrote a couple of paragraphs mentionning these papers as well (now commented). Also the old but very complete reviews by \cite{olsson_1965_distance}}
%\COMMENT{\ML: I agree, it seems that the "first gravity model" was suggested by Carey and popularized by Zipf. In any case we should try to be consistent with the population-level section.} \\
%\COMMENT{\FS:
%\begin{itemize}
%\item I suggest to edit/add one or two sentences to improve the part about the origins of the Gravity model. Can the authors of the comments above take care of this? \\
%\item I think what Ronaldo wrote about the relationship between mobility and social interactions (now commented) could go in the conclusions. 
%\end{itemize}
%}

To provide context for what is to follow, we briefly describe a selection of influential historical contributions, keeping in mind ``Stigler's law of eponymy'' which states: ``No scientific discovery is named after its original discoverer''%(due to the statistician Stephen Stigler
~\cite{stigler_1997_statistics}). However, if we restrict ourselves to the recognition of \emph{distance} as a primary factor in determining movement and interactions between places, then it is reasonable to start with the work of Ernst Georg Ravenstein~\cite{ravenstein_1885_laws}.\\
%(\COMMENT{\TL: same remark than above, as Carrothers' (1956) and Olsson's (1965) reviews of human interactions models mention Carey's paper as the seminal one.}).  

\noindent {\bf Laws of Migration.}~Ravenstein was a German-English geographer who made important contributions to cartography as well as providing one of the first rough estimates of a ``maximal'' global population based on resource consumption. He was also one of the first to attempt an explanation and prediction of migration patterns within and between countries. Considering the effect of distance as well as the type of migrant (male or female, old or young) as primary factors, he posited the following seven laws:
\begin{enumerate}
\item Most migrants only travel short distances, and ``currents of migrations'' are in the direction of the great centers of commerce and industry given that these can absorb the migrants.
\item The process of absorption occurs in the following manner: inhabitants of the areas immediately surrounding a rapidly growing town flock to it, thus leaving gaps in the rural areas that are filled by migrants of more remote districts, creating migration flows that reach to ``the most remote corner of the kingdom''.
\item The process of dispersion is inverse to that of absorption, and exhibits similar features.
\item Each main current of migration produces a compensating counter-current.
\item Migrants traveling long distances generally go by preference to one of the great centers of commerce or industry.
\item The natives of towns are less migratory than those of the rural parts of the country.
\item Females are more migratory than males.
\end{enumerate}
Ravenstein added another two laws in 1889 \cite{ravenstein_1889_laws}: 
\begin{enumerate}
  \setcounter{enumi}{7}
\item Towns grow more by immigration than by natural increase.
\item The volume of migration increases as transport improves and industry grows.
\end{enumerate}
%

%\COMMENT{\HB: I think the seminality of Ravensteins' Laws of Migration is also in his attempt to provide a structured (although descriptive) characterization of human displacements -- more in the mobility modeling realm -- in which the distance/population relation is one of the elements. The other laws describe the emergence of flows and counter-flows of migration, gender inequality in migration, rural/urban flows etc. Although  his laws are not believed true anymore, they provided a more general framework to characterize human migrations accounting for other socio-demographic factors.  }

While the laws are non-quantitative and observational in character, Ravenstein correctly identified socio-economic factors as well as distance-constraints to be the essential ingredients behind modern population movement. Consequently, his laws stimulated an enormous volume of work, and although they have been refined and adjusted over the years, the essential ingredients of his formulation remain relevant even today.\\

\noindent{\bf Law of Intervening Opportunities.} One of the most important refinements was made in the 1940's by the American sociologist Samuel Stouffer~\cite{stouffer_1940_intervening}. Roughly speaking, Stouffer was looking to expand upon Ravenstein's observations regarding migrants moving shorter distances and flocking to commercial centers. To account for this, he proposed that \emph{the number of people going a given distance is directly proportional to the number of opportunities at that distance and inversely proportional to the number of intervening opportunities}. In other words, trips between two locations are driven primarily by relative accessibility of socio-economic opportunities that lie between those two locations. In this context, \emph{opportunity} is defined as a potential destination for the termination of a traveler's journey, whereas an \emph{intervening opportunity} is one that the traveler rejects in favor of continuing on. In Stouffer's original formulation, this can be mathematically expressed as
\begin{equation}
\frac{d y(r)}{dr} \propto \frac{1}{x}\frac{d x(r)}{dr},
\label{io_eq1}
\end{equation}
where $y(r)$ is the cumulative number of migrants that move a distance $r$ from their original location, and $x(r)$ is the cumulative number of intervening opportunities. 
%\COMMENT{\FS here we use $r$ for distance, whereas in the next paragraph about Zipf $r$ is the rank and distance is $d$. I'd suggest to use a unique symbol for distance, but we have to decide if that should be $r$ or $d$ or something else.}
%\COMMENT{\HB Well observed, Filippo.  $r$ is also used in other places as distance. I fixed it replacing $d$ with $r$ for distance and $r$ for $z$ for the frequency rank.}
Assuming that $x$ itself is a continuous function $x(r)$ of distance, then the expression above can be integrated to yield
\begin{equation}
y(r) = \log x(r) + C,
\label{io_eq2}
 \end{equation}
where $C$ is some constant denoting the number of opportunities at the origin location. Thus, the relationship between mobility and distance is indirect and is established only through an auxiliary dependence via the intervening opportunities: the higher the number of intervening opportunities between two locations at distance $r$, the smaller the number of migrants that would travel that distance.
% Furthermore, the logarithmic connection between the number of migrants and the number of opportunities at distance $r$ implies some kind of \emph{distance decay} but as an auxiliary function of other factors. 
This may explain why rural migrants may flock to urban centers over large distances, whereas those already in commercial centers are comparatively more stationary.\\

\noindent {\bf Distance-Decay and the Gravity Law.} Around the same time as Stouffer, the Harvard Philologist, George Kingsley Zipf, was expanding upon his famous observation of the rank-frequency dependence in linguistics; the eponymous Zipf's law, where the frequency of a word ranked $z$---in terms of usage---has the statistical dependence $f_z \sim 1/z$~\cite{zipf_1936_psychobiology}. Zipf found that this relation was expandable to other realms of society, specifically the size of cities~\cite{zipf_1940_generalized}, where the occurrence of a city with population $P$ and consequent rank $z$ also follows the relation 
%
\begin{equation}
P_z \sim 1/z^{\alpha}.
\label{eq:zipflaw}
\end{equation}
%
Broadly speaking, Zipf's argument for this relation was due to the tension between two competing factors. The first, which he called \emph{Force of Diversification} relates to the likelihood of populations living close to the source of raw materials (commodities) in order to minimize the cost of transportation to production centers. The second effect, referred to as the \emph{Force of Unification}, is the tendency of populations to aggregate in urban centers due to the minimization of work required to transport finished products to consumers. While the former leads to the formation of multiple urban centers (given that the commodity sources are not localized in one part of a country) with smaller populations, the latter has the competing effect of urban agglomeration in a \emph{few} centers of large population. Assuming some kind of equilibrium between these quasi-forces, the rank-frequency relation of \equationname~\eqref{eq:zipflaw} with $\alpha = 1$ naturally follows~\cite{zipf_1941_national}. Deviations from equilibrium, where one force dominates over the other, then leads to a change in the exponent.  

Carrying the argument further, under somewhat unrealistic assumptions of equitable share of national income as well as urban centers being self-sufficient (i.e. production and consumption at equal levels), the share of any center $i$ in the total flow of goods is proportional to its population $P_i$. Therefore, the flow of goods between two centers is proportional to the product of their populations. Finally, if one would like to minimize the cost and work associated with transportation of goods, this flow must be inversely proportional to the distance between centers. Putting all this together we arrive at the relation
%
\begin{equation}
w_{ij} \propto \frac{P_i P_j}{r_{ij}},
\label{eq:dist-decay}
\end{equation}
%
where $w_{ij}$ represents the flow of goods between two population centers $i,j$ and $r_{ij}$ is the distance between these two centers. Zipf tested his theory on both freight and population movement data and got good qualitative agreement. This formulation is of course quite different form Stouffer's explanation as the effect of distance is quite explicit in \equationname~\eqref{eq:dist-decay}. Indeed, the form of \equationname~\eqref{eq:dist-decay} is such that it leads to a distance-decay effect suppressing long-range movement. Furthermore, the flow of populations and goods is seen to be as a result of some dynamic equilibrium between the cost of transportation, manufacture, and distribution of goods and services. The references to ``forces'' as well as its functional dependence on distance eventually led to Zipf's formulation being dubbed the \emph{Gravity law} in analogy to Newtonian mechanics.

The common theme connecting Ravenstein, Stouffer and Zipf of course is the geographical distance, though its functional effect on movement is quite different in the Intervening Opportunities and Gravity Law models. Nevertheless, both these models were quite influential on subsequent work, setting off two major strains of parallel research as well as attempts at unification.\\ 

\noindent {\bf Time geography and ICT Data.} Measuring, understanding and forecasting the displacements of individuals in space and time has long been part of the program of quantitative and theoretical geography, a branch of geography ``born'' academically in the 1950's \cite{berry_1993_geographys,adams_2001_quantitative}. While the first efforts in capturing human displacements focused on the aggregated levels of flows between spatial units, some also focused on individual trajectories. Torsten H\"agerstrand, a Swedish geographer, laid down in the early 1950's the basis of \emph{time geography}, and brought a number of conceptual and graphical tools to formalize the trajectories of individuals through space and time. His seminal work \cite{hagerstraand_1970_what} remains famous for its proposal to represent individual trajectories in a cube (also known as the ``space-time aquarium''), in which the horizontal plane represents the geographical space, while the vertical axis represents time (as depicted in Fig.~\ref{fig:timegeo-cube}). %\figurename~\ref{fig:aquarium}).\
%\begin{figure}[tpb]
%  \centering
% \includegraphics[width=.6\textwidth]{Figures_Intro/aquarium}
%  \caption{A simple example of a spacetime aquarium as proposed by H\"agerstrand.}
%\label{fig:aquarium}
%\end{figure}
%%
H\"agerstrand proposed a number of graphical conventions (a ``notation system" in his own words) to picture the constraints imposed by social life on individual daily trajectories. He also provided means to represent the co-presence and synchronization of several individuals in space, and more generally a set of (essentially graphical) conventions useful to represent the structure and behavior of individual human mobility. 
%A more complex example of such a representation is shown in \figurename~\ref{fig:timegeo-cube}.
\begin{figure}[tpb]
  \centering
\includegraphics[width=.8\textwidth]{Figures_Intro/TimeGeographyWithTraces}
  \caption{The cubes of time geography, as first proposed by Torsten H\"agerstrand in \cite{hagerstraand_1970_what}. The geographical space is represented by the 2D plan, while time is figured by the vertical axis. (Left) The two curves represent the daily space-time trajectories of two individuals living in the same neighborhood and working in the same place. (Right) The geographical footprints continuously and passively produced by individuals through the use of their ICT devices allow to approximate their trajectories. While these re-constructed trajectories are partial and contain errors that might mislead the understanding of underlying trajectories, they are nonetheless more precise nowadays than they were 10 years ago, and produced by a constantly growing number of individuals worldwide.}
\label{fig:timegeo-cube}
\end{figure}

Time geography was naturally invoked (and somewhat rediscovered) in the 1990's, when the modeling of human mobility shifted towards individual-based simulation (micro/multi-agent/agent-based simulation)~\cite{chardonnel_2007_time}. This can be understood in the twin contexts of increased computing power and the development of more expressive programming frameworks, allowing for semantically richer and more ambitious models of human dynamics. However, while the models increased in complexity, they were somewhat artificial;
%as they weren't systematically supported by
relevant data of comparable complexity and resolution was not available for their calibration. Indeed, while the models progressed, the data lagged behind and the best one had to work with was longitudinal survey data collected since the 1970's. 

The beginning of the 21$^\text{st}$ century saw the introduction and subsequent widespread adoption of mobile phones, as well as the pervasive usage of Global Positioning System (GPS). This led to an exponential increase in data-generation on human movement. Coupled with further progress in computing power and sophisticated data-mining methods, it enabled to capture the movement not just of populations at finer levels of spatial granularity, but potentially of \emph{individuals}. In particular, the large volume and frequency of Call Detail Records (CDRs) from mobile phones (see section \ref{sec:cdr}) enabled the analysis of human movement at very fine \emph{temporal} scales. Thus statistical information on mobility became available, on the scale of hours to decades; from the individual, through communities to the level of country-wide populations.

\medskip

This review predominantly focuses on these later developments. As we will see throughout the text, these new sources of data have rejuvenated the scientific interest for human mobility, opening the door to new questions and measures, as well as enhancing and validating the insight gained from studying traditional data sources. For reasons which are probably technical, historical, and societal, research teams in physics and computer science have broadly ``invested" in these new georeferenced (meta)data resulting from human activity~\cite{osullivan_2015_do}. This is evidenced by a quick search on the online website arXiv\footnote{http://arxiv.org} for physics papers that include the terms ``Human Mobility'' or ``Mobility Patterns''. As seen in \figurename~\ref{fig:pubs}, since about 2004-2005, the number of such papers displays an almost exponential increase. One must keep in mind that this is probably an incomplete sample, and is underestimating the volume of research in this field. It is likely that one would see an even more dramatic trend if a more comprehensive list of publications %(such as those from the Computer Science community) 
were to be included.

\begin{figure}[t!]
%\centering
\includegraphics[width=0.9\textwidth]{Figures_Intro/arxiv_mobility}
\caption{Fraction of papers on arXiv.org with mention to the terms "Human Mobility or "Mobility Patterns" from 2004 to 2015. The growth in the number of published papers displays a dramatic increase around 2008. Data source: arXiv.org (last date accessed: Feb 16, 2017).}
\label{fig:pubs}
\end{figure}

While this growing interest of the Physics community in human mobility has multiple reasons, it is somewhat to be expected given the traditional interest of statistical physicists in studying emergent collective properties of systems of many interacting particles and in describing the dynamics of randomly moving particles undergoing (anomalous) diffusion. The availability of data at multiple spatio-temporal scales is helping uncover robust statistical patterns as well as aiding the development of phenomenological theories of human mobility, that are well suited to the tools, methods and paradigms of \emph{statistical physics}. 

\subsection{Scope and Limitations}

The recent pace of developments and the volume of work published are such that a comprehensive review of the findings appeared necessary. Moreover, the study of human mobility is a highly interdisciplinary endeavor and progress has occurred in parallel across different academic communities, sometimes with one unaware of the others' works. This is reflected in different terminology of common metrics and similar results being cast in different and seemingly unrelated contexts. Consequently with this review, we also aim at \emph{(i)} bringing disparate communities together by unifying the findings in a common context, and \emph {(ii)} providing new researchers in the field with
% (who might otherwise be intimated by the pace and diversity of developments) 
an accessible starting point and a minimal set of tools, metrics, concepts and models.
% such that they too can join in this exciting field.

%The field of Social Network research is a huge enterprise in its own right and it is not the focus of this review. Instead, we only touch upon those aspects that are directly associated with human mobility. Elucidating the connection between social networks and human movement and our treatment of this will topic is brief. 

% \COMMENT{\TL Should we keep the last couple of sentences since we now have a section dedicated to the topic?}
% \COMMENT{\HB I think we should keep it, just to acknowledge that although the role of social ties to human migration can be traced back at least to 1986,  in our review we're focusing on the more recent empirical efforts.}

One must note that humans, of course, do not move in a vacuum. Just as the diffusion of molecules in materials is mediated by the structural and thermodynamic properties of the material, similarly, spatio-temporal patterns of human movement are necessarily shaped by spatial constraints and limitations of geography. Examples of this are the topography of urban centers or the pattern of roads and transportation infrastructure, the properties of which are studied under the aegis of \emph{spatial networks}~\cite{barthelemy_2011_spatial}. Yet, this survey is \emph{not} about Spatial Networks, nor do we give details on the so-called ``science of cities" (e.g., city forms, properties of urban transport networks, urban scaling laws, etc. \cite{barthelemy_2016_structure}); these two aspects will be considered only when made necessary for the reader to understand the concepts related to human mobility.

% \GG {\bf Want to say here that this review is not concerned with spatial networks nor is it a review of urban systems (City Science). Instead it only touches upon specific aspects of each that are related to human mobility, Need some help with this.}
% \RM {\bf I left your suggestion untouched. I'm not entirely sure we have to say what you're suggesting. Why? Do you think people would expect us to talk about City Science in more depth? or Spatial Networks? I took a shot above. See if this is enough.}

\subsection{Organization}

We begin in \sectionname~\ref{sec:data} with a comprehensive (but by no means exhaustive) list of data sources used for empirical studies. In \sectionname~\ref{sec:scaling} we introduce key metrics, measures, spatio-temporal scales, as well as an overview of the fundamental physics behind the study of human mobility.  In \sectionname~\ref{sec:models} we describe the state-of-the-art families of models (both generative and phenomenological)  that best describe the empirical observations of human mobility. The models are categorized according to scale, starting from the level of individuals, through to the collective level of populations, and finally a mixture of the two incorporating the concept of (inter) modality.  Continuing with the theme of scales, in \sectionname~\ref{sec:applications} we describe some selected applications of the framework and models ranging from intra-urban movement to flows between countries and continents including the case of epidemic spreading, transportation systems, and a brief digression on new results on virtual mobility (web browsing). Finally, we conclude in \sectionname~\ref{sec:conclusion} with challenges and future directions for the field. We also added Appendix~\ref{sec:comp} where we provide descriptions of some basic tools and algorithms (agent-based modeling, random walkers, etc.) that may be of use to researchers making initial forays into the field.


%On the Social Networks front, most of the theoretical grounds derive from contributions by groups in different areas. In the 1930s, Moreno proposed a graphical representation of a social relationship, called \emph{sociograms} \cite{moreno_1934_who}. Later, Erd\H{o}s and R\'enyi publish a seminal paper in the area proposing, not only the first network model, namely the \emph{random-graph model}, but also the theoretical groundwork for several other network models based on the random graphs theory \cite{holland_1981_exponential,newman_2001_random,goodreau_2007_advances}. 

%In the later 1960s, empirical studies carried out by Milgram investigated the \emph{small-world problem}, measuring the shortest path lengths between two individuals in the U.S \cite{milgram_1967_small}. In this work, Milgram empirically found that the average shortest path length (i.e. the minimum number of social ties connecting any two individuals of a population) in social networks is around~6 \cite{milgram_1967_small,travers_1969_experimental}. Despite the limited amount of data, Milgram's findings were empirically validated by \cite{dodds_2003_experimental} on a global-scale experiment, whose results suggested median path lengths from 5 to 7. 

%Although the correlation between propinquity and social interactions has long been known by sociologists and social psychologists \cite{stewart_1941_inverse,olsson_1965_distance}, it was only in the last decade that these two phenomena began to be studied together producing an extensive literature on the subject. Such studies come mainly from Physics and Computer Science research, and most of them exhibit the presence of power-law relationships in the dynamics of both human mobility and social networks \cite{nagel_1995_emergent,albert_2002_statistical,brockmann_2006_scaling,gonzalez_2006_system,gonzalez_2008_understanding,song_2010_modelling,simini_2012_universal}.
%Scholars agree that the social relationships are often driven by spatial proximity \cite{mok_2007_did,wang_2012_understanding,grabowicz_2014_entangling}. On the other hand, the way in which people move is also known to be influenced by social factors. In fact, recent studies have shown that a good approximation for a persons' position can be obtained from the position of her friends \cite{backstrom_2010_find,boldrini_2010_hcmm,grabowicz_2014_entangling}. 
%However, our knowledge on the interplay between social networks and human mobility patterns is very limited, partly due to the lack of large-scale data that captures simultaneously the individuals' spatio-temporal information and the social interactions among them. Moreover, all the models based on the interplay between these two phenomena assume one is the outcome of the other. For example, models that predict a person's position from their acquaintances assume that human trajectories are influenced by the social relationships. Conversely, several social network models assume that social interactions are physically constrained, hence the social network structure is an outcome of human mobility patterns. Yet, the origin and nature of this interplay remain an open question.
