\section{Relation Confidence Estimation Module}
As introduced in the main paper, the RCE module is an important parts of our method.
To demonstrate the effectiveness of the RCE module, we further introduce the learning details of the RCE module and performance comparison with the similar model proposed by previous works.

\subsection{Learning}
We use a supervised learning strategy to train the RCE module of BGNN, in which the predicate class labels (which predicate category and whether it is valid predicate or background) are used for supervision. 
Different from the cross-entropy loss $\mathcal{L}_p$ used for the final predicate predictions, we develop a multi-task loss $\mathcal{L}_{rce}$ for the RCE module. Specifically, we use two confidence predictions from the RCE: multi-categories confidence score $\mathbf{s}^m \in \mathbb{R}^{C_p}$ and binary confidence score $s^b$. We define two focal losses, $\mathcal{L}_{m},  \mathcal{L}_{b}$ on the confidence predictions of $M$ predicate proposals $\{\mathbf{s}^m_1, ... \mathbf{s}^m_M\}, \{s^b_1, ... s^b_M\}$, respectively. 
Formally:
\begin{align}
\mathcal{L}_{m} &= -\alpha\frac{1}{M}\sum_k^{M}\sum_i^{|C_{p}|} \mathbf{y}_{k, i} (1- \mathbf{s}^m_{k,i})^{\gamma} \cdot \log(\mathbf{s}^m_{k,i}) \\
\mathcal{L}_{b} &= -\alpha\frac{1}{M}\sum_k^{M} y'_k(1- s^b_{k})^{\gamma} \cdot \log(s^b_{k})
\end{align}

%The $\mathbf{y}_{k}$ is the one-hot vector as same as the labels for $\mathcal{L}_p$, the $y'_k$ is the binary for indicating the positive relationship proposals.
where $\mathbf{y}_{k}$ is one-hot vector and $y'_k$ is binary label of positive predicate proposals. $\alpha,\gamma$ are the hyper-parameters.


\subsection{Performance}
To demonstrate the effectiveness of the RCE module on removing negative predicate proposals, we use the \textbf{AUC} to measure its performance, and compare it with two alternatives: \textit{production of entities prediction score} and \textit{relation proposal network} proposed by Graph-RCNN.
The AUC of those three methods are \textbf{0.839, 0.629, 0.671}, respectively on the validation set of VG, which indicates RCE module is more effective than previous works.
%, which also significantly better than previous methods.


\begin{figure}
   \centering
   \includegraphics[width=0.95\linewidth]{imgs/qualitative_results.pdf}
   \caption{\textbf{Qualitative comparisons between our method and GPS-Net$\dagger$ in the SGGen setting.}
   The predicates in \textit{body} and \textit{tail} categories group are marked as \textcolor{red}{red} color.
   We also show the reasonable relationships detected by models which are not included in GT.}  
   \label{fig:qualitative}  
\end{figure}

