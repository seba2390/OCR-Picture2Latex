\begingroup
\setlength{\tabcolsep}{3.5pt} % Default value: 6pt
\renewcommand{\arraystretch}{1.06}
\begin{table}[ht!]
\centering
\caption{Examples of misleading instances included in the \textsc{SupNatInst} dataset. The misleading part is highlighted in \textcolor{red}{red}, along with explanations for why it is misleading.}
\begin{tabular}{@{}ll}
\toprule
\textbf{Task ID}
& task199\_mnli\_classification
\\
\hline
\textbf{Instance ID} 
& task199-9f3e5f6241ce41c7ba83bf9a2b6be11d
\\ 
\midrule
\textbf{Definition} 
& 
\begin{tabular}[c]{@{}l@{}}
In this task, you're given a pair of sentences, sentence 1 and sentence 2. Your \\job is to determine if the two sentences clearly agree/disagree with each other, \\or if this can't be determined. Indicate your answer as yes or no respectively.
\end{tabular}
\\ 
\midrule
\textbf{Instance}
&
\begin{tabular}[c]{@{}l@{}}
Sentence 1: Ninety-five percent of the total amount of sulfur dioxide \\allowances allocated each year under Section 423 will be allocated based on \\the amount of sulfur dioxide allowances allocated under the Acid Rain Program \\for 2010 and thereafter and that are held in allowance accounts in the Allowance \\Tracking System on the date 180 days after enactment. Sentence 2: Most of the \\sulfur dioxide that is allowed are controlled by the Acid Rain Program.
\end{tabular} 
\\
\midrule
\textbf{Golden Labels}
&
\begin{tabular}[c]{@{}l@{}}
[\textcolor{red}{"no"}, "yes"] \textcolor{red}{\qquad\# An additional incorrect label is included.}
\end{tabular}
\\
\hline\hline
\textbf{Task ID}
& task020\_mctaco\_span\_based\_question
\\
\midrule
\textbf{Instance ID} 
& task020-3b643de54d474381854c2499cd185a74
\\ 
\midrule
\textbf{Definition} 
& 
\begin{tabular}[c]{@{}l@{}}
The answer will be 'yes' if the provided sentence contains an explicit mention \\that answers the given question. Otherwise, the answer should be 'no'. Instances \\ where the answer is implied from the sentence using "instinct" or "common \\sense" (as opposed to being written explicitly in the sentence) should be labeled \\as 'no'.
\end{tabular}
\\ 
\midrule
\textbf{Instance}
&
\begin{tabular}[c]{@{}l@{}}
Sentence: Jerry goes out to the pier and casts his favorite bait : cheese .  
\\
Question: How much time did Jerry spend at the pier?
\end{tabular} 
\\
\midrule
\textbf{Golden Labels}
&
\begin{tabular}[c]{@{}l@{}}
[\textcolor{red}{"No."}] \textcolor{red}{\qquad\# Unexpected capital letter and period are included.}
\end{tabular} 
\\
\hline\hline
\textbf{Task ID}
& task738\_perspectrum\_classification
\\
\midrule
\textbf{Instance ID} 
& task738-7bde66d008f6449f90b430dac8a78257
\\ 
\midrule
\textbf{Definition} 
& 
\begin{tabular}[c]{@{}l@{}}
In this task you will be given a claim and a perspective. You should determine \\whether that perspective \textcolor{red}{supports} or \textcolor{red}{undermines} the claim. If the perspective \\could possibly convince someone with different view, it is \textcolor{red}{supporting}, otherwise \\it is \textcolor{red}{undermining}.
\end{tabular}
\\ 
\midrule
\textbf{Instance}
&
\begin{tabular}[c]{@{}l@{}}
claim: Domestic intelligence agencies have a legitimate role to play in democracy.
\\
perspective: The government does not have the right to spy on its citizens
\end{tabular} 
\\
\midrule
\textbf{Golden Labels}
&
\begin{tabular}[c]{@{}l@{}}
[\textcolor{red}{"undermine"}] \textcolor{red}{\qquad\# The label space is ambiguously defined.}
\end{tabular} 
\\
\hline\hline
\textbf{Task ID}
& task1624\_disfl\_qa\_question\_yesno\_classification
\\
\midrule
\textbf{Instance ID} 
& task1624-ef6dff90c9ae4c5fb513d3e0c4f16f19
\\ 
\midrule
\textbf{Definition} 
& 
\begin{tabular}[c]{@{}l@{}}
In this task you are given a disfluent question, a proper question and a context. \\ A disfluent question is a question that has some interruptions in it while framing \\and a proper question is the correct form of the question without any disfluency. \\\textcolor{red}{Classify whether the question is answerable or not} based on the given context.
\end{tabular}
\\ 
\midrule
\textbf{Instance}
&
\begin{tabular}[c]{@{}l@{}}
proper question: Who was the last known European to visit China and return? 
\\disfluent question: What did Polo, no I'm sorry, who was the last known European to 
\\ visit China and return? 
\\context: The first recorded travels by Europeans to China and back date from this time. \\The most famous traveler of the period was the Venetian Marco Polo, \textit{\textcolor{gray}{... \{Skip\}...}}\\Some suggest that Marco Polo acquired much of his knowledge through contact with \\Persian traders since many of the places he named were in Persian.
\end{tabular} 
\\
\midrule
\textbf{Golden Labels}
&
\begin{tabular}[c]{@{}l@{}}
[\textcolor{red}{"No"}] \textcolor{red}{\qquad\# The explicit label space is not defined.}
\end{tabular}
\\
\bottomrule
\end{tabular}
\\
\end{table}
\endgroup
