% - What problem we address?
We propose a technique that tackles action detection in multimodal videos under a realistic and challenging condition in which only limited training data and partially observed modalities are available. 
% - Why other existing approaches cannot address this problem well?
Common methods in transfer learning do not take advantage of the extra modalities potentially available in the source domain. On the other hand, previous work on multimodal learning only focuses on a single domain or task and does not handle the modality discrepancy between training and testing.
% - What we do in this paper? Specify the core ideas and contributions.
In this work, we propose a method termed graph distillation that incorporates rich privileged information from a large-scale multimodal dataset in the source domain, and improves the learning in the target domain where training data and modalities are scarce. 
% - What's the results of this? How do we evaluate the success?
We evaluate our approach on action classification and detection tasks in multimodal videos, and show that our model outperforms the state-of-the-art by a large margin on the NTU RGB+D and PKU-MMD benchmarks. The code is released at \url{http://alan.vision/eccv18_graph/}.

