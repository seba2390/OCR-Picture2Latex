% ****** Start of file apssamp.tex ******
%
%   This file is part of the APS files in the REVTeX 4 distribution.
%   Version 4.0 of REVTeX, August 2001
%
%   Copyright (c) 2001 The American Physical Society
%
%   See the REVTeX 4 README file for restrictions and more information.
%
% TeX'ing this file requires that you have AMS-LaTeX 2.0 installed
% as well as the rest of the prerequisites for REVTeX 4.0
%
% See the REVTeX 4 README file
% It also requires running BibTeX. The commands are as follows:
%
%  1)  latex apssamp.tex
%  2)  bibtex apssamp
%  3)  latex apssamp.tex
%  4)  latex apssamp.tex
%
%\documentclass{article}
\documentclass[twocolumn, a4paper, superscriptaddress,nofootinbib, accepted=2020-08-07, hyperref]{quantumarticle}
%\documentclass[pra,twocolumn, superscriptaddress,nofootinbib]{revtex4}
\pdfoutput=1
\usepackage{enumitem}
\usepackage{graphicx}
\usepackage{bm}
\usepackage{amsmath}
\usepackage{amssymb}
\usepackage{underscore}
\usepackage{color}
%\usepackage{subcaption}
\usepackage{float}
\usepackage[caption = false]{subfig}
\usepackage{amsmath,amsthm,amsfonts,amssymb,amscd}
\newcommand{\kh}[1]{\textcolor{blue}{ #1}}
\newcommand{\com}[1]{\textcolor{magenta}{ comment: #1}}
\def\tred{\textcolor{red}}
\def\Real{{\hbox{\Bbb R}}} \def\C{{\hbox {\Bbb C}}}
\def\spec{R_{\alpha\beta}}
%\def\id {{\hbox{\Bbb I}}}
\def\lipa{lipa}
\newcommand{\EE}{{{\mathbb E}}}
%\usepackage[colorlinks,citecolor=red,urlcolor=blue,bookmarks=false,hypertexnames=true]{hyperref}\documentclass[10pt]{•}
\usepackage{hyperref}
\hypersetup{
    colorlinks=true,
    linkcolor=blue,
    filecolor=magenta,      
    urlcolor=violet,
}

\def\squareforqed{\hbox{\rlap{$\sqcap$}$\sqcup$}}
\def\qed{\ifmmode\squareforqed\else{\unskip\nobreak\hfil
\penalty50\hskip1em\null\nobreak\hfil\squareforqed
\parfillskip=0pt\finalhyphendemerits=0\endgraf}\fi}

\def\kvec{{\bbox{k}}}
\def\lvec{{\bbox{l}}}
\def\duzomniejsze{<\kern-.7mm<}
\def\duzowieksze{>\kern-.7mm>}
\def\intlarge{\mathop{\int}\limits}
\def\textbf#1{{\bf #1}}
\def\beq{\begin{equation}}
\def\eeq{\end{equation}}
\def\be{\begin{equation}}
\def\ee{\end{equation}}
\def\ben{\begin{eqnarray}}
\def\een{\end{eqnarray}}
\def\beqa{\begin{eqnarray}}
\def\eeqa{\end{eqnarray}}
\def\eea{\end{array}}
\def\bea{\begin{array}}
\def\bs{{\backslash}}
\newcommand{\bei}{\begin{itemize}}
\newcommand{\eei}{\end{itemize}}
\newcommand{\bee}{\begin{enumerate}}
\newcommand{\eee}{\end{enumerate}}
\newcommand{\nc}{\newcommand}
\nc{\1}{{\openone}}

\def\xcal{{\cal X}}
\def\ycal{{\cal Y}}
\def\acal{{\cal A}}
\def\hcal{{\cal H}}
\def\dcal{{\cal D}}
\def\pcal{{\cal P}}
\def\ccal{{\cal C}}
\def\bcal{{\cal B}}

\def\tr{{\rm Tr}}
\def\id{{\rm I}}
\def\ra{\rangle}
\def\la{\langle}
\def\>{\rangle}
\def\<{\langle}
\def\blacksquare{\vrule height 4pt width 3pt depth2pt}
\def\ic{I_{coh}}
\def\ot{\otimes}
\def\rhoab{\varrho_{AB}}
\def\sigmaab{\sigma_{AB}}
\def\rhoa{\varrho_{A}}
\def\sigmaa{\sigma_{A}}
\def\rhob{\varrho_{B}}
\def\sigmab{\sigma_{B}}
\def\logneg{E_{N}}
\def\disone{E_D^\to}

\def\wobie{\rightleftharpoons}
\def\sigmasws{\sigma_{sws}}
\def\tosym{S_{sym}}
\def\sym{{\cal S}_E^{as}}
\def\asym{{\cal A}_E^{as}}

\def\asymfin{{\cal A}_E}

\def\abba{S_{swap}}
\def\id{\mathbb{I}}
\def\rateprzezsym{symmetry}
\def\rateswapa{swap-symmetry}

\newtheorem{lemma}{Lemma}
\newtheorem{cor}{Corollary}
\newtheorem{prop}{Proposition}
\newtheorem{theorem}{Theorem}
\newtheorem{dfn}{Definition}
\newtheorem{obs}{Observation}
\newtheorem{Conjecture}{Conjecture}
\newtheorem{rem}{Remark}
\newtheorem{example}{Example}
\newtheorem{fact}{Fact}
\newtheorem{claim}{Claim}
\newtheorem*{definition}{Definition}

\def\bed{\begin{definition}}
\def\eed{\end{definition}}
\def\bel{\begin{lemma}}
\def\eel{\end{lemma}}
\def\bet{\begin{theorem}}
\def\eet{\end{theorem}}
\def\be{\begin{equation}}
\def\ee{\end{equation}}

\allowdisplaybreaks

%\documentclass[prl,superscriptaddress,twocolumn]{revtex4}
%%\documentclass[aps,11pt,twoside, nofootinbib, superscriptaddress]{revtex4}
%\usepackage{amsthm}
%\usepackage{amsmath,latexsym,amssymb,verbatim,enumerate,graphicx}
%%\usepackage{hyperref}
%\usepackage{color}
%
\usepackage{mathtools}
\newcommand{\defeq}{\vcentcolon=}
%\newcommand{\eqdef}{=\vcentcolon}
%
%
%\renewcommand{\familydefault}{ppl}
%\renewcommand{\baselinestretch}{1.0}
% 
%%\usepackage{xypic}
%%\usepackage{theorem}
%\newtheorem{definition}{Definition} %[section]
%\newtheorem{prop}[definition]{Proposition}
%\newtheorem{lemma}[definition]{Lemma}
%\newtheorem{algorithm}[definition]{Algorithm}
%\newtheorem{fact}[definition]{Fact}
%%\newtheorem{theorem}[definition]{Theorem}
%\newtheorem{thm}[definition]{Theorem}
%\newtheorem{corollary}[definition]{Corollary}
%\newtheorem{conjecture}[definition]{Conjecture}
%\newtheorem{obs}[definition]{Observation}
%\newtheorem{remark}[definition]{Remark}
%\newtheorem{claim}[definition]{Claim}
%
%\newtheorem*{rep@theorem}{\rep@title}
%\newcommand{\newreptheorem}[2]{%
%\newenvironment{rep#1}[1]{%
% \def\rep@title{#2 \ref{##1} (restatement)}%
% \begin{rep@theorem}}%
% {\end{rep@theorem}}}
%\makeatother
%
%\newreptheorem{thm}{Theorem}
%\newreptheorem{lem}{Lemma}
%
%
%\def\ba#1\ea{\begin{align}#1\end{align}}
%\def\ban#1\ean{\begin{align*}#1\end{align*}}
%
%%Michal
%\newcommand{\ot}{\otimes}
%%\newcommand{\<}{\langle}
%%\newcommand{\>}{\rangle}
%\newcommand{\be}{\begin{equation}}
%\newcommand{\ee}{\end{equation}}
%\def\hcal{{\cal H}}
%
%\def\tred{\textcolor{black}}
%\def\tgr{\textcolor{black}}
%\def\good{\text{Good}}
%\def\Extr{\text{Extr}}
%\def\cor{\text{Cor}}
%
%\def\Eveinput{\mathbb{W}}
%\def\Eveoutput{z}
%\def\dmaxw{d_{\max}^{w}}  % not used anymore
%\def\dmax{d_{\max}^{\text{no } w}} % not used anymore
%%\def\decintro{d_c}
%\def\dcintro{d_{\text{comp}}}
%%\def\dcours{\tilde d_c}
%\def\dcours{\tilde d_{\text{comp}}}
%\def\dnow{d}
%\def\dwithw{d'}
%%\def\dcintroII{d_c^{\text{II}}}
%\def\dcintroII{d_{\text{comp}}^{\,\text{II}}}
%\def\dcoursII{\tilde d_{\text{comp}}^{\,\text{II}}}
%\def\dnowII{d_{\,\text{II}}}
%\def\dwithwII{d'_{\,\text{II}}}
%\def\ein{\text{\tt in}_E}
%\def\eout{\text{\tt out}_E}
%\def\sv{\text{\tt sv}}
%\def\abin{\text{\tt in}}
%\def\about{\text{\tt out}}
%\def\sve{\text{\tt sv}_E}
%\def\svin{\text{\tt sv}_1}
%\def\svhash{\text{\tt sv}_2}
%
%
%%\newcommand{\bea}{\begin{array}}
%%\newcommand{\eea}{\end{array}}
%
%
%\def\locHam{{\sc local Hamiltonian}}
%\def\gappedlocHam{{\sc Gapped local Hamiltonian}}
%\def\partition{{\sc Partition Function}}
%\def\spec{{\sc Spectral Density}}
%\def\BQP{{\sf{BQP}}}
%\def\PH{{\sf{PH}}}
%\def\COLORING{{\sf{COLORING}}}
%\def\FACTORING{{\sf{FACTORING}}}
%\def\MAJORITY{{\sf{MAJORITY}}}
%\def\FC{\textsc{Fourier Checking}}
%\def\CC{\textsc{Circuit Checking}}
%%\def\FC{{\sf{Fourier \hspace{0.08 cm} Checking}}}
%%\def\CC{{\sf{Circuit \hspace{0.08 cm} Checking}}}
%\def\BPP{{\sf{BPP}}}
%\def\QCMA{{\sf{QCMA}}}
%\def\SZK{{\sf{SZK}}}
%\def\QMA{{\sf{QMA}}}
%\def\UQCMA{{\sf{UQCMA}}}
%\def\BellQMA{{\sf{BellQMA}}}
%\def\LOCCQMA{{\sf{LOCCQMA}}}
%\def\BPTIME{{\sf{BPTIME}}}
%\def\DQC{{\sf{DQC1}}}
%\def\NP{{\sf{NP}}}
%\def\UNP{{\sf{UNP}}}
%\def\UcoNP{{\sf{Uco-NP}}}
%\def\MA{{\sf{MA}}}
%\def\A{{\sf{A}}}
%\def\PSPACE{{\sf{PSPACE}}}
%\def\P{{\sf{P}}}
%\def\U{{\sf{U}}}
%\def\PP{{\sf{PP}}}
%%\def\poly{{\sf{poly}}}
%\def\SAT{{\sf{SAT}}}
%\def\YES{{\sf{YES}}}
%\def\NO{{\sf{NO}}}
%\def\RFS{{\sf{RFS }}}
%\def\benum{\begin{enumerate}}
%\def\eenum{\end{enumerate}}
%
%%\renewcommand{\thetheorem}{\Roman{theorem}}
%
%\def\lpolskie{\l}
%
%\def\nn{\nonumber}
%\def\norm#1{ {|\hspace{-.022in}|#1|\hspace{-.022in}|} }
%\def\Norm#1{ {\big|\hspace{-.022in}\big| #1 \big|\hspace{-.022in}\big|} }
%\def\NOrm#1{ {\Big|\hspace{-.022in}\Big| #1 \Big|\hspace{-.022in}\Big|} }
%\def\NORM#1{ {\left|\hspace{-.022in}\left| #1 \right|\hspace{-.022in}\right|} }
%
%\def\squareforqed{\hbox{\rlap{$\sqcap$}$\sqcup$}}
%\def\qed{\ifmmode\squareforqed\else{\unskip\nobreak\hfil
%\penalty50\hskip1em\null\nobreak\hfil\squareforqed
%\parfillskip=0pt\finalhyphendemerits=0\endgraf}\fi}
%\def\endenv{\ifmmode\;\else{\unskip\nobreak\hfil
%\penalty50\hskip1em\null\nobreak\hfil\;
%\parfillskip=0pt\finalhyphendemerits=0\endgraf}\fi}
%%\newenvironment{proof}{\noindent \textbf{{Proof~} }}{\qed}
%%\newenvironment{proof}[1][Proof]{\noindent \textbf{{#1~} }}{\qed}
%%\newenvironment{remark}{\noindent \textbf{{Remark~}}}{\qed}
%\newenvironment{example}{\noindent \textbf{{Example~}}}{\qed}
%
%%\font\gensymbols=drgen10
%%\def\male{{\gensymbols\char"1A}}
%%\def\female{{\gensymbols\char"19}}
%
%\newcommand{\exampleTitle}[1]{\textsl{}bf{(#1)}}
%\newcommand{\remarkTitle}[1]{\textbf{(#1)}}
%\newcommand{\proofComment}[1]{\exampleTitle{#1}}
%
%\DeclareUnicodeCharacter{2009}{\,} 

%
%\newcommand{\bra}[1]{\langle #1|}
%\newcommand{\ket}[1]{|#1\rangle}
%\newcommand{\braket}[2]{\langle #1|#2\rangle}
%\newcommand{\tr}{\text{tr}}
%
%%\newcommand{\nn}{\mathbb{N}}
%\newcommand{\bbR}{\mathbb{R}}
%\newcommand{\bbC}{\mathbb{C}}
%\newcommand{\bbU}{\mathbb{U}}
%\newcommand{\id}{\mathbb{I}}
%\newcommand{\proofend}{\hfill$\square$\par\vskip24pt}
%%\newcommand{\ben}{\begin{equation}}
%%\newcommand{\een}{\end{equation}}
%
%\newcommand{\<}{\langle}
%\renewcommand{\>}{\rangle}
%\newcommand{\I}{{\rm I}}
%\def\L{\left}
%\def\R{\right}
%
%\def\id{{\operatorname{id}}}
%%\def\I {{\hbox{\Bbb I}}}
%%\def\I{{\rm I}}
%\DeclareMathOperator{\ad}{ad}
%\DeclareMathOperator{\sym}{sym}
%\DeclareMathOperator{\rank}{rank}
%\DeclareMathOperator{\supp}{supp}
%\def\be{\begin{equation}}
%\def\ee{\end{equation}}
%\def\ben{\begin{eqnarray}}
%\def\een{\end{eqnarray}}
%\def\ot{\otimes}
%\def\tred{\textcolor{red}}
%\def\bei{\begin{itemize}}
%\def\eei{\end{itemize}}
%
%\def\E{{\mathbb{E}}}
%\def\C{{\mathbb{C}}}
%\def\F{{\mathbb{F}}}
%\def\I{{\mathbb{I}}}
%
%%\defR_1{W_1}
%%\defR_2{W_2}
%
%\def\ep{\epsilon}
%\def\eps{\epsilon}
%\def\ecal{{\cal E}}
%\def\hcal{{\cal H}}
%\def\gcal{{\cal G}}
%
%\newcommand{\mbp}{$\spadesuit$}
%
%% Align := properly in math mode
%\mathchardef\ordinarycolon\mathcode`\:
%\mathcode`\:=\string"8000
%\def\vcentcolon{\mathrel{\mathop\ordinarycolon}}
%\begingroup \catcode`\:=\active
%  \lowercase{\endgroup
%  \let :\vcentcolon
%  }
%
%\newcommand{\nc}{\newcommand}
%%\nc{\rnc}{\renewcommand} \nc{\beq}{\begin{equation}}
%%\nc{\eeq}{{\end{equation}}} \nc{\bea}{\begin{eqnarray}}
%%\nc{\eea}{\end{eqnarray}} \nc{\beqa}{\begin{eqnarray}}
%%\nc{\eeqa}{\end{eqnarray}} \nc{\lbar}[1]{\overline{#1}}
%%\nc{\bra}[1]{\langle#1|} \nc{\ket}[1]{|#1\rangle}
%%\nc{\ketbra}[2]{|#1\rangle\!\langle#2|}
%%\nc{\braket}[2]{\langle#1|#2\rangle}
% \nc{\proj}[1]{|#1\rangle\!\langle #1 |} 
%\nc{\avg}[1]{\langle#1\rangle}
%\def\Bra#1{{\big\langle #1 \big|  }}
%\def\Ket#1{{\big| #1 \big\rangle }}
%\def\BRA#1{{\left\langle #1 \right|  }}
%\def\KET#1{{\left| #1 \right\rangle }}
%%\rnc{\max}{\operatorname{max}} \nc{\rank}{\operatorname{rank}}
%\nc{\conv}{\operatorname{conv}}
%\nc{\smfrac}[2]{\mbox{$\frac{#1}{#2}$}} \nc{\Tr}{\operatorname{Tr}}
%\nc{\ox}{\otimes} \nc{\dg}{\dagger} \nc{\dn}{\downarrow}
%\nc{\lmax}{\lambda_{\text{max}}}
%\nc{\lmin}{\lambda_{\text{min}}}
%
%
%
%
%\nc{\csupp}{{\operatorname{csupp}}}
%\nc{\qsupp}{{\operatorname{qsupp}}} \nc{\var}{\operatorname{var}}
%\nc{\rar}{\rightarrow} \nc{\lrar}{\longrightarrow}
%\nc{\poly}{\operatorname{poly}}
%\nc{\polylog}{\operatorname{polylog}} \nc{\Lip}{\operatorname{Lip}}
%%\nc{\1}{\openone} \nc{\supp}{{\operatorname{supp}}}
%%\nc{\ep}{\epsilon}
%\nc{\Om}{\Omega}
%\nc{\wt}[1]{\widetilde{#1}}
%
%\def\>{\rangle}
%\def\<{\langle}
%
%\def\a{\alpha}
%\def\b{\beta}
%\def\g{\gamma}
%\def\d{\delta}
%\def\e{\epsilon}
%\def\ve{\varepsilon}
%\def\z{\zeta}
%%\def\h{\eta}
%\def\t{\theta}
%%\def\i{s}
%\def\k{\kappa}
%%\def\l{\lambda}
%\def\m{\mu}
%\def\n{\nu}
%\def\x{\xi}
%\def\p{\pi}
%\def\r{\rho}
%\def\s{\sigma}
%\def\ta{\tau}
%\def\u{\upsilon}
%\def\ph{\varphi}
%\def\ps{\psi}
%\def\o{\omega}
%\def\om{\omega}
%
%\def\G{\Gamma}
%\def\D{\Delta}
%\def\T{\Theta}
%%\def\L{\Lambda}
%\def\X{\Xi}
%
%\def\S{\Sigma}
%\def\Ph{\Phi}
%\def\Ps{\Psi}
%\def\O{\Omega}
%\def\Psd{{\Psi_{\!\delta}}}
%\def\Pst{{\widetilde{\Psi}}}
%\def\Pstd{{\widetilde{\Psi}_{\!\delta}}}
%
%\def\tred{\textcolor{red}}
%\def\EE{\mathbb{E}}
%\def\cor{\text{Cor}}
%
%\def\tred{\textcolor{red}}
%\def\cor{\text{Cor}}
%%\def\bu{{\it \textbf{u}}}
%\def\bu{{\textbf{u}}}
%\def\bx{\textbf{x}}
%%\def\pxu{P_{\textbf{x}_{< l}, \textbf{u}_{< l}, \Eveinput = w}(\textbf{x}_l | \textbf{u}_l, \Eveoutput)}
%\def\pxu{P_{{x}_{< l},{u}_{< l}}({x}_l | {u}_l)}
%\def\pxutot {P(x_1,\ldots,x_n|u_1,\ldots,u_n)}
%%\def\pxutot {P(\bx_1,\ldots,\bx_n|\bu_1,\ldots,\bu_n,\Eveinput=w, \mathbb{Z} = \Eveoutput)}
%%\def\xuseq{\Eveoutput,w,x_1,u_1,\ldots,x_n,u_n}
%\def\xuseq{x_1,u_1,\ldots,x_n,u_n}
%\def\useq{u_1,\ldots,u_n}
%\def\xseq{x_1,...,x_n}
%\def\Bell{B}
%
%
%\nc{\glneq}{{\raisebox{0.6ex}{$>$}  \hspace*{-1.8ex} \raisebox{-0.6ex}{$<$}}}
%\nc{\gleq}{{\raisebox{0.6ex}{$\geq$}\hspace*{-1.8ex} \raisebox{-0.6ex}{$\leq$}}}
%
%
%%\nc{\id}{{\operatorname{id}}}
%
%\nc{\vholder}[1]{\rule{0pt}{#1}}
%\nc{\wh}[1]{\widehat{#1}}
%\nc{\h}[1]{\widehat{#1}}
%
%\nc{\ob}[1]{#1}
%
%\def\beq{\begin {equation}}
%\def\eeq{\end {equation}}
%
%%Michal defs
%\def\be{\begin{equation}}
%\def\ee{\end{equation}}
%
%
%\def\Ubf{\textbf{U}}
%\def\Xbf{\textbf{X}}
%\def\Ubar{\bar{U}}
%\def\Mwithj{\tilde M^j_{}}
%\def\Mk{\tilde M_{\neq k}}
%\def\Mo{\tilde M_{\neq 1}}
%\def\Mj{\tilde M_{\neq j}}
%\def\Mwithjk{M^j_{\neq k}}
%\def\Mwithjo{\tilde M^j_{\neq 1}}
%\def\Mwithjj{\tilde M^j_{\neq j}}
%\def\Setadef{S_{deF}^{\eta}}
%%\def\setadefnc{S^{\xi}_{deF}}
%%\def\setadefnco{S^{\xi,1}_{deF}}
%%\def\setadefncj{S^{\xi,j}_{deF}}
%%\def\setadefnck{S^{\xi,k}_{deF}}
%\def\setadefnc{S^{\xi}}
%\def\setadefnco{S^{\xi,1}}
%\def\setadefncj{S^{\xi,j}}
%\def\setadefnck{S^{\xi,k}}
%\def\Ddef{D_{\epdef}}
%%\de\deltadef{\delta_{deF}}
%\def\epazdef{\ep_{Az}}
%\def\faildef{\text{FAIL}^{\delta_f}_{deF}}
%\def\distdef{d_{deF}}
%\def\epdef{\ep_{deF}}
%%\def\ACCdef{\text{ACC}^{\delta,{deF}}}
%%\def\ACCdef{\text{ACC}_{{deF}}}
%\def\ACC{\text{ACC}}
%\def\ACCdef{\text{ACC}}
%\def\ACCdefo{\text{ACC}^1}
%\def\ACCdefj{\text{ACC}^j}
%\def\ACCdefk{\text{ACC}^k}
%%\def\ACCd{\text{ACC}^\delta}
%\def\ACCd{\text{ACC}}
%\def\ACCu{\text{ACC}_{\Ubf}}
%\def\ACCud{\text{ACC}_{\Ubf}^{\delta}}
%\def\FAIL{\text{FAIL}}
%\def\FAILd{\text{FAIL}^{\delta_f}}
%
%\def\Xnot{X_{\not=1,j}}
%
%\def\FAILdnc{\text{FAIL}^{\delta_f}_{}}
%\def\Xgoodnc{X^{\textbf{U}}_{\text{good}}}
%\def\Xbadnc{X^{\textbf{U}}_{\text{bad}}}
%
%%\def\FAILU{\text{F}}
%\def\Lazuma{L}
%%\def\Lazumadef{\overline{L}_{deF}}
%\def\Lazumadef{\overline{L}}
%\def\Lazumanc{\overline{L}}
%\def\Lunif{\overline{L}_{\text{unif}}}
%\def\Bunif{\overline{B}_i^{\text{unif}}}
%\def\Lunifz{\overline{L}_{z,\text{unif}}}
%\def\Bunifz{\overline{B}_i^{z,\text{unif}}}
%\def\Zgood{Z_{\text{good}}}
%\def\Xgood{X^{(z,u,e)}_{\text{good}}}
%\def\Xbad{X^{(z,u,e)}_{\text{bad}}}
%\def\Azuma{A}
%\def\Azumad{A^{\delta_e}}
%\def\Azumadc{(A^{\delta_e})^c}
%\def\Azumadnc{A^{\delta_{Az}}}
%\def\Azumadcnc{(A^{{\delta_e}})^c}
%\def\Ad{A^{\delta_e}}
%\def\Adef{A^{\delta_{Az}}}
%\def\Adefo{A^{\delta_{Az},1}}
%\def\Adefj{A^{\delta_{Az},j}}
%\def\Adefk{A^{\delta_{Az},k}}
%
%
%\def\Seta{S^{\eta}}
%\def\Setac{(S^\eta)^c}
%\def\Rest{Rest}
%
%\def\ACCdeF{ACC}
%\def\deltadeF{\delta}
%
%
%\nc{\eq}[1]{(\ref{eq:#1})} 
%\nc{\eqs}[2]{\eq{#1} and \eq{#2}}
%
%\nc{\eqn}[1]{Eq.~(\ref{eqn:#1})}
%\nc{\eqns}[2]{Eqs.~(\ref{eqn:#1}) and (\ref{eqn:#2})}
%\newcommand{\fig}[1]{Fig.~\ref{fig:#1}}
%\newcommand{\secref}[1]{Section~\ref{sec:#1}}
%\newcommand{\appref}[1]{Appendix~\ref{sec:#1}}
%\newcommand{\lemref}[1]{Lemma~\ref{lem:#1}}
%\newcommand{\thmref}[1]{Theorem~\ref{thm:#1}}
%\newcommand{\propref}[1]{Proposition~\ref{prop:#1}}
%\newcommand{\protoref}[1]{Protocol~\ref{proto:#1}}
%\newcommand{\defref}[1]{Definition~\ref{def:#1}}
%\newcommand{\corref}[1]{Corollary~\ref{cor:#1}}
%
%\newcommand{\mh}[1]{\textcolor{blue}{\tt michal: #1}}
%
%\nc{\region}{\cS\cW}
%%%%%%%%%%%%%%%%%%%%%%%%%%%
%\newcommand{\xxx}{$\clubsuit$}
%\newcommand{\yyy}{$\spadesuit$}
%%\nofiles
%
%\newenvironment{protocol*}[1]
%  {
%    \begin{center}
%      \hrulefill\\
%      \textbf{#1}
%  }
%  {
%    \vspace{-1\baselineskip}
%    \hrulefill
%    \end{center}
%  }
\begin{document}

%\preprint{}

\title{Gadget structures in proofs of the Kochen-Specker theorem}
%\title{Randomness Witnessing from local contextuality via coloring gadgets}
\author{Ravishankar Ramanathan}
%\email{ravi@cs.hku.hk}
\affiliation{Department of Computer Science, The University of Hong Kong, Pokfulam Road, Hong Kong}
%\affiliation{Laboratoire d'Information Quantique, Universit\'{e} Libre de Bruxelles, Belgium}
%\affiliation{National Quantum Information Center of Gda\'{n}sk, 81-824 Sopot, Poland}
%\affiliation{Institute of Theoretical Physics and Astrophysics, University of Gda\'{n}sk, 80-952 Gda\'{n}sk, Poland}

\author{Monika Rosicka}
\affiliation{Institute of Theoretical Physics and Astrophysics and the National Quantum Information Centre, Faculty of Mathematics, Physics and Informatics, 
	University of Gdansk, 80-308 Gdansk, Poland.}

\author{Karol Horodecki}
\affiliation{Institute of Informatics Faculty of Mathematics, Physics and Informatics,
		University of Gdansk, 80-308 Gdansk, Poland}
\affiliation{International Centre for Theory of Quantum Technologies,
		University of Gdansk, Wita Stwosza 63, 80-308 Gdansk, Poland}	
%\affiliation{National Quantum Information Center of Gda\'{n}sk, 81-824 Sopot, Poland}
%\affiliation{Institute of Informatics, University of Gda\'{n}sk, 80-952 Gda\'{n}sk, Poland}

\author{Stefano Pironio}
\affiliation{Laboratoire d'Information Quantique, Universit\'{e} Libre de Bruxelles, Belgium}

\author{Micha{\l} Horodecki}
\affiliation{International Centre for Theory of Quantum Technologies, University of Gda\'{n}sk, Wita Stwosza 63, 80-308 Gda\'{n}sk, Poland}
\affiliation{Institute of Theoretical Physics and Astrophysics and the National Quantum Information Centre, Faculty of Mathematics, Physics and Informatics, 
	University of Gdansk, 80-308 Gdansk, Poland.}


\author{Pawe{\l} Horodecki}
\affiliation{International Centre for Theory of Quantum Technologies, University of Gda\'{n}sk, Wita Stwosza 63, 80-308 Gda\'{n}sk, Poland}
\affiliation{Faculty of Applied Physics and Mathematics, National Quantum Information Centre, Gda\'{n}sk University of Technology, Gabriela Narutowicza 11/12, 80-233 Gda\'{n}sk, Poland} 
%\affiliation{Faculty of Applied Physics and Mathematics
%	and the National Quantum Information Centre, Gdansk University of
%	Technology, 80-233 Gdansk, Poland.}
%\affiliation{Faculty of Applied Physics and Mathematics, Technical University of Gda\'{n}sk, 80-233 Gda\'{n}sk, Poland}



\begin{abstract}
	The Kochen-Specker theorem is a fundamental result in quantum foundations that has spawned massive interest since its inception. We show that within every Kochen-Specker graph, there exist interesting subgraphs which we term $01$-gadgets, that capture the essential contradiction necessary to prove the Kochen-Specker theorem, i.e,. every Kochen-Specker graph contains a $01$-gadget and from every $01$-gadget one can construct a proof of the Kochen-Specker theorem. Moreover, we show that the $01$-gadgets form a fundamental primitive that can be used to formulate state-independent and state-dependent statistical Kochen-Specker arguments as well as to give simple constructive proofs of an ``extended'' Kochen-Specker theorem first considered by Pitowsky in \cite{Pitowsky}.  
\end{abstract}

\maketitle

%\date{\today}

\section{Introduction}
According to the quantum formalism, a projective measurement $M$ is described by a set $M=\{V_1,\ldots,V_m\}$ of projectors $V_i$ in a complex Hilbert space, that are orthogonal, $V_iV_j=\delta_{ij} V_i$, and sum to the identity, $\sum_i V_i=I$. Each $V_i$ corresponds to a possible outcome $i$ of the measurement $M$ and determines the probability of this outcome when measuring a state $|\psi\rangle$ through the formula $\text{Pr}_{\psi}(i\mid M)=\langle\psi|V_i|\psi\rangle$. 

If two physically distinct measurements $M=\{V_1,\ldots,V_m\}$ and $M'=\{V'_1,\ldots,V'_{m'}\}$ share a common projector, i.e., $V_i=V'_{i'}=V$ for some outcome $i$ of $M$ and $i'$ of $M'$, it then follows that
\begin{equation}\label{eq:meas}
\text{Pr}_{\psi}(i\mid M)=\text{Pr}_{\psi}(i'\mid M')=\langle\psi|V|\psi\rangle\,.
\end{equation}
In other words, though quantum measurements are defined by \emph{sets} of projectors, the outcome probabilities of these measurements are determined by the \emph{individual} projectors alone, independently of the broader set -- or the \emph{context} -- to which they belong. We say that the probability assignment is \emph{non-contextual}.

The Kocken-Specker (KS) theorem \cite{KS} is a cornerstone result in the foundations of quantum mechanics, establishing that, in Hilbert spaces of dimension greater than two, it is not possible to find a \emph{deterministic} outcome assignment that is non-contextual. Deterministic means that all outcome probabilities should take only the values $0$ or $1$. Non-contextual means, as above, that these probabilities are not directly assigned to the measurements themselves, but to the individual projectors from which they are composed, independently of the context to which the projectors belong. More formally, the KS theorem establishes that it is not possible to find a rule $f$ such that
\begin{equation}\label{eq:measlambda}
\text{Pr}_{f}(i\mid M)=\text{Pr}_{f}(i'\mid M')=f(V)\in\{0,1\}\,,
\end{equation}
which would provide a deterministic analogue of a quantum state.

The most common way to prove the KS theorem involves a set $\mathcal{S}=\{V_1,\ldots,V_n\}$ of rank-one projectors in a complex Hilbert space.
We can represent these projectors by the vectors (strictly speaking, the rays) onto which they project and thus view $\mathcal{S}$ as a set of vectors $\mathcal{S}=\{|v_1\rangle,\ldots,|v_n\rangle\}\subset\mathbb{C}^d$. Consider an assignment $f:\mathcal{S}\rightarrow \{0,1\}$ that associates to each $|v_i\rangle$ in $\mathcal{S}$ a probability $f(|v_i\rangle)\in\{0,1\}$.
To interpret the $f(|v_i\rangle)$ as valid measurement outcome probabilities, they should satisfy the two following conditions:
\begin{flalign}\label{eq:01rule}
\begin{minipage}{.42\textwidth}
%{\parbox{.36\textwidth}{
\begin{itemize}
		\item $\sum_{|v\rangle\in \mathcal{O}} f(|v\rangle)\leq 1$ for every set $\mathcal{O}\subseteq \mathcal{S}$ of mutually orthogonal vectors;	
		\item $\sum_{|v\rangle\in \mathcal{B}} f(|v\rangle)=1$ for every set $\mathcal{B}\subseteq\mathcal{S}$ of $d$ mutually orthogonal vectors.
		\end{itemize}
\end{minipage}&&
\end{flalign}
The first condition is required because if a set of vectors are mutually orthogonal, they may be part of the same measurement, but then their corresponding probabilities must sum at most to 1. The second condition follows from the fact that if $d$ vectors are mutually orthogonal in $\mathbb{C}^d$, they form a complete basis, and then their corresponding probabilities must exactly sum to one. Note that the first condition implies in particular that any two  vectors $|v_1\rangle$ and $|v_2\rangle$ in $\mathcal{S}$ that are orthogonal cannot both be assigned the value 1 by $f$.

We call any assignment $f : \mathcal{S} \rightarrow \{0,1\}$ satisfying the above two conditions, a $\{0,1\}$-coloring of $\mathcal{S}$. The Kocken-Specker theorem states that if $d\geq 3$, there exist sets of vectors that are not $\{0,1\}$-colorable, thus establishing the impossibility of a non-contextual deterministic outcome assignment. We call such $\{0,1\}$-uncolorable sets, KS sets. In their original proof, Kochen and Specker describe a set $\mathcal{S}$ of 117 vectors in $\mathbb{C}^d$ dimension $d=3$ \cite{KS}. The minimal KS set contains 18 vectors in dimension $d=4$ \cite{CEG96, Cab08}. 	

%\footnote{Note that strictly speaking a $\{0,1\}$-coloring is a coloring of the rank-one projectors $|v_i\rangle\langle v_i|$ corresponding to the vectors $|v_i\rangle$. It thus follows that vectors differing only by a  phase are assigned the same value, i.e., $f(|v_i\rangle)=f(e^{i\alpha}|v_i\rangle)$. Alternatively, we can implicitly assume that the set $\mathcal{S}$ that we consider here do not contain vectors differing only by a phase.} 

In this paper, we identify within KS sets interesting subsets which we term $01$-gadgets. Such $01$-gadgets are $\{0,1\}$-colorable and thus do not represent by themselves KS sets. However, they do not admit arbitrary $\{0,1\}$-coloring: in any $\{0,1\}$-coloring of a $01$-gadget, there exist two non-orthogonal vectors $|v_1\rangle$ and $|v_2\rangle$ that cannot both be assigned the color 1. We show that such $01$-gadgets form the essence of the KS contradiction, in the sense that every KS set contains a $01$-gadget and from every $01$-gadget one can construct a KS set.

Besides being useful in the construction of KS sets, we show that $01$-gadgets also form a fundamental primitive in constructing statistical KS arguments \`a la Clifton \cite{Clifton93} and state-independent non-contextuality inequalities as introduced in \cite{YO12}. Moreover, we show that an ``extended" Kochen-Specker theorem considered by Pitowsky \cite{Pitowsky} and Abbott et al. \cite{ACS15, ACCS12} can be easily proven using an extension of the notion of $01$-gadgets. We give simple constructive proofs of these different results. 

Certain $01$-gadgets have already been studied previously in the literature, as they possess other interesting properties. In particular, $01$-gadgets were also used in \cite{Arends09} to show that the problem of checking whether certain families of graphs (which represent natural candidates for KS sets) are $\{0,1\}$-colorable is NP-complete, a result which we refine in the present paper. Specific $01$-gadgets have already been studied in the literature, for instance as 'definite prediction sets' in \cite{CA96} and recently as 'true-implies-false sets' in \cite{APSS18} where also minimal constructions in several dimensions were explored. A first method to produce different $01$-gadgets was also shown in \cite{CG95}. 

%Besides being deeply linked to KS sets, we show that $01$-gadgets also give rise to non-locality proofs. It is well-known that every Kochen-Specker proof gives rise to a two-party non-local pseudo-telepathy game \cite{HR83, BBT05, RW04}  %\tred{Shouldn't we also cite an old paper from the 80s?}. 
%We show that analogously, $01$-gadgets give rise to two-party Hardy paradoxes and we use them to construct Hardy paradoxes with the non-zero probability taking any value in $(0,1]$. Besides its fundamental interest, this construction finds application in device-independent protocols for randomness amplification, which we explicitly show by bounding the randomness that can be certified against a no-signaling adversary for a certain class of our new Hardy paradoxes.


This paper is organized as follows. In section~\ref{sec:prel}, we introduce some notation and elementary concepts, in particular the representation of KS sets as graphs. In section~\ref{sec:gadget}, we define the notion of $01$-gadgets and establish their relation to KS sets. In section~\ref{sec:constr}, we give several constructions of $01$-gadgets and associated KS sets. In section~\ref{sec:real}, we show how $01$-gadgets can be used to construct statistical KS arguments. In section~\ref{sec:ext}, we also show a simple constructive proof of the extended Kochen-Specker theorem of Pitowsky \cite{Pitowsky} and Abbott et al. \cite{ACCS12} using a notion of extended $01$-gadgets which we introduce.  In section~\ref{sec:compl}, we show that $01$-gadgets can be used to establish the NP-completeness of $\{0,1\}$-coloring of the family of graphs relevant for KS proofs. 
%In section~\ref{sec:hardy}, we show how to construct Hardy paradoxes from $01$-gadgets and study the corresponding randomness in section~\ref{sec:rand}. 
%In section~\ref{sec:color}, we introduce coloring gadgets and establish their relation to KS graphs. In section~\ref{sec:SIC-graphs}, we show how in certain dimensions, the chromatic number gives a simple condition for identifying Kochen-Specker graphs. 
We finish by a general discussion and conclusion in section~\ref{sec:concl}. 




%In this paper we consider the traditional formulation of the KS theorem and show that despite these not possessing the property of value indefiniteness stated above, one may still extract randomness from them. Moreover, by considering the non-local Kochen-Specker game corresponding to the KS sets, one may also obtain this randomness in a device-independent fashion. Furthermore, the same device-independent extraction of randomness can be done even with much smaller sets of vectors that only provide a state-dependent test of contextuality. Note that the non-local task corresponding to a state-dependent set is no longer a Kochen-Specker game.    
%Consequently, as shown in \cite{}, there  sets of vectors satsifying Theorem \ref{thm:ACS14} is a 


\section{Preliminaries}\label{sec:prel}
Much of the reasoning involving KS sets is usually carried out using a graph representation of KS sets defined below. We thus start by reminding some basic graph-theoretic definitions.

\paragraph*{Graphs.}
Throughout the paper, we will deal with simple undirected finite graphs $G$, i.e., finite graphs without loops, multi-edges or directed edges. We denote $V(G)$ the vertices of $G$ and $E(G)$ the edges of $G$. If two vertices $v_1,v_2$ are connected by an edge, we say that they are adjacent, and write $v_1\sim v_2$.


A subgraph $H$ of $G$ (denoted $H < G$) is a graph formed from a subset of vertices and edges of $G$, i.e., $V(H) \subseteq V(G)$ and $E(H) \subseteq E(G)$. An induced subgraph $K$ of $G$ (denoted $K \lhd G$) is a subgraph that includes all the edges in $G$ whose endpoints belong to the vertex subset $V(K) \subseteq V(G)$, i.e., $E(K) \subseteq E(G)$ with $(v_1, v_2) \in E(K)$ iff $(v_1, v_2) \in E(G)$ for all $v_1, v_2 \in V(K)$.  

A clique in the graph $G$ is a subset of vertices $Q \subset V(G)$ such that every pair of vertices in $Q$ is connected by an edge, i.e., $\forall v_1, v_2 \in Q$ we have $v_1 \sim v_2$. A maximal clique in $G$ is a clique that is not a subset of a larger clique in $G$. 
A maximum clique in $G$ is a clique that is of maximum size in $G$. The clique number $\omega(G)$ of $G$ is the cardinality of a maximum clique in $G$.
 
\paragraph*{Orthogonality graphs.}
The use of graphs in the context of the KS theorem comes from the fact that it is convenient to represent the orthogonality relations in a KS set $\mathcal{S}$ by a graph $G_\mathcal{S}$, known as its orthogonality graph \cite{CSW10, CSW14}. In such a graph, each vector $|v_i\rangle$ in $\mathcal{S}$ is represented by a vertex $v_i$ of $ G_{\mathcal{S}}$ and two vertices $v_1, v_2$ of $G_{\mathcal{S}}$ are connected by an edge if the associated vectors $|v_1 \rangle, |v_2 \rangle$ are orthogonal, i.e. $v_1\sim v_2$ if $\langle v_1 | v_2 \rangle = 0$ (for instance the graph in Fig.~\ref{fig:Clifton} is the orthogonality graph of the set of vectors given by eq.~(\ref{eq:Clif-orth-rep})).

It follows that in an orthogonality graph $G_{\mathcal{S}}$, a clique corresponds to a set of mutually orthogonal vectors in $\mathcal{S}$. If $\mathcal{S}\subset\mathbb{C}^d$ contains a basis set of $d$ orthogonal vectors, then the maximum clique in $G_{\mathcal{S}}$ is of size $\omega(G_{\mathcal{S}}) = d$. 

\paragraph*{Coloring of graphs.}
The problem of $\{0,1\}$-coloring $\mathcal{S}$ thus translates into the problem of coloring the vertices of its orthogonality graph $G_\mathcal{S}$ such that vertices connected by an edge cannot both be assigned the color 1 and maximum cliques have exactly one vertex of color~1. Formally, we say that an arbitrary graph $G$ is $\{0,1\}$-colorable if there exists an assignment $f : V(G) \rightarrow \{0,1\}$ such that 
\begin{flalign}\label{eq:01rulegraph}
\begin{minipage}{0.42\textwidth}
\begin{itemize}
	\item $\sum_{v \in Q} f(v) \leq 1$ for every clique $Q \subset V(G)$;
	\item $\sum_{v \in Q_{\max}} f(v) = 1$ for every maximum clique $Q_{\max} \subset V(G)$.
\end{itemize}
\end{minipage} &&
\end{flalign}
The KS theorem is then equivalent to the statement that there exist for any $d\geq 3$, finite sets of vectors $\mathcal{S} \subset \mathbb{C}^d$ (the KS sets)	such that their orthogonality graph $G_{\mathcal{S}}$ is not $\{0,1\}$-colorable. 
Deciding if a given graph $G$ admits a $\{0,1\}$-coloring is NP-complete \cite{Arends09}.
Note that any graph $G$ that is not $\{0,1\}$-colorable must contain at least two cliques of maximum size $\omega(G)$. Indeed, if a graph $G$ contains a single clique of maximum size it always admits a $\{0,1\}$-coloring consisting in assigning the value 0 to all its vertices, except for one vertex in the maximum clique that is assigned the value 1.

\paragraph*{Orthogonal representations.}
For a given graph $G$, an orthogonal representation $\mathcal{S}$ of $G$ in dimension $d$ is a set of non-zero vectors $\mathcal{S}=\{|v_i \rangle\}$ in $\mathbb{C}^d$ obeying the orthogonality conditions imposed by the edges of the graph, i.e., $v_1 \sim v_2 \Rightarrow \langle v_1|v_2 \rangle=0$ \cite{Lovasz87}. We denote by $d(G)$ the minimum dimension of an orthogonal representation of $G$ and we say that $G$ has dimension $d(G)$. Obviously, $d(G)\geq \omega(G)$. A \emph{faithful} orthogonal representation of $G$ is given by a set of vectors $\mathcal{S}=\{|v_i \rangle\}$ that in addition obey the condition that non-adjacent vertices are assigned non-orthogonal vectors, i.e., $v_1 \sim v_2 \Leftrightarrow  \langle v_1|v_2 \rangle=0$ and that distinct vertices are assigned different vectors, i.e., $v_1 \neq v_2 \Leftrightarrow |v_1 \rangle \neq |v_2 \rangle$.
% We will also be concerned with a notion of \textit{strictly faithful orthogonal representation} of $G$ where in addition to the above two conditions, we impose the requirement that distinct vertices are assigned different vectors, i.e., $v_1 \neq v_2 \Leftrightarrow |v_1 \rangle \neq |v_2 \rangle$. 
We denote by $d^*(G)$ the minimum dimension of such a faithful orthogonal representation of $G$ and we say that $G$ has faithful dimension $d^*(G)$.

Given a graph $G$ of dimension $d(G)$, the orthogonality graph $G_\mathcal{S}$ of the minimal orthogonal representation $\mathcal{S}$ of $G$ has faithful dimension $d^*(G_\mathcal{S})=d(G)$. The graph $G_\mathcal{S}$ can be seen as obtained from $G$ by adding edges (between vertices that are non-adjacent in $G$, but corresponding to vectors in $\mathcal{S}$ that are nevertheless orthogonal) and by identifying certain vertices (those that correspond to identical vectors in $\mathcal{S}$). We say that $G_\mathcal{S}$ is the faithful version of $G$.


\paragraph*{KS graphs.}
While the non-$\{0,1\}$-colorability of a set $S$ translates into the non-$\{0,1\}$-colorability of its orthogonality graph $G_\mathcal{S}$, the non-$\{0,1\}$-colorability of an arbitrary graph $G$ translates into the non-$\{0,1\}$-colorability of one of its orthogonal representations only if this representation has the minimal dimension $d(G)=\omega(G)$. Indeed, it is only under this condition that the requirement that $\sum_{v \in Q_{\max}} f(v) = 1$ in the definition of the $\{0,1\}$-coloring of the graph $G$ gives rise to the corresponding requirement that $\sum_{v\in Q_{\max}} f(|v\rangle)  = 1$ for its orthogonal representation (if the dimension $d$ is larger than $\omega(G)=|Q_{\max}|$, the $|Q_{\max}|<d$ mutually orthogonal vectors $\{|v\rangle:v\in Q_{\max}\}$ in $\mathbb{C}^d$ do not form a basis).

If a graph $G$ is not $\{0,1\}$-colorable and has dimension $d(G)=\omega(G)$, it thus follows that its minimal orthogonal representation $\mathcal{S}$ forms a KS set. If in addition $d^*(G)=\omega(G)$, we say that $G$ is a KS graph (this last condition can always be obtained by considering the faithful version of $G$, i.e., the orthogonality graph $G_\mathcal{S}$ of its minimal orthogonal representation $\mathcal{S}$).

%Note that the orthogonality graph $G_\mathcal{S}$ of the KS set $\mathcal{S}$ has the additional property that $d^*(G_\mathcal{S})=\omega(G_\mathcal{S})$, i.e., its minimal orthogonal representation (given by $\mathcal{S}$) is faithful. The graph $G_\mathcal{S}$ can be seen as obtained from $G$ by adding edges (between vertices that are non-adjacent in $G$, but corresponding to vectors in $\mathcal{S}$ that are nevertheless orthogonal) and by identifying certain vertices (those that correspond to identical vectors in $\mathcal{S}$). We call a graph $G$ that is not $\{0,1\}$-colorable and satisfying  $d^*(G)=\omega(G)$ a KS graph.
%then we say that it is a KS graph. In this case, its minimal orthogonal representation forms a KS set. Obviously, the orthogonality graph $G_\mathcal{S}$ of a KS set $\mathcal{S}$ is a KS graph. Note that $G_\mathcal{S}$ has the additional property that $d^*(G_\mathcal{S})=\omega(G_\mathcal{S})$, since it admits $\mathcal{S}$ as a faithful orthogonal representation. The minimal orthogonal representation $\mathcal{T}$ of an arbitrary KS graph $G$ is not necessarily faithful, however. But its orthogonality graph $G_\mathcal{T}$ is a KS graph satisfying the property that $d^*(G_\mathcal{T})=\omega(G_\mathcal{T})=\omega(G)$. The graph $G_\mathcal{T}$ can be seen as obtained from $G$ by adding edges (between vertices that are non-adjacent in $G$, but corresponding to vectors in $\mathcal{T}$ that are nevertheless orthogonal) and by identifying certain vertices (those that correspond to identical vectors in $\mathcal{T}$).


The problem of finding KS sets can thus be reduced to the problem of finding KS graphs. But as we have noticed above, deciding if a graph is $\{0,1\}$-colorable is NP-complete. In addition, while finding an orthogonal representation for a given graph can be expressed as finding a solution to a system of polynomial equations, efficient numerical methods for finding such representations are still lacking. Thus, finding KS sets in arbitrary dimensions is a difficult problem towards which a huge amount of effort has been expended \cite{CA96}. In particular, ``records'' of minimal Kochen-Specker systems in different dimensions have been studied \cite{CEG96}, the minimal KS system in dimension four is the $18$-vector system due to Cabello et al. \cite{CEG96, Cab08} while lower bounds on the size of minimal KS systems in other dimensions have also been established.

%A generalization of the KS sets due to Renner and Wolf \cite{RW04} called \textit{weak Kochen Specker sets} also serves to prove the KS theorem. A weak KS set \cite{RW04} is a set of (unit) vectors $S \subset \mathbb{C}^d$ such that for any function $f: S \rightarrow \{0,1\}$ satisfying $\sum_{|u \rangle \in b} f(|v \rangle) = 1$ for all orthogonal bases $b \subset S$, there exist two orthogonal unit vectors $|u_1 \rangle \perp |u_2 \rangle \in S$ such that $f(|u_1 \rangle) = f( |u_2 \rangle) = 1$. It can be readily seen that every KS set is a weak KS set and in \cite{RW04} it was shown that every weak KS set can be completed to a KS set by adding $O(|S|^2)$ vectors so that both types of sets serve to prove the KS theorem in dimensions $d \geq 3$.   





\section{$01$-gadgets and the Kochen-Specker theorem}\label{sec:gadget}
We now introduce the notion of $01$-gadgets that play a crucial role in constructions of KS sets. 
\begin{dfn}
	A $01$-gadget in dimension $d$ is a $\{0,1\}$-colorable set $\mathcal{S}_\text{gad}\subset\mathbb{C}^d$ of vectors containing two distinguished vectors $|v_1\rangle$ and $|v_2\rangle$ that are non-orthogonal, but for which $f(|v_1\rangle)+f(|v_2\rangle)\leq 1$ in every $\{0,1\}$-coloring $f$ of $\mathcal{S}_\text{gad}$.
\end{dfn}
In other words, while a $01$-gadget $\mathcal{S}_\text{gad}$ admits a $\{0,1\}$-coloring, in any such coloring the two distinguished non-orthogonal vertices cannot both be assigned the value $1$ (as if they were actually orthogonal).
We can give an equivalent, alternative definition of a gadget as a graph.
\begin{dfn}
	A $01$-gadget in dimension $d$ is a $\{0,1\}$-colorable graph $G_\text{gad}$ with faithful dimension $d^*(G_\text{gad})=\omega(G_\text{gad})=d$ and with two distinguished non-adjacent vertices $v_1 \nsim v_2$ such that $f(v_1)+f(v_2)\leq 1$ in every $\{0,1\}$-coloring $f$ of $G_\text{gad}$.
\end{dfn}
%Note that in the above definition, the last condition (the non-orthogonality of $|v_1\rangle$ and $|v_2\rangle$) is automatically satisfied if the graph $G_\text{gad}$ satisfies $d^*(G_\text{gad})=\omega(G_\text{gad})$.
In the following when we refer to a $01$-gadget, we freely alternate between the equivalent set or graph definitions.

An example of a $01$-gadget in dimension 3 is given by the following set of 8 vectors in $\mathbb{C}^3$:
\begin{eqnarray}
	\label{eq:Clif-orth-rep}
&&	| u_1 \rangle = \frac{1}{\sqrt{3}}(-1,1,1), \; \; |u_2 \rangle = \frac{1}{\sqrt{2}}(1,1,0), \nonumber \\ 
&& |u_3 \rangle = \frac{1}{\sqrt{2}} (0,1,-1), |u_4 \rangle = (0,0,1), \nonumber \\
&&	|u_5 \rangle = (1,0,0), \; \; |u_6 \rangle = \frac{1}{\sqrt{2}}(1,-1,0), \nonumber \\
&& |u_7 \rangle = \frac{1}{\sqrt{2}}(0,1,1), \; \; |u_8 \rangle = \frac{1}{\sqrt{3}}(1,1,1), 
	\end{eqnarray}
where the two distinguished vectors are $|v_1\rangle=|u_1\rangle$ and $|v_2\rangle=|u_8\rangle$. Its  orthogonality graph is represented in Fig.~\ref{fig:Clifton}. It is easily seen from this graph representation that the vertices $u_1$ and $u_8$ cannot both be assigned the value 1, as this then necessarily leads to the adjacent vertices $u_4$ and $u_5$ to be both assigned the value 1, in contradiction with the $\{0,1\}$-coloring rules. This graph was identified by Clifton, following work by Stairs \cite{Clifton93, Stairs}, and used by him to construct statistical proofs of the Kochen-Specker theorem. We will refer to it as the Clifton gadget $G_{\text{Clif}}$. The Clifton gadget and similar gadgets were termed ``definite prediction sets" in \cite{CA96}. 
\begin{figure}[t]
	\centerline{\includegraphics[scale=0.33]{fig-clifton.pdf}}
	\caption{The $8$-vertex ``Clifton" graph that was used by Kochen and Specker in their construction of the $117$ vector KS set. The two distinguished vertices are $u_1$ and $u_8$.}
	\label{fig:Clifton}
\end{figure}

We identify the role played by 01-gadgets in the construction of Kochen-Specker sets via the following theorem. 
\begin{theorem}
	\label{prop:KS-gadg}
	For any Kochen-Specker graph $G_\text{KS}$, there exists a subgraph $G_{\text{gad}}<G_{\text{KS}}$ with $\omega(G_\text{gad})=\omega(G_\text{KS})$ that is a $01$-gadget. Moreover, given a $01$-gadget $G_{\text{gad}}$, one can construct a KS graph $G_\text{KS}$ with $\omega(G_\text{KS})=	\omega(G_{\text{gad}})$. 
\end{theorem} 
The demonstration of our theorem is constructive, it allows to build a $01$-gadget from a KS graph and conversely.
The $01$-gadget in the original $117$-vector proof by Kochen-Specker is the Clifton graph in Fig.~\ref{fig:Clifton}.
A $16$-vertex $01$-gadget in dimension 4 that is an induced subgraph of the $18$-vertex KS graph introduced in \cite{CEG96} is represented in Fig. \ref{fig:cab-KS-gadget}.
\begin{figure}[t]
	\centerline{\includegraphics[scale=0.39]{cabello-KSproof-gadget-4.pdf}}
	\caption{A $16$ vertex coloring gadget (also a $101$-gadget) that is a subgraph of the $18$ vertex Kochen-Specker graph in dimension $d=4$ found by Cabello et al. \cite{CEG96}. The $9$ edge colors denote $9$ cliques in the graph, with the maximum clique being of size $\omega(G) = 4$. The distinguished vertices $u_1, u_6$ are denoted by black circles.}
	\label{fig:cab-KS-gadget}
\end{figure} 

\begin{proof}
We start by showing the first part of the Theorem: that one can construct a $01$-gadget $G_{\text{gad}}$ from any KS graph $G_\text{KS}$.  Given $G_\text{KS}$, which by definition is not $\{0,1\}$-colorable, we first construct, by deleting vertices one at a time, an induced subgraph $G_{\text{crit}}$ that is vertex-critical. By vertex-critical, we mean that $(i)$ $G_{\text{crit}}$ is not $\{0,1\}$-colorable, but $(ii)$ any subgraph obtained from it by deleting a supplementary vertex does admit a $\{0,1\}$-coloring. Observe that in the process of constructing $G_{\text{crit}}$ we are able to preserve the maximum clique size, i.e., $\omega(G_{\text{crit}}) = \omega(G_{\text{KS}})$. This is because we are able to delete vertices from all but two maximum cliques, simply because at least two maximum cliques must exist in a graph that is not $\{0,1\}$-colorable. Observe also that $G_{\text{crit}}$ is itself a KS graph, since the faithful orthogonal representation of $G_\text{KS}$ in dimension $d=\omega(G_{\text{	KS}})$ provides an orthogonal representation of $G_\text{crit}$ in the same dimension. 

We consider three cases: $(i)$ there exists a vertex in $G_{\text{crit}}$ that belongs to a single maximum clique, $(ii)$ all vertices in $G_{\text{crit}}$ belong to at least two maximum cliques, and there exists a vertex that belong to exactly two maximum cliques; $(iii)$, all vertices in $G_{\text{crit}}$ belong to at least three maximum cliques. In the first two cases, which happen to be the case encountered in all known KS graphs, we will be able to prove that the $01$-gadget appears as an induced subgraph while in the third case, the $01$-gadget appears as a subgraph that may not necessarily be induced. 

In case $(i)$, let $v$ be one of the vertices having the property that it belongs to a single maximum clique. We denote this clique $Q_1 \subset G_{\mathcal{S}}^{\text{crit}}$. Deleting $v$ leads to a graph $G_{\text{crit}} \setminus v$ that is $\{0,1\}$-colorable by definition. However, observe that in any coloring $f$ of $G_{\text{crit}} \setminus v$, all the vertices in $Q_1 \setminus v$ are assigned the value $0$ by $f$. This is because, if one of these vertices were assigned value $1$, then one could obtain a valid coloring of $G_{\text{crit}}$ from $f$ by defining $f(v) = 0$. Choose a vertex $v_1\in Q_1 \setminus v$ and any other non-adjacent vertex $v_2\in G_{\text{crit}} \setminus v$. Then $G_{\text{crit}} \setminus v$ is the required $01$-gadget with $v_1, v_2$ playing the role of the distinguished vertices.  

In case $(ii)$, let $v$ be one of the vertices having the property that it belongs to exactly two maximum cliques, which we denote $Q_1, Q_2 \subset G_{\text{crit}}$. Again, deleting $v$ leads to a graph $G_{\text{crit}} \setminus v$ that is $\{0,1\}$-colorable. However, in any coloring $f$ of $G_{\text{crit}} \setminus v$, it cannot be that a value $f(v_1)=1$ and a value $f(v_2)=1$ are simultaneously assigned to a vertex $v_1 \in Q_1 \setminus v$ and a vertex $v_2 \in Q_2 \setminus v$. This is again because if there was such a coloring $f$, then one could obtain a valid coloring for $G_{\text{crit}}$ by defining $f(v) = 0$, in contradiction with the criticality of $G_{\text{crit}}$. Choose $v_1 \in Q_1 \setminus v$ and $v_2 \in Q_2 \setminus v$ such that $v_1$ and $v_2$ are not adjacent. Two such vertices must exist. Indeed, if all vertices $Q_1 \setminus v$ where adjacent to all vertices of $Q_2 \setminus v$, then the maximum clique size would be strictly greater than $\omega(G_\text{crit})$. Therefore, we have that $G_{\text{crit}} \setminus v$ is the required $01$-gadget with $v_1, v_2$ the distinguished vertices. 

Finally, we consider the case $(iii)$ where each vertex in $G_{\text{crit}}$ belongs to at least three maximum cliques. In this case, we cannot proceed as above where we remove a certain vertex $v$ and pick vertices from two maximal cliques containing $v$, because we can no longer guarantee that these two vertices cannot simultaneously be assigned the value 1 (we can only guarantee that a certain $t$-uple of vertices, each one picked from the $t$ maximum cliques to which $v$ belongs, cannot all simultaneously be assigned the value 1, which may be thought of as a generalization of the $01$-gadget property to $t$ distinguished vertices in place of two). Instead, we proceed as follows. 
We start by deleting edges of $G_{\text{crit}}$ one at a time, to construct a new graph $G'_{\text{crit}}$ that is edge-critical. By edge-critical, we mean, similarly to the construction above, that $G'_{\text{crit}}$ is not $\{0,1\}$-colorable, but any graph obtained from it by deleting a supplementary edge (and thus also by deleting a supplementary vertex) does admit a $\{0,1\}$-coloring. As above, we are able to preserve the maximum clique size in the process, i.e., $\omega(G'_{\text{crit}}) = \omega(G_{\text{crit}})=\omega(G_\text{KS})$, and $G'_{\text{crit}}$ is still a non-$\{0,1\}$-colorable KS graph.
% admitting an orthogonal representation in dimension $d=\omega(G'_{\text{crit}})$.

Case (iii a): If the resulting graph $G'_{\text{crit}}$ is as in the cases $(i)$ and $(ii)$ above, we proceed as before to construct a $01$-gadget from a graph $G'_{\text{crit}}\setminus v$, with the caveat that choosing two non-adjacent vertices $v_1$ and $v_2$ in $G'_{\text{crit}}$ does not necessarily guarantee that they correspond to non-orthogonal vectors in the natural representation induced by the one of $G_\text{KS}$. This is because we have been removing edges from $G_\text{crit}$ to construct $G'_\text{crit}$. However, we can always choose two vertices $v_1$ and $v_2$ that were non-adjacent in the original graph $G_\text{KS}$ and that thus correspond to non-orthogonal vectors. Again, this is because otherwise the maximum clique size of $G_\text{KS}$ would be greater than $\omega(G_\text{KS})$. Now, in any $\{0,1\}$-coloring of $G'_{\text{crit}}\setminus v$, we cannot have both $f(v_1) = 1$ and $f(v_2) = 1$, so that $G'_{\text{crit}}\setminus v$ forms a subgraph of $G$ that is a $01$-gadget. But notice that the $\{0,1\}$-colorings of $G_{\text{crit}}\setminus v$ are a subset of the $\{0,1\}$-colorings of $G'_{\text{crit}}\setminus v$. So that we cannot have both $f(v_1) = 1$ and $f(v_2) = 1$ in any $\{0,1\}$-coloring of $G_{\text{crit}}\setminus v$ as well. So that the $01$-gadget is given by $G_{\text{crit}}\setminus v$ in this case with $v_1, v_2$ the distinguished vertices.

%is then given by 

%$G'_{\text{crit}}\setminus v$, which is a subgraph of $G$, but not necessarily an induced subgraph. 

%can then be completed by taking the faithful version of $G'_{\text{crit}}\setminus v$.

Case(iii b): If the resulting graph $G'_{\text{crit}}$ is not as in the cases $(i)$ and $(ii)$ above, we proceed as follows. 
Let $v$ be an arbitrary vertex of $G'_{\text{crit}}$. By assumption, this vertex belong to at least two maximun cliques $Q_1, Q_2$ (and actually even at least a third one). Delete all the edges $(v,v')$ from $Q_1$ where $v' \in Q_1$ to form $G'_{\text{crit}} \setminus E_v(Q_1)$ (where $E_v(Q_1)$ denotes the edges incident on $v$ in $Q_1$) which is $\{0,1\}$-colorable by definition. In any such coloring $f$, either $f(v)=0$ or $f(v)=1$. In the first case, we must necessarily have that $f(v')=0$ for all $v' \in Q_1\setminus v$, since otherwise the coloring $f$ would also define a valid coloring for $G'_{\text{crit}}$. In the second case, we have $f(v'')=0$ for all $v''\in Q_2\setminus v$ by definition of a coloring. We thus conclude that it cannot be simultaneously the case that $f(v')=f(v'')=1$ for $v'\in Q_1\setminus v$ and $v''\in Q_2\setminus v$. Choose $v_1 \in Q_1$ and $v_2\in Q_2$ non-adjacent in $G_\text{KS}$, which is always possible by the same argument as given before. The faithful version of the graph $G'_{\text{crit}} \setminus E_v(Q_1)$ forms the required $01$-gadget with $v_1, v_2$ being the distinguished vertices. Indeed, by the preceding argument, one can restore edges from $G_{\text{crit}}$ to the graph $G'_{\text{crit}} \setminus E_v(Q_1)$ to obtain the $01$-gadget so long as the graph is $\{0,1\}$-colorable, an instance of this is the graph $G'_{\text{crit}} \setminus (v, v_1)$. 


We now proceed to prove the second part of the statement. Starting from a gadget graph we give a construction of a KS graph. The construction generalizes the original Kochen-Specker construction of \cite{KS} to arbitrary dimensions and arbitrary repeating gadget units. Given $G_{\text{gad}}$, we know that there exists a faithful orthogonal representation $\{| v_i \rangle\}_{i=1}^{n}$ in a Hilbert space of dimension $d = \omega(G_{\text{gad}})$ with $n = |V(G_{\text{gad}})|$. Let $v_1, v_2$ denote the distinguished vertices, and let $|v_2^{\perp} \rangle$ denote a vector orthogonal to $|v_2 \rangle$ that lies in the plane $\text{span}(|v_1 \rangle, |v_2 \rangle)$, spanned by the vectors $|v_1 \rangle$ and $|v_2 \rangle$, with $\theta = \arccos{|\langle v_1| v_2^{\perp} \rangle|}>0$ by definition of a $01$-gadget. We consider the following cases: $(i)$ $\frac{\pi}{2 \theta}$ is rational and can be written as $\frac{p}{q}$ with $q$ an odd integer, $(ii)$ $\frac{\pi}{2 \theta}$ is rational and is given by $\frac{p}{q}$ with $q$ an even integer, or alternatively, $\frac{\pi}{2 \theta}$ is irrational.  
 
\emph{Case $(i)$: $\frac{\pi}{2 \theta}$ is rational and is given by $\frac{p}{q}$ with $q$ an odd integer.}
	 Recall that $|v_2^{\perp} \rangle$ is orthogonal to $|v_2 \rangle$ in the plane $\text{span}(|v_1 \rangle, |v_2 \rangle)$. In the subspace orthogonal to $\text{span}(|v_1 \rangle, |v_2 \rangle)$, choose a basis consisting of $d-2$ mutually orthogonal vectors $|w_1 \rangle, \dots, |w_{d-2} \rangle$. Denoting $G'_{\text{gad}}$ as the orthogonality graph of the entire set of these vectors $\{| v_i \rangle\}_{i=1}^{n} \bigcup \{|v_2^{\perp} \rangle, |w_1 \rangle, \dots, |w_{d-2} \rangle\}$, we obtain a gadget graph that can be used as a building block in a Kochen-Specker type construction. In particular, the crucial property of $G'_{\text{gad}}$ is that in any $\{0,1\}$-coloring $f$, $f(v_1) = 1 \Rightarrow f(v_2^{\perp}) = 1$. This can be seen as follows: $f(v_1) = 1$ implies, by the $\{0,1\}$-coloring rules, that $f(w_i) = 0$ for all $i \in [d-2]$. Moreover, by the gadget property, we have $f(v_2) = 0$, and this imposes $f(v_2^{\perp}) = 1$ to satisfy the requirement that exactly one of the vertices in the maximum clique $(v_2,v_2^{\perp},w_1,\dots, w_{d-2})$ is assigned value $1$. 
	
	As in the original KS construction of \cite{KS}, we construct a chain of $p+1$ copies $G'^{(i)}_{\text{gad}}$ $(i=0,1,\ldots,p\}$ of $G'_{\text{gad}}$ so that $p \theta = q \frac{\pi}{2}$ is an odd integral multiple of $\frac{\pi}{2}$. These copies are obtained from the realization of $G'_{\text{gad}}$ by successive applications of a unitary $\mathcal{U}$, i.e., $|v_j^{(i)} \rangle = \mathcal{U}^{i} | v_j \rangle$ for $i=0,1,\ldots,p$ and $j=1,\ldots,n$ and similarly for the other vectors in $G'_{\text{gad}}$. This unitary operator $\mathcal{U}$ is defined as
	\begin{eqnarray}
	\mathcal{U} = |v_2^{\perp} \rangle \langle v_1 | - | v_2 \rangle \langle v_1^{\perp}| + \textbf{1}_W, 
	\end{eqnarray}
	where $| v_1^{\perp} \rangle$ denotes the vector orthogonal to $|v_1 \rangle$ in the plane $\text{span}(|v_1 \rangle, |v_2 \rangle)$ and where $\textbf{1}_W$ denotes the identity on the subspace orthogonal to $\text{span}(|v_1 \rangle, |v_2 \rangle)$. Writing $|v_2^{\perp} \rangle = \alpha | v_1 \rangle + \beta | v_1^{\perp} \rangle$ for some $\alpha, \beta \in \mathbb{C}$, we see that applying once $\mathcal{U}$ to the faithful realization of $G'_{\text{gad}}$ gives
	\begin{eqnarray}
	\mathcal{U} | v_1 \rangle = | v_2^{\perp} \rangle, \nonumber \\
	\mathcal{U} | v_2^{\perp} \rangle = \alpha | v_2^{\perp} \rangle - \beta | v_2 \rangle.
	\end{eqnarray}
	We have evidently $| \langle v_2^{\perp} | \mathcal{U} | v_2^{\perp} \rangle| = |\langle v_1 | v_2^{\perp} \rangle|$ and that 
	\begin{eqnarray}
	\arccos|\langle v_1 | \mathcal{U}| v_2^{\perp} \rangle| = 2 \arccos|\langle v_1 | v_2^{\perp} \rangle| = 2 \theta.
	\end{eqnarray}
We thus have that under successive applications of $\mathcal{U}$, $|v_1^{(0)}\rangle\rightarrow |v_1^{(1)}\rangle=|v_2^{\perp,(0)}\rangle$, $|v_2^{\perp,(0)}\rangle\rightarrow |v_2^{\perp,(1)}\rangle$, $|v_1^{(1)}\rangle\rightarrow |v_1^{(2)}\rangle=|v_2^{\perp,(1)}\rangle$, $|v_2^{\perp,(1)}\rangle\rightarrow |v_2^{\perp,(2)}\rangle$, and so on, with $|v_1^{(p)} \rangle \perp |v_1^{(0)} \rangle$. Furthermore, in any $\{0,1\}$-coloring $f$ of the graph union $\bigcup_{i} G'^{(i)}_{\text{gad}}$, $f(v_1^{(0)})=1 \Rightarrow f(v_1^{(p)}) = 1$. A similar construction of $d-1$ copies of $\bigcup_{i} G'^{(i)}_{\text{gad}}$ gives rise to a graph with a clique formed by the vertices $v_1^{(0)}, v_1^{(p)}$ and the $d-2$ vectors that complete the basis. The resulting graph is a Kochen-Specker graph since in any $\{0,1\}$-coloring, if any of the vertices in this maximal clique is assigned value $1$ then so are all of them, giving rise to a contradiction. We thus obtain a finite system of vectors given by the union of the vector sets in each of the graphs, that gives rise to a proof of the Kochen-Specker theorem in dimension $\omega(G_{\text{gad}})$.

\emph{Case $(ii)$: $\frac{\pi}{2 \theta}$ is rational and is given by $\frac{p}{q}$ with $q$ an even integer, or alternatively, $\frac{\pi}{2 \theta}$ is irrational.}
	
	In this case, we construct from $G_{\text{gad}}$ a larger gadget $\tilde{G}_{\text{gad}}$ with the property that the angle $\tilde{\theta}$ between the distinguished vectors obeys $\frac{\pi}{2 \tilde{\theta}} = \frac{\tilde{p}}{\tilde{q}} \in \mathbb{Q}$, with $\tilde{q}$ an odd integer. 
	As in the previous case, we let $|v_2^{\perp} \rangle$ be the vector orthogonal to $|v_2 \rangle$ in the plane $\text{span}(|v_1 \rangle, |v_2 \rangle)$, and $| v_1^{\perp} \rangle$ be the vector orthogonal to $|v_1 \rangle$ in this plane, so that $|v_2^{\perp} \rangle = \alpha | v_1 \rangle + \beta | v_1^{\perp} \rangle$,
%	\begin{eqnarray}
%	|v_1 \rangle = \alpha | u \rangle + \beta | u^{\perp, 2} \rangle,
%	\end{eqnarray}
	for some $\alpha, \beta \in \mathbb{C}$. We also consider a basis $\{|w_1 \rangle, \dots, |w_{d-2} \rangle \}$ for the subspace orthogonal to $\text{span}(|v_1 \rangle, |v_2 \rangle)$ and denote $G'_{\text{gad}}$ as the orthogonality graph of the set of vectors $\{|v_i \rangle\}_{i=1}^n \bigcup \{|v_2^{\perp} \rangle, |w_1 \rangle, \dots, |w_{d-2}\rangle\}$.
	
	Let $\mathcal{U}$ denote a unitary operator transforming $|v_1 \rangle$ to $|v_2^{\perp} \rangle$, i.e., $\mathcal{U}$ is of the form 
	\begin{eqnarray}
	\mathcal{U} &=& |v_2^{\perp} \rangle \langle v_1| - | v'_2 \rangle \langle v_1^{\perp}| +  \nonumber \\
&&+  |w'_1 \rangle \langle w_1 |+\dots+ |w'_{d-2} \rangle \langle w_{d-2} | 
	\end{eqnarray}
	with $|v'_2\rangle$, $|w'_1\rangle,\ldots,|w'_{d-2}\rangle$ orthogonal to $|v_2^{\perp} \rangle$ and orthogonal to each other.	% with $| v_1^{\perp,2} \rangle = - | u_2 \rangle$. 
	%We choose $\mathcal{V}$ such that $\mathcal{V} | v_2 \rangle$ does not lie entirely in the plane $\text{span}(|v_1 \rangle, |v_2 \rangle)$. 
	Applying $\mathcal{U}$ to the orthogonal representation of the gadget gives that 
	\begin{eqnarray}
	\mathcal{U} | v_1 \rangle &=& | v_2^{\perp} \rangle, \nonumber \\
	\mathcal{U} | v_2^{\perp} \rangle &=& \alpha |v_2^{\perp} \rangle - \beta |v'_2 \rangle
	% \nonumber \\
	%&=& \alpha | v_1 \rangle - \beta \mathcal{V} | u_2 \rangle.
	\end{eqnarray} 
	%Let $| v_2 \rangle$ denote a vector orthogonal to $\mathcal{U} | v \rangle$ in the plane $\text{span}(|u \rangle, \mathcal{U} | v \rangle)$, and let $\tilde{\theta} = \arccos|\langle u | \mathcal{U} | v_2 \rangle|$. 
	Let $\tilde{\theta} = \arccos| \langle v_1 | \mathcal{U} | v_2^{\perp} \rangle |$.
	We choose $|v'_2\rangle$ and thereby $\mathcal{U}$ such that $\frac{\pi}{2 \tilde{\theta}} = \frac{\tilde{p}}{\tilde{q}} \in \mathbb{Q}$ with $\tilde{q}$ an odd integer.  Now construct $G'_{\text{gad}}$ as the orthogonality graph of the set of vectors 
	\begin{eqnarray}
	\{|v_i \rangle\}_{i=1}^{n} \bigcup \{|v_2^{\perp} \rangle, |w_1 \rangle, \dots, |w_{d-2}\rangle\} \bigcup \nonumber \\ \{\mathcal{U}|v_i \rangle\}_{i=2}^{n} \bigcup \{\mathcal{U} |v_2^{\perp} \rangle, |w'_1 \rangle, \dots, |w'_{d-2} \rangle\}.
	\end{eqnarray}  
	We have thus concatenated two gadgets to form the new gadget $G'_{\text{gad}}$ with the property that if $f(|v_1 \rangle) = 1$ then also $f(|v_2^{\perp} \rangle) = 1$ and consequently also $f(\mathcal{U} | v_2^{\perp} \rangle) = 1$. We are now in the same position as in the previous case i.e., we may construct a chain of $\tilde{p}+1$ copies $G'^{(i)}_{\text{gad}}$ of $G'_{\text{gad}}$ and follow the steps as in the previous case to construct the entire KS set in dimension $\omega(G_{\text{gad}})$.

In both cases, we thus obtain a construction of a Kochen-Specker set in dimension $\omega(G_{\text{gad}})$, completing the proof. 
\end{proof}

We remark that the above Theorem does not guarantee that the $01$-gadgets appear as \textit{induced} subgraphs in KS graphs; this is the case only when every vertex in the $\{0,1\}$-edge-critical subgraph of the KS graph does not belong to three or more maximum cliques (cases $(i), (ii)$ and $(iii a)$ in the proof). As such, in the case $(iii b)$ where every vertex in the $\{0,1\}$-edge-critical subgraph of the KS graph belongs to at least three maximum cliques, the subgraphs may not correspond to vector subsets of the original KS vector set. We leave it as an interesting open question whether $01$-gadgets always appear as vector subsets of the KS vector sets in this case as well. We also note that constructions similar to that given in the proof of the second part of Theorem \ref{prop:KS-gadg} have appeared in \cite{BBCP09}. 








\section{Other 01-gadgets and KS sets constructions}\label{sec:constr}
In this section, we make some interesting observations about $01$-gadgets and provide new constructions of $01$-gadgets that will be used in the next sections. 

\begin{lemma}
	\label{lem:min-gad}
	For any $d\geq 3$, there exists a $01$-gadget in dimension $d$ consisting of $5+d$ vertices.    
\end{lemma}
\begin{proof}
For $d=3$, a 8-vertex $01$-gadget is simply given by the Clifton gadget $G_{\text{Clif}}$. In higher dimensions, a new $01$-gadget $G'_{\text{Clif}}$ can be obtained by adding $d-3$ vertices to $G_{\text{Clif}}$ with edges joining the additional vertices to each other and to each of the $8$ vertices in $G_{\text{Clif}}$. Clearly, a faithful representation of $G'_{\text{Clif}}$ can be obtained by supplementing the $3$-dimensional representation of $G_{\text{Clif}}$ with $d-3$ mutually orthogonal vectors in the complementary subspace. The construction preserves the property that a $\{0,1\}$-coloring of $G'_{\text{Clif}}$ exists and that the two distinguished vertices $v_1,v_2$ of $G_{\text{Clif}}$, now viewed 	as vertices of $G'_{\text{Clif}}$, cannot both be assigned the value $1$ in any $\{0,1\}$ coloring. 
\end{proof}


The $8$-vertex Clifton gadget $G_{\text{Clif}}$ was shown to be the minimal $01$-gadget in dimension 3 \cite{Arends09}. This result was obtained by an exhaustive search over all non-isomorphic square-free graphs of up to $7$ vertices. 
It is an open question to prove if the simple construction in Lemma \ref{lem:min-gad} gives the minimal $01$-gadgets in dimension $d>3$ or whether even smaller gadgets exist in these higher dimensions. 

In the Clifton gadget $G_{\text{Clif}}$ the overlap between the two distinguised vertices is $|\langle v_1|v_2\rangle|=1/3$. The following Lemma shows that one can reduce this overlap at the expense of increasing the dimension by one. 
\begin{lemma}
	\label{lem:gad-realize}
	Let $G$ be a $01$-gadget in dimension $d$ with distinguished vectors $|u_1\rangle, |u_2\rangle$. Then there exists a $01$-gadget $G'$ in dimension $d+1$ with distinguished vertices $|v_1\rangle, |v_2\rangle$ for any choice of the overlap $0<|\langle v_1|v_2\rangle|\leq |\langle u_1|u_2\rangle|$.
\end{lemma}
\begin{proof}
Let $\{|u_i\rangle\}_{i=1}^n\subset \mathbb{C}^d$ be the set of $n$ vectors forming the gadget $G$. We define $G'$ as the set of $n+1$ vectors $\{|v_i\rangle\}_{i=0}^n$ in $\mathbb{C}^{d+1}$ defined as follows. For given $|u_i \rangle \in \mathbb{C}^{d}$, let $|\tilde{u}_i \rangle \in \mathbb{C}^{d+1}$ be the vector obtained by padding a $0$ to the end of $|u_i \rangle$. Define the vectors $|v_i \rangle$ as
	%\begin{eqnarray}
	\[
	|v_i \rangle := \left\{\begin{array}{lr}
	(0,\dots,0, 1)^T, & \text{for } i = 0\\
	\mathcal{N}\left(|\tilde{u}_1\rangle + x (0,\dots,0, 1)^T\right), & \text{for } i=1\\
	|\tilde{u}_i \rangle & \text{for } i = 2, \dots, n
	\end{array}\right.
	\] 
	%\end{eqnarray}
	with a free parameter $x \in \mathbb{R}$ and corresponding normalization factor $\mathcal{N}$. 
Now, notice that the orthogonality relations between the set of vectors $|v_1 \rangle, \dots, |v_{n} \rangle$ is the same as the orthogonality relations between the set of vectors $|u_1 \rangle, \dots, |u_{n} \rangle$. The only additional orthogonality relations in $G'$ involve $|v_0\rangle$, which is orthogonal to all other vectors but $|v_1\rangle$. By this property, it follows that if $f(|v_0\rangle)=0$ in a coloring of $G'$, then the coloring of the remaining vectors $|v_1 \rangle, \dots, |v_{n} \rangle$ is constrained exactly as for $|u_1 \rangle, \dots, |u_{n|} \rangle$ in $G$. In particular, we cannot have simultaneously $f(|v_1\rangle)=f(|v_2\rangle)=1$. Now simply observe that if $f(|v_2\rangle)=1$, we must have necessarily have $f(|v_0\rangle)=0$ since $|v_0 \rangle \perp |v_2 \rangle$ and thus $|v_1\rangle$ cannot also satisfy $f(|v_1\rangle) = 1$. In other words, $G'$ is a $01$-gadget with $|v_1\rangle, |v_2\rangle$ playing the role of the distinguished vertices. Finally, we see that by varying the free parameter $x \in \mathbb{R}$, we get  any overlap $0< |\langle v_1|v_2\rangle|\leq |\langle u_1|u_2\rangle|$ between the distinguished vertices. 
\end{proof}



We now show the following.  

\begin{theorem}
	\label{prop:fin-gadg-const}
	Let $|v_1 \rangle$ and $|v_2 \rangle$ be any two distinct non-orthogonal vectors in $\mathbb{C}^d$ with $d \geq 3$. Then there exists a 01-gadget in dimension $d$ with $|v_1 \rangle$ and $|v_2 \rangle$ being the two distinguished vertices.
\end{theorem}   
While the existence of such a construction can be anticipated from the Kochen-Specker construction from Theorem~\ref{prop:KS-gadg}, we give a construction with much fewer vectors based on the $43$-vertex graph of Fig. \ref{fig:gadg-id-vec}.
\begin{proof}
The construction is based on the $43$-vertex graph $G$ of Fig. \ref{fig:gadg-id-vec}. We first show the construction for $\mathbb{C}^3$, and then straightforwardly extend it to $\mathbb{C}^d$ for $d > 3$. Suppose thus that we are given $|v_1 \rangle, |v_2 \rangle \in \mathbb{C}^3$. We consider two cases: (i) $0 < | \langle v_1 | v_2 \rangle| \leq \frac{1}{\sqrt{2}}$ and (ii)  $\frac{1}{\sqrt{2}} < | \langle v_1 | v_2 \rangle| \leq 1$. 
	
Case (i): $0 < | \langle v_1 | v_2 \rangle| \leq \frac{1}{\sqrt{2}}$.
	Suppose without loss of generality that $|v_1 \rangle = (1,0,0)^T$ and $|v_{2} \rangle = \frac{1}{\sqrt{1+x^2}}(x,1,0)^T$ with $0 < x \leq 1$.
	In this case, the induced subgraph $G_{\text{ind}}$ of $G$  consisting of the vertex set $V(G_{\text{ind}}) = \{1,\dots,22\}$ and $E(G_{\text{ind}}) = \{(u_i, u_j): 1 \leq i,j \leq 22, (u_i,u_j) \in E(G)\}$ will suffice to construct the gadget with $u_1$ and $u_{22}$ the two distinguished vertices, corresponding to $|v_1 \rangle$ and $|v_2 \rangle$. First, it is easily verified from the graph that in any $\{0,1\}$-coloring $f$,  $f(u_1)$ and $f(u_{22})$ cannot both be assigned the value 1. It thus only remains to provide an orthogonal representation of the graph $G_{\text{ind}}$. Such a representation is given by the following set of (non-normalized) vectors:
	\begin{eqnarray}
	&&|u_1 \rangle = (1,0,0)^T; \; \; |u_2 \rangle = (0,1,-1)^T; \; \; |u_3 \rangle = (0,1,0)^T; \nonumber \\
	&&|u_4 \rangle = (0,y,1)^T; \; \; |u_5 \rangle = (2 x,1,1)^T; \; \; |u_6 \rangle = (-1,0,2 x)^T; \nonumber \\
	&&|u_7 \rangle = (-2 x,0,-1)^T; \; \; |u_8 \rangle = (x,1,-2 x^2)^T; \nonumber \\
	&& |u_9 \rangle = (2x^3, 2 x^2,1+x^2)^T; \nonumber \\
	&& |u_{10} \rangle = (-(1+x^2),0,2x^3)^T; \nonumber \\
	&&|u_{11} \rangle = (2 x^3, 0, 1+x^2)^T; \nonumber\\
	&& |u_{12} \rangle = (x(1+x^2), 1+x^2,-2x^4)^T; \nonumber \\
	&&|u_{13} \rangle = (2 x^5, 2x^4, (1+x^2)^2)^T; \nonumber \\
	&& |u_{14} \rangle = (-(1+x^2)^2, 0, 2x^5)^T; \nonumber \\
	&&|u_{15} \rangle = (2x^5, 0,(1+x^2)^2)^T; \nonumber \\
	&& |u_{16} \rangle = (x(1+x^2)^2, (1+x^2)^2, -2x^6)^T; \nonumber \\
	&&|u_{17} \rangle = (2x^7, 2x^6, (1+x^2)^3)^T; \nonumber \\
	&& |u_{18} \rangle = (-x(1+ y^2), -1, y)^T; \nonumber \\
	&&|u_{19} \rangle = (1,-x,-x)^T; \; \; |u_{20} \rangle = (1,-x,0)^T; \nonumber \\
	&&|u_{21} \rangle = (1,-x,x y)^T; \; \; |u_{22} \rangle = (x,1,0)^T;
	\end{eqnarray}
	with 
	\begin{eqnarray}
	y = \frac{(1+x^2)^3 + \sqrt{(1+x^2)^6 - 16 x^{14} (1+x^2)}}{4 x^8}\,.
	\end{eqnarray}
	
It is easily verified that this set of vectors satisfy all the orthogonality relations encoded by the induced subgraph $G_{\text{ind}}$ we are considering.
	
Case (ii):  $\frac{1}{\sqrt{2}} < | \langle v_1 | v_2 \rangle| \leq 1$.
	Suppose without loss of generality that $|v_1 \rangle = (1,0,0)^T$ and $|v_{2} \rangle = (1+x,1-x,0)^T/\sqrt{2+2x^2}$ with $0 < x \leq 1$. In this case, we consider the entire $43$-vertex graph $G$ from Fig. \ref{fig:gadg-id-vec}, with  $u_1$ and $u_{42}$ the two distinguished vertices, corresponding to $|v_1 \rangle$ and $|v_2 \rangle$. Again, it is easily seen that in any $\{0,1\}$-coloring $f$,  $f(u_1)$ and $f(u_{42})$ cannot both be assigned the value 1. It thus only remains to provide an orthogonal representation of the graph $G$. 
	
The graph $G$ can be seen as being composed from $(i)$ the induced subgraph $G_{\text{ind}}$ with vertices $u_1,\ldots,u_{22}$ considered above, $(ii)$ an isomorphic subgraph $G'_{\text{ind}}$ with vertices $u'_1=u_{20},u'_2=u_{23},\ldots,u'_{22}=u_{42}$, $(iii)$ the vertex $u_{43}$ connected to $u_1$, $u_{20}$, $u_{22}$, $u_{42}$. 

The first 22 vectors $u_1,\ldots,u_{22}$ of $G_{\text{ind}}$ are chosen as above with $x = 1$ and $y = 2 + \sqrt{2}$. 
The 22 vectors $u'_1,\ldots,u'_{22}$ of $G'_{\text{ind}}$ are also obtained from the above solution, but with $0<x\leq 1$ a free parameter, and after applying first a unitary $U$ that maps  $(1,0,0)$ to $(1,-1,0)/\sqrt{2}$ and $(0,1,0)$ to $(1,1,0)\sqrt{2}$ and leave invariant $(0,0,1)$. We thus have $|u_1\rangle=|v_1\rangle=(1,0,0)^T$ and  $|u_{42}\rangle=|v_2\rangle=(1+x,1-x,0)^T/\sqrt{2+2x^2}$ as assumed.

By construction, the orthogonality relations of the subgraphs $G_{\text{ind}}$ and $G'_{\text{ind}}$ are satisfied. We also have that the vectors common to the two subgraphs are indeed identical, namely $|u_{20}\rangle=(1,-1,0)^T$ and $|u_{22}\rangle=(1,1,0)^T$. Furthemore, choosing $|u_{43}\rangle=(0,0,1)^T$, we also have that $|u_{43}\rangle$ is orthogonal to $|u_1\rangle, |u_{20}\rangle, |u_{22}\rangle$, and $|u_{42}\rangle$ as required. 

This completes the construction of the gadget for $\mathbb{C}^3$. Now, one may simply consider the same set of vectors as being embedded in any $\mathbb{C}^d$ (with additional vectors $(0,0,0,1,0,\dots,0)^T$, $(0,0,0,0,1,0,\dots,0)^T$ etc.) to construct a gadget in this dimension.
	\end{proof}
\begin{figure}[t]
	\centerline{\includegraphics[scale=0.38]{fig-45-3.pdf}}
	\caption{The 43 vertex $01$-gadget used in the proof of Theorem~\ref{prop:fin-gadg-const}.}
	\label{fig:gadg-id-vec}
\end{figure}

%\subsection{Further proofs of the KS theorem using $01$-gadgets.}
Theorem \ref{prop:fin-gadg-const} allows to construct new KS graphs than the one given in the proof of Theorem \ref{prop:KS-gadg}. Some of such constructions in dimension 3 are shown in Fig. \ref{fig:KS-proofs}. A crucial role in these is played by the repeating unit $G_0$ shown in Fig. \ref{fig:KS-proofs} (a). This unit is given by a set of basis vectors $\{|u_1 \rangle, |u_2 \rangle, |u_3 \rangle\}$ all connected via appropriate $01$-gadgets to a central vector $|v_1 \rangle$. In any $\{0,1\}$-coloring $f$ of $G_0$, one of the three basis vectors must be assigned the value $1$, so that we necessarily have $f(|v_1 \rangle) = 0$. In other words, $G_0$ is a graph in which a particular vector necessarily takes value $0$ in any $\{0,1\}$-coloring. Note that this property is also shown by the graph in Fig. \ref{fig:gadg-id-vec} 
%\tred{this is like graph of figure 3}. 

Note that from $G_0$, one can also construct an orthogonality graph $G_1$ in which a particular vector necessarily takes values 1 in any $\{0,1\}$-coloring. Indeed, consider two copies of $G_0$ with the respective central vectors $|v_1 \rangle$ and $|v_2 \rangle$ orthogonal to each other, so that $f(|v_1 \rangle) = f(|v_2 \rangle) = 0$. Then, in any $\{0,1\}$-coloring of the resulting graph $G_1$, the third basis vector $|v_3 \rangle \perp |v_1 \rangle, |v_2 \rangle$ necessarily obeys $f(|v_3 \rangle) = 1$.  

\begin{figure}
	\centerline{\includegraphics[scale=0.35]{fig-KS-proofs.pdf}}
	\caption{Graphs with the dashed edges denoting $01$-gadgets. (a) In any $\{0,1\}$-coloring of the graph $G_0$, the central vertex is necessarily assigned value $0$. (b) Three copies of $G_0$ with the central vertices forming a basis in $\mathbb{C}^3$ so that the resulting graph $G_{KS1}$ forms a Kochen-Specker proof. (c) Another proof of the KS theorem $G_{KS2}$ is obtained by connecting every pair of vectors in two bases by a $01$-gadget.}
	\label{fig:KS-proofs}
\end{figure} 

In Fig. \ref{fig:KS-proofs} (b), a KS proof in $\mathbb{C}^3$ is based on the unit $G_0$, repeated three times with a basis set of central vectors $|v_1 \rangle, |v_2 \rangle, |v_3 \rangle$. By the property of $G_0$ in any $\{0,1\}$-coloring, all these three basis vectors are assigned value $0$ leading to a KS contradiction.
 In Fig. \ref{fig:KS-proofs} (c), the construction is based on two basis sets $\{|u_1 \rangle, |u_2 \rangle, |u_3 \rangle\}$ and $\{|v_1 \rangle, |v_2 \rangle, |v_3 \rangle\}$ with an appropriate $01$-gadget connecting every pair $|u_i \rangle, |v_j \rangle$ for $i, j =1, 2, 3$. So that assigning value $1$ to any of the vectors in one basis, necessarily implies that all of the vectors in the other basis are assigned value $0$, leading to a contradiction. Furthermore, the construction can be readily extended to derive KS graphs using any frustrated graph. 
%For instance, consider the frustrated triangle shown in Fig. \ref{} (d), where each edge in $\{(v_1, v_2), (v_2, v_3), (v_3, v_1)\}$ imposes a constraint $f(v_i) \neq f(v_j) \in \{0,1\}$. The configuration can be turned into a Kochen-Specker proof by considering two copies of $G_1$ with $v_1, v_2$ taking value $1$        




\section{Statistical KS arguments based on $01$-gadgets}\label{sec:real}
The KS theorem can be seen as a proof that no non-contextual deterministic hidden-variable interpretation of quantum theory is possible. In a deterministic hidden-variable model, we aim to reproduce the quantum probabilities
\begin{equation}
\text{Pr}_{\psi}(i|M)=\sum_\lambda q_{\psi}(\lambda) f_\lambda(i|M)
\end{equation}
in term of hidden-variables $\lambda$, where a distribution $q_{\psi}(\lambda)$ over the hidden-variables is associated to each quantum state $|\psi\rangle$, and where for each $\lambda$, the model predicts with certainty that one of the outcomes $i$ will occur for each measurement $M$, i.e., the hidden measurement outcome probabilities $f_\lambda(i|M)$ satisfy $f_\lambda(i|M)\in\{0,1\}$. Furthermore, the model is non-contextual if, as in the quantum case, the probabilistic assignment to the outcome $i$ of the (projective) measurement $M$, only depends on the corresponding projector $V_i$, independently of the wider context provided by the full description of the measurement $M=\{V_1,V_2,\ldots,V_n\}$. In other words in a non-contextual deterministic hidden-variable, we aim to write for every projector $V$:
\begin{equation}
\langle\psi|V|\psi\rangle=\sum_\lambda q_{\psi}(\lambda) f_\lambda(V)\,,
\end{equation}
where $f_\lambda(V)\in\{0,1\}$. Obviously, we should also require for consistency that $\sum_{i\in \mathcal{O}}f(V_i)\leq 1$ for any set $\mathcal{O}$ of mutually orthogonal projectors, with equality when the projectors in $\mathcal{O}$ sum to the identity. 

No-go theorems against such models, i.e., ``proofs of contextuality" , are usually obtained by considering a finite set $\mathcal{S}=\{|v_1\rangle,\ldots,|v_n\rangle\} \subset \mathbb{C}^d$ of rank-one projectors $V_i$, represented as vectors through $V_i=|v_i\rangle\langle v_i|$. Specializing to this case, a non-contextual hidden variable model should satisfy for each $|v_i\rangle$ in $\mathcal{S}$ and each $|\psi\rangle$ in $\mathbb{C}^d$,
\begin{equation}\label{eq:nchv}
|\langle\psi|v_i\rangle|^2=\sum_\lambda q_{\psi}(\lambda) f_\lambda(|v_i\rangle)\,,
\end{equation}
where the $f_\lambda:\mathcal{S}\rightarrow \{0,1\}$ are $\{0,1\}$-colorings of $\mathcal{S}$.

At least three types of no-go theorems, from strongest to weakest, against such non-contextual hidden-variable models can be constructed.

The first types correspond to Kochen-Specker theorems. They establish that for certain sets $\mathcal{S}$, it is not possible to consistently define $\{0,1\}$-colorings $f_\lambda$ of $\mathcal{S}$, even before attempting to use them to reproduce the quantum probabilities. This is what we have discussed until now. 

In the second type of proofs, a $\{0,1\}$-coloring of $\mathcal{S}$ is not excluded. But it can be shown that for any such coloring $f_\lambda$ of $\mathcal{S}$, a certain inequality  $\sum_i c_i f_\lambda(|v_i\rangle)\leq c_0$ must necessarily be satisfied, while in the quantum case, it happens that $\sum_i c_i |v_i\rangle\langle v_i|>c_0 \mathbb{I}$. In other words, though it is possible to find a $\{0,1\}$ assignment $f_\lambda(|v_i\rangle)$ to each projector $|v_i\rangle\langle v_i|$ in $\mathcal{S}$ that is compatible with the orthogonality relations among such projectors, any such assignment fails to reproduce some more complex relation of the type $\sum_i c_i |v_i\rangle\langle v_i|>c_0 \mathbb{I}$ satisfied by these projectors. This immediately implies a contradiction with eq.~(\ref{eq:nchv}), since in the quantum case we have for any $|\psi\rangle$, $\sum_i c_i |\langle\psi|v_i\rangle|^2>c_0$, while according to a non-contextual hidden variable model, we would have $\sum_i c_i |\langle\psi|v_i\rangle|^2=\sum_\lambda q_{\psi}(\lambda)\left[\sum_i c_i f_\lambda(|v_i\rangle)\right]\leq \sum_\lambda q_{|\psi\rangle}(\lambda)c_0 \leq c_0$. Such no-go theorems are referred to as ``statistical state-independent" KS arguments and were introduced by Yu and Oh \cite{YO12}.

Finally, for certain sets $\mathcal{S}$, it is possible to find valid $\{0,1\}$-colorings that do not lead to any type of contradictions of the second type above. However, it is not possible to take mixtures of such colorings, as in eq.~(\ref{eq:nchv}), to reproduce the predictions of certain quantum states $|\psi\rangle$. Such no-go theorems are referred to as ``statistical state-dependent" KS arguments and were introduced by Clifton in \cite{Clifton93}.

While we have seen in the previous section how proofs of the KS theorem can be constructed using $01$-gadgets, in this section we show how to use them to build statistical state-independent and state-dependent KS arguments

\subsection{State-independent KS arguments}
In \cite{YO12}, Yu and Oh introduced a set of 13 vectors in $\mathbb{C}^ 3$ that provides a state-independent proof of contextuality, despite not being a KS set. We show how using Theorem~\ref{prop:fin-gadg-const}, it is possible to construct other state-independent proofs of contextuality based on $01$-gadgets.  To do this, we make use of the following lemma.

\begin{lemma}
	\label{lem:d-simplex}
	Let $|u_i \rangle$, for $i =1, \dots, d+1$ be the unit vectors denoting the vertices of a $d$-dimensional simplex embedded in $\mathbb{R}^d$. Then 
	\begin{eqnarray}
	\sum_{i=1}^{d+1} |u_i \rangle \langle u_i | = \frac{d+1}{d} \mathbb{I}.
	\end{eqnarray}
\end{lemma}
\begin{proof}
Since $|u_i \rangle$ form the vertices of the $d$-simplex, we have $ \langle u_i | u_j \rangle = -\frac{1}{d}$ for any $i \neq j \in \{1, \dots, d+1\}$. It then follows 
	\begin{eqnarray}
	\left(\sum_{i=1}^{d+1} \langle u_i | \right) \left(\sum_{j=1}^{d+1} | u_i \rangle \right) = (d+1) + d(d+1)\left(-\frac{1}{d} \right) = 0, \nonumber
	\end{eqnarray}
	so that 
	\begin{eqnarray}
	O := \sum_{i=1}^{d+1} |u_i \rangle \langle u_i| = -\sum_{i \neq j = 1}^{d+1} | u_i \rangle \langle u_j|
	\end{eqnarray}
	This then implies that
	\begin{eqnarray}
	O^{2} = O - \frac{1}{d} \sum_{i \neq j =1}^{d+1} |u_i \rangle \langle u_j| = \frac{d+1}{d} O.
	\end{eqnarray}
    Moreover, $O$ is invertible, since $\text{span}(\{|u_i \rangle\}_{i=1}^{d+1}) = \mathbb{R}^d$ so that we obtain $O = \frac{d+1}{d} \mathbb{I}$.
\end{proof}
Now, state-independent KS arguments for $\mathbb{C}^d$ are straightforwardly constructed as follows. For every pair of vectors $|u_i \rangle, |u_j \rangle$ of the $d$-simplex, consider a $01$-gadget $\mathcal{S}_{ij}$ with $|u_i\rangle$, $|u_j\rangle$ the distinguished vertices. Since $|u_i\rangle$ and $|u_j\rangle$ are non-orthogonals, such gadgets exists, as implied by Theorem~\ref{prop:fin-gadg-const}. The resulting set of vectors $\mathcal{S}=\cup_{ij} \mathcal{S}_{ij}$ exhibits state-independent contextuality. Indeed, by the property of the $01$-gadgets, only one of the vectors $|u_i \rangle$ for $i =1, \dots, d+1$ can be assigned the value $1$ in any $\{0,1\}$-coloring of $\mathcal{S}$. It thus follows that
\begin{eqnarray}
\sum_{i=1}^{d+1} f(|u_i \rangle) \leq 1,.
\end{eqnarray}
On the other hand, from Lemma \ref{lem:d-simplex}, every state $|\psi\rangle$ from $\mathbb{C}^d$ achieves the value $\sum_{i=1}^{d+1} |\langle\psi|u_i\rangle|^2=\frac{d+1}{d}>1$.

While we have used the $d+1$ vertices of a $d$-simplex in the construction above, we observe that any set $\{|u_i \rangle\}$ of vectors in $\mathbb{C}^d$ such that $\sum_{i} |\langle \psi | u_i \rangle|^2 > 1$ for all $| \psi \rangle \in \mathbb{C}^d$ can be utilized in the construction, although such a set clearly needs to contain at least $d+1$ vectors. 

\subsection{State-dependent KS arguments}

The relation between state-dependent KS arguments and 01-gadgets is even more direct than in the above construction. Actually, the first state-dependent KS argument introduced by Clifton in \cite{Clifton93} was precisely based on the set of vectors (\ref{eq:Clif-orth-rep}) forming the Clifton gadget $G_\text{gad}$. His argument was as follows. In every non-contextual hidden-variable model attempting to replicate the quantum probabilities associated to the projectors of the Clifton gadget, we should have $|\langle\psi|u_1\rangle|^2+|\langle\psi|u_8\rangle|^2=\sum_\lambda q_\psi(\lambda) \left(f_\lambda(|u_1\rangle)+f_\lambda(|u_8\rangle)\right)\leq 1$, by the gadget property. However, if we take $|\psi\rangle=|u_1\rangle$, we find that according to the quantum predictions $|\langle u_1|u_1\rangle|^2+|\langle u_1|u_8\rangle|^2=1+|\langle u_1|u_8\rangle|^2>1$ since $|\langle u_1|u_8\rangle|^2>0$ as $|u_1\rangle$ and $|u_8\rangle$ are non-orthogonal. Other state-dependent proofs based on inequalities have since been developed, with the smallest involving five vectors \cite{KCBS08}. The first state-independent statistical KS argument was presented in \cite{Cab08} and the proof that any KS set give can be converted in a state-independent statistical KS argument was presented in \cite{BBCP09}.



Obviously, the argument used by Clifton for the particular set of vectors he introduced, immediately carries over to any $01$-gadget. Thus every $01$-gadget serves as a proof of state-dependent contextuality.

Note that it was realized in \cite{CDLP14} that a class of graphs, known as perfect graphs, define a class of graphs that cannot serve as proofs of (even state-dependent) contextuality. That is, for any orthogonal representation $\{|v_j \rangle\} \subset \mathbb{C}^d$ of a perfect graph and for any pure state $|\psi \rangle \in \mathbb{C}^d$, the outcome probabilities $|\langle \psi|v_j\rangle|^2$ admit a non-contextual hidden variable model of the form (\ref{eq:nchv}). Since a non-contextual hidden variable model is not possible for a $01$-gadget, we deduce that no perfect graph is a $01$-gadget. Perfect graphs are a well-known class of graphs which by the strong perfect graph theorem \cite{CRST06} can be characterized as those graphs that do not contain odd cycles and anti-cycles of length greater than three as induced subgraphs. %It was realized in \cite{CDLP14} that the perfect graphs are exactly the graphs that do not allow for contextuality, so that the non-perfect graphs are the basic exclusivity graphs required for quantum correlations. 

Finally, remark that the argument due to Clifton presented above works not only for the state $|\psi\rangle=|u_1\rangle$, but for any state $|\psi \rangle \in \mathbb{C}^3$ which obeys $| \langle \psi | u_1 \rangle|^2 + |\langle \psi | u_8 \rangle|^2 > 1$. More generally, we now present a $01$-gadget which serves to prove state-dependent contextuality for all but a measure zero set of states in $\mathbb{C}^3$. 

This construction is based on the gadget $G$ of Fig.~\ref{fig:gadg-id-vec} with the 43 vector orthogonal representation presented in the proof of Theorem~\ref{prop:fin-gadg-const}. Note that if we take $x=1$ in this representation, then the two distinguished vectors $|u_1\rangle$ and $|u_{42}\rangle$ actually coincide and are both equal to $(1,0,0)$ (i.e., the two distinguished vertices $u_1$ and $u_{42}$ should actually be identified). Therefore in any $\{0,1\}$-coloring $f$ of $G$, $2f(|u_1\rangle)=f(|u_1\rangle)+f(|u_{42}\rangle)\leq 1$, i.e. the vector $|v_1\rangle$ is assigned value $0$. This implies that $G$ witnesses state-dependent contextuality of all states in $\mathbb{C}^3$ but for a measure zero set of states $|\psi \rangle$ that are orthogonal to $|v_1\rangle=(1,0,0)$.

The construction that we just described is based on 42 vectors. It is actually possible to find a slightly smaller construction based on the following 40 vectors:
\begin{eqnarray}
&&|u_1 \rangle = (1,-1,0)^T; \; \; |u_2 \rangle = (1,1,1)^T;\nonumber \\
&&|u_3 \rangle = (1,1,0)^T; \; \; |u_4 \rangle = (1,1,b)^T; \nonumber \\
&&|u_5 \rangle = (-2,1,1)^T; \; \; |u_6 \rangle = (1,-1,3)^T; \nonumber \\
&&|u_7 \rangle = (3,-3,-2)^T; \; \; |u_8 \rangle = (2,0,3)^T;\nonumber \\
&& |u_9 \rangle = (-3,0,2)^T; \;\;|u_{10} \rangle = (-2,2,-3)^T; \nonumber \\
&& |u_{11} \rangle = (3,-3,-4)^T; \; \; |u_{12} \rangle = (4,0,3)^T; \nonumber \\
&&|u_{13} \rangle = (-3,0,4)^T; \; \; |u_{14} \rangle = (-4,4,-3)^T;  \nonumber \\
&&|u_{15} \rangle = (3,-3,-8)^T; \; \; |u_{16} \rangle = (8,0,3)^T; \nonumber \\
&& |u_{17} \rangle = (-3,0,8)^T; \;\;|u_{18} \rangle = (-8,4+\sqrt{7},-3)^T;\nonumber \\
&&|u_{19} \rangle = (0,1,-1)^T; |u_{20} \rangle = (0,1,0)^T; \nonumber \\
&& |u_{21} \rangle = (0,-3+8b,-16-3b)^T; \;\;|u_{22} \rangle = (1,0,0)^T; \nonumber \\
&&|u_{23} \rangle = (1,0,-1)^T; \;\; |u_{24} \rangle = (2-\sqrt{2},0,1)^T; \nonumber \\
&&|u_{25} \rangle = (1,-2,1)^T; \; \; |u_{26} \rangle = (0,1,2)^T; \nonumber \\
&& |u_{27} \rangle = (0,2,-1)^T; \;\; |u_{28} \rangle = (1,-1,-2)^T; \nonumber \\
&& |u_{29} \rangle = (1,-1,1)^T; \; \; |u_{30} \rangle = (0,1,1)^T; \nonumber \\
&&|u_{31} \rangle = (0,1,-1)^T; \;\; |u_{32} \rangle = (-1,1,1)^T; \nonumber \\
&&|u_{33} \rangle = (-1,1,-2)^T; \;\; |u_{34} \rangle = (0,2,1)^T; \nonumber \\
&& |u_{35} \rangle = (0,1,-2)^T; \; \; |u_{36} \rangle = (2,-2,-1)^T; \nonumber \\
&&|u_{37} \rangle = (1,-1,4)^T; \; \;  |u_{38} \rangle = (-2-\sqrt{2},6-\sqrt{2},2)^T;  \nonumber \\
&& |u_{39} \rangle = |u_2 \rangle;\; \; |u_{40} \rangle = |u_3 \rangle; \; \; |u_{41} \rangle = (1,1,-2+\sqrt{2})^T; \nonumber \\
&& |u_{42} \rangle = |u_1 \rangle; |u_{43} \rangle = (0,0,1)^T; \nonumber 
\end{eqnarray}
with $b = \frac{-4+\sqrt{7}}{3}$, and where we have the following identities $|u_{1}\rangle=|u_{42}\rangle$, $|u_2\rangle=|u_{39}\rangle$,  $|u_3\rangle=|u_{40}\rangle$. It can be verified that the graph in Fig.~\ref{fig:gadg-id-vec} where we identify the vertices $u_{1}$ and $u_{42}$, $u_2$ and $u_{39}$,  $u_3$ and $u_{40}$, is the orthogonality graph of these 40 vectors. These 40 vectors thus form a $01$-gadget, where as above the vector $|u_1\rangle=(1,-1,0)$ can only be assigned the value $0$, implying that it can serve as a state-dependent contextuality proof for any vector in $\mathbb{C}^3$ that is not orthogonal to $(1,-1,0)$. We leave it as an open question whether this set of $40$ vectors is the minimal set with this property. 

 
 


\section{Proofs of the extended Kochen-Specker theorem using $01$-gadgets}\label{sec:ext}
In this section, we consider a stronger variant of the KS theorem due to Pitowsky \cite{Pitowsky} and Hrushovski and Pitowsky \cite{HP03}. While the KS theorem is concerned with $\{0,1\}$-colorings where all projectors (or vectors) in a given set $S$ must be assigned a value in $\{0,1\}$, we consider here more general assignments where any real value in $[0,1]$ is allowed to the members of $S$. Specifically, given a set of vectors $\mathcal{S}=\{|v_1\rangle,\ldots,|v_n\rangle\}\subset\mathbb{C}^d$, we say that $f:\mathcal{S}\rightarrow [0,1]$ is a $[0,1]$-assignment if $f$ satisfies the same rules (\ref{eq:01rule}) as it does for $\{0,1\}$-colorings.
Both $\{0,1\}$-colorings and $[0,1]$-assignments can be interpreted as assigning a probability to the projectors corresponding to each of the elements of $S$. But while the assignment is constrained to be deterministic in the case of $\{0,1\}$-colorings since these probabilities can only take the values $0$ or $1$, the probabilistic assignment may be completely general (hence non-deterministic) for $[0,1]$-assignments. In particular, for any given quantum state $|\psi\rangle$, the Born rule $f(|v_i\rangle)=|\langle \psi|v_i\rangle|^2$ defines a valid $[0,1]$-assignment.

Hrushovski and Pitowsky \cite{HP03}, following earlier work by Pitowsky in \cite{Pitowsky}, proved the following theorem, which they call the ``logical indeterminacy principle".
\begin{theorem}[\cite{HP03}]
	\label{thm:HP03}
	Let $|v_1 \rangle$ and $|v_2 \rangle$ be two non-orthogonal vectors in $\mathbb{C}^d$ with $d \geq 3$. Then there is a finite set of vectors $S \subset \mathbb{C}^d$ with $|v_1 \rangle, |v_2 \rangle \in S$ such that 
	for any $[0,1]$-assignment, it holds that $f(|v_1 \rangle), f(|v_2 \rangle) \in \{0, 1\}$ if and only if $f(|v_1\rangle) = f(|v_2\rangle) = 0$. 
\end{theorem}
Thus for any two non-orthogonal vectors $|v_1\rangle$ and $|v_2\rangle$, at least one of the probabilities associated to the  vectors $|v_1\rangle$ or $|v_2\rangle$ must be strictly between zero and one, unless they are both equal to zero. A corollary of this result, observed in \cite{ACCS12, ACS14, ACS14-2} is that if $f(|v_1 \rangle) = 1$ (this should, for instance, necessarily be the case if we attempt to reproduce the quantum probabilities for measurements performed on the state $|\psi\rangle=|v_1\rangle$), then $f(|v_2 \rangle) \neq 0,1$, showing that one can localise the ``value-indefiniteness" of quantum observables that the KS theorem implies. Theorem~\ref{thm:HP03} therefore provides a stronger variant of the KS theorem, and we will refer to it as the \emph{extended KS theorem}. 

The proof of Theorem \ref{thm:HP03} given in \cite{HP03} was obtained as a corollary of Gleason's theorem \cite{Gleason}.
A more explicit constructive proof was given by Abbott, Calude and Svozil \cite{ACCS12, ACS14}, where they also noted that significantly none of the known KS sets serves to prove Theorem~\ref{thm:HP03}. Note that an earlier proof of the extended KS theorem was also given in \cite{Pitowsky}. All these existing proofs of the extended KS theorem involve complicated constructions with no systematic procedure for obtaining the requisite sets of vectors. In this subsection, we will provide a simple systematic method for obtaining in a constructive way these extended KS sets. 

In order to prove the extended KS theorem, we need gadgets of a special kind, which are defined as  usual $01$-gadgets apart from the fact that the condition that the two distinguished vertices cannot both be assigned the value $1$ in any $\{0,1\}$-colorings should also hold for any $[0,1]$-assignments. That is, we simply replace `$\{0,1\}$-coloring' by `$[0,1]$-assignment' and $f(|v_1\rangle)+f(|v_2\rangle)\leq 1$ by $f(|v_1\rangle)+f(|v_2\rangle)<2$ in Definition 1, and similarly for Definition 2. We call such new gadgets `extended $01$-gadgets'.  It is easily verified that the Clifton gadget in Fig. \ref{fig:Clifton} and the $16$-vertex gadget in Fig. \ref{fig:cab-KS-gadget} obey this additional restriction. 

Our first aim will be to construct such extended $01$-gadgets for any two given non-orthogonal vectors $|v_1 \rangle, |v_2 \rangle \in \mathbb{C}^d$ for $d \geq 3$. This is the content of the following Theorem, which generalizes Theorem~\ref{prop:fin-gadg-const}.

\begin{theorem}
	\label{prop:101-gadg-const}
	Let $|v_1 \rangle$ and $|v_2 \rangle$ be any two distinct non-orthogonal vectors in $\mathbb{C}^d$ with $d \geq 3$. Then there exists an extended 01-gadget in dimension $d$ with $|v_1 \rangle$ and $|v_2 \rangle$ being the two distinguished vertices.
\end{theorem}
\begin{proof}
	We begin with the construction for $d=3$ and generalize it to higher dimensions naturally. The construction is an iterative procedure based on the Clifton gadget $G_{\text{Clif}}$ given in Fig. \ref{fig:Clifton}.
	
Firstly, as stated previously, it is readily seen that $G_{\text{Clif}}$ is actually an extended $01$-gadget with $u_1, u_8$ the two distinguished vertices, i.e., any $[0,1]$-assignment $f : V(G_{\text{Clif}}) \rightarrow [0,1]$ cannot be such that $f(u_1)=f(u_8)=1$. 
Further, it is known that the $\mathbb{R}^3$ realization of $G_{\text{Clif}}$ given by (\ref{eq:Clif-orth-rep})  achieves the (minimal possible) separation of $\theta_1 = \arccos{|\langle u_1 | u_8 \rangle|} = \arccos{1/3}$ between the two end vertices \cite{Stanford}. 
%In fact, as can be seen with a little algebra, this separation is minimal for the Clifton gadget. 

\begin{figure}
	\centerline{\includegraphics[scale=0.23]{gadg-construct.pdf}}
	\caption{An iterative construction of an extended $01$-gadget for which the two distinguished vertices $u_1$ and $u_8$ are such that in the limit of large number of iterations $k$, $ |\langle u_1^{(k)} | u_8^{(k)}\rangle|\in [0,1[$.}
	\label{fig:Gad-construct}
\end{figure} 

We now describe a nesting procedure that at each step decreases the angle between the vectors corresponding to the two outer vertices. The procedure works as follows. Replace the edge $(u_4, u_5)$ in $G_{\text{Clif}}$ by $G'_{\text{Clif}}$, a copy of $G_{\text{Clif}}$ where we identify $u'_1=u_4$ and $u'_8=u_5$. The new graph thus obtained has $14$ vertices and $21$ edges. The operation has the property that in any $[0,1]$-assignment $f$, an assignment of value $1$ to the two outer vertices of the new graph (i.e. $u_1,u_8$) leads to a similar assignment to the two outer vertices of the inner copy of $G_{\text{Clif}}$ (i.e. $u'_1,u'_8$) thereby giving rise to a contradiction. In other words, the newly constructed graph is once again an extended $01$-gadget.  This procedure can be repeated an arbitrary number of times, as illustrated in Fig. \ref{fig:Gad-construct}, leading to an extended $01$-gadget formed from $k$ nested Clifford graphs $G_{\text{Clif}}^1,G_{\text{Clif}}^2,G_{\text{Clif}}^2,\ldots,G_{\text{Clif}}^k$ where $G_{\text{Clif}}^1$ corresponds to the most inner graph and $G_{\text{Clif}}^k$ to the most outer graph.
We now show that the total graph at the $k$-th iteration is an orthogonality graph where the overlap $|\langle u_1^{(k)} | u_8^{(k)} \rangle |$ between the two outer vertices $u_1^k,u_8^k$ can be chosen to take any value in $[0,\frac{k}{k+2}]$, thus spanning any possible value in $[0,1[$ for $k$ sufficiently large. 
Setting $|v_1 \rangle = |u_1^{(k)} \rangle$ and $|v_2 \rangle = |u_8^{(k)} \rangle$ with $k$ depending on the overlap of the given vectors $| \langle v_1 | v_2 \rangle|$, then gives the required gadget and proves the Theorem.



Suppose that at the $k$-th step of the iteration, the vectors representing the two outer vertices of the ``inner" gadget from the $k-1$-th step are 
\begin{eqnarray}
&&|u_4^{(k)} \rangle = |u_1^{(k-1)} \rangle = (1,0,0), \nonumber \\
&&|u_5^{(k)} \rangle = |u_8^{(k-1)} \rangle = \frac{1}{\sqrt{1+x_k^2}}(x_k,1,0),
\end{eqnarray} 
without loss of generality, so that the overlap between these vectors is $| \langle u_4^{(k)} | u_5^{(k)} \rangle| = \frac{x_k}{\sqrt{1+x_k^2}}$, where for simplicity of the construction we take $x_k \in \mathbb{R}^+_0$. The remaining vectors then in general have the following (non-normalized) orthogonal representation in $\mathbb{R}^3$
\begin{eqnarray}
&&|u_8^{(k)} \rangle = (a_k,b_k,c_k), \; \; |u_6^{(k)} \rangle = (0,-c_k,b_k), \nonumber \\
&& |u_7^{(k)} \rangle = (c_k,-c_k x_k, -a_k + b_k x_k), \; \; |u_2^{(k)} \rangle = (0,b_k,c_k), \; \;  \nonumber \\
&&|u_3^{(k)} \rangle = (-a_k + b_k x_k, a_k x_k - b_k x_k^2, -c_k -c_k x_k^2), \nonumber \\
&& |u_1 \rangle = (-b_k c_k- a_k c_k x_k, - a_k c_k + b_k c_k x_k,a_k b_k - b_k^2 x_k), \nonumber \\
\end{eqnarray}
with $a_k,b_k,c_k \in \mathbb{R}$. This gives an overlap of
\begin{widetext}
\begin{eqnarray}
\label{eq:overlap}
| \langle u_1^{(k)} | u_8^{(k)} \rangle | = \frac{|-a_k c_k (b_k + a_k x_k)|}{\sqrt{(a_k^2 + b_k^2 + c_k^2)(c_k^2(b_k + a_k x_k)^2 + b_k^2(a_k - b_k x_k)^2 + (a_k c_k - b_k c_k x_k)^2)}}.
\end{eqnarray}
\end{widetext}
A direct optimization of this expression with respect to the parameters $a_k,b_k,c_k$ gives the choice $b_k=1$, $c_k=1$, $a_k= x_k + \sqrt{1+x_k^2}$. So that the overlap between the two outer vertices at the $k$-th step of the iteration is given by 
\begin{eqnarray}
\label{eq:iter-koverlap}
| \langle u_1^{(k)} | u_8^{(k)} \rangle | = \frac{1}{3+4 x_k(x_k-\sqrt{1+x_k^2})} =: \frac{x_{k+1}}{\sqrt{1+x_{k+1}^2}}.
\end{eqnarray}
With the initial overlap for $k=1$ of $1/3$ and corresponding initial $x$ values of $x_1 = 0$ and $x_2=\frac{1}{2 \sqrt{2}}$, we can now evaluate the expression for the overlap for any $k>1$. We find that the overlap at the $k$-th step is $\frac{k}{k+2}$. This is readily seen by an inductive argument. The base claim is clear, suppose that at the $k$-th step the overlap is given by $\frac{x_{k+1}}{\sqrt{1+x_{k+1}^2}} = \frac{k}{k+2}$, i.e., $x_{k+1} = \frac{k}{2\sqrt{k+1}}$. Substituting in Eq.\ref{eq:iter-koverlap}, we obtain $\frac{x_{k+2}}{\sqrt{1+x_{k+2}^2}} = \frac{k+1}{k+3}=\frac{(k+1)}{(k+1)+2}$. Moreover, we see that choosing $b_k=1, c_k=1$, the overlap expression (\ref{eq:overlap}) is a continuous function of $a_k$ for any fixed $x_k$ with the minimum value of $0$ achieved at $a_k = 0$. Thus, every intermediate overlap in $[0,\frac{k}{k+2}]$ between the two outer vectors is also achievable by appropriate choice of $a_k$ for the fixed value of $x_k, b_k, c_k$. This completes the construction of the gadget for $\mathbb{C}^3$ (possibly by taking its faithful version in the graph representation).

Now, one may simply consider the same set of vectors as being embedded in any $\mathbb{C}^d$ (with additional vectors$(0,0,0,1,0,\dots,0)^T$, $(0,0,0,0,1,0,\dots,0)^T$ etc.) to construct a gadget in this dimension.  
%The proof in general dimensions is seen by a lifting argument as follows.  
\end{proof}

\begin{figure}
	\centerline{\includegraphics[scale=0.38]{fig-ext-gad.pdf}}
	\caption{An alternative construction of an extended $01$-gadget for which the two distinguished vertices $v_1$ and $v_2$ are such that in the limit of large number $t$ of the repeating unit of four vectors, $| \langle v_1 | v_2 \rangle|$ can take any value in $[0,1[$.}
	\label{fig:ext-gad}
\end{figure} 

In fact, the construction above is not unique. We give an alternative set of vectors that also serves to prove Theorem~\ref{prop:101-gadg-const}. The construction is shown in Fig. \ref{fig:ext-gad}. Suppose we are given two distinct non-orthogonal vectors $|v_1 \rangle = (1,0,0)^T$, $|v_2 \rangle = (x,\sqrt{1-x^2},0)^T$, with $0 < x < 1$. We begin by adding the following set of vectors with a parameter $y \in \mathbb{R}$:
	\begin{eqnarray}
&&|v_3 \rangle = (0,x,-\sqrt{1-x^2})^T; \nonumber \\
&&|v_4 \rangle = (-(1-x^2),x \sqrt{1-x^2},x^2)^T; \nonumber \\
&&|v_5 \rangle = (x,(1-x^2)\sqrt{1-x^2},x(1-x^2))^T; \nonumber \\
&&|v_6 \rangle = (0,y,\sqrt{1-y^2})^T; \nonumber \\
&&|v_7 \rangle = (-\sqrt{(1-x^2)(1-y^2)},x\sqrt{1-y^2}, x y)^T; \nonumber \\
&&|v_8 \rangle = (x,(1-y^2)\sqrt{1-x^2},y\sqrt{(1-x^2)(1-y^2)})^T; \nonumber \\
&&|v_9 \rangle = (0,1,0)^T; \; \; |v_{10} \rangle = (-\sqrt{1-x^2},x,0)^T.
\end{eqnarray}    
The remaining vectors are obtained using a repeating unit consisting of four vectors:
\begin{eqnarray}
&&|v_{7 + 4t} \rangle = (-(1-x^2),0,x^{2(t-1)})^T; \nonumber \\
&&|v_{8 + 4t} \rangle = (x^{2(t-1)},0,1-x^2)^T; \nonumber \\
&&|v_{9+4t} \rangle = (-x(1-x^2),-(1-x^2)\sqrt{1-x^2},x^{2t-1})^T; \nonumber \\
&&|v_{10+4t} \rangle = (x^{2t},x^{2t-1}\sqrt{1-x^2},1-x^2)^T; 
\end{eqnarray}
repeated $t$ times for an integer $t \geq 1$ depending on $x$. Choosing the parameter $y$ as 
\begin{eqnarray}
y = \sqrt{\frac{(1-x^2)^2+2 x^{4t-2} - \sqrt{(1-x^2)((1-x^2)^3 - 4 x^{4t})}}{2(1-x^2)(1-x^2+x^{4t-2})}}, \nonumber
\end{eqnarray}
we find that $y \in \mathbb{R}$, for $t$ satisfying $(1-x^2)^3 \geq 4 x^{4t}$. We see that as $t$ increases this inequality can be satisfied for larger values of $x$, and for any $0 < x < 1$ as $t \rightarrow \infty$. From the orthogonality graph of this set of vectors $S$ shown in Fig. \ref{fig:ext-gad}, it is clear that there cannot be any assignement $f: S \rightarrow [0,1]$ such that $f(|v_1 \rangle)=f(|v_2\rangle)=1$, giving an extended $01$-gadget. 

 

While the construction in Theorem~\ref{prop:101-gadg-const} and that in the previous paragraph work for any two distinct vectors, given two such vectors it is of great interest to find the minimal extended $01$-gadget with these vectors as the distinguished vertices. While this question is the foundational analog for extended KS systems of the question of finding minimal KS sets, it is also of practical interest in obtaining Hardy paradoxes with optimal values of the non-zero probability, and extracting randomness from the gadgets \cite{R17}. 

We now show how the extended $01$-gadgets can be used to construct proofs of the extended KS Theorem~\ref{thm:HP03}. 





\begin{proof}(Theorem \ref{thm:HP03})
We present the construction for $d=3$, the proof for higher dimensions will follow in an analogous fashion. The idea is encapsulated by Fig. \ref{fig:Ext-KS}. Suppose we are given two distinct non-orthogonal vectors $|v_1 \rangle$ and $|v_2 \rangle$ in $\mathbb{C}^d$. We begin by constructing an appropriate extended $01$-gadget $G_{v_1,v_2}$, depending on $| \langle v_1 | v_2 \rangle|$, with the corresponding $v_1, v_2$ being the distinguished vertices. \ 

Let $|v_3 \rangle = |v_1 \rangle \times |v_2 \rangle$ denote the vector orthogonal to the plane $\text{span}(|v_1 \rangle, |v_2\rangle)$ spanned by $|v_1 \rangle$ and $|v_2 \rangle$, where $\times$ denotes the cross product of the vectors. Let $|v_4 \rangle$ be the vector in the plane $\text{span}(|v_1 \rangle, |v_2\rangle)$ orthogonal to $|v_1 \rangle$, and $|v_5 \rangle$ denote the vector in this plane orthogonal to $|v_2 \rangle$, so that $\{|v_1 \rangle, |v_3 \rangle, |v_4 \rangle\}$, $\{|v_2 \rangle, |v_3 \rangle, |v_5 \rangle\}$ form orthogonal bases in $\mathbb{C}^3$. We construct appropriate extended $01$-gadgets $G_{v_1, v_5}$ and $G_{v_2, v_4}$ depending on $| \langle v_1 | v_5 \rangle|$ and $| \langle v_2 | v_4 \rangle|$. In $G_{v_1, v_5}$ the vertices $v_1, v_5$ corresponding to the vectors $|v_1 \rangle, |v_5 \rangle$ play the role of the distinguished vertices and similarly in $G_{v_2, v_4}$. Let $G_{\text{Pit}}$ denote the orthogonality graph of the entire set of vectors ${G_{v_1, v_2}} \bigcup {G_{v_1, v_5}} \bigcup {G_{v_2, v_4}} \bigcup |v_3 \rangle$. 

We have that in any assignment $f: V(G_{\text{Pit}}) \rightarrow [0,1]$ for which $f(v_1), f(v_2) \in \{0,1\}$, if $f(v_1) = 1, f(v_2) = 1$, then we obtain a contradiction by the property of the extended $01$-gadget $G_{v_1,v_2}$. On the other hand, if $f(v_1) = 1, f(v_2) = 0$, then since $|v_1 \rangle \perp |v_3 \rangle$ we have $f(v_3) = 0$, and by the property of the extended $01$-gadget $G_{v_1, v_5}$ we have $f(v_5) = 0$. This gives a contradiction since $v_2, v_3, v_5$ form a maximum clique. Similarly, if $f(v_1) = 0, f(v_2) = 1$, then since $|v_2 \rangle \perp |v_3 \rangle$ we have $f(v_3) = 0$, and by the property of the extended $01$-gadget $G_{v_2, v_4}$ we have $f(v_4) = 0$. This also gives a contradiction since $v_1, v_3, v_4$ form a maximum clique. Therefore, we have any assignment $f: V(G_{\text{Pit}}) \rightarrow [0,1]$ which obeys $f(v_1), f(v_2) \in \{0,1\}$ also must necessarily obey $f(v_1) = f(v_2) = 0$. This completes the proof.

\begin{figure}
	\centerline{\includegraphics[scale=0.32]{fig-ext-KS.pdf}}
	\caption{A constructive proof of the extended Kochen-Specker theorem \ref{thm:HP03} using the extended $01$-gadgets. Given vectors $|v_1 \rangle, |v_2 \rangle \in \mathbb{C}^d$, we obtain vector $|v_3 \rangle \perp \text{span}(|v_1 \rangle, |v_2 \rangle)$ and two other vectors $|v_4 \rangle, |v_5 \rangle$ in the plane $\text{span}(|v_1 \rangle, |v_2 \rangle)$ with the orthogonality relations indicated in the left figure. Dashed edges between two vertices indicate an extended $01$-gadget from Theorem \ref{prop:101-gadg-const} with the two vertices being distinguished.}
	\label{fig:Ext-KS}
\end{figure} 
\end{proof}
\color{black}
%The construction works via the basic unit $G_0$ shown in Fig. \ref{}. As shown in the Figure, the unit $G_0$ is obtained by connecting the vectors $|v_1 \rangle, |v_2 \rangle, |v_3 \rangle$ from a basis in $\mathbb{C}^3$ to a single vector $|v_0 \rangle$ via three extended $01$-gadgets from Prop. \ref{prop:101-gadg-const}. In the Figure, we show the construction for the basis vectors $(1,0,0)^T, (0,1,0)^T, (0,0,1)^T$ each connected via an extended $01$-gadget to $(1,1,1)^T$. 

\subsection{Discussion} Intuitively, with respect to any $\{0,1\}$ coloring, a $01$-gadget behaves like a ''virtual edge" between its two special vertices, with this edge also obeying the rule that at most one of its incident vertices may be assigned the color $1$. Moreover, in Theorem \ref{prop:fin-gadg-const} we have shown that $01$-gadgets may be constructed with any two non-orthogonal vectors as the special vertices. Starting from a given set of vectors, this allows us to connect any two non-orthogonal vectors by an appropriate $01$-gadget, which imposes additional constraints on the $\{0,1\}$-colorings of the resulting set of vectors. By appropriately adding such virtual edges, we are eventually able to obtain a set of vectors that gives a Kochen-Specker contradiction. Moreover, it turns out that the statistical proofs of the Kochen-Specker theorem can also be interpreted in the same manner. For instance, the famous Yu-Oh graph of \cite{YO12} can be interpreted as six $01$-gadgets connecting the vectors $(1,1,1)^T, (1,1,-1)^T, (1,-1,1)^T$ and $(-1,1,1)^T$. These four vectors thus form a ''virtual clique", with the property that in any $\{0,1\}$-coloring of the Yu-Oh set, the sum of the values attributed to these four vectors cannot exceed one. On the other hand, any quantum state has overlap with these four vectors summing to $4/3$ providing a statistical contradiction. Similar considerations also apply to the extended Kochen-Specker theorem of Pitowsky by means of extended $01$-gadgets.     

% \tred{What this sentence means: It is also an open question whether such a construction works for every $101$-gadget, i.e., given such a gadget, does such an iterative construction always give rise to a gadget with increasing overlap between the vectors corresponding to the distinguished vertices?}

%It is also possible to obtain a finite gadget with the property that the two distinguished vertices are represented by vectors with any overlap in $[0,1]$.\tred{You mean that there exists a 101-gadget $G$ admitting orthogonal representations where the two distinguished vertices are represented by vectors with any overlap in $[0,1]$?} Such a construction is illustrated in Fig. \ref{fig:gadg-id-vec} in the Appendix where the upper- and lower-most vertices are the distinguished vertices. \tred{Is the construction given?} In the extreme case, these two vertices are represented by the same vector (in which case, in order to obtain a faithful representation, we identify the two vertices as a single vertex adjacent to every vertex that is a neighbor of either one of the two distinguished vertices). The exact set of vectors in this extreme case is given in the Appendix, an algebraic argument to show that it is possible to obtain a representation in $\mathbb{C}^3$ with every angle between the two vectors is given in \cite{R17}. For a proof that such a representation exists in $\mathbb{C}^4$, see Observation \ref{lem:gad-realize}. It is an open question to find the smallest $101$-gadget with this property.   
%\begin{figure}[t]
%	\centerline{\includegraphics[scale=0.37]{fig-45-3.pdf}}
%	\caption{A finite $101$-gadget for which a representation (in $\mathbb{C}^3$) exists in which the two distinguished vertices $u_1,u_{42}$ are represented by the same vector (note that the orthogonality graph for this representation has the two vertices $u_1$ and $u_{42}$ identified with each other). The faithful orthogonal representation of this graph is given in the Appendix. An interesting open question is to find the minimal gadget with this property.}
%	\label{fig:gadg-id-vec}
%\end{figure}	





%by showing that for every such system $G_{\mathcal{S}}$ has such a $G_{\text{gadget}}$ as an induced subgraph and moreover that given such a $G_{\text{gadget}}$ one can construct a Kochen-Specker vector system in dimension $\omega(G_{\text{gadget}})$. 
%numerical methods  strictly faithful orthogonal representations of a graph in a given dimension 


 



%\section{Quantum Realizations of the Generalized quasi-edge graphs}


%\begin{prop}
%For a coloring gadget $G$ of order $2$ with $\omega(G) = \xi(G)$ and distinguished vertices $u, v \in V(G)$, in any quantum box $\mathbf{B}_{Q}(G)$ constructed using projectors of dimension $\xi(G)$ for the vertices of $G$, we have that $P_{Q}(u) + P_{Q}(v) < 2$.  
%\end{prop}
%We have seen in the previous section that the two distinguished vertices $u, v$ in a generalized quasi-edge cannot be assigned the same color in any optimal coloring of the graph. We shall now prove that in any quantum realization of the graph, these two vertices cannot also be assigned the same projector. This theorem shall form the fundamental statement from which we will use the state-independent contextuality proofs for randomness generation. To elaborate, the fact that the two vertices cannot be assigned the same projector implies that there is no quantum state for which two measurements containing these two projectors yield a deterministic outcome corresponding to these projectors. This will then be used to produce a min-entropy source out of the outputs of these measurements, which together with another independent min-entropy source used to choose the measurements will produce uniformly random bits upon application of a hash function.
%
%\begin{theorem}
%Consider any generalized quasi-edge graph $G$ with $\chi(G) = \omega(G)$ and two distinguished vertices $u, v \in V(G)$. In any quantum realization of $G$ with vectors $|\psi \rangle  \in \mathbb{C}^{\omega(G)}$ we have that $|\psi_u \rangle \neq | \psi_v \rangle$. 
%\end{theorem}
%
%\begin{proof}
%The key idea is to consider graph coloring as an integer semi-definite programming problem which when relaxed to arbitrary reals leads to a representation of the vertices of the graph with vectors. 
%
%Let us first recall the following integer semi-definite programming formulation of the chromatic number formulated by Dukanovic and Rendl in \cite{DR08}. An $m$-coloring of the graph $c : V(G) \rightarrow \{1,\dots, m\}$ defines a coloring relation $R$ on the graph with two vertices equivalent under the relation if they belong to the same color class, i.e., $u R v \leftrightarrow c(u) = c(v)$. The $m$-coloring $c$ is represented by a matrix $X$ with entries $X_{u,v} = 1$ if $c(u) = c(v)$. A permutation of the vertices $P$ can be found such that $P^T X P$ is a direct sum of blocks of the all-ones matrix $J_{n_i}$, i.e., $P^T X P = \oplus_{i=1}^{m} J_{n_i}$ with $n_i$ being the number of vertices belonging to color class $i$. Dukanovic and Rendl then show that for any symmetric 0-1 matrix $X$, there exists a permutation matrix $P$ bringing it to the above block-diagonal form for given $m$ if and only if $diag(X) = e$ and $(t X - J \succeq 0 \leftrightarrow t \geq m)$, where $e$ denotes the all-ones vector and $J$ denotes the all-ones matrix of size $|V(G)| \times |V(G)|$.  
%
%From the above, we see that the chromatic number can be formulated as an integer semi-definite program:
%% for $X$ a $|V(G)| \times |V(G)|$ matrix and $J$ the all-ones matrix of this size, 
%%\theta(G) = \left\{ 
%%\begin{tabular}{ c|c c } 
%%$\min t$ & $t X - J \succeq 0$ \\
%%   &$X_{i,i} = 1 \; \; \forall i \in V(G)$ \\
%%   &$X_{i,j} = 0 \; \; \forall (i,j) \in E(G)$ \\
%%   &$X_{i,j} \in \{0,1\}$\\
%% \end{tabular} 
%% \right\}
%\begin{eqnarray}
%\label{chi-integer-sdp}
%\chi(G) &=& \min t \nonumber \\ 
%&& t X - J \succeq 0 \nonumber \\
%&& X_{i,i} = 1 \; \; \forall i \in V(G) \nonumber \\
%&& X_{i,j} = 0 \; \; \forall (i,j) \in E(G) \nonumber \\
%&& X_{i,j} \in \{0,1\}
%\end{eqnarray}
%%The positive semi-definite matrix $X$ in the program above corresponds to a coloring matrix, i.e., a block diagonal matrix with the number of blocks equal to the number of colors, two vertices belonging to the same block if and only if they are assigned the same color in the proper coloring.
%At this point, let us note that in any optimal coloring of the GQE, the distinguished vertices $u$ and $v$ belong to different color classes so that \textit{all} optimal solutions $X$ to the above program for the GQE have $X_{u,v} = 0$.  
% 
%The corresponding relaxation of the integer constraints to arbitrary reals leads to the Lov\'{a}sz-theta function $\theta(\bar{G})$ of the complement graph $\bar{G}$ 
%\begin{eqnarray}
%\label{theta-sdp}
%\theta(\bar{G}) &=& \min t \nonumber \\ 
%&& t X - J \succeq 0 \nonumber \\
%&& X_{i,i} = 1 \; \; \forall i \in V(G) \nonumber \\
%&& X_{i,j} = 0 \; \; \forall (i,j) \in E(G) \nonumber \\
%\end{eqnarray}
%%\begin{eqnarray}
%%\chi(G) &=& \min \{t :  \text{s. t.} \; \sum_j \lambda_j \chi_{I_j} = \mathbf{1}_V,
%%\end{eqnarray} 
%For the GQE, we have that $\omega(G) = \chi(G)$ so that by the well-known sandwich theorem (for any graph $H$, it holds that $\omega(H) \leq \theta(\bar{H}) \leq \chi(H)$), we have that for the GQE $\omega(G) = \theta(\bar{G}) = \chi(G)$ so that the optimal solutions to the above semidefinite program for $\theta(\bar{G})$ also achieve the value $\omega(G)$.  
%Our task is to show that all feasible solutions $X$ to the semidefinite program for $\theta(\bar{G})$ that achieve value $\omega(G)$ have $X_{u,v} \neq 1$, which would imply that the vectors assigned to the vertices in this SDP solution have $|\psi_u \rangle \neq |\psi_v \rangle$. 
%
%Suppose by contradiction that there exists a feasible solution $X$ satisfying $X_{u,v} = 1$, so that $X$ satisfies $\omega(G) X - J \succeq 0$, $X_{i,i} = 1$ for all $i \in V(G)$ and $X_{i,j} = 0$ for all $(i,j) \in E(G)$ in addition to $X_{u,v} = 1$ for the distinguished vertices $u$ and $v$. Notice that the diagonal elements in the feasible matrix $X$ are equal to $1$ so that all the entries $X_{i,j} \in [-1,1]$ so that one can write $X$ as a convex combination of matrices with integral entries belonging to the set $\{-1, 0, 1\}$. That is, we can write $X = \sum_{k} p_k \tilde{X}^k$ with $\sum_k p_k = 1$ and $\tilde{X}^k_{i,j} \in \{-1, 0, 1\}$, also note that $X_{u,v} = 1$ implies $\tilde{X}^k_{u,v} = 1$ for all $k$.  
%
%Now by assumption $\omega(G) X - J \succeq 0$ so that $\sum_k p_k (\omega(G) \tilde{X}^k - J) \succeq 0$. Also, note    
%
%
%
%
%\end{proof}
%%\begin{prop}
%%For any coloring gadget $G$ with $\omega(G) = \xi(G)$ and $\alpha(G) < \chi_f(\bar{G})$, there exists a protocol obtaining secure randomness against any quantum adversary. If, in addition, $\chi_f(\bar{G}) < \chi(\bar{G})$, then there exists a protocol obtaining secure randomness against any no-signaling adversary.
%%Moreover, there exists a protocol obtaining secure randomness against a no-signaling adversary using a coloring gadget $G$ of order $2$ if $\alpha(G) < \chi_f(\bar{G}) < \chi(\bar{G})$.
%\end{prop}

\section{Computational complexity of $\{0,1\}$-colorings}\label{sec:compl}
\label{sec:Comp-complexity}



Clearly, complete graphs of size $d+1$ cannot be faithfully realized in $\mathbb{C}^d$, but there also exist certain other graphs that cannot be faithfully realized in $\mathbb{C}^d$. The well-known example is the four-cycle (square) graph in $\mathbb{C}^3$, this can be seen by the following simple argument. Suppose a pair of vertices in opposite corners of the square is assigned without loss of generality the vectors $| 0 \rangle$ and $\alpha |0 \rangle + \beta | 1 \rangle$, with $\alpha, \beta \in \mathbb{C}$. Since these vectors span a plane and the remaining pair of vertices are both required to be orthogonal to this plane, these latter vectors are both equal up to a phase to $| 2 \rangle$, contradicting the requirement of faithfulness. There exist analogous graphs that are not faithfully realizable in higher dimensions, some of which are shown in Fig. \ref{fig:forbidden-graphs}. 
%\tred{Move all this to caption:}
%In the Fig. \ref{fig:forbidden-graphs}, graph (i) is the square graph which is not faithfully realizable in $\mathbb{C}^3$ as explained above. Graph (iv) is the graph from \cite{C11} which was verified to be not faithfully realizable in dimension three despite being square-free. Graph (ii) is not faithfully realizable in $\mathbb{C}^4$, which can be seen as arising from the fact that the induced square subgraph is not faithfully realizable in $\mathbb{C}^3$ and the additional vertex being adjacent to all vertices of the square, the vector corresponding to this vertex occupies an orthogonal subspace to that spanned by the square. Graph (iii) is similarly not realizable in $\mathbb{C}^5$ this time owing to the presence of two vertices (which themselves cannot be represented by identical vectors) that are adjacent to all the vertices of the square. It is clear that the construction can be extended to higher dimensions. 
\begin{figure}
	\centerline{\includegraphics[scale=0.32]{forbidden-graphs.pdf}}
	\caption{Examples of forbidden subgraphs in dimensions $3,4$ and $5$. Graph (i) is the square graph which is not faithfully realizable in $\mathbb{C}^3$ as explained in the text. Graph (iv) is the graph from \cite{C11} which was verified to be not faithfully realizable in dimension three despite being square-free. Graph (ii) is not faithfully realizable in $\mathbb{C}^4$, which can be seen as arising from the fact that the induced square subgraph is not faithfully realizable in $\mathbb{C}^3$ and the additional vertex being adjacent to all vertices of the square, the vector corresponding to this vertex occupies an orthogonal subspace to that spanned by the square. Graph (iii) is similarly not realizable in $\mathbb{C}^5$ this time owing to the presence of two vertices (which themselves cannot be represented by identical vectors) that are adjacent to all the vertices of the square. It is clear that the construction can be extended to higher dimensions.}
	\label{fig:forbidden-graphs}
\end{figure} 
In searching for Kochen-Specker vector systems in $\mathbb{C}^d$, it is therefore crucial to reduce the size of the search by restricting to non-isomorphic graphs which do not contain these forbidden graphs as subgraphs. Indeed, searching over non-isomorphic square-free graphs lead to the proof that the smallest Kochen-Specker vector system in $\mathbb{C}^3$ is of size at least $18$ \cite{Arends09}. 

	
Let us denote the set of forbidden graphs in $\mathbb{C}^d$ as $\{G_{\text{fbd}}\}$. We show, following the proof by Arends et al. \cite{Arends09, AOW11} for the square-free case, that the problem of checking $\{0,1\}$-colorability of $\{G_{\text{fbd}}\}$-free graphs is NP-complete. Here, by a $\{G_{\text{fbd}}\}$-free graph we mean a graph that does not contain any of the forbidden graphs as subgraphs. %\tred{What does that sentence means?: The $01$-gadgets turn out to be exactly the structures that enable to show that the problem of checking $\{0,1\}$-colorability of $\{G_{\text{fbd}}\}$-free graphs is NP-complete.} 
\begin{theorem}[see also \cite{Arends09}]\label{theo:NP}
	Checking $\{0,1\}$-colorability of $\{G_{\text{fbd}}\}$-free graphs is NP-complete. 
\end{theorem}
The proof is based on a reduction to the well-known graph coloring problem that uses $01$-gadgets in a crucial manner. Let us first recall the usual notion of coloring of a graph used in the proof. A proper coloring $c$ of a graph $G$ is an assignment of one among $n$ colors to each of the vertices of the graph $c : V(G) \rightarrow [n]$ ($[n]:=\{1,\dots,n\}$) such that no pair of adjacent vertices are assigned the same color. If such a coloring exists, we say that $G$ is $n$-colorable.

\begin{proof}
	The proof generalizes and simplifies that for the analogous question of $\{0,1\}$-colorability of square-free graphs in \cite{Arends09}, with the difference being that we directly use the constructions of $01$-gadgets from the previous sections. Firstly, we know that checking $\{0,1\}$-colorability of a $\{G_{\text{fbd}}\}$-free graph is in NP because the problem of checking an arbitrary graph for $\{0,1\}$-colorability is in NP \cite{Arends09}. Suppose we are given a graph $G$. The idea is to construct a new graph $H$ which is $\{G_{\text{fbd}}\}$-free such that the problem of $\omega(G)$-colorability of $G$ is equivalent to the problem of $\{0,1\}$-colorability of $H$. Provided the construction is achievable in polynomial time, this gives a reduction from the $\{0,1\}$-colorability problem to the $\omega(G)$-colorability problem (for $\omega(G) \geq 3$) which is known to be NP-complete \cite{CLRS01}. 
	
	The construction goes as follows. Replace every vertex $v \in V(G)$ by a clique of size $\omega(G)$ in $H$ and label the corresponding vertices $v_i \in V(H)$ for $i \in [\omega(G)]$. For every edge $(u,v) \in E(G)$, connect the corresponding vertices $(u_i,v_i)$ by a $01$-gadget $\Gamma_{(u_i,v_i)}$ in $H$. The exact form of the gadget $\Gamma_{(u_i,v_i)}$ is left unspecified at the moment, for the polynomial time reduction it is only important that it is finite (i.e., $|V(\Gamma_{(u_i,v_i)})|$ and $|E(\Gamma_{(u_i,v_i)})|$ are finite), so that $|V(H)| \leq \omega(G) (|V(\Gamma_{(u_i,v_i)})|_{max}-1) |V(G)|$ and $|E(H)| \leq \omega(G) | V(G)| + |E(\Gamma_{(u_i,v_i)})|_{max} |E(G)|$, i.e., $|V(H)| = O(|V(G)|)$ and $|E(H)| = O(|E(G)| + |V(G)|)$. 
	
	We first verify that $H$ is $\{G_{\text{fbd}}\}$-free. We do this by showing that $H$ is in fact faithfully realizable in dimension $\omega(G)$ and consequently free of the forbidden subgraphs for that dimension. For the vertices $v \in V(G)$, the actual representation of the vertices $v_i \in V(H)$ is chosen independent of the exact structure of the graph, i.e., for any $G$ with $|V(G)| = n$, we choose a fixed faithful orthogonal representation $\{|v_i \rangle\}$ for $v \in V(G)$ and $i \in [\omega(G)]$. Indeed, to show the realizability of the rest of $H$, it suffices to show the realizability of the vertices $v_1$ for $v \in V(G)$, since the representation for the remaining vertices $v_i$ for $i \geq 2$ can be readily obtained by a cyclic permutation $\Pi_i: | j \rangle \mapsto | j + i \rangle$ with the sum taken modulo $\omega(G)$. The structure of the graph is then incorporated by means of an appropriate choice of the gadgets $\Gamma_{(u_i,v_i)}$.  The crucial idea behind the construction is that there exist finite sized gadgets (with faithful representations) for any two distinct vertices as shown in Prop. \ref{prop:101-gadg-const}. So that for any edge $(u,v) \in E(G)$, we use a gadget $\Gamma(u_1,v_1)$ from Prop. \ref{prop:101-gadg-const} (the same gadget is used for the other pairs $(u_i,v_i)$) corresponding to the required overlap $|\langle u_1 | v_1 \rangle|$. Now, since the representation is faithful, we do not have different vertices represented by the same vector. As such, the construction from Prop. \ref{prop:101-gadg-const} yields a finite sized gadget for any pair of vertices $(u_i,v_1)$.   
	%Pick a vertex $v \in V(G)$, let the corresponding vertices $v_i \in V(H)$ for $i \in [\omega(G)]$ be represented by the vectors in the computational basis $\{|1 \rangle, \dots, |\omega(G)  \rangle\}$. Now, to show the realizability of the rest of $H$, it is clear that it suffices to show the realizability of the vertices $v_1$ for $v \in V(G)$, since the realizability of the remaining vertices $v_i$ for $i \geq 2$ is obtainable by a cyclic permutation $\Pi_i: | j \rangle \mapsto | j + i \rangle$ with the sum taken modulo $\omega(G)$.  
	
	The proof that checking $\{0,1\}$-colorability of the  $\{G_{\text{fbd}}\}$-free graph $H$ is equivalent to checking the $\omega(G)$-colorability of $G$ (which is $NP$-complete) follows along analogous lines to the proof in \cite{Arends09} and we present it here for completeness. Firstly, we show that $H$ is $\{0,1\}$-colorable if $G$ is $\omega(G)$-colorable. Consider the intermediate situation when we form a graph $G'$ by replacing every vertex $v \in V(G)$ by a clique of size $\omega(G)$ and labeling the corresponding vertices $v_i \in V(G')$ for $i \in [\omega(G)]$. For every edge $(u,v) \in E(G)$, connect the corresponding vertices $(u_i,v_i)$ by an edge in $G'$. The strategy is to show that if $G$ is $\omega(G)$-colorable, then $G'$ admits a valid $\{0,1\}$-assignment. Suppose $G$ is $\omega(G)$-colorable, and $c: V(G) \rightarrow [\omega(G)]$ is an optimal coloring. We define the $\{0,1\}$-coloring of $G'$ by 
	%	\begin{eqnarray}
	\[
	c'(v_i) = \left\{\begin{array}{lr}
	1, & \text{for } i = c(v)\\
	0, & \text{else} 
	\end{array}\right. 
	\]
	%	\end{eqnarray} 
	The fact that this is a valid $\{0,1\}$-coloring of $G'$ follows the proof of Lemma 1 in \cite{Arends09}. We now derive the $\{0,1\}$-coloring of $H$ from that of $G'$ by seeing that each of the gadgets in Prop. \ref{prop:101-gadg-const} can be $\{0,1\}$-colored in all three cases, when the distinguished vertices $u_i, v_i$ have the assignments: (i) $f(u_i) = 0, f(v_i) = 0$, (ii) $f(u_i) = 0, f(v_i) = 1$, and $f(u_i) = 1, f(v_i) = 0$. This is done by checking that such a valid $\{0,1\}$-coloring exists for the Clifton gadget in Fig. \ref{fig:Clifton} in each of the three cases. The $\{0,1\}$-coloring can be extended to the entire gadget iteratively by following the procedure shown in the proof of Prop. \ref{prop:101-gadg-const}. This gives a valid $\{0,1\}$-coloring of $H$. 
	
	We now show that a valid $\{0,1\}$-coloring of $H$ also implies that $G$ is $\omega(G)$-colorable. Let $f: V(H) \rightarrow \{0,1\}$ be a valid $\{0,1\}$ assignment of $H$. For every $v \in V(G)$, by the fact that we have a valid $\{0,1\}$-coloring, exactly one of the vertices $v_i  \in V(H)$ is assigned value $1$, i.e., $f(v_i) = 1$. One can then define a $\omega(G)$-coloring $c: V(G) \rightarrow [\omega(G)]$ by $c(v) = i \leftrightarrow c(v_i) = 1$ for every $v \in V(G)$. It is clear that this is a valid coloring since if $(u,v) \in E(G)$ we have by the property of the gadget that at most one of $u_i, v_i$ is assigned value $1$, i.e., either $f(u_i) = 0$ or  $f(v_i) = 0$. Thus, the $\{0,1\}$-colorability of the $\{G_{\text{fbd}}\}$-free graph $H$ is equivalent to the $\omega(G)$-colorability of $G$. From \cite{CLRS01}, we know that for $\omega(G) \geq 3$, this problem is NP-complete, which finishes the proof. 
\end{proof}



%), we explain an important connection between $\{0,1\}$-coloring and the usual graph coloring. It is well-known that every $\omega(G)$-colorable graph $G$ is also $\{0,1\}$-colorable (given a $\omega(G)$-coloring $f: V(G) \rightarrow [\omega(G)]$ of $G$, we simply define its $\{0,1\}$-coloring $g: V(G) \rightarrow \{0,1\}$ by $g(u) = 1$ if $f(u) = 1$ and $0$ otherwise). Moreover, there exist graphs (such as the $13$-vertex graph found by Yu and Oh \cite{YO12}) that are not $\omega(G)$-colorable despite admitting a $\{0,1\}$-coloring. 
%Another example of such a graph is shown in Fig. \ref{}
% we show that not every $101$-colorable graph is $\omega(G)$-colorable so that the two concepts are not equivalent. 



%\begin{lemma}
%	A $\omega(G)$-colorable graph $G$ is $101$-colorable. There exist $101$-colorable graphs that are not $\omega(G)$-colorable.
%\end{lemma}
%\begin{proof}
%	Given an $\omega(G)$-coloring $f: V(G) \rightarrow [\omega(G)]$, we simply define a $101$-coloring $g: V(G) \rightarrow \{0,1\}$ by $g(u) = 1$ if $f(u) = 1$ and $0$ otherwise. Clearly, this is a $101$-coloring since for every edge $(u,v) \in E(G)$, we have $f(u) \neq f(v)$ so that $f(u) = 0 \vee f(v) = 0$. Moreover, every maximum clique contains a vertex $u$ with $f(u) = 1$ and therefore $g(u) = 1$. 
%	
%	An explicit example of a $101$-colorable graph that is not $\omega(G)$-colorable is shown in Fig. \ref{. }. 
%\end{proof}



It is also interesting to examine the complexity of identifying $01$-gadgets. In this case, it appears to be necessary to enumerate all $\{0,1\}$-colorings of a given graph and to check $O(n^2)$ vertices to identify the possible distinguished vertices. Note that for a graph with $n$ vertices there are $2^n$ possible $\{0,1\}$-colorings so that it is not apparent whether even a polynomially checkable certificate exists for this problem. Peeters  in \cite{P96} gave a polynomial time reduction preserving graph planarity of the problem of testing $\xi(G) \leq 3$ to the problem of testing whether the chromatic number $\chi(G)$ is less than or equal to $3$, which is a well-known NP-complete problem, so that it is hard to check whether $d(G) \leq 3$ already for the case of planar graphs.  
%\tred{Is it possible to explain a little bit this last sentence, especially given that $\chi(G)$ was not defined.}
%Furthermore, the problem of checking strict faithfuly realizability of a graph in $\mathbb{C}^d$ is in PSPACE, and it is unknown whether an efficient algorithm exists for it. We leave the complexity of identifying $101$-gadgets as an open question. 

%\section{Two-party Hardy paradoxes from $101$-gadgets}\label{sec:hardy}
%It is well-known that single-party non-contextuality inequalities cannot directly be applied for device-independent information processing. A central reason for this lack of applicability is that in essence, the experimental test of a non-contextuality inequality tends to make the assumption that the same projector is being measured in multiple distinct contexts, which is contrary to the device-independent requirement. One way to overcome this is to construct a two-party inequality based on the local contextuality, so that the spatial correlations between two parties ensures that the local projectors remain the same in different contexts \cite{RBHH+15}. Indeed, it is known that every Kochen-Specker system gives rise to a two-party Bell inequality that is algebraically violated by quantum mechanical correlations (as such this is also called a pseudo-telepathy game) \cite{BBT05}. On the other hand, the $101$-gadgets admit a $\{0,1\}$ assignment, and as such do not translate to a two-party pseudo-telepathy game. Nevertheless, we show that one can construct a Hardy paradox based on the local (state-dependent) gadget that ensures the condition and in \cite{MH17} show that this gives rise to a remarkable device-independent application of local contextuality where the randomness essentially arises from the contextuality exhibited by the $101$-gadget. 
%
%Hardy's paradox is considered the simplest proof of non-locality \tred{by whom? I wouldn't  say that.}, which proves Bell's theorem under the condition considered by Einstein, Podolsky and Rosen, namely that one party's measurement outcome allows this party to predict \textit{with certainty} the other party's measurement outcome \tred{would remove this last bit. I don't understand what it means}. In its simplest form, in the $(2,2,2)$ Bell scenario with two binary measurements per party, the paradox is formulated by the following four constraints:
%\begin{eqnarray}
%\label{eq:Hardy-paradox}
%(i) \; P_{A,B|X,Y}(0, 1|1, 2) &=& 0, \nonumber \\
%(ii) \; P_{A,B|X,Y}(1, 0|2, 1) &=& 0, \nonumber \\
%(iii)\; P_{A,B|X,Y}(0, 0|2, 2) &=& 0, \nonumber \\
%(iv)\; P_{A,B|X,Y}(0, 0|1, 1) &>& 0. 
%\end{eqnarray} 
%While classically, it can be readily verified that conditions $(i)-(iii)$ impose $P_{A,B|X,Y}(1 1|1, 1) = 0$, there exist a suitable two-qubit non-maximally entangled state and dichotomic measurements such that all four conditions are obeyed. 
%%In particular, the optimal quantum strategy gives rise to the value $P_{A,B|U,V}(0, 0|1, 1)$ ($\approx 0.09$)
%\tred{Not necessary:
%Explicitly, Alice and Bob perform on the state
%\begin{eqnarray}
%|\psi_{\theta} \rangle = \frac{1}{\sqrt{1+\cos{(\theta)}^2}} \left[\cos{(\theta)} \left( |01 \rangle + |10 \rangle \right) + \sin{(\theta)} | 11 \rangle \right], 
%\end{eqnarray}
%measurements in the basis
%\begin{eqnarray}
%&&\{|0 \rangle, |1 \rangle\} \; \; \;  \;  x,y=2, \nonumber \\
%&&\{\sin{\theta} | 0 \rangle - \cos{\theta} |1 \rangle, \cos{\theta} | 0 \rangle + \sin{\theta} | 1 \rangle\} \; \;  \; \; x,y = 1. \nonumber \\
%\end{eqnarray}
%The constraints $(i)-(iv)$ are satisfied for any value $0 < \theta < \pi/2$, and the optimal value of $P_{A,B|X,Y}(0, 0|1, 1)$ ($\approx 0.09$) is achieved at $\theta = \arccos{\left(\sqrt{\frac{\sqrt{5}-1}{2}}\right)}$.}
%Hardy's paradox is thus a proof of ``non-locality without inequalities". Yet it is also a probabilistic proof, in the sense that the difference between classical and
%quantum worlds in the paradox lies in the possibility of occurrence of some type of events.
%
%
%
%
%
%The Hardy paradox corresponding to the Clifton gadget belongs to the class $(2, 7, 3)$ indicating that it involves two parties Alice and Bob, each making one of seven possible measurements and obtaining one of three possible outcomes. We label the measurement settings of Alice $x$ and those of Bob $y$ with $x, y \in \{1, \dots, 7\}$. These measurement settings are obtained as follows. We first complete the possible measurement bases in the graph to obtain the set $Q_{\text{max}}$ of maximum cliques as shown in the Fig.\ref{fig:Clifton2}, for the specific gadget under consideration this set is given as
%\begin{eqnarray}
%Q_{\text{max}} = \{(1,2,9), (1,3,10), (2,4,6), (3,5,7), \nonumber \\
%(4,5,11), (6,8,12), (7,8,13)\}.
%\end{eqnarray}
%The measurement settings $x,y$ of the two parties respectively, correspond to the maximum cliques in $Q_{\text{max}}$, so $x, y  \in \{1, \dots, 7\}$. The corresponding outcomes of Alice are labeled $a$ and those of Bob $b$ with $a, b \in \{1, 2, 3\}$. The two parties observe a set of conditional probability distributions $\{P(a, b|x, y)\}$. Let $\{|u_{z,c} \rangle\}$ denote a faithful orthogonal representation of $G$ such as in Eq.(\ref{eq:Clif-orth-rep}). Consider the set $\mathcal{S}_B$ be defined as follows
%\begin{eqnarray}
%\label{eq:SB-Hardy}
%\mathcal{S}_B := \{((x,a),(y,b)) : \langle u_{(x,a)} |u_{(y,b)} \rangle = 0 \} \nonumber \\ \bigcup \; \; \{((x^*,a^*),(y^*,b^*))\}.
%\end{eqnarray}
%Here the input-output pair $((x^*,a^*),(y^*,b^*))$ corresponds to the distinguished vertices $u_{(x^*,a^*)}, u_{(y^*,b^*)}$ of the gadget (one possible choice for the specific gadget under consideration is $((x^*,a^*),(y^*,b^*)) = ((1,1),(7,2))$) and will give rise to the non-zero probability in the Hardy paradox.
%The correlation expression is then given by
%\begin{equation}
%\label{bip-Bell-ineq}
%\textbf{B} \cdot \{P(a, b| x, y)\} = \sum_{a,b, x, y} \textbf{B}(a,b, x, y) P(a, b | x, y),
%\end{equation}
%where $\textbf{B}$ is the indicator vector with entries 
%\begin{equation} \label{eq:bell-indicator}
%\textbf{B}(a,b,x,y) = \left\{
%\begin{array}{lr}
%1 & : ((x,a), (y,b)) \in \mathcal{S}_B \\
%0 & : \text{otherwise} 
%\end{array}
%\right.
%\end{equation} 
%Since a $101$-coloring exists for the gadget, it follows that the classical value of the correlation expression in Eq.(\ref{bip-Bell-ineq}) is exactly $0$. Crucially, the probability $P(a^*,b^*|x^*,y^*)$ for the distinguished vertices $u_{(x^*,a^*)}, u_{(y^*,b^*)}$ is also zero due to the property of the gadget that in any $\{0,1\}$ assignment both these vertices cannot be assigned value $1$. The local hidden variable model achieving this value is precisely given for each of the parties by the $101$-coloring of the gadget, i.e., each of the parties outputs the value in $\{0,1\}$ for each of the projectors in a measurement setting (maximum clique) given by the $101$-coloring.  
%
%\begin{figure}[t]
%	\centerline{\includegraphics[scale=0.33]{fig-clifton-2.pdf}}
%	\caption{The $8$-vertex ``Clifton" graph that was used by Kochen and Specker in their construction of the $117$ vector KS set, shown with all cliques completed to be maximum cliques. These cliques serve as inputs in the corresponding Hardy paradox as explained in the text.}
%	\label{fig:Clifton2}
%\end{figure} 
%%In any classical theory, the expression in Eq.(\ref{bip-Bell-ineq}) evaluates to $0$ owing to the property of the gadget. 
%
%The quantum strategy is given as follows. 
%\begin{enumerate}
%	\item The players' share a maximally entangled state  $| \psi \rangle  = \frac{1}{\sqrt{d}} \sum_{i=0}^{d-1} | i, i \rangle$ (for the specific gadget under consideration $d=3$). 
%	
%	\item Upon receiving the input $x,y$ the players measure in the basis $\{\Pi_x\} = \{|u_{(x,a)} \rangle \langle u_{(x,a)} | \}_{a=1,\dots,d}$ and $\{\Sigma_y \} = \{|u_{(y,b)} \rangle \langle u_{(y,b)} | \}_{b=1,\dots,d}$  corresponding to the received clique. They return the outcomes $a,b$ of the measurement. 
%\end{enumerate}
%
%%
%% In the quantum strategy, the projectors corresponding to the measurements are given as $\Pi_{(x,a)} = |u_{(x,a)} \rangle \langle u_{(x,a)} |$, with $|u_{(x,a)} \rangle$ given by the faithful orthogonal representation in Eq.(\ref{eq:Clif-orth-rep}), and the parties measure on the maximally entangled state of two qutrits $| \psi \rangle  = \frac{1}{\sqrt{3}} \sum_{i=0,1,2} | i, i \rangle$. Let $\mathcal{S}_B$ denote the set of two-party input-output pairs corresponding to orthogonal projectors on each side, i.e., 
%Now for all $((x,a),(y,b)) \in \mathcal{S}_B \setminus \{((x^*,a^*),(y^*,b^*))\}$, we have that  
%\begin{eqnarray}
%\langle \psi | \left(|u_{(x,a)} \rangle \otimes | u_{(y,b)} \rangle \right) &=& \frac{1}{\sqrt{d}} \sum_{i=0}^{d-1} \langle i | u_{(x,a)} \rangle \langle i | u_{(y,b)} \rangle \nonumber \\
%&=& \frac{1}{\sqrt{d}} \overline{\langle u_{(x,a)} |  \overline{u}_{(y,b)} \rangle} = 0, 
%\end{eqnarray}
%since the corresponding vectors are orthogonal. On the other hand, for the distinguished vertices, from the fact that a faithful representation of the gadget exists, we have that the corresponding probability is non-zero. For the specific gadget under consideration we have from the orthogonal representation in Eq.(\ref{eq:Clif-orth-rep}) that  $P(a^*,b^*|x^*,y^*) = \frac{1}{27}$. 
%
%
%The above Hardy paradox generalises in a straightforward manner to any $101$-gadget. 
%%Despite the fact that the classical and quantum values of the correlation expression are equal for general $101$-gadgets, they nevertheless are useful for device-independent applications owing to the following property. Any classical strategy achieving the optimal classical value obeys $P(a,b|x,y) = 0$ for the distinguished vertices $u_{(x,a)}$ and $u_{(y,b)}$ of the $101$-gadget. However, in the quantum strategy the corresponding value for the distinguished vertices is given by $P(a,b|x,y) = |\langle u_{(x,a)} | u_{(y,b)} \rangle|^2$ which evaluates to a maximum value of $\frac{1}{9}$ for the Clifton gadget using the orthogonal representation in Eq.{(\ref{eq:Clif-orth-rep})}. In other words, every $101$-gadget gives rise to a Hardy paradox \cite{} via the procedure of converting to a correlation expression outlined above. Therefore, 
%In a device-independent protocol, in addition to testing for the correlation expression corresponding to $\mathcal{S}_B \setminus \{((x^*,a^*),(y^*,b^*))\}$ (that evaluates to $0$ in both classical and quantum theories), we incorporate a partial tomographic procedure that tests for the particular probability $P(a^*,b^*|x^*,y^*)$ corresponding to the distinguished vertices $u_{(x^*,a^*)}$ and $u_{(y^*,b^*)}$. As seen above, in the classical case, this expression evaluates to $0$ while in the quantum scenario, this probability is strictly bounded from above so that the outputs correspond to a source of min-entropy. Moreover, for a large class of $101$-gadgets, we show that this probability is also strictly bounded below unity for general no-signaling strategies so that we obtain a source of min-entropy even in the case of a powerful eavesdropper who has access to such strategies. We outline a  protocol for randomness amplification from the contextuality of $101$-gadgets in the next section.  
%
%It has been an open question as to how large the non-zero probability in the Hardy paradox can get and whether Hardy paradoxes can be constructed for maximally entangled states \cite{ACY16}. From the construction of the correlation expression for each $101$-gadget outlined in this section (which gives the analogous constraints to (i) - (iii) in the original Hardy paradox (\ref{eq:Hardy-paradox})), and the construction of the $101$-gadget for any two distinct vectors in $\mathbb{C}^d$ for $d \geq 3$ given in Proposition \ref{prop:101-gadg-const}, we obtain Hardy paradoxes with the non-zero probability taking on the spectrum of values in $(0,1/3]$ at the expense of complexity of the gadget. 
%
%In fact, it is possible to obtain Hardy paradoxes for the entire spectrum $(0,1]$ as we now show. To do this, we work in $\mathbb{R}^4$, i.e., we augment the gadget in $\mathbb{R}^3$ with the additional vertex $(0,0,0,1)^T$ to obtain a gadget in dimension four. As shown in Prop. \ref{prop:101-gadg-const} it is possible to obtain a gadget with any two distinct vectors as distinguished vertices $u_1, u_2$ in $\mathbb{R}^4$ by embedding the gadget in $\mathbb{R}^3$ in this manner. We now form four copies of the gadget $G^{(1)}, G^{(2)}, G^{(3)}, G^{(4)}$ in $\mathbb{C}^4$, with the corresponding vectors in each copy being mutually orthogonal, i.e., 
%$|u_k^{(1)} \rangle, |u_k^{(2)} \rangle, |u_k^{(3)} \rangle, |u_k^{(4)} \rangle$ form a complete basis (the vertices form a maximum clique). That such a construction is always possible is based on the fact that in dimensions $4$ and $8$ there exist division algebras \cite{Cohn03} (the quaternions and octonions) so that one can rotate each vector $(a,b,c,d) \in \mathbb{R}^4$ in a set of vectors, to the orthogonal vectors $(b,-a,-d,c), (c,d,-a,-b), (d,-c,b,-a)$ by multiplication by the orthogonal units $i,j$ or $k$ of the algebra, i.e., to every vector $(a,b,c,d) \in \mathbb{R}^4$, one can associate the real orthogonal matrix
%\begin{eqnarray} 
%\label{eq:real-ort-mat}
%\begin{pmatrix} a & b & c & d\\ b & -a & d & -c \\ c & -d & -a & b \\ d & c & -b & -a \end{pmatrix}
%\end{eqnarray}
%A similar construction also exists in dimension eight by means of the octonions. 
%Let $G$ denote the newly formed orthogonality graph with $V(G) = V(G^{(1)}) \bigcup V(G^{(2)}) \bigcup V(G^{(3)}) \bigcup V(G^{(4)})$ and the edge set formed by the orthogonality constraints defined by the three faithful representations of the copies as before. The construction is illustrated by means of the $8$-vertex Clifton gadget in Fig. \ref{fig:hardy-(0,1]}.
%
%\begin{figure}[t]
%	\centerline{\includegraphics[scale=0.39]{hardy-(0,1].pdf}}
%	\caption{An illustration of the Hardy paradox construction in dimension four by means of the real orthogonal matrix in Eq.(\ref{eq:real-ort-mat}). We embed the Clifton graph in dimension four by adding the extra vertex $u_0$ corresponding to the vector $(0,0,0,1)^T$. We then obtain four copies of the resulting gadget in dimension four by multiplication by the orthogonal unit as in Eq.(\ref{eq:real-ort-mat}) (not all resulting edges are shown for clarity). The two distinguished vertices $u_1$ and $u_8$ give rise in this manner to distinguished measurement bases given as $\{|u_{1/8} \rangle, |v_{1/8} \rangle, |w_{1/8} \rangle, |x_{1/8} \rangle\}$. In the limiting scenario of the gadget constructed in Prop. \ref{prop:101-gadg-const} measurements on a maximally entangled ququart state by Alice and Bob in these respective bases give rise to a Hardy paradox with non-zero probability in $(0,1]$ as explained in the text.}
%	\label{fig:hardy-(0,1]}
%\end{figure} 
%
%Now, we form the set $Q_{\text{max}}(G)$ of maximum cliques in $G$ as before, with the measurement settings $x,y$ of the two parties corresponding to these maximum cliques. In particular, we now denote $x^*,y^*$ as the maximum cliques formed by the four orthgonal vectors corresponding to the distinguished vertices $u_1$ and $u_2$, i.e., $u_1^{(1)}, \dots, u_1^{(4)}$ and $u_2^{(1)}, \dots, u_2^{(4)}$. From a faithful representation of the gadget $\{|u_{z,c} \rangle\}$, we form the constraint set $\mathcal{S}_B$
%\begin{eqnarray}
%\mathcal{S}_B & := &\{((x,a),(y,b)) : \langle u_{(x,a)} | u_{(y,b)} \rangle = 0\} \nonumber \\ &&\bigcup  \cup_{k=1}^{4} \{((x^*,u_1^{(k)}),(y^*,u_2^{(k)}))\}.
%\end{eqnarray}
%Now, the input-output pairs $((x^*,u_1^{(k)}),(y^*,u_2^{(k)}))$ correspond to the four pairs of distinguished vertices of the gadget in $\mathbb{R}^4$ and will give rise to the non-zero probabilities in the Hardy paradox. In the quantum strategy, the players share a two-ququart maximally entangled state $\frac{1}{2} \sum_{i=0}^3 | i, i  \rangle$ and measure in the basis $\{\Pi_x\} = \{| u_{(x,a)} \rangle \langle u_{(x,a)} |\}_{a=1,\dots, 4}$ and $\{\Sigma_y \} = \{|u_{(y,b)} \rangle \langle u_{(y,b)} | \}_{b=1,\dots,d}$  corresponding to the received clique. They return the outcomes $a,b$ of the measurement. As before, we have for all $((x,a),(y,b)) \in \mathcal{S}_B \setminus \cup_{k=1}^{4} \{((x^*,u_1^{(k)}),(y^*,u_2^{(k)}))\}$ that 
%\begin{eqnarray}
%\langle \psi | \left(|u_{(x,a)} \rangle \otimes | u_{(y,b)} \rangle \right) &=& \frac{1}{2} \overline{\langle u_{(x,a)} |  \overline{u}_{(y,b)} \rangle} = 0.
%\end{eqnarray}
%By the properties of the gadgets, in any classical theory we have that for each of the distinguished pairs, the probability $P(u_1^{(k)}, u_2^{(k)}|x^*, y^*) = 0$ for all $k =1, \dots, 4$. In the quantum strategy for the measurements $x^*, y^*$ however, we have that 
%\begin{eqnarray}
%\label{eq:quant-prob}
%P(u_1^{(k)}, u_2^{(k)}|x^*, y^*) = \frac{1}{4} | \langle u_1^{(k)} |  \overline{u}_{2}^{(k)} \rangle|^2 
%\end{eqnarray} 
%Choosing a gadget with the distinguished vectors asymptotically identical as in Prop. \ref{prop:101-gadg-const}, we see that the probability $P(u_1^{(k)}, u_2^{(k)}|x^*, y^*)  = \frac{1}{4}$ for all $k = 1, \dots, 4$. We thus obtain a Hardy paradox with the maximum possible contradiction, i.e., a set of events which have probability $0$ in any classical theory, which however are certain to occur in quantum theory. Moreover, from Eq.(\ref{eq:quant-prob}) and the gadget construction in Prop. \ref{prop:101-gadg-const}, we see that one can obtain Hardy paradoxes with the non-zero probability in the entire spectrum $(0,1]$. We therefore obtain the following proposition.
%
%\begin{prop}
%	There exist Hardy paradoxes for the maximally entangled state $\frac{1}{\sqrt{d}} \sum_{i=0}^{d-1} |i, i \rangle$ for all $d \geq 3$ with the non-zero probability taking any value in $(0,\frac{1}{d}]$. In dimensions four and eight, there exist Hardy paradoxes for the maximally entangled state with the non-zero probability taking any value in $(0,1]$.
%\end{prop}
%%Alternatively, one can also see that it is possible to form a non-local game with inputs as projectors (i.e. binary outputs) to obtain an alternative version of the Hardy paradox. 
%In \cite{R17}, we show an even simpler non-local game with finite number of inputs and outputs based on coloring gadgets that yields a Hardy paradox achieving all values in $(0,1]$ for the non-zero probability.




%\section{Randomness from $101$-gadgets}\label{sec:rando}
%\label{sec:rand}
%%The previous section focused on coloring properties of graphs and established an interesting property of the Kochen-Specker graphs with respect to their coloring. Namely, we established the existence of certain subgraphs that we called generalized quasi-edges in the Kochen-Specker graphs that lead to their chromatic number being larger than their clique number. We now show that this same coloring property also translates to a guarantee of the presence of randomness in these $101$-gadgets (when these graphs are realizable by rank-one projectors in a Hilbert space of dimension equal to their clique number). This randomness can be utilized in a device-independent protocol for the process of randomness amplification of weak Santha-Vazirani sources \cite{} by requiring two spatially separated parties to measure all the projectors in the $101$-gadget. Note that while the presence of generalized quasi-edges in Kochen-Specker graphs implies that one can construct a device-independent protocol for randomness amplification using any Kochen-Specker game \cite{}, the set of rank-one projectors that realize generalized quasi-edges themselves do not exhibit \textit{state-independent contextuality} so that the non-local ``games" they give rise to are no longer Kochen-Specker games. 
%%As such, 
%% same subgraphs also guarantee the presence of randomness in certain \textit{specific output-input pairs} when measuring the 
%% Kochen-Specker set.  this randomness can be extracted 
%%
%%In order to extract randomness, we need concrete quantum realizations of graphs, i.e., an orthogonal representation of the graph using a set of vectors $\{ |v_i \rangle \} \in \mathbb{C}^d$ for some fixed dimension $d$. Having such a set, we need to find a suitable state $| \psi \rangle \in \mathbb{C}^d$ on which to measure the corresponding projectors $P_{v_{i}} = | v_i \rangle \langle v_i |$, Finally, we need to construct a device-independent protocol where this set of state and measurements can be used to extract randomness. Therefore, the existence of GQE's of the type in Fig. \ref{fig:qedge-not-gadg} where no randomness exists, does not by itself preclude the link between GQE's and randomness. Indeed, for the particular example graph shown in Fig. \ref{fig:qedge-not-gadg} with $\omega(G) = \chi(G) = 3$, it can be readily verified that the presence of a $4$-cycle subgraph precludes its quantum realization in dimension $d=3$ \cite{Arends09}.
%% a property of the Kochen-Specker with respect to their coloring. 
%
%In what follows, we will show how the $101$-gadgets studied previously can be applied to obtain randomness using a protocol outlined by us previously in \cite{RBHH+15}. In particular we will study the task of randomness amplification of a class of weak sources called Santha-Vazirani sources \cite{SV84} using certain $101$-gadgets that only exhibit state-dependent contextuality. To illustrate the procedure we use the concrete example of the $8$-vertex Clifton graph from Fig. \ref{fig:Clifton}. Denote $C$ as the set of cliques in the graph, and $C_{\text{max}}$ as the set of maximum cliques in the graph. We can then directly reformulate the property of the gadget as the statement that the solution to the following integer programming problem is $1$. 
%\begin{eqnarray}
%\label{lin-prog1}
%&&\max: f(u_1) + f(u_8) \nonumber \\
%&&\sum_{u \in Q} f(u) \leq 1, \; \; \forall Q \in C \nonumber \\
%&&\sum_{u \in Q} f(u) = 1, \; \; \forall Q \in C_{\text{max}} \nonumber \\
%&&f(u) \in \{0,1\}, \; \; \forall u \in V(G).
%\end{eqnarray}  
%Relaxing the integrality constraints to $f(u) \in [0,1]$, the solution to the resulting linear programming problem is also bounded strictly below $2$ for a large class of $101$-gadgets. In this specific case of the Clifton gadget, it is readily seen that the solution to the linear program is $1\frac{1}{2}$. This observation forms a basis for extracting randomness using these gadgets in a secure manner against general non-signaling adversaries via the techniques outlined in \cite{RBHH+15,BRGH+16}. 
%
%To do this, we use the construction of the two-party Hardy paradoxes from $101$-gadgets given in previous sections. The inputs $x,y$ to Alice and Bob are the cliques of the gadget, with every maximal clique extended to form a maximum clique of size $d(G)$. In the case of the Clifton gadget, $x,y \in \{1,\dots,7\}$. In the randomness protocol, Alice and Bob choose these inputs using $\lceil |\log{|X|} \rceil = \lceil | \log{|Y|}| \rceil$ (= $3$ bits in the case of the Clifton gadget) bits taken from a weak source of randomness such as the $\epsilon$-SV source \cite{RBHH+15}. They output $a,b \in d(G)$ (= $3$ in the Clifton gadget) and check that the zero probability events in the Hardy paradox do not occur during any run of their experiment. Once this test is passed, they perform a further check to count the number of times the event $(a^*,b^*|x^*, y^*)$ corresponding to the non-zero probability in the Hardy paradox occurs. If this number is larger than some threshold value, these outputs constitute a weak min-entropy source of randomnesss, which can be extracted by the methods outlined in \cite{RBHH+15}. We give a more clear and detailed practical protocol for obtaining randomness using Hardy paradoxes in a forthcoming paper \cite{MH17}. Here, we simply illustrate this concept by bounding the guessing probability $P_{A,B|X,Y}(a^*, b^*|x^*, y^*)$ in the case of the Clifton gadget for a general no-signaling adversary. To elaborate, suppose that in a Bell experiment where the inputs are chosen using an $\epsilon$-SV source, the observed sum of probabilities for the zero probability events in the Hardy paradox arising from the Clifton gadget is $\delta > 0$. In the following proposition, we bound the non-zero probability event $P_{A,B|X,Y}(a^*,b^*|x^*, y^*)$ for any no-signaling box as a function of $\delta$. 
%
%
%
%%\begin{prop}
%%For any generalized quasi-edge $G$ with $\omega(G) = \xi(G)$,
%% there exists a protocol obtaining secure randomness against any quantum adversary. If, in addition for 
%%and with $\chi_f(\bar{G}) < \chi_f(\bar{G'})+2$ for $G' = G \setminus (u,v)$, there exists a protocol for randomness amplification secure against any no-signaling adversary.
%%Moreover, there exists a protocol obtaining secure randomness against a no-signaling adversary using a coloring gadget $G$ of order $2$ if $\alpha(G) < \chi_f(\bar{G}) < \chi(\bar{G})$.
%%\end{prop}
%
%
%\begin{prop}
%	Let $\mathcal{S}_B$ denote the set of input-outputs pairs from Eq.(\ref{eq:SB-Hardy}) for the Hardy paradox constructed from the Clifton gadget. Consider a two-party no-signaling box $\{P_{A,B|X,Y}(a,b| x,y)\}$ satisfying 
%	\begin{equation}
%	\sum_{\substack{((x,a),(y,b)) \in \mathcal{S}_B \\  ((x,a),(y,b)) \neq ((x^*,a^*),(y^*,b^*))}} \nu_{X,Y}(x,y) P_{A,B|X,Y}(a,b|x,y) \leq \delta,
%	\end{equation}
%	for some constant $\delta \geq 0$, where $\nu_{X,Y}$ is the probability distribution of the inputs taken from the $\epsilon$-SV source satisfying 
%	\begin{eqnarray}
%	(1/2-\epsilon)^6 \leq \nu_{X,Y}(x,y) \leq (1/2+\epsilon)^6.
%	\end{eqnarray}
%	%en by Eq. \eqref{SV-Bell} with $B(\textbf{x},\textbf{u})$ given by Eq. (\ref{eq:bell-indicator}). 
%	Then  for the distinguished input-output pair $((x^*,a^*),(y^*,b^*))$ we have
%	\begin{equation} \label{SV-output-bound}
%	P_{A,B|X,Y}( a^*, b^*| x^*, y^*) \leq   \frac14 \left(3+ \frac{2 \delta}{(\frac12 -\epsilon)^{6} }  \right).          
%	\end{equation}
%\end{prop}
%
%\begin{proof}
%	Let us denote
%	\begin{eqnarray}
%	&&\overline{B}^{SV} := \nonumber \\ 
%	&&\sum_{\substack{((x,a),(y,b)) \in \mathcal{S}_B \\  ((x,a),(y,b)) \neq ((x^*,a^*),(y^*,b^*))}} \nu_{X,Y}(x,y) P_{A,B|X,Y}(a,b|x,y), \nonumber \\
%	\end{eqnarray}
%	as the observed value for the zero-probability events in the Hardy paradox when the settings are chosen with the $\epsilon$-SV source  
%	and 
%	\begin{eqnarray}
%	&&\overline{B}^{U} := \nonumber \\
%	&& \sum_{\substack{((x,a),(y,b)) \in \mathcal{S}_B \\  ((x,a),(y,b)) \neq ((x^*,a^*),(y^*,b^*))}} \mu_{X,Y}(x,y) P_{A,B|X,Y}(a,b|x,y), \nonumber \\
%	\end{eqnarray}
%	as the observed value for the zero-probability events in the Hardy paradox when the settings are chosen with a uniform source $\mu_{X,Y}$. 
%	From the definition of an $\varepsilon$-SV source we have 
%	\begin{equation}
%	\left(\frac{1}{2} - \varepsilon \right)^{6} \leq \nu_{X,Y}(\textbf{x,y}) \leq \left(\frac{1}{2} + \varepsilon \right)^{6}.
%	\end{equation}
%	so that 
%	\begin{equation}
%	\label{SV-uniform}
%	\frac{1}{(\frac{1}{2} + \varepsilon)^{6} |\textsl{X}||\textsl{Y}|} \overline{B}^{SV} \leq \overline{B}^{U} \leq \frac{1}{(\frac{1}{2} - \varepsilon)^{6} |\textsl{X}||\textsl{Y}|}, \overline{B}^{SV}
%	\end{equation}
%	with $|\textsl{X}| = |\textsl{Y}| = 7$ in the case of the Clifton gadget. 
%	%Then the claim follows from lemma \ref{lin-prog}, relating  $\overline{B}^{U}$ with 
%	%%$\Pr \left( \textbf{x}^* | \textbf{u}^*, \mathbb{Z} = z, \mathbb{W} = w  \right)$ by use of linear programming. 
%	%$\Pr \left( \textbf{x}^* | \textbf{u}^* \right)$ by use of linear programming. 
%	%\end{proof}
%	%
%	%\subsubsection{Bounding output probability by linear programming}
%	%
%	%Let us show that for the specific Bell inequality we consider, when the value of the Bell expression is small there is weak randomness. Consider a bipartite no-signaling box $P(\textbf{x} | \textbf{u})$ that obtains a value $\delta$ for the Bell expression in Eq. \eqref{bip-Bell-ineq}. 
%	%The following lemma shows that for the particular measurement setting $\textbf{u}^*$ and outcome $\textbf{x}^*$, the probability $P(\textbf{x}^* | \textbf{u}^*)$ is bounded from above by a function of $\delta$.
%	
%	%\begin{lemma} \label{lin-prog}
%	We can therefore work with the Bell value for uniformly chosen settings, relating it to the Bell value with SV source settings through Eq. (\ref{SV-uniform}). For $\overline{B}^{SV}\leq \delta$, Eq.(\ref{SV-uniform}) gives that $\overline{B}^{U} \leq \frac{\delta}{(\frac{1}{2} - \varepsilon)^{6} |\textsl{X}||\textsl{Y}|} =: \frac{\tilde{\delta}}{|\textsl{X}||\textsl{Y}|}$. 
%	
%	Consider a bipartite no-signaling box $P_{A,B|X,Y}(a,b|x,y)$ satisfying 
%	\begin{equation} 
%	\label{goodindividual}
%	\overline{B}^U  \leq \frac{\tilde{\delta}}{|\textsl{X}||\textsl{Y}|}.
%	\end{equation}
%	%with $\textbf{B}$ the indicator vector for the Bell expression in Eq. \eqref{bip-Bell-ineq} and $|\textsl{U}| = 7 \times 7 = 49$ the number of settings in the Bell expression. 
%	%For any measurement setting $\textbf{u}^*$ and any output $\textbf{x}^*$, we have
%	%\begin{equation} \label{prod-dist}
%	%\Pr \left( \textbf{x}^* | \textbf{u}^*\right) \leq  \frac{1+ 2 \delta}{3}.          
%	%\end{equation}
%	%\end{lemma}
%	
%	%\begin{proof}
%	%Consider any measurement setting $\textbf{u}^*$ and any corresponding output $\textbf{x}^*$ for this setting. 
%	The maximum probability for the chosen output and input 
%	%$\Pr \left( \textbf{x}^* | \textbf{u}^* \right)$ 
%	for the given (uniform) Bell value can be computed by the following linear program
%	\begin{eqnarray}
%	\label{lin-prog1}
%	&&\max_{ \{ P \}}: \textit{M}^T \cdot \{ P_{A,B|X,Y}( a,b| x,y) \} \nonumber \\
%	&&s.t. \; \; \textit{A} \cdot \{ P_{A,B|X,Y}(a,b |x,y) \} \leq \textit{c}.
%	\end{eqnarray}
%	Here, the indicator vector $\textit{M}$ for the Clifton gadget is a $7^2 \times 3^2$ element vector with entries 
%	%\begin{equation}
%	$M(a,b,x,y) = = 1$ if $((x,a),(y,b)) \in \mathcal{S}_B \setminus ((x^*,a^*),(y^*,b^*))$ and $0$ otherwise. 
%	The constraints on the box $\{P_{A,B|X,Y}(a,b| x,y)\}$ written as a vector with $3^2 \times 7^2$ entries are given by the matrix $\textit{A}$ and the vector $\textit{c}$. These encode the no-signaling constraints between the two parties, the normalization and the positivity constraints on the probabilities $P_{A,B|X,Y}(a,b |x,y)$. In addition, $\textit{A}$ and $\textit{c}$ also encode the condition that $\overline{B}^U \leq \frac{\tilde{\delta}}{|\textsl{X}||\textsl{Y}|}$ for a constant $\tilde{\delta} \geq 0$. \
%	%Analogous programs can be formulated for each of the $2^4$ measurement settings appearing in the Bell inequality in Eq. (\ref{Bell-ineq}) and each of the $2^4$ corresponding outputs.
%	
%	The solution to the primal linear program in Eq. (\ref{lin-prog1}) can be bounded by a feasible solution to the dual program (by the duality theorem of linear programming) which is written as
%	\begin{eqnarray}
%	\label{dual-lin-prog1}
%	&&\min_{ \lambda}: \textit{c}^T \cdot \lambda \nonumber \\
%	&& s.t. \; \; \; \textit{A}^T \cdot \lambda= \textit{M}, \nonumber \\
%	&&\; \; \; \; \; \; \; \;  \lambda(x,y,a,b) \geq 0 \; \; \; \; \forall x,y,a,b.
%	\end{eqnarray}
%	We find a feasible $\lambda$ satisfying the constraints to the dual program above that gives $\textit{c}^T \lambda\leq \frac14 (3 + 2 \tilde{\delta} )$. \footnote{The explicit vector $\lambda$ that is feasible for the dual program in Eq. (\ref{dual-lin-prog1}) and gives the bound can be computed by standard techniques and is available upon request.} We therefore obtain by strong duality of linear programming that
%	\begin{equation}
%	P_{A,B|X,Y}( a^*,b^*| x^*,y^*) \leq  \frac14 (3+ 2 \tilde{\delta}).
%	\end{equation}
%	Noting that $\tilde{\delta} =\frac{\delta}{(\frac{1}{2} - \varepsilon)^{6}}$, we obtain the required bound.
%\end{proof}

%\section{Coloring gadgets and the Kochen-Specker theorem.}\label{sec:color}
%In this section, we identify another class of substructures within Kochen-Specker graphs in relation to their coloring properties, which we term coloring gadgets.  
%%We define a coloring gadget to be a graph whose chromatic number equals its clique number, such that the graph contains two non-adjacent vertices $v_1, v_2$ with the property that in any (not necessarily optimal) coloring of the graph if $v_1$ and $v_2$ are assigned the same color, then this color is not assigned to any of the vertices in some maximum clique of the graph. We will use $\hat{v}_1, \hat{v}_2$ to denote the distinguished vertices in the gadget with this property. Formally, we have
%%
%%\begin{dfn}
%%	A \textit{coloring gadget} is defined to be a graph $G$ with $\chi(G) = \omega(G)$ and the property that $\exists \hat{v}_1 \nsim \hat{v}_2 \in V(G)$
%%	% with $(\hat{v}_1, \hat{v}_2) \notin E(G)$ 
%%	such that in any proper coloring $c : V(G) \rightarrow [n]$ if $c(\hat{v}_1) = c(\hat{v}_2)$ then $\exists$ maximum clique $Q \subset G$ with $c(v) \neq c(\hat{v}_1) \; \; \forall v \in Q$. 
%%%\end{dfn}
%%Recall that a proper coloring of a graph is by definition a partition of its vertex set into independent sets, one for each color. The above definition of a coloring gadget is seen to imply that any independent set $I \subset G$ containing the two distinguished vertices $\hat{v}_1$ and $\hat{v}_2$ (i.e., $\hat{v}_1, \hat{v}_2 \in I$) cannot also include one vertex from every maximum clique in the graph $G$. An example of a coloring gadget is provided by the famous $8$-vertex Clifton graph shown in Fig. \ref{fig:Clifton}. The Clifton graph can be recognized clearly as the sub-graph from which Kochen and Specker built their famous construction of $117$ vectors to prove their theorem \cite{KS}. This graph was used by Clifton \cite{Clifton93} to provide a \textit{statistical} KS argument in dimension $d=3$. The relation of the $101$-KS ``colorability" of the Clifton graph with its 3-colorability in the sense of graph coloring was studied in \cite{Arends09, AOW11}. There, the notion of `gadgets' was coined for graphs of this type, and the gadgets were used to show that the problem of checking for the $101$-colorability of a square-free graph is NP-complete. Our notion of coloring gadgets is a generalization of these gadgets to $\chi(G) \geq 3$.    
%\begin{figure}[t]
%	\centerline{\includegraphics[scale=0.35]{quasi-edge-not-gadget.pdf}}
%	\caption{(Left) A coloring gadget $G$ is shown with along its optimal $3$-coloring. It can be readily verified that the top and bottom vertices $u, v$ cannot be assigned the same color in any proper $3$-coloring. (Right) A $\{0,1\}$-coloring (non-contextual deterministic box $\bar{B}_{\text{nc}}^{det}(G)$) that assigns value $1$ to both $u$ and $v$.}
%	\label{fig:qedge-not-gadg}
%\end{figure} 
%%\begin{figure}[t]
%%	\centerline{\includegraphics[scale=0.39]{cabello-KSproof-gadget-4.pdf}}
%%	\caption{A $16$ vertex coloring gadget (also a $101$-gadget) that is a subgraph of the $18$ vertex Kochen-Specker graph in dimension $d=4$ found by Cabello et al. \cite{CEG96}. The $9$ edge colors (not to be confused with the vertex graph coloring in the rest of the paper)  denote $9$ cliques in the graph, with the maximum clique being of size $\omega(G) = 4$. An optimal graph coloring $(\chi(G) = 4)$ is also exhibited. The distinguished vertices $u_1, u_6$ are denoted by black circles. It can be verified that $u_1$ and $u_6$ cannot be assigned the same color in any optimal $4$-coloring of this graph, see an argument in the main text. Prop. \ref{lem:quasi-edge-subgr} shows that a coloring gadget can be found as a subgraph in any Kochen-Specker graph.}
%%	\label{fig:cab-KS-gadget}
%%\end{figure} 
%%\begin{definition}
%%We define a \textit{coloring gadget} of order $k$ to be a graph $G = (V(G), E(G))$ with the following properties:
%%\begin{itemize}
%%%\item $\chi(G) = \omega(G)$, and
%%\item there exists an independent set $S \subset V(G)$ in $G$ of size $|S| = k$ such that in any proper coloring $c : V(G) \rightarrow [n]$, if $c(u) = c(v) \; \forall u, v \in S$, then $\exists$ maximum clique $Q \in G$ with $c(w) \neq c(u) \; \; \forall w \in Q, u \in S$. 
%%%\begin{eqnarray}
%%%c(u) = c(v) \; \forall u, v \in S \Longrightarrow \exists \; \text{maximum clique} \;  Q \in G \; \text{s.t.} \; c(w) \neq c(u) \; \; \forall w \in Q, u \in S.
%%%\end{eqnarray} 
%%%$c(u) = c(v) \; \forall u, v \in S$, then $\exists$ maximum clique $Q \in G$ with $c(w) \neq c(u) \; \; \forall w \in Q, u \in S$. 
%%
%%\end{itemize}
%%\end{definition}
%%In other words, the independent set cannot be extended to include one vertex from every maximum clique in the graph $G$. 
%%In particular, in a coloring gadget of order $2$, we have that in any optimal coloring of the graph with $\chi(G)$ colors $c: V(G) \rightarrow [\chi(G)]$, there exist two distinguished vertices $u, v \in V(G)$ with $(u,v) \notin E(G)$ and $c(u) \neq c(v)$. These vertices are said to form a "quasi-edge" in the literature \cite{}. 
%%We first observe the following property of coloring gadgets which follows from their definition. 
%%\begin{obs}
%%	\label{obs:coloring}
%%	In any optimal coloring of the coloring gadget with $\chi(G) = \omega(G)$ colors, the two distinguished vertices $\hat{v}_1, \hat{v}_2$ are assigned different colors, i.e., in any optimal coloring $c: V(G) \rightarrow [\chi(G)]$ we have $c(\hat{v}_1) \neq c(\hat{v}_2)$.
%%\end{obs}
%%\begin{proof}
%%	By the property of the coloring gadget, if $c(\hat{v}_1) = c(\hat{v}_2)$ in a proper coloring, all the vertices $v \in V(Q)$ in some maximum clique $Q \subset G$ have $c(v) \neq c(\hat{v}_1)$. Since $|Q| = \omega(G)$ by the definition of a maximum clique, we need $\omega(G)$ colors to properly color all the vertices of $Q$. Together with the additional color for the distinguished vertices, we have that the number of colors in the proper coloring $c$ is at least $\omega(G) + 1$, which implies that the coloring is not optimal. 
%%\end{proof}
% We define a \textit{coloring gadget} as a graph containing two distinguished vertices $u, v$ such that in any optimal coloring $c: V(G) \rightarrow [\chi(G)]$ we have $c(u) \neq c(v)$, so that the coloring gadgets can be viewed as generalizations of $01$-gadgets. 
% %Formally, we define a coloring gadget as follows.
%\begin{dfn}
%	A \textit{coloring gadget} is a graph $G$ with $\chi(G) = \omega(G)$ and two distinguished vertices $\hat{v}_1 \nsim  \hat{v}_2 \in V(G)$ 
%	%with $(\hat{v}_1, \hat{v}_2) \notin E(G)$ 
%	such that in any optimal coloring $c : V(G) \rightarrow [\chi(G)]$, it holds that $c(\hat{v}_1) \neq c(\hat{v}_2)$.  
%	%if $c(u) = c(v)$ then $\exists$ maximum clique $Q \subset G$ with $c(w) \neq c(u) \; \; \forall w \in V(Q)$. 
%\end{dfn}
%There is a closely related notion of \textit{quasi-edges} that originates from the 3-coloring problem (a variant of the more famous 4-coloring problem) for planar graphs in graph theory \cite{Steinberg93, AM78, AM80}. In particular, Aksionov and Mel'nikov \cite{AM78, AM80} define a quasi-edge to be a graph embedded in the plane containing a pair of nonadjacent vertices $\hat{v}_1$ and $\hat{v}_2$ lying in the outer face which are assigned different colors in any 3-coloring of the graph.
%
%It is worth noting that the $8$-vertex Clifton graph from Fig. \ref{fig:Clifton} (which has $\chi(G) = \omega(G) = 3$) is also a coloring gadget, which was proven to be minimal in \cite{AM78, AM80, Arends09}. Another example of a coloring gadget (that is also a $01$-gadget) with $\chi(G) = \omega(G) = 4$ is the subgraph of the $18$-vertex KS graph found by Cabello shown in Fig. \ref{fig:cab-KS-gadget}. That the vertices $u_1, u_6$ cannot be assigned the same color in a $4$-coloring of this graph $c : V(G) \rightarrow \{1,2,3,4\}$ can be seen as follows. Suppose $c(u_1) = c(u_6) = 1$. Then all the vertices $u_2, \dots, u_5$ and $u_7, \dots, u_{10}$ cannot be assigned this color due to their adjacency to either $u_1$ or $u_6$. Now, one of $(u_{13},u_{14}, u_{15}, u_{16})$ must be assigned color $1$ as well as one of $(u_{11}, u_{13})$, $(u_{11}, u_{14})$, $(u_{12}, u_{15})$ and $(u_{12}, u_{16})$. These five constraints cannot all be satisfied simultaneously. 
%
%
%Let us note that in general, the $01$-gadgets that come from the Kochen-Specker problem do not correspond precisely to the notion of coloring gadgets from the graph coloring problem. An example of a coloring gadget that is not a $01$-gadget is shown in Fig. \ref{fig:qedge-not-gadg}. Moreover, the well-known $13$-vertex Yu-Oh graph shown in Fig. \ref{fig:YO} provides an example of a $01$-gadget that is not a coloring gadget (this graph has $\chi(G) > \omega(G)$). Nevertheless, it is readily seen that a $01$-gadget with $\chi(G) = \omega(G)$ is a coloring gadget. To see this, notice that if an optimal coloring $c : V(G) \rightarrow [\chi(G)]$ exists in which $c(v_1) = c(v_2) = 1$ for the distinguished vertices $v_1$ and $v_2$, then one can also obtain a $\{0,1\}$-coloring $f : V(G) \rightarrow \{0,1\}$ as $f(v) = 1 \Leftrightarrow c(v) = 1$ and $f(u) = 0$ otherwise. This $\{0,1\}$ assignment violates the $01$-gadget property, showing the claim. 
%
%\begin{figure}[t]
%	\centerline{\includegraphics[scale=0.42]{YO-101-not-coloring-gadget.pdf}}
%	\caption{The $13$-vertex Yu-Oh graph \cite{YO12} that has chromatic number greater than clique number. No pair of vertices among $u_4, u_5, u_6, u_7$ can be simultaneously assigned the value $1$ in any $\{0,1\}$-coloring.}
%	\label{fig:YO}
%\end{figure}
%%We study both notions in this paper in characterizing the randomness from Kochen-Specker proofs and more general state-dependent contextuality. 
%%\begin{rem}
%%	All coloring gadget graphs are GQEs by Observation \ref{obs:coloring} but not all GQEs are coloring gadgets, an example of a GQE that is not a coloring gadget is presented in Fig. \ref{fig:qedge-not-gadg}.
%%\end{rem}
%%\begin{proof}
%%The fact that all coloring gadgets are generalized quasi-edges follows from Observation \ref{obs:coloring}. Fig. \ref{} shows an example of a generalized quasi-edge that is not a coloring gadget.
%%\end{proof}
%%
%%
%%We first establish the connection between coloring gadget graphs and the corresponding set of (non-contextual) boxes with the following lemma. 
%%\begin{lemma}
%%	\label{lem:col-gad-nc}
%%	For any coloring gadget $G$ with distinguished vertices $\hat{v}_1, \hat{v}_2 \in V(G)$, no normalized non-contextual box $\bar{B}_{\text{nc}}(G)$ exists with $P_{\text{nc}}(\hat{v}_1) = P_{\text{nc}}(\hat{v}_2) = 1$. Conversely, any graph $G$ with $\chi(G) = \omega(G)$ containing two distinct vertices $\hat{v}_1, \hat{v}_2 \in V(G)$ with the property that no normalized non-contextual box $\bar{B}_{\text{nc}}(G)$ exists with $P_{\text{nc}}(\hat{v}_1) = P_{\text{nc}}(\hat{v}_2) = 1$ is a coloring gadget.  
%%\end{lemma}
%%\begin{proof}
%%	Suppose by contradiction that there exists a normalized non-contextual box $\bar{B}_{\text{nc}}(G)$ with $P_{\text{nc}}(\hat{v}_1) = P_{\text{nc}}(\hat{v}_2) = 1$. It is readily seen by considering the sum of probabilities in a maximum clique that in a convex decomposition of $\bar{B}_{\text{nc}}(G)$ into non-contextual deterministic boxes, every box appearing in the decomposition must be normalized. Moreover, for any such convex decomposition into normalized non-contextual deterministic boxes,
%%	\begin{equation}
%%	\bar{B}_{\text{nc}}(G) = \sum_{i} p_i \bar{B}_{\text{nc}}^{\text{det}, i}(G), \; \; \; \sum_i p_i = 1, \; \; p_i \geq 0,
%%	\end{equation}
%%	we have 
%%	%by simply considering the decomposition of $P_L(u)$ and $P_L(v)$ 
%%	that for all $i$ there is $P_{\text{nc}}^{det,i}(\hat{v}_1) = P_{\text{nc}}^{det,i}(\hat{v}_2) = 1$, so that it suffices to show the contradiction for normalized non-contextual deterministic boxes.
%%	
%%	Let us consider one such normalized non-contextual deterministic box $\bar{B}_{\text{nc}}^{det}(G)$ which has $P_{\text{nc}}^{det}(\hat{v}_1) = P_{\text{nc}}^{det}(\hat{v}_2) = 1$. Denote by $I$ the set of vertices which are assigned value unity in this box, i.e., $I = \{v \in V(G) \vert P_{\text{nc}}^{det}(v) = 1\}$. By the property of the consistent assignment, we have that $I$ constitutes an independent set of the graph $G$ and by assumption $\hat{v}_1, \hat{v}_2 \in I$. Also, note that due to the assumption of normalization of the box, there must exist for each maximum clique $Q$ in $G$, one vertex $v \in Q$ that gets assigned value $1$, so that by construction such a vertex is also contained in $I$. Now, consider a proper coloring $c$ of the graph with $I$ being one of the color classes (i.e., the independent set corresponding to one of the colors in the coloring). In this proper coloring, 
%%	%by the requirement that $\sum_{u \in Q} P_L^{det,i^*}(u) = 1$ for all maximum cliques in the local deterministic box, 
%%	we have that for every maximum clique $Q$, $\exists v \in Q$ with $v \in I$, a contradiction with the definition of the coloring gadget. 
%%	
%%	To see the converse, suppose that there exists a graph $G$ with $\chi(G) = \omega(G)$ and two vertices $\hat{v}_1, \hat{v}_2 \in V(G)$ with the property that no normalized non-contextual box $\bar{B}_{\text{nc}}(G)$ exists satisfying $P_{\text{nc}}(\hat{v}_1) = P_{\text{nc}}(\hat{v}_2) = 1$. This implies that any probability assignment $P \in \{0, 1\}$ to the vertices of the graph satisfying $P(\hat{v}_1) = P(\hat{v}_2) = 1$ fails to satisfy $\sum_{v \in Q} P(v) = 1$ for some maximum clique $Q$, i.e., assigns value $0$ to all the vertices of $Q$. In other words, no independent set $I \subset G$ exists with $\hat{v}_1, \hat{v}_2 \in I$ and such that for each maximum clique $Q \subset G$, $\exists v \in Q$ with also $v \in I$. Noting that independent sets correspond to color classes gives the required coloring gadget property of $G$. 
%%\end{proof}
%%
%%\begin{rem} 
%%	One cannot substitute generalized quasi-edges in place of coloring gadgets in Lemma \ref{lem:col-gad-nc}, an example of a GQE $G$ and corresponding box $\bar{B}_{\text{nc}}(G)$ with $P_{\text{nc}}(\hat{v}_1) = P_{\text{nc}}(\hat{v}_2) = 1$ is shown in Fig.\ref{fig:qedge-not-gadg}.  
%%	%There exist quasi-edge graphs $G$ with $\chi(G) = \omega(G)$ that are not coloring gadgets, consequently local boxes exist assigning value $1$ to the two distinguished vertices of these quasi-edges.
%%\end{rem}
%
%%\begin{rem}
%%	There is no loss of generality in assuming that for any coloring gadget $G$ of order $2$, every edge not incident on one of the distinguished vertices belongs to a maximum clique. In other words, any vertex belonging only to cliques containing a distinguished vertex can be removed without altering the gadget property. 
%%\end{rem}
%%Let us now consider generalized quasi-edges as subgraphs needed for the construction of graphs with chromatic number larger than clique number. 
%%The graphs that are of most interest to us are the physically relevant orthogonality graphs of the Kochen-Specker sets of vectors. In fact, there are also sets of vectors which while not strictly proving the Kochen-Specker theorem, nevertheless can be used to test \textit{state-independent contextuality} in a Hilbert space of fixed dimension $d \geq 3$. The first such set was discovered by Yu and Oh for dimension $d = 3$ in \cite{YO12}. In what follows, we will use the term \textit{Kochen-Specker graph} to denote any such set of vectors that serves as a proof of state-independent contextuality, including the Kochen-Specker sets as well as sets such as in \cite{YO12}. 
%\begin{prop}
%	\label{lem:quasi-edge-subgr}
%	Every graph $G$ with $\chi(G) > \omega(G) \geq 3$ contains a coloring gadget $K$ as a subgraph with $\chi(K) = \omega(K) = \omega(G)$. In particular, every Kochen-Specker graph $G$ in dimension $d$ contains a coloring gadget $K$ as a subgraph with $\chi(K) = \omega(K) = d$. 
%\end{prop}
%
%
%
%
%
%%
%%\begin{cor}
%%	\label{cor:KS-gadget}
%%	Every Kochen-Specker graph $G$ in dimension $d$ contains a generalized quasi-edge $K$ as a subgraph with $\chi(K) = \omega(K) = d$. 
%%\end{cor}
%%\begin{proof}
%%	By the results in \cite{RH14, CKB15}, we know that $\chi(G) > \omega(G)$ is a necessary condition for an orthogonality graph to represent a set of projectors with the Kochen-Specker (KS) property, i.e., in \cite{RH14, CKB15} the following Lemma was proved. 
%%	\begin{lemma}[\cite{RH14, CKB15}]
%%		\label{lem:KS-chi}
%%		A necessary condition for an orthogonality graph $G$ to represent a set of Kochen-Specker set of rank-one projectors $\{P_{v_j}\}$ with $|v_j \rangle \in \mathbb{C}^d$ is that $\chi_f(G) > d$.
%%	\end{lemma}
%%	For any graph $G$, we know that $\chi_f(G) \leq \chi(G)$ so that by Lemma \ref{lem:KS-chi} we have that the orthogonality graph of any Kochen-Specker set must satisfy $\chi(G) > d$. Also, $\omega(G) = d$ for the Kochen-Specker graph since the size of a maximum clique is equal to the size of the basis in $\mathbb{C}^d$, so that $\chi(G) > \omega(G)$ for the KS graphs. By Lemma \ref{lem:quasi-edge-subgr}, we see that every Kochen-Specker graph contains a generalized quasi-edge as a subgraph.  
%%\end{proof}
%
%%\begin{lemma}
%%For any coloring gadget $G$ of order $2$ with distinguished vertices $u, v \in V(G)$, no local box $\mathbf{B}_L(G)$ exists satisfying the following constraints:
%%\begin{itemize}
%%\item Normalization of probabilities for every maximum clique $Q \in G$, and
%%\item $P_L(u) = P_L(v) = 1$.
%%\end{itemize}
%%\end{lemma}
%
%%\begin{rem}
%	Note that the proof of Prop. \ref{lem:quasi-edge-subgr} does not guarantee that the coloring gadget $K$ is an \textit{induced} subgraph of $G$. 
%	%An example of a coloring (as well as a $01$-) gadget within the $18$-vertex graph of Cabello et al. \cite{Cab08} is shown in Fig. \ref{fig:cab-KS-gadget}.  
%	We also observe the following.
%\begin{obs} 
%	Let $G$ be a graph with $\chi(G) > \omega(G) \geq 3$. 
%	If there exists a vertex-critical induced subgraph $\tilde{H}$ of $G$ with $\chi(\tilde{H}) = \omega(G)+1$ and $\delta(\tilde{H}) = \omega(G)$, then $G$ contains a coloring gadget $K$ as an \textit{induced} subgraph with $\chi(K) = \omega(G)$. 
%\end{obs}
%
%
%Kochen-Specker sets have been related to pseudo-telepathy games via a graph coloring game \cite{BBT05, CMNSW07}. In the two-player coloring game for a graph $G$ \cite{BBT05}, two players claim that they have a proper $n$-coloring for $G$. To check this, the referee forbids communication between the players and separately sends two vertices $u,v$ to Alice and Bob to which they reply with the colors $a, b \in [n]$ respectively. The winning constraints of the game are then the following:
%\begin{itemize}
%	\item If $u = v$, then $a = b$.
%	
%	\item If $(u,v) \in E(G)$, then $a \neq b$.  
%\end{itemize}  
%To win the game with certainty classically, the players both have to use a proper coloring $c : V(G) \rightarrow [n]$, so that they can only win if $n \geq \chi(G)$. In contrast, for (weak) Kochen-Specker sets it was shown in \cite{SS12} that one can have a quantum winning strategy with the corresponding parameter called the quantum chromatic number $\chi_q(G)$ being smaller, i.e., with $\chi_q(G) < \chi(G)$. Moreover, weak KS sets were shown to characterize all graphs with this property \cite{SS12}.  
%
%The coloring gadgets within Kochen-Specker proofs lead to Hardy paradoxes via a modification of the graph coloring game \cite{CMNSW07} with the following additional constraint, namely:
%
%\begin{itemize}
%	\item If $u = u^*$ and $v = v^*$ then $a = b$, 
%\end{itemize}
%with $u^*, v^*$ denoting distinguished vertices of the gadget.
%In other words, the Hardy paradox associated to the coloring gadget is given by
%
%\begin{enumerate}
%	\item $P_{A,B|U,V}(a \neq b|u,u) = 0$.
%	
%	\item $P_{A,B|U,V}(a = b|u,v) = 0, \; \; \; \forall (u,v) \in E(G)$.
%	
%	\item $P_{A,B|U,V}(a = b|u^*, v^*) > 0$.
%\end{enumerate}
%We have that for $n = \chi(G)$, the players' classical strategy to satisfy the constraints (1) and (2) is to respond with an optimal coloring $c : V(G) \rightarrow [\chi(G)]$. In which case, by the definition of the coloring gadget, in the classical strategy we have for the constraint (3) that $P_{A,B|X,Y}(a = b|u^*, v^*) = 0$. In contrast, a quantum strategy exists given by the faithful orthogonal representation of the coloring gadget, in which this probability is equal to ($\frac{1}{d}$) times the square of the inner product between the vectors representing the distinguished vertices. 
%%In dimensions four and eight, by the construction given in the previous section, we obtain Hardy paradoxes with the non-zero probability in $(0,1]$. Notice that in the extreme limit where this probability is unity, the corresponding set of projectors forms a weak Kochen-Specker set by the result of \cite{SS12}. Further investigations into this are carried out in \cite{R17}.


%\section{Identifying state-independent contextual graphs.}
%\label{sec:SIC-graphs}
%While the general problem of identifying graphs that are not $\{0,1\}$-colorable was seen to be NP-hard in Section \ref{sec:Comp-complexity}, in some cases the problem may be simplified. For instance, in the well-known parity proofs of the KS theorem, the non-colorability is easy to see. To elaborate, a KS parity set is one in which the vectors comprise an odd number of complete contexts, in such a way that each vector belongs to an even
%number of the contexts. Then a $\{0,1\}$-coloring of such a set would need to have an odd number of 1's, one for each context,
%but an even number in order to obey non-contextuality, making such a coloring impossible \cite{WA17}. 
%
%In \cite{RH14, CKB15}, a simple necessary condition was given, in terms of graph colorings, to identify the orthogonality graph of a state-independent contextual set of vectors in $\mathbb{C}^d$. This condition was simply given by $\chi_f(G) > d$ and consequently $\chi(G) > d$. On the other hand, this condition is not sufficient, since one can readily construct graphs with $\chi(G) > d$ for which no orthogonal representation exists, by a state-independent contextual set of vectors \cite{CKB15}. In this section, we show that in $\mathbb{C}^4, \mathbb{C}^8$, any graph $G$ that satisfies $\chi(G) > d$ gives rise to a KS proof by taking $d$ copies of $G$. This identification of state-independent contextual graphs in dimensions $4, 8$ is encapsulated by the following proposition.  
%
%\begin{prop}
%	Suppose $S = \{|v_i \rangle\} \subset \mathbb{R}^d$ for $d= 4, 8$ denotes a set of vectors such that $\chi(G_S) > d$. Then, a unitary $U$ exists such that $\bigcup_{k=0}^{d-1} \{U^k |v_i \rangle\}$ is a weak Kochen-Specker set. 
%\end{prop} 
%
%\begin{proof}
%	The proof is based on the fact that in dimensions $4$ and $8$ there exist division algebras \cite{Cohn03} (the quaternions and octonions) so that one can rotate each vector $(a_1,a_2,a_3,a_4) \in \mathbb{R}^4$ in a set of vectors, to the orthogonal vectors $(a_2,-a_1,-a_4,a_3), (a_3,a_4,-a_1,-a_2), (a_4,-a_3,a_2,-a_1)$ by multiplication by the orthogonal units $i,j$ or $k$ of the algebra, i.e., to every vector $(a_1,a_2,a_3,a_4) \in \mathbb{R}^4$, one can associate the real orthogonal matrix
%	\begin{eqnarray} 
%	\label{eq:real-ort-mat}
%	\begin{pmatrix} a_1 & a_2 & a_3 & a_4\\ a_2 & -a_1 & a_4 & -a_3 \\ a_3 & -a_4 & -a_1 & a_2 \\ a_4 & a_3 & -a_2 & -a_1 \end{pmatrix}
%	\end{eqnarray}
%	A similar construction also exists in dimension eight by means of the octonions, i.e., to every vector $(a_1, a_2, a_3, a_4, a_5, a_6, a_7, a_8) \in \mathbb{R}^8$, one can associate the real orthogonal matrix
%	\begin{eqnarray} 
%	\label{eq:real-ort-mat-2}
%	\begin{pmatrix} a_1 & a_2 & a_3 & a_4 & a_5 & a_6 & a_7 & a_8\\ a_2 & -a_1 & a_4 & -a_3 & a_6 & -a_5 & -a_8 & a_7\\ a_3 & -a_4 & -a_1 & a_2 & a_7 & a_8 & -a_5 & -a_6 \\ a_4 & a_3 & -a_2 & -a_1 & a_8 & -a_7 & a_6 & -a_5 \\ a_5 & -a_6 & -a_7 & -a_8 & -a_1 & a_2 & a_3 & a_4\\ a_6 & a_5 & -a_8 & a_7 & -a_2 & -a_1 & -a_4 & a_3\\ a_7 & a_8 & a_5 & -a_6 & -a_3 & a_4 & -a_1 & -a_2\\a_8 & -a_7 & a_6 & a_5 & -a_4 & -a_3 & a_2 & -a_1 \end{pmatrix}
%	\end{eqnarray}  
%	So that again one can rotate each vector $(a_1,a_2,a_3,a_4,a_5,a_6,a_7,a_8) \in \mathbb{R}^8$ into a set of basis vectors (given by the rows of the above matrix) using a unitary $\mathcal{U}_8$. 
%	We would now like to show that the set $\tilde{S} := \bigcup_{k=0}^{d-1} \{\mathcal{U}_d^k |v_i \rangle\}$ constitutes a Kochen-Specker set for $d= 4,8$ when the graph $G_S$ for $S = \{|v_i \rangle \}$ obeys $\chi(G) > d$. 
%	
%	The idea now is to construct a pseudo-telepathy game based on $\tilde{S}$. Then, by the result of Renner and Wolf in \cite{RW04}, it will follow that $\tilde{S}$ constitutes a weak Kochen-Specker set. To this end, we construct a graph $\tilde{G}$ from $G_{S}$ for which Alice and Bob play the graph coloring game. We begin by removing vertices and edges from $G_{S}$ until we obtain a vertex-critical and edge-critical graph $G'$ that has $\chi(G') > d$, and the removal of any more vertices or edges renders the chromatic number equal to $d$. Observe that by the criticality of $G'$, any vertex $w \in V(G')$ has at least $d$ neighbors, i.e., $|N(w)| \geq d$. Duplicate a vertex $w$ to obtain two new vertices $w_1, w_2$ in $\tilde{G}$ such that $w_1$ has only one neighbor, and $w_2$ is adjacent to the rest of the neighbors of $w$, i.e., $|N(w_1)| = 1$ and $|N(w_2)| = |N(w)| - 1$. We have that $\chi(\tilde{G}) = d$  and that in any optimal coloring $c$ of $\tilde{G}$, $c(w_1) \neq c(w_2)$. 
%	
%	Now consider the following two-player coloring game for $\tilde{G}$. The referee sends vertices $u,v \in V(\tilde{G})$ to Alice and Bob to which they reply with the colors $a, b \in [\chi(\tilde{G})]$, respectively. The winning constraints are given as follows:
%	   \begin{itemize}
%	   	\item If $u = v$, then $a = b$.
%	   	
%	   	\item If $(u,v) \in E(G)$, then $a \neq b$. 
%	   	
%	   	\item If $(u,v) = (w_1, w_2)$, then $c(w_1) = c(w_2)$. 
%	   \end{itemize} 
%   	As we have seen, by the construction of $\tilde{G}$, in any optimal coloring $c$ of $\tilde{G}$, $c(w_1) \neq c(w_2)$ so that the final condition above cannot be met. In other words, no classical strategy exists to satisfy the constraints of the above coloring game for $\tilde{G}$. 
%   	
%   	On the other hand, an orthogonal representation of $\tilde{G}$ exists given by the appropriate subset of $S$, although note that the representation is not faithful since both $w_1$ and $w_2$ are represented by the same vector $|w \rangle$. This orthogonal representation can be used to construct a quantum strategy for Alice and Bob that wins the game with probability $1$. In the optimal quantum strategy, Alice and Bob share a maximally entangled state of dimension $d$, when Bob (Alice) is given vertex $v \in \tilde{G}$, (s)he measures her half of the state in the basis given by $\{\mathcal{U}_d^k|v \rangle\}_{k=0}^{d-1}$, and returns the corresponding outcome as his (her) output $b (a)$. It is easy to see that this strategy satisfies all the constraints so that we have constructed a pseudo-telepathy game using the set of vectors $\tilde{S}$. By the results of \cite{RW04}, we infer that $\tilde{S}$ is a weak Kochen-Specker set for $\mathbb{C}^d$.  
%   
%\end{proof}
%\color{black}

\section{Randomness from $01$-gadgets}
In this section, we give a brief outline of how $01$-gadgets may be linked to device-independent randomness certification.
%, deferring the formulation of a fully device-independent protocol to a separate paper \cite{R17}. We shall describe the idea in terms of the Clifton gadget in Fig. \ref{fig:Clifton} although it can be readily extended to any $01$-gadget. 
Namely, when two parties Alice and Bob perform locally the measurements from the Clifton gadget on their half of a maximally entangled state (in $\mathbb{C}^3 \otimes \mathbb{C}^3$), we will show that some specific outcome of their joint measurements has probability bounded from above and below (and this holds in all no-signaling theories). This can be
inserted into a fully device-independent protocol as given in \cite{PRL-rand}, the details are deferred to a separate paper \cite{R17}.
To show how the Clifton gadget can be used for randomness amplification we first consider a non-contextual assignment
of probabilities to its vertices $v$ satisfying
\be
\label{eq:klik}
\sum_{v\in \text{clique}}p_v\leq 1,\quad \sum_{v\in \text{maximum clique}} p_v=1
\ee
This is the same requirement as Eq.(\ref{eq:01rule}), but we now assign not necessarily zeros and ones, but probabilities (i.e., values in $[0,1]$ rather than in $\{0,1\}$). Recall that such an assignment was also considered in our discussion of the extended Kochen-Specker theorem in Section \ref{sec:ext}.  
Now, since the gadgets are $\{0,1\}$ colorable, such an assignment of zeros and ones is possible, although in the $\{0,1\}$ assignment, it is not possible to assign $1$'s to both vertices $1$ and $8$. 
%However, as we have explained below equation \eqref{eq:Clif-orth-rep}, it is not possible to assign $1$'s to both vertices $1$ and $8$. 
Here, we will first show, that even if we assign probabilities, we still cannot have $p_1=p_8=1$, and we will provide a quantitative bound for this.
Indeed, let us write  Eq.\eqref{eq:klik} explicitly for the cliques in question from the Clifton gadget in Fig. \ref{fig:Clifton}:
\ben
\label{eq:nonmax}
&&p_1+p_2\leq 1, \quad p_1+p_6\leq 1, \quad p_4+p_5\leq 1, \nonumber \\
&&\quad p_7+p_8\leq 1, \quad p_3+p_2\leq 1
\een
for non-maximal cliques and 
\be
\label{eq:max}
p_2+p_3+p_4=1, \quad  p_5+p_6+p_7=1
\ee
for the two maximum cliques. 
We sum up all the inequalities \eqref{eq:nonmax}, and get 
\be
2p_1 + p_2+p_3+p_4+p_5+p_6+p_7+ 2 p_8 \leq 5.
\ee
Using \eqref{eq:max} we then obtain
\be
\label{eq:nsbound}
p_1+p_8\leq \frac32.
\ee
To exploit this feature for randomness amplification, 
we consider a maximally entangled state shared by two parties. The parties will measure observables composed of the 
projectors given by the quantum representation (if the clique is not maximal, one simply adds a third orthogonal projector to obtain a complete measurement).
Recall here that a set of eight projectors $P_v=|u_v\>\<u_v|$ that is compatible with the Clifton graph
%(i.e. if pair of projectors is orthogonal, we have an edge) 
is given in Eq.\eqref{eq:Clif-orth-rep}.
Projectors of Alice will be denoted $A_v$ and those of Bob $B_v$, 
and the probability of obtaining outcome $v$, while measuring observable containing $v$, will be denoted by $p(A_v=1)$.
We correspondingly denote by  $p(A_v=0)$ the probability that the outcome $v$ was not obtained. Clearly $p(A_v=1)+p(A_v=0)=1$.
Now, we shall show using no-signaling (which will impose non-contextuality), that the probability $p(A_1=1,B_8=1)$ 
is bounded from above. 
To see this, we apply Eq.\eqref{eq:nsbound} to Alice's observables and get
\ben
&&p(A_1=1)+p(A_8=1)\leq \frac32
%&&p(B_1=1)+p(B_8=1)\leq \frac32
\een
From the correlations in the maximally entangled state, we have that 
\be
p(A_8=1)=p(B_8=1)
\ee
giving 
\be
\label{eq:AB32}
p(A_1 = 1)+p(B_8 = 1)\leq \frac32.
\ee
Now, from no-signaling we have 
\ben
&&p(A_1=1)=  p(A_1 = 1, B_8 = 0) + p(A_1 = 1, B_8 = 1), \nonumber \\
&&p(B_8 = 1) =  p(A_1 = 0, B_8 = 1) + p(A_1 = 1, B_8 = 1).  \nonumber\\
\een
Summing these and applying \eqref{eq:AB32} we get 
\ben
&&p(A_1 = 1, B_8 = 0) + p(A_1 = 1, B_8 = 1)   +  \nonumber \\
&& p(A_1 = 0, B_8 = 1) + p(A_1 = 1, B_8 = 1) \leq   \frac32 
\een
and hence 
\be
p(A_1=1,B_8=1)\leq \frac34.
\ee
Thus we have obtained, that the probability of the event $(A_1,B_8)=(1,1)$ is bounded from above. 
We have also the lower bound
\be
p(A_1=1,B_8=1) = \frac13  |\<u_1|u_8\>|^2 \geq \frac1{27}.
\ee
Thus 
\be
\frac1{27}\leq p(A_1=1,B_8=1) \leq \frac34
\ee
Therefore, the outcome $(A_1,B_8)=(1,1)$ has randomness, which can be used in a randomness amplification scheme employing the protocol of \cite{PRL-rand}.
The lower bound is $\frac1{27}$ in noiseless conditions, and assuming we have exactly measured the specified projectors. 
In a real experiment, this value may be different, but if the noise is low enough it should be close to $\frac1{27}$.
Also the upper bound, relies on perfect correlations, which in a real experiment may be imperfect. Thus in noisy conditions, we will 
have less stringent lower and upper bounds, though these are certifiable by statistics from the experiment. 
Note that crucially we have not used explicitly Bell inequalities, nor even the KS paradox. We have simply made use of the perfect correlations between the parties and the local  $01$-gadget structure of Alice and Bob's observables. \\

\section{Conclusion and Open Questions}\label{sec:concl}
In this paper, we have shown that there exist interesting subgraphs of the Kochen-Specker graphs that we termed $01$-gadgets that encapsulate the main contradiction necessary to prove the  Kochen-Specker theorem. Furthermore, as a main technical contribution, we have shown that the fundamental structures identified here, lead to clean constructions of state-independent statistical proofs of the KS theorem, of which the famous Yu and Oh proof is a particular case. The proofs given here provide a new perspective on these results, and serve as a useful tool to construct minimal KS sets, since efforts may be concentrated on the $01$-gadget subgraphs. 
%we have provided simple constructive proofs of statistical KS arguments and state-independent non-contextuality inequalities for any $d \geq 3$ using these gadgets. 
An extended notion of $01$-gadgets also helped to provide simple constructive proofs of the extended Kochen-Specker theorem \cite{Pitowsky}. The gadgets enable a proof of the NP-completeness of checking $\{0,1\}$-colorability of graphs free from the forbidden subgraphs from Hilbert spaces of any dimension. Practically, the gadgets open up a highly important application of contextuality to practical device-independent randomness generation, which we study in a companion paper \cite{R17} where we provide an explicit device-independent protocol for randomness amplification based on \cite{BRGH+16, PRL-rand, WBGH+16} and Hardy paradoxes constructed using $01$-gadgets. %Another interesting class of subgraphs of Kochen-Specker graphs were also found in relation to their coloring properties, that we termed coloring gadgets. Finally, in dimensions four and eight, we gave a simple condition for identifying Kochen-Specker graphs based on their coloring properties. 
%Both $01$ and coloring gadgets give rise to Hardy paradoxes, by means of the Kochen-Specker and the coloring game, respectively \cite{MH17}. 


%The fundamental structures identified here, led to a very clean construction of state independent statistical proofs of KS theorem, of which the famous Yu and Oh proof is a particular case. The proof presented in this contribution gives a new perspective on this kind of result that makes it much more easy to follow. I also believe that from this work many proofs of minimality of KS sets will follow, since we can now concentrate our efforts in this type of subgraph.

An open question, is to find, for given overlap  $| \langle v_1 | v_2 \rangle|$, the minimal $01$-gadget and extended $01$-gadget with the corresponding vertices $v_1,v_2$ playing the role of the distinguished vertices. An answer to this question would have applications for randomness generation from contextuality \cite{R17}. 
%Another interesting question is to study the relationship between the $01$ and coloring gadgets and to construct further interesting examples of these.
Another open question is whether all state-independent contextual graphs (including those going beyond KS sets such as that of Yu and Oh \cite{YO12}) contain $01$-gadgets as subgraphs, or even possibly as induced subgraphs. Finally, while it is known that in $\mathbb{C}^3$ KS sets cannot be constructed using rational vectors \cite{GN08}, it would be very interesting to study quantum realizations of $01$-gadgets using rational vectors, to build statistical KS arguments and state-independent non-contextuality inequalities using these.  

% A further interesting question concerns the possible generalizations of these Hardy paradoxes to multi-party Bell scenarios \cite{ACY16}.  

\textit{Acknowledgments.-}
We are grateful to Andrzej Grudka, Waldemar K\l obus and David Roberson for useful discussions.
R.R. acknowledges support from the research project  ``Causality in quantum theory: foundations and applications'' of the Fondation Wiener-Anspach and from the Interuniversity Attraction Poles 5 program of the Belgian Science Policy Office under the grant IAP P7-35 photonics@be. This work is supported by the Start-up Fund 'Device-Independent Quantum Communication Networks' from The University of Hong Kong. This work was supported by the National Natural Science Foundation of China through grant 11675136, the Hong Kong Research Grant Council through grant 17300918, and the John Templeton Foundation through grants 60609, Quantum Causal Structures, and 61466, The Quantum Information Structure of Spacetime (qiss.fr). M. R. is supported by the National Science Centre, Poland, grant OPUS 9. 2015/17/B/ST2/01945. MH and PH are supported by the John Templeton Foundation. The opinions expressed in this publication are those of the authors and do not necessarily reflect the views of the John Templeton Foundation. PH also acknowledges partial support from the Foundation for Polish Science (IRAP project, ICTQT, contract no. 2018/MAB/5, co-financed by EU within the Smart Growth Operational Programme). KH acknowledges support from the grant Sonata Bis 5 (grant number: 2015/18/E/ST2/00327) from the National Science Centre. S.P. is a Research Associate of the Fonds de la Recherche Scientifique (F.R.S.-FNRS). We acknowledge support from the EU Quantum Flagship project QRANGE.


%to be composed of two distinguished vertices and maximum cliques 
%\begin{dfn}
%A \textit{coloring gadget} of order $k$ is a graph $G$ containing an independent set $S\subset V$ of $k$ vertices such that in any proper coloring $c:V(G)\mapsto\{1,2,...,\chi(G)\}$ with $\chi$ colors $\left|c(S)\right|>1$.
%\end{dfn}

%\begin{thm}
%Every graph such that $\chi>\omega$ contains a coloring gadget as a subgraph.
%\end{thm}

%\begin{proof}
%\end{proof}
%
%\begin{dfn}
%A \textit{randomness gadget} of order $k$ (in a given dimension $d$) is a graph $G$ containing an independent set $S\subset V$ of $k$ vertices such that in any assignment $f:V(G)\mapsto\{0,1\}$ assigning at most one $1$ to each clique and exactly one $1$ to every clique of size $d$ there exists a vertex $v\in S$ such that $f(v)=0$.
%\end{dfn}
%
%\begin{dfn}
%[quantum randomness gadget]
%\end{dfn}
%
%[randomness, k-party randomness (using gadgets of order k)]
%
%\begin{lemma}
%Every randomness gadget is a coloring gadget.
%\end{lemma}
%
%\begin{proof}
%[if G is not a coloring gadget then it is not a randomness gadget]
%
%\end{proof}
%
%\begin{obs}
%Not every coloring gadget is a randomness gadget.
%\end{obs}
%
%[example]
%
%\begin{obs}
%There are no quantum randomness gadgets in dimension 2.
%\end{obs}
%
%\begin{dfn}
%[perfect graph]
%\end{dfn}
%
%\begin{conj}
%No perfect graph is a quantum randomness gadget.
%\end{conj}
%
%Perfect graphs can, however, be coloring and randomness gadgets.

%An interesting connection between graph planarity and Peres-Mermin type proofs of the Kochen-Specker theorem was discovered in \cite{Ark12}. 

\begin{thebibliography}{99}

%\bibitem{Bell}
%J. S. Bell. 
%\textit{On the problem of hidden variables in quantum mechanics}. 
%Reviews of Modern Physics \textbf{38}, 447 (1966).

\bibitem{KS}
S. Kochen and E.P. Specker.
\textit{The Problem of Hidden Variables in Quantum Mechanics}.
\href{https://www.jstor.org/stable/24902153}{Journal of Mathematics and Mechanics Vol. 17, No. 1, 59 (1967)}.


\bibitem{ACS15}
A. A. Abbott, C. S. Calude and K. Svozil.
\textit{A variant of the Kochen-Specker theorem localising value indefiniteness}.
\href{https://doi.org/10.1063/1.4931658}{Journal of Mathematical Physics 56, 102201 (2015)}. 

\bibitem{ACCS12}
A. A. Abbott, C. S. Calude, J. Conder and K. Svozil.
\textit{Strong Kochen-Specker theorem and incomputability of quantum randomness}.
\href{https://doi.org/10.1103/PhysRevA.86.062109}{Phys. Rev. A 86(6), 062109 (2012)}.

\bibitem{KCBS08}
A. A. Klyachko, M. A. Can, S. Binicoglu and A. S. Shumovsky. \textit{Simple Test for Hidden Variables in Spin-1 Systems}.
\href{https://doi.org/10.1103/PhysRevLett.101.020403}{Phys. Rev. Lett. 101, 020403 (2008)}.

\bibitem{GN08}
C. D. Godsil and M. W. Newman. 
\textit{Coloring an orthogonality graph}. 
\href{https://doi.org/10.1137/050639715}{SIAM J. Discrete Math., 22(2): 683 (2008)}. 
%arXiv:math/0509151.

%\bibitem{MSS13}
%L. Mancinska, G. Scarpa and S. Severini.
%\textit{A Generalization of Kochen-Specker Sets Relates Quantum Coloring to Entanglement-Assisted Channel Capacity}.
%IEEE Transactions on Information Theory \textbf{59}, 6 (2013). 

\bibitem{CSW14}
A. Cabello, S. Severini and A. Winter.
\textit{Graph-Theoretic Approach to Quantum Correlations}.
\href{https://doi.org/10.1103/PhysRevLett.112.040401}{Phys. Rev. Lett. 112, 040401 (2014)}. 

\bibitem{CSW10}
A. Cabello, S. Severini and A. Winter.
\textit{(Non-)Contextuality of Physical Theories as an Axiom}
\href{http://www.mittag-leffler.se/sites/default/files/IML-1011f-08.pdf}{Mittag-Leffler-2010fall Report No. 8 (2010)}.
%arXiv:1010.2163 (2010). 


%\bibitem{WA17}
%M. Waegell, P. K. Aravind.
%\textit{The Minimum Complexity of Kochen-Specker Sets Does Not Scale with Dimension}. 
%Phys. Rev. A \textbf{95}, 050101 (2017). 

\bibitem{ACS14-2}
A. A. Abbott, C. S. Calude and K. Svozil.
\textit{A quantum random number generator certified by value indefiniteness}.
\href{https://doi.org/10.1017/S0960129512000692}{Mathematical Structures in Computer Science  Vol. 24, Issue 3 (Developments of the Concepts of Randomness, Statistic and Probability), e240303 (2014)}.

%Mathematical Structures in Computer Science 24 (Special Issue 3), e240303 (2014).

%\bibitem{RW04}
%R. Renner and S. Wolf. 
%\textit{Quantum pseudo-telepathy and the Kochen-Specker theorem}.
%In Proc. Int. Symp. Inf. Theory, pp. 322-329 (2004).
%
%\bibitem{RH14}
%R. Ramanathan and P. Horodecki.
%\textit{Necessary and sufficient conditions for state-independent measurement contextual scenarios}.
%Phys. Rev. Lett. \textbf{112}, 040404 (2014).


%\bibitem{CKB15}
%A. Cabello, M. Kleinmann and C. Budroni. 
%\textit{Necessary and sufficient condition for quantum state-independent contextuality}.
%Phys. Rev. Lett. \textbf{114}, 250402 (2015).



\bibitem{ACS14}
A. A. Abbott, C. S. Calude and K. Svozil.
\textit{Value-indefinite observables are almost everywhere}.
\href{https://doi.org/10.1103/PhysRevA.89.032109}{Phys. Rev. A 89, 032109 (2014)}.

\bibitem{PRL-rand}
R. Ramanathan, F. G. S. L. Brandao, K. Horodecki, M. Horodecki, P. Horodecki and H. Wojewodka.
%R. Ramanathan, F. G. S. L. Brand\~{a}o, K. Horodecki, M. Horodecki, P. Horodecki, and H. Wojew\'{o}dka.
\textit{Randomness Amplification under Minimal Fundamental Assumptions on the Devices}. \href{https://doi.org/10.1103/PhysRevLett.117.230501}{Phys. Rev. Lett. 117, 230501 (2016)}.

%\bibitem{RBHH+15}
%R. Ramanathan, F. G. S. L. Brand\~{a}o, K. Horodecki, M. Horodecki, P. Horodecki and H. Wojew\'{o}dka.
%\textit{Randomness amplification against no-signaling adversaries using two devices}.
%Phys. Rev. Lett. \textbf{117}, 230501 (2016). 
%arXiv:1504.06313 (2015).

\bibitem{BRGH+16}
F.G.S.L. Brand\~{a}o, R. Ramanathan, A. Grudka, K. Horodecki, M. Horodecki, P. Horodecki, T. Szarek, and H. Wojew\'{o}dka.
\textit{Robust Device-Independent Randomness Amplification with Few Devices}.
\href{https://doi.org/10.1038/ncomms11345}{Nat. Comm. 7, 11345 (2016)}.

\bibitem{WBGH+16}
H. Wojew\'{o}dka, F. G. S. L. Brand\~{a}o, A. Grudka, M. Horodecki, K. Horodecki, P. Horodecki, M. Paw\'{l}owski, R. Ramanathan. \textit{Amplifying the Randomness of Weak Sources Correlated with Devices}. \href{10.1109/TIT.2017.2738010}{IEEE Trans. on Inf. Theory. Vol. 63, No. 11, pp. 7592-7611 (2017)}.
%Vol. PP, Issue 99 (2017). 

%\bibitem{JT95}
%T. R. Jensen and B. Toft. 
%\textit{Graph coloring problems}. 
%New York: Wiley-Interscience, ISBN 0-471-02865-7 (1995).

%\bibitem{AM78}
%V. A. Aksionov and L. S. Mel'nikov.
%\textit{Essay on the theme: the three-color problem.}
%Combinatorics, Colloquia Mathematica Societatis J\'{a}nos Bolyai, \textbf{18}, 23 (1978). 

%\bibitem{AM80}
%V. A. Aksionov and L. S. Mel'nikov.
%\textit{Some counterexamples associated with the three-color problem.}
%Journal of Combinatorial Theory, Series B, \textbf{28}, 1 (1980). 

\bibitem{CDLP14}
A. Cabello, L. E. Danielsen, A. J. Lopez-Tarrida, J. R. Portillo. \textit{Basic exclusivity graphs in quantum correlations}. \href{https://doi.org/10.1103/PhysRevA.88.032104}{Phys. Rev. A 88, 032104 (2013)}.

\bibitem{C11}
A. Cabello. \textit{State-independent quantum contextuality and maximum nonlocality}.
\href{https://arxiv.org/abs/1112.5149}{arXiv:1112.5149 (2011)}.

\bibitem{Arends09}
F. Arends. 
\textit{A lower bound on the size of the smallest Kochen-Specker vector
system}. 
%Master’s thesis, Oxford University (2009), 
\href{http://www.cs.ox.ac.uk/people/joel.ouaknine/download/arends09.pdf}{Master’s thesis, Oxford University (2009)}.
%www.cs.ox.ac.uk/people/joel.ouaknine/download/arends09.pdf.

\bibitem{AOW11}
F. Arends, J. Ouaknine, and C. W. Wampler. 
\textit{On Searching for Small Kochen-Specker Vector Systems}. 
\href{https://doi.org/10.1007/978-3-642-25870-1_4}{In: Kolman P., Kratochvíl J. (eds) Graph-Theoretic Concepts in Computer Science. WG 2011. Lecture Notes in Computer Science, vol 6986. Springer, Berlin, Heidelberg (2011)}.

\bibitem{Clifton93}
R. K. Clifton. 
\textit{Getting Contextual and Nonlocal Elements-of-Reality the Easy Way}. 
\href{https://doi.org/10.1119/1.17239}{American Journal of Physics, 61: 443 (1993)}.

\bibitem{CEG96}
A. Cabello, J. Estebaranz and G. Garc\'{i}a-Alcaine.
\textit{Bell-Kochen-Specker Theorem: A Proof with 18 vectors}. 
\href{https://doi.org/10.1016/0375-9601(96)00134-X}{Phys. Lett. A Vol. 212, Issue 4, 183 (1996)}.

%\bibitem{Cohn03}
%P. M. Cohn. \textit{Basic algebra: groups, rings, and fields.} Springer (2003). 


\bibitem{Stanford}
C. Held. \textit{The Kochen-Specker Theorem}. \href{https://plato.stanford.edu/entries/kochen-specker/}{The Stanford Encyclopedia of Philosophy (Fall 2016 Edition), Edward N. Zalta (ed.), (2016)}. 

\bibitem{Cab08}
A. Cabello.
\textit{Experimentally Testable State-Independent Quantum Contextuality}.
\href{https://doi.org/10.1103/PhysRevLett.101.210401}{Phys. Rev. Lett. 101, 210401 (2008)}.

%\bibitem{GW02}
%V. Galliard and S. Wolf. \textit{Pseudo-telepathy, entanglement,
%and graph colorings}. Proc. ISIT 2002, p. 101
%(2002).

%\bibitem{GTW03}
%V. Galliard, A. Tapp and S. Wolf. \textit{The impossibility
%of pseudotelepathy without quantum entanglement}.
%Proc. ISIT 2003, p. 457 (2003).

\bibitem{CA96}
A. Cabello and G. Garc\'{i}a-Alcaine.
\textit{Bell-Kochen-Specker Theorem for any finite dimension $n \geq 3$}. 
\href{https://doi.org/10.1088/0305-4470/29/5/016}{J. Phys A: Math. and Gen., 29, 1025, (1996)}.  

%\bibitem{Steinberg93}
%R. Steinberg.
%\textit{The State of the Three Color Problem. Quo Vadis, Graph Theory? — A Source Book for Challenges and Directions.}
%Annals of Discrete Mathematics, \textbf{55}, 211 (1993).

\bibitem{Pitowsky}
I. Pitowsky.
\textit{Infinite and finite Gleason's theorems and the logic of indeterminacy}.
\href{https://doi.org/10.1063/1.532334}{Journal of Mathematical Physics 39, 218 (1998)}.

\bibitem{HP03}
E. Hrushovski and I. Pitowsky.
\textit{Generalizations of Kochen and Specker's theorem and the effectiveness of Gleason's theorem}.
\href{https://doi.org/10.1016/j.shpsb.2003.10.002}{Studies in History and Philosophy of Science Part B: Studies in History and Philosophy of Modern Physics Vol. 35, Issue 2, 177-194 (2004)}.
%arXiv: 0307139 (2003). 

\bibitem{Gleason}
A. M. Gleason. 
\textit{Measures on the Closed Subspaces of a Hilbert Space}.
\href{https://www.jstor.org/stable/24900629}{Journal of Mathematics and Mechanics Vol. 6, No. 6, 885 (1957)}.

\bibitem{YO12}
S. Yu and C. H. Oh.
\textit{State-Independent Proof of Kochen-Specker Theorem with 13 Rays}.
\href{https://doi.org/10.1103/PhysRevLett.108.030402}{Phys. Rev. Lett. 108, 030402 (2012)}.

%\bibitem{Ark12}
%A. Arkhipov.
%\textit{Extending and Characterizing Quantum Magic Games.} 
%arXiv:1209.3819 (2012).

%\bibitem{DR08}
%I. Dukanovic and F. Rendl.
%\textit{A semidefinite programming-based heuristic for graph coloring.}
%Disc. App. Math. \textbf{156}, 180 (2008).

\bibitem{Stairs}
A. Stairs. \textit{Quantum Logic, Realism, and Value Definiteness}. \href{https://www.jstor.org/stable/187557}{Philos. Sci. 50, 4, 578 (1983)}.

%\bibitem{KS2}
%S. Kochen and E.P. Specker, in Symposium on the Theory of
%Models, Proceedings of the 1963 International Symposium at
%Berkeley, edited by J.W. Addison et al. ~North-Holland, Amsterdam,
%(1965).
% p. 177; 

%reprinted in The Logico-Algebraic Approach
%to Quantum Mechanics. Volume I: Historical Evolution,
%edited by C.A. Hooker ~Reidel, Dordrecht, (1975), p. 263,
%and in E.P. Specker, Selecta ~Birkha¨user Verlag, Basel, 1990!,
%p. 209.

%\bibitem{HR83}
%P. Heywood and M.L.G. Redhead. \textit{Nonlocality and the Kochen-Specker paradox}.
%Found. of Phys. vol. \textbf{13}, 481 (1983).

\bibitem{CLRS01}
T. H. Cormen, C. E. Leiserson, R. L. Rivest, and C. Stein. \href{https://mitpress.mit.edu/books/introduction-algorithms-second-edition}{\textit{Introduction to Algorithms}, Second Edition. The MIT Press (2001)}.

%\bibitem{BBT05}
%G. Brassard, A. Broadbent and A. Tapp. \textit{Quantum Pseudo-Telepathy}. 
%Found. of Phys., vol. \textbf{35}, 11, 1877 (2005).

\bibitem{Lovasz87}
L. Lovasz, M. Saks, A. Schrijver. \textit{Orthogonal representation and connectivity
of graphs}. \href{https://doi.org/10.1016/0024-3795(89)90475-8}{Linear Algebra and its applications, 4, 114-115, 439 (1987)}. 

\bibitem{R17}
R. Ramanathan, M. Horodecki, S. Pironio, K. Horodecki, and P. Horodecki. \textit{Generic randomness amplification schemes using Hardy paradoxes}. \href{https://arxiv.org/abs/1810.11648}{arXiv: 1810.11648 (2018)}.

%\bibitem{PRL-rand}
%R. Ramanathan, F. G. S. L. Brand\~{a}o, K. Horodecki, M. Horodecki, P. Horodecki, and H. Wojew\'{o}dka.
%\textit{Randomness Amplification under Minimal Fundamental Assumptions on the Devices}. \href{https://doi.org/10.1103/PhysRevLett.117.230501}{Phys. Rev. Lett. 117, 230501 (2016)}.

%\bibitem{SV84}
%M. Santha and U. V. Vazirani. Generating Quasi-Random Sequences from Slightly-Random Sources. Proceedings of the 25th IEEE Symposium on Foundations of Computer Science (FOCS'84), 434 (1984).

\bibitem{CRST06}
M. Chudnovsky, N. Robertson, P. Seymour and R. Thomas.  \textit{The strong perfect graph theorem}. \href{https://doi.org/10.4007/annals.2006.164.51}{Annals of Mathematics, 164 (1): 51 (2006)}.

%\bibitem{ACY16}
%S. Abramsky, C. M. Constantin and S. Ying. 
%\textit{Hardy is (almost) everywhere: nonlocality without inequalities for almost all entangled multipartite states}.
%Information and Computation vol. \textbf{250}, 3 (2016).

%\bibitem{CMNSW07}
%P. J. Cameron, A. Montanaro, M. W. Newman, S. Severini and A. Winter. \textit{On the quantum chromatic number of a graph}. Electronic Journal of Combinatorics \textbf{14}(1)  (2007). 

%\bibitem{MH17}
%R. Ramanathan et al. \textit{in preparation}. 

%\bibitem{Cab98}
%A. Cabello. 
%\textit{Ladder proof of nonlocality without inequalities and without probabilities}.
%Phys. Rev. A \textbf{58}, 1687 (1998).

%\bibitem{BBMH97}
%D. Boschi, S. Branca, F. De Martini, and L. Hardy.
%\textit{Ladder Proof of Nonlocality without Inequalities: Theoretical and Experimental Results}.
%Phys. Rev. Lett. \textbf{79}, 2755 (1997).

\bibitem{P96}
R. Peeters.
\textit{Orthogonal representations over finite fields and the chromatic number of graphs}. \href{https://doi.org/10.1007/BF01261326}{Combinatorica 16, 3, 417 (1996)}. 

\bibitem{CG95}
A Cabello and G Garcia-Alcaine. \textit{A hidden-variables versus quantum mechanics experiment}. \href{https://doi.org/10.1088/0305-4470/28/13/016}{Journal of Phys. A: Math. and General, 28, No. 13 (1995)}.

\bibitem{BBCP09}
P. Badzi\c{a}g, I. Bengtsson, A. Cabello, and I. Pitowsky. \textit{Universality of State-Independent Violation of Correlation Inequalities for Noncontextual Theories}. \href{https://doi.org/10.1103/PhysRevLett.103.050401}{Phys. Rev. Lett. 103, 050401 (2009)}.

\bibitem{APSS18}
A. Cabello, J. R. Portillo, A. Sol\'{i}s, and K. Svozil. \textit{Minimal true-implies-false and true-implies-true sets of propositions in noncontextual hidden-variable theories}.
%https://doi.org/ https://journals.aps.org/pra/abstract/10.1103/PhysRevA.98.012106
\href{https://doi.org/10.1103/PhysRevA.98.012106}{Phys. Rev. A 98, 012106 (2018)}.

%\bibitem{Cab01}
%A. Cabello. 
%\textit{Bell's Theorem without Inequalities and without Probabilities for Two Observers}.
%Phys. Rev. Lett. \textbf{86}, 1911 (2001).

%\bibitem{SS12}
%G. Scarpa and S. Severini
%\textit{Kochen-Specker Sets and the Rank-1 Quantum Chromatic Number}
%IEEE Trans. on Inf. Theory, vol. \textbf{58}, no. 4, (2012).

%\bibitem{AGA+12}
%L. Aolita, R. Gallego, A. Ac\'{i}n, A. Chiuri, G. Vallone, P. Mataloni, and A. Cabello
%\textit{Fully nonlocal quantum correlations}.
%Phys. Rev. A \textbf{85}, 032107 (2012).


\end{thebibliography}






%
%\textit{\textbf{Proposition 5.}
%	Every graph $G$ with $\chi(G) > \omega(G) \geq 3$ contains a coloring gadget $K$ as a subgraph with $\chi(K) = \omega(K) = \omega(G)$. In particular, every Kochen-Specker graph $G$ in dimension $d$ contains a coloring gadget $K$ as a subgraph with $\chi(K) = \omega(K) = d$. }
%
%\begin{proof}
%	Consider the graph $G$ with $\chi(G) > \omega(G) \geq 3$. 
%	Let $Q \subset V(G)$ be a maximum clique of the graph $G$ (of size $\omega(G)$) and let $\tilde{G} \defeq G \setminus Q$, i.e., $\tilde{G}$ denotes the induced subgraph of $G$ obtained by deleting the maximum clique $Q$. We investigate separately the two possibilities: 
%	\ben 
%	&&(1) \; \chi(\tilde{G}) > \omega(G), \; \; \text{and} \nonumber \\
%	&&(2) \; \chi(\tilde{G}) \leq \omega(G). 
%	\een
%	Here, note that $\chi(\tilde{G})$ denotes the chromatic number of $\tilde{G}$ while $\omega(G)$ denotes the size of the maximum clique in $G$.\\
%	
%	\underline{Case (1): $\chi(\tilde{G}) > \omega(G) \geq 3$.} 
%	
%	In this case, let us consider an edge-critical subgraph $\tilde{H}$ of $\tilde{G}$, with $\chi(\tilde{H}) = \omega(G) +1$. Recall that a graph is said to be edge-critical if removing any edge results in a decrease in its chromatic number. A k-edge-critical subgraph of a graph $\Gamma$ is an edge-critical subgraph $\Gamma_0 \subseteq \Gamma$, such that $\chi(\Gamma_0) = k$. Each graph $\Gamma$ contains at least one $k$-edge-critical subgraph (also at least one $k$-vertex-critical subgraph) for $1 \leq k \leq \chi(\Gamma)$ \cite{JT95}. 
%	
%	\begin{claim}
%		\label{claim:non-triangle} 
%		$\exists$ three distinct vertices $u, v, w \in \tilde{H}$ such that $(u,w) \in E(\tilde{H})$, $(v,w) \in E(\tilde{H})$ and $(u,v) \notin E(G)$.
%	\end{claim}
%	%\textbf{Proof of Claim:}
%	\begin{proof}
%		Firstly observe that every edge-critical graph is connected \cite{JT95}, if not since the chromatic number of a disconnected graph with two disconnected components $\Gamma = \Gamma_1 \cup \Gamma_2$ is given as $\chi(\Gamma) = \max \{\chi(\Gamma_1), \chi(\Gamma_2) \}$, one may delete the component that does not yield the maximum without decreasing the chromatic number.   
%		Since $\tilde{H}$ is an edge-critical graph, it is connected. 
%		%Therefore, there must exist three vertices $u, v, w \in V(\tilde{H})$ with $(u, v) \in E(\tilde{H})$ and $(v,w) \in E(\tilde{H})$. 
%		
%		Let us now suppose that the claim does not hold, i.e., for all triples $u, v, w \in \tilde{H}$ obeying $(u,w) \in E(\tilde{H})$ and $(v,w) \in E(\tilde{H})$, we have that $(u,v) \in E(G)$. Let us show that this will result in a contradiction. Begin with a vertex $v_1 \in \tilde{H}$, since $\tilde{H}$ is connected there must exist vertex $v_2 \in \tilde{H}$ such that there is an edge $(v_1, v_2) \in E(\tilde{H})$. Similarly, since $\tilde{H}$ is connected, there is a third distinct vertex $v_3 \in \tilde{H}$ such that $(v_2, v_3) \in E(\tilde{H})$ (or alternatively $(v_1, v_3) \in E(\tilde{H})$). Now by the supposition that the claim does not hold, we know that $(v_1, v_3) \in E(G)$ (or alternatively $(v_2, v_3) \in E(\tilde{H})$), so that since $E(\tilde{H}) \subseteq E(G)$ we have that $v_1, v_2, v_3$ form a triangle in $G$. Now, we continue to observe that since $\tilde{H}$ is connected, there must exist a fourth distinct vertex $v_4 \in \tilde{H}$ with $(v_3, v_4) \in E(G)$ (or alternatively $(v_1, v_4) \in E(\tilde{H})$ or $(v_2, v_4) \in E(\tilde{H})$) so that the vertices $v_1, v_2, v_3, v_4$ form a clique of size four in $G$ and so on. This procedure gives that the induced subgraph of $G$ on the vertex set $V(\tilde{H})$ must be a clique.  
%		
%		But by assumption we know that $\chi(\tilde{H}) = \omega(G) + 1$, so that $|V(\tilde{H})| \geq \omega(G) + 1$. Therefore, the procedure outlined above must give a clique of size $\geq \omega(G) + 1$ in $G$ which is a contradiction to $\omega(G)$ being the size of the largest clique in $G$. 
%		% there do not exist three vertices $u, v, w \in V(\tilde{H})$ with $(u,v) \in E(\tilde{H})$, $(v,w) \in E(\tilde{H})$ and $(u,w) \notin E(G)$. Since $\tilde{H}$ is a subgraph of $G$, $(u,v) \in E(\tilde{H})$ implies $(u,v) \in E(G)$ and similarly $(v,w) \in E(G)$. 
%		%Since we have supposed that the claim does not hold, this implies that any three distinct vertices $u, v, w \in V(\tilde{H})$ form a triangle, so that $\tilde{H}$ is a complete graph. Now, recall that by assumption $\chi(\tilde{H}) = \omega(G) + 1$. 
%		%the contrary case is that $(u,v), (v,w), (u,w) \in E(\tilde{H}^{cr}) \; \; \forall u,v,w \in V(\tilde{H}^{cr})$. 
%		%This implies that $\tilde{H}$ forms a clique of size $\omega(G) + 1$, a contradiction to $\tilde{H} \leq G$. Also, for precisely the same reason, $(u,w) \notin E(G)$. This shows the claim.
%	\end{proof}
%	%\qed
%	From Claim \ref{claim:non-triangle}, we have three vertices $u, v, w$ in $\tilde{H}$ such that $(u,w) \in E(\tilde{H})$, $(v,w) \in E(\tilde{H})$ and $(u,v) \notin E(G)$. Consider 
%	%For the three vertices $u,v,w$ as in the Claim, 
%	the graph $\tilde{H}'$ formed by deleting the edge $(u,w)$ from $\tilde{H}$, i.e., $V(\tilde{H}') = V(\tilde{H})$ and $E(\tilde{H}') = E(\tilde{H}) \setminus (u,w)$. Since $\tilde{H}$ is edge-critical with $\chi(\tilde{H}) = \omega(G) + 1$, we have that $\chi(\tilde{H}') = \omega(G)$. Moreover, by construction, we have that in any optimal coloring $c: V(\tilde{H}') \rightarrow [\chi(\tilde{H}')]$, there is
%	\begin{equation}
%	c(u) = c(w) \; \; (\neq c(v)).
%	% \; \; \text{with} (u,v) \notin E(\tilde{H}').
%	\end{equation}   
%	If this were not the case, i.e., if $c(u) \neq c(w)$ then we could insert the edge $(u,w)$ and obtain a coloring of $\tilde{H}$ with $\omega(G)$ colors which would be a contradiction to $\chi(\tilde{H}) = \omega(G) + 1$. 
%	%This is because any coloring $c'$ of $\tilde{H}'$ with $c'(u) \neq c'(v)$ is also a coloring of $\tilde{H}$ which has $\chi(\tilde{H}) > \chi(\tilde{H}')$.
%	
%	We now consider the subgraph $K$ of $G$ with vertex set $V(K) = V(Q) \cup V(\tilde{H}')$ and edge set $E(K) = E(Q) \cup E(\tilde{H}') \cup S$ where $S = \{(v_1,v_2) \in E(G) \vert v_1 \in V(Q), v_2 \in V(\tilde{H}')\}$. We have that either $\chi(K) = \omega(G)$, in which case $K$ is the required coloring gadget subgraph of $G$, since $\chi(K) = \omega(K) = \omega(G)$ and we have the distinguished vertices $u,v$ with $(u,v) \notin E(K)$ with the requisite coloring property. Note that here the inclusion of the maximum clique $Q \subset K$ ensures that $\omega(K) = \omega(G)$. Alternatively, in case $\chi(K) > \omega(G)$, we further construct $\hat{K}$ by deleting edges $(v_!,v_2) \in S$ from $K$ until $\chi(\hat{K}) = \omega(\hat{K}) = \omega(G)$, and see that $\hat{K}$ is a subgraph of $G$ that is a coloring gadget, again with the same distinguished vertices $u \nsim v$. \\
%	
%	\underline{Case (2): $\chi(\tilde{G}) \leq \omega(G)$.} 
%	%Here, we have $\chi(Q) = \omega(G)$ and $\chi(G) > \omega(G) $. Moreover, 
%	Recall that $Q$ is a maximum clique of the graph and $\tilde{G}$ is the graph obtained from $G$ by deleting $Q$. We have that $V(G) = V(Q) \cup V(\tilde{G})$ and $E(G) = E(Q) \cup E(\tilde{G}) \cup \tilde{S}$, where $\tilde{S} = \{(v_1,v_2) \in E(G)| v_1 \in V(Q), v_2 \in V(\tilde{G})\}$, and we also know that $\chi(G) > \omega(G)$. Since $\chi(\tilde{G}) \leq \omega(G)$ and clearly $\chi(Q) = \omega(G)$, we can construct an edge-critical graph $\tilde{H}$ of $G$ with $\chi(\tilde{H}) = \omega(G) + 1$ by deleting edges belonging to $\tilde{S}$ from the graph $G$. Now, similar to Case (1), we may identify three vertices in the edge-critical graph $\tilde{H}$ with the following property. 
%	% until we obtain an edge-critical graph $\tilde{H}$ with $\chi(\tilde{H}) = \omega(G) +1$ after which deleting any more such edges results in a sub-graph with chromatic number $\omega(G)$.  \\
%	\begin{claim}
%		\label{claim:not-triangle-2}
%		$\exists$ three distinct vertices $u, v, w \in V(\tilde{H})$ with $v, w \in V(Q)$ and $u \in V(\tilde{G})$ such that $(u,w) \in E(\tilde{H})$, and $(u,v) \notin E(G)$. 
%	\end{claim} 
%	\begin{proof} 
%		Let us suppose that the claim does not hold, i.e., for all triples $v, w \in V(Q), u \in V(\tilde{G})$ obeying $(u,w) \in E(\tilde{H})$ we have that $(u,v) \in E(G)$. Since all the vertices in the maximum clique $Q$ are connected, this gives that the vertex $u$ is adjacent in $G$ to every vertex in $Q$. In other words, we have a clique $Q \cup u$ of size $\omega(G) +1$ in $G$, which is a contradiction.  
%		%Suppose not. In the contrary scenario, $\forall w \in V(\tilde{G})$ such that $(v,w) \in E(\tilde{H})$ we would have $(u,w) \in E(\tilde{H}) \; \; \forall u \in Q$. This would imply that $\omega(\tilde{H}) > \omega(G)$ a contradiction to $\tilde{H}$ being a subgraph of $G$. By the same argument, we also have that $(u,w) \notin E(G)$. 
%	\end{proof}
%	By Claim \ref{claim:not-triangle-2}, we have three vertices $v, w \in V(Q)$ and $u \in V(\tilde{G})$ such that $(u,w) \in E(\tilde{H})$, and $(u,v) \notin E(G)$. Form the subgraph $K$ by deleting the edge $(u,w)$ from $\tilde{H}$, i.e., $V(K) = V(\tilde{H})$ and $E(K) = E(\tilde{H}) \setminus (u,w)$. Since $\tilde{H}$ is an edge-critical graph with $\chi(\tilde{H}) = \omega(G) +1$, we have that $\chi(K) = \omega(G) = \omega(K)$, where the last equality is due to the fact that the maximum clique $Q$ belongs to $K$. Furthermore, in any optimal coloring $c: V(K) \rightarrow [\chi(K)]$, we have $c(u) = c(w) (\neq c(v))$. This gives that $K$ is the requisite coloring gadget subgraph of $G$ with the distinguished vertices $u \nsim v$ in $G$ such that in any optimal coloring of $K$ with $\chi(K) = \omega(K) = \omega(G)$ colors, $u$ and $v$ are assigned distinct colors. 
%	%Consider the three vertices $u,v,w$ as in the Claim above and form a graph $H$ by deleting the edge $(v,w)$, i.e., $H = \tilde{H}^{cr} \setminus (v,w)$. By the critical property, we have that $\chi(H) = \omega(G)$; furthermore $\omega(H) = \omega(G)$ since $Q \subset H$ and in any optimal coloring $c: V(H) \rightarrow [\chi(H)]$ there is $c(u) \neq c(v) = c(w)$ with $(u,w) \notin E(H)$. This gives that $H$ is the required quasi-edge subgraph of $G$.   
%	
%	Finally, by the results in \cite{RH14, CKB15}, we know that $\chi(G) > \omega(G)$ is a necessary condition for an orthogonality graph to represent a set of projectors with the Kochen-Specker (KS) property, i.e., in \cite{RH14} (see also \cite{CKB15}) some of us showed the following. 
%	\begin{lemma}[\cite{RH14, CKB15}]
%		\label{lem:KS-chi}
%		A necessary condition for an orthogonality graph $G$ to represent a set of Kochen-Specker set of rank-one projectors $\{P_{v_j}\}$ with $|v_j \rangle \in \mathbb{C}^d$ is that $\chi_f(G) > d$.
%	\end{lemma}
%	For any graph $G$, we know that $\chi_f(G) \leq \chi(G)$ so that by Lemma \ref{lem:KS-chi} we have that the orthogonality graph of any Kochen-Specker set must satisfy $\chi(G) > d$. Also, $\omega(G) = d$ for the Kochen-Specker graph since the size of a maximum clique is equal to the size of the basis in $\mathbb{C}^d$, so that $\chi(G) > \omega(G)$ for the KS graphs. By Lemma \ref{lem:quasi-edge-subgr}, we see that every Kochen-Specker graph contains a coloring gadget as a subgraph, completing the proof.
%\end{proof}
%
%\textit{\textbf{Observation 3.} 
%	Let $G$ be a graph with $\chi(G) > \omega(G) \geq 3$. 
%	If there exists a vertex-critical induced subgraph $\tilde{H}$ of $G$ with $\chi(\tilde{H}) = \omega(G)+1$ and $\delta(\tilde{H}) = \omega(G)$, then $G$ contains a coloring gadget $K$ as an \textit{induced} subgraph with $\chi(K) = \omega(G)$.}
%\begin{proof}
%	Consider a vertex-critical induced subgraph $\tilde{H}$ of $G$ with $\chi(\tilde{H}) = \omega(G) +1$. Vertex criticality ensures that for each vertex $u \in V(\tilde{H})$, there is an optimal coloring $c: V(\tilde{H}) \rightarrow [\omega(G) +1]$ with $u$ being a singleton color class \cite{JT95}. This also gives that the minimum degree of $\tilde{H}$ is at least $\omega(G)$, i.e., $\delta(\tilde{H}) \geq \omega(G)$. Now by assumption $\delta(\tilde{H}) = \omega(G)$, let $v \in V(\tilde{H})$ denote the vertex of this minimum degree. In any optimal coloring of $\tilde{H}$ with $v$ belonging to a singleton class, each vertex $w$ that is adjacent to $v$ (i.e. $w \in N(v)$ with the neighborhood $N(v) := \{w | (v,w) \in E(\tilde{H})\}$) belongs to a distinct color class. Note that $|N(v)| = \omega(G)$ so that $\exists$ distinct vertices $w, w' \in N(v)$ with $(w,w') \notin E(\tilde{H})$. 
%	
%	Form the induced subgraph $K$ of $\tilde{H}$ by deleting vertex $v$, i.e., $K$ is the induced subgraph of $\tilde{H}$ on the vertex set $V(K) = V(\tilde{H}) \setminus v$. Note that since $\tilde{H}$ is an induced subgraph of $G$, $K$ is also an induced subgraph of $G$. We have that $\chi(K) = \omega(G)$ by vertex criticality of $\tilde{H}$. Also, in any optimal coloring $c : V(K) \rightarrow [\omega(G)]$, we have that each vertex $w$ that belonged to the set $N(v)$ in $\tilde{H}$ must be assigned a distinct color and by the argument above there must exist two distinct non-adjacent vertices $w \nsim w'$ in this set. Therefore, $K$ is the requisite coloring gadget that is an induced subgraph of $G$.   
%	% so that in any coloring of $K \defeq G \setminus v$, all the  
%	% so that $\exists$ distinct vertices $w, w' \in N(v)$ with $(w,w') \notin E(\tilde{H})$.    
%\end{proof}

%\begin{figure}[t]
%	\centerline{\includegraphics[scale=0.38]{fig-45-3.pdf}}
%	\caption{A finite $101$-gadget for which a representation (in $\mathbb{C}^3$) exists in which the two distinguished vertices $u_1,u_{42}$ are represented by the same vector (note that the orthogonality graph for this representation has the two vertices $u_1$ and $u_{42}$ identified with each other). The faithful orthogonal representation of this graph is given in the Appendix. An interesting open question is to find the minimal gadget with this property.}
%	\label{fig:gadg-id-vec}
%\end{figure}	

%Finally, we present in Fig.\ref{fig:gadg-id-vec} a finite ($43$ vertex) $101$-gadget for which a representation in $\mathbb{C}^3$ exists in which the two distinguished vertices are represented by the same vector. Note that in this case, the two vertices $u_{1}$ and $u_{42}$ are identified with each other. Furthermore, we also identify the vertices $u_2$ with $u_{39}$ and $u_3$ with $u_{40}$ in this representation to make it faithful. The set of vectors representing this graph are given as follows:
%\begin{eqnarray}
%&&|u_1 \rangle = (1,-1,0); \; \; |u_2 \rangle = (1,1,1); \; \; |u_3 \rangle = (1,1,0); \nonumber \\
%&&|u_4 \rangle = (1,1,b); \; \; |u_5 \rangle = (-2,1,1); \; \; |u_6 \rangle = (1,-1,3); \nonumber \\
%&&|u_7 \rangle = (3,-3,-2); \; \; |u_8 \rangle = (2,0,3); \; \; |u_9 \rangle = (-3,0,2); \nonumber \\
%&&|u_{10} \rangle = (-2,2,-3); \; \; |u_{11} \rangle = (3,-3,-4); \; \; |u_{12} \rangle = (4,0,3); \nonumber \\
%&&|u_{13} \rangle = (-3,0,4); \; \; |u_{14} \rangle = (-4,4,-3); \; \; |u_{15} \rangle = (3,-3,-8); \nonumber \\
%&&|u_{16} \rangle = (8,0,3); \; \; |u_{17} \rangle = (-3,0,8); \; \; |u_{18} \rangle = (-8,4+\sqrt{7},-3); \nonumber \\
%&&|u_{19} \rangle = (0,1,-1); \; \; |u_{20} \rangle = (0,1,0); \; \; |u_{21} \rangle = (0,-3+8b,-16-3b); \nonumber \\
%&&|u_{22} \rangle = (1,0,0); \; \; |u_{23} \rangle = (1,0,-1); \;\; |u_{24} \rangle = (2-\sqrt{2},0,1); \nonumber \\
%&&|u_{25} \rangle = (1,-2,1); \; \; |u_{26} \rangle = (0,1,2); \; \; |u_{27} \rangle = (0,2,-1); \nonumber \\
%&&|u_{28} \rangle = (1,-1,-2); \; \; |u_{29} \rangle = (1,-1,1); \; \; |u_{30} \rangle = (0,1,1); \nonumber \\
%&&|u_{31} \rangle = (0,1,-1); \;\; |u_{32} \rangle = (-1,1,1); \; \; |u_{33} \rangle = (-1,1,-2); \nonumber \\
%&&|u_{34} \rangle = (0,2,1); \; \; |u_{35} \rangle = (0,1,-2); \; \; |u_{36} \rangle = (2,-2,-1); \nonumber \\
%&&|u_{37} \rangle = (1,-1,4); \; \;  |u_{38} \rangle = (-2-\sqrt{2},6-\sqrt{2},2); \; \; |u_{39} \rangle = |u_2 \rangle; \nonumber \\
%&&|u_{40} \rangle = |u_3 \rangle; \; \; |u_{41} \rangle = (1,1,-2+\sqrt{2}); \; \; |u_{42} \rangle = |u_1 \rangle; |u_{43} \rangle = (0,0,1); \nonumber \\
%\end{eqnarray}
%with $b = \frac{-4+\sqrt{7}}{3}$. It is an open question whether this set of $40$ vectors is the minimal set with this property. 


%\subsection{Boxes}
%For a set of vectors which admits a \{0,1\}-coloring, the convex hull of such \{0,1\} colorings forms the set of non-contextual boxes $B_{nc}$. In other words, the non-contextual boxes are obtained as convex mixtures of deterministic non-contextual \{0,1\} assignments to the vectors. In terms of the corresponding orthogonality graph $G$, the set of non-contextual boxes is closely related to the well-known stable set of the graph STAB($G$) \cite{CSW14, CSW10} which is the convex hull of the indicator vectors corresponding to the independent sets of the graph. This relation can be seen as follows: every non-contextual deterministic box may be written as a vector of length $|V(G)|$ with \{0,1\} entries. Then all the vertices that receive value $1$ are mutually non-adjacent and thus correspond to an independent set in the graph. Therefore, every non-contextual box lies within STAB(G). However, while all the non-contextual boxes are normalized (each maximum clique receives a $1$ by the Kochen-Specker rules), an independent set does not need to contain a vector belonging to every maximum clique, so STAB(G) also contains non-normalized boxes. 
%
%Analogously, the set of all valid probability distributions (with real entries in $[0,1]$ and the sum of entries in each maximum clique being equal to $1$) forms the set of consistent boxes $B_c(G)$. Formally, we say
%\begin{dfn} 
%A consistent box $B_c(G)$ associated to graph $G$ is a set of probability assignments $P_c: V(G) \rightarrow [0,1]$ such that for any clique $Q \subset V(G)$ we have $\sum_{v \in Q} P_c(v) \leq 1$, with equality required to be achieved on all maximum cliques. 
%\end{dfn}
%It is then apparent that a non-contextual deterministic box is nothing but a consistent box where all the entries are in \{0,1\}. The convex hull of all consistent boxes corresponding to a graph has also been considered previously, and is denoted by QSTAB(G), again with the caveat that in QSTAB(G) one does not impose the requirement that the sum of entries in every maximum clique be equal to $1$. 
%
%The quantum boxes $B_q(G)$ form a convex set sandwiched between the set of non-contextual and consistent boxes. A quantum box $B_q(G)$ is obtained from an orthogonal representation of the graph $G$ and a state $| \psi \rangle \in \mathbb{C}^d$, the probabilities $P_q(v)$ defining the quantum box are given by
%%\be 
%$P_q(v) = \vert \langle v | \psi \rangle \vert^2$.
%%\ee
%For a complete basis set $b$ of vectors $|v \rangle \in S$, $\sum_{v \in b} P_q(v) = 1$ which corresponds to the normalization condition for the quantum boxes. The convex hull of quantum boxes for a given graph forms the set known as $\text{TH}(G)$ \cite{CSW14, CSW10}, again up to the normalization factor.

%\tred{Is the following necessary? Can't we simplify it?}
%
%\tred{We also consider the notion of consistent boxes, assignments of probabilities to the vertices of the graph $G$ in a consistent manner, i.e., each vertex is assigned a single probability value so that the probability of the corresponding outcome is independent of the context (the clique in the graph) and such that the sum of the probability assignments in a clique does not exceed unity.  
%%We will also require that the sum of the probability assignments in a maximum clique is equal to unity, which is the normalization condition. 
%Such a probability assignment to the vertices of the graph $G$ will give rise to a \textit{consistent box} that we denote $B_c(G)$. 
%\begin{dfn} 
%A consistent box $B_c(G)$ associated to graph $G$ is a set of probability assignments $P_c: V(G) \rightarrow [0,1]$ such that for any clique $Q \subset V(G)$ we have $\sum_{v \in Q} P_c(v) \leq 1$.
%% with equality required to be achieved on all maximum cliques. 
%\end{dfn}
%We will also consider the subset of consistent boxes that are normalized $\bar{B}_c(G)$, i.e., boxes with the property that the sum of the probability assignments in every maximum clique is equal to unity, i.e., $\sum_{v \in Q} P_c(v) = 1$ for $Q$ being a maximum clique, which is the normalization condition. 
%The non-contextual (classical) deterministic boxes $B_{\text{nc}}^{\text{det}}(G)$ are consistent boxes which have the probability assignments strictly from $\{0, 1\}$. Mixtures of non-contextual deterministic boxes give rise to general non-contextual boxes $B_{\text{nc}}(G)$. Formally, we have
%%The non-contextual (classical) deterministic boxes 
%\begin{dfn}
%A non-contextual deterministic box $B_{\text{nc}}^{\text{det}}(G)$ is a consistent box with the additional restriction on the probability assignment: $P_{\text{nc}}^{\text{det}}(u) \in \{0,1\} \; \; \forall u \in V(G)$. A non-contextual box $B_{\text{nc}}(G)$ is a convex combination of non-contextual deterministic boxes $B_{\text{nc}}^{\text{det}}(G)$, i.e., 
%\begin{equation}
%B_{\text{nc}}^{\text{det}}(G) = \sum_{i} p_i B_{\text{nc}}^{\text{det}, i}(G),
%\end{equation}  
%for $p_i \geq 0$ with $\sum_{i} p_i = 1$. 
%\end{dfn}
%The convex hull of non-contextual deterministic boxes is known in the graph theory literature as $\text{STAB}(G)$ (up to the normalization). This is due to the fact that every non-contextual deterministic box can be written out as a vector with the $1$'s in the vector indicating the vertices corresponding to a stable set in the graph \cite{CSW14, CSW10}. 
%We will also consider the subset of non-contextual deterministic boxes that are normalized $\bar{B}_{\text{nc}}^{\text{det}}(G)$, i.e., boxes for which the assignment $P_{\text{nc}}^{\text{det}}(u) \in \{0,1\}$ to the vertices is such that every maximum clique contains a vertex that is assigned value $1$. Note that such a normalized box $\bar{B}_{\text{nc}}^{\text{det}}(G)$ does not always exist for a graph, in particular this will be the case for the Kochen-Specker graphs, i.e., orthogonality graphs of Kochen-Specker sets of vectors. This is the restatement of the well-known that Kochen-Specker sets do not admit a $101$-coloring, on which more below. The corresponding mixtures of normalized non-contextual deterministic boxes are denoted $\bar{B}_{\text{nc}}(G)$.   
%The quantum boxes $B_q(G)$ form a convex set sandwiched between the set of non-contextual and consistent boxes. A quantum box $B_q(G)$ is obtained from an orthogonal representation of the graph $G$ and a state $| \psi \rangle \in \mathbb{C}^d$, the probabilities $P_q(v)$ defining the quantum box $B_q(G)$ are given by
%%\be 
%$P_q(v) = \vert \langle v | \psi \rangle \vert^2$.
%%\ee
%For a basis set $b$ of vectors $|v \rangle \in S$, $\sum_{v \in b} P_q(v) = 1$ which corresponds to the normalization condition for the quantum boxes. The convex hull of quantum boxes forms the set known as $\text{TH}(G)$ \cite{CSW14, CSW10}. 
%%by considering a representation of the graph using 
%A generalization of the KS sets due to Renner and Wolf \cite{RW04} called \textit{weak KS sets} also serves to prove the Kochen-Specker theorem. A weak KS set \cite{RW04} is a set of (unit) vectors $S \subset \mathbb{C}^d$ such that for any function $f: S \rightarrow \{0,1\}$ satisfying $\sum_{|u \rangle \in b} f(|v \rangle) = 1$ for all orthogonal bases $b \subset S$, there exist two orthogonal unit vectors $|u_1 \rangle, |u_2 \rangle \in S$ such that $f(|u_1 \rangle) = f( |u_2 \rangle) = 1$. It can be readily seen that every KS set is a weak KS set and in \cite{RW04} it was shown that every weak KS set can be completed to a KS set by adding $O(|S|^2)$ vectors so that both types of sets serve to prove the KS theorem in any specific dimension $d \geq 3$.   
%}

\end{document}