\documentclass[12pt]{article}
%\documentstyle[epsf,12pt]{arti\end{equation}
\usepackage{latexsym}
\usepackage{amsmath}
\usepackage{amssymb}
\usepackage{epsfig,graphics}
\usepackage{graphicx}
\usepackage{booktabs}
\usepackage{multirow}
\usepackage{caption}
\usepackage{subcaption}

\newcommand{\be}{\begin{equation}}
\newcommand{\ee}{\end{equation}}
\newcommand{\beq}{\begin{eqnarray}}
\newcommand{\eeq}{\end{eqnarray}}
\newcommand{\otoprule}{\midrule[\heavyrulewidth]}

%\newcommand{\singlespace}{
%  \renewcommand{\baselinestretch}{1}\large\normalsize}
%\newcommand{\doublespace}{
%  \renewcommand{\baselinestretch}{1.6}\large\normalsize}

 
\topmargin=-.35in 
\textheight=8.60in
\oddsidemargin=0.0in
\textwidth=6.6in

%\newlength{\figsize}
%\figsize = 0.7\textwidth

\begin{document}

\begin{titlepage}


%\vspace*{0.3in}

\begin{center}
  {\large\bf SU(N) gauge theories in 3+1 dimensions: \\
  glueball spectrum, string tensions and topology} \\
  \vspace*{.35in}
{Andreas Athenodorou$^{a,b}$ and Michael Teper$^{c,d}$\\
  \vspace*{.35in}
  $^{a}$ Dipartimento di Fisica, Universit\`a di Pisa and INFN, Sezione di Pisa, \\
  Largo Pontecorvo 3, 56127 Pisa, Italy. \\
\vspace*{.10in}
$^{b}$ Computation-based Science and Technology Research Center, The Cyprus Institute,\\
20 Kavafi Str., Nicosia 2121, Cyprus \\
\vspace*{.10in}
$^{c}$Rudolf Peierls Centre for Theoretical Physics, University of Oxford,\\
Parks Road, Oxford OX1 3PU, UK\\
\vspace*{.10in}
$^{d}$All Souls College, University of Oxford,\\
High Street, Oxford OX1 4AL, UK}
%
\end{center}



\vspace*{0.2in}


\begin{center}
{\bf Abstract}
\end{center}

We calculate the low-lying glueball spectrum, several string tensions and some properties of
topology and the running coupling for $SU(N)$ lattice gauge theories in $3+1$ dimensions. We
do so for $2 \leq N \leq 12$, using lattice simulations with the Wilson plaquette action,
and for glueball states in all the representations of the cubic rotation group, for
both values of parity and charge conjugation. We extrapolate these results to
the continuum limit of each theory and then to $N=\infty$. For a number of these states
we are able to identify their continuum spins with very little ambiguity.
We calculate the fundamental string tension and $k=2$ string tension and investigate
the $N$ dependence of the ratio. Using the string tension as the scale, we calculate
the running of a lattice coupling and confirm that $g^2(a)\propto 1/N$  for
constant physics as $N\to\infty$. We fit our calculated values of $a\surd\sigma$
with the 3-loop $\beta$-function, and extract a value for $\Lambda_{\overline{MS}}$,
in units of the string tension, for all our values of $N$, including $SU(3)$. We use these
fits to provide analytic formulae for estimating the string tension at a given lattice coupling.
We calculate the topological charge $Q$ for $N\leq 6$ where it fluctuates
sufficiently for a plausible estimate of the continuum topological susceptibility.
We also calculate the renormalisation of the lattice topological charge, $Z_Q(\beta)$, for
all our $SU(N)$ gauge theories, using a standard definition of the charge, and we provide
interpolating formulae, which may be useful in estimating the renormalisation of
the lattice $\theta$ parameter. We provide quantitative results for how
the topological charge `freezes' with decreasing lattice spacing and with increasing $N$.
Although we are able to show that within our typical errors our glueball and string tension
results are insensitive to the freezing of $Q$ at larger $N$ and $\beta$, we choose to
perform our calculations with a typical distribution of $Q$ imposed upon the fields
so as to further reduce any potential systematic errors.


\vspace*{0.15in}

\leftline{{\it E-mail:} a.athenodorou@cyi.ac.cy, mike.teper@physics.ox.ac.uk}


\end{titlepage}

\setcounter{page}{1}
\newpage
\pagestyle{plain}

\tableofcontents

%
%
%
%
\section{Introduction}
\label{section_intro}

In this paper we calculate various physical properties of $SU(N)$ 
gauge theories in $3+1$ dimensions. We do so by performing calculations 
in the corresponding lattice gauge theories over a sufficient range of lattice
spacings, $a$, and with enough precision that we can obtain plausible
continuum extrapolations. We also wish to be able to extrapolate to the
theoretically interesting $N=\infty$ limit and to compare this to the
physically interesting $SU(3)$ theory. To do so we have performed
our calculations for $N=2,3,4,5,6,8,10,12$ gauge theories. 

Our main aim is to provide a calculation of the low-lying `glueball' mass spectrum
for all quantum numbers and all $N$. This means calculating the lowest states in all 
the irreducible representions, $R$, of the rotation group of a cubic lattice, and for
both values of parity $P$ and charge conjugation parity $C$. We also calculate the
confining string tension, $\sigma$, so as to provide a useful scale for the glueball
masses. In addition we calculate a number of other interesting quantities.
However these are side-products of our glueball calculations rather than being
dedicated calculations of these quantities. The first is the $k=2$ string tension, 
$\sigma_{k=2}$ and the nature of its approach to $N=\infty$. The second is an estimate 
of the scale parameter $\Lambda_{\overline{MS}}$ for all our $SU(N)$ gauge theories. 
Thirdly, in order to monitor the topological `freezing' as $a$ becomes small
at fixed $N$, or $N$ becomes large at fixed $a$, we calculate the interval (in units
of our Monte Carlo steps) between changes in the topological charge, as well
as the renormalisation of that charge, and also the topological susceptibility.


Our calculations of the glueball spectrum are intended to make necessary improvements to
previous work. We recall that the pioneering glueball calculations in
%
\cite{BLMT_N,BLMTUW_N}
%
were restricted to obtaining the continuum masses of the lightest and first excited 
$J^{PC}=0^{++}$ glueballs and the lightest $2^{++}$ glueball, 
in units of the string tension, with the continuum spin assignments being based on plausible
assumptions. This sufficed to provide the first fully non-perturbative demonstration that 
the basic physical quantities of the  $SU(3)$ gauge theory are `close to' those of $SU(\infty)$, 
as had long been hoped, but it did not provide the kind of detailed $N=\infty$ glueball spectrum 
that would be useful in testing theoretical approaches. Such a detailed spectrum was
subsequently provided in 
%
\cite{BLARER_N}
%
where the lightest masses in all the  irreducible representations of the rotation group of 
the cubic lattice were calculated for $N\in [2,8]$ and extrapolated to $N=\infty$. In addition 
a serious effort was made in that work to identify those states that might be multi-glueball 
`scattering' states or finite volume artifacts (based on conjugate pairs of winding flux tubes), 
rather than being the single glueballs that one would obtain in an infinite volume.  
The important drawback of this calculation is that it was made at a fixed value of 
the lattice spacing $a$ (`fixed' in units of the deconfinement temperature), corresponding 
to a value of $\beta \simeq 5.895$ in $SU(3)$. This corresponds to a coarse lattice 
discretisation, typically between the two largest lattice spacings used in our present work.
This has two important adverse consequences. The first is that it leaves uncertain
the values of the masses in the desired continuum limit. (Although the calculations in
%
\cite{BLARER_N}
%
and earlier $SU(3)$ calculations left room for being optimistic about the limited
size of any lattice corrections.) The second problem is that since the lattice 
masses $aM$ are large when $a$ is large, as in that paper, it makes the extraction
of heavier glueball masses much more ambiguous and the statistical errors much larger
than they would be at smaller values of $a$. A corollary is that it renders
continuum spin assignments more ambiguous, except for the very lightest glueballs,
since the spin assignment depends on observing near-degeneracies amongst states 
in appropriate irreducible representations of the lattice rotation group, and for this 
it is clearly essential to achieve an adequate precision in the various mass estimates. 
In addition to this earlier work, there recently appeared a potentially more relevant paper 
%
\cite{SpN_Lucini} 
%
when our work was largely completed.
This paper provides a pioneering calculation of the masses of the 
lightest glueballs in all the irreducible representations, of a number of continuum 
$Sp(2N)$ gauge theories with an extrapolation to $N=\infty$. In principle this extrapolation 
is equally relevant to the $N\to\infty$ limit of $SU(N)$ gauge theories, as pointed out in
%
\cite{SpN_Lucini},
%
since the  $Sp(2N)$ and $SU(N)$ gauge theories share a common (perturbative) planar limit.
In practice, however, the usefulness for $SU(N\to\infty)$ turns out to be very limited. Firstly, 
$Sp(2N)$ has no $C=-$ sector, so there are no predictions for half the $SU(N\to\infty)$
spectrum. Secondly the limited range of $N$ in
%
\cite{SpN_Lucini},
%
coupled with the fact that in $Sp(2N)$ the leading
correction to $N=\infty$ is $O(1/N)$ renders the large-$N$ extrapolations less convincing 
and much less precise when compared to  $SU(N)$ where the leading correction is $O(1/N^2)$.
The net result is that the errors on the $SU(N\to\infty)$ glueball masses obtained in 
%
\cite{SpN_Lucini},
%
are larger by 
a factor of $\sim 5 - 20$ compared to the values obtained in this paper, and the mass
estimates are mainly for just the ground states in each channel. This means that
the evidence for the assignment of a continuum spin $J$ only appears plausible for
the $J^P=0^+,0^-,2^+,2^-$ ground states. Of course, none of these comments detract
from the success of
%
\cite{SpN_Lucini} 
%
in their primary aim of elucidating the mass spectrum of $Sp(2N)$  gauge theories.

In this paper we provide a calculation of the masses of the ground states and some excited
states in all the irreducible representations $R$ of the rotation group of our cubic lattice 
with an extrapolation of these masses (in units of the string tension) to the continuum limit.
To make this extrapolation  more reliable and more precise we extend the range of
our calculations to much smaller lattice spacings than earlier work. This enables
us to extract the masses of the heavier ground states and most of the first few
excited states much more reliably than earlier calculations.
As an important by-product, all this will allow us to make a significant number 
of continuum spin assignments after the extrapolation to the continuum limit. 
We recall that the assignment of a continuum spin typically depends on observing 
near-degeneracies amongst both ground and excited states in the various irreducible 
representations of the lattice rotation group, and this requires both precision 
in the mass estimates and small enough lattice spacings for the masses of the
heavier excited states to be plausibly estimated. We also extend the range 
of our calculations to larger values of $N$. This is not only to make the 
extrapolation to $N=\infty$ somewhat more reliable and precise but also, and perhaps 
more importantly, to help in excluding from our $N\to\infty$ glueball spectrum any 
multiglueball scattering states and  finite volume states, since these states 
will decouple from our single trace operators as $N\uparrow$.


This kind of calculation is of course standard in the case of $SU(3)$, see for example
%
\cite{CMMT-1989,CM-UKQCD-1993,MP-1999,HM_Thesis,HMMT-2004,MP-2005,AAMT-2020},
%
and what we have attempted to do is to bring the $SU(N)$ calculations towards a similar level
of sophistication. The computational cost of performing calculations at larger
$N$ means that further improvement to our work is still desirable. 
(It would also be very interesting to see the existing calculations at very large $N$ using 
space-time reduction
%
\cite{MGPAGAMO-2020}
%
extended to calculations of the glueball spectrum, just as they have been in
$2+1$ dimensions
%
\cite{MGPAGAMKMO-2018}.)
%
Nonetheless we are able in this paper to provide the first calculation of the masses 
of the ground states in all the $R^{PC}$ channels of the continuum $SU(N\to\infty)$ gauge 
theory,  as well as some excited states in most channels.


The plan of the paper is as follows. In the next section we introduce our lattice
setup and describe how we calculate energies from correlators. We discuss some of
the main systematic errors affecting these calculations and how we deal with them,
with a particular focus on the rapid loss of tunneling between sectors
of differing topology as $N\uparrow$. In Section~\ref{section_strings} we describe in 
detail our calculation
of the confining string tension, $\sigma$, which we will later use as the physical scale in
which to express our glueball masses. As a side product we also calculate the string tension, 
$\sigma_{k=2}$, of the lightest flux tubes carrying $k=2$ units of fundamental flux.
In Section~\ref{subsection_kstringlargeN} we study the approach
of the ratio $\sigma_{k=2}/\sigma$ to the $N=\infty$ limit so as to address the
old controversy concerning the power of  $1/N$ of the leading correction. In
Section~\ref{section_coupling} we show how our precise calculation of the fundamental 
string tension, $a^2\sigma$, as a function of the lattice coupling 
enables us to confirm the expected scaling of $g^2$ with $N$, and motivates 
(lattice improved) perturbative fits that allow us, in Section~\ref{subsection_Lambda}, 
to estimate a value for $\Lambda_{\overline{MS}}$ as a function of $N$. We also provide, 
in Section~\ref{subsection_interpol}, some analytic interpolation/extrapolation formulae 
for the variation of the string tension
with the coupling, which may be of use in other calculations.
We then turn, in Section~\ref{section_glueballs}, to our main
calculation in this paper, which is that of the low-lying glueball spectrum.
We calculate the masses on the lattice, extrapolate to the continuum limit, and
then extrapolate to $N=\infty$. Although these states are classified according to
the representations of the rotation group of our cubic lattice, we are able to
identify the continuum spins in many cases, as described in Section~\ref{subsection_spins}.
In doing all this we need to address the
problem of the extra states that winding modes introduce into the glueball spectrum and
also the possible presence of multi-glueball 'scattering' states.
We complete Section~\ref{section_glueballs} with a brief comparison of our results with
those of some earlier calculations. We then return,
in Section~\ref{section_topology}, to some of the properties of the topological
fluctuations in our lattice fields. After illustrating in Section~\ref{subsection_Qcooling}
how our cooling algorithm reproduces the topological charge of a lattice field, we
calculate in Section~\ref{subsection_Qtunneling} the rate of topological freezing with increasing
$N$ and with decreasing $a(\beta)$ and compare our results to the simplest theoretical expectations.
In Section~\ref{subsection_Qsusc} we provide our results for the topological
susceptibility in the continuum limit of $SU(N\leq 6)$ gauge theories and in the $N\to\infty$
limit. Then, in Section~\ref{subsection_QZ}, we calculate the multiplicative renormalisation of
our lattice topological charge for each $SU(N)$ group and provide interpolating formulae
which may be useful in calculations with a $\theta$ parameter in the lattice action.
Section~\ref{section_conclusion} summarises our main results. 


Finally we remark that in parallel with the present calculations most of our $SU(3)$
calculations, which are of particular physical interest, have recently been published
separately 
%
\cite{AAMT-2020}.
%



%
%
%
%
\section{Calculating on a lattice}
\label{section_lattice} 

%
%
\subsection{lattice setup}
\label{subsection_lattice_setup}


We work on hypercubic lattices of size $L_s^3L_t$ with lattice spacing $a$ and with
periodic boundary conditions on the fields. The Euclidean time extent, $aL_t$, is
always chosen large enough that we are in the confining phase of the theory, at a
temperature that is well below the deconfining phase transition
%
\cite{BLMTUW_Tc}.
%
Our fields are $SU(N)$ matrices, $U_l$,
assigned to the links $l$ of the lattice. The Euclidean path integral is 
%
\begin{equation}
Z=\int {\cal{D}}U \exp\{- \beta S[U]\},
\label{eqn_Z}
\end{equation}
%
where ${\cal{D}}U$ is the Haar measure and we use the standard plaquette action,
%
\begin{equation}
\beta S = \beta \sum_p \left\{1-\frac{1}{N} {\text{ReTr}} U_p\right\}  
\quad ; \quad \beta=\frac{2N}{g^2_L}.
\label{eqn_S}
\end{equation}
%
Here $U_p$ is the ordered product of link matrices around the plaquette $p$. We write
$\beta=2N/g^2_L$ since in this way we recover the usual continuum action when we
take the continuum limit of the lattice theory and replace $g^2_L\to g^2$.
The subscript $L$ reminds us that this coupling is defined in a specific coupling
scheme  corresponding to the lattice and the plaquette action. Since $g^2_L$ is the
bare coupling corresponding to a short distance cut-off $a$, it provides a definition
of the running coupling $g^2_L(a)=g^2_L$ on the length scale $a$. Since the theory is
asymptotically free $g^2_L(a)\to 0$ as $a\to 0$ and hence $\beta\to \infty$ as $a\to 0$
and so we can decrease the cut-off
$a$ and so approach the desired continuum limit of the theory by increasing $\beta$. Our
calculations of this lattice path integral are carried out via a standard Monte Carlo using a
mixture of Cabibbo-Marinari heat bath and over-relaxation sweeps through the lattice.
We typically perform $\sim 2\times 10^6$ sweeps at each value of $\beta$ at each lattice size,
and we typically calculate correlators every $\sim 25$ sweeps and the topological charge
every $50$ or $100$ sweeps. Naturally we choose values of $\beta$ that place us on the
weak coupling branch of the lattice theory. We discuss the `bulk' transition that separates weak
coupling from strong coupling, and becomes a first order transition for $N\geq 5$
%
\cite{BLMTUW05},
%
in Section~\ref{subsection_bulk}. There are of course other possible choices for the action,
and we briefly comment upon our choice in Section~\ref{subsection_comparisons}, where
we compare our results to those of other recent calculations.


We simulate $SU(N)$ lattice gauge theories for $N=2,3,4,5,6,8,10,12$ over a range of values
of $\beta$ so as to be able to plausibly obtain, by extrapolation, the glueball spectra and
string tensions of the corresponding continuum gauge theories. A summary of the basic
parameters of our calculations is given in Tables~\ref{table_param_SU2}-\ref{table_param_SU12}.
For each of our $SU(N)$ calculations we show the values of $\beta$, the lattice sizes, the
average plaquette, the string tension, $a^2\sigma$, and the mass gap, $am_G$. In the header of
each table we show the approximate spatial size in units of the string tension. Since
finite volume corrections are expected to decrease with increasing $N$ we decrease the
lattice volume as we increase $N$. This expectation needs to be confirmed by explicit
calculations which we shall provide in Section~\ref{subsection_massV}.


%
%
\subsection{energies and correlators}
\label{subsection_lattice_energies}

We calculate energies from correlation functions in the standard way. Suppose we wish to
calculate glueball masses in some representation $R^{PC}$. If one picks an operator $\phi(t)$
with quantum numbers $R^{PC}$ and momentum $p=0$ then the correlator will provide us with
the energies $E_i$ of states with those quantum numbers 
%
\begin{equation}
  C(t) = \langle \phi^{\dagger}(t)\phi(0)\rangle
  = \sum_{n=0} |c_n|^2 \exp\{-E_nt\} \stackrel{t\to\infty}{\longrightarrow}
  |c_0|^2 \exp\{-E_0t\},
\label{eqn_C}
\end{equation}
%
where $E_n\leq E_{n+1}$ and $E_0$ is the lightest state with $R^{PC}$ quantum numbers --
which will often be the lightest glueball in that sector. (If $\phi$ has vacuum quantum
numbers then one uses the vacuum subtracted operator.)
Since on the lattice time is measured in lattice units, $t=an_t$, 
what we obtain is the value, $aE_0$, of the mass in lattice units. If we calculate two masses,
$aM$ and $a\mu$, in this way then the lattice spacing drops out of the ratio and if we
calculate the ratio for several values of $a(\beta)$ we can extrapolate to the continuum
limit in the standard way, using
%
\begin{equation}
 \frac{aM(a)}{a\mu(a)} = \frac{M(a)}{\mu(a)} \simeq \frac{M(0)}{\mu(0)} + ca^2\mu^2 
\label{eqn_cont}
\end{equation}
%
once  $a(\beta)$ is small enough. (This standard tree level extrapolation could be improved with
perturbative corrections
%
\cite{Sommer-cutoff}
%
and with higher order power corrections but our calculations are not so extensive and precise as to
motivate such modifications.)

The starting point for an operator $\phi$ is the trace of a closed loop on the lattice.
For an operator that projects onto glueballs the loop should be contractible.
A non-contractible loop that closes across the periodic spatial boundary
will project onto a confining flux tube that winds around that spatial torus.
For the contractible glueball loop one takes a suitable linear combination of rotations
of that loop for it to be in the desired representation $R$ of the lattice rotation
group and together with the parity inverse this allows us to form an operator
with parity $P$. The real and imaginary parts of the original loop will correspond to
$C=+$ and $C=-$ respectively. Summing this linear combination over all spatial sites
at time $t$ gives us the operator with $p=0$.

The statistical errors on $C(t)$ are roughly independent of $t$ while its value decreases
exponentially with $t$, so if we are to estimate $aE_0$ from $C(t)$ in eqn(\ref{eqn_C})
we need to be able to do this at small $t$. (The fluctuations of $\phi^{\dagger}(t)\phi(0)$
involve the higher order correlator $(\phi^{\dagger}(t)\phi(0))^2$ which typically has a vacuum
channel and this is independent of $t$.) For this to be possible we need the operator
$\phi$ to have a large overlap onto the state $|n=0\rangle$ that corresponds to $E_0$,
i.e. that $|c_0|^2/\sum_n |c_n|^2 \sim 1$. One can achieve this using iteratively `blocked'
link matrices and loops
%
\cite{MT-block1,MT-block2}
%
as described in detail in, for example,
%
\cite{BLMTUW_N}.
%
To monitor the approach of  $C(t)$ to the asymptotic exponential decay in eqn(\ref{eqn_C})
it is useful to define an effective energy
%
\begin{equation}
  \frac{C(an_t)}{C(a(n_t-1))} = \exp\{-aE_{eff}(n_t)\}.
\label{eqn_Eeff}
\end{equation}
%
(In practice we use a cosh modification of this definition to take into account the
periodicity in the temporal direction.)
If, within errors,  $aE_{eff}(n_t) = \mathrm{const}$ for $n_t\geq \tilde{n}_t$ then
we can use $aE_{eff}(\tilde{n}_t)$ as an estimate for $aE_0$, or we can do a simple
exponential fit to $C(t)$ for $n_t\geq \tilde{n}_t-1$ to estimate $aE_0$. 


To calculate not just the ground state energy but also some usefully precise excited state
energies from $C(t)$ in eqn(\ref{eqn_C}) would require a precision that is at present
unachievable. Instead the standard strategy is to use a variational calculation. One chooses
some basis of $n_0$ operators $\{\phi_i(t): i=1,..,n_0\}$, calculates the cross-correlators
$C_{ij}(t)= \langle \phi^{\dagger}_i(t)\phi_j(0)\rangle$ and finds the linear combination,
$\Phi = \Phi_0$, that maximises $C(t)= \langle \Phi^{\dagger}(t)\Phi(0)\rangle$
for some suitable small $t=t_0$. This is then our best estimate of the wave-functional
of the ground state $|n=0\rangle$. We then extract $aE_0$ from the asymptotic
exponential decay of $C(t)= \langle \Phi^{\dagger}_0(t)\Phi_0(0)\rangle$. To calculate
the first excited energy $aE_1$ we consider the subspace of $\{\phi_i: i=1,..,n_0\}$
that is orthogonal to $\Phi_0$ and repeat the above steps. This gives our best
estimate $\Phi_1$ for  $|n=1\rangle$ and we estimate $aE_1$ from the asymptotic
exponential decay of the correlator of $\Phi_1(t)$. Similarly by considering
the subspace of the operator vector space that is orthogonal to $\Phi_0$ and $\Phi_1$ 
we can obtain estimates for the next excited state and so on. As with any variational
calculation, the accuracy of such estimates will depend on having a large enough
basis of operators.

%
%
\subsection{systematic errors}
\label{subsection_lattice_systematics}

In addition to the statistical errors in the calculation of energies and masses, which
can be estimated quite reliably, there are systematic errors that are harder to control.
We will now briefly point to some of these, leaving a more detailed discussion till later on.

Since the error to signal ratio on $C(n_t)$ increases at least as fast as
$\propto \exp\{+aE_0n_t\}$, the range of $n_t$ which is useful rapidly decreases
when the energy $aE_0$ increases, and an obvious systematic error is that we begin our
fit of the asymptotic exponential decay at too small a value of $t=an_t$.
Given the positivity properties of our correlators, this means that as the true
value of $E_0$ increases, so does our overestimate of its value. For any particular
state this problem will become less severe as we decrease $a$, and this provides
some check on this error.

The correlator of an excited state, using our best variational choice of operator,
may contain small contributions from lighter excitations since our basis is far
from complete. So at large enough $n_t$ these will dominate the correlator and
the effective mass will drop below that of the mass of the state of interest.
This does not appear to be a severe problem for us because the pattern of our statistical
errors, which grow rapidly with $n_t$, prevents us from obtaining values at large $n_t$.
In any case we attempt to identify at most an intermediate effective mass plateau
from which we extract the desired mass.

If a state has too small an overlap onto our basis of operators then, given the
exponential growth with $t$ of the statistical error on the effective energy, the
state will be, at best, assigned a mass that is much too high and so it will typically not
appear in its correct order in the mass spectrum. If so this means that there will be
a missing state in our calculated glueball spectrum and this will not be helpful if,
for example, one wishes to check theoretical models of the spectrum against the lattice
spectrum. To reduce the possibility of this occurring one needs to use a large basis of
operators, just as in any variational calculation.

The states we are primarily interested in are single particle glueball states rather
than multi-glueball scattering states. (In our finite spatial volume multiglueball states
are interacting at all times but for simplicity we refer to them as scattering states.)
We use single trace operators which should ensure that at large $N$ our correlators
receive no multi-glueball contributions but this is not necessarily the case
at smaller $N$. This is of course related to the fact that at smaller $N$ 
heavier glueball states may have substantial decay widths. Later in the paper we will
provide an exploratory analysis of this issue by combining single and double trace
operators.

The spectrum in a finite spatial volume will differ from the spectrum in an infinite
volume. One type of correction to a glueball propagator comes, for example, from a virtual
glueball loop where one line of the two in the loop encircles the spatial torus. Such
corrections are exponentially small in, typically, $M_GL$ where $M_G$ is the mass gap
(the lightest scalar glueball) and $L$ is the spatial size
%
\cite{Luscher-V}.
%
These corrections will be very small in our calculations. For the intermediate volumes
on which we work a much more relevant finite volume correction comes from a confining
flux tube that winds once around one of the spatial tori. Such a flux tube is often
referred to as a torelon. By itself a torelon has zero overlap onto local particle states.
However a state composed of two torelons, where one is conjugated, will in general
have a nonzero overlap. Such a `ditorelon' will not only shift the energies of
the usual particle glueballs but will introduce extra states into the glueball spectrum.
These extra states can be identified through their volume dependence and in some
other ways as we shall see later on.

In addition to the usual gradual loss of ergodicity as the lattice spacing decreases,
the tunnelling between sectors of differing topological charge $Q$ suffers a much more
rapid supression not only as $a\to 0$ at fixed $N$ but also as $N$ increases at fixed $a$.
This affects a significant part of our calculations and we discuss next how we deal with it
in practice.


%
%
\subsection{topological freezing}
\label{subsection_Qfreezing}

We begin by briefly summarising the reason that one loses ergodicity in the
topology of lattice gauge fields much faster than in other typical quantities 
as $a\to 0$ at fixed $N$ and as $N\to\infty$ at fixed $a$, when using a local 
Monte Carlo algorithm,. We then describe where this impacts on our particular
calculations and why this inpact should be minor on theoretical grounds. 
We provide explicit calculations to confirm this expectation and, finally, 
we describe the additional procedure that we follow to minimise any remaining
systematic errors. It is important to note that all these arguments are for pure 
gauge theories and would have to be reconsidered in the presence of light fermions
as in $QCD$.  

  
In a periodic four-volume a continuum gauge field of topological charge $Q$ cannot be smoothly
deformed into a gauge field of charge $Q^\prime\neq Q$. So in a sequence of lattice fields
that have been generated using a Monte Carlo algorithm that is local any change in
the value of $Q$ should be increasingly suppressed as we approach the continuum limit.
This statement can be made quantitative as we do below. But before that we note that
there is a separate suppression that occurs at any fixed small value of $a$
when we increase $N$. To smoothly deform a lattice field from, say, $Q=1$ to $Q=0$
in a finite volume with periodic gauge potentials, the core of an original
extended instanton must gradually shrink until it disappears within a hypercube,
leaving a simple gauge singularity centered on that hypercube.
Before it does so it will be a small instanton (assuming $a$ is small) which
we can describe using standard semiclassical methods. So its effective action 
will be $S_I(\rho) \sim 8\pi^2/g^2(\rho)$ where $\rho$ is the size of the core
and provides the relevant scale, and the probability of it existing
in a field will be $\propto \exp\{-S_I(\rho)\} \sim \exp\{- 8\pi^2/g^2(\rho)\}$
per unit volume up to a $1/\rho^4$ volume factor and some less important factors
%
\cite{tHooft_Q,Coleman_Q}.
%
Now at large $N$ the 't Hooft coupling $\lambda = g^2N$ needs to be kept constant
for a smooth limit
%
\cite{tHooft_N,Coleman_N}.
%
In terms of the running coupling this means $\lambda(l) = g^2(l)N$
is independent of $N$ up to corrections
that are powers of $1/N$, where $l$ is measured in some physical units, e.g.
the mass gap. So the weight of a small instanton vanishes as
$\propto \exp\{- 8\pi^2N/\lambda(\rho)\}$, i.e. it vanishes exponentially
with $N$
%
\cite{Witten_Q,Teper_Q}.
%
Since the process of going `smoothly' from a field with $Q=1$ to a field with
$Q=0$ necessarily involves at an intermediate stage of the process the presence of
fields containing such small instantons, we immediately infer that the change in
topology must be at least exponentially suppressed with increasing $N$ once $a$ is small.
Such a strong suppression has indeed been observed in earlier lattice calculations
%
\cite{BLMT_N,LDD_K2}.
%
Returning to the $a\to 0$ limit at fixed $N$ we note that one can use the one-loop expression
for $g^2(\rho)$ and this leads to a suppression of the tunnelling between
topological sectors that is the appropriate power of $\rho$ and hence of $a$, once
$\rho \sim O(a)$.

In our calculations we normally start a Monte Carlo sequence with the trivial $Q=0$
gauge field $U_l=I, \forall l$, and `thermalise' with some tens of thousands of sweeps
to reach the `equilibriated' lattice gauge fields which we can begin to use for our
calculations. As described above, for larger $N$ and smaller $a$ the topological 
tunnelling can become
sufficiently rare that the fields remain at $Q=0$ even after our attempted
equilibriation. In practice this proves not to be an issue for any of our $SU(2)$, $SU(3)$,
$SU(4)$ and $SU(5)$ calculations. For $SU(6)$ it becomes a problem for the smallest values
of $a$ and for $SU(N\geq 8)$ it is a problem for all but our largest values of $a$.
That is to say, within our statistics and our range of $a$ topological freezing
is largely an issue to do with $N\uparrow$ rather than with $a\downarrow$.
This distinction is important since the currently accepted theoretical picture
%
\cite{Witten_98}
%
tells us that we are sitting in a vacuum with periodicity in the $\theta$ parameter of
$2\pi N$ rather than $2\pi$. (It is other interlaced $\theta$-vacua that provide the
expected $2\pi$ periodicity.) This implies that higher moments in $Q$ disappear as inverse
powers of $N$, and so does the correlation between the topological charge and gluonic 
operators. Lattice calculations 
%
\cite{LDDGMEV_Q}
%
have confirmed this expectation. This tells us that at large enough $N$ the loss of
ergodicity should not affect physical quantities such as glueball masses. 

Of course we do not know in advance what value of $N$ is `large enough' and so it
is useful to perform
an explicit check of how such freezing affects the physics we are interested in.
For this we choose to work in $SU(8)$ at $\beta=45.50$ where changes in topology
are extremely rare, but which is close to the value $N=6$ where we begin to encounter
a serious loss of topological ergodicity in our calculations.
We begin with a comparison of two sets of about $2\times 10^6$ fields
on a $14^320$ lattice, with glueball and flux tube `measurements' taken every
25'th update. One set of fields has $Q=0$ throughout while the other has what one expects
to be the correct distribution of $Q$ imposed through about 40 different starting
configurations as described above. We calculate glueball and flux tube correlators
in exactly the same way in both ensembles of fields and perform the usual variational
calculation for each representation $R^{PC}$. Note that here we include all the states
that we obtain and make no attempt to identify any of the finite volume ditorelon states
alluded to in Section~\ref{subsection_lattice_systematics}.
In most, but not all, cases the effective
energy plateau in the effective energies, $aE_{eff}(t)$, defined in eqn(\ref{eqn_Eeff})
begin at $t=2a$. So we show in Table~\ref{table_GKvsQ_SU8_l14} the values of 
$aE_{eff}(t=2a)$ that we obtain in each of the two sets of fields, for the ground
states and some of the lowest lying excited states. We see that the two sets of values
are remarkably consistent within errors, with only a small handful of heavier states
differing by slightly more than $2\sigma$. In addition to the glueball effective
energies we also show those for the lightest two states of a flux tube that  winds
around the spatial torus. We do so for the flux in the fundamental representation
($k=1$) and for the flux in the $(k=2)=(k=1)\otimes (k=1)$ representation.
Here too we see consistency within statistical errors. In addition to this study
on the $14^320$ lattice we perform a similar one on a smaller $12^320$ lattices at
the same $\beta$. This is the lattice size, in physical units, that is used in
our calculations later on in this paper, and we recall that in general the local
effects of constraining the topological charge should increase with decreasing volume
(see e.g.
%
\cite{Aoki}).
%
%Here, in Table~\ref{table_GKvsQ_SU8_l12}, we provide our actual energy estimates 
%rather than the effective energy at $t=2a$. (In almost all cases the numbers are similar.) 
Again we have consistency within errors for almost all the states.
The conclusion of these explicit calculations is that within our typical statistical
errors the physical quantities we calculate in this paper are not affected by
constraining the total topological charge to $Q=0$. 

Although both the theory and our above explicit checks suggest that the loss of
topological ergodicity in our calculations will have very little effect on
our measured glueball spectrum, we implement the following additional procedure that
is designed to reduce any remaining biases. Suppose we are confronted with 
topological freezing on a lattice of size $L_s^3L_t$ at some value of $\beta$ in some
$SU(N)$ theory. Now instead of starting with a $Q=0$ field with $U_l=I, \forall l$,
we produce a set of sequences in parallel which start with  fields that have various
values of $Q$. We choose the ensemble of these starting fields to roughly reproduce
the expected distribution. What is `expected' we infer from what we observe at the lower
values of $N$ where there is no topological freezing and also from the values obtained
at the same value of $N$ but at the larger values of $a(\beta)$ (if any) where the freezing
has not yet set in. This is a plausible strategy since we find that in those cases
where the distribution of $Q$ can be determined there is little variation
in the distribution of $Q$ with either $N$ or $a$ on an equal physical volume.
To produce such a starting field with some
$Q$ we generate a sequence of $SU(3)$ fields on the same size lattice as our $SU(N)$ one
and we pick out a field with the desired charge $Q$. We `cool' this field
to produce a field with minimal fluctuations apart from the net topological charge.
(See Section~\ref{section_topology} for details.)
We then embed this lattice field of $3\times 3$ complex
matrices into a suitable corner of our set of $N\times N$ unit matrices, so as to produce
a set of $SU(N)$ matrices with charge $Q$. If we have topological freezing then this
starting distribution of $Q$ will be maintained in our set of Monte Carlo sequences.
This method has some disadvantages but also some advantages. One advantage is that
the lack of topological tunneling is a better representation of the continuum
theory: we are sampling disconnected sectors. After all, the field configurations near the
point of topological tunnelling have no continuum correspondence and are a lattice
artifact. The major disadvantage is that we necessarily have a rather limited set of
parallel sequences (typically 20 or 40) and hence values of $Q$, and so only the most
probable values of $Q$ are properly represented. Lattice fields with values of $Q$ that have
a low probability simply do not appear. That is to say we are missing the
tail of the $Q$ distribution, and so any physics that depends on 
higher moments of $Q$ will be something we cannot address here.
However theory tells us that the correlation between such higher moments of $Q$
and our gluonic operators will be suppressed as a correpondingly higher power of $N$,
so any bias should be a small fraction of what, as we have seen above, will be at most
a very small shift to the glueball masses. 


In conclusion, our explicit calculations, the theoretical arguments, and our
additional procedure of imposing upon our fields something quite close to the 
expected distribution of $Q$ makes us confident that within our typical statistical 
errors any systematic error from the freezing of topology is negligible in our 
glueball calculations. 





%
%
%
%
\section{String tensions}
\label{section_strings} 


Let us label the sites of the hypercubic lattice by the integers $n_x,n_y,n_z,n_t$.
To project onto a fundamental ($k=1$) flux tube the simplest operator we can use is
$\mathrm{Tr}(l_f)$ where $l_f$ is the product of link matrices along a minimal path
encircling  the spatial torus. This is just the usual spatial Polyakov loop. For example,
if the spatial lattice size is $L$ then the Polyakov loop in the $x$ direction is
%
\begin{equation}
 l_f(n_y,n_z,n_t) = \prod_{n_x=1}^{n_x=L} U_x(n_x,n_y,n_z,n_t).
  \label{eqn_Poly}
\end{equation}
%
We also use such Polyakov loops composed of blocked links and we translate the operator
along the $x$ direction.
Clearly the parities of such states are positive. This loop is translationally
invariant along the flux tube direction and so the momentum along the flux tube is
zero; $p_{\parallel}=0$. It is rotationally invariant around its axis, so $J=0$.
Because of the usual centre symmetry (see below) the $C=+$ and $C=-$ flux tubes are
degenerate. We then sum $l_f(n_y,n_z,n_t)$ over $n_y$ and $n_z$ so that the transverse
momentum $p_{\perp}=0$. Thus the lightest state onto which this final operator projects
should be the ground state of the winding flux tube that carries fundamental flux.
To project onto a $k=2$ flux tube we do the same using linear combinations of the
operators $\mathrm{Tr}(l_fl_f)$ and $\mathrm{Tr}(l_f)\mathrm{Tr}(l_f)$. For $k=3$ we use linear
combinations of $\mathrm{Tr}(l_fl_fl_f)$, $\mathrm{Tr}(l_fl_f)\mathrm{Tr}(l_f)$ and
$\mathrm{Tr}(l_f)\mathrm{Tr}(l_f)\mathrm{Tr}(l_f)$, and so on for higher $k$.
The defining property of a $k$-string that winds around the spatial torus in say the
$x$ direction is that under a transformation of the link matrices
$U_x(n_x,n_y,n_z,n_t) \to \exp\{i2\pi /N\} U_x(n_x,n_y,n_z,n_t)$ for all $n_y,n_z,n_t$
and at some fixed value of $n_x$, it will acquire a factor $\exp\{i2k\pi /N\}$.
Since this transformation is a symmetry of the theory (the action and measure are
unchanged) and at low temperatures the symmetry is not spontaneously broken,
any two operators for which this factor of $\exp\{i2k\pi /N\}$ differ will
have zero mutual overlap. So we have separate sectors of flux tubes labelled
by $k$ with the fundamental $k=1$ strings existing for $N\geq 2$,
the $k=2$ strings for $N\geq 4$, the $k=3$ strings for $N\geq 6$, and so on.
We note that our basis of operators for a $k$-string could easily be extended by
multiplying the operators by factors such as $\mathrm{Tr}(l_f)\mathrm{Tr}(l_f^\dagger)$,
but while this might be useful to improve the overlap onto massive excitations
it would be very surprising if it improved the overlap onto the ground state
which is what we are interested in here.


%
%
\subsection{finite volume corrections}
\label{subsection_KcorrnV}


As described above, we calculate the ground state energy of a flux tube that winds once around
a spatial torus. If the lattice has size $l^3l_t$, this flux tube has length $l$ and we generically
denote the energy by $E_k(l)$, where $k=1$ corresponds to a flux tube carrying fundamental flux
and $k$ to a $k$-string (carrying $k$ units of fundamental flux). The string tension $\sigma_k$
is the energy per unit length of a very long flux tube
%
\begin{equation}
\sigma_k = \lim_{l\to\infty}\frac{E_k(l)}{l}.
  \label{eqn_K}
\end{equation}
%
In practice our tori are finite and so we expect that there will be corrections to the
leading linear dependence of $E_k(l)$.
In our calculations we will normally estimate the asymptotic string tension
$\sigma_k$ using the `Nambu-Goto' formula
%
\begin{equation}
E_k(l) = \sigma_k l\left( 1 - \frac{2\pi}{3\sigma_k l^2}\right)^{\frac{1}{2}}.
  \label{eqn_NG}
\end{equation}
%
When expanded in powers of $1/\sigma_kl^2$ this formula generates all the known universal terms
%
\cite{string_1,string_2,string_3,string_4}
%
and, at least for fundamental flux, proves to accurately describe numerical calculations for the range
of $l$ relevant to our calculations
%
\cite{AABBMT_K}.
%
However, strictly speaking, these results hold for flux tubes that are effectively in an infinite
transverse volume, while in our case, in order to maintain the rotational symmetry needed for our glueball
calculations, the transverse size is $l\times l$ and so decreases when we decrease $l$.
We therefore need to perform some explicit checks, i.e. to check whether eqn(\ref{eqn_NG}) encodes all
the finite-$l$ corrections that are visible within our typical statistical errors.


We see from eqn(\ref{eqn_NG}) that the relevant scale is $l$ in units of the string tension,
i.e. $l\surd\sigma$. These scales are listed in the headers of
Tables~\ref{table_param_SU2}-\ref{table_param_SU12} and they vary from $l\surd\sigma\sim 4$
for lower $N$ to $l\surd\sigma\sim 2.6$ for our highest values of $N$. As far as glueball
masses are concerned the theoretical justification for this decrease in the volume
is that we expect finite volume corrections to decrease as $N$ grows and the practical
reason is that the expense of the $SU(N)$ matrix product calculations grows as $\propto N^3$. 
In order to check whether there are significant corrections to eqn(\ref{eqn_NG}) at the values
of $l$ we use, we calculate $\sigma$ on larger (sometimes smaller) lattices at selected values of
$\beta$. These calculations are summarised in Table~\ref{table_V_k1_SUN}. For $N=8,10,12$
the scale $l\surd\sigma\sim 2.6$ corresponds to $l=12$ in the Table, and we see that the
value of $a\surd\sigma$ is the same for the longer $l=14$ flux tube, suggesting that there are no
significant finite $l$ corrections to eqn(\ref{eqn_NG}) for these values of $N$. The same
conclusion holds for the other groups in Table~\ref{table_V_k1_SUN}. All this provides evidence
that the fundamental string tensions quoted in Tables~\ref{table_param_SU2}-\ref{table_param_SU12}
do not suffer significant finite volume errors, at least within our typical statistical errors.

The energies of $k\geq 2$ flux tubes are larger and therefore so are the statistical
and systematic errors in estimating them. Since the energies grow with $k$, we have
only attempted a finite volume study for $k=2$ flux tubes. This is presented in
Table~\ref{table_V_k2_SUN}. The message here is more nuanced than for $k=1$: while any finite $l$
corrections to eqn(\ref{eqn_NG}) appear to be insignificant when $l\surd\sigma \gtrsim 3$,
and hence for our $N \leq 6$ string tension calculations, there is strong evidence that
there is a correction of $\simeq 1.9(5)\%$ to the values obtained with $l\surd\sigma \sim 2.6$,
which are the values we emply for $N\geq 8$. This is something
we will need to consider when we estimate the errors on our $k=2$ string tensions. Presumably
the $k=3$ and $k=4$ string tensions show corrections at least as large, but we do not
have any finite $l$ study of these. 


%
%
\subsection{fundamental ($k=1$) string tensions}
\label{subsection_k1strings}

We calculate the ground state energies of the fundamental flux tube for the lattices, couplings
and gauge groups listed in Tables~\ref{table_param_SU2}-\ref{table_param_SU12}.
We then use eqn(\ref{eqn_NG}) to extract our estimates of the infinite volume string tensions.
The energies are obtained by identifying effective energy plateaux as described in
Section~\ref{subsection_lattice_energies}.
As an example we show in Fig.\ref{fig_EeffK1_SU8} the effective energies
that lead to the $SU(8)$ string tension estimates in Table~\ref{table_param_SU8}.
As one increases $\beta$ and hence decreases $a$, the energy in lattice units decreases
so that we can obtain precise effective energies over a larger range of $n_t$
which in turn makes it easier to identify a plateau and so estimate the corresponding
value of $E_{eff}(n_t\to\infty)$. Although we attempt to show in Fig.\ref{fig_EeffK1_SU8}
the error bands on our estimates of the corresponding flux tube energies, these are
almost invisible because the errors are too small on the scale used in the plot.
We therefore replot, in Fig.\ref{fig_EeffK1b_SU8}, the effective energies for $\beta=47.75$,
which is the calculation that is the closest to the desired continuum limit,
with a sufficient rescaling to expose
the errors on the effective masses. On the plot we show the best estimate for the
energy obtained from a fit to the correlation function, together with the
energies corresponding to $\pm 1$ standard deviations.


For the corresponding $SU(3)$ plots we refer the reader
to Fig.1 of
%
\cite{AAMT-2020}.
%
The fact that the $SU(3)$ lattices are (much) larger than the $SU(8)$ ones at similar
values of $a\surd\sigma$ means that the flux tubes are more massive, so the correlators
decrease faster with $n_t$ and the extraction of $E_{eff}(n_t\to\infty)$ is less
compelling -- despite the fact that the calculations extend closer to the continuum
limit in the case of $SU(3)$. These $SU(3)$ plots are typical of our  $N \leq 4$
calculations, where the lattices are chosen to be particularly large, while the $SU(8)$
plot is typical of  $N \geq 8$ where the spatial volumes used are smaller.

In the case of $SU(2)$ and $SU(3)$ we have also calculated the string tension on
smaller volumes which are still large enough for the flux tube calculations even if too
small for reliable glueball calculations.
This enables us to extend the calculations to smaller values of
$a\surd\sigma$ at modest computational cost. The purpose of these calculations
is to feed into our analysis of the running coupling later on in this paper. We list
the results of the calculations in Tables~\ref{table_Ksmall_SU2},\ref{table_Ksmall_SU3}.
%As an example we show on Fig.\ref{fig_EeffK1sm_SU3} the effective energy
%plots for $SU(3)$. Comparing to  Fig.1 of
%
%\cite{AAMT-2020}
%
%shows the practical advantage of working on these smaller lattices.




%
%
\subsection{$k=2$ string tensions}
\label{subsection_k2strings}

One expects that as $N\to\infty$ the $k=2$ flux tube will become two non-interacting $k=1$
flux tubes and we will have $\sigma_{k=2}\to 2\sigma_{k=1}$. However, as
earlier calculations have shown
%
\cite{BLMT_K2,AABBMT_K,LDD_K2},
%
at lower values of $N$ one finds that $\sigma_{k=2}$ is substantially less than $2\sigma_{k=1}$,
so that one can think of it as behaving like a bound state of two fundamental flux tubes.
This suggests that at lower values of $N$ the $k=2$ flux tube can be treated as a single
`string' with finite volume corrections well described by eqn(\ref{eqn_NG}). As we increase
$N$ the $k=2$ flux tube should increasingly look like two fundamental flux tubes that
are loosely bound with the binding energy vanishing as $N\to\infty$. Here one might
expect each of these two flux tubes to have finite volume corrections given by
eqn(\ref{eqn_NG}) so that the overall finite volume behaviour of the energy of a
$k=2$ flux tube becomes significantly different:
%
\begin{equation}
  E_{k=2}(l) \stackrel{\mathrm{small}\, N}{=}
  \sigma_{k=2} l\left( 1 - \frac{2\pi}{3\sigma_{k=2} l^2}\right)^{\frac{1}{2}}
  \stackrel{N\to\infty}{\longrightarrow}
  \sigma_{k=2} l\left( 1 - \frac{4\pi}{3\sigma_{k=2} l^2}\right)^{\frac{1}{2}}
  \label{eqn_NGk2}
\end{equation}
%
with $\sigma_{k=2}=2\sigma_{k=1}$. One can expect a smooth transition between these
two behaviours which clearly creates ambiguities in extracting the string tension
$a^2\sigma_{k=2}$ from the flux tube energy $aE_{k=2}(l)$.

%In Fig.~\ref{fig_EeffK2_SU4SU8} we show some representative effective energy plots
%for $k=2$ flux tubes in $SU(4)$ and $SU(8)$. The reason that the $SU(4)$ energies
%are larger than the $SU(8)$ ones is that they have been obtained on larger
%lattices, so that the flux tubes are longer and hence more massive;
%typically $l\surd\sigma\sim 4$ for $SU(4)$ versus $l\surd\sigma\sim 2.6$ for $SU(8)$.
%It is clear that one can reliably identify effective energy plateaux, particularly
%as $\beta$ increases and one approaches the continuum limit.

Assuming for now that we can treat the $k=2$ flux tube as a single string with
finite volume corrections as given by  eqn(\ref{eqn_NG}) we calculate $\sigma_{k=2}$
on our various ensembles of lattice fields. Using the previously calculated
values of $\sigma_f \equiv \sigma_{k=1}$ we form the dimensionless ratio
$\sigma_{k=2}/\sigma_f$ which we then extrapolate to the continuum limit using
eqn(\ref{eqn_cont}). We show the extrapolations in Fig.~\ref{fig_k2k1_cont}.
For $N\leq 6$ these appear to be under good control, but for larger $N$ that
is less clear. One would expect the slope of the extrapolation to vary smoothly
with $N$, and that is certainly what one sees for $N=4,5,6$. But then
there is a violent break in the behaviour between $N=6$ and $N=8$
and wild oscillations when comparing  the $N=8,10,12$ slopes.
It may be that this is due to our use of smaller spatial volumes for $N\geq 8$.
In any case, we calculate the continuum limits from these fits and the
results for our various gauge groups are presented
in the second column of Table~\ref{table_sigmak2}. As remarked earlier, there
is good evidence from Table~\ref{table_V_k2_SUN} that for the smaller lattices
that we have used for $N\geq 8$ we should apply an additional finite volume correction of
$\simeq 1.9(5)\%$. Doing this leads to the values in the third column of
Table~\ref{table_sigmak2}, which are the values we consider to be more reliable.
There is a caveat here: at our largest values of $N$ the values of $\sigma_{k=2}/\sigma_f$
are quite close to the asymptotic value of two, and so we might expect that we are in
the range of $N$ where the stronger finite volume behaviour displayed in eqn(\ref{eqn_NGk2}) 
is setting in. To settle this question would require a dedicated calculation that is
beyond the scope of this paper.

%For completeness we present in Table~\ref{table_sigmak3k4} our continuum extrapolations
%of the $k=3$ and $k=4$ string tensions.
We have also obtained continuum extrapolations of the $k=3$ and $k=4$ string tensions.
These have been obtained from the corresponding flux tube energies using eqn(\ref{eqn_NG})
and this undoubtedly underestimates the finite volume corrections. Indeed as $N\to\infty$
we expect $\sigma_k\to k \sigma_f$ as the $k$-string becomes $k$ non-interacting
fundamental strings, and then we expect a version of eqn(\ref{eqn_NGk2}),
%
\begin{equation}
  E_k(l) \stackrel{N\to\infty}{\longrightarrow}
  k\sigma_f l\left( 1 + \frac{2\pi}{3\sigma_f l^2}\right)^{\frac{1}{2}}
  =
  \sigma_k l\left( 1 + \frac{2k\pi}{3\sigma_k l^2}\right)^{\frac{1}{2}}.
  \label{eqn_NGk}
\end{equation}
%
Given this uncertainty and the increasing ambiguity with increasing $k$ of extracting
an effective energy plateau (because of their increasing energies), we do not
discuss these values any further here.



%
%
\subsection{$N\to\infty$ extrapolations}
\label{subsection_kstringlargeN} 

An interesting question about the ratio $\sigma_{k=2}/\sigma_f$ concerns its dependence
on $N$. On general grounds we expect the leading correction to the asymptotic value
to be $O(1/N^2)$. However there are old ideas under the label of `Casimir Scaling',
that the (string) tension of a flux tube should be proportional to the smallest quadratic
Casimir of the representations that contribute to that flux, as though the flux tube
joining two sources behaved, in this respect, just like one gluon exchange. For
the $k$ flux tube the relevant representation is the totally antisymmetric one and this
predicts
%
\begin{equation}
 \frac{\sigma_k}{\sigma_f}
 \stackrel{CS}{=}
 \frac{k(N-k)}{N-1}
 \stackrel{N\to\infty}{=}
  k-\frac{k(k-1)}{N}-\frac{k(k-1)}{N^2} +O\left(\frac{1}{N^3}\right)
  \label{eqn_sigkCS}
\end{equation}
%
so that the leading correction is $O(1/N)$ rather than $O(1/N^2)$. Previous lattice
calculations have often favoured Casimir scaling as a good approximation
%
\cite{CSlattice}
%
so it is interesting to test this idea against our $k=2$ values listed in Table~\ref{table_sigmak2}.
In Fig.\ref{fig_k2k1_N} we plot our continuum values of  $\sigma_{k=2}/\sigma_f$ against $1/N^2$.
In addition we know that $\sigma_{k=2}/\sigma_f=2$ at $N=\infty$ so we impose this as a constraint
in our fits. In Fig.\ref{fig_k2k1_N} we show our best fit in powers of $1/N^2$
%
\begin{equation}
 \frac{\sigma_{k=2}}{\sigma_f}
 =
 2.0-\frac{1.28(19)}{N}-\frac{4.78(90)}{N^2}
  \label{eqn_k2k1N}
\end{equation}
%
with a $\chi^2$ per degree of freedom of $\sim 0.5$ and an alternative best fit in powers of $1/N^2$
%
\begin{equation}
 \frac{\sigma_{k=2}}{\sigma_f}
 =
 2.0-\frac{14.43(60)}{N^2}+\frac{73.8(12.1)}{N^4}
  \label{eqn_k2k1NN}
\end{equation}
%
with a $\chi^2$ per degree of freedom of $\sim 2.2$. The former fit is clearly better, but the
latter cannot be entirely excluded. On the other hand while the fit in powers of $1/N$ is very good,
the coefficients are very different from those obtained if we expand the Casimir scaling
prediction, as in eqn(\ref{eqn_sigkCS}). One may speculate that for the range
of $N$ where the $k=2$ flux tube is a bound state the $N$ dependence of the
string tension is best described in powers of $1/N$ while once $N$ is large enough
that it has become a weakly interacting pair of fundamental flux tubes, the dependance
will be best described in powers of $1/N^2$. At which $N$ this transition occurs will depend 
on the length $l$ of the flux tube. If we use the formula in eqn(\ref{eqn_NG}) for the
fundamental and $k=2$ flux tubes and if we assume that $\sigma_{k=2}=2\sigma - O(1/N)$,
then one finds that the lightest state will be the weakly interacting pair of fundamental 
flux tubes for $l\leq l_c$ where $l_c\surd\sigma \propto N^{\frac{1}{2}}$. (This essentially 
reflects a competition between the leading linear terms and the $O(1/l)$ Luscher corrections.)
That is to say, for any fixed length $l$,  the (very) asymptotic leading
correction is $O(1/N^2)$ as expected by the usual large-$N$ counting. However one has to
be careful in which order one takes the $l\to\infty$ and $N\to\infty$ limits.




%
%
%
%
\section{Running coupling}
\label{section_coupling}

A question of theoretical interest is whether our calculations show that  we should keep the
\mbox{'t Hooft} coupling $g^2N$ fixed as $N\to \infty$ in order to have a smooth large-$N$ limit.
A question of phenomenological interest is whether we can estimate the scale $\Lambda$
of our $SU(N)$ gauge theories, from the calculated running of the gauge coupling, and
in particular whether we can do so for $SU(3)$. Finally something that is often useful in
lattice calculations is to have an interpolation function for $a(\beta)$. These are
the three issues we address in this section. A summary of the technical background to the
perturbative calculations has been placed in Appendix~\ref{section_appendix_couplings}.


%
%
\subsection{scaling with N}
\label{subsection_couplingN}


Since the lattice coupling $g^2_L$ defined in eqn(\ref{eqn_S}) provides a definition of the running coupling
on the scale $a$, and our above calculation of the (fundamental) string tension at various $\beta$ enables
us to express $a$ in physical units, i.e. as $a\surd\sigma$, we can use these calculations to address
some questions about the properties of the running coupling in $SU(N)$ gauge theories. Before doing so
we recall that this lattice coupling, corresponding to the particular coupling scheme defined by the lattice
and the plaquette action, is well-known to be a `poor' definition of a running coupling in the sense that
higher order corrections will typically be very large. This is indicated, for example, by the relationship
between the scale parameters $\Lambda_L$ and $\Lambda_{\overline{MS}}$ in this scheme and the
standard ${\overline{MS}}$ scheme 
%
\cite{Hasenfratz,Dashen-Gross}:
%
\begin{equation}
\frac{\Lambda_{\overline{MS}}}{\Lambda_L}
=
38.853 \exp\left\{-\frac{3\pi^2}{11N^2}\right\}.
\label{eqn_lamLlamMS}
\end{equation}
%
This is a long-standing issue that has led to the formulation of a number of improved couplings.
(For a review see
%
\cite{gimp_review}.)
%
Here we shall use the `mean-field' improved coupling of Parisi 
%
\cite{MF_Parisi},
%
\be
\frac{1}{g^2_I} = \frac{1}{g^2_L}
\langle \frac{1}{N}\mathrm{Tr}U_p \rangle,
\label{eqn_gI}
\ee
%
which has a nice physical motivation as the effective coupling experienced by a background field
(in a simple approximation). Denoting the corresponding scale by $\Lambda_I$, one finds
%
\be
\frac{\Lambda_{\overline{MS}}}{\Lambda_I}
=
\frac{\Lambda_L}{\Lambda_I}
\frac{\Lambda_{\overline{MS}}}{\Lambda_L}
=
\exp\left\{ -\frac{w_1}{2b_0} \right\}
\times
38.853 \exp\left\{-\frac{3\pi^2}{11N^2}\right\}
\simeq
2.633
\label{eqn_lamIlamMS}
\ee
%
using eqn(\ref{eqn_lamLlamMS}) and eqns(\ref{eqn_b0b1},\ref{eqn_plaq_pert}), with the
value of $\Lambda_L/\Lambda_I$ being obtained using eqn(\ref{eqn_aa0f}) for each $\Lambda$,
with $g^2_I$ and $g^2_L$ related by eqns(\ref{eqn_gIgL},\ref{eqn_plaq_pert}), and then
taking $g^2\to 0$. This already suggests that $g^2_I$ has the potential to be more-or-less
as good as $g^2_{\overline{MS}}$. One can also show
%
\cite{CAMTAT}
%
that this coupling tracks quite accurately the Schrodinger-functional coupling
%
\cite{SF1,SF2}
%
over a very wide range of scales. (See
%
\cite{CAMTAT}
%
for a detailed discussion.) So from now on, in this section, we shall use
$g^2_I(a)$ as our lattice running coupling.

The main question we address concerns the $N$ dependence of $g^2_I(a)$. The usual large $N$
counting tells us that we expect to approach constant physics as $N\to\infty$ if we
keep $g^2N$ fixed. For the running coupling this means that if we plot $g^2_I(a)N$
against the calculated values of $a\surd\sigma$ in our $SU(N)$ gauge theories, they
should approach a common envelope as $N\to\infty$. The interesting question is whether
the approach is slow or fast, indicating that our values of $N$ are `far from' or `near to'
$N=\infty$ respectively. In Fig.~\ref{fig_ggINK_suN} we plot our calculated values of the
running 't Hooft coupling $g^2_I(a)N$ against $a\surd\sigma$ for all our values of $N$.
The results are quite remarkable: even $SU(2)$ is very close to $SU(\infty)$ in this respect,
and it is only when the lattice spacing becomes large that appreciable differences appear.
%To put this into perspective, we show in Fig.~\ref{fig_ggIK_suN} the corresponding
%plot of $g^2_I(a)$ versus $a\surd\sigma$.
It is also interesting to see what happens if
one uses the poor $g^2_L(a)$ coupling instead of $g^2_I(a)$, and one finds that
%This is shown in Fig.~\ref{fig_ggNK_suN}:
while the convergence to a large $N$ limit is still evident,
the corrections at lower $N$ are substantial indicating that, not surprisingly, higher order
non-planar contributions are important using this coupling scheme. Finally we comment
that these conclusions are not unexpected or novel: similar analyses have appeared in
for example 
%
\cite{BLMT_N,CAMTAT},
%
albeit usually with less accuracy and over more limited ranges of $N$ and $a\surd\sigma$.



%
%
\subsection{perturbative running and ${\mathrm{\Lambda_{\overline{MS}}}}$}
\label{subsection_Lambda}



The second question we address is whether the running coupling dependence displayed in 
Fig.~\ref{fig_ggINK_suN} can be described by the usual perturbative $\beta$-function
for at least some $N$ once $a$ is small. If so we can extract a value of $\Lambda_I$
for each such $N$ and a corresponding value of $\Lambda_{\overline{MS}}$ using
eqn(\ref{eqn_lamIlamMS}). Our analysis will broadly follow that of
%
\cite{CAMTAT}
%
and we refer the reader to that paper for background and context; here we merely outline
the calculation, with more details in Appendix~\ref{section_appendix_couplings}. To fix
our notation we begin with the standard $\beta$-function for the lattice bare coupling $g_I^2$:
%
\be
\beta(g_I) = -\frac{\partial g^2_I}{\partial\log a^2}
=
- b_0  g^4_I - b_1  g^6_I - b^I_2  g^8_I + ...
+ O(a^2),
\label{eqn_bfunction}
\ee
%
where the scheme independent coefficients $b_0,b_1$ and the scheme dependent $b^I_2$
are given in eqn(\ref{eqn_b0b1}) and eqn(\ref{eqn_b2I}) of
Appendix~\ref{section_appendix_couplings}.
As shown in Appendix~\ref{section_appendix_couplings} this motivates the following
3-loop  expression for $a$:
%
\be
a \sqrt\sigma(a)
\stackrel{3 loop}{=} 
\frac{\sqrt\sigma(0)}{\Lambda_I}
\left( 1 + c_{\sigma} a^2\sigma \right) 
\left(b_0g_I^2(a)\right)^{-\frac{b_1}{2b^2_0}}
e^{-\frac{1}{2b_0g_I^2(a)}}
e^{-\frac{1}{2} \int^{g_I^2(a)}_0 dg^2
\left(\frac{b_0b^I_2-b^2_1-b_1b^I_2g^2}{b^3_0+b^2_0b_1g^2+b^2_0b^I_2g^4}
\right) }.
\label{eqn_agI3loop}
\ee
%
Here the first factor on the right after the coefficient is a non-perturbative tree-level lattice
spacing correction.
The motivation is that we could just as well use some physical mass, $\mu(a)$, as a scale in place of
$\surd\sigma(a)$ and since $\mu(a)/\surd\sigma(a)=\mu(a=0)/\surd\sigma(a=0)(1+O(a^2))$ it
must be the case that we have, in general, a factor $(1+O(a^2))$ multiplying the perturbative expression
%
\cite{CAMTAT,CAMTAT2}.
%
Here we choose to use $\sigma$ as our scale for $a^2$: using some other $\mu^2$ would only change
the $O(a^4)$ term and we will assume that $a$ is small enough that we can neglect any $O(a^4)$ terms.
We evaluate numerically the integral in the exponential for any given value of $g_I(a)$:
for such a smooth integrand any simple technique will be able to give accurate results.
We now describe the result of fitting this function to our values of $a\surd\sigma$, as listed in
Tables~\ref{table_Ksmall_SU2},\ref{table_Ksmall_SU3} for $SU(2)$ and $SU(3)$ and in
Tables~\ref{table_param_SU4}--\ref{table_param_SU12} for $SU(4)$ to $SU(12)$.
In performing these fits one can either use the measured values of $a^2\sigma$ in the
$(1+c_{\sigma}a^2\sigma)$ lattice correction factor, or the value calculated from the
formula itself. In the latter case eqn(\ref{eqn_agI3loop}) becomes a quadratic equation
for $a\surd\sigma$. Not surprisingly, for the range of $g_I^2(a)$ where  eqn(\ref{eqn_agI3loop})
provides a good fit to our calculated string tensions, the difference between the
two methods is insignificant, and we choose here to use the measured value in the correction term.

The results of our fits using eqn(\ref{eqn_agI3loop}) are shown in Table~\ref{table_Lambda_fitN}.
For each value of $N$ we fit our lattice values of $a\surd\sigma$ discarding the largest
values until we obtain a reasonably acceptable fit. The range of lattice spacings for the fits
is listed for each $N$, together with the $\chi^2$ per degree of freedom, $\chi^2/n_{df}$.
In the few cases labelled by $\ast$ in  Table~\ref{table_Lambda_fitN} the fits are very poor
with $\chi^2/n_{df} \geq 3$, and so we quadruple the stated
error in those cases in the hope that this encodes the increased uncertainty.
Referring to  Tables~\ref{table_param_SU2}--\ref{table_param_SU12}
we see that the number of values used within a given fit varies from 6 in $SU(3)$ to 4 in
$SU(10)$ and $SU(12)$. Since we have 2 parameters in our fit, 4 points is a very small number
for a fit. Moreover, for $SU(10)$ and $SU(12)$ the lattice spacings do not extend to values as small 
as for lower $N$. Hence one should treat the resulting fits with extra caution.

In Table~\ref{table_Lambda_fitN}
we also list the fitted values of the perturbative scale parameter, $\Lambda_I$, in units of the
continuum string tension. We then convert this to the corresponding $\Lambda_{\overline{MS}}$
scale in the widely used $\overline{MS}$ coupling scheme, using the $N$-independent relation
$\Lambda_{\overline{MS}}/\Lambda_I \simeq 2.633$ from eqn(\ref{eqn_lamIlamMS})
and the results are listed in the last column of Table~\ref{table_Lambda_fitN}, again
in units of the string tension. The quoted errors on our values of $\Lambda_I$ are very small,
but they are purely statistical and their smallness reflects the very small errors on the
string tensions that are fitted. The systematic errors involved in, for example, the
truncation of the beta-function may be much larger. To get some measure of this
error we also perform an extra fit using the exact 2-loop running,
%
\be
a \sqrt\sigma(a) \stackrel{2 loop}{=}  c_I \left( 1 + c_{\sigma,I} a^2\sigma \right)
e^{-\frac{1}{2b_0 g_I^2}}
\left(\frac{b_1}{b_0^2}+\frac{1}{b_0 g_I^2}
\right)^{\frac{b_1}{2b_0^2}}.
\label{eqn_agI2loop}
\ee
%
The resulting 2-loop values of $\Lambda_I$ are also listed in  Table~\ref{table_Lambda_fitN}.
We see that $\Lambda^{2loop}_I$ is typically about $10\%$ smaller than the $\Lambda^{3loop}_I$.
We have decided to use half of this difference as a measure of the error associated
with dropping 4-loop and higher coefficients from the 3-loop calculation. This is
added as a second error, within square brackets, to the resulting 3-loop values for 
$\Lambda_{\overline{MS}}$ that are listed in the last right-hand column of
Table~\ref{table_Lambda_fitN}. Unlike the statistical errors, this error will affect
the results for all values of $N$ in the same direction. 

In Fig.~\ref{fig_LamMS_N} we display the values that we obtain from our 3-loop fits for
$\Lambda_{\overline{MS}}/\surd\sigma$ as a function of $N$. (We use only the statistical
errors in these fits.) It is clear that the variation
with $N$ is very weak. If we fit all our values for $N \leq 12$ we find the fit,
%
\be
\frac{\Lambda_{\overline{MS}}}{\surd\sigma}
  = 
  0.5055(7)[250] + \frac{0.306(12)}{N^2}, \quad \chi^2/n_{df}=2.70, \qquad N\in[2,12],  
\label{eqn_LamMS_N}
\ee
%
which is displayed in Fig.~\ref{fig_LamMS_N}. Although the $\chi^2/n_{df}$ is not good,
the calculated values appear to be scattered around the fit in a random pattern.
Other possible fits include
\beq
\frac{\Lambda_{\overline{MS}}}{\surd\sigma}
  & = &
  0.5067(11)[250] + \frac{0.258(12)}{N^2}, \quad \chi^2/n_{df}=1.65, \qquad N\in[4,12], \nonumber \\
\frac{\Lambda_{\overline{MS}}}{\surd\sigma}
  & = &
  0.5060(8)[250] + \frac{0.303(13)}{N^2}, \quad \chi^2/n_{df}=3.81, \qquad N\in[2,8], \nonumber \\
\frac{\Lambda_{\overline{MS}}}{\surd\sigma}
  & = &
  0.5093(16)[250] + \frac{0.206(38)}{N^2}, \quad \chi^2/n_{df}=0.44, \qquad N\in[4,8].
\label{eqn_LamMS_N_B}
\eeq
%
In all the above fits we have included within square brackets an estimate of $\sim 5\%$ for
the systematic error associated with the truncation of the perturbative expansion. We note
that this systematic error is much larger than the differences between the various fits above,
and is the dominant source of uncertainty.

To finish we turn briefly to the case of $SU(3)$ where the value of  $\Lambda_{\overline{MS}}$
has phenomenological interest. Here we can attempt to transform our value into $\mathrm{MeV}$
units as follows. We begin with its value in units of the string tension, as listed in
Table~\ref{table_Lambda_fitN}, and we then transform this into a value in terms of the Sommer
length scale $r_0$
%
\cite{Sommer-r0}
%
using a recent calculation
%
\cite{AAMT-2020}
%
that gives $r_0\surd\sigma = 1.160(6)$. A recent review
%
\cite{Sommer-r0b}
%
of calculations in lattice QCD with light quarks concludes that
$r_0=0.472(5)\mathrm{fm}=1/418(5) \mathrm{MeV}^{-1}$.
The usual expectation is that the value of $r_0$ is not very sensitive to the inclusion of light
quarks, so we use this $\mathrm{MeV}$ value in the pure gauge theory. (This is of course the
arguable step.) Doing so we arrive at
%
\be
\left.\frac{\Lambda_{\overline{MS}}}{\surd\sigma}\right|_{SU3}
= 0.5424(13)[185]
\Longrightarrow
r_0 \Lambda_{\overline{MS}} = 0.629(4)[22]
\Longrightarrow
\Lambda_{\overline{MS}} \stackrel{SU3}{=} 263(4)[9] \mathrm{MeV},
\label{eqn_LamMS_SU3}
\ee
%
where the first error is statistical and the second is systematic.
This value is consistent with the values recently obtained in the dedicated calculations
%
\cite{LamMS_SU3a,LamMS_SU3b}
%
that use very different methods, and our value has a similar accuracy. This adds
confidence that our calculations of $\Lambda_{\overline{MS}}$ for the other $SU(N)$
groups are also reliable.




%
%
\subsection{interpolating and extrapolating functions for $a(\beta)$}
%\subsection{interpolating functions for $a(\beta)$}
\label{subsection_interpol}

It can often be useful to know the value of $a(\beta)$, in physical  units, at some value
of $\beta$. This can be provided, for example, by the value of $a\surd\sigma$. However
calculations of $a\surd\sigma$ are obtained at a number of discrete values of $\beta$
within some finite range $\beta_0\leq\beta\leq\beta_1$ and the $\beta$ value of interest
may lie outside this range, or may be within this range but not at one of the discrete
values where  $a\surd\sigma$ has been calculated. In the latter case one needs to find an
interpolating function that will work in the range $\beta \in [\beta_0,\beta_1]$ and since
$a\surd\sigma$ typically varies smoothly in this range
%(as in Fig.\ref{fig_ggIK_suN})
this is easy to do: a few sensibly chosen terms from almost any complete set of functions
will work adequately. However if the $\beta$ value of interest lies outside the
range $[\beta_0,\beta_1]$ and, in particular, if it is at some weaker coupling, then
extrapolating such a `random' interpolating fit will invariably work badly
unless the original interpolating function has a form motivated by weak coupling
perturbation theory, in which case it should (in principle) be reasonably accurate.
In the section above we have used precisely such functions. We will here
present a fit in an explicit form so that the reader can readily employ it.

Our fit will be in terms of the mean-field improved coupling $g^2_I$ defined
in eqn(\ref{eqn_gI}). This requires that one know the value of the average plaquette;
this is always calculated in a Monte Carlo but one needs to record the
average, which is normally done so as to provide a first check on the calculation.
(And if not, one can obtain a value with adequate accuracy very quickly
using small lattices and modest statistics.)
The interpolating function we will use is a variation on the ones used above
and is as follows:
%
\be
a \sqrt\sigma(a) =  c_0 \left( 1 + c_{\sigma} a^2\sigma(a) \right) F_{2l+}(g_I),
\label{eqn_agI1}
\ee
%
where
%
\be
F_{2l+}(g_I)
=
e^{-\frac{1}{2b_0 g_I^2}}
\left(\frac{b_1}{b_0^2}+\frac{1}{b_0 g_I^2}
\right)^{\frac{b_1}{2b_0^2}} 
e^{-\frac{b^I_2}{2b_0^2}g_I^2}.
\label{eqn_agI2}
\ee
%
The first two factors of $F_{2l+}(g_I)$ constitute the exact dependence when $\beta(g)$ is
truncated to the first two terms. The last factor on the right is the extra dependence
if one keeps in the exponent the leading $O(b^I_2)$ term to $O(g^2_I)$.
That is to say, it is `more' than 2 loops, but `less' than 3 loops. Hence the subscript
$2l+$ on $F(g)$. This is of course an arbitrary truncation with no guarantee that it
does indeed do better than the 2-loop one; however we use it because, as we shall see
below, it turns out to be very close to the 3-loop result, but without the need to
perform any numerical integrations. Note that here we choose to use in the $\propto a^2\sigma$
correction term the value given by fitting the formula to our data since this is the
mode in which it needs to be used in an extrapolation.

To make use of eqn(\ref{eqn_agI2}) we need to fit the constants $c_0$ and $c_{\sigma}$ to our
calculated values of $a\surd\sigma$, at each $N$. Once the constants $c_0$ and $c_{\sigma}$
have been fitted we can solve eqn(\ref{eqn_agI1}), which is quadratic in $a\surd\sigma$, at
any value of $\beta_I$:
%
\be
a \sqrt\sigma(a) =  \frac{1}{2c_0c_{\sigma}F_{2l+}(\beta_I)}
  \left(1-\left[1 - 4c^2_0c_{\sigma}F_{2l+}(\beta_I)^2\right]\right).
\label{eqn_aKgI}
\ee
%
In  eqn(\ref{eqn_agI2}) the values of $b_0,b_1,b^I_2$ need to be specified. The values of
$b_0$ and $b_1$ are  universal and are given in eqn(\ref{eqn_b0b1}). The value of $b^I_2$
is as given in eqn(\ref{eqn_b2I}) 
%
\be
b^I_2=b^L_2+w_2b_0-w_1b_1,
\label{eqn_b2Ib}
\ee
where we use eqns(\ref{eqn_b2MS},\ref{eqn_b2Lb}) in eqn(\ref{eqn_b2La}) to obtain the
explicit expression for $b^L_2$ and then inserting that together with the functions in
eqn(\ref{eqn_plaq_pert}) and eqn(\ref{eqn_b0b1}) into eqn(\ref{eqn_b2Ib}) we
obtain the explicit expression for $b^I_2$ for any $N$. 

It only remains now to give our fitted values of $c_0$ and $c_{\sigma}$ for each of the
$SU(N)$ groups for which we have calculated the string tension. This we do in
Table~\ref{table_interp_gI}. There we show for each $N$ our best fits to these
two parameters as well as the fitted range and the $\chi^2$ per degree of freedom
of the fit. In performing these fits we systematically drop the largest values of
$a\surd\sigma$ from the fit until we obtain an acceptable $\chi^2$. This is
appropriate since our interpolating function is based on weak-coupling
perturbation theory. Note that being based on weak coupling the resulting function
is not designed to work for values of $a$ or $g^2$ greater than the range
within which it provides a good fit. 


As an aside it needs to be emphasised that all the above fits are only relevant
if one uses the Wilson plaquette action.
As a second aside, we note that if one wants $a(\beta)$ in terms of some physical
scale $\mu$ instead of $\surd\sigma$ then one can straighforwardly modify the
above formula for $a\surd\sigma$ to one for $a\mu$ by using the relation
$\mu/\surd\sigma = d_0 + d_1 a^2\sigma$ if it accurately holds in the
range of couplings of interest.

%
%
%
\section{Glueball masses}
\label{section_glueballs} 

%
%
\subsection{quantum numbers}
\label{subsection_quantumG}

The glueballs are colour singlets and so our glueball operator is obtained by taking the
ordered product of $SU(N)$ link matrices around a contractible loop and then taking the trace.
To retain the exact positivity of the correlators we use loops that contain only spatial links.
The real part of the trace projects on $C=+$ and the imaginary part on $C=-$.
We sum all spatial translations of the loop so as to obtain an operator with momentum $p=0$.
We take all rotations of the loop and construct the linear combinations that transform
according to the irreducible representations, $R$, of the rotational symmetry group  of our cubic
spatial lattice. We always choose to use a cubic lattice volume that respects these symmetries.
For each loop we also construct its parity inverse so that taking linear combinations
we can construct operators of both parities, $P=\pm$. The correlators of such operators will
project onto glueballs with $p=0$ and the $R^{PC}$ quantum numbers of the operators concerned.

The irreducible representations $R$ of our subgroup of the full rotation group are usually
labelled as $A_1,A_2,E,T_1,T_2$. The $A_1$ is a singlet and rotationally symmetric, so it
will contain the $J=0$ state in the continuum limit. The $A_2$ is also a singlet, while the
$E$ is a doublet and $T_1$ and $T_2$ are both triplets. In Section~\ref{subsection_spins} we
will outline the detailed relationship between these representations and the continuum spin $J$.
Since, for example, the three states transforming as the triplet of $T_2$ are degenerate on
the lattice, we average their values and treat them as one state in our tables of glueball
masses and we do the same with the $T_1$ triplets and the $E$ doublets. (Just as we would
treat the 5 states of a continuum $J=2$ glueball as one entry.)

%
%
\subsection{finite volume effects}
\label{subsection_massV} 

For reasons of computational economy we wish to calculate on lattice sizes that are small 
but, at the same time, large enough that any finite volume corrections remain smaller than
our typical statistical errors. Since the computional cost of calculating in $SU(N)$
gauge theories grows roughly $\propto N^3$ (the multiplication of two $N \times N$
matrices) and since finite volume corrections are expected to decrease as powers of $1/N$,
we reduce the size in physical units of our lattices as we increase $N$, as shown
in Tables~\ref{table_param_SU2}-\ref{table_param_SU12}. 

There are two important types of finite volume corrections. The first can be thought of as arising
when the propagating glueball emits a virtual glueball which propagates around the spatial
torus. The resulting shift in the mass of the propagating glueball decreases exponentially
in $m_Gl$ where $m_G$ is the mass gap and $l$ is the length of the spatial torus
%
\cite{Luscher-V}.
%
As we see from Tables~\ref{table_param_SU2}-\ref{table_param_SU12} the value of $am_G \times l/a$
is quite large in all of our calculations, so we can expect this correction to be small.

The second type of finite volume correction consists of states composed of multiple flux
tubes winding around a spatial torus in a (centre) singlet state. The lightest of these will be
a state composed of one winding flux tube together with a conjugate winding flux tube,
which we refer to as a `ditorelon'. (A single winding flux tube is usually
referred to as a `torelon'.) Since it can have a non-zero overlap onto the contractible
loops that we use as our glueball operators, it can appear as a state in our calculated
glueball spectrum. Neglecting interactions, the lightest ditorelon will consist
of each flux tube in its ground state with zero momentum and will have an energy, $E_d$, that
is twice that of the flux tube ground state, $E_d=2E_f$. Interactions will shift the energy
but this shift should be small on our volumes so we shall use $E_d\simeq 2E_f$ as a rough guide
in searching for these states. This ground state ditorelon has simple rotational properties
and only contribute to the $A_1^{++}$ and $E^{++}$ representations. If we allow one or
both of the component flux tubes to be excited and/or to have non-zero equal and opposite
transverse momenta we can populate other representations and produce towers of states.
However these excited ditorelon states will be considerably heavier on the lattice volumes
we employ and so we will not consider them any further in this paper, although they certainly
warrant further study.

The first of the above corrections leads to small shifts in the masses of the glueballs.
The second leads to extra states in the glueball spectrum. The signature of such an extra
ditorelon state is that its mass grows roughly linearly with the lattice size:
$aE_D \simeq 2aE_f \simeq 2 a^2\sigma_f L$ where $L$ is the relevant spatial size in lattice
units and  $\sigma_f$ is the (fundamental) string tension. So to test for finite volume effects
we perform calculations at the same value of $\beta$ on different lattice sizes and compare the
glueball spectra. To identify any ditorelon states we look for extra states in the $A_1^{++}$
and $E^{++}$ spectra whose masses increase roughly linearly with the volume. Since the mass
shift associated with the first kind of correction decreases exponentially with the lattice
size any shift on a significantly larger volume should be much smaller than on the smaller volume;
so to check that it is negligible compared to our statistical errors we simply compare the masses
of the states that are not ditorelons on the different volumes. As we see
from Tables~\ref{table_param_SU2}-\ref{table_param_SU12} the lattice sizes we use
fall into three groups: $l\surd\sigma \sim 4.0$ for $SU(2),SU(3),SU(4)$;
 $l\surd\sigma \sim 3.1$ for $SU(5),SU(6)$; $l\surd\sigma \sim 2.6$ for $SU(8),SU(10),SU(12)$.
We will exhibit a finite volume analysis for a representative of each of these three groups.

We begin with $SU(2)$. In Table~\ref{table_GvsV_SU2} we list the low-lying glueball spectra
that we obtain on 3 lattice sizes, together with the energies of the ground and
first excited states of the winding flux tubes. In physical units the $L=12,14,20$ spatial
lattice sizes correspond to $l\surd\sigma \simeq 2.9,3.4,4.8$ respectively and we recall
that the typical lattice size we use in $SU(2)$ is $l\surd\sigma \simeq 4$.
For the glueball states we list the
effective energy at $t=2a$ or, where applicable, the energy of the fit to the effective
energy plateau when that begins at $t=2a$. (These two measures differ very little in practice.)
These are obtained from the correlators of our variationally selected best operators.
For most of our lighter states the value of $aE_{eff}(t=2a)$ is very close to our best estimate
of the mass. For these finite volume comparisons we prefer this measure to the mass itself because
it serves to minimise the statistical errors and makes any finite volume corrections more visible.
Putting aside the $R^P=A_1^{+}$ and the $E^{+}$ spectra for the moment, we see that most of the glueball
energies on the $L=14$ and $L=20$ lattices are consistent within one standard deviation and all within
two standard deviations. The energies from the $L=12$ lattice are broadly consistent but there are
now some examples, such as the $A_2^+$ and $E^-$ ground states, where there appear to be significant
differences. We now return to the $A_1^{+}$ and the $E^{+}$ spectra listed in Table~\ref{table_GvsV_SU2}
to see if there is any evidence of the extra ditorelon states. We do indeed see these on the
$L=12$ and $L=14$ lattices in both the $A_1^{+}$ and the $E^{+}$ spectra: these states are displayed
in the Table with no corresponding entries at other lattice sizes. Their
effective energies increase with
$L$ and are just a little heavier than twice the flux tube energy. For the $L=20$ lattice
any ditorelon state would be much heavier than any of the states shown: at such energies the
spectrum is denser, the errors are larger, and so identifying an `extra' state becomes ambiguous.
The remaining states in the $A_1^+$ and the $E^{+}$ spectra are broadly consistent across the
lattice sizes, except for the lightest $E^+$ on $L=12$ and the first excited $A_1^+$ on $L=14$,
both of which are quite close to their respective ditorelons and possibly mixing with them.
Also the 4th state in the $L=12$ $A_1^+$ spectrum shows a shift. Since the typical lattice size we
use in $SU(2)$ is $l\surd\sigma \sim 4$ (except for the relatively unimportant calculations
at the largest values of $a(\beta)$) we can estimate the ditorelon states to have energies
$E_D \sim 2\sigma l \sim 2 (l\surd\sigma)\surd\sigma \sim 8\surd\sigma$ and we shall therefore
only perform continuum extrapolations of $A_1^{+}$ and $E^{+}$ states that are lighter than
$\sim 8\surd\sigma$, although it is still the case that the heaviest states in these channels
may be perturbed by ditorelon contributions. As for the other channels, since the size
$l\surd\sigma \sim 4$ falls between our $L=14$ and $L=20$ lattices
at $\beta=2.427$, we can conclude from the above that there should be no finite volume
corrections that would be visible outside our statistical errors.
We have performed similar finite volume checks in $SU(3)$ in our earlier paper
\cite{AAMT-2020}
and we refer the reader to that paper for details, in particular for a check of the $C=-$ states.
In the case of $SU(4)$ we have performed a finite volume analysis comparing the spectra on
$18^320$ and $22^4$ lattices at $\beta=11.02$, which has helped us identify the
positions of the ditorelon states in the spectra at other values of $\beta$, and to
remove these states from our spectra. With all these checks we have some confidence
that our $SU(2)$, $SU(3)$ and $SU(4)$ continuum spectra will not be afflicted by significant
finite volume corrections.


Our $SU(5)$ and $SU(6)$ calculations are on significantly smaller volumes, typically with
$l\surd\sigma \sim 3.1$, and therefore deserve a separate finite volume study to the one above.
%In Table~\ref{table_GvsV_SU5}
To do so we compare the low-lying glueball spectra that we obtain in $SU(5)$ on $14^320$ and $18^318$
lattices at $\beta=17.46$, together with the energies of the ground and first excited states of
the fundamental and $k=2$ winding flux tubes. In physical units these $L=14$ and $L=18$ spatial
lattice sizes correspond to $l\surd\sigma \simeq 3.06$ and $3.93$ respectively and we recall
that the typical lattice size we use in $SU(5)$ and $SU(6)$ is $l\surd\sigma \simeq 3.1$ i.e. close
to that of the smaller of our two lattices. As in $SU(2)$
we readily identify the extra ditorelon states in the $A_1^{++}$ and $E^{++}$ representations,
which have energies very close to twice that of the winding fundamental flux tube.
(The ditorelon on the $L=18$ lattice will lie outside our energy range.) They appear
to be well separated from neighbouring glueball states and so we can readily identify and
exclude them from our $SU(5)$ and $SU(6)$ glueball spectra and exclude them from our
continuum limits. The other energies
on our two lattice sizes are mostly within errors of each other, with a few out by up to two
standard deviations, and two states a little more than that. Given the large number of states
being compared, such rare discrepancies are inevitable and we can  claim that
we see no significant finite volume corrections to any of our listed states.

We now turn to our largest $N$ calculations where we use the even smaller spatial lattice size
$l\surd\sigma \simeq 2.6$. Our finite volume study is in $SU(12)$ on $12^320$ and $14^320$ lattices
at $\beta=103.03$, where the $12^320$ lattice represents our typical physical size, and the
energies are listed in Table~\ref{table_GvsV_SU12}. In contrast to the previous tables, we
list under the $A_1^{++}$ and $E^{++}$ representations our best estimate of the true glueball
spectrum with the ditorelons removed. We will return to the identification of the latter shortly.
We observe that most of the energies are the same within errors and all are within two standard
deviations. That is to say, any finite volume shifts in the energies are within our statistical
errors at these values of $N$, despite the fact that the $L=12$ lattice volume is quite small.

The situation with respect to the ditorelons is more complex. The first complication is that on
the  $12^320$ lattice the $A_1^{++}$ ditorelon is nearly degenerate with the first excited  $A_1^{++}$
glueball, so that these states may well mix even if the overlaps are small due to the large-$N$
suppression. Moreover the same occurs in the $E^{++}$ representation where the
ditorelon is nearly degenerate with the ground state $E^{++}$ glueball.
The second complication arises from the fact that the overlap of the ditorelon double trace
operators onto the single trace operators that we used to calculate the glueball spectrum
is suppressed since $N$ is large. This should
be an advantage and indeed if the overlap is small enough then the ditorelon will not appear
in the spectrum obtained from the single trace operators so that there is no issue.
However it may well appear as a minor component of
a state that at first sight is quite massive but then its $aE_{eff}(t)$ drops towards the
ditorelon energy as $t$ increases and the more massive components die away. This 
can lead to ambiguities and indeed does on both our lattice sizes.

Since all our $SU(12)$ glueball calculations, and also those in $SU(10)$ and $SU(8)$,
are on lattices of roughly the same physical size as our $12^320$ one at $\beta=103.03$, we need
to address these problems with the ditorelons. We do so as follows. First we find that
if in constructing our glueball operators we use blocked links whose extent is smaller than
the lattice size then the overlap of ditorelons onto these operators is small enough that
we do not see any ditorelon state in the $A_1^{++}$ spectrum but we do see it embedded as a
small component in more massive states in the $E^{++}$ spectrum. Since a link at blocking level
$bl$ joins sites that are $2^{bl-1}a$ apart, this means keeping to blocking levels $bl=1-4$
on our $l=12a$ and $l=14a$ lattices. If we now do something that may appear less reasonable and
include $bl=5$ blocked links, which join lattice sites $16a$ apart, so that the operators
formed out of these links wrap multiply around the spatial torus in all spatial directions,
we find that the ditorelon overlaps are much larger and the ditorelon states now appear quite
clearly in the resulting spectrum. The results of these various calculations
are shown in Table~\ref{table_GvsV_SU12B}. As usual the energies shown are $aE_{eff}(t=2a)$
and so are often slightly higher than the value of the effective energy plateau.
Consider first the $A_1^{++}$ spectra on the $12^320$ lattice. We see that the second and third
states in the $bl=1-5$ spectrum are nearly degenerate, with masses close to what one might
expect for the ditorelon, i.e. $2aE_f\sim 0.96$, but only one of them appears in the $bl=1-4$
spectrum. Thus we infer that one of those two (or a mixture of the two) is a ditorelon.
On the $14^320$ lattice we also see an extra state in the $bl=1-5$ spectrum, as compared
to the $bl=1-4$ spectrum, but now the state is heavier, as we would expect for a ditorelon 
since  $2aE_f\sim 1.2$ on this lattice. Turning to the $E^{++}$ representation on the $14^320$
lattice we find an extra state using the $bl=1-5$ basis, as compared to using the $bl=1-4$ basis,
with roughly the expected mass of a ditorelon.
On the  $12^320$ lattice things are a bit different: the second state in the $bl=1-5$ spectrum
has no obvious partner in the $bl=1-4$ spectrum, but there is a state in the latter spectrum
where $aE_{ff}(t)$ decreases rapidly with increasing $t$ to a comparable value. This is
the second entry in the table and the mass from the `plateau', indicated in square brackets,
is similar to the energy of the second state in the $bl=1-5$ spectrum. We
interpret this as follows: the $E^{++}$ ditorelon has a substantial overlap onto the $bl=1-5$
basis and only a small overlap onto the $bl=1-4$ basis, but large enough to appear within
what appears to be a higher excited state. This mass is somewhat smaller than that of the
state we identified as a ditorelon on the  $14^320$ lattice which is what one expects.
We note that unlike the states we identify as ditorelons, most of the other states
have nearly the same energies on the two lattice sizes, as one expects for states that
are not ditorelons. Recalling that the lattice sizes we use in our $N\geq 8$ calculations
are of the same physical size as our $12^320$ lattice at  $\beta=103.03$, we can use
the above results in all those cases to identify ditorelon states and remove them from
our glueball spectra. The same type of technique is useful for $N=5,6$.

A final comment is that the rotational symmetries of the ditorelon states
-- and indeed multitorelon states -- remain those of our finite volume, i.e. $\pi/2$ rotations,
even in the continuum limit. That is to say, even in that limit
they will fall into the representations of the cubic subgroup of the continuum rotation group.
The genuine glueballs, on the other hand, will fall into representations of the full continuum
rotation group in the continuum limit once the volume is large enough, up to corrections
that are exponentially small in the spatial size, and this difference can also be useful
in distinguishing ditorelons from genuine glueballs.


%
%
\subsection{lattice masses}
\label{subsection_latticemass} 

As described earlier, we calculate glueball masses from the correlators of suitable $p=0$ operators.
These operators are chosen to have the desired $R^{PC}$ quantum numbers, where
$R \in \{A_1,A_2,E,T_1,T_2\}$ labels the irreducible representation of the cubic subgroup of the
rotation group and $P=\pm$ and $C=\pm$ label parity and charge conjugation. We typically
start with a set of 12 different closed loops on the lattice. (For $N=2,3$ we used 27, but
we then observed that we can get almost equally good spectra by using a suitable subset of 12 loops.)
We calculate all 24 rotations of the loop and construct linear combinations of the traces that
tranform as $R$. We do so separately using the real and imaginary parts of the traces,
which gives us operators with $C=\pm$ respectively. We also calculate the parity inverses of
each of these twelve loops, and of their rotations, and by adding and subtracting appropriate
operators from these two sets we form operators for each $R$ with $P=\pm$. To ensure that
we have non-trivial operators for each set of quantum numbers it is useful to include
some loops that have no symmetry under rotations and parity inversion. With our particular
choice of 12 (or 27) closed loops we are able to construct the number of independent operators
shown in Table~\ref{table_numops_N}. This is the number of operators at each blocking level and 
we typically use 4 or 5 blocking levels in our calculations. This makes for quite a large 
basis of operators for all $R^{PC}$ quantum numbers, and indeed a very large number for most
of them. However this large number is slightly deceptive as at any given $\beta$ only two or perhaps
three blocking levels make an important contribution to the low-lying spectrum. In addition
the states in the $E$ representation are doubly degenerate and in the $T_1$ and $T_2$ are
triply degenerate, so one should divide the corresponding numbers in Table~\ref{table_numops_N}
by 2 or 3 in order to estimate the number of different energy levels accessible to
the basis.

For completeness we list our 12 basic loops in Table~\ref{table_loops}.
The links are labelled by $1,2,3$ to indicate their spatial directions, with
negative signs for backward going links, and the path ordering is from left to right.
So, for example, the plaquette in the $2-3$ plane would appear as $\{2,3,-2,-3\}$.
Our list includes the plaquette, three 6 link loops, four 8 link loops
and four 10 link loops.  We also show for each loop which representations
it contributes to once we include all the rotations and the inverses, doing so for the
real ($C=+$) and  imaginary ($C=-$) parts of the traces separately. This basis is
used for $N\geq 4$. For $SU(2)$ and $SU(3)$ we use a larger basis of 27 loops
that includes the 12 listed here. For some finite size studies we have used
a reduced set of 8 loops. 

We use these bases of operators in our variational calculations of the glueball spectra.
In each $R^{PC}$ sector we calculate a number of the lowest masses from the
correlation functions of the operators that our variational procedure selects as being
the best operators for those states within our basis. This relies on identifying an effective
mass plateau in the correlator, as described in Section~\ref{subsection_lattice_energies},
and performing an exponential fit to the data points on that plateau. So an important
question is: how reliably can we identify such a plateau? We will illustrate this
with our calculations in $SU(8)$ on the $20^330$ lattice at $\beta=47.75$. This
is at our smallest value of $a(\beta)$ which, being the closest to the continuum
limit, is one of the most important values in our subsequent extrapolations to the
continuum theory. Also it provides the best resolution in $t$ of our correlation
functions. We begin with the effective masses, $aE_{eff}(t=an_t)$, of the three
lightest $A_1^{++}$ states and of the two lightest $A_1^{-+}$ states, as shown in
Fig.\ref{fig_MeffA1_SU8}. As we will see below, the lightest two $A_1^{++}$ and
$A_1^{-+}$ states become the lightest two $J^{PC}=0^{++}$ and $0^{-+}$ glueballs
in the continuum limit. The third $A_1^{++}$ state is most likely part of the nonet of
spin states making up the $4^{++}$ glueball ground state (see below), although if not
then it would probably be the third scalar glueball. 
It is clear from Fig.\ref{fig_MeffA1_SU8}
that we have plausible effective mass plateaux for all the states, with the
weakest case being the excited $A_1^{-+}$ state. The solid lines show our
best mass estimates as obtained from fits to the correlation functions,
together with their error bands. The error band on the lightest  $A_1^{++}$
state, which is the mass gap of the gauge theory, is invisible on this plot,
so given the importance of this state we replot it in  Fig.\ref{fig_MeffA1b_SU8}
with an axis rescaling that exposes the errors. Here the solid line is the
best mass estimateand the dahsed lines bound the $\pm 1$ stadard deviation
error band. Fig.\ref{fig_MeffA1_SU8} illustrates
the obvious fact that as the masses get larger, the error to signal ratio
grows at any fixed value of $n_t$, so that at large enough masses the effective 
energies becomes too imprecise to unambiguously indicate the values of $n_t$
where the effective mass plateau begins. In this case we can turn to our
calculations of the excited $A_1^{-+}$ state at the smallest lattice spacings
in $SU(2)$ and $SU(3)$, where there is much better evidence for the effective
mass plateau beginning at $aM_{eff}(n_t=3a)$, and use that in estimating that
the effective mass plateau in this $SU(8)$ calculations begins at $n_t=3a$.
This is the type of argument we use for a number of the heavier states
at larger values of $N$ where our calculations do not extend to very small
values of $a(\beta)$.



As an aside, we also show in this
plot the effective mass plot for the ditorelon. It shows every sign of plateauing
to a value not far from $\sim 2aE_f\simeq 0.59$ as one would expect. It is clearly
important to make sure that one excludes such a state from the scalar glueball spectrum,
as we have done here, since it is located near the first excited glueball state
and would create a false level ordering if included.

We turn next to states which become $J=2$ glueballs in the continuum limit.
As we shall see below, the five components of a $J=2$ state are spread over
the two components of an $E$ energy level and the three of a $T_2$. As always
we average the masses of the `degenerate' doublets of $R=E$ and the triplets of
$R=T_1$ and $T_2$ to provide single effective masses in each case. What we show
in Fig.\ref{fig_MeffET2_SU8} are the resulting effective mass plots for a number
of such pairs of $E$ and $T_2$ states. The equality of the $E^{++}$ and $T_2^{++}$ masses
for the lightest two pairs is convincing, as it is for the lightest  $E^{-+}$
and $T_2^{-+}$. It is also plausible for the lightest $E^{--},T_2^{--}$
pair, and the lightest  $E^{+-},T_2^{+-}$ pair, and there is some convergence
for the second  $E^{-+},T_2^{-+}$ pair, even if these more massive states
do not show unambiguous plateaux. Again we show the ditorelon, this time in
the $E^{++}$ representation. In this spectrum it is nearly degenerate with
the  $E^{++}$  ground state. The fact that it does not appear in a nearly
degenerate $T_2^{++}$ state confirms that it is a finite volume state,
reflecting the limited rotational symmetries of the spatial volume.

%In Fig.\ref{fig_MeffT1_SU8} we show the effective mass plots for a number of
If we examine the effective mass plots for the $T_1^{PC}$ states that we shall
later argue approach $J=1$ in the continuum limit, we see that
while the lightest $T_1^{+-}$ glueball has a well-defined effective mass plateau,
this is less evident for the more massive states.
%Finally, in Fig.\ref{fig_MeffA1MM_SU8} we display
Finally we remark that if we plot the  effective masses of our heaviest states,
such as those of the lightest five $A_1^{--}$ states,
%as an example of the very heaviest states that we attempt to calculate.
the evidence for the effective mass plateaux is not strong, although one can
speculate on the presence of some the plateaux.
This will clearly result in substantial systematic errors on the corresponding mass
estimates, which we cannot readily quantify.

One lesson of the effective mass plots is that the heavier the state
the less reliable will be our error estimate. However these plots also tell us at what
$t=an_t$ the effective mass plateau begins for those states where this can be 
identified. Since the iterative blocking means that when  we vary $\beta$ our variationally
selected operators are of roughly constant physical size and shape, we can assume
that the overlap of a given state onto our basis will be roughly independent of $\beta$
and hence that the effective mass plateau will begin at a value of $t=a(\beta)n_t$ that
is roughly constant in physical units, i.e. at smaller $n_t$ as $a(\beta)$ grows.
So at larger values of $a(\beta)$ where $aE_{eff}(t)$ may be too large for us to identify 
a plateau, we can nonetheless use the value of $aE_{eff}(t_0)$, where $t_0$ is the
value where our above calculations tell us that a plateau begins, as an estimate of the mass.
This is something we do in our calculations, where appropriate.

We turn now to the glueball spectra that we obtain by the methods described above.
We obtain the spectra for the lightest glueballs in each $R^{PC}$ sector for
each of our gauge groups. As an example we display in \ref{table_Mlat_RPC_SU8}
our results for $SU(8)$.
%Tables~\ref{table_Mlat_R_SU2}-\ref{table_Mlat_RPC_SU12}.
Here we have removed the finite volume
ditorelon states in the $A_1^{++}$ and $E^{++}$ sectors whenever they are present, 
as discussed in Section~\ref{subsection_massV}, so that what
we present in the Table is our best estimate of the infinite volume glueball spectrum.
%We do not present the spectra for $SU(3)$ since these have been published separately in
A similar Table for $SU(3)$ has been published separately in
%
\cite{AAMT-2020}.
%
%(We have in fact calculated a more extensive spectrum than that shown, at the smaller values
%of $a(\beta)$ where that becomes possible.)
All this assumes of course that none of these states
is a multiglueball state. Since we use single trace operators, their overlap onto
multiglueball states should decrease with increasing $N$ and in any case they would be
quite heavy. Some explicit calculations that provide evidence that such mutliglueball
states do not appear in our spectra are described in Section~\ref{subsection_scattstates}.

The reader will note that in some cases what we list as a higher excited state has a lower mass
than what is listed as a lower excited state. This typically involves states that are
nearly degenerate. Our procedure is to order the states according to the values of their
effective masses at $t=a$ which is the value of $t$ at which our variational calculation
typically operates. This nearly always corresponds to final mass estimates (from the plateaux)
that are in the same order, but very occasionally this not the case.
In the very few cases where the difference is well outside the statistical errors
we invert the ordering, but otherwise we do not. In addition it may be that
the continuum extrapolation leads to a level inversion; again we only take that on board
if the difference is well outside the errors. We have a good number of states that are nearly
degenerate with a neighbouring state and in these cases the variational procedure itself is
likely to mix them, and if that is done differently at different values of $\beta$ it can
lead to poor extrapolations to the continuum limit.

Finally we remark that
the ensembles of lattice fields used at each $N$ and $\beta$ have a reasonable distribution
of topological charge, either because the tunnelling between topological charge sectors is
sufficiently frequent, or because we have imposed such a distribution on the initial lattice
fields of the collection of sequences that make up the total ensemble, as discussed in
Section~\ref{subsection_Qfreezing}.


%
%
\subsection{strong-to-weak coupling transition}
\label{subsection_bulk}

Before moving on to the continuum extrapolation of the lattice masses that we have calculated
above, we briefly remark on the `bulk' transition that interpolates in $\beta\propto 1/g^2$
between the strong and weak coupling regimes. From eqns(\ref{eqn_Z},\ref{eqn_S}) we see that the
naive expectation is that at strong coupling, $\beta\to 0$, the natural expansion of a
lattice quantity is in positive powers of $\beta$ while at weak coupling, $\beta\to \infty$, 
the natural expansion is in positive powers of $1/\beta$. On the weak coupling side
asymptotic freedom promotes the dependence of physical quantities to exponentials,
$\propto \exp\{-c\beta\}$, and it is important to make sure that our calculations
are indeed in the weak coupling regime.

As remarked in Section~\ref{subsection_lattice_setup}, for $SU(N\leq 4)$ the transition
is known to be a smooth crossover, while for $SU(5)$ it is weakly first order,
and for $SU(N\geq 6)$ it is strongly first order. For $SU(N\leq 4)$ the location
of the crossover coincides with a dip in the mass gap which then drives a peak in the
specific heat. (The specific heat is proportional to the sum of the plaquette-plaquette correlator
and so peaks where the correlation length has a peak, i.e. where the mass gap has a dip.)
For $SU(2)$ this peak is around $\beta\sim 2.15$ as we see, for example, in Fig.8 of
%
\cite{KIGSMT-SU2-1983},
%
while for $SU(3)$ the peak is around $\beta\sim 5.4$ as we see, for example, in Fig.4a of
%
\cite{KIGSMT-SU3-1983}.
%
For $SU(4)$ the dip in the mass gap can be seen in Fig.1 of
%
\cite{BLMT_N}
%
and is located  around $\beta\sim 10.45$. For larger $N$ the transition has been shown in
%
\cite{BLMTUW_Tc}
%
to be first order, and strongly first order for $N\geq 6$.
In that case we have a strong hysteresis, i.e. if we lower $\beta$ slowly from large values 
then the transition occurs at $\beta=\beta_b^{\downarrow}$, while if we increase  $\beta$ slowly
from small values then the transition occurs at $\beta=\beta_b^{\uparrow}$, with 
$\beta_b^{\uparrow}$ significantly larger than $\beta_b^{\downarrow}$.
Values for these transitions can be found listed in Table 16 of 
%
\cite{BLMTUW_Tc}.
%
As described in that paper, calculations performed in the weak coupling false vacuum just
above $\beta=\beta_b^{\downarrow}$, and well below $\beta=\beta_b^{\uparrow}$, show no sign of
being affected by the simultaneous presence of the deeper true vacuum elsewhere
in field space. This fact has been exploited in previous large $N$ calculations.
For example all but one of the $\beta$ values in the $SU(8)$ calculations in
%
\cite{BLMTUW_N}
%
lie within this hysteresis window, as does the $SU(8)$ calculation in
%
\cite{BLARER_N}.
%
As is clear from Tables~\ref{table_param_SU2}-\ref{table_param_SU12}
all our calculations have been performed on the weak coupling side
of the cross-over or transition. In addition we note that
none of our $SU(5)$ or  $SU(6)$
calculations fall within the hysteresis window, only one of our $SU(8)$
calculations lies within this window, while some of the $SU(10)$ and about
half of the $SU(12)$ calculations lie within the window. 


As remarked above, the main effect of the crossover at smaller $N$ is an anomalous
dip in the value of the mass gap, i.e. that of the lightest scalar glueball $am_{0^{++}}$.
This dip is caused by a nearby critical point which lies at the end of a first-order
transition line in an extended fundamental and adjoint coupling plane,
at which critical point the scalar glueball mass vanishes. For a detailed
calculation in $SU(3)$ see for example
%
\cite{Heller_bulk}.
%
(The fundamental coupling is the one that which appears in our path integral.)
As $N$ increases the first-order line intersects the fundamental axis and provides the
first-order bulk transition separating weak and strong coupling.
However until the critical point has moved far away from the fundamental
axis there may still be a dip in the mass gap near the first-order transition.
Indeed one can see some plausible evidence for this occuring in the behaviour of
the $SU(6)$ and $SU(8)$ mass gaps listed in 
%
\cite{BLMTUW_N}.
%
Since there is no theoretical reason to expect that this dip can be encoded in
the weak coupling expansion of the lattice action in powers of $a^2$, we minimise
the risk to the continuum extrapolation of the scalar glueball mass by simply excluding
from the continuum fit the value obtained at the largest value of $\beta$ when
including this value would require adding an additional $a^4$ correction to the fit.
As it turns out, we need to do this for all our $SU(N)$ groups except for $SU(2)$.


%
%
\subsection{continuum masses}
\label{subsection_massratios} 

For each of our $SU(N)$ lattice gauge theories  we now have the low-lying glueball spectra
for a range of values of $a(\beta)$. These are all given in lattice units as $aM$, and to
transform that to physical units we can take the ratio to the string tension, $a\surd\sigma$,
that we have simultaneously calculated. We can then extrapolate this ratio to the
continuum limit using the standard Symanzik effective action analysis
%
\cite{Symanzik_cont}
%
that tells us that for our lattice action the leading correction at tree-level is $O(a^2)$:
%
\be
\frac{aM(a)}{a\surd\sigma(a)} = \frac{M(a)}{\surd\sigma(a)}
=
\frac{M(0)}{\surd\sigma(0)} + a^2\sigma(a) + O(a^4).
\label{eqn_MKcont}
\ee
%
Here we have used the calculated string tension, $a^2\sigma(a)$, as the $O(a^2)$
correction. Clearly we could use any other calculated energy, and this would
differ at $O(a^4)$ in eqn(\ref{eqn_MKcont}). It is convenient to use $a^2\sigma(a)$
since its calculated value has very small errors.

The results of these continuum extrapolations are listed in
Tables~\ref{table_MK_R_SU2}-\ref{table_MK_R_SU12} for the gauge groups ranging
from $SU2)$ to $SU(12)$. (As always, each $E$ doublet, and each $T_2$ or $T_1$
triplet appears as a single state in our tables and discussions.)
A few entries are accompanied by a star denoting a
poor fit, $2.5 < \chi^2/n_{df} < 3.5$, or a double star denoting a very poor fit,
$\chi^2/n_{df} \geq 3.5$. For $SU(10)$ and $SU(12)$ we have only 5 values of $\beta$
and since the masses at the coarsest value of $\beta$ often have to be discarded
from the fit (not surprisingly) and since we are fitting 2 parameters,
we often have only 2 degrees of freedom. For lower $N$ we usually
have 3 and sometimes 4 degrees of freedom, which gives much more confidence in
the extrapolations. All of which is to say that while the fits  for larger $N$ are
certainly not trivial, it would be good to be able to do better in future calculations.

We now illustrate the quality of these linear continuum extrapolations for our most interesting
glueball states. We do so for $SU(4)$. In Fig.\ref{fig_MJ02ppK_cont_SU4} we show our
extrapolations of the lightest two $A_1^{++}$, $E^{++}$ and  $T_2^{++}$ states.
These states are of particular importance because, as we shall see latter on, they correspond
to the lightest two $J^{PC}=0^{++}$ and $2^{++}$ states. We see that all the linear fits
are convincing, even if in some cases we have to exclude from the fit the value at the
largest $a(\beta)$. In Fig.\ref{fig_MJ02mpK_cont_SU4} we show the corresponding plot for $P=-$
which, as we shall see latter on, correspond to the lightest two $J^{PC}=0^{-+}$ and $2^{-+}$
states. The lightest states have very plausible continuum extrapolations, although
the excited states, which are heavier than those for $P=+$, begin to show a
large scatter indicating a poor fit. In Fig.\ref{fig_MJ1K_cont_SU4} we show the
extrapolations of various $T_1^{PC}$ states that we shall later argue correspond to $J=1$,
and again we see fits that appear convincing for the lighter states and quite
plausible for the heavier states. 
%Finally, we show in Fig.\ref{fig_MJ2J3K_cont_SU4}
Even for the heaviest states to which we can plausibly assign a continuum spin, such as the
states that pair up to give the  $J^{PC}=2^{--}$ and  $2^{+-}$ ground states and the states
that provide the seven components of the $3^{+-}$ ground  state, where the
larger errors lead to a greater scatter of points, continuum fits linear in $a^2\sigma$
are still plausible.




%
%
\subsection{$N\to\infty$ extrapolation}
\label{subsection_masslargeN} 

Amongst the various $SU(N)$ calculations, the one that is most interesting
from a phenomenological point of view is the $SU(3)$ one, and that is why we
devoted a separate paper to that case
%
\cite{AAMT-2020}.
%
From a theoretical point of view however the most interesting glueball spectra
are those of the $SU(N\to\infty)$ theory since the theoretical simplifications
in that limit make it the most likely case to be accessible to analytic
solution, whether complete or partial.

To obtain the $N=\infty$ spectrum from our results so far, we use the fact that
in the pure gauge theory the leading correction is $O(1/N^2)$. So we can extrapolate
the continuum mass ratios in Tables~\ref{table_MK_R_SU2}-\ref{table_MK_R_SU12}
using
%
\be
\left.\frac{M_i}{\surd\sigma}\right|_{N}
=
\left.\frac{M_i}{\surd\sigma}\right|_{\infty}
+ \frac{c_i}{N^2} + O\left(\frac{1}{N^4}\right).
\label{eqn_MKN}
\ee
%
The results of these extrapolations are presented in Table~\ref{table_MK_R_SUN}.
In this table the stars point to poor fits exactly as described earlier for the continuum fits.
The very poor fit for the first excited $A_1^{++}$ is due to a large mismatch between
the $SU(5)$ and $SU(6)$ mass estimates which may well be due to an inadequate treatment
of the ditorelon influence. (We can obtain a good fit to $N\geq 6$ and this
gives a mass $\sim 5.85(9)$ which  is the same within errors.)
Most of the fits are to $N\geq 2$ or $N\geq 3$ but some are over a more restricted range of
$N$ and this is indicated by a dagger.These states are the  $A_2^{++}$ ground state, fitted to $N\geq 4$,
the $T_2^{-+}$ second excited  state, also fitted to $N\geq 4$, the $T_2^{--}$ ground  state,
again fitted to $N\geq 4$, and finally the $A_2^{+-}$ ground state which is more arguable
since it was fitted to $N\leq 8$. From the practical point of view the most important
extrapolations are for those states to which we are able to assign a continuum spin
(in the next section). We therefore show these extrapolations in
Figs.\ref{fig_M0pp0mpK_N},\ref{fig_MJ2PCK_N},\ref{fig_MJ1K_N}
%,\ref{fig_MJ3K_N}
%for states with $J=0,2,1,3$ respectively.
for states with $J=0,2,1$ respectively.





%
%
\subsection{continuum spins}
\label{subsection_spins} 

So far we have used the representations of the rotational symmetry of our cubic spatial
lattice to label our glueball states. However as we approach the continuum limit these
states will approach the continuum glueball states and these belong to representations of
the continuum rotational symmetry, i.e. they fall into degenerate multiplets of $2J+1$ states
where $J$ is the spin. In determining the continuum limit of the low-lying glueball spectrum,
it is clearly more useful to be able to assign the states to a given spin $J$, rather
than to the representations of the cubic subgroup which have a much less fine `resolution'
since all of $J=0,1,2,...,\infty$ are mapped to just 5 cubic representations. 
How the $2J+1$ states for a given $J$ are distributed amongst the representations of the
cubic symmetry subgroup is given, for the relevant low values of $J$, in Table~\ref{table_J_R}.
So, for example, we see that the seven states
corresponding to a $J=3$ glueball will be distributed over a singlet $A_2$, a degenerate
triplet $T_1$ and a degenerate triplet $T_2$, so seven states in total. These $A_2$, $T_1$ and $T_2$
states will be split by $O(a^2)$ lattice spacing corrections, generated by irrelevant operators.
So once $a$ is small enough these states will be nearly degenerate and one can use this
near-degeneracy to identify the continuum spin. 

This strategy works for the lightest states but becomes rapidly unrealistic for heavier
states. The latter will have larger statistical errors and will fall amongst other states
that become more densely packed as the energy increases, so that identifying apparent
near-degeneracies between states in different representations becomes highly ambiguous.
There exist more sytematic ways of assigning continuum spin to lattice states, such as
%
\cite{HMMT-2004,HM_Thesis,PCSDMT-2019},
%
but these are beyond the scope of this work. Accordingly we shall limit ourselves to the
lightest states where any ambiguity in identifying near-degeneracies is either small
or non-existent.

The states whose spin $J$  we feel confident in identifying are listed in
Table~\ref{table_M_J_R} and the resulting continuum masses are listed in
Table~\ref{table_MKJ_N2-5} and Table~\ref{table_MKJ_N6-12}.
We will now briefly illustrate the argument for the assignments,
taking $SU(8)$ as a typical example. Consider the masses in the $SU(8)$ column of
Table~\ref{table_MKJ_N6-12}. We have obtained these by interpreting the masses listed in
Table~\ref{table_MK_R_SU8} as follows. The ground state of $R^{PC}=A_1^{++}$ is much lighter than
any other mass and so must be the singlet $J^{PC}=0^{++}$ ground state. The first excited
$A_1^{++}$ state certainly has no `nearly-degenerate' partners amongst all the other $P,C=+,+$
states and so it must be the first excited $0^{++}$ glueball. The ground states of the
doubly degenerate $E^{++}$ and the triply degenerate $T_2^{++}$ states are nearly degenerate
and there are no other states with similar masses, so together they must provide the
five components of the $J^{PC}=2^{++}$ ground state. Similarly the first excited
$E^{++}$ doublet and $T_2^{++}$ triplet have no nearly degenerate companion states elsewhere,
and so they provide the first excited $J^{PC}=2^{++}$ state. The argument for the
ground and first excited $0^{-+}$ states is equally straightforward, as it is for
the $E^{-+}$ and the $T_2^{-+}$ ground states forming the $2^{-+}$ ground state.
The first excited $E^{-+}$ and the $T_2^{-+}$ states are consistent with forming the
first excited  $2^{-+}$ state, but here there is some ambiguity: it is also possible that
these states pair with the second excited $A_1^{-+}$ state and the $T_1^{-+}$ ground state
to make up the  $4^{-+}$ ground state. This ambiguity arises because the errors are
sufficiently large that both possibilities can be entertained. Here we can resolve the
ambiguity by observing that for neighbouring values of $N$, $SU(6)$ and  $SU(10)$,
the ambiguity is not present and the first excited $E^{-+}$ and the $T_2^{-+}$ states
form the first excited  $2^{-+}$ state, strongly suggesting that this is also the
case for $SU(8)$. This choice is not as solid as the earlier choices and fortunately
it is not an argument we need to make very often in coming to our choices of $J$.
With this choice we can plausibly assign the lightest $T_1^{-+}$ state
to the $1^{-+}$ ground state. (The second excited $A_1^{-+}$ state has a similar mass
but there are no further partners allowing us to assign it elsewhere according
to Table~\ref{table_J_R}.) Similar arguments to the above lead to the $J^{+-}$
and $J^{--}$ choices in  Table~\ref{table_M_J_R} and the corresponding masses
in Table~\ref{table_MKJ_N6-12}.
Determining the spin $J$ in the above way, clearly requires us not to miss any
intermediate states in the mass range of interest. For this one needs a
good enough overlap onto all the low-lying states and for that one needs a large
basis of operators, which has been our goal in these calculations. 

Using the arguments sketched above we infer the $J^{PC}$ glueball masses
listed in Tables~\ref{table_MKJ_N2-5},\ref{table_MKJ_N6-12}, from the
various masses listed in Tables~\ref{table_MK_R_SU2}-\ref{table_MK_R_SU12}.
In addition by inspecting the remaining states in
Tables~\ref{table_MK_R_SU2}-\ref{table_MK_R_SU12} we can infer some
lower bounds on the ground states in some other channels, and these
are listed in Table~\ref{table_MKJ_lowbound}. These estimates are
fairly rough and at that level apply to all our $SU(N)$ theories.


The most interesting glueball spectra are those for $SU(3)$ and for $SU(\infty)$:
the former for its potential phenomenological relevance and the latter for
its potential theoretical accessibility. 
The $SU(3)$ theory is sufficiently close to the real world of $QCD$ that it makes
sense to attempt to present the masses in physical units, using scales that
one believes to be relatively insensitive to the presence of quarks. We refer the
reader to
\cite{AAMT-2020}
for a more detailed discussion. Here we simply recall from our earlier discussion
that the Sommer scale $r_0$,
which is defined by the heavy quark potential at intermediate distances, is
believed to vary weakly with quark masses, with a physical value of
$r_0 \simeq 0.472(5)\mathrm{fm}$
%
\cite{Sommer-r0b}.
%
One can extract from the published data the value $r_0\surd\sigma=1.160(6)$
%
\cite{AAMT-2020}
%
This allows us to translate masses in units of $\surd\sigma$ to units of $r_0$
and finally to units of $\mathrm{GeV}$. Doing so we obtain the masses listed in
Table~\ref{table_MJ_N3}. This Table is similar to that in
%
\cite{AAMT-2020}
%
except that we have included lower bounds for most of the masses where
we previously lacked entries. 

We now turn to our main focus: the $N\to\infty$ limit.
We extrapolate the masses in Tables~\ref{table_MKJ_N2-5},\ref{table_MKJ_N6-12}
with $O(1/N^2)$ corrections, as in eqn(\ref{eqn_MKN}).
The extrapolations for most of these states are displayed
in Figs.\ref{fig_M0pp0mpK_N},\ref{fig_MJ2PCK_N},\ref{fig_MJ1K_N}.
%,\ref{fig_MJ3K_N}.
This leads to the masses (in units of the string tension) listed in
Table~\ref{table_MK_J_SUN}. In making these extrapolations we have sometimes 
excluded the mass for the smallest value of $N$ ($N=2$ for $C=+$ and $N=3$ for $C=-$),
in order to achieve a significantly better fit. This we have done for the $2^{++}$,
$1^{-+}$ and $1^{--}$ ground states. This presumably reflects the need for at
least a further $1/N^4$ correction, but our number of data points is too small
for this. In some cases we obtain better fits by excluding the $N=12$ values,
and although this could be argued for on the basis that our range of $a$ for the
continuum extrapolation is relatively limited in that case, the danger of
`cherry-picking' has led us to avoid doing so.

The main features of these continuum spectra are that the lightest glueball
is the $J^{PC}=0^{++}$ scalar, with the $J^{PC}=2^{++}$ tensor about $50\%$ heavier,
and with the  $J^{PC}=0^{-+}$ pseudoscalar just slightly above that. The next
state is the $J^{PC}=1^{+-}$ vector, which is special in that it is the only
relatively light $C=-$ glueball, with most $C=-$ states being very heavy.
Just a little heavier than this vector is the first excited $0^{++}$ and then the
$2^{-+}$ ground state. The $N$ dependence of nearly all these states is weak and readily
absorbed into a $O(1/N^2)$ correction down to at least $SU(3)$; all this confirming
that the $SU(3)$ gauge theory is indeed `close to' $SU(\infty)$.


%
%
\subsection{scattering states}
\label{subsection_scattstates}

A given operator will project onto all states with the quantum numbers of that operator.
In particular our single trace glueball operators will project onto multi-glueball states
in addition to the glueballs that we are interested in. This means that some of the states
in the `glueball' spectra that we have calculated may in fact be multi-glueball
states. In the ideal case of a very large spatial volume these will be scattering states
where at sufficiently large times the states are far apart 


There is some ambiguity here of course. Sufficiently heavy glueballs
will be unstable, and will decay into multiglueball states but to the extent
that their decay width is not too large, we still regard them as glueball states.
Indeed as $N$ increases the decay widths and the overlap of multiglueball
states onto our single trace basis of glueball operators
should decrease. So although we expect
this to be an issue primarily at smaller $N$, it is clearly important to address.
We do so here in some detail because it has been addressed only briefly and
occasionally in past glueball calculations.

A full analysis of scattering states and glueball decays would involve different
techniques to those used in this paper and so are outside the scope of this work.
Here we will, instead, perform some exploratory calculations to obtain an indication
of the impact of multiglueball states on our calculated glueball spectra. We limit
ourselves to considering states consisting of two glueballs since these are the lightest
ones and our calculated glueball spectra do not extend to masses that are much higher.
If there is weak mixing between double trace and single trace operators then we can
introduce two glueball scattering states by using double trace operators, such as
%
\be
\phi_{ab}(t) = (\phi_a(t)-\langle \phi_a\rangle)(\phi_b(t)-\langle \phi_b\rangle)
-\langle(\phi_a-\langle \phi_a\rangle)(\phi_b-\langle \phi_b\rangle)\rangle.
\label{eqn_phiab}
\ee
%
Here $\phi_a$ and $\phi_b$ are single trace operators. The vacuum subtractions ensure
that $\phi_{ab}$ does not have a projection onto the vacuum or a `trivial' projection
onto single glueball states through terms such as $\langle \phi_a\rangle \phi_b(t)$.
One can calculate the mass spectrum obtained when one adds such double trace operators
to the basis of single trace operators and compare it to that obtained using just the
single trace operators. If the resulting spectra are the same then this strongly
suggests that the spectrum obtained using single trace operators already includes
two glueball states. If, on the other hand, the spectrum obtained using the expanded basis
produces extra states that can be plausibly interpreted as two glueball scattering states,
then this suggests that the states obtained with our original basis of single trace
operators does not contain such states and that categorising them as single glueball
states is probably correct.

The potential number of operators $\phi_{ab}$ is clearly too huge to be practical
here and so we severely limit the number for our study as follows. Firstly, we take
both $\phi_a$ and $\phi_b$ to have zero momentum, so both the total and relative
momenta are zero. Of course this does not mean that the relative momentum of the
two glueballs has to be zero, since the relative momentum is not a conserved
quantum number in an interacting system, but one naively expects that the main overlap
will be onto zero relative momentum. Secondly we only keep the 2 or 3 blocking
levels that are most important at the $\beta$ we calculate. (We systematically avoid using
the largest blocking levels so as to exclude unwanted ditorelon states.) Thirdly, we take the
same blocking levels for $\phi_a$ and $\phi_b$. For $\phi_a$ we use only the rotationally
invariant sum of (blocked) plaquettes. That is to say it is in the $A_1^{++}$
representation. So $\phi_{ab}$  will be in the same representation as $\phi_b$. For
$\phi_b$ we take 3 different loops chosen so that we can have projections onto all
representations including $P=\pm$ and $C=\pm$. The lightest energy of the
corresponding asymptotic two glueball state would be the sum of the lightest
$A_1^{++}$ mass plus the lightest mass in the representation of $\phi_b$
if we were in an infinite spatial volume. Since our spatial volume is finite
the glueballs are interacting at all times and there will be a shift in the total
energy. However, to the extent that our spatial volume is not very small, this
shift should be small and we will use the naive sum as the quantity against
which to compare our supposed scattering states.

We perform calculations in $SU(3)$, which is the most physically interesting case
amongst our lower $N$ calculations, and in $SU(8)$ which is representative of our
large $N$ calculations. In $SU(3)$ we work at $\beta=6.235$ on the same $26^4$ lattice
used in our above glueball calculations. For completeness we have also carried out
some calculations on $18^326$ and $34^326$ lattices at the same value of $\beta$.
In $SU(8)$ we work at $\beta=46.70$ on a $16^324$ lattice which, again, is the same
as that used in our glueball calculations. The spatial volumes at larger $N$ are
smaller than those at smaller $N$, taking advantage of the expected suppression of
finite volume corrections with increasing $N$, and this is the reason for this
additional calculation in  $SU(8)$.

We calculate correlators of the double trace operators with each other as
well as with the single trace operators, since all of these are needed in the
variational calculation using the basis that combines single and double trace
operators. Because we need high statistics for glueball calculations, we calculate
the correlators at the same time as we generate the lattice fields. At that stage
we do not know the values of the vacuum subtractions in eqn(\ref{eqn_phiab})
and so we need to calculate a number of other correlators so that at the later
analysis stage we can reconstruct the desired subtracted correlators. The relevant
expressions are given in Appendix~\ref{section_appendix_scattstates}.

We begin with our $SU(3)$ calculation. In Fig.~\ref{fig_MeffG+GGA1++l26n_SU3} we
show the effective masses of the lightest few states in the $A_1^{++}$ representations
obtained using the basis of single trace operators (open circles) and the same basis
extended with our double trace operators (filled points). We see quite clearly
that the two sets of states match each other well except for one state, whose effective
masses are represented by a $\blacklozenge$, and which clearly has no partner amongst
the states from the single trace basis. This state appears to asymptote, at larger
$t$, to an energy that is close to twice the lightest $A_1^{++}$ mass, i.e. that of the
lightest free two glueball state, and so we infer that this is indeed the
lightest two glueball state. Apart from this state, the lightest 4 states in the
two bases match each other quite precisely. That is to say, we have good evidence
that the $A_1^{++}$ continuum masses listed in Table~\ref{table_MK_R_SU3} are indeed
those of single glueball states.

It is interesting to ask what kind of $A_1^{++}$ energy spectrum one obtains if one
uses a basis consisting only of the double trace operators. The result is shown
in Fig.~\ref{fig_MeffGGA1++l26n_SU3}. We see that the state that is lightest at small $t$
is consistent with being the lightest 2 glueball state, and that our variational
wave-functional has a good overlap onto that state. What is equally interesting is
the state that is the first excited state at small $t$ and whose effective mass
drops monotonically below that of our two glueball state as $t$ increases, consistent
with heading towards the mass of the lightest glueball. This presumably represents
the overlap of our double trace basis onto the lightest glueball. A rough estimate suggests
that this variationally selected operator has an overlap (squared) of $\sim 10\%$
which again points to little mixing of scattering states into our low-lying
glueball spectrum.

Clearly we would like to repeat our above analysis of the  $A_1^{++}$ spectrum for all 
the other representations. However once the lightest state in some $R^{PC}$ representation
is much heavier than in the above case the energy of the predicted scattering state
lies in a dense part of the spectrum, all the energies have much larger statistical errors
and we cannot follow the effective energies beyond the smallest values of $t$. That is to
say, the analysis becomes hopelessly ambiguous. We will therefore restrict ourselves to
a few cases where the energies are still manageably small. That is to say, we will
look at the $T_2^{++}$ which contains 3 components of the lightest $J=2^{++}$ state,
the  $A_1^{-+}$ where the lightest state is the interesting $0^{-+}$ pseudoscalar
(the lightest $P=-$ glueball),
the  $E^{-+}$ where the lightest state is the $2^{-+}$ pseudotensor, and the $T_1^{+-}$
which is the lightest $C=-$ state and where the lightest 3 component state is the
$C=-$ vector. We do not analyse any $R^{--}$ states since they are all too heavy.

We begin with the $T_2^{++}$ representation. Here our double trace operator
would project onto a two glueball state consisting of the lightest $A_1^{++}$
and $T_2^{++}$ glueballs if there were no mixing. Neglecting interactions
its energy is just the sum of these masses. This case differs from the $A_1^{++}$
representation discussed above because this is not the lightest scattering state
one can construct. Two $A_1^{++}$ glueballs with two units of angular momentum
are lighter, although one can expect an angular momentum threshold suppression
factor that would need to be quantified. Such a state requires the use of operators
with non-zero momentum which would mean extending our basis of operators well beyond
our choice for this exploratory study. It is plausible that what we learn using
our heavier two glueball states would also apply to these lighter ones. So
%in Fig.~\ref{fig_MeffG+GGT2++l26n_SU3} we show for the $T_2^{++}$ representation the
we carry out for the $T_2^{++}$ representation the analogue of the analysis
in Fig.~\ref{fig_MeffG+GGA1++l26n_SU3} for the $A_1^{++}$. Once again we find
that the inclusion of our double trace operators leads to an extra state
%, labelled by $\blacklozenge$,
although because of the denser packing of the states it is less
prominent than in the  $A_1^{++}$ case. However the effective energy of this state is
consistent with decreasing towards that of a scattering state composed of the lightest
$A_1^{++}$ and $T_2^{++}$  glueball, albeit with substantial statistical uncertainty.
So it is very plausible that it is a scattering state composed of the lightest  $A_1^{++}$
and $T_2^{++}$ glueballs.
The spectrum from the double trace basis
%is shown in Fig.~\ref{fig_MeffGGT2++l26n_SU3} Here we see
shows quite clearly that the state that is lightest at small $t$ has an energy
roughly equal to the sum of the lightest $A_1^{++}$ and $T_2^{++}$ glueballs, showing
it to be a scattering state. It is also clear that the projection of these
double trace operators onto the lightest $T_2^{++}$ glueball is very small.
All this makes it plausible that the five or six lightest states in the $T_2^{++}$ channel
%in Fig.~\ref{fig_MeffG+GGT2++l26n_SU3}
are single glueballs. However we recall our above caveat that there are lighter scattering
states that could, in principle, behave differently.

With the $T_1^{+-}$ representation our double trace operators project onto the
lightest two glueball state since the lightest  $T_1^{+-}$ is the lightest $C=-$ state.
%In Fig.~\ref{fig_MeffG+GGT1+-l26_SU3} we see
Here we again find 
that the addition of the double trace operators to the single trace basis does
produce an extra state and that the effective energy of that state appears
to be decreasing towards the energy of the lightest scattering state, up to the
point where the errors grow too large to allow a statement. From this
%plot
we infer
that the lightest five $T_1^{+-}$ states obtained using the single trace basis are
not scattering states. In addition we find that in the
%In  Fig.~\ref{fig_MeffGGT1+-l26_SU3} we show the
spectrum obtained using just the double trace operators the
lightest state appears to approach the expected scattering state and that
there apears to be no significant component of the lightest single glueballs
present in this basis, providing further evidence for little overlap
between our single and double trace operators.

In the case of the  $A_1^{-+}$ representation our double trace operators naturally
project onto a state with both an $A_1^{++}$ and an $A_1^{-+}$ glueball. Just as
for the $A_1^{++}$ this should be the lightest possible scattering state: although
one can obtain $P=-$ through unit angular momentum one would need a glueball
with non-zero spin to take us back to an overall $A_1$ state.
We show in Fig.~\ref{fig_MeffG+GGA1-+l26n_SU3}
how the inclusion of our double trace operators affects the spectrum. While the
ground and first excited states are unaffected, there is clearly an extra state
just above these, and a significant shift in the mass of the next state above this.
The effective energy of the (probable) extra state descends rapidly towards
the energy of the free scattering state, but the rapidly growing errors prevent
us from telling if it asymptotes to this value or continues to decrease.
For further evidence we plot in Fig.~\ref{fig_MeffGGA1-+l26n_SU3} the spectrum
obtained in the basis with only the double trace operators. Here we have a quite
dramatic contrast to what we have seen above in the $A_1^{++}$, $T_2^{++}$
and $T_1^{+-}$ representations: the lightest state is clearly the ground state
of the single trace spectrum. This tells us that here we have a large overlap
between single trace and double trace operators. So here we can have legitimate concerns
about the presence of multiglueball states in the low-lying spectrum obtained
using single trace operators.

The $E^{-+}$ representation contains the second lightest state in the $P,C=-,+$
sector, after the $A_1^{-+}$  ground state. Neglecting interactions, our double trace
operators would project onto the two glueball states composed of an $A_1^{++}$ and
an $E^{-+}$ glueball. There are scattering states that are slightly lighter
but they should have angular momentum threshold factors that effectively
cancel that advantage. (For example, the lightest  $A_1^{++}$ and $A_1^{-+}$
glueballs with two units of angular momentum.)
%In Fig.~\ref{fig_MeffG+GGE-+l26_SU3} we see
We find, when plotting the effective masses,
that the addition of the double trace operators to the single trace basis does
produce an extra state. However this state is nearly degenerate with another state
so it is unclear which is the extra one. (It is quite possible that our variational
procedure mixes the two.) In either case the effective energy of that state appears
to be decreasing towards the energy of the lightest scattering state, up to the
point where the errors grow too large to allow a statement. The spectrum of
states with just the double trace basis
%, shown in  Fig.~\ref{fig_MeffGGE-+l26_SU3},
shows quite clearly that the lightest state is the scattering state and that there
is no significant overlap of our double trace operators onto the low-lying spectrum
obtained with the single trace operators.

In summary, the above study in $SU(3)$ shows that two glueball states have very little
overlap onto most of the spectrum one obtains from single trace operators for the
lighter and hence most interesting states in our calculated glueball spectra. This is
something of a surprise in $SU(3)$ since one would not expect to be able to
invoke a large-$N$ suppression for $N=3$. The one apparent exception concerns
the $A_1^{-+}$ representation which contains the $0^{-+}$ pseudoscalar which
is of particular theoretical interest. This clearly merits a much more detailed
investigation.

We now turn to a similar study in $SU(8)$. This we should do since for $N\geq 8$
we use considerably smaller spatial volumes than for $SU(3)$, taking advantage of the
expected large-$N$ suppression of finite volume corrections. We perform our study on
a $16^324$ lattice at $\beta=46.70$ which corresponds to a lattice spacing that
is slightly larger than that of the above $SU(3)$ study if measured in units of the
string tension, $\sigma$, although equal when measured in units of the mass gap.
Our results are very much the same as for  $SU(3)$:
in the $A_1^{++},T_2^{++},T_1^{+-}$ and $E^{-+}$ representations there appear to
be very small overlaps between our would-be two glueball operators and the
single trace operators in the low-lying part of the spectrum that is of interest
to us in this paper. The striking exception, as in the case of $SU(3)$, is the 
$A_1^{-+}$ representation. This is intriguing both because this representation
contains the interesting $J^{PC}=0^{-+}$ pseudoscalar glueball and because one
would have naively expected the large-$N$ suppression of such overlaps between
single and double trace operators to have taken effect by $N=8$. There is
perhaps some hint of this in comparing the approaches to the respective $A_1^{-+}$ masses
of the ground states.

These results for $SU(3)$ and $SU(8)$ provide some reassurance that our spectra
in the glueball calculations of this paper are indeed those of single glueballs,
at least for the lower lying spectrum that is of the greatest physical and
theoretical interest.


%
%
\subsection{some comparisons}
\label{subsection_comparisons}

It is useful to compare our glueball results against those of earlier calculations both as
a check on the reliability of our errors, particularly the systematic errors that are the most
difficult to control. To reduce the comparison to manageable proportions we limit
ourselves to the continuum limit of the $J^{PC}=0^{++},2^{++},0^{-+},2^{-+},1^{+-},3^{+-},2^{--}$
ground states and to calculations that have appeared after the year 2000. The first five states
are the lightest ground states and are easy to identify, while the sixth involves a more
challenging identification of near-degeneracies amongst the cubic irreducible representations,
and the last is an example of the much heavier states over which we have less control.
We will also limit ourselves to $SU(3)$, where there are several calculations, and to $SU(8)$
as a representative of our larger $N$ calculations.

We list the $SU(3)$ and $SU(8)$ comparisons in Tables~\ref{table_MK_J_SU3_comp} and
\ref{table_MK_J_SU8_comp}. The calculations in
%
\cite{BLMT_N,BLMTUW_N,HM_Thesis}
%
are broadly of the same type as those in the present paper, making a comparison straightforward.
The calculations of
%
\cite{BLMT_N,BLMTUW_N}
%
were designed to explore the approach to the large $N$ limit and focussed upon the masses of
the $J^{PC}=0^{++},2^{++}$ ground states and the string tension. The primary goal of
%
\cite{HM_Thesis}
%
was to calculate the masses of higher spin glueballs which cannot be identified using the
naive approach of searching for near-degeneracies, as used in the present paper, and this required
the application of novel dedicated techniques. The calculations in 
%
\cite{MP-2005}
%
are significantly different to all of these, in that they use an alternative `improved' lattice
action and anisotropic lattices with a temporal lattice
spacing, $a_t$, smaller  than the spatial lattice spacing, $a_s$, by a factor of (roughly) 3 or 5.
In addition the continuum glueball masses are expressed in units of the Sommer scale $r_0$
rather than the string tension $\sigma$. We have used the value $r_0\surd\sigma = 1.160(6)$,
as fitted in
%
\cite{AAMT-2020},
%
to translate the values given for $r_oM_G$ in 
%
\cite{MP-2005}
%
to the values $M_G/\surd\sigma$ listed in Table~\ref{table_MK_J_SU3_comp}.

We begin with the $SU(3)$ comparison in Table~\ref{table_MK_J_SU3_comp}. Overall we see
that the results of the different calculations agree within about 2 standard deviations for
all the states. In particular the values in 
%
\cite{HM_Thesis}
%
are within one standard deviation of our values. On the lighter ground states our errors
are far smaller than those of the earlier calculations. However on the two heaviest states
the errors in
%
\cite{MP-2005}
%
are similar to ours, demonstrating the expected advantage of highly anisotropic lattices
for calculating the masses of heavy glueballs. Focussing now on possible differences,
we note that all the mass estimates from
%
\cite{MP-2005}
%
are slightly above ours, and indeed above those of
%
\cite{HM_Thesis}
%
It is possible that this has to do with the scale $r_0$. This scale is determined as the
spatial distance at which the heavy quark potential has a certain value of its derivative.
Thus its value is expressed in units of the spatial lattice spacing, i.e. as $r_0/a_s$,
and this has to be expressed as  $r_0/a_t$ in order to serve as a scale for a lattice
glueball mass, $a_tM_G$. This rescaling is clearly sensitive to even small shifts of the
anisotropy from the tree level values of 3 or 5. Moreover the values of $a_s$ used in
%
\cite{MP-2005}
%
are large which means that $r_0/a_s$ is not large, and this makes the accurate determination
of its value more delicate. So it is possible that all this is enough to lead to a $1-2\%$
shift in the estimates of $M/\surd\sigma$ in this calculation. Turning now to
%
\cite{BLMT_N,BLMTUW_N}
%
we see that the $0^{++}$ mass is about two standard deviations higher than our value
(or that of 
%
\cite{HM_Thesis}
%
).
A possible source of this is the fact that these older calculations have a more limited
range of higher $\beta$ values and include in their continuum extrapolations the smaller values
of $\beta$ where there is some effect of the dip in the mass gap associated with the
strong-to-weak coupling bulk transition, as discussed in Section~\ref{subsection_bulk}.
This dip in the mass at larger lattice spacings may then bias the straight line fit
in $a^2\sigma$ so that its $a=0$ intercept is a little too high. In the present
calculations the statistical errors are small enough that a good straight line fit
is no longer possible if one includes such large $a$ values and by excluding them any
such bias becomes smaller and the continuum intercept is a little lower. This discussion
is only relevant to the mass gap, since the other glueballs appear to be insensitive
to the bulk transition and have, for the most part, a small dependence on the
size of the lattice spacing.

Turning now to the $SU(8)$ comparison in Table~\ref{table_MK_J_SU8_comp}, we see that
for the states other than the $0^{++}$ the masses from the different calculations agree
within one standard deviation, although the errors on the earlier calculations are typically
about ten times larger than ours, leaving room for potential discrepancies.
The mass gap shows a signficant discrepancy but both of the earlier calculations include
masses calculated close to the bulk transition at $\beta=\beta_b^\downarrow$ where
the mass gap appears to have a significant dip, as discussed in Section~\ref{subsection_bulk},
which may well explain this minor difference.

The above comparisons naturally raise the question of why we chose to use the traditional Wilson
plaquette action instead of some alternative `improved' action, such as the one used in 
%
\cite{MP-2005}.
%
In glueball calculations the main improvement observed, for $SU(3)$, is to
largely remove the dip in the mass gap as one transitions from strong to weak coupling.
This means that calculations at coarser lattice spacings, and so on smaller lattices, may be
used in the continuum extrapolation of the mass gap. However a larger lattice spacing
means that the correlator signals tends to disappear into the statistical noise before one
has a well defined effective mass plateau. To remedy this one can use an anisotropic
lattice action with the temporal spacing $a_t$ much less than the spatial $a_s$.
However using contemporary resources there is no need to worry about the dip in the mass gap:
one can easily do calculations far from the dip, at much weaker coupling, as we have seen
in this paper. Moreover if one works at coarse lattice spacings using such an improved action 
one does not know in advance whether the behaviour of other glueballs will be improved
or worsened. And in any case, as we have seen in
Figs~\ref{fig_MJ02ppK_cont_SU4},\ref{fig_MJ02mpK_cont_SU4},\ref{fig_MJ1K_cont_SU4},
the lattice corrections to the glueball masses are already quite modest with
the standard Wilson plaquette  action. An advantage of the simple plaquette action
is that it is known to have exact reflection positivity and hence the transfer matrix has
real positive eigenvalues between zero and one, just as if it was the exponential of
a Hamiltonian. This means that one does not need to be
cautious in applying a variational calculation of the glueball spectrum. Of course
one can use an anisotropic plaquette action, which will also possess reflection positivity.
Such a calculation would produce a finer resolution in time of the correlator and so
could be very useful for calculating  heavier glueball masses. However it has a disadvantage
in setting the scale as described just above. One needs to know the relation between
$a_s$ and $a_t$ and since the chosen tree level anisotropy will receive corrections, this need
to be estimated, and the error on this estimate may be significant. Thus our suggestion for a
better calculation than the present one is that one does two calculations in parallel. One 
calculation should be with an isotropic action, and this will provide very accurate continuum
values for the light glueballs, including the mass gap $M_G$, in units of, for example,
the string tension, $\sigma$, or the Sommer scale $r_0$. The second calculation should be
with a strongly anisotropic action and provided that one has a large enough basis of operators,
this should provide accurate calculations of the heavier glueball masses in units of, say,
the mass gap. Here it is important that the spatial lattice spacings should extend to small
values, just as in the isotropic calculations, in order to guarantee the credibility of the
continuum extrapolations. Putting together both calculations allows all the masses to be expressed
in units of $\sigma$ or $r_0$. The action used could be either the plaquette action or some
improved action, although maintaining reflection positivity is desirable from a practical
point of view.


%
%
%
%
\section{Topological fluctuations}
\label{section_topology}

Euclidean $D=4$ $SU(N)$ gauge fields possess non-trivial
topological properties, characterised by a topological charge $Q$ which is integer-valued 
in a space-time volume with periodic boundary conditions. This charge can be expressed
as the integral over Euclidean space-time of a topological charge density, $Q(x)$, where
%
\begin{equation}
Q(x)= \frac{1}{32\pi^2} \epsilon_{\mu\nu\rho\sigma}
\mathrm{Tr}\{F_{\mu\nu}(x)F_{\rho\sigma}(x)\}.
\label{eqn_Q_cont}
\end{equation}
Since the plaquette matrix $U_{\mu\nu}(x) = 1 + a^2 F_{\mu\nu}(x) + ....$ on sufficiently
smooth fields, one can write a lattice topological charge density $Q_L(x)$ on
such fields as
%
\begin{equation}
Q_L(x) \equiv {\frac{1}{32\pi^2}} \epsilon_{\mu\nu\rho\sigma}
\mathrm{Tr}\{U_{\mu\nu}(x)U_{\rho\sigma}(x)\}
= a^4 Q(x) +O(a^6).        
\label{eqn_Q_lat}
\end{equation}
However this definition lacks the reflection antisymmetry of the continuum operator
in eqn(\ref{eqn_Q_cont}), since all the plaquettes $U_{\mu\nu}(x)$ are defined as forward
going in terms of our coordinate basis, so to recover these symmetry properties
we use the version of this operator that is antisymmetrised with respect to forward
and backward directions
\cite{DiVecchia-FFD}
(although this is unnecessary on smooth fields). 


The fluctuations of $Q_L(x)$ are related to the expectation value of the composite operator
$Q_L^2(x)$ whose operator product expansion contains the unit operator
\cite{DiVecchia-FFD},
so these fluctuations are powerlike in $1/\beta$. On the other hand, the average of
$Q_L(x)$ is $O(a^4)$ and
hence exponentially suppressed in $\beta$. Thus as $\beta$ increases the fluctuations
around $Q_L(x)$ and $Q_L=\sum_x Q_L(x)$ diverge compared to the physically interesting
mean values. In addition the composite operator $Q_L(x)$ also receives a multiplicative
renormalisation $Z(\beta)$ such that $Z(\beta) \ll 1$  at accessible values of $\beta$
\cite{Pisa_ZQ}.

In practice all this means that one cannot extract the topological charge of a typical lattice
gauge field by directly calculating $Q_L=\sum_x Q_L(x)$ on that gauge field. However we note that
the fluctuations obscuring the value of $Q$ are ultraviolet, while the physically relevant
topological charge is on physical length scales. Thus if we perform a very limited local smoothening
of the fields to suppress the ultraviolet fluctuations, this should not affect physics
on long distance scales, and the value of $Q_L=\sum_x Q_L(x)$ calculated on these smoothened
fields should provide a reliable estimate of $Q$. Moreover, recalling that the total
topological charge of a gauge field is unchanged under smooth deformations, we can expect
that even under a moderately large amount of continued smoothening the value of  $Q_L$ will not
change, even though $Q_L(x)$ itself does gradually change. One convenient way to smoothen the
gauge fields is to locally minimise the action. Such a `cooling' of the original `hot'
lattice gauge field
\cite{MT-cool}
involves sweeping through the lattice one link at a time, precisely like the Monte Carlo
except that one chooses the new link matrix to be the one that minimises the total
action of the plaquettes containing that link matrix. This is a standard technique
that one can find described in more detail in, for example,
\cite{DSMT-Q}.
An alternative and attractive smoothing method with perturbatively proven
renormalisation properties is the gradient flow
%
\cite{Luscher:2010iy,Luscher:2011bx,Luscher:2013vga}
%
which has been shown to be numerically equivalent to cooling
%
\cite{Bonati:2014tqa,Alexandrou:2015yba,Alexandrou:2017hqw}.
%
Cooling typically performs nearly two orders of magnitude faster than the gradient flow and
since we are aiming for large statistics we adopt the simplest cooling method.
After the first couple of cooling sweeps the fields are already quite smooth, as we shall see below.
Since we are minimising the action and since in the continuum the minimum action field with a given
$Q$ is a multi-instanton field, we expect that under systematic cooling the lattice field will be driven
to become some multi-instanton field, which one can see by calculating the distribution $ Q_L(x)$ on such
a field. Of course, because of the discretisation of space-time the topological properties of lattice fields
are only approximately like those of a continuum field. One can deform a large instanton by gradually
shrinking its non-trivial core and on a lattice this core can shrink to within a hypercube. At this
point what was an instanton has been transformed into a gauge singularity and the value of $Q$
will now differ from its original value by one unit. (Equally, one can gradually grow an instanton
out of a hypercube.) This process can occur during cooling but it can equally occur during
the course of our Monte Carlo. In the latter case, it is these changes in $Q$ that allow us
to sample all possible values of $Q$ and hence maintain the ergodicity in $Q$ of our Markov process.
As $\beta\uparrow$ the distance between physical and ultraviolet scales grows and these
changes in $Q$ become increasingly suppressed -- in fact more strongly than the usual critical
slowing down (see below). This is as it should be: after all, the changes in topological charge
are no more than a (useful) lattice artifact. 

We have outlined in Section~\ref{subsection_Qfreezing} the specific reasons for the suppression
of changes in $Q_L$ when $a(\beta)\to 0$ at fixed $N$ and when $N\to \infty$ at fixed $a(\beta)$,
and how we deal with this problem in our glueball and string tension calculations.
Below, in Section~\ref{subsection_Qcooling}, we will give some evidence that the value of $Q_L$
on a cooled lattice field does indeed reflect the topology of the original lattice field.
Then, in Section~\ref{subsection_Qtunneling}, we will give some results on the rate of critical
slowing down, both as $a(\beta) \to 0$ and as $N\to\infty$. In Section~\ref{subsection_Qsusc}
we present our results for the continuum topological susceptibility in those cases, $2\leq N\leq 6$,
where the topological freezing is not too serious an obstacle. Finally in Section~\ref{subsection_QZ}
we present our numerical results for the factor $Z(\beta)$ that relates the value of $Q_L$
before cooling to its value, $Q_I$, after cooling. Our study of lattice topology has some
limitations of course: it is not a dedicated study but is constrained by being a byproduct of our
glueball and string tension calculations. Finally we remark that it is only in the calculations
of the topological susceptibility that the topological freezing is an obstacle; in our other calculations
in this Section it does not matter whether the topological charge we study is introduced by hand
or appears spontaneously.
  
%
%
\subsection{topology and cooling}
\label{subsection_Qcooling} 

As we cool our lattice fields the fluctuations in the measured value of $Q_L$ decrease and
it rapidly settles down to a value that is close to an integer. As a typical example we plot in
Fig.\ref{fig_Qcool20_su5} the number of fields with a given topological charge $Q_L$ after
2 coolings sweeps and the number after 20 cooling sweeps, taken from sequences of $SU(5)$ fields
on a $16^320$ lattice at $\beta=17.63$. With respect to the calculations in this paper, this 
corresponds to an intermediate value of $N$, and of $a(\beta)$, and of the lattice volume. As we
see, after 20 cooling sweeps the values of $Q_L$ are very strongly peaked close to 
integer values. The reason that the peak is not at an integer, even when the cooling has
erased the high frequency fluctuations, is that for an instanton of size $\rho$ on a lattice,  
we only obtain $Q_L \to 1$ in the limit $\rho/a \to \infty$, with deviations from unity
that are powers of $a/\rho$. However, as is clear from Fig.\ref{fig_Qcool20_su5}, there is 
no significant ambiguity in assigning to each field after 20 cooling sweeps an integer valued 
topological charge which we label $Q_I$. We can expect $Q_I$ to provide our most reliable
estimate of the topological charge $Q$ of the original lattice gauge field. For the calculations
with the largest values of $a(\beta)$ identifying the value $Q_I$ can be
less  clear-cut, but such lattices are of little importance in determining continuum physics.
We also see from Fig.\ref{fig_Qcool20_su5} that even after only 2 cooling sweeps the fields
fall into groups that only overlap slightly. Since 2 sweeps cannot affect anything other 
than the most local fluctuations, we can assume that this segregation into differing topological 
sectors directly reflects the topology of the gauge fields prior to any cooling. 

As a second example we show in Fig.\ref{fig_Qcool20_su8} the same type of plot for several sequences 
of $SU(8)$ fields generated at $\beta=47.75$ on a $20^330$ lattice. This corresponds to our
smallest lattice spacing in $SU(8)$. Here we see an even sharper peaking after 20 cooling sweeps
and, more interestingly, a very clear separation between the various sectors even after only 2 cooling 
sweeps. In contrast to the $SU(5)$ example, in this case there is essentially no tunnelling between
topological sectors, and the observed distribution of $Q$ has been imposed on the starting
configurations for the various sequences.  

The loss of ergodicity with respect to the topological charge is illustrated for $SU(8)$
in Fig.\ref{fig_Qseq_su8} where we show the values of $Q_L$ after 2 and 20 cooling sweeps
taken every 100 Monte Carlo sweeps for two sequences of 50000 sweeps generated at $\beta=47.75$.
In one sequence we have $Q_L\sim 1$ and in the other we have $Q_L\sim 2$. It is interesting
that even after only 2 cooling sweeps the separation in the measured values of $Q_L$ is
unambiguous. A similar plot for $SU(5)$ fields generated at $\beta=17.63$ is shown in 
Fig.\ref{fig_Qseq_su5}. Here the value of $Q_L$ remains unchanged over subsequences that
are typically a few thousand sweeps long, but the changes are sufficiently frequent that
in our overall ensemble of $\sim 2\times 10^6$ fields we may regard $Q_L$ as ergodic.
But just as for $SU(8)$ the values of $Q_L$ after only 2 cooling sweeps track the
values after 20 cooling sweeps.

The close relationship that we observe in Fig.\ref{fig_Qseq_su8} and Fig.\ref{fig_Qseq_su5}
between the values of $Q_L$ after 2 and 20 cooling sweeps is confirmed explicitly in
Fig.\ref{fig_Qcool20c2_su8}. Here we have taken all the $\beta=47.75$ $SU(8)$ fields on which
we have calculated $Q_L$ and we have extracted the three subsets that correspond to
$Q_L \simeq 1,2,3$ after 20 cooling sweeps. For each subset we plot the values that
$Q_L$ takes on fields after 2 cooling sweeps. As we see in Fig.\ref{fig_Qcool20c2_su8}
the three distributions do not overlap. So we can assign to each field a value of $Q$ that
is completely unambiguous (at least for our statistics) on the basis of the value of $Q_L$
measured after only 2 cooling sweeps, where the long-distance physics should be essentially
unchanged from that of the original uncooled lattice fields. As an aside, we show
in Fig.\ref{fig_Qcool20c1_su8} the same plot as in Fig.\ref{fig_Qcool20c2_su8}, but
with values of $Q_L$ calculated after only 1 cooling sweep. Here the distributions are
broader, so that one can no longer use the value of $Q_L$ to unambiguously assign the
field its value of $Q$, but the fact that even  after 1 cooling sweep the values
of $Q_L$ strongly reflect the value of $Q$ after 20 cooling sweeps is evident. 

It is instructive to see  how the measured values of $Q_L$ vary with the number of cooling
sweeps $n_c$. As an example, we take an ensemble of fields that have a topological charge
$Q=2$, as determined by the value of $Q_L$ after 20 cooling sweeps. For this same ensemble
we calculate how the average value of $Q_L$ varies with $n_c$ and also how the standard
deviation of its fluctuations vary with $n_c$. These quantities, labelled $\overline{Q}_L$
and $\sigma_{Q_L}$ respectively, are shown in Table~\ref{table_Q_nc_SU8} for three different
values of $\beta$ in $SU(8)$. As we see, in the uncooled ($n_c=0$) fields the
fluctuations $\sigma_{Q_{L}}$ are so large compared to the average charge $\bar{Q}_L(n_c=0)$
of those fields, that one cannot hope to estimate for individual fields the true value of $Q$ from
the value of $Q_L$ at $n_c=0$. However we also see that even after only 1 cooling sweep the value of
$\bar{Q}_L/\sigma_{Q_{L}}$ increases by a factor $\sim 20$ and after 2 cooling sweeps by $\sim 80$,
so that one can, with rapidly increasing reliability, assign a value of $Q$ to an individual
field on the basis of the value of $Q_L(n_c)$ at the first few cooling sweeps. What we also see
in Table~\ref{table_Q_nc_SU8} is that beyond the lowest few values of $n_c$ there is little
difference between the 3 ensembles, despite the fact that the lattice spacing changes
by a factor $\sim 1.7$. For larger $n_c$ the value of $Q_L(n_c)$ decreases slightly with
decreasing $\beta$ as one expects because the typical instanton size will decrease in lattice
units and the discretisation corrections are negative. For $n_c=0$ the decrease in $Q_L$
is larger and reflects the $\beta$-dependence of the renormalisation factor $Z(\beta)$ which
is driven by high frequency fluctuations, as analysed more quantitatively in
Section~\ref{subsection_QZ}. The rapid increase of $\sigma_{Q_L}$ with increasing $\beta$
is driven by two competing factors: the high frequency fluctuations per lattice site
decrease as an inverse power of $\beta$, but since these fluctuations are roughly
uncorrelated across lattice sites, there is a factor proportional to the square root
of the lattice volume in lattice units which increases exponentially with $\beta$ if our
volumes are roughly constant in physical units. At larger $\beta$ the latter factor
will dominate and it therefore becomes rapidly harder to relate $Q$ from $Q_L(n_c=0)$
as we approach the continuum limit. On the other hand, after the first few cooling sweeps
the high frequency fluctuations have been largely erased and the fluctuations of $Q_L$ no
longer increase with increasing $\beta$, so that it becomes easier to identify the
value of $Q$ from the value of $Q_L$.

%As a footnote to the above we show in Table~\ref{table_Q_nc_SUN} the same quantities as in
%Table~\ref{table_Q_nc_SU8} but now for different $SU(N)$ groups. The calculations are at
As a footnote to the above we have calculated the same quantities as in
Table~\ref{table_Q_nc_SU8} but now for different $SU(N)$ groups. The calculations are at
roughly the same value of $a\surd\sigma$ as for the $SU(8)$ fields at $\beta=46.70$ in
Table~\ref{table_Q_nc_SU8}, i.e. at roughly the same value of the 't Hooft coupling, so that
the contribution of the high frequency fluctuations is roughly the same. We find that the
%So, as we see, the
variation in  $\sigma_{Q_L}(n_c=0)$ with $N$ is consistent with being due to the difference
in the square root of the lattice volumes. At larger $n_c$ the value of $\sigma_{Q_L}(n_c)$
decreases with increasing $N$, which tells us that larger-scale non-perturbative fluctuations,
such as in the instanton size, are decreasing. Finally, the main practical point is that there
is very little variation with $N$ of $\overline{Q}_L$ at any value of $n_c$.





%
%
\subsection{tunneling between topological sectors}
\label{subsection_Qtunneling} 

In the continuum theory $Q$ cannot change under continuous deformations of the fields
unlike most other quantities, such as glueball correlation functions, so one expects to lose
ergodicity in $Q$ much faster than in such other quantities as one approaches the continuum
limit in a lattice calculation using a local Monte Carlo algorithm such as the heat bath.
In this section we shall provide our data on this `freezing' of $Q$ as a function
of $a$ and $N$ and then compare this to some theoretical expectations.

As described earlier, the value of $Q$ changes if the core of an instanton shrinks and
disappears within a hypercube (or the reverse of this process). When $a$ is small enough
such instantons can be described by dilute gas calculations since these very small
instantons are very rare and the effect on them of other background field fluctuations that exist on
physical length scales will be negligible. So the basic process is for $Q$ to change by one unit.
Therefore we use as our measure of topological freezing the average number of sweeps  between
changes of $Q$ by one unit. We call this $\tau_Q$. Since the probability of such a process
is clearly proportional to the space-time volume, which is not something we have tried to keep
constant in our calculations, we rescale our measured values of $\tau_Q$ to a standard physical
volume, which we choose to be $V_0=l^4$ with $l\surd\sigma=3.0$,
so that we can compare the results of different calculations.
Our calculations of $Q$ have not been performed after every sweep, but typically with
gaps of  25 or 50 or 100 sweeps depending on the calculation. Our estimates of $\tau_Q$ can
only be reliable if they are much larger than this gap, since otherwise there could be multiple
changes of $Q$ within the gaps that we are missing, and so we do not includes those ensembles
where this issue arises -- typically at the coarser lattice spacings. However even if $\tau_Q$
appears to be much larger than the gap, we will occasionally see that $\Delta Q$, the change in
$Q$ across a gap, is greater than unity. In this case we assume that there have been $|\Delta Q|$
jumps within the gap and we make that part of our final estimate of  $\tau_Q$. In practice
this makes a very small difference to our results. By counting the number of changes of $Q$ in our
total sequence of lattice fields we can obtain the average distance between such changes, $\tau_Q$,
once we have renormalised to our standard volume $V_0$.

As $\tau_Q \to \infty$ the above definition is adequate. However when $\tau_Q$ is not very large
one can worry about the unwanted contribution of near-dislocations that occur across
a measurement of $Q$. We have in mind 
a small instanton that appears out of a hypercube shortly before a measurement, survives the
cooling (because of its environment)  and so contributes to the value of $Q$, but then quickly
disappears without becoming a larger physical instanton long before the next measurement.
In a sequence of measurements of $Q$ such an event would be characterised by a jump  $\Delta Q = \pm 1$
at one measurement followed by  the opposite jump,  $\Delta Q = \mp 1$, at the next measurement.
Of course it could be that when this happens we are seeing two independent events, with an
instanton appearing from one hypercube and after the measurement an anti-instanton appearing
out of a quite different hypercube. The characteristic of the latter events is that the signs of
the changes in $Q$ at neighbouring measurements are uncorrelated. Correcting for these we obtain
the measure  $\Tilde{\tau}_Q$ (normalised to our standard volume) which excludes this
estimate of near-dislocations. Whether this estimate is entirely reliable is arguable, so it is
reassuring that the differences between $\Tilde{\tau}_Q$ and $\tau_Q$ are not large,
and which one we use does not alter our conclusions below.

We present in Table~\ref{table_tauQ_SUN} our results for $\tau_Q$  and  $\Tilde{\tau}_Q$
from the sequences of fields generated in our glueball and string tension calculations.
The excluded values of $\beta$ either correspond to cases where $\tau_Q$ is not much
larger than the gap between measurements, and this includes all of our $SU(2)$ calculations,
or where $\tau_Q$ has become so large that we see no changes in $Q$ at all, which includes
almost all of our $SU(10)$ and $SU(12)$ calculations. The prominent qualitative features of our
results are that for any given gauge group both  $\tau_Q$  and  $\Tilde{\tau}_Q$ increase rapidly
as $a(\beta)$ decreases and, at roughly equal values of $a(\beta)$, they increase rapidly
as $N$ increases.

One can formulate some theoretical expectations for the behaviour of $\tau_Q$ as one decreases
$a$ and increases $N$. Just before shrinking through a hypercube an instanton will be very small
with size $\rho \sim a$ where the number density, $D(\rho)d\rho$  can be estimated using the
standard semiclassical formula
%
\cite{Coleman_Q}
%
\begin{equation}
D(\rho) \propto \frac{1}{\rho^5}
  \frac{1}{g^{4N}}
  \exp\left\{-\frac{8\pi^2}{g^2(\rho)}\right\}
    \stackrel{N\to\infty}{\propto}
 \frac{1}{\rho^5}\left\{ 
  \exp\left\{-\frac{8\pi^2}{g^2(\rho)N}
    \right\}\right\}^N
    \stackrel{\rho = a}{\propto}
 \left(a\Lambda\right)^{\frac{11N}{3}-5},  
\label{eqn_DI}
\end{equation}
%
where $\Lambda$ is the dynamical length scale of the theory. This is of
course a very asymptotic expression: we have neglected the powers of $g$ because they only
contribute powers of $\log(a\Lambda)$, and we have used the 1-loop expression for $g^2(a)$ which
is, as we have seen in Section~\ref{section_coupling}, inadequate for our range of
bare couplings. We also note that this expression only tells us what is the probability
of a very small instanton. In addition there will be a factor for the small instanton with
$\rho \sim O(a)$ to finally shrink completely within a hypercube: this `tunneling event' may
contribute an important factor depending on the lattice action being used.

The first qualitative feature of eqn(\ref{eqn_DI}) is that if we increase $N$ at
fixed $a$, we should find 
%
\begin{equation}
  \tau_Q \propto \frac{1}{D(\rho)} \propto \left\{\frac{1}{a\Lambda}\right\}^{\frac{11N}{3}-5}
\longrightarrow
  \ln\{\tau_Q\} = b + c N,
\label{eqn_IN}
\end{equation}
%
where $c$ depends on the value of $a$ and $b$ is some undetermined constant. In our
simulations we have some that correspond to almost equal values of $a\surd\sigma$, and hence of
$a\Lambda$, across several values of $N$. These are for $N=3,4,5,6$ at $a\surd\sigma \simeq 0.15$
and for $N=8,10,12$ at  $a\surd\sigma \simeq 0.33$. In Fig.~\ref{fig_tauQ_suN}
we plot the values of $\ln\{\tilde{\tau}_Q\}$ against $N$ and we see that the
behaviour is roughly linear as predicted from eqns(\ref{eqn_DI},\ref{eqn_IN}).

The second qualitative feature of eqn(\ref{eqn_IN}) is that if we vary $a$ at
fixed $N$, we should find 
%
\begin{equation}
  \tau_Q \propto \frac{1}{D(\rho)} \propto \left\{\frac{1}{a\Lambda}\right\}^{\frac{11N}{3}-5}
\longrightarrow
  \ln\{\tau_Q\} = b + \left\{\frac{11N}{3}-5\right\} \ln \left\{\frac{1}{a\Lambda}\right\}.
\label{eqn_Ia}
\end{equation}
In  Fig.~\ref{fig_tauQ_KsuN} we show plots of $\ln\{\tau_Q\}$ versus  $\ln\{1/a\surd\sigma\}$
for $N\in[3,8]$ and we see that the plots are roughly linear as predicted by
eqns(\ref{eqn_DI},\ref{eqn_Ia}). (As an aside, the fact that the $SU(2)$ values of $\tau_Q$ never
become large enough to be useful is no surprise given that the asymptotic dependence predicted
by eqn(\ref{eqn_Ia}) is quite weak, $\tau_Q\propto a^{7/3}$.)
The fitted coefficients of the $\ln\{1/a\surd\sigma\}$
term are listed in Table~\ref{table_tauQ_a} and compared to the value in eqn(\ref{eqn_Ia}).
We do not expect a good agreement since we know that the one loop expression for $g^2(a)$ is a poor
approximation in our range of bare couplings but it is interesting that our calculated values
listed in Table~\ref{table_tauQ_a}  do reflect the trend of the asymptotic theoretical values.


%
%
\subsection{topological susceptibility}
\label{subsection_Qsusc}

The simplest topological quantity that one can calculate is the topological susceptibility,
$\chi_t = \langle Q^2 \rangle / V$, where $V$ is the space-time volume. This is a
particularly interesting quantity because of the way it enters into estimates of the physical
$\eta^{\prime}$ mass through the Witten-Veneziano sum rule
%
\cite{Veneziano_eta,Witten_eta}.
%
There have been many calculations of this quantity and here we will add to these our calculations
for the gauge groups $SU(2)$, $SU(3)$, $SU(4)$, $SU(5)$ and for those of our $SU(6)$
ensembles where there are enough fluctuations in the value of $Q$ to make an estimate plausible.

We typically calculate the topological charge $Q$ after every 25 or 50 or 100 Monte Carlo
sweeps on most of the lattice ensembles that we use for our glueball and/or string
tension calculations. We estimate the value of $Q$ from the value of the lattice $Q_L$ after 20
cooling sweeps. The assignment of an integer value, $Q_I$, after 20 cooling sweeps is, as we
have seen, largely unambiguous. In Tables~\ref{table_QQ_SU2},~\ref{table_QQ_SU3_SU4} and
~\ref{table_QQ_SU5_SU6} we list our values for $Q^2$ on the lattices and at the couplings shown.

To obtain the continuum limit we perform a conventional extrapolation
%
\be
\left.\frac{\chi_t^{\frac{1}{4}}}{\surd\sigma}\right|_a
=
\left.\frac{\chi_t^{\frac{1}{4}}}{\surd\sigma}\right|_0
+ c a^2\sigma,
\label{eqn_chi_cont}
\ee
%
where we systematically remove from the fit the values corresponding to the largest $a$ until we
get an acceptable fit. We show the resulting continuum extrapolations for $N=2,3,4,5$ in
Fig.\ref{fig_khiIK_cont} and list the resulting continuum values in Table~\ref{table_khiK_SUN_cont}.
For $SU(2)$ and $SU(3)$ we do not include the value at the smallest $a$ since it is clearly too low
and it is plausible that it is due to a gradual loss of ergodicity in $Q$ accompanied by an
increasingly unreliable estimate of the errors. We see from Fig.\ref{fig_khiIK_cont} that
As we increase $N$ our fits are able to include values from increasingly coarse $a(\beta)$.
This is presumably related to the fact that the `bulk' cross-over/transition between
strong and weak coupling becomes rapidly sharper as $N$ increases
%
\cite{BLMTUW05}.
%
(It is particularly smooth for $SU(2)$.)
As shown in Table~\ref{table_khiK_SUN_cont} all the fits are acceptable.
We have performed separate continuum extrapolations of the susceptibilities obtained from
the non-integer lattice charges, $Q_L$, and the integer charges, $Q_I$, labelling these
$\chi_L$ and $\chi_I$ respectively. In the continuum limit these should be the same and we see
from Table~\ref{table_khiK_SUN_cont} that this is indeed so, within errors, for all except
the case of $SU(2)$, where the difference between the two values can be taken as a
systematic error that is additional to the statistical errors. 

Once we have the continuum susceptibilites we can extrapolate them to $N=\infty$  as shown
in Fig.\ref{fig_khiK_N}:
%
\be
\left.\frac{\chi_I^{\frac{1}{4}}}{\surd\sigma}\right|_N
=
0.3681(28) + \frac{0.471(15)}{N^2},
\label{eqn_chiIN}
\ee
%
and 
%
\be
\left.\frac{\chi_L^{\frac{1}{4}}}{\surd\sigma}\right|_N
=
0.3655(27) + \frac{0.448(15)}{N^2}.
\label{eqn_chiN}
\ee
%
The two are within errors as one would expect. We note that this value is consistent within errors
with the $N=\infty$ extrapolation in
%
\cite{bonanno_Q}
%
which uses a novel technique
%
\cite{hasenbusch_Q}
%
for ameliorating the problem of topological freezing.


Finally we return to the case of $SU(3)$ since it also has some phenomenological interest.
Here our analysis differs slightly from our earlier analysis in
%
\cite{AAMT-2020}
%
and our final result
%
\be
\left.\frac{\chi_I^{\frac{1}{4}}}{\surd\sigma}\right|_{SU(3)}
=
0.4246(36)
\label{eqn_chiIN3}
\ee
%
is about one standard deviation higher. To transform this into physical units
we translate from units in terms of $\surd\sigma$ to the standard scale $r_0$
and then to $\mathrm{MeV}$ units just as we did in eqn(\ref{eqn_LamMS_SU3}) of
Section~\ref{subsection_Lambda}, giving
%
\be
r_0\surd\sigma = 1.160(6) \Longrightarrow   \chi_I^{\frac{1}{4}} = 206(4)\mathrm{MeV}.
\label{eqn_chiIN3_MeV}
\ee
%
This is within errors of the value we presented in
%
\cite{AAMT-2020}
%
and the value $\chi_I^{\frac{1}{4}} \stackrel{SU3}{=} 208(6)\mathrm{MeV}$ obtained in
%
\cite{MLFP_SU3_10}
%
using the gradient flow technique. As an aside we note that the value of  
$\chi_I^{\frac{1}{4}}/\surd\sigma$ at $N=\infty$, as given in eqn(\ref{eqn_chiIN}),
is $\sim 13\%$ lower than the $SU(3)$ value, i.e. $\sim 179$`MeV' if
we simply rescale the value in eqn(\ref{eqn_chiIN3_MeV}).




%
%
\subsection{${Z_Q(\beta)}$ and lattice $\theta$ parameter}
\label{subsection_QZ} 

Consider the ensemble of Monte Carlo lattice gauge fields that correspond to an integer valued 
topological charge $Q$. If we calculate the charge $Q_L$ of each of these fields, prior to
any cooling, the average value will be related to $Q$ by
%
\cite{Pisa_ZQ}
%
\be
<Q_L> = Z_Q(\beta) Q,
\label{eqn_ZQ}
\ee
%
where  $Z_Q(\beta)$ will depend (weakly) on $N$ and negligibly on the lattice volume (as long
as it is not very small). Since the deviation of  $Z_Q(\beta)$ from unity is driven by high
frequency lattice fluctuations, it is of little physical interest in itself. However its
value is important if, for example, one wishes to study the $\theta$ dependence of the theory by
adding a term $i\theta Q$ to the continuum action and, correspondingly, a term $i\theta_L Q_L$
to the lattice action, since one sees that
%
\cite{Pisa_theta}
%
\be
\theta_L \simeq  Z^{-1}_Q(\beta) \theta.
\label{eqn_theta}
\ee
%
Primarily for this reason we have calculated  $Z_Q(\beta)$ in our lattice calculations
and will also provide interpolating functions that will give  $Z_Q(\beta)$ at values
of $\beta$ that lie between our measured values.

A first estimate for $Z_Q(\beta)$ can be obtained in perturbation theory, giving at one loop
%
\cite{Pisa_ZQ}
%
\be
Z_Q(\beta) \stackrel{\beta\to\infty}{=} 1 - (0.6612 N^2 - 0.5)\frac{1}{\beta} + O(\beta^{-2})
\label{eqn_Zpert}
\ee
%
which already tells us that for our range of $\beta$ we will have $Z_Q(\beta) \ll 1$. The
fact that the one loop correction is so large tells us that the one loop estimate is likely
to be not very accurate, and indeed that proves to be the case. Moreover as we increase the
lattice spacing the typical instanton becomes smaller and the value of $Q_L$ acquires
significant corrections that are powers of $a(\beta)$ and which are additional to any
perturbative corrections. So in constructing our interpolating function for $Z_Q$ we
simply use the form
%
\be
Z^{int}_Q(\beta) = 1 -  z_0 g^2N - z_1 (g^2N)^2 \quad ; \quad g^2N = \frac{2N^2}{\beta}, 
\label{eqn_Zint}
\ee
%
where $g^2N$ is the 't Hooft coupling, and we make no attempt 
to constrain the parameters $z_0$ and $z_1$ to perturbative values.
This turns out to be an adequate fitting function to our values of  $Z_Q(\beta)$. It
is however important to note that while this works as an interpolating function, it is
likely to fail increasingly badly the further one uses it away from the fitted range of $\beta$
as an extrapolating function.

Our values for $Z_Q(\beta)$ are obtained from fits such as those shown in Fig.\ref{fig_ZQ_su8},
and are listed in Tables~\ref{table_ZQA} and \ref{table_ZQB}.
Interpolating these values with eqn(\ref{eqn_Zint}) gives the values for $z_0$ and $z_1$ listed
in Table~\ref{table_ZQint}. Physically the most relevant interpolating function is the one
for $SU(3)$:
%
\be
Z^{int}_Q(\beta) \stackrel{su3}{=} 1 - 0.162(10)g^2N - 0.0425(31) (g^2N)^2 \quad , \quad \chi^2/n_{df} = 0.62
\label{eqn_Zintsu3}
\ee
%
In the case of $SU(2)$ we have only a few entries because most of our earliest calculations
did not include calculating $Q_L$ on the fields prior to cooling. In addition in the case
of $SU(2)$ the identification of an integer $Q$ after 20 cooling sweeps possesses small
but significant ambiguities, which rapidly disappear as $N$ increases. We can also
fit our interpolating functions in $N$, thus obtaining
%
\be
z_0 = 0.179(12) - \frac{0.08(15)}{N^2}   \quad , \quad \chi^2/n_{df} = 1.00
\label{eqn_Z0suN}
\ee
%
and
%
\be
z_1 = 0.0482(46) - \frac{0.072(50)}{N^2}   \quad , \quad \chi^2/n_{df} = 1.13
\label{eqn_Z1suN}
\ee
%
which should be reliable over a wide range of $N$ as long as we are not too far outside
the range of the t'Hooft coupling $\lambda=g^2N$ of our calculations in
Tables~\ref{table_ZQA} and \ref{table_ZQB}. Finally we remind the reader that all these
results for $Z_Q(\beta)$ only apply to calculations with the standard Wilson plaquette action
and with the definition of the lattice topological charge $Q_L$ used here.





%
%
%
%
\section{Conclusions}
%\section{Discussion}
\label{section_conclusion} 

Our primary goal in this paper has been to calculate the glueball spectra of a range
of $SU(N)$ gauge theories, in the continuum limit, with enough precision to obtain plausible
extrapolations to the theoretically interesting $N=\infty$ limit. This provides
the first calculation of the masses of the ground states 
in all the $R^{PC}$ channels, as well as some excited states in most channels,
in the continuum limit of the $N\to\infty$ gauge theory.

Our results, for $N=2,3,4,5,6,8,10,12$, were obtained using standard lattice gauge theory 
methods. Although the issue of topological freezing at larger $N$ in $SU(N)$ gauge theories 
is not expected to be important for glueball spectra
%
\cite{Witten_98,LDDGMEV_Q},
%
we confirmed this explicitly in some extensive $SU(8)$ calculations, and in addition
we chose to minimise any remnant bias at larger $N$ by modifying the usual update algorithm, 
explicitly imposing the expected distribution of topological charge on the starting 
lattice fields of our ensemble of Monte Carlo sequences.
We employed a large basis of single-trace glueball operators, which allowed us
to calculate the ground state and some excited states for each of the $R^{PC}$ channels,
where $R$ labels the representation of the rotation symmetry group appropriate
to our cubic lattice, and $P,C$ label the parity and charge conjugation. The large basis
gives us confidence that we are not missing any low-lying states and this in turn
allows us to match near-degeneracies between states with different $R$ so as to
assign continuum spin quantum numbers to a significant number of glueball states.

Our results have greatly extended existing calculations while largely confirming existing
results; in particular the important conclusion that $SU(3)$ is `close to' $SU(\infty)$.
As before, one finds that the lightest glueball is the $J^{PC}=0^{++}$ scalar ground state,
with a mass that ranges from $M_{0^{++}}\sim 3.41\surd\sigma$ for $SU(3)$ to
$M_{0^{++}}\sim 3.07\surd\sigma$ for $SU(\infty)$, where $\sigma$ is the string tension,
and that the next heavier glueballs are the tensor with a mass $M_{2^{++}} \simeq 1.5 M_{0^{++}}$,
and the pseudoscalar, $0^{-+}$, which is nearly degenerate with the tensor. One then
has the $1^{+-}$ with $M_{1^{+-}} \sim 1.85 M_{0^{++}}$, and this is the only relatively light
$C=-$ state. At roughly the same mass is the first excited $0^{++}$ and then the lightest
pesudotensor with $M_{2^{-+}} \sim 1.95 M_{0^{++}}$. All other states are heavier than twice
the lightest scalar, with most of the $C=-$ ground states being very much heavier than that.
One sees a number of near-degerenacies which may or may not be significant.
The continuum glueball masses (in units of the string tension) for the various $R^{PC}$
channels are listed in Tables~\ref{table_MK_R_SU2}-\ref{table_MK_R_SUN} and for the
$J^{PC}$ channels in Tables~\ref{table_MKJ_N2-5}-\ref{table_MK_J_SUN}.

Since our calculations are on a finite spatial volume we have had to identify and exclude
the `ditorelon' states composed of a pair of mutually conjugate flux tubes that wind
around our periodic spatial torus. These are states whose projection onto our single
trace operators will vanish as $N\to\infty$. We also need to exclude any multiglueball
scattering states. Our (albeit partial) explicit calculations using the corresponding
double trace operators strongly suggest that these states do not appear in the glueball
spectra that we claim to obtain using our single trace basis.

In calculating the glueball spectra we have also calculated a number of other quantities
that could be calculated simultaneously. Our calculations of the string tension were
primarily intended to provide a scale in which to express our glueball masses.
However they also provided a scale for the lattice spacing $a$ at each value of the
bare lattice coupling, $g_L(a)$, which we were able to use to obtain values of the
dynamical scale $\Lambda_{\overline{MS}}$ for all our values of $N$. These improved
upon earlier calculations of this kind
and for $SU(3)$ provided a value in physical units of
$\Lambda_{\overline{MS}} \simeq 263(4)[9]{\mathrm{MeV}}$ which
is consistent with values obtained using more dedicated methods
At the same time we were able to confirm that keeping fixed the running 't Hooft coupling 
$g^2(a)N$, with $a$ being kept fixed in units of the string tension, is the way to
approach the $N=\infty$ limit along lines of constant physics.

In addition to the fundamental string tension we calculate the tension of $k=2$ flux tubes,
in order to analyse the way that $\sigma_{k=2}/\sigma$ approaches $N=\infty$. We find that
a leading $O(1/N)$ correction works better than the $O(1/N^2)$ expected from standard
large-$N$ counting, but that the latter cannot be completely excluded. We speculate
that the  $O(1/N)$ behaviour is sub-asymptotic, with the $O(1/N^2)$ correction 
settling in for flux tubes of length $l\leq l_c$ once $N$ is large enough that the $k=2$ 
flux tube becomes a pair of weakly interacting fundamental flux tubes for $l\leq l_c$
with  $l_c\uparrow$ as  $N\uparrow$. That is to say, the $l,N\to\infty$ limit is
not uniform.

Since for our glueball calculations  we need to monitor the onset of the freezing
of the topology of our lattice fields, we have performed extensive calculations of the
topological charge along with the glueball calculations. Using these calculations 
we obtain values for the continuum limit of the topological susceptibility for the
$SU(N\leq 6)$ gauge theories. The freezing of topology means that we have no values
for $N > 6$ or for our smaller $SU(6)$ lattice spacings. Nonetheless this does not
prevent us achieving a usefully precise value of the $N=\infty$ topological suseptibility,
$\chi_I^{\frac{1}{4}}/\surd\sigma = 0.3681(28)$. We have also calculated the renormalisation
factor $Z_Q(\beta)$ that relates the value of our particular (but standard) lattice topological
charge, as obtained on the Monte carlo generated lattice gauge fields, to the true integer
valued topological charge of those fields. This is useful if one wishes to include
a topological $\theta$-term in the action, and so we also include functions that
interpolate between our values of $\beta$. We can do so for all our $SU(N)$ groups
because these calculations can be equally well determined using ensembles of fields
where the topological charge has been inserted through the initial fields of the
Markovian chains.

The present study could be improved in several ways. A definitive study of ditorelon states
and the lightest multiglueball states for all $R^{PC}$ sectors would be useful.
The heaviest states need a finer resolution in the correlation functions for the
mass identification to become completely unambiguous: this could be achieved by using
an anisotropic lattice such that the timelike lattice spacing is much smaller than the
spacelike one, a technique that has occasionally been put to good purpose in the past
%
\cite{KIGSMT-SU2-1983,KIGSMT-SU3-1983,MP-1999,MP-2005}.
%
Perhaps most important would be the incorporation of more effective techniques for
determining the continuum spins of the glueball states, as for example in
%
\cite{PCSDMT-2019}
%
for 2 space dimensions and in
%
\cite{HM_Thesis}
%
for our case here of 3 spatial dimensions.


%
%
%
%
\section*{Acknowledgements}

AA has been financially supported by the European Union's Horizon 2020 research and innovation
programme ``Tips in SCQFT'' under the Marie Sk\l odowska-Curie grant agreement No. 791122.
MT acknowledges support by Oxford Theoretical Physics and All Souls College. The numerical
computations were carried out on the computing cluster in Oxford Theoretical Physics.

\clearpage



\begin{appendix}
\setcounter{table}{0}
\renewcommand{\thetable}{A\arabic{table}}


\section{Lattice running couplings}
\label{section_appendix_couplings}

For pure gauge theories the perturbative $\beta$ function for the running coupling
in a coupling scheme $s$ is given by
%
\be
\beta(g_s)
=
-a\frac{\partial g_s}{\partial a}
=
-b_0  g^3_s - b_1  g^5_s - b^s_2  g^7_s + ...,
\label{eqn_bfunction2}
\ee
%
where $a$ is the length scale on which the coupling is calculated and on which it
depends.  The first two coefficients $b_0$ and $b_1$ are scheme independent while the
coefficients $b^s_{j\geq 2}$ depend on the scheme. Integrating between scales $a_0$
and $a$, we obtain
%
\be
\frac{a}{a_0} 
=
\exp{\left(-\int^{g(a)}_{g(a_0)} \frac{dg}{\beta(g)}\right)}.
\label{eqn_aa0}
\ee
%
(The label $s$ on $g$ is to be understood.)
The integrand is singular as $g\to 0$ and for any calculations it is convenient
to separate out the singular pieces. Since the issue arise for  $g\to 0$ we
can expand $1/\beta(g)$ in powers of $g$ and we then readily see that we can
separate out the singular terms as follows,
%
\beq
\frac{a}{a_0} 
& = &
e^{+\int^{g(a)}_{g(a_0)} dg \left(
 - \frac{b_1}{b^2_0 g} + \frac{1}{b_0 g^3}
 \right) }
e^{-\int^{g(a)}_{g(a_0)} dg \left(\frac{1}{\beta(g)}
 - \frac{b_1}{b^2_0 g} + \frac{1}{b_0 g^3}
 \right) }       \nonumber \\
& = &
\left(\frac{g^2(a)}{g^2(a_0)}\right)^{-\frac{b_1}{2b^2_0}}
e^{-\frac{1}{2b_0}\left(\frac{1}{g^2(a)}-\frac{1}{g^2(a_0)}\right)}
e^{-\int^{g(a)}_{g(a_0)} dg \left(\frac{1}{\beta(g)}
 - \frac{b_1}{b^2_0 g} + \frac{1}{b_0 g^3}
 \right) }.
\label{eqn_aa0b}
\eeq
%
In the second line we have integrated the `singular' terms, and the remaining
integral will now be finite as $g(a_0) \to 0$. So we can break up the integral as
%
\be
e^{-\int^{g(a)}_{g(a_0)} dg \left(\frac{1}{\beta(g)}
 - \frac{b_1}{b^2_0 g} + \frac{1}{b_0 g^3}
 \right) }
=
e^{+\int^{g(a_0)}_0 dg \left(\frac{1}{\beta(g)}
 - \frac{b_1}{b^2_0 g} + \frac{1}{b_0 g^3}
 \right) }
e^{-\int^{g(a)}_0 dg \left(\frac{1}{\beta(g)}
 - \frac{b_1}{b^2_0 g} + \frac{1}{b_0 g^3}
 \right) }
\label{eqn_aa0c}
\ee
where each integral will be well-defined since the singularities at $g=0$
have been removed. Separating the terms in $a$ and $a_0$ in
eqns(\ref{eqn_aa0b},\ref{eqn_aa0c}) we can write
%
\be
\frac{a}{a_0}  = \frac{F(g(a))}{F(g(a_0))},
\label{eqn_aa0d}
\ee
%
where we define
%
\be
F(g)
\equiv
\left(b_0g^2\right)^{-\frac{b_1}{2b^2_0}}
e^{-\frac{1}{2b_0g^2}}
e^{-\int^{g}_0 dg \left(\frac{1}{\beta(g)}
 - \frac{b_1}{b^2_0 g} + \frac{1}{b_0 g^3}
 \right) }.
\label{eqn_aa0e}
\ee
%
Note that the factor of $b_0$ that we have inserted in the first term on the right
of eqn(\ref{eqn_aa0e}) will cancel in eqn(\ref{eqn_aa0d}) so we are free to insert
it if we wish. We see from eqn(\ref{eqn_aa0d}) that $a/F(g^2(a))$ is
independent of the scale $a$ on which the coupling is defined and is a constant.
So we can now define a dynamical energy scale $\Lambda$ by
%
\beq
\Lambda \equiv \frac{a_0}{F(g(a_0))}
\Longrightarrow
a & = & \frac{1}{\Lambda} F(g(a)) \nonumber \\
& = &
\frac{1}{\Lambda} 
\left(b_0g^2\right)^{-\frac{b_1}{2b^2_0}}
e^{-\frac{1}{2b_0g^2}}
e^{-\int^{g(a)}_0 dg \left(\frac{1}{\beta(g)}
 - \frac{b_1}{b^2_0 g} + \frac{1}{b_0 g^3}
 \right) }.
\label{eqn_aa0f}
\eeq
%
The scale $\Lambda$ defined here coincides with the conventional $\Lambda$ parameter
that appears in the standard 2-loop expression for the running coupling.
(It is to ensure this equality that we inserted the factor of $b_0$ above.)
This scale will clearly depend on the coupling scheme and within a given scheme
the value we obtain for it will depend on our approximation to $\beta(g)$. 

For our lattice action  only the first 3 coefficients in the
$\beta$-function are known. In that case, collecting terms, it is convenient
to rewrite eqn(\ref{eqn_aa0f}) as
%
\be
a
\stackrel{3 loop}{=}
\frac{1}{\Lambda} 
\left(b_0g^2(a)\right)^{-\frac{b_1}{2b^2_0}}
e^{-\frac{1}{2b_0g^2(a)}}
e^{-\frac{1}{2} \int^{g^2(a)}_0 dg^2
\left(\frac{b_0b_2-b^2_1-b_1b_2g^2}{b^3_0+b^2_0b_1g^2+b^2_0b_2g^4}
\right) }
\equiv
\frac{1}{\Lambda} F_{3l}(g(a)),
\label{eqn_ag3loop}
\ee
%
where we denote by $F_{3l}(g)$ the 3-loop approximation to $F(g)$.
The integrand is a smooth function of $g^2$ and so the integral can be calculated
accurately for any given $g(a)$ using any elementary numerical integration
method. If we retain only the first 2 coefficients of $\beta(g)$ then we can
do the integral analytically to obtain
%
\be
a
\stackrel{2 loop}{=}
\frac{1}{\Lambda} 
e^{-\frac{1}{2b_0g(a)^2}}
\left(\frac{b_1}{b^2_0}+\frac{1}{b_0g^2(a)}\right)^{\frac{b_1}{2b^2_0}}
\equiv
\frac{1}{\Lambda} F_{2l}(g(a))
\label{eqn_ag2loop}
\ee
%
as the exact 2-loop result, where we denote by $F_{2l}(g)$ the 2-loop approximation to
$F(g)$.

The above perturbative expressions for $a\Lambda$ can be turned into expressions
for $a\mu$ where $\mu$ is some physical mass or energy:
%
\be
a\mu
=
\left.a\Lambda \frac{\mu}{\Lambda}\right|_{a}
=
\left.\frac{\mu}{\Lambda}\right|_{a=0} (1 + c_{\mu}a^2\mu^2 + O(a^4)) F(g(a)),
\label{eqn_amu}
\ee
%
where $F(g(a))$ is defined in eqn(\ref{eqn_aa0e}) and $c_{\mu}$ is an unknown constant.
Here we use the standard tree-level expansion for a dimensionless ratio of physical
energy scales, which here is $\mu/\Lambda$. This expression marries perturbative
(logarithmic)
and power-like dependences on $a$ in a plausible way. It is of course arguable: for
example the perturbative expansion is at best asymptotic and this can introduce other
power-like corrections. In practice we shall use this for the string tension,
$\mu=\surd\sigma$, and we will drop $O(a^4)$ and higher order terms.
That is to say we will attempt to fit the calculated string tensions with
%
\be
a\surd\sigma
=
\left.\frac{\surd\sigma}{\Lambda}\right|_{a=0} (1 + c_{\sigma}a^2\sigma) F_{3l}(g(a))
\label{eqn_aK}
\ee
%
and we shall be doing so in the mean-field coupling scheme $s=I$. To calculate
$F_{3l}(g(a))$ in that scheme we only need to know the coefficient $b_2^{s=I}$,
since $b_0$ and $b_1$ are scheme independent with values
%
\be
b_0={\frac{1}{(4\pi)^2}} {\frac{11}{3}} N, \quad 
b_1={\frac{1}{(4\pi)^4}} {\frac{34}{3}} N^2.
\label{eqn_b0b1}
\ee
% 
We can begin with the well-known value of $b_2^{s=\overline{MS}}$
%
\cite{MSbar_4loop}
%
\be
b^{\overline{MS}}_2={\frac{1}{(4\pi)^6}} {\frac{2857}{54}} N^3.
\label{eqn_b2MS}
\ee
%
To obtain $b^s_2$ in the improved lattice coupling scheme $I$ we first transform
to the plaquette action lattice coupling scheme, $s=L$, using
%
\cite{betaL_3loop}
%
\be
b^L_2=2b^{\overline{MS}}_2 - b_1l_0 +b_0l_1,
\label{eqn_b2La}
\ee
% 
where
%
\be
l_0=\frac{1}{8N} - 0.16995599 N, \qquad
l_1=-\frac{3}{128N^2}+0.018127763-0.0079101185N^2.
\label{eqn_b2Lb}
\ee
% 
The lattice coupling, $g^2_L$, satisfies the $\beta$-function in eqn(\ref{eqn_bfunction2}) with
$s=L$, and the mean-field coupling will satisfy a $\beta$-function with the same $b_0$ and $b_1$
coefficients but with a different coefficient, $b^I_2$, of the $g^7$ power. To
determine $b^I_2$ one can use the expression for  $\langle \mathrm{Tr} U_p \rangle$
as a power series in $g^2_L$ to write $g^2_I$ as
%
\be
g^2_I = g^2_L \langle \frac{1}{N} \mathrm{Tr} U_p \rangle
=  g^2_L \left(1-w_1g^2_L-w_2g^4_L- ...\right),
\label{eqn_gIgL}
\ee
% 
where
%
\cite{plaq_pert}
%
\be
w_1=\frac{(N^2-1)}{8N}, \qquad
w_2=\left(N^2-1\right)\left(0.0051069297 -\frac{1}{128N^2}\right).
\label{eqn_plaq_pert}
\ee
% 
We now insert the expression for $g^2_I$ in eqn(\ref{eqn_gIgL}) into the $\beta$-function
for $g^2_L$ in eqn(\ref{eqn_bfunction2}) giving us, after some elementary manipulation,
%
\be
b^I_2=b^L_2+w_2b_0-w_1b_1.
\label{eqn_b2I}
\ee
% eqn
So: using eqns(\ref{eqn_b2MS},\ref{eqn_b2Lb}) in eqn(\ref{eqn_b2La}) we obtain the
explicit expression for $b^L_2$ and then inserting that together with the functions in
eqn(\ref{eqn_plaq_pert}) and eqn(\ref{eqn_b0b1}) into eqn(\ref{eqn_b2I}) we
obtain the explicit expression for $b^I_2$ for any $N$. We now have the explicit
expressions for $b_0$, $b_1$ and $b^I_2$ which allow us to calculate the
value of $F_{3l}(g_I(a))$ in eqn(\ref{eqn_aK}) for any value of $N$ and $g_I(a)$.





\clearpage

\setcounter{table}{0}
\renewcommand{\thetable}{B\arabic{table}}

\section{Scattering states}
\label{section_appendix_scattstates}

We will restrict ourselves to states of two glueballs. We probe such states with product
operators $\phi_a(t)\phi_b(t)$ where $\phi_a$ and $\phi_b$ are typical single loop zero momentum operators
that are expected to project primarily onto single glueballs with chosen quantum numbers. Both the
individual and product operators will in general need vacuum subtraction for this to be the case. As usual
we will have some basis of single loop operators so the generic correlator will be of the form
%
\be
C_4(t) = \langle \phi'_a(t)\phi'_b(t) \phi'_c(0)\phi'_d(0) \rangle
-\langle \phi'_a\phi'_b\rangle \langle \phi'_c\phi'_d\rangle,
\label{eqn_corGGGGa}
\ee
%
where $\phi'_a(t) = \phi_a(t) - \langle \phi_a\rangle$ etc. This equation subtracts any vacuum
contribution to the individual operators as well as to their products. We have taken the operators
to be real. If, for example, we were to include operators  $\phi_i(t),\phi_j(t)$ with opposite
non-zero momenta then these would be complex. In that case we should change
$\phi_a(t),\phi_b(t) \to \phi^{\dagger}_a(t),\phi^{\dagger}_b(t)$ in eqn(\ref{eqn_corGGGGa}) and below.

Our glueball calculations require high statistics so we calculate our correlators during the
generation of the sequence of lattice fields. At this stage we can only calculate the correlators
of the fields $\phi$ without any vacuum subtraction -- we will only be able to calculate the
vacuum expectation values at the end of the computer simulation.
A short calculation tells us that the correlator of the $\phi'_i$ fields in eqn(\ref{eqn_corGGGGa})
can be written in terms of the correlators of the unsubtracted $\phi_i$ fields  as follows:
%
\begin{multline}
C_4(t) = \langle \phi_a(t)\phi_b(t) \phi_c(0)\phi_d(0) \rangle \\
-\langle\phi_a\rangle \langle\phi_b(t)\phi_c(0)\phi_d(0) \rangle
-\langle\phi_b\rangle \langle\phi_a(t)\phi_c(0)\phi_d(0) \rangle\\
-\langle\phi_c\rangle \langle\phi_a(t)\phi_b(t)\phi_d(0) \rangle
-\langle\phi_d\rangle \langle\phi_a(t)\phi_b(t)\phi_c(0) \rangle \\
+\langle\phi_a\rangle \langle\phi_c\rangle \langle \phi_b(t)\phi_d(0) \rangle
+\langle\phi_b\rangle \langle\phi_c\rangle \langle \phi_a(t)\phi_d(0) \rangle \\
+\langle\phi_a\rangle \langle\phi_d\rangle \langle \phi_b(t)\phi_c(0) \rangle
+\langle\phi_b\rangle \langle\phi_d\rangle \langle \phi_a(t)\phi_c(0) \rangle \\
-\langle\phi_a\phi_b\rangle \langle\phi_c\phi_d\rangle
+2\langle\phi_a\phi_b\rangle \langle\phi_c\rangle \langle\phi_d\rangle
+2\langle\phi_c\phi_d\rangle \langle\phi_a\rangle \langle\phi_b\rangle
-4\langle\phi_a\rangle \langle\phi_b\rangle \langle\phi_c\rangle \langle\phi_d\rangle.
\label{eqn_corGGGGb}
\end{multline}
%
Of course when some of the operators have non-vacuum quantum numbers, then the corresponding
vacuum expectation values will vanish and the above expression will simplify in obvious ways.

In addition to the above we may also be interested in the overlap of single loop operators,
which mainly project onto single glueballs, with the above product operators, which one
expects to mainly project onto two glueballs. That is to say correlators such as
%
\be
C_3(t) = \langle \phi'_a(t)\phi'_b(t) \phi'_c(0) \rangle
-\langle \phi'_a\phi'_b\rangle \langle \phi'_c\rangle
= \langle \phi'_a(t)\phi'_b(t) \phi'_c(0) \rangle
\label{eqn_corGGGa}
\ee
%
since $\langle \phi'_c\rangle=0$ by definition. In terms of the unsubtracted operators we find
%
\begin{multline}
C_3(t) = \langle \phi_a(t)\phi_b(t) \phi_c(0) \rangle 
-\langle\phi_a\rangle \langle\phi_b(t)\phi_c(0) \rangle
-\langle\phi_b\rangle \langle\phi_a(t)\phi_c(0) \rangle \\
-\langle\phi_c\rangle \langle\phi_a\phi_b\rangle
+2\langle\phi_a\rangle \langle\phi_b\rangle \langle\phi_c\rangle. 
\label{eqn_corGGGb}
\end{multline}
%

In this paper we can only calculate light masses with any reliability so we restrict ourselves
to double loop operators where one loop is in the $A_1^{++}$ representation and therefore has
some projection onto the lightest glueball. The second operator will then be in the representation
$R^{PC}$ in which we happen to be interested. If $R^{PC} \neq A_1^{++}$ then eqn(\ref{eqn_corGGGGb})
simplifies drastically 
%
\begin{multline}
C_4(t) = \langle \phi_a(t)\phi_b(t) \phi_c(0)\phi_d(0) \rangle \\
-\langle\phi_a\rangle \langle\phi_b(t)\phi_c(0)\phi_d(0) \rangle
-\langle\phi_c\rangle \langle\phi_a(t)\phi_b(t)\phi_d(0) \rangle
+\langle\phi_a\rangle \langle\phi_c\rangle \langle \phi_b(t)\phi_d(0) \rangle
-\langle\phi_a\phi_b\rangle \langle\phi_c\phi_d\rangle
\label{eqn_corGGGGd}
\end{multline}
%
since with $\phi_a,\phi_c$ being $A_1^{++}$ and $\phi_b,\phi_d$ not being $A_1^{++}$ means
that not only $\langle \phi_b\rangle = \langle \phi_d\rangle = 0$ but also that
products like  $\langle \phi_a \phi_b\rangle$ are zero. Similarly if  $\phi_a$ in
eqn(\ref{eqn_corGGGa}) is $A_1^{++}$ and  $\phi_b$ is in some $R^{PC} \neq A_1^{++}$ then
$\phi_c$ must be in the same $R^{PC} \neq A_1^{++}$ for $C_3(t)$ not to be zero.
In that case we will have
%
\be
C_3(t) = \langle \phi_a(t)\phi_b(t) \phi_c(0) \rangle 
-\langle\phi_a\rangle \langle\phi_b(t)\phi_c(0) \rangle.
\label{eqn_corGGGc}
\ee
%
If however $\phi_b$ and $\phi_c$ are in $A_1^{++}$ then we need the full expression.


%\clearpage


\end{appendix}

\clearpage




%
%
%
\begin{thebibliography}{99}


 \bibitem{BLMT_N}
B. Lucini, M. Teper,
  `{\it  SU(N) gauge theories in four dimensions: exploring the approach to N = infinity}',
   JHEP 0106:050,2001 [arXiv:hep-lat/0103027].

\bibitem{BLMTUW_N}
B. Lucini, M. Teper, U. Wenger,
`{\it Glueballs and k-strings in SU(N) gauge theories: Calculations with improved operators}',
JHEP 0406:012,2004 [arXiv:hep-lat/0404008].

\bibitem{BLARER_N}
B. Lucini, A. Rago, E. Rinaldi,
`{\it Glueball masses in the large N limit}',
JHEP 1008:119,2010 [arXiv:1007.3879].

\bibitem{SpN_Lucini}
  E. Bennett, J. Holligan, D.K. Hong, J-W Lee, C-J D. Lee, B. Lucini, M. Piai, D. Vadacchino,
  `{\it Glueballs and strings in Sp(2N) Yang-Mills theories}',
   Phys.Rev.D 103 (2021) 054509 [arXiv:2010.15781].
 
\bibitem{CMMT-1989}
  C. Michael, M. Teper,
  `{\it The Glueball Spectrum in SU(3)}',
  Nucl.Phys.B 314 (1989) 347

\bibitem{CM-UKQCD-1993}
  G.S. Bali, K. Schilling, A. Hulsebos, A.C. Irving, C. Michael, P.W. Stephenson,
  `{\it A comprehensive lattice study of SU(3) glueballs}',
  Phys.Lett.B309(1993)378 [arXiv:hep-lat/9304012].
  
\bibitem{MP-1999}
  C.J. Morningstar, M.J. Peardon,
  `{\it The Glueball spectrum from an anisotropic lattice study}',
  Phys.Rev.D 60 (1999) 034509 [arXiv:hep-lat/9901004].
  
\bibitem{HMMT-2004}
  H.B. Meyer, M. Teper,
  `{\it Glueball Regge trajectories and the Pomeron -- a lattice study}',
  Phys.Lett.B605(2005)344 [arXiv:hep-ph/0409183].
  
\bibitem{HM_Thesis}
  H.B. Meyer,
  `{\it Glueball Regge trajectories}',
  Oxford D.Phil Thesis 2004 [arXiv:hep-lat/0508002].

\bibitem{MP-2005}
  Y. Chen, A. Alexandru, S.J. Dong, T. Draper, I. Horvath, F.X. Lee, K.F. Liu, N. Mathur,
C. Morningstar, M. Peardon, S. Tamhankar, B.L. Young, J.B. Zhang,
  `{\it Glueball Spectrum and Matrix Elements on Anisotropic Lattices}',
  Phys.Rev.D 73 (2006) 014516 [arXiv:hep-lat/0510074].
  
\bibitem{AAMT-2020}
  A. Athenodorou, M. Teper,
  `{\it The glueball spectrum of SU(3) gauge theory in 3+1 dimensions}',
  JHEP 11 (2020) 172 [arXiv:2007.06422].

\bibitem{MGPAGAMO-2020}
  M.G. Perez, A. Gonzalez-Arroyo, M. Okawa,
  `{\it Meson spectrum in the large N limit}',
  JHEP 04 (2021) 230 [arXiv:2011.13061].

\bibitem{MGPAGAMKMO-2018}
  M.G. Perez, A. Gonzalez-Arroyo, M. Koren, M. Okawa,
  `{\it The spectrum of 2+1 dimensional Yang-Mills theory on a twisted spatial torus}',
  JHEP 07 (2018) 169 [arXiv:1807.03481].

\bibitem{BLMTUW_Tc}
  B. Lucini, M. Teper, U. Wenger,
  `{\it The high temperature phase transition in SU(N) gauge theories }',
  JHEP 0401 (2004) 061 [arXiv:hep-lat/0307017].

\bibitem{BLMTUW05}
  B. Lucini, M. Teper, U. Wenger,
  `{\it Properties of the deconfining phase transition in SU(N) gauge theories}',
  JHEP 0502 (2005) 033 [arXiv:hep-lat/0502003].


\bibitem{Sommer-cutoff}
  N. Husung, P. Marquard, R. Sommer,
  `{\it Asymptotic behavior of cutoff effects in Yang-Mills theory and in Wilson's lattice QCD}',
  Eur.Phys.J.C. 80(2020)3,200, [arXiv:1912.08498].
  
\bibitem{MT-block1}
  M. Teper,
  `{\it An improved method for lattice glueball calculations}',
  Phys. Lett. B183 (1987) 345.
  
\bibitem{MT-block2}
  M. Teper, 
  `{\it The scalar and tensor glueball masses in lattice gauge theory}',
  Phys. Lett. B185 (1987) 121.

\bibitem{Luscher-V}
  M. L\"uscher,
  `{\it Volume Dependence of the Energy Spectrum in Massive Quantum Field Theories. 1. Stable Particle States}',
  Commun.Math.Phys. 104 (1986) 177.

  
\bibitem{tHooft_Q}
  G. 't Hooft,
 `{\it Computation of the quantum effects due to a four-dimensional pseudoparticle}',
  Phys. Rev. D 14, 3432 (1976); Erratum Phys. Rev. D 18, 2199 (1978).
  
\bibitem{Coleman_Q}
  S. Coleman,
  `{\it The uses of instantons}',
  Chapter 7 in `{\it Aspects of Symmetry}',  Cambridge University Press, 1985.

  
\bibitem{tHooft_N}
  G. 't Hooft,
 `{\it A planar diagram theory for strong interactions}'. 
  Nucl. Phys. B72 (1974) 461.

\bibitem{Coleman_N}
  S. Coleman,
  `{\it 1/N}',
  Chapter 8 in `{\it Aspects of Symmetry}',  Cambridge University Press, 1985.

\bibitem{Witten_N}
  E. Witten,
  `{\it Baryons in the 1/n expansion}',
  Nucl.Phys.B 160 (1979) 57.

  
\bibitem{Witten_Q}
  E. Witten,
  `{\it Instantons, the Quark Model, and the 1/n Expansion}',
  Nucl.Phys.B 149 (1979) 285

\bibitem{Teper_Q}
  M. Teper, 
  `{\it Instantons and the 1/N expansion}',
  Z.Phys.C 5 (1980) 233.

\bibitem{LDD_K2}
  L. Del Debbio, H. Panagopoulos, P. Rossi, E. Vicari,
  `{\it k string tensions in SU(N) gauge theories }',
 Phys.Rev.D 65 (2002) 021501 [arXiv:hep-th/0106185].


\bibitem{Witten_98}
  E. Witten,
  `{\it Theta dependence in the large N limit of four-dimensional gauge theories }',
  Phys.Rev.Lett. 81 (1998) 2862 [arXiv:hep-th/9807109].
  

\bibitem{LDDGMEV_Q}
  L. Del Debbio, H. Panagopoulos, E.Vicari,
  `{\it theta dependence of SU(N) gauge theories}',
  JHEP 08 (2002) 044 [arXiv:hep-th/0204125].

\bibitem{Aoki}
  S. Aoki, H. Fukaya, S. Hashimoto, T. Onogi,
  `{\it Finite volume QCD at fixed topological charge}',
  Phys. Rev. D76 (2007) 054508, [arXiv:0707.0396].
  
\bibitem{string_1}
  O. Aharony and Z. Komargodski,
  `{\it The Effective Theory of Long Strings}',
  JHEP 1305 (2013) 118 [arXiv:1302.6257].

\bibitem{string_2}
  S. Dubovsky, R. Flauger, V. Gorbenko,
  `{\it Flux Tube Spectra from Approximate Integrability at Low Energies}',
  J.Exp.Theor.Phys. 120 (2015) 399 [arXiv:1404.0037].

\bibitem{string_3}
  M. Luscher, P. Weisz,
  `{\it String excitation energies in SU(N) gauge theories beyond the free-string approximation}',
  JHEP 0407:014 (2004), [arXiv:hep-th/0406205].
  itv player
  
\bibitem{string_4}
  J.M. Drummond
  `{\it Universal subleading spectrum of effective string theory}',
  arXiv:hep-th/0411017.

\bibitem{AABBMT_K}
  A. Athenodorou, B. Bringoltz, M. Teper,
  `{\it Closed flux tubes and their string description in D=3+1 SU(N) gauge theories }',
  JHEP 02 (2011) 030 [arXiv:1007.4720].

\bibitem{BLMT_K2}
  B. Lucini, M. Teper,
  `{\it Confining strings in SU(N) gauge theories }',
  Phys.Rev.D64 (2001) 105019 [arXiv:hep-lat/0107007]


\bibitem{Hasenfratz}
  A. Hasenfratz and P. Hasenfratz,
  `{\it The Connection Between the Lambda Parameters of Lattice and Continuum QCD}',
  Phys. Lett. 93B (1980)165.

  
\bibitem{Dashen-Gross}
  R. Dashen and D. Gross,
`{\it The Relationship Between Lattice and Continuum Definitions of the Gauge Theory Coupling}',
  Phys. Rev. D23 (1981) 2340

\bibitem{CSlattice}
  J. Ambjorn, P. Olesen, C. Peterson,
  `{\it Stochastic confinement and dimensional reduction: (I). Four-dimensional SU(2) lattice gauge theory}',
  Nucl. Phys. B240 (1984) 189, 533.
    
\bibitem{gimp_review}
  G. P. Lepage,
  `{\it Redesigning Lattice QCD }',
  Schladming Winter School lectures 2006  [arXiv:hep-lat/9607076].

\bibitem{MF_Parisi}
  G. Parisi,
  `{\it Recent Progresses in Gauge Theories}',
  20th International Conference on High-Energy Physics 1980, AIP Conf.Proc. 68 (1980) 1531.

\bibitem{CAMTAT}
  C. Allton, M. Teper, A. Trivini,
  `{\it On the running of the bare coupling in SU(N) lattice gauge theories}',
  JHEP 07 (2008) 021 [arXiv:0803.1092].

\bibitem{SF1}
 M. L\"uscher, R. Sommer, P. Weisz and U. Wolff,
`{\it A Precise Determination of the Running Coupling in the SU(3) Yang-Mills Theory}',
 Nucl. Phys. B413 (1994) 48 [hep-lat/9309005].

\bibitem{SF2}
  S. Capitani, M. L\"uscher, R. Sommer and H. Wittig,
  `{\it Non-perturbative quark mass renormalization in quenched lattice QCD }',
  Nucl. Phys. B544 (1999) 669 [hep-lat/9810063].

\bibitem{CAMTAT2}
  C. Allton,
  `{\it Lattice Monte Carlo data versus Perturbation Theory}',
  arXiv:hep-lat/9610016.

\bibitem{Sommer-r0}
  R. Sommer,
  `{\it A new way to set the energy scale in lattice gauge theories and its applications to the static force and $\alpha_s$ in SU(2) Yang-Mills theory}',
  Nucl. Phys. B411 (1994) 839 [arXiv:hep-lat/9310022].

\bibitem{Sommer-r0b}
  R. Sommer,
  `{Scale setting in lattice QCD}',
  arXiv:1401.3270.

\bibitem{LamMS_SU3a}
  Ken-Ichi Ishikawa, I Kanamori, Y. Murakami, A. Nakamura, M. Okawa, R. Ueno,
  `{\it Non-perturbative determination of the $\Lambda$-parameter in the pure SU(3) gauge theory
    from the twisted gradient flow coupling }',
  JHEP 12 (2017) 067 [arXiv:1702.06289].
  
\bibitem{LamMS_SU3b}
  N. Husung, M. Koren, P. Krah, R. Sommer,
  `{\it SU(3) Yang Mills theory at small distances and fine lattices}',
  EPJ Web Conf. 175 (2018) 14024  [arXiv:1711.01860].

\bibitem{KIGSMT-SU2-1983}
  K. Ishikawa, G. Schierholz, M. Teper,
  `{\it Calculation of the Glueball Mass Spectrum of SU(2) and SU(3) Non-Abelian
    Lattice Gauge Theories I: Introduction and SU(2)}'
  Z.Phys.C 19 (1983) 327.

\bibitem{KIGSMT-SU3-1983}
  K. Ishikawa, A. Sato, G. Schierholz, M. Teper,
  `{\it Calculation of the Glueball Mass Spectrum of SU(2) and SU(3) Non-Abelian
    Lattice Gauge Theories II: SU(3)}'
  Z.Phys.C 21 (1983) 167.

\bibitem{Heller_bulk}
  U. Heller
  `{\it SU(3) lattice gauge theory in the fundamental-adjoint plane and scaling along the
    Wilson axis}'
  Phys. Lett. B362(1995)123 [arXiv:hep-lat/9508009].

 \bibitem{Symanzik_cont}
   P. Weisz,
   `{\it Renormalization and lattice artifacts }',
   Les Houches Summer School 2009, arXiv:1004.3462.

\bibitem{PCSDMT-2019}
  P. Conkey, S. Dubovsky, M. Teper,
  `{\it Glueball Spins in D=3 Yang-Mills}',
  JHEP 10 (2019) 175 [arXiv:1909.07430].

\bibitem{DiVecchia-FFD}
  P.Di Vecchia, K.Fabricius, G.C.Rossi, G.Veneziano,
  `{\it   Preliminary evidence for $U_A(1)$ breaking in QCD from lattice calculations}',
  Nucl. Phys. B192 (1981) 392.
  
\bibitem{Pisa_ZQ}
  M. Campostrini, A. Di Giacomo and H. Panagopoulos,
  `{\it   The Topological Susceptibility on the Lattice}',
  Phys. Lett. B212 (1988) 206

  
\bibitem{MT-cool}
  M. Teper,
  `{\it Instantons in the Quantized SU(2) Vacuum: A Lattice Monte Carlo Investigation}',
  Phys. Lett. B162 (1985) 357.

  
\bibitem{DSMT-Q}
  D. Smith and M. Teper,
  `{\it Topological Structure of the SU(3) Vacuum}',
  Phys. Rev. D 58, 014505 (1998) [arXiv:hep-lat/9801008]


\bibitem{Luscher:2010iy}
  M. L\"uscher,
  `{\it Properties and uses of the Wilson flow in lattice QCD}',
  JHEP 1008 (2010) 071, [arXiv:1006.4518]. 

\bibitem{Luscher:2011bx}
  M. L\"uscher and P. Weisz,
  `{\it Perturbative analysis of the gradient flow in non-abelian gauge theories}',
  JHEP 1102 (2011) 051 [arXiv:1101.0963].

\bibitem{Luscher:2013vga}
  M. L\"uscher,
  `{\it Future applications of the Yang-Mills gradient flow in lattice QCD}',
  PoS LATTICE2013, (2004) 016 [arXiv:1308.5598]. 

\bibitem{Bonati:2014tqa}
  C. Bonati and M. D'Elia,
  `{\it Comparison of the gradient flow with cooling in SU(3) pure gauge theory}',
  Phys. Rev. D 89 (2014) 10, 105005 [arXiv:1401.2441].

\bibitem{Alexandrou:2015yba}
  C. Alexandrou, A. Athenodorou, K. Jansen,
  `{\it Topological charge using cooling and the gradient flow}',
  Phys. Rev. D92 (12) (2015) 125014 [arXiv:1509.04259].

\bibitem{Alexandrou:2017hqw}
  C. Alexandrou, A. Athenodorou, K. Cichy, A. Dromard, E. Garcia-Ramos, K. Jansen, U. Wenger and F. Zimmermann,
  `{\it Comparison of topological charge definitions in Lattice QCD}',
  Eur. Phys. J. C 80 (2020) no.5, 424 [arXiv:1708.00696].

   
\bibitem{BL_Q}
  G. Cossu, D. Lancaster, B. Lucini, R. Pellegrini, A. Rago,
  `{\it Ergodic sampling of the topological charge using the density of states }',
  arXiv:2102.03630.

\bibitem{Veneziano_eta}
  G. Veneziano,
  `{\it U(1) Without Instantons}',
  Nucl. Phys. B159 (1979) 213.
  
\bibitem{Witten_eta}
  E. Witten,
  `{\it Current Algebra Theorems for the U(1) Goldstone Boson}',
  Nucl. Phys. B156 (1979) 269.


\bibitem{bonanno_Q}
  C. Bonanno, C. Bonati, M. D'Elia,
  `{\it Large-N SU(N) Yang-Mills theories with milder topological freezing}',
  JHEP 03 (2021) 111, [arXiv:2012.14000].

  
\bibitem{hasenbusch_Q}
  M. Hasenbusch,
  `{\it  Fighting topological freezing in the two-dimensional CPN-1 model}',
  Phys.Rev.D96, 054504 (2017) [arXiv:1706.04443].

\bibitem{MLFP_SU3_10}
  M. L\"uscher, F. Palombi,
  `{\it Universality of the topological susceptibility in the SU(3) gauge theory}',
  JHEP 09 (2010) 110 [arXiv:1008.0732].


\bibitem{Pisa_theta}
  C. Bonati, M. D'Elia, P. Rossi, E.Vicari,
  `{\it $\theta$  dependence of 4D SU(N) gauge theories in the large-N limit }',
  Phys. Rev. D 94, 085017 (2016) [arXiv:1607.06360].

  
\bibitem{MSbar_4loop}
  T. van Ritbergen, J.A.M. Vermaseren, S.A. Larin,
  `{\it The four-loop beta-function in Quantum Chromodynamics}',
  Phys.Lett. B400 (1997) 379 [arXiv:hep-ph/9701390].

\bibitem{betaL_3loop}
  C. Cristou, A. Feo, H. Panagopoulos, E. Vicari,
  `{\it The three-loop beta function of SU(N) lattice gauge theories with Wilson fermions}',
  Nucl.Phys. B525 (1998) 387; Erratum-ibid. B608 (2001) 479 [arXiv:hep-lat/9801007].

\bibitem{plaq_pert}
  B. Alles, A. Feo, H. Panagopoulos,
  `{\it Asymptotic scaling corrections in QCD with Wilson fermions from the 3-loop
    average plaquette}',
  Phys.Lett. B426 (1998) 361; Erratum-ibid. B553 (2003) 337 [arXiv:hep-lat/9801003].

 
  
\end{thebibliography}



\clearpage


\begin{table}[htb]
\centering
\begin{tabular}{|cc|ccc|} \hline
\multicolumn{5}{|c|}{SU(2) ; $l\surd\sigma\sim 4$ } \\ \hline
$\beta$ & $l^3l_t$ & $\tfrac{1}{2}\text{ReTr}\langle U_p\rangle$ & 
$a\surd\sigma$ & $am_G$  \\ \hline
 2.2986 & $12^316$  & 0.6018259(46)  & 0.36778(69)  & 1.224(16)  \\
 2.3714 & $14^316$  & 0.6226998(36)  & 0.29023(50)  & 1.025(12)  \\
 2.427  & $20^316$  & 0.6364293(15)  & 0.24013(41)  & 0.8469(76)  \\
 2.509  & $22^320$  & 0.6537214(10)  & 0.18011(22)  & 0.6563(56)  \\
 2.60   & $30^4$    & 0.6700089(5)   & 0.13283(30)  & 0.5001(41)  \\
 2.70   & $40^4$    & 0.6855713(3)   & 0.09737(23)  & 0.3652(35)  \\ \hline
\end{tabular}
\caption{Parameters of the main $SU(2)$ calculations: the inverse coupling, $\beta$, the lattice size, the
  average plaquette, the string tension, $\sigma$, and the  mass gap, $m_G$.}
\label{table_param_SU2}
\end{table}

\begin{table}[htb]
\centering
\begin{tabular}{|cc|ccc|} \hline
\multicolumn{5}{|c|}{SU(3) ; $l\surd\sigma\sim 4$}  \\ \hline
$\beta$ & $l^3l_t$ & $\tfrac{1}{3}\text{ReTr}\langle U_p\rangle$ & 
$a\surd\sigma$ & $am_G$  \\ \hline
5.6924 & $10^316$ & 0.5475112(71) & 0.3999(58)  & 0.987(9)    \\
5.80   & $12^316$ & 0.5676412(36) & 0.31666(66) & 0.908(12)   \\
5.8941 & $14^316$ & 0.5810697(18) & 0.26118(37) & 0.7991(92) \\
5.99   & $18^4$ &  0.5925636(11)  & 0.21982(77) & 0.7045(65)  \\
6.0625 & $20^4$ &  0.6003336(10)  & 0.19472(54) & 0.6365(43)  \\
6.235  & $26^4$ &  0.6167723(6)   & 0.15003(30) & 0.4969(29)  \\
6.3380 & $30^4$ &  0.6255952(4)   & 0.12928(27) & 0.4276(37)  \\
6.50   & $38^4$ &  0.6383531(3)   & 0.10383(24) & 0.3474(22)  \\ \hline
\end{tabular}
\caption{Parameters of the main $SU(3)$ calculations: the inverse coupling, $\beta$, the lattice size, the
  average plaquette, the string tension, $\sigma$, and the  mass gap, $m_G$.}
\label{table_param_SU3}
\end{table}


\begin{table}[htb]
\centering
\begin{tabular}{|cc|ccc|} \hline
\multicolumn{5}{|c|}{SU(4) ; $l\surd\sigma\sim 4$}  \\ \hline
$\beta$ & $l^3l_t$ & $\tfrac{1}{4}\text{ReTr}\langle U_p\rangle$ & 
$a\surd\sigma$ & $am_G$  \\ \hline
10.70  & $12^316$ & 0.5540665(24)  & 0.3021(5)   & 0.8406(48)  \\
10.85  & $14^320$ & 0.5664268(15)  & 0.25426(38)  & 0.7611(54)  \\
11.02  & $18^320$ & 0.5782610(11)  & 0.21434(28)  & 0.6605(33)  \\
11.20  & $22^4$   & 0.5893298(6)  & 0.18149(49)  & 0.5709(34)  \\
11.40  & $26^4$   & 0.6004374(4)  & 0.15305(34)  & 0.4864(30)  \\
11.60  & $30^4$   & 0.6106057(3)  & 0.13065(21)  & 0.4132(44)  \\ \hline
\end{tabular}
\caption{Parameters of the main $SU(4)$ calculations: the inverse coupling, $\beta$, the lattice size, the
  average plaquette, the string tension, $\sigma$, and the  mass gap, $m_G$.}
\label{table_param_SU4}
\end{table}

\begin{table}[htb]
\centering
\begin{tabular}{|cc|ccc|} \hline
\multicolumn{5}{|c|}{SU(5) ; $l\surd\sigma\sim 3.1$} \\ \hline
 $\beta$ & $l^3l_t$ & $\tfrac{1}{5}\text{ReTr}\langle U_p\rangle$ & 
$a\surd\sigma$ & $am_G$  \\ \hline
16.98   & $10^316$  & 0.5454873(28)  & 0.3033(8)    & 0.8241(68)  \\
17.22   & $12^316$  & 0.5587002(18)  & 0.2546(6)    & 0.7517(51)  \\
17.43   & $14^320$  & 0.5685281(10)  & 0.22217(37)  & 0.6751(44)  \\
17.63   & $16^320$  & 0.5769707(9)   & 0.19636(35)  & 0.5961(79)  \\
18.04   & $20^324$  & 0.5924012(6)   & 0.15622(38)  & 0.4783(44)  \\
18.375  & $24^330$  & 0.6036547(4)   & 0.13106(30)  & 0.4078(38)  \\ \hline
\end{tabular}
\caption{Parameters of the main $SU(5)$ calculations: the inverse coupling, $\beta$, the lattice size, the
  average plaquette, the string tension, $\sigma$, and the  mass gap, $m_G$.}
\label{table_param_SU5}
\end{table}

\begin{table}[htb]
\centering
\begin{tabular}{|cc|ccc|} \hline
\multicolumn{5}{|c|}{SU(6) ; $l\surd\sigma\sim 3.1$} \\ \hline
$\beta$ & $l^3l_t$ & $\tfrac{1}{6}\text{ReTr}\langle U_p\rangle$ & 
$a\surd\sigma$ & $am_G$  \\ \hline
24.67    & $10^316$  & 0.5409011(28)  & 0.30658(34)  & 0.8240(41)  \\
25.05    & $12^316$  & 0.5557062(13)  & 0.25177(23)  & 0.7395(50)  \\
25.32    & $14^320$  & 0.5646185(10)  & 0.22208(35)  & 0.6673(32)  \\
25.55    & $16^320$  & 0.5715585(9)   & 0.20153(34)  & 0.6112(41)  \\
26.22    & $20^324$  & 0.5894540(4)   & 0.15480(36)  & 0.4751(53)  \\
26.71    & $24^330$  & 0.6009861(3)   & 0.12867(27)  & 0.3886(37)  \\ \hline
\end{tabular}
\caption{Parameters of the main $SU(6)$ calculations: the inverse coupling, $\beta$, the lattice size, the
  average plaquette, the string tension, $\sigma$, and the  mass gap, $m_G$.}
\label{table_param_SU6}
\end{table}




\begin{table}[htb]
\centering
\begin{tabular}{|cc|ccc|} \hline
\multicolumn{5}{|c|}{SU(8) ; $l\surd\sigma\sim 2.6$} \\ \hline
 $\beta$ & $l^3l_t$ & $\tfrac{1}{8}\text{ReTr}\langle U_p\rangle$ & 
$a\surd\sigma$ & $am_G$  \\ \hline
44.10  & $8^316$   & 0.5318034(31) & 0.32589(62)  & 0.8246(66)  \\
44.85  & $10^316$  & 0.5497960(15) & 0.25791(40)  & 0.7461(53)  \\
45.50  & $12^320$  & 0.5622253(9)  & 0.21851(45)  & 0.6409(38)  \\
46.10  & $14^320$  & 0.5723242(8)  & 0.18932(38)  & 0.5617(43)  \\
46.70  & $16^324$  & 0.5815072(5)  & 0.16557(38)  & 0.4909(43)  \\
47.75  & $20^330$  & 0.5959878(3)  & 0.13253(26)  & 0.4075(28)  \\ \hline
\end{tabular}
\caption{Parameters of the main $SU(8)$ calculations: the inverse coupling, $\beta$, the lattice size, the
  average plaquette, the string tension, $\sigma$, and the  mass gap, $m_G$.}
\label{table_param_SU8}
\end{table}



\begin{table}[htb]
\centering
\begin{tabular}{|cc|ccc|} \hline
\multicolumn{5}{|c|}{SU(10) ; $l\surd\sigma\sim 2.6$} \\ \hline
 $\beta$ & $l^3l_t$ & $\tfrac{1}{10}\text{ReTr}\langle U_p\rangle$ & 
$a\surd\sigma$ & $am_G$  \\ \hline
69.20    & $8^316$   & 0.5292925(29) & 0.33024(35)  & 0.8282(60)  \\
70.38    & $10^316$  & 0.5478565(13) & 0.25987(30)  & 0.7435(42)  \\
71.38    & $12^320$  & 0.5602903(9)  & 0.21988(32)  & 0.6451(44)  \\
72.40    & $14^320$  & 0.5713707(6)  & 0.18845(20)  & 0.5549(59)  \\
73.35    & $16^324$  & 0.5807004(4)  & 0.16399(19)  & 0.4952(44)  \\ \hline
\end{tabular}
\caption{Parameters of the main $SU(10)$ calculations: the inverse coupling, $\beta$, the lattice size, the
  average plaquette, the string tension, $\sigma$, and the  mass gap, $m_G$.}
\label{table_param_SU10}
\end{table}


\begin{table}[htb]
\centering
\begin{tabular}{|cc|ccc|} \hline
\multicolumn{5}{|c|}{SU(12) ; $l\surd\sigma\sim 2.6$} \\ \hline
$\beta$ & $l^3l_t$ & $\tfrac{1}{12}\text{ReTr}\langle U_p\rangle$ & 
$a\surd\sigma$ & $am_G$  \\ \hline
 99.86   & $8^316$   & 0.5275951(27)   & 0.33341(40)  & 0.8243(52)  \\
 101.55  & $10^316$  & 0.5464461(12)   & 0.26162(32)  & 0.7384(51)  \\
 103.03  & $12^320$  & 0.55936304(61)  & 0.21915(25)  & 0.6432(32)  \\
 104.55  & $14^320$  & 0.57087665(49)  & 0.18663(38)  & 0.5521(42)  \\
 105.95  & $16^324$  & 0.58043063(32)  & 0.16197(27)  & 0.4920(37)  \\ \hline
\end{tabular}
\caption{Parameters of the main $SU(12)$ calculations: the inverse coupling, $\beta$, the lattice size, the
  average plaquette, the string tension, $\sigma$, and the  mass gap, $m_G$.}
\label{table_param_SU12}
\end{table}


\begin{table}[htb]
\centering
\begin{tabular}{|c|c|c||c|c|c|} \hline
\multicolumn{6}{|c|}{$aE_{eff}(t=2a)$ : $SU(8)$ at $\beta=45.50$ on $14^320$} \\ \hline
  $R^{PC}$   & all $Q$ & $Q=0$ &  $R^{PC}$   & all $Q$ & $Q=0$  \\ \hline \hline
$A_1^{++}$  & 0.6512(21) & 0.6497(25) & $A_1^{-+}$ & 1.119(8)  & 1.098(9)  \\
          &  1.199(9)  & 1.209(9)  &          & 1.581(23)  & 1.587(21)  \\
          &  1.243(12) & 1.236(11) &          & 1.908(43)  & 1.937(51)  \\
          &  1.565(21) & 1.551(23) &          &            &           \\ \hline
$A_2^{++}$  &  1.668(27) & 1.642(25) & $A_2^{-+}$ & 2.21(8)  & 2.51(11)  \\
          &  1.983(48) & 2.037(48) &          &  2.30(9)  & 2.23(11)  \\ \hline
$E^{++}$   & 1.020(4)  & 1.020(5)   & $E^{-+}$  & 1.354(9)  & 1.356(11)  \\
          & 1.230(8)  & 1.233(8)   &          & 1.757(24)  & 1.765(25)  \\
          & 1.435(10) & 1.424(9)   &          & 2.060(51)  & 2.157(50)  \\ \hline
$T_1^{++}$  & 1.670(14) & 1.673(18) & $T_1^{-+}$ & 1.841(25)   & 1.861(24)  \\
          & 1.728(20) & 1.708(19) &          & 1.976(27)   & 1.950(26)  \\
          & 2.011(27) & 2.041(35) &          & 1.925(28)   & 1.928(28)  \\
          & 2.053(35) & 2.037(31) &          &             &           \\ \hline
$T_2^{++}$  & 1.032(4)  & 1.033(3) & $T_2^{-+}$ & 1.360(8)    &  1.356(9) \\
          & 1.460(9)  & 1.447(8)  &          & 1.727(15)   & 1.714(17)  \\
          & 1.670(15) & 1.648(18) &          & 1.881(20)   & 1.900(24)  \\
          & 1.692(14) & 1.691(17) &          &             &          \\ \hline
$A_1^{+-}$  & 2.10(7)  & 2.10(8)   & $A_1^{--}$ & 2.10(7)   & 2.10(8)  \\
          & 2.33(9)   & 2.41(12)   &         & 2.41(12)  & 2.40(13)  \\ \hline
$A_2^{+-}$  & 1.557(18) & 1.577(21) & $A_2^{--}$ & 2.00(5)  & 2.06(7)  \\
          & 1.852(29)  & 1.754(26) &          & 2.15(8)  & 2.25(9)  \\
          & 2.13(6)    & 2.27(10)  &          &          &           \\ \hline
$E^{+-}$   & 1.980(35)  & 1.981(33) & $E^{--}$ & 1.690(18) & 1.686(21)  \\
          & 2.155(48)  & 2.119(41) &          & 2.072(39) & 1.998(42)  \\
          & 2.198(54)  & 2.087(50) &          & 2.198(57) & 2.204(66)  \\ \hline
$T_1^{+-}$  & 1.266(5)   & 1.271(6)  & $T_1^{--}$ & 1.738(18) & 1.734(18)  \\
          & 1.542(9)   & 1.557(10) &          & 1.973(29) &  1.920(26) \\
          & 1.656(13)  & 1.648(11) &          & 2.088(32) &  2.001(31) \\
          & 1.859(18)  & 1.860(19) &          &           &    \\ \hline
$T_2^{+-}$  & 1.571(11)  & 1.543(11) & $T_2^{--}$ & 1.721(15) & 1.715(21)  \\
          & 1.880(19)  & 1.881(20) &          & 1.888(23) & 1.889(28)  \\
          & 1.900(20)  & 1.915(27) &          & 2.043(27) & 2.006(26)  \\
          & 1.996(34)  & 1.994(30) &          &           &          \\ \hline \hline
$l_{k=1}$  & 0.5963(11) & 0.5932(14) & $l_{k=2}$ & 1.1204(33) & 1.1235(28)  \\
          & 1.2954(33) & 1.2918(42) &           & 1.3876(57)  & 1.3810(64)  \\ \hline
\end{tabular}
\caption{Comparison of glueball and flux tube energies obtained on fields with
  topological charge $Q=0$ against fields with a `normal' distribution of $Q$.
  In $SU(8)$ on a $14^320$ lattice at $\beta=45.50$. Energies extracted
  from best (variationally selected)  correlators between $t=a$ and $t=2a$.
  Glueballs labelled by representation of cubic rotation symmetry $R$, parity $P$ and charge
  conjugation $C$. Flux tubes are fundamental, $l_{k=1}$, and $k=2$, $l_{k=2}$.}
\label{table_GKvsQ_SU8_l14}
\end{table}


%\caption{Comparison of glueball and flux tube energies obtained on fields with
%  topological charge $Q=0$ against fields with a `normal' distribution of $Q$.
%  In $SU(8)$ on a $12^320$ lattice at $\beta=45.50$. Energies extracted
%  from best (variationally selected)  correlators.
%  Glueballs labelled by representation of cubic rotation symmetry $R$, parity $P$ and charge
%  conjugation $C$. Flux tubes are fundamental, $l_{k=1}$, and $k=2$, $l_{k=2}$.}
%\label{table_GKvsQ_SU8_l12}


\clearpage

\begin{table}[htb]
\centering
\begin{tabular}{|cc|cc|cc|} \hline
  \multicolumn{2}{|c|}{SU(2) $\beta=2.427$} & \multicolumn{2}{|c|}{SU(3) $\beta=6.235$}
   & \multicolumn{2}{|c|}{SU(6) $\beta=25.35$} \\ \hline
  $l$ & $a\surd\sigma$ & $l$ & $a\surd\sigma$ & $l$ & $a\surd\sigma$ \\ \hline
12   & 0.2380(3)  & 18  & 0.1491(5)  & 12  & 0.2196(5)    \\
14   & 0.2387(4)  & 26  & 0.1499(4)  & 14  & 0.2200(7)    \\
16   & 0.2390(8)  & 34  & 0.1506(6)  & 18  & 0.2181(17)    \\
20   & 0.2399(10) &     &   &     &     \\
24   & 0.2396(16) &     &   &     &     \\ \hline
  \multicolumn{2}{|c|}{SU(4) $\beta=11.02$} & \multicolumn{2}{|c|}{SU(4) $\beta=11.60$}
   & \multicolumn{2}{|c|}{SU(8) $\beta=45.50$} \\ \hline
  $l$ & $a\surd\sigma$ & $l$ & $a\surd\sigma$ & $l$ & $a\surd\sigma$ \\ \hline
 18  & 0.2143(3)  & 24  & 0.1301(5)  & 12  &  0.2187(4)   \\
 22  & 0.2142(6)  & 30  & 0.1307(3)  & 14  &  0.2189(4)   \\ \hline
  \multicolumn{2}{|c|}{SU(10) $\beta=71.38$} & \multicolumn{2}{|c|}{SU(12) $\beta=103.03$}
   &  \multicolumn{2}{|c|}{}  \\ \hline
   $l$ & $a\surd\sigma$ & $l$ & $a\surd\sigma$ &  & \\ \hline
 12  & 0.2199(4)  & 12  & 0.2198(2)  &   &     \\
 14  & 0.2195(5)  & 14  & 0.2202(7)  &   &     \\ \hline
\end{tabular}
\caption{String tensions obtained using eqn(\ref{eqn_NG}), for  (fundamental) flux tubes
  of length $l$ for various groups. A test of finite volume corrections.}
\label{table_V_k1_SUN}
\end{table}


\begin{table}[htb]
\centering
\begin{tabular}{|cc|cc|cc|} \hline
  \multicolumn{2}{|c|}{SU(4) $\beta=11.02$} & \multicolumn{2}{|c|}{SU(4) $\beta=11.60$}
   & \multicolumn{2}{|c|}{SU(6) $\beta=25.35$} \\ \hline
  $l$ & $a\surd\sigma_{k=2}$ & $l$ & $a\surd\sigma_{k=2}$ & $l$ & $a\surd\sigma_{k=2}$ \\ \hline
 18  & 0.2490(12)  & 24  & 0.1523(5)  & 12  & 0.2779(9) \\
 22  & 0.2480(24)  & 30  & 0.1534(5)  & 14  & 0.2846(6) \\
     &             &     &            & 18  & 0.2841(8) \\  \hline
 \multicolumn{2}{|c|}{SU(8) $\beta=45.50$} & \multicolumn{2}{|c|}{SU(10) $\beta=71.38$} &
 \multicolumn{2}{|c|}{SU(12) $\beta=103.03$} \\ \hline
  $l$ & $a\surd\sigma_{k=2}$ & $l$ & $a\surd\sigma_{k=2}$ & $l$ & $a\surd\sigma_{k=2}$ \\ \hline
 12  & 0.2833(9)  & 12  & 0.2901(10)  & 12  & 0.2925(11)    \\
 14  & 0.2885(24) & 14  & 0.2946(22)  & 14  & 0.2987(23)    \\ \hline
\end{tabular}
\caption{String tensions obtained using eqn(\ref{eqn_NG})  for  $k=2$  flux tubes
  of length $l$ for various groups. Testing finite volume corrections.}
\label{table_V_k2_SUN}
\end{table}


\begin{table}[htb]
\centering
\begin{tabular}{|cc|ccc|} \hline
\multicolumn{5}{|c||}{SU(2)} \\ \hline
$\beta$ & lattice & $\tfrac{1}{2}\text{ReTr}\langle U_p\rangle$ & $aE_f$ & $a\surd\sigma$ \\ \hline
2.2986  & $8^316$   & 0.6018323(78)  & 0.9310(55) & 0.36590(93)   \\
2.3714  & $10^316$  & 0.6227129(41)  & 0.7159(41) & 0.28779(71)   \\
2.427   & $12^316$  & 0.6364295(26)  & 0.5831(32) & 0.23750(56)   \\
2.452   & $14^320$  & 0.6420346(28)  & 0.5852(51) & 0.21791(83)   \\
2.509   & $16^320$  & 0.6537206(12)  & 0.4399(15) & 0.17857(26)   \\
2.60    & $22^330$  & 0.6700085(8)   & 0.3370(17) & 0.13279(29)   \\
2.65    & $26^334$  & 0.6780431(5)   & 0.2914(15) & 0.11342(26)   \\
2.70    & $28^340$  & 0.6855710(4)   & 0.2228(13) & 0.09698(24)   \\
2.75    & $34^346$  & 0.6926656(2)   & 0.2048(16) & 0.08375(21)   \\
2.80    & $40^354$  & 0.6993804(2)   & 0.1817(16) & 0.07242(28)   \\ \hline
\end{tabular}
\caption{Energies of periodic flux tubes of length $l$ in $SU(2)$ and derived string tensions
  at the couplings $\beta=4/g^2$ on the lattices $l^3l_t$.}
\label{table_Ksmall_SU2}
\end{table}



\begin{table}[htb]
\centering
\begin{tabular}{|cc|ccc|} \hline
\multicolumn{5}{|c|}{SU(3)} \\ \hline
$\beta$ & lattice & $\tfrac{1}{3}\text{ReTr}\langle U_p\rangle$ & $aE_f$ & $a\surd\sigma$ \\ \hline
5.6924 & $8^316$  & 0.547503(7)   & 1.1588(37) & 0.4010(23)  \\
5.80   & $10^316$ & 0.567642(5)   & 0.8862(26) & 0.31603(84) \\
5.8941 & $12^316$ & 0.581069(4)   & 0.7298(47) & 0.2613(14) \\
5.99   & $14^320$ & 0.5925655(13) & 0.5984(26) & 0.21959(76) \\
6.0625 & $14^320$ & 0.6003331(21) & 0.4517(25) & 0.19509(65) \\
6.235  & $18^326$ & 0.6167715(13) & 0.3369(26) & 0.14899(59) \\
6.338  & $22^330$ & 0.6255948(8)  & 0.3182(19) & 0.12948(44) \\
6.50   & $26^338$ & 0.6383532(5)  & 0.2334(17) & 0.10319(41) \\
6.60   & $32^340$ & 0.64566194(21) & 0.2255(15) & 0.09024(26)  \\
6.70   & $36^344$ & 0.65260033(18) & 0.1933(16) & 0.07898(28)  \\ \hline
\end{tabular}
\caption{Energies of periodic flux tubes of length $l$ in $SU(3)$ and derived string tensions
  at the couplings $\beta=6/g^2$ on the lattices $l^3l_t$.}
\label{table_Ksmall_SU3}
\end{table}


\begin{table}[htb]
\centering
\begin{tabular}{|c||c|c||c|} \hline
\multicolumn{4}{|c|}{continuum $k=2$ string tensions} \\ \hline
 group   &  $\sigma_k/\sigma_f$:NG  &  $\sigma_k/\sigma_f$:$l\to\infty$ & Casimir scaling \\ \hline
 $SU(4)$   & 1.381(14)  & 1.381(14)  & 1.333  \\
 $SU(5)$   & 1.551(11)  & 1.551(11)  & 1.500  \\
 $SU(6)$   & 1.654(13)  & 1.654(13)  & 1.600  \\
 $SU(8)$   & 1.731(11)  & 1.794(28)  & 1.714  \\
 $SU(10)$  & 1.733(15)  & 1.796(29)  & 1.778   \\
 $SU(12)$  & 1.792(16)  & 1.857(29)  & 1.818   \\ \hline
\end{tabular}
\caption{Ratio of the $k=2$ string tension to the fundamental for various $SU(N)$ gauge theories.
  Values labelled NG are obtained using eqn(\ref{eqn_NG}). Values labelled $l\to\infty$ denote our
  best estimates in that limit. The values corresponding to `Casimir scaling'
  are shown for comparison.}
\label{table_sigmak2}
\end{table}


%\caption{Ratios of the $k=3$ and $k=4$ string tensions to the fundamental
%  for various $SU(N)$ gauge theories.
%  Values are obtained using eqn(\ref{eqn_NG}). Values labelled $l\to\infty$ denote our
%  best estimates in that limit. The values corresponding to `Casimir scaling'
%  are shown for comparison.}
%\label{table_sigmak3k4}



\begin{table}
\begin{center}
\begin{tabular}{|c|ccc|c|c|}\hline
  $N$ & $a\sqrt{\sigma}\in$ & $\sqrt{\sigma}/\Lambda^{3loop}_I$ & $\chi^2/n_{df}$
  & $\sqrt{\sigma}/\Lambda^{2loop}_I$ & $\Lambda^{3loop}_{\overline{MS}}/\sqrt{\sigma}$ \\ \hline  
2  & [0.133,0.072] & 4.535(16)  & 0.42 & 4.914(16) &  0.5806(21)[570]  \\
3  & [0.195,0.079] & 4.855(11)  & 1.40 & 5.210(20) &  0.5424(13)[185]  \\ 
4  & [0.254,0.131] & 5.043(10)  & 0.48 & 5.532(11) &  0.5222(11)[230]  \\ 
5  & [0.255,0.131] & 5.090(14)  & 0.31 & 5.622(15) &  0.5174(15)[245]  \\ 
6  & [0.252,0.129] & 5.105(11)  & 1.62 & 5.664(48)$^{\ast}$ & 0.5158(11)[250]   \\ 
8  & [0.258,0.133] & 5.148(17)  & 0.13 & 5.727(20) &  0.5115(17)[250]  \\ 
10 & [0.260,0.164] & 5.221(40)$^{\ast}$  & 3.30  & 5.845(44)$^{\ast}$ & 0.5044(20)[270]   \\ 
12 & [0.262,0.162] & 5.193(13)  & 0.45 & 5.823(15) &  0.5071(13)[270]  \\ \hline 
\end{tabular}
\caption{Results for $\Lambda_I$ in units of the string tension using the exact 3-loop
  $\beta$-function, with the 2-loop result for comparison. Poor fits denoted by $\ast$.
  Resulting $\Lambda_{\overline{MS}}$ is shown with statistical errors and an estimate
  of the (correlated) systematic error in square brackets.}
\label{table_Lambda_fitN}
\end{center}
\end{table}


\begin{table}
\begin{center}
\begin{tabular}{|c|c|c|c|c|c|c|}\hline 
$N$ &  $c_0$ & $c_{\sigma}$ & $a\sqrt{\sigma}\in$ & $\beta\in$ & $\chi^2/n_{df}$ & $n_{df}$ \\ \hline
2  & 4.510(15)  & 4.98(25)  & [0.133,0.072] & [2.60,2.80]   &  0.40 & 3   \\
3  & 4.827(12)  & 2.48(11)  & [0.195,0.079] & [5.99,6.70]   &  1.29 & 4 \\
4  & 5.017(8)   & 1.622(40) & [0.302,0.131] & [10.70,11.60] &  0.98 & 4  \\ 
5  & 5.068(11)  & 1.435(56) & [0.303,0.131] & [16.98,18.375] & 0.38 & 4  \\ 
6  & 5.064(12)  & 1.541(68) & [0.252,0.129] & [25.05,26.71]  & 1.45 & 3  \\ 
8  & 5.143(9)   & 1.265(34) & [0.326,0.133] & [44.10,47.75]  & 1.75 & 4  \\
10 & 5.176(10)  & 1.285(35) & [0.260,0.164] & [70.38,73.35]  & 3.06 & 2   \\
12 & 5.148(12)  & 1.387(43) & [0.262,0.162] & [101.55,105.95] & 0.41 & 2  \\  \hline
\end{tabular}
\caption{Fitted values of the coefficients $c_{0}$ and  $c_{\sigma}$ that
  determine the interpolation function in eqn(\ref{eqn_aKgI}) for each of our $SU(N)$
  lattice gauge theories. In each case the range in $a\surd\sigma$ and of $\beta$ of
  the fit is shown as is the $\chi^2$ per degree of freedom and the number
  of degrees of freedom.}
\label{table_interp_gI}
\end{center}
\end{table}





\begin{table}[htb]
\centering
\begin{tabular}{|c|c|c|c|} \hline
\multicolumn{4}{|c|}{$aE_{eff}(t=2a)$ : $SU(2)$ at $\beta=2.427$} \\ \hline
  $R^{P}$  &  $12^316$ & $14^316$  &  $20^316$  \\ \hline
$A_1^{+}$  & 0.840(11)   & 0.854(8) & 0.847(8)  \\
           & 1.225(26)   &           &            \\ 
           & 1.418(17)   & 1.367(15) & 1.408(19)  \\ 
           &             & 1.593(27) &            \\ 
           & 1.611(36)   & 1.721(51) & 1.780(45)  \\ 
           & 1.871(40)   & 1.839(46) & 1.872(46)  \\  \hline
$A_2^{+}$  & 1.680(40)   & 1.805(38) & 1.942(72)  \\  \hline
$E^{+}$    & 1.220(9)    & 1.254(14) & 1.284(13)  \\
           & 1.317(17)   &           &            \\ 
           &             & 1.549(20) &            \\ 
           & 1.632(30)   & 1.672(33) &  1.672(40) \\ 
           & 1.774(31)   & 1.820(39) &  1.832(42) \\   \hline
$T_1^{+}$  & 1.925(42)   & 1.875(27) & 1.855(37)  \\ 
           & 1.977(48)   & 1.972(34) & 1.916(50)  \\  \hline
$T_2^{+}$  & 1.297(12)   & 1.278(14) & 1.289(11)  \\ 
           & 1.739(20)   & 1.701(30) & 1.672(21)  \\ 
           & 1.774(34)   & 1.849(22) & 1.877(30)  \\  \hline
$A_1^{-}$  & 1.471(30)   & 1.540(30) & 1.526(36)  \\
           & 1.71(4)     & 1.87(9)   & 1.94(9)  \\  \hline
$A_2^{-}$  & 2.31(16)   & 2.21(17)   & 2.04(12)  \\  \hline
$E^{-}$    & 1.585(31)   & 1.680(28) & 1.652(34)  \\
           &  2.00(6)    & 2.04(7)   & 2.02(7)  \\ \hline
$T_1^{-}$  &  2.18(6)    & 2.15(6)   & 2.09(7)  \\ \hline
$T_2^{-}$  &  1.665(25)  & 1.632(33) & 1.620(21)  \\
           &  2.04(5)    & 2.00(4)   & 2.01(6)  \\ \hline
$l_{k=1}$   & 0.5804(38)   & 0.7140(41) & 1.0996(39)  \\
           & 1.367(23)    & 1.428(32)  & 1.677(9)  \\ \hline
\end{tabular}
\caption{Comparison of glueball  effective energies from $t=2a$ and flux tube energies ($l1$)
  obtained on $12^316$, $14^316$  and $20^316$ lattices at $\beta=2.427$ in $SU(2)$.
  Unmatched states are ditorelons.
  Glueballs labelled by representation of cubic rotation symmetry $R$ and parity $P$.
  Fundamental flux tube energies , $l_{k=1}$, are ground and first excited states.}
\label{table_GvsV_SU2}
\end{table}



\clearpage


%\caption{Comparison of glueball effective energies from $t=2a$  and flux tube energies ($l1$,$l2$)
%  obtained on $14^320$ and $18^318$ lattices at $\beta=17.46$ in $SU(5)$. Unmatched states
%  are ditorelons.
%  Glueballs labelled by representation of cubic rotation symmetry $R$, parity $P$ and charge
%  conjugation $C$. Flux tube energies are fundamental, $l_{k=1}$, and $k=2$, $l_{k=2}$.}
%\label{table_GvsV_SU5}
%\end{table}




%use bl=1-4 and 1-5 to identify extra states and remove;
% then take aE from t=2a plateau
\begin{table}[htb]
\centering
\begin{tabular}{|c|c|c||c|c|c|} \hline
\multicolumn{6}{|c|}{$aE_{eff}(t=2a)$ : $SU(12)$ at $\beta=103.03$} \\ \hline
  $R^{PC}$   & $12^320$ & $14^320$  &  $R^{PC}$   & $12^320$ & $14^320$  \\ \hline  \hline 
$A_1^{++}$  & 0.6434(32) & 0.6499(50)  &  $A_1^{-+}$ & 1.079(8) & 1.097(12) \\
          & 1.190(9)   & 1.200(14)   &            & 1.567(22)  & 1.591(33)  \\
          & 1.495(19)  & 1.579(25)   &            & 1.859(38)  & 2.037(91) \\
          & 1.604(26)  & 1.585(33)   &            &            &   \\ 
          & 1.660(26)  & 1.687(43)   &            &            &   \\ \hline
$A_2^{++}$  & 1.629(30) & 1.639(39)  & $A_2^{-+}$ & 2.14(10) & 2.21(11)  \\
          & 1.91(5)    & 2.11(10)    &          & 2.20(9)   & 2.30(13)  \\ \hline
$E^{++}$   & 1.021(5)   & 1.026(7)   & $E^{-+}$  & 1.346(10)  & 1.355(15)  \\
          & 1.429(14)  &  1.464(16)  &          & 1.723(21)  & 1.747(32)  \\
          & 1.610(19)  &  1.632(28)  &          & 2.105(33)  & 2.178(74)  \\ \hline
$T_1^{++}$  & 1.635(17) & 1.692(21) & $T_1^{-+}$ & 1.865(21)  & 1.885(39)  \\
          & 1.712(19) & 1.776(26)  &          & 1.945(30)  & 1.930(36)  \\
          & 2.090(32) & 2.062(47)  &          & 2.012(28)  & 2.026(49)  \\ \hline
$T_2^{++}$  & 1.040(4) & 1.031(7)  & $T_2^{-+}$ & 1.357(8) & 1.366(12)  \\
          & 1.465(9) & 1.473(16)  &          & 1.730(16)  & 1.748(23)  \\
          & 1.597(14) & 1.659(21)  &          & 1.891(26)  & 1.905(44)  \\
          & 1.652(14) & 1.676(25)  &          & 1.932(29)  & 1.975(40)  \\ \hline
$A_1^{+-}$  & 2.001(52) & 2.06(11)  & $A_1^{--}$ & 2.18(8)  & 2.09(11)  \\
          & 2.12(8)   & 2.23(13)   &            & 2.23(10) & 2.30(12)  \\ \hline
$A_2^{+-}$  & 1.564(22)  & 1.603(30)  & $A_2^{--}$ & 1.932(52) & 2.026(70)  \\
          & 1.823(36)  & 1.882(53)  &          & 2.209(85)  & 2.33(17)  \\
          & 1.97(8)    & 2.19(10)   &          &            &   \\ \hline
$E^{+-}$   & 1.903(29) & 2.011(43)  & $E^{--}$ & 1.682(17) & 1.677(30)  \\
          & 2.140(51) & 2.044(70)  &          & 2.053(44) & 2.087(54)  \\
          &           &            &          & 2.11(5)   & 2.26(9)   \\ \hline
$T_1^{+-}$  & 1.269(6)  & 1.268(8)  & $T_1^{--}$ & 1.696(18)  & 1.693(22)  \\ 
          & 1.543(12)  & 1.546(16)  &          & 1.921(25)  & 1.897(44)  \\
          & 1.678(16)  & 1.690(18)  &          & 1.976(33)  & 2.023(45)  \\
          & 1.850(21)  & 1.880(27)  &          & 2.21(6)    & 2.18(6) \\ \hline
$T_2^{+-}$  & 1.534(13)  & 1.541(15) & $T_2^{--}$ & 1.719(20) & 1.766(27)  \\
          & 1.878(24)  & 1.870(27)  &          & 1.893(29)  & 1.935(31)  \\
          & 1.850(22)  & 1.896(36)  &          & 2.014(32)  & 2.035(37)  \\
          & 1.988(32)  & 1.987(43)  &          & 2.19(5)    & 2.19(6)  \\ \hline \hline
$l_{k=1}$  & 0.4812(13) & 0.5993(40) &  $l_{k=2}$ & 0.935(8)   & 1.172(18)  \\ \hline
\end{tabular}
\caption{Comparison of glueball effective energies from $t=2a$  and flux tube energies ($l1$,$l2$)
  obtained on $12^320$ and $14^320$ lattices at $\beta=103.03$ in $SU(12)$. 
  Glueballs labelled by representation of cubic rotation symmetry $R$, parity $P$ and charge
  conjugation $C$. Flux tube energies are fundamental, $l_{k=1}$, and $k=2$, $l_{k=2}$.}
\label{table_GvsV_SU12}
\end{table}



%use bl=1-4 and 1-5 to identify extra states and remove;
% then take aE from t=2a plateau
\begin{table}[htb]
\centering
\begin{tabular}{|c|cc|cc|} \hline
\multicolumn{5}{|c|}{$aE_{eff}(t=2a)$ : $SU(12)$ at $\beta=103.03$} \\ \hline
$R^{PC}$   & \multicolumn{2}{|c|}{$12^320$} &  \multicolumn{2}{|c|}{$14^320$} \\ 
          & $bl=1-4$  &  $bl=1-5$   &  $bl=1-4$  &  $bl=1-5$ \\ \hline 
$A_1^{++}$ & 0.6434(32) & 0.6432(32) & 0.6499(50) & 0.6497(50)  \\
          &            & 1.172(10)  &            &           \\
          & 1.190(9)   & 1.179(14)  & 1.200(14)  & 1.200(14) \\
          &            &            &            & 1.424(23) \\ 
          & 1.462(20)  & 1.495(19)  & 1.588(23)  & 1.433(29) \\ 
          & 1.604(26)  & 1.601(29)  & 1.585(33)  & 1.571(35) \\
          & 1.660(26)  & 1.666(30)  & 1.687(43)  & 1.659(40) \\ 
          & 1.753(25)  & 1.780(31)  & 1.786(54)  & 1.780(52) \\ \hline 
$E^{++}$   & 1.021(5)    & 1.019(5)  & 1.026(7)   & 1.026(7) \\
          & 1.525(13)[1.01(14)] & 1.118(7)  &    &          \\
          &             &           &            & 1.273(20) \\
          & 1.429(14)   & 1,373(12)  &  1.464(16)  & 1.446(16) \\
          & 1.610(19)   & 1.610(18)  &  1.635(29)  & 1.632(28) \\
          & 1.657(19)   & 1.650(18)  &  1.663(24)  & 1.666(25) \\ \hline 
\end{tabular}
\caption{Comparison of $A_1^{++}$ and $E^{++}$ glueball effective energies from $t=2a$  
  obtained on $12^320$ and $14^320$ lattices at $\beta=103.03$ in $SU(12)$.
  Using operators up to blocking levels 4 and 5 respectively, as shown.}
\label{table_GvsV_SU12B}
\end{table}



\begin{table}[htb]
\centering
\begin{tabular}{|c|c|c|c|c|} \hline
\multicolumn{5}{|c|}{number of operators: $N\in [4,12]$} \\ \hline
  $R$   & P=+,C=+ & P=-,C=+ &  P=+,C=-   &  P=-,C=-   \\ \hline
$A_1$  & 12 & 7   & 6   & 8 \\
$A_2$  &  7 & 5   & 7   & 7 \\ 
$E$    & 36 & 24  & 24  & 30 \\
$T_1$  & 48 & 54  & 66  & 51 \\
$T_2$  & 60 & 60  & 60  & 54 \\ \hline
\multicolumn{5}{|c|}{number of operators: $N=2,3$} \\ \hline
  $R$   & P=+,C=+ & P=-,C=+ &  P=+,C=-   &  P=-,C=-   \\ \hline
$A_1$  & 27  & 9  & 8   & 11 \\ 
$A_2$  & 14  & 6  & 13  & 11 \\ 
$E$    & 80  & 30 & 40  & 44 \\ 
$T_1$  & 78  & 84(75) & 132 & 81 \\ 
$T_2$  & 108 & 96(87) & 108 & 84 \\ \hline
\end{tabular}
\caption{Number of operators in rotational representation , $R$, with parity, $P$, and charge conjugation, $C$,
  as used in our various $SU(N)$ calculations, at each blocking level. Some numbers for $SU(2)$
  differ from $SU(3)$ and are in brackets; also no $C=-$ for $SU(2)$} 
\label{table_numops_N}
\end{table}


\begin{table}[htb]
\centering
\begin{tabular}{|c|c|} \hline
\multicolumn{2}{|c|}{operator loops} \\  \hline
  $loop$   &   $R^{PC}$     \\ \hline
  \{2,3,-2,-3\}  &   $A_1^{++},E^{++}$ \\ 
                 &   $T_1^{+-}$ \\ \hline
  \{1,2,2,-1,-2,-2\} &   $A_1^{++},A_2^{++},E^{++}$   \\ 
                 &   $T_1^{+-},T_2^{+-}$  \\ \hline
  \{1,2,3,-1,-2,-3\} &   $A_1^{++},T_2^{++}$   \\ 
                 &  $A_2^{+-},T_1^{+-}$   \\ \hline
  \{1,3,2,-3,-1,-2\} &   $A_1^{++},E^{++},T_2^{++},T_1^{-+},T_2^{-+}$   \\ 
                 &  $T_1^{+-},T_2^{+-},A_1^{--},E^{--},T_2^{--}$   \\ \hline
  \{1,2,2,-1,3,-2,-3,-2\} &  $A_1^{++},A_2^{++},E^{++},T_1^{++},T_2^{++},A_1^{-+},A_2^{-+},E^{-+},T_1^{-+},T_2^{-+}$    \\ 
                 &    $A_1^{+-},A_2^{+-},E^{+-},T_1^{+-},T_2^{+-},A_1^{--},A_2^{--},E^{--},T_1^{--},T_2^{--}$  \\ \hline
  \{1,3,-1,-3,-1,-2,1,2\} &  $A_1^{++},E^{++},T_1^{++},T_2^{++},A_1^{-+},E^{-+},T_1^{-+},T_2^{-+}$    \\ 
                 &   $A_1^{+-},E^{+-},T_1^{+-},T_2^{+-},A_1^{--},E^{--},T_1^{--},T_2^{--}$  \\ \hline
  \{1,2,3,-1,-3,-3,-2,3\} &  $A_1^{++},E^{++},T_1^{++},T_2^{++},A_1^{-+},E^{-+},T_1^{-+},T_2^{-+}$    \\ 
                 &   $A_2^{+-},E^{+-},T_1^{+-},T_2^{+-},A_2^{--},E^{--},T_1^{--},T_2^{--}$  \\ \hline
  \{1,3,1,2,-3,-1,-1,-2\} &   $A_1^{++},A_2^{++},E^{++},T_1^{++},T_2^{++},A_1^{-+},A_2^{-+},E^{-+},T_1^{-+},T_2^{-+}$   \\ 
                 &   $A_1^{+-},A_2^{+-},E^{+-},T_1^{+-},T_2^{+-},A_1^{--},A_2^{--},E^{--},T_1^{--},T_2^{--}$  \\ \hline
  \{1,2,2,2,-1,3,-2,-3,-2,-2\} &  $A_1^{++},A_2^{++},E^{++},T_1^{++},T_2^{++},A_1^{-+},A_2^{-+},E^{-+},T_1^{-+},T_2^{-+}$    \\ 
                 &   $A_1^{+-},A_2^{+-},E^{+-},T_1^{+-},T_2^{+-},A_1^{--},A_2^{--},E^{--},T_1^{--},T_2^{--}$  \\ \hline
  \{1,2,2,2,-1,-2,3,-2,-3,-2\} &   $A_1^{++},A_2^{++},E^{++},T_1^{++},T_2^{++},T_1^{-+},T_2^{-+}$   \\ 
                 &  $T_1^{+-},T_2^{+-},A_1^{--},A_2^{--},E^{--},T_1^{--},T_2^{--}$    \\ \hline
  \{-3,1,3,1,2,-3,-1,3,-1,-2\} &   $A_1^{++},A_2^{++},E^{++},T_1^{++},T_2^{++},A_1^{-+},A_2^{-+},E^{-+},T_1^{-+},T_2^{-+}$    \\ 
                 &   $A_1^{+-},A_2^{+-},E^{+-},T_1^{+-},T_2^{+-},A_1^{--},A_2^{--},E^{--},T_1^{--},T_2^{--}$  \\ \hline
  \{-3,1,3,1,2,3,-1,-3,-2,-1\} &    $A_1^{++},A_2^{++},E^{++},T_1^{++},T_2^{++},A_1^{-+},A_2^{-+},E^{-+},T_1^{-+},T_2^{-+}$   \\  
                 &   $A_1^{+-},A_2^{+-},E^{+-},T_1^{+-},T_2^{+-},A_1^{--},A_2^{--},E^{--},T_1^{--},T_2^{--}$  \\ \hline
\end{tabular}
\caption{The 12 loops used as the basis of our glueball calculations for $N\geq 4$. These generate contributions
  to the representations as shown.}
\label{table_loops}
\end{table}


%\caption{$SU(2)$ lattice glueball masses for all $R^{P}$ representations, and the flux tube energy $aE_f$.} 
%\label{table_Mlat_R_SU2}




%\caption{$SU(4)$ lattice glueball masses for all $R^{P}$ and $C=+$ representations, with
%  the fundamental, $aE_f$, and $k=2$, $aE_{k=2}$, flux tube energies.} 
%\label{table_Mlat_RC+_SU4}



%\caption{$SU(4)$ lattice glueball masses for all $R^{P}$ and $C=-$ representations.} 
%\label{table_Mlat_RC-_SU4}




%\caption{$SU(5)$ lattice glueball masses for all $R^{PC}$ representations, with
%  the fundamental, $aE_f$, and $k=2$, $aE_{k=2}$, flux tube energies.} 
%\label{table_Mlat_RPC_SU5}



%\caption{$SU(6)$ lattice glueball masses for all $R^{PC}$ representations, with
%  the fundamental, $aE_f$, and $k=2$, $aE_{k=2}$, flux tube energies.} 
%\label{table_Mlat_RPC_SU6}







\begin{table}[htb]
\centering
\begin{tabular}{|c|c|c|c|c|c|c|} \hline
\multicolumn{7}{|c|}{$SU(8):aM_G$} \\ \hline
 $R^{PC}$ & $\beta=44.10$ & $\beta=44.85$ & $\beta=45.50$ & $\beta=46.10$ & $\beta=46.70$  & $\beta=47.75$    \\ 
          & $8^316$ & $10^316$ & $12^320$  & $14^320$ & $16^324$ & $20^330$    \\ \hline \hline
$A^{++}_1$ & 0.8246(66) & 0.7461(53) & 0.6409(38) & 0.5617(43) & 0.4909(43) & 0.4075(28)   \\
          & 1.615(26)  & 1.191(50)  & 1.112(37) & 1.081(22) & 0.930(13) & 0.755(10)   \\
          & 2.01(6)    & 1.764(39)  & 1.42(9)  & 1.252(41) & 1.146(27)  & 0.936(18)   \\ \hline
$A^{++}_2$ & 2.42(11)   & 1.828(46)  & 1.615(25) & 1.408(26) & 1.180(27) & 0.969(16)    \\ \hline
$E^{++}$   & 1.504(13)  & 1.184(8)  & 1.0063(51) & 0.8813(48) & 0.7676(67) & 0.6192(69)   \\
          &  2.145(41) & 1.689(18) & 1.442(11)  & 1.299(34)  & 1.070(16)  & 0.831(8)  \\ 
          &            & 1.855(28) & 1.579(15)  & 1.393(13) & 1.184(22)  &  0.954(11)  \\ \hline
$T^{++}_1$ & 2.50(8)   & 1.970(29)  & 1.517(27)  & 1.397(40) & 1.204(21)  & 0.982(7)   \\
          &           & 1.932(29)  & 1.536(90)  & 1,392(43) & 1.238(29)  & 0.996(10) \\ \hline
$T^{++}_2$ & 1.536(12)  & 1.191(22)  & 1.033(10) & 0.8795(58) & 0.7725(64) & 0.6195(22)   \\
          & 2.092(41)  & 1.689(15)  & 1.428(39) & 1.188(21) & 1.101(12)  & 0.870(16)   \\ 
          & 2.41(8)  & 1.875(28)  & 1.621(76)  & 1.357(34) & 1.162(14)  & 0.963(8)   \\ 
          &         & 1.889(30)  & 1.52(8)  & 1.344(35) & 1.201(18)  & 0.975(24)   \\ \hline
$A^{-+}_1$ & 1.618(33) & 1.256(13)  & 1.009(20)  & 0.902(17) & 0.8021(95) & 0.6365(81)   \\
          & 2.21(10)  & 1.75(5)   & 1.42(8)   & 1.23(5)  & 1.074(82)   &  0.963(35)  \\ \hline
$A^{-+}_2$ &           & 2.27(13)  & 2.07(6)   & 1.857(38) & 1.588(32)  &  1.257(33) \\ \hline
$E^{-+}$   & 1.994(56) & 1.573(17) & 1.295(41)  & 1.131(21) & 1.013(11)  & 0.8082(65)   \\
          & 2.72(14)  & 2.019(54)  & 1.734(20) & 1.498(14) & 1.320(9)   & 1.101(15)  \\ \hline
$T^{-+}_1$ &           & 2.29(15)  & 1.844(23)  & 1.571(65) & 1.324(32)  & 1.108(11)   \\ \hline
c          &           &           & 1.990(24)  & 1.639(12) & 1.447(39)  & 1.142(18)  \\ \hline
$T^{-+}_2$ & 1.959(26)  & 1.47(5)  & 1.271(24)  & 1.146(16) & 0.982(10)  & 0.807(10)  \\
          & 2.62(11)  & 2.043(43) & 1.741(18)  & 1.506(11) & 1.298(25)  & 1.065(11)  \\ \hline
$A^{+-}_1$ &          & 2.47(16)  & 2.103(73)  & 1.831(40) & 1.552(19)  & 1.209(32)   \\ \hline
$A^{+-}_2$ &  2.25(10) & 1.834(37)  & 1.529(18) & 1.275(44) & 1.176(26)  & 0.928(13)   \\ \hline
$E^{+-}$   &           & 2.196(52) & 1.921(33) & 1.659(26) & 1.463(38) & 1.117(18)    \\ \hline
$T^{+-}_1$ & 1.898(26)  & 1.510(12)  & 1.281(7)  & 1.102(6) & 0.9556(82)  & 0.7706(46)   \\
          & 2.23(6)   & 1.812(27)  & 1.539(13)  & 1.328(6) & 1.1724(41)  & 0.9269(78)   \\
          &          & 1.914(21)  & 1.648(14)  & 1.437(10) & 1.221(16)  & 0.980(8)   \\ \hline
$T^{+-}_2$ & 2.33(6)  & 1.781(21)  & 1.544(15)  & 1.294(30) & 1.133(14)  & 0.922(7)   \\
          & 2.53(12) & 2.237(46)  & 1.811(22)  & 1.50(6)  & 1.327(32)  &  1.093(15)  \\
          & 2.53(10) & 2.265(50)  & 1.815(20)  & 1,618(18) & 1.343(33) &  1.119(13)  \\ \hline
$A^{--}_1$ &         & 2.62(18)   & 2.05(7)   & 1.888(35) & 1.633(23)  &  1.254(36)  \\ \hline
$A^{--}_2$ &        & 2.22(10)  & 1.94(6)  & 1.74(17)  & 1.368(67)   &  1.114(24) \\ \hline
$E^{--}$   & 2.42(9) & 2.012(40)  & 1.663(19)  & 1.441(11) & 1.250(31)  & 1.009(14)   \\ \hline
$T^{--}_1$ & 2.55(9) & 1.906(30)  & 1.564(69)  & 1.414(42) & 1.186(22)  & 0.974(29)   \\ \hline
$T^{--}_2$ & 2.48(7) & 1.981(32)  & 1.709(16)  & 1.473(10) & 1.2920(56) & 1.037(9)  \\
          &         & 2.24(5)   & 1.896(22)  & 1.601(13) & 1.4108(81)  & 1.109(14)  \\ \hline \hline
$aE_f$    & 0.7067(33)  & 0.5506(21)  & 0.4778(24)  & 0.4204(20) & 0.3674(20)  & 0.2943(14)  \\ 
$aE_{k=2}$ & 1.268(20)  & 0.9948(85)  & 0.8716(61)  & 0.7923(69) & 0.6735(46)  & 0.5483(29)  \\ \hline
\end{tabular}
\caption{$SU(8)$ lattice glueball masses for all $R^{PC}$ representations, with
  the fundamental, $aE_f$, and $k=2$, $aE_{k=2}$, flux tube energies.} 
\label{table_Mlat_RPC_SU8}
\end{table}



%\caption{$SU(10)$ lattice glueball masses for all $R^{PC}$ representations, with
%  the fundamental, $aE_f$, and $k=2$, $aE_{k=2}$, flux tube energies.} 
%\label{table_Mlat_RPC_SU10}


\clearpage


%\caption{$SU(12)$ lattice glueball masses for all $R^{PC}$ representations, with
%  the fundamental, $aE_f$, and $k=2$, $aE_{k=2}$, flux tube energies.} 
%\label{table_Mlat_RPC_SU12}




\begin{table}[htb]
\centering
\begin{tabular}{|c|c|c|c|c|} \hline
\multicolumn{3}{|c|}{$SU(2)$: $M_G/\surd\sigma$  continuum limit} \\ \hline
  $R$   & P=+,C=+ & P=-,C=+     \\ \hline
$A_1$  &  3.781(23)  &  6.017(61)  \\
    &  6.126(38)  &  8.00(15)   \\ 
    &  7.54(10)   &             \\ \hline
$A_2$  &  7.77(18)   &  9.50(18)$^\star$ \\
    &  8.56(21)   &             \\ \hline
$E$   &  5.343(30)  &  7.037(67)  \\
    &  6.967(62)  &  8.574(83)  \\
    &  7.722(82)  &  9.58(16)   \\ \hline
$T_1$  &  8.14(10)   &  9.06(13)   \\
    &  8.46(12)   &  9.40(13)   \\
    &  9.67(9)    &  9.83(16)   \\ \hline
$T_2$  &  5.353(23)  &  6.997(65)  \\
    &  7.218(52)  &  8.468(86)  \\
    &  8.23(10)   &  9.47(10)   \\ \hline
\end{tabular}
\caption{$SU(2)$ continuum limit of glueball masses in units of the string tension,
  for all representations, $R$, of the rotation symmetry of a cube, for
  both values of parity, $P$.
  Ground states and some excited states. Stars
  indicate poor fits (see text).}
\label{table_MK_R_SU2}
\end{table}





\begin{table}[htb]
\centering
\begin{tabular}{|c|c|c|c|c|} \hline
\multicolumn{5}{|c|}{$SU(3)$: $M_G/\surd\sigma$  continuum limit} \\ \hline
  $R$   & P=+,C=+ & P=-,C=+ &  P=+,C=-   &  P=-,C=-   \\ \hline
$A_1$  & 3.405(21)  & 5.276(45) & 9.32(28)  &  9.93(49)  \\
    & 5.855(41)  & 7.29(13)  &           &  10.03(47) \\
    & 7.38(11)   & 9.18(26)  &           &           \\ 
    & 7.515(50)  & 9.37(22)  &           &            \\  \hline
$A_2$  & 7.705(85)  & 9.80(22)  & 7.384(90) &  8.96(15)  \\
    & 8.61(20)   & 11.17(30) & 8.94(10)  &  10.21(20) \\ 
    &            &           & 8.90(21)  &             \\  \hline
$E$   & 4.904(20)  & 6.211(56) & 8.77(12)  &  7.91(10)  \\
    & 6.728(47)  & 8.23(9)   & 9.03(23)  &  9.39(18)  \\ 
    & 7.49(9)    & 9.47(16)  & 10.39(21) &  10.40(22) \\ 
    & 7.531(60)  &           &           &            \\  \hline
$T_1$  & 7.698(80)  & 8.48(12)  & 6.065(40) &  8.31(10)  \\
    & 7.72(11)   & 8.57(13)  & 7.21(8)   &  9.30(14)$^{\star\star}$  \\ 
    & 9.31(11)$^\star$ & 8.66(15) & 7.824(56) & 9.72(15)  \\ 
    &            & 9.56(28)  & 8.92(10)  &            \\  \hline
$T_2$  & 4.884(19)  & 6.393(45) & 7.220(86) &  8.198(80) \\
    & 6.814(31)  & 8.15(7)   & 8.72(11)  &  8.99(11)$^{\star\star}$  \\ 
    & 7.716(70)$^{\star\star}$  & 9.23(12)$^\star$ & 9.060(80) &  9.69(13)  \\ 
    & 7.677(71)  &           &  &   \\  \hline
\end{tabular}
\caption{$SU(3)$ continuum limit of glueball masses in units of the string tension,
  for all representations, $R$, of the rotation symmetry of a cube, for
  both values of parity, $P$,
  and charge conjugation, $C$. Ground states and some excited states. Stars
  indicate poor fits (see text).}
\label{table_MK_R_SU3}
\end{table}


\begin{table}[htb]
\centering
\begin{tabular}{|c|c|c|c|c|} \hline
\multicolumn{5}{|c|}{$SU(4)$: $M_G/\surd\sigma$  continuum limit} \\ \hline
  $R$   & P=+,C=+ & P=-,C=+ &  P=+,C=-   &  P=-,C=-   \\ \hline
$A_1$  & 3.271(27)  & 5.020(46) & 9.22(22)$^\star$  & 10.27(28)  \\
    & 5.827(62)  & 7.33(11)  & 10.43(21) & 9.95(29)  \\
    & 7.50(11)   & 8.98(23)  &           &   \\
    & 7.73(8)    &           &           &   \\ \hline
$A_2$  & 7.32(12)   & 9.67(19)  & 6.87(26)  & 8.89(13)  \\
    & 8.42(22)   & 10.69(45)$^\star$ & 8.66(16) & 10.99(35)  \\
    &            &           & 9.57(19)  &   \\ \hline
$E$   & 4.721(27)  & 6.130(52) & 8.73(20)  & 7.80(11)  \\
    & 6.702(45)  & 7.91(13)  & 9.28(19)  & 9.55(13)$^\star$  \\
    & 7.271(86)  & 9.13(22)  & 9.58(20)  & 10.06(26)  \\
    & 7.586(84)  &           &           &   \\ \hline
$T_1$  & 7.42(12)   & 8.47(11)  & 5.956(42) & 7.603(84)  \\
    & 7.50(9)    & 8.59(16)  & 7.11(8)   & 8.92(12)$^{\star\star}$  \\
    & 9.10(18)   & 8.62(17)$^{\star\star}$ & 7.508(74) & 9.86(22) \\
    & 9.54(23)   &           & 8.91(13)  &            \\ \hline
$T_2$  & 4.750(16)  & 6.203(33) & 7.010(68) & 7.787(96)  \\
    & 6.687(51)  & 8.05(10)$^{\star\star}$ & 8.53(11) & 8.42(14)  \\
    & 7.411(81)  & 8.25(16)  & 8.86(8)   & 9.80(17)    \\
    & 7.492(68)  &           & 8.94(20)  &             \\ \hline
\end{tabular}
\caption{$SU(4)$ continuum limit of glueball masses in units of the string tension,
  for all representations, $R$, of the rotation symmetry of a cube, for
  both values of parity, $P$, and charge conjugation, $C$.
  Ground states and some excited states. Stars
  indicate poor fits (see text).}
\label{table_MK_R_SU4}
\end{table}

\begin{table}[htb]
\centering
\begin{tabular}{|c|c|c|c|c|} \hline
\multicolumn{5}{|c|}{$SU(5)$: $M_G/\surd\sigma$  continuum limit} \\ \hline
  $R$   & P=+,C=+ & P=-,C=+ &  P=+,C=-    &  P=-,C=-   \\ \hline
$A_1$  & 3.156(31)  & 4.832(40)  & 9.04(22)  & 10.36(18)  \\
    & 5.689(53)  & 7.38(11)   & 9.48(30)  &  9.70(36)  \\
    & 7.36(14)   & 8.03(30)$^\star$  &     &             \\
    & 7.53(14)   &            &           &             \\ \hline
$A_2$  & 7.31(14)   & 9.80(20)   & 7.333(59) & 8.05(23)  \\
    & 8.85(32)   & 9.93(57)   & 8.75(19)  & 11.03(37)  \\ \hline
$E$   & 4.692(22)  & 6.152(60)  & 8.54(19)  & 8.000(51)  \\
    & 6.590(62)  & 8.19(23)   & 9.45(18)  & 9.02(22)   \\
    & 7.31(10)   & 9.51(17)   & 9.53(20)  & 10.22(13)$^{\star\star}$  \\
    & 7.28(10)   &            &           &             \\ \hline
$T_1$  & 7.396(73)  & 8.29(13)   & 5.915(45) & 7.62(15)  \\
    & 7.19(12)   & 8.56(15)   & 7.018(62) & 8.53(17)$^{\star\star}$  \\
    & 8.99(22)   & 8.59(16)   & 7.624(74) & 9.91(26)    \\
    &            &            & 8.36(14)  &             \\ \hline
$T_2$  & 4.686(30)  & 6.208(47)  & 7.051(72) & 7.87(11)     \\
    & 6.621(55)  & 7.97(11)   & 8.28(15)$^\star$ & 8.52(9)  \\
    & 7.338(84)  & 8.45(16)   & 8.44(16)  & 10.17(9)$^{\star}$  \\
    & 7.53(10)   &            & 9.18(13)  &   \\ \hline
\end{tabular}
\caption{$SU(5)$ continuum limit of glueball masses in units of the string tension,
  for all representations, $R$, of the rotation symmetry of a cube, for
  both values of parity, $P$, and charge conjugation, $C$.
  Ground states and some excited states. Stars
  indicate poor fits (see text).}
\label{table_MK_R_SU5}
\end{table}

\begin{table}[htb]
\centering
\begin{tabular}{|c|c|c|c|c|} \hline
\multicolumn{5}{|c|}{$SU(6)$: $M_G/\surd\sigma$  continuum limit} \\ \hline
  $R$   & P=+,C=+ & P=-,C=+ &  P=+,C=-   &  P=-,C=-   \\ \hline
$A_1$  & 3.102(32)  & 4.967(43) & 9.37(22)  & 10.46(17)  \\
    & 6.020(57)  & 7.14(13)  & 10.58(21) & 10.53(20)$^{\star\star}$  \\
    & 7.51(12)   & 8.72(31)  &           &          \\
    & 7.59(13)$^\star$  &     &           &          \\ \hline
$A_2$  & 7.46(13)   & 9.37(25)  & 7.169(96) & 8.71(16)$^\star$  \\
    & 9.58(24)$^{\star\star}$ & 10.48(43) & 9.00(11) & 10.87(38)    \\
    &            &           & 9.12(20)  &           \\ \hline
$E$   & 4.706(30)  & 6.098(55) & 8.80(11)  & 8.106(50)  \\
    & 6.43(11)   & 7.93(15)  & 8.95(17)  & 9.48(33)   \\
    & 7.06(13)   & 9.58(19)  & 9.34(19)  & 9.88(11)   \\
    & 7.19(17)   &           &           &            \\ \hline
$T_1$  & 7.435(80)  & 8.405(88) & 5.847(43) & 7.41(13)   \\
    & 7.57(10)   & 8.75(11)  & 7.066(80) & 8.90(11)    \\
    & 9.17(19)   & 9.23(11)$^\star$  & 7.552(85) & 9.86(26)$^\star$  \\
    &            &           & 8.46(11)  &             \\ \hline
$T_2$  & 4.649(28)  & 6.157(50) & 6.997(80) & 7.64(14)   \\
    & 6.577(71)  & 7.92(14)  & 8.20(19)  & 8.51(9)   \\
    & 7.189(72)  & 8.25(11)$^\star$ & 8.864(61) & 10.20(9)  \\
    & 7.02(10)$^\star$   &           & 9.15(11)  &            \\ \hline
\end{tabular}
\caption{$SU(6)$ continuum limit of glueball masses in units of the string tension,
  for all representations, $R$, of the rotation symmetry of a cube, for
  both values of parity, $P$, and charge conjugation, $C$.
  Ground states and some excited states. Stars
  indicate poor fits (see text).}
\label{table_MK_R_SU6}
\end{table}

\clearpage

\begin{table}[htb]
\centering
\begin{tabular}{|c|c|c|c|c|} \hline
\multicolumn{5}{|c|}{$SU(8)$: $M_G/\surd\sigma$  continuum limit} \\ \hline
  $R$   & P=+,C=+ & P=-,C=+ &  P=+,C=-   &  P=-,C=-   \\ \hline
$A_1$  & 3.099(26)  & 4.755(58) & 8.97(31)  & 9.69(35)  \\
    & 5.87(7)    & 6.90(24)  & 10.06(24) & 9.50(39)  \\
    & 7.18(13)   & 8.25(26)  &           &           \\
    & 7.61(7)    &           &           &           \\ \hline
$A_2$  & 7.28(14)   & 9.80(29)  & 7.00(12)  & 8.26(28)  \\
    & 9.06(34)   & 11.28(25) & 8.60(18)  & 11.40(23)  \\
    &            &           & 8.87(48)  &            \\ \hline
$E$   & 4.658(32)  & 6.091(59) & 8.53(19)  & 7.60(11)    \\
    & 6.32(7)$^\star$  & 8.23(13) & 8.94(24)$^{\star\star}$ & 9.18(25)  \\
    & 7.26(11)   & 9.46(24)  & 8.81(35)  & 9.57(29)   \\ \hline
$T_1$  & 7.318(74)  & 8.24(15)  & 5.801(38) & 7.09(17)  \\
    & 7.51(11)   & 8.60(27)  & 7.089(55) & 8.35(18)  \\
    & 9.21(22)   & 8.46(22)  & 7.43(9)$^\star$ & 9.60(25) \\
    & 9.21(25)   &           & 8.43(12)  &   \\ \hline
$T_2$  & 4.661(21)  & 5.995(61) & 6.908(67) & 7.863(66)  \\
    & 6.599(84)  & 8.00(11)  & 8.01(16)  & 8.31(12)  \\
    & 7.16(8)    & 8.38(16)  & 8.51(12)$^\star$ & 9.40(24)  \\
    & 7.23(16)   &           & 8.92(15)  &   \\ \hline
\end{tabular}
\caption{$SU(8)$ continuum limit of glueball masses in units of the string tension,
  for all representations, $R$, of the rotation symmetry of a cube, for
  both values of parity, $P$, and charge conjugation, $C$.
  Ground states and some excited states. Stars
  indicate poor fits (see text).}
\label{table_MK_R_SU8}
\end{table}

\begin{table}[htb]
\centering
\begin{tabular}{|c|c|c|c|c|} \hline
\multicolumn{5}{|c|}{$SU(10)$: $M_G/\surd\sigma$  continuum limit} \\ \hline
  $R$   & P=+,C=+ & P=-,C=+ &  P=+,C=-   &  P=-,C=-   \\ \hline
$A_1$  & 3.102(37) & 4.835(60) & 8.92(37)  & 10.04(36)  \\
    & 5.99(13)  & 7.03(14)  & 8.73(60)  &  9.79(55)  \\
    & 7.15(17)  & 8.90(68)  &           &            \\
    & 7.87(13)  &           &           &            \\ \hline
$A_2$  & 7.20(18)  & 8.99(35)  & 6.78(15)  & 9.06(44)   \\
    & 8.35(50)  & 11.74(70) & 7.83(44)  & 10.70(70)  \\
    &           &           & 8.29(103)?&   \\ \hline
$E$   & 4.587(27) & 6.12(10)  & 8.64(20)  & 7.60(14)  \\
    & 6.49(9)   & 8.03(12)  & 9.04(30)  & 9.59(33)  \\
    & 7.34(12)  & 9.36(46)  & 9.05(30)  & 9.92(22) \\ \hline
$T_1$  & 7.14(17)  & 8.49(28)  & 5.776(41) & 7.27(17)  \\
    & 7.84(27)  & 8.11(62)  & 7.039(62) & 9.00(24)$^\star$  \\
    & 9.32(18)$^{\star\star}$ & 9.53(100)$^\star$ & 7.79(13) & 10.17(52)  \\
    & 9.69(22)$^{\star\star}$ &   & 8.75(21)$^{\star\star}$ &          \\ \hline
$T_2$  & 4.600(30) & 5.94(8)   & 7.071(55) & 7.89(16)  \\
    & 6.638(82) & 7.52(17)  & 7.60(24)  & 7.82(26)$^\star$  \\
    & 7.00(18)  & 7.88(30)$^{\star\star}$ & 8.60(13) &  9.26(36) \\
    & 7.19(22)  &           & 8.77(28)  &           \\ \hline
\end{tabular}
\caption{$SU(10)$ continuum limit of glueball masses in units of the string tension,
  for all representations, $R$, of the rotation symmetry of a cube, for
  both values of parity, $P$, and charge conjugation, $C$.
  Ground states and some excited states. Stars
  indicate poor fits (see text).}
\label{table_MK_R_SU10}
\end{table}

\begin{table}[htb]
\centering
\begin{tabular}{|c|c|c|c|c|} \hline
\multicolumn{5}{|c|}{$SU(12)$: $M_G/\surd\sigma$  continuum limit} \\ \hline
  $R$   & P=+,C=+ & P=-,C=+ &  P=+,C=-   &  P=-,C=-   \\ \hline
$A_1$  & 3.151(33)  & 4.696(65) & 9.63(25)   & 9.57(32)  \\
    & 5.914(82)  & 6.94(19)  & 10.50(51)  & 9.90(58)  \\
    & 7.07(18)   & 9.56(44)$^\star$ &      &           \\
    & 7.61(19)   &           &            &          \\ \hline
$A_2$  & 7.74(22)   & 10.16(26) & 6.49(17)   & 9.03(22)  \\
    & 9.64(26)   & 12.09(74) & 8.90(22)   & 10.66(60)  \\
    &            &           & 9.46(46)   &   \\ \hline
$E$   & 4.647(33)  & 6.22(10)  & 8.52(22)   & 7.80(14)  \\
    & 6.613(80)  & 8.28(12)$^\star$ & 9.56(43) & 9.88(20) \\
    & 7.67(13)   &           &            & 10.24(29)    \\ \hline
$T_1$  & 6.64(27)   & 8.59(12)  & 5.741(60)  & 7.50(17)$^\star$  \\
    & 7.59(21)   & 8.95(16)  & 7.005(78)  & 8.83(14)     \\
    & 9.34(18)$^{\star\star}$ & 9.30(30)$^\star$ & 7.48(14) & 9.51(48)  \\
    &            &           & 8.80(23)   &   \\ \hline
$T_2$  & 4.645(33)  & 5.97(7)   & 6.88(10) &  8.20(13)      \\
    & 6.764(57)  & 8.05(15)  & 8.16(25) &  8.27(31)$^\star$ \\
    & 7.35(15)$^\star$ & 8.89(22) & 8.48(25) & 10.52(21)$^{\star\star}$  \\
    & 7.20(19)   &           & 8.21(30) &                \\ \hline
\end{tabular}
\caption{$SU(12)$ continuum limit of glueball masses in units of the string tension,
  for all representations, $R$, of the rotation symmetry of a cube, for
  both values of parity, $P$, and charge conjugation, $C$.
  Ground states and some excited states. Stars
  indicate poor fits (see text).}
\label{table_MK_R_SU12}
\end{table}

\clearpage

\begin{table}[htb]
\centering
\begin{tabular}{|c|c|c|c|c|} \hline
\multicolumn{5}{|c|}{$SU(\infty)$:  $M_G/\surd\sigma$ } \\ \hline
  $R$   & P=+,C=+ & P=-,C=+ &  P=+,C=-   &  P=-,C=-   \\ \hline
$A_1$  & 3.072(14)         & 4.711(25)            & 9.26(16)        &  10.10(18) \\
    & 5.805(31)$^{\star\star}$ & 7.050(68)             &                &  10.14(23) \\
    & 7.294(63)         &                       &                &   \\ \hline
$A_2$  & 7.40(12)$^\dagger$ & 9.73(12)              & 7.142(75)$^{\dagger\star}$  & 8.61(13)$^\star$ \\
    & 9.14(14)$^\star$   & 11.12(24)             & 8.77(10)      &  11.38(21)   \\ \hline
$E$   & 4.582(14)         & 6.108(44)             & 8.63(10)      &  7.951(53)$^{\star\star}$ \\
    & 6.494(33)$^\star$  & 8.051(60)             & 9.14(15)      &  9.55(13)                   \\
    & 7.266(50)$^\star$  &                       &               &  9.84(12)                 \\ \hline
$T_1$  & 7.250(47)         & 8.412(76)             & 5.760(25)     & 7.134(86)  \\
    & 7.337(60)$^\star$  & 8.79(10)              & 7.020(39)     & 8.65(9)  \\
    & 9.142(82)         & 9.08(12)$^{\star\star}$ & 7.470(55)     & 9.81(17)  \\ 
    &                   &                       & 8.422(84)     &             \\ \hline
$T_2$  & 4.578(11)         & 5.965(28)             & 6.957(41)     &  7.96(8)$^\dagger$ \\
    & 6.579(30)         & 7.883(57)             & 7.93(11)      &  8.22(8)            \\
    & 7.121(45)         & 8.45(14)$^\dagger$     & 8.63(7)$\star$ & 10.26(10)$^{\star\star}$  \\
    & 7.122(76)         &     &     &  \\ \hline
\end{tabular}
\caption{Continuum glueball masses in units of the string tension,
  in the limit $N\to\infty$. Fits are to $N\geq 2$ or $N\geq 3$ except
  for values labelled with a $\dagger$, and $\star$ indicates a  poor fit,
  as explained in the text. 
  Labels are $R$ for the representations of the rotation symmetry of a cube,
  $P$ for parity and  $C$ for charge conjugation.}
\label{table_MK_R_SUN}
\end{table}









\begin{table}[htb]
\centering
\begin{tabular}{|ccc|} \hline
\multicolumn{3}{|c|}{continuum $J \sim$ cubic $R$} \\ \hline
$J$    &        &  cubic $R$  \\  \hline
 0   & $\sim$ & $A_1$   \\
 1   & $\sim$ & $T_1$      \\
 2   & $\sim$ & $E+T_2$     \\
 3   & $\sim$ & $A_2+T_1+T_2$     \\
 4   & $\sim$ & $A_1+E+T_1+T_2$     \\
 5   & $\sim$ & $E+2T_1+T_2$      \\
 6   & $\sim$ & $A_1+A_2+E+T_1+2T_2$     \\
 7   & $\sim$ & $A_2+E+2T_1+2T_2$     \\
 8   & $\sim$ & $A_1+2E+2T_1+2T_2$     \\ \hline
\end{tabular}
\caption{Projection of continuum spin $J$ states onto the cubic representations $R$.}
\label{table_J_R}
\end{table}




\begin{table}[htb]
\centering
\begin{tabular}{|llc|} \hline
\multicolumn{3}{|c|}{continuum $J^{PC}$ from cubic $R$} \\ \hline
$J^{PC}$    &        &  cubic $R^{PC}$  \\  \hline
$0^{++}$gs  & $\sim$ & $A_1^{++}$gs     \\ 
$0^{++}$ex1 & $\sim$ & $A_1^{++}$ex1    \\ 
$2^{++}$gs  & $\sim$ & $E^{++}$gs + $T_2^{++}$gs    \\  
$2^{++}$ex1 & $\sim$ & $E^{++}$ex1 + $T_2^{++}$ex1    \\ 
$3^{++}$gs  & $\sim$* & $A_2^{++}$gs + $T_1^{++}$gs(ex1) + $T_2^{++}$ex3(ex2)    \\ 
$4^{++}$gs  & $\sim$* & $A_1^{++}$ex2 + $E^{++}$ex2 +$T_1^{++}$ex1(gs) + $T_2^{++}$ex2(ex3)   \\ \hline
 $0^{-+}$gs  &  $\sim$  &  $A_1^{-+}$gs     \\ 
 $0^{-+}$ex1 &  $\sim$  &  $A_1^{-+}$ex1    \\
 $2^{-+}$gs  &  $\sim$  &  $E^{-+}$gs + $T_2^{-+}$gs   \\
 $2^{-+}$ex1 &  $\sim$  &  $E^{-+}$ex1 + $T_2^{-+}$ex1   \\ 
 $1^{-+}$gs  &  $\sim$  & $T_1^{-+}$gs    \\ \hline
 $2^{+-}$gs  &  $\sim$  &  $E^{+-}$gs + $T_2^{+-}$ex2   \\
 $1^{+-}$gs  &  $\sim$  & $T_1^{+-}$gs   \\ 
 $1^{+-}$ex1 &  $\sim$  & $T_1^{+-}$ex2   \\ 
 $3^{+-}$gs  &  $\sim$  & $A_2^{+-}$gs + $T_1^{+-}$ex1 + $T_2^{+-}$gs   \\  \hline
 $2^{--}$gs  &  $\sim$  & $E^{--}$gs + $T_2^{--}$gs \\
 $1^{--}$gs  &  $\sim$  & $T_1^{--}$gs   \\ \hline
\end{tabular}
\caption{Identification of continuum $J^{PC}$ states from the results for the cubic representations
  in  Tables~\ref{table_MK_R_SU2}-\ref{table_MK_R_SUN}.
  Ground state denoted by $gs$, $i$'th excited state by $exi$. Where
  there is some ambiguity, a single star denotes 'likely'
  while two stars indicate 'significant uncertainty'.}
\label{table_M_J_R}
\end{table}


\begin{table}[htb]
\centering
\begin{tabular}{|l|c|c|c|c|} \hline
\multicolumn{5}{|c|}{$M(J^{PC})/\surd\sigma$ continuum limit} \\ \hline
  $J^{PC}$      & $SU(2)$ & $SU(3)$ & $SU(4)$ & $SU(5)$ \\ \hline
 $0^{++}$ {gs}   & 3.781(23)  & 3.405(21)  & 3.271(27)  & 3.156(31)  \\
 $0^{++}$ {ex1}  & 6.126(38)  & 5.855(41)  & 5.827(62)  & 5.689(53)  \\
 $2^{++}$ {gs}   & 5.349(20)  & 4.894(22)  & 4.742(15)  & 4.690(20)  \\
 $2^{++}$ {ex1}  & 7.22(6)$^+$ & 6.788(40) & 6.694(40)  & 6.607(45)  \\
 $3^{++}$ {gs}   & 8.13(8)$^*$ & 7.71(9)$^*$ &          & 7.29(9)     \\ 
 $4^{++}$ {gs}   &            & 7.60(12)$^*$ & 7.36(9)$^*$ & 7.41(10) \\ \hline
 $0^{-+}$ {gs}   & 6.017(61)  & 5.276(45)  & 5.020(46)  & 4.832(40)  \\ 
 $0^{-+}$ {ex1}  & 8.00(15)   & 7.29(13)   & 7.33(11)   & 7.38(11)  \\
 $2^{-+}$ {gs}   & 7.017(50)  & 6.32(9)    & 6.182(33)  & 6.187(50)  \\
 $2^{-+}$ {ex1}  & 8.521(65)  & 8.18(8)    & 7.91(13)$^*$ & 8.02(11)  \\
 $1^{-+}$ {gs}   & 9.06(13)   & 8.48(12)   & 8.47(11)   & 8.29(13)  \\ \hline
 $2^{+-}$ {gs}   &            & 8.91(15)   & 8.84(11)   &  8.49(15)  \\
 $1^{+-}$ {gs}   &            & 6.065(40)  & 5.956(42)  & 5.915(45)  \\
 $1^{+-}$ {ex1}  &            & 7.82(6)    & 7.51(8)    & 7.62(8)  \\
 $3^{+-}$ {gs}   &            & 7.27(12)   & 7.03(7)    & 7.13(7)  \\  \hline
 $2^{--}$ {gs}   &            & 8.08(15)   & 7.79(9)   & 7.97(6)  \\
 $1^{--}$ {gs}   &            & 8.31(10)   & 7.60(9)   & 7.62(15)  \\ \hline
\end{tabular}
\caption{Continuum limit of glueball masses, in units of the string tension,
  for those $J^{PC}$ representations we can identify. Ground state denoted by $gs$,
  $i$'th excited state by $exi$. Stars explained in text.}
% su2 2++ex1 + = only T2 for J=2
% su2 3++ * = A2 a little low
% su4 4++ * = could be 3++
% su4 2-+ ex1 * = use E: T2 poor fit
\label{table_MKJ_N2-5}
\end{table}



\begin{table}[htb]
\centering
\begin{tabular}{|l|c|c|c|c|} \hline
\multicolumn{5}{|c|}{$M(J^{PC})/\surd\sigma$ continuum limit} \\ \hline
  $J^{PC}$      & $SU(6)$ & $SU(8)$ & $SU(10)$ & $SU(12)$ \\ \hline
 $0^{++}$ {gs}   & 3.102(32)  & 3.099(26)  & 3.102(37)  & 3.151(33)  \\
 $0^{++}$ {ex1}  & 6.020(57)  & 5.87(7)    & 5.99(13)   & 5.914(82)  \\
 $2^{++}$ {gs}   & 4.678(30)  & 4.660(20)  & 4.594(25)  & 4.646(30)  \\
 $2^{++}$ {ex1}  & 6.54(7)    & 6.60(9)$^*$ & 6.57(9)   & 6.71(6)  \\
 $3^{++}$ {gs}   & 7.44(8)$^*$ & 7.34(11)$^*$ & 7.14(18) &   \\ 
 $4^{++}$ {gs}   &            & 7.20(7)$^*$ & 7.32(15)  &   \\ \hline
 $0^{-+}$ {gs}   & 4.967(43)  & 4.755(58)  & 4.835(60)  & 4.696(65)  \\ 
 $0^{-+}$ {ex1}  & 7.14(13)   & 6.90(24)   & 7.03(14)   & 6.94(19)  \\
 $2^{-+}$ {gs}   & 6.148(50)  & 6.043(55)  & 6.01(8)   &  6.05(7) \\
 $2^{-+}$ {ex1}  & 7.92(14)   & 8.10(11)   & 7.86(12)$^*$ & 8.16(15)  \\
 $1^{-+}$ {gs}   & 8.41(9)    & 8.24(15)   & 8.49(28)  & 8.59(12)  \\ \hline
 $2^{+-}$ {gs}   & 8.83(9)    & 8.52(12)   & 8.62(12)  & 8.50(17)    \\
 $1^{+-}$ {gs}   & 5.847(43)  & 5.801(38)  & 5.776(41) & 5.741(60)  \\
 $1^{+-}$ {ex1}  & 7.55(9)    & 7.43(9)$^*$ & 7.79(13) & 7.48(14)  \\
 $3^{+-}$ {gs}   & 7.067(70)  & 6.999(70)  & 7.035(50) & 6.89(10)  \\ \hline
 $2^{--}$ {gs}  & 8.05(5)$^*$ & 7.78(7)   & 7.74(15)  & 8.00(15)  \\
 $1^{--}$ {gs}   & 7.41(13)   & 7.09(17)   & 7.27(17)  & 7.50(17)  \\ \hline
\end{tabular}
\caption{Continuum limit of glueball masses, in units of the string tension,
  for those $J^{PC}$ representations we can identify. Ground state denoted by $gs$,
  $i$'th excited state by $exi$. Stars explained in text.}
% su6 3++gs + = no T2 
% su6 2--gs + = big E,T2 gap 
% su8 3++,4++ * = which T1,T2 excitations to which?
% su8 1+- ex1 poor fit
% su10 2-+ ex1 E,T2 gap
\label{table_MKJ_N6-12}
\end{table}

\clearpage


\begin{table}[htb]
\centering
\begin{tabular}{|cccccc|} \hline
\multicolumn{6}{|c|}{$M(J^{PC})/\surd\sigma$ lower bounds} \\ \hline
  $1^{++}$ &  $3^{-+}$ & $0^{+-}$& $2^{+-}$ & $0^{--}$ & $3^{--}$ \\ \hline
 $\ge 9.0$  & $\ge 9.7$  &$\ge 9.2$   & $\ge 8.6$  &$\ge 10.0$   & $\ge 8.6$   \\ \hline
\end{tabular}
\caption{Lower bounds on masses of some of the low-$J$ ground states not
  appearing in Tables~\ref{table_MKJ_N2-5},\ref{table_MKJ_N6-12}.}
\label{table_MKJ_lowbound}
\end{table}


\begin{table}[htb]
\centering
\begin{tabular}{|l|l|l|l|l|} \hline
\multicolumn{5}{|c|}{$M_G \, GeV \quad SU(3)$} \\ \hline
  $J$   & P=+,C=+ & P=-,C=+ &  P=+,C=-   &  P=-,C=-   \\ \hline
 0 gs   & 1.653(26)   & 2.561(40) & $\ge 4.52(15)$ &  $\ge 4.81(24)$   \\
 0 ex1  & 2.842(40)   & 3.54(8)   &               &    \\
 2 gs   & 2.376(32)   & 3.07(6)   & 4.24(8)*      &  3.92(9)   \\
 2 ex1  & 3.30(5)     & 3.97(7)   & $\ge 4.38(13)$ & $\ge 4.55(11)$  \\
 1 gs   & $\ge 4.52(6)$ & 4.12(8)  & 2.944(42)    &  4.03(7)  \\ 
 1 ex1  &             & 4.16(8)*  & 3.80(6)       & $\ge 4.51(9)$    \\
 1 ex2  &             & 4.20(9)*  &               &    \\
 3 gs   & 3.74(7)*    &  $\ge 4.75(13)$ & 3.53(8)   & $\ge 4.35(9)$   \\
 4 gs   & 3.69(8)*    &  $\ge 4.45(14)$ & 4.38(8)** & $\ge 4.81(24)$   \\ \hline
\end{tabular}
\caption{Continuum limit of $SU(3)$ glueball masses, in physical $GeV$ units
  for those $J^{PC}$ representations we can identify, with lower bounds in those
  cases where this is not possible. Ground state denoted by $gs$,
  first excited state by $ex$. Stars denote ambiguity.}
\label{table_MJ_N3}
    \end{table}



\begin{table}[htb]
\centering
\begin{tabular}{|c|c|c|c|c|} \hline
\multicolumn{5}{|c|}{$M_J/\surd\sigma$ ; $SU(\infty)$} \\ \hline
  $J$      & P=+,C=+   & P=-,C=+   &  P=+,C=-   &  P=-,C=-   \\ \hline
 $0$ {gs}  & 3.072(14) & 4.711(26) &  $\ge 9.26(16)$     & $\ge 10.10(18)$    \\ 
 $0$ {ex}  & 5.845(50) & 7.050(68) &                     &    \\ 
 $2$ {gs}  & 4.599(14) & 6.031(38) & 8.566(76)           &  7.910(56)  \\ 
 $2$ {ex}  & 6.582(36) & 7.936(54) &  $\ge 9.14(15)$     & $\ge 9.55(13)$    \\ 
 $1$ {gs}  & $\ge 9.14(9)$ & 8.415(76)  & 5.760(25)      & 7.26(11)   \\ 
 $1$ {ex}  &           &                & 7.473(57)      & $\ge 8.65(9) $  \\ 
 $3$ {gs}  & 7.263(56) & $\ge 9.73(12)$ & 6.988(41)      & $\ge 8.61(13) $   \\ 
 $4$ {gs}  & 7.182(71) & $\ge 8.79(10)$ & $\ge 9.26(16)$ & $\ge 10.10(18)$   \\  \hline
\end{tabular}
\caption{Large $N$ extrapolation of continuum glueball masses, in units of the string tension,
  for those $J^{PC}$ representations we can identify, with lower bounds in those
  cases where this is not possible. Ground state denoted by $gs$,
  first excited state by $ex$.}
\label{table_MK_J_SUN}
\end{table}


\begin{table}[htb]
\centering
\begin{tabular}{|c|c|c|c|c|c|c|c|} \hline
\multicolumn{8}{|c|}{$M/\surd\sigma$ ; continuum $SU(3)$} \\ \hline
  paper     & $0^{++}$ & $2^{++}$ & $0^{-+}$ & $2^{-+}$ & $1^{+-}$ & $3^{+-}$ & $2^{--}$    \\ \hline
 this work                      & 3.405(21)  & 4.894(22) &  5.276(45) & 6.32(9)  & 6.065(40) & 7.27(12) & 8.08(15)  \\
 2005: ref\cite{HM_Thesis}      & 3.347(68)  & 4.891(65) &  5.11(14)  & 6.32(11) & 6.06(15)  & 7.43(20) & 8.32(29)  \\
 2005: ref\cite{MP-2005}$^\star$ & 3.59(15)   & 5.03(11)  &  5.39(11) &  6.40(14) & 6.27(11)  & 7.58(12) & 8.42(14)  \\
 2004: ref\cite{BLMTUW_N}       & 3.55(7)    & 4.78(9)   &  &  &  &   &   \\
 2001: ref\cite{BLMT_N}         & 3.61(9)    & 5.13(22)  &  &  &  &   &   \\  \hline
\end{tabular}
\caption{Some glueball masses in units of the string tension in the continuum limit of the $SU(3)$ gauge theory:
  a comparison between this work and some earlier work. The starred calculation includes our tansformation of
the units from the Sommer scale to the string tension.}
\label{table_MK_J_SU3_comp}
\end{table}


\begin{table}[htb]
\centering
\begin{tabular}{|c|c|c|c|c|c|c|c|} \hline
\multicolumn{8}{|c|}{$M/\surd\sigma$ ; continuum $SU(8)$} \\ \hline
  paper     & $0^{++}$ & $2^{++}$ & $0^{-+}$ & $2^{-+}$ & $1^{+-}$ & $3^{+-}$ & $2^{--}$    \\ \hline
 this work                 & 3.099(26)  & 4.660(20) & 4.755(58) & 6.043(55) & 5.801(38) & 7.00(7)  & 7.78(7)  \\
 2005: ref\cite{HM_Thesis} & 3.32(15)   & 4.65(19)  & 4.72(32)  & 5.67(40)  & 5.70(29)  & 7.74(79) & 7.3(1.4)  \\
 2004: ref\cite{BLMTUW_N}  & 3.55(12)   & 4.73(22) &  &  &  &   &   \\  \hline
\end{tabular}
\caption{Some glueball masses in units of the string tension in the continuum limit of the $SU(8)$ gauge theory:
  a comparison between this work and some earlier work.}
\label{table_MK_J_SU8_comp}
\end{table}




\begin{table}[htb]
\centering
\begin{tabular}{|c|lc|lc|lc|} \hline
\multicolumn{7}{|c|}{$Q_L(n_c)$ : $SU(8)$, $Q=2$} \\ \hline
       & \multicolumn{2}{|c|}{$\beta=45.50,12^320$} & \multicolumn{2}{|c|}{$\beta=46.70,16^324$} & \multicolumn{2}{|c|}{$\beta=47.75,20^330$}  \\
 $n_c$ &  $\bar{Q}_L$ &  $\sigma_{Q_{L}}$ & $\bar{Q}_L$ & $\sigma_{Q_{L}}$  & $\bar{Q}_L$ & $\sigma_{Q_{L}}$ \\ \hline
  0   & 0.275(21) & 0.836 & 0.333(23) & 1.282 &  0.410(35) & 1.956  \\
  1   & 1.199(6)  & 0.202 & 1.286(5)  & 0.265 &  1.345(7)  & 0.351  \\
  2   & 1.575(3)  & 0.070 & 1.631(2)  & 0.082 &  1.665(3)  & 0.103  \\
  3   & 1.699(2)  & 0.041 & 1.7465(7) & 0.044 &  1.7743(9) & 0.053  \\
  4   & 1.7576(7) & 0.030 & 1.8005(6) & 0.030 &  1.8254(6) & 0.035  \\
  8   & 1.8437(3) & 0.016 & 1.8781(3) & 0.014 &  1.8970(2) & 0.014  \\
  12  & 1.8744(3) & 0.013 & 1.9046(2) & 0.010 &  1.9212(1) & 0.009  \\ 
  16  & 1.8912(3) & 0.011 & 1.9186(2) & 0.008 &  1.9339(1) & 0.007  \\ 
  20  & 1.9021(3) & 0.010 & 1.9275(1) & 0.007 &  1.9418(1) & 0.006  \\   \hline
\end{tabular}
\caption{Lattice topological charge $Q_L$ as function of number of cooling sweeps, $n_c$, for fields
  which have $Q_L\simeq 2$ after 20 cooling sweeps: the average value, with error, and standard deviation.
  For small, intermediate and large $\beta$, in $SU(8)$, all with similar volumes in physical units.}
\label{table_Q_nc_SU8}
\end{table}



%\caption{Lattice topological charge $Q_L$ as function of number of cooling sweeps, $n_c$, for fields
%  which have $Q_L\simeq 2$ after 20 cooling sweeps: the average value, with error, and standard deviation.
%  For $SU(3),SU(5),SU(12)$ lattice fields at similar lattice spacings (in physical units), and similar
%  to the $SU(8)$ at $\beta=46.70$ in Table~\ref{table_Q_nc_SU8}.}
%\label{table_Q_nc_SUN}
%\end{table}



\begin{table}[htb]
\centering
\begin{tabular}{|ccc|ccc|} \hline
\multicolumn{6}{|c|}{topology tunnelling time} \\ \hline
$\beta$ &  $\tau_Q$ & $\Tilde{\tau}_Q$ & $\beta$ &  $\tau_Q$ & $\Tilde{\tau}_Q$ \\ \hline
\multicolumn{3}{|c|}{$SU(3)$} & \multicolumn{3}{|c|}{$SU(4)$} \\ \hline
 6.235 & 100(2)   & 112(2)   & 11.20 & 499(9)   & 520(9)    \\    
 6.338 & 247(6)   & 283(6)   & 11.40 & 2173(42) & 2275(45)    \\  
 6.50  & 1056(44) & 1236(50) & 11.60 & 12674(636) & 13371(690)  \\
 6.60  & 3399(188) & 3731(190) &     &          &               \\
 6.70  & 7973(674) & 8973(744) &     &          &              \\ \hline
\multicolumn{3}{|c|}{$SU(5)$} & \multicolumn{3}{|c|}{$SU(6)$} \\ \hline
17.22  & 189(4)   &  196(4)   & 25.05 & 1158(31)   & 1177(31)    \\
17.43  & 695(13)  &  721(13)  & 25.32 & 5992(262)  & 6105(272)   \\
17.63  & 2686(90) & 2749(93)  & 25.55 & $23.6(1.8)\times 10^3$  &  $24.4(1.8)\times 10^3$   \\
18.04  & $35.7(3.7)\times 10^3$ & $37.0(3.7)\times 10^3$ & 26.22 & $1.43(99)\times 10^6$  & $1.43(99)\times 10^6$ \\
18.375 & $40(17)\times 10^4$    & $44(19)\times10^4$    &       &  & \\ \hline
\multicolumn{3}{|c|}{$SU(8)$} & \multicolumn{3}{|c|}{} \\ \hline
 44.10 & 514(10)                & 528(10)                  &   &  &  \\
 44.85 & $26.3(3.5)\times 10^3$ & $26.3(3.5)\times 10^3$   &  &   &    \\ 
 45.50 &  $7.1(3.5)\times 10^5$ & $7.1(3.5)\times 10^5$    &  &   &   \\
 45.50 & $12.5(5.1)\times 10^5$ & $12.5(5.1)\times 10^5$   &  &   &  \\ \hline
\multicolumn{3}{|c|}{$SU(10)$} & \multicolumn{3}{|c|}{$SU(12)$} \\ \hline
  69.20  & 4793(211) & 4831(214)  & 99.86  & $4.83(58)\times 10^4$ &  $4.83(58)\times 10^4$   \\  \hline
\end{tabular}
\caption{Average number of sweeps, $\tau_Q$, between $\Delta Q= \pm 1$ changes normalised to
  a standard space-time volume of $(3/\surd\sigma)^4$, with $\Tilde{\tau}_Q$ including a correction for
`near-dislocations'.}
\label{table_tauQ_SUN}
\end{table}

\begin{table}[htb]
\centering
\begin{tabular}{|c|c|c|} \hline
\multicolumn{3}{|c|}{$\ln\{\tau_Q\} = b - c(N)\ln\{a\surd\sigma\}$} \\ \hline
$N$ & $c$ &  $11N/3 - 5$ \\ \hline
3  & 6.82(7)   &  6.0 \\
4  & 9.27(12)  &  9.6$\dot{6}$ \\
5  & 11.13(20) &  13.3$\dot{3}$ \\
6  & 13.50(30) &  17.0 \\
8  & 17.58(43) &  24.3$\dot{3}$ \\ \hline
 \end{tabular}
\caption{Fitted values of $c(N)$ from Fig.\ref{fig_tauQ_KsuN} compared to the asymptotic
  dilute gas prediction in eqn(\ref{eqn_Ia}).}
\label{table_tauQ_a}
\end{table}


\begin{table}[htb]
\centering
\begin{tabular}{|c|c|c|c|} \hline
\multicolumn{4}{|c|}{$SU(2)$ topology} \\ \hline
$\beta$  &  lattice  & $\langle Q_L^2 \rangle$ & $\langle Q_I^2 \rangle$ \\ \hline
2.2986  & $8^316$  & 2.266(9)  & 3.057(12)    \\ 
        & $12^316$ & 7.694(42) & 9.600(49)    \\   \hline
2.3714  & $10^316$ & 2.705(9)  & 3.544(12)    \\ 
        & $14^316$ & 7.343(61) & 9.220(77)    \\   \hline
2.427   & $12^316$ & 2.824(7)  & 3.551(8)     \\ 
        & $16^4$   & 6.635(56) & 8.066(64)    \\ 
        & $20^316$ & 13.09(19) & 15.41(22)   \\  
        & $24^316$ & 22.56(20) & 26.18(23)    \\   \hline
2.509   & $16^320$ & 3.411(13) & 4.067(16)    \\ 
        & $22^320$ & 8.822(58) & 10.137(65)   \\   \hline
2.60    & $22^330$ & 4.491(31)  & 5.109(33)   \\ 
        & $30^4$   & 11.48(10)  & 12.65(12)   \\  \hline
2.65    & $26^334$ & 4.702(41)  & 5.268(45)   \\  \hline
2.70    & $28^340$ & 3.786(36)  & 4.181(39)   \\ 
        & $40^4$   & 11.07(14)  & 12.09(14)   \\ \hline
2.75    & $34^346$ & 4.328(88)  & 4.711(94)   \\   \hline
2.80    & $40^354$ & 4.43(12)   & 4.76(12)   \\   \hline
\end{tabular}
\caption{Average values of $Q^2_L$, after 20 cooling sweeps, in $SU(2)$ for various values of $\beta$ and lattice sizes.
  $Q_I$ is the projection of $Q_L$ to an integer value.}
\label{table_QQ_SU2}
\end{table}



\begin{table}[htb]
\centering
\begin{tabular}{|c|c|c|c||c|c|c|c|} \hline
\multicolumn{4}{|c|}{$SU(3)$ topology} & \multicolumn{4}{|c|}{$SU(4)$ topology} \\ \hline
$\beta$  &  lattice  & $\langle Q^2_L \rangle$ & $\langle Q_I^2 \rangle$
& $\beta$  &  lattice  & $\langle Q^2_L \rangle$ & $\langle Q_I^2 \rangle$ \\ \hline
5.6924 & $8^316$  & 4.034(21) &  5.452(29)  & 10.70 & $12^316$  & 5.813(37) & 7.084(48) \\
5.80   & $10^316$ & 4.021(13) &  5.111(16)  & 10.85 & $14^320$  & 5.718(45) & 6.674(53) \\
5.8941 & $12^316$ & 3.570(35) &  4.352(43)  & 11.02 & $18^320$  & 6.010(95) & 6.78(11) \\
5.99   & $14^320$ & 3.790(48) &  4.428(56)  & 11.20 & $22^4$    & 5.97(33)  & 6.57(36) \\
6.0625 & $14^320$ & 2.312(22) &  2.649(26)  & 11.40 & $26^4$    & 6.19(67)  & 6.68(73) \\
6.235  & $18^326$ & 2.32(12)  &  2.55(13)   & 11.60 & $30^4$    & 5.52(73)  & 5.80(78) \\
6.3380 & $22^330$ & 2.72(13)  &  2.94(14)   &  &  &  &  \\
6.50   & $26^338$ & 2.09(20)  &  2.23(21)   &  &  &  &  \\
6.60   & $32^340$ & 2.60(35)  &  2.74(37)   &  &  &  &  \\
6.70   & $36^344$ & 1.54(30)  &  1.61(31)   &  &  &  &  \\ \hline
\end{tabular}
\caption{Average values of $Q_L^2$, after 20 cooling sweeps, in $SU(3)$ and in  $SU(4)$
  for the values of $\beta$ and lattices shown.
  $Q_I$ is the projection of $Q_L$ to an integer value.}
\label{table_QQ_SU3_SU4}
\end{table}



\begin{table}[htb]
\centering
\begin{tabular}{|c|c|c|c||c|c|c|c|} \hline
\multicolumn{4}{|c|}{$SU(5)$ topology} & \multicolumn{4}{|c|}{$SU(6)$ topology} \\ \hline
$\beta$  &  lattice  & $\langle Q^2_L \rangle$ & $\langle Q_I^2 \rangle$
& $\beta$  &  lattice  & $\langle Q^2_L \rangle$ & $\langle Q_I^2 \rangle$ \\ \hline
16.98   & $10^316$  & 3.273(25)  & 3.906(30) & 24.67   & $10^316$  & 3.178(45)  & 3.747(53) \\
17.22   & $12^316$  & 2.682((60) & 3.073(68) & 25.05   & $12^316$  & 2.50(13)  & 2.83(15) \\
17.43   & $14^320$  & 2.95(13)   & 3.30(14)  & 25.32   & $14^320$  & 2.76(32)  & 3.07(35) \\
17.63   & $16^320$  & 2.86(31)   & 3.15(35)  & 25.35   & $14^320$  & 2.91(60)  & 3.24(67) \\
18.04   & $20^324$  & 3.44(1.27) & 3.71(1.36)& 25.35   & $18^318$  & 5.83(78)  & 6.48(87) \\
18.375  & $24^330$  & 2.21(86)   & 2.35(92)  &    &   &   &  \\ \hline
\end{tabular}
\caption{Average values of $Q^2_L$, after 20 cooling sweeps, in $SU(5)$ and in  $SU(6)$
  for the values of $\beta$ and lattices shown.
  $Q_I$ is the projection of $Q_L$ to an integer value.}
\label{table_QQ_SU5_SU6}
\end{table}



\begin{table}[htb]
\centering
\begin{tabular}{|c|ccc|ccc|} \hline
\multicolumn{7}{|c|}{continuum topological susceptibility} \\ \hline
group  &  $\chi_L^{1/4}/\surd\sigma$ & $\beta\in$ & $\chi^2/n_{df}$ & $\chi_I^{1/4}/\surd\sigma$ & $\beta\in$ & $\chi^2/n_{df}$  \\ \hline
SU(2) & 0.4773(14) & [2.509,2.75]   & 0.46 & 0.4857(14) & [2.509,2.75]   & 0.55 \\
SU(3) & 0.4196(35) & [5.8941,6.60]  & 0.85 & 0.4246(36) & [5.8941,6.60]  & 0.89 \\
SU(4) & 0.3925(25) & [10.70,11.60]  & 0.19 & 0.3964(27) & [10.70,11.60]  & 0.26 \\
SU(5) & 0.3786(59) & [16.98,18.375] & 0.44 & 0.3818(60) & [16.98,18.375] & 0.45  \\
SU(6) & 0.386(13)  & [24.67,25.35]  & 0.29 & 0.390(12)  & [24.67,25.35]  & 0.30 \\ \hline
SU($\infty$) & 0.3655(27)  &        & 0.92 & 0.3681(28) &                & 0.93 \\ \hline
\end{tabular}
\caption{Continuum limit of the topological susceptibility in units of the string tension for the
  gauge groups shown; $\chi_L$ is from $Q^2_L$ and $\chi_I$ is from $Q^2_I$. Fitted range of $\beta$
  also shown, as is the chi-squared per degree of freedom, $\chi^2/n_{df}$.}
\label{table_khiK_SUN_cont}
\end{table}




\begin{table}[htb]
\centering
\begin{tabular}{|ll|ll|ll|ll|} \hline
\multicolumn{8}{|c|}{$\langle{Q}_{hot}\rangle_{Q=Q_I} = Z_Q(\beta) Q_I$} \\ \hline
\multicolumn{2}{|c|}{$SU(2)$} &\multicolumn{2}{|c|}{$SU(3)$} & \multicolumn{2}{|c|}{$SU(4)$} & \multicolumn{2}{|c|}{$SU(5)$} \\ \hline
$\beta$ & $Z_Q(\beta)$ & $\beta$ & $Z_Q(\beta)$ & $\beta$ & $Z_Q(\beta)$ & $\beta$ & $Z_Q(\beta)$  \\ \hline
2.452 & 0.1386(23) &  5.6924 & 0.0646(9)  & 10.70 & 0.0948(14) & 16.98  & 0.0992(15) \\
2.65  & 0.233(12)  &  5.80   & 0.0877(10) & 10.85 & 0.1146(20) & 17.22  & 0.1137(20) \\
2.70  & 0.239(17)  &  5.99   & 0.1306(26) & 11.02 & 0.1347(35) & 17.43  & 0.1348(26) \\
2.75  & 0.258(18)  &  6.0625 & 0.1462(31) & 11.02 & 0.1397(29) & 17.63  & 0.1476(31) \\
2.80  & 0.239(22)  &  6.235  & 0.1808(54) & 11.20 & 0.1524(46) & 18.04  & 0.1587(48) \\
      &            &  6.338  & 0.2044(65) & 11.40 & 0.1710(66) & 18.375 & 0.190(11) \\
      &            &  6.50   & 0.231(10)  & 11.60 & 0.184(10)  &  &  \\
      &            &  6.60   & 0.232(19)  &  &  &  &  \\
      &            &  6.70   & 0.241(28)  &  &  &  &  \\ \hline
\end{tabular}
\caption{Multiplicative renormalisation factor, $Z_Q(\beta)$, relating 
  the average lattice topological charge, $Q_{hot}$,  calculated on the rough
  Monte Carlo fields, and the integer valued topological charge $Q_I$ calculated
  after 20 `cooling' sweeps of those fields. For the gauge groups and $\beta$ shown.}
\label{table_ZQA}
\end{table}



\begin{table}[htb]
\centering
\begin{tabular}{|ll|ll|ll|ll|} \hline
\multicolumn{8}{|c|}{$\langle{Q}_{hot}\rangle_{Q=Q_I} = Z_Q(\beta) Q_I$} \\ \hline
\multicolumn{2}{|c|}{$SU(6)$} &\multicolumn{2}{|c|}{$SU(8)$} & \multicolumn{2}{|c|}{$SU(10)$} & \multicolumn{2}{|c|}{$SU(12)$} \\ \hline
$\beta$ & $Z_Q(\beta)$ & $\beta$ & $Z_Q(\beta)$ & $\beta$ & $Z_Q(\beta)$ & $\beta$ & $Z_Q(\beta)$  \\ \hline
24.67 & 0.0976(20) & 44.10  & 0.0912(19)  & 69.20 & 0.0930(19) & 99.86  & 0.0899(17) \\
25.05 & 0.1132(29) & 44.85  & 0.1194(31)  & 70.38 & 0.1165(27) & 101.55 & 0.1129(26) \\
25.32 & 0.1316(40) & 45.50  & 0.1254(35)  & 71.38 & 0.1283(37) & 103.03 & 0.1292(33) \\
25.55 & 0.1444(56) & 46.10  & 0.1491(50)  & 72.40 & 0.1525(41) & 104.55 & 0.1566(46) \\
26.22 & 0.1716(75) & 46.70  & 0.1617(54)  & 73.35 & 0.1731(52) & 105.95 & 0.1561(66) \\
26.71 & 0.170(12)  & 47.75  & 0.1866(88)  &  &  &  & \\ \hline
\end{tabular}
\caption{Multiplicative renormalisation factor, $Z_Q(\beta)$, relating 
  the average lattice topological charge, $Q_{hot}$,  calculated on the rough
  Monte Carlo fields, and the integer valued topological charge $Q_I$ calculated
  after 20 `cooling' sweeps of those fields. For the gauge groups and $\beta$ shown.}
\label{table_ZQB}
\end{table}



\begin{table}[htb]
\centering
\begin{tabular}{|c|c|c|c|} \hline
\multicolumn{4}{|c|}{$Z^{int}_Q = 1 - z_0 g^2N - z_1 (g^2N)^2$} \\ \hline
$N$ & $z_0$ & $z_1$ & $\chi^2/n_{df}$ \\ \hline
 2 & 0.190(30) & 0.023(9)   & 1.17  \\
 3 & 0.162(10) & 0.0425(31) & 0.62  \\
 4 & 0.156(20) & 0.047(7)   & 1.32  \\
 5 & 0.203(21) & 0.035(7)   & 2.76  \\
 6 & 0.205(30) & 0.036(11)  & 1.37  \\
 8  & 0.187(24) & 0.043(9)   & 1.71   \\
 10 & 0.141(44) & 0.060(16)  & 1.05 \\
 12 & 0.182(24) & 0.071(22)  & 2.22 \\ \hline
\end{tabular}
\caption{Interpolating funtions for the multiplicative renormalisation factor, $Z_Q(\beta)$,
  for our $SU(N)$ calculations, with $g^2N=2N^2/\beta$.}
\label{table_ZQint}
\end{table}




\clearpage




%\begin{figure}[htb]
%\begin	{center}
%\leavevmode
%\input	{plot_EeffK1_SU8.tex}
%\end	{center}
%\caption{Effective energies of the ground state of a fundamental $k=1$ flux tube winding
%  around a spatial torus, extracted from
%  the best correlator $C(t)$ between $t=an_t$ and $t=a(n_t+1)$. For $SU(8)$ and at
%  $\beta=44.10, 44.85, 45.50, 46.10, 46.70, 47.75$ in descending order. Lines are our estimates of
%  the $t\to\infty$ asymptotic energies. The nearly invisible bands around those lines denote the errors
%  on those estimates.}
%\label{fig_EeffK1_SU8}
%\end{figure}



\begin{figure}[htb]
\begin	{center}
\leavevmode
% GNUPLOT: LaTeX picture with Postscript
\begingroup%
\makeatletter%
\newcommand{\GNUPLOTspecial}{%
  \@sanitize\catcode`\%=14\relax\special}%
\setlength{\unitlength}{0.0500bp}%
\begin{picture}(7200,7560)(0,0)%
  {\GNUPLOTspecial{"
%!PS-Adobe-2.0 EPSF-2.0
%%Title: plot_EeffK1_SU8.tex
%%Creator: gnuplot 5.2 patchlevel 8
%%CreationDate: Fri Oct 29 07:43:39 2021
%%DocumentFonts: 
%%BoundingBox: 0 0 360 378
%%EndComments
%%BeginProlog
/gnudict 256 dict def
gnudict begin
%
% The following true/false flags may be edited by hand if desired.
% The unit line width and grayscale image gamma correction may also be changed.
%
/Color true def
/Blacktext true def
/Solid false def
/Dashlength 1 def
/Landscape false def
/Level1 false def
/Level3 false def
/Rounded false def
/ClipToBoundingBox false def
/SuppressPDFMark false def
/TransparentPatterns false def
/gnulinewidth 5.000 def
/userlinewidth gnulinewidth def
/Gamma 1.0 def
/BackgroundColor {-1.000 -1.000 -1.000} def
%
/vshift -93 def
/dl1 {
  10.0 Dashlength userlinewidth gnulinewidth div mul mul mul
  Rounded { currentlinewidth 0.75 mul sub dup 0 le { pop 0.01 } if } if
} def
/dl2 {
  10.0 Dashlength userlinewidth gnulinewidth div mul mul mul
  Rounded { currentlinewidth 0.75 mul add } if
} def
/hpt_ 31.5 def
/vpt_ 31.5 def
/hpt hpt_ def
/vpt vpt_ def
/doclip {
  ClipToBoundingBox {
    newpath 0 0 moveto 360 0 lineto 360 378 lineto 0 378 lineto closepath
    clip
  } if
} def
%
% Gnuplot Prolog Version 5.2 (Dec 2017)
%
%/SuppressPDFMark true def
%
/M {moveto} bind def
/L {lineto} bind def
/R {rmoveto} bind def
/V {rlineto} bind def
/N {newpath moveto} bind def
/Z {closepath} bind def
/C {setrgbcolor} bind def
/f {rlineto fill} bind def
/g {setgray} bind def
/Gshow {show} def   % May be redefined later in the file to support UTF-8
/vpt2 vpt 2 mul def
/hpt2 hpt 2 mul def
/Lshow {currentpoint stroke M 0 vshift R 
	Blacktext {gsave 0 setgray textshow grestore} {textshow} ifelse} def
/Rshow {currentpoint stroke M dup stringwidth pop neg vshift R
	Blacktext {gsave 0 setgray textshow grestore} {textshow} ifelse} def
/Cshow {currentpoint stroke M dup stringwidth pop -2 div vshift R 
	Blacktext {gsave 0 setgray textshow grestore} {textshow} ifelse} def
/UP {dup vpt_ mul /vpt exch def hpt_ mul /hpt exch def
  /hpt2 hpt 2 mul def /vpt2 vpt 2 mul def} def
/DL {Color {setrgbcolor Solid {pop []} if 0 setdash}
 {pop pop pop 0 setgray Solid {pop []} if 0 setdash} ifelse} def
/BL {stroke userlinewidth 2 mul setlinewidth
	Rounded {1 setlinejoin 1 setlinecap} if} def
/AL {stroke userlinewidth 2 div setlinewidth
	Rounded {1 setlinejoin 1 setlinecap} if} def
/UL {dup gnulinewidth mul /userlinewidth exch def
	dup 1 lt {pop 1} if 10 mul /udl exch def} def
/PL {stroke userlinewidth setlinewidth
	Rounded {1 setlinejoin 1 setlinecap} if} def
3.8 setmiterlimit
% Classic Line colors (version 5.0)
/LCw {1 1 1} def
/LCb {0 0 0} def
/LCa {0 0 0} def
/LC0 {1 0 0} def
/LC1 {0 1 0} def
/LC2 {0 0 1} def
/LC3 {1 0 1} def
/LC4 {0 1 1} def
/LC5 {1 1 0} def
/LC6 {0 0 0} def
/LC7 {1 0.3 0} def
/LC8 {0.5 0.5 0.5} def
% Default dash patterns (version 5.0)
/LTB {BL [] LCb DL} def
/LTw {PL [] 1 setgray} def
/LTb {PL [] LCb DL} def
/LTa {AL [1 udl mul 2 udl mul] 0 setdash LCa setrgbcolor} def
/LT0 {PL [] LC0 DL} def
/LT1 {PL [2 dl1 3 dl2] LC1 DL} def
/LT2 {PL [1 dl1 1.5 dl2] LC2 DL} def
/LT3 {PL [6 dl1 2 dl2 1 dl1 2 dl2] LC3 DL} def
/LT4 {PL [1 dl1 2 dl2 6 dl1 2 dl2 1 dl1 2 dl2] LC4 DL} def
/LT5 {PL [4 dl1 2 dl2] LC5 DL} def
/LT6 {PL [1.5 dl1 1.5 dl2 1.5 dl1 1.5 dl2 1.5 dl1 6 dl2] LC6 DL} def
/LT7 {PL [3 dl1 3 dl2 1 dl1 3 dl2] LC7 DL} def
/LT8 {PL [2 dl1 2 dl2 2 dl1 6 dl2] LC8 DL} def
/SL {[] 0 setdash} def
/Pnt {stroke [] 0 setdash gsave 1 setlinecap M 0 0 V stroke grestore} def
/Dia {stroke [] 0 setdash 2 copy vpt add M
  hpt neg vpt neg V hpt vpt neg V
  hpt vpt V hpt neg vpt V closepath stroke
  Pnt} def
/Pls {stroke [] 0 setdash vpt sub M 0 vpt2 V
  currentpoint stroke M
  hpt neg vpt neg R hpt2 0 V stroke
 } def
/Box {stroke [] 0 setdash 2 copy exch hpt sub exch vpt add M
  0 vpt2 neg V hpt2 0 V 0 vpt2 V
  hpt2 neg 0 V closepath stroke
  Pnt} def
/Crs {stroke [] 0 setdash exch hpt sub exch vpt add M
  hpt2 vpt2 neg V currentpoint stroke M
  hpt2 neg 0 R hpt2 vpt2 V stroke} def
/TriU {stroke [] 0 setdash 2 copy vpt 1.12 mul add M
  hpt neg vpt -1.62 mul V
  hpt 2 mul 0 V
  hpt neg vpt 1.62 mul V closepath stroke
  Pnt} def
/Star {2 copy Pls Crs} def
/BoxF {stroke [] 0 setdash exch hpt sub exch vpt add M
  0 vpt2 neg V hpt2 0 V 0 vpt2 V
  hpt2 neg 0 V closepath fill} def
/TriUF {stroke [] 0 setdash vpt 1.12 mul add M
  hpt neg vpt -1.62 mul V
  hpt 2 mul 0 V
  hpt neg vpt 1.62 mul V closepath fill} def
/TriD {stroke [] 0 setdash 2 copy vpt 1.12 mul sub M
  hpt neg vpt 1.62 mul V
  hpt 2 mul 0 V
  hpt neg vpt -1.62 mul V closepath stroke
  Pnt} def
/TriDF {stroke [] 0 setdash vpt 1.12 mul sub M
  hpt neg vpt 1.62 mul V
  hpt 2 mul 0 V
  hpt neg vpt -1.62 mul V closepath fill} def
/DiaF {stroke [] 0 setdash vpt add M
  hpt neg vpt neg V hpt vpt neg V
  hpt vpt V hpt neg vpt V closepath fill} def
/Pent {stroke [] 0 setdash 2 copy gsave
  translate 0 hpt M 4 {72 rotate 0 hpt L} repeat
  closepath stroke grestore Pnt} def
/PentF {stroke [] 0 setdash gsave
  translate 0 hpt M 4 {72 rotate 0 hpt L} repeat
  closepath fill grestore} def
/Circle {stroke [] 0 setdash 2 copy
  hpt 0 360 arc stroke Pnt} def
/CircleF {stroke [] 0 setdash hpt 0 360 arc fill} def
/C0 {BL [] 0 setdash 2 copy moveto vpt 90 450 arc} bind def
/C1 {BL [] 0 setdash 2 copy moveto
	2 copy vpt 0 90 arc closepath fill
	vpt 0 360 arc closepath} bind def
/C2 {BL [] 0 setdash 2 copy moveto
	2 copy vpt 90 180 arc closepath fill
	vpt 0 360 arc closepath} bind def
/C3 {BL [] 0 setdash 2 copy moveto
	2 copy vpt 0 180 arc closepath fill
	vpt 0 360 arc closepath} bind def
/C4 {BL [] 0 setdash 2 copy moveto
	2 copy vpt 180 270 arc closepath fill
	vpt 0 360 arc closepath} bind def
/C5 {BL [] 0 setdash 2 copy moveto
	2 copy vpt 0 90 arc
	2 copy moveto
	2 copy vpt 180 270 arc closepath fill
	vpt 0 360 arc} bind def
/C6 {BL [] 0 setdash 2 copy moveto
	2 copy vpt 90 270 arc closepath fill
	vpt 0 360 arc closepath} bind def
/C7 {BL [] 0 setdash 2 copy moveto
	2 copy vpt 0 270 arc closepath fill
	vpt 0 360 arc closepath} bind def
/C8 {BL [] 0 setdash 2 copy moveto
	2 copy vpt 270 360 arc closepath fill
	vpt 0 360 arc closepath} bind def
/C9 {BL [] 0 setdash 2 copy moveto
	2 copy vpt 270 450 arc closepath fill
	vpt 0 360 arc closepath} bind def
/C10 {BL [] 0 setdash 2 copy 2 copy moveto vpt 270 360 arc closepath fill
	2 copy moveto
	2 copy vpt 90 180 arc closepath fill
	vpt 0 360 arc closepath} bind def
/C11 {BL [] 0 setdash 2 copy moveto
	2 copy vpt 0 180 arc closepath fill
	2 copy moveto
	2 copy vpt 270 360 arc closepath fill
	vpt 0 360 arc closepath} bind def
/C12 {BL [] 0 setdash 2 copy moveto
	2 copy vpt 180 360 arc closepath fill
	vpt 0 360 arc closepath} bind def
/C13 {BL [] 0 setdash 2 copy moveto
	2 copy vpt 0 90 arc closepath fill
	2 copy moveto
	2 copy vpt 180 360 arc closepath fill
	vpt 0 360 arc closepath} bind def
/C14 {BL [] 0 setdash 2 copy moveto
	2 copy vpt 90 360 arc closepath fill
	vpt 0 360 arc} bind def
/C15 {BL [] 0 setdash 2 copy vpt 0 360 arc closepath fill
	vpt 0 360 arc closepath} bind def
/Rec {newpath 4 2 roll moveto 1 index 0 rlineto 0 exch rlineto
	neg 0 rlineto closepath} bind def
/Square {dup Rec} bind def
/Bsquare {vpt sub exch vpt sub exch vpt2 Square} bind def
/S0 {BL [] 0 setdash 2 copy moveto 0 vpt rlineto BL Bsquare} bind def
/S1 {BL [] 0 setdash 2 copy vpt Square fill Bsquare} bind def
/S2 {BL [] 0 setdash 2 copy exch vpt sub exch vpt Square fill Bsquare} bind def
/S3 {BL [] 0 setdash 2 copy exch vpt sub exch vpt2 vpt Rec fill Bsquare} bind def
/S4 {BL [] 0 setdash 2 copy exch vpt sub exch vpt sub vpt Square fill Bsquare} bind def
/S5 {BL [] 0 setdash 2 copy 2 copy vpt Square fill
	exch vpt sub exch vpt sub vpt Square fill Bsquare} bind def
/S6 {BL [] 0 setdash 2 copy exch vpt sub exch vpt sub vpt vpt2 Rec fill Bsquare} bind def
/S7 {BL [] 0 setdash 2 copy exch vpt sub exch vpt sub vpt vpt2 Rec fill
	2 copy vpt Square fill Bsquare} bind def
/S8 {BL [] 0 setdash 2 copy vpt sub vpt Square fill Bsquare} bind def
/S9 {BL [] 0 setdash 2 copy vpt sub vpt vpt2 Rec fill Bsquare} bind def
/S10 {BL [] 0 setdash 2 copy vpt sub vpt Square fill 2 copy exch vpt sub exch vpt Square fill
	Bsquare} bind def
/S11 {BL [] 0 setdash 2 copy vpt sub vpt Square fill 2 copy exch vpt sub exch vpt2 vpt Rec fill
	Bsquare} bind def
/S12 {BL [] 0 setdash 2 copy exch vpt sub exch vpt sub vpt2 vpt Rec fill Bsquare} bind def
/S13 {BL [] 0 setdash 2 copy exch vpt sub exch vpt sub vpt2 vpt Rec fill
	2 copy vpt Square fill Bsquare} bind def
/S14 {BL [] 0 setdash 2 copy exch vpt sub exch vpt sub vpt2 vpt Rec fill
	2 copy exch vpt sub exch vpt Square fill Bsquare} bind def
/S15 {BL [] 0 setdash 2 copy Bsquare fill Bsquare} bind def
/D0 {gsave translate 45 rotate 0 0 S0 stroke grestore} bind def
/D1 {gsave translate 45 rotate 0 0 S1 stroke grestore} bind def
/D2 {gsave translate 45 rotate 0 0 S2 stroke grestore} bind def
/D3 {gsave translate 45 rotate 0 0 S3 stroke grestore} bind def
/D4 {gsave translate 45 rotate 0 0 S4 stroke grestore} bind def
/D5 {gsave translate 45 rotate 0 0 S5 stroke grestore} bind def
/D6 {gsave translate 45 rotate 0 0 S6 stroke grestore} bind def
/D7 {gsave translate 45 rotate 0 0 S7 stroke grestore} bind def
/D8 {gsave translate 45 rotate 0 0 S8 stroke grestore} bind def
/D9 {gsave translate 45 rotate 0 0 S9 stroke grestore} bind def
/D10 {gsave translate 45 rotate 0 0 S10 stroke grestore} bind def
/D11 {gsave translate 45 rotate 0 0 S11 stroke grestore} bind def
/D12 {gsave translate 45 rotate 0 0 S12 stroke grestore} bind def
/D13 {gsave translate 45 rotate 0 0 S13 stroke grestore} bind def
/D14 {gsave translate 45 rotate 0 0 S14 stroke grestore} bind def
/D15 {gsave translate 45 rotate 0 0 S15 stroke grestore} bind def
/DiaE {stroke [] 0 setdash vpt add M
  hpt neg vpt neg V hpt vpt neg V
  hpt vpt V hpt neg vpt V closepath stroke} def
/BoxE {stroke [] 0 setdash exch hpt sub exch vpt add M
  0 vpt2 neg V hpt2 0 V 0 vpt2 V
  hpt2 neg 0 V closepath stroke} def
/TriUE {stroke [] 0 setdash vpt 1.12 mul add M
  hpt neg vpt -1.62 mul V
  hpt 2 mul 0 V
  hpt neg vpt 1.62 mul V closepath stroke} def
/TriDE {stroke [] 0 setdash vpt 1.12 mul sub M
  hpt neg vpt 1.62 mul V
  hpt 2 mul 0 V
  hpt neg vpt -1.62 mul V closepath stroke} def
/PentE {stroke [] 0 setdash gsave
  translate 0 hpt M 4 {72 rotate 0 hpt L} repeat
  closepath stroke grestore} def
/CircE {stroke [] 0 setdash 
  hpt 0 360 arc stroke} def
/Opaque {gsave closepath 1 setgray fill grestore 0 setgray closepath} def
/DiaW {stroke [] 0 setdash vpt add M
  hpt neg vpt neg V hpt vpt neg V
  hpt vpt V hpt neg vpt V Opaque stroke} def
/BoxW {stroke [] 0 setdash exch hpt sub exch vpt add M
  0 vpt2 neg V hpt2 0 V 0 vpt2 V
  hpt2 neg 0 V Opaque stroke} def
/TriUW {stroke [] 0 setdash vpt 1.12 mul add M
  hpt neg vpt -1.62 mul V
  hpt 2 mul 0 V
  hpt neg vpt 1.62 mul V Opaque stroke} def
/TriDW {stroke [] 0 setdash vpt 1.12 mul sub M
  hpt neg vpt 1.62 mul V
  hpt 2 mul 0 V
  hpt neg vpt -1.62 mul V Opaque stroke} def
/PentW {stroke [] 0 setdash gsave
  translate 0 hpt M 4 {72 rotate 0 hpt L} repeat
  Opaque stroke grestore} def
/CircW {stroke [] 0 setdash 
  hpt 0 360 arc Opaque stroke} def
/BoxFill {gsave Rec 1 setgray fill grestore} def
/Density {
  /Fillden exch def
  currentrgbcolor
  /ColB exch def /ColG exch def /ColR exch def
  /ColR ColR Fillden mul Fillden sub 1 add def
  /ColG ColG Fillden mul Fillden sub 1 add def
  /ColB ColB Fillden mul Fillden sub 1 add def
  ColR ColG ColB setrgbcolor} def
/BoxColFill {gsave Rec PolyFill} def
/PolyFill {gsave Density fill grestore grestore} def
/h {rlineto rlineto rlineto closepath gsave fill grestore stroke} bind def
%
% PostScript Level 1 Pattern Fill routine for rectangles
% Usage: x y w h s a XX PatternFill
%	x,y = lower left corner of box to be filled
%	w,h = width and height of box
%	  a = angle in degrees between lines and x-axis
%	 XX = 0/1 for no/yes cross-hatch
%
/PatternFill {gsave /PFa [ 9 2 roll ] def
  PFa 0 get PFa 2 get 2 div add PFa 1 get PFa 3 get 2 div add translate
  PFa 2 get -2 div PFa 3 get -2 div PFa 2 get PFa 3 get Rec
  TransparentPatterns {} {gsave 1 setgray fill grestore} ifelse
  clip
  currentlinewidth 0.5 mul setlinewidth
  /PFs PFa 2 get dup mul PFa 3 get dup mul add sqrt def
  0 0 M PFa 5 get rotate PFs -2 div dup translate
  0 1 PFs PFa 4 get div 1 add floor cvi
	{PFa 4 get mul 0 M 0 PFs V} for
  0 PFa 6 get ne {
	0 1 PFs PFa 4 get div 1 add floor cvi
	{PFa 4 get mul 0 2 1 roll M PFs 0 V} for
 } if
  stroke grestore} def
%
/languagelevel where
 {pop languagelevel} {1} ifelse
dup 2 lt
	{/InterpretLevel1 true def
	 /InterpretLevel3 false def}
	{/InterpretLevel1 Level1 def
	 2 gt
	    {/InterpretLevel3 Level3 def}
	    {/InterpretLevel3 false def}
	 ifelse }
 ifelse
%
% PostScript level 2 pattern fill definitions
%
/Level2PatternFill {
/Tile8x8 {/PaintType 2 /PatternType 1 /TilingType 1 /BBox [0 0 8 8] /XStep 8 /YStep 8}
	bind def
/KeepColor {currentrgbcolor [/Pattern /DeviceRGB] setcolorspace} bind def
<< Tile8x8
 /PaintProc {0.5 setlinewidth pop 0 0 M 8 8 L 0 8 M 8 0 L stroke} 
>> matrix makepattern
/Pat1 exch def
<< Tile8x8
 /PaintProc {0.5 setlinewidth pop 0 0 M 8 8 L 0 8 M 8 0 L stroke
	0 4 M 4 8 L 8 4 L 4 0 L 0 4 L stroke}
>> matrix makepattern
/Pat2 exch def
<< Tile8x8
 /PaintProc {0.5 setlinewidth pop 0 0 M 0 8 L
	8 8 L 8 0 L 0 0 L fill}
>> matrix makepattern
/Pat3 exch def
<< Tile8x8
 /PaintProc {0.5 setlinewidth pop -4 8 M 8 -4 L
	0 12 M 12 0 L stroke}
>> matrix makepattern
/Pat4 exch def
<< Tile8x8
 /PaintProc {0.5 setlinewidth pop -4 0 M 8 12 L
	0 -4 M 12 8 L stroke}
>> matrix makepattern
/Pat5 exch def
<< Tile8x8
 /PaintProc {0.5 setlinewidth pop -2 8 M 4 -4 L
	0 12 M 8 -4 L 4 12 M 10 0 L stroke}
>> matrix makepattern
/Pat6 exch def
<< Tile8x8
 /PaintProc {0.5 setlinewidth pop -2 0 M 4 12 L
	0 -4 M 8 12 L 4 -4 M 10 8 L stroke}
>> matrix makepattern
/Pat7 exch def
<< Tile8x8
 /PaintProc {0.5 setlinewidth pop 8 -2 M -4 4 L
	12 0 M -4 8 L 12 4 M 0 10 L stroke}
>> matrix makepattern
/Pat8 exch def
<< Tile8x8
 /PaintProc {0.5 setlinewidth pop 0 -2 M 12 4 L
	-4 0 M 12 8 L -4 4 M 8 10 L stroke}
>> matrix makepattern
/Pat9 exch def
/Pattern1 {PatternBgnd KeepColor Pat1 setpattern} bind def
/Pattern2 {PatternBgnd KeepColor Pat2 setpattern} bind def
/Pattern3 {PatternBgnd KeepColor Pat3 setpattern} bind def
/Pattern4 {PatternBgnd KeepColor Landscape {Pat5} {Pat4} ifelse setpattern} bind def
/Pattern5 {PatternBgnd KeepColor Landscape {Pat4} {Pat5} ifelse setpattern} bind def
/Pattern6 {PatternBgnd KeepColor Landscape {Pat9} {Pat6} ifelse setpattern} bind def
/Pattern7 {PatternBgnd KeepColor Landscape {Pat8} {Pat7} ifelse setpattern} bind def
} def
%
%
%End of PostScript Level 2 code
%
/PatternBgnd {
  TransparentPatterns {} {gsave 1 setgray fill grestore} ifelse
} def
%
% Substitute for Level 2 pattern fill codes with
% grayscale if Level 2 support is not selected.
%
/Level1PatternFill {
/Pattern1 {0.250 Density} bind def
/Pattern2 {0.500 Density} bind def
/Pattern3 {0.750 Density} bind def
/Pattern4 {0.125 Density} bind def
/Pattern5 {0.375 Density} bind def
/Pattern6 {0.625 Density} bind def
/Pattern7 {0.875 Density} bind def
} def
%
% Now test for support of Level 2 code
%
Level1 {Level1PatternFill} {Level2PatternFill} ifelse
%
/Symbol-Oblique /Symbol findfont [1 0 .167 1 0 0] makefont
dup length dict begin {1 index /FID eq {pop pop} {def} ifelse} forall
currentdict end definefont pop
%
Level1 SuppressPDFMark or 
{} {
/SDict 10 dict def
systemdict /pdfmark known not {
  userdict /pdfmark systemdict /cleartomark get put
} if
SDict begin [
  /Title (plot_EeffK1_SU8.tex)
  /Subject (gnuplot plot)
  /Creator (gnuplot 5.2 patchlevel 8)
%  /Producer (gnuplot)
%  /Keywords ()
  /CreationDate (Fri Oct 29 07:43:39 2021)
  /DOCINFO pdfmark
end
} ifelse
%
% Support for boxed text - Ethan A Merritt Sep 2016
%
/InitTextBox { userdict /TBy2 3 -1 roll put userdict /TBx2 3 -1 roll put
           userdict /TBy1 3 -1 roll put userdict /TBx1 3 -1 roll put
	   /Boxing true def } def
/ExtendTextBox { dup type /stringtype eq
    { Boxing { gsave dup false charpath pathbbox
      dup TBy2 gt {userdict /TBy2 3 -1 roll put} {pop} ifelse
      dup TBx2 gt {userdict /TBx2 3 -1 roll put} {pop} ifelse
      dup TBy1 lt {userdict /TBy1 3 -1 roll put} {pop} ifelse
      dup TBx1 lt {userdict /TBx1 3 -1 roll put} {pop} ifelse
      grestore } if }
    {} ifelse} def
/PopTextBox { newpath TBx1 TBxmargin sub TBy1 TBymargin sub M
               TBx1 TBxmargin sub TBy2 TBymargin add L
	       TBx2 TBxmargin add TBy2 TBymargin add L
	       TBx2 TBxmargin add TBy1 TBymargin sub L closepath } def
/DrawTextBox { PopTextBox stroke /Boxing false def} def
/FillTextBox { gsave PopTextBox fill grestore /Boxing false def} def
0 0 0 0 InitTextBox
/TBxmargin 20 def
/TBymargin 20 def
/Boxing false def
/textshow { ExtendTextBox Gshow } def
%
end
%%EndProlog
%%Page: 1 1
gnudict begin
gsave
doclip
0 0 translate
0.050 0.050 scale
0 setgray
newpath
BackgroundColor 0 lt 3 1 roll 0 lt exch 0 lt or or not {BackgroundColor C 1.000 0 0 7200.00 7560.00 BoxColFill} if
1.000 UL
LTb
LCb setrgbcolor
2100 896 M
63 0 V
4532 0 R
-63 0 V
stroke
LTb
LCb setrgbcolor
2100 1605 M
63 0 V
4532 0 R
-63 0 V
stroke
LTb
LCb setrgbcolor
2100 2314 M
63 0 V
4532 0 R
-63 0 V
stroke
LTb
LCb setrgbcolor
2100 3024 M
63 0 V
4532 0 R
-63 0 V
stroke
LTb
LCb setrgbcolor
2100 3733 M
63 0 V
4532 0 R
-63 0 V
stroke
LTb
LCb setrgbcolor
2100 4442 M
63 0 V
4532 0 R
-63 0 V
stroke
LTb
LCb setrgbcolor
2100 5151 M
63 0 V
4532 0 R
-63 0 V
stroke
LTb
LCb setrgbcolor
2100 5861 M
63 0 V
4532 0 R
-63 0 V
stroke
LTb
LCb setrgbcolor
2100 6570 M
63 0 V
4532 0 R
-63 0 V
stroke
LTb
LCb setrgbcolor
2100 7279 M
63 0 V
4532 0 R
-63 0 V
stroke
LTb
LCb setrgbcolor
2100 896 M
0 63 V
0 6320 R
0 -63 V
stroke
LTb
LCb setrgbcolor
2428 896 M
0 63 V
0 6320 R
0 -63 V
stroke
LTb
LCb setrgbcolor
2756 896 M
0 63 V
0 6320 R
0 -63 V
stroke
LTb
LCb setrgbcolor
3085 896 M
0 63 V
0 6320 R
0 -63 V
stroke
LTb
LCb setrgbcolor
3413 896 M
0 63 V
0 6320 R
0 -63 V
stroke
LTb
LCb setrgbcolor
3741 896 M
0 63 V
0 6320 R
0 -63 V
stroke
LTb
LCb setrgbcolor
4069 896 M
0 63 V
0 6320 R
0 -63 V
stroke
LTb
LCb setrgbcolor
4398 896 M
0 63 V
0 6320 R
0 -63 V
stroke
LTb
LCb setrgbcolor
4726 896 M
0 63 V
0 6320 R
0 -63 V
stroke
LTb
LCb setrgbcolor
5054 896 M
0 63 V
0 6320 R
0 -63 V
stroke
LTb
LCb setrgbcolor
5382 896 M
0 63 V
0 6320 R
0 -63 V
stroke
LTb
LCb setrgbcolor
5710 896 M
0 63 V
0 6320 R
0 -63 V
stroke
LTb
LCb setrgbcolor
6039 896 M
0 63 V
0 6320 R
0 -63 V
stroke
LTb
LCb setrgbcolor
6367 896 M
0 63 V
0 6320 R
0 -63 V
stroke
LTb
LCb setrgbcolor
6695 896 M
0 63 V
0 6320 R
0 -63 V
stroke
LTb
LCb setrgbcolor
1.000 UL
LTb
LCb setrgbcolor
2100 7279 N
0 -6383 V
4595 0 V
0 6383 V
-4595 0 V
Z stroke
1.000 UP
1.000 UL
LTb
LCb setrgbcolor
LCb setrgbcolor
LTb
LCb setrgbcolor
LTb
1.000 UL
LTb
0.58 0.00 0.83 C
gsave 2100 2973 N 4595 0 V 0 20 V -4595 0 V 0.20 PolyFill
1.500 UP
1.000 UL
LTb
0.58 0.00 0.83 C
2264 3087 M
0 6 V
328 -91 R
0 8 V
329 -25 R
0 9 V
328 -15 R
0 11 V
328 -12 R
0 15 V
328 -17 R
0 20 V
328 -30 R
0 28 V
329 -33 R
0 34 V
328 -38 R
0 47 V
328 -57 R
0 60 V
328 -71 R
0 78 V
328 -117 R
0 93 V
329 -65 R
0 107 V
2264 3090 CircleF
2592 3006 CircleF
2921 2990 CircleF
3249 2985 CircleF
3577 2985 CircleF
3905 2986 CircleF
4233 2980 CircleF
4562 2978 CircleF
4890 2981 CircleF
5218 2977 CircleF
5546 2975 CircleF
5874 2943 CircleF
6203 2978 CircleF
1.500 UL
LTb
0.58 0.00 0.83 C
2100 2983 M
46 0 V
47 0 V
46 0 V
47 0 V
46 0 V
46 0 V
47 0 V
46 0 V
47 0 V
46 0 V
47 0 V
46 0 V
46 0 V
47 0 V
46 0 V
47 0 V
46 0 V
46 0 V
47 0 V
46 0 V
47 0 V
46 0 V
47 0 V
46 0 V
46 0 V
47 0 V
46 0 V
47 0 V
46 0 V
46 0 V
47 0 V
46 0 V
47 0 V
46 0 V
46 0 V
47 0 V
46 0 V
47 0 V
46 0 V
47 0 V
46 0 V
46 0 V
47 0 V
46 0 V
47 0 V
46 0 V
46 0 V
47 0 V
46 0 V
47 0 V
46 0 V
47 0 V
46 0 V
46 0 V
47 0 V
46 0 V
47 0 V
46 0 V
46 0 V
47 0 V
46 0 V
47 0 V
46 0 V
47 0 V
46 0 V
46 0 V
47 0 V
46 0 V
47 0 V
46 0 V
46 0 V
47 0 V
46 0 V
47 0 V
46 0 V
46 0 V
47 0 V
46 0 V
47 0 V
46 0 V
47 0 V
46 0 V
46 0 V
47 0 V
46 0 V
47 0 V
46 0 V
46 0 V
47 0 V
46 0 V
47 0 V
46 0 V
47 0 V
46 0 V
46 0 V
47 0 V
46 0 V
47 0 V
46 0 V
stroke
1.000 UL
LTb
0.58 0.00 0.83 C
gsave 2100 3487 N 4595 0 V 0 29 V -4595 0 V 0.20 PolyFill
1.500 UP
1.000 UL
LTb
0.58 0.00 0.83 C
2264 3626 M
0 6 V
328 -110 R
0 10 V
329 -34 R
0 15 V
328 -18 R
0 18 V
328 -34 R
0 27 V
328 -27 R
0 36 V
328 -53 R
0 52 V
329 -11 R
0 75 V
328 -119 R
0 105 V
328 -118 R
0 125 V
328 -168 R
0 164 V
2264 3629 Circle
2592 3527 Circle
2921 3506 Circle
3249 3504 Circle
3577 3493 Circle
3905 3497 Circle
4233 3488 Circle
4562 3540 Circle
4890 3511 Circle
5218 3508 Circle
5546 3485 Circle
1.500 UL
LTb
0.58 0.00 0.83 C
2100 3502 M
46 0 V
47 0 V
46 0 V
47 0 V
46 0 V
46 0 V
47 0 V
46 0 V
47 0 V
46 0 V
47 0 V
46 0 V
46 0 V
47 0 V
46 0 V
47 0 V
46 0 V
46 0 V
47 0 V
46 0 V
47 0 V
46 0 V
47 0 V
46 0 V
46 0 V
47 0 V
46 0 V
47 0 V
46 0 V
46 0 V
47 0 V
46 0 V
47 0 V
46 0 V
46 0 V
47 0 V
46 0 V
47 0 V
46 0 V
47 0 V
46 0 V
46 0 V
47 0 V
46 0 V
47 0 V
46 0 V
46 0 V
47 0 V
46 0 V
47 0 V
46 0 V
47 0 V
46 0 V
46 0 V
47 0 V
46 0 V
47 0 V
46 0 V
46 0 V
47 0 V
46 0 V
47 0 V
46 0 V
47 0 V
46 0 V
46 0 V
47 0 V
46 0 V
47 0 V
46 0 V
46 0 V
47 0 V
46 0 V
47 0 V
46 0 V
46 0 V
47 0 V
46 0 V
47 0 V
46 0 V
47 0 V
46 0 V
46 0 V
47 0 V
46 0 V
47 0 V
46 0 V
46 0 V
47 0 V
46 0 V
47 0 V
46 0 V
47 0 V
46 0 V
46 0 V
47 0 V
46 0 V
47 0 V
46 0 V
stroke
1.000 UL
LTb
0.58 0.00 0.83 C
gsave 2100 3863 N 4595 0 V 0 29 V -4595 0 V 0.20 PolyFill
1.500 UP
1.000 UL
LTb
0.58 0.00 0.83 C
2264 4065 M
0 11 V
328 -151 R
0 13 V
329 -48 R
0 16 V
328 -34 R
0 21 V
328 -49 R
0 35 V
328 -24 R
0 50 V
328 -81 R
0 65 V
329 -77 R
0 104 V
328 -142 R
0 142 V
2264 4070 BoxF
2592 3932 BoxF
2921 3898 BoxF
3249 3882 BoxF
3577 3862 BoxF
3905 3880 BoxF
4233 3856 BoxF
4562 3864 BoxF
4890 3845 BoxF
1.500 UL
LTb
0.58 0.00 0.83 C
2100 3878 M
46 0 V
47 0 V
46 0 V
47 0 V
46 0 V
46 0 V
47 0 V
46 0 V
47 0 V
46 0 V
47 0 V
46 0 V
46 0 V
47 0 V
46 0 V
47 0 V
46 0 V
46 0 V
47 0 V
46 0 V
47 0 V
46 0 V
47 0 V
46 0 V
46 0 V
47 0 V
46 0 V
47 0 V
46 0 V
46 0 V
47 0 V
46 0 V
47 0 V
46 0 V
46 0 V
47 0 V
46 0 V
47 0 V
46 0 V
47 0 V
46 0 V
46 0 V
47 0 V
46 0 V
47 0 V
46 0 V
46 0 V
47 0 V
46 0 V
47 0 V
46 0 V
47 0 V
46 0 V
46 0 V
47 0 V
46 0 V
47 0 V
46 0 V
46 0 V
47 0 V
46 0 V
47 0 V
46 0 V
47 0 V
46 0 V
46 0 V
47 0 V
46 0 V
47 0 V
46 0 V
46 0 V
47 0 V
46 0 V
47 0 V
46 0 V
46 0 V
47 0 V
46 0 V
47 0 V
46 0 V
47 0 V
46 0 V
46 0 V
47 0 V
46 0 V
47 0 V
46 0 V
46 0 V
47 0 V
46 0 V
47 0 V
46 0 V
47 0 V
46 0 V
46 0 V
47 0 V
46 0 V
47 0 V
46 0 V
stroke
1.000 UL
LTb
0.58 0.00 0.83 C
gsave 2100 4268 N 4595 0 V 0 34 V -4595 0 V 0.20 PolyFill
1.500 UP
1.000 UL
LTb
0.58 0.00 0.83 C
2264 4447 M
0 9 V
328 -154 R
0 13 V
329 -48 R
0 23 V
328 -30 R
0 24 V
328 0 R
0 47 V
328 -96 R
0 73 V
328 -58 R
0 101 V
329 -90 R
0 178 V
328 -205 R
0 230 V
2264 4451 Box
2592 4308 Box
2921 4278 Box
3249 4272 Box
3577 4308 Box
3905 4272 Box
4233 4300 Box
4562 4350 Box
4890 4349 Box
1.500 UL
LTb
0.58 0.00 0.83 C
2100 4285 M
46 0 V
47 0 V
46 0 V
47 0 V
46 0 V
46 0 V
47 0 V
46 0 V
47 0 V
46 0 V
47 0 V
46 0 V
46 0 V
47 0 V
46 0 V
47 0 V
46 0 V
46 0 V
47 0 V
46 0 V
47 0 V
46 0 V
47 0 V
46 0 V
46 0 V
47 0 V
46 0 V
47 0 V
46 0 V
46 0 V
47 0 V
46 0 V
47 0 V
46 0 V
46 0 V
47 0 V
46 0 V
47 0 V
46 0 V
47 0 V
46 0 V
46 0 V
47 0 V
46 0 V
47 0 V
46 0 V
46 0 V
47 0 V
46 0 V
47 0 V
46 0 V
47 0 V
46 0 V
46 0 V
47 0 V
46 0 V
47 0 V
46 0 V
46 0 V
47 0 V
46 0 V
47 0 V
46 0 V
47 0 V
46 0 V
46 0 V
47 0 V
46 0 V
47 0 V
46 0 V
46 0 V
47 0 V
46 0 V
47 0 V
46 0 V
46 0 V
47 0 V
46 0 V
47 0 V
46 0 V
47 0 V
46 0 V
46 0 V
47 0 V
46 0 V
47 0 V
46 0 V
46 0 V
47 0 V
46 0 V
47 0 V
46 0 V
47 0 V
46 0 V
46 0 V
47 0 V
46 0 V
47 0 V
46 0 V
stroke
1.000 UL
LTb
0.58 0.00 0.83 C
gsave 2100 4786 N 4595 0 V 0 30 V -4595 0 V 0.20 PolyFill
1.500 UP
1.000 UL
LTb
0.58 0.00 0.83 C
2264 4968 M
0 9 V
328 -151 R
0 13 V
329 -29 R
0 18 V
328 -62 R
0 37 V
328 -36 R
0 56 V
328 -79 R
0 94 V
328 -113 R
0 180 V
2264 4972 DiaF
2592 4832 DiaF
2921 4819 DiaF
3249 4784 DiaF
3577 4795 DiaF
3905 4791 DiaF
4233 4815 DiaF
1.500 UL
LTb
0.58 0.00 0.83 C
2100 4801 M
46 0 V
47 0 V
46 0 V
47 0 V
46 0 V
46 0 V
47 0 V
46 0 V
47 0 V
46 0 V
47 0 V
46 0 V
46 0 V
47 0 V
46 0 V
47 0 V
46 0 V
46 0 V
47 0 V
46 0 V
47 0 V
46 0 V
47 0 V
46 0 V
46 0 V
47 0 V
46 0 V
47 0 V
46 0 V
46 0 V
47 0 V
46 0 V
47 0 V
46 0 V
46 0 V
47 0 V
46 0 V
47 0 V
46 0 V
47 0 V
46 0 V
46 0 V
47 0 V
46 0 V
47 0 V
46 0 V
46 0 V
47 0 V
46 0 V
47 0 V
46 0 V
47 0 V
46 0 V
46 0 V
47 0 V
46 0 V
47 0 V
46 0 V
46 0 V
47 0 V
46 0 V
47 0 V
46 0 V
47 0 V
46 0 V
46 0 V
47 0 V
46 0 V
47 0 V
46 0 V
46 0 V
47 0 V
46 0 V
47 0 V
46 0 V
46 0 V
47 0 V
46 0 V
47 0 V
46 0 V
47 0 V
46 0 V
46 0 V
47 0 V
46 0 V
47 0 V
46 0 V
46 0 V
47 0 V
46 0 V
47 0 V
46 0 V
47 0 V
46 0 V
46 0 V
47 0 V
46 0 V
47 0 V
46 0 V
stroke
1.000 UL
LTb
0.58 0.00 0.83 C
gsave 2100 5885 N 4595 0 V 0 46 V -4595 0 V 0.20 PolyFill
1.500 UP
1.000 UL
LTb
0.58 0.00 0.83 C
2264 6115 M
0 10 V
328 -184 R
0 18 V
329 -57 R
0 36 V
328 -77 R
0 77 V
328 -170 R
0 157 V
328 -355 R
0 293 V
328 -357 R
0 503 V
2264 6120 Dia
2592 5950 Dia
2921 5920 Dia
3249 5900 Dia
3577 5846 Dia
3905 5717 Dia
4233 5758 Dia
1.500 UL
LTb
0.58 0.00 0.83 C
2100 5908 M
46 0 V
47 0 V
46 0 V
47 0 V
46 0 V
46 0 V
47 0 V
46 0 V
47 0 V
46 0 V
47 0 V
46 0 V
46 0 V
47 0 V
46 0 V
47 0 V
46 0 V
46 0 V
47 0 V
46 0 V
47 0 V
46 0 V
47 0 V
46 0 V
46 0 V
47 0 V
46 0 V
47 0 V
46 0 V
46 0 V
47 0 V
46 0 V
47 0 V
46 0 V
46 0 V
47 0 V
46 0 V
47 0 V
46 0 V
47 0 V
46 0 V
46 0 V
47 0 V
46 0 V
47 0 V
46 0 V
46 0 V
47 0 V
46 0 V
47 0 V
46 0 V
47 0 V
46 0 V
46 0 V
47 0 V
46 0 V
47 0 V
46 0 V
46 0 V
47 0 V
46 0 V
47 0 V
46 0 V
47 0 V
46 0 V
46 0 V
47 0 V
46 0 V
47 0 V
46 0 V
46 0 V
47 0 V
46 0 V
47 0 V
46 0 V
46 0 V
47 0 V
46 0 V
47 0 V
46 0 V
47 0 V
46 0 V
46 0 V
47 0 V
46 0 V
47 0 V
46 0 V
46 0 V
47 0 V
46 0 V
47 0 V
46 0 V
47 0 V
46 0 V
46 0 V
47 0 V
46 0 V
47 0 V
46 0 V
stroke
2.000 UL
LTb
LCb setrgbcolor
1.000 UL
LTb
LCb setrgbcolor
2100 7279 N
0 -6383 V
4595 0 V
0 6383 V
-4595 0 V
Z stroke
1.000 UP
1.000 UL
LTb
LCb setrgbcolor
stroke
grestore
end
showpage
  }}%
\fontsize{14}{\baselineskip}\selectfont
  \put(4397,196){\makebox(0,0){{$n_t$}}}%
  \put(546,6047){\makebox(0,0){{$aE_{eff}^{k=1}$}}}%
  \put(6695,616){\makebox(0,0){\strut{}\ {$14$}}}%
  \put(6367,616){\makebox(0,0){\strut{}\ {$13$}}}%
  \put(6039,616){\makebox(0,0){\strut{}\ {$12$}}}%
  \put(5710,616){\makebox(0,0){\strut{}\ {$11$}}}%
  \put(5382,616){\makebox(0,0){\strut{}\ {$10$}}}%
  \put(5054,616){\makebox(0,0){\strut{}\ {$9$}}}%
  \put(4726,616){\makebox(0,0){\strut{}\ {$8$}}}%
  \put(4398,616){\makebox(0,0){\strut{}\ {$7$}}}%
  \put(4069,616){\makebox(0,0){\strut{}\ {$6$}}}%
  \put(3741,616){\makebox(0,0){\strut{}\ {$5$}}}%
  \put(3413,616){\makebox(0,0){\strut{}\ {$4$}}}%
  \put(3085,616){\makebox(0,0){\strut{}\ {$3$}}}%
  \put(2756,616){\makebox(0,0){\strut{}\ {$2$}}}%
  \put(2428,616){\makebox(0,0){\strut{}\ {$1$}}}%
  \put(2100,616){\makebox(0,0){\strut{}\ {$0$}}}%
  \put(1932,7279){\makebox(0,0)[r]{\strut{}\ \ {$0.9$}}}%
  \put(1932,6570){\makebox(0,0)[r]{\strut{}\ \ {$0.8$}}}%
  \put(1932,5861){\makebox(0,0)[r]{\strut{}\ \ {$0.7$}}}%
  \put(1932,5151){\makebox(0,0)[r]{\strut{}\ \ {$0.6$}}}%
  \put(1932,4442){\makebox(0,0)[r]{\strut{}\ \ {$0.5$}}}%
  \put(1932,3733){\makebox(0,0)[r]{\strut{}\ \ {$0.4$}}}%
  \put(1932,3024){\makebox(0,0)[r]{\strut{}\ \ {$0.3$}}}%
  \put(1932,2314){\makebox(0,0)[r]{\strut{}\ \ {$0.2$}}}%
  \put(1932,1605){\makebox(0,0)[r]{\strut{}\ \ {$0.1$}}}%
  \put(1932,896){\makebox(0,0)[r]{\strut{}\ \ {$0$}}}%
\end{picture}%
\endgroup
\endinput

\end	{center}
\caption{Effective energies of the ground state of a fundamental $k=1$ flux tube winding
  around a spatial torus, extracted from
  the best correlator $C(t)$ between $t=an_t$ and $t=a(n_t+1)$. For $SU(8)$ and at
  $\beta=44.10, 44.85, 45.50, 46.10, 46.70, 47.75$ in descending order. Lines are our estimates of
  the $t\to\infty$ asymptotic energies. The nearly invisible bands around those lines denote the errors
  on those estimates.}
\label{fig_EeffK1_SU8}
\end{figure}



\begin{figure}[htb]
\begin	{center}
\leavevmode
% GNUPLOT: LaTeX picture with Postscript
\begingroup%
\makeatletter%
\newcommand{\GNUPLOTspecial}{%
  \@sanitize\catcode`\%=14\relax\special}%
\setlength{\unitlength}{0.0500bp}%
\begin{picture}(7200,7560)(0,0)%
  {\GNUPLOTspecial{"
%!PS-Adobe-2.0 EPSF-2.0
%%Title: plot_EeffK1b_SU8.tex
%%Creator: gnuplot 5.0 patchlevel 3
%%CreationDate: Mon Nov  8 09:38:58 2021
%%DocumentFonts: 
%%BoundingBox: 0 0 360 378
%%EndComments
%%BeginProlog
/gnudict 256 dict def
gnudict begin
%
% The following true/false flags may be edited by hand if desired.
% The unit line width and grayscale image gamma correction may also be changed.
%
/Color true def
/Blacktext true def
/Solid false def
/Dashlength 1 def
/Landscape false def
/Level1 false def
/Level3 false def
/Rounded false def
/ClipToBoundingBox false def
/SuppressPDFMark false def
/TransparentPatterns false def
/gnulinewidth 5.000 def
/userlinewidth gnulinewidth def
/Gamma 1.0 def
/BackgroundColor {-1.000 -1.000 -1.000} def
%
/vshift -66 def
/dl1 {
  10.0 Dashlength userlinewidth gnulinewidth div mul mul mul
  Rounded { currentlinewidth 0.75 mul sub dup 0 le { pop 0.01 } if } if
} def
/dl2 {
  10.0 Dashlength userlinewidth gnulinewidth div mul mul mul
  Rounded { currentlinewidth 0.75 mul add } if
} def
/hpt_ 31.5 def
/vpt_ 31.5 def
/hpt hpt_ def
/vpt vpt_ def
/doclip {
  ClipToBoundingBox {
    newpath 0 0 moveto 360 0 lineto 360 378 lineto 0 378 lineto closepath
    clip
  } if
} def
%
% Gnuplot Prolog Version 5.1 (Oct 2015)
%
%/SuppressPDFMark true def
%
/M {moveto} bind def
/L {lineto} bind def
/R {rmoveto} bind def
/V {rlineto} bind def
/N {newpath moveto} bind def
/Z {closepath} bind def
/C {setrgbcolor} bind def
/f {rlineto fill} bind def
/g {setgray} bind def
/Gshow {show} def   % May be redefined later in the file to support UTF-8
/vpt2 vpt 2 mul def
/hpt2 hpt 2 mul def
/Lshow {currentpoint stroke M 0 vshift R 
	Blacktext {gsave 0 setgray textshow grestore} {textshow} ifelse} def
/Rshow {currentpoint stroke M dup stringwidth pop neg vshift R
	Blacktext {gsave 0 setgray textshow grestore} {textshow} ifelse} def
/Cshow {currentpoint stroke M dup stringwidth pop -2 div vshift R 
	Blacktext {gsave 0 setgray textshow grestore} {textshow} ifelse} def
/UP {dup vpt_ mul /vpt exch def hpt_ mul /hpt exch def
  /hpt2 hpt 2 mul def /vpt2 vpt 2 mul def} def
/DL {Color {setrgbcolor Solid {pop []} if 0 setdash}
 {pop pop pop 0 setgray Solid {pop []} if 0 setdash} ifelse} def
/BL {stroke userlinewidth 2 mul setlinewidth
	Rounded {1 setlinejoin 1 setlinecap} if} def
/AL {stroke userlinewidth 2 div setlinewidth
	Rounded {1 setlinejoin 1 setlinecap} if} def
/UL {dup gnulinewidth mul /userlinewidth exch def
	dup 1 lt {pop 1} if 10 mul /udl exch def} def
/PL {stroke userlinewidth setlinewidth
	Rounded {1 setlinejoin 1 setlinecap} if} def
3.8 setmiterlimit
% Classic Line colors (version 5.0)
/LCw {1 1 1} def
/LCb {0 0 0} def
/LCa {0 0 0} def
/LC0 {1 0 0} def
/LC1 {0 1 0} def
/LC2 {0 0 1} def
/LC3 {1 0 1} def
/LC4 {0 1 1} def
/LC5 {1 1 0} def
/LC6 {0 0 0} def
/LC7 {1 0.3 0} def
/LC8 {0.5 0.5 0.5} def
% Default dash patterns (version 5.0)
/LTB {BL [] LCb DL} def
/LTw {PL [] 1 setgray} def
/LTb {PL [] LCb DL} def
/LTa {AL [1 udl mul 2 udl mul] 0 setdash LCa setrgbcolor} def
/LT0 {PL [] LC0 DL} def
/LT1 {PL [2 dl1 3 dl2] LC1 DL} def
/LT2 {PL [1 dl1 1.5 dl2] LC2 DL} def
/LT3 {PL [6 dl1 2 dl2 1 dl1 2 dl2] LC3 DL} def
/LT4 {PL [1 dl1 2 dl2 6 dl1 2 dl2 1 dl1 2 dl2] LC4 DL} def
/LT5 {PL [4 dl1 2 dl2] LC5 DL} def
/LT6 {PL [1.5 dl1 1.5 dl2 1.5 dl1 1.5 dl2 1.5 dl1 6 dl2] LC6 DL} def
/LT7 {PL [3 dl1 3 dl2 1 dl1 3 dl2] LC7 DL} def
/LT8 {PL [2 dl1 2 dl2 2 dl1 6 dl2] LC8 DL} def
/SL {[] 0 setdash} def
/Pnt {stroke [] 0 setdash gsave 1 setlinecap M 0 0 V stroke grestore} def
/Dia {stroke [] 0 setdash 2 copy vpt add M
  hpt neg vpt neg V hpt vpt neg V
  hpt vpt V hpt neg vpt V closepath stroke
  Pnt} def
/Pls {stroke [] 0 setdash vpt sub M 0 vpt2 V
  currentpoint stroke M
  hpt neg vpt neg R hpt2 0 V stroke
 } def
/Box {stroke [] 0 setdash 2 copy exch hpt sub exch vpt add M
  0 vpt2 neg V hpt2 0 V 0 vpt2 V
  hpt2 neg 0 V closepath stroke
  Pnt} def
/Crs {stroke [] 0 setdash exch hpt sub exch vpt add M
  hpt2 vpt2 neg V currentpoint stroke M
  hpt2 neg 0 R hpt2 vpt2 V stroke} def
/TriU {stroke [] 0 setdash 2 copy vpt 1.12 mul add M
  hpt neg vpt -1.62 mul V
  hpt 2 mul 0 V
  hpt neg vpt 1.62 mul V closepath stroke
  Pnt} def
/Star {2 copy Pls Crs} def
/BoxF {stroke [] 0 setdash exch hpt sub exch vpt add M
  0 vpt2 neg V hpt2 0 V 0 vpt2 V
  hpt2 neg 0 V closepath fill} def
/TriUF {stroke [] 0 setdash vpt 1.12 mul add M
  hpt neg vpt -1.62 mul V
  hpt 2 mul 0 V
  hpt neg vpt 1.62 mul V closepath fill} def
/TriD {stroke [] 0 setdash 2 copy vpt 1.12 mul sub M
  hpt neg vpt 1.62 mul V
  hpt 2 mul 0 V
  hpt neg vpt -1.62 mul V closepath stroke
  Pnt} def
/TriDF {stroke [] 0 setdash vpt 1.12 mul sub M
  hpt neg vpt 1.62 mul V
  hpt 2 mul 0 V
  hpt neg vpt -1.62 mul V closepath fill} def
/DiaF {stroke [] 0 setdash vpt add M
  hpt neg vpt neg V hpt vpt neg V
  hpt vpt V hpt neg vpt V closepath fill} def
/Pent {stroke [] 0 setdash 2 copy gsave
  translate 0 hpt M 4 {72 rotate 0 hpt L} repeat
  closepath stroke grestore Pnt} def
/PentF {stroke [] 0 setdash gsave
  translate 0 hpt M 4 {72 rotate 0 hpt L} repeat
  closepath fill grestore} def
/Circle {stroke [] 0 setdash 2 copy
  hpt 0 360 arc stroke Pnt} def
/CircleF {stroke [] 0 setdash hpt 0 360 arc fill} def
/C0 {BL [] 0 setdash 2 copy moveto vpt 90 450 arc} bind def
/C1 {BL [] 0 setdash 2 copy moveto
	2 copy vpt 0 90 arc closepath fill
	vpt 0 360 arc closepath} bind def
/C2 {BL [] 0 setdash 2 copy moveto
	2 copy vpt 90 180 arc closepath fill
	vpt 0 360 arc closepath} bind def
/C3 {BL [] 0 setdash 2 copy moveto
	2 copy vpt 0 180 arc closepath fill
	vpt 0 360 arc closepath} bind def
/C4 {BL [] 0 setdash 2 copy moveto
	2 copy vpt 180 270 arc closepath fill
	vpt 0 360 arc closepath} bind def
/C5 {BL [] 0 setdash 2 copy moveto
	2 copy vpt 0 90 arc
	2 copy moveto
	2 copy vpt 180 270 arc closepath fill
	vpt 0 360 arc} bind def
/C6 {BL [] 0 setdash 2 copy moveto
	2 copy vpt 90 270 arc closepath fill
	vpt 0 360 arc closepath} bind def
/C7 {BL [] 0 setdash 2 copy moveto
	2 copy vpt 0 270 arc closepath fill
	vpt 0 360 arc closepath} bind def
/C8 {BL [] 0 setdash 2 copy moveto
	2 copy vpt 270 360 arc closepath fill
	vpt 0 360 arc closepath} bind def
/C9 {BL [] 0 setdash 2 copy moveto
	2 copy vpt 270 450 arc closepath fill
	vpt 0 360 arc closepath} bind def
/C10 {BL [] 0 setdash 2 copy 2 copy moveto vpt 270 360 arc closepath fill
	2 copy moveto
	2 copy vpt 90 180 arc closepath fill
	vpt 0 360 arc closepath} bind def
/C11 {BL [] 0 setdash 2 copy moveto
	2 copy vpt 0 180 arc closepath fill
	2 copy moveto
	2 copy vpt 270 360 arc closepath fill
	vpt 0 360 arc closepath} bind def
/C12 {BL [] 0 setdash 2 copy moveto
	2 copy vpt 180 360 arc closepath fill
	vpt 0 360 arc closepath} bind def
/C13 {BL [] 0 setdash 2 copy moveto
	2 copy vpt 0 90 arc closepath fill
	2 copy moveto
	2 copy vpt 180 360 arc closepath fill
	vpt 0 360 arc closepath} bind def
/C14 {BL [] 0 setdash 2 copy moveto
	2 copy vpt 90 360 arc closepath fill
	vpt 0 360 arc} bind def
/C15 {BL [] 0 setdash 2 copy vpt 0 360 arc closepath fill
	vpt 0 360 arc closepath} bind def
/Rec {newpath 4 2 roll moveto 1 index 0 rlineto 0 exch rlineto
	neg 0 rlineto closepath} bind def
/Square {dup Rec} bind def
/Bsquare {vpt sub exch vpt sub exch vpt2 Square} bind def
/S0 {BL [] 0 setdash 2 copy moveto 0 vpt rlineto BL Bsquare} bind def
/S1 {BL [] 0 setdash 2 copy vpt Square fill Bsquare} bind def
/S2 {BL [] 0 setdash 2 copy exch vpt sub exch vpt Square fill Bsquare} bind def
/S3 {BL [] 0 setdash 2 copy exch vpt sub exch vpt2 vpt Rec fill Bsquare} bind def
/S4 {BL [] 0 setdash 2 copy exch vpt sub exch vpt sub vpt Square fill Bsquare} bind def
/S5 {BL [] 0 setdash 2 copy 2 copy vpt Square fill
	exch vpt sub exch vpt sub vpt Square fill Bsquare} bind def
/S6 {BL [] 0 setdash 2 copy exch vpt sub exch vpt sub vpt vpt2 Rec fill Bsquare} bind def
/S7 {BL [] 0 setdash 2 copy exch vpt sub exch vpt sub vpt vpt2 Rec fill
	2 copy vpt Square fill Bsquare} bind def
/S8 {BL [] 0 setdash 2 copy vpt sub vpt Square fill Bsquare} bind def
/S9 {BL [] 0 setdash 2 copy vpt sub vpt vpt2 Rec fill Bsquare} bind def
/S10 {BL [] 0 setdash 2 copy vpt sub vpt Square fill 2 copy exch vpt sub exch vpt Square fill
	Bsquare} bind def
/S11 {BL [] 0 setdash 2 copy vpt sub vpt Square fill 2 copy exch vpt sub exch vpt2 vpt Rec fill
	Bsquare} bind def
/S12 {BL [] 0 setdash 2 copy exch vpt sub exch vpt sub vpt2 vpt Rec fill Bsquare} bind def
/S13 {BL [] 0 setdash 2 copy exch vpt sub exch vpt sub vpt2 vpt Rec fill
	2 copy vpt Square fill Bsquare} bind def
/S14 {BL [] 0 setdash 2 copy exch vpt sub exch vpt sub vpt2 vpt Rec fill
	2 copy exch vpt sub exch vpt Square fill Bsquare} bind def
/S15 {BL [] 0 setdash 2 copy Bsquare fill Bsquare} bind def
/D0 {gsave translate 45 rotate 0 0 S0 stroke grestore} bind def
/D1 {gsave translate 45 rotate 0 0 S1 stroke grestore} bind def
/D2 {gsave translate 45 rotate 0 0 S2 stroke grestore} bind def
/D3 {gsave translate 45 rotate 0 0 S3 stroke grestore} bind def
/D4 {gsave translate 45 rotate 0 0 S4 stroke grestore} bind def
/D5 {gsave translate 45 rotate 0 0 S5 stroke grestore} bind def
/D6 {gsave translate 45 rotate 0 0 S6 stroke grestore} bind def
/D7 {gsave translate 45 rotate 0 0 S7 stroke grestore} bind def
/D8 {gsave translate 45 rotate 0 0 S8 stroke grestore} bind def
/D9 {gsave translate 45 rotate 0 0 S9 stroke grestore} bind def
/D10 {gsave translate 45 rotate 0 0 S10 stroke grestore} bind def
/D11 {gsave translate 45 rotate 0 0 S11 stroke grestore} bind def
/D12 {gsave translate 45 rotate 0 0 S12 stroke grestore} bind def
/D13 {gsave translate 45 rotate 0 0 S13 stroke grestore} bind def
/D14 {gsave translate 45 rotate 0 0 S14 stroke grestore} bind def
/D15 {gsave translate 45 rotate 0 0 S15 stroke grestore} bind def
/DiaE {stroke [] 0 setdash vpt add M
  hpt neg vpt neg V hpt vpt neg V
  hpt vpt V hpt neg vpt V closepath stroke} def
/BoxE {stroke [] 0 setdash exch hpt sub exch vpt add M
  0 vpt2 neg V hpt2 0 V 0 vpt2 V
  hpt2 neg 0 V closepath stroke} def
/TriUE {stroke [] 0 setdash vpt 1.12 mul add M
  hpt neg vpt -1.62 mul V
  hpt 2 mul 0 V
  hpt neg vpt 1.62 mul V closepath stroke} def
/TriDE {stroke [] 0 setdash vpt 1.12 mul sub M
  hpt neg vpt 1.62 mul V
  hpt 2 mul 0 V
  hpt neg vpt -1.62 mul V closepath stroke} def
/PentE {stroke [] 0 setdash gsave
  translate 0 hpt M 4 {72 rotate 0 hpt L} repeat
  closepath stroke grestore} def
/CircE {stroke [] 0 setdash 
  hpt 0 360 arc stroke} def
/Opaque {gsave closepath 1 setgray fill grestore 0 setgray closepath} def
/DiaW {stroke [] 0 setdash vpt add M
  hpt neg vpt neg V hpt vpt neg V
  hpt vpt V hpt neg vpt V Opaque stroke} def
/BoxW {stroke [] 0 setdash exch hpt sub exch vpt add M
  0 vpt2 neg V hpt2 0 V 0 vpt2 V
  hpt2 neg 0 V Opaque stroke} def
/TriUW {stroke [] 0 setdash vpt 1.12 mul add M
  hpt neg vpt -1.62 mul V
  hpt 2 mul 0 V
  hpt neg vpt 1.62 mul V Opaque stroke} def
/TriDW {stroke [] 0 setdash vpt 1.12 mul sub M
  hpt neg vpt 1.62 mul V
  hpt 2 mul 0 V
  hpt neg vpt -1.62 mul V Opaque stroke} def
/PentW {stroke [] 0 setdash gsave
  translate 0 hpt M 4 {72 rotate 0 hpt L} repeat
  Opaque stroke grestore} def
/CircW {stroke [] 0 setdash 
  hpt 0 360 arc Opaque stroke} def
/BoxFill {gsave Rec 1 setgray fill grestore} def
/Density {
  /Fillden exch def
  currentrgbcolor
  /ColB exch def /ColG exch def /ColR exch def
  /ColR ColR Fillden mul Fillden sub 1 add def
  /ColG ColG Fillden mul Fillden sub 1 add def
  /ColB ColB Fillden mul Fillden sub 1 add def
  ColR ColG ColB setrgbcolor} def
/BoxColFill {gsave Rec PolyFill} def
/PolyFill {gsave Density fill grestore grestore} def
/h {rlineto rlineto rlineto gsave closepath fill grestore} bind def
%
% PostScript Level 1 Pattern Fill routine for rectangles
% Usage: x y w h s a XX PatternFill
%	x,y = lower left corner of box to be filled
%	w,h = width and height of box
%	  a = angle in degrees between lines and x-axis
%	 XX = 0/1 for no/yes cross-hatch
%
/PatternFill {gsave /PFa [ 9 2 roll ] def
  PFa 0 get PFa 2 get 2 div add PFa 1 get PFa 3 get 2 div add translate
  PFa 2 get -2 div PFa 3 get -2 div PFa 2 get PFa 3 get Rec
  TransparentPatterns {} {gsave 1 setgray fill grestore} ifelse
  clip
  currentlinewidth 0.5 mul setlinewidth
  /PFs PFa 2 get dup mul PFa 3 get dup mul add sqrt def
  0 0 M PFa 5 get rotate PFs -2 div dup translate
  0 1 PFs PFa 4 get div 1 add floor cvi
	{PFa 4 get mul 0 M 0 PFs V} for
  0 PFa 6 get ne {
	0 1 PFs PFa 4 get div 1 add floor cvi
	{PFa 4 get mul 0 2 1 roll M PFs 0 V} for
 } if
  stroke grestore} def
%
/languagelevel where
 {pop languagelevel} {1} ifelse
dup 2 lt
	{/InterpretLevel1 true def
	 /InterpretLevel3 false def}
	{/InterpretLevel1 Level1 def
	 2 gt
	    {/InterpretLevel3 Level3 def}
	    {/InterpretLevel3 false def}
	 ifelse }
 ifelse
%
% PostScript level 2 pattern fill definitions
%
/Level2PatternFill {
/Tile8x8 {/PaintType 2 /PatternType 1 /TilingType 1 /BBox [0 0 8 8] /XStep 8 /YStep 8}
	bind def
/KeepColor {currentrgbcolor [/Pattern /DeviceRGB] setcolorspace} bind def
<< Tile8x8
 /PaintProc {0.5 setlinewidth pop 0 0 M 8 8 L 0 8 M 8 0 L stroke} 
>> matrix makepattern
/Pat1 exch def
<< Tile8x8
 /PaintProc {0.5 setlinewidth pop 0 0 M 8 8 L 0 8 M 8 0 L stroke
	0 4 M 4 8 L 8 4 L 4 0 L 0 4 L stroke}
>> matrix makepattern
/Pat2 exch def
<< Tile8x8
 /PaintProc {0.5 setlinewidth pop 0 0 M 0 8 L
	8 8 L 8 0 L 0 0 L fill}
>> matrix makepattern
/Pat3 exch def
<< Tile8x8
 /PaintProc {0.5 setlinewidth pop -4 8 M 8 -4 L
	0 12 M 12 0 L stroke}
>> matrix makepattern
/Pat4 exch def
<< Tile8x8
 /PaintProc {0.5 setlinewidth pop -4 0 M 8 12 L
	0 -4 M 12 8 L stroke}
>> matrix makepattern
/Pat5 exch def
<< Tile8x8
 /PaintProc {0.5 setlinewidth pop -2 8 M 4 -4 L
	0 12 M 8 -4 L 4 12 M 10 0 L stroke}
>> matrix makepattern
/Pat6 exch def
<< Tile8x8
 /PaintProc {0.5 setlinewidth pop -2 0 M 4 12 L
	0 -4 M 8 12 L 4 -4 M 10 8 L stroke}
>> matrix makepattern
/Pat7 exch def
<< Tile8x8
 /PaintProc {0.5 setlinewidth pop 8 -2 M -4 4 L
	12 0 M -4 8 L 12 4 M 0 10 L stroke}
>> matrix makepattern
/Pat8 exch def
<< Tile8x8
 /PaintProc {0.5 setlinewidth pop 0 -2 M 12 4 L
	-4 0 M 12 8 L -4 4 M 8 10 L stroke}
>> matrix makepattern
/Pat9 exch def
/Pattern1 {PatternBgnd KeepColor Pat1 setpattern} bind def
/Pattern2 {PatternBgnd KeepColor Pat2 setpattern} bind def
/Pattern3 {PatternBgnd KeepColor Pat3 setpattern} bind def
/Pattern4 {PatternBgnd KeepColor Landscape {Pat5} {Pat4} ifelse setpattern} bind def
/Pattern5 {PatternBgnd KeepColor Landscape {Pat4} {Pat5} ifelse setpattern} bind def
/Pattern6 {PatternBgnd KeepColor Landscape {Pat9} {Pat6} ifelse setpattern} bind def
/Pattern7 {PatternBgnd KeepColor Landscape {Pat8} {Pat7} ifelse setpattern} bind def
} def
%
%
%End of PostScript Level 2 code
%
/PatternBgnd {
  TransparentPatterns {} {gsave 1 setgray fill grestore} ifelse
} def
%
% Substitute for Level 2 pattern fill codes with
% grayscale if Level 2 support is not selected.
%
/Level1PatternFill {
/Pattern1 {0.250 Density} bind def
/Pattern2 {0.500 Density} bind def
/Pattern3 {0.750 Density} bind def
/Pattern4 {0.125 Density} bind def
/Pattern5 {0.375 Density} bind def
/Pattern6 {0.625 Density} bind def
/Pattern7 {0.875 Density} bind def
} def
%
% Now test for support of Level 2 code
%
Level1 {Level1PatternFill} {Level2PatternFill} ifelse
%
/Symbol-Oblique /Symbol findfont [1 0 .167 1 0 0] makefont
dup length dict begin {1 index /FID eq {pop pop} {def} ifelse} forall
currentdict end definefont pop
%
Level1 SuppressPDFMark or 
{} {
/SDict 10 dict def
systemdict /pdfmark known not {
  userdict /pdfmark systemdict /cleartomark get put
} if
SDict begin [
  /Title (plot_EeffK1b_SU8.tex)
  /Subject (gnuplot plot)
  /Creator (gnuplot 5.0 patchlevel 3)
  /Author (mteper)
%  /Producer (gnuplot)
%  /Keywords ()
  /CreationDate (Mon Nov  8 09:38:58 2021)
  /DOCINFO pdfmark
end
} ifelse
%
% Support for boxed text - Ethan A Merritt May 2005
%
/InitTextBox { userdict /TBy2 3 -1 roll put userdict /TBx2 3 -1 roll put
           userdict /TBy1 3 -1 roll put userdict /TBx1 3 -1 roll put
	   /Boxing true def } def
/ExtendTextBox { Boxing
    { gsave dup false charpath pathbbox
      dup TBy2 gt {userdict /TBy2 3 -1 roll put} {pop} ifelse
      dup TBx2 gt {userdict /TBx2 3 -1 roll put} {pop} ifelse
      dup TBy1 lt {userdict /TBy1 3 -1 roll put} {pop} ifelse
      dup TBx1 lt {userdict /TBx1 3 -1 roll put} {pop} ifelse
      grestore } if } def
/PopTextBox { newpath TBx1 TBxmargin sub TBy1 TBymargin sub M
               TBx1 TBxmargin sub TBy2 TBymargin add L
	       TBx2 TBxmargin add TBy2 TBymargin add L
	       TBx2 TBxmargin add TBy1 TBymargin sub L closepath } def
/DrawTextBox { PopTextBox stroke /Boxing false def} def
/FillTextBox { gsave PopTextBox 1 1 1 setrgbcolor fill grestore /Boxing false def} def
0 0 0 0 InitTextBox
/TBxmargin 20 def
/TBymargin 20 def
/Boxing false def
/textshow { ExtendTextBox Gshow } def
%
% redundant definitions for compatibility with prologue.ps older than 5.0.2
/LTB {BL [] LCb DL} def
/LTb {PL [] LCb DL} def
end
%%EndProlog
%%Page: 1 1
gnudict begin
gsave
doclip
0 0 translate
0.050 0.050 scale
0 setgray
newpath
BackgroundColor 0 lt 3 1 roll 0 lt exch 0 lt or or not {BackgroundColor C 1.000 0 0 7200.00 7560.00 BoxColFill} if
1.000 UL
LTb
LCb setrgbcolor
1580 1753 M
63 0 V
5196 0 R
-63 0 V
stroke
LTb
LCb setrgbcolor
1580 2866 M
63 0 V
5196 0 R
-63 0 V
stroke
LTb
LCb setrgbcolor
1580 3980 M
63 0 V
5196 0 R
-63 0 V
stroke
LTb
LCb setrgbcolor
1580 5093 M
63 0 V
5196 0 R
-63 0 V
stroke
LTb
LCb setrgbcolor
1580 6206 M
63 0 V
5196 0 R
-63 0 V
stroke
LTb
LCb setrgbcolor
1580 640 M
0 63 V
0 6616 R
0 -63 V
stroke
LTb
LCb setrgbcolor
1985 640 M
0 63 V
0 6616 R
0 -63 V
stroke
LTb
LCb setrgbcolor
2389 640 M
0 63 V
0 6616 R
0 -63 V
stroke
LTb
LCb setrgbcolor
2794 640 M
0 63 V
0 6616 R
0 -63 V
stroke
LTb
LCb setrgbcolor
3198 640 M
0 63 V
0 6616 R
0 -63 V
stroke
LTb
LCb setrgbcolor
3603 640 M
0 63 V
0 6616 R
0 -63 V
stroke
LTb
LCb setrgbcolor
4007 640 M
0 63 V
0 6616 R
0 -63 V
stroke
LTb
LCb setrgbcolor
4412 640 M
0 63 V
0 6616 R
0 -63 V
stroke
LTb
LCb setrgbcolor
4816 640 M
0 63 V
0 6616 R
0 -63 V
stroke
LTb
LCb setrgbcolor
5221 640 M
0 63 V
0 6616 R
0 -63 V
stroke
LTb
LCb setrgbcolor
5625 640 M
0 63 V
0 6616 R
0 -63 V
stroke
LTb
LCb setrgbcolor
6030 640 M
0 63 V
0 6616 R
0 -63 V
stroke
LTb
LCb setrgbcolor
6434 640 M
0 63 V
0 6616 R
0 -63 V
stroke
LTb
LCb setrgbcolor
6839 640 M
0 63 V
0 6616 R
0 -63 V
stroke
LTb
LCb setrgbcolor
1.000 UL
LTb
LCb setrgbcolor
1580 7319 N
0 -6679 V
5259 0 V
0 6679 V
-5259 0 V
Z stroke
1.000 UP
1.000 UL
LTb
LCb setrgbcolor
LCb setrgbcolor
LTb
LCb setrgbcolor
LTb
1.500 UP
1.000 UL
LTb
0.58 0.00 0.83 C 1782 5972 M
0 178 V
2187 3305 M
0 231 V
404 -763 R
0 271 V
405 -478 R
0 369 V
404 -403 R
0 481 V
405 -545 R
0 637 V
405 -949 R
0 882 V
4614 2025 M
0 1046 V
5019 1902 M
0 1447 V
5423 1566 M
0 1906 V
5828 1239 M
0 2422 V
6232 640 M
0 2275 V
6637 872 M
0 3361 V
1782 6061 CircleF
2187 3421 CircleF
2591 2909 CircleF
2996 2751 CircleF
3400 2773 CircleF
3805 2786 CircleF
4210 2597 CircleF
4614 2548 CircleF
5019 2626 CircleF
5423 2519 CircleF
5828 2450 CircleF
6232 1453 CircleF
6637 2552 CircleF
1.500 UL
LTb
0.58 0.00 0.83 C 1580 2710 M
53 0 V
53 0 V
53 0 V
53 0 V
54 0 V
53 0 V
53 0 V
53 0 V
53 0 V
53 0 V
53 0 V
53 0 V
54 0 V
53 0 V
53 0 V
53 0 V
53 0 V
53 0 V
53 0 V
53 0 V
54 0 V
53 0 V
53 0 V
53 0 V
53 0 V
53 0 V
53 0 V
53 0 V
54 0 V
53 0 V
53 0 V
53 0 V
53 0 V
53 0 V
53 0 V
53 0 V
53 0 V
54 0 V
53 0 V
53 0 V
53 0 V
53 0 V
53 0 V
53 0 V
53 0 V
54 0 V
53 0 V
53 0 V
53 0 V
53 0 V
53 0 V
53 0 V
53 0 V
54 0 V
53 0 V
53 0 V
53 0 V
53 0 V
53 0 V
53 0 V
53 0 V
54 0 V
53 0 V
53 0 V
53 0 V
53 0 V
53 0 V
53 0 V
53 0 V
53 0 V
54 0 V
53 0 V
53 0 V
53 0 V
53 0 V
53 0 V
53 0 V
53 0 V
54 0 V
53 0 V
53 0 V
53 0 V
53 0 V
53 0 V
53 0 V
53 0 V
54 0 V
53 0 V
53 0 V
53 0 V
53 0 V
53 0 V
53 0 V
53 0 V
54 0 V
53 0 V
53 0 V
53 0 V
53 0 V
stroke
2.500 UL
LTb
LT1
0.34 0.71 0.91 C 1580 3022 M
53 0 V
53 0 V
53 0 V
53 0 V
54 0 V
53 0 V
53 0 V
53 0 V
53 0 V
53 0 V
53 0 V
53 0 V
54 0 V
53 0 V
53 0 V
53 0 V
53 0 V
53 0 V
53 0 V
53 0 V
54 0 V
53 0 V
53 0 V
53 0 V
53 0 V
53 0 V
53 0 V
53 0 V
54 0 V
53 0 V
53 0 V
53 0 V
53 0 V
53 0 V
53 0 V
53 0 V
53 0 V
54 0 V
53 0 V
53 0 V
53 0 V
53 0 V
53 0 V
53 0 V
53 0 V
54 0 V
53 0 V
53 0 V
53 0 V
53 0 V
53 0 V
53 0 V
53 0 V
54 0 V
53 0 V
53 0 V
53 0 V
53 0 V
53 0 V
53 0 V
53 0 V
54 0 V
53 0 V
53 0 V
53 0 V
53 0 V
53 0 V
53 0 V
53 0 V
53 0 V
54 0 V
53 0 V
53 0 V
53 0 V
53 0 V
53 0 V
53 0 V
53 0 V
54 0 V
53 0 V
53 0 V
53 0 V
53 0 V
53 0 V
53 0 V
53 0 V
54 0 V
53 0 V
53 0 V
53 0 V
53 0 V
53 0 V
53 0 V
53 0 V
54 0 V
53 0 V
53 0 V
53 0 V
53 0 V
stroke
LTb
LT1
0.90 0.62 0.00 C 1580 2399 M
53 0 V
53 0 V
53 0 V
53 0 V
54 0 V
53 0 V
53 0 V
53 0 V
53 0 V
53 0 V
53 0 V
53 0 V
54 0 V
53 0 V
53 0 V
53 0 V
53 0 V
53 0 V
53 0 V
53 0 V
54 0 V
53 0 V
53 0 V
53 0 V
53 0 V
53 0 V
53 0 V
53 0 V
54 0 V
53 0 V
53 0 V
53 0 V
53 0 V
53 0 V
53 0 V
53 0 V
53 0 V
54 0 V
53 0 V
53 0 V
53 0 V
53 0 V
53 0 V
53 0 V
53 0 V
54 0 V
53 0 V
53 0 V
53 0 V
53 0 V
53 0 V
53 0 V
53 0 V
54 0 V
53 0 V
53 0 V
53 0 V
53 0 V
53 0 V
53 0 V
53 0 V
54 0 V
53 0 V
53 0 V
53 0 V
53 0 V
53 0 V
53 0 V
53 0 V
53 0 V
54 0 V
53 0 V
53 0 V
53 0 V
53 0 V
53 0 V
53 0 V
53 0 V
54 0 V
53 0 V
53 0 V
53 0 V
53 0 V
53 0 V
53 0 V
53 0 V
54 0 V
53 0 V
53 0 V
53 0 V
53 0 V
53 0 V
53 0 V
53 0 V
54 0 V
53 0 V
53 0 V
53 0 V
53 0 V
stroke
2.000 UL
LTb
LCb setrgbcolor
1.000 UL
LTb
LCb setrgbcolor
1580 7319 N
0 -6679 V
5259 0 V
0 6679 V
-5259 0 V
Z stroke
1.000 UP
1.000 UL
LTb
LCb setrgbcolor
stroke
grestore
end
showpage
  }}%
  \put(4209,140){\makebox(0,0){\large{$n_t$}}}%
  \put(160,4979){\makebox(0,0){\Large{$aE_{eff}^{k=1}$}}}%
  \put(6839,440){\makebox(0,0){\strut{}\ {$13$}}}%
  \put(6434,440){\makebox(0,0){\strut{}\ {$12$}}}%
  \put(6030,440){\makebox(0,0){\strut{}\ {$11$}}}%
  \put(5625,440){\makebox(0,0){\strut{}\ {$10$}}}%
  \put(5221,440){\makebox(0,0){\strut{}\ {$9$}}}%
  \put(4816,440){\makebox(0,0){\strut{}\ {$8$}}}%
  \put(4412,440){\makebox(0,0){\strut{}\ {$7$}}}%
  \put(4007,440){\makebox(0,0){\strut{}\ {$6$}}}%
  \put(3603,440){\makebox(0,0){\strut{}\ {$5$}}}%
  \put(3198,440){\makebox(0,0){\strut{}\ {$4$}}}%
  \put(2794,440){\makebox(0,0){\strut{}\ {$3$}}}%
  \put(2389,440){\makebox(0,0){\strut{}\ {$2$}}}%
  \put(1985,440){\makebox(0,0){\strut{}\ {$1$}}}%
  \put(1580,440){\makebox(0,0){\strut{}\ {$0$}}}%
  \put(1460,6206){\makebox(0,0)[r]{\strut{}\ \ {$0.31$}}}%
  \put(1460,5093){\makebox(0,0)[r]{\strut{}\ \ {$0.305$}}}%
  \put(1460,3980){\makebox(0,0)[r]{\strut{}\ \ {$0.3$}}}%
  \put(1460,2866){\makebox(0,0)[r]{\strut{}\ \ {$0.295$}}}%
  \put(1460,1753){\makebox(0,0)[r]{\strut{}\ \ {$0.29$}}}%
\end{picture}%
\endgroup
\endinput

\end	{center}
\caption{
  Effective energy of the ground state of a fundamental $k=1$ flux tube winding
  around a spatial torus, as in Fig.\ref{fig_EeffK1_SU8}, for $\beta=47.75$ in $SU(8)$,
  with a rescaling sufficient to expose the errors. The solid line is the best
  estimate from a fit to the correlation function, and the dashed lines show
  the $\pm 1$ standard deviation fits.}
\label{fig_EeffK1b_SU8}
\end{figure}


%\begin{figure}[htb]
%\begin	{center}
%\leavevmode
%\input	{plot_EeffK1sm_SU3.tex}
%\end	{center}
%\caption{Effective energies of the ground state of a flux tube winding around a spatial torus, extracted from
%  the best correlator $C(t)$ between $t=an_t$ and $t=a(n_t+1)$. For $SU(3)$ and at
%  $\beta=5.6924, 5.80, 5.8941, 5.99, 6.0625, 6,235, 6.338, 6.50$,  6.60$, 6.70$
%  in descending order, on the lattices
%  listed in Table~\ref{table_Ksmall_SU3}.
%  Lines are our estimates of the $t\to\infty$ asymptotic energies.}
%\label{fig_EeffK1sm_SU3}
%\end{figure}






%\begin{figure}[htb]
%\begin	{center}
%\leavevmode
%\input	{plot_EeffK2_SU4SU8.tex}
%\end	{center}
%\caption{Effective energies of the ground state of a $k=2$ flux tube winding around a spatial torus, 
%  extracted from the best correlator $C(t)$ between $t=an_t$ and $t=a(n_t+1)$. For $SU(8)$ at
%  $\beta=44.85, 46.10, 47.75$ (filled points), in descending order, and similarly for $SU(4)$
%  at $\beta=10.85, 11.20, 11.60$ (open points). Lines are our estimates of
%  the $t\to\infty$ asymptotic energies.}
%\label{fig_EeffK2_SU4SU8}
%\end{figure}






\begin{figure}[htb]
\begin	{center}
\leavevmode
% GNUPLOT: LaTeX picture with Postscript
\begingroup%
\makeatletter%
\newcommand{\GNUPLOTspecial}{%
  \@sanitize\catcode`\%=14\relax\special}%
\setlength{\unitlength}{0.0500bp}%
\begin{picture}(7200,7560)(0,0)%
  {\GNUPLOTspecial{"
%!PS-Adobe-2.0 EPSF-2.0
%%Title: plot_k2k1_cont.tex
%%Creator: gnuplot 5.0 patchlevel 3
%%CreationDate: Sun Dec 27 21:21:32 2020
%%DocumentFonts: 
%%BoundingBox: 0 0 360 378
%%EndComments
%%BeginProlog
/gnudict 256 dict def
gnudict begin
%
% The following true/false flags may be edited by hand if desired.
% The unit line width and grayscale image gamma correction may also be changed.
%
/Color true def
/Blacktext true def
/Solid false def
/Dashlength 1 def
/Landscape false def
/Level1 false def
/Level3 false def
/Rounded false def
/ClipToBoundingBox false def
/SuppressPDFMark false def
/TransparentPatterns false def
/gnulinewidth 5.000 def
/userlinewidth gnulinewidth def
/Gamma 1.0 def
/BackgroundColor {-1.000 -1.000 -1.000} def
%
/vshift -66 def
/dl1 {
  10.0 Dashlength userlinewidth gnulinewidth div mul mul mul
  Rounded { currentlinewidth 0.75 mul sub dup 0 le { pop 0.01 } if } if
} def
/dl2 {
  10.0 Dashlength userlinewidth gnulinewidth div mul mul mul
  Rounded { currentlinewidth 0.75 mul add } if
} def
/hpt_ 31.5 def
/vpt_ 31.5 def
/hpt hpt_ def
/vpt vpt_ def
/doclip {
  ClipToBoundingBox {
    newpath 0 0 moveto 360 0 lineto 360 378 lineto 0 378 lineto closepath
    clip
  } if
} def
%
% Gnuplot Prolog Version 5.1 (Oct 2015)
%
%/SuppressPDFMark true def
%
/M {moveto} bind def
/L {lineto} bind def
/R {rmoveto} bind def
/V {rlineto} bind def
/N {newpath moveto} bind def
/Z {closepath} bind def
/C {setrgbcolor} bind def
/f {rlineto fill} bind def
/g {setgray} bind def
/Gshow {show} def   % May be redefined later in the file to support UTF-8
/vpt2 vpt 2 mul def
/hpt2 hpt 2 mul def
/Lshow {currentpoint stroke M 0 vshift R 
	Blacktext {gsave 0 setgray textshow grestore} {textshow} ifelse} def
/Rshow {currentpoint stroke M dup stringwidth pop neg vshift R
	Blacktext {gsave 0 setgray textshow grestore} {textshow} ifelse} def
/Cshow {currentpoint stroke M dup stringwidth pop -2 div vshift R 
	Blacktext {gsave 0 setgray textshow grestore} {textshow} ifelse} def
/UP {dup vpt_ mul /vpt exch def hpt_ mul /hpt exch def
  /hpt2 hpt 2 mul def /vpt2 vpt 2 mul def} def
/DL {Color {setrgbcolor Solid {pop []} if 0 setdash}
 {pop pop pop 0 setgray Solid {pop []} if 0 setdash} ifelse} def
/BL {stroke userlinewidth 2 mul setlinewidth
	Rounded {1 setlinejoin 1 setlinecap} if} def
/AL {stroke userlinewidth 2 div setlinewidth
	Rounded {1 setlinejoin 1 setlinecap} if} def
/UL {dup gnulinewidth mul /userlinewidth exch def
	dup 1 lt {pop 1} if 10 mul /udl exch def} def
/PL {stroke userlinewidth setlinewidth
	Rounded {1 setlinejoin 1 setlinecap} if} def
3.8 setmiterlimit
% Classic Line colors (version 5.0)
/LCw {1 1 1} def
/LCb {0 0 0} def
/LCa {0 0 0} def
/LC0 {1 0 0} def
/LC1 {0 1 0} def
/LC2 {0 0 1} def
/LC3 {1 0 1} def
/LC4 {0 1 1} def
/LC5 {1 1 0} def
/LC6 {0 0 0} def
/LC7 {1 0.3 0} def
/LC8 {0.5 0.5 0.5} def
% Default dash patterns (version 5.0)
/LTB {BL [] LCb DL} def
/LTw {PL [] 1 setgray} def
/LTb {PL [] LCb DL} def
/LTa {AL [1 udl mul 2 udl mul] 0 setdash LCa setrgbcolor} def
/LT0 {PL [] LC0 DL} def
/LT1 {PL [2 dl1 3 dl2] LC1 DL} def
/LT2 {PL [1 dl1 1.5 dl2] LC2 DL} def
/LT3 {PL [6 dl1 2 dl2 1 dl1 2 dl2] LC3 DL} def
/LT4 {PL [1 dl1 2 dl2 6 dl1 2 dl2 1 dl1 2 dl2] LC4 DL} def
/LT5 {PL [4 dl1 2 dl2] LC5 DL} def
/LT6 {PL [1.5 dl1 1.5 dl2 1.5 dl1 1.5 dl2 1.5 dl1 6 dl2] LC6 DL} def
/LT7 {PL [3 dl1 3 dl2 1 dl1 3 dl2] LC7 DL} def
/LT8 {PL [2 dl1 2 dl2 2 dl1 6 dl2] LC8 DL} def
/SL {[] 0 setdash} def
/Pnt {stroke [] 0 setdash gsave 1 setlinecap M 0 0 V stroke grestore} def
/Dia {stroke [] 0 setdash 2 copy vpt add M
  hpt neg vpt neg V hpt vpt neg V
  hpt vpt V hpt neg vpt V closepath stroke
  Pnt} def
/Pls {stroke [] 0 setdash vpt sub M 0 vpt2 V
  currentpoint stroke M
  hpt neg vpt neg R hpt2 0 V stroke
 } def
/Box {stroke [] 0 setdash 2 copy exch hpt sub exch vpt add M
  0 vpt2 neg V hpt2 0 V 0 vpt2 V
  hpt2 neg 0 V closepath stroke
  Pnt} def
/Crs {stroke [] 0 setdash exch hpt sub exch vpt add M
  hpt2 vpt2 neg V currentpoint stroke M
  hpt2 neg 0 R hpt2 vpt2 V stroke} def
/TriU {stroke [] 0 setdash 2 copy vpt 1.12 mul add M
  hpt neg vpt -1.62 mul V
  hpt 2 mul 0 V
  hpt neg vpt 1.62 mul V closepath stroke
  Pnt} def
/Star {2 copy Pls Crs} def
/BoxF {stroke [] 0 setdash exch hpt sub exch vpt add M
  0 vpt2 neg V hpt2 0 V 0 vpt2 V
  hpt2 neg 0 V closepath fill} def
/TriUF {stroke [] 0 setdash vpt 1.12 mul add M
  hpt neg vpt -1.62 mul V
  hpt 2 mul 0 V
  hpt neg vpt 1.62 mul V closepath fill} def
/TriD {stroke [] 0 setdash 2 copy vpt 1.12 mul sub M
  hpt neg vpt 1.62 mul V
  hpt 2 mul 0 V
  hpt neg vpt -1.62 mul V closepath stroke
  Pnt} def
/TriDF {stroke [] 0 setdash vpt 1.12 mul sub M
  hpt neg vpt 1.62 mul V
  hpt 2 mul 0 V
  hpt neg vpt -1.62 mul V closepath fill} def
/DiaF {stroke [] 0 setdash vpt add M
  hpt neg vpt neg V hpt vpt neg V
  hpt vpt V hpt neg vpt V closepath fill} def
/Pent {stroke [] 0 setdash 2 copy gsave
  translate 0 hpt M 4 {72 rotate 0 hpt L} repeat
  closepath stroke grestore Pnt} def
/PentF {stroke [] 0 setdash gsave
  translate 0 hpt M 4 {72 rotate 0 hpt L} repeat
  closepath fill grestore} def
/Circle {stroke [] 0 setdash 2 copy
  hpt 0 360 arc stroke Pnt} def
/CircleF {stroke [] 0 setdash hpt 0 360 arc fill} def
/C0 {BL [] 0 setdash 2 copy moveto vpt 90 450 arc} bind def
/C1 {BL [] 0 setdash 2 copy moveto
	2 copy vpt 0 90 arc closepath fill
	vpt 0 360 arc closepath} bind def
/C2 {BL [] 0 setdash 2 copy moveto
	2 copy vpt 90 180 arc closepath fill
	vpt 0 360 arc closepath} bind def
/C3 {BL [] 0 setdash 2 copy moveto
	2 copy vpt 0 180 arc closepath fill
	vpt 0 360 arc closepath} bind def
/C4 {BL [] 0 setdash 2 copy moveto
	2 copy vpt 180 270 arc closepath fill
	vpt 0 360 arc closepath} bind def
/C5 {BL [] 0 setdash 2 copy moveto
	2 copy vpt 0 90 arc
	2 copy moveto
	2 copy vpt 180 270 arc closepath fill
	vpt 0 360 arc} bind def
/C6 {BL [] 0 setdash 2 copy moveto
	2 copy vpt 90 270 arc closepath fill
	vpt 0 360 arc closepath} bind def
/C7 {BL [] 0 setdash 2 copy moveto
	2 copy vpt 0 270 arc closepath fill
	vpt 0 360 arc closepath} bind def
/C8 {BL [] 0 setdash 2 copy moveto
	2 copy vpt 270 360 arc closepath fill
	vpt 0 360 arc closepath} bind def
/C9 {BL [] 0 setdash 2 copy moveto
	2 copy vpt 270 450 arc closepath fill
	vpt 0 360 arc closepath} bind def
/C10 {BL [] 0 setdash 2 copy 2 copy moveto vpt 270 360 arc closepath fill
	2 copy moveto
	2 copy vpt 90 180 arc closepath fill
	vpt 0 360 arc closepath} bind def
/C11 {BL [] 0 setdash 2 copy moveto
	2 copy vpt 0 180 arc closepath fill
	2 copy moveto
	2 copy vpt 270 360 arc closepath fill
	vpt 0 360 arc closepath} bind def
/C12 {BL [] 0 setdash 2 copy moveto
	2 copy vpt 180 360 arc closepath fill
	vpt 0 360 arc closepath} bind def
/C13 {BL [] 0 setdash 2 copy moveto
	2 copy vpt 0 90 arc closepath fill
	2 copy moveto
	2 copy vpt 180 360 arc closepath fill
	vpt 0 360 arc closepath} bind def
/C14 {BL [] 0 setdash 2 copy moveto
	2 copy vpt 90 360 arc closepath fill
	vpt 0 360 arc} bind def
/C15 {BL [] 0 setdash 2 copy vpt 0 360 arc closepath fill
	vpt 0 360 arc closepath} bind def
/Rec {newpath 4 2 roll moveto 1 index 0 rlineto 0 exch rlineto
	neg 0 rlineto closepath} bind def
/Square {dup Rec} bind def
/Bsquare {vpt sub exch vpt sub exch vpt2 Square} bind def
/S0 {BL [] 0 setdash 2 copy moveto 0 vpt rlineto BL Bsquare} bind def
/S1 {BL [] 0 setdash 2 copy vpt Square fill Bsquare} bind def
/S2 {BL [] 0 setdash 2 copy exch vpt sub exch vpt Square fill Bsquare} bind def
/S3 {BL [] 0 setdash 2 copy exch vpt sub exch vpt2 vpt Rec fill Bsquare} bind def
/S4 {BL [] 0 setdash 2 copy exch vpt sub exch vpt sub vpt Square fill Bsquare} bind def
/S5 {BL [] 0 setdash 2 copy 2 copy vpt Square fill
	exch vpt sub exch vpt sub vpt Square fill Bsquare} bind def
/S6 {BL [] 0 setdash 2 copy exch vpt sub exch vpt sub vpt vpt2 Rec fill Bsquare} bind def
/S7 {BL [] 0 setdash 2 copy exch vpt sub exch vpt sub vpt vpt2 Rec fill
	2 copy vpt Square fill Bsquare} bind def
/S8 {BL [] 0 setdash 2 copy vpt sub vpt Square fill Bsquare} bind def
/S9 {BL [] 0 setdash 2 copy vpt sub vpt vpt2 Rec fill Bsquare} bind def
/S10 {BL [] 0 setdash 2 copy vpt sub vpt Square fill 2 copy exch vpt sub exch vpt Square fill
	Bsquare} bind def
/S11 {BL [] 0 setdash 2 copy vpt sub vpt Square fill 2 copy exch vpt sub exch vpt2 vpt Rec fill
	Bsquare} bind def
/S12 {BL [] 0 setdash 2 copy exch vpt sub exch vpt sub vpt2 vpt Rec fill Bsquare} bind def
/S13 {BL [] 0 setdash 2 copy exch vpt sub exch vpt sub vpt2 vpt Rec fill
	2 copy vpt Square fill Bsquare} bind def
/S14 {BL [] 0 setdash 2 copy exch vpt sub exch vpt sub vpt2 vpt Rec fill
	2 copy exch vpt sub exch vpt Square fill Bsquare} bind def
/S15 {BL [] 0 setdash 2 copy Bsquare fill Bsquare} bind def
/D0 {gsave translate 45 rotate 0 0 S0 stroke grestore} bind def
/D1 {gsave translate 45 rotate 0 0 S1 stroke grestore} bind def
/D2 {gsave translate 45 rotate 0 0 S2 stroke grestore} bind def
/D3 {gsave translate 45 rotate 0 0 S3 stroke grestore} bind def
/D4 {gsave translate 45 rotate 0 0 S4 stroke grestore} bind def
/D5 {gsave translate 45 rotate 0 0 S5 stroke grestore} bind def
/D6 {gsave translate 45 rotate 0 0 S6 stroke grestore} bind def
/D7 {gsave translate 45 rotate 0 0 S7 stroke grestore} bind def
/D8 {gsave translate 45 rotate 0 0 S8 stroke grestore} bind def
/D9 {gsave translate 45 rotate 0 0 S9 stroke grestore} bind def
/D10 {gsave translate 45 rotate 0 0 S10 stroke grestore} bind def
/D11 {gsave translate 45 rotate 0 0 S11 stroke grestore} bind def
/D12 {gsave translate 45 rotate 0 0 S12 stroke grestore} bind def
/D13 {gsave translate 45 rotate 0 0 S13 stroke grestore} bind def
/D14 {gsave translate 45 rotate 0 0 S14 stroke grestore} bind def
/D15 {gsave translate 45 rotate 0 0 S15 stroke grestore} bind def
/DiaE {stroke [] 0 setdash vpt add M
  hpt neg vpt neg V hpt vpt neg V
  hpt vpt V hpt neg vpt V closepath stroke} def
/BoxE {stroke [] 0 setdash exch hpt sub exch vpt add M
  0 vpt2 neg V hpt2 0 V 0 vpt2 V
  hpt2 neg 0 V closepath stroke} def
/TriUE {stroke [] 0 setdash vpt 1.12 mul add M
  hpt neg vpt -1.62 mul V
  hpt 2 mul 0 V
  hpt neg vpt 1.62 mul V closepath stroke} def
/TriDE {stroke [] 0 setdash vpt 1.12 mul sub M
  hpt neg vpt 1.62 mul V
  hpt 2 mul 0 V
  hpt neg vpt -1.62 mul V closepath stroke} def
/PentE {stroke [] 0 setdash gsave
  translate 0 hpt M 4 {72 rotate 0 hpt L} repeat
  closepath stroke grestore} def
/CircE {stroke [] 0 setdash 
  hpt 0 360 arc stroke} def
/Opaque {gsave closepath 1 setgray fill grestore 0 setgray closepath} def
/DiaW {stroke [] 0 setdash vpt add M
  hpt neg vpt neg V hpt vpt neg V
  hpt vpt V hpt neg vpt V Opaque stroke} def
/BoxW {stroke [] 0 setdash exch hpt sub exch vpt add M
  0 vpt2 neg V hpt2 0 V 0 vpt2 V
  hpt2 neg 0 V Opaque stroke} def
/TriUW {stroke [] 0 setdash vpt 1.12 mul add M
  hpt neg vpt -1.62 mul V
  hpt 2 mul 0 V
  hpt neg vpt 1.62 mul V Opaque stroke} def
/TriDW {stroke [] 0 setdash vpt 1.12 mul sub M
  hpt neg vpt 1.62 mul V
  hpt 2 mul 0 V
  hpt neg vpt -1.62 mul V Opaque stroke} def
/PentW {stroke [] 0 setdash gsave
  translate 0 hpt M 4 {72 rotate 0 hpt L} repeat
  Opaque stroke grestore} def
/CircW {stroke [] 0 setdash 
  hpt 0 360 arc Opaque stroke} def
/BoxFill {gsave Rec 1 setgray fill grestore} def
/Density {
  /Fillden exch def
  currentrgbcolor
  /ColB exch def /ColG exch def /ColR exch def
  /ColR ColR Fillden mul Fillden sub 1 add def
  /ColG ColG Fillden mul Fillden sub 1 add def
  /ColB ColB Fillden mul Fillden sub 1 add def
  ColR ColG ColB setrgbcolor} def
/BoxColFill {gsave Rec PolyFill} def
/PolyFill {gsave Density fill grestore grestore} def
/h {rlineto rlineto rlineto gsave closepath fill grestore} bind def
%
% PostScript Level 1 Pattern Fill routine for rectangles
% Usage: x y w h s a XX PatternFill
%	x,y = lower left corner of box to be filled
%	w,h = width and height of box
%	  a = angle in degrees between lines and x-axis
%	 XX = 0/1 for no/yes cross-hatch
%
/PatternFill {gsave /PFa [ 9 2 roll ] def
  PFa 0 get PFa 2 get 2 div add PFa 1 get PFa 3 get 2 div add translate
  PFa 2 get -2 div PFa 3 get -2 div PFa 2 get PFa 3 get Rec
  TransparentPatterns {} {gsave 1 setgray fill grestore} ifelse
  clip
  currentlinewidth 0.5 mul setlinewidth
  /PFs PFa 2 get dup mul PFa 3 get dup mul add sqrt def
  0 0 M PFa 5 get rotate PFs -2 div dup translate
  0 1 PFs PFa 4 get div 1 add floor cvi
	{PFa 4 get mul 0 M 0 PFs V} for
  0 PFa 6 get ne {
	0 1 PFs PFa 4 get div 1 add floor cvi
	{PFa 4 get mul 0 2 1 roll M PFs 0 V} for
 } if
  stroke grestore} def
%
/languagelevel where
 {pop languagelevel} {1} ifelse
dup 2 lt
	{/InterpretLevel1 true def
	 /InterpretLevel3 false def}
	{/InterpretLevel1 Level1 def
	 2 gt
	    {/InterpretLevel3 Level3 def}
	    {/InterpretLevel3 false def}
	 ifelse }
 ifelse
%
% PostScript level 2 pattern fill definitions
%
/Level2PatternFill {
/Tile8x8 {/PaintType 2 /PatternType 1 /TilingType 1 /BBox [0 0 8 8] /XStep 8 /YStep 8}
	bind def
/KeepColor {currentrgbcolor [/Pattern /DeviceRGB] setcolorspace} bind def
<< Tile8x8
 /PaintProc {0.5 setlinewidth pop 0 0 M 8 8 L 0 8 M 8 0 L stroke} 
>> matrix makepattern
/Pat1 exch def
<< Tile8x8
 /PaintProc {0.5 setlinewidth pop 0 0 M 8 8 L 0 8 M 8 0 L stroke
	0 4 M 4 8 L 8 4 L 4 0 L 0 4 L stroke}
>> matrix makepattern
/Pat2 exch def
<< Tile8x8
 /PaintProc {0.5 setlinewidth pop 0 0 M 0 8 L
	8 8 L 8 0 L 0 0 L fill}
>> matrix makepattern
/Pat3 exch def
<< Tile8x8
 /PaintProc {0.5 setlinewidth pop -4 8 M 8 -4 L
	0 12 M 12 0 L stroke}
>> matrix makepattern
/Pat4 exch def
<< Tile8x8
 /PaintProc {0.5 setlinewidth pop -4 0 M 8 12 L
	0 -4 M 12 8 L stroke}
>> matrix makepattern
/Pat5 exch def
<< Tile8x8
 /PaintProc {0.5 setlinewidth pop -2 8 M 4 -4 L
	0 12 M 8 -4 L 4 12 M 10 0 L stroke}
>> matrix makepattern
/Pat6 exch def
<< Tile8x8
 /PaintProc {0.5 setlinewidth pop -2 0 M 4 12 L
	0 -4 M 8 12 L 4 -4 M 10 8 L stroke}
>> matrix makepattern
/Pat7 exch def
<< Tile8x8
 /PaintProc {0.5 setlinewidth pop 8 -2 M -4 4 L
	12 0 M -4 8 L 12 4 M 0 10 L stroke}
>> matrix makepattern
/Pat8 exch def
<< Tile8x8
 /PaintProc {0.5 setlinewidth pop 0 -2 M 12 4 L
	-4 0 M 12 8 L -4 4 M 8 10 L stroke}
>> matrix makepattern
/Pat9 exch def
/Pattern1 {PatternBgnd KeepColor Pat1 setpattern} bind def
/Pattern2 {PatternBgnd KeepColor Pat2 setpattern} bind def
/Pattern3 {PatternBgnd KeepColor Pat3 setpattern} bind def
/Pattern4 {PatternBgnd KeepColor Landscape {Pat5} {Pat4} ifelse setpattern} bind def
/Pattern5 {PatternBgnd KeepColor Landscape {Pat4} {Pat5} ifelse setpattern} bind def
/Pattern6 {PatternBgnd KeepColor Landscape {Pat9} {Pat6} ifelse setpattern} bind def
/Pattern7 {PatternBgnd KeepColor Landscape {Pat8} {Pat7} ifelse setpattern} bind def
} def
%
%
%End of PostScript Level 2 code
%
/PatternBgnd {
  TransparentPatterns {} {gsave 1 setgray fill grestore} ifelse
} def
%
% Substitute for Level 2 pattern fill codes with
% grayscale if Level 2 support is not selected.
%
/Level1PatternFill {
/Pattern1 {0.250 Density} bind def
/Pattern2 {0.500 Density} bind def
/Pattern3 {0.750 Density} bind def
/Pattern4 {0.125 Density} bind def
/Pattern5 {0.375 Density} bind def
/Pattern6 {0.625 Density} bind def
/Pattern7 {0.875 Density} bind def
} def
%
% Now test for support of Level 2 code
%
Level1 {Level1PatternFill} {Level2PatternFill} ifelse
%
/Symbol-Oblique /Symbol findfont [1 0 .167 1 0 0] makefont
dup length dict begin {1 index /FID eq {pop pop} {def} ifelse} forall
currentdict end definefont pop
%
Level1 SuppressPDFMark or 
{} {
/SDict 10 dict def
systemdict /pdfmark known not {
  userdict /pdfmark systemdict /cleartomark get put
} if
SDict begin [
  /Title (plot_k2k1_cont.tex)
  /Subject (gnuplot plot)
  /Creator (gnuplot 5.0 patchlevel 3)
  /Author (mteper)
%  /Producer (gnuplot)
%  /Keywords ()
  /CreationDate (Sun Dec 27 21:21:32 2020)
  /DOCINFO pdfmark
end
} ifelse
%
% Support for boxed text - Ethan A Merritt May 2005
%
/InitTextBox { userdict /TBy2 3 -1 roll put userdict /TBx2 3 -1 roll put
           userdict /TBy1 3 -1 roll put userdict /TBx1 3 -1 roll put
	   /Boxing true def } def
/ExtendTextBox { Boxing
    { gsave dup false charpath pathbbox
      dup TBy2 gt {userdict /TBy2 3 -1 roll put} {pop} ifelse
      dup TBx2 gt {userdict /TBx2 3 -1 roll put} {pop} ifelse
      dup TBy1 lt {userdict /TBy1 3 -1 roll put} {pop} ifelse
      dup TBx1 lt {userdict /TBx1 3 -1 roll put} {pop} ifelse
      grestore } if } def
/PopTextBox { newpath TBx1 TBxmargin sub TBy1 TBymargin sub M
               TBx1 TBxmargin sub TBy2 TBymargin add L
	       TBx2 TBxmargin add TBy2 TBymargin add L
	       TBx2 TBxmargin add TBy1 TBymargin sub L closepath } def
/DrawTextBox { PopTextBox stroke /Boxing false def} def
/FillTextBox { gsave PopTextBox 1 1 1 setrgbcolor fill grestore /Boxing false def} def
0 0 0 0 InitTextBox
/TBxmargin 20 def
/TBymargin 20 def
/Boxing false def
/textshow { ExtendTextBox Gshow } def
%
% redundant definitions for compatibility with prologue.ps older than 5.0.2
/LTB {BL [] LCb DL} def
/LTb {PL [] LCb DL} def
end
%%EndProlog
%%Page: 1 1
gnudict begin
gsave
doclip
0 0 translate
0.050 0.050 scale
0 setgray
newpath
BackgroundColor 0 lt 3 1 roll 0 lt exch 0 lt or or not {BackgroundColor C 1.000 0 0 7200.00 7560.00 BoxColFill} if
1.000 UL
LTb
LCb setrgbcolor
1140 640 M
63 0 V
5636 0 R
-63 0 V
stroke
LTb
LCb setrgbcolor
1140 1594 M
63 0 V
5636 0 R
-63 0 V
stroke
LTb
LCb setrgbcolor
1140 2548 M
63 0 V
5636 0 R
-63 0 V
stroke
LTb
LCb setrgbcolor
1140 3502 M
63 0 V
5636 0 R
-63 0 V
stroke
LTb
LCb setrgbcolor
1140 4457 M
63 0 V
5636 0 R
-63 0 V
stroke
LTb
LCb setrgbcolor
1140 5411 M
63 0 V
5636 0 R
-63 0 V
stroke
LTb
LCb setrgbcolor
1140 6365 M
63 0 V
5636 0 R
-63 0 V
stroke
LTb
LCb setrgbcolor
1140 7319 M
63 0 V
5636 0 R
-63 0 V
stroke
LTb
LCb setrgbcolor
1140 640 M
0 63 V
0 6616 R
0 -63 V
stroke
LTb
LCb setrgbcolor
2017 640 M
0 63 V
0 6616 R
0 -63 V
stroke
LTb
LCb setrgbcolor
2894 640 M
0 63 V
0 6616 R
0 -63 V
stroke
LTb
LCb setrgbcolor
3770 640 M
0 63 V
0 6616 R
0 -63 V
stroke
LTb
LCb setrgbcolor
4647 640 M
0 63 V
0 6616 R
0 -63 V
stroke
LTb
LCb setrgbcolor
5524 640 M
0 63 V
0 6616 R
0 -63 V
stroke
LTb
LCb setrgbcolor
6401 640 M
0 63 V
0 6616 R
0 -63 V
stroke
LTb
LCb setrgbcolor
1.000 UL
LTb
LCb setrgbcolor
1140 7319 N
0 -6679 V
5699 0 V
0 6679 V
-5699 0 V
Z stroke
1.000 UP
1.000 UL
LTb
LCb setrgbcolor
LCb setrgbcolor
LTb
LCb setrgbcolor
LTb
1.500 UP
1.000 UL
LTb
0.58 0.00 0.83 C 5141 1904 M
0 613 V
3974 1934 M
0 325 V
3154 1944 M
0 240 V
-570 5 R
0 275 V
2167 2173 M
0 229 V
1888 2149 M
0 291 V
5141 2210 CircleF
3974 2096 CircleF
3154 2064 CircleF
2584 2326 CircleF
2167 2288 CircleF
1888 2295 CircleF
1.500 UL
LTb
0.58 0.00 0.83 C 1140 2363 M
58 -5 V
57 -5 V
58 -5 V
57 -6 V
58 -5 V
57 -5 V
58 -5 V
58 -5 V
57 -6 V
58 -5 V
57 -5 V
58 -5 V
57 -6 V
58 -5 V
57 -5 V
58 -5 V
58 -6 V
57 -5 V
58 -5 V
57 -5 V
58 -5 V
57 -6 V
58 -5 V
58 -5 V
57 -5 V
58 -5 V
57 -6 V
58 -5 V
57 -5 V
58 -5 V
58 -6 V
57 -5 V
58 -5 V
57 -5 V
58 -5 V
57 -6 V
58 -5 V
57 -5 V
58 -5 V
58 -5 V
57 -6 V
58 -5 V
57 -5 V
58 -5 V
57 -5 V
58 -6 V
58 -5 V
57 -5 V
58 -5 V
57 -5 V
58 -6 V
57 -5 V
58 -5 V
58 -5 V
57 -5 V
58 -5 V
57 -6 V
58 -5 V
57 -5 V
58 -5 V
58 -5 V
57 -6 V
58 -5 V
57 -5 V
58 -5 V
57 -5 V
58 -5 V
57 -6 V
58 -5 V
58 -5 V
57 -5 V
58 -5 V
57 -5 V
58 -6 V
57 -5 V
58 -5 V
58 -5 V
57 -5 V
58 -5 V
57 -6 V
58 -5 V
57 -5 V
58 -5 V
58 -5 V
57 -5 V
58 -5 V
57 -6 V
58 -5 V
57 -5 V
58 -5 V
57 -5 V
58 -5 V
58 -5 V
57 -6 V
58 -5 V
57 -5 V
58 -5 V
57 -5 V
58 -5 V
1.500 UP
stroke
1.000 UL
LTb
0.58 0.00 0.83 C 5172 3527 M
0 417 V
3982 3708 M
0 276 V
3304 3846 M
0 244 V
2830 3859 M
0 191 V
2210 3854 M
0 233 V
1893 3821 M
0 178 V
5172 3735 Circle
3982 3846 Circle
3304 3968 Circle
2830 3955 Circle
2210 3970 Circle
1893 3910 Circle
1.500 UL
LTb
0.58 0.00 0.83 C 1140 3989 M
58 -2 V
57 -2 V
58 -3 V
57 -2 V
58 -2 V
57 -3 V
58 -2 V
58 -2 V
57 -2 V
58 -3 V
57 -2 V
58 -2 V
57 -3 V
58 -2 V
57 -2 V
58 -3 V
58 -2 V
57 -2 V
58 -2 V
57 -3 V
58 -2 V
57 -2 V
58 -3 V
58 -2 V
57 -2 V
58 -3 V
57 -2 V
58 -2 V
57 -2 V
58 -3 V
58 -2 V
57 -2 V
58 -3 V
57 -2 V
58 -2 V
57 -3 V
58 -2 V
57 -2 V
58 -2 V
58 -3 V
57 -2 V
58 -2 V
57 -3 V
58 -2 V
57 -2 V
58 -3 V
58 -2 V
57 -2 V
58 -2 V
57 -3 V
58 -2 V
57 -2 V
58 -3 V
58 -2 V
57 -2 V
58 -2 V
57 -3 V
58 -2 V
57 -2 V
58 -3 V
58 -2 V
57 -2 V
58 -2 V
57 -3 V
58 -2 V
57 -2 V
58 -3 V
57 -2 V
58 -2 V
58 -3 V
57 -2 V
58 -2 V
57 -2 V
58 -3 V
57 -2 V
58 -2 V
58 -3 V
57 -2 V
58 -2 V
57 -2 V
58 -3 V
57 -2 V
58 -2 V
58 -3 V
57 -2 V
58 -2 V
57 -2 V
58 -3 V
57 -2 V
58 -2 V
57 -3 V
58 -2 V
58 -2 V
57 -2 V
58 -3 V
57 -2 V
58 -2 V
57 -3 V
58 -2 V
1.500 UP
stroke
1.000 UL
LTb
0.58 0.00 0.83 C 5260 4473 M
0 611 V
3919 4599 M
0 306 V
-617 77 R
0 294 V
2920 4925 M
0 231 V
2190 4770 M
0 176 V
-324 -67 R
0 238 V
5260 4778 BoxF
3919 4752 BoxF
3302 5129 BoxF
2920 5040 BoxF
2190 4858 BoxF
1866 4998 BoxF
1.500 UL
LTb
0.58 0.00 0.83 C 1140 4975 M
58 -2 V
57 -1 V
58 -1 V
57 -1 V
58 -2 V
57 -1 V
58 -1 V
58 -2 V
57 -1 V
58 -1 V
57 -1 V
58 -2 V
57 -1 V
58 -1 V
57 -2 V
58 -1 V
58 -1 V
57 -1 V
58 -2 V
57 -1 V
58 -1 V
57 -2 V
58 -1 V
58 -1 V
57 -1 V
58 -2 V
57 -1 V
58 -1 V
57 -2 V
58 -1 V
58 -1 V
57 -1 V
58 -2 V
57 -1 V
58 -1 V
57 -2 V
58 -1 V
57 -1 V
58 -1 V
58 -2 V
57 -1 V
58 -1 V
57 -2 V
58 -1 V
57 -1 V
58 -1 V
58 -2 V
57 -1 V
58 -1 V
57 -2 V
58 -1 V
57 -1 V
58 -1 V
58 -2 V
57 -1 V
58 -1 V
57 -2 V
58 -1 V
57 -1 V
58 -1 V
58 -2 V
57 -1 V
58 -1 V
57 -2 V
58 -1 V
57 -1 V
58 -1 V
57 -2 V
58 -1 V
58 -1 V
57 -2 V
58 -1 V
57 -1 V
58 -1 V
57 -2 V
58 -1 V
58 -1 V
57 -2 V
58 -1 V
57 -1 V
58 -1 V
57 -2 V
58 -1 V
58 -1 V
57 -2 V
58 -1 V
57 -1 V
58 -1 V
57 -2 V
58 -1 V
57 -1 V
58 -2 V
58 -1 V
57 -1 V
58 -1 V
57 -2 V
58 -1 V
57 -1 V
58 -2 V
1.500 UP
stroke
1.000 UL
LTb
0.58 0.00 0.83 C 5796 4744 M
0 464 V
4056 4910 M
0 261 V
-823 -61 R
0 241 V
-522 248 R
0 291 V
2342 5210 M
0 247 V
-432 15 R
0 203 V
5796 4976 Box
4056 5041 Box
3233 5230 Box
2711 5745 Box
2342 5333 Box
1910 5573 Box
1.500 UL
LTb
0.58 0.00 0.83 C 1140 5705 M
58 -11 V
57 -12 V
58 -11 V
57 -11 V
58 -11 V
57 -11 V
58 -11 V
58 -11 V
57 -11 V
58 -11 V
57 -11 V
58 -11 V
57 -11 V
58 -11 V
57 -11 V
58 -11 V
58 -11 V
57 -11 V
58 -11 V
57 -11 V
58 -11 V
57 -11 V
58 -11 V
58 -11 V
57 -11 V
58 -11 V
57 -11 V
58 -11 V
57 -11 V
58 -11 V
58 -11 V
57 -11 V
58 -11 V
57 -11 V
58 -11 V
57 -11 V
58 -11 V
57 -11 V
58 -11 V
58 -10 V
57 -11 V
58 -11 V
57 -11 V
58 -11 V
57 -11 V
58 -11 V
58 -11 V
57 -11 V
58 -11 V
57 -11 V
58 -10 V
57 -11 V
58 -11 V
58 -11 V
57 -11 V
58 -11 V
57 -11 V
58 -11 V
57 -10 V
58 -11 V
58 -11 V
57 -11 V
58 -11 V
57 -11 V
58 -11 V
57 -10 V
58 -11 V
57 -11 V
58 -11 V
58 -11 V
57 -10 V
58 -11 V
57 -11 V
58 -11 V
57 -11 V
58 -10 V
58 -11 V
57 -11 V
58 -11 V
57 -11 V
58 -10 V
57 -11 V
58 -11 V
58 -11 V
57 -10 V
58 -11 V
57 -11 V
58 -11 V
57 -11 V
58 -10 V
57 -11 V
58 -11 V
58 -10 V
57 -11 V
58 -11 V
57 -11 V
58 -10 V
57 -11 V
58 -11 V
1.500 UP
stroke
1.000 UL
LTb
0.58 0.00 0.83 C 5921 5594 M
0 506 V
4100 5603 M
0 346 V
3260 5679 M
0 231 V
-563 -43 R
0 258 V
2317 5567 M
0 192 V
5921 5847 TriUF
4100 5776 TriUF
3260 5795 TriUF
2697 5996 TriUF
2317 5663 TriUF
1.500 UL
LTb
0.58 0.00 0.83 C 1140 5727 M
58 2 V
57 2 V
58 2 V
57 2 V
58 2 V
57 2 V
58 1 V
58 2 V
57 2 V
58 2 V
57 2 V
58 2 V
57 2 V
58 2 V
57 1 V
58 2 V
58 2 V
57 2 V
58 2 V
57 2 V
58 2 V
57 2 V
58 2 V
58 1 V
57 2 V
58 2 V
57 2 V
58 2 V
57 2 V
58 2 V
58 2 V
57 1 V
58 2 V
57 2 V
58 2 V
57 2 V
58 2 V
57 2 V
58 2 V
58 2 V
57 1 V
58 2 V
57 2 V
58 2 V
57 2 V
58 2 V
58 2 V
57 2 V
58 1 V
57 2 V
58 2 V
57 2 V
58 2 V
58 2 V
57 2 V
58 2 V
57 2 V
58 1 V
57 2 V
58 2 V
58 2 V
57 2 V
58 2 V
57 2 V
58 2 V
57 2 V
58 1 V
57 2 V
58 2 V
58 2 V
57 2 V
58 2 V
57 2 V
58 2 V
57 2 V
58 1 V
58 2 V
57 2 V
58 2 V
57 2 V
58 2 V
57 2 V
58 2 V
58 2 V
57 1 V
58 2 V
57 2 V
58 2 V
57 2 V
58 2 V
57 2 V
58 2 V
58 2 V
57 1 V
58 2 V
57 2 V
58 2 V
57 2 V
58 2 V
1.500 UP
stroke
1.000 UL
LTb
0.58 0.00 0.83 C 6013 5903 M
0 604 V
4141 5729 M
0 289 V
-895 27 R
0 274 V
-579 -76 R
0 269 V
2290 6007 M
0 224 V
6013 6205 TriU
4141 5873 TriU
3246 6182 TriU
2667 6378 TriU
2290 6119 TriU
1.500 UL
LTb
0.58 0.00 0.83 C 1140 6290 M
58 -5 V
57 -4 V
58 -4 V
57 -4 V
58 -4 V
57 -5 V
58 -4 V
58 -4 V
57 -4 V
58 -4 V
57 -4 V
58 -5 V
57 -4 V
58 -4 V
57 -4 V
58 -4 V
58 -5 V
57 -4 V
58 -4 V
57 -4 V
58 -4 V
57 -4 V
58 -5 V
58 -4 V
57 -4 V
58 -4 V
57 -4 V
58 -5 V
57 -4 V
58 -4 V
58 -4 V
57 -4 V
58 -4 V
57 -5 V
58 -4 V
57 -4 V
58 -4 V
57 -4 V
58 -4 V
58 -5 V
57 -4 V
58 -4 V
57 -4 V
58 -4 V
57 -5 V
58 -4 V
58 -4 V
57 -4 V
58 -4 V
57 -4 V
58 -5 V
57 -4 V
58 -4 V
58 -4 V
57 -4 V
58 -4 V
57 -5 V
58 -4 V
57 -4 V
58 -4 V
58 -4 V
57 -4 V
58 -5 V
57 -4 V
58 -4 V
57 -4 V
58 -4 V
57 -4 V
58 -4 V
58 -5 V
57 -4 V
58 -4 V
57 -4 V
58 -4 V
57 -4 V
58 -5 V
58 -4 V
57 -4 V
58 -4 V
57 -4 V
58 -4 V
57 -4 V
58 -5 V
58 -4 V
57 -4 V
58 -4 V
57 -4 V
58 -4 V
57 -5 V
58 -4 V
57 -4 V
58 -4 V
58 -4 V
57 -4 V
58 -4 V
57 -5 V
58 -4 V
57 -4 V
58 -4 V
stroke
2.000 UL
LTb
LCb setrgbcolor
1.000 UL
LTb
LCb setrgbcolor
1140 7319 N
0 -6679 V
5699 0 V
0 6679 V
-5699 0 V
Z stroke
1.000 UP
1.000 UL
LTb
LCb setrgbcolor
stroke
grestore
end
showpage
  }}%
  \put(3989,140){\makebox(0,0){\large{$a^2\sigma_f$}}}%
  \put(200,4979){\makebox(0,0){\Large{$\frac{\sigma_{k=2}}{\sigma_f}$}}}%
  \put(6401,440){\makebox(0,0){\strut{}\ {$0.12$}}}%
  \put(5524,440){\makebox(0,0){\strut{}\ {$0.1$}}}%
  \put(4647,440){\makebox(0,0){\strut{}\ {$0.08$}}}%
  \put(3770,440){\makebox(0,0){\strut{}\ {$0.06$}}}%
  \put(2894,440){\makebox(0,0){\strut{}\ {$0.04$}}}%
  \put(2017,440){\makebox(0,0){\strut{}\ {$0.02$}}}%
  \put(1140,440){\makebox(0,0){\strut{}\ {$0$}}}%
  \put(1020,7319){\makebox(0,0)[r]{\strut{}\ \ {$1.9$}}}%
  \put(1020,6365){\makebox(0,0)[r]{\strut{}\ \ {$1.8$}}}%
  \put(1020,5411){\makebox(0,0)[r]{\strut{}\ \ {$1.7$}}}%
  \put(1020,4457){\makebox(0,0)[r]{\strut{}\ \ {$1.6$}}}%
  \put(1020,3502){\makebox(0,0)[r]{\strut{}\ \ {$1.5$}}}%
  \put(1020,2548){\makebox(0,0)[r]{\strut{}\ \ {$1.4$}}}%
  \put(1020,1594){\makebox(0,0)[r]{\strut{}\ \ {$1.3$}}}%
  \put(1020,640){\makebox(0,0)[r]{\strut{}\ \ {$1.2$}}}%
\end{picture}%
\endgroup
\endinput

\end	{center}
\caption{$k=2$ string tensions, $\sigma_k$, in $SU(N)$ gauge theories
  for $N=4,5,6,8,10,12$ in ascending order, in units of the $k=1$ fundamental string
  tension, $\sigma_f$. Lines are extrapolations to the continuum limits.}
\label{fig_k2k1_cont}
\end{figure}



\begin{figure}[htb]
\begin	{center}
\leavevmode
% GNUPLOT: LaTeX picture with Postscript
\begingroup%
\makeatletter%
\newcommand{\GNUPLOTspecial}{%
  \@sanitize\catcode`\%=14\relax\special}%
\setlength{\unitlength}{0.0500bp}%
\begin{picture}(7200,7560)(0,0)%
  {\GNUPLOTspecial{"
%!PS-Adobe-2.0 EPSF-2.0
%%Title: plot_k2k1_N.tex
%%Creator: gnuplot 5.0 patchlevel 3
%%CreationDate: Mon Dec 21 20:59:44 2020
%%DocumentFonts: 
%%BoundingBox: 0 0 360 378
%%EndComments
%%BeginProlog
/gnudict 256 dict def
gnudict begin
%
% The following true/false flags may be edited by hand if desired.
% The unit line width and grayscale image gamma correction may also be changed.
%
/Color true def
/Blacktext true def
/Solid false def
/Dashlength 1 def
/Landscape false def
/Level1 false def
/Level3 false def
/Rounded false def
/ClipToBoundingBox false def
/SuppressPDFMark false def
/TransparentPatterns false def
/gnulinewidth 5.000 def
/userlinewidth gnulinewidth def
/Gamma 1.0 def
/BackgroundColor {-1.000 -1.000 -1.000} def
%
/vshift -66 def
/dl1 {
  10.0 Dashlength userlinewidth gnulinewidth div mul mul mul
  Rounded { currentlinewidth 0.75 mul sub dup 0 le { pop 0.01 } if } if
} def
/dl2 {
  10.0 Dashlength userlinewidth gnulinewidth div mul mul mul
  Rounded { currentlinewidth 0.75 mul add } if
} def
/hpt_ 31.5 def
/vpt_ 31.5 def
/hpt hpt_ def
/vpt vpt_ def
/doclip {
  ClipToBoundingBox {
    newpath 0 0 moveto 360 0 lineto 360 378 lineto 0 378 lineto closepath
    clip
  } if
} def
%
% Gnuplot Prolog Version 5.1 (Oct 2015)
%
%/SuppressPDFMark true def
%
/M {moveto} bind def
/L {lineto} bind def
/R {rmoveto} bind def
/V {rlineto} bind def
/N {newpath moveto} bind def
/Z {closepath} bind def
/C {setrgbcolor} bind def
/f {rlineto fill} bind def
/g {setgray} bind def
/Gshow {show} def   % May be redefined later in the file to support UTF-8
/vpt2 vpt 2 mul def
/hpt2 hpt 2 mul def
/Lshow {currentpoint stroke M 0 vshift R 
	Blacktext {gsave 0 setgray textshow grestore} {textshow} ifelse} def
/Rshow {currentpoint stroke M dup stringwidth pop neg vshift R
	Blacktext {gsave 0 setgray textshow grestore} {textshow} ifelse} def
/Cshow {currentpoint stroke M dup stringwidth pop -2 div vshift R 
	Blacktext {gsave 0 setgray textshow grestore} {textshow} ifelse} def
/UP {dup vpt_ mul /vpt exch def hpt_ mul /hpt exch def
  /hpt2 hpt 2 mul def /vpt2 vpt 2 mul def} def
/DL {Color {setrgbcolor Solid {pop []} if 0 setdash}
 {pop pop pop 0 setgray Solid {pop []} if 0 setdash} ifelse} def
/BL {stroke userlinewidth 2 mul setlinewidth
	Rounded {1 setlinejoin 1 setlinecap} if} def
/AL {stroke userlinewidth 2 div setlinewidth
	Rounded {1 setlinejoin 1 setlinecap} if} def
/UL {dup gnulinewidth mul /userlinewidth exch def
	dup 1 lt {pop 1} if 10 mul /udl exch def} def
/PL {stroke userlinewidth setlinewidth
	Rounded {1 setlinejoin 1 setlinecap} if} def
3.8 setmiterlimit
% Classic Line colors (version 5.0)
/LCw {1 1 1} def
/LCb {0 0 0} def
/LCa {0 0 0} def
/LC0 {1 0 0} def
/LC1 {0 1 0} def
/LC2 {0 0 1} def
/LC3 {1 0 1} def
/LC4 {0 1 1} def
/LC5 {1 1 0} def
/LC6 {0 0 0} def
/LC7 {1 0.3 0} def
/LC8 {0.5 0.5 0.5} def
% Default dash patterns (version 5.0)
/LTB {BL [] LCb DL} def
/LTw {PL [] 1 setgray} def
/LTb {PL [] LCb DL} def
/LTa {AL [1 udl mul 2 udl mul] 0 setdash LCa setrgbcolor} def
/LT0 {PL [] LC0 DL} def
/LT1 {PL [2 dl1 3 dl2] LC1 DL} def
/LT2 {PL [1 dl1 1.5 dl2] LC2 DL} def
/LT3 {PL [6 dl1 2 dl2 1 dl1 2 dl2] LC3 DL} def
/LT4 {PL [1 dl1 2 dl2 6 dl1 2 dl2 1 dl1 2 dl2] LC4 DL} def
/LT5 {PL [4 dl1 2 dl2] LC5 DL} def
/LT6 {PL [1.5 dl1 1.5 dl2 1.5 dl1 1.5 dl2 1.5 dl1 6 dl2] LC6 DL} def
/LT7 {PL [3 dl1 3 dl2 1 dl1 3 dl2] LC7 DL} def
/LT8 {PL [2 dl1 2 dl2 2 dl1 6 dl2] LC8 DL} def
/SL {[] 0 setdash} def
/Pnt {stroke [] 0 setdash gsave 1 setlinecap M 0 0 V stroke grestore} def
/Dia {stroke [] 0 setdash 2 copy vpt add M
  hpt neg vpt neg V hpt vpt neg V
  hpt vpt V hpt neg vpt V closepath stroke
  Pnt} def
/Pls {stroke [] 0 setdash vpt sub M 0 vpt2 V
  currentpoint stroke M
  hpt neg vpt neg R hpt2 0 V stroke
 } def
/Box {stroke [] 0 setdash 2 copy exch hpt sub exch vpt add M
  0 vpt2 neg V hpt2 0 V 0 vpt2 V
  hpt2 neg 0 V closepath stroke
  Pnt} def
/Crs {stroke [] 0 setdash exch hpt sub exch vpt add M
  hpt2 vpt2 neg V currentpoint stroke M
  hpt2 neg 0 R hpt2 vpt2 V stroke} def
/TriU {stroke [] 0 setdash 2 copy vpt 1.12 mul add M
  hpt neg vpt -1.62 mul V
  hpt 2 mul 0 V
  hpt neg vpt 1.62 mul V closepath stroke
  Pnt} def
/Star {2 copy Pls Crs} def
/BoxF {stroke [] 0 setdash exch hpt sub exch vpt add M
  0 vpt2 neg V hpt2 0 V 0 vpt2 V
  hpt2 neg 0 V closepath fill} def
/TriUF {stroke [] 0 setdash vpt 1.12 mul add M
  hpt neg vpt -1.62 mul V
  hpt 2 mul 0 V
  hpt neg vpt 1.62 mul V closepath fill} def
/TriD {stroke [] 0 setdash 2 copy vpt 1.12 mul sub M
  hpt neg vpt 1.62 mul V
  hpt 2 mul 0 V
  hpt neg vpt -1.62 mul V closepath stroke
  Pnt} def
/TriDF {stroke [] 0 setdash vpt 1.12 mul sub M
  hpt neg vpt 1.62 mul V
  hpt 2 mul 0 V
  hpt neg vpt -1.62 mul V closepath fill} def
/DiaF {stroke [] 0 setdash vpt add M
  hpt neg vpt neg V hpt vpt neg V
  hpt vpt V hpt neg vpt V closepath fill} def
/Pent {stroke [] 0 setdash 2 copy gsave
  translate 0 hpt M 4 {72 rotate 0 hpt L} repeat
  closepath stroke grestore Pnt} def
/PentF {stroke [] 0 setdash gsave
  translate 0 hpt M 4 {72 rotate 0 hpt L} repeat
  closepath fill grestore} def
/Circle {stroke [] 0 setdash 2 copy
  hpt 0 360 arc stroke Pnt} def
/CircleF {stroke [] 0 setdash hpt 0 360 arc fill} def
/C0 {BL [] 0 setdash 2 copy moveto vpt 90 450 arc} bind def
/C1 {BL [] 0 setdash 2 copy moveto
	2 copy vpt 0 90 arc closepath fill
	vpt 0 360 arc closepath} bind def
/C2 {BL [] 0 setdash 2 copy moveto
	2 copy vpt 90 180 arc closepath fill
	vpt 0 360 arc closepath} bind def
/C3 {BL [] 0 setdash 2 copy moveto
	2 copy vpt 0 180 arc closepath fill
	vpt 0 360 arc closepath} bind def
/C4 {BL [] 0 setdash 2 copy moveto
	2 copy vpt 180 270 arc closepath fill
	vpt 0 360 arc closepath} bind def
/C5 {BL [] 0 setdash 2 copy moveto
	2 copy vpt 0 90 arc
	2 copy moveto
	2 copy vpt 180 270 arc closepath fill
	vpt 0 360 arc} bind def
/C6 {BL [] 0 setdash 2 copy moveto
	2 copy vpt 90 270 arc closepath fill
	vpt 0 360 arc closepath} bind def
/C7 {BL [] 0 setdash 2 copy moveto
	2 copy vpt 0 270 arc closepath fill
	vpt 0 360 arc closepath} bind def
/C8 {BL [] 0 setdash 2 copy moveto
	2 copy vpt 270 360 arc closepath fill
	vpt 0 360 arc closepath} bind def
/C9 {BL [] 0 setdash 2 copy moveto
	2 copy vpt 270 450 arc closepath fill
	vpt 0 360 arc closepath} bind def
/C10 {BL [] 0 setdash 2 copy 2 copy moveto vpt 270 360 arc closepath fill
	2 copy moveto
	2 copy vpt 90 180 arc closepath fill
	vpt 0 360 arc closepath} bind def
/C11 {BL [] 0 setdash 2 copy moveto
	2 copy vpt 0 180 arc closepath fill
	2 copy moveto
	2 copy vpt 270 360 arc closepath fill
	vpt 0 360 arc closepath} bind def
/C12 {BL [] 0 setdash 2 copy moveto
	2 copy vpt 180 360 arc closepath fill
	vpt 0 360 arc closepath} bind def
/C13 {BL [] 0 setdash 2 copy moveto
	2 copy vpt 0 90 arc closepath fill
	2 copy moveto
	2 copy vpt 180 360 arc closepath fill
	vpt 0 360 arc closepath} bind def
/C14 {BL [] 0 setdash 2 copy moveto
	2 copy vpt 90 360 arc closepath fill
	vpt 0 360 arc} bind def
/C15 {BL [] 0 setdash 2 copy vpt 0 360 arc closepath fill
	vpt 0 360 arc closepath} bind def
/Rec {newpath 4 2 roll moveto 1 index 0 rlineto 0 exch rlineto
	neg 0 rlineto closepath} bind def
/Square {dup Rec} bind def
/Bsquare {vpt sub exch vpt sub exch vpt2 Square} bind def
/S0 {BL [] 0 setdash 2 copy moveto 0 vpt rlineto BL Bsquare} bind def
/S1 {BL [] 0 setdash 2 copy vpt Square fill Bsquare} bind def
/S2 {BL [] 0 setdash 2 copy exch vpt sub exch vpt Square fill Bsquare} bind def
/S3 {BL [] 0 setdash 2 copy exch vpt sub exch vpt2 vpt Rec fill Bsquare} bind def
/S4 {BL [] 0 setdash 2 copy exch vpt sub exch vpt sub vpt Square fill Bsquare} bind def
/S5 {BL [] 0 setdash 2 copy 2 copy vpt Square fill
	exch vpt sub exch vpt sub vpt Square fill Bsquare} bind def
/S6 {BL [] 0 setdash 2 copy exch vpt sub exch vpt sub vpt vpt2 Rec fill Bsquare} bind def
/S7 {BL [] 0 setdash 2 copy exch vpt sub exch vpt sub vpt vpt2 Rec fill
	2 copy vpt Square fill Bsquare} bind def
/S8 {BL [] 0 setdash 2 copy vpt sub vpt Square fill Bsquare} bind def
/S9 {BL [] 0 setdash 2 copy vpt sub vpt vpt2 Rec fill Bsquare} bind def
/S10 {BL [] 0 setdash 2 copy vpt sub vpt Square fill 2 copy exch vpt sub exch vpt Square fill
	Bsquare} bind def
/S11 {BL [] 0 setdash 2 copy vpt sub vpt Square fill 2 copy exch vpt sub exch vpt2 vpt Rec fill
	Bsquare} bind def
/S12 {BL [] 0 setdash 2 copy exch vpt sub exch vpt sub vpt2 vpt Rec fill Bsquare} bind def
/S13 {BL [] 0 setdash 2 copy exch vpt sub exch vpt sub vpt2 vpt Rec fill
	2 copy vpt Square fill Bsquare} bind def
/S14 {BL [] 0 setdash 2 copy exch vpt sub exch vpt sub vpt2 vpt Rec fill
	2 copy exch vpt sub exch vpt Square fill Bsquare} bind def
/S15 {BL [] 0 setdash 2 copy Bsquare fill Bsquare} bind def
/D0 {gsave translate 45 rotate 0 0 S0 stroke grestore} bind def
/D1 {gsave translate 45 rotate 0 0 S1 stroke grestore} bind def
/D2 {gsave translate 45 rotate 0 0 S2 stroke grestore} bind def
/D3 {gsave translate 45 rotate 0 0 S3 stroke grestore} bind def
/D4 {gsave translate 45 rotate 0 0 S4 stroke grestore} bind def
/D5 {gsave translate 45 rotate 0 0 S5 stroke grestore} bind def
/D6 {gsave translate 45 rotate 0 0 S6 stroke grestore} bind def
/D7 {gsave translate 45 rotate 0 0 S7 stroke grestore} bind def
/D8 {gsave translate 45 rotate 0 0 S8 stroke grestore} bind def
/D9 {gsave translate 45 rotate 0 0 S9 stroke grestore} bind def
/D10 {gsave translate 45 rotate 0 0 S10 stroke grestore} bind def
/D11 {gsave translate 45 rotate 0 0 S11 stroke grestore} bind def
/D12 {gsave translate 45 rotate 0 0 S12 stroke grestore} bind def
/D13 {gsave translate 45 rotate 0 0 S13 stroke grestore} bind def
/D14 {gsave translate 45 rotate 0 0 S14 stroke grestore} bind def
/D15 {gsave translate 45 rotate 0 0 S15 stroke grestore} bind def
/DiaE {stroke [] 0 setdash vpt add M
  hpt neg vpt neg V hpt vpt neg V
  hpt vpt V hpt neg vpt V closepath stroke} def
/BoxE {stroke [] 0 setdash exch hpt sub exch vpt add M
  0 vpt2 neg V hpt2 0 V 0 vpt2 V
  hpt2 neg 0 V closepath stroke} def
/TriUE {stroke [] 0 setdash vpt 1.12 mul add M
  hpt neg vpt -1.62 mul V
  hpt 2 mul 0 V
  hpt neg vpt 1.62 mul V closepath stroke} def
/TriDE {stroke [] 0 setdash vpt 1.12 mul sub M
  hpt neg vpt 1.62 mul V
  hpt 2 mul 0 V
  hpt neg vpt -1.62 mul V closepath stroke} def
/PentE {stroke [] 0 setdash gsave
  translate 0 hpt M 4 {72 rotate 0 hpt L} repeat
  closepath stroke grestore} def
/CircE {stroke [] 0 setdash 
  hpt 0 360 arc stroke} def
/Opaque {gsave closepath 1 setgray fill grestore 0 setgray closepath} def
/DiaW {stroke [] 0 setdash vpt add M
  hpt neg vpt neg V hpt vpt neg V
  hpt vpt V hpt neg vpt V Opaque stroke} def
/BoxW {stroke [] 0 setdash exch hpt sub exch vpt add M
  0 vpt2 neg V hpt2 0 V 0 vpt2 V
  hpt2 neg 0 V Opaque stroke} def
/TriUW {stroke [] 0 setdash vpt 1.12 mul add M
  hpt neg vpt -1.62 mul V
  hpt 2 mul 0 V
  hpt neg vpt 1.62 mul V Opaque stroke} def
/TriDW {stroke [] 0 setdash vpt 1.12 mul sub M
  hpt neg vpt 1.62 mul V
  hpt 2 mul 0 V
  hpt neg vpt -1.62 mul V Opaque stroke} def
/PentW {stroke [] 0 setdash gsave
  translate 0 hpt M 4 {72 rotate 0 hpt L} repeat
  Opaque stroke grestore} def
/CircW {stroke [] 0 setdash 
  hpt 0 360 arc Opaque stroke} def
/BoxFill {gsave Rec 1 setgray fill grestore} def
/Density {
  /Fillden exch def
  currentrgbcolor
  /ColB exch def /ColG exch def /ColR exch def
  /ColR ColR Fillden mul Fillden sub 1 add def
  /ColG ColG Fillden mul Fillden sub 1 add def
  /ColB ColB Fillden mul Fillden sub 1 add def
  ColR ColG ColB setrgbcolor} def
/BoxColFill {gsave Rec PolyFill} def
/PolyFill {gsave Density fill grestore grestore} def
/h {rlineto rlineto rlineto gsave closepath fill grestore} bind def
%
% PostScript Level 1 Pattern Fill routine for rectangles
% Usage: x y w h s a XX PatternFill
%	x,y = lower left corner of box to be filled
%	w,h = width and height of box
%	  a = angle in degrees between lines and x-axis
%	 XX = 0/1 for no/yes cross-hatch
%
/PatternFill {gsave /PFa [ 9 2 roll ] def
  PFa 0 get PFa 2 get 2 div add PFa 1 get PFa 3 get 2 div add translate
  PFa 2 get -2 div PFa 3 get -2 div PFa 2 get PFa 3 get Rec
  TransparentPatterns {} {gsave 1 setgray fill grestore} ifelse
  clip
  currentlinewidth 0.5 mul setlinewidth
  /PFs PFa 2 get dup mul PFa 3 get dup mul add sqrt def
  0 0 M PFa 5 get rotate PFs -2 div dup translate
  0 1 PFs PFa 4 get div 1 add floor cvi
	{PFa 4 get mul 0 M 0 PFs V} for
  0 PFa 6 get ne {
	0 1 PFs PFa 4 get div 1 add floor cvi
	{PFa 4 get mul 0 2 1 roll M PFs 0 V} for
 } if
  stroke grestore} def
%
/languagelevel where
 {pop languagelevel} {1} ifelse
dup 2 lt
	{/InterpretLevel1 true def
	 /InterpretLevel3 false def}
	{/InterpretLevel1 Level1 def
	 2 gt
	    {/InterpretLevel3 Level3 def}
	    {/InterpretLevel3 false def}
	 ifelse }
 ifelse
%
% PostScript level 2 pattern fill definitions
%
/Level2PatternFill {
/Tile8x8 {/PaintType 2 /PatternType 1 /TilingType 1 /BBox [0 0 8 8] /XStep 8 /YStep 8}
	bind def
/KeepColor {currentrgbcolor [/Pattern /DeviceRGB] setcolorspace} bind def
<< Tile8x8
 /PaintProc {0.5 setlinewidth pop 0 0 M 8 8 L 0 8 M 8 0 L stroke} 
>> matrix makepattern
/Pat1 exch def
<< Tile8x8
 /PaintProc {0.5 setlinewidth pop 0 0 M 8 8 L 0 8 M 8 0 L stroke
	0 4 M 4 8 L 8 4 L 4 0 L 0 4 L stroke}
>> matrix makepattern
/Pat2 exch def
<< Tile8x8
 /PaintProc {0.5 setlinewidth pop 0 0 M 0 8 L
	8 8 L 8 0 L 0 0 L fill}
>> matrix makepattern
/Pat3 exch def
<< Tile8x8
 /PaintProc {0.5 setlinewidth pop -4 8 M 8 -4 L
	0 12 M 12 0 L stroke}
>> matrix makepattern
/Pat4 exch def
<< Tile8x8
 /PaintProc {0.5 setlinewidth pop -4 0 M 8 12 L
	0 -4 M 12 8 L stroke}
>> matrix makepattern
/Pat5 exch def
<< Tile8x8
 /PaintProc {0.5 setlinewidth pop -2 8 M 4 -4 L
	0 12 M 8 -4 L 4 12 M 10 0 L stroke}
>> matrix makepattern
/Pat6 exch def
<< Tile8x8
 /PaintProc {0.5 setlinewidth pop -2 0 M 4 12 L
	0 -4 M 8 12 L 4 -4 M 10 8 L stroke}
>> matrix makepattern
/Pat7 exch def
<< Tile8x8
 /PaintProc {0.5 setlinewidth pop 8 -2 M -4 4 L
	12 0 M -4 8 L 12 4 M 0 10 L stroke}
>> matrix makepattern
/Pat8 exch def
<< Tile8x8
 /PaintProc {0.5 setlinewidth pop 0 -2 M 12 4 L
	-4 0 M 12 8 L -4 4 M 8 10 L stroke}
>> matrix makepattern
/Pat9 exch def
/Pattern1 {PatternBgnd KeepColor Pat1 setpattern} bind def
/Pattern2 {PatternBgnd KeepColor Pat2 setpattern} bind def
/Pattern3 {PatternBgnd KeepColor Pat3 setpattern} bind def
/Pattern4 {PatternBgnd KeepColor Landscape {Pat5} {Pat4} ifelse setpattern} bind def
/Pattern5 {PatternBgnd KeepColor Landscape {Pat4} {Pat5} ifelse setpattern} bind def
/Pattern6 {PatternBgnd KeepColor Landscape {Pat9} {Pat6} ifelse setpattern} bind def
/Pattern7 {PatternBgnd KeepColor Landscape {Pat8} {Pat7} ifelse setpattern} bind def
} def
%
%
%End of PostScript Level 2 code
%
/PatternBgnd {
  TransparentPatterns {} {gsave 1 setgray fill grestore} ifelse
} def
%
% Substitute for Level 2 pattern fill codes with
% grayscale if Level 2 support is not selected.
%
/Level1PatternFill {
/Pattern1 {0.250 Density} bind def
/Pattern2 {0.500 Density} bind def
/Pattern3 {0.750 Density} bind def
/Pattern4 {0.125 Density} bind def
/Pattern5 {0.375 Density} bind def
/Pattern6 {0.625 Density} bind def
/Pattern7 {0.875 Density} bind def
} def
%
% Now test for support of Level 2 code
%
Level1 {Level1PatternFill} {Level2PatternFill} ifelse
%
/Symbol-Oblique /Symbol findfont [1 0 .167 1 0 0] makefont
dup length dict begin {1 index /FID eq {pop pop} {def} ifelse} forall
currentdict end definefont pop
%
Level1 SuppressPDFMark or 
{} {
/SDict 10 dict def
systemdict /pdfmark known not {
  userdict /pdfmark systemdict /cleartomark get put
} if
SDict begin [
  /Title (plot_k2k1_N.tex)
  /Subject (gnuplot plot)
  /Creator (gnuplot 5.0 patchlevel 3)
  /Author (mteper)
%  /Producer (gnuplot)
%  /Keywords ()
  /CreationDate (Mon Dec 21 20:59:44 2020)
  /DOCINFO pdfmark
end
} ifelse
%
% Support for boxed text - Ethan A Merritt May 2005
%
/InitTextBox { userdict /TBy2 3 -1 roll put userdict /TBx2 3 -1 roll put
           userdict /TBy1 3 -1 roll put userdict /TBx1 3 -1 roll put
	   /Boxing true def } def
/ExtendTextBox { Boxing
    { gsave dup false charpath pathbbox
      dup TBy2 gt {userdict /TBy2 3 -1 roll put} {pop} ifelse
      dup TBx2 gt {userdict /TBx2 3 -1 roll put} {pop} ifelse
      dup TBy1 lt {userdict /TBy1 3 -1 roll put} {pop} ifelse
      dup TBx1 lt {userdict /TBx1 3 -1 roll put} {pop} ifelse
      grestore } if } def
/PopTextBox { newpath TBx1 TBxmargin sub TBy1 TBymargin sub M
               TBx1 TBxmargin sub TBy2 TBymargin add L
	       TBx2 TBxmargin add TBy2 TBymargin add L
	       TBx2 TBxmargin add TBy1 TBymargin sub L closepath } def
/DrawTextBox { PopTextBox stroke /Boxing false def} def
/FillTextBox { gsave PopTextBox 1 1 1 setrgbcolor fill grestore /Boxing false def} def
0 0 0 0 InitTextBox
/TBxmargin 20 def
/TBymargin 20 def
/Boxing false def
/textshow { ExtendTextBox Gshow } def
%
% redundant definitions for compatibility with prologue.ps older than 5.0.2
/LTB {BL [] LCb DL} def
/LTb {PL [] LCb DL} def
end
%%EndProlog
%%Page: 1 1
gnudict begin
gsave
doclip
0 0 translate
0.050 0.050 scale
0 setgray
newpath
BackgroundColor 0 lt 3 1 roll 0 lt exch 0 lt or or not {BackgroundColor C 1.000 0 0 7200.00 7560.00 BoxColFill} if
1.000 UL
LTb
LCb setrgbcolor
1140 640 M
63 0 V
5636 0 R
-63 0 V
stroke
LTb
LCb setrgbcolor
1140 1594 M
63 0 V
5636 0 R
-63 0 V
stroke
LTb
LCb setrgbcolor
1140 2548 M
63 0 V
5636 0 R
-63 0 V
stroke
LTb
LCb setrgbcolor
1140 3502 M
63 0 V
5636 0 R
-63 0 V
stroke
LTb
LCb setrgbcolor
1140 4457 M
63 0 V
5636 0 R
-63 0 V
stroke
LTb
LCb setrgbcolor
1140 5411 M
63 0 V
5636 0 R
-63 0 V
stroke
LTb
LCb setrgbcolor
1140 6365 M
63 0 V
5636 0 R
-63 0 V
stroke
LTb
LCb setrgbcolor
1140 7319 M
63 0 V
5636 0 R
-63 0 V
stroke
LTb
LCb setrgbcolor
1140 640 M
0 63 V
0 6616 R
0 -63 V
stroke
LTb
LCb setrgbcolor
1852 640 M
0 63 V
0 6616 R
0 -63 V
stroke
LTb
LCb setrgbcolor
2565 640 M
0 63 V
0 6616 R
0 -63 V
stroke
LTb
LCb setrgbcolor
3277 640 M
0 63 V
0 6616 R
0 -63 V
stroke
LTb
LCb setrgbcolor
3990 640 M
0 63 V
0 6616 R
0 -63 V
stroke
LTb
LCb setrgbcolor
4702 640 M
0 63 V
0 6616 R
0 -63 V
stroke
LTb
LCb setrgbcolor
5414 640 M
0 63 V
0 6616 R
0 -63 V
stroke
LTb
LCb setrgbcolor
6127 640 M
0 63 V
0 6616 R
0 -63 V
stroke
LTb
LCb setrgbcolor
6839 640 M
0 63 V
0 6616 R
0 -63 V
stroke
LTb
LCb setrgbcolor
1.000 UL
LTb
LCb setrgbcolor
1140 7319 N
0 -6679 V
5699 0 V
0 6679 V
-5699 0 V
Z stroke
1.000 UP
1.000 UL
LTb
LCb setrgbcolor
LCb setrgbcolor
LTb
LCb setrgbcolor
LTb
1.500 UP
1.000 UL
LTb
0.58 0.00 0.83 C 5592 2391 M
0 133 V
3990 3216 M
0 105 V
-871 377 R
0 124 V
-866 477 R
0 258 V
1852 4299 M
0 277 V
-218 14 R
0 277 V
5592 2458 CircleF
3990 3269 CircleF
3119 3760 CircleF
2253 4428 CircleF
1852 4437 CircleF
1634 4729 CircleF
1.500 UL
LTb
LCb setrgbcolor
1140 5411 M
58 -193 V
57 -90 V
58 -74 V
57 -65 V
58 -59 V
57 -56 V
58 -52 V
58 -50 V
57 -49 V
58 -46 V
57 -46 V
58 -44 V
57 -43 V
58 -42 V
57 -41 V
58 -41 V
58 -39 V
57 -40 V
58 -38 V
57 -38 V
58 -38 V
57 -37 V
58 -37 V
58 -36 V
57 -36 V
58 -36 V
57 -35 V
58 -35 V
57 -35 V
58 -34 V
58 -35 V
57 -33 V
58 -34 V
57 -34 V
58 -33 V
57 -33 V
58 -33 V
57 -32 V
58 -33 V
58 -32 V
57 -32 V
58 -32 V
57 -32 V
58 -31 V
57 -32 V
58 -31 V
58 -31 V
57 -31 V
58 -31 V
57 -31 V
58 -31 V
57 -30 V
58 -31 V
58 -30 V
57 -30 V
58 -30 V
57 -30 V
58 -30 V
57 -30 V
58 -30 V
58 -29 V
57 -30 V
58 -29 V
57 -30 V
58 -29 V
57 -29 V
58 -29 V
57 -29 V
58 -29 V
58 -29 V
57 -29 V
58 -28 V
57 -29 V
58 -29 V
57 -28 V
58 -28 V
58 -29 V
57 -28 V
58 -28 V
57 -29 V
58 -28 V
57 -28 V
58 -28 V
58 -28 V
57 -28 V
58 -27 V
57 -28 V
58 -28 V
57 -28 V
58 -27 V
57 -28 V
58 -27 V
58 -28 V
57 -27 V
58 -27 V
57 -28 V
58 -27 V
57 -27 V
58 -27 V
stroke
LTa
LCa setrgbcolor
1140 5411 M
58 -56 V
57 -55 V
58 -54 V
57 -54 V
58 -54 V
57 -53 V
58 -52 V
58 -53 V
57 -51 V
58 -52 V
57 -50 V
58 -51 V
57 -50 V
58 -49 V
57 -49 V
58 -49 V
58 -48 V
57 -47 V
58 -47 V
57 -47 V
58 -46 V
57 -46 V
58 -45 V
58 -45 V
57 -44 V
58 -44 V
57 -44 V
58 -43 V
57 -42 V
58 -42 V
58 -42 V
57 -41 V
58 -41 V
57 -40 V
58 -40 V
57 -39 V
58 -39 V
57 -38 V
58 -38 V
58 -38 V
57 -37 V
58 -36 V
57 -36 V
58 -36 V
57 -35 V
58 -35 V
58 -34 V
57 -34 V
58 -33 V
57 -33 V
58 -32 V
57 -32 V
58 -32 V
58 -31 V
57 -30 V
58 -31 V
57 -29 V
58 -29 V
57 -29 V
58 -28 V
58 -28 V
57 -28 V
58 -26 V
57 -27 V
58 -26 V
57 -25 V
58 -25 V
57 -25 V
58 -24 V
58 -24 V
57 -23 V
58 -23 V
57 -22 V
58 -22 V
57 -21 V
58 -21 V
58 -21 V
57 -20 V
58 -19 V
57 -19 V
58 -19 V
57 -18 V
58 -18 V
58 -17 V
57 -17 V
58 -16 V
57 -16 V
58 -15 V
57 -15 V
58 -15 V
57 -14 V
58 -13 V
58 -13 V
57 -13 V
58 -12 V
57 -12 V
58 -11 V
57 -11 V
58 -10 V
stroke
2.000 UL
LTb
LCb setrgbcolor
1.000 UL
LTb
LCb setrgbcolor
1140 7319 N
0 -6679 V
5699 0 V
0 6679 V
-5699 0 V
Z stroke
1.000 UP
1.000 UL
LTb
LCb setrgbcolor
stroke
grestore
end
showpage
  }}%
  \put(3989,140){\makebox(0,0){\large{$1/N^2$}}}%
  \put(200,4979){\makebox(0,0){\Large{$\frac{\sigma_{k=2}}{\sigma_f}$}}}%
  \put(6839,440){\makebox(0,0){\strut{}\ {$0.08$}}}%
  \put(6127,440){\makebox(0,0){\strut{}\ {$0.07$}}}%
  \put(5414,440){\makebox(0,0){\strut{}\ {$0.06$}}}%
  \put(4702,440){\makebox(0,0){\strut{}\ {$0.05$}}}%
  \put(3990,440){\makebox(0,0){\strut{}\ {$0.04$}}}%
  \put(3277,440){\makebox(0,0){\strut{}\ {$0.03$}}}%
  \put(2565,440){\makebox(0,0){\strut{}\ {$0.02$}}}%
  \put(1852,440){\makebox(0,0){\strut{}\ {$0.01$}}}%
  \put(1140,440){\makebox(0,0){\strut{}\ {$0$}}}%
  \put(1020,7319){\makebox(0,0)[r]{\strut{}\ \ {$2.4$}}}%
  \put(1020,6365){\makebox(0,0)[r]{\strut{}\ \ {$2.2$}}}%
  \put(1020,5411){\makebox(0,0)[r]{\strut{}\ \ {$2$}}}%
  \put(1020,4457){\makebox(0,0)[r]{\strut{}\ \ {$1.8$}}}%
  \put(1020,3502){\makebox(0,0)[r]{\strut{}\ \ {$1.6$}}}%
  \put(1020,2548){\makebox(0,0)[r]{\strut{}\ \ {$1.4$}}}%
  \put(1020,1594){\makebox(0,0)[r]{\strut{}\ \ {$1.2$}}}%
  \put(1020,640){\makebox(0,0)[r]{\strut{}\ \ {$1$}}}%
\end{picture}%
\endgroup
\endinput

\end	{center}
\caption{Continuum limit of $k=2$ string tension, $\sigma_k$, in units of
  the $k=1$ fundamental string tension, $\sigma_f$, for our $SU(N)$ gauge
  theories. Solid line is the best fit in powers of $1/N$ and dashed line
  is the best fit in powers of $1/N^2$, with the constraint that the ratio
  is 2 at $N=\infty$.}
\label{fig_k2k1_N}
\end{figure}




\begin{figure}[htb]
\begin	{center}
\leavevmode
% GNUPLOT: LaTeX picture with Postscript
\begingroup%
\makeatletter%
\newcommand{\GNUPLOTspecial}{%
  \@sanitize\catcode`\%=14\relax\special}%
\setlength{\unitlength}{0.0500bp}%
\begin{picture}(7200,7560)(0,0)%
  {\GNUPLOTspecial{"
%!PS-Adobe-2.0 EPSF-2.0
%%Title: plot_ggINK_suN.tex
%%Creator: gnuplot 5.0 patchlevel 3
%%CreationDate: Mon Aug 30 17:01:03 2021
%%DocumentFonts: 
%%BoundingBox: 0 0 360 378
%%EndComments
%%BeginProlog
/gnudict 256 dict def
gnudict begin
%
% The following true/false flags may be edited by hand if desired.
% The unit line width and grayscale image gamma correction may also be changed.
%
/Color true def
/Blacktext true def
/Solid false def
/Dashlength 1 def
/Landscape false def
/Level1 false def
/Level3 false def
/Rounded false def
/ClipToBoundingBox false def
/SuppressPDFMark false def
/TransparentPatterns false def
/gnulinewidth 5.000 def
/userlinewidth gnulinewidth def
/Gamma 1.0 def
/BackgroundColor {-1.000 -1.000 -1.000} def
%
/vshift -66 def
/dl1 {
  10.0 Dashlength userlinewidth gnulinewidth div mul mul mul
  Rounded { currentlinewidth 0.75 mul sub dup 0 le { pop 0.01 } if } if
} def
/dl2 {
  10.0 Dashlength userlinewidth gnulinewidth div mul mul mul
  Rounded { currentlinewidth 0.75 mul add } if
} def
/hpt_ 31.5 def
/vpt_ 31.5 def
/hpt hpt_ def
/vpt vpt_ def
/doclip {
  ClipToBoundingBox {
    newpath 0 0 moveto 360 0 lineto 360 378 lineto 0 378 lineto closepath
    clip
  } if
} def
%
% Gnuplot Prolog Version 5.1 (Oct 2015)
%
%/SuppressPDFMark true def
%
/M {moveto} bind def
/L {lineto} bind def
/R {rmoveto} bind def
/V {rlineto} bind def
/N {newpath moveto} bind def
/Z {closepath} bind def
/C {setrgbcolor} bind def
/f {rlineto fill} bind def
/g {setgray} bind def
/Gshow {show} def   % May be redefined later in the file to support UTF-8
/vpt2 vpt 2 mul def
/hpt2 hpt 2 mul def
/Lshow {currentpoint stroke M 0 vshift R 
	Blacktext {gsave 0 setgray textshow grestore} {textshow} ifelse} def
/Rshow {currentpoint stroke M dup stringwidth pop neg vshift R
	Blacktext {gsave 0 setgray textshow grestore} {textshow} ifelse} def
/Cshow {currentpoint stroke M dup stringwidth pop -2 div vshift R 
	Blacktext {gsave 0 setgray textshow grestore} {textshow} ifelse} def
/UP {dup vpt_ mul /vpt exch def hpt_ mul /hpt exch def
  /hpt2 hpt 2 mul def /vpt2 vpt 2 mul def} def
/DL {Color {setrgbcolor Solid {pop []} if 0 setdash}
 {pop pop pop 0 setgray Solid {pop []} if 0 setdash} ifelse} def
/BL {stroke userlinewidth 2 mul setlinewidth
	Rounded {1 setlinejoin 1 setlinecap} if} def
/AL {stroke userlinewidth 2 div setlinewidth
	Rounded {1 setlinejoin 1 setlinecap} if} def
/UL {dup gnulinewidth mul /userlinewidth exch def
	dup 1 lt {pop 1} if 10 mul /udl exch def} def
/PL {stroke userlinewidth setlinewidth
	Rounded {1 setlinejoin 1 setlinecap} if} def
3.8 setmiterlimit
% Classic Line colors (version 5.0)
/LCw {1 1 1} def
/LCb {0 0 0} def
/LCa {0 0 0} def
/LC0 {1 0 0} def
/LC1 {0 1 0} def
/LC2 {0 0 1} def
/LC3 {1 0 1} def
/LC4 {0 1 1} def
/LC5 {1 1 0} def
/LC6 {0 0 0} def
/LC7 {1 0.3 0} def
/LC8 {0.5 0.5 0.5} def
% Default dash patterns (version 5.0)
/LTB {BL [] LCb DL} def
/LTw {PL [] 1 setgray} def
/LTb {PL [] LCb DL} def
/LTa {AL [1 udl mul 2 udl mul] 0 setdash LCa setrgbcolor} def
/LT0 {PL [] LC0 DL} def
/LT1 {PL [2 dl1 3 dl2] LC1 DL} def
/LT2 {PL [1 dl1 1.5 dl2] LC2 DL} def
/LT3 {PL [6 dl1 2 dl2 1 dl1 2 dl2] LC3 DL} def
/LT4 {PL [1 dl1 2 dl2 6 dl1 2 dl2 1 dl1 2 dl2] LC4 DL} def
/LT5 {PL [4 dl1 2 dl2] LC5 DL} def
/LT6 {PL [1.5 dl1 1.5 dl2 1.5 dl1 1.5 dl2 1.5 dl1 6 dl2] LC6 DL} def
/LT7 {PL [3 dl1 3 dl2 1 dl1 3 dl2] LC7 DL} def
/LT8 {PL [2 dl1 2 dl2 2 dl1 6 dl2] LC8 DL} def
/SL {[] 0 setdash} def
/Pnt {stroke [] 0 setdash gsave 1 setlinecap M 0 0 V stroke grestore} def
/Dia {stroke [] 0 setdash 2 copy vpt add M
  hpt neg vpt neg V hpt vpt neg V
  hpt vpt V hpt neg vpt V closepath stroke
  Pnt} def
/Pls {stroke [] 0 setdash vpt sub M 0 vpt2 V
  currentpoint stroke M
  hpt neg vpt neg R hpt2 0 V stroke
 } def
/Box {stroke [] 0 setdash 2 copy exch hpt sub exch vpt add M
  0 vpt2 neg V hpt2 0 V 0 vpt2 V
  hpt2 neg 0 V closepath stroke
  Pnt} def
/Crs {stroke [] 0 setdash exch hpt sub exch vpt add M
  hpt2 vpt2 neg V currentpoint stroke M
  hpt2 neg 0 R hpt2 vpt2 V stroke} def
/TriU {stroke [] 0 setdash 2 copy vpt 1.12 mul add M
  hpt neg vpt -1.62 mul V
  hpt 2 mul 0 V
  hpt neg vpt 1.62 mul V closepath stroke
  Pnt} def
/Star {2 copy Pls Crs} def
/BoxF {stroke [] 0 setdash exch hpt sub exch vpt add M
  0 vpt2 neg V hpt2 0 V 0 vpt2 V
  hpt2 neg 0 V closepath fill} def
/TriUF {stroke [] 0 setdash vpt 1.12 mul add M
  hpt neg vpt -1.62 mul V
  hpt 2 mul 0 V
  hpt neg vpt 1.62 mul V closepath fill} def
/TriD {stroke [] 0 setdash 2 copy vpt 1.12 mul sub M
  hpt neg vpt 1.62 mul V
  hpt 2 mul 0 V
  hpt neg vpt -1.62 mul V closepath stroke
  Pnt} def
/TriDF {stroke [] 0 setdash vpt 1.12 mul sub M
  hpt neg vpt 1.62 mul V
  hpt 2 mul 0 V
  hpt neg vpt -1.62 mul V closepath fill} def
/DiaF {stroke [] 0 setdash vpt add M
  hpt neg vpt neg V hpt vpt neg V
  hpt vpt V hpt neg vpt V closepath fill} def
/Pent {stroke [] 0 setdash 2 copy gsave
  translate 0 hpt M 4 {72 rotate 0 hpt L} repeat
  closepath stroke grestore Pnt} def
/PentF {stroke [] 0 setdash gsave
  translate 0 hpt M 4 {72 rotate 0 hpt L} repeat
  closepath fill grestore} def
/Circle {stroke [] 0 setdash 2 copy
  hpt 0 360 arc stroke Pnt} def
/CircleF {stroke [] 0 setdash hpt 0 360 arc fill} def
/C0 {BL [] 0 setdash 2 copy moveto vpt 90 450 arc} bind def
/C1 {BL [] 0 setdash 2 copy moveto
	2 copy vpt 0 90 arc closepath fill
	vpt 0 360 arc closepath} bind def
/C2 {BL [] 0 setdash 2 copy moveto
	2 copy vpt 90 180 arc closepath fill
	vpt 0 360 arc closepath} bind def
/C3 {BL [] 0 setdash 2 copy moveto
	2 copy vpt 0 180 arc closepath fill
	vpt 0 360 arc closepath} bind def
/C4 {BL [] 0 setdash 2 copy moveto
	2 copy vpt 180 270 arc closepath fill
	vpt 0 360 arc closepath} bind def
/C5 {BL [] 0 setdash 2 copy moveto
	2 copy vpt 0 90 arc
	2 copy moveto
	2 copy vpt 180 270 arc closepath fill
	vpt 0 360 arc} bind def
/C6 {BL [] 0 setdash 2 copy moveto
	2 copy vpt 90 270 arc closepath fill
	vpt 0 360 arc closepath} bind def
/C7 {BL [] 0 setdash 2 copy moveto
	2 copy vpt 0 270 arc closepath fill
	vpt 0 360 arc closepath} bind def
/C8 {BL [] 0 setdash 2 copy moveto
	2 copy vpt 270 360 arc closepath fill
	vpt 0 360 arc closepath} bind def
/C9 {BL [] 0 setdash 2 copy moveto
	2 copy vpt 270 450 arc closepath fill
	vpt 0 360 arc closepath} bind def
/C10 {BL [] 0 setdash 2 copy 2 copy moveto vpt 270 360 arc closepath fill
	2 copy moveto
	2 copy vpt 90 180 arc closepath fill
	vpt 0 360 arc closepath} bind def
/C11 {BL [] 0 setdash 2 copy moveto
	2 copy vpt 0 180 arc closepath fill
	2 copy moveto
	2 copy vpt 270 360 arc closepath fill
	vpt 0 360 arc closepath} bind def
/C12 {BL [] 0 setdash 2 copy moveto
	2 copy vpt 180 360 arc closepath fill
	vpt 0 360 arc closepath} bind def
/C13 {BL [] 0 setdash 2 copy moveto
	2 copy vpt 0 90 arc closepath fill
	2 copy moveto
	2 copy vpt 180 360 arc closepath fill
	vpt 0 360 arc closepath} bind def
/C14 {BL [] 0 setdash 2 copy moveto
	2 copy vpt 90 360 arc closepath fill
	vpt 0 360 arc} bind def
/C15 {BL [] 0 setdash 2 copy vpt 0 360 arc closepath fill
	vpt 0 360 arc closepath} bind def
/Rec {newpath 4 2 roll moveto 1 index 0 rlineto 0 exch rlineto
	neg 0 rlineto closepath} bind def
/Square {dup Rec} bind def
/Bsquare {vpt sub exch vpt sub exch vpt2 Square} bind def
/S0 {BL [] 0 setdash 2 copy moveto 0 vpt rlineto BL Bsquare} bind def
/S1 {BL [] 0 setdash 2 copy vpt Square fill Bsquare} bind def
/S2 {BL [] 0 setdash 2 copy exch vpt sub exch vpt Square fill Bsquare} bind def
/S3 {BL [] 0 setdash 2 copy exch vpt sub exch vpt2 vpt Rec fill Bsquare} bind def
/S4 {BL [] 0 setdash 2 copy exch vpt sub exch vpt sub vpt Square fill Bsquare} bind def
/S5 {BL [] 0 setdash 2 copy 2 copy vpt Square fill
	exch vpt sub exch vpt sub vpt Square fill Bsquare} bind def
/S6 {BL [] 0 setdash 2 copy exch vpt sub exch vpt sub vpt vpt2 Rec fill Bsquare} bind def
/S7 {BL [] 0 setdash 2 copy exch vpt sub exch vpt sub vpt vpt2 Rec fill
	2 copy vpt Square fill Bsquare} bind def
/S8 {BL [] 0 setdash 2 copy vpt sub vpt Square fill Bsquare} bind def
/S9 {BL [] 0 setdash 2 copy vpt sub vpt vpt2 Rec fill Bsquare} bind def
/S10 {BL [] 0 setdash 2 copy vpt sub vpt Square fill 2 copy exch vpt sub exch vpt Square fill
	Bsquare} bind def
/S11 {BL [] 0 setdash 2 copy vpt sub vpt Square fill 2 copy exch vpt sub exch vpt2 vpt Rec fill
	Bsquare} bind def
/S12 {BL [] 0 setdash 2 copy exch vpt sub exch vpt sub vpt2 vpt Rec fill Bsquare} bind def
/S13 {BL [] 0 setdash 2 copy exch vpt sub exch vpt sub vpt2 vpt Rec fill
	2 copy vpt Square fill Bsquare} bind def
/S14 {BL [] 0 setdash 2 copy exch vpt sub exch vpt sub vpt2 vpt Rec fill
	2 copy exch vpt sub exch vpt Square fill Bsquare} bind def
/S15 {BL [] 0 setdash 2 copy Bsquare fill Bsquare} bind def
/D0 {gsave translate 45 rotate 0 0 S0 stroke grestore} bind def
/D1 {gsave translate 45 rotate 0 0 S1 stroke grestore} bind def
/D2 {gsave translate 45 rotate 0 0 S2 stroke grestore} bind def
/D3 {gsave translate 45 rotate 0 0 S3 stroke grestore} bind def
/D4 {gsave translate 45 rotate 0 0 S4 stroke grestore} bind def
/D5 {gsave translate 45 rotate 0 0 S5 stroke grestore} bind def
/D6 {gsave translate 45 rotate 0 0 S6 stroke grestore} bind def
/D7 {gsave translate 45 rotate 0 0 S7 stroke grestore} bind def
/D8 {gsave translate 45 rotate 0 0 S8 stroke grestore} bind def
/D9 {gsave translate 45 rotate 0 0 S9 stroke grestore} bind def
/D10 {gsave translate 45 rotate 0 0 S10 stroke grestore} bind def
/D11 {gsave translate 45 rotate 0 0 S11 stroke grestore} bind def
/D12 {gsave translate 45 rotate 0 0 S12 stroke grestore} bind def
/D13 {gsave translate 45 rotate 0 0 S13 stroke grestore} bind def
/D14 {gsave translate 45 rotate 0 0 S14 stroke grestore} bind def
/D15 {gsave translate 45 rotate 0 0 S15 stroke grestore} bind def
/DiaE {stroke [] 0 setdash vpt add M
  hpt neg vpt neg V hpt vpt neg V
  hpt vpt V hpt neg vpt V closepath stroke} def
/BoxE {stroke [] 0 setdash exch hpt sub exch vpt add M
  0 vpt2 neg V hpt2 0 V 0 vpt2 V
  hpt2 neg 0 V closepath stroke} def
/TriUE {stroke [] 0 setdash vpt 1.12 mul add M
  hpt neg vpt -1.62 mul V
  hpt 2 mul 0 V
  hpt neg vpt 1.62 mul V closepath stroke} def
/TriDE {stroke [] 0 setdash vpt 1.12 mul sub M
  hpt neg vpt 1.62 mul V
  hpt 2 mul 0 V
  hpt neg vpt -1.62 mul V closepath stroke} def
/PentE {stroke [] 0 setdash gsave
  translate 0 hpt M 4 {72 rotate 0 hpt L} repeat
  closepath stroke grestore} def
/CircE {stroke [] 0 setdash 
  hpt 0 360 arc stroke} def
/Opaque {gsave closepath 1 setgray fill grestore 0 setgray closepath} def
/DiaW {stroke [] 0 setdash vpt add M
  hpt neg vpt neg V hpt vpt neg V
  hpt vpt V hpt neg vpt V Opaque stroke} def
/BoxW {stroke [] 0 setdash exch hpt sub exch vpt add M
  0 vpt2 neg V hpt2 0 V 0 vpt2 V
  hpt2 neg 0 V Opaque stroke} def
/TriUW {stroke [] 0 setdash vpt 1.12 mul add M
  hpt neg vpt -1.62 mul V
  hpt 2 mul 0 V
  hpt neg vpt 1.62 mul V Opaque stroke} def
/TriDW {stroke [] 0 setdash vpt 1.12 mul sub M
  hpt neg vpt 1.62 mul V
  hpt 2 mul 0 V
  hpt neg vpt -1.62 mul V Opaque stroke} def
/PentW {stroke [] 0 setdash gsave
  translate 0 hpt M 4 {72 rotate 0 hpt L} repeat
  Opaque stroke grestore} def
/CircW {stroke [] 0 setdash 
  hpt 0 360 arc Opaque stroke} def
/BoxFill {gsave Rec 1 setgray fill grestore} def
/Density {
  /Fillden exch def
  currentrgbcolor
  /ColB exch def /ColG exch def /ColR exch def
  /ColR ColR Fillden mul Fillden sub 1 add def
  /ColG ColG Fillden mul Fillden sub 1 add def
  /ColB ColB Fillden mul Fillden sub 1 add def
  ColR ColG ColB setrgbcolor} def
/BoxColFill {gsave Rec PolyFill} def
/PolyFill {gsave Density fill grestore grestore} def
/h {rlineto rlineto rlineto gsave closepath fill grestore} bind def
%
% PostScript Level 1 Pattern Fill routine for rectangles
% Usage: x y w h s a XX PatternFill
%	x,y = lower left corner of box to be filled
%	w,h = width and height of box
%	  a = angle in degrees between lines and x-axis
%	 XX = 0/1 for no/yes cross-hatch
%
/PatternFill {gsave /PFa [ 9 2 roll ] def
  PFa 0 get PFa 2 get 2 div add PFa 1 get PFa 3 get 2 div add translate
  PFa 2 get -2 div PFa 3 get -2 div PFa 2 get PFa 3 get Rec
  TransparentPatterns {} {gsave 1 setgray fill grestore} ifelse
  clip
  currentlinewidth 0.5 mul setlinewidth
  /PFs PFa 2 get dup mul PFa 3 get dup mul add sqrt def
  0 0 M PFa 5 get rotate PFs -2 div dup translate
  0 1 PFs PFa 4 get div 1 add floor cvi
	{PFa 4 get mul 0 M 0 PFs V} for
  0 PFa 6 get ne {
	0 1 PFs PFa 4 get div 1 add floor cvi
	{PFa 4 get mul 0 2 1 roll M PFs 0 V} for
 } if
  stroke grestore} def
%
/languagelevel where
 {pop languagelevel} {1} ifelse
dup 2 lt
	{/InterpretLevel1 true def
	 /InterpretLevel3 false def}
	{/InterpretLevel1 Level1 def
	 2 gt
	    {/InterpretLevel3 Level3 def}
	    {/InterpretLevel3 false def}
	 ifelse }
 ifelse
%
% PostScript level 2 pattern fill definitions
%
/Level2PatternFill {
/Tile8x8 {/PaintType 2 /PatternType 1 /TilingType 1 /BBox [0 0 8 8] /XStep 8 /YStep 8}
	bind def
/KeepColor {currentrgbcolor [/Pattern /DeviceRGB] setcolorspace} bind def
<< Tile8x8
 /PaintProc {0.5 setlinewidth pop 0 0 M 8 8 L 0 8 M 8 0 L stroke} 
>> matrix makepattern
/Pat1 exch def
<< Tile8x8
 /PaintProc {0.5 setlinewidth pop 0 0 M 8 8 L 0 8 M 8 0 L stroke
	0 4 M 4 8 L 8 4 L 4 0 L 0 4 L stroke}
>> matrix makepattern
/Pat2 exch def
<< Tile8x8
 /PaintProc {0.5 setlinewidth pop 0 0 M 0 8 L
	8 8 L 8 0 L 0 0 L fill}
>> matrix makepattern
/Pat3 exch def
<< Tile8x8
 /PaintProc {0.5 setlinewidth pop -4 8 M 8 -4 L
	0 12 M 12 0 L stroke}
>> matrix makepattern
/Pat4 exch def
<< Tile8x8
 /PaintProc {0.5 setlinewidth pop -4 0 M 8 12 L
	0 -4 M 12 8 L stroke}
>> matrix makepattern
/Pat5 exch def
<< Tile8x8
 /PaintProc {0.5 setlinewidth pop -2 8 M 4 -4 L
	0 12 M 8 -4 L 4 12 M 10 0 L stroke}
>> matrix makepattern
/Pat6 exch def
<< Tile8x8
 /PaintProc {0.5 setlinewidth pop -2 0 M 4 12 L
	0 -4 M 8 12 L 4 -4 M 10 8 L stroke}
>> matrix makepattern
/Pat7 exch def
<< Tile8x8
 /PaintProc {0.5 setlinewidth pop 8 -2 M -4 4 L
	12 0 M -4 8 L 12 4 M 0 10 L stroke}
>> matrix makepattern
/Pat8 exch def
<< Tile8x8
 /PaintProc {0.5 setlinewidth pop 0 -2 M 12 4 L
	-4 0 M 12 8 L -4 4 M 8 10 L stroke}
>> matrix makepattern
/Pat9 exch def
/Pattern1 {PatternBgnd KeepColor Pat1 setpattern} bind def
/Pattern2 {PatternBgnd KeepColor Pat2 setpattern} bind def
/Pattern3 {PatternBgnd KeepColor Pat3 setpattern} bind def
/Pattern4 {PatternBgnd KeepColor Landscape {Pat5} {Pat4} ifelse setpattern} bind def
/Pattern5 {PatternBgnd KeepColor Landscape {Pat4} {Pat5} ifelse setpattern} bind def
/Pattern6 {PatternBgnd KeepColor Landscape {Pat9} {Pat6} ifelse setpattern} bind def
/Pattern7 {PatternBgnd KeepColor Landscape {Pat8} {Pat7} ifelse setpattern} bind def
} def
%
%
%End of PostScript Level 2 code
%
/PatternBgnd {
  TransparentPatterns {} {gsave 1 setgray fill grestore} ifelse
} def
%
% Substitute for Level 2 pattern fill codes with
% grayscale if Level 2 support is not selected.
%
/Level1PatternFill {
/Pattern1 {0.250 Density} bind def
/Pattern2 {0.500 Density} bind def
/Pattern3 {0.750 Density} bind def
/Pattern4 {0.125 Density} bind def
/Pattern5 {0.375 Density} bind def
/Pattern6 {0.625 Density} bind def
/Pattern7 {0.875 Density} bind def
} def
%
% Now test for support of Level 2 code
%
Level1 {Level1PatternFill} {Level2PatternFill} ifelse
%
/Symbol-Oblique /Symbol findfont [1 0 .167 1 0 0] makefont
dup length dict begin {1 index /FID eq {pop pop} {def} ifelse} forall
currentdict end definefont pop
%
Level1 SuppressPDFMark or 
{} {
/SDict 10 dict def
systemdict /pdfmark known not {
  userdict /pdfmark systemdict /cleartomark get put
} if
SDict begin [
  /Title (plot_ggINK_suN.tex)
  /Subject (gnuplot plot)
  /Creator (gnuplot 5.0 patchlevel 3)
  /Author (mteper)
%  /Producer (gnuplot)
%  /Keywords ()
  /CreationDate (Mon Aug 30 17:01:03 2021)
  /DOCINFO pdfmark
end
} ifelse
%
% Support for boxed text - Ethan A Merritt May 2005
%
/InitTextBox { userdict /TBy2 3 -1 roll put userdict /TBx2 3 -1 roll put
           userdict /TBy1 3 -1 roll put userdict /TBx1 3 -1 roll put
	   /Boxing true def } def
/ExtendTextBox { Boxing
    { gsave dup false charpath pathbbox
      dup TBy2 gt {userdict /TBy2 3 -1 roll put} {pop} ifelse
      dup TBx2 gt {userdict /TBx2 3 -1 roll put} {pop} ifelse
      dup TBy1 lt {userdict /TBy1 3 -1 roll put} {pop} ifelse
      dup TBx1 lt {userdict /TBx1 3 -1 roll put} {pop} ifelse
      grestore } if } def
/PopTextBox { newpath TBx1 TBxmargin sub TBy1 TBymargin sub M
               TBx1 TBxmargin sub TBy2 TBymargin add L
	       TBx2 TBxmargin add TBy2 TBymargin add L
	       TBx2 TBxmargin add TBy1 TBymargin sub L closepath } def
/DrawTextBox { PopTextBox stroke /Boxing false def} def
/FillTextBox { gsave PopTextBox 1 1 1 setrgbcolor fill grestore /Boxing false def} def
0 0 0 0 InitTextBox
/TBxmargin 20 def
/TBymargin 20 def
/Boxing false def
/textshow { ExtendTextBox Gshow } def
%
% redundant definitions for compatibility with prologue.ps older than 5.0.2
/LTB {BL [] LCb DL} def
/LTb {PL [] LCb DL} def
end
%%EndProlog
%%Page: 1 1
gnudict begin
gsave
doclip
0 0 translate
0.050 0.050 scale
0 setgray
newpath
BackgroundColor 0 lt 3 1 roll 0 lt exch 0 lt or or not {BackgroundColor C 1.000 0 0 7200.00 7560.00 BoxColFill} if
1.000 UL
LTb
LCb setrgbcolor
1140 640 M
63 0 V
5636 0 R
-63 0 V
stroke
LTb
LCb setrgbcolor
1140 1753 M
63 0 V
5636 0 R
-63 0 V
stroke
LTb
LCb setrgbcolor
1140 2866 M
63 0 V
5636 0 R
-63 0 V
stroke
LTb
LCb setrgbcolor
1140 3980 M
63 0 V
5636 0 R
-63 0 V
stroke
LTb
LCb setrgbcolor
1140 5093 M
63 0 V
5636 0 R
-63 0 V
stroke
LTb
LCb setrgbcolor
1140 6206 M
63 0 V
5636 0 R
-63 0 V
stroke
LTb
LCb setrgbcolor
1140 7319 M
63 0 V
5636 0 R
-63 0 V
stroke
LTb
LCb setrgbcolor
1140 640 M
0 63 V
0 6616 R
0 -63 V
stroke
LTb
LCb setrgbcolor
1773 640 M
0 63 V
0 6616 R
0 -63 V
stroke
LTb
LCb setrgbcolor
2406 640 M
0 63 V
0 6616 R
0 -63 V
stroke
LTb
LCb setrgbcolor
3040 640 M
0 63 V
0 6616 R
0 -63 V
stroke
LTb
LCb setrgbcolor
3673 640 M
0 63 V
0 6616 R
0 -63 V
stroke
LTb
LCb setrgbcolor
4306 640 M
0 63 V
0 6616 R
0 -63 V
stroke
LTb
LCb setrgbcolor
4939 640 M
0 63 V
0 6616 R
0 -63 V
stroke
LTb
LCb setrgbcolor
5573 640 M
0 63 V
0 6616 R
0 -63 V
stroke
LTb
LCb setrgbcolor
6206 640 M
0 63 V
0 6616 R
0 -63 V
stroke
LTb
LCb setrgbcolor
6839 640 M
0 63 V
0 6616 R
0 -63 V
stroke
LTb
LCb setrgbcolor
1.000 UL
LTb
LCb setrgbcolor
1140 7319 N
0 -6679 V
5699 0 V
0 6679 V
-5699 0 V
Z stroke
1.000 UP
1.000 UL
LTb
LCb setrgbcolor
LCb setrgbcolor
LTb
LCb setrgbcolor
LTb
1.500 UP
1.000 UL
LTb
0.58 0.00 0.83 C 5774 5723 CircleF
4785 4909 CircleF
4148 4379 CircleF
3900 4161 CircleF
3401 3707 CircleF
2822 3072 CircleF
2576 2760 CircleF
2368 2470 CircleF
2201 2198 CircleF
2057 1943 CircleF
1.500 UP
1.000 UL
LTb
0.58 0.00 0.83 C 6218 5706 Circle
5142 5020 Circle
4449 4549 Circle
3921 4138 Circle
3611 3859 Circle
3027 3269 Circle
2780 2955 Circle
2447 2506 Circle
2283 2252 Circle
2140 2013 Circle
1.500 UP
1.000 UL
LTb
0.58 0.00 0.83 C 4966 4865 BoxF
4360 4440 BoxF
3854 4028 BoxF
3438 3641 BoxF
3078 3256 BoxF
2795 2906 BoxF
1.500 UP
1.000 UL
LTb
0.58 0.00 0.83 C 4981 4866 Box
4364 4418 Box
3954 4081 Box
3627 3791 Box
3118 3264 Box
2800 2883 Box
1.500 UP
1.000 UL
LTb
0.58 0.00 0.83 C 5023 4860 DiaF
4329 4363 DiaF
3953 4060 DiaF
3692 3825 DiaF
3100 3219 DiaF
2770 2834 DiaF
1.500 UP
1.000 UL
LTb
0.58 0.00 0.83 C 5267 4999 Dia
4406 4405 Dia
3907 3988 Dia
3538 3649 Dia
3237 3342 Dia
2818 2861 Dia
1.500 UP
1.000 UL
LTb
0.58 0.00 0.83 C 5322 5005 TriUF
4431 4396 TriUF
3925 3981 TriUF
3527 3612 TriUF
3217 3301 TriUF
1.500 UP
1.000 UL
LTb
0.58 0.00 0.83 C 5362 5018 TriU
4453 4402 TriU
3915 3973 TriU
3504 3591 TriU
3191 3274 TriU
0.500 UL
LTb
LCb setrgbcolor
2210 1976 M
138 222 V
148 223 V
160 223 V
171 222 V
184 223 V
197 223 V
212 222 V
226 223 V
244 223 V
262 222 V
284 223 V
308 222 V
338 223 V
372 223 V
417 222 V
474 223 V
494 199 V
stroke
LTa
LCa setrgbcolor
2132 1976 M
128 222 V
139 223 V
151 223 V
163 222 V
176 223 V
190 223 V
208 222 V
226 223 V
248 223 V
275 222 V
307 223 V
350 222 V
408 223 V
494 223 V
641 222 V
603 137 V
stroke
2.000 UL
LTb
LCb setrgbcolor
1.000 UL
LTb
LCb setrgbcolor
1140 7319 N
0 -6679 V
5699 0 V
0 6679 V
-5699 0 V
Z stroke
1.000 UP
1.000 UL
LTb
LCb setrgbcolor
stroke
grestore
end
showpage
  }}%
  \put(3989,140){\makebox(0,0){\large{$a\surd\sigma$}}}%
  \put(200,4779){\makebox(0,0){\Large{$g^2_I(a)N$}}}%
  \put(6839,440){\makebox(0,0){\strut{}\ {$0.45$}}}%
  \put(6206,440){\makebox(0,0){\strut{}\ {$0.4$}}}%
  \put(5573,440){\makebox(0,0){\strut{}\ {$0.35$}}}%
  \put(4939,440){\makebox(0,0){\strut{}\ {$0.3$}}}%
  \put(4306,440){\makebox(0,0){\strut{}\ {$0.25$}}}%
  \put(3673,440){\makebox(0,0){\strut{}\ {$0.2$}}}%
  \put(3040,440){\makebox(0,0){\strut{}\ {$0.15$}}}%
  \put(2406,440){\makebox(0,0){\strut{}\ {$0.1$}}}%
  \put(1773,440){\makebox(0,0){\strut{}\ {$0.05$}}}%
  \put(1140,440){\makebox(0,0){\strut{}\ {$0$}}}%
  \put(1020,7319){\makebox(0,0)[r]{\strut{}\ \ {$6.5$}}}%
  \put(1020,6206){\makebox(0,0)[r]{\strut{}\ \ {$6$}}}%
  \put(1020,5093){\makebox(0,0)[r]{\strut{}\ \ {$5.5$}}}%
  \put(1020,3980){\makebox(0,0)[r]{\strut{}\ \ {$5$}}}%
  \put(1020,2866){\makebox(0,0)[r]{\strut{}\ \ {$4.5$}}}%
  \put(1020,1753){\makebox(0,0)[r]{\strut{}\ \ {$4$}}}%
  \put(1020,640){\makebox(0,0)[r]{\strut{}\ \ {$3.5$}}}%
\end{picture}%
\endgroup
\endinput

\end	{center}
\caption{Running (mean-field improved) 't Hooft coupling on the lattice scale $a$, expressed in
  units of the string tension, for
  $SU(2)$, $\bullet$, $SU(3)$, $\circ$, $SU(4)$, $\blacksquare$,
  $SU(5)$, $\square$, $SU(6)$,  $\blacklozenge$, $SU(8)$, $\lozenge$,
  $SU(10)$, $\blacktriangle$, $SU(12)$, $\vartriangle$.
  Solid and dashed lines are (improved) perturbative fits to $SU(8)$ and $SU(3)$
  respectively.}
\label{fig_ggINK_suN}
\end{figure}

%\begin{figure}[htb]
%\begin	{center}
%\leavevmode
%\input	{plot_ggIK_suN.tex}
%\end	{center}
%\caption{Running coupling (mean-field improved) on the lattice scale $a$, expressed in
%  units of the string tension, for
%  $SU(2)$, $\bullet$, $SU(3)$, $\circ$, $SU(4)$, $\blacksquare$,
%  $SU(5)$, $\square$, $SU(6)$, $\blacklozenge$, $SU(8)$, $\lozenge$,
%  $SU(10)$, $\blacktriangle$, $SU(12)$, $\vartriangle$.}
%%\label{fig_ggIK_suN}
%\end{figure}





%\begin{figure}[htb]
%\begin	{center}
%\leavevmode
%\input	{plot_ggNK_suN.tex}
%\end	{center}
%\caption{As in Fig.~\ref{fig_ggINK_suN} but without mean-field improvement.}
%\label{fig_ggNK_suN}
%\end{figure}


\begin{figure}[htb]
\begin	{center}
\leavevmode
% GNUPLOT: LaTeX picture with Postscript
\begingroup%
\makeatletter%
\newcommand{\GNUPLOTspecial}{%
  \@sanitize\catcode`\%=14\relax\special}%
\setlength{\unitlength}{0.0500bp}%
\begin{picture}(7200,7560)(0,0)%
  {\GNUPLOTspecial{"
%!PS-Adobe-2.0 EPSF-2.0
%%Title: plot_LamMS_N.tex
%%Creator: gnuplot 5.0 patchlevel 3
%%CreationDate: Mon Mar 22 13:34:03 2021
%%DocumentFonts: 
%%BoundingBox: 0 0 360 378
%%EndComments
%%BeginProlog
/gnudict 256 dict def
gnudict begin
%
% The following true/false flags may be edited by hand if desired.
% The unit line width and grayscale image gamma correction may also be changed.
%
/Color true def
/Blacktext true def
/Solid false def
/Dashlength 1 def
/Landscape false def
/Level1 false def
/Level3 false def
/Rounded false def
/ClipToBoundingBox false def
/SuppressPDFMark false def
/TransparentPatterns false def
/gnulinewidth 5.000 def
/userlinewidth gnulinewidth def
/Gamma 1.0 def
/BackgroundColor {-1.000 -1.000 -1.000} def
%
/vshift -66 def
/dl1 {
  10.0 Dashlength userlinewidth gnulinewidth div mul mul mul
  Rounded { currentlinewidth 0.75 mul sub dup 0 le { pop 0.01 } if } if
} def
/dl2 {
  10.0 Dashlength userlinewidth gnulinewidth div mul mul mul
  Rounded { currentlinewidth 0.75 mul add } if
} def
/hpt_ 31.5 def
/vpt_ 31.5 def
/hpt hpt_ def
/vpt vpt_ def
/doclip {
  ClipToBoundingBox {
    newpath 0 0 moveto 360 0 lineto 360 378 lineto 0 378 lineto closepath
    clip
  } if
} def
%
% Gnuplot Prolog Version 5.1 (Oct 2015)
%
%/SuppressPDFMark true def
%
/M {moveto} bind def
/L {lineto} bind def
/R {rmoveto} bind def
/V {rlineto} bind def
/N {newpath moveto} bind def
/Z {closepath} bind def
/C {setrgbcolor} bind def
/f {rlineto fill} bind def
/g {setgray} bind def
/Gshow {show} def   % May be redefined later in the file to support UTF-8
/vpt2 vpt 2 mul def
/hpt2 hpt 2 mul def
/Lshow {currentpoint stroke M 0 vshift R 
	Blacktext {gsave 0 setgray textshow grestore} {textshow} ifelse} def
/Rshow {currentpoint stroke M dup stringwidth pop neg vshift R
	Blacktext {gsave 0 setgray textshow grestore} {textshow} ifelse} def
/Cshow {currentpoint stroke M dup stringwidth pop -2 div vshift R 
	Blacktext {gsave 0 setgray textshow grestore} {textshow} ifelse} def
/UP {dup vpt_ mul /vpt exch def hpt_ mul /hpt exch def
  /hpt2 hpt 2 mul def /vpt2 vpt 2 mul def} def
/DL {Color {setrgbcolor Solid {pop []} if 0 setdash}
 {pop pop pop 0 setgray Solid {pop []} if 0 setdash} ifelse} def
/BL {stroke userlinewidth 2 mul setlinewidth
	Rounded {1 setlinejoin 1 setlinecap} if} def
/AL {stroke userlinewidth 2 div setlinewidth
	Rounded {1 setlinejoin 1 setlinecap} if} def
/UL {dup gnulinewidth mul /userlinewidth exch def
	dup 1 lt {pop 1} if 10 mul /udl exch def} def
/PL {stroke userlinewidth setlinewidth
	Rounded {1 setlinejoin 1 setlinecap} if} def
3.8 setmiterlimit
% Classic Line colors (version 5.0)
/LCw {1 1 1} def
/LCb {0 0 0} def
/LCa {0 0 0} def
/LC0 {1 0 0} def
/LC1 {0 1 0} def
/LC2 {0 0 1} def
/LC3 {1 0 1} def
/LC4 {0 1 1} def
/LC5 {1 1 0} def
/LC6 {0 0 0} def
/LC7 {1 0.3 0} def
/LC8 {0.5 0.5 0.5} def
% Default dash patterns (version 5.0)
/LTB {BL [] LCb DL} def
/LTw {PL [] 1 setgray} def
/LTb {PL [] LCb DL} def
/LTa {AL [1 udl mul 2 udl mul] 0 setdash LCa setrgbcolor} def
/LT0 {PL [] LC0 DL} def
/LT1 {PL [2 dl1 3 dl2] LC1 DL} def
/LT2 {PL [1 dl1 1.5 dl2] LC2 DL} def
/LT3 {PL [6 dl1 2 dl2 1 dl1 2 dl2] LC3 DL} def
/LT4 {PL [1 dl1 2 dl2 6 dl1 2 dl2 1 dl1 2 dl2] LC4 DL} def
/LT5 {PL [4 dl1 2 dl2] LC5 DL} def
/LT6 {PL [1.5 dl1 1.5 dl2 1.5 dl1 1.5 dl2 1.5 dl1 6 dl2] LC6 DL} def
/LT7 {PL [3 dl1 3 dl2 1 dl1 3 dl2] LC7 DL} def
/LT8 {PL [2 dl1 2 dl2 2 dl1 6 dl2] LC8 DL} def
/SL {[] 0 setdash} def
/Pnt {stroke [] 0 setdash gsave 1 setlinecap M 0 0 V stroke grestore} def
/Dia {stroke [] 0 setdash 2 copy vpt add M
  hpt neg vpt neg V hpt vpt neg V
  hpt vpt V hpt neg vpt V closepath stroke
  Pnt} def
/Pls {stroke [] 0 setdash vpt sub M 0 vpt2 V
  currentpoint stroke M
  hpt neg vpt neg R hpt2 0 V stroke
 } def
/Box {stroke [] 0 setdash 2 copy exch hpt sub exch vpt add M
  0 vpt2 neg V hpt2 0 V 0 vpt2 V
  hpt2 neg 0 V closepath stroke
  Pnt} def
/Crs {stroke [] 0 setdash exch hpt sub exch vpt add M
  hpt2 vpt2 neg V currentpoint stroke M
  hpt2 neg 0 R hpt2 vpt2 V stroke} def
/TriU {stroke [] 0 setdash 2 copy vpt 1.12 mul add M
  hpt neg vpt -1.62 mul V
  hpt 2 mul 0 V
  hpt neg vpt 1.62 mul V closepath stroke
  Pnt} def
/Star {2 copy Pls Crs} def
/BoxF {stroke [] 0 setdash exch hpt sub exch vpt add M
  0 vpt2 neg V hpt2 0 V 0 vpt2 V
  hpt2 neg 0 V closepath fill} def
/TriUF {stroke [] 0 setdash vpt 1.12 mul add M
  hpt neg vpt -1.62 mul V
  hpt 2 mul 0 V
  hpt neg vpt 1.62 mul V closepath fill} def
/TriD {stroke [] 0 setdash 2 copy vpt 1.12 mul sub M
  hpt neg vpt 1.62 mul V
  hpt 2 mul 0 V
  hpt neg vpt -1.62 mul V closepath stroke
  Pnt} def
/TriDF {stroke [] 0 setdash vpt 1.12 mul sub M
  hpt neg vpt 1.62 mul V
  hpt 2 mul 0 V
  hpt neg vpt -1.62 mul V closepath fill} def
/DiaF {stroke [] 0 setdash vpt add M
  hpt neg vpt neg V hpt vpt neg V
  hpt vpt V hpt neg vpt V closepath fill} def
/Pent {stroke [] 0 setdash 2 copy gsave
  translate 0 hpt M 4 {72 rotate 0 hpt L} repeat
  closepath stroke grestore Pnt} def
/PentF {stroke [] 0 setdash gsave
  translate 0 hpt M 4 {72 rotate 0 hpt L} repeat
  closepath fill grestore} def
/Circle {stroke [] 0 setdash 2 copy
  hpt 0 360 arc stroke Pnt} def
/CircleF {stroke [] 0 setdash hpt 0 360 arc fill} def
/C0 {BL [] 0 setdash 2 copy moveto vpt 90 450 arc} bind def
/C1 {BL [] 0 setdash 2 copy moveto
	2 copy vpt 0 90 arc closepath fill
	vpt 0 360 arc closepath} bind def
/C2 {BL [] 0 setdash 2 copy moveto
	2 copy vpt 90 180 arc closepath fill
	vpt 0 360 arc closepath} bind def
/C3 {BL [] 0 setdash 2 copy moveto
	2 copy vpt 0 180 arc closepath fill
	vpt 0 360 arc closepath} bind def
/C4 {BL [] 0 setdash 2 copy moveto
	2 copy vpt 180 270 arc closepath fill
	vpt 0 360 arc closepath} bind def
/C5 {BL [] 0 setdash 2 copy moveto
	2 copy vpt 0 90 arc
	2 copy moveto
	2 copy vpt 180 270 arc closepath fill
	vpt 0 360 arc} bind def
/C6 {BL [] 0 setdash 2 copy moveto
	2 copy vpt 90 270 arc closepath fill
	vpt 0 360 arc closepath} bind def
/C7 {BL [] 0 setdash 2 copy moveto
	2 copy vpt 0 270 arc closepath fill
	vpt 0 360 arc closepath} bind def
/C8 {BL [] 0 setdash 2 copy moveto
	2 copy vpt 270 360 arc closepath fill
	vpt 0 360 arc closepath} bind def
/C9 {BL [] 0 setdash 2 copy moveto
	2 copy vpt 270 450 arc closepath fill
	vpt 0 360 arc closepath} bind def
/C10 {BL [] 0 setdash 2 copy 2 copy moveto vpt 270 360 arc closepath fill
	2 copy moveto
	2 copy vpt 90 180 arc closepath fill
	vpt 0 360 arc closepath} bind def
/C11 {BL [] 0 setdash 2 copy moveto
	2 copy vpt 0 180 arc closepath fill
	2 copy moveto
	2 copy vpt 270 360 arc closepath fill
	vpt 0 360 arc closepath} bind def
/C12 {BL [] 0 setdash 2 copy moveto
	2 copy vpt 180 360 arc closepath fill
	vpt 0 360 arc closepath} bind def
/C13 {BL [] 0 setdash 2 copy moveto
	2 copy vpt 0 90 arc closepath fill
	2 copy moveto
	2 copy vpt 180 360 arc closepath fill
	vpt 0 360 arc closepath} bind def
/C14 {BL [] 0 setdash 2 copy moveto
	2 copy vpt 90 360 arc closepath fill
	vpt 0 360 arc} bind def
/C15 {BL [] 0 setdash 2 copy vpt 0 360 arc closepath fill
	vpt 0 360 arc closepath} bind def
/Rec {newpath 4 2 roll moveto 1 index 0 rlineto 0 exch rlineto
	neg 0 rlineto closepath} bind def
/Square {dup Rec} bind def
/Bsquare {vpt sub exch vpt sub exch vpt2 Square} bind def
/S0 {BL [] 0 setdash 2 copy moveto 0 vpt rlineto BL Bsquare} bind def
/S1 {BL [] 0 setdash 2 copy vpt Square fill Bsquare} bind def
/S2 {BL [] 0 setdash 2 copy exch vpt sub exch vpt Square fill Bsquare} bind def
/S3 {BL [] 0 setdash 2 copy exch vpt sub exch vpt2 vpt Rec fill Bsquare} bind def
/S4 {BL [] 0 setdash 2 copy exch vpt sub exch vpt sub vpt Square fill Bsquare} bind def
/S5 {BL [] 0 setdash 2 copy 2 copy vpt Square fill
	exch vpt sub exch vpt sub vpt Square fill Bsquare} bind def
/S6 {BL [] 0 setdash 2 copy exch vpt sub exch vpt sub vpt vpt2 Rec fill Bsquare} bind def
/S7 {BL [] 0 setdash 2 copy exch vpt sub exch vpt sub vpt vpt2 Rec fill
	2 copy vpt Square fill Bsquare} bind def
/S8 {BL [] 0 setdash 2 copy vpt sub vpt Square fill Bsquare} bind def
/S9 {BL [] 0 setdash 2 copy vpt sub vpt vpt2 Rec fill Bsquare} bind def
/S10 {BL [] 0 setdash 2 copy vpt sub vpt Square fill 2 copy exch vpt sub exch vpt Square fill
	Bsquare} bind def
/S11 {BL [] 0 setdash 2 copy vpt sub vpt Square fill 2 copy exch vpt sub exch vpt2 vpt Rec fill
	Bsquare} bind def
/S12 {BL [] 0 setdash 2 copy exch vpt sub exch vpt sub vpt2 vpt Rec fill Bsquare} bind def
/S13 {BL [] 0 setdash 2 copy exch vpt sub exch vpt sub vpt2 vpt Rec fill
	2 copy vpt Square fill Bsquare} bind def
/S14 {BL [] 0 setdash 2 copy exch vpt sub exch vpt sub vpt2 vpt Rec fill
	2 copy exch vpt sub exch vpt Square fill Bsquare} bind def
/S15 {BL [] 0 setdash 2 copy Bsquare fill Bsquare} bind def
/D0 {gsave translate 45 rotate 0 0 S0 stroke grestore} bind def
/D1 {gsave translate 45 rotate 0 0 S1 stroke grestore} bind def
/D2 {gsave translate 45 rotate 0 0 S2 stroke grestore} bind def
/D3 {gsave translate 45 rotate 0 0 S3 stroke grestore} bind def
/D4 {gsave translate 45 rotate 0 0 S4 stroke grestore} bind def
/D5 {gsave translate 45 rotate 0 0 S5 stroke grestore} bind def
/D6 {gsave translate 45 rotate 0 0 S6 stroke grestore} bind def
/D7 {gsave translate 45 rotate 0 0 S7 stroke grestore} bind def
/D8 {gsave translate 45 rotate 0 0 S8 stroke grestore} bind def
/D9 {gsave translate 45 rotate 0 0 S9 stroke grestore} bind def
/D10 {gsave translate 45 rotate 0 0 S10 stroke grestore} bind def
/D11 {gsave translate 45 rotate 0 0 S11 stroke grestore} bind def
/D12 {gsave translate 45 rotate 0 0 S12 stroke grestore} bind def
/D13 {gsave translate 45 rotate 0 0 S13 stroke grestore} bind def
/D14 {gsave translate 45 rotate 0 0 S14 stroke grestore} bind def
/D15 {gsave translate 45 rotate 0 0 S15 stroke grestore} bind def
/DiaE {stroke [] 0 setdash vpt add M
  hpt neg vpt neg V hpt vpt neg V
  hpt vpt V hpt neg vpt V closepath stroke} def
/BoxE {stroke [] 0 setdash exch hpt sub exch vpt add M
  0 vpt2 neg V hpt2 0 V 0 vpt2 V
  hpt2 neg 0 V closepath stroke} def
/TriUE {stroke [] 0 setdash vpt 1.12 mul add M
  hpt neg vpt -1.62 mul V
  hpt 2 mul 0 V
  hpt neg vpt 1.62 mul V closepath stroke} def
/TriDE {stroke [] 0 setdash vpt 1.12 mul sub M
  hpt neg vpt 1.62 mul V
  hpt 2 mul 0 V
  hpt neg vpt -1.62 mul V closepath stroke} def
/PentE {stroke [] 0 setdash gsave
  translate 0 hpt M 4 {72 rotate 0 hpt L} repeat
  closepath stroke grestore} def
/CircE {stroke [] 0 setdash 
  hpt 0 360 arc stroke} def
/Opaque {gsave closepath 1 setgray fill grestore 0 setgray closepath} def
/DiaW {stroke [] 0 setdash vpt add M
  hpt neg vpt neg V hpt vpt neg V
  hpt vpt V hpt neg vpt V Opaque stroke} def
/BoxW {stroke [] 0 setdash exch hpt sub exch vpt add M
  0 vpt2 neg V hpt2 0 V 0 vpt2 V
  hpt2 neg 0 V Opaque stroke} def
/TriUW {stroke [] 0 setdash vpt 1.12 mul add M
  hpt neg vpt -1.62 mul V
  hpt 2 mul 0 V
  hpt neg vpt 1.62 mul V Opaque stroke} def
/TriDW {stroke [] 0 setdash vpt 1.12 mul sub M
  hpt neg vpt 1.62 mul V
  hpt 2 mul 0 V
  hpt neg vpt -1.62 mul V Opaque stroke} def
/PentW {stroke [] 0 setdash gsave
  translate 0 hpt M 4 {72 rotate 0 hpt L} repeat
  Opaque stroke grestore} def
/CircW {stroke [] 0 setdash 
  hpt 0 360 arc Opaque stroke} def
/BoxFill {gsave Rec 1 setgray fill grestore} def
/Density {
  /Fillden exch def
  currentrgbcolor
  /ColB exch def /ColG exch def /ColR exch def
  /ColR ColR Fillden mul Fillden sub 1 add def
  /ColG ColG Fillden mul Fillden sub 1 add def
  /ColB ColB Fillden mul Fillden sub 1 add def
  ColR ColG ColB setrgbcolor} def
/BoxColFill {gsave Rec PolyFill} def
/PolyFill {gsave Density fill grestore grestore} def
/h {rlineto rlineto rlineto gsave closepath fill grestore} bind def
%
% PostScript Level 1 Pattern Fill routine for rectangles
% Usage: x y w h s a XX PatternFill
%	x,y = lower left corner of box to be filled
%	w,h = width and height of box
%	  a = angle in degrees between lines and x-axis
%	 XX = 0/1 for no/yes cross-hatch
%
/PatternFill {gsave /PFa [ 9 2 roll ] def
  PFa 0 get PFa 2 get 2 div add PFa 1 get PFa 3 get 2 div add translate
  PFa 2 get -2 div PFa 3 get -2 div PFa 2 get PFa 3 get Rec
  TransparentPatterns {} {gsave 1 setgray fill grestore} ifelse
  clip
  currentlinewidth 0.5 mul setlinewidth
  /PFs PFa 2 get dup mul PFa 3 get dup mul add sqrt def
  0 0 M PFa 5 get rotate PFs -2 div dup translate
  0 1 PFs PFa 4 get div 1 add floor cvi
	{PFa 4 get mul 0 M 0 PFs V} for
  0 PFa 6 get ne {
	0 1 PFs PFa 4 get div 1 add floor cvi
	{PFa 4 get mul 0 2 1 roll M PFs 0 V} for
 } if
  stroke grestore} def
%
/languagelevel where
 {pop languagelevel} {1} ifelse
dup 2 lt
	{/InterpretLevel1 true def
	 /InterpretLevel3 false def}
	{/InterpretLevel1 Level1 def
	 2 gt
	    {/InterpretLevel3 Level3 def}
	    {/InterpretLevel3 false def}
	 ifelse }
 ifelse
%
% PostScript level 2 pattern fill definitions
%
/Level2PatternFill {
/Tile8x8 {/PaintType 2 /PatternType 1 /TilingType 1 /BBox [0 0 8 8] /XStep 8 /YStep 8}
	bind def
/KeepColor {currentrgbcolor [/Pattern /DeviceRGB] setcolorspace} bind def
<< Tile8x8
 /PaintProc {0.5 setlinewidth pop 0 0 M 8 8 L 0 8 M 8 0 L stroke} 
>> matrix makepattern
/Pat1 exch def
<< Tile8x8
 /PaintProc {0.5 setlinewidth pop 0 0 M 8 8 L 0 8 M 8 0 L stroke
	0 4 M 4 8 L 8 4 L 4 0 L 0 4 L stroke}
>> matrix makepattern
/Pat2 exch def
<< Tile8x8
 /PaintProc {0.5 setlinewidth pop 0 0 M 0 8 L
	8 8 L 8 0 L 0 0 L fill}
>> matrix makepattern
/Pat3 exch def
<< Tile8x8
 /PaintProc {0.5 setlinewidth pop -4 8 M 8 -4 L
	0 12 M 12 0 L stroke}
>> matrix makepattern
/Pat4 exch def
<< Tile8x8
 /PaintProc {0.5 setlinewidth pop -4 0 M 8 12 L
	0 -4 M 12 8 L stroke}
>> matrix makepattern
/Pat5 exch def
<< Tile8x8
 /PaintProc {0.5 setlinewidth pop -2 8 M 4 -4 L
	0 12 M 8 -4 L 4 12 M 10 0 L stroke}
>> matrix makepattern
/Pat6 exch def
<< Tile8x8
 /PaintProc {0.5 setlinewidth pop -2 0 M 4 12 L
	0 -4 M 8 12 L 4 -4 M 10 8 L stroke}
>> matrix makepattern
/Pat7 exch def
<< Tile8x8
 /PaintProc {0.5 setlinewidth pop 8 -2 M -4 4 L
	12 0 M -4 8 L 12 4 M 0 10 L stroke}
>> matrix makepattern
/Pat8 exch def
<< Tile8x8
 /PaintProc {0.5 setlinewidth pop 0 -2 M 12 4 L
	-4 0 M 12 8 L -4 4 M 8 10 L stroke}
>> matrix makepattern
/Pat9 exch def
/Pattern1 {PatternBgnd KeepColor Pat1 setpattern} bind def
/Pattern2 {PatternBgnd KeepColor Pat2 setpattern} bind def
/Pattern3 {PatternBgnd KeepColor Pat3 setpattern} bind def
/Pattern4 {PatternBgnd KeepColor Landscape {Pat5} {Pat4} ifelse setpattern} bind def
/Pattern5 {PatternBgnd KeepColor Landscape {Pat4} {Pat5} ifelse setpattern} bind def
/Pattern6 {PatternBgnd KeepColor Landscape {Pat9} {Pat6} ifelse setpattern} bind def
/Pattern7 {PatternBgnd KeepColor Landscape {Pat8} {Pat7} ifelse setpattern} bind def
} def
%
%
%End of PostScript Level 2 code
%
/PatternBgnd {
  TransparentPatterns {} {gsave 1 setgray fill grestore} ifelse
} def
%
% Substitute for Level 2 pattern fill codes with
% grayscale if Level 2 support is not selected.
%
/Level1PatternFill {
/Pattern1 {0.250 Density} bind def
/Pattern2 {0.500 Density} bind def
/Pattern3 {0.750 Density} bind def
/Pattern4 {0.125 Density} bind def
/Pattern5 {0.375 Density} bind def
/Pattern6 {0.625 Density} bind def
/Pattern7 {0.875 Density} bind def
} def
%
% Now test for support of Level 2 code
%
Level1 {Level1PatternFill} {Level2PatternFill} ifelse
%
/Symbol-Oblique /Symbol findfont [1 0 .167 1 0 0] makefont
dup length dict begin {1 index /FID eq {pop pop} {def} ifelse} forall
currentdict end definefont pop
%
Level1 SuppressPDFMark or 
{} {
/SDict 10 dict def
systemdict /pdfmark known not {
  userdict /pdfmark systemdict /cleartomark get put
} if
SDict begin [
  /Title (plot_LamMS_N.tex)
  /Subject (gnuplot plot)
  /Creator (gnuplot 5.0 patchlevel 3)
  /Author (mteper)
%  /Producer (gnuplot)
%  /Keywords ()
  /CreationDate (Mon Mar 22 13:34:03 2021)
  /DOCINFO pdfmark
end
} ifelse
%
% Support for boxed text - Ethan A Merritt May 2005
%
/InitTextBox { userdict /TBy2 3 -1 roll put userdict /TBx2 3 -1 roll put
           userdict /TBy1 3 -1 roll put userdict /TBx1 3 -1 roll put
	   /Boxing true def } def
/ExtendTextBox { Boxing
    { gsave dup false charpath pathbbox
      dup TBy2 gt {userdict /TBy2 3 -1 roll put} {pop} ifelse
      dup TBx2 gt {userdict /TBx2 3 -1 roll put} {pop} ifelse
      dup TBy1 lt {userdict /TBy1 3 -1 roll put} {pop} ifelse
      dup TBx1 lt {userdict /TBx1 3 -1 roll put} {pop} ifelse
      grestore } if } def
/PopTextBox { newpath TBx1 TBxmargin sub TBy1 TBymargin sub M
               TBx1 TBxmargin sub TBy2 TBymargin add L
	       TBx2 TBxmargin add TBy2 TBymargin add L
	       TBx2 TBxmargin add TBy1 TBymargin sub L closepath } def
/DrawTextBox { PopTextBox stroke /Boxing false def} def
/FillTextBox { gsave PopTextBox 1 1 1 setrgbcolor fill grestore /Boxing false def} def
0 0 0 0 InitTextBox
/TBxmargin 20 def
/TBymargin 20 def
/Boxing false def
/textshow { ExtendTextBox Gshow } def
%
% redundant definitions for compatibility with prologue.ps older than 5.0.2
/LTB {BL [] LCb DL} def
/LTb {PL [] LCb DL} def
end
%%EndProlog
%%Page: 1 1
gnudict begin
gsave
doclip
0 0 translate
0.050 0.050 scale
0 setgray
newpath
BackgroundColor 0 lt 3 1 roll 0 lt exch 0 lt or or not {BackgroundColor C 1.000 0 0 7200.00 7560.00 BoxColFill} if
1.000 UL
LTb
LCb setrgbcolor
1260 640 M
63 0 V
5516 0 R
-63 0 V
stroke
LTb
LCb setrgbcolor
1260 1976 M
63 0 V
5516 0 R
-63 0 V
stroke
LTb
LCb setrgbcolor
1260 3312 M
63 0 V
5516 0 R
-63 0 V
stroke
LTb
LCb setrgbcolor
1260 4647 M
63 0 V
5516 0 R
-63 0 V
stroke
LTb
LCb setrgbcolor
1260 5983 M
63 0 V
5516 0 R
-63 0 V
stroke
LTb
LCb setrgbcolor
1260 7319 M
63 0 V
5516 0 R
-63 0 V
stroke
LTb
LCb setrgbcolor
1260 640 M
0 63 V
0 6616 R
0 -63 V
stroke
LTb
LCb setrgbcolor
2190 640 M
0 63 V
0 6616 R
0 -63 V
stroke
LTb
LCb setrgbcolor
3120 640 M
0 63 V
0 6616 R
0 -63 V
stroke
LTb
LCb setrgbcolor
4050 640 M
0 63 V
0 6616 R
0 -63 V
stroke
LTb
LCb setrgbcolor
4979 640 M
0 63 V
0 6616 R
0 -63 V
stroke
LTb
LCb setrgbcolor
5909 640 M
0 63 V
0 6616 R
0 -63 V
stroke
LTb
LCb setrgbcolor
6839 640 M
0 63 V
0 6616 R
0 -63 V
stroke
LTb
LCb setrgbcolor
1.000 UL
LTb
LCb setrgbcolor
1260 7319 N
0 -6679 V
5579 0 V
0 6679 V
-5579 0 V
Z stroke
1.000 UP
1.000 UL
LTb
LCb setrgbcolor
LCb setrgbcolor
LTb
LCb setrgbcolor
LTb
1.500 UP
1.000 UL
LTb
0.58 0.00 0.83 C 5909 5409 M
0 112 V
3326 4410 M
0 69 V
2422 3875 M
0 59 V
2004 3736 M
0 81 V
1777 3704 M
0 59 V
1550 3573 M
0 91 V
1446 3322 M
0 214 V
-57 -69 R
0 69 V
5909 5465 CircleF
3326 4444 CircleF
2422 3905 CircleF
2004 3776 CircleF
1777 3734 CircleF
1550 3619 CircleF
1446 3429 CircleF
1389 3501 CircleF
1.500 UL
LTb
0.58 0.00 0.83 C 1260 3459 M
56 24 V
57 25 V
56 25 V
56 25 V
57 24 V
56 25 V
56 25 V
57 25 V
56 24 V
57 25 V
56 25 V
56 25 V
57 25 V
56 24 V
56 25 V
57 25 V
56 25 V
56 24 V
57 25 V
56 25 V
56 25 V
57 25 V
56 24 V
56 25 V
57 25 V
56 25 V
57 24 V
56 25 V
56 25 V
57 25 V
56 25 V
56 24 V
57 25 V
56 25 V
56 25 V
57 24 V
56 25 V
56 25 V
57 25 V
56 24 V
56 25 V
57 25 V
56 25 V
57 25 V
56 24 V
56 25 V
57 25 V
56 25 V
56 24 V
57 25 V
56 25 V
56 25 V
57 25 V
56 24 V
56 25 V
57 25 V
56 25 V
57 24 V
56 25 V
56 25 V
57 25 V
56 24 V
56 25 V
57 25 V
56 25 V
56 25 V
57 24 V
56 25 V
56 25 V
57 25 V
56 24 V
56 25 V
57 25 V
56 25 V
57 25 V
56 24 V
56 25 V
57 25 V
56 25 V
56 24 V
57 25 V
56 25 V
56 25 V
57 24 V
56 25 V
56 25 V
57 25 V
56 25 V
56 24 V
57 25 V
56 25 V
57 25 V
56 24 V
56 25 V
57 25 V
56 25 V
56 25 V
57 24 V
56 25 V
stroke
2.000 UL
LTb
LCb setrgbcolor
1.000 UL
LTb
LCb setrgbcolor
1260 7319 N
0 -6679 V
5579 0 V
0 6679 V
-5579 0 V
Z stroke
1.000 UP
1.000 UL
LTb
LCb setrgbcolor
stroke
grestore
end
showpage
  }}%
  \put(4049,140){\makebox(0,0){\large{$1/N^2$}}}%
  \put(200,4979){\makebox(0,0){\Large{$\frac{\Lambda_{\overline{MS}}}{\surd\sigma}$}}}%
  \put(6839,440){\makebox(0,0){\strut{}\ {$0.3$}}}%
  \put(5909,440){\makebox(0,0){\strut{}\ {$0.25$}}}%
  \put(4979,440){\makebox(0,0){\strut{}\ {$0.2$}}}%
  \put(4050,440){\makebox(0,0){\strut{}\ {$0.15$}}}%
  \put(3120,440){\makebox(0,0){\strut{}\ {$0.1$}}}%
  \put(2190,440){\makebox(0,0){\strut{}\ {$0.05$}}}%
  \put(1260,440){\makebox(0,0){\strut{}\ {$0$}}}%
  \put(1140,7319){\makebox(0,0)[r]{\strut{}\ \ {$0.65$}}}%
  \put(1140,5983){\makebox(0,0)[r]{\strut{}\ \ {$0.6$}}}%
  \put(1140,4647){\makebox(0,0)[r]{\strut{}\ \ {$0.55$}}}%
  \put(1140,3312){\makebox(0,0)[r]{\strut{}\ \ {$0.5$}}}%
  \put(1140,1976){\makebox(0,0)[r]{\strut{}\ \ {$0.45$}}}%
  \put(1140,640){\makebox(0,0)[r]{\strut{}\ \ {$0.4$}}}%
\end{picture}%
\endgroup
\endinput

\end	{center}
\caption{Values of the scale parameter $\Lambda_{\overline{MS}}$
  in units of the string tension in our $SU(N)$ gauge theories.}
\label{fig_LamMS_N}
\end{figure}



\clearpage



%\begin{figure}[htb]
%\begin	{center}
%\leavevmode
%\input	{plot_MeffA1_SU8.tex}
%\end	{center}
%\caption{Effective masses for the lightest three $A_1^{++}$ ($\bullet$) and the lightest
%  two $A_1^{-+}$ ($\circ$) glueball states, as well as the main $A_1^{++}$ ditorelon state ($\ast$).
%  Lines are mass estimates. All on a $20^330$ lattice at $\beta=47.75$ in $SU(8)$.
%  In the continuum limit the lightest two glueball states in each sector become the lightest
%  two $J^{PC}=0^{++}$ and $J^{PC}=0^{-+}$ glueballs. The ditorelon disappears in the thermodynamic
%  limit.}
%\label{fig_MeffA1_SU8}
%\end{figure}


\begin{figure}[htb]
\begin	{center}
\leavevmode
% GNUPLOT: LaTeX picture with Postscript
\begingroup%
\makeatletter%
\newcommand{\GNUPLOTspecial}{%
  \@sanitize\catcode`\%=14\relax\special}%
\setlength{\unitlength}{0.0500bp}%
\begin{picture}(7200,7560)(0,0)%
  {\GNUPLOTspecial{"
%!PS-Adobe-2.0 EPSF-2.0
%%Title: plot_MeffA1_SU8.tex
%%Creator: gnuplot 5.2 patchlevel 8
%%CreationDate: Sun Oct 31 19:07:00 2021
%%DocumentFonts: 
%%BoundingBox: 0 0 360 378
%%EndComments
%%BeginProlog
/gnudict 256 dict def
gnudict begin
%
% The following true/false flags may be edited by hand if desired.
% The unit line width and grayscale image gamma correction may also be changed.
%
/Color true def
/Blacktext true def
/Solid false def
/Dashlength 1 def
/Landscape false def
/Level1 false def
/Level3 false def
/Rounded false def
/ClipToBoundingBox false def
/SuppressPDFMark false def
/TransparentPatterns false def
/gnulinewidth 5.000 def
/userlinewidth gnulinewidth def
/Gamma 1.0 def
/BackgroundColor {-1.000 -1.000 -1.000} def
%
/vshift -93 def
/dl1 {
  10.0 Dashlength userlinewidth gnulinewidth div mul mul mul
  Rounded { currentlinewidth 0.75 mul sub dup 0 le { pop 0.01 } if } if
} def
/dl2 {
  10.0 Dashlength userlinewidth gnulinewidth div mul mul mul
  Rounded { currentlinewidth 0.75 mul add } if
} def
/hpt_ 31.5 def
/vpt_ 31.5 def
/hpt hpt_ def
/vpt vpt_ def
/doclip {
  ClipToBoundingBox {
    newpath 0 0 moveto 360 0 lineto 360 378 lineto 0 378 lineto closepath
    clip
  } if
} def
%
% Gnuplot Prolog Version 5.2 (Dec 2017)
%
%/SuppressPDFMark true def
%
/M {moveto} bind def
/L {lineto} bind def
/R {rmoveto} bind def
/V {rlineto} bind def
/N {newpath moveto} bind def
/Z {closepath} bind def
/C {setrgbcolor} bind def
/f {rlineto fill} bind def
/g {setgray} bind def
/Gshow {show} def   % May be redefined later in the file to support UTF-8
/vpt2 vpt 2 mul def
/hpt2 hpt 2 mul def
/Lshow {currentpoint stroke M 0 vshift R 
	Blacktext {gsave 0 setgray textshow grestore} {textshow} ifelse} def
/Rshow {currentpoint stroke M dup stringwidth pop neg vshift R
	Blacktext {gsave 0 setgray textshow grestore} {textshow} ifelse} def
/Cshow {currentpoint stroke M dup stringwidth pop -2 div vshift R 
	Blacktext {gsave 0 setgray textshow grestore} {textshow} ifelse} def
/UP {dup vpt_ mul /vpt exch def hpt_ mul /hpt exch def
  /hpt2 hpt 2 mul def /vpt2 vpt 2 mul def} def
/DL {Color {setrgbcolor Solid {pop []} if 0 setdash}
 {pop pop pop 0 setgray Solid {pop []} if 0 setdash} ifelse} def
/BL {stroke userlinewidth 2 mul setlinewidth
	Rounded {1 setlinejoin 1 setlinecap} if} def
/AL {stroke userlinewidth 2 div setlinewidth
	Rounded {1 setlinejoin 1 setlinecap} if} def
/UL {dup gnulinewidth mul /userlinewidth exch def
	dup 1 lt {pop 1} if 10 mul /udl exch def} def
/PL {stroke userlinewidth setlinewidth
	Rounded {1 setlinejoin 1 setlinecap} if} def
3.8 setmiterlimit
% Classic Line colors (version 5.0)
/LCw {1 1 1} def
/LCb {0 0 0} def
/LCa {0 0 0} def
/LC0 {1 0 0} def
/LC1 {0 1 0} def
/LC2 {0 0 1} def
/LC3 {1 0 1} def
/LC4 {0 1 1} def
/LC5 {1 1 0} def
/LC6 {0 0 0} def
/LC7 {1 0.3 0} def
/LC8 {0.5 0.5 0.5} def
% Default dash patterns (version 5.0)
/LTB {BL [] LCb DL} def
/LTw {PL [] 1 setgray} def
/LTb {PL [] LCb DL} def
/LTa {AL [1 udl mul 2 udl mul] 0 setdash LCa setrgbcolor} def
/LT0 {PL [] LC0 DL} def
/LT1 {PL [2 dl1 3 dl2] LC1 DL} def
/LT2 {PL [1 dl1 1.5 dl2] LC2 DL} def
/LT3 {PL [6 dl1 2 dl2 1 dl1 2 dl2] LC3 DL} def
/LT4 {PL [1 dl1 2 dl2 6 dl1 2 dl2 1 dl1 2 dl2] LC4 DL} def
/LT5 {PL [4 dl1 2 dl2] LC5 DL} def
/LT6 {PL [1.5 dl1 1.5 dl2 1.5 dl1 1.5 dl2 1.5 dl1 6 dl2] LC6 DL} def
/LT7 {PL [3 dl1 3 dl2 1 dl1 3 dl2] LC7 DL} def
/LT8 {PL [2 dl1 2 dl2 2 dl1 6 dl2] LC8 DL} def
/SL {[] 0 setdash} def
/Pnt {stroke [] 0 setdash gsave 1 setlinecap M 0 0 V stroke grestore} def
/Dia {stroke [] 0 setdash 2 copy vpt add M
  hpt neg vpt neg V hpt vpt neg V
  hpt vpt V hpt neg vpt V closepath stroke
  Pnt} def
/Pls {stroke [] 0 setdash vpt sub M 0 vpt2 V
  currentpoint stroke M
  hpt neg vpt neg R hpt2 0 V stroke
 } def
/Box {stroke [] 0 setdash 2 copy exch hpt sub exch vpt add M
  0 vpt2 neg V hpt2 0 V 0 vpt2 V
  hpt2 neg 0 V closepath stroke
  Pnt} def
/Crs {stroke [] 0 setdash exch hpt sub exch vpt add M
  hpt2 vpt2 neg V currentpoint stroke M
  hpt2 neg 0 R hpt2 vpt2 V stroke} def
/TriU {stroke [] 0 setdash 2 copy vpt 1.12 mul add M
  hpt neg vpt -1.62 mul V
  hpt 2 mul 0 V
  hpt neg vpt 1.62 mul V closepath stroke
  Pnt} def
/Star {2 copy Pls Crs} def
/BoxF {stroke [] 0 setdash exch hpt sub exch vpt add M
  0 vpt2 neg V hpt2 0 V 0 vpt2 V
  hpt2 neg 0 V closepath fill} def
/TriUF {stroke [] 0 setdash vpt 1.12 mul add M
  hpt neg vpt -1.62 mul V
  hpt 2 mul 0 V
  hpt neg vpt 1.62 mul V closepath fill} def
/TriD {stroke [] 0 setdash 2 copy vpt 1.12 mul sub M
  hpt neg vpt 1.62 mul V
  hpt 2 mul 0 V
  hpt neg vpt -1.62 mul V closepath stroke
  Pnt} def
/TriDF {stroke [] 0 setdash vpt 1.12 mul sub M
  hpt neg vpt 1.62 mul V
  hpt 2 mul 0 V
  hpt neg vpt -1.62 mul V closepath fill} def
/DiaF {stroke [] 0 setdash vpt add M
  hpt neg vpt neg V hpt vpt neg V
  hpt vpt V hpt neg vpt V closepath fill} def
/Pent {stroke [] 0 setdash 2 copy gsave
  translate 0 hpt M 4 {72 rotate 0 hpt L} repeat
  closepath stroke grestore Pnt} def
/PentF {stroke [] 0 setdash gsave
  translate 0 hpt M 4 {72 rotate 0 hpt L} repeat
  closepath fill grestore} def
/Circle {stroke [] 0 setdash 2 copy
  hpt 0 360 arc stroke Pnt} def
/CircleF {stroke [] 0 setdash hpt 0 360 arc fill} def
/C0 {BL [] 0 setdash 2 copy moveto vpt 90 450 arc} bind def
/C1 {BL [] 0 setdash 2 copy moveto
	2 copy vpt 0 90 arc closepath fill
	vpt 0 360 arc closepath} bind def
/C2 {BL [] 0 setdash 2 copy moveto
	2 copy vpt 90 180 arc closepath fill
	vpt 0 360 arc closepath} bind def
/C3 {BL [] 0 setdash 2 copy moveto
	2 copy vpt 0 180 arc closepath fill
	vpt 0 360 arc closepath} bind def
/C4 {BL [] 0 setdash 2 copy moveto
	2 copy vpt 180 270 arc closepath fill
	vpt 0 360 arc closepath} bind def
/C5 {BL [] 0 setdash 2 copy moveto
	2 copy vpt 0 90 arc
	2 copy moveto
	2 copy vpt 180 270 arc closepath fill
	vpt 0 360 arc} bind def
/C6 {BL [] 0 setdash 2 copy moveto
	2 copy vpt 90 270 arc closepath fill
	vpt 0 360 arc closepath} bind def
/C7 {BL [] 0 setdash 2 copy moveto
	2 copy vpt 0 270 arc closepath fill
	vpt 0 360 arc closepath} bind def
/C8 {BL [] 0 setdash 2 copy moveto
	2 copy vpt 270 360 arc closepath fill
	vpt 0 360 arc closepath} bind def
/C9 {BL [] 0 setdash 2 copy moveto
	2 copy vpt 270 450 arc closepath fill
	vpt 0 360 arc closepath} bind def
/C10 {BL [] 0 setdash 2 copy 2 copy moveto vpt 270 360 arc closepath fill
	2 copy moveto
	2 copy vpt 90 180 arc closepath fill
	vpt 0 360 arc closepath} bind def
/C11 {BL [] 0 setdash 2 copy moveto
	2 copy vpt 0 180 arc closepath fill
	2 copy moveto
	2 copy vpt 270 360 arc closepath fill
	vpt 0 360 arc closepath} bind def
/C12 {BL [] 0 setdash 2 copy moveto
	2 copy vpt 180 360 arc closepath fill
	vpt 0 360 arc closepath} bind def
/C13 {BL [] 0 setdash 2 copy moveto
	2 copy vpt 0 90 arc closepath fill
	2 copy moveto
	2 copy vpt 180 360 arc closepath fill
	vpt 0 360 arc closepath} bind def
/C14 {BL [] 0 setdash 2 copy moveto
	2 copy vpt 90 360 arc closepath fill
	vpt 0 360 arc} bind def
/C15 {BL [] 0 setdash 2 copy vpt 0 360 arc closepath fill
	vpt 0 360 arc closepath} bind def
/Rec {newpath 4 2 roll moveto 1 index 0 rlineto 0 exch rlineto
	neg 0 rlineto closepath} bind def
/Square {dup Rec} bind def
/Bsquare {vpt sub exch vpt sub exch vpt2 Square} bind def
/S0 {BL [] 0 setdash 2 copy moveto 0 vpt rlineto BL Bsquare} bind def
/S1 {BL [] 0 setdash 2 copy vpt Square fill Bsquare} bind def
/S2 {BL [] 0 setdash 2 copy exch vpt sub exch vpt Square fill Bsquare} bind def
/S3 {BL [] 0 setdash 2 copy exch vpt sub exch vpt2 vpt Rec fill Bsquare} bind def
/S4 {BL [] 0 setdash 2 copy exch vpt sub exch vpt sub vpt Square fill Bsquare} bind def
/S5 {BL [] 0 setdash 2 copy 2 copy vpt Square fill
	exch vpt sub exch vpt sub vpt Square fill Bsquare} bind def
/S6 {BL [] 0 setdash 2 copy exch vpt sub exch vpt sub vpt vpt2 Rec fill Bsquare} bind def
/S7 {BL [] 0 setdash 2 copy exch vpt sub exch vpt sub vpt vpt2 Rec fill
	2 copy vpt Square fill Bsquare} bind def
/S8 {BL [] 0 setdash 2 copy vpt sub vpt Square fill Bsquare} bind def
/S9 {BL [] 0 setdash 2 copy vpt sub vpt vpt2 Rec fill Bsquare} bind def
/S10 {BL [] 0 setdash 2 copy vpt sub vpt Square fill 2 copy exch vpt sub exch vpt Square fill
	Bsquare} bind def
/S11 {BL [] 0 setdash 2 copy vpt sub vpt Square fill 2 copy exch vpt sub exch vpt2 vpt Rec fill
	Bsquare} bind def
/S12 {BL [] 0 setdash 2 copy exch vpt sub exch vpt sub vpt2 vpt Rec fill Bsquare} bind def
/S13 {BL [] 0 setdash 2 copy exch vpt sub exch vpt sub vpt2 vpt Rec fill
	2 copy vpt Square fill Bsquare} bind def
/S14 {BL [] 0 setdash 2 copy exch vpt sub exch vpt sub vpt2 vpt Rec fill
	2 copy exch vpt sub exch vpt Square fill Bsquare} bind def
/S15 {BL [] 0 setdash 2 copy Bsquare fill Bsquare} bind def
/D0 {gsave translate 45 rotate 0 0 S0 stroke grestore} bind def
/D1 {gsave translate 45 rotate 0 0 S1 stroke grestore} bind def
/D2 {gsave translate 45 rotate 0 0 S2 stroke grestore} bind def
/D3 {gsave translate 45 rotate 0 0 S3 stroke grestore} bind def
/D4 {gsave translate 45 rotate 0 0 S4 stroke grestore} bind def
/D5 {gsave translate 45 rotate 0 0 S5 stroke grestore} bind def
/D6 {gsave translate 45 rotate 0 0 S6 stroke grestore} bind def
/D7 {gsave translate 45 rotate 0 0 S7 stroke grestore} bind def
/D8 {gsave translate 45 rotate 0 0 S8 stroke grestore} bind def
/D9 {gsave translate 45 rotate 0 0 S9 stroke grestore} bind def
/D10 {gsave translate 45 rotate 0 0 S10 stroke grestore} bind def
/D11 {gsave translate 45 rotate 0 0 S11 stroke grestore} bind def
/D12 {gsave translate 45 rotate 0 0 S12 stroke grestore} bind def
/D13 {gsave translate 45 rotate 0 0 S13 stroke grestore} bind def
/D14 {gsave translate 45 rotate 0 0 S14 stroke grestore} bind def
/D15 {gsave translate 45 rotate 0 0 S15 stroke grestore} bind def
/DiaE {stroke [] 0 setdash vpt add M
  hpt neg vpt neg V hpt vpt neg V
  hpt vpt V hpt neg vpt V closepath stroke} def
/BoxE {stroke [] 0 setdash exch hpt sub exch vpt add M
  0 vpt2 neg V hpt2 0 V 0 vpt2 V
  hpt2 neg 0 V closepath stroke} def
/TriUE {stroke [] 0 setdash vpt 1.12 mul add M
  hpt neg vpt -1.62 mul V
  hpt 2 mul 0 V
  hpt neg vpt 1.62 mul V closepath stroke} def
/TriDE {stroke [] 0 setdash vpt 1.12 mul sub M
  hpt neg vpt 1.62 mul V
  hpt 2 mul 0 V
  hpt neg vpt -1.62 mul V closepath stroke} def
/PentE {stroke [] 0 setdash gsave
  translate 0 hpt M 4 {72 rotate 0 hpt L} repeat
  closepath stroke grestore} def
/CircE {stroke [] 0 setdash 
  hpt 0 360 arc stroke} def
/Opaque {gsave closepath 1 setgray fill grestore 0 setgray closepath} def
/DiaW {stroke [] 0 setdash vpt add M
  hpt neg vpt neg V hpt vpt neg V
  hpt vpt V hpt neg vpt V Opaque stroke} def
/BoxW {stroke [] 0 setdash exch hpt sub exch vpt add M
  0 vpt2 neg V hpt2 0 V 0 vpt2 V
  hpt2 neg 0 V Opaque stroke} def
/TriUW {stroke [] 0 setdash vpt 1.12 mul add M
  hpt neg vpt -1.62 mul V
  hpt 2 mul 0 V
  hpt neg vpt 1.62 mul V Opaque stroke} def
/TriDW {stroke [] 0 setdash vpt 1.12 mul sub M
  hpt neg vpt 1.62 mul V
  hpt 2 mul 0 V
  hpt neg vpt -1.62 mul V Opaque stroke} def
/PentW {stroke [] 0 setdash gsave
  translate 0 hpt M 4 {72 rotate 0 hpt L} repeat
  Opaque stroke grestore} def
/CircW {stroke [] 0 setdash 
  hpt 0 360 arc Opaque stroke} def
/BoxFill {gsave Rec 1 setgray fill grestore} def
/Density {
  /Fillden exch def
  currentrgbcolor
  /ColB exch def /ColG exch def /ColR exch def
  /ColR ColR Fillden mul Fillden sub 1 add def
  /ColG ColG Fillden mul Fillden sub 1 add def
  /ColB ColB Fillden mul Fillden sub 1 add def
  ColR ColG ColB setrgbcolor} def
/BoxColFill {gsave Rec PolyFill} def
/PolyFill {gsave Density fill grestore grestore} def
/h {rlineto rlineto rlineto closepath gsave fill grestore stroke} bind def
%
% PostScript Level 1 Pattern Fill routine for rectangles
% Usage: x y w h s a XX PatternFill
%	x,y = lower left corner of box to be filled
%	w,h = width and height of box
%	  a = angle in degrees between lines and x-axis
%	 XX = 0/1 for no/yes cross-hatch
%
/PatternFill {gsave /PFa [ 9 2 roll ] def
  PFa 0 get PFa 2 get 2 div add PFa 1 get PFa 3 get 2 div add translate
  PFa 2 get -2 div PFa 3 get -2 div PFa 2 get PFa 3 get Rec
  TransparentPatterns {} {gsave 1 setgray fill grestore} ifelse
  clip
  currentlinewidth 0.5 mul setlinewidth
  /PFs PFa 2 get dup mul PFa 3 get dup mul add sqrt def
  0 0 M PFa 5 get rotate PFs -2 div dup translate
  0 1 PFs PFa 4 get div 1 add floor cvi
	{PFa 4 get mul 0 M 0 PFs V} for
  0 PFa 6 get ne {
	0 1 PFs PFa 4 get div 1 add floor cvi
	{PFa 4 get mul 0 2 1 roll M PFs 0 V} for
 } if
  stroke grestore} def
%
/languagelevel where
 {pop languagelevel} {1} ifelse
dup 2 lt
	{/InterpretLevel1 true def
	 /InterpretLevel3 false def}
	{/InterpretLevel1 Level1 def
	 2 gt
	    {/InterpretLevel3 Level3 def}
	    {/InterpretLevel3 false def}
	 ifelse }
 ifelse
%
% PostScript level 2 pattern fill definitions
%
/Level2PatternFill {
/Tile8x8 {/PaintType 2 /PatternType 1 /TilingType 1 /BBox [0 0 8 8] /XStep 8 /YStep 8}
	bind def
/KeepColor {currentrgbcolor [/Pattern /DeviceRGB] setcolorspace} bind def
<< Tile8x8
 /PaintProc {0.5 setlinewidth pop 0 0 M 8 8 L 0 8 M 8 0 L stroke} 
>> matrix makepattern
/Pat1 exch def
<< Tile8x8
 /PaintProc {0.5 setlinewidth pop 0 0 M 8 8 L 0 8 M 8 0 L stroke
	0 4 M 4 8 L 8 4 L 4 0 L 0 4 L stroke}
>> matrix makepattern
/Pat2 exch def
<< Tile8x8
 /PaintProc {0.5 setlinewidth pop 0 0 M 0 8 L
	8 8 L 8 0 L 0 0 L fill}
>> matrix makepattern
/Pat3 exch def
<< Tile8x8
 /PaintProc {0.5 setlinewidth pop -4 8 M 8 -4 L
	0 12 M 12 0 L stroke}
>> matrix makepattern
/Pat4 exch def
<< Tile8x8
 /PaintProc {0.5 setlinewidth pop -4 0 M 8 12 L
	0 -4 M 12 8 L stroke}
>> matrix makepattern
/Pat5 exch def
<< Tile8x8
 /PaintProc {0.5 setlinewidth pop -2 8 M 4 -4 L
	0 12 M 8 -4 L 4 12 M 10 0 L stroke}
>> matrix makepattern
/Pat6 exch def
<< Tile8x8
 /PaintProc {0.5 setlinewidth pop -2 0 M 4 12 L
	0 -4 M 8 12 L 4 -4 M 10 8 L stroke}
>> matrix makepattern
/Pat7 exch def
<< Tile8x8
 /PaintProc {0.5 setlinewidth pop 8 -2 M -4 4 L
	12 0 M -4 8 L 12 4 M 0 10 L stroke}
>> matrix makepattern
/Pat8 exch def
<< Tile8x8
 /PaintProc {0.5 setlinewidth pop 0 -2 M 12 4 L
	-4 0 M 12 8 L -4 4 M 8 10 L stroke}
>> matrix makepattern
/Pat9 exch def
/Pattern1 {PatternBgnd KeepColor Pat1 setpattern} bind def
/Pattern2 {PatternBgnd KeepColor Pat2 setpattern} bind def
/Pattern3 {PatternBgnd KeepColor Pat3 setpattern} bind def
/Pattern4 {PatternBgnd KeepColor Landscape {Pat5} {Pat4} ifelse setpattern} bind def
/Pattern5 {PatternBgnd KeepColor Landscape {Pat4} {Pat5} ifelse setpattern} bind def
/Pattern6 {PatternBgnd KeepColor Landscape {Pat9} {Pat6} ifelse setpattern} bind def
/Pattern7 {PatternBgnd KeepColor Landscape {Pat8} {Pat7} ifelse setpattern} bind def
} def
%
%
%End of PostScript Level 2 code
%
/PatternBgnd {
  TransparentPatterns {} {gsave 1 setgray fill grestore} ifelse
} def
%
% Substitute for Level 2 pattern fill codes with
% grayscale if Level 2 support is not selected.
%
/Level1PatternFill {
/Pattern1 {0.250 Density} bind def
/Pattern2 {0.500 Density} bind def
/Pattern3 {0.750 Density} bind def
/Pattern4 {0.125 Density} bind def
/Pattern5 {0.375 Density} bind def
/Pattern6 {0.625 Density} bind def
/Pattern7 {0.875 Density} bind def
} def
%
% Now test for support of Level 2 code
%
Level1 {Level1PatternFill} {Level2PatternFill} ifelse
%
/Symbol-Oblique /Symbol findfont [1 0 .167 1 0 0] makefont
dup length dict begin {1 index /FID eq {pop pop} {def} ifelse} forall
currentdict end definefont pop
%
Level1 SuppressPDFMark or 
{} {
/SDict 10 dict def
systemdict /pdfmark known not {
  userdict /pdfmark systemdict /cleartomark get put
} if
SDict begin [
  /Title (plot_MeffA1_SU8.tex)
  /Subject (gnuplot plot)
  /Creator (gnuplot 5.2 patchlevel 8)
%  /Producer (gnuplot)
%  /Keywords ()
  /CreationDate (Sun Oct 31 19:07:00 2021)
  /DOCINFO pdfmark
end
} ifelse
%
% Support for boxed text - Ethan A Merritt Sep 2016
%
/InitTextBox { userdict /TBy2 3 -1 roll put userdict /TBx2 3 -1 roll put
           userdict /TBy1 3 -1 roll put userdict /TBx1 3 -1 roll put
	   /Boxing true def } def
/ExtendTextBox { dup type /stringtype eq
    { Boxing { gsave dup false charpath pathbbox
      dup TBy2 gt {userdict /TBy2 3 -1 roll put} {pop} ifelse
      dup TBx2 gt {userdict /TBx2 3 -1 roll put} {pop} ifelse
      dup TBy1 lt {userdict /TBy1 3 -1 roll put} {pop} ifelse
      dup TBx1 lt {userdict /TBx1 3 -1 roll put} {pop} ifelse
      grestore } if }
    {} ifelse} def
/PopTextBox { newpath TBx1 TBxmargin sub TBy1 TBymargin sub M
               TBx1 TBxmargin sub TBy2 TBymargin add L
	       TBx2 TBxmargin add TBy2 TBymargin add L
	       TBx2 TBxmargin add TBy1 TBymargin sub L closepath } def
/DrawTextBox { PopTextBox stroke /Boxing false def} def
/FillTextBox { gsave PopTextBox fill grestore /Boxing false def} def
0 0 0 0 InitTextBox
/TBxmargin 20 def
/TBymargin 20 def
/Boxing false def
/textshow { ExtendTextBox Gshow } def
%
end
%%EndProlog
%%Page: 1 1
gnudict begin
gsave
doclip
0 0 translate
0.050 0.050 scale
0 setgray
newpath
BackgroundColor 0 lt 3 1 roll 0 lt exch 0 lt or or not {BackgroundColor C 1.000 0 0 7200.00 7560.00 BoxColFill} if
1.000 UL
LTb
LCb setrgbcolor
2100 896 M
63 0 V
4532 0 R
-63 0 V
stroke
LTb
LCb setrgbcolor
2100 1387 M
63 0 V
4532 0 R
-63 0 V
stroke
LTb
LCb setrgbcolor
2100 1878 M
63 0 V
4532 0 R
-63 0 V
stroke
LTb
LCb setrgbcolor
2100 2369 M
63 0 V
4532 0 R
-63 0 V
stroke
LTb
LCb setrgbcolor
2100 2860 M
63 0 V
4532 0 R
-63 0 V
stroke
LTb
LCb setrgbcolor
2100 3351 M
63 0 V
4532 0 R
-63 0 V
stroke
LTb
LCb setrgbcolor
2100 3842 M
63 0 V
4532 0 R
-63 0 V
stroke
LTb
LCb setrgbcolor
2100 4333 M
63 0 V
4532 0 R
-63 0 V
stroke
LTb
LCb setrgbcolor
2100 4824 M
63 0 V
4532 0 R
-63 0 V
stroke
LTb
LCb setrgbcolor
2100 5315 M
63 0 V
4532 0 R
-63 0 V
stroke
LTb
LCb setrgbcolor
2100 5806 M
63 0 V
4532 0 R
-63 0 V
stroke
LTb
LCb setrgbcolor
2100 6297 M
63 0 V
4532 0 R
-63 0 V
stroke
LTb
LCb setrgbcolor
2100 6788 M
63 0 V
4532 0 R
-63 0 V
stroke
LTb
LCb setrgbcolor
2100 7279 M
63 0 V
4532 0 R
-63 0 V
stroke
LTb
LCb setrgbcolor
2100 896 M
0 63 V
0 6320 R
0 -63 V
stroke
LTb
LCb setrgbcolor
2560 896 M
0 63 V
0 6320 R
0 -63 V
stroke
LTb
LCb setrgbcolor
3019 896 M
0 63 V
0 6320 R
0 -63 V
stroke
LTb
LCb setrgbcolor
3479 896 M
0 63 V
0 6320 R
0 -63 V
stroke
LTb
LCb setrgbcolor
3938 896 M
0 63 V
0 6320 R
0 -63 V
stroke
LTb
LCb setrgbcolor
4398 896 M
0 63 V
0 6320 R
0 -63 V
stroke
LTb
LCb setrgbcolor
4857 896 M
0 63 V
0 6320 R
0 -63 V
stroke
LTb
LCb setrgbcolor
5317 896 M
0 63 V
0 6320 R
0 -63 V
stroke
LTb
LCb setrgbcolor
5776 896 M
0 63 V
0 6320 R
0 -63 V
stroke
LTb
LCb setrgbcolor
6236 896 M
0 63 V
0 6320 R
0 -63 V
stroke
LTb
LCb setrgbcolor
6695 896 M
0 63 V
0 6320 R
0 -63 V
stroke
LTb
LCb setrgbcolor
1.000 UL
LTb
LCb setrgbcolor
2100 7279 N
0 -6383 V
4595 0 V
0 6383 V
-4595 0 V
Z stroke
1.000 UP
1.000 UL
LTb
LCb setrgbcolor
LCb setrgbcolor
LTb
LCb setrgbcolor
LTb
1.000 UL
LTb
0.58 0.00 0.83 C
gsave 2100 2883 N 4595 0 V 0 28 V -4595 0 V 0.20 PolyFill
1.500 UP
1.000 UL
LTb
0.58 0.00 0.83 C
2330 3025 M
0 9 V
459 -123 R
0 12 V
460 -43 R
0 19 V
459 -27 R
0 26 V
460 -4 R
0 38 V
459 -63 R
0 62 V
460 -23 R
0 123 V
459 -85 R
0 177 V
460 -292 R
0 310 V
2330 3029 CircleF
2789 2917 CircleF
3249 2889 CircleF
3708 2885 CircleF
4168 2913 CircleF
4627 2900 CircleF
5087 2969 CircleF
5546 3034 CircleF
6006 2986 CircleF
1.500 UL
LTb
0.58 0.00 0.83 C
2100 2899 M
46 0 V
47 0 V
46 0 V
47 0 V
46 0 V
46 0 V
47 0 V
46 0 V
47 0 V
46 0 V
47 0 V
46 0 V
46 0 V
47 0 V
46 0 V
47 0 V
46 0 V
46 0 V
47 0 V
46 0 V
47 0 V
46 0 V
47 0 V
46 0 V
46 0 V
47 0 V
46 0 V
47 0 V
46 0 V
46 0 V
47 0 V
46 0 V
47 0 V
46 0 V
46 0 V
47 0 V
46 0 V
47 0 V
46 0 V
47 0 V
46 0 V
46 0 V
47 0 V
46 0 V
47 0 V
46 0 V
46 0 V
47 0 V
46 0 V
47 0 V
46 0 V
47 0 V
46 0 V
46 0 V
47 0 V
46 0 V
47 0 V
46 0 V
46 0 V
47 0 V
46 0 V
47 0 V
46 0 V
47 0 V
46 0 V
46 0 V
47 0 V
46 0 V
47 0 V
46 0 V
46 0 V
47 0 V
46 0 V
47 0 V
46 0 V
46 0 V
47 0 V
46 0 V
47 0 V
46 0 V
47 0 V
46 0 V
46 0 V
47 0 V
46 0 V
47 0 V
46 0 V
46 0 V
47 0 V
46 0 V
47 0 V
46 0 V
47 0 V
46 0 V
46 0 V
47 0 V
46 0 V
47 0 V
46 0 V
stroke
1.000 UL
LTb
0.58 0.00 0.83 C
gsave 2100 4677 N 4595 0 V 0 108 V -4595 0 V 0.20 PolyFill
1.500 UP
1.000 UL
LTb
0.58 0.00 0.83 C
2330 5108 M
0 16 V
459 -302 R
0 30 V
460 -140 R
0 67 V
459 -130 R
0 160 V
460 -328 R
0 349 V
2330 5116 CircleF
2789 4837 CircleF
3249 4746 CircleF
3708 4729 CircleF
4168 4656 CircleF
1.500 UL
LTb
0.58 0.00 0.83 C
2100 4731 M
46 0 V
47 0 V
46 0 V
47 0 V
46 0 V
46 0 V
47 0 V
46 0 V
47 0 V
46 0 V
47 0 V
46 0 V
46 0 V
47 0 V
46 0 V
47 0 V
46 0 V
46 0 V
47 0 V
46 0 V
47 0 V
46 0 V
47 0 V
46 0 V
46 0 V
47 0 V
46 0 V
47 0 V
46 0 V
46 0 V
47 0 V
46 0 V
47 0 V
46 0 V
46 0 V
47 0 V
46 0 V
47 0 V
46 0 V
47 0 V
46 0 V
46 0 V
47 0 V
46 0 V
47 0 V
46 0 V
46 0 V
47 0 V
46 0 V
47 0 V
46 0 V
47 0 V
46 0 V
46 0 V
47 0 V
46 0 V
47 0 V
46 0 V
46 0 V
47 0 V
46 0 V
47 0 V
46 0 V
47 0 V
46 0 V
46 0 V
47 0 V
46 0 V
47 0 V
46 0 V
46 0 V
47 0 V
46 0 V
47 0 V
46 0 V
46 0 V
47 0 V
46 0 V
47 0 V
46 0 V
47 0 V
46 0 V
46 0 V
47 0 V
46 0 V
47 0 V
46 0 V
46 0 V
47 0 V
46 0 V
47 0 V
46 0 V
47 0 V
46 0 V
46 0 V
47 0 V
46 0 V
47 0 V
46 0 V
stroke
1.000 UL
LTb
0.58 0.00 0.83 C
gsave 2100 5713 N 4595 0 V 0 157 V -4595 0 V 0.20 PolyFill
1.000 UL
LTb
0.00 0.62 0.45 C
gsave 2560 5452 N 4135 0 V 0 344 V -4135 0 V 0.20 PolyFill
1.500 UP
1.000 UL
LTb
0.58 0.00 0.83 C
2330 6436 M
0 17 V
459 -492 R
0 48 V
460 -288 R
0 141 V
459 -245 R
0 328 V
460 43 R
0 1004 V
2330 6444 CircleF
2789 5985 CircleF
3249 5791 CircleF
3708 5781 CircleF
4168 6490 CircleF
1.500 UL
LTb
0.58 0.00 0.83 C
2100 5791 M
46 0 V
47 0 V
46 0 V
47 0 V
46 0 V
46 0 V
47 0 V
46 0 V
47 0 V
46 0 V
47 0 V
46 0 V
46 0 V
47 0 V
46 0 V
47 0 V
46 0 V
46 0 V
47 0 V
46 0 V
47 0 V
46 0 V
47 0 V
46 0 V
46 0 V
47 0 V
46 0 V
47 0 V
46 0 V
46 0 V
47 0 V
46 0 V
47 0 V
46 0 V
46 0 V
47 0 V
46 0 V
47 0 V
46 0 V
47 0 V
46 0 V
46 0 V
47 0 V
46 0 V
47 0 V
46 0 V
46 0 V
47 0 V
46 0 V
47 0 V
46 0 V
47 0 V
46 0 V
46 0 V
47 0 V
46 0 V
47 0 V
46 0 V
46 0 V
47 0 V
46 0 V
47 0 V
46 0 V
47 0 V
46 0 V
46 0 V
47 0 V
46 0 V
47 0 V
46 0 V
46 0 V
47 0 V
46 0 V
47 0 V
46 0 V
46 0 V
47 0 V
46 0 V
47 0 V
46 0 V
47 0 V
46 0 V
46 0 V
47 0 V
46 0 V
47 0 V
46 0 V
46 0 V
47 0 V
46 0 V
47 0 V
46 0 V
47 0 V
46 0 V
46 0 V
47 0 V
46 0 V
47 0 V
46 0 V
stroke
1.000 UL
LTb
0.00 0.62 0.45 C
gsave 2100 4014 N 4595 0 V 0 79 V -4595 0 V 0.20 PolyFill
1.500 UP
1.000 UL
LTb
0.00 0.62 0.45 C
2330 4540 M
0 12 V
459 -372 R
0 22 V
460 -148 R
0 49 V
459 -96 R
0 81 V
460 -183 R
0 113 V
459 -117 R
0 261 V
2330 4546 Circle
2789 4191 Circle
3249 4078 Circle
3708 4048 Circle
4168 3961 Circle
4627 4031 Circle
1.500 UL
LTb
0.00 0.62 0.45 C
2100 4054 M
46 0 V
47 0 V
46 0 V
47 0 V
46 0 V
46 0 V
47 0 V
46 0 V
47 0 V
46 0 V
47 0 V
46 0 V
46 0 V
47 0 V
46 0 V
47 0 V
46 0 V
46 0 V
47 0 V
46 0 V
47 0 V
46 0 V
47 0 V
46 0 V
46 0 V
47 0 V
46 0 V
47 0 V
46 0 V
46 0 V
47 0 V
46 0 V
47 0 V
46 0 V
46 0 V
47 0 V
46 0 V
47 0 V
46 0 V
47 0 V
46 0 V
46 0 V
47 0 V
46 0 V
47 0 V
46 0 V
46 0 V
47 0 V
46 0 V
47 0 V
46 0 V
47 0 V
46 0 V
46 0 V
47 0 V
46 0 V
47 0 V
46 0 V
46 0 V
47 0 V
46 0 V
47 0 V
46 0 V
47 0 V
46 0 V
46 0 V
47 0 V
46 0 V
47 0 V
46 0 V
46 0 V
47 0 V
46 0 V
47 0 V
46 0 V
46 0 V
47 0 V
46 0 V
47 0 V
46 0 V
47 0 V
46 0 V
46 0 V
47 0 V
46 0 V
47 0 V
46 0 V
46 0 V
47 0 V
46 0 V
47 0 V
46 0 V
47 0 V
46 0 V
46 0 V
47 0 V
46 0 V
47 0 V
46 0 V
1.500 UP
stroke
1.000 UL
LTb
0.00 0.62 0.45 C
2330 6417 M
0 17 V
459 -582 R
0 49 V
460 -256 R
0 119 V
459 -314 R
0 345 V
460 218 R
0 1194 V
2330 6426 Circle
2789 5877 Circle
3249 5705 Circle
3708 5623 Circle
4168 6610 Circle
1.500 UL
LTb
0.00 0.62 0.45 C
2560 5639 M
4135 0 V
1.500 UP
stroke
1.000 UL
LTb
0.58 0.00 0.83 C
2330 5424 M
0 18 V
459 -759 R
0 41 V
460 -305 R
0 74 V
459 -261 R
0 164 V
460 -405 R
0 257 V
459 -326 R
0 382 V
2330 5433 Star
2789 4703 Star
3249 4456 Star
3708 4314 Star
4168 4120 Star
4627 4113 Star
2.000 UL
LTb
LCb setrgbcolor
1.000 UL
LTb
LCb setrgbcolor
2100 7279 N
0 -6383 V
4595 0 V
0 6383 V
-4595 0 V
Z stroke
1.000 UP
1.000 UL
LTb
LCb setrgbcolor
stroke
grestore
end
showpage
  }}%
\fontsize{14}{\baselineskip}\selectfont
  \put(4397,196){\makebox(0,0){{$n_t$}}}%
  \put(378,5487){\makebox(0,0){{$aM_{eff}^{A_1}$}}}%
  \put(6695,616){\makebox(0,0){\strut{}\ {$10$}}}%
  \put(6236,616){\makebox(0,0){\strut{}\ {$9$}}}%
  \put(5776,616){\makebox(0,0){\strut{}\ {$8$}}}%
  \put(5317,616){\makebox(0,0){\strut{}\ {$7$}}}%
  \put(4857,616){\makebox(0,0){\strut{}\ {$6$}}}%
  \put(4398,616){\makebox(0,0){\strut{}\ {$5$}}}%
  \put(3938,616){\makebox(0,0){\strut{}\ {$4$}}}%
  \put(3479,616){\makebox(0,0){\strut{}\ {$3$}}}%
  \put(3019,616){\makebox(0,0){\strut{}\ {$2$}}}%
  \put(2560,616){\makebox(0,0){\strut{}\ {$1$}}}%
  \put(2100,616){\makebox(0,0){\strut{}\ {$0$}}}%
  \put(1932,7279){\makebox(0,0)[r]{\strut{}\ \ {$1.3$}}}%
  \put(1932,6788){\makebox(0,0)[r]{\strut{}\ \ {$1.2$}}}%
  \put(1932,6297){\makebox(0,0)[r]{\strut{}\ \ {$1.1$}}}%
  \put(1932,5806){\makebox(0,0)[r]{\strut{}\ \ {$1$}}}%
  \put(1932,5315){\makebox(0,0)[r]{\strut{}\ \ {$0.9$}}}%
  \put(1932,4824){\makebox(0,0)[r]{\strut{}\ \ {$0.8$}}}%
  \put(1932,4333){\makebox(0,0)[r]{\strut{}\ \ {$0.7$}}}%
  \put(1932,3842){\makebox(0,0)[r]{\strut{}\ \ {$0.6$}}}%
  \put(1932,3351){\makebox(0,0)[r]{\strut{}\ \ {$0.5$}}}%
  \put(1932,2860){\makebox(0,0)[r]{\strut{}\ \ {$0.4$}}}%
  \put(1932,2369){\makebox(0,0)[r]{\strut{}\ \ {$0.3$}}}%
  \put(1932,1878){\makebox(0,0)[r]{\strut{}\ \ {$0.2$}}}%
  \put(1932,1387){\makebox(0,0)[r]{\strut{}\ \ {$0.1$}}}%
  \put(1932,896){\makebox(0,0)[r]{\strut{}\ \ {$0$}}}%
\end{picture}%
\endgroup
\endinput

\end	{center}
\caption{Effective masses for the lightest three $A_1^{++}$ ($\bullet$) and the lightest
  two $A_1^{-+}$ ($\circ$) glueball states, as well as the main $A_1^{++}$ ditorelon state ($\ast$).
  Lines are our best glueball mass estimates, with bands corresponding to $\pm 1$ standard deviations.
  All on a $20^330$ lattice at $\beta=47.75$ in $SU(8)$.
  In the continuum limit the lightest two glueball states in each sector become the lightest
  two $J^{PC}=0^{++}$ and $J^{PC}=0^{-+}$ glueballs. The ditorelon disappears in the thermodynamic
  limit.}
\label{fig_MeffA1_SU8}
\end{figure}


\begin{figure}[htb]
\begin	{center}
\leavevmode
% GNUPLOT: LaTeX picture with Postscript
\begingroup%
\makeatletter%
\newcommand{\GNUPLOTspecial}{%
  \@sanitize\catcode`\%=14\relax\special}%
\setlength{\unitlength}{0.0500bp}%
\begin{picture}(7200,7560)(0,0)%
  {\GNUPLOTspecial{"
%!PS-Adobe-2.0 EPSF-2.0
%%Title: plot_MeffA1b_SU8.tex
%%Creator: gnuplot 5.0 patchlevel 3
%%CreationDate: Sun Nov  7 12:53:53 2021
%%DocumentFonts: 
%%BoundingBox: 0 0 360 378
%%EndComments
%%BeginProlog
/gnudict 256 dict def
gnudict begin
%
% The following true/false flags may be edited by hand if desired.
% The unit line width and grayscale image gamma correction may also be changed.
%
/Color true def
/Blacktext true def
/Solid false def
/Dashlength 1 def
/Landscape false def
/Level1 false def
/Level3 false def
/Rounded false def
/ClipToBoundingBox false def
/SuppressPDFMark false def
/TransparentPatterns false def
/gnulinewidth 5.000 def
/userlinewidth gnulinewidth def
/Gamma 1.0 def
/BackgroundColor {-1.000 -1.000 -1.000} def
%
/vshift -66 def
/dl1 {
  10.0 Dashlength userlinewidth gnulinewidth div mul mul mul
  Rounded { currentlinewidth 0.75 mul sub dup 0 le { pop 0.01 } if } if
} def
/dl2 {
  10.0 Dashlength userlinewidth gnulinewidth div mul mul mul
  Rounded { currentlinewidth 0.75 mul add } if
} def
/hpt_ 31.5 def
/vpt_ 31.5 def
/hpt hpt_ def
/vpt vpt_ def
/doclip {
  ClipToBoundingBox {
    newpath 0 0 moveto 360 0 lineto 360 378 lineto 0 378 lineto closepath
    clip
  } if
} def
%
% Gnuplot Prolog Version 5.1 (Oct 2015)
%
%/SuppressPDFMark true def
%
/M {moveto} bind def
/L {lineto} bind def
/R {rmoveto} bind def
/V {rlineto} bind def
/N {newpath moveto} bind def
/Z {closepath} bind def
/C {setrgbcolor} bind def
/f {rlineto fill} bind def
/g {setgray} bind def
/Gshow {show} def   % May be redefined later in the file to support UTF-8
/vpt2 vpt 2 mul def
/hpt2 hpt 2 mul def
/Lshow {currentpoint stroke M 0 vshift R 
	Blacktext {gsave 0 setgray textshow grestore} {textshow} ifelse} def
/Rshow {currentpoint stroke M dup stringwidth pop neg vshift R
	Blacktext {gsave 0 setgray textshow grestore} {textshow} ifelse} def
/Cshow {currentpoint stroke M dup stringwidth pop -2 div vshift R 
	Blacktext {gsave 0 setgray textshow grestore} {textshow} ifelse} def
/UP {dup vpt_ mul /vpt exch def hpt_ mul /hpt exch def
  /hpt2 hpt 2 mul def /vpt2 vpt 2 mul def} def
/DL {Color {setrgbcolor Solid {pop []} if 0 setdash}
 {pop pop pop 0 setgray Solid {pop []} if 0 setdash} ifelse} def
/BL {stroke userlinewidth 2 mul setlinewidth
	Rounded {1 setlinejoin 1 setlinecap} if} def
/AL {stroke userlinewidth 2 div setlinewidth
	Rounded {1 setlinejoin 1 setlinecap} if} def
/UL {dup gnulinewidth mul /userlinewidth exch def
	dup 1 lt {pop 1} if 10 mul /udl exch def} def
/PL {stroke userlinewidth setlinewidth
	Rounded {1 setlinejoin 1 setlinecap} if} def
3.8 setmiterlimit
% Classic Line colors (version 5.0)
/LCw {1 1 1} def
/LCb {0 0 0} def
/LCa {0 0 0} def
/LC0 {1 0 0} def
/LC1 {0 1 0} def
/LC2 {0 0 1} def
/LC3 {1 0 1} def
/LC4 {0 1 1} def
/LC5 {1 1 0} def
/LC6 {0 0 0} def
/LC7 {1 0.3 0} def
/LC8 {0.5 0.5 0.5} def
% Default dash patterns (version 5.0)
/LTB {BL [] LCb DL} def
/LTw {PL [] 1 setgray} def
/LTb {PL [] LCb DL} def
/LTa {AL [1 udl mul 2 udl mul] 0 setdash LCa setrgbcolor} def
/LT0 {PL [] LC0 DL} def
/LT1 {PL [2 dl1 3 dl2] LC1 DL} def
/LT2 {PL [1 dl1 1.5 dl2] LC2 DL} def
/LT3 {PL [6 dl1 2 dl2 1 dl1 2 dl2] LC3 DL} def
/LT4 {PL [1 dl1 2 dl2 6 dl1 2 dl2 1 dl1 2 dl2] LC4 DL} def
/LT5 {PL [4 dl1 2 dl2] LC5 DL} def
/LT6 {PL [1.5 dl1 1.5 dl2 1.5 dl1 1.5 dl2 1.5 dl1 6 dl2] LC6 DL} def
/LT7 {PL [3 dl1 3 dl2 1 dl1 3 dl2] LC7 DL} def
/LT8 {PL [2 dl1 2 dl2 2 dl1 6 dl2] LC8 DL} def
/SL {[] 0 setdash} def
/Pnt {stroke [] 0 setdash gsave 1 setlinecap M 0 0 V stroke grestore} def
/Dia {stroke [] 0 setdash 2 copy vpt add M
  hpt neg vpt neg V hpt vpt neg V
  hpt vpt V hpt neg vpt V closepath stroke
  Pnt} def
/Pls {stroke [] 0 setdash vpt sub M 0 vpt2 V
  currentpoint stroke M
  hpt neg vpt neg R hpt2 0 V stroke
 } def
/Box {stroke [] 0 setdash 2 copy exch hpt sub exch vpt add M
  0 vpt2 neg V hpt2 0 V 0 vpt2 V
  hpt2 neg 0 V closepath stroke
  Pnt} def
/Crs {stroke [] 0 setdash exch hpt sub exch vpt add M
  hpt2 vpt2 neg V currentpoint stroke M
  hpt2 neg 0 R hpt2 vpt2 V stroke} def
/TriU {stroke [] 0 setdash 2 copy vpt 1.12 mul add M
  hpt neg vpt -1.62 mul V
  hpt 2 mul 0 V
  hpt neg vpt 1.62 mul V closepath stroke
  Pnt} def
/Star {2 copy Pls Crs} def
/BoxF {stroke [] 0 setdash exch hpt sub exch vpt add M
  0 vpt2 neg V hpt2 0 V 0 vpt2 V
  hpt2 neg 0 V closepath fill} def
/TriUF {stroke [] 0 setdash vpt 1.12 mul add M
  hpt neg vpt -1.62 mul V
  hpt 2 mul 0 V
  hpt neg vpt 1.62 mul V closepath fill} def
/TriD {stroke [] 0 setdash 2 copy vpt 1.12 mul sub M
  hpt neg vpt 1.62 mul V
  hpt 2 mul 0 V
  hpt neg vpt -1.62 mul V closepath stroke
  Pnt} def
/TriDF {stroke [] 0 setdash vpt 1.12 mul sub M
  hpt neg vpt 1.62 mul V
  hpt 2 mul 0 V
  hpt neg vpt -1.62 mul V closepath fill} def
/DiaF {stroke [] 0 setdash vpt add M
  hpt neg vpt neg V hpt vpt neg V
  hpt vpt V hpt neg vpt V closepath fill} def
/Pent {stroke [] 0 setdash 2 copy gsave
  translate 0 hpt M 4 {72 rotate 0 hpt L} repeat
  closepath stroke grestore Pnt} def
/PentF {stroke [] 0 setdash gsave
  translate 0 hpt M 4 {72 rotate 0 hpt L} repeat
  closepath fill grestore} def
/Circle {stroke [] 0 setdash 2 copy
  hpt 0 360 arc stroke Pnt} def
/CircleF {stroke [] 0 setdash hpt 0 360 arc fill} def
/C0 {BL [] 0 setdash 2 copy moveto vpt 90 450 arc} bind def
/C1 {BL [] 0 setdash 2 copy moveto
	2 copy vpt 0 90 arc closepath fill
	vpt 0 360 arc closepath} bind def
/C2 {BL [] 0 setdash 2 copy moveto
	2 copy vpt 90 180 arc closepath fill
	vpt 0 360 arc closepath} bind def
/C3 {BL [] 0 setdash 2 copy moveto
	2 copy vpt 0 180 arc closepath fill
	vpt 0 360 arc closepath} bind def
/C4 {BL [] 0 setdash 2 copy moveto
	2 copy vpt 180 270 arc closepath fill
	vpt 0 360 arc closepath} bind def
/C5 {BL [] 0 setdash 2 copy moveto
	2 copy vpt 0 90 arc
	2 copy moveto
	2 copy vpt 180 270 arc closepath fill
	vpt 0 360 arc} bind def
/C6 {BL [] 0 setdash 2 copy moveto
	2 copy vpt 90 270 arc closepath fill
	vpt 0 360 arc closepath} bind def
/C7 {BL [] 0 setdash 2 copy moveto
	2 copy vpt 0 270 arc closepath fill
	vpt 0 360 arc closepath} bind def
/C8 {BL [] 0 setdash 2 copy moveto
	2 copy vpt 270 360 arc closepath fill
	vpt 0 360 arc closepath} bind def
/C9 {BL [] 0 setdash 2 copy moveto
	2 copy vpt 270 450 arc closepath fill
	vpt 0 360 arc closepath} bind def
/C10 {BL [] 0 setdash 2 copy 2 copy moveto vpt 270 360 arc closepath fill
	2 copy moveto
	2 copy vpt 90 180 arc closepath fill
	vpt 0 360 arc closepath} bind def
/C11 {BL [] 0 setdash 2 copy moveto
	2 copy vpt 0 180 arc closepath fill
	2 copy moveto
	2 copy vpt 270 360 arc closepath fill
	vpt 0 360 arc closepath} bind def
/C12 {BL [] 0 setdash 2 copy moveto
	2 copy vpt 180 360 arc closepath fill
	vpt 0 360 arc closepath} bind def
/C13 {BL [] 0 setdash 2 copy moveto
	2 copy vpt 0 90 arc closepath fill
	2 copy moveto
	2 copy vpt 180 360 arc closepath fill
	vpt 0 360 arc closepath} bind def
/C14 {BL [] 0 setdash 2 copy moveto
	2 copy vpt 90 360 arc closepath fill
	vpt 0 360 arc} bind def
/C15 {BL [] 0 setdash 2 copy vpt 0 360 arc closepath fill
	vpt 0 360 arc closepath} bind def
/Rec {newpath 4 2 roll moveto 1 index 0 rlineto 0 exch rlineto
	neg 0 rlineto closepath} bind def
/Square {dup Rec} bind def
/Bsquare {vpt sub exch vpt sub exch vpt2 Square} bind def
/S0 {BL [] 0 setdash 2 copy moveto 0 vpt rlineto BL Bsquare} bind def
/S1 {BL [] 0 setdash 2 copy vpt Square fill Bsquare} bind def
/S2 {BL [] 0 setdash 2 copy exch vpt sub exch vpt Square fill Bsquare} bind def
/S3 {BL [] 0 setdash 2 copy exch vpt sub exch vpt2 vpt Rec fill Bsquare} bind def
/S4 {BL [] 0 setdash 2 copy exch vpt sub exch vpt sub vpt Square fill Bsquare} bind def
/S5 {BL [] 0 setdash 2 copy 2 copy vpt Square fill
	exch vpt sub exch vpt sub vpt Square fill Bsquare} bind def
/S6 {BL [] 0 setdash 2 copy exch vpt sub exch vpt sub vpt vpt2 Rec fill Bsquare} bind def
/S7 {BL [] 0 setdash 2 copy exch vpt sub exch vpt sub vpt vpt2 Rec fill
	2 copy vpt Square fill Bsquare} bind def
/S8 {BL [] 0 setdash 2 copy vpt sub vpt Square fill Bsquare} bind def
/S9 {BL [] 0 setdash 2 copy vpt sub vpt vpt2 Rec fill Bsquare} bind def
/S10 {BL [] 0 setdash 2 copy vpt sub vpt Square fill 2 copy exch vpt sub exch vpt Square fill
	Bsquare} bind def
/S11 {BL [] 0 setdash 2 copy vpt sub vpt Square fill 2 copy exch vpt sub exch vpt2 vpt Rec fill
	Bsquare} bind def
/S12 {BL [] 0 setdash 2 copy exch vpt sub exch vpt sub vpt2 vpt Rec fill Bsquare} bind def
/S13 {BL [] 0 setdash 2 copy exch vpt sub exch vpt sub vpt2 vpt Rec fill
	2 copy vpt Square fill Bsquare} bind def
/S14 {BL [] 0 setdash 2 copy exch vpt sub exch vpt sub vpt2 vpt Rec fill
	2 copy exch vpt sub exch vpt Square fill Bsquare} bind def
/S15 {BL [] 0 setdash 2 copy Bsquare fill Bsquare} bind def
/D0 {gsave translate 45 rotate 0 0 S0 stroke grestore} bind def
/D1 {gsave translate 45 rotate 0 0 S1 stroke grestore} bind def
/D2 {gsave translate 45 rotate 0 0 S2 stroke grestore} bind def
/D3 {gsave translate 45 rotate 0 0 S3 stroke grestore} bind def
/D4 {gsave translate 45 rotate 0 0 S4 stroke grestore} bind def
/D5 {gsave translate 45 rotate 0 0 S5 stroke grestore} bind def
/D6 {gsave translate 45 rotate 0 0 S6 stroke grestore} bind def
/D7 {gsave translate 45 rotate 0 0 S7 stroke grestore} bind def
/D8 {gsave translate 45 rotate 0 0 S8 stroke grestore} bind def
/D9 {gsave translate 45 rotate 0 0 S9 stroke grestore} bind def
/D10 {gsave translate 45 rotate 0 0 S10 stroke grestore} bind def
/D11 {gsave translate 45 rotate 0 0 S11 stroke grestore} bind def
/D12 {gsave translate 45 rotate 0 0 S12 stroke grestore} bind def
/D13 {gsave translate 45 rotate 0 0 S13 stroke grestore} bind def
/D14 {gsave translate 45 rotate 0 0 S14 stroke grestore} bind def
/D15 {gsave translate 45 rotate 0 0 S15 stroke grestore} bind def
/DiaE {stroke [] 0 setdash vpt add M
  hpt neg vpt neg V hpt vpt neg V
  hpt vpt V hpt neg vpt V closepath stroke} def
/BoxE {stroke [] 0 setdash exch hpt sub exch vpt add M
  0 vpt2 neg V hpt2 0 V 0 vpt2 V
  hpt2 neg 0 V closepath stroke} def
/TriUE {stroke [] 0 setdash vpt 1.12 mul add M
  hpt neg vpt -1.62 mul V
  hpt 2 mul 0 V
  hpt neg vpt 1.62 mul V closepath stroke} def
/TriDE {stroke [] 0 setdash vpt 1.12 mul sub M
  hpt neg vpt 1.62 mul V
  hpt 2 mul 0 V
  hpt neg vpt -1.62 mul V closepath stroke} def
/PentE {stroke [] 0 setdash gsave
  translate 0 hpt M 4 {72 rotate 0 hpt L} repeat
  closepath stroke grestore} def
/CircE {stroke [] 0 setdash 
  hpt 0 360 arc stroke} def
/Opaque {gsave closepath 1 setgray fill grestore 0 setgray closepath} def
/DiaW {stroke [] 0 setdash vpt add M
  hpt neg vpt neg V hpt vpt neg V
  hpt vpt V hpt neg vpt V Opaque stroke} def
/BoxW {stroke [] 0 setdash exch hpt sub exch vpt add M
  0 vpt2 neg V hpt2 0 V 0 vpt2 V
  hpt2 neg 0 V Opaque stroke} def
/TriUW {stroke [] 0 setdash vpt 1.12 mul add M
  hpt neg vpt -1.62 mul V
  hpt 2 mul 0 V
  hpt neg vpt 1.62 mul V Opaque stroke} def
/TriDW {stroke [] 0 setdash vpt 1.12 mul sub M
  hpt neg vpt 1.62 mul V
  hpt 2 mul 0 V
  hpt neg vpt -1.62 mul V Opaque stroke} def
/PentW {stroke [] 0 setdash gsave
  translate 0 hpt M 4 {72 rotate 0 hpt L} repeat
  Opaque stroke grestore} def
/CircW {stroke [] 0 setdash 
  hpt 0 360 arc Opaque stroke} def
/BoxFill {gsave Rec 1 setgray fill grestore} def
/Density {
  /Fillden exch def
  currentrgbcolor
  /ColB exch def /ColG exch def /ColR exch def
  /ColR ColR Fillden mul Fillden sub 1 add def
  /ColG ColG Fillden mul Fillden sub 1 add def
  /ColB ColB Fillden mul Fillden sub 1 add def
  ColR ColG ColB setrgbcolor} def
/BoxColFill {gsave Rec PolyFill} def
/PolyFill {gsave Density fill grestore grestore} def
/h {rlineto rlineto rlineto gsave closepath fill grestore} bind def
%
% PostScript Level 1 Pattern Fill routine for rectangles
% Usage: x y w h s a XX PatternFill
%	x,y = lower left corner of box to be filled
%	w,h = width and height of box
%	  a = angle in degrees between lines and x-axis
%	 XX = 0/1 for no/yes cross-hatch
%
/PatternFill {gsave /PFa [ 9 2 roll ] def
  PFa 0 get PFa 2 get 2 div add PFa 1 get PFa 3 get 2 div add translate
  PFa 2 get -2 div PFa 3 get -2 div PFa 2 get PFa 3 get Rec
  TransparentPatterns {} {gsave 1 setgray fill grestore} ifelse
  clip
  currentlinewidth 0.5 mul setlinewidth
  /PFs PFa 2 get dup mul PFa 3 get dup mul add sqrt def
  0 0 M PFa 5 get rotate PFs -2 div dup translate
  0 1 PFs PFa 4 get div 1 add floor cvi
	{PFa 4 get mul 0 M 0 PFs V} for
  0 PFa 6 get ne {
	0 1 PFs PFa 4 get div 1 add floor cvi
	{PFa 4 get mul 0 2 1 roll M PFs 0 V} for
 } if
  stroke grestore} def
%
/languagelevel where
 {pop languagelevel} {1} ifelse
dup 2 lt
	{/InterpretLevel1 true def
	 /InterpretLevel3 false def}
	{/InterpretLevel1 Level1 def
	 2 gt
	    {/InterpretLevel3 Level3 def}
	    {/InterpretLevel3 false def}
	 ifelse }
 ifelse
%
% PostScript level 2 pattern fill definitions
%
/Level2PatternFill {
/Tile8x8 {/PaintType 2 /PatternType 1 /TilingType 1 /BBox [0 0 8 8] /XStep 8 /YStep 8}
	bind def
/KeepColor {currentrgbcolor [/Pattern /DeviceRGB] setcolorspace} bind def
<< Tile8x8
 /PaintProc {0.5 setlinewidth pop 0 0 M 8 8 L 0 8 M 8 0 L stroke} 
>> matrix makepattern
/Pat1 exch def
<< Tile8x8
 /PaintProc {0.5 setlinewidth pop 0 0 M 8 8 L 0 8 M 8 0 L stroke
	0 4 M 4 8 L 8 4 L 4 0 L 0 4 L stroke}
>> matrix makepattern
/Pat2 exch def
<< Tile8x8
 /PaintProc {0.5 setlinewidth pop 0 0 M 0 8 L
	8 8 L 8 0 L 0 0 L fill}
>> matrix makepattern
/Pat3 exch def
<< Tile8x8
 /PaintProc {0.5 setlinewidth pop -4 8 M 8 -4 L
	0 12 M 12 0 L stroke}
>> matrix makepattern
/Pat4 exch def
<< Tile8x8
 /PaintProc {0.5 setlinewidth pop -4 0 M 8 12 L
	0 -4 M 12 8 L stroke}
>> matrix makepattern
/Pat5 exch def
<< Tile8x8
 /PaintProc {0.5 setlinewidth pop -2 8 M 4 -4 L
	0 12 M 8 -4 L 4 12 M 10 0 L stroke}
>> matrix makepattern
/Pat6 exch def
<< Tile8x8
 /PaintProc {0.5 setlinewidth pop -2 0 M 4 12 L
	0 -4 M 8 12 L 4 -4 M 10 8 L stroke}
>> matrix makepattern
/Pat7 exch def
<< Tile8x8
 /PaintProc {0.5 setlinewidth pop 8 -2 M -4 4 L
	12 0 M -4 8 L 12 4 M 0 10 L stroke}
>> matrix makepattern
/Pat8 exch def
<< Tile8x8
 /PaintProc {0.5 setlinewidth pop 0 -2 M 12 4 L
	-4 0 M 12 8 L -4 4 M 8 10 L stroke}
>> matrix makepattern
/Pat9 exch def
/Pattern1 {PatternBgnd KeepColor Pat1 setpattern} bind def
/Pattern2 {PatternBgnd KeepColor Pat2 setpattern} bind def
/Pattern3 {PatternBgnd KeepColor Pat3 setpattern} bind def
/Pattern4 {PatternBgnd KeepColor Landscape {Pat5} {Pat4} ifelse setpattern} bind def
/Pattern5 {PatternBgnd KeepColor Landscape {Pat4} {Pat5} ifelse setpattern} bind def
/Pattern6 {PatternBgnd KeepColor Landscape {Pat9} {Pat6} ifelse setpattern} bind def
/Pattern7 {PatternBgnd KeepColor Landscape {Pat8} {Pat7} ifelse setpattern} bind def
} def
%
%
%End of PostScript Level 2 code
%
/PatternBgnd {
  TransparentPatterns {} {gsave 1 setgray fill grestore} ifelse
} def
%
% Substitute for Level 2 pattern fill codes with
% grayscale if Level 2 support is not selected.
%
/Level1PatternFill {
/Pattern1 {0.250 Density} bind def
/Pattern2 {0.500 Density} bind def
/Pattern3 {0.750 Density} bind def
/Pattern4 {0.125 Density} bind def
/Pattern5 {0.375 Density} bind def
/Pattern6 {0.625 Density} bind def
/Pattern7 {0.875 Density} bind def
} def
%
% Now test for support of Level 2 code
%
Level1 {Level1PatternFill} {Level2PatternFill} ifelse
%
/Symbol-Oblique /Symbol findfont [1 0 .167 1 0 0] makefont
dup length dict begin {1 index /FID eq {pop pop} {def} ifelse} forall
currentdict end definefont pop
%
Level1 SuppressPDFMark or 
{} {
/SDict 10 dict def
systemdict /pdfmark known not {
  userdict /pdfmark systemdict /cleartomark get put
} if
SDict begin [
  /Title (plot_MeffA1b_SU8.tex)
  /Subject (gnuplot plot)
  /Creator (gnuplot 5.0 patchlevel 3)
  /Author (mteper)
%  /Producer (gnuplot)
%  /Keywords ()
  /CreationDate (Sun Nov  7 12:53:53 2021)
  /DOCINFO pdfmark
end
} ifelse
%
% Support for boxed text - Ethan A Merritt May 2005
%
/InitTextBox { userdict /TBy2 3 -1 roll put userdict /TBx2 3 -1 roll put
           userdict /TBy1 3 -1 roll put userdict /TBx1 3 -1 roll put
	   /Boxing true def } def
/ExtendTextBox { Boxing
    { gsave dup false charpath pathbbox
      dup TBy2 gt {userdict /TBy2 3 -1 roll put} {pop} ifelse
      dup TBx2 gt {userdict /TBx2 3 -1 roll put} {pop} ifelse
      dup TBy1 lt {userdict /TBy1 3 -1 roll put} {pop} ifelse
      dup TBx1 lt {userdict /TBx1 3 -1 roll put} {pop} ifelse
      grestore } if } def
/PopTextBox { newpath TBx1 TBxmargin sub TBy1 TBymargin sub M
               TBx1 TBxmargin sub TBy2 TBymargin add L
	       TBx2 TBxmargin add TBy2 TBymargin add L
	       TBx2 TBxmargin add TBy1 TBymargin sub L closepath } def
/DrawTextBox { PopTextBox stroke /Boxing false def} def
/FillTextBox { gsave PopTextBox 1 1 1 setrgbcolor fill grestore /Boxing false def} def
0 0 0 0 InitTextBox
/TBxmargin 20 def
/TBymargin 20 def
/Boxing false def
/textshow { ExtendTextBox Gshow } def
%
% redundant definitions for compatibility with prologue.ps older than 5.0.2
/LTB {BL [] LCb DL} def
/LTb {PL [] LCb DL} def
end
%%EndProlog
%%Page: 1 1
gnudict begin
gsave
doclip
0 0 translate
0.050 0.050 scale
0 setgray
newpath
BackgroundColor 0 lt 3 1 roll 0 lt exch 0 lt or or not {BackgroundColor C 1.000 0 0 7200.00 7560.00 BoxColFill} if
1.000 UL
LTb
LCb setrgbcolor
1460 640 M
63 0 V
5316 0 R
-63 0 V
stroke
LTb
LCb setrgbcolor
1460 2866 M
63 0 V
5316 0 R
-63 0 V
stroke
LTb
LCb setrgbcolor
1460 5093 M
63 0 V
5316 0 R
-63 0 V
stroke
LTb
LCb setrgbcolor
1460 7319 M
63 0 V
5316 0 R
-63 0 V
stroke
LTb
LCb setrgbcolor
1460 640 M
0 63 V
0 6616 R
0 -63 V
stroke
LTb
LCb setrgbcolor
1998 640 M
0 63 V
0 6616 R
0 -63 V
stroke
LTb
LCb setrgbcolor
2536 640 M
0 63 V
0 6616 R
0 -63 V
stroke
LTb
LCb setrgbcolor
3074 640 M
0 63 V
0 6616 R
0 -63 V
stroke
LTb
LCb setrgbcolor
3612 640 M
0 63 V
0 6616 R
0 -63 V
stroke
LTb
LCb setrgbcolor
4150 640 M
0 63 V
0 6616 R
0 -63 V
stroke
LTb
LCb setrgbcolor
4687 640 M
0 63 V
0 6616 R
0 -63 V
stroke
LTb
LCb setrgbcolor
5225 640 M
0 63 V
0 6616 R
0 -63 V
stroke
LTb
LCb setrgbcolor
5763 640 M
0 63 V
0 6616 R
0 -63 V
stroke
LTb
LCb setrgbcolor
6301 640 M
0 63 V
0 6616 R
0 -63 V
stroke
LTb
LCb setrgbcolor
6839 640 M
0 63 V
0 6616 R
0 -63 V
stroke
LTb
LCb setrgbcolor
1.000 UL
LTb
LCb setrgbcolor
1460 7319 N
0 -6679 V
5379 0 V
0 6679 V
-5379 0 V
Z stroke
1.000 UP
1.000 UL
LTb
LCb setrgbcolor
LCb setrgbcolor
LTb
LCb setrgbcolor
LTb
1.500 UP
1.000 UL
LTb
0.58 0.00 0.83 C 1729 6605 M
0 208 V
2267 4031 M
0 271 V
538 -988 R
0 434 V
538 -601 R
0 588 V
538 -96 R
0 859 V
4418 3073 M
0 1412 V
538 -541 R
0 2789 V
5494 4809 M
0 2510 V
6032 2210 M
0 5109 V
1729 6709 CircleF
2267 4167 CircleF
2805 3531 CircleF
3343 3441 CircleF
3881 4069 CircleF
4418 3779 CircleF
4956 5339 CircleF
5494 6818 CircleF
6032 5723 CircleF
2.500 UL
LTb
0.58 0.00 0.83 C 1460 3701 M
54 0 V
55 0 V
54 0 V
54 0 V
55 0 V
54 0 V
54 0 V
55 0 V
54 0 V
54 0 V
55 0 V
54 0 V
54 0 V
55 0 V
54 0 V
54 0 V
55 0 V
54 0 V
54 0 V
55 0 V
54 0 V
54 0 V
55 0 V
54 0 V
54 0 V
55 0 V
54 0 V
54 0 V
55 0 V
54 0 V
54 0 V
55 0 V
54 0 V
54 0 V
55 0 V
54 0 V
54 0 V
55 0 V
54 0 V
54 0 V
55 0 V
54 0 V
54 0 V
55 0 V
54 0 V
54 0 V
55 0 V
54 0 V
54 0 V
55 0 V
54 0 V
54 0 V
55 0 V
54 0 V
54 0 V
55 0 V
54 0 V
54 0 V
55 0 V
54 0 V
54 0 V
55 0 V
54 0 V
54 0 V
55 0 V
54 0 V
54 0 V
55 0 V
54 0 V
54 0 V
55 0 V
54 0 V
54 0 V
55 0 V
54 0 V
54 0 V
55 0 V
54 0 V
54 0 V
55 0 V
54 0 V
54 0 V
55 0 V
54 0 V
54 0 V
55 0 V
54 0 V
54 0 V
55 0 V
54 0 V
54 0 V
55 0 V
54 0 V
54 0 V
55 0 V
54 0 V
54 0 V
55 0 V
54 0 V
stroke
LTb
LT1
0.34 0.71 0.91 C 1460 4013 M
54 0 V
55 0 V
54 0 V
54 0 V
55 0 V
54 0 V
54 0 V
55 0 V
54 0 V
54 0 V
55 0 V
54 0 V
54 0 V
55 0 V
54 0 V
54 0 V
55 0 V
54 0 V
54 0 V
55 0 V
54 0 V
54 0 V
55 0 V
54 0 V
54 0 V
55 0 V
54 0 V
54 0 V
55 0 V
54 0 V
54 0 V
55 0 V
54 0 V
54 0 V
55 0 V
54 0 V
54 0 V
55 0 V
54 0 V
54 0 V
55 0 V
54 0 V
54 0 V
55 0 V
54 0 V
54 0 V
55 0 V
54 0 V
54 0 V
55 0 V
54 0 V
54 0 V
55 0 V
54 0 V
54 0 V
55 0 V
54 0 V
54 0 V
55 0 V
54 0 V
54 0 V
55 0 V
54 0 V
54 0 V
55 0 V
54 0 V
54 0 V
55 0 V
54 0 V
54 0 V
55 0 V
54 0 V
54 0 V
55 0 V
54 0 V
54 0 V
55 0 V
54 0 V
54 0 V
55 0 V
54 0 V
54 0 V
55 0 V
54 0 V
54 0 V
55 0 V
54 0 V
54 0 V
55 0 V
54 0 V
54 0 V
55 0 V
54 0 V
54 0 V
55 0 V
54 0 V
54 0 V
55 0 V
54 0 V
stroke
LTb
LT1
0.90 0.62 0.00 C 1460 3390 M
54 0 V
55 0 V
54 0 V
54 0 V
55 0 V
54 0 V
54 0 V
55 0 V
54 0 V
54 0 V
55 0 V
54 0 V
54 0 V
55 0 V
54 0 V
54 0 V
55 0 V
54 0 V
54 0 V
55 0 V
54 0 V
54 0 V
55 0 V
54 0 V
54 0 V
55 0 V
54 0 V
54 0 V
55 0 V
54 0 V
54 0 V
55 0 V
54 0 V
54 0 V
55 0 V
54 0 V
54 0 V
55 0 V
54 0 V
54 0 V
55 0 V
54 0 V
54 0 V
55 0 V
54 0 V
54 0 V
55 0 V
54 0 V
54 0 V
55 0 V
54 0 V
54 0 V
55 0 V
54 0 V
54 0 V
55 0 V
54 0 V
54 0 V
55 0 V
54 0 V
54 0 V
55 0 V
54 0 V
54 0 V
55 0 V
54 0 V
54 0 V
55 0 V
54 0 V
54 0 V
55 0 V
54 0 V
54 0 V
55 0 V
54 0 V
54 0 V
55 0 V
54 0 V
54 0 V
55 0 V
54 0 V
54 0 V
55 0 V
54 0 V
54 0 V
55 0 V
54 0 V
54 0 V
55 0 V
54 0 V
54 0 V
55 0 V
54 0 V
54 0 V
55 0 V
54 0 V
54 0 V
55 0 V
54 0 V
stroke
2.000 UL
LTb
LCb setrgbcolor
1.000 UL
LTb
LCb setrgbcolor
1460 7319 N
0 -6679 V
5379 0 V
0 6679 V
-5379 0 V
Z stroke
1.000 UP
1.000 UL
LTb
LCb setrgbcolor
stroke
grestore
end
showpage
  }}%
  \put(4149,140){\makebox(0,0){\large{$n_t$}}}%
  \put(160,4979){\makebox(0,0){\Large{$aM_{eff}^{A_1^{++}}$}}}%
  \put(6839,440){\makebox(0,0){\strut{}\ {$10$}}}%
  \put(6301,440){\makebox(0,0){\strut{}\ {$9$}}}%
  \put(5763,440){\makebox(0,0){\strut{}\ {$8$}}}%
  \put(5225,440){\makebox(0,0){\strut{}\ {$7$}}}%
  \put(4687,440){\makebox(0,0){\strut{}\ {$6$}}}%
  \put(4150,440){\makebox(0,0){\strut{}\ {$5$}}}%
  \put(3612,440){\makebox(0,0){\strut{}\ {$4$}}}%
  \put(3074,440){\makebox(0,0){\strut{}\ {$3$}}}%
  \put(2536,440){\makebox(0,0){\strut{}\ {$2$}}}%
  \put(1998,440){\makebox(0,0){\strut{}\ {$1$}}}%
  \put(1460,440){\makebox(0,0){\strut{}\ {$0$}}}%
  \put(1340,7319){\makebox(0,0)[r]{\strut{}\ \ {$0.44$}}}%
  \put(1340,5093){\makebox(0,0)[r]{\strut{}\ \ {$0.42$}}}%
  \put(1340,2866){\makebox(0,0)[r]{\strut{}\ \ {$0.4$}}}%
  \put(1340,640){\makebox(0,0)[r]{\strut{}\ \ {$0.38$}}}%
\end{picture}%
\endgroup
\endinput

\end	{center}
\caption{Effective mass plot for the ground state $A_1^{++}$ ($\bullet$) 
  on a $20^330$ lattice at $\beta=47.75$ in $SU(8)$, as in Fig.\ref{fig_MeffA1_SU8},
  but rescaled so as to expose the errors on the effective masses. The straight line
  is the best estimate for the mass obtained by fitting the correlation function, and
  the two dashed lines bound the $\pm 1$ standard deviation error band on this mass.}
\label{fig_MeffA1b_SU8}
\end{figure}


\begin{figure}[htb]
\begin	{center}
\leavevmode
% GNUPLOT: LaTeX picture with Postscript
\begingroup%
\makeatletter%
\newcommand{\GNUPLOTspecial}{%
  \@sanitize\catcode`\%=14\relax\special}%
\setlength{\unitlength}{0.0500bp}%
\begin{picture}(7200,7560)(0,0)%
  {\GNUPLOTspecial{"
%!PS-Adobe-2.0 EPSF-2.0
%%Title: plot_MeffET2_SU8.tex
%%Creator: gnuplot 5.0 patchlevel 3
%%CreationDate: Fri Mar  5 17:32:40 2021
%%DocumentFonts: 
%%BoundingBox: 0 0 360 378
%%EndComments
%%BeginProlog
/gnudict 256 dict def
gnudict begin
%
% The following true/false flags may be edited by hand if desired.
% The unit line width and grayscale image gamma correction may also be changed.
%
/Color true def
/Blacktext true def
/Solid false def
/Dashlength 1 def
/Landscape false def
/Level1 false def
/Level3 false def
/Rounded false def
/ClipToBoundingBox false def
/SuppressPDFMark false def
/TransparentPatterns false def
/gnulinewidth 5.000 def
/userlinewidth gnulinewidth def
/Gamma 1.0 def
/BackgroundColor {-1.000 -1.000 -1.000} def
%
/vshift -66 def
/dl1 {
  10.0 Dashlength userlinewidth gnulinewidth div mul mul mul
  Rounded { currentlinewidth 0.75 mul sub dup 0 le { pop 0.01 } if } if
} def
/dl2 {
  10.0 Dashlength userlinewidth gnulinewidth div mul mul mul
  Rounded { currentlinewidth 0.75 mul add } if
} def
/hpt_ 31.5 def
/vpt_ 31.5 def
/hpt hpt_ def
/vpt vpt_ def
/doclip {
  ClipToBoundingBox {
    newpath 0 0 moveto 360 0 lineto 360 378 lineto 0 378 lineto closepath
    clip
  } if
} def
%
% Gnuplot Prolog Version 5.1 (Oct 2015)
%
%/SuppressPDFMark true def
%
/M {moveto} bind def
/L {lineto} bind def
/R {rmoveto} bind def
/V {rlineto} bind def
/N {newpath moveto} bind def
/Z {closepath} bind def
/C {setrgbcolor} bind def
/f {rlineto fill} bind def
/g {setgray} bind def
/Gshow {show} def   % May be redefined later in the file to support UTF-8
/vpt2 vpt 2 mul def
/hpt2 hpt 2 mul def
/Lshow {currentpoint stroke M 0 vshift R 
	Blacktext {gsave 0 setgray textshow grestore} {textshow} ifelse} def
/Rshow {currentpoint stroke M dup stringwidth pop neg vshift R
	Blacktext {gsave 0 setgray textshow grestore} {textshow} ifelse} def
/Cshow {currentpoint stroke M dup stringwidth pop -2 div vshift R 
	Blacktext {gsave 0 setgray textshow grestore} {textshow} ifelse} def
/UP {dup vpt_ mul /vpt exch def hpt_ mul /hpt exch def
  /hpt2 hpt 2 mul def /vpt2 vpt 2 mul def} def
/DL {Color {setrgbcolor Solid {pop []} if 0 setdash}
 {pop pop pop 0 setgray Solid {pop []} if 0 setdash} ifelse} def
/BL {stroke userlinewidth 2 mul setlinewidth
	Rounded {1 setlinejoin 1 setlinecap} if} def
/AL {stroke userlinewidth 2 div setlinewidth
	Rounded {1 setlinejoin 1 setlinecap} if} def
/UL {dup gnulinewidth mul /userlinewidth exch def
	dup 1 lt {pop 1} if 10 mul /udl exch def} def
/PL {stroke userlinewidth setlinewidth
	Rounded {1 setlinejoin 1 setlinecap} if} def
3.8 setmiterlimit
% Classic Line colors (version 5.0)
/LCw {1 1 1} def
/LCb {0 0 0} def
/LCa {0 0 0} def
/LC0 {1 0 0} def
/LC1 {0 1 0} def
/LC2 {0 0 1} def
/LC3 {1 0 1} def
/LC4 {0 1 1} def
/LC5 {1 1 0} def
/LC6 {0 0 0} def
/LC7 {1 0.3 0} def
/LC8 {0.5 0.5 0.5} def
% Default dash patterns (version 5.0)
/LTB {BL [] LCb DL} def
/LTw {PL [] 1 setgray} def
/LTb {PL [] LCb DL} def
/LTa {AL [1 udl mul 2 udl mul] 0 setdash LCa setrgbcolor} def
/LT0 {PL [] LC0 DL} def
/LT1 {PL [2 dl1 3 dl2] LC1 DL} def
/LT2 {PL [1 dl1 1.5 dl2] LC2 DL} def
/LT3 {PL [6 dl1 2 dl2 1 dl1 2 dl2] LC3 DL} def
/LT4 {PL [1 dl1 2 dl2 6 dl1 2 dl2 1 dl1 2 dl2] LC4 DL} def
/LT5 {PL [4 dl1 2 dl2] LC5 DL} def
/LT6 {PL [1.5 dl1 1.5 dl2 1.5 dl1 1.5 dl2 1.5 dl1 6 dl2] LC6 DL} def
/LT7 {PL [3 dl1 3 dl2 1 dl1 3 dl2] LC7 DL} def
/LT8 {PL [2 dl1 2 dl2 2 dl1 6 dl2] LC8 DL} def
/SL {[] 0 setdash} def
/Pnt {stroke [] 0 setdash gsave 1 setlinecap M 0 0 V stroke grestore} def
/Dia {stroke [] 0 setdash 2 copy vpt add M
  hpt neg vpt neg V hpt vpt neg V
  hpt vpt V hpt neg vpt V closepath stroke
  Pnt} def
/Pls {stroke [] 0 setdash vpt sub M 0 vpt2 V
  currentpoint stroke M
  hpt neg vpt neg R hpt2 0 V stroke
 } def
/Box {stroke [] 0 setdash 2 copy exch hpt sub exch vpt add M
  0 vpt2 neg V hpt2 0 V 0 vpt2 V
  hpt2 neg 0 V closepath stroke
  Pnt} def
/Crs {stroke [] 0 setdash exch hpt sub exch vpt add M
  hpt2 vpt2 neg V currentpoint stroke M
  hpt2 neg 0 R hpt2 vpt2 V stroke} def
/TriU {stroke [] 0 setdash 2 copy vpt 1.12 mul add M
  hpt neg vpt -1.62 mul V
  hpt 2 mul 0 V
  hpt neg vpt 1.62 mul V closepath stroke
  Pnt} def
/Star {2 copy Pls Crs} def
/BoxF {stroke [] 0 setdash exch hpt sub exch vpt add M
  0 vpt2 neg V hpt2 0 V 0 vpt2 V
  hpt2 neg 0 V closepath fill} def
/TriUF {stroke [] 0 setdash vpt 1.12 mul add M
  hpt neg vpt -1.62 mul V
  hpt 2 mul 0 V
  hpt neg vpt 1.62 mul V closepath fill} def
/TriD {stroke [] 0 setdash 2 copy vpt 1.12 mul sub M
  hpt neg vpt 1.62 mul V
  hpt 2 mul 0 V
  hpt neg vpt -1.62 mul V closepath stroke
  Pnt} def
/TriDF {stroke [] 0 setdash vpt 1.12 mul sub M
  hpt neg vpt 1.62 mul V
  hpt 2 mul 0 V
  hpt neg vpt -1.62 mul V closepath fill} def
/DiaF {stroke [] 0 setdash vpt add M
  hpt neg vpt neg V hpt vpt neg V
  hpt vpt V hpt neg vpt V closepath fill} def
/Pent {stroke [] 0 setdash 2 copy gsave
  translate 0 hpt M 4 {72 rotate 0 hpt L} repeat
  closepath stroke grestore Pnt} def
/PentF {stroke [] 0 setdash gsave
  translate 0 hpt M 4 {72 rotate 0 hpt L} repeat
  closepath fill grestore} def
/Circle {stroke [] 0 setdash 2 copy
  hpt 0 360 arc stroke Pnt} def
/CircleF {stroke [] 0 setdash hpt 0 360 arc fill} def
/C0 {BL [] 0 setdash 2 copy moveto vpt 90 450 arc} bind def
/C1 {BL [] 0 setdash 2 copy moveto
	2 copy vpt 0 90 arc closepath fill
	vpt 0 360 arc closepath} bind def
/C2 {BL [] 0 setdash 2 copy moveto
	2 copy vpt 90 180 arc closepath fill
	vpt 0 360 arc closepath} bind def
/C3 {BL [] 0 setdash 2 copy moveto
	2 copy vpt 0 180 arc closepath fill
	vpt 0 360 arc closepath} bind def
/C4 {BL [] 0 setdash 2 copy moveto
	2 copy vpt 180 270 arc closepath fill
	vpt 0 360 arc closepath} bind def
/C5 {BL [] 0 setdash 2 copy moveto
	2 copy vpt 0 90 arc
	2 copy moveto
	2 copy vpt 180 270 arc closepath fill
	vpt 0 360 arc} bind def
/C6 {BL [] 0 setdash 2 copy moveto
	2 copy vpt 90 270 arc closepath fill
	vpt 0 360 arc closepath} bind def
/C7 {BL [] 0 setdash 2 copy moveto
	2 copy vpt 0 270 arc closepath fill
	vpt 0 360 arc closepath} bind def
/C8 {BL [] 0 setdash 2 copy moveto
	2 copy vpt 270 360 arc closepath fill
	vpt 0 360 arc closepath} bind def
/C9 {BL [] 0 setdash 2 copy moveto
	2 copy vpt 270 450 arc closepath fill
	vpt 0 360 arc closepath} bind def
/C10 {BL [] 0 setdash 2 copy 2 copy moveto vpt 270 360 arc closepath fill
	2 copy moveto
	2 copy vpt 90 180 arc closepath fill
	vpt 0 360 arc closepath} bind def
/C11 {BL [] 0 setdash 2 copy moveto
	2 copy vpt 0 180 arc closepath fill
	2 copy moveto
	2 copy vpt 270 360 arc closepath fill
	vpt 0 360 arc closepath} bind def
/C12 {BL [] 0 setdash 2 copy moveto
	2 copy vpt 180 360 arc closepath fill
	vpt 0 360 arc closepath} bind def
/C13 {BL [] 0 setdash 2 copy moveto
	2 copy vpt 0 90 arc closepath fill
	2 copy moveto
	2 copy vpt 180 360 arc closepath fill
	vpt 0 360 arc closepath} bind def
/C14 {BL [] 0 setdash 2 copy moveto
	2 copy vpt 90 360 arc closepath fill
	vpt 0 360 arc} bind def
/C15 {BL [] 0 setdash 2 copy vpt 0 360 arc closepath fill
	vpt 0 360 arc closepath} bind def
/Rec {newpath 4 2 roll moveto 1 index 0 rlineto 0 exch rlineto
	neg 0 rlineto closepath} bind def
/Square {dup Rec} bind def
/Bsquare {vpt sub exch vpt sub exch vpt2 Square} bind def
/S0 {BL [] 0 setdash 2 copy moveto 0 vpt rlineto BL Bsquare} bind def
/S1 {BL [] 0 setdash 2 copy vpt Square fill Bsquare} bind def
/S2 {BL [] 0 setdash 2 copy exch vpt sub exch vpt Square fill Bsquare} bind def
/S3 {BL [] 0 setdash 2 copy exch vpt sub exch vpt2 vpt Rec fill Bsquare} bind def
/S4 {BL [] 0 setdash 2 copy exch vpt sub exch vpt sub vpt Square fill Bsquare} bind def
/S5 {BL [] 0 setdash 2 copy 2 copy vpt Square fill
	exch vpt sub exch vpt sub vpt Square fill Bsquare} bind def
/S6 {BL [] 0 setdash 2 copy exch vpt sub exch vpt sub vpt vpt2 Rec fill Bsquare} bind def
/S7 {BL [] 0 setdash 2 copy exch vpt sub exch vpt sub vpt vpt2 Rec fill
	2 copy vpt Square fill Bsquare} bind def
/S8 {BL [] 0 setdash 2 copy vpt sub vpt Square fill Bsquare} bind def
/S9 {BL [] 0 setdash 2 copy vpt sub vpt vpt2 Rec fill Bsquare} bind def
/S10 {BL [] 0 setdash 2 copy vpt sub vpt Square fill 2 copy exch vpt sub exch vpt Square fill
	Bsquare} bind def
/S11 {BL [] 0 setdash 2 copy vpt sub vpt Square fill 2 copy exch vpt sub exch vpt2 vpt Rec fill
	Bsquare} bind def
/S12 {BL [] 0 setdash 2 copy exch vpt sub exch vpt sub vpt2 vpt Rec fill Bsquare} bind def
/S13 {BL [] 0 setdash 2 copy exch vpt sub exch vpt sub vpt2 vpt Rec fill
	2 copy vpt Square fill Bsquare} bind def
/S14 {BL [] 0 setdash 2 copy exch vpt sub exch vpt sub vpt2 vpt Rec fill
	2 copy exch vpt sub exch vpt Square fill Bsquare} bind def
/S15 {BL [] 0 setdash 2 copy Bsquare fill Bsquare} bind def
/D0 {gsave translate 45 rotate 0 0 S0 stroke grestore} bind def
/D1 {gsave translate 45 rotate 0 0 S1 stroke grestore} bind def
/D2 {gsave translate 45 rotate 0 0 S2 stroke grestore} bind def
/D3 {gsave translate 45 rotate 0 0 S3 stroke grestore} bind def
/D4 {gsave translate 45 rotate 0 0 S4 stroke grestore} bind def
/D5 {gsave translate 45 rotate 0 0 S5 stroke grestore} bind def
/D6 {gsave translate 45 rotate 0 0 S6 stroke grestore} bind def
/D7 {gsave translate 45 rotate 0 0 S7 stroke grestore} bind def
/D8 {gsave translate 45 rotate 0 0 S8 stroke grestore} bind def
/D9 {gsave translate 45 rotate 0 0 S9 stroke grestore} bind def
/D10 {gsave translate 45 rotate 0 0 S10 stroke grestore} bind def
/D11 {gsave translate 45 rotate 0 0 S11 stroke grestore} bind def
/D12 {gsave translate 45 rotate 0 0 S12 stroke grestore} bind def
/D13 {gsave translate 45 rotate 0 0 S13 stroke grestore} bind def
/D14 {gsave translate 45 rotate 0 0 S14 stroke grestore} bind def
/D15 {gsave translate 45 rotate 0 0 S15 stroke grestore} bind def
/DiaE {stroke [] 0 setdash vpt add M
  hpt neg vpt neg V hpt vpt neg V
  hpt vpt V hpt neg vpt V closepath stroke} def
/BoxE {stroke [] 0 setdash exch hpt sub exch vpt add M
  0 vpt2 neg V hpt2 0 V 0 vpt2 V
  hpt2 neg 0 V closepath stroke} def
/TriUE {stroke [] 0 setdash vpt 1.12 mul add M
  hpt neg vpt -1.62 mul V
  hpt 2 mul 0 V
  hpt neg vpt 1.62 mul V closepath stroke} def
/TriDE {stroke [] 0 setdash vpt 1.12 mul sub M
  hpt neg vpt 1.62 mul V
  hpt 2 mul 0 V
  hpt neg vpt -1.62 mul V closepath stroke} def
/PentE {stroke [] 0 setdash gsave
  translate 0 hpt M 4 {72 rotate 0 hpt L} repeat
  closepath stroke grestore} def
/CircE {stroke [] 0 setdash 
  hpt 0 360 arc stroke} def
/Opaque {gsave closepath 1 setgray fill grestore 0 setgray closepath} def
/DiaW {stroke [] 0 setdash vpt add M
  hpt neg vpt neg V hpt vpt neg V
  hpt vpt V hpt neg vpt V Opaque stroke} def
/BoxW {stroke [] 0 setdash exch hpt sub exch vpt add M
  0 vpt2 neg V hpt2 0 V 0 vpt2 V
  hpt2 neg 0 V Opaque stroke} def
/TriUW {stroke [] 0 setdash vpt 1.12 mul add M
  hpt neg vpt -1.62 mul V
  hpt 2 mul 0 V
  hpt neg vpt 1.62 mul V Opaque stroke} def
/TriDW {stroke [] 0 setdash vpt 1.12 mul sub M
  hpt neg vpt 1.62 mul V
  hpt 2 mul 0 V
  hpt neg vpt -1.62 mul V Opaque stroke} def
/PentW {stroke [] 0 setdash gsave
  translate 0 hpt M 4 {72 rotate 0 hpt L} repeat
  Opaque stroke grestore} def
/CircW {stroke [] 0 setdash 
  hpt 0 360 arc Opaque stroke} def
/BoxFill {gsave Rec 1 setgray fill grestore} def
/Density {
  /Fillden exch def
  currentrgbcolor
  /ColB exch def /ColG exch def /ColR exch def
  /ColR ColR Fillden mul Fillden sub 1 add def
  /ColG ColG Fillden mul Fillden sub 1 add def
  /ColB ColB Fillden mul Fillden sub 1 add def
  ColR ColG ColB setrgbcolor} def
/BoxColFill {gsave Rec PolyFill} def
/PolyFill {gsave Density fill grestore grestore} def
/h {rlineto rlineto rlineto gsave closepath fill grestore} bind def
%
% PostScript Level 1 Pattern Fill routine for rectangles
% Usage: x y w h s a XX PatternFill
%	x,y = lower left corner of box to be filled
%	w,h = width and height of box
%	  a = angle in degrees between lines and x-axis
%	 XX = 0/1 for no/yes cross-hatch
%
/PatternFill {gsave /PFa [ 9 2 roll ] def
  PFa 0 get PFa 2 get 2 div add PFa 1 get PFa 3 get 2 div add translate
  PFa 2 get -2 div PFa 3 get -2 div PFa 2 get PFa 3 get Rec
  TransparentPatterns {} {gsave 1 setgray fill grestore} ifelse
  clip
  currentlinewidth 0.5 mul setlinewidth
  /PFs PFa 2 get dup mul PFa 3 get dup mul add sqrt def
  0 0 M PFa 5 get rotate PFs -2 div dup translate
  0 1 PFs PFa 4 get div 1 add floor cvi
	{PFa 4 get mul 0 M 0 PFs V} for
  0 PFa 6 get ne {
	0 1 PFs PFa 4 get div 1 add floor cvi
	{PFa 4 get mul 0 2 1 roll M PFs 0 V} for
 } if
  stroke grestore} def
%
/languagelevel where
 {pop languagelevel} {1} ifelse
dup 2 lt
	{/InterpretLevel1 true def
	 /InterpretLevel3 false def}
	{/InterpretLevel1 Level1 def
	 2 gt
	    {/InterpretLevel3 Level3 def}
	    {/InterpretLevel3 false def}
	 ifelse }
 ifelse
%
% PostScript level 2 pattern fill definitions
%
/Level2PatternFill {
/Tile8x8 {/PaintType 2 /PatternType 1 /TilingType 1 /BBox [0 0 8 8] /XStep 8 /YStep 8}
	bind def
/KeepColor {currentrgbcolor [/Pattern /DeviceRGB] setcolorspace} bind def
<< Tile8x8
 /PaintProc {0.5 setlinewidth pop 0 0 M 8 8 L 0 8 M 8 0 L stroke} 
>> matrix makepattern
/Pat1 exch def
<< Tile8x8
 /PaintProc {0.5 setlinewidth pop 0 0 M 8 8 L 0 8 M 8 0 L stroke
	0 4 M 4 8 L 8 4 L 4 0 L 0 4 L stroke}
>> matrix makepattern
/Pat2 exch def
<< Tile8x8
 /PaintProc {0.5 setlinewidth pop 0 0 M 0 8 L
	8 8 L 8 0 L 0 0 L fill}
>> matrix makepattern
/Pat3 exch def
<< Tile8x8
 /PaintProc {0.5 setlinewidth pop -4 8 M 8 -4 L
	0 12 M 12 0 L stroke}
>> matrix makepattern
/Pat4 exch def
<< Tile8x8
 /PaintProc {0.5 setlinewidth pop -4 0 M 8 12 L
	0 -4 M 12 8 L stroke}
>> matrix makepattern
/Pat5 exch def
<< Tile8x8
 /PaintProc {0.5 setlinewidth pop -2 8 M 4 -4 L
	0 12 M 8 -4 L 4 12 M 10 0 L stroke}
>> matrix makepattern
/Pat6 exch def
<< Tile8x8
 /PaintProc {0.5 setlinewidth pop -2 0 M 4 12 L
	0 -4 M 8 12 L 4 -4 M 10 8 L stroke}
>> matrix makepattern
/Pat7 exch def
<< Tile8x8
 /PaintProc {0.5 setlinewidth pop 8 -2 M -4 4 L
	12 0 M -4 8 L 12 4 M 0 10 L stroke}
>> matrix makepattern
/Pat8 exch def
<< Tile8x8
 /PaintProc {0.5 setlinewidth pop 0 -2 M 12 4 L
	-4 0 M 12 8 L -4 4 M 8 10 L stroke}
>> matrix makepattern
/Pat9 exch def
/Pattern1 {PatternBgnd KeepColor Pat1 setpattern} bind def
/Pattern2 {PatternBgnd KeepColor Pat2 setpattern} bind def
/Pattern3 {PatternBgnd KeepColor Pat3 setpattern} bind def
/Pattern4 {PatternBgnd KeepColor Landscape {Pat5} {Pat4} ifelse setpattern} bind def
/Pattern5 {PatternBgnd KeepColor Landscape {Pat4} {Pat5} ifelse setpattern} bind def
/Pattern6 {PatternBgnd KeepColor Landscape {Pat9} {Pat6} ifelse setpattern} bind def
/Pattern7 {PatternBgnd KeepColor Landscape {Pat8} {Pat7} ifelse setpattern} bind def
} def
%
%
%End of PostScript Level 2 code
%
/PatternBgnd {
  TransparentPatterns {} {gsave 1 setgray fill grestore} ifelse
} def
%
% Substitute for Level 2 pattern fill codes with
% grayscale if Level 2 support is not selected.
%
/Level1PatternFill {
/Pattern1 {0.250 Density} bind def
/Pattern2 {0.500 Density} bind def
/Pattern3 {0.750 Density} bind def
/Pattern4 {0.125 Density} bind def
/Pattern5 {0.375 Density} bind def
/Pattern6 {0.625 Density} bind def
/Pattern7 {0.875 Density} bind def
} def
%
% Now test for support of Level 2 code
%
Level1 {Level1PatternFill} {Level2PatternFill} ifelse
%
/Symbol-Oblique /Symbol findfont [1 0 .167 1 0 0] makefont
dup length dict begin {1 index /FID eq {pop pop} {def} ifelse} forall
currentdict end definefont pop
%
Level1 SuppressPDFMark or 
{} {
/SDict 10 dict def
systemdict /pdfmark known not {
  userdict /pdfmark systemdict /cleartomark get put
} if
SDict begin [
  /Title (plot_MeffET2_SU8.tex)
  /Subject (gnuplot plot)
  /Creator (gnuplot 5.0 patchlevel 3)
  /Author (mteper)
%  /Producer (gnuplot)
%  /Keywords ()
  /CreationDate (Fri Mar  5 17:32:40 2021)
  /DOCINFO pdfmark
end
} ifelse
%
% Support for boxed text - Ethan A Merritt May 2005
%
/InitTextBox { userdict /TBy2 3 -1 roll put userdict /TBx2 3 -1 roll put
           userdict /TBy1 3 -1 roll put userdict /TBx1 3 -1 roll put
	   /Boxing true def } def
/ExtendTextBox { Boxing
    { gsave dup false charpath pathbbox
      dup TBy2 gt {userdict /TBy2 3 -1 roll put} {pop} ifelse
      dup TBx2 gt {userdict /TBx2 3 -1 roll put} {pop} ifelse
      dup TBy1 lt {userdict /TBy1 3 -1 roll put} {pop} ifelse
      dup TBx1 lt {userdict /TBx1 3 -1 roll put} {pop} ifelse
      grestore } if } def
/PopTextBox { newpath TBx1 TBxmargin sub TBy1 TBymargin sub M
               TBx1 TBxmargin sub TBy2 TBymargin add L
	       TBx2 TBxmargin add TBy2 TBymargin add L
	       TBx2 TBxmargin add TBy1 TBymargin sub L closepath } def
/DrawTextBox { PopTextBox stroke /Boxing false def} def
/FillTextBox { gsave PopTextBox 1 1 1 setrgbcolor fill grestore /Boxing false def} def
0 0 0 0 InitTextBox
/TBxmargin 20 def
/TBymargin 20 def
/Boxing false def
/textshow { ExtendTextBox Gshow } def
%
% redundant definitions for compatibility with prologue.ps older than 5.0.2
/LTB {BL [] LCb DL} def
/LTb {PL [] LCb DL} def
end
%%EndProlog
%%Page: 1 1
gnudict begin
gsave
doclip
0 0 translate
0.050 0.050 scale
0 setgray
newpath
BackgroundColor 0 lt 3 1 roll 0 lt exch 0 lt or or not {BackgroundColor C 1.000 0 0 7200.00 7560.00 BoxColFill} if
1.000 UL
LTb
LCb setrgbcolor
1340 640 M
63 0 V
5436 0 R
-63 0 V
stroke
LTb
LCb setrgbcolor
1340 1308 M
63 0 V
5436 0 R
-63 0 V
stroke
LTb
LCb setrgbcolor
1340 1976 M
63 0 V
5436 0 R
-63 0 V
stroke
LTb
LCb setrgbcolor
1340 2644 M
63 0 V
5436 0 R
-63 0 V
stroke
LTb
LCb setrgbcolor
1340 3312 M
63 0 V
5436 0 R
-63 0 V
stroke
LTb
LCb setrgbcolor
1340 3979 M
63 0 V
5436 0 R
-63 0 V
stroke
LTb
LCb setrgbcolor
1340 4647 M
63 0 V
5436 0 R
-63 0 V
stroke
LTb
LCb setrgbcolor
1340 5315 M
63 0 V
5436 0 R
-63 0 V
stroke
LTb
LCb setrgbcolor
1340 5983 M
63 0 V
5436 0 R
-63 0 V
stroke
LTb
LCb setrgbcolor
1340 6651 M
63 0 V
5436 0 R
-63 0 V
stroke
LTb
LCb setrgbcolor
1340 7319 M
63 0 V
5436 0 R
-63 0 V
stroke
LTb
LCb setrgbcolor
1340 640 M
0 63 V
0 6616 R
0 -63 V
stroke
LTb
LCb setrgbcolor
2027 640 M
0 63 V
0 6616 R
0 -63 V
stroke
LTb
LCb setrgbcolor
2715 640 M
0 63 V
0 6616 R
0 -63 V
stroke
LTb
LCb setrgbcolor
3402 640 M
0 63 V
0 6616 R
0 -63 V
stroke
LTb
LCb setrgbcolor
4090 640 M
0 63 V
0 6616 R
0 -63 V
stroke
LTb
LCb setrgbcolor
4777 640 M
0 63 V
0 6616 R
0 -63 V
stroke
LTb
LCb setrgbcolor
5464 640 M
0 63 V
0 6616 R
0 -63 V
stroke
LTb
LCb setrgbcolor
6152 640 M
0 63 V
0 6616 R
0 -63 V
stroke
LTb
LCb setrgbcolor
6839 640 M
0 63 V
0 6616 R
0 -63 V
stroke
LTb
LCb setrgbcolor
1.000 UL
LTb
LCb setrgbcolor
1340 7319 N
0 -6679 V
5499 0 V
0 6679 V
-5499 0 V
Z stroke
1.000 UP
1.000 UL
LTb
LCb setrgbcolor
LCb setrgbcolor
LTb
LCb setrgbcolor
LTb
1.500 UP
1.000 UL
LTb
0.58 0.00 0.83 C 1718 2352 M
0 10 V
687 -266 R
0 22 V
688 -94 R
0 31 V
687 -56 R
0 56 V
688 -86 R
0 106 V
687 -227 R
0 221 V
687 -125 R
0 403 V
688 -570 R
0 573 V
1718 4581 M
0 20 V
687 -502 R
0 34 V
688 -297 R
0 88 V
687 -296 R
0 189 V
688 -653 R
0 480 V
687 -482 R
0 1302 V
1718 2357 CircleF
2405 2107 CircleF
3093 2040 CircleF
3780 2027 CircleF
4468 2022 CircleF
5155 1958 CircleF
5842 2145 CircleF
6530 2064 CircleF
1718 4591 CircleF
2405 4116 CircleF
3093 3880 CircleF
3780 3722 CircleF
4468 3404 CircleF
5155 3813 CircleF
1.500 UL
LTb
0.58 0.00 0.83 C 1340 2031 M
56 0 V
55 0 V
56 0 V
55 0 V
56 0 V
55 0 V
56 0 V
55 0 V
56 0 V
55 0 V
56 0 V
56 0 V
55 0 V
56 0 V
55 0 V
56 0 V
55 0 V
56 0 V
55 0 V
56 0 V
55 0 V
56 0 V
56 0 V
55 0 V
56 0 V
55 0 V
56 0 V
55 0 V
56 0 V
55 0 V
56 0 V
55 0 V
56 0 V
56 0 V
55 0 V
56 0 V
55 0 V
56 0 V
55 0 V
56 0 V
55 0 V
56 0 V
55 0 V
56 0 V
56 0 V
55 0 V
56 0 V
55 0 V
56 0 V
55 0 V
56 0 V
55 0 V
56 0 V
55 0 V
56 0 V
56 0 V
55 0 V
56 0 V
55 0 V
56 0 V
55 0 V
56 0 V
55 0 V
56 0 V
55 0 V
56 0 V
56 0 V
55 0 V
56 0 V
55 0 V
56 0 V
55 0 V
56 0 V
55 0 V
56 0 V
55 0 V
56 0 V
56 0 V
55 0 V
56 0 V
55 0 V
56 0 V
55 0 V
56 0 V
55 0 V
56 0 V
55 0 V
56 0 V
56 0 V
55 0 V
56 0 V
55 0 V
56 0 V
55 0 V
56 0 V
55 0 V
56 0 V
55 0 V
56 0 V
stroke
LTb
0.58 0.00 0.83 C 1340 3846 M
56 0 V
55 0 V
56 0 V
55 0 V
56 0 V
55 0 V
56 0 V
55 0 V
56 0 V
55 0 V
56 0 V
56 0 V
55 0 V
56 0 V
55 0 V
56 0 V
55 0 V
56 0 V
55 0 V
56 0 V
55 0 V
56 0 V
56 0 V
55 0 V
56 0 V
55 0 V
56 0 V
55 0 V
56 0 V
55 0 V
56 0 V
55 0 V
56 0 V
56 0 V
55 0 V
56 0 V
55 0 V
56 0 V
55 0 V
56 0 V
55 0 V
56 0 V
55 0 V
56 0 V
56 0 V
55 0 V
56 0 V
55 0 V
56 0 V
55 0 V
56 0 V
55 0 V
56 0 V
55 0 V
56 0 V
56 0 V
55 0 V
56 0 V
55 0 V
56 0 V
55 0 V
56 0 V
55 0 V
56 0 V
55 0 V
56 0 V
56 0 V
55 0 V
56 0 V
55 0 V
56 0 V
55 0 V
56 0 V
55 0 V
56 0 V
55 0 V
56 0 V
56 0 V
55 0 V
56 0 V
55 0 V
56 0 V
55 0 V
56 0 V
55 0 V
56 0 V
55 0 V
56 0 V
56 0 V
55 0 V
56 0 V
55 0 V
56 0 V
55 0 V
56 0 V
55 0 V
56 0 V
55 0 V
56 0 V
1.500 UP
stroke
1.000 UL
LTb
0.58 0.00 0.83 C 1649 2469 M
0 8 V
688 -263 R
0 16 V
687 -116 R
0 26 V
687 -100 R
0 46 V
688 -94 R
0 77 V
687 -127 R
0 160 V
688 -297 R
0 312 V
687 -437 R
0 423 V
1649 4468 M
0 12 V
688 -467 R
0 29 V
687 -184 R
0 64 V
687 -174 R
0 177 V
688 -533 R
0 390 V
687 -404 R
0 918 V
1649 2473 Circle
2337 2222 Circle
3024 2127 Circle
3711 2063 Circle
4399 2031 Circle
5086 2022 Circle
5774 1961 Circle
6461 1892 Circle
1649 4474 Circle
2337 4027 Circle
3024 3890 Circle
3711 3836 Circle
4399 3587 Circle
5086 3837 Circle
1.500 UL
LTb
0.58 0.00 0.83 C 1340 2099 M
56 0 V
55 0 V
56 0 V
55 0 V
56 0 V
55 0 V
56 0 V
55 0 V
56 0 V
55 0 V
56 0 V
56 0 V
55 0 V
56 0 V
55 0 V
56 0 V
55 0 V
56 0 V
55 0 V
56 0 V
55 0 V
56 0 V
56 0 V
55 0 V
56 0 V
55 0 V
56 0 V
55 0 V
56 0 V
55 0 V
56 0 V
55 0 V
56 0 V
56 0 V
55 0 V
56 0 V
55 0 V
56 0 V
55 0 V
56 0 V
55 0 V
56 0 V
55 0 V
56 0 V
56 0 V
55 0 V
56 0 V
55 0 V
56 0 V
55 0 V
56 0 V
55 0 V
56 0 V
55 0 V
56 0 V
56 0 V
55 0 V
56 0 V
55 0 V
56 0 V
55 0 V
56 0 V
55 0 V
56 0 V
55 0 V
56 0 V
56 0 V
55 0 V
56 0 V
55 0 V
56 0 V
55 0 V
56 0 V
55 0 V
56 0 V
55 0 V
56 0 V
56 0 V
55 0 V
56 0 V
55 0 V
56 0 V
55 0 V
56 0 V
55 0 V
56 0 V
55 0 V
56 0 V
56 0 V
55 0 V
56 0 V
55 0 V
56 0 V
55 0 V
56 0 V
55 0 V
56 0 V
55 0 V
56 0 V
stroke
LTb
0.58 0.00 0.83 C 1340 3846 M
56 0 V
55 0 V
56 0 V
55 0 V
56 0 V
55 0 V
56 0 V
55 0 V
56 0 V
55 0 V
56 0 V
56 0 V
55 0 V
56 0 V
55 0 V
56 0 V
55 0 V
56 0 V
55 0 V
56 0 V
55 0 V
56 0 V
56 0 V
55 0 V
56 0 V
55 0 V
56 0 V
55 0 V
56 0 V
55 0 V
56 0 V
55 0 V
56 0 V
56 0 V
55 0 V
56 0 V
55 0 V
56 0 V
55 0 V
56 0 V
55 0 V
56 0 V
55 0 V
56 0 V
56 0 V
55 0 V
56 0 V
55 0 V
56 0 V
55 0 V
56 0 V
55 0 V
56 0 V
55 0 V
56 0 V
56 0 V
55 0 V
56 0 V
55 0 V
56 0 V
55 0 V
56 0 V
55 0 V
56 0 V
55 0 V
56 0 V
56 0 V
55 0 V
56 0 V
55 0 V
56 0 V
55 0 V
56 0 V
55 0 V
56 0 V
55 0 V
56 0 V
56 0 V
55 0 V
56 0 V
55 0 V
56 0 V
55 0 V
56 0 V
55 0 V
56 0 V
55 0 V
56 0 V
56 0 V
55 0 V
56 0 V
55 0 V
56 0 V
55 0 V
56 0 V
55 0 V
56 0 V
55 0 V
56 0 V
stroke
LTb
0.58 0.00 0.83 C 1340 3779 M
56 0 V
55 0 V
56 0 V
55 0 V
56 0 V
55 0 V
56 0 V
55 0 V
56 0 V
55 0 V
56 0 V
56 0 V
55 0 V
56 0 V
55 0 V
56 0 V
55 0 V
56 0 V
55 0 V
56 0 V
55 0 V
56 0 V
56 0 V
55 0 V
56 0 V
55 0 V
56 0 V
55 0 V
56 0 V
55 0 V
56 0 V
55 0 V
56 0 V
56 0 V
55 0 V
56 0 V
55 0 V
56 0 V
55 0 V
56 0 V
55 0 V
56 0 V
55 0 V
56 0 V
56 0 V
55 0 V
56 0 V
55 0 V
56 0 V
55 0 V
56 0 V
55 0 V
56 0 V
55 0 V
56 0 V
56 0 V
55 0 V
56 0 V
55 0 V
56 0 V
55 0 V
56 0 V
55 0 V
56 0 V
55 0 V
56 0 V
56 0 V
55 0 V
56 0 V
55 0 V
56 0 V
55 0 V
56 0 V
55 0 V
56 0 V
55 0 V
56 0 V
56 0 V
55 0 V
56 0 V
55 0 V
56 0 V
55 0 V
56 0 V
55 0 V
56 0 V
55 0 V
56 0 V
56 0 V
55 0 V
56 0 V
55 0 V
56 0 V
55 0 V
56 0 V
55 0 V
56 0 V
55 0 V
56 0 V
1.500 UP
stroke
1.000 UL
LTb
0.58 0.00 0.83 C 1718 3947 M
0 17 V
687 -459 R
0 33 V
688 -207 R
0 63 V
687 -99 R
0 159 V
688 -276 R
0 407 V
687 24 R
0 917 V
1718 6425 M
0 20 V
687 -876 R
0 76 V
688 -426 R
0 148 V
687 -119 R
0 599 V
1718 3955 BoxF
2405 3522 BoxF
3093 3362 BoxF
3780 3374 BoxF
4468 3382 BoxF
5155 4068 BoxF
1718 6435 BoxF
2405 5607 BoxF
3093 5293 BoxF
3780 5547 BoxF
1.500 UL
LTb
0.58 0.00 0.83 C 1340 3366 M
56 0 V
55 0 V
56 0 V
55 0 V
56 0 V
55 0 V
56 0 V
55 0 V
56 0 V
55 0 V
56 0 V
56 0 V
55 0 V
56 0 V
55 0 V
56 0 V
55 0 V
56 0 V
55 0 V
56 0 V
55 0 V
56 0 V
56 0 V
55 0 V
56 0 V
55 0 V
56 0 V
55 0 V
56 0 V
55 0 V
56 0 V
55 0 V
56 0 V
56 0 V
55 0 V
56 0 V
55 0 V
56 0 V
55 0 V
56 0 V
55 0 V
56 0 V
55 0 V
56 0 V
56 0 V
55 0 V
56 0 V
55 0 V
56 0 V
55 0 V
56 0 V
55 0 V
56 0 V
55 0 V
56 0 V
56 0 V
55 0 V
56 0 V
55 0 V
56 0 V
55 0 V
56 0 V
55 0 V
56 0 V
55 0 V
56 0 V
56 0 V
55 0 V
56 0 V
55 0 V
56 0 V
55 0 V
56 0 V
55 0 V
56 0 V
55 0 V
56 0 V
56 0 V
55 0 V
56 0 V
55 0 V
56 0 V
55 0 V
56 0 V
55 0 V
56 0 V
55 0 V
56 0 V
56 0 V
55 0 V
56 0 V
55 0 V
56 0 V
55 0 V
56 0 V
55 0 V
56 0 V
55 0 V
56 0 V
stroke
LTb
0.58 0.00 0.83 C 1340 5322 M
56 0 V
55 0 V
56 0 V
55 0 V
56 0 V
55 0 V
56 0 V
55 0 V
56 0 V
55 0 V
56 0 V
56 0 V
55 0 V
56 0 V
55 0 V
56 0 V
55 0 V
56 0 V
55 0 V
56 0 V
55 0 V
56 0 V
56 0 V
55 0 V
56 0 V
55 0 V
56 0 V
55 0 V
56 0 V
55 0 V
56 0 V
55 0 V
56 0 V
56 0 V
55 0 V
56 0 V
55 0 V
56 0 V
55 0 V
56 0 V
55 0 V
56 0 V
55 0 V
56 0 V
56 0 V
55 0 V
56 0 V
55 0 V
56 0 V
55 0 V
56 0 V
55 0 V
56 0 V
55 0 V
56 0 V
56 0 V
55 0 V
56 0 V
55 0 V
56 0 V
55 0 V
56 0 V
55 0 V
56 0 V
55 0 V
56 0 V
56 0 V
55 0 V
56 0 V
55 0 V
56 0 V
55 0 V
56 0 V
55 0 V
56 0 V
55 0 V
56 0 V
56 0 V
55 0 V
56 0 V
55 0 V
56 0 V
55 0 V
56 0 V
55 0 V
56 0 V
55 0 V
56 0 V
56 0 V
55 0 V
56 0 V
55 0 V
56 0 V
55 0 V
56 0 V
55 0 V
56 0 V
55 0 V
56 0 V
1.500 UP
stroke
1.000 UL
LTb
0.58 0.00 0.83 C 1649 3981 M
0 12 V
688 -479 R
0 29 V
687 -162 R
0 70 V
687 -162 R
0 134 V
688 -423 R
0 202 V
687 -107 R
0 551 V
1649 6050 M
0 18 V
688 -771 R
0 42 V
687 -285 R
0 139 V
687 -455 R
0 364 V
1649 3987 Box
2337 3529 Box
3024 3416 Box
3711 3356 Box
4399 3101 Box
5086 3370 Box
1649 6059 Box
2337 5318 Box
3024 5123 Box
3711 4920 Box
1.500 UL
LTb
0.58 0.00 0.83 C 1340 3398 M
56 0 V
55 0 V
56 0 V
55 0 V
56 0 V
55 0 V
56 0 V
55 0 V
56 0 V
55 0 V
56 0 V
56 0 V
55 0 V
56 0 V
55 0 V
56 0 V
55 0 V
56 0 V
55 0 V
56 0 V
55 0 V
56 0 V
56 0 V
55 0 V
56 0 V
55 0 V
56 0 V
55 0 V
56 0 V
55 0 V
56 0 V
55 0 V
56 0 V
56 0 V
55 0 V
56 0 V
55 0 V
56 0 V
55 0 V
56 0 V
55 0 V
56 0 V
55 0 V
56 0 V
56 0 V
55 0 V
56 0 V
55 0 V
56 0 V
55 0 V
56 0 V
55 0 V
56 0 V
55 0 V
56 0 V
56 0 V
55 0 V
56 0 V
55 0 V
56 0 V
55 0 V
56 0 V
55 0 V
56 0 V
55 0 V
56 0 V
56 0 V
55 0 V
56 0 V
55 0 V
56 0 V
55 0 V
56 0 V
55 0 V
56 0 V
55 0 V
56 0 V
56 0 V
55 0 V
56 0 V
55 0 V
56 0 V
55 0 V
56 0 V
55 0 V
56 0 V
55 0 V
56 0 V
56 0 V
55 0 V
56 0 V
55 0 V
56 0 V
55 0 V
56 0 V
55 0 V
56 0 V
55 0 V
56 0 V
stroke
LTb
0.58 0.00 0.83 C 1340 5082 M
56 0 V
55 0 V
56 0 V
55 0 V
56 0 V
55 0 V
56 0 V
55 0 V
56 0 V
55 0 V
56 0 V
56 0 V
55 0 V
56 0 V
55 0 V
56 0 V
55 0 V
56 0 V
55 0 V
56 0 V
55 0 V
56 0 V
56 0 V
55 0 V
56 0 V
55 0 V
56 0 V
55 0 V
56 0 V
55 0 V
56 0 V
55 0 V
56 0 V
56 0 V
55 0 V
56 0 V
55 0 V
56 0 V
55 0 V
56 0 V
55 0 V
56 0 V
55 0 V
56 0 V
56 0 V
55 0 V
56 0 V
55 0 V
56 0 V
55 0 V
56 0 V
55 0 V
56 0 V
55 0 V
56 0 V
56 0 V
55 0 V
56 0 V
55 0 V
56 0 V
55 0 V
56 0 V
55 0 V
56 0 V
55 0 V
56 0 V
56 0 V
55 0 V
56 0 V
55 0 V
56 0 V
55 0 V
56 0 V
55 0 V
56 0 V
55 0 V
56 0 V
56 0 V
55 0 V
56 0 V
55 0 V
56 0 V
55 0 V
56 0 V
55 0 V
56 0 V
55 0 V
56 0 V
56 0 V
55 0 V
56 0 V
55 0 V
56 0 V
55 0 V
56 0 V
55 0 V
56 0 V
55 0 V
56 0 V
1.500 UP
stroke
1.000 UL
LTb
0.58 0.00 0.83 C 1718 6637 M
0 23 V
687 -823 R
0 71 V
688 -556 R
0 196 V
687 -567 R
0 749 V
1718 6648 TriDF
2405 5872 TriDF
3093 5450 TriDF
3780 5355 TriDF
1.500 UL
LTb
0.58 0.00 0.83 C 1340 5429 M
56 0 V
55 0 V
56 0 V
55 0 V
56 0 V
55 0 V
56 0 V
55 0 V
56 0 V
55 0 V
56 0 V
56 0 V
55 0 V
56 0 V
55 0 V
56 0 V
55 0 V
56 0 V
55 0 V
56 0 V
55 0 V
56 0 V
56 0 V
55 0 V
56 0 V
55 0 V
56 0 V
55 0 V
56 0 V
55 0 V
56 0 V
55 0 V
56 0 V
56 0 V
55 0 V
56 0 V
55 0 V
56 0 V
55 0 V
56 0 V
55 0 V
56 0 V
55 0 V
56 0 V
56 0 V
55 0 V
56 0 V
55 0 V
56 0 V
55 0 V
56 0 V
55 0 V
56 0 V
55 0 V
56 0 V
56 0 V
55 0 V
56 0 V
55 0 V
56 0 V
55 0 V
56 0 V
55 0 V
56 0 V
55 0 V
56 0 V
56 0 V
55 0 V
56 0 V
55 0 V
56 0 V
55 0 V
56 0 V
55 0 V
56 0 V
55 0 V
56 0 V
56 0 V
55 0 V
56 0 V
55 0 V
56 0 V
55 0 V
56 0 V
55 0 V
56 0 V
55 0 V
56 0 V
56 0 V
55 0 V
56 0 V
55 0 V
56 0 V
55 0 V
56 0 V
55 0 V
56 0 V
55 0 V
56 0 V
1.500 UP
stroke
1.000 UL
LTb
0.58 0.00 0.83 C 1649 6381 M
0 17 V
688 -769 R
0 57 V
687 -329 R
0 175 V
687 -814 R
0 524 V
1649 6389 TriD
2337 5658 TriD
3024 5445 TriD
3711 4980 TriD
1.500 UL
LTb
0.58 0.00 0.83 C 1340 5442 M
56 0 V
55 0 V
56 0 V
55 0 V
56 0 V
55 0 V
56 0 V
55 0 V
56 0 V
55 0 V
56 0 V
56 0 V
55 0 V
56 0 V
55 0 V
56 0 V
55 0 V
56 0 V
55 0 V
56 0 V
55 0 V
56 0 V
56 0 V
55 0 V
56 0 V
55 0 V
56 0 V
55 0 V
56 0 V
55 0 V
56 0 V
55 0 V
56 0 V
56 0 V
55 0 V
56 0 V
55 0 V
56 0 V
55 0 V
56 0 V
55 0 V
56 0 V
55 0 V
56 0 V
56 0 V
55 0 V
56 0 V
55 0 V
56 0 V
55 0 V
56 0 V
55 0 V
56 0 V
55 0 V
56 0 V
56 0 V
55 0 V
56 0 V
55 0 V
56 0 V
55 0 V
56 0 V
55 0 V
56 0 V
55 0 V
56 0 V
56 0 V
55 0 V
56 0 V
55 0 V
56 0 V
55 0 V
56 0 V
55 0 V
56 0 V
55 0 V
56 0 V
56 0 V
55 0 V
56 0 V
55 0 V
56 0 V
55 0 V
56 0 V
55 0 V
56 0 V
55 0 V
56 0 V
56 0 V
55 0 V
56 0 V
55 0 V
56 0 V
55 0 V
56 0 V
55 0 V
56 0 V
55 0 V
56 0 V
1.500 UP
stroke
1.000 UL
LTb
0.58 0.00 0.83 C 1718 5496 M
0 17 V
687 -577 R
0 54 V
688 -344 R
0 141 V
687 -265 R
0 438 V
1718 5504 TriUF
2405 4963 TriUF
3093 4716 TriUF
3780 4741 TriUF
1.500 UL
LTb
0.58 0.00 0.83 C 1340 4708 M
56 0 V
55 0 V
56 0 V
55 0 V
56 0 V
55 0 V
56 0 V
55 0 V
56 0 V
55 0 V
56 0 V
56 0 V
55 0 V
56 0 V
55 0 V
56 0 V
55 0 V
56 0 V
55 0 V
56 0 V
55 0 V
56 0 V
56 0 V
55 0 V
56 0 V
55 0 V
56 0 V
55 0 V
56 0 V
55 0 V
56 0 V
55 0 V
56 0 V
56 0 V
55 0 V
56 0 V
55 0 V
56 0 V
55 0 V
56 0 V
55 0 V
56 0 V
55 0 V
56 0 V
56 0 V
55 0 V
56 0 V
55 0 V
56 0 V
55 0 V
56 0 V
55 0 V
56 0 V
55 0 V
56 0 V
56 0 V
55 0 V
56 0 V
55 0 V
56 0 V
55 0 V
56 0 V
55 0 V
56 0 V
55 0 V
56 0 V
56 0 V
55 0 V
56 0 V
55 0 V
56 0 V
55 0 V
56 0 V
55 0 V
56 0 V
55 0 V
56 0 V
56 0 V
55 0 V
56 0 V
55 0 V
56 0 V
55 0 V
56 0 V
55 0 V
56 0 V
55 0 V
56 0 V
56 0 V
55 0 V
56 0 V
55 0 V
56 0 V
55 0 V
56 0 V
55 0 V
56 0 V
55 0 V
56 0 V
1.500 UP
stroke
1.000 UL
LTb
0.58 0.00 0.83 C 1649 5618 M
0 16 V
688 -585 R
0 43 V
687 -249 R
0 112 V
687 -223 R
0 298 V
1649 5626 TriU
2337 5070 TriU
3024 4899 TriU
3711 4881 TriU
1.500 UL
LTb
0.58 0.00 0.83 C 1340 4895 M
56 0 V
55 0 V
56 0 V
55 0 V
56 0 V
55 0 V
56 0 V
55 0 V
56 0 V
55 0 V
56 0 V
56 0 V
55 0 V
56 0 V
55 0 V
56 0 V
55 0 V
56 0 V
55 0 V
56 0 V
55 0 V
56 0 V
56 0 V
55 0 V
56 0 V
55 0 V
56 0 V
55 0 V
56 0 V
55 0 V
56 0 V
55 0 V
56 0 V
56 0 V
55 0 V
56 0 V
55 0 V
56 0 V
55 0 V
56 0 V
55 0 V
56 0 V
55 0 V
56 0 V
56 0 V
55 0 V
56 0 V
55 0 V
56 0 V
55 0 V
56 0 V
55 0 V
56 0 V
55 0 V
56 0 V
56 0 V
55 0 V
56 0 V
55 0 V
56 0 V
55 0 V
56 0 V
55 0 V
56 0 V
55 0 V
56 0 V
56 0 V
55 0 V
56 0 V
55 0 V
56 0 V
55 0 V
56 0 V
55 0 V
56 0 V
55 0 V
56 0 V
56 0 V
55 0 V
56 0 V
55 0 V
56 0 V
55 0 V
56 0 V
55 0 V
56 0 V
55 0 V
56 0 V
56 0 V
55 0 V
56 0 V
55 0 V
56 0 V
55 0 V
56 0 V
55 0 V
56 0 V
55 0 V
56 0 V
1.500 UP
stroke
1.000 UL
LTb
0.58 0.00 0.83 C 1684 3559 M
0 22 V
2371 2575 M
0 29 V
687 -352 R
0 64 V
688 -207 R
0 106 V
687 -158 R
0 187 V
688 -402 R
0 311 V
1684 3570 Star
2371 2590 Star
3058 2284 Star
3746 2162 Star
4433 2151 Star
5121 1998 Star
1.500 UL
LTb
0.58 0.00 0.83 C 1340 2031 M
56 0 V
55 0 V
56 0 V
55 0 V
56 0 V
55 0 V
56 0 V
55 0 V
56 0 V
55 0 V
56 0 V
56 0 V
55 0 V
56 0 V
55 0 V
56 0 V
55 0 V
56 0 V
55 0 V
56 0 V
55 0 V
56 0 V
56 0 V
55 0 V
56 0 V
55 0 V
56 0 V
55 0 V
56 0 V
55 0 V
56 0 V
55 0 V
56 0 V
56 0 V
55 0 V
56 0 V
55 0 V
56 0 V
55 0 V
56 0 V
55 0 V
56 0 V
55 0 V
56 0 V
56 0 V
55 0 V
56 0 V
55 0 V
56 0 V
55 0 V
56 0 V
55 0 V
56 0 V
55 0 V
56 0 V
56 0 V
55 0 V
56 0 V
55 0 V
56 0 V
55 0 V
56 0 V
55 0 V
56 0 V
55 0 V
56 0 V
56 0 V
55 0 V
56 0 V
55 0 V
56 0 V
55 0 V
56 0 V
55 0 V
56 0 V
55 0 V
56 0 V
56 0 V
55 0 V
56 0 V
55 0 V
56 0 V
55 0 V
56 0 V
55 0 V
56 0 V
55 0 V
56 0 V
56 0 V
55 0 V
56 0 V
55 0 V
56 0 V
55 0 V
56 0 V
55 0 V
56 0 V
55 0 V
56 0 V
stroke
2.000 UL
LTb
LCb setrgbcolor
1.000 UL
LTb
LCb setrgbcolor
1340 7319 N
0 -6679 V
5499 0 V
0 6679 V
-5499 0 V
Z stroke
1.000 UP
1.000 UL
LTb
LCb setrgbcolor
stroke
grestore
end
showpage
  }}%
  \put(4089,140){\makebox(0,0){\large{$n_t$}}}%
  \put(160,4979){\makebox(0,0){\Large{$aM_{eff}^{ET_2}$}}}%
  \put(6839,440){\makebox(0,0){\strut{}\ {$8$}}}%
  \put(6152,440){\makebox(0,0){\strut{}\ {$7$}}}%
  \put(5464,440){\makebox(0,0){\strut{}\ {$6$}}}%
  \put(4777,440){\makebox(0,0){\strut{}\ {$5$}}}%
  \put(4090,440){\makebox(0,0){\strut{}\ {$4$}}}%
  \put(3402,440){\makebox(0,0){\strut{}\ {$3$}}}%
  \put(2715,440){\makebox(0,0){\strut{}\ {$2$}}}%
  \put(2027,440){\makebox(0,0){\strut{}\ {$1$}}}%
  \put(1340,440){\makebox(0,0){\strut{}\ {$0$}}}%
  \put(1220,7319){\makebox(0,0)[r]{\strut{}\ \ {$1.4$}}}%
  \put(1220,6651){\makebox(0,0)[r]{\strut{}\ \ {$1.3$}}}%
  \put(1220,5983){\makebox(0,0)[r]{\strut{}\ \ {$1.2$}}}%
  \put(1220,5315){\makebox(0,0)[r]{\strut{}\ \ {$1.1$}}}%
  \put(1220,4647){\makebox(0,0)[r]{\strut{}\ \ {$1$}}}%
  \put(1220,3979){\makebox(0,0)[r]{\strut{}\ \ {$0.9$}}}%
  \put(1220,3312){\makebox(0,0)[r]{\strut{}\ \ {$0.8$}}}%
  \put(1220,2644){\makebox(0,0)[r]{\strut{}\ \ {$0.7$}}}%
  \put(1220,1976){\makebox(0,0)[r]{\strut{}\ \ {$0.6$}}}%
  \put(1220,1308){\makebox(0,0)[r]{\strut{}\ \ {$0.5$}}}%
  \put(1220,640){\makebox(0,0)[r]{\strut{}\ \ {$0.4$}}}%
\end{picture}%
\endgroup
\endinput

\end	{center}
\caption{Effective masses for the lightest two $E^{++}$ ($\bullet$) and $T_2^{++}$ ($\circ$) 
  glueballs; the lightest two  $E^{-+}$ ($\blacksquare$) and $T_2^{-+}$ ($\square$) glueballs;
  the lightest $E^{+-}$ ($\blacktriangledown$) and $T_2^{+-}$ ($\triangledown$) glueballs;
  and the lightest $E^{--}$ ($\blacktriangle$) and $T_2^{--}$ ($\vartriangle$) glueballs.
  The state labelled by $\ast$ is, mainly, the $E^{++}$ ditorelon.
  Lines are mass estimates. All on a $20^330$ lattice at $\beta=47.75$ in $SU(8)$.
  In the continuum limit each of the $E$ doublets and corresponding $T_2$ triplets will
  pair up to give the five states of a $J=2$ glueball.}
\label{fig_MeffET2_SU8}
\end{figure}


%\begin{figure}[htb]
%\begin	{center}
%\leavevmode
%\input	{plot_MeffT1_SU8.tex}
%\end	{center}
%\caption{Effective masses for the lightest two $T_1^{+-}$ ($\bullet$) 
%  glueballs and the lightest $T_1^{-+}$ ($\blacksquare$) and $T_1^{--}$ ($\square$) glueballs.
%  Lines are mass estimates. All on a $20^330$ lattice at $\beta=47.75$ in $SU(8)$.
%  In the continuum limit each of these $T_1$ triplets will
%  give the three states of a $J=1$ glueball.}
%\label{fig_MeffT1_SU8}
%\end{figure}

%\begin{figure}[htb]
%\begin	{center}
%\leavevmode
%\input	{plot_MeffA1MM_SU8.tex}
%\end	{center}
%\caption{Effective masses for the lightest five $A_1^{--}$ glueballs.
%  Lines are mass estimates. All on a $20^330$ lattice at $\beta=47.75$ in $SU(8)$.
%  As examples of our heaviest states.}
%\label{fig_MeffA1MM_SU8}
%\end{figure}



\begin{figure}[htb]
\begin	{center}
\leavevmode
% GNUPLOT: LaTeX picture with Postscript
\begingroup%
\makeatletter%
\newcommand{\GNUPLOTspecial}{%
  \@sanitize\catcode`\%=14\relax\special}%
\setlength{\unitlength}{0.0500bp}%
\begin{picture}(7200,7560)(0,0)%
  {\GNUPLOTspecial{"
%!PS-Adobe-2.0 EPSF-2.0
%%Title: plot_MJ02ppK_cont_SU4.tex
%%Creator: gnuplot 5.0 patchlevel 3
%%CreationDate: Sun Mar  7 08:50:42 2021
%%DocumentFonts: 
%%BoundingBox: 0 0 360 378
%%EndComments
%%BeginProlog
/gnudict 256 dict def
gnudict begin
%
% The following true/false flags may be edited by hand if desired.
% The unit line width and grayscale image gamma correction may also be changed.
%
/Color true def
/Blacktext true def
/Solid false def
/Dashlength 1 def
/Landscape false def
/Level1 false def
/Level3 false def
/Rounded false def
/ClipToBoundingBox false def
/SuppressPDFMark false def
/TransparentPatterns false def
/gnulinewidth 5.000 def
/userlinewidth gnulinewidth def
/Gamma 1.0 def
/BackgroundColor {-1.000 -1.000 -1.000} def
%
/vshift -66 def
/dl1 {
  10.0 Dashlength userlinewidth gnulinewidth div mul mul mul
  Rounded { currentlinewidth 0.75 mul sub dup 0 le { pop 0.01 } if } if
} def
/dl2 {
  10.0 Dashlength userlinewidth gnulinewidth div mul mul mul
  Rounded { currentlinewidth 0.75 mul add } if
} def
/hpt_ 31.5 def
/vpt_ 31.5 def
/hpt hpt_ def
/vpt vpt_ def
/doclip {
  ClipToBoundingBox {
    newpath 0 0 moveto 360 0 lineto 360 378 lineto 0 378 lineto closepath
    clip
  } if
} def
%
% Gnuplot Prolog Version 5.1 (Oct 2015)
%
%/SuppressPDFMark true def
%
/M {moveto} bind def
/L {lineto} bind def
/R {rmoveto} bind def
/V {rlineto} bind def
/N {newpath moveto} bind def
/Z {closepath} bind def
/C {setrgbcolor} bind def
/f {rlineto fill} bind def
/g {setgray} bind def
/Gshow {show} def   % May be redefined later in the file to support UTF-8
/vpt2 vpt 2 mul def
/hpt2 hpt 2 mul def
/Lshow {currentpoint stroke M 0 vshift R 
	Blacktext {gsave 0 setgray textshow grestore} {textshow} ifelse} def
/Rshow {currentpoint stroke M dup stringwidth pop neg vshift R
	Blacktext {gsave 0 setgray textshow grestore} {textshow} ifelse} def
/Cshow {currentpoint stroke M dup stringwidth pop -2 div vshift R 
	Blacktext {gsave 0 setgray textshow grestore} {textshow} ifelse} def
/UP {dup vpt_ mul /vpt exch def hpt_ mul /hpt exch def
  /hpt2 hpt 2 mul def /vpt2 vpt 2 mul def} def
/DL {Color {setrgbcolor Solid {pop []} if 0 setdash}
 {pop pop pop 0 setgray Solid {pop []} if 0 setdash} ifelse} def
/BL {stroke userlinewidth 2 mul setlinewidth
	Rounded {1 setlinejoin 1 setlinecap} if} def
/AL {stroke userlinewidth 2 div setlinewidth
	Rounded {1 setlinejoin 1 setlinecap} if} def
/UL {dup gnulinewidth mul /userlinewidth exch def
	dup 1 lt {pop 1} if 10 mul /udl exch def} def
/PL {stroke userlinewidth setlinewidth
	Rounded {1 setlinejoin 1 setlinecap} if} def
3.8 setmiterlimit
% Classic Line colors (version 5.0)
/LCw {1 1 1} def
/LCb {0 0 0} def
/LCa {0 0 0} def
/LC0 {1 0 0} def
/LC1 {0 1 0} def
/LC2 {0 0 1} def
/LC3 {1 0 1} def
/LC4 {0 1 1} def
/LC5 {1 1 0} def
/LC6 {0 0 0} def
/LC7 {1 0.3 0} def
/LC8 {0.5 0.5 0.5} def
% Default dash patterns (version 5.0)
/LTB {BL [] LCb DL} def
/LTw {PL [] 1 setgray} def
/LTb {PL [] LCb DL} def
/LTa {AL [1 udl mul 2 udl mul] 0 setdash LCa setrgbcolor} def
/LT0 {PL [] LC0 DL} def
/LT1 {PL [2 dl1 3 dl2] LC1 DL} def
/LT2 {PL [1 dl1 1.5 dl2] LC2 DL} def
/LT3 {PL [6 dl1 2 dl2 1 dl1 2 dl2] LC3 DL} def
/LT4 {PL [1 dl1 2 dl2 6 dl1 2 dl2 1 dl1 2 dl2] LC4 DL} def
/LT5 {PL [4 dl1 2 dl2] LC5 DL} def
/LT6 {PL [1.5 dl1 1.5 dl2 1.5 dl1 1.5 dl2 1.5 dl1 6 dl2] LC6 DL} def
/LT7 {PL [3 dl1 3 dl2 1 dl1 3 dl2] LC7 DL} def
/LT8 {PL [2 dl1 2 dl2 2 dl1 6 dl2] LC8 DL} def
/SL {[] 0 setdash} def
/Pnt {stroke [] 0 setdash gsave 1 setlinecap M 0 0 V stroke grestore} def
/Dia {stroke [] 0 setdash 2 copy vpt add M
  hpt neg vpt neg V hpt vpt neg V
  hpt vpt V hpt neg vpt V closepath stroke
  Pnt} def
/Pls {stroke [] 0 setdash vpt sub M 0 vpt2 V
  currentpoint stroke M
  hpt neg vpt neg R hpt2 0 V stroke
 } def
/Box {stroke [] 0 setdash 2 copy exch hpt sub exch vpt add M
  0 vpt2 neg V hpt2 0 V 0 vpt2 V
  hpt2 neg 0 V closepath stroke
  Pnt} def
/Crs {stroke [] 0 setdash exch hpt sub exch vpt add M
  hpt2 vpt2 neg V currentpoint stroke M
  hpt2 neg 0 R hpt2 vpt2 V stroke} def
/TriU {stroke [] 0 setdash 2 copy vpt 1.12 mul add M
  hpt neg vpt -1.62 mul V
  hpt 2 mul 0 V
  hpt neg vpt 1.62 mul V closepath stroke
  Pnt} def
/Star {2 copy Pls Crs} def
/BoxF {stroke [] 0 setdash exch hpt sub exch vpt add M
  0 vpt2 neg V hpt2 0 V 0 vpt2 V
  hpt2 neg 0 V closepath fill} def
/TriUF {stroke [] 0 setdash vpt 1.12 mul add M
  hpt neg vpt -1.62 mul V
  hpt 2 mul 0 V
  hpt neg vpt 1.62 mul V closepath fill} def
/TriD {stroke [] 0 setdash 2 copy vpt 1.12 mul sub M
  hpt neg vpt 1.62 mul V
  hpt 2 mul 0 V
  hpt neg vpt -1.62 mul V closepath stroke
  Pnt} def
/TriDF {stroke [] 0 setdash vpt 1.12 mul sub M
  hpt neg vpt 1.62 mul V
  hpt 2 mul 0 V
  hpt neg vpt -1.62 mul V closepath fill} def
/DiaF {stroke [] 0 setdash vpt add M
  hpt neg vpt neg V hpt vpt neg V
  hpt vpt V hpt neg vpt V closepath fill} def
/Pent {stroke [] 0 setdash 2 copy gsave
  translate 0 hpt M 4 {72 rotate 0 hpt L} repeat
  closepath stroke grestore Pnt} def
/PentF {stroke [] 0 setdash gsave
  translate 0 hpt M 4 {72 rotate 0 hpt L} repeat
  closepath fill grestore} def
/Circle {stroke [] 0 setdash 2 copy
  hpt 0 360 arc stroke Pnt} def
/CircleF {stroke [] 0 setdash hpt 0 360 arc fill} def
/C0 {BL [] 0 setdash 2 copy moveto vpt 90 450 arc} bind def
/C1 {BL [] 0 setdash 2 copy moveto
	2 copy vpt 0 90 arc closepath fill
	vpt 0 360 arc closepath} bind def
/C2 {BL [] 0 setdash 2 copy moveto
	2 copy vpt 90 180 arc closepath fill
	vpt 0 360 arc closepath} bind def
/C3 {BL [] 0 setdash 2 copy moveto
	2 copy vpt 0 180 arc closepath fill
	vpt 0 360 arc closepath} bind def
/C4 {BL [] 0 setdash 2 copy moveto
	2 copy vpt 180 270 arc closepath fill
	vpt 0 360 arc closepath} bind def
/C5 {BL [] 0 setdash 2 copy moveto
	2 copy vpt 0 90 arc
	2 copy moveto
	2 copy vpt 180 270 arc closepath fill
	vpt 0 360 arc} bind def
/C6 {BL [] 0 setdash 2 copy moveto
	2 copy vpt 90 270 arc closepath fill
	vpt 0 360 arc closepath} bind def
/C7 {BL [] 0 setdash 2 copy moveto
	2 copy vpt 0 270 arc closepath fill
	vpt 0 360 arc closepath} bind def
/C8 {BL [] 0 setdash 2 copy moveto
	2 copy vpt 270 360 arc closepath fill
	vpt 0 360 arc closepath} bind def
/C9 {BL [] 0 setdash 2 copy moveto
	2 copy vpt 270 450 arc closepath fill
	vpt 0 360 arc closepath} bind def
/C10 {BL [] 0 setdash 2 copy 2 copy moveto vpt 270 360 arc closepath fill
	2 copy moveto
	2 copy vpt 90 180 arc closepath fill
	vpt 0 360 arc closepath} bind def
/C11 {BL [] 0 setdash 2 copy moveto
	2 copy vpt 0 180 arc closepath fill
	2 copy moveto
	2 copy vpt 270 360 arc closepath fill
	vpt 0 360 arc closepath} bind def
/C12 {BL [] 0 setdash 2 copy moveto
	2 copy vpt 180 360 arc closepath fill
	vpt 0 360 arc closepath} bind def
/C13 {BL [] 0 setdash 2 copy moveto
	2 copy vpt 0 90 arc closepath fill
	2 copy moveto
	2 copy vpt 180 360 arc closepath fill
	vpt 0 360 arc closepath} bind def
/C14 {BL [] 0 setdash 2 copy moveto
	2 copy vpt 90 360 arc closepath fill
	vpt 0 360 arc} bind def
/C15 {BL [] 0 setdash 2 copy vpt 0 360 arc closepath fill
	vpt 0 360 arc closepath} bind def
/Rec {newpath 4 2 roll moveto 1 index 0 rlineto 0 exch rlineto
	neg 0 rlineto closepath} bind def
/Square {dup Rec} bind def
/Bsquare {vpt sub exch vpt sub exch vpt2 Square} bind def
/S0 {BL [] 0 setdash 2 copy moveto 0 vpt rlineto BL Bsquare} bind def
/S1 {BL [] 0 setdash 2 copy vpt Square fill Bsquare} bind def
/S2 {BL [] 0 setdash 2 copy exch vpt sub exch vpt Square fill Bsquare} bind def
/S3 {BL [] 0 setdash 2 copy exch vpt sub exch vpt2 vpt Rec fill Bsquare} bind def
/S4 {BL [] 0 setdash 2 copy exch vpt sub exch vpt sub vpt Square fill Bsquare} bind def
/S5 {BL [] 0 setdash 2 copy 2 copy vpt Square fill
	exch vpt sub exch vpt sub vpt Square fill Bsquare} bind def
/S6 {BL [] 0 setdash 2 copy exch vpt sub exch vpt sub vpt vpt2 Rec fill Bsquare} bind def
/S7 {BL [] 0 setdash 2 copy exch vpt sub exch vpt sub vpt vpt2 Rec fill
	2 copy vpt Square fill Bsquare} bind def
/S8 {BL [] 0 setdash 2 copy vpt sub vpt Square fill Bsquare} bind def
/S9 {BL [] 0 setdash 2 copy vpt sub vpt vpt2 Rec fill Bsquare} bind def
/S10 {BL [] 0 setdash 2 copy vpt sub vpt Square fill 2 copy exch vpt sub exch vpt Square fill
	Bsquare} bind def
/S11 {BL [] 0 setdash 2 copy vpt sub vpt Square fill 2 copy exch vpt sub exch vpt2 vpt Rec fill
	Bsquare} bind def
/S12 {BL [] 0 setdash 2 copy exch vpt sub exch vpt sub vpt2 vpt Rec fill Bsquare} bind def
/S13 {BL [] 0 setdash 2 copy exch vpt sub exch vpt sub vpt2 vpt Rec fill
	2 copy vpt Square fill Bsquare} bind def
/S14 {BL [] 0 setdash 2 copy exch vpt sub exch vpt sub vpt2 vpt Rec fill
	2 copy exch vpt sub exch vpt Square fill Bsquare} bind def
/S15 {BL [] 0 setdash 2 copy Bsquare fill Bsquare} bind def
/D0 {gsave translate 45 rotate 0 0 S0 stroke grestore} bind def
/D1 {gsave translate 45 rotate 0 0 S1 stroke grestore} bind def
/D2 {gsave translate 45 rotate 0 0 S2 stroke grestore} bind def
/D3 {gsave translate 45 rotate 0 0 S3 stroke grestore} bind def
/D4 {gsave translate 45 rotate 0 0 S4 stroke grestore} bind def
/D5 {gsave translate 45 rotate 0 0 S5 stroke grestore} bind def
/D6 {gsave translate 45 rotate 0 0 S6 stroke grestore} bind def
/D7 {gsave translate 45 rotate 0 0 S7 stroke grestore} bind def
/D8 {gsave translate 45 rotate 0 0 S8 stroke grestore} bind def
/D9 {gsave translate 45 rotate 0 0 S9 stroke grestore} bind def
/D10 {gsave translate 45 rotate 0 0 S10 stroke grestore} bind def
/D11 {gsave translate 45 rotate 0 0 S11 stroke grestore} bind def
/D12 {gsave translate 45 rotate 0 0 S12 stroke grestore} bind def
/D13 {gsave translate 45 rotate 0 0 S13 stroke grestore} bind def
/D14 {gsave translate 45 rotate 0 0 S14 stroke grestore} bind def
/D15 {gsave translate 45 rotate 0 0 S15 stroke grestore} bind def
/DiaE {stroke [] 0 setdash vpt add M
  hpt neg vpt neg V hpt vpt neg V
  hpt vpt V hpt neg vpt V closepath stroke} def
/BoxE {stroke [] 0 setdash exch hpt sub exch vpt add M
  0 vpt2 neg V hpt2 0 V 0 vpt2 V
  hpt2 neg 0 V closepath stroke} def
/TriUE {stroke [] 0 setdash vpt 1.12 mul add M
  hpt neg vpt -1.62 mul V
  hpt 2 mul 0 V
  hpt neg vpt 1.62 mul V closepath stroke} def
/TriDE {stroke [] 0 setdash vpt 1.12 mul sub M
  hpt neg vpt 1.62 mul V
  hpt 2 mul 0 V
  hpt neg vpt -1.62 mul V closepath stroke} def
/PentE {stroke [] 0 setdash gsave
  translate 0 hpt M 4 {72 rotate 0 hpt L} repeat
  closepath stroke grestore} def
/CircE {stroke [] 0 setdash 
  hpt 0 360 arc stroke} def
/Opaque {gsave closepath 1 setgray fill grestore 0 setgray closepath} def
/DiaW {stroke [] 0 setdash vpt add M
  hpt neg vpt neg V hpt vpt neg V
  hpt vpt V hpt neg vpt V Opaque stroke} def
/BoxW {stroke [] 0 setdash exch hpt sub exch vpt add M
  0 vpt2 neg V hpt2 0 V 0 vpt2 V
  hpt2 neg 0 V Opaque stroke} def
/TriUW {stroke [] 0 setdash vpt 1.12 mul add M
  hpt neg vpt -1.62 mul V
  hpt 2 mul 0 V
  hpt neg vpt 1.62 mul V Opaque stroke} def
/TriDW {stroke [] 0 setdash vpt 1.12 mul sub M
  hpt neg vpt 1.62 mul V
  hpt 2 mul 0 V
  hpt neg vpt -1.62 mul V Opaque stroke} def
/PentW {stroke [] 0 setdash gsave
  translate 0 hpt M 4 {72 rotate 0 hpt L} repeat
  Opaque stroke grestore} def
/CircW {stroke [] 0 setdash 
  hpt 0 360 arc Opaque stroke} def
/BoxFill {gsave Rec 1 setgray fill grestore} def
/Density {
  /Fillden exch def
  currentrgbcolor
  /ColB exch def /ColG exch def /ColR exch def
  /ColR ColR Fillden mul Fillden sub 1 add def
  /ColG ColG Fillden mul Fillden sub 1 add def
  /ColB ColB Fillden mul Fillden sub 1 add def
  ColR ColG ColB setrgbcolor} def
/BoxColFill {gsave Rec PolyFill} def
/PolyFill {gsave Density fill grestore grestore} def
/h {rlineto rlineto rlineto gsave closepath fill grestore} bind def
%
% PostScript Level 1 Pattern Fill routine for rectangles
% Usage: x y w h s a XX PatternFill
%	x,y = lower left corner of box to be filled
%	w,h = width and height of box
%	  a = angle in degrees between lines and x-axis
%	 XX = 0/1 for no/yes cross-hatch
%
/PatternFill {gsave /PFa [ 9 2 roll ] def
  PFa 0 get PFa 2 get 2 div add PFa 1 get PFa 3 get 2 div add translate
  PFa 2 get -2 div PFa 3 get -2 div PFa 2 get PFa 3 get Rec
  TransparentPatterns {} {gsave 1 setgray fill grestore} ifelse
  clip
  currentlinewidth 0.5 mul setlinewidth
  /PFs PFa 2 get dup mul PFa 3 get dup mul add sqrt def
  0 0 M PFa 5 get rotate PFs -2 div dup translate
  0 1 PFs PFa 4 get div 1 add floor cvi
	{PFa 4 get mul 0 M 0 PFs V} for
  0 PFa 6 get ne {
	0 1 PFs PFa 4 get div 1 add floor cvi
	{PFa 4 get mul 0 2 1 roll M PFs 0 V} for
 } if
  stroke grestore} def
%
/languagelevel where
 {pop languagelevel} {1} ifelse
dup 2 lt
	{/InterpretLevel1 true def
	 /InterpretLevel3 false def}
	{/InterpretLevel1 Level1 def
	 2 gt
	    {/InterpretLevel3 Level3 def}
	    {/InterpretLevel3 false def}
	 ifelse }
 ifelse
%
% PostScript level 2 pattern fill definitions
%
/Level2PatternFill {
/Tile8x8 {/PaintType 2 /PatternType 1 /TilingType 1 /BBox [0 0 8 8] /XStep 8 /YStep 8}
	bind def
/KeepColor {currentrgbcolor [/Pattern /DeviceRGB] setcolorspace} bind def
<< Tile8x8
 /PaintProc {0.5 setlinewidth pop 0 0 M 8 8 L 0 8 M 8 0 L stroke} 
>> matrix makepattern
/Pat1 exch def
<< Tile8x8
 /PaintProc {0.5 setlinewidth pop 0 0 M 8 8 L 0 8 M 8 0 L stroke
	0 4 M 4 8 L 8 4 L 4 0 L 0 4 L stroke}
>> matrix makepattern
/Pat2 exch def
<< Tile8x8
 /PaintProc {0.5 setlinewidth pop 0 0 M 0 8 L
	8 8 L 8 0 L 0 0 L fill}
>> matrix makepattern
/Pat3 exch def
<< Tile8x8
 /PaintProc {0.5 setlinewidth pop -4 8 M 8 -4 L
	0 12 M 12 0 L stroke}
>> matrix makepattern
/Pat4 exch def
<< Tile8x8
 /PaintProc {0.5 setlinewidth pop -4 0 M 8 12 L
	0 -4 M 12 8 L stroke}
>> matrix makepattern
/Pat5 exch def
<< Tile8x8
 /PaintProc {0.5 setlinewidth pop -2 8 M 4 -4 L
	0 12 M 8 -4 L 4 12 M 10 0 L stroke}
>> matrix makepattern
/Pat6 exch def
<< Tile8x8
 /PaintProc {0.5 setlinewidth pop -2 0 M 4 12 L
	0 -4 M 8 12 L 4 -4 M 10 8 L stroke}
>> matrix makepattern
/Pat7 exch def
<< Tile8x8
 /PaintProc {0.5 setlinewidth pop 8 -2 M -4 4 L
	12 0 M -4 8 L 12 4 M 0 10 L stroke}
>> matrix makepattern
/Pat8 exch def
<< Tile8x8
 /PaintProc {0.5 setlinewidth pop 0 -2 M 12 4 L
	-4 0 M 12 8 L -4 4 M 8 10 L stroke}
>> matrix makepattern
/Pat9 exch def
/Pattern1 {PatternBgnd KeepColor Pat1 setpattern} bind def
/Pattern2 {PatternBgnd KeepColor Pat2 setpattern} bind def
/Pattern3 {PatternBgnd KeepColor Pat3 setpattern} bind def
/Pattern4 {PatternBgnd KeepColor Landscape {Pat5} {Pat4} ifelse setpattern} bind def
/Pattern5 {PatternBgnd KeepColor Landscape {Pat4} {Pat5} ifelse setpattern} bind def
/Pattern6 {PatternBgnd KeepColor Landscape {Pat9} {Pat6} ifelse setpattern} bind def
/Pattern7 {PatternBgnd KeepColor Landscape {Pat8} {Pat7} ifelse setpattern} bind def
} def
%
%
%End of PostScript Level 2 code
%
/PatternBgnd {
  TransparentPatterns {} {gsave 1 setgray fill grestore} ifelse
} def
%
% Substitute for Level 2 pattern fill codes with
% grayscale if Level 2 support is not selected.
%
/Level1PatternFill {
/Pattern1 {0.250 Density} bind def
/Pattern2 {0.500 Density} bind def
/Pattern3 {0.750 Density} bind def
/Pattern4 {0.125 Density} bind def
/Pattern5 {0.375 Density} bind def
/Pattern6 {0.625 Density} bind def
/Pattern7 {0.875 Density} bind def
} def
%
% Now test for support of Level 2 code
%
Level1 {Level1PatternFill} {Level2PatternFill} ifelse
%
/Symbol-Oblique /Symbol findfont [1 0 .167 1 0 0] makefont
dup length dict begin {1 index /FID eq {pop pop} {def} ifelse} forall
currentdict end definefont pop
%
Level1 SuppressPDFMark or 
{} {
/SDict 10 dict def
systemdict /pdfmark known not {
  userdict /pdfmark systemdict /cleartomark get put
} if
SDict begin [
  /Title (plot_MJ02ppK_cont_SU4.tex)
  /Subject (gnuplot plot)
  /Creator (gnuplot 5.0 patchlevel 3)
  /Author (mteper)
%  /Producer (gnuplot)
%  /Keywords ()
  /CreationDate (Sun Mar  7 08:50:42 2021)
  /DOCINFO pdfmark
end
} ifelse
%
% Support for boxed text - Ethan A Merritt May 2005
%
/InitTextBox { userdict /TBy2 3 -1 roll put userdict /TBx2 3 -1 roll put
           userdict /TBy1 3 -1 roll put userdict /TBx1 3 -1 roll put
	   /Boxing true def } def
/ExtendTextBox { Boxing
    { gsave dup false charpath pathbbox
      dup TBy2 gt {userdict /TBy2 3 -1 roll put} {pop} ifelse
      dup TBx2 gt {userdict /TBx2 3 -1 roll put} {pop} ifelse
      dup TBy1 lt {userdict /TBy1 3 -1 roll put} {pop} ifelse
      dup TBx1 lt {userdict /TBx1 3 -1 roll put} {pop} ifelse
      grestore } if } def
/PopTextBox { newpath TBx1 TBxmargin sub TBy1 TBymargin sub M
               TBx1 TBxmargin sub TBy2 TBymargin add L
	       TBx2 TBxmargin add TBy2 TBymargin add L
	       TBx2 TBxmargin add TBy1 TBymargin sub L closepath } def
/DrawTextBox { PopTextBox stroke /Boxing false def} def
/FillTextBox { gsave PopTextBox 1 1 1 setrgbcolor fill grestore /Boxing false def} def
0 0 0 0 InitTextBox
/TBxmargin 20 def
/TBymargin 20 def
/Boxing false def
/textshow { ExtendTextBox Gshow } def
%
% redundant definitions for compatibility with prologue.ps older than 5.0.2
/LTB {BL [] LCb DL} def
/LTb {PL [] LCb DL} def
end
%%EndProlog
%%Page: 1 1
gnudict begin
gsave
doclip
0 0 translate
0.050 0.050 scale
0 setgray
newpath
BackgroundColor 0 lt 3 1 roll 0 lt exch 0 lt or or not {BackgroundColor C 1.000 0 0 7200.00 7560.00 BoxColFill} if
1.000 UL
LTb
LCb setrgbcolor
1100 640 M
63 0 V
5676 0 R
-63 0 V
stroke
LTb
LCb setrgbcolor
1100 1382 M
63 0 V
5676 0 R
-63 0 V
stroke
LTb
LCb setrgbcolor
1100 2124 M
63 0 V
5676 0 R
-63 0 V
stroke
LTb
LCb setrgbcolor
1100 2866 M
63 0 V
5676 0 R
-63 0 V
stroke
LTb
LCb setrgbcolor
1100 3608 M
63 0 V
5676 0 R
-63 0 V
stroke
LTb
LCb setrgbcolor
1100 4351 M
63 0 V
5676 0 R
-63 0 V
stroke
LTb
LCb setrgbcolor
1100 5093 M
63 0 V
5676 0 R
-63 0 V
stroke
LTb
LCb setrgbcolor
1100 5835 M
63 0 V
5676 0 R
-63 0 V
stroke
LTb
LCb setrgbcolor
1100 6577 M
63 0 V
5676 0 R
-63 0 V
stroke
LTb
LCb setrgbcolor
1100 7319 M
63 0 V
5676 0 R
-63 0 V
stroke
LTb
LCb setrgbcolor
1100 640 M
0 63 V
0 6616 R
0 -63 V
stroke
LTb
LCb setrgbcolor
1674 640 M
0 63 V
0 6616 R
0 -63 V
stroke
LTb
LCb setrgbcolor
2248 640 M
0 63 V
0 6616 R
0 -63 V
stroke
LTb
LCb setrgbcolor
2822 640 M
0 63 V
0 6616 R
0 -63 V
stroke
LTb
LCb setrgbcolor
3396 640 M
0 63 V
0 6616 R
0 -63 V
stroke
LTb
LCb setrgbcolor
3970 640 M
0 63 V
0 6616 R
0 -63 V
stroke
LTb
LCb setrgbcolor
4543 640 M
0 63 V
0 6616 R
0 -63 V
stroke
LTb
LCb setrgbcolor
5117 640 M
0 63 V
0 6616 R
0 -63 V
stroke
LTb
LCb setrgbcolor
5691 640 M
0 63 V
0 6616 R
0 -63 V
stroke
LTb
LCb setrgbcolor
6265 640 M
0 63 V
0 6616 R
0 -63 V
stroke
LTb
LCb setrgbcolor
6839 640 M
0 63 V
0 6616 R
0 -63 V
stroke
LTb
LCb setrgbcolor
1.000 UL
LTb
LCb setrgbcolor
1100 7319 N
0 -6679 V
5739 0 V
0 6679 V
-5739 0 V
Z stroke
1.000 UP
1.000 UL
LTb
LCb setrgbcolor
LCb setrgbcolor
LTb
LCb setrgbcolor
LTb
1.500 UP
1.000 UL
LTb
0.58 0.00 0.83 C 6338 2693 M
0 24 V
4810 2845 M
0 33 V
-1074 37 R
0 24 V
-746 20 R
0 31 V
-546 -7 R
0 31 V
-364 -52 R
0 50 V
6338 4270 M
0 104 V
4810 4542 M
0 59 V
-1074 3 R
0 194 V
2990 4654 M
0 141 V
-546 14 R
0 118 V
2080 4751 M
0 98 V
6338 2705 CircleF
4810 2861 CircleF
3736 2927 CircleF
2990 2974 CircleF
2444 2999 CircleF
2080 2987 CircleF
6338 4322 CircleF
4810 4571 CircleF
3736 4701 CircleF
2990 4725 CircleF
2444 4868 CircleF
2080 4800 CircleF
1.500 UL
LTb
0.58 0.00 0.83 C 1100 3067 M
58 -3 V
58 -3 V
58 -3 V
58 -3 V
58 -3 V
58 -3 V
58 -3 V
58 -4 V
58 -3 V
58 -3 V
58 -3 V
58 -3 V
58 -3 V
58 -3 V
58 -4 V
58 -3 V
57 -3 V
58 -3 V
58 -3 V
58 -3 V
58 -3 V
58 -3 V
58 -4 V
58 -3 V
58 -3 V
58 -3 V
58 -3 V
58 -3 V
58 -3 V
58 -4 V
58 -3 V
58 -3 V
58 -3 V
58 -3 V
58 -3 V
58 -3 V
58 -3 V
58 -4 V
58 -3 V
58 -3 V
58 -3 V
58 -3 V
58 -3 V
58 -3 V
58 -4 V
58 -3 V
58 -3 V
58 -3 V
58 -3 V
57 -3 V
58 -3 V
58 -3 V
58 -4 V
58 -3 V
58 -3 V
58 -3 V
58 -3 V
58 -3 V
58 -3 V
58 -4 V
58 -3 V
58 -3 V
58 -3 V
58 -3 V
58 -3 V
58 -3 V
58 -3 V
58 -4 V
58 -3 V
58 -3 V
58 -3 V
58 -3 V
58 -3 V
58 -3 V
58 -4 V
58 -3 V
58 -3 V
58 -3 V
58 -3 V
58 -3 V
58 -3 V
58 -3 V
57 -4 V
58 -3 V
58 -3 V
58 -3 V
58 -3 V
58 -3 V
58 -3 V
58 -4 V
58 -3 V
58 -3 V
58 -3 V
58 -3 V
58 -3 V
58 -3 V
58 -3 V
58 -4 V
58 -3 V
stroke
LTb
0.58 0.00 0.83 C 1100 4964 M
58 -6 V
58 -7 V
58 -6 V
58 -7 V
58 -6 V
58 -7 V
58 -6 V
58 -7 V
58 -6 V
58 -7 V
58 -6 V
58 -7 V
58 -6 V
58 -7 V
58 -7 V
58 -6 V
57 -7 V
58 -6 V
58 -7 V
58 -6 V
58 -7 V
58 -6 V
58 -7 V
58 -6 V
58 -7 V
58 -6 V
58 -7 V
58 -6 V
58 -7 V
58 -6 V
58 -7 V
58 -6 V
58 -7 V
58 -6 V
58 -7 V
58 -6 V
58 -7 V
58 -7 V
58 -6 V
58 -7 V
58 -6 V
58 -7 V
58 -6 V
58 -7 V
58 -6 V
58 -7 V
58 -6 V
58 -7 V
58 -6 V
57 -7 V
58 -6 V
58 -7 V
58 -6 V
58 -7 V
58 -6 V
58 -7 V
58 -6 V
58 -7 V
58 -6 V
58 -7 V
58 -7 V
58 -6 V
58 -7 V
58 -6 V
58 -7 V
58 -6 V
58 -7 V
58 -6 V
58 -7 V
58 -6 V
58 -7 V
58 -6 V
58 -7 V
58 -6 V
58 -7 V
58 -6 V
58 -7 V
58 -6 V
58 -7 V
58 -6 V
58 -7 V
58 -6 V
57 -7 V
58 -7 V
58 -6 V
58 -7 V
58 -6 V
58 -7 V
58 -6 V
58 -7 V
58 -6 V
58 -7 V
58 -6 V
58 -7 V
58 -6 V
58 -7 V
58 -6 V
58 -7 V
58 -6 V
1.500 UP
stroke
1.000 UL
LTb
0.58 0.00 0.83 C 6338 4167 M
0 65 V
4810 4127 M
0 42 V
-1074 -2 R
0 29 V
-746 -37 R
0 34 V
-546 -73 R
0 54 V
-364 -65 R
0 44 V
6338 5317 M
0 167 V
-1528 50 R
0 95 V
3736 5504 M
0 77 V
-746 4 R
0 79 V
-546 -96 R
0 45 V
2080 5479 M
0 92 V
6338 4199 DiaF
4810 4148 DiaF
3736 4182 DiaF
2990 4176 DiaF
2444 4147 DiaF
2080 4131 DiaF
6338 5401 DiaF
4810 5581 DiaF
3736 5543 DiaF
2990 5624 DiaF
2444 5591 DiaF
2080 5525 DiaF
1.500 UL
LTb
0.58 0.00 0.83 C 1100 4144 M
58 0 V
58 1 V
58 0 V
58 1 V
58 0 V
58 1 V
58 0 V
58 1 V
58 0 V
58 1 V
58 0 V
58 1 V
58 0 V
58 1 V
58 0 V
58 1 V
57 1 V
58 0 V
58 1 V
58 0 V
58 1 V
58 0 V
58 1 V
58 0 V
58 1 V
58 0 V
58 1 V
58 0 V
58 1 V
58 0 V
58 1 V
58 1 V
58 0 V
58 1 V
58 0 V
58 1 V
58 0 V
58 1 V
58 0 V
58 1 V
58 0 V
58 1 V
58 0 V
58 1 V
58 0 V
58 1 V
58 0 V
58 1 V
58 1 V
57 0 V
58 1 V
58 0 V
58 1 V
58 0 V
58 1 V
58 0 V
58 1 V
58 0 V
58 1 V
58 0 V
58 1 V
58 0 V
58 1 V
58 1 V
58 0 V
58 1 V
58 0 V
58 1 V
58 0 V
58 1 V
58 0 V
58 1 V
58 0 V
58 1 V
58 0 V
58 1 V
58 0 V
58 1 V
58 0 V
58 1 V
58 1 V
58 0 V
57 1 V
58 0 V
58 1 V
58 0 V
58 1 V
58 0 V
58 1 V
58 0 V
58 1 V
58 0 V
58 1 V
58 0 V
58 1 V
58 1 V
58 0 V
58 1 V
58 0 V
stroke
LTb
0.58 0.00 0.83 C 1100 5614 M
58 -2 V
58 -1 V
58 -1 V
58 -1 V
58 -1 V
58 -1 V
58 -2 V
58 -1 V
58 -1 V
58 -1 V
58 -1 V
58 -1 V
58 -2 V
58 -1 V
58 -1 V
58 -1 V
57 -1 V
58 -2 V
58 -1 V
58 -1 V
58 -1 V
58 -1 V
58 -1 V
58 -2 V
58 -1 V
58 -1 V
58 -1 V
58 -1 V
58 -2 V
58 -1 V
58 -1 V
58 -1 V
58 -1 V
58 -1 V
58 -2 V
58 -1 V
58 -1 V
58 -1 V
58 -1 V
58 -1 V
58 -2 V
58 -1 V
58 -1 V
58 -1 V
58 -1 V
58 -2 V
58 -1 V
58 -1 V
58 -1 V
57 -1 V
58 -1 V
58 -2 V
58 -1 V
58 -1 V
58 -1 V
58 -1 V
58 -1 V
58 -2 V
58 -1 V
58 -1 V
58 -1 V
58 -1 V
58 -2 V
58 -1 V
58 -1 V
58 -1 V
58 -1 V
58 -1 V
58 -2 V
58 -1 V
58 -1 V
58 -1 V
58 -1 V
58 -1 V
58 -2 V
58 -1 V
58 -1 V
58 -1 V
58 -1 V
58 -2 V
58 -1 V
58 -1 V
57 -1 V
58 -1 V
58 -1 V
58 -2 V
58 -1 V
58 -1 V
58 -1 V
58 -1 V
58 -1 V
58 -2 V
58 -1 V
58 -1 V
58 -1 V
58 -1 V
58 -2 V
58 -1 V
58 -1 V
1.500 UP
stroke
1.000 UL
LTb
0.58 0.00 0.83 C 6338 4199 M
0 45 V
4810 4183 M
0 36 V
3736 4189 M
0 37 V
-746 -86 R
0 52 V
-546 -20 R
0 53 V
-364 -78 R
0 42 V
6338 5516 M
0 143 V
4810 5473 M
0 118 V
3736 5577 M
0 70 V
2990 5439 M
0 142 V
-546 -25 R
0 99 V
-364 -91 R
0 66 V
6338 4221 Dia
4810 4201 Dia
3736 4207 Dia
2990 4166 Dia
2444 4198 Dia
2080 4168 Dia
6338 5587 Dia
4810 5532 Dia
3736 5612 Dia
2990 5510 Dia
2444 5605 Dia
2080 5597 Dia
1.500 UL
LTb
0.58 0.00 0.83 C 1100 4165 M
58 1 V
58 0 V
58 1 V
58 1 V
58 0 V
58 1 V
58 0 V
58 1 V
58 1 V
58 0 V
58 1 V
58 1 V
58 0 V
58 1 V
58 0 V
58 1 V
57 1 V
58 0 V
58 1 V
58 1 V
58 0 V
58 1 V
58 1 V
58 0 V
58 1 V
58 0 V
58 1 V
58 1 V
58 0 V
58 1 V
58 1 V
58 0 V
58 1 V
58 0 V
58 1 V
58 1 V
58 0 V
58 1 V
58 1 V
58 0 V
58 1 V
58 1 V
58 0 V
58 1 V
58 0 V
58 1 V
58 1 V
58 0 V
58 1 V
57 1 V
58 0 V
58 1 V
58 0 V
58 1 V
58 1 V
58 0 V
58 1 V
58 1 V
58 0 V
58 1 V
58 0 V
58 1 V
58 1 V
58 0 V
58 1 V
58 1 V
58 0 V
58 1 V
58 1 V
58 0 V
58 1 V
58 0 V
58 1 V
58 1 V
58 0 V
58 1 V
58 1 V
58 0 V
58 1 V
58 0 V
58 1 V
58 1 V
57 0 V
58 1 V
58 1 V
58 0 V
58 1 V
58 1 V
58 0 V
58 1 V
58 0 V
58 1 V
58 1 V
58 0 V
58 1 V
58 1 V
58 0 V
58 1 V
58 0 V
stroke
LTb
0.58 0.00 0.83 C 1100 5602 M
58 0 V
58 0 V
58 -1 V
58 0 V
58 0 V
58 -1 V
58 0 V
58 -1 V
58 0 V
58 0 V
58 -1 V
58 0 V
58 0 V
58 -1 V
58 0 V
58 -1 V
57 0 V
58 0 V
58 -1 V
58 0 V
58 0 V
58 -1 V
58 0 V
58 -1 V
58 0 V
58 0 V
58 -1 V
58 0 V
58 0 V
58 -1 V
58 0 V
58 -1 V
58 0 V
58 0 V
58 -1 V
58 0 V
58 0 V
58 -1 V
58 0 V
58 -1 V
58 0 V
58 0 V
58 -1 V
58 0 V
58 0 V
58 -1 V
58 0 V
58 -1 V
58 0 V
57 0 V
58 -1 V
58 0 V
58 0 V
58 -1 V
58 0 V
58 -1 V
58 0 V
58 0 V
58 -1 V
58 0 V
58 0 V
58 -1 V
58 0 V
58 -1 V
58 0 V
58 0 V
58 -1 V
58 0 V
58 0 V
58 -1 V
58 0 V
58 -1 V
58 0 V
58 0 V
58 -1 V
58 0 V
58 0 V
58 -1 V
58 0 V
58 -1 V
58 0 V
58 0 V
57 -1 V
58 0 V
58 0 V
58 -1 V
58 0 V
58 -1 V
58 0 V
58 0 V
58 -1 V
58 0 V
58 0 V
58 -1 V
58 0 V
58 -1 V
58 0 V
58 0 V
58 -1 V
stroke
2.000 UL
LTb
LCb setrgbcolor
1.000 UL
LTb
LCb setrgbcolor
1100 7319 N
0 -6679 V
5739 0 V
0 6679 V
-5739 0 V
Z stroke
1.000 UP
1.000 UL
LTb
LCb setrgbcolor
stroke
grestore
end
showpage
  }}%
  \put(3969,140){\makebox(0,0){\large{$a^2\sigma$}}}%
  \put(160,4979){\makebox(0,0){\Large{$\frac{M_{J^{++}}}{\surd\sigma}$}}}%
  \put(6839,440){\makebox(0,0){\strut{}\ {$0.1$}}}%
  \put(6265,440){\makebox(0,0){\strut{}\ {$0.09$}}}%
  \put(5691,440){\makebox(0,0){\strut{}\ {$0.08$}}}%
  \put(5117,440){\makebox(0,0){\strut{}\ {$0.07$}}}%
  \put(4543,440){\makebox(0,0){\strut{}\ {$0.06$}}}%
  \put(3970,440){\makebox(0,0){\strut{}\ {$0.05$}}}%
  \put(3396,440){\makebox(0,0){\strut{}\ {$0.04$}}}%
  \put(2822,440){\makebox(0,0){\strut{}\ {$0.03$}}}%
  \put(2248,440){\makebox(0,0){\strut{}\ {$0.02$}}}%
  \put(1674,440){\makebox(0,0){\strut{}\ {$0.01$}}}%
  \put(1100,440){\makebox(0,0){\strut{}\ {$0$}}}%
  \put(980,7319){\makebox(0,0)[r]{\strut{}\ \ {$9$}}}%
  \put(980,6577){\makebox(0,0)[r]{\strut{}\ \ {$8$}}}%
  \put(980,5835){\makebox(0,0)[r]{\strut{}\ \ {$7$}}}%
  \put(980,5093){\makebox(0,0)[r]{\strut{}\ \ {$6$}}}%
  \put(980,4351){\makebox(0,0)[r]{\strut{}\ \ {$5$}}}%
  \put(980,3608){\makebox(0,0)[r]{\strut{}\ \ {$4$}}}%
  \put(980,2866){\makebox(0,0)[r]{\strut{}\ \ {$3$}}}%
  \put(980,2124){\makebox(0,0)[r]{\strut{}\ \ {$2$}}}%
  \put(980,1382){\makebox(0,0)[r]{\strut{}\ \ {$1$}}}%
  \put(980,640){\makebox(0,0)[r]{\strut{}\ \ {$0$}}}%
\end{picture}%
\endgroup
\endinput

\end	{center}
\caption{Lightest two glueball masses in the $A_1^{++}$ ($\bullet$), $E^{++}$ ($\blacklozenge$)
  and $T_2^{++}$ ($\lozenge$) sectors, in units of the string tension. Lines are linear
  extrapolations to the continuum limit. In that limit the  $A_1^{++}$ states become the
  lightest two $J^{PC}=0^{++}$ scalar glueballs while the doublet $E^{++}$ and triplet $T_2^{++}$
  pair up to give the five components of each of the lightest two $J^{PC}=2^{++}$ glueballs.
  All in $SU(4)$.}
\label{fig_MJ02ppK_cont_SU4}
\end{figure}



\begin{figure}[htb]
\begin	{center}
\leavevmode
% GNUPLOT: LaTeX picture with Postscript
\begingroup%
\makeatletter%
\newcommand{\GNUPLOTspecial}{%
  \@sanitize\catcode`\%=14\relax\special}%
\setlength{\unitlength}{0.0500bp}%
\begin{picture}(7200,7560)(0,0)%
  {\GNUPLOTspecial{"
%!PS-Adobe-2.0 EPSF-2.0
%%Title: plot_MJ02mpK_cont_SU4.tex
%%Creator: gnuplot 5.0 patchlevel 3
%%CreationDate: Sun Mar  7 09:25:32 2021
%%DocumentFonts: 
%%BoundingBox: 0 0 360 378
%%EndComments
%%BeginProlog
/gnudict 256 dict def
gnudict begin
%
% The following true/false flags may be edited by hand if desired.
% The unit line width and grayscale image gamma correction may also be changed.
%
/Color true def
/Blacktext true def
/Solid false def
/Dashlength 1 def
/Landscape false def
/Level1 false def
/Level3 false def
/Rounded false def
/ClipToBoundingBox false def
/SuppressPDFMark false def
/TransparentPatterns false def
/gnulinewidth 5.000 def
/userlinewidth gnulinewidth def
/Gamma 1.0 def
/BackgroundColor {-1.000 -1.000 -1.000} def
%
/vshift -66 def
/dl1 {
  10.0 Dashlength userlinewidth gnulinewidth div mul mul mul
  Rounded { currentlinewidth 0.75 mul sub dup 0 le { pop 0.01 } if } if
} def
/dl2 {
  10.0 Dashlength userlinewidth gnulinewidth div mul mul mul
  Rounded { currentlinewidth 0.75 mul add } if
} def
/hpt_ 31.5 def
/vpt_ 31.5 def
/hpt hpt_ def
/vpt vpt_ def
/doclip {
  ClipToBoundingBox {
    newpath 0 0 moveto 360 0 lineto 360 378 lineto 0 378 lineto closepath
    clip
  } if
} def
%
% Gnuplot Prolog Version 5.1 (Oct 2015)
%
%/SuppressPDFMark true def
%
/M {moveto} bind def
/L {lineto} bind def
/R {rmoveto} bind def
/V {rlineto} bind def
/N {newpath moveto} bind def
/Z {closepath} bind def
/C {setrgbcolor} bind def
/f {rlineto fill} bind def
/g {setgray} bind def
/Gshow {show} def   % May be redefined later in the file to support UTF-8
/vpt2 vpt 2 mul def
/hpt2 hpt 2 mul def
/Lshow {currentpoint stroke M 0 vshift R 
	Blacktext {gsave 0 setgray textshow grestore} {textshow} ifelse} def
/Rshow {currentpoint stroke M dup stringwidth pop neg vshift R
	Blacktext {gsave 0 setgray textshow grestore} {textshow} ifelse} def
/Cshow {currentpoint stroke M dup stringwidth pop -2 div vshift R 
	Blacktext {gsave 0 setgray textshow grestore} {textshow} ifelse} def
/UP {dup vpt_ mul /vpt exch def hpt_ mul /hpt exch def
  /hpt2 hpt 2 mul def /vpt2 vpt 2 mul def} def
/DL {Color {setrgbcolor Solid {pop []} if 0 setdash}
 {pop pop pop 0 setgray Solid {pop []} if 0 setdash} ifelse} def
/BL {stroke userlinewidth 2 mul setlinewidth
	Rounded {1 setlinejoin 1 setlinecap} if} def
/AL {stroke userlinewidth 2 div setlinewidth
	Rounded {1 setlinejoin 1 setlinecap} if} def
/UL {dup gnulinewidth mul /userlinewidth exch def
	dup 1 lt {pop 1} if 10 mul /udl exch def} def
/PL {stroke userlinewidth setlinewidth
	Rounded {1 setlinejoin 1 setlinecap} if} def
3.8 setmiterlimit
% Classic Line colors (version 5.0)
/LCw {1 1 1} def
/LCb {0 0 0} def
/LCa {0 0 0} def
/LC0 {1 0 0} def
/LC1 {0 1 0} def
/LC2 {0 0 1} def
/LC3 {1 0 1} def
/LC4 {0 1 1} def
/LC5 {1 1 0} def
/LC6 {0 0 0} def
/LC7 {1 0.3 0} def
/LC8 {0.5 0.5 0.5} def
% Default dash patterns (version 5.0)
/LTB {BL [] LCb DL} def
/LTw {PL [] 1 setgray} def
/LTb {PL [] LCb DL} def
/LTa {AL [1 udl mul 2 udl mul] 0 setdash LCa setrgbcolor} def
/LT0 {PL [] LC0 DL} def
/LT1 {PL [2 dl1 3 dl2] LC1 DL} def
/LT2 {PL [1 dl1 1.5 dl2] LC2 DL} def
/LT3 {PL [6 dl1 2 dl2 1 dl1 2 dl2] LC3 DL} def
/LT4 {PL [1 dl1 2 dl2 6 dl1 2 dl2 1 dl1 2 dl2] LC4 DL} def
/LT5 {PL [4 dl1 2 dl2] LC5 DL} def
/LT6 {PL [1.5 dl1 1.5 dl2 1.5 dl1 1.5 dl2 1.5 dl1 6 dl2] LC6 DL} def
/LT7 {PL [3 dl1 3 dl2 1 dl1 3 dl2] LC7 DL} def
/LT8 {PL [2 dl1 2 dl2 2 dl1 6 dl2] LC8 DL} def
/SL {[] 0 setdash} def
/Pnt {stroke [] 0 setdash gsave 1 setlinecap M 0 0 V stroke grestore} def
/Dia {stroke [] 0 setdash 2 copy vpt add M
  hpt neg vpt neg V hpt vpt neg V
  hpt vpt V hpt neg vpt V closepath stroke
  Pnt} def
/Pls {stroke [] 0 setdash vpt sub M 0 vpt2 V
  currentpoint stroke M
  hpt neg vpt neg R hpt2 0 V stroke
 } def
/Box {stroke [] 0 setdash 2 copy exch hpt sub exch vpt add M
  0 vpt2 neg V hpt2 0 V 0 vpt2 V
  hpt2 neg 0 V closepath stroke
  Pnt} def
/Crs {stroke [] 0 setdash exch hpt sub exch vpt add M
  hpt2 vpt2 neg V currentpoint stroke M
  hpt2 neg 0 R hpt2 vpt2 V stroke} def
/TriU {stroke [] 0 setdash 2 copy vpt 1.12 mul add M
  hpt neg vpt -1.62 mul V
  hpt 2 mul 0 V
  hpt neg vpt 1.62 mul V closepath stroke
  Pnt} def
/Star {2 copy Pls Crs} def
/BoxF {stroke [] 0 setdash exch hpt sub exch vpt add M
  0 vpt2 neg V hpt2 0 V 0 vpt2 V
  hpt2 neg 0 V closepath fill} def
/TriUF {stroke [] 0 setdash vpt 1.12 mul add M
  hpt neg vpt -1.62 mul V
  hpt 2 mul 0 V
  hpt neg vpt 1.62 mul V closepath fill} def
/TriD {stroke [] 0 setdash 2 copy vpt 1.12 mul sub M
  hpt neg vpt 1.62 mul V
  hpt 2 mul 0 V
  hpt neg vpt -1.62 mul V closepath stroke
  Pnt} def
/TriDF {stroke [] 0 setdash vpt 1.12 mul sub M
  hpt neg vpt 1.62 mul V
  hpt 2 mul 0 V
  hpt neg vpt -1.62 mul V closepath fill} def
/DiaF {stroke [] 0 setdash vpt add M
  hpt neg vpt neg V hpt vpt neg V
  hpt vpt V hpt neg vpt V closepath fill} def
/Pent {stroke [] 0 setdash 2 copy gsave
  translate 0 hpt M 4 {72 rotate 0 hpt L} repeat
  closepath stroke grestore Pnt} def
/PentF {stroke [] 0 setdash gsave
  translate 0 hpt M 4 {72 rotate 0 hpt L} repeat
  closepath fill grestore} def
/Circle {stroke [] 0 setdash 2 copy
  hpt 0 360 arc stroke Pnt} def
/CircleF {stroke [] 0 setdash hpt 0 360 arc fill} def
/C0 {BL [] 0 setdash 2 copy moveto vpt 90 450 arc} bind def
/C1 {BL [] 0 setdash 2 copy moveto
	2 copy vpt 0 90 arc closepath fill
	vpt 0 360 arc closepath} bind def
/C2 {BL [] 0 setdash 2 copy moveto
	2 copy vpt 90 180 arc closepath fill
	vpt 0 360 arc closepath} bind def
/C3 {BL [] 0 setdash 2 copy moveto
	2 copy vpt 0 180 arc closepath fill
	vpt 0 360 arc closepath} bind def
/C4 {BL [] 0 setdash 2 copy moveto
	2 copy vpt 180 270 arc closepath fill
	vpt 0 360 arc closepath} bind def
/C5 {BL [] 0 setdash 2 copy moveto
	2 copy vpt 0 90 arc
	2 copy moveto
	2 copy vpt 180 270 arc closepath fill
	vpt 0 360 arc} bind def
/C6 {BL [] 0 setdash 2 copy moveto
	2 copy vpt 90 270 arc closepath fill
	vpt 0 360 arc closepath} bind def
/C7 {BL [] 0 setdash 2 copy moveto
	2 copy vpt 0 270 arc closepath fill
	vpt 0 360 arc closepath} bind def
/C8 {BL [] 0 setdash 2 copy moveto
	2 copy vpt 270 360 arc closepath fill
	vpt 0 360 arc closepath} bind def
/C9 {BL [] 0 setdash 2 copy moveto
	2 copy vpt 270 450 arc closepath fill
	vpt 0 360 arc closepath} bind def
/C10 {BL [] 0 setdash 2 copy 2 copy moveto vpt 270 360 arc closepath fill
	2 copy moveto
	2 copy vpt 90 180 arc closepath fill
	vpt 0 360 arc closepath} bind def
/C11 {BL [] 0 setdash 2 copy moveto
	2 copy vpt 0 180 arc closepath fill
	2 copy moveto
	2 copy vpt 270 360 arc closepath fill
	vpt 0 360 arc closepath} bind def
/C12 {BL [] 0 setdash 2 copy moveto
	2 copy vpt 180 360 arc closepath fill
	vpt 0 360 arc closepath} bind def
/C13 {BL [] 0 setdash 2 copy moveto
	2 copy vpt 0 90 arc closepath fill
	2 copy moveto
	2 copy vpt 180 360 arc closepath fill
	vpt 0 360 arc closepath} bind def
/C14 {BL [] 0 setdash 2 copy moveto
	2 copy vpt 90 360 arc closepath fill
	vpt 0 360 arc} bind def
/C15 {BL [] 0 setdash 2 copy vpt 0 360 arc closepath fill
	vpt 0 360 arc closepath} bind def
/Rec {newpath 4 2 roll moveto 1 index 0 rlineto 0 exch rlineto
	neg 0 rlineto closepath} bind def
/Square {dup Rec} bind def
/Bsquare {vpt sub exch vpt sub exch vpt2 Square} bind def
/S0 {BL [] 0 setdash 2 copy moveto 0 vpt rlineto BL Bsquare} bind def
/S1 {BL [] 0 setdash 2 copy vpt Square fill Bsquare} bind def
/S2 {BL [] 0 setdash 2 copy exch vpt sub exch vpt Square fill Bsquare} bind def
/S3 {BL [] 0 setdash 2 copy exch vpt sub exch vpt2 vpt Rec fill Bsquare} bind def
/S4 {BL [] 0 setdash 2 copy exch vpt sub exch vpt sub vpt Square fill Bsquare} bind def
/S5 {BL [] 0 setdash 2 copy 2 copy vpt Square fill
	exch vpt sub exch vpt sub vpt Square fill Bsquare} bind def
/S6 {BL [] 0 setdash 2 copy exch vpt sub exch vpt sub vpt vpt2 Rec fill Bsquare} bind def
/S7 {BL [] 0 setdash 2 copy exch vpt sub exch vpt sub vpt vpt2 Rec fill
	2 copy vpt Square fill Bsquare} bind def
/S8 {BL [] 0 setdash 2 copy vpt sub vpt Square fill Bsquare} bind def
/S9 {BL [] 0 setdash 2 copy vpt sub vpt vpt2 Rec fill Bsquare} bind def
/S10 {BL [] 0 setdash 2 copy vpt sub vpt Square fill 2 copy exch vpt sub exch vpt Square fill
	Bsquare} bind def
/S11 {BL [] 0 setdash 2 copy vpt sub vpt Square fill 2 copy exch vpt sub exch vpt2 vpt Rec fill
	Bsquare} bind def
/S12 {BL [] 0 setdash 2 copy exch vpt sub exch vpt sub vpt2 vpt Rec fill Bsquare} bind def
/S13 {BL [] 0 setdash 2 copy exch vpt sub exch vpt sub vpt2 vpt Rec fill
	2 copy vpt Square fill Bsquare} bind def
/S14 {BL [] 0 setdash 2 copy exch vpt sub exch vpt sub vpt2 vpt Rec fill
	2 copy exch vpt sub exch vpt Square fill Bsquare} bind def
/S15 {BL [] 0 setdash 2 copy Bsquare fill Bsquare} bind def
/D0 {gsave translate 45 rotate 0 0 S0 stroke grestore} bind def
/D1 {gsave translate 45 rotate 0 0 S1 stroke grestore} bind def
/D2 {gsave translate 45 rotate 0 0 S2 stroke grestore} bind def
/D3 {gsave translate 45 rotate 0 0 S3 stroke grestore} bind def
/D4 {gsave translate 45 rotate 0 0 S4 stroke grestore} bind def
/D5 {gsave translate 45 rotate 0 0 S5 stroke grestore} bind def
/D6 {gsave translate 45 rotate 0 0 S6 stroke grestore} bind def
/D7 {gsave translate 45 rotate 0 0 S7 stroke grestore} bind def
/D8 {gsave translate 45 rotate 0 0 S8 stroke grestore} bind def
/D9 {gsave translate 45 rotate 0 0 S9 stroke grestore} bind def
/D10 {gsave translate 45 rotate 0 0 S10 stroke grestore} bind def
/D11 {gsave translate 45 rotate 0 0 S11 stroke grestore} bind def
/D12 {gsave translate 45 rotate 0 0 S12 stroke grestore} bind def
/D13 {gsave translate 45 rotate 0 0 S13 stroke grestore} bind def
/D14 {gsave translate 45 rotate 0 0 S14 stroke grestore} bind def
/D15 {gsave translate 45 rotate 0 0 S15 stroke grestore} bind def
/DiaE {stroke [] 0 setdash vpt add M
  hpt neg vpt neg V hpt vpt neg V
  hpt vpt V hpt neg vpt V closepath stroke} def
/BoxE {stroke [] 0 setdash exch hpt sub exch vpt add M
  0 vpt2 neg V hpt2 0 V 0 vpt2 V
  hpt2 neg 0 V closepath stroke} def
/TriUE {stroke [] 0 setdash vpt 1.12 mul add M
  hpt neg vpt -1.62 mul V
  hpt 2 mul 0 V
  hpt neg vpt 1.62 mul V closepath stroke} def
/TriDE {stroke [] 0 setdash vpt 1.12 mul sub M
  hpt neg vpt 1.62 mul V
  hpt 2 mul 0 V
  hpt neg vpt -1.62 mul V closepath stroke} def
/PentE {stroke [] 0 setdash gsave
  translate 0 hpt M 4 {72 rotate 0 hpt L} repeat
  closepath stroke grestore} def
/CircE {stroke [] 0 setdash 
  hpt 0 360 arc stroke} def
/Opaque {gsave closepath 1 setgray fill grestore 0 setgray closepath} def
/DiaW {stroke [] 0 setdash vpt add M
  hpt neg vpt neg V hpt vpt neg V
  hpt vpt V hpt neg vpt V Opaque stroke} def
/BoxW {stroke [] 0 setdash exch hpt sub exch vpt add M
  0 vpt2 neg V hpt2 0 V 0 vpt2 V
  hpt2 neg 0 V Opaque stroke} def
/TriUW {stroke [] 0 setdash vpt 1.12 mul add M
  hpt neg vpt -1.62 mul V
  hpt 2 mul 0 V
  hpt neg vpt 1.62 mul V Opaque stroke} def
/TriDW {stroke [] 0 setdash vpt 1.12 mul sub M
  hpt neg vpt 1.62 mul V
  hpt 2 mul 0 V
  hpt neg vpt -1.62 mul V Opaque stroke} def
/PentW {stroke [] 0 setdash gsave
  translate 0 hpt M 4 {72 rotate 0 hpt L} repeat
  Opaque stroke grestore} def
/CircW {stroke [] 0 setdash 
  hpt 0 360 arc Opaque stroke} def
/BoxFill {gsave Rec 1 setgray fill grestore} def
/Density {
  /Fillden exch def
  currentrgbcolor
  /ColB exch def /ColG exch def /ColR exch def
  /ColR ColR Fillden mul Fillden sub 1 add def
  /ColG ColG Fillden mul Fillden sub 1 add def
  /ColB ColB Fillden mul Fillden sub 1 add def
  ColR ColG ColB setrgbcolor} def
/BoxColFill {gsave Rec PolyFill} def
/PolyFill {gsave Density fill grestore grestore} def
/h {rlineto rlineto rlineto gsave closepath fill grestore} bind def
%
% PostScript Level 1 Pattern Fill routine for rectangles
% Usage: x y w h s a XX PatternFill
%	x,y = lower left corner of box to be filled
%	w,h = width and height of box
%	  a = angle in degrees between lines and x-axis
%	 XX = 0/1 for no/yes cross-hatch
%
/PatternFill {gsave /PFa [ 9 2 roll ] def
  PFa 0 get PFa 2 get 2 div add PFa 1 get PFa 3 get 2 div add translate
  PFa 2 get -2 div PFa 3 get -2 div PFa 2 get PFa 3 get Rec
  TransparentPatterns {} {gsave 1 setgray fill grestore} ifelse
  clip
  currentlinewidth 0.5 mul setlinewidth
  /PFs PFa 2 get dup mul PFa 3 get dup mul add sqrt def
  0 0 M PFa 5 get rotate PFs -2 div dup translate
  0 1 PFs PFa 4 get div 1 add floor cvi
	{PFa 4 get mul 0 M 0 PFs V} for
  0 PFa 6 get ne {
	0 1 PFs PFa 4 get div 1 add floor cvi
	{PFa 4 get mul 0 2 1 roll M PFs 0 V} for
 } if
  stroke grestore} def
%
/languagelevel where
 {pop languagelevel} {1} ifelse
dup 2 lt
	{/InterpretLevel1 true def
	 /InterpretLevel3 false def}
	{/InterpretLevel1 Level1 def
	 2 gt
	    {/InterpretLevel3 Level3 def}
	    {/InterpretLevel3 false def}
	 ifelse }
 ifelse
%
% PostScript level 2 pattern fill definitions
%
/Level2PatternFill {
/Tile8x8 {/PaintType 2 /PatternType 1 /TilingType 1 /BBox [0 0 8 8] /XStep 8 /YStep 8}
	bind def
/KeepColor {currentrgbcolor [/Pattern /DeviceRGB] setcolorspace} bind def
<< Tile8x8
 /PaintProc {0.5 setlinewidth pop 0 0 M 8 8 L 0 8 M 8 0 L stroke} 
>> matrix makepattern
/Pat1 exch def
<< Tile8x8
 /PaintProc {0.5 setlinewidth pop 0 0 M 8 8 L 0 8 M 8 0 L stroke
	0 4 M 4 8 L 8 4 L 4 0 L 0 4 L stroke}
>> matrix makepattern
/Pat2 exch def
<< Tile8x8
 /PaintProc {0.5 setlinewidth pop 0 0 M 0 8 L
	8 8 L 8 0 L 0 0 L fill}
>> matrix makepattern
/Pat3 exch def
<< Tile8x8
 /PaintProc {0.5 setlinewidth pop -4 8 M 8 -4 L
	0 12 M 12 0 L stroke}
>> matrix makepattern
/Pat4 exch def
<< Tile8x8
 /PaintProc {0.5 setlinewidth pop -4 0 M 8 12 L
	0 -4 M 12 8 L stroke}
>> matrix makepattern
/Pat5 exch def
<< Tile8x8
 /PaintProc {0.5 setlinewidth pop -2 8 M 4 -4 L
	0 12 M 8 -4 L 4 12 M 10 0 L stroke}
>> matrix makepattern
/Pat6 exch def
<< Tile8x8
 /PaintProc {0.5 setlinewidth pop -2 0 M 4 12 L
	0 -4 M 8 12 L 4 -4 M 10 8 L stroke}
>> matrix makepattern
/Pat7 exch def
<< Tile8x8
 /PaintProc {0.5 setlinewidth pop 8 -2 M -4 4 L
	12 0 M -4 8 L 12 4 M 0 10 L stroke}
>> matrix makepattern
/Pat8 exch def
<< Tile8x8
 /PaintProc {0.5 setlinewidth pop 0 -2 M 12 4 L
	-4 0 M 12 8 L -4 4 M 8 10 L stroke}
>> matrix makepattern
/Pat9 exch def
/Pattern1 {PatternBgnd KeepColor Pat1 setpattern} bind def
/Pattern2 {PatternBgnd KeepColor Pat2 setpattern} bind def
/Pattern3 {PatternBgnd KeepColor Pat3 setpattern} bind def
/Pattern4 {PatternBgnd KeepColor Landscape {Pat5} {Pat4} ifelse setpattern} bind def
/Pattern5 {PatternBgnd KeepColor Landscape {Pat4} {Pat5} ifelse setpattern} bind def
/Pattern6 {PatternBgnd KeepColor Landscape {Pat9} {Pat6} ifelse setpattern} bind def
/Pattern7 {PatternBgnd KeepColor Landscape {Pat8} {Pat7} ifelse setpattern} bind def
} def
%
%
%End of PostScript Level 2 code
%
/PatternBgnd {
  TransparentPatterns {} {gsave 1 setgray fill grestore} ifelse
} def
%
% Substitute for Level 2 pattern fill codes with
% grayscale if Level 2 support is not selected.
%
/Level1PatternFill {
/Pattern1 {0.250 Density} bind def
/Pattern2 {0.500 Density} bind def
/Pattern3 {0.750 Density} bind def
/Pattern4 {0.125 Density} bind def
/Pattern5 {0.375 Density} bind def
/Pattern6 {0.625 Density} bind def
/Pattern7 {0.875 Density} bind def
} def
%
% Now test for support of Level 2 code
%
Level1 {Level1PatternFill} {Level2PatternFill} ifelse
%
/Symbol-Oblique /Symbol findfont [1 0 .167 1 0 0] makefont
dup length dict begin {1 index /FID eq {pop pop} {def} ifelse} forall
currentdict end definefont pop
%
Level1 SuppressPDFMark or 
{} {
/SDict 10 dict def
systemdict /pdfmark known not {
  userdict /pdfmark systemdict /cleartomark get put
} if
SDict begin [
  /Title (plot_MJ02mpK_cont_SU4.tex)
  /Subject (gnuplot plot)
  /Creator (gnuplot 5.0 patchlevel 3)
  /Author (mteper)
%  /Producer (gnuplot)
%  /Keywords ()
  /CreationDate (Sun Mar  7 09:25:32 2021)
  /DOCINFO pdfmark
end
} ifelse
%
% Support for boxed text - Ethan A Merritt May 2005
%
/InitTextBox { userdict /TBy2 3 -1 roll put userdict /TBx2 3 -1 roll put
           userdict /TBy1 3 -1 roll put userdict /TBx1 3 -1 roll put
	   /Boxing true def } def
/ExtendTextBox { Boxing
    { gsave dup false charpath pathbbox
      dup TBy2 gt {userdict /TBy2 3 -1 roll put} {pop} ifelse
      dup TBx2 gt {userdict /TBx2 3 -1 roll put} {pop} ifelse
      dup TBy1 lt {userdict /TBy1 3 -1 roll put} {pop} ifelse
      dup TBx1 lt {userdict /TBx1 3 -1 roll put} {pop} ifelse
      grestore } if } def
/PopTextBox { newpath TBx1 TBxmargin sub TBy1 TBymargin sub M
               TBx1 TBxmargin sub TBy2 TBymargin add L
	       TBx2 TBxmargin add TBy2 TBymargin add L
	       TBx2 TBxmargin add TBy1 TBymargin sub L closepath } def
/DrawTextBox { PopTextBox stroke /Boxing false def} def
/FillTextBox { gsave PopTextBox 1 1 1 setrgbcolor fill grestore /Boxing false def} def
0 0 0 0 InitTextBox
/TBxmargin 20 def
/TBymargin 20 def
/Boxing false def
/textshow { ExtendTextBox Gshow } def
%
% redundant definitions for compatibility with prologue.ps older than 5.0.2
/LTB {BL [] LCb DL} def
/LTb {PL [] LCb DL} def
end
%%EndProlog
%%Page: 1 1
gnudict begin
gsave
doclip
0 0 translate
0.050 0.050 scale
0 setgray
newpath
BackgroundColor 0 lt 3 1 roll 0 lt exch 0 lt or or not {BackgroundColor C 1.000 0 0 7200.00 7560.00 BoxColFill} if
1.000 UL
LTb
LCb setrgbcolor
1220 640 M
63 0 V
5556 0 R
-63 0 V
stroke
LTb
LCb setrgbcolor
1220 1594 M
63 0 V
5556 0 R
-63 0 V
stroke
LTb
LCb setrgbcolor
1220 2548 M
63 0 V
5556 0 R
-63 0 V
stroke
LTb
LCb setrgbcolor
1220 3502 M
63 0 V
5556 0 R
-63 0 V
stroke
LTb
LCb setrgbcolor
1220 4457 M
63 0 V
5556 0 R
-63 0 V
stroke
LTb
LCb setrgbcolor
1220 5411 M
63 0 V
5556 0 R
-63 0 V
stroke
LTb
LCb setrgbcolor
1220 6365 M
63 0 V
5556 0 R
-63 0 V
stroke
LTb
LCb setrgbcolor
1220 7319 M
63 0 V
5556 0 R
-63 0 V
stroke
LTb
LCb setrgbcolor
1220 640 M
0 63 V
0 6616 R
0 -63 V
stroke
LTb
LCb setrgbcolor
1782 640 M
0 63 V
0 6616 R
0 -63 V
stroke
LTb
LCb setrgbcolor
2344 640 M
0 63 V
0 6616 R
0 -63 V
stroke
LTb
LCb setrgbcolor
2906 640 M
0 63 V
0 6616 R
0 -63 V
stroke
LTb
LCb setrgbcolor
3468 640 M
0 63 V
0 6616 R
0 -63 V
stroke
LTb
LCb setrgbcolor
4030 640 M
0 63 V
0 6616 R
0 -63 V
stroke
LTb
LCb setrgbcolor
4591 640 M
0 63 V
0 6616 R
0 -63 V
stroke
LTb
LCb setrgbcolor
5153 640 M
0 63 V
0 6616 R
0 -63 V
stroke
LTb
LCb setrgbcolor
5715 640 M
0 63 V
0 6616 R
0 -63 V
stroke
LTb
LCb setrgbcolor
6277 640 M
0 63 V
0 6616 R
0 -63 V
stroke
LTb
LCb setrgbcolor
6839 640 M
0 63 V
0 6616 R
0 -63 V
stroke
LTb
LCb setrgbcolor
1.000 UL
LTb
LCb setrgbcolor
1220 7319 N
0 -6679 V
5619 0 V
0 6679 V
-5619 0 V
Z stroke
1.000 UP
1.000 UL
LTb
LCb setrgbcolor
LCb setrgbcolor
LTb
LCb setrgbcolor
LTb
1.500 UP
1.000 UL
LTb
0.58 0.00 0.83 C 6348 2631 M
0 146 V
4853 2622 M
0 113 V
3801 2411 M
0 188 V
-730 -28 R
0 149 V
2536 2540 M
0 95 V
-357 -71 R
0 85 V
6348 4293 M
0 386 V
-1495 78 R
0 376 V
3801 4521 M
0 232 V
3071 4636 M
0 152 V
2536 4379 M
0 239 V
-357 163 R
0 205 V
6348 2704 CircleF
4853 2678 CircleF
3801 2505 CircleF
3071 2646 CircleF
2536 2587 CircleF
2179 2606 CircleF
6348 4486 CircleF
4853 4945 CircleF
3801 4637 CircleF
3071 4712 CircleF
2536 4498 CircleF
2179 4883 CircleF
1.500 UL
LTb
0.58 0.00 0.83 C 1220 2567 M
57 2 V
57 1 V
56 2 V
57 1 V
57 2 V
57 1 V
56 1 V
57 2 V
57 1 V
57 2 V
56 1 V
57 2 V
57 1 V
57 2 V
56 1 V
57 1 V
57 2 V
57 1 V
56 2 V
57 1 V
57 2 V
57 1 V
56 1 V
57 2 V
57 1 V
57 2 V
56 1 V
57 2 V
57 1 V
57 2 V
56 1 V
57 1 V
57 2 V
57 1 V
57 2 V
56 1 V
57 2 V
57 1 V
57 1 V
56 2 V
57 1 V
57 2 V
57 1 V
56 2 V
57 1 V
57 2 V
57 1 V
56 1 V
57 2 V
57 1 V
57 2 V
56 1 V
57 2 V
57 1 V
57 2 V
56 1 V
57 1 V
57 2 V
57 1 V
56 2 V
57 1 V
57 2 V
57 1 V
56 1 V
57 2 V
57 1 V
57 2 V
57 1 V
56 2 V
57 1 V
57 2 V
57 1 V
56 1 V
57 2 V
57 1 V
57 2 V
56 1 V
57 2 V
57 1 V
57 1 V
56 2 V
57 1 V
57 2 V
57 1 V
56 2 V
57 1 V
57 2 V
57 1 V
56 1 V
57 2 V
57 1 V
57 2 V
56 1 V
57 2 V
57 1 V
57 2 V
56 1 V
57 1 V
57 2 V
stroke
LTb
0.58 0.00 0.83 C 1220 4776 M
57 -2 V
57 -2 V
56 -2 V
57 -2 V
57 -2 V
57 -2 V
56 -2 V
57 -2 V
57 -2 V
57 -2 V
56 -2 V
57 -2 V
57 -2 V
57 -2 V
56 -2 V
57 -2 V
57 -2 V
57 -3 V
56 -2 V
57 -2 V
57 -2 V
57 -2 V
56 -2 V
57 -2 V
57 -2 V
57 -2 V
56 -2 V
57 -2 V
57 -2 V
57 -2 V
56 -2 V
57 -2 V
57 -2 V
57 -2 V
57 -2 V
56 -2 V
57 -2 V
57 -2 V
57 -2 V
56 -2 V
57 -2 V
57 -2 V
57 -3 V
56 -2 V
57 -2 V
57 -2 V
57 -2 V
56 -2 V
57 -2 V
57 -2 V
57 -2 V
56 -2 V
57 -2 V
57 -2 V
57 -2 V
56 -2 V
57 -2 V
57 -2 V
57 -2 V
56 -2 V
57 -2 V
57 -2 V
57 -2 V
56 -2 V
57 -2 V
57 -2 V
57 -2 V
57 -3 V
56 -2 V
57 -2 V
57 -2 V
57 -2 V
56 -2 V
57 -2 V
57 -2 V
57 -2 V
56 -2 V
57 -2 V
57 -2 V
57 -2 V
56 -2 V
57 -2 V
57 -2 V
57 -2 V
56 -2 V
57 -2 V
57 -2 V
57 -2 V
56 -2 V
57 -2 V
57 -2 V
57 -3 V
56 -2 V
57 -2 V
57 -2 V
57 -2 V
56 -2 V
57 -2 V
57 -2 V
1.500 UP
stroke
1.000 UL
LTb
0.58 0.00 0.83 C 6348 3591 M
0 166 V
4853 3676 M
0 114 V
3801 3688 M
0 91 V
3071 3527 M
0 172 V
2536 3527 M
0 110 V
-357 -10 R
0 98 V
6348 4947 M
0 505 V
4853 5252 M
0 271 V
3801 5264 M
0 171 V
3071 5250 M
0 112 V
2536 5251 M
0 264 V
2179 4978 M
0 526 V
6348 3674 DiaF
4853 3733 DiaF
3801 3734 DiaF
3071 3613 DiaF
2536 3582 DiaF
2179 3676 DiaF
6348 5199 DiaF
4853 5388 DiaF
3801 5349 DiaF
3071 5306 DiaF
2536 5383 DiaF
2179 5241 DiaF
1.500 UL
LTb
0.58 0.00 0.83 C 1220 3626 M
57 2 V
57 1 V
56 1 V
57 2 V
57 1 V
57 1 V
56 1 V
57 2 V
57 1 V
57 1 V
56 2 V
57 1 V
57 1 V
57 1 V
56 2 V
57 1 V
57 1 V
57 2 V
56 1 V
57 1 V
57 2 V
57 1 V
56 1 V
57 1 V
57 2 V
57 1 V
56 1 V
57 2 V
57 1 V
57 1 V
56 1 V
57 2 V
57 1 V
57 1 V
57 2 V
56 1 V
57 1 V
57 1 V
57 2 V
56 1 V
57 1 V
57 2 V
57 1 V
56 1 V
57 1 V
57 2 V
57 1 V
56 1 V
57 2 V
57 1 V
57 1 V
56 1 V
57 2 V
57 1 V
57 1 V
56 2 V
57 1 V
57 1 V
57 1 V
56 2 V
57 1 V
57 1 V
57 2 V
56 1 V
57 1 V
57 1 V
57 2 V
57 1 V
56 1 V
57 2 V
57 1 V
57 1 V
56 1 V
57 2 V
57 1 V
57 1 V
56 2 V
57 1 V
57 1 V
57 1 V
56 2 V
57 1 V
57 1 V
57 2 V
56 1 V
57 1 V
57 1 V
57 2 V
56 1 V
57 1 V
57 2 V
57 1 V
56 1 V
57 2 V
57 1 V
57 1 V
56 1 V
57 2 V
57 1 V
stroke
LTb
0.58 0.00 0.83 C 1220 5320 M
57 0 V
57 0 V
56 1 V
57 0 V
57 0 V
57 0 V
56 0 V
57 0 V
57 1 V
57 0 V
56 0 V
57 0 V
57 0 V
57 0 V
56 1 V
57 0 V
57 0 V
57 0 V
56 0 V
57 0 V
57 1 V
57 0 V
56 0 V
57 0 V
57 0 V
57 0 V
56 1 V
57 0 V
57 0 V
57 0 V
56 0 V
57 0 V
57 1 V
57 0 V
57 0 V
56 0 V
57 0 V
57 0 V
57 1 V
56 0 V
57 0 V
57 0 V
57 0 V
56 0 V
57 1 V
57 0 V
57 0 V
56 0 V
57 0 V
57 0 V
57 1 V
56 0 V
57 0 V
57 0 V
57 0 V
56 0 V
57 1 V
57 0 V
57 0 V
56 0 V
57 0 V
57 0 V
57 1 V
56 0 V
57 0 V
57 0 V
57 0 V
57 0 V
56 1 V
57 0 V
57 0 V
57 0 V
56 0 V
57 0 V
57 1 V
57 0 V
56 0 V
57 0 V
57 0 V
57 0 V
56 1 V
57 0 V
57 0 V
57 0 V
56 0 V
57 0 V
57 1 V
57 0 V
56 0 V
57 0 V
57 0 V
57 0 V
56 1 V
57 0 V
57 0 V
57 0 V
56 0 V
57 0 V
57 1 V
1.500 UP
stroke
1.000 UL
LTb
0.58 0.00 0.83 C 6348 3617 M
0 159 V
4853 3638 M
0 77 V
3801 3658 M
0 187 V
3071 3501 M
0 150 V
-535 48 R
0 89 V
2179 3660 M
0 52 V
6348 5357 M
0 443 V
4853 5369 M
0 293 V
3801 5171 M
0 108 V
3071 4816 M
0 328 V
-535 238 R
0 264 V
2179 5430 M
0 134 V
6348 3696 Dia
4853 3677 Dia
3801 3751 Dia
3071 3576 Dia
2536 3743 Dia
2179 3686 Dia
6348 5578 Dia
4853 5515 Dia
3801 5225 Dia
3071 4980 Dia
2536 5514 Dia
2179 5497 Dia
1.500 UL
LTb
0.58 0.00 0.83 C 1220 3696 M
57 0 V
57 0 V
56 0 V
57 -1 V
57 0 V
57 0 V
56 0 V
57 0 V
57 0 V
57 -1 V
56 0 V
57 0 V
57 0 V
57 0 V
56 -1 V
57 0 V
57 0 V
57 0 V
56 0 V
57 0 V
57 -1 V
57 0 V
56 0 V
57 0 V
57 0 V
57 0 V
56 -1 V
57 0 V
57 0 V
57 0 V
56 0 V
57 -1 V
57 0 V
57 0 V
57 0 V
56 0 V
57 0 V
57 -1 V
57 0 V
56 0 V
57 0 V
57 0 V
57 -1 V
56 0 V
57 0 V
57 0 V
57 0 V
56 0 V
57 -1 V
57 0 V
57 0 V
56 0 V
57 0 V
57 0 V
57 -1 V
56 0 V
57 0 V
57 0 V
57 0 V
56 -1 V
57 0 V
57 0 V
57 0 V
56 0 V
57 0 V
57 -1 V
57 0 V
57 0 V
56 0 V
57 0 V
57 0 V
57 -1 V
56 0 V
57 0 V
57 0 V
57 0 V
56 -1 V
57 0 V
57 0 V
57 0 V
56 0 V
57 0 V
57 -1 V
57 0 V
56 0 V
57 0 V
57 0 V
57 0 V
56 -1 V
57 0 V
57 0 V
57 0 V
56 0 V
57 -1 V
57 0 V
57 0 V
56 0 V
57 0 V
57 0 V
stroke
LTb
0.58 0.00 0.83 C 1220 5455 M
57 -3 V
57 -3 V
56 -3 V
57 -3 V
57 -3 V
57 -3 V
56 -3 V
57 -3 V
57 -3 V
57 -3 V
56 -2 V
57 -3 V
57 -3 V
57 -3 V
56 -3 V
57 -3 V
57 -3 V
57 -3 V
56 -3 V
57 -3 V
57 -3 V
57 -2 V
56 -3 V
57 -3 V
57 -3 V
57 -3 V
56 -3 V
57 -3 V
57 -3 V
57 -3 V
56 -3 V
57 -3 V
57 -3 V
57 -2 V
57 -3 V
56 -3 V
57 -3 V
57 -3 V
57 -3 V
56 -3 V
57 -3 V
57 -3 V
57 -3 V
56 -3 V
57 -2 V
57 -3 V
57 -3 V
56 -3 V
57 -3 V
57 -3 V
57 -3 V
56 -3 V
57 -3 V
57 -3 V
57 -3 V
56 -2 V
57 -3 V
57 -3 V
57 -3 V
56 -3 V
57 -3 V
57 -3 V
57 -3 V
56 -3 V
57 -3 V
57 -3 V
57 -3 V
57 -2 V
56 -3 V
57 -3 V
57 -3 V
57 -3 V
56 -3 V
57 -3 V
57 -3 V
57 -3 V
56 -3 V
57 -3 V
57 -2 V
57 -3 V
56 -3 V
57 -3 V
57 -3 V
57 -3 V
56 -3 V
57 -3 V
57 -3 V
57 -3 V
56 -3 V
57 -3 V
57 -2 V
57 -3 V
56 -3 V
57 -3 V
57 -3 V
57 -3 V
56 -3 V
57 -3 V
57 -3 V
stroke
2.000 UL
LTb
LCb setrgbcolor
1.000 UL
LTb
LCb setrgbcolor
1220 7319 N
0 -6679 V
5619 0 V
0 6679 V
-5619 0 V
Z stroke
1.000 UP
1.000 UL
LTb
LCb setrgbcolor
stroke
grestore
end
showpage
  }}%
  \put(4029,140){\makebox(0,0){\large{$a^2\sigma$}}}%
  \put(160,4979){\makebox(0,0){\Large{$\frac{M_{J^{-+}}}{\surd\sigma}$}}}%
  \put(6839,440){\makebox(0,0){\strut{}\ {$0.1$}}}%
  \put(6277,440){\makebox(0,0){\strut{}\ {$0.09$}}}%
  \put(5715,440){\makebox(0,0){\strut{}\ {$0.08$}}}%
  \put(5153,440){\makebox(0,0){\strut{}\ {$0.07$}}}%
  \put(4591,440){\makebox(0,0){\strut{}\ {$0.06$}}}%
  \put(4030,440){\makebox(0,0){\strut{}\ {$0.05$}}}%
  \put(3468,440){\makebox(0,0){\strut{}\ {$0.04$}}}%
  \put(2906,440){\makebox(0,0){\strut{}\ {$0.03$}}}%
  \put(2344,440){\makebox(0,0){\strut{}\ {$0.02$}}}%
  \put(1782,440){\makebox(0,0){\strut{}\ {$0.01$}}}%
  \put(1220,440){\makebox(0,0){\strut{}\ {$0$}}}%
  \put(1100,7319){\makebox(0,0)[r]{\strut{}\ \ {$10$}}}%
  \put(1100,6365){\makebox(0,0)[r]{\strut{}\ \ {$9$}}}%
  \put(1100,5411){\makebox(0,0)[r]{\strut{}\ \ {$8$}}}%
  \put(1100,4457){\makebox(0,0)[r]{\strut{}\ \ {$7$}}}%
  \put(1100,3502){\makebox(0,0)[r]{\strut{}\ \ {$6$}}}%
  \put(1100,2548){\makebox(0,0)[r]{\strut{}\ \ {$5$}}}%
  \put(1100,1594){\makebox(0,0)[r]{\strut{}\ \ {$4$}}}%
  \put(1100,640){\makebox(0,0)[r]{\strut{}\ \ {$3$}}}%
\end{picture}%
\endgroup
\endinput

\end	{center}
\caption{Lightest two glueball masses in the $A_1^{-+}$ ($\bullet$), $E^{-+}$ ($\blacklozenge$)
  and $T_2^{-+}$ ($\lozenge$) sectors, in units of the string tension. Lines are linear
  extrapolations to the continuum limit. In that limit the  $A_1^{-+}$ states become the
  lightest two $J^{PC}=0^{-+}$ pseudoscalar glueballs while the doublet $E^{--+}$ and triplet $T_2^{-+}$
  pair up to give the five components of each of the lightest two $J^{PC}=2^{-+}$ glueballs.
  All in $SU(4)$.}
\label{fig_MJ02mpK_cont_SU4}
\end{figure}



\begin{figure}[htb]
\begin	{center}
\leavevmode
% GNUPLOT: LaTeX picture with Postscript
\begingroup%
\makeatletter%
\newcommand{\GNUPLOTspecial}{%
  \@sanitize\catcode`\%=14\relax\special}%
\setlength{\unitlength}{0.0500bp}%
\begin{picture}(7200,7560)(0,0)%
  {\GNUPLOTspecial{"
%!PS-Adobe-2.0 EPSF-2.0
%%Title: plot_MJ1K_cont_SU4.tex
%%Creator: gnuplot 5.0 patchlevel 3
%%CreationDate: Sun Mar  7 12:13:18 2021
%%DocumentFonts: 
%%BoundingBox: 0 0 360 378
%%EndComments
%%BeginProlog
/gnudict 256 dict def
gnudict begin
%
% The following true/false flags may be edited by hand if desired.
% The unit line width and grayscale image gamma correction may also be changed.
%
/Color true def
/Blacktext true def
/Solid false def
/Dashlength 1 def
/Landscape false def
/Level1 false def
/Level3 false def
/Rounded false def
/ClipToBoundingBox false def
/SuppressPDFMark false def
/TransparentPatterns false def
/gnulinewidth 5.000 def
/userlinewidth gnulinewidth def
/Gamma 1.0 def
/BackgroundColor {-1.000 -1.000 -1.000} def
%
/vshift -66 def
/dl1 {
  10.0 Dashlength userlinewidth gnulinewidth div mul mul mul
  Rounded { currentlinewidth 0.75 mul sub dup 0 le { pop 0.01 } if } if
} def
/dl2 {
  10.0 Dashlength userlinewidth gnulinewidth div mul mul mul
  Rounded { currentlinewidth 0.75 mul add } if
} def
/hpt_ 31.5 def
/vpt_ 31.5 def
/hpt hpt_ def
/vpt vpt_ def
/doclip {
  ClipToBoundingBox {
    newpath 0 0 moveto 360 0 lineto 360 378 lineto 0 378 lineto closepath
    clip
  } if
} def
%
% Gnuplot Prolog Version 5.1 (Oct 2015)
%
%/SuppressPDFMark true def
%
/M {moveto} bind def
/L {lineto} bind def
/R {rmoveto} bind def
/V {rlineto} bind def
/N {newpath moveto} bind def
/Z {closepath} bind def
/C {setrgbcolor} bind def
/f {rlineto fill} bind def
/g {setgray} bind def
/Gshow {show} def   % May be redefined later in the file to support UTF-8
/vpt2 vpt 2 mul def
/hpt2 hpt 2 mul def
/Lshow {currentpoint stroke M 0 vshift R 
	Blacktext {gsave 0 setgray textshow grestore} {textshow} ifelse} def
/Rshow {currentpoint stroke M dup stringwidth pop neg vshift R
	Blacktext {gsave 0 setgray textshow grestore} {textshow} ifelse} def
/Cshow {currentpoint stroke M dup stringwidth pop -2 div vshift R 
	Blacktext {gsave 0 setgray textshow grestore} {textshow} ifelse} def
/UP {dup vpt_ mul /vpt exch def hpt_ mul /hpt exch def
  /hpt2 hpt 2 mul def /vpt2 vpt 2 mul def} def
/DL {Color {setrgbcolor Solid {pop []} if 0 setdash}
 {pop pop pop 0 setgray Solid {pop []} if 0 setdash} ifelse} def
/BL {stroke userlinewidth 2 mul setlinewidth
	Rounded {1 setlinejoin 1 setlinecap} if} def
/AL {stroke userlinewidth 2 div setlinewidth
	Rounded {1 setlinejoin 1 setlinecap} if} def
/UL {dup gnulinewidth mul /userlinewidth exch def
	dup 1 lt {pop 1} if 10 mul /udl exch def} def
/PL {stroke userlinewidth setlinewidth
	Rounded {1 setlinejoin 1 setlinecap} if} def
3.8 setmiterlimit
% Classic Line colors (version 5.0)
/LCw {1 1 1} def
/LCb {0 0 0} def
/LCa {0 0 0} def
/LC0 {1 0 0} def
/LC1 {0 1 0} def
/LC2 {0 0 1} def
/LC3 {1 0 1} def
/LC4 {0 1 1} def
/LC5 {1 1 0} def
/LC6 {0 0 0} def
/LC7 {1 0.3 0} def
/LC8 {0.5 0.5 0.5} def
% Default dash patterns (version 5.0)
/LTB {BL [] LCb DL} def
/LTw {PL [] 1 setgray} def
/LTb {PL [] LCb DL} def
/LTa {AL [1 udl mul 2 udl mul] 0 setdash LCa setrgbcolor} def
/LT0 {PL [] LC0 DL} def
/LT1 {PL [2 dl1 3 dl2] LC1 DL} def
/LT2 {PL [1 dl1 1.5 dl2] LC2 DL} def
/LT3 {PL [6 dl1 2 dl2 1 dl1 2 dl2] LC3 DL} def
/LT4 {PL [1 dl1 2 dl2 6 dl1 2 dl2 1 dl1 2 dl2] LC4 DL} def
/LT5 {PL [4 dl1 2 dl2] LC5 DL} def
/LT6 {PL [1.5 dl1 1.5 dl2 1.5 dl1 1.5 dl2 1.5 dl1 6 dl2] LC6 DL} def
/LT7 {PL [3 dl1 3 dl2 1 dl1 3 dl2] LC7 DL} def
/LT8 {PL [2 dl1 2 dl2 2 dl1 6 dl2] LC8 DL} def
/SL {[] 0 setdash} def
/Pnt {stroke [] 0 setdash gsave 1 setlinecap M 0 0 V stroke grestore} def
/Dia {stroke [] 0 setdash 2 copy vpt add M
  hpt neg vpt neg V hpt vpt neg V
  hpt vpt V hpt neg vpt V closepath stroke
  Pnt} def
/Pls {stroke [] 0 setdash vpt sub M 0 vpt2 V
  currentpoint stroke M
  hpt neg vpt neg R hpt2 0 V stroke
 } def
/Box {stroke [] 0 setdash 2 copy exch hpt sub exch vpt add M
  0 vpt2 neg V hpt2 0 V 0 vpt2 V
  hpt2 neg 0 V closepath stroke
  Pnt} def
/Crs {stroke [] 0 setdash exch hpt sub exch vpt add M
  hpt2 vpt2 neg V currentpoint stroke M
  hpt2 neg 0 R hpt2 vpt2 V stroke} def
/TriU {stroke [] 0 setdash 2 copy vpt 1.12 mul add M
  hpt neg vpt -1.62 mul V
  hpt 2 mul 0 V
  hpt neg vpt 1.62 mul V closepath stroke
  Pnt} def
/Star {2 copy Pls Crs} def
/BoxF {stroke [] 0 setdash exch hpt sub exch vpt add M
  0 vpt2 neg V hpt2 0 V 0 vpt2 V
  hpt2 neg 0 V closepath fill} def
/TriUF {stroke [] 0 setdash vpt 1.12 mul add M
  hpt neg vpt -1.62 mul V
  hpt 2 mul 0 V
  hpt neg vpt 1.62 mul V closepath fill} def
/TriD {stroke [] 0 setdash 2 copy vpt 1.12 mul sub M
  hpt neg vpt 1.62 mul V
  hpt 2 mul 0 V
  hpt neg vpt -1.62 mul V closepath stroke
  Pnt} def
/TriDF {stroke [] 0 setdash vpt 1.12 mul sub M
  hpt neg vpt 1.62 mul V
  hpt 2 mul 0 V
  hpt neg vpt -1.62 mul V closepath fill} def
/DiaF {stroke [] 0 setdash vpt add M
  hpt neg vpt neg V hpt vpt neg V
  hpt vpt V hpt neg vpt V closepath fill} def
/Pent {stroke [] 0 setdash 2 copy gsave
  translate 0 hpt M 4 {72 rotate 0 hpt L} repeat
  closepath stroke grestore Pnt} def
/PentF {stroke [] 0 setdash gsave
  translate 0 hpt M 4 {72 rotate 0 hpt L} repeat
  closepath fill grestore} def
/Circle {stroke [] 0 setdash 2 copy
  hpt 0 360 arc stroke Pnt} def
/CircleF {stroke [] 0 setdash hpt 0 360 arc fill} def
/C0 {BL [] 0 setdash 2 copy moveto vpt 90 450 arc} bind def
/C1 {BL [] 0 setdash 2 copy moveto
	2 copy vpt 0 90 arc closepath fill
	vpt 0 360 arc closepath} bind def
/C2 {BL [] 0 setdash 2 copy moveto
	2 copy vpt 90 180 arc closepath fill
	vpt 0 360 arc closepath} bind def
/C3 {BL [] 0 setdash 2 copy moveto
	2 copy vpt 0 180 arc closepath fill
	vpt 0 360 arc closepath} bind def
/C4 {BL [] 0 setdash 2 copy moveto
	2 copy vpt 180 270 arc closepath fill
	vpt 0 360 arc closepath} bind def
/C5 {BL [] 0 setdash 2 copy moveto
	2 copy vpt 0 90 arc
	2 copy moveto
	2 copy vpt 180 270 arc closepath fill
	vpt 0 360 arc} bind def
/C6 {BL [] 0 setdash 2 copy moveto
	2 copy vpt 90 270 arc closepath fill
	vpt 0 360 arc closepath} bind def
/C7 {BL [] 0 setdash 2 copy moveto
	2 copy vpt 0 270 arc closepath fill
	vpt 0 360 arc closepath} bind def
/C8 {BL [] 0 setdash 2 copy moveto
	2 copy vpt 270 360 arc closepath fill
	vpt 0 360 arc closepath} bind def
/C9 {BL [] 0 setdash 2 copy moveto
	2 copy vpt 270 450 arc closepath fill
	vpt 0 360 arc closepath} bind def
/C10 {BL [] 0 setdash 2 copy 2 copy moveto vpt 270 360 arc closepath fill
	2 copy moveto
	2 copy vpt 90 180 arc closepath fill
	vpt 0 360 arc closepath} bind def
/C11 {BL [] 0 setdash 2 copy moveto
	2 copy vpt 0 180 arc closepath fill
	2 copy moveto
	2 copy vpt 270 360 arc closepath fill
	vpt 0 360 arc closepath} bind def
/C12 {BL [] 0 setdash 2 copy moveto
	2 copy vpt 180 360 arc closepath fill
	vpt 0 360 arc closepath} bind def
/C13 {BL [] 0 setdash 2 copy moveto
	2 copy vpt 0 90 arc closepath fill
	2 copy moveto
	2 copy vpt 180 360 arc closepath fill
	vpt 0 360 arc closepath} bind def
/C14 {BL [] 0 setdash 2 copy moveto
	2 copy vpt 90 360 arc closepath fill
	vpt 0 360 arc} bind def
/C15 {BL [] 0 setdash 2 copy vpt 0 360 arc closepath fill
	vpt 0 360 arc closepath} bind def
/Rec {newpath 4 2 roll moveto 1 index 0 rlineto 0 exch rlineto
	neg 0 rlineto closepath} bind def
/Square {dup Rec} bind def
/Bsquare {vpt sub exch vpt sub exch vpt2 Square} bind def
/S0 {BL [] 0 setdash 2 copy moveto 0 vpt rlineto BL Bsquare} bind def
/S1 {BL [] 0 setdash 2 copy vpt Square fill Bsquare} bind def
/S2 {BL [] 0 setdash 2 copy exch vpt sub exch vpt Square fill Bsquare} bind def
/S3 {BL [] 0 setdash 2 copy exch vpt sub exch vpt2 vpt Rec fill Bsquare} bind def
/S4 {BL [] 0 setdash 2 copy exch vpt sub exch vpt sub vpt Square fill Bsquare} bind def
/S5 {BL [] 0 setdash 2 copy 2 copy vpt Square fill
	exch vpt sub exch vpt sub vpt Square fill Bsquare} bind def
/S6 {BL [] 0 setdash 2 copy exch vpt sub exch vpt sub vpt vpt2 Rec fill Bsquare} bind def
/S7 {BL [] 0 setdash 2 copy exch vpt sub exch vpt sub vpt vpt2 Rec fill
	2 copy vpt Square fill Bsquare} bind def
/S8 {BL [] 0 setdash 2 copy vpt sub vpt Square fill Bsquare} bind def
/S9 {BL [] 0 setdash 2 copy vpt sub vpt vpt2 Rec fill Bsquare} bind def
/S10 {BL [] 0 setdash 2 copy vpt sub vpt Square fill 2 copy exch vpt sub exch vpt Square fill
	Bsquare} bind def
/S11 {BL [] 0 setdash 2 copy vpt sub vpt Square fill 2 copy exch vpt sub exch vpt2 vpt Rec fill
	Bsquare} bind def
/S12 {BL [] 0 setdash 2 copy exch vpt sub exch vpt sub vpt2 vpt Rec fill Bsquare} bind def
/S13 {BL [] 0 setdash 2 copy exch vpt sub exch vpt sub vpt2 vpt Rec fill
	2 copy vpt Square fill Bsquare} bind def
/S14 {BL [] 0 setdash 2 copy exch vpt sub exch vpt sub vpt2 vpt Rec fill
	2 copy exch vpt sub exch vpt Square fill Bsquare} bind def
/S15 {BL [] 0 setdash 2 copy Bsquare fill Bsquare} bind def
/D0 {gsave translate 45 rotate 0 0 S0 stroke grestore} bind def
/D1 {gsave translate 45 rotate 0 0 S1 stroke grestore} bind def
/D2 {gsave translate 45 rotate 0 0 S2 stroke grestore} bind def
/D3 {gsave translate 45 rotate 0 0 S3 stroke grestore} bind def
/D4 {gsave translate 45 rotate 0 0 S4 stroke grestore} bind def
/D5 {gsave translate 45 rotate 0 0 S5 stroke grestore} bind def
/D6 {gsave translate 45 rotate 0 0 S6 stroke grestore} bind def
/D7 {gsave translate 45 rotate 0 0 S7 stroke grestore} bind def
/D8 {gsave translate 45 rotate 0 0 S8 stroke grestore} bind def
/D9 {gsave translate 45 rotate 0 0 S9 stroke grestore} bind def
/D10 {gsave translate 45 rotate 0 0 S10 stroke grestore} bind def
/D11 {gsave translate 45 rotate 0 0 S11 stroke grestore} bind def
/D12 {gsave translate 45 rotate 0 0 S12 stroke grestore} bind def
/D13 {gsave translate 45 rotate 0 0 S13 stroke grestore} bind def
/D14 {gsave translate 45 rotate 0 0 S14 stroke grestore} bind def
/D15 {gsave translate 45 rotate 0 0 S15 stroke grestore} bind def
/DiaE {stroke [] 0 setdash vpt add M
  hpt neg vpt neg V hpt vpt neg V
  hpt vpt V hpt neg vpt V closepath stroke} def
/BoxE {stroke [] 0 setdash exch hpt sub exch vpt add M
  0 vpt2 neg V hpt2 0 V 0 vpt2 V
  hpt2 neg 0 V closepath stroke} def
/TriUE {stroke [] 0 setdash vpt 1.12 mul add M
  hpt neg vpt -1.62 mul V
  hpt 2 mul 0 V
  hpt neg vpt 1.62 mul V closepath stroke} def
/TriDE {stroke [] 0 setdash vpt 1.12 mul sub M
  hpt neg vpt 1.62 mul V
  hpt 2 mul 0 V
  hpt neg vpt -1.62 mul V closepath stroke} def
/PentE {stroke [] 0 setdash gsave
  translate 0 hpt M 4 {72 rotate 0 hpt L} repeat
  closepath stroke grestore} def
/CircE {stroke [] 0 setdash 
  hpt 0 360 arc stroke} def
/Opaque {gsave closepath 1 setgray fill grestore 0 setgray closepath} def
/DiaW {stroke [] 0 setdash vpt add M
  hpt neg vpt neg V hpt vpt neg V
  hpt vpt V hpt neg vpt V Opaque stroke} def
/BoxW {stroke [] 0 setdash exch hpt sub exch vpt add M
  0 vpt2 neg V hpt2 0 V 0 vpt2 V
  hpt2 neg 0 V Opaque stroke} def
/TriUW {stroke [] 0 setdash vpt 1.12 mul add M
  hpt neg vpt -1.62 mul V
  hpt 2 mul 0 V
  hpt neg vpt 1.62 mul V Opaque stroke} def
/TriDW {stroke [] 0 setdash vpt 1.12 mul sub M
  hpt neg vpt 1.62 mul V
  hpt 2 mul 0 V
  hpt neg vpt -1.62 mul V Opaque stroke} def
/PentW {stroke [] 0 setdash gsave
  translate 0 hpt M 4 {72 rotate 0 hpt L} repeat
  Opaque stroke grestore} def
/CircW {stroke [] 0 setdash 
  hpt 0 360 arc Opaque stroke} def
/BoxFill {gsave Rec 1 setgray fill grestore} def
/Density {
  /Fillden exch def
  currentrgbcolor
  /ColB exch def /ColG exch def /ColR exch def
  /ColR ColR Fillden mul Fillden sub 1 add def
  /ColG ColG Fillden mul Fillden sub 1 add def
  /ColB ColB Fillden mul Fillden sub 1 add def
  ColR ColG ColB setrgbcolor} def
/BoxColFill {gsave Rec PolyFill} def
/PolyFill {gsave Density fill grestore grestore} def
/h {rlineto rlineto rlineto gsave closepath fill grestore} bind def
%
% PostScript Level 1 Pattern Fill routine for rectangles
% Usage: x y w h s a XX PatternFill
%	x,y = lower left corner of box to be filled
%	w,h = width and height of box
%	  a = angle in degrees between lines and x-axis
%	 XX = 0/1 for no/yes cross-hatch
%
/PatternFill {gsave /PFa [ 9 2 roll ] def
  PFa 0 get PFa 2 get 2 div add PFa 1 get PFa 3 get 2 div add translate
  PFa 2 get -2 div PFa 3 get -2 div PFa 2 get PFa 3 get Rec
  TransparentPatterns {} {gsave 1 setgray fill grestore} ifelse
  clip
  currentlinewidth 0.5 mul setlinewidth
  /PFs PFa 2 get dup mul PFa 3 get dup mul add sqrt def
  0 0 M PFa 5 get rotate PFs -2 div dup translate
  0 1 PFs PFa 4 get div 1 add floor cvi
	{PFa 4 get mul 0 M 0 PFs V} for
  0 PFa 6 get ne {
	0 1 PFs PFa 4 get div 1 add floor cvi
	{PFa 4 get mul 0 2 1 roll M PFs 0 V} for
 } if
  stroke grestore} def
%
/languagelevel where
 {pop languagelevel} {1} ifelse
dup 2 lt
	{/InterpretLevel1 true def
	 /InterpretLevel3 false def}
	{/InterpretLevel1 Level1 def
	 2 gt
	    {/InterpretLevel3 Level3 def}
	    {/InterpretLevel3 false def}
	 ifelse }
 ifelse
%
% PostScript level 2 pattern fill definitions
%
/Level2PatternFill {
/Tile8x8 {/PaintType 2 /PatternType 1 /TilingType 1 /BBox [0 0 8 8] /XStep 8 /YStep 8}
	bind def
/KeepColor {currentrgbcolor [/Pattern /DeviceRGB] setcolorspace} bind def
<< Tile8x8
 /PaintProc {0.5 setlinewidth pop 0 0 M 8 8 L 0 8 M 8 0 L stroke} 
>> matrix makepattern
/Pat1 exch def
<< Tile8x8
 /PaintProc {0.5 setlinewidth pop 0 0 M 8 8 L 0 8 M 8 0 L stroke
	0 4 M 4 8 L 8 4 L 4 0 L 0 4 L stroke}
>> matrix makepattern
/Pat2 exch def
<< Tile8x8
 /PaintProc {0.5 setlinewidth pop 0 0 M 0 8 L
	8 8 L 8 0 L 0 0 L fill}
>> matrix makepattern
/Pat3 exch def
<< Tile8x8
 /PaintProc {0.5 setlinewidth pop -4 8 M 8 -4 L
	0 12 M 12 0 L stroke}
>> matrix makepattern
/Pat4 exch def
<< Tile8x8
 /PaintProc {0.5 setlinewidth pop -4 0 M 8 12 L
	0 -4 M 12 8 L stroke}
>> matrix makepattern
/Pat5 exch def
<< Tile8x8
 /PaintProc {0.5 setlinewidth pop -2 8 M 4 -4 L
	0 12 M 8 -4 L 4 12 M 10 0 L stroke}
>> matrix makepattern
/Pat6 exch def
<< Tile8x8
 /PaintProc {0.5 setlinewidth pop -2 0 M 4 12 L
	0 -4 M 8 12 L 4 -4 M 10 8 L stroke}
>> matrix makepattern
/Pat7 exch def
<< Tile8x8
 /PaintProc {0.5 setlinewidth pop 8 -2 M -4 4 L
	12 0 M -4 8 L 12 4 M 0 10 L stroke}
>> matrix makepattern
/Pat8 exch def
<< Tile8x8
 /PaintProc {0.5 setlinewidth pop 0 -2 M 12 4 L
	-4 0 M 12 8 L -4 4 M 8 10 L stroke}
>> matrix makepattern
/Pat9 exch def
/Pattern1 {PatternBgnd KeepColor Pat1 setpattern} bind def
/Pattern2 {PatternBgnd KeepColor Pat2 setpattern} bind def
/Pattern3 {PatternBgnd KeepColor Pat3 setpattern} bind def
/Pattern4 {PatternBgnd KeepColor Landscape {Pat5} {Pat4} ifelse setpattern} bind def
/Pattern5 {PatternBgnd KeepColor Landscape {Pat4} {Pat5} ifelse setpattern} bind def
/Pattern6 {PatternBgnd KeepColor Landscape {Pat9} {Pat6} ifelse setpattern} bind def
/Pattern7 {PatternBgnd KeepColor Landscape {Pat8} {Pat7} ifelse setpattern} bind def
} def
%
%
%End of PostScript Level 2 code
%
/PatternBgnd {
  TransparentPatterns {} {gsave 1 setgray fill grestore} ifelse
} def
%
% Substitute for Level 2 pattern fill codes with
% grayscale if Level 2 support is not selected.
%
/Level1PatternFill {
/Pattern1 {0.250 Density} bind def
/Pattern2 {0.500 Density} bind def
/Pattern3 {0.750 Density} bind def
/Pattern4 {0.125 Density} bind def
/Pattern5 {0.375 Density} bind def
/Pattern6 {0.625 Density} bind def
/Pattern7 {0.875 Density} bind def
} def
%
% Now test for support of Level 2 code
%
Level1 {Level1PatternFill} {Level2PatternFill} ifelse
%
/Symbol-Oblique /Symbol findfont [1 0 .167 1 0 0] makefont
dup length dict begin {1 index /FID eq {pop pop} {def} ifelse} forall
currentdict end definefont pop
%
Level1 SuppressPDFMark or 
{} {
/SDict 10 dict def
systemdict /pdfmark known not {
  userdict /pdfmark systemdict /cleartomark get put
} if
SDict begin [
  /Title (plot_MJ1K_cont_SU4.tex)
  /Subject (gnuplot plot)
  /Creator (gnuplot 5.0 patchlevel 3)
  /Author (mteper)
%  /Producer (gnuplot)
%  /Keywords ()
  /CreationDate (Sun Mar  7 12:13:18 2021)
  /DOCINFO pdfmark
end
} ifelse
%
% Support for boxed text - Ethan A Merritt May 2005
%
/InitTextBox { userdict /TBy2 3 -1 roll put userdict /TBx2 3 -1 roll put
           userdict /TBy1 3 -1 roll put userdict /TBx1 3 -1 roll put
	   /Boxing true def } def
/ExtendTextBox { Boxing
    { gsave dup false charpath pathbbox
      dup TBy2 gt {userdict /TBy2 3 -1 roll put} {pop} ifelse
      dup TBx2 gt {userdict /TBx2 3 -1 roll put} {pop} ifelse
      dup TBy1 lt {userdict /TBy1 3 -1 roll put} {pop} ifelse
      dup TBx1 lt {userdict /TBx1 3 -1 roll put} {pop} ifelse
      grestore } if } def
/PopTextBox { newpath TBx1 TBxmargin sub TBy1 TBymargin sub M
               TBx1 TBxmargin sub TBy2 TBymargin add L
	       TBx2 TBxmargin add TBy2 TBymargin add L
	       TBx2 TBxmargin add TBy1 TBymargin sub L closepath } def
/DrawTextBox { PopTextBox stroke /Boxing false def} def
/FillTextBox { gsave PopTextBox 1 1 1 setrgbcolor fill grestore /Boxing false def} def
0 0 0 0 InitTextBox
/TBxmargin 20 def
/TBymargin 20 def
/Boxing false def
/textshow { ExtendTextBox Gshow } def
%
% redundant definitions for compatibility with prologue.ps older than 5.0.2
/LTB {BL [] LCb DL} def
/LTb {PL [] LCb DL} def
end
%%EndProlog
%%Page: 1 1
gnudict begin
gsave
doclip
0 0 translate
0.050 0.050 scale
0 setgray
newpath
BackgroundColor 0 lt 3 1 roll 0 lt exch 0 lt or or not {BackgroundColor C 1.000 0 0 7200.00 7560.00 BoxColFill} if
1.000 UL
LTb
LCb setrgbcolor
1220 640 M
63 0 V
5556 0 R
-63 0 V
stroke
LTb
LCb setrgbcolor
1220 1594 M
63 0 V
5556 0 R
-63 0 V
stroke
LTb
LCb setrgbcolor
1220 2548 M
63 0 V
5556 0 R
-63 0 V
stroke
LTb
LCb setrgbcolor
1220 3502 M
63 0 V
5556 0 R
-63 0 V
stroke
LTb
LCb setrgbcolor
1220 4457 M
63 0 V
5556 0 R
-63 0 V
stroke
LTb
LCb setrgbcolor
1220 5411 M
63 0 V
5556 0 R
-63 0 V
stroke
LTb
LCb setrgbcolor
1220 6365 M
63 0 V
5556 0 R
-63 0 V
stroke
LTb
LCb setrgbcolor
1220 7319 M
63 0 V
5556 0 R
-63 0 V
stroke
LTb
LCb setrgbcolor
1220 640 M
0 63 V
0 6616 R
0 -63 V
stroke
LTb
LCb setrgbcolor
1782 640 M
0 63 V
0 6616 R
0 -63 V
stroke
LTb
LCb setrgbcolor
2344 640 M
0 63 V
0 6616 R
0 -63 V
stroke
LTb
LCb setrgbcolor
2906 640 M
0 63 V
0 6616 R
0 -63 V
stroke
LTb
LCb setrgbcolor
3468 640 M
0 63 V
0 6616 R
0 -63 V
stroke
LTb
LCb setrgbcolor
4030 640 M
0 63 V
0 6616 R
0 -63 V
stroke
LTb
LCb setrgbcolor
4591 640 M
0 63 V
0 6616 R
0 -63 V
stroke
LTb
LCb setrgbcolor
5153 640 M
0 63 V
0 6616 R
0 -63 V
stroke
LTb
LCb setrgbcolor
5715 640 M
0 63 V
0 6616 R
0 -63 V
stroke
LTb
LCb setrgbcolor
6277 640 M
0 63 V
0 6616 R
0 -63 V
stroke
LTb
LCb setrgbcolor
6839 640 M
0 63 V
0 6616 R
0 -63 V
stroke
LTb
LCb setrgbcolor
1.000 UL
LTb
LCb setrgbcolor
1220 7319 N
0 -6679 V
5619 0 V
0 6679 V
-5619 0 V
Z stroke
1.000 UP
1.000 UL
LTb
LCb setrgbcolor
LCb setrgbcolor
LTb
LCb setrgbcolor
LTb
1.500 UP
1.000 UL
LTb
0.58 0.00 0.83 C 6348 3231 M
0 103 V
-1495 42 R
0 76 V
-1052 22 R
0 56 V
-730 -79 R
0 61 V
2536 3367 M
0 78 V
-357 -25 R
0 74 V
6348 4871 M
0 291 V
4853 4914 M
0 197 V
3801 4930 M
0 100 V
3071 4742 M
0 297 V
-535 -63 R
0 141 V
2179 4839 M
0 133 V
6348 3282 CircleF
4853 3414 CircleF
3801 3502 CircleF
3071 3482 CircleF
2536 3406 CircleF
2179 3457 CircleF
6348 5016 CircleF
4853 5013 CircleF
3801 4980 CircleF
3071 4891 CircleF
2536 5047 CircleF
2179 4905 CircleF
1.500 UL
LTb
0.58 0.00 0.83 C 1220 3460 M
57 0 V
57 1 V
56 0 V
57 0 V
57 0 V
57 0 V
56 0 V
57 0 V
57 0 V
57 0 V
56 0 V
57 0 V
57 0 V
57 0 V
56 0 V
57 0 V
57 0 V
57 0 V
56 0 V
57 0 V
57 0 V
57 0 V
56 0 V
57 1 V
57 0 V
57 0 V
56 0 V
57 0 V
57 0 V
57 0 V
56 0 V
57 0 V
57 0 V
57 0 V
57 0 V
56 0 V
57 0 V
57 0 V
57 0 V
56 0 V
57 0 V
57 0 V
57 0 V
56 0 V
57 0 V
57 0 V
57 1 V
56 0 V
57 0 V
57 0 V
57 0 V
56 0 V
57 0 V
57 0 V
57 0 V
56 0 V
57 0 V
57 0 V
57 0 V
56 0 V
57 0 V
57 0 V
57 0 V
56 0 V
57 0 V
57 0 V
57 0 V
57 0 V
56 1 V
57 0 V
57 0 V
57 0 V
56 0 V
57 0 V
57 0 V
57 0 V
56 0 V
57 0 V
57 0 V
57 0 V
56 0 V
57 0 V
57 0 V
57 0 V
56 0 V
57 0 V
57 0 V
57 0 V
56 0 V
57 0 V
57 0 V
57 1 V
56 0 V
57 0 V
57 0 V
57 0 V
56 0 V
57 0 V
57 0 V
stroke
LTb
0.58 0.00 0.83 C 1220 4941 M
57 1 V
57 1 V
56 1 V
57 1 V
57 1 V
57 1 V
56 1 V
57 1 V
57 1 V
57 1 V
56 1 V
57 1 V
57 1 V
57 1 V
56 1 V
57 1 V
57 1 V
57 1 V
56 1 V
57 1 V
57 1 V
57 1 V
56 0 V
57 1 V
57 1 V
57 1 V
56 1 V
57 1 V
57 1 V
57 1 V
56 1 V
57 1 V
57 1 V
57 1 V
57 1 V
56 1 V
57 1 V
57 1 V
57 1 V
56 1 V
57 1 V
57 1 V
57 1 V
56 1 V
57 1 V
57 1 V
57 1 V
56 1 V
57 1 V
57 1 V
57 1 V
56 0 V
57 1 V
57 1 V
57 1 V
56 1 V
57 1 V
57 1 V
57 1 V
56 1 V
57 1 V
57 1 V
57 1 V
56 1 V
57 1 V
57 1 V
57 1 V
57 1 V
56 1 V
57 1 V
57 1 V
57 1 V
56 1 V
57 1 V
57 1 V
57 1 V
56 1 V
57 1 V
57 1 V
57 1 V
56 0 V
57 1 V
57 1 V
57 1 V
56 1 V
57 1 V
57 1 V
57 1 V
56 1 V
57 1 V
57 1 V
57 1 V
56 1 V
57 1 V
57 1 V
57 1 V
56 1 V
57 1 V
57 1 V
1.500 UP
stroke
1.000 UL
LTb
0.58 0.00 0.83 C 6348 5787 M
0 468 V
4853 5800 M
0 286 V
3801 5803 M
0 197 V
-730 -20 R
0 114 V
2536 5594 M
0 289 V
2179 5701 M
0 191 V
6348 6021 DiaF
4853 5943 DiaF
3801 5901 DiaF
3071 6037 DiaF
2536 5739 DiaF
2179 5796 DiaF
1.500 UL
LTb
0.58 0.00 0.83 C 1220 5857 M
57 3 V
57 2 V
56 3 V
57 2 V
57 2 V
57 3 V
56 2 V
57 3 V
57 2 V
57 3 V
56 2 V
57 2 V
57 3 V
57 2 V
56 3 V
57 2 V
57 3 V
57 2 V
56 2 V
57 3 V
57 2 V
57 3 V
56 2 V
57 3 V
57 2 V
57 2 V
56 3 V
57 2 V
57 3 V
57 2 V
56 3 V
57 2 V
57 2 V
57 3 V
57 2 V
56 3 V
57 2 V
57 3 V
57 2 V
56 2 V
57 3 V
57 2 V
57 3 V
56 2 V
57 3 V
57 2 V
57 2 V
56 3 V
57 2 V
57 3 V
57 2 V
56 3 V
57 2 V
57 2 V
57 3 V
56 2 V
57 3 V
57 2 V
57 3 V
56 2 V
57 2 V
57 3 V
57 2 V
56 3 V
57 2 V
57 3 V
57 2 V
57 2 V
56 3 V
57 2 V
57 3 V
57 2 V
56 3 V
57 2 V
57 2 V
57 3 V
56 2 V
57 3 V
57 2 V
57 3 V
56 2 V
57 2 V
57 3 V
57 2 V
56 3 V
57 2 V
57 3 V
57 2 V
56 2 V
57 3 V
57 2 V
57 3 V
56 2 V
57 3 V
57 2 V
57 2 V
56 3 V
57 2 V
57 3 V
1.500 UP
stroke
1.000 UL
LTb
0.58 0.00 0.83 C 6348 4776 M
0 335 V
-1495 89 R
0 196 V
3801 5193 M
0 126 V
-730 -93 R
0 391 V
2536 4939 M
0 190 V
-357 -64 R
0 119 V
6348 4944 Dia
4853 5298 Dia
3801 5256 Dia
3071 5422 Dia
2536 5034 Dia
2179 5124 Dia
1.500 UL
LTb
0.58 0.00 0.83 C 1220 5032 M
57 5 V
57 4 V
56 5 V
57 4 V
57 5 V
57 5 V
56 4 V
57 5 V
57 4 V
57 5 V
56 4 V
57 5 V
57 5 V
57 4 V
56 5 V
57 4 V
57 5 V
57 5 V
56 4 V
57 5 V
57 4 V
57 5 V
56 5 V
57 4 V
57 5 V
57 4 V
56 5 V
57 5 V
57 4 V
57 5 V
56 4 V
57 5 V
57 5 V
57 4 V
57 5 V
56 4 V
57 5 V
57 5 V
57 4 V
56 5 V
57 4 V
57 5 V
57 5 V
56 4 V
57 5 V
57 4 V
57 5 V
56 5 V
57 4 V
57 5 V
57 4 V
56 5 V
57 5 V
57 4 V
57 5 V
56 4 V
57 5 V
57 5 V
57 4 V
56 5 V
57 4 V
57 5 V
57 5 V
56 4 V
57 5 V
57 4 V
57 5 V
57 5 V
56 4 V
57 5 V
57 4 V
57 5 V
56 5 V
57 4 V
57 5 V
57 4 V
56 5 V
57 5 V
57 4 V
57 5 V
56 4 V
57 5 V
57 5 V
57 4 V
56 5 V
57 4 V
57 5 V
57 5 V
56 4 V
57 5 V
57 4 V
57 5 V
56 5 V
57 4 V
57 5 V
57 4 V
56 5 V
57 5 V
57 4 V
stroke
2.000 UL
LTb
LCb setrgbcolor
1.000 UL
LTb
LCb setrgbcolor
1220 7319 N
0 -6679 V
5619 0 V
0 6679 V
-5619 0 V
Z stroke
1.000 UP
1.000 UL
LTb
LCb setrgbcolor
stroke
grestore
end
showpage
  }}%
  \put(4029,140){\makebox(0,0){\large{$a^2\sigma$}}}%
  \put(160,4979){\makebox(0,0){\Large{$\frac{M_{J=1}}{\surd\sigma}$}}}%
  \put(6839,440){\makebox(0,0){\strut{}\ {$0.1$}}}%
  \put(6277,440){\makebox(0,0){\strut{}\ {$0.09$}}}%
  \put(5715,440){\makebox(0,0){\strut{}\ {$0.08$}}}%
  \put(5153,440){\makebox(0,0){\strut{}\ {$0.07$}}}%
  \put(4591,440){\makebox(0,0){\strut{}\ {$0.06$}}}%
  \put(4030,440){\makebox(0,0){\strut{}\ {$0.05$}}}%
  \put(3468,440){\makebox(0,0){\strut{}\ {$0.04$}}}%
  \put(2906,440){\makebox(0,0){\strut{}\ {$0.03$}}}%
  \put(2344,440){\makebox(0,0){\strut{}\ {$0.02$}}}%
  \put(1782,440){\makebox(0,0){\strut{}\ {$0.01$}}}%
  \put(1220,440){\makebox(0,0){\strut{}\ {$0$}}}%
  \put(1100,7319){\makebox(0,0)[r]{\strut{}\ \ {$10$}}}%
  \put(1100,6365){\makebox(0,0)[r]{\strut{}\ \ {$9$}}}%
  \put(1100,5411){\makebox(0,0)[r]{\strut{}\ \ {$8$}}}%
  \put(1100,4457){\makebox(0,0)[r]{\strut{}\ \ {$7$}}}%
  \put(1100,3502){\makebox(0,0)[r]{\strut{}\ \ {$6$}}}%
  \put(1100,2548){\makebox(0,0)[r]{\strut{}\ \ {$5$}}}%
  \put(1100,1594){\makebox(0,0)[r]{\strut{}\ \ {$4$}}}%
  \put(1100,640){\makebox(0,0)[r]{\strut{}\ \ {$3$}}}%
\end{picture}%
\endgroup
\endinput

\end	{center}
\caption{Lightest two glueball masses in the $T_1^{+-}$ ($\bullet$) representation
  and the lightest ones in the $T_1^{-+}$ ($\blacklozenge$)
  and $T_1^{--}$ ($\lozenge$) representations, in units of the string tension. Lines are linear
  extrapolations to the continuum limit. In that limit the  $T_1^{+-}$ states become the
  lightest two $J^{PC}=1^{+-}$ glueballs while the other two becomes the
  $1^{-+}$ and $1^{--}$ ground state glueballs.  All in $SU(4)$.}
\label{fig_MJ1K_cont_SU4}
\end{figure}

%\begin{figure}[htb]
%\begin	{center}
%\leavevmode
%\input	{plot_MJ2J3K_cont_SU4.tex}
%\end	{center}
%\caption{Masses of the lightest states in the $E^{--}$ ($\blacksquare$), $T_2^{--}$ ($\square$) and 
%  $E^{+-}$ ($\bullet$) representations and of the second excited state in the $T_2^{+-}$ ($\circ$).
%  Also lightest $A_2^{+-}$ ($\lozenge$) and $T_2^{+-}$ ($\vartriangle$) and first excited
%  $T_1^{+-}$ ($\triangledown$). All in units of the string tension. Lines are linear
%  extrapolations to the continuum limit. In that limit the $E^{--}$ and $T_2^{--}$ states
%  pair up to give the lightest $J^{PC}=2^{--}$ glueball, as do the $E^{+-}$ and second
%  excited  $T_2^{+-}$ to give the lightest $J^{PC}=2^{+-}$ glueball. The other three states
%  (a singlet and two triplets) converge to give the seven components of the lightest
%  $J^{PC}=3^{+-}$ glueball. All in $SU(4)$.}
%\label{fig_MJ2J3K_cont_SU4}
%\end{figure}




%which 4++ fit to use - check! too iffy to include?
\begin{figure}[htb]
\begin	{center}
\leavevmode
% GNUPLOT: LaTeX picture with Postscript
\begingroup%
\makeatletter%
\newcommand{\GNUPLOTspecial}{%
  \@sanitize\catcode`\%=14\relax\special}%
\setlength{\unitlength}{0.0500bp}%
\begin{picture}(7200,7560)(0,0)%
  {\GNUPLOTspecial{"
%!PS-Adobe-2.0 EPSF-2.0
%%Title: plot_M0pp0mpK_N.tex
%%Creator: gnuplot 5.0 patchlevel 3
%%CreationDate: Wed Mar  3 16:51:34 2021
%%DocumentFonts: 
%%BoundingBox: 0 0 360 378
%%EndComments
%%BeginProlog
/gnudict 256 dict def
gnudict begin
%
% The following true/false flags may be edited by hand if desired.
% The unit line width and grayscale image gamma correction may also be changed.
%
/Color true def
/Blacktext true def
/Solid false def
/Dashlength 1 def
/Landscape false def
/Level1 false def
/Level3 false def
/Rounded false def
/ClipToBoundingBox false def
/SuppressPDFMark false def
/TransparentPatterns false def
/gnulinewidth 5.000 def
/userlinewidth gnulinewidth def
/Gamma 1.0 def
/BackgroundColor {-1.000 -1.000 -1.000} def
%
/vshift -66 def
/dl1 {
  10.0 Dashlength userlinewidth gnulinewidth div mul mul mul
  Rounded { currentlinewidth 0.75 mul sub dup 0 le { pop 0.01 } if } if
} def
/dl2 {
  10.0 Dashlength userlinewidth gnulinewidth div mul mul mul
  Rounded { currentlinewidth 0.75 mul add } if
} def
/hpt_ 31.5 def
/vpt_ 31.5 def
/hpt hpt_ def
/vpt vpt_ def
/doclip {
  ClipToBoundingBox {
    newpath 0 0 moveto 360 0 lineto 360 378 lineto 0 378 lineto closepath
    clip
  } if
} def
%
% Gnuplot Prolog Version 5.1 (Oct 2015)
%
%/SuppressPDFMark true def
%
/M {moveto} bind def
/L {lineto} bind def
/R {rmoveto} bind def
/V {rlineto} bind def
/N {newpath moveto} bind def
/Z {closepath} bind def
/C {setrgbcolor} bind def
/f {rlineto fill} bind def
/g {setgray} bind def
/Gshow {show} def   % May be redefined later in the file to support UTF-8
/vpt2 vpt 2 mul def
/hpt2 hpt 2 mul def
/Lshow {currentpoint stroke M 0 vshift R 
	Blacktext {gsave 0 setgray textshow grestore} {textshow} ifelse} def
/Rshow {currentpoint stroke M dup stringwidth pop neg vshift R
	Blacktext {gsave 0 setgray textshow grestore} {textshow} ifelse} def
/Cshow {currentpoint stroke M dup stringwidth pop -2 div vshift R 
	Blacktext {gsave 0 setgray textshow grestore} {textshow} ifelse} def
/UP {dup vpt_ mul /vpt exch def hpt_ mul /hpt exch def
  /hpt2 hpt 2 mul def /vpt2 vpt 2 mul def} def
/DL {Color {setrgbcolor Solid {pop []} if 0 setdash}
 {pop pop pop 0 setgray Solid {pop []} if 0 setdash} ifelse} def
/BL {stroke userlinewidth 2 mul setlinewidth
	Rounded {1 setlinejoin 1 setlinecap} if} def
/AL {stroke userlinewidth 2 div setlinewidth
	Rounded {1 setlinejoin 1 setlinecap} if} def
/UL {dup gnulinewidth mul /userlinewidth exch def
	dup 1 lt {pop 1} if 10 mul /udl exch def} def
/PL {stroke userlinewidth setlinewidth
	Rounded {1 setlinejoin 1 setlinecap} if} def
3.8 setmiterlimit
% Classic Line colors (version 5.0)
/LCw {1 1 1} def
/LCb {0 0 0} def
/LCa {0 0 0} def
/LC0 {1 0 0} def
/LC1 {0 1 0} def
/LC2 {0 0 1} def
/LC3 {1 0 1} def
/LC4 {0 1 1} def
/LC5 {1 1 0} def
/LC6 {0 0 0} def
/LC7 {1 0.3 0} def
/LC8 {0.5 0.5 0.5} def
% Default dash patterns (version 5.0)
/LTB {BL [] LCb DL} def
/LTw {PL [] 1 setgray} def
/LTb {PL [] LCb DL} def
/LTa {AL [1 udl mul 2 udl mul] 0 setdash LCa setrgbcolor} def
/LT0 {PL [] LC0 DL} def
/LT1 {PL [2 dl1 3 dl2] LC1 DL} def
/LT2 {PL [1 dl1 1.5 dl2] LC2 DL} def
/LT3 {PL [6 dl1 2 dl2 1 dl1 2 dl2] LC3 DL} def
/LT4 {PL [1 dl1 2 dl2 6 dl1 2 dl2 1 dl1 2 dl2] LC4 DL} def
/LT5 {PL [4 dl1 2 dl2] LC5 DL} def
/LT6 {PL [1.5 dl1 1.5 dl2 1.5 dl1 1.5 dl2 1.5 dl1 6 dl2] LC6 DL} def
/LT7 {PL [3 dl1 3 dl2 1 dl1 3 dl2] LC7 DL} def
/LT8 {PL [2 dl1 2 dl2 2 dl1 6 dl2] LC8 DL} def
/SL {[] 0 setdash} def
/Pnt {stroke [] 0 setdash gsave 1 setlinecap M 0 0 V stroke grestore} def
/Dia {stroke [] 0 setdash 2 copy vpt add M
  hpt neg vpt neg V hpt vpt neg V
  hpt vpt V hpt neg vpt V closepath stroke
  Pnt} def
/Pls {stroke [] 0 setdash vpt sub M 0 vpt2 V
  currentpoint stroke M
  hpt neg vpt neg R hpt2 0 V stroke
 } def
/Box {stroke [] 0 setdash 2 copy exch hpt sub exch vpt add M
  0 vpt2 neg V hpt2 0 V 0 vpt2 V
  hpt2 neg 0 V closepath stroke
  Pnt} def
/Crs {stroke [] 0 setdash exch hpt sub exch vpt add M
  hpt2 vpt2 neg V currentpoint stroke M
  hpt2 neg 0 R hpt2 vpt2 V stroke} def
/TriU {stroke [] 0 setdash 2 copy vpt 1.12 mul add M
  hpt neg vpt -1.62 mul V
  hpt 2 mul 0 V
  hpt neg vpt 1.62 mul V closepath stroke
  Pnt} def
/Star {2 copy Pls Crs} def
/BoxF {stroke [] 0 setdash exch hpt sub exch vpt add M
  0 vpt2 neg V hpt2 0 V 0 vpt2 V
  hpt2 neg 0 V closepath fill} def
/TriUF {stroke [] 0 setdash vpt 1.12 mul add M
  hpt neg vpt -1.62 mul V
  hpt 2 mul 0 V
  hpt neg vpt 1.62 mul V closepath fill} def
/TriD {stroke [] 0 setdash 2 copy vpt 1.12 mul sub M
  hpt neg vpt 1.62 mul V
  hpt 2 mul 0 V
  hpt neg vpt -1.62 mul V closepath stroke
  Pnt} def
/TriDF {stroke [] 0 setdash vpt 1.12 mul sub M
  hpt neg vpt 1.62 mul V
  hpt 2 mul 0 V
  hpt neg vpt -1.62 mul V closepath fill} def
/DiaF {stroke [] 0 setdash vpt add M
  hpt neg vpt neg V hpt vpt neg V
  hpt vpt V hpt neg vpt V closepath fill} def
/Pent {stroke [] 0 setdash 2 copy gsave
  translate 0 hpt M 4 {72 rotate 0 hpt L} repeat
  closepath stroke grestore Pnt} def
/PentF {stroke [] 0 setdash gsave
  translate 0 hpt M 4 {72 rotate 0 hpt L} repeat
  closepath fill grestore} def
/Circle {stroke [] 0 setdash 2 copy
  hpt 0 360 arc stroke Pnt} def
/CircleF {stroke [] 0 setdash hpt 0 360 arc fill} def
/C0 {BL [] 0 setdash 2 copy moveto vpt 90 450 arc} bind def
/C1 {BL [] 0 setdash 2 copy moveto
	2 copy vpt 0 90 arc closepath fill
	vpt 0 360 arc closepath} bind def
/C2 {BL [] 0 setdash 2 copy moveto
	2 copy vpt 90 180 arc closepath fill
	vpt 0 360 arc closepath} bind def
/C3 {BL [] 0 setdash 2 copy moveto
	2 copy vpt 0 180 arc closepath fill
	vpt 0 360 arc closepath} bind def
/C4 {BL [] 0 setdash 2 copy moveto
	2 copy vpt 180 270 arc closepath fill
	vpt 0 360 arc closepath} bind def
/C5 {BL [] 0 setdash 2 copy moveto
	2 copy vpt 0 90 arc
	2 copy moveto
	2 copy vpt 180 270 arc closepath fill
	vpt 0 360 arc} bind def
/C6 {BL [] 0 setdash 2 copy moveto
	2 copy vpt 90 270 arc closepath fill
	vpt 0 360 arc closepath} bind def
/C7 {BL [] 0 setdash 2 copy moveto
	2 copy vpt 0 270 arc closepath fill
	vpt 0 360 arc closepath} bind def
/C8 {BL [] 0 setdash 2 copy moveto
	2 copy vpt 270 360 arc closepath fill
	vpt 0 360 arc closepath} bind def
/C9 {BL [] 0 setdash 2 copy moveto
	2 copy vpt 270 450 arc closepath fill
	vpt 0 360 arc closepath} bind def
/C10 {BL [] 0 setdash 2 copy 2 copy moveto vpt 270 360 arc closepath fill
	2 copy moveto
	2 copy vpt 90 180 arc closepath fill
	vpt 0 360 arc closepath} bind def
/C11 {BL [] 0 setdash 2 copy moveto
	2 copy vpt 0 180 arc closepath fill
	2 copy moveto
	2 copy vpt 270 360 arc closepath fill
	vpt 0 360 arc closepath} bind def
/C12 {BL [] 0 setdash 2 copy moveto
	2 copy vpt 180 360 arc closepath fill
	vpt 0 360 arc closepath} bind def
/C13 {BL [] 0 setdash 2 copy moveto
	2 copy vpt 0 90 arc closepath fill
	2 copy moveto
	2 copy vpt 180 360 arc closepath fill
	vpt 0 360 arc closepath} bind def
/C14 {BL [] 0 setdash 2 copy moveto
	2 copy vpt 90 360 arc closepath fill
	vpt 0 360 arc} bind def
/C15 {BL [] 0 setdash 2 copy vpt 0 360 arc closepath fill
	vpt 0 360 arc closepath} bind def
/Rec {newpath 4 2 roll moveto 1 index 0 rlineto 0 exch rlineto
	neg 0 rlineto closepath} bind def
/Square {dup Rec} bind def
/Bsquare {vpt sub exch vpt sub exch vpt2 Square} bind def
/S0 {BL [] 0 setdash 2 copy moveto 0 vpt rlineto BL Bsquare} bind def
/S1 {BL [] 0 setdash 2 copy vpt Square fill Bsquare} bind def
/S2 {BL [] 0 setdash 2 copy exch vpt sub exch vpt Square fill Bsquare} bind def
/S3 {BL [] 0 setdash 2 copy exch vpt sub exch vpt2 vpt Rec fill Bsquare} bind def
/S4 {BL [] 0 setdash 2 copy exch vpt sub exch vpt sub vpt Square fill Bsquare} bind def
/S5 {BL [] 0 setdash 2 copy 2 copy vpt Square fill
	exch vpt sub exch vpt sub vpt Square fill Bsquare} bind def
/S6 {BL [] 0 setdash 2 copy exch vpt sub exch vpt sub vpt vpt2 Rec fill Bsquare} bind def
/S7 {BL [] 0 setdash 2 copy exch vpt sub exch vpt sub vpt vpt2 Rec fill
	2 copy vpt Square fill Bsquare} bind def
/S8 {BL [] 0 setdash 2 copy vpt sub vpt Square fill Bsquare} bind def
/S9 {BL [] 0 setdash 2 copy vpt sub vpt vpt2 Rec fill Bsquare} bind def
/S10 {BL [] 0 setdash 2 copy vpt sub vpt Square fill 2 copy exch vpt sub exch vpt Square fill
	Bsquare} bind def
/S11 {BL [] 0 setdash 2 copy vpt sub vpt Square fill 2 copy exch vpt sub exch vpt2 vpt Rec fill
	Bsquare} bind def
/S12 {BL [] 0 setdash 2 copy exch vpt sub exch vpt sub vpt2 vpt Rec fill Bsquare} bind def
/S13 {BL [] 0 setdash 2 copy exch vpt sub exch vpt sub vpt2 vpt Rec fill
	2 copy vpt Square fill Bsquare} bind def
/S14 {BL [] 0 setdash 2 copy exch vpt sub exch vpt sub vpt2 vpt Rec fill
	2 copy exch vpt sub exch vpt Square fill Bsquare} bind def
/S15 {BL [] 0 setdash 2 copy Bsquare fill Bsquare} bind def
/D0 {gsave translate 45 rotate 0 0 S0 stroke grestore} bind def
/D1 {gsave translate 45 rotate 0 0 S1 stroke grestore} bind def
/D2 {gsave translate 45 rotate 0 0 S2 stroke grestore} bind def
/D3 {gsave translate 45 rotate 0 0 S3 stroke grestore} bind def
/D4 {gsave translate 45 rotate 0 0 S4 stroke grestore} bind def
/D5 {gsave translate 45 rotate 0 0 S5 stroke grestore} bind def
/D6 {gsave translate 45 rotate 0 0 S6 stroke grestore} bind def
/D7 {gsave translate 45 rotate 0 0 S7 stroke grestore} bind def
/D8 {gsave translate 45 rotate 0 0 S8 stroke grestore} bind def
/D9 {gsave translate 45 rotate 0 0 S9 stroke grestore} bind def
/D10 {gsave translate 45 rotate 0 0 S10 stroke grestore} bind def
/D11 {gsave translate 45 rotate 0 0 S11 stroke grestore} bind def
/D12 {gsave translate 45 rotate 0 0 S12 stroke grestore} bind def
/D13 {gsave translate 45 rotate 0 0 S13 stroke grestore} bind def
/D14 {gsave translate 45 rotate 0 0 S14 stroke grestore} bind def
/D15 {gsave translate 45 rotate 0 0 S15 stroke grestore} bind def
/DiaE {stroke [] 0 setdash vpt add M
  hpt neg vpt neg V hpt vpt neg V
  hpt vpt V hpt neg vpt V closepath stroke} def
/BoxE {stroke [] 0 setdash exch hpt sub exch vpt add M
  0 vpt2 neg V hpt2 0 V 0 vpt2 V
  hpt2 neg 0 V closepath stroke} def
/TriUE {stroke [] 0 setdash vpt 1.12 mul add M
  hpt neg vpt -1.62 mul V
  hpt 2 mul 0 V
  hpt neg vpt 1.62 mul V closepath stroke} def
/TriDE {stroke [] 0 setdash vpt 1.12 mul sub M
  hpt neg vpt 1.62 mul V
  hpt 2 mul 0 V
  hpt neg vpt -1.62 mul V closepath stroke} def
/PentE {stroke [] 0 setdash gsave
  translate 0 hpt M 4 {72 rotate 0 hpt L} repeat
  closepath stroke grestore} def
/CircE {stroke [] 0 setdash 
  hpt 0 360 arc stroke} def
/Opaque {gsave closepath 1 setgray fill grestore 0 setgray closepath} def
/DiaW {stroke [] 0 setdash vpt add M
  hpt neg vpt neg V hpt vpt neg V
  hpt vpt V hpt neg vpt V Opaque stroke} def
/BoxW {stroke [] 0 setdash exch hpt sub exch vpt add M
  0 vpt2 neg V hpt2 0 V 0 vpt2 V
  hpt2 neg 0 V Opaque stroke} def
/TriUW {stroke [] 0 setdash vpt 1.12 mul add M
  hpt neg vpt -1.62 mul V
  hpt 2 mul 0 V
  hpt neg vpt 1.62 mul V Opaque stroke} def
/TriDW {stroke [] 0 setdash vpt 1.12 mul sub M
  hpt neg vpt 1.62 mul V
  hpt 2 mul 0 V
  hpt neg vpt -1.62 mul V Opaque stroke} def
/PentW {stroke [] 0 setdash gsave
  translate 0 hpt M 4 {72 rotate 0 hpt L} repeat
  Opaque stroke grestore} def
/CircW {stroke [] 0 setdash 
  hpt 0 360 arc Opaque stroke} def
/BoxFill {gsave Rec 1 setgray fill grestore} def
/Density {
  /Fillden exch def
  currentrgbcolor
  /ColB exch def /ColG exch def /ColR exch def
  /ColR ColR Fillden mul Fillden sub 1 add def
  /ColG ColG Fillden mul Fillden sub 1 add def
  /ColB ColB Fillden mul Fillden sub 1 add def
  ColR ColG ColB setrgbcolor} def
/BoxColFill {gsave Rec PolyFill} def
/PolyFill {gsave Density fill grestore grestore} def
/h {rlineto rlineto rlineto gsave closepath fill grestore} bind def
%
% PostScript Level 1 Pattern Fill routine for rectangles
% Usage: x y w h s a XX PatternFill
%	x,y = lower left corner of box to be filled
%	w,h = width and height of box
%	  a = angle in degrees between lines and x-axis
%	 XX = 0/1 for no/yes cross-hatch
%
/PatternFill {gsave /PFa [ 9 2 roll ] def
  PFa 0 get PFa 2 get 2 div add PFa 1 get PFa 3 get 2 div add translate
  PFa 2 get -2 div PFa 3 get -2 div PFa 2 get PFa 3 get Rec
  TransparentPatterns {} {gsave 1 setgray fill grestore} ifelse
  clip
  currentlinewidth 0.5 mul setlinewidth
  /PFs PFa 2 get dup mul PFa 3 get dup mul add sqrt def
  0 0 M PFa 5 get rotate PFs -2 div dup translate
  0 1 PFs PFa 4 get div 1 add floor cvi
	{PFa 4 get mul 0 M 0 PFs V} for
  0 PFa 6 get ne {
	0 1 PFs PFa 4 get div 1 add floor cvi
	{PFa 4 get mul 0 2 1 roll M PFs 0 V} for
 } if
  stroke grestore} def
%
/languagelevel where
 {pop languagelevel} {1} ifelse
dup 2 lt
	{/InterpretLevel1 true def
	 /InterpretLevel3 false def}
	{/InterpretLevel1 Level1 def
	 2 gt
	    {/InterpretLevel3 Level3 def}
	    {/InterpretLevel3 false def}
	 ifelse }
 ifelse
%
% PostScript level 2 pattern fill definitions
%
/Level2PatternFill {
/Tile8x8 {/PaintType 2 /PatternType 1 /TilingType 1 /BBox [0 0 8 8] /XStep 8 /YStep 8}
	bind def
/KeepColor {currentrgbcolor [/Pattern /DeviceRGB] setcolorspace} bind def
<< Tile8x8
 /PaintProc {0.5 setlinewidth pop 0 0 M 8 8 L 0 8 M 8 0 L stroke} 
>> matrix makepattern
/Pat1 exch def
<< Tile8x8
 /PaintProc {0.5 setlinewidth pop 0 0 M 8 8 L 0 8 M 8 0 L stroke
	0 4 M 4 8 L 8 4 L 4 0 L 0 4 L stroke}
>> matrix makepattern
/Pat2 exch def
<< Tile8x8
 /PaintProc {0.5 setlinewidth pop 0 0 M 0 8 L
	8 8 L 8 0 L 0 0 L fill}
>> matrix makepattern
/Pat3 exch def
<< Tile8x8
 /PaintProc {0.5 setlinewidth pop -4 8 M 8 -4 L
	0 12 M 12 0 L stroke}
>> matrix makepattern
/Pat4 exch def
<< Tile8x8
 /PaintProc {0.5 setlinewidth pop -4 0 M 8 12 L
	0 -4 M 12 8 L stroke}
>> matrix makepattern
/Pat5 exch def
<< Tile8x8
 /PaintProc {0.5 setlinewidth pop -2 8 M 4 -4 L
	0 12 M 8 -4 L 4 12 M 10 0 L stroke}
>> matrix makepattern
/Pat6 exch def
<< Tile8x8
 /PaintProc {0.5 setlinewidth pop -2 0 M 4 12 L
	0 -4 M 8 12 L 4 -4 M 10 8 L stroke}
>> matrix makepattern
/Pat7 exch def
<< Tile8x8
 /PaintProc {0.5 setlinewidth pop 8 -2 M -4 4 L
	12 0 M -4 8 L 12 4 M 0 10 L stroke}
>> matrix makepattern
/Pat8 exch def
<< Tile8x8
 /PaintProc {0.5 setlinewidth pop 0 -2 M 12 4 L
	-4 0 M 12 8 L -4 4 M 8 10 L stroke}
>> matrix makepattern
/Pat9 exch def
/Pattern1 {PatternBgnd KeepColor Pat1 setpattern} bind def
/Pattern2 {PatternBgnd KeepColor Pat2 setpattern} bind def
/Pattern3 {PatternBgnd KeepColor Pat3 setpattern} bind def
/Pattern4 {PatternBgnd KeepColor Landscape {Pat5} {Pat4} ifelse setpattern} bind def
/Pattern5 {PatternBgnd KeepColor Landscape {Pat4} {Pat5} ifelse setpattern} bind def
/Pattern6 {PatternBgnd KeepColor Landscape {Pat9} {Pat6} ifelse setpattern} bind def
/Pattern7 {PatternBgnd KeepColor Landscape {Pat8} {Pat7} ifelse setpattern} bind def
} def
%
%
%End of PostScript Level 2 code
%
/PatternBgnd {
  TransparentPatterns {} {gsave 1 setgray fill grestore} ifelse
} def
%
% Substitute for Level 2 pattern fill codes with
% grayscale if Level 2 support is not selected.
%
/Level1PatternFill {
/Pattern1 {0.250 Density} bind def
/Pattern2 {0.500 Density} bind def
/Pattern3 {0.750 Density} bind def
/Pattern4 {0.125 Density} bind def
/Pattern5 {0.375 Density} bind def
/Pattern6 {0.625 Density} bind def
/Pattern7 {0.875 Density} bind def
} def
%
% Now test for support of Level 2 code
%
Level1 {Level1PatternFill} {Level2PatternFill} ifelse
%
/Symbol-Oblique /Symbol findfont [1 0 .167 1 0 0] makefont
dup length dict begin {1 index /FID eq {pop pop} {def} ifelse} forall
currentdict end definefont pop
%
Level1 SuppressPDFMark or 
{} {
/SDict 10 dict def
systemdict /pdfmark known not {
  userdict /pdfmark systemdict /cleartomark get put
} if
SDict begin [
  /Title (plot_M0pp0mpK_N.tex)
  /Subject (gnuplot plot)
  /Creator (gnuplot 5.0 patchlevel 3)
  /Author (mteper)
%  /Producer (gnuplot)
%  /Keywords ()
  /CreationDate (Wed Mar  3 16:51:34 2021)
  /DOCINFO pdfmark
end
} ifelse
%
% Support for boxed text - Ethan A Merritt May 2005
%
/InitTextBox { userdict /TBy2 3 -1 roll put userdict /TBx2 3 -1 roll put
           userdict /TBy1 3 -1 roll put userdict /TBx1 3 -1 roll put
	   /Boxing true def } def
/ExtendTextBox { Boxing
    { gsave dup false charpath pathbbox
      dup TBy2 gt {userdict /TBy2 3 -1 roll put} {pop} ifelse
      dup TBx2 gt {userdict /TBx2 3 -1 roll put} {pop} ifelse
      dup TBy1 lt {userdict /TBy1 3 -1 roll put} {pop} ifelse
      dup TBx1 lt {userdict /TBx1 3 -1 roll put} {pop} ifelse
      grestore } if } def
/PopTextBox { newpath TBx1 TBxmargin sub TBy1 TBymargin sub M
               TBx1 TBxmargin sub TBy2 TBymargin add L
	       TBx2 TBxmargin add TBy2 TBymargin add L
	       TBx2 TBxmargin add TBy1 TBymargin sub L closepath } def
/DrawTextBox { PopTextBox stroke /Boxing false def} def
/FillTextBox { gsave PopTextBox 1 1 1 setrgbcolor fill grestore /Boxing false def} def
0 0 0 0 InitTextBox
/TBxmargin 20 def
/TBymargin 20 def
/Boxing false def
/textshow { ExtendTextBox Gshow } def
%
% redundant definitions for compatibility with prologue.ps older than 5.0.2
/LTB {BL [] LCb DL} def
/LTb {PL [] LCb DL} def
end
%%EndProlog
%%Page: 1 1
gnudict begin
gsave
doclip
0 0 translate
0.050 0.050 scale
0 setgray
newpath
BackgroundColor 0 lt 3 1 roll 0 lt exch 0 lt or or not {BackgroundColor C 1.000 0 0 7200.00 7560.00 BoxColFill} if
1.000 UL
LTb
LCb setrgbcolor
1020 640 M
63 0 V
5756 0 R
-63 0 V
stroke
LTb
LCb setrgbcolor
1020 1308 M
63 0 V
5756 0 R
-63 0 V
stroke
LTb
LCb setrgbcolor
1020 1976 M
63 0 V
5756 0 R
-63 0 V
stroke
LTb
LCb setrgbcolor
1020 2644 M
63 0 V
5756 0 R
-63 0 V
stroke
LTb
LCb setrgbcolor
1020 3312 M
63 0 V
5756 0 R
-63 0 V
stroke
LTb
LCb setrgbcolor
1020 3980 M
63 0 V
5756 0 R
-63 0 V
stroke
LTb
LCb setrgbcolor
1020 4647 M
63 0 V
5756 0 R
-63 0 V
stroke
LTb
LCb setrgbcolor
1020 5315 M
63 0 V
5756 0 R
-63 0 V
stroke
LTb
LCb setrgbcolor
1020 5983 M
63 0 V
5756 0 R
-63 0 V
stroke
LTb
LCb setrgbcolor
1020 6651 M
63 0 V
5756 0 R
-63 0 V
stroke
LTb
LCb setrgbcolor
1020 7319 M
63 0 V
5756 0 R
-63 0 V
stroke
LTb
LCb setrgbcolor
1020 640 M
0 63 V
0 6616 R
0 -63 V
stroke
LTb
LCb setrgbcolor
1990 640 M
0 63 V
0 6616 R
0 -63 V
stroke
LTb
LCb setrgbcolor
2960 640 M
0 63 V
0 6616 R
0 -63 V
stroke
LTb
LCb setrgbcolor
3930 640 M
0 63 V
0 6616 R
0 -63 V
stroke
LTb
LCb setrgbcolor
4899 640 M
0 63 V
0 6616 R
0 -63 V
stroke
LTb
LCb setrgbcolor
5869 640 M
0 63 V
0 6616 R
0 -63 V
stroke
LTb
LCb setrgbcolor
6839 640 M
0 63 V
0 6616 R
0 -63 V
stroke
LTb
LCb setrgbcolor
1.000 UL
LTb
LCb setrgbcolor
1020 7319 N
0 -6679 V
5819 0 V
0 6679 V
-5819 0 V
Z stroke
1.000 UP
1.000 UL
LTb
LCb setrgbcolor
LCb setrgbcolor
LTb
LCb setrgbcolor
LTb
1.500 UP
1.000 UL
LTb
0.58 0.00 0.83 C 5869 3150 M
0 31 V
3175 2900 M
0 28 V
2232 2807 M
0 36 V
1796 2727 M
0 42 V
-237 -79 R
0 43 V
-236 -41 R
0 35 V
-109 -40 R
0 50 V
-59 -14 R
0 44 V
5869 3165 CircleF
3175 2914 CircleF
2232 2825 CircleF
1796 2748 CircleF
1559 2712 CircleF
1323 2710 CircleF
1214 2712 CircleF
1155 2745 CircleF
1.500 UL
LTb
0.58 0.00 0.83 C 1020 2692 M
59 6 V
59 5 V
58 6 V
59 6 V
59 6 V
59 6 V
58 5 V
59 6 V
59 6 V
59 6 V
59 5 V
58 6 V
59 6 V
59 6 V
59 6 V
58 5 V
59 6 V
59 6 V
59 6 V
59 6 V
58 5 V
59 6 V
59 6 V
59 6 V
58 6 V
59 5 V
59 6 V
59 6 V
59 6 V
58 5 V
59 6 V
59 6 V
59 6 V
58 6 V
59 5 V
59 6 V
59 6 V
59 6 V
58 6 V
59 5 V
59 6 V
59 6 V
58 6 V
59 5 V
59 6 V
59 6 V
59 6 V
58 6 V
59 5 V
59 6 V
59 6 V
58 6 V
59 6 V
59 5 V
59 6 V
59 6 V
58 6 V
59 6 V
59 5 V
59 6 V
58 6 V
59 6 V
59 5 V
59 6 V
59 6 V
58 6 V
59 6 V
59 5 V
59 6 V
58 6 V
59 6 V
59 6 V
59 5 V
59 6 V
58 6 V
59 6 V
59 6 V
59 5 V
58 6 V
59 6 V
59 6 V
59 5 V
59 6 V
58 6 V
59 6 V
59 6 V
59 5 V
58 6 V
59 6 V
59 6 V
59 6 V
59 5 V
58 6 V
59 6 V
59 6 V
59 5 V
58 6 V
59 6 V
59 6 V
1.500 UP
stroke
1.000 UL
LTb
0.58 0.00 0.83 C 5869 4706 M
0 51 V
3175 4523 M
0 55 V
-943 -88 R
0 83 V
1796 4404 M
0 71 V
-237 148 R
0 76 V
1323 4514 M
0 93 V
-109 -52 R
0 165 V
-59 -185 R
0 110 V
5869 4732 BoxF
3175 4551 BoxF
2232 4532 BoxF
1796 4440 BoxF
1559 4661 BoxF
1323 4561 BoxF
1214 4637 BoxF
1155 4590 BoxF
1.500 UL
LTb
0.58 0.00 0.83 C 1020 4518 M
59 2 V
59 2 V
58 3 V
59 2 V
59 2 V
59 3 V
58 2 V
59 2 V
59 2 V
59 3 V
59 2 V
58 2 V
59 3 V
59 2 V
59 2 V
58 2 V
59 3 V
59 2 V
59 2 V
59 3 V
58 2 V
59 2 V
59 2 V
59 3 V
58 2 V
59 2 V
59 3 V
59 2 V
59 2 V
58 2 V
59 3 V
59 2 V
59 2 V
58 3 V
59 2 V
59 2 V
59 2 V
59 3 V
58 2 V
59 2 V
59 3 V
59 2 V
58 2 V
59 2 V
59 3 V
59 2 V
59 2 V
58 3 V
59 2 V
59 2 V
59 2 V
58 3 V
59 2 V
59 2 V
59 3 V
59 2 V
58 2 V
59 2 V
59 3 V
59 2 V
58 2 V
59 3 V
59 2 V
59 2 V
59 2 V
58 3 V
59 2 V
59 2 V
59 3 V
58 2 V
59 2 V
59 2 V
59 3 V
59 2 V
58 2 V
59 3 V
59 2 V
59 2 V
58 3 V
59 2 V
59 2 V
59 2 V
59 3 V
58 2 V
59 2 V
59 3 V
59 2 V
58 2 V
59 2 V
59 3 V
59 2 V
59 2 V
58 3 V
59 2 V
59 2 V
59 2 V
58 3 V
59 2 V
59 2 V
1.500 UP
stroke
1.000 UL
LTb
0.58 0.00 0.83 C 5869 4618 M
0 81 V
3175 4134 M
0 60 V
2232 3962 M
0 62 V
1796 3841 M
0 53 V
-237 35 R
0 57 V
1323 3777 M
0 78 V
-109 -26 R
0 80 V
-59 -176 R
0 87 V
5869 4659 Circle
3175 4164 Circle
2232 3993 Circle
1796 3867 Circle
1559 3957 Circle
1323 3816 Circle
1214 3869 Circle
1155 3776 Circle
1.500 UL
LTb
0.58 0.00 0.83 C 1020 3786 M
59 11 V
59 10 V
58 11 V
59 10 V
59 11 V
59 10 V
58 10 V
59 11 V
59 10 V
59 11 V
59 10 V
58 11 V
59 10 V
59 10 V
59 11 V
58 10 V
59 11 V
59 10 V
59 11 V
59 10 V
58 10 V
59 11 V
59 10 V
59 11 V
58 10 V
59 10 V
59 11 V
59 10 V
59 11 V
58 10 V
59 11 V
59 10 V
59 10 V
58 11 V
59 10 V
59 11 V
59 10 V
59 11 V
58 10 V
59 10 V
59 11 V
59 10 V
58 11 V
59 10 V
59 11 V
59 10 V
59 10 V
58 11 V
59 10 V
59 11 V
59 10 V
58 10 V
59 11 V
59 10 V
59 11 V
59 10 V
58 11 V
59 10 V
59 10 V
59 11 V
58 10 V
59 11 V
59 10 V
59 11 V
59 10 V
58 10 V
59 11 V
59 10 V
59 11 V
58 10 V
59 11 V
59 10 V
59 10 V
59 11 V
58 10 V
59 11 V
59 10 V
59 10 V
58 11 V
59 10 V
59 11 V
59 10 V
59 11 V
58 10 V
59 10 V
59 11 V
59 10 V
58 11 V
59 10 V
59 11 V
59 10 V
59 10 V
58 11 V
59 10 V
59 11 V
59 10 V
58 11 V
59 10 V
59 10 V
1.500 UP
stroke
1.000 UL
LTb
0.58 0.00 0.83 C 5869 5883 M
0 200 V
3175 5422 M
0 174 V
2232 5462 M
0 147 V
1796 5496 M
0 147 V
1559 5322 M
0 174 V
1323 5088 M
0 321 V
1214 5242 M
0 187 V
-59 -281 R
0 254 V
5869 5983 Box
3175 5509 Box
2232 5536 Box
1796 5569 Box
1559 5409 Box
1323 5249 Box
1214 5335 Box
1155 5275 Box
1.500 UL
LTb
0.58 0.00 0.83 C 1020 5349 M
59 7 V
59 8 V
58 7 V
59 7 V
59 8 V
59 7 V
58 8 V
59 7 V
59 8 V
59 7 V
59 7 V
58 8 V
59 7 V
59 8 V
59 7 V
58 8 V
59 7 V
59 7 V
59 8 V
59 7 V
58 8 V
59 7 V
59 8 V
59 7 V
58 7 V
59 8 V
59 7 V
59 8 V
59 7 V
58 8 V
59 7 V
59 7 V
59 8 V
58 7 V
59 8 V
59 7 V
59 8 V
59 7 V
58 7 V
59 8 V
59 7 V
59 8 V
58 7 V
59 8 V
59 7 V
59 7 V
59 8 V
58 7 V
59 8 V
59 7 V
59 8 V
58 7 V
59 7 V
59 8 V
59 7 V
59 8 V
58 7 V
59 8 V
59 7 V
59 7 V
58 8 V
59 7 V
59 8 V
59 7 V
59 8 V
58 7 V
59 7 V
59 8 V
59 7 V
58 8 V
59 7 V
59 8 V
59 7 V
59 7 V
58 8 V
59 7 V
59 8 V
59 7 V
58 7 V
59 8 V
59 7 V
59 8 V
59 7 V
58 8 V
59 7 V
59 7 V
59 8 V
58 7 V
59 8 V
59 7 V
59 8 V
59 7 V
58 7 V
59 8 V
59 7 V
59 8 V
58 7 V
59 8 V
59 7 V
1.500 UP
stroke
1.000 UL
LTb
0.58 0.00 0.83 C 5869 5611 M
0 134 V
3175 5496 M
0 147 V
-943 -67 R
0 147 V
1796 5462 M
0 187 V
-237 -73 R
0 160 V
1323 5349 M
0 173 V
1214 5302 M
0 227 V
-59 -287 R
0 240 V
5869 5678 DiaF
3175 5569 DiaF
2232 5649 DiaF
1796 5556 DiaF
1559 5656 DiaF
1323 5436 DiaF
1214 5415 DiaF
1155 5362 DiaF
1.500 UL
LTb
0.58 0.00 0.83 C 1020 5405 M
59 13 V
59 13 V
58 13 V
59 14 V
59 13 V
59 13 V
58 13 V
59 13 V
59 14 V
59 13 V
59 13 V
58 13 V
59 13 V
59 14 V
59 13 V
58 13 V
59 13 V
59 13 V
59 14 V
59 13 V
58 13 V
59 13 V
59 13 V
59 14 V
58 13 V
59 13 V
59 13 V
59 13 V
59 13 V
58 14 V
59 13 V
59 13 V
59 13 V
58 13 V
59 14 V
59 13 V
59 13 V
59 13 V
58 13 V
59 14 V
59 13 V
59 13 V
58 13 V
59 13 V
59 14 V
59 13 V
59 13 V
58 13 V
59 13 V
59 14 V
59 13 V
58 13 V
59 13 V
59 13 V
59 14 V
59 13 V
58 13 V
59 13 V
59 13 V
59 14 V
58 13 V
59 13 V
59 13 V
59 13 V
59 14 V
58 13 V
59 13 V
59 13 V
59 13 V
58 14 V
59 13 V
59 13 V
59 13 V
59 13 V
58 14 V
59 13 V
59 13 V
59 13 V
58 13 V
59 13 V
59 14 V
59 13 V
59 13 V
58 13 V
59 13 V
59 14 V
59 13 V
58 13 V
59 13 V
59 13 V
59 14 V
59 13 V
58 13 V
59 13 V
59 13 V
59 14 V
58 13 V
59 13 V
59 13 V
stroke
2.000 UL
LTb
LCb setrgbcolor
1.000 UL
LTb
LCb setrgbcolor
1020 7319 N
0 -6679 V
5819 0 V
0 6679 V
-5819 0 V
Z stroke
1.000 UP
1.000 UL
LTb
LCb setrgbcolor
stroke
grestore
end
showpage
  }}%
  \put(3929,140){\makebox(0,0){\large{$1/N^2$}}}%
  \put(200,4979){\makebox(0,0){\Large{$\frac{M_{J=0}}{\surd\sigma}$}}}%
  \put(6839,440){\makebox(0,0){\strut{}\ {$0.3$}}}%
  \put(5869,440){\makebox(0,0){\strut{}\ {$0.25$}}}%
  \put(4899,440){\makebox(0,0){\strut{}\ {$0.2$}}}%
  \put(3930,440){\makebox(0,0){\strut{}\ {$0.15$}}}%
  \put(2960,440){\makebox(0,0){\strut{}\ {$0.1$}}}%
  \put(1990,440){\makebox(0,0){\strut{}\ {$0.05$}}}%
  \put(1020,440){\makebox(0,0){\strut{}\ {$0$}}}%
  \put(900,7319){\makebox(0,0)[r]{\strut{}\ \ {$10$}}}%
  \put(900,6651){\makebox(0,0)[r]{\strut{}\ \ {$9$}}}%
  \put(900,5983){\makebox(0,0)[r]{\strut{}\ \ {$8$}}}%
  \put(900,5315){\makebox(0,0)[r]{\strut{}\ \ {$7$}}}%
  \put(900,4647){\makebox(0,0)[r]{\strut{}\ \ {$6$}}}%
  \put(900,3980){\makebox(0,0)[r]{\strut{}\ \ {$5$}}}%
  \put(900,3312){\makebox(0,0)[r]{\strut{}\ \ {$4$}}}%
  \put(900,2644){\makebox(0,0)[r]{\strut{}\ \ {$3$}}}%
  \put(900,1976){\makebox(0,0)[r]{\strut{}\ \ {$2$}}}%
  \put(900,1308){\makebox(0,0)[r]{\strut{}\ \ {$1$}}}%
  \put(900,640){\makebox(0,0)[r]{\strut{}\ \ {$0$}}}%
\end{picture}%
\endgroup
\endinput

\end	{center}
\caption{Continuum masses of the lightest ($\bullet$) and first excited ($\blacksquare$)
  $J^{PC}=0^{++}$ scalars and of the lightest ($\circ$) and first excited ($\square$)
  $0^{-+}$ pseudoscalars, in units of the string tension. The state denoted by
  $\blacklozenge$ is either the $4^{++}$ ground state or the second excited $0^{++}$.
  With extrapolations from values in the range $N\in[2,12]$ to $N=\infty$.}
\label{fig_M0pp0mpK_N}
\end{figure}


\begin{figure}[htb]
\begin	{center}
\leavevmode
% GNUPLOT: LaTeX picture with Postscript
\begingroup%
\makeatletter%
\newcommand{\GNUPLOTspecial}{%
  \@sanitize\catcode`\%=14\relax\special}%
\setlength{\unitlength}{0.0500bp}%
\begin{picture}(7200,7560)(0,0)%
  {\GNUPLOTspecial{"
%!PS-Adobe-2.0 EPSF-2.0
%%Title: plot_MJ2PCK_N.tex
%%Creator: gnuplot 5.0 patchlevel 3
%%CreationDate: Thu Mar  4 11:54:35 2021
%%DocumentFonts: 
%%BoundingBox: 0 0 360 378
%%EndComments
%%BeginProlog
/gnudict 256 dict def
gnudict begin
%
% The following true/false flags may be edited by hand if desired.
% The unit line width and grayscale image gamma correction may also be changed.
%
/Color true def
/Blacktext true def
/Solid false def
/Dashlength 1 def
/Landscape false def
/Level1 false def
/Level3 false def
/Rounded false def
/ClipToBoundingBox false def
/SuppressPDFMark false def
/TransparentPatterns false def
/gnulinewidth 5.000 def
/userlinewidth gnulinewidth def
/Gamma 1.0 def
/BackgroundColor {-1.000 -1.000 -1.000} def
%
/vshift -66 def
/dl1 {
  10.0 Dashlength userlinewidth gnulinewidth div mul mul mul
  Rounded { currentlinewidth 0.75 mul sub dup 0 le { pop 0.01 } if } if
} def
/dl2 {
  10.0 Dashlength userlinewidth gnulinewidth div mul mul mul
  Rounded { currentlinewidth 0.75 mul add } if
} def
/hpt_ 31.5 def
/vpt_ 31.5 def
/hpt hpt_ def
/vpt vpt_ def
/doclip {
  ClipToBoundingBox {
    newpath 0 0 moveto 360 0 lineto 360 378 lineto 0 378 lineto closepath
    clip
  } if
} def
%
% Gnuplot Prolog Version 5.1 (Oct 2015)
%
%/SuppressPDFMark true def
%
/M {moveto} bind def
/L {lineto} bind def
/R {rmoveto} bind def
/V {rlineto} bind def
/N {newpath moveto} bind def
/Z {closepath} bind def
/C {setrgbcolor} bind def
/f {rlineto fill} bind def
/g {setgray} bind def
/Gshow {show} def   % May be redefined later in the file to support UTF-8
/vpt2 vpt 2 mul def
/hpt2 hpt 2 mul def
/Lshow {currentpoint stroke M 0 vshift R 
	Blacktext {gsave 0 setgray textshow grestore} {textshow} ifelse} def
/Rshow {currentpoint stroke M dup stringwidth pop neg vshift R
	Blacktext {gsave 0 setgray textshow grestore} {textshow} ifelse} def
/Cshow {currentpoint stroke M dup stringwidth pop -2 div vshift R 
	Blacktext {gsave 0 setgray textshow grestore} {textshow} ifelse} def
/UP {dup vpt_ mul /vpt exch def hpt_ mul /hpt exch def
  /hpt2 hpt 2 mul def /vpt2 vpt 2 mul def} def
/DL {Color {setrgbcolor Solid {pop []} if 0 setdash}
 {pop pop pop 0 setgray Solid {pop []} if 0 setdash} ifelse} def
/BL {stroke userlinewidth 2 mul setlinewidth
	Rounded {1 setlinejoin 1 setlinecap} if} def
/AL {stroke userlinewidth 2 div setlinewidth
	Rounded {1 setlinejoin 1 setlinecap} if} def
/UL {dup gnulinewidth mul /userlinewidth exch def
	dup 1 lt {pop 1} if 10 mul /udl exch def} def
/PL {stroke userlinewidth setlinewidth
	Rounded {1 setlinejoin 1 setlinecap} if} def
3.8 setmiterlimit
% Classic Line colors (version 5.0)
/LCw {1 1 1} def
/LCb {0 0 0} def
/LCa {0 0 0} def
/LC0 {1 0 0} def
/LC1 {0 1 0} def
/LC2 {0 0 1} def
/LC3 {1 0 1} def
/LC4 {0 1 1} def
/LC5 {1 1 0} def
/LC6 {0 0 0} def
/LC7 {1 0.3 0} def
/LC8 {0.5 0.5 0.5} def
% Default dash patterns (version 5.0)
/LTB {BL [] LCb DL} def
/LTw {PL [] 1 setgray} def
/LTb {PL [] LCb DL} def
/LTa {AL [1 udl mul 2 udl mul] 0 setdash LCa setrgbcolor} def
/LT0 {PL [] LC0 DL} def
/LT1 {PL [2 dl1 3 dl2] LC1 DL} def
/LT2 {PL [1 dl1 1.5 dl2] LC2 DL} def
/LT3 {PL [6 dl1 2 dl2 1 dl1 2 dl2] LC3 DL} def
/LT4 {PL [1 dl1 2 dl2 6 dl1 2 dl2 1 dl1 2 dl2] LC4 DL} def
/LT5 {PL [4 dl1 2 dl2] LC5 DL} def
/LT6 {PL [1.5 dl1 1.5 dl2 1.5 dl1 1.5 dl2 1.5 dl1 6 dl2] LC6 DL} def
/LT7 {PL [3 dl1 3 dl2 1 dl1 3 dl2] LC7 DL} def
/LT8 {PL [2 dl1 2 dl2 2 dl1 6 dl2] LC8 DL} def
/SL {[] 0 setdash} def
/Pnt {stroke [] 0 setdash gsave 1 setlinecap M 0 0 V stroke grestore} def
/Dia {stroke [] 0 setdash 2 copy vpt add M
  hpt neg vpt neg V hpt vpt neg V
  hpt vpt V hpt neg vpt V closepath stroke
  Pnt} def
/Pls {stroke [] 0 setdash vpt sub M 0 vpt2 V
  currentpoint stroke M
  hpt neg vpt neg R hpt2 0 V stroke
 } def
/Box {stroke [] 0 setdash 2 copy exch hpt sub exch vpt add M
  0 vpt2 neg V hpt2 0 V 0 vpt2 V
  hpt2 neg 0 V closepath stroke
  Pnt} def
/Crs {stroke [] 0 setdash exch hpt sub exch vpt add M
  hpt2 vpt2 neg V currentpoint stroke M
  hpt2 neg 0 R hpt2 vpt2 V stroke} def
/TriU {stroke [] 0 setdash 2 copy vpt 1.12 mul add M
  hpt neg vpt -1.62 mul V
  hpt 2 mul 0 V
  hpt neg vpt 1.62 mul V closepath stroke
  Pnt} def
/Star {2 copy Pls Crs} def
/BoxF {stroke [] 0 setdash exch hpt sub exch vpt add M
  0 vpt2 neg V hpt2 0 V 0 vpt2 V
  hpt2 neg 0 V closepath fill} def
/TriUF {stroke [] 0 setdash vpt 1.12 mul add M
  hpt neg vpt -1.62 mul V
  hpt 2 mul 0 V
  hpt neg vpt 1.62 mul V closepath fill} def
/TriD {stroke [] 0 setdash 2 copy vpt 1.12 mul sub M
  hpt neg vpt 1.62 mul V
  hpt 2 mul 0 V
  hpt neg vpt -1.62 mul V closepath stroke
  Pnt} def
/TriDF {stroke [] 0 setdash vpt 1.12 mul sub M
  hpt neg vpt 1.62 mul V
  hpt 2 mul 0 V
  hpt neg vpt -1.62 mul V closepath fill} def
/DiaF {stroke [] 0 setdash vpt add M
  hpt neg vpt neg V hpt vpt neg V
  hpt vpt V hpt neg vpt V closepath fill} def
/Pent {stroke [] 0 setdash 2 copy gsave
  translate 0 hpt M 4 {72 rotate 0 hpt L} repeat
  closepath stroke grestore Pnt} def
/PentF {stroke [] 0 setdash gsave
  translate 0 hpt M 4 {72 rotate 0 hpt L} repeat
  closepath fill grestore} def
/Circle {stroke [] 0 setdash 2 copy
  hpt 0 360 arc stroke Pnt} def
/CircleF {stroke [] 0 setdash hpt 0 360 arc fill} def
/C0 {BL [] 0 setdash 2 copy moveto vpt 90 450 arc} bind def
/C1 {BL [] 0 setdash 2 copy moveto
	2 copy vpt 0 90 arc closepath fill
	vpt 0 360 arc closepath} bind def
/C2 {BL [] 0 setdash 2 copy moveto
	2 copy vpt 90 180 arc closepath fill
	vpt 0 360 arc closepath} bind def
/C3 {BL [] 0 setdash 2 copy moveto
	2 copy vpt 0 180 arc closepath fill
	vpt 0 360 arc closepath} bind def
/C4 {BL [] 0 setdash 2 copy moveto
	2 copy vpt 180 270 arc closepath fill
	vpt 0 360 arc closepath} bind def
/C5 {BL [] 0 setdash 2 copy moveto
	2 copy vpt 0 90 arc
	2 copy moveto
	2 copy vpt 180 270 arc closepath fill
	vpt 0 360 arc} bind def
/C6 {BL [] 0 setdash 2 copy moveto
	2 copy vpt 90 270 arc closepath fill
	vpt 0 360 arc closepath} bind def
/C7 {BL [] 0 setdash 2 copy moveto
	2 copy vpt 0 270 arc closepath fill
	vpt 0 360 arc closepath} bind def
/C8 {BL [] 0 setdash 2 copy moveto
	2 copy vpt 270 360 arc closepath fill
	vpt 0 360 arc closepath} bind def
/C9 {BL [] 0 setdash 2 copy moveto
	2 copy vpt 270 450 arc closepath fill
	vpt 0 360 arc closepath} bind def
/C10 {BL [] 0 setdash 2 copy 2 copy moveto vpt 270 360 arc closepath fill
	2 copy moveto
	2 copy vpt 90 180 arc closepath fill
	vpt 0 360 arc closepath} bind def
/C11 {BL [] 0 setdash 2 copy moveto
	2 copy vpt 0 180 arc closepath fill
	2 copy moveto
	2 copy vpt 270 360 arc closepath fill
	vpt 0 360 arc closepath} bind def
/C12 {BL [] 0 setdash 2 copy moveto
	2 copy vpt 180 360 arc closepath fill
	vpt 0 360 arc closepath} bind def
/C13 {BL [] 0 setdash 2 copy moveto
	2 copy vpt 0 90 arc closepath fill
	2 copy moveto
	2 copy vpt 180 360 arc closepath fill
	vpt 0 360 arc closepath} bind def
/C14 {BL [] 0 setdash 2 copy moveto
	2 copy vpt 90 360 arc closepath fill
	vpt 0 360 arc} bind def
/C15 {BL [] 0 setdash 2 copy vpt 0 360 arc closepath fill
	vpt 0 360 arc closepath} bind def
/Rec {newpath 4 2 roll moveto 1 index 0 rlineto 0 exch rlineto
	neg 0 rlineto closepath} bind def
/Square {dup Rec} bind def
/Bsquare {vpt sub exch vpt sub exch vpt2 Square} bind def
/S0 {BL [] 0 setdash 2 copy moveto 0 vpt rlineto BL Bsquare} bind def
/S1 {BL [] 0 setdash 2 copy vpt Square fill Bsquare} bind def
/S2 {BL [] 0 setdash 2 copy exch vpt sub exch vpt Square fill Bsquare} bind def
/S3 {BL [] 0 setdash 2 copy exch vpt sub exch vpt2 vpt Rec fill Bsquare} bind def
/S4 {BL [] 0 setdash 2 copy exch vpt sub exch vpt sub vpt Square fill Bsquare} bind def
/S5 {BL [] 0 setdash 2 copy 2 copy vpt Square fill
	exch vpt sub exch vpt sub vpt Square fill Bsquare} bind def
/S6 {BL [] 0 setdash 2 copy exch vpt sub exch vpt sub vpt vpt2 Rec fill Bsquare} bind def
/S7 {BL [] 0 setdash 2 copy exch vpt sub exch vpt sub vpt vpt2 Rec fill
	2 copy vpt Square fill Bsquare} bind def
/S8 {BL [] 0 setdash 2 copy vpt sub vpt Square fill Bsquare} bind def
/S9 {BL [] 0 setdash 2 copy vpt sub vpt vpt2 Rec fill Bsquare} bind def
/S10 {BL [] 0 setdash 2 copy vpt sub vpt Square fill 2 copy exch vpt sub exch vpt Square fill
	Bsquare} bind def
/S11 {BL [] 0 setdash 2 copy vpt sub vpt Square fill 2 copy exch vpt sub exch vpt2 vpt Rec fill
	Bsquare} bind def
/S12 {BL [] 0 setdash 2 copy exch vpt sub exch vpt sub vpt2 vpt Rec fill Bsquare} bind def
/S13 {BL [] 0 setdash 2 copy exch vpt sub exch vpt sub vpt2 vpt Rec fill
	2 copy vpt Square fill Bsquare} bind def
/S14 {BL [] 0 setdash 2 copy exch vpt sub exch vpt sub vpt2 vpt Rec fill
	2 copy exch vpt sub exch vpt Square fill Bsquare} bind def
/S15 {BL [] 0 setdash 2 copy Bsquare fill Bsquare} bind def
/D0 {gsave translate 45 rotate 0 0 S0 stroke grestore} bind def
/D1 {gsave translate 45 rotate 0 0 S1 stroke grestore} bind def
/D2 {gsave translate 45 rotate 0 0 S2 stroke grestore} bind def
/D3 {gsave translate 45 rotate 0 0 S3 stroke grestore} bind def
/D4 {gsave translate 45 rotate 0 0 S4 stroke grestore} bind def
/D5 {gsave translate 45 rotate 0 0 S5 stroke grestore} bind def
/D6 {gsave translate 45 rotate 0 0 S6 stroke grestore} bind def
/D7 {gsave translate 45 rotate 0 0 S7 stroke grestore} bind def
/D8 {gsave translate 45 rotate 0 0 S8 stroke grestore} bind def
/D9 {gsave translate 45 rotate 0 0 S9 stroke grestore} bind def
/D10 {gsave translate 45 rotate 0 0 S10 stroke grestore} bind def
/D11 {gsave translate 45 rotate 0 0 S11 stroke grestore} bind def
/D12 {gsave translate 45 rotate 0 0 S12 stroke grestore} bind def
/D13 {gsave translate 45 rotate 0 0 S13 stroke grestore} bind def
/D14 {gsave translate 45 rotate 0 0 S14 stroke grestore} bind def
/D15 {gsave translate 45 rotate 0 0 S15 stroke grestore} bind def
/DiaE {stroke [] 0 setdash vpt add M
  hpt neg vpt neg V hpt vpt neg V
  hpt vpt V hpt neg vpt V closepath stroke} def
/BoxE {stroke [] 0 setdash exch hpt sub exch vpt add M
  0 vpt2 neg V hpt2 0 V 0 vpt2 V
  hpt2 neg 0 V closepath stroke} def
/TriUE {stroke [] 0 setdash vpt 1.12 mul add M
  hpt neg vpt -1.62 mul V
  hpt 2 mul 0 V
  hpt neg vpt 1.62 mul V closepath stroke} def
/TriDE {stroke [] 0 setdash vpt 1.12 mul sub M
  hpt neg vpt 1.62 mul V
  hpt 2 mul 0 V
  hpt neg vpt -1.62 mul V closepath stroke} def
/PentE {stroke [] 0 setdash gsave
  translate 0 hpt M 4 {72 rotate 0 hpt L} repeat
  closepath stroke grestore} def
/CircE {stroke [] 0 setdash 
  hpt 0 360 arc stroke} def
/Opaque {gsave closepath 1 setgray fill grestore 0 setgray closepath} def
/DiaW {stroke [] 0 setdash vpt add M
  hpt neg vpt neg V hpt vpt neg V
  hpt vpt V hpt neg vpt V Opaque stroke} def
/BoxW {stroke [] 0 setdash exch hpt sub exch vpt add M
  0 vpt2 neg V hpt2 0 V 0 vpt2 V
  hpt2 neg 0 V Opaque stroke} def
/TriUW {stroke [] 0 setdash vpt 1.12 mul add M
  hpt neg vpt -1.62 mul V
  hpt 2 mul 0 V
  hpt neg vpt 1.62 mul V Opaque stroke} def
/TriDW {stroke [] 0 setdash vpt 1.12 mul sub M
  hpt neg vpt 1.62 mul V
  hpt 2 mul 0 V
  hpt neg vpt -1.62 mul V Opaque stroke} def
/PentW {stroke [] 0 setdash gsave
  translate 0 hpt M 4 {72 rotate 0 hpt L} repeat
  Opaque stroke grestore} def
/CircW {stroke [] 0 setdash 
  hpt 0 360 arc Opaque stroke} def
/BoxFill {gsave Rec 1 setgray fill grestore} def
/Density {
  /Fillden exch def
  currentrgbcolor
  /ColB exch def /ColG exch def /ColR exch def
  /ColR ColR Fillden mul Fillden sub 1 add def
  /ColG ColG Fillden mul Fillden sub 1 add def
  /ColB ColB Fillden mul Fillden sub 1 add def
  ColR ColG ColB setrgbcolor} def
/BoxColFill {gsave Rec PolyFill} def
/PolyFill {gsave Density fill grestore grestore} def
/h {rlineto rlineto rlineto gsave closepath fill grestore} bind def
%
% PostScript Level 1 Pattern Fill routine for rectangles
% Usage: x y w h s a XX PatternFill
%	x,y = lower left corner of box to be filled
%	w,h = width and height of box
%	  a = angle in degrees between lines and x-axis
%	 XX = 0/1 for no/yes cross-hatch
%
/PatternFill {gsave /PFa [ 9 2 roll ] def
  PFa 0 get PFa 2 get 2 div add PFa 1 get PFa 3 get 2 div add translate
  PFa 2 get -2 div PFa 3 get -2 div PFa 2 get PFa 3 get Rec
  TransparentPatterns {} {gsave 1 setgray fill grestore} ifelse
  clip
  currentlinewidth 0.5 mul setlinewidth
  /PFs PFa 2 get dup mul PFa 3 get dup mul add sqrt def
  0 0 M PFa 5 get rotate PFs -2 div dup translate
  0 1 PFs PFa 4 get div 1 add floor cvi
	{PFa 4 get mul 0 M 0 PFs V} for
  0 PFa 6 get ne {
	0 1 PFs PFa 4 get div 1 add floor cvi
	{PFa 4 get mul 0 2 1 roll M PFs 0 V} for
 } if
  stroke grestore} def
%
/languagelevel where
 {pop languagelevel} {1} ifelse
dup 2 lt
	{/InterpretLevel1 true def
	 /InterpretLevel3 false def}
	{/InterpretLevel1 Level1 def
	 2 gt
	    {/InterpretLevel3 Level3 def}
	    {/InterpretLevel3 false def}
	 ifelse }
 ifelse
%
% PostScript level 2 pattern fill definitions
%
/Level2PatternFill {
/Tile8x8 {/PaintType 2 /PatternType 1 /TilingType 1 /BBox [0 0 8 8] /XStep 8 /YStep 8}
	bind def
/KeepColor {currentrgbcolor [/Pattern /DeviceRGB] setcolorspace} bind def
<< Tile8x8
 /PaintProc {0.5 setlinewidth pop 0 0 M 8 8 L 0 8 M 8 0 L stroke} 
>> matrix makepattern
/Pat1 exch def
<< Tile8x8
 /PaintProc {0.5 setlinewidth pop 0 0 M 8 8 L 0 8 M 8 0 L stroke
	0 4 M 4 8 L 8 4 L 4 0 L 0 4 L stroke}
>> matrix makepattern
/Pat2 exch def
<< Tile8x8
 /PaintProc {0.5 setlinewidth pop 0 0 M 0 8 L
	8 8 L 8 0 L 0 0 L fill}
>> matrix makepattern
/Pat3 exch def
<< Tile8x8
 /PaintProc {0.5 setlinewidth pop -4 8 M 8 -4 L
	0 12 M 12 0 L stroke}
>> matrix makepattern
/Pat4 exch def
<< Tile8x8
 /PaintProc {0.5 setlinewidth pop -4 0 M 8 12 L
	0 -4 M 12 8 L stroke}
>> matrix makepattern
/Pat5 exch def
<< Tile8x8
 /PaintProc {0.5 setlinewidth pop -2 8 M 4 -4 L
	0 12 M 8 -4 L 4 12 M 10 0 L stroke}
>> matrix makepattern
/Pat6 exch def
<< Tile8x8
 /PaintProc {0.5 setlinewidth pop -2 0 M 4 12 L
	0 -4 M 8 12 L 4 -4 M 10 8 L stroke}
>> matrix makepattern
/Pat7 exch def
<< Tile8x8
 /PaintProc {0.5 setlinewidth pop 8 -2 M -4 4 L
	12 0 M -4 8 L 12 4 M 0 10 L stroke}
>> matrix makepattern
/Pat8 exch def
<< Tile8x8
 /PaintProc {0.5 setlinewidth pop 0 -2 M 12 4 L
	-4 0 M 12 8 L -4 4 M 8 10 L stroke}
>> matrix makepattern
/Pat9 exch def
/Pattern1 {PatternBgnd KeepColor Pat1 setpattern} bind def
/Pattern2 {PatternBgnd KeepColor Pat2 setpattern} bind def
/Pattern3 {PatternBgnd KeepColor Pat3 setpattern} bind def
/Pattern4 {PatternBgnd KeepColor Landscape {Pat5} {Pat4} ifelse setpattern} bind def
/Pattern5 {PatternBgnd KeepColor Landscape {Pat4} {Pat5} ifelse setpattern} bind def
/Pattern6 {PatternBgnd KeepColor Landscape {Pat9} {Pat6} ifelse setpattern} bind def
/Pattern7 {PatternBgnd KeepColor Landscape {Pat8} {Pat7} ifelse setpattern} bind def
} def
%
%
%End of PostScript Level 2 code
%
/PatternBgnd {
  TransparentPatterns {} {gsave 1 setgray fill grestore} ifelse
} def
%
% Substitute for Level 2 pattern fill codes with
% grayscale if Level 2 support is not selected.
%
/Level1PatternFill {
/Pattern1 {0.250 Density} bind def
/Pattern2 {0.500 Density} bind def
/Pattern3 {0.750 Density} bind def
/Pattern4 {0.125 Density} bind def
/Pattern5 {0.375 Density} bind def
/Pattern6 {0.625 Density} bind def
/Pattern7 {0.875 Density} bind def
} def
%
% Now test for support of Level 2 code
%
Level1 {Level1PatternFill} {Level2PatternFill} ifelse
%
/Symbol-Oblique /Symbol findfont [1 0 .167 1 0 0] makefont
dup length dict begin {1 index /FID eq {pop pop} {def} ifelse} forall
currentdict end definefont pop
%
Level1 SuppressPDFMark or 
{} {
/SDict 10 dict def
systemdict /pdfmark known not {
  userdict /pdfmark systemdict /cleartomark get put
} if
SDict begin [
  /Title (plot_MJ2PCK_N.tex)
  /Subject (gnuplot plot)
  /Creator (gnuplot 5.0 patchlevel 3)
  /Author (mteper)
%  /Producer (gnuplot)
%  /Keywords ()
  /CreationDate (Thu Mar  4 11:54:35 2021)
  /DOCINFO pdfmark
end
} ifelse
%
% Support for boxed text - Ethan A Merritt May 2005
%
/InitTextBox { userdict /TBy2 3 -1 roll put userdict /TBx2 3 -1 roll put
           userdict /TBy1 3 -1 roll put userdict /TBx1 3 -1 roll put
	   /Boxing true def } def
/ExtendTextBox { Boxing
    { gsave dup false charpath pathbbox
      dup TBy2 gt {userdict /TBy2 3 -1 roll put} {pop} ifelse
      dup TBx2 gt {userdict /TBx2 3 -1 roll put} {pop} ifelse
      dup TBy1 lt {userdict /TBy1 3 -1 roll put} {pop} ifelse
      dup TBx1 lt {userdict /TBx1 3 -1 roll put} {pop} ifelse
      grestore } if } def
/PopTextBox { newpath TBx1 TBxmargin sub TBy1 TBymargin sub M
               TBx1 TBxmargin sub TBy2 TBymargin add L
	       TBx2 TBxmargin add TBy2 TBymargin add L
	       TBx2 TBxmargin add TBy1 TBymargin sub L closepath } def
/DrawTextBox { PopTextBox stroke /Boxing false def} def
/FillTextBox { gsave PopTextBox 1 1 1 setrgbcolor fill grestore /Boxing false def} def
0 0 0 0 InitTextBox
/TBxmargin 20 def
/TBymargin 20 def
/Boxing false def
/textshow { ExtendTextBox Gshow } def
%
% redundant definitions for compatibility with prologue.ps older than 5.0.2
/LTB {BL [] LCb DL} def
/LTb {PL [] LCb DL} def
end
%%EndProlog
%%Page: 1 1
gnudict begin
gsave
doclip
0 0 translate
0.050 0.050 scale
0 setgray
newpath
BackgroundColor 0 lt 3 1 roll 0 lt exch 0 lt or or not {BackgroundColor C 1.000 0 0 7200.00 7560.00 BoxColFill} if
1.000 UL
LTb
LCb setrgbcolor
1020 640 M
63 0 V
5756 0 R
-63 0 V
stroke
LTb
LCb setrgbcolor
1020 1594 M
63 0 V
5756 0 R
-63 0 V
stroke
LTb
LCb setrgbcolor
1020 2548 M
63 0 V
5756 0 R
-63 0 V
stroke
LTb
LCb setrgbcolor
1020 3502 M
63 0 V
5756 0 R
-63 0 V
stroke
LTb
LCb setrgbcolor
1020 4457 M
63 0 V
5756 0 R
-63 0 V
stroke
LTb
LCb setrgbcolor
1020 5411 M
63 0 V
5756 0 R
-63 0 V
stroke
LTb
LCb setrgbcolor
1020 6365 M
63 0 V
5756 0 R
-63 0 V
stroke
LTb
LCb setrgbcolor
1020 7319 M
63 0 V
5756 0 R
-63 0 V
stroke
LTb
LCb setrgbcolor
1020 640 M
0 63 V
0 6616 R
0 -63 V
stroke
LTb
LCb setrgbcolor
1990 640 M
0 63 V
0 6616 R
0 -63 V
stroke
LTb
LCb setrgbcolor
2960 640 M
0 63 V
0 6616 R
0 -63 V
stroke
LTb
LCb setrgbcolor
3930 640 M
0 63 V
0 6616 R
0 -63 V
stroke
LTb
LCb setrgbcolor
4899 640 M
0 63 V
0 6616 R
0 -63 V
stroke
LTb
LCb setrgbcolor
5869 640 M
0 63 V
0 6616 R
0 -63 V
stroke
LTb
LCb setrgbcolor
6839 640 M
0 63 V
0 6616 R
0 -63 V
stroke
LTb
LCb setrgbcolor
1.000 UL
LTb
LCb setrgbcolor
1020 7319 N
0 -6679 V
5819 0 V
0 6679 V
-5819 0 V
Z stroke
1.000 UP
1.000 UL
LTb
LCb setrgbcolor
LCb setrgbcolor
LTb
LCb setrgbcolor
LTb
1.500 UP
1.000 UL
LTb
0.58 0.00 0.83 C 5869 2862 M
0 38 V
3175 2426 M
0 42 V
2232 2288 M
0 28 V
-436 -83 R
0 39 V
-237 -60 R
0 58 V
-236 -65 R
0 38 V
1214 2137 M
0 48 V
-59 -3 R
0 57 V
5869 2881 CircleF
3175 2447 CircleF
2232 2302 CircleF
1796 2253 CircleF
1559 2241 CircleF
1323 2224 CircleF
1214 2161 CircleF
1155 2211 CircleF
1.500 UL
LTb
0.58 0.00 0.83 C 1020 2166 M
59 7 V
59 7 V
58 7 V
59 8 V
59 7 V
59 7 V
58 7 V
59 8 V
59 7 V
59 7 V
59 7 V
58 8 V
59 7 V
59 7 V
59 7 V
58 8 V
59 7 V
59 7 V
59 7 V
59 7 V
58 8 V
59 7 V
59 7 V
59 7 V
58 8 V
59 7 V
59 7 V
59 7 V
59 8 V
58 7 V
59 7 V
59 7 V
59 8 V
58 7 V
59 7 V
59 7 V
59 8 V
59 7 V
58 7 V
59 7 V
59 8 V
59 7 V
58 7 V
59 7 V
59 7 V
59 8 V
59 7 V
58 7 V
59 7 V
59 8 V
59 7 V
58 7 V
59 7 V
59 8 V
59 7 V
59 7 V
58 7 V
59 8 V
59 7 V
59 7 V
58 7 V
59 8 V
59 7 V
59 7 V
59 7 V
58 8 V
59 7 V
59 7 V
59 7 V
58 7 V
59 8 V
59 7 V
59 7 V
59 7 V
58 8 V
59 7 V
59 7 V
59 7 V
58 8 V
59 7 V
59 7 V
59 7 V
59 8 V
58 7 V
59 7 V
59 7 V
59 8 V
58 7 V
59 7 V
59 7 V
59 8 V
59 7 V
58 7 V
59 7 V
59 7 V
59 8 V
58 7 V
59 7 V
59 7 V
1.500 UP
stroke
1.000 UL
LTb
0.58 0.00 0.83 C 5869 4609 M
0 115 V
3175 4216 M
0 76 V
2232 4126 M
0 77 V
1796 4039 M
0 86 V
1559 3951 M
0 133 V
-236 -95 R
0 172 V
1214 3960 M
0 172 V
-59 -9 R
0 114 V
5869 4666 Circle
3175 4254 Circle
2232 4165 Circle
1796 4082 Circle
1559 4018 Circle
1323 4075 Circle
1214 4046 Circle
1155 4180 Circle
1.500 UL
LTb
0.58 0.00 0.83 C 1020 4058 M
59 5 V
59 5 V
58 5 V
59 5 V
59 5 V
59 5 V
58 5 V
59 5 V
59 5 V
59 5 V
59 5 V
58 5 V
59 5 V
59 5 V
59 5 V
58 5 V
59 5 V
59 5 V
59 5 V
59 5 V
58 5 V
59 5 V
59 5 V
59 5 V
58 5 V
59 5 V
59 5 V
59 5 V
59 5 V
58 5 V
59 5 V
59 5 V
59 5 V
58 5 V
59 5 V
59 5 V
59 4 V
59 5 V
58 5 V
59 5 V
59 5 V
59 5 V
58 5 V
59 5 V
59 5 V
59 5 V
59 5 V
58 5 V
59 5 V
59 5 V
59 5 V
58 5 V
59 5 V
59 5 V
59 5 V
59 5 V
58 5 V
59 5 V
59 5 V
59 5 V
58 5 V
59 5 V
59 5 V
59 5 V
59 5 V
58 5 V
59 5 V
59 5 V
59 5 V
58 5 V
59 5 V
59 5 V
59 5 V
59 5 V
58 5 V
59 5 V
59 5 V
59 5 V
58 5 V
59 5 V
59 5 V
59 5 V
59 5 V
58 5 V
59 5 V
59 5 V
59 5 V
58 5 V
59 5 V
59 5 V
59 5 V
59 5 V
58 5 V
59 5 V
59 5 V
59 5 V
58 5 V
59 5 V
59 5 V
1.500 UP
stroke
1.000 UL
LTb
0.58 0.00 0.83 C 5869 4425 M
0 95 V
3175 3722 M
0 172 V
2232 3645 M
0 63 V
-436 -75 R
0 96 V
1559 3596 M
0 95 V
1323 3475 M
0 105 V
1214 3436 M
0 152 V
-59 -105 R
0 134 V
5869 4473 BoxF
3175 3808 BoxF
2232 3676 BoxF
1796 3681 BoxF
1559 3644 BoxF
1323 3527 BoxF
1214 3512 BoxF
1155 3550 BoxF
1.500 UL
LTb
0.58 0.00 0.83 C 1020 3532 M
59 8 V
59 7 V
58 8 V
59 8 V
59 8 V
59 7 V
58 8 V
59 8 V
59 7 V
59 8 V
59 8 V
58 8 V
59 7 V
59 8 V
59 8 V
58 7 V
59 8 V
59 8 V
59 8 V
59 7 V
58 8 V
59 8 V
59 7 V
59 8 V
58 8 V
59 7 V
59 8 V
59 8 V
59 8 V
58 7 V
59 8 V
59 8 V
59 7 V
58 8 V
59 8 V
59 8 V
59 7 V
59 8 V
58 8 V
59 7 V
59 8 V
59 8 V
58 8 V
59 7 V
59 8 V
59 8 V
59 7 V
58 8 V
59 8 V
59 8 V
59 7 V
58 8 V
59 8 V
59 7 V
59 8 V
59 8 V
58 8 V
59 7 V
59 8 V
59 8 V
58 7 V
59 8 V
59 8 V
59 8 V
59 7 V
58 8 V
59 8 V
59 7 V
59 8 V
58 8 V
59 8 V
59 7 V
59 8 V
59 8 V
58 7 V
59 8 V
59 8 V
59 7 V
58 8 V
59 8 V
59 8 V
59 7 V
59 8 V
58 8 V
59 7 V
59 8 V
59 8 V
58 8 V
59 7 V
59 8 V
59 8 V
59 7 V
58 8 V
59 8 V
59 8 V
59 7 V
58 8 V
59 8 V
59 7 V
1.500 UP
stroke
1.000 UL
LTb
0.58 0.00 0.83 C 5869 5846 M
0 124 V
3175 5506 M
0 153 V
2232 5201 M
0 248 V
1796 5325 M
0 210 V
1559 5201 M
0 267 V
-236 -67 R
0 210 V
1214 5163 M
0 229 V
-59 28 R
0 286 V
5869 5908 Box
3175 5582 Box
2232 5325 Box
1796 5430 Box
1559 5334 Box
1323 5506 Box
1214 5277 Box
1155 5563 Box
1.500 UL
LTb
0.58 0.00 0.83 C 1020 5350 M
59 6 V
59 7 V
58 7 V
59 6 V
59 7 V
59 6 V
58 7 V
59 7 V
59 6 V
59 7 V
59 6 V
58 7 V
59 7 V
59 6 V
59 7 V
58 7 V
59 6 V
59 7 V
59 6 V
59 7 V
58 7 V
59 6 V
59 7 V
59 7 V
58 6 V
59 7 V
59 6 V
59 7 V
59 7 V
58 6 V
59 7 V
59 7 V
59 6 V
58 7 V
59 6 V
59 7 V
59 7 V
59 6 V
58 7 V
59 6 V
59 7 V
59 7 V
58 6 V
59 7 V
59 7 V
59 6 V
59 7 V
58 6 V
59 7 V
59 7 V
59 6 V
58 7 V
59 7 V
59 6 V
59 7 V
59 6 V
58 7 V
59 7 V
59 6 V
59 7 V
58 7 V
59 6 V
59 7 V
59 6 V
59 7 V
58 7 V
59 6 V
59 7 V
59 7 V
58 6 V
59 7 V
59 6 V
59 7 V
59 7 V
58 6 V
59 7 V
59 6 V
59 7 V
58 7 V
59 6 V
59 7 V
59 7 V
59 6 V
58 7 V
59 6 V
59 7 V
59 7 V
58 6 V
59 7 V
59 7 V
59 6 V
59 7 V
58 6 V
59 7 V
59 7 V
59 6 V
58 7 V
59 7 V
59 6 V
1.500 UP
stroke
1.000 UL
LTb
0.58 0.00 0.83 C 3175 6136 M
0 286 V
2232 6107 M
0 210 V
1796 5706 M
0 344 V
-237 48 R
0 210 V
1323 5792 M
0 229 V
1214 5888 M
0 229 V
-59 -391 R
0 324 V
3175 6279 Star
2232 6212 Star
1796 5878 Star
1559 6203 Star
1323 5907 Star
1214 6002 Star
1155 5888 Star
1.500 UL
LTb
0.58 0.00 0.83 C 1020 5951 M
59 10 V
59 10 V
58 11 V
59 10 V
59 10 V
59 11 V
58 10 V
59 10 V
59 11 V
59 10 V
59 10 V
58 11 V
59 10 V
59 10 V
59 11 V
58 10 V
59 10 V
59 10 V
59 11 V
59 10 V
58 10 V
59 11 V
59 10 V
59 10 V
58 11 V
59 10 V
59 10 V
59 11 V
59 10 V
58 10 V
59 11 V
59 10 V
59 10 V
58 11 V
59 10 V
59 10 V
59 10 V
59 11 V
58 10 V
59 10 V
59 11 V
59 10 V
58 10 V
59 11 V
59 10 V
59 10 V
59 11 V
58 10 V
59 10 V
59 11 V
59 10 V
58 10 V
59 11 V
59 10 V
59 10 V
59 10 V
58 11 V
59 10 V
59 10 V
59 11 V
58 10 V
59 10 V
59 11 V
59 10 V
59 10 V
58 11 V
59 10 V
59 10 V
59 11 V
58 10 V
59 10 V
59 11 V
59 10 V
59 10 V
58 10 V
59 11 V
59 10 V
59 10 V
58 11 V
59 10 V
59 10 V
59 11 V
59 10 V
58 10 V
59 11 V
59 10 V
59 10 V
58 11 V
59 10 V
59 10 V
59 11 V
59 10 V
58 10 V
59 10 V
59 11 V
59 10 V
58 10 V
59 11 V
59 10 V
1.500 UP
stroke
1.000 UL
LTb
0.58 0.00 0.83 C 3175 5344 M
0 286 V
2232 5124 M
0 172 V
-436 29 R
0 114 V
-237 -28 R
0 95 V
1323 5134 M
0 134 V
1214 5020 M
0 286 V
-59 -38 R
0 286 V
3175 5487 DiaF
2232 5210 DiaF
1796 5382 DiaF
1559 5458 DiaF
1323 5201 DiaF
1214 5163 DiaF
1155 5411 DiaF
1.500 UL
LTb
0.58 0.00 0.83 C 1020 5325 M
59 2 V
59 3 V
58 3 V
59 2 V
59 3 V
59 3 V
58 2 V
59 3 V
59 2 V
59 3 V
59 3 V
58 2 V
59 3 V
59 3 V
59 2 V
58 3 V
59 2 V
59 3 V
59 3 V
59 2 V
58 3 V
59 3 V
59 2 V
59 3 V
58 2 V
59 3 V
59 3 V
59 2 V
59 3 V
58 3 V
59 2 V
59 3 V
59 2 V
58 3 V
59 3 V
59 2 V
59 3 V
59 3 V
58 2 V
59 3 V
59 2 V
59 3 V
58 3 V
59 2 V
59 3 V
59 3 V
59 2 V
58 3 V
59 2 V
59 3 V
59 3 V
58 2 V
59 3 V
59 3 V
59 2 V
59 3 V
58 2 V
59 3 V
59 3 V
59 2 V
58 3 V
59 3 V
59 2 V
59 3 V
59 2 V
58 3 V
59 3 V
59 2 V
59 3 V
58 3 V
59 2 V
59 3 V
59 2 V
59 3 V
58 3 V
59 2 V
59 3 V
59 3 V
58 2 V
59 3 V
59 2 V
59 3 V
59 3 V
58 2 V
59 3 V
59 3 V
59 2 V
58 3 V
59 2 V
59 3 V
59 3 V
59 2 V
58 3 V
59 3 V
59 2 V
59 3 V
58 2 V
59 3 V
59 3 V
stroke
2.000 UL
LTb
LCb setrgbcolor
1.000 UL
LTb
LCb setrgbcolor
1020 7319 N
0 -6679 V
5819 0 V
0 6679 V
-5819 0 V
Z stroke
1.000 UP
1.000 UL
LTb
LCb setrgbcolor
stroke
grestore
end
showpage
  }}%
  \put(3929,140){\makebox(0,0){\large{$1/N^2$}}}%
  \put(200,4979){\makebox(0,0){\Large{$\frac{M_{J=2}}{\surd\sigma}$}}}%
  \put(6839,440){\makebox(0,0){\strut{}\ {$0.3$}}}%
  \put(5869,440){\makebox(0,0){\strut{}\ {$0.25$}}}%
  \put(4899,440){\makebox(0,0){\strut{}\ {$0.2$}}}%
  \put(3930,440){\makebox(0,0){\strut{}\ {$0.15$}}}%
  \put(2960,440){\makebox(0,0){\strut{}\ {$0.1$}}}%
  \put(1990,440){\makebox(0,0){\strut{}\ {$0.05$}}}%
  \put(1020,440){\makebox(0,0){\strut{}\ {$0$}}}%
  \put(900,7319){\makebox(0,0)[r]{\strut{}\ \ {$10$}}}%
  \put(900,6365){\makebox(0,0)[r]{\strut{}\ \ {$9$}}}%
  \put(900,5411){\makebox(0,0)[r]{\strut{}\ \ {$8$}}}%
  \put(900,4457){\makebox(0,0)[r]{\strut{}\ \ {$7$}}}%
  \put(900,3502){\makebox(0,0)[r]{\strut{}\ \ {$6$}}}%
  \put(900,2548){\makebox(0,0)[r]{\strut{}\ \ {$5$}}}%
  \put(900,1594){\makebox(0,0)[r]{\strut{}\ \ {$4$}}}%
  \put(900,640){\makebox(0,0)[r]{\strut{}\ \ {$3$}}}%
\end{picture}%
\endgroup
\endinput

\end	{center}
\caption{Continuum masses of the lightest ($\bullet$) and first excited ($\circ$)
  $J^{PC}=2^{++}$ tensors, the lightest ($\blacksquare$) and first excited ($\square$)
  $2^{-+}$ pseudotensors, the lightest $2^{+-}$ ($\ast$), and the lightest
  $2^{--}$ ($\blacklozenge$), all in units of the string tension.
  With extrapolations to $N=\infty$ from $N\leq 12$.}
\label{fig_MJ2PCK_N}
\end{figure}

\begin{figure}[htb]
\begin	{center}
\leavevmode
% GNUPLOT: LaTeX picture with Postscript
\begingroup%
\makeatletter%
\newcommand{\GNUPLOTspecial}{%
  \@sanitize\catcode`\%=14\relax\special}%
\setlength{\unitlength}{0.0500bp}%
\begin{picture}(7200,7560)(0,0)%
  {\GNUPLOTspecial{"
%!PS-Adobe-2.0 EPSF-2.0
%%Title: plot_MJ1K_N.tex
%%Creator: gnuplot 5.0 patchlevel 3
%%CreationDate: Thu Mar  4 20:36:22 2021
%%DocumentFonts: 
%%BoundingBox: 0 0 360 378
%%EndComments
%%BeginProlog
/gnudict 256 dict def
gnudict begin
%
% The following true/false flags may be edited by hand if desired.
% The unit line width and grayscale image gamma correction may also be changed.
%
/Color true def
/Blacktext true def
/Solid false def
/Dashlength 1 def
/Landscape false def
/Level1 false def
/Level3 false def
/Rounded false def
/ClipToBoundingBox false def
/SuppressPDFMark false def
/TransparentPatterns false def
/gnulinewidth 5.000 def
/userlinewidth gnulinewidth def
/Gamma 1.0 def
/BackgroundColor {-1.000 -1.000 -1.000} def
%
/vshift -66 def
/dl1 {
  10.0 Dashlength userlinewidth gnulinewidth div mul mul mul
  Rounded { currentlinewidth 0.75 mul sub dup 0 le { pop 0.01 } if } if
} def
/dl2 {
  10.0 Dashlength userlinewidth gnulinewidth div mul mul mul
  Rounded { currentlinewidth 0.75 mul add } if
} def
/hpt_ 31.5 def
/vpt_ 31.5 def
/hpt hpt_ def
/vpt vpt_ def
/doclip {
  ClipToBoundingBox {
    newpath 0 0 moveto 360 0 lineto 360 378 lineto 0 378 lineto closepath
    clip
  } if
} def
%
% Gnuplot Prolog Version 5.1 (Oct 2015)
%
%/SuppressPDFMark true def
%
/M {moveto} bind def
/L {lineto} bind def
/R {rmoveto} bind def
/V {rlineto} bind def
/N {newpath moveto} bind def
/Z {closepath} bind def
/C {setrgbcolor} bind def
/f {rlineto fill} bind def
/g {setgray} bind def
/Gshow {show} def   % May be redefined later in the file to support UTF-8
/vpt2 vpt 2 mul def
/hpt2 hpt 2 mul def
/Lshow {currentpoint stroke M 0 vshift R 
	Blacktext {gsave 0 setgray textshow grestore} {textshow} ifelse} def
/Rshow {currentpoint stroke M dup stringwidth pop neg vshift R
	Blacktext {gsave 0 setgray textshow grestore} {textshow} ifelse} def
/Cshow {currentpoint stroke M dup stringwidth pop -2 div vshift R 
	Blacktext {gsave 0 setgray textshow grestore} {textshow} ifelse} def
/UP {dup vpt_ mul /vpt exch def hpt_ mul /hpt exch def
  /hpt2 hpt 2 mul def /vpt2 vpt 2 mul def} def
/DL {Color {setrgbcolor Solid {pop []} if 0 setdash}
 {pop pop pop 0 setgray Solid {pop []} if 0 setdash} ifelse} def
/BL {stroke userlinewidth 2 mul setlinewidth
	Rounded {1 setlinejoin 1 setlinecap} if} def
/AL {stroke userlinewidth 2 div setlinewidth
	Rounded {1 setlinejoin 1 setlinecap} if} def
/UL {dup gnulinewidth mul /userlinewidth exch def
	dup 1 lt {pop 1} if 10 mul /udl exch def} def
/PL {stroke userlinewidth setlinewidth
	Rounded {1 setlinejoin 1 setlinecap} if} def
3.8 setmiterlimit
% Classic Line colors (version 5.0)
/LCw {1 1 1} def
/LCb {0 0 0} def
/LCa {0 0 0} def
/LC0 {1 0 0} def
/LC1 {0 1 0} def
/LC2 {0 0 1} def
/LC3 {1 0 1} def
/LC4 {0 1 1} def
/LC5 {1 1 0} def
/LC6 {0 0 0} def
/LC7 {1 0.3 0} def
/LC8 {0.5 0.5 0.5} def
% Default dash patterns (version 5.0)
/LTB {BL [] LCb DL} def
/LTw {PL [] 1 setgray} def
/LTb {PL [] LCb DL} def
/LTa {AL [1 udl mul 2 udl mul] 0 setdash LCa setrgbcolor} def
/LT0 {PL [] LC0 DL} def
/LT1 {PL [2 dl1 3 dl2] LC1 DL} def
/LT2 {PL [1 dl1 1.5 dl2] LC2 DL} def
/LT3 {PL [6 dl1 2 dl2 1 dl1 2 dl2] LC3 DL} def
/LT4 {PL [1 dl1 2 dl2 6 dl1 2 dl2 1 dl1 2 dl2] LC4 DL} def
/LT5 {PL [4 dl1 2 dl2] LC5 DL} def
/LT6 {PL [1.5 dl1 1.5 dl2 1.5 dl1 1.5 dl2 1.5 dl1 6 dl2] LC6 DL} def
/LT7 {PL [3 dl1 3 dl2 1 dl1 3 dl2] LC7 DL} def
/LT8 {PL [2 dl1 2 dl2 2 dl1 6 dl2] LC8 DL} def
/SL {[] 0 setdash} def
/Pnt {stroke [] 0 setdash gsave 1 setlinecap M 0 0 V stroke grestore} def
/Dia {stroke [] 0 setdash 2 copy vpt add M
  hpt neg vpt neg V hpt vpt neg V
  hpt vpt V hpt neg vpt V closepath stroke
  Pnt} def
/Pls {stroke [] 0 setdash vpt sub M 0 vpt2 V
  currentpoint stroke M
  hpt neg vpt neg R hpt2 0 V stroke
 } def
/Box {stroke [] 0 setdash 2 copy exch hpt sub exch vpt add M
  0 vpt2 neg V hpt2 0 V 0 vpt2 V
  hpt2 neg 0 V closepath stroke
  Pnt} def
/Crs {stroke [] 0 setdash exch hpt sub exch vpt add M
  hpt2 vpt2 neg V currentpoint stroke M
  hpt2 neg 0 R hpt2 vpt2 V stroke} def
/TriU {stroke [] 0 setdash 2 copy vpt 1.12 mul add M
  hpt neg vpt -1.62 mul V
  hpt 2 mul 0 V
  hpt neg vpt 1.62 mul V closepath stroke
  Pnt} def
/Star {2 copy Pls Crs} def
/BoxF {stroke [] 0 setdash exch hpt sub exch vpt add M
  0 vpt2 neg V hpt2 0 V 0 vpt2 V
  hpt2 neg 0 V closepath fill} def
/TriUF {stroke [] 0 setdash vpt 1.12 mul add M
  hpt neg vpt -1.62 mul V
  hpt 2 mul 0 V
  hpt neg vpt 1.62 mul V closepath fill} def
/TriD {stroke [] 0 setdash 2 copy vpt 1.12 mul sub M
  hpt neg vpt 1.62 mul V
  hpt 2 mul 0 V
  hpt neg vpt -1.62 mul V closepath stroke
  Pnt} def
/TriDF {stroke [] 0 setdash vpt 1.12 mul sub M
  hpt neg vpt 1.62 mul V
  hpt 2 mul 0 V
  hpt neg vpt -1.62 mul V closepath fill} def
/DiaF {stroke [] 0 setdash vpt add M
  hpt neg vpt neg V hpt vpt neg V
  hpt vpt V hpt neg vpt V closepath fill} def
/Pent {stroke [] 0 setdash 2 copy gsave
  translate 0 hpt M 4 {72 rotate 0 hpt L} repeat
  closepath stroke grestore Pnt} def
/PentF {stroke [] 0 setdash gsave
  translate 0 hpt M 4 {72 rotate 0 hpt L} repeat
  closepath fill grestore} def
/Circle {stroke [] 0 setdash 2 copy
  hpt 0 360 arc stroke Pnt} def
/CircleF {stroke [] 0 setdash hpt 0 360 arc fill} def
/C0 {BL [] 0 setdash 2 copy moveto vpt 90 450 arc} bind def
/C1 {BL [] 0 setdash 2 copy moveto
	2 copy vpt 0 90 arc closepath fill
	vpt 0 360 arc closepath} bind def
/C2 {BL [] 0 setdash 2 copy moveto
	2 copy vpt 90 180 arc closepath fill
	vpt 0 360 arc closepath} bind def
/C3 {BL [] 0 setdash 2 copy moveto
	2 copy vpt 0 180 arc closepath fill
	vpt 0 360 arc closepath} bind def
/C4 {BL [] 0 setdash 2 copy moveto
	2 copy vpt 180 270 arc closepath fill
	vpt 0 360 arc closepath} bind def
/C5 {BL [] 0 setdash 2 copy moveto
	2 copy vpt 0 90 arc
	2 copy moveto
	2 copy vpt 180 270 arc closepath fill
	vpt 0 360 arc} bind def
/C6 {BL [] 0 setdash 2 copy moveto
	2 copy vpt 90 270 arc closepath fill
	vpt 0 360 arc closepath} bind def
/C7 {BL [] 0 setdash 2 copy moveto
	2 copy vpt 0 270 arc closepath fill
	vpt 0 360 arc closepath} bind def
/C8 {BL [] 0 setdash 2 copy moveto
	2 copy vpt 270 360 arc closepath fill
	vpt 0 360 arc closepath} bind def
/C9 {BL [] 0 setdash 2 copy moveto
	2 copy vpt 270 450 arc closepath fill
	vpt 0 360 arc closepath} bind def
/C10 {BL [] 0 setdash 2 copy 2 copy moveto vpt 270 360 arc closepath fill
	2 copy moveto
	2 copy vpt 90 180 arc closepath fill
	vpt 0 360 arc closepath} bind def
/C11 {BL [] 0 setdash 2 copy moveto
	2 copy vpt 0 180 arc closepath fill
	2 copy moveto
	2 copy vpt 270 360 arc closepath fill
	vpt 0 360 arc closepath} bind def
/C12 {BL [] 0 setdash 2 copy moveto
	2 copy vpt 180 360 arc closepath fill
	vpt 0 360 arc closepath} bind def
/C13 {BL [] 0 setdash 2 copy moveto
	2 copy vpt 0 90 arc closepath fill
	2 copy moveto
	2 copy vpt 180 360 arc closepath fill
	vpt 0 360 arc closepath} bind def
/C14 {BL [] 0 setdash 2 copy moveto
	2 copy vpt 90 360 arc closepath fill
	vpt 0 360 arc} bind def
/C15 {BL [] 0 setdash 2 copy vpt 0 360 arc closepath fill
	vpt 0 360 arc closepath} bind def
/Rec {newpath 4 2 roll moveto 1 index 0 rlineto 0 exch rlineto
	neg 0 rlineto closepath} bind def
/Square {dup Rec} bind def
/Bsquare {vpt sub exch vpt sub exch vpt2 Square} bind def
/S0 {BL [] 0 setdash 2 copy moveto 0 vpt rlineto BL Bsquare} bind def
/S1 {BL [] 0 setdash 2 copy vpt Square fill Bsquare} bind def
/S2 {BL [] 0 setdash 2 copy exch vpt sub exch vpt Square fill Bsquare} bind def
/S3 {BL [] 0 setdash 2 copy exch vpt sub exch vpt2 vpt Rec fill Bsquare} bind def
/S4 {BL [] 0 setdash 2 copy exch vpt sub exch vpt sub vpt Square fill Bsquare} bind def
/S5 {BL [] 0 setdash 2 copy 2 copy vpt Square fill
	exch vpt sub exch vpt sub vpt Square fill Bsquare} bind def
/S6 {BL [] 0 setdash 2 copy exch vpt sub exch vpt sub vpt vpt2 Rec fill Bsquare} bind def
/S7 {BL [] 0 setdash 2 copy exch vpt sub exch vpt sub vpt vpt2 Rec fill
	2 copy vpt Square fill Bsquare} bind def
/S8 {BL [] 0 setdash 2 copy vpt sub vpt Square fill Bsquare} bind def
/S9 {BL [] 0 setdash 2 copy vpt sub vpt vpt2 Rec fill Bsquare} bind def
/S10 {BL [] 0 setdash 2 copy vpt sub vpt Square fill 2 copy exch vpt sub exch vpt Square fill
	Bsquare} bind def
/S11 {BL [] 0 setdash 2 copy vpt sub vpt Square fill 2 copy exch vpt sub exch vpt2 vpt Rec fill
	Bsquare} bind def
/S12 {BL [] 0 setdash 2 copy exch vpt sub exch vpt sub vpt2 vpt Rec fill Bsquare} bind def
/S13 {BL [] 0 setdash 2 copy exch vpt sub exch vpt sub vpt2 vpt Rec fill
	2 copy vpt Square fill Bsquare} bind def
/S14 {BL [] 0 setdash 2 copy exch vpt sub exch vpt sub vpt2 vpt Rec fill
	2 copy exch vpt sub exch vpt Square fill Bsquare} bind def
/S15 {BL [] 0 setdash 2 copy Bsquare fill Bsquare} bind def
/D0 {gsave translate 45 rotate 0 0 S0 stroke grestore} bind def
/D1 {gsave translate 45 rotate 0 0 S1 stroke grestore} bind def
/D2 {gsave translate 45 rotate 0 0 S2 stroke grestore} bind def
/D3 {gsave translate 45 rotate 0 0 S3 stroke grestore} bind def
/D4 {gsave translate 45 rotate 0 0 S4 stroke grestore} bind def
/D5 {gsave translate 45 rotate 0 0 S5 stroke grestore} bind def
/D6 {gsave translate 45 rotate 0 0 S6 stroke grestore} bind def
/D7 {gsave translate 45 rotate 0 0 S7 stroke grestore} bind def
/D8 {gsave translate 45 rotate 0 0 S8 stroke grestore} bind def
/D9 {gsave translate 45 rotate 0 0 S9 stroke grestore} bind def
/D10 {gsave translate 45 rotate 0 0 S10 stroke grestore} bind def
/D11 {gsave translate 45 rotate 0 0 S11 stroke grestore} bind def
/D12 {gsave translate 45 rotate 0 0 S12 stroke grestore} bind def
/D13 {gsave translate 45 rotate 0 0 S13 stroke grestore} bind def
/D14 {gsave translate 45 rotate 0 0 S14 stroke grestore} bind def
/D15 {gsave translate 45 rotate 0 0 S15 stroke grestore} bind def
/DiaE {stroke [] 0 setdash vpt add M
  hpt neg vpt neg V hpt vpt neg V
  hpt vpt V hpt neg vpt V closepath stroke} def
/BoxE {stroke [] 0 setdash exch hpt sub exch vpt add M
  0 vpt2 neg V hpt2 0 V 0 vpt2 V
  hpt2 neg 0 V closepath stroke} def
/TriUE {stroke [] 0 setdash vpt 1.12 mul add M
  hpt neg vpt -1.62 mul V
  hpt 2 mul 0 V
  hpt neg vpt 1.62 mul V closepath stroke} def
/TriDE {stroke [] 0 setdash vpt 1.12 mul sub M
  hpt neg vpt 1.62 mul V
  hpt 2 mul 0 V
  hpt neg vpt -1.62 mul V closepath stroke} def
/PentE {stroke [] 0 setdash gsave
  translate 0 hpt M 4 {72 rotate 0 hpt L} repeat
  closepath stroke grestore} def
/CircE {stroke [] 0 setdash 
  hpt 0 360 arc stroke} def
/Opaque {gsave closepath 1 setgray fill grestore 0 setgray closepath} def
/DiaW {stroke [] 0 setdash vpt add M
  hpt neg vpt neg V hpt vpt neg V
  hpt vpt V hpt neg vpt V Opaque stroke} def
/BoxW {stroke [] 0 setdash exch hpt sub exch vpt add M
  0 vpt2 neg V hpt2 0 V 0 vpt2 V
  hpt2 neg 0 V Opaque stroke} def
/TriUW {stroke [] 0 setdash vpt 1.12 mul add M
  hpt neg vpt -1.62 mul V
  hpt 2 mul 0 V
  hpt neg vpt 1.62 mul V Opaque stroke} def
/TriDW {stroke [] 0 setdash vpt 1.12 mul sub M
  hpt neg vpt 1.62 mul V
  hpt 2 mul 0 V
  hpt neg vpt -1.62 mul V Opaque stroke} def
/PentW {stroke [] 0 setdash gsave
  translate 0 hpt M 4 {72 rotate 0 hpt L} repeat
  Opaque stroke grestore} def
/CircW {stroke [] 0 setdash 
  hpt 0 360 arc Opaque stroke} def
/BoxFill {gsave Rec 1 setgray fill grestore} def
/Density {
  /Fillden exch def
  currentrgbcolor
  /ColB exch def /ColG exch def /ColR exch def
  /ColR ColR Fillden mul Fillden sub 1 add def
  /ColG ColG Fillden mul Fillden sub 1 add def
  /ColB ColB Fillden mul Fillden sub 1 add def
  ColR ColG ColB setrgbcolor} def
/BoxColFill {gsave Rec PolyFill} def
/PolyFill {gsave Density fill grestore grestore} def
/h {rlineto rlineto rlineto gsave closepath fill grestore} bind def
%
% PostScript Level 1 Pattern Fill routine for rectangles
% Usage: x y w h s a XX PatternFill
%	x,y = lower left corner of box to be filled
%	w,h = width and height of box
%	  a = angle in degrees between lines and x-axis
%	 XX = 0/1 for no/yes cross-hatch
%
/PatternFill {gsave /PFa [ 9 2 roll ] def
  PFa 0 get PFa 2 get 2 div add PFa 1 get PFa 3 get 2 div add translate
  PFa 2 get -2 div PFa 3 get -2 div PFa 2 get PFa 3 get Rec
  TransparentPatterns {} {gsave 1 setgray fill grestore} ifelse
  clip
  currentlinewidth 0.5 mul setlinewidth
  /PFs PFa 2 get dup mul PFa 3 get dup mul add sqrt def
  0 0 M PFa 5 get rotate PFs -2 div dup translate
  0 1 PFs PFa 4 get div 1 add floor cvi
	{PFa 4 get mul 0 M 0 PFs V} for
  0 PFa 6 get ne {
	0 1 PFs PFa 4 get div 1 add floor cvi
	{PFa 4 get mul 0 2 1 roll M PFs 0 V} for
 } if
  stroke grestore} def
%
/languagelevel where
 {pop languagelevel} {1} ifelse
dup 2 lt
	{/InterpretLevel1 true def
	 /InterpretLevel3 false def}
	{/InterpretLevel1 Level1 def
	 2 gt
	    {/InterpretLevel3 Level3 def}
	    {/InterpretLevel3 false def}
	 ifelse }
 ifelse
%
% PostScript level 2 pattern fill definitions
%
/Level2PatternFill {
/Tile8x8 {/PaintType 2 /PatternType 1 /TilingType 1 /BBox [0 0 8 8] /XStep 8 /YStep 8}
	bind def
/KeepColor {currentrgbcolor [/Pattern /DeviceRGB] setcolorspace} bind def
<< Tile8x8
 /PaintProc {0.5 setlinewidth pop 0 0 M 8 8 L 0 8 M 8 0 L stroke} 
>> matrix makepattern
/Pat1 exch def
<< Tile8x8
 /PaintProc {0.5 setlinewidth pop 0 0 M 8 8 L 0 8 M 8 0 L stroke
	0 4 M 4 8 L 8 4 L 4 0 L 0 4 L stroke}
>> matrix makepattern
/Pat2 exch def
<< Tile8x8
 /PaintProc {0.5 setlinewidth pop 0 0 M 0 8 L
	8 8 L 8 0 L 0 0 L fill}
>> matrix makepattern
/Pat3 exch def
<< Tile8x8
 /PaintProc {0.5 setlinewidth pop -4 8 M 8 -4 L
	0 12 M 12 0 L stroke}
>> matrix makepattern
/Pat4 exch def
<< Tile8x8
 /PaintProc {0.5 setlinewidth pop -4 0 M 8 12 L
	0 -4 M 12 8 L stroke}
>> matrix makepattern
/Pat5 exch def
<< Tile8x8
 /PaintProc {0.5 setlinewidth pop -2 8 M 4 -4 L
	0 12 M 8 -4 L 4 12 M 10 0 L stroke}
>> matrix makepattern
/Pat6 exch def
<< Tile8x8
 /PaintProc {0.5 setlinewidth pop -2 0 M 4 12 L
	0 -4 M 8 12 L 4 -4 M 10 8 L stroke}
>> matrix makepattern
/Pat7 exch def
<< Tile8x8
 /PaintProc {0.5 setlinewidth pop 8 -2 M -4 4 L
	12 0 M -4 8 L 12 4 M 0 10 L stroke}
>> matrix makepattern
/Pat8 exch def
<< Tile8x8
 /PaintProc {0.5 setlinewidth pop 0 -2 M 12 4 L
	-4 0 M 12 8 L -4 4 M 8 10 L stroke}
>> matrix makepattern
/Pat9 exch def
/Pattern1 {PatternBgnd KeepColor Pat1 setpattern} bind def
/Pattern2 {PatternBgnd KeepColor Pat2 setpattern} bind def
/Pattern3 {PatternBgnd KeepColor Pat3 setpattern} bind def
/Pattern4 {PatternBgnd KeepColor Landscape {Pat5} {Pat4} ifelse setpattern} bind def
/Pattern5 {PatternBgnd KeepColor Landscape {Pat4} {Pat5} ifelse setpattern} bind def
/Pattern6 {PatternBgnd KeepColor Landscape {Pat9} {Pat6} ifelse setpattern} bind def
/Pattern7 {PatternBgnd KeepColor Landscape {Pat8} {Pat7} ifelse setpattern} bind def
} def
%
%
%End of PostScript Level 2 code
%
/PatternBgnd {
  TransparentPatterns {} {gsave 1 setgray fill grestore} ifelse
} def
%
% Substitute for Level 2 pattern fill codes with
% grayscale if Level 2 support is not selected.
%
/Level1PatternFill {
/Pattern1 {0.250 Density} bind def
/Pattern2 {0.500 Density} bind def
/Pattern3 {0.750 Density} bind def
/Pattern4 {0.125 Density} bind def
/Pattern5 {0.375 Density} bind def
/Pattern6 {0.625 Density} bind def
/Pattern7 {0.875 Density} bind def
} def
%
% Now test for support of Level 2 code
%
Level1 {Level1PatternFill} {Level2PatternFill} ifelse
%
/Symbol-Oblique /Symbol findfont [1 0 .167 1 0 0] makefont
dup length dict begin {1 index /FID eq {pop pop} {def} ifelse} forall
currentdict end definefont pop
%
Level1 SuppressPDFMark or 
{} {
/SDict 10 dict def
systemdict /pdfmark known not {
  userdict /pdfmark systemdict /cleartomark get put
} if
SDict begin [
  /Title (plot_MJ1K_N.tex)
  /Subject (gnuplot plot)
  /Creator (gnuplot 5.0 patchlevel 3)
  /Author (mteper)
%  /Producer (gnuplot)
%  /Keywords ()
  /CreationDate (Thu Mar  4 20:36:22 2021)
  /DOCINFO pdfmark
end
} ifelse
%
% Support for boxed text - Ethan A Merritt May 2005
%
/InitTextBox { userdict /TBy2 3 -1 roll put userdict /TBx2 3 -1 roll put
           userdict /TBy1 3 -1 roll put userdict /TBx1 3 -1 roll put
	   /Boxing true def } def
/ExtendTextBox { Boxing
    { gsave dup false charpath pathbbox
      dup TBy2 gt {userdict /TBy2 3 -1 roll put} {pop} ifelse
      dup TBx2 gt {userdict /TBx2 3 -1 roll put} {pop} ifelse
      dup TBy1 lt {userdict /TBy1 3 -1 roll put} {pop} ifelse
      dup TBx1 lt {userdict /TBx1 3 -1 roll put} {pop} ifelse
      grestore } if } def
/PopTextBox { newpath TBx1 TBxmargin sub TBy1 TBymargin sub M
               TBx1 TBxmargin sub TBy2 TBymargin add L
	       TBx2 TBxmargin add TBy2 TBymargin add L
	       TBx2 TBxmargin add TBy1 TBymargin sub L closepath } def
/DrawTextBox { PopTextBox stroke /Boxing false def} def
/FillTextBox { gsave PopTextBox 1 1 1 setrgbcolor fill grestore /Boxing false def} def
0 0 0 0 InitTextBox
/TBxmargin 20 def
/TBymargin 20 def
/Boxing false def
/textshow { ExtendTextBox Gshow } def
%
% redundant definitions for compatibility with prologue.ps older than 5.0.2
/LTB {BL [] LCb DL} def
/LTb {PL [] LCb DL} def
end
%%EndProlog
%%Page: 1 1
gnudict begin
gsave
doclip
0 0 translate
0.050 0.050 scale
0 setgray
newpath
BackgroundColor 0 lt 3 1 roll 0 lt exch 0 lt or or not {BackgroundColor C 1.000 0 0 7200.00 7560.00 BoxColFill} if
1.000 UL
LTb
LCb setrgbcolor
1020 640 M
63 0 V
5756 0 R
-63 0 V
stroke
LTb
LCb setrgbcolor
1020 1753 M
63 0 V
5756 0 R
-63 0 V
stroke
LTb
LCb setrgbcolor
1020 2866 M
63 0 V
5756 0 R
-63 0 V
stroke
LTb
LCb setrgbcolor
1020 3980 M
63 0 V
5756 0 R
-63 0 V
stroke
LTb
LCb setrgbcolor
1020 5093 M
63 0 V
5756 0 R
-63 0 V
stroke
LTb
LCb setrgbcolor
1020 6206 M
63 0 V
5756 0 R
-63 0 V
stroke
LTb
LCb setrgbcolor
1020 7319 M
63 0 V
5756 0 R
-63 0 V
stroke
LTb
LCb setrgbcolor
1020 640 M
0 63 V
0 6616 R
0 -63 V
stroke
LTb
LCb setrgbcolor
1990 640 M
0 63 V
0 6616 R
0 -63 V
stroke
LTb
LCb setrgbcolor
2960 640 M
0 63 V
0 6616 R
0 -63 V
stroke
LTb
LCb setrgbcolor
3930 640 M
0 63 V
0 6616 R
0 -63 V
stroke
LTb
LCb setrgbcolor
4899 640 M
0 63 V
0 6616 R
0 -63 V
stroke
LTb
LCb setrgbcolor
5869 640 M
0 63 V
0 6616 R
0 -63 V
stroke
LTb
LCb setrgbcolor
6839 640 M
0 63 V
0 6616 R
0 -63 V
stroke
LTb
LCb setrgbcolor
1.000 UL
LTb
LCb setrgbcolor
1020 7319 N
0 -6679 V
5819 0 V
0 6679 V
-5819 0 V
Z stroke
1.000 UP
1.000 UL
LTb
LCb setrgbcolor
LCb setrgbcolor
LTb
LCb setrgbcolor
LTb
1.500 UP
1.000 UL
LTb
0.58 0.00 0.83 C 3175 2894 M
0 89 V
2232 2771 M
0 93 V
1796 2722 M
0 100 V
1559 2648 M
0 96 V
1323 2597 M
0 96 V
1214 2571 M
0 92 V
-59 -152 R
0 134 V
3175 2939 CircleF
2232 2817 CircleF
1796 2772 CircleF
1559 2696 CircleF
1323 2645 CircleF
1214 2617 CircleF
1155 2578 CircleF
1.500 UL
LTb
0.58 0.00 0.83 C 1020 2599 M
59 10 V
59 10 V
58 10 V
59 9 V
59 10 V
59 10 V
58 10 V
59 10 V
59 9 V
59 10 V
59 10 V
58 10 V
59 10 V
59 9 V
59 10 V
58 10 V
59 10 V
59 10 V
59 9 V
59 10 V
58 10 V
59 10 V
59 10 V
59 9 V
58 10 V
59 10 V
59 10 V
59 10 V
59 9 V
58 10 V
59 10 V
59 10 V
59 10 V
58 9 V
59 10 V
59 10 V
59 10 V
59 10 V
58 9 V
59 10 V
59 10 V
59 10 V
58 10 V
59 9 V
59 10 V
59 10 V
59 10 V
58 10 V
59 10 V
59 9 V
59 10 V
58 10 V
59 10 V
59 10 V
59 9 V
59 10 V
58 10 V
59 10 V
59 10 V
59 9 V
58 10 V
59 10 V
59 10 V
59 10 V
59 9 V
58 10 V
59 10 V
59 10 V
59 10 V
58 9 V
59 10 V
59 10 V
59 10 V
59 10 V
58 9 V
59 10 V
59 10 V
59 10 V
58 10 V
59 9 V
59 10 V
59 10 V
59 10 V
58 10 V
59 9 V
59 10 V
59 10 V
58 10 V
59 10 V
59 9 V
59 10 V
59 10 V
58 10 V
59 10 V
59 9 V
59 10 V
58 10 V
59 10 V
59 10 V
1.500 UP
stroke
1.000 UL
LTb
0.58 0.00 0.83 C 3175 4826 M
0 133 V
2232 4458 M
0 178 V
-436 -55 R
0 178 V
1559 4492 M
0 200 V
1323 4358 M
0 200 V
-109 156 R
0 290 V
-59 -646 R
0 312 V
3175 4892 Circle
2232 4547 Circle
1796 4670 Circle
1559 4592 Circle
1323 4458 Circle
1214 4859 Circle
1155 4514 Circle
1.500 UL
LTb
0.58 0.00 0.83 C 1020 4506 M
59 9 V
59 10 V
58 9 V
59 10 V
59 9 V
59 10 V
58 9 V
59 10 V
59 9 V
59 10 V
59 9 V
58 9 V
59 10 V
59 9 V
59 10 V
58 9 V
59 10 V
59 9 V
59 10 V
59 9 V
58 9 V
59 10 V
59 9 V
59 10 V
58 9 V
59 10 V
59 9 V
59 10 V
59 9 V
58 9 V
59 10 V
59 9 V
59 10 V
58 9 V
59 10 V
59 9 V
59 10 V
59 9 V
58 10 V
59 9 V
59 9 V
59 10 V
58 9 V
59 10 V
59 9 V
59 10 V
59 9 V
58 10 V
59 9 V
59 9 V
59 10 V
58 9 V
59 10 V
59 9 V
59 10 V
59 9 V
58 10 V
59 9 V
59 9 V
59 10 V
58 9 V
59 10 V
59 9 V
59 10 V
59 9 V
58 10 V
59 9 V
59 10 V
59 9 V
58 9 V
59 10 V
59 9 V
59 10 V
59 9 V
58 10 V
59 9 V
59 10 V
59 9 V
58 9 V
59 10 V
59 9 V
59 10 V
59 9 V
58 10 V
59 9 V
59 10 V
59 9 V
58 9 V
59 10 V
59 9 V
59 10 V
59 9 V
58 10 V
59 9 V
59 10 V
59 9 V
58 10 V
59 9 V
59 9 V
1.500 UP
stroke
1.000 UL
LTb
0.58 0.00 0.83 C 5869 6128 M
0 289 V
3175 5493 M
0 268 V
2232 5493 M
0 245 V
1796 5271 M
0 289 V
1559 5443 M
0 201 V
1323 5193 M
0 334 V
1214 5326 M
0 624 V
-59 -334 R
0 267 V
5869 6273 Star
3175 5627 Star
2232 5616 Star
1796 5415 Star
1559 5543 Star
1323 5360 Star
1214 5638 Star
1155 5749 Star
1.500 UL
LTb
0.58 0.00 0.83 C 1020 5555 M
59 1 V
59 1 V
58 1 V
59 1 V
59 1 V
59 1 V
58 1 V
59 1 V
59 1 V
59 2 V
59 1 V
58 1 V
59 1 V
59 1 V
59 1 V
58 1 V
59 1 V
59 1 V
59 1 V
59 2 V
58 1 V
59 1 V
59 1 V
59 1 V
58 1 V
59 1 V
59 1 V
59 1 V
59 1 V
58 2 V
59 1 V
59 1 V
59 1 V
58 1 V
59 1 V
59 1 V
59 1 V
59 1 V
58 1 V
59 1 V
59 2 V
59 1 V
58 1 V
59 1 V
59 1 V
59 1 V
59 1 V
58 1 V
59 1 V
59 1 V
59 2 V
58 1 V
59 1 V
59 1 V
59 1 V
59 1 V
58 1 V
59 1 V
59 1 V
59 1 V
58 2 V
59 1 V
59 1 V
59 1 V
59 1 V
58 1 V
59 1 V
59 1 V
59 1 V
58 1 V
59 1 V
59 2 V
59 1 V
59 1 V
58 1 V
59 1 V
59 1 V
59 1 V
58 1 V
59 1 V
59 1 V
59 2 V
59 1 V
58 1 V
59 1 V
59 1 V
59 1 V
58 1 V
59 1 V
59 1 V
59 1 V
59 1 V
58 2 V
59 1 V
59 1 V
59 1 V
58 1 V
59 1 V
59 1 V
1.500 UP
stroke
1.000 UL
LTb
0.58 0.00 0.83 C 3175 5326 M
0 223 V
2232 4547 M
0 201 V
1796 4503 M
0 334 V
1559 4291 M
0 290 V
1323 3890 M
0 379 V
1214 4091 M
0 378 V
-59 -122 R
0 378 V
3175 5438 DiaF
2232 4647 DiaF
1796 4670 DiaF
1559 4436 DiaF
1323 4080 DiaF
1214 4280 DiaF
1155 4536 DiaF
1.500 UL
LTb
0.58 0.00 0.83 C 1020 4269 M
59 19 V
59 19 V
58 20 V
59 19 V
59 19 V
59 19 V
58 19 V
59 19 V
59 20 V
59 19 V
59 19 V
58 19 V
59 19 V
59 20 V
59 19 V
58 19 V
59 19 V
59 19 V
59 20 V
59 19 V
58 19 V
59 19 V
59 19 V
59 20 V
58 19 V
59 19 V
59 19 V
59 19 V
59 20 V
58 19 V
59 19 V
59 19 V
59 19 V
58 20 V
59 19 V
59 19 V
59 19 V
59 19 V
58 19 V
59 20 V
59 19 V
59 19 V
58 19 V
59 19 V
59 20 V
59 19 V
59 19 V
58 19 V
59 19 V
59 20 V
59 19 V
58 19 V
59 19 V
59 19 V
59 20 V
59 19 V
58 19 V
59 19 V
59 19 V
59 20 V
58 19 V
59 19 V
59 19 V
59 19 V
59 20 V
58 19 V
59 19 V
59 19 V
59 19 V
58 19 V
59 20 V
59 19 V
59 19 V
59 19 V
58 19 V
59 20 V
59 19 V
59 19 V
58 19 V
59 19 V
59 20 V
59 19 V
59 19 V
58 19 V
59 19 V
59 20 V
59 19 V
58 19 V
59 19 V
59 19 V
59 20 V
59 19 V
58 19 V
59 19 V
59 19 V
59 20 V
58 19 V
59 19 V
59 19 V
stroke
2.000 UL
LTb
LCb setrgbcolor
1.000 UL
LTb
LCb setrgbcolor
1020 7319 N
0 -6679 V
5819 0 V
0 6679 V
-5819 0 V
Z stroke
1.000 UP
1.000 UL
LTb
LCb setrgbcolor
stroke
grestore
end
showpage
  }}%
  \put(3929,140){\makebox(0,0){\large{$1/N^2$}}}%
  \put(200,4979){\makebox(0,0){\Large{$\frac{M_{J=1}}{\surd\sigma}$}}}%
  \put(6839,440){\makebox(0,0){\strut{}\ {$0.3$}}}%
  \put(5869,440){\makebox(0,0){\strut{}\ {$0.25$}}}%
  \put(4899,440){\makebox(0,0){\strut{}\ {$0.2$}}}%
  \put(3930,440){\makebox(0,0){\strut{}\ {$0.15$}}}%
  \put(2960,440){\makebox(0,0){\strut{}\ {$0.1$}}}%
  \put(1990,440){\makebox(0,0){\strut{}\ {$0.05$}}}%
  \put(1020,440){\makebox(0,0){\strut{}\ {$0$}}}%
  \put(900,7319){\makebox(0,0)[r]{\strut{}\ \ {$10$}}}%
  \put(900,6206){\makebox(0,0)[r]{\strut{}\ \ {$9$}}}%
  \put(900,5093){\makebox(0,0)[r]{\strut{}\ \ {$8$}}}%
  \put(900,3980){\makebox(0,0)[r]{\strut{}\ \ {$7$}}}%
  \put(900,2866){\makebox(0,0)[r]{\strut{}\ \ {$6$}}}%
  \put(900,1753){\makebox(0,0)[r]{\strut{}\ \ {$5$}}}%
  \put(900,640){\makebox(0,0)[r]{\strut{}\ \ {$4$}}}%
\end{picture}%
\endgroup
\endinput

\end	{center}
\caption{Continuum masses of the lightest ($\bullet$) and first excited ($\circ$) $J^{PC}=1^{+-}$
  glueballs, as well as the lightest $1^{-+}$ ($\ast$) and $1^{--}$ ($\blacklozenge$)
  glueballs. in units of the string tension, with extrapolations to $N=\infty$.}
\label{fig_MJ1K_N}
\end{figure}

%\begin{figure}[htb]
%\begin	{center}
%\leavevmode
%\input	{plot_MJ3K_N.tex}
%\end	{center}
%\caption{Continuum masses of the lightest $J^{PC}=3^{++}$ ($\bullet$) and  $J^{PC}=3^{+-}$
%  ($\circ$) glueballs,in units of the string tension, with extrapolations to $N=\infty$.}
%\label{fig_MJ3K_N}
%\end{figure}



\begin{figure}[htb]
\begin	{center}
\leavevmode
% GNUPLOT: LaTeX picture with Postscript
\begingroup%
\makeatletter%
\newcommand{\GNUPLOTspecial}{%
  \@sanitize\catcode`\%=14\relax\special}%
\setlength{\unitlength}{0.0500bp}%
\begin{picture}(7200,7560)(0,0)%
  {\GNUPLOTspecial{"
%!PS-Adobe-2.0 EPSF-2.0
%%Title: plot_MeffG+GGA1++l26n_SU3.tex
%%Creator: gnuplot 5.0 patchlevel 3
%%CreationDate: Sun Feb  7 15:09:02 2021
%%DocumentFonts: 
%%BoundingBox: 0 0 360 378
%%EndComments
%%BeginProlog
/gnudict 256 dict def
gnudict begin
%
% The following true/false flags may be edited by hand if desired.
% The unit line width and grayscale image gamma correction may also be changed.
%
/Color true def
/Blacktext true def
/Solid false def
/Dashlength 1 def
/Landscape false def
/Level1 false def
/Level3 false def
/Rounded false def
/ClipToBoundingBox false def
/SuppressPDFMark false def
/TransparentPatterns false def
/gnulinewidth 5.000 def
/userlinewidth gnulinewidth def
/Gamma 1.0 def
/BackgroundColor {-1.000 -1.000 -1.000} def
%
/vshift -66 def
/dl1 {
  10.0 Dashlength userlinewidth gnulinewidth div mul mul mul
  Rounded { currentlinewidth 0.75 mul sub dup 0 le { pop 0.01 } if } if
} def
/dl2 {
  10.0 Dashlength userlinewidth gnulinewidth div mul mul mul
  Rounded { currentlinewidth 0.75 mul add } if
} def
/hpt_ 31.5 def
/vpt_ 31.5 def
/hpt hpt_ def
/vpt vpt_ def
/doclip {
  ClipToBoundingBox {
    newpath 0 0 moveto 360 0 lineto 360 378 lineto 0 378 lineto closepath
    clip
  } if
} def
%
% Gnuplot Prolog Version 5.1 (Oct 2015)
%
%/SuppressPDFMark true def
%
/M {moveto} bind def
/L {lineto} bind def
/R {rmoveto} bind def
/V {rlineto} bind def
/N {newpath moveto} bind def
/Z {closepath} bind def
/C {setrgbcolor} bind def
/f {rlineto fill} bind def
/g {setgray} bind def
/Gshow {show} def   % May be redefined later in the file to support UTF-8
/vpt2 vpt 2 mul def
/hpt2 hpt 2 mul def
/Lshow {currentpoint stroke M 0 vshift R 
	Blacktext {gsave 0 setgray textshow grestore} {textshow} ifelse} def
/Rshow {currentpoint stroke M dup stringwidth pop neg vshift R
	Blacktext {gsave 0 setgray textshow grestore} {textshow} ifelse} def
/Cshow {currentpoint stroke M dup stringwidth pop -2 div vshift R 
	Blacktext {gsave 0 setgray textshow grestore} {textshow} ifelse} def
/UP {dup vpt_ mul /vpt exch def hpt_ mul /hpt exch def
  /hpt2 hpt 2 mul def /vpt2 vpt 2 mul def} def
/DL {Color {setrgbcolor Solid {pop []} if 0 setdash}
 {pop pop pop 0 setgray Solid {pop []} if 0 setdash} ifelse} def
/BL {stroke userlinewidth 2 mul setlinewidth
	Rounded {1 setlinejoin 1 setlinecap} if} def
/AL {stroke userlinewidth 2 div setlinewidth
	Rounded {1 setlinejoin 1 setlinecap} if} def
/UL {dup gnulinewidth mul /userlinewidth exch def
	dup 1 lt {pop 1} if 10 mul /udl exch def} def
/PL {stroke userlinewidth setlinewidth
	Rounded {1 setlinejoin 1 setlinecap} if} def
3.8 setmiterlimit
% Classic Line colors (version 5.0)
/LCw {1 1 1} def
/LCb {0 0 0} def
/LCa {0 0 0} def
/LC0 {1 0 0} def
/LC1 {0 1 0} def
/LC2 {0 0 1} def
/LC3 {1 0 1} def
/LC4 {0 1 1} def
/LC5 {1 1 0} def
/LC6 {0 0 0} def
/LC7 {1 0.3 0} def
/LC8 {0.5 0.5 0.5} def
% Default dash patterns (version 5.0)
/LTB {BL [] LCb DL} def
/LTw {PL [] 1 setgray} def
/LTb {PL [] LCb DL} def
/LTa {AL [1 udl mul 2 udl mul] 0 setdash LCa setrgbcolor} def
/LT0 {PL [] LC0 DL} def
/LT1 {PL [2 dl1 3 dl2] LC1 DL} def
/LT2 {PL [1 dl1 1.5 dl2] LC2 DL} def
/LT3 {PL [6 dl1 2 dl2 1 dl1 2 dl2] LC3 DL} def
/LT4 {PL [1 dl1 2 dl2 6 dl1 2 dl2 1 dl1 2 dl2] LC4 DL} def
/LT5 {PL [4 dl1 2 dl2] LC5 DL} def
/LT6 {PL [1.5 dl1 1.5 dl2 1.5 dl1 1.5 dl2 1.5 dl1 6 dl2] LC6 DL} def
/LT7 {PL [3 dl1 3 dl2 1 dl1 3 dl2] LC7 DL} def
/LT8 {PL [2 dl1 2 dl2 2 dl1 6 dl2] LC8 DL} def
/SL {[] 0 setdash} def
/Pnt {stroke [] 0 setdash gsave 1 setlinecap M 0 0 V stroke grestore} def
/Dia {stroke [] 0 setdash 2 copy vpt add M
  hpt neg vpt neg V hpt vpt neg V
  hpt vpt V hpt neg vpt V closepath stroke
  Pnt} def
/Pls {stroke [] 0 setdash vpt sub M 0 vpt2 V
  currentpoint stroke M
  hpt neg vpt neg R hpt2 0 V stroke
 } def
/Box {stroke [] 0 setdash 2 copy exch hpt sub exch vpt add M
  0 vpt2 neg V hpt2 0 V 0 vpt2 V
  hpt2 neg 0 V closepath stroke
  Pnt} def
/Crs {stroke [] 0 setdash exch hpt sub exch vpt add M
  hpt2 vpt2 neg V currentpoint stroke M
  hpt2 neg 0 R hpt2 vpt2 V stroke} def
/TriU {stroke [] 0 setdash 2 copy vpt 1.12 mul add M
  hpt neg vpt -1.62 mul V
  hpt 2 mul 0 V
  hpt neg vpt 1.62 mul V closepath stroke
  Pnt} def
/Star {2 copy Pls Crs} def
/BoxF {stroke [] 0 setdash exch hpt sub exch vpt add M
  0 vpt2 neg V hpt2 0 V 0 vpt2 V
  hpt2 neg 0 V closepath fill} def
/TriUF {stroke [] 0 setdash vpt 1.12 mul add M
  hpt neg vpt -1.62 mul V
  hpt 2 mul 0 V
  hpt neg vpt 1.62 mul V closepath fill} def
/TriD {stroke [] 0 setdash 2 copy vpt 1.12 mul sub M
  hpt neg vpt 1.62 mul V
  hpt 2 mul 0 V
  hpt neg vpt -1.62 mul V closepath stroke
  Pnt} def
/TriDF {stroke [] 0 setdash vpt 1.12 mul sub M
  hpt neg vpt 1.62 mul V
  hpt 2 mul 0 V
  hpt neg vpt -1.62 mul V closepath fill} def
/DiaF {stroke [] 0 setdash vpt add M
  hpt neg vpt neg V hpt vpt neg V
  hpt vpt V hpt neg vpt V closepath fill} def
/Pent {stroke [] 0 setdash 2 copy gsave
  translate 0 hpt M 4 {72 rotate 0 hpt L} repeat
  closepath stroke grestore Pnt} def
/PentF {stroke [] 0 setdash gsave
  translate 0 hpt M 4 {72 rotate 0 hpt L} repeat
  closepath fill grestore} def
/Circle {stroke [] 0 setdash 2 copy
  hpt 0 360 arc stroke Pnt} def
/CircleF {stroke [] 0 setdash hpt 0 360 arc fill} def
/C0 {BL [] 0 setdash 2 copy moveto vpt 90 450 arc} bind def
/C1 {BL [] 0 setdash 2 copy moveto
	2 copy vpt 0 90 arc closepath fill
	vpt 0 360 arc closepath} bind def
/C2 {BL [] 0 setdash 2 copy moveto
	2 copy vpt 90 180 arc closepath fill
	vpt 0 360 arc closepath} bind def
/C3 {BL [] 0 setdash 2 copy moveto
	2 copy vpt 0 180 arc closepath fill
	vpt 0 360 arc closepath} bind def
/C4 {BL [] 0 setdash 2 copy moveto
	2 copy vpt 180 270 arc closepath fill
	vpt 0 360 arc closepath} bind def
/C5 {BL [] 0 setdash 2 copy moveto
	2 copy vpt 0 90 arc
	2 copy moveto
	2 copy vpt 180 270 arc closepath fill
	vpt 0 360 arc} bind def
/C6 {BL [] 0 setdash 2 copy moveto
	2 copy vpt 90 270 arc closepath fill
	vpt 0 360 arc closepath} bind def
/C7 {BL [] 0 setdash 2 copy moveto
	2 copy vpt 0 270 arc closepath fill
	vpt 0 360 arc closepath} bind def
/C8 {BL [] 0 setdash 2 copy moveto
	2 copy vpt 270 360 arc closepath fill
	vpt 0 360 arc closepath} bind def
/C9 {BL [] 0 setdash 2 copy moveto
	2 copy vpt 270 450 arc closepath fill
	vpt 0 360 arc closepath} bind def
/C10 {BL [] 0 setdash 2 copy 2 copy moveto vpt 270 360 arc closepath fill
	2 copy moveto
	2 copy vpt 90 180 arc closepath fill
	vpt 0 360 arc closepath} bind def
/C11 {BL [] 0 setdash 2 copy moveto
	2 copy vpt 0 180 arc closepath fill
	2 copy moveto
	2 copy vpt 270 360 arc closepath fill
	vpt 0 360 arc closepath} bind def
/C12 {BL [] 0 setdash 2 copy moveto
	2 copy vpt 180 360 arc closepath fill
	vpt 0 360 arc closepath} bind def
/C13 {BL [] 0 setdash 2 copy moveto
	2 copy vpt 0 90 arc closepath fill
	2 copy moveto
	2 copy vpt 180 360 arc closepath fill
	vpt 0 360 arc closepath} bind def
/C14 {BL [] 0 setdash 2 copy moveto
	2 copy vpt 90 360 arc closepath fill
	vpt 0 360 arc} bind def
/C15 {BL [] 0 setdash 2 copy vpt 0 360 arc closepath fill
	vpt 0 360 arc closepath} bind def
/Rec {newpath 4 2 roll moveto 1 index 0 rlineto 0 exch rlineto
	neg 0 rlineto closepath} bind def
/Square {dup Rec} bind def
/Bsquare {vpt sub exch vpt sub exch vpt2 Square} bind def
/S0 {BL [] 0 setdash 2 copy moveto 0 vpt rlineto BL Bsquare} bind def
/S1 {BL [] 0 setdash 2 copy vpt Square fill Bsquare} bind def
/S2 {BL [] 0 setdash 2 copy exch vpt sub exch vpt Square fill Bsquare} bind def
/S3 {BL [] 0 setdash 2 copy exch vpt sub exch vpt2 vpt Rec fill Bsquare} bind def
/S4 {BL [] 0 setdash 2 copy exch vpt sub exch vpt sub vpt Square fill Bsquare} bind def
/S5 {BL [] 0 setdash 2 copy 2 copy vpt Square fill
	exch vpt sub exch vpt sub vpt Square fill Bsquare} bind def
/S6 {BL [] 0 setdash 2 copy exch vpt sub exch vpt sub vpt vpt2 Rec fill Bsquare} bind def
/S7 {BL [] 0 setdash 2 copy exch vpt sub exch vpt sub vpt vpt2 Rec fill
	2 copy vpt Square fill Bsquare} bind def
/S8 {BL [] 0 setdash 2 copy vpt sub vpt Square fill Bsquare} bind def
/S9 {BL [] 0 setdash 2 copy vpt sub vpt vpt2 Rec fill Bsquare} bind def
/S10 {BL [] 0 setdash 2 copy vpt sub vpt Square fill 2 copy exch vpt sub exch vpt Square fill
	Bsquare} bind def
/S11 {BL [] 0 setdash 2 copy vpt sub vpt Square fill 2 copy exch vpt sub exch vpt2 vpt Rec fill
	Bsquare} bind def
/S12 {BL [] 0 setdash 2 copy exch vpt sub exch vpt sub vpt2 vpt Rec fill Bsquare} bind def
/S13 {BL [] 0 setdash 2 copy exch vpt sub exch vpt sub vpt2 vpt Rec fill
	2 copy vpt Square fill Bsquare} bind def
/S14 {BL [] 0 setdash 2 copy exch vpt sub exch vpt sub vpt2 vpt Rec fill
	2 copy exch vpt sub exch vpt Square fill Bsquare} bind def
/S15 {BL [] 0 setdash 2 copy Bsquare fill Bsquare} bind def
/D0 {gsave translate 45 rotate 0 0 S0 stroke grestore} bind def
/D1 {gsave translate 45 rotate 0 0 S1 stroke grestore} bind def
/D2 {gsave translate 45 rotate 0 0 S2 stroke grestore} bind def
/D3 {gsave translate 45 rotate 0 0 S3 stroke grestore} bind def
/D4 {gsave translate 45 rotate 0 0 S4 stroke grestore} bind def
/D5 {gsave translate 45 rotate 0 0 S5 stroke grestore} bind def
/D6 {gsave translate 45 rotate 0 0 S6 stroke grestore} bind def
/D7 {gsave translate 45 rotate 0 0 S7 stroke grestore} bind def
/D8 {gsave translate 45 rotate 0 0 S8 stroke grestore} bind def
/D9 {gsave translate 45 rotate 0 0 S9 stroke grestore} bind def
/D10 {gsave translate 45 rotate 0 0 S10 stroke grestore} bind def
/D11 {gsave translate 45 rotate 0 0 S11 stroke grestore} bind def
/D12 {gsave translate 45 rotate 0 0 S12 stroke grestore} bind def
/D13 {gsave translate 45 rotate 0 0 S13 stroke grestore} bind def
/D14 {gsave translate 45 rotate 0 0 S14 stroke grestore} bind def
/D15 {gsave translate 45 rotate 0 0 S15 stroke grestore} bind def
/DiaE {stroke [] 0 setdash vpt add M
  hpt neg vpt neg V hpt vpt neg V
  hpt vpt V hpt neg vpt V closepath stroke} def
/BoxE {stroke [] 0 setdash exch hpt sub exch vpt add M
  0 vpt2 neg V hpt2 0 V 0 vpt2 V
  hpt2 neg 0 V closepath stroke} def
/TriUE {stroke [] 0 setdash vpt 1.12 mul add M
  hpt neg vpt -1.62 mul V
  hpt 2 mul 0 V
  hpt neg vpt 1.62 mul V closepath stroke} def
/TriDE {stroke [] 0 setdash vpt 1.12 mul sub M
  hpt neg vpt 1.62 mul V
  hpt 2 mul 0 V
  hpt neg vpt -1.62 mul V closepath stroke} def
/PentE {stroke [] 0 setdash gsave
  translate 0 hpt M 4 {72 rotate 0 hpt L} repeat
  closepath stroke grestore} def
/CircE {stroke [] 0 setdash 
  hpt 0 360 arc stroke} def
/Opaque {gsave closepath 1 setgray fill grestore 0 setgray closepath} def
/DiaW {stroke [] 0 setdash vpt add M
  hpt neg vpt neg V hpt vpt neg V
  hpt vpt V hpt neg vpt V Opaque stroke} def
/BoxW {stroke [] 0 setdash exch hpt sub exch vpt add M
  0 vpt2 neg V hpt2 0 V 0 vpt2 V
  hpt2 neg 0 V Opaque stroke} def
/TriUW {stroke [] 0 setdash vpt 1.12 mul add M
  hpt neg vpt -1.62 mul V
  hpt 2 mul 0 V
  hpt neg vpt 1.62 mul V Opaque stroke} def
/TriDW {stroke [] 0 setdash vpt 1.12 mul sub M
  hpt neg vpt 1.62 mul V
  hpt 2 mul 0 V
  hpt neg vpt -1.62 mul V Opaque stroke} def
/PentW {stroke [] 0 setdash gsave
  translate 0 hpt M 4 {72 rotate 0 hpt L} repeat
  Opaque stroke grestore} def
/CircW {stroke [] 0 setdash 
  hpt 0 360 arc Opaque stroke} def
/BoxFill {gsave Rec 1 setgray fill grestore} def
/Density {
  /Fillden exch def
  currentrgbcolor
  /ColB exch def /ColG exch def /ColR exch def
  /ColR ColR Fillden mul Fillden sub 1 add def
  /ColG ColG Fillden mul Fillden sub 1 add def
  /ColB ColB Fillden mul Fillden sub 1 add def
  ColR ColG ColB setrgbcolor} def
/BoxColFill {gsave Rec PolyFill} def
/PolyFill {gsave Density fill grestore grestore} def
/h {rlineto rlineto rlineto gsave closepath fill grestore} bind def
%
% PostScript Level 1 Pattern Fill routine for rectangles
% Usage: x y w h s a XX PatternFill
%	x,y = lower left corner of box to be filled
%	w,h = width and height of box
%	  a = angle in degrees between lines and x-axis
%	 XX = 0/1 for no/yes cross-hatch
%
/PatternFill {gsave /PFa [ 9 2 roll ] def
  PFa 0 get PFa 2 get 2 div add PFa 1 get PFa 3 get 2 div add translate
  PFa 2 get -2 div PFa 3 get -2 div PFa 2 get PFa 3 get Rec
  TransparentPatterns {} {gsave 1 setgray fill grestore} ifelse
  clip
  currentlinewidth 0.5 mul setlinewidth
  /PFs PFa 2 get dup mul PFa 3 get dup mul add sqrt def
  0 0 M PFa 5 get rotate PFs -2 div dup translate
  0 1 PFs PFa 4 get div 1 add floor cvi
	{PFa 4 get mul 0 M 0 PFs V} for
  0 PFa 6 get ne {
	0 1 PFs PFa 4 get div 1 add floor cvi
	{PFa 4 get mul 0 2 1 roll M PFs 0 V} for
 } if
  stroke grestore} def
%
/languagelevel where
 {pop languagelevel} {1} ifelse
dup 2 lt
	{/InterpretLevel1 true def
	 /InterpretLevel3 false def}
	{/InterpretLevel1 Level1 def
	 2 gt
	    {/InterpretLevel3 Level3 def}
	    {/InterpretLevel3 false def}
	 ifelse }
 ifelse
%
% PostScript level 2 pattern fill definitions
%
/Level2PatternFill {
/Tile8x8 {/PaintType 2 /PatternType 1 /TilingType 1 /BBox [0 0 8 8] /XStep 8 /YStep 8}
	bind def
/KeepColor {currentrgbcolor [/Pattern /DeviceRGB] setcolorspace} bind def
<< Tile8x8
 /PaintProc {0.5 setlinewidth pop 0 0 M 8 8 L 0 8 M 8 0 L stroke} 
>> matrix makepattern
/Pat1 exch def
<< Tile8x8
 /PaintProc {0.5 setlinewidth pop 0 0 M 8 8 L 0 8 M 8 0 L stroke
	0 4 M 4 8 L 8 4 L 4 0 L 0 4 L stroke}
>> matrix makepattern
/Pat2 exch def
<< Tile8x8
 /PaintProc {0.5 setlinewidth pop 0 0 M 0 8 L
	8 8 L 8 0 L 0 0 L fill}
>> matrix makepattern
/Pat3 exch def
<< Tile8x8
 /PaintProc {0.5 setlinewidth pop -4 8 M 8 -4 L
	0 12 M 12 0 L stroke}
>> matrix makepattern
/Pat4 exch def
<< Tile8x8
 /PaintProc {0.5 setlinewidth pop -4 0 M 8 12 L
	0 -4 M 12 8 L stroke}
>> matrix makepattern
/Pat5 exch def
<< Tile8x8
 /PaintProc {0.5 setlinewidth pop -2 8 M 4 -4 L
	0 12 M 8 -4 L 4 12 M 10 0 L stroke}
>> matrix makepattern
/Pat6 exch def
<< Tile8x8
 /PaintProc {0.5 setlinewidth pop -2 0 M 4 12 L
	0 -4 M 8 12 L 4 -4 M 10 8 L stroke}
>> matrix makepattern
/Pat7 exch def
<< Tile8x8
 /PaintProc {0.5 setlinewidth pop 8 -2 M -4 4 L
	12 0 M -4 8 L 12 4 M 0 10 L stroke}
>> matrix makepattern
/Pat8 exch def
<< Tile8x8
 /PaintProc {0.5 setlinewidth pop 0 -2 M 12 4 L
	-4 0 M 12 8 L -4 4 M 8 10 L stroke}
>> matrix makepattern
/Pat9 exch def
/Pattern1 {PatternBgnd KeepColor Pat1 setpattern} bind def
/Pattern2 {PatternBgnd KeepColor Pat2 setpattern} bind def
/Pattern3 {PatternBgnd KeepColor Pat3 setpattern} bind def
/Pattern4 {PatternBgnd KeepColor Landscape {Pat5} {Pat4} ifelse setpattern} bind def
/Pattern5 {PatternBgnd KeepColor Landscape {Pat4} {Pat5} ifelse setpattern} bind def
/Pattern6 {PatternBgnd KeepColor Landscape {Pat9} {Pat6} ifelse setpattern} bind def
/Pattern7 {PatternBgnd KeepColor Landscape {Pat8} {Pat7} ifelse setpattern} bind def
} def
%
%
%End of PostScript Level 2 code
%
/PatternBgnd {
  TransparentPatterns {} {gsave 1 setgray fill grestore} ifelse
} def
%
% Substitute for Level 2 pattern fill codes with
% grayscale if Level 2 support is not selected.
%
/Level1PatternFill {
/Pattern1 {0.250 Density} bind def
/Pattern2 {0.500 Density} bind def
/Pattern3 {0.750 Density} bind def
/Pattern4 {0.125 Density} bind def
/Pattern5 {0.375 Density} bind def
/Pattern6 {0.625 Density} bind def
/Pattern7 {0.875 Density} bind def
} def
%
% Now test for support of Level 2 code
%
Level1 {Level1PatternFill} {Level2PatternFill} ifelse
%
/Symbol-Oblique /Symbol findfont [1 0 .167 1 0 0] makefont
dup length dict begin {1 index /FID eq {pop pop} {def} ifelse} forall
currentdict end definefont pop
%
Level1 SuppressPDFMark or 
{} {
/SDict 10 dict def
systemdict /pdfmark known not {
  userdict /pdfmark systemdict /cleartomark get put
} if
SDict begin [
  /Title (plot_MeffG+GGA1++l26n_SU3.tex)
  /Subject (gnuplot plot)
  /Creator (gnuplot 5.0 patchlevel 3)
  /Author (mteper)
%  /Producer (gnuplot)
%  /Keywords ()
  /CreationDate (Sun Feb  7 15:09:02 2021)
  /DOCINFO pdfmark
end
} ifelse
%
% Support for boxed text - Ethan A Merritt May 2005
%
/InitTextBox { userdict /TBy2 3 -1 roll put userdict /TBx2 3 -1 roll put
           userdict /TBy1 3 -1 roll put userdict /TBx1 3 -1 roll put
	   /Boxing true def } def
/ExtendTextBox { Boxing
    { gsave dup false charpath pathbbox
      dup TBy2 gt {userdict /TBy2 3 -1 roll put} {pop} ifelse
      dup TBx2 gt {userdict /TBx2 3 -1 roll put} {pop} ifelse
      dup TBy1 lt {userdict /TBy1 3 -1 roll put} {pop} ifelse
      dup TBx1 lt {userdict /TBx1 3 -1 roll put} {pop} ifelse
      grestore } if } def
/PopTextBox { newpath TBx1 TBxmargin sub TBy1 TBymargin sub M
               TBx1 TBxmargin sub TBy2 TBymargin add L
	       TBx2 TBxmargin add TBy2 TBymargin add L
	       TBx2 TBxmargin add TBy1 TBymargin sub L closepath } def
/DrawTextBox { PopTextBox stroke /Boxing false def} def
/FillTextBox { gsave PopTextBox 1 1 1 setrgbcolor fill grestore /Boxing false def} def
0 0 0 0 InitTextBox
/TBxmargin 20 def
/TBymargin 20 def
/Boxing false def
/textshow { ExtendTextBox Gshow } def
%
% redundant definitions for compatibility with prologue.ps older than 5.0.2
/LTB {BL [] LCb DL} def
/LTb {PL [] LCb DL} def
end
%%EndProlog
%%Page: 1 1
gnudict begin
gsave
doclip
0 0 translate
0.050 0.050 scale
0 setgray
newpath
BackgroundColor 0 lt 3 1 roll 0 lt exch 0 lt or or not {BackgroundColor C 1.000 0 0 7200.00 7560.00 BoxColFill} if
1.000 UL
LTb
LCb setrgbcolor
1340 640 M
63 0 V
5436 0 R
-63 0 V
stroke
LTb
LCb setrgbcolor
1340 1475 M
63 0 V
5436 0 R
-63 0 V
stroke
LTb
LCb setrgbcolor
1340 2310 M
63 0 V
5436 0 R
-63 0 V
stroke
LTb
LCb setrgbcolor
1340 3145 M
63 0 V
5436 0 R
-63 0 V
stroke
LTb
LCb setrgbcolor
1340 3980 M
63 0 V
5436 0 R
-63 0 V
stroke
LTb
LCb setrgbcolor
1340 4814 M
63 0 V
5436 0 R
-63 0 V
stroke
LTb
LCb setrgbcolor
1340 5649 M
63 0 V
5436 0 R
-63 0 V
stroke
LTb
LCb setrgbcolor
1340 6484 M
63 0 V
5436 0 R
-63 0 V
stroke
LTb
LCb setrgbcolor
1340 7319 M
63 0 V
5436 0 R
-63 0 V
stroke
LTb
LCb setrgbcolor
1340 640 M
0 63 V
0 6616 R
0 -63 V
stroke
LTb
LCb setrgbcolor
2126 640 M
0 63 V
0 6616 R
0 -63 V
stroke
LTb
LCb setrgbcolor
2911 640 M
0 63 V
0 6616 R
0 -63 V
stroke
LTb
LCb setrgbcolor
3697 640 M
0 63 V
0 6616 R
0 -63 V
stroke
LTb
LCb setrgbcolor
4482 640 M
0 63 V
0 6616 R
0 -63 V
stroke
LTb
LCb setrgbcolor
5268 640 M
0 63 V
0 6616 R
0 -63 V
stroke
LTb
LCb setrgbcolor
6053 640 M
0 63 V
0 6616 R
0 -63 V
stroke
LTb
LCb setrgbcolor
6839 640 M
0 63 V
0 6616 R
0 -63 V
stroke
LTb
LCb setrgbcolor
1.000 UL
LTb
LCb setrgbcolor
1340 7319 N
0 -6679 V
5499 0 V
0 6679 V
-5499 0 V
Z stroke
1.000 UP
1.000 UL
LTb
LCb setrgbcolor
LCb setrgbcolor
LTb
LCb setrgbcolor
LTb
1.500 UP
1.000 UL
LTb
0.58 0.00 0.83 C 1764 2800 M
0 13 V
786 -96 R
0 19 V
785 -41 R
0 32 V
786 -39 R
0 47 V
785 -72 R
0 77 V
786 -80 R
0 120 V
786 -131 R
0 197 V
1764 4449 M
0 17 V
786 -199 R
0 37 V
785 -119 R
0 95 V
786 -315 R
0 217 V
1764 5714 M
0 21 V
786 -353 R
0 60 V
785 -101 R
0 210 V
1764 5752 M
0 20 V
786 -350 R
0 72 V
785 -356 R
0 189 V
1764 6117 M
0 27 V
786 -500 R
0 84 V
785 -228 R
0 258 V
1764 6751 M
0 34 V
786 -491 R
0 110 V
785 -586 R
0 351 V
1764 2806 CircleF
2550 2726 CircleF
3335 2711 CircleF
4121 2711 CircleF
4906 2702 CircleF
5692 2720 CircleF
6478 2748 CircleF
1764 4457 CircleF
2550 4286 CircleF
3335 4232 CircleF
4121 4073 CircleF
1764 5725 CircleF
2550 5412 CircleF
3335 5446 CircleF
1764 5762 CircleF
2550 5458 CircleF
3335 5232 CircleF
1764 6130 CircleF
2550 5686 CircleF
3335 5629 CircleF
1764 6768 CircleF
2550 6349 CircleF
3335 5993 CircleF
2.000 UP
1.000 UL
LTb
0.58 0.00 0.83 C 1764 5082 M
0 36 V
786 -226 R
0 71 V
785 -204 R
0 176 V
786 -253 R
0 395 V
1764 5100 DiaF
2550 4927 DiaF
3335 4847 DiaF
4121 4879 DiaF
1.500 UP
1.000 UL
LTb
0.58 0.00 0.83 C 1701 2803 M
0 12 V
786 -96 R
0 18 V
786 -40 R
0 31 V
785 -40 R
0 47 V
786 -70 R
0 78 V
785 -88 R
0 120 V
786 -125 R
0 196 V
1701 4455 M
0 17 V
786 -199 R
0 37 V
786 -120 R
0 94 V
785 -308 R
0 219 V
1701 5715 M
0 21 V
786 -353 R
0 60 V
786 -97 R
0 211 V
1701 5745 M
0 22 V
786 -359 R
0 72 V
786 -401 R
0 189 V
1701 5987 M
0 25 V
786 -551 R
0 82 V
786 -167 R
0 199 V
1701 6777 M
0 35 V
786 -607 R
0 112 V
786 -638 R
0 297 V
1701 2809 Circle
2487 2728 Circle
3273 2712 Circle
4058 2712 Circle
4844 2704 Circle
5629 2715 Circle
6415 2748 Circle
1701 4463 Circle
2487 4292 Circle
3273 4237 Circle
4058 4085 Circle
1701 5725 Circle
2487 5413 Circle
3273 5451 Circle
1701 5756 Circle
2487 5444 Circle
3273 5174 Circle
1701 5999 Circle
2487 5502 Circle
3273 5475 Circle
1701 6794 Circle
2487 6261 Circle
3273 5828 Circle
1.000 UL
LTb
0.58 0.00 0.83 C 1340 4748 M
56 0 V
55 0 V
56 0 V
55 0 V
56 0 V
55 0 V
56 0 V
55 0 V
56 0 V
55 0 V
56 0 V
56 0 V
55 0 V
56 0 V
55 0 V
56 0 V
55 0 V
56 0 V
55 0 V
56 0 V
55 0 V
56 0 V
56 0 V
55 0 V
56 0 V
55 0 V
56 0 V
55 0 V
56 0 V
55 0 V
56 0 V
55 0 V
56 0 V
56 0 V
55 0 V
56 0 V
55 0 V
56 0 V
55 0 V
56 0 V
55 0 V
56 0 V
55 0 V
56 0 V
56 0 V
55 0 V
56 0 V
55 0 V
56 0 V
55 0 V
56 0 V
55 0 V
56 0 V
55 0 V
56 0 V
56 0 V
55 0 V
56 0 V
55 0 V
56 0 V
55 0 V
56 0 V
55 0 V
56 0 V
55 0 V
56 0 V
56 0 V
55 0 V
56 0 V
55 0 V
56 0 V
55 0 V
56 0 V
55 0 V
56 0 V
55 0 V
56 0 V
56 0 V
55 0 V
56 0 V
55 0 V
56 0 V
55 0 V
56 0 V
55 0 V
56 0 V
55 0 V
56 0 V
56 0 V
55 0 V
56 0 V
55 0 V
56 0 V
55 0 V
56 0 V
55 0 V
56 0 V
55 0 V
56 0 V
stroke
2.000 UL
LTb
LCb setrgbcolor
1.000 UL
LTb
LCb setrgbcolor
1340 7319 N
0 -6679 V
5499 0 V
0 6679 V
-5499 0 V
Z stroke
1.000 UP
1.000 UL
LTb
LCb setrgbcolor
stroke
grestore
end
showpage
  }}%
  \put(4089,140){\makebox(0,0){\large{$t/a$}}}%
  \put(160,5179){\makebox(0,0){\Large{$aE_{eff}(t)$}}}%
  \put(6839,440){\makebox(0,0){\strut{}\ {$7$}}}%
  \put(6053,440){\makebox(0,0){\strut{}\ {$6$}}}%
  \put(5268,440){\makebox(0,0){\strut{}\ {$5$}}}%
  \put(4482,440){\makebox(0,0){\strut{}\ {$4$}}}%
  \put(3697,440){\makebox(0,0){\strut{}\ {$3$}}}%
  \put(2911,440){\makebox(0,0){\strut{}\ {$2$}}}%
  \put(2126,440){\makebox(0,0){\strut{}\ {$1$}}}%
  \put(1340,440){\makebox(0,0){\strut{}\ {$0$}}}%
  \put(1220,7319){\makebox(0,0)[r]{\strut{}\ \ {$1.6$}}}%
  \put(1220,6484){\makebox(0,0)[r]{\strut{}\ \ {$1.4$}}}%
  \put(1220,5649){\makebox(0,0)[r]{\strut{}\ \ {$1.2$}}}%
  \put(1220,4814){\makebox(0,0)[r]{\strut{}\ \ {$1$}}}%
  \put(1220,3980){\makebox(0,0)[r]{\strut{}\ \ {$0.8$}}}%
  \put(1220,3145){\makebox(0,0)[r]{\strut{}\ \ {$0.6$}}}%
  \put(1220,2310){\makebox(0,0)[r]{\strut{}\ \ {$0.4$}}}%
  \put(1220,1475){\makebox(0,0)[r]{\strut{}\ \ {$0.2$}}}%
  \put(1220,640){\makebox(0,0)[r]{\strut{}\ \ {$0$}}}%
\end{picture}%
\endgroup
\endinput

\end	{center}
\caption{Effective masses for the lightest few glueballs in the `scalar' $A_1^{++}$
  representation, for the single trace operators ($\circ$) and for the same set
  augmented with double trace operators (filled points), with points shifted for clarity.
  The extra `scattering' state amongst the latter is shown as $\blacklozenge$. 
  Horizontal line indicates twice the mass of the lightest glueball.
  On the $26^326$ lattice at $\beta=6.235$ in SU(3).}
\label{fig_MeffG+GGA1++l26n_SU3}
\end{figure}




\begin{figure}[htb]
\begin	{center}
\leavevmode
% GNUPLOT: LaTeX picture with Postscript
\begingroup%
\makeatletter%
\newcommand{\GNUPLOTspecial}{%
  \@sanitize\catcode`\%=14\relax\special}%
\setlength{\unitlength}{0.0500bp}%
\begin{picture}(7200,7560)(0,0)%
  {\GNUPLOTspecial{"
%!PS-Adobe-2.0 EPSF-2.0
%%Title: plot_MeffGGA1++l26n_SU3.tex
%%Creator: gnuplot 5.0 patchlevel 3
%%CreationDate: Fri Feb  5 21:12:30 2021
%%DocumentFonts: 
%%BoundingBox: 0 0 360 378
%%EndComments
%%BeginProlog
/gnudict 256 dict def
gnudict begin
%
% The following true/false flags may be edited by hand if desired.
% The unit line width and grayscale image gamma correction may also be changed.
%
/Color true def
/Blacktext true def
/Solid false def
/Dashlength 1 def
/Landscape false def
/Level1 false def
/Level3 false def
/Rounded false def
/ClipToBoundingBox false def
/SuppressPDFMark false def
/TransparentPatterns false def
/gnulinewidth 5.000 def
/userlinewidth gnulinewidth def
/Gamma 1.0 def
/BackgroundColor {-1.000 -1.000 -1.000} def
%
/vshift -66 def
/dl1 {
  10.0 Dashlength userlinewidth gnulinewidth div mul mul mul
  Rounded { currentlinewidth 0.75 mul sub dup 0 le { pop 0.01 } if } if
} def
/dl2 {
  10.0 Dashlength userlinewidth gnulinewidth div mul mul mul
  Rounded { currentlinewidth 0.75 mul add } if
} def
/hpt_ 31.5 def
/vpt_ 31.5 def
/hpt hpt_ def
/vpt vpt_ def
/doclip {
  ClipToBoundingBox {
    newpath 0 0 moveto 360 0 lineto 360 378 lineto 0 378 lineto closepath
    clip
  } if
} def
%
% Gnuplot Prolog Version 5.1 (Oct 2015)
%
%/SuppressPDFMark true def
%
/M {moveto} bind def
/L {lineto} bind def
/R {rmoveto} bind def
/V {rlineto} bind def
/N {newpath moveto} bind def
/Z {closepath} bind def
/C {setrgbcolor} bind def
/f {rlineto fill} bind def
/g {setgray} bind def
/Gshow {show} def   % May be redefined later in the file to support UTF-8
/vpt2 vpt 2 mul def
/hpt2 hpt 2 mul def
/Lshow {currentpoint stroke M 0 vshift R 
	Blacktext {gsave 0 setgray textshow grestore} {textshow} ifelse} def
/Rshow {currentpoint stroke M dup stringwidth pop neg vshift R
	Blacktext {gsave 0 setgray textshow grestore} {textshow} ifelse} def
/Cshow {currentpoint stroke M dup stringwidth pop -2 div vshift R 
	Blacktext {gsave 0 setgray textshow grestore} {textshow} ifelse} def
/UP {dup vpt_ mul /vpt exch def hpt_ mul /hpt exch def
  /hpt2 hpt 2 mul def /vpt2 vpt 2 mul def} def
/DL {Color {setrgbcolor Solid {pop []} if 0 setdash}
 {pop pop pop 0 setgray Solid {pop []} if 0 setdash} ifelse} def
/BL {stroke userlinewidth 2 mul setlinewidth
	Rounded {1 setlinejoin 1 setlinecap} if} def
/AL {stroke userlinewidth 2 div setlinewidth
	Rounded {1 setlinejoin 1 setlinecap} if} def
/UL {dup gnulinewidth mul /userlinewidth exch def
	dup 1 lt {pop 1} if 10 mul /udl exch def} def
/PL {stroke userlinewidth setlinewidth
	Rounded {1 setlinejoin 1 setlinecap} if} def
3.8 setmiterlimit
% Classic Line colors (version 5.0)
/LCw {1 1 1} def
/LCb {0 0 0} def
/LCa {0 0 0} def
/LC0 {1 0 0} def
/LC1 {0 1 0} def
/LC2 {0 0 1} def
/LC3 {1 0 1} def
/LC4 {0 1 1} def
/LC5 {1 1 0} def
/LC6 {0 0 0} def
/LC7 {1 0.3 0} def
/LC8 {0.5 0.5 0.5} def
% Default dash patterns (version 5.0)
/LTB {BL [] LCb DL} def
/LTw {PL [] 1 setgray} def
/LTb {PL [] LCb DL} def
/LTa {AL [1 udl mul 2 udl mul] 0 setdash LCa setrgbcolor} def
/LT0 {PL [] LC0 DL} def
/LT1 {PL [2 dl1 3 dl2] LC1 DL} def
/LT2 {PL [1 dl1 1.5 dl2] LC2 DL} def
/LT3 {PL [6 dl1 2 dl2 1 dl1 2 dl2] LC3 DL} def
/LT4 {PL [1 dl1 2 dl2 6 dl1 2 dl2 1 dl1 2 dl2] LC4 DL} def
/LT5 {PL [4 dl1 2 dl2] LC5 DL} def
/LT6 {PL [1.5 dl1 1.5 dl2 1.5 dl1 1.5 dl2 1.5 dl1 6 dl2] LC6 DL} def
/LT7 {PL [3 dl1 3 dl2 1 dl1 3 dl2] LC7 DL} def
/LT8 {PL [2 dl1 2 dl2 2 dl1 6 dl2] LC8 DL} def
/SL {[] 0 setdash} def
/Pnt {stroke [] 0 setdash gsave 1 setlinecap M 0 0 V stroke grestore} def
/Dia {stroke [] 0 setdash 2 copy vpt add M
  hpt neg vpt neg V hpt vpt neg V
  hpt vpt V hpt neg vpt V closepath stroke
  Pnt} def
/Pls {stroke [] 0 setdash vpt sub M 0 vpt2 V
  currentpoint stroke M
  hpt neg vpt neg R hpt2 0 V stroke
 } def
/Box {stroke [] 0 setdash 2 copy exch hpt sub exch vpt add M
  0 vpt2 neg V hpt2 0 V 0 vpt2 V
  hpt2 neg 0 V closepath stroke
  Pnt} def
/Crs {stroke [] 0 setdash exch hpt sub exch vpt add M
  hpt2 vpt2 neg V currentpoint stroke M
  hpt2 neg 0 R hpt2 vpt2 V stroke} def
/TriU {stroke [] 0 setdash 2 copy vpt 1.12 mul add M
  hpt neg vpt -1.62 mul V
  hpt 2 mul 0 V
  hpt neg vpt 1.62 mul V closepath stroke
  Pnt} def
/Star {2 copy Pls Crs} def
/BoxF {stroke [] 0 setdash exch hpt sub exch vpt add M
  0 vpt2 neg V hpt2 0 V 0 vpt2 V
  hpt2 neg 0 V closepath fill} def
/TriUF {stroke [] 0 setdash vpt 1.12 mul add M
  hpt neg vpt -1.62 mul V
  hpt 2 mul 0 V
  hpt neg vpt 1.62 mul V closepath fill} def
/TriD {stroke [] 0 setdash 2 copy vpt 1.12 mul sub M
  hpt neg vpt 1.62 mul V
  hpt 2 mul 0 V
  hpt neg vpt -1.62 mul V closepath stroke
  Pnt} def
/TriDF {stroke [] 0 setdash vpt 1.12 mul sub M
  hpt neg vpt 1.62 mul V
  hpt 2 mul 0 V
  hpt neg vpt -1.62 mul V closepath fill} def
/DiaF {stroke [] 0 setdash vpt add M
  hpt neg vpt neg V hpt vpt neg V
  hpt vpt V hpt neg vpt V closepath fill} def
/Pent {stroke [] 0 setdash 2 copy gsave
  translate 0 hpt M 4 {72 rotate 0 hpt L} repeat
  closepath stroke grestore Pnt} def
/PentF {stroke [] 0 setdash gsave
  translate 0 hpt M 4 {72 rotate 0 hpt L} repeat
  closepath fill grestore} def
/Circle {stroke [] 0 setdash 2 copy
  hpt 0 360 arc stroke Pnt} def
/CircleF {stroke [] 0 setdash hpt 0 360 arc fill} def
/C0 {BL [] 0 setdash 2 copy moveto vpt 90 450 arc} bind def
/C1 {BL [] 0 setdash 2 copy moveto
	2 copy vpt 0 90 arc closepath fill
	vpt 0 360 arc closepath} bind def
/C2 {BL [] 0 setdash 2 copy moveto
	2 copy vpt 90 180 arc closepath fill
	vpt 0 360 arc closepath} bind def
/C3 {BL [] 0 setdash 2 copy moveto
	2 copy vpt 0 180 arc closepath fill
	vpt 0 360 arc closepath} bind def
/C4 {BL [] 0 setdash 2 copy moveto
	2 copy vpt 180 270 arc closepath fill
	vpt 0 360 arc closepath} bind def
/C5 {BL [] 0 setdash 2 copy moveto
	2 copy vpt 0 90 arc
	2 copy moveto
	2 copy vpt 180 270 arc closepath fill
	vpt 0 360 arc} bind def
/C6 {BL [] 0 setdash 2 copy moveto
	2 copy vpt 90 270 arc closepath fill
	vpt 0 360 arc closepath} bind def
/C7 {BL [] 0 setdash 2 copy moveto
	2 copy vpt 0 270 arc closepath fill
	vpt 0 360 arc closepath} bind def
/C8 {BL [] 0 setdash 2 copy moveto
	2 copy vpt 270 360 arc closepath fill
	vpt 0 360 arc closepath} bind def
/C9 {BL [] 0 setdash 2 copy moveto
	2 copy vpt 270 450 arc closepath fill
	vpt 0 360 arc closepath} bind def
/C10 {BL [] 0 setdash 2 copy 2 copy moveto vpt 270 360 arc closepath fill
	2 copy moveto
	2 copy vpt 90 180 arc closepath fill
	vpt 0 360 arc closepath} bind def
/C11 {BL [] 0 setdash 2 copy moveto
	2 copy vpt 0 180 arc closepath fill
	2 copy moveto
	2 copy vpt 270 360 arc closepath fill
	vpt 0 360 arc closepath} bind def
/C12 {BL [] 0 setdash 2 copy moveto
	2 copy vpt 180 360 arc closepath fill
	vpt 0 360 arc closepath} bind def
/C13 {BL [] 0 setdash 2 copy moveto
	2 copy vpt 0 90 arc closepath fill
	2 copy moveto
	2 copy vpt 180 360 arc closepath fill
	vpt 0 360 arc closepath} bind def
/C14 {BL [] 0 setdash 2 copy moveto
	2 copy vpt 90 360 arc closepath fill
	vpt 0 360 arc} bind def
/C15 {BL [] 0 setdash 2 copy vpt 0 360 arc closepath fill
	vpt 0 360 arc closepath} bind def
/Rec {newpath 4 2 roll moveto 1 index 0 rlineto 0 exch rlineto
	neg 0 rlineto closepath} bind def
/Square {dup Rec} bind def
/Bsquare {vpt sub exch vpt sub exch vpt2 Square} bind def
/S0 {BL [] 0 setdash 2 copy moveto 0 vpt rlineto BL Bsquare} bind def
/S1 {BL [] 0 setdash 2 copy vpt Square fill Bsquare} bind def
/S2 {BL [] 0 setdash 2 copy exch vpt sub exch vpt Square fill Bsquare} bind def
/S3 {BL [] 0 setdash 2 copy exch vpt sub exch vpt2 vpt Rec fill Bsquare} bind def
/S4 {BL [] 0 setdash 2 copy exch vpt sub exch vpt sub vpt Square fill Bsquare} bind def
/S5 {BL [] 0 setdash 2 copy 2 copy vpt Square fill
	exch vpt sub exch vpt sub vpt Square fill Bsquare} bind def
/S6 {BL [] 0 setdash 2 copy exch vpt sub exch vpt sub vpt vpt2 Rec fill Bsquare} bind def
/S7 {BL [] 0 setdash 2 copy exch vpt sub exch vpt sub vpt vpt2 Rec fill
	2 copy vpt Square fill Bsquare} bind def
/S8 {BL [] 0 setdash 2 copy vpt sub vpt Square fill Bsquare} bind def
/S9 {BL [] 0 setdash 2 copy vpt sub vpt vpt2 Rec fill Bsquare} bind def
/S10 {BL [] 0 setdash 2 copy vpt sub vpt Square fill 2 copy exch vpt sub exch vpt Square fill
	Bsquare} bind def
/S11 {BL [] 0 setdash 2 copy vpt sub vpt Square fill 2 copy exch vpt sub exch vpt2 vpt Rec fill
	Bsquare} bind def
/S12 {BL [] 0 setdash 2 copy exch vpt sub exch vpt sub vpt2 vpt Rec fill Bsquare} bind def
/S13 {BL [] 0 setdash 2 copy exch vpt sub exch vpt sub vpt2 vpt Rec fill
	2 copy vpt Square fill Bsquare} bind def
/S14 {BL [] 0 setdash 2 copy exch vpt sub exch vpt sub vpt2 vpt Rec fill
	2 copy exch vpt sub exch vpt Square fill Bsquare} bind def
/S15 {BL [] 0 setdash 2 copy Bsquare fill Bsquare} bind def
/D0 {gsave translate 45 rotate 0 0 S0 stroke grestore} bind def
/D1 {gsave translate 45 rotate 0 0 S1 stroke grestore} bind def
/D2 {gsave translate 45 rotate 0 0 S2 stroke grestore} bind def
/D3 {gsave translate 45 rotate 0 0 S3 stroke grestore} bind def
/D4 {gsave translate 45 rotate 0 0 S4 stroke grestore} bind def
/D5 {gsave translate 45 rotate 0 0 S5 stroke grestore} bind def
/D6 {gsave translate 45 rotate 0 0 S6 stroke grestore} bind def
/D7 {gsave translate 45 rotate 0 0 S7 stroke grestore} bind def
/D8 {gsave translate 45 rotate 0 0 S8 stroke grestore} bind def
/D9 {gsave translate 45 rotate 0 0 S9 stroke grestore} bind def
/D10 {gsave translate 45 rotate 0 0 S10 stroke grestore} bind def
/D11 {gsave translate 45 rotate 0 0 S11 stroke grestore} bind def
/D12 {gsave translate 45 rotate 0 0 S12 stroke grestore} bind def
/D13 {gsave translate 45 rotate 0 0 S13 stroke grestore} bind def
/D14 {gsave translate 45 rotate 0 0 S14 stroke grestore} bind def
/D15 {gsave translate 45 rotate 0 0 S15 stroke grestore} bind def
/DiaE {stroke [] 0 setdash vpt add M
  hpt neg vpt neg V hpt vpt neg V
  hpt vpt V hpt neg vpt V closepath stroke} def
/BoxE {stroke [] 0 setdash exch hpt sub exch vpt add M
  0 vpt2 neg V hpt2 0 V 0 vpt2 V
  hpt2 neg 0 V closepath stroke} def
/TriUE {stroke [] 0 setdash vpt 1.12 mul add M
  hpt neg vpt -1.62 mul V
  hpt 2 mul 0 V
  hpt neg vpt 1.62 mul V closepath stroke} def
/TriDE {stroke [] 0 setdash vpt 1.12 mul sub M
  hpt neg vpt 1.62 mul V
  hpt 2 mul 0 V
  hpt neg vpt -1.62 mul V closepath stroke} def
/PentE {stroke [] 0 setdash gsave
  translate 0 hpt M 4 {72 rotate 0 hpt L} repeat
  closepath stroke grestore} def
/CircE {stroke [] 0 setdash 
  hpt 0 360 arc stroke} def
/Opaque {gsave closepath 1 setgray fill grestore 0 setgray closepath} def
/DiaW {stroke [] 0 setdash vpt add M
  hpt neg vpt neg V hpt vpt neg V
  hpt vpt V hpt neg vpt V Opaque stroke} def
/BoxW {stroke [] 0 setdash exch hpt sub exch vpt add M
  0 vpt2 neg V hpt2 0 V 0 vpt2 V
  hpt2 neg 0 V Opaque stroke} def
/TriUW {stroke [] 0 setdash vpt 1.12 mul add M
  hpt neg vpt -1.62 mul V
  hpt 2 mul 0 V
  hpt neg vpt 1.62 mul V Opaque stroke} def
/TriDW {stroke [] 0 setdash vpt 1.12 mul sub M
  hpt neg vpt 1.62 mul V
  hpt 2 mul 0 V
  hpt neg vpt -1.62 mul V Opaque stroke} def
/PentW {stroke [] 0 setdash gsave
  translate 0 hpt M 4 {72 rotate 0 hpt L} repeat
  Opaque stroke grestore} def
/CircW {stroke [] 0 setdash 
  hpt 0 360 arc Opaque stroke} def
/BoxFill {gsave Rec 1 setgray fill grestore} def
/Density {
  /Fillden exch def
  currentrgbcolor
  /ColB exch def /ColG exch def /ColR exch def
  /ColR ColR Fillden mul Fillden sub 1 add def
  /ColG ColG Fillden mul Fillden sub 1 add def
  /ColB ColB Fillden mul Fillden sub 1 add def
  ColR ColG ColB setrgbcolor} def
/BoxColFill {gsave Rec PolyFill} def
/PolyFill {gsave Density fill grestore grestore} def
/h {rlineto rlineto rlineto gsave closepath fill grestore} bind def
%
% PostScript Level 1 Pattern Fill routine for rectangles
% Usage: x y w h s a XX PatternFill
%	x,y = lower left corner of box to be filled
%	w,h = width and height of box
%	  a = angle in degrees between lines and x-axis
%	 XX = 0/1 for no/yes cross-hatch
%
/PatternFill {gsave /PFa [ 9 2 roll ] def
  PFa 0 get PFa 2 get 2 div add PFa 1 get PFa 3 get 2 div add translate
  PFa 2 get -2 div PFa 3 get -2 div PFa 2 get PFa 3 get Rec
  TransparentPatterns {} {gsave 1 setgray fill grestore} ifelse
  clip
  currentlinewidth 0.5 mul setlinewidth
  /PFs PFa 2 get dup mul PFa 3 get dup mul add sqrt def
  0 0 M PFa 5 get rotate PFs -2 div dup translate
  0 1 PFs PFa 4 get div 1 add floor cvi
	{PFa 4 get mul 0 M 0 PFs V} for
  0 PFa 6 get ne {
	0 1 PFs PFa 4 get div 1 add floor cvi
	{PFa 4 get mul 0 2 1 roll M PFs 0 V} for
 } if
  stroke grestore} def
%
/languagelevel where
 {pop languagelevel} {1} ifelse
dup 2 lt
	{/InterpretLevel1 true def
	 /InterpretLevel3 false def}
	{/InterpretLevel1 Level1 def
	 2 gt
	    {/InterpretLevel3 Level3 def}
	    {/InterpretLevel3 false def}
	 ifelse }
 ifelse
%
% PostScript level 2 pattern fill definitions
%
/Level2PatternFill {
/Tile8x8 {/PaintType 2 /PatternType 1 /TilingType 1 /BBox [0 0 8 8] /XStep 8 /YStep 8}
	bind def
/KeepColor {currentrgbcolor [/Pattern /DeviceRGB] setcolorspace} bind def
<< Tile8x8
 /PaintProc {0.5 setlinewidth pop 0 0 M 8 8 L 0 8 M 8 0 L stroke} 
>> matrix makepattern
/Pat1 exch def
<< Tile8x8
 /PaintProc {0.5 setlinewidth pop 0 0 M 8 8 L 0 8 M 8 0 L stroke
	0 4 M 4 8 L 8 4 L 4 0 L 0 4 L stroke}
>> matrix makepattern
/Pat2 exch def
<< Tile8x8
 /PaintProc {0.5 setlinewidth pop 0 0 M 0 8 L
	8 8 L 8 0 L 0 0 L fill}
>> matrix makepattern
/Pat3 exch def
<< Tile8x8
 /PaintProc {0.5 setlinewidth pop -4 8 M 8 -4 L
	0 12 M 12 0 L stroke}
>> matrix makepattern
/Pat4 exch def
<< Tile8x8
 /PaintProc {0.5 setlinewidth pop -4 0 M 8 12 L
	0 -4 M 12 8 L stroke}
>> matrix makepattern
/Pat5 exch def
<< Tile8x8
 /PaintProc {0.5 setlinewidth pop -2 8 M 4 -4 L
	0 12 M 8 -4 L 4 12 M 10 0 L stroke}
>> matrix makepattern
/Pat6 exch def
<< Tile8x8
 /PaintProc {0.5 setlinewidth pop -2 0 M 4 12 L
	0 -4 M 8 12 L 4 -4 M 10 8 L stroke}
>> matrix makepattern
/Pat7 exch def
<< Tile8x8
 /PaintProc {0.5 setlinewidth pop 8 -2 M -4 4 L
	12 0 M -4 8 L 12 4 M 0 10 L stroke}
>> matrix makepattern
/Pat8 exch def
<< Tile8x8
 /PaintProc {0.5 setlinewidth pop 0 -2 M 12 4 L
	-4 0 M 12 8 L -4 4 M 8 10 L stroke}
>> matrix makepattern
/Pat9 exch def
/Pattern1 {PatternBgnd KeepColor Pat1 setpattern} bind def
/Pattern2 {PatternBgnd KeepColor Pat2 setpattern} bind def
/Pattern3 {PatternBgnd KeepColor Pat3 setpattern} bind def
/Pattern4 {PatternBgnd KeepColor Landscape {Pat5} {Pat4} ifelse setpattern} bind def
/Pattern5 {PatternBgnd KeepColor Landscape {Pat4} {Pat5} ifelse setpattern} bind def
/Pattern6 {PatternBgnd KeepColor Landscape {Pat9} {Pat6} ifelse setpattern} bind def
/Pattern7 {PatternBgnd KeepColor Landscape {Pat8} {Pat7} ifelse setpattern} bind def
} def
%
%
%End of PostScript Level 2 code
%
/PatternBgnd {
  TransparentPatterns {} {gsave 1 setgray fill grestore} ifelse
} def
%
% Substitute for Level 2 pattern fill codes with
% grayscale if Level 2 support is not selected.
%
/Level1PatternFill {
/Pattern1 {0.250 Density} bind def
/Pattern2 {0.500 Density} bind def
/Pattern3 {0.750 Density} bind def
/Pattern4 {0.125 Density} bind def
/Pattern5 {0.375 Density} bind def
/Pattern6 {0.625 Density} bind def
/Pattern7 {0.875 Density} bind def
} def
%
% Now test for support of Level 2 code
%
Level1 {Level1PatternFill} {Level2PatternFill} ifelse
%
/Symbol-Oblique /Symbol findfont [1 0 .167 1 0 0] makefont
dup length dict begin {1 index /FID eq {pop pop} {def} ifelse} forall
currentdict end definefont pop
%
Level1 SuppressPDFMark or 
{} {
/SDict 10 dict def
systemdict /pdfmark known not {
  userdict /pdfmark systemdict /cleartomark get put
} if
SDict begin [
  /Title (plot_MeffGGA1++l26n_SU3.tex)
  /Subject (gnuplot plot)
  /Creator (gnuplot 5.0 patchlevel 3)
  /Author (mteper)
%  /Producer (gnuplot)
%  /Keywords ()
  /CreationDate (Fri Feb  5 21:12:30 2021)
  /DOCINFO pdfmark
end
} ifelse
%
% Support for boxed text - Ethan A Merritt May 2005
%
/InitTextBox { userdict /TBy2 3 -1 roll put userdict /TBx2 3 -1 roll put
           userdict /TBy1 3 -1 roll put userdict /TBx1 3 -1 roll put
	   /Boxing true def } def
/ExtendTextBox { Boxing
    { gsave dup false charpath pathbbox
      dup TBy2 gt {userdict /TBy2 3 -1 roll put} {pop} ifelse
      dup TBx2 gt {userdict /TBx2 3 -1 roll put} {pop} ifelse
      dup TBy1 lt {userdict /TBy1 3 -1 roll put} {pop} ifelse
      dup TBx1 lt {userdict /TBx1 3 -1 roll put} {pop} ifelse
      grestore } if } def
/PopTextBox { newpath TBx1 TBxmargin sub TBy1 TBymargin sub M
               TBx1 TBxmargin sub TBy2 TBymargin add L
	       TBx2 TBxmargin add TBy2 TBymargin add L
	       TBx2 TBxmargin add TBy1 TBymargin sub L closepath } def
/DrawTextBox { PopTextBox stroke /Boxing false def} def
/FillTextBox { gsave PopTextBox 1 1 1 setrgbcolor fill grestore /Boxing false def} def
0 0 0 0 InitTextBox
/TBxmargin 20 def
/TBymargin 20 def
/Boxing false def
/textshow { ExtendTextBox Gshow } def
%
% redundant definitions for compatibility with prologue.ps older than 5.0.2
/LTB {BL [] LCb DL} def
/LTb {PL [] LCb DL} def
end
%%EndProlog
%%Page: 1 1
gnudict begin
gsave
doclip
0 0 translate
0.050 0.050 scale
0 setgray
newpath
BackgroundColor 0 lt 3 1 roll 0 lt exch 0 lt or or not {BackgroundColor C 1.000 0 0 7200.00 7560.00 BoxColFill} if
1.000 UL
LTb
LCb setrgbcolor
1340 640 M
63 0 V
5436 0 R
-63 0 V
stroke
LTb
LCb setrgbcolor
1340 1308 M
63 0 V
5436 0 R
-63 0 V
stroke
LTb
LCb setrgbcolor
1340 1976 M
63 0 V
5436 0 R
-63 0 V
stroke
LTb
LCb setrgbcolor
1340 2644 M
63 0 V
5436 0 R
-63 0 V
stroke
LTb
LCb setrgbcolor
1340 3312 M
63 0 V
5436 0 R
-63 0 V
stroke
LTb
LCb setrgbcolor
1340 3980 M
63 0 V
5436 0 R
-63 0 V
stroke
LTb
LCb setrgbcolor
1340 4647 M
63 0 V
5436 0 R
-63 0 V
stroke
LTb
LCb setrgbcolor
1340 5315 M
63 0 V
5436 0 R
-63 0 V
stroke
LTb
LCb setrgbcolor
1340 5983 M
63 0 V
5436 0 R
-63 0 V
stroke
LTb
LCb setrgbcolor
1340 6651 M
63 0 V
5436 0 R
-63 0 V
stroke
LTb
LCb setrgbcolor
1340 7319 M
63 0 V
5436 0 R
-63 0 V
stroke
LTb
LCb setrgbcolor
1340 640 M
0 63 V
0 6616 R
0 -63 V
stroke
LTb
LCb setrgbcolor
2440 640 M
0 63 V
0 6616 R
0 -63 V
stroke
LTb
LCb setrgbcolor
3540 640 M
0 63 V
0 6616 R
0 -63 V
stroke
LTb
LCb setrgbcolor
4639 640 M
0 63 V
0 6616 R
0 -63 V
stroke
LTb
LCb setrgbcolor
5739 640 M
0 63 V
0 6616 R
0 -63 V
stroke
LTb
LCb setrgbcolor
6839 640 M
0 63 V
0 6616 R
0 -63 V
stroke
LTb
LCb setrgbcolor
1.000 UL
LTb
LCb setrgbcolor
1340 7319 N
0 -6679 V
5499 0 V
0 6679 V
-5499 0 V
Z stroke
1.000 UP
1.000 UL
LTb
LCb setrgbcolor
LCb setrgbcolor
LTb
LCb setrgbcolor
LTb
1.500 UP
1.000 UL
LTb
0.58 0.00 0.83 C 1890 4100 M
0 31 V
2990 3899 M
0 55 V
4090 3750 M
0 116 V
5189 3559 M
0 287 V
6289 3361 M
0 643 V
1890 4115 CircleF
2990 3927 CircleF
4090 3808 CircleF
5189 3702 CircleF
6289 3682 CircleF
1.500 UP
1.000 UL
LTb
0.58 0.00 0.83 C 1846 5169 M
0 36 V
2946 4503 M
0 88 V
4046 3810 M
0 165 V
5145 3021 M
0 365 V
6245 2179 M
0 645 V
1846 5187 Circle
2946 4547 Circle
4046 3892 Circle
5145 3204 Circle
6245 2502 Circle
1.500 UP
1.000 UL
LTb
0.58 0.00 0.83 C 1934 6705 M
0 45 V
3034 6173 M
0 170 V
4133 6109 M
0 1172 V
1934 6727 BoxF
3034 6258 BoxF
4133 6695 BoxF
1.500 UL
LTb
0.58 0.00 0.83 C 1340 2283 M
56 0 V
55 0 V
56 0 V
55 0 V
56 0 V
55 0 V
56 0 V
55 0 V
56 0 V
55 0 V
56 0 V
56 0 V
55 0 V
56 0 V
55 0 V
56 0 V
55 0 V
56 0 V
55 0 V
56 0 V
55 0 V
56 0 V
56 0 V
55 0 V
56 0 V
55 0 V
56 0 V
55 0 V
56 0 V
55 0 V
56 0 V
55 0 V
56 0 V
56 0 V
55 0 V
56 0 V
55 0 V
56 0 V
55 0 V
56 0 V
55 0 V
56 0 V
55 0 V
56 0 V
56 0 V
55 0 V
56 0 V
55 0 V
56 0 V
55 0 V
56 0 V
55 0 V
56 0 V
55 0 V
56 0 V
56 0 V
55 0 V
56 0 V
55 0 V
56 0 V
55 0 V
56 0 V
55 0 V
56 0 V
55 0 V
56 0 V
56 0 V
55 0 V
56 0 V
55 0 V
56 0 V
55 0 V
56 0 V
55 0 V
56 0 V
55 0 V
56 0 V
56 0 V
55 0 V
56 0 V
55 0 V
56 0 V
55 0 V
56 0 V
55 0 V
56 0 V
55 0 V
56 0 V
56 0 V
55 0 V
56 0 V
55 0 V
56 0 V
55 0 V
56 0 V
55 0 V
56 0 V
55 0 V
56 0 V
stroke
LTb
0.58 0.00 0.83 C 1340 3926 M
56 0 V
55 0 V
56 0 V
55 0 V
56 0 V
55 0 V
56 0 V
55 0 V
56 0 V
55 0 V
56 0 V
56 0 V
55 0 V
56 0 V
55 0 V
56 0 V
55 0 V
56 0 V
55 0 V
56 0 V
55 0 V
56 0 V
56 0 V
55 0 V
56 0 V
55 0 V
56 0 V
55 0 V
56 0 V
55 0 V
56 0 V
55 0 V
56 0 V
56 0 V
55 0 V
56 0 V
55 0 V
56 0 V
55 0 V
56 0 V
55 0 V
56 0 V
55 0 V
56 0 V
56 0 V
55 0 V
56 0 V
55 0 V
56 0 V
55 0 V
56 0 V
55 0 V
56 0 V
55 0 V
56 0 V
56 0 V
55 0 V
56 0 V
55 0 V
56 0 V
55 0 V
56 0 V
55 0 V
56 0 V
55 0 V
56 0 V
56 0 V
55 0 V
56 0 V
55 0 V
56 0 V
55 0 V
56 0 V
55 0 V
56 0 V
55 0 V
56 0 V
56 0 V
55 0 V
56 0 V
55 0 V
56 0 V
55 0 V
56 0 V
55 0 V
56 0 V
55 0 V
56 0 V
56 0 V
55 0 V
56 0 V
55 0 V
56 0 V
55 0 V
56 0 V
55 0 V
56 0 V
55 0 V
56 0 V
stroke
2.000 UL
LTb
LCb setrgbcolor
1.000 UL
LTb
LCb setrgbcolor
1340 7319 N
0 -6679 V
5499 0 V
0 6679 V
-5499 0 V
Z stroke
1.000 UP
1.000 UL
LTb
LCb setrgbcolor
stroke
grestore
end
showpage
  }}%
  \put(4089,140){\makebox(0,0){\large{$t/a$}}}%
  \put(160,5179){\makebox(0,0){\Large{$aE_{eff}(t)$}}}%
  \put(6839,440){\makebox(0,0){\strut{}\ {$5$}}}%
  \put(5739,440){\makebox(0,0){\strut{}\ {$4$}}}%
  \put(4639,440){\makebox(0,0){\strut{}\ {$3$}}}%
  \put(3540,440){\makebox(0,0){\strut{}\ {$2$}}}%
  \put(2440,440){\makebox(0,0){\strut{}\ {$1$}}}%
  \put(1340,440){\makebox(0,0){\strut{}\ {$0$}}}%
  \put(1220,7319){\makebox(0,0)[r]{\strut{}\ \ {$2$}}}%
  \put(1220,6651){\makebox(0,0)[r]{\strut{}\ \ {$1.8$}}}%
  \put(1220,5983){\makebox(0,0)[r]{\strut{}\ \ {$1.6$}}}%
  \put(1220,5315){\makebox(0,0)[r]{\strut{}\ \ {$1.4$}}}%
  \put(1220,4647){\makebox(0,0)[r]{\strut{}\ \ {$1.2$}}}%
  \put(1220,3980){\makebox(0,0)[r]{\strut{}\ \ {$1$}}}%
  \put(1220,3312){\makebox(0,0)[r]{\strut{}\ \ {$0.8$}}}%
  \put(1220,2644){\makebox(0,0)[r]{\strut{}\ \ {$0.6$}}}%
  \put(1220,1976){\makebox(0,0)[r]{\strut{}\ \ {$0.4$}}}%
  \put(1220,1308){\makebox(0,0)[r]{\strut{}\ \ {$0.2$}}}%
  \put(1220,640){\makebox(0,0)[r]{\strut{}\ \ {$0$}}}%
\end{picture}%
\endgroup
\endinput

\end	{center}
\caption{Effective masses for the lightest three states
  in the `scalar' $A_1^{++}$ representation, for the double trace operators.
  Lower horizontal line indicates the mass of the lightest glueball, and
  upper  horizontal line indicates twice the mass of the lightest glueball.
  On the $26^326$ lattice at $\beta=6.235$ in SU(3).}
\label{fig_MeffGGA1++l26n_SU3}
\end{figure}





%\begin{figure}[htb]
%\begin	{center}
%\leavevmode
%\input	{plot_MeffG+GGT2++l26n_SU3.tex}
%\end	{center}
%\caption{Effective masses for the lightest few glueballs in the 'tensor' $T_2^{++}$
%  representation, for the single trace operators ($\circ$) and for the same set
%  augmented with double trace operators (filled points), with points shifted for clarity.
%  The extra `scattering' state amongst the latter is shown as $\blacklozenge$. 
%  Horizontal line indicates  the sum of the masses of the lightest
%  $A_1^{++}$ and $T_2^{++}$ glueballs.
%  On the $26^326$ lattice at $\beta=6.235$ in SU(3).}
%\label{fig_MeffG+GGT2++l26n_SU3}
%\end{figure}


%\begin{figure}[htb]
%\begin	{center}
%\leavevmode
%\input	{plot_MeffGGT2++l26n_SU3.tex}
%\end	{center}
%\caption{Effective masses for the lightest three states
%  in the `tensor' $T_2^{++}$ representation using the double trace operators.
%  Lower horizontal line indicates the mass of the lightest $T_2^{++}$ glueball, and
%  upper  horizontal line indicates the sum of the masses of the lightest
%  $A_1^{++}$ and $T_2^{++}$ glueballs.
%  On the $26^326$ lattice at $\beta=6.235$ in SU(3).}
%\label{fig_MeffGGT2++l26n_SU3}
%\end{figure}




%\begin{figure}[htb]
%\begin	{center}
%\leavevmode
%\input	{plot_MeffG+GGT1+-l26_SU3.tex}
%\end	{center}
%\caption{Effective masses for the lightest few glueballs in the $C=-$ 'vector' $T_1^{+-}$
%  representation, for the single trace operators ($\circ$) and for the same set
%  augmented with double trace operators (filled points), with points shifted for clarity.
%  The extra `scattering' state amongst the latter is shown as $\blacklozenge$. 
%  Horizontal line indicates  the sum of the masses of the lightest
%  $A_1^{++}$ and $T_1^{+-}$ glueballs.
%  On the $26^326$ lattice at $\beta=6.235$ in SU(3).}
%\label{fig_MeffG+GGT1+-l26_SU3}
%\end{figure}


%\begin{figure}[htb]
%\begin	{center}
%\leavevmode
%\input	{plot_MeffGGT1+-l26_SU3.tex}
%\end	{center}
%\caption{Effective masses for the lightest three states
%  in the $C=-$ `vector' $T_1^{+-}$ representation, for the double trace operators.
%  Lower horizontal line indicates the mass of the lightest $T_1^{+-}$ glueball, and
%  upper  horizontal line indicates the sum of the masses of the lightest
%  $A_1^{++}$ and $T_1^{+-}$ glueballs.
%%  On the $26^326$ lattice at $\beta=6.235$ in SU(3).}
%\label{fig_MeffGGT1+-l26_SU3}
%\end{figure}



\clearpage


\begin{figure}[htb]
\begin	{center}
\leavevmode
% GNUPLOT: LaTeX picture with Postscript
\begingroup%
\makeatletter%
\newcommand{\GNUPLOTspecial}{%
  \@sanitize\catcode`\%=14\relax\special}%
\setlength{\unitlength}{0.0500bp}%
\begin{picture}(7200,7560)(0,0)%
  {\GNUPLOTspecial{"
%!PS-Adobe-2.0 EPSF-2.0
%%Title: plot_MeffG+GGA1-+l26n_SU3.tex
%%Creator: gnuplot 5.0 patchlevel 3
%%CreationDate: Sun Feb  7 16:36:24 2021
%%DocumentFonts: 
%%BoundingBox: 0 0 360 378
%%EndComments
%%BeginProlog
/gnudict 256 dict def
gnudict begin
%
% The following true/false flags may be edited by hand if desired.
% The unit line width and grayscale image gamma correction may also be changed.
%
/Color true def
/Blacktext true def
/Solid false def
/Dashlength 1 def
/Landscape false def
/Level1 false def
/Level3 false def
/Rounded false def
/ClipToBoundingBox false def
/SuppressPDFMark false def
/TransparentPatterns false def
/gnulinewidth 5.000 def
/userlinewidth gnulinewidth def
/Gamma 1.0 def
/BackgroundColor {-1.000 -1.000 -1.000} def
%
/vshift -66 def
/dl1 {
  10.0 Dashlength userlinewidth gnulinewidth div mul mul mul
  Rounded { currentlinewidth 0.75 mul sub dup 0 le { pop 0.01 } if } if
} def
/dl2 {
  10.0 Dashlength userlinewidth gnulinewidth div mul mul mul
  Rounded { currentlinewidth 0.75 mul add } if
} def
/hpt_ 31.5 def
/vpt_ 31.5 def
/hpt hpt_ def
/vpt vpt_ def
/doclip {
  ClipToBoundingBox {
    newpath 0 0 moveto 360 0 lineto 360 378 lineto 0 378 lineto closepath
    clip
  } if
} def
%
% Gnuplot Prolog Version 5.1 (Oct 2015)
%
%/SuppressPDFMark true def
%
/M {moveto} bind def
/L {lineto} bind def
/R {rmoveto} bind def
/V {rlineto} bind def
/N {newpath moveto} bind def
/Z {closepath} bind def
/C {setrgbcolor} bind def
/f {rlineto fill} bind def
/g {setgray} bind def
/Gshow {show} def   % May be redefined later in the file to support UTF-8
/vpt2 vpt 2 mul def
/hpt2 hpt 2 mul def
/Lshow {currentpoint stroke M 0 vshift R 
	Blacktext {gsave 0 setgray textshow grestore} {textshow} ifelse} def
/Rshow {currentpoint stroke M dup stringwidth pop neg vshift R
	Blacktext {gsave 0 setgray textshow grestore} {textshow} ifelse} def
/Cshow {currentpoint stroke M dup stringwidth pop -2 div vshift R 
	Blacktext {gsave 0 setgray textshow grestore} {textshow} ifelse} def
/UP {dup vpt_ mul /vpt exch def hpt_ mul /hpt exch def
  /hpt2 hpt 2 mul def /vpt2 vpt 2 mul def} def
/DL {Color {setrgbcolor Solid {pop []} if 0 setdash}
 {pop pop pop 0 setgray Solid {pop []} if 0 setdash} ifelse} def
/BL {stroke userlinewidth 2 mul setlinewidth
	Rounded {1 setlinejoin 1 setlinecap} if} def
/AL {stroke userlinewidth 2 div setlinewidth
	Rounded {1 setlinejoin 1 setlinecap} if} def
/UL {dup gnulinewidth mul /userlinewidth exch def
	dup 1 lt {pop 1} if 10 mul /udl exch def} def
/PL {stroke userlinewidth setlinewidth
	Rounded {1 setlinejoin 1 setlinecap} if} def
3.8 setmiterlimit
% Classic Line colors (version 5.0)
/LCw {1 1 1} def
/LCb {0 0 0} def
/LCa {0 0 0} def
/LC0 {1 0 0} def
/LC1 {0 1 0} def
/LC2 {0 0 1} def
/LC3 {1 0 1} def
/LC4 {0 1 1} def
/LC5 {1 1 0} def
/LC6 {0 0 0} def
/LC7 {1 0.3 0} def
/LC8 {0.5 0.5 0.5} def
% Default dash patterns (version 5.0)
/LTB {BL [] LCb DL} def
/LTw {PL [] 1 setgray} def
/LTb {PL [] LCb DL} def
/LTa {AL [1 udl mul 2 udl mul] 0 setdash LCa setrgbcolor} def
/LT0 {PL [] LC0 DL} def
/LT1 {PL [2 dl1 3 dl2] LC1 DL} def
/LT2 {PL [1 dl1 1.5 dl2] LC2 DL} def
/LT3 {PL [6 dl1 2 dl2 1 dl1 2 dl2] LC3 DL} def
/LT4 {PL [1 dl1 2 dl2 6 dl1 2 dl2 1 dl1 2 dl2] LC4 DL} def
/LT5 {PL [4 dl1 2 dl2] LC5 DL} def
/LT6 {PL [1.5 dl1 1.5 dl2 1.5 dl1 1.5 dl2 1.5 dl1 6 dl2] LC6 DL} def
/LT7 {PL [3 dl1 3 dl2 1 dl1 3 dl2] LC7 DL} def
/LT8 {PL [2 dl1 2 dl2 2 dl1 6 dl2] LC8 DL} def
/SL {[] 0 setdash} def
/Pnt {stroke [] 0 setdash gsave 1 setlinecap M 0 0 V stroke grestore} def
/Dia {stroke [] 0 setdash 2 copy vpt add M
  hpt neg vpt neg V hpt vpt neg V
  hpt vpt V hpt neg vpt V closepath stroke
  Pnt} def
/Pls {stroke [] 0 setdash vpt sub M 0 vpt2 V
  currentpoint stroke M
  hpt neg vpt neg R hpt2 0 V stroke
 } def
/Box {stroke [] 0 setdash 2 copy exch hpt sub exch vpt add M
  0 vpt2 neg V hpt2 0 V 0 vpt2 V
  hpt2 neg 0 V closepath stroke
  Pnt} def
/Crs {stroke [] 0 setdash exch hpt sub exch vpt add M
  hpt2 vpt2 neg V currentpoint stroke M
  hpt2 neg 0 R hpt2 vpt2 V stroke} def
/TriU {stroke [] 0 setdash 2 copy vpt 1.12 mul add M
  hpt neg vpt -1.62 mul V
  hpt 2 mul 0 V
  hpt neg vpt 1.62 mul V closepath stroke
  Pnt} def
/Star {2 copy Pls Crs} def
/BoxF {stroke [] 0 setdash exch hpt sub exch vpt add M
  0 vpt2 neg V hpt2 0 V 0 vpt2 V
  hpt2 neg 0 V closepath fill} def
/TriUF {stroke [] 0 setdash vpt 1.12 mul add M
  hpt neg vpt -1.62 mul V
  hpt 2 mul 0 V
  hpt neg vpt 1.62 mul V closepath fill} def
/TriD {stroke [] 0 setdash 2 copy vpt 1.12 mul sub M
  hpt neg vpt 1.62 mul V
  hpt 2 mul 0 V
  hpt neg vpt -1.62 mul V closepath stroke
  Pnt} def
/TriDF {stroke [] 0 setdash vpt 1.12 mul sub M
  hpt neg vpt 1.62 mul V
  hpt 2 mul 0 V
  hpt neg vpt -1.62 mul V closepath fill} def
/DiaF {stroke [] 0 setdash vpt add M
  hpt neg vpt neg V hpt vpt neg V
  hpt vpt V hpt neg vpt V closepath fill} def
/Pent {stroke [] 0 setdash 2 copy gsave
  translate 0 hpt M 4 {72 rotate 0 hpt L} repeat
  closepath stroke grestore Pnt} def
/PentF {stroke [] 0 setdash gsave
  translate 0 hpt M 4 {72 rotate 0 hpt L} repeat
  closepath fill grestore} def
/Circle {stroke [] 0 setdash 2 copy
  hpt 0 360 arc stroke Pnt} def
/CircleF {stroke [] 0 setdash hpt 0 360 arc fill} def
/C0 {BL [] 0 setdash 2 copy moveto vpt 90 450 arc} bind def
/C1 {BL [] 0 setdash 2 copy moveto
	2 copy vpt 0 90 arc closepath fill
	vpt 0 360 arc closepath} bind def
/C2 {BL [] 0 setdash 2 copy moveto
	2 copy vpt 90 180 arc closepath fill
	vpt 0 360 arc closepath} bind def
/C3 {BL [] 0 setdash 2 copy moveto
	2 copy vpt 0 180 arc closepath fill
	vpt 0 360 arc closepath} bind def
/C4 {BL [] 0 setdash 2 copy moveto
	2 copy vpt 180 270 arc closepath fill
	vpt 0 360 arc closepath} bind def
/C5 {BL [] 0 setdash 2 copy moveto
	2 copy vpt 0 90 arc
	2 copy moveto
	2 copy vpt 180 270 arc closepath fill
	vpt 0 360 arc} bind def
/C6 {BL [] 0 setdash 2 copy moveto
	2 copy vpt 90 270 arc closepath fill
	vpt 0 360 arc closepath} bind def
/C7 {BL [] 0 setdash 2 copy moveto
	2 copy vpt 0 270 arc closepath fill
	vpt 0 360 arc closepath} bind def
/C8 {BL [] 0 setdash 2 copy moveto
	2 copy vpt 270 360 arc closepath fill
	vpt 0 360 arc closepath} bind def
/C9 {BL [] 0 setdash 2 copy moveto
	2 copy vpt 270 450 arc closepath fill
	vpt 0 360 arc closepath} bind def
/C10 {BL [] 0 setdash 2 copy 2 copy moveto vpt 270 360 arc closepath fill
	2 copy moveto
	2 copy vpt 90 180 arc closepath fill
	vpt 0 360 arc closepath} bind def
/C11 {BL [] 0 setdash 2 copy moveto
	2 copy vpt 0 180 arc closepath fill
	2 copy moveto
	2 copy vpt 270 360 arc closepath fill
	vpt 0 360 arc closepath} bind def
/C12 {BL [] 0 setdash 2 copy moveto
	2 copy vpt 180 360 arc closepath fill
	vpt 0 360 arc closepath} bind def
/C13 {BL [] 0 setdash 2 copy moveto
	2 copy vpt 0 90 arc closepath fill
	2 copy moveto
	2 copy vpt 180 360 arc closepath fill
	vpt 0 360 arc closepath} bind def
/C14 {BL [] 0 setdash 2 copy moveto
	2 copy vpt 90 360 arc closepath fill
	vpt 0 360 arc} bind def
/C15 {BL [] 0 setdash 2 copy vpt 0 360 arc closepath fill
	vpt 0 360 arc closepath} bind def
/Rec {newpath 4 2 roll moveto 1 index 0 rlineto 0 exch rlineto
	neg 0 rlineto closepath} bind def
/Square {dup Rec} bind def
/Bsquare {vpt sub exch vpt sub exch vpt2 Square} bind def
/S0 {BL [] 0 setdash 2 copy moveto 0 vpt rlineto BL Bsquare} bind def
/S1 {BL [] 0 setdash 2 copy vpt Square fill Bsquare} bind def
/S2 {BL [] 0 setdash 2 copy exch vpt sub exch vpt Square fill Bsquare} bind def
/S3 {BL [] 0 setdash 2 copy exch vpt sub exch vpt2 vpt Rec fill Bsquare} bind def
/S4 {BL [] 0 setdash 2 copy exch vpt sub exch vpt sub vpt Square fill Bsquare} bind def
/S5 {BL [] 0 setdash 2 copy 2 copy vpt Square fill
	exch vpt sub exch vpt sub vpt Square fill Bsquare} bind def
/S6 {BL [] 0 setdash 2 copy exch vpt sub exch vpt sub vpt vpt2 Rec fill Bsquare} bind def
/S7 {BL [] 0 setdash 2 copy exch vpt sub exch vpt sub vpt vpt2 Rec fill
	2 copy vpt Square fill Bsquare} bind def
/S8 {BL [] 0 setdash 2 copy vpt sub vpt Square fill Bsquare} bind def
/S9 {BL [] 0 setdash 2 copy vpt sub vpt vpt2 Rec fill Bsquare} bind def
/S10 {BL [] 0 setdash 2 copy vpt sub vpt Square fill 2 copy exch vpt sub exch vpt Square fill
	Bsquare} bind def
/S11 {BL [] 0 setdash 2 copy vpt sub vpt Square fill 2 copy exch vpt sub exch vpt2 vpt Rec fill
	Bsquare} bind def
/S12 {BL [] 0 setdash 2 copy exch vpt sub exch vpt sub vpt2 vpt Rec fill Bsquare} bind def
/S13 {BL [] 0 setdash 2 copy exch vpt sub exch vpt sub vpt2 vpt Rec fill
	2 copy vpt Square fill Bsquare} bind def
/S14 {BL [] 0 setdash 2 copy exch vpt sub exch vpt sub vpt2 vpt Rec fill
	2 copy exch vpt sub exch vpt Square fill Bsquare} bind def
/S15 {BL [] 0 setdash 2 copy Bsquare fill Bsquare} bind def
/D0 {gsave translate 45 rotate 0 0 S0 stroke grestore} bind def
/D1 {gsave translate 45 rotate 0 0 S1 stroke grestore} bind def
/D2 {gsave translate 45 rotate 0 0 S2 stroke grestore} bind def
/D3 {gsave translate 45 rotate 0 0 S3 stroke grestore} bind def
/D4 {gsave translate 45 rotate 0 0 S4 stroke grestore} bind def
/D5 {gsave translate 45 rotate 0 0 S5 stroke grestore} bind def
/D6 {gsave translate 45 rotate 0 0 S6 stroke grestore} bind def
/D7 {gsave translate 45 rotate 0 0 S7 stroke grestore} bind def
/D8 {gsave translate 45 rotate 0 0 S8 stroke grestore} bind def
/D9 {gsave translate 45 rotate 0 0 S9 stroke grestore} bind def
/D10 {gsave translate 45 rotate 0 0 S10 stroke grestore} bind def
/D11 {gsave translate 45 rotate 0 0 S11 stroke grestore} bind def
/D12 {gsave translate 45 rotate 0 0 S12 stroke grestore} bind def
/D13 {gsave translate 45 rotate 0 0 S13 stroke grestore} bind def
/D14 {gsave translate 45 rotate 0 0 S14 stroke grestore} bind def
/D15 {gsave translate 45 rotate 0 0 S15 stroke grestore} bind def
/DiaE {stroke [] 0 setdash vpt add M
  hpt neg vpt neg V hpt vpt neg V
  hpt vpt V hpt neg vpt V closepath stroke} def
/BoxE {stroke [] 0 setdash exch hpt sub exch vpt add M
  0 vpt2 neg V hpt2 0 V 0 vpt2 V
  hpt2 neg 0 V closepath stroke} def
/TriUE {stroke [] 0 setdash vpt 1.12 mul add M
  hpt neg vpt -1.62 mul V
  hpt 2 mul 0 V
  hpt neg vpt 1.62 mul V closepath stroke} def
/TriDE {stroke [] 0 setdash vpt 1.12 mul sub M
  hpt neg vpt 1.62 mul V
  hpt 2 mul 0 V
  hpt neg vpt -1.62 mul V closepath stroke} def
/PentE {stroke [] 0 setdash gsave
  translate 0 hpt M 4 {72 rotate 0 hpt L} repeat
  closepath stroke grestore} def
/CircE {stroke [] 0 setdash 
  hpt 0 360 arc stroke} def
/Opaque {gsave closepath 1 setgray fill grestore 0 setgray closepath} def
/DiaW {stroke [] 0 setdash vpt add M
  hpt neg vpt neg V hpt vpt neg V
  hpt vpt V hpt neg vpt V Opaque stroke} def
/BoxW {stroke [] 0 setdash exch hpt sub exch vpt add M
  0 vpt2 neg V hpt2 0 V 0 vpt2 V
  hpt2 neg 0 V Opaque stroke} def
/TriUW {stroke [] 0 setdash vpt 1.12 mul add M
  hpt neg vpt -1.62 mul V
  hpt 2 mul 0 V
  hpt neg vpt 1.62 mul V Opaque stroke} def
/TriDW {stroke [] 0 setdash vpt 1.12 mul sub M
  hpt neg vpt 1.62 mul V
  hpt 2 mul 0 V
  hpt neg vpt -1.62 mul V Opaque stroke} def
/PentW {stroke [] 0 setdash gsave
  translate 0 hpt M 4 {72 rotate 0 hpt L} repeat
  Opaque stroke grestore} def
/CircW {stroke [] 0 setdash 
  hpt 0 360 arc Opaque stroke} def
/BoxFill {gsave Rec 1 setgray fill grestore} def
/Density {
  /Fillden exch def
  currentrgbcolor
  /ColB exch def /ColG exch def /ColR exch def
  /ColR ColR Fillden mul Fillden sub 1 add def
  /ColG ColG Fillden mul Fillden sub 1 add def
  /ColB ColB Fillden mul Fillden sub 1 add def
  ColR ColG ColB setrgbcolor} def
/BoxColFill {gsave Rec PolyFill} def
/PolyFill {gsave Density fill grestore grestore} def
/h {rlineto rlineto rlineto gsave closepath fill grestore} bind def
%
% PostScript Level 1 Pattern Fill routine for rectangles
% Usage: x y w h s a XX PatternFill
%	x,y = lower left corner of box to be filled
%	w,h = width and height of box
%	  a = angle in degrees between lines and x-axis
%	 XX = 0/1 for no/yes cross-hatch
%
/PatternFill {gsave /PFa [ 9 2 roll ] def
  PFa 0 get PFa 2 get 2 div add PFa 1 get PFa 3 get 2 div add translate
  PFa 2 get -2 div PFa 3 get -2 div PFa 2 get PFa 3 get Rec
  TransparentPatterns {} {gsave 1 setgray fill grestore} ifelse
  clip
  currentlinewidth 0.5 mul setlinewidth
  /PFs PFa 2 get dup mul PFa 3 get dup mul add sqrt def
  0 0 M PFa 5 get rotate PFs -2 div dup translate
  0 1 PFs PFa 4 get div 1 add floor cvi
	{PFa 4 get mul 0 M 0 PFs V} for
  0 PFa 6 get ne {
	0 1 PFs PFa 4 get div 1 add floor cvi
	{PFa 4 get mul 0 2 1 roll M PFs 0 V} for
 } if
  stroke grestore} def
%
/languagelevel where
 {pop languagelevel} {1} ifelse
dup 2 lt
	{/InterpretLevel1 true def
	 /InterpretLevel3 false def}
	{/InterpretLevel1 Level1 def
	 2 gt
	    {/InterpretLevel3 Level3 def}
	    {/InterpretLevel3 false def}
	 ifelse }
 ifelse
%
% PostScript level 2 pattern fill definitions
%
/Level2PatternFill {
/Tile8x8 {/PaintType 2 /PatternType 1 /TilingType 1 /BBox [0 0 8 8] /XStep 8 /YStep 8}
	bind def
/KeepColor {currentrgbcolor [/Pattern /DeviceRGB] setcolorspace} bind def
<< Tile8x8
 /PaintProc {0.5 setlinewidth pop 0 0 M 8 8 L 0 8 M 8 0 L stroke} 
>> matrix makepattern
/Pat1 exch def
<< Tile8x8
 /PaintProc {0.5 setlinewidth pop 0 0 M 8 8 L 0 8 M 8 0 L stroke
	0 4 M 4 8 L 8 4 L 4 0 L 0 4 L stroke}
>> matrix makepattern
/Pat2 exch def
<< Tile8x8
 /PaintProc {0.5 setlinewidth pop 0 0 M 0 8 L
	8 8 L 8 0 L 0 0 L fill}
>> matrix makepattern
/Pat3 exch def
<< Tile8x8
 /PaintProc {0.5 setlinewidth pop -4 8 M 8 -4 L
	0 12 M 12 0 L stroke}
>> matrix makepattern
/Pat4 exch def
<< Tile8x8
 /PaintProc {0.5 setlinewidth pop -4 0 M 8 12 L
	0 -4 M 12 8 L stroke}
>> matrix makepattern
/Pat5 exch def
<< Tile8x8
 /PaintProc {0.5 setlinewidth pop -2 8 M 4 -4 L
	0 12 M 8 -4 L 4 12 M 10 0 L stroke}
>> matrix makepattern
/Pat6 exch def
<< Tile8x8
 /PaintProc {0.5 setlinewidth pop -2 0 M 4 12 L
	0 -4 M 8 12 L 4 -4 M 10 8 L stroke}
>> matrix makepattern
/Pat7 exch def
<< Tile8x8
 /PaintProc {0.5 setlinewidth pop 8 -2 M -4 4 L
	12 0 M -4 8 L 12 4 M 0 10 L stroke}
>> matrix makepattern
/Pat8 exch def
<< Tile8x8
 /PaintProc {0.5 setlinewidth pop 0 -2 M 12 4 L
	-4 0 M 12 8 L -4 4 M 8 10 L stroke}
>> matrix makepattern
/Pat9 exch def
/Pattern1 {PatternBgnd KeepColor Pat1 setpattern} bind def
/Pattern2 {PatternBgnd KeepColor Pat2 setpattern} bind def
/Pattern3 {PatternBgnd KeepColor Pat3 setpattern} bind def
/Pattern4 {PatternBgnd KeepColor Landscape {Pat5} {Pat4} ifelse setpattern} bind def
/Pattern5 {PatternBgnd KeepColor Landscape {Pat4} {Pat5} ifelse setpattern} bind def
/Pattern6 {PatternBgnd KeepColor Landscape {Pat9} {Pat6} ifelse setpattern} bind def
/Pattern7 {PatternBgnd KeepColor Landscape {Pat8} {Pat7} ifelse setpattern} bind def
} def
%
%
%End of PostScript Level 2 code
%
/PatternBgnd {
  TransparentPatterns {} {gsave 1 setgray fill grestore} ifelse
} def
%
% Substitute for Level 2 pattern fill codes with
% grayscale if Level 2 support is not selected.
%
/Level1PatternFill {
/Pattern1 {0.250 Density} bind def
/Pattern2 {0.500 Density} bind def
/Pattern3 {0.750 Density} bind def
/Pattern4 {0.125 Density} bind def
/Pattern5 {0.375 Density} bind def
/Pattern6 {0.625 Density} bind def
/Pattern7 {0.875 Density} bind def
} def
%
% Now test for support of Level 2 code
%
Level1 {Level1PatternFill} {Level2PatternFill} ifelse
%
/Symbol-Oblique /Symbol findfont [1 0 .167 1 0 0] makefont
dup length dict begin {1 index /FID eq {pop pop} {def} ifelse} forall
currentdict end definefont pop
%
Level1 SuppressPDFMark or 
{} {
/SDict 10 dict def
systemdict /pdfmark known not {
  userdict /pdfmark systemdict /cleartomark get put
} if
SDict begin [
  /Title (plot_MeffG+GGA1-+l26n_SU3.tex)
  /Subject (gnuplot plot)
  /Creator (gnuplot 5.0 patchlevel 3)
  /Author (mteper)
%  /Producer (gnuplot)
%  /Keywords ()
  /CreationDate (Sun Feb  7 16:36:24 2021)
  /DOCINFO pdfmark
end
} ifelse
%
% Support for boxed text - Ethan A Merritt May 2005
%
/InitTextBox { userdict /TBy2 3 -1 roll put userdict /TBx2 3 -1 roll put
           userdict /TBy1 3 -1 roll put userdict /TBx1 3 -1 roll put
	   /Boxing true def } def
/ExtendTextBox { Boxing
    { gsave dup false charpath pathbbox
      dup TBy2 gt {userdict /TBy2 3 -1 roll put} {pop} ifelse
      dup TBx2 gt {userdict /TBx2 3 -1 roll put} {pop} ifelse
      dup TBy1 lt {userdict /TBy1 3 -1 roll put} {pop} ifelse
      dup TBx1 lt {userdict /TBx1 3 -1 roll put} {pop} ifelse
      grestore } if } def
/PopTextBox { newpath TBx1 TBxmargin sub TBy1 TBymargin sub M
               TBx1 TBxmargin sub TBy2 TBymargin add L
	       TBx2 TBxmargin add TBy2 TBymargin add L
	       TBx2 TBxmargin add TBy1 TBymargin sub L closepath } def
/DrawTextBox { PopTextBox stroke /Boxing false def} def
/FillTextBox { gsave PopTextBox 1 1 1 setrgbcolor fill grestore /Boxing false def} def
0 0 0 0 InitTextBox
/TBxmargin 20 def
/TBymargin 20 def
/Boxing false def
/textshow { ExtendTextBox Gshow } def
%
% redundant definitions for compatibility with prologue.ps older than 5.0.2
/LTB {BL [] LCb DL} def
/LTb {PL [] LCb DL} def
end
%%EndProlog
%%Page: 1 1
gnudict begin
gsave
doclip
0 0 translate
0.050 0.050 scale
0 setgray
newpath
BackgroundColor 0 lt 3 1 roll 0 lt exch 0 lt or or not {BackgroundColor C 1.000 0 0 7200.00 7560.00 BoxColFill} if
1.000 UL
LTb
LCb setrgbcolor
1340 640 M
63 0 V
5436 0 R
-63 0 V
stroke
LTb
LCb setrgbcolor
1340 1276 M
63 0 V
5436 0 R
-63 0 V
stroke
LTb
LCb setrgbcolor
1340 1912 M
63 0 V
5436 0 R
-63 0 V
stroke
LTb
LCb setrgbcolor
1340 2548 M
63 0 V
5436 0 R
-63 0 V
stroke
LTb
LCb setrgbcolor
1340 3184 M
63 0 V
5436 0 R
-63 0 V
stroke
LTb
LCb setrgbcolor
1340 3820 M
63 0 V
5436 0 R
-63 0 V
stroke
LTb
LCb setrgbcolor
1340 4457 M
63 0 V
5436 0 R
-63 0 V
stroke
LTb
LCb setrgbcolor
1340 5093 M
63 0 V
5436 0 R
-63 0 V
stroke
LTb
LCb setrgbcolor
1340 5729 M
63 0 V
5436 0 R
-63 0 V
stroke
LTb
LCb setrgbcolor
1340 6365 M
63 0 V
5436 0 R
-63 0 V
stroke
LTb
LCb setrgbcolor
1340 7001 M
63 0 V
5436 0 R
-63 0 V
stroke
LTb
LCb setrgbcolor
1340 640 M
0 63 V
0 6616 R
0 -63 V
stroke
LTb
LCb setrgbcolor
2440 640 M
0 63 V
0 6616 R
0 -63 V
stroke
LTb
LCb setrgbcolor
3540 640 M
0 63 V
0 6616 R
0 -63 V
stroke
LTb
LCb setrgbcolor
4639 640 M
0 63 V
0 6616 R
0 -63 V
stroke
LTb
LCb setrgbcolor
5739 640 M
0 63 V
0 6616 R
0 -63 V
stroke
LTb
LCb setrgbcolor
6839 640 M
0 63 V
0 6616 R
0 -63 V
stroke
LTb
LCb setrgbcolor
1.000 UL
LTb
LCb setrgbcolor
1340 7319 N
0 -6679 V
5499 0 V
0 6679 V
-5499 0 V
Z stroke
1.000 UP
1.000 UL
LTb
LCb setrgbcolor
LCb setrgbcolor
LTb
LCb setrgbcolor
LTb
1.500 UP
1.000 UL
LTb
0.58 0.00 0.83 C 1934 3423 M
0 12 V
3034 3221 M
0 31 V
4133 3151 M
0 56 V
1100 51 R
0 135 V
6333 3047 M
0 343 V
1934 4504 M
0 18 V
3034 4179 M
0 50 V
4133 4129 M
0 140 V
5233 3874 M
0 403 V
1934 5794 M
0 27 V
3034 5178 M
0 100 V
4133 4634 M
0 385 V
1934 5944 M
0 31 V
3034 5211 M
0 109 V
4133 4889 M
0 543 V
1934 6310 M
0 33 V
3034 5696 M
0 145 V
4133 5431 M
0 706 V
1934 3429 CircleF
3034 3237 CircleF
4133 3179 CircleF
5233 3326 CircleF
6333 3218 CircleF
1934 4513 CircleF
3034 4204 CircleF
4133 4199 CircleF
5233 4075 CircleF
1934 5807 CircleF
3034 5228 CircleF
4133 4826 CircleF
1934 5959 CircleF
3034 5266 CircleF
4133 5161 CircleF
1934 6326 CircleF
3034 5769 CircleF
4133 5784 CircleF
2.000 UP
1.000 UL
LTb
0.58 0.00 0.83 C 1890 5489 M
0 27 V
2990 4993 M
0 104 V
4090 4564 M
0 343 V
1890 5503 DiaF
2990 5045 DiaF
4090 4735 DiaF
1.500 UP
1.000 UL
LTb
0.58 0.00 0.83 C 1846 3423 M
0 12 V
2946 3221 M
0 31 V
4046 3151 M
0 55 V
1099 51 R
0 135 V
6245 3045 M
0 343 V
1846 4510 M
0 18 V
2946 4185 M
0 51 V
1100 -98 R
0 135 V
5145 3877 M
0 402 V
1846 5678 M
0 29 V
2946 5116 M
0 83 V
4046 4580 M
0 376 V
1846 5918 M
0 29 V
2946 5203 M
0 114 V
4046 4931 M
0 551 V
1846 6310 M
0 34 V
2946 5697 M
0 145 V
4046 5463 M
0 715 V
1846 3429 Circle
2946 3237 Circle
4046 3179 Circle
5145 3325 Circle
6245 3216 Circle
1846 4519 Circle
2946 4211 Circle
4046 4206 Circle
5145 4078 Circle
1846 5692 Circle
2946 5157 Circle
4046 4768 Circle
1846 5933 Circle
2946 5260 Circle
4046 5207 Circle
1846 6327 Circle
2946 5770 Circle
4046 5820 Circle
1.500 UL
LTb
0.58 0.00 0.83 C 1340 4711 M
56 0 V
55 0 V
56 0 V
55 0 V
56 0 V
55 0 V
56 0 V
55 0 V
56 0 V
55 0 V
56 0 V
56 0 V
55 0 V
56 0 V
55 0 V
56 0 V
55 0 V
56 0 V
55 0 V
56 0 V
55 0 V
56 0 V
56 0 V
55 0 V
56 0 V
55 0 V
56 0 V
55 0 V
56 0 V
55 0 V
56 0 V
55 0 V
56 0 V
56 0 V
55 0 V
56 0 V
55 0 V
56 0 V
55 0 V
56 0 V
55 0 V
56 0 V
55 0 V
56 0 V
56 0 V
55 0 V
56 0 V
55 0 V
56 0 V
55 0 V
56 0 V
55 0 V
56 0 V
55 0 V
56 0 V
56 0 V
55 0 V
56 0 V
55 0 V
56 0 V
55 0 V
56 0 V
55 0 V
56 0 V
55 0 V
56 0 V
56 0 V
55 0 V
56 0 V
55 0 V
56 0 V
55 0 V
56 0 V
55 0 V
56 0 V
55 0 V
56 0 V
56 0 V
55 0 V
56 0 V
55 0 V
56 0 V
55 0 V
56 0 V
55 0 V
56 0 V
55 0 V
56 0 V
56 0 V
55 0 V
56 0 V
55 0 V
56 0 V
55 0 V
56 0 V
55 0 V
56 0 V
55 0 V
56 0 V
stroke
2.000 UL
LTb
LCb setrgbcolor
1.000 UL
LTb
LCb setrgbcolor
1340 7319 N
0 -6679 V
5499 0 V
0 6679 V
-5499 0 V
Z stroke
1.000 UP
1.000 UL
LTb
LCb setrgbcolor
stroke
grestore
end
showpage
  }}%
  \put(4089,140){\makebox(0,0){\large{$t/a$}}}%
  \put(160,5179){\makebox(0,0){\Large{$aE_{eff}(t)$}}}%
  \put(6839,440){\makebox(0,0){\strut{}\ {$5$}}}%
  \put(5739,440){\makebox(0,0){\strut{}\ {$4$}}}%
  \put(4639,440){\makebox(0,0){\strut{}\ {$3$}}}%
  \put(3540,440){\makebox(0,0){\strut{}\ {$2$}}}%
  \put(2440,440){\makebox(0,0){\strut{}\ {$1$}}}%
  \put(1340,440){\makebox(0,0){\strut{}\ {$0$}}}%
  \put(1220,7001){\makebox(0,0)[r]{\strut{}\ \ {$2$}}}%
  \put(1220,6365){\makebox(0,0)[r]{\strut{}\ \ {$1.8$}}}%
  \put(1220,5729){\makebox(0,0)[r]{\strut{}\ \ {$1.6$}}}%
  \put(1220,5093){\makebox(0,0)[r]{\strut{}\ \ {$1.4$}}}%
  \put(1220,4457){\makebox(0,0)[r]{\strut{}\ \ {$1.2$}}}%
  \put(1220,3820){\makebox(0,0)[r]{\strut{}\ \ {$1$}}}%
  \put(1220,3184){\makebox(0,0)[r]{\strut{}\ \ {$0.8$}}}%
  \put(1220,2548){\makebox(0,0)[r]{\strut{}\ \ {$0.6$}}}%
  \put(1220,1912){\makebox(0,0)[r]{\strut{}\ \ {$0.4$}}}%
  \put(1220,1276){\makebox(0,0)[r]{\strut{}\ \ {$0.2$}}}%
  \put(1220,640){\makebox(0,0)[r]{\strut{}\ \ {$0$}}}%
\end{picture}%
\endgroup
\endinput

\end	{center}
\caption{Effective masses for the lightest few glueballs in the `pseudoscalar' $A_1^{-+}$
  representation, for the single trace operators ($\circ$) and for the same set
  augmented with double trace operators (filled points), with points shifted for clarity.
  The likely extra `scattering' state amongst the latter is shown as $\blacklozenge$. 
  Horizontal line indicates the sum of the lightest $A_1^{++}$ and $A_1^{-+}$ glueball masses.
  On the $26^326$ lattice at $\beta=6.235$ in SU(3).}
\label{fig_MeffG+GGA1-+l26n_SU3}
\end{figure}

\begin{figure}[htb]
\begin	{center}
\leavevmode
% GNUPLOT: LaTeX picture with Postscript
\begingroup%
\makeatletter%
\newcommand{\GNUPLOTspecial}{%
  \@sanitize\catcode`\%=14\relax\special}%
\setlength{\unitlength}{0.0500bp}%
\begin{picture}(7200,7560)(0,0)%
  {\GNUPLOTspecial{"
%!PS-Adobe-2.0 EPSF-2.0
%%Title: plot_MeffGGA1-+l26n_SU3.tex
%%Creator: gnuplot 5.0 patchlevel 3
%%CreationDate: Sun Feb  7 16:46:12 2021
%%DocumentFonts: 
%%BoundingBox: 0 0 360 378
%%EndComments
%%BeginProlog
/gnudict 256 dict def
gnudict begin
%
% The following true/false flags may be edited by hand if desired.
% The unit line width and grayscale image gamma correction may also be changed.
%
/Color true def
/Blacktext true def
/Solid false def
/Dashlength 1 def
/Landscape false def
/Level1 false def
/Level3 false def
/Rounded false def
/ClipToBoundingBox false def
/SuppressPDFMark false def
/TransparentPatterns false def
/gnulinewidth 5.000 def
/userlinewidth gnulinewidth def
/Gamma 1.0 def
/BackgroundColor {-1.000 -1.000 -1.000} def
%
/vshift -66 def
/dl1 {
  10.0 Dashlength userlinewidth gnulinewidth div mul mul mul
  Rounded { currentlinewidth 0.75 mul sub dup 0 le { pop 0.01 } if } if
} def
/dl2 {
  10.0 Dashlength userlinewidth gnulinewidth div mul mul mul
  Rounded { currentlinewidth 0.75 mul add } if
} def
/hpt_ 31.5 def
/vpt_ 31.5 def
/hpt hpt_ def
/vpt vpt_ def
/doclip {
  ClipToBoundingBox {
    newpath 0 0 moveto 360 0 lineto 360 378 lineto 0 378 lineto closepath
    clip
  } if
} def
%
% Gnuplot Prolog Version 5.1 (Oct 2015)
%
%/SuppressPDFMark true def
%
/M {moveto} bind def
/L {lineto} bind def
/R {rmoveto} bind def
/V {rlineto} bind def
/N {newpath moveto} bind def
/Z {closepath} bind def
/C {setrgbcolor} bind def
/f {rlineto fill} bind def
/g {setgray} bind def
/Gshow {show} def   % May be redefined later in the file to support UTF-8
/vpt2 vpt 2 mul def
/hpt2 hpt 2 mul def
/Lshow {currentpoint stroke M 0 vshift R 
	Blacktext {gsave 0 setgray textshow grestore} {textshow} ifelse} def
/Rshow {currentpoint stroke M dup stringwidth pop neg vshift R
	Blacktext {gsave 0 setgray textshow grestore} {textshow} ifelse} def
/Cshow {currentpoint stroke M dup stringwidth pop -2 div vshift R 
	Blacktext {gsave 0 setgray textshow grestore} {textshow} ifelse} def
/UP {dup vpt_ mul /vpt exch def hpt_ mul /hpt exch def
  /hpt2 hpt 2 mul def /vpt2 vpt 2 mul def} def
/DL {Color {setrgbcolor Solid {pop []} if 0 setdash}
 {pop pop pop 0 setgray Solid {pop []} if 0 setdash} ifelse} def
/BL {stroke userlinewidth 2 mul setlinewidth
	Rounded {1 setlinejoin 1 setlinecap} if} def
/AL {stroke userlinewidth 2 div setlinewidth
	Rounded {1 setlinejoin 1 setlinecap} if} def
/UL {dup gnulinewidth mul /userlinewidth exch def
	dup 1 lt {pop 1} if 10 mul /udl exch def} def
/PL {stroke userlinewidth setlinewidth
	Rounded {1 setlinejoin 1 setlinecap} if} def
3.8 setmiterlimit
% Classic Line colors (version 5.0)
/LCw {1 1 1} def
/LCb {0 0 0} def
/LCa {0 0 0} def
/LC0 {1 0 0} def
/LC1 {0 1 0} def
/LC2 {0 0 1} def
/LC3 {1 0 1} def
/LC4 {0 1 1} def
/LC5 {1 1 0} def
/LC6 {0 0 0} def
/LC7 {1 0.3 0} def
/LC8 {0.5 0.5 0.5} def
% Default dash patterns (version 5.0)
/LTB {BL [] LCb DL} def
/LTw {PL [] 1 setgray} def
/LTb {PL [] LCb DL} def
/LTa {AL [1 udl mul 2 udl mul] 0 setdash LCa setrgbcolor} def
/LT0 {PL [] LC0 DL} def
/LT1 {PL [2 dl1 3 dl2] LC1 DL} def
/LT2 {PL [1 dl1 1.5 dl2] LC2 DL} def
/LT3 {PL [6 dl1 2 dl2 1 dl1 2 dl2] LC3 DL} def
/LT4 {PL [1 dl1 2 dl2 6 dl1 2 dl2 1 dl1 2 dl2] LC4 DL} def
/LT5 {PL [4 dl1 2 dl2] LC5 DL} def
/LT6 {PL [1.5 dl1 1.5 dl2 1.5 dl1 1.5 dl2 1.5 dl1 6 dl2] LC6 DL} def
/LT7 {PL [3 dl1 3 dl2 1 dl1 3 dl2] LC7 DL} def
/LT8 {PL [2 dl1 2 dl2 2 dl1 6 dl2] LC8 DL} def
/SL {[] 0 setdash} def
/Pnt {stroke [] 0 setdash gsave 1 setlinecap M 0 0 V stroke grestore} def
/Dia {stroke [] 0 setdash 2 copy vpt add M
  hpt neg vpt neg V hpt vpt neg V
  hpt vpt V hpt neg vpt V closepath stroke
  Pnt} def
/Pls {stroke [] 0 setdash vpt sub M 0 vpt2 V
  currentpoint stroke M
  hpt neg vpt neg R hpt2 0 V stroke
 } def
/Box {stroke [] 0 setdash 2 copy exch hpt sub exch vpt add M
  0 vpt2 neg V hpt2 0 V 0 vpt2 V
  hpt2 neg 0 V closepath stroke
  Pnt} def
/Crs {stroke [] 0 setdash exch hpt sub exch vpt add M
  hpt2 vpt2 neg V currentpoint stroke M
  hpt2 neg 0 R hpt2 vpt2 V stroke} def
/TriU {stroke [] 0 setdash 2 copy vpt 1.12 mul add M
  hpt neg vpt -1.62 mul V
  hpt 2 mul 0 V
  hpt neg vpt 1.62 mul V closepath stroke
  Pnt} def
/Star {2 copy Pls Crs} def
/BoxF {stroke [] 0 setdash exch hpt sub exch vpt add M
  0 vpt2 neg V hpt2 0 V 0 vpt2 V
  hpt2 neg 0 V closepath fill} def
/TriUF {stroke [] 0 setdash vpt 1.12 mul add M
  hpt neg vpt -1.62 mul V
  hpt 2 mul 0 V
  hpt neg vpt 1.62 mul V closepath fill} def
/TriD {stroke [] 0 setdash 2 copy vpt 1.12 mul sub M
  hpt neg vpt 1.62 mul V
  hpt 2 mul 0 V
  hpt neg vpt -1.62 mul V closepath stroke
  Pnt} def
/TriDF {stroke [] 0 setdash vpt 1.12 mul sub M
  hpt neg vpt 1.62 mul V
  hpt 2 mul 0 V
  hpt neg vpt -1.62 mul V closepath fill} def
/DiaF {stroke [] 0 setdash vpt add M
  hpt neg vpt neg V hpt vpt neg V
  hpt vpt V hpt neg vpt V closepath fill} def
/Pent {stroke [] 0 setdash 2 copy gsave
  translate 0 hpt M 4 {72 rotate 0 hpt L} repeat
  closepath stroke grestore Pnt} def
/PentF {stroke [] 0 setdash gsave
  translate 0 hpt M 4 {72 rotate 0 hpt L} repeat
  closepath fill grestore} def
/Circle {stroke [] 0 setdash 2 copy
  hpt 0 360 arc stroke Pnt} def
/CircleF {stroke [] 0 setdash hpt 0 360 arc fill} def
/C0 {BL [] 0 setdash 2 copy moveto vpt 90 450 arc} bind def
/C1 {BL [] 0 setdash 2 copy moveto
	2 copy vpt 0 90 arc closepath fill
	vpt 0 360 arc closepath} bind def
/C2 {BL [] 0 setdash 2 copy moveto
	2 copy vpt 90 180 arc closepath fill
	vpt 0 360 arc closepath} bind def
/C3 {BL [] 0 setdash 2 copy moveto
	2 copy vpt 0 180 arc closepath fill
	vpt 0 360 arc closepath} bind def
/C4 {BL [] 0 setdash 2 copy moveto
	2 copy vpt 180 270 arc closepath fill
	vpt 0 360 arc closepath} bind def
/C5 {BL [] 0 setdash 2 copy moveto
	2 copy vpt 0 90 arc
	2 copy moveto
	2 copy vpt 180 270 arc closepath fill
	vpt 0 360 arc} bind def
/C6 {BL [] 0 setdash 2 copy moveto
	2 copy vpt 90 270 arc closepath fill
	vpt 0 360 arc closepath} bind def
/C7 {BL [] 0 setdash 2 copy moveto
	2 copy vpt 0 270 arc closepath fill
	vpt 0 360 arc closepath} bind def
/C8 {BL [] 0 setdash 2 copy moveto
	2 copy vpt 270 360 arc closepath fill
	vpt 0 360 arc closepath} bind def
/C9 {BL [] 0 setdash 2 copy moveto
	2 copy vpt 270 450 arc closepath fill
	vpt 0 360 arc closepath} bind def
/C10 {BL [] 0 setdash 2 copy 2 copy moveto vpt 270 360 arc closepath fill
	2 copy moveto
	2 copy vpt 90 180 arc closepath fill
	vpt 0 360 arc closepath} bind def
/C11 {BL [] 0 setdash 2 copy moveto
	2 copy vpt 0 180 arc closepath fill
	2 copy moveto
	2 copy vpt 270 360 arc closepath fill
	vpt 0 360 arc closepath} bind def
/C12 {BL [] 0 setdash 2 copy moveto
	2 copy vpt 180 360 arc closepath fill
	vpt 0 360 arc closepath} bind def
/C13 {BL [] 0 setdash 2 copy moveto
	2 copy vpt 0 90 arc closepath fill
	2 copy moveto
	2 copy vpt 180 360 arc closepath fill
	vpt 0 360 arc closepath} bind def
/C14 {BL [] 0 setdash 2 copy moveto
	2 copy vpt 90 360 arc closepath fill
	vpt 0 360 arc} bind def
/C15 {BL [] 0 setdash 2 copy vpt 0 360 arc closepath fill
	vpt 0 360 arc closepath} bind def
/Rec {newpath 4 2 roll moveto 1 index 0 rlineto 0 exch rlineto
	neg 0 rlineto closepath} bind def
/Square {dup Rec} bind def
/Bsquare {vpt sub exch vpt sub exch vpt2 Square} bind def
/S0 {BL [] 0 setdash 2 copy moveto 0 vpt rlineto BL Bsquare} bind def
/S1 {BL [] 0 setdash 2 copy vpt Square fill Bsquare} bind def
/S2 {BL [] 0 setdash 2 copy exch vpt sub exch vpt Square fill Bsquare} bind def
/S3 {BL [] 0 setdash 2 copy exch vpt sub exch vpt2 vpt Rec fill Bsquare} bind def
/S4 {BL [] 0 setdash 2 copy exch vpt sub exch vpt sub vpt Square fill Bsquare} bind def
/S5 {BL [] 0 setdash 2 copy 2 copy vpt Square fill
	exch vpt sub exch vpt sub vpt Square fill Bsquare} bind def
/S6 {BL [] 0 setdash 2 copy exch vpt sub exch vpt sub vpt vpt2 Rec fill Bsquare} bind def
/S7 {BL [] 0 setdash 2 copy exch vpt sub exch vpt sub vpt vpt2 Rec fill
	2 copy vpt Square fill Bsquare} bind def
/S8 {BL [] 0 setdash 2 copy vpt sub vpt Square fill Bsquare} bind def
/S9 {BL [] 0 setdash 2 copy vpt sub vpt vpt2 Rec fill Bsquare} bind def
/S10 {BL [] 0 setdash 2 copy vpt sub vpt Square fill 2 copy exch vpt sub exch vpt Square fill
	Bsquare} bind def
/S11 {BL [] 0 setdash 2 copy vpt sub vpt Square fill 2 copy exch vpt sub exch vpt2 vpt Rec fill
	Bsquare} bind def
/S12 {BL [] 0 setdash 2 copy exch vpt sub exch vpt sub vpt2 vpt Rec fill Bsquare} bind def
/S13 {BL [] 0 setdash 2 copy exch vpt sub exch vpt sub vpt2 vpt Rec fill
	2 copy vpt Square fill Bsquare} bind def
/S14 {BL [] 0 setdash 2 copy exch vpt sub exch vpt sub vpt2 vpt Rec fill
	2 copy exch vpt sub exch vpt Square fill Bsquare} bind def
/S15 {BL [] 0 setdash 2 copy Bsquare fill Bsquare} bind def
/D0 {gsave translate 45 rotate 0 0 S0 stroke grestore} bind def
/D1 {gsave translate 45 rotate 0 0 S1 stroke grestore} bind def
/D2 {gsave translate 45 rotate 0 0 S2 stroke grestore} bind def
/D3 {gsave translate 45 rotate 0 0 S3 stroke grestore} bind def
/D4 {gsave translate 45 rotate 0 0 S4 stroke grestore} bind def
/D5 {gsave translate 45 rotate 0 0 S5 stroke grestore} bind def
/D6 {gsave translate 45 rotate 0 0 S6 stroke grestore} bind def
/D7 {gsave translate 45 rotate 0 0 S7 stroke grestore} bind def
/D8 {gsave translate 45 rotate 0 0 S8 stroke grestore} bind def
/D9 {gsave translate 45 rotate 0 0 S9 stroke grestore} bind def
/D10 {gsave translate 45 rotate 0 0 S10 stroke grestore} bind def
/D11 {gsave translate 45 rotate 0 0 S11 stroke grestore} bind def
/D12 {gsave translate 45 rotate 0 0 S12 stroke grestore} bind def
/D13 {gsave translate 45 rotate 0 0 S13 stroke grestore} bind def
/D14 {gsave translate 45 rotate 0 0 S14 stroke grestore} bind def
/D15 {gsave translate 45 rotate 0 0 S15 stroke grestore} bind def
/DiaE {stroke [] 0 setdash vpt add M
  hpt neg vpt neg V hpt vpt neg V
  hpt vpt V hpt neg vpt V closepath stroke} def
/BoxE {stroke [] 0 setdash exch hpt sub exch vpt add M
  0 vpt2 neg V hpt2 0 V 0 vpt2 V
  hpt2 neg 0 V closepath stroke} def
/TriUE {stroke [] 0 setdash vpt 1.12 mul add M
  hpt neg vpt -1.62 mul V
  hpt 2 mul 0 V
  hpt neg vpt 1.62 mul V closepath stroke} def
/TriDE {stroke [] 0 setdash vpt 1.12 mul sub M
  hpt neg vpt 1.62 mul V
  hpt 2 mul 0 V
  hpt neg vpt -1.62 mul V closepath stroke} def
/PentE {stroke [] 0 setdash gsave
  translate 0 hpt M 4 {72 rotate 0 hpt L} repeat
  closepath stroke grestore} def
/CircE {stroke [] 0 setdash 
  hpt 0 360 arc stroke} def
/Opaque {gsave closepath 1 setgray fill grestore 0 setgray closepath} def
/DiaW {stroke [] 0 setdash vpt add M
  hpt neg vpt neg V hpt vpt neg V
  hpt vpt V hpt neg vpt V Opaque stroke} def
/BoxW {stroke [] 0 setdash exch hpt sub exch vpt add M
  0 vpt2 neg V hpt2 0 V 0 vpt2 V
  hpt2 neg 0 V Opaque stroke} def
/TriUW {stroke [] 0 setdash vpt 1.12 mul add M
  hpt neg vpt -1.62 mul V
  hpt 2 mul 0 V
  hpt neg vpt 1.62 mul V Opaque stroke} def
/TriDW {stroke [] 0 setdash vpt 1.12 mul sub M
  hpt neg vpt 1.62 mul V
  hpt 2 mul 0 V
  hpt neg vpt -1.62 mul V Opaque stroke} def
/PentW {stroke [] 0 setdash gsave
  translate 0 hpt M 4 {72 rotate 0 hpt L} repeat
  Opaque stroke grestore} def
/CircW {stroke [] 0 setdash 
  hpt 0 360 arc Opaque stroke} def
/BoxFill {gsave Rec 1 setgray fill grestore} def
/Density {
  /Fillden exch def
  currentrgbcolor
  /ColB exch def /ColG exch def /ColR exch def
  /ColR ColR Fillden mul Fillden sub 1 add def
  /ColG ColG Fillden mul Fillden sub 1 add def
  /ColB ColB Fillden mul Fillden sub 1 add def
  ColR ColG ColB setrgbcolor} def
/BoxColFill {gsave Rec PolyFill} def
/PolyFill {gsave Density fill grestore grestore} def
/h {rlineto rlineto rlineto gsave closepath fill grestore} bind def
%
% PostScript Level 1 Pattern Fill routine for rectangles
% Usage: x y w h s a XX PatternFill
%	x,y = lower left corner of box to be filled
%	w,h = width and height of box
%	  a = angle in degrees between lines and x-axis
%	 XX = 0/1 for no/yes cross-hatch
%
/PatternFill {gsave /PFa [ 9 2 roll ] def
  PFa 0 get PFa 2 get 2 div add PFa 1 get PFa 3 get 2 div add translate
  PFa 2 get -2 div PFa 3 get -2 div PFa 2 get PFa 3 get Rec
  TransparentPatterns {} {gsave 1 setgray fill grestore} ifelse
  clip
  currentlinewidth 0.5 mul setlinewidth
  /PFs PFa 2 get dup mul PFa 3 get dup mul add sqrt def
  0 0 M PFa 5 get rotate PFs -2 div dup translate
  0 1 PFs PFa 4 get div 1 add floor cvi
	{PFa 4 get mul 0 M 0 PFs V} for
  0 PFa 6 get ne {
	0 1 PFs PFa 4 get div 1 add floor cvi
	{PFa 4 get mul 0 2 1 roll M PFs 0 V} for
 } if
  stroke grestore} def
%
/languagelevel where
 {pop languagelevel} {1} ifelse
dup 2 lt
	{/InterpretLevel1 true def
	 /InterpretLevel3 false def}
	{/InterpretLevel1 Level1 def
	 2 gt
	    {/InterpretLevel3 Level3 def}
	    {/InterpretLevel3 false def}
	 ifelse }
 ifelse
%
% PostScript level 2 pattern fill definitions
%
/Level2PatternFill {
/Tile8x8 {/PaintType 2 /PatternType 1 /TilingType 1 /BBox [0 0 8 8] /XStep 8 /YStep 8}
	bind def
/KeepColor {currentrgbcolor [/Pattern /DeviceRGB] setcolorspace} bind def
<< Tile8x8
 /PaintProc {0.5 setlinewidth pop 0 0 M 8 8 L 0 8 M 8 0 L stroke} 
>> matrix makepattern
/Pat1 exch def
<< Tile8x8
 /PaintProc {0.5 setlinewidth pop 0 0 M 8 8 L 0 8 M 8 0 L stroke
	0 4 M 4 8 L 8 4 L 4 0 L 0 4 L stroke}
>> matrix makepattern
/Pat2 exch def
<< Tile8x8
 /PaintProc {0.5 setlinewidth pop 0 0 M 0 8 L
	8 8 L 8 0 L 0 0 L fill}
>> matrix makepattern
/Pat3 exch def
<< Tile8x8
 /PaintProc {0.5 setlinewidth pop -4 8 M 8 -4 L
	0 12 M 12 0 L stroke}
>> matrix makepattern
/Pat4 exch def
<< Tile8x8
 /PaintProc {0.5 setlinewidth pop -4 0 M 8 12 L
	0 -4 M 12 8 L stroke}
>> matrix makepattern
/Pat5 exch def
<< Tile8x8
 /PaintProc {0.5 setlinewidth pop -2 8 M 4 -4 L
	0 12 M 8 -4 L 4 12 M 10 0 L stroke}
>> matrix makepattern
/Pat6 exch def
<< Tile8x8
 /PaintProc {0.5 setlinewidth pop -2 0 M 4 12 L
	0 -4 M 8 12 L 4 -4 M 10 8 L stroke}
>> matrix makepattern
/Pat7 exch def
<< Tile8x8
 /PaintProc {0.5 setlinewidth pop 8 -2 M -4 4 L
	12 0 M -4 8 L 12 4 M 0 10 L stroke}
>> matrix makepattern
/Pat8 exch def
<< Tile8x8
 /PaintProc {0.5 setlinewidth pop 0 -2 M 12 4 L
	-4 0 M 12 8 L -4 4 M 8 10 L stroke}
>> matrix makepattern
/Pat9 exch def
/Pattern1 {PatternBgnd KeepColor Pat1 setpattern} bind def
/Pattern2 {PatternBgnd KeepColor Pat2 setpattern} bind def
/Pattern3 {PatternBgnd KeepColor Pat3 setpattern} bind def
/Pattern4 {PatternBgnd KeepColor Landscape {Pat5} {Pat4} ifelse setpattern} bind def
/Pattern5 {PatternBgnd KeepColor Landscape {Pat4} {Pat5} ifelse setpattern} bind def
/Pattern6 {PatternBgnd KeepColor Landscape {Pat9} {Pat6} ifelse setpattern} bind def
/Pattern7 {PatternBgnd KeepColor Landscape {Pat8} {Pat7} ifelse setpattern} bind def
} def
%
%
%End of PostScript Level 2 code
%
/PatternBgnd {
  TransparentPatterns {} {gsave 1 setgray fill grestore} ifelse
} def
%
% Substitute for Level 2 pattern fill codes with
% grayscale if Level 2 support is not selected.
%
/Level1PatternFill {
/Pattern1 {0.250 Density} bind def
/Pattern2 {0.500 Density} bind def
/Pattern3 {0.750 Density} bind def
/Pattern4 {0.125 Density} bind def
/Pattern5 {0.375 Density} bind def
/Pattern6 {0.625 Density} bind def
/Pattern7 {0.875 Density} bind def
} def
%
% Now test for support of Level 2 code
%
Level1 {Level1PatternFill} {Level2PatternFill} ifelse
%
/Symbol-Oblique /Symbol findfont [1 0 .167 1 0 0] makefont
dup length dict begin {1 index /FID eq {pop pop} {def} ifelse} forall
currentdict end definefont pop
%
Level1 SuppressPDFMark or 
{} {
/SDict 10 dict def
systemdict /pdfmark known not {
  userdict /pdfmark systemdict /cleartomark get put
} if
SDict begin [
  /Title (plot_MeffGGA1-+l26n_SU3.tex)
  /Subject (gnuplot plot)
  /Creator (gnuplot 5.0 patchlevel 3)
  /Author (mteper)
%  /Producer (gnuplot)
%  /Keywords ()
  /CreationDate (Sun Feb  7 16:46:12 2021)
  /DOCINFO pdfmark
end
} ifelse
%
% Support for boxed text - Ethan A Merritt May 2005
%
/InitTextBox { userdict /TBy2 3 -1 roll put userdict /TBx2 3 -1 roll put
           userdict /TBy1 3 -1 roll put userdict /TBx1 3 -1 roll put
	   /Boxing true def } def
/ExtendTextBox { Boxing
    { gsave dup false charpath pathbbox
      dup TBy2 gt {userdict /TBy2 3 -1 roll put} {pop} ifelse
      dup TBx2 gt {userdict /TBx2 3 -1 roll put} {pop} ifelse
      dup TBy1 lt {userdict /TBy1 3 -1 roll put} {pop} ifelse
      dup TBx1 lt {userdict /TBx1 3 -1 roll put} {pop} ifelse
      grestore } if } def
/PopTextBox { newpath TBx1 TBxmargin sub TBy1 TBymargin sub M
               TBx1 TBxmargin sub TBy2 TBymargin add L
	       TBx2 TBxmargin add TBy2 TBymargin add L
	       TBx2 TBxmargin add TBy1 TBymargin sub L closepath } def
/DrawTextBox { PopTextBox stroke /Boxing false def} def
/FillTextBox { gsave PopTextBox 1 1 1 setrgbcolor fill grestore /Boxing false def} def
0 0 0 0 InitTextBox
/TBxmargin 20 def
/TBymargin 20 def
/Boxing false def
/textshow { ExtendTextBox Gshow } def
%
% redundant definitions for compatibility with prologue.ps older than 5.0.2
/LTB {BL [] LCb DL} def
/LTb {PL [] LCb DL} def
end
%%EndProlog
%%Page: 1 1
gnudict begin
gsave
doclip
0 0 translate
0.050 0.050 scale
0 setgray
newpath
BackgroundColor 0 lt 3 1 roll 0 lt exch 0 lt or or not {BackgroundColor C 1.000 0 0 7200.00 7560.00 BoxColFill} if
1.000 UL
LTb
LCb setrgbcolor
1340 1011 M
63 0 V
5436 0 R
-63 0 V
stroke
LTb
LCb setrgbcolor
1340 1753 M
63 0 V
5436 0 R
-63 0 V
stroke
LTb
LCb setrgbcolor
1340 2495 M
63 0 V
5436 0 R
-63 0 V
stroke
LTb
LCb setrgbcolor
1340 3237 M
63 0 V
5436 0 R
-63 0 V
stroke
LTb
LCb setrgbcolor
1340 3980 M
63 0 V
5436 0 R
-63 0 V
stroke
LTb
LCb setrgbcolor
1340 4722 M
63 0 V
5436 0 R
-63 0 V
stroke
LTb
LCb setrgbcolor
1340 5464 M
63 0 V
5436 0 R
-63 0 V
stroke
LTb
LCb setrgbcolor
1340 6206 M
63 0 V
5436 0 R
-63 0 V
stroke
LTb
LCb setrgbcolor
1340 6948 M
63 0 V
5436 0 R
-63 0 V
stroke
LTb
LCb setrgbcolor
1340 640 M
0 63 V
0 6616 R
0 -63 V
stroke
LTb
LCb setrgbcolor
2440 640 M
0 63 V
0 6616 R
0 -63 V
stroke
LTb
LCb setrgbcolor
3540 640 M
0 63 V
0 6616 R
0 -63 V
stroke
LTb
LCb setrgbcolor
4639 640 M
0 63 V
0 6616 R
0 -63 V
stroke
LTb
LCb setrgbcolor
5739 640 M
0 63 V
0 6616 R
0 -63 V
stroke
LTb
LCb setrgbcolor
6839 640 M
0 63 V
0 6616 R
0 -63 V
stroke
LTb
LCb setrgbcolor
1.000 UL
LTb
LCb setrgbcolor
1340 7319 N
0 -6679 V
5499 0 V
0 6679 V
-5499 0 V
Z stroke
1.000 UP
1.000 UL
LTb
LCb setrgbcolor
LCb setrgbcolor
LTb
LCb setrgbcolor
LTb
1.500 UP
1.000 UL
LTb
0.58 0.00 0.83 C 1890 2668 M
0 18 V
2990 2202 M
0 39 V
4090 1941 M
0 101 V
1099 -92 R
0 241 V
6289 1590 M
0 515 V
1890 5649 M
0 41 V
2990 4940 M
0 181 V
4090 4926 M
0 1198 V
1890 6295 M
0 55 V
2990 5223 M
0 274 V
4090 4756 M
0 1403 V
1890 2677 CircleF
2990 2221 CircleF
4090 1991 CircleF
5189 2071 CircleF
6289 1848 CircleF
1890 5669 CircleF
2990 5031 CircleF
4090 5525 CircleF
1890 6323 CircleF
2990 5360 CircleF
4090 5457 CircleF
1.500 UP
1.000 UL
LTb
0.58 0.00 0.83 C 1.500 UP
1.000 UL
LTb
0.58 0.00 0.83 C 1.500 UL
LTb
0.58 0.00 0.83 C 1340 1683 M
56 0 V
55 0 V
56 0 V
55 0 V
56 0 V
55 0 V
56 0 V
55 0 V
56 0 V
55 0 V
56 0 V
56 0 V
55 0 V
56 0 V
55 0 V
56 0 V
55 0 V
56 0 V
55 0 V
56 0 V
55 0 V
56 0 V
56 0 V
55 0 V
56 0 V
55 0 V
56 0 V
55 0 V
56 0 V
55 0 V
56 0 V
55 0 V
56 0 V
56 0 V
55 0 V
56 0 V
55 0 V
56 0 V
55 0 V
56 0 V
55 0 V
56 0 V
55 0 V
56 0 V
56 0 V
55 0 V
56 0 V
55 0 V
56 0 V
55 0 V
56 0 V
55 0 V
56 0 V
55 0 V
56 0 V
56 0 V
55 0 V
56 0 V
55 0 V
56 0 V
55 0 V
56 0 V
55 0 V
56 0 V
55 0 V
56 0 V
56 0 V
55 0 V
56 0 V
55 0 V
56 0 V
55 0 V
56 0 V
55 0 V
56 0 V
55 0 V
56 0 V
56 0 V
55 0 V
56 0 V
55 0 V
56 0 V
55 0 V
56 0 V
55 0 V
56 0 V
55 0 V
56 0 V
56 0 V
55 0 V
56 0 V
55 0 V
56 0 V
55 0 V
56 0 V
55 0 V
56 0 V
55 0 V
56 0 V
stroke
LTb
0.58 0.00 0.83 C 1340 3527 M
56 0 V
55 0 V
56 0 V
55 0 V
56 0 V
55 0 V
56 0 V
55 0 V
56 0 V
55 0 V
56 0 V
56 0 V
55 0 V
56 0 V
55 0 V
56 0 V
55 0 V
56 0 V
55 0 V
56 0 V
55 0 V
56 0 V
56 0 V
55 0 V
56 0 V
55 0 V
56 0 V
55 0 V
56 0 V
55 0 V
56 0 V
55 0 V
56 0 V
56 0 V
55 0 V
56 0 V
55 0 V
56 0 V
55 0 V
56 0 V
55 0 V
56 0 V
55 0 V
56 0 V
56 0 V
55 0 V
56 0 V
55 0 V
56 0 V
55 0 V
56 0 V
55 0 V
56 0 V
55 0 V
56 0 V
56 0 V
55 0 V
56 0 V
55 0 V
56 0 V
55 0 V
56 0 V
55 0 V
56 0 V
55 0 V
56 0 V
56 0 V
55 0 V
56 0 V
55 0 V
56 0 V
55 0 V
56 0 V
55 0 V
56 0 V
55 0 V
56 0 V
56 0 V
55 0 V
56 0 V
55 0 V
56 0 V
55 0 V
56 0 V
55 0 V
56 0 V
55 0 V
56 0 V
56 0 V
55 0 V
56 0 V
55 0 V
56 0 V
55 0 V
56 0 V
55 0 V
56 0 V
55 0 V
56 0 V
stroke
2.000 UL
LTb
LCb setrgbcolor
1.000 UL
LTb
LCb setrgbcolor
1340 7319 N
0 -6679 V
5499 0 V
0 6679 V
-5499 0 V
Z stroke
1.000 UP
1.000 UL
LTb
LCb setrgbcolor
stroke
grestore
end
showpage
  }}%
  \put(4089,140){\makebox(0,0){\large{$t/a$}}}%
  \put(160,5179){\makebox(0,0){\Large{$aE_{eff}(t)$}}}%
  \put(6839,440){\makebox(0,0){\strut{}\ {$5$}}}%
  \put(5739,440){\makebox(0,0){\strut{}\ {$4$}}}%
  \put(4639,440){\makebox(0,0){\strut{}\ {$3$}}}%
  \put(3540,440){\makebox(0,0){\strut{}\ {$2$}}}%
  \put(2440,440){\makebox(0,0){\strut{}\ {$1$}}}%
  \put(1340,440){\makebox(0,0){\strut{}\ {$0$}}}%
  \put(1220,6948){\makebox(0,0)[r]{\strut{}\ \ {$2.2$}}}%
  \put(1220,6206){\makebox(0,0)[r]{\strut{}\ \ {$2$}}}%
  \put(1220,5464){\makebox(0,0)[r]{\strut{}\ \ {$1.8$}}}%
  \put(1220,4722){\makebox(0,0)[r]{\strut{}\ \ {$1.6$}}}%
  \put(1220,3980){\makebox(0,0)[r]{\strut{}\ \ {$1.4$}}}%
  \put(1220,3237){\makebox(0,0)[r]{\strut{}\ \ {$1.2$}}}%
  \put(1220,2495){\makebox(0,0)[r]{\strut{}\ \ {$1$}}}%
  \put(1220,1753){\makebox(0,0)[r]{\strut{}\ \ {$0.8$}}}%
  \put(1220,1011){\makebox(0,0)[r]{\strut{}\ \ {$0.6$}}}%
\end{picture}%
\endgroup
\endinput

\end	{center}
\caption{Effective masses for the lightest three states
  in the `pseudoscalar' $A_1^{-+}$ representation, for the double trace operators.
  Lower horizontal line indicates the mass of the lightest $A_1^{-+}$ glueball, and
  upper  horizontal line indicates the sum of the masses of the lightest
  $A_1^{++}$ and $A_1^{-+}$ glueballs.
  On the $26^326$ lattice at $\beta=6.235$ in SU(3).}
\label{fig_MeffGGA1-+l26n_SU3}
\end{figure}



%\begin{figure}[htb]
%\begin	{center}
%\leavevmode
%\input	{plot_MeffG+GGE-+l26n_SU3.tex}
%\end	{center}
%\caption{Effective masses for the lightest few glueballs in the 'pseudo-tensor' $E^{-+}$
%  representation, for the single trace operators ($\circ$) and for the same set
%  augmented with double trace operators (filled points), with points shifted for clarity.
%  The extra `scattering' state amongst the latter is either $\blacklozenge$
%  or $\blacktriangledown$ or a mixture of these. 
%  Horizontal line indicates  the sum of the masses of the lightest
%  $A_1^{++}$ and $E^{-+}$ glueballs.
%  On the $26^326$ lattice at $\beta=6.235$ in SU(3).}
%%\label{fig_MeffG+GGE-+l26_SU3}
%\end{figure}




%\begin{figure}[htb]
%\begin	{center}
%\leavevmode
%\input	{plot_MeffGGE-+l26n_SU3.tex}
%\end	{center}
%\caption{Effective masses for the lightest three states
%  in the 'pseudo-tensor' $E^{-+}$, for the double trace operators.
%  Lower horizontal line indicates the mass of the lightest $E^{-+}$ glueball, and
%  upper  horizontal line indicates the sum of the masses of the lightest
%  $A_1^{++}$ and $E^{-+}$ glueballs.
%  On the $26^326$ lattice at $\beta=6.235$ in SU(3).}
%\label{fig_MeffGGE-+l26_SU3}
%\end{figure}



\clearpage



\begin{figure}[htb]
\begin	{center}
\leavevmode
% GNUPLOT: LaTeX picture with Postscript
\begingroup%
\makeatletter%
\newcommand{\GNUPLOTspecial}{%
  \@sanitize\catcode`\%=14\relax\special}%
\setlength{\unitlength}{0.0500bp}%
\begin{picture}(9360,7560)(0,0)%
  {\GNUPLOTspecial{"
%!PS-Adobe-2.0 EPSF-2.0
%%Title: plot_Qcool_su5b17.63.tex
%%Creator: gnuplot 5.0 patchlevel 3
%%CreationDate: Thu Apr 29 13:45:53 2021
%%DocumentFonts: 
%%BoundingBox: 0 0 468 378
%%EndComments
%%BeginProlog
/gnudict 256 dict def
gnudict begin
%
% The following true/false flags may be edited by hand if desired.
% The unit line width and grayscale image gamma correction may also be changed.
%
/Color true def
/Blacktext true def
/Solid false def
/Dashlength 1 def
/Landscape false def
/Level1 false def
/Level3 false def
/Rounded false def
/ClipToBoundingBox false def
/SuppressPDFMark false def
/TransparentPatterns false def
/gnulinewidth 5.000 def
/userlinewidth gnulinewidth def
/Gamma 1.0 def
/BackgroundColor {-1.000 -1.000 -1.000} def
%
/vshift -66 def
/dl1 {
  10.0 Dashlength userlinewidth gnulinewidth div mul mul mul
  Rounded { currentlinewidth 0.75 mul sub dup 0 le { pop 0.01 } if } if
} def
/dl2 {
  10.0 Dashlength userlinewidth gnulinewidth div mul mul mul
  Rounded { currentlinewidth 0.75 mul add } if
} def
/hpt_ 31.5 def
/vpt_ 31.5 def
/hpt hpt_ def
/vpt vpt_ def
/doclip {
  ClipToBoundingBox {
    newpath 0 0 moveto 468 0 lineto 468 378 lineto 0 378 lineto closepath
    clip
  } if
} def
%
% Gnuplot Prolog Version 5.1 (Oct 2015)
%
%/SuppressPDFMark true def
%
/M {moveto} bind def
/L {lineto} bind def
/R {rmoveto} bind def
/V {rlineto} bind def
/N {newpath moveto} bind def
/Z {closepath} bind def
/C {setrgbcolor} bind def
/f {rlineto fill} bind def
/g {setgray} bind def
/Gshow {show} def   % May be redefined later in the file to support UTF-8
/vpt2 vpt 2 mul def
/hpt2 hpt 2 mul def
/Lshow {currentpoint stroke M 0 vshift R 
	Blacktext {gsave 0 setgray textshow grestore} {textshow} ifelse} def
/Rshow {currentpoint stroke M dup stringwidth pop neg vshift R
	Blacktext {gsave 0 setgray textshow grestore} {textshow} ifelse} def
/Cshow {currentpoint stroke M dup stringwidth pop -2 div vshift R 
	Blacktext {gsave 0 setgray textshow grestore} {textshow} ifelse} def
/UP {dup vpt_ mul /vpt exch def hpt_ mul /hpt exch def
  /hpt2 hpt 2 mul def /vpt2 vpt 2 mul def} def
/DL {Color {setrgbcolor Solid {pop []} if 0 setdash}
 {pop pop pop 0 setgray Solid {pop []} if 0 setdash} ifelse} def
/BL {stroke userlinewidth 2 mul setlinewidth
	Rounded {1 setlinejoin 1 setlinecap} if} def
/AL {stroke userlinewidth 2 div setlinewidth
	Rounded {1 setlinejoin 1 setlinecap} if} def
/UL {dup gnulinewidth mul /userlinewidth exch def
	dup 1 lt {pop 1} if 10 mul /udl exch def} def
/PL {stroke userlinewidth setlinewidth
	Rounded {1 setlinejoin 1 setlinecap} if} def
3.8 setmiterlimit
% Classic Line colors (version 5.0)
/LCw {1 1 1} def
/LCb {0 0 0} def
/LCa {0 0 0} def
/LC0 {1 0 0} def
/LC1 {0 1 0} def
/LC2 {0 0 1} def
/LC3 {1 0 1} def
/LC4 {0 1 1} def
/LC5 {1 1 0} def
/LC6 {0 0 0} def
/LC7 {1 0.3 0} def
/LC8 {0.5 0.5 0.5} def
% Default dash patterns (version 5.0)
/LTB {BL [] LCb DL} def
/LTw {PL [] 1 setgray} def
/LTb {PL [] LCb DL} def
/LTa {AL [1 udl mul 2 udl mul] 0 setdash LCa setrgbcolor} def
/LT0 {PL [] LC0 DL} def
/LT1 {PL [2 dl1 3 dl2] LC1 DL} def
/LT2 {PL [1 dl1 1.5 dl2] LC2 DL} def
/LT3 {PL [6 dl1 2 dl2 1 dl1 2 dl2] LC3 DL} def
/LT4 {PL [1 dl1 2 dl2 6 dl1 2 dl2 1 dl1 2 dl2] LC4 DL} def
/LT5 {PL [4 dl1 2 dl2] LC5 DL} def
/LT6 {PL [1.5 dl1 1.5 dl2 1.5 dl1 1.5 dl2 1.5 dl1 6 dl2] LC6 DL} def
/LT7 {PL [3 dl1 3 dl2 1 dl1 3 dl2] LC7 DL} def
/LT8 {PL [2 dl1 2 dl2 2 dl1 6 dl2] LC8 DL} def
/SL {[] 0 setdash} def
/Pnt {stroke [] 0 setdash gsave 1 setlinecap M 0 0 V stroke grestore} def
/Dia {stroke [] 0 setdash 2 copy vpt add M
  hpt neg vpt neg V hpt vpt neg V
  hpt vpt V hpt neg vpt V closepath stroke
  Pnt} def
/Pls {stroke [] 0 setdash vpt sub M 0 vpt2 V
  currentpoint stroke M
  hpt neg vpt neg R hpt2 0 V stroke
 } def
/Box {stroke [] 0 setdash 2 copy exch hpt sub exch vpt add M
  0 vpt2 neg V hpt2 0 V 0 vpt2 V
  hpt2 neg 0 V closepath stroke
  Pnt} def
/Crs {stroke [] 0 setdash exch hpt sub exch vpt add M
  hpt2 vpt2 neg V currentpoint stroke M
  hpt2 neg 0 R hpt2 vpt2 V stroke} def
/TriU {stroke [] 0 setdash 2 copy vpt 1.12 mul add M
  hpt neg vpt -1.62 mul V
  hpt 2 mul 0 V
  hpt neg vpt 1.62 mul V closepath stroke
  Pnt} def
/Star {2 copy Pls Crs} def
/BoxF {stroke [] 0 setdash exch hpt sub exch vpt add M
  0 vpt2 neg V hpt2 0 V 0 vpt2 V
  hpt2 neg 0 V closepath fill} def
/TriUF {stroke [] 0 setdash vpt 1.12 mul add M
  hpt neg vpt -1.62 mul V
  hpt 2 mul 0 V
  hpt neg vpt 1.62 mul V closepath fill} def
/TriD {stroke [] 0 setdash 2 copy vpt 1.12 mul sub M
  hpt neg vpt 1.62 mul V
  hpt 2 mul 0 V
  hpt neg vpt -1.62 mul V closepath stroke
  Pnt} def
/TriDF {stroke [] 0 setdash vpt 1.12 mul sub M
  hpt neg vpt 1.62 mul V
  hpt 2 mul 0 V
  hpt neg vpt -1.62 mul V closepath fill} def
/DiaF {stroke [] 0 setdash vpt add M
  hpt neg vpt neg V hpt vpt neg V
  hpt vpt V hpt neg vpt V closepath fill} def
/Pent {stroke [] 0 setdash 2 copy gsave
  translate 0 hpt M 4 {72 rotate 0 hpt L} repeat
  closepath stroke grestore Pnt} def
/PentF {stroke [] 0 setdash gsave
  translate 0 hpt M 4 {72 rotate 0 hpt L} repeat
  closepath fill grestore} def
/Circle {stroke [] 0 setdash 2 copy
  hpt 0 360 arc stroke Pnt} def
/CircleF {stroke [] 0 setdash hpt 0 360 arc fill} def
/C0 {BL [] 0 setdash 2 copy moveto vpt 90 450 arc} bind def
/C1 {BL [] 0 setdash 2 copy moveto
	2 copy vpt 0 90 arc closepath fill
	vpt 0 360 arc closepath} bind def
/C2 {BL [] 0 setdash 2 copy moveto
	2 copy vpt 90 180 arc closepath fill
	vpt 0 360 arc closepath} bind def
/C3 {BL [] 0 setdash 2 copy moveto
	2 copy vpt 0 180 arc closepath fill
	vpt 0 360 arc closepath} bind def
/C4 {BL [] 0 setdash 2 copy moveto
	2 copy vpt 180 270 arc closepath fill
	vpt 0 360 arc closepath} bind def
/C5 {BL [] 0 setdash 2 copy moveto
	2 copy vpt 0 90 arc
	2 copy moveto
	2 copy vpt 180 270 arc closepath fill
	vpt 0 360 arc} bind def
/C6 {BL [] 0 setdash 2 copy moveto
	2 copy vpt 90 270 arc closepath fill
	vpt 0 360 arc closepath} bind def
/C7 {BL [] 0 setdash 2 copy moveto
	2 copy vpt 0 270 arc closepath fill
	vpt 0 360 arc closepath} bind def
/C8 {BL [] 0 setdash 2 copy moveto
	2 copy vpt 270 360 arc closepath fill
	vpt 0 360 arc closepath} bind def
/C9 {BL [] 0 setdash 2 copy moveto
	2 copy vpt 270 450 arc closepath fill
	vpt 0 360 arc closepath} bind def
/C10 {BL [] 0 setdash 2 copy 2 copy moveto vpt 270 360 arc closepath fill
	2 copy moveto
	2 copy vpt 90 180 arc closepath fill
	vpt 0 360 arc closepath} bind def
/C11 {BL [] 0 setdash 2 copy moveto
	2 copy vpt 0 180 arc closepath fill
	2 copy moveto
	2 copy vpt 270 360 arc closepath fill
	vpt 0 360 arc closepath} bind def
/C12 {BL [] 0 setdash 2 copy moveto
	2 copy vpt 180 360 arc closepath fill
	vpt 0 360 arc closepath} bind def
/C13 {BL [] 0 setdash 2 copy moveto
	2 copy vpt 0 90 arc closepath fill
	2 copy moveto
	2 copy vpt 180 360 arc closepath fill
	vpt 0 360 arc closepath} bind def
/C14 {BL [] 0 setdash 2 copy moveto
	2 copy vpt 90 360 arc closepath fill
	vpt 0 360 arc} bind def
/C15 {BL [] 0 setdash 2 copy vpt 0 360 arc closepath fill
	vpt 0 360 arc closepath} bind def
/Rec {newpath 4 2 roll moveto 1 index 0 rlineto 0 exch rlineto
	neg 0 rlineto closepath} bind def
/Square {dup Rec} bind def
/Bsquare {vpt sub exch vpt sub exch vpt2 Square} bind def
/S0 {BL [] 0 setdash 2 copy moveto 0 vpt rlineto BL Bsquare} bind def
/S1 {BL [] 0 setdash 2 copy vpt Square fill Bsquare} bind def
/S2 {BL [] 0 setdash 2 copy exch vpt sub exch vpt Square fill Bsquare} bind def
/S3 {BL [] 0 setdash 2 copy exch vpt sub exch vpt2 vpt Rec fill Bsquare} bind def
/S4 {BL [] 0 setdash 2 copy exch vpt sub exch vpt sub vpt Square fill Bsquare} bind def
/S5 {BL [] 0 setdash 2 copy 2 copy vpt Square fill
	exch vpt sub exch vpt sub vpt Square fill Bsquare} bind def
/S6 {BL [] 0 setdash 2 copy exch vpt sub exch vpt sub vpt vpt2 Rec fill Bsquare} bind def
/S7 {BL [] 0 setdash 2 copy exch vpt sub exch vpt sub vpt vpt2 Rec fill
	2 copy vpt Square fill Bsquare} bind def
/S8 {BL [] 0 setdash 2 copy vpt sub vpt Square fill Bsquare} bind def
/S9 {BL [] 0 setdash 2 copy vpt sub vpt vpt2 Rec fill Bsquare} bind def
/S10 {BL [] 0 setdash 2 copy vpt sub vpt Square fill 2 copy exch vpt sub exch vpt Square fill
	Bsquare} bind def
/S11 {BL [] 0 setdash 2 copy vpt sub vpt Square fill 2 copy exch vpt sub exch vpt2 vpt Rec fill
	Bsquare} bind def
/S12 {BL [] 0 setdash 2 copy exch vpt sub exch vpt sub vpt2 vpt Rec fill Bsquare} bind def
/S13 {BL [] 0 setdash 2 copy exch vpt sub exch vpt sub vpt2 vpt Rec fill
	2 copy vpt Square fill Bsquare} bind def
/S14 {BL [] 0 setdash 2 copy exch vpt sub exch vpt sub vpt2 vpt Rec fill
	2 copy exch vpt sub exch vpt Square fill Bsquare} bind def
/S15 {BL [] 0 setdash 2 copy Bsquare fill Bsquare} bind def
/D0 {gsave translate 45 rotate 0 0 S0 stroke grestore} bind def
/D1 {gsave translate 45 rotate 0 0 S1 stroke grestore} bind def
/D2 {gsave translate 45 rotate 0 0 S2 stroke grestore} bind def
/D3 {gsave translate 45 rotate 0 0 S3 stroke grestore} bind def
/D4 {gsave translate 45 rotate 0 0 S4 stroke grestore} bind def
/D5 {gsave translate 45 rotate 0 0 S5 stroke grestore} bind def
/D6 {gsave translate 45 rotate 0 0 S6 stroke grestore} bind def
/D7 {gsave translate 45 rotate 0 0 S7 stroke grestore} bind def
/D8 {gsave translate 45 rotate 0 0 S8 stroke grestore} bind def
/D9 {gsave translate 45 rotate 0 0 S9 stroke grestore} bind def
/D10 {gsave translate 45 rotate 0 0 S10 stroke grestore} bind def
/D11 {gsave translate 45 rotate 0 0 S11 stroke grestore} bind def
/D12 {gsave translate 45 rotate 0 0 S12 stroke grestore} bind def
/D13 {gsave translate 45 rotate 0 0 S13 stroke grestore} bind def
/D14 {gsave translate 45 rotate 0 0 S14 stroke grestore} bind def
/D15 {gsave translate 45 rotate 0 0 S15 stroke grestore} bind def
/DiaE {stroke [] 0 setdash vpt add M
  hpt neg vpt neg V hpt vpt neg V
  hpt vpt V hpt neg vpt V closepath stroke} def
/BoxE {stroke [] 0 setdash exch hpt sub exch vpt add M
  0 vpt2 neg V hpt2 0 V 0 vpt2 V
  hpt2 neg 0 V closepath stroke} def
/TriUE {stroke [] 0 setdash vpt 1.12 mul add M
  hpt neg vpt -1.62 mul V
  hpt 2 mul 0 V
  hpt neg vpt 1.62 mul V closepath stroke} def
/TriDE {stroke [] 0 setdash vpt 1.12 mul sub M
  hpt neg vpt 1.62 mul V
  hpt 2 mul 0 V
  hpt neg vpt -1.62 mul V closepath stroke} def
/PentE {stroke [] 0 setdash gsave
  translate 0 hpt M 4 {72 rotate 0 hpt L} repeat
  closepath stroke grestore} def
/CircE {stroke [] 0 setdash 
  hpt 0 360 arc stroke} def
/Opaque {gsave closepath 1 setgray fill grestore 0 setgray closepath} def
/DiaW {stroke [] 0 setdash vpt add M
  hpt neg vpt neg V hpt vpt neg V
  hpt vpt V hpt neg vpt V Opaque stroke} def
/BoxW {stroke [] 0 setdash exch hpt sub exch vpt add M
  0 vpt2 neg V hpt2 0 V 0 vpt2 V
  hpt2 neg 0 V Opaque stroke} def
/TriUW {stroke [] 0 setdash vpt 1.12 mul add M
  hpt neg vpt -1.62 mul V
  hpt 2 mul 0 V
  hpt neg vpt 1.62 mul V Opaque stroke} def
/TriDW {stroke [] 0 setdash vpt 1.12 mul sub M
  hpt neg vpt 1.62 mul V
  hpt 2 mul 0 V
  hpt neg vpt -1.62 mul V Opaque stroke} def
/PentW {stroke [] 0 setdash gsave
  translate 0 hpt M 4 {72 rotate 0 hpt L} repeat
  Opaque stroke grestore} def
/CircW {stroke [] 0 setdash 
  hpt 0 360 arc Opaque stroke} def
/BoxFill {gsave Rec 1 setgray fill grestore} def
/Density {
  /Fillden exch def
  currentrgbcolor
  /ColB exch def /ColG exch def /ColR exch def
  /ColR ColR Fillden mul Fillden sub 1 add def
  /ColG ColG Fillden mul Fillden sub 1 add def
  /ColB ColB Fillden mul Fillden sub 1 add def
  ColR ColG ColB setrgbcolor} def
/BoxColFill {gsave Rec PolyFill} def
/PolyFill {gsave Density fill grestore grestore} def
/h {rlineto rlineto rlineto gsave closepath fill grestore} bind def
%
% PostScript Level 1 Pattern Fill routine for rectangles
% Usage: x y w h s a XX PatternFill
%	x,y = lower left corner of box to be filled
%	w,h = width and height of box
%	  a = angle in degrees between lines and x-axis
%	 XX = 0/1 for no/yes cross-hatch
%
/PatternFill {gsave /PFa [ 9 2 roll ] def
  PFa 0 get PFa 2 get 2 div add PFa 1 get PFa 3 get 2 div add translate
  PFa 2 get -2 div PFa 3 get -2 div PFa 2 get PFa 3 get Rec
  TransparentPatterns {} {gsave 1 setgray fill grestore} ifelse
  clip
  currentlinewidth 0.5 mul setlinewidth
  /PFs PFa 2 get dup mul PFa 3 get dup mul add sqrt def
  0 0 M PFa 5 get rotate PFs -2 div dup translate
  0 1 PFs PFa 4 get div 1 add floor cvi
	{PFa 4 get mul 0 M 0 PFs V} for
  0 PFa 6 get ne {
	0 1 PFs PFa 4 get div 1 add floor cvi
	{PFa 4 get mul 0 2 1 roll M PFs 0 V} for
 } if
  stroke grestore} def
%
/languagelevel where
 {pop languagelevel} {1} ifelse
dup 2 lt
	{/InterpretLevel1 true def
	 /InterpretLevel3 false def}
	{/InterpretLevel1 Level1 def
	 2 gt
	    {/InterpretLevel3 Level3 def}
	    {/InterpretLevel3 false def}
	 ifelse }
 ifelse
%
% PostScript level 2 pattern fill definitions
%
/Level2PatternFill {
/Tile8x8 {/PaintType 2 /PatternType 1 /TilingType 1 /BBox [0 0 8 8] /XStep 8 /YStep 8}
	bind def
/KeepColor {currentrgbcolor [/Pattern /DeviceRGB] setcolorspace} bind def
<< Tile8x8
 /PaintProc {0.5 setlinewidth pop 0 0 M 8 8 L 0 8 M 8 0 L stroke} 
>> matrix makepattern
/Pat1 exch def
<< Tile8x8
 /PaintProc {0.5 setlinewidth pop 0 0 M 8 8 L 0 8 M 8 0 L stroke
	0 4 M 4 8 L 8 4 L 4 0 L 0 4 L stroke}
>> matrix makepattern
/Pat2 exch def
<< Tile8x8
 /PaintProc {0.5 setlinewidth pop 0 0 M 0 8 L
	8 8 L 8 0 L 0 0 L fill}
>> matrix makepattern
/Pat3 exch def
<< Tile8x8
 /PaintProc {0.5 setlinewidth pop -4 8 M 8 -4 L
	0 12 M 12 0 L stroke}
>> matrix makepattern
/Pat4 exch def
<< Tile8x8
 /PaintProc {0.5 setlinewidth pop -4 0 M 8 12 L
	0 -4 M 12 8 L stroke}
>> matrix makepattern
/Pat5 exch def
<< Tile8x8
 /PaintProc {0.5 setlinewidth pop -2 8 M 4 -4 L
	0 12 M 8 -4 L 4 12 M 10 0 L stroke}
>> matrix makepattern
/Pat6 exch def
<< Tile8x8
 /PaintProc {0.5 setlinewidth pop -2 0 M 4 12 L
	0 -4 M 8 12 L 4 -4 M 10 8 L stroke}
>> matrix makepattern
/Pat7 exch def
<< Tile8x8
 /PaintProc {0.5 setlinewidth pop 8 -2 M -4 4 L
	12 0 M -4 8 L 12 4 M 0 10 L stroke}
>> matrix makepattern
/Pat8 exch def
<< Tile8x8
 /PaintProc {0.5 setlinewidth pop 0 -2 M 12 4 L
	-4 0 M 12 8 L -4 4 M 8 10 L stroke}
>> matrix makepattern
/Pat9 exch def
/Pattern1 {PatternBgnd KeepColor Pat1 setpattern} bind def
/Pattern2 {PatternBgnd KeepColor Pat2 setpattern} bind def
/Pattern3 {PatternBgnd KeepColor Pat3 setpattern} bind def
/Pattern4 {PatternBgnd KeepColor Landscape {Pat5} {Pat4} ifelse setpattern} bind def
/Pattern5 {PatternBgnd KeepColor Landscape {Pat4} {Pat5} ifelse setpattern} bind def
/Pattern6 {PatternBgnd KeepColor Landscape {Pat9} {Pat6} ifelse setpattern} bind def
/Pattern7 {PatternBgnd KeepColor Landscape {Pat8} {Pat7} ifelse setpattern} bind def
} def
%
%
%End of PostScript Level 2 code
%
/PatternBgnd {
  TransparentPatterns {} {gsave 1 setgray fill grestore} ifelse
} def
%
% Substitute for Level 2 pattern fill codes with
% grayscale if Level 2 support is not selected.
%
/Level1PatternFill {
/Pattern1 {0.250 Density} bind def
/Pattern2 {0.500 Density} bind def
/Pattern3 {0.750 Density} bind def
/Pattern4 {0.125 Density} bind def
/Pattern5 {0.375 Density} bind def
/Pattern6 {0.625 Density} bind def
/Pattern7 {0.875 Density} bind def
} def
%
% Now test for support of Level 2 code
%
Level1 {Level1PatternFill} {Level2PatternFill} ifelse
%
/Symbol-Oblique /Symbol findfont [1 0 .167 1 0 0] makefont
dup length dict begin {1 index /FID eq {pop pop} {def} ifelse} forall
currentdict end definefont pop
%
Level1 SuppressPDFMark or 
{} {
/SDict 10 dict def
systemdict /pdfmark known not {
  userdict /pdfmark systemdict /cleartomark get put
} if
SDict begin [
  /Title (plot_Qcool_su5b17.63.tex)
  /Subject (gnuplot plot)
  /Creator (gnuplot 5.0 patchlevel 3)
  /Author (mteper)
%  /Producer (gnuplot)
%  /Keywords ()
  /CreationDate (Thu Apr 29 13:45:53 2021)
  /DOCINFO pdfmark
end
} ifelse
%
% Support for boxed text - Ethan A Merritt May 2005
%
/InitTextBox { userdict /TBy2 3 -1 roll put userdict /TBx2 3 -1 roll put
           userdict /TBy1 3 -1 roll put userdict /TBx1 3 -1 roll put
	   /Boxing true def } def
/ExtendTextBox { Boxing
    { gsave dup false charpath pathbbox
      dup TBy2 gt {userdict /TBy2 3 -1 roll put} {pop} ifelse
      dup TBx2 gt {userdict /TBx2 3 -1 roll put} {pop} ifelse
      dup TBy1 lt {userdict /TBy1 3 -1 roll put} {pop} ifelse
      dup TBx1 lt {userdict /TBx1 3 -1 roll put} {pop} ifelse
      grestore } if } def
/PopTextBox { newpath TBx1 TBxmargin sub TBy1 TBymargin sub M
               TBx1 TBxmargin sub TBy2 TBymargin add L
	       TBx2 TBxmargin add TBy2 TBymargin add L
	       TBx2 TBxmargin add TBy1 TBymargin sub L closepath } def
/DrawTextBox { PopTextBox stroke /Boxing false def} def
/FillTextBox { gsave PopTextBox 1 1 1 setrgbcolor fill grestore /Boxing false def} def
0 0 0 0 InitTextBox
/TBxmargin 20 def
/TBymargin 20 def
/Boxing false def
/textshow { ExtendTextBox Gshow } def
%
% redundant definitions for compatibility with prologue.ps older than 5.0.2
/LTB {BL [] LCb DL} def
/LTb {PL [] LCb DL} def
end
%%EndProlog
%%Page: 1 1
gnudict begin
gsave
doclip
0 0 translate
0.050 0.050 scale
0 setgray
newpath
BackgroundColor 0 lt 3 1 roll 0 lt exch 0 lt or or not {BackgroundColor C 1.000 0 0 9360.00 7560.00 BoxColFill} if
1.000 UL
LTb
LCb setrgbcolor
1380 640 M
63 0 V
7555 0 R
-63 0 V
stroke
LTb
LCb setrgbcolor
1380 1308 M
63 0 V
7555 0 R
-63 0 V
stroke
LTb
LCb setrgbcolor
1380 1976 M
63 0 V
7555 0 R
-63 0 V
stroke
LTb
LCb setrgbcolor
1380 2644 M
63 0 V
7555 0 R
-63 0 V
stroke
LTb
LCb setrgbcolor
1380 3312 M
63 0 V
7555 0 R
-63 0 V
stroke
LTb
LCb setrgbcolor
1380 3980 M
63 0 V
7555 0 R
-63 0 V
stroke
LTb
LCb setrgbcolor
1380 4647 M
63 0 V
7555 0 R
-63 0 V
stroke
LTb
LCb setrgbcolor
1380 5315 M
63 0 V
7555 0 R
-63 0 V
stroke
LTb
LCb setrgbcolor
1380 5983 M
63 0 V
7555 0 R
-63 0 V
stroke
LTb
LCb setrgbcolor
1380 6651 M
63 0 V
7555 0 R
-63 0 V
stroke
LTb
LCb setrgbcolor
1380 7319 M
63 0 V
7555 0 R
-63 0 V
stroke
LTb
LCb setrgbcolor
1803 640 M
0 63 V
0 6616 R
0 -63 V
stroke
LTb
LCb setrgbcolor
2650 640 M
0 63 V
0 6616 R
0 -63 V
stroke
LTb
LCb setrgbcolor
3496 640 M
0 63 V
0 6616 R
0 -63 V
stroke
LTb
LCb setrgbcolor
4343 640 M
0 63 V
0 6616 R
0 -63 V
stroke
LTb
LCb setrgbcolor
5189 640 M
0 63 V
0 6616 R
0 -63 V
stroke
LTb
LCb setrgbcolor
6035 640 M
0 63 V
0 6616 R
0 -63 V
stroke
LTb
LCb setrgbcolor
6882 640 M
0 63 V
0 6616 R
0 -63 V
stroke
LTb
LCb setrgbcolor
7728 640 M
0 63 V
0 6616 R
0 -63 V
stroke
LTb
LCb setrgbcolor
8575 640 M
0 63 V
0 6616 R
0 -63 V
stroke
LTb
LCb setrgbcolor
1.000 UL
LTb
LCb setrgbcolor
1380 7319 N
0 -6679 V
7618 0 V
0 6679 V
-7618 0 V
Z stroke
1.000 UP
1.000 UL
LTb
LCb setrgbcolor
LCb setrgbcolor
LTb
LCb setrgbcolor
LTb
1.500 UP
1.000 UL
LTb
0.58 0.00 0.83 C 1465 640 M
0 1 V
42 -1 R
0 1 V
85 -1 R
0 1 V
42 1 R
0 3 V
42 -3 R
0 3 V
43 8 R
0 7 V
42 -4 R
0 7 V
42 13 R
0 11 V
43 -11 R
0 11 V
42 -21 R
0 8 V
42 -20 R
0 7 V
43 -15 R
0 5 V
42 -9 R
0 4 V
42 -6 R
0 1 V
169 -1 R
0 1 V
43 2 R
0 3 V
42 4 R
0 6 V
42 19 R
0 10 V
43 20 R
0 14 V
42 33 R
0 18 V
42 13 R
0 20 V
43 -27 R
0 20 V
42 -38 R
0 18 V
42 -66 R
0 14 V
43 -57 R
0 9 V
42 -25 R
0 6 V
42 -17 R
0 2 V
43 -1 R
0 2 V
84 -3 R
0 1 V
43 1 R
0 4 V
42 5 R
0 7 V
42 22 R
0 12 V
43 44 R
0 17 V
42 95 R
0 24 V
42 106 R
0 31 V
43 30 R
0 34 V
42 -25 R
0 34 V
42 -137 R
0 29 V
43 -159 R
0 22 V
42 -120 R
0 15 V
42 -62 R
0 10 V
42 -37 R
0 3 V
43 -5 R
0 3 V
42 -2 R
0 3 V
42 -3 R
0 3 V
43 5 R
0 6 V
42 4 R
0 8 V
42 47 R
0 15 V
43 116 R
0 24 V
42 199 R
0 34 V
42 182 R
0 42 V
43 120 R
0 47 V
42 -126 R
0 44 V
stroke 3877 1412 M
42 -238 R
0 39 V
43 -290 R
0 28 V
42 -197 R
0 18 V
42 -91 R
0 11 V
43 -40 R
0 6 V
42 -15 R
0 4 V
42 -5 R
0 3 V
43 -2 R
0 3 V
42 6 R
0 6 V
42 33 R
0 13 V
43 83 R
0 21 V
42 143 R
0 30 V
42 285 R
0 42 V
43 344 R
0 53 V
42 54 R
0 55 V
42 -122 R
0 54 V
42 -364 R
0 45 V
43 -397 R
0 32 V
42 -263 R
0 21 V
42 -126 R
0 12 V
43 -59 R
0 6 V
42 -10 R
0 3 V
42 -4 R
0 4 V
43 -4 R
0 3 V
42 11 R
0 7 V
42 33 R
0 13 V
43 88 R
0 21 V
42 223 R
0 34 V
42 284 R
0 45 V
43 352 R
0 56 V
42 63 R
0 58 V
42 -197 R
0 55 V
43 -427 R
0 44 V
42 -372 R
0 33 V
42 -267 R
0 21 V
43 -131 R
0 13 V
42 -50 R
0 7 V
42 -18 R
0 4 V
43 -6 R
0 4 V
42 -1 R
0 4 V
42 4 R
0 6 V
43 40 R
0 14 V
42 104 R
0 22 V
42 177 R
0 32 V
43 318 R
0 45 V
42 228 R
0 52 V
42 40 R
0 54 V
42 -218 R
0 51 V
43 -345 R
0 42 V
42 -347 R
0 30 V
42 -226 R
0 20 V
43 -103 R
0 12 V
stroke 6078 701 M
42 -53 R
0 5 V
42 -10 R
0 4 V
43 -7 R
0 1 V
42 2 R
0 3 V
42 4 R
0 6 V
43 15 R
0 10 V
42 64 R
0 17 V
42 62 R
0 22 V
43 172 R
0 33 V
42 85 R
0 37 V
42 16 R
0 39 V
43 -140 R
0 35 V
42 -220 R
0 27 V
42 -139 R
0 20 V
43 -105 R
0 14 V
42 -64 R
0 8 V
42 -21 R
0 4 V
43 -5 R
0 4 V
42 -5 R
0 2 V
42 -1 R
0 3 V
43 -2 R
0 4 V
42 -2 R
0 4 V
42 12 R
0 8 V
43 32 R
0 14 V
42 20 R
0 17 V
42 34 R
0 20 V
42 -13 R
0 21 V
43 -61 R
0 18 V
42 -60 R
0 14 V
42 -43 R
0 11 V
43 -40 R
0 6 V
42 -10 R
0 5 V
42 -14 R
0 1 V
127 -1 R
0 1 V
43 1 R
0 4 V
42 -4 R
0 3 V
42 4 R
0 6 V
43 9 R
0 9 V
42 11 R
0 11 V
42 2 R
0 13 V
43 1 R
0 14 V
42 -30 R
0 13 V
42 -39 R
0 10 V
43 -30 R
0 6 V
42 -8 R
0 5 V
42 -12 R
0 2 V
43 -2 R
0 1 V
127 -1 R
0 1 V
42 -1 R
0 2 V
42 1 R
0 3 V
42 -4 R
0 4 V
43 1 R
0 5 V
stroke 8448 652 M
42 -5 R
0 5 V
42 7 R
0 8 V
43 -14 R
0 7 V
42 -15 R
0 4 V
42 -6 R
0 3 V
43 -4 R
0 3 V
42 -5 R
0 2 V
42 -2 R
0 2 V
1465 641 Circle
1507 641 Circle
1592 641 Circle
1634 643 Circle
1676 643 Circle
1719 657 Circle
1761 659 Circle
1803 681 Circle
1846 681 Circle
1888 670 Circle
1930 658 Circle
1973 649 Circle
2015 644 Circle
2057 641 Circle
2226 641 Circle
2269 645 Circle
2311 653 Circle
2353 680 Circle
2396 712 Circle
2438 761 Circle
2480 793 Circle
2523 786 Circle
2565 767 Circle
2607 717 Circle
2650 671 Circle
2692 654 Circle
2734 641 Circle
2777 642 Circle
2861 641 Circle
2904 644 Circle
2946 655 Circle
2988 686 Circle
3031 745 Circle
3073 860 Circle
3115 993 Circle
3158 1056 Circle
3200 1065 Circle
3242 960 Circle
3285 826 Circle
3327 725 Circle
3369 675 Circle
3411 645 Circle
3454 643 Circle
3496 643 Circle
3538 643 Circle
3581 653 Circle
3623 664 Circle
3665 723 Circle
3708 858 Circle
3750 1086 Circle
3792 1306 Circle
3835 1470 Circle
3877 1390 Circle
3919 1194 Circle
3962 937 Circle
4004 763 Circle
4046 687 Circle
4089 655 Circle
4131 645 Circle
4173 643 Circle
4216 645 Circle
4258 655 Circle
4300 697 Circle
4343 798 Circle
4385 966 Circle
4427 1287 Circle
4470 1679 Circle
4512 1787 Circle
4554 1719 Circle
4596 1405 Circle
4639 1046 Circle
4681 810 Circle
4723 700 Circle
4766 650 Circle
4808 645 Circle
4850 644 Circle
4893 643 Circle
4935 659 Circle
4977 703 Circle
5020 808 Circle
5062 1058 Circle
5104 1381 Circle
5147 1784 Circle
5189 1904 Circle
5231 1763 Circle
5274 1386 Circle
5316 1053 Circle
5358 812 Circle
5401 699 Circle
5443 659 Circle
5485 646 Circle
5528 644 Circle
5570 647 Circle
5612 656 Circle
5655 706 Circle
5697 828 Circle
5739 1032 Circle
5782 1389 Circle
5824 1665 Circle
5866 1758 Circle
5908 1592 Circle
5951 1294 Circle
5993 983 Circle
6035 782 Circle
6078 695 Circle
6120 651 Circle
6162 645 Circle
6205 641 Circle
6247 645 Circle
6289 653 Circle
6332 676 Circle
6374 753 Circle
6416 835 Circle
6459 1035 Circle
6501 1155 Circle
6543 1208 Circle
6586 1105 Circle
6628 916 Circle
6670 801 Circle
6713 713 Circle
6755 660 Circle
6797 645 Circle
6840 644 Circle
6882 642 Circle
6924 643 Circle
6967 645 Circle
7009 647 Circle
7051 665 Circle
7094 708 Circle
7136 743 Circle
7178 796 Circle
7220 804 Circle
7263 762 Circle
7305 718 Circle
7347 687 Circle
7390 656 Circle
7432 651 Circle
7474 641 Circle
7601 641 Circle
7644 644 Circle
7686 643 Circle
7728 652 Circle
7771 669 Circle
7813 689 Circle
7855 703 Circle
7898 718 Circle
7940 701 Circle
7982 674 Circle
8025 652 Circle
8067 649 Circle
8109 641 Circle
8152 641 Circle
8279 641 Circle
8321 641 Circle
8363 645 Circle
8405 644 Circle
8448 649 Circle
8490 649 Circle
8532 663 Circle
8575 657 Circle
8617 647 Circle
8659 645 Circle
8702 643 Circle
8744 641 Circle
8786 641 Circle
1.500 UP
1.000 UL
LTb
0.58 0.00 0.83 C 1930 815 M
0 22 V
43 346 R
0 39 V
42 -538 R
0 11 V
42 -53 R
0 3 V
127 -5 R
0 1 V
381 -1 R
0 1 V
85 -1 R
0 2 V
42 -2 R
0 2 V
42 342 R
0 31 V
43 1354 R
0 69 V
2819 719 M
0 15 V
42 -89 R
0 4 V
43 -9 R
0 1 V
42 0 R
0 2 V
550 -3 R
0 2 V
42 437 R
0 35 V
43 2909 R
0 96 V
3623 807 M
0 22 V
42 -182 R
0 5 V
43 -11 R
0 2 V
42 -3 R
0 1 V
339 -1 R
0 1 V
127 -1 R
0 1 V
42 -1 R
0 1 V
42 4 R
0 4 V
43 364 R
0 32 V
42 4763 R
0 118 V
4427 882 M
0 26 V
43 -263 R
0 4 V
42 -8 R
0 2 V
42 -3 R
0 1 V
85 -1 R
0 1 V
296 -1 R
0 1 V
127 -1 R
0 1 V
42 6 R
0 5 V
43 297 R
0 29 V
42 5307 R
0 123 V
5231 936 M
0 29 V
43 -320 R
0 4 V
42 -8 R
0 3 V
42 -4 R
0 1 V
127 -1 R
0 1 V
339 -1 R
0 1 V
42 -1 R
0 2 V
42 7 R
0 6 V
43 229 R
0 26 V
42 4757 R
0 117 V
6035 996 M
0 31 V
43 -385 R
0 4 V
635 -2 R
0 4 V
42 101 R
0 17 V
42 2246 R
0 81 V
6840 933 M
0 29 V
stroke 6840 962 M
7305 640 M
0 1 V
85 -1 R
0 1 V
127 0 R
0 2 V
42 27 R
0 9 V
42 553 R
0 40 V
43 -485 R
0 20 V
7771 640 M
0 1 V
550 -1 R
0 1 V
42 12 R
0 7 V
42 208 R
0 26 V
43 -163 R
0 16 V
42 -107 R
0 2 V
1930 826 CircleF
1973 1202 CircleF
2015 689 CircleF
2057 643 CircleF
2184 641 CircleF
2565 641 CircleF
2650 641 CircleF
2692 641 CircleF
2734 1000 CircleF
2777 2403 CircleF
2819 726 CircleF
2861 647 CircleF
2904 641 CircleF
2946 642 CircleF
3496 641 CircleF
3538 1096 CircleF
3581 4071 CircleF
3623 818 CircleF
3665 649 CircleF
3708 642 CircleF
3750 641 CircleF
4089 641 CircleF
4216 641 CircleF
4258 641 CircleF
4300 647 CircleF
4343 1029 CircleF
4385 5867 CircleF
4427 895 CircleF
4470 647 CircleF
4512 642 CircleF
4554 641 CircleF
4639 641 CircleF
4935 641 CircleF
5062 641 CircleF
5104 649 CircleF
5147 963 CircleF
5189 6346 CircleF
5231 951 CircleF
5274 647 CircleF
5316 643 CircleF
5358 641 CircleF
5485 641 CircleF
5824 641 CircleF
5866 641 CircleF
5908 652 CircleF
5951 897 CircleF
5993 5725 CircleF
6035 1011 CircleF
6078 644 CircleF
6713 646 CircleF
6755 757 CircleF
6797 3052 CircleF
6840 948 CircleF
7305 641 CircleF
7390 641 CircleF
7517 642 CircleF
7559 675 CircleF
7601 1252 CircleF
7644 797 CircleF
7771 641 CircleF
8321 641 CircleF
8363 657 CircleF
8405 881 CircleF
8448 739 CircleF
8490 641 CircleF
2.000 UL
LTb
LCb setrgbcolor
1.000 UL
LTb
LCb setrgbcolor
1380 7319 N
0 -6679 V
7618 0 V
0 6679 V
-7618 0 V
Z stroke
1.000 UP
1.000 UL
LTb
LCb setrgbcolor
stroke
grestore
end
showpage
  }}%
  \put(5189,140){\makebox(0,0){\large{$Q_L$}}}%
  \put(200,4979){\makebox(0,0){\Large{$N(Q_L)$}}}%
  \put(8575,440){\makebox(0,0){\strut{}\ {$4$}}}%
  \put(7728,440){\makebox(0,0){\strut{}\ {$3$}}}%
  \put(6882,440){\makebox(0,0){\strut{}\ {$2$}}}%
  \put(6035,440){\makebox(0,0){\strut{}\ {$1$}}}%
  \put(5189,440){\makebox(0,0){\strut{}\ {$0$}}}%
  \put(4343,440){\makebox(0,0){\strut{}\ {$-1$}}}%
  \put(3496,440){\makebox(0,0){\strut{}\ {$-2$}}}%
  \put(2650,440){\makebox(0,0){\strut{}\ {$-3$}}}%
  \put(1803,440){\makebox(0,0){\strut{}\ {$-4$}}}%
  \put(1260,7319){\makebox(0,0)[r]{\strut{}\ \ {$10000$}}}%
  \put(1260,6651){\makebox(0,0)[r]{\strut{}\ \ {$9000$}}}%
  \put(1260,5983){\makebox(0,0)[r]{\strut{}\ \ {$8000$}}}%
  \put(1260,5315){\makebox(0,0)[r]{\strut{}\ \ {$7000$}}}%
  \put(1260,4647){\makebox(0,0)[r]{\strut{}\ \ {$6000$}}}%
  \put(1260,3980){\makebox(0,0)[r]{\strut{}\ \ {$5000$}}}%
  \put(1260,3312){\makebox(0,0)[r]{\strut{}\ \ {$4000$}}}%
  \put(1260,2644){\makebox(0,0)[r]{\strut{}\ \ {$3000$}}}%
  \put(1260,1976){\makebox(0,0)[r]{\strut{}\ \ {$2000$}}}%
  \put(1260,1308){\makebox(0,0)[r]{\strut{}\ \ {$1000$}}}%
  \put(1260,640){\makebox(0,0)[r]{\strut{}\ \ {$0$}}}%
\end{picture}%
\endgroup
\endinput

\end	{center}
\caption{The number of lattice fields with topological charge $Q_L$ after 2 ($\circ$) and after 20 ($\bullet$)
cooling sweeps, from sequences of $SU(5)$ fields generated at $\beta=17.63$. $N(Q_L)=0$ points suppressed.}
\label{fig_Qcool20_su5}
\end{figure}


\begin{figure}[htb]
\begin	{center}
\leavevmode
% GNUPLOT: LaTeX picture with Postscript
\begingroup%
\makeatletter%
\newcommand{\GNUPLOTspecial}{%
  \@sanitize\catcode`\%=14\relax\special}%
\setlength{\unitlength}{0.0500bp}%
\begin{picture}(9360,7560)(0,0)%
  {\GNUPLOTspecial{"
%!PS-Adobe-2.0 EPSF-2.0
%%Title: plot_Qcool_su8b47.75.tex
%%Creator: gnuplot 5.0 patchlevel 3
%%CreationDate: Thu Apr 29 13:46:56 2021
%%DocumentFonts: 
%%BoundingBox: 0 0 468 378
%%EndComments
%%BeginProlog
/gnudict 256 dict def
gnudict begin
%
% The following true/false flags may be edited by hand if desired.
% The unit line width and grayscale image gamma correction may also be changed.
%
/Color true def
/Blacktext true def
/Solid false def
/Dashlength 1 def
/Landscape false def
/Level1 false def
/Level3 false def
/Rounded false def
/ClipToBoundingBox false def
/SuppressPDFMark false def
/TransparentPatterns false def
/gnulinewidth 5.000 def
/userlinewidth gnulinewidth def
/Gamma 1.0 def
/BackgroundColor {-1.000 -1.000 -1.000} def
%
/vshift -66 def
/dl1 {
  10.0 Dashlength userlinewidth gnulinewidth div mul mul mul
  Rounded { currentlinewidth 0.75 mul sub dup 0 le { pop 0.01 } if } if
} def
/dl2 {
  10.0 Dashlength userlinewidth gnulinewidth div mul mul mul
  Rounded { currentlinewidth 0.75 mul add } if
} def
/hpt_ 31.5 def
/vpt_ 31.5 def
/hpt hpt_ def
/vpt vpt_ def
/doclip {
  ClipToBoundingBox {
    newpath 0 0 moveto 468 0 lineto 468 378 lineto 0 378 lineto closepath
    clip
  } if
} def
%
% Gnuplot Prolog Version 5.1 (Oct 2015)
%
%/SuppressPDFMark true def
%
/M {moveto} bind def
/L {lineto} bind def
/R {rmoveto} bind def
/V {rlineto} bind def
/N {newpath moveto} bind def
/Z {closepath} bind def
/C {setrgbcolor} bind def
/f {rlineto fill} bind def
/g {setgray} bind def
/Gshow {show} def   % May be redefined later in the file to support UTF-8
/vpt2 vpt 2 mul def
/hpt2 hpt 2 mul def
/Lshow {currentpoint stroke M 0 vshift R 
	Blacktext {gsave 0 setgray textshow grestore} {textshow} ifelse} def
/Rshow {currentpoint stroke M dup stringwidth pop neg vshift R
	Blacktext {gsave 0 setgray textshow grestore} {textshow} ifelse} def
/Cshow {currentpoint stroke M dup stringwidth pop -2 div vshift R 
	Blacktext {gsave 0 setgray textshow grestore} {textshow} ifelse} def
/UP {dup vpt_ mul /vpt exch def hpt_ mul /hpt exch def
  /hpt2 hpt 2 mul def /vpt2 vpt 2 mul def} def
/DL {Color {setrgbcolor Solid {pop []} if 0 setdash}
 {pop pop pop 0 setgray Solid {pop []} if 0 setdash} ifelse} def
/BL {stroke userlinewidth 2 mul setlinewidth
	Rounded {1 setlinejoin 1 setlinecap} if} def
/AL {stroke userlinewidth 2 div setlinewidth
	Rounded {1 setlinejoin 1 setlinecap} if} def
/UL {dup gnulinewidth mul /userlinewidth exch def
	dup 1 lt {pop 1} if 10 mul /udl exch def} def
/PL {stroke userlinewidth setlinewidth
	Rounded {1 setlinejoin 1 setlinecap} if} def
3.8 setmiterlimit
% Classic Line colors (version 5.0)
/LCw {1 1 1} def
/LCb {0 0 0} def
/LCa {0 0 0} def
/LC0 {1 0 0} def
/LC1 {0 1 0} def
/LC2 {0 0 1} def
/LC3 {1 0 1} def
/LC4 {0 1 1} def
/LC5 {1 1 0} def
/LC6 {0 0 0} def
/LC7 {1 0.3 0} def
/LC8 {0.5 0.5 0.5} def
% Default dash patterns (version 5.0)
/LTB {BL [] LCb DL} def
/LTw {PL [] 1 setgray} def
/LTb {PL [] LCb DL} def
/LTa {AL [1 udl mul 2 udl mul] 0 setdash LCa setrgbcolor} def
/LT0 {PL [] LC0 DL} def
/LT1 {PL [2 dl1 3 dl2] LC1 DL} def
/LT2 {PL [1 dl1 1.5 dl2] LC2 DL} def
/LT3 {PL [6 dl1 2 dl2 1 dl1 2 dl2] LC3 DL} def
/LT4 {PL [1 dl1 2 dl2 6 dl1 2 dl2 1 dl1 2 dl2] LC4 DL} def
/LT5 {PL [4 dl1 2 dl2] LC5 DL} def
/LT6 {PL [1.5 dl1 1.5 dl2 1.5 dl1 1.5 dl2 1.5 dl1 6 dl2] LC6 DL} def
/LT7 {PL [3 dl1 3 dl2 1 dl1 3 dl2] LC7 DL} def
/LT8 {PL [2 dl1 2 dl2 2 dl1 6 dl2] LC8 DL} def
/SL {[] 0 setdash} def
/Pnt {stroke [] 0 setdash gsave 1 setlinecap M 0 0 V stroke grestore} def
/Dia {stroke [] 0 setdash 2 copy vpt add M
  hpt neg vpt neg V hpt vpt neg V
  hpt vpt V hpt neg vpt V closepath stroke
  Pnt} def
/Pls {stroke [] 0 setdash vpt sub M 0 vpt2 V
  currentpoint stroke M
  hpt neg vpt neg R hpt2 0 V stroke
 } def
/Box {stroke [] 0 setdash 2 copy exch hpt sub exch vpt add M
  0 vpt2 neg V hpt2 0 V 0 vpt2 V
  hpt2 neg 0 V closepath stroke
  Pnt} def
/Crs {stroke [] 0 setdash exch hpt sub exch vpt add M
  hpt2 vpt2 neg V currentpoint stroke M
  hpt2 neg 0 R hpt2 vpt2 V stroke} def
/TriU {stroke [] 0 setdash 2 copy vpt 1.12 mul add M
  hpt neg vpt -1.62 mul V
  hpt 2 mul 0 V
  hpt neg vpt 1.62 mul V closepath stroke
  Pnt} def
/Star {2 copy Pls Crs} def
/BoxF {stroke [] 0 setdash exch hpt sub exch vpt add M
  0 vpt2 neg V hpt2 0 V 0 vpt2 V
  hpt2 neg 0 V closepath fill} def
/TriUF {stroke [] 0 setdash vpt 1.12 mul add M
  hpt neg vpt -1.62 mul V
  hpt 2 mul 0 V
  hpt neg vpt 1.62 mul V closepath fill} def
/TriD {stroke [] 0 setdash 2 copy vpt 1.12 mul sub M
  hpt neg vpt 1.62 mul V
  hpt 2 mul 0 V
  hpt neg vpt -1.62 mul V closepath stroke
  Pnt} def
/TriDF {stroke [] 0 setdash vpt 1.12 mul sub M
  hpt neg vpt 1.62 mul V
  hpt 2 mul 0 V
  hpt neg vpt -1.62 mul V closepath fill} def
/DiaF {stroke [] 0 setdash vpt add M
  hpt neg vpt neg V hpt vpt neg V
  hpt vpt V hpt neg vpt V closepath fill} def
/Pent {stroke [] 0 setdash 2 copy gsave
  translate 0 hpt M 4 {72 rotate 0 hpt L} repeat
  closepath stroke grestore Pnt} def
/PentF {stroke [] 0 setdash gsave
  translate 0 hpt M 4 {72 rotate 0 hpt L} repeat
  closepath fill grestore} def
/Circle {stroke [] 0 setdash 2 copy
  hpt 0 360 arc stroke Pnt} def
/CircleF {stroke [] 0 setdash hpt 0 360 arc fill} def
/C0 {BL [] 0 setdash 2 copy moveto vpt 90 450 arc} bind def
/C1 {BL [] 0 setdash 2 copy moveto
	2 copy vpt 0 90 arc closepath fill
	vpt 0 360 arc closepath} bind def
/C2 {BL [] 0 setdash 2 copy moveto
	2 copy vpt 90 180 arc closepath fill
	vpt 0 360 arc closepath} bind def
/C3 {BL [] 0 setdash 2 copy moveto
	2 copy vpt 0 180 arc closepath fill
	vpt 0 360 arc closepath} bind def
/C4 {BL [] 0 setdash 2 copy moveto
	2 copy vpt 180 270 arc closepath fill
	vpt 0 360 arc closepath} bind def
/C5 {BL [] 0 setdash 2 copy moveto
	2 copy vpt 0 90 arc
	2 copy moveto
	2 copy vpt 180 270 arc closepath fill
	vpt 0 360 arc} bind def
/C6 {BL [] 0 setdash 2 copy moveto
	2 copy vpt 90 270 arc closepath fill
	vpt 0 360 arc closepath} bind def
/C7 {BL [] 0 setdash 2 copy moveto
	2 copy vpt 0 270 arc closepath fill
	vpt 0 360 arc closepath} bind def
/C8 {BL [] 0 setdash 2 copy moveto
	2 copy vpt 270 360 arc closepath fill
	vpt 0 360 arc closepath} bind def
/C9 {BL [] 0 setdash 2 copy moveto
	2 copy vpt 270 450 arc closepath fill
	vpt 0 360 arc closepath} bind def
/C10 {BL [] 0 setdash 2 copy 2 copy moveto vpt 270 360 arc closepath fill
	2 copy moveto
	2 copy vpt 90 180 arc closepath fill
	vpt 0 360 arc closepath} bind def
/C11 {BL [] 0 setdash 2 copy moveto
	2 copy vpt 0 180 arc closepath fill
	2 copy moveto
	2 copy vpt 270 360 arc closepath fill
	vpt 0 360 arc closepath} bind def
/C12 {BL [] 0 setdash 2 copy moveto
	2 copy vpt 180 360 arc closepath fill
	vpt 0 360 arc closepath} bind def
/C13 {BL [] 0 setdash 2 copy moveto
	2 copy vpt 0 90 arc closepath fill
	2 copy moveto
	2 copy vpt 180 360 arc closepath fill
	vpt 0 360 arc closepath} bind def
/C14 {BL [] 0 setdash 2 copy moveto
	2 copy vpt 90 360 arc closepath fill
	vpt 0 360 arc} bind def
/C15 {BL [] 0 setdash 2 copy vpt 0 360 arc closepath fill
	vpt 0 360 arc closepath} bind def
/Rec {newpath 4 2 roll moveto 1 index 0 rlineto 0 exch rlineto
	neg 0 rlineto closepath} bind def
/Square {dup Rec} bind def
/Bsquare {vpt sub exch vpt sub exch vpt2 Square} bind def
/S0 {BL [] 0 setdash 2 copy moveto 0 vpt rlineto BL Bsquare} bind def
/S1 {BL [] 0 setdash 2 copy vpt Square fill Bsquare} bind def
/S2 {BL [] 0 setdash 2 copy exch vpt sub exch vpt Square fill Bsquare} bind def
/S3 {BL [] 0 setdash 2 copy exch vpt sub exch vpt2 vpt Rec fill Bsquare} bind def
/S4 {BL [] 0 setdash 2 copy exch vpt sub exch vpt sub vpt Square fill Bsquare} bind def
/S5 {BL [] 0 setdash 2 copy 2 copy vpt Square fill
	exch vpt sub exch vpt sub vpt Square fill Bsquare} bind def
/S6 {BL [] 0 setdash 2 copy exch vpt sub exch vpt sub vpt vpt2 Rec fill Bsquare} bind def
/S7 {BL [] 0 setdash 2 copy exch vpt sub exch vpt sub vpt vpt2 Rec fill
	2 copy vpt Square fill Bsquare} bind def
/S8 {BL [] 0 setdash 2 copy vpt sub vpt Square fill Bsquare} bind def
/S9 {BL [] 0 setdash 2 copy vpt sub vpt vpt2 Rec fill Bsquare} bind def
/S10 {BL [] 0 setdash 2 copy vpt sub vpt Square fill 2 copy exch vpt sub exch vpt Square fill
	Bsquare} bind def
/S11 {BL [] 0 setdash 2 copy vpt sub vpt Square fill 2 copy exch vpt sub exch vpt2 vpt Rec fill
	Bsquare} bind def
/S12 {BL [] 0 setdash 2 copy exch vpt sub exch vpt sub vpt2 vpt Rec fill Bsquare} bind def
/S13 {BL [] 0 setdash 2 copy exch vpt sub exch vpt sub vpt2 vpt Rec fill
	2 copy vpt Square fill Bsquare} bind def
/S14 {BL [] 0 setdash 2 copy exch vpt sub exch vpt sub vpt2 vpt Rec fill
	2 copy exch vpt sub exch vpt Square fill Bsquare} bind def
/S15 {BL [] 0 setdash 2 copy Bsquare fill Bsquare} bind def
/D0 {gsave translate 45 rotate 0 0 S0 stroke grestore} bind def
/D1 {gsave translate 45 rotate 0 0 S1 stroke grestore} bind def
/D2 {gsave translate 45 rotate 0 0 S2 stroke grestore} bind def
/D3 {gsave translate 45 rotate 0 0 S3 stroke grestore} bind def
/D4 {gsave translate 45 rotate 0 0 S4 stroke grestore} bind def
/D5 {gsave translate 45 rotate 0 0 S5 stroke grestore} bind def
/D6 {gsave translate 45 rotate 0 0 S6 stroke grestore} bind def
/D7 {gsave translate 45 rotate 0 0 S7 stroke grestore} bind def
/D8 {gsave translate 45 rotate 0 0 S8 stroke grestore} bind def
/D9 {gsave translate 45 rotate 0 0 S9 stroke grestore} bind def
/D10 {gsave translate 45 rotate 0 0 S10 stroke grestore} bind def
/D11 {gsave translate 45 rotate 0 0 S11 stroke grestore} bind def
/D12 {gsave translate 45 rotate 0 0 S12 stroke grestore} bind def
/D13 {gsave translate 45 rotate 0 0 S13 stroke grestore} bind def
/D14 {gsave translate 45 rotate 0 0 S14 stroke grestore} bind def
/D15 {gsave translate 45 rotate 0 0 S15 stroke grestore} bind def
/DiaE {stroke [] 0 setdash vpt add M
  hpt neg vpt neg V hpt vpt neg V
  hpt vpt V hpt neg vpt V closepath stroke} def
/BoxE {stroke [] 0 setdash exch hpt sub exch vpt add M
  0 vpt2 neg V hpt2 0 V 0 vpt2 V
  hpt2 neg 0 V closepath stroke} def
/TriUE {stroke [] 0 setdash vpt 1.12 mul add M
  hpt neg vpt -1.62 mul V
  hpt 2 mul 0 V
  hpt neg vpt 1.62 mul V closepath stroke} def
/TriDE {stroke [] 0 setdash vpt 1.12 mul sub M
  hpt neg vpt 1.62 mul V
  hpt 2 mul 0 V
  hpt neg vpt -1.62 mul V closepath stroke} def
/PentE {stroke [] 0 setdash gsave
  translate 0 hpt M 4 {72 rotate 0 hpt L} repeat
  closepath stroke grestore} def
/CircE {stroke [] 0 setdash 
  hpt 0 360 arc stroke} def
/Opaque {gsave closepath 1 setgray fill grestore 0 setgray closepath} def
/DiaW {stroke [] 0 setdash vpt add M
  hpt neg vpt neg V hpt vpt neg V
  hpt vpt V hpt neg vpt V Opaque stroke} def
/BoxW {stroke [] 0 setdash exch hpt sub exch vpt add M
  0 vpt2 neg V hpt2 0 V 0 vpt2 V
  hpt2 neg 0 V Opaque stroke} def
/TriUW {stroke [] 0 setdash vpt 1.12 mul add M
  hpt neg vpt -1.62 mul V
  hpt 2 mul 0 V
  hpt neg vpt 1.62 mul V Opaque stroke} def
/TriDW {stroke [] 0 setdash vpt 1.12 mul sub M
  hpt neg vpt 1.62 mul V
  hpt 2 mul 0 V
  hpt neg vpt -1.62 mul V Opaque stroke} def
/PentW {stroke [] 0 setdash gsave
  translate 0 hpt M 4 {72 rotate 0 hpt L} repeat
  Opaque stroke grestore} def
/CircW {stroke [] 0 setdash 
  hpt 0 360 arc Opaque stroke} def
/BoxFill {gsave Rec 1 setgray fill grestore} def
/Density {
  /Fillden exch def
  currentrgbcolor
  /ColB exch def /ColG exch def /ColR exch def
  /ColR ColR Fillden mul Fillden sub 1 add def
  /ColG ColG Fillden mul Fillden sub 1 add def
  /ColB ColB Fillden mul Fillden sub 1 add def
  ColR ColG ColB setrgbcolor} def
/BoxColFill {gsave Rec PolyFill} def
/PolyFill {gsave Density fill grestore grestore} def
/h {rlineto rlineto rlineto gsave closepath fill grestore} bind def
%
% PostScript Level 1 Pattern Fill routine for rectangles
% Usage: x y w h s a XX PatternFill
%	x,y = lower left corner of box to be filled
%	w,h = width and height of box
%	  a = angle in degrees between lines and x-axis
%	 XX = 0/1 for no/yes cross-hatch
%
/PatternFill {gsave /PFa [ 9 2 roll ] def
  PFa 0 get PFa 2 get 2 div add PFa 1 get PFa 3 get 2 div add translate
  PFa 2 get -2 div PFa 3 get -2 div PFa 2 get PFa 3 get Rec
  TransparentPatterns {} {gsave 1 setgray fill grestore} ifelse
  clip
  currentlinewidth 0.5 mul setlinewidth
  /PFs PFa 2 get dup mul PFa 3 get dup mul add sqrt def
  0 0 M PFa 5 get rotate PFs -2 div dup translate
  0 1 PFs PFa 4 get div 1 add floor cvi
	{PFa 4 get mul 0 M 0 PFs V} for
  0 PFa 6 get ne {
	0 1 PFs PFa 4 get div 1 add floor cvi
	{PFa 4 get mul 0 2 1 roll M PFs 0 V} for
 } if
  stroke grestore} def
%
/languagelevel where
 {pop languagelevel} {1} ifelse
dup 2 lt
	{/InterpretLevel1 true def
	 /InterpretLevel3 false def}
	{/InterpretLevel1 Level1 def
	 2 gt
	    {/InterpretLevel3 Level3 def}
	    {/InterpretLevel3 false def}
	 ifelse }
 ifelse
%
% PostScript level 2 pattern fill definitions
%
/Level2PatternFill {
/Tile8x8 {/PaintType 2 /PatternType 1 /TilingType 1 /BBox [0 0 8 8] /XStep 8 /YStep 8}
	bind def
/KeepColor {currentrgbcolor [/Pattern /DeviceRGB] setcolorspace} bind def
<< Tile8x8
 /PaintProc {0.5 setlinewidth pop 0 0 M 8 8 L 0 8 M 8 0 L stroke} 
>> matrix makepattern
/Pat1 exch def
<< Tile8x8
 /PaintProc {0.5 setlinewidth pop 0 0 M 8 8 L 0 8 M 8 0 L stroke
	0 4 M 4 8 L 8 4 L 4 0 L 0 4 L stroke}
>> matrix makepattern
/Pat2 exch def
<< Tile8x8
 /PaintProc {0.5 setlinewidth pop 0 0 M 0 8 L
	8 8 L 8 0 L 0 0 L fill}
>> matrix makepattern
/Pat3 exch def
<< Tile8x8
 /PaintProc {0.5 setlinewidth pop -4 8 M 8 -4 L
	0 12 M 12 0 L stroke}
>> matrix makepattern
/Pat4 exch def
<< Tile8x8
 /PaintProc {0.5 setlinewidth pop -4 0 M 8 12 L
	0 -4 M 12 8 L stroke}
>> matrix makepattern
/Pat5 exch def
<< Tile8x8
 /PaintProc {0.5 setlinewidth pop -2 8 M 4 -4 L
	0 12 M 8 -4 L 4 12 M 10 0 L stroke}
>> matrix makepattern
/Pat6 exch def
<< Tile8x8
 /PaintProc {0.5 setlinewidth pop -2 0 M 4 12 L
	0 -4 M 8 12 L 4 -4 M 10 8 L stroke}
>> matrix makepattern
/Pat7 exch def
<< Tile8x8
 /PaintProc {0.5 setlinewidth pop 8 -2 M -4 4 L
	12 0 M -4 8 L 12 4 M 0 10 L stroke}
>> matrix makepattern
/Pat8 exch def
<< Tile8x8
 /PaintProc {0.5 setlinewidth pop 0 -2 M 12 4 L
	-4 0 M 12 8 L -4 4 M 8 10 L stroke}
>> matrix makepattern
/Pat9 exch def
/Pattern1 {PatternBgnd KeepColor Pat1 setpattern} bind def
/Pattern2 {PatternBgnd KeepColor Pat2 setpattern} bind def
/Pattern3 {PatternBgnd KeepColor Pat3 setpattern} bind def
/Pattern4 {PatternBgnd KeepColor Landscape {Pat5} {Pat4} ifelse setpattern} bind def
/Pattern5 {PatternBgnd KeepColor Landscape {Pat4} {Pat5} ifelse setpattern} bind def
/Pattern6 {PatternBgnd KeepColor Landscape {Pat9} {Pat6} ifelse setpattern} bind def
/Pattern7 {PatternBgnd KeepColor Landscape {Pat8} {Pat7} ifelse setpattern} bind def
} def
%
%
%End of PostScript Level 2 code
%
/PatternBgnd {
  TransparentPatterns {} {gsave 1 setgray fill grestore} ifelse
} def
%
% Substitute for Level 2 pattern fill codes with
% grayscale if Level 2 support is not selected.
%
/Level1PatternFill {
/Pattern1 {0.250 Density} bind def
/Pattern2 {0.500 Density} bind def
/Pattern3 {0.750 Density} bind def
/Pattern4 {0.125 Density} bind def
/Pattern5 {0.375 Density} bind def
/Pattern6 {0.625 Density} bind def
/Pattern7 {0.875 Density} bind def
} def
%
% Now test for support of Level 2 code
%
Level1 {Level1PatternFill} {Level2PatternFill} ifelse
%
/Symbol-Oblique /Symbol findfont [1 0 .167 1 0 0] makefont
dup length dict begin {1 index /FID eq {pop pop} {def} ifelse} forall
currentdict end definefont pop
%
Level1 SuppressPDFMark or 
{} {
/SDict 10 dict def
systemdict /pdfmark known not {
  userdict /pdfmark systemdict /cleartomark get put
} if
SDict begin [
  /Title (plot_Qcool_su8b47.75.tex)
  /Subject (gnuplot plot)
  /Creator (gnuplot 5.0 patchlevel 3)
  /Author (mteper)
%  /Producer (gnuplot)
%  /Keywords ()
  /CreationDate (Thu Apr 29 13:46:56 2021)
  /DOCINFO pdfmark
end
} ifelse
%
% Support for boxed text - Ethan A Merritt May 2005
%
/InitTextBox { userdict /TBy2 3 -1 roll put userdict /TBx2 3 -1 roll put
           userdict /TBy1 3 -1 roll put userdict /TBx1 3 -1 roll put
	   /Boxing true def } def
/ExtendTextBox { Boxing
    { gsave dup false charpath pathbbox
      dup TBy2 gt {userdict /TBy2 3 -1 roll put} {pop} ifelse
      dup TBx2 gt {userdict /TBx2 3 -1 roll put} {pop} ifelse
      dup TBy1 lt {userdict /TBy1 3 -1 roll put} {pop} ifelse
      dup TBx1 lt {userdict /TBx1 3 -1 roll put} {pop} ifelse
      grestore } if } def
/PopTextBox { newpath TBx1 TBxmargin sub TBy1 TBymargin sub M
               TBx1 TBxmargin sub TBy2 TBymargin add L
	       TBx2 TBxmargin add TBy2 TBymargin add L
	       TBx2 TBxmargin add TBy1 TBymargin sub L closepath } def
/DrawTextBox { PopTextBox stroke /Boxing false def} def
/FillTextBox { gsave PopTextBox 1 1 1 setrgbcolor fill grestore /Boxing false def} def
0 0 0 0 InitTextBox
/TBxmargin 20 def
/TBymargin 20 def
/Boxing false def
/textshow { ExtendTextBox Gshow } def
%
% redundant definitions for compatibility with prologue.ps older than 5.0.2
/LTB {BL [] LCb DL} def
/LTb {PL [] LCb DL} def
end
%%EndProlog
%%Page: 1 1
gnudict begin
gsave
doclip
0 0 translate
0.050 0.050 scale
0 setgray
newpath
BackgroundColor 0 lt 3 1 roll 0 lt exch 0 lt or or not {BackgroundColor C 1.000 0 0 9360.00 7560.00 BoxColFill} if
1.000 UL
LTb
LCb setrgbcolor
1260 640 M
63 0 V
7675 0 R
-63 0 V
stroke
LTb
LCb setrgbcolor
1260 1197 M
63 0 V
7675 0 R
-63 0 V
stroke
LTb
LCb setrgbcolor
1260 1753 M
63 0 V
7675 0 R
-63 0 V
stroke
LTb
LCb setrgbcolor
1260 2310 M
63 0 V
7675 0 R
-63 0 V
stroke
LTb
LCb setrgbcolor
1260 2866 M
63 0 V
7675 0 R
-63 0 V
stroke
LTb
LCb setrgbcolor
1260 3423 M
63 0 V
7675 0 R
-63 0 V
stroke
LTb
LCb setrgbcolor
1260 3980 M
63 0 V
7675 0 R
-63 0 V
stroke
LTb
LCb setrgbcolor
1260 4536 M
63 0 V
7675 0 R
-63 0 V
stroke
LTb
LCb setrgbcolor
1260 5093 M
63 0 V
7675 0 R
-63 0 V
stroke
LTb
LCb setrgbcolor
1260 5649 M
63 0 V
7675 0 R
-63 0 V
stroke
LTb
LCb setrgbcolor
1260 6206 M
63 0 V
7675 0 R
-63 0 V
stroke
LTb
LCb setrgbcolor
1260 6762 M
63 0 V
7675 0 R
-63 0 V
stroke
LTb
LCb setrgbcolor
1260 7319 M
63 0 V
7675 0 R
-63 0 V
stroke
LTb
LCb setrgbcolor
1690 640 M
0 63 V
0 6616 R
0 -63 V
stroke
LTb
LCb setrgbcolor
2550 640 M
0 63 V
0 6616 R
0 -63 V
stroke
LTb
LCb setrgbcolor
3409 640 M
0 63 V
0 6616 R
0 -63 V
stroke
LTb
LCb setrgbcolor
4269 640 M
0 63 V
0 6616 R
0 -63 V
stroke
LTb
LCb setrgbcolor
5129 640 M
0 63 V
0 6616 R
0 -63 V
stroke
LTb
LCb setrgbcolor
5989 640 M
0 63 V
0 6616 R
0 -63 V
stroke
LTb
LCb setrgbcolor
6849 640 M
0 63 V
0 6616 R
0 -63 V
stroke
LTb
LCb setrgbcolor
7708 640 M
0 63 V
0 6616 R
0 -63 V
stroke
LTb
LCb setrgbcolor
8568 640 M
0 63 V
0 6616 R
0 -63 V
stroke
LTb
LCb setrgbcolor
1.000 UL
LTb
LCb setrgbcolor
1260 7319 N
0 -6679 V
7738 0 V
0 6679 V
-7738 0 V
Z stroke
1.000 UP
1.000 UL
LTb
LCb setrgbcolor
LCb setrgbcolor
LTb
LCb setrgbcolor
LTb
1.500 UP
1.000 UL
LTb
0.58 0.00 0.83 C 1991 641 M
0 3 V
43 -3 R
0 4 V
43 13 R
0 10 V
43 -1 R
0 12 V
43 24 R
0 18 V
43 3 R
0 21 V
43 -21 R
0 21 V
43 -6 R
0 22 V
43 -49 R
0 19 V
43 -72 R
0 11 V
43 -22 R
0 7 V
43 -14 R
0 4 V
43 -5 R
0 2 V
86 -2 R
0 2 V
86 -2 R
0 2 V
43 -2 R
0 2 V
43 5 R
0 6 V
43 -4 R
0 8 V
43 14 R
0 13 V
43 13 R
0 17 V
43 5 R
0 20 V
43 8 R
0 22 V
43 -47 R
0 21 V
43 -49 R
0 17 V
43 -40 R
0 13 V
43 -26 R
0 10 V
43 -26 R
0 5 V
214 -6 R
0 3 V
43 8 R
0 9 V
43 1 R
0 11 V
43 51 R
0 21 V
43 91 R
0 31 V
43 65 R
0 37 V
43 74 R
0 44 V
43 -38 R
0 44 V
43 -152 R
0 37 V
43 -147 R
0 30 V
43 -125 R
0 22 V
43 -71 R
0 15 V
43 -53 R
0 8 V
43 -13 R
0 5 V
129 -9 R
0 2 V
43 5 R
0 6 V
43 21 R
0 14 V
43 60 R
0 23 V
43 137 R
0 36 V
43 191 R
0 48 V
43 272 R
0 62 V
43 39 R
0 65 V
43 -118 R
0 64 V
43 -281 R
0 55 V
43 -330 R
0 42 V
43 -250 R
0 28 V
stroke 4570 831 M
43 -126 R
0 19 V
43 -68 R
0 10 V
43 -23 R
0 5 V
86 -8 R
0 2 V
43 -2 R
0 2 V
43 17 R
0 11 V
43 39 R
0 18 V
43 95 R
0 30 V
43 172 R
0 42 V
43 282 R
0 58 V
43 273 R
0 69 V
43 -6 R
0 71 V
43 -128 R
0 69 V
43 -393 R
0 58 V
43 -400 R
0 42 V
43 -274 R
0 27 V
43 -127 R
0 15 V
43 -45 R
0 10 V
43 -27 R
0 2 V
129 1 R
0 5 V
43 9 R
0 10 V
43 45 R
0 19 V
43 112 R
0 31 V
43 179 R
0 44 V
43 240 R
0 57 V
43 143 R
0 65 V
43 -26 R
0 65 V
43 -280 R
0 58 V
43 -276 R
0 49 V
43 -353 R
0 31 V
43 -141 R
0 21 V
43 -90 R
0 11 V
43 -28 R
0 5 V
129 -7 R
0 3 V
43 -3 R
0 3 V
43 14 R
0 10 V
43 25 R
0 16 V
43 72 R
0 26 V
43 120 R
0 36 V
43 88 R
0 44 V
43 68 R
0 49 V
43 -133 R
0 46 V
43 -117 R
0 41 V
43 -181 R
0 33 V
43 -171 R
0 21 V
43 -72 R
0 14 V
43 -47 R
0 6 V
43 -11 R
0 4 V
43 -4 R
0 3 V
86 -4 R
0 2 V
43 0 R
0 5 V
43 -1 R
0 6 V
stroke 7064 652 M
42 18 R
0 13 V
43 29 R
0 19 V
43 22 R
0 23 V
43 26 R
0 28 V
43 19 R
0 32 V
43 -44 R
0 31 V
43 -102 R
0 24 V
43 -101 R
0 16 V
43 -44 R
0 11 V
43 -23 R
0 7 V
43 -13 R
0 5 V
43 -7 R
0 3 V
1991 642 Circle
2034 643 Circle
2077 663 Circle
2120 673 Circle
2163 712 Circle
2206 735 Circle
2249 735 Circle
2292 750 Circle
2335 721 Circle
2378 664 Circle
2421 651 Circle
2464 643 Circle
2507 641 Circle
2593 641 Circle
2679 641 Circle
2722 641 Circle
2765 650 Circle
2808 653 Circle
2851 678 Circle
2894 706 Circle
2937 729 Circle
2980 758 Circle
3023 732 Circle
3066 702 Circle
3109 678 Circle
3152 663 Circle
3195 644 Circle
3409 642 Circle
3452 657 Circle
3495 668 Circle
3538 735 Circle
3581 851 Circle
3624 951 Circle
3667 1065 Circle
3710 1071 Circle
3753 959 Circle
3796 846 Circle
3839 747 Circle
3882 695 Circle
3925 653 Circle
3968 647 Circle
4097 641 Circle
4140 650 Circle
4183 681 Circle
4226 759 Circle
4269 926 Circle
4312 1159 Circle
4355 1486 Circle
4398 1588 Circle
4441 1535 Circle
4484 1313 Circle
4527 1032 Circle
4570 817 Circle
4613 715 Circle
4656 661 Circle
4699 646 Circle
4785 641 Circle
4828 641 Circle
4871 664 Circle
4914 718 Circle
4957 837 Circle
5000 1045 Circle
5043 1377 Circle
5086 1713 Circle
5129 1778 Circle
5172 1720 Circle
5215 1390 Circle
5258 1040 Circle
5301 800 Circle
5344 695 Circle
5387 662 Circle
5430 641 Circle
5559 646 Circle
5602 662 Circle
5645 721 Circle
5688 858 Circle
5731 1075 Circle
5774 1366 Circle
5817 1569 Circle
5860 1608 Circle
5903 1390 Circle
5946 1168 Circle
5989 855 Circle
6032 739 Circle
6075 666 Circle
6118 646 Circle
6247 642 Circle
6290 642 Circle
6333 663 Circle
6376 701 Circle
6419 794 Circle
6462 945 Circle
6505 1073 Circle
6548 1188 Circle
6591 1102 Circle
6634 1028 Circle
6677 885 Circle
6720 740 Circle
6763 686 Circle
6806 649 Circle
6849 643 Circle
6892 642 Circle
6978 641 Circle
7021 644 Circle
7064 649 Circle
7106 677 Circle
7149 721 Circle
7192 765 Circle
7235 816 Circle
7278 865 Circle
7321 853 Circle
7364 778 Circle
7407 697 Circle
7450 667 Circle
7493 652 Circle
7536 646 Circle
7579 642 Circle
1.500 UP
1.000 UL
LTb
0.58 0.00 0.83 C 1776 1112 M
0 47 V
43 -466 R
0 16 V
774 -67 R
0 5 V
43 520 R
0 50 V
816 1590 R
0 99 V
3495 647 M
0 6 V
774 1191 R
0 74 V
43 2425 R
0 130 V
817 2206 R
0 165 V
5172 640 M
0 2 V
774 3726 R
0 130 V
43 -2678 R
0 73 V
6763 648 M
0 7 V
43 2701 R
0 111 V
7622 1693 M
0 69 V
7665 660 M
0 11 V
1776 1135 CircleF
1819 701 CircleF
2593 644 CircleF
2636 1192 CircleF
3452 2856 CircleF
3495 650 CircleF
4269 1881 CircleF
4312 4408 CircleF
5129 6761 CircleF
5172 641 CircleF
5946 4433 CircleF
5989 1857 CircleF
6763 651 CircleF
6806 3412 CircleF
7622 1728 CircleF
7665 666 CircleF
2.000 UL
LTb
LCb setrgbcolor
1.000 UL
LTb
LCb setrgbcolor
1260 7319 N
0 -6679 V
7738 0 V
0 6679 V
-7738 0 V
Z stroke
1.000 UP
1.000 UL
LTb
LCb setrgbcolor
stroke
grestore
end
showpage
  }}%
  \put(5129,140){\makebox(0,0){\large{$Q_L$}}}%
  \put(200,4979){\makebox(0,0){\Large{$N(Q_L)$}}}%
  \put(8568,440){\makebox(0,0){\strut{}\ {$4$}}}%
  \put(7708,440){\makebox(0,0){\strut{}\ {$3$}}}%
  \put(6849,440){\makebox(0,0){\strut{}\ {$2$}}}%
  \put(5989,440){\makebox(0,0){\strut{}\ {$1$}}}%
  \put(5129,440){\makebox(0,0){\strut{}\ {$0$}}}%
  \put(4269,440){\makebox(0,0){\strut{}\ {$-1$}}}%
  \put(3409,440){\makebox(0,0){\strut{}\ {$-2$}}}%
  \put(2550,440){\makebox(0,0){\strut{}\ {$-3$}}}%
  \put(1690,440){\makebox(0,0){\strut{}\ {$-4$}}}%
  \put(1140,7319){\makebox(0,0)[r]{\strut{}\ \ {$6000$}}}%
  \put(1140,6762){\makebox(0,0)[r]{\strut{}\ \ {$5500$}}}%
  \put(1140,6206){\makebox(0,0)[r]{\strut{}\ \ {$5000$}}}%
  \put(1140,5649){\makebox(0,0)[r]{\strut{}\ \ {$4500$}}}%
  \put(1140,5093){\makebox(0,0)[r]{\strut{}\ \ {$4000$}}}%
  \put(1140,4536){\makebox(0,0)[r]{\strut{}\ \ {$3500$}}}%
  \put(1140,3980){\makebox(0,0)[r]{\strut{}\ \ {$3000$}}}%
  \put(1140,3423){\makebox(0,0)[r]{\strut{}\ \ {$2500$}}}%
  \put(1140,2866){\makebox(0,0)[r]{\strut{}\ \ {$2000$}}}%
  \put(1140,2310){\makebox(0,0)[r]{\strut{}\ \ {$1500$}}}%
  \put(1140,1753){\makebox(0,0)[r]{\strut{}\ \ {$1000$}}}%
  \put(1140,1197){\makebox(0,0)[r]{\strut{}\ \ {$500$}}}%
  \put(1140,640){\makebox(0,0)[r]{\strut{}\ \ {$0$}}}%
\end{picture}%
\endgroup
\endinput

\end	{center}
\caption{The number of lattice fields with topological charge $Q_L$ after 2 ($\circ$) and after 20 ($\bullet$)
cooling sweeps, from sequences of $SU(8)$ fields generated at $\beta=47.75$. $N(Q_L)=0$ points suppressed.}
\label{fig_Qcool20_su8}
\end{figure}

\begin{figure}[htb]
\begin	{center}
\leavevmode
% GNUPLOT: LaTeX picture with Postscript
\begingroup%
\makeatletter%
\newcommand{\GNUPLOTspecial}{%
  \@sanitize\catcode`\%=14\relax\special}%
\setlength{\unitlength}{0.0500bp}%
\begin{picture}(9360,7560)(0,0)%
  {\GNUPLOTspecial{"
%!PS-Adobe-2.0 EPSF-2.0
%%Title: plot_Qseq_su8b47.75.tex
%%Creator: gnuplot 5.0 patchlevel 3
%%CreationDate: Sat Mar 27 10:51:08 2021
%%DocumentFonts: 
%%BoundingBox: 0 0 468 378
%%EndComments
%%BeginProlog
/gnudict 256 dict def
gnudict begin
%
% The following true/false flags may be edited by hand if desired.
% The unit line width and grayscale image gamma correction may also be changed.
%
/Color true def
/Blacktext true def
/Solid false def
/Dashlength 1 def
/Landscape false def
/Level1 false def
/Level3 false def
/Rounded false def
/ClipToBoundingBox false def
/SuppressPDFMark false def
/TransparentPatterns false def
/gnulinewidth 5.000 def
/userlinewidth gnulinewidth def
/Gamma 1.0 def
/BackgroundColor {-1.000 -1.000 -1.000} def
%
/vshift -66 def
/dl1 {
  10.0 Dashlength userlinewidth gnulinewidth div mul mul mul
  Rounded { currentlinewidth 0.75 mul sub dup 0 le { pop 0.01 } if } if
} def
/dl2 {
  10.0 Dashlength userlinewidth gnulinewidth div mul mul mul
  Rounded { currentlinewidth 0.75 mul add } if
} def
/hpt_ 31.5 def
/vpt_ 31.5 def
/hpt hpt_ def
/vpt vpt_ def
/doclip {
  ClipToBoundingBox {
    newpath 0 0 moveto 468 0 lineto 468 378 lineto 0 378 lineto closepath
    clip
  } if
} def
%
% Gnuplot Prolog Version 5.1 (Oct 2015)
%
%/SuppressPDFMark true def
%
/M {moveto} bind def
/L {lineto} bind def
/R {rmoveto} bind def
/V {rlineto} bind def
/N {newpath moveto} bind def
/Z {closepath} bind def
/C {setrgbcolor} bind def
/f {rlineto fill} bind def
/g {setgray} bind def
/Gshow {show} def   % May be redefined later in the file to support UTF-8
/vpt2 vpt 2 mul def
/hpt2 hpt 2 mul def
/Lshow {currentpoint stroke M 0 vshift R 
	Blacktext {gsave 0 setgray textshow grestore} {textshow} ifelse} def
/Rshow {currentpoint stroke M dup stringwidth pop neg vshift R
	Blacktext {gsave 0 setgray textshow grestore} {textshow} ifelse} def
/Cshow {currentpoint stroke M dup stringwidth pop -2 div vshift R 
	Blacktext {gsave 0 setgray textshow grestore} {textshow} ifelse} def
/UP {dup vpt_ mul /vpt exch def hpt_ mul /hpt exch def
  /hpt2 hpt 2 mul def /vpt2 vpt 2 mul def} def
/DL {Color {setrgbcolor Solid {pop []} if 0 setdash}
 {pop pop pop 0 setgray Solid {pop []} if 0 setdash} ifelse} def
/BL {stroke userlinewidth 2 mul setlinewidth
	Rounded {1 setlinejoin 1 setlinecap} if} def
/AL {stroke userlinewidth 2 div setlinewidth
	Rounded {1 setlinejoin 1 setlinecap} if} def
/UL {dup gnulinewidth mul /userlinewidth exch def
	dup 1 lt {pop 1} if 10 mul /udl exch def} def
/PL {stroke userlinewidth setlinewidth
	Rounded {1 setlinejoin 1 setlinecap} if} def
3.8 setmiterlimit
% Classic Line colors (version 5.0)
/LCw {1 1 1} def
/LCb {0 0 0} def
/LCa {0 0 0} def
/LC0 {1 0 0} def
/LC1 {0 1 0} def
/LC2 {0 0 1} def
/LC3 {1 0 1} def
/LC4 {0 1 1} def
/LC5 {1 1 0} def
/LC6 {0 0 0} def
/LC7 {1 0.3 0} def
/LC8 {0.5 0.5 0.5} def
% Default dash patterns (version 5.0)
/LTB {BL [] LCb DL} def
/LTw {PL [] 1 setgray} def
/LTb {PL [] LCb DL} def
/LTa {AL [1 udl mul 2 udl mul] 0 setdash LCa setrgbcolor} def
/LT0 {PL [] LC0 DL} def
/LT1 {PL [2 dl1 3 dl2] LC1 DL} def
/LT2 {PL [1 dl1 1.5 dl2] LC2 DL} def
/LT3 {PL [6 dl1 2 dl2 1 dl1 2 dl2] LC3 DL} def
/LT4 {PL [1 dl1 2 dl2 6 dl1 2 dl2 1 dl1 2 dl2] LC4 DL} def
/LT5 {PL [4 dl1 2 dl2] LC5 DL} def
/LT6 {PL [1.5 dl1 1.5 dl2 1.5 dl1 1.5 dl2 1.5 dl1 6 dl2] LC6 DL} def
/LT7 {PL [3 dl1 3 dl2 1 dl1 3 dl2] LC7 DL} def
/LT8 {PL [2 dl1 2 dl2 2 dl1 6 dl2] LC8 DL} def
/SL {[] 0 setdash} def
/Pnt {stroke [] 0 setdash gsave 1 setlinecap M 0 0 V stroke grestore} def
/Dia {stroke [] 0 setdash 2 copy vpt add M
  hpt neg vpt neg V hpt vpt neg V
  hpt vpt V hpt neg vpt V closepath stroke
  Pnt} def
/Pls {stroke [] 0 setdash vpt sub M 0 vpt2 V
  currentpoint stroke M
  hpt neg vpt neg R hpt2 0 V stroke
 } def
/Box {stroke [] 0 setdash 2 copy exch hpt sub exch vpt add M
  0 vpt2 neg V hpt2 0 V 0 vpt2 V
  hpt2 neg 0 V closepath stroke
  Pnt} def
/Crs {stroke [] 0 setdash exch hpt sub exch vpt add M
  hpt2 vpt2 neg V currentpoint stroke M
  hpt2 neg 0 R hpt2 vpt2 V stroke} def
/TriU {stroke [] 0 setdash 2 copy vpt 1.12 mul add M
  hpt neg vpt -1.62 mul V
  hpt 2 mul 0 V
  hpt neg vpt 1.62 mul V closepath stroke
  Pnt} def
/Star {2 copy Pls Crs} def
/BoxF {stroke [] 0 setdash exch hpt sub exch vpt add M
  0 vpt2 neg V hpt2 0 V 0 vpt2 V
  hpt2 neg 0 V closepath fill} def
/TriUF {stroke [] 0 setdash vpt 1.12 mul add M
  hpt neg vpt -1.62 mul V
  hpt 2 mul 0 V
  hpt neg vpt 1.62 mul V closepath fill} def
/TriD {stroke [] 0 setdash 2 copy vpt 1.12 mul sub M
  hpt neg vpt 1.62 mul V
  hpt 2 mul 0 V
  hpt neg vpt -1.62 mul V closepath stroke
  Pnt} def
/TriDF {stroke [] 0 setdash vpt 1.12 mul sub M
  hpt neg vpt 1.62 mul V
  hpt 2 mul 0 V
  hpt neg vpt -1.62 mul V closepath fill} def
/DiaF {stroke [] 0 setdash vpt add M
  hpt neg vpt neg V hpt vpt neg V
  hpt vpt V hpt neg vpt V closepath fill} def
/Pent {stroke [] 0 setdash 2 copy gsave
  translate 0 hpt M 4 {72 rotate 0 hpt L} repeat
  closepath stroke grestore Pnt} def
/PentF {stroke [] 0 setdash gsave
  translate 0 hpt M 4 {72 rotate 0 hpt L} repeat
  closepath fill grestore} def
/Circle {stroke [] 0 setdash 2 copy
  hpt 0 360 arc stroke Pnt} def
/CircleF {stroke [] 0 setdash hpt 0 360 arc fill} def
/C0 {BL [] 0 setdash 2 copy moveto vpt 90 450 arc} bind def
/C1 {BL [] 0 setdash 2 copy moveto
	2 copy vpt 0 90 arc closepath fill
	vpt 0 360 arc closepath} bind def
/C2 {BL [] 0 setdash 2 copy moveto
	2 copy vpt 90 180 arc closepath fill
	vpt 0 360 arc closepath} bind def
/C3 {BL [] 0 setdash 2 copy moveto
	2 copy vpt 0 180 arc closepath fill
	vpt 0 360 arc closepath} bind def
/C4 {BL [] 0 setdash 2 copy moveto
	2 copy vpt 180 270 arc closepath fill
	vpt 0 360 arc closepath} bind def
/C5 {BL [] 0 setdash 2 copy moveto
	2 copy vpt 0 90 arc
	2 copy moveto
	2 copy vpt 180 270 arc closepath fill
	vpt 0 360 arc} bind def
/C6 {BL [] 0 setdash 2 copy moveto
	2 copy vpt 90 270 arc closepath fill
	vpt 0 360 arc closepath} bind def
/C7 {BL [] 0 setdash 2 copy moveto
	2 copy vpt 0 270 arc closepath fill
	vpt 0 360 arc closepath} bind def
/C8 {BL [] 0 setdash 2 copy moveto
	2 copy vpt 270 360 arc closepath fill
	vpt 0 360 arc closepath} bind def
/C9 {BL [] 0 setdash 2 copy moveto
	2 copy vpt 270 450 arc closepath fill
	vpt 0 360 arc closepath} bind def
/C10 {BL [] 0 setdash 2 copy 2 copy moveto vpt 270 360 arc closepath fill
	2 copy moveto
	2 copy vpt 90 180 arc closepath fill
	vpt 0 360 arc closepath} bind def
/C11 {BL [] 0 setdash 2 copy moveto
	2 copy vpt 0 180 arc closepath fill
	2 copy moveto
	2 copy vpt 270 360 arc closepath fill
	vpt 0 360 arc closepath} bind def
/C12 {BL [] 0 setdash 2 copy moveto
	2 copy vpt 180 360 arc closepath fill
	vpt 0 360 arc closepath} bind def
/C13 {BL [] 0 setdash 2 copy moveto
	2 copy vpt 0 90 arc closepath fill
	2 copy moveto
	2 copy vpt 180 360 arc closepath fill
	vpt 0 360 arc closepath} bind def
/C14 {BL [] 0 setdash 2 copy moveto
	2 copy vpt 90 360 arc closepath fill
	vpt 0 360 arc} bind def
/C15 {BL [] 0 setdash 2 copy vpt 0 360 arc closepath fill
	vpt 0 360 arc closepath} bind def
/Rec {newpath 4 2 roll moveto 1 index 0 rlineto 0 exch rlineto
	neg 0 rlineto closepath} bind def
/Square {dup Rec} bind def
/Bsquare {vpt sub exch vpt sub exch vpt2 Square} bind def
/S0 {BL [] 0 setdash 2 copy moveto 0 vpt rlineto BL Bsquare} bind def
/S1 {BL [] 0 setdash 2 copy vpt Square fill Bsquare} bind def
/S2 {BL [] 0 setdash 2 copy exch vpt sub exch vpt Square fill Bsquare} bind def
/S3 {BL [] 0 setdash 2 copy exch vpt sub exch vpt2 vpt Rec fill Bsquare} bind def
/S4 {BL [] 0 setdash 2 copy exch vpt sub exch vpt sub vpt Square fill Bsquare} bind def
/S5 {BL [] 0 setdash 2 copy 2 copy vpt Square fill
	exch vpt sub exch vpt sub vpt Square fill Bsquare} bind def
/S6 {BL [] 0 setdash 2 copy exch vpt sub exch vpt sub vpt vpt2 Rec fill Bsquare} bind def
/S7 {BL [] 0 setdash 2 copy exch vpt sub exch vpt sub vpt vpt2 Rec fill
	2 copy vpt Square fill Bsquare} bind def
/S8 {BL [] 0 setdash 2 copy vpt sub vpt Square fill Bsquare} bind def
/S9 {BL [] 0 setdash 2 copy vpt sub vpt vpt2 Rec fill Bsquare} bind def
/S10 {BL [] 0 setdash 2 copy vpt sub vpt Square fill 2 copy exch vpt sub exch vpt Square fill
	Bsquare} bind def
/S11 {BL [] 0 setdash 2 copy vpt sub vpt Square fill 2 copy exch vpt sub exch vpt2 vpt Rec fill
	Bsquare} bind def
/S12 {BL [] 0 setdash 2 copy exch vpt sub exch vpt sub vpt2 vpt Rec fill Bsquare} bind def
/S13 {BL [] 0 setdash 2 copy exch vpt sub exch vpt sub vpt2 vpt Rec fill
	2 copy vpt Square fill Bsquare} bind def
/S14 {BL [] 0 setdash 2 copy exch vpt sub exch vpt sub vpt2 vpt Rec fill
	2 copy exch vpt sub exch vpt Square fill Bsquare} bind def
/S15 {BL [] 0 setdash 2 copy Bsquare fill Bsquare} bind def
/D0 {gsave translate 45 rotate 0 0 S0 stroke grestore} bind def
/D1 {gsave translate 45 rotate 0 0 S1 stroke grestore} bind def
/D2 {gsave translate 45 rotate 0 0 S2 stroke grestore} bind def
/D3 {gsave translate 45 rotate 0 0 S3 stroke grestore} bind def
/D4 {gsave translate 45 rotate 0 0 S4 stroke grestore} bind def
/D5 {gsave translate 45 rotate 0 0 S5 stroke grestore} bind def
/D6 {gsave translate 45 rotate 0 0 S6 stroke grestore} bind def
/D7 {gsave translate 45 rotate 0 0 S7 stroke grestore} bind def
/D8 {gsave translate 45 rotate 0 0 S8 stroke grestore} bind def
/D9 {gsave translate 45 rotate 0 0 S9 stroke grestore} bind def
/D10 {gsave translate 45 rotate 0 0 S10 stroke grestore} bind def
/D11 {gsave translate 45 rotate 0 0 S11 stroke grestore} bind def
/D12 {gsave translate 45 rotate 0 0 S12 stroke grestore} bind def
/D13 {gsave translate 45 rotate 0 0 S13 stroke grestore} bind def
/D14 {gsave translate 45 rotate 0 0 S14 stroke grestore} bind def
/D15 {gsave translate 45 rotate 0 0 S15 stroke grestore} bind def
/DiaE {stroke [] 0 setdash vpt add M
  hpt neg vpt neg V hpt vpt neg V
  hpt vpt V hpt neg vpt V closepath stroke} def
/BoxE {stroke [] 0 setdash exch hpt sub exch vpt add M
  0 vpt2 neg V hpt2 0 V 0 vpt2 V
  hpt2 neg 0 V closepath stroke} def
/TriUE {stroke [] 0 setdash vpt 1.12 mul add M
  hpt neg vpt -1.62 mul V
  hpt 2 mul 0 V
  hpt neg vpt 1.62 mul V closepath stroke} def
/TriDE {stroke [] 0 setdash vpt 1.12 mul sub M
  hpt neg vpt 1.62 mul V
  hpt 2 mul 0 V
  hpt neg vpt -1.62 mul V closepath stroke} def
/PentE {stroke [] 0 setdash gsave
  translate 0 hpt M 4 {72 rotate 0 hpt L} repeat
  closepath stroke grestore} def
/CircE {stroke [] 0 setdash 
  hpt 0 360 arc stroke} def
/Opaque {gsave closepath 1 setgray fill grestore 0 setgray closepath} def
/DiaW {stroke [] 0 setdash vpt add M
  hpt neg vpt neg V hpt vpt neg V
  hpt vpt V hpt neg vpt V Opaque stroke} def
/BoxW {stroke [] 0 setdash exch hpt sub exch vpt add M
  0 vpt2 neg V hpt2 0 V 0 vpt2 V
  hpt2 neg 0 V Opaque stroke} def
/TriUW {stroke [] 0 setdash vpt 1.12 mul add M
  hpt neg vpt -1.62 mul V
  hpt 2 mul 0 V
  hpt neg vpt 1.62 mul V Opaque stroke} def
/TriDW {stroke [] 0 setdash vpt 1.12 mul sub M
  hpt neg vpt 1.62 mul V
  hpt 2 mul 0 V
  hpt neg vpt -1.62 mul V Opaque stroke} def
/PentW {stroke [] 0 setdash gsave
  translate 0 hpt M 4 {72 rotate 0 hpt L} repeat
  Opaque stroke grestore} def
/CircW {stroke [] 0 setdash 
  hpt 0 360 arc Opaque stroke} def
/BoxFill {gsave Rec 1 setgray fill grestore} def
/Density {
  /Fillden exch def
  currentrgbcolor
  /ColB exch def /ColG exch def /ColR exch def
  /ColR ColR Fillden mul Fillden sub 1 add def
  /ColG ColG Fillden mul Fillden sub 1 add def
  /ColB ColB Fillden mul Fillden sub 1 add def
  ColR ColG ColB setrgbcolor} def
/BoxColFill {gsave Rec PolyFill} def
/PolyFill {gsave Density fill grestore grestore} def
/h {rlineto rlineto rlineto gsave closepath fill grestore} bind def
%
% PostScript Level 1 Pattern Fill routine for rectangles
% Usage: x y w h s a XX PatternFill
%	x,y = lower left corner of box to be filled
%	w,h = width and height of box
%	  a = angle in degrees between lines and x-axis
%	 XX = 0/1 for no/yes cross-hatch
%
/PatternFill {gsave /PFa [ 9 2 roll ] def
  PFa 0 get PFa 2 get 2 div add PFa 1 get PFa 3 get 2 div add translate
  PFa 2 get -2 div PFa 3 get -2 div PFa 2 get PFa 3 get Rec
  TransparentPatterns {} {gsave 1 setgray fill grestore} ifelse
  clip
  currentlinewidth 0.5 mul setlinewidth
  /PFs PFa 2 get dup mul PFa 3 get dup mul add sqrt def
  0 0 M PFa 5 get rotate PFs -2 div dup translate
  0 1 PFs PFa 4 get div 1 add floor cvi
	{PFa 4 get mul 0 M 0 PFs V} for
  0 PFa 6 get ne {
	0 1 PFs PFa 4 get div 1 add floor cvi
	{PFa 4 get mul 0 2 1 roll M PFs 0 V} for
 } if
  stroke grestore} def
%
/languagelevel where
 {pop languagelevel} {1} ifelse
dup 2 lt
	{/InterpretLevel1 true def
	 /InterpretLevel3 false def}
	{/InterpretLevel1 Level1 def
	 2 gt
	    {/InterpretLevel3 Level3 def}
	    {/InterpretLevel3 false def}
	 ifelse }
 ifelse
%
% PostScript level 2 pattern fill definitions
%
/Level2PatternFill {
/Tile8x8 {/PaintType 2 /PatternType 1 /TilingType 1 /BBox [0 0 8 8] /XStep 8 /YStep 8}
	bind def
/KeepColor {currentrgbcolor [/Pattern /DeviceRGB] setcolorspace} bind def
<< Tile8x8
 /PaintProc {0.5 setlinewidth pop 0 0 M 8 8 L 0 8 M 8 0 L stroke} 
>> matrix makepattern
/Pat1 exch def
<< Tile8x8
 /PaintProc {0.5 setlinewidth pop 0 0 M 8 8 L 0 8 M 8 0 L stroke
	0 4 M 4 8 L 8 4 L 4 0 L 0 4 L stroke}
>> matrix makepattern
/Pat2 exch def
<< Tile8x8
 /PaintProc {0.5 setlinewidth pop 0 0 M 0 8 L
	8 8 L 8 0 L 0 0 L fill}
>> matrix makepattern
/Pat3 exch def
<< Tile8x8
 /PaintProc {0.5 setlinewidth pop -4 8 M 8 -4 L
	0 12 M 12 0 L stroke}
>> matrix makepattern
/Pat4 exch def
<< Tile8x8
 /PaintProc {0.5 setlinewidth pop -4 0 M 8 12 L
	0 -4 M 12 8 L stroke}
>> matrix makepattern
/Pat5 exch def
<< Tile8x8
 /PaintProc {0.5 setlinewidth pop -2 8 M 4 -4 L
	0 12 M 8 -4 L 4 12 M 10 0 L stroke}
>> matrix makepattern
/Pat6 exch def
<< Tile8x8
 /PaintProc {0.5 setlinewidth pop -2 0 M 4 12 L
	0 -4 M 8 12 L 4 -4 M 10 8 L stroke}
>> matrix makepattern
/Pat7 exch def
<< Tile8x8
 /PaintProc {0.5 setlinewidth pop 8 -2 M -4 4 L
	12 0 M -4 8 L 12 4 M 0 10 L stroke}
>> matrix makepattern
/Pat8 exch def
<< Tile8x8
 /PaintProc {0.5 setlinewidth pop 0 -2 M 12 4 L
	-4 0 M 12 8 L -4 4 M 8 10 L stroke}
>> matrix makepattern
/Pat9 exch def
/Pattern1 {PatternBgnd KeepColor Pat1 setpattern} bind def
/Pattern2 {PatternBgnd KeepColor Pat2 setpattern} bind def
/Pattern3 {PatternBgnd KeepColor Pat3 setpattern} bind def
/Pattern4 {PatternBgnd KeepColor Landscape {Pat5} {Pat4} ifelse setpattern} bind def
/Pattern5 {PatternBgnd KeepColor Landscape {Pat4} {Pat5} ifelse setpattern} bind def
/Pattern6 {PatternBgnd KeepColor Landscape {Pat9} {Pat6} ifelse setpattern} bind def
/Pattern7 {PatternBgnd KeepColor Landscape {Pat8} {Pat7} ifelse setpattern} bind def
} def
%
%
%End of PostScript Level 2 code
%
/PatternBgnd {
  TransparentPatterns {} {gsave 1 setgray fill grestore} ifelse
} def
%
% Substitute for Level 2 pattern fill codes with
% grayscale if Level 2 support is not selected.
%
/Level1PatternFill {
/Pattern1 {0.250 Density} bind def
/Pattern2 {0.500 Density} bind def
/Pattern3 {0.750 Density} bind def
/Pattern4 {0.125 Density} bind def
/Pattern5 {0.375 Density} bind def
/Pattern6 {0.625 Density} bind def
/Pattern7 {0.875 Density} bind def
} def
%
% Now test for support of Level 2 code
%
Level1 {Level1PatternFill} {Level2PatternFill} ifelse
%
/Symbol-Oblique /Symbol findfont [1 0 .167 1 0 0] makefont
dup length dict begin {1 index /FID eq {pop pop} {def} ifelse} forall
currentdict end definefont pop
%
Level1 SuppressPDFMark or 
{} {
/SDict 10 dict def
systemdict /pdfmark known not {
  userdict /pdfmark systemdict /cleartomark get put
} if
SDict begin [
  /Title (plot_Qseq_su8b47.75.tex)
  /Subject (gnuplot plot)
  /Creator (gnuplot 5.0 patchlevel 3)
  /Author (mteper)
%  /Producer (gnuplot)
%  /Keywords ()
  /CreationDate (Sat Mar 27 10:51:08 2021)
  /DOCINFO pdfmark
end
} ifelse
%
% Support for boxed text - Ethan A Merritt May 2005
%
/InitTextBox { userdict /TBy2 3 -1 roll put userdict /TBx2 3 -1 roll put
           userdict /TBy1 3 -1 roll put userdict /TBx1 3 -1 roll put
	   /Boxing true def } def
/ExtendTextBox { Boxing
    { gsave dup false charpath pathbbox
      dup TBy2 gt {userdict /TBy2 3 -1 roll put} {pop} ifelse
      dup TBx2 gt {userdict /TBx2 3 -1 roll put} {pop} ifelse
      dup TBy1 lt {userdict /TBy1 3 -1 roll put} {pop} ifelse
      dup TBx1 lt {userdict /TBx1 3 -1 roll put} {pop} ifelse
      grestore } if } def
/PopTextBox { newpath TBx1 TBxmargin sub TBy1 TBymargin sub M
               TBx1 TBxmargin sub TBy2 TBymargin add L
	       TBx2 TBxmargin add TBy2 TBymargin add L
	       TBx2 TBxmargin add TBy1 TBymargin sub L closepath } def
/DrawTextBox { PopTextBox stroke /Boxing false def} def
/FillTextBox { gsave PopTextBox 1 1 1 setrgbcolor fill grestore /Boxing false def} def
0 0 0 0 InitTextBox
/TBxmargin 20 def
/TBymargin 20 def
/Boxing false def
/textshow { ExtendTextBox Gshow } def
%
% redundant definitions for compatibility with prologue.ps older than 5.0.2
/LTB {BL [] LCb DL} def
/LTb {PL [] LCb DL} def
end
%%EndProlog
%%Page: 1 1
gnudict begin
gsave
doclip
0 0 translate
0.050 0.050 scale
0 setgray
newpath
BackgroundColor 0 lt 3 1 roll 0 lt exch 0 lt or or not {BackgroundColor C 1.000 0 0 9360.00 7560.00 BoxColFill} if
1.000 UL
LTb
LCb setrgbcolor
1140 640 M
63 0 V
7795 0 R
-63 0 V
stroke
LTb
LCb setrgbcolor
1140 1976 M
63 0 V
7795 0 R
-63 0 V
stroke
LTb
LCb setrgbcolor
1140 3312 M
63 0 V
7795 0 R
-63 0 V
stroke
LTb
LCb setrgbcolor
1140 4647 M
63 0 V
7795 0 R
-63 0 V
stroke
LTb
LCb setrgbcolor
1140 5983 M
63 0 V
7795 0 R
-63 0 V
stroke
LTb
LCb setrgbcolor
1140 7319 M
63 0 V
7795 0 R
-63 0 V
stroke
LTb
LCb setrgbcolor
1140 640 M
0 63 V
0 6616 R
0 -63 V
stroke
LTb
LCb setrgbcolor
2712 640 M
0 63 V
0 6616 R
0 -63 V
stroke
LTb
LCb setrgbcolor
4283 640 M
0 63 V
0 6616 R
0 -63 V
stroke
LTb
LCb setrgbcolor
5855 640 M
0 63 V
0 6616 R
0 -63 V
stroke
LTb
LCb setrgbcolor
7426 640 M
0 63 V
0 6616 R
0 -63 V
stroke
LTb
LCb setrgbcolor
8998 640 M
0 63 V
0 6616 R
0 -63 V
stroke
LTb
LCb setrgbcolor
1.000 UL
LTb
LCb setrgbcolor
1140 7319 N
0 -6679 V
7858 0 V
0 6679 V
-7858 0 V
Z stroke
1.000 UP
1.000 UL
LTb
LCb setrgbcolor
LCb setrgbcolor
LTb
LCb setrgbcolor
LTb
1.500 UP
1.000 UL
LTb
0.58 0.00 0.83 C 1156 2891 M
15 312 R
16 -238 R
16 42 R
16 34 R
15 82 R
16 -626 R
16 550 R
15 -498 R
16 681 R
16 -14 R
16 -758 R
15 -51 R
16 286 R
16 7 R
15 377 R
16 -314 R
16 -318 R
16 298 R
15 128 R
16 -218 R
16 201 R
15 58 R
16 -402 R
16 189 R
16 34 R
15 -4 R
16 69 R
16 239 R
15 129 R
16 -250 R
16 -232 R
16 567 R
0 1 V
15 -283 R
16 -432 R
16 358 R
15 77 R
16 161 R
16 -165 R
16 -341 R
15 443 R
16 -130 R
0 1 V
16 -372 R
16 310 R
15 -299 R
16 271 R
16 -125 R
0 1 V
15 404 R
16 -103 R
16 113 R
16 -43 R
15 -341 R
16 169 R
16 -289 R
15 -40 R
16 478 R
16 -262 R
16 34 R
15 40 R
16 -564 R
16 957 R
15 -430 R
16 -5 R
16 240 R
16 -170 R
15 -78 R
16 43 R
16 -196 R
15 364 R
16 -119 R
16 -43 R
16 -239 R
15 111 R
16 333 R
16 -353 R
15 34 R
16 -178 R
16 137 R
16 333 R
15 42 R
16 98 R
16 -244 R
15 41 R
16 67 R
0 1 V
16 -34 R
16 -594 R
15 231 R
16 626 R
16 -496 R
15 -124 R
16 131 R
16 74 R
16 246 R
15 -209 R
16 49 R
16 -56 R
15 -159 R
16 -416 R
16 522 R
16 180 R
15 175 R
16 -397 R
16 184 R
15 -515 R
16 -58 R
16 390 R
0 1 V
stroke 2806 2820 M
16 -310 R
15 576 R
16 -163 R
16 -356 R
15 359 R
16 -41 R
16 198 R
16 86 R
15 -282 R
16 223 R
16 -203 R
15 60 R
16 184 R
16 -285 R
16 -213 R
15 571 R
16 -498 R
16 110 R
16 -552 R
15 701 R
16 544 R
16 -735 R
15 437 R
16 -99 R
16 -1025 R
16 486 R
15 560 R
16 -573 R
16 202 R
15 -124 R
16 509 R
16 -331 R
16 454 R
15 -90 R
16 -325 R
16 120 R
15 341 R
16 -698 R
16 -218 R
16 627 R
15 -302 R
16 -235 R
16 316 R
15 -152 R
16 142 R
16 350 R
16 -217 R
15 -184 R
16 -256 R
16 665 R
15 436 R
16 -718 R
16 381 R
16 -858 R
15 289 R
16 -99 R
0 1 V
16 -329 R
15 765 R
16 -770 R
16 267 R
16 894 R
15 -902 R
16 101 R
16 -73 R
15 653 R
16 -211 R
16 -652 R
16 673 R
15 -127 R
16 395 R
16 -378 R
15 -76 R
16 232 R
16 346 R
16 -806 R
15 358 R
16 -323 R
16 16 R
15 634 R
16 -760 R
16 37 R
16 476 R
15 -28 R
16 -231 R
16 236 R
15 -269 R
16 770 R
16 -789 R
16 448 R
15 -471 R
16 104 R
16 151 R
15 -343 R
16 894 R
16 -642 R
16 724 R
15 -751 R
16 -487 R
16 1088 R
15 -737 R
16 183 R
16 -65 R
16 282 R
15 -383 R
16 -165 R
16 168 R
16 -85 R
15 757 R
16 -413 R
16 -261 R
15 -45 R
16 531 R
16 -712 R
16 365 R
15 -64 R
16 147 R
16 -481 R
15 420 R
16 132 R
16 -248 R
16 217 R
15 -550 R
16 184 R
16 338 R
15 113 R
16 -365 R
16 89 R
16 -69 R
15 59 R
0 1 V
stroke 4833 2793 M
16 362 R
16 -34 R
15 -319 R
16 -150 R
0 1 V
16 365 R
16 -214 R
15 -388 R
16 617 R
0 1 V
16 109 R
15 -268 R
16 293 R
16 -328 R
16 -121 R
15 288 R
16 222 R
16 -542 R
15 -126 R
16 -35 R
16 243 R
16 -174 R
15 274 R
16 305 R
16 -571 R
15 268 R
16 -143 R
16 281 R
16 -203 R
15 98 R
16 177 R
16 20 R
15 31 R
16 192 R
16 -400 R
0 1 V
16 -392 R
15 327 R
16 79 R
16 233 R
15 -426 R
16 179 R
16 -81 R
16 181 R
15 -124 R
16 -129 R
16 -110 R
15 55 R
16 47 R
16 -172 R
16 455 R
15 -715 R
16 376 R
16 214 R
15 46 R
16 -160 R
16 117 R
16 -406 R
15 -8 R
16 197 R
16 380 R
0 1 V
16 -189 R
15 240 R
16 -908 R
16 769 R
15 99 R
16 -24 R
16 -452 R
16 312 R
15 187 R
16 -70 R
16 -140 R
15 29 R
16 101 R
16 -234 R
16 131 R
15 -382 R
16 -160 R
16 695 R
15 -484 R
16 329 R
16 -181 R
16 -5 R
15 -91 R
16 174 R
0 1 V
16 -5 R
0 1 V
15 119 R
16 -95 R
16 -29 R
0 1 V
16 228 R
15 53 R
16 -373 R
16 -63 R
15 13 R
16 762 R
16 -431 R
16 -625 R
15 119 R
16 475 R
16 468 R
15 -737 R
16 253 R
16 -195 R
16 -205 R
15 293 R
16 -320 R
16 165 R
15 -154 R
16 140 R
16 440 R
16 -430 R
15 331 R
16 -319 R
16 -456 R
15 916 R
16 8 R
16 80 R
16 -429 R
15 98 R
16 -197 R
16 -94 R
15 -237 R
16 413 R
16 451 R
16 -133 R
15 -397 R
16 253 R
16 -86 R
15 280 R
16 -629 R
16 251 R
16 321 R
15 -714 R
16 531 R
16 19 R
15 -155 R
16 -27 R
16 -177 R
16 253 R
15 60 R
16 1 R
16 -76 R
16 365 R
15 35 R
16 -737 R
16 -23 R
15 -112 R
16 117 R
16 269 R
16 416 R
15 35 R
16 -610 R
16 -28 R
15 277 R
16 -68 R
16 72 R
16 87 R
15 -110 R
16 -339 R
16 383 R
15 92 R
16 507 R
16 -310 R
16 -484 R
15 617 R
16 -650 R
16 501 R
15 -412 R
16 -36 R
16 126 R
16 -235 R
15 343 R
16 77 R
16 -324 R
15 136 R
16 319 R
16 -70 R
16 -357 R
15 -195 R
16 -357 R
16 73 R
15 1077 R
16 -643 R
16 -306 R
16 444 R
15 -79 R
16 166 R
16 174 R
15 -368 R
16 481 R
16 -462 R
16 -52 R
15 293 R
16 -386 R
16 694 R
15 -127 R
16 -709 R
16 221 R
16 346 R
0 1 V
stroke 7914 2999 M
15 -176 R
16 233 R
16 -98 R
15 -9 R
16 -285 R
16 247 R
16 -485 R
0 1 V
15 773 R
16 -381 R
0 1 V
16 161 R
15 90 R
16 155 R
16 -542 R
16 82 R
15 345 R
16 -523 R
16 113 R
15 397 R
16 -319 R
16 50 R
16 -71 R
15 217 R
0 1 V
16 -10 R
16 199 R
15 -295 R
16 485 R
16 -388 R
16 -327 R
15 -255 R
16 230 R
16 259 R
16 -494 R
15 537 R
16 -40 R
16 -18 R
15 -115 R
16 76 R
16 -127 R
16 -113 R
15 46 R
16 74 R
16 66 R
15 -202 R
16 322 R
16 528 R
16 -1153 R
15 345 R
16 535 R
16 -555 R
15 48 R
16 308 R
16 -257 R
16 502 R
15 32 R
16 -570 R
16 125 R
15 -86 R
16 -314 R
16 100 R
16 918 R
15 -306 R
16 -464 R
16 -173 R
15 499 R
16 31 R
16 -409 R
0 1 V
16 66 R
15 45 R
16 797 R
1156 2891 Circle
1171 3203 Circle
1187 2965 Circle
1203 3007 Circle
1219 3041 Circle
1234 3123 Circle
1250 2497 Circle
1266 3047 Circle
1281 2549 Circle
1297 3230 Circle
1313 3216 Circle
1329 2458 Circle
1344 2407 Circle
1360 2693 Circle
1376 2700 Circle
1391 3077 Circle
1407 2763 Circle
1423 2445 Circle
1439 2743 Circle
1454 2871 Circle
1470 2653 Circle
1486 2854 Circle
1501 2912 Circle
1517 2510 Circle
1533 2699 Circle
1549 2733 Circle
1564 2729 Circle
1580 2798 Circle
1596 3037 Circle
1611 3166 Circle
1627 2916 Circle
1643 2684 Circle
1659 3251 Circle
1674 2969 Circle
1690 2537 Circle
1706 2895 Circle
1721 2972 Circle
1737 3133 Circle
1753 2968 Circle
1769 2627 Circle
1784 3070 Circle
1800 2941 Circle
1816 2569 Circle
1832 2879 Circle
1847 2580 Circle
1863 2851 Circle
1879 2727 Circle
1894 3131 Circle
1910 3028 Circle
1926 3141 Circle
1942 3098 Circle
1957 2757 Circle
1973 2926 Circle
1989 2637 Circle
2004 2597 Circle
2020 3075 Circle
2036 2813 Circle
2052 2847 Circle
2067 2887 Circle
2083 2323 Circle
2099 3280 Circle
2114 2850 Circle
2130 2845 Circle
2146 3085 Circle
2162 2915 Circle
2177 2837 Circle
2193 2880 Circle
2209 2684 Circle
2224 3048 Circle
2240 2929 Circle
2256 2886 Circle
2272 2647 Circle
2287 2758 Circle
2303 3091 Circle
2319 2738 Circle
2334 2772 Circle
2350 2594 Circle
2366 2731 Circle
2382 3064 Circle
2397 3106 Circle
2413 3204 Circle
2429 2960 Circle
2444 3001 Circle
2460 3068 Circle
2476 3035 Circle
2492 2441 Circle
2507 2672 Circle
2523 3298 Circle
2539 2802 Circle
2554 2678 Circle
2570 2809 Circle
2586 2883 Circle
2602 3129 Circle
2617 2920 Circle
2633 2969 Circle
2649 2913 Circle
2664 2754 Circle
2680 2338 Circle
2696 2860 Circle
2712 3040 Circle
2727 3215 Circle
2743 2818 Circle
2759 3002 Circle
2774 2487 Circle
2790 2429 Circle
2806 2819 Circle
2822 2510 Circle
2837 3086 Circle
2853 2923 Circle
2869 2567 Circle
2884 2926 Circle
2900 2885 Circle
2916 3083 Circle
2932 3169 Circle
2947 2887 Circle
2963 3110 Circle
2979 2907 Circle
2994 2967 Circle
3010 3151 Circle
3026 2866 Circle
3042 2653 Circle
3057 3224 Circle
3073 2726 Circle
3089 2836 Circle
3105 2284 Circle
3120 2985 Circle
3136 3529 Circle
3152 2794 Circle
3167 3231 Circle
3183 3132 Circle
3199 2107 Circle
3215 2593 Circle
3230 3153 Circle
3246 2580 Circle
3262 2782 Circle
3277 2658 Circle
3293 3167 Circle
3309 2836 Circle
3325 3290 Circle
3340 3200 Circle
3356 2875 Circle
3372 2995 Circle
3387 3336 Circle
3403 2638 Circle
3419 2420 Circle
3435 3047 Circle
3450 2745 Circle
3466 2510 Circle
3482 2826 Circle
3497 2674 Circle
3513 2816 Circle
3529 3166 Circle
3545 2949 Circle
3560 2765 Circle
3576 2509 Circle
3592 3174 Circle
3607 3610 Circle
3623 2892 Circle
3639 3273 Circle
3655 2415 Circle
3670 2704 Circle
3686 2605 Circle
3702 2277 Circle
3717 3042 Circle
3733 2272 Circle
3749 2539 Circle
3765 3433 Circle
3780 2531 Circle
3796 2632 Circle
3812 2559 Circle
3827 3212 Circle
3843 3001 Circle
3859 2349 Circle
3875 3022 Circle
3890 2895 Circle
3906 3290 Circle
3922 2912 Circle
3937 2836 Circle
3953 3068 Circle
3969 3414 Circle
3985 2608 Circle
4000 2966 Circle
4016 2643 Circle
4032 2659 Circle
4047 3293 Circle
4063 2533 Circle
4079 2570 Circle
4095 3046 Circle
4110 3018 Circle
4126 2787 Circle
4142 3023 Circle
4157 2754 Circle
4173 3524 Circle
4189 2735 Circle
4205 3183 Circle
4220 2712 Circle
4236 2816 Circle
4252 2967 Circle
4267 2624 Circle
4283 3518 Circle
4299 2876 Circle
4315 3600 Circle
4330 2849 Circle
4346 2362 Circle
4362 3450 Circle
4377 2713 Circle
4393 2896 Circle
4409 2831 Circle
4425 3113 Circle
4440 2730 Circle
4456 2565 Circle
4472 2733 Circle
4488 2648 Circle
4503 3405 Circle
4519 2992 Circle
4535 2731 Circle
4550 2686 Circle
4566 3217 Circle
4582 2505 Circle
4598 2870 Circle
4613 2806 Circle
4629 2953 Circle
4645 2472 Circle
4660 2892 Circle
4676 3024 Circle
4692 2776 Circle
4708 2993 Circle
4723 2443 Circle
4739 2627 Circle
4755 2965 Circle
4770 3078 Circle
4786 2713 Circle
4802 2802 Circle
4818 2733 Circle
4833 2793 Circle
4849 3155 Circle
4865 3121 Circle
4880 2802 Circle
4896 2653 Circle
4912 3018 Circle
4928 2804 Circle
4943 2416 Circle
4959 3033 Circle
4975 3143 Circle
4990 2875 Circle
5006 3168 Circle
5022 2840 Circle
5038 2719 Circle
5053 3007 Circle
5069 3229 Circle
5085 2687 Circle
5100 2561 Circle
5116 2526 Circle
5132 2769 Circle
5148 2595 Circle
5163 2869 Circle
5179 3174 Circle
5195 2603 Circle
5210 2871 Circle
5226 2728 Circle
5242 3009 Circle
5258 2806 Circle
5273 2904 Circle
5289 3081 Circle
5305 3101 Circle
5320 3132 Circle
5336 3324 Circle
5352 2924 Circle
5368 2533 Circle
5383 2860 Circle
5399 2939 Circle
5415 3172 Circle
5430 2746 Circle
5446 2925 Circle
5462 2844 Circle
5478 3025 Circle
5493 2901 Circle
5509 2772 Circle
5525 2662 Circle
5540 2717 Circle
5556 2764 Circle
5572 2592 Circle
5588 3047 Circle
5603 2332 Circle
5619 2708 Circle
5635 2922 Circle
5650 2968 Circle
5666 2808 Circle
5682 2925 Circle
5698 2519 Circle
5713 2511 Circle
5729 2708 Circle
5745 3089 Circle
5761 2900 Circle
5776 3140 Circle
5792 2232 Circle
5808 3001 Circle
5823 3100 Circle
5839 3076 Circle
5855 2624 Circle
5871 2936 Circle
5886 3123 Circle
5902 3053 Circle
5918 2913 Circle
5933 2942 Circle
5949 3043 Circle
5965 2809 Circle
5981 2940 Circle
5996 2558 Circle
6012 2398 Circle
6028 3093 Circle
6043 2609 Circle
6059 2938 Circle
6075 2757 Circle
6091 2752 Circle
6106 2661 Circle
6122 2836 Circle
6138 2832 Circle
6153 2951 Circle
6169 2856 Circle
6185 2828 Circle
6201 3056 Circle
6216 3109 Circle
6232 2736 Circle
6248 2673 Circle
6263 2686 Circle
6279 3448 Circle
6295 3017 Circle
6311 2392 Circle
6326 2511 Circle
6342 2986 Circle
6358 3454 Circle
6373 2717 Circle
6389 2970 Circle
6405 2775 Circle
6421 2570 Circle
6436 2863 Circle
6452 2543 Circle
6468 2708 Circle
6483 2554 Circle
6499 2694 Circle
6515 3134 Circle
6531 2704 Circle
6546 3035 Circle
6562 2716 Circle
6578 2260 Circle
6593 3176 Circle
6609 3184 Circle
6625 3264 Circle
6641 2835 Circle
6656 2933 Circle
6672 2736 Circle
6688 2642 Circle
6703 2405 Circle
6719 2818 Circle
6735 3269 Circle
6751 3136 Circle
6766 2739 Circle
6782 2992 Circle
6798 2906 Circle
6813 3186 Circle
6829 2557 Circle
6845 2808 Circle
6861 3129 Circle
6876 2415 Circle
6892 2946 Circle
6908 2965 Circle
6923 2810 Circle
6939 2783 Circle
6955 2606 Circle
6971 2859 Circle
6986 2919 Circle
7002 2920 Circle
7018 2844 Circle
7034 3209 Circle
7049 3244 Circle
7065 2507 Circle
7081 2484 Circle
7096 2372 Circle
7112 2489 Circle
7128 2758 Circle
7144 3174 Circle
7159 3209 Circle
7175 2599 Circle
7191 2571 Circle
7206 2848 Circle
7222 2780 Circle
7238 2852 Circle
7254 2939 Circle
7269 2829 Circle
7285 2490 Circle
7301 2873 Circle
7316 2965 Circle
7332 3472 Circle
7348 3162 Circle
7364 2678 Circle
7379 3295 Circle
7395 2645 Circle
7411 3146 Circle
7426 2734 Circle
7442 2698 Circle
7458 2824 Circle
7474 2589 Circle
7489 2932 Circle
7505 3009 Circle
7521 2685 Circle
7536 2821 Circle
7552 3140 Circle
7568 3070 Circle
7584 2713 Circle
7599 2518 Circle
7615 2161 Circle
7631 2234 Circle
7646 3311 Circle
7662 2668 Circle
7678 2362 Circle
7694 2806 Circle
7709 2727 Circle
7725 2893 Circle
7741 3067 Circle
7756 2699 Circle
7772 3180 Circle
7788 2718 Circle
7804 2666 Circle
7819 2959 Circle
7835 2573 Circle
7851 3267 Circle
7866 3140 Circle
7882 2431 Circle
7898 2652 Circle
7914 2998 Circle
7929 2823 Circle
7945 3056 Circle
7961 2958 Circle
7976 2949 Circle
7992 2664 Circle
8008 2911 Circle
8024 2426 Circle
8039 3200 Circle
8055 2819 Circle
8071 2981 Circle
8086 3071 Circle
8102 3226 Circle
8118 2684 Circle
8134 2766 Circle
8149 3111 Circle
8165 2588 Circle
8181 2701 Circle
8196 3098 Circle
8212 2779 Circle
8228 2829 Circle
8244 2758 Circle
8259 2976 Circle
8275 2966 Circle
8291 3165 Circle
8306 2870 Circle
8322 3355 Circle
8338 2967 Circle
8354 2640 Circle
8369 2385 Circle
8385 2615 Circle
8401 2874 Circle
8417 2380 Circle
8432 2917 Circle
8448 2877 Circle
8464 2859 Circle
8479 2744 Circle
8495 2820 Circle
8511 2693 Circle
8527 2580 Circle
8542 2626 Circle
8558 2700 Circle
8574 2766 Circle
8589 2564 Circle
8605 2886 Circle
8621 3414 Circle
8637 2261 Circle
8652 2606 Circle
8668 3141 Circle
8684 2586 Circle
8699 2634 Circle
8715 2942 Circle
8731 2685 Circle
8747 3187 Circle
8762 3219 Circle
8778 2649 Circle
8794 2774 Circle
8809 2688 Circle
8825 2374 Circle
8841 2474 Circle
8857 3392 Circle
8872 3086 Circle
8888 2622 Circle
8904 2449 Circle
8919 2948 Circle
8935 2979 Circle
8951 2570 Circle
8967 2637 Circle
8982 2682 Circle
8998 3479 Circle
1.500 UP
1.000 UL
LTb
0.58 0.00 0.83 C 1156 3248 M
15 -6 R
16 -17 R
16 5 R
16 37 R
0 1 V
15 -49 R
16 19 R
16 12 R
15 -26 R
16 14 R
16 14 R
16 -33 R
15 -13 R
16 29 R
16 3 R
15 5 R
16 -7 R
16 2 R
16 15 R
15 -32 R
16 10 R
16 -14 R
15 14 R
16 28 R
16 -39 R
16 29 R
15 -12 R
16 10 R
0 1 V
16 -16 R
0 1 V
15 16 R
16 -9 R
16 -1 R
16 -19 R
15 23 R
16 13 R
16 -34 R
15 18 R
16 6 R
16 -9 R
16 -2 R
15 6 R
16 -7 R
16 -6 R
16 26 R
15 -24 R
16 -5 R
16 4 R
15 29 R
16 5 R
16 -24 R
16 2 R
15 -15 R
16 8 R
16 3 R
15 -17 R
16 22 R
16 -44 R
16 42 R
15 7 R
16 -12 R
16 -16 R
15 34 R
16 -31 R
16 12 R
16 10 R
15 -20 R
16 3 R
16 -5 R
15 22 R
16 -16 R
16 39 R
16 -38 R
15 -5 R
16 2 R
16 2 R
0 1 V
15 -19 R
16 31 R
16 4 R
16 -25 R
15 -10 R
16 47 R
16 5 R
15 -18 R
16 -18 R
16 2 R
0 1 V
16 -18 R
15 42 R
16 -7 R
16 -25 R
15 8 R
16 -8 R
16 21 R
16 8 R
15 -7 R
16 6 R
16 -12 R
15 8 R
16 -4 R
16 0 R
16 0 R
15 -19 R
16 25 R
16 -4 R
15 -27 R
16 -1 R
16 30 R
16 8 R
15 -26 R
16 17 R
16 -6 R
15 -23 R
16 19 R
16 -3 R
16 8 R
15 2 R
16 -5 R
16 13 R
15 -14 R
16 7 R
16 -27 R
16 15 R
15 4 R
16 -7 R
16 -15 R
0 1 V
stroke 3089 3213 M
16 30 R
15 -9 R
16 4 R
16 -21 R
15 30 R
0 1 V
16 -16 R
0 1 V
16 13 R
16 -7 R
15 -5 R
16 0 R
16 15 R
15 -14 R
16 4 R
16 5 R
16 4 R
15 -15 R
16 1 R
16 -10 R
15 -19 R
16 27 R
0 1 V
16 16 R
16 -27 R
15 -11 R
16 42 R
16 -17 R
15 25 R
16 -2 R
16 -27 R
16 -6 R
15 9 R
16 9 R
16 -32 R
15 52 R
16 -32 R
16 0 R
16 6 R
15 -9 R
16 -4 R
16 3 R
15 1 R
16 -15 R
16 21 R
16 -16 R
15 40 R
0 1 V
16 -34 R
16 17 R
15 -15 R
16 0 R
16 10 R
16 -20 R
15 -4 R
16 22 R
16 -24 R
15 30 R
16 -33 R
16 0 R
16 16 R
15 27 R
16 -19 R
16 31 R
15 -30 R
16 0 R
16 4 R
16 -5 R
15 4 R
16 -13 R
16 3 R
15 1 R
16 -22 R
16 -3 R
16 11 R
15 26 R
16 -4 R
16 10 R
15 8 R
16 8 R
16 -40 R
16 -8 R
15 22 R
16 22 R
16 -16 R
15 10 R
16 -26 R
16 14 R
16 -13 R
15 15 R
16 -14 R
0 1 V
16 4 R
16 7 R
15 5 R
16 -10 R
16 29 R
15 -15 R
16 15 R
16 -17 R
0 1 V
16 -33 R
15 11 R
16 -10 R
16 19 R
15 -12 R
16 -26 R
16 27 R
0 1 V
stroke 4692 3225 M
16 -11 R
15 -1 R
16 26 R
16 -18 R
15 22 R
16 -3 R
16 -6 R
16 -4 R
15 -24 R
16 40 R
16 -17 R
15 15 R
16 -26 R
16 -4 R
16 25 R
15 4 R
16 -8 R
16 -27 R
15 28 R
16 -10 R
16 37 R
16 -44 R
15 4 R
16 14 R
16 -36 R
15 24 R
16 17 R
16 -11 R
16 6 R
15 13 R
16 -38 R
16 16 R
15 11 R
16 24 R
16 -41 R
16 18 R
15 -5 R
16 -13 R
16 18 R
15 -3 R
16 -17 R
0 1 V
16 13 R
16 23 R
15 -22 R
16 0 R
16 19 R
15 -22 R
0 1 V
16 -1 R
0 1 V
16 0 R
16 10 R
15 7 R
16 -13 R
16 -14 R
15 2 R
16 1 R
16 2 R
0 1 V
16 -8 R
15 37 R
16 -23 R
16 20 R
15 5 R
16 -9 R
16 -5 R
16 1 R
15 5 R
16 2 R
16 -25 R
16 -8 R
15 7 R
16 -8 R
16 52 R
15 -23 R
16 -5 R
16 -31 R
16 24 R
15 17 R
16 0 R
16 -19 R
15 17 R
16 -23 R
16 14 R
16 -7 R
15 1 R
16 16 R
16 -7 R
15 -5 R
16 3 R
16 -8 R
16 3 R
15 22 R
16 -22 R
16 0 R
15 12 R
16 -40 R
16 19 R
16 -12 R
15 21 R
16 -17 R
16 -9 R
15 9 R
16 31 R
16 0 R
16 -22 R
15 3 R
16 2 R
16 -31 R
15 40 R
16 -25 R
16 -3 R
16 2 R
15 17 R
16 8 R
16 -4 R
15 -4 R
16 -7 R
16 7 R
16 -15 R
15 25 R
16 1 R
16 3 R
15 -20 R
16 -20 R
16 30 R
16 -5 R
15 1 R
16 -1 R
16 -3 R
0 1 V
stroke 6688 3233 M
15 13 R
16 -3 R
16 -17 R
16 -12 R
15 -3 R
16 31 R
16 -9 R
15 -10 R
16 53 R
16 -23 R
16 -8 R
15 4 R
16 -5 R
16 9 R
15 -1 R
16 -4 R
16 -19 R
16 37 R
15 -52 R
16 20 R
16 -10 R
16 -6 R
15 -13 R
16 31 R
16 -6 R
15 11 R
16 -6 R
16 -4 R
16 0 R
15 9 R
16 -12 R
16 3 R
15 24 R
16 -24 R
16 16 R
16 -53 R
15 43 R
16 -5 R
16 -2 R
15 11 R
16 15 R
16 -23 R
16 4 R
15 6 R
0 1 V
16 3 R
0 1 V
16 -9 R
15 -14 R
16 -8 R
16 22 R
16 -25 R
15 34 R
16 -12 R
16 -1 R
15 -26 R
16 -15 R
16 35 R
16 6 R
15 -19 R
0 1 V
16 13 R
16 -1 R
15 3 R
0 1 V
16 3 R
16 -23 R
16 40 R
15 -35 R
16 17 R
16 -12 R
15 -3 R
16 27 R
16 -29 R
16 8 R
15 -17 R
16 2 R
16 31 R
15 17 R
16 -7 R
16 -10 R
16 -4 R
15 7 R
16 -1 R
16 -13 R
15 8 R
16 0 R
16 -10 R
16 -26 R
15 44 R
16 -8 R
16 25 R
15 -41 R
16 -8 R
16 -14 R
16 25 R
15 30 R
16 -7 R
16 -6 R
15 -15 R
16 -14 R
16 11 R
16 39 R
15 -10 R
16 -12 R
16 -20 R
15 -4 R
16 21 R
16 -19 R
16 12 R
15 0 R
0 1 V
stroke 8369 3233 M
16 -9 R
16 18 R
16 -18 R
0 1 V
15 2 R
16 20 R
0 1 V
16 -8 R
15 1 R
16 -9 R
16 -14 R
16 -10 R
0 1 V
15 13 R
16 27 R
16 -31 R
15 22 R
16 -1 R
16 -21 R
16 17 R
15 -36 R
16 60 R
0 1 V
16 -23 R
15 -5 R
0 1 V
16 7 R
16 -9 R
16 20 R
0 1 V
15 -3 R
16 -28 R
16 28 R
15 -2 R
0 1 V
16 -12 R
16 -3 R
16 13 R
15 -2 R
16 -8 R
16 -16 R
15 32 R
16 -21 R
16 1 R
0 1 V
16 -8 R
15 -10 R
16 26 R
1156 3248 CircleF
1171 3242 CircleF
1187 3225 CircleF
1203 3230 CircleF
1219 3268 CircleF
1234 3219 CircleF
1250 3238 CircleF
1266 3250 CircleF
1281 3224 CircleF
1297 3238 CircleF
1313 3252 CircleF
1329 3219 CircleF
1344 3206 CircleF
1360 3235 CircleF
1376 3238 CircleF
1391 3243 CircleF
1407 3236 CircleF
1423 3238 CircleF
1439 3253 CircleF
1454 3221 CircleF
1470 3231 CircleF
1486 3217 CircleF
1501 3231 CircleF
1517 3259 CircleF
1533 3220 CircleF
1549 3249 CircleF
1564 3237 CircleF
1580 3247 CircleF
1596 3233 CircleF
1611 3249 CircleF
1627 3240 CircleF
1643 3239 CircleF
1659 3220 CircleF
1674 3243 CircleF
1690 3256 CircleF
1706 3222 CircleF
1721 3240 CircleF
1737 3246 CircleF
1753 3237 CircleF
1769 3235 CircleF
1784 3241 CircleF
1800 3234 CircleF
1816 3228 CircleF
1832 3254 CircleF
1847 3230 CircleF
1863 3225 CircleF
1879 3229 CircleF
1894 3258 CircleF
1910 3263 CircleF
1926 3239 CircleF
1942 3241 CircleF
1957 3226 CircleF
1973 3234 CircleF
1989 3237 CircleF
2004 3220 CircleF
2020 3242 CircleF
2036 3198 CircleF
2052 3240 CircleF
2067 3247 CircleF
2083 3235 CircleF
2099 3219 CircleF
2114 3253 CircleF
2130 3222 CircleF
2146 3234 CircleF
2162 3244 CircleF
2177 3224 CircleF
2193 3227 CircleF
2209 3222 CircleF
2224 3244 CircleF
2240 3228 CircleF
2256 3267 CircleF
2272 3229 CircleF
2287 3224 CircleF
2303 3226 CircleF
2319 3229 CircleF
2334 3210 CircleF
2350 3241 CircleF
2366 3245 CircleF
2382 3220 CircleF
2397 3210 CircleF
2413 3257 CircleF
2429 3262 CircleF
2444 3244 CircleF
2460 3226 CircleF
2476 3229 CircleF
2492 3211 CircleF
2507 3253 CircleF
2523 3246 CircleF
2539 3221 CircleF
2554 3229 CircleF
2570 3221 CircleF
2586 3242 CircleF
2602 3250 CircleF
2617 3243 CircleF
2633 3249 CircleF
2649 3237 CircleF
2664 3245 CircleF
2680 3241 CircleF
2696 3241 CircleF
2712 3241 CircleF
2727 3222 CircleF
2743 3247 CircleF
2759 3243 CircleF
2774 3216 CircleF
2790 3215 CircleF
2806 3245 CircleF
2822 3253 CircleF
2837 3227 CircleF
2853 3244 CircleF
2869 3238 CircleF
2884 3215 CircleF
2900 3234 CircleF
2916 3231 CircleF
2932 3239 CircleF
2947 3241 CircleF
2963 3236 CircleF
2979 3249 CircleF
2994 3235 CircleF
3010 3242 CircleF
3026 3215 CircleF
3042 3230 CircleF
3057 3234 CircleF
3073 3227 CircleF
3089 3212 CircleF
3105 3243 CircleF
3120 3234 CircleF
3136 3238 CircleF
3152 3217 CircleF
3167 3247 CircleF
3183 3233 CircleF
3199 3246 CircleF
3215 3239 CircleF
3230 3234 CircleF
3246 3234 CircleF
3262 3249 CircleF
3277 3235 CircleF
3293 3239 CircleF
3309 3244 CircleF
3325 3248 CircleF
3340 3233 CircleF
3356 3234 CircleF
3372 3224 CircleF
3387 3205 CircleF
3403 3233 CircleF
3419 3249 CircleF
3435 3222 CircleF
3450 3211 CircleF
3466 3253 CircleF
3482 3236 CircleF
3497 3261 CircleF
3513 3259 CircleF
3529 3232 CircleF
3545 3226 CircleF
3560 3235 CircleF
3576 3244 CircleF
3592 3212 CircleF
3607 3264 CircleF
3623 3232 CircleF
3639 3232 CircleF
3655 3238 CircleF
3670 3229 CircleF
3686 3225 CircleF
3702 3228 CircleF
3717 3229 CircleF
3733 3214 CircleF
3749 3235 CircleF
3765 3219 CircleF
3780 3260 CircleF
3796 3226 CircleF
3812 3243 CircleF
3827 3228 CircleF
3843 3228 CircleF
3859 3238 CircleF
3875 3218 CircleF
3890 3214 CircleF
3906 3236 CircleF
3922 3212 CircleF
3937 3242 CircleF
3953 3209 CircleF
3969 3209 CircleF
3985 3225 CircleF
4000 3252 CircleF
4016 3233 CircleF
4032 3264 CircleF
4047 3234 CircleF
4063 3234 CircleF
4079 3238 CircleF
4095 3233 CircleF
4110 3237 CircleF
4126 3224 CircleF
4142 3227 CircleF
4157 3228 CircleF
4173 3206 CircleF
4189 3203 CircleF
4205 3214 CircleF
4220 3240 CircleF
4236 3236 CircleF
4252 3246 CircleF
4267 3254 CircleF
4283 3262 CircleF
4299 3222 CircleF
4315 3214 CircleF
4330 3236 CircleF
4346 3258 CircleF
4362 3242 CircleF
4377 3252 CircleF
4393 3226 CircleF
4409 3240 CircleF
4425 3227 CircleF
4440 3242 CircleF
4456 3229 CircleF
4472 3233 CircleF
4488 3240 CircleF
4503 3245 CircleF
4519 3235 CircleF
4535 3264 CircleF
4550 3249 CircleF
4566 3264 CircleF
4582 3247 CircleF
4598 3215 CircleF
4613 3226 CircleF
4629 3216 CircleF
4645 3235 CircleF
4660 3223 CircleF
4676 3197 CircleF
4692 3225 CircleF
4708 3214 CircleF
4723 3213 CircleF
4739 3239 CircleF
4755 3221 CircleF
4770 3243 CircleF
4786 3240 CircleF
4802 3234 CircleF
4818 3230 CircleF
4833 3206 CircleF
4849 3246 CircleF
4865 3229 CircleF
4880 3244 CircleF
4896 3218 CircleF
4912 3214 CircleF
4928 3239 CircleF
4943 3243 CircleF
4959 3235 CircleF
4975 3208 CircleF
4990 3236 CircleF
5006 3226 CircleF
5022 3263 CircleF
5038 3219 CircleF
5053 3223 CircleF
5069 3237 CircleF
5085 3201 CircleF
5100 3225 CircleF
5116 3242 CircleF
5132 3231 CircleF
5148 3237 CircleF
5163 3250 CircleF
5179 3212 CircleF
5195 3228 CircleF
5210 3239 CircleF
5226 3263 CircleF
5242 3222 CircleF
5258 3240 CircleF
5273 3235 CircleF
5289 3222 CircleF
5305 3240 CircleF
5320 3237 CircleF
5336 3220 CircleF
5352 3234 CircleF
5368 3257 CircleF
5383 3235 CircleF
5399 3235 CircleF
5415 3254 CircleF
5430 3233 CircleF
5446 3233 CircleF
5462 3233 CircleF
5478 3243 CircleF
5493 3250 CircleF
5509 3237 CircleF
5525 3223 CircleF
5540 3225 CircleF
5556 3226 CircleF
5572 3229 CircleF
5588 3221 CircleF
5603 3258 CircleF
5619 3235 CircleF
5635 3255 CircleF
5650 3260 CircleF
5666 3251 CircleF
5682 3246 CircleF
5698 3247 CircleF
5713 3252 CircleF
5729 3254 CircleF
5745 3229 CircleF
5761 3221 CircleF
5776 3228 CircleF
5792 3220 CircleF
5808 3272 CircleF
5823 3249 CircleF
5839 3244 CircleF
5855 3213 CircleF
5871 3237 CircleF
5886 3254 CircleF
5902 3254 CircleF
5918 3235 CircleF
5933 3252 CircleF
5949 3229 CircleF
5965 3243 CircleF
5981 3236 CircleF
5996 3237 CircleF
6012 3253 CircleF
6028 3246 CircleF
6043 3241 CircleF
6059 3244 CircleF
6075 3236 CircleF
6091 3239 CircleF
6106 3261 CircleF
6122 3239 CircleF
6138 3239 CircleF
6153 3251 CircleF
6169 3211 CircleF
6185 3230 CircleF
6201 3218 CircleF
6216 3239 CircleF
6232 3222 CircleF
6248 3213 CircleF
6263 3222 CircleF
6279 3253 CircleF
6295 3253 CircleF
6311 3231 CircleF
6326 3234 CircleF
6342 3236 CircleF
6358 3205 CircleF
6373 3245 CircleF
6389 3220 CircleF
6405 3217 CircleF
6421 3219 CircleF
6436 3236 CircleF
6452 3244 CircleF
6468 3240 CircleF
6483 3236 CircleF
6499 3229 CircleF
6515 3236 CircleF
6531 3221 CircleF
6546 3246 CircleF
6562 3247 CircleF
6578 3250 CircleF
6593 3230 CircleF
6609 3210 CircleF
6625 3240 CircleF
6641 3235 CircleF
6656 3236 CircleF
6672 3235 CircleF
6688 3233 CircleF
6703 3246 CircleF
6719 3243 CircleF
6735 3226 CircleF
6751 3214 CircleF
6766 3211 CircleF
6782 3242 CircleF
6798 3233 CircleF
6813 3223 CircleF
6829 3276 CircleF
6845 3253 CircleF
6861 3245 CircleF
6876 3249 CircleF
6892 3244 CircleF
6908 3253 CircleF
6923 3252 CircleF
6939 3248 CircleF
6955 3229 CircleF
6971 3266 CircleF
6986 3214 CircleF
7002 3234 CircleF
7018 3224 CircleF
7034 3218 CircleF
7049 3205 CircleF
7065 3236 CircleF
7081 3230 CircleF
7096 3241 CircleF
7112 3235 CircleF
7128 3231 CircleF
7144 3231 CircleF
7159 3240 CircleF
7175 3228 CircleF
7191 3231 CircleF
7206 3255 CircleF
7222 3231 CircleF
7238 3247 CircleF
7254 3194 CircleF
7269 3237 CircleF
7285 3232 CircleF
7301 3230 CircleF
7316 3241 CircleF
7332 3256 CircleF
7348 3233 CircleF
7364 3237 CircleF
7379 3243 CircleF
7395 3247 CircleF
7411 3239 CircleF
7426 3225 CircleF
7442 3217 CircleF
7458 3239 CircleF
7474 3214 CircleF
7489 3248 CircleF
7505 3236 CircleF
7521 3235 CircleF
7536 3209 CircleF
7552 3194 CircleF
7568 3229 CircleF
7584 3235 CircleF
7599 3216 CircleF
7615 3230 CircleF
7631 3229 CircleF
7646 3233 CircleF
7662 3236 CircleF
7678 3213 CircleF
7694 3253 CircleF
7709 3218 CircleF
7725 3235 CircleF
7741 3223 CircleF
7756 3220 CircleF
7772 3247 CircleF
7788 3218 CircleF
7804 3226 CircleF
7819 3209 CircleF
7835 3211 CircleF
7851 3242 CircleF
7866 3259 CircleF
7882 3252 CircleF
7898 3242 CircleF
7914 3238 CircleF
7929 3245 CircleF
7945 3244 CircleF
7961 3231 CircleF
7976 3239 CircleF
7992 3239 CircleF
8008 3229 CircleF
8024 3203 CircleF
8039 3247 CircleF
8055 3239 CircleF
8071 3264 CircleF
8086 3223 CircleF
8102 3215 CircleF
8118 3201 CircleF
8134 3226 CircleF
8149 3256 CircleF
8165 3249 CircleF
8181 3243 CircleF
8196 3228 CircleF
8212 3214 CircleF
8228 3225 CircleF
8244 3264 CircleF
8259 3254 CircleF
8275 3242 CircleF
8291 3222 CircleF
8306 3218 CircleF
8322 3239 CircleF
8338 3220 CircleF
8354 3232 CircleF
8369 3233 CircleF
8385 3224 CircleF
8401 3242 CircleF
8417 3225 CircleF
8432 3227 CircleF
8448 3247 CircleF
8464 3240 CircleF
8479 3241 CircleF
8495 3232 CircleF
8511 3218 CircleF
8527 3208 CircleF
8542 3222 CircleF
8558 3249 CircleF
8574 3218 CircleF
8589 3240 CircleF
8605 3239 CircleF
8621 3218 CircleF
8637 3235 CircleF
8652 3199 CircleF
8668 3260 CircleF
8684 3237 CircleF
8699 3233 CircleF
8715 3240 CircleF
8731 3231 CircleF
8747 3251 CircleF
8762 3249 CircleF
8778 3221 CircleF
8794 3249 CircleF
8809 3247 CircleF
8825 3236 CircleF
8841 3233 CircleF
8857 3246 CircleF
8872 3244 CircleF
8888 3236 CircleF
8904 3220 CircleF
8919 3252 CircleF
8935 3231 CircleF
8951 3233 CircleF
8967 3225 CircleF
8982 3215 CircleF
8998 3241 CircleF
1.500 UP
1.000 UL
LTb
0.58 0.00 0.83 C 1156 4889 M
15 790 R
16 -678 R
16 408 R
16 265 R
15 -664 R
16 -70 R
0 1 V
16 -144 R
15 581 R
16 -852 R
16 222 R
16 1096 R
0 1 V
15 -637 R
16 42 R
16 238 R
15 -199 R
16 24 R
16 -398 R
16 69 R
15 182 R
16 -63 R
16 225 R
15 -583 R
16 253 R
16 507 R
16 -776 R
15 681 R
16 18 R
16 -9 R
15 -726 R
16 368 R
16 256 R
16 404 R
15 -904 R
16 394 R
16 -267 R
0 1 V
15 514 R
16 -161 R
16 -142 R
16 135 R
15 -460 R
0 1 V
16 396 R
0 1 V
16 174 R
0 1 V
16 -328 R
15 -200 R
16 -31 R
16 164 R
15 -53 R
16 -148 R
16 195 R
16 146 R
15 -308 R
16 472 R
16 -230 R
15 -105 R
16 231 R
16 -43 R
16 69 R
15 -379 R
16 -45 R
16 669 R
15 -419 R
16 -175 R
0 1 V
16 -120 R
16 311 R
15 -78 R
16 85 R
16 -102 R
15 438 R
16 8 R
16 -375 R
16 561 R
15 -242 R
16 92 R
16 -9 R
15 -390 R
16 -363 R
16 568 R
16 -222 R
0 1 V
15 277 R
16 -216 R
16 -395 R
15 130 R
16 -120 R
16 183 R
16 -83 R
15 247 R
16 -246 R
16 580 R
15 -313 R
16 -389 R
16 336 R
16 -133 R
15 397 R
16 -112 R
0 1 V
16 245 R
15 -487 R
16 335 R
16 -57 R
16 -13 R
0 1 V
stroke 2712 5218 M
15 215 R
16 -575 R
16 170 R
15 507 R
16 -645 R
16 218 R
16 -322 R
15 418 R
16 306 R
16 -603 R
15 -66 R
16 93 R
16 46 R
16 -675 R
15 1092 R
16 -65 R
16 41 R
0 1 V
15 -114 R
16 -170 R
16 -613 R
16 388 R
15 107 R
16 231 R
16 -173 R
16 -191 R
15 286 R
16 729 R
16 -812 R
15 475 R
0 1 V
16 -441 R
16 159 R
16 197 R
15 -359 R
16 5 R
16 13 R
15 -286 R
16 612 R
16 -164 R
16 -304 R
15 10 R
16 338 R
0 1 V
16 -304 R
15 397 R
16 256 R
16 -608 R
16 -71 R
15 -321 R
0 1 V
16 432 R
0 1 V
16 -723 R
15 842 R
16 -55 R
16 381 R
16 -643 R
15 323 R
16 -148 R
16 -57 R
15 162 R
16 -300 R
16 108 R
16 -184 R
15 371 R
16 148 R
16 -133 R
0 1 V
15 -104 R
16 138 R
16 181 R
16 -594 R
15 -23 R
16 468 R
16 -175 R
15 -38 R
16 -121 R
16 569 R
16 -348 R
15 -155 R
16 -67 R
16 164 R
15 204 R
16 -836 R
16 380 R
16 657 R
15 -699 R
16 281 R
16 -236 R
15 776 R
16 -488 R
16 -79 R
16 -265 R
0 1 V
15 459 R
16 -82 R
16 162 R
15 -154 R
16 -333 R
16 77 R
16 606 R
0 1 V
15 -237 R
16 -352 R
16 492 R
15 -1 R
16 -460 R
16 -62 R
16 85 R
15 -173 R
16 168 R
16 320 R
15 8 R
0 1 V
stroke 4377 5291 M
16 163 R
0 1 V
16 -592 R
0 1 V
16 312 R
15 -20 R
16 -271 R
16 -165 R
16 652 R
15 -299 R
16 -281 R
0 1 V
16 21 R
15 607 R
16 -622 R
16 156 R
16 319 R
15 -331 R
16 283 R
16 -288 R
15 -69 R
16 510 R
16 -348 R
16 -303 R
15 379 R
16 -404 R
16 63 R
15 357 R
16 -177 R
16 99 R
16 44 R
15 -63 R
0 1 V
16 391 R
16 -513 R
15 250 R
0 1 V
16 120 R
16 -514 R
16 570 R
15 66 R
16 -50 R
16 -256 R
0 1 V
15 187 R
16 -24 R
16 -540 R
16 815 R
15 -579 R
0 1 V
16 391 R
16 -328 R
15 320 R
16 53 R
16 -370 R
16 318 R
15 -296 R
16 464 R
16 -23 R
15 -315 R
16 60 R
16 -666 R
16 250 R
15 300 R
0 1 V
16 -368 R
16 163 R
15 -1 R
16 322 R
16 -313 R
16 -56 R
15 42 R
16 491 R
16 26 R
15 -323 R
16 -1 R
16 136 R
16 -338 R
15 436 R
16 -461 R
16 9 R
15 -78 R
16 -321 R
16 31 R
16 738 R
15 -372 R
0 1 V
16 299 R
0 1 V
16 96 R
15 -48 R
16 -113 R
16 -201 R
16 442 R
15 144 R
16 -599 R
16 -69 R
16 222 R
0 1 V
15 -312 R
16 294 R
16 165 R
15 152 R
16 115 R
16 -686 R
16 283 R
15 -302 R
16 213 R
16 47 R
15 -162 R
16 -159 R
16 292 R
16 418 R
0 1 V
stroke 5981 5409 M
15 -683 R
0 1 V
16 617 R
16 -367 R
15 382 R
16 16 R
16 -178 R
0 1 V
16 -37 R
15 -761 R
16 663 R
16 -290 R
15 236 R
16 -55 R
16 -521 R
16 757 R
15 93 R
16 -151 R
16 115 R
15 -66 R
16 -290 R
16 174 R
16 -451 R
15 772 R
16 -605 R
16 89 R
0 1 V
15 1073 R
16 -483 R
16 -408 R
16 -462 R
15 567 R
16 89 R
16 -3 R
15 -644 R
16 40 R
16 -40 R
16 892 R
15 176 R
16 -355 R
16 -332 R
15 -61 R
16 217 R
16 88 R
16 -104 R
15 213 R
0 1 V
16 -576 R
16 -224 R
15 773 R
16 -269 R
16 15 R
16 -212 R
15 -58 R
16 71 R
0 1 V
16 659 R
15 -600 R
16 285 R
16 -104 R
0 1 V
16 241 R
15 -182 R
16 -158 R
16 -285 R
15 -37 R
0 1 V
16 231 R
16 361 R
16 222 R
15 -528 R
16 506 R
16 -249 R
0 1 V
16 -385 R
15 303 R
16 127 R
16 63 R
15 15 R
16 -559 R
0 1 V
16 627 R
16 -235 R
0 1 V
15 -475 R
16 -10 R
16 605 R
0 1 V
15 -731 R
16 591 R
16 170 R
16 -180 R
15 260 R
16 -229 R
16 -92 R
15 165 R
16 331 R
16 -645 R
16 225 R
15 -231 R
16 119 R
16 67 R
15 -62 R
16 117 R
16 57 R
16 -363 R
15 -123 R
16 412 R
0 1 V
stroke 7505 5159 M
16 373 R
15 -452 R
16 -179 R
16 337 R
16 40 R
15 57 R
16 -203 R
16 108 R
15 -238 R
16 -266 R
16 188 R
16 18 R
15 205 R
0 1 V
16 173 R
16 406 R
15 -517 R
16 -183 R
16 -266 R
16 -106 R
15 550 R
16 -5 R
16 -40 R
15 180 R
16 -501 R
16 151 R
16 87 R
15 -17 R
16 -118 R
16 -252 R
15 -286 R
16 635 R
16 208 R
16 -122 R
15 458 R
16 -755 R
16 110 R
15 329 R
16 109 R
0 1 V
16 -660 R
16 106 R
15 35 R
16 445 R
16 -241 R
15 -305 R
16 521 R
16 -107 R
16 -35 R
15 -87 R
16 -492 R
16 726 R
15 -465 R
16 555 R
16 -243 R
16 -169 R
15 -1 R
16 270 R
16 -449 R
0 1 V
16 -84 R
15 559 R
16 -655 R
16 864 R
15 506 R
16 -633 R
16 -312 R
16 757 R
15 -568 R
16 -543 R
16 -196 R
0 1 V
15 586 R
16 169 R
16 456 R
16 -259 R
15 13 R
16 127 R
16 -879 R
15 112 R
16 723 R
16 -654 R
16 219 R
15 600 R
16 -718 R
16 -138 R
0 1 V
15 762 R
16 -447 R
0 1 V
16 -666 R
16 627 R
15 623 R
16 -695 R
16 7 R
15 -273 R
16 264 R
16 130 R
16 -275 R
0 1 V
15 -258 R
16 613 R
1156 4889 Box
1171 5679 Box
1187 5001 Box
1203 5409 Box
1219 5674 Box
1234 5010 Box
1250 4940 Box
1266 4797 Box
1281 5378 Box
1297 4526 Box
1313 4748 Box
1329 5845 Box
1344 5208 Box
1360 5250 Box
1376 5488 Box
1391 5289 Box
1407 5313 Box
1423 4915 Box
1439 4984 Box
1454 5166 Box
1470 5103 Box
1486 5328 Box
1501 4745 Box
1517 4998 Box
1533 5505 Box
1549 4729 Box
1564 5410 Box
1580 5428 Box
1596 5419 Box
1611 4693 Box
1627 5061 Box
1643 5317 Box
1659 5721 Box
1674 4817 Box
1690 5211 Box
1706 4944 Box
1721 5459 Box
1737 5298 Box
1753 5156 Box
1769 5291 Box
1784 4831 Box
1800 5228 Box
1816 5403 Box
1832 5076 Box
1847 4876 Box
1863 4845 Box
1879 5009 Box
1894 4956 Box
1910 4808 Box
1926 5003 Box
1942 5149 Box
1957 4841 Box
1973 5313 Box
1989 5083 Box
2004 4978 Box
2020 5209 Box
2036 5166 Box
2052 5235 Box
2067 4856 Box
2083 4811 Box
2099 5480 Box
2114 5061 Box
2130 4887 Box
2146 4767 Box
2162 5078 Box
2177 5000 Box
2193 5085 Box
2209 4983 Box
2224 5421 Box
2240 5429 Box
2256 5054 Box
2272 5615 Box
2287 5373 Box
2303 5465 Box
2319 5456 Box
2334 5066 Box
2350 4703 Box
2366 5271 Box
2382 5049 Box
2397 5327 Box
2413 5111 Box
2429 4716 Box
2444 4846 Box
2460 4726 Box
2476 4909 Box
2492 4826 Box
2507 5073 Box
2523 4827 Box
2539 5407 Box
2554 5094 Box
2570 4705 Box
2586 5041 Box
2602 4908 Box
2617 5305 Box
2633 5193 Box
2649 5439 Box
2664 4952 Box
2680 5287 Box
2696 5230 Box
2712 5218 Box
2727 5433 Box
2743 4858 Box
2759 5028 Box
2774 5535 Box
2790 4890 Box
2806 5108 Box
2822 4786 Box
2837 5204 Box
2853 5510 Box
2869 4907 Box
2884 4841 Box
2900 4934 Box
2916 4980 Box
2932 4305 Box
2947 5397 Box
2963 5332 Box
2979 5374 Box
2994 5260 Box
3010 5090 Box
3026 4477 Box
3042 4865 Box
3057 4972 Box
3073 5203 Box
3089 5030 Box
3105 4839 Box
3120 5125 Box
3136 5854 Box
3152 5042 Box
3167 5518 Box
3183 5077 Box
3199 5236 Box
3215 5433 Box
3230 5074 Box
3246 5079 Box
3262 5092 Box
3277 4806 Box
3293 5418 Box
3309 5254 Box
3325 4950 Box
3340 4960 Box
3356 5298 Box
3372 4995 Box
3387 5392 Box
3403 5648 Box
3419 5040 Box
3435 4969 Box
3450 4648 Box
3466 5082 Box
3482 4359 Box
3497 5201 Box
3513 5146 Box
3529 5527 Box
3545 4884 Box
3560 5207 Box
3576 5059 Box
3592 5002 Box
3607 5164 Box
3623 4864 Box
3639 4972 Box
3655 4788 Box
3670 5159 Box
3686 5307 Box
3702 5175 Box
3717 5071 Box
3733 5209 Box
3749 5390 Box
3765 4796 Box
3780 4773 Box
3796 5241 Box
3812 5066 Box
3827 5028 Box
3843 4907 Box
3859 5476 Box
3875 5128 Box
3890 4973 Box
3906 4906 Box
3922 5070 Box
3937 5274 Box
3953 4438 Box
3969 4818 Box
3985 5475 Box
4000 4776 Box
4016 5057 Box
4032 4821 Box
4047 5597 Box
4063 5109 Box
4079 5030 Box
4095 4765 Box
4110 5225 Box
4126 5143 Box
4142 5305 Box
4157 5151 Box
4173 4818 Box
4189 4895 Box
4205 5502 Box
4220 5265 Box
4236 4913 Box
4252 5405 Box
4267 5404 Box
4283 4944 Box
4299 4882 Box
4315 4967 Box
4330 4794 Box
4346 4962 Box
4362 5282 Box
4377 5290 Box
4393 5454 Box
4409 4864 Box
4425 5176 Box
4440 5156 Box
4456 4885 Box
4472 4720 Box
4488 5372 Box
4503 5073 Box
4519 4792 Box
4535 4814 Box
4550 5421 Box
4566 4799 Box
4582 4955 Box
4598 5274 Box
4613 4943 Box
4629 5226 Box
4645 4938 Box
4660 4869 Box
4676 5379 Box
4692 5031 Box
4708 4728 Box
4723 5107 Box
4739 4703 Box
4755 4766 Box
4770 5123 Box
4786 4946 Box
4802 5045 Box
4818 5089 Box
4833 5027 Box
4849 5418 Box
4865 4905 Box
4880 5156 Box
4896 5276 Box
4912 4762 Box
4928 5332 Box
4943 5398 Box
4959 5348 Box
4975 5092 Box
4990 5280 Box
5006 5256 Box
5022 4716 Box
5038 5531 Box
5053 4952 Box
5069 5344 Box
5085 5016 Box
5100 5336 Box
5116 5389 Box
5132 5019 Box
5148 5337 Box
5163 5041 Box
5179 5505 Box
5195 5482 Box
5210 5167 Box
5226 5227 Box
5242 4561 Box
5258 4811 Box
5273 5111 Box
5289 4744 Box
5305 4907 Box
5320 4906 Box
5336 5228 Box
5352 4915 Box
5368 4859 Box
5383 4901 Box
5399 5392 Box
5415 5418 Box
5430 5095 Box
5446 5094 Box
5462 5230 Box
5478 4892 Box
5493 5328 Box
5509 4867 Box
5525 4876 Box
5540 4798 Box
5556 4477 Box
5572 4508 Box
5588 5246 Box
5603 4874 Box
5619 5175 Box
5635 5271 Box
5650 5223 Box
5666 5110 Box
5682 4909 Box
5698 5351 Box
5713 5495 Box
5729 4896 Box
5745 4827 Box
5761 5049 Box
5776 4738 Box
5792 5032 Box
5808 5197 Box
5823 5349 Box
5839 5464 Box
5855 4778 Box
5871 5061 Box
5886 4759 Box
5902 4972 Box
5918 5019 Box
5933 4857 Box
5949 4698 Box
5965 4990 Box
5981 5409 Box
5996 4726 Box
6012 5344 Box
6028 4977 Box
6043 5359 Box
6059 5375 Box
6075 5197 Box
6091 5161 Box
6106 4400 Box
6122 5063 Box
6138 4773 Box
6153 5009 Box
6169 4954 Box
6185 4433 Box
6201 5190 Box
6216 5283 Box
6232 5132 Box
6248 5247 Box
6263 5181 Box
6279 4891 Box
6295 5065 Box
6311 4614 Box
6326 5386 Box
6342 4781 Box
6358 4870 Box
6373 5944 Box
6389 5461 Box
6405 5053 Box
6421 4591 Box
6436 5158 Box
6452 5247 Box
6468 5244 Box
6483 4600 Box
6499 4640 Box
6515 4600 Box
6531 5492 Box
6546 5668 Box
6562 5313 Box
6578 4981 Box
6593 4920 Box
6609 5137 Box
6625 5225 Box
6641 5121 Box
6656 5335 Box
6672 4759 Box
6688 4535 Box
6703 5308 Box
6719 5039 Box
6735 5054 Box
6751 4842 Box
6766 4784 Box
6782 4856 Box
6798 5515 Box
6813 4915 Box
6829 5200 Box
6845 5096 Box
6861 5338 Box
6876 5156 Box
6892 4998 Box
6908 4713 Box
6923 4677 Box
6939 4908 Box
6955 5269 Box
6971 5491 Box
6986 4963 Box
7002 5469 Box
7018 5220 Box
7034 4836 Box
7049 5139 Box
7065 5266 Box
7081 5329 Box
7096 5344 Box
7112 4786 Box
7128 5413 Box
7144 5179 Box
7159 4704 Box
7175 4694 Box
7191 5300 Box
7206 4569 Box
7222 5160 Box
7238 5330 Box
7254 5150 Box
7269 5410 Box
7285 5181 Box
7301 5089 Box
7316 5254 Box
7332 5585 Box
7348 4940 Box
7364 5165 Box
7379 4934 Box
7395 5053 Box
7411 5120 Box
7426 5058 Box
7442 5175 Box
7458 5232 Box
7474 4869 Box
7489 4746 Box
7505 5158 Box
7521 5532 Box
7536 5080 Box
7552 4901 Box
7568 5238 Box
7584 5278 Box
7599 5335 Box
7615 5132 Box
7631 5240 Box
7646 5002 Box
7662 4736 Box
7678 4924 Box
7694 4942 Box
7709 5148 Box
7725 5321 Box
7741 5727 Box
7756 5210 Box
7772 5027 Box
7788 4761 Box
7804 4655 Box
7819 5205 Box
7835 5200 Box
7851 5160 Box
7866 5340 Box
7882 4839 Box
7898 4990 Box
7914 5077 Box
7929 5060 Box
7945 4942 Box
7961 4690 Box
7976 4404 Box
7992 5039 Box
8008 5247 Box
8024 5125 Box
8039 5583 Box
8055 4828 Box
8071 4938 Box
8086 5267 Box
8102 5376 Box
8118 4717 Box
8134 4823 Box
8149 4858 Box
8165 5303 Box
8181 5062 Box
8196 4757 Box
8212 5278 Box
8228 5171 Box
8244 5136 Box
8259 5049 Box
8275 4557 Box
8291 5283 Box
8306 4818 Box
8322 5373 Box
8338 5130 Box
8354 4961 Box
8369 4960 Box
8385 5230 Box
8401 4782 Box
8417 4698 Box
8432 5257 Box
8448 4602 Box
8464 5466 Box
8479 5972 Box
8495 5339 Box
8511 5027 Box
8527 5784 Box
8542 5216 Box
8558 4673 Box
8574 4477 Box
8589 5064 Box
8605 5233 Box
8621 5689 Box
8637 5430 Box
8652 5443 Box
8668 5570 Box
8684 4691 Box
8699 4803 Box
8715 5526 Box
8731 4872 Box
8747 5091 Box
8762 5691 Box
8778 4973 Box
8794 4835 Box
8809 5598 Box
8825 5152 Box
8841 4486 Box
8857 5113 Box
8872 5736 Box
8888 5041 Box
8904 5048 Box
8919 4775 Box
8935 5039 Box
8951 5169 Box
8967 4895 Box
8982 4637 Box
8998 5250 Box
1.500 UP
1.000 UL
LTb
0.58 0.00 0.83 C 1156 5835 M
0 1 V
15 2 R
16 -5 R
16 -12 R
16 36 R
15 -12 R
16 0 R
16 -20 R
15 21 R
16 -19 R
16 -6 R
16 -9 R
15 13 R
16 9 R
16 -14 R
15 -25 R
16 30 R
16 5 R
16 6 R
0 1 V
15 -31 R
16 30 R
16 -6 R
15 -1 R
16 8 R
16 15 R
16 -23 R
15 15 R
16 -28 R
0 1 V
16 10 R
15 -13 R
16 -17 R
0 1 V
16 46 R
16 0 R
15 -20 R
16 13 R
16 8 R
15 -32 R
16 4 R
16 9 R
16 4 R
15 -14 R
16 23 R
0 1 V
16 8 R
0 1 V
16 -10 R
15 11 R
16 -46 R
16 39 R
15 -18 R
16 4 R
16 -10 R
16 2 R
15 3 R
16 -12 R
16 -1 R
15 -25 R
16 48 R
16 -27 R
16 3 R
15 48 R
16 -27 R
16 1 R
15 -20 R
0 1 V
16 -6 R
16 56 R
16 -29 R
15 -34 R
16 41 R
16 14 R
15 -54 R
16 0 R
0 1 V
16 16 R
16 52 R
15 -39 R
16 -9 R
16 5 R
15 6 R
16 -9 R
16 9 R
16 -15 R
15 44 R
16 -17 R
16 -30 R
15 24 R
16 25 R
16 -52 R
16 39 R
15 -31 R
16 -2 R
16 -4 R
15 1 R
16 15 R
16 -5 R
16 14 R
15 -16 R
16 25 R
0 1 V
16 -21 R
15 4 R
16 -34 R
16 37 R
16 -16 R
15 20 R
16 -14 R
16 1 R
15 -27 R
16 20 R
16 -1 R
16 23 R
15 -9 R
16 -10 R
16 -9 R
15 2 R
16 3 R
16 9 R
16 27 R
15 -23 R
16 -14 R
16 12 R
15 -8 R
16 21 R
16 -15 R
16 -10 R
15 14 R
16 41 R
16 -39 R
16 0 R
15 -14 R
16 -4 R
16 17 R
15 -6 R
16 3 R
16 -22 R
16 7 R
15 1 R
16 17 R
16 2 R
15 -3 R
16 28 R
16 -13 R
16 -27 R
15 18 R
16 -9 R
0 1 V
stroke 3356 5829 M
16 19 R
0 1 V
15 -21 R
0 1 V
16 11 R
16 -22 R
16 7 R
15 20 R
16 -42 R
16 25 R
15 6 R
16 -13 R
16 25 R
16 -34 R
15 -5 R
16 23 R
16 -6 R
0 1 V
15 6 R
16 -2 R
16 -15 R
16 -25 R
0 1 V
15 36 R
16 12 R
16 -18 R
15 -4 R
16 -4 R
16 -6 R
16 25 R
15 11 R
16 -28 R
16 9 R
15 1 R
16 4 R
16 -9 R
16 1 R
15 -5 R
16 7 R
16 17 R
0 1 V
15 -13 R
16 -4 R
16 15 R
16 0 R
15 1 R
0 1 V
16 2 R
16 -9 R
15 11 R
16 -6 R
16 -32 R
16 22 R
15 -17 R
16 8 R
16 -1 R
15 -6 R
16 -25 R
16 49 R
16 -7 R
15 29 R
16 -8 R
16 -54 R
15 17 R
16 -10 R
16 13 R
16 34 R
15 -27 R
16 -25 R
16 61 R
15 -50 R
16 15 R
16 -14 R
16 10 R
0 1 V
15 20 R
16 -21 R
0 1 V
16 -1 R
16 -6 R
15 11 R
16 -31 R
16 50 R
15 -13 R
16 28 R
0 1 V
16 -40 R
16 6 R
15 2 R
16 12 R
16 -30 R
15 16 R
16 11 R
16 -19 R
16 2 R
0 1 V
15 -6 R
16 -23 R
16 31 R
15 -14 R
0 1 V
16 27 R
16 -26 R
16 32 R
0 1 V
stroke 4818 5844 M
15 -30 R
16 -8 R
16 10 R
0 1 V
15 36 R
16 -27 R
16 -11 R
16 5 R
15 21 R
16 0 R
16 5 R
15 -20 R
16 17 R
16 -5 R
16 2 R
15 -17 R
16 -5 R
16 20 R
15 -2 R
16 -11 R
16 -11 R
16 9 R
15 15 R
16 -26 R
0 1 V
16 38 R
15 -39 R
16 -10 R
16 52 R
16 9 R
15 -25 R
16 13 R
16 -24 R
15 6 R
16 -39 R
16 32 R
16 8 R
15 -10 R
0 1 V
16 9 R
16 -37 R
15 19 R
16 12 R
16 27 R
16 -6 R
15 -22 R
16 26 R
16 -9 R
15 -6 R
16 7 R
16 -42 R
16 28 R
15 -12 R
16 -7 R
16 -4 R
15 32 R
16 -26 R
16 1 R
16 32 R
15 -3 R
16 -24 R
16 18 R
16 -5 R
15 0 R
16 5 R
16 -2 R
15 3 R
16 -34 R
16 23 R
16 -21 R
15 20 R
16 -5 R
16 13 R
15 -20 R
16 -5 R
16 11 R
16 -5 R
15 20 R
16 -21 R
16 0 R
15 26 R
16 -12 R
16 -13 R
0 1 V
16 46 R
15 -41 R
16 8 R
16 -11 R
15 -2 R
16 15 R
16 -18 R
16 9 R
15 -2 R
16 14 R
16 -4 R
0 1 V
15 -29 R
16 19 R
16 21 R
0 1 V
16 -16 R
15 -27 R
0 1 V
16 38 R
0 1 V
stroke 6342 5841 M
16 -17 R
15 10 R
16 -30 R
0 1 V
16 19 R
16 -9 R
15 -1 R
16 -5 R
16 24 R
15 9 R
16 -16 R
16 20 R
16 -17 R
15 12 R
16 -16 R
16 -6 R
15 8 R
16 36 R
16 -39 R
0 1 V
16 19 R
15 -7 R
16 2 R
16 -21 R
15 3 R
16 15 R
16 17 R
16 -49 R
15 28 R
16 -14 R
16 4 R
15 10 R
0 1 V
16 -9 R
16 10 R
16 3 R
15 15 R
16 -39 R
16 18 R
15 9 R
0 1 V
16 -8 R
16 20 R
16 -5 R
15 -53 R
16 41 R
0 1 V
16 -18 R
16 -20 R
15 15 R
16 38 R
16 -21 R
15 14 R
16 -45 R
0 1 V
16 55 R
16 -20 R
0 1 V
15 -21 R
0 1 V
16 0 R
16 -4 R
15 38 R
16 -10 R
16 5 R
16 -2 R
15 -4 R
16 -32 R
16 36 R
15 -11 R
16 -8 R
16 -11 R
16 19 R
15 5 R
16 -18 R
16 10 R
15 -15 R
16 29 R
16 -28 R
16 -11 R
15 29 R
16 6 R
16 -1 R
15 3 R
16 -15 R
16 23 R
16 -14 R
0 1 V
15 10 R
16 -4 R
16 -26 R
15 11 R
16 0 R
16 16 R
0 1 V
16 -2 R
15 13 R
16 -22 R
16 10 R
0 1 V
15 8 R
16 -31 R
16 26 R
16 -10 R
15 -31 R
16 25 R
16 15 R
15 -20 R
16 12 R
16 -9 R
16 -4 R
15 -2 R
16 -3 R
16 -3 R
15 -3 R
16 1 R
16 52 R
16 -31 R
15 -10 R
16 -24 R
16 3 R
15 23 R
16 -18 R
16 3 R
16 -19 R
15 22 R
16 12 R
16 -19 R
15 4 R
16 20 R
16 11 R
16 -34 R
15 20 R
0 1 V
stroke 8259 5832 M
16 -1 R
0 1 V
16 0 R
15 -16 R
16 -6 R
16 18 R
16 20 R
15 -31 R
16 5 R
16 -4 R
16 -11 R
15 22 R
16 -27 R
16 40 R
15 -17 R
16 -7 R
16 28 R
16 -31 R
15 25 R
16 -32 R
16 22 R
15 3 R
16 -12 R
16 -13 R
0 1 V
16 22 R
15 17 R
16 -48 R
0 1 V
16 6 R
15 35 R
16 -36 R
16 -7 R
16 40 R
15 -1 R
16 6 R
16 -13 R
15 -9 R
16 -17 R
16 13 R
16 2 R
0 1 V
15 5 R
16 -14 R
16 -13 R
15 24 R
16 -9 R
16 13 R
16 -3 R
15 -9 R
16 11 R
1156 5835 BoxF
1171 5838 BoxF
1187 5833 BoxF
1203 5821 BoxF
1219 5857 BoxF
1234 5845 BoxF
1250 5845 BoxF
1266 5825 BoxF
1281 5846 BoxF
1297 5827 BoxF
1313 5821 BoxF
1329 5812 BoxF
1344 5825 BoxF
1360 5834 BoxF
1376 5820 BoxF
1391 5795 BoxF
1407 5825 BoxF
1423 5830 BoxF
1439 5837 BoxF
1454 5806 BoxF
1470 5836 BoxF
1486 5830 BoxF
1501 5829 BoxF
1517 5837 BoxF
1533 5852 BoxF
1549 5829 BoxF
1564 5844 BoxF
1580 5816 BoxF
1596 5827 BoxF
1611 5814 BoxF
1627 5798 BoxF
1643 5844 BoxF
1659 5844 BoxF
1674 5824 BoxF
1690 5837 BoxF
1706 5845 BoxF
1721 5813 BoxF
1737 5817 BoxF
1753 5826 BoxF
1769 5830 BoxF
1784 5816 BoxF
1800 5839 BoxF
1816 5849 BoxF
1832 5839 BoxF
1847 5850 BoxF
1863 5804 BoxF
1879 5843 BoxF
1894 5825 BoxF
1910 5829 BoxF
1926 5819 BoxF
1942 5821 BoxF
1957 5824 BoxF
1973 5812 BoxF
1989 5811 BoxF
2004 5786 BoxF
2020 5834 BoxF
2036 5807 BoxF
2052 5810 BoxF
2067 5858 BoxF
2083 5831 BoxF
2099 5832 BoxF
2114 5812 BoxF
2130 5807 BoxF
2146 5863 BoxF
2162 5834 BoxF
2177 5800 BoxF
2193 5841 BoxF
2209 5855 BoxF
2224 5801 BoxF
2240 5802 BoxF
2256 5818 BoxF
2272 5870 BoxF
2287 5831 BoxF
2303 5822 BoxF
2319 5827 BoxF
2334 5833 BoxF
2350 5824 BoxF
2366 5833 BoxF
2382 5818 BoxF
2397 5862 BoxF
2413 5845 BoxF
2429 5815 BoxF
2444 5839 BoxF
2460 5864 BoxF
2476 5812 BoxF
2492 5851 BoxF
2507 5820 BoxF
2523 5818 BoxF
2539 5814 BoxF
2554 5815 BoxF
2570 5830 BoxF
2586 5825 BoxF
2602 5839 BoxF
2617 5823 BoxF
2633 5849 BoxF
2649 5828 BoxF
2664 5832 BoxF
2680 5798 BoxF
2696 5835 BoxF
2712 5819 BoxF
2727 5839 BoxF
2743 5825 BoxF
2759 5826 BoxF
2774 5799 BoxF
2790 5819 BoxF
2806 5818 BoxF
2822 5841 BoxF
2837 5832 BoxF
2853 5822 BoxF
2869 5813 BoxF
2884 5815 BoxF
2900 5818 BoxF
2916 5827 BoxF
2932 5854 BoxF
2947 5831 BoxF
2963 5817 BoxF
2979 5829 BoxF
2994 5821 BoxF
3010 5842 BoxF
3026 5827 BoxF
3042 5817 BoxF
3057 5831 BoxF
3073 5872 BoxF
3089 5833 BoxF
3105 5833 BoxF
3120 5819 BoxF
3136 5815 BoxF
3152 5832 BoxF
3167 5826 BoxF
3183 5829 BoxF
3199 5807 BoxF
3215 5814 BoxF
3230 5815 BoxF
3246 5832 BoxF
3262 5834 BoxF
3277 5831 BoxF
3293 5859 BoxF
3309 5846 BoxF
3325 5819 BoxF
3340 5837 BoxF
3356 5829 BoxF
3372 5849 BoxF
3387 5829 BoxF
3403 5840 BoxF
3419 5818 BoxF
3435 5825 BoxF
3450 5845 BoxF
3466 5803 BoxF
3482 5828 BoxF
3497 5834 BoxF
3513 5821 BoxF
3529 5846 BoxF
3545 5812 BoxF
3560 5807 BoxF
3576 5830 BoxF
3592 5825 BoxF
3607 5831 BoxF
3623 5829 BoxF
3639 5814 BoxF
3655 5790 BoxF
3670 5826 BoxF
3686 5838 BoxF
3702 5820 BoxF
3717 5816 BoxF
3733 5812 BoxF
3749 5806 BoxF
3765 5831 BoxF
3780 5842 BoxF
3796 5814 BoxF
3812 5823 BoxF
3827 5824 BoxF
3843 5828 BoxF
3859 5819 BoxF
3875 5820 BoxF
3890 5815 BoxF
3906 5822 BoxF
3922 5839 BoxF
3937 5827 BoxF
3953 5823 BoxF
3969 5838 BoxF
3985 5838 BoxF
4000 5839 BoxF
4016 5842 BoxF
4032 5833 BoxF
4047 5844 BoxF
4063 5838 BoxF
4079 5806 BoxF
4095 5828 BoxF
4110 5811 BoxF
4126 5819 BoxF
4142 5818 BoxF
4157 5812 BoxF
4173 5787 BoxF
4189 5836 BoxF
4205 5829 BoxF
4220 5858 BoxF
4236 5850 BoxF
4252 5796 BoxF
4267 5813 BoxF
4283 5803 BoxF
4299 5816 BoxF
4315 5850 BoxF
4330 5823 BoxF
4346 5798 BoxF
4362 5859 BoxF
4377 5809 BoxF
4393 5824 BoxF
4409 5810 BoxF
4425 5820 BoxF
4440 5841 BoxF
4456 5820 BoxF
4472 5820 BoxF
4488 5814 BoxF
4503 5825 BoxF
4519 5794 BoxF
4535 5844 BoxF
4550 5831 BoxF
4566 5860 BoxF
4582 5820 BoxF
4598 5826 BoxF
4613 5828 BoxF
4629 5840 BoxF
4645 5810 BoxF
4660 5826 BoxF
4676 5837 BoxF
4692 5818 BoxF
4708 5820 BoxF
4723 5815 BoxF
4739 5792 BoxF
4755 5823 BoxF
4770 5810 BoxF
4786 5837 BoxF
4802 5811 BoxF
4818 5843 BoxF
4833 5814 BoxF
4849 5806 BoxF
4865 5816 BoxF
4880 5853 BoxF
4896 5826 BoxF
4912 5815 BoxF
4928 5820 BoxF
4943 5841 BoxF
4959 5841 BoxF
4975 5846 BoxF
4990 5826 BoxF
5006 5843 BoxF
5022 5838 BoxF
5038 5840 BoxF
5053 5823 BoxF
5069 5818 BoxF
5085 5838 BoxF
5100 5836 BoxF
5116 5825 BoxF
5132 5814 BoxF
5148 5823 BoxF
5163 5838 BoxF
5179 5812 BoxF
5195 5851 BoxF
5210 5812 BoxF
5226 5802 BoxF
5242 5854 BoxF
5258 5863 BoxF
5273 5838 BoxF
5289 5851 BoxF
5305 5827 BoxF
5320 5833 BoxF
5336 5794 BoxF
5352 5826 BoxF
5368 5834 BoxF
5383 5825 BoxF
5399 5834 BoxF
5415 5797 BoxF
5430 5816 BoxF
5446 5828 BoxF
5462 5855 BoxF
5478 5849 BoxF
5493 5827 BoxF
5509 5853 BoxF
5525 5844 BoxF
5540 5838 BoxF
5556 5845 BoxF
5572 5803 BoxF
5588 5831 BoxF
5603 5819 BoxF
5619 5812 BoxF
5635 5808 BoxF
5650 5840 BoxF
5666 5814 BoxF
5682 5815 BoxF
5698 5847 BoxF
5713 5844 BoxF
5729 5820 BoxF
5745 5838 BoxF
5761 5833 BoxF
5776 5833 BoxF
5792 5838 BoxF
5808 5836 BoxF
5823 5839 BoxF
5839 5805 BoxF
5855 5828 BoxF
5871 5807 BoxF
5886 5827 BoxF
5902 5822 BoxF
5918 5835 BoxF
5933 5815 BoxF
5949 5810 BoxF
5965 5821 BoxF
5981 5816 BoxF
5996 5836 BoxF
6012 5815 BoxF
6028 5815 BoxF
6043 5841 BoxF
6059 5829 BoxF
6075 5816 BoxF
6091 5863 BoxF
6106 5822 BoxF
6122 5830 BoxF
6138 5819 BoxF
6153 5817 BoxF
6169 5832 BoxF
6185 5814 BoxF
6201 5823 BoxF
6216 5821 BoxF
6232 5835 BoxF
6248 5831 BoxF
6263 5803 BoxF
6279 5822 BoxF
6295 5843 BoxF
6311 5828 BoxF
6326 5802 BoxF
6342 5841 BoxF
6358 5824 BoxF
6373 5834 BoxF
6389 5804 BoxF
6405 5824 BoxF
6421 5815 BoxF
6436 5814 BoxF
6452 5809 BoxF
6468 5833 BoxF
6483 5842 BoxF
6499 5826 BoxF
6515 5846 BoxF
6531 5829 BoxF
6546 5841 BoxF
6562 5825 BoxF
6578 5819 BoxF
6593 5827 BoxF
6609 5863 BoxF
6625 5825 BoxF
6641 5844 BoxF
6656 5837 BoxF
6672 5839 BoxF
6688 5818 BoxF
6703 5821 BoxF
6719 5836 BoxF
6735 5853 BoxF
6751 5804 BoxF
6766 5832 BoxF
6782 5818 BoxF
6798 5822 BoxF
6813 5833 BoxF
6829 5824 BoxF
6845 5834 BoxF
6861 5837 BoxF
6876 5852 BoxF
6892 5813 BoxF
6908 5831 BoxF
6923 5841 BoxF
6939 5833 BoxF
6955 5853 BoxF
6971 5848 BoxF
6986 5795 BoxF
7002 5837 BoxF
7018 5819 BoxF
7034 5799 BoxF
7049 5814 BoxF
7065 5852 BoxF
7081 5831 BoxF
7096 5845 BoxF
7112 5800 BoxF
7128 5856 BoxF
7144 5837 BoxF
7159 5816 BoxF
7175 5817 BoxF
7191 5813 BoxF
7206 5851 BoxF
7222 5841 BoxF
7238 5846 BoxF
7254 5844 BoxF
7269 5840 BoxF
7285 5808 BoxF
7301 5844 BoxF
7316 5833 BoxF
7332 5825 BoxF
7348 5814 BoxF
7364 5833 BoxF
7379 5838 BoxF
7395 5820 BoxF
7411 5830 BoxF
7426 5815 BoxF
7442 5844 BoxF
7458 5816 BoxF
7474 5805 BoxF
7489 5834 BoxF
7505 5840 BoxF
7521 5839 BoxF
7536 5842 BoxF
7552 5827 BoxF
7568 5850 BoxF
7584 5837 BoxF
7599 5847 BoxF
7615 5843 BoxF
7631 5817 BoxF
7646 5828 BoxF
7662 5828 BoxF
7678 5845 BoxF
7694 5843 BoxF
7709 5856 BoxF
7725 5834 BoxF
7741 5845 BoxF
7756 5853 BoxF
7772 5822 BoxF
7788 5848 BoxF
7804 5838 BoxF
7819 5807 BoxF
7835 5832 BoxF
7851 5847 BoxF
7866 5827 BoxF
7882 5839 BoxF
7898 5830 BoxF
7914 5826 BoxF
7929 5824 BoxF
7945 5821 BoxF
7961 5818 BoxF
7976 5815 BoxF
7992 5816 BoxF
8008 5868 BoxF
8024 5837 BoxF
8039 5827 BoxF
8055 5803 BoxF
8071 5806 BoxF
8086 5829 BoxF
8102 5811 BoxF
8118 5814 BoxF
8134 5795 BoxF
8149 5817 BoxF
8165 5829 BoxF
8181 5810 BoxF
8196 5814 BoxF
8212 5834 BoxF
8228 5845 BoxF
8244 5811 BoxF
8259 5831 BoxF
8275 5831 BoxF
8291 5832 BoxF
8306 5816 BoxF
8322 5810 BoxF
8338 5828 BoxF
8354 5848 BoxF
8369 5817 BoxF
8385 5822 BoxF
8401 5818 BoxF
8417 5807 BoxF
8432 5829 BoxF
8448 5802 BoxF
8464 5842 BoxF
8479 5825 BoxF
8495 5818 BoxF
8511 5846 BoxF
8527 5815 BoxF
8542 5840 BoxF
8558 5808 BoxF
8574 5830 BoxF
8589 5833 BoxF
8605 5821 BoxF
8621 5808 BoxF
8637 5831 BoxF
8652 5848 BoxF
8668 5800 BoxF
8684 5807 BoxF
8699 5842 BoxF
8715 5806 BoxF
8731 5799 BoxF
8747 5839 BoxF
8762 5838 BoxF
8778 5844 BoxF
8794 5831 BoxF
8809 5822 BoxF
8825 5805 BoxF
8841 5818 BoxF
8857 5820 BoxF
8872 5826 BoxF
8888 5812 BoxF
8904 5799 BoxF
8919 5823 BoxF
8935 5814 BoxF
8951 5827 BoxF
8967 5824 BoxF
8982 5815 BoxF
8998 5826 BoxF
2.000 UL
LTb
LCb setrgbcolor
1.000 UL
LTb
LCb setrgbcolor
1140 7319 N
0 -6679 V
7858 0 V
0 6679 V
-7858 0 V
Z stroke
1.000 UP
1.000 UL
LTb
LCb setrgbcolor
stroke
grestore
end
showpage
  }}%
  \put(5069,140){\makebox(0,0){\large{sweeps}}}%
  \put(200,4979){\makebox(0,0){\Large{$Q_L$}}}%
  \put(8998,440){\makebox(0,0){\strut{}\ {$50000$}}}%
  \put(7426,440){\makebox(0,0){\strut{}\ {$40000$}}}%
  \put(5855,440){\makebox(0,0){\strut{}\ {$30000$}}}%
  \put(4283,440){\makebox(0,0){\strut{}\ {$20000$}}}%
  \put(2712,440){\makebox(0,0){\strut{}\ {$10000$}}}%
  \put(1140,440){\makebox(0,0){\strut{}\ {$0$}}}%
  \put(1020,7319){\makebox(0,0)[r]{\strut{}\ \ {$2.5$}}}%
  \put(1020,5983){\makebox(0,0)[r]{\strut{}\ \ {$2$}}}%
  \put(1020,4647){\makebox(0,0)[r]{\strut{}\ \ {$1.5$}}}%
  \put(1020,3312){\makebox(0,0)[r]{\strut{}\ \ {$1$}}}%
  \put(1020,1976){\makebox(0,0)[r]{\strut{}\ \ {$0.5$}}}%
  \put(1020,640){\makebox(0,0)[r]{\strut{}\ \ {$0$}}}%
\end{picture}%
\endgroup
\endinput

\end	{center}
\caption{The lattice topological charge $Q_L$ for two sequences of $SU(8)$ lattice fields,
  calculated after 2 ($\circ,\square$) and 20 ($\bullet,\blacksquare$) cooling sweeps.
  Calculations of $Q_L$ made every 100 Monte Carlo sweeps for each sequence of 50000 sweeps,
  at $\beta=47.75$ on a $20^330$ lattice.}
\label{fig_Qseq_su8}
\end{figure}


\begin{figure}[htb]
\begin	{center}
\leavevmode
% GNUPLOT: LaTeX picture with Postscript
\begingroup%
\makeatletter%
\newcommand{\GNUPLOTspecial}{%
  \@sanitize\catcode`\%=14\relax\special}%
\setlength{\unitlength}{0.0500bp}%
\begin{picture}(9360,7560)(0,0)%
  {\GNUPLOTspecial{"
%!PS-Adobe-2.0 EPSF-2.0
%%Title: plot_Qseq_su5b17.63b.tex
%%Creator: gnuplot 5.0 patchlevel 3
%%CreationDate: Fri Mar 26 13:16:27 2021
%%DocumentFonts: 
%%BoundingBox: 0 0 468 378
%%EndComments
%%BeginProlog
/gnudict 256 dict def
gnudict begin
%
% The following true/false flags may be edited by hand if desired.
% The unit line width and grayscale image gamma correction may also be changed.
%
/Color true def
/Blacktext true def
/Solid false def
/Dashlength 1 def
/Landscape false def
/Level1 false def
/Level3 false def
/Rounded false def
/ClipToBoundingBox false def
/SuppressPDFMark false def
/TransparentPatterns false def
/gnulinewidth 5.000 def
/userlinewidth gnulinewidth def
/Gamma 1.0 def
/BackgroundColor {-1.000 -1.000 -1.000} def
%
/vshift -66 def
/dl1 {
  10.0 Dashlength userlinewidth gnulinewidth div mul mul mul
  Rounded { currentlinewidth 0.75 mul sub dup 0 le { pop 0.01 } if } if
} def
/dl2 {
  10.0 Dashlength userlinewidth gnulinewidth div mul mul mul
  Rounded { currentlinewidth 0.75 mul add } if
} def
/hpt_ 31.5 def
/vpt_ 31.5 def
/hpt hpt_ def
/vpt vpt_ def
/doclip {
  ClipToBoundingBox {
    newpath 0 0 moveto 468 0 lineto 468 378 lineto 0 378 lineto closepath
    clip
  } if
} def
%
% Gnuplot Prolog Version 5.1 (Oct 2015)
%
%/SuppressPDFMark true def
%
/M {moveto} bind def
/L {lineto} bind def
/R {rmoveto} bind def
/V {rlineto} bind def
/N {newpath moveto} bind def
/Z {closepath} bind def
/C {setrgbcolor} bind def
/f {rlineto fill} bind def
/g {setgray} bind def
/Gshow {show} def   % May be redefined later in the file to support UTF-8
/vpt2 vpt 2 mul def
/hpt2 hpt 2 mul def
/Lshow {currentpoint stroke M 0 vshift R 
	Blacktext {gsave 0 setgray textshow grestore} {textshow} ifelse} def
/Rshow {currentpoint stroke M dup stringwidth pop neg vshift R
	Blacktext {gsave 0 setgray textshow grestore} {textshow} ifelse} def
/Cshow {currentpoint stroke M dup stringwidth pop -2 div vshift R 
	Blacktext {gsave 0 setgray textshow grestore} {textshow} ifelse} def
/UP {dup vpt_ mul /vpt exch def hpt_ mul /hpt exch def
  /hpt2 hpt 2 mul def /vpt2 vpt 2 mul def} def
/DL {Color {setrgbcolor Solid {pop []} if 0 setdash}
 {pop pop pop 0 setgray Solid {pop []} if 0 setdash} ifelse} def
/BL {stroke userlinewidth 2 mul setlinewidth
	Rounded {1 setlinejoin 1 setlinecap} if} def
/AL {stroke userlinewidth 2 div setlinewidth
	Rounded {1 setlinejoin 1 setlinecap} if} def
/UL {dup gnulinewidth mul /userlinewidth exch def
	dup 1 lt {pop 1} if 10 mul /udl exch def} def
/PL {stroke userlinewidth setlinewidth
	Rounded {1 setlinejoin 1 setlinecap} if} def
3.8 setmiterlimit
% Classic Line colors (version 5.0)
/LCw {1 1 1} def
/LCb {0 0 0} def
/LCa {0 0 0} def
/LC0 {1 0 0} def
/LC1 {0 1 0} def
/LC2 {0 0 1} def
/LC3 {1 0 1} def
/LC4 {0 1 1} def
/LC5 {1 1 0} def
/LC6 {0 0 0} def
/LC7 {1 0.3 0} def
/LC8 {0.5 0.5 0.5} def
% Default dash patterns (version 5.0)
/LTB {BL [] LCb DL} def
/LTw {PL [] 1 setgray} def
/LTb {PL [] LCb DL} def
/LTa {AL [1 udl mul 2 udl mul] 0 setdash LCa setrgbcolor} def
/LT0 {PL [] LC0 DL} def
/LT1 {PL [2 dl1 3 dl2] LC1 DL} def
/LT2 {PL [1 dl1 1.5 dl2] LC2 DL} def
/LT3 {PL [6 dl1 2 dl2 1 dl1 2 dl2] LC3 DL} def
/LT4 {PL [1 dl1 2 dl2 6 dl1 2 dl2 1 dl1 2 dl2] LC4 DL} def
/LT5 {PL [4 dl1 2 dl2] LC5 DL} def
/LT6 {PL [1.5 dl1 1.5 dl2 1.5 dl1 1.5 dl2 1.5 dl1 6 dl2] LC6 DL} def
/LT7 {PL [3 dl1 3 dl2 1 dl1 3 dl2] LC7 DL} def
/LT8 {PL [2 dl1 2 dl2 2 dl1 6 dl2] LC8 DL} def
/SL {[] 0 setdash} def
/Pnt {stroke [] 0 setdash gsave 1 setlinecap M 0 0 V stroke grestore} def
/Dia {stroke [] 0 setdash 2 copy vpt add M
  hpt neg vpt neg V hpt vpt neg V
  hpt vpt V hpt neg vpt V closepath stroke
  Pnt} def
/Pls {stroke [] 0 setdash vpt sub M 0 vpt2 V
  currentpoint stroke M
  hpt neg vpt neg R hpt2 0 V stroke
 } def
/Box {stroke [] 0 setdash 2 copy exch hpt sub exch vpt add M
  0 vpt2 neg V hpt2 0 V 0 vpt2 V
  hpt2 neg 0 V closepath stroke
  Pnt} def
/Crs {stroke [] 0 setdash exch hpt sub exch vpt add M
  hpt2 vpt2 neg V currentpoint stroke M
  hpt2 neg 0 R hpt2 vpt2 V stroke} def
/TriU {stroke [] 0 setdash 2 copy vpt 1.12 mul add M
  hpt neg vpt -1.62 mul V
  hpt 2 mul 0 V
  hpt neg vpt 1.62 mul V closepath stroke
  Pnt} def
/Star {2 copy Pls Crs} def
/BoxF {stroke [] 0 setdash exch hpt sub exch vpt add M
  0 vpt2 neg V hpt2 0 V 0 vpt2 V
  hpt2 neg 0 V closepath fill} def
/TriUF {stroke [] 0 setdash vpt 1.12 mul add M
  hpt neg vpt -1.62 mul V
  hpt 2 mul 0 V
  hpt neg vpt 1.62 mul V closepath fill} def
/TriD {stroke [] 0 setdash 2 copy vpt 1.12 mul sub M
  hpt neg vpt 1.62 mul V
  hpt 2 mul 0 V
  hpt neg vpt -1.62 mul V closepath stroke
  Pnt} def
/TriDF {stroke [] 0 setdash vpt 1.12 mul sub M
  hpt neg vpt 1.62 mul V
  hpt 2 mul 0 V
  hpt neg vpt -1.62 mul V closepath fill} def
/DiaF {stroke [] 0 setdash vpt add M
  hpt neg vpt neg V hpt vpt neg V
  hpt vpt V hpt neg vpt V closepath fill} def
/Pent {stroke [] 0 setdash 2 copy gsave
  translate 0 hpt M 4 {72 rotate 0 hpt L} repeat
  closepath stroke grestore Pnt} def
/PentF {stroke [] 0 setdash gsave
  translate 0 hpt M 4 {72 rotate 0 hpt L} repeat
  closepath fill grestore} def
/Circle {stroke [] 0 setdash 2 copy
  hpt 0 360 arc stroke Pnt} def
/CircleF {stroke [] 0 setdash hpt 0 360 arc fill} def
/C0 {BL [] 0 setdash 2 copy moveto vpt 90 450 arc} bind def
/C1 {BL [] 0 setdash 2 copy moveto
	2 copy vpt 0 90 arc closepath fill
	vpt 0 360 arc closepath} bind def
/C2 {BL [] 0 setdash 2 copy moveto
	2 copy vpt 90 180 arc closepath fill
	vpt 0 360 arc closepath} bind def
/C3 {BL [] 0 setdash 2 copy moveto
	2 copy vpt 0 180 arc closepath fill
	vpt 0 360 arc closepath} bind def
/C4 {BL [] 0 setdash 2 copy moveto
	2 copy vpt 180 270 arc closepath fill
	vpt 0 360 arc closepath} bind def
/C5 {BL [] 0 setdash 2 copy moveto
	2 copy vpt 0 90 arc
	2 copy moveto
	2 copy vpt 180 270 arc closepath fill
	vpt 0 360 arc} bind def
/C6 {BL [] 0 setdash 2 copy moveto
	2 copy vpt 90 270 arc closepath fill
	vpt 0 360 arc closepath} bind def
/C7 {BL [] 0 setdash 2 copy moveto
	2 copy vpt 0 270 arc closepath fill
	vpt 0 360 arc closepath} bind def
/C8 {BL [] 0 setdash 2 copy moveto
	2 copy vpt 270 360 arc closepath fill
	vpt 0 360 arc closepath} bind def
/C9 {BL [] 0 setdash 2 copy moveto
	2 copy vpt 270 450 arc closepath fill
	vpt 0 360 arc closepath} bind def
/C10 {BL [] 0 setdash 2 copy 2 copy moveto vpt 270 360 arc closepath fill
	2 copy moveto
	2 copy vpt 90 180 arc closepath fill
	vpt 0 360 arc closepath} bind def
/C11 {BL [] 0 setdash 2 copy moveto
	2 copy vpt 0 180 arc closepath fill
	2 copy moveto
	2 copy vpt 270 360 arc closepath fill
	vpt 0 360 arc closepath} bind def
/C12 {BL [] 0 setdash 2 copy moveto
	2 copy vpt 180 360 arc closepath fill
	vpt 0 360 arc closepath} bind def
/C13 {BL [] 0 setdash 2 copy moveto
	2 copy vpt 0 90 arc closepath fill
	2 copy moveto
	2 copy vpt 180 360 arc closepath fill
	vpt 0 360 arc closepath} bind def
/C14 {BL [] 0 setdash 2 copy moveto
	2 copy vpt 90 360 arc closepath fill
	vpt 0 360 arc} bind def
/C15 {BL [] 0 setdash 2 copy vpt 0 360 arc closepath fill
	vpt 0 360 arc closepath} bind def
/Rec {newpath 4 2 roll moveto 1 index 0 rlineto 0 exch rlineto
	neg 0 rlineto closepath} bind def
/Square {dup Rec} bind def
/Bsquare {vpt sub exch vpt sub exch vpt2 Square} bind def
/S0 {BL [] 0 setdash 2 copy moveto 0 vpt rlineto BL Bsquare} bind def
/S1 {BL [] 0 setdash 2 copy vpt Square fill Bsquare} bind def
/S2 {BL [] 0 setdash 2 copy exch vpt sub exch vpt Square fill Bsquare} bind def
/S3 {BL [] 0 setdash 2 copy exch vpt sub exch vpt2 vpt Rec fill Bsquare} bind def
/S4 {BL [] 0 setdash 2 copy exch vpt sub exch vpt sub vpt Square fill Bsquare} bind def
/S5 {BL [] 0 setdash 2 copy 2 copy vpt Square fill
	exch vpt sub exch vpt sub vpt Square fill Bsquare} bind def
/S6 {BL [] 0 setdash 2 copy exch vpt sub exch vpt sub vpt vpt2 Rec fill Bsquare} bind def
/S7 {BL [] 0 setdash 2 copy exch vpt sub exch vpt sub vpt vpt2 Rec fill
	2 copy vpt Square fill Bsquare} bind def
/S8 {BL [] 0 setdash 2 copy vpt sub vpt Square fill Bsquare} bind def
/S9 {BL [] 0 setdash 2 copy vpt sub vpt vpt2 Rec fill Bsquare} bind def
/S10 {BL [] 0 setdash 2 copy vpt sub vpt Square fill 2 copy exch vpt sub exch vpt Square fill
	Bsquare} bind def
/S11 {BL [] 0 setdash 2 copy vpt sub vpt Square fill 2 copy exch vpt sub exch vpt2 vpt Rec fill
	Bsquare} bind def
/S12 {BL [] 0 setdash 2 copy exch vpt sub exch vpt sub vpt2 vpt Rec fill Bsquare} bind def
/S13 {BL [] 0 setdash 2 copy exch vpt sub exch vpt sub vpt2 vpt Rec fill
	2 copy vpt Square fill Bsquare} bind def
/S14 {BL [] 0 setdash 2 copy exch vpt sub exch vpt sub vpt2 vpt Rec fill
	2 copy exch vpt sub exch vpt Square fill Bsquare} bind def
/S15 {BL [] 0 setdash 2 copy Bsquare fill Bsquare} bind def
/D0 {gsave translate 45 rotate 0 0 S0 stroke grestore} bind def
/D1 {gsave translate 45 rotate 0 0 S1 stroke grestore} bind def
/D2 {gsave translate 45 rotate 0 0 S2 stroke grestore} bind def
/D3 {gsave translate 45 rotate 0 0 S3 stroke grestore} bind def
/D4 {gsave translate 45 rotate 0 0 S4 stroke grestore} bind def
/D5 {gsave translate 45 rotate 0 0 S5 stroke grestore} bind def
/D6 {gsave translate 45 rotate 0 0 S6 stroke grestore} bind def
/D7 {gsave translate 45 rotate 0 0 S7 stroke grestore} bind def
/D8 {gsave translate 45 rotate 0 0 S8 stroke grestore} bind def
/D9 {gsave translate 45 rotate 0 0 S9 stroke grestore} bind def
/D10 {gsave translate 45 rotate 0 0 S10 stroke grestore} bind def
/D11 {gsave translate 45 rotate 0 0 S11 stroke grestore} bind def
/D12 {gsave translate 45 rotate 0 0 S12 stroke grestore} bind def
/D13 {gsave translate 45 rotate 0 0 S13 stroke grestore} bind def
/D14 {gsave translate 45 rotate 0 0 S14 stroke grestore} bind def
/D15 {gsave translate 45 rotate 0 0 S15 stroke grestore} bind def
/DiaE {stroke [] 0 setdash vpt add M
  hpt neg vpt neg V hpt vpt neg V
  hpt vpt V hpt neg vpt V closepath stroke} def
/BoxE {stroke [] 0 setdash exch hpt sub exch vpt add M
  0 vpt2 neg V hpt2 0 V 0 vpt2 V
  hpt2 neg 0 V closepath stroke} def
/TriUE {stroke [] 0 setdash vpt 1.12 mul add M
  hpt neg vpt -1.62 mul V
  hpt 2 mul 0 V
  hpt neg vpt 1.62 mul V closepath stroke} def
/TriDE {stroke [] 0 setdash vpt 1.12 mul sub M
  hpt neg vpt 1.62 mul V
  hpt 2 mul 0 V
  hpt neg vpt -1.62 mul V closepath stroke} def
/PentE {stroke [] 0 setdash gsave
  translate 0 hpt M 4 {72 rotate 0 hpt L} repeat
  closepath stroke grestore} def
/CircE {stroke [] 0 setdash 
  hpt 0 360 arc stroke} def
/Opaque {gsave closepath 1 setgray fill grestore 0 setgray closepath} def
/DiaW {stroke [] 0 setdash vpt add M
  hpt neg vpt neg V hpt vpt neg V
  hpt vpt V hpt neg vpt V Opaque stroke} def
/BoxW {stroke [] 0 setdash exch hpt sub exch vpt add M
  0 vpt2 neg V hpt2 0 V 0 vpt2 V
  hpt2 neg 0 V Opaque stroke} def
/TriUW {stroke [] 0 setdash vpt 1.12 mul add M
  hpt neg vpt -1.62 mul V
  hpt 2 mul 0 V
  hpt neg vpt 1.62 mul V Opaque stroke} def
/TriDW {stroke [] 0 setdash vpt 1.12 mul sub M
  hpt neg vpt 1.62 mul V
  hpt 2 mul 0 V
  hpt neg vpt -1.62 mul V Opaque stroke} def
/PentW {stroke [] 0 setdash gsave
  translate 0 hpt M 4 {72 rotate 0 hpt L} repeat
  Opaque stroke grestore} def
/CircW {stroke [] 0 setdash 
  hpt 0 360 arc Opaque stroke} def
/BoxFill {gsave Rec 1 setgray fill grestore} def
/Density {
  /Fillden exch def
  currentrgbcolor
  /ColB exch def /ColG exch def /ColR exch def
  /ColR ColR Fillden mul Fillden sub 1 add def
  /ColG ColG Fillden mul Fillden sub 1 add def
  /ColB ColB Fillden mul Fillden sub 1 add def
  ColR ColG ColB setrgbcolor} def
/BoxColFill {gsave Rec PolyFill} def
/PolyFill {gsave Density fill grestore grestore} def
/h {rlineto rlineto rlineto gsave closepath fill grestore} bind def
%
% PostScript Level 1 Pattern Fill routine for rectangles
% Usage: x y w h s a XX PatternFill
%	x,y = lower left corner of box to be filled
%	w,h = width and height of box
%	  a = angle in degrees between lines and x-axis
%	 XX = 0/1 for no/yes cross-hatch
%
/PatternFill {gsave /PFa [ 9 2 roll ] def
  PFa 0 get PFa 2 get 2 div add PFa 1 get PFa 3 get 2 div add translate
  PFa 2 get -2 div PFa 3 get -2 div PFa 2 get PFa 3 get Rec
  TransparentPatterns {} {gsave 1 setgray fill grestore} ifelse
  clip
  currentlinewidth 0.5 mul setlinewidth
  /PFs PFa 2 get dup mul PFa 3 get dup mul add sqrt def
  0 0 M PFa 5 get rotate PFs -2 div dup translate
  0 1 PFs PFa 4 get div 1 add floor cvi
	{PFa 4 get mul 0 M 0 PFs V} for
  0 PFa 6 get ne {
	0 1 PFs PFa 4 get div 1 add floor cvi
	{PFa 4 get mul 0 2 1 roll M PFs 0 V} for
 } if
  stroke grestore} def
%
/languagelevel where
 {pop languagelevel} {1} ifelse
dup 2 lt
	{/InterpretLevel1 true def
	 /InterpretLevel3 false def}
	{/InterpretLevel1 Level1 def
	 2 gt
	    {/InterpretLevel3 Level3 def}
	    {/InterpretLevel3 false def}
	 ifelse }
 ifelse
%
% PostScript level 2 pattern fill definitions
%
/Level2PatternFill {
/Tile8x8 {/PaintType 2 /PatternType 1 /TilingType 1 /BBox [0 0 8 8] /XStep 8 /YStep 8}
	bind def
/KeepColor {currentrgbcolor [/Pattern /DeviceRGB] setcolorspace} bind def
<< Tile8x8
 /PaintProc {0.5 setlinewidth pop 0 0 M 8 8 L 0 8 M 8 0 L stroke} 
>> matrix makepattern
/Pat1 exch def
<< Tile8x8
 /PaintProc {0.5 setlinewidth pop 0 0 M 8 8 L 0 8 M 8 0 L stroke
	0 4 M 4 8 L 8 4 L 4 0 L 0 4 L stroke}
>> matrix makepattern
/Pat2 exch def
<< Tile8x8
 /PaintProc {0.5 setlinewidth pop 0 0 M 0 8 L
	8 8 L 8 0 L 0 0 L fill}
>> matrix makepattern
/Pat3 exch def
<< Tile8x8
 /PaintProc {0.5 setlinewidth pop -4 8 M 8 -4 L
	0 12 M 12 0 L stroke}
>> matrix makepattern
/Pat4 exch def
<< Tile8x8
 /PaintProc {0.5 setlinewidth pop -4 0 M 8 12 L
	0 -4 M 12 8 L stroke}
>> matrix makepattern
/Pat5 exch def
<< Tile8x8
 /PaintProc {0.5 setlinewidth pop -2 8 M 4 -4 L
	0 12 M 8 -4 L 4 12 M 10 0 L stroke}
>> matrix makepattern
/Pat6 exch def
<< Tile8x8
 /PaintProc {0.5 setlinewidth pop -2 0 M 4 12 L
	0 -4 M 8 12 L 4 -4 M 10 8 L stroke}
>> matrix makepattern
/Pat7 exch def
<< Tile8x8
 /PaintProc {0.5 setlinewidth pop 8 -2 M -4 4 L
	12 0 M -4 8 L 12 4 M 0 10 L stroke}
>> matrix makepattern
/Pat8 exch def
<< Tile8x8
 /PaintProc {0.5 setlinewidth pop 0 -2 M 12 4 L
	-4 0 M 12 8 L -4 4 M 8 10 L stroke}
>> matrix makepattern
/Pat9 exch def
/Pattern1 {PatternBgnd KeepColor Pat1 setpattern} bind def
/Pattern2 {PatternBgnd KeepColor Pat2 setpattern} bind def
/Pattern3 {PatternBgnd KeepColor Pat3 setpattern} bind def
/Pattern4 {PatternBgnd KeepColor Landscape {Pat5} {Pat4} ifelse setpattern} bind def
/Pattern5 {PatternBgnd KeepColor Landscape {Pat4} {Pat5} ifelse setpattern} bind def
/Pattern6 {PatternBgnd KeepColor Landscape {Pat9} {Pat6} ifelse setpattern} bind def
/Pattern7 {PatternBgnd KeepColor Landscape {Pat8} {Pat7} ifelse setpattern} bind def
} def
%
%
%End of PostScript Level 2 code
%
/PatternBgnd {
  TransparentPatterns {} {gsave 1 setgray fill grestore} ifelse
} def
%
% Substitute for Level 2 pattern fill codes with
% grayscale if Level 2 support is not selected.
%
/Level1PatternFill {
/Pattern1 {0.250 Density} bind def
/Pattern2 {0.500 Density} bind def
/Pattern3 {0.750 Density} bind def
/Pattern4 {0.125 Density} bind def
/Pattern5 {0.375 Density} bind def
/Pattern6 {0.625 Density} bind def
/Pattern7 {0.875 Density} bind def
} def
%
% Now test for support of Level 2 code
%
Level1 {Level1PatternFill} {Level2PatternFill} ifelse
%
/Symbol-Oblique /Symbol findfont [1 0 .167 1 0 0] makefont
dup length dict begin {1 index /FID eq {pop pop} {def} ifelse} forall
currentdict end definefont pop
%
Level1 SuppressPDFMark or 
{} {
/SDict 10 dict def
systemdict /pdfmark known not {
  userdict /pdfmark systemdict /cleartomark get put
} if
SDict begin [
  /Title (plot_Qseq_su5b17.63b.tex)
  /Subject (gnuplot plot)
  /Creator (gnuplot 5.0 patchlevel 3)
  /Author (mteper)
%  /Producer (gnuplot)
%  /Keywords ()
  /CreationDate (Fri Mar 26 13:16:27 2021)
  /DOCINFO pdfmark
end
} ifelse
%
% Support for boxed text - Ethan A Merritt May 2005
%
/InitTextBox { userdict /TBy2 3 -1 roll put userdict /TBx2 3 -1 roll put
           userdict /TBy1 3 -1 roll put userdict /TBx1 3 -1 roll put
	   /Boxing true def } def
/ExtendTextBox { Boxing
    { gsave dup false charpath pathbbox
      dup TBy2 gt {userdict /TBy2 3 -1 roll put} {pop} ifelse
      dup TBx2 gt {userdict /TBx2 3 -1 roll put} {pop} ifelse
      dup TBy1 lt {userdict /TBy1 3 -1 roll put} {pop} ifelse
      dup TBx1 lt {userdict /TBx1 3 -1 roll put} {pop} ifelse
      grestore } if } def
/PopTextBox { newpath TBx1 TBxmargin sub TBy1 TBymargin sub M
               TBx1 TBxmargin sub TBy2 TBymargin add L
	       TBx2 TBxmargin add TBy2 TBymargin add L
	       TBx2 TBxmargin add TBy1 TBymargin sub L closepath } def
/DrawTextBox { PopTextBox stroke /Boxing false def} def
/FillTextBox { gsave PopTextBox 1 1 1 setrgbcolor fill grestore /Boxing false def} def
0 0 0 0 InitTextBox
/TBxmargin 20 def
/TBymargin 20 def
/Boxing false def
/textshow { ExtendTextBox Gshow } def
%
% redundant definitions for compatibility with prologue.ps older than 5.0.2
/LTB {BL [] LCb DL} def
/LTb {PL [] LCb DL} def
end
%%EndProlog
%%Page: 1 1
gnudict begin
gsave
doclip
0 0 translate
0.050 0.050 scale
0 setgray
newpath
BackgroundColor 0 lt 3 1 roll 0 lt exch 0 lt or or not {BackgroundColor C 1.000 0 0 9360.00 7560.00 BoxColFill} if
1.000 UL
LTb
LCb setrgbcolor
1020 640 M
63 0 V
7915 0 R
-63 0 V
stroke
LTb
LCb setrgbcolor
1020 1753 M
63 0 V
7915 0 R
-63 0 V
stroke
LTb
LCb setrgbcolor
1020 2866 M
63 0 V
7915 0 R
-63 0 V
stroke
LTb
LCb setrgbcolor
1020 3980 M
63 0 V
7915 0 R
-63 0 V
stroke
LTb
LCb setrgbcolor
1020 5093 M
63 0 V
7915 0 R
-63 0 V
stroke
LTb
LCb setrgbcolor
1020 6206 M
63 0 V
7915 0 R
-63 0 V
stroke
LTb
LCb setrgbcolor
1020 7319 M
63 0 V
7915 0 R
-63 0 V
stroke
LTb
LCb setrgbcolor
1020 640 M
0 63 V
0 6616 R
0 -63 V
stroke
LTb
LCb setrgbcolor
2616 640 M
0 63 V
0 6616 R
0 -63 V
stroke
LTb
LCb setrgbcolor
4211 640 M
0 63 V
0 6616 R
0 -63 V
stroke
LTb
LCb setrgbcolor
5807 640 M
0 63 V
0 6616 R
0 -63 V
stroke
LTb
LCb setrgbcolor
7402 640 M
0 63 V
0 6616 R
0 -63 V
stroke
LTb
LCb setrgbcolor
8998 640 M
0 63 V
0 6616 R
0 -63 V
stroke
LTb
LCb setrgbcolor
1.000 UL
LTb
LCb setrgbcolor
1020 7319 N
0 -6679 V
7978 0 V
0 6679 V
-7978 0 V
Z stroke
1.000 UP
1.000 UL
LTb
LCb setrgbcolor
LCb setrgbcolor
LTb
LCb setrgbcolor
LTb
1.500 UP
1.000 UL
LTb
0.58 0.00 0.83 C 1028 4759 M
16 323 R
16 -290 R
16 173 R
16 132 R
16 -128 R
16 -139 R
16 30 R
16 -49 R
16 188 R
16 -196 R
15 45 R
16 -109 R
16 31 R
16 -149 R
16 456 R
16 -214 R
16 121 R
16 -95 R
16 43 R
16 -58 R
16 -12 R
16 -12 R
16 -18 R
16 23 R
16 36 R
16 -105 R
16 171 R
16 -259 R
16 142 R
16 -68 R
16 72 R
16 64 R
16 -100 R
15 47 R
16 54 R
16 -113 R
16 160 R
16 -32 R
16 -124 R
16 100 R
16 4 R
16 -115 R
16 -195 R
16 417 R
16 -198 R
16 -109 R
16 -116 R
16 309 R
16 97 R
16 -112 R
16 836 R
16 74 R
16 36 R
16 -65 R
16 -149 R
16 122 R
15 -39 R
16 -133 R
16 145 R
16 91 R
16 -57 R
16 26 R
16 -72 R
16 -20 R
16 247 R
16 -135 R
16 115 R
16 799 R
16 -51 R
16 -144 R
16 223 R
16 -139 R
16 114 R
16 12 R
16 -130 R
16 -89 R
16 202 R
16 -93 R
16 -80 R
15 137 R
16 28 R
16 -4 R
16 7 R
16 -62 R
16 96 R
16 -117 R
16 -109 R
16 231 R
0 1 V
16 -28 R
16 -10 R
16 -181 R
0 1 V
16 26 R
16 96 R
16 -162 R
16 63 R
16 191 R
16 -258 R
16 138 R
16 184 R
0 1 V
16 -287 R
16 -42 R
15 169 R
16 -34 R
16 214 R
16 -24 R
16 -225 R
16 -29 R
16 268 R
16 26 R
16 -180 R
16 -58 R
16 -32 R
0 1 V
stroke 2815 6564 M
16 138 R
16 -8 R
16 -111 R
16 237 R
0 1 V
16 -310 R
16 177 R
16 -14 R
16 -195 R
16 116 R
16 215 R
16 -131 R
0 1 V
16 110 R
15 -97 R
16 159 R
16 -378 R
16 154 R
16 -129 R
16 190 R
16 -76 R
16 -21 R
16 173 R
16 -139 R
16 33 R
16 -18 R
16 11 R
16 -1055 R
16 146 R
16 264 R
16 -259 R
16 -186 R
16 290 R
0 1 V
16 -86 R
16 127 R
16 -223 R
16 29 R
15 0 R
16 117 R
16 128 R
16 -39 R
16 -262 R
16 23 R
16 100 R
16 -73 R
16 -798 R
16 687 R
16 216 R
16 -223 R
16 -741 R
16 30 R
16 763 R
16 159 R
16 1 R
16 -264 R
16 386 R
16 -126 R
16 -19 R
16 -128 R
15 221 R
16 -16 R
16 -907 R
16 32 R
16 -160 R
16 -170 R
16 118 R
16 82 R
16 94 R
16 -68 R
16 -17 R
16 -107 R
16 95 R
16 -153 R
16 145 R
16 65 R
16 -60 R
16 -175 R
16 315 R
16 -52 R
16 -195 R
16 372 R
16 -184 R
15 -109 R
16 -73 R
16 119 R
16 61 R
16 66 R
16 -128 R
16 142 R
16 3 R
16 -137 R
16 -120 R
16 240 R
16 -254 R
16 -63 R
16 100 R
16 874 R
16 -4 R
16 236 R
16 -132 R
16 -168 R
16 -1 R
16 -1 R
16 238 R
16 -71 R
15 32 R
16 105 R
16 50 R
16 -181 R
16 9 R
16 -22 R
0 1 V
stroke 4554 5724 M
16 42 R
16 -40 R
0 1 V
16 129 R
16 -83 R
16 -1045 R
16 71 R
16 54 R
16 114 R
16 -42 R
16 -12 R
16 -6 R
16 -179 R
16 162 R
16 133 R
16 -219 R
16 115 R
16 -158 R
15 146 R
16 53 R
16 -78 R
16 -90 R
16 127 R
16 -209 R
16 254 R
16 -939 R
16 -909 R
16 20 R
0 -1 V
16 -86 R
16 -29 R
16 52 R
16 281 R
16 -299 R
16 54 R
16 117 R
16 -126 R
16 90 R
16 -203 R
16 65 R
16 -17 R
15 7 R
16 51 R
16 -188 R
16 96 R
16 101 R
16 136 R
16 -183 R
16 166 R
16 -992 R
16 -112 R
16 327 R
16 -280 R
16 241 R
16 -271 R
16 -882 R
16 179 R
16 -38 R
16 211 R
16 -417 R
16 275 R
0 -1 V
16 -138 R
0 -1 V
16 49 R
16 -116 R
15 122 R
16 -171 R
16 113 R
16 1037 R
16 -120 R
16 -176 R
16 105 R
16 9 R
16 -31 R
16 49 R
16 -87 R
16 61 R
16 39 R
16 101 R
16 114 R
16 -250 R
16 107 R
16 17 R
16 -65 R
16 -80 R
16 83 R
16 264 R
16 -302 R
15 89 R
16 -46 R
16 26 R
16 -61 R
16 -50 R
16 205 R
16 -66 R
16 -39 R
16 73 R
16 51 R
16 -246 R
16 938 R
16 164 R
16 24 R
16 626 R
16 312 R
16 -165 R
16 -85 R
16 67 R
16 36 R
16 95 R
16 -297 R
16 159 R
15 -52 R
16 29 R
16 142 R
16 -183 R
16 119 R
16 -114 R
16 -105 R
16 252 R
16 792 R
16 62 R
16 60 R
16 -29 R
16 29 R
16 -145 R
16 22 R
16 -714 R
16 -95 R
16 -36 R
16 -153 R
16 -55 R
16 235 R
16 -70 R
15 -35 R
16 -17 R
16 -42 R
16 38 R
16 -8 R
16 95 R
16 -8 R
16 166 R
16 -66 R
16 -67 R
16 -7 R
16 -121 R
16 -29 R
16 162 R
16 -15 R
16 36 R
16 10 R
16 -312 R
16 44 R
16 149 R
16 64 R
16 -54 R
16 -48 R
15 -141 R
16 290 R
16 -93 R
16 -110 R
16 33 R
16 172 R
16 -55 R
16 707 R
16 165 R
16 62 R
16 -261 R
0 1 V
stroke 7171 4699 M
16 255 R
16 31 R
16 -123 R
16 19 R
16 -121 R
16 22 R
16 186 R
16 -68 R
16 -117 R
16 106 R
16 -237 R
16 223 R
15 -140 R
16 151 R
16 -203 R
16 99 R
16 64 R
16 -42 R
16 -867 R
16 -29 R
16 82 R
16 -71 R
16 -48 R
16 88 R
16 -22 R
16 -90 R
16 30 R
16 81 R
16 82 R
16 -222 R
16 1 R
16 -616 R
16 -127 R
16 -158 R
15 214 R
16 866 R
16 -152 R
16 43 R
16 -34 R
16 -20 R
16 56 R
16 58 R
16 -699 R
16 -71 R
16 -229 R
16 -62 R
16 125 R
16 100 R
16 -178 R
16 948 R
16 252 R
16 -389 R
16 129 R
16 197 R
16 -159 R
16 208 R
16 670 R
15 1 R
16 -43 R
16 5 R
16 153 R
16 307 R
16 -310 R
16 -132 R
16 -35 R
16 62 R
16 50 R
16 63 R
16 -177 R
16 231 R
16 -90 R
16 -64 R
16 309 R
16 -352 R
16 109 R
16 65 R
16 -119 R
16 -23 R
16 81 R
16 18 R
15 -136 R
16 36 R
16 49 R
16 89 R
16 -18 R
16 124 R
16 -193 R
16 929 R
16 133 R
16 -115 R
16 4 R
16 164 R
16 -319 R
16 343 R
16 -114 R
16 -30 R
16 9 R
16 55 R
16 -301 R
0 1 V
16 224 R
16 -57 R
16 -17 R
16 -51 R
15 12 R
16 113 R
16 -169 R
16 43 R
16 116 R
16 -84 R
16 130 R
16 -51 R
16 -253 R
16 55 R
16 14 R
1028 4759 Circle
1044 5082 Circle
1060 4792 Circle
1076 4965 Circle
1092 5097 Circle
1108 4969 Circle
1124 4830 Circle
1140 4860 Circle
1156 4811 Circle
1172 4999 Circle
1188 4803 Circle
1203 4848 Circle
1219 4739 Circle
1235 4770 Circle
1251 4621 Circle
1267 5077 Circle
1283 4863 Circle
1299 4984 Circle
1315 4889 Circle
1331 4932 Circle
1347 4874 Circle
1363 4862 Circle
1379 4850 Circle
1395 4832 Circle
1411 4855 Circle
1427 4891 Circle
1443 4786 Circle
1459 4957 Circle
1475 4698 Circle
1491 4840 Circle
1507 4772 Circle
1523 4844 Circle
1539 4908 Circle
1555 4808 Circle
1570 4855 Circle
1586 4909 Circle
1602 4796 Circle
1618 4956 Circle
1634 4924 Circle
1650 4800 Circle
1666 4900 Circle
1682 4904 Circle
1698 4789 Circle
1714 4594 Circle
1730 5011 Circle
1746 4813 Circle
1762 4704 Circle
1778 4588 Circle
1794 4897 Circle
1810 4994 Circle
1826 4882 Circle
1842 5718 Circle
1858 5792 Circle
1874 5828 Circle
1890 5763 Circle
1906 5614 Circle
1922 5736 Circle
1937 5697 Circle
1953 5564 Circle
1969 5709 Circle
1985 5800 Circle
2001 5743 Circle
2017 5769 Circle
2033 5697 Circle
2049 5677 Circle
2065 5924 Circle
2081 5789 Circle
2097 5904 Circle
2113 6703 Circle
2129 6652 Circle
2145 6508 Circle
2161 6731 Circle
2177 6592 Circle
2193 6706 Circle
2209 6718 Circle
2225 6588 Circle
2241 6499 Circle
2257 6701 Circle
2273 6608 Circle
2289 6528 Circle
2304 6665 Circle
2320 6693 Circle
2336 6689 Circle
2352 6696 Circle
2368 6634 Circle
2384 6730 Circle
2400 6613 Circle
2416 6504 Circle
2432 6735 Circle
2448 6708 Circle
2464 6698 Circle
2480 6518 Circle
2496 6544 Circle
2512 6640 Circle
2528 6478 Circle
2544 6541 Circle
2560 6732 Circle
2576 6474 Circle
2592 6612 Circle
2608 6796 Circle
2624 6510 Circle
2640 6468 Circle
2655 6637 Circle
2671 6603 Circle
2687 6817 Circle
2703 6793 Circle
2719 6568 Circle
2735 6539 Circle
2751 6807 Circle
2767 6833 Circle
2783 6653 Circle
2799 6595 Circle
2815 6563 Circle
2831 6702 Circle
2847 6694 Circle
2863 6583 Circle
2879 6821 Circle
2895 6511 Circle
2911 6688 Circle
2927 6674 Circle
2943 6479 Circle
2959 6595 Circle
2975 6810 Circle
2991 6679 Circle
3007 6790 Circle
3022 6693 Circle
3038 6852 Circle
3054 6474 Circle
3070 6628 Circle
3086 6499 Circle
3102 6689 Circle
3118 6613 Circle
3134 6592 Circle
3150 6765 Circle
3166 6626 Circle
3182 6659 Circle
3198 6641 Circle
3214 6652 Circle
3230 5597 Circle
3246 5743 Circle
3262 6007 Circle
3278 5748 Circle
3294 5562 Circle
3310 5853 Circle
3326 5767 Circle
3342 5894 Circle
3358 5671 Circle
3374 5700 Circle
3389 5700 Circle
3405 5817 Circle
3421 5945 Circle
3437 5906 Circle
3453 5644 Circle
3469 5667 Circle
3485 5767 Circle
3501 5694 Circle
3517 4896 Circle
3533 5583 Circle
3549 5799 Circle
3565 5576 Circle
3581 4835 Circle
3597 4865 Circle
3613 5628 Circle
3629 5787 Circle
3645 5788 Circle
3661 5524 Circle
3677 5910 Circle
3693 5784 Circle
3709 5765 Circle
3725 5637 Circle
3740 5858 Circle
3756 5842 Circle
3772 4935 Circle
3788 4967 Circle
3804 4807 Circle
3820 4637 Circle
3836 4755 Circle
3852 4837 Circle
3868 4931 Circle
3884 4863 Circle
3900 4846 Circle
3916 4739 Circle
3932 4834 Circle
3948 4681 Circle
3964 4826 Circle
3980 4891 Circle
3996 4831 Circle
4012 4656 Circle
4028 4971 Circle
4044 4919 Circle
4060 4724 Circle
4076 5096 Circle
4092 4912 Circle
4107 4803 Circle
4123 4730 Circle
4139 4849 Circle
4155 4910 Circle
4171 4976 Circle
4187 4848 Circle
4203 4990 Circle
4219 4993 Circle
4235 4856 Circle
4251 4736 Circle
4267 4976 Circle
4283 4722 Circle
4299 4659 Circle
4315 4759 Circle
4331 5633 Circle
4347 5629 Circle
4363 5865 Circle
4379 5733 Circle
4395 5565 Circle
4411 5564 Circle
4427 5563 Circle
4443 5801 Circle
4459 5730 Circle
4474 5762 Circle
4490 5867 Circle
4506 5917 Circle
4522 5736 Circle
4538 5745 Circle
4554 5723 Circle
4570 5766 Circle
4586 5727 Circle
4602 5856 Circle
4618 5773 Circle
4634 4728 Circle
4650 4799 Circle
4666 4853 Circle
4682 4967 Circle
4698 4925 Circle
4714 4913 Circle
4730 4907 Circle
4746 4728 Circle
4762 4890 Circle
4778 5023 Circle
4794 4804 Circle
4810 4919 Circle
4826 4761 Circle
4841 4907 Circle
4857 4960 Circle
4873 4882 Circle
4889 4792 Circle
4905 4919 Circle
4921 4710 Circle
4937 4964 Circle
4953 4025 Circle
4969 3116 Circle
4985 3135 Circle
5001 3049 Circle
5017 3020 Circle
5033 3072 Circle
5049 3353 Circle
5065 3054 Circle
5081 3108 Circle
5097 3225 Circle
5113 3099 Circle
5129 3189 Circle
5145 2986 Circle
5161 3051 Circle
5177 3034 Circle
5192 3041 Circle
5208 3092 Circle
5224 2904 Circle
5240 3000 Circle
5256 3101 Circle
5272 3237 Circle
5288 3054 Circle
5304 3220 Circle
5320 2228 Circle
5336 2116 Circle
5352 2443 Circle
5368 2163 Circle
5384 2404 Circle
5400 2133 Circle
5416 1251 Circle
5432 1430 Circle
5448 1392 Circle
5464 1603 Circle
5480 1186 Circle
5496 1461 Circle
5512 1321 Circle
5528 1370 Circle
5544 1254 Circle
5559 1376 Circle
5575 1205 Circle
5591 1318 Circle
5607 2355 Circle
5623 2235 Circle
5639 2059 Circle
5655 2164 Circle
5671 2173 Circle
5687 2142 Circle
5703 2191 Circle
5719 2104 Circle
5735 2165 Circle
5751 2204 Circle
5767 2305 Circle
5783 2419 Circle
5799 2169 Circle
5815 2276 Circle
5831 2293 Circle
5847 2228 Circle
5863 2148 Circle
5879 2231 Circle
5895 2495 Circle
5911 2193 Circle
5926 2282 Circle
5942 2236 Circle
5958 2262 Circle
5974 2201 Circle
5990 2151 Circle
6006 2356 Circle
6022 2290 Circle
6038 2251 Circle
6054 2324 Circle
6070 2375 Circle
6086 2129 Circle
6102 3067 Circle
6118 3231 Circle
6134 3255 Circle
6150 3881 Circle
6166 4193 Circle
6182 4028 Circle
6198 3943 Circle
6214 4010 Circle
6230 4046 Circle
6246 4141 Circle
6262 3844 Circle
6278 4003 Circle
6293 3951 Circle
6309 3980 Circle
6325 4122 Circle
6341 3939 Circle
6357 4058 Circle
6373 3944 Circle
6389 3839 Circle
6405 4091 Circle
6421 4883 Circle
6437 4945 Circle
6453 5005 Circle
6469 4976 Circle
6485 5005 Circle
6501 4860 Circle
6517 4882 Circle
6533 4168 Circle
6549 4073 Circle
6565 4037 Circle
6581 3884 Circle
6597 3829 Circle
6613 4064 Circle
6629 3994 Circle
6644 3959 Circle
6660 3942 Circle
6676 3900 Circle
6692 3938 Circle
6708 3930 Circle
6724 4025 Circle
6740 4017 Circle
6756 4183 Circle
6772 4117 Circle
6788 4050 Circle
6804 4043 Circle
6820 3922 Circle
6836 3893 Circle
6852 4055 Circle
6868 4040 Circle
6884 4076 Circle
6900 4086 Circle
6916 3774 Circle
6932 3818 Circle
6948 3967 Circle
6964 4031 Circle
6980 3977 Circle
6996 3929 Circle
7011 3788 Circle
7027 4078 Circle
7043 3985 Circle
7059 3875 Circle
7075 3908 Circle
7091 4080 Circle
7107 4025 Circle
7123 4732 Circle
7139 4897 Circle
7155 4959 Circle
7171 4698 Circle
7187 4954 Circle
7203 4985 Circle
7219 4862 Circle
7235 4881 Circle
7251 4760 Circle
7267 4782 Circle
7283 4968 Circle
7299 4900 Circle
7315 4783 Circle
7331 4889 Circle
7347 4652 Circle
7363 4875 Circle
7378 4735 Circle
7394 4886 Circle
7410 4683 Circle
7426 4782 Circle
7442 4846 Circle
7458 4804 Circle
7474 3937 Circle
7490 3908 Circle
7506 3990 Circle
7522 3919 Circle
7538 3871 Circle
7554 3959 Circle
7570 3937 Circle
7586 3847 Circle
7602 3877 Circle
7618 3958 Circle
7634 4040 Circle
7650 3818 Circle
7666 3819 Circle
7682 3203 Circle
7698 3076 Circle
7714 2918 Circle
7729 3132 Circle
7745 3998 Circle
7761 3846 Circle
7777 3889 Circle
7793 3855 Circle
7809 3835 Circle
7825 3891 Circle
7841 3949 Circle
7857 3250 Circle
7873 3179 Circle
7889 2950 Circle
7905 2888 Circle
7921 3013 Circle
7937 3113 Circle
7953 2935 Circle
7969 3883 Circle
7985 4135 Circle
8001 3746 Circle
8017 3875 Circle
8033 4072 Circle
8049 3913 Circle
8065 4121 Circle
8081 4791 Circle
8096 4792 Circle
8112 4749 Circle
8128 4754 Circle
8144 4907 Circle
8160 5214 Circle
8176 4904 Circle
8192 4772 Circle
8208 4737 Circle
8224 4799 Circle
8240 4849 Circle
8256 4912 Circle
8272 4735 Circle
8288 4966 Circle
8304 4876 Circle
8320 4812 Circle
8336 5121 Circle
8352 4769 Circle
8368 4878 Circle
8384 4943 Circle
8400 4824 Circle
8416 4801 Circle
8432 4882 Circle
8448 4900 Circle
8463 4764 Circle
8479 4800 Circle
8495 4849 Circle
8511 4938 Circle
8527 4920 Circle
8543 5044 Circle
8559 4851 Circle
8575 5780 Circle
8591 5913 Circle
8607 5798 Circle
8623 5802 Circle
8639 5966 Circle
8655 5647 Circle
8671 5990 Circle
8687 5876 Circle
8703 5846 Circle
8719 5855 Circle
8735 5910 Circle
8751 5610 Circle
8767 5834 Circle
8783 5777 Circle
8799 5760 Circle
8815 5709 Circle
8830 5721 Circle
8846 5834 Circle
8862 5665 Circle
8878 5708 Circle
8894 5824 Circle
8910 5740 Circle
8926 5870 Circle
8942 5819 Circle
8958 5566 Circle
8974 5621 Circle
8990 5635 Circle
1.500 UP
1.000 UL
LTb
0.58 0.00 0.83 C 1028 5041 M
16 -17 R
16 30 R
16 -3 R
16 -13 R
16 17 R
16 3 R
0 1 V
16 -31 R
16 15 R
16 9 R
16 -10 R
15 -11 R
16 -7 R
16 -32 R
16 35 R
16 22 R
16 8 R
16 -30 R
16 24 R
16 -11 R
16 14 R
16 -15 R
16 -37 R
0 1 V
16 39 R
16 -10 R
16 -5 R
16 -18 R
16 32 R
16 13 R
16 -11 R
16 -9 R
16 11 R
16 3 R
16 -8 R
15 -7 R
16 15 R
16 5 R
16 -19 R
16 21 R
16 -1 R
16 -10 R
16 -4 R
16 -5 R
16 -68 R
16 71 R
16 12 R
16 11 R
16 -33 R
16 29 R
16 -7 R
16 -29 R
16 1080 R
16 -22 R
16 30 R
16 -34 R
16 9 R
16 16 R
15 7 R
16 2 R
16 11 R
16 -10 R
16 -54 R
16 72 R
16 -43 R
16 22 R
16 41 R
16 -55 R
16 82 R
16 1023 R
16 -7 R
16 -24 R
16 25 R
16 -15 R
16 16 R
16 -40 R
16 -18 R
16 35 R
16 22 R
16 -55 R
16 49 R
15 -43 R
16 9 R
16 18 R
16 -6 R
0 1 V
16 19 R
0 1 V
16 -28 R
16 -29 R
16 27 R
16 -15 R
16 5 R
16 16 R
16 12 R
16 7 R
0 1 V
16 -22 R
16 -15 R
16 2 R
16 38 R
16 -24 R
16 4 R
16 -14 R
16 -37 R
16 86 R
15 -38 R
16 -21 R
16 16 R
16 -2 R
16 11 R
16 -6 R
0 1 V
stroke 2735 7160 M
16 12 R
16 -1 R
16 -26 R
16 32 R
16 -7 R
16 -11 R
16 14 R
0 1 V
16 -30 R
16 15 R
16 -31 R
16 7 R
16 61 R
16 -34 R
16 27 R
16 -21 R
16 -5 R
16 0 R
15 -3 R
16 9 R
16 -8 R
0 1 V
16 23 R
16 -36 R
16 33 R
16 -38 R
16 29 R
16 -7 R
16 -2 R
16 -42 R
16 56 R
16 -9 R
16 -1043 R
16 -21 R
16 -1 R
16 -15 R
0 1 V
16 8 R
16 43 R
16 -33 R
16 -12 R
0 1 V
16 -24 R
16 14 R
15 2 R
16 23 R
16 -3 R
16 19 R
16 -31 R
16 1 R
16 8 R
16 -2 R
16 -1039 R
16 1032 R
16 -9 R
0 1 V
16 -78 R
16 -999 R
16 30 R
16 1058 R
16 12 R
16 11 R
16 -25 R
16 3 R
16 -12 R
16 29 R
16 -30 R
0 1 V
15 13 R
16 14 R
16 -1052 R
16 -27 R
16 36 R
16 -40 R
16 -14 R
16 34 R
16 -11 R
16 -5 R
16 15 R
16 -17 R
16 13 R
16 -22 R
16 11 R
16 4 R
16 7 R
16 4 R
16 -38 R
16 32 R
16 9 R
16 -2 R
16 -16 R
15 4 R
16 -21 R
16 18 R
16 -16 R
16 5 R
16 44 R
16 -9 R
16 -25 R
16 15 R
16 -36 R
16 34 R
16 -12 R
16 18 R
16 -11 R
16 1040 R
16 31 R
16 31 R
16 -30 R
16 -6 R
16 -13 R
16 -2 R
16 3 R
16 12 R
15 5 R
16 -79 R
16 90 R
16 -17 R
16 -15 R
16 -5 R
16 -11 R
16 15 R
16 19 R
16 -10 R
16 -1089 R
16 13 R
16 23 R
16 29 R
16 -43 R
16 13 R
16 18 R
0 1 V
stroke 4730 5068 M
16 -25 R
16 16 R
16 9 R
16 -29 R
16 9 R
16 -2 R
15 -13 R
16 17 R
16 -10 R
16 -20 R
16 6 R
16 11 R
16 18 R
16 -1075 R
16 -1085 R
16 37 R
16 -7 R
16 4 R
16 -11 R
16 6 R
16 -13 R
16 10 R
16 -26 R
16 10 R
16 21 R
16 -11 R
16 15 R
16 -20 R
15 14 R
16 -19 R
16 14 R
16 -3 R
16 -4 R
16 27 R
16 -33 R
16 26 R
16 -1106 R
16 16 R
16 1 R
16 3 R
16 1 R
16 1 R
5416 814 M
16 -19 R
16 -4 R
16 0 R
16 4 R
16 -3 R
16 -15 R
16 64 R
16 -36 R
15 -4 R
16 -13 R
16 23 R
16 1044 R
16 -10 R
16 15 R
16 8 R
16 -5 R
16 -10 R
16 20 R
0 -1 V
16 -16 R
16 -4 R
16 -27 R
16 -4 R
0 -1 V
16 85 R
16 -43 R
16 -12 R
16 5 R
16 22 R
16 -59 R
16 25 R
16 55 R
16 -54 R
15 18 R
16 -7 R
16 0 R
16 -36 R
16 37 R
16 -1 R
16 2 R
16 1 R
16 -2 R
16 -24 R
16 25 R
16 1046 R
16 12 R
16 6 R
16 1044 R
16 48 R
16 -17 R
16 -31 R
16 33 R
16 -28 R
16 26 R
16 -9 R
16 -56 R
15 36 R
16 10 R
16 4 R
16 -1 R
16 8 R
16 -7 R
16 -25 R
16 56 R
16 1034 R
16 -15 R
16 26 R
16 -10 R
16 -3 R
16 -21 R
16 21 R
16 -1076 R
16 19 R
16 -6 R
16 2 R
16 -6 R
16 9 R
16 -7 R
15 -13 R
16 15 R
16 3 R
16 -47 R
16 26 R
16 13 R
16 -3 R
16 15 R
16 34 R
16 -67 R
16 3 R
16 19 R
16 33 R
16 -33 R
16 4 R
16 2 R
16 4 R
16 -22 R
16 19 R
16 -11 R
16 15 R
16 13 R
16 -39 R
15 42 R
16 -16 R
16 25 R
16 -42 R
16 16 R
16 -13 R
16 -29 R
16 1077 R
16 38 R
16 -26 R
16 30 R
16 -22 R
16 -20 R
16 25 R
16 -18 R
16 15 R
16 2 R
16 -4 R
16 -15 R
16 30 R
16 -13 R
16 -34 R
16 65 R
15 -40 R
16 -1 R
16 -6 R
16 3 R
16 4 R
16 -10 R
16 -1021 R
16 -33 R
16 26 R
16 -27 R
16 -1 R
16 15 R
16 -1 R
16 -27 R
16 24 R
16 1 R
16 8 R
16 -9 R
16 -9 R
16 -1039 R
16 9 R
16 -36 R
15 15 R
16 1049 R
16 16 R
16 0 R
16 -12 R
16 11 R
16 31 R
16 -39 R
16 -1036 R
16 -21 R
16 -13 R
16 -11 R
16 17 R
16 15 R
16 -20 R
16 1057 R
16 21 R
16 -15 R
16 30 R
16 -34 R
16 -24 R
16 39 R
16 1081 R
15 -18 R
16 -23 R
16 29 R
16 5 R
16 11 R
16 -28 R
16 22 R
16 -7 R
16 -9 R
16 -16 R
16 38 R
16 -37 R
16 5 R
16 -13 R
16 7 R
16 26 R
16 -13 R
16 -8 R
16 -5 R
16 27 R
16 -21 R
16 -14 R
16 24 R
15 -18 R
16 31 R
16 -44 R
16 -21 R
16 47 R
16 25 R
16 -4 R
16 1031 R
16 31 R
16 -16 R
16 1 R
16 14 R
16 -11 R
16 24 R
16 -28 R
16 25 R
16 2 R
16 -65 R
16 45 R
16 -20 R
16 25 R
16 -7 R
16 -6 R
15 -19 R
16 25 R
16 -22 R
16 -17 R
16 35 R
16 -15 R
16 -2 R
16 33 R
16 -39 R
16 33 R
16 -13 R
1028 5041 CircleF
1044 5024 CircleF
1060 5054 CircleF
1076 5051 CircleF
1092 5038 CircleF
1108 5055 CircleF
1124 5058 CircleF
1140 5028 CircleF
1156 5043 CircleF
1172 5052 CircleF
1188 5042 CircleF
1203 5031 CircleF
1219 5024 CircleF
1235 4992 CircleF
1251 5027 CircleF
1267 5049 CircleF
1283 5057 CircleF
1299 5027 CircleF
1315 5051 CircleF
1331 5040 CircleF
1347 5054 CircleF
1363 5039 CircleF
1379 5003 CircleF
1395 5042 CircleF
1411 5032 CircleF
1427 5027 CircleF
1443 5009 CircleF
1459 5041 CircleF
1475 5054 CircleF
1491 5043 CircleF
1507 5034 CircleF
1523 5045 CircleF
1539 5048 CircleF
1555 5040 CircleF
1570 5033 CircleF
1586 5048 CircleF
1602 5053 CircleF
1618 5034 CircleF
1634 5055 CircleF
1650 5054 CircleF
1666 5044 CircleF
1682 5040 CircleF
1698 5035 CircleF
1714 4967 CircleF
1730 5038 CircleF
1746 5050 CircleF
1762 5061 CircleF
1778 5028 CircleF
1794 5057 CircleF
1810 5050 CircleF
1826 5021 CircleF
1842 6101 CircleF
1858 6079 CircleF
1874 6109 CircleF
1890 6075 CircleF
1906 6084 CircleF
1922 6100 CircleF
1937 6107 CircleF
1953 6109 CircleF
1969 6120 CircleF
1985 6110 CircleF
2001 6056 CircleF
2017 6128 CircleF
2033 6085 CircleF
2049 6107 CircleF
2065 6148 CircleF
2081 6093 CircleF
2097 6175 CircleF
2113 7198 CircleF
2129 7191 CircleF
2145 7167 CircleF
2161 7192 CircleF
2177 7177 CircleF
2193 7193 CircleF
2209 7153 CircleF
2225 7135 CircleF
2241 7170 CircleF
2257 7192 CircleF
2273 7137 CircleF
2289 7186 CircleF
2304 7143 CircleF
2320 7152 CircleF
2336 7170 CircleF
2352 7164 CircleF
2368 7185 CircleF
2384 7157 CircleF
2400 7128 CircleF
2416 7155 CircleF
2432 7140 CircleF
2448 7145 CircleF
2464 7161 CircleF
2480 7173 CircleF
2496 7181 CircleF
2512 7159 CircleF
2528 7144 CircleF
2544 7146 CircleF
2560 7184 CircleF
2576 7160 CircleF
2592 7164 CircleF
2608 7150 CircleF
2624 7113 CircleF
2640 7199 CircleF
2655 7161 CircleF
2671 7140 CircleF
2687 7156 CircleF
2703 7154 CircleF
2719 7165 CircleF
2735 7159 CircleF
2751 7172 CircleF
2767 7171 CircleF
2783 7145 CircleF
2799 7177 CircleF
2815 7170 CircleF
2831 7159 CircleF
2847 7174 CircleF
2863 7144 CircleF
2879 7159 CircleF
2895 7128 CircleF
2911 7135 CircleF
2927 7196 CircleF
2943 7162 CircleF
2959 7189 CircleF
2975 7168 CircleF
2991 7163 CircleF
3007 7163 CircleF
3022 7160 CircleF
3038 7169 CircleF
3054 7161 CircleF
3070 7185 CircleF
3086 7149 CircleF
3102 7182 CircleF
3118 7144 CircleF
3134 7173 CircleF
3150 7166 CircleF
3166 7164 CircleF
3182 7122 CircleF
3198 7178 CircleF
3214 7169 CircleF
3230 6126 CircleF
3246 6105 CircleF
3262 6104 CircleF
3278 6090 CircleF
3294 6098 CircleF
3310 6141 CircleF
3326 6108 CircleF
3342 6097 CircleF
3358 6073 CircleF
3374 6087 CircleF
3389 6089 CircleF
3405 6112 CircleF
3421 6109 CircleF
3437 6128 CircleF
3453 6097 CircleF
3469 6098 CircleF
3485 6106 CircleF
3501 6104 CircleF
3517 5065 CircleF
3533 6097 CircleF
3549 6089 CircleF
3565 6011 CircleF
3581 5012 CircleF
3597 5042 CircleF
3613 6100 CircleF
3629 6112 CircleF
3645 6123 CircleF
3661 6098 CircleF
3677 6101 CircleF
3693 6089 CircleF
3709 6118 CircleF
3725 6089 CircleF
3740 6102 CircleF
3756 6116 CircleF
3772 5064 CircleF
3788 5037 CircleF
3804 5073 CircleF
3820 5033 CircleF
3836 5019 CircleF
3852 5053 CircleF
3868 5042 CircleF
3884 5037 CircleF
3900 5052 CircleF
3916 5035 CircleF
3932 5048 CircleF
3948 5026 CircleF
3964 5037 CircleF
3980 5041 CircleF
3996 5048 CircleF
4012 5052 CircleF
4028 5014 CircleF
4044 5046 CircleF
4060 5055 CircleF
4076 5053 CircleF
4092 5037 CircleF
4107 5041 CircleF
4123 5020 CircleF
4139 5038 CircleF
4155 5022 CircleF
4171 5027 CircleF
4187 5071 CircleF
4203 5062 CircleF
4219 5037 CircleF
4235 5052 CircleF
4251 5016 CircleF
4267 5050 CircleF
4283 5038 CircleF
4299 5056 CircleF
4315 5045 CircleF
4331 6085 CircleF
4347 6116 CircleF
4363 6147 CircleF
4379 6117 CircleF
4395 6111 CircleF
4411 6098 CircleF
4427 6096 CircleF
4443 6099 CircleF
4459 6111 CircleF
4474 6116 CircleF
4490 6037 CircleF
4506 6127 CircleF
4522 6110 CircleF
4538 6095 CircleF
4554 6090 CircleF
4570 6079 CircleF
4586 6094 CircleF
4602 6113 CircleF
4618 6103 CircleF
4634 5014 CircleF
4650 5027 CircleF
4666 5050 CircleF
4682 5079 CircleF
4698 5036 CircleF
4714 5049 CircleF
4730 5068 CircleF
4746 5043 CircleF
4762 5059 CircleF
4778 5068 CircleF
4794 5039 CircleF
4810 5048 CircleF
4826 5046 CircleF
4841 5033 CircleF
4857 5050 CircleF
4873 5040 CircleF
4889 5020 CircleF
4905 5026 CircleF
4921 5037 CircleF
4937 5055 CircleF
4953 3980 CircleF
4969 2895 CircleF
4985 2932 CircleF
5001 2925 CircleF
5017 2929 CircleF
5033 2918 CircleF
5049 2924 CircleF
5065 2911 CircleF
5081 2921 CircleF
5097 2895 CircleF
5113 2905 CircleF
5129 2926 CircleF
5145 2915 CircleF
5161 2930 CircleF
5177 2910 CircleF
5192 2924 CircleF
5208 2905 CircleF
5224 2919 CircleF
5240 2916 CircleF
5256 2912 CircleF
5272 2939 CircleF
5288 2906 CircleF
5304 2932 CircleF
5320 1826 CircleF
5336 1842 CircleF
5352 1843 CircleF
5368 1846 CircleF
5384 1847 CircleF
5400 1848 CircleF
5416 814 CircleF
5432 795 CircleF
5448 791 CircleF
5464 791 CircleF
5480 795 CircleF
5496 792 CircleF
5512 777 CircleF
5528 841 CircleF
5544 805 CircleF
5559 801 CircleF
5575 788 CircleF
5591 811 CircleF
5607 1855 CircleF
5623 1845 CircleF
5639 1860 CircleF
5655 1868 CircleF
5671 1863 CircleF
5687 1853 CircleF
5703 1872 CircleF
5719 1856 CircleF
5735 1852 CircleF
5751 1825 CircleF
5767 1821 CircleF
5783 1905 CircleF
5799 1862 CircleF
5815 1850 CircleF
5831 1855 CircleF
5847 1877 CircleF
5863 1818 CircleF
5879 1843 CircleF
5895 1898 CircleF
5911 1844 CircleF
5926 1862 CircleF
5942 1855 CircleF
5958 1855 CircleF
5974 1819 CircleF
5990 1856 CircleF
6006 1855 CircleF
6022 1857 CircleF
6038 1858 CircleF
6054 1856 CircleF
6070 1832 CircleF
6086 1857 CircleF
6102 2903 CircleF
6118 2915 CircleF
6134 2921 CircleF
6150 3965 CircleF
6166 4013 CircleF
6182 3996 CircleF
6198 3965 CircleF
6214 3998 CircleF
6230 3970 CircleF
6246 3996 CircleF
6262 3987 CircleF
6278 3931 CircleF
6293 3967 CircleF
6309 3977 CircleF
6325 3981 CircleF
6341 3980 CircleF
6357 3988 CircleF
6373 3981 CircleF
6389 3956 CircleF
6405 4012 CircleF
6421 5046 CircleF
6437 5031 CircleF
6453 5057 CircleF
6469 5047 CircleF
6485 5044 CircleF
6501 5023 CircleF
6517 5044 CircleF
6533 3968 CircleF
6549 3987 CircleF
6565 3981 CircleF
6581 3983 CircleF
6597 3977 CircleF
6613 3986 CircleF
6629 3979 CircleF
6644 3966 CircleF
6660 3981 CircleF
6676 3984 CircleF
6692 3937 CircleF
6708 3963 CircleF
6724 3976 CircleF
6740 3973 CircleF
6756 3988 CircleF
6772 4022 CircleF
6788 3955 CircleF
6804 3958 CircleF
6820 3977 CircleF
6836 4010 CircleF
6852 3977 CircleF
6868 3981 CircleF
6884 3983 CircleF
6900 3987 CircleF
6916 3965 CircleF
6932 3984 CircleF
6948 3973 CircleF
6964 3988 CircleF
6980 4001 CircleF
6996 3962 CircleF
7011 4004 CircleF
7027 3988 CircleF
7043 4013 CircleF
7059 3971 CircleF
7075 3987 CircleF
7091 3974 CircleF
7107 3945 CircleF
7123 5022 CircleF
7139 5060 CircleF
7155 5034 CircleF
7171 5064 CircleF
7187 5042 CircleF
7203 5022 CircleF
7219 5047 CircleF
7235 5029 CircleF
7251 5044 CircleF
7267 5046 CircleF
7283 5042 CircleF
7299 5027 CircleF
7315 5057 CircleF
7331 5044 CircleF
7347 5010 CircleF
7363 5075 CircleF
7378 5035 CircleF
7394 5034 CircleF
7410 5028 CircleF
7426 5031 CircleF
7442 5035 CircleF
7458 5025 CircleF
7474 4004 CircleF
7490 3971 CircleF
7506 3997 CircleF
7522 3970 CircleF
7538 3969 CircleF
7554 3984 CircleF
7570 3983 CircleF
7586 3956 CircleF
7602 3980 CircleF
7618 3981 CircleF
7634 3989 CircleF
7650 3980 CircleF
7666 3971 CircleF
7682 2932 CircleF
7698 2941 CircleF
7714 2905 CircleF
7729 2920 CircleF
7745 3969 CircleF
7761 3985 CircleF
7777 3985 CircleF
7793 3973 CircleF
7809 3984 CircleF
7825 4015 CircleF
7841 3976 CircleF
7857 2940 CircleF
7873 2919 CircleF
7889 2906 CircleF
7905 2895 CircleF
7921 2912 CircleF
7937 2927 CircleF
7953 2907 CircleF
7969 3964 CircleF
7985 3985 CircleF
8001 3970 CircleF
8017 4000 CircleF
8033 3966 CircleF
8049 3942 CircleF
8065 3981 CircleF
8081 5062 CircleF
8096 5044 CircleF
8112 5021 CircleF
8128 5050 CircleF
8144 5055 CircleF
8160 5066 CircleF
8176 5038 CircleF
8192 5060 CircleF
8208 5053 CircleF
8224 5044 CircleF
8240 5028 CircleF
8256 5066 CircleF
8272 5029 CircleF
8288 5034 CircleF
8304 5021 CircleF
8320 5028 CircleF
8336 5054 CircleF
8352 5041 CircleF
8368 5033 CircleF
8384 5028 CircleF
8400 5055 CircleF
8416 5034 CircleF
8432 5020 CircleF
8448 5044 CircleF
8463 5026 CircleF
8479 5057 CircleF
8495 5013 CircleF
8511 4992 CircleF
8527 5039 CircleF
8543 5064 CircleF
8559 5060 CircleF
8575 6091 CircleF
8591 6122 CircleF
8607 6106 CircleF
8623 6107 CircleF
8639 6121 CircleF
8655 6110 CircleF
8671 6134 CircleF
8687 6106 CircleF
8703 6131 CircleF
8719 6133 CircleF
8735 6068 CircleF
8751 6113 CircleF
8767 6093 CircleF
8783 6118 CircleF
8799 6111 CircleF
8815 6105 CircleF
8830 6086 CircleF
8846 6111 CircleF
8862 6089 CircleF
8878 6072 CircleF
8894 6107 CircleF
8910 6092 CircleF
8926 6090 CircleF
8942 6123 CircleF
8958 6084 CircleF
8974 6117 CircleF
8990 6104 CircleF
2.000 UL
LTb
LCb setrgbcolor
1.000 UL
LTb
LCb setrgbcolor
1020 7319 N
0 -6679 V
7978 0 V
0 6679 V
-7978 0 V
Z stroke
1.000 UP
1.000 UL
LTb
LCb setrgbcolor
stroke
grestore
end
showpage
  }}%
  \put(5009,140){\makebox(0,0){\large{sweeps}}}%
  \put(200,4979){\makebox(0,0){\Large{$Q_L$}}}%
  \put(8998,440){\makebox(0,0){\strut{}\ {$50000$}}}%
  \put(7402,440){\makebox(0,0){\strut{}\ {$40000$}}}%
  \put(5807,440){\makebox(0,0){\strut{}\ {$30000$}}}%
  \put(4211,440){\makebox(0,0){\strut{}\ {$20000$}}}%
  \put(2616,440){\makebox(0,0){\strut{}\ {$10000$}}}%
  \put(1020,440){\makebox(0,0){\strut{}\ {$0$}}}%
  \put(900,7319){\makebox(0,0)[r]{\strut{}\ \ {$3$}}}%
  \put(900,6206){\makebox(0,0)[r]{\strut{}\ \ {$2$}}}%
  \put(900,5093){\makebox(0,0)[r]{\strut{}\ \ {$1$}}}%
  \put(900,3980){\makebox(0,0)[r]{\strut{}\ \ {$0$}}}%
  \put(900,2866){\makebox(0,0)[r]{\strut{}\ \ {$-1$}}}%
  \put(900,1753){\makebox(0,0)[r]{\strut{}\ \ {$-2$}}}%
  \put(900,640){\makebox(0,0)[r]{\strut{}\ \ {$-3$}}}%
\end{picture}%
\endgroup
\endinput

\end	{center}
\caption{The lattice topological charge $Q_L$ for a sequence of $SU(5)$ lattice fields after
2 ($\circ$) and 20 ($\bullet$) cooling sweeps. Calculations of $Q_L$ every 100 Monte Carlo sweeps
over a sequence of 50000 sweeps, at $\beta=17.63$ on a $16^320$ lattice.}
\label{fig_Qseq_su5}
\end{figure}



\begin{figure}[htb]
\begin	{center}
\leavevmode
% GNUPLOT: LaTeX picture with Postscript
\begingroup%
\makeatletter%
\newcommand{\GNUPLOTspecial}{%
  \@sanitize\catcode`\%=14\relax\special}%
\setlength{\unitlength}{0.0500bp}%
\begin{picture}(9360,7560)(0,0)%
  {\GNUPLOTspecial{"
%!PS-Adobe-2.0 EPSF-2.0
%%Title: plot_Qcool20c2_su8b47.75.tex
%%Creator: gnuplot 5.0 patchlevel 3
%%CreationDate: Thu Apr 29 13:50:40 2021
%%DocumentFonts: 
%%BoundingBox: 0 0 468 378
%%EndComments
%%BeginProlog
/gnudict 256 dict def
gnudict begin
%
% The following true/false flags may be edited by hand if desired.
% The unit line width and grayscale image gamma correction may also be changed.
%
/Color true def
/Blacktext true def
/Solid false def
/Dashlength 1 def
/Landscape false def
/Level1 false def
/Level3 false def
/Rounded false def
/ClipToBoundingBox false def
/SuppressPDFMark false def
/TransparentPatterns false def
/gnulinewidth 5.000 def
/userlinewidth gnulinewidth def
/Gamma 1.0 def
/BackgroundColor {-1.000 -1.000 -1.000} def
%
/vshift -66 def
/dl1 {
  10.0 Dashlength userlinewidth gnulinewidth div mul mul mul
  Rounded { currentlinewidth 0.75 mul sub dup 0 le { pop 0.01 } if } if
} def
/dl2 {
  10.0 Dashlength userlinewidth gnulinewidth div mul mul mul
  Rounded { currentlinewidth 0.75 mul add } if
} def
/hpt_ 31.5 def
/vpt_ 31.5 def
/hpt hpt_ def
/vpt vpt_ def
/doclip {
  ClipToBoundingBox {
    newpath 0 0 moveto 468 0 lineto 468 378 lineto 0 378 lineto closepath
    clip
  } if
} def
%
% Gnuplot Prolog Version 5.1 (Oct 2015)
%
%/SuppressPDFMark true def
%
/M {moveto} bind def
/L {lineto} bind def
/R {rmoveto} bind def
/V {rlineto} bind def
/N {newpath moveto} bind def
/Z {closepath} bind def
/C {setrgbcolor} bind def
/f {rlineto fill} bind def
/g {setgray} bind def
/Gshow {show} def   % May be redefined later in the file to support UTF-8
/vpt2 vpt 2 mul def
/hpt2 hpt 2 mul def
/Lshow {currentpoint stroke M 0 vshift R 
	Blacktext {gsave 0 setgray textshow grestore} {textshow} ifelse} def
/Rshow {currentpoint stroke M dup stringwidth pop neg vshift R
	Blacktext {gsave 0 setgray textshow grestore} {textshow} ifelse} def
/Cshow {currentpoint stroke M dup stringwidth pop -2 div vshift R 
	Blacktext {gsave 0 setgray textshow grestore} {textshow} ifelse} def
/UP {dup vpt_ mul /vpt exch def hpt_ mul /hpt exch def
  /hpt2 hpt 2 mul def /vpt2 vpt 2 mul def} def
/DL {Color {setrgbcolor Solid {pop []} if 0 setdash}
 {pop pop pop 0 setgray Solid {pop []} if 0 setdash} ifelse} def
/BL {stroke userlinewidth 2 mul setlinewidth
	Rounded {1 setlinejoin 1 setlinecap} if} def
/AL {stroke userlinewidth 2 div setlinewidth
	Rounded {1 setlinejoin 1 setlinecap} if} def
/UL {dup gnulinewidth mul /userlinewidth exch def
	dup 1 lt {pop 1} if 10 mul /udl exch def} def
/PL {stroke userlinewidth setlinewidth
	Rounded {1 setlinejoin 1 setlinecap} if} def
3.8 setmiterlimit
% Classic Line colors (version 5.0)
/LCw {1 1 1} def
/LCb {0 0 0} def
/LCa {0 0 0} def
/LC0 {1 0 0} def
/LC1 {0 1 0} def
/LC2 {0 0 1} def
/LC3 {1 0 1} def
/LC4 {0 1 1} def
/LC5 {1 1 0} def
/LC6 {0 0 0} def
/LC7 {1 0.3 0} def
/LC8 {0.5 0.5 0.5} def
% Default dash patterns (version 5.0)
/LTB {BL [] LCb DL} def
/LTw {PL [] 1 setgray} def
/LTb {PL [] LCb DL} def
/LTa {AL [1 udl mul 2 udl mul] 0 setdash LCa setrgbcolor} def
/LT0 {PL [] LC0 DL} def
/LT1 {PL [2 dl1 3 dl2] LC1 DL} def
/LT2 {PL [1 dl1 1.5 dl2] LC2 DL} def
/LT3 {PL [6 dl1 2 dl2 1 dl1 2 dl2] LC3 DL} def
/LT4 {PL [1 dl1 2 dl2 6 dl1 2 dl2 1 dl1 2 dl2] LC4 DL} def
/LT5 {PL [4 dl1 2 dl2] LC5 DL} def
/LT6 {PL [1.5 dl1 1.5 dl2 1.5 dl1 1.5 dl2 1.5 dl1 6 dl2] LC6 DL} def
/LT7 {PL [3 dl1 3 dl2 1 dl1 3 dl2] LC7 DL} def
/LT8 {PL [2 dl1 2 dl2 2 dl1 6 dl2] LC8 DL} def
/SL {[] 0 setdash} def
/Pnt {stroke [] 0 setdash gsave 1 setlinecap M 0 0 V stroke grestore} def
/Dia {stroke [] 0 setdash 2 copy vpt add M
  hpt neg vpt neg V hpt vpt neg V
  hpt vpt V hpt neg vpt V closepath stroke
  Pnt} def
/Pls {stroke [] 0 setdash vpt sub M 0 vpt2 V
  currentpoint stroke M
  hpt neg vpt neg R hpt2 0 V stroke
 } def
/Box {stroke [] 0 setdash 2 copy exch hpt sub exch vpt add M
  0 vpt2 neg V hpt2 0 V 0 vpt2 V
  hpt2 neg 0 V closepath stroke
  Pnt} def
/Crs {stroke [] 0 setdash exch hpt sub exch vpt add M
  hpt2 vpt2 neg V currentpoint stroke M
  hpt2 neg 0 R hpt2 vpt2 V stroke} def
/TriU {stroke [] 0 setdash 2 copy vpt 1.12 mul add M
  hpt neg vpt -1.62 mul V
  hpt 2 mul 0 V
  hpt neg vpt 1.62 mul V closepath stroke
  Pnt} def
/Star {2 copy Pls Crs} def
/BoxF {stroke [] 0 setdash exch hpt sub exch vpt add M
  0 vpt2 neg V hpt2 0 V 0 vpt2 V
  hpt2 neg 0 V closepath fill} def
/TriUF {stroke [] 0 setdash vpt 1.12 mul add M
  hpt neg vpt -1.62 mul V
  hpt 2 mul 0 V
  hpt neg vpt 1.62 mul V closepath fill} def
/TriD {stroke [] 0 setdash 2 copy vpt 1.12 mul sub M
  hpt neg vpt 1.62 mul V
  hpt 2 mul 0 V
  hpt neg vpt -1.62 mul V closepath stroke
  Pnt} def
/TriDF {stroke [] 0 setdash vpt 1.12 mul sub M
  hpt neg vpt 1.62 mul V
  hpt 2 mul 0 V
  hpt neg vpt -1.62 mul V closepath fill} def
/DiaF {stroke [] 0 setdash vpt add M
  hpt neg vpt neg V hpt vpt neg V
  hpt vpt V hpt neg vpt V closepath fill} def
/Pent {stroke [] 0 setdash 2 copy gsave
  translate 0 hpt M 4 {72 rotate 0 hpt L} repeat
  closepath stroke grestore Pnt} def
/PentF {stroke [] 0 setdash gsave
  translate 0 hpt M 4 {72 rotate 0 hpt L} repeat
  closepath fill grestore} def
/Circle {stroke [] 0 setdash 2 copy
  hpt 0 360 arc stroke Pnt} def
/CircleF {stroke [] 0 setdash hpt 0 360 arc fill} def
/C0 {BL [] 0 setdash 2 copy moveto vpt 90 450 arc} bind def
/C1 {BL [] 0 setdash 2 copy moveto
	2 copy vpt 0 90 arc closepath fill
	vpt 0 360 arc closepath} bind def
/C2 {BL [] 0 setdash 2 copy moveto
	2 copy vpt 90 180 arc closepath fill
	vpt 0 360 arc closepath} bind def
/C3 {BL [] 0 setdash 2 copy moveto
	2 copy vpt 0 180 arc closepath fill
	vpt 0 360 arc closepath} bind def
/C4 {BL [] 0 setdash 2 copy moveto
	2 copy vpt 180 270 arc closepath fill
	vpt 0 360 arc closepath} bind def
/C5 {BL [] 0 setdash 2 copy moveto
	2 copy vpt 0 90 arc
	2 copy moveto
	2 copy vpt 180 270 arc closepath fill
	vpt 0 360 arc} bind def
/C6 {BL [] 0 setdash 2 copy moveto
	2 copy vpt 90 270 arc closepath fill
	vpt 0 360 arc closepath} bind def
/C7 {BL [] 0 setdash 2 copy moveto
	2 copy vpt 0 270 arc closepath fill
	vpt 0 360 arc closepath} bind def
/C8 {BL [] 0 setdash 2 copy moveto
	2 copy vpt 270 360 arc closepath fill
	vpt 0 360 arc closepath} bind def
/C9 {BL [] 0 setdash 2 copy moveto
	2 copy vpt 270 450 arc closepath fill
	vpt 0 360 arc closepath} bind def
/C10 {BL [] 0 setdash 2 copy 2 copy moveto vpt 270 360 arc closepath fill
	2 copy moveto
	2 copy vpt 90 180 arc closepath fill
	vpt 0 360 arc closepath} bind def
/C11 {BL [] 0 setdash 2 copy moveto
	2 copy vpt 0 180 arc closepath fill
	2 copy moveto
	2 copy vpt 270 360 arc closepath fill
	vpt 0 360 arc closepath} bind def
/C12 {BL [] 0 setdash 2 copy moveto
	2 copy vpt 180 360 arc closepath fill
	vpt 0 360 arc closepath} bind def
/C13 {BL [] 0 setdash 2 copy moveto
	2 copy vpt 0 90 arc closepath fill
	2 copy moveto
	2 copy vpt 180 360 arc closepath fill
	vpt 0 360 arc closepath} bind def
/C14 {BL [] 0 setdash 2 copy moveto
	2 copy vpt 90 360 arc closepath fill
	vpt 0 360 arc} bind def
/C15 {BL [] 0 setdash 2 copy vpt 0 360 arc closepath fill
	vpt 0 360 arc closepath} bind def
/Rec {newpath 4 2 roll moveto 1 index 0 rlineto 0 exch rlineto
	neg 0 rlineto closepath} bind def
/Square {dup Rec} bind def
/Bsquare {vpt sub exch vpt sub exch vpt2 Square} bind def
/S0 {BL [] 0 setdash 2 copy moveto 0 vpt rlineto BL Bsquare} bind def
/S1 {BL [] 0 setdash 2 copy vpt Square fill Bsquare} bind def
/S2 {BL [] 0 setdash 2 copy exch vpt sub exch vpt Square fill Bsquare} bind def
/S3 {BL [] 0 setdash 2 copy exch vpt sub exch vpt2 vpt Rec fill Bsquare} bind def
/S4 {BL [] 0 setdash 2 copy exch vpt sub exch vpt sub vpt Square fill Bsquare} bind def
/S5 {BL [] 0 setdash 2 copy 2 copy vpt Square fill
	exch vpt sub exch vpt sub vpt Square fill Bsquare} bind def
/S6 {BL [] 0 setdash 2 copy exch vpt sub exch vpt sub vpt vpt2 Rec fill Bsquare} bind def
/S7 {BL [] 0 setdash 2 copy exch vpt sub exch vpt sub vpt vpt2 Rec fill
	2 copy vpt Square fill Bsquare} bind def
/S8 {BL [] 0 setdash 2 copy vpt sub vpt Square fill Bsquare} bind def
/S9 {BL [] 0 setdash 2 copy vpt sub vpt vpt2 Rec fill Bsquare} bind def
/S10 {BL [] 0 setdash 2 copy vpt sub vpt Square fill 2 copy exch vpt sub exch vpt Square fill
	Bsquare} bind def
/S11 {BL [] 0 setdash 2 copy vpt sub vpt Square fill 2 copy exch vpt sub exch vpt2 vpt Rec fill
	Bsquare} bind def
/S12 {BL [] 0 setdash 2 copy exch vpt sub exch vpt sub vpt2 vpt Rec fill Bsquare} bind def
/S13 {BL [] 0 setdash 2 copy exch vpt sub exch vpt sub vpt2 vpt Rec fill
	2 copy vpt Square fill Bsquare} bind def
/S14 {BL [] 0 setdash 2 copy exch vpt sub exch vpt sub vpt2 vpt Rec fill
	2 copy exch vpt sub exch vpt Square fill Bsquare} bind def
/S15 {BL [] 0 setdash 2 copy Bsquare fill Bsquare} bind def
/D0 {gsave translate 45 rotate 0 0 S0 stroke grestore} bind def
/D1 {gsave translate 45 rotate 0 0 S1 stroke grestore} bind def
/D2 {gsave translate 45 rotate 0 0 S2 stroke grestore} bind def
/D3 {gsave translate 45 rotate 0 0 S3 stroke grestore} bind def
/D4 {gsave translate 45 rotate 0 0 S4 stroke grestore} bind def
/D5 {gsave translate 45 rotate 0 0 S5 stroke grestore} bind def
/D6 {gsave translate 45 rotate 0 0 S6 stroke grestore} bind def
/D7 {gsave translate 45 rotate 0 0 S7 stroke grestore} bind def
/D8 {gsave translate 45 rotate 0 0 S8 stroke grestore} bind def
/D9 {gsave translate 45 rotate 0 0 S9 stroke grestore} bind def
/D10 {gsave translate 45 rotate 0 0 S10 stroke grestore} bind def
/D11 {gsave translate 45 rotate 0 0 S11 stroke grestore} bind def
/D12 {gsave translate 45 rotate 0 0 S12 stroke grestore} bind def
/D13 {gsave translate 45 rotate 0 0 S13 stroke grestore} bind def
/D14 {gsave translate 45 rotate 0 0 S14 stroke grestore} bind def
/D15 {gsave translate 45 rotate 0 0 S15 stroke grestore} bind def
/DiaE {stroke [] 0 setdash vpt add M
  hpt neg vpt neg V hpt vpt neg V
  hpt vpt V hpt neg vpt V closepath stroke} def
/BoxE {stroke [] 0 setdash exch hpt sub exch vpt add M
  0 vpt2 neg V hpt2 0 V 0 vpt2 V
  hpt2 neg 0 V closepath stroke} def
/TriUE {stroke [] 0 setdash vpt 1.12 mul add M
  hpt neg vpt -1.62 mul V
  hpt 2 mul 0 V
  hpt neg vpt 1.62 mul V closepath stroke} def
/TriDE {stroke [] 0 setdash vpt 1.12 mul sub M
  hpt neg vpt 1.62 mul V
  hpt 2 mul 0 V
  hpt neg vpt -1.62 mul V closepath stroke} def
/PentE {stroke [] 0 setdash gsave
  translate 0 hpt M 4 {72 rotate 0 hpt L} repeat
  closepath stroke grestore} def
/CircE {stroke [] 0 setdash 
  hpt 0 360 arc stroke} def
/Opaque {gsave closepath 1 setgray fill grestore 0 setgray closepath} def
/DiaW {stroke [] 0 setdash vpt add M
  hpt neg vpt neg V hpt vpt neg V
  hpt vpt V hpt neg vpt V Opaque stroke} def
/BoxW {stroke [] 0 setdash exch hpt sub exch vpt add M
  0 vpt2 neg V hpt2 0 V 0 vpt2 V
  hpt2 neg 0 V Opaque stroke} def
/TriUW {stroke [] 0 setdash vpt 1.12 mul add M
  hpt neg vpt -1.62 mul V
  hpt 2 mul 0 V
  hpt neg vpt 1.62 mul V Opaque stroke} def
/TriDW {stroke [] 0 setdash vpt 1.12 mul sub M
  hpt neg vpt 1.62 mul V
  hpt 2 mul 0 V
  hpt neg vpt -1.62 mul V Opaque stroke} def
/PentW {stroke [] 0 setdash gsave
  translate 0 hpt M 4 {72 rotate 0 hpt L} repeat
  Opaque stroke grestore} def
/CircW {stroke [] 0 setdash 
  hpt 0 360 arc Opaque stroke} def
/BoxFill {gsave Rec 1 setgray fill grestore} def
/Density {
  /Fillden exch def
  currentrgbcolor
  /ColB exch def /ColG exch def /ColR exch def
  /ColR ColR Fillden mul Fillden sub 1 add def
  /ColG ColG Fillden mul Fillden sub 1 add def
  /ColB ColB Fillden mul Fillden sub 1 add def
  ColR ColG ColB setrgbcolor} def
/BoxColFill {gsave Rec PolyFill} def
/PolyFill {gsave Density fill grestore grestore} def
/h {rlineto rlineto rlineto gsave closepath fill grestore} bind def
%
% PostScript Level 1 Pattern Fill routine for rectangles
% Usage: x y w h s a XX PatternFill
%	x,y = lower left corner of box to be filled
%	w,h = width and height of box
%	  a = angle in degrees between lines and x-axis
%	 XX = 0/1 for no/yes cross-hatch
%
/PatternFill {gsave /PFa [ 9 2 roll ] def
  PFa 0 get PFa 2 get 2 div add PFa 1 get PFa 3 get 2 div add translate
  PFa 2 get -2 div PFa 3 get -2 div PFa 2 get PFa 3 get Rec
  TransparentPatterns {} {gsave 1 setgray fill grestore} ifelse
  clip
  currentlinewidth 0.5 mul setlinewidth
  /PFs PFa 2 get dup mul PFa 3 get dup mul add sqrt def
  0 0 M PFa 5 get rotate PFs -2 div dup translate
  0 1 PFs PFa 4 get div 1 add floor cvi
	{PFa 4 get mul 0 M 0 PFs V} for
  0 PFa 6 get ne {
	0 1 PFs PFa 4 get div 1 add floor cvi
	{PFa 4 get mul 0 2 1 roll M PFs 0 V} for
 } if
  stroke grestore} def
%
/languagelevel where
 {pop languagelevel} {1} ifelse
dup 2 lt
	{/InterpretLevel1 true def
	 /InterpretLevel3 false def}
	{/InterpretLevel1 Level1 def
	 2 gt
	    {/InterpretLevel3 Level3 def}
	    {/InterpretLevel3 false def}
	 ifelse }
 ifelse
%
% PostScript level 2 pattern fill definitions
%
/Level2PatternFill {
/Tile8x8 {/PaintType 2 /PatternType 1 /TilingType 1 /BBox [0 0 8 8] /XStep 8 /YStep 8}
	bind def
/KeepColor {currentrgbcolor [/Pattern /DeviceRGB] setcolorspace} bind def
<< Tile8x8
 /PaintProc {0.5 setlinewidth pop 0 0 M 8 8 L 0 8 M 8 0 L stroke} 
>> matrix makepattern
/Pat1 exch def
<< Tile8x8
 /PaintProc {0.5 setlinewidth pop 0 0 M 8 8 L 0 8 M 8 0 L stroke
	0 4 M 4 8 L 8 4 L 4 0 L 0 4 L stroke}
>> matrix makepattern
/Pat2 exch def
<< Tile8x8
 /PaintProc {0.5 setlinewidth pop 0 0 M 0 8 L
	8 8 L 8 0 L 0 0 L fill}
>> matrix makepattern
/Pat3 exch def
<< Tile8x8
 /PaintProc {0.5 setlinewidth pop -4 8 M 8 -4 L
	0 12 M 12 0 L stroke}
>> matrix makepattern
/Pat4 exch def
<< Tile8x8
 /PaintProc {0.5 setlinewidth pop -4 0 M 8 12 L
	0 -4 M 12 8 L stroke}
>> matrix makepattern
/Pat5 exch def
<< Tile8x8
 /PaintProc {0.5 setlinewidth pop -2 8 M 4 -4 L
	0 12 M 8 -4 L 4 12 M 10 0 L stroke}
>> matrix makepattern
/Pat6 exch def
<< Tile8x8
 /PaintProc {0.5 setlinewidth pop -2 0 M 4 12 L
	0 -4 M 8 12 L 4 -4 M 10 8 L stroke}
>> matrix makepattern
/Pat7 exch def
<< Tile8x8
 /PaintProc {0.5 setlinewidth pop 8 -2 M -4 4 L
	12 0 M -4 8 L 12 4 M 0 10 L stroke}
>> matrix makepattern
/Pat8 exch def
<< Tile8x8
 /PaintProc {0.5 setlinewidth pop 0 -2 M 12 4 L
	-4 0 M 12 8 L -4 4 M 8 10 L stroke}
>> matrix makepattern
/Pat9 exch def
/Pattern1 {PatternBgnd KeepColor Pat1 setpattern} bind def
/Pattern2 {PatternBgnd KeepColor Pat2 setpattern} bind def
/Pattern3 {PatternBgnd KeepColor Pat3 setpattern} bind def
/Pattern4 {PatternBgnd KeepColor Landscape {Pat5} {Pat4} ifelse setpattern} bind def
/Pattern5 {PatternBgnd KeepColor Landscape {Pat4} {Pat5} ifelse setpattern} bind def
/Pattern6 {PatternBgnd KeepColor Landscape {Pat9} {Pat6} ifelse setpattern} bind def
/Pattern7 {PatternBgnd KeepColor Landscape {Pat8} {Pat7} ifelse setpattern} bind def
} def
%
%
%End of PostScript Level 2 code
%
/PatternBgnd {
  TransparentPatterns {} {gsave 1 setgray fill grestore} ifelse
} def
%
% Substitute for Level 2 pattern fill codes with
% grayscale if Level 2 support is not selected.
%
/Level1PatternFill {
/Pattern1 {0.250 Density} bind def
/Pattern2 {0.500 Density} bind def
/Pattern3 {0.750 Density} bind def
/Pattern4 {0.125 Density} bind def
/Pattern5 {0.375 Density} bind def
/Pattern6 {0.625 Density} bind def
/Pattern7 {0.875 Density} bind def
} def
%
% Now test for support of Level 2 code
%
Level1 {Level1PatternFill} {Level2PatternFill} ifelse
%
/Symbol-Oblique /Symbol findfont [1 0 .167 1 0 0] makefont
dup length dict begin {1 index /FID eq {pop pop} {def} ifelse} forall
currentdict end definefont pop
%
Level1 SuppressPDFMark or 
{} {
/SDict 10 dict def
systemdict /pdfmark known not {
  userdict /pdfmark systemdict /cleartomark get put
} if
SDict begin [
  /Title (plot_Qcool20c2_su8b47.75.tex)
  /Subject (gnuplot plot)
  /Creator (gnuplot 5.0 patchlevel 3)
  /Author (mteper)
%  /Producer (gnuplot)
%  /Keywords ()
  /CreationDate (Thu Apr 29 13:50:40 2021)
  /DOCINFO pdfmark
end
} ifelse
%
% Support for boxed text - Ethan A Merritt May 2005
%
/InitTextBox { userdict /TBy2 3 -1 roll put userdict /TBx2 3 -1 roll put
           userdict /TBy1 3 -1 roll put userdict /TBx1 3 -1 roll put
	   /Boxing true def } def
/ExtendTextBox { Boxing
    { gsave dup false charpath pathbbox
      dup TBy2 gt {userdict /TBy2 3 -1 roll put} {pop} ifelse
      dup TBx2 gt {userdict /TBx2 3 -1 roll put} {pop} ifelse
      dup TBy1 lt {userdict /TBy1 3 -1 roll put} {pop} ifelse
      dup TBx1 lt {userdict /TBx1 3 -1 roll put} {pop} ifelse
      grestore } if } def
/PopTextBox { newpath TBx1 TBxmargin sub TBy1 TBymargin sub M
               TBx1 TBxmargin sub TBy2 TBymargin add L
	       TBx2 TBxmargin add TBy2 TBymargin add L
	       TBx2 TBxmargin add TBy1 TBymargin sub L closepath } def
/DrawTextBox { PopTextBox stroke /Boxing false def} def
/FillTextBox { gsave PopTextBox 1 1 1 setrgbcolor fill grestore /Boxing false def} def
0 0 0 0 InitTextBox
/TBxmargin 20 def
/TBymargin 20 def
/Boxing false def
/textshow { ExtendTextBox Gshow } def
%
% redundant definitions for compatibility with prologue.ps older than 5.0.2
/LTB {BL [] LCb DL} def
/LTb {PL [] LCb DL} def
end
%%EndProlog
%%Page: 1 1
gnudict begin
gsave
doclip
0 0 translate
0.050 0.050 scale
0 setgray
newpath
BackgroundColor 0 lt 3 1 roll 0 lt exch 0 lt or or not {BackgroundColor C 1.000 0 0 9360.00 7560.00 BoxColFill} if
1.000 UL
LTb
LCb setrgbcolor
1260 640 M
63 0 V
7675 0 R
-63 0 V
stroke
LTb
LCb setrgbcolor
1260 1531 M
63 0 V
7675 0 R
-63 0 V
stroke
LTb
LCb setrgbcolor
1260 2421 M
63 0 V
7675 0 R
-63 0 V
stroke
LTb
LCb setrgbcolor
1260 3312 M
63 0 V
7675 0 R
-63 0 V
stroke
LTb
LCb setrgbcolor
1260 4202 M
63 0 V
7675 0 R
-63 0 V
stroke
LTb
LCb setrgbcolor
1260 5093 M
63 0 V
7675 0 R
-63 0 V
stroke
LTb
LCb setrgbcolor
1260 5983 M
63 0 V
7675 0 R
-63 0 V
stroke
LTb
LCb setrgbcolor
1260 6874 M
63 0 V
7675 0 R
-63 0 V
stroke
LTb
LCb setrgbcolor
1260 640 M
0 63 V
0 6616 R
0 -63 V
stroke
LTb
LCb setrgbcolor
2365 640 M
0 63 V
0 6616 R
0 -63 V
stroke
LTb
LCb setrgbcolor
3471 640 M
0 63 V
0 6616 R
0 -63 V
stroke
LTb
LCb setrgbcolor
4576 640 M
0 63 V
0 6616 R
0 -63 V
stroke
LTb
LCb setrgbcolor
5682 640 M
0 63 V
0 6616 R
0 -63 V
stroke
LTb
LCb setrgbcolor
6787 640 M
0 63 V
0 6616 R
0 -63 V
stroke
LTb
LCb setrgbcolor
7893 640 M
0 63 V
0 6616 R
0 -63 V
stroke
LTb
LCb setrgbcolor
8998 640 M
0 63 V
0 6616 R
0 -63 V
stroke
LTb
LCb setrgbcolor
1.000 UL
LTb
LCb setrgbcolor
1260 7319 N
0 -6679 V
7738 0 V
0 6679 V
-7738 0 V
Z stroke
1.000 UP
1.000 UL
LTb
LCb setrgbcolor
LCb setrgbcolor
LTb
LCb setrgbcolor
LTb
1.500 UP
1.000 UL
LTb
0.58 0.00 0.83 C 2587 640 M
0 9 V
110 -9 R
0 9 V
111 68 R
0 42 V
110 155 R
0 75 V
111 380 R
0 118 V
110 689 R
0 170 V
111 1127 R
0 229 V
110 1092 R
0 277 V
111 -23 R
0 285 V
110 -513 R
0 278 V
3692 3525 M
0 232 V
3802 2154 M
0 169 V
3913 1228 M
0 107 V
4024 827 M
0 62 V
4134 709 M
0 40 V
4245 640 M
0 9 V
2587 644 Circle
2697 644 Circle
2808 738 Circle
2918 952 Circle
3029 1428 Circle
3139 2261 Circle
3250 3588 Circle
3360 4932 Circle
3471 5191 Circle
3581 4959 Circle
3692 3641 Circle
3802 2238 Circle
3913 1281 Circle
4024 858 Circle
4134 729 Circle
4245 644 Circle
1.500 UP
1.000 UL
LTb
0.58 0.00 0.83 C 4576 652 M
0 20 V
111 37 R
0 40 V
110 178 R
0 76 V
111 447 R
0 125 V
110 718 R
0 176 V
111 960 R
0 228 V
111 572 R
0 258 V
110 -105 R
0 263 V
5461 3525 M
0 232 V
5571 2654 M
0 193 V
5682 1437 M
0 124 V
5792 994 M
0 84 V
5903 721 M
0 43 V
6013 652 M
0 20 V
4576 662 BoxF
4687 729 BoxF
4797 965 BoxF
4908 1513 BoxF
5018 2381 BoxF
5129 3543 BoxF
5240 4358 BoxF
5350 4514 BoxF
5461 3641 BoxF
5571 2751 BoxF
5682 1499 BoxF
5792 1036 BoxF
5903 742 BoxF
6013 662 BoxF
1.500 UP
1.000 UL
LTb
0.58 0.00 0.83 C 6345 643 M
0 12 V
111 -12 R
0 12 V
110 58 R
0 41 V
111 98 R
0 66 V
110 284 R
0 105 V
111 479 R
0 148 V
110 350 R
0 176 V
111 272 R
0 197 V
110 -532 R
0 182 V
111 -468 R
0 166 V
110 -724 R
0 133 V
7561 998 M
0 85 V
7671 794 M
0 57 V
7782 663 M
0 25 V
111 -42 R
0 15 V
110 -21 R
0 9 V
6345 649 Box
6456 649 Box
6566 733 Box
6677 885 Box
6787 1254 Box
6898 1860 Box
7008 2372 Box
7119 2831 Box
7229 2488 Box
7340 2194 Box
7450 1620 Box
7561 1041 Box
7671 823 Box
7782 676 Box
7893 653 Box
8003 644 Box
2.000 UL
LTb
LCb setrgbcolor
1.000 UL
LTb
LCb setrgbcolor
1260 7319 N
0 -6679 V
7738 0 V
0 6679 V
-7738 0 V
Z stroke
1.000 UP
1.000 UL
LTb
LCb setrgbcolor
stroke
grestore
end
showpage
  }}%
  \put(5129,140){\makebox(0,0){\large{$Q_L$}}}%
  \put(200,4979){\makebox(0,0){\Large{$N(Q_L)$}}}%
  \put(8998,440){\makebox(0,0){\strut{}\ {$2.5$}}}%
  \put(7893,440){\makebox(0,0){\strut{}\ {$2$}}}%
  \put(6787,440){\makebox(0,0){\strut{}\ {$1.5$}}}%
  \put(5682,440){\makebox(0,0){\strut{}\ {$1$}}}%
  \put(4576,440){\makebox(0,0){\strut{}\ {$0.5$}}}%
  \put(3471,440){\makebox(0,0){\strut{}\ {$0$}}}%
  \put(2365,440){\makebox(0,0){\strut{}\ {$-0.5$}}}%
  \put(1260,440){\makebox(0,0){\strut{}\ {$-1$}}}%
  \put(1140,6874){\makebox(0,0)[r]{\strut{}\ \ {$1400$}}}%
  \put(1140,5983){\makebox(0,0)[r]{\strut{}\ \ {$1200$}}}%
  \put(1140,5093){\makebox(0,0)[r]{\strut{}\ \ {$1000$}}}%
  \put(1140,4202){\makebox(0,0)[r]{\strut{}\ \ {$800$}}}%
  \put(1140,3312){\makebox(0,0)[r]{\strut{}\ \ {$600$}}}%
  \put(1140,2421){\makebox(0,0)[r]{\strut{}\ \ {$400$}}}%
  \put(1140,1531){\makebox(0,0)[r]{\strut{}\ \ {$200$}}}%
  \put(1140,640){\makebox(0,0)[r]{\strut{}\ \ {$0$}}}%
\end{picture}%
\endgroup
\endinput

\end	{center}
\caption{The distribution in the topological charge $Q_L$ as obtained after 2 cooling sweeps,
for fields that after 20 cooling sweeps have topological charges $Q=0$ ($\circ$),
$Q=1$ ($\blacksquare$) and $Q=2$ ($\square$). $N(Q_L)=0$ points suppressed. 
From the same sequences of $SU(8)$ fields generated 
at $\beta=47.75$ plotted in Fig.\ref{fig_Qcool20_su8}.} 
\label{fig_Qcool20c2_su8}
\end{figure}



\begin{figure}[htb]
\begin	{center}
\leavevmode
% GNUPLOT: LaTeX picture with Postscript
\begingroup%
\makeatletter%
\newcommand{\GNUPLOTspecial}{%
  \@sanitize\catcode`\%=14\relax\special}%
\setlength{\unitlength}{0.0500bp}%
\begin{picture}(9360,7560)(0,0)%
  {\GNUPLOTspecial{"
%!PS-Adobe-2.0 EPSF-2.0
%%Title: plot_Qcool20c1_su8b47.75.tex
%%Creator: gnuplot 5.0 patchlevel 3
%%CreationDate: Thu Apr 29 13:50:30 2021
%%DocumentFonts: 
%%BoundingBox: 0 0 468 378
%%EndComments
%%BeginProlog
/gnudict 256 dict def
gnudict begin
%
% The following true/false flags may be edited by hand if desired.
% The unit line width and grayscale image gamma correction may also be changed.
%
/Color true def
/Blacktext true def
/Solid false def
/Dashlength 1 def
/Landscape false def
/Level1 false def
/Level3 false def
/Rounded false def
/ClipToBoundingBox false def
/SuppressPDFMark false def
/TransparentPatterns false def
/gnulinewidth 5.000 def
/userlinewidth gnulinewidth def
/Gamma 1.0 def
/BackgroundColor {-1.000 -1.000 -1.000} def
%
/vshift -66 def
/dl1 {
  10.0 Dashlength userlinewidth gnulinewidth div mul mul mul
  Rounded { currentlinewidth 0.75 mul sub dup 0 le { pop 0.01 } if } if
} def
/dl2 {
  10.0 Dashlength userlinewidth gnulinewidth div mul mul mul
  Rounded { currentlinewidth 0.75 mul add } if
} def
/hpt_ 31.5 def
/vpt_ 31.5 def
/hpt hpt_ def
/vpt vpt_ def
/doclip {
  ClipToBoundingBox {
    newpath 0 0 moveto 468 0 lineto 468 378 lineto 0 378 lineto closepath
    clip
  } if
} def
%
% Gnuplot Prolog Version 5.1 (Oct 2015)
%
%/SuppressPDFMark true def
%
/M {moveto} bind def
/L {lineto} bind def
/R {rmoveto} bind def
/V {rlineto} bind def
/N {newpath moveto} bind def
/Z {closepath} bind def
/C {setrgbcolor} bind def
/f {rlineto fill} bind def
/g {setgray} bind def
/Gshow {show} def   % May be redefined later in the file to support UTF-8
/vpt2 vpt 2 mul def
/hpt2 hpt 2 mul def
/Lshow {currentpoint stroke M 0 vshift R 
	Blacktext {gsave 0 setgray textshow grestore} {textshow} ifelse} def
/Rshow {currentpoint stroke M dup stringwidth pop neg vshift R
	Blacktext {gsave 0 setgray textshow grestore} {textshow} ifelse} def
/Cshow {currentpoint stroke M dup stringwidth pop -2 div vshift R 
	Blacktext {gsave 0 setgray textshow grestore} {textshow} ifelse} def
/UP {dup vpt_ mul /vpt exch def hpt_ mul /hpt exch def
  /hpt2 hpt 2 mul def /vpt2 vpt 2 mul def} def
/DL {Color {setrgbcolor Solid {pop []} if 0 setdash}
 {pop pop pop 0 setgray Solid {pop []} if 0 setdash} ifelse} def
/BL {stroke userlinewidth 2 mul setlinewidth
	Rounded {1 setlinejoin 1 setlinecap} if} def
/AL {stroke userlinewidth 2 div setlinewidth
	Rounded {1 setlinejoin 1 setlinecap} if} def
/UL {dup gnulinewidth mul /userlinewidth exch def
	dup 1 lt {pop 1} if 10 mul /udl exch def} def
/PL {stroke userlinewidth setlinewidth
	Rounded {1 setlinejoin 1 setlinecap} if} def
3.8 setmiterlimit
% Classic Line colors (version 5.0)
/LCw {1 1 1} def
/LCb {0 0 0} def
/LCa {0 0 0} def
/LC0 {1 0 0} def
/LC1 {0 1 0} def
/LC2 {0 0 1} def
/LC3 {1 0 1} def
/LC4 {0 1 1} def
/LC5 {1 1 0} def
/LC6 {0 0 0} def
/LC7 {1 0.3 0} def
/LC8 {0.5 0.5 0.5} def
% Default dash patterns (version 5.0)
/LTB {BL [] LCb DL} def
/LTw {PL [] 1 setgray} def
/LTb {PL [] LCb DL} def
/LTa {AL [1 udl mul 2 udl mul] 0 setdash LCa setrgbcolor} def
/LT0 {PL [] LC0 DL} def
/LT1 {PL [2 dl1 3 dl2] LC1 DL} def
/LT2 {PL [1 dl1 1.5 dl2] LC2 DL} def
/LT3 {PL [6 dl1 2 dl2 1 dl1 2 dl2] LC3 DL} def
/LT4 {PL [1 dl1 2 dl2 6 dl1 2 dl2 1 dl1 2 dl2] LC4 DL} def
/LT5 {PL [4 dl1 2 dl2] LC5 DL} def
/LT6 {PL [1.5 dl1 1.5 dl2 1.5 dl1 1.5 dl2 1.5 dl1 6 dl2] LC6 DL} def
/LT7 {PL [3 dl1 3 dl2 1 dl1 3 dl2] LC7 DL} def
/LT8 {PL [2 dl1 2 dl2 2 dl1 6 dl2] LC8 DL} def
/SL {[] 0 setdash} def
/Pnt {stroke [] 0 setdash gsave 1 setlinecap M 0 0 V stroke grestore} def
/Dia {stroke [] 0 setdash 2 copy vpt add M
  hpt neg vpt neg V hpt vpt neg V
  hpt vpt V hpt neg vpt V closepath stroke
  Pnt} def
/Pls {stroke [] 0 setdash vpt sub M 0 vpt2 V
  currentpoint stroke M
  hpt neg vpt neg R hpt2 0 V stroke
 } def
/Box {stroke [] 0 setdash 2 copy exch hpt sub exch vpt add M
  0 vpt2 neg V hpt2 0 V 0 vpt2 V
  hpt2 neg 0 V closepath stroke
  Pnt} def
/Crs {stroke [] 0 setdash exch hpt sub exch vpt add M
  hpt2 vpt2 neg V currentpoint stroke M
  hpt2 neg 0 R hpt2 vpt2 V stroke} def
/TriU {stroke [] 0 setdash 2 copy vpt 1.12 mul add M
  hpt neg vpt -1.62 mul V
  hpt 2 mul 0 V
  hpt neg vpt 1.62 mul V closepath stroke
  Pnt} def
/Star {2 copy Pls Crs} def
/BoxF {stroke [] 0 setdash exch hpt sub exch vpt add M
  0 vpt2 neg V hpt2 0 V 0 vpt2 V
  hpt2 neg 0 V closepath fill} def
/TriUF {stroke [] 0 setdash vpt 1.12 mul add M
  hpt neg vpt -1.62 mul V
  hpt 2 mul 0 V
  hpt neg vpt 1.62 mul V closepath fill} def
/TriD {stroke [] 0 setdash 2 copy vpt 1.12 mul sub M
  hpt neg vpt 1.62 mul V
  hpt 2 mul 0 V
  hpt neg vpt -1.62 mul V closepath stroke
  Pnt} def
/TriDF {stroke [] 0 setdash vpt 1.12 mul sub M
  hpt neg vpt 1.62 mul V
  hpt 2 mul 0 V
  hpt neg vpt -1.62 mul V closepath fill} def
/DiaF {stroke [] 0 setdash vpt add M
  hpt neg vpt neg V hpt vpt neg V
  hpt vpt V hpt neg vpt V closepath fill} def
/Pent {stroke [] 0 setdash 2 copy gsave
  translate 0 hpt M 4 {72 rotate 0 hpt L} repeat
  closepath stroke grestore Pnt} def
/PentF {stroke [] 0 setdash gsave
  translate 0 hpt M 4 {72 rotate 0 hpt L} repeat
  closepath fill grestore} def
/Circle {stroke [] 0 setdash 2 copy
  hpt 0 360 arc stroke Pnt} def
/CircleF {stroke [] 0 setdash hpt 0 360 arc fill} def
/C0 {BL [] 0 setdash 2 copy moveto vpt 90 450 arc} bind def
/C1 {BL [] 0 setdash 2 copy moveto
	2 copy vpt 0 90 arc closepath fill
	vpt 0 360 arc closepath} bind def
/C2 {BL [] 0 setdash 2 copy moveto
	2 copy vpt 90 180 arc closepath fill
	vpt 0 360 arc closepath} bind def
/C3 {BL [] 0 setdash 2 copy moveto
	2 copy vpt 0 180 arc closepath fill
	vpt 0 360 arc closepath} bind def
/C4 {BL [] 0 setdash 2 copy moveto
	2 copy vpt 180 270 arc closepath fill
	vpt 0 360 arc closepath} bind def
/C5 {BL [] 0 setdash 2 copy moveto
	2 copy vpt 0 90 arc
	2 copy moveto
	2 copy vpt 180 270 arc closepath fill
	vpt 0 360 arc} bind def
/C6 {BL [] 0 setdash 2 copy moveto
	2 copy vpt 90 270 arc closepath fill
	vpt 0 360 arc closepath} bind def
/C7 {BL [] 0 setdash 2 copy moveto
	2 copy vpt 0 270 arc closepath fill
	vpt 0 360 arc closepath} bind def
/C8 {BL [] 0 setdash 2 copy moveto
	2 copy vpt 270 360 arc closepath fill
	vpt 0 360 arc closepath} bind def
/C9 {BL [] 0 setdash 2 copy moveto
	2 copy vpt 270 450 arc closepath fill
	vpt 0 360 arc closepath} bind def
/C10 {BL [] 0 setdash 2 copy 2 copy moveto vpt 270 360 arc closepath fill
	2 copy moveto
	2 copy vpt 90 180 arc closepath fill
	vpt 0 360 arc closepath} bind def
/C11 {BL [] 0 setdash 2 copy moveto
	2 copy vpt 0 180 arc closepath fill
	2 copy moveto
	2 copy vpt 270 360 arc closepath fill
	vpt 0 360 arc closepath} bind def
/C12 {BL [] 0 setdash 2 copy moveto
	2 copy vpt 180 360 arc closepath fill
	vpt 0 360 arc closepath} bind def
/C13 {BL [] 0 setdash 2 copy moveto
	2 copy vpt 0 90 arc closepath fill
	2 copy moveto
	2 copy vpt 180 360 arc closepath fill
	vpt 0 360 arc closepath} bind def
/C14 {BL [] 0 setdash 2 copy moveto
	2 copy vpt 90 360 arc closepath fill
	vpt 0 360 arc} bind def
/C15 {BL [] 0 setdash 2 copy vpt 0 360 arc closepath fill
	vpt 0 360 arc closepath} bind def
/Rec {newpath 4 2 roll moveto 1 index 0 rlineto 0 exch rlineto
	neg 0 rlineto closepath} bind def
/Square {dup Rec} bind def
/Bsquare {vpt sub exch vpt sub exch vpt2 Square} bind def
/S0 {BL [] 0 setdash 2 copy moveto 0 vpt rlineto BL Bsquare} bind def
/S1 {BL [] 0 setdash 2 copy vpt Square fill Bsquare} bind def
/S2 {BL [] 0 setdash 2 copy exch vpt sub exch vpt Square fill Bsquare} bind def
/S3 {BL [] 0 setdash 2 copy exch vpt sub exch vpt2 vpt Rec fill Bsquare} bind def
/S4 {BL [] 0 setdash 2 copy exch vpt sub exch vpt sub vpt Square fill Bsquare} bind def
/S5 {BL [] 0 setdash 2 copy 2 copy vpt Square fill
	exch vpt sub exch vpt sub vpt Square fill Bsquare} bind def
/S6 {BL [] 0 setdash 2 copy exch vpt sub exch vpt sub vpt vpt2 Rec fill Bsquare} bind def
/S7 {BL [] 0 setdash 2 copy exch vpt sub exch vpt sub vpt vpt2 Rec fill
	2 copy vpt Square fill Bsquare} bind def
/S8 {BL [] 0 setdash 2 copy vpt sub vpt Square fill Bsquare} bind def
/S9 {BL [] 0 setdash 2 copy vpt sub vpt vpt2 Rec fill Bsquare} bind def
/S10 {BL [] 0 setdash 2 copy vpt sub vpt Square fill 2 copy exch vpt sub exch vpt Square fill
	Bsquare} bind def
/S11 {BL [] 0 setdash 2 copy vpt sub vpt Square fill 2 copy exch vpt sub exch vpt2 vpt Rec fill
	Bsquare} bind def
/S12 {BL [] 0 setdash 2 copy exch vpt sub exch vpt sub vpt2 vpt Rec fill Bsquare} bind def
/S13 {BL [] 0 setdash 2 copy exch vpt sub exch vpt sub vpt2 vpt Rec fill
	2 copy vpt Square fill Bsquare} bind def
/S14 {BL [] 0 setdash 2 copy exch vpt sub exch vpt sub vpt2 vpt Rec fill
	2 copy exch vpt sub exch vpt Square fill Bsquare} bind def
/S15 {BL [] 0 setdash 2 copy Bsquare fill Bsquare} bind def
/D0 {gsave translate 45 rotate 0 0 S0 stroke grestore} bind def
/D1 {gsave translate 45 rotate 0 0 S1 stroke grestore} bind def
/D2 {gsave translate 45 rotate 0 0 S2 stroke grestore} bind def
/D3 {gsave translate 45 rotate 0 0 S3 stroke grestore} bind def
/D4 {gsave translate 45 rotate 0 0 S4 stroke grestore} bind def
/D5 {gsave translate 45 rotate 0 0 S5 stroke grestore} bind def
/D6 {gsave translate 45 rotate 0 0 S6 stroke grestore} bind def
/D7 {gsave translate 45 rotate 0 0 S7 stroke grestore} bind def
/D8 {gsave translate 45 rotate 0 0 S8 stroke grestore} bind def
/D9 {gsave translate 45 rotate 0 0 S9 stroke grestore} bind def
/D10 {gsave translate 45 rotate 0 0 S10 stroke grestore} bind def
/D11 {gsave translate 45 rotate 0 0 S11 stroke grestore} bind def
/D12 {gsave translate 45 rotate 0 0 S12 stroke grestore} bind def
/D13 {gsave translate 45 rotate 0 0 S13 stroke grestore} bind def
/D14 {gsave translate 45 rotate 0 0 S14 stroke grestore} bind def
/D15 {gsave translate 45 rotate 0 0 S15 stroke grestore} bind def
/DiaE {stroke [] 0 setdash vpt add M
  hpt neg vpt neg V hpt vpt neg V
  hpt vpt V hpt neg vpt V closepath stroke} def
/BoxE {stroke [] 0 setdash exch hpt sub exch vpt add M
  0 vpt2 neg V hpt2 0 V 0 vpt2 V
  hpt2 neg 0 V closepath stroke} def
/TriUE {stroke [] 0 setdash vpt 1.12 mul add M
  hpt neg vpt -1.62 mul V
  hpt 2 mul 0 V
  hpt neg vpt 1.62 mul V closepath stroke} def
/TriDE {stroke [] 0 setdash vpt 1.12 mul sub M
  hpt neg vpt 1.62 mul V
  hpt 2 mul 0 V
  hpt neg vpt -1.62 mul V closepath stroke} def
/PentE {stroke [] 0 setdash gsave
  translate 0 hpt M 4 {72 rotate 0 hpt L} repeat
  closepath stroke grestore} def
/CircE {stroke [] 0 setdash 
  hpt 0 360 arc stroke} def
/Opaque {gsave closepath 1 setgray fill grestore 0 setgray closepath} def
/DiaW {stroke [] 0 setdash vpt add M
  hpt neg vpt neg V hpt vpt neg V
  hpt vpt V hpt neg vpt V Opaque stroke} def
/BoxW {stroke [] 0 setdash exch hpt sub exch vpt add M
  0 vpt2 neg V hpt2 0 V 0 vpt2 V
  hpt2 neg 0 V Opaque stroke} def
/TriUW {stroke [] 0 setdash vpt 1.12 mul add M
  hpt neg vpt -1.62 mul V
  hpt 2 mul 0 V
  hpt neg vpt 1.62 mul V Opaque stroke} def
/TriDW {stroke [] 0 setdash vpt 1.12 mul sub M
  hpt neg vpt 1.62 mul V
  hpt 2 mul 0 V
  hpt neg vpt -1.62 mul V Opaque stroke} def
/PentW {stroke [] 0 setdash gsave
  translate 0 hpt M 4 {72 rotate 0 hpt L} repeat
  Opaque stroke grestore} def
/CircW {stroke [] 0 setdash 
  hpt 0 360 arc Opaque stroke} def
/BoxFill {gsave Rec 1 setgray fill grestore} def
/Density {
  /Fillden exch def
  currentrgbcolor
  /ColB exch def /ColG exch def /ColR exch def
  /ColR ColR Fillden mul Fillden sub 1 add def
  /ColG ColG Fillden mul Fillden sub 1 add def
  /ColB ColB Fillden mul Fillden sub 1 add def
  ColR ColG ColB setrgbcolor} def
/BoxColFill {gsave Rec PolyFill} def
/PolyFill {gsave Density fill grestore grestore} def
/h {rlineto rlineto rlineto gsave closepath fill grestore} bind def
%
% PostScript Level 1 Pattern Fill routine for rectangles
% Usage: x y w h s a XX PatternFill
%	x,y = lower left corner of box to be filled
%	w,h = width and height of box
%	  a = angle in degrees between lines and x-axis
%	 XX = 0/1 for no/yes cross-hatch
%
/PatternFill {gsave /PFa [ 9 2 roll ] def
  PFa 0 get PFa 2 get 2 div add PFa 1 get PFa 3 get 2 div add translate
  PFa 2 get -2 div PFa 3 get -2 div PFa 2 get PFa 3 get Rec
  TransparentPatterns {} {gsave 1 setgray fill grestore} ifelse
  clip
  currentlinewidth 0.5 mul setlinewidth
  /PFs PFa 2 get dup mul PFa 3 get dup mul add sqrt def
  0 0 M PFa 5 get rotate PFs -2 div dup translate
  0 1 PFs PFa 4 get div 1 add floor cvi
	{PFa 4 get mul 0 M 0 PFs V} for
  0 PFa 6 get ne {
	0 1 PFs PFa 4 get div 1 add floor cvi
	{PFa 4 get mul 0 2 1 roll M PFs 0 V} for
 } if
  stroke grestore} def
%
/languagelevel where
 {pop languagelevel} {1} ifelse
dup 2 lt
	{/InterpretLevel1 true def
	 /InterpretLevel3 false def}
	{/InterpretLevel1 Level1 def
	 2 gt
	    {/InterpretLevel3 Level3 def}
	    {/InterpretLevel3 false def}
	 ifelse }
 ifelse
%
% PostScript level 2 pattern fill definitions
%
/Level2PatternFill {
/Tile8x8 {/PaintType 2 /PatternType 1 /TilingType 1 /BBox [0 0 8 8] /XStep 8 /YStep 8}
	bind def
/KeepColor {currentrgbcolor [/Pattern /DeviceRGB] setcolorspace} bind def
<< Tile8x8
 /PaintProc {0.5 setlinewidth pop 0 0 M 8 8 L 0 8 M 8 0 L stroke} 
>> matrix makepattern
/Pat1 exch def
<< Tile8x8
 /PaintProc {0.5 setlinewidth pop 0 0 M 8 8 L 0 8 M 8 0 L stroke
	0 4 M 4 8 L 8 4 L 4 0 L 0 4 L stroke}
>> matrix makepattern
/Pat2 exch def
<< Tile8x8
 /PaintProc {0.5 setlinewidth pop 0 0 M 0 8 L
	8 8 L 8 0 L 0 0 L fill}
>> matrix makepattern
/Pat3 exch def
<< Tile8x8
 /PaintProc {0.5 setlinewidth pop -4 8 M 8 -4 L
	0 12 M 12 0 L stroke}
>> matrix makepattern
/Pat4 exch def
<< Tile8x8
 /PaintProc {0.5 setlinewidth pop -4 0 M 8 12 L
	0 -4 M 12 8 L stroke}
>> matrix makepattern
/Pat5 exch def
<< Tile8x8
 /PaintProc {0.5 setlinewidth pop -2 8 M 4 -4 L
	0 12 M 8 -4 L 4 12 M 10 0 L stroke}
>> matrix makepattern
/Pat6 exch def
<< Tile8x8
 /PaintProc {0.5 setlinewidth pop -2 0 M 4 12 L
	0 -4 M 8 12 L 4 -4 M 10 8 L stroke}
>> matrix makepattern
/Pat7 exch def
<< Tile8x8
 /PaintProc {0.5 setlinewidth pop 8 -2 M -4 4 L
	12 0 M -4 8 L 12 4 M 0 10 L stroke}
>> matrix makepattern
/Pat8 exch def
<< Tile8x8
 /PaintProc {0.5 setlinewidth pop 0 -2 M 12 4 L
	-4 0 M 12 8 L -4 4 M 8 10 L stroke}
>> matrix makepattern
/Pat9 exch def
/Pattern1 {PatternBgnd KeepColor Pat1 setpattern} bind def
/Pattern2 {PatternBgnd KeepColor Pat2 setpattern} bind def
/Pattern3 {PatternBgnd KeepColor Pat3 setpattern} bind def
/Pattern4 {PatternBgnd KeepColor Landscape {Pat5} {Pat4} ifelse setpattern} bind def
/Pattern5 {PatternBgnd KeepColor Landscape {Pat4} {Pat5} ifelse setpattern} bind def
/Pattern6 {PatternBgnd KeepColor Landscape {Pat9} {Pat6} ifelse setpattern} bind def
/Pattern7 {PatternBgnd KeepColor Landscape {Pat8} {Pat7} ifelse setpattern} bind def
} def
%
%
%End of PostScript Level 2 code
%
/PatternBgnd {
  TransparentPatterns {} {gsave 1 setgray fill grestore} ifelse
} def
%
% Substitute for Level 2 pattern fill codes with
% grayscale if Level 2 support is not selected.
%
/Level1PatternFill {
/Pattern1 {0.250 Density} bind def
/Pattern2 {0.500 Density} bind def
/Pattern3 {0.750 Density} bind def
/Pattern4 {0.125 Density} bind def
/Pattern5 {0.375 Density} bind def
/Pattern6 {0.625 Density} bind def
/Pattern7 {0.875 Density} bind def
} def
%
% Now test for support of Level 2 code
%
Level1 {Level1PatternFill} {Level2PatternFill} ifelse
%
/Symbol-Oblique /Symbol findfont [1 0 .167 1 0 0] makefont
dup length dict begin {1 index /FID eq {pop pop} {def} ifelse} forall
currentdict end definefont pop
%
Level1 SuppressPDFMark or 
{} {
/SDict 10 dict def
systemdict /pdfmark known not {
  userdict /pdfmark systemdict /cleartomark get put
} if
SDict begin [
  /Title (plot_Qcool20c1_su8b47.75.tex)
  /Subject (gnuplot plot)
  /Creator (gnuplot 5.0 patchlevel 3)
  /Author (mteper)
%  /Producer (gnuplot)
%  /Keywords ()
  /CreationDate (Thu Apr 29 13:50:30 2021)
  /DOCINFO pdfmark
end
} ifelse
%
% Support for boxed text - Ethan A Merritt May 2005
%
/InitTextBox { userdict /TBy2 3 -1 roll put userdict /TBx2 3 -1 roll put
           userdict /TBy1 3 -1 roll put userdict /TBx1 3 -1 roll put
	   /Boxing true def } def
/ExtendTextBox { Boxing
    { gsave dup false charpath pathbbox
      dup TBy2 gt {userdict /TBy2 3 -1 roll put} {pop} ifelse
      dup TBx2 gt {userdict /TBx2 3 -1 roll put} {pop} ifelse
      dup TBy1 lt {userdict /TBy1 3 -1 roll put} {pop} ifelse
      dup TBx1 lt {userdict /TBx1 3 -1 roll put} {pop} ifelse
      grestore } if } def
/PopTextBox { newpath TBx1 TBxmargin sub TBy1 TBymargin sub M
               TBx1 TBxmargin sub TBy2 TBymargin add L
	       TBx2 TBxmargin add TBy2 TBymargin add L
	       TBx2 TBxmargin add TBy1 TBymargin sub L closepath } def
/DrawTextBox { PopTextBox stroke /Boxing false def} def
/FillTextBox { gsave PopTextBox 1 1 1 setrgbcolor fill grestore /Boxing false def} def
0 0 0 0 InitTextBox
/TBxmargin 20 def
/TBymargin 20 def
/Boxing false def
/textshow { ExtendTextBox Gshow } def
%
% redundant definitions for compatibility with prologue.ps older than 5.0.2
/LTB {BL [] LCb DL} def
/LTb {PL [] LCb DL} def
end
%%EndProlog
%%Page: 1 1
gnudict begin
gsave
doclip
0 0 translate
0.050 0.050 scale
0 setgray
newpath
BackgroundColor 0 lt 3 1 roll 0 lt exch 0 lt or or not {BackgroundColor C 1.000 0 0 9360.00 7560.00 BoxColFill} if
1.000 UL
LTb
LCb setrgbcolor
1140 640 M
63 0 V
7795 0 R
-63 0 V
stroke
LTb
LCb setrgbcolor
1140 1976 M
63 0 V
7795 0 R
-63 0 V
stroke
LTb
LCb setrgbcolor
1140 3312 M
63 0 V
7795 0 R
-63 0 V
stroke
LTb
LCb setrgbcolor
1140 4647 M
63 0 V
7795 0 R
-63 0 V
stroke
LTb
LCb setrgbcolor
1140 5983 M
63 0 V
7795 0 R
-63 0 V
stroke
LTb
LCb setrgbcolor
1140 7319 M
63 0 V
7795 0 R
-63 0 V
stroke
LTb
LCb setrgbcolor
1140 640 M
0 63 V
0 6616 R
0 -63 V
stroke
LTb
LCb setrgbcolor
2263 640 M
0 63 V
0 6616 R
0 -63 V
stroke
LTb
LCb setrgbcolor
3385 640 M
0 63 V
0 6616 R
0 -63 V
stroke
LTb
LCb setrgbcolor
4508 640 M
0 63 V
0 6616 R
0 -63 V
stroke
LTb
LCb setrgbcolor
5630 640 M
0 63 V
0 6616 R
0 -63 V
stroke
LTb
LCb setrgbcolor
6753 640 M
0 63 V
0 6616 R
0 -63 V
stroke
LTb
LCb setrgbcolor
7875 640 M
0 63 V
0 6616 R
0 -63 V
stroke
LTb
LCb setrgbcolor
8998 640 M
0 63 V
0 6616 R
0 -63 V
stroke
LTb
LCb setrgbcolor
1.000 UL
LTb
LCb setrgbcolor
1140 7319 N
0 -6679 V
7858 0 V
0 6679 V
-7858 0 V
Z stroke
1.000 UP
1.000 UL
LTb
LCb setrgbcolor
LCb setrgbcolor
LTb
LCb setrgbcolor
LTb
1.500 UP
1.000 UL
LTb
0.58 0.00 0.83 C 1140 687 M
0 66 V
112 -76 R
0 60 V
113 -6 R
0 85 V
1477 709 M
0 76 V
112 86 R
0 125 V
112 45 R
0 160 V
113 -37 R
0 181 V
112 -169 R
0 183 V
112 230 R
0 239 V
112 -201 R
0 243 V
113 61 R
0 276 V
112 183 R
0 320 V
112 -38 R
0 344 V
112 -164 R
0 358 V
113 -229 R
0 367 V
112 239 R
0 411 V
112 94 R
0 443 V
112 102 R
0 475 V
113 -955 R
0 447 V
112 -525 R
0 442 V
112 206 R
0 481 V
112 -948 R
0 454 V
113 -752 R
0 435 V
112 330 R
0 481 V
3834 3755 M
0 422 V
112 -486 R
0 417 V
113 -508 R
0 411 V
4171 2904 M
0 361 V
112 -181 R
0 375 V
4395 2429 M
0 323 V
113 -553 R
0 302 V
112 -786 R
0 254 V
112 -279 R
0 251 V
112 -490 R
0 222 V
113 -546 R
0 175 V
5069 967 M
0 147 V
5181 943 M
0 142 V
5294 720 M
0 80 V
112 -23 R
0 100 V
5518 754 M
0 92 V
5630 667 M
0 53 V
113 -43 R
0 60 V
112 -80 R
0 46 V
112 -46 R
0 46 V
337 -63 R
0 27 V
1140 720 Circle
1252 707 Circle
1365 773 Circle
1477 747 Circle
1589 934 Circle
1701 1121 Circle
1814 1254 Circle
1926 1268 Circle
2038 1709 Circle
2150 1749 Circle
2263 2069 Circle
2375 2550 Circle
2487 2844 Circle
2599 3031 Circle
2712 3165 Circle
2824 3792 Circle
2936 4313 Circle
3048 4874 Circle
3161 4380 Circle
3273 4300 Circle
3385 4968 Circle
3497 4487 Circle
3610 4180 Circle
3722 4968 Circle
3834 3966 Circle
3946 3899 Circle
4059 3806 Circle
4171 3084 Circle
4283 3271 Circle
4395 2590 Circle
4508 2350 Circle
4620 1842 Circle
4732 1815 Circle
4844 1562 Circle
4957 1214 Circle
5069 1041 Circle
5181 1014 Circle
5294 760 Circle
5406 827 Circle
5518 800 Circle
5630 693 Circle
5743 707 Circle
5855 680 Circle
5967 680 Circle
6304 653 Circle
1.500 UP
1.000 UL
LTb
0.58 0.00 0.83 C 2038 648 M
0 37 V
112 -28 R
0 46 V
113 -55 R
0 37 V
112 -45 R
0 27 V
112 -10 R
0 46 V
112 -55 R
0 37 V
113 -28 R
0 46 V
112 6 R
0 76 V
112 -42 R
0 88 V
112 -19 R
0 110 V
113 -39 R
0 128 V
112 -80 R
0 139 V
112 -54 R
0 156 V
112 54 R
0 191 V
113 21 R
0 220 V
112 70 R
0 255 V
112 -255 R
0 255 V
112 203 R
0 301 V
113 -135 R
0 316 V
112 -8 R
0 342 V
112 -34 R
0 366 V
112 46 R
0 397 V
113 -500 R
0 389 V
112 361 R
0 439 V
112 -556 R
0 432 V
112 -444 R
0 430 V
113 -508 R
0 426 V
112 -530 R
0 419 V
112 -393 R
0 421 V
113 -744 R
0 399 V
112 -747 R
0 374 V
112 -503 R
0 365 V
112 -570 R
0 348 V
113 -502 R
0 335 V
112 -847 R
0 290 V
112 -379 R
0 281 V
112 -674 R
0 239 V
113 -528 R
0 202 V
112 -251 R
0 194 V
6416 992 M
0 151 V
6528 871 M
0 125 V
6641 800 M
0 107 V
6753 789 M
0 103 V
6865 777 M
0 100 V
6977 667 M
0 53 V
113 -43 R
0 60 V
112 -60 R
0 60 V
112 -70 R
0 53 V
112 -72 R
0 37 V
113 -45 R
0 27 V
336 -27 R
0 27 V
2038 667 BoxF
2150 680 BoxF
2263 667 BoxF
2375 653 BoxF
2487 680 BoxF
2599 667 BoxF
2712 680 BoxF
2824 747 BoxF
2936 787 BoxF
3048 867 BoxF
3161 947 BoxF
3273 1001 BoxF
3385 1094 BoxF
3497 1321 BoxF
3610 1548 BoxF
3722 1855 BoxF
3834 1855 BoxF
3946 2336 BoxF
4059 2510 BoxF
4171 2831 BoxF
4283 3151 BoxF
4395 3579 BoxF
4508 3472 BoxF
4620 4247 BoxF
4732 4126 BoxF
4844 4113 BoxF
4957 4033 BoxF
5069 3926 BoxF
5181 3953 BoxF
5294 3619 BoxF
5406 3258 BoxF
5518 3124 BoxF
5630 2911 BoxF
5743 2750 BoxF
5855 2216 BoxF
5967 2123 BoxF
6079 1709 BoxF
6192 1401 BoxF
6304 1348 BoxF
6416 1067 BoxF
6528 934 BoxF
6641 854 BoxF
6753 840 BoxF
6865 827 BoxF
6977 693 BoxF
7090 707 BoxF
7202 707 BoxF
7314 693 BoxF
7426 667 BoxF
7539 653 BoxF
7875 653 BoxF
1.500 UP
1.000 UL
LTb
0.58 0.00 0.83 C 3834 640 M
0 27 V
112 -27 R
0 27 V
113 -10 R
0 46 V
112 -46 R
0 46 V
112 -55 R
0 37 V
112 35 R
0 80 V
4508 677 M
0 60 V
112 -28 R
0 76 V
112 -8 R
0 100 V
112 66 R
0 142 V
4957 967 M
0 147 V
112 -110 R
0 153 V
112 118 R
0 199 V
113 -298 R
0 183 V
112 29 R
0 214 V
112 -101 R
0 228 V
112 -102 R
0 243 V
113 -218 R
0 246 V
112 84 R
0 281 V
112 -154 R
0 294 V
112 -281 R
0 295 V
113 -52 R
0 317 V
112 -343 R
0 315 V
112 -251 R
0 321 V
112 -487 R
0 306 V
113 -293 R
0 307 V
112 -115 R
0 323 V
112 -426 R
0 315 V
112 -748 R
0 273 V
113 -312 R
0 270 V
112 -434 R
0 252 V
112 -567 R
0 214 V
112 -227 R
0 212 V
113 -423 R
0 181 V
112 -255 R
0 169 V
7763 980 M
0 148 V
7875 992 M
0 151 V
7988 789 M
0 103 V
8100 731 M
0 85 V
112 -96 R
0 80 V
112 -80 R
0 80 V
8437 648 M
0 37 V
112 -8 R
0 60 V
112 -70 R
0 53 V
225 -80 R
0 27 V
112 -27 R
0 27 V
3834 653 Box
3946 653 Box
4059 680 Box
4171 680 Box
4283 667 Box
4395 760 Box
4508 707 Box
4620 747 Box
4732 827 Box
4844 1014 Box
4957 1041 Box
5069 1081 Box
5181 1375 Box
5294 1268 Box
5406 1495 Box
5518 1615 Box
5630 1749 Box
5743 1775 Box
5855 2123 Box
5967 2256 Box
6079 2270 Box
6192 2523 Box
6304 2497 Box
6416 2563 Box
6528 2390 Box
6641 2403 Box
6753 2603 Box
6865 2497 Box
6977 2042 Box
7090 2002 Box
7202 1829 Box
7314 1495 Box
7426 1481 Box
7539 1254 Box
7651 1174 Box
7763 1054 Box
7875 1067 Box
7988 840 Box
8100 773 Box
8212 760 Box
8324 760 Box
8437 667 Box
8549 707 Box
8661 693 Box
8886 653 Box
8998 653 Box
2.000 UL
LTb
LCb setrgbcolor
1.000 UL
LTb
LCb setrgbcolor
1140 7319 N
0 -6679 V
7858 0 V
0 6679 V
-7858 0 V
Z stroke
1.000 UP
1.000 UL
LTb
LCb setrgbcolor
stroke
grestore
end
showpage
  }}%
  \put(5069,140){\makebox(0,0){\large{$Q_L$}}}%
  \put(200,4979){\makebox(0,0){\Large{$N(Q_L)$}}}%
  \put(8998,440){\makebox(0,0){\strut{}\ {$2.5$}}}%
  \put(7875,440){\makebox(0,0){\strut{}\ {$2$}}}%
  \put(6753,440){\makebox(0,0){\strut{}\ {$1.5$}}}%
  \put(5630,440){\makebox(0,0){\strut{}\ {$1$}}}%
  \put(4508,440){\makebox(0,0){\strut{}\ {$0.5$}}}%
  \put(3385,440){\makebox(0,0){\strut{}\ {$0$}}}%
  \put(2263,440){\makebox(0,0){\strut{}\ {$-0.5$}}}%
  \put(1140,440){\makebox(0,0){\strut{}\ {$-1$}}}%
  \put(1020,7319){\makebox(0,0)[r]{\strut{}\ \ {$500$}}}%
  \put(1020,5983){\makebox(0,0)[r]{\strut{}\ \ {$400$}}}%
  \put(1020,4647){\makebox(0,0)[r]{\strut{}\ \ {$300$}}}%
  \put(1020,3312){\makebox(0,0)[r]{\strut{}\ \ {$200$}}}%
  \put(1020,1976){\makebox(0,0)[r]{\strut{}\ \ {$100$}}}%
  \put(1020,640){\makebox(0,0)[r]{\strut{}\ \ {$0$}}}%
\end{picture}%
\endgroup
\endinput

\end	{center}
\caption{The distribution in the topological charge $Q_L$ as obtained after only 1 cooling sweep,
for fields that after 20 cooling sweeps have topological charges $Q=0$ ($\circ$),
$Q=1$ ($\blacksquare$) and $Q=2$ ($\square$).  $N(Q_L)=0$ points suppressed. 
From the same sequence of $SU(8)$ fields generated 
at $\beta=47.75$ plotted in Fig.\ref{fig_Qcool20_su8}.} 
\label{fig_Qcool20c1_su8}
\end{figure}


\begin{figure}[htb]
\begin	{center}
\leavevmode
% GNUPLOT: LaTeX picture with Postscript
\begingroup%
\makeatletter%
\newcommand{\GNUPLOTspecial}{%
  \@sanitize\catcode`\%=14\relax\special}%
\setlength{\unitlength}{0.0500bp}%
\begin{picture}(7200,7560)(0,0)%
  {\GNUPLOTspecial{"
%!PS-Adobe-2.0 EPSF-2.0
%%Title: plot_tauQ_suN.tex
%%Creator: gnuplot 5.0 patchlevel 3
%%CreationDate: Tue May 18 14:50:17 2021
%%DocumentFonts: 
%%BoundingBox: 0 0 360 378
%%EndComments
%%BeginProlog
/gnudict 256 dict def
gnudict begin
%
% The following true/false flags may be edited by hand if desired.
% The unit line width and grayscale image gamma correction may also be changed.
%
/Color true def
/Blacktext true def
/Solid false def
/Dashlength 1 def
/Landscape false def
/Level1 false def
/Level3 false def
/Rounded false def
/ClipToBoundingBox false def
/SuppressPDFMark false def
/TransparentPatterns false def
/gnulinewidth 5.000 def
/userlinewidth gnulinewidth def
/Gamma 1.0 def
/BackgroundColor {-1.000 -1.000 -1.000} def
%
/vshift -66 def
/dl1 {
  10.0 Dashlength userlinewidth gnulinewidth div mul mul mul
  Rounded { currentlinewidth 0.75 mul sub dup 0 le { pop 0.01 } if } if
} def
/dl2 {
  10.0 Dashlength userlinewidth gnulinewidth div mul mul mul
  Rounded { currentlinewidth 0.75 mul add } if
} def
/hpt_ 31.5 def
/vpt_ 31.5 def
/hpt hpt_ def
/vpt vpt_ def
/doclip {
  ClipToBoundingBox {
    newpath 0 0 moveto 360 0 lineto 360 378 lineto 0 378 lineto closepath
    clip
  } if
} def
%
% Gnuplot Prolog Version 5.1 (Oct 2015)
%
%/SuppressPDFMark true def
%
/M {moveto} bind def
/L {lineto} bind def
/R {rmoveto} bind def
/V {rlineto} bind def
/N {newpath moveto} bind def
/Z {closepath} bind def
/C {setrgbcolor} bind def
/f {rlineto fill} bind def
/g {setgray} bind def
/Gshow {show} def   % May be redefined later in the file to support UTF-8
/vpt2 vpt 2 mul def
/hpt2 hpt 2 mul def
/Lshow {currentpoint stroke M 0 vshift R 
	Blacktext {gsave 0 setgray textshow grestore} {textshow} ifelse} def
/Rshow {currentpoint stroke M dup stringwidth pop neg vshift R
	Blacktext {gsave 0 setgray textshow grestore} {textshow} ifelse} def
/Cshow {currentpoint stroke M dup stringwidth pop -2 div vshift R 
	Blacktext {gsave 0 setgray textshow grestore} {textshow} ifelse} def
/UP {dup vpt_ mul /vpt exch def hpt_ mul /hpt exch def
  /hpt2 hpt 2 mul def /vpt2 vpt 2 mul def} def
/DL {Color {setrgbcolor Solid {pop []} if 0 setdash}
 {pop pop pop 0 setgray Solid {pop []} if 0 setdash} ifelse} def
/BL {stroke userlinewidth 2 mul setlinewidth
	Rounded {1 setlinejoin 1 setlinecap} if} def
/AL {stroke userlinewidth 2 div setlinewidth
	Rounded {1 setlinejoin 1 setlinecap} if} def
/UL {dup gnulinewidth mul /userlinewidth exch def
	dup 1 lt {pop 1} if 10 mul /udl exch def} def
/PL {stroke userlinewidth setlinewidth
	Rounded {1 setlinejoin 1 setlinecap} if} def
3.8 setmiterlimit
% Classic Line colors (version 5.0)
/LCw {1 1 1} def
/LCb {0 0 0} def
/LCa {0 0 0} def
/LC0 {1 0 0} def
/LC1 {0 1 0} def
/LC2 {0 0 1} def
/LC3 {1 0 1} def
/LC4 {0 1 1} def
/LC5 {1 1 0} def
/LC6 {0 0 0} def
/LC7 {1 0.3 0} def
/LC8 {0.5 0.5 0.5} def
% Default dash patterns (version 5.0)
/LTB {BL [] LCb DL} def
/LTw {PL [] 1 setgray} def
/LTb {PL [] LCb DL} def
/LTa {AL [1 udl mul 2 udl mul] 0 setdash LCa setrgbcolor} def
/LT0 {PL [] LC0 DL} def
/LT1 {PL [2 dl1 3 dl2] LC1 DL} def
/LT2 {PL [1 dl1 1.5 dl2] LC2 DL} def
/LT3 {PL [6 dl1 2 dl2 1 dl1 2 dl2] LC3 DL} def
/LT4 {PL [1 dl1 2 dl2 6 dl1 2 dl2 1 dl1 2 dl2] LC4 DL} def
/LT5 {PL [4 dl1 2 dl2] LC5 DL} def
/LT6 {PL [1.5 dl1 1.5 dl2 1.5 dl1 1.5 dl2 1.5 dl1 6 dl2] LC6 DL} def
/LT7 {PL [3 dl1 3 dl2 1 dl1 3 dl2] LC7 DL} def
/LT8 {PL [2 dl1 2 dl2 2 dl1 6 dl2] LC8 DL} def
/SL {[] 0 setdash} def
/Pnt {stroke [] 0 setdash gsave 1 setlinecap M 0 0 V stroke grestore} def
/Dia {stroke [] 0 setdash 2 copy vpt add M
  hpt neg vpt neg V hpt vpt neg V
  hpt vpt V hpt neg vpt V closepath stroke
  Pnt} def
/Pls {stroke [] 0 setdash vpt sub M 0 vpt2 V
  currentpoint stroke M
  hpt neg vpt neg R hpt2 0 V stroke
 } def
/Box {stroke [] 0 setdash 2 copy exch hpt sub exch vpt add M
  0 vpt2 neg V hpt2 0 V 0 vpt2 V
  hpt2 neg 0 V closepath stroke
  Pnt} def
/Crs {stroke [] 0 setdash exch hpt sub exch vpt add M
  hpt2 vpt2 neg V currentpoint stroke M
  hpt2 neg 0 R hpt2 vpt2 V stroke} def
/TriU {stroke [] 0 setdash 2 copy vpt 1.12 mul add M
  hpt neg vpt -1.62 mul V
  hpt 2 mul 0 V
  hpt neg vpt 1.62 mul V closepath stroke
  Pnt} def
/Star {2 copy Pls Crs} def
/BoxF {stroke [] 0 setdash exch hpt sub exch vpt add M
  0 vpt2 neg V hpt2 0 V 0 vpt2 V
  hpt2 neg 0 V closepath fill} def
/TriUF {stroke [] 0 setdash vpt 1.12 mul add M
  hpt neg vpt -1.62 mul V
  hpt 2 mul 0 V
  hpt neg vpt 1.62 mul V closepath fill} def
/TriD {stroke [] 0 setdash 2 copy vpt 1.12 mul sub M
  hpt neg vpt 1.62 mul V
  hpt 2 mul 0 V
  hpt neg vpt -1.62 mul V closepath stroke
  Pnt} def
/TriDF {stroke [] 0 setdash vpt 1.12 mul sub M
  hpt neg vpt 1.62 mul V
  hpt 2 mul 0 V
  hpt neg vpt -1.62 mul V closepath fill} def
/DiaF {stroke [] 0 setdash vpt add M
  hpt neg vpt neg V hpt vpt neg V
  hpt vpt V hpt neg vpt V closepath fill} def
/Pent {stroke [] 0 setdash 2 copy gsave
  translate 0 hpt M 4 {72 rotate 0 hpt L} repeat
  closepath stroke grestore Pnt} def
/PentF {stroke [] 0 setdash gsave
  translate 0 hpt M 4 {72 rotate 0 hpt L} repeat
  closepath fill grestore} def
/Circle {stroke [] 0 setdash 2 copy
  hpt 0 360 arc stroke Pnt} def
/CircleF {stroke [] 0 setdash hpt 0 360 arc fill} def
/C0 {BL [] 0 setdash 2 copy moveto vpt 90 450 arc} bind def
/C1 {BL [] 0 setdash 2 copy moveto
	2 copy vpt 0 90 arc closepath fill
	vpt 0 360 arc closepath} bind def
/C2 {BL [] 0 setdash 2 copy moveto
	2 copy vpt 90 180 arc closepath fill
	vpt 0 360 arc closepath} bind def
/C3 {BL [] 0 setdash 2 copy moveto
	2 copy vpt 0 180 arc closepath fill
	vpt 0 360 arc closepath} bind def
/C4 {BL [] 0 setdash 2 copy moveto
	2 copy vpt 180 270 arc closepath fill
	vpt 0 360 arc closepath} bind def
/C5 {BL [] 0 setdash 2 copy moveto
	2 copy vpt 0 90 arc
	2 copy moveto
	2 copy vpt 180 270 arc closepath fill
	vpt 0 360 arc} bind def
/C6 {BL [] 0 setdash 2 copy moveto
	2 copy vpt 90 270 arc closepath fill
	vpt 0 360 arc closepath} bind def
/C7 {BL [] 0 setdash 2 copy moveto
	2 copy vpt 0 270 arc closepath fill
	vpt 0 360 arc closepath} bind def
/C8 {BL [] 0 setdash 2 copy moveto
	2 copy vpt 270 360 arc closepath fill
	vpt 0 360 arc closepath} bind def
/C9 {BL [] 0 setdash 2 copy moveto
	2 copy vpt 270 450 arc closepath fill
	vpt 0 360 arc closepath} bind def
/C10 {BL [] 0 setdash 2 copy 2 copy moveto vpt 270 360 arc closepath fill
	2 copy moveto
	2 copy vpt 90 180 arc closepath fill
	vpt 0 360 arc closepath} bind def
/C11 {BL [] 0 setdash 2 copy moveto
	2 copy vpt 0 180 arc closepath fill
	2 copy moveto
	2 copy vpt 270 360 arc closepath fill
	vpt 0 360 arc closepath} bind def
/C12 {BL [] 0 setdash 2 copy moveto
	2 copy vpt 180 360 arc closepath fill
	vpt 0 360 arc closepath} bind def
/C13 {BL [] 0 setdash 2 copy moveto
	2 copy vpt 0 90 arc closepath fill
	2 copy moveto
	2 copy vpt 180 360 arc closepath fill
	vpt 0 360 arc closepath} bind def
/C14 {BL [] 0 setdash 2 copy moveto
	2 copy vpt 90 360 arc closepath fill
	vpt 0 360 arc} bind def
/C15 {BL [] 0 setdash 2 copy vpt 0 360 arc closepath fill
	vpt 0 360 arc closepath} bind def
/Rec {newpath 4 2 roll moveto 1 index 0 rlineto 0 exch rlineto
	neg 0 rlineto closepath} bind def
/Square {dup Rec} bind def
/Bsquare {vpt sub exch vpt sub exch vpt2 Square} bind def
/S0 {BL [] 0 setdash 2 copy moveto 0 vpt rlineto BL Bsquare} bind def
/S1 {BL [] 0 setdash 2 copy vpt Square fill Bsquare} bind def
/S2 {BL [] 0 setdash 2 copy exch vpt sub exch vpt Square fill Bsquare} bind def
/S3 {BL [] 0 setdash 2 copy exch vpt sub exch vpt2 vpt Rec fill Bsquare} bind def
/S4 {BL [] 0 setdash 2 copy exch vpt sub exch vpt sub vpt Square fill Bsquare} bind def
/S5 {BL [] 0 setdash 2 copy 2 copy vpt Square fill
	exch vpt sub exch vpt sub vpt Square fill Bsquare} bind def
/S6 {BL [] 0 setdash 2 copy exch vpt sub exch vpt sub vpt vpt2 Rec fill Bsquare} bind def
/S7 {BL [] 0 setdash 2 copy exch vpt sub exch vpt sub vpt vpt2 Rec fill
	2 copy vpt Square fill Bsquare} bind def
/S8 {BL [] 0 setdash 2 copy vpt sub vpt Square fill Bsquare} bind def
/S9 {BL [] 0 setdash 2 copy vpt sub vpt vpt2 Rec fill Bsquare} bind def
/S10 {BL [] 0 setdash 2 copy vpt sub vpt Square fill 2 copy exch vpt sub exch vpt Square fill
	Bsquare} bind def
/S11 {BL [] 0 setdash 2 copy vpt sub vpt Square fill 2 copy exch vpt sub exch vpt2 vpt Rec fill
	Bsquare} bind def
/S12 {BL [] 0 setdash 2 copy exch vpt sub exch vpt sub vpt2 vpt Rec fill Bsquare} bind def
/S13 {BL [] 0 setdash 2 copy exch vpt sub exch vpt sub vpt2 vpt Rec fill
	2 copy vpt Square fill Bsquare} bind def
/S14 {BL [] 0 setdash 2 copy exch vpt sub exch vpt sub vpt2 vpt Rec fill
	2 copy exch vpt sub exch vpt Square fill Bsquare} bind def
/S15 {BL [] 0 setdash 2 copy Bsquare fill Bsquare} bind def
/D0 {gsave translate 45 rotate 0 0 S0 stroke grestore} bind def
/D1 {gsave translate 45 rotate 0 0 S1 stroke grestore} bind def
/D2 {gsave translate 45 rotate 0 0 S2 stroke grestore} bind def
/D3 {gsave translate 45 rotate 0 0 S3 stroke grestore} bind def
/D4 {gsave translate 45 rotate 0 0 S4 stroke grestore} bind def
/D5 {gsave translate 45 rotate 0 0 S5 stroke grestore} bind def
/D6 {gsave translate 45 rotate 0 0 S6 stroke grestore} bind def
/D7 {gsave translate 45 rotate 0 0 S7 stroke grestore} bind def
/D8 {gsave translate 45 rotate 0 0 S8 stroke grestore} bind def
/D9 {gsave translate 45 rotate 0 0 S9 stroke grestore} bind def
/D10 {gsave translate 45 rotate 0 0 S10 stroke grestore} bind def
/D11 {gsave translate 45 rotate 0 0 S11 stroke grestore} bind def
/D12 {gsave translate 45 rotate 0 0 S12 stroke grestore} bind def
/D13 {gsave translate 45 rotate 0 0 S13 stroke grestore} bind def
/D14 {gsave translate 45 rotate 0 0 S14 stroke grestore} bind def
/D15 {gsave translate 45 rotate 0 0 S15 stroke grestore} bind def
/DiaE {stroke [] 0 setdash vpt add M
  hpt neg vpt neg V hpt vpt neg V
  hpt vpt V hpt neg vpt V closepath stroke} def
/BoxE {stroke [] 0 setdash exch hpt sub exch vpt add M
  0 vpt2 neg V hpt2 0 V 0 vpt2 V
  hpt2 neg 0 V closepath stroke} def
/TriUE {stroke [] 0 setdash vpt 1.12 mul add M
  hpt neg vpt -1.62 mul V
  hpt 2 mul 0 V
  hpt neg vpt 1.62 mul V closepath stroke} def
/TriDE {stroke [] 0 setdash vpt 1.12 mul sub M
  hpt neg vpt 1.62 mul V
  hpt 2 mul 0 V
  hpt neg vpt -1.62 mul V closepath stroke} def
/PentE {stroke [] 0 setdash gsave
  translate 0 hpt M 4 {72 rotate 0 hpt L} repeat
  closepath stroke grestore} def
/CircE {stroke [] 0 setdash 
  hpt 0 360 arc stroke} def
/Opaque {gsave closepath 1 setgray fill grestore 0 setgray closepath} def
/DiaW {stroke [] 0 setdash vpt add M
  hpt neg vpt neg V hpt vpt neg V
  hpt vpt V hpt neg vpt V Opaque stroke} def
/BoxW {stroke [] 0 setdash exch hpt sub exch vpt add M
  0 vpt2 neg V hpt2 0 V 0 vpt2 V
  hpt2 neg 0 V Opaque stroke} def
/TriUW {stroke [] 0 setdash vpt 1.12 mul add M
  hpt neg vpt -1.62 mul V
  hpt 2 mul 0 V
  hpt neg vpt 1.62 mul V Opaque stroke} def
/TriDW {stroke [] 0 setdash vpt 1.12 mul sub M
  hpt neg vpt 1.62 mul V
  hpt 2 mul 0 V
  hpt neg vpt -1.62 mul V Opaque stroke} def
/PentW {stroke [] 0 setdash gsave
  translate 0 hpt M 4 {72 rotate 0 hpt L} repeat
  Opaque stroke grestore} def
/CircW {stroke [] 0 setdash 
  hpt 0 360 arc Opaque stroke} def
/BoxFill {gsave Rec 1 setgray fill grestore} def
/Density {
  /Fillden exch def
  currentrgbcolor
  /ColB exch def /ColG exch def /ColR exch def
  /ColR ColR Fillden mul Fillden sub 1 add def
  /ColG ColG Fillden mul Fillden sub 1 add def
  /ColB ColB Fillden mul Fillden sub 1 add def
  ColR ColG ColB setrgbcolor} def
/BoxColFill {gsave Rec PolyFill} def
/PolyFill {gsave Density fill grestore grestore} def
/h {rlineto rlineto rlineto gsave closepath fill grestore} bind def
%
% PostScript Level 1 Pattern Fill routine for rectangles
% Usage: x y w h s a XX PatternFill
%	x,y = lower left corner of box to be filled
%	w,h = width and height of box
%	  a = angle in degrees between lines and x-axis
%	 XX = 0/1 for no/yes cross-hatch
%
/PatternFill {gsave /PFa [ 9 2 roll ] def
  PFa 0 get PFa 2 get 2 div add PFa 1 get PFa 3 get 2 div add translate
  PFa 2 get -2 div PFa 3 get -2 div PFa 2 get PFa 3 get Rec
  TransparentPatterns {} {gsave 1 setgray fill grestore} ifelse
  clip
  currentlinewidth 0.5 mul setlinewidth
  /PFs PFa 2 get dup mul PFa 3 get dup mul add sqrt def
  0 0 M PFa 5 get rotate PFs -2 div dup translate
  0 1 PFs PFa 4 get div 1 add floor cvi
	{PFa 4 get mul 0 M 0 PFs V} for
  0 PFa 6 get ne {
	0 1 PFs PFa 4 get div 1 add floor cvi
	{PFa 4 get mul 0 2 1 roll M PFs 0 V} for
 } if
  stroke grestore} def
%
/languagelevel where
 {pop languagelevel} {1} ifelse
dup 2 lt
	{/InterpretLevel1 true def
	 /InterpretLevel3 false def}
	{/InterpretLevel1 Level1 def
	 2 gt
	    {/InterpretLevel3 Level3 def}
	    {/InterpretLevel3 false def}
	 ifelse }
 ifelse
%
% PostScript level 2 pattern fill definitions
%
/Level2PatternFill {
/Tile8x8 {/PaintType 2 /PatternType 1 /TilingType 1 /BBox [0 0 8 8] /XStep 8 /YStep 8}
	bind def
/KeepColor {currentrgbcolor [/Pattern /DeviceRGB] setcolorspace} bind def
<< Tile8x8
 /PaintProc {0.5 setlinewidth pop 0 0 M 8 8 L 0 8 M 8 0 L stroke} 
>> matrix makepattern
/Pat1 exch def
<< Tile8x8
 /PaintProc {0.5 setlinewidth pop 0 0 M 8 8 L 0 8 M 8 0 L stroke
	0 4 M 4 8 L 8 4 L 4 0 L 0 4 L stroke}
>> matrix makepattern
/Pat2 exch def
<< Tile8x8
 /PaintProc {0.5 setlinewidth pop 0 0 M 0 8 L
	8 8 L 8 0 L 0 0 L fill}
>> matrix makepattern
/Pat3 exch def
<< Tile8x8
 /PaintProc {0.5 setlinewidth pop -4 8 M 8 -4 L
	0 12 M 12 0 L stroke}
>> matrix makepattern
/Pat4 exch def
<< Tile8x8
 /PaintProc {0.5 setlinewidth pop -4 0 M 8 12 L
	0 -4 M 12 8 L stroke}
>> matrix makepattern
/Pat5 exch def
<< Tile8x8
 /PaintProc {0.5 setlinewidth pop -2 8 M 4 -4 L
	0 12 M 8 -4 L 4 12 M 10 0 L stroke}
>> matrix makepattern
/Pat6 exch def
<< Tile8x8
 /PaintProc {0.5 setlinewidth pop -2 0 M 4 12 L
	0 -4 M 8 12 L 4 -4 M 10 8 L stroke}
>> matrix makepattern
/Pat7 exch def
<< Tile8x8
 /PaintProc {0.5 setlinewidth pop 8 -2 M -4 4 L
	12 0 M -4 8 L 12 4 M 0 10 L stroke}
>> matrix makepattern
/Pat8 exch def
<< Tile8x8
 /PaintProc {0.5 setlinewidth pop 0 -2 M 12 4 L
	-4 0 M 12 8 L -4 4 M 8 10 L stroke}
>> matrix makepattern
/Pat9 exch def
/Pattern1 {PatternBgnd KeepColor Pat1 setpattern} bind def
/Pattern2 {PatternBgnd KeepColor Pat2 setpattern} bind def
/Pattern3 {PatternBgnd KeepColor Pat3 setpattern} bind def
/Pattern4 {PatternBgnd KeepColor Landscape {Pat5} {Pat4} ifelse setpattern} bind def
/Pattern5 {PatternBgnd KeepColor Landscape {Pat4} {Pat5} ifelse setpattern} bind def
/Pattern6 {PatternBgnd KeepColor Landscape {Pat9} {Pat6} ifelse setpattern} bind def
/Pattern7 {PatternBgnd KeepColor Landscape {Pat8} {Pat7} ifelse setpattern} bind def
} def
%
%
%End of PostScript Level 2 code
%
/PatternBgnd {
  TransparentPatterns {} {gsave 1 setgray fill grestore} ifelse
} def
%
% Substitute for Level 2 pattern fill codes with
% grayscale if Level 2 support is not selected.
%
/Level1PatternFill {
/Pattern1 {0.250 Density} bind def
/Pattern2 {0.500 Density} bind def
/Pattern3 {0.750 Density} bind def
/Pattern4 {0.125 Density} bind def
/Pattern5 {0.375 Density} bind def
/Pattern6 {0.625 Density} bind def
/Pattern7 {0.875 Density} bind def
} def
%
% Now test for support of Level 2 code
%
Level1 {Level1PatternFill} {Level2PatternFill} ifelse
%
/Symbol-Oblique /Symbol findfont [1 0 .167 1 0 0] makefont
dup length dict begin {1 index /FID eq {pop pop} {def} ifelse} forall
currentdict end definefont pop
%
Level1 SuppressPDFMark or 
{} {
/SDict 10 dict def
systemdict /pdfmark known not {
  userdict /pdfmark systemdict /cleartomark get put
} if
SDict begin [
  /Title (plot_tauQ_suN.tex)
  /Subject (gnuplot plot)
  /Creator (gnuplot 5.0 patchlevel 3)
  /Author (mteper)
%  /Producer (gnuplot)
%  /Keywords ()
  /CreationDate (Tue May 18 14:50:17 2021)
  /DOCINFO pdfmark
end
} ifelse
%
% Support for boxed text - Ethan A Merritt May 2005
%
/InitTextBox { userdict /TBy2 3 -1 roll put userdict /TBx2 3 -1 roll put
           userdict /TBy1 3 -1 roll put userdict /TBx1 3 -1 roll put
	   /Boxing true def } def
/ExtendTextBox { Boxing
    { gsave dup false charpath pathbbox
      dup TBy2 gt {userdict /TBy2 3 -1 roll put} {pop} ifelse
      dup TBx2 gt {userdict /TBx2 3 -1 roll put} {pop} ifelse
      dup TBy1 lt {userdict /TBy1 3 -1 roll put} {pop} ifelse
      dup TBx1 lt {userdict /TBx1 3 -1 roll put} {pop} ifelse
      grestore } if } def
/PopTextBox { newpath TBx1 TBxmargin sub TBy1 TBymargin sub M
               TBx1 TBxmargin sub TBy2 TBymargin add L
	       TBx2 TBxmargin add TBy2 TBymargin add L
	       TBx2 TBxmargin add TBy1 TBymargin sub L closepath } def
/DrawTextBox { PopTextBox stroke /Boxing false def} def
/FillTextBox { gsave PopTextBox 1 1 1 setrgbcolor fill grestore /Boxing false def} def
0 0 0 0 InitTextBox
/TBxmargin 20 def
/TBymargin 20 def
/Boxing false def
/textshow { ExtendTextBox Gshow } def
%
% redundant definitions for compatibility with prologue.ps older than 5.0.2
/LTB {BL [] LCb DL} def
/LTb {PL [] LCb DL} def
end
%%EndProlog
%%Page: 1 1
gnudict begin
gsave
doclip
0 0 translate
0.050 0.050 scale
0 setgray
newpath
BackgroundColor 0 lt 3 1 roll 0 lt exch 0 lt or or not {BackgroundColor C 1.000 0 0 7200.00 7560.00 BoxColFill} if
1.000 UL
LTb
LCb setrgbcolor
1020 640 M
63 0 V
5756 0 R
-63 0 V
stroke
LTb
LCb setrgbcolor
1020 1475 M
63 0 V
5756 0 R
-63 0 V
stroke
LTb
LCb setrgbcolor
1020 2310 M
63 0 V
5756 0 R
-63 0 V
stroke
LTb
LCb setrgbcolor
1020 3145 M
63 0 V
5756 0 R
-63 0 V
stroke
LTb
LCb setrgbcolor
1020 3980 M
63 0 V
5756 0 R
-63 0 V
stroke
LTb
LCb setrgbcolor
1020 4814 M
63 0 V
5756 0 R
-63 0 V
stroke
LTb
LCb setrgbcolor
1020 5649 M
63 0 V
5756 0 R
-63 0 V
stroke
LTb
LCb setrgbcolor
1020 6484 M
63 0 V
5756 0 R
-63 0 V
stroke
LTb
LCb setrgbcolor
1020 7319 M
63 0 V
5756 0 R
-63 0 V
stroke
LTb
LCb setrgbcolor
1020 640 M
0 63 V
0 6616 R
0 -63 V
stroke
LTb
LCb setrgbcolor
1436 640 M
0 63 V
0 6616 R
0 -63 V
stroke
LTb
LCb setrgbcolor
1851 640 M
0 63 V
0 6616 R
0 -63 V
stroke
LTb
LCb setrgbcolor
2267 640 M
0 63 V
0 6616 R
0 -63 V
stroke
LTb
LCb setrgbcolor
2683 640 M
0 63 V
0 6616 R
0 -63 V
stroke
LTb
LCb setrgbcolor
3098 640 M
0 63 V
0 6616 R
0 -63 V
stroke
LTb
LCb setrgbcolor
3514 640 M
0 63 V
0 6616 R
0 -63 V
stroke
LTb
LCb setrgbcolor
3930 640 M
0 63 V
0 6616 R
0 -63 V
stroke
LTb
LCb setrgbcolor
4345 640 M
0 63 V
0 6616 R
0 -63 V
stroke
LTb
LCb setrgbcolor
4761 640 M
0 63 V
0 6616 R
0 -63 V
stroke
LTb
LCb setrgbcolor
5176 640 M
0 63 V
0 6616 R
0 -63 V
stroke
LTb
LCb setrgbcolor
5592 640 M
0 63 V
0 6616 R
0 -63 V
stroke
LTb
LCb setrgbcolor
6008 640 M
0 63 V
0 6616 R
0 -63 V
stroke
LTb
LCb setrgbcolor
6423 640 M
0 63 V
0 6616 R
0 -63 V
stroke
LTb
LCb setrgbcolor
6839 640 M
0 63 V
0 6616 R
0 -63 V
stroke
LTb
LCb setrgbcolor
1.000 UL
LTb
LCb setrgbcolor
1020 7319 N
0 -6679 V
5819 0 V
0 6679 V
-5819 0 V
Z stroke
1.000 UP
1.000 UL
LTb
LCb setrgbcolor
LCb setrgbcolor
LTb
LCb setrgbcolor
LTb
1.500 UP
1.000 UL
LTb
0.58 0.00 0.83 C 2267 2602 M
0 15 V
416 1242 R
0 16 V
415 1116 R
0 79 V
416 1267 R
0 439 V
2267 2610 CircleF
2683 3867 CircleF
3098 5030 CircleF
3514 6556 CircleF
1.500 UL
LTb
0.58 0.00 0.83 C 1611 640 M
56 168 V
58 177 V
59 177 V
59 176 V
59 177 V
58 176 V
59 177 V
59 177 V
59 176 V
59 177 V
58 177 V
59 176 V
59 177 V
59 176 V
58 177 V
59 177 V
59 176 V
59 177 V
59 177 V
58 176 V
59 177 V
59 176 V
59 177 V
58 177 V
59 176 V
59 177 V
59 177 V
59 176 V
58 177 V
59 176 V
59 177 V
59 177 V
58 176 V
59 177 V
59 176 V
59 177 V
59 177 V
50 152 V
1.500 UP
stroke
1.000 UL
LTb
0.58 0.00 0.83 C 4345 3249 M
0 16 V
831 898 R
0 36 V
832 896 R
0 94 V
4345 3257 Circle
5176 4181 Circle
6008 5142 Circle
1.500 UL
LTb
0.58 0.00 0.83 C 2010 640 M
9 10 V
59 66 V
59 66 V
59 66 V
58 66 V
59 65 V
59 66 V
59 66 V
58 66 V
59 66 V
59 66 V
59 66 V
59 65 V
58 66 V
59 66 V
59 66 V
59 66 V
58 66 V
59 66 V
59 65 V
59 66 V
59 66 V
58 66 V
59 66 V
59 66 V
59 66 V
58 65 V
59 66 V
59 66 V
59 66 V
59 66 V
58 66 V
59 66 V
59 65 V
59 66 V
58 66 V
59 66 V
59 66 V
59 66 V
59 66 V
58 65 V
59 66 V
59 66 V
59 66 V
58 66 V
59 66 V
59 66 V
59 65 V
59 66 V
58 66 V
59 66 V
59 66 V
59 66 V
58 66 V
59 65 V
59 66 V
59 66 V
59 66 V
58 66 V
59 66 V
59 66 V
59 65 V
58 66 V
59 66 V
59 66 V
59 66 V
59 66 V
58 65 V
59 66 V
59 66 V
59 66 V
58 66 V
59 66 V
59 66 V
59 65 V
59 66 V
58 66 V
59 66 V
59 66 V
59 66 V
58 66 V
59 65 V
59 66 V
stroke
2.000 UL
LTb
LCb setrgbcolor
1.000 UL
LTb
LCb setrgbcolor
1020 7319 N
0 -6679 V
5819 0 V
0 6679 V
-5819 0 V
Z stroke
1.000 UP
1.000 UL
LTb
LCb setrgbcolor
stroke
grestore
end
showpage
  }}%
  \put(3929,140){\makebox(0,0){\large{$N$}}}%
  \put(200,4979){\makebox(0,0){\Large{$\ln\{\Tilde{\tau}_Q\}$}}}%
  \put(6839,440){\makebox(0,0){\strut{}\ {$14$}}}%
  \put(6423,440){\makebox(0,0){\strut{}\ {$13$}}}%
  \put(6008,440){\makebox(0,0){\strut{}\ {$12$}}}%
  \put(5592,440){\makebox(0,0){\strut{}\ {$11$}}}%
  \put(5176,440){\makebox(0,0){\strut{}\ {$10$}}}%
  \put(4761,440){\makebox(0,0){\strut{}\ {$9$}}}%
  \put(4345,440){\makebox(0,0){\strut{}\ {$8$}}}%
  \put(3930,440){\makebox(0,0){\strut{}\ {$7$}}}%
  \put(3514,440){\makebox(0,0){\strut{}\ {$6$}}}%
  \put(3098,440){\makebox(0,0){\strut{}\ {$5$}}}%
  \put(2683,440){\makebox(0,0){\strut{}\ {$4$}}}%
  \put(2267,440){\makebox(0,0){\strut{}\ {$3$}}}%
  \put(1851,440){\makebox(0,0){\strut{}\ {$2$}}}%
  \put(1436,440){\makebox(0,0){\strut{}\ {$1$}}}%
  \put(1020,440){\makebox(0,0){\strut{}\ {$0$}}}%
  \put(900,7319){\makebox(0,0)[r]{\strut{}\ \ {$16$}}}%
  \put(900,6484){\makebox(0,0)[r]{\strut{}\ \ {$14$}}}%
  \put(900,5649){\makebox(0,0)[r]{\strut{}\ \ {$12$}}}%
  \put(900,4814){\makebox(0,0)[r]{\strut{}\ \ {$10$}}}%
  \put(900,3980){\makebox(0,0)[r]{\strut{}\ \ {$8$}}}%
  \put(900,3145){\makebox(0,0)[r]{\strut{}\ \ {$6$}}}%
  \put(900,2310){\makebox(0,0)[r]{\strut{}\ \ {$4$}}}%
  \put(900,1475){\makebox(0,0)[r]{\strut{}\ \ {$2$}}}%
  \put(900,640){\makebox(0,0)[r]{\strut{}\ \ {$0$}}}%
\end{picture}%
\endgroup
\endinput

\end	{center}
\caption{Correlation length $\Tilde{\tau}_Q$, the average number of sweeps between changes of $Q$ by $\pm 1$,
  measured for the $SU(N)$ lattice topological charge
  for $a\surd\sigma \sim 0.15$ ($\bullet$) and for $a\surd\sigma \sim 0.33$ ($\circ$).
  Lines are fits $\Tilde{\tau}_Q = b\exp\{cN\}$.}
\label{fig_tauQ_suN}
\end{figure}


\begin{figure}[htb]
\begin	{center}
\leavevmode
% GNUPLOT: LaTeX picture with Postscript
\begingroup%
\makeatletter%
\newcommand{\GNUPLOTspecial}{%
  \@sanitize\catcode`\%=14\relax\special}%
\setlength{\unitlength}{0.0500bp}%
\begin{picture}(7200,7560)(0,0)%
  {\GNUPLOTspecial{"
%!PS-Adobe-2.0 EPSF-2.0
%%Title: plot_tauQ_KsuN.tex
%%Creator: gnuplot 5.0 patchlevel 3
%%CreationDate: Tue May 18 14:50:52 2021
%%DocumentFonts: 
%%BoundingBox: 0 0 360 378
%%EndComments
%%BeginProlog
/gnudict 256 dict def
gnudict begin
%
% The following true/false flags may be edited by hand if desired.
% The unit line width and grayscale image gamma correction may also be changed.
%
/Color true def
/Blacktext true def
/Solid false def
/Dashlength 1 def
/Landscape false def
/Level1 false def
/Level3 false def
/Rounded false def
/ClipToBoundingBox false def
/SuppressPDFMark false def
/TransparentPatterns false def
/gnulinewidth 5.000 def
/userlinewidth gnulinewidth def
/Gamma 1.0 def
/BackgroundColor {-1.000 -1.000 -1.000} def
%
/vshift -66 def
/dl1 {
  10.0 Dashlength userlinewidth gnulinewidth div mul mul mul
  Rounded { currentlinewidth 0.75 mul sub dup 0 le { pop 0.01 } if } if
} def
/dl2 {
  10.0 Dashlength userlinewidth gnulinewidth div mul mul mul
  Rounded { currentlinewidth 0.75 mul add } if
} def
/hpt_ 31.5 def
/vpt_ 31.5 def
/hpt hpt_ def
/vpt vpt_ def
/doclip {
  ClipToBoundingBox {
    newpath 0 0 moveto 360 0 lineto 360 378 lineto 0 378 lineto closepath
    clip
  } if
} def
%
% Gnuplot Prolog Version 5.1 (Oct 2015)
%
%/SuppressPDFMark true def
%
/M {moveto} bind def
/L {lineto} bind def
/R {rmoveto} bind def
/V {rlineto} bind def
/N {newpath moveto} bind def
/Z {closepath} bind def
/C {setrgbcolor} bind def
/f {rlineto fill} bind def
/g {setgray} bind def
/Gshow {show} def   % May be redefined later in the file to support UTF-8
/vpt2 vpt 2 mul def
/hpt2 hpt 2 mul def
/Lshow {currentpoint stroke M 0 vshift R 
	Blacktext {gsave 0 setgray textshow grestore} {textshow} ifelse} def
/Rshow {currentpoint stroke M dup stringwidth pop neg vshift R
	Blacktext {gsave 0 setgray textshow grestore} {textshow} ifelse} def
/Cshow {currentpoint stroke M dup stringwidth pop -2 div vshift R 
	Blacktext {gsave 0 setgray textshow grestore} {textshow} ifelse} def
/UP {dup vpt_ mul /vpt exch def hpt_ mul /hpt exch def
  /hpt2 hpt 2 mul def /vpt2 vpt 2 mul def} def
/DL {Color {setrgbcolor Solid {pop []} if 0 setdash}
 {pop pop pop 0 setgray Solid {pop []} if 0 setdash} ifelse} def
/BL {stroke userlinewidth 2 mul setlinewidth
	Rounded {1 setlinejoin 1 setlinecap} if} def
/AL {stroke userlinewidth 2 div setlinewidth
	Rounded {1 setlinejoin 1 setlinecap} if} def
/UL {dup gnulinewidth mul /userlinewidth exch def
	dup 1 lt {pop 1} if 10 mul /udl exch def} def
/PL {stroke userlinewidth setlinewidth
	Rounded {1 setlinejoin 1 setlinecap} if} def
3.8 setmiterlimit
% Classic Line colors (version 5.0)
/LCw {1 1 1} def
/LCb {0 0 0} def
/LCa {0 0 0} def
/LC0 {1 0 0} def
/LC1 {0 1 0} def
/LC2 {0 0 1} def
/LC3 {1 0 1} def
/LC4 {0 1 1} def
/LC5 {1 1 0} def
/LC6 {0 0 0} def
/LC7 {1 0.3 0} def
/LC8 {0.5 0.5 0.5} def
% Default dash patterns (version 5.0)
/LTB {BL [] LCb DL} def
/LTw {PL [] 1 setgray} def
/LTb {PL [] LCb DL} def
/LTa {AL [1 udl mul 2 udl mul] 0 setdash LCa setrgbcolor} def
/LT0 {PL [] LC0 DL} def
/LT1 {PL [2 dl1 3 dl2] LC1 DL} def
/LT2 {PL [1 dl1 1.5 dl2] LC2 DL} def
/LT3 {PL [6 dl1 2 dl2 1 dl1 2 dl2] LC3 DL} def
/LT4 {PL [1 dl1 2 dl2 6 dl1 2 dl2 1 dl1 2 dl2] LC4 DL} def
/LT5 {PL [4 dl1 2 dl2] LC5 DL} def
/LT6 {PL [1.5 dl1 1.5 dl2 1.5 dl1 1.5 dl2 1.5 dl1 6 dl2] LC6 DL} def
/LT7 {PL [3 dl1 3 dl2 1 dl1 3 dl2] LC7 DL} def
/LT8 {PL [2 dl1 2 dl2 2 dl1 6 dl2] LC8 DL} def
/SL {[] 0 setdash} def
/Pnt {stroke [] 0 setdash gsave 1 setlinecap M 0 0 V stroke grestore} def
/Dia {stroke [] 0 setdash 2 copy vpt add M
  hpt neg vpt neg V hpt vpt neg V
  hpt vpt V hpt neg vpt V closepath stroke
  Pnt} def
/Pls {stroke [] 0 setdash vpt sub M 0 vpt2 V
  currentpoint stroke M
  hpt neg vpt neg R hpt2 0 V stroke
 } def
/Box {stroke [] 0 setdash 2 copy exch hpt sub exch vpt add M
  0 vpt2 neg V hpt2 0 V 0 vpt2 V
  hpt2 neg 0 V closepath stroke
  Pnt} def
/Crs {stroke [] 0 setdash exch hpt sub exch vpt add M
  hpt2 vpt2 neg V currentpoint stroke M
  hpt2 neg 0 R hpt2 vpt2 V stroke} def
/TriU {stroke [] 0 setdash 2 copy vpt 1.12 mul add M
  hpt neg vpt -1.62 mul V
  hpt 2 mul 0 V
  hpt neg vpt 1.62 mul V closepath stroke
  Pnt} def
/Star {2 copy Pls Crs} def
/BoxF {stroke [] 0 setdash exch hpt sub exch vpt add M
  0 vpt2 neg V hpt2 0 V 0 vpt2 V
  hpt2 neg 0 V closepath fill} def
/TriUF {stroke [] 0 setdash vpt 1.12 mul add M
  hpt neg vpt -1.62 mul V
  hpt 2 mul 0 V
  hpt neg vpt 1.62 mul V closepath fill} def
/TriD {stroke [] 0 setdash 2 copy vpt 1.12 mul sub M
  hpt neg vpt 1.62 mul V
  hpt 2 mul 0 V
  hpt neg vpt -1.62 mul V closepath stroke
  Pnt} def
/TriDF {stroke [] 0 setdash vpt 1.12 mul sub M
  hpt neg vpt 1.62 mul V
  hpt 2 mul 0 V
  hpt neg vpt -1.62 mul V closepath fill} def
/DiaF {stroke [] 0 setdash vpt add M
  hpt neg vpt neg V hpt vpt neg V
  hpt vpt V hpt neg vpt V closepath fill} def
/Pent {stroke [] 0 setdash 2 copy gsave
  translate 0 hpt M 4 {72 rotate 0 hpt L} repeat
  closepath stroke grestore Pnt} def
/PentF {stroke [] 0 setdash gsave
  translate 0 hpt M 4 {72 rotate 0 hpt L} repeat
  closepath fill grestore} def
/Circle {stroke [] 0 setdash 2 copy
  hpt 0 360 arc stroke Pnt} def
/CircleF {stroke [] 0 setdash hpt 0 360 arc fill} def
/C0 {BL [] 0 setdash 2 copy moveto vpt 90 450 arc} bind def
/C1 {BL [] 0 setdash 2 copy moveto
	2 copy vpt 0 90 arc closepath fill
	vpt 0 360 arc closepath} bind def
/C2 {BL [] 0 setdash 2 copy moveto
	2 copy vpt 90 180 arc closepath fill
	vpt 0 360 arc closepath} bind def
/C3 {BL [] 0 setdash 2 copy moveto
	2 copy vpt 0 180 arc closepath fill
	vpt 0 360 arc closepath} bind def
/C4 {BL [] 0 setdash 2 copy moveto
	2 copy vpt 180 270 arc closepath fill
	vpt 0 360 arc closepath} bind def
/C5 {BL [] 0 setdash 2 copy moveto
	2 copy vpt 0 90 arc
	2 copy moveto
	2 copy vpt 180 270 arc closepath fill
	vpt 0 360 arc} bind def
/C6 {BL [] 0 setdash 2 copy moveto
	2 copy vpt 90 270 arc closepath fill
	vpt 0 360 arc closepath} bind def
/C7 {BL [] 0 setdash 2 copy moveto
	2 copy vpt 0 270 arc closepath fill
	vpt 0 360 arc closepath} bind def
/C8 {BL [] 0 setdash 2 copy moveto
	2 copy vpt 270 360 arc closepath fill
	vpt 0 360 arc closepath} bind def
/C9 {BL [] 0 setdash 2 copy moveto
	2 copy vpt 270 450 arc closepath fill
	vpt 0 360 arc closepath} bind def
/C10 {BL [] 0 setdash 2 copy 2 copy moveto vpt 270 360 arc closepath fill
	2 copy moveto
	2 copy vpt 90 180 arc closepath fill
	vpt 0 360 arc closepath} bind def
/C11 {BL [] 0 setdash 2 copy moveto
	2 copy vpt 0 180 arc closepath fill
	2 copy moveto
	2 copy vpt 270 360 arc closepath fill
	vpt 0 360 arc closepath} bind def
/C12 {BL [] 0 setdash 2 copy moveto
	2 copy vpt 180 360 arc closepath fill
	vpt 0 360 arc closepath} bind def
/C13 {BL [] 0 setdash 2 copy moveto
	2 copy vpt 0 90 arc closepath fill
	2 copy moveto
	2 copy vpt 180 360 arc closepath fill
	vpt 0 360 arc closepath} bind def
/C14 {BL [] 0 setdash 2 copy moveto
	2 copy vpt 90 360 arc closepath fill
	vpt 0 360 arc} bind def
/C15 {BL [] 0 setdash 2 copy vpt 0 360 arc closepath fill
	vpt 0 360 arc closepath} bind def
/Rec {newpath 4 2 roll moveto 1 index 0 rlineto 0 exch rlineto
	neg 0 rlineto closepath} bind def
/Square {dup Rec} bind def
/Bsquare {vpt sub exch vpt sub exch vpt2 Square} bind def
/S0 {BL [] 0 setdash 2 copy moveto 0 vpt rlineto BL Bsquare} bind def
/S1 {BL [] 0 setdash 2 copy vpt Square fill Bsquare} bind def
/S2 {BL [] 0 setdash 2 copy exch vpt sub exch vpt Square fill Bsquare} bind def
/S3 {BL [] 0 setdash 2 copy exch vpt sub exch vpt2 vpt Rec fill Bsquare} bind def
/S4 {BL [] 0 setdash 2 copy exch vpt sub exch vpt sub vpt Square fill Bsquare} bind def
/S5 {BL [] 0 setdash 2 copy 2 copy vpt Square fill
	exch vpt sub exch vpt sub vpt Square fill Bsquare} bind def
/S6 {BL [] 0 setdash 2 copy exch vpt sub exch vpt sub vpt vpt2 Rec fill Bsquare} bind def
/S7 {BL [] 0 setdash 2 copy exch vpt sub exch vpt sub vpt vpt2 Rec fill
	2 copy vpt Square fill Bsquare} bind def
/S8 {BL [] 0 setdash 2 copy vpt sub vpt Square fill Bsquare} bind def
/S9 {BL [] 0 setdash 2 copy vpt sub vpt vpt2 Rec fill Bsquare} bind def
/S10 {BL [] 0 setdash 2 copy vpt sub vpt Square fill 2 copy exch vpt sub exch vpt Square fill
	Bsquare} bind def
/S11 {BL [] 0 setdash 2 copy vpt sub vpt Square fill 2 copy exch vpt sub exch vpt2 vpt Rec fill
	Bsquare} bind def
/S12 {BL [] 0 setdash 2 copy exch vpt sub exch vpt sub vpt2 vpt Rec fill Bsquare} bind def
/S13 {BL [] 0 setdash 2 copy exch vpt sub exch vpt sub vpt2 vpt Rec fill
	2 copy vpt Square fill Bsquare} bind def
/S14 {BL [] 0 setdash 2 copy exch vpt sub exch vpt sub vpt2 vpt Rec fill
	2 copy exch vpt sub exch vpt Square fill Bsquare} bind def
/S15 {BL [] 0 setdash 2 copy Bsquare fill Bsquare} bind def
/D0 {gsave translate 45 rotate 0 0 S0 stroke grestore} bind def
/D1 {gsave translate 45 rotate 0 0 S1 stroke grestore} bind def
/D2 {gsave translate 45 rotate 0 0 S2 stroke grestore} bind def
/D3 {gsave translate 45 rotate 0 0 S3 stroke grestore} bind def
/D4 {gsave translate 45 rotate 0 0 S4 stroke grestore} bind def
/D5 {gsave translate 45 rotate 0 0 S5 stroke grestore} bind def
/D6 {gsave translate 45 rotate 0 0 S6 stroke grestore} bind def
/D7 {gsave translate 45 rotate 0 0 S7 stroke grestore} bind def
/D8 {gsave translate 45 rotate 0 0 S8 stroke grestore} bind def
/D9 {gsave translate 45 rotate 0 0 S9 stroke grestore} bind def
/D10 {gsave translate 45 rotate 0 0 S10 stroke grestore} bind def
/D11 {gsave translate 45 rotate 0 0 S11 stroke grestore} bind def
/D12 {gsave translate 45 rotate 0 0 S12 stroke grestore} bind def
/D13 {gsave translate 45 rotate 0 0 S13 stroke grestore} bind def
/D14 {gsave translate 45 rotate 0 0 S14 stroke grestore} bind def
/D15 {gsave translate 45 rotate 0 0 S15 stroke grestore} bind def
/DiaE {stroke [] 0 setdash vpt add M
  hpt neg vpt neg V hpt vpt neg V
  hpt vpt V hpt neg vpt V closepath stroke} def
/BoxE {stroke [] 0 setdash exch hpt sub exch vpt add M
  0 vpt2 neg V hpt2 0 V 0 vpt2 V
  hpt2 neg 0 V closepath stroke} def
/TriUE {stroke [] 0 setdash vpt 1.12 mul add M
  hpt neg vpt -1.62 mul V
  hpt 2 mul 0 V
  hpt neg vpt 1.62 mul V closepath stroke} def
/TriDE {stroke [] 0 setdash vpt 1.12 mul sub M
  hpt neg vpt 1.62 mul V
  hpt 2 mul 0 V
  hpt neg vpt -1.62 mul V closepath stroke} def
/PentE {stroke [] 0 setdash gsave
  translate 0 hpt M 4 {72 rotate 0 hpt L} repeat
  closepath stroke grestore} def
/CircE {stroke [] 0 setdash 
  hpt 0 360 arc stroke} def
/Opaque {gsave closepath 1 setgray fill grestore 0 setgray closepath} def
/DiaW {stroke [] 0 setdash vpt add M
  hpt neg vpt neg V hpt vpt neg V
  hpt vpt V hpt neg vpt V Opaque stroke} def
/BoxW {stroke [] 0 setdash exch hpt sub exch vpt add M
  0 vpt2 neg V hpt2 0 V 0 vpt2 V
  hpt2 neg 0 V Opaque stroke} def
/TriUW {stroke [] 0 setdash vpt 1.12 mul add M
  hpt neg vpt -1.62 mul V
  hpt 2 mul 0 V
  hpt neg vpt 1.62 mul V Opaque stroke} def
/TriDW {stroke [] 0 setdash vpt 1.12 mul sub M
  hpt neg vpt 1.62 mul V
  hpt 2 mul 0 V
  hpt neg vpt -1.62 mul V Opaque stroke} def
/PentW {stroke [] 0 setdash gsave
  translate 0 hpt M 4 {72 rotate 0 hpt L} repeat
  Opaque stroke grestore} def
/CircW {stroke [] 0 setdash 
  hpt 0 360 arc Opaque stroke} def
/BoxFill {gsave Rec 1 setgray fill grestore} def
/Density {
  /Fillden exch def
  currentrgbcolor
  /ColB exch def /ColG exch def /ColR exch def
  /ColR ColR Fillden mul Fillden sub 1 add def
  /ColG ColG Fillden mul Fillden sub 1 add def
  /ColB ColB Fillden mul Fillden sub 1 add def
  ColR ColG ColB setrgbcolor} def
/BoxColFill {gsave Rec PolyFill} def
/PolyFill {gsave Density fill grestore grestore} def
/h {rlineto rlineto rlineto gsave closepath fill grestore} bind def
%
% PostScript Level 1 Pattern Fill routine for rectangles
% Usage: x y w h s a XX PatternFill
%	x,y = lower left corner of box to be filled
%	w,h = width and height of box
%	  a = angle in degrees between lines and x-axis
%	 XX = 0/1 for no/yes cross-hatch
%
/PatternFill {gsave /PFa [ 9 2 roll ] def
  PFa 0 get PFa 2 get 2 div add PFa 1 get PFa 3 get 2 div add translate
  PFa 2 get -2 div PFa 3 get -2 div PFa 2 get PFa 3 get Rec
  TransparentPatterns {} {gsave 1 setgray fill grestore} ifelse
  clip
  currentlinewidth 0.5 mul setlinewidth
  /PFs PFa 2 get dup mul PFa 3 get dup mul add sqrt def
  0 0 M PFa 5 get rotate PFs -2 div dup translate
  0 1 PFs PFa 4 get div 1 add floor cvi
	{PFa 4 get mul 0 M 0 PFs V} for
  0 PFa 6 get ne {
	0 1 PFs PFa 4 get div 1 add floor cvi
	{PFa 4 get mul 0 2 1 roll M PFs 0 V} for
 } if
  stroke grestore} def
%
/languagelevel where
 {pop languagelevel} {1} ifelse
dup 2 lt
	{/InterpretLevel1 true def
	 /InterpretLevel3 false def}
	{/InterpretLevel1 Level1 def
	 2 gt
	    {/InterpretLevel3 Level3 def}
	    {/InterpretLevel3 false def}
	 ifelse }
 ifelse
%
% PostScript level 2 pattern fill definitions
%
/Level2PatternFill {
/Tile8x8 {/PaintType 2 /PatternType 1 /TilingType 1 /BBox [0 0 8 8] /XStep 8 /YStep 8}
	bind def
/KeepColor {currentrgbcolor [/Pattern /DeviceRGB] setcolorspace} bind def
<< Tile8x8
 /PaintProc {0.5 setlinewidth pop 0 0 M 8 8 L 0 8 M 8 0 L stroke} 
>> matrix makepattern
/Pat1 exch def
<< Tile8x8
 /PaintProc {0.5 setlinewidth pop 0 0 M 8 8 L 0 8 M 8 0 L stroke
	0 4 M 4 8 L 8 4 L 4 0 L 0 4 L stroke}
>> matrix makepattern
/Pat2 exch def
<< Tile8x8
 /PaintProc {0.5 setlinewidth pop 0 0 M 0 8 L
	8 8 L 8 0 L 0 0 L fill}
>> matrix makepattern
/Pat3 exch def
<< Tile8x8
 /PaintProc {0.5 setlinewidth pop -4 8 M 8 -4 L
	0 12 M 12 0 L stroke}
>> matrix makepattern
/Pat4 exch def
<< Tile8x8
 /PaintProc {0.5 setlinewidth pop -4 0 M 8 12 L
	0 -4 M 12 8 L stroke}
>> matrix makepattern
/Pat5 exch def
<< Tile8x8
 /PaintProc {0.5 setlinewidth pop -2 8 M 4 -4 L
	0 12 M 8 -4 L 4 12 M 10 0 L stroke}
>> matrix makepattern
/Pat6 exch def
<< Tile8x8
 /PaintProc {0.5 setlinewidth pop -2 0 M 4 12 L
	0 -4 M 8 12 L 4 -4 M 10 8 L stroke}
>> matrix makepattern
/Pat7 exch def
<< Tile8x8
 /PaintProc {0.5 setlinewidth pop 8 -2 M -4 4 L
	12 0 M -4 8 L 12 4 M 0 10 L stroke}
>> matrix makepattern
/Pat8 exch def
<< Tile8x8
 /PaintProc {0.5 setlinewidth pop 0 -2 M 12 4 L
	-4 0 M 12 8 L -4 4 M 8 10 L stroke}
>> matrix makepattern
/Pat9 exch def
/Pattern1 {PatternBgnd KeepColor Pat1 setpattern} bind def
/Pattern2 {PatternBgnd KeepColor Pat2 setpattern} bind def
/Pattern3 {PatternBgnd KeepColor Pat3 setpattern} bind def
/Pattern4 {PatternBgnd KeepColor Landscape {Pat5} {Pat4} ifelse setpattern} bind def
/Pattern5 {PatternBgnd KeepColor Landscape {Pat4} {Pat5} ifelse setpattern} bind def
/Pattern6 {PatternBgnd KeepColor Landscape {Pat9} {Pat6} ifelse setpattern} bind def
/Pattern7 {PatternBgnd KeepColor Landscape {Pat8} {Pat7} ifelse setpattern} bind def
} def
%
%
%End of PostScript Level 2 code
%
/PatternBgnd {
  TransparentPatterns {} {gsave 1 setgray fill grestore} ifelse
} def
%
% Substitute for Level 2 pattern fill codes with
% grayscale if Level 2 support is not selected.
%
/Level1PatternFill {
/Pattern1 {0.250 Density} bind def
/Pattern2 {0.500 Density} bind def
/Pattern3 {0.750 Density} bind def
/Pattern4 {0.125 Density} bind def
/Pattern5 {0.375 Density} bind def
/Pattern6 {0.625 Density} bind def
/Pattern7 {0.875 Density} bind def
} def
%
% Now test for support of Level 2 code
%
Level1 {Level1PatternFill} {Level2PatternFill} ifelse
%
/Symbol-Oblique /Symbol findfont [1 0 .167 1 0 0] makefont
dup length dict begin {1 index /FID eq {pop pop} {def} ifelse} forall
currentdict end definefont pop
%
Level1 SuppressPDFMark or 
{} {
/SDict 10 dict def
systemdict /pdfmark known not {
  userdict /pdfmark systemdict /cleartomark get put
} if
SDict begin [
  /Title (plot_tauQ_KsuN.tex)
  /Subject (gnuplot plot)
  /Creator (gnuplot 5.0 patchlevel 3)
  /Author (mteper)
%  /Producer (gnuplot)
%  /Keywords ()
  /CreationDate (Tue May 18 14:50:52 2021)
  /DOCINFO pdfmark
end
} ifelse
%
% Support for boxed text - Ethan A Merritt May 2005
%
/InitTextBox { userdict /TBy2 3 -1 roll put userdict /TBx2 3 -1 roll put
           userdict /TBy1 3 -1 roll put userdict /TBx1 3 -1 roll put
	   /Boxing true def } def
/ExtendTextBox { Boxing
    { gsave dup false charpath pathbbox
      dup TBy2 gt {userdict /TBy2 3 -1 roll put} {pop} ifelse
      dup TBx2 gt {userdict /TBx2 3 -1 roll put} {pop} ifelse
      dup TBy1 lt {userdict /TBy1 3 -1 roll put} {pop} ifelse
      dup TBx1 lt {userdict /TBx1 3 -1 roll put} {pop} ifelse
      grestore } if } def
/PopTextBox { newpath TBx1 TBxmargin sub TBy1 TBymargin sub M
               TBx1 TBxmargin sub TBy2 TBymargin add L
	       TBx2 TBxmargin add TBy2 TBymargin add L
	       TBx2 TBxmargin add TBy1 TBymargin sub L closepath } def
/DrawTextBox { PopTextBox stroke /Boxing false def} def
/FillTextBox { gsave PopTextBox 1 1 1 setrgbcolor fill grestore /Boxing false def} def
0 0 0 0 InitTextBox
/TBxmargin 20 def
/TBymargin 20 def
/Boxing false def
/textshow { ExtendTextBox Gshow } def
%
% redundant definitions for compatibility with prologue.ps older than 5.0.2
/LTB {BL [] LCb DL} def
/LTb {PL [] LCb DL} def
end
%%EndProlog
%%Page: 1 1
gnudict begin
gsave
doclip
0 0 translate
0.050 0.050 scale
0 setgray
newpath
BackgroundColor 0 lt 3 1 roll 0 lt exch 0 lt or or not {BackgroundColor C 1.000 0 0 7200.00 7560.00 BoxColFill} if
1.000 UL
LTb
LCb setrgbcolor
1020 640 M
63 0 V
5756 0 R
-63 0 V
stroke
LTb
LCb setrgbcolor
1020 1475 M
63 0 V
5756 0 R
-63 0 V
stroke
LTb
LCb setrgbcolor
1020 2310 M
63 0 V
5756 0 R
-63 0 V
stroke
LTb
LCb setrgbcolor
1020 3145 M
63 0 V
5756 0 R
-63 0 V
stroke
LTb
LCb setrgbcolor
1020 3980 M
63 0 V
5756 0 R
-63 0 V
stroke
LTb
LCb setrgbcolor
1020 4814 M
63 0 V
5756 0 R
-63 0 V
stroke
LTb
LCb setrgbcolor
1020 5649 M
63 0 V
5756 0 R
-63 0 V
stroke
LTb
LCb setrgbcolor
1020 6484 M
63 0 V
5756 0 R
-63 0 V
stroke
LTb
LCb setrgbcolor
1020 7319 M
63 0 V
5756 0 R
-63 0 V
stroke
LTb
LCb setrgbcolor
1020 640 M
0 63 V
0 6616 R
0 -63 V
stroke
LTb
LCb setrgbcolor
1667 640 M
0 63 V
0 6616 R
0 -63 V
stroke
LTb
LCb setrgbcolor
2313 640 M
0 63 V
0 6616 R
0 -63 V
stroke
LTb
LCb setrgbcolor
2960 640 M
0 63 V
0 6616 R
0 -63 V
stroke
LTb
LCb setrgbcolor
3606 640 M
0 63 V
0 6616 R
0 -63 V
stroke
LTb
LCb setrgbcolor
4253 640 M
0 63 V
0 6616 R
0 -63 V
stroke
LTb
LCb setrgbcolor
4899 640 M
0 63 V
0 6616 R
0 -63 V
stroke
LTb
LCb setrgbcolor
5546 640 M
0 63 V
0 6616 R
0 -63 V
stroke
LTb
LCb setrgbcolor
6192 640 M
0 63 V
0 6616 R
0 -63 V
stroke
LTb
LCb setrgbcolor
6839 640 M
0 63 V
0 6616 R
0 -63 V
stroke
LTb
LCb setrgbcolor
1.000 UL
LTb
LCb setrgbcolor
1020 7319 N
0 -6679 V
5819 0 V
0 6679 V
-5819 0 V
Z stroke
1.000 UP
1.000 UL
LTb
LCb setrgbcolor
LCb setrgbcolor
LTb
LCb setrgbcolor
LTb
1.500 UP
1.000 UL
LTb
0.58 0.00 0.83 C 3942 2602 M
0 15 V
454 371 R
0 17 V
733 590 R
0 34 V
434 423 R
0 42 V
431 312 R
0 67 V
3942 2610 CircleF
4396 2997 CircleF
5129 3612 CircleF
5563 4073 CircleF
5994 4440 CircleF
1.500 UL
LTb
0.58 0.00 0.83 C 1713 640 M
12 10 V
59 52 V
59 52 V
59 52 V
58 52 V
59 51 V
59 52 V
59 52 V
59 52 V
58 52 V
59 51 V
59 52 V
59 52 V
58 52 V
59 52 V
59 51 V
59 52 V
59 52 V
58 52 V
59 52 V
59 51 V
59 52 V
58 52 V
59 52 V
59 52 V
59 51 V
59 52 V
58 52 V
59 52 V
59 52 V
59 51 V
58 52 V
59 52 V
59 52 V
59 52 V
59 51 V
58 52 V
59 52 V
59 52 V
59 52 V
58 51 V
59 52 V
59 52 V
59 52 V
59 52 V
58 51 V
59 52 V
59 52 V
59 52 V
58 52 V
59 51 V
59 52 V
59 52 V
59 52 V
58 52 V
59 51 V
59 52 V
59 52 V
58 52 V
59 52 V
59 51 V
59 52 V
59 52 V
58 52 V
59 52 V
59 51 V
59 52 V
58 52 V
59 52 V
59 52 V
59 51 V
59 52 V
58 52 V
59 52 V
59 52 V
59 51 V
58 52 V
59 52 V
59 52 V
59 52 V
59 51 V
58 52 V
59 52 V
59 52 V
59 52 V
58 51 V
59 52 V
59 52 V
1.500 UP
stroke
1.000 UL
LTb
0.58 0.00 0.83 C 3304 3243 M
0 15 V
551 601 R
0 16 V
512 710 R
0 42 V
3304 3251 Circle
3855 3867 Circle
4367 4606 Circle
1.500 UL
LTb
0.58 0.00 0.83 C 1133 640 M
5 5 V
58 70 V
59 71 V
59 70 V
59 70 V
58 71 V
59 70 V
59 71 V
59 70 V
59 70 V
58 71 V
59 70 V
59 70 V
59 71 V
58 70 V
59 70 V
59 71 V
59 70 V
59 70 V
58 71 V
59 70 V
59 71 V
59 70 V
58 70 V
59 71 V
59 70 V
59 70 V
59 71 V
58 70 V
59 70 V
59 71 V
59 70 V
58 70 V
59 71 V
59 70 V
59 71 V
59 70 V
58 70 V
59 71 V
59 70 V
59 70 V
58 71 V
59 70 V
59 70 V
59 71 V
59 70 V
58 70 V
59 71 V
59 70 V
59 71 V
58 70 V
59 70 V
59 71 V
59 70 V
59 70 V
58 71 V
59 70 V
59 70 V
59 71 V
58 70 V
59 70 V
59 71 V
59 70 V
59 71 V
58 70 V
59 70 V
59 71 V
59 70 V
58 70 V
59 71 V
59 70 V
59 70 V
59 71 V
58 70 V
59 70 V
59 71 V
59 70 V
58 71 V
59 70 V
59 70 V
59 71 V
59 70 V
58 70 V
59 71 V
59 70 V
59 70 V
58 71 V
59 70 V
59 70 V
59 71 V
59 70 V
58 71 V
59 70 V
59 70 V
59 71 V
50 60 V
1.500 UP
stroke
1.000 UL
LTb
0.58 0.00 0.83 C 4399 4515 M
0 74 V
4399 4552 Circle
1.500 UP
1.000 UL
LTb
0.58 0.00 0.83 C 2650 3380 M
0 14 V
400 538 R
0 28 V
739 1031 R
0 79 V
568 854 R
0 288 V
2650 3387 BoxF
3050 3946 BoxF
3789 5030 BoxF
4357 6068 BoxF
1.500 UL
LTb
0.58 0.00 0.83 C 1020 1041 M
59 84 V
59 85 V
58 84 V
59 85 V
59 84 V
59 85 V
58 84 V
59 85 V
59 84 V
59 85 V
59 84 V
58 85 V
59 84 V
59 85 V
59 84 V
58 85 V
59 84 V
59 85 V
59 84 V
59 85 V
58 84 V
59 85 V
59 84 V
59 84 V
58 85 V
59 84 V
59 85 V
59 84 V
59 85 V
58 84 V
59 85 V
59 84 V
59 85 V
58 84 V
59 85 V
59 84 V
59 85 V
59 84 V
58 85 V
59 84 V
59 85 V
59 84 V
58 85 V
59 84 V
59 85 V
59 84 V
59 85 V
58 84 V
59 85 V
59 84 V
59 85 V
58 84 V
59 85 V
59 84 V
59 85 V
59 84 V
58 85 V
59 84 V
59 85 V
59 84 V
58 85 V
59 84 V
59 85 V
59 84 V
59 85 V
58 84 V
59 85 V
59 84 V
59 85 V
58 84 V
59 84 V
59 85 V
59 84 V
59 85 V
18 26 V
1.500 UP
stroke
1.000 UL
LTb
0.58 0.00 0.83 C 2246 3581 M
0 21 V
406 658 R
0 37 V
314 531 R
0 59 V
852 1450 R
0 439 V
2246 3592 Box
2652 4279 Box
2966 4858 Box
3818 6556 Box
1.500 UL
LTb
0.58 0.00 0.83 C 1020 1452 M
59 102 V
59 103 V
58 102 V
59 103 V
59 102 V
59 103 V
58 102 V
59 102 V
59 103 V
59 102 V
59 103 V
58 102 V
59 103 V
59 102 V
59 102 V
58 103 V
59 102 V
59 103 V
59 102 V
59 103 V
58 102 V
59 102 V
59 103 V
59 102 V
58 103 V
59 102 V
59 103 V
59 102 V
59 102 V
58 103 V
59 102 V
59 103 V
59 102 V
58 103 V
59 102 V
59 102 V
59 103 V
59 102 V
58 103 V
59 102 V
59 103 V
59 102 V
58 102 V
59 103 V
59 102 V
59 103 V
59 102 V
58 103 V
59 102 V
59 102 V
59 103 V
58 102 V
59 103 V
59 102 V
59 103 V
59 102 V
58 103 V
17 28 V
1.500 UP
stroke
1.000 UL
LTb
0.58 0.00 0.83 C 1412 3249 M
0 16 V
756 1571 R
0 105 V
536 1158 R
0 334 V
1412 3257 DiaF
2168 4888 DiaF
2704 6266 DiaF
1.500 UL
LTb
0.58 0.00 0.83 C 1020 2367 M
59 133 V
59 134 V
58 133 V
59 133 V
59 134 V
59 133 V
58 134 V
59 133 V
59 133 V
59 134 V
59 133 V
58 133 V
59 134 V
59 133 V
59 134 V
58 133 V
59 133 V
59 134 V
59 133 V
59 133 V
58 134 V
59 133 V
59 134 V
59 133 V
58 133 V
59 134 V
59 133 V
59 133 V
59 134 V
58 133 V
59 134 V
59 133 V
59 133 V
58 134 V
59 133 V
59 133 V
59 134 V
7 17 V
1.500 UP
stroke
1.000 UL
LTb
0.58 0.00 0.83 C 2736 6357 M
0 286 V
2736 6500 DiaF
2.000 UL
LTb
LCb setrgbcolor
1.000 UL
LTb
LCb setrgbcolor
1020 7319 N
0 -6679 V
5819 0 V
0 6679 V
-5819 0 V
Z stroke
1.000 UP
1.000 UL
LTb
LCb setrgbcolor
stroke
grestore
end
showpage
  }}%
  \put(3929,140){\makebox(0,0){\large{$\ln\{1/a\surd\sigma\}$}}}%
  \put(200,4979){\makebox(0,0){\Large{$\ln\{\Tilde{\tau}_Q\}$}}}%
  \put(6839,440){\makebox(0,0){\strut{}\ {$2.8$}}}%
  \put(6192,440){\makebox(0,0){\strut{}\ {$2.6$}}}%
  \put(5546,440){\makebox(0,0){\strut{}\ {$2.4$}}}%
  \put(4899,440){\makebox(0,0){\strut{}\ {$2.2$}}}%
  \put(4253,440){\makebox(0,0){\strut{}\ {$2$}}}%
  \put(3606,440){\makebox(0,0){\strut{}\ {$1.8$}}}%
  \put(2960,440){\makebox(0,0){\strut{}\ {$1.6$}}}%
  \put(2313,440){\makebox(0,0){\strut{}\ {$1.4$}}}%
  \put(1667,440){\makebox(0,0){\strut{}\ {$1.2$}}}%
  \put(1020,440){\makebox(0,0){\strut{}\ {$1$}}}%
  \put(900,7319){\makebox(0,0)[r]{\strut{}\ \ {$16$}}}%
  \put(900,6484){\makebox(0,0)[r]{\strut{}\ \ {$14$}}}%
  \put(900,5649){\makebox(0,0)[r]{\strut{}\ \ {$12$}}}%
  \put(900,4814){\makebox(0,0)[r]{\strut{}\ \ {$10$}}}%
  \put(900,3980){\makebox(0,0)[r]{\strut{}\ \ {$8$}}}%
  \put(900,3145){\makebox(0,0)[r]{\strut{}\ \ {$6$}}}%
  \put(900,2310){\makebox(0,0)[r]{\strut{}\ \ {$4$}}}%
  \put(900,1475){\makebox(0,0)[r]{\strut{}\ \ {$2$}}}%
  \put(900,640){\makebox(0,0)[r]{\strut{}\ \ {$0$}}}%
\end{picture}%
\endgroup
\endinput

\end	{center}
\caption{Variation of $\Tilde{\tau}_Q$, the average number of sweeps between changes of $Q$ by $\pm 1$,
  against the lattice spacing and normalised to our standard space-time volume $V=(3/\surd\sigma)^4$.
  For $SU(3)$ ($\bullet$), $SU(4)$ ($\circ$), $SU(5)$ ($\blacksquare$),
  $SU(6)$ ($\square$), $SU(8)$ ($\blacklozenge$). The two pairs of points nearly overlapping
  are obtained at same $a$ but from different volumes. Lines are fits
  $\Tilde{\tau}_Q = b\{1/a\surd\sigma\}^c$.}
\label{fig_tauQ_KsuN}
\end{figure}




\begin{figure}[htb]
\begin	{center}
\leavevmode
% GNUPLOT: LaTeX picture with Postscript
\begingroup%
\makeatletter%
\newcommand{\GNUPLOTspecial}{%
  \@sanitize\catcode`\%=14\relax\special}%
\setlength{\unitlength}{0.0500bp}%
\begin{picture}(7200,7560)(0,0)%
  {\GNUPLOTspecial{"
%!PS-Adobe-2.0 EPSF-2.0
%%Title: plot_khiIK_cont.tex
%%Creator: gnuplot 5.0 patchlevel 3
%%CreationDate: Tue Apr 13 15:19:11 2021
%%DocumentFonts: 
%%BoundingBox: 0 0 360 378
%%EndComments
%%BeginProlog
/gnudict 256 dict def
gnudict begin
%
% The following true/false flags may be edited by hand if desired.
% The unit line width and grayscale image gamma correction may also be changed.
%
/Color true def
/Blacktext true def
/Solid false def
/Dashlength 1 def
/Landscape false def
/Level1 false def
/Level3 false def
/Rounded false def
/ClipToBoundingBox false def
/SuppressPDFMark false def
/TransparentPatterns false def
/gnulinewidth 5.000 def
/userlinewidth gnulinewidth def
/Gamma 1.0 def
/BackgroundColor {-1.000 -1.000 -1.000} def
%
/vshift -66 def
/dl1 {
  10.0 Dashlength userlinewidth gnulinewidth div mul mul mul
  Rounded { currentlinewidth 0.75 mul sub dup 0 le { pop 0.01 } if } if
} def
/dl2 {
  10.0 Dashlength userlinewidth gnulinewidth div mul mul mul
  Rounded { currentlinewidth 0.75 mul add } if
} def
/hpt_ 31.5 def
/vpt_ 31.5 def
/hpt hpt_ def
/vpt vpt_ def
/doclip {
  ClipToBoundingBox {
    newpath 0 0 moveto 360 0 lineto 360 378 lineto 0 378 lineto closepath
    clip
  } if
} def
%
% Gnuplot Prolog Version 5.1 (Oct 2015)
%
%/SuppressPDFMark true def
%
/M {moveto} bind def
/L {lineto} bind def
/R {rmoveto} bind def
/V {rlineto} bind def
/N {newpath moveto} bind def
/Z {closepath} bind def
/C {setrgbcolor} bind def
/f {rlineto fill} bind def
/g {setgray} bind def
/Gshow {show} def   % May be redefined later in the file to support UTF-8
/vpt2 vpt 2 mul def
/hpt2 hpt 2 mul def
/Lshow {currentpoint stroke M 0 vshift R 
	Blacktext {gsave 0 setgray textshow grestore} {textshow} ifelse} def
/Rshow {currentpoint stroke M dup stringwidth pop neg vshift R
	Blacktext {gsave 0 setgray textshow grestore} {textshow} ifelse} def
/Cshow {currentpoint stroke M dup stringwidth pop -2 div vshift R 
	Blacktext {gsave 0 setgray textshow grestore} {textshow} ifelse} def
/UP {dup vpt_ mul /vpt exch def hpt_ mul /hpt exch def
  /hpt2 hpt 2 mul def /vpt2 vpt 2 mul def} def
/DL {Color {setrgbcolor Solid {pop []} if 0 setdash}
 {pop pop pop 0 setgray Solid {pop []} if 0 setdash} ifelse} def
/BL {stroke userlinewidth 2 mul setlinewidth
	Rounded {1 setlinejoin 1 setlinecap} if} def
/AL {stroke userlinewidth 2 div setlinewidth
	Rounded {1 setlinejoin 1 setlinecap} if} def
/UL {dup gnulinewidth mul /userlinewidth exch def
	dup 1 lt {pop 1} if 10 mul /udl exch def} def
/PL {stroke userlinewidth setlinewidth
	Rounded {1 setlinejoin 1 setlinecap} if} def
3.8 setmiterlimit
% Classic Line colors (version 5.0)
/LCw {1 1 1} def
/LCb {0 0 0} def
/LCa {0 0 0} def
/LC0 {1 0 0} def
/LC1 {0 1 0} def
/LC2 {0 0 1} def
/LC3 {1 0 1} def
/LC4 {0 1 1} def
/LC5 {1 1 0} def
/LC6 {0 0 0} def
/LC7 {1 0.3 0} def
/LC8 {0.5 0.5 0.5} def
% Default dash patterns (version 5.0)
/LTB {BL [] LCb DL} def
/LTw {PL [] 1 setgray} def
/LTb {PL [] LCb DL} def
/LTa {AL [1 udl mul 2 udl mul] 0 setdash LCa setrgbcolor} def
/LT0 {PL [] LC0 DL} def
/LT1 {PL [2 dl1 3 dl2] LC1 DL} def
/LT2 {PL [1 dl1 1.5 dl2] LC2 DL} def
/LT3 {PL [6 dl1 2 dl2 1 dl1 2 dl2] LC3 DL} def
/LT4 {PL [1 dl1 2 dl2 6 dl1 2 dl2 1 dl1 2 dl2] LC4 DL} def
/LT5 {PL [4 dl1 2 dl2] LC5 DL} def
/LT6 {PL [1.5 dl1 1.5 dl2 1.5 dl1 1.5 dl2 1.5 dl1 6 dl2] LC6 DL} def
/LT7 {PL [3 dl1 3 dl2 1 dl1 3 dl2] LC7 DL} def
/LT8 {PL [2 dl1 2 dl2 2 dl1 6 dl2] LC8 DL} def
/SL {[] 0 setdash} def
/Pnt {stroke [] 0 setdash gsave 1 setlinecap M 0 0 V stroke grestore} def
/Dia {stroke [] 0 setdash 2 copy vpt add M
  hpt neg vpt neg V hpt vpt neg V
  hpt vpt V hpt neg vpt V closepath stroke
  Pnt} def
/Pls {stroke [] 0 setdash vpt sub M 0 vpt2 V
  currentpoint stroke M
  hpt neg vpt neg R hpt2 0 V stroke
 } def
/Box {stroke [] 0 setdash 2 copy exch hpt sub exch vpt add M
  0 vpt2 neg V hpt2 0 V 0 vpt2 V
  hpt2 neg 0 V closepath stroke
  Pnt} def
/Crs {stroke [] 0 setdash exch hpt sub exch vpt add M
  hpt2 vpt2 neg V currentpoint stroke M
  hpt2 neg 0 R hpt2 vpt2 V stroke} def
/TriU {stroke [] 0 setdash 2 copy vpt 1.12 mul add M
  hpt neg vpt -1.62 mul V
  hpt 2 mul 0 V
  hpt neg vpt 1.62 mul V closepath stroke
  Pnt} def
/Star {2 copy Pls Crs} def
/BoxF {stroke [] 0 setdash exch hpt sub exch vpt add M
  0 vpt2 neg V hpt2 0 V 0 vpt2 V
  hpt2 neg 0 V closepath fill} def
/TriUF {stroke [] 0 setdash vpt 1.12 mul add M
  hpt neg vpt -1.62 mul V
  hpt 2 mul 0 V
  hpt neg vpt 1.62 mul V closepath fill} def
/TriD {stroke [] 0 setdash 2 copy vpt 1.12 mul sub M
  hpt neg vpt 1.62 mul V
  hpt 2 mul 0 V
  hpt neg vpt -1.62 mul V closepath stroke
  Pnt} def
/TriDF {stroke [] 0 setdash vpt 1.12 mul sub M
  hpt neg vpt 1.62 mul V
  hpt 2 mul 0 V
  hpt neg vpt -1.62 mul V closepath fill} def
/DiaF {stroke [] 0 setdash vpt add M
  hpt neg vpt neg V hpt vpt neg V
  hpt vpt V hpt neg vpt V closepath fill} def
/Pent {stroke [] 0 setdash 2 copy gsave
  translate 0 hpt M 4 {72 rotate 0 hpt L} repeat
  closepath stroke grestore Pnt} def
/PentF {stroke [] 0 setdash gsave
  translate 0 hpt M 4 {72 rotate 0 hpt L} repeat
  closepath fill grestore} def
/Circle {stroke [] 0 setdash 2 copy
  hpt 0 360 arc stroke Pnt} def
/CircleF {stroke [] 0 setdash hpt 0 360 arc fill} def
/C0 {BL [] 0 setdash 2 copy moveto vpt 90 450 arc} bind def
/C1 {BL [] 0 setdash 2 copy moveto
	2 copy vpt 0 90 arc closepath fill
	vpt 0 360 arc closepath} bind def
/C2 {BL [] 0 setdash 2 copy moveto
	2 copy vpt 90 180 arc closepath fill
	vpt 0 360 arc closepath} bind def
/C3 {BL [] 0 setdash 2 copy moveto
	2 copy vpt 0 180 arc closepath fill
	vpt 0 360 arc closepath} bind def
/C4 {BL [] 0 setdash 2 copy moveto
	2 copy vpt 180 270 arc closepath fill
	vpt 0 360 arc closepath} bind def
/C5 {BL [] 0 setdash 2 copy moveto
	2 copy vpt 0 90 arc
	2 copy moveto
	2 copy vpt 180 270 arc closepath fill
	vpt 0 360 arc} bind def
/C6 {BL [] 0 setdash 2 copy moveto
	2 copy vpt 90 270 arc closepath fill
	vpt 0 360 arc closepath} bind def
/C7 {BL [] 0 setdash 2 copy moveto
	2 copy vpt 0 270 arc closepath fill
	vpt 0 360 arc closepath} bind def
/C8 {BL [] 0 setdash 2 copy moveto
	2 copy vpt 270 360 arc closepath fill
	vpt 0 360 arc closepath} bind def
/C9 {BL [] 0 setdash 2 copy moveto
	2 copy vpt 270 450 arc closepath fill
	vpt 0 360 arc closepath} bind def
/C10 {BL [] 0 setdash 2 copy 2 copy moveto vpt 270 360 arc closepath fill
	2 copy moveto
	2 copy vpt 90 180 arc closepath fill
	vpt 0 360 arc closepath} bind def
/C11 {BL [] 0 setdash 2 copy moveto
	2 copy vpt 0 180 arc closepath fill
	2 copy moveto
	2 copy vpt 270 360 arc closepath fill
	vpt 0 360 arc closepath} bind def
/C12 {BL [] 0 setdash 2 copy moveto
	2 copy vpt 180 360 arc closepath fill
	vpt 0 360 arc closepath} bind def
/C13 {BL [] 0 setdash 2 copy moveto
	2 copy vpt 0 90 arc closepath fill
	2 copy moveto
	2 copy vpt 180 360 arc closepath fill
	vpt 0 360 arc closepath} bind def
/C14 {BL [] 0 setdash 2 copy moveto
	2 copy vpt 90 360 arc closepath fill
	vpt 0 360 arc} bind def
/C15 {BL [] 0 setdash 2 copy vpt 0 360 arc closepath fill
	vpt 0 360 arc closepath} bind def
/Rec {newpath 4 2 roll moveto 1 index 0 rlineto 0 exch rlineto
	neg 0 rlineto closepath} bind def
/Square {dup Rec} bind def
/Bsquare {vpt sub exch vpt sub exch vpt2 Square} bind def
/S0 {BL [] 0 setdash 2 copy moveto 0 vpt rlineto BL Bsquare} bind def
/S1 {BL [] 0 setdash 2 copy vpt Square fill Bsquare} bind def
/S2 {BL [] 0 setdash 2 copy exch vpt sub exch vpt Square fill Bsquare} bind def
/S3 {BL [] 0 setdash 2 copy exch vpt sub exch vpt2 vpt Rec fill Bsquare} bind def
/S4 {BL [] 0 setdash 2 copy exch vpt sub exch vpt sub vpt Square fill Bsquare} bind def
/S5 {BL [] 0 setdash 2 copy 2 copy vpt Square fill
	exch vpt sub exch vpt sub vpt Square fill Bsquare} bind def
/S6 {BL [] 0 setdash 2 copy exch vpt sub exch vpt sub vpt vpt2 Rec fill Bsquare} bind def
/S7 {BL [] 0 setdash 2 copy exch vpt sub exch vpt sub vpt vpt2 Rec fill
	2 copy vpt Square fill Bsquare} bind def
/S8 {BL [] 0 setdash 2 copy vpt sub vpt Square fill Bsquare} bind def
/S9 {BL [] 0 setdash 2 copy vpt sub vpt vpt2 Rec fill Bsquare} bind def
/S10 {BL [] 0 setdash 2 copy vpt sub vpt Square fill 2 copy exch vpt sub exch vpt Square fill
	Bsquare} bind def
/S11 {BL [] 0 setdash 2 copy vpt sub vpt Square fill 2 copy exch vpt sub exch vpt2 vpt Rec fill
	Bsquare} bind def
/S12 {BL [] 0 setdash 2 copy exch vpt sub exch vpt sub vpt2 vpt Rec fill Bsquare} bind def
/S13 {BL [] 0 setdash 2 copy exch vpt sub exch vpt sub vpt2 vpt Rec fill
	2 copy vpt Square fill Bsquare} bind def
/S14 {BL [] 0 setdash 2 copy exch vpt sub exch vpt sub vpt2 vpt Rec fill
	2 copy exch vpt sub exch vpt Square fill Bsquare} bind def
/S15 {BL [] 0 setdash 2 copy Bsquare fill Bsquare} bind def
/D0 {gsave translate 45 rotate 0 0 S0 stroke grestore} bind def
/D1 {gsave translate 45 rotate 0 0 S1 stroke grestore} bind def
/D2 {gsave translate 45 rotate 0 0 S2 stroke grestore} bind def
/D3 {gsave translate 45 rotate 0 0 S3 stroke grestore} bind def
/D4 {gsave translate 45 rotate 0 0 S4 stroke grestore} bind def
/D5 {gsave translate 45 rotate 0 0 S5 stroke grestore} bind def
/D6 {gsave translate 45 rotate 0 0 S6 stroke grestore} bind def
/D7 {gsave translate 45 rotate 0 0 S7 stroke grestore} bind def
/D8 {gsave translate 45 rotate 0 0 S8 stroke grestore} bind def
/D9 {gsave translate 45 rotate 0 0 S9 stroke grestore} bind def
/D10 {gsave translate 45 rotate 0 0 S10 stroke grestore} bind def
/D11 {gsave translate 45 rotate 0 0 S11 stroke grestore} bind def
/D12 {gsave translate 45 rotate 0 0 S12 stroke grestore} bind def
/D13 {gsave translate 45 rotate 0 0 S13 stroke grestore} bind def
/D14 {gsave translate 45 rotate 0 0 S14 stroke grestore} bind def
/D15 {gsave translate 45 rotate 0 0 S15 stroke grestore} bind def
/DiaE {stroke [] 0 setdash vpt add M
  hpt neg vpt neg V hpt vpt neg V
  hpt vpt V hpt neg vpt V closepath stroke} def
/BoxE {stroke [] 0 setdash exch hpt sub exch vpt add M
  0 vpt2 neg V hpt2 0 V 0 vpt2 V
  hpt2 neg 0 V closepath stroke} def
/TriUE {stroke [] 0 setdash vpt 1.12 mul add M
  hpt neg vpt -1.62 mul V
  hpt 2 mul 0 V
  hpt neg vpt 1.62 mul V closepath stroke} def
/TriDE {stroke [] 0 setdash vpt 1.12 mul sub M
  hpt neg vpt 1.62 mul V
  hpt 2 mul 0 V
  hpt neg vpt -1.62 mul V closepath stroke} def
/PentE {stroke [] 0 setdash gsave
  translate 0 hpt M 4 {72 rotate 0 hpt L} repeat
  closepath stroke grestore} def
/CircE {stroke [] 0 setdash 
  hpt 0 360 arc stroke} def
/Opaque {gsave closepath 1 setgray fill grestore 0 setgray closepath} def
/DiaW {stroke [] 0 setdash vpt add M
  hpt neg vpt neg V hpt vpt neg V
  hpt vpt V hpt neg vpt V Opaque stroke} def
/BoxW {stroke [] 0 setdash exch hpt sub exch vpt add M
  0 vpt2 neg V hpt2 0 V 0 vpt2 V
  hpt2 neg 0 V Opaque stroke} def
/TriUW {stroke [] 0 setdash vpt 1.12 mul add M
  hpt neg vpt -1.62 mul V
  hpt 2 mul 0 V
  hpt neg vpt 1.62 mul V Opaque stroke} def
/TriDW {stroke [] 0 setdash vpt 1.12 mul sub M
  hpt neg vpt 1.62 mul V
  hpt 2 mul 0 V
  hpt neg vpt -1.62 mul V Opaque stroke} def
/PentW {stroke [] 0 setdash gsave
  translate 0 hpt M 4 {72 rotate 0 hpt L} repeat
  Opaque stroke grestore} def
/CircW {stroke [] 0 setdash 
  hpt 0 360 arc Opaque stroke} def
/BoxFill {gsave Rec 1 setgray fill grestore} def
/Density {
  /Fillden exch def
  currentrgbcolor
  /ColB exch def /ColG exch def /ColR exch def
  /ColR ColR Fillden mul Fillden sub 1 add def
  /ColG ColG Fillden mul Fillden sub 1 add def
  /ColB ColB Fillden mul Fillden sub 1 add def
  ColR ColG ColB setrgbcolor} def
/BoxColFill {gsave Rec PolyFill} def
/PolyFill {gsave Density fill grestore grestore} def
/h {rlineto rlineto rlineto gsave closepath fill grestore} bind def
%
% PostScript Level 1 Pattern Fill routine for rectangles
% Usage: x y w h s a XX PatternFill
%	x,y = lower left corner of box to be filled
%	w,h = width and height of box
%	  a = angle in degrees between lines and x-axis
%	 XX = 0/1 for no/yes cross-hatch
%
/PatternFill {gsave /PFa [ 9 2 roll ] def
  PFa 0 get PFa 2 get 2 div add PFa 1 get PFa 3 get 2 div add translate
  PFa 2 get -2 div PFa 3 get -2 div PFa 2 get PFa 3 get Rec
  TransparentPatterns {} {gsave 1 setgray fill grestore} ifelse
  clip
  currentlinewidth 0.5 mul setlinewidth
  /PFs PFa 2 get dup mul PFa 3 get dup mul add sqrt def
  0 0 M PFa 5 get rotate PFs -2 div dup translate
  0 1 PFs PFa 4 get div 1 add floor cvi
	{PFa 4 get mul 0 M 0 PFs V} for
  0 PFa 6 get ne {
	0 1 PFs PFa 4 get div 1 add floor cvi
	{PFa 4 get mul 0 2 1 roll M PFs 0 V} for
 } if
  stroke grestore} def
%
/languagelevel where
 {pop languagelevel} {1} ifelse
dup 2 lt
	{/InterpretLevel1 true def
	 /InterpretLevel3 false def}
	{/InterpretLevel1 Level1 def
	 2 gt
	    {/InterpretLevel3 Level3 def}
	    {/InterpretLevel3 false def}
	 ifelse }
 ifelse
%
% PostScript level 2 pattern fill definitions
%
/Level2PatternFill {
/Tile8x8 {/PaintType 2 /PatternType 1 /TilingType 1 /BBox [0 0 8 8] /XStep 8 /YStep 8}
	bind def
/KeepColor {currentrgbcolor [/Pattern /DeviceRGB] setcolorspace} bind def
<< Tile8x8
 /PaintProc {0.5 setlinewidth pop 0 0 M 8 8 L 0 8 M 8 0 L stroke} 
>> matrix makepattern
/Pat1 exch def
<< Tile8x8
 /PaintProc {0.5 setlinewidth pop 0 0 M 8 8 L 0 8 M 8 0 L stroke
	0 4 M 4 8 L 8 4 L 4 0 L 0 4 L stroke}
>> matrix makepattern
/Pat2 exch def
<< Tile8x8
 /PaintProc {0.5 setlinewidth pop 0 0 M 0 8 L
	8 8 L 8 0 L 0 0 L fill}
>> matrix makepattern
/Pat3 exch def
<< Tile8x8
 /PaintProc {0.5 setlinewidth pop -4 8 M 8 -4 L
	0 12 M 12 0 L stroke}
>> matrix makepattern
/Pat4 exch def
<< Tile8x8
 /PaintProc {0.5 setlinewidth pop -4 0 M 8 12 L
	0 -4 M 12 8 L stroke}
>> matrix makepattern
/Pat5 exch def
<< Tile8x8
 /PaintProc {0.5 setlinewidth pop -2 8 M 4 -4 L
	0 12 M 8 -4 L 4 12 M 10 0 L stroke}
>> matrix makepattern
/Pat6 exch def
<< Tile8x8
 /PaintProc {0.5 setlinewidth pop -2 0 M 4 12 L
	0 -4 M 8 12 L 4 -4 M 10 8 L stroke}
>> matrix makepattern
/Pat7 exch def
<< Tile8x8
 /PaintProc {0.5 setlinewidth pop 8 -2 M -4 4 L
	12 0 M -4 8 L 12 4 M 0 10 L stroke}
>> matrix makepattern
/Pat8 exch def
<< Tile8x8
 /PaintProc {0.5 setlinewidth pop 0 -2 M 12 4 L
	-4 0 M 12 8 L -4 4 M 8 10 L stroke}
>> matrix makepattern
/Pat9 exch def
/Pattern1 {PatternBgnd KeepColor Pat1 setpattern} bind def
/Pattern2 {PatternBgnd KeepColor Pat2 setpattern} bind def
/Pattern3 {PatternBgnd KeepColor Pat3 setpattern} bind def
/Pattern4 {PatternBgnd KeepColor Landscape {Pat5} {Pat4} ifelse setpattern} bind def
/Pattern5 {PatternBgnd KeepColor Landscape {Pat4} {Pat5} ifelse setpattern} bind def
/Pattern6 {PatternBgnd KeepColor Landscape {Pat9} {Pat6} ifelse setpattern} bind def
/Pattern7 {PatternBgnd KeepColor Landscape {Pat8} {Pat7} ifelse setpattern} bind def
} def
%
%
%End of PostScript Level 2 code
%
/PatternBgnd {
  TransparentPatterns {} {gsave 1 setgray fill grestore} ifelse
} def
%
% Substitute for Level 2 pattern fill codes with
% grayscale if Level 2 support is not selected.
%
/Level1PatternFill {
/Pattern1 {0.250 Density} bind def
/Pattern2 {0.500 Density} bind def
/Pattern3 {0.750 Density} bind def
/Pattern4 {0.125 Density} bind def
/Pattern5 {0.375 Density} bind def
/Pattern6 {0.625 Density} bind def
/Pattern7 {0.875 Density} bind def
} def
%
% Now test for support of Level 2 code
%
Level1 {Level1PatternFill} {Level2PatternFill} ifelse
%
/Symbol-Oblique /Symbol findfont [1 0 .167 1 0 0] makefont
dup length dict begin {1 index /FID eq {pop pop} {def} ifelse} forall
currentdict end definefont pop
%
Level1 SuppressPDFMark or 
{} {
/SDict 10 dict def
systemdict /pdfmark known not {
  userdict /pdfmark systemdict /cleartomark get put
} if
SDict begin [
  /Title (plot_khiIK_cont.tex)
  /Subject (gnuplot plot)
  /Creator (gnuplot 5.0 patchlevel 3)
  /Author (mteper)
%  /Producer (gnuplot)
%  /Keywords ()
  /CreationDate (Tue Apr 13 15:19:11 2021)
  /DOCINFO pdfmark
end
} ifelse
%
% Support for boxed text - Ethan A Merritt May 2005
%
/InitTextBox { userdict /TBy2 3 -1 roll put userdict /TBx2 3 -1 roll put
           userdict /TBy1 3 -1 roll put userdict /TBx1 3 -1 roll put
	   /Boxing true def } def
/ExtendTextBox { Boxing
    { gsave dup false charpath pathbbox
      dup TBy2 gt {userdict /TBy2 3 -1 roll put} {pop} ifelse
      dup TBx2 gt {userdict /TBx2 3 -1 roll put} {pop} ifelse
      dup TBy1 lt {userdict /TBy1 3 -1 roll put} {pop} ifelse
      dup TBx1 lt {userdict /TBx1 3 -1 roll put} {pop} ifelse
      grestore } if } def
/PopTextBox { newpath TBx1 TBxmargin sub TBy1 TBymargin sub M
               TBx1 TBxmargin sub TBy2 TBymargin add L
	       TBx2 TBxmargin add TBy2 TBymargin add L
	       TBx2 TBxmargin add TBy1 TBymargin sub L closepath } def
/DrawTextBox { PopTextBox stroke /Boxing false def} def
/FillTextBox { gsave PopTextBox 1 1 1 setrgbcolor fill grestore /Boxing false def} def
0 0 0 0 InitTextBox
/TBxmargin 20 def
/TBymargin 20 def
/Boxing false def
/textshow { ExtendTextBox Gshow } def
%
% redundant definitions for compatibility with prologue.ps older than 5.0.2
/LTB {BL [] LCb DL} def
/LTb {PL [] LCb DL} def
end
%%EndProlog
%%Page: 1 1
gnudict begin
gsave
doclip
0 0 translate
0.050 0.050 scale
0 setgray
newpath
BackgroundColor 0 lt 3 1 roll 0 lt exch 0 lt or or not {BackgroundColor C 1.000 0 0 7200.00 7560.00 BoxColFill} if
1.000 UL
LTb
LCb setrgbcolor
1260 640 M
63 0 V
5516 0 R
-63 0 V
stroke
LTb
LCb setrgbcolor
1260 1976 M
63 0 V
5516 0 R
-63 0 V
stroke
LTb
LCb setrgbcolor
1260 3312 M
63 0 V
5516 0 R
-63 0 V
stroke
LTb
LCb setrgbcolor
1260 4647 M
63 0 V
5516 0 R
-63 0 V
stroke
LTb
LCb setrgbcolor
1260 5983 M
63 0 V
5516 0 R
-63 0 V
stroke
LTb
LCb setrgbcolor
1260 7319 M
63 0 V
5516 0 R
-63 0 V
stroke
LTb
LCb setrgbcolor
1260 640 M
0 63 V
0 6616 R
0 -63 V
stroke
LTb
LCb setrgbcolor
1880 640 M
0 63 V
0 6616 R
0 -63 V
stroke
LTb
LCb setrgbcolor
2500 640 M
0 63 V
0 6616 R
0 -63 V
stroke
LTb
LCb setrgbcolor
3120 640 M
0 63 V
0 6616 R
0 -63 V
stroke
LTb
LCb setrgbcolor
3740 640 M
0 63 V
0 6616 R
0 -63 V
stroke
LTb
LCb setrgbcolor
4359 640 M
0 63 V
0 6616 R
0 -63 V
stroke
LTb
LCb setrgbcolor
4979 640 M
0 63 V
0 6616 R
0 -63 V
stroke
LTb
LCb setrgbcolor
5599 640 M
0 63 V
0 6616 R
0 -63 V
stroke
LTb
LCb setrgbcolor
6219 640 M
0 63 V
0 6616 R
0 -63 V
stroke
LTb
LCb setrgbcolor
6839 640 M
0 63 V
0 6616 R
0 -63 V
stroke
LTb
LCb setrgbcolor
1.000 UL
LTb
LCb setrgbcolor
1260 7319 N
0 -6679 V
5579 0 V
0 6679 V
-5579 0 V
Z stroke
1.000 UP
1.000 UL
LTb
LCb setrgbcolor
LCb setrgbcolor
LTb
LCb setrgbcolor
LTb
1.500 UP
1.000 UL
LTb
0.58 0.00 0.83 C 5410 2746 M
0 55 V
3827 3921 M
0 58 V
-819 593 R
0 57 V
-760 533 R
0 43 V
-442 114 R
0 59 V
-147 43 R
0 79 V
-108 -49 R
0 87 V
-74 -157 R
0 153 V
-55 -362 R
0 184 V
5410 2773 Circle
3827 3950 Circle
3008 4600 Circle
2248 5184 Circle
1806 5348 Circle
1659 5460 Circle
1551 5494 Circle
1477 5457 Circle
1422 5264 Circle
1.500 UL
LTb
0.58 0.00 0.83 C 1260 5601 M
56 -24 V
57 -24 V
56 -24 V
56 -24 V
57 -24 V
56 -24 V
56 -23 V
57 -24 V
56 -24 V
57 -24 V
56 -24 V
56 -24 V
57 -24 V
56 -24 V
56 -24 V
57 -24 V
56 -24 V
56 -24 V
57 -24 V
56 -24 V
56 -24 V
57 -24 V
56 -24 V
56 -24 V
57 -24 V
56 -23 V
57 -24 V
56 -24 V
56 -24 V
57 -24 V
56 -24 V
56 -24 V
57 -24 V
56 -24 V
56 -24 V
57 -24 V
56 -24 V
56 -24 V
57 -24 V
56 -24 V
56 -24 V
57 -24 V
56 -24 V
57 -24 V
56 -23 V
56 -24 V
57 -24 V
56 -24 V
56 -24 V
57 -24 V
56 -24 V
56 -24 V
57 -24 V
56 -24 V
56 -24 V
57 -24 V
56 -24 V
57 -24 V
56 -24 V
56 -24 V
57 -24 V
56 -24 V
56 -24 V
57 -23 V
56 -24 V
56 -24 V
57 -24 V
56 -24 V
56 -24 V
57 -24 V
56 -24 V
56 -24 V
57 -24 V
56 -24 V
57 -24 V
56 -24 V
56 -24 V
57 -24 V
56 -24 V
56 -24 V
57 -24 V
56 -24 V
56 -23 V
57 -24 V
56 -24 V
56 -24 V
57 -24 V
56 -24 V
56 -24 V
57 -24 V
56 -24 V
57 -24 V
56 -24 V
56 -24 V
57 -24 V
56 -24 V
56 -24 V
57 -24 V
56 -24 V
1.500 UP
stroke
1.000 UL
LTb
0.58 0.00 0.83 C 6244 3263 M
0 126 V
4355 3896 M
0 62 V
-979 52 R
0 135 V
-621 -43 R
0 108 V
2440 3992 M
0 95 V
1948 3957 M
0 302 V
1780 3852 M
0 276 V
1590 3437 M
0 511 V
-78 -429 R
0 727 V
1453 2237 M
0 909 V
6244 3326 CircleF
4355 3927 CircleF
3376 4077 CircleF
2755 4156 CircleF
2440 4040 CircleF
1948 4108 CircleF
1780 3990 CircleF
1590 3692 CircleF
1512 3883 CircleF
1453 2692 CircleF
1.500 UL
LTb
0.58 0.00 0.83 C 1260 3969 M
56 4 V
57 4 V
56 5 V
56 4 V
57 4 V
56 4 V
56 5 V
57 4 V
56 4 V
57 5 V
56 4 V
56 4 V
57 4 V
56 5 V
56 4 V
57 4 V
56 4 V
56 5 V
57 4 V
56 4 V
56 5 V
57 4 V
56 4 V
56 4 V
57 5 V
56 4 V
57 4 V
56 5 V
56 4 V
57 4 V
56 4 V
56 5 V
57 4 V
56 4 V
56 4 V
57 5 V
56 4 V
56 4 V
57 5 V
56 4 V
56 4 V
57 4 V
56 5 V
57 4 V
56 4 V
56 4 V
57 5 V
56 4 V
56 4 V
57 5 V
56 4 V
56 4 V
57 4 V
56 5 V
56 4 V
57 4 V
56 4 V
57 5 V
56 4 V
56 4 V
57 5 V
56 4 V
56 4 V
57 4 V
56 5 V
56 4 V
57 4 V
56 4 V
56 5 V
57 4 V
56 4 V
56 5 V
57 4 V
56 4 V
57 4 V
56 5 V
56 4 V
57 4 V
56 5 V
56 4 V
57 4 V
56 4 V
56 5 V
57 4 V
56 4 V
56 4 V
57 5 V
56 4 V
56 4 V
57 5 V
56 4 V
57 4 V
56 4 V
56 5 V
57 4 V
56 4 V
56 4 V
57 5 V
56 4 V
1.500 UP
stroke
1.000 UL
LTb
0.58 0.00 0.83 C 4089 3789 M
0 49 V
3264 3632 M
0 54 V
2684 3463 M
0 91 V
2281 3191 M
0 293 V
1986 3134 M
0 569 V
1789 2864 M
0 678 V
4089 3813 Dia
3264 3659 Dia
2684 3508 Dia
2281 3338 Dia
1986 3419 Dia
1789 3203 Dia
1.500 UL
LTb
0.58 0.00 0.83 C 1260 3215 M
56 12 V
57 12 V
56 12 V
56 12 V
57 12 V
56 12 V
56 12 V
57 12 V
56 12 V
57 12 V
56 12 V
56 12 V
57 12 V
56 12 V
56 12 V
57 12 V
56 12 V
56 12 V
57 12 V
56 12 V
56 12 V
57 12 V
56 12 V
56 12 V
57 12 V
56 12 V
57 12 V
56 12 V
56 12 V
57 12 V
56 12 V
56 12 V
57 12 V
56 12 V
56 12 V
57 12 V
56 12 V
56 12 V
57 12 V
56 12 V
56 12 V
57 12 V
56 12 V
57 12 V
56 12 V
56 12 V
57 12 V
56 12 V
56 12 V
57 12 V
56 12 V
56 12 V
57 12 V
56 12 V
56 12 V
57 12 V
56 12 V
57 12 V
56 12 V
56 12 V
57 12 V
56 12 V
56 12 V
57 12 V
56 12 V
56 12 V
57 12 V
56 12 V
56 12 V
57 12 V
56 12 V
56 12 V
57 12 V
56 12 V
57 12 V
56 12 V
56 12 V
57 12 V
56 12 V
56 12 V
57 12 V
56 12 V
56 12 V
57 12 V
56 12 V
56 12 V
57 12 V
56 12 V
56 12 V
57 12 V
56 12 V
57 12 V
56 12 V
56 12 V
57 12 V
56 12 V
56 12 V
57 12 V
56 12 V
1.500 UP
stroke
1.000 UL
LTb
0.58 0.00 0.83 C 4111 3603 M
0 68 V
3269 3336 M
0 127 V
2790 3101 M
0 223 V
2455 3057 M
0 562 V
2017 3042 M
0 1843 V
1792 1715 M
0 1713 V
4111 3637 DiaF
3269 3400 DiaF
2790 3212 DiaF
2455 3338 DiaF
2017 3963 DiaF
1792 2571 DiaF
1.500 UL
LTb
0.58 0.00 0.83 C 1260 2825 M
56 16 V
57 16 V
56 16 V
56 16 V
57 17 V
56 16 V
56 16 V
57 16 V
56 16 V
57 16 V
56 16 V
56 16 V
57 16 V
56 16 V
56 16 V
57 16 V
56 16 V
56 16 V
57 16 V
56 16 V
56 16 V
57 16 V
56 16 V
56 16 V
57 16 V
56 16 V
57 16 V
56 16 V
56 16 V
57 16 V
56 16 V
56 16 V
57 16 V
56 16 V
56 16 V
57 16 V
56 16 V
56 16 V
57 17 V
56 16 V
56 16 V
57 16 V
56 16 V
57 16 V
56 16 V
56 16 V
57 16 V
56 16 V
56 16 V
57 16 V
56 16 V
56 16 V
57 16 V
56 16 V
56 16 V
57 16 V
56 16 V
57 16 V
56 16 V
56 16 V
57 16 V
56 16 V
56 16 V
57 16 V
56 16 V
56 16 V
57 16 V
56 16 V
56 16 V
57 16 V
56 16 V
56 16 V
57 17 V
56 16 V
57 16 V
56 16 V
56 16 V
57 16 V
56 16 V
56 16 V
57 16 V
56 16 V
56 16 V
57 16 V
56 16 V
56 16 V
57 16 V
56 16 V
56 16 V
57 16 V
56 16 V
57 16 V
56 16 V
56 16 V
57 16 V
56 16 V
56 16 V
57 16 V
56 16 V
stroke
2.000 UL
LTb
LCb setrgbcolor
1.000 UL
LTb
LCb setrgbcolor
1260 7319 N
0 -6679 V
5579 0 V
0 6679 V
-5579 0 V
Z stroke
1.000 UP
1.000 UL
LTb
LCb setrgbcolor
stroke
grestore
end
showpage
  }}%
  \put(4049,140){\makebox(0,0){\large{$a^2\sigma$}}}%
  \put(200,4979){\makebox(0,0){\Large{$\frac{\chi^{1/4}}{\surd\sigma}$}}}%
  \put(6839,440){\makebox(0,0){\strut{}\ {$0.18$}}}%
  \put(6219,440){\makebox(0,0){\strut{}\ {$0.16$}}}%
  \put(5599,440){\makebox(0,0){\strut{}\ {$0.14$}}}%
  \put(4979,440){\makebox(0,0){\strut{}\ {$0.12$}}}%
  \put(4359,440){\makebox(0,0){\strut{}\ {$0.1$}}}%
  \put(3740,440){\makebox(0,0){\strut{}\ {$0.08$}}}%
  \put(3120,440){\makebox(0,0){\strut{}\ {$0.06$}}}%
  \put(2500,440){\makebox(0,0){\strut{}\ {$0.04$}}}%
  \put(1880,440){\makebox(0,0){\strut{}\ {$0.02$}}}%
  \put(1260,440){\makebox(0,0){\strut{}\ {$0$}}}%
  \put(1140,7319){\makebox(0,0)[r]{\strut{}\ \ {$0.55$}}}%
  \put(1140,5983){\makebox(0,0)[r]{\strut{}\ \ {$0.5$}}}%
  \put(1140,4647){\makebox(0,0)[r]{\strut{}\ \ {$0.45$}}}%
  \put(1140,3312){\makebox(0,0)[r]{\strut{}\ \ {$0.4$}}}%
  \put(1140,1976){\makebox(0,0)[r]{\strut{}\ \ {$0.35$}}}%
  \put(1140,640){\makebox(0,0)[r]{\strut{}\ \ {$0.3$}}}%
\end{picture}%
\endgroup
\endinput

\end	{center}
\caption{Continuum extrapolations of the topological susceptibility in units of the string tension for
  continuum $SU(2)$, $\circ$, $SU(3)$, $\bullet$, $SU(4)$, $\lozenge$, and $SU(5)$, $\blacklozenge$,
  gauge theories.}
\label{fig_khiIK_cont}
\end{figure}




\begin{figure}[htb]
\begin	{center}
\leavevmode
% GNUPLOT: LaTeX picture with Postscript
\begingroup%
\makeatletter%
\newcommand{\GNUPLOTspecial}{%
  \@sanitize\catcode`\%=14\relax\special}%
\setlength{\unitlength}{0.0500bp}%
\begin{picture}(7200,7560)(0,0)%
  {\GNUPLOTspecial{"
%!PS-Adobe-2.0 EPSF-2.0
%%Title: plot_khiK_N.tex
%%Creator: gnuplot 5.0 patchlevel 3
%%CreationDate: Tue Apr 13 09:16:49 2021
%%DocumentFonts: 
%%BoundingBox: 0 0 360 378
%%EndComments
%%BeginProlog
/gnudict 256 dict def
gnudict begin
%
% The following true/false flags may be edited by hand if desired.
% The unit line width and grayscale image gamma correction may also be changed.
%
/Color true def
/Blacktext true def
/Solid false def
/Dashlength 1 def
/Landscape false def
/Level1 false def
/Level3 false def
/Rounded false def
/ClipToBoundingBox false def
/SuppressPDFMark false def
/TransparentPatterns false def
/gnulinewidth 5.000 def
/userlinewidth gnulinewidth def
/Gamma 1.0 def
/BackgroundColor {-1.000 -1.000 -1.000} def
%
/vshift -66 def
/dl1 {
  10.0 Dashlength userlinewidth gnulinewidth div mul mul mul
  Rounded { currentlinewidth 0.75 mul sub dup 0 le { pop 0.01 } if } if
} def
/dl2 {
  10.0 Dashlength userlinewidth gnulinewidth div mul mul mul
  Rounded { currentlinewidth 0.75 mul add } if
} def
/hpt_ 31.5 def
/vpt_ 31.5 def
/hpt hpt_ def
/vpt vpt_ def
/doclip {
  ClipToBoundingBox {
    newpath 0 0 moveto 360 0 lineto 360 378 lineto 0 378 lineto closepath
    clip
  } if
} def
%
% Gnuplot Prolog Version 5.1 (Oct 2015)
%
%/SuppressPDFMark true def
%
/M {moveto} bind def
/L {lineto} bind def
/R {rmoveto} bind def
/V {rlineto} bind def
/N {newpath moveto} bind def
/Z {closepath} bind def
/C {setrgbcolor} bind def
/f {rlineto fill} bind def
/g {setgray} bind def
/Gshow {show} def   % May be redefined later in the file to support UTF-8
/vpt2 vpt 2 mul def
/hpt2 hpt 2 mul def
/Lshow {currentpoint stroke M 0 vshift R 
	Blacktext {gsave 0 setgray textshow grestore} {textshow} ifelse} def
/Rshow {currentpoint stroke M dup stringwidth pop neg vshift R
	Blacktext {gsave 0 setgray textshow grestore} {textshow} ifelse} def
/Cshow {currentpoint stroke M dup stringwidth pop -2 div vshift R 
	Blacktext {gsave 0 setgray textshow grestore} {textshow} ifelse} def
/UP {dup vpt_ mul /vpt exch def hpt_ mul /hpt exch def
  /hpt2 hpt 2 mul def /vpt2 vpt 2 mul def} def
/DL {Color {setrgbcolor Solid {pop []} if 0 setdash}
 {pop pop pop 0 setgray Solid {pop []} if 0 setdash} ifelse} def
/BL {stroke userlinewidth 2 mul setlinewidth
	Rounded {1 setlinejoin 1 setlinecap} if} def
/AL {stroke userlinewidth 2 div setlinewidth
	Rounded {1 setlinejoin 1 setlinecap} if} def
/UL {dup gnulinewidth mul /userlinewidth exch def
	dup 1 lt {pop 1} if 10 mul /udl exch def} def
/PL {stroke userlinewidth setlinewidth
	Rounded {1 setlinejoin 1 setlinecap} if} def
3.8 setmiterlimit
% Classic Line colors (version 5.0)
/LCw {1 1 1} def
/LCb {0 0 0} def
/LCa {0 0 0} def
/LC0 {1 0 0} def
/LC1 {0 1 0} def
/LC2 {0 0 1} def
/LC3 {1 0 1} def
/LC4 {0 1 1} def
/LC5 {1 1 0} def
/LC6 {0 0 0} def
/LC7 {1 0.3 0} def
/LC8 {0.5 0.5 0.5} def
% Default dash patterns (version 5.0)
/LTB {BL [] LCb DL} def
/LTw {PL [] 1 setgray} def
/LTb {PL [] LCb DL} def
/LTa {AL [1 udl mul 2 udl mul] 0 setdash LCa setrgbcolor} def
/LT0 {PL [] LC0 DL} def
/LT1 {PL [2 dl1 3 dl2] LC1 DL} def
/LT2 {PL [1 dl1 1.5 dl2] LC2 DL} def
/LT3 {PL [6 dl1 2 dl2 1 dl1 2 dl2] LC3 DL} def
/LT4 {PL [1 dl1 2 dl2 6 dl1 2 dl2 1 dl1 2 dl2] LC4 DL} def
/LT5 {PL [4 dl1 2 dl2] LC5 DL} def
/LT6 {PL [1.5 dl1 1.5 dl2 1.5 dl1 1.5 dl2 1.5 dl1 6 dl2] LC6 DL} def
/LT7 {PL [3 dl1 3 dl2 1 dl1 3 dl2] LC7 DL} def
/LT8 {PL [2 dl1 2 dl2 2 dl1 6 dl2] LC8 DL} def
/SL {[] 0 setdash} def
/Pnt {stroke [] 0 setdash gsave 1 setlinecap M 0 0 V stroke grestore} def
/Dia {stroke [] 0 setdash 2 copy vpt add M
  hpt neg vpt neg V hpt vpt neg V
  hpt vpt V hpt neg vpt V closepath stroke
  Pnt} def
/Pls {stroke [] 0 setdash vpt sub M 0 vpt2 V
  currentpoint stroke M
  hpt neg vpt neg R hpt2 0 V stroke
 } def
/Box {stroke [] 0 setdash 2 copy exch hpt sub exch vpt add M
  0 vpt2 neg V hpt2 0 V 0 vpt2 V
  hpt2 neg 0 V closepath stroke
  Pnt} def
/Crs {stroke [] 0 setdash exch hpt sub exch vpt add M
  hpt2 vpt2 neg V currentpoint stroke M
  hpt2 neg 0 R hpt2 vpt2 V stroke} def
/TriU {stroke [] 0 setdash 2 copy vpt 1.12 mul add M
  hpt neg vpt -1.62 mul V
  hpt 2 mul 0 V
  hpt neg vpt 1.62 mul V closepath stroke
  Pnt} def
/Star {2 copy Pls Crs} def
/BoxF {stroke [] 0 setdash exch hpt sub exch vpt add M
  0 vpt2 neg V hpt2 0 V 0 vpt2 V
  hpt2 neg 0 V closepath fill} def
/TriUF {stroke [] 0 setdash vpt 1.12 mul add M
  hpt neg vpt -1.62 mul V
  hpt 2 mul 0 V
  hpt neg vpt 1.62 mul V closepath fill} def
/TriD {stroke [] 0 setdash 2 copy vpt 1.12 mul sub M
  hpt neg vpt 1.62 mul V
  hpt 2 mul 0 V
  hpt neg vpt -1.62 mul V closepath stroke
  Pnt} def
/TriDF {stroke [] 0 setdash vpt 1.12 mul sub M
  hpt neg vpt 1.62 mul V
  hpt 2 mul 0 V
  hpt neg vpt -1.62 mul V closepath fill} def
/DiaF {stroke [] 0 setdash vpt add M
  hpt neg vpt neg V hpt vpt neg V
  hpt vpt V hpt neg vpt V closepath fill} def
/Pent {stroke [] 0 setdash 2 copy gsave
  translate 0 hpt M 4 {72 rotate 0 hpt L} repeat
  closepath stroke grestore Pnt} def
/PentF {stroke [] 0 setdash gsave
  translate 0 hpt M 4 {72 rotate 0 hpt L} repeat
  closepath fill grestore} def
/Circle {stroke [] 0 setdash 2 copy
  hpt 0 360 arc stroke Pnt} def
/CircleF {stroke [] 0 setdash hpt 0 360 arc fill} def
/C0 {BL [] 0 setdash 2 copy moveto vpt 90 450 arc} bind def
/C1 {BL [] 0 setdash 2 copy moveto
	2 copy vpt 0 90 arc closepath fill
	vpt 0 360 arc closepath} bind def
/C2 {BL [] 0 setdash 2 copy moveto
	2 copy vpt 90 180 arc closepath fill
	vpt 0 360 arc closepath} bind def
/C3 {BL [] 0 setdash 2 copy moveto
	2 copy vpt 0 180 arc closepath fill
	vpt 0 360 arc closepath} bind def
/C4 {BL [] 0 setdash 2 copy moveto
	2 copy vpt 180 270 arc closepath fill
	vpt 0 360 arc closepath} bind def
/C5 {BL [] 0 setdash 2 copy moveto
	2 copy vpt 0 90 arc
	2 copy moveto
	2 copy vpt 180 270 arc closepath fill
	vpt 0 360 arc} bind def
/C6 {BL [] 0 setdash 2 copy moveto
	2 copy vpt 90 270 arc closepath fill
	vpt 0 360 arc closepath} bind def
/C7 {BL [] 0 setdash 2 copy moveto
	2 copy vpt 0 270 arc closepath fill
	vpt 0 360 arc closepath} bind def
/C8 {BL [] 0 setdash 2 copy moveto
	2 copy vpt 270 360 arc closepath fill
	vpt 0 360 arc closepath} bind def
/C9 {BL [] 0 setdash 2 copy moveto
	2 copy vpt 270 450 arc closepath fill
	vpt 0 360 arc closepath} bind def
/C10 {BL [] 0 setdash 2 copy 2 copy moveto vpt 270 360 arc closepath fill
	2 copy moveto
	2 copy vpt 90 180 arc closepath fill
	vpt 0 360 arc closepath} bind def
/C11 {BL [] 0 setdash 2 copy moveto
	2 copy vpt 0 180 arc closepath fill
	2 copy moveto
	2 copy vpt 270 360 arc closepath fill
	vpt 0 360 arc closepath} bind def
/C12 {BL [] 0 setdash 2 copy moveto
	2 copy vpt 180 360 arc closepath fill
	vpt 0 360 arc closepath} bind def
/C13 {BL [] 0 setdash 2 copy moveto
	2 copy vpt 0 90 arc closepath fill
	2 copy moveto
	2 copy vpt 180 360 arc closepath fill
	vpt 0 360 arc closepath} bind def
/C14 {BL [] 0 setdash 2 copy moveto
	2 copy vpt 90 360 arc closepath fill
	vpt 0 360 arc} bind def
/C15 {BL [] 0 setdash 2 copy vpt 0 360 arc closepath fill
	vpt 0 360 arc closepath} bind def
/Rec {newpath 4 2 roll moveto 1 index 0 rlineto 0 exch rlineto
	neg 0 rlineto closepath} bind def
/Square {dup Rec} bind def
/Bsquare {vpt sub exch vpt sub exch vpt2 Square} bind def
/S0 {BL [] 0 setdash 2 copy moveto 0 vpt rlineto BL Bsquare} bind def
/S1 {BL [] 0 setdash 2 copy vpt Square fill Bsquare} bind def
/S2 {BL [] 0 setdash 2 copy exch vpt sub exch vpt Square fill Bsquare} bind def
/S3 {BL [] 0 setdash 2 copy exch vpt sub exch vpt2 vpt Rec fill Bsquare} bind def
/S4 {BL [] 0 setdash 2 copy exch vpt sub exch vpt sub vpt Square fill Bsquare} bind def
/S5 {BL [] 0 setdash 2 copy 2 copy vpt Square fill
	exch vpt sub exch vpt sub vpt Square fill Bsquare} bind def
/S6 {BL [] 0 setdash 2 copy exch vpt sub exch vpt sub vpt vpt2 Rec fill Bsquare} bind def
/S7 {BL [] 0 setdash 2 copy exch vpt sub exch vpt sub vpt vpt2 Rec fill
	2 copy vpt Square fill Bsquare} bind def
/S8 {BL [] 0 setdash 2 copy vpt sub vpt Square fill Bsquare} bind def
/S9 {BL [] 0 setdash 2 copy vpt sub vpt vpt2 Rec fill Bsquare} bind def
/S10 {BL [] 0 setdash 2 copy vpt sub vpt Square fill 2 copy exch vpt sub exch vpt Square fill
	Bsquare} bind def
/S11 {BL [] 0 setdash 2 copy vpt sub vpt Square fill 2 copy exch vpt sub exch vpt2 vpt Rec fill
	Bsquare} bind def
/S12 {BL [] 0 setdash 2 copy exch vpt sub exch vpt sub vpt2 vpt Rec fill Bsquare} bind def
/S13 {BL [] 0 setdash 2 copy exch vpt sub exch vpt sub vpt2 vpt Rec fill
	2 copy vpt Square fill Bsquare} bind def
/S14 {BL [] 0 setdash 2 copy exch vpt sub exch vpt sub vpt2 vpt Rec fill
	2 copy exch vpt sub exch vpt Square fill Bsquare} bind def
/S15 {BL [] 0 setdash 2 copy Bsquare fill Bsquare} bind def
/D0 {gsave translate 45 rotate 0 0 S0 stroke grestore} bind def
/D1 {gsave translate 45 rotate 0 0 S1 stroke grestore} bind def
/D2 {gsave translate 45 rotate 0 0 S2 stroke grestore} bind def
/D3 {gsave translate 45 rotate 0 0 S3 stroke grestore} bind def
/D4 {gsave translate 45 rotate 0 0 S4 stroke grestore} bind def
/D5 {gsave translate 45 rotate 0 0 S5 stroke grestore} bind def
/D6 {gsave translate 45 rotate 0 0 S6 stroke grestore} bind def
/D7 {gsave translate 45 rotate 0 0 S7 stroke grestore} bind def
/D8 {gsave translate 45 rotate 0 0 S8 stroke grestore} bind def
/D9 {gsave translate 45 rotate 0 0 S9 stroke grestore} bind def
/D10 {gsave translate 45 rotate 0 0 S10 stroke grestore} bind def
/D11 {gsave translate 45 rotate 0 0 S11 stroke grestore} bind def
/D12 {gsave translate 45 rotate 0 0 S12 stroke grestore} bind def
/D13 {gsave translate 45 rotate 0 0 S13 stroke grestore} bind def
/D14 {gsave translate 45 rotate 0 0 S14 stroke grestore} bind def
/D15 {gsave translate 45 rotate 0 0 S15 stroke grestore} bind def
/DiaE {stroke [] 0 setdash vpt add M
  hpt neg vpt neg V hpt vpt neg V
  hpt vpt V hpt neg vpt V closepath stroke} def
/BoxE {stroke [] 0 setdash exch hpt sub exch vpt add M
  0 vpt2 neg V hpt2 0 V 0 vpt2 V
  hpt2 neg 0 V closepath stroke} def
/TriUE {stroke [] 0 setdash vpt 1.12 mul add M
  hpt neg vpt -1.62 mul V
  hpt 2 mul 0 V
  hpt neg vpt 1.62 mul V closepath stroke} def
/TriDE {stroke [] 0 setdash vpt 1.12 mul sub M
  hpt neg vpt 1.62 mul V
  hpt 2 mul 0 V
  hpt neg vpt -1.62 mul V closepath stroke} def
/PentE {stroke [] 0 setdash gsave
  translate 0 hpt M 4 {72 rotate 0 hpt L} repeat
  closepath stroke grestore} def
/CircE {stroke [] 0 setdash 
  hpt 0 360 arc stroke} def
/Opaque {gsave closepath 1 setgray fill grestore 0 setgray closepath} def
/DiaW {stroke [] 0 setdash vpt add M
  hpt neg vpt neg V hpt vpt neg V
  hpt vpt V hpt neg vpt V Opaque stroke} def
/BoxW {stroke [] 0 setdash exch hpt sub exch vpt add M
  0 vpt2 neg V hpt2 0 V 0 vpt2 V
  hpt2 neg 0 V Opaque stroke} def
/TriUW {stroke [] 0 setdash vpt 1.12 mul add M
  hpt neg vpt -1.62 mul V
  hpt 2 mul 0 V
  hpt neg vpt 1.62 mul V Opaque stroke} def
/TriDW {stroke [] 0 setdash vpt 1.12 mul sub M
  hpt neg vpt 1.62 mul V
  hpt 2 mul 0 V
  hpt neg vpt -1.62 mul V Opaque stroke} def
/PentW {stroke [] 0 setdash gsave
  translate 0 hpt M 4 {72 rotate 0 hpt L} repeat
  Opaque stroke grestore} def
/CircW {stroke [] 0 setdash 
  hpt 0 360 arc Opaque stroke} def
/BoxFill {gsave Rec 1 setgray fill grestore} def
/Density {
  /Fillden exch def
  currentrgbcolor
  /ColB exch def /ColG exch def /ColR exch def
  /ColR ColR Fillden mul Fillden sub 1 add def
  /ColG ColG Fillden mul Fillden sub 1 add def
  /ColB ColB Fillden mul Fillden sub 1 add def
  ColR ColG ColB setrgbcolor} def
/BoxColFill {gsave Rec PolyFill} def
/PolyFill {gsave Density fill grestore grestore} def
/h {rlineto rlineto rlineto gsave closepath fill grestore} bind def
%
% PostScript Level 1 Pattern Fill routine for rectangles
% Usage: x y w h s a XX PatternFill
%	x,y = lower left corner of box to be filled
%	w,h = width and height of box
%	  a = angle in degrees between lines and x-axis
%	 XX = 0/1 for no/yes cross-hatch
%
/PatternFill {gsave /PFa [ 9 2 roll ] def
  PFa 0 get PFa 2 get 2 div add PFa 1 get PFa 3 get 2 div add translate
  PFa 2 get -2 div PFa 3 get -2 div PFa 2 get PFa 3 get Rec
  TransparentPatterns {} {gsave 1 setgray fill grestore} ifelse
  clip
  currentlinewidth 0.5 mul setlinewidth
  /PFs PFa 2 get dup mul PFa 3 get dup mul add sqrt def
  0 0 M PFa 5 get rotate PFs -2 div dup translate
  0 1 PFs PFa 4 get div 1 add floor cvi
	{PFa 4 get mul 0 M 0 PFs V} for
  0 PFa 6 get ne {
	0 1 PFs PFa 4 get div 1 add floor cvi
	{PFa 4 get mul 0 2 1 roll M PFs 0 V} for
 } if
  stroke grestore} def
%
/languagelevel where
 {pop languagelevel} {1} ifelse
dup 2 lt
	{/InterpretLevel1 true def
	 /InterpretLevel3 false def}
	{/InterpretLevel1 Level1 def
	 2 gt
	    {/InterpretLevel3 Level3 def}
	    {/InterpretLevel3 false def}
	 ifelse }
 ifelse
%
% PostScript level 2 pattern fill definitions
%
/Level2PatternFill {
/Tile8x8 {/PaintType 2 /PatternType 1 /TilingType 1 /BBox [0 0 8 8] /XStep 8 /YStep 8}
	bind def
/KeepColor {currentrgbcolor [/Pattern /DeviceRGB] setcolorspace} bind def
<< Tile8x8
 /PaintProc {0.5 setlinewidth pop 0 0 M 8 8 L 0 8 M 8 0 L stroke} 
>> matrix makepattern
/Pat1 exch def
<< Tile8x8
 /PaintProc {0.5 setlinewidth pop 0 0 M 8 8 L 0 8 M 8 0 L stroke
	0 4 M 4 8 L 8 4 L 4 0 L 0 4 L stroke}
>> matrix makepattern
/Pat2 exch def
<< Tile8x8
 /PaintProc {0.5 setlinewidth pop 0 0 M 0 8 L
	8 8 L 8 0 L 0 0 L fill}
>> matrix makepattern
/Pat3 exch def
<< Tile8x8
 /PaintProc {0.5 setlinewidth pop -4 8 M 8 -4 L
	0 12 M 12 0 L stroke}
>> matrix makepattern
/Pat4 exch def
<< Tile8x8
 /PaintProc {0.5 setlinewidth pop -4 0 M 8 12 L
	0 -4 M 12 8 L stroke}
>> matrix makepattern
/Pat5 exch def
<< Tile8x8
 /PaintProc {0.5 setlinewidth pop -2 8 M 4 -4 L
	0 12 M 8 -4 L 4 12 M 10 0 L stroke}
>> matrix makepattern
/Pat6 exch def
<< Tile8x8
 /PaintProc {0.5 setlinewidth pop -2 0 M 4 12 L
	0 -4 M 8 12 L 4 -4 M 10 8 L stroke}
>> matrix makepattern
/Pat7 exch def
<< Tile8x8
 /PaintProc {0.5 setlinewidth pop 8 -2 M -4 4 L
	12 0 M -4 8 L 12 4 M 0 10 L stroke}
>> matrix makepattern
/Pat8 exch def
<< Tile8x8
 /PaintProc {0.5 setlinewidth pop 0 -2 M 12 4 L
	-4 0 M 12 8 L -4 4 M 8 10 L stroke}
>> matrix makepattern
/Pat9 exch def
/Pattern1 {PatternBgnd KeepColor Pat1 setpattern} bind def
/Pattern2 {PatternBgnd KeepColor Pat2 setpattern} bind def
/Pattern3 {PatternBgnd KeepColor Pat3 setpattern} bind def
/Pattern4 {PatternBgnd KeepColor Landscape {Pat5} {Pat4} ifelse setpattern} bind def
/Pattern5 {PatternBgnd KeepColor Landscape {Pat4} {Pat5} ifelse setpattern} bind def
/Pattern6 {PatternBgnd KeepColor Landscape {Pat9} {Pat6} ifelse setpattern} bind def
/Pattern7 {PatternBgnd KeepColor Landscape {Pat8} {Pat7} ifelse setpattern} bind def
} def
%
%
%End of PostScript Level 2 code
%
/PatternBgnd {
  TransparentPatterns {} {gsave 1 setgray fill grestore} ifelse
} def
%
% Substitute for Level 2 pattern fill codes with
% grayscale if Level 2 support is not selected.
%
/Level1PatternFill {
/Pattern1 {0.250 Density} bind def
/Pattern2 {0.500 Density} bind def
/Pattern3 {0.750 Density} bind def
/Pattern4 {0.125 Density} bind def
/Pattern5 {0.375 Density} bind def
/Pattern6 {0.625 Density} bind def
/Pattern7 {0.875 Density} bind def
} def
%
% Now test for support of Level 2 code
%
Level1 {Level1PatternFill} {Level2PatternFill} ifelse
%
/Symbol-Oblique /Symbol findfont [1 0 .167 1 0 0] makefont
dup length dict begin {1 index /FID eq {pop pop} {def} ifelse} forall
currentdict end definefont pop
%
Level1 SuppressPDFMark or 
{} {
/SDict 10 dict def
systemdict /pdfmark known not {
  userdict /pdfmark systemdict /cleartomark get put
} if
SDict begin [
  /Title (plot_khiK_N.tex)
  /Subject (gnuplot plot)
  /Creator (gnuplot 5.0 patchlevel 3)
  /Author (mteper)
%  /Producer (gnuplot)
%  /Keywords ()
  /CreationDate (Tue Apr 13 09:16:49 2021)
  /DOCINFO pdfmark
end
} ifelse
%
% Support for boxed text - Ethan A Merritt May 2005
%
/InitTextBox { userdict /TBy2 3 -1 roll put userdict /TBx2 3 -1 roll put
           userdict /TBy1 3 -1 roll put userdict /TBx1 3 -1 roll put
	   /Boxing true def } def
/ExtendTextBox { Boxing
    { gsave dup false charpath pathbbox
      dup TBy2 gt {userdict /TBy2 3 -1 roll put} {pop} ifelse
      dup TBx2 gt {userdict /TBx2 3 -1 roll put} {pop} ifelse
      dup TBy1 lt {userdict /TBy1 3 -1 roll put} {pop} ifelse
      dup TBx1 lt {userdict /TBx1 3 -1 roll put} {pop} ifelse
      grestore } if } def
/PopTextBox { newpath TBx1 TBxmargin sub TBy1 TBymargin sub M
               TBx1 TBxmargin sub TBy2 TBymargin add L
	       TBx2 TBxmargin add TBy2 TBymargin add L
	       TBx2 TBxmargin add TBy1 TBymargin sub L closepath } def
/DrawTextBox { PopTextBox stroke /Boxing false def} def
/FillTextBox { gsave PopTextBox 1 1 1 setrgbcolor fill grestore /Boxing false def} def
0 0 0 0 InitTextBox
/TBxmargin 20 def
/TBymargin 20 def
/Boxing false def
/textshow { ExtendTextBox Gshow } def
%
% redundant definitions for compatibility with prologue.ps older than 5.0.2
/LTB {BL [] LCb DL} def
/LTb {PL [] LCb DL} def
end
%%EndProlog
%%Page: 1 1
gnudict begin
gsave
doclip
0 0 translate
0.050 0.050 scale
0 setgray
newpath
BackgroundColor 0 lt 3 1 roll 0 lt exch 0 lt or or not {BackgroundColor C 1.000 0 0 7200.00 7560.00 BoxColFill} if
1.000 UL
LTb
LCb setrgbcolor
1140 640 M
63 0 V
5636 0 R
-63 0 V
stroke
LTb
LCb setrgbcolor
1140 1475 M
63 0 V
5636 0 R
-63 0 V
stroke
LTb
LCb setrgbcolor
1140 2310 M
63 0 V
5636 0 R
-63 0 V
stroke
LTb
LCb setrgbcolor
1140 3145 M
63 0 V
5636 0 R
-63 0 V
stroke
LTb
LCb setrgbcolor
1140 3980 M
63 0 V
5636 0 R
-63 0 V
stroke
LTb
LCb setrgbcolor
1140 4814 M
63 0 V
5636 0 R
-63 0 V
stroke
LTb
LCb setrgbcolor
1140 5649 M
63 0 V
5636 0 R
-63 0 V
stroke
LTb
LCb setrgbcolor
1140 6484 M
63 0 V
5636 0 R
-63 0 V
stroke
LTb
LCb setrgbcolor
1140 7319 M
63 0 V
5636 0 R
-63 0 V
stroke
LTb
LCb setrgbcolor
1140 640 M
0 63 V
0 6616 R
0 -63 V
stroke
LTb
LCb setrgbcolor
2090 640 M
0 63 V
0 6616 R
0 -63 V
stroke
LTb
LCb setrgbcolor
3040 640 M
0 63 V
0 6616 R
0 -63 V
stroke
LTb
LCb setrgbcolor
3990 640 M
0 63 V
0 6616 R
0 -63 V
stroke
LTb
LCb setrgbcolor
4939 640 M
0 63 V
0 6616 R
0 -63 V
stroke
LTb
LCb setrgbcolor
5889 640 M
0 63 V
0 6616 R
0 -63 V
stroke
LTb
LCb setrgbcolor
6839 640 M
0 63 V
0 6616 R
0 -63 V
stroke
LTb
LCb setrgbcolor
1.000 UL
LTb
LCb setrgbcolor
1140 7319 N
0 -6679 V
5699 0 V
0 6679 V
-5699 0 V
Z stroke
1.000 UP
1.000 UL
LTb
LCb setrgbcolor
LCb setrgbcolor
LTb
LCb setrgbcolor
LTb
1.500 UP
1.000 UL
LTb
0.58 0.00 0.83 C 5870 4683 M
0 24 V
3232 4155 M
0 60 V
2308 3927 M
0 45 V
1881 3777 M
0 101 V
-232 -82 R
0 200 V
5870 4695 CircleF
3232 4185 CircleF
2308 3949 CircleF
1881 3828 CircleF
1649 3896 CircleF
1.500 UL
LTb
0.58 0.00 0.83 C 1140 3713 M
58 12 V
57 12 V
58 12 V
57 12 V
58 12 V
57 12 V
58 12 V
58 12 V
57 11 V
58 12 V
57 12 V
58 12 V
57 12 V
58 12 V
57 12 V
58 12 V
58 12 V
57 12 V
58 12 V
57 11 V
58 12 V
57 12 V
58 12 V
58 12 V
57 12 V
58 12 V
57 12 V
58 12 V
57 12 V
58 12 V
58 12 V
57 11 V
58 12 V
57 12 V
58 12 V
57 12 V
58 12 V
57 12 V
58 12 V
58 12 V
57 12 V
58 12 V
57 12 V
58 11 V
57 12 V
58 12 V
58 12 V
57 12 V
58 12 V
57 12 V
58 12 V
57 12 V
58 12 V
58 12 V
57 12 V
58 11 V
57 12 V
58 12 V
57 12 V
58 12 V
58 12 V
57 12 V
58 12 V
57 12 V
58 12 V
57 12 V
58 12 V
57 11 V
58 12 V
58 12 V
57 12 V
58 12 V
57 12 V
58 12 V
57 12 V
58 12 V
58 12 V
57 12 V
58 12 V
57 11 V
58 12 V
57 12 V
58 12 V
58 12 V
57 12 V
58 12 V
57 12 V
58 12 V
57 12 V
58 12 V
57 12 V
58 11 V
58 12 V
57 12 V
58 12 V
57 12 V
58 12 V
57 12 V
58 12 V
1.500 UP
stroke
1.000 UL
LTb
0.58 0.00 0.83 C 5908 4613 M
0 24 V
3270 4114 M
0 58 V
2346 3896 M
0 42 V
1919 3752 M
0 98 V
-232 -96 R
0 217 V
5908 4625 Circle
3270 4143 Circle
2346 3917 Circle
1919 3801 Circle
1687 3863 Circle
1.500 UL
LTb
0.58 0.00 0.83 C 1140 3691 M
58 12 V
57 11 V
58 11 V
57 12 V
58 11 V
57 11 V
58 12 V
58 11 V
57 11 V
58 12 V
57 11 V
58 11 V
57 12 V
58 11 V
57 11 V
58 12 V
58 11 V
57 11 V
58 12 V
57 11 V
58 11 V
57 12 V
58 11 V
58 11 V
57 12 V
58 11 V
57 11 V
58 12 V
57 11 V
58 11 V
58 12 V
57 11 V
58 11 V
57 12 V
58 11 V
57 11 V
58 12 V
57 11 V
58 11 V
58 12 V
57 11 V
58 11 V
57 12 V
58 11 V
57 12 V
58 11 V
58 11 V
57 12 V
58 11 V
57 11 V
58 12 V
57 11 V
58 11 V
58 12 V
57 11 V
58 11 V
57 12 V
58 11 V
57 11 V
58 12 V
58 11 V
57 11 V
58 12 V
57 11 V
58 11 V
57 12 V
58 11 V
57 11 V
58 12 V
58 11 V
57 11 V
58 12 V
57 11 V
58 11 V
57 12 V
58 11 V
58 11 V
57 12 V
58 11 V
57 11 V
58 12 V
57 11 V
58 11 V
58 12 V
57 11 V
58 11 V
57 12 V
58 11 V
57 11 V
58 12 V
57 11 V
58 11 V
58 12 V
57 11 V
58 11 V
57 12 V
58 11 V
57 11 V
58 12 V
stroke
2.000 UL
LTb
LCb setrgbcolor
1.000 UL
LTb
LCb setrgbcolor
1140 7319 N
0 -6679 V
5699 0 V
0 6679 V
-5699 0 V
Z stroke
1.000 UP
1.000 UL
LTb
LCb setrgbcolor
stroke
grestore
end
showpage
  }}%
  \put(3989,140){\makebox(0,0){\large{$1/N^2$}}}%
  \put(200,4979){\makebox(0,0){\Large{$\frac{\chi^{1/4}}{\surd\sigma}$}}}%
  \put(6839,440){\makebox(0,0){\strut{}\ {$0.3$}}}%
  \put(5889,440){\makebox(0,0){\strut{}\ {$0.25$}}}%
  \put(4939,440){\makebox(0,0){\strut{}\ {$0.2$}}}%
  \put(3990,440){\makebox(0,0){\strut{}\ {$0.15$}}}%
  \put(3040,440){\makebox(0,0){\strut{}\ {$0.1$}}}%
  \put(2090,440){\makebox(0,0){\strut{}\ {$0.05$}}}%
  \put(1140,440){\makebox(0,0){\strut{}\ {$0$}}}%
  \put(1020,7319){\makebox(0,0)[r]{\strut{}\ \ {$0.8$}}}%
  \put(1020,6484){\makebox(0,0)[r]{\strut{}\ \ {$0.7$}}}%
  \put(1020,5649){\makebox(0,0)[r]{\strut{}\ \ {$0.6$}}}%
  \put(1020,4814){\makebox(0,0)[r]{\strut{}\ \ {$0.5$}}}%
  \put(1020,3980){\makebox(0,0)[r]{\strut{}\ \ {$0.4$}}}%
  \put(1020,3145){\makebox(0,0)[r]{\strut{}\ \ {$0.3$}}}%
  \put(1020,2310){\makebox(0,0)[r]{\strut{}\ \ {$0.2$}}}%
  \put(1020,1475){\makebox(0,0)[r]{\strut{}\ \ {$0.1$}}}%
  \put(1020,640){\makebox(0,0)[r]{\strut{}\ \ {$0$}}}%
\end{picture}%
\endgroup
\endinput

\end	{center}
\caption{Topological susceptibility in units of the string tension for
  continuum $SU(N)$ gauge theories with $N=2,3,4,5,6$. For integer valued
  charge, $\bullet$, and for non-integer lattice charge, $\circ$. Points
  shifted slightly for clarity. Lines are extrapolations to $N=\infty$. }
\label{fig_khiK_N}
\end{figure}




\begin{figure}[htb]
\begin	{center}
\leavevmode
% GNUPLOT: LaTeX picture with Postscript
\begingroup%
\makeatletter%
\newcommand{\GNUPLOTspecial}{%
  \@sanitize\catcode`\%=14\relax\special}%
\setlength{\unitlength}{0.0500bp}%
\begin{picture}(7200,7560)(0,0)%
  {\GNUPLOTspecial{"
%!PS-Adobe-2.0 EPSF-2.0
%%Title: plot_ZQ_su8.tex
%%Creator: gnuplot 5.0 patchlevel 3
%%CreationDate: Wed Mar 31 19:43:38 2021
%%DocumentFonts: 
%%BoundingBox: 0 0 360 378
%%EndComments
%%BeginProlog
/gnudict 256 dict def
gnudict begin
%
% The following true/false flags may be edited by hand if desired.
% The unit line width and grayscale image gamma correction may also be changed.
%
/Color true def
/Blacktext true def
/Solid false def
/Dashlength 1 def
/Landscape false def
/Level1 false def
/Level3 false def
/Rounded false def
/ClipToBoundingBox false def
/SuppressPDFMark false def
/TransparentPatterns false def
/gnulinewidth 5.000 def
/userlinewidth gnulinewidth def
/Gamma 1.0 def
/BackgroundColor {-1.000 -1.000 -1.000} def
%
/vshift -66 def
/dl1 {
  10.0 Dashlength userlinewidth gnulinewidth div mul mul mul
  Rounded { currentlinewidth 0.75 mul sub dup 0 le { pop 0.01 } if } if
} def
/dl2 {
  10.0 Dashlength userlinewidth gnulinewidth div mul mul mul
  Rounded { currentlinewidth 0.75 mul add } if
} def
/hpt_ 31.5 def
/vpt_ 31.5 def
/hpt hpt_ def
/vpt vpt_ def
/doclip {
  ClipToBoundingBox {
    newpath 0 0 moveto 360 0 lineto 360 378 lineto 0 378 lineto closepath
    clip
  } if
} def
%
% Gnuplot Prolog Version 5.1 (Oct 2015)
%
%/SuppressPDFMark true def
%
/M {moveto} bind def
/L {lineto} bind def
/R {rmoveto} bind def
/V {rlineto} bind def
/N {newpath moveto} bind def
/Z {closepath} bind def
/C {setrgbcolor} bind def
/f {rlineto fill} bind def
/g {setgray} bind def
/Gshow {show} def   % May be redefined later in the file to support UTF-8
/vpt2 vpt 2 mul def
/hpt2 hpt 2 mul def
/Lshow {currentpoint stroke M 0 vshift R 
	Blacktext {gsave 0 setgray textshow grestore} {textshow} ifelse} def
/Rshow {currentpoint stroke M dup stringwidth pop neg vshift R
	Blacktext {gsave 0 setgray textshow grestore} {textshow} ifelse} def
/Cshow {currentpoint stroke M dup stringwidth pop -2 div vshift R 
	Blacktext {gsave 0 setgray textshow grestore} {textshow} ifelse} def
/UP {dup vpt_ mul /vpt exch def hpt_ mul /hpt exch def
  /hpt2 hpt 2 mul def /vpt2 vpt 2 mul def} def
/DL {Color {setrgbcolor Solid {pop []} if 0 setdash}
 {pop pop pop 0 setgray Solid {pop []} if 0 setdash} ifelse} def
/BL {stroke userlinewidth 2 mul setlinewidth
	Rounded {1 setlinejoin 1 setlinecap} if} def
/AL {stroke userlinewidth 2 div setlinewidth
	Rounded {1 setlinejoin 1 setlinecap} if} def
/UL {dup gnulinewidth mul /userlinewidth exch def
	dup 1 lt {pop 1} if 10 mul /udl exch def} def
/PL {stroke userlinewidth setlinewidth
	Rounded {1 setlinejoin 1 setlinecap} if} def
3.8 setmiterlimit
% Classic Line colors (version 5.0)
/LCw {1 1 1} def
/LCb {0 0 0} def
/LCa {0 0 0} def
/LC0 {1 0 0} def
/LC1 {0 1 0} def
/LC2 {0 0 1} def
/LC3 {1 0 1} def
/LC4 {0 1 1} def
/LC5 {1 1 0} def
/LC6 {0 0 0} def
/LC7 {1 0.3 0} def
/LC8 {0.5 0.5 0.5} def
% Default dash patterns (version 5.0)
/LTB {BL [] LCb DL} def
/LTw {PL [] 1 setgray} def
/LTb {PL [] LCb DL} def
/LTa {AL [1 udl mul 2 udl mul] 0 setdash LCa setrgbcolor} def
/LT0 {PL [] LC0 DL} def
/LT1 {PL [2 dl1 3 dl2] LC1 DL} def
/LT2 {PL [1 dl1 1.5 dl2] LC2 DL} def
/LT3 {PL [6 dl1 2 dl2 1 dl1 2 dl2] LC3 DL} def
/LT4 {PL [1 dl1 2 dl2 6 dl1 2 dl2 1 dl1 2 dl2] LC4 DL} def
/LT5 {PL [4 dl1 2 dl2] LC5 DL} def
/LT6 {PL [1.5 dl1 1.5 dl2 1.5 dl1 1.5 dl2 1.5 dl1 6 dl2] LC6 DL} def
/LT7 {PL [3 dl1 3 dl2 1 dl1 3 dl2] LC7 DL} def
/LT8 {PL [2 dl1 2 dl2 2 dl1 6 dl2] LC8 DL} def
/SL {[] 0 setdash} def
/Pnt {stroke [] 0 setdash gsave 1 setlinecap M 0 0 V stroke grestore} def
/Dia {stroke [] 0 setdash 2 copy vpt add M
  hpt neg vpt neg V hpt vpt neg V
  hpt vpt V hpt neg vpt V closepath stroke
  Pnt} def
/Pls {stroke [] 0 setdash vpt sub M 0 vpt2 V
  currentpoint stroke M
  hpt neg vpt neg R hpt2 0 V stroke
 } def
/Box {stroke [] 0 setdash 2 copy exch hpt sub exch vpt add M
  0 vpt2 neg V hpt2 0 V 0 vpt2 V
  hpt2 neg 0 V closepath stroke
  Pnt} def
/Crs {stroke [] 0 setdash exch hpt sub exch vpt add M
  hpt2 vpt2 neg V currentpoint stroke M
  hpt2 neg 0 R hpt2 vpt2 V stroke} def
/TriU {stroke [] 0 setdash 2 copy vpt 1.12 mul add M
  hpt neg vpt -1.62 mul V
  hpt 2 mul 0 V
  hpt neg vpt 1.62 mul V closepath stroke
  Pnt} def
/Star {2 copy Pls Crs} def
/BoxF {stroke [] 0 setdash exch hpt sub exch vpt add M
  0 vpt2 neg V hpt2 0 V 0 vpt2 V
  hpt2 neg 0 V closepath fill} def
/TriUF {stroke [] 0 setdash vpt 1.12 mul add M
  hpt neg vpt -1.62 mul V
  hpt 2 mul 0 V
  hpt neg vpt 1.62 mul V closepath fill} def
/TriD {stroke [] 0 setdash 2 copy vpt 1.12 mul sub M
  hpt neg vpt 1.62 mul V
  hpt 2 mul 0 V
  hpt neg vpt -1.62 mul V closepath stroke
  Pnt} def
/TriDF {stroke [] 0 setdash vpt 1.12 mul sub M
  hpt neg vpt 1.62 mul V
  hpt 2 mul 0 V
  hpt neg vpt -1.62 mul V closepath fill} def
/DiaF {stroke [] 0 setdash vpt add M
  hpt neg vpt neg V hpt vpt neg V
  hpt vpt V hpt neg vpt V closepath fill} def
/Pent {stroke [] 0 setdash 2 copy gsave
  translate 0 hpt M 4 {72 rotate 0 hpt L} repeat
  closepath stroke grestore Pnt} def
/PentF {stroke [] 0 setdash gsave
  translate 0 hpt M 4 {72 rotate 0 hpt L} repeat
  closepath fill grestore} def
/Circle {stroke [] 0 setdash 2 copy
  hpt 0 360 arc stroke Pnt} def
/CircleF {stroke [] 0 setdash hpt 0 360 arc fill} def
/C0 {BL [] 0 setdash 2 copy moveto vpt 90 450 arc} bind def
/C1 {BL [] 0 setdash 2 copy moveto
	2 copy vpt 0 90 arc closepath fill
	vpt 0 360 arc closepath} bind def
/C2 {BL [] 0 setdash 2 copy moveto
	2 copy vpt 90 180 arc closepath fill
	vpt 0 360 arc closepath} bind def
/C3 {BL [] 0 setdash 2 copy moveto
	2 copy vpt 0 180 arc closepath fill
	vpt 0 360 arc closepath} bind def
/C4 {BL [] 0 setdash 2 copy moveto
	2 copy vpt 180 270 arc closepath fill
	vpt 0 360 arc closepath} bind def
/C5 {BL [] 0 setdash 2 copy moveto
	2 copy vpt 0 90 arc
	2 copy moveto
	2 copy vpt 180 270 arc closepath fill
	vpt 0 360 arc} bind def
/C6 {BL [] 0 setdash 2 copy moveto
	2 copy vpt 90 270 arc closepath fill
	vpt 0 360 arc closepath} bind def
/C7 {BL [] 0 setdash 2 copy moveto
	2 copy vpt 0 270 arc closepath fill
	vpt 0 360 arc closepath} bind def
/C8 {BL [] 0 setdash 2 copy moveto
	2 copy vpt 270 360 arc closepath fill
	vpt 0 360 arc closepath} bind def
/C9 {BL [] 0 setdash 2 copy moveto
	2 copy vpt 270 450 arc closepath fill
	vpt 0 360 arc closepath} bind def
/C10 {BL [] 0 setdash 2 copy 2 copy moveto vpt 270 360 arc closepath fill
	2 copy moveto
	2 copy vpt 90 180 arc closepath fill
	vpt 0 360 arc closepath} bind def
/C11 {BL [] 0 setdash 2 copy moveto
	2 copy vpt 0 180 arc closepath fill
	2 copy moveto
	2 copy vpt 270 360 arc closepath fill
	vpt 0 360 arc closepath} bind def
/C12 {BL [] 0 setdash 2 copy moveto
	2 copy vpt 180 360 arc closepath fill
	vpt 0 360 arc closepath} bind def
/C13 {BL [] 0 setdash 2 copy moveto
	2 copy vpt 0 90 arc closepath fill
	2 copy moveto
	2 copy vpt 180 360 arc closepath fill
	vpt 0 360 arc closepath} bind def
/C14 {BL [] 0 setdash 2 copy moveto
	2 copy vpt 90 360 arc closepath fill
	vpt 0 360 arc} bind def
/C15 {BL [] 0 setdash 2 copy vpt 0 360 arc closepath fill
	vpt 0 360 arc closepath} bind def
/Rec {newpath 4 2 roll moveto 1 index 0 rlineto 0 exch rlineto
	neg 0 rlineto closepath} bind def
/Square {dup Rec} bind def
/Bsquare {vpt sub exch vpt sub exch vpt2 Square} bind def
/S0 {BL [] 0 setdash 2 copy moveto 0 vpt rlineto BL Bsquare} bind def
/S1 {BL [] 0 setdash 2 copy vpt Square fill Bsquare} bind def
/S2 {BL [] 0 setdash 2 copy exch vpt sub exch vpt Square fill Bsquare} bind def
/S3 {BL [] 0 setdash 2 copy exch vpt sub exch vpt2 vpt Rec fill Bsquare} bind def
/S4 {BL [] 0 setdash 2 copy exch vpt sub exch vpt sub vpt Square fill Bsquare} bind def
/S5 {BL [] 0 setdash 2 copy 2 copy vpt Square fill
	exch vpt sub exch vpt sub vpt Square fill Bsquare} bind def
/S6 {BL [] 0 setdash 2 copy exch vpt sub exch vpt sub vpt vpt2 Rec fill Bsquare} bind def
/S7 {BL [] 0 setdash 2 copy exch vpt sub exch vpt sub vpt vpt2 Rec fill
	2 copy vpt Square fill Bsquare} bind def
/S8 {BL [] 0 setdash 2 copy vpt sub vpt Square fill Bsquare} bind def
/S9 {BL [] 0 setdash 2 copy vpt sub vpt vpt2 Rec fill Bsquare} bind def
/S10 {BL [] 0 setdash 2 copy vpt sub vpt Square fill 2 copy exch vpt sub exch vpt Square fill
	Bsquare} bind def
/S11 {BL [] 0 setdash 2 copy vpt sub vpt Square fill 2 copy exch vpt sub exch vpt2 vpt Rec fill
	Bsquare} bind def
/S12 {BL [] 0 setdash 2 copy exch vpt sub exch vpt sub vpt2 vpt Rec fill Bsquare} bind def
/S13 {BL [] 0 setdash 2 copy exch vpt sub exch vpt sub vpt2 vpt Rec fill
	2 copy vpt Square fill Bsquare} bind def
/S14 {BL [] 0 setdash 2 copy exch vpt sub exch vpt sub vpt2 vpt Rec fill
	2 copy exch vpt sub exch vpt Square fill Bsquare} bind def
/S15 {BL [] 0 setdash 2 copy Bsquare fill Bsquare} bind def
/D0 {gsave translate 45 rotate 0 0 S0 stroke grestore} bind def
/D1 {gsave translate 45 rotate 0 0 S1 stroke grestore} bind def
/D2 {gsave translate 45 rotate 0 0 S2 stroke grestore} bind def
/D3 {gsave translate 45 rotate 0 0 S3 stroke grestore} bind def
/D4 {gsave translate 45 rotate 0 0 S4 stroke grestore} bind def
/D5 {gsave translate 45 rotate 0 0 S5 stroke grestore} bind def
/D6 {gsave translate 45 rotate 0 0 S6 stroke grestore} bind def
/D7 {gsave translate 45 rotate 0 0 S7 stroke grestore} bind def
/D8 {gsave translate 45 rotate 0 0 S8 stroke grestore} bind def
/D9 {gsave translate 45 rotate 0 0 S9 stroke grestore} bind def
/D10 {gsave translate 45 rotate 0 0 S10 stroke grestore} bind def
/D11 {gsave translate 45 rotate 0 0 S11 stroke grestore} bind def
/D12 {gsave translate 45 rotate 0 0 S12 stroke grestore} bind def
/D13 {gsave translate 45 rotate 0 0 S13 stroke grestore} bind def
/D14 {gsave translate 45 rotate 0 0 S14 stroke grestore} bind def
/D15 {gsave translate 45 rotate 0 0 S15 stroke grestore} bind def
/DiaE {stroke [] 0 setdash vpt add M
  hpt neg vpt neg V hpt vpt neg V
  hpt vpt V hpt neg vpt V closepath stroke} def
/BoxE {stroke [] 0 setdash exch hpt sub exch vpt add M
  0 vpt2 neg V hpt2 0 V 0 vpt2 V
  hpt2 neg 0 V closepath stroke} def
/TriUE {stroke [] 0 setdash vpt 1.12 mul add M
  hpt neg vpt -1.62 mul V
  hpt 2 mul 0 V
  hpt neg vpt 1.62 mul V closepath stroke} def
/TriDE {stroke [] 0 setdash vpt 1.12 mul sub M
  hpt neg vpt 1.62 mul V
  hpt 2 mul 0 V
  hpt neg vpt -1.62 mul V closepath stroke} def
/PentE {stroke [] 0 setdash gsave
  translate 0 hpt M 4 {72 rotate 0 hpt L} repeat
  closepath stroke grestore} def
/CircE {stroke [] 0 setdash 
  hpt 0 360 arc stroke} def
/Opaque {gsave closepath 1 setgray fill grestore 0 setgray closepath} def
/DiaW {stroke [] 0 setdash vpt add M
  hpt neg vpt neg V hpt vpt neg V
  hpt vpt V hpt neg vpt V Opaque stroke} def
/BoxW {stroke [] 0 setdash exch hpt sub exch vpt add M
  0 vpt2 neg V hpt2 0 V 0 vpt2 V
  hpt2 neg 0 V Opaque stroke} def
/TriUW {stroke [] 0 setdash vpt 1.12 mul add M
  hpt neg vpt -1.62 mul V
  hpt 2 mul 0 V
  hpt neg vpt 1.62 mul V Opaque stroke} def
/TriDW {stroke [] 0 setdash vpt 1.12 mul sub M
  hpt neg vpt 1.62 mul V
  hpt 2 mul 0 V
  hpt neg vpt -1.62 mul V Opaque stroke} def
/PentW {stroke [] 0 setdash gsave
  translate 0 hpt M 4 {72 rotate 0 hpt L} repeat
  Opaque stroke grestore} def
/CircW {stroke [] 0 setdash 
  hpt 0 360 arc Opaque stroke} def
/BoxFill {gsave Rec 1 setgray fill grestore} def
/Density {
  /Fillden exch def
  currentrgbcolor
  /ColB exch def /ColG exch def /ColR exch def
  /ColR ColR Fillden mul Fillden sub 1 add def
  /ColG ColG Fillden mul Fillden sub 1 add def
  /ColB ColB Fillden mul Fillden sub 1 add def
  ColR ColG ColB setrgbcolor} def
/BoxColFill {gsave Rec PolyFill} def
/PolyFill {gsave Density fill grestore grestore} def
/h {rlineto rlineto rlineto gsave closepath fill grestore} bind def
%
% PostScript Level 1 Pattern Fill routine for rectangles
% Usage: x y w h s a XX PatternFill
%	x,y = lower left corner of box to be filled
%	w,h = width and height of box
%	  a = angle in degrees between lines and x-axis
%	 XX = 0/1 for no/yes cross-hatch
%
/PatternFill {gsave /PFa [ 9 2 roll ] def
  PFa 0 get PFa 2 get 2 div add PFa 1 get PFa 3 get 2 div add translate
  PFa 2 get -2 div PFa 3 get -2 div PFa 2 get PFa 3 get Rec
  TransparentPatterns {} {gsave 1 setgray fill grestore} ifelse
  clip
  currentlinewidth 0.5 mul setlinewidth
  /PFs PFa 2 get dup mul PFa 3 get dup mul add sqrt def
  0 0 M PFa 5 get rotate PFs -2 div dup translate
  0 1 PFs PFa 4 get div 1 add floor cvi
	{PFa 4 get mul 0 M 0 PFs V} for
  0 PFa 6 get ne {
	0 1 PFs PFa 4 get div 1 add floor cvi
	{PFa 4 get mul 0 2 1 roll M PFs 0 V} for
 } if
  stroke grestore} def
%
/languagelevel where
 {pop languagelevel} {1} ifelse
dup 2 lt
	{/InterpretLevel1 true def
	 /InterpretLevel3 false def}
	{/InterpretLevel1 Level1 def
	 2 gt
	    {/InterpretLevel3 Level3 def}
	    {/InterpretLevel3 false def}
	 ifelse }
 ifelse
%
% PostScript level 2 pattern fill definitions
%
/Level2PatternFill {
/Tile8x8 {/PaintType 2 /PatternType 1 /TilingType 1 /BBox [0 0 8 8] /XStep 8 /YStep 8}
	bind def
/KeepColor {currentrgbcolor [/Pattern /DeviceRGB] setcolorspace} bind def
<< Tile8x8
 /PaintProc {0.5 setlinewidth pop 0 0 M 8 8 L 0 8 M 8 0 L stroke} 
>> matrix makepattern
/Pat1 exch def
<< Tile8x8
 /PaintProc {0.5 setlinewidth pop 0 0 M 8 8 L 0 8 M 8 0 L stroke
	0 4 M 4 8 L 8 4 L 4 0 L 0 4 L stroke}
>> matrix makepattern
/Pat2 exch def
<< Tile8x8
 /PaintProc {0.5 setlinewidth pop 0 0 M 0 8 L
	8 8 L 8 0 L 0 0 L fill}
>> matrix makepattern
/Pat3 exch def
<< Tile8x8
 /PaintProc {0.5 setlinewidth pop -4 8 M 8 -4 L
	0 12 M 12 0 L stroke}
>> matrix makepattern
/Pat4 exch def
<< Tile8x8
 /PaintProc {0.5 setlinewidth pop -4 0 M 8 12 L
	0 -4 M 12 8 L stroke}
>> matrix makepattern
/Pat5 exch def
<< Tile8x8
 /PaintProc {0.5 setlinewidth pop -2 8 M 4 -4 L
	0 12 M 8 -4 L 4 12 M 10 0 L stroke}
>> matrix makepattern
/Pat6 exch def
<< Tile8x8
 /PaintProc {0.5 setlinewidth pop -2 0 M 4 12 L
	0 -4 M 8 12 L 4 -4 M 10 8 L stroke}
>> matrix makepattern
/Pat7 exch def
<< Tile8x8
 /PaintProc {0.5 setlinewidth pop 8 -2 M -4 4 L
	12 0 M -4 8 L 12 4 M 0 10 L stroke}
>> matrix makepattern
/Pat8 exch def
<< Tile8x8
 /PaintProc {0.5 setlinewidth pop 0 -2 M 12 4 L
	-4 0 M 12 8 L -4 4 M 8 10 L stroke}
>> matrix makepattern
/Pat9 exch def
/Pattern1 {PatternBgnd KeepColor Pat1 setpattern} bind def
/Pattern2 {PatternBgnd KeepColor Pat2 setpattern} bind def
/Pattern3 {PatternBgnd KeepColor Pat3 setpattern} bind def
/Pattern4 {PatternBgnd KeepColor Landscape {Pat5} {Pat4} ifelse setpattern} bind def
/Pattern5 {PatternBgnd KeepColor Landscape {Pat4} {Pat5} ifelse setpattern} bind def
/Pattern6 {PatternBgnd KeepColor Landscape {Pat9} {Pat6} ifelse setpattern} bind def
/Pattern7 {PatternBgnd KeepColor Landscape {Pat8} {Pat7} ifelse setpattern} bind def
} def
%
%
%End of PostScript Level 2 code
%
/PatternBgnd {
  TransparentPatterns {} {gsave 1 setgray fill grestore} ifelse
} def
%
% Substitute for Level 2 pattern fill codes with
% grayscale if Level 2 support is not selected.
%
/Level1PatternFill {
/Pattern1 {0.250 Density} bind def
/Pattern2 {0.500 Density} bind def
/Pattern3 {0.750 Density} bind def
/Pattern4 {0.125 Density} bind def
/Pattern5 {0.375 Density} bind def
/Pattern6 {0.625 Density} bind def
/Pattern7 {0.875 Density} bind def
} def
%
% Now test for support of Level 2 code
%
Level1 {Level1PatternFill} {Level2PatternFill} ifelse
%
/Symbol-Oblique /Symbol findfont [1 0 .167 1 0 0] makefont
dup length dict begin {1 index /FID eq {pop pop} {def} ifelse} forall
currentdict end definefont pop
%
Level1 SuppressPDFMark or 
{} {
/SDict 10 dict def
systemdict /pdfmark known not {
  userdict /pdfmark systemdict /cleartomark get put
} if
SDict begin [
  /Title (plot_ZQ_su8.tex)
  /Subject (gnuplot plot)
  /Creator (gnuplot 5.0 patchlevel 3)
  /Author (mteper)
%  /Producer (gnuplot)
%  /Keywords ()
  /CreationDate (Wed Mar 31 19:43:38 2021)
  /DOCINFO pdfmark
end
} ifelse
%
% Support for boxed text - Ethan A Merritt May 2005
%
/InitTextBox { userdict /TBy2 3 -1 roll put userdict /TBx2 3 -1 roll put
           userdict /TBy1 3 -1 roll put userdict /TBx1 3 -1 roll put
	   /Boxing true def } def
/ExtendTextBox { Boxing
    { gsave dup false charpath pathbbox
      dup TBy2 gt {userdict /TBy2 3 -1 roll put} {pop} ifelse
      dup TBx2 gt {userdict /TBx2 3 -1 roll put} {pop} ifelse
      dup TBy1 lt {userdict /TBy1 3 -1 roll put} {pop} ifelse
      dup TBx1 lt {userdict /TBx1 3 -1 roll put} {pop} ifelse
      grestore } if } def
/PopTextBox { newpath TBx1 TBxmargin sub TBy1 TBymargin sub M
               TBx1 TBxmargin sub TBy2 TBymargin add L
	       TBx2 TBxmargin add TBy2 TBymargin add L
	       TBx2 TBxmargin add TBy1 TBymargin sub L closepath } def
/DrawTextBox { PopTextBox stroke /Boxing false def} def
/FillTextBox { gsave PopTextBox 1 1 1 setrgbcolor fill grestore /Boxing false def} def
0 0 0 0 InitTextBox
/TBxmargin 20 def
/TBymargin 20 def
/Boxing false def
/textshow { ExtendTextBox Gshow } def
%
% redundant definitions for compatibility with prologue.ps older than 5.0.2
/LTB {BL [] LCb DL} def
/LTb {PL [] LCb DL} def
end
%%EndProlog
%%Page: 1 1
gnudict begin
gsave
doclip
0 0 translate
0.050 0.050 scale
0 setgray
newpath
BackgroundColor 0 lt 3 1 roll 0 lt exch 0 lt or or not {BackgroundColor C 1.000 0 0 7200.00 7560.00 BoxColFill} if
1.000 UL
LTb
LCb setrgbcolor
1460 640 M
63 0 V
5316 0 R
-63 0 V
stroke
LTb
LCb setrgbcolor
1460 1475 M
63 0 V
5316 0 R
-63 0 V
stroke
LTb
LCb setrgbcolor
1460 2310 M
63 0 V
5316 0 R
-63 0 V
stroke
LTb
LCb setrgbcolor
1460 3145 M
63 0 V
5316 0 R
-63 0 V
stroke
LTb
LCb setrgbcolor
1460 3980 M
63 0 V
5316 0 R
-63 0 V
stroke
LTb
LCb setrgbcolor
1460 4814 M
63 0 V
5316 0 R
-63 0 V
stroke
LTb
LCb setrgbcolor
1460 5649 M
63 0 V
5316 0 R
-63 0 V
stroke
LTb
LCb setrgbcolor
1460 6484 M
63 0 V
5316 0 R
-63 0 V
stroke
LTb
LCb setrgbcolor
1460 7319 M
63 0 V
5316 0 R
-63 0 V
stroke
LTb
LCb setrgbcolor
1759 640 M
0 63 V
0 6616 R
0 -63 V
stroke
LTb
LCb setrgbcolor
2357 640 M
0 63 V
0 6616 R
0 -63 V
stroke
LTb
LCb setrgbcolor
2954 640 M
0 63 V
0 6616 R
0 -63 V
stroke
LTb
LCb setrgbcolor
3552 640 M
0 63 V
0 6616 R
0 -63 V
stroke
LTb
LCb setrgbcolor
4150 640 M
0 63 V
0 6616 R
0 -63 V
stroke
LTb
LCb setrgbcolor
4747 640 M
0 63 V
0 6616 R
0 -63 V
stroke
LTb
LCb setrgbcolor
5345 640 M
0 63 V
0 6616 R
0 -63 V
stroke
LTb
LCb setrgbcolor
5943 640 M
0 63 V
0 6616 R
0 -63 V
stroke
LTb
LCb setrgbcolor
6540 640 M
0 63 V
0 6616 R
0 -63 V
stroke
LTb
LCb setrgbcolor
1.000 UL
LTb
LCb setrgbcolor
1460 7319 N
0 -6679 V
5379 0 V
0 6679 V
-5379 0 V
Z stroke
1.000 UP
1.000 UL
LTb
LCb setrgbcolor
LCb setrgbcolor
LTb
LCb setrgbcolor
LTb
1.500 UP
1.000 UL
LTb
0.58 0.00 0.83 C 1759 2239 M
0 242 V
598 288 R
0 159 V
597 237 R
0 101 V
598 267 R
0 75 V
598 280 R
0 50 V
597 413 R
0 50 V
598 246 R
0 59 V
598 413 R
0 100 V
597 255 R
0 267 V
1759 2360 BoxF
2357 2848 BoxF
2954 3216 BoxF
3552 3570 BoxF
4150 3913 BoxF
4747 4376 BoxF
5345 4677 BoxF
5943 5169 BoxF
6540 5608 BoxF
1.500 UL
LTb
0.58 0.00 0.83 C 1460 2266 M
54 35 V
55 35 V
54 34 V
54 35 V
55 34 V
54 35 V
54 35 V
55 34 V
54 35 V
54 34 V
55 35 V
54 35 V
54 34 V
55 35 V
54 34 V
54 35 V
55 35 V
54 34 V
54 35 V
55 35 V
54 34 V
54 35 V
55 34 V
54 35 V
54 35 V
55 34 V
54 35 V
54 34 V
55 35 V
54 35 V
54 34 V
55 35 V
54 34 V
54 35 V
55 35 V
54 34 V
54 35 V
55 34 V
54 35 V
54 35 V
55 34 V
54 35 V
54 35 V
55 34 V
54 35 V
54 34 V
55 35 V
54 35 V
54 34 V
55 35 V
54 34 V
54 35 V
55 35 V
54 34 V
54 35 V
55 34 V
54 35 V
54 35 V
55 34 V
54 35 V
54 35 V
55 34 V
54 35 V
54 34 V
55 35 V
54 35 V
54 34 V
55 35 V
54 34 V
54 35 V
55 35 V
54 34 V
54 35 V
55 34 V
54 35 V
54 35 V
55 34 V
54 35 V
54 34 V
55 35 V
54 35 V
54 34 V
55 35 V
54 35 V
54 34 V
55 35 V
54 34 V
54 35 V
55 35 V
54 34 V
54 35 V
55 34 V
54 35 V
54 35 V
55 34 V
54 35 V
54 34 V
55 35 V
54 35 V
1.500 UP
stroke
1.000 UL
LTb
0.58 0.00 0.83 C 2357 2360 M
0 217 V
597 317 R
0 117 V
598 438 R
0 109 V
598 342 R
0 92 V
597 518 R
0 100 V
598 430 R
0 175 V
598 251 R
0 142 V
2357 2468 Circle
2954 2953 Circle
3552 3504 Circle
4150 3946 Circle
4747 4560 Circle
5345 5127 Circle
5943 5537 Circle
1.500 UL
LTb
0.58 0.00 0.83 C 1460 1626 M
54 47 V
55 48 V
54 47 V
54 48 V
55 48 V
54 47 V
54 48 V
55 47 V
54 48 V
54 47 V
55 48 V
54 47 V
54 48 V
55 47 V
54 48 V
54 48 V
55 47 V
54 48 V
54 47 V
55 48 V
54 47 V
54 48 V
55 47 V
54 48 V
54 48 V
55 47 V
54 48 V
54 47 V
55 48 V
54 47 V
54 48 V
55 47 V
54 48 V
54 47 V
55 48 V
54 48 V
54 47 V
55 48 V
54 47 V
54 48 V
55 47 V
54 48 V
54 47 V
55 48 V
54 48 V
54 47 V
55 48 V
54 47 V
54 48 V
55 47 V
54 48 V
54 47 V
55 48 V
54 47 V
54 48 V
55 48 V
54 47 V
54 48 V
55 47 V
54 48 V
54 47 V
55 48 V
54 47 V
54 48 V
55 48 V
54 47 V
54 48 V
55 47 V
54 48 V
54 47 V
55 48 V
54 47 V
54 48 V
55 47 V
54 48 V
54 48 V
55 47 V
54 48 V
54 47 V
55 48 V
54 47 V
54 48 V
55 47 V
54 48 V
54 48 V
55 47 V
54 48 V
54 47 V
55 48 V
54 47 V
54 48 V
55 47 V
54 48 V
54 47 V
55 48 V
54 48 V
54 47 V
55 48 V
54 47 V
1.500 UP
stroke
1.000 UL
LTb
0.58 0.00 0.83 C 1759 1316 M
0 359 V
598 -200 R
0 467 V
597 226 R
0 317 V
598 835 R
0 142 V
598 492 R
0 167 V
597 376 R
0 200 V
598 577 R
0 192 V
598 417 R
0 284 V
1759 1496 CircleF
2357 1709 CircleF
2954 2326 CircleF
3552 3391 CircleF
4150 4038 CircleF
4747 4597 CircleF
5345 5370 CircleF
5943 6025 CircleF
1.500 UL
LTb
0.58 0.00 0.83 C 1460 942 M
54 61 V
55 62 V
54 61 V
54 61 V
55 62 V
54 61 V
54 62 V
55 61 V
54 61 V
54 62 V
55 61 V
54 61 V
54 62 V
55 61 V
54 61 V
54 62 V
55 61 V
54 62 V
54 61 V
55 61 V
54 62 V
54 61 V
55 61 V
54 62 V
54 61 V
55 61 V
54 62 V
54 61 V
55 62 V
54 61 V
54 61 V
55 62 V
54 61 V
54 61 V
55 62 V
54 61 V
54 61 V
55 62 V
54 61 V
54 62 V
55 61 V
54 61 V
54 62 V
55 61 V
54 61 V
54 62 V
55 61 V
54 61 V
54 62 V
55 61 V
54 62 V
54 61 V
55 61 V
54 62 V
54 61 V
55 61 V
54 62 V
54 61 V
55 61 V
54 62 V
54 61 V
55 62 V
54 61 V
54 61 V
55 62 V
54 61 V
54 61 V
55 62 V
54 61 V
54 61 V
55 62 V
54 61 V
54 62 V
55 61 V
54 61 V
54 62 V
55 61 V
54 61 V
54 62 V
55 61 V
54 61 V
54 62 V
55 61 V
54 62 V
54 61 V
55 61 V
54 62 V
54 61 V
55 61 V
54 62 V
54 61 V
55 61 V
54 62 V
54 61 V
55 62 V
54 61 V
54 61 V
55 62 V
54 61 V
stroke
2.000 UL
LTb
LCb setrgbcolor
1.000 UL
LTb
LCb setrgbcolor
1460 7319 N
0 -6679 V
5379 0 V
0 6679 V
-5379 0 V
Z stroke
1.000 UP
1.000 UL
LTb
LCb setrgbcolor
stroke
grestore
end
showpage
  }}%
  \put(4149,140){\makebox(0,0){\large{$Q$}}}%
  \put(160,4979){\makebox(0,0){\Large{$\langle Q_L\rangle_{n_c=0}$}}}%
  \put(6540,440){\makebox(0,0){\strut{}\ {$4$}}}%
  \put(5943,440){\makebox(0,0){\strut{}\ {$3$}}}%
  \put(5345,440){\makebox(0,0){\strut{}\ {$2$}}}%
  \put(4747,440){\makebox(0,0){\strut{}\ {$1$}}}%
  \put(4150,440){\makebox(0,0){\strut{}\ {$0$}}}%
  \put(3552,440){\makebox(0,0){\strut{}\ {$-1$}}}%
  \put(2954,440){\makebox(0,0){\strut{}\ {$-2$}}}%
  \put(2357,440){\makebox(0,0){\strut{}\ {$-3$}}}%
  \put(1759,440){\makebox(0,0){\strut{}\ {$-4$}}}%
  \put(1340,7319){\makebox(0,0)[r]{\strut{}\ \ {$0.8$}}}%
  \put(1340,6484){\makebox(0,0)[r]{\strut{}\ \ {$0.6$}}}%
  \put(1340,5649){\makebox(0,0)[r]{\strut{}\ \ {$0.4$}}}%
  \put(1340,4814){\makebox(0,0)[r]{\strut{}\ \ {$0.2$}}}%
  \put(1340,3980){\makebox(0,0)[r]{\strut{}\ \ {$0$}}}%
  \put(1340,3145){\makebox(0,0)[r]{\strut{}\ \ {$-0.2$}}}%
  \put(1340,2310){\makebox(0,0)[r]{\strut{}\ \ {$-0.4$}}}%
  \put(1340,1475){\makebox(0,0)[r]{\strut{}\ \ {$-0.6$}}}%
  \put(1340,640){\makebox(0,0)[r]{\strut{}\ \ {$-0.8$}}}%
\end{picture}%
\endgroup
\endinput

\end	{center}
\caption{Average value of topological charge on lattice fields that
  have charge $Q$ after 20 cooling sweeps. In $SU(8)$ at $\beta =44.10$, $\blacksquare$,
  $\beta =45.50$, $\circ$, and $\beta =46.70$, $\bullet$. Slope of fits gives the
  renormalisation factor $Z_Q(\beta)$.} 
\label{fig_ZQ_su8}
\end{figure}





\end{document}

