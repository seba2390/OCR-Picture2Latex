\section{Quadrotor Dynamics}
\label{sec: quad_eval}

In the following, we define the dynamics of a quadrotor as a control-affine system. A control-affine system can be represented as:
%
\begin{equation}
\dot{\mathbf{x}} = \mathbf{f}(\mathbf{x}) + \mathbf{g}(\mathbf{x})\mathbf{u}
\end{equation}
%
where $\mathbf{x}$ is the state vector, $\mathbf{u}$ is the control input, and $\mathbf{f}(\mathbf{x})$ and $\mathbf{g}(\mathbf{x})$ are vector fields. The state $\bfx$ is defined as a combination of position, velocity, rotation matrix, and angular velocity:
%
\begin{equation}
\mathbf{x} = [\mathbf{p}, \mathrm{R}, \mathbf{v}, \boldsymbol{\omega}]^\top \in \bbR^{18}
\end{equation}
%
where $\mathbf{p} \in \mathbb{R}^3$ is the position vector, $\mathrm{R} \in \mathbb{R}^{3\times 3}$ is the rotation matrix, $\mathbf{v} \in \mathbb{R}^3$ is the linear velocity vector, and $\boldsymbol{\omega} \in \mathbb{R}^3$ is the angular velocity vector.

The control input $\mathbf{u}$ is defined as:
%
\begin{equation}
\mathbf{u} = [\mathbf{F}, \boldsymbol{\tau}]^\top \in \bbR^4
\end{equation}
%
where $\mathbf{F} \in \mathbb{R}$ is the total thrust force along the $z$-axis, and $\boldsymbol{\tau} \in \mathbb{R}^3$ is the torque vector.

The dynamics of the quadrotor can be written as:

\begin{equation}
\begin{aligned}
\dot{\mathbf{p}} &= \mathbf{v} \\
\dot{\mathrm{R}} &= \mathrm{R} \mathrm{S}(\boldsymbol{\omega}) \\
\dot{\mathbf{v}} &= \frac{1}{m}(\mathrm{R} \mathbf{F}\mathbf{e}_3) - g\mathbf{e}_3 \\
\dot{\boldsymbol{\omega}} &= \mathrm{J}^{-1}(\boldsymbol{\tau} - \boldsymbol{\omega} \times \mathrm{J}\boldsymbol{\omega})
\end{aligned}
\end{equation}
%
where $m$ is the mass of the quadrotor, $\mathrm{J}$ is the inertia matrix, $g$ is the gravitational constant, $\mathbf{e}_3$ is the unit vector along the $z$-axis, $\times$ denotes the cross product between vectors, and $\mathrm{S}(\cdot)$ denotes the skew-symmetric matrix. Thus, we can define the components as follows:
%
\begin{equation}
\mathbf{f}(\mathbf{x}) =
\begin{bmatrix}
\mathbf{v} \\
\mathrm{vec}(\mathrm{R}\mathrm{S}(\boldsymbol{\omega}))\\
-g\mathbf{e}_3 \\
\mathrm{J}^{-1}(- \boldsymbol{\omega} \times \mathrm{J} \boldsymbol{\omega})
\end{bmatrix} \in \bbR^{18},
\mathbf{g}(\mathbf{x}) =
\begin{bmatrix}
\mathbf{0}_{3 \times 4} \\
\mathbf{0}_{9 \times 4} \\
\frac{1}{m}\mathbf{R} \mathbf{e}_3 \otimes \mathbf{e}_1^\top \\
\begin{bmatrix}
\mathbf{0}_{3 \times 1} & \mathrm{J}^{-1}
\end{bmatrix}
\end{bmatrix} \in \bbR^{18 \times 4},
\mathbf{u} =
\begin{bmatrix}
\bfF \\
\boldsymbol{\tau}
\end{bmatrix} \in \bbR^{4}. \notag
\end{equation}
%
where $\otimes$ denotes the Kronecker product, and $\mathbf{e}_1^\top = [1, 0, 0, 0]$. 

