\section{Analytic distance between ellipse and polygon}
\label{sec: analytic_distance}


In this section, we derive an analytic formula for computing the distance between a polygon and an ellipse. From this distance function, we also compute the associated partial derivatives, which enable the formulation of CBFs to ensure safe autonomy.


% For notation, the orientation of a 2D body is given by the parameter $0 \leq \theta < 2\pi$ for counter-clockwise rotation. For convenience, it is also expressed as 
% \begin{equation}
% \label{eq: rotation}
%     \bfR(\theta) = \begin{bmatrix} \cos \theta & -\sin \theta \\ \sin \theta & \cos \theta \end{bmatrix},
% \end{equation}
% and $\bfR$ might be written instead when $\theta$ is understood.

% For an arbitrary elliptical obstacle $\calE(\bfq, \bfR(\theta), a, b)$ in the inertial frame, $\bfq$ is the center of mass, and $\bfR$ denotes the rotation matrix of the ellipse. In its body-fixed frame, $a$ and $b$ are the lengths of the semi-axes of the ellipse along the $x$-axis and $y$-axis, respectively. Similarly, let $\calP( \tilde{\bfq}, \tilde{\bfR}(\tilde{\theta}), \{\tilde{\bfp}_i\}_{i=0}^{M-1})$ denote the polygon of interest where $\Tilde{\bfq}$ as the center of mass and $\Tilde{\bfR}$ as the orientation in the inertial frame. In its fixed-body frame, $\{\tilde{\bfp}_i\}$ are the vertices of the polygonal robot with line segments $\tilde{\bfd}_i = \tilde{\bfp}_{[i+1]_M} - \tilde{\bfp}_i$ for $i = 0, 1, \ldots, M-1$ where $[\cdot]_M$ is the $M$-modulus. 

We consider the mobile robot's body $S(\bfx)$ to be described as a polygon, denoted by $\calP( \tilde{\bfq}, \tilde{\bfR}(\tilde{\theta}), \{\tilde{\bfp}_i\}_{i=0}^{M-1})$. Here, $\Tilde{\bfq}$ denotes the center of mass and $\Tilde{\bfR}$ denotes the orientation in the inertial frame. In its fixed-body frame, $\{\tilde{\bfp}_i\}$ denotes the vertices of the polygonal robot with line segments $\tilde{\bfd}_i = \tilde{\bfp}_{[i+1]_M} - \tilde{\bfp}_i$ for $i = 0, 1, \ldots, M-1$ where $[\cdot]_M$ is the $M$-modulus. 

For convenience, denote $\calE$ and $\calP$ as the bodies in the inertial frame, and we assume their intersection is empty. Now, denote $\calE'$ and $\calP'$ as the respective bodies in the body-fixed frame of the elliptical obstacle. As a result, 
\begin{equation}
\label{eq: distance_frames}
    d(\calE, \calP) = d(\calE',\calP')
\end{equation}
by isometric transformation.
%\NA{Why is this important? I think this is well known.}.
%\NA{I am not sure what the need for this is. The computation in this paragraph, i.e., in eq. (10) is very standard and can just be mentioned directly, i.e., if $\tilde{\bfp}_i$ are the vertices of the robot polygon and $(\bfq,\bfR)$ and $(\tilde{\bfq},\tilde{\bfR})$ are the poses of an obstacle and the robot, then the coordinates of the vertices in the obstacle body frame are $\bfp_i'$.}.
Furthemore, let $\tilde{\bfp}_i$ be a vertex in the robot's frame. Then in the inertial frame, it becomes $\bfp_i = \Tilde{\bfq} + \tilde{\bfR} \Tilde{\bfp}_i$. In the obstacle's frame, it is 
\begin{align}
    \bfp_i' 
        = \bfR^\top(\bfp_i - \bfq)
        = \bfR^\top \tilde{\bfR} \Tilde{\bfp}_i + \bfR^\top(\Tilde{\bfq} -\bfq),
\label{eq: vertices_ellipse_frame}
\end{align}
In short, $\{\tilde{\bfp}_i\}$ are vertices in the robot's frame, $\{\bfp_i\}$ are vertices in the inertial frame, and $\{\bfp_i'\}$ are vertices in the obstacle's frame.

The distance function is
\begin{equation}
\label{eq: polygon_ellipse_1}
        d(\calE', \calP') := \min_{0 \leq i < M} d(\calE', \bfd_i'),
\end{equation}
%
which is the distance between the ellipse $\calE'$ and each line segment $\bfd_i'$. We write each segment as
\begin{equation}
\label{eq: line_segment_i}
    l_i'(\tau) = (1-\tau)\bfp_i' + \tau \bfp_{[i+1]_{M}}',
\end{equation}
for $\tau \in [0,1]$. This further simplifies the function to
\begin{equation}
\label{eq: polygon_line_seg_1}
    d(\calE', \bfd_i') = \min_{\tau \in [0,1]} d(\calE', l_i'(\tau)).
\end{equation}

Now, there are essentially two group of computations for the distance in \eqref{eq: polygon_line_seg_1}: one is the distances between the ellipse $\calE'$ and the endpoints of $\bfd_i'$; the other is the distance between the ellipse $\calE'$ and the infinite line $l_i'(\tau)$ for arbitrary $\tau$ with the caveat that the minimizing argument occurs at $\tau^* \in (0,1)$. The two computations are detailed in the procedures which follow our next proposition.


\begin{proposition}
    Let $\calE'$ be an ellipse and $l_i'$ be a line segment in the frame of the ellipse. Denote $\tau^*$ as the argument of the minimum in~\eqref{eq: polygon_line_seg_1}. Then, the distance
    \begin{equation}
        d(\calE', \bfd_i') = 
        \left\{
            \begin{array}{ll}
            \| \bfp_i' - \underline{\bfp_i'} \|,     &  \text{if } \tau^* = 0, \\
            \| \bfp_{[i+1]_{M}}' - \underline{\bfp_{[i+1]_{M}}'} \|,      & \text{if } \tau^* = 1,  \\
            \| l_i'(\tau^*) - \underline{l_i'(\tau^*)} \|,    & \text{if } \tau^* \in (0,1).
            \end{array}
        \right.
    \end{equation}
The points $\underline{\bfp_i'}$ and $\underline{\bfp_{[i+1]_{M}}'}$ on the ellipse are determined using \textbf{Procedure 1}, and both $l_i'(\tau^*)$ and $\underline{l_i'(\tau^*)}$ (on the ellipse) are determined using \textbf{Procedure 2}.
\label{prop: distance}
\end{proposition}


\begin{comment}
In addition, we will need to compute the necessary partial derivatives\NA{Finish describing the distance computation first. Bring upthe partial derivatives after.} with respect to the state of the ellipse, $(\bfq, \bfR)$  or the state of the polygonal robot $(\tilde{\bfq}, \tilde{\bfR})$. These computations are greatly simplified when a signed distance function (SDF) of the ellipse (in its body-fixed frame) is defined~\cite{Osher2003}. Hence, we focus on the SDF next, with a view towards computing $d(\calE', \bfd_i')$ and its partial derivatives.

Consider the body-fixed frame of the ellipse. The SDF of the ellipse $\psi_\calE: \mathbb{R}^2 \to \mathbb{R}$ is defined as 
\begin{equation}
\label{eq: SDF}
    \psi_\calE(\bfp') 
    = \left\{
    \begin{array}{ll}
        d(\calE',\bfp'), & \text{if } \bfp' \in \calE^c,  \\
        -d(\calE',\bfp'), & \text{if } \bfp' \in \calE.
    \end{array}
    \right.
\end{equation}
Furthermore, $\|\nabla \psi_\calE (\bfp')\| = 1$ for all $\bfp'$ except on the boundary of the ellipse and its center of mass, the origin.
\end{comment}
\begin{comment}
First, we present the formula for computing $d(\calE', \bar{\bfd}_i')$: Define the unit normal of the infinite line of $\bfd_i'$ as
\begin{equation}
\label{eq: unit_normal_di}
    \hat{\bfn}_i' = \frac{1}{\| \bfd_i' \|}(-d'_{i,y},d'_{i,x}).
\end{equation}
Then, we have the formula with dependent variables $(\bfq, \bfR)$:
\begin{equation}
\label{eq: distance_ellipse} 
        d(\calE', \bar{\bfd}_i') := \hat{\bfn}_i'^\top \bfq - \| \bfI_{\calE} \bfR^\top \hat{\bfn}_i'\|,
\end{equation}
where $\bfI_{\calE} = \text{diag}(a,b)$.  The partial derivative with respect to translation $\bfq$ is
\begin{equation}
\label{eq: ellipse_to_line_partials_translation}
   \frac{\partial d}{\partial \bfq} = \left( \frac{\partial d}{\partial x}, \frac{\partial d}{\partial y}\right) = \hat{\bfn}_i'.
\end{equation}
Next, the partial derivative with respect to orientation via counter-clockwise rotation $\theta$ is computed. Recall that 
\[
    \bfR(\theta) = \begin{bmatrix} \cos \theta & -\sin \theta \\ \sin \theta & \cos \theta \end{bmatrix},
\]
so
\begin{equation}
\label{eq: ellipse_to_line_partials_orientation}
\frac{\partial d}{\partial \theta} := \frac{ \hat{\bfn}_i'^\top \frac{\partial \bfR}{\partial \theta} \bfI_{\calE}^2 \bfR\hat{\bfn}_i'}{\| \bfI_{\calE} \bfR \hat{\bfn}_i'\|}.
\end{equation}

First, we derive the formulas for computing $d(\calE, \bar{\bfd})$. We first apply a $SE(2)$ transformation $T_{\bar{\bfd}, \calE}$on the system of the line $\bar{\bfd}$ and the ellipse $\calE$, such that the $\bar{\bfd}$ becomes the $x$-axis and the ellipse lies above the $x$-axis. 

Since the distance is invariant to $SE(2)$ transformations, it can be computed analytically from the transformed ellipse to the $x$-axis using the following formula:
\begin{equation}
\label{eq: distance_ellipse_x_axis} 
        d(\calE, \bar{\bfd}) = d(\calE', \bar{\bfd}') := \bfn^\top \bfq_1 - \| \bfI_{\calE} \bfR_1 \bfn\|
\end{equation}
%
where $\bfn^\top = (0,1)$ and $\bfI_{\calE} = \text{diag}(a,b)$, and $\bfq_1$, $\bfR_1$ denotes the transformed center of mass and rotation matrix, respectively. 

We next compute the partial derivatives with respect to the translations and rotations of the ellipse. We observe that the distance function~\eqref{eq: distance_ellipse_x_axis} does not depend on $x$ values and has a linear dependence on $y$, meaning that 
%
\begin{equation}
\label{eq: ellipse_to_line_partials_1}    \frac{\partial d}{\partial x} = 0, \quad \frac{\partial d}{\partial y} = 1.
\end{equation}
%
Furthermore, the gradient with respect to the orientation $\theta$ can be computed as:
\begin{equation}
\label{eq: distance_grad_ellipse_orientation_1}
\frac{\partial d}{\partial \theta} := \frac{(\bfI_{\calE} \frac{\partial \bfR}{\partial \theta} \bfn)^\top (\bfI_{\calE} \bfR \bfn)}{\| \bfI_{\calE} \bfR \bfn\|}.
\end{equation}
%
By the symmetric property of the ellipse and the $x$-axis, note that the gradients with respect to the translations and rotations of the $x$-axis can be regarded as computing the distance gradients of some transformations on the ellipse.

Here, we show how to compute the distance between the ellipse $\mathcal{E}$ and a general point $\bfp = (p_x, p_y)$, which is in the exterior of the ellipse; we shall compute this in the body-fixed frame: $d(\calE', \bfp)$. Essentially, we are deriving the signed distance function (SDF) for the exterior points of the ellipse ~\cite{Osher2003}. Formally, the SDF $\psi_\calE: \mathbb{R}^2 \to \mathbb{R}$ is defined as 
\begin{equation}
\label{eq: SDF}
    \psi_\calE(\bfp') 
    = \left\{
    \begin{array}{ll}
        d(\calE',\bfp'), & \text{if } \bfp' \in \calE^c,  \\
        -d(\calE',\bfp'), & \text{if } \bfp' \in \calE.
    \end{array}
    \right.
\end{equation}
Furthermore, $\|\nabla \psi_\calE (\bfp)\| = 1$ for all $\bfp$ except on the boundary of the ellipse and its center, the origin. We will later use these tools and facts to derive the partial derivatives.
\end{comment}

\begin{method}
Let $\bfp' = (p_x', p_y')$ be one of the endpoints for the line segment $\bfd_i'$. Recall that the ellipse is defined by its semi-axes along $x$-axis and $y$-axis, denoted by $a$ and $b$, respectively. Then, the points on the ellipse are parameterized by 
\begin{equation}
x(t) = a\cos(t), \quad y(t) = b\sin(t),
\end{equation}
for $0 \leq t \leq 2 \pi$. The goal is to determine the point $(x(t), y(t))$ on the ellipse that is closest to the point $\bfp'$, so it is a minimization problem of the squared Euclidean distance:
%
\begin{equation}
d^2(t) = (p_x' - a\cos(t))^2 + (p_y' - b\sin(t))^2.
\end{equation}
%
To find the minimum distance, we determine the critical point(s) by solving for $0 = \frac{d}{dt} d^2(t)$, which simplified to
%
\begin{equation}
0 = (b^2 - a^2)\cos t \sin t  + a p_x' \sin t - b p_y' \cos t.
\end{equation}
%
Using single-variable optimization, we substitute 
\begin{equation}
    \cos t = \lambda, \quad \sin t = \sqrt{1-\lambda},
\end{equation}
%
and this yields $b p_y' \lambda = \sqrt{1-\lambda^2}((b^2-a^2)\lambda+a p_x')$, which is a quartic equation in $\lambda$. Furthermore, a monic quartic can be derived, which gives the following simplified coefficients:
\begin{equation}
    0 = \lambda^4 + 2m\lambda^3 + (m^2 + n^2 -1) \lambda^2 -2m \lambda -m^2,
\end{equation}
where
%
\begin{equation}
    m = p_x' \frac{a}{b^2 - a^2}, \quad n = p_y' \frac{b}{b^2 - a^2}.
\end{equation}
%
From this point, the real root(s) of the equation can be solved analytically following Cardano's and Ferrari's solution for the quartic equations \cite{cardano2011artis}. 
% (The solution is coded\NA{This is informal. The right word is implemented. Is it necessary to provide this reference?} in \cite{ellipsesdf} with further simplifications using the symmetries of the coefficients.) 
Let $\underline{t}$ be the solution so that $\underline{\bfp'} = (x(\underline{t}),y(\underline{t}))$ is a point on the ellipse and is closest to $\bfp'$. Hence, 
\begin{equation}
\label{eq: ellipse_to_point}
    d(\calE', \bfd_i') = \| \bfp' - \underline{\bfp'} \|
\end{equation}
where $\bfp'$ is either $\bfp_i'$ or $\bfp_{[i+1]_{M}}'$. This concludes \textbf{Procedure 1}.
\end{method}

\begin{method}
We compute the distance between the ellipse $\calE'$ and the infinite line $l_i'(\tau)$ whose minimizing point occurs at $\tau^* \in (0,1)$.

First define the unit normal of the infinite line as
\begin{equation}
\label{eq: unit_normal_di}
    \hat{\bfn}_i' = \frac{1}{\| \bfd_i' \|}(-d'_{i,y},d'_{i,x}).
\end{equation}
Denote $\underline{l_i'(\tau^*)}$ as the point on the ellipse that is closest to the $l_i'(\tau^*)$. In fact, this point $\underline{l_i'(\tau^*)}$ must have a tangent line at the ellipse which is parallel to $l_i'$; this also means the normal at $\underline{l_i'(\tau^*)}$ is $\pm \hat{\bfn}_i'$, so we can compute this point on the ellipse up to a sign:
\begin{equation}
\label{eq:closest_point_to_line}
    \underline{l_i'(\tau^*)} = \pm \frac{I_\epsilon^2 \hat{\bfn}_i'}{\| I_\epsilon \hat{\bfn}_i' \|},
\end{equation}
where $I_\epsilon = \text{diag}(a,b)$. The correct sign is chosen when we are looking at the sign of the constant $C$ in the line equation $Ax + By + C = 0$ of $l_i'$. In particular, 
\begin{equation}
\label{eq:line_equation_constant}
    C = -\hat{\bfn}_i'^\top \bfp_i'.
\end{equation}
If $C > 0$, then $\underline{l_i'(\tau^*)} = -\frac{I_\epsilon^2 \hat{\bfn}_i'}{\| I_\epsilon \hat{\bfn}_i' \|}$; otherwise, if $C < 0$, then $\underline{l_i'(\tau^*)} =  \frac{I_\epsilon^2 \hat{\bfn}_i'}{\| I_\epsilon \hat{\bfn}_i' \|}$.


Finally, we can determine $l_i'(\tau^*)$ on the line segment $\bfd_i'$ using projection:
%
\begin{equation}
\label{eq: ellipse_to_line_closestPoint}
    l_i'(\tau^*) = \bfp_i' + \text{proj}_{\bfd_i'} (\underline{l_i'(\tau^*)} - \bfp_i').
\end{equation}
Here, we are done with \textbf{Procedure 2}. 
\end{method}


We turn our attention to compute the partial derivatives of 
$d(\calE', \calP')$ with respect to either $(\bfq,\bfR)$, the configuration of our obstacle, or $(\tilde{\bfq}, \tilde{\bfR})$, the configuration of the polygonal robot.


Now, in general, both procedures above compute the distance using the Euclidean norm between two unique points: one point $\bfp'$ on a line segment of the robot, and the other $\underline{\bfp'}$ on the ellipse. This is, in fact, equivalent to the SDF of the ellipse evaluated at $\bfp'$ by the uniqueness of these two points. Therefore, let $\bfp' = l_i(\tau^*)$ for some $0 \leq i < M$, then 
\begin{equation}
    d(\calE', \calP') = \psi_\calE(\bfp') = \psi_\calE(l_i(\tau^*)).
\end{equation}
Then, its gradient with respect to $\bfp'$ is
\begin{equation}
\label{eq: ellipse_to_point_gradient}
    \nabla \psi_\calE(\bfp') = \frac{\bfp' - \underline{\bfp'}}{\| \bfp' - \underline{\bfp'} \|}.
\end{equation}
However, notice that $\bfp'$ is a point transformed from the polygonal robot's frame using \eqref{eq: vertices_ellipse_frame}, which depends on both the configurations of the elliptical obstacle and the polygonal robot. Hence the partial derivatives can be computed using the chain rule:


\begin{proposition}
    Let $\calE'$ and $\calP'$ be the bodies of the elliptical obstacle and robot, respectively, in the obstacle's frame. Let $\bfp'$ and $\underline{\bfp'}$ be determined from Proposition \ref{prop: distance}, then 
    \begin{align}
    \frac{\partial d}{\partial \bfq} 
    &= \left( \frac{\partial d}{\partial q_x},\frac{\partial d}{\partial q_y} \right)
    = -\bfR \nabla \psi_\calE(\bfp') \label{eq: ellipse_to_point_partial_1}, \\
    \frac{\partial d}{\partial \bfR} 
    &= \nabla \psi_\calE(\bfp')\otimes (\tilde{\bfR} \tilde{\bfp} + (\tilde{\bfq} - \bfq))\label{eq: ellipse_to_point_partial_2}, \\
    \frac{\partial d}{\partial \tilde{\bfq}} 
    &= \left( \frac{\partial d}{\partial \tilde{q}_x},\frac{\partial d}{\partial \tilde{q}_y} \right)
    = \bfR \nabla \psi_\calE(\bfp') \label{eq: ellipse_to_point_partial_3}, \\
    \frac{\partial d}{\partial \tilde{\bfR}} 
    &= \bfR (\nabla \psi_\calE(\bfp') \otimes \tilde{\bfp}) \label{eq: ellipse_to_point_partial_4}.
\end{align}
Furthermore, Eqs. \eqref{eq: ellipse_to_point_partial_2} and \eqref{eq: ellipse_to_point_partial_4} are derivatives with respect to the rotation matrices; one may compute the derivatives with respect to the rotation angle as
\begin{align}
    \begin{split}
    \frac{\partial d}{\partial \theta} 
        &= \nabla \psi_\calE(\bfp')^\top \left[\frac{\partial \bfR}{\partial \theta}^\top (\tilde{\bfR} \tilde{\bfp} + (\tilde{\bfq} - \bfq)) \right] \\
        &= \text{tr}\left[ \frac{\partial d}{\partial \bfR} \frac{\partial \bfR}{\partial \theta} \right],
    \end{split} \label{eq: ellipse_to_point_partial_2_v2} \\
    \begin{split}
    \frac{\partial d}{\partial \tilde{\theta}}  
        &= \nabla \psi_\calE(\bfp')^\top \left[ \bfR^\top \frac{\partial \tilde{\bfR}}{\partial \tilde{\theta}} \tilde{\bfp} \right] 
        = \text{tr}\left[ \frac{\partial d}{\partial \tilde{\bfR}} \frac{\partial \tilde{\bfR}}{\partial \tilde{\theta}}^\top \right].
    \end{split}
\end{align}
\end{proposition}

\begin{comment}
Consequently, the partial derivatives will be the same as equations \eqref{eq: ellipse_to_point_partial_1}--\eqref{eq: ellipse_to_point_partial_4}, where 
\begin{equation}
\label{eq: ellipse_to_point_gradienSpecial}
    \nabla \psi_\calE(\bfp_{\Delta_i}') = \frac{\bfp_{\Delta_i}' - \underline{\bfp_{\Delta_i}'}}{\| \bfp_{\Delta_i}' - \underline{\bfp_{\Delta_i}'} \|},
\end{equation}
and $\tilde{\bfp}_{\Delta_i} = \tilde{\bfR}^\top \bfR \bfp_{\Delta_i}'  + \tilde{\bfR}^\top(\bfq -\tilde{\bfq})$ in the robot's frame.
\end{comment}

Following both propositions above, we can finally implement CBF using the distance function 
\begin{equation}
\label{eq: distance_ellipse_polygon_final}
    \Phi(\bfq, \bfR, \tilde{\bfq}, \tilde{\bfR}) = d(\calE, \calP) = d(\calE',\calP')
\end{equation}
for the elliptical obstacle $\calE(\bfq, \bfR, a, b)$ and polygonal robot $\calP(\tilde{\bfq}, \tilde{\bfR}, \{\tilde{\bfp}_i\})$. Furthermore, $\Phi$ is differentiable with respect to both $(\bfq, \bfR)$ and $(\tilde{\bfq}, \tilde{\bfR})$.


\begin{comment}

 $\calE(\bfq, \bfR(\theta), a, b)$ in the inertial frame, $\bfq$ is the center of mass, and $\bfR$ denotes the orientation of the ellipse. In its body-fixed frame, $a$ and $b$ are the lengths of the semi-axes of the ellipse along the $x$-axis and $y$-axis, respectively. Similarly, let $\calP( \tilde{\bfq}, \tilde{\bfR}(\tilde{\theta}), \{\tilde{\bfp}_i\}_{i=0}^{M-1})$


Following each case shown above, we can define an analytic distance (collision-checking) function from an ellipse $\calE(\bfq, \bfR, a, b)$ to a polygon-shaped robot $\calP(\bfx)$ with states $\bfx$ based on~\eqref{eq: polygon_ellipse_1}:
\begin{equation}
\label{eq: distance_ellipse_polygon_final}
    \Phi(\bfq, \bfR, a, b, \calP(\bfx)) 
\end{equation}
%
Note that the function $\Phi(\bfq, \bfR, a, b, \bfx)$ is differentiable with respect to $\bfq$ and $\bfR$. In the following, we discuss that the partial derivatives of $\Phi$ with respect to $\bfx$ can also be computed analytically. 

Considering the distance function $\Phi(\bfq, \bfR, \bfx)$ that defines the distance between the polygon $\calP(\bfx)$ and the ellipse $\calE(\bfq, \bfR)$, we can observe that the effect of applying translations and rotations to the polygon can be equivalently represented by applying specific transformations to the ellipse while keeping the distance invariant. Therefore, the gradient of $\Phi$ with respect to $\bfx$ can be expressed using the chain rule:

\begin{equation}
\frac{\partial \Phi}{\partial \bfx} = \frac{\partial \Phi}{\partial \bfq} \frac{\partial \bfq}{\partial \bfx} + \frac{\partial \Phi}{\partial \bfR} \frac{\partial \bfR}{\partial \bfx},
\end{equation}
%
where $\frac{\partial \bfq}{\partial \bfx}$ and $\frac{\partial \bfR}{\partial \bfx}$ represent the gradients of the equivalent transformations applied to the ellipse's position and orientation, respectively, with respect to the changes in the robot state $\bfx$. By evaluating these gradients and combining them with the partial derivatives of $\Phi$ with respect to $\bfq$ and $\bfR$, we can compute the gradient of $\Phi$ with respect to $\bfx$.
\end{comment}




