\section{Evaluation}
\label{sec: evaluate}


% \begin{figure*}[h]
%     \centering
%     % First row
%     \includegraphics[width=0.19\textwidth, trim={0.5cm 0.2cm 2cm 0.0cm},clip]{fig/SE(2)_R2_compare/distance_plot_0.00.png}
%     \includegraphics[width=0.19\textwidth, trim={0.5cm 0.2cm 2cm 0.0cm},clip]{fig/SE(2)_R2_compare/distance_plot_0.79.png}
%     \includegraphics[width=0.19\textwidth, trim={0.5cm 0.2cm 2cm 0.0cm},clip]{fig/SE(2)_R2_compare/distance_plot_1.57.png}
%     \includegraphics[width=0.19\textwidth, trim={0.5cm 0.2cm 2cm 0.0cm},clip]{fig/SE(2)_R2_compare/distance_plot_3.14.png}
%     \includegraphics[width=0.19\textwidth, trim={0.5cm 0.2cm 2cm 0.0cm},clip]{fig/SE(2)_R2_compare/distance_plot_4.71.png}
%     \caption{Comparative analysis of the $SE(2)$ and $\bbR^2$ signed distance functions for elliptical obstacles. The cyan triangle represents the rigid-body robot, with its orientation varying across the sequence. The importance of considering robot orientation in distance computations becomes evident: while the $SE(2)$ function accounts for this orientation, the $\bbR^2$ approximation treats the robot as an encapsulating circle with radius $1$. Level sets at distances $0.2$ and $2$ are depicted for both functions.}
%     \label{fig:se2_r2_compare}
% \end{figure*}

% \begin{figure*}[h]
%     \centering
%     % First row
%     \subcaptionbox{Initial Pose \label{fig:2a}}
%     {\includegraphics[width=0.19\textwidth, trim={3cm 0.2cm 3cm 0.0cm},clip]{fig/unicycle/snapshot_step_0.png}}
%     \subcaptionbox{Time t = 1.66 sec \label{fig:2b}}
%     {\includegraphics[width=0.19\textwidth, trim={3cm 0.2cm 3cm 0.0cm},clip]{fig/unicycle/snapshot_step_83.png}}
%     \subcaptionbox{Time t = 4.12 sec \label{fig:2c}}
%     {\includegraphics[width=0.19\textwidth, trim={3cm 0.2cm 3cm 0.0cm},clip]{fig/unicycle/snapshot_step_206.png}}
%     \subcaptionbox{Final Pose \label{fig:2d}}
%     {\includegraphics[width=0.19\textwidth, trim={3cm 0.2cm 3cm 0.0cm},clip]{fig/unicycle/snapshot_step_305.png}}
%     \subcaptionbox{Circular Robot \label{fig:2e}}
%     {\includegraphics[width=0.19\textwidth, trim={3cm 0.2cm 3cm 0.0cm},clip]{fig/unicycle/circular_snapshot_step_451.png}}
    
%     % Space between rows
%     %\vspace{-1mm}
    
%     % Second row
%     % \includegraphics[width=0.19\textwidth, trim={3cm 0.2cm 3cm 0.0cm},clip]{fig/unicycle/circular_snapshot_0.png}
%     % \includegraphics[width=0.19\textwidth, trim={3cm 0.2cm 3cm 0.0cm},clip]{fig/unicycle/circular_snapshot_40.png}
%     % \includegraphics[width=0.19\textwidth, trim={3cm 0.2cm 3cm 0.0cm},clip]{fig/unicycle/circular_snapshot_82.png}
%     % \includegraphics[width=0.19\textwidth, trim={3cm 0.2cm 3cm 0.0cm},clip]{fig/unicycle/circular_snapshot_124.png}
%     % \includegraphics[width=0.19\textwidth, trim={3cm 0.2cm 3cm 0.0cm},clip]{fig/unicycle/circular_snapshot_176.png}
    
%     \caption{Safe navigation in a dynamical elliptical environment. (a) shows the initial pose of the triangular robot and the environment. (b) shows the triangular robot passing through the narrow space between two moving ellipses. (c) shows the robot adjusts its pose to avoid the moving obstacle. (d) shows the final pose of the robot that reaches the goal region and the current environment. In (e), we plot the trajectory of navigating a circular robot in the same environment.}
%     \label{fig:safe_navigation}
% \end{figure*}

In this section, we show the efficacy of our proposed CBF construction techniques using simulation examples, focusing on ground-robot navigation and 2-D robot arm control.


Fig.~\ref{fig:se2_r2_compare} contrasts the $SE(2)$ distance function with the $\mathbb{R}^2$ counterpart by visualizing their level sets. Our proposed $SE(2)$ approach incorporates the orientation of the rigid-body robot, yielding notably improved results, particularly when the robot is close to obstacles.







To highlight the significance of accurate robot shape representation, we draw a comparison with a baseline circular robot CBF formulation. In Fig.~\ref{fig:safe_navigation}, we compare safe navigation using our proposed $SE(2)$ CBF approach with a regular $\mathbb{R}^2$ CBF approach. For both methods,  we set ${\bfk}(\bfx) = [v_{\max}, 0]^\top$ where $v_{\max} = 3.0$ is the maximum linear velocity. The remaining parameters were $\lambda = 100$, $\alpha_V(V(\bfx)) = 2V(\bfx)$, and $\alpha_h(h(\bfx, t)) = 3 h(\bfx, t)$.

We demonstrate safe navigation to a goal state. In Fig.~\ref{fig:2a}, the triangular robot starts the navigation with position centered at $(0,0)$ and orientation $\theta = \pi / 4$. In Fig.~\ref{fig:2b}, the robot adeptly navigates the narrow passage between two dynamic obstacles. In Fig.~\ref{fig:2d}, we see that the robot is able to reach the goal region without collision. In Fig.~\ref{fig:2e}, when the robot is conservatively modeled as a circle navigating the identical environment, it is evident that the robot has to opt for a more circuitous route to circumvent obstacles. This is due to its inability to traverse certain constricted spaces, as illustrated in Fig.~\ref{fig:2b}. These outcomes underscore the superior performance of our $SE(2)$ CBF methodology. Another advantage of the $SE(2)$ formulation lies in its assurance of a uniformly relative degree of $1$ for the constructed CBF, obviating the need to model a point off the wheel axis~\cite{cortes2017coordinated}.

% \begin{figure*}[h]
%     \centering
%     % First row
%     \subcaptionbox{Initial Pose \label{fig:3a}}
%     {\includegraphics[width=0.19\textwidth, trim={0.5cm 0.2cm 2cm 0.0cm},clip]{fig/robot_arm/arm_snapshot_0.png}}
%     \subcaptionbox{Time t = 4.12 sec \label{fig:3b}}
%     {\includegraphics[width=0.19\textwidth, trim={0.5cm 0.2cm 2cm 0.0cm},clip]{fig/robot_arm/arm_snapshot_206.png}}
%     \subcaptionbox{Time t = 4.92 sec \label{fig:3c}}
%     {\includegraphics[width=0.19\textwidth, trim={0.5cm 0.2cm 2cm 0.0cm},clip]{fig/robot_arm/arm_snapshot_246.png}}
%     \subcaptionbox{Time t = 6.28 sec  \label{fig:3d}}
%     {\includegraphics[width=0.19\textwidth, trim={0.5cm 0.2cm 2cm 0.0cm},clip]{fig/robot_arm/arm_snapshot_314.png}}
%     \subcaptionbox{Final Pose \label{fig:3e}}
%     {\includegraphics[width=0.19\textwidth, trim={0.5cm 0.2cm 2cm 0.0cm},clip]{fig/robot_arm/arm_snapshot_383.png}}
%     \caption{Safe stabilization of a 3-joint robot arm. The green circle denotes the goal region, and the gray box denotes the base of the arm. The arm is shown in blue and the trajectory of its end-effector is shown in red. The trajectories of the moving elliptical obstacles are shown in purple.}
%     \label{fig:robot_arm_safety}
% \end{figure*}

In the following set of experiments, we consider safe stabilization of a 3-joint robot arm in a dynamical elliptical environment. We set ${\bfk}(\bfx) = [0, 0, 0]^\top$ and restrict the joint control bounds with $|\omega_i | \leq 3$. In Fig.~\ref{fig:robot_arm_safety}, the robot arm is able to elude the mobile ellipses by nimbly adjusting its pose. In Fig.~\ref{fig:robot_arm_control_input}, we show the control inputs of each joint over time. We see that when the robot arm is close to the obstacles, it is able to take large control inputs in adjusting its pose. In Fig.~\ref{fig:robot_arm_LF_BF}, we show the CLF and CBF values over time. A consistently positive CBF value throughout the trajectory signifies safety assurance, while the decreasing CLF values indicates the convergence to the desired state. Moreover, the CLF value may increase when the arm is close to obstacles (i.e. CBF value is low), this comes from the relaxation of the CLF-CBF QP to ensure the feasibility of the program. 





\begin{figure}
    \centering
    \includegraphics[width = 0.48\textwidth]{fig/robot_arm/control_input_history_3_links.png}
    \caption{Control input of the 3-joint robot arm. }
    \label{fig:robot_arm_control_input}
\end{figure}

\begin{figure}
    \centering
    \includegraphics[width = 0.48\textwidth]{fig/robot_arm/lyapunov_and_barrier_history_3_links.png}
    \caption{Lyapunov function and barrier function values over time.}
    \label{fig:robot_arm_LF_BF}
\end{figure}