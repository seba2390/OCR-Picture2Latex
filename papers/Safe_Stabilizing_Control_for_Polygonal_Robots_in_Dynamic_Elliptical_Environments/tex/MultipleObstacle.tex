\section{Polygon-Shaped Robot Safe Navigation in Dynamic Ellipse Environments}
\label{sec: multiple_moving_ellipses_env}

In this section, using the distance formula in \eqref{eq: distance_ellipse_polygon_final} and assuming known motion of the ellipse obstacles, we derive time-varying control barrier functions to ensure safety for polygon-shaped robots operating in dynamic elliptical environments. 

\subsection{Time-Varying Control Barrier Function Constraints}

We assume that there are a total of $N$ elliptical obstacles in the environment, each having a rigid-body motion with linear velocity $v_i$ and angular velocity $\omega_i$ around its center of mass. We define a time-varying CBF 
\begin{equation}
\label{eq: tv_cbf_define}
    h(\bfx, t) := \min_{i=1}^{N} \Phi(\bfq_i(t), \bfR_i(t), \tilde{\bfq}, \tilde{\bfR}),
\end{equation}
%
where $\Phi$ is the collision function  and $\bfq_i(t)$ and $\bfR_i(t)$ denotes the position and orientation of the $i$-th ellipse at time $t$, respectively.

Based on the TV-CBF definition, we construct the safety constraint as follows. We first identify the index $k$ corresponding to the minimum collision function:

\begin{equation}
k = \arg\min_{i=1}^{N} \Phi(\bfq_i(t), \bfR_i(t), \tilde{\bfq}, \tilde{\bfR}), 
\end{equation}
%
and we can compute the time derivative of the CBF as:
\begin{equation}
\label{eq: time_derivative}
\frac{\partial h(\bfx, t)}{\partial t} = \frac{\partial \Phi(\bfq_k, \bfR_k)}{\partial \bfq_k} \frac{\partial \bfq_k}{\partial t} + \frac{\partial \Phi(\bfq_k, \bfR_k)}{\partial \bfR_k} \frac{\partial \bfR_k}{\partial t}.
\end{equation}
%
% \begin{proposition}\label{prop:well-defined}
% Let $e_1 := v_i -\tilde{v}_i \in \bbR$ be the error in the approximation of linear velocity $v$, and let $e_2 := \omega_i - \tilde{\omega}_i \in \bbR$ be the error in the approximation of the angular velocity. Assume there are available known bounds,
% %
% % \begin{equation}
% % \label{eq: error_bound}
% % \begin{aligned}
%     $|e_1| \leq e_v , \quad 
%     |e_2|  \leq e_{\omega}$.
% % \end{aligned}
% % \end{equation}
% %
% Let
% \begin{align}
%     \mathcal{K}_{\tilde{h}}(\boldsymbol{x}) &:= \{\bfu \in \mathcal{U} \mid  \mathcal{L}_f h(\bfx, t) + \mathcal{L}_g h(\bfx, t)\bfu + \frac{\partial h(\bfx,t)}{\partial t} + \alpha_h(h(\bfx,t)).  -
%      \label{eq: tvcbc_set}
%     \\
%     & \; \|f(\boldsymbol{x}) + g(\boldsymbol{x})\boldsymbol{u}\|e_{\nabla h}(\boldsymbol{x}) + \alpha_h(\tilde{h}(\boldsymbol{x}) - e_h(\boldsymbol{x})) \geq 0\} .
%     \notag 
% \end{align}
% %is non-empty for all $\boldsymbol{x} \in \mathcal{S}$.
% Then, any locally Lipschitz continuous controller $\bfu: \mathcal{X} \mapsto \calU$ such that $\boldsymbol{u}(\boldsymbol{x}) \in \mathcal{K}_{\tilde{h}}(\boldsymbol{x})$ guarantees that the safe set $\mathcal{S}$ is forward invariant.
% \end{proposition}

\begin{figure*}[h]
    \centering
    % First row
    \includegraphics[width=0.19\textwidth, trim={0.5cm 0.2cm 2cm 0.0cm},clip]{fig/SE2_R2_compare/distance_plot_00.png}
    \includegraphics[width=0.19\textwidth, trim={0.5cm 0.2cm 2cm 0.0cm},clip]{fig/SE2_R2_compare/distance_plot_79.png}
    \includegraphics[width=0.19\textwidth, trim={0.5cm 0.2cm 2cm 0.0cm},clip]{fig/SE2_R2_compare/distance_plot_157.png}
    \includegraphics[width=0.19\textwidth, trim={0.5cm 0.2cm 2cm 0.0cm},clip]{fig/SE2_R2_compare/distance_plot_314.png}
    \includegraphics[width=0.19\textwidth, trim={0.5cm 0.2cm 2cm 0.0cm},clip]{fig/SE2_R2_compare/distance_plot_471.png}
    \caption{Comparative analysis of the $SE(2)$ and $\bbR^2$ signed distance functions for elliptical obstacles. The cyan triangle represents the rigid-body robot, with its orientation varying across the sequence. The importance of considering robot orientation in distance computations becomes evident: while the $SE(2)$ function accounts for this orientation, the $\bbR^2$ approximation treats the robot as an encapsulating circle with radius $1$. Level sets at distances $0.2$ and $2$ are depicted for both functions.}
    \label{fig:se2_r2_compare}
\end{figure*}

% \begin{figure*}[h]
%     \centering
%     % First row
%     \includepdf[pages={1}, scale=0.5, pagecommand={\includegraphics[width=0.19\textwidth, trim={0.5cm 0.2cm 2cm 0.0cm},clip]{fig/SE(2)_R2_compare/distance_plot_00.PNG}}]{}
%     \includepdf[pages={1}, scale=0.5, pagecommand={\includegraphics[width=0.19\textwidth, trim={0.5cm 0.2cm 2cm 0.0cm},clip]{fig/SE(2)_R2_compare/distance_plot_79.PNG}}]{}
%     \includepdf[pages={1}, scale=0.5, pagecommand={\includegraphics[width=0.19\textwidth, trim={0.5cm 0.2cm 2cm 0.0cm},clip]{fig/SE(2)_R2_compare/distance_plot_157.PNG}}]{}
%     \includepdf[pages={1}, scale=0.5, pagecommand={\includegraphics[width=0.19\textwidth, trim={0.5cm 0.2cm 2cm 0.0cm},clip]{fig/SE(2)_R2_compare/distance_plot_314.PNG}}]{}
%     \includepdf[pages={1}, scale=0.5, pagecommand={\includegraphics[width=0.19\textwidth, trim={0.5cm 0.2cm 2cm 0.0cm},clip]{fig/SE(2)_R2_compare/distance_plot_471.PNG}}]{}
%     \caption{Comparative analysis of the $SE(2)$ and $\bbR^2$ signed distance functions for elliptical obstacles. The cyan triangle represents the rigid-body robot, with its orientation varying across the sequence. The importance of considering robot orientation in distance computations becomes evident: while the $SE(2)$ function accounts for this orientation, the $\bbR^2$ approximation treats the robot as an encapsulating circle with radius $1$. Level sets at distances $0.2$ and $2$ are depicted for both functions.}
%     \label{fig:se2_r2_compare}
% \end{figure*}



Now, by utilizing the known motion of the $k$-th ellipse with linear and angular velocity $v_k$ and $\omega_k$, we can express the CBC condition as:
\begin{equation*}
\label{eq:tvcbc_explicit}
\begin{aligned}
\textit{CBC}(\bfx,\bfu, t) & :=
\left[\frac{\partial \Phi(\bfq_k, \bfR_k, \bfx)}{\partial \bfx} \right]^\top F(\bfx)\ubfu + \frac{\partial \Phi(\bfq_k, \bfR_k)}{\partial \bfq_k} v_k \\
&+ \frac{\partial \Phi(\bfq_k, \bfR_k)}{\partial \bfR_k} \frac{\partial \bfR_k}{\partial \theta_k} \omega_k + \alpha_h(\Phi(\bfq_k, \bfR_k, \bfx)) \geq 0.
\end{aligned}
\end{equation*}
%
% Finally, the safety constraint is formulated by ensuring that the CBC condition holds for all $(\bfx, t) \in \calX \times [t_0, t_1]$:
% \begin{equation}\label{eq:tv_cbf}
% \sup_{\bfu\in \mathcal{U}} \textit{CBC}(\bfx,\bfu, t) \geq 0, \quad \forall \; (\bfx,t) \in \calX \times [t_0, t_1].
% \end{equation}


\subsection{Ground-Robot Navigation}

\begin{figure*}[h]
    \centering
    % First row
    \subcaptionbox{Initial Pose \label{fig:2a}}
    {\includegraphics[width=0.19\textwidth, trim={3cm 0.2cm 3cm 0.0cm},clip]{fig/unicycle/snapshot_step_0.png}}
    \subcaptionbox{Time t = 1.66 sec \label{fig:2b}}
    {\includegraphics[width=0.19\textwidth, trim={3cm 0.2cm 3cm 0.0cm},clip]{fig/unicycle/snapshot_step_83.png}}
    \subcaptionbox{Time t = 4.12 sec \label{fig:2c}}
    {\includegraphics[width=0.19\textwidth, trim={3cm 0.2cm 3cm 0.0cm},clip]{fig/unicycle/snapshot_step_206.png}}
    \subcaptionbox{Final Pose \label{fig:2d}}
    {\includegraphics[width=0.19\textwidth, trim={3cm 0.2cm 3cm 0.0cm},clip]{fig/unicycle/snapshot_step_305.png}}
    \subcaptionbox{Circular Robot \label{fig:2e}}
    {\includegraphics[width=0.19\textwidth, trim={3cm 0.2cm 3cm 0.0cm},clip]{fig/unicycle/circular_snapshot_step_451.png}}
    
    % Space between rows
    %\vspace{-1mm}
    
    % Second row
    % \includegraphics[width=0.19\textwidth, trim={3cm 0.2cm 3cm 0.0cm},clip]{fig/unicycle/circular_snapshot_0.png}
    % \includegraphics[width=0.19\textwidth, trim={3cm 0.2cm 3cm 0.0cm},clip]{fig/unicycle/circular_snapshot_40.png}
    % \includegraphics[width=0.19\textwidth, trim={3cm 0.2cm 3cm 0.0cm},clip]{fig/unicycle/circular_snapshot_82.png}
    % \includegraphics[width=0.19\textwidth, trim={3cm 0.2cm 3cm 0.0cm},clip]{fig/unicycle/circular_snapshot_124.png}
    % \includegraphics[width=0.19\textwidth, trim={3cm 0.2cm 3cm 0.0cm},clip]{fig/unicycle/circular_snapshot_176.png}
    
    \caption{Safe navigation in a dynamical elliptical environment. (a) shows the initial pose of the triangular robot and the environment. (b) shows the triangular robot passing through the narrow space between two moving ellipses. (c) shows the robot adjusts its pose to avoid the moving obstacle. (d) shows the final pose of the robot that reaches the goal region and the current environment. In (e), we plot the trajectory of navigating a circular robot in the same environment.}
    \label{fig:safe_navigation}
\end{figure*}

Suppose the robot has a polygonal shape with $\{\tilde{\bfp}_i\}$ denoting the vertices, and the robot motion is governed by unicycle kinematics,
\begin{equation}
\label{eq: unicycle_model}
\begin{bmatrix} \dot{x} \\ \dot{y} \\ \dot{\theta} \end{bmatrix} = \begin{bmatrix}\cos(\theta) &0 \\ \sin(\theta)  &0\\ 0 &1  \end{bmatrix} \begin{bmatrix}
    v \\ \omega
\end{bmatrix},
\end{equation}
%
where $v$, $\omega$ represent the robot linear and angular velocity, respectively. The state and input are $\bfx := [x , y ,\theta]^\top \in \mathbb{R}^2 \times [-\pi,\pi)$, $\bfu := [v,\omega]^\top \in \mathbb{R}^2$. The CLF for the unicycle model is defined as a quadratic form $
V(\bfx) = (\bfx - \bfx^*)^\top \mathbf{Q} (\bfx - \bfx^*)$, where $\bfx^*$ denotes the desired equilibrium and $\bfQ$ is a positive-definite matrix~\cite{unicycle_clf}. We then define the goal region $\calG$ as a disk centered at the 2D position of the desired state $\bfx^*$, with a radius $r$.
 
By writing the robot's position as $ \tilde{\bfq} = [x,y]^\top $ and its orientation via the rotation matrix $ \tilde{\bfR}(\theta) $, we write the shape $S(\bfx)$ of the robot in terms of its state:
\begin{equation}
    S(\bfx) := \text{conv}\{\tilde{\bfq} + \tilde{\bfR}(\theta) \tilde{\bfp_i}\} 
\end{equation}
%
where $\tilde{\bfp_i}$ denotes the vertices of the polygon and $\text{conv}\{ \cdot \}$ denotes the convex hull of points. With this definition, we can derive the CBF for the polygon-shaped unicycle model, as in \eqref{eq: tv_cbf_define}.

% Based on our discussions in Sec.~\ref{sec: analytic_distance}, we have an analytic distance function $\Phi: \calX \mapsto \bbR$, which takes the robot's pose and shape into account. 

% In this section, we show that this distance function is a valid candidate CBF for system dynamics in~\eqref{eq: unicycle_model}. We denote $h(\bfx) := \Phi(\bfx)$. 

% \begin{align}
% \dot{h}(\bfx) = \dot{\Phi}(\bfx) = \nabla \Phi(\bfx)^\top \dot{\bfx} = \nabla \Phi(\bfx)^\top \begin{bmatrix} v  \cos(\theta) \\ v \sin(\theta) \\ \omega \end{bmatrix}. \label{eq: cbf_derivative}
% \end{align}
% %
% In order to ensure safety, we want this rate of change to be nonnegative, i.e., $\dot{h}(\bfx) \geq 0$.

% Now, we consider two cases:

% \begin{itemize}
%     \item When the angle between $\nabla_{\bfq} \Phi(\bfx)$ and the robot's orientation $\theta$ is less than $\pi/2$, this implies $\nabla_{\bfq} \Phi(\bfx)^\top \begin{bmatrix} \cos(\theta) \ \sin(\theta) \end{bmatrix} > 0$. In this case, we can select $v>0$ and $\omega=0$ to ensure $\dot{h} > 0$.
%     \item When the angle between $\nabla_{\bfq} \Phi(\bfx)$ and the robot's orientation $\theta$ is greater than or equal to $\pi/2$, this implies $\nabla_{\bfq} \Phi(\bfx)^\top \begin{bmatrix} \cos(\theta) \ \sin(\theta) \end{bmatrix} \leq 0$. In this case, we can select $v=0$ and $\omega$ such that $\nabla_{\theta} \Phi(\bfx) \omega > 0$, which ensures $\dot{h} > 0$.
% \end{itemize}

% In either case, we can always find a control input $\bfu$ that ensures $\dot{h}(\bfx) > 0$. Hence, we have shown that the CBF defined by the analytic distance function $\Phi$ is a valid candidate for the system dynamics in \eqref{eq: unicycle_model}.



\subsection{K-joint Robot Arm Safe Stabilizing Control}

\begin{figure*}[h]
    \centering
    % First row
    \subcaptionbox{Initial Pose \label{fig:3a}}
    {\includegraphics[width=0.19\textwidth, trim={0.5cm 0.2cm 2cm 0.0cm},clip]{fig/robot_arm/arm_snapshot_0.png}}
    \subcaptionbox{Time t = 4.12 sec \label{fig:3b}}
    {\includegraphics[width=0.19\textwidth, trim={0.5cm 0.2cm 2cm 0.0cm},clip]{fig/robot_arm/arm_snapshot_206.png}}
    \subcaptionbox{Time t = 4.92 sec \label{fig:3c}}
    {\includegraphics[width=0.19\textwidth, trim={0.5cm 0.2cm 2cm 0.0cm},clip]{fig/robot_arm/arm_snapshot_246.png}}
    \subcaptionbox{Time t = 6.28 sec  \label{fig:3d}}
    {\includegraphics[width=0.19\textwidth, trim={0.5cm 0.2cm 2cm 0.0cm},clip]{fig/robot_arm/arm_snapshot_314.png}}
    \subcaptionbox{Final Pose \label{fig:3e}}
    {\includegraphics[width=0.19\textwidth, trim={0.5cm 0.2cm 2cm 0.0cm},clip]{fig/robot_arm/arm_snapshot_383.png}}
    \caption{Safe stabilization of a 3-joint robot arm. The green circle denotes the goal region, and the gray box denotes the base of the arm. The arm is shown in blue and the trajectory of its end-effector is shown in red. The trajectories of the moving elliptical obstacles are shown in purple.}
    \label{fig:robot_arm_safety}
\end{figure*}

In this section, we discuss methods for controlling a 2D K-joint robot arm in a dynamical ellipse environment by utilizing our proposed CBF construction approach. For such robots, the links are intrinsically interconnected due to kinematic chaining. This means that controlling any one link will influence the pose of all subsequent links. 


The dynamics of the robot arm are captured by:
%
\begin{equation}
\label{eq: M_d_arm_dynamics}
    \dot{\boldsymbol{\theta}} = \boldsymbol{\omega},
\end{equation}
%
where \( \boldsymbol{\theta} = [\tilde{\theta}_1, \tilde{\theta}_2, \ldots, \tilde{\theta}_K]^\top \) and \( \boldsymbol{\omega} = [\omega_1, \omega_2, \ldots, \omega_K]^\top \).


For the robot arm, each link has an associated 2D shape, denoted as $ S_i (\boldsymbol{\theta}) $, which depends on the state of the arm. The overall shape of the robot arm, is given by the union of these shapes $S(\boldsymbol{\theta}) = \bigcup_{i=1}^{K} S_i(\boldsymbol{\theta})$. For simplicity, we assume each $S_i $ is a line segment.

For each link \(i\), its state in $SE(2)$ consists of a position $\tilde{\bfq}_i = [x_i,y_i]^\top$: 
\begin{align}
x_i &= x_{i-1} + L_i \cos\left(\sum_{j=1}^{i} \tilde{\theta}_j\right), \\
y_i &= y_{i-1} + L_i \sin\left(\sum_{j=1}^{i} \tilde{\theta}_j\right),
\end{align}
%
and a rotation matrix $\tilde{\bfR}_i$ corresponding to $\underline{\theta_i} := \sum_{j=1}^{i}\tilde{\theta}_j$. For simplicity, we suppose $ x_1 = 0 $ and $ y_1 = 0 $, and $ L_i $ represents the length of the $i$-th link. The robot state can also be represented as multiple $ SE(2) $ configurations corresponding to each link, from $ (\tilde{\mathbf{q}}_1, \tilde{\bfR}_1) $ to $ (\tilde{\mathbf{q}}_{K}, \tilde{\bfR}_{K}) $. Additionally, we denote $\tilde{\mathbf{q}}_{K+1}$ as the end effector.

We define the CLF for the K-joint robot arm as 
%
$
V(\bftheta) = (\bftheta - \bftheta^*)^\top \mathbf{Q} (\bftheta - \bftheta^*),
$
%
where \( \mathbf{Q} \) is a positive-definite matrix, and \( \bftheta^*\) is the desired joint states. The goal region $\calG$ is specified as a disk centered at the position of the end effector corresponding to state $\bftheta^*$, with a defined radius $r$.

For safety assurance, the CBF is constructed using the distance between the robot arm and nearby elliptical obstacles:
%
\begin{equation}
h(\bftheta) = \min_{i = 1, \ldots, N, \; j = 1, \ldots, M} \Phi(\bfq_i, \bfR_i, \tilde{\mathbf{q}}_j, \tilde{\bfR}_j).
\end{equation}

% \NA{It is not clear how the orientation of the $i$-th link in the work space and its polygonal shape are related to the joint angles. This presentation makes it seem that the link orientation is the same as the joint angle, which is not true. I am missing some relationship between link shape, link pose, and joint angles, e.g., given by forward kinematics like Ch. 3.2 here \url{https://www.cse.lehigh.edu/~trink/Courses/RoboticsII/reading/murray-li-sastry-94-complete.pdf}.} \NA{What happens when there are singularities in the forward kinematics? It this CBF relative degree 1? See the discussion on manipulability in Ch. 5 of Lynch and Park.}
%











