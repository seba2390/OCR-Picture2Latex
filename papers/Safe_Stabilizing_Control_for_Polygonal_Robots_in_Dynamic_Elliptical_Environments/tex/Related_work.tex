\section{Related Work}
\label{sec: relate_work}

Obstacle avoidance, in both static and dynamic environments, has persistently been a central issue in robotics. Over the years, various algorithms and methodologies have been proposed to address this challenge.

At the planning level, several motion planning algorithms have been developed to provide a feasible path that ensures obstacle avoidance, including prominent approaches like $\textbf{A}^*$~\cite{A_star_planning},$\textbf{RRT}^*$~\cite{RRT_star}, and their variants~\cite{informed_rrt_star, neural_rrt_star}. However, those approaches typically assume the existence of a low-level tracking controller and may not be applicable in dynamic environments. and may not be applicable in dynamic environments. A significant contribution to the field was made by Khatib~\cite{potential-field}, who introduced artificial potential fields to enable collision avoidance during not only the motion planning stage but also the real-time control of a mobile robot. Later, Rimon and Koditschek \cite{navigation-function} developed navigation functions, a particular form of artificial potential functions. These functions strive to ensure collision avoidance and stabilization towards a goal configuration simultaneously. 
% Meanwhile, Fox \cite{Fox1997TheDW} introduced the dynamics window concept, an influential approach to obstacle avoidance that proactively filters out unsafe control actions. 
In recent years, research has delved into the domain of trajectory generation and optimization, with innovative algorithms proposed for quadrotor safe navigation~\cite{mellinger_snap_2011, zhou2019robust, tordesillas2019faster}. In parallel, the rise of learning-based approaches~\cite{michels2005high, pfeiffer2018reinforced, loquercio2021learning} has added a new direction to the field, utilizing machine learning to facilitate both planning and real-time obstacle avoidance. Despite their promise, these methods often face challenges in dynamic environments and in providing safety guarantees.


In the field of safe control synthesis, integrating control Lyapunov functions (CLFs) and control barrier functions (CBFs) into a quadratic program (QP) has proven to be a reliable and efficient strategy for formulating safe stabilizing controls across a wide array of robotic tasks~\cite{glotfelter2017nonsmooth, grandia_2021_legged, wang2017_aerial}. While CBF-based methodologies have been deployed for obstacle avoidance~\cite{srinivasan2020synthesis, Long_learningcbf_ral21, almubarak2022safety, dawson2022learning, abdi2023safe}, such strategies typically simplify the robot as a point or circle and assume static environments when constructing CBFs for control synthesis. Some recent advances have also explored the use of time-varying CBFs to facilitate safe control in dynamic environments~\cite{he2021rule, molnar2022safety, hamdipoor2023safe}. However, this concept has yet to be thoroughly investigated in the context of obstacle avoidance for rigid-body robots. For the safe autonomy of robot arms, Koptev \textit{et al}.\cite{Koptev2023_neural_joint_control} introduced a neural network approach to approximate the signed distance function of a robot arm and use it for safe reactive control in dynamical environment. 




Researchers~\cite{ding2022configurationaware} proposed a configuration-aware control approach for the robot arm by integrating geometric restrictions with Control Barrier Functions. 



Mostly related to our work, Thirugnanam \textit{et al}. \cite{discrete_polytope_cbf} introduced a discrete CBF constraint between polytopes and further incorporated the constraint in a model predictive control framework to enable safe navigation. The authors \cite{polytopic_cbf} also extended the formulation for continuous-time systems but the CBF computation between polytopes still remained numerical, requiring a duality-based formulation with non-smooth CBFs. 




\textbf{Contributions}: 1) We present an analytic distance formula in $SE(2)$ for elliptical and polygonal objects, enabling closed-form calculations for distance and its gradient. 2) We introduce a novel time-varying control barrier function tailored for robots described by one or multiple $SE(2)$ configurations based on the analytic distance. Its efficacy is validated through applications in ground robot navigation and multi-link robot arm demonstrations.












