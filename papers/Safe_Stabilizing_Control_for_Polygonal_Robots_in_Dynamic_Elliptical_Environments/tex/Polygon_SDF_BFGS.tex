\section{Collision-Checking Function Construction in Non-Convex Environments}
\label{sec: method}

In this section, we describe our approach to defining a collision-checking function for the system~\eqref{eq: dynamic} using an estimated environment Signed Distance Function (SDF) model $\tilde{\varphi}$ and a polygonal robot shape.

We define a polygon $\bfP(\bfq, \theta)$ for the robot state $\bfx = [\bfq, \theta]^\top$, representing the 2-dimensional physical domain associated with the robot state. The polygon is specified with $M$ edges $\{l_i\}_{i=1}^M$ and $M$ vertices $\{\bfp_i\}_{i=1}^M$.


We assume a Gaussian Processes (GP) estimation of the SDF environment, i.e. $\varphi(\bfq) \sim \mathcal{GP}(\tilde{\varphi}(\bfq),K_\varphi(\bfq,\bfq'))$, which can be obtained using \cite{gpis_lee}. 

Given the polygon $\bfP(\bfq, \theta)$ and the estimated continuously differentiable SDF $\tilde{\varphi}(\bfq)$, we compute a collision-checking function $\Phi: SE(2) \mapsto \bbR$ using the optimization program:

%
\begin{equation}
\label{eq: optimize_collision_function}
    \Phi(\bfq, \theta) := \min_{\bfy \in \bfP(\bfq, \theta)} \tilde{\varphi}(\bfy).
\end{equation}
%
Since the polygon is convex, we can restrict the search to the $M$ edges of the polygon, i.e, 
\begin{equation}
\label{eq: optimize_collision_edges}
    \Phi(\bfq, \theta) := \min_{\bfy \in \bfP(\bfq, \theta)} \tilde{\varphi}(\bfy) = \min_{i} \min_{\bfy \in l_i} \varphi(\bfy).
\end{equation}

However, due to the non-convexity of the SDF $\tilde{\varphi}$, it is not possible to obtain a closed-form expression for the optimizer in \eqref{eq: optimize_collision_edges}.
Therefore, we apply the Broyden–Fletcher–Goldfarb–Shanno (BFGS) \cite{NoceWrig06, Fletcher1988PracticalMO} algorithm to get the optimal solution. To ensure that the optimal value obtained from the BFGS algorithm is the global optimal solution, we apply the BFGS algorithm with $N = 10$ uniform  initial guesses on each edge $l_i$.

However, due to the non-convexity of the environmental SDF $\tilde{\varphi}$, the collision-checking function $\Phi$ may be non-differentiable, and the theory of CBF in Definition.~\ref{def: cbf} cannot be directly used.




% \subsection{SDF to CBF}

% This section shows that the constructed nonsmooth collision checking function $\Phi: SE(2) \mapsto \bbR$ is a valid candidate NCBF for the robotic system of a relative-degree one. 











