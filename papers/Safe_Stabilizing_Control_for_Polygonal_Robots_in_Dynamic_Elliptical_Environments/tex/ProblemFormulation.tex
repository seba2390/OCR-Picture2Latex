\section{Problem Formulation}\label{sec:problem}

%\subsection{Notations}
% \footnotetext[1]{\textbf{Notation. }
% The sets of non-negative real and natural numbers are denoted $\bbR_{\geq 0}$ and $\bbN$. For $N \in \bbN$, $[N] := \{1,2, \dots N\}$. The orientation of a 2D body is denoted by $0 \leq \theta < 2\pi$ for counter-clockwise rotation. We denote the corresponding rotation matrix as 
% % \begin{equation}
% % \label{eq: rotation}
%     $\bfR(\theta) = \begin{bmatrix} \cos \theta & -\sin \theta \\ \sin \theta & \cos \theta \end{bmatrix}.$
% % \end{equation
% The $L_2$ norm for a vector $\bfx$ is denoted by $\|\bfx\|$. The gradient of a differentiable function $V$ is denoted by $\nabla V$, and its Lie derivative along a vector field $f$ by $\calL_f V  = \nabla V \cdot f$. A continuous function $\alpha: [0,a)\rightarrow [0,\infty )$ is of class $\calK$ if it is strictly increasing and $\alpha(0) = 0$. A continuous function $\alpha:\mathbb{R} \rightarrow \mathbb{R}$ is of extended class $\calK_{\infty}$ if it is of class $\calK$ and $\lim_{r \rightarrow \infty} \alpha(r) = \infty$.}

% \NA{The footnote should be referenced somewhere. It might be better to just introduce the notation in the main text whenever it is needed.}

Consider a robot with dynamics governed by a non-linear control-affine system,
%
\begin{equation}
\label{eq: dynamic}
\begin{aligned}
    &\dot{\bfx} = f(\bfx) + g(\bfx) \bfu ,
\end{aligned}
\end{equation}
%
where $\bfx \in \calX \subseteq \mathbb{R}^{n}$ is the robot state and $\bfu \in  \mathbb{R}^{m}$ is the control input. Assume that $f : \mathbb{R}^{n} \mapsto \mathbb{R}^{n}$ and $g : \mathbb{R}^{n} \mapsto \mathbb{R}^{n \times m}$ are continuously differentiable functions. We assume the robot operates in a 2D workspace with a state-dependent shape $S(\bfx) \subset \bbR^2$. 


% \NA{Move this paragraph later, now that we define the robot shape $S(\bfx)$, it is not necessary to define the polygon here. It may be better to use a different letter than $S$ since (a) we use calligraphic fonts for sets and (b) we are defining $\calS$ as the safe region. One idea is to define $\calS(\bfx)$ as the robot shape and $\calF$ as the free space.}

We assume the $\bbR^2$ workspace is partitioned into a closed safe (free) region $\mathcal{F}(t)$ and an open unsafe region $\mathcal{O}(t)$ such that  $\mathcal{F}(t) \cap \mathcal{O}(t) = \emptyset$ and $\bbR^2 = \mathcal{F}(t) \cup \mathcal{O}(t)$. We assume the unsafe set $\mathcal{O}(t)$ is characterized by a collection of dynamical elliptical obstacles with known rigid-body motions, denoted as $\{\calE(\bfq_i(t), \bfR(\theta_i(t)), a_i, b_i)\}_{i=1}^N$. Here, $\bfq_i$ denotes the center of mass and $\bfR_i$ denotes the rotation matrix of the ellipse. In its body-fixed frame, $a_i$ and $b_i$ are the lengths of the semi-axes of the ellipse along the $x$-axis and $y$-axis, respectively.

\begin{problem*}
Given a robot with shape $S(\bfx)$ governed by dynamics \eqref{eq: dynamic} that can perfectly determine its state, the objective is to stabilize the robot safely within a goal region $\calG \subset \bbR^2$ such that $S(\bfx(t)) \cap \calO(t) = \emptyset$ for all $t \in \bbR_{\geq 0}$.
\end{problem*}



% In this section we review some notation and preliminaries on distributionally robust chance-constrained programming, control Lyapunov and barrier functions, and their use in systems with uncertainty

% \subsection{Control Lyapunov Function}
% The notion of a control Lyapunov function (CLF) was introduced in \cite{Artstein1983StabilizationWR, SONTAG1989117} to verify the stabilizability of control-affine systems \eqref{eq: dynamic}. Specifically, a (exponentially stabilizing) CLF $V: \calX \mapsto \bbR$ is defined as follows,
% %
% \begin{definition}
% A function $V \in \mathbb{C}^1(\calX,\mathbb{R})$ is a \emph{control Lyapunov function (CLF)} on $\calX$ for system \eqref{eq: dynamic} if $V(\bfx)>0, \forall \bfx \in \calX \setminus \{\boldsymbol{0} \}, V(\boldsymbol{0}) = 0$, and it satisfies:
% %
% \begin{equation}\label{eq: clf}
%     \inf_{\bfu \in \bbR^m} \text{CLC}(\bfx,\bfu) \leq 0, \quad \forall \bfx \in \calX,
% \end{equation}
% %
% where $\text{CLC}(\bfx,\bfu) := \mathcal{L}_f V(\bfx) + \mathcal{L}_g V(\bfx)\bfu + \alpha_V( V(\bfx))$
% is the \emph{control Lyapunov condition} (CLC) defined for some class $K$ function $\alpha_V$.
% \end{definition}


% \subsection{Control Barrier Function}

% To facilitate safe control synthesis, we consider a \textcolor{blue}{time-varying set $\calC(t)$} defined as the super zero-level set of a continuously differentiable function $h: \calX \times \bbR \mapsto \bbR$:
% %
% \begin{equation}
% \label{eq: safe_set}  
%     \calC(t) := \{\bfx \in \calX \subseteq \bbR^n: h(\bfx, t) \geq 0 \}.
% \end{equation}
% %
% Safety of the system \eqref{eq: dynamic} can then be ensured by keeping the state $\bfx$ within the safe set $\calC(t)$.

% \begin{definition}
% \label{def: tv_cbf}
% A function $h: \mathbb{R}^n \times \bbR_{\geq 0} \mapsto {\mathbb{R}}$ is a valid time-varying \emph{control barrier function (CBF)} on $\mathcal{X} \subseteq \mathbb{R}^n$ for \eqref{eq: dynamic} if there exists an extended class $\mathcal{K}_{\infty}$ function $\alpha_h$ with:
% %
% \begin{equation}\label{eq:tv_cbf}
%     \sup_{\bfu\in \mathcal{U}} \text{CBC}(\bfx,\bfu, t) \geq 0, \quad \forall \; (\bfx,t) \in \calX \times \bbR_{\geq 0},
% \end{equation}
% where the \emph{control barrier condition (CBC)} is:
% \begin{equation}
% \label{eq:tvcbc_define}
% \begin{aligned}
%     &\text{CBC}(\bfx,\bfu, t) := \dot{h}(\bfx, t) + \alpha_h(h(\bfx,t)) \\
%     & = \mathcal{L}_f h(\bfx, t) + \mathcal{L}_g h(\bfx, t)\bfu + \frac{\partial h(\bfx,t)}{\partial t} + \alpha_h(h(\bfx,t)).
% \end{aligned}
% \end{equation}
% \end{definition}


% % \textcolor{blue}{Note that the above definition assumes that the TV-CBF depends only first order in time explicitly. If $h$ has a higher order time dependency, we would need to modify \eqref{eq:tvcbc_define} to include higher order time derivatives.}

% Definition~\ref{def: tv_cbf} allows us to consider the following set of control values that render the safe set $\calC$ forward invariant, 
% \begin{equation}
% \label{eq: safe_control_set}
%     K_{\text{CBF}}(\bfx, t) := \left \{  \bfu \in \calU: \text{CBC}(\bfx,\bfu, t) \geq 0 \right \}. 
% \end{equation}

% % \begin{lemma}
% % \label{lemma: 1}
% % Suppose $\alpha: \bbR_{\geq 0} \mapsto \bbR_{\geq 0}$ is a locally Lipschitz continuous class $\calK$ function and $\eta : [t_0, t_1] \mapsto \bbR$ is an absolutely continuous function. If $\dot{\eta}(t) \geq - \alpha(\eta(t))$ for every $t \in [t_0, t_1]$, and $\eta(t_0) \geq 0$, then $\eta(t) \geq 0$ for all $t \in [t_0, t_1]$.
% % \end{lemma}

% % \begin{theorem}
% % \label{theorem: tv_cbc_validity}
% % Assume that $\bfu(\bfx, t) \in K_{\text{CBF}}(\bfx, t)$ is locally Lipschitz continuous in $\bfx$ and piecewise continuous in $t$ and that the unique solutions to \eqref{eq: dynamic} are defined over $[t_0, t_1]$. Then, if $h(\bfx, t)$ is a valid time-varying control barrier function, the set $\calC(t)$ is forward invariant under the control law $\bfu(\bfx, t)$.

% % \begin{proof}
% % We start by assuming that $\bfx(t_0) \in \calC(t_0)$, which implies that $h(\bfx(t_0), t_0) \geq 0$. Given that $h(\bfx, t)$ is a valid TV-CBF and hence $\bfu(\bfx, t) \in K_{\text{CBC}}(\bfx, t) \neq \emptyset$ results in a solution $\bfx$ to \eqref{eq: dynamic} with initial condition $\bfx(t_0)$ that satisfies $\dot{h}(\bfx(t), t) \geq -\alpha_h(h(\bfx(t), t))$ for all $t \in [t_0, t_1]$. Note that the solution $\bfx : [t_0, t_1] \mapsto \mathbb{R}^n$ exists by assumption. We then set $\eta(t) := h(\bfx(t), t)$. Applying Lemma~\ref{lemma: 1}, we deduce that $\eta(t) \geq 0$ for all $t \in [t_0, t_1]$. Hence, $\calC(t)$ is forward invariant, since $h(\mathbf{x}(t), t) \geq 0$ leads to $\bfx(t) \in \calC(t)$.

% % \end{proof}
% % \end{theorem}


% % %
% % Theorem.~\ref{theorem: tv_cbc_validity} implies that if there exists a TV-CBF for \eqref{eq: dynamic}, then the robot state evolves within the safe set $\calC(t)$ under a locally Lipschitz control law $\bfk(\bfx, t) \in K_{\text{CBF}}(\bfx, t)$.


% Suppose we are given a baseline feedback controller $\bfu = \bfk(\bfx)$ for the control-affine systems \eqref{eq: dynamic}, and we aim to ensure the safety and stability of the system.  \textcolor{blue}{By observing that both the CLC and CBC constraints are affine in the control input $\bfu$, a quadratic program (QP) can be formulated for online synthesis of a safe stabilizing controller for \eqref{eq: dynamic}:
% %
% \begin{equation}
% \begin{aligned}
% \label{eq: clf_cbf_qp}
%     \bfu(\bfx) = &\argmin_{\bfu \in \bbR^m,\delta \in \bbR} \| \bfu - \bfk(\bfx)\|^2 + \lambda \delta^2,  \\
%     \mathrm{s.t.} \, \,  &\text{CLC}(\bfx,\bfu) \leq \delta,  \text{CBC}(\bfx,\bfu, t) \geq 0,
% \end{aligned}
% \end{equation}
% %
% where $\delta \geq 0$ denotes a slack variable that relaxes the CLF constraints to ensure the feasibility of the QP, controlled by the scaling factor $\lambda > 0$.}
















