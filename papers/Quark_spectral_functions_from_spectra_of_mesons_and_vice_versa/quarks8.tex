\documentclass[aps,prd,superscriptaddress,eqsecnum,amsfonts,showpacs,epsfig]{revtex4}
%\documentstyle[preprint,floats,eqsecnum,aps,epsfig]{revtex}
\usepackage{epsfig}



\newcommand{\be}{\begin{equation}}
\newcommand{\ee}{\end{equation}}
\newcommand{\bea}{\begin{eqnarray}}
\newcommand{\eea}{\end{eqnarray}}
\newcommand{\al}{\alpha}
\newcommand{\nn}{\nonumber}
\newcommand{\ep}{i\epsilon}
\newcommand{\om}{\omega}
\newcommand{\hm}{\hat{\omega}}
\newcommand{\hz}{\hat{z}}
%%%%%%%%%%%%%%%%%%%%%%%%%%%%%%%%%%%%%%%%%%%%%%%%%%%%%%%%%%%%%%%%%%%%%%%
\begin{document}

%\preprint{ \parbox{1.5in}{\leftline{hep-th/0111433}}}

\title{Quark spectral functions from spectra of mesons and vice versa}

\author{V. \v{S}auli}  


\email{sauli@ujf.cas.cz}
\affiliation{Department of Theoretical Physics, Institute of Nuclear Physics Rez near Prague, CAS, Czech Republic  }
%   

\begin{abstract}
Within the QCD functional formalism, having the approximations  controlled by physical masses and decays  of pseudoscalar mesons,
we extract spectral  function of quarks from which the  meson are composed.
 We choose the pion for the case of light quarks and $\eta_c(N)$ for the extraction of charm quark spectral function. For this purpose we solved the  ladder-rainbow approximation of the  spectral Dyson-Schwinger equations for quarks coupled to Bethe-Salpeter equation for the pion and  the pseudoscalar charmonia. We   begin with indefinite  gauge fixing method for class of covariant linear gauges  and search for an optimal value of the gauge fixing parameter in presented  truncation of Dyson-Schwinger equations system. We found an evidence that Yennie gauge is particularly suited when working with the ladder-rainbow approximation and get the spectral quark functions with high accuracy, simultaneously getting known experimental proparties of mesons. Purely continuous spectral functions were obtained  and their connection with confinement is  discussed.
\end{abstract}

\pacs{11.10.St, 11.15.Tk}
\maketitle





\section{Introduction}

QCD is  a rigid part of the Standard Model  for more then half century and it has passed many nontrivial experimental tests.
The properties of  hadron resonances are well known from various phenomenological models, however using the QCD degrees of freedom
 -the quarks and gluons fields- this subject is little understood  and  available nonperturbative studies 
are not yet  developed. The knowledge of correlation functions  at time-like momentum region is crucial for the first principle
determination of hadronic resonances and understanding of production of hadrons \cite{s3,s4}. In this respect, the Lattice theory provides quite indirect answer, since being formulated in the Euclidean space where it is also solve. To the date, the  analytical continuation of lattice data for propagators to the timelike Minkowski subspace are challenging \cite{DORS2020,LD2022} , but it still represents an ill defined  numerical problem, offering the results with limited precision. 



  A complementary to aforementioned lattice reconstructions is the spectral functional formalism approach, where the analytical continuation is performed from very beginning and the set of Dyson-Schwinger equations (DSEs) is solved for spectral functions in Minkowski space. Such method was appreciated quite recently \cite{s1,s2,SCLK2019,s3,MS2021,HPW2022b} and it includes the topics of spectral renormalization - primary or secondary subtractions technique performed  at the timelike momentum scale. A  Yang-Mills sector of $SU(3)$ gauge theory was considered in \cite{HPPW2021,HPW2022} bringing a new insights  in  the conventional Landau gauge.  A meaningful comparison to  lattice data was not available  when  one of the first  spectral DSEs study \cite{CO1982}  was suggested. In order to get agreement with recent lattice data, the importance of transverse vertices  in pure gluodynamics was shown  \cite{s2}. In the Euclidean space formalism the calculations of hadrons as a bound state of quarks, antiquarks and gluons is evolved nowadays beyond the first fancy approximations, see for instance \cite{ESWAF2016,HFS2020}.  For a general review of present knowledge of QCD propagators  in DSEs formalism see \cite{HUB2020}.  The purpose of presented paper is not a jump to bandwagon or chasing the  train of DSEs scheduled in the Euclidean space \cite{ROWI1994,ALSM353,HUB2020}, but to push theory of spectral DSEs in its own direction. In presented study, the popular ladder-rainbow (LR) truncation of the DSEs system, exploited for years for its simplicity in practice \cite{MARO1997,Sa2008,CK2010,QCLRW2012,FSV2014,TGK2015,HGK2017,HGKL2017,YCKRSX2019} either in the Euclidean and in the Minkowski space,
   is used as a first approximation for various purpose.         
 

    Since the relativity is  less urgent for mutual interaction of heavy quarks $Q=c,b$ inside heavy mesons, nonrelativistic quantum mechanic was widely used to describe quarkonia and their transitions ( for a review see \cite{EGMR2007}) instead. 
In addition to perturbative Coulomb  ``one gluon exchange'' potential, the linear rising potential has been proposed to explain spectra of excited quarkonia \cite{KOSU1974}. Ignoring quark  mixing,    models  based on presence of linear potential reasonably  describe static  spectra of strangeonia \cite{strange2021} as well.  A theory behind phenomenology of string linear potential relies on 
  observation of flux tubes between infinitely heavy color sources \cite{BSS1995}. This technique lies aside 
 of quark-antiquark scattering kernel used in the LR  DSE heavy quarkonia studies \cite{saulieta,saulipsy,HPGK2015}). To match the two different approaches -the DSEs and Wilsonian static quark potential together, is longstanding desire but  unfinished story (for the attempt see \cite{BMCCO2009}).
 

 
 Actually, to the author best knowledge, there is not known truncation of QCD DSEs, which would lead to  the  string picture of confinement.  In DSEs formalism, the string-like interaction is either introduced  by hand \cite{saulieta,saulipsy} or 
  avoided  by using alternative non-string interaction \cite{HPGK2015}, which still  provides observed structure of excited hadrons.
Also here,  in order to describe spectra of excited charmonia here, wee need to go ``beyond one gluon exchange'' in a way  it does not destroy the solution with desired  analytical
 property. Enchantingly, it turns out the pion and heavy quarkonium (pseudoscalar charmonia) can be described within almost the same functional form of the  DSE  kernel.  
   


Before presenting the details of truncation, which complies with the existence of quark spectral function, let  us mention here the  so called hindered transitions , which were measured at various channels \cite{hind,hind1,hind2}.
  The  large  discrepancy between  measured rates and nonrelativistic theory predictions 
were usually  attributed to missing relativistic corrections.
 To explain the quarkonia and  their  transitions, a very recent  treatments based either on  DSEs formalism, nonrelativistic  quantum mechanic or other techniques \cite{GUGEBH2021,JUHUCHA2021,BG2020,SG2020,BGPSS2019,SJS2018,LLMV2018,DLGZ2017,BS2013,LZ2011}
 still represent a plethora of  different approaches with not completely clear connection  to QCD.
To this point, the  formalism of spectral DSEs we  present here is based on QCD degrees of freedom, it is systematically  improvable via truncation of DSEs, and  it  leads  to the known dispersion relations for  hadronic form factors.
 
  In the next  Section we describe  the calculation scheme for  the quark propagator. In the Section III we describe the method of solution of the pion Bethe-Salpeter equation (BSE) and present numerical results in section IV.  The results for pseudoscalar charmonia and extracted c-quark spectral function are shown in the Section V. We discuss the limitations of our  method  and conclude in the last Section VI.
 

 
 \section{Truncation of SDEs system for the pion}
 
 For purpose of completeness we write down all necessary equations here. 
The  quark propagator $S$ can be conventionally parametrized as
\be   \
S(q,\mu)=[A(q^2,\mu)\not q- B_q(q^2,\mu)]^{-1}
\ee
where  $A,B$ are two scalar functions characterizing completely  quark propagator when 
the electroweak interaction is turned off.
The DSE for the quark propagator in MOM renormalization scheme can be  written as
%
\bea  
A(q^2,\mu)&=&a(\mu)-Tr \frac{\not q}{4q^2}\Sigma_r(q)+ Tr \frac{\not q}{q^2}\Sigma_r(q)|_{q^2=\mu^2}
\nn \\
B(q^2,\mu)&=&b(\mu)+\frac{Tr}{4}\Sigma_r(q)- \frac{Tr}{4} \Sigma_r(q)|_{q^2=\mu^2}
\nn \\
 \label{gap}
\Sigma_r(q)&=&i\frac{4}{3}\int\frac{d^D k}{(2\pi)^D
} \gamma_{\mu} S(q-k) \Gamma_{\nu}(k,q) G^{\mu\nu}(k) \, ,
\eea
%
where $Tr$ is for Trace over the dirac indices.
 The inverse of $A$, is traditionally called the quark renormalization function, while the 
rate $B/A$ represents the renormalization scheme invariant  dynamical quark function $M(q^2)$. The MOM renormalization scheme 
turns to be particularly useful for  evaluation of spectral functions, leaving also a simple meaning for the constants $a(\mu)$ and $b(\mu)$, they represent values of renormalized functions $A(\mu^2,\mu)$ and $B(\mu^2,mu)$  respectively.

The quark selfenergy $\Sigma_r$ could be conveniently regularized in symmetry preserving way before indicated subtractions are made. 
For this purpose we have used dimensional regularization in (\cite{s1}) as indicating by non-integer spacetime dimension $D=4-\epsilon$.



To renormalize we  take $\Re a(\mu)=1$ and  $ \Re b(\mu)=300 MeV$, being thus approximately the  constituent quark mass at the timelike subtracting point $\mu^2=0.5 GeV^2$.  The imaginary parts of functions $A,B$ are not  arbitrary  and their values need to be find as a solution. Requiring all renormalization constants are real , one can alternatively  subtract only the real parts of projected selfenergy in the Eq. (\ref{gap}) and set  a real renormalization condition to be real. Of course, choosing the scale $\mu$ spacelike ($\mu^2<0$ within our metric convention), all renormalized functions must be real at the scale $\mu$. Needless to say, a numerically working calculation scheme, which is  equivalent to the standard scheme used in the Euclidean space, but provides results in entire Minkowski space, is the art of state of presented 
evaluation method here. For this purpose we continue  the method developed in the  paper \cite{s1}. 
  
In the Eq. (\ref{gap}) $G $ stands for dressed gluon propagator and   $\Gamma(k,q)$ represents  the dressed quark-gluon vertex, their color and Dirac indices are suppressed (and we will not write Lorentz indices in the text if not needed).  Assuming the full vertices have the same color structure as  the classical ones, it provides the prefactor $4/3$, which is explicitly shown.
The popular LRA is used and for the product of the quark-gluon vertex $\Gamma$ and the  gluon propagator $G$ in the DSE we take
%
\be \label{kernel}
\Gamma_{\nu} G^{\mu\nu}(p)=\gamma_{\nu}N(\xi)\left[-g^{\mu\nu}+\frac{p^{\mu}p^{\nu}}{p^2}\right]\int do \frac{\rho_T(o)}{p^2-o+\ep}
 -\not p \frac{\xi g^2 p^{\mu} }{(p^2)^2} \, ,
\ee
where $\rho_T$ is the gluon spectral function obtained with Landau gauge in the paper (\cite{s2}) and  $N(\xi)$ stands for  momentum 
independent constant factor, $\xi$ is the  gauge fixing parameter.

In this report we do not solve the gluonic spectral  DSE (in which  way we  get  rid  of the ghost kernels) and
in  order to reduce the number of numerical integration in our  study we have used a simplified (UV finite) fit for the gluonic spectral function 
\be \label{prase}
\rho_T(o)=-\delta(o-m_g^2)+\delta(o-\Lambda^2)
\ee
with $m_g=0.6 GeV$ and $\Lambda=2 GeV$ as found in \cite{s2}.
  The extension of the study  \cite{s2}  to more realistic system of  QCD DSEs  is undergoing.
The parameters $N(\xi)$ as well as the gauge parameter $\xi$ were varied to find the solution which complies with
\

{\bf 1.} spectral property of the quark propagator
\

{\bf 2.}  meson properties.  
 \  

To ensure the first point, the   DSE is solved  for the  spectral functions $\sigma$ for which purpose 
we  follow  the method  established and  described in details in the paper \cite{s1}. Thorough obtained spectral function the method provides the solution of the equation (\ref{gap}) in the entire Minkowski space, confirming
the spectral representation for the quark propagator exists in the standard form:
%
\be  \label{spec1}
S(p,\mu)=\int_0^{\infty} d o \frac{\not p \sigma_v(o)+\sigma_s(o)}{p^2-o+\ep}\, \, .
\ee

Two introduced spectral function stands for the dirac and the scalar part of the quark propagator
\bea
S_v(p^2)&=&\frac{A(p^2)}{p^2 A^2(p^2)-B^2(p^2)}=\int_0^{\infty} d o \frac{ \sigma_v(o)}{p^2-o+\ep}
\\
 S_s(p^2)&=&\frac{B(p^2)}{p^2 A^2(p^2)-B^2(p^2)}=\int_0^{\infty} d o \frac{ \sigma_v(o)}{p^2-o+\ep} \, ,
\label{spec2}
\eea
 where  another standard labeling  for two scalar propagator functions $S_v$ and $S_s$ was used; obviously $S=\not p S_v+S_s$  . 
 The renormalization scale is not shown for purpose of brevity and we will omit it in further text. The spectral function , likewise the propagator, is a scheme dependent object. In gauge theory like QCD, it nontrivialy 
 depends on the gauge parameter as well. 
 
Since analytical properties are  not automatically guaranteed in solutions for strongly coupled theories like QCD
thus to comply with the point 1, we require the spectral deviation  introduced in \cite{s1,s2} should  be vanishing.
  To  simultaneously  satisfies the point 2, the Bethe-Salpeter equation (BSE) for the pion  must produce the correct pion mass   
 $m_{\pi}=140 MeV$ and pionic decay $f_{\pi}=90 MeV$. 
This gives us   the parameter space for modeling of the interaction kernel of our LR approximation, but it also limits the physical  renormalization 
 conditions for the quark propagators. 


%
 \begin{figure}
\centerline{\includegraphics[width=8.6cm]{sigmapseudo2.eps}}
\caption{Quark spectral functions, solid line stand for the functions $\sigma_d$, , dashed line for $\sigma_s$ plotted against the energy . The left two blobs are for light quarks, the one on the right for the charm quark.}
\label{jedna}
{\mbox{-------------------------------------------------------------------------------------}}
\end{figure}
 


\section{Pion }

The homogeneous BSE for mesons is the field theoretical equation for the quark-antiquark bound state. In a dense notation it reads
\be
\Gamma=\int_k S\Gamma S K
\ee
where $S$ is  propagator of bound state component, i.e. the quark propagator  for either  quark or antiquark, here $\Gamma$
is the Bethe-Salpeter vertex function and $K$ is the quark-antiquark-quark-antiquark irreducible interaction kernel.
For the pion case, this kernel has been taken within the same  approximation as for  the DSE for the 
light quarks, which consistently preserves Goldsteone character of the pseudoscalar mesons in the chiral limit.


%Ladder-rainbow approximation is widely used  in the literature \cite{}  and we follow it here mainly for its simplicity.

 
The dominant Bethe-Salpeter vertex component $\Gamma =\gamma_5 A$
 was used and BSE was solved by eigenvalue method in complex momentum space. 
Such single component approximation is working well 
not only for the ground state \cite{FNW2008} but (with a slight modification) for the excited states as well \cite{saulipsy}. 

Introducing an auxiliary 
eigenvalue function $\lambda(P,A)$ with suited dependence on the mass of the bound state $P^2=m_{\pi}^2$,    
then the BSE, after the 3d angular integration and some trivial algebra,  reads in our approximation
%
\bea \label{num}
A(p_E,P)=\lambda(P,A)\int\limits_{-\infty}^{\infty} dk_4 \int\limits_0^{\infty} d {\bf k} \frac{\bf k^2}{\bf p^2}  A(k_E,P)(S_v(k_+)S_v(k_-)(k_E^2+m_{\pi}^2)+S_s(k_+)S_s(k_-))[K_g(k,p)+K_{\xi}(k,p)]
\eea

As  one can see the rhs. of BSE involves
 the product of the scalar functions
$S_v(k_+)S_v(k_-)$ and $S_s(k_+)S_s(k_-)$ evaluated at complex valued momentum $k_{\pm}=k\pm P/2$. Here $k$ (or $k_E$) is a  relative Euclidean momentum, while the total momentum  $P_E=(im_{\pi},0)$ in the rest frame of the pion. The propagator $S$ is obviously  getting complex for complex arguments. For explicit evaluation  we have used the spectral representation (\ref{spec2}) and  have implemented additional integration over the spectral variable into the BSE kernel.  Note, both  propagator functions products  appearing in the Eq. (\ref{num}) stay real in the izospin approximation.

The kernel $K_g$ and $K_{\xi}$ are usual logs stemming from the "gluon propagator" integrated over the 3d spacelike angles. Thus for instance 
\be 
K_{\xi}=C_A\frac{g^2\xi}{(16\pi^3)} \ln\frac{f+2{\bf k p}}{f-2{\bf k p}}
\nn
\ee 
with $f$ defined as $f=k_E^2+p_E^2+2 k_4p_4$.

The following eigenvalue function has been found  particularly useful 
%
\be
\lambda^{-2}=\frac{8}{P^2}\int_{-\infty}^{\infty} dk_4 \int_0^{\infty} d k A^2(k_E,P)/\sqrt{C+k^2_E} \, .
\ee
for the evaluation with numerical value of the constant$C$,   $C=1 GeV^2$ .  
The BSE (\ref{num}) has been solved by  iterations, providing the original BSE solution when
$\lambda(P,A)=1$ which must coincide exactly with desired vanishment of difference between two 
consecutive iterations ( $\sigma^2$  labels its quadrat in the fig. 3 of presented paper) . 


\section{Results for the light quarks from the pion }




The resulting spectral functions for the the light quarks are shown in the figure 1. We work in the izospin limit and ignore electromagnetic interaction,
thus the spectral function  for the $u$ quark  is identical to the $d$ quark one.
According to  broad shapes of both functions $\sigma_{v,s}$, 
they describe confined objects- the light quark excitations. The quarks  continuously change colors inside hadrons by  exchanging  gluons, hence a  width of the main peak can be interpreted as  the inverse of  mean time $\tau _{u,d}\simeq 0.2 GeV^{-1}$, which the quark of given flavor spent  with a given color.
The appearance of the Dirac delta function, if it was there, would mean a colored quark can escape to the detector and no confinement exists.
Being obviously opposite , the time $\tau_f$,  similarly to partial width of the quark weak decays, does not represent observable. Non-trivially the broad shape
of the quark spectral function should be  reflected in shape of timelike pionic form factors in gauge fixing independent manner. At least,  as a mathematical consequence, the quark thresholds vanishes, as `` they are washout'' at evaluated form factors.
Such  behavior  is intuitively expected, and in fact it  has been   mimic in \cite{BFGIKL2010,GGIL2021,DDIL2022} by the introduction of certain  infrared cutoff  in the Feynman(Schwinger) parameter in  various evaluations of hadronic form factors.



 \begin{figure}
\centerline{\includegraphics[width=8.0cm]{sigmapseudo.eps}}
\caption{Dimensionless quark spectral functions $ o\sigma_v(o)$ (solid line)  and $\sqrt(o)\sigma_s(o)$ (dashed line)
 for the light and charm quark. At larger (smaller) energy scale the broad peak for the charm flavor  (u,d)  quark spectral function develops.}
\label{dva}
{\mbox{-------------------------------------------------------------------------------------}}
\end{figure}
%

The solution for the pion was found by using method described in the previous section. 
The following rate of couplings 
%
\be \label{rate}
\frac{g^2\xi}{N(\xi)}=3 \, 
\ee
provides one particularly convergent solution in our LRA.  To get the pion, the following numerical  value $\frac{4N(\xi)}{3(4\pi)^2}=\frac{16}{3}$
was taken here.

We have  improved numeric by splitting  integrals  into two Gaussian integrators in order to 
 perform the integration over the  dominant peak in the quark spectral function
efficiently. The first Gaussian integration is mapped from the zero to the position of the right arm of the peak, while  the second Gaussian integration  has been  mapped to the infinite remnant of the spectral integration region.  
The deviation from assumed analyticity   as  established in \cite{s1,s2} ( it is called $\sigma$ there, it should not to be confused with other $\sigma$'s used in the text herein) can be in principle arbitrarily  minimized. The  value $\sigma^2=10^{-6}$ has been  achieved and seems to be  limited only by precision of principal value integral appearing in the definition of deviation  $\sigma$.





Numerical codes, either for BSE and DSE are available for public \cite{mujweb}. 
At this place we should aware the reader that  
after the successful search of one solution, the convergence to  the next excited state is not an automatically ensured
 and search could be repeated at suspicious interval of bound state masses.
 This note is particularly important when concerning narrow states e.g quarkonia considered in the next section. 
 The model here, according to broad resonant character of experimentally known  $\pi$*, does not offer excited 
 pion states form the solution of homogeneous BSE. 


\section{ c-quark spectral function from $\eta_c(n)$ quarkonia }


Almost identical functional form of  QCD DSE kernel that govern the interaction between quark-antiquark in the 
light meson was used to calculate  the heavy pseudoscalar quarkonia.
Subtle changes and numerical solution for the charmonium l system are described in this Section.

According the flavor non-universality of the quark-gluon vertex,
a certain softening of the interaction  is expected.
We have used    dimensionless quantities  $g; \xi$ identical to  the case of the pion but we 
have change the dimensional parameters, including the renormalization scale as well.
We have found that the  use of a single scale parameter $r=0.721$ is enough for changes in the kernel. I.e. we take now  
 \be
m_g^c=r m_g=0.433\,  GeV \, \, ; \, \, \Lambda^c=r \Lambda =1.442 \,  GeV ;
\ee 
now for a charm quark  DSE and charmonium BSE. 

The renormalized values of  renormalization function and mass are taken  such that $\Re a_c(\mu)=1$  and $Re b_c(\mu)=r 1.3  GeV$
 at the timelike renormalization point  $\mu^2=r^2 0.5 GeV^2$.
 We should remind that the use the timelike renormalization scale is a technical must.
 Without this option we could not get a precise and stable solution for the spectral functions of c-quark propagator. 
 
 


Elsewhere more important diamond diagrams (the diagrams with interrupted quark horizontal  lines by gluon lines) should  contribute to the kernel with substantially small effect (note $M(\eta_c)\simeq M(J/\psi)$. 
Instead of evaluating these  complicated  diagrams 
we mimic their small  effect and  insert the following prefactor
\be \label{pref}
f_{\eta}=\frac{1}{\sqrt{2}}\sqrt{1+\frac{M(\eta_c(2))^2}{P^2}}.
\ee
in lhs. of BSE, in which  we have incorporated  very mild  total  momentum $P$ of ,
into the game.
 

The resulting charm quark spectral functions are added into the fig. \ref{jedna} for comparison. 
Since the spectral functions are dimensinfull object, we introduce the dimensionless quantity  $\sqrt(o)\sigma_s(o)$ and $o\sigma_v(o)$ for a better comparison of spectral functions of different flavors. These object are compared in the figure \ref{dva}. 
The on-shell singularity is washout to a broad peak  and heavy  free quark  excitation does not exists at all.
A picture of confinement that emerge in spectral framework of DSEs is very the same for the light as well for the heavy quarks.
 The change of the kernel when going from the pion to  the charmonium case  tells us about the importance of the quark-gluon vertex.
   




The spectra of bottomonia can be obtained by a similar fashion, however our two mutually beating poles  turns to be cruel approximation at bottomonium scale and slightly differen approximation seems to be required. We plan to perform  more comprehensive study of $BB$ system in the future.
 

%
\begin{table}[t]
\begin{center}
\small{
\begin{tabular}{|c|c|} \hline \hline 
BSE  & EXP.    \\
\hline \hline
2980 & 2980  \\
\hline
3442   & 3638 \\
\hline
4150 & 3810 \\
\hline
4720 & --   \\
\hline \hline
\end{tabular}}
%
\caption[99]{Comparison with PDG data (second column) and calculated spectrum.
\label{tabbse2}}
\end{center}
\end{table}



The obtained masses are tabled in the Tab \ref{tabbse2} further predicted and  not yet observed excited $\eta_{c}$ charmonia states  we only list here: 5436,6186,7030MeV,...
Our calculations  ignore interaction with lighter quarks/meson  these states are narrow here, however they become a broad resonance 
in the Nature, making our predictions less and more academic.    We  show our ultimate  numerical search for  eigenvalue $\lambda$  and the iteration difference $\sigma^2$ (now for BSE)  in the  figure \ref{tri}.  A single point shown in this figure costed one day of work of recent  single processor. Even working with multiprocessor machines the   reader can imagine the time consumed before the truncation of DSE/BSE has been established.  
%
\begin{figure}
\centerline{\includegraphics[width=8.0cm]{lambda.eps}}
\caption{The eigenvalue $\lambda$ and the numerical error $\sigma$ from the solution of BSE the  solution for the ground state and the the first three excited state of pseudoscalar  charmonium. The definitions can be find in \cite{saulipsy}, the bound states are for $\lambda \rightarrow 1$  $\sigma\rightarrow 0$ when satisfied simultaneously for the meson mass $M$. }
\label{tri}
{\mbox{-------------------------------------------------------------------------------------}}
\end{figure}
 

\section{Conclusion}

We  have solved coupled set of spectral quark Dyson-Schwinger  and Bethe-Salpeter equations for the pion and  have extended the method to the heavy quarks sector represented by  pseudoscalar charmonia. 

Facing the resulting spectral functions   we get   
simple picture of confinement of the light as well as the heavy quarks:  quarks are never on-shell inside the hadrons,  the inverse of quark propagator never gets zero for a real momenta. The sharp  singularity is completely washed out due to the specific presence of imaginary part in selfenergy function.  The imaginary parts is  gradually growing from the anomalous thresholds- the zero momenta. 



For the pion case the solution presented here has been already obtained for kindred model, albeit the renormalization and the kernels  slightly differ numerically. That description of both - the light and heavy meson systems - is possible within  DSEs/BSEs formalism is not surprising fact \cite{HPGK2015}. However the use of  almost (up to the scaling) identical kernel  for the charmonium and for the pion case is astonishing. 
Notably, the interaction kernel  does not exactly follow our conventional historical wisdom: it does not lead to static  color coulomb plus linear potential in the nonrelativistic limit.  Nevertheless, the spectrum of charmonia is recovered.  It suggests that c-quark, albeit its mass (peak position of spectral function) is quite above $\Lambda_{QCD}$ is not heavy enough to be considered as ``infinitely heavy  source'' of color flux tube \cite{BSS1995}. Wether a simple picture obtained here survives the   extension to other flavor sector of mesons can be checked in future studies.  

 The gauge term, which is not contributing  for on-shell scattering fermions at all, turns to be important part due to the unavoidable offshellness of confined quarks.  The form of kernel suggest that our  gauge choice is very close to the Yennie gauge $\xi=3$, known because of  cancellation of infrared infinities in perturbation theory applied to on-shell fermion scattering. However, here we are not beating infrared divergences,  here the value of $\xi$  comes out as result   due to  perspective of  convergence of LR approximation.  During the preparation of this paper, new study of the solely  quark mass function dependence on the gauge fixing parametr has appeared  \cite{LSBBO2023}.  A structure of the quark-gluon vertex ensuring the approximate gauge independence of the quark condensate for $\xi$ between the Landau and Feynman gauge has been found. The  gauge invariance  of hadron spectra  and comparison to a simple $LR$ approximation of BSE for wider range of $\xi$ are expected questions to be answered in a not so distant future.
 
 Obviously, use of spectral representation can be seen as heavy hammer tool  for calculations of form factor for spacelike argument.
 There, the convenient calculation within the use of Euclidean metric works sufficiently irrespective of analytical details of the kernels.  There is a little advantage   when Isgur-Wise functions \cite{IW1989,IW1990} are calculated within the use of presented formalism as well. There are likely other quantities  insensitive to the issue of confinement especially if vertices and  quarks lines lie outside the timelike domain of momenta. 
 The methodology of calculation of form factors at resonant region is a right challenge for formalism of spectral DSEs . Within the truncation presented here one can get the celebrated dispersion relation form \cite{CAGA1961} for Vacuum Hadron Polarization as well as one can enjoy the resulting  dispersion relation for  the electromagnetic meson form factor \cite{s3}.  The form of  gauge invariant  quark propagator has been established  in \cite{ACS2023} via Wilsonian lines formulation, which paths the direct way to observable hadroproductions.  Establishing a connection between this and the standard gauge fixed  quark propagator used here  can be useful in various respect.   
 

At last but  not at least we could stress again the  strategy of our indefinite gauge method. Actually, we have found that in our approximation, leaving the Landau gauge  and considering other value of gauge parameter $\xi$ is really advantageous. Of course,
Of course, if  quark spectral representation exists in a given gauge, 
it is natural to expect that  it exists in some other gauges as well. However to reach  meson spectra  requires very different effort when one goes from one gauge choice to another one. 
 
 Further reason to do that, is a well known fact that Ladder-rainbow approximation, within the use of Landau gauge  lattice gluon propagator,
 does not have a proper strength and lead unavoidably to non-QCD spectra. Actually, that such interaction is too weak   can be  seen  from the DSE solution alone.  It has been actually  checked that  decreasing the gauge  fixing parameters, one gradually observes the growth of the particle like peak in the quark spectral function. The peak is narrowing and the Dirac delta function is formed  after passing through critical value $\xi g^2$, which shows  inefficiency
of  the approximation to reflect confinement correctly.
  At some critical point one actually gets non-confined quark  propagator with familiar form
%
\be \label{SRNC}
S_{NC}(p)=\frac{R}{\not p-m_p}+\int_{th}^{\infty} do \frac{\sigma_v(o)\not p+\sigma_s(o)}{(p^2-o+\ep)}, 
\ee
where two  continuous spectral functions  $\sigma_{v,s}$ are nonzero only from the threshold. 
 Such solutions typically arise at non-confining theory like QED, being 
preserved for not large coupling in toy quantum field models \cite{SAJHP2003}.
Even taking the prefactor of the quark-gluon vertex as a free parameter, the authors of  \cite{HPW2022b}  obtained particle like pole with the residuum value $R\simeq 0.75$ in the Eq. (\ref{SRNC}) by solving the spectral DSE propagator within   lattice Landau gauge gluon data. 
 In order to have method working already at LR approximation  we decided to use nontrivial value ($\xi\ne 0$) instead.


Here we argue, that match of  the LR trucation of QCD DSEs/BSEs  and  simultaneous gain of  the correct form of the quark and gluon spectral function  is very likely impossible in the Landau gauge.  There are known working scheme in  Landau gauge showing the importance of other QCD vertices- a difficult task 
to involve them correctly remains to be done in the  spectral DSE approach.  Oppositely, the existence of gauges where higher vertices  could play only subdominat role  and simple truncation of DSEs (like here) could be checked more seriously by other methods.  If not confirmed, the results  presented here
can an accidental luck that happen in  pseudoscalar meson channel only.


 %%%%%%%%%%%%%%%%%%%%%%%%%%%%%%%%%%%%%%%%%%%%%%%
\begin{thebibliography}{00}
\bibitem{s3}
%pipi creace
V. Sauli, Phys. Rev. D  {\bf 106}, 3, 034030 (2022).

\bibitem{s4}
%Gauge technique approximation to the πγπγ production and the pion transition form factor
V. Sauli, Phys. Rev. D {\bf 102}1, 014049 (2020).

\bibitem{DORS2020}
% Spectral representation of lattice gluon and ghost propagators at zero temperature
D. Dudal, O. Oliveira, M. Roelfs, P. Silva,   Nucl. Phys. B {\bf 952}, 114912 (2020).

\bibitem{LD2022}
 T. Lechien, D. Dudal, SciPost Phys.  {\bf 13}, 4 097 (2022).

\bibitem{s1}
%The quark spectral functions and the Hadron Vacuum Polarization from application of DSEs in Minkowski space
V. Sauli, Few Body Syst. 61 (2020).
  
\bibitem{s2}
%Confinement within the use of Minkowski integral representation
V. Sauli, Phys. Rev. D {\bf 106}, 9, 094022 (2022).

\bibitem{SCLK2019}
% Quark propagator in Minkowski space
E.L. Solis, C.S.R. Costa, V.V. Luiz, G. Krein, Few Body Syst. {\bf 60} 3, 49 (2019).

\bibitem{HPW2022b}
%On the quark spectral function in QCD
J. Horak, J. M. Pawlowski, N. Wink, ArXiv:2210.07597. 

\bibitem{MS2021}
%Fermion and Photon gap-equations in Minkowski space within the Nakanishi Integral Representation method
C. Mezrag, G. Salmè, Eur. Phys. J. {\bf C 81} 1, 34 (2021).

\bibitem{HPW2022}
%On the complex structure of Yang-Mills theory
J. Horak, J. M. Pawlowski, N. Wink, ArXiv:2202.09333

\bibitem{HPPW2021}
%Ghost spectral function from the spectral Dyson-Schwinger equation
J. Horak, J. Papavassiliou , J. M. Pawlowski, N. Wink, Phys. Rev. {\bf D 104},074017 (2021).

\bibitem{CO1982}
J.M. Cornwall, Phys. Rev. {\bf D 26}, 1453 (1982).

\bibitem{ROWI1994} 
 C.D. Roberts and  A.G. Williams, Prog. Part. Nucl. Phys. {\bf 33}, 477 (1994).
   
\bibitem{ALSM353}  
R. Alkofer, L. von Smekal, Phys. Rept. {\bf 353} 281 (2001).
  
\bibitem{HPGK2015}
%Spectra of heavy quarkonia in a Bethe-Salpeter-equation approach
T. Hilger, C. Popovici, M. Gomez-Rocha, A. Krassnigg, Phys. Rev. D {\bf 91} 3,034013 (2015).
% popsana je metoda eigenvalue- tiskni si

\bibitem{saulieta}
V. Sauli, Phys. Rev. D {\bf 90}, 016005 (2014).

\bibitem{saulipsy}
V.Sauli,Phys. Rev. D {\bf 86}, 096004 (2012).

\bibitem{ESWAF2016}
%   Baryons as relativistic three-quark bound states
G. Eichmann, H. Sanchis-Alepuz, R. Williams, R. Alkofer, C. S. Fischer, Prog. Part. Nucl. Phys. {\bf 91}, 1-100 (2016).

\bibitem{HFS2020}
 M. Q. Huber, C. S. Fischer, H. Sanchis-Alepuz,  Eur. Phys. J. C {\bf 80}, 11,1077 (2020).
%  Spectrum of scalar and pseudoscalar glueballs from functional methods
 
\bibitem{HUB2020}
%  Nonperturbative properties of Yang–Mills theories
 M. Q. Huber,  Phys. Rept. 879, 1-92 (2020).
    
 \bibitem{MARO1997}
P. Maris, C. D. Roberts, Phys. Rev. C {\bf 56},3369 (1997).
%   Pi- and K meson Bethe-Salpeter amplitudes

\bibitem{Sa2008}
V. Sauli, J. Phys. G {\bf 35}, 035005 (2008).

\bibitem{CK2010}
%Solving Bethe-Salpeter equation for two fermions in Minkowski space
 J. Carbonell, V. A. Karmanov, Eur. Phys. J. A {\bf 46},387 (2010).

\bibitem{QCLRW2012}
S. Qin, L. Chang, Y. Liu, C. D. Roberts, D. J. Wilson,    Phys. Rev. C {\bf 85},  035202 (2012).

\bibitem{FSV2014}
%Quantitative studies of the homogeneous Bethe-Salpeter Equation in Minkowski space
T. Frederico, G. Salme, M. Viviani, Phys. Rev. D {\bf 89}, 016010 (2014).

\bibitem{TGK2015}
T. Hilger, M. Gomez-Rocha, A. Krassnigg,    Phys. Rev. D {\bf 91} 11, 114004 (2015).
%Masses of JPC=1−+JPC=1−+ exotic quarkonia in a Bethe-Salpeter-equation approach

\bibitem{HGK2017}
 T. Hilger, M. Gomez-Rocha, A. Krassnigg,   Eur. Phys. J. C {\bf 77} 9, 625 (2017).
%Light-quarkonium spectra and orbital-angular-momentum decomposition in a Bethe–Salpeter-equation approach

\bibitem{HGKL2017}
T. Hilger , M. Gómez-Rocha, A. Krassnigg, W. Lucha, Eur. Phys. J. A {\bf 53} 10,213  (2017).

\bibitem{YCKRSX2019}
P. Yin, C. Chen, G. Krein, C.D. Roberts, J. Segovia, S. Xu,  Phys. Rev. D {\bf 100},  034008 (2019).

\bibitem{EGMR2007}
 %Quarkonia and their transitions
E. Eichten, S. Godfrey, H. Mahlke, J. L. Rosner, Rev. Mod. Phys. {\bf 80} 1161, (2008).

\bibitem{KOSU1974}
 J. Kogut and L. Susskind, Phys. Rev. D{\bf 10}  3468 (1974).

\bibitem{strange2021}
% Mass spectrum and strong decays of strangeonium in a constituent quark model
Qi Li, Long-Cheng Gui, Ming-Sheng Liu, Qi-Fang Lü, Xian-Hui Zhong,    Chin. Phys. {\bf C 45} 2, 023116 (2021).

\bibitem{BSS1995}
 G.S. Bali. K. Schilling and C. Schlichter, Phys. Rev. D{\bf 51}, 5165-5198 (1995).
% Observing long color flux tubes in $SU(2)$ lattice gauge theory. ArXiv:hep-lat/9409005.
 

\bibitem{BMCCO2009} 
%Schwinger-Dyson equations and the quark-antiquark static potential,
P.Bicudo, G. Marques, M. Cardoso, N. Cardoso, O. Oliveira,    PoS QCD-TNT09 003 (2009); ArXive: 0912.1274.

\bibitem{hind}
R. E. Mitchell et al. [CLEO Collaboration], Phys. Rev. Lett. 102, 011801 (2009).

\bibitem{hind1}
%Observation of the hindered electromagnetic Dalitz decay ψ(3686)→e+e−ηcψ(3686)→e+e−ηc​
M. Ablikim et al. BESIII Collaboration, Phys. Rev. {\bf D 106}, 112002 (2022).

\bibitem{hind2}
%Branching fraction measurements of ψ(3686)→γχcJ
BESIII collaboration: M. AblikimBESIII Collaboration et al., Phys. Rev. {\bf D 96}, 3, 032001 (2017).

\bibitem{GUGEBH2021}
%E1 and M1 radiative transitions involving heavy-light axial, pseudoscalar, 
%and vector quarkonia in the framework of the Bethe-Salpeter equation
V. Guleria, E. Gebrehana, S. Bhatnagar, Phys. Rev. D {\bf 104}, 9, 094045 (2021).

\bibitem{JUHUCHA2021}
%Revisiting the PP-wave charmonium radiative decays hc→γη(′)hc​→γη(′) with relativistic corrections
Jun-Kang He, Hua-Zhong, Chao-Jie Fan, Phys. Rev. D {\bf 103}, 11, 114006 (2021).

\bibitem{BG2020}
%Radiative decays in charmonium beyond the p/m approximation
R. Bruschini, P. González, Phys. Rev. D {\bf 101},1 014027 (2020).
    
\bibitem{SG2020}
%Radiative decays of heavy-light quarkonia through M1M1 and E1E1 transitions in the framework of the Bethe-Salpeter equation
S. Bhatnagar, E. Gebrehana, Phys. Rev. D {\bf 102}, 9, 094024 (2020).
    
\bibitem{BGPSS2019}
%γ∗γ∗→ηc​(1S,2S) transition form factors for spacelike photons
I. Babiarz, V. P. Goncalves, R. Pasechnik, W. Schäfer, A. Szczurek, Phys. Rev. {\bf D 100}   054018 (2019).

\bibitem{SJS2018}
%QQˉ​ ( Q∈{b,c}Q∈{b,c} ) spectroscopy using the Cornell potential
N.R. Soni, B.R. Joshi, R.P. Shah, H.R. Chauhan, J.N. Pandya,
Eur.Phys.J. C {\bf 78},  592 (2018).

\bibitem{LLMV2018}
%Radiative transitions between 0−+0−+ and 1−−1−− heavy quarkonia on the light front
Meijian Li, Yang Li, P. Maris, J. P. Vary, Phys. Rev.D {\bf  98} 3, 034024 (2018).

\bibitem{DLGZ2017}
%Charmonium spectrum and their electromagnetic transitions with higher multipole contributions
Wei-Jun Deng, Hui Liu, Long-Cheng Gui, Xian-Hui Zhong, Phys. Rev.D {\bf  95}  3, 034026 (2017).
    
\bibitem{BS2013} 
%Lattice QCD study of the radiative decays J/ψ→ηcγJ/ψ→ηc​γ and hc→ηcγhc​→ηc​γ
D. Becirevic, F. Sanfilippo, JHEP 01, 028 (2013).
 
\bibitem{LZ2011}
% Revisit the radiative decays of J/ψJ/ψ and ψ′→γηc(γηc′)ψ′→γηc​(γηc′​)
Gang Li, Qiang Zhao,   Phys. Rev. D{\bf  84}  074005 (2011).  
 
 \bibitem{BBCOS2015}
%Lattice gluon propagator in renormalizable ξξ gauges
 P. Bicudo, D. Binosi, N. Cardoso, O. Oliveira, P. J. Silva,  Phys. Rev. D {\bf 92} 11, 114514 (2015).

\bibitem{BBCOS2016}
%Gauge fixing and the gluon propagator in renormalizable xi gauges
 P. Bicudo, D. Binosi, N. Cardoso, O. Oliveira, P. J. Silva, talk at  Lattice 2015,        PoS LATTICE2015 (2016) 317;  e-Print: 1509.06737.

\bibitem{NAHP2021}
M Napetschnig, R. Alkofer, M. Q. Huber, J. M. Pawlowski Phys. Rev. D{\bf  104}, 054003 (2021). 

\bibitem{FAOLSI2020}
% The analytic structure of the lattice Landau gauge gluon and ghost propagators
 A. F. Falcão, O. Oliveira, P. J. Silva, Phys. Rev. D {\bf 102}, 114518 (2020). 

\bibitem{mujweb}
author web pages at: gemma.ujf.cas.cz

\bibitem{BFGIKL2010}
% Relativistic constituent quark model with infrared confinement
T. Branz, A. Faessler, T. Gutsche, M. A. Ivanov, J. G. Körner, and V. E. Lyubovitskij, Phys. Rev. D {\bf 81}, 034010 (2010).

\bibitem{GGIL2021}
%Radiative transitions of charmonium states in the covariant confined quark model
G. Ganbold, T. Gutsche, M. A. Ivanov, and V. E. Lyubovitskij, Phys. Rev. D {\bf 104}, 094048 (2021).

\bibitem{DDIL2022}
S. Dubnicka, A. Z. Dubnickova, M.A. Ivanov , A. Liptaj, Phys. Rev. D {\bf 106}, 033006 (2022).

\bibitem{FNW2008}
%On Gribov's supercriticality picture of quark confinement
C.S. Fischer, D. Nickel, R. Williams, Eur. Phys. J. C{\bf 60}, 49 (2009).

%Gauge dependence of the quark gap equation: An exploratory study
\bibitem{LSBBO2023}
 J. R. Lessa, F. E. Serna, B. El-Bennich, A. Bashir, O. Oliveira, Phys. Rev. D{\bf  107}, 074017 (2023).

\bibitem{IW1989}
N. Isgur and M. Wise , Phys. Lett.  B{\bf 232} (1989). 

\bibitem{IW1990}
N. Isgur and M. Wise, Phys. Lett. B{\bf 237}, 527 (1990). 

\bibitem{SAJHP2003}
V. Sauli, JHEP 02,001 (2003).

\bibitem{CAGA1961}
N. Cabibbo and R. Gatto, Phys. Rev. {\bf 224} , 1577 (1961).

\bibitem{ACS2023}
 A. Accardi, C. S. R. Costa, A. Signori, ArXiv: 2307.10152 .

\end{thebibliography}
%
\end{document}







       
