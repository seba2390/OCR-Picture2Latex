
%% bare_conf.tex
%% V1.4
%% 2012/12/27
%% by Michael Shell
%% See:
%% http://www.michaelshell.org/
%% for current contact information.
%%
%% This is a skeleton file demonstrating the use of IEEEtran.cls
%% (requires IEEEtran.cls version 1.8 or later) with an IEEE conference paper.
%%
%% Support sites:
%% http://www.michaelshell.org/tex/ieeetran/
%% http://www.ctan.org/tex-archive/macros/latex/contrib/IEEEtran/
%% and
%% http://www.ieee.org/

%%*************************************************************************
%% Legal Notice:
%% This code is offered as-is without any warranty either expressed or
%% implied; without even the implied warranty of MERCHANTABILITY or
%% FITNESS FOR A PARTICULAR PURPOSE! 
%% User assumes all risk.
%% In no event shall IEEE or any contributor to this code be liable for
%% any damages or losses, including, but not limited to, incidental,
%% consequential, or any other damages, resulting from the use or misuse
%% of any information contained here.
%%
%% All comments are the opinions of their respective authors and are not
%% necessarily endorsed by the IEEE.
%%
%% This work is distributed under the LaTeX Project Public License (LPPL)
%% ( http://www.latex-project.org/ ) version 1.3, and may be freely used,
%% distributed and modified. A copy of the LPPL, version 1.3, is included
%% in the base LaTeX documentation of all distributions of LaTeX released
%% 2003/12/01 or later.
%% Retain all contribution notices and credits.
%% ** Modified files should be clearly indicated as such, including  **
%% ** renaming them and changing author support contact information. **
%%
%% File list of work: IEEEtran.cls, IEEEtran_HOWTO.pdf, bare_adv.tex,
%%                    bare_conf.tex, bare_jrnl.tex, bare_jrnl_compsoc.tex,
%%                    bare_jrnl_transmag.tex
%%*************************************************************************

% *** Authors should verify (and, if needed, correct) their LaTeX system  ***
% *** with the testflow diagnostic prior to trusting their LaTeX platform ***
% *** with production work. IEEE's font choices can trigger bugs that do  ***
% *** not appear when using other class files.                            ***
% The testflow support page is at:
% http://www.michaelshell.org/tex/testflow/



% Note that the a4paper option is mainly intended so that authors in
% countries using A4 can easily print to A4 and see how their papers will
% look in print - the typesetting of the document will not typically be
% affected with changes in paper size (but the bottom and side margins will).
% Use the testflow package mentioned above to verify correct handling of
% both paper sizes by the user's LaTeX system.
%
% Also note that the "draftcls" or "draftclsnofoot", not "draft", option
% should be used if it is desired that the figures are to be displayed in
% draft mode.
%
\documentclass[conference]{IEEEtran}
% Add the compsoc option for Computer Society conferences.
%
% If IEEEtran.cls has not been installed into the LaTeX system files,
% manually specify the path to it like:
% \documentclass[conference]{../sty/IEEEtran}





% Some very useful LaTeX packages include:
% (uncomment the ones you want to load)


% *** MISC UTILITY PACKAGES ***
%
%\usepackage{ifpdf}
% Heiko Oberdiek's ifpdf.sty is very useful if you need conditional
% compilation based on whether the output is pdf or dvi.
% usage:
% \ifpdf
%   % pdf code
% \else
%   % dvi code
% \fi
% The latest version of ifpdf.sty can be obtained from:
% http://www.ctan.org/tex-archive/macros/latex/contrib/oberdiek/
% Also, note that IEEEtran.cls V1.7 and later provides a builtin
% \ifCLASSINFOpdf conditional that works the same way.
% When switching from latex to pdflatex and vice-versa, the compiler may
% have to be run twice to clear warning/error messages.






% *** CITATION PACKAGES ***
%
%\usepackage{cite}
% cite.sty was written by Donald Arseneau
% V1.6 and later of IEEEtran pre-defines the format of the cite.sty package
% \cite{} output to follow that of IEEE. Loading the cite package will
% result in citation numbers being automatically sorted and properly
% "compressed/ranged". e.g., [1], [9], [2], [7], [5], [6] without using
% cite.sty will become [1], [2], [5]--[7], [9] using cite.sty. cite.sty's
% \cite will automatically add leading space, if needed. Use cite.sty's
% noadjust option (cite.sty V3.8 and later) if you want to turn this off
% such as if a citation ever needs to be enclosed in parenthesis.
% cite.sty is already installed on most LaTeX systems. Be sure and use
% version 4.0 (2003-05-27) and later if using hyperref.sty. cite.sty does
% not currently provide for hyperlinked citations.
% The latest version can be obtained at:
% http://www.ctan.org/tex-archive/macros/latex/contrib/cite/
% The documentation is contained in the cite.sty file itself.






% *** GRAPHICS RELATED PACKAGES ***
%
\ifCLASSINFOpdf
  % \usepackage[pdftex]{graphicx}
  % declare the path(s) where your graphic files are
  % \graphicspath{{../pdf/}{../jpeg/}}
  % and their extensions so you won't have to specify these with
  % every instance of \includegraphics
  % \DeclareGraphicsExtensions{.pdf,.jpeg,.png}
\else
  % or other class option (dvipsone, dvipdf, if not using dvips). graphicx
  % will default to the driver specified in the system graphics.cfg if no
  % driver is specified.
  % \usepackage[dvips]{graphicx}
  % declare the path(s) where your graphic files are
  % \graphicspath{{../eps/}}
  % and their extensions so you won't have to specify these with
  % every instance of \includegraphics
  % \DeclareGraphicsExtensions{.eps}
\fi
% graphicx was written by David Carlisle and Sebastian Rahtz. It is
% required if you want graphics, photos, etc. graphicx.sty is already
% installed on most LaTeX systems. The latest version and documentation
% can be obtained at: 
% http://www.ctan.org/tex-archive/macros/latex/required/graphics/
% Another good source of documentation is "Using Imported Graphics in
% LaTeX2e" by Keith Reckdahl which can be found at:
% http://www.ctan.org/tex-archive/info/epslatex/
%
% latex, and pdflatex in dvi mode, support graphics in encapsulated
% postscript (.eps) format. pdflatex in pdf mode supports graphics
% in .pdf, .jpeg, .png and .mps (metapost) formats. Users should ensure
% that all non-photo figures use a vector format (.eps, .pdf, .mps) and
% not a bitmapped formats (.jpeg, .png). IEEE frowns on bitmapped formats
% which can result in "jaggedy"/blurry rendering of lines and letters as
% well as large increases in file sizes.
%
% You can find documentation about the pdfTeX application at:
% http://www.tug.org/applications/pdftex





% *** MATH PACKAGES ***
%
\usepackage[cmex10]{amsmath}
% A popular package from the American Mathematical Society that provides
% many useful and powerful commands for dealing with mathematics. If using
% it, be sure to load this package with the cmex10 option to ensure that
% only type 1 fonts will utilized at all point sizes. Without this option,
% it is possible that some math symbols, particularly those within
% footnotes, will be rendered in bitmap form which will result in a
% document that can not be IEEE Xplore compliant!
%
% Also, note that the amsmath package sets \interdisplaylinepenalty to 10000
% thus preventing page breaks from occurring within multiline equations. Use:
%\interdisplaylinepenalty=2500
% after loading amsmath to restore such page breaks as IEEEtran.cls normally
% does. amsmath.sty is already installed on most LaTeX systems. The latest
% version and documentation can be obtained at:
% http://www.ctan.org/tex-archive/macros/latex/required/amslatex/math/





% *** SPECIALIZED LIST PACKAGES ***
%
%\usepackage{algorithmic}
% algorithmic.sty was written by Peter Williams and Rogerio Brito.
% This package provides an algorithmic environment fo describing algorithms.
% You can use the algorithmic environment in-text or within a figure
% environment to provide for a floating algorithm. Do NOT use the algorithm
% floating environment provided by algorithm.sty (by the same authors) or
% algorithm2e.sty (by Christophe Fiorio) as IEEE does not use dedicated
% algorithm float types and packages that provide these will not provide
% correct IEEE style captions. The latest version and documentation of
% algorithmic.sty can be obtained at:
% http://www.ctan.org/tex-archive/macros/latex/contrib/algorithms/
% There is also a support site at:
% http://algorithms.berlios.de/index.html
% Also of interest may be the (relatively newer and more customizable)
% algorithmicx.sty package by Szasz Janos:
% http://www.ctan.org/tex-archive/macros/latex/contrib/algorithmicx/




% *** ALIGNMENT PACKAGES ***
%
%\usepackage{array}
% Frank Mittelbach's and David Carlisle's array.sty patches and improves
% the standard LaTeX2e array and tabular environments to provide better
% appearance and additional user controls. As the default LaTeX2e table
% generation code is lacking to the point of almost being broken with
% respect to the quality of the end results, all users are strongly
% advised to use an enhanced (at the very least that provided by array.sty)
% set of table tools. array.sty is already installed on most systems. The
% latest version and documentation can be obtained at:
% http://www.ctan.org/tex-archive/macros/latex/required/tools/


% IEEEtran contains the IEEEeqnarray family of commands that can be used to
% generate multiline equations as well as matrices, tables, etc., of high
% quality.




% *** SUBFIGURE PACKAGES ***
%\ifCLASSOPTIONcompsoc
%  \usepackage[caption=false,font=normalsize,labelfont=sf,textfont=sf]{subfig}
%\else
%  \usepackage[caption=false,font=footnotesize]{subfig}
%\fi
% subfig.sty, written by Steven Douglas Cochran, is the modern replacement
% for subfigure.sty, the latter of which is no longer maintained and is
% incompatible with some LaTeX packages including fixltx2e. However,
% subfig.sty requires and automatically loads Axel Sommerfeldt's caption.sty
% which will override IEEEtran.cls' handling of captions and this will result
% in non-IEEE style figure/table captions. To prevent this problem, be sure
% and invoke subfig.sty's "caption=false" package option (available since
% subfig.sty version 1.3, 2005/06/28) as this is will preserve IEEEtran.cls
% handling of captions.
% Note that the Computer Society format requires a larger sans serif font
% than the serif footnote size font used in traditional IEEE formatting
% and thus the need to invoke different subfig.sty package options depending
% on whether compsoc mode has been enabled.
%
% The latest version and documentation of subfig.sty can be obtained at:
% http://www.ctan.org/tex-archive/macros/latex/contrib/subfig/




% *** FLOAT PACKAGES ***
%
%\usepackage{fixltx2e}
% fixltx2e, the successor to the earlier fix2col.sty, was written by
% Frank Mittelbach and David Carlisle. This package corrects a few problems
% in the LaTeX2e kernel, the most notable of which is that in current
% LaTeX2e releases, the ordering of single and double column floats is not
% guaranteed to be preserved. Thus, an unpatched LaTeX2e can allow a
% single column figure to be placed prior to an earlier double column
% figure. The latest version and documentation can be found at:
% http://www.ctan.org/tex-archive/macros/latex/base/


%\usepackage{stfloats}
% stfloats.sty was written by Sigitas Tolusis. This package gives LaTeX2e
% the ability to do double column floats at the bottom of the page as well
% as the top. (e.g., "\begin{figure*}[!b]" is not normally possible in
% LaTeX2e). It also provides a command:
%\fnbelowfloat
% to enable the placement of footnotes below bottom floats (the standard
% LaTeX2e kernel puts them above bottom floats). This is an invasive package
% which rewrites many portions of the LaTeX2e float routines. It may not work
% with other packages that modify the LaTeX2e float routines. The latest
% version and documentation can be obtained at:
% http://www.ctan.org/tex-archive/macros/latex/contrib/sttools/
% Do not use the stfloats baselinefloat ability as IEEE does not allow
% \baselineskip to stretch. Authors submitting work to the IEEE should note
% that IEEE rarely uses double column equations and that authors should try
% to avoid such use. Do not be tempted to use the cuted.sty or midfloat.sty
% packages (also by Sigitas Tolusis) as IEEE does not format its papers in
% such ways.
% Do not attempt to use stfloats with fixltx2e as they are incompatible.
% Instead, use Morten Hogholm'a dblfloatfix which combines the features
% of both fixltx2e and stfloats:
%
% \usepackage{dblfloatfix}
% The latest version can be found at:
% http://www.ctan.org/tex-archive/macros/latex/contrib/dblfloatfix/




% *** PDF, URL AND HYPERLINK PACKAGES ***
%
%\usepackage{url}
% url.sty was written by Donald Arseneau. It provides better support for
% handling and breaking URLs. url.sty is already installed on most LaTeX
% systems. The latest version and documentation can be obtained at:
% http://www.ctan.org/tex-archive/macros/latex/contrib/url/
% Basically, \url{my_url_here}.




% *** Do not adjust lengths that control margins, column widths, etc. ***
% *** Do not use packages that alter fonts (such as pslatex).         ***
% There should be no need to do such things with IEEEtran.cls V1.6 and later.
% (Unless specifically asked to do so by the journal or conference you plan
% to submit to, of course. )a

% correct bad hyphenation here
%\hyphenation{op-tical net-works semi-conduc-tor}
\usepackage{amsmath}
\usepackage{amsfonts}
\usepackage{amssymb}
\usepackage{graphicx}%Allows you to import images
\usepackage{adjustbox}
\usepackage[T1]{fontenc}
\usepackage{caption}
\usepackage{subcaption}
\newcommand{\RNum}[1]{\uppercase\expandafter{\romannumeral #1\relax}}
\usepackage{mathtools}
\usepackage{array}
\newcolumntype{P}[1]{>{\centering\arraybackslash}p{#1}}
\begin{document}
%
% paper title
% can use linebreaks \\ within to get better formatting as desired
% Do not put math or special symbols in the title.
\title{\textit{Removal of Narrowband Interference (PLI in ECG Signal) Using Ramanujan Periodic Transform (RPT)}}
% author names and affiliations
% use a multiple column layout for up to three different
% affiliations
\author{\IEEEauthorblockN{Basheeruddin Shah Shaik${^1}$, Vijay Kumar Chakka${^2}$, Srikanth Goli${^3}$, A. Satyanarayana Reddy${^4}$}
\IEEEauthorblockA{${^1{^,}^2{^,}^3}$Department of Electrical Engineering\\
${^4}$Department of Mathematics\\
Shiv Nadar University\\
Greater Noida, India\\
${^1}$bs600@snu.edu.in,${^2}$vijay.chakka@snu.edu.in,${^3}$gs499@snu.edu.in,${^4}$satyanarayana.reddy@snu.edu.in}
}
% conference papers do not typically use \thanks and this command
% is locked out in conference mode. If really needed, such as for
% the acknowledgment of grants, issue a \IEEEoverridecommandlockouts
% after \documentclass
% for over three affiliations, or if they all won't fit within the width
% of the page, use this alternative format:
% 
%\author{\IEEEauthorblockN{Michael Shell\IEEEauthorrefmark{1},
%Homer Simpson\IEEEauthorrefmark{2},
%James Kirk\IEEEauthorrefmark{3}, 
%Montgomery Scott\IEEEauthorrefmark{3} and
%Eldon Tyrell\IEEEauthorrefmark{4}}
%\IEEEauthorblockA{\IEEEauthorrefmark{1}School of Electrical and Computer Engineering\\
%Georgia Institute of Technology,
%Atlanta, Georgia 30332--0250\\ Email: see http://www.michaelshell.org/contact.html}
%\IEEEauthorblockA{\IEEEauthorrefmark{2}Twentieth Century Fox, Springfield, USA\\
%Email: homer@thesimpsons.com}
%\IEEEauthorblockA{\IEEEauthorrefmark{3}Starfleet Academy, San Francisco, California 96678-2391\\
%Telephone: (800) 555--1212, Fax: (888) 555--1212}
%\IEEEauthorblockA{\IEEEauthorrefmark{4}Tyrell Inc., 123 Replicant Street, Los Angeles, California 90210--4321}}




% use for special paper notices
%\IEEEspecialpapernotice{(Invited Paper)}




% make the title area
\maketitle

% As a general rule, do not put math, special symbols or citations
% in the abstract
\begin{abstract}
Suppression of interference from narrowband frequency signals play vital role in many signal processing and communication applications. A transform based method for suppression of narrow band interference in a biomedical signal is proposed. As a specific example Electrocardiogram (ECG) is considered for the analysis. ECG is one of the widely used biomedical signal. ECG signal is often contaminated with baseline wander noise, powerline interference (PLI) and artifacts (bioelectric signals), which complicates the processing of raw ECG signal. This work proposes an approach using Ramanujan periodic transform for reducing PLI and is tested on a subject data from MIT-BIH Arrhythmia database. A sum ($E$) of Euclidean error per block ($e_i$) is used as measure to quantify the suppression capability of RPT and notch filter based methods. The transformation is performed for different lengths ($N$), namely $36$, $72$, $108$, $144$, $180$. Every doubling of $N$-points results in  $50{\%}$ reduction in error ($E$). 
\end{abstract}
\begin{IEEEkeywords}
 Ramanujan sum; RPT; PLI; Ramanujan Space; ECG; Notch filter.
\end{IEEEkeywords}
% no keywordste of art
% For peer review papers, you can put extra information on the cover
% page as needed:
% \ifCLASSOPTIONpeerreview
% \begin{center} \bfseries EDICS Category: 3-BBND \end{center}
% \fi
%
% For peerreview papers, this IEEEtran command inserts a page break and
% creates the second title. It will be ignored for other modes.
\IEEEpeerreviewmaketitle
\section{Introduction}
% no \IEEEPARstart
Interference caused in any system/signal by narrow band of frequencies known as narrowband interference. In the past several years there has been steady increase in research involving reduction of narrowband interference \cite{1094725}. The basic idea for any transformation technique is representing an input signal as a linear combination of orthogonal/linearly independent basis functions. A finite set of orthogonal complex exponential basis functions are used in finite length Fourier transform. A set of Ramanujan sums are used as basis functions for Ramanujan FIR Transform (RFT). Ramanujan sums and its linearly independent circular shifts forms basis functions for Ramanujan Periodic Transform (RPT) \cite{6839030}. One of the major advantage of RPT and RFT is having an integer basis.

Electrocardiogram (ECG) is an electrical record of depolarization and re-polarization activity of heart. Nobel laureate Willem Einthoven recorded an ECG signal in 1903 by placing electrodes on limbs and chest. Preprocessing of bio-potential signals such as ECG is necessary in applications like diagnosis of disease \cite{hosseini2006}, and QRS complex delineation for feature extraction\cite{1275572}, \cite{7383914}, to eliminate the contaminated artifacts, noises and interference. Powerline interference is an interference caused in ECG signal due to Electromagnetic interference (EMI) of frequency 50/60Hz from supply lines. The classical method for elimination of power line interference is passing the signal through a filter which reduces the interference, in the state of art, many methods exist to reduce the power line interference, few of them are based on adaptive filtering \cite{1451965}, using wavelets \cite{5670602}, based on state space recursive least squares filtering \cite{6530021}, and a well known method, design of adaptive and non-adaptive notch filter \cite{477707}. In this paper a method based on RPT is used to remove the power line interference.

In Section \RNum{2} a detailed description about Ramanujan sum, Ramanujan space and Ramanujan periodic transform is described. An overview of proposed methodology and simulation results are presented in section \RNum{3}. Summary of the overall paper and future work is discussed in Section \RNum{4}.

\section{RAMANUJAN PERIODIC TRANSFORM}
For a fixed positive integer $m$, the great Indian Mathematician Srinivasa Ramanujan introduced an trigonometrical summation, now called 
as Ramanujan sum \cite{Ramanujan}, denoted $S_m(n)$ and is defined as
%\subsection{Ramanujan Sum}
%India famous mathematician Srinivasa Ramanujan introduced a trigonometric summation known as Ramanujan sum \cite{Ramanujan}, for a given value of $m$, the form of summation is shown in \eqref{eq1},
\begin{equation}
\label{eq1}
 s_m(n) = \sum_{k=1,(k,m)=1}^{m}e^{\frac{j2{\pi}kn}{m}},
\end{equation}
where $(k,m)$ denotes the greatest common divisor (gcd) between $k$ and $m$. It is easy to check that $S_m(n)$ is a periodic and integer valued  function with period $m,$ thus it is a finite sequence.  For example, 
$S_3(n)=2,-1,-1.$
%Consider $m=3$ then,
%\begin{equation}
%\label{eq2}
%\begin{aligned}
% s_3(n) & = e^{\frac{j2{\pi}n}{3}}+e^{\frac{j4{\pi}n}{3}}, {0}\leq{n}\leq {2}\\
% \Rightarrow s_3(n) & = {\{2, -1, -1\}}
%\end{aligned}
%\end{equation}
This attracts many signal processing researchers to use $s_m (n)$ as basis function. 

Let $\mathbb{N}$ and $\mathbb{C}$ denote the set of natural numbers (starting from $0$) and complex numbers respectively. Then a mapping 
$f:\mathbb{N}\to \mathbb{C}$ is called an arithmetic function. For example, for 
$n \in \mathbb{N}$, let $\phi(n)$ denote the number of positive integers not exceeding $n$
that are relatively prime to $n$. That is 
$\phi(n)=\#\{m|1\leq m\leq n, (m,n)=1\}.$  The arithmetic function $\phi(n)$ is called Euler's Phi-function or Euler's totient function.
Ramanujan's motivation for introducing this summation is to represent any arithmetic function as a linear combination of Ramanujan sums. In this paper we treat an arithmetic function as a sequence and if the arithmetic function is periodic, then it is a finite sequence.
% Few examples of arithmetic functions are Mobius function $\mu(n)$, Euler's totient function $\phi(n)$. For a given n, totient function gives the information about number of integers $k$ in ${1}\leq{l}\leq{n}$ satisfying $(l,n)=1$, for example $\phi(3)= 2$. 
For a given $m$ it is easy to verify $s_m(0) = \phi(m)$. Consider any two Ramanujan sums $s_{m_1}(n),s_{m_2}(n),$ they are orthogonal to each other over the sequence length $l$, where $l = lcm(m_1,m_2)$, i.e.,
\begin{equation}
\label{eq3}
\sum_{n=0}^{l-1}s_{m_1}(n)s_{m_2}(n-k) = 0,{\quad} {m_1}\neq{m_2},
\end{equation} for any integer $k$, where $0\leq k \leq {l-1}$.
 
\subsection{Ramanujan Space}
The column space of integer circulant matrix $D_m$ of size $m{\times}m$ forms the Ramanujan space $V_m$ \cite{6839014}. The first column of the matrix $D_m$ is the sequence $ s_m(n)$, the remaining $m-1$ columns are circular down shift of the previous columns. Let $m=3$, then,
\begin{equation}
D_3 = \begin{bmatrix}
s_3(0) & s_3(2) & s_3(1) \\
s_3(1) & s_3(0) & s_3(2) \\
s_3(2) & s_3(1) & s_3(0)
\end{bmatrix} = \begin{bmatrix}
2 & -1 & -1 \\
-1 & 2 & -1 \\
-1 & -1 & 2
\end{bmatrix}.
\end{equation}
From theorem 5 in \cite{6839014}, it is clear that any consecutive $\phi(m)$ columns of $D_m$ matrix forms the basis for $V_m$, but 
 not  an orthogonal basis unless $m$ is power of $2.$
%These basis vectors are not orthogonal to each other (unless m is a power of 2) but they are linearly independent. 
Then any sequence $x(n)\in{V_m}$ can be expressed as follows,
\begin{equation}
x(n) = \sum_{k=0}^{\phi{(m)}-1}{\beta}_{k}s_m(n-k).
\end{equation}
It is shown in Theorem $10$ in \cite{6839014} that if we add two periodic signals  from Ramanujan space $V_m$, then the period of 
resultant signal is the LCM of periods of individual signals.  It is not true 
for any periodic signals, there the period of 
resultant signal is a divisor of  LCM of periods of individual signals.
%when two periodic signals are added then the period of resultant signal is LCM ($L$) of the individual periods or a divisor of $L$, but if we add the signals from Ramanujan space $V_m$ the period of resultant signal is exactly equal to LCM ($L$) of the individual periods (Theorem 10 in \cite{6839014}). 

If signals from different Ramanujan spaces are added i.e., 
\begin{equation}
x(n) = \sum_{{i}=1}^{k}{x_{m_i}(n)} ,{\quad}x_{m_i}(n)\in V_{m_i},
\end{equation}
then the period of $x(n)$ is equal to $N$, assuming that none of the $x_m (n)$ is identically zero (Theorem 12 in \cite{6839014}), where $N=lcm(m_1,m_2,\dotsc ,m_k)$. These are the two important properties of sequences belongs to Ramanujan spaces, are helpful for understanding RPT, as explained in subsection D.
\subsection{Relation between Ramanujan Spaces and DFT Matrix}
According to factorization property of $D_m$  \cite{6839014},
\begin{equation}
\label{eq:DM}
D_m=W W^H,
\end{equation}
where $W$ is an $m\times \phi(m)$ matrix obtained from the DFT matrix $F$ by choosing those columns such that the column numbers are relatively prime to $m.$ So the columns of $W$ acts as a basis for $V_m.$
%\begin{equation}
%D_m =W_{m{\times}{\phi}(m)}{W_{{\phi}(m){\times}m}^H}
%\end{equation}
%where $W$ is a sub-matrix obtained from DFT matrix ($F$). The columns of $W$ matrix are obtained from columns of $F$, whose column index gcd with $m$ is $1$.
%For example consider $m=2$, then
%\begin{equation}
%B_2 = \begin{bmatrix}
%1 & -1 & \\
%-1 & 1 &  \\
%\end{bmatrix} = \begin{bmatrix}
%1 \\
%e^{\frac{-j2\pi}{2}} \\
%\end{bmatrix}\begin{bmatrix}
%1 & e^{\frac{j2\pi}{2}}
%\end{bmatrix}
%\end{equation}
%So, the matrix $W$ acts like a basis for any sequence belongs to Ramanujan space. 
If $x(n){\in}V_m$ then $x(n)$ can be expressed in any one of the following forms,
\begin{equation}
x(n) = \sum_{k=0}^{\phi{(m)}-1}{\beta}_{k}s_m(n-k) = \sum_{1{\leq}k{\leq}m,(k,m)=1} {\alpha_k}{e^\frac{-j2{\pi}kn}{m}},
\end{equation}
each term (in complex exponential summation) has period $m$, and the frequencies are,
\begin{equation}
\label{eq10}
w_k = \frac{2{\pi}k}{m},{\quad}{1{\leq}k{\leq}m},\quad {(k,m)=1}.
\end{equation}
So, \textit{\bfseries{Ramanujan space having period $\bold{m}$ has different frequency components}}.
\subsection{RPT}
In this transformation, a representation of input signal $x(n)$ of length $N$ as a linear combination of signals belongs to Ramanujan space $V_{m_i}$ (basis) is used \cite{6839030}, i.e.,
\begin{equation}
x(n) = \sum_{{m_i}|N}\sum_{k=0}^{\phi{(m_i)}-k}{\beta_i}_{k}s_{m_i}(n-k),
\end{equation}
where $m_i |N$ means the summation is executed for those $m_i$ values, which are divisors of $N$. Let us consider $N = 4$ length input sequence, then the divisors $(m_i)$ are $1$,$2$ and $4$, i.e., the input signal is represented as a linear combination of signals from Ramanujan spaces $V_1$,$V_2$, and $V_4$. 
\begin{equation}
\begin{aligned}
x(n) & = x_1 (n)  + x_2 (n)  + x_4 (n),\\ & \quad {x_1}(n) {\in} {V_1},{\quad} { x_2(n)} {\in}{V_2} {\quad}{ \&}{\quad} { x_4(n)} {\in} {V_4}.\\
 & = \sum_{l=0}^{\phi{(1)}-1}{\beta_1}_{l}s_{1}(n-l)+\sum_{l=0}^{\phi{(2)}-1}{\beta_2}_{l}s_{2}(n-l)\\
& +\sum_{l=0}^{\phi{(4)}-1}{\beta_4}_{l}s_{4}(n-l),{\quad} 0{\leq}n{\leq}3\\
& = \beta_{10}{s_1}(n)+\beta_{20}{s_2}(n)+\beta_{40}{s_4}(n)+\beta_{41}{s_4}(n-1)\\
x & = \underbrace{\begin{bmatrix}
{s_1}(0) & {s_2}(0) & {s_4}(0) & {s_4}(3)\\
{s_1}(0) & {s_2}(1) & {s_4}(1) & {s_4}(0)\\
{s_1}(0) & {s_2}(0) & {s_4}(2) & {s_4}(1)\\
\underbrace{{s_1}(0)}_{R_1} & \underbrace{{s_2}(1)}_{R_2} & \underbrace{{s_4}(3)}_{R_4} & \underbrace{{s_4}(2)}_{R_4}
\end{bmatrix}}_{T_4}
\underbrace{\begin{bmatrix}
\beta_{10}\\
\beta_{20}\\
\beta_{40}\\
\beta_{41}
\end{bmatrix}}_{\beta}\\ & = {[{R_1}\quad {R_2}\quad {R_4}]}{\beta}\\ & x = {T_4}{\beta}.
\end{aligned}
\end{equation}
This synthesis representation of $x(n)$ is known as Ramanujan Periodic Representation (RPR). In general for a given $N$ length sequence, the RPR is,
\begin{equation}
\label{RPR}
x(n) = {T_N}\beta(K),
\end{equation}
%In the above matrix $T_4$ each column has an integer number of periods of respective Ramanujan sum, like column $1$ is a repetition of $s_1$ $4$ times, column 2 is a repetition of $s_2$ $2$ times, column $3$ and $4$ are repetition of $s_4$ $1$ time, this is because all ${m_i}$are divisors of $N$. 
where the values of $\beta$ are known as Ramanujan coefficients and $T_N$ is the transformation matrix. From Orthogonality property (Equation  \eqref{eq3}) the column space of $R_{m_i}$ is orthogonal to $R_{m_k}$ for $i{\neq}k$, i.e.,
\begin{equation}
\label{orthogonal}
{R_{m_i}^H}{R_{m_k}} = 0,{\quad} k{\neq}i.
\end{equation}
The projection of input signal onto a specific space can be calculated using equation \eqref{Projection}.
\begin{equation}
\label{Projection}
x_{m_{i}} = \underbrace{R_{m_{i}}({R_{m_{i}}}^{H}R_{m_{i}})^{-1}{R_{m_{i}}}^{H}}_{{Projection}{\quad}{matrix}}x.
\end{equation}

%In general any $R_{m_i}$ consists of $\phi({m_i})$ number of columns, where first column is repetition of $s_{m_i}$ for $\frac{N}{m_i}$ number times, remaining $\phi({m_i})-1$ columns are circular down shift of the previous column, these $\phi({m_i})$ columns acts as basis vectors (linearly independent) for the space $V_{m_i}$ (Section \RNum{2}-B).%
Let $m_1$,$m_2$,$...$,$m_k$,$N$ are the possible divisors of $N$, then the sum of totient functions of all these divisors is equal to length of the signal \cite{6839030}, i.e.,
\begin{equation}
\Rightarrow \sum_{{m_i}|N}\phi(m_i) = N,
\end{equation}
from \eqref{RPR} the analysis equation can be written as
%From equation \eqref{orthogonal} and above statement, it is clear that the matrix $T_N$ is a full rank matrix, one can easily verify that $T_N$ is orthogonal matrix if $N$ is a power of two. The matrix form of transformation from $x$ to $\beta$ can be written as
\begin{equation}
\label{RPT}
\beta(K) = {T_N}^{-1}x(n),
\end{equation}
this transformation is known as \textit{Ramanujan periodic transform (RPT)}, also equations \eqref{RPR} and \eqref{RPT} forms Ramanujan transformation pair. If $N$ is a power of $2$, then equation  \eqref{RPT} modified as,
\begin{equation}
\beta(K) = {T_N}^{T}x(n).
\end{equation}

\section{Proposed Method for Narrowband Interference Removal and Simulation Results}
In the proposed methodology, a $N$ length narrowband interfered data ($x(n)$) is projected onto Ramanujan spaces, which are divisors of $N$. The window length is calculated based on the period of narrowband interference. 
%The total energy of the signal is equal to sum of projection energies onto different Ramanujan spaces.%
A significant amount of projection energy in a Ramanujan space indicates the presence of periodic component of that space in the signal. In order to reduce the interference, first determine all periods of narrowband interference, then identify the matched Ramanujan spaces and the coefficients ($\beta(K)$) belongs to these spaces. Then force these coefficients to zero, this can be achieved by multiplying with a window function ($W(K)$), whose values are zeros for these spaces otherwise one. Then reconstruct (using equation \eqref{RPT}) the signal ($r(n)$) using these modified Ramanujan coefficients (${\beta(K)}{W(K))}$. As an example, the entire procedure described above is applied on ECG signal to remove the $50/60$Hz powerline interference. Block diagram of the proposed methodology for analysing ECG signal is shown in Fig. \ref{f1}. 
%The algorithm consists of few stages like Data Acquisition along with adding PLI to the data, Period Estimation from the data, Implementing RPT, Processing in frequency domain using a Window Function, reconstruction of the signal using IRPT, then Error Calculation.
\begin{figure}[!h]
\centering
 \includegraphics[width=3.5in,height=1.5in]{RPT_blockdiagram.png} 
\caption[Block diagram of PLI reduction algorithm using RPT]{Block diagram of PLI reduction algorithm using RPT}
\label{f1}
\end{figure}

The brief description of each stage is presented  in the following sub-sections.
\subsection{Data Acquisition}
MIT-BIH Arrhythmia database from physionet website \cite{Physionet} is used for the analysis it consists of two channel ECG recordings of total $48$ subjects, with a sampling frequency of $360$Hz. 
%ECG-ID database consists of total $310$ recordings obtained from $90$ persons, with a sampling frequency of $500$Hz. 
The proposed methodology is analysed on a single record. To obtain a PLI ECG data a pure sinusoidal signal of $50$Hz frequency is added to the entire record. A typical powerline interfered ECG signal (Record 100 in MIT-BIH Arrhythmia) up to 500 samples is shown in Fig.\ref{Powerline interfered ECG signal}.   
\begin{figure}[!h]
\centering
\includegraphics[width=3.5in,height=1.6in]{Powerline_Interfered_ECG_Signal.png} 
\caption[Powerline interfered ECG signal]{Powerline interfered ECG signal}
	\label{Powerline interfered ECG signal}
\end{figure}
%A typical ECG signal of the record 100 in the MIT-BIH database, up to 5000 samples is shown in Fig. \ref{Original ECG signal}.
%\begin{figure}[!h]
%\centering
%\includegraphics[width=3.5in,height=1.6in]{RPT_images/Preprocessing/Raw_ECG_Signal.png} 
%\caption[Original ECG signal]{Raw ECG signal}
%	\label{Original ECG signal}
%\end{figure}
\subsection{Period Estimation}
%Period estimation plays a vital for the entire analysis, because the signal energy is preserved (i.e. projecting the signal) into the spaces which are divisors of length of the signal, by knowing the period one can estimate on which spaces the signal should be projected.% 
To estimate the period of an unknown signal, one need to decompose the signal into different Ramanujan spaces. Then it is easy to estimate the period by calculating projection energies ($|x_{m_i}|^2$). Since PLI is a $50$Hz sinusoid, with the sampling frequency of $360$Hz, corresponds to a period of $36$ (with minimum possible integer value of $k = 5$ in  \eqref{eq10}), 
%so, the signal should be projected into $V_{36}$ to know the energy contribution of $50$Hz frequency component in the entire signal energy. 
%Similarly for ECG-ID database the sampling frequency is $500$Hz corresponds to a period of $10$.
\begin{figure}[!h]
\centering
 \includegraphics[width=3.5in,height=2.2in]{Proj_figures.png} 
\caption[Decomposition of ECG signal with PLI into different periodic signals corresponding to their respective Ramanujan spaces]{Decomposition of ECG signal with PLI into different periodic signals corresponding to their respective Ramanujan spaces}
\label{Decomposition of ECG signal with PLI into different periodic signals corresponding to their respective Ramanujan spaces}
\end{figure}


\subsection{Implementation of RPT}
Now the entire signal should be divided in to number blocks such that period $36$ should be one of the divisors of block size. So, the minimal possible block size is $36$. In the current analysis a block size of $36$ is considered. The projections $x_{m_i}$ on each $m_i$ (which is a divisor of block size) is calculated using \eqref{Projection}. Figure \ref{Decomposition of ECG signal with PLI into different periodic signals corresponding to their respective Ramanujan spaces} shows the decomposition of ECG signal with PLI into different periodic signals corresponding to their respective Ramanujan spaces.

Space wise normalized RPT is implemented on each block of data. To calculate normalized RPT ($\hat{T_N}$), the following procedure is considered. Let $N = 4$, then, 
\begin{equation}
\label{matrix}
T_4 = \begin{bmatrix}
1 & 1 & 2 & 0\\
1 & -1 & 0 & 2 \\
1 & 1 & -2 & 0\\
1 & -1 & 0 & -2
\end{bmatrix}.
\end{equation}
%As early discussed if $N$ value is power of two then $T_N$ is an orthogonal matrix, i.e. it must satisfy the condition 
%\begin{equation}
%{T_N}^{T}{T_N} = I
%\end{equation}
According to \eqref{matrix}, the multiplication yields to
\begin{equation}
{T_4}^{T}{T_4} = \begin{bmatrix}
4 & 0 & 0 & 0\\
0 & 4 & 0 & 0 \\
0 & 0 & 8 & 0\\
\underbrace{0}_{V_1} & \underbrace{0}_{V_2} & \underbrace{0}_{V_4} & \underbrace{8}_{V_4}
\end{bmatrix}.
\end{equation}
The diagonal elements in the above matrix representing the space wise energy of $N$ length Ramanujan sum with period $m$ (where $m$ is a divisor of $N$). To overcome this issue a space wise normalization is performed, for example the first column ($\in{V_1}$) in $T_4$ is divided with $\sqrt{4}$, second column ($\in{V_2}$) is divided with $\sqrt{4}$, third and fourth ($\in{V_4}$) are divided with $\sqrt{8}$. By using above normalized $\hat{T_4}$, the multiplication yields to, 
%\begin{equation}
%\small{{T_N}^{T}{T_N}=\begin{bmatrix}
%0.5 & 0.5 & 0.7071 & 0\\
%0.5 & -0.5 & 0 & 0.7071 \\
%0.5 & 0.5 & -0.7071 & 0\\
%0.5 & -0.5 & 0 & -0.7071
%\end{bmatrix}\begin{bmatrix}
%0.5 & 0.5 & 0.5 & 0.5\\
%0.5 & -0.5 & 0.5 & -0.5 \\
%0.7071 & 0 & -0.7071 & 0\\
%0 & 0.7071 & 0 & -0.7071
%\end{bmatrix}=
%\end{equation}
\begin{equation}
\small{{\hat{T_4}}^{T}{\hat{T_4}}=
\begin{bmatrix}
1 & 0 & 0 & 0\\
0 & 1 & 0 & 0 \\
0 & 0 & 1 & 0\\
0 & 0 & 0 & 1
\end{bmatrix}},
\end{equation} 
%With this normalization if the value of $N$ is not a power of two then also the diagonal elements value is equal to $1$. 
%by using this normalized $\hat{T_N}$ matrix, RPT is  performed on each block of data, the obtained coefficients are known as Ramanujan Coefficients $(X(K))$, which are need to be analysed to reduce the PLI.
input signal $x(n)$ is transformed by this $\hat{T_N}$ matrix to generate the Ramanujan periodic coefficients ($\beta(K)$). These are manipulated to reduce the PLI.

\subsection{Frequency Domain Analysis and Implementation of IRPT}
Now the obtained $36$ Ramanujan coefficients ($\beta$) should be separated space wise.
%for further analysis, specifically how many belongs to $V_{36}$ to know this calculate $\phi(36)$ whose value is equal to $12$, so
The last $12$ coefficients in $36$ belongs to $V_{36}$. In order to reduce PLI, the coefficients in $V_{36}$ should be equal to zero.  For this purpose a window sequence $(W(K))$ of length $36$ is defined as,
\begin{equation}
\label{window}
W(K) = \{\underbrace{1}_{0},\underbrace{1}_{1},\underbrace{1}_{2},\cdots,\underbrace{1}_{23},\underbrace{0}_{24},\cdots,\underbrace{0}_{35}\},
\end{equation}
which is multiplied with  $\beta(K)$, resulting in modified Ramanujan coefficients (${\beta(K)}W(K)$) are reconstructed back as signal ($r(n)$) by using  Ramanujan periodic representation  \eqref{RPR} known as Inverse Ramanujan periodic transform (IRPT). The reconstructed (PLI reduced) ECG data for a particular block of $36$ samples is shown in 
\ref{PLI reduction using RPT}. The magnitude response of ECG signal with PLI, before and after applying RPT is shown in Fig. \ref{Magnitude Response of original ECG Signal and With, Without (reduced using RPT) PLI} 

\begin{figure}[!h]
\centering
 \includegraphics[width=3.5in,height=1.6in]{PLI_Reduction_Using_RPT.png} 
\caption[PLI reduction using RPT]{PLI reduction using RPT}
\label{PLI reduction using RPT}
\end{figure}
\begin{figure}[!h]
\centering
 \includegraphics[width=3.5in,height=1.6in]{Magnitude_Response_PLI_Reduced_ECG_Signal.png} 
\caption[Magnitude Response of original ECG Signal and With, Without (reduced using RPT) PLI]{Magnitude Response of original ECG Signal and With, Without (reduced using RPT) PLI}
\label{Magnitude Response of original ECG Signal and With, Without (reduced using RPT) PLI}
\end{figure}
%As stated earlier, the \textit{Ramanujan space has different frequency components all with the same period $m$}, i.e. not only $50$Hz frequency component, interference from its harmonic frequency components also can be reduced. To prove this consider an ECG signal and add a pure sinusoidal signal of $50$Hz and its harmonics upto $5^{th}$ order ( up to $250$Hz), with an exponential decay ($e^{-an}$, $a = 0.5$ is considered) in amplitude. In overall frequency components $50$Hz, $250$Hz corresponds to period $36$ and $100$Hz, $150$Hz, $200$Hz components corresponds to periods $18$, $12$, $9$ respectively. To reduce these harmonic components in the signal, we have to make the signal energy in these subspaces to zero. The analysis similar to reduction of PLI is performed on this signal, for better understanding of results $180$ samples ($5$ blocks) of signal is shown in Fig.\ref{PLI and Its Harmonic Reduction Using RPT}. The magnitude response of a single block of data is shown in Fig.\ref{Magnitude Response of PLI and Its Harmonic Reduced ECG Signal}.
%\begin{figure}[!h]
%\centering
% \includegraphics[width=3.5in,height=1.6in]{RPT_images/PLI_and_Its_Harmonic_Reduction_Using_RPT.png} 
%\caption[PLI and Its Harmonic Reduction Using RPT]{PLI and its harmonic reduction using RPT}
%\label{PLI and Its Harmonic Reduction Using RPT}
%\end{figure}
%\begin{figure}[!h]
%\centering
% \includegraphics[width=3.5in,height=1.6in]{RPT_images/Magnitude_Response_PLI_and_Its_Harmonic_Reduced_ECG_Signal.png} 
%\caption[Magnitude Response of PLI and Its Harmonic Reduced ECG Signal]{Magnitude response of PLI and its harmonic reduced ECG signal}
%\label{Magnitude Response of PLI and Its Harmonic Reduced ECG Signal}
%\end{figure}

%An important observation can be made from Fig.\ref{PLI and Its Harmonic Reduction Using RPT}, whenever there is a QRS peak in the ECG signal, RPT algorithm is unable to approximate the original signal with minimum error, because from literature the frequency range of QRS complex waveform is $8-20$Hz \cite{shah}, the frequency components $10$Hz, $20$Hz corresponds to periods $36$ and $18$ respectively, and in methodology the signal energy in these periodic spaces are made as zero. Our future works includes, how to overcome this issue by selecting a proper window (tending not to choose ones and zeros as window function values).

\subsection{PLI Reduction using Notch Filter}
The results of proposed algorithm are compared with the results obtained using notch filter, by using Euclidean error as a measure. The relation between notch frequency ($f_0$), quality factor ($Q$), and bandwidth ($\delta{f}$) is mathematically represented in equation \eqref{notch} \cite{6530021}.
\begin{equation}
\label{notch}
Q = \frac{f_0}{\delta{f}}.
\end{equation}
Notch filter with a higher attenuation level will remove the PLI noise very effectively, to achieve higher attenuation the $Q$ factor should be decreased. In this work a second order IIR notch filter with notch frequency of $50$Hz, having a quality factor of $1$ ($\delta{f} = 50$Hz) is designed. Fig. \ref{Magnitude Response Notch Filter}. shows the magnitude response of notch filer.
\begin{figure}[!h]
\centering
 \includegraphics[width=3.5in,height=1.6in]{Magnitude_Response_Notch_Filter.png} 
\caption[Magnitude Response Notch Filter]{Magnitude response notch filter}
\label{Magnitude Response Notch Filter}
\end{figure}

The analysis using notch filter is also performed block by block (similar to RPT), so the reduction of PLI in the first block of data using notch filter is shown in Fig. \ref{PLI Reduction Using Notch Filter}.
\begin{figure}[!h]
\centering
 \includegraphics[width=3.5in,height=1.6in]{PLI_Reduction_Using_Notch_Filter.png} 
\caption[PLI Reduction Using Notch Filter]{PLI reduction using notch filter}
\label{PLI Reduction Using Notch Filter}
\end{figure}
The magnitude response of the PLI reduced ECG signal using notch filter is shown in Fig.\ref{Magnitude Response PLI Reduced ECG Signal Using Notch Filter}.
\begin{figure}[!h]
\centering
 \includegraphics[width=3.5in,height=1.6in]{Magnitude_Response_PLI_Reduced_ECG_Signal_Using_Notch_Filter.png} 
\caption[Magnitude Response PLI Reduced ECG Signal Using Notch Filter]{Magnitude response PLI reduced ECG signal using notch filter}
\label{Magnitude Response PLI Reduced ECG Signal Using Notch Filter}
\end{figure}

\subsection{Euclidean Error Measurement}
For each block of data an error is calculated by using the equation \eqref{Error}.
\begin{equation}
\label{Error}
Error (e_i) = \sum_{n=0}^{N-1}|{x_i}(n)-{r_i}(n)|^2.
\end{equation}
Where ${x_i}(n)$ is the $i^{th}$ block input ECG signal with out any interference, ${r_i}(n)$ is the $i^{th}$ block interference reduced signal and $N$ is the block size. For the first block of data ${e_i}$ using RPT method is 0.0268, using Notch filter is 0.6282.

A length of $524288$ samples of ECG data is considered from the record 100 of MIT-BIH database, and it is converted to $36\times14564$ ($14564\times36 = 524304$, by appending few zeros to the considered signal) after deciding the block size. Now ${e_i}$ obtained using two methods is calculated for a total of $14564$ blocks. Then an $E$, defined in \eqref{E}, is calculated for the total blocks.
\begin{equation}
\label{E}
E = \sum_{i=1}^{14564}{e_i}.
\end{equation} 
%in this, the error for first $250$ blocks is displayed in Fig.\ref{Error per block in both RPT and Notch Filter techniques}.
%\begin{figure}[!h]
%\centering
% \includegraphics[width=3.5in,height=1.6in]{RPT_images/Error_RPT_Vs_Notch.png} 
%\caption[Error per block in both RPT and Notch Filter techniques]{Error per block in both RPT and Notch Filter techniques}
%\label{Error per block in both RPT and Notch Filter techniques}
%\end{figure}
%%Similarly error in both techniques when ECG signal interfered with $50$Hz and its harmonics (up to $5^{th}$ order) is shown in Fig. \ref{Error per block in both RPT and Notch Filter techniques when the signal is interfered with 50Hz and its harmonics}.
%%\begin{figure}[!h]
%%\centering
%% \includegraphics[width=3.5in,height=1.6in]{RPT_images/Error_RPT_Vs_Notch_Harmonics.png} 
%%\caption[Error per block in both RPT and Notch Filter techniques when the signal is interfered with 50Hz and its harmonics]{Error per block in both RPT and Notch Filter techniques when the signal is interfered with 50Hz and its harmonics}
%%\label{Error per block in both RPT and Notch Filter techniques when the signal is interfered with 50Hz and its harmonics}
%%\end{figure}
%The high error in RPT method represents the presence of QRS complex waveform in ECG signal. Whenever there is a QRS peak in the ECG signal, RPT algorithm is unable to approximate the original signal with minimum error, because from literature the frequency range of QRS complex waveform is $8-20$Hz \cite{shah}, the frequency component $10$Hz,  corresponds to period $36$, and in methodology the signal coefficients in this periodic space is made as zero. 
The above procedure is repeated for different window lengths (different $N$-point RPT's).For each length, an $E$ is calculated, and summarized in TABLE
\ref{tab:Comparison of total error between RPT and Notch methods for different block sizes}.
\begin{table}[h]
\centering
\caption{Comparison of $E$ between RPT and Notch methods for different block sizes}
\label{tab:Comparison of total error between RPT and Notch methods for different block sizes}
\begin{adjustbox}{max width=\textwidth}
\renewcommand{\arraystretch}{1.25}
\begin{tabular}{|P{2.5cm}|P{2.5cm}|P{2.5cm}|}\hline 
\bfseries{\small{Block Size}}	&	 \bfseries{\small{RPT}}	&	\bfseries{\small{Notch Filtering}} \\    \hline
\small{36}	&	\small{7280}	&	\small{13652}	\\	\hline
\small{72}	&	\small{3501}	&	\small{8119}	\\	\hline
\small{108}	&	\small{2369}	&	\small{6266}	\\	\hline
\small{144}	&	\small{1782}	&	\small{5347}	\\	\hline
\small{180}	&	\small{1440}	&	\small{4805}	\\	\hline
\end{tabular}
\end{adjustbox}
\end{table}

\section{Conclusion}
In this work, a transform domain analysis based reduction of narrowband interference has been proposed using RPT. From TABLE \ref{tab:Comparison of total error between RPT and Notch methods for different block sizes}, it has been shown that RPT is reducing PLI with minimum $E$, in comparison with notch filter technique for different data window lengths. Every doubling of data length results in approximately $50\%$ reduction in error ($E$) for the case of RPT based method.
\section*{Acknowledgement}
The authors would like to thank Dr. Krishnan Rajkumar for his continuous support and many useful discussions.
% trigger a \newpage just before the given reference
% number - used to balance the columns on the last page
% adjust value as needed - may need to be readjusted if
% the document is modified later
%\IEEEtriggeratref{8}
% The "triggered" command can be changed if desired:
%\IEEEtriggercmd{\enlargethispage{-5in}}

% references section

% can use a bibliography generated by BibTeX as a .bbl file
% BibTeX documentation can be easily obtained at:
% http://www.ctan.org/tex-archive/biblio/bibtex/contrib/doc/
% The IEEEtran BibTeX style support page is at:
% http://www.michaelshell.org/tex/ieeetran/bibtex/
%\bibliographystyle{IEEEtran}
% argument is your BibTeX string definitions and bibliography database(s)
%\bibliography{IEEEabrv,../bib/paper}
%
% <OR> manually copy in the resultant .bbl file
% set second argument of \begin to the number of references
% (used to reserve space for the reference number labels box)
\begin{thebibliography}{10}
\bibitem{1094725} L. Milstein and P. Das, {\em An analysis of a Real-Time Transform
Domain Filtering Digital Communication System - Part I: Narrow 
Band Interference Rejection}, IEEE Transactions on Communications,
vol. 28, pp. 816-824, June 1980.
\bibitem{Ramanujan}  S. Ramanujan, {\em On certain trigonometrical sums and their applications
in the theory of numbers}, Trans. Cambridge Philos. Soc, vol. 22,
no. 13, pp. 259-276, 1918.
\bibitem{hosseini2006} H. G. Hosseini, D. Luo, and K. J. Reynolds, {\em The comparison of
different feed forward neural network architectures for ecg signal
diagnosis},  Medical engineering \& physics, vol. 28, no. 4, pp. 372-378, 2006.
\bibitem{1275572} J. P. Martinez, R. Almeida, S. Olmos, A. P. Rocha, and P. Laguna, {\em A
wavelet-based ECG delineator: evaluation on standard databases},  IEEE
Transactions on Biomedical Engineering, vol. 51, pp. 570-581, April
2004.
\bibitem{7383914} B. S. Shaik, G. V. S. S. K. R. Naganjaneyulu, and A. V. Narasimhadhan,
{\em A novel approach for QRS delineation in ECG signal based on
chirplet transform},  in Electronics, Computing and Communication
Technologies (CONECCT), 2015 IEEE International Conference on,
pp. 1-5, July 2015.
\bibitem{1451965}B. Widrow, J. R. Glover, J. M. McCool, J. Kaunitz, C. S. Williams,
R. H. Hearn, J. R. Zeidler, J. E. Dong, and R. C. Goodlin, {\em Adaptive
noise cancelling: Principles and Applications},  Proceedings of the IEEE,
vol. 63, pp. 1692-1716, Dec 1975.
\bibitem{5670602} L. N. Sharma, S. Dandapat, and A. Mahanta, {\em Multiscale wavelet
energies and relative energy based denoising of ECG signal},  in
Communication Control and Computing Technologies (ICCCCT), 2010
IEEE International Conference on, pp. 491-495, Oct 2010.
\bibitem{6530021} M. Butt, N. Razzaq, I. Sadiq, M. Salman, and T. Zaidi,{\em Power
Line Interference removal from ECG signal using SSRLS algorithm},
in Signal Processing and its Applications (CSPA), 2013 IEEE 9th
International Colloquium on, pp. 95-98, March 2013.
\bibitem{477707} P. S. Hamilton, {\em A comparison of adaptive and nonadaptive filters for
reduction of power line interference in the ECG},  IEEE Transactions
on Biomedical Engineering, vol. 43, pp. 105-109, Jan 1996.
\bibitem{6839014} P. P. Vaidyanathan, {\em Ramanujan Sums in the Context of Signal
Processing-Part I: Fundamentals},  IEEE Transactions on Signal Pro-
cessing, vol. 62, pp. 4145-4157, Aug 2014.
\bibitem{6839030} P. P. Vaidyanathan, {\em Ramanujan Sums in the Context of Signal
Processing-Part II: FIR Representations and Applications}, IEEE Trans
actions on Signal Processing, vol. 62, pp. 4158-4172, Aug 2014.
\bibitem{Physionet} G. Moody, R. Mark, and A. Goldberger, {\em PhysioNet: A Web-based
resource for the study of physiologic signals}, IEEE Engineering in
Medicine and Biology Magazine, vol. 20, pp. 70-75, May 2001.
\end{thebibliography}
% that's all folks
\end{document}