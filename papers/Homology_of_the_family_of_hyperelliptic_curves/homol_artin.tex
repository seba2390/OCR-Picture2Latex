We collect here some of the results
% from \cite{calmar} 
concerning the homology of $\Ga{n}$ and $\Gb{n}$ with constant coefficients and with coefficients in abelian local systems.
We follow the notation used in \cite{calmar}. 

Given an element $x \in \{0,1\}^n$ we can write it as a list of $0$'s and $1$'s. We identify such an element $x$ with a string of $0$'s and $1$'s.

Recall the definition of the following $q$-analog and $q,t$-analog polynomials with integer coefficients:
$$ 
[0]_q :=1, \qquad [m]_q := 1 + q + \cdots + q^{m-1} = \frac{q^m-1}{q-1} \mbox{ for }m\geq 1,
$$
$$
[m]_q! := \prod_{i=1}^m [m]_q, \qquad [2m]_{q,t} := [m]_q (1+tq^{m-1}),
$$
$$
[2m]_{q,t}!! := \prod_{i=1}^m [2i]_{q,t}\ =\ [ m ]_q!
\prod_{i=0}^{m-1} (1+tq^i),
$$
$$
\qbin{m}{i}_{q}\!\!: = \frac{[m]_q!}{[i]_q!
	[m-i]_q!},
%$$
\qquad
%$$
\qbin{m}{i}_{q,t}'\!\!: = \frac{[2m]_{q,t}!!}{[2i]_{q,t}!! [m-i]_q!} \ =
\qbin{m}{i}_{q}\prod_{j=i}^{m-1}(1+tq^j).
$$
In the following we specialize $q = -1$ and we write $\qbin{m}{i}_{-1}'$ for $\qbin{m}{i}_{-1,t}'$.

%The Artin group $\Ga{n}$ has a natural action on the module $\md$  by mapping each standard generator to multiplication by $(-t)$.


The homology of the Artin group of type $\Ga{n}$ with constant coefficients over the module $M$ is computed by the following complex $(C_i(\Ga{n}, \md), \partial)$.
\begin{df}
$$
C_i(\Ga{n}, \md) := \bigoplus_{|x| = i} \md.x 
$$
where the boundary is defined by:
$$
\partial 1^l = \sum_{h=0}^l-1 (-1)^h \qbin{l+1}{h+1}_{-1} 1^{h}01^{l-h-1}
$$
and if $A$ and $B$ are two strings
$$
\partial A0B = (\partial A)0B + (-1)^{|A|} A0 \partial B.
$$
\end{df}

Assume that the group $\Gb{n}$ acts on the module $\md = \ring[t^\pmu]$ mapping the first standard generator to multiplication by $(-t)$ and all other generators to multiplication by $1$. Then the homology of $\Gb{n}$ with coefficients on $\md$ is computed by the following complex $(C_*(\Gb{n}, \md),\DB) $.
\begin{df}\label{def:complesso}
%Let $\md$ be a $\Z[t^\pmu]$-module such that the first standard generator acts by multiplication by $(-t)$ and all other generators to multiplication by $1$. 
The complex $C_*(\Gb{n}, \md)$ is given by:
$$
C_i(\Gb{n}, \md) := \bigoplus_{|x| = i} \md.x 
$$
where $\md.x$ is a copy of the module $\md$ generated by an element $x$ and $x \in \{0,1\}^n$ is a list of length $n$ and $|x| :=  |\{i \in 1, \ldots, n \mid x_i = 1\}|$.

When we represent an element $x$ that generates $C_*(\Gb{n}, \md)$ as a string of $0$'s and $1$'s, we put a line over the first element of the string, since it plays a special role, different from that of the complex $C_*(\Ga{n}, \md)$.

The boundary $\DB x$ for $C_*(\Gb{n}, \md)$ is defined by linearity from the following relations:
$$
\DB \overline{0}A = \overline{0}\DA A,
$$
$$
\DB \; \overline{1}1^{l-1} = 
\qbin{l}{0}_{-1}' \overline{0}1^{l-1} + \sum_{h=1}^{l-1} (-1)^{h}\qbin{l}{h}_{-1}' 
\overline{1}1^{h-1}01^{l-h-1}
$$
and
$$
\DB A0B = (\DB A)0B + (-1)^{|A|}A0\DA B.
$$
\end{df}

Hence we have:
\begin{thm}[\cite{salvetti}]
$$
H_*(\Ga{n}; \md) = H_*(C_*(\Ga{n}, \md), \DA)
$$
$$
H_*(\Gb{n}; \md) = H_*(C_*(\Gb{n}, \md), \DB)
$$
\end{thm}


Let $\Br(n) = \Ga{n-1}$ be the classical Artin braid group in $n$ strands. We recall the description of the homology of these groups
according to the results of \cite{cohen, fuks, vain}. 
We shall adopt a notation coherent with \cite{dps} (see also \cite{cal06}) for the description of 
the algebraic complex and the generators. Let $\F$ be a field.
The direct sum of the homology of $\Br(n)$ for $n \in \N = \Z_{\geq 0}$ is considered as a bigraded ring $\oplus_{d,n} H_d(\Br(n), \F)$ 
where the product structure $$H_{d_1}(\Br(n_1), \F) \times H_{d_2}(\Br(n_2), \F) \to H_{d_1+d_2}(\Br(n_1+n_2), \F)$$
is induced by the map $\Br(n_1) \times \Br(n_2) \to \Br(n_1 + n_2)$ that juxtaposes braids (see \cite{cohen_braids, cal06}). 


\subsection{Braid homology over $\Q$}\label{ss:braid_homol}

The homology of the braid group with rational coefficients has a very simple description:
$$
H_d(\Br(n), \Q)= \left( \Q[x_0, x_1]/(x_1^2)\right)_{\deg = n, \dim= d}
$$
where $\deg x_i = i+1 $ and $\dim x_i = i$.
In the Salvetti complex for the classical braid group (see \cite{salvetti, dps} ) the element $x_0$ is represented by the string 0 and $x_1$ is 
represented by the string 10. In the representation of a monomial $x_0^ax_{1}^b$ we drop the last 0.

For example the generator of $H_1(\Br(4),\Q)$ is the monomial $x_0^2x_1$ and we can also write it as a string in the form $001$ (instead of $0010$, dropping the last $0$).

We denote by $A(\Q)$ the module $\Q[x_0, x_1]/(x_1^2)[t^\pmu]$.


\subsection{Braid homology over $\F_2$}

With coefficients in $\F_2$ we have:

$$
H_d(\Br(n), \F_2)= \F_2[x_0, x_1, x_2, x_3, \ldots]_{\deg = n, \dim= d}
$$
where the generator $x_i, i \in \N,$ has degree $\deg x_i = 2^i$ and homological dimension 
$\dim x_i = 2^i-1$.

In the Salvetti complex the element $x_i$ is represented by a string of $2^i -1$ 1's 
followed by one 0. In the representation of a monomial $x_{i_1}\cdots x_{i_k}$ we drop the last 0. 


We denote by $A(\F_2)$ the module $\F_2[x_0, x_1, x_2, x_3, \cdots][t^\pmu]$.

\subsection{Braid homology over $\F_p$, $p>2$}

With coefficients in $\F_p$, with $p$ an odd prime, we have:
$$
H_d(\Br(n), \F_p)= \left( \F_p[h, y_1, y_2, y_3, \ldots] \otimes \Lambda[x_0, x_1, x_2, x_3, \ldots] 
\right)_{\deg = n, \dim= d}
$$
where the second factor in the tensor product is the exterior algebra over the field $\F_p$ 
with generators $x_i, i \in\N$. 
The generator $h$ has degree $\deg h = 1$ and homological dimension $\dim h=0$.
The generator $y_i, i \in \N$ has degree $\deg y_i = 2p^i$ and homological dimension 
$\dim y_i = 2p^i-2$.
The generator $x_i, i \in \N$ has degree $\deg x_i = 2p^i$ and homological dimension 
$\dim x_i = 2p^i-1$.

In the Salvetti complex the element $h$ is represented by the string $0$, the element $x_i$ is 
represented by a string of $2p^i -1$ $1$'s followed by one $0$. %

We remark that the term $\DA(x_i)$ is divisible by $p$. In fact, with generic coefficients (see \cite{cal06}), the differential $\DA(x_i)$ is given by a sum of terms with coefficients all divisible by the cyclotomic polynomial $\varphi_{2p^i}(q)$. Specializing to the trivial local system, with integer coefficients we have that all terms are divisible by $\varphi_{2p^i}(-1) = p$.

The element $y_i$ is represented by the following term (the differential is computed over 
the integers and then, 
after dividing by $p$, we  consider the result modulo $p$):
$$
\frac{\DA(x_i)}{p}.
$$

\subsection{Homology of the Artin group of type $B$}
Now let us recall some results on the homology of the groups $\Gb{n}$ with coefficients on the module $\F[t^\pmu]$, where the first standard generator of  $\Gb{n}$ acts with multiplication by $(-t)$ and all the others generators acts with multiplication by $1$.

In order to describe the elements of $C_*(\Gb{n};\md)$ we write $z_i$ for $\overline{1}1^{i-1}0$. Hence, if $x$ is a generator of $C_*(\Gb{n};\md)$ of the form $x = \overline{1}1^{i-1}0y$, then we write $x = z_iy$.

From \cite[Sec.~4.2]{calmar} we have
\begin{prop} \label{prop:hrazionale}
Let  $\F$ be a field of characteristic $p=0$. The $\F[t]$-module $\oplus_{i,n} H_i(\Gb{n}; \F[t^{\pmu}])$ has a basis given by
$$
\frac{\DB (z_{2i+1}x_0^{j-1})}{1+t},
$$
$$
\frac{\DB (z_{2i+1}x_0^{j-1}x_1)}{1+t},
$$
and
$$
\frac{\DB (z_{2i+2})}{1-t^2},
$$
where the first and the second kind of generators have torsion of order $(1+t)$, while the third kind of generators have torsion of order $(1-t^2)$.
\end{prop}

From the description given in \cite[Thm.~4.5, Thm.~4.12]{calmar} we have the following results.

\begin{prop}\label{prop:homol_p2}
For $p=2$ the  $\F[t]$-module $\oplus_{i,n} H_i(\Gb{n}; \F[t^\pmu])$ has a basis given by
\begin{equation}\label{generators_21}
%\gs(z_c, x_0 x_{i_1} \cdots x_{i_k}):=
\frac{\DB(z_{2^{h+1}(2m+1)+1}x_{i_1} \cdots x_{i_k})}{(1+t)}.
\end{equation}
and
\begin{equation}\label{generators_022}
\frac{\DB(z_{2^{h+1}(2m+1)+2^i}x_{i_1} \cdots x_{i_k})}{(1-t^2)^{2^{i-1}}}.
\end{equation}
where $i \leq i_1 \leq \cdots i_k$, $i \leq h$ and the first kind of generators have
torsion of order $(1+t)$ while the second kind of generators have torsion or order $(1-t^2)^{2^{i-1}}$.\end{prop}

\begin{prop}\label{prop:homol_pdisp}
For $p >2$ or $p=0$, lef $\F$ be a field of characteristic $p$. For $n$ odd the homology $H_i(\Gb{n}; \F[t^\pmu])$ is an $\F[t]$-module with torsion of order $(1+t)$.
\end{prop}
The Proposition \ref{prop:homol_pdisp} above is a consequence of Proposition \ref{prop:hrazionale} and \cite[Thm.~4.12]{calmar}, since 
both results provide a description of the module $\oplus_{i,n} H_i(\Gb{n}; \F[t^\pmu])$ with generators with torsion of order $(1+t)$ or $(1-t^2)^k$ for suitable exponents $k$'s. One can verify that when $n$ is odd the torsion of the generators of degree $n$ is always $(1+t)$.

Next we compute the homology groups  and $H_*(\Gb{n}; \F_2[t]/(1-t^2))$ 
%and the Bockstein homomorphism $\beta_2$ 
using the explicit description of \cite[Sec.~4.3 and 4.4]{calmar}.
As a special case of \cite[Prop.~4.7]{calmar} we have 
%\begin{prop}
the isomorphism
\begin{equation}\label{eq:decomposizione}
H_i(\Gb{n}; \F_2[t]/(1-t^2)) = h_{i}(n,2) \oplus h_{i}'(n,2)
\end{equation}
where the two summands are determined by the following exact sequence:
$$
0 \to h_{i}'(n,2) \to H_i(\Gb{n}; \F_2[t^{\pmu}]) \stackrel{(1+t^2)}{\longrightarrow} H_i(\Gb{n}; \F_2[t^{\pmu}]) \to h_{i}(n,2) \to 0.
$$
%\end{prop}
For odd $n$ all the elements of $H_i(\Gb{n}; \F_2[t^{\pmu}])$ are multiple of $x_0$ and hence have $(1+t)$-torsion (see Proposition \ref{prop:homol_p2}). Hence the multiplication by 
$(1+t^2)$ is the zero map and the generators of  $h_{i}'(n,2)$ and $h_{i}(n,2)$ are  in bijection with a set of generators of $H_i(\Gb{n}; \F_2[t^{\pmu}])$.

In particular, from a direct computation (see \cite[\S 4.4]{calmar}) we have:
\begin{prop} \label{prop:generators}
For odd $n$ the homology $H_*(\Gb{n}; \F_2[t]/(1-t^2))$ is generated, as an $\F_2[t]$-module, by the classes of the form 
\begin{equation}\label{generators1}
\gp (z_c, x_0 x_{i_1} \cdots x_{i_k}) := (1-t) z_{c+1} x_{i_1} \cdots x_{i_k}.
\end{equation}
that correspond to the generators of $h_{i}'(n,2)$
and
\begin{equation}\label{generators2}
\gs(z_c, x_0 x_{i_1} \cdots x_{i_k}):=\frac{\DB(z_{c+1}x_{i_1} \cdots x_{i_k})}{(1+t)}.
\end{equation}
that correspond to generators of $h_{i}(n,2)$. Here $0 \leq i_1 \leq \cdots i_k$, $c$ is even and both kind of generators have
torsion $(1+t)$.
\end{prop}

%%
%% Sposto da qui
%%
%%	
%\todo{sistemare }
Here we do not provide a description of the generators of $H_i(\Gb{n}; \F[t^\pmu])$ for a field $\F$ of characteristic $p>2$ and we also avoid a detailed presentation of a set of generators of  $H_i(\Gb{n}; \F[t]/(1+t))$ and $H_i(\Gb{n}; \F[t]/(1-t^2))$ for a generic field $\F$. For such a description, we refer to \cite{calmar}.

Since it will be useful in Section \ref{s:4tor} and in particular in the proof of Lemma \ref{lem:tau_torsion}, we provide sets of elements $\cB'$, $\cB''$ of the $\Z$-modules $H_i(\confm{n};\Z)\simeq H_i(\Gb{n}; \Z[t]/(1+t))$ and $H_i(\confmd{n};\Z)\simeq H_i(\Gb{n}; \Z[t]/(1-t^2))$ for $n$ odd, such that the following two condition are satisfied:
\begin{enumerate}[label=(\roman*)]
	\item $\cB'$ (resp.~$\cB''$) induces a base of the homology of $H_i(\confm{n};\Q) $ (resp.~$H_i(\confmd{n};\Q)$);
	%\item the elements in $\cB'$ (resp.~$\cB''$) generate the $2$-torsion submodule in $H_i(\confm{n};\Z)$ (resp.~$H_i(\confmd{n};\Z)$) with integer coefficients;
	\item the images of the elements of $\cB'$ (resp.~$\cB''$) in $H_i(\confm{n};\Z_p)$ (resp.~$H_i(\confmd{n};\Z_p)$) are
	linearly independent for any prime $p$.
\end{enumerate}
\begin{df}\label{def:BB}
Let $n$ be an odd integer. We define the sets $\cB' \subset H_i(\confm{n};\Z)$ (for $e=1$) and $\cB'' \subset H_i(\confmd{n};\Z)$ (for $e=2$) given by the following elements:
$$ 
\omega_{2i,j,0}^{(e)}:= \frac{\DB(z_{2i+1}x_0^{j-1})}{(1+t)} \mbox{ and } \widetilde{\omega}_{2i,j,0}^{(e)}:= \frac{(1-(-t)^e)z_{2i+1}x_0^{j-1}}{(1+t)} \qquad \mbox{ for }j>0;
$$
and
$$
\omega_{2i,j,1}^{(e)}:=\frac{\DB(z_{2i+1}x_0^{j-1}x_1)}{(1+t)} \mbox{ and } \widetilde{\omega}_{2i,j,1}^{(e)}:=\frac{(1-(-t)^e)z_{2i+1}x_0^{j-1}x_1}{(1+t)} \qquad \mbox{ for }j>0.
$$
\end{df}
%where for $e=1$ we have elements in $\cB'_1$, while for $e=2$ we have elements in $\cB''_1$.
It follows from \cite[\S4.2]{calmar}
that the elements above provide a basis for $H_*(\confm{n};\Q) $ (resp.~$H_*(\confmd{n};\Q)$) for $n$ odd (condition (i)). 
%\todo{spiegare richiamando anche Prop.~\ref{prop:hrazionale}}

For condition (ii), one can check that the elements in $\cB'$ and $\cB''$ define, mod $2$, a subset of the bases of  $H_i(\Gb{n}; \Z_2[t]/(1+t))$ and $H_i(\Gb{n}; \Z_2[t]/(1-t^2))$ given in \cite[\S4.4]{calmar} and, mod $p$ for an odd prime, a subset of the bases of $H_i(\Gb{n}; \Z_p[t]/(1+t))$ and $H_i(\Gb{n}; \Z_p[t]/(1-t^2))$  given in \cite[\S 4.6]{calmar}.

Hence the following claim follows:
\begin{prop}\label{prop:BB}
For $n$ odd the elements of $\cB'$ (resp.~$\cB''$) are a free set of generator of a maximal free $\Z$-submodule of $H_i(\confm{n};\Z)$ (resp.~$H_i(\confmd{n};\Z)$).
\end{prop}

\begin{prop}\label{prop:hrazionale_minus_t}
Let $n$ be an even integer. Let $\F$ be a field of characteristic $0$.
%The Poincar\'e polynomial of 
%$H_*(\confm{n};\F) = H_*(\Gb{n}; \F[t]/(1+t))$ is $(1+q)(1+q+\ldots+q^{n-1})$
%and a base of the homology is given by the following generators
%$$ \frac{\DB (z_{2i+1}x_0^{j-1}x_1)}{1+t}, z_{2i+1}x_0^{j-1}x_1, \frac{\DB (z_{2i+2})}{1+t},  z_{2i+2}.$$
The Poincar\'e polynomial of 
$H_*(\Gb{n}; \F[t]/(1-t))$ is $(1+q)q^{n-1}
$
and a base of the homology is given by the following generators
$$\frac{\DB (z_{n})}{1-t},  z_{n}.$$
\end{prop}	
\begin{proof}
This follows from Proposition \ref{prop:hrazionale} and by studying the long exact sequence associated to $$
0 \to \F[t^\pmu] \stackrel{(1-t)}{\longrightarrow} \F[t^\pmu] \to \F[t]/(1-t) \to 0
$$ as in \cite[\S~4.2]{calmar}.
\end{proof}
%where for $e=1$ we have elements in $\cB'_2$, while for $e=2$ we have elements in $\cB''_2$.
%%
%% Sposto fino a qui
%%
%%	