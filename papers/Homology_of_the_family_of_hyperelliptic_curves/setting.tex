
Along all this paper, when not specified, the homology is understood to be computed with constant coefficients over a ring $\ring$. 
Let $p$ be a prime or $0$ and let $\F$ be a field of characteristic $p$. 
We write $\md$ for the $\ring$-module of Laurent series $\ring[t, t^{-1}] $. 

%We assume $n$ to be an odd integer.

Let $\Ga{n-1}  = \Br_n$ be the classical braid group on $n$ strands and let $\Gb{n}$ be the 
Artin group of type $\mathrm{B}$.
%$\Gb{n} = \Gbn$.

We write $\confn$ for the configuration space of $n$ unordered points in the unitary disk  $\disk:= \{z \in \C | \, |z|  < 1\}$.
% $\conf_n = \confn$, 
A generic element of $\confn$ is an unordered set of $n$ distinct points $\P = \{x_1, \ldots, x_n\} \subset \disk$. In particular $\conf_1 = \disk$. The fundamental group of $\conf_n$ is the classical braid group on $n$ strands $\Ga{n-1}  = \Br_n$ and we recall that the space $\conf_n$ is a $K(\Br_n,1)$ (see \cite{fa_neu_62}).

Given an element $\P \in \confn$, we can consider the set of points $$\surf_n:= \{(z,y) \in  \disk \times \C | y^2 = (z-x_1)\cdots(z-x_n) \}.$$ This is a connected oriented surface with one boundary component for $n$ odd and with two boundary components if $n$ is even. The genus of $\surf_n$ is $g = \frac{n-1}{2}$ for odd $n$ and $g = \frac{n-2}{2}$ for $n$ even.

Hence we define the space
$$
\totsp_n := \{(\P, z,y ) \in \conf_n  \times \disk \times \C| y^2 = (z-x_1)\cdots(z-x_n) \}.
$$
Notice that $E_n$ has a natural projection $\pi:E_n\to \conf_n$ that maps $(\P, y, z) \mapsto \P$. The fiber of $\pi$ is the surface $\surf_n$.

It is natural to consider the complement of the $n$-points set in the disk: $\disk \setminus \P$. We have that $H_1(\disk \setminus \P)$ has rank $n$. The surface $\surf_n$ is a double covering of $\disk$ ramified along $\P$, hence it is natural to identify $\P$ as a subset of $\surf_n$. 
We define $\ddiskP:= \surf_n \setminus \P$ as the double covering of $\disk \setminus \P$ induced by $\surf_n \to \disk$.
Notice that for $n$ odd $H_1(\ddiskP)$ has rank $2n-1$. 

There is a projection 
$\totsp_n \stackrel{p}{\longrightarrow} \conf_n \times \disk 
$ given by $p:(\P,z,y) \mapsto (\P, z)$.
Hence $\totsp_n$ is a double covering of $\conf_n \times \disk$ ramified along the space $\confm{n-1} := \{ (\P, z) \in \conf_n \times \disk | z \in \P\}$. This is the configuration space of $n-1$ unordered distinct points in $\disk$ with one additional distinct marked point. In particular the complement of
$\confm{n-1} \subset \conf_n \times \disk$ is $\confm{n}$, so the complement of
$p^{-1}(\confm{n-1})$ in $\totsp_n$ is a double covering of $\confm{n}$ that we call $\confmd{n}$.
The fundamental group of $\confm{n}$ is the Artin groups $\Gb{n}$. Moreover (see for example \cite{bri_73}) the space $\confm{n}$ is a $K(\Gbn,1)$.

%Ramified double covering of $\C \setminus \P$: $\surf_n$. Notice that $\surf_n$ is an open surface of genus $g= (n-1)/2$ and one boundary component and $H_1(\surf_n)$ has rank $2g = n-1$.

%Configurations space of $n$ points in $\disk$ with an additional market point: $\confm{n} = \confmn$.

%Double covering of $\confmn: \confmd{n} = \confmdn$.

%Double covering of $\confmn$ with ramification points: $\totsp_n$.

%Fibration: $\surf_n \to \totsp_n \stackrel{\pi}{\to} \conf_n$. 
\begin{rem}\label{rem:globalsection}
Notice that the covering $\surf_n \into \totsp_n \stackrel{\pi}{\to} \conf_n$ admits a global section (see Definition \ref{def:section}) and hence $H_*(\totsp_n) = H_*(\totsp_n,\conf_n) \oplus H_*(\conf_n)$ and $$H_i(\totsp_n, \conf_n) = H_{i-1}(\conf_n; H_1(\surf_n)).$$
\end{rem}


We recall that the $\Z$-module $H_1(\surf_n)$  is endowed with a symplectic form given by the cap product. Moreover the action of $\pi_1(\conf_n)$ on $H_1(\surf_n)$  associated to the covering $\pi$ preserves this form. This monodromy representation is induced by the embedding of the braid group  $\pi_1(\conf_n)$  in the mapping class group of the surface $\surf_n$. Such a monodromy representation maps the standard genyerators of the braid groups to Dehn twists and is called geometric monodromy (see \cite{per_van_92, waj_99}).
Hence we can consider $H_1(\surf_n)$ as a $\pi_1(\conf_n)= \Br_n$-representation; we write also $\sym{g} :=H_1(\surf_n)$, where $g =\frac{n-1}{2}$ for $n$ odd, and $g = \frac{n-2}{2}$ for $n$ even.

The braid group $\Br_n = \Ga{n-1}$ maps on the permutation group $\perm_n$ on $n$ letters. Hence the group $\Br_n $ has a natural representation on $\Z^n$ by permuting cohordinates. We write $\st_n$ for this representation of $\Br_n$.


