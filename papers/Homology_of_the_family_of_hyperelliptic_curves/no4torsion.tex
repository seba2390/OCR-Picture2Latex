In this section we will show that the homology $H_{i}(\Br_n; H_1(\surf_n))$ for $n$ odd actually has only $2$-torsion.

In order to prove this we will consider the following short exact sequence associated to \eqref{eq:bia2}, with coefficients in $\Z_2$ and in $\Z$.
$$
0 \to \coker \iota \to H_{i}(\Br_n; H_1(\surf_n)) \to \ker \iota \to 0
$$

Let us fix the number $n$. From Theorem \ref{thm:4tor} we can assume that $H_{i}(\Br_n; H_1(\surf_n; \Z)) = \Z_2^{a_i} \oplus \Z_4^{b_i}$.
Since the modules $\ker \iota$ and $\coker \iota$ have no $4$-torsion, we can assume $$\coker (\iota_i :H_{i}(\Br_n; H_1(S^1 \times P;\Z)) \to H_{i}(\Br_n; H_1(\ddiskP;\Z) ) = \Z_2^{u_i}$$ and $$\ker (\iota_i :H_{i}(\Br_n; H_1(S^1 \times P;\Z)) \to H_{i}(\Br_n; H_1(\ddiskP;\Z) ) = \Z_2^{v_i}$$
and clearly we have 
$$
u_i + v_{i-1} = a_i + 2 b_i.
$$
Moreover, with coefficients in $\Z_2$, we have
$$
H_{i}(\Br_n; H_1(\surf_n; \Z_2)) = \Z_2^{a_i+a_{i-1} + b_i + b_{i-1}}.
$$
Let $$\overline{u}_i := \rk \coker (\iota_i :H_{i}(\Br_n; H_1(S^1 \times P ;\Z_2)) \to H_{i}(\Br_n; H_1(\ddiskP ;\Z_2) ) $$
and 
$$\overline{v}_i := \rk \ker (\iota_i :H_{i}(\Br_n; H_1(S^1 \times P ;\Z_2)) \to H_{i}(\Br_n; H_1(\ddiskP;\Z_2) ). $$
It follows that $H_*(\Br_n; H_1(\surf_n; \Z))$ has no $4$-torsion if and only if 
$$2 \sum_i (u_i + v_i) = \sum_i (\overline{u}_i  + \overline{v}_i ). $$

Hence we can compute the rank of the modules above.

A basis of the homology $H_{i}(\Br_n; H_1(S^1 \times P;\Z_2))$ is given as follows. 
Following \cite{calmar}, the homology $H_*(\Gb{n}; \F_2[t]/(1+t))$ for $n$ odd is generated, as an $\F_2[t]$-module, by the classes of the form
\begin{equation}\label{generators1Z2}
\gpu (z_c, x_0 x_{i_1} \cdots x_{i_k}) :=z_{c+1} x_{i_1} \cdots x_{i_k}
\end{equation}
 and
\begin{equation}\label{generators2Z2}
\gsu(z_c, x_0 x_{i_1} \cdots x_{i_k}):=\frac{\DB(z_{c+1}x_{i_1} \cdots x_{i_k})}{(1+t)}
\end{equation}
where we assume $0 \leq i_1 \leq \cdots i_k$ and $c$ even.


In particular the image of $\mu_*$ if generated by all elements $\gpu (z_c, x_0 x_{i_1} \cdots x_{i_k})$ and all elements $\gsu(z_c, x_0 x_{i_1} \cdots x_{i_k})$ with $c>0$.


As seen in Section \ref{s:4tor} the homology $H_*(\Gb{n}; \F_2[t]/(1-t^2))$ for $n$ odd is generated, as an $\F_2[t]$-module, by the following classes already introduced in \eqref{generators1} and \eqref{generators2}:
\begin{equation*}
\gp (z_c, x_0 x_{i_1} \cdots x_{i_k}) :=(1-t)z_{c+1} x_{i_1} \cdots x_{i_k}
\end{equation*}
and
\begin{equation*}
\gs(z_c, x_0 x_{i_1} \cdots x_{i_k}):=\frac{\DB(z_{c+1}x_{i_1} \cdots x_{i_k})}{(1+t)}
\end{equation*}
where we assume $0 \leq i_1 \leq \cdots i_k$ and $c$ even.
The kernel of $J$ is generated by the classes 
$\gs(z_c, x_0 x_{i_1} \cdots x_{i_k})$ for $c=0$ as one can see that these classes generate the image of $s_*$.

The map $\tau$ acts as follows:
\begin{eqnarray*}
\tau: \gpu (z_c, x_0 x_{i_1} \cdots x_{i_k}) & \mapsto & \gp (z_c, x_0 x_{i_1} \cdots x_{i_k});\\
\tau: \gsu (z_c, x_0 x_{i_1} \cdots x_{i_k}) &\mapsto & 0.
\end{eqnarray*}

\begin{rem}\label{rem:basi_mod_2}
A basis of $\coker (\iota_i :H_{i}(\Br_n; H_1(S^1 \times P;\Z_2)) \to H_{i}(\Br_n; H_1(\ddiskP;\Z_2) )$ is
given by the elements $\gs (z_c, x_0 x_{i_1} \cdots x_{i_k})$ with $c>0$, of degree $n$ and homological dimension $i+1$, while a basis of $ \ker (\iota_i :H_{i}(\Br_n; H_1(S^1 \times P;\Z_2)) \to H_{i}(\Br_n; H_1(\ddiskP;\Z_2) )$ is given by the elements $\gsu (z_c, x_0 x_{i_1} \cdots x_{i_k})$ with $c>0$, of degree $n$ and homological dimension $i+1$.
In particular, in order to have odd degree $c$ must be even.
Clearly this two basis are in bijection and we have $\overline{u}_i = \overline{v}_i$.
\end{rem}
 
%$\rk \coker (\iota_i :H_{i}(\Br_n; H_1(S^1 \times P);\Z_2) \to H_{i}(\Br_n; H_1(\ddiskP);\Z_2) ) $
%$\rk \ker (\iota_i :H_{i}(\Br_n; H_1(S^1 \times P);\Z_2) \to H_{i}(\Br_n; H_1(\ddiskP);\Z_2) ). $

In order to describe the corresponding map with integer coefficient we recall that in Section \ref{s:4tor} we described basis
 $\cB'$, $\cB''$ generating the homology of $H_i(\confm{n};\Q) $ and $H_i(\confmd{n};\Q)$ and spanning a maximal free $\Z$-submodule of $H_i(\confm{n};\Z) $ and $H_i(\confmd{n};\Z)$. The action of $\tau$ with respect to these basis is given in equations (\ref{rat_gen1}--\ref{rat_gen4}). The elements of $\cB'$ (resp.~$\cB''$) of the form $\omega_{2i,j,\epsilon}^{(1)}$ (resp.~$\omega_{2i,j,\epsilon}^{(2)}$) map, modulo $2$, to elements of the form $\gsu$ (resp.~$\gs$) and in particular the elements the form $\omega_{2i,j,\epsilon}^{(1)}$ (resp.~$\omega_{2i,j,\epsilon}^{(2)}$) with $i=0$ map to elements of the form $\gsu(z_c, \ldots)$ (resp.~$\gs(z_c, \ldots)$) with $c=0$.
The elements of $\cB'$ (resp.~$\cB''$) of the form $\widetilde{\omega}_{2i,j,\epsilon}^{(1)}$ (resp.~$\widetilde{\omega}_{2i,j,\epsilon}^{(2)}$) map, modulo $2$, to elements of the form $\gpu$ (resp.~$\gp$).
Let $$w_i = | \{ \omega_{0,j,\epsilon}^{(1)} \in H_i(\confm{n};\Q)\} | = 
| \{ \omega_{0,j,\epsilon}^{(2)} \in H_i(\confmd{n};\Q)\} |.
$$
From the Universal Coefficients Theorem and from the description of $\tau$ given in equations (\ref{rat_gen1}--\ref{rat_gen4}) we have that
$$
\sum_i u_i = \sum_i  \frac{ \overline{u}_i - w_i}{2} +\sum_i  w_i
$$
and
$$
\sum_i v_i =  \sum_i  \frac{\overline{v}_i - w_i}{2}.
$$
Then it is straightforward to see that 
$$2 \sum_i (u_i + v_i) = \sum_i (\overline{u}_i  + \overline{v}_i ) $$
and hence we have proved the following result.
\begin{thm} \label{th:no4tor}
For odd $n$ the homology $H_{i}(\Br_n; H_1(\surf_n;\Z))$ has only $2$-torsion. \qed
\end{thm}

%$\coker (\iota_i :H_{i}(\Br_n; H_1(S^1 \times P);\Z) \to H_{i}(\Br_n; H_1(\ddiskP);\Z) ) = \Z_2^{u_i}$

%$\ker (\iota_i :H_{i}(\Br_n; H_1(S^1 \times P);\Z) \to H_{i}(\Br_n; H_1(\ddiskP);\Z) ) = \Z_2^{v_i}$

