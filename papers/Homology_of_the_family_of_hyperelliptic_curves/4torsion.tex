In this section we focus on the torsion of order $2^j$ in the integral homology $H_{i}(\Br_n; H_1(\surf_n))$. 
We prove that torsion appears with order at most $4$. 

\begin{lem} \label{lem:no2tor1}
Let $\ring = \Z$. The homology $H_i(\confm{n}) \simeq H_i(\Gb{n}; \ring[t]/(1+t))$ has no $4$-torsion.
\end{lem}
\begin{proof}
%In \cite[Thm.~C]{gorj} Gorjunov proves the splitting
As seen in Remark \ref{rem:collapsss}, we have a splitting
$$
H_q(\confm{n}; \Z) = \bigoplus_{i=0}^\infty H_{q-i} (\conf_{n-i}; \Z).
$$ 
Moreover we have that the integer cohomology of braid group has no $p^2$ torsion for any prime $p$ (\cite[Thm.~3]{vain}). Hence the result follows from the universal coefficients formula.
\end{proof}

\begin{lem} \label{lem:no2tor2}
Let $R = \Z$. For $n$ odd, the homology $H_i(\confmd{n}) \simeq H_i(\Gb{n}; \ring[t]/(1-t^2))$ has no $4$-torsion. 
\end{lem}
\begin{proof} 
It will suffice to show that the dimension over $\F_2$ of the homology of the complex $(H_*(\Gb{n}; \F_2[t]/(1-t^2)), \beta_2)$, where $\beta_2$ is  the Bockstein homomorphism, is the same as the dimension over $\Q$ of $H_*(\Gb{n}; \Q[t]/(1-t^2))$ (see \cite[Thm.~3E.4]{hatcher_at}).


According to \cite[Thm.~6.1, case $r=2$ and $n$ odd]{lehr04}, the Poincar\'e polynomial of the homology groups
$H_*(\Gb{n}; \Q[t]/(1-t^2)) 
%= H_*(B(4,2,n); \Q)
$
is
$$
P(\Gb{n},t)= 
%\left\{
%\begin{array}{cl}
(1+t)(1+t+t^2+ \cdots + t^{n-1}). 
%& 
%\mbox{ if } n \mbox{ is odd,} \\
%(1+t)(1+t+t^2+ \cdots + t^{n-1}) + (t^{n-1} +t^n) & \mbox{ otherwise.}
%\end{array}
%\right. 
$$




The explicit computation of the Bockstein homomorphism $\beta_2$ of the homology group $H_*(\Gb{n}; \F_2[t]/(1-t^2))$ is the following. Let $$0 \to \Z_2[t]/(1-t^2) \stackrel{i_2}{\longrightarrow} \Z_4[t]/(1-t^2) \stackrel{\pi_2}{\longrightarrow} \Z_2[t]/(1-t^2) \to 0$$ be the short exact sequence of coefficients. Then (see Prop.~\ref{prop:generators})
$$(1-t) z_{c+1} x_{i_1} \cdots x_{i_k} = \pi_2 ((1-t) z_{c+1} x_{i_1} \cdots x_{i_k} )$$
and
$$
\DB (1-t) z_{c+1} x_{i_1} \cdots x_{i_k} = \sum_{
	\scsc{j=1, \ldots, k}{i_j>1}
}
2 (1-t) z_{c+1} x_{i_1} \cdots x_{i_j-1}^2 \cdots x_{i_k} =
$$
$$
= i_2 \left(\sum_{
	\scsc{j=1, \ldots, k}{i_j>1}
}
(1-t) z_{c+1} x_{i_1} \cdots x_{i_j-1}^2 \cdots x_{i_k} \right)
$$ and hence, using the notation introduced in Proposition \ref{prop:generators} we have
\begin{equation}\label{bockstein1}
\beta_2 \gp (z_c, x_0x_{i_1} \cdots x_{i_k})   =\sum_{
	\scsc{j=1, \ldots, k}{i_j>1}
} \gp (z_c, x_0 x_{i_1} \cdots x_{i_j-1}^2 \cdots x_{i_k}) 
\end{equation}
for all generators of the form given in \eqref{generators1}.
Moreover 
$$\frac{\DB(z_{c+1}x_{i_1} \cdots x_{i_k})}{(1+t)} = \pi_2 \left( \frac{1}{(1+t)}\left( \DB(z_{c+1}x_{i_1} \cdots x_{i_k}) - 2\sum_{
	\scsc{j=1, \ldots, k}{i_j>1}
}
z_{c+1} x_{i_1} \cdots x_{i_j-1}^2 \cdots x_{i_k} ) \right) \right) $$
and
$$
\DB \left( \frac{1}{(1+t)}\left( \DB(z_{c+1}x_{i_1} \cdots x_{i_k}) - 2\sum_{
		\scsc{j=1, \ldots, k}{i_j>1}
	}
	z_{c+1} x_{i_1} \cdots x_{i_j-1}^2 \cdots x_{i_k} ) \right) \right) =
$$
$$
=\DB \left( \frac{1}{(1+t)}\left(- 2\sum_{
	\scsc{j=1, \ldots, k}{i_j>1}
}
z_{c+1} x_{i_1} \cdots x_{i_j-1}^2 \cdots x_{i_k} ) \right) \right) =
$$
$$
= -2 \sum_{
	\scsc{j=1, \ldots, k}{i_j>1}
} \frac{\DB(z_{c+1} x_{i_1} \cdots x_{i_j-1}^2 \cdots x_{i_k} ) }{1+t}=
i_2 \left( \sum_{
	\scsc{j=1, \ldots, k}{i_j>1}
} \frac{\DB(z_{c+1} x_{i_1} \cdots x_{i_j-1}^2 \cdots x_{i_k} ) }{1+t} \right)
$$
hence we have
\begin{equation}\label{bockstein2}
\beta_2 \gs (z_c, x_0x_{i_1} \cdots x_{i_k}) =\sum_{
	\scsc{j=1, \ldots, k}{i_j>1}
}  \gs (z_c, x_{0} x_{i_1}  \cdots x_{i_j-1}^2 \cdots  x_{i_k})
\end{equation}
for all generators of the form given in \eqref{generators2}.

\begin{df} \label{def:modules}
Let $a,b$ be two non-negative integers, with  $a \in \N_{>0}$, $b \in \{0,1\}.$ 
Let $I = (i_1, \ldots, i_k)$, $J = (j_1, \ldots, j_h)$, where we assume that:
\begin{enumerate}[label=(\roman*)]
	\item $j_1 < \cdots <j_h$,
%	\item[ii)] $\min I \geq 1$,
	\item $\min J \geq 2$,
	\item for all $s \in 1, \ldots, k$ there exists an integer $t \in 1, \ldots, h$ such that $i_s+1 = j_t$ 
\end{enumerate}  
We define the following sub-modules of $H_*(\Gb{n}; \F_2[t]/(1-t^2))$:
%\begin{enumerate}
%\item[i)] 
$$\MM{c,a,b,I,J} := \langle
\gs (z_c, x_0^a x_1^b x_{i_1}^2 \cdots x_{i_k}^2\epsilon(x_{j_1})\cdots \epsilon(x_{j_h})) | \mbox{ where }\epsilon(x_{j_t}) = x_{j_t} \mbox{ or }  x_{j_t-1}^2 \rangle.$$
and
$$\MMp{c,a,b,I,J} := \langle
\gp (z_c, x_0^a x_1^b x_{i_1}^2 \cdots x_{i_k}^2\epsilon(x_{j_1})\cdots \epsilon(x_{j_h})) | \mbox{ where }\epsilon(x_{j_t}) = x_{j_t} \mbox{ or }  x_{j_t-1}^2 \rangle.$$
%\end{enumerate}
\end{df}
\begin{rem} \label{rem:Jvuoto}
Notice that the condition (iii) above implies that if $J= \emptyset$ then also $I = \emptyset$ and hence $\MM{c,a,b,I,J}$ and $\MMp{c,a,b,I,J}$ have rank $1$ concentrated in degree $c+b$ and $c+b+1$ respectively. 
\end{rem}

The module $\MM{c,a,b,I,J}$ and $\MMp{c,a,b,I,J}$ are  free $\Z_2[t]/(1+t)$-modules, closed for $\beta_2$, as follows from formulas \eqref{bockstein1} and \eqref{bockstein2}.

If $J \neq \emptyset$, the complexes $(\MM{c,a,b,I,J}, \beta_2)$ and $(\MMp{c,a,b,I,J}, \beta_2)$ are acyclic. This fact can be proven by the same argument used in \cite[Lem.~4.4]{cal06}.
The argument can be expressed with the following statement:
\begin{lem} \label{lem:boolean}
Let $\bool{J}$ be the chain complex with $\F_2$ coefficients associated to the boolean lattice of the subsets of $J$. The complexes $(\MM{c,a,b,I,J}, \beta_2)$ and $(\MMp{c,a,b,I,J}, \beta_2)$ are isomorphic to $\bool{J}$. In particular if $J \neq \emptyset$ the complexes $(\MM{c,a,b,I,J}, \beta_2)$ and $(\MMp{c,a,b,I,J}, \beta_2)$ are acyclic.
\end{lem}
\begin{proof}[Proof of the Lemma \ref{lem:boolean}.]
Let first construct an isomorphism $\theta$ between the boolean complex $\bool{J}$ and the complex $\MM{c,a,b,I,J}$. We can map the generator $e_K$ of $\bool{J}$ associated to a subset $K$ of $J$ to the element $$ \theta(e_k):= \gs (z_c, x_0^a x_1^b x_{i_1}^2 \cdots x_{i_k}^2\epsilon(x_{j_1})\cdots \epsilon(x_{j_h})),$$ where $\epsilon(x_{j_t}) = x_{j_t}$ if $j_t \notin K$ and $\epsilon(x_{j_t}) =  x_{j_t-1}^2 $ if $j_t \in K$. It is easy to check that $\theta \circ d = \beta_2 \circ \theta$. Similarly we can construct an isomorphism $\theta': \bool{J} \to \MMp{c,a,b,I,J}$ with
$$
\theta'(e_J):= \gp (z_c, x_0^a x_1^b x_{i_1}^2 \cdots x_{i_k}^2\epsilon(x_{j_1})\cdots \epsilon(x_{j_h}))
$$
and check that $\theta' \circ d = \beta_2 \circ \theta'$.
\end{proof}


We recall from Proposition \ref{prop:generators} that the elements of the form $\gp (z_c, x_0 x_{i_1} \cdots x_{i_k})$ and $\gs (z_c, x_0 x_{i_1} \cdots x_{i_k})$ are a free set of generators of $H_*(\Gb{n}; \F_2[t]/(1-t^2))$. 



We claim that for any pair of distinct modules $\MM{c,a,b,I,J}$ or $\MMp{c,a,b,I,J}$ we have disjoint set of generators.
%\todo{dimostrare} 
The case of modules of the form $\MM{c,a,b,I,J}$ can be proved as follows (the case of modules of the form $\MMp{c,a,b,I,J}$ is analogous).  
Let
$$
\gs_0 = \gs (z_c, x_0^a x_1^b x_{i_1}^2 \cdots x_{i_k}^2\epsilon(x_{j_1})\cdots \epsilon(x_{j_h}))
$$
with $\epsilon(x_{j_t}) = x_{j_t} \mbox{ or }  x_{j_t-1}^2$ be a generator of $\MM{c,a,b,I,J}$. We can choose an element $\gs_1$ of the form 
$$
\gs_1 = \gs (z_c, x_0^a x_1^b x_{i_1}^2 \cdots x_{i_k}^2\epsilon'(x_{j_1})\cdots \epsilon'(x_{j_h}))
$$
such that $\gs_0$ appears as a summand in $\beta_2(\gs_1).$ The constrains on the multi-indexes $I$ and $J$ and formula \eqref{bockstein2} imply that such an element $\gs_1$ exists if and only for at least one index $l$ we have that $\epsilon(x_{j_l}) = x_{j_l-1}^2$. If such an element $\gs_1$ exists we have that $\gs_1 \in \MM{c,a,b,I,J}$ and we say that $\gs_0$ \emph{lifts} to $\gs_1$. Hence in a finite number of steps we have that $\gs_0$ lifts to $$\gs (z_c, x_0^a x_1^b x_{i_1}^2 \cdots x_{i_k}^2x_{j_1}\cdots x_{j_h}).
$$  This implies that an element $\gs_0$ does not belong at the same time to two different modules
$\MM{c,a,b,I,J}$ and $\MM{c',a',b',I',J'}.$ 

Therefore the modules $\MM{c,a,b,I,J}$ for all admissible $c,a,b,I,J$ are in direct sum and the modules $\MMp{c,a,b,I,J}$ for all admissible $c,a,b,I,J$ are in direct sum.
Moreover every generator 
$\gp (z_c, x_0 x_{l_1} \cdots x_{l_k})$ or $\gs (z_c, x_0 x_{l_1} \cdots x_{l_k})$ appears in a suitable 
complex  $\MM{c,a,b,I,J}$ or $\MMp{c,a,b,I,J}$. 
%\todo{dimostrare}
Let us prove this in the case of a generator of the form $\gp (z_c, x_0 x_{l_1} \cdots x_{l_k})$, the other case being analogous. We can write the monomial $x_0x_{l_1} \cdots x_{l_k}$ as
$$
x_{q_1}^{p_{q_1}} \cdots x_{q_r}^{p_{q_r}} 
$$
with $q_1 < q_2 < \cdots < q_r$. %, where we assume $p_{q_s} > 0$ for all $s$ and we set $p_s = 0 $ if $s \neq q_t$ for all $t$.
Then we define the strictly ordered multi-index $J$ as follows: $j \in J$ if and only if $j>1$ and one of the following conditions is satisfied:
\begin{enumerate}
	\item  $p_j$ is odd;
	\item $p_j=0$ and $p_{j-1}$ is even and non-zero.
\end{enumerate}
Moreover if $p_j$ is odd we set $\epsilon(x_j) = x_j$, otherwise we set $\epsilon(x_j) = x_{j-1}^2$.
Next we define the multi-index $I$ suitably in the unique way such that 
$$
x_{q_1}^{p_{q_1}} \cdots x_{q_r}^{p_{q_r}} = x_0^{p_0} x_1^{p_1} x_{i_1}^2 \cdots x_{i_k}^2\epsilon(x_{j_1})\cdots \epsilon(x_{j_h}).
$$
It is straightforward to check that 
$$
\gs (z_c, x_0^{p_0} x_1^{p_1} x_{i_1}^2 \cdots x_{i_k}^2\epsilon(x_{j_1})\cdots \epsilon(x_{j_h})) = \gs (z_c,  x_{q_1}^{p_{q_1}} \cdots x_{q_r}^{p_{q_r}} ) = 
$$
$$
=\gs (z_c, x_0 x_{l_1} \cdots x_{l_k})
$$
and that the multi-indexes $I$ and $J$ satisfies the condition of Definition \ref{def:modules}.

Hence 
%let $\MMM$ be 
$h_{i}(n,2)$ is the direct sum of all admissible modules $\MM{c,a,b,I,J}$ and 
%let $\MMMp$ be 
$h_{i}'(n,2)$ is
the direct sum of all admissible modules $\MMp{c,a,b,I,J}$ and we recall (equation \eqref{eq:decomposizione}) that
$$
H_*(\Gb{n}; \F_2[t]/(1-t^2)) = h_{i}(n,2) \oplus h_{i}'(n,2).
$$
This direct sum decomposition  implies that the homology $H_{\beta_2}$ of the complex $(H_*(\Gb{n};$ $\F_2[t]/(1-t^2)), \beta_2)$ is given as follows:
$$
H_{\beta_2} = \bigoplus \MM{c,a,b,\emptyset, \emptyset}
\oplus
\bigoplus \MMp{c,a,b,\emptyset, \emptyset}
$$
for $c$ even, $a \in \N_{>0}$, $b \in \{0,1\}$. In fact if $J = \emptyset$ then also $I = \emptyset$ and for all non-empty $J$ we have from Lemma \ref{lem:boolean} that the complexes $(\MM{c,a,b,I,J},\beta_2)$ and $(\MMp{c,a,b,I,J},\beta_2)$ are acyclic.

Using Remark \ref{rem:Jvuoto}  it is easy to check that the complex $H_{\beta_2}$ has Poincar\'e polynomial $(1+t)(1+t+t^2+ \cdots + t^{n-1})$, hence the lemma follows.
\end{proof}


\begin{lem} \label{lem:tau_torsion}
Let $\ring = \Z$. For $n$ odd the cokernel of the homomorphism
$$\tau:H_i(\confm{n}) \to H_i(\confmd{n})
$$
has no $4$-torsion.
\end{lem}
\begin{proof}
We recall that in Definition \ref{def:BB} we introduced the sets $\cB' \subset H_i(\confm{n};\Z)$  and $\cB'' \subset H_i(\confmd{n};\Z)$. As stated in Proposition \ref{prop:BB}, these are free sets of generators of a maximal free $\Z$-submodule of $H_i(\confm{n};\Z)$  and $H_i(\confmd{n};\Z)$ respectively.

To describe the cokernel we see that the map 
$$\tau:H_i(\confm{n}; \Z) \to H_i(\confmd{n}; \Z)
$$ 
acts on the elements of $\cB'$ as follows:
\begin{eqnarray}
\tau: \omega_{2i,j,0}^{(1)} & \mapsto  (1 - t ) &\omega_{2i,j,0}^{(2)} = 2 \omega_{2i,j,0}^{(2)}, \label{rat_gen1}\\
\tau: \widetilde{\omega}_{2i,j,0}^{(1)} & \mapsto \phantom{(1-t)}& \widetilde{\omega}_{2i,j,0}^{(2)},\label{rat_gen2}\\
\tau: \omega_{2i,j,1}^{(1)}  & \mapsto (1 - t )& \omega_{2i,j,1}^{(2)} = 2 \omega_{2i,j,1}^{(2)},\label{rat_gen3}\\
\tau: \widetilde{\omega}_{2i,j,1}^{(1)} & \mapsto \phantom{(1-t)}& \widetilde{\omega}_{2i,j,1}^{(2)}.\label{rat_gen4}
\end{eqnarray}
Hence the homomorphism $\tau$ acts diagonally and maps each element of $\cB'$ to $1$ or $2$ times the corresponding element of $\cB''$.

Since the $\Z$-modules $H_i(\confm{n};\Z)$ and $H_i(\confmd{n};\Z)$ have no $4$-torsion and  $\tau$ is an isomorphism mod $p$ for any odd prime, it follows that, with integer coefficients, the cokernel of $\tau$ has no $4$-torsion.
\end{proof}


%\begin{rem}
%Let $\ring$ be a field of characteristic $p$. For $p=2$ and $n$ odd the spectral sequences $E^2_{ij}(\confmd{n})$ and $E^2_{ij}(\confmd{n}, \conf_n)$ collapses at the second page. This follows since for $p=2$ the action of the group $\Gb{n}$ on the module $\ring[t]/(1-t^2)$ is trivial. \todo{questo sembra FALSO...}
%\end{rem}


%\begin{lem}
%Let $\ring$ be a field of characteristic $2$ and let us consider homology with coefficient in $R$. The map $\overline{J}_{ij}^2:E^2_{ij}(\confmd{n}) \to E^2_{ij}(\confmd{n}, \conf_n)$ on spectral sequences is surjective. Moreover $\overline{J}_{ij}$ is an isomorphism for $i \neq 0$
%\end{lem}
%\begin{proof}
%The double covering $\pi: \confmd{n} \to \conf_n$ has a section $s$. Hence the homology map $s_*$ is injective and the short exact sequence
%$$
%0 \to H_i(\conf_n) \stackrel{s_*}{\longrightarrow} H_i(\confmd{n}) \stackrel{J}{\to} H_i(\confmd{n}, \conf_n) \to 0
%$$
%is split. Hence $H_i(\confmd{n}) \simeq H_i(\conf_n) \oplus H_i(\confmd{n}, \conf_n)$.
%Moreover the analogous section exists for $\pi: \confm{n} \to \conf_n$ and the same argument implies that we have the splitting $H_i(\confm{n}) \simeq H_i(\conf_n) \oplus H_i(\confm{n}, \conf_n)$.	
%	
%Let $\sigma: \confmd{n} \to \confmd{n}$ the only nontrivial automorphism of the double covering $\confmd{n} \to \confm{n}$.
%We have two sections $s,s': \conf_n \to \confmd{n}$ such that $\sigma s = s'$, $\sigma s' = s$.
%Hence we can include $S:\conf_n \times \{1, -1\} \into \confmd{n}$ and the restriction of $\sigma$ exchanges the components $\conf_n \times \{1\}$ and $\conf_n \times \{-1\}$. Hence we can understand the inclusion $s:\conf_n \into \confmd{n}$ in homology via the following diagram:
%$$
%H_i(\conf_n) \stackrel{\Id \otimes 1}{\longrightarrow} H_i(\conf_n) \otimes \ring[t]/(1-t^2) \stackrel{\Sigma}{\to} H_i(\confmd{n}) 
%$$
%and the composition is injective.
%
%We consider the spectral sequences 
%$E^2_{ij}(\confmd{n}) \Rightarrow  H_{i+j}(\confmd{n}, \conf_n)$ and $ E^2_{ij}(\confmd{n}, \conf_n) \Rightarrow H_{i+j}(\confmd{n}, \conf_n)$ already introduced.
%
%The map $\Sigma$ induce the isomorphism between  $H_i(\conf_n) \otimes \ring[t]/(1-t^2)$ and the first column of $E^2_{ij}(\confmd{n})$. Hence the inclusion $s:\conf_n \into \confmd{n}$ induces an injective homomorphism from $H_i(\conf_n) $ to the first column of $E_{ij}^2(\confmd{n})$.
%
%Hence
%%Since the map $J$ is surjective, 
%we have that $\overline{J}_{ij}^2$ is an isomorphism for $i \neq 0$. 
%\end{proof}

Let us consider homology with coefficient in the ring $\ring = \Z$. As stated in Corollary \ref{cor:image_mu}, the image of  $\mu_*$ is the submodule $H_i(\confm{n}, \conf_n)$ in $H_i(\confm{n})$ and we consider the composition $\iota = J \circ \tau \circ \mu_*: H_{i-1}(\confm{n-1}) \otimes H_1(S^1) \to H_i(\confmd{n}, \conf_n)$.



\begin{lem} \label{lem:comp_no_4}
For odd $n$ the cokernel of the composition $J \circ \tau \circ \mu_*$ has no $4$-torsion.
\end{lem}
\begin{proof}
First we can consider the homomorphism $\overline{s}_*:H_*(\conf_n) \to H_*(\confm{n})$ induced by the inclusion $\overline{s}:\conf_n \into \confm{n}$, the decomposition $H_*(\confmd{n}) = H_*(\conf_n) \oplus H_*(\confmd{n}, \conf_n)$, with $\pi_1$ and $\pi_2= J$ respectively the projections on the first and the second summand,  and hence the map  $\tau_{11}:H_*(\conf_n) \to H_*(\conf_n)$ defined by the composition
$$
H_*(\conf_n) \stackrel{\overline{s}_*}{\longrightarrow} H_*(\confm{n}) \stackrel{\tau}{\longrightarrow} H_*(\confmd{n}) \stackrel{\pi_1}{\longrightarrow} H_*(\conf_n).
$$
We can consider the following diagram:
$$
\xymatrix{ H_*(\conf_n) \ar[d]^{\overline{s}_*} \ar[drr]^(0.65){s_*+s'_*} \ar[rr]_(0.44){\tau_{|\overline{s}_*H_*(\conf_n)}} \ar@/^1pc/[rrr]^{\tau_{11}}
	&& H_*(s(\conf_{n})  \sqcup s'(\conf_{n}))\ar[d]^{i_*} \ar[r] & H_*(\conf_n) \\ H_*(\confm{n}) \ar[rr]^\tau && H_*(\confmd{n}) \ar[ur]^{\pi_1}  }
$$
From Theorem \ref{teo:tau_restricted} we have that that $\tau_{11} = 2 \Id_{H_*(\conf_n)}$ and
%Moreover we claim that 
$\pi_2 \tau\overline{s}_*(H_*(\conf_n))=0$. %This follows since $s_*$ is the left inverse of $\pi_1$ and hence for $x \in H_*(\conf_n)$ we have that $\pi_2(\tau(\overline{s}_*(x))) = \pi_2(s_*(x) + s'_*(x) ) = 2 \pi_2(s_*(x))=0$.
Now, let $x_2 \in H_*(\confmd{n}, \conf_n)$ and let $\tau_{22}: H_*(\confm{n}, \conf_n) \to H_*(\confmd{n}, \conf_n)$ be the map induced by $\tau$ by restricting to $H_*(\confm{n}, \conf_n)$ and projecting to $H_*(\confmd{n}, \conf_n)$. If there exists $y \in H_*(\confm{n})$ such that $\pi_2 \tau (y) = 4 x_2$, then let $x_1:= \pi_1(\tau(y))$. We can consider $-x_1 + 2 y \in H_*(\confm{n})$ and we have that $\tau(-x_1 + 2 y) = -2 x_1 + 2 (x_1 +4x_2) = 8 x_2$. Since the cokernel of $\tau$ has only $2$-torsion it follows that $2x_2 = 0$ in $\coker \tau$ and finally, since $\pi_2 \tau \overline{s}_*(H_*(\conf_n))= 0$, $2x_2 = 0$ in $\coker \tau_{22}$.
\end{proof}


%hence the image is included in the kernel of the projection to $H_i(\conf_n)$. The homomorphism $\tau$ maps  $H_i(\confm{n}, \conf_n)$ to the submodule $H_i(\confmd{n}, \conf_n)$ of  $H_i(\confmd{n})$, because this is the kernel of the projection to $H_i(\conf_n)$.
%Hence, since from Lemma \ref{lem:tau_torsion}  the cokernel of the homomorphism $\tau$ has no $4$-torsion, the same holds for the cokernel of the composition $J \circ \tau \circ \mu_*: H_{i-1}(\confm{n-1}) \otimes H_1(S^1) \to H_i(\confmd{n}, \conf_n)$.

From Lemma \ref{lem:no2tor1} and \ref{lem:comp_no_4} we have that, with integer coefficients, the kernel and the cokernel of the map $$\iota:H_{i-1}(\Br_n; H_1(S^1 \times P)) \to H_{i-1}(\Br_n; H_1(\ddiskP))$$ in diagram \eqref{eq:bia2} have no $4$-torsion. 
Hence we have (see also \cite{bianchi}):
\begin{thm}\label{thm:4tor}
For $n$ odd the homology $H_{i}(\Br_n; H_1(\surf_n))$ computed with coefficients in the ring $\ring = \Z$ has torsion of order at most $4$. \qed
\end{thm}
%The map $\tau$, restricted to the image of $\mu_*$, maps suitable generators of the $\Z$-module $H_i(\confm{n})$ to $1$ or $2$ times suitable generators of the $\Z$-module $H_i(\confmd{n})$.