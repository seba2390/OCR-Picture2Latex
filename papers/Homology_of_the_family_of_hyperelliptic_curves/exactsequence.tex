% \emph{[inizia Filippo coi teoremi principali]}

% \emph{-ricostruire la decomposizione data da Bianchi e arrivare alla successione di Mayer-Vietoris} \vskip 10pt




We recall that the double covering $\pi: \confmd{n} \to \conf_n$ has a continuous section $s$  (see also \cite[p. 30]{bianchi}) that we can define as follows.
\begin{df} \label{def:section}
Given a monic polynomial $p$ with $n$ distinct roots $x_1, \ldots, x_n$ such that $|x_i| <1$ for all $i$ we can map
$$s: p \mapsto \left(p, z:= \frac{(\max_i |x_i|) +1}{2}, \sqrt{p(z)}) \right)$$
where we choose $\sqrt{p(z)}$ as a continuous function as follows: since $p(z) = \prod_i(z-x_i)$ is a product of complex numbers with $\Re (z-x_i) >0$, we can choose $\sqrt{z-x_i}$ to be the unique square root with $\Re \sqrt{z-x_i} >0$ and hence we can define $\sqrt{p(z)} := \prod_i \sqrt{z-x_i}$.
\end{df}
Clearly the section $s:\conf_n \to \confmd{n}$ is a continuous lifting of the section  $\overline{s}:\conf_n \to \confm{n}$ that maps $p \mapsto (p, z)$.
The homology map $s_*$ is injective and the short exact sequence
$$
0 \to H_i(\conf_n) \stackrel{s_*}{\longrightarrow} H_i(\confmd{n}) \stackrel{J}{\to} H_i(\confmd{n}, \conf_n) \to 0
$$
is split. Hence $H_i(\confmd{n}) \simeq H_i(\conf_n) \oplus H_i(\confmd{n}, \conf_n)$.
Moreover the same argument applied to the section $\overline{s}$ for $\pi: \confm{n} \to \conf_n$ implies that we have the splitting $H_i(\confm{n}) \simeq H_i(\conf_n) \oplus H_i(\confm{n}, \conf_n)$.	

\begin{rem}\label{rem:Cn_sections}
Let $\sigma: \confmd{n} \to \confmd{n}$ the only nontrivial automorphism of the double covering $\confmd{n} \to \confm{n}$.
We can obviously define another section  $s': \conf_n \to \confmd{n}$ such that $\sigma s = s'$, $\sigma s' = s$.
Hence we can include $S:\conf_n \times \{1, -1\} \into \confmd{n}$ and the restriction of $\sigma$ exchanges the components $\conf_n \times \{1\}$ and $\conf_n \times \{-1\}$. Hence we can understand the inclusion $s:\conf_n \into \confmd{n}$ in homology via the following diagram:
$$
H_i(\conf_n) \stackrel{\Id \otimes 1}{\longrightarrow} H_i(\conf_n) \otimes \ring[t]/(1-t^2) \stackrel{S_*}{\to} H_i(\confmd{n}) 
$$
and the composition is injective.
\end{rem}

\begin{rem} \label{rem:s_sprime}
If $n$ is odd the two sections $s,s'$ are homotopic. In fact we can define in a unique way a continuous family of maps
$
s_t:\conf_n \to \confmd{n}
$
such that $s_0 = s$ and
$$
s_t: \mapsto \left(p, z_t:= e^{2\pi i t}\frac{(\max_i |x_i|) +1}{2}, \sqrt{p(z_t)} \right)
$$
Since $p(z_t)$ is a product of $n$ factors it is clear that  for $n$ odd we have $s_1 = s'$.
As a consequence $s_* = s'_*$.\end{rem}

According to Bianchi (\cite{bianchi}) we consider the following decomposition.
The space $\totsp_n$ is the union of the double covering $\confmd{n}$ and a subset diffeomorphic to $\confm{n-1}$. 
Let $N$ be a small tubolar neighborhood of $\confm{n-1}$ in $\totsp_n$ and let $M$ be the closure of its complement in $\totsp_n$. The complement $M$ is homotopy equivalent to $\confmd{n}$, while $\partial N = M \cap N$ is diffeomorphic to $\confm{n-1} \times S^1$. Moreover $M$ contains a subspace $\conf_n = s(\conf_n) \subset \totsp_n $. 




Hence there is a relative Mayer-Vietoris long exact sequence 
$$
\cdots \to H_i(\partial N) \to H_i (N) \oplus H_i(M, \conf_n) \to H_i(\totsp_n, \conf_n) \to \cdots
$$
that is equivalent to the following long exact sequence:
$$
\cdots \to H_i(\confm{n-1}) \oplus H_{i-1}(\confm{n-1}) \otimes H_1(S^1)\stackrel{\iota}{\to} H_i (\confm{n-1}) \oplus H_i(\confmd{n}, \conf_n) \to H_i(\totsp_n, \conf_n) \to \cdots
$$
Notice that from Kunneth decomposition the restriction of the map $\iota$ induces an isomorphism between the terms (again, see also \cite[Lem.~58]{bianchi}):
$$
\iota: H_i(\confm{n-1}) \to H_i(\confm{n-1})
$$
and the restriction of $\iota$ to the second summand $H_{i-1}(\confm{n-1}) \otimes H_1(S^1)$ maps to zero if we project to the term $H_i(\confm{n-1})$.
Hence we can simplify our exact sequence as follows:
\begin{equation}\label{eq:bia1}
\cdots \to  H_{i-1}(\confm{n-1}) \otimes H_1(S^1)\stackrel{\iota}{\to} H_i(\confmd{n}, \conf_n) \to H_i(\totsp_n, \conf_n) \to \cdots
\end{equation}


% \emph{-descrivere le identificazioni, anche in termini di inclusioni e quozienti di successioni spettrali} \vskip 10pt

We already know that $\confm{n}$ is a classifying space for the Artin group of type $B_n$, and hence we have:
$$
H_i(\confm{n-1}) = H_i(\Gb{n-1}).
$$
Moreover the forgetful map $\confm{n-1} \to \conf_n$ is a covering with generic fibre given by the discrete set $P \in \conf_n$ and hence induces the isomorphism (see \cite[Lem.~8]{bianchi}):
$$
H_i(\Br_{n}; \st_n) = H_i(\conf_{n}; \st_n) \simeq H_i(\confm{n-1}).
$$
Recall that $\partial N = S^1 \times \confm{n-1}$. Hence with respect to the fibration $\partial N \to \conf_n$, with fiber $S^1 \times P$, the monodromy action of $\Br_n = \pi_1(\conf_n)$ on $H_1(S^1 \times P)$ is exactly the permutation action. It follows that we have isomorphic $\Br_n$-representations:
 $H_1(S^1 \times P) \simeq \st_n$.

As we already noticed in Remark \ref{rem:globalsection}:
$$
H_i (\totsp_n, \conf_n) \simeq H_{i-1}(\Br_n; H_1(\surf_n))
$$

Finally the term $ H_i(\confmd{n}, \conf_n)$ is isomorphic to $H_{i-1}(\Br_n; H_1(\ddiskP))$.

Then we can rewrite \eqref{eq:bia1} as follows:
\begin{equation}\label{eq:bia2}
\cdots \to H_{i-1}(\Br_n; H_1(S^1 \times P)) \stackrel{\iota}{\to} H_{i-1}(\Br_n; H_1(\ddiskP)) \to  H_{i-1}(\Br_n; H_1(\surf_n)) \to \cdots
\end{equation}

Actually we can see that \eqref{eq:bia2} is the long exact sequence for the homology of the group $\Br_n$ associated to the short exact sequence of coefficients
$$
0 \to H_1(S^1 \times P) \to H_1(\disk \times P) \oplus H_1(\ddiskP) \to H_1(\surf_n) \to 0
$$
that follows from the Mayer-Vietoris long exact sequence associated to the decomposition: $\surf_n = \disk \times P \cup \ddiskP$, with, up to homotopy, $\disk \times P \cap \ddiskP \simeq S^1 \times P$.

\begin{df} \label{df:mu}
We write $p= (p_1, \{p_2, \ldots, p_n\})$ for a point in $\confm{n-1}$ and   $\overline{p} := \{p_1, \ldots, p_n\}$. Let $$\delta(\overline{p}):= \frac{1}{2}\min \left( \{ |p_1 - p_i|,  2 \leq i  \leq n\} \cup \{ 1-|p_1|\}
\right).$$
We define the map $\mu$ is given  by 
$$
\confm{n-1} \times S^1 \ni (p,e^{it}) \mapsto (p_1 + \delta(\overline{p})e^{it}, \overline{p}) \in \confm{n}
$$
\end{df}


In order to understand $H_{i-1}(\Br_n; H_1(S^1 \times P)) \stackrel{\iota}{\to} H_{i-1}(\Br_n; H_1(\ddiskP))$ we can consider the commuting diagram
\begin{equation}\label{diag:trerighe0}
\begin{tabular}{c}
\xymatrix @R=2pc @C=2pc {
H_{i-1}(\Br_n; H_1(S^1 \times P)) \ar[r]^\iota \ar[d]^\simeq & H_{i-1}(\Br_n; H_1(\ddiskP))\ar[d]^\simeq  \\
H_{i-1}(\confm{n-1}) \otimes H_1(S^1) \ar[d]^{\mu_*} \ar[r] & H_i(\confmd{n}, \conf_n) \\
H_i(\confm{n}) 
%\ar[r]^{\tau} 
&H_i(\confmd{n})  \ar[u]_J
}
\end{tabular}
\end{equation}
where $\mu_*$ is induced by the map $\mu: \confm{n-1} \times S^1 \to \confm{n}$
and $J$ is induced by the inclusion $\confmd{n} \to (\confmd{n}, \conf_n)$.

%\emph{Leggendo Bianchi si capisce che $\mu$ \`e definita nel teorema 12 e poi nella sezione 3. Basterebbe quindi mostrare che la decomposizione che si trova dal teorema 12 corrisponde a quella della ss data dalla filtrazione del complesso.}


In \cite[Thm.~12]{bianchi} Bianchi shows that $H_*(\confm{n}) = H_*(\conf_n) \oplus H_{*-1}(\confm{n-1})$, where the first summand is the image of the map induced by the natural inclusion $\overline{s}: \conf_n \to \confm{n}$
and the projection to the first summand corresponds to the map induced by the forgetful map $r:\confm{n} \to \conf_n$. This argument is also implicit in \cite[Ch. 1,~\S 5]{vas} since Vassiliev shows that $H_i(\conf_n; H_1(\disk \setminus \P)) \simeq H_i(\confm{n-1})$ and from the spectral sequence associated to the projection $r:\confm{n} \to \conf_n$ (with section $\bar s$) we get $H_i(\confm{n}, \conf_n) =  H_{i-1}(\conf_n; H_1(\disk \setminus \P))
$. 

In order to provide an explicit description of the homology homomorphism induced by $\mu$, we give another proof of this splitting.
We have the short exact sequence:
$$
0 \to H_i(\conf_n) \stackrel{\overline{s}_*}{\longrightarrow} H_i(\confm{n}) \to H_i(\confm{n}, \conf_n) \to 0
$$
and since $\overline{s}$ is a section of $r$ we have that the exact sequence splits and 
$
H_i(\confm{n}, \conf_n) = \ker [r_*: H_i(\confm{n}) \to H_i(\conf_n)].
$

Let now define the decomposition 
$\conf_n \times \disk = \confm{n} \cup \confm{n-1} \times \disk$, where we naturally identify $\confm{n}$ with a subset of $\conf_n \times \disk$ mapping
$$
p= (p_1, \{p_2, \ldots, p_{n+1}\}) \mapsto (\{p_2, \ldots, p_{n+1}\}, p_1)
$$
and where we identify $\confm{n-1}\times \disk$ with a subset of $\conf_n \times \disk$ mapping
$$
(p',q)= ((p'_1, \{p'_2, \ldots, p'_{n}\}),q) \mapsto (\{p'_1, \ldots, p'_n\}, p_1+\delta(\overline{p}')q).
$$
Clearly $\confm{n} \cap \confm{n-1} \times \disk \simeq \confm{n-1} \times S^1$.
Hence we get the associated Mayer-Vietoris exact sequence:
$$
\cdots \to H_i(\confm{n-1} \times S^1) \to H_i(\confm{n}) \oplus H_i(\confm{n-1}) \to H_i(\conf_n) \to \cdots.
$$
Since $H_i(\confm{n})$ decomposes as $H_i(\confm{n}) = H_i(\confm{n},\conf_n) \oplus H_i(\conf_n)$ and the map $H_i(\confm{n}) \to H_i(\conf_n)$ is surjective, with kernel given by $H_i(\confm{n}, \conf_n)$, we have the isomorphism
$$
H_i(\confm{n-1} \times S^1) \simeq  H_i(\confm{n},\conf_n) \oplus H_i(\confm{n-1}).
$$
Finally notice that $H_i(\confm{n-1} \times S^1) = H_i(\confm{n-1}) \oplus  H_{i-1}(\confm{n-1}) \otimes H_1(S^1)$ and $H_{i-1}(\confm{n-1}) \otimes H_1(S^1)$ has trivial projection onto $H_i(\confm{n-1})$ since it factors through 
$$
H_{i-1}(\confm{n-1}) \otimes H_1(S^1) \to H_{i-1}(\confm{n-1}) \otimes H_1(\disk) \to H_i(\confm{n-1}).
$$
Recalling the definition of $\mu$ we obtain:
\begin{prop} \label{prop:isodellesemplicifazioni}The following groups are isomorphic
$$
H_{i-1}(\confm{n-1}) \simeq H_i(\confm{n},\conf_n)
$$
and the isomorphism is induced by the map $\mu$.
\end{prop}

The increasing filtration $$\Filt^i:= \langle  A \mid A \mbox{ is a string that contains at least one } 0 \mbox{ among the first } i \mbox{ entries.}\rangle$$ of the complex $C_*(\Gb{n}, \md)$ introduced in Definition \ref{def:complesso} 
induces a spectral sequence 
\begin{equation}\label{eq:ss_bn}
E_{ij}^2 = H_j(\Br(n-i); \md) \Rightarrow H_{i+j} (\Gb{n}; \md).
\end{equation}
\begin{rem}\label{rem:collapsss}
If the action of $\Gb{n}$ on the module $\md$ is trivial the spectral sequence collapses at $E^2$. This fact can be proved with a quite technical argument: in fact one can see that all the non-zero differentials of the spectral sequence are divided by a coefficient $(1+t)$, where $-t$ correspond to the action of the first standard generator of the group $\Gb{n}$ on the module $\md$. In particular, if the action is trivial, all the differentials of the spectral sequence are trivial (see \cite{calmar} for a detailed analysis of this spectral sequence). Another more elementary argument is the following: the splitting $H_*(\confm{n}) \simeq H_*(\conf_n) \oplus H_{*-1}(\confm{n-1})$ induces the decomposition (see also \cite{gorj}) $$
H_*(\confm{n}) = \oplus_{i \geq 0} H_{*-i} (\conf_{n-i})
$$ that is isomorphic to the $E^2$-term of the spectral sequence above
and the same argument proves the splitting for any system of coefficients where the group $\Gb{n}$ acts trivially.
\end{rem}
As a consequence of the previous remark the $E^\infty$ term of the spectral sequence given in \eqref{eq:ss_bn} is isomorphic to the homology $H_*(\Gb{n}; \md)$.
Moreover the maps $\overline{s}, r$ induce respectively the inclusion of the first column and the projection onto the first column of the spectral sequence.
%TODO 
%aggiungere dettagli o referenze

Hence from the previous description of the spectral sequence and from Proposition \ref{prop:isodellesemplicifazioni} we have the following result.
\begin{prop} \label{prop:image_mu}
The image of $\mu_*$ corresponds to the direct sum of all the columns but the first in the spectral sequence \eqref{eq:ss_bn} for $H_*(\confm{n})$.
\end{prop} 
\begin{rem}\label{rem:Q_mu}
Let $\F$ be a field of characteristic $0$. Then for $i>1$ the map
$$
\mu_*: H_{i-1}(\confm{n-1}) \otimes H_1(S^1) \to H_i(\confm{n})
$$
is an isomorphism. 
\end{rem}
%and induces the splitting
%$$
%H_i(\confm{n}) = H_i(\conf_n) \oplus H_{i-1}(\confm{n-1}).
%$$
%
\begin{prop}\label{prop:tau1}
Let $\tau: H_i(\confm{n}) \to H_i(\confmd{n})$ be the transfer map induced by the double covering $\confmd{n} \to \confm{n}$.
The following diagram commutes:
\begin{equation}\label{diag:trerighe}
\begin{tabular}{c}
\xymatrix @R=2pc @C=2pc {
	H_{i-1}(\Br_n; H_1(S^1 \times P)) \ar[r]^\iota \ar[d]^\simeq & H_{i-1}(\Br_n; H_1(\ddiskP))\ar[d]^\simeq  \\
	H_{i-1}(\confm{n-1}) \otimes H_1(S^1) \ar[d]^{\mu_*} \ar[r] & H_i(\confmd{n}, \conf_n) \\
	H_i(\confm{n}) \ar[r]^{\tau} &H_i(\confmd{n})  \ar[u]_J
}
\end{tabular}
\end{equation}
\end{prop}
\begin{proof}

Let $\sigma_k$ be a $k$-simplex in $\confm{n-1}$. Then $p = \sigma_k(q)$ is a point in $\confm{n-1}$ and we write $p= (p_1, \{p_2, \ldots, p_n\})$. Let 
$$
\Sigma_p = \{(x,y) \in \C^2 \mid y^2 = (x-p_1)\cdots(x-p_n)\}
$$ be the double covering of $\C$ ramified around $\overline{p} = \{p_1, \ldots, p_n\}$. Then if we consider the projection $\pi: \totsp_n \to \conf_n$ we have $\Sigma_p = \pi^{-1} (\overline{p})$.

Let $\epsilon$ be the automorphism of $\totsp_n$ that maps $( \overline{p}, x,y) \mapsto (\overline{p},x,-y)$. It is clear that the restriction of $\epsilon$ to $\confmd{n}$ is the automorphism $\sigma: \confmd{n} \to \confmd{n}$.

Let $D_p$ be the intersection $N \cap \Sigma_p$. We can assume that $D_p$ is diffeomorphic to a closed disk and the restriction of the projection $\pi_x: (x, y) \mapsto x$ to $D_p$ is a double covering of a small disk around $p_1$ in $\C \setminus \{p_2, \ldots, p_n\}$ ramified in $p_1$.

Let $\overline{\sigma}_k$ be the projection of $\sigma_k$ to $\conf_n$. The restriction of the tubolar neighborhood $N$ to $\overline{\sigma}_k$ is a trivial bundle. So we can define a parametrization $\gamma_p:[0,1]\to \partial D_p$ that is continuous in $p$. Then $\gamma_p$ represents a generator of $H_1(\partial D_p)$.

Let $\zeta$ be the standard generator of $H_1(S^1)$.
Hence the map $H_{i-1}(\confm{n-1}) \otimes H_1(S^1) \to H_i(\confmd{n}, \conf_n) $ is induced by mapping 
$\sigma_k \otimes \zeta \mapsto  \sigma_k \times \gamma $ defined as  $(\sigma_k \times \gamma) (q, t) = ( \overline{\sigma}_k(q), \gamma_{\sigma_k(q)}(t))$.

We can replace $\gamma_p$ by $\gamma'_p + \gamma''_p$ where we define $\gamma'_p (t) := \gamma_p(t/2)$ and $\gamma''_p(t) := \gamma_p((1+t)/2)$, both for $t \in [0,1]$. 

Hence we have that the map $H_{i-1}(\confm{n-1}) \otimes H_1(S^1) \to H_i(\confmd{n}, \conf_n) $ is induced by mapping 
$\sigma_k\otimes \zeta \mapsto  \sigma_k \times (\gamma' + \gamma'')$ defined as above. Moreover $\epsilon(\sigma_k \times \gamma') = \sigma_k \times \gamma''$ and $\epsilon(\sigma_k \times \gamma'') = \sigma_k \times \gamma'$.

Recall that $\mu(p,e^{it}) =  (p_1 + \delta(\overline{p})e^{it}, \overline{p})$. Hence, up to a suitable choice of  the tubular neighborhood $N$ and of the parametrization $\gamma$, we can assume that
$$
p_1 + \delta(\overline{p})e^{it} = \pi_x(\gamma'_{p}(t)).
$$
This implies that $\mu_*(\sigma_k\otimes \zeta) = \sigma_k \times \pi_x(\gamma')= \sigma_k \times \pi_x(\gamma'')$, where we define $ (\sigma_k \times \pi_x(\gamma'))(q,t) = (p_1 + \delta(\overline{p})e^{it},\overline{p})$.

It is now clear that $\sigma_k \times \gamma'$ and $\sigma_k \times \gamma''$ are both liftings of $\mu_*(\sigma_k\otimes \zeta)$ and since $\epsilon$ exchanges the two liftings, we have that 
the map $\tau: H_i(\confm{n}) \to H_i(\confmd{n})$ is the transfer map induced by the double covering $\confmd{n} \to \confm{n}$.
\end{proof}
In the case of $n$ odd a different proof of Proposition \ref{prop:tau1} can be found in \cite[Lem.~58]{bianchi}.

\begin{prop}\label{prop:commut_tau}
The following diagram commutes:
$$
\begin{tabular}{c}
\xymatrix @R=2pc @C=2pc {
H_i(\confm{n})\ar[d]^\simeq \ar[r]^{\tau} &H_i(\confmd{n}) \ar[d]^\simeq \\
H_i(\Gb{n}; \ring[t]/(1+t)) \ar[r]^{1-t} & H_i(\Gb{n}; \ring[t]/(1-t^2))	.
}
\end{tabular}
$$
where in the bottom row we are considering the map induced by the $(1-t)$-multiplication map $C_*(\Gb{n}, \ring[t]/(1+t)) \to C_*(\Gb{n}, \ring[t]/(1-t^2))$ 
\end{prop}
\begin{proof}
%TODO inserire la dimostrazione

The complex $C_*(\Gb{n}, \ring[\Z]) = C_*(\Gb{n}, \ring[t^\pmu])$ computes the homology of the infinite cyclic cover $\confmd{n}^\Z$ associated to the homomorphism $\Gb{n} \to \Z$ that maps the first standard generator to multiplication by $(-t)$ and all other generators to multiplication by $1$.

Hence the complex $C_*(\Gb{n}, \ring[\Z_2]) = C_*(\Gb{n}, \ring[t]/(1-t^2))$ computes the homology of the double cover $\confmd{n}$ and $C_*(\Gb{n}, R) = C_*(\Gb{n}, \ring[t]/(1+t))$ computes the homology of $\confm{n}$.

Since the non-trivial monodromy associated to the double cover $\confmd{n} \to \confm{n}$ is induced by the first generator of $\Gb{n}$, the transfer of a cycle  in $C_*(\Gb{n}, \ring[t]/(1+t))$ to a cycle in $C_*(\Gb{n}, \ring[t]/(1-t^2))$ is given by the multiplication by $(1 -t)$.
\end{proof}

%\section{title}
%\vskip 10pt \emph{-interpretazioni in termini di calcoli di Callegaro-Marin}\vskip 10pt

\begin{rem} \label{rem:J}
Recall the isomorphism $H_i(\confmd{n}) \simeq H_i(\Gb{n}; \ring[t]/(1-t^2))$. Let $\ring$ be a field of characteristic $p$. For $p \neq 2$ the second term decomposes as $H_i(\Gb{n}; \ring[t]/(1+t)) \oplus H_i(\Gb{n}; \ring[t]/(1-t)) $ and moreover for $n$ odd  the term $H_i(\Gb{n}; \ring[t]/(1-t))$ is trivial. Again, this follows from the fact that the module $H_i(\Gb{n}; \ring[t^\pmu])$ has $(1+t)$-torsion and from the homology long exact sequence associated to
	$$
	0 \to \ring[t^\pmu] \stackrel{1-t}{\longrightarrow} \ring[t^\pmu ] \to \ring[t]/(1-t) \to 0.
	$$
	%to\cite[Sec.~4.5]{calmar}). 
	In particular the homology groups $H_i(\confmd{n}) \simeq H_i(\Gb{n}; \ring[t]/(1-t^2))$ and $H_i(\confm{n}) \simeq H_i(\Gb{n}; \ring[t]/(1+t))$ are isomorphic and the isomorphis is induced by the quotient map $\ring[t]/(1-t^2) \to \ring[t]/(1+t)$.
	Hence 
	we can consider the commuting diagram
	$$
	\begin{tabular}{c}
	\xymatrix @R=2pc @C=2pc {
		0 \ar[r]& H_i(\conf_n) \ar[d]^\simeq \ar[r] & H_i(\confmd{n})	\ar[d]^\simeq \ar[r]^J & H_i(\confmd{n}, \conf_n) \ar[d]^\simeq  \ar[r] & 0\\
		0 \ar[r] & H_i(\conf_n) \ar[r] & H_i(\confm{n}) \ar[r]^{\overline{J}} & H_i(\confm{n}, \conf_n) \ar[r] & 0.
	}
	% H_i(\confm{n})
	% H_i(\confmd{n})
	% H_i(\Gb{n}; \ring[t]/(1+t))
	% H_i(\Gb{n}; \ring[t]/(1-t^2))
	\end{tabular}
	$$
	where the last vertical map is an isomorphism from the five lemma. 
	From the right square we have that the map $J$ corresponds to the homomorphism
	$$
\overline{J}:	H_i(\Gb{n}; \ring[t]/(1+t)) \to H_i((\Gb{n}, \Ga{n-1}); \ring[t]/(1+t))
	$$
	associated to the inclusion $\Ga{n-1} \into \Gb{n}$ induced by $\conf_n \into \confm{n}$.  From the short exaxt sequence in the second row of the diagram above we have that the homomorphism $$\overline{J}: H_i(\confm{n}) = H_i(\confm{n}, \conf_n) \oplus H_i(\conf_n) \to H_i(\confm{n}, \conf_n)$$ is the projection on the first term of the direct sum.
\end{rem}

\begin{lem}\label{lem:tau_invertible}
If $p$ is an odd prime or $p=0$ and $\ring$ is a field of characteristic $p$, then the homomorphism
$$
\overline{\tau}:H_i(\Gb{n}; \ring[t]/(1+t)) \stackrel{1-t}{\longrightarrow} H_i(\Gb{n}; \ring[t]/(1-t^2))
$$
is invertible for $n$ odd.
\end{lem}
\begin{proof}
This follows since, for odd $n$, the homology group $H_i(\Gb{n}; \ring[t^\pmu])$ has $(1+t)$-torsion (see Proposition \ref{prop:homol_pdisp}
%, \cite[Sec.~4.5]{calmar}
). 
For $p \neq 2$ we have that  $H_i(\Gb{n}; \ring[t]/(1-t)) = 0$ (see Remark \ref{rem:J}) and hence 
$$
H_i(\Gb{n}; \ring[t]/(1-t^2)) \simeq H_i(\Gb{n}; \ring[t]/(1+t))
$$
and the map $\overline{\tau}$ is equivalent to the multiplication map
$$
H_i(\Gb{n}; \ring[t]/(1+t)) \stackrel{1-t}{\longrightarrow} H_i(\Gb{n}; \ring[t]/(1+t))
$$
and $(1-t)$ is invertible.
\end{proof}
\begin{rem} \label{rem:Q_tau_J}
The decompositions $H_i(\confmd{n}) \simeq H_i(\Gb{n}; \ring[t]/(1-t^2)) = H_i(\Gb{n}; \ring[t]/(1+t)) \oplus H_i(\Gb{n}; \ring[t]/(1-t))$ and $H_i(\confm{n}) = H_i(\confm{n}, \conf_n) \oplus H_i(\conf_n)$ give the following consequences for $n$ odd and $\F$ a field of characteristic $0$. Since $H_i(\Gb{n}; \ring[t]/(1-t))$ has Poincar\'e polynomial $(1+q)q^{n-1}$ (Proposition \ref{prop:hrazionale_minus_t}) and since $H_*(\conf_n)$ has Poincar\`e polynomial $(1+q)$ (see \S~\ref{ss:braid_homol}), we have that the map
$$
J: H_i(\confmd{n})  \to H_i(\confmd{n}, \conf_n) 
$$
is an isomorphism for $i>1$.
Moreover the argument of Lemma \ref{lem:tau_invertible} implies that
the map
$$\tau: H_*(\confm{n}) \to H_*(\confmd{n})$$ is injective and its cokernel has Poincar\'e polynomial $(1+q)q^{n-1}$.
\end{rem}

\begin{prop} \label{prop:coker_n_pari}
Let $n$ be even and $\F$ a field of characteristic $0$. Then for $i>1$ the map 
$$
\iota: H_{i-1}(\Br_n; H_1(S^1 \times P))  \to H_{i-1}(\Br_n; H_1(\ddiskP))
$$
is injective and its cokernel has rank $1$ for $i = n-1, n-2$ and $0$ otherwise.
\end{prop}
\begin{proof}
The result follows from Remark \ref{rem:Q_mu} and Remark \ref{rem:Q_tau_J}.
\end{proof}
\begin{thm} \label{teo:tau_restricted}
%\todo{controllare (ripetizione di Lemma \ref{lem:comp_no_4}?)}
Consider the decomposition $H_i(\confm{n}) = H_i(\confm{n}, \conf_n) \oplus H_i(\conf_n)$ associated to the inclusion $\overline{s}:\conf_n \into \confm{n}$ and $H_i(\confmd{n}) = H_i(\confmd{n}, \conf_n) \oplus H_i(\conf_n)$ associated to the inclusion $s:\conf_n \into \confmd{n}$. If $n$ is odd the following inclusion holds: $$\tau(H_*(\conf_n)) \subset H_*(\conf_n)$$
and for $x \in H_*(\conf_n)$ we have that $\tau(x) = 2x$.
\end{thm}

\begin{proof}
Since $\tau$ is the transfer map, we can consider the following diagram:
	$$
\begin{tabular}{c}
\xymatrix @R=2pc @C=2pc {
s\conf_n  \sqcup s' \conf_n \ar[d] \ar[r] & \conf_n \ar[d]^{\overline{s}} \\
\confmd{n} \ar[r]& \confm{n}
}
\end{tabular}
$$
where the left vertical map is the natural inclusion and the horizontal maps are $2:1$ projections that induces the transfer map $\tau$ and its restriction to $H_*(\conf_n)$. 
Then we have the following commuting diagram in homology:
$$
\xymatrix{ 
H_*(\conf_n) 
\ar[d]^{\overline{s}_*} \ar[drr]^(0.65){s_*+s'_*} \ar[rr]_(0.44){\tau_{|\overline{s}_*H_*(\conf_n)}} 
&& 
H_*(s(\conf_{n})  \sqcup s'(\conf_{n}))
\ar[d]^{i_*} 
\\
H_*(\confm{n}) 
\ar[rr]^\tau 
&& 
H_*(\confmd{n}) 
 }
$$
Hence given a cycle $x \in H_*(\conf_n)$ we have that %$\tau_{11}(\alpha) = \pi_1(s_*(\alpha)+s'_*(\alpha)) = 2 s_* \alpha$ 
$\tau \overline{s}_* x = s_* x + s'_* x = 2 s_* x \in H_*(\conf_n)$
where the last equality follows because for $n$ odd we have that $s_*=s'_*$ (see Remark \ref{rem:s_sprime}).
%
%Since for $n$ odd the maps $s$ and $s'$ are homotopic, it follows that for $x \in H_*(\conf_n)$ we have $\tau x = s_* x + s'_* x = 2 s_* x \in H_*(\conf_n)$.
\end{proof}

As a consequence of the previous result and Lemma \ref{lem:tau_invertible} we obtain the following.
\begin{cor} \label{cor:tau_restricted}
	If $p$ is an odd prime or $p=0$ and $\ring$ is a field of characteristic $p$ and $n$ is odd, then the projection on $H_i(\confmd{n},\conf_n)$ of the restriction of the homomorphism $\tau$ to $H_i(\confm{n},\conf_n)$
	$$
	\tau_|:H_i(\confm{n},\conf_n) \to H_i(\confmd{n},\conf_n)
	$$
	is an isomorphism.
\end{cor}
\begin{proof}
The corollary follows since under the stated conditions the homomorphism $\tau$ is an isomorphism (Lemma \ref{lem:tau_invertible}) and its restriction to $H_i(\conf_n)$ maps to $H_i(\conf_n)$.
\end{proof}

In order to prove the result concerning the odd torsion of the homology $H_i(\Br_n; \sym{g})$ we need to understand the maps 
%$\mu_*$ and 
$J$ in the diagram \eqref{diag:trerighe}.

We have the following result.
	
\begin{lem}\label{lem:ss_incl}
%nella ss la mappa di inclusione è l'iso con la prima colonna
Let $E^2_{ij} =  H_j(\Br(n-i);\ring[t]/(1+t)) \Rightarrow H_i(\Gb{n}; \ring[t]/(1+t))$ be the spectral sequence induced by the
filtration $\Filt$ described above. The inclusion $\Ga{n-1} \into \Gb{n}$ induces the isomorphism of the term $H_j(\Ga{n-1}; \ring[t]/(1+t)) = H_j(\Br(n);\ring[t]/(1+t)) $ with the submodule $E^2_{0j}$ for all $j$.
\end{lem}
\begin{proof}
The Lemma follows since the inclusion $\conf_n \into \confm{n}$ maps the $i$-th standard generator of $\Ga{n-1}$ to the $(i+1)$-st standard generator of $\Gb{n}$. Hence the image of the complex that computes the homology of $\Ga{n-1}$ is the first term of the filtration $\Filt$.
\end{proof}
\begin{cor}\label{cor:image_mu}
According to the decomposition $H_*(\confm{n})= H_*(\conf_n) \oplus H_*(\confm{n}, \conf_n)$ induced by the section $\conf_n \into \confm{n}$  and the projection $\pi: \confm{n} \to \conf_n$ we have that the image of $\mu_*$ corresponds to $H_*(\confm{n}, \conf_n)$.
\end{cor}
\begin{proof}
This follows from Lemma \ref{lem:ss_incl} and Proposition \ref{prop:image_mu}, since, as seen in Remark \ref{rem:collapsss}, the spectral sequence of Lemma \ref{lem:ss_incl} collapses at the page $E^2$.
\end{proof}

%\begin{cor}\label{cor:J}
%Let $n$ be an odd integer. We consider the homology with coefficient in a field of characteristic $p \neq 2$.
%The map $J: H_i(\confmd{n}) \to H_i(\confmd{n}, \conf_n)$ corresponds to a map $\overline{J}$ 
%
%of spectral sequences on the spectral sequences associated to the filtration $\Filt$. The map $\overline{J}^2$ is surjective and restricts to an isomorphism  $\overline{J}_{ij}^2:E^2_{ij}(\confmd{n}) \to E^2_{ij}(\confmd{n}, \conf_n)$ for all $i \neq 0$.
%Moreover for $i=0$ we have $E^2_{ij}(\confmd{n}, \conf_n) = 0$.
%% J è l'iso su tutte le col tranne la prima per n dispari, p non 2
%\end{cor}
%\begin{proof}
%%As follows from construction given in the proof of Proposition \ref{prop:commut_tau}, the action of the automorphism $\sigma: \confmd{n} \to \confmd{n}$ induces the multiplication by $(-t)$ on $H_*(\confmd{n})$.
%%Since for $n$ odd the sections $s:\conf_n \to \confmd{n}$ and $s':\conf_n \to \confmd{n}$ are homotopic (Remark \ref{rem:s_sprime}) and $\sigma s = s'$, the image of $H_*(\conf_n)$ in $H_*(\confmd{n})$ is a trivial $\sigma_*$-module and hence an $R[t]/(1+t)$ submodule.
%%The maps of the short exact sequence
%%$$
%%0 \to H_i(\conf_n) \to H_i(\confmd{n}) \stackrel{J}{\to} H_i(\confmd{n}, \conf_n) \to 0
%%$$ 
%%associated to the pair $(\confmd{n}, \conf_n)$ 
%We can consider the commuting diagram
%$$
%\begin{tabular}{c}
%\xymatrix @R=2pc @C=2pc {
%\conf_n \ar[r]^s \ar[dr]^{\overline{s}} & \confmd{n} \ar[d]\\
% & \confm{n}
%}
%\end{tabular}
%$$
%where the vertical map induces an isomorphism in homology. Hence we have a map from the homology of $\conf_n$ to the first column of the spectral sequence $E^2_{ij}(\confmd{n})$ defined according to Remark \ref{rem:Cn_sections} and Lemma \ref{lem:ss_incl} and a map from the spectral sequence $E^2_{ij}(\confmd{n})$ to the spectral sequence $E^2_{ij}(\confmd{n}, \conf_n)$. Since for $n$ odd and $p \neq 2$ we have $H_i(\confmd{n}) \simeq H_i(\confm{n})$ (as follows from Remark \ref{rem:J}), $E^2_{ij}(\confmd{n}) \simeq E^2_{ij}(\confm{n})$ 
%and $E^2_{ij}(\confmd{n}, \conf_n) \simeq E^2_{ij}(\confm{n}, \conf_n)$ (and the isomorphisms are all induced by the projection $\confmd{n} \to \confm{n}$) we can apply Lemma \ref{lem:ss_incl}. It follows that $E^2_{0j}(\confmd{n}, \conf_n) = 0$ and $E^2_{ij}(\confmd{n}) \simeq E^2_{ij}(\confmd{n}, \conf_n)$ for $i \neq 0$, where the last isomorphism is induced by the map $J$.
%The last statement follows from Lemma \ref{lem:ss_incl}
%%Hence the result follows. 
%\end{proof}
\begin{thm}
Let $n$ be an odd integer and let $g = (n-1)/2$ and let $\sym{g}= H_1(\surf_n;\Z)$ be the integral symplectic representation of the braid group $\Br_n$.
Then the homology  $H_i(\Br_n; \sym{g})$ is a torsion $\Z$-module
with only $2^j$-torsion.
% and the torsion of its elements is a power of $2$.
%and for any odd prime $p$ 
%the homology $H_i(\Br_n; \sym{g})$
%has no $p$-torsion.	
\end{thm}
\begin{proof}
From the description of the map $\mu_*$ (Proposition \ref{prop:isodellesemplicifazioni} and Corollary \ref{cor:image_mu}), the results about the map $\tau$ (Proposition \ref{prop:tau1}, Lemma \ref{lem:tau_invertible}, Corollary \ref{cor:tau_restricted}) and 
%Corollary \ref{cor:J} 
Remark \ref{rem:J} concerning the map $J$
we have that the map $\iota$ in diagram \eqref{diag:trerighe} is an isomorphism for $n$ odd and $p \neq 2$.
The result follows from the exact sequence of diagram \eqref{eq:bia2}.
\end{proof}