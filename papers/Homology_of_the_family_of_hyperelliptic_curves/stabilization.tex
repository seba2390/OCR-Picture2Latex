

\subsection{Stablization results}
Applying the results of Wahl and Randal-Willians (\cite{W-RW}) for the stability of family of groups with twisted coefficients  it is possible to prove  that the groups $H_i(\Br_n; H_1(\surf_n))$ stabilize for all $i$. 
In particular (see see \cite[Thm.~52]{bianchi}) the map $H_i(\Br_n; H_1(\surf_n)) \to H_i(\Br_{n+1}; H_1(\surf_{n+1}))$ is an epimorphism for $i \leq \frac{n}{2}-1$ and an isomorphism for $i \leq \frac{n}{2}-2$.
In this section  we will prove a slightly sharper result using the explicit description of the homology.

\begin{df}
	The stabilization map $\stab: \confm{n} \to \confm{n+1}$ is given by 
	$$
	(p_1, \{p_2, \ldots, p_{n+1}\}) \mapsto (p_1, \{p_2, \ldots, p_{n+1}, \frac{1+\max_{1 \leq i \leq n+1} (|p_i|)}{2} \}) 
	$$
\end{df}

%In the notation of \cite{calmar} we have $H_i(\Gb{n}; \F[t]/(1+t)))$ = $H_i(\mathrm{B}(2,1,n); \F))$ and $H_i(\Gb{n}; \F[t]/(1-t^2))) = H_i(\mathrm{B}(4,2,n); \F))$.
We recall that $\confm{n}$ is a classifying space for $\Gb{n}$, that is the Artin group of type $\mathrm{B}$. Moreover we recall from \cite[Cor.~4.17, 4.18,4.19]{calmar} (notations $H_i(\mathrm{B}(2,1,n); \F))$ and $H_i(\mathrm{B}(4,2,n); \F))$ were there used respectively for $H_i(\Gb{n}; \F[t]/(1+t))$ and $H_i(\Gb{n}; \F[t]/(1-t^2))$):
\begin{prop}\label{prop:stab_rank}
Let $p$ be a prime or $0$. Let $\F$ be a field of characteristic $p$. Let us consider the stabilization homomorphisms
$$
\stab_*:H_i(\Gb{n}; \F[t]/(1+t)) \to H_i(\Gb{n+1}; \F[t]/(1+t))
$$
and
$$
\stab_*:H_i(\Gb{n}; \F[t]/(1-t^2)) \to H_i(\Gb{n+1}; \F[t]/(1-t^2)).
$$
\begin{enumerate}[label=\alph*)]
\item If $p=2$ $\stab_*$ is epimorphism for $2i \leq n$ and isomorphisms for $2i < n$.
\item If $p>2$ $\stab_*$ is epimorphism for $\frac{p(i-1)}{p-1}+2 \leq n$ and isomorphisms for $\frac{p(i-1)}{p-1}+2 < n$.
\item If $p=0$ $\stab_*$ is epimorphism for $i+1 \leq n$ and isomorphisms for $i+1 < n$.
\end{enumerate}
\end{prop}
%\subsection{Stabilization an the map $\mu_*$}

The map $\mu$ commutes, up to homotopy, with the stabilization map $\stab: \confm{n} \to \confm{n+1}$:
$$
\xymatrix{
\confm{n-1} \times S^1 \ar[d]^{\stab \times \Id} \ar[r]^\mu& \confm{n} \ar[d]^\stab\\
\confm{n} \times S^1 \ar[r]^\mu & \confm{n+1}
}
$$

%\subsection{Stabilization and the map $\tau$}

The map $\tau$ naturally commutes with the stabilization homomorphism $\stab_*$ in homology, since $\tau$ is given by the multiplication by $(1-t)$.

%\subsection{Stabilization and the map $J$}

We can also define a geometric stabilization map $\stab: \confmd{n} \to \confmd{n+1}$ as follows:
$$
\gstab:(P, z, y) \mapsto (P \cup \{p_{\infty}\}, z, y \sqrt{z-p_{\infty}})
$$
where we set $p_\infty:= \frac{\max ( \{|p_i|, p_i \in P \} \cup \{|z|\})+1}{2}$ and since $\Re(z-p_{\infty}) < 0$ we choose
$\sqrt{z-p_{\infty}}$ to be the unique square root with $\Im(\sqrt{z-p_{\infty}}) >0$

The following diagram is homotopy commutative:
$$
\xymatrix{
	\conf_n \ar[d]^{\stab} \ar[r]^s& \confmd{n} \ar[d]^\gstab\\
	\conf_{n+1}  \ar[r]^s & \confmd{n+1}
}
$$
and this imply that $J$ commutes with the stabilization homomorphism $\gstab_*$.

We also need to prove that the following diagram commutes:
$$
\xymatrix{
	H_*(\Gb{n};\ring[t]/(1-t^2)) \ar[d]^{\stab_*} \ar[r]^(0.63){\simeq}& H_*(\confmd{n}) \ar[d]^{\gstab_*}\\
	H_*(\Gb{n+1};\ring[t]/(1-t^2))  \ar[r]^(0.63){\simeq} & H_*(\confmd{n+1})
}
$$
This is true since the homomorphism $$\stab_*:H_*(\Gb{n};\ring[t]/(1-t^2)) \to H_*(\Gb{n+1};\ring[t]/(1-t^2))$$ is induced by the map $\stab: \confm{n} \to \confm{n+1}$ previously defined and it is obtained applying the Shapiro lemma to $\confm{n} = k(\Gb{n},1)$, with $\ring[t]/(1-t^2)) = \ring[\Z_2] = \ring[\pi_1(\confm{n})/\pi_1(\confmd{n})]$.
It is straightforward to check that the diagram
$$
\xymatrix{
 \confmd{n} \ar[d]^\gstab \ar[r]& 	\confm{n}\ar[d]^{\stab} \\
 \confmd{n+1} \ar[r] & 	\confm{n+1}
}
$$
commutes, where the horizontal maps are the usual double coverings. Finally we have proved the following result.
\begin{lem}\label{lem:commutative_stab}
The following diagram is commutative
\begin{equation}\label{diag:stabilization}
\begin{tabular}{c}
\xymatrix @R=2pc @C=2pc {
%H_{i-1}(\Br_n; H_1(S^1 \times P)) \ar[r]^\iota \ar[d]^\simeq & H_{i-1}(\Br_n; H_1(\ddiskP))\ar[d]^\simeq  \\
H_{i-1}(\confm{n-1}) \otimes H_1(S^1) \ar[d]^{\stab_* \otimes \Id} \ar[r]^(0.6){\mu_*}  &	H_i(\confm{n}) \ar[r]^{\tau} \ar[d]^{\stab_*} &H_i(\confmd{n})  \ar[r]^(0.45)J \ar[d]^{\gstab_*}& H_i(\confmd{n}, \conf_n) \ar[d]^{\gstab_*} \\
H_{i-1}(\confm{n}) \otimes H_1(S^1) \ar[r]^(0.6){\mu_*}  &	H_i(\confm{n+1}) \ar[r]^{\tau} &H_i(\confmd{n+1})  \ar[r]^(0.45)J & H_i(\confmd{n+1}, \conf_{n+1}) 
}
\end{tabular}
\end{equation}
\end{lem}

\begin{thm}\label{thm:stabilization}
Consider homology with integer coefficients. The homomorphism 
$$
H_i(\Br_n; H_1(\surf_n)) \to H_i(\Br_{n+1}; H_1(\surf_{n+1}))
$$
is an epimorphism for $i \leq \frac{n}{2}-1 $
and an isomorphism for $i < \frac{n}{2}-1$.

For $n$ even $H_i(\Br_n; H_1(\surf_n))$ has no $p$ torsion (for $p > 2$) when $\frac{pi}{p-1}+3 \leq n$ and no free part for $i+3 \leq n$. In particular for $n$ even,  when $\frac{3i}{2}+3 \leq n$ the group $H_i(\Br_n; H_1(\surf_n))$ has only $2$-torsion.
\end{thm}
\begin{proof}
The maps in the diagram  \eqref{diag:stabilization} with $\Z_p$ coefficients fits in the map of long exact sequences
$$
%\begin{tabular}{c}
\xymatrix @R=1.5pc @C=0.8pc {
\cdots \ar[r] &	H_{i-1}(\confm{n-1};\Z_p) \otimes H_1(S^1) \ar[d]^{\stab_* \otimes \Id} \ar[r]^(0.6){\iota}  &	H_i(\confmd{n}, \conf_n;\Z_p) \ar[d]^{\gstab_*} \ar[r] & H_{i-1} (\Br_n; H_1(\surf_n;\Z_p)) \ar[d]^{\stab_*} \ar[r] & \cdots\\
\cdots \ar[r] &	H_{i-1}(\confm{n};\Z_p) \otimes H_1(S^1) \ar[r]^(0.55){\iota}  & H_i(\confmd{n+1}, \conf_{n+1};\Z_p) \ar[r]  & H_{i-1} (\Br_{n+1}; H_1(\surf_n;\Z_p)) \ar[r] & \cdots
}
%\end{tabular}
$$
that near $H_{i-1} (\Br_n; H_1(\surf_n;\Z_p))$
looks as follows:
\begin{equation}\label{diag:stabilization2}
\begin{tabular}{l}
\xymatrix @R=1.5pc @C=0.8pc {
	\!\!\cdots\ar[r]\! &	\!H_i(\confmd{n}, \conf_n;\Z_p)\! \ar[d]^{\gstab_*} \ar[r] &\!H_{i-1} (\Br_n; H_1(\surf_n;\Z_p))\! \ar[d]^{\stab_*} \ar[r] &\!H_{i-2}(\confm{n-1};\Z_p)\! \otimes\!H_1(S^1)\! \ar[d]^{\stab_* \otimes \Id} \ar[r] &\!\cdots\!\!\\
	\!\!\cdots\!\ar[r]\!& \!H_i(\confmd{n+1}, \conf_{n+1};\Z_p)\! \ar[r]  & \!H_{i-1} (\Br_{n+1}; \!H_1(\surf_n;\Z_p))\!\ar[r] &	\!H_{i-2}(\confm{n};\Z_p) \! \otimes\!H_1(S^1)\!\ar[r]  & \!\cdots\!\!
}
\end{tabular}
\end{equation}
For $p=2$, from Proposition \ref{prop:stab_rank} and Lemma \ref{lem:commutative_stab} we have that the vertical map $\gstab_*$ on the left of diagram \eqref{diag:stabilization2} is an epimorphism for for $i \leq \frac{n}{2}$ and isomorphisms for $i < \frac{n}{2}$. The vertical map $\stab_* \otimes \Id$ on the right of diagram \eqref{diag:stabilization2} is an isomorphism for $i \leq \frac{n}{2}$.

This implies that $$
\stab_*:H_i(\Br_n; H_1(\surf_n; \Z_2)) \to H_i(\Br_{n+1}; H_1(\surf_{n+1}; \Z_2))
$$
is epimorphism for $i \leq \frac{n}{2} -1$
and an isomorphism for $i < \frac{n}{2}-1$. 

For $p>2$, from Proposition \ref{prop:stab_rank} and Lemma \ref{lem:commutative_stab} we have that the vertical map $\gstab_*$ on the left of diagram \eqref{diag:stabilization2} is an epimorphism for $\frac{p(i-1)}{p-1}+2 \leq n$ and isomorphisms for $\frac{p(i-1)}{p-1}+2 < n$. The vertical map $\stab_* \otimes \Id$ on the right of diagram \eqref{diag:stabilization2} is an isomorphism  for $\frac{p(i-1)}{p-1}+2 \leq n$. We notice that actually these conditions are weaker that the ones for $p=2$.

This implies that for $p>2$ $$
\stab_*:H_i(\Br_n; H_1(\surf_n; \Z_p)) \to H_i(\Br_{n+1}; H_1(\surf_{n+1}; \Z_p))
$$
is epimorphism for $\frac{pi}{p-1}+2 \leq n$
and an isomorphism for $\frac{pi}{p-1}+2 < n$. 

The same argument for $p=0$ shows that
$$
\stab_*:H_i(\Br_n; H_1(\surf_n; \Q)) \to H_i(\Br_{n+1}; H_1(\surf_{n+1}; \Q))
$$
is an epimorphism for $i+2 \leq n$
and an isomorphism for $i+2< n$. 

From the Universal Coefficients Theorem for homology we get that the homomorphism 
$$
H_i(\Br_n; H_1(\surf_n)) \to H_i(\Br_{n+1}; H_1(\surf_{n+1}))
$$
is an epimorphism for $i \leq \frac{n}{2}-1 $
and an isomorphism for $i < \frac{n}{2}-1$.


Since the integer  homology $H_i(\Br_n; H_1(\surf_n))$ has only $2$-torsion for $n$ odd, 
the stabilization implies that for $n$ even $H_i(\Br_n; H_1(\surf_n))$ has no $p$ torsion (for $p > 2$) for $\frac{pi}{p-1}+3 \leq n$ and no free part for $i+3 \leq n$.

In particular, for $n$ odd,  $\frac{3i}{2}+3 \leq n$ we have that $H_i(\Br_n; H_1(\surf_n))$ has only $2$-torsion.
\end{proof}

\begin{thm}\label{thm:unstable}
For $n$ even the groups $H_i(\Br_n;H_1(\surf_n;\Z))$ are torsion, except for $i=n-1, n-2$ where $H_i(\Br_n;H_1(\surf_n;\Q)) = \Q.$
\end{thm}
\begin{proof}
The result follows from Proposition \ref{prop:coker_n_pari} and, for $i<2$, from the stabilization Theorem \ref{thm:stabilization}. For $n=4$ the result follows from a direct computation.
\end{proof}

In Table \ref{tab:conti} we present some computations of the groups $H_i(\Br_n;H_1(\surf_n))$ (with integer coefficients). The computations are obtained using an Axiom implementation of the complex introduced in \cite{salvetti}.
\begin{table*}[htb] 
\renewcommand{\arraystretch}{1.2}
\setlength\extrarowheight{3pt}
\begin{center}
\begin{tabular}{|c|c|c|c|c|c|c|c|c|c|c|c|}
\hline
\backslashbox{$n$}{$i$}
& 1 & 2 & 3 & 4 & 5 & 6 & 7 & 8 & 9 & 10 & 11\\
\hline
$3$ & $\Z_2$ &&&&&&&&&&\\
\hline
$4$ & $\Z_2^2$ & $\Z$&$\Z$&&&&&&&&\\
\hline
$5$
 & 
\cellcolor{lightgray!30}
$\Z_2$ & 
$\Z_2$
&$\Z_2$&&&&&&&&\\ 
\hline
$6$ & $\Z_2$ & $\Z_2^2$&$\Z_2^2  \Z_3$&$\Z$&$\Z$&&&&&&\\
\hline
$7$ & $\Z_2$ & \chl $\Z_2$&$\Z_2^2 $&$\Z_2^2$&$\Z_2$&&&&&&\\
\hline
$8$ & $\Z_2$ & $\Z_2$&$\Z_2^3 $&$\Z_2^3\Z_3$&$\Z_2^3\Z_3$&$\Z$&$\Z$&&&&\\
\hline
$9$ & $\Z_2$ & $\Z_2$&\chl $\Z_2^2 $&$\Z_2^3$&$\Z_2^3$&$\Z_2^2$&$\Z_2$&&&&\\
\hline
${10}$ & $\Z_2$ & $\Z_2$&$\Z_2^2 $&$\Z_2^4$&$\Z_2^4$&$\Z_2^4\Z_3$&$\Z_2^3 \Z_3\Z_5$&$\Z$&$\Z$&&\\
\hline
${11}$ & $\Z_2$ & $\Z_2$&$\Z_2^2 $&\chl $\Z_2^3$&$\Z_2^4$&$\Z_2^4$&$\Z_2^4 $&$\Z_2^3$&$\Z_2$&&\\
\hline
${12}$ & $\Z_2$ & $\Z_2$&$\Z_2^2 $&$\Z_2^3$&
% $\Z_2^5$&$\Z_2^5$ 
$\Z_2^5$ & $\Z_2^5$
&$\Z_2^6\Z_3$&$\Z_2^6\Z_3\Z_5$&$\Z_2^3\Z_3\Z_5$&$\Z$&$\Z$\\
\hline
${13}$ & $\Z_2$ & $\Z_2$&$\Z_2^2 $&$\Z_2^3$&\chl $\Z_2^4$&$\Z_2^5$&$\Z_2^6$&$\Z_2^6$&$\Z_2^5$&$\Z_2^3$&$\Z_2$\\
\hline
\end{tabular}
\end{center}
\caption{Computations of $H_i(\Br_n;H_1(\surf_n))$. For each column the first stable group is highlighted.}\label{tab:conti}
\end{table*}

\subsection{Poincar\'e polynomials}
We use the previous results to compute explicitly the Poincar\'e polynomials for odd $n$.

%-ogni generatore da luogo ad una classe stabile. infatti la stabilizzazione ($n \mapsto n+2$ ) viene data dalla moltiplicazione per $x_0^2$ che commuta con $\tau$.

\begin{thm}\label{thm:poincare}
For odd $n$, 
the rank of $H_i(\Br_n;H_1(\surf_n;\Z))$ as a $\Z_2$-module is the coefficient of $q^it^n$ in the series
 $$
\widetilde{P}_2(q,t)=\frac{qt^3}{(1-t^2q^2)} \prod_{i \geq 0} \frac{1}{1-q^{2^i-1}t^{2^i}}
$$
In particular the series $\widetilde{P}_2(q,t)$ is 
the Poincar\'e series of the homology group  %$$\oplus_{n}H_*(\Br_{2n+1};H_1(\surf_{2n+1};\Z))$$ 
$$\bigoplus_{n \mbox{\scriptsize odd}}H_*(\Br_{n};H_1(\surf_n;\Z))$$
as a $\Z_2$-module. 
%is the coefficient of $t^n$ of the series:	
\end{thm}

\begin{proof}
Let $P_2(\Br_n,H_1(\surf_n))(q)$ be the Poincar\'e polynomial for $H_*(\Br_n;H_1(\surf_n;\Z))$ as a $\Z_2$-module. 
Since we already know that for $n$ odd the homology $H_i(\Br_n;H_1(\surf_n;\Z))$ has only $2$-torsion, we can obtain $P_2(\Br_n,H_1(\surf_n))(q)$ from the Universal Coefficients Theorem as follows. We compute the Poincar\'e polynomial 
$
P_2(\Br_n, H_1(\surf_n;\Z_2))(q)
$
for
$H_i(\Br_n;H_1(\surf_n;\Z_2))
$
and we divide by $1+q$.

In order to compute $
P_2(\Br_n, H_i(\surf_n;\Z_2))(q)
$ 
we consider the short exact sequence
$$
0 \to \coker \iota_i \to H_{i}(\Br_n; H_1(\surf_n:\Z_2)) \to \ker \iota_{i-1} \to 0
$$
where we recall that $\iota_i$ is the map
$$\iota_i:H_{i}(\Br_n; H_1(S^1 \times P:\Z_2)) \to H_{i}(\Br_n; H_1(\ddiskP:\Z_2)).$$
Now it follows from the Remark \ref{rem:basi_mod_2} and from the description in Section \ref{sec:no4tor} that for a fixed odd $n$ the ranks of $\ker \iota$ and $\coker \iota$
are respectively the coefficients of $q^it^n$ in the following series:
$$
P_2(\coker \iota) = P_2(\ker \iota)  =
\frac{qt^3}{1-t^2q^2} \prod_{j \geq 0} \frac{1}{1-q^{2^j-1}t^{2^j}}
$$
Clearly the Polynomial for 
$H_{i}(\Br_n; H_1(\surf_n:\Z_2))$ is given by the coefficient of $t^n$ in the sum
$$
 P_2(\coker \iota) + qP_2(\ker \iota) 
$$
and hence dividing by $(1+q)$ we get our result. 
\end{proof}

%\begin{cor}
The same argument of the previous proof can be applied in stable rank. From the Remark \ref{rem:basi_mod_2} the Stable Poincar\'e polynomial of both $\coker \iota$ and $\ker \iota$ with $\Z_2$ coefficients is the following:
$$
\frac{q}{1-q^2} \prod_{j \geq 1} \frac{1}{1-q^{2^j-1}}
$$
%\end{cor}
Since there is no free part in $H_i(\Br_n;H_1(\surf_n;\Z))$ all these groups have only $2$-torsion. In particular for integer coefficients we get the following statement.
\begin{thm}\label{thm:stablepoincare}
The Poincar\'e polynomial of the stable homology $H_i(\Br_n;H_1(\surf_n;\Z))$ as a $\Z_2$-module is the following:
$$
P_2(\Br;H_1(\Sigma))(q) = \frac{q}{1-q^2} \prod_{j \geq 1} \frac{1}{1-q^{2^j-1}}
$$
\end{thm}
An explicit computation of the first terms of the stable series $P_2(\Br;H_1(\Sigma))(q)$ gives
$$
q+q^2+ 2q^3 + 3q^4 + 4q^5 + 5q^6 + 7q^7 + 9 q^8 + 11q^9 + 14q^{10} + 17 q^{11} + \ldots
$$
	
