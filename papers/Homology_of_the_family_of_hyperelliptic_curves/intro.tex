\begin{comment}
\documentclass[a4paper, 12pt, pdftex,reqno
%,draft% comment this for final version
]{amsart}
\usepackage[hang, flushmargin, bottom]{footmisc} % places footnotes at page bottom

\usepackage[hyperindex,breaklinks,pdftex, bookmarks]{hyperref}
%\usepackage{showkeys} % comment this for final version

\usepackage[all,  curve, frame, arc, color]{xy}

\usepackage{mathtools}

%\usepackage{mathptmx}       % selects Times Roman as basic font
\usepackage{helvet}         % selects Helvetica as sans-serif font
\usepackage{courier}        % selects Courier as typewriter font
%\usepackage{type1cm}        % activate if the above 3 fonts are
                            % not available on your system
%
%\usepackage{makeidx}         % allows index generation
\usepackage{graphicx}        % standard LaTeX graphics tool
                             % when including figure files
\usepackage{multicol}        % used for the two-column index

\makeindex             % used for the subject index
                       % please use the style svind.ist with
                       % your makeindex program

%%%%%%%%%%%%%%%%%%%%%%%%%%%%%%%%%%%%%%%%%%%%%%%%%%%%%%%%%%%%%%%%%%%%%%%%%%%%%%%%%%%%%%%%%

%pacchetti
%\setlength{\oddsidemargin}{-0.01in}
%\setlength{\textwidth}{5.5in}
\usepackage[utf8x]{inputenc}
\usepackage[english]{babel}
%\usepackage[T1]{fontenc}
\usepackage{amsmath} % tolto per svmult
\usepackage{amscd}
\usepackage{amsfonts}
\usepackage{amssymb}
\usepackage{amsthm} %tolto per svmult
\usepackage{mathabx}
\usepackage{mathrsfs}
\usepackage{graphicx}
\usepackage{textcomp}
\usepackage{url}
%\usepackage{hyperref}
%\hypersetup{colorlinks=true,linkcolor=blue}
\usepackage{amsopn,latexsym,amscd}
%\usepackage[pdftex]{graphicx}
%\usepackage[dvips]{epsfig}
\usepackage{color}
%\usepackage[textwidth=2cm, textsize=tiny]{todonotes}
\usepackage[disable]{todonotes}
\setlength{\marginparwidth}{2cm}
\usepackage{fullpage}
\usepackage{graphicx}

%comandi
  
\makeatletter
\newcommand{\eqnum}{\refstepcounter{equation}\textup{\tagform@{\theequation}}}
\makeatother

%teoremi, proposizioni, lemmi, ecc...
\newtheorem{thm}{Theorem}[subsection]
\newtheorem{lem}[thm]{Lemma}  %tolto per svmult

\newtheorem{prop}[thm]{Proposition}
\newtheorem{cor}[thm]{Corollary}


\newtheorem{question}[thm]{Question}

\newtheorem{Rthm}{Theorem}
\renewcommand*{\theRthm}{\Alph{Rthm}}


\theoremstyle{definition}
\newtheorem{df}[thm]{Definition}
\newtheorem{scholium}[thm]{Scholium}
\newtheorem{notation}[thm]{Notation}
\newtheorem{es}[thm]{Example}
%
\theoremstyle{remark}
\newtheorem{rem}[thm]{Remark}
%\newtheorem{rem}[thm]{Remark}


%insiemi numerici classici
\newcommand{\Z}{\mathbb{Z}}
\newcommand{\C}{{\mathbb{C}}}
\newcommand{\R}{{\mathbb{R}}}
\newcommand{\Q}{{\mathbb{Q}}}
\newcommand{\N}{{\mathbb{N}}}
\newcommand{\pp}{{\mathbb{P}}}
%lettere utili
\newcommand{\G}{{\mathcal{G}}}
%\newcommand{\Fc}{\mathcal{F}}
\newcommand{\rk}{{\operatorname{rk}}}
\newcommand{\codim}{{\operatorname{codim}}}
\newcommand{\Aut}{{\operatorname{Aut}}}
\newcommand{\out}{{\operatorname{Out}}}
\newcommand{\into}[0]{\hookrightarrow}
\newcommand{\id}{\mathrm{id}}
%%%%
%%%%

\newcommand{\Hom}{\operatorname{Hom}}


%groups

\newcommand{\Ga}[1]{\operatorname{G_{A_{#1}}}}
\newcommand{\Gb}[1]{\operatorname{G_{B_{#1}}}}
\newcommand{\Gbn}{\operatorname{G_{B_n}}}
\newcommand{\Br}{\operatorname{Br}}

%spaces

\newcommand{\disk}{\mathrm{D}}
\newcommand{\ddiskP}{\widetilde{\disk \setminus \P}}
\newcommand{\conf}{\operatorname{C}}
\newcommand{\confn}{\conf_n}
\newcommand{\confm}[1]{\operatorname{C_{1,#1}}}
\newcommand{\confmn}{\confm{n}}
\newcommand{\confmd}[1]{\widetilde{\operatorname{C_{1,#1}}}}
\newcommand{\confmdn}{\confmd{n}}


\newcommand{\totsp}{\mathrm{E}}

\renewcommand{\P}{\mathrm{P}}

\newcommand{\surf}{\mathrm{X}}
\newcommand{\compl}{\disk \setminus }
\newcommand{\ddisk}{def}

\end{comment}
%\begin{document}
	
%The Hurwitz spaces of  curves are, roughly speaking, spaces of curves of given genus, each of them being a $d-$covering of the complex line ramified over $n$ distinct points ($n$ and $d$ fixed). Precise definitions can be (slightly) different (\cite{fulton}, \cite{evw}).
 In this paper we consider %the particular case of 
 the family of hyperelliptic curves 
$$
\totsp_{n} := \{(\P, z,y ) \in \conf_n  \times \disk \times \C| y^2 = (z-x_1)\cdots(z-x_n) \}.
$$ 
where $\disk$ is the unit open disk in $\C, $ \  $\conf_n$ is the configuration space of $n$ distinct unordered points in $\disk$ and $\P=\{x_1,\dots,x_n\}\in \conf_n.$  
Each curve $\surf_n$ in the family is a ramified double covering of the disk $\disk$ and there is a fibration  $\pi:\totsp_n\to \conf_n$ which takes  $\surf_n$ onto its set of ramification points. Clearly $\totsp_n$ is a universal family over the Hurwitz space $H^{n,2}$ (for precise definitions see \cite{fulton}, \cite{evw}).
%Even though Hurwitz spaces are very classical and studied objects, it seems that their cohomology is not completely known in general. 

The aim of this paper is to compute the integral homology of the space $\totsp_n.$ 
The rational homology of $\totsp_n$ is known, having been computed in \cite{chen} by using \cite{cms_tams}.
The bundle $\pi:\totsp_n\to \conf_n$ has a global section, so $H_*(E_n)$ splits into a direct sum $H_*(\conf_n)\oplus H_*(\totsp_n,\conf_n)$ and by the Serre spectral sequence $H_*(\totsp_n,\conf_n)=H_{*-1}(\Br_n;H_1(\surf_n)).$ 
We use here that $\conf_n$ is a classifying space for the braid group $\Br_n.$ The action of the braid group over the homology of the surface is geometrical: the braid group embeds (see \cite{per_van_92, waj_99}) into the mapping class group of the surface (with one or two boundary components according to $n$ odd or even respectively) by taking the standard generators into particular Dehn twists.
In this paper we actually compute the homology of this symplectic representation of the braid groups. Since the homology of the braid groups is well-know (see for example \cite{fuks, vain, cohen}) we obtain a description of the homology of $\totsp_n.$ 
It would be natural to extend the computation to the homology of the braid group $\Br_n$ with coefficients in the symmetric powers of  $H_1(\surf_n)$. In the case of $n=3$ a complete computation (in cohomology) can be found in \cite{ccs}.

Some experimental computations given in \cite{msv2012} have led us to conjecture that $H_{*}(\Br_n; H_1(\surf_n;\Z))$  is only $2$-torsion for odd $n.$ 

Our main results are the following 

\begin{thm} (see Theorem \ref{th:no4tor}, \ref{thm:poincare})

For odd $n:$
\begin{enumerate}
\item  the integral homology $H_{i}(\Br_n; H_1(\surf_n;\Z))$ has only $2$-torsion. 
\item the rank of $H_i(\Br_n;H_1(\surf_n;\Z))$ as a $\Z_2$-module is the coefficient of $q^it^n$ in the series
 $$
\widetilde{P}_2(q,t)=\frac{qt^3}{(1-t^2q^2)} \prod_{i \geq 0} \frac{1}{1-q^{2^i-1}t^{2^i}}
$$
In particular the series $\widetilde{P}_2(q,t)$ is 
the Poincar\'e series of the homology group  %$$\oplus_{n}H_*(\Br_{2n+1};H_1(\surf_{2n+1};\Z))$$ 
$$\bigoplus_{n \mbox{\scriptsize odd}}H_*(\Br_{n};H_1(\surf_n;\Z))$$
as a $\Z_2$-module. 
\end{enumerate}
\end{thm}

\begin{thm} (see Theorem \ref{thm:stabilization}, \ref{thm:stablepoincare})
Consider homology with integer coefficients. 
\begin{enumerate}
\item The homomorphism 
$$
H_i(\Br_n; H_1(\surf_n)) \to H_i(\Br_{n+1}; H_1(\surf_{n+1}))
$$
is an epimorphism for $i \leq \frac{n}{2}-1 $
and an isomorphism for $i < \frac{n}{2}-1$.

\item For $n$ even $H_i(\Br_n; H_1(\surf_n))$ has no $p$ torsion (for $p > 2$) when $\frac{pi}{p-1}+3 \leq n$ and no free part for $i+3 \leq n$. In particular for $n$ even,  when $\frac{3i}{2}+3 \leq n$ the group $H_i(\Br_n; H_1(\surf_n))$ has only $2$-torsion.
\item 
The Poincar\'e polynomial of the stable homology $H_i(\Br_n;H_1(\surf_n;\Z))$ as a $\Z_2$-module is the following:
$$
P_2(\Br;H_1(\Sigma))(q) = \frac{q}{1-q^2} \prod_{j \geq 1} \frac{1}{1-q^{2^j-1}}
$$
\end{enumerate}
\end{thm}

We also find unstable free components in the top and top-1 dimension for even $n$ (Theorem \ref{thm:unstable}). 

The main tools that we use are the following. 

First, we use here some of the geometrical ideas in \cite{bianchi}, where  the author shows that the $H_*(\Br_n;H_1(\surf_n))$ is at most $4$-torsion using some exact sequences obtained from a Mayer-Vietoris decomposition. 

Second, we identify the homology groups which appear in the exact sequences with local homology groups of the configuration space $\confm{n}$ of $n+1$ points with one distinguished point. Such spaces are the classifying space of the  Artin groups of type $\mathrm{B},$ so we can use some of  the homology computations given in \cite{calmar}: our results heavily rely on these computations and we collect most of those we need in Section \ref{sec:homol_artin}.

Some explicit computations are provided in Table \ref{tab:conti}.
%\todo{Inserire risultato finale e teoremi di sezione \ref{sec:stab}}
%\end{document}

