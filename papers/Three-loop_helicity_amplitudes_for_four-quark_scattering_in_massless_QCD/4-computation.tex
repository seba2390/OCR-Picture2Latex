% !TEX root = 0-qqQQmain.tex

\section{Computation}
\label{computation}

We perform our calculations in dimensional regularization with $d = 4 - 2 \epsilon$ dimensions for all internal momenta and gluon fields. UV and IR singularities will then manifest themselves as poles in the dimensional regulator $\epsilon$.
In order to compute the  helicity amplitudes $\mathbfcal{H}_1$ and $\mathbfcal{H}_2$,  we begin by producing all relevant Feynman diagrams for the process in eq.~\eqref{s_channel} with \texttt{QGRAF}~\cite{Nogueira:1991ex}. Only 1 diagram contributes at tree level,  9 diagrams at one loop, 158 diagrams at two loops and 3584 at three loops. We give a few representative samples of the three-loop diagrams in figure~\ref{diagrams}. 
\begingroup
\centering
\begin{figure}[htbp]
\centering
\subfigure[]{\includegraphics[width=1.3in]{fig/Diag1}}\label{fig:1a}
\subfigure[] {\includegraphics[width=1.3in]{fig/Diag2}}\label{fig:1b}
\subfigure[]{\includegraphics[width=1.3in]{fig/Diag3}}\label{fig:1c}



\subfigure[]{\includegraphics[width=1.3in]{fig/Diag4}}\label{fig:1d}
\subfigure[] {\includegraphics[width=1.3in]{fig/Q4}}\label{fig:1e}
\subfigure[]{\includegraphics[width=1.3in]{fig/Q3}}\label{fig:1f}
\caption{Sample three loop diagrams
contributing to the process $q\bar q \rightarrow Q \bar Q $.} \label{diagrams}
\end{figure}
\endgroup
We use \texttt{FORM} \cite{Vermaseren:2000nd} to apply the tensor projectors of eq.~\eqref{eq:p12} to the diagrams, perform the Dirac traces and the colour algebra.  The latter can be boiled down to a repeated application of the identities

\begin{align}
  {({T^a})}_{ij} {({T^a})}_{kh} &= \frac12
  \left({\delta}_{ih} {\delta}_{kj} - \frac{1}{N_c} {\delta}_{ij} {\delta}_{kh}\right)\,, \quad
f^{abc} = - 2 \: i \: {\rm Tr}(T^a[T^b,T^c])\,
\end{align}
After the colour algebra  has been performed,  the quark colour indices can only appear via the two independent rank-4 tensor structures defined in \eqref{colour_structures}.   They appear in the amplitude accompanied by coefficients of the type 
\begin{equation}
n_f^a \: N_c^b \quad\text{with~}    a=0,\ldots,3,~ b=-4,\ldots,3\,.
\end{equation} 
Terms in the amplitude with different $a$ and $b$ are separately gauge invariant so there cannot be any gauge cancellations among them.
Because of this, we compute them separately, which allows
us to deal with smaller expressions.

After performing the colour and Dirac algebras we can express the helicity amplitudes as linear combinations
of scalar Feynman integrals with rational coefficients depending on the Mandelstam invariants $s$,  $t$ and the dimensional
regulator $\epsilon$. 
At $L$ loops, we write the integrals appearing in the amplitudes as
\begin{equation}\label{integrals}
\mathcal{I}^\text{top}_{n_1,n_2,...,n_N} = \mu_0^{2L\epsilon} e^{L \epsilon \gamma_E}  \int \prod_{i=1}^L \left( \frac{\mathrm{d}^d k_i}{i \pi^{\frac{d}{2}}} \right) \frac{1}{D_1^{n_1}D_2^{n_2} \dots D_N^{n_N}}
\end{equation}
where $\gamma_E \approx 0.5772$  is the Euler constant and $\mu_0$ is the dimensional regularisation scale. Here the factor $e^{L \epsilon \gamma_E}$ is purely conventional and it is chosen for later convenience, while the factor $\mu_0^{2L\epsilon}$ ensures that the integrals have integer mass dimension.
In general,  for a given process with $E$ independent external momenta and $L$ loops one needs $L(L+1)/2 + L E $ independent denominators to describe all possible scalar products of loop momenta with loop or external momenta.  
In our case $E=3$, and therefore we need $4$ denominators at one loop,  $9$ denominators at two loops and $15$ 
at three loops. A specific complete set of $\{D_i\}$ at a given loop order
is usually referred to as
an ``integral family''. 
Having in mind the calculation of the three-loop scattering amplitudes, it is useful to organise the relevant integrals 
in as few integral families as possible, up to permutation of the external momenta. 
While one family is sufficient at one loop, we need one planar family and one non-planar one at two loops 
and one
planar and two non-planar ones at three loops. 
We report them in Tabs\eqref{table:1}, \eqref{table:2} and \eqref{table:3} .
There, we indicate the loop momenta with $k_1$, $k_2$ and $k_3$.  We name PL the families corresponding to planar graphs and 
NPL,  NPL1,  NPL2 the ones corresponding to non-planar graphs. 
The different number of loop momenta for the planar topologies eliminates any ambiguity
in the naming convention.

\begin{table}
\centering
\begin{tabular}{ c || c  }
Family & PL \\
\hline\hline
$D_1 $& $k_1^2$  \\ 
$D_2 $& $(k_1 - p_1)^2$  \\  
$D_3 $& $(k_1 - p_1-p_2)^2$    \\
$D_4 $& $(k_1 - p_1-p_2-p_3)^2$  
\end{tabular}
\caption{Planar 1-loop integral family.}
\label{table:1}
\end{table}

\begin{table}
\centering
\begin{tabular}{ c || c | c }
Family & PL & NPL\\
\hline\hline
$D_1  $& $k_1^2$  &   $k_1^2$ \\ 
$D_2 $& $k_2^2$ &  $k_2^2$ \\  
$D_3 $& $(k_1 - k_2)^2$  &  $ (k_1-k_2)^2$  \\
$D_4 $& $(k_1 - p_1)^2$ &  $ (k_1-p_1)^2$ \\
$D_5 $& $(k_2 - p_1)^2$ &  $(k_2-p_1)^2$ \\
$D_6$& $(k_1 - p_1-p_2)^2$ &  $(k_1 - p_1-p_2)^2$ \\
$D_7 $& $(k_2 - p_1-p_2)^2$ &  $(k_1 - k_2+ p_3)^2$ \\
$D_8$ & $(k_1 - p_1-p_2-p_3)^2$ &  $(k_2 - p_1-p_2-p_3)^2$ \\
$D_9$ & $(k_2 - p_1-p_2-p_3)^2$ &  $(k_1-k_2-p_1-p_2)^2$ \\
\end{tabular}
\caption{Planar and non-planar 2-loop integral families.}
\label{table:2}
\end{table}


\begin{table}
\centering
\begin{tabular}{ c || c | c |c}
Family & PL & NPL1 & NPL2\\
\hline\hline
$D_1  $& $k_1^2$  & $k_1^2$ & $k_1^2$ \\ 
$D_2 $& $k_2^2$ & $k_1^2$ & $k_1^2$ \\  
$D_3 $& $k_3^2$ & $k_3^2$  & $k_3^2$    \\
$D_4 $& $(k_1 - p_1)^2$ & $(k_1 - p_1)^2$ & $(k_1 - p_1)^2$ \\
$D_5 $& $(k_2 - p_1)^2$ & $(k_2 - p_1)^2 $ & $(k_2 - p_1)^2$   \\
$D_6$& $(k_3 - p_1)^2$ & $(k_3 - p_1)^2$ & $(k_3 - p_1)^2$ \\
$D_7 $& $(k_1 - p_1-p_2)^2$ & $(k_1 - p_1-p_2)^2$ & $(k_1 - p_1-p_2)^2$ \\
$D_8$ & $(k_2 - p_1-p_2)^2$ & $(k_2 - p_1-p_2)^2$  & $(k_3 - p_1-p_2)^2$  \\
$D_9$ & $(k_3 - p_1-p_2)^2$ & $(k_3 - p_1-p_2)^2$ & $(k_1 - k_2)^2$ \\
$D_{10}$& $(k_1 - p_1-p_2-p_3)^2$ & $(k_1 - p_1-p_2-p_3)^2$ & $(k_2 - k_3)^2$ \\
$D_{11}$ & $(k_2 - p_1-p_2-p_3)^2$ & $(k_2 - p_1-p_2-p_3)^2$  & $(k_1 -k_2 -p_3)^2$  \\
$D_{12}$ & $(k_3 - p_1-p_2-p_3)^2$ & $(k_3 - p_1-p_2-p_3)^2$ & $(k_2 - k_3+p_1+p_2+p_3)^2$ \\
$D_{13}$ & $(k_1 - k_2)^2$ & $(k_1 - k_2)^2$ & $(k_2 + p_3)^2$\\
$D_{14}$ & $(k_1 - k_3)^2$ & $(k_2 - k_3)^2$  & $(k_1 - k_3)^2$  \\
$D_{15}$ & $(k_2 - k_3)^2$ & $(k_1 - k_2 + k_3)^2$ & $(k_2 - p_1 - p_2)^2$ \\
\end{tabular}
\caption{Planar and non-planar 3-loop integral families.}
\label{table:3}
\end{table}

As it is well known, the integrals in eq.~\eqref{integrals} are not
all independent and, instead, various types of relations can be
established among them, most notably by the algorithmic exploitation
of symmetry relations and integration-by-parts
identities~\cite{Tkachov:1981wb,Chetyrkin:1981qh}.  The latter in
particular allow one to reduce the number of independent integrals by
potentially several orders of magnitudes and to express the amplitude in terms of
a relatively small number of so-called master integrals.  While the
reduction to master integrals for $2 \to 2$ massless scattering up to
two loops can be performed very easily with automated tools, going one
order higher involves a considerable increase in complexity.  In
practice, at three loops we proceed as follows. Once the amplitude has
been expressed in terms of scalar integrals, we first use
\texttt{Reduze 2}~\cite{Studerus:2009ye,vonManteuffel:2012np} to find
trivially vanishing integrals, non-trivial symmetry relations among
the various integrals and corresponding ones obtained by permutations
of the external momenta.  This step already reduces the size of the
expressions significantly as well as the number of integrals within them.  We then use
\texttt{Finred}, an in-house implementation of the Laporta
algorithm~\cite{Laporta:2001dd} augmented by the use of finite-field
arithmetics~\cite{vonManteuffel:2014ixa,
  vonManteuffel:2016xki,Peraro:2016wsq,Peraro:2019svx} and
syzygy-based
techniques~\cite{Gluza:2010ws,Schabinger:2011dz,Ita:2015tya,Larsen:2015ped,Bohm:2017qme,Agarwal:2020dye},
in order to solve the system of integration-by-parts
identities satisfied by the remaining integrals.
This allows us to express the three-loop helicity
amplitudes for this process in terms of the basis of master integrals
computed in ref.~\cite{Henn:2020lye}. In this way we find that the
physical three-loop scattering amplitudes can be expressed in terms of
the same $486$ master integrals necessary for the calculation of
diphoton production at three loops~\cite{Caola:2020dfu}.
