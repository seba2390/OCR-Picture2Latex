% !TEX root = 0-qqQQmain.tex

\section{Crossed Channels and Equal Flavour Amplitudes} \label{extra_results}
In the previous sections, we presented the computation of the helicity amplitudes 
for the scattering of four quarks of different flavour as in eq.~\eqref{s_channel}.  
We discuss here how to use this result to derive the helicity amplitudes for all other 2-to-2 quark scattering processes, both for different and equal flavours.
We start by listing all of these processes.
\begin{itemize}
\item Different flavour quarks:
 \begin{align}
 & { q}(p_1) \;+ \;\bar { q}(p_2)  \; \longrightarrow \;  { \bar Q}(-p_3) \;+ \; {   Q}(-p_4) \; ,    \label{s_channel_new} \\[8pt]
&  { q}(p_1) \;+ \; { Q}(p_2)  \; \longrightarrow \;  {  q}(-p_3) \;+ \; {   Q}(-p_4) \; ,  \label{t_channel}   \\[8pt]
& { q}(p_1) \;+ \;\bar { Q}(p_2)  \; \longrightarrow \;  { \bar Q}(-p_3) \;+ \;{ q}(-p_4)  \; ,\label{u_channel} \\[8pt]
&  { \bar q}(p_1) \;+ \; { \bar Q}(p_2)  \; \longrightarrow \;  { \bar q}(-p_3) \;+ \; {  \bar  Q}(-p_4) \; .  \label{anti_t_channel} 
 \end{align}
 \item Equal flavour quarks:
\begin{align}
 & { q}(p_1) \;+ \;\bar { q}(p_2)  \; \longrightarrow \;  { \bar q}(-p_3) \;+ \; {   q}(-p_4) \; ,    \label{s_channel_equal} \\[8pt]
 & { q}(p_1) \;+ \; { q}(p_2)  \; \longrightarrow \;  {  q}(-p_3) \;+ \; {  q}(-p_4) \; ,  \label{t_channel_equal}  \\[8pt]
  & { \bar q}(p_1) \;+ \; { \bar q}(p_2)  \; \longrightarrow \;  { \bar  q}(-p_3) \;+ \; { \bar  q}(-p_4) \; .  \label{anti_t_channel_equal}  
\end{align}
\end{itemize}
Here, the first process \eqref{s_channel_new} is the one we have already computed,  
$i.e.$\ process \eqref{s_channel}, see eqs.~\eqref{H_ij} and~\eqref{H1H2def}.

In the following we will consider only two helicity amplitudes for each process, since,  as already discussed,  the other two can be obtained by a parity transformation\footnote{For these processes,
a parity transformation only
acts on the external spinors by the following transformation: $ \langle ij \rangle  \leftrightarrow [ji]$.} which flips the signs of helicities.
We start with the case $q \neq Q$.  From the helicity amplitudes \eqref{H_ij},  one can find those for  \eqref{t_channel} and \eqref{u_channel}  by performing appropriate crossings of particles from the initial to the final state.  This can be achieved by analytically continuing the results for \eqref{s_channel_new} to the appropriate kinematical region (see figure~\ref{KinematicsPath} for reference) and then renaming the momenta of the particles involved in the crossing to match the definitions of the processes above.  Amplitudes for the process \eqref{anti_t_channel} are then simply found by acting with a charge conjugation transformation\footnote{See the formulas below for the explicit action of charge conjugation on the helicity amplitudes. } on \eqref{t_channel}.
Details on how to perform the analytic continuation of the transcendental functions appearing in the scattering amplitudes are described for example in \cite{Anastasiou:2000mf}.  However,  for convenience of the reader,  we outline the method in Appendix \ref{sec:appC}.  
In the following we will abandon the colour-vector notation for helicity amplitudes as the crossing of particles also changes the indices of the colour basis \eqref{colour_structures}. Instead, we adopt the more conventional notation
\begin{equation}
\mathbfcal{H}_i =
\begin{pmatrix}
{\mathcal{H}}_i^{[1]}  \\
{\mathcal{H}}_i^{[2]} 
\end{pmatrix} 
\longrightarrow  \mathcal{H}_i = \mathcal{H}_i^{[1]} \mathcal{C}_1 +  \mathcal{H}_i^{[2]} \mathcal{C}_2.
\end{equation}
This way the crossing acts on the colour structure of the amplitude by simply exchanging indices in $\mathcal{C}_1$ and $\mathcal{C}_2$.
With this notation and following the procedure described above, we find\footnote{We recall that the notation $\bar{\mathcal{A}}$ is defined in eq.~\eqref{definition_A_bar}.}  for processes \eqref{t_channel} and \eqref{anti_t_channel}
\begin{align}
\bar{\mathcal{A}}_{++--}^{\scriptstyle qQ\rightarrow qQ} &=\bar{\mathcal{A}}_{++--}^{\scriptstyle \bar q\bar Q\rightarrow \bar q\bar Q} = \mathcal{H}_1|_{p_2 \leftrightarrow p_3}  \;  \frac{ \langle 12 \rangle}{\langle 34 \rangle},   \\[8pt]
\bar{\mathcal{A}}_{+--+}^{\scriptstyle qQ\rightarrow qQ} &= \bar{\mathcal{A}}_{+--+}^{\scriptstyle \bar q\bar Q\rightarrow \bar q\bar Q} =\mathcal{H}_2|_{p_2 \leftrightarrow p_3}  \; \frac{ \langle 14 \rangle}{\langle 32 \rangle}\,, \label{anti_t_channel_2}
\end{align}
and for process \eqref{u_channel} 
\begin{align}
\bar{\mathcal{A}}_{+-+-}^{\scriptstyle q\bar Q\rightarrow \bar Q q } &= \mathcal{H}_1|_{p_2 \leftrightarrow p_4}  \;  \frac{ \langle 13 \rangle}{\langle 42 \rangle} ,   \\ \bar{\mathcal{A}}_{++--}^{\scriptstyle q \bar Q\rightarrow \bar Q q } &= \mathcal{H}_2|_{p_2 \leftrightarrow p_4}  \;  \frac{ \langle 12 \rangle}{\langle 43 \rangle} \;.
\end{align}
We now turn to the processes with $q=Q$. 
To obtain the amplitude for \eqref{s_channel_equal}, we can use the fact that the Feynman diagrams contributing to this process are exactly the sum of the ones of processes \eqref{s_channel_new} and \eqref{u_channel}.  
Accounting for a relative minus sign between the two contributions to the amplitude due to the exchange of two identical fermions, we write, similarly to \cite{Glover:2004si},
\begin{equation}\label{equal_quarks_relation}
\bar{\mathcal{A}}_{\lambda_1\lambda_2\lambda_3\lambda_4}^{\scriptstyle q\bar q \rightarrow \bar qq} = \bar{\mathcal{A}}_{\lambda_1\lambda_2\lambda_3\lambda_4}^{\scriptstyle q\bar q \rightarrow \bar QQ}  - \bar{\mathcal{A}}_{\lambda_1\lambda_2\lambda_3\lambda_4}^{\scriptstyle q\bar Q \rightarrow \bar Q q}\,.
\end{equation}
For the two independent helicity choices we obtain
\begin{align}
&\bar{\mathcal{A}}_{+-+-}^{\scriptstyle q\bar q \rightarrow \bar  q q} = \Big[ \mathcal{H}_1 - \mathcal{H}_1|_{p_2 \leftrightarrow p_4} \Big] \; \frac{ \langle 13 \rangle}{\langle 24 \rangle},
 \label{s_equal_quarks_1}\\[8pt]
&\bar{\mathcal{A}}_{+--+}^{\scriptstyle q\bar q \rightarrow \bar  qq } = \mathcal{H}_2 \; \frac{ \langle 14 \rangle}{\langle 23 \rangle} ,\label{s_equal_quarks_2}
\end{align}
where in the second equation we used $ \bar{\mathcal{A}}_{+--+}^{\scriptstyle q\bar Q \rightarrow \bar Q q}  = 0$.
In the same way, we can find the amplitudes for processes \eqref{t_channel_equal} and \eqref{anti_t_channel_equal} as a sum of those for process \eqref{t_channel} and the ones obtained by crossing $p_3 \leftrightarrow p_4$ in \eqref{t_channel}. 
Explicitly, we obtain
\begin{equation}\label{equal_quarks_relation_2}
\bar{\mathcal{A}}_{\lambda_1\lambda_2\lambda_3\lambda_4}^{\scriptstyle q q \rightarrow q q} = \bar{\mathcal{A}}_{\lambda_1\lambda_2\lambda_3\lambda_4}^{\scriptstyle \bar q \bar q \rightarrow \bar q \bar q} = \bar{\mathcal{A}}_{\lambda_1\lambda_2\lambda_3\lambda_4}^{\scriptstyle qQ\rightarrow qQ}  - \bar{\mathcal{A}} ^{\scriptstyle qQ\rightarrow qQ}_{{\lambda_1\lambda_2\lambda_4\lambda_3}}\big|_{p_3 \leftrightarrow p_4},
\end{equation}
where we stress the exchange of helicity labels in the second term of the r.h.s.
This implies
\begin{align}
\bar{\mathcal{A}}_{++--}^{\scriptstyle q q \rightarrow q q} &=\bar{\mathcal{A}}_{++--}^{\scriptstyle \bar q \bar q \rightarrow \bar q \bar q} = \Big[ \mathcal{H}_1|_{p_2 \leftrightarrow p_3}  + \mathcal{H}_1|_{p_2 \rightarrow p_4 \rightarrow p_3 \rightarrow p_2}   \Big]  \; \frac{ \langle 12 \rangle}{\langle 34 \rangle}  \;,  \label{t_equal_quarks_1}\\[8pt]
\bar{\mathcal{A}}_{+--+}^{\scriptstyle qq \rightarrow q q} &=\bar{\mathcal{A}}_{+--+}^{\scriptstyle \bar q\bar q \rightarrow \bar q \bar q} = \mathcal{H}_2|_{p_2 \leftrightarrow p_3} \; \frac{ \langle 14 \rangle}{\langle 32 \rangle} , \label{t_equal_quarks_2}
\end{align}
where we have used that $\bar{\mathcal{A}}^{\scriptstyle qQ\rightarrow qQ}_{{+-+-}}=0$.\\

We point out that the formulas of this section are valid for the bare amplitudes as well as for the UV renormalised and the finite remainders after IR subtraction.
We thus arrive at the following representation for the finite remainders of the full helicity amplitudes:
\begin{align}
    \mathcal{A}^{\text{process}}_{\lambda_1\lambda_2\lambda_3\lambda_4\,\text{fin}}
    &= 4 \pi \alpha_s \Phi^{\text{process}}_{\lambda_1\lambda_2\lambda_3\lambda_4}
    \sum_{L=0}^3 \left(\frac{\alpha_s}{4\pi}\right)^L \mathcal{H}^{\text{process}\;(L)}_{\lambda_1\lambda_2\lambda_3\lambda_4,\,\text{fin}} 
    + \mathcal{O}(\alpha_s^5)\,
\end{align}
where $\Phi^{\text{process}}_{\lambda_1\lambda_2\lambda_3\lambda_4}$ are the corresponding spinor phases, see e.g.\ eq.~\eqref{H_ij}.
We list the spinor phases and provide explicit analytical results for the form factors $\mathcal{H}^{\text{process}\;(L)}_{\lambda_1\lambda_2\lambda_3\lambda_4,\,\text{fin}}$ in the ancillary files.
