% !TEX root = 0-qqQQmain.tex

\section{Conclusions}
\label{conclusions}
In this paper we presented the analytic calculation of the three-loop
helicity amplitudes for the process $q \bar q
\rightarrow Q \bar{Q}$ and all other related $2 \to 2$ quark
scattering processes in massless QCD.  Our results were obtained
through a tensor decomposition of the scattering amplitude in the 't
Hooft Veltman scheme, which allowed us to work with a minimal number
of independent basis tensors, exactly matching the number of
independent helicity amplitudes.  This method is particularly convenient when
applied to processes with multiple external fermion lines, where the
number of basis tensors in conventional dimensional regularisation
is known to increase with the number of
loops.
We employed modern integration-by-parts reductions based on finite field
arithmetic and syzygy techniques to perform the demanding mapping
of the three-loop integrals to a basis of master integrals.
For the master integrals, analytical solutions in terms of harmonic polylogarithms
were already available~\cite{Henn:2020lye}. 
We point out that the finite remainders
at three loops are remarkably compact with file sizes 
of $\sim 1$ MB for each partonic channel. 
We provide the analytical results as ancillary files attached to the 
\texttt{arXiv} submission of this manuscript.

This is the first three-loop calculation in full QCD for a process
that involves the scattering of four coloured particles.  In
particular, our results confirm, for the first time, the structure of the colour
quadrupole contribution to the IR poles of QCD scattering
amplitudes with non-trivial colour flow.  We performed various checks
on our results, both comparing them with previous analytical
calculations at one and two loops, and numerically with
\texttt{OpenLoops 2}.  The first test highlighted a
subtlety in the definition of the \emph{bare} helicity amplitudes 
in 't Hooft-Veltman scheme for processes
which involve the scattering of more than one pair of fermions.
In this case, using the standard projectors defined in
conventional dimensional regularisation, different results 
can be obtained depending on whether the 
$\gamma$ algebra to fix the helicity amplitudes is performed 
in $4$ or in $d=4-2 \epsilon$ dimensions. 
While no ambiguity arises at the level of finite remainders in four dimensions,
extra care should be paid when comparing divergent results.

The four-quark scattering processes presented here are arguably the least
complex among the ones involving four coloured particles. However, the
calculation of the three-loop amplitudes for $q \bar q \rightarrow gg$, $gg \rightarrow gg$ and the processes related by
crossings require no new concepts and can be performed following the steps described in this paper.
