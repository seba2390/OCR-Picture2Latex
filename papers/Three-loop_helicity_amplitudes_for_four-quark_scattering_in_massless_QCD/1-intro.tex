% !TEX root = 0-qqQQmain.tex

\section{Introduction}


The success of the collider particle physics program, whose main
player today is the Large Hadron Collider (LHC) at CERN, relies
heavily on our ability to model with high precision and accuracy the
scattering of high energetic protons in Quantum Chromodynamics (QCD).
Thanks to asymptotic freedom and the factorization properties of QCD,
this intrinsically non-perturbative problem can be treated with
perturbative methods, supplemented by non-perturbative information about
the distribution of partons in the proton. Within this picture, an important
role is played by higher order perturbative QCD calculations, which allow
for a reliable and precise description of a wide range of collider processes
and observables.

Thanks to a concerted effort in the high-energy community over the
last few years, it is currently possible to compute predictions for
many interesting reactions to second order in the strong coupling
expansion, $i.e.$ to what is usually referred to as
next-to-next-to-leading order (NNLO). 
This has required, on the one hand, 
major advances in computational techniques
for multi-loop scattering amplitudes~\cite{Tkachov:1981wb,Chetyrkin:1981qh,Hodges:2009hk,Gluza:2010ws,Ita:2015tya,Larsen:2015ped,Bohm:2017qme,Badger:2016uuq,vonManteuffel:2014ixa,Peraro:2016wsq,Peraro:2019svx,Guan:2019bcx,Pak:2011xt,Abreu:2019odu,Heller:2021qkz,Kotikov:1990kg,Bern:1993kr,Remiddi:1997ny,Gehrmann:1999as,Papadopoulos:2014lla,Dixon:1996wi,Henn:2013pwa,Primo:2016ebd,Goncharov,Remiddi:1999ew,Goncharov:2001iea,Goncharov:2010jf,Brown:2008um,Ablinger:2013cf,Panzer:2014caa,Duhr:2011zq,Duhr:2012fh,Duhr:2019tlz},
which, notably, have recently made it possible to compute
various $2 \to 3$ processes up to two loops in QCD~\cite{Badger:2017jhb,Abreu:2017hqn,Abreu:2018aqd,Abreu:2018zmy,Abreu:2018jgq,Abreu:2019rpt,Abreu:2020cwb,Chicherin:2018yne,Chicherin:2019xeg,Chawdhry:2020for,DeLaurentis:2020qle,Chawdhry:2018awn,Abreu:2020xvt,Agarwal:2021grm,Badger:2021nhg,Abreu:2021fuk,Agarwal:2021vdh,Chawdhry:2021mkw,Badger:2021imn,Gehrmann:2015bfy,Papadopoulos:2015jft,Gehrmann:2018yef,Chicherin:2018mue,Chicherin:2020oor}.
On the other hand, the use of these amplitudes to perform phenomenological
studies for the relevant processes at NNLO~\cite{Chawdhry:2019bji,Kallweit:2020gcp,Chawdhry:2021hkp,Czakon:2021mjy} has required the
development of so-called subtraction or slicing frameworks~\cite{GehrmannDeRidder:2005cm,Czakon:2010td,Caola:2017dug,Magnea:2018hab,Herzog:2018ily,DelDuca:2016ily,Cacciari:2015jma,Catani:2007vq,Gaunt:2015pea,Boughezal:2015dva}
to properly deal with the intricate IR divergences that appear in QCD reactions. 

Beyond NNLO, predictions at third order in the
perturbative couplings, i.e.\ at N$^3$LO, are
known only for a handful of important LHC processes~\cite{Anastasiou:2015vya,Duhr:2019kwi,
Dulat:2018bfe,Mistlberger:2018etf,Dreyer:2016oyx,Dreyer:2018qbw,
Billis:2021ecs,Chen:2021isd,Chen:2021vtu}. 
In particular, N$^3$LO results are currently available only for reactions that
require at most three-point three-loop integrals.
Given the remarkable success of the
experimental program at the LHC, it is desirable to extend these
calculations to more complex processes. A particularly interesting
one is di-jet production.  In fact, jets are ubiquitous at
hadron colliders, so understanding their dynamics is of great
interest.
Moreover, di-jet production is the first massless $2\to2$
process that has a non-trivial colour structure. This makes it an
ideal ground for studying the structure of perturbative QCD. For
example, it is by now well known that when four or more coloured
partons interact, starting at the three-loop order, non-trivial
colour correlations can affect the pattern of IR divergences,
generating new structures~\cite{Almelid:2015jia} beyond the standard dipole
formula~\cite{Sterman:2002qn,Aybat:2006wq,Aybat:2006mz,Becher:2009cu,Gardi:2009qi,Becher:2009qa,Dixon:2009gx}. Also, the
non-trivial colour structure may create subtle violations of the
factorization framework that is at the very core of theoretical
predictions at hadronic
colliders~\cite{Catani:2011st,Forshaw:2012bi,Forshaw:2006fk,Becher:2021zkk}.
This makes jet production at hadronic colliders an extremely interesting
process to investigate at higher orders. 

A key ingredient for the study of jet production at N$^3$LO is
provided by the virtual three-loop corrections to the scattering
amplitudes for the production of two jets in massless QCD.  Modulo
crossings, there are three main partonic channels that need to be
computed: four-gluon scattering, the scattering of two quarks and two
gluons, and the scattering of four quarks.  All ingredients necessary
for the calculation of the two-loop QCD corrections to these processes have
been known for a long time~\cite{Smirnov:1999gc,Tausk:1999vh,Glover:2001af,Anastasiou:2002zn,Glover:2003cm}, which 
have made it possible to compute the relevant scattering amplitudes~\cite{Anastasiou:2000kg,Bern:2003ck,Glover:2004si,DeFreitas:2004kmi}. Also, in view of extending these
calculations to three loops, results for the two loop helicity
amplitudes up to order $\epsilon^2$ have been
obtained~\cite{Ahmed:2019qtg}.  For what concerns the three loop
results, instead, the relevant master integrals have been computed in
ref.~\cite{Henn:2020lye}, and have then been used to obtain the first
three loop results for $2 \to 2$ scattering amplitudes in
supersymmetric theories~\cite{Henn:2016jdu,Henn:2019rgj}.  More
recently also the first three loop corrections to the production of
two photons in full QCD have been obtained~\cite{Caola:2020dfu}.

In this paper, we move one step further and consider one of the three classes of partonic processes  listed above, 
namely the scattering of four massless quarks. This particular process is interesting not only
because it allows us, for the first time, to check the full structure of IR divergences at three loops in QCD, 
but also because it involves two external spinor structures. In fact, this property makes the use of the standard form factors
method for the calculation of the helicity amplitudes particularly cumbersome, due to the fact that the  $\gamma$-algebra does not
close in  $d$ space-time dimensions. For the calculation of the helicity amplitudes we then make use of
a different approach, recently described in refs~\cite{Peraro:2019cjj,Peraro:2020sfm}, which allows us to calculate the helicity amplitudes
in a simpler way, corresponding to the 't Hooft-Veltman scheme (tHV)~\cite{tHooft:1972tcz} for the processes considered here.\footnote{See also ref.~\cite{Heller:2020owb} for an application of similar ideas in the case of a chiral theory, and ref.~\cite{Chen:2019wyb} for an alternative approach.} 
In doing this, we also expose some
subtleties in the usual approach to compute helicity amplitudes in 
tHV with the standard form factor method.
The rest of the paper is organised as follows. 
We start in section~\ref{Kinematics} by establishing the notation for the calculation of the
fundamental partonic channel $q\bar{q} \to Q \bar{Q}$, from which all other channels can by obtained by crossing.
We continue in section~\ref{The Scattering Amplitude},
where we describe the colour and tensor decomposition  of the scattering amplitude, and show how to 
efficiently compute the helicity amplitudes without considering evanescent Lorentz structures.
In section~\ref{computation} we provide details on our computational set up,
and in section~\ref{subtraction} we discuss the renormalisation and the infrared structure of the three-loop scattering amplitudes.
In section~\ref{results} we discuss our final results for the main partonic channel.
In section~\ref{extra_results} we then explain how to obtain all other partonic channels from our calculation, both for quarks of equal and different flavour.
Finally we conclude in section~\ref{conclusions}.  In 
appendices~\ref{sec:appB} and~\ref{sec:appA}
we provide some details on the structure of infrared divergences up at three loops, in particular focusing on the explicit derivation of the quadrupole terms, which appear for the first time with the scattering of at least four coloured partons at three loops.
In appendix~\ref{sec:appC}, we review the analytical continuation of the amplitudes to different regions of phase space.
