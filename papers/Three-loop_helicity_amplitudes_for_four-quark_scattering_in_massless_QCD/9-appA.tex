% !TEX root = 0-qqQQmain.tex

\section{Details on the IR Structure}
\label{sec:appB}
In this appendix we provide details on the calculation of the IR subtraction operators given in eqs. \eqref{I1}, \eqref{I2} and \eqref{I3}.  We recall that one can write the finite remainder of the scattering amplitude by means of a multiplicative colour-space operator $\mathbfcal{Z}$ 
\begin{equation}\label{Zeta_A_B}
\mathbfcal{A}_\text{fin} (\{p\},\mu)= \lim_{\epsilon \rightarrow 0} \mathbfcal{Z}^{-1}(\epsilon,\{p\},\mu) \; \mathbfcal{A}_{\text{ren}}(\epsilon,\{p\}) \; ,
\end{equation}
which we can apply similarly to the form factors in eq.~\eqref{hel_ampls_ren}: 
\begin{equation}\label{Zeta_H_B}
\mathbfcal{H}_{i,\:\text{fin}} (\epsilon,\{p\})= \lim_{\epsilon \rightarrow 0} \mathbfcal{Z}^{-1}(\epsilon,\{p\},\mu) \; \mathbfcal{H}_{i,\:\text{ren}}(\epsilon,\{p\}) \; .
\end{equation}

For the case of four-quark scattering considered in this paper, the
$\mathbfcal{Z}$ colour operator can be represented as a 2-by-2 matrix,
which mixes the colour structures defined in
eq.~\eqref{colour_structures}.
It is possible to write the $\mathbfcal Z$ operator
as~\cite{Becher:2009qa,Becher:2009cu}
\begin{equation}\label{exponentiation_B}
\mathbfcal{Z} (\epsilon,\{p\},\mu) = \mathbb{P} \exp \left[
  \int_\mu^\infty \frac{\mathrm{d} \mu'}{\mu'}
  \mathbf{\Gamma}(\{p\},\mu')\right] \; ,
\end{equation}
where $\mathbb{P}$ is the \textit{path-ordering} symbol, {\it i.e.}\
operators are ordered from left to right with decreasing values of $\mu'$.
To proceed, it is  useful to decompose the 
anomalous dimension operator according to
\begin{equation}\label{anomalous_operator_B}
\mathbf{\Gamma}(\{p\},\mu) =  \mathbf{\Gamma}_{\text{dipole}}(\{p\},\mu)  + \mathbf{\Delta}_4 (\{p\}) 
\end{equation}
into a dipole and a quadrupole contribution.

The dipole matrix was given in eq.~\eqref{dipole}. We repeat its form here for convenience
\begin{equation}\label{dipole_B}
\mathbf{\Gamma}_{\text{dipole}}(\{p\},\mu)  =  \sum_{1\leq i < j \leq 4} \mathbf{T}^a_i \; \mathbf{T}^a_j  \; \gamma^\text{cusp}(\as) \; \log\left(\frac{\mu^2}{-s_{ij}-i\delta}\right)  \; + \; \sum_{i=1}^4 \; \gamma^i(\as) \; ,
\end{equation}
where $\as=\as(\mu)$, $\gamma^\text{cusp}(\as)$ is the cusp anomalous
dimension and $\gamma^i$ the anomalous dimension of the $i$-th
external particle, which only depends on whether $i$ is a quark or a
gluon.  The perturbative expansion for these quantities are given up
to order $\mathcal{O}(\as^3)$ in Appendix \ref{sec:appA}.
In eq.~\eqref{dipole_B}, the $\mathbf T^a_i$ colour operators are
related to the
$SU(N_c)$ generator associated with the external particle $i$.
Specifically, following the conventions of \cite{Catani:1996vz} they are defined as
\begin{alignat}{2}
(\mathbf{T}^a_i)_{b_i c_i} &=    \;- i {f ^a}_{b_i c_i} &~&\text{for a gluon},\\
(\mathbf{T}^a_i)_{i_i j_i} &=  \; + T^a_{i_i j_i} &&\text{for a final(initial) state quark(anti-quark)},\\
(\mathbf{T}^a_i)_{i_i j_i} &=  \; - T^a_{j_i i_i} &&\text{for a initial(final) state quark(anti-quark)}.
\end{alignat}
We now consider how \eqref{dipole_B} can be written for our process \eqref{s_channel} as a matrix in colour space. The operator $\mathbf{T}^a_1 \; \mathbf{T}^a_2$ acts on our colour space basis tensors as follows 
\begin{align}
 \left( \mathbf{T}^a_1 \; \mathbf{T}^a_2 \;  \mathcal{C}_1 \right)_{i_1 i_2 i_3 i_4} &= - T^a_{j_1 i_1} T^a_{i_2 j_2} \; (\delta_{ j_1 i_4} \delta_{j_2 i_3})=- \; \frac{1}{2} \left( \delta_{ i_1 i_2} \delta_{i_3 i_4 } - \frac{1}{N_c} \delta_{ i_1 i_4} \delta_{i_2 i_3}  \right), \\[10pt]
\left( \mathbf{T}^a_1 \; \mathbf{T}^a_2 \; \mathcal{C}_2  \right)_{i_1 i_2 i_3 i_4} &= - T^a_{j_1 i_1} T^a_{i_2 j_2} \; (\delta_{ j_1 j_2} \delta_{i_3 i_4 })= -C_F  \;  \delta_{ i_1 i_2} \delta_{i_3 i_4 }.
\end{align}
We change to component notation with respect to our $\mathcal{C}_i$ basis by identifying
\begin{equation}\label{ref1_start}
v_1 \;  \delta_{ i_1 i_4} \delta_{i_2 i_3} + v_2 \; \delta_{ i_1 i_2} \delta_{i_3 i_4 } \rightarrow \begin{pmatrix}
v_1 \\
v_2
\end{pmatrix}
\end{equation}
and obtain the matrix representation
\begin{equation}
\mathbf{T}^a_1 \; \mathbf{T}^a_2 = \begin{pmatrix}
\frac{1}{2N_c} & 0 \\
-\frac{1}{2} & -C_F 
\end{pmatrix} \; .
\end{equation}
Similarly, we find for the first sum in eq.~\eqref{dipole_B} the matrix representation
\begin{equation}\label{dipole_matrix_form_B}
  \sum_{1\leq i <  j \leq 4} \mathbf{T}^a_i \; \mathbf{T}^a_j
  \; \gamma^\text{cusp}(\as)\;  \log\left(\frac{\mu^2}{-s_{ij}-i\delta}\right) = \gamma^\text{cusp} (\as) \begin{pmatrix}
A(s,u) & B(s,u) \\[8pt]
B(u,s) & A(u,s)
\end{pmatrix} \; ,
\end{equation}  
with 
\begin{align}
A(s,u)&=-2\left\{C_F \log\left(\frac{\mu^2}{-u-i\delta}\right)+\frac{1}{2N_c}\left[ \log\left(\frac{\mu^2}{-t-i\delta}\right) - \log\left(\frac{\mu^2}{-s-i\delta}\right) \right] \right\} \: , \label{A_B}\\[10pt]
B(s,u)&= -\left[ \log\left(\frac{\mu^2}{-u-i\delta}\right) - \log\left(\frac{\mu^2}{-t-i\delta}\right) \right] \; , \label{ref1_end}
\end{align}
where $t=-s-u$ is implied.
In these equations, a small positive imaginary part is associated with the Mandelstam invariants to fix branch cut ambiguities.

The dipole anomalous matrix that we have just discussed is sufficient
to reproduce the IR poles of our scattering amplitude up to two
loops.  Starting at three loops, however, new four-parton colour
correlations contained in the matrix $\mathbf{\Delta}_4$ spoil the
simple dipole picture.  To compute $\mathbf{\Delta}_4$ explicitly,
we start by writing
\begin{equation}
\mathbf{\Delta}_4 (\{p\}) = \sum_{L=3}^\infty \left(\frac{\as}{4 \pi}\right)^L \; \mathbf{\Delta}^{(L)}_4 (\{p\}) \; ,  \label{delta4_B}
\end{equation}
such that, at the three-loop level, we are only interested in the first term of the expansion.
%
By analytically continuing eq.~(7) of Ref.~\cite{Almelid:2015jia} to the kinematical region with $s>0$, $t,u < 0$,  we find
\begin{align} \label{eq:quadrupole}
\mathbf{\Delta}^{(3)}_4 &= \;  128 \; f_{abe} \: f_{cde} \;  \left[ \mathbf{T}^a_1\:\mathbf{T}^c_2\:\mathbf{T}^b_3\:\mathbf{T}^d_4\: D_1(x) - \mathbf{T}^a_4\:\mathbf{T}^b_1\:\mathbf{T}^c_2\:\mathbf{T}^d_3\: D_2(x) \right] \nonumber\\
&\quad - 16 \; C \; f_{abe} \: f_{cde}\;  \sum_{i=1}^4 \sum_{\substack{1\leq j < k \leq4 \\ j,k\neq i}}  \left\{ \mathbf{T}^a_i,\mathbf{T}^d_i \right\} \; \mathbf{T}^b_j \; \mathbf{T}^c_k \; ,
\end{align}
with $x=-t/s$ given in eq.~\eqref{variables}, the constant $C$ in eq.~\eqref{C}, and the functions $D_1$, $D_2$ in eqs.~\eqref{D1}, \eqref{D2}. We note that all of the four partons of the kinematically dependent parts of eq.~\eqref{eq:quadrupole} are correlated through colour, while the kinematically independent part has a three parton colour correlation.
We also note that, at this level, the quadrupole corrections to the {\it{anomalous dimension}} matrix can be generated solely by gluon interactions, and hence it is expected to be universal for all gauge theories involving the gluon in the particle content. Two of the representative diagrams that give rise to the quadrupole corrections are those in panels (e) and (f) of figure~\ref{diagrams}. 
We can highlight contributions to the quadrupole divergence, and in particular to the colour structures in the first and second line of eq.~\eqref{eq:quadrupole}, by drawing the representative diagrams in figure~\ref{diagrams_red},
where the black lines represent a tree level diagram and the red lines the gluons responsible for the soft quadrupole corrections.
%
\begin{figure}
\centering
\subfigure[]{\includegraphics[width=0.35\textwidth]{fig/Quadrupole4Ts}}\label{fig:2a}
\subfigure[] {\includegraphics[width=0.35\textwidth]{fig/Quadrupole3Ts}}\label{fig:2b}
\caption{Diagrams contributing to the quadrupole IR divergence, reinterpreted as tree level diagrams plus virtual gluons. Diagrams (a) and (b) give a contribution to the colour structures in the first and second line of eq.~\eqref{eq:quadrupole}, respectively.} \label{diagrams_red}
\end{figure}
%
We stress here that in reference~\cite{Almelid:2015jia}
the authors give results for $\mathbf{\Delta}^{(3)}_4$ prior to imposing momentum conservation.
In this formulation, there exists a kinematical region where $\mathbf{\Delta}^{(3)}_4$ 
is purely real, which corresponds to $s_{ij}<0$ for all $i,j$.  
Consequently, we first analytically continue to the region in which $s= s_{12} > 0$ and then impose momentum conservation.

Just like for the dipole case
it is useful to represent the quadrupole operator as a matrix acting 
in colour space.  This is immediate to do for the first line of $\mathbf{\Delta}_4^{(3)}$ in eq.~\eqref{eq:quadrupole},
since each operator $\mathbf{T}$ that contributes to it acts on a different parton 
and thus commutes with all  other colour operators.
More care is required instead to manipulate the second line of 
eq.~\eqref{eq:quadrupole} since it contains operators acting on the same particle: 
this means that the ordering of the $\mathbf{T}_i$ is relevant.  
Here, the computation can be performed following again the procedure employed for the dipole case, see 
eqs.~\eqref{ref1_start}---\eqref{ref1_end}. The
result is presented in eq.~\eqref{delta_4_main}.

With the dipole and quadrupole contributions at hand,
it is easy to compute the $\mathbfcal{Z}$ opperator using eq.~\eqref{exponentiation_B} and to perform the exponentiation as a series expansion in $\as$.
To do so, we follow ref.~\cite{Becher:2009qa} and
introduce the short-hand
\begin{equation}
  \Gamma' = \frac{\partial}{\partial \log \mu} \mathbf{\Gamma}_{\text{dipole}} =
  \sum_{1\leq i < j \leq 4} \mathbf{T}^a_i \; \mathbf{T}^a_j\,
  \gamma^\text{cusp}(\as) = - 4 C_F\, \gamma^\text{cusp}(\as),
\end{equation}
and the $\as$ expansions
\begin{equation}
\mathbf{\Gamma}_{\text{dipole}} = \sum_{n=0}^\infty \mathbf{\Gamma}_n \;  \left( \frac{\as}{4 \pi} \right)^{n+1}, \quad  \quad \Gamma '=\sum_{n=0}^\infty \Gamma'_n \;  \left( \frac{\as}{4 \pi} \right)^{n+1}\; .
\end{equation}
In terms of these quantities, it is easy to find
\begin{align}\label{logZ_B}
&\log \mathbfcal{Z} = \frac{\as}{4 \pi}  \left[ \frac{\Gamma'_0}{4 \epsilon^2} + \frac{\mathbf{\Gamma}_0}{2 \epsilon}\right] \nonumber\\[8pt]
& + \left( \frac{\as}{4 \pi} \right)^2  \left[- \frac{3 \beta_0 \Gamma'_0}{16 \epsilon^3} + \frac{\Gamma'_1 - 4 \beta_0 \mathbf{\Gamma}_0}{16 \epsilon^2} + \frac{\mathbf{\Gamma}_1}{4 \epsilon}  \right] \\[8pt]
&+\left(\frac{\as}{4 \pi}\right)^3 \left[ \frac{11 \beta_0^2 \Gamma_0'}{72 \epsilon^4}  - \frac{5 \beta_0  \Gamma_1' + 8 \beta_1 \Gamma_0' - 12 \beta_0^2 \mathbf{\Gamma}_0}{72 \epsilon^3} + \frac{\Gamma_2' - 6 \beta_0 \mathbf{\Gamma}_1 - 6 \beta_1 \mathbf{\Gamma}_0}{36 \epsilon^2}  + \frac{\mathbf{\Gamma}_2 + \mathbf{\Delta}^{(3)}_4}{6 \epsilon}\right] \; .\nonumber
\end{align} 
From this result, it is straightforward to obtain the perturbative
expansion of $\mathbfcal Z$,
\begin{equation}
\mathbfcal{Z} = 1 + \left( \frac{\as}{4\pi} \right)
\mathbfcal{Z}_1+ \left( \frac{\as}{4\pi} \right)^2 \mathbfcal{Z}_2 +
\left( \frac{\as}{4\pi} \right)^3 \mathbfcal{Z}_3 + \mathcal O(\as^4)\;,
\end{equation}
and of its inverse $\mathbfcal Z^{-1}$,
\begin{equation}\label{Zinverse_B}
\mathbfcal{Z}^{-1} = 1 - \left( \frac{\as}{4\pi}  \right) \mathbfcal{I}_1- \left( \frac{\as}{4\pi}  \right)^2 \mathbfcal{I}_2 - \left( \frac{\as}{4\pi}  \right)^3 \mathbfcal{I}_3 + \mathcal O(\as^4)\;.
\end{equation}
The explicit form of the $\mathbfcal Z_i$ and $\mathbfcal I_1$ coefficients
are presented in eqs.~(\ref{I1}-\ref{Z3}).
