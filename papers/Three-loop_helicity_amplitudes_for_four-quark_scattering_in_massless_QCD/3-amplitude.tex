% !TEX root = 0-qqQQmain.tex

\section{The Scattering Amplitude}\label{The Scattering Amplitude}

\subsection{Lorentz Decomposition} 

We compute the scattering amplitude for the process \eqref{s_channel}
up to three loops in QCD, i.e.\ up
to order $\calO(\asb^4)$ , where $\asb$ is the bare strong coupling constant.  
We find it convenient to write the amplitude as 
\begin{equation}\label{definition_A_bar}
\tensor{\mathcal{A}}{_{i_1} _{i_2} _{i_3} _{i_4}} (q_1\bar q_2 \rightarrow \bar Q_3  Q_4)  =  ( 4 \pi \asb ) \;  \tensor{\bar{\mathcal{A}}}{_{i_1} _{i_2} _{i_3} _{i_4}}  \; ,
\end{equation}
where we have made the external quarks colour indices $i_j$ explicit.
The colour space for our process can be spanned
using the two tensor structures\footnote{We point out that $\mathcal C_1$ and
  $\mathcal C_2$ are not orthogonal.}
\begin{equation}\label{colour_structures}
\mathcal{C}_1 = {\delta}_{ i_1 i_4} {\delta}_{ i_2 i_3}
\;, \quad\quad  \mathcal{C}_2 = {\delta}_{ i_1 i_2} {\delta}_{ i_3 i_4}\;.
\end{equation}
By decomposing $\bar{\mathcal{A}}_{i_1i_2i_3i_4} $ with respect to the
$\mathcal{C}_j$, $j=1,2$ basis, we can write it as a vector in
colour space:
\begin{equation}
\tensor{\bar{\mathcal{A}}}{_{i_1} _{i_2} _{i_3} _{i_4}}  =  \bar{\mathcal{A}}^{[1]} \: \mathcal{C}_1  +  \bar{\mathcal{A}}^{[2]} \: \mathcal{C}_2        \quad \longrightarrow \quad \bar{\mathbfcal{A}}  = \begin{pmatrix}
\bar{\mathcal{A}}^{[1]} \\
\bar{\mathcal{A}}^{[2]}
\end{pmatrix} \; ,
\end{equation}
where we indicate all colour-space vectors in boldface.
It is important to notice that the components of the colour vector $\bar{\mathbfcal{A}}$ are independently gauge invariant,  since a gauge transformation cannot mix different colour contributions to the amplitude.  From now on we will work in the vector notation defined above, unless explicitly stated.\\

Turning to the spin structure of the process,  
we further decompose the scattering amplitude in terms of 
in terms of a basis of Lorentz structures (``tensors'') $T_i$
\begin{equation}\label{decomp_tensors}
\bar{\mathbfcal{A}}  = \sum_{i=1}^{N_L}  \mathbfcal{F}_i \;  T_i \;,
\end{equation}
where the $ \mathbfcal{F}_i$ are scalar \textit{form factors} that
only depends on the Mandelstam invariants and
$N_L$ is the number of elements of the basis of Lorentz structures.
In our calculation, we employ dimensional regularization to deal with
ultraviolet (UV) and infrared (IR) divergences. This makes the
decomposition eq.~\eqref{decomp_tensors} subtle.  Indeed,
if one
works in Conventional Dimensional Regularisation (CDR), one finds 
that, since the $\gamma$-algebra in $d$
dimensions does not close,
the number $N_L$ of independent structures depends on the loop
order~\cite{Glover:2004si}.  However, since we are ultimately
interested in computing helicity amplitudes in four dimensions, we
find it convenient to work in a scheme, where we can 
ignore evanescent Lorentz structures right from the start.
In this approach, it is possible to show
that the number of Lorentz structures which are physically
relevant is the same at any number of loops ($N_L = N$), and it
equals the number of independent helicity
amplitudes~\cite{Peraro:2019cjj,Peraro:2020sfm}.
Specifically, we consider all internal momenta and polarizations in $d$ dimensions, but restrict momenta and polarizations of the external quarks
to a $4$-dimensional subspace.
A convenient choice for the two independent Lorentz structures describing our process is~\cite{Peraro:2020sfm}:
\begin{align} \label{tensors}
T_1 &= {\bar u} (p_2)\: \gamma_\alpha \: { u} (p_1) \times {\bar u} (p_4) \: \gamma^\alpha \:{ u}(p_3) \; ,
% \\[8pt] 
\quad
T_2 = {\bar u} (p_2)\: \slashed p_3 \: { u} (p_1) \times   {\bar u} (p_4) \: \slashed p_2 \:{ u}(p_3) \;.
\end{align} 
These two Lorentz structures are then sufficient at any loop order.

In order to isolate the form factors from the rest of the amplitude, we define tensor projectors $P_i $ satisfying
\begin{equation}
P_i \cdot T_j = \delta_{ij} \; ,
\end{equation}  
where the dot products indicates the sum over the polarisations of the external quarks, $P_i \cdot T_j = \sum_{\rm pol} P_i T_j$.  
It then follows from eq.~\eqref{decomp_tensors}  that
$
 P_i \cdot \bar{\mathbfcal{A}}  = \mathbfcal{F}_i .
$ 
 For the choice eq.~\eqref{tensors}, the explicit form of the projectors is
 \begin{equation}
   \begin{split}
     P_1 &= \frac{1}{(d-3)4s^2} \; T_1^\dagger +  \frac{t-u}{(d-3)s^2tu}\;  T_2^\dagger \; ,\\
     P_2 &= \frac{t-u}{(d-3)s^2tu} \; T_1^\dagger + \frac{(d-4) s^2+ 2t^2 + 2 u^2}{(d-3) 4 s^2 t^2 u^2} \; T_2^\dagger \; .
   \end{split}
   \label{eq:p12}
 \end{equation}
We recall here that the main advantage of working with scalar form factors $\mathbfcal{F}_i $ 
is that, by construction, they only contain scalar integrals, 
because all the Lorentz tensor structure has been factorised out by the basis tensors ${T_i}$. 

 
\subsection{Helicity Amplitudes} 
Ultimately, we are interested in computing the helicity amplitudes 
$\mathbfcal{A}_{\lambda_1 \lambda_2 \lambda_3 \lambda_4}$, 
for the process in eq.~  \eqref{s_channel},
where we indicate with $\lambda_j$ the helicity of the (anti)particle with momentum $p_j$.
Since the quarks are massless, helicity is conserved along the quark lines and there are only four different
possibilities that we need to consider
\begin{equation}
(\lambda_1,\lambda_2,\lambda_3,\lambda_4) = (+,-,+,-),(+,-,-,+),(-,+,+,-),(-,+,-,+) \; .  
\end{equation}
Moreover, the symmetries of the process allow us to compute only the first two helicity amplitudes $(+,-,+,-),(+,-,-,+)$ and then obtain the other two by acting on the result with a parity transformation, which flips the signs of the external helicities: 
\begin{equation}
(\lambda_1,\lambda_2,\lambda_3,\lambda_4)  \xrightarrow{P} (-\lambda_1,-\lambda_2,-\lambda_3,-\lambda_4) \; .
\end{equation}
In what follows, we will focus on the two independent configurations
$(\lambda_1,\lambda_2,\lambda_3,\lambda_4) = (+,-,+,-),(+,-,-,+)$.\\
We adopt the following definition for helicity states of spin-$\frac{1}{2}$ fermions
\begin{align}
  |p \rangle = {\ u} (p,+) = \left[ \frac{1}{2} (1 + \gamma_5) \right] \: {u}(p),  \quad \quad   |p] = {\ u} (p,-) = \left[\frac{1}{2} (1 - \gamma_5) \right] \: {u}(p),  \\[10pt]
   \langle p| = {\ \bar u} (p,+) = {\ \bar u}(p) \left[\frac{1}{2} (1 + \gamma_5) \right], \:  \quad \quad   [ p | = {\ \bar  u} (p,-) =  {\bar u}(p)\left[ \frac{1}{2} (1 - \gamma_5) \right],
\end{align}
where we use the well known \textit{spinor-helicity formalism}~\cite{Dixon:1996wi}, and indicate with $\pm$ the projection of the (anti-)particle spin along its four-momentum.

Using this notation and starting from the general structure of the amplitude given in eqs~\eqref{decomp_tensors} and~\eqref{tensors},
we obtain 
\begin{align}
\bar{\mathbfcal{A}}_{+-+-}^{\scriptstyle q \bar q\rightarrow \bar Q Q} &=  
\mathbfcal{H}_1 \frac{\langle 13 \rangle}{\langle 24 \rangle} \,, \quad 
\bar{\mathbfcal{A}}_{+--+}^{\scriptstyle q \bar q\rightarrow \bar Q Q} =  
\mathbfcal{H}_2 \frac{\langle 14 \rangle}{\langle 2 3 \rangle}, \label{H_ij}
\end{align}
where 
\begin{equation}\label{H1H2def}
\mathbfcal{H}_1 = 2t \mathbfcal{F}_1  - tu \mathbfcal{F}_2,\quad \mathbfcal{H}_2 = 2u \mathbfcal{F}_1  + tu \mathbfcal{F}_2.
\end{equation}
In eqs.~\eqref{H_ij} we have introduced an explicit label for the 
process that we are considering, which will turn out to be useful later on when
we describe the other partonic channels.

Since we expect that final analytic results for the 
helicity amplitudes should display the maximum degree of simplicity,
in what follows we focus directly on the two 
linear combinations $\mathbfcal{H}_1$ and $\mathbfcal{H}_2$.
We write their expansion in terms of the bare coupling $\asb$ as
\begin{equation}\label{full_amplitude}
\mathbfcal{H}_i =  \mathbfcal{H}^{(0)}_i +\left(  \frac{\asb }{4 \pi}  \right)\: \:\mathbfcal{H}^{(1)}_i+\left(  \frac{\asb }{4 \pi}  \right)^2\:  \: \mathbfcal{H}^{(2)}_i+\left( \frac{\asb }{4 \pi}  \right)^3 \:  \:\mathbfcal{H}^{(3)}_i + \calO\left( \asb^4 \right).
\end{equation} 
In the next section, we discuss the computation of  $\mathbfcal{H}_1$ and $\mathbfcal{H}_2$ up to three loops in QCD.
