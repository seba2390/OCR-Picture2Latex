% !TEX root = 0-qqQQmain.tex

\section{Results} \label{results}

After the ultraviolet and infrared pole subtraction described in the previous section, we arrive at the main result of this paper, the fully analytical expressions for the finite remainders of the helicity amplitudes for process \eqref{s_channel}.
As previously mentioned,  the helicity amplitudes for other $2\to2$ quark processes with different initial states, including the equal-flavour case $q=Q$, can be obtained by a combination of analytical continuation and momenta renaming from the ones for our main process \eqref{s_channel}.  We  discuss this in more detail in Section \ref{extra_results}.
We provide all finite remainders  in electronic format as ancillary files attached
to the \texttt{arXiv} submission of this manuscript.

\subsection{Checks}
We have performed various checks on our results. First of all, we have verified that the IR poles of our scattering
amplitudes  follow the pattern predicted in refs~\cite{Becher:2009qa,Becher:2009cu,Almelid:2015jia} up to three loops.
We have then checked the finite part of our one loop amplitudes for all different partonic channels against the automated one-loop generator \texttt{OpenLoops}~\cite{Cascioli:2011va,Buccioni:2019sur}.
Finally, we have checked our one- and two-loop amplitudes through to order $\epsilon^4$ and $\epsilon^2$, respectively,
 against the results presented in refs~\cite{Glover:2004si,Ahmed:2019qtg}.

In order to successfully perform this check, one has to pay particular attention when comparing the amplitudes before IR subtraction, 
due to a subtlety in the  dimensional regularisation scheme 
used in refs~\cite{Glover:2004si,Ahmed:2019qtg}. This is due to the fact that the tensor structures
used to decompose the scattering amplitude in those references contain an explicit dependence on the dimensional-regulation
parameter $\epsilon$, even if the external states are taken to be in four dimensions, as in the 't Hooft-Veltman prescription.
While ignoring this dependence does not change the finite remainder of the scattering amplitudes after UV and IR poles have been subtracted,
it does change the bare results.
We illustrate this point explicitly for the one loop case. 
In refs \cite{Glover:2004si,Ahmed:2019qtg},  the following four tensors are used to decompose the amplitude at one loop
in CDR\footnote{Note that here even $\widetilde T_1$  and  $\widetilde T_2$ do not coincide with our definitions.}

\begin{align} 
\label{tensors_henn_D1} \widetilde T_1 &= {\bar u} (p_2)\: \slashed p_3 \: { u} (p_1) \times   {\bar u} (p_4) \: \slashed p_1 \:{ u}(p_3) \; , \\
\label{tensors_henn_D2} \widetilde T_2 &= {\bar u} (p_2)\: \gamma^\alpha \: { u} (p_1) \times {\bar u} (p_4) \: \gamma_\alpha \:{ u}(p_3) \; , \\
\label{tensors_henn_D3} \widetilde T_3 &= {\bar u} (p_2)\: \slashed p_3   \gamma^\mu   \gamma^\nu \: { u} (p_1) \times   {\bar u} (p_4) \:
 \slashed p_1 \gamma_\mu  \gamma_\nu  \:{ u}(p_3) \; , \\
\label{tensors_henn_D4} \widetilde T_4 &= {\bar u} (p_2)\: \gamma^\alpha \gamma^\mu  \gamma^\nu \: { u} (p_1) \times {\bar u} (p_4) \: \gamma_\alpha  \gamma_\mu   \gamma_\nu \:{ u}(p_3) \; , 
\end{align} 
where we stress that this decomposition is loop dependent and is only sufficient up to one-loop order, $L \leq 1$
\begin{equation}\label{decomp_tensors_henn}
\bar{\mathbfcal{A}}^{L \leq 1}  =   \widetilde{\mathbfcal{F}}_1 \; \widetilde{T}_1 + \widetilde{\mathbfcal{F}}_2 \; \widetilde{T}_2 + \widetilde{\mathbfcal{F}}_3 \; \widetilde{T}_3 + \widetilde{\mathbfcal{F}}_4 \; \widetilde{T}_4.
\end{equation}

Importantly, in the 't Hooft-Veltman scheme the vector indices of the $\gamma$ matrices in eqs.~\eqref{tensors_henn_D1}---\eqref{tensors_henn_D4} that are not 
explicitly contracted with four-dimensional vector fields 
are in general to be taken in 
$d$ dimensions. While this makes no difference for the 
first two tensors, the second two~(\ref{tensors_henn_D3},\ref{tensors_henn_D4})  
depend explicitly on $d$ and are responsible for an ambiguity 
in the way the helicity amplitudes are defined.
Let us consider for example the fourth tensors, eq.~\eqref{tensors_henn_D4}. 
If we consider the $\gamma$ matrices to carry $d$ dimensional vector indices we can
split the four dimensional part from the
$\epsilon$-dependent one by writing 
$\gamma^\mu_d = \gamma^\mu_4 + \gamma^{\mu}_{-2 \epsilon}$ and then use the equation 
\begin{equation}
    {\rm Tr}[\gamma_4^{\mu_1} \dots \gamma_4^{\mu_n} \gamma_{-2\epsilon}^{\nu_1} \dots \gamma_{-2\epsilon}^{\nu_m} ] = \frac{1}{4}     {\rm Tr}[\gamma_4^{\mu_1} \dots \gamma_4^{\mu_n} ]     {\rm Tr}[ \gamma_{-2\epsilon}^{\nu_1} \dots \gamma_{-2\epsilon}^{\nu_m} ] 
\end{equation}
as done in ref.~\cite{Cullen:2010jv} to extract the $(-2\epsilon)$-dimensional dependence of the $\gamma$-strings. All traces can then be evaluated as usual using the Clifford algebra relation $\{\gamma_\mu,\gamma_\nu\} = 2 g_{\mu \nu}$ for $d$-dimensional indices $\mu,\nu$. In this case, the dimensional splitting procedure simply amounts to $\epsilon$ dependent coefficients of the 4-dimensional $\gamma$-strings. For instance, taking the first string of $\widetilde{T}_4$ we find  
\begin{align}
    {\bar u} (p_2)\: \gamma^\alpha \gamma^\mu  \gamma^\nu \: { u} (p_1)& = g_{-2\epsilon}^{\mu \nu}  {\bar u} (p_2)\: \gamma_4^\alpha \: { u} (p_1) \nonumber \\
    &+ g_{-2\epsilon}^{\mu \alpha}  {\bar u} (p_2)\: \gamma_4^\nu \: { u} (p_1) \nonumber \\ 
    &-  g_{-2\epsilon}^{\nu \alpha}  {\bar u} (p_2)\: \gamma_4^\mu \: { u} (p_1),
\end{align}
where ${\rm Tr}[\gamma_{-2\epsilon}^\mu \gamma_{-2\epsilon}^\nu] = g_{-2\epsilon}^{\mu \nu} $ is the $(-2\epsilon)$-dimensional part of the metric tensor. Repeating the exercise for the other fermion string and then fixing the external helicities to $(+,-,+,-)$ 
we find
\begin{equation}
\widetilde T_4|_{(+,-,+,-)} = (32 - 12 \epsilon) \; [24]\langle 3 1 \rangle\,.
\end{equation}
Similarly, by repeating the same exercise for both helicities as we did 
in eqs.~\eqref{H_ij},  
but this time starting from the tensor decomposition in eq.~\eqref{decomp_tensors_henn},
we find
\begin{align}
\mathbfcal{H}_1 &=  -tu\: \left[ \widetilde{\mathbfcal{F}}_1 + 8 \widetilde{\mathbfcal{F}}_3 \right]
+2 t\left[ \widetilde{\mathbfcal{F}}_2 + 16 \widetilde{\mathbfcal{F}}_4 \right]  
+ 2 \epsilon \left[ t u\: \widetilde{\mathbfcal{F}}_3 - 6t \widetilde{\mathbfcal{F}}_4 \right] \: ,  \label{matrix_H1}\\
\mathbfcal{H}_2 &=  -tu \: \left[ \widetilde{\mathbfcal{F}}_1 + 4 \widetilde{\mathbfcal{F}}_3\right] 
- 2 u\left[ \widetilde{\mathbfcal{F}}_2 + 4 \widetilde{\mathbfcal{F}}_4 \right] 
+ 2\epsilon \left[ t u \:\widetilde{\mathbfcal{F}}_3 + 6 u\:\widetilde{\mathbfcal{F}}_4 \right] \; .\label{matrix_H2}
\end{align}
It is instructive to compare these formulas to the corresponding ones 
obtained in our approach, see eq.~\eqref{H1H2def}.
Our expressions for the helicity amplitudes, despite not displaying 
any explicit dependence on the parameter
$\epsilon$,  are exact in the 't Hooft-Veltman scheme. 
Notice, in particular, that the two tensors onto which we decompose the
amplitude,
${T}_1$ and ${T}_2$ in eq.~\eqref{tensors}, 
have been chosen such that it makes no practical difference
whether the $\gamma$ algebra to fix the helicity amplitudes
is performed $4$ or in $d=4 - 2 \epsilon$ dimensions. 
Upon substituting the form factors provided in refs~\cite{Glover:2004si,Ahmed:2019qtg} in eqs.~\eqref{matrix_H1} and \eqref{matrix_H2} (and in the corresponding
generalisations for the two-loop corrections),
we find perfect agreement up to weight six with 
our results for the \emph{bare helicity amplitudes}.

We stress, nevertheless, that the results for the bare helicity amplitudes as provided in refs~\cite{Glover:2004si,Ahmed:2019qtg} are obtained 
by setting $\epsilon=0$ in the coefficients of the form factors 
of \eqref{matrix_H1} and \eqref{matrix_H2} before substituting
the results for the form factors. This amounts to having assumed that the $\gamma$ matrices in eq.~\eqref{decomp_tensors_henn} are purely four-dimensional.  
This produces a difference for the bare amplitudes with respect to ours 
of order $\mathcal{O}(\epsilon)$ 
at one loop and $\mathcal{O}(1/\epsilon)$ at 
two loops.\footnote{Note that in this approach, one needs two more tensors 
at two loops, which depend quadratically on $\epsilon$.}  
However, it is easy to see that, as long as this choice is made 
consistently to all orders,  one obtains the same results for the 
finite remainders in $d=4$. 
In fact, one can imagine to first subtract UV and IR poles at the level of 
the individual form factors $\widetilde{\mathbfcal{F}}_j$ and, only afterwards,
substitute the finite form factors in eqs.~\eqref{matrix_H1} and \eqref{matrix_H2}, and fix $\epsilon = 0$. 
It is then obvious that the finite remainder cannot depend on 
the $\epsilon$-suppressed contributions in eqs.~\eqref{matrix_H1} 
and \eqref{matrix_H2}.
 We have verified the last statement directly, finding perfect agreement with refs~\cite{Glover:2004si,Ahmed:2019qtg} at the level
 of the finite remainders. 

\subsection{Numerical Evaluation}

\begin{figure}
\center
\includegraphics[width=1\textwidth]{fig/allH_1.pdf}
\caption{Real (left) and imaginary (right) parts of the form factors $\mathbfcal{H}_{1,\text{fin}}^{[i],(L)}$ relevant for helicities $(+,-,+,-)$.   Colour components $[i]$ and number of loops $(L)$ are specified in the legends.}
\label{allH_1}
\end{figure} 

\begin{figure}
\center
\includegraphics[width=1\textwidth]{fig/allH_2.pdf}
\caption{Real (left) and imaginary (right) parts of the form factors $\mathbfcal{H}_{2,\text{fin}}^{[i],(L)}$ relevant for helicities $(+,-,-,+)$.   Colour components $[i]$ and number of loops $(L)$ are specified in the legends. }
\label{allH_2}
\end{figure} 
We present numerical results for the finite form factors defined in  \eqref{one}, \eqref{two} and \eqref{three} calculated in the physical region $0<x<1$ for the process in eq.~\eqref{s_channel}, $q\bar{q}\to Q\bar{Q}$. To evaluate our results numerically we made use of the \texttt{Mathematica} package \texttt{PolyLogTools} \cite{Duhr:2019tlz},
which in turn uses the \texttt{Ginac} library~\cite{Bauer:2000cp,cln,Vollinga:2004sn}.
For the various parameters we use the following values:
\begin{equation}
N_c = 3, \quad n_f = 5, \quad \as = 0.118, \quad  \mu^2 = s.
\end{equation}
We show results for $\mathbfcal{H}_{1,\text{fin}}$ corresponding to helicities $(+,-,+,-)$ in figure~\ref{allH_1} and for $\mathbfcal{H}_{2,\text{fin}}$ corresponding to helicities $(+,-,-,+)$ in figure~\ref{allH_2}.
In the figures, we present the two colour components of the form factors individually, where we recall that our colour decomposition reads: 
\begin{equation}
{\mathbfcal{H}}_{i,\text{fin}}^{(L)}  
= 
\begin{pmatrix}
{\mathcal{H}}_{i,\text{fin}}^{[1],(L)}  \\
{\mathcal{H}}_{i,\text{fin}}^{[2],(L)} 
\end{pmatrix} \quad i=1,2 \; .
\end{equation}
Here, the colour index $[1]$ is related to the colour structure  ${\delta}_{ i_1 i_4} {\delta}_{i_2 i_3}$ while the index $[2]$ refers to the coefficient of ${\delta}_{ i_1 i_2} {\delta}_{ i_3 i_4}$.
Lastly, the index $(L)$ refers to the number of loops of the corresponding amplitude.  


