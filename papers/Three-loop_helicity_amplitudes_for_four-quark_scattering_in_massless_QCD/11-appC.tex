% !TEX root = 0-qqQQmain.tex

\section{Analytic Continuation} \label{sec:appC}
In this appendix, we briefly outline the analytical continuation of the scattering amplitudes needed to obtain the crossed channels presented in section~\ref{extra_results}. The kinematic regions and the connecting paths used in the following are shown in figure~\ref{KinematicsPath}.
\begin{figure}
\center
\includegraphics[scale=0.3]{fig/BranchPaths.pdf}
\caption{Paths in phase space to obtain the crossed amplitudes.}
\label{KinematicsPath}
\end{figure}
\paragraph{Process $ q Q \rightarrow  q Q$:} To derive the partonic channel~\eqref{t_channel} from our calculation of process~\eqref{s_channel_new}, one must effectively continue our results from region \texttt{I} to region \texttt{II}, where $p_1$ and $p_3$ are incoming, see figure \ref{KinematicsPath}. After the analytic continuation, one can then rename $p_2 \leftrightarrow p_3$.  The connecting path crosses two branch cuts, one at $t=0$ and one at $s=0$.  
After crossing each branch cut, we wish to maintain the amplitude written in terms of explicitly real HPLs.
This is achieved by the change of variable $x \rightarrow -x_a - i\delta$ for the first branch cut and $x_a \rightarrow -1/y -i \delta$ for the second, where the variables used have been defined in eqs.~\eqref{variables}.
We stress here, that the sign of the infinitesimal imaginary parts are chosen by imposing
that each Mandelstam invariant that changes sign effectively carries a small \emph{positive} imaginary part.
Once we are in region \texttt{II}, with the amplitude written as a function of $y$, 
we can perform the renaming  $p_2 \leftrightarrow p_3$,
\begin{equation}
y = x|_{p_2 \leftrightarrow p_3} \; \xrightarrow{p_2 \leftrightarrow p_3} \;x \; ,
\end{equation}
to match the usual conventions.
\paragraph{Process $ q \bar Q \rightarrow  q \bar Q$:} To obtain process \eqref{u_channel} from process \eqref{s_channel_new}, we move from region \texttt{I} to region \texttt{III},  and then exchange  $p_1$ with $p_4$.  This time, we first cross the branch cut at $u=0$, and then the one at $s=0$.  It is convenient to perform a preliminary change of variables $x =1-x'$ prior to crossing the cuts.  We can then use exactly the same changes of variables as for the previous case: to cross the $u=0$ branch cut we use $x' \rightarrow -x_b - i\delta$, while for the $s=0$ branch cut we use $x_b \rightarrow -1/z -i \delta$, such that $0<z<1$ in region \texttt{III}. We can then perform the renaming $p_2 \leftrightarrow p_4$,
\begin{equation}
z = x|_{p_2 \leftrightarrow p_4} \; \xrightarrow{p_2 \leftrightarrow p_4} \; x \; .
\end{equation}
As for the previous case, the signs of the imaginary parts are always chosen
such that the Mandelstam invariants that change sign 
effectively carry a small \emph{positive} imaginary part.
We have performed all analytic continuations explicitly and employed \texttt{PolyLogTools}~\cite{Duhr:2019tlz} for the required transformations of the harmonic polylogarithms.
