% !TEX root = 0-qqQQmain.tex

\section{Kinematics and Notation}\label{Kinematics}
We consider the massless quark-antiquark scattering process 
\begin{equation}
  { q}(p_1) \;+ \;\bar { q}(p_2)  \; \longrightarrow \;  { \bar Q}(-p_3) \;+ \; {   Q}(-p_4) \; ,    \label{s_channel} 
\end{equation}
where $q$ and $Q$ are in general differently-flavoured quarks
and $p_1^2=p_2^2=p_3^2=p_4^2=0$. From
the master process in eq.~\eqref{s_channel} one can obtain all $2 \to
2$ quark scattering amplitudes with arbitrary initial states, including equal-flavour scattering
$q=Q$.  We discuss in detail how this can be achieved in section~\ref{extra_results}.\\ The minus signs appearing in the final-state
momenta imply that all momenta are taken to be incoming,
\begin{equation}
p_1^\mu + p_2^\mu  + p_3^\mu + p_4^\mu = 0.
\end{equation}
This choice is convenient when performing the required crossings to obtain
all the relevant partonic channels.
The kinematics of the process eq.~\eqref{s_channel} can be parametrized in
terms of Mandelstam invariants, defined as
\begin{align}
%& s_{ij} = (p_i + p_j)^2, \\[10pt]
s% = s_{12}
 = (p_1 + p_2)^2, \quad
t %= s_{13}
 = (p_1 + p_3)^2, \quad
u %= s_{23}
 = (p_2 + p_3)^2,
%\\[10pt]
%&
\quad u = -t-s \,.
\end{align}
We also find it convenient to define the dimensionless ratios 
\begin{equation}\label{variables}
x=-\frac{t}{s}\,, 
\quad\quad y = x|_{p_2 \leftrightarrow p_3} = -\frac{s}{t} , \quad\quad z = x|_{p_2 \leftrightarrow p_4} = -\frac{t}{u}.
\end{equation}
The variables $y$, $z$ are not needed for the computation of process
\eqref{s_channel}, but will be convenient to describe all the other
processes which can be derived from it using crossing symmetry.  In
terms of these variables the physical scattering region is given by $s >
0\,, \; t,u < 0$ which imply $0 < x < 1\,.$
For our calculation, we work in QCD with $n_f$ massless quarks and 
$N_c$ colours. We denote the
generators of the fundamental representation of $SU(N_c)$ 
by $(T^a)_{ij}$, with $\tr[T^aT^b] =
\frac12 \delta_{ab}$. 
Also, we indicate with $C_F$ and $C_A$ the quadratic Casimir constants.
They are defined through
$({T^a})_{ij} ({T^a})_{jk} =
C_F \delta_{ik}$
and 
$f_{acd}f_{bcd} = C_A\delta_{ab}$
where $f_{abc}$ are the 
structure constants.
These definitions imply
\begin{equation}
  C_F=\frac{N_c^2-1}{2N_c},~~~~ C_A = N_c,
\end{equation}
for the algebra of $SU(N_c)$.

