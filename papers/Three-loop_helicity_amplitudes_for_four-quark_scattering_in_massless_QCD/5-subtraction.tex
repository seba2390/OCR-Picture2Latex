% !TEX root = 0-qqQQmain.tex

\section{Ultraviolet and Infrared Subtraction}
\label{subtraction}

The result of the computation described in the previous section are the divergent helicity amplitudes for the process \eqref{s_channel} to $\mathcal{O}(\asb^4)$.  
In the following, we describe the UV renormalisation and IR subtraction of the divergent amplitude for this process. 

\subsection{Ultraviolet Renormalisation}
We work in massless QCD with an arbitrary number of fermion flavours $n_f$.  We adopt the standard $\overline{\mathrm{MS}}$ renormalisation scheme,  where the bare coupling $\asb$ is written in terms of the renormalised coupling $\as(\mu)$ in the following way
\begin{align}
\label{bare_to_phys}
\asb \: \mu_0^{2\epsilon} \: S_\epsilon &= \as \: \mu^{2\epsilon} Z[\alpha_s(\mu)] 
\end{align}
where $S_\epsilon = (4 \pi)^{-\epsilon} e^{-\gamma_E \epsilon}$, $\mu$ is the renormalisation scale (for the rest of the paper we set $\mu_0=\mu$) and
\begin{align}
Z[\alpha_s]  =  1 -  \left( \frac{\as}{4 \pi} \right) \frac{ \beta_0 }{\epsilon } + \left(\frac{\as}{4 \pi}\right)^2 \left( \frac{\beta_0^2}{\epsilon^2} - \frac{\beta_1 }{2 \epsilon} \right)- 
\left(\frac{\as}{4 \pi}\right)^3  \left( \frac{\beta_0^3}{\epsilon^3} - \frac{ 7}{6} \frac{\beta_0 \beta_1}{\epsilon^2}+ \frac{\beta_2}{3 \epsilon} \right)  + \mathcal{O}(\as^4)\,.
\end{align}
The $\beta$-function coefficients are defined through
\begin{align}
\frac{d \as }{d \log \mu} &= \beta(\as,\epsilon) = \beta(\as) - 2\epsilon \as \; , \qquad
\beta(\as) = -2 \as \sum_ {n=0}^\infty \beta_n \left(\frac{\as}{4 \pi}\right)^{n+1} \; ,
\end{align}
where in this equation $\as\equiv\as(\mu)$. To the relevant order, they read
\begin{align}
\beta_0 &= \frac{11}{3} C_A - \frac{2}{3}\: n_f \; , \nonumber\\%[10pt]
\beta_1 &= \frac{1}{3} \left(34 \: C_A^2-10\: C_A \:n_f \right)-2 \:C_F\: n_f \; ,\\ %[10pt]
\beta_2 &= -\frac{1415 \:C_A^2 \:n_f}{54}+\frac{2857\: C_A^3}{54}-\frac{205\: C_A\: C_F\:
   n_f}{18}+\frac{79 \:C_A \:n_f^2}{54}+C_F^2 \:n_f+\frac{11\: C_F \:n_f^2}{9} \; .\nonumber
\end{align}
By inserting  eq.~\eqref{bare_to_phys} in the $\as$ expansion for the helicity amplitudes~\eqref{full_amplitude}, we obtain the renormalised helicity amplitudes
\begin{align}
\mathbfcal{H}_{i,\text{ren}}^{(0)} &=  \mathbfcal{H}_i^{(0)} \, , \\ %[8pt]
\mathbfcal{H}_{i,\: \text{ren}}^{(1)} & =\mathbfcal{H}_i^{(1)}- \frac{\beta_0 }{\epsilon}   \mathbfcal{H}_i^{(0)} \, , \\ %[8pt]
\mathbfcal{H}_{i,\: \text{ren}}^{(2)} &= \mathbfcal{H}_i^{(2)}  - \frac{2 \beta_0 }{\epsilon}  \mathbfcal{H}_i^{(1)}  + \frac{ \left(2 \beta_0^2- \beta_1\epsilon\right)}{2 \epsilon^2} \mathbfcal{H}_i^{(0)} \, ,\\ %[8pt]
\mathbfcal{H}_{i,\: \text{ren}}^{(3)} & = \mathbfcal{H}_i^{(3)} -\frac{3 \beta_0}{\epsilon}    \mathbfcal{H}_i^{(2)} +\frac{ \left(3 \beta_0^2-\beta_1 \epsilon\right)}{\epsilon^2}  \mathbfcal{H}_i^{(1)} + \frac{ \left(7 \beta_1 \beta_0
   \epsilon -6 \beta_0^3-2 \beta_2 \epsilon^2 \right)}{6 \epsilon^3} \mathbfcal{H}_i^{(0)}  \; .
\label{hel_ampls_ren}
\end{align}
These are free from UV poles, but they still contain poles in $\epsilon$ of IR origin. We discuss how to subtract them in the next subsection.


\subsection{Infrared Subtraction}
While the structure of IR singularities of scattering
amplitudes in massless QCD up to two-loop order has been known for a
long time~\cite{Catani:1998bh}, its generalisation to three- and
higher-loop order has been understood only more recently, in
particular in the case where four or more coloured partons participate
to the scattering
process~\cite{Becher:2009cu,Becher:2009qa,Almelid:2015jia}.  In
particular, it has been shown that IR singularities are in one-to-one
correspondence to the UV poles of operator matrix elements in
SCET~\cite{Becher:2009cu,Becher:2009qa}.  Therefore, UV
renormalisation in SCET corresponds to IR subtraction in QCD and one
can write the finite remainder of the scattering amplitude by means of
a multiplicative colour-space operator $\mathbfcal{Z}$
as
\begin{equation}\label{Zeta_H}
\mathbfcal{H}_{i,\:\text{fin}} (\epsilon,\{p\})= \lim_{\epsilon \rightarrow 0} \mathbfcal{Z}^{-1}(\epsilon,\{p\},\mu) \; \mathbfcal{H}_{i,\:\text{ren}}(\epsilon,\{p\}) \; ,
\end{equation}
where $\{p\}$  stands for the dependence on the external kinematics. 
We point out that,  since we are working with vectors in colour-space as defined in Section \ref{The Scattering Amplitude}, the $\mathbfcal{Z}$ colour operator can be represented as a 2 by 2 matrix which mixes the colour structures defined in eq.~\eqref{colour_structures}. 
Solving a renormalisation group equation one finds that $\mathbfcal{Z}$ can be rewritten as 
\begin{equation}\label{exponentiation}
\mathbfcal{Z} (\epsilon,\{p\},\mu) = \mathbb{P} \exp \left[ \int_\mu^\infty \frac{\mathrm{d} \mu'}{\mu'}  \mathbf{\Gamma}(\{p\},\mu')\right] = \sum_{n=0}^{\infty} \left( \frac{\as}{4 \pi} \right)^n \mathbfcal{Z}_n \; ,
\end{equation}
with $\mathbb{P}$ the \textit{path-ordering} symbol, $i.e.$ operators are ordered from left to right with decreasing values of $\mu'$.
Following the notation of \cite{Almelid:2015jia}, where $\mathbf{\Gamma}$ was first computed up to three loops,  the anomalous dimension operator for 4 coloured external particles is written as 
\begin{equation}\label{anomalous_operator}
\mathbf{\Gamma}(\{p\},\mu) =  \mathbf{\Gamma}_{\text{dipole}}(\{p\},\mu)  + \mathbf{\Delta}_4 (\{p\}) \; .
\end{equation}
Above,  $ \mathbf{\Gamma}_\text{dipole}$ represents the well known dipole colour correlations between two coloured external legs, namely 
\begin{equation}\label{dipole}
\mathbf{\Gamma}_{\text{dipole}}(\{p\},\mu)  =  \sum_{1\leq i < j \leq 4} \mathbf{T}^a_i \; \mathbf{T}^a_j \; \gamma^\text{cusp}(\as) \; \log\left(\frac{\mu^2}{-s_{ij}-i \delta}\right)  \; + \; \sum_{i=1}^4 \; \gamma^i(\as) \; ,
\end{equation}
with $\mathbf{T}^a_i$ the $i$-th particle $SU(N_c)$ generator and from now on we use the shorthand $\alpha_s = \alpha_s(\mu)$ to indicate the renormalised coupling at scale $\mu$.
The constants $\gamma^\text{cusp}$~\cite{Korchemsky:1987wg,Moch:2004pa,Vogt:2004mw,Grozin:2014hna,Henn:2019swt,Huber:2019fxe,vonManteuffel:2020vjv}
% concept, massless 3-loop, angle dependent 3-loop, 4-loop
and $\gamma^i$~\cite{Ravindran:2004mb,Moch:2005id,Moch:2005tm,Agarwal:2021zft}
% concept, 3-loop, 4-loop
are given in Appendix \ref{sec:appA}.  It is also useful to define the expansions 
\begin{equation}
\mathbf{\Gamma}_{\text{dipole}} = \sum_{n=0}^\infty \mathbf{\Gamma}_n \;  \left( \frac{\as}{4 \pi} \right)^{n+1}\,, \quad  \quad \Gamma '= \frac{\partial \mathbf{\Gamma}_{\text{dipole}}}{\partial \log \mu} = \sum_{n=0}^\infty \Gamma'_n \;  \left( \frac{\as}{4 \pi} \right)^{n+1}\; .
\end{equation} 
The operator $\mathbf{\Delta}_4$ appears instead for the first time at three loops
and contains quadrupole colour correlations among all four external legs.
It can also be expanded in $\as$ as
\begin{equation}
\mathbf{\Delta}_4 (\{p\}) = \sum_{L=3}^\infty \left(\frac{\as}{4 \pi}\right)^L \; \mathbf{\Delta}^{(L)}_4 (\{p\}) \; .  \label{delta4}
\end{equation}
In terms of these quantities one can organise the IR poles of the helicity amplitudes as 
\begin{align}
\mathbfcal{H}_{i,\:\text{fin}}^{(0)} &= \mathbfcal{H}_{i}^{(0)} \; ,  \label{tree}\\ %[10pt]
\mathbfcal{H}_{i,\:\text{fin}}^{(1)} &= \mathbfcal{H}_{i,\: \text{ren}}^{(1)} - \mathbfcal{I}_1 \; \mathbfcal{H}_{i,\: \text{ren}}^{(0)} \; , \label{one}  \\ %[10pt]
\mathbfcal{H}_{i,\:\text{fin}}^{(2)} &= \mathbfcal{H}_{i,\: \text{ren}}^{(2)} - \mathbfcal{I}_2\; \mathbfcal{H}_{i,\: \text{ren}}^{(0)} - \mathbfcal{I}_1\; \mathbfcal{H}_{i,\: \text{ren}}^{(1)} \; ,\label{two}  \\ %[10pt]
\mathbfcal{H}_{i,\:\text{fin}}^{(3)} &= \mathbfcal{H}_{i,\: \text{ren}}^{(3)} - \mathbfcal{I}_3\; \mathbfcal{H}_{i,\: \text{ren}}^{(0)} - \mathbfcal{I}_2\; \mathbfcal{H}_{i,\: \text{ren}}^{(1)} - \mathbfcal{I}_1 \;\mathbfcal{H}_{i,\: \text{ren}}^{(2)} \; ,\label{three}  
\end{align}
where the IR subtraction operators read
\begin{align}
\mathbfcal{I}_{1}  &= \mathbfcal{Z}_1 \label{I1}\,,\\%[10pt]
\mathbfcal{I}_{2} &=  \mathbfcal{Z}_2 - \mathbfcal{Z}_1^2  \label{I2}\,, \\ %[10pt]
 \mathbfcal{I}_{3} &= \mathbfcal{Z}_3    -   2\mathbfcal{Z}_1  \mathbfcal{Z}_2 + \mathbfcal{Z}_1^3 + \mathbf{\Delta}_4^{(3)}  \;,\label{I3}
\end{align}
with 
\begin{align}
 \mathbfcal{Z}_1 &=  \frac{\Gamma'_0}{4 \epsilon^2} + \frac{\mathbf{\Gamma}_0}{2 \epsilon} \,,   \label{Z1}\\%[8pt]
\mathbfcal{Z}_2 &=   \frac{{\Gamma_0'}^2}{32 \epsilon^4} + \frac{\Gamma'_0}{8 \epsilon^3} \left( \mathbf{\Gamma}_0 - \frac{3}{2} \beta_0  \right) +  \frac{\mathbf{\Gamma}_0}{8 \epsilon^2}(\mathbf{\Gamma}_0 - 2 \beta_0)  + \frac{\Gamma_1'}{16 \epsilon^2} + \frac{\mathbf{\Gamma}_1}{4 \epsilon}\,, \label{Z2}\\ %[8pt]
\mathbfcal{Z}_3 &=  \frac{{\Gamma'_0}^3}{384 \epsilon^6}  + \frac{{\Gamma'_0}^2}{64 \epsilon^5}(\mathbf{\Gamma}_0 - 3 \beta_0) + \frac{\Gamma_0'}{32 \epsilon^4} \left( \mathbf{\Gamma}_0 - \frac{4}{3} \beta_0 \right) \left( \mathbf{\Gamma}_0 - \frac{11}{3} \beta_0 \right)  + \frac{\Gamma_0' \Gamma_1'}{64 \epsilon^4}  \nonumber\\% [8pt]
& \quad +\frac{\mathbf{\Gamma}_0}{48\epsilon^3}(\mathbf{\Gamma}_0 - 2 \beta_0)(\mathbf{\Gamma}_0 - 4 \beta_0) + \frac{\Gamma'_0}{16 \epsilon^3} \left( \mathbf{\Gamma}_1 - \frac{16}{9} \beta_1\right) + \frac{\Gamma_1'}{32 \epsilon^3} \left( \mathbf{\Gamma}_0 - \frac{20}{9} \beta_0 \right)  \nonumber\\ %[8pt]
&\quad + \frac{\mathbf{\Gamma}_0 \mathbf{\Gamma}_1}{8 \epsilon^2} - \frac{\beta_0 \mathbf{\Gamma}_1 + \beta_1 \mathbf{\Gamma}_0}{6 \epsilon^2} + \frac{\Gamma_2'}{36 \epsilon^2 } + \frac{\mathbf{\Gamma}_2 + \mathbf{\Delta}_4^{(3)}}{6 \epsilon} \; ,\label{Z3}
\end{align}
together with the quadrupole contribution $\mathbf{\Delta}_4^{(3)}$, 
which is non-diagonal in colour space and reads explicitly
\begin{equation}\label{delta_4_main}
\mathbf{\Delta}_4^{(3)} = {\small
\begin{pmatrix}
 - 8 N_c \left[2 \left(D_2+2 D_1 \right) + 3 C \right] & 8 N_c^2 \left[2 D_2 - C\right]+32 \left[C-  D_1- D_2 \right] \\[8pt]
 8 N_c^2 \left[2 D_2- C\right]+32(C+ D_1) & 8 N_c \left[2 \left(D_2+2 D_1\right)-3 C\right] 
\end{pmatrix}
} \; .
\end{equation}
We stress here that, even if their $x$ dependence has been left implicit for clarity, $D_1$ and $D_2$ are non trivial functions of the kinematics.
More explicitly, the abbreviations in~\eqref{delta_4_main} read
\begin{align}
\label{C}
C &=  \zeta_5 + 2 \zeta_2 \zeta_3\,,
\\[2ex]
D_1 &= -2 \textit{G}_{1,4}-\textit{G}_{2,3}-\textit{G}_{3,2}+2 \textit{G}_{1,1,3}+2 \textit{G}_{1,2,2}-2 \textit{G}_{1,3,0}-\textit{G}_{2,2,0}-\textit{G}_{3,1,0} \nonumber\\
&\quad +2 \textit{G}_{1,1,2,0}-2 \textit{G}_{1,2,0,0}+ 2 \textit{G}_{1,2,1,0}+4 \textit{G}_{1,0,0,0,0}-2 \textit{G}_{1,1,0,0,0}+\frac{1}{2}\zeta_5  - 5 \zeta_2 \zeta_3  \nonumber \\
 &\quad  + \zeta_2 [5 \textit{G}_{3}+5 \textit{G}_{2,0}+2 \textit{G}_{1,0,0}-6 (\textit{G}_{1,2}+\textit{G}_{1,1,0})] + \zeta_3 (\textit{G}_{2}+2 \textit{G}_{1,0}-2 \textit{G}_{1,1}) \nonumber\\
&\quad - i \pi  [-\zeta_3 \textit{G}_{0}+\textit{G}_{2,2}+\textit{G}_{3,0}+\textit{G}_{3,1}+ \textit{G}_{2,0,0}+2 (\textit{G}_{1,3}-\textit{G}_{1,1,2}-\textit{G}_{1,2,1}-\textit{G}_{1,0,0,0})]\nonumber\\
&\quad + i \pi \zeta_2 (-\textit{G}_{2}+2 (\textit{G}_{1,1}+\textit{G}_{1,0}))- 11 i\pi \zeta_4 \, ,\label{D1}\\[10pt]
D_2 &=  2 \textit{G}_{2,3}+2 \textit{G}_{3,2}-\textit{G}_{1,1,3}-\textit{G}_{1,2,2}-2 \textit{G}_{2,1,2}+2 \textit{G}_{2,2,0}-2 \textit{G}_{2,2,1} \nonumber\\
&\quad +2 \textit{G}_{3,1,0}-2 \textit{G}_{3,1,1}-\textit{G}_{1,1,2,0}- \textit{G}_{1,2,1,0}-2 \textit{G}_{2,1,1,0}+4 \textit{G}_{2,1,1,1}-\zeta_5 +4 \zeta_2 \zeta_3 \nonumber \\
&\quad + \zeta_3 \textit{G}_{1,1}+\zeta_2 [-6 \textit{G}_{3}-6 \textit{G}_{2,0}+2 \textit{G}_{2,1}+5 (\textit{G}_{1,2}+\textit{G}_{1,1,0})] \nonumber\\
&\quad + i \pi  (\zeta_3 \textit{G}_{1}+2 \textit{G}_{3,0}-\textit{G}_{1,1,2}-\textit{G}_{1,2,0}-\textit{G}_{1,2,1}+2 \textit{G}_{2,0,0}-2 \textit{G}_{2,1,0} \nonumber\\
&\quad +2 \textit{G}_{2,1,1}-\textit{G}_{1,1,0,0})+ i \pi \zeta_2  (4 \textit{G}_{2}-\textit{G}_{1,1}) \, .  \label{D2}
\end{align}
where $D_1$ and $D_2$ are expressed in terms of generalized polylogarithms of argument $x$ with letters 0 and 1.
In eqs.~(\ref{D1},\ref{D2}) we have suppressed the $x$ dependence and used a compact index notation for the harmonic polylogarithms similar to \cite{Remiddi:1999ew,Maitre:2005uu},
\begin{align}
\label{compactindex}
G_{a_1,\dots,a_n,\footnotesize\underbrace{ 0,\dots,0}_{n_0}} &= G(\underbrace{0,\dots,0}_{|a_1|-1},\sgn(a_1),\dots,\underbrace{0,\dots,0}_{|a_n|-1},\sgn(a_n),\underbrace{0,\dots,0}_{n_0};x),
\end{align}
for $a_i \in \mathbbm{Z} \setminus\!\{0\}$.
The $G$ functions on the r.h.s.\ of eq.~\eqref{compactindex} are the generalized polylogarithms
\begin{align}
G(w_1,w_2,\dots,w_n; x)&= \int_0^x \frac{dt}{t-w_1} G(w_2,\dots,w_n; t) \quad \text{if at least one $w_i\neq 0$}, \\
G(\underbrace{0,\ldots,0}_{n}; x) &= \frac{1}{n!}\log^{n}(x),
\end{align}
and we use this notation also to present our explicit analytical results for the finite remainders.
We note here that both $D_1$ and $D_2$ are of uniform transcendental of weight five and are independent of the matter content of the theory, $i.e.$\ they do not depend
explicitly on $n_f$. 
More details on the derivation of these formulas are provided in Appendix~\ref{sec:appB}.
