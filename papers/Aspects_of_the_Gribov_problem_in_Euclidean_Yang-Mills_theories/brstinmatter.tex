

%%%%%%%%%%%%%%%%%%%%%%%%%%%%%%%%%%%%%%%%%%%%%%%%%%%%
\chapter{Quark confinement and BRST soft breaking}
\chaptermark{Quark confinement and BRST breaking}
\label{brstonmatter}
%%%%%%%%%%%%%%%%%%%%%%%%%%%%%%%%%%%%%%%%%%%%%%%%%%%%



Being restricted to the matter sector of Nature, the directly observed particles are called
hadrons, which are colorless composition of quarks. For instance, protons are elements of the
baryon set of particles (composed of three quarks) whose mass is about
$\unit{938}{\mega\electronvolt}$, while quark's mass is of order $\sim
\unit{1}{\mega\electronvolt}$. So, where does come from the huge difference of mass? How does
it happen? Nature gives us clues of an spontaneous symmetry breaking (of an approximate
symmetry), known as \emph{chiral symmetry breaking}, with a mass term being dynamically
generated.

Besides chiral symmetry breaking, there exists the confinement scenario, where quarks cannot 
be asymptotically detected as free particles or, in other words, they do not belong to the
physical particle spectrum of Nature. In a general sense, the chiral phase transition is not
directly related to the deconfinement transition; it is known that for the two-flavors and
three colors scenario the chiral phase transition takes place at a critical temperature of
about $\unit{170}{\mega\electronvolt}$, while the deconfinement second order phase transition
occur at a temperature around $\unit{270}{\mega\electronvolt}$,
\cite{Fukushima:2002bk,Fukushima:thesis,Fukushima:2010bq,Bazavov:2009zn,Fukushima:2003fw,Banks:1979yr}.

In this chapter we propose, and analyze, an effective model to the matter sector, by means of
introducing a non-local mass term to the matter field, leading to a soft breaking of the
BRST symmetry, in analogy to what happens in the gauge sector. In a sense, this non-local mass
term would represent a generalization of the \emph{horizon function} of the GZ scheme of the
gauge field applied to the matter field. We provide a general analysis of this procedure by
specializing the matter sector to the scalar field and to the quark field. A comparison is made
with the most recent lattice data of the matter field and we could find a clear agreement
between them. The matter field, in this scenario, is deprived of an asymptotic physical
particle interpretation, since its propagator displays positivity violation, so not satisfying
every reality condition of Osterwalder-Schrader, just the same as the gauge field. The ${\cal
N}=1$ supersymmetric case is presented at the of the chapter as an example.


The content of the second chapter, concerning the Gribov and Gribov-Zwanziger mechanism of
quantizing the gauge field, is useful to the comprehension of the present one. More precisely,
the fate of BRST symmetry breaking due to the non-local horizon function must be kept in mind.
Therefore, to the benefit of the reader, some recurrent expressions will be rewritten here,
preventing going back and forth to the second chapter from being repeated overmuch. The first
one is the GZ action, which reads
\begin{equation} 
S_{GZ} = S_{YM} + S_{gf} + S_0+S_\gamma  \;, 
\label{sgz5}
\end{equation}
with
\begin{equation}
S_0 =\int d^{4}x \left( {\bar \varphi}^{ac}_{\mu} (\partial_\nu D^{ab}_{\nu} )
\varphi^{bc}_{\mu} - {\bar \omega}^{ac}_{\mu}  (\partial_\nu D^{ab}_{\nu} ) \omega^{bc}_{\mu}
- gf^{amb} (\partial_\nu  {\bar \omega}^{ac}_{\mu} ) (D^{mp}_{\nu}c^p) \varphi^{bc}_{\mu}
\right) \;, \label{s05}
\end{equation}
and 
\begin{equation}
S_\gamma =\; \gamma^{2} \int d^{4}x \left( gf^{abc}A^{a}_{\mu}(\varphi^{bc}_{\mu} + {\bar
\varphi}^{bc}_{\mu})\right)-4 \gamma^4V (N^2-1)\,.
\label{hfl5}
\end{equation} 
The RGZ action can written as
\begin{equation}
S_{RGZ} = S_{GZ} + \int d^4x \left(  \frac{m^2}{2} A^a_\mu A^a_\mu  - \mu^2 \left( {\bar
\varphi}^{ab}_{\mu}  { \varphi}^{ab}_{\mu} -  {\bar \omega}^{ab}_{\mu}  { \omega}^{ab}_{\mu}
\right)   \right)  \;.
\label{rgz5}
\end{equation}
The soft breaking of the BRST symmetry can be directly seen from the aplication of the BRST
transformation $s$ on the (R)GZ action, given by the BRST transformation of each field that is
given in \eqref{brst0}. At the
end one gets
\begin{equation}
s S_{GZ} = \gamma^2 \Delta  \;, 
\label{brstbrr5}
\end{equation}
with
\begin{equation}
\Delta = \int d^{4}x \left( - gf^{abc} (D_\mu^{am}c^m) (\varphi^{bc}_{\mu} + {\bar
\varphi}^{bc}_{\mu})   + g f^{abc}A^a_\mu \omega^{bc}_\mu            \right)  \;.
\label{brstb15}
\end{equation}
Finally, the gluon and ghost propagators read as,
\begin{eqnarray} 
\langle  A^a_\mu(k)  A^b_\nu(-k) \rangle  & = &  \delta^{ab}  \left(\delta_{\mu\nu} -
\frac{k_\mu k_\nu}{k^2}     \right)   {\cal D}(k^2) \;, 
\label{glrgz5} \\
{\cal D}(k^2) & = & \frac{k^2 +\mu^2}{k^4 + (\mu^2+m^2)k^2 + 2Ng^2\gamma^4 + \mu^2 m^2}  \;.
\label{Dg5}
\end{eqnarray} 
and
\begin{equation}
{\cal G}^{ab}(k^2) = \langle  {\bar c}^{a} (k)  c^b(-k) \rangle \Big|_{k\sim 0} \; \sim
\frac{\delta^{ab}}{k^2}   \,.
\label{ghrgz5} 
\end{equation}

Moreover, despite the soft breaking, eq.\eqref{brstbrr5}, a set of BRST  invariant composite operators whose correlation functions exhibit the K{\"a}ll{\'e}n-Lehmann spectral representation with positive spectral densities can be consistently introduced \cite{Baulieu:2009ha}. 

Although a satisfactory understanding of the physical meaning of the soft breaking of the BRST
symmetry in presence of the Gribov horizon and of its relationship with confinement is still
lacking, it is worth  underlining here that the first concrete numerical lattice evidence  of
the existence of such breaking has been provided by the authors of \cite{Cucchieri:2014via},
where the Bose-ghost propagator 
\begin{eqnarray}
{\cal Q}^{abcd}_{\;\;\mu\nu} ~=~ \langle \omega^{ab}_{\mu} \bar{\omega}^{cd}_{\nu} +
\varphi^{ab}_{\mu}
\bar{\varphi}^{cd}_{\nu}   \rangle
\label{ss1}
\end{eqnarray}
has being numerically computed on the lattice formulation, since it can be written as
\begin{eqnarray}
{\cal Q}^{abcd}_{\;\;\mu\nu} ~=~ \langle s\varphi^{ab}_{\mu}\bar{\omega}^{cd}_{\nu} \rangle\,,
\label{ss1}
\end{eqnarray}
which is evidently a BRST exact correlation function. So, if it is non-zero, it is a signal of
the (soft) BRST symmetry breaking. As $(\bar{\omega},\omega)$ and $(\bar{\varphi},\varphi)$
are localizing auxiliary fields of the GZ framework, thus there must be a non-local version of
the Boson-ghost propagator, and indeed there is. Evaluating \eqref{ss1} is equivalent to
measuring
\begin{eqnarray} 
\langle {\cal R}^{ab}_{\;\;\mu}(x)  {\cal R}^{cd}_{\;\;\nu}(y) \rangle    \;, 
\label{rr} 
\end{eqnarray}
with
\begin{eqnarray}
{\cal R}^{ac}_{\;\;\mu}(x) & = &  \int d^4z\;  ({\cal M}^{-1})^{ad} (x,z) \; g f^{dec}
A^{e}_\mu(z)  \;, 
\label{ra} 
\end{eqnarray} 
where $\cal M$ accounts for the inverse of the Faddeev-Popov operator.
The relation of the correlation function \eqref{rr} with the breaking of the BRST symmetry can
be understood by observing that, within the local formulation of the Gribov-Zwanziger
framework, expression \eqref{rr} corresponds exactly to the Bose-ghost propagator \eqref{ss1}.
In fact, integrating out the auxiliary fields $(\bar{\omega}_\mu^{ab}, \omega_\mu^{ab}, \bar{\varphi}_\mu^{ab},\varphi_\mu^{ab})$ in expression 
\begin{equation}
\int [{\cal D} {\Phi}] \; \left( \omega^{ab}_\mu(x) {\bar \omega}^{cd}_\nu(y) + \varphi^{ab}_\mu(x) {\bar \varphi}^{cd}_\nu(y) \right)   \; e^{-S_{GZ}} \;,  \label{loce}
\end{equation}
one ends up with
\begin{equation} 
\frac{ \int [{\cal D} {\Phi}] \;   \left( s \left( \varphi^{ab}_\mu(x) {\bar
\omega}^{cd}_\nu(y)  \right)   \right) \; e^{-S_{GZ}} }{ \int [{\cal D \phi}]    \;
e^{-S_{GZ}}}  =\gamma^4 \; \frac{  \int {\cal D}A\; \delta(\partial A) \; \left( det{\cal M}
\right) \; {\cal R}^{ab}_{\;\;\mu}(x)  {\cal R}^{cd}_{\;\;\nu}(y)  \; e^{-(S_{YM}+\gamma^4 H(A)
)} } {\int {\cal D}A\; \delta(\partial A) \; \left( det{\cal M} \right)   \;
e^{-(S_{YM}+\gamma^4 H(A) )} } \;. 
\label{brstbra}
\end{equation}
This equation shows  that the investigation of the correlation function \eqref{rr} with a
cutoff at the Gribov horizon is directly related to the existence of the BRST breaking. This is
precisely what has been done in \cite{Cucchieri:2014via}, where the correlator \eqref{rr} has
been shown to be non-vanishing, see Fig.1 of \cite{Cucchieri:2014via}. Moreover, from
\cite{Cucchieri:2014via}, it turns out that in the deep infrared the Fourier transform of the
correlation function \eqref{rr} is deeply enhanced, see Fig.2 of \cite{Cucchieri:2014via},
behaving as $\frac{1}{k^4}$, namely 
\begin{equation} 
\langle \tilde{\cal R}^{ab}_{\;\;\mu}(k)  \tilde {\cal R}^{cd}_{\;\;\nu}(-k)  \rangle  \sim
\frac{1}{k^4} \;.  
\label{enhanc}
\end{equation}  
As observed in \cite{Cucchieri:2014via}, this behaviour can be  understood by making use of the
analysis \cite{Zwanziger:2010iz}, {\it i.e.} of the cluster decomposition 
\begin{equation}
 \langle {\tilde {\cal R}} ^{ab}_{\;\;\mu}(k)  {\tilde {\cal R}}^{cd}_{\;\;\nu}(-k)  \rangle   \sim  g^2 {\cal G}^2(k^2) {\cal D}(k^2) \;, \label{clust} 
\end{equation} 
where ${\cal D}(k^2)$ and ${\cal G}(k^2)$ correspond to the   gluon and ghost propagators,
eqs.\eqref{Dg5},\eqref{ghrgz5}. A non-enhanced ghost propagator, {\it i.e.}  ${\cal G}(k^2)
\Big|_{k\sim 0} \sim \frac{1}{k^2}$, and an infrared finite gluon propagator, {\it i.e.} ${\cal
D}(0) \neq 0$, nicely yield the behaviour of eq.\eqref{enhanc}. 

Thus, we are going to show in this chapter that the quantity ${\cal R}$, eq.\eqref{ra}, and
the correlation function  $ \langle {\cal R}(x)  {\cal R}(y)  \rangle $, eq.\eqref{rr}, can be
consistently generalized to the case of matter fields, {\it e.g.} for the quarks and scalar
fields.




%-------------------------------------------------------------
\section{A horizon-like term to the matter field: the $\langle {\tilde {\cal R}}(k) {\tilde
{\cal R}}(-k) \rangle$ in the light of lattice data}
\sectionmark{A horizon-like term to the matter field}
%-------------------------------------------------------------


Let $F^{i}$ denote a generic matter field in a  given
representation of $SU(N)$, specified by the generators $(T^a)^{ij}$, $a=1,..,(N^2-1)$, and let
${\cal R}^{ai}(x)$ stand for the quantity
\begin{equation}
{\cal R}^{ai}(x)  =  g \int d^4z\;  [{\cal M}^{-1}]^{ab} (x,z)   \;(T^b)^{ij} \;F^{j}(z)   \label{rmatter}  \;, 
\end{equation}
which is a convolution of the inverse Faddeev-Popov operator with a given colored matter field,
being clearly the matter counterpart of the operator ${\cal R}^{ab}_{\;\;\mu}$ in the pure
gauge case. We shall be able to prove that, in analogy with the case of the gauge field
$A^a_\mu$, a non-trivial correlation function 
\begin{equation} 
 \langle {\cal R}^{ai}(x)  {\cal R}^{bj}(y)  \rangle    \;, \label{rcm} 
\end{equation} 
can be obtained from a local and renormalizable action  which is constructed by adding to the
starting conventional matter action a non-local term which shares great similarity with the
horizon function $H(A)$, eq.\eqref{hf2}, namely 
\begin{equation}
{g^{2}}   \int d^{4}x\;d^{4}y\; F^{i}(x) (T^a)^{ij} \left[ {\cal M}^{-1}\right]^{ab} (x,y)
(T^b)^{jk} F^{k} (y) \;.    
\label{hmatter}
\end{equation} 
The introduction of such \emph{horizon-like} functional into the sector of the matter field of
the action has the physical meaning of a non-local mass term, due to the inverse of the FP
operator, which would account for non-perturbative features of the matter sector. Therefore,
the proposed non-local effective action would looks like,
\begin{eqnarray}
S_{non-loc} &=& \iint d^4x\,d^{4}y\; \Biggl\{
\frac{1}{4}F^a_{\mu\nu}(x) F^a_{\mu\nu}(y)
+ b^{a}(x)\partial_{\mu}A^{a}_{\mu}(y)
+ \bar{c}^{a}(x)\partial_{\mu}D^{ab}_{\mu}(x,y)c^{b}(y)
\nonumber \\
&+&
\text{T}[F^{i}](x,y)
- U[F^{i}](x,y)
+ {g^{2}}\gamma^{4} f^{abc}A_{\mu}^{b}(x)\left[ {\cal M}^{-1}\right]^{ad}
(x,y)f^{dec}A_{\mu}^{e}(y) 
\nonumber \\
&+&
{g^{2}}\sigma^{4}  F^{i}(x) (T^a)^{ij} \left[ {\cal M}^{-1}\right]^{ab} (x,y)
(T^b)^{jk} F^{k} (y)
-\gamma^{4}4(N^{2}-1)
\Biggr\}\,,
\label{mnlocalact}
\end{eqnarray}
whence $\text{T}[F^{i}](x,y)$ accounts for the kinetic term of the matter sector, and
$U[F^{i}](x,y)$ stands for the potential term. The by-hand introduced parameter $\sigma^{2}$
has dimension of $\pmb{[mass]}^{2}$, just as the GZ parameter $\gamma^{2}$, although being a
free parameter of the theory.

As it happens in the case of the Gribov-Zwanziger theory, the non-local action
\eqref{mnlocalact} can be cast in a local form by means of the introduction of suitable
auxiliary fields. The resulting local action enjoys a large set of Ward identities which
guarantee its renormalizabilty (take a look at the next chapter). The introduction of the term \eqref{hmatter} deeply modifies
the infrared behavior (IR) of the correlation functions of the matter fields, while keeping
safe the well known UV perturbative results. One of the most interesting outcomes of this
procedure is that the matter's propagators are of the confining type, displaying positivity
violation, while being in good agreement with the available lattice data, as in the case of the
scalar matter fields, \cite{Maas:2011yx,Maas:2010nc}, and of quarks
\cite{Furui:2006ks,Parappilly:2005ei}.

Moreover, relying on the numerical data for the two-point correlation functions of quark and
scalar fields, the \emph{vev} \eqref{rcm} turns out to be non-vanishing and, interestingly
enough, it appears to behave exactly as the Boson-ghost propagator \eqref{enhanc} of the gauge
sector in the  deep IR, \emph{i.e.}
\begin{equation} 
 \langle {\tilde {\cal R}} ^{ai}(k)  {\tilde {\cal R}}^{bj}(-k)  \rangle  ~\sim~ \frac{1}{k^4}
\;. 
\label{menhanc}
\end{equation} 
Furthermore, just as in the case of the gauge sector, expression \eqref{rcm} signals the
existence of a (soft) BRST breaking in the matter field sector of the theory.


In the next section we shall show how the correlation function $\langle {\cal R}^{ai}(x) {\cal R}^{bj}(y)  \rangle $ can be obtained from a local and renormalizable action exhibiting a soft breaking of the BRST invariance in the matter sector.






%-----------------------------------------------------------------------------------------
\section{The local version of the proposed model and the analysis of $\langle {\tilde
{\cal R}}(k) {\tilde {\cal R}}(-k) \rangle$}
\sectionmark{The local version of the proposed model}
\label{localhorizoninmatter}
%-----------------------------------------------------------------------------------------

Useful quantities in QFT can only be obtained through a local (and renormalizable) action, such
as the $n$-point correlation functions, \emph{vev} of composite operators and the vacuum
energy. Therefore, since we have proposed an effective non-local action for the matter field,
in order to describe non-perturbative features of matter, it is very important to check, and
prove, that the proposed action can be recast in a local form. To achieve this goal, a couple
of auxiliary fields must be introduced, just as in the gauge sector. Furthermore, after
properly localizing the action, the propagator of the matter field will be derived in a Refined
theory, where dynamical condensates of the auxiliary fields are taken into account. Naturally,
the existence --- energetically favorable --- of such condensates is also checked. This
procedure will be developed in both example cases, for the scalar field and for the quark
field.


%Now that we have established that available lattice data for different propagators of
%gauge-interacting matter seems to be qualitatively compatible with the non-trivial $\langle
%{\tilde {\cal R}}(k) {\tilde {\cal R}}(-k) \rangle\sim 1/k^4$ behavior, let us discuss how the
%correlation function \eqref{rcm} can be obtained through a local and renormalizable action. In
%this section, the example of a real scalar field $\phi^a$ in the adjoint representation of the
%gauge group will be worked out in detail. 








\subsection{The scalar field in the adjoint representation } 

We start by considering the following non-local action  
\begin{eqnarray}
\label{acs}
S^{\phi} &=& \int d^4x\; \left(
 \frac{1}{2}(D^{ab}_{\mu}\phi^{b})^{2} + \frac{m^2_{\phi}}{2} \phi^a \phi^a 
+ \frac{\lambda}{4!}(\phi^{a}\phi^{a})^{2} \right)   + 
\nonumber \\
&+&
{g^{2}} \sigma^4  \int d^{4}x\;d^{4}y\;
f^{abc}\phi^{b}(x)\left[ {\cal M}^{-1}\right]^{ad} (x,y)f^{dec}\phi^{e}(y) \;, 
\end{eqnarray}
where, once again, $\sigma$ is a massive parameter which, to some extent, plays a role akin to that of the
Gribov parameter $\gamma^2$ of the Gribov-Zwanziger action. eq.\eqref{sgz2}. It should be
noticed that, despite of any mathematical similarity with the Gribov-Zwanziger's parameter,
$\gamma^{2}$, $\sigma^{2}$ has no dynamical origin, nor geometrical interpretation, until now.
However, they indeed share algebraic similarities, so that
most of the tools already known from GZ framework can be used here for the matter sector.

Following, then, the same procedure adopted in the case of the Gribov-Zwanziger action, it is
not difficult to show that the non-local action \eqref{acs} can be cast in a local form. This
is achieved by introducing  a set of auxiliary fields $(\tilde{\eta}^{ab},\eta^{ab})$,
$(\tilde{\theta}^{ab},\theta^{ab})$, where $(\tilde{\eta}^{ab},\eta^{ab})$ are commuting fields
while  $(\tilde{\theta}^{ab},\theta^{ab})$ are anti-commuting. For the local version of
\eqref{acs} one gets 
\begin{equation}
S_{loc}^{\phi}   =     S_{0}^{\phi} + S_{\sigma}   \;, \label{lphi} 
\end{equation}
with 
\begin{eqnarray}
\label{act0}
S_{0}^{\phi}  ~=~  \int d^4x\; \bigg(  \frac{1}{2}  (D^{ab}_{\mu}\phi^{b})^{2}  + \frac{m^2_{\phi}}{2} \phi^a \phi^a
+  \frac{\lambda}{4!}(\phi^{a}\phi^{a})^{2}
+ \tilde{\eta}^{ac}(\partial_{\mu} D^{ab}_{\mu})\eta^{bc} -
\nonumber \\
-
\tilde{\theta}^{ac}(\partial_{\mu} D^{ab}_{\mu})\theta^{bc}      -
gf^{abc}(\partial_{\mu}\tilde{\theta}^{ae})(D^{bd}_{\mu}c^{d})\eta^{ce}  \bigg)  \;  
\end{eqnarray}
and
\begin{equation}
S_{\sigma}  =  \sigma^{2}g  \int d^4x   \; f^{abc}\phi^{a}(\eta^{bc} + \tilde{\eta}^{bc}) \;.     \label{ss}
\end{equation}
As in the case of the Gribov-Zwanziger action, the auxiliary fields $(\tilde{\eta}^{ab},\eta^{ab})$, $(\tilde{\theta}^{ab},\theta^{ab})$ appear quadratically, so that they can be easily integrated out, giving back precisely the non-local starting expression \eqref{acs}. Moreover, in full analogy with the Gribov-Zwanziger case, the local action $S_{loc}^{\phi}$ exhibits a soft breaking of the BRST symmetry. In fact, making use of eqs.\eqref{brst1} and of 
\begin{eqnarray}
&&
s\phi^{a}=-gf^{abc}\phi^{b}c^{c} \;,    \nonumber \\
&&
s\tilde{\theta}^{ab} = \tilde{\eta}^{ab}\;, \qquad s\tilde{\eta}^{ab} =0\;, \nonumber \\
&&
s\eta^{ab}=\theta^{ab}\;, \qquad s\theta^{ab}=0\;, 
\end{eqnarray}
it follows that 
\begin{equation}
s  S_{loc}^{\phi} = \sigma^2 \Delta^{\phi}   \;, \label{bs}
\end{equation}
 where 
\begin{equation}
 \Delta^{\phi}  = g \int d^4x\; f^{abc} \left( -g f^{amn} \phi^{m} c^n (\eta^{bc} + \tilde{\eta}^{bc}) + \phi^a \theta^{bc}     \right)   \;. 
\label{dphi}
\end{equation}
Being of dimension two in the fields (smaller than the space-time dimension $4$, in general),
the breaking term  $\Delta^{\phi} $ \eqref{dphi} is in fact a soft breaking. 

Now the local action \eqref{lphi} is added to the Gribov-Zwanziger action \eqref{sgz2},
obtaining 
\begin{eqnarray}
\label{actlc}
S_{ loc} &=& \int d^4x\; \Biggl\{
\frac{1}{4}F^a_{\mu\nu} F^a_{\mu\nu}
+ b^{a}\partial_{\mu}A^{a}_{\mu}
+ \bar{c}^{a}\partial_{\mu}D^{ab}_{\mu}c^{b}
+ \frac{1}{2}(D^{ab}_{\mu}\phi^{b})^{2}  + \frac{m^2_{\phi}}{2} \phi^a \phi^a
+ \frac{\lambda}{4!}(\phi^{a}\phi^{a})^{2}
\nonumber \\
&&
+ \varphi^{ac}_{\nu}\partial_{\mu}D^{ab}_{\mu}\bar{\varphi}^{bc}_{\nu}
- \omega^{ac}_{\nu}\partial_{\mu}D^{ab}_{\mu}\bar{\omega}^{ac}_{\nu}
+ \gamma^{2}gf^{abc}A^{a}_{\mu}(\varphi^{bc}_{\mu} + \bar{\varphi}^{bc}_{\mu})
- gf^{abc}(\partial_{\mu}\bar{\omega}^{ae}_{\nu})(D^{bd}_{\mu}c^{d})\varphi^{ce}_\nu
\nonumber \\
&&
- \gamma^{4}4(N^{2}-1)
+ \tilde{\eta}^{ac}(\partial_{\mu} D^{ab}_{\mu})\eta^{bc}
- \tilde{\theta}^{ac}(\partial_{\mu} D^{ab}_{\mu})\theta^{bc}
+ \sigma^{2}gf^{abc}\phi^{a}(\eta^{bc} + \tilde{\eta}^{bc})
\nonumber \\
&&
- gf^{abc}(\partial_{\mu}\tilde{\theta}^{ae})(D^{bd}_{\mu}c^{d})\eta^{ce}
\Biggr\}
\;.
\end{eqnarray}
As it happens in the case of the Gribov-Zwanziger action, the local action  $S_{ loc}$ can be
proven to be renormalizable to all orders. This important property follows from the existence
of a large set of Ward identities which can be derived in the matter scalar sector and which
restrict very much the possible allowed counterterms.  For the sake of completeness, the
Appendix \ref{ARscalaraction} has been devoted to the detailed algebraic proof of the
renormalizability of the local action \eqref{actlc}. 

As in the case of the Gribov-Zwanziger action, expression
\eqref{actlc} is well suited to investigate the correlation function 
\begin{equation} 
 \langle {\cal R}^{ab}(x)  {\cal R}^{cd}(y)  \rangle    \;, \label{cphi} 
\end{equation} 
\begin{equation}
{\cal R}^{ab}(x)  =  g \int d^4z\;  ({\cal M}^{-1})^{ac} (x,z)   \; f^{cdb} \phi^{d}(z)   \label{rmsc}  \;, 
\end{equation}
and its relation with the soft BRST breaking in the scalar field sector, eq.\eqref{bs}. In fact, repeating the same reasoning of eqs.\eqref{ss}, \eqref{loce},\eqref{brstbra}, one is led to consider the exact BRST correlation function in the matter scalar field sector
\begin{equation} 
\langle \; s ( \eta^{ab}(x) {\tilde \theta}^{cd}(y)  \; ) \rangle_{S_{ loc}}  = \langle   \theta^{ab}(x) {\tilde \theta}^{cd}(y) + \eta^{ab}(x) {\tilde \eta}^{cd}(y)      \rangle_{S_{ loc}} \;.
\end{equation}
Integrating out the auxiliary fields $(\tilde{\theta}^{ab}, \theta^{ab}, \tilde{\eta}^{ab},\eta^{ab})$ in expression 
\begin{equation}
\int [{\cal D} {\Phi}] \; \left( \theta^{ab}(x) {\tilde \theta}^{cd}(y) + \eta^{ab}(x) {\tilde \eta}^{cd}(y) \right)   \; e^{-S_{loc}} \;,  \label{locphi}
\end{equation}
gives
\begin{equation} 
\frac{ \int [{\cal D} {\Phi}] \;   \left( s \left( \eta^{ab}(x) {\tilde \theta}^{cd}(y)
\right)   \right) \; e^{-S_{loc}} }{ \int [{\cal D} {\Phi}]    \; e^{-S_{loc}}}  =\sigma^4 \;
\frac{  \int {\cal D}A {\cal D}{\phi}\; \delta(\partial A) \; \left( det{\cal M} \right) \;
{\cal R}^{ab}(x)  {\cal R}^{cd}(y)  \; e^{-(S_{YM}+\gamma^4 H(A) + S^{\phi})} } {\int {\cal D}A
{\cal D}{\phi} \; \delta(\partial A) \; \left( det{\cal M} \right)   \; e^{-(S_{YM}+\gamma^4
H(A) + S^{\phi})}  } \;,\label{brstphi}
\end{equation}
showing that, in analogy with the case of the gauge field,  the correlation function \eqref{cphi}  with a cutoff at the Gribov horizon is directly related to the existence of the BRST breaking in the matter sector. 

We can now have a look at the two-point correlation function of the scalar field. Nevertheless, before that, an additional effect has to be taken into account. In very strict analogy with the case of the Refined Gribov-Zwanziger action, eq.\eqref{rgz}, the soft breaking of the BRST symmetry occurring in the scalar matter sector, eq.\eqref{bs}, implies the existence of a non-vanishing BRST exact dimension two condensate, namely 
\begin{equation}
\langle s ( \tilde{\theta}^{ab}(x)  {\eta}^{ab}(x) ) \rangle = \langle ( \tilde{\eta}^{ab}(x)  {\eta}^{ab}(x)  -  \tilde{\theta}^{ab}(x)  {\theta}^{ab}(x) ) \rangle \neq 0 \;.  \label{condphi}
\end{equation}
In order to show that expression \eqref{condphi} in non-vanishing, we couple the operator $( \tilde{\eta}^{ab}(x)  {\eta}^{ab}(x)  -  \tilde{\theta}^{ab}(x)  {\theta}^{ab}(x) ) $ to the local action $S_{ loc}$, eq.\eqref{actlc}, by means of a constant external source $J$, 
\begin{equation}
S_{ loc} - J \int d^4x\;  ( \tilde{\eta}^{ab}(x)  {\eta}^{ab}(x)  -  \tilde{\theta}^{ab}(x)  {\theta}^{ab}(x) )  \;, \label{cj}
\end{equation}
and we evaluate the vacuum energy $\mathcal{E}(J)$ in the presence of $J$, namely 
\begin{equation}
 e^{-V\mathcal{E}(J)}= \int {\cal D}{\Phi} \; e^{ -\left( S_{ loc} - J \int d^4x\;  ( \tilde{\eta}^{ab}(x)  {\eta}^{ab}(x)  -  \tilde{\theta}^{ab}(x)  {\theta}^{ab}(x) ) \right) }   \;. \label{ej}
\end{equation}
Thus, the condensate $\langle ( \tilde{\eta}^{ab}(x)  {\eta}^{ab}(x)  -  \tilde{\theta}^{ab}(x)  {\theta}^{ab}(x) ) \rangle$  is obtained by differentiating $\mathcal{E}(J)$ with respect to $J$ and setting $J=0$ at the end, {\it i.e.}
\begin{equation}
\frac{\partial \mathcal{E}(J)}{\partial J} \Big|_{J=0} = - \langle ( \tilde{\eta}^{ab}(x)  {\eta}^{ab}(x)  -  \tilde{\theta}^{ab}(x)  {\theta}^{ab}(x) ) \rangle    \;. \label{vj}
\end{equation}
Employing dimensional regularisation, to the first order, we have 
\begin{equation}
\mathcal{E}(J) = \frac{(N^2-1)}{2} \int \frac{ d^dk}{(2\pi)^d} \; \log\left( k^2 +m^2_{\phi} +\frac{2N\sigma^4 g^2}{k^2+J} \right)  \; + \;{\hat {\cal E}}   \;, \label{fo}
\end{equation}
where ${\hat {\cal E}} $ stands for the part of the vacuum energy which is independent from $J$. Differentiating eq.\eqref{fo} with respect to $J$ and setting $J=0$, we get 
\begin{equation}
 \langle ( \tilde{\eta}^{ab}(x)  {\eta}^{ab}(x)  -  \tilde{\theta}^{ab}(x)  {\theta}^{ab}(x) ) \rangle = (N^2-1) N \sigma^4 g^2  \int \frac{ d^dk}{(2\pi)^d} \frac{1}{k^2} 
 \frac{1}{k^4 + m^2_\phi \;k^2 +  2 N \sigma^4 g^2}   \neq 0 \;. \label{vcondphi}
\end{equation}
Notice that the integral in the right hand side of eq.\eqref{vcondphi} is ultraviolet convergent in $d=4$. Expression  \eqref{vcondphi} shows that, as long as the parameter $\sigma$ in non-vanishing, the condensate $\langle ( \tilde{\eta}^{ab}(x)  {\eta}^{ab}(x)  -  \tilde{\theta}^{ab}(x)  {\theta}^{ab}(x) ) \rangle$ is dynamically generated. The effect of the condensate 
 \eqref{condphi}  can be taken into account by adding to the action $S_{ loc}$ the novel term 
\begin{equation}
\mu^2_\phi \int d^4x \; s ( \tilde{\theta}^{ab} {\eta}^{ab} )  = \mu^2_\phi \int d^4x\;  ( \tilde{\eta}^{ab}  {\eta}^{ab}  -  \tilde{\theta}^{ab}  {\theta}^{ab} )   \;,  \label{accondphi}
\end{equation}
giving rise to the  Refined action 
\begin{equation}
{\tilde S}_{Ref} =  S_{ loc} +  \int d^4x \left(  \frac{m^2}{2} A^a_\mu A^a_\mu  - \mu^2 \left( {\bar \varphi}^{ab}_{\mu}  { \varphi}^{ab}_{\mu} -  {\bar \omega}^{ab}_{\mu}  { \omega}^{ab}_{\mu} \right)   \right)    - \mu^2_\phi \int d^4x\;  \left( \tilde{\eta}^{ab}  {\eta}^{ab}  -  \tilde{\theta}^{ab}  {\theta}^{ab} \right)  \;. \label{refphi}
\end{equation}
Finally, for the propagator of the scalar field, we get 
\begin{equation} 
\langle  \phi^a(k)  \phi^b(-k) \rangle   =   \delta^{ab}   \frac{k^2 +\mu^2_{\phi}}{k^4 +
(\mu^2_\phi+m^2_{\phi})k^2 + 2Ng^2\sigma^4 + \mu^2_\phi m^2_\phi}  \;. 
\label{phiprop}
\end{equation} 

In the subsection \ref{latticefit} we are going to fit this perturbative propagator to the
correspondent lattice data, so that the free parameters of the theory can be estimated.





\subsection{The quark field} 
In this subsection we generalise the previous construction to the case of quark fields. The  starting non-local action \eqref{acs} is now given by 
\begin{eqnarray}
\label{apsi}
S^{\psi} &=& \int d^4x\; \left( {\bar \psi}^{i} \gamma_\mu D_{\mu}^{ij} \psi^{j} - m_{\psi}  {\bar \psi}^{i} \psi^{i}  \right) 
 \nonumber \\
 &-&
 M^3 g^2   \int d^{4}x\;d^{4}y\;   {\bar \psi}^{i}(x)  (T^a)^{ij} \left[ {\cal M}^{-1}\right]^{ab} (x,y)  (T^b)^{jk} \psi^{k}(y) \;, 
\end{eqnarray}
where the massive parameter $M$ is the analogue of the parameter $\sigma$ of the scalar field and 
\begin{equation}
D^{ij}_\mu = \delta^{ij} \partial_\mu - i g  (T^a)^{ij} A^a_\mu \;, \label{covf}
\end{equation}
is the covariant derivative in the fundamental representation, specified by the generators $(T^a)^{ij}$. As in the previous case, the non-local action \eqref{apsi} can be cast in local form through the introduction of a suitable set of auxiliary fields: $({\bar \lambda}^{ai}, {\lambda}^{ai})$ and $({\bar \xi}^{ai}, {\xi}^{ai})$. The fields $({\bar \lambda}^{ai}, {\lambda}^{ai})$ are Dirac spinors  with two color indices $(a,i)$ belonging, respectively,  to the adjoint and to the  fundamental representation. Similarly, $({\bar \xi}^{ai}, {\xi}^{ai})$ are a pair of spinor fields with ghost number $(-1,1)$. The spinors  $({\bar \lambda}^{ai}, {\lambda}^{ai})$ are anti-commuting, while $({\bar \xi}^{ai}, {\xi}^{ai})$ are commuting. For the local version of the action, we get 
\begin{equation}
S^{\psi}_{loc} = S_0 + S_M \;, \label{locpsi}
\end{equation}
where 
\begin{eqnarray}
S_0  =  \int d^4x\; \left( {\bar \psi}^{i} \gamma_\mu D_{\mu}^{ij} \psi^{j} - m_{\psi}  {\bar \psi}^{i} \psi^{i}   +  {\bar \lambda}^{ai}( -\partial_\mu D^{ab}_\mu) \lambda^{bi} 
+ {\bar \xi}^{ai}( -\partial_\mu D^{ab}_{\mu} ) \xi^{bi}  \right.
\nonumber \\
\left. -(\partial_\mu {\bar \xi}^{ai}) g f^{acb} (D^{cm}_\mu c^m) \lambda^{bi}  \right)  \;, \label{szpsi}
\end{eqnarray}
and 
\begin{equation}
S_M = g M^{3/2} \int d^4x \; \left(   {\bar \lambda}^{ai} (T^a)^{ij} \psi^{j} +  {\bar \psi}^{i} (T^a)^{ij} \lambda^{aj}    \right)    \;. \label{sM}
\end{equation}
The non-local action $S^{\psi}$ is easily recovered by integrating out the auxiliary fields $({\bar \lambda}^{ai}, {\lambda}^{ai})$ and $({\bar \xi}^{ai}, {\xi}^{ai})$. As in the case of the scalar field, the term $S_M$ induces a soft breaking of the BRST symmetry. In fact, from 
\begin{eqnarray}
s \psi^{i} & = & -ig c^a (T^a)^{ij} \psi^{j} \;, \nonumber \\
s {\bar \psi}^{i} & = & -ig {\bar \psi}^{j} c^a (T^a)^{ji} \;, \nonumber \\
s {\bar \xi}^{ai} & = & {\bar \lambda}^{ai} \;, \qquad s {\bar \lambda}^{ai} = 0\;, \nonumber \\
s {\lambda}^{ai} & = & {\xi}^{ai} \;, \qquad s{\xi}^{ai}=0 \;,  \label{spsi}
\end{eqnarray}
one easily checks that 
\begin{equation}
s S^{\psi}_{loc} = s  S_M = M^{3/2}  \Delta^M   \;, \label{sbM} 
\end{equation}
where
\begin{equation}
\Delta^M = \int d^4x \; \left(  ig^2  {\bar \lambda}^{ai} (T^a)^{ij} c^b (T^b)^{jk}\psi^{k} -ig^2 {\bar \psi}^{k} c^b (T^b)^{ki}(T^a)^{ij} \lambda^{aj}  
- g {\bar \psi}^{i} (T^a)^{ij} \xi^{aj}   \right)   \;. \label{dm}
\end{equation}
Again, being of dimension $5/2$ in the fields, $\Delta^M$ is a soft breaking. In the present case, for the quantity  \eqref{rmatter} we have 
\begin{eqnarray}
{\cal R}^{ai}_{\;\alpha} (x)  & = &  g \int d^4z\;  ({\cal M}^{-1})^{ab} (x,z)   \;(T^b)^{ij} \psi^{j}_{\alpha} (z)     \;, \nonumber \\
{\bar {\cal R}}^{bj}_{\;\beta} (x)  & = &  g \int d^4z\;  ({\cal M}^{-1})^{bc } (x,z)  {\bar \psi}^{k}_{\beta}(z)  \;(T^c)^{kj}     \;, \label{rpsi}
\end{eqnarray}
where we have explicitated  the Dirac indices $\alpha,\beta=1,2,3,4$. 

As in the case of the scalar field, the action $S^{\psi}_{loc} $ can be added to the Gribov-Zwanziger action. The resulting action, $(S_{GZ} + S^{\psi}_{loc})$, turns out to be renormalizable. Although we shall not give here the details of the proof of the renormalizability of the action $(S_{GZ} + S^{\psi}_{loc})$, it is worth mentioning that it can be given by following the framework already outlined in \cite{Baulieu:2009xr}, where a similar non-local spinor action has been considered. 

Proceeding now as in the case of the scalar field, one finds 
%
\begin{eqnarray} 
	&&
	\frac{ \int [{\cal D} {\Phi}] \;   \left[ s \left( {\bar \xi}^{ai}_{\alpha}(x) {\lambda}^{bj}_\beta(y)  \right)   \right] \; e^{-(S_{GZ}+S^{\psi}_{loc})} }{ \int [{\cal D} {\Phi}]    \; e^{-(S_{GZ}+S^{\psi}_{loc})}}  =
	\nonumber \\
	&=&
	M^3 \; \frac{  \int {\cal D}A {\cal D}{\psi} {\cal D}{\bar \psi} \; \delta(\partial A)  \left( det{\cal M} \right) {\cal R}^{ai}_{\;\alpha}(x)  {\bar {\cal R}}^{bj}_{\;\beta}(y)  \; e^{-(S_{YM}+\gamma^4 H(A) + S^{\psi})} } {\int {\cal D}A {\cal D}{\psi} {\cal D}{\bar \psi} \; \delta(\partial A) \; \left( det{\cal M} \right)   \; e^{-(S_{YM}+\gamma^4 H(A) + S^{\psi})}  } \;,\label{brstpsi}
\end{eqnarray}
%
showing that  the correlation function $\langle {\cal R}^{ai}_{\;\alpha}(x)  {\bar {\cal R}}^{bj}_{\;\beta}(y) \rangle$  with a cutoff at the Gribov horizon is  related to the existence of the BRST breaking, eq.\eqref{sbM}. 

Let us end this section by discussing the two-point correlation function of the quark field. As before, an additional effect has to be taken into account. Also here,  the soft breaking of the BRST symmetry, eq.\eqref{sbM}, implies the existence of a non-vanishing BRST exact dimension two condensate, namely 
%
\begin{equation}
\langle s ( {\bar {\xi}}^{ai}(x)  {\lambda}^{ai}(x) ) \rangle = \langle ( {\bar \lambda}^{ai}(x)  {\lambda}^{ai}(x)  + {\bar  \xi}^{ai}(x)  {\xi}^{ai}(x) ) \rangle \neq 0 \;,  \label{condpsi}
\end{equation}
%
whose effect can be taken into account by adding to the action $S^{\psi}_{loc}$ the term 
%
\begin{equation}
\mu^2_\psi \int d^4x \; s   ( {\bar {\xi}}^{ai}(x)  {\lambda}^{ai}(x) )   = \mu^2_\psi \int d^4x\;  ( {\bar \lambda}^{ai}(x)  {\lambda}^{ai}(x)  + {\bar  \xi}^{ai}(x)  {\xi}^{ai}(x) )  \;.  \label{accondpsi}
\end{equation}
%
Therefore, including the dimension two condensates, we end up with the Refined action 
%
\begin{equation}
	{\tilde S}_{Ref}^{\psi} =  S_{RGZ} + S^{\psi}_{loc} +  \mu^2_\psi \int d^4x\;  \left[ {\bar \lambda}^{ai}(x)  {\lambda}^{ai}(x)  + {\bar  \xi}^{ai}(x)  {\xi}^{ai}(x) \right]   \;. \label{refpsi}
\end{equation}
%
Finally, for the propagator of the quark field, we get 
%
\begin{equation} 
\langle  \psi^{i}(k)  {\bar \psi}^{j}(-k) \rangle   =   \delta^{ij} \;  \frac{-ik_\mu
\gamma_\mu + {\cal A}(k^2)}{k^2 + {\cal A}^2(k^2)}  \;, \label{psiprop}
\end{equation} 
%
 where 
%
\begin{equation}
{\cal A}(k^2) = m_{\psi} + \frac{g^2 M^3 C_F}{k^2+\mu^2_\psi} \;, \label{A}
\end{equation}
%
and 
%
\begin{equation}
 (T^a)^{ij} (T^a)^{jk} = \delta^{ik} C_F \;, \qquad C_F= \frac{N^2-1}{2N}  \;.    \label{norm}
\end{equation}
%
%Expression \eqref{psiprop} is of the the same kind employed  to fit the lattice data.  


In the following section \ref{latticefit} we are going to fit our perturbative Refined matter
propagators \eqref{phiprop} and \eqref{psiprop} with the most recent curves of scalar and quark
propagators from the lattice data. As will be shown, for the fitted parameters of the theory
both, the scalar and quark, propagators exhibit positivity violation, so that they do not
belong to the spectrum of the asymptotically free physical particles of the theory; they are
said to be confined.

















%---------------------------------------------------------------------------------
\section{Analysis of $\langle {\tilde {\cal R}}(k) {\tilde {\cal R}}(-k) \rangle$ in the light
of the available lattice data}
\sectionmark{Analysis of $\langle {\tilde {\cal R}}(k) {\tilde {\cal R}}(-k) \rangle$}
\label{latticefit}
%---------------------------------------------------------------------------------



In the present section we present a discussion of the correlation function \eqref{rcm}  in
the case of  quark and scalar fields, relying on the available lattice data for the quark and
scalar propagators. This will be done by working out in detail the case of a scalar field in
the adjoint representation. We shall also discuss how $\langle {\cal R}^{ai}(x)  {\cal
R}^{bj}(y)  \rangle $ encodes information on the soft  breaking of the BRST symmetry. In the
same section we generalize the previous construction  to the case of quark fields. The final Appendix collects  the details of the algebraic proof of
the renormalizability of the local action obtained by the addition of the term  \eqref{hmatter}
in the case of a scalar matter field in the adjoint representation.  


Let us, then, investigate the correlation function $\langle {\tilde {\cal R}}(k) {\tilde {\cal R}}(-k) \rangle$, that signals soft BRST breaking in the matter sector, in  light of available lattice data for gauge-interacting matter propagators in the Landau gauge.

As in the pure gauge case, one may rely on the general cluster decomposition property in order
to obtain the leading behavior in the deep infrared region. The point is that, in one side we
have the highly non-local operator
\begin{eqnarray}
{\cal R}^{ai}(x)  {\cal R}^{bj}(y) ~=~ g^{2} \iint d^{4}z\,d^{4}z'\,\, [{\cal
M}^{-1}]^{ad} (x,z)   \;(T^d)^{il} \;F^{j}(z) \;(T^e)^{jl} \;F^{j}(z')  \,[{\cal M}^{-1}]^{be}
(y,z')\,,
\end{eqnarray}
whose non-locality stems from the squared inverse of the FP operator. In the other side we have
\begin{eqnarray}
\langle c^{a}(y)\bar{c}^{b}(x) \rangle ~=~ [{\cal M}^{-1}]^{ab} (x,y)\,,
\end{eqnarray}
so that the non-local operator may be rewritten as,
\begin{eqnarray}
{\cal R}^{ai}(x)  {\cal R}^{bj}(y) ~=~ \iint d^{4}z\,d^{4}z' 
\langle c^{a}(z)\bar{c}^{d}(x) \rangle  \langle c^{e}(z')\bar{c}^{b}(y) \rangle \;(T^d)^{il}
\;F^{j}(z) \;(T^e)^{jl} \;F^{j}(z')  \,.
\label{RRop}
\end{eqnarray}
Therefore, the \emph{vev} of this operator can be written as
\begin{eqnarray}
\langle {{\cal R}}^{ai}(x) {{\cal R}}^{bj}(y)\rangle 
~=~ 
{g^{2}}   \int d^{4}z\;d^{4}z'\; \langle \bar{c}^{a}(x) c^{a'}(z) F^{i'}(z) (T^{a'})^{i'i}
\bar{c}^b(y) c^{b'}(z') (T^{b'})^{j'j} F^{j'} (z') 
\rangle
\end{eqnarray}
whence the cluster decomposition principle applies to the ghost and matter propagators,
yielding to
\begin{eqnarray}
\langle {{\cal R}}^{ai}(x) {{\cal R}}^{bj}(y)\rangle 
&=& g^2  (T^{a})^{i'i} (T^{b})^{i'j}\int d^4k {\rm e}^{ik(x-y)}{\cal G}(k^{2})D(k^2)+
\nonumber\\
&+&
{g^{2}}   \int d^{4}z\;d^{4}z'\; \langle \bar{c}^{a}(x) c^{a'}(z) F^{i'}(z) (T^{a'})^{i'i}
\bar{c}^b(y) c^{b'}(z') (T^{b'})^{j'j} F^{j'} (z') 
\rangle_{1PI}
\;.
\label{<RR>}
\end{eqnarray}
The cluster decomposition principle could be seen as a reflection of the non-locality of the
operator ${{\cal R}}^{ai}(x) {{\cal R}}^{bj}(y)$. Notice that the operator \eqref{RRop} depends
on two non-local quantities,$\langle c^{a}(z)\bar{c}^{d}(x) \rangle$ and $\langle
c^{e}(z')\bar{c}^{b}(y) \rangle$, measured in two unrelated points $x$ and $y$.

On equation \eqref{<RR>} ${\cal G}(k^2)$ is the ghost propagator, while $D(k^2)$ now stands for the propagator of
the associated matter field. The one-particle-irreducible (1PI) contribution above (the second
term) becomes subleading in the IR limit, since in this case the points $x$ and $y$ are largely
separated and
the cluster decomposition applies. This can also be seen diagrammatically: since the external legs
are ghosts, these corrections will involve at least two ghost-gluon vertices, that carry a
derivative coupling.
In fact, as a consequence of the transversality  of the gluon propagator, factorization of the
external momentum takes place, implying the subleading character of the 1PI contributions.

Therefore, in the limit $k\to 0$, the (full) ghost and matter propagators alone dictate the momentum-dependence of the correlation function $\langle {\tilde {\cal R}}(k) {\tilde {\cal R}}(-k) \rangle$, i.e.
\begin{eqnarray}
\langle \tilde{\cal R}^{ai}(k) \tilde{\cal R}^{bj}(-k)
\rangle
&\sim& g^2 {\cal G}^2(k)D(k^2)
\;.
\end{eqnarray}
Having in mind the non-enhanced ghost propagator, ${\cal G}(k^2)\sim 1/k^2$ (as observed in high-precision pure gauge simulations in the Landau gauge \cite{Cucchieri:2007rg,Cucchieri:2008fc,Cucchieri:2011ig}), it is straightforward to conclude that a finite zero-momentum value for the matter propagators is a sufficient condition for a $\sim 1/k^4$  behavior of the correlation function $\langle {\tilde {\cal R}}(k) {\tilde {\cal R}}(-k) \rangle$ in the deep IR.

As we shall see in the following subsections, both scalar and fermion propagators display, when coupled to non-Abelian gauge fields, a shape compatible with a finite zero-momentum value in the currently available lattice data. We expect thus a $\sim 1/k^4$ behavior of  the correlation function $\langle {\tilde {\cal R}}(k) {\tilde {\cal R}}(-k) \rangle$ in the matter sector, being in this sense a universal property associated with the Faddeev-Popov operator -- when coupled to any colored field --  in confining Yang-Mills theories that can be easily probed in the future via direct lattice measurements.

Moreover, fits of the lattice data are presented for adjoint scalars in
subsection \ref{scalars} and for fermions in subsection \ref{quarks}. This analysis shows that the propagators for gauge-interacting scalars and fermions are compatible not only with a finite zero-momentum limit, but also with a complete analytical form that can be extracted from an implementation of soft BRST breaking in the matter sector to be presented below, in Sect.3. 




\subsection{The scalar field in the adjoint representation }
\label{scalars}

In this subsection we consider real scalar fields coupled to a confining Yang-Mills theory:
\begin{eqnarray}
{\cal L} ~=~ \frac{1}{4} F_{\mu\nu}^aF_{\mu\nu}^a +\frac{1}{2} [D_{\mu}^{ab}\phi^b]^2 + \frac{m_{\phi}^2}{2}\phi^a\phi^a +\frac{\lambda}{4!} [\phi^a\phi^a]^2  + {\cal L}_{GF}\;,
\end{eqnarray}
where  ${\cal L}_{GF}$ is the Landau gauge fixing term and $\phi$ is a real scalar field in the adjoint representation of $SU(N)$ and there is no Higgs mechanism, namely $\langle\phi\rangle = 0$.  

We are interested in analyzing the infrared non-perturbative regime, focussing especially on
the adjoint scalar propagator. We resort to the lattice implementation of this system:
currently available in the quenched approximation with the specific setup described in
\cite{Maas:2010nc}. Preliminary and unpublished data points for larger lattice sizes (with
lattice cutoff $a^{-1}=4.94 $ GeV and $N=30$ lattice sites) \cite{axel} are displayed in Fig. 1
for different values of the bare scalar mass ($m_{bare}=0,\,1,\,10$ GeV). It should be noticed
that this data is unrenormalized in the lattice sense. The renormalization procedure that fixes
the data to a known renormalization scheme and the resulting points will be discussed below.

%
\begin{figure}[h!]
%\vspace{1cm}
\center
\includegraphics[width=9cm]{fit-unrenorm.pdf}
\caption{Unrenormalized propagator for different bare masses of the scalar field:
$m_{bare}=0$(top, black), 1 and 10 GeV (bottom, red). The points are preliminary and
unpublished lattice data from quenched simulations \cite{axel} (for lattice cutoff $a^{-1}=4.54
$ GeV, $N=30$ and $\beta=2.698$; cf. also \cite{Maas:2010nc} for more details on the lattice
setup and measurements) and the curves are the corresponding fits, whose parameter values can
be found in Table \ref{table:par}.}
\label{scalarlattfig}
\end{figure}
%


First of all, the data tends to show a finite zero-momentum value for the scalar propagator,
irrespective of its bare mass. This indicates   -- together with the well-stablished
non-enhanced ghost propagator -- that the correlation function $\langle \tilde{R} \tilde
R\rangle_k$ is indeed non-vanishing in the IR limit, presenting the power-law enhancement
$\sim1/k^4$ that we have anticipated above.

The curves in Fig. \ref{scalarlattfig} further show that the data is compatible with fits of a
propagator of the same type as the one we found in \eqref{phiprop},
%
\begin{eqnarray}
D(p) ~=~ Z\,
\frac{p^2+\mu_{\phi}^2}{p^4+p^2(m_{\phi}^2+\mu_{\phi}^2)+\sigma^4+m_{\phi}^2\mu_{\phi}^2}
\,,\label{RGZfit}
\end{eqnarray}
%
where $Z,\mu_{\phi},m_{\phi},\sigma$ are the fit parameters, whose values are presented in
Table \ref{table:par}. In this case we may extrapolate the fits in order to obtain the specific values at
zero momentum: $D(p=0)\approx 0.028,\, 0.027,\, 0.0073$ GeV$^{-2}$, for the bare mass $m_{bare}
= \unit{0,1,10}{\giga\electronvolt}$, respectively, so that the non-trivial IR limit is clear.
Moreover, the $\sigma$ parameter -- which is directly related to the non-vanishing of the
\emph{vev} of an exact BRST local operator, $\langle s(\eta^{ab}(x)\tilde{\theta}^{cd}(x))
\rangle \neq 0$ \eqref{brstphi} -- seems to be non-vanishing.  It is also interesting to point
out that the obtained fits correspond to a combination of two complex-conjugate poles for all
values of bare scalar mass, indicating the absence of a K\"all\'en-Lehmann spectral
representation for this two-point function and the presence of positivity violation. In this
sense the adjoint scalar propagators  consistently represent confined degrees of freedom, that
do not exhibit a physical propagating pole.
    
    \begin{table}[ht]
 \caption{Fit parameters for the unrenormalized propagator in powers of GeV.}
 \vspace{0.3cm}
  \centering
   \begin{tabular}{c ||c| c| c| c||c}
    $m_{bare}$ & $\mu_{\phi}^2$ & $m_{\phi}^2$& $\sigma^4$ & $Z$&$\chi^2/\textrm{dof}$\\
    \hline\hline 
    0 & 120
    & 0 & 4913 & 1.137 &0.31\\
      \hline 
    1 & 46 & 34  & 644 & 1.28 & 1.84\\
      \hline 
    10 & 88
    & 158 & 1267 & 1.26 & 0.10
     \end{tabular}
     \label{table:par} 
    \end{table}
    
An important issue to be addressed is the possibility of scheme dependence of those findings. To check for this, we have also analyzed the scalar propagators after renormalization in another scheme.
As usual, renormalization is implemented through the inclusion of mass $\delta m_{\phi}$ and wave-function renormalization $\delta Z $ counterterms:
%%
\begin{eqnarray}
D_{ren}^{-1}(p)&=& D^{-1}(p) +\delta m_{\phi}^2 +\delta Z (p^2+m_{bare}^2)
\,,
\end{eqnarray}
%
%
where the counterterms are obtained by imposing the following renormalization conditions (for $\Lambda=2$ GeV):
\begin{enumerate}
	\item[i)] $\partial_{p^2}D_{ren}^{-1}(p=\Lambda)=1 $;
	\item[ii)] $D_{ren}^{-1}(p=\Lambda)=\Lambda^2+m_{bare}^2$.
\end{enumerate}
The fit functions were used to compute the counterterms and the renormalized points are obtained from the original lattice data by adding the same counterterms\footnote{Direct renormalization of lattice data was avoided, since we did not have access to the measurement of $\partial_{p^2}D$ and the number of data points available was not sufficient for a reliable numerical derivative to be computed.}. Results are shown in Fig. 2 and Table 2.


    %
\begin{figure}[h!]
%\vspace{1cm}
\center
\includegraphics[width=10cm]{fit-renorm.pdf}
\caption{Renormalized propagator for different bare masses of the scalar field: $m_{bare}=0$(top, black), 1 and 10 GeV (bottom, red). The points are obtained from the unrenormalized lattice data  \cite{axel,Maas:2010nc} displayed in Fig.1.}
\end{figure}
%

The renormalized propagator may be rewritten in the form \eqref{RGZfit}, with redefined parameters $m_{\phi}',\sigma',Z'$:
\begin{eqnarray}
D_{ren}(p)&=&
Z'\, \frac{p^2+\mu_{\phi}^2}{p^4+p^2(m_{\phi}'^2+\mu_{\phi}^2)+\sigma'^4+m_{\phi}'^2\mu_{\phi}^2}
\end{eqnarray}    
    
        \begin{table}[ht]
 \caption{Counterterms, redefined fit parameters and zero-momentum values of the renormalized propagator in powers of GeV.}
 \vspace{0.3cm}
  \centering
   \begin{tabular}{c ||c| c|| c| c|c||c}
    $m_{bare}$ & $\delta m_{\phi}^2$ & $\delta Z$& $m_{\phi}'^2$ & $\sigma'^4$ & $Z'$& $D_{ren}(p=0)$\\
   \hline\hline 
    0 & -35.98    & 0.40 & -28.09 & 3374.32 & 0.781&26.7\\
      \hline 
    1 & -36.49 & 0.416 & -8.18  & 420.84 & 0.834 &0.94\\
      \hline 
    10 &  -69.69    & 0.322 & 79.19 & 902.23 & 0.894 &0.01
     \end{tabular}
     \label{table:CT} 
    \end{table}



All the interesting qualitative properties observed in the unrenormalized data remain valid, namely: $(i)$ finite IR limit, $(ii)$ compatibility with 4-parameter fits of the same form, with non-trivial $\sigma$ values, $(iii)$ the fit parameters yield complex-conjugate poles, so that the renormalized propagator is still compatible with positivity violation and confinement.

We underline that the present analysis for the scalar fields is meant to be a preliminary study
of the propagator. As such, the results are still at the qualitative level. A more quantitative
analysis would require further simulations with improved statistics and even larger lattices.


\subsection{The quark field}
\label{quarks}

Now we consider the case of gauge-interacting fermionic fields coupled to a confining Yang-Mills theory. Of course, the case of QCD is the emblematic example. 
We will verify that the same qualitative properties shown above for scalar fields can also be found in this case, indicating that the IR enhancement of the correlation function $\langle\tilde {\cal R}\tilde {\cal R}\rangle\sim 1/k^4$ seems to be universally present in the confined matter sector.

The fermionic propagator is decomposed as usual,
\begin{eqnarray}
{\cal S}(p)=Z(p^2)\frac{-ip_{\mu}\gamma_{\mu}+{\cal A}(p^2)}{p^2+{\cal A}(p^2)}\,,
\end{eqnarray}
and our interest resides solely on the mass function ${\cal A}(p^2)$, whose lattice data will be analyzed here.


\begin{figure}[h!]
   \centering
       \includegraphics[width=9cm]{massas.png}
               \caption{Lattice quark mass function \cite{Parappilly:2005ei} with its fit ${\cal A}(p^2)$. Figure extracted from \cite{Dudal:2013vha}; fit obtained by O. Oliveira \cite{Orlando}.}
\end{figure}


As already discussed and shown in \cite{Dudal:2013vha}, the data of \cite{Parappilly:2005ei} for the mass function of the propagator of degenerate up ($u$) and down ($d$) quarks with current mass $\mu=0.014~\text{GeV}$ can be fitted excellently with
\begin{equation}\label{fit}
{\cal A}(p^2)=\frac{M^3}{p^2+m^2}+\mu~\text{with}~M^3=0.1960(84)~\text{GeV}^3\,, m^2=0.639(46)~\text{GeV}^2 \quad (\chi^2/\text{d.o.f.}~=~1.18)\,.
\end{equation}
as can be seen in Fig. 3.
 
The quark propagator presents clearly a finite IR limit. This is, in fact,  well-known in QCD
as dynamical mass generation and is intimately related to chiral symmetry breaking.
Interestingly enough, this is also a sufficient condition -- supposing a non-enhanced ghost
propagator -- for the soft BRST breaking in the quark sector through the IR enhancement of the
correlation function $\langle \tilde{\cal R}\tilde{\cal R}\rangle$. Again, we predict a $\sim
1/k^4$ IR scaling for this observable, now in the quark sector. This suggests a close relation
between soft BRST breaking and chiral symmetry breaking, and may provide an interesting
underlying connection between confinement and chiral symmetry breaking.









\section{Discussions about the results\label{conc4}}

One of the striking features of the (R)GZ formulation of non-perturbative Euclidean continuum
Yang-Mills theories is the appearance of the soft breaking of the BRST symmetry, which seems to
be deeply related to gluon confinement. Recently, direct lattice investigations
\cite{Cucchieri:2014via} have confirmed the existence of such a breaking, through the analysis
of the Boson-ghost correlation function:
\begin{eqnarray}
\langle \tilde {\cal R}^{ab}_{\;\;\mu}(k)\tilde {\cal
R}^{cd}_{\;\;\nu}(-k)\rangle&\stackrel{k\to 0}{\sim}&\frac{1}{k^4} \,,
\label{RRgluon}
\end{eqnarray}
with
\begin{eqnarray}
{\cal R}^{ac}_{\;\;\mu}(x)&=&g \int d^4z ({\cal M}^{-1})^{ad}(x,z) f^{dec}A^e_{\mu}(z)
\,.
\end{eqnarray}
As pointed in \cite{Cucchieri:2014via}, the non-vanishing of such correlator signals the
breaking of the BRST invariance, since it is related with the \emph{vev} of an exact BRST local
operator,
\begin{eqnarray}
\langle s(\varphi^{ab}_{\mu}(x)\bar{\omega}^{cd}_{\nu}(y))  \rangle_{GZ} ~=~ \gamma^{4} \langle
{\cal R}^{ab}_{\;\;\mu}(x) {\cal R}^{cd}_{\;\;\nu}(y)\rangle_{GZ}\,.
\end{eqnarray}
Interestingly enough, the behavior \eqref{RRgluon}, in the gauge sector, is in quite good agreement with the RGZ framework.

Inspired by the gauge sector, we proposed an effective model for the matter sector, where a
similar structure to that one of the gauge sector can be consistently implemented. It must be
clear that we had no geometrical motivation, or any ambiguity issues in the matter field
quantization procedure, that would have led us to this effective model. However, it seems to be
reasonable that, in some sense, non-perturbative features of the gauge field play any role in
the non-perturbative feature of the matter field \footnote{More on the interplay between
non-perturbative features of the gauge sector and quark confinement will be treated in the
chapter \ref{Ploop1}.}. Therefore, within the framework of (R)GZ quantization procedure, the
fate of restricting the space of configuration of the gauge field to the first Gribov region,
$\Omega$, may be reflected in the matter sector.

Two main interesting cases were considered in this chapter, the adjoint scalar and the quark
fields. In these cases we could show that it is possible to construct an analogous operator ${\cal
R}^{ai}_{\;\;F}$ for matter field,
\begin{eqnarray}
{\cal R}^{ai}_{\;\;F}(x)  &=&  g \int d^4z\;  ({\cal M}^{-1})^{ab} (x,z)   \;(T^b)^{ij}
\;F^{j}(z)  
\,,
\end{eqnarray}
so that the correlation function $\langle{\cal R}_{\;\;F}{\cal R}_{\;\;F}\rangle$ is
non-vanishing and, from the available lattice data, seems to behave like the Boson-ghost
propagator in the IR regime, \eqref{RRgluon}, namely
\begin{eqnarray}
\langle \tilde {\cal R}^{ai}_{\;\;F}(k)\tilde {\cal R}^{bj}_{\;\;F}(-k)\rangle&\stackrel{k\to 0}{\sim}&\frac{1}{k^4} 
\,.
\end{eqnarray}
Again, the non-vanishing of $\langle{\cal R}_{\;\;F}{\cal R}_{\;\;F}\rangle$ indicates the soft
breaking of the BRST symmetry in the matter sector, since the \emph{vev} of ${\cal
R}_{\;\;F}{\cal R}_{\;\;F}$ can be written in terms the \emph{vev} of a BRST exact local
operator of the localizing fields,
\begin{eqnarray}
\langle s(\eta^{ab}(x)\bar{\theta}^{cd}(y))  \rangle ~=~ \gamma^{4} \langle
{\cal R}^{ab}(x) {\cal R}^{cd}(y)\rangle\,.
\end{eqnarray}
In this sense, the correlation function $\langle{\cal R}_{\;\;F}{\cal R}_{\;\;F}\rangle$ could
be regarded as a direct signature for BRST breaking, being accessible both analytically as well
as through numerical lattice simulations. 

Concerning the analytic side, we have been able to construct a local and renormalizable action
including matter fields which accommodates the non-trivial correlation functions $\langle{\cal
R}_{\;\;F}{\cal R}_{\;\;F}\rangle$. Our analysis further suggests that the inverse of the
Faddeev-Popov operator ${\cal M}^{-1}$, whose existence is guaranteed by the restriction to
the first Gribov region $\Omega$ of the gauge field, couples in a universal way to any coloured
field $G^i$ ({\it e.g.} gluon and matter fields),
\begin{eqnarray}
{\cal R}^{ai}_{\;\;G}(x)  &=&  g \int d^4z\;  ({\cal M}^{-1})^{ab} (x,z)   \;(T^b)^{ij}
\;G^{j}(z) \,,
 \label{RG}
\end{eqnarray}
giving rise to a non-vanishing correlation function 
\begin{eqnarray}
\langle \tilde {\cal R}_{\;\;G}(k)\tilde {\cal R}_{\;\;G}(-k)\rangle&\stackrel{k\to
0}{\sim}&\frac{1}{k^4} \,.
\label{RRG}
\end{eqnarray}

The construction carried out here was restricted to the Landau gauge, although something
similar could be developed in other gauges, \emph{e.g.} the Maximal Abelian Gauge
\cite{Capri:2015pxa}, or even in the wider class of Linear Covariant Gauges, in a framework that
lives invariant the action under a non-perturbative version of the BRST symmetry
\cite{Capri:2015ixa,Capri:2015nzw,Capri:2016aqq}.































