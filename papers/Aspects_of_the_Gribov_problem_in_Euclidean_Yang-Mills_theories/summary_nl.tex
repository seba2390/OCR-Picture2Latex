

%-----------------------------------
\chapter*{Nederlandse Samenvatting}
\addcontentsline{toc}{chapter}{Dutch Summary}
%-----------------------------------



De inhoud van deze thesis is gebaseerd op de artikels
\cite{Capri:2013oja,Capri:2013gha,Capri:2012ah,Capri:2012cr,Capri:2014jqa,Canfora:2015yia,Capri:2015mna}
en gaat over de studie van bepaalde aspecten van het Gribov probleem in Euclidische Yang-Mills
theorie\"en qin de aanwezigheid van materievelden. In het bijzonder geven we enkele sterke
aanwijzingen, zowel wiskundig als fysisch, dat verder suggereert dat er een wederzijdse invloed
op de ijkveldynamica is ten gevolge van de materievelden, en omgekeerd, in het bijzonder in het
lage energie gebied (het zogenaamde infrarood regime). Meerbepaald claimen we dat er een effect
van de vectorbosonsector in de Natuur (cf. het Standaardmodel)  van de sterke/zwakke interactie
is op de materiesector, gekoppeld aan de sterke/zwakke interactie.

 

Deze thesis bevat 3 delen, volgend op 2 inleidende hoofdstukken. In een eerste deel, zijnde
Hoofdstuk 3, vindt de lezer een analytische studie van een Yang-Mills ijktheorie gekoppeld aan
een scalair Higgsveld, waarbij we rekening houden met het bestaan van Gribov ijkkopie\"en in de
kwantisatieprocedure. Dit laatste kwantisatiemechanisme, opgesteld door Gribov met als doel
ijkequivalente infinitesimale ijkkopie\"en te verwijderen, laat ons toe om een interessante
alternatieve visie op confinement te ontwikkelen, met name dat de deeltjespropagatoren complex
toegevoegde polen vertoont. Dit leidt tot wat men een schending van positiviteit noemt, hetgeen
betekent dat de deeltjes niet re\"eel waarneembaar kunnen zijn (de \"Osterwalder-Schrader
criteria zijn niet vervuld).
Rekening houden met de verschillen tussen en bijzonderheden van elk model, proberen we ook
telkens een parallel te trekken tussen onze analyse en de befaamde Fradkin-Shenker resultaten
over Yang-Mills-Higgs ijktheorie\"en.

 

In het tweede deel van de thesis stellen we een effectief model voor een zachte breking van de
BRST symmetrie voor, dit door een niet-lokale term in de materievelden aan de actie toe te
voegen. Deze speelt een rol gelijkaardig aan de Gribov-Zwanziger term die Gribov's
kwantisatieprocedure tot op alle ordes in de expansieparameter implementeert. We tonen o.a. aan
dat een consistente beschrijving van confined materie kan ge\"interpreteerd worden in termen van
een systematische zachte BRST breking. Dientengevolge is ons model in acceptabele overeenkomst
met recent rooster QCD data, met eveneens een materieveldpropagator die een schending van
positiviteit vertoont. Daarnaast is ons model ook veilig in het UV gebied, vermits we bewijzen
dat dergelijk confinement mechanisme ---analoog aan Gribov's originele analyse--- geen nieuwe
UV divergente termen kan genereren dan degene reeds aanwezig in de (niet-effectieve) theory.
Voor details verwijzen we hier naar de Hoofdstukken 4 en 5.

 

Het derde en tevens laatste deel van de thesis (Hoofdstuk 6) behandelt eindige
temperatuuraspecten van statische quarks, dit opnieuw gebruik maken van het Gribov-Zwanziger
kwantisatieformalisme. Met als bedoeling inzicht te verkrijgen in de confinement-deconfinement
fasetransitie, voeren we de Polyakovlus in via het achtergrondveldformalisme, uitgebreid met
Gribov-Zwanziger weliswaar. De statische quark fasetransitie kan zo onderzocht worden vermits
de Polyakovlus een ijkinvariante ordeparameter is. We bekijken de vacu\"umverwachtingswaarde in
verschillende omstandigheden, in het bijzonder om na te gaan hoe deze de dynamische Gribov
massaparameter be\"invloedt. We vinden het mooie resultaat dat deze massaschaal duidelijk
gevoelig is aan de fasetransitie, te zien aan het sterk veranderende gedrag rond de kritische
temperatuur. We vinden echter ook een gebied bij lage temperatuur dat thermodynamisch gezien
instabiel is. We eindigen met een korte discussie annex analyse wat een meer verfijnde
(Refined) Gribov-Zwanziger analyse hieraan zou kunnen verhelpen.
