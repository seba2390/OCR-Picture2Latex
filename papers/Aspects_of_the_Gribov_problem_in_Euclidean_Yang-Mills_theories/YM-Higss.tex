

\chapter{The Yang-Mills $+$ Higgs field theory}
\label{The Yang-Mills $+$ Higgs field theory}
%%%%%%%%%%%%%%%%%%%%%%%%%%%%%%%%%%%%%%%%%%%%%%%%%%%%%%%%%%%%%%%%%%%%





\section{Introduction}
\label{phasediag}


As mentioned in the Introduction, the confinement feature of QCD seems to be directly linked
to the existence of a (remnant) global symmetry: in the Linear Sigma model (LSM) that is the
$SO(N)$ symmetry, with $N$ standing for the number of flavors of the scalar field; in the
Yang-Mills theory coupled to a (static) matter field, such as the Higgs field, the center
symmetry $Z_{N}$ is the remaining global symmetry that has to be checked.

In this chapter we are going to discuss the specific model of Yang-Mills theory coupled to the
Higgs field. Specifically, the $SU(2)+$ Higgs and the Electroweak $SU(2)\times U(1)+$ Higgs
gauge theory will be analyzed. Here we adopt a perturbative analytical approach, by accounting
for non-perturbative effects through the quantization mechanism proposed by
Gribov\footnote{For details about the Gribov and Gribov-Zwanziger approaches the reader is
pointed to chapter \ref{usefulbkground} and advised to read references cited therein.}. Of
course, this is not the first time that such a model is studied. Instead of that, there
exist a vast bibliography concerning this topic, hanging from lattice works
\cite{Fradkin:1978dv,Lang:1981qg,Langguth:1985dr,Azcoiti:1987ua,Caudy:2007sf,Bonati:2009pf,Maas:2010nc,Maas:2012zf,Greensite:2011zz}
to the mean-field approach \cite{Horowitz:1983sr,Damgaard:1985nb,Baier:1986sa}.

Before effectively entering into details of our approach, we would like to briefly discuss the
work by Fradkin-Shenker on the lattice \cite{Fradkin:1978dv}, where the gauge field was
considered to be coupled to the Higgs field frozen in its state of vacuum configuration. For
such an end, in the next subsection we are going to present a summary of their work, with
details that may help us to understand differences between their discrete and our analytical
approach, leaving us in comfortable position to compare both results. Our results are shortly
exposed at the end of each section of this chapter. At the end of the chapter our conclusions
are exposed with a comparison between the referred lattice work of Fradkin-Shenker.







%-----------------------------------------
\subsection{Fradkin \& Shenker's results}
\label{FSresults}
%-----------------------------------------


By making use of a discrete space-time, called lattice, Fradkin \& Shenker reported a work on
the study of phase diagrams of gauge theories coupled to Higgs fields, \cite{Fradkin:1978dv}.
%
In order to properly address the feature of phase transition of gauge-Higgs theories, the
radial part of the scalar fields is considered to be frozen at its vacuum configuration state,
\begin{eqnarray}
{\phi}^{2} ~=~ \nu^2\,,
\label{untgauge}
\end{eqnarray}
by imposing the unitary gauge. The action describing this lattice Yang-Mills + Higgs fields
theory is given by
\begin{eqnarray}
S[\phi(\vec{r});U_{\mu}(\vec{r})] &=& \frac{K}{2} \sum_{(\vec{r},\,\mu\nu)} \Tr \bigg[
U_{\mu}(\vec{r}) U_{\nu}(\vec{r} + \hat{e}_{\mu}) U^{\dagger}_{\mu}(\vec{r} + \hat{e}_{\nu})
U^{\dagger}_{\nu}(\vec{r}) \bigg] + c.c. +
\nonumber \\
&&
\frac{\beta}{2} \sum_{\vec{r},\,\mu} \bigg[ \phi(\vec{r})\cdot D\left\{ U_{\mu}(\vec{r})
\right\} \cdot \phi^{\dagger}(\vec{r} + \hat{e}_{\mu}) + c.c. \bigg]\,,
\label{latticeaction}
\end{eqnarray}
making use of their notation, where $K$ stands for the inverse of the squared coupling
constant, $K = 1/g^{2}$, and $\beta$ stands for the squared fixed norm of the
scalar fields, $\beta = \nu^{2}$. The lattice is composed of sites, labeled by $\vec{r}$, and
of links, whose starting point is on the lattice site $\vec{r}$ with ending point on $\vec{r} +
\hat{e}_{\mu}$, with $\hat{e}_{\mu}$ denoting the fundamental direction vector. So a link is
labeled by $(\vec{r},\,\mu)$. 
In the action \eqref{latticeaction}, $U_{\mu}(\vec{r})$ denotes the gauge group element that
lives on the lattice link $(\vec{r},\,\mu)$, while $\phi(\vec{r})$ accounts for the scalar
fields living on the lattice site $\vec{r}$; $D\left\{U_{\mu} (\vec{r})\right\}$ accounts for
the group representation of the gauge link $U_{\mu}(\vec{r})$, and summation is taken over the
plaquette $(\vec{r},\,\mu\nu)$, which is defined as
\begin{eqnarray}
U_{\mu}(\vec{r}) U_{\nu}(\vec{r} + \hat{e}_{\mu}) U^{\dagger}_{\mu}(\vec{r} + \hat{e}_{\nu})
U^{\dagger}_{\nu}(\vec{r})\,.
\end{eqnarray}

In the example case of the (\emph{compact}) Electromagnetism quantum field theory, QED, the gauge
link is given by
\begin{eqnarray}
U_{\mu}(\vec{r}) ~=~ \exp\left[ i aeA_{\mu}(\vec{r})  \right]\,,
\end{eqnarray}
with $a$ denoting the lattice spacing --- the size of the link between two neighbor sites ---
and $e$ the electromagnetic coupling constant. The continuum limit is recovered for $a \to 0$
and the gauge sector of the action \eqref{latticeaction} goes as
\begin{eqnarray}
S[A] ~=~ \int d^{4}x \frac12 \Tr\bigg[ F_{\mu\nu}F_{\mu\nu} \bigg]\,,
\end{eqnarray}
with $F_{\mu\nu}$ standing for the electromagnetic field strength: $F_{\mu\nu} =
\p_{\mu}A_{\nu} - \p_{\nu}A_{\mu}$.

For the general case, the gauge group link $U_{\mu}(\vec{r})$ should transform under the $SU(N)$ gauge transformation as
\begin{eqnarray}
U_{\mu}(\vec{r}) ~\to~ G(\vec{r}) U_{\mu}(\vec{r}) G^{\dagger}(\vec{r})\,,
\label{gtlink}
\end{eqnarray}
while the scalar field transforms as 
\begin{eqnarray}
\phi(\vec{r}) ~\to~ D\left\{G(\vec{r})\right\}\phi(\vec{r})\,,
\label{gtscalar}
\end{eqnarray}
for $G(\vec{r}) \in SU(N)$. The lattice action \eqref{latticeaction} is left invariant under
such gauge transformation, \eqref{gtlink} and \eqref{gtscalar}. Particularly, the trace taken
over any gauge link $U_{\mu}(\vec{r})$ of the $SU(2)$ gauge group is real. In the general case
of $SU(N)$ gauge group the complex conjugate term (\emph{c.c.}) has to be added to the action,
so to end up with a real trace \cite{Fradkin:1978dv,Greensite:2011zz,Montvay:1994cy}.

When the unitary gauge is imposed, by choosing configurations of the scalar fields obeying
\footnote{The unitary gauge is not necessarily given by \eqref{untgauge}. Take a look at the
next subsection \ref{suitablegauge} for more details} \eqref{untgauge}, the gauge symmetry
is broken down to a \emph{local} group of symmetry, named the center symmetry $Z_{N}$. This is
a subgroup of the broken gauge group $SU(N)$ and whose elements commute with every element
of $SU(N)$. For instance, in the Georgi-Glashow model the
$SU(2)$ gauge group is spontaneously broken to the Abelian $U(1)$ group, after fixing the
unitary gauge, leaving the center symmetry $Z_{2}$ unbroken. As mentioned in the Introduction,
the confinement phase transition is to be understood as the ordered/disordered magnetic phase
transition related to the center symmetry $Z_{N}$.

Fradkin \& Shenker make use of the Wilson loop in order to probe for phase transition,
although such obsevable is not sensible to the center symmetry breaking at infinite volume. The
Wilson loop is, in fact, an order parameter (as discussed in the introduction) in the sense
that it is a gauge invariant quantity and is sensible to the existence of three different
phases, since it is a measure of the self-energy of static quarks. Namely, the Wilson loop is
defined on the lattice by
\begin{eqnarray}
W ~=~ \left\langle \Tr \left[ \prod_{(\vec{r},\,\mu)\in\Gamma} U_{\mu}(\vec{r}) \right]
\right\rangle\,,
\end{eqnarray}
where $\Gamma$ is the set of links forming closed loops.

Fradkin \& Shenker could find that for the Higgs fields in a trivial representation of the
$SU(N)$ gauge group, such as the adjoint representation, the gauge symmetry is broken
after fixing the unitary gauge, with the Higgs field frozen in its vacuum configuration.
However, the center symmetry $Z_{N}$ will always be left intact \cite{Fradkin:1978dv}. By
varying the parameters of the theory, $\beta=\nu^{2}$ and $K=1/g^{2}$, they could find three possible phases,
by probing the Wilson loop. Namely,
\begin{itemize}
\item[\emph{i}.] A Higgs-mechanism-type phase, with massive gauge bosons and a \emph{perimeter
law fall-off} for the Wilson loop. This region corresponds to large $\beta$ and $K$ values;

\item[\emph{ii}.] An intermediate phase, called \emph{free-charge} or \emph{Coulomb phase},
where the Wilson loop indicates a finite-energy between two static sources, and massless gauge
bosons;

\item[\emph{iii.}] A confined phase, where the Wilson loop develops an \emph{area law
fall-off}, the gauge bosons are massive with no free charges.
\end{itemize}
It should be emphasized that such spectrum concerns the Higgs fields in the adjoint
representation, see Figure \ref{fig1}.

\begin{figure}
\begin{center}
\includegraphics[width=5cm]{fradkin_shenker.pdf}
\caption{Taken from Fradkin \& Shenker's work \cite{Fradkin:1978dv}. This phase diagram
corresponds to the non-Abelian gauge theory coupled to Higgs fields in the adjoint
representation. It also corresponds to the spectrum of the compact Abelian scenario.}
\label{fig1}
\end{center}
\end{figure}

For the Higgs filds in the fundamental representation, which is a non-trivial representation of
the gauge group, the situation is completely different. As they say, the unitary gauge
completely breaks the gauge symmetry, so that the center symmetry does not survive
\cite{Fradkin:1978dv}. In this case they found that the confinement-like regime and the
Higgs-like regime \emph{belong to the same phase of the theory}, on the whole configuration
space of the parameters. That is, there is no phase transition between the confinement- and
Higgs-like regimes. Furthermore, the transition between any two points of the confined regime
and of the Higgs-like regime are smoothly connected, which means that the \emph{vev} of local
composite operators develops continuously throughout the entire configuration space (some care
should be taken to the vicinity of $\beta =K=\infty$). By means of the Osterwalder-Seiler's
proof to the special case of fixed length of the Higgs field, they could prove the analyticity
of the whole configuration space. Those results were obtained for non-Abelian gauge fields
coupled to Higgs fileds in the fundamental representation. Things are considerably different
for the Abelian + Higgs gauge theory (for details, see \cite{Fradkin:1978dv}).




















%-----------------------------------------------------------
\subsection{A suitable gauge choice, but not the unitary one}
\label{suitablegauge}
%-----------------------------------------------------------


Before properly starting to analyze the proposed model, let us state a few words on general features that are useful in this work. As mentioned before, the considered $SU(2)$ and $SU(2) \times U(1)$ Yang-Mills gauge theories are coupled to a scalar Higgs field. The Higgs field is considered in either the fundamental and the adjoint representation: in the $SU(2)$ case both the fundamental and the adjoint representation are analysed; while in the $SU(2)\times U(1)$ case only the the fundamental representation will be considered.

Usually the unitary gauge arises as a good choice when the Higgs mechanism is being treated, since in this gauge physical excitations are evident. However, instead of fixing the unitary gauge, we are going to choose the more general $\R_{\xi}$ gauge, whereby the unitary gauge is a special limit, $\xi \to \infty$, and the Landau gauge can be recovered when $\xi \to 0$. In the case of interest, the Landau gauge is imposed at the end of each computation. 
%The unitary gauge is defined by rotating the Higgs scalar field so that this ``rotated field'' becomes orthogonal to the Goldstone modes. It means,
%\begin{eqnarray}
%\tilde{\Phi}_{n} ~=~ \sum_{m}  U_{nm}(x) \Phi_{m}   \;,
%\label{rotatedHfield}
%\end{eqnarray}
%such that
%\begin{eqnarray}
%0   ~=~  \sum_{m,n}\tilde{\Phi}_{n}(x)(\tau^{\alpha})_{nm}\nu_{m}   \;,
%\label{orthogonHfield}
%\end{eqnarray}
%where $\sum_{m}(\tau^{\alpha})_{nm}\nu_{m} $ defines the Goldstone modes ($m,n ~=~ 1,2,...,N$) and $\tau^{a}$ accounts for the $SU(N)$ generators, although here, specifically, the index ``$a$'' runs over the broken symmetry directions. Evidently, the unitary gauge, defined by \eqref{rotatedHfield} and \eqref{orthogonHfield}, is a gauge fixing procedure established in the scalar field, instead of in the vector gauge field by itself.

%Alternatively, instead of impose conditions \eqref{rotatedHfield} and \eqref{orthogonHfield}, we are going to use the condition
The $\R_{\xi}$ gauge condition reads,
\begin{eqnarray}
f^{a} ~=~ \p_{\mu}A^{a}_{\mu} -i\xi  \sum_{m,n} \varphi_{m}(x)(\tau^{a})_{mn} \nu_{n}   \;.
\label{Rxiguage}
\end{eqnarray}
Note that in eq. \eqref{Rxiguage} $i\varphi_{m}(x)(\tau^{a})_{mn} \nu_{n}$ is a possible form for the $b^{a}$ field, defined in \eqref{notideal}. The $\varphi$ field is defined as a small fluctuation around the vacuum configuration of the Higgs field $\Phi$,
\begin{eqnarray}
\Phi(x) ~=~ \varphi(x) + \nu  \;,
\end{eqnarray}
with vacuum expectation value $\langle \varphi \rangle = 0$. In order to apply the gauge condition \eqref{Rxiguage} we should follow the standard process described in many text books, \cite{Weinberg:1996kr,Peskin:1995ev,Srednicki:2007qs}. Rather than a Dirac delta function, as in eq. \eqref{genfp}, one should put a Gaussian function,
\begin{eqnarray}
\delta(f^{a}(A)) \to \exp \left( -\frac{1}{2\xi}\int \d^{d}x\; f^{a}f^{a} \right)   \;.
\label{higgsgauge}
\end{eqnarray}
In the limiting case of $\xi \to 0$ the Gaussian term \eqref{higgsgauge} oscillates very fast
around $f^{a} =0$ so that the Gaussian term \eqref{higgsgauge} behaves like a delta function,
ensuring the desired Landau gauge. On the other side, if $\xi \to \infty$ then we have the
unitary gauge. Needless to say, the limit $\xi \to 0$, recovering the Landau gauge, should be
applied at the very end of each computation\footnote{It is perhaps worthwhile pointing out here
that the Landau gauge is also a special case of the 't Hooft $R_\xi$ gauges, which have proven
their usefulness as being renormalizable and offering a way to get rid of the unwanted
propagator mixing between (massive) gauge bosons and associated Goldstone modes, $\sim A_\mu
\p_\mu \phi$. The latter term indeed vanishes upon using the gauge field transversality. The
upshot of specifically using the Landau gauge is that it allows to take into account potential
non-perturbative effects related to the gauge copy ambiguity.}.

In the present work we should deal with the Higgs field frozen at its vacuum configuration --- $i.e.$, $\Phi = \nu$. It is equivalent to replacing every Higgs field $\Phi$ in the action with its vacuum expectation value $\nu$. Since the $\R_{\xi}$ gauge fixing condition \eqref{Rxiguage} only depends on the vector gauge field $A_{\mu}$ and on the fluctuation of the scalar field $\varphi$ multiplied by the gauge parameter $\xi$, the Gribov procedure remains valid for the limit case $\xi \to 0$.
















%%%%%%%%%%%%%%%%%%%%%%%%%%%%%%%%%%%%%%%%%%%%%%%%%%%%%%%%%%%%%%%%%%%%%
%\chapter{The $SU(2)+$Higgs field}
%\label{gfwithbroksym}
%%%%%%%%%%%%%%%%%%%%%%%%%%%%%%%%%%%%%%%%%%%%%%%%%%%%%%%%%%%%%%%%%%%%%


\section{$SU(2) +$Higgs field in the Fundamental representation}

In the present section nonperturbative effects of the $SU(2)+$Higgs model will be considered, by taking into account the existence of Gribov copies. The fundamental representation of the Higgs field, in $d=3$ and $d=4$, will be studied first. Subsequently, its adjoint representation, in $d=3$ and $d=4$, will be considered.
% Hereinafter, one should assume the Higgs field to be frozen at its vacuum configuration, as done in \cite{Fradkin:1978dv} --- in practice, the Higgs field self-coupling $\lambda$ will be made large enough in equation \eqref{SYMhiggs} --- and the Landau gauge limit, $\xi \to 0$, shall be taken at the very end of each computation.

Working in Euclidean spacetime, the starting action of the current model reads
\begin{equation}
S=\int \d^{d}x\left(\frac{1}{4}F_{\mu \nu }^{a}F_{\mu \nu }^{a}+
(D_{\mu }^{ij}\Phi ^{j})^{\dagger}( D_{\mu }^{ik}\Phi ^{k})+\frac{\lambda }{2}\left(
\Phi ^{\dagger}\Phi-\nu ^{2}\right) ^{2} - \frac{(\partial _{\mu }A_{\mu}^{a})^{2}}{2\xi} +\bar{c}^{a}\partial _{\mu }D_{\mu }^{ab}c^{b}\right)  \;, 
\label{SYMhiggs}
\end{equation}
where the covariant derivative is given by
\begin{equation}
D_{\mu }^{ij}\Phi^{j} =\partial _{\mu }\Phi^{i} -ig \frac{(\tau^a)^{ij}}{2}A_{\mu }^{a}\Phi^{j} \;,
\end{equation}
and the vacuum expectation value of the scalar field is $\left\langle \Phi^{i} \right\rangle ~=~ \nu\delta^{2i} $, with $i=1,2$,
%Introduced in the last section, $({\bar c}^a, c^a)$ are the anti-commuting Faddeev-Popov ghosts. The indices $i,j=1,2$ refer to the fundamental representation, and $\tau^a, a=1,2,3$, are the Pauli matrices.  The vacuum configuration, which minimizes the energy, is achieved by a constant scalar field parameterized as
%\begin{equation}
%\langle \Phi \rangle  = \left( \begin{array}{ccc}
%                                          0  \\
%                                          \nu
%                                          \end{array} \right)  \;,  \label{vevf}
%\end{equation}
so that all components of the gauge field acquire the same mass, $m^2= \frac{g^2\nu^2}{2}$.

By following the procedure described in the subsection \ref{Gribov's issue}, the restriction to
the first Gribov region $\Omega$ relies on the computation of the ghost form factor, given by
equation \eqref{nopole}, and on the enforcement of the no-pole condition,
\eqref{ghstprop00}--\eqref{nopolestepfunct}. Since the presence of the scalar Higgs field does
not influence the procedure of quantizing the gauge field, due to the Landau gauge chosen with
the Higgs field frozen at its vacuum configuration\footnote{Take a look at the subsection
\ref{suitablegauge}.}, the non-local Gribov term, which is proportional to $\beta$, is not affected and one ends up with the action
\begin{eqnarray}
S\;' = S + \beta^* \sigma(0,A) -\beta^*   \;,
\label{effctactsu2}
\end{eqnarray}
with $S$ given by eq.\eqref{SYMhiggs}, the ghost form factor given by \eqref{nopole} and
$\beta^{\ast}$ stands for the Gribov parameter that solves the gap equation \eqref{gapeq},
where the function $f(\beta)$ is given in the following subsection.
%The no-pole condition is translated to the action by the insertion of a new parameter, called Gribov parameter, which is dynamically fixed by the gap equation, equation \eqref{gapeq}. The ghost form factor for the $SU(2)$ case becomes,
%\begin{eqnarray}
%\sigma(0,A) &=& \frac{1}{V} \frac{1}{d} \frac{2g^2}{3} \int\frac{ \d^d q}{(2 \pi)^d} A_\alpha^\ell(-q) A_\alpha^{\ell} (q) \frac{ 1 }{q^2}    \;.
%\label{nopole}
%\end{eqnarray}








\subsection{The gluon propagator and the gap equation}

In order to compute the gluon propagator up to one-loop order in perturbation theory let us
follow the steps described in the subsection \ref{A sign of confinement from gluon
propagator}. The condition of freezing the scalar field to its vacuum configuration is
equivalent to considering $\lambda$ large enough, so that the potential term of the scalar
field becomes a delta function of $\left(\Phi ^{\dagger}\Phi-\nu ^{2}\right)$: the quadratic
terms of the action \eqref{SYMhiggs} reads,
\begin{eqnarray}
S\,_{quad} ~=~ \int \d^{d}x\left( \frac{1}{4} { \left(  \partial_\mu A^a_\nu -\partial_\nu
A^a_\mu  \right)} ^2 - \frac{(\partial _{\mu }A_{\mu}^{a})^{2}}{2\xi}  +
\frac{g^{2}\nu^{2}}{4}A_{\mu }^{a}A_{\mu }^{a}  \right)
%+  \frac{2g^2\beta}{3dV}  \frac{ A_{\mu}^{\ell} A_{\mu}^{\ell} }{(-\p^2)}  \right) - \beta
\;.
\label{quadf}
\end{eqnarray}
%\int\frac{ \d^d q}{(2 \pi)^4} 

After implementing the Gribov's restriction of the gauge field configuration space
to the first Gribov region $\Omega$, and changing to the Fourier momentum space, one gets the following partition function

\begin{equation}
Z_{quad} ~=~ \int \frac{\d\beta  e^{\beta  }}{2\pi i\beta } [\d A] \; \exp \left\{  -\frac{1}{2} \int \frac{\d^{d}q}{(2\pi )^{d}}A_{\mu }^{a }(q)\mathcal{P}_{\mu \nu }^{ab }A_{\nu }^{b }(-q)   \right\}\;,  
\label{Zq1f}
\end{equation}
with
\begin{equation}
\mathcal{P}_{\mu \nu }^{ab } =\delta ^{ab }\left[  \delta _{\mu \nu }\left( q^{2}+\frac{\nu^{2}g^{2}}{2}\right) +\left( \frac{1}{\xi } -1  \right) q_{\mu }q_{\nu } +  \frac{4g^2\beta}{3dV} \frac{1}{q^2} \delta _{\mu \nu }  \right]   \;.  
\label{Pf}
\end{equation}

Computing the inverse of \eqref{Pf} and taking $\xi \to 0$ at the very end, so recovering the
Landau gauge, only the transversal component of will survive and the gauge propagators may be
identified as

%Therefore, we have
%By changing to Fourier space and after a few algebraic manipulations, one gets for the partition function,
%The gluon propagator may be taken immediately from the partition function \eqref{Zq1f} by constructing the generating functional. After all, the whole process boils down to finding the inverse of the operator \eqref{Pf} and taking $\xi \to 0$. At the end we must have,
\begin{equation}
\left\langle A_{\mu}^{a}(q)A_{\nu }^{b}(-q)\right\rangle ~=~  \delta^{ab}\frac{ q^2}{q^{4} + \frac{g^{2}\nu^{2}}{2} q^2
+  \frac{ 4g^2\beta^* }{3dV} }\left( \delta _{\mu\nu }-\frac{q_{\mu }q_{\nu }}{q^{2}}\right)   \;,
\label{propf0}
\end{equation}
whereby $\beta^{\ast}$ solves the gap equation, which is obtained by the means of the subsection \ref{Gribovimplementation}. After computing the Gaussian integral of the partition function and taking the trace over all indices, one ends up with the following partition function,
%At first one should integreat out the partition function \eqref{Zq1f} and, after that, take the trace over each index's space of the operator $\mathcal{P}_{\mu \nu }^{ab }$. After taking the trace one gets,
%The gap equation, needed to attach a dynamical meaning to the Gribov parameter $\beta$, may be derived from \eqref{Zq1f} by integrating out the gauge field --- which is simple since it is a Gaussian integral and, additionally, this process is quite the similar to that one developed in section \ref{Gribovimplementation} --- leading to
%\begin{equation}
%Z_{quad}  ~=~ \int {\frac{\d \beta}{2\pi i} }
%e^{({\beta} -\ln\beta)} \left(\det\mathcal{P}^{ab}_{\mu\nu}\right)^{-\frac{1}{2}} \;.
%\label{Zq2f}
%\end{equation}
%Replicating the process detailed in \eqref{Zq2f00} we could find
%\begin{equation}
%\left( \det \mathcal{P}_{\mu \nu }^{ab }\right) ^{-\frac{1}{2}} ~=~ \exp \left[ -\frac{3(d-1)V}{2}\int {\frac{\d^{d}q}{(2\pi )^{d}}\ln \left(
%q^{2}  +   \frac{g^{2}\nu^{2}}{2}  +  \frac{4g^2\beta}{3dV}   \frac{1}{q^{2}}\right) }\right] \;,
%\label{traceoperator}
%\end{equation}
%so that the partition function now reads,
%by taking the trace over all indices of the operator $\mathcal{P}^{ab}_{\mu\nu}$. Particularly, the factor $d-1$ comes from the trace over the space-time indices after taking the the Landau gauge $\xi \to 0$. Therefore, the partition function reads
\begin{equation}  \label{Zf}
Z_{quad} ~=~ \int \frac{\d\beta}{2\pi i}\; e^{f(\beta)} ~=~ \e^{-V{\cal E}_{v}}   \;,
\end{equation}
whereby one reads the free-energy,
\begin{equation}
f(\beta)  ~=~  \beta - \ln \beta - \frac{3(d-1)V}{2} \int \frac{\d^d k}{(2\pi)^d} \; \ln\left( k^2 + \frac{g^2\nu^2}{2} +  \frac{4g^2\beta}{3dV} \frac{1}{k^2} \right)  \;, \label{ff}
\end{equation}
which is equivalent to \eqref{minusfreenergy}. In the thermodynamic limit the integral of equation \eqref{Zf} may be solved through the saddle-point approximation \eqref{gapeq} leading to the gap equation\footnote{We remind here that the derivative of the term ${\ln\beta}$  in expression \eqref{ff} will be neglected, for the derivation of the gap equation, eq.\eqref{gapf}, when taking the thermodynamic limit.}
\begin{equation}
\frac{2(d-1)}{d}g^2 \int \frac{\d^d q}{(2\pi)^d} \frac{1}{ q^{4} + \frac{g^{2}\nu^{2}}{2} q^2  +   \frac{2g^2\beta^{\ast}}{3dV} }  = 1  \;.
\label{gapf}
\end{equation}

In what follows the special case of $d=3$ and $d=4$ Euclidean space-time will be considered,
and the gap equation will be solved in both situations with a subsequent analysis of the gauge
field propagator, paying special attention to their pole: the applicability of Gribov's
confinement criterion (\emph{e.g.} the existence of complex conjugate poles) will be studied in
each space-time situation.







\subsection{The $d=3$ case}

Now, let us proceed with the solution  of the gap equation \eqref{gapf}. Since we are working in $d=3$, the gap equation contain a finite integral, easy to be computed, leading to
\begin{equation}
 \frac{4g^2}{3dV} \beta^{\ast} = \frac{1}{4} \left( \frac{g^2\nu^2}{2} -  \frac{g^4}{9\pi^2} \right)^2 \;. \label{solf}
\end{equation}
As done in the previous section, the analysis of the gluon propagator could be simplified by making explicit use of its poles. Namely,
\begin{equation}
\left\langle A_{\mu }^{a}(q)A_{\nu }^{b}(-q)\right\rangle
=  \frac{\delta^{ab}}{m^2_+-m^2_-}\left(  \frac{m^2_+}{q^2+m^2_+} -\frac{m^2_{-}}{q^2+m^2_-}  \right)
\left( \delta _{\mu\nu }-\frac{q_{\mu }q_{\nu }}{q^{2}}\right)  \;,  \label{decf}
\end{equation}
with
\begin{equation}
m^2_+ = \frac{1}{2} \left( \frac{g^2\nu^2}{2}  + \sqrt{\frac{g^6}{9\pi^2} \left(\nu^2-\frac{g^2}{9\pi^2}\right)}\; \right) \;,  \qquad m^2_- = \frac{1}{2} \left( \frac{g^2\nu^2}{2}  - \sqrt{\frac{g^6}{9\pi^2} \left(\nu^2-\frac{g^2}{9\pi^2}\right)} \;\right)
\label{massesf} \;.
\end{equation}
%and
%\begin{equation}
%{\cal F}_{+} = \frac{m^2_+}{m^2_+-m^2_-}  \;, \qquad {\cal F}_{-} = \frac{m^2_-}{m^2_+-m^2_-}  \label{residf} \;.
%\end{equation}
In this way, we may distinguish two regions in the $(\nu^2,g^2)$ plane:
\begin{itemize}
\item[\it i)] when $g^2 < 9\pi^2 \nu^2$ both masses $(m^2_+,m^2_-)$ are positive, as well as the residues. The gluon propagator, eq.\eqref{decf}, decomposes into two Yukawa modes. However, due to the relative minus sign in expression \eqref{decf} only the heaviest mode with mass $m^2_+$ represents a physical mode. We see thus that, for $g^2 < 9\pi^2 \nu^2$, all components of the gauge field exhibit a physical massive mode with mass $m^2_+$. This region is what can be called a Higgs phase. 

Let us also notice that, for the particular value $g^2=\frac{9\pi^2}{2}\nu^2$, corresponding to a vanishing Gribov parameter $\beta=0$, the unphysical Yukawa mode in expression  \eqref{decf} disappears, as $m^2_-~=~0$. As a consequence, the gluon propagator reduces to that of a single physical mode with mass $\frac{9\pi^2}{4}\nu^4$.
%, namely
%\begin{equation}
%\left\langle A_{\mu }^{a}(q)A_{\nu }^{b}(-q)\right\rangle
%=\delta^{ab}   \left( \delta _{\mu\nu }-\frac{q_{\mu }q_{\nu }}{q^{2}}\right) \frac{1}{q^2 +\frac{9\pi^2}{4}\nu^4 } \;.  \label{decff}
%\end{equation}
\item[\it ii)] when $g^2> 9 \pi^2 \nu^2$, the masses $(m^2_+,m^2_-)$ become complex. In this region, the gluon propagator, eq.\eqref{decf}, becomes of the Gribov type, displaying complex conjugate poles. All components of the gauge field become thus unphysical. This region corresponds to the confining phase.
\end{itemize}
%Summarizing, when the Higgs field is in the fundamental representation, a Higgs phase is detected for $g^2 < 9\pi^2 \nu^2$. When $g^2> 9 \pi^2 \nu^2$, the confining phase emerges. Concerning the trustworthiness of the results, completely analogous comments as in the adjoint case apply here as well.














\subsection{The $d=4$ case}
With quite the same process as for the $d=3$ case, let us analyse the poles of the gauge field propagator by solving the gap equation for $d=4$. To that end the following decomposition becomes useful
\begin{equation}
q^4 +  \frac{g^{2}\nu^{2}}{2} q^2
+ \frac{g^2}{3} \beta   ~=~ (q^2+m^2_+) (q^2+m^2_-) \;,  
\label{dec1}
\end{equation}
with
\begin{equation}
m^2_+ ~=~ \frac{1}{2} \left(\frac{g^2 \nu^2}{2} + \sqrt{\frac{g^4\nu^4}{4}  -\frac{4g^2}{3} \beta^*} \;  \right) \;,  \qquad    m^2_- ~=~ \frac{1}{2} \left(\frac{g^2 \nu^2}{2} -\sqrt{\frac{g^4\nu^4}{4}  -\frac{4g^2}{3} \beta^*} \;  \right)  \;.
\label{roots1}
\end{equation}
Making use of the $\MSbar$ renormalization scheme in $d=4-\varepsilon$ 
%and of the standard integral
%\begin{equation}
%\int \frac{\d^d p}{(2\pi)^d} \frac{1}{p^2+\rho^2} ~=~ - \frac{\rho^2}{16\pi^2} \frac{2}{\bar \beta} + \frac{\rho^2}{16\pi^2} \left( \ln\frac{\rho^2}{{\omu}^2} - 1  \right) \;,
%\label{intd2f}
%\end{equation}
the gap equation \eqref{gapf} becomes
\begin{equation}
\left[1 + \frac{m^2_{-}}{m^2_+ -  m^2_-}\; \ln\left( \frac{m^2_-}{{\omu}^2} \right)  - \frac{m^2_+}{m^2_+ -  m^2_-}\; \ln\left( \frac{m^2_+}{{\omu}^2} \right)  \right] ~=~ \frac{32\pi^2}{3g^2}  \;. 
\label{gapf1}
\end{equation}
After a suitable manipulation we get a more concise expression for the gap equation
%In order to analyze this equation we rewrite it in a more suitable way, {\it i.e.}
%\begin{eqnarray}
% \frac{m^2_-}{m^2_+ -  m^2_-}\; \ln\left( \frac{m^2_-}{\omu^2} \right)  - \frac{m^2_+}{m^2_+ -  m^2_-}\; \ln\left( \frac{m^2_+}{\omu^2} \right)   
%&=& - \frac{m^2_+ - m^2_-}{m^2_+ -  m^2_-} \left( 1- \frac{32\pi^2}{3g^2} \right)  
%\nonumber \\
%&=&  \frac{m^2_+ - m^2_-}{m^2_+ -  m^2_-}\; \ln \left(  e^{- \left( 1- \frac{32\pi^2}{3g^2} \right)}  \right)   \;,  
%\label{gapf2}
%\end{eqnarray}
%so that
%\begin{equation}
% m^2_-\; \ln\left( \frac{m^2_-}{\omu^2  e^{\left( 1- \frac{32\pi^2}{3g^2} \right)} } \right) ~=~  m^2_+\; \ln\left( \frac{m^2_+}{\omu^2  e^{\left( 1- \frac{32\pi^2}{3g^2} \right)} } \right) \;,
%\label{gapf3}
%\end{equation}
%whose final form can be written as
\begin{equation}
2 \sqrt{1-\zeta}\; \ln(a) = -  \left( 1 +  \sqrt{1-\zeta} \right) \; \ln\left( 1 +  \sqrt{1-\zeta} \right) +  \left( 1 -  \sqrt{1-\zeta} \right) \; \ln\left( 1 - \sqrt{1-\zeta} \right)  \;, 
\label{gapd1}
\end{equation}
where we have introduced the dimensionless variables
\begin{equation}
a ~=~ \frac{g^2 \nu^2}{4 \omu^2 e^{\left( 1 -\frac{32\pi^2}{3g^2} \right)}} \;, \qquad  \qquad \zeta ~=~ \frac{16}{3} \frac{\beta^*}{g^2\nu^4}  \geq0 \;, \label{vb1}
\end{equation}
with $0 \le \zeta < 1$ in order to have two real, positive, distinct roots $(m^2_+, m^2_-)$.
% It is worth underlining that the renormalization scale $\omu$ could be exchanged in favour of the invariant scale  $\lms$, defined at one-loop as
%\begin{equation}
%\lms^2  =  \omu^2 \;e^{\frac{1}{\beta_0} \frac{1}{ g^2(\omu) } }  \label{Lambda} \;,
%\end{equation}
%with $\beta_0$ given by \cite{Gross:1973ju,Pickering:2001aq}
%\begin{equation}
%\beta  = - g^3 \beta_0 + O(g^5) \;, \qquad  \beta_0 = \frac{1}{16\pi^2} \left( \frac{11}{3} N - \frac{1}{6} T  \right)  \;,
%\end{equation}
%where $T$ is the Casimir of the representation of the Higgs field equaling $T=\frac{1}{2}$, resp.~$T=2$, for the fundamental, resp.~adjoint, representation of $SU(2)$.
For $\zeta >1$, the roots $(m^2_+, m^2_-)$ become complex conjugate, and the gap equation takes the form
\begin{equation}
2 \sqrt{\zeta -1} \; \ln(a) =   -2 \; \arctan\left({\sqrt{\zeta-1}}\; \right)   - \sqrt{\zeta-1} \; \ln\;\zeta  \;. \label{d1}
\end{equation}
Moreover, it is worth noticing that both expressions \eqref{gapd1},\eqref{d1} involve only one function, {\it i.e.} they can be written as
\begin{equation}
2 \; \ln(a) = g(\zeta)  \;, \label{g}
\end{equation}
where for $g(\zeta)$ we might take
\begin{equation}
g(\zeta) =    \frac{1}{ \sqrt{1-\zeta}} \left(
- \left( 1 +  \sqrt{1-\zeta} \right) \; \ln\left( 1 +  \sqrt{1-\zeta} \right) +  \left( 1 -  \sqrt{1-\zeta} \right) \; \ln\left( 1 - \sqrt{1-\zeta} \right)  \right) \;,
\label{gex}
\end{equation}
which is a real function of the variable $\zeta \ge 0$. Expression \eqref{d1} is easily obtained from \eqref{gapd1} by rewriting it in the region $\zeta>1$.  In particular, it turns out that the function $g(\zeta)\leq -2\ln 2$ for all $\zeta \ge 0$, and strictly decreasing. As consequence, for each value of $a<\frac{1}{2}$, equation \eqref{g} has always a unique solution with $\zeta>0$.  Moreover, it is easy to check that $g(1)=-2$. Therefore,  we can distinguish ultimately three regions, namely
\begin{itemize}
\item[(a)]  when $a>\frac{1}{2}$,  eq.\eqref{g}  has no solution for $\zeta$. Since the gap equation \eqref{gapf} has been  obtained by applying the saddle point approximation in the thermodynamic limit, we are forced to set $\beta^{\ast}=0$.  This means that, when $a>\frac{1}{2}$, the dynamics of the system is such that the restriction to the Gribov region cannot be consistently implemented.  As a consequence, the standard Higgs mechanism takes place, yielding three components of the gauge field with mass $m^{2} ~=~ \frac{g^{2}\nu^{2}}{2}$. Note that, for sufficiently weak coupling $g^2$, $a$ will unavoidably be larger than $\frac{1}{2}$.
%, according to
%\begin{equation}
%\left\langle A_{\mu }^{a}(q)A_{\nu }^{b}(-q)\right\rangle
%=\delta^{ab}\frac{1}{q^{2} + \frac{g^{2}\nu^{2}}{2}  }\left( \delta _{\mu
%\nu }-\frac{q_{\mu }q_{\nu }}{q^{2}}\right)  \label{propfY} \;.
%\end{equation}


\item[(b)] when $\frac{1}{e}<a<\frac{1}{2}$, equation  \eqref{g} has a solution for  $0 \le \zeta <1$. In this region, the roots $(m^2_+, m^2_-)$  are real and the gluon propagator decomposes into the sum of two terms of the Yukawa type:
\begin{equation}
\left\langle A_{\mu }^{a }(q)A_{\nu }^{b }(-q)\right\rangle
=\frac{\delta^{ab}}{m^2_+-m^2_-}  \left(   \frac{m^2_+}{q^2+m^2_+} -   \frac{m^2_-}{q^2+m^2_-}   \right) 
 \left( \delta _{\mu
\nu }-\frac{q_{\mu }q_{\nu }}{q^{2}}\right)  \label{ffin} \;.
\end{equation}
%where
%\begin{equation}
%{\cal F}_+ = \frac{m^2_+}{m^2_+-m^2_-}  \;, \qquad {\cal F}_- = \frac{m^2_-}{m^2_+-m^2_-} \;. \label{rf}
%\end{equation}
Moreover, due to the relative minus sign in eq.\eqref{ffin} only the component proportional to
$m^{2}_{+}$ represents a physical mode.
\item[(c)]  for $a<\frac{1}{e}$, equation \eqref{g} has a solution for  $\zeta>1$. This
scenario will always be realized if $g^2$ gets sufficiently large, i.e.~at strong coupling. In
this region the roots  $(m^2_+, m^2_-)$  become complex conjugate and the gauge boson
propagator is of the Gribov type, displaying complex poles.  As usual, this can be interpreted
as the confining region.
\end{itemize}
In summary, we clearly notice that at sufficiently weak coupling, the standard Higgs mechanism will definitely take place, as $a>\frac{1}{2}$, whereas for sufficiently strong coupling, we always end up in a confining phase because then $a<\frac{1}{2}$.

Having obtained these results, it is instructive to go back where we originally started. For a fundamental Higgs, all gauge bosons acquire a mass that screens the propagator in the infrared. This effect, combined with a sufficiently small coupling constant, will lead to a severely suppressed ghost self energy, i.e.~the average of \eqref{nopole} (to be understood after renormalization, of course). If the latter quantity will a priori not exceed the value of 1 under certain conditions --- {\it i.e.}, satisfying the no-pole condition --- the theory is already well inside the Gribov region and there is no need to implement the restriction. Actually, the failure of the Gribov restriction for $a>\frac{1}{2}$ is exactly because it is simply not possible to enforce that $\sigma(0)=1$. Perturbation theory in the Higgs sector is \emph{in se} already consistent with the restriction within the 1st Gribov horizon. Let us verify this explicitly by taking the average of \eqref{nopole} with, as tree level input propagator, a transverse Yukawa gauge field with mass $m^2=\frac{g^2\nu^2}{2}$. Using that there are 3 transverse directions\footnote{We have been a bit sloppy in this paper with the use of dimensional regularization. In principle, there are $3-\epsilon$ transverse polarizations in $d=4-\epsilon$ dimensions. Positive powers in $\epsilon$ can (and will) combine with the divergences in $\epsilon^{-1}$ to change the finite terms. However, as already pointed out before, a careful renormalization analysis of the Gribov restriction is possible, see e.g.~\cite{Dudal:2010fq,Vandersickel:2012tz} and this will also reveal that the ``1'' in the Gribov gap equation will receive finite renormalizations, compatible with the finite renormalization in e.g.~$\sigma(0)$, basically absorbable  in the definition of $a$. } in $4d$, we have
%\begin{eqnarray}
%\sigma(0) =\frac{3g^{2}}{2}\int \frac{d^{4}q}{(2\pi )^{4}}\frac{1}{q^2(q^2+\frac{g^2\nu^2}{2})}=-\frac{3}{\nu^2}\int\frac{d^4q}{(2\pi^4)}\frac{1}{q^2+\frac{g^2\nu^2}{2}}=-\frac{3g^2}{32\pi^2}\left(\ln\frac{g^2\nu^2}{2\omu^2}-1\right)\;.
%\end{eqnarray}
%Introducing $a$ as in \eqref{vb1}, we may reexpress the latter result as
\begin{eqnarray}
\sigma(0) =1-\frac{3g^2}{32\pi^2}\ln(2a)\;.
\end{eqnarray}
For $a>\frac{1}{2}$, the logarithm is positive and it is then evident that $\sigma(0)$ will not cross $1$, indicating that the theory already is well within the first Gribov horizon.

Another interesting remark is at place concerning the transition in terms of a varying value of $a$. If $a$ crosses $\frac{1}{e}$, the imaginary part of the complex conjugate roots becomes smoothly zero, leaving us with 2 coinciding real roots, which then split when $a$ grows. At $a=\frac{1}{2}$, one of the roots and its accompanying residue vanishes, to leave us with a single massive gauge boson. We thus observe that all these transitions are continuous, something which is in qualitative correspondence with the theoretical lattice predictions of the classic work \cite{Fradkin:1978dv} for a fundamental Higgs field that is ``frozen'' ($\lambda\to\infty$). Concerning the somewhat strange intermediate phase, {\it i.e.}  the one with a Yukawa propagator with a negative residue, eq.\eqref{ffin}, we can investigate in future work in more detail the asymptotic spectrum based on the BRST tools developed in \cite{Dudal:2012sb} when the local action formulation of the Gribov restriction is implemented. Recent works on the lattice confirm the existence of a cross-over region, where there is no line separating the ``phases'', as e.g. \cite{Maas:2013aia,Maas:2014pba} where the authors work in the non-aligned minimal Landau gauge and observe the transition between a QCD-like phase and a Higgs-like phase, in a region away from the cross-over region.













\subsection{The vacuum energy in the fundamental representation}


Let us look at the vacuum energy ${\cal E}_v$ of the system, which  can easily be read off from expression \eqref{Zf}, namely
\begin{equation}
{\cal E}_v = - \beta^* + \frac{9}{2} \int \frac{d^4k}{(2\pi)^4} \; \ln\left( k^2 + \frac{g^2\nu^2}{2} +\frac{\beta^*}{3} \frac{g^2}{k^2} \right)  \;, \label{ev}
\end{equation}
where $\beta^*$ is given by the gap equation \eqref{gapf}. Making use of the $\MSbar$ renormalization scheme, the vacuum energy may be written as:
%Making use of
%\begin{equation}
%\int \frac{d^dp}{(2\pi)^d} \; \ln(p^2 + m^2)  = - \frac{m^4}{32\pi^2} \left( \frac{2}{\bar \varepsilon}  -  \ln{\frac{m^2}{{\bar \mu}^2}} + \frac{3}{2} \right) \;,  \label{intdl}
%\end{equation}
%it is very easy to write down the vacuum energy:
\begin{itemize}
\item  for $a<\frac{1}{2}$, we have
\begin{eqnarray}
\frac{8}{9 g^4\nu^4}\; {\cal E}_v & = &  \frac{1}{32\pi^2} \left( 1 - \frac{32\pi^2}{3g^2} \right) - \frac{1}{2} \frac{\zeta}{32\pi^2} + \frac{1}{4}\frac{1}{32\pi^2} \left(  (4-2\zeta)\left( \ln( a) -\frac{3}{2} \right) \right)  \\ \nonumber & + &  \frac{1}{4}\frac{1}{32\pi^2} \left(   \left( 1+ \sqrt{1-\zeta} \right)^2 \ln  \left( 1+ \sqrt{1-\zeta} \right)
+ \left( 1- \sqrt{1-\zeta} \right)^2 \ln  \left( 1- \sqrt{1-\zeta} \right)
 \right)     \;, \label{v1}
\end{eqnarray}
where $\zeta$ is obtained through eqs.\eqref{g},\eqref{gex}. 
%Let us also give the expressions of the first two derivatives of ${\cal E}_v(a)$  with respect to $a$:
% From the gap equation \eqref{g},  we easily get
%\begin{equation}
%\frac{\partial \zeta}{\partial a} = \frac{2}{a} \frac{1}{g'(\zeta)}  \;. \label{saa}
%\end{equation}
%Therefore
%\begin{eqnarray}
%\frac{ \partial }{\partial a} \left[\frac{8}{9 g^4\nu^4}\; {\cal E}_v \right] & = & \frac{1}{64\pi^2} \frac{1}{a} (2-\zeta) \;, \nonumber \\
%\frac{ \partial^2 }{\partial a^2} \left[\frac{8}{9 g^4\nu^4}\; {\cal E}_v \right] & =& \frac{1}{64\pi^2} \frac{1}{a^2} \left(\zeta - 2 - \frac{2}{g'({\zeta})} \right)\;. \label{d12}
%\end{eqnarray}




\item for $a>\frac{1}{2}$,
\begin{equation}
\frac{8}{9 g^4\nu^4}\; {\cal E}_v  =   \frac{1}{32\pi^2} \left( 1 - \frac{32\pi^2}{3g^2} \right)  + \frac{1}{32\pi^2} \left(  \left( \ln( a) -\frac{3}{2} \right) \right)   +   \frac{1}{32\pi^2} \ln2    \;. \label{v2}
\end{equation}
\end{itemize}
%Owing to the fact that $g'(0) =- \infty$, it turns out that
From these expressions we could check that the vacuum energy ${\cal E}_v(a)$ is a continuous  function of the variable $a$, as well as its first and second derivative, and that the third derivative develops a jump at $a=\frac{1}{2}$. We might be tempted to interpret this is indicating a third order phase transition at $a=\frac{1}{2}$. The latter value actually corresponds to a line in the $(g^2,\nu)$ plane according to the functional relation \eqref{vb1}. However, we should be cautious to blindly interpret this value. It is important to take a closer look at the validity of our results in the light of the made assumptions. More precisely, we implemented the restriction to the horizon in a first order approximation, which can only be meaningful if the effective coupling constant is sufficiently small, while simultaneously emerging logarithms should be controlled as well. In the absence of propagating matter, the expansion parameter is provided by $y\equiv\frac{g^2N}{16\pi^2}$ as in pure gauge theory. The size of the logarithmic terms in the vacuum energy (that ultimately defines the gap equations) are set by $m_+^2\ln\frac{m_+^2}{\omu^2}$ and $m_-^2\ln\frac{m_-^2}{\omu^2}$. A good choice for the renormalization scale would thus be $\omu^2\sim |m_+^2|$: for (positive) real masses, a fortiori we have $m_-^2<m_+^2$ and the second log will not get excessively large either because $m_-^2$ gets small and the pre-factor is thus small, or $m_-^2$ is of the order of $m_+^2$ and the log itself small. For complex conjugate masses, the size of the log is set by the (equal) modulus of $m_\pm^2$ and thus both small by our choice of scale.

Let us now consider the trustworthiness, if any, of the $a=\frac{1}{2}$ phase transition point.  For $a\sim\frac{1}{2}$, we already know that $\zeta\sim 0$, so a perfect choice is $\omu^2\sim m_+^2\sim\frac{g^2\nu^2}{2}$. Doing so, the $a$-equation corresponds to
\begin{equation}\label{aeq}
    \frac{1}{2}\sim e^{-1+\frac{4}{3y}}
\end{equation}
so that $y\sim 4$. Evidently, this number is thus far too big to associate any meaning to the ``phase transition'' at $a=\frac{1}{2}$. Notice that there is no problem for the $a$ small and $a$ large region. If $\nu^2$ is sufficiently large and we set $\omu^2\sim \frac{g^2\nu^2}{2}$ we have a small $y$, leading to a large $a$, i.e.~the weak coupling limit without Gribov parameter and normal Higgs-like physics. The logs are also well-tempered.    For a small $\nu^2$, the choice $\omu^2\sim\sqrt{g^2\theta^*}$ will lead to
\begin{equation}\label{aeq2}
a\sim (\textrm{small number})e^{-1+\frac{4}{3y}}
\end{equation}
so that a small $a$  can now be compatible with a small $y$, leading to a Gribov parameter dominating the Higgs induced mass, the ``small number'' corresponds to $\frac{g^2\nu^2}{\sqrt{g^2\theta^*}}$. Due to the choice of $\omu^2$, the logs are again under control in this case.

Within the current approximation, we are thus forced to conclude that only for sufficiently small or large values of the parameter $a$ we can probe the theory in a controllable fashion. Nevertheless, this is sufficient to ensure the existence of a Higgs-like phase at large Higgs condensate, and a confinement-like region for small Higgs condensate. The intermediate $a$-region is more difficult to interpret due to the occurrence  of large logs and/or effective coupling. Notice that this also might make the emergence of this double Yukawa phase at $a=\frac{1}{e}\approx 0.37$ not well established at this point.























\section{The $SU(2) +$Higgs field in the ajoint representation}
\label{Adjrep}

The Yang-Mills $+$ Higgs action with the scalar field in its adjoint representation may be
written as
\begin{eqnarray}
S ~=~ \int \,\d^{d}x \left[ \frac14 F^{a}_{\mu\nu}F^{a}_{\mu\nu} +
D^{ab}_{\mu}\Phi^{b}D^{ac}_{\mu}\Phi^{c} + \frac{\lambda}{2}\left( \Phi^{\dagger}\Phi - \nu^{2}
\right)^{2} - \frac{(\p A)^{2}}{\xi} + \bar{c}^{a}\p D^{ab}c^{b}
 \right]\;.
\end{eqnarray}
In the adjoint case the vacuum configuration that minimizes the energy is achieved by a
constant scalar field satisfying
\begin{equation}
\left\langle \Phi ^{a}\right\rangle ~=~ \nu \delta ^{a3}   \;,
\label{higgsv}
\end{equation}
leading to the standard Higgs mechanism. One should pay attention that the condition of 
degenerated vacuum, $\left\langle \Phi ^{a}\right\rangle \neq 0$, \eqref{higgsv} does not
automatically means that the \emph{unitary gauge} is being adopted. As has been emphasized
through out this chapter, the Higgs field is being considered to be frozen in its vacuum
configuration, which allows us to choose, under such hypothesis, the Landau gauge. Details
concerning this statement can be found in standard textbooks
\cite{Weinberg:1996kr,Peskin:1995ev,Ryder:1985wq} as well as in the section
\ref{suitablegauge}.

Just as in the fundamental case, fixing the scalar field in its vacuum configuration is
equivalent to consider a large enough value for the self-coupling $\lambda$, so that the
potential energy amounts to a delta function of $\left(\Phi ^{\dagger}\Phi-\nu ^{2}\right)$.
Thus, for the quadratic terms of the action, we have
\begin{equation}
S_{quad}=\int \d^{d}x\left( \frac{1}{4} { \left(  \partial_\mu A^a_\nu -\partial_\nu A^a_\mu  \right)} ^2 + b^a \partial_\mu A^a_\mu
+ \frac{g^{2}\nu ^{2}}{2}\left( A_{\mu }^{1}A_{\mu }^{1}+A_{\mu }^{2}A_{\mu
}^{2}\right)  \right)  \;. 
\label{quad}
\end{equation}


Following the standard procedure, before implementing the Gribov framework, one should notice
that the action \eqref{quad} has two independent sector, the \emph{diagonal} and the
\emph{off-diagonal} ones, corresponding respectively to the quadratic terms of $A_{\mu}^{3}$
and $A_{\mu}^{\alpha}$, with $\alpha=1,2$ (Greek letters should account for $1$ and $2$ in the
colour space). The existence of such split in the gauge sector reflects the breaking of the
gauge field, due to the gauge fixing after freezing the scalar field as 
\[
\left\langle \Phi^{a}\right\rangle ~=~ \nu \delta ^{a3}\,,
\]
leading to the existence of two massive vector modes, and a massless one. These massive vector
bosons and massless one may inferred from the following propagators,
\begin{equation}
\label{gluonoff}
\left\langle A_{\mu }^{\alpha }(p)A_{\nu }^{\beta }(-p)\right\rangle =\frac{\delta ^{\alpha \beta }}{p^{2}+m_{H}^{2}}\left( \delta _{\mu \nu }-%
\frac{p_{\mu }p_{\nu }}{p^{2}}\right) \;,
\end{equation}
from what, $m_{H}^{2} ~=~ g^{2}\nu ^{2}$ is the acquired mass after the symmetry breaking. The
massless mode amounts to the third component $A_\mu^3$, namely,
\begin{equation}
\left\langle A_{\mu }^{3}(p)A_{\nu }^{3}(-p)\right\rangle =\frac{1}{p^{2}}%
\left( \delta _{\mu \nu }-\frac{p_{\mu }p_{\nu }}{p^{2}}\right) \;. 
\label{zm}
\end{equation}

However, as was pointed out by Polyakov  \cite{Polyakov:1976fu}, the theory exhibits a different behaviour. The action \eqref{SYMhiggs} admits classical solitonic solutions, known as the 't Hooft-Polyakov monopoles\footnote{ These configurations are instantons in Euclidean space-time.} which play a relevant role in the dynamics of the model. In fact, it turns out that these configurations give rise to a monopole condensation at weak coupling, leading to a confinement of the third component $A^3_\mu$, rather than to a Higgs type behaviour, eq.\eqref{zm}, a feature also confirmed by lattice numerical simulations  \cite{Nadkarni:1989na,Hart:1996ac}.

Since our aim is that of analysing the nonperturbative dynamics of the Georgi-Glashow model by taking into account the Gribov copies, let's follow the procedure described in the subsection \ref{Gribovimplementation}. Due to the presence of the Higgs field in the adjoint representation, causing a breaking of the global gauge symmetry, the ghost two-point function has to be decomposed into two sectors, diagonal and off-diagonal:
% Then, for the connected two-point ghost function $\mathcal{G}^{ab}(k;A)$ at first order in the gauge fields,  one finds  
%\begin{equation}
%\mathcal{G}^{ab}(k;A)  ~=~ \frac{1}{k^{2}}  \left( \delta ^{ab}-g^{2}\frac{k_{\mu}k_{\nu }}{k^{2}}  
%\int \frac{\d^{d}q}{(2\pi )^{d}} \;  \varepsilon^{amc}\varepsilon^{cnb}  \frac{1}{(k-q)^{2}}  \left( A_{\mu }^{m}(q)A_{\nu}^{n}(-q)\right) \right)  \label{Ghost} \;,
%\end{equation}
%where use has been made of the transversality condition $q_{\mu}A_{\mu}(q)=0$. 

%In order to correctly take into account the presence of the Higgs vacuum, eq.\eqref{higgsv}, we decompose  $\mathcal{G}^{ab}(k;A)$ into
\begin{equation}
\mathcal{G}^{ab}(k,A)=\left(
\begin{array}{cc}
\delta^{\alpha \beta}\mathcal{G}_{off}(k;A) & 0 \\
0 & \mathcal{G}_{diag}(k;A)
\end{array}
\right)
\end{equation}
where
\begin{eqnarray}
\mathcal{G}_{off}(k;A) 
%&=&  \frac{1}{k^{2}}   \left(   1+ \frac{g^{2}}{V}  \frac{k_{\mu }k_{\nu }}{2k^{2}}  
%\int \frac{\d^{d}q}{(2\pi )^{d}}  \;  \frac{1}{(q-k)^{2}}  \left( A_{\mu}^{\alpha }(q)A_{\nu }^{\alpha }(-q)+2A_{\mu }^{3}(q)A_{\nu }^{3}(-q)\right) \right)  
%\nonumber \\
&=&  \frac{1}{k^{2}}   \left( 1+\sigma _{off}(k;A)\right)       \approx \frac{1}{k^{2}}  \left( \frac{1}{1-\sigma
_{off}(k;A)}\right) 
\label{Goff} \;,  \\[5mm]
\mathcal{G}_{diag}(k;A) 
%&=&  \frac{1}{k^{2}}  \left( 1 +  \frac{g^{2}}{V} \frac{k_{\mu }k_{\nu}}{k^{2}}
%\int \frac{\d^{d}q}{(2\pi )^{d}} \;  \frac{1}{(q-k)^{2}}  \left( A_{\mu}^{\alpha }(q)A_{\nu }^{\alpha }(-q)\right) \right)  
%\nonumber \\
&=&  \frac{1}{k^{2}}  \left( 1+\sigma _{diag}(k;A)\right)  \approx \frac{1}{k^{2}}\left( \frac{1}{1-\sigma
_{diag}(k;A)}\right) \label{Gdiag} \;.
\end{eqnarray}

As we know, the quantities $\sigma_{off}(k;A), \; \sigma_{diag}(k;A)$ turn out to be  decreasing functions of the momentum $k$ and making use of the gauge field transversality, we have
\begin{eqnarray}
\sigma _{off}(0;A) &=&  \frac{g^{2}}{Vd}  \int \frac{\d^{d}q}{(2\pi )^{d}} \; \frac{\left( A_{\mu }^{3}(q)A_{\mu }^{3}(-q)+\frac{1}{2}A_{\mu }^{\alpha}(q)A_{\mu }^{\alpha }(-q)\right) }{q^{2}}  \;, 
\nonumber  \\
\sigma _{diag}(0;A) &=& \frac{g^{2}}{Vd}  \int {\frac{\d^{d}q}{(2\pi )^{d}} \; \frac{ \left( A_{\mu }^{\alpha }(q) A_{\mu }^{\alpha }(-q)\right) }{q^{2}}}    \;.
\label{sigma}
\end{eqnarray}
Once again, these expressions were obtained by taking the limit $k \rightarrow 0$ of eqs.\eqref{Goff},\eqref{Gdiag}, and by making use of the property
\begin{eqnarray}
A_{\mu }^{a}(q)A_{\nu }^{a}(-q) &=&\left( \delta _{\mu \nu }-\frac{q_{\mu }q_{\nu }%
}{q^{2}}\right) \omega (A)(q)   \nonumber \\
&\Rightarrow &\omega (A)(q)=\frac{1}{2}A_{\lambda }^{a}(q)A_{\lambda }^{a}(-q)
\end{eqnarray}
which follows from the transversality of the gauge field, $q_\mu A^a_\mu(q)=0$. Also, it is useful to remind that, for an arbitrary function $\mathcal{F}(p^2)$, we have
\begin{equation}
\int \frac{d^{3}p}{(2\pi )^{3}}\left( \delta _{\mu \nu }-\frac{%
p_{\mu }p_{\nu }}{p^{2}}\right) \mathcal{F}(p^2)=\mathcal{A}\;\delta _{\mu \nu } \,.
\end{equation}%
Therefore, the no-pole condition for the ghost function $\mathcal{G}^{ab}(k,A)$ is implemented by imposing that \cite{Gribov:1977wm,Vandersickel:2012tz,Sobreiro:2005ec}
\begin{eqnarray}
\sigma _{off}(0;A) &\leq &1\;, \nonumber \\
\sigma _{diag}(0;A) &\leq &1   \label{np} \;.
\end{eqnarray}

%By taking the limit $k \to 0$ of eqs.\eqref{Goff} and \eqref{Gdiag} and by making use of the property
%\begin{eqnarray}
%A_{\mu }^{a}(q)A_{\nu }^{a}(-q) &=&  \left( \delta _{\mu \nu }-\frac{q_{\mu }q_{\nu }}{q^{2}}\right) \omega (A)(q)   \nonumber \\
%&\Rightarrow &\omega (A)(q) ~=~ \frac{1}{d-1}A_{\lambda }^{a}(q)A_{\lambda }^{a}(-q)\;,
%\end{eqnarray}
%which follows from the transversality of the gauge field, $q_\mu A^a_\mu(q)=0$, we are able find the following expressions for $\sigma_{off}(0;A)$, and $\sigma_{diag}(0;A)$,

%Besides that, it is useful to remind that, for an arbitrary function $\mathcal{F}(p^2)$, we have
%\begin{equation}
%\int \frac{\d^{d}p}{(2\pi )^{d}}  \;  \left( \delta _{\mu \nu }-\frac{p_{\mu }p_{\nu }}{p^{2}} \right) 
%\mathcal{F}(p^2)  ~=~  \mathcal{A}  \;  \delta _{\mu \nu }  \label{a}
%\end{equation}
%where, upon contracting both sides with $\delta_{\mu\nu}$,
%\begin{equation}
%\mathcal{A}=\frac{d-1}{d}\int \frac{\d^{d}p}{(2\pi )^{d}}\mathcal{F}(p^2).
%\end{equation}

After that two different parameters are needed in order to implement the no-pole condition in
the action, so restricting the path integral to the first Gribov region. Thus, we are led to
the following action accounting for the Gribov ambiguities,
\begin{eqnarray}
S\;' = S + \beta^* \left( \sigma_{off}(0,A) - 1 \right) + \omega^* \left( \sigma_{diag}(0,A) - 1 \right)   \;.
\label{effctactadjoit1}
\end{eqnarray}
In the action \eqref{effctactadjoit1} $\beta^*$ and $\omega^*$ are given dynamically through its own gap equation.









\subsection{The gluon propagator and the gap equation}

In order to obtain the partition function associated to the action \eqref{effctactadjoit1}, the
first step is to consider the standard Yang-Mills partition function within the first Gribov
region, $\Omega$. Namely, this restricted partition function reads
\cite{Gribov:1977wm,Sobreiro:2005ec,Vandersickel:2012tz},
\begin{equation}
Z=\int {[DA_{\mu }]\delta (\partial A)(\det\mathcal{M})\theta (1-\sigma
_{diag}(A))\theta (1-\sigma _{off}(A))e^{-S_{YM}}}\,.
\end{equation}
Since we are interested in the study of the gluon propagators, we shall consider the quadratic
approximation for the partition function, namely  
\begin{eqnarray}
Z_{quad} &=&\int \frac{d\beta }{2\pi i\beta }\frac{d\omega }{2\pi i\omega }%
DA_{\mu }e^{\beta (1-\sigma _{diag}(0,A))}e^{\omega \left( 1-\sigma
_{off}(0,A)\right) }  \nonumber  \label{Zq} \\
&\times &e^{-\frac{1}{4}\int d^{d}x(\partial _{\mu }A_{\nu }^{a}-\partial
_{\nu }A_{\mu }^{a})^{2}-\frac{1}{2\xi }\int {d^{d}x(\partial _{\mu }A_{\mu
}^{a})^{2}-}\frac{{g^{2}\nu ^{2}}}{2}\int {d^{d}xA_{\mu }^{\alpha }A_{\mu
}^{\alpha }}} \;,
\end{eqnarray}
where use has been made of the integral representation
\begin{equation}
\theta(x) = \int_{-i \infty +\epsilon}^{i\infty +\epsilon} \frac{d\beta}{2\pi i \beta} \; e^{\beta x}  \;. \label{step}
\end{equation}
The partition function accounting only for quadratic terms of the action
\eqref{effctactadjoit1} can be written as
%Just as was made before, the Gribov gap equation would be obtained by integrating out the equation
\begin{equation}
Z_{quad} ~=~ \int \frac{\d\beta e^{\beta }}{2\pi i\beta }\frac{\d\omega e^{\omega }} {2\pi i\omega }[\d A^{\alpha }][\d A^{3}] \; e^{-\frac{1}{2}  \int \frac{\d^{d}q}{(2\pi )^{d}}  \;  A_{\mu }^{\alpha }(q)\mathcal{P}_{\mu \nu }^{\alpha
\beta }A_{\nu }^{\beta }(-q)-\frac{1}{2}  \int \frac{\d^{d}q}{(2\pi )^{d}} \;  A_{\mu }^{3}(q)\mathcal{Q}_{\mu \nu }A_{\nu }^{3}(-q)},  
\label{Zq1}
\end{equation}
with
\begin{eqnarray}
\mathcal{P}_{\mu \nu }^{\alpha \beta } ~=~  \delta^{\alpha \beta } \left(\delta _{\mu \nu }
\left( q^{2}+\nu^{2}g^{2}\right) +\left( \frac{1}{\xi } -1 \right) q_{\mu }q_{\nu } +
\frac{2g^{2}}{Vd}\left( \beta +\frac{\omega }{2} \right) \frac{1}{q^{2}}\delta _{\mu \nu
}\right)  \label{P}  \;,
\end{eqnarray}
and
\begin{eqnarray}
\mathcal{Q}_{\mu \nu } ~=~ \delta_{\mu \nu }\left( q^{2} - \frac{2\omega
g^{2}}{Vd}\frac{1}{q^{2}}\right) +\left( \frac{1}{\xi }-1\right) q_{\mu }q_{\nu } \;.
\label{Q}
\end{eqnarray}
The parameter $\xi$ stands for the usual gauge fixing parameter, to be put to zero at the end in order to recover the Landau gauge. Evaluating  the inverse of the expressions \eqref{Q} and taking the limit $\xi\rightarrow 0$, the gluon propagators become
\begin{eqnarray}
\left\langle A_{\mu }^{3}(q)A_{\nu }^{3}(-q)\right\rangle &=&\frac{q^{2}}{q^{4}+  \frac{2\omega g^{2}}{Vd} }  \left( \delta _{\mu \nu }-\frac{q_{\mu }q_{\nu}}{q^{2}}\right)  \label{Pdiag} \;, 
\\
\left\langle A_{\mu}^{\alpha}(q) A_{\nu}^{\beta }(-q)\right\rangle  &=&  \delta^{\alpha\beta} 
\frac{q^{2}}{q^{2} \left( q^{2}+g^{2}\nu^{2}\right) +  \frac{2g^{2}}{Vd} \left(  \beta + \frac{\omega}{2}\right)}  
\left(\delta_{\mu\nu} - \frac{q_{\mu}q_{\nu}}{q^{2}}\right)  \;.
\label{NPoff} 
\end{eqnarray}
The off-diagonal sector of the gluon propagator can be put in a more convenient form, where its
poles are explicitly written,
\begin{eqnarray}
\left\langle A_{\mu}^{\alpha}(q) A_{\nu}^{\beta }(-q)\right\rangle 
% &=&  \delta^{\alpha\beta} 
%\frac{q^{2}}{q^{2} \left( q^{2}+g^{2}\nu^{2}\right) +  \frac{2g^{2}}{Vd} \left(  \beta + \frac{\omega}{2}\right)}  
%\left(\delta_{\mu\nu} - \frac{q_{\mu}q_{\nu}}{q^{2}}\right)
%\nonumber  \\
&=&  \frac{\delta ^{\alpha \beta }}{m^2_+-m^2_-}   \left(  \frac{m^2_+}{q^2+m^2_+} -\frac{m^2_{-}}{q^2+m^2_-}  \right)
\left( \delta _{\mu\nu }-\frac{q_{\mu }q_{\nu }}{q^{2}}\right)  \;,  
\label{NPoff_f1}
\end{eqnarray}
with
\begin{equation}
m^2_+ = \frac{g^2\nu^2 + \sqrt{g^4 \nu^4 - 4 \tau}}{2}  \;,  \qquad m^2_- = \frac{g^2\nu^2 - \sqrt{g^4 \nu^4 - 4 \tau}}{2} 
\;, \qquad  \tau = \frac{2g^2}{Vd} \left( \beta +\frac{\omega}{2} \right)  \;.\label{masses}
\end{equation}



Since the Gribov parameters $(\beta, \omega)$ are fixed dynamically through the gap equation,
now we should integrate out the gauge field from equation \eqref{Zq1} and make use of the
saddle-point approximation, in the thermodynamic limit, which will gives us two gap equations,
enabling us to express $\beta$ and $\omega$ in terms of the parameters of the starting model,
{\it i.e.} the gauge coupling constant $g$ and the {\it vev} of the Higgs field $\nu$. That is,
firstly, we integrate out the gauge fields, obtaining
\begin{equation}
Z_{quad}=\int{\frac{d\beta}{2\pi i\beta}\frac{d\omega}{2\pi i\omega}}%
e^{\beta}e^{\omega}\left(\det\mathcal{Q}_{\mu\nu}\right)^{-\frac{1}{2}%
}\left(\det\mathcal{P}^{\alpha\beta}_{\mu\nu}\right)^{-\frac{1}{2}} \;.
\label{Zq2}
\end{equation}
By making use of the following property of functional determinants,
\begin{equation}
\left(\det \mathcal{A}_{\mu\nu}^{ab}\right)^{-\frac{1}{2}}=e^{-\frac{1}{2}%
\ln \det \mathcal{A}_{\mu\nu}^{ab}}=e^{-\frac{1}{2}Tr \ln \mathcal{A}%
_{\mu\nu}^{ab}} \;,
\end{equation}
for those determinants in expression (\ref{Zq2}), one gets
\begin{eqnarray}
\left( \det \mathcal{Q}_{\mu \nu }\right) ^{-\frac{1}{2}} &=&\exp \left[
-\int {\frac{d^{d}q}{(2\pi )^{d}}\ln \left( q^{2}+\frac{2\omega
g^{2}}{Vd}\frac{1}{q^{2}}\right) }\right] \;, \nonumber \\
\left( \det \mathcal{P}_{\mu \nu }^{\alpha \beta }\right) ^{-\frac{1}{2}}
&=&\exp \left[ -2\int {\frac{d^{d}q}{(2\pi )^{d}}\ln \left(
(q^{2}+g^{2}\nu^{2}) + \frac{g^{2}}{Vd} \left( 2\beta + \omega \right)
\frac{1}{q^{2}}\right) }\right] \;.
\end{eqnarray}
At the end, we have
%Performing the Gaussian integral of the partition function \eqref{Zq1} we get\footnote{In order
%to obtain eq. \eqref{Zq4} we had to take the functional trace of $\ln \mathcal{P}$ and $\ln
%\mathcal{Q}$ following the process described in subsection \ref{Gribovimplementation}. For more
%details take a look at \cite{Vandersickel:2012tz}.}
\begin{equation}  \label{Zq3}
Z_{quad}=\int{\frac{\d\beta}{2\pi i}\frac{\d\omega}{2\pi i}}e^{f(\omega,\beta)} \;,
\end{equation}
with
\begin{eqnarray}\label{Zq4}
f(\omega ,\beta ) &=& \beta +\omega -\ln \beta -\ln \omega - \frac{(d-1)V}{2}  \int
{\frac{\d^{d}q}{(2\pi )^{d}}  \; \ln \left( q^{2}+\frac{2\omega
g^{2}}{Vd}\frac{1}{q^{2}}\right)
} \nonumber  \\
&-&  \frac{2(d-1)V}{2}  \int {\frac{\d^{d}q}{(2\pi)^{d}}  \; \ln \left((q^{2}+g^{2}\nu^{2}) + \frac{g^{2}}{Vd} \left( 2\beta + \omega \right)
\frac{1}{q^{2}}\right) } \;.
\end{eqnarray}
Since in the thermodynamic limit, as mentioned in the section \ref{Gribovimplementation}, the
integral \eqref{Zq3} can be solved through the saddle point approximation,
%\begin{equation}
%Z_{quad}\approx e^{f(\beta^*,\omega^*)} \;,
%\end{equation}
%with $\beta^*$ and $\omega^*$ being determined by the stationary conditions
\begin{equation}
\frac{\partial f}{\partial \beta^*}=\frac{\partial f}{\partial \omega^*}=0 \;,
\end{equation}
 leading to
\cite{Gribov:1977wm,Sobreiro:2005ec,Vandersickel:2012tz}
\begin{equation}
Z_{quad}\approx e^{f(\beta^*,\omega^*)} \;,
\end{equation}
one gets the following two gap equations
%\footnote{We remind here that the terms $\log\beta$ and $\log \omega$ can be neglected in the derivation of the gap equations, eqs.\eqref{gap1} \eqref{gap2}, when taking the thermodynamic limit \cite{Gribov:1977wm,Sobreiro:2005ec,Vandersickel:2012tz}.}:
\begin{eqnarray}
\frac{4(d-1)g^{2}}{2d} \int \frac{\d^{d}q}{(2\pi )^{d}} \frac{1}{q^{4}+\frac{2\omega^{\ast}g^{2}}{d}}  &=&1 \;, 
\label{gap1} \\
\frac{4(d-1)g^{2}}{2d}  \int \frac{\d^{d}q}{(2\pi )^{d}}\left( \frac{1}{q^{2}(q^{2}+g^{2}\nu^{2})+g^{2}\left( \frac{2\beta ^{\ast}}{d}+\frac{\omega^{\ast}}{d}\right) }\right) &=&1 \;.
\label{gap2}
\end{eqnarray}
Therefore, $\beta^*$ and $\omega^*$ can be expressed in terms of the parameters $\nu,g$. To solve the gap equations the denominator of eq.\eqref{gap2} can be decomposed into its poles, which is similar to \eqref{NPoff}--\eqref{masses}.

Let us assume the particular cases of $d=3$ and $d=4$ Euclidean space-times. In the light of
the gap equation in each situation, we will analyse what happens to the diagonal and
off-diagonal propagators.






%\subsubsection{Gluon propagators}

%Just as already done before, let's decompose the gluon propagator in order to simplify the analysis of its poles. In particular, the propagator \eqref{Pdiag} does not need to be decomposed, since it is quite similar to Gribov's propagator \eqref{Gribovprop1}. By the other side, the analysis of the off-diagonal sector propagator \eqref{NPoff} becomes easier if it is written as


%and
%\begin{equation}
%{\cal R}_{+} =   \;, \qquad {\cal R}_{-} = \frac{m^2_-}{m^2_+-m^2_-}  \label{residues} \;.
%\end{equation}















\subsection{The $d=3$ case}
\label{3dsu2}


%The first integral is easy to compute, giving $\omega^* $ as a function of $g$
%only

%In order to solve the second gap equation \eqref{gap2}  we compute the roots of the
%denominator:
%\begin{equation}
%m_{\pm }^{2}=\frac{-g^{2}\nu^{2}\pm \sqrt{g^{4}\nu^{4}-4\tau }}{2}  \label{p+-}  \;.
%\end{equation}%
%Notice that the roots are real when
%\begin{equation}
% \tau \leq \frac{g^{4}\nu ^{4}}{4}  \qquad \text{with} \qquad \tau =\beta^* \frac{2g^{2}}{3}+\omega^* \frac{g^{2}}{3} \;.
%\label{cond}
%\end{equation}
%After decomposition in partial fractions,  equation \eqref{gap2}  becomes
%\begin{eqnarray}
%\frac{4g^{2}\pi }{(2\pi )^{3}(q_{+}^{2}-q_{-}^{2})}\int_{0}^{\infty
%}dq\left( \frac{q^{2}}{(q^{2}-q_{+}^{2})}-\frac{q^{2}}{(q^{2}-q_{-}^{2})}%
%\right) &=&1 \;.
%\end{eqnarray}%
%Using the principal value prescription, this yields the final (finite) result
%\begin{equation}
%\frac{ig^{2}}{(4\pi )}\frac{1}{q_{+}-q_{-}}=1 \;.  \label{one}
%\end{equation}
%Making use of expression \eqref{p+-}, equation \eqref{one} gives $\tau$ as function of the parameters $(\nu,g)$, {\it i.e.}





%\subsubsection{Analysis of the gluon propagator}

In the three-dimensional case both gap equations cause not many difficulties to be solved, as there are no divergences to be treated. Namely, the first gap equation, eq.\eqref{gap1}, leads to the following result,
\begin{eqnarray}
\omega^*(g)= \frac{3}{2^{11}\pi^4} \;g^6        \;,  \label{omega}
\end{eqnarray}
while the second one, given by eq.\eqref{gap2}, leads to
\begin{equation}
\tau =\beta^* \frac{2g^{2}}{3}+\omega^* \frac{g^{2}}{3}  = \left[ \frac{1}{2}g^{2}\nu^{2}-\frac{g^{4}}{32\pi^2 }\right] ^{2}  \;. \label{feq}
\end{equation}

Now we can look at the gluon propagators, \eqref{Pdiag} and \eqref{NPoff}, and analyse the different regions in the $(g,\nu)$ plane. Let us start by the diagonal component $A^3_\mu$. Namely, we have
\begin{equation}
\left\langle A_{\mu }^{3}(q)A_{\nu }^{3}(-q)\right\rangle =\frac{q^{2}}{%
q^{4}+\frac{2\omega^* g^{2}}{3}}\left( \delta _{\mu \nu }-\frac{q_{\mu }q_{\nu
}}{q^{2}}\right)  \label{Pdiagf} \;.
% \qquad \omega^*(g)= \frac{3}{2^{11}\pi^4} \;g^6  \;.
\end{equation}
One observes that expression \eqref{Pdiagf} turns out to be independent from the $vev$ $\nu$ of the Higgs field, while displaying two complex conjugate poles. This gauge component is thus of the Gribov type. In other words, the mode $A^3_\mu$ is always confined, for all values of the parameters $g, \nu$. Concerning now the off-diagonal gluon propagator \eqref{NPoff}, after decomposing it into two Yukawa modes  \eqref{NPoff_f1}, we could find the following regions in the $(g,\nu)$ plane:

\begin{itemize}
\item[\it i)] 
when $g^2 < 32 \pi^2 \nu^2$, corresponding to $\tau < \frac{g^2\nu^2}{4}$, both masses $m^2_+, m^2_-$ are real, positive and  different from each other. Moreover, due to the presence of the relative minus sign in expression \eqref{NPoff_f1}, only the heaviest mode with mass $m^2_+$ represents a physical excitation --- {\it i.e.}, despite the existence of two real positive poles, $m^2_{+}$ and $m^{2}_{-}$, only the contribution related to the $m^{2}_{+}$ pole has physical meaning.

It is also worth observing that, for the particular value $g=16 \pi^2 \nu^2$, corresponding to $\tau=0$, the unphysical mode in the decomposition \eqref{NPoff_f1} disappears. Thus, for that particular value of the gauge coupling, the off-diagonal propagator reduces to a single physical Yukawa mode with mass $16\pi^2\nu^4$.
%, {\it i.e.}
%\begin{equation}
%\left\langle A_{\mu }^{\alpha }(q)A_{\nu }^{\beta }(-q)\right\rangle
%= \delta ^{\alpha \beta }  \left(  \frac{1}{q^2+16\pi^2\nu^4} \right)
%\left( \delta _{\mu\nu }-\frac{q_{\mu }q_{\nu }}{q^{2}}\right)  \;,  \label{off-Yuk}
%\end{equation}
\item[\it ii)] when $g^2>32\pi^2\nu^2$, corresponding to $\tau> \frac{g^2\nu^2}{4}$, all masses become complex and the off-diagonal propagator becomes of the Gribov type with two complex conjugate poles. This region, called Gribov region since all modes are of Gribov type, corresponds to a phase in which all gauge modes are said to be confined.
\end{itemize}
In summary, when the Higgs field is in the adjoint representation we could find two distinct regions. For $g^2<32\pi^2\nu^2$ the $A_3$ mode is confined while the off-diagonal propagator displays a physical Yukawa mode with mass $m^2_+$. This phase is referred to as the $U(1)$ symmetric phase \cite{Nadkarni:1989na,Hart:1996ac}. When $g^2>32\pi^2\nu^2$ all propagators are of the Gribov type, displaying complex conjugate poles leading to a confinement interpretation. According to \cite{Nadkarni:1989na,Hart:1996ac} this regime is referred to as the $SU(2)$ confined phase. 

Since our results were obtained in a semi-classical approximation ({\it i.e.}, lowest order in the loop expansion), let us comment on the validity of such approximation. In general, the perturbation theory is reliable when the \emph{effective coupling constant} is sufficiently small. The effective coupling depends, in $3d$, on the factor $\frac{g^2}{(4\pi)^{3/2}}$. However, since $g^2$ has mass dimension $1$ the effective coupling is not complete yet. In the presence of a mass scale $M$, the perturbative series --- for e.g.~the gap equation --- will organize itself automatically in a series in $G^2/M$. Let us analyse, for example, the case where $g^2 < 32 \pi^2 \nu^2$, the called the ``Higgs phase''. In this case the effective coupling will be sufficiently small when $\frac{g^2}{\nu^2(4\pi)^{3/2}}$ is small compared\footnote{The Higgs mass $\nu^2$ is then the only mass scale entering the game.} to $1$. Such condition is not at odds with the retrieved condition $g^2 < 32 \pi^2 \nu^2$. Next, assuming the coupling $g^2$ to get large compared to $\nu^2$, thereby entering the confinement phase with $cc$ masses, $g^2$ dominates everything, leading to a Gribov mass scale $\tau\propto g^8$, and an appropriate power of the latter will secure a small effective expansion parameter consistent with the condition $g^2 > 32 \pi^2 \nu^2$. We thus find that at sufficiently small and large values of $\frac{g^2}{\nu^2}$ our approximation and results are trustworthy.





















\subsection{The $d=4$ case}

Let us start by considering the second gap equation, eq.\eqref{gap2}. Performing the decomposition described in eq.\eqref{masses} the referred gap equation becomes of the same form as the one obtained in the fundamental $d=4$ case, eq.\eqref{gapf1}. The difference between the fundamental and adjoint $d=4$ cases appears in the definition of the mass parameter $m^{2}_{\pm}$ (see eq.\eqref{roots1} and eq.\eqref{masses}, respectively). Namely,
%\begin{equation}
%q^4 + g^2 \nu^2 q^2 + \tau = (q^2+m^2_+) (q^2+m^2_-) \;, \qquad \tau = g^2 \left( \frac{\beta^*}{2}+\frac{\omega^*}{4} \right)  \;, \label{dec}
%\end{equation}
%with $m^{2}_{\pm}$ given by equation \eqref{masses}. Let us discuss first the case of two real,  positive, different roots, namely $0 \le \tau <  \frac{g^4\nu^4}{4}$. From
%\eqref{intd2f}, we reexpress the gap equation \eqref{gap2} as
\begin{equation}
\left(1 + \frac{m^2_-}{m^2_+ -  m^2_-}\; \ln\left( \frac{m^2_-}{{\omu}^2} \right)  - \frac{m^2_+}{m^2_+ -  m^2_-}\; \ln\left( \frac{m^2_+}{{\omu}^2} \right)  \right) = \frac{32\pi^2}{3g^2}  \;. \label{gap2bb}
\end{equation}
Introducing now the dimensionless variables\footnote{We introduced the renormalization group invariant scale $\lms$. }
\begin{eqnarray}
b ~=~  \frac{g^2\nu^2}{2 {\bar \mu}^2\; e^{\left(1-\frac{32\pi^2}{3g^2}\right)} } =\frac{1}{2\; e^{\left( 1 -\frac{272 \pi^2}{21 g^2} \right)}}  \;\frac{g^2\nu^2}{\lms^2  } \;, \qquad \text{and} \qquad 
 \xi  ~=~  \frac{4\tau}{g^4\nu^4}  \geq 0\;,
\label{vb}
\end{eqnarray}
with $0 \le \xi < 1 \;$. Proceeding as in the fundamental $d=4$ case, eq. \eqref{gap2bb} can be recast in the following form
\begin{equation}
2 \sqrt{1-\xi}\; \ln(b) = -  \left( 1 +  \sqrt{1-\xi} \right) \; \ln\left( 1 +  \sqrt{1-\xi} \right) +  \left( 1 -  \sqrt{1-\xi} \right) \; \ln\left( 1 - \sqrt{1-\xi} \right)  \;. \label{gapdd}
\end{equation}
or compactly,
\begin{equation}
2\ln b=g(\xi)\;. \label{gapddbis}
\end{equation}
Also here, in the adjoint $d=4$ case, eq.\eqref{gapddbis} remains valid also for complex conjugate roots, viz.~$\xi>1$. We are then led to the following cases.


\subsubsection{When $b<\frac{1}{2}$}
Using the properties of $g(\xi)$, it turns out that eq.\eqref{gapdd} admits a unique solution for $\xi$, which can be explicitly constructed with a numerical approach. More precisely, when the mass scale $g^2\nu^2$ is sufficiently smaller  than $\lms^2$, {\it i.e.}
\begin{equation}
g^2 \nu^2 < 2 \; e^{\left( 1 -\frac{272 \pi^2}{21 g^2} \right)} \; \lms^2   \;, \label{mh1}
\end{equation}
we have what can be called the $U(1)$ confined phase. In fact, in this regime the gap equation \eqref{gap1} leads to a non-null $\omega^*$, so that the diagonal component of the gauge field is said to be of the Gribov type, {\it i.e.} with confinement interpretation.
%and \eqref{gap2} can be rewritten as
%\begin{eqnarray}
%3 \left( \frac{g^{2}}{2}\right) \int \frac{d^{4}q}{(2\pi )^{4}}\left( \frac{1}{%
%q^{4}+g^2 \frac{\omega^*}{2}}\right) &=&1\;, \label{ggapp1} \\
%3 \left( \frac{g^{2}}{2}\right) \int \frac{d^{4}q}{(2\pi )^{4}}\left( \frac{1%
%}{q^{2}(q^{2}+g^{2}\nu^{2})+g^2 (\frac{\beta^*}{2}+\frac{\omega^*}{4})}
%\right) &=&1\label{ggapp2} \;.
%\end{eqnarray}
%Moreover, making use of
%\begin{equation}
%\int \frac{d^dp}{(2\pi)^d} \frac{1}{p^4 + m^4}  = \frac{2}{\bar \varepsilon} \frac{1}{16\pi^2} - \frac{1}{16\pi^2} \left( \ln\frac{m^2}{\omu^2} - 1 \right) \;,  \label{intd1}
%\end{equation}
%the gap equation \eqref{ggapp1} gives
%\begin{equation}
%{\left( \frac{g^2 \omega^*}{2} \right)}^{1/2} = {\bar \mu}^2  e^{ \left(1 - \frac{32\pi^2}{3g^2} \right)}  = \lms^2\; e^{\left( 1 -\frac{272 \pi^2}{21 g^2} \right)}  \;.
%\end{equation}
%Therefore, for $b<\frac{1}{2}$, the $A^3_\mu$ component of the gauge field gets confined, exhibiting a Gribov-type propagator with complex poles, namely
%\begin{equation}
%\left\langle A_{\mu }^{3}(q)A_{\nu }^{3}(-q)\right\rangle =\frac{q^{2}}{%
%q^{4}+\frac{\omega^* g^{2}}{2}}\left( \delta _{\mu \nu }-\frac{q_{\mu }q_{\nu
%}}{q^{2}}\right)  \;. \label{Pdiag_conf}
%\end{equation}

On the other side, the second gap equation \eqref{gapdd} splits this region in the two subregions:
\begin{itemize}
\item[(i)]  when $\frac{1}{e}<b<\frac{1}{2}$ equation  \eqref{gapdd} has a single solution with $0 \le \xi <1$. In this region, the roots $(m^2_+, m^2_-)$  are thus real and  the off-diagonal propagator decomposes into the sum of two Yukawa propagators.
%\begin{equation}
%\left\langle A_{\mu }^{\alpha }(q)A_{\nu }^{\beta }(-q)\right\rangle
%=\delta ^{\alpha \beta } \left(  \frac{{\cal R}_+}{m^2+m^2_+} -    \frac{{\cal R}_-}{m^2+m^2_-}   \right)
%\left( \delta _{\mu
%\nu }-\frac{q_{\mu }q_{\nu }}{q^{2}}\right)  \label{NPoff_fin} \;,
%\end{equation}
%where
%\begin{equation}
%{\cal R}_+ = \frac{m^2_+}{m^2_+-m^2_-}  \;, \qquad {\cal R}_- = \frac{m^2_-}{m^2_+ - m^2_-} \;. \label{r}
%\end{equation}

However, due to the relative minus sign in eq.\eqref{NPoff_f1},  only the component with $m^{2}_+ $ pole can be associated to a physical mode, analogously as in the fundamental case. Due to the confinement of the third component $A^3_\mu$, this phase is recognized as the $U(1)$ confining phase. It is worth observing that it is also present in the $3d$ case, with terminology coined in \cite{Nadkarni:1989na}, see also \cite{Capri:2012cr}.

\item[(ii)]  for $b<\frac{1}{e}$, equation \eqref{gapdd} has a solution for  $\xi>1$. In this
region the roots  $(m^2_+, m^2_-)$  become complex conjugate and the off-diagonal gluon
propagator is of the Gribov type, displaying complex poles.  In this region all gauge fields
display a propagator of the Gribov type. This is recognized as the $SU(2)$ {\it confined-like}
regime.
\end{itemize}
Similarly, the above regions are continuously connected when $b$ varies. In particular, for $b \stackrel{<}{\to} \frac{1}{2}$, we obtain $\xi=0$ as solution.



\subsubsection{The case $b>\frac{1}{2}$}
Let us consider now the case in which $b>\frac{1}{2}$. Here, there is no solution of the equation \eqref{gapdd} for the parameter $\xi$, as it follows by observing that the left hand side of eq.\eqref{gapdd} is always positive, while the right hand side is always negative. This has a deep physical consequence. It means that for a Higgs mass $m_{Higgs}^2 = g^2\nu^2$ sufficiently larger than $\lms^2$, {\it i.e.}
\begin{equation}
g^2 \nu^2 > 2 \; e^{\left( 1 -\frac{272 \pi^2}{21 g^2} \right)} \; \lms^2   \;, \label{mh}
\end{equation}
the gap equation \eqref{gap2} is inconsistent. It is then important to realize that this is actually the gap equation obtained by acting with $\frac{\p}{\p \beta}$ on the vacuum energy  ${\cal E}_v= -  f(\omega ,\beta )$.
So, we are forced to set $\beta^*=0$, and confront the remaining $\omega$-equation, viz.~eq.\eqref{gap1},
%:
%\begin{equation}
%\frac{3}{2}\left( \frac{g^{2}}{2}\right) \int \frac{d^{4}q}{(2\pi )^{4}}\left( \frac{1}{q^{4}+\frac{%
%\omega ^{\ast }g^{2}}{2}}+\frac{1}{q^{2}(q^{2}+g^{2}\nu^{2})+\frac{g^2\omega ^{\ast }}{4} }\right) =1 \;, \label{gap1n}
%\end{equation}
which can be transformed into
\begin{eqnarray}
4 \; \ln(b) & = &  \frac{1}{\sqrt{1-\xi}} \left[ -\left( 1 +  \sqrt{1-\xi} \right) \; \ln\left( 1 +  \sqrt{1-\xi} \right) +  \left( 1 -  \sqrt{1-\xi} \right) \; \ln\left( 1 - \sqrt{1-\xi} \right) \right.  \nonumber \\
 &\; \;&-\left. \sqrt{1-\xi}\; \ln\xi - \sqrt{1-\xi} \ln 2\right]\equiv h(\xi)
 \label{gapddn}
\end{eqnarray}
after a little algebra, where $\xi=\frac{\omega^*}{g^4\nu^4}$. The behaviour of $h(\xi)$ for $\xi\geq0$ is more complicated than that of $g(\xi)$. Because of the $-\ln\xi$ contribution, $h(\xi)$ becomes more and more positive when $\xi$ approaches zero. In fact, $h(\xi)$ strictly decreases from $+\infty$ to $-\infty$ for $\xi$ ranging from 0 to $+\infty$.

It is interesting to consider first the limiting case $b \stackrel{>}{\to} \frac{1}{2}$, yielding $\xi\approx 1.0612$.   So, there is a discontinuous jump in $\xi$ (i.e.~the Gribov parameter for fixed $v$) when the parameter $b$ crosses the boundary value $\frac{1}{2}$.

We were able to separate the $b>\frac{1}{2}$ region as follows:
\begin{itemize}
\item[(a)] For $\frac{1}{2}<b<\frac{1}{\sqrt{\sqrt{2}e}}\approx 0.51$, we have a unique solution $\xi>1$, i.e.~we are in the confining region again, with all gauge bosons displaying a Gribov type of propagator with complex conjugate poles.
\item[(b)] For $\frac{1}{\sqrt{\sqrt{2}e}}< b<\infty$, we have a unique solution $\xi<1$, indicating again a combination of two Yukawa modes for the off-diagonal gauge bosons. The ``photon'' is still of the Gribov type, thus confined.
\end{itemize}
Completely analogous as in the fundamental case, it can be checked by addressing the averages of the expressions \eqref{sigma} that for $b>\frac{1}{2}$ and $\omega$ obeying the gap equation with $\beta=0$, we are already within the Gribov horizon, making the introduction of the second Gribov parameter $\beta$ obsolete.

It is obvious that the transitions in the adjoint case are far more intricate than in the earlier studied fundamental case. First of all, we notice that the ``photon'' (diagonal gauge boson) is confined according to its Gribov propagator. There is never a Coulomb phase for $b<\infty$. The latter finding can be understood again from the viewpoint of the ghost self-energy. If the diagonal gluon would remain Coulomb (massless), the off-diagonal ghost self-energy, cfr.~eq.\eqref{sigma}, will contain an untamed infrared contribution from this massless photon\footnote{The ``photon'' indeed keeps it coupling to the charged (= off-diagonal) ghosts, as can be read off directly from the Faddeev-Popov term $c^a \p_\mu D_\mu^{ab}c^b$.}, leading to an off-diagonal ghost self-energy that will cross the value 1 at a momentum $k^2>0$, indicative of trespassing the first Gribov horizon. This crossing will not be prevented at any finite value of the Higgs condensate $\nu$, thus we are forced to impose at any time a nonvanishing Gribov parameter $\omega$. Treating the gauge copy problem for the adjoint Higgs sector will screen (rather confine) the a priori massless ``photon''.

An interesting limiting case is that of infinite Higgs condensate, also considered in the lattice study of \cite{Brower:1982yn}. Assuming $\nu\to\infty$, we have $b\to\infty$ according to its definition \eqref{vb}. Expanding the gap equation \eqref{gapddn} around $\xi=0^+$, we find the limiting equation $b^4=\frac{1}{\xi}$, or equivalently $\omega^*\propto \lms^8/g^4\nu^4$. Said otherwise, we find that also the second Gribov parameter vanishes in the limit of infinite Higgs condensate. As a consequence, the photon becomes truly massless in this limit. This result provides ---in our opinion--- a kind of continuum version of the existence of the Coulomb phase in the same limit as in the lattice version of the model probed in \cite{Brower:1982yn}. It is instructive to link this back to the off-diagonal no pole function, see Eq.~\eqref{sigma}, as we have argued in the proceeding paragraph that the massless photon leads to $\sigma_{off}(0)>1$ upon taking averages. However, there is an intricate combination of the limits $\nu\to\infty$, $\omega^*\to 0$ preventing such a problem here. Indeed, we find in these limits, again using dimensional regularization in the $\MSbar$ scheme, that
\begin{align}\label{dv}
  \sigma_{off}(0) &= \frac{3g^2}4 \left(\int \frac{d^4q}{(2\pi)^4} \frac1{q^4+\frac{\omega^\ast g^2}2} + \int \frac{d^4q}{(2\pi)^4} \frac1{q^2(q^2+g^2\nu^2)+\frac{\omega^\ast g^2}4}\right) \nonumber\\
	&= -\frac{3g^2}{128\pi^2} \left(\tfrac12 \ln\frac{\omega^\ast g^2}{2\overline\mu^4} + \ln\frac{g^2\nu^2}{\overline\mu^2}-2\right) \nonumber \\
	&= -\frac{3g^2}{128\pi^2} \left(\tfrac12 \ln\frac{\omega^\ast g^6\nu^4}{2\overline\mu^8}-2\right) \stackrel{b^4=\xi^{-1}}{\longrightarrow} -\frac{3g^2}{64\pi^2} \ln 8g^2 + \frac12.
\end{align}
The latter quantity is always smaller than $1$ for $g^2$ positive, meaning that we did not cross the Gribov horizon. This observation confirm in an explicit way the intuitive reasoning also found in section 3.4 of \cite{Lenz:2000zt}, at least in the limit $\nu\to\infty$. The subtle point in the above analysis is that it is not allowed to naively throw away the 2nd integral in the first line of \eqref{dv} for $\nu\to\infty$. There is a logarithmic $\ln\nu$ ($\nu\to\infty$) divergence that conspires with the $\ln \omega^*$ ($\omega^*\to0$) divergence of the 1st integral to yield the final reported result. This displays that, as usual, certain care is needed when taking infinite mass limits in Feynman integrals. 








\subsection{The vacuum energy in the adjoint representation}
As done in the case of the fundamental representation, let us work out the expression of the vacuum energy ${\cal E}_v$, for which we have the one loop  integral representation given by eq.\eqref{Zq4} multiplied by $-1$.
%\begin{eqnarray}
%{\cal E}_v= -\beta^*-\omega^*+\frac{3}{2}\int \frac{d^4q}{(2\pi)^4}\ln(q^4+2\tau')+3\int \frac{d^4q}{(2\pi)^4} \ln(q^4+g^2\nu^2q^2+\tau)
%\end{eqnarray}
%where
%\begin{equation}\label{ddvac1}
%    \tau'=\frac{g^2\omega^*}{4}\,,\qquad \tau=g^2\left(\frac{\beta^*}{2}+\frac{\omega^*}{4}\right).
%\end{equation}
%Based on \eqref{intdl} and on
%\begin{equation}\label{ddvac2}
%    \int\frac{d^4q}{(2\pi)^4}\ln(q^4+m^4)=-\frac{m^4}{32\pi^2}\left(\ln\frac{m^4}{\omu^4}-3\right),
%\end{equation}
%we find
Making use of the $\MSbar$ renormalization scheme in $d = 4 - \varepsilon$ the vacuum energy becomes
%\begin{eqnarray}\label{ddvac3}
%{\cal E}_v= -\frac{2}{g^2}(\tau+\tau')-\frac{3\tau'}{32\pi^2}\left(\ln\frac{2\tau'}{\omu^4}-3\right)+\frac{3}{32\pi^2}\left(m_-^4\left(\ln\frac{m_-^2}{\omu^2}-\frac{3}{2}\right)+q_+^4\left(\ln\frac{m_+^2}{\omu^2}-\frac{3}{2}\right)\right),
%\end{eqnarray}

%We can introduce $b$ via its definition \eqref{vb} to write after simplification
\begin{eqnarray}\label{ddvac4}
\frac{{\cal E}_v}{g^4\nu^4}&=& -\frac{1}{g^2}-\frac{3\xi'}{128\pi^2}\left(\ln(2b^2\xi')-1\right)+\frac{3(4-2\xi)}{128\pi^2}\left(\ln b-\frac{1}{2}\right)\nonumber\\
&&+\frac{3}{128\pi^2}\left(\left(1-\sqrt{1-\xi}\right)^2\ln\left(1-\sqrt{1-\xi}\right)+\left(1+\sqrt{1-\xi}\right)^2\ln\left(1+\sqrt{1-\xi}\right)\right)\;,
\end{eqnarray}
where $b$ was introduced via its definition \eqref{vb}, while
\begin{equation}\label{ddvac5}
    \xi'=\frac{4\tau'}{g^4\nu^4}\,,\qquad \xi=\frac{4\tau}{g^4\nu^4} 
\qquad \text{and} \qquad 
\tau'=\frac{g^2\omega^*}{4}\,,\qquad \tau=g^2\left(\frac{\beta^*}{2}+\frac{\omega^*}{4}\right)    \;.
\end{equation}

Since we found scenarios completely different for $b<1/2$ and $b>1/2$ with the scalar Higgs
field in its adjoint representation, it becomes of great importance analysing the plot of the
vacuum energy as a function of $b$. From Figure \ref{BS-d4} one can easily find out a clear jump for $b=1/2$, which can be seen as a reflection of the discontinuity of the parameter  $\xi$.
%First, for $b<1/2$, we can use the  $\omega$-gap equation \eqref{ggapp1} to establish $\xi'=\frac{1}{2b^2}$, and thus
%\begin{eqnarray}\label{ddvac4bis}
%\frac{{\cal E}_v}{g^4\nu^4}&=& -\frac{1}{g^2}+\frac{3}{256b^2\pi^2}+\frac{3(4-2\xi)}{128\pi^2}\left(\ln b-\frac{1}{2}\right)\nonumber\\
%&&+\frac{3}{128\pi^2}\left(\left(1-\sqrt{1-\xi}\right)^2\ln\left(1-\sqrt{1-\xi}\right)+\left(1+\sqrt{1-\xi}\right)^2\ln\left(1+\sqrt{1-\xi}\right)\right)  \;,
%\end{eqnarray}
%where $\xi$ is determined by the equations  \eqref{gapdd},\eqref{gapddbis}.
%For $b>1/2$, we can remember that $\beta=0$ and thus $\xi=\xi'$, in which case we can easily obtain
%\begin{eqnarray}
%\frac{{\cal E}_v}{g^4 \nu^4} & = & - \left( \frac{1}{g^2} + \frac{3}{64\pi^2}  \right)   + \frac{3}{64\pi^2} \xi+ \frac{3}{32\pi^2} \frac{1}{4} (4-2\xi) \ln(b) \nonumber \\
%&+& \frac{3}{32\pi^2} \frac{1}{4} \left( \left(1+\sqrt{1-\xi}\right)^2\;\ln\left(1+\sqrt{1-\xi}\right) + \left(1-\sqrt{1-\xi}\right)^2\;\ln\left(1-\sqrt{1-\xi}\right) \right) \nonumber \\
%&-& \frac{3}{32\pi^2} \frac{1}{4} \left(2\xi \ln(\sqrt{\xi}) +2\xi \ln\sqrt{2} + 2\xi \ln(b)   \right)   \;, \label{Evv2}
%\end{eqnarray}
%with $\xi$ now given by eq.\eqref{gapddn}.
%It is worth noticing that the discontinuity in the parameter $\xi$ directly reflects itself in a discontinuity in the vacuum energy, as is clear from the plot of Fig.\ref{BS-d4}.
\begin{figure}[h!]
\center
%%\framebox[79mm]{\rule[-26mm]{0mm}{52mm}}
\includegraphics[width=8cm]{vacuum_energy-adjoint.png}
\caption{Plot of the vacuum energy in the adjoint representation as a function of the parameter $b$. The discontinuity at $b=\frac{1}{2}$ is evident. }
\label{BS-d4}
\end{figure}

Investigating the functional \eqref{ddvac4} in terms of $\xi$ and $\xi'$, it is numerically (graphically) rapidly established there is always a solution to the gap equations $\frac{\p \mathcal{E}_v }{\p \xi}=\frac{\p \mathcal{E}_v }{\p \xi'}=0$ for $b<\frac{1}{2}$, but the solution $\xi^*$ is pushed towards the boundary $\xi=0$ if $b$ approaches $\frac{1}{2}$, to subsequently disappear for $b>\frac{1}{2}$ \footnote{The gap solutions correspond to a local maximum, as identified by analysing the Hessian matrix of 2nd derivatives.}. In that case, we are forced to return on our steps as in the fundamental case and conclude that $\beta=0$, leaving us with a single variable $\xi=\xi'$ and a new vacuum functional to extremize. There is, a priori, no reason for these 2 intrinsically different vacuum functionals be smoothly joined at $b=\frac{1}{2}$. This situation is clearly different from what happens when a potential has e.g.~2 different local minima with different energy, where at a first order transition the two minima both become global minima, thereafter changing their role of local vs.~global. Evidently, the vacuum energy does not jump since it is by definition equal at the transition.

Nevertheless, a completely analogous analysis as for the fundamental case will learn that $b=\frac{1}{2}$ is beyond the range of validity of our approximation\footnote{A little more care is needed as the appearance of two Gribov scales complicate the log structure. However, for small $b$ the Gribov masses will dominate over the Higgs condensate and we can take  $\omu$ of the order of the Gribov masses to control the logs and get a small coupling. For large $b$, we have $\beta^*=0$ and a small $\omega^*$: the first log will be kept small by its pre-factor and the other logs can be managed by taking $\omu$ of the order of the Higgs condensate.}. The small and large $b$ results can again be shown to be valid, so at large $b$ ($\sim$ large Higgs condensate) we have a mixture of off-diagonal Yukawa and confined diagonal modes and at small $b$ ($\sim$ small Higgs condensate) we are in a confined phase. In any case we have that the diagonal gauge boson is \emph{not} Coulomb-like, its infrared behaviour is suppressed as it feels the presence of the Gribov horizon.






























%-------------------------------------------------------------------------------------------
\section{$SU(2)\times U(1)+$Higgs field in the fundamental representation}
\label{The Electroweak theory}
%-------------------------------------------------------------------------------------------



From now on in this work only the fundamental case of the Higgs field will be treated, for reasons relying on the physical relevance of the fundamental representation of this field. As a first step, we are going to present, as in the previous sections, general results for $d$-dimension. Afterwards, the $3$ and $4$-dimensional cases will be considered in the subsections \ref{d=3} and \ref{d=4}. The starting action of the $SU(2) \times U(1)+$Higgs field reads
\begin{eqnarray}
S ~=~ \int \d^{d}x  \;  \bigg(\frac{1}{4}  F_{\mu \nu }^{a} F_{\mu \nu }^{a}  +  \frac{1}{4} B_{\mu\nu} B_{\mu\nu} +{\bar c}^a \partial_\mu D^{ab}_\mu c^b - \frac{(\partial_\mu A^a_\mu)^{2}}{2\xi} 
 + {\bar c}\partial^2 c  - \frac{(\partial_\mu B_\mu)^{2}}{2\xi}  +
\nonumber \\
+
(D_{\mu }^{ij}\Phi^{j})^{\dagger}( D_{\mu }^{ik}\Phi^{k})+\frac{\lambda }{2}\left(\Phi^{\dagger}\Phi - \nu^{2}\right)^{2}   \bigg)  \;,
\label{Sf}
\end{eqnarray}
where the covariant derivative is defined by
\begin{equation}
D_{\mu }^{ij}\Phi^{j} =\partial _{\mu }\Phi^{i} - \frac{ig'}{2}B_{\mu}\Phi^{i} -   ig \frac{(\tau^a)^{ij}}{2}A_{\mu }^{a}\Phi^{j}  \;.
\end{equation}
and the vacuum expectation value (\textit{vev}) of the Higgs field is $\langle \Phi^{i} \rangle ~=~ \nu\delta^{2i}$.
%%\begin{equation}
%\langle \Phi \rangle  = \left( \begin{array}{ccc}
%                                          0  \\
%                                          \nu
%                                          \end{array} \right)  \;.
%\label{vevf}
%%\end{equation}
The indices $i,j=1,2$ refer to the fundamental representation of $SU(2)$ and $\tau^a, a=1,2,3$, are the Pauli matrices. The coupling constants $g$ and $g'$ refer to the groups $SU(2)$ and $U(1)$, respectively. The field strengths $F^a_{\mu\nu}$ and $B_{\mu\nu}$ are given by
\begin{equation}
F^a_{\mu\nu} = \partial_\mu A^a_\nu -\partial_\nu A^a_\mu + g \varepsilon^{abc} A^b_\mu A^c_\nu \;, \qquad B_{\mu \nu} = \partial_\mu B_\nu -\partial_\nu B_\mu  \;.
\label{fs}
\end{equation}

In order to obtain the boson propagators only quadratic terms of the starting action are required and, due to the new covariant derivative, this quadratic action is not diagonal any more. To diagonalize this action one could introduce a set of new fields, related to the standard ones by
%To obtain the  gauge boson propagators, we consider the quadratic part of the action (\ref{Sf}), given by
%\begin{multline}
%S_{quad} = \int\!\! \d^{d}x \; \frac{1}{2}A_{\mu}^{\alpha}\left[ \left(-\partial_{\mu}\partial_{\mu} + \frac{\nu^{2}}{2}g^{2}\right)\delta_{\mu\nu} + \partial_{\mu}\partial_{\nu} \right]A_{\nu}^{\alpha} + 
%\int\!\! \d^{d}x\, \frac{1}{2}B_{\mu}\left[ \left(-\partial_{\mu}\partial_{\mu} + \frac{\nu^{2}}{2}g'^{2}\right)\delta_{\mu\nu} + \partial_{\mu}\partial_{\nu}\right]B_{\nu} \\
%+ \int\!\! \d^{d}x\, \frac{1}{2}A_{\mu}^{3}\left[ \left(-\partial_{\mu}\partial_{\mu} + \frac{\nu^{2}}{2}g^{2}\right)\delta_{\mu\nu} + \partial_{\mu}\partial_{\nu}\right]A_{\nu}^{3} - \frac{1}{4}
%\int\!\! \d^{d}x\, \nu^{2}g\,g'\,A^{3}_{\mu}B_{\mu} - \frac{1}{4} 
%\int\!\! \d^{d}x\, \nu^{2}g\,g'\,B_{\mu}A^{3}_{\mu} \;.
%\label{Squad}
%\end{multline}
%In order to diagonalize expression \eqref{Squad} we introduce the following fields
\begin{subequations} \begin{gather}
W^+_\mu = \frac{1}{\sqrt{2}} \left( A^1_\mu + iA^2_\mu \right) \;, \qquad W^-_\mu = \frac{1}{\sqrt{2}} \left( A^1_\mu - iA^2_\mu \right)  \;,
\label{ws} \\
Z_\mu =\frac{1}{\sqrt{g^2+g'^2} } \left(  -g A^3_\mu + g' B_\mu \right) \qquad \text{and}\qquad A_\mu =\frac{1}{\sqrt{g^2+g'^2} } \left(  g' A^3_\mu + gB_\mu \right) \;.
\label{za}
\end{gather} 
\end{subequations}
The inverse relation can be easily obtained.
%Let us also give, for further use, the inverse combinations:
%\begin{subequations} \begin{gather}
%A^1_\mu = \frac{1}{\sqrt{2}} \left( W^+_\mu + W^-_\mu \right) \;, \qquad A^2_\mu = \frac{1}{i\sqrt{2}} \left( W^+_\mu - W^-_\mu \right) \;,
%\label{iw} \\
%B_\mu =\frac{1}{\sqrt{g^2+g'^2} } \left(  g A_\mu + g' Z_\mu \right) \qquad \text{and}\qquad A^3_\mu =\frac{1}{\sqrt{g^2+g'^2} } \left(  g' A_\mu - gZ_\mu \right) \;.
%\label{iza}
%\end{gather} \end{subequations}
With this new set of fields the quadratic part of the action reads,
\begin{eqnarray}
S_{quad} &=&  \int d^3 x   \left( \frac{1}{2} (\partial_\mu W^+_\nu - \partial_\nu W^+_\mu)(\partial_\mu W^-_\nu - \partial_\nu W^-_\mu)  + \frac{g^2\nu^2}{2}W^+_\mu W^-_\mu   \right) 
\nonumber \\
&+& \int d^3x  \left(  \frac{1}{4} (\partial_\mu Z_\nu - \partial_\nu Z_\mu)^2  + \frac{(g^2+g'^2)\nu^2}{4}Z_\mu Z _\mu  +    \frac{1}{4} (\partial_\mu A_\nu - \partial_\nu A_\mu)^2  \right)  \;,
\label{qd}
\end{eqnarray}
from which we can read off the masses of the fields $W^+$, $W^-$, and $Z$:
\begin{equation}
m^2_W = \frac{g^2\nu^2}{2} \;, \qquad m^2_Z =  \frac{(g^2+g'^2)\nu^2}{2}  \;. \label{ms}
\end{equation}

The restriction to the Gribov region $\Omega$ still is needed and the procedure here becomes quite similar to what was carried out in the section \ref{Adjrep}. Due to the breaking of the global gauge invariance, caused by the Higgs field (through the covariant derivatives), the ghost sector can be split up in two different sectors, diagonal and off-diagonal. Namely, the ghost propagator reads,



\begin{equation}
\mathcal{G}^{ab}(k;A) ~=~ \left(
  \begin{array}{ll}
   \delta^{\alpha \beta} \mathcal{G}_{off}(k;A) & \,\,\,\,\,\,\,\,0 \\
   \,\,\,\;\;\;\;\;\;0 & \mathcal{G}_{diag}(k;A)
  \end{array}
\right).
\label{gh prop offdiag}
\end{equation}
By expliciting the ghost form factor we have
\begin{eqnarray}
\mathcal{G}_{off}(k;A) 
~\simeq~  \frac{1}{k^{2}} \left( \frac{1}{1 - \sigma_{off}(k;A)} \right) \;,
\label{gh off}
\end{eqnarray}
and
\begin{eqnarray}
\mathcal{G}_{diag}(k;A) 
~\simeq ~ \frac{1}{k^{2}} \left( \frac{1}{1 - \sigma_{diag}(k;A)} \right)\;,
\label{gh diag}
\end{eqnarray}
where
\begin{subequations} \begin{equation}
\sigma_{off}(0;A) ~=~ \frac{g^{2}}{dV} \int\!\! \frac{\d^{d}p}{(2\pi)^{d}} \;  \frac{1}{p^{2}} \left( \frac{1}{2} A^{\alpha}_{\mu}(p)A^{\alpha}_{\mu}(-p) + A^{3}_{\mu}(p)A^{3}_{\mu}(-p)\right)  \;,
\label{sigma off}
\end{equation}
and
\begin{equation}
\sigma_{diag}(0;A) ~=~ \frac{g^{2}}{dV} \int\!\! \frac{\d^{d}p}{(2\pi)^{d}} \; \frac{1}{p^{2}} A^{\alpha}_{\mu}(p)A^{\alpha}_{\mu}(-p)\;.
\label{sigma diag}
\end{equation} \end{subequations}
In order to obtain expressions  \eqref{sigma off} and \eqref{sigma diag}, where $V$ denotes the (infinte) space-time volume, the transversality of the gluon field and the property that $\sigma(k;A)_{off}$ and $\sigma(k;A)_{diga}$ are decreasing functions of $k$ were used\footnote{For more details concerning the ghost computation see \cite{Capri:2013oja,Capri:2013gha,Capri:2012ah,Vandersickel:2012tz}}. From equations \eqref{gh off} and \eqref{sigma off} one can easily read off the two no-pole conditions. Namely,
\begin{subequations} \begin{equation}
\sigma_{off}(0;A) < 1  \;,
\label{sigmaoffnopole}
\end{equation}
and
\begin{equation}
\sigma_{diag}(0;A) < 1\;.
\label{sigmadiagnopole}
\end{equation} \end{subequations}

At the end, the partition function restricted to the first Gribov region $\Omega$ reads,
\begin{eqnarray}
Z &=&  \int \frac{\d \omega}{2\pi i \omega}\frac{\d \beta}{2\pi i \beta} [\d A] [\d B]  \; \; e^{\omega (1-\sigma_{off})} \, e^{\beta (1-\sigma_{diag})} e^{-S}\;.
\label{ptionfucnt22}
\end{eqnarray}














%--------------------------------------------------------
\subsection{The gluon propagator and the gap equation}
%--------------------------------------------------------



The perturbative computation at the semi-classical level requires only quadratic terms of the full action, defined in eq.\eqref{ptionfucnt22} (with $S$ given by eq.\eqref{Sf}), yielding a Gaussian integral over the fields. Inserting external fields to obtain the boson propagators, one gets, after taking the limit $\xi \to 0$, the following propagators,

%As explained in section \ref{introductiontogribov}, the implementation of the restriction to the first Gribov region $\Omega$ in the functional integral concerns the application of the no-pole conditions \eqref{sigmaoffnopole} and \eqref{sigmadiagnopole}, which is encoded into Heaviside step functions. By making use of its integral representation,
%\begin{equation}
%\theta(x) = \int_{-i\infty +\epsilon}^{i\infty +\epsilon} \frac{d\omega}{2\pi i \omega} \; e^{\omega x}   \;, 
%\label{rp}
%\end{equation}
%we could get for the functional integral,

%By computing up to one-loop order in perturbation theory, we have
%\begin{eqnarray}
%Z_{quad} &=& \mathcal{N'} \int \frac{\d \omega}{2\pi i \omega}\frac{\d \beta}{2\pi i \beta} [\d A^{\alpha}] [\d A^{3}] [\d B] \; e^{\omega}\, e^{\beta} \exp \left\{-\frac{1}{2} \int\!\! \frac{\d^{d}p}{(2\pi)^{d}}\, A^{\alpha}_{\mu}(p) \left[ \left(p^{2} + \frac{\nu^{2}}{2}g^{2} + \right. \right. \right. 
%\nonumber \\
% &{}& \left. \left. + \frac{2g^{2}}{dV}\left(\beta +\frac{\omega}{2}\right)\frac{1}{p^{2}}\right) \left(\delta_{\mu\nu} - \frac{p_{\mu}p_{\nu}}{p^{2}}\right)\right]A^{\alpha}_{\nu}(-p) + A^{3}_{\mu}(p) \left[ \left(p^{2} + \frac{\nu^{2}}{2}g^{2} + \frac{2g^{2}}{dV}\frac{\omega}{p^{2}}\right) \times \right.
%\nonumber \\
%&{}& \left. \times \left(\delta_{\mu\nu} - \frac{p_{\mu}p_{\nu}}{p^{2}}\right) \right]A^{3}_{\nu}(-p)  + B_{\mu}(p) \left[ \left(p^{2} + \frac{\nu^{2}}{2}g'^{2}\right)\left(\delta_{\mu\nu} - \frac{p_{\mu}p_{\nu}}{p^{2}}\right) \right] B_{\nu}(-p) - 
%\nonumber \\
%&{}& \left. - A^{3}_{\mu}(p) \left[\nu^{2}g\,g'\left(\delta_{\mu\nu} - \frac{p_{\mu}p_{\nu}}{p^{2}}\right) \right]B_{\nu}(-p) \right\}\;.
%\label{Zquad1}
%\end{eqnarray}
%It turns out to be convenient to perform the following change of variables
%\begin{subequations} \begin{equation}
%\beta \to \beta -\frac{\omega}{2}
%\end{equation}
%and
%\begin{equation}
%\omega \to \omega\;.
%\end{equation} \end{subequations}
%Therefore, equation \eqref{Zquad1} can be rewritten as,
%\begin{eqnarray}
%Z_{quad} &=& \mathcal{N'} \int \frac{\d \omega}{2\pi i}\frac{\d \beta}{2\pi i} [\d A^{\alpha}] [\d A^{3}] [\d B] \;\delta (\partial A^\alpha) \; \delta (\partial A^3)\; \delta({\partial B}) \;\; e^{-\ln\left(\beta\omega-\frac{\omega^{2}}{2}\right)} e^{\beta}\,e^{\frac{\omega}{2}}\;  \times 
%\nonumber \\
%&{ }& \times \exp \left[-\frac{1}{2} \int \frac{\d^{d}p}{(2\pi)^{d}} A^{\alpha}_{\mu}(p)\, Q^{\alpha \beta}_{\mu \nu}\, A^{\beta}_{\nu}(-p)\right] \times   \exp \left[-\frac{1}{2} \int \frac{\d^{d}p}{(2\pi)^{d}} \left(
% \begin{array}{cc}
%  A^{3}_{\mu}(p) & B_{\mu}(p)
% \end{array} \right)
%\mathcal{P}_{\mu \nu} \left(
%\begin{array}{ll}
%A^{3}_{\nu}(-p) \\
%B_{\nu}(-p)
%\end{array} \right) \right]  \nonumber \\
%\label{Zquad2}
%\end{eqnarray}
%where
%\begin{subequations} \begin{equation}
%Q^{\alpha \beta}_{\mu \nu} ~=~ \left[ p^{2} + \frac{\nu^{2} g^{2}}{2} + \frac{2g^{2}\beta}{dV} \frac{1}{p^{2}} \right] \delta^{\alpha \beta} \left( \delta_{\mu \nu} - \frac{p_{\mu} p_{\nu}}{p^{2}}\right) \label{Qmunu}
%\end{equation}
%and
%\begin{equation}
%\mathcal{P}_{\mu \nu} ~=~ \left(
%                        \begin{array}{cc}
%                         p^{2} + \frac{\nu^{2}}{2}g^{2} + \frac{2 g^{2} \omega }{dV} \frac{1}{p^{2}} & - \frac{\nu^{2}}{2}g\,g' \\
%                         - \frac{\nu^{2}}{2}g\,g' & p^{2} + \frac{\nu^{2}}{2}{g'}^2
%                        \end{array} \right) \left( \delta_{\mu \nu} - \frac{p_{\mu} p_{\nu}}{p^{2}} \right)\;.
%\label{P mu nu}
%\end{equation} \end{subequations}
%From expression \eqref{Zquad2} we can easily deduce the two-point correlation functions of the fields $A^\alpha_\mu$, $A^3_\mu$ and $B_\mu$, namely
\begin{subequations} \label{propsaandb} \begin{gather}
\langle  A^{\alpha}_\mu(p) A^{\beta}_\nu(-p) \rangle ~=~ \frac{p^2}{p^4 + \frac{\nu^2g^2}{2} p^2 + \frac{2g^2\beta}{dV}} \; \delta^{\alpha \beta} \left( \delta_{\mu\nu} - \frac{p_\mu p_\nu}{p^2} \right)  \;,   
\label{aalpha} \\
\langle  A^{3}_\mu(p) A^{3}_\nu(-p) \rangle ~=~ \frac{p^2 \left(p^2 +\frac{\nu^2}{2} g'^{2}\right)}{p^6 + \frac{\nu^2}{2} p^4 \left(g^2 +g'^2 \right)  + \frac{2\omega g^2}{dV} \left( p^2 + \frac{\nu^2 g'^2}{2} \right)} \;  \left( \delta_{\mu\nu} - \frac{p_\mu p_\nu}{p^2} \right)  \;,   \label{a3a3}
\\ %
\langle  B_\mu(p) B_\nu(-p) \rangle ~=~ \frac{ \left(p^4 +\frac{\nu^2}{2} g^{2} p^2+\frac{2\omega g^2}{dV}  \right)}{p^6 + \frac{\nu^2}{2} p^4 \left(g^2 +g'^2 \right)  + \frac{2\omega g^2}{dV} \left( p^2 +  \frac{\nu^2 g'^2}{2} \right)} \;  \left( \delta_{\mu\nu} - \frac{p_\mu p_\nu}{p^2} \right)  \;,   \label{bb}
\\
\langle  A^3_\mu(p) B_\nu(-p) \rangle ~=~  \frac{ \frac{\nu^2}{2} g g'  p^2}{p^6 + \frac{\nu^2}{2} p^4 \left(g^2 +g'^2 \right)  + \frac{2\omega g^2}{dV} \left( p^2 + \frac{\nu^2 g'^2}{2} \right)} \;  \left( \delta_{\mu\nu} - \frac{p_\mu p_\nu}{p^2} \right)  \;.   \label{ba3}
\end{gather} \end{subequations}
Moving to the fields $W^{+}_\mu, W^{-}_\mu, Z_\mu, A_\mu$, one obtains 
\begin{subequations} \label{propszandgamma} \begin{gather}
\langle  W^{+}_\mu(p) W^{-}_\nu(-p) \rangle ~=~ \frac{p^2}{p^4 + \frac{\nu^2g^2}{2} p^2 + \frac{2g^2\beta}{dV} } \;  \left( \delta_{\mu\nu} - \frac{p_\mu p_\nu}{p^2} \right)  \;,   \label{ww}
\\
\langle  Z_\mu(p) Z_\nu(-p) \rangle ~=~ \frac{\left( p^4 +\frac{2\omega}{dV} \frac{g^2 g'^2}{g^2+g'^2}  \right)}{p^6 + \frac{\nu^2}{2} p^4 \left(g^2 +g'^2 \right)  + \frac{2\omega g^2}{dV} \left( p^2 + \frac{\nu^2 g'^2}{2} \right) } \;  \left( \delta_{\mu\nu} - \frac{p_\mu p_\nu}{p^2} \right)  \;,   \label{zz}
\\
\langle  A_\mu(p) A_\nu(-p) \rangle ~=~ \frac{\left( p^4 +\frac{\nu^2}{2} p^2 (g^2+g'^2) +\frac{2\omega}{dV} \frac{g^4}{g^2+g'^2}\right)}{p^6 + \frac{\nu^2}{2} p^4 \left(g^2 +g'^2 \right)  + \frac{2\omega g^2}{dV} \left( p^2 + \frac{\nu^2 g'^2}{2} \right) } \;  \left( \delta_{\mu\nu} - \frac{p_\mu p_\nu}{p^2} \right)  \;,   \label{aa}
\\
\langle  A_\mu(p) Z_\nu(-p) \rangle ~=~ \frac{\frac{2\omega}{dV} \frac{g^3 g'}{g^2+g'^2} }{p^6 + \frac{\nu^2}{2} p^4 \left(g^2 +g'^2 \right)  + \frac{2\omega g^2}{dV} \left( p^2 + \frac{\nu^2 g'^2}{2} \right)} \;  \left( \delta_{\mu\nu} - \frac{p_\mu p_\nu}{p^2} \right)  \;.   \label{az}
\end{gather} \end{subequations}
As expected, all propagators get deeply modified in the IR by the presence of the Gribov parameters $\beta$ and $\omega$. Notice, in particular, that due to the parameter $\omega$ a mixing between the fields $A_\mu$ and $Z_\mu$ arises, eq.\eqref{az}. As such, the original photon and the boson $Z$ loose their distinct particle interpretation.  Moreover, it is straightforward to check that in the limit $\beta \rightarrow 0$ and $\omega \rightarrow 0$, the standards propagators are recovered.
%, {\it i.e.}
%\begin{subequations} \begin{gather}
%\langle  W^{+}_\mu(p) W^{-}_\nu(-p) \rangle {\big |}_{\beta=0} ~=~ \frac{1}{p^2 + \frac{\nu^2g^2}{2} } \;  \left( \delta_{\mu\nu} - \frac{p_\mu p_\nu}{p^2} \right)  \;,   \label{ww0}
%\\
%\langle  Z_\mu(p) Z_\nu(-p) \rangle  {\big |}_{\omega=0} ~=~ \frac{1}{p^2 + \frac{\nu^2}{2} \left(g^2 +g'^2 \right)} \;  \left( \delta_{\mu\nu} - \frac{p_\mu p_\nu}{p^2} \right)  \;,   \label{zz0}
%\\
%\langle  A_\mu(p) A_\nu(-p) \rangle{\big |}_{\omega=0}  ~=~ \frac{1}{p^2}   \;  \left( \delta_{\mu\nu} - \frac{p_\mu p_\nu}{p^2} \right)  \;,   \label{aa0}
%\\
%\langle  A_\mu(p) Z_\nu(-p) \rangle {\big |}_{\omega=0} = 0    \;.   \label{az0}
%\end{gather} 
%\end{subequations}

Let us now proceed by deriving the gap equations which will enable us to (dynamically) fix the Gribov parameters, $\beta$ and $\omega$, as function of $g$, $g'$ and $\nu^2$. Thus, performing the path integral of eq.\eqref{ptionfucnt22}, in the semi-classical level, we get
%\begin{equation}
%Z_{quad} ~=~ \mathcal{N'} \int \frac{\d \omega}{2\pi i}\frac{\d \beta}{2\pi i}\,e^{-\ln\left(\beta\omega -\frac{\omega^{2}}{2}\right)} e^{\frac{\omega}{2}}\,e^{\beta} \left[\det Q_{\mu \nu}^{\alpha \beta} \right]^{-1/2}\, \left[ \det \mathcal{P}_{\mu \nu} \right]^{-1/2},  \label{Zquad3}
%\end{equation}
%with
%\begin{subequations} \begin{equation}
%\left[\det Q_{\mu \nu}^{\alpha \beta} \right]^{-1/2} ~=~ \exp \left[ -\frac{2(d-1)}{2} \int \frac{\d^{d}p}{(2\pi)^{d}}  \;  \log \left(p^{2} + \frac{g^{2}\nu^{2}}{2} + \frac{2g^{2}\beta}{dV} \frac{1}{p^{2}}\right) \right] \label{calc_det1}
%\end{equation}
%and
%\begin{equation}
%\left[ \det \mathcal{P}_{\mu \nu} \right]^{-1/2} ~=~ \exp \left[ - \frac{(d-1)}{2} \int \frac{\d^{d}p}{(2\pi)^{d}} \log \lambda_{+}(p, \omega)\, \lambda_{-}(p, \omega) \right].   \label{calc_det2}
%\end{equation} \end{subequations}
%where $\lambda_{\pm}$ are the eigenvalues of the $2\times 2$ matrix of eq.\eqref{P mu nu}, {\it i.e.}
%\begin{equation}
%\lambda_{\pm} ~=~ \frac{\left( p^{4} + \frac{\nu^{2}}{4} p^{2}(g^{2} + g'^{2}) + \frac{g^{2}\omega}{dV} \right) \pm \sqrt{\left[ \frac{\nu^{2}}{4}(g^{2} + g'^{2})p^{2} + \frac{g^{2}\omega}{dV}\right]^{2} - \frac{\omega}{3}\nu^{2}g^{2}\,g'^{2}p^{2}}}{p^{2}}  \;. \label{ev}
%\end{equation}
%Thus,
%\begin{equation}
%Z_{quad} = \mathcal{N} \int \frac{\d \omega}{2\pi i}\frac{\d \beta}{2\pi i} e^{f(\omega, \beta)} \;, \label{Zf eq}
%\end{equation}
%where
\begin{eqnarray}
f(\omega, \beta) ~=~ \frac{\omega}{2} + \beta - \frac{2(d-1)}{2} \int \frac{\d^{d}p}{(2\pi)^{d}} \; \log \left(p^{2} + \frac{\nu^{2}}{2}g^{2} + \frac{2g^{2}\beta}{dV} \frac{1}{p^{2}}\right) -
\nonumber \\
- \frac{(d-1)}{2} \int \frac{\d^{d}p}{(2\pi)^{d}} \; \log \lambda_{+}(p, \omega)\, \lambda_{-}(p, \omega)\;. 
\label{f eq}
\end{eqnarray}
In eq.\eqref{f eq}, $f(\omega,\beta)$ is defined according to eq.\eqref{Zq3} and
\begin{equation}
\lambda_{\pm} ~=~ \frac{\left( p^{4} + \frac{\nu^{2}}{4} p^{2}(g^{2} + g'^{2}) + \frac{g^{2}\omega}{dV} \right) \pm \sqrt{\left[ \frac{\nu^{2}}{4}(g^{2} + g'^{2})p^{2} + \frac{g^{2}\omega}{dV}\right]^{2} - \frac{\omega}{3}\nu^{2}g^{2}\,g'^{2}p^{2}}}{p^{2}}  \;. 
\label{ev1}
\end{equation}
Making use of the thermodynamic limit, where the saddle point approximation takes place, we have the two gap equations given by\footnote{For more details see \cite{Capri:2013oja,Capri:2013gha,Capri:2012ah}.}

%Following Gribov's framework \cite{Gribov:1977wm,Sobreiro:2005ec,Vandersickel:2012tz}, expression \eqref{Zf eq} is evaluated in a saddle point approximation, {\it i.e.}
%\begin{equation}
%Z_{quad} \simeq e^{f(\omega^{\ast}, \beta^{\ast})}\;,
%\label{Zf eq1}
%\end{equation}
%where $(\beta^*, \omega^*)$ are determined by the stationarity conditions
%\begin{equation}
%\frac{\partial f(\omega, \beta)}{\partial \beta}\bigg{|}_{\beta^{\ast}, \omega^{\ast}} ~=~ \frac{\partial f(\omega, \beta)}{\partial \omega}\bigg{|}_{\beta^{\ast}, \omega^{\ast}} ~=~ 0 \;, \nonumber
%\label{saddle condition}
%\end{equation}
%from which we get the two gap equations: the first one, from the $\omega$ derivative,
%\begin{subequations} \begin{multline}
%\frac{2(d-1)}{2d} g^{2} \int \frac{\d^{d}p}{(2\pi)^{d}}\; \frac{1}{\left [ p^{4} + \frac{\nu^{2}}{4}(g^{2}+g'^{2})p^{2} + \frac{g^{2}\omega^{\ast}}{dV}\right]^{2} - \left[ \frac{\nu^{2}}{4}(g^{2}+g'^{2})p^{2} + \frac{g^{2}\omega^{\ast}}{dV} \right]^{2} + \frac{\nu^{2}g^{2}g'^{2}\omega^{\ast}}{dV} p^{2}} \times 
%\\
%\left\{ \left[ 1+ \frac{\frac{\nu^{2}}{4}(g^{2}+g'^{2})p^{2} + \frac{g^{2}\omega^{\ast}}{dV}  - \frac{\nu^{2}}{2}g'^{2}p^{2}}{\sqrt{\left[ \frac{\nu^{2}}{4}(g^{2}+g'^{2})p^{2} + \frac{g^{2}\omega^{\ast}}{dV} \right]^{2} - \frac{\nu^{2}g^{2}g'^{2}\omega^{\ast}}{dV}p^{2}}}\right] \right. \times 
%\\
%\left[ p^{4} + \frac{\nu^{2}}{4}(g^{2}+g'^{2})p^{2} +  \frac{g^{2}\omega^{\ast}}{dV} - \sqrt{\left[ \frac{\nu^{2}}{4}(g^{2}+g'^{2})p^{2} + \frac{g^{2}\omega^{\ast}}{dV}  \right]^{2} - \frac{\nu^{2}g^{2}g'^{2}\omega^{\ast}}{dV}  p^{2}}\right] + 
%\\
%\left[ 1-\frac{\frac{\nu^{2}}{4}(g^{2}+g'^{2})p^{2} + \frac{g^{4}\omega^{\ast}}{3}  -  \frac{\nu^{2}}{2}g'^{2}p^{2}}{\sqrt{\left[ \frac{\nu^{2}}{4}(g^{2}+g'^{2})p^{2} + \frac{g^{2}\omega^{\ast}}{dV}  \right]^{2} - \frac{\nu^{2}g^{2}g'^{2}\omega^{\ast}}{dV}  p^{2}}} \right] \times 
%\\
%\left. \left[ p^{4}+\frac{\nu^{2}}{4}(g^{2}+g'^{2})p^{2} + \frac{g^{2}\omega^{\ast}}{dV} + \sqrt{\left[ \frac{\nu^{2}}{4}(g^{2}+g'^{2})p^{2}  +  \frac{g^{2}\omega^{\ast}}{dV}  \right]^{2} - \frac{\nu^{2}g^{2}g'^{2}\omega^{\ast}}{dV} p^{2}} \right] \right\}  ~=~ 1\;,
%\label{omega gap eq1}
%\end{multline}
%and the second one, from the $\beta$ derivative,
\begin{equation}
\frac{4(d-1)}{2d}g^{2} \int \frac{\d^{d}p}{(2\pi)^{d}} \;  \frac{1}{p^{4}+\frac{g^{2}\nu^{2}}{2}p^{2} + \frac{2g^{2}\beta^{\ast}}{dV} } ~=~ 1   \;,
\label{beta gap eq}
\end{equation} 
and
%In particular, after a little algebra, eq.\eqref{omega gap eq1} can be considerably simplified, yielding
\begin{equation}
\frac{2(d-1)}{d}g^{2}\int\!\! \frac{\d^{d}p}{(2\pi)^{d}} \; \frac{p^{2} + \frac{\nu^{2}}{2}g'^{2}}{p^{6} + \frac{\nu^{2}}{2}(g^{2} + g'^{2})p^{4} + \frac{2\omega^{\ast}g^{2}}{dV}p^{2} + \frac{\nu^{2}g^{2}\,g'^{2}\omega^{\ast}}{dV} } ~=~ 1 \;.
\label{omega gap eq}
\end{equation}
%

















%\subsubsection{The limit $g' \to 0$.}
%An important check to be done is the case where $g'=0$, which must recover the results of \cite{Capri:2012ah} with the Higgs field in the fundamental representation, obtaining a $U(1)$ massless gauge field decoupled from the $SU(2)$ gauge sector. This decoupling can be easily seen just by setting $g'=0$ in the propagator expressions \eqref{a3a3} - \eqref{ba3} and \eqref{aalpha}:
%\begin{subequations} \label{propgzro} \begin{eqnarray}
%\langle A^{\alpha}_{\mu}(p)A^{\beta}_{\nu}(-p)\rangle &=& \frac{p^{2}}{p^{4} + \frac{\nu^{2}}{2}g^{2}p^{2} + \frac{\beta}{2}g^{2}} \delta^{\alpha\beta}T_{\mu\nu}(p^2)\;,\quad \langle A^{3}_{\mu}(p)A^{3}_{\nu}(-p)\rangle = \frac{p^{2}}{p^{4}+\frac{\nu^{2}}{2}g^{2}p^{2}+\frac{\omega^{2}}{2}g^{2}}T_{\mu\nu}(p^2)\;, \\
%\langle B_{\mu}(p)B_{\nu}(-p)\rangle &=&  \frac{1}{p^{2}} T_{\mu\nu}(p^2)\;,\quad \langle A^{3}_{\mu}(p)B_{\nu}(-p)\rangle = 0\;.
%\end{eqnarray} \end{subequations}
%Also, as in the last section, one should be able to write the propagators in terms of the fields $W^{\pm}$, $Z$ and $A$ obtaining
%\begin{subequations} \label{ppgzro} \begin{eqnarray}
%\langle W^{+}_{\mu}(p)W^{-}_{\nu}(-p)\rangle &=& \frac{p^{2}}{p^{4} + \frac{\nu^{2}}{2}g^{2}p^{2} + \frac{\beta}{2}g^{2}} \delta^{\alpha\beta}T_{\mu\nu}(p^2))\;, \label{ww1} \quad \langle Z_{\mu}(p)Z_{\nu}(-p)\rangle = \frac{p^{2}}{p^{4}+\frac{\nu^{2}}{2}g^{2}p^{2}+\frac{\omega^{2}}{2}g^{2}} T_{\mu\nu}(p^2)\;, \\
%\langle A_{\mu}(p)A_{\nu}(-p)\rangle &=& \frac{1}{p^{2}} T_{\mu\nu}(p^2)\;, \quad
%\langle A_{\mu}(p)Z_{\nu}(-p)\rangle = 0\;.
%\end{eqnarray} \end{subequations}
%These propagators, \eqref{propgzro} and \eqref{ppgzro}, could also be derived by taking $g'=0$ in the quadratic partition function, or even in the generating functional \eqref{Zquad1}, and following the steps of the last section.





Given the difficulties in solving the gap equations \eqref{beta gap eq} and \eqref{omega gap eq}, we propose an alternative approach to probe the gluon propagators in the parameter space $\nu$, $g$ and $g'$. Instead of explicitly solve the gap equations, let us search for the necessity to implement the Gribov restriction. For that we mean to compute $\langle \sigma_{off}(0) \rangle $ and $\langle \sigma_{diag}(0) \rangle$ with the gauge field propagators unchanged by the Gribov terms, {\it i.e.}, before applying the Gribov restriction. Therefore, if $\langle \sigma_{off}(0;A) \rangle  < 1$ and $\langle \sigma_{diag}(0;A) \rangle < 1$ already in this case (without Gribov restrictions), then we would say that there is no need to restrict the domain of integration to $\Omega$. In that case we have, immediately, $\beta^* = \omega^* =0$ and the standard Higgs procedure takes place. Namely, the expression of each ghost form factor is
\begin{eqnarray}
\langle \sigma_{off}(0) \rangle & = &  \frac{(d-1)g^{2}}{d}  \int\!\!  \frac{\d^d p}{(2\pi)^d} \frac{1}{p^{2}}\left(\frac{1}{p^{2} + \frac{\nu^{2}}{2}g^{2}} + \frac{1}{p^{2} + \frac{\nu^{2}}{2}(g^{2}+g'^{2})} \right)  \;.
\label{sgoff1}
\end{eqnarray}
and
\begin{equation}
\langle \sigma_{diag}(0) \rangle ~=~ \frac{2(d-1)g^{2}}{d} \int\!\!\frac{\d^{d}p}{(2\pi)^{d}}\frac{1}{p^{2}}\left(\frac{1}{p^{2} + \frac{\nu^{2}}{2}g^{2}}\right)  \;.
\label{sgdiag1}
\end{equation}





%\subsubsection{About $\sigma_\text{off}(0)$ and $\sigma_\text{diag}(0)$ without the Gribov parameters}
%\label{Evaluation of the ghost form factors}

%Specifically in the next two sections we propose a different approach to check if there exist values of the Higgs condensate $\nu$ and of the coupling $(g,\;g')$ for which both $\langle \sigma_{off}(0) \rangle $ and $\langle \sigma_{diag}(0) \rangle$  already satisfy the no-pole condition
%\begin{equation}
%\langle \sigma_{off}(0;A) \rangle  < 1  \;, \qquad   \langle \sigma_{diag}(0;A) \rangle < 1\;,
%\label{n-pole}
%\end{equation}
%in which case $\beta^\ast$ and/or $\omega^\ast$ could be immediately set equal to zero. Therefore, we present here the expression of both quantities for the $d$-dimensional case. Thus,

%Analogously, for $\langle \sigma_{diag}(0)\rangle$ one gets


%Now that we have in hands all expressions needed to analyze the gluon propagator, let us specialize them to the $d=3$ and $d=4$ cases. With that, we are able to provide a map of the parameter space displaying regions where we have sign of confinement, regions where the Higgs mechanism takes place unaltered and mixed regions where the propagators have ``confined'' and ``deconfined'' contributions in its expressions.




















\subsection{The $d=3$ case} 
\label{d=3}

In the three-dimensional case things become easier since there is no divergences to treat. Therefore, computing the ghost form factors \eqref{sgoff1} and \eqref{sgdiag1} we led to the following conditions 


%Before discussing the gap equations equations \eqref{beta gap eq} and \eqref{omega gap eq}, it is worthwhile to evaluate the vacuum expectation values of the ghost form factors  $\sigma_{off}(0)$ and $\sigma_{diag}(0)$, eqs. \eqref{sigma off} and \eqref{sigma diag}, without taking into account the restriction to the Gribov region, {\it i.e.}~without the presence of the two Gribov parameters $(\beta^{\ast},\omega^{\ast})$. This will enable us to verify if there exist values of the Higgs condensate $\nu$ and of the couplings $(g,g')$ for which both $\langle \sigma_{off}(0) \rangle $ and $\langle \sigma_{diag}(0) \rangle$  already satisfy the no-pole condition
%\begin{equation}
%\langle \sigma_{off}(0;A) \rangle  < 1  \;, \qquad   \langle \sigma_{diag}(0;A) \rangle < 1\;,
%\label{n-pole}
%\end{equation}
%in which case $\beta^\ast$ and/or $\omega^\ast$ could be immediately set equal to zero.

%Let us start by considering $\langle \sigma_{off}(0)\rangle$. From eqs.\eqref{sigma off} and \eqref{AB gluon prop} we easily obtain
%\begin{eqnarray}
%\langle \sigma_{off}(0) \rangle & = &  \frac{2g^{2}}{3}\int\!\!\frac{d^3 p}{(2\pi)^3} \frac{1}{p^{2}}\left(\frac{1}{p^{2} + \frac{\nu^{2}}{2}g^{2}} + \frac{1}{p^{2} + \frac{\nu^{2}}{2}(g^{2}+g'^{2})} \right)  \nonumber \\
% & =& \frac{g^{2}}{3\pi^{2}}\int_{0}^{\infty} \!\!dp\left(\frac{1}{p^{2} + \frac{\nu^{2}}{2}g^{2}} + \frac{1}{p^{2} + \frac{\nu^{2}}{2}(g^{2}+g'^{2})}\right)\;.
%\label{sgoff1}
%\end{eqnarray}
%Analogously, for $\langle \sigma_{diag}(0)\rangle$ one gets
%\begin{equation}
%\langle \sigma_{diag}(0) \rangle = \frac{4g^{2}}{3}\int\!\!\frac{d^{3}p}{(2\pi)^{3}}\frac{1}{p^{2}}\left(\frac{1}{p^{2} + \frac{\nu^{2}}{2}g^{2}}\right)
%= \frac{2g^{2}}{3\pi^{2}}\int_{0}^{\infty}\!\! \frac{dp}{p^{2} + \frac{\nu^{2}}{2}g^{2}}\;.
%\label{sgdiag1}
%\end{equation}
% As
%\begin{eqnarray}
%\int_{0}^{\infty} \frac{dp}{p^{2} + m^{2}} &=& \frac{\pi}{2m}\;,
%\label{decomp1}
%\end{eqnarray}
%we found for $\langle \sigma_{off}(0)\rangle$ and $\langle \sigma_{diag}(0)\rangle$
%\begin{subequations} \label{rmn1} \begin{eqnarray}
%\langle \sigma_{off}(0)\rangle &=& \frac{g}{3\sqrt{2}\pi\nu}(1 + \cos(\theta_{W})) \;, \\
%\langle \sigma_{diag}(0)\rangle &=& \frac{2g}{3\sqrt{2}\pi\nu}\;,
%\end{eqnarray} \end{subequations} 
%where
%\begin{equation}
%\cos(\theta_{W}) = \frac{g}{\sqrt{g^{2}+g'^{2}}}
%\label{thW}
%\end{equation}
%is the Weinberg angle. For the integration domain of the Yang--Mills field to be the configuration space inside the first Gribov horizon, we need that $\langle \sigma_{off}(0)\rangle$ and $\langle \sigma_{diag}(0)\rangle$  be less than one. Thus,


\begin{subequations} 
\label{conds} 
\begin{eqnarray}
(1+\cos(\theta_{W}))\frac{g}{\nu} &<& 3\sqrt{2}\pi \label{firstcond} \\
2\frac{g}{\nu} &<& 3\sqrt{2}\pi \label{secondcond} \;,
\end{eqnarray} 
\end{subequations}
where $\theta(W)$ stands for the Weinberg angle,
\begin{equation}
\cos(\theta_{W}) = \frac{g}{\sqrt{g^{2}+g'^{2}}}\;.
\label{thW}
\end{equation}
These two conditions make phase space fall apart in three regions, as depicted in \figurename\ \ref{regionsdiag1}.
\begin{itemize}
	\item If $g/\nu<3\pi/\sqrt2$, neither Gribov parameter is necessary to make the integration cut off at the Gribov horizon. In this regime the theory is unmodified from the usual perturbative electroweak theory.
	\item In the intermediate case $3\pi/\sqrt2<g/\nu<3\sqrt2\pi/(1+\cos\theta_W)$ only one of the two Gribov parameters,  $\beta$, is necessary. The off-diagonal ($W$) gauge bosons will see their propagators modified due to the presence of a non-zero $\beta$, while the $Z$ boson and the photon $A$ remain untouched.
	\item In the third phase, when $g/\nu>3\sqrt2\pi/(1+\cos\theta_W)$, both Gribov parameters are needed, and all propagators are influenced by them. The off-diagonal gauge bosons are confined. The behaviour of the diagonal gauge bosons depends on the values of the couplings, and the third phase falls apart into two parts, as detailed in section \ref{sect7}.
\end{itemize}
Note that here in the $3$-dimensional $SU(2)\times U(1)+$Higgs case, as well as in the $3d$ $SU(2)+$Higgs treated in section \ref{3dsu2}, an effective coupling constant becomes of utmost importance when discussing the trustworthiness of the our semi-classical results.

\begin{figure}\begin{center}
\includegraphics[width=.25\textwidth]{fourregions.pdf}
\caption{There appear to be four regions in phase space. The region I is defined by condition \eqref{secondcond} and is characterized by ordinary Yang--Mills--Higgs behaviour (massive $W$ and $Z$ bosons, massless photon). The region II is defined by \eqref{firstcond} while excluding all points of region I --- this region only has electrically neutral excitations, as the $W$ bosons are confined (see Section \ref{sect6}); the massive $Z$ and the massless photon are unmodified from ordinary Yang--Mills--Higgs behaviour. Region III has confined $W$ bosons, while both photon and $Z$ particles are massive due to influence from the Gribov horizon; furthermore there is a negative-norm state. In region IV all $SU(2)$ bosons are confined and only a massive photon is left. Mark that the tip of region III is hard to deal with numerically --- the discontinuity shown in the diagram is probably an artefact due to this difficulty.  Details are collected in Section \ref{sect7}. \label{regionsdiag1}}
\end{center}\end{figure}

















\subsubsection{The off-diagonal ($W$) gauge bosons} 
\label{sect6}
Let us first look at the behaviour of the off-diagonal bosons under the influence of the Gribov horizon. The propagator \eqref{ww}  only contains the $\beta$ Gribov parameter, meaning that $\omega$ need not be considered here.

In the regime $g/\nu<3\pi/\sqrt2$ (region I in \figurename\ \ref{regionsdiag1}) the parameter $\beta$ is not necessary, due to the ghost form factor $\langle\sigma_{diag}(0)\rangle$ always being smaller than one. In this case, the off-diagonal boson propagator is simply of massive type, with mass parameter $\frac{\nu^{2}}{2}g^{2}$.
%\begin{equation}
%  \langle W^{+}_{\mu}(p)W^{-}_{\nu}(-p) \rangle = \frac{1}{p^{2} + \frac{\nu^{2}}{2}g^{2}}\left(\delta_{\mu\nu} - \frac{p_{\mu}p_{\nu}}{p^{2}}\right) \;.
%\end{equation}

In the case that $g/\nu>3\pi/\sqrt2$ (regions II, III, and IV in \figurename\ \ref{regionsdiag1}), the relevant ghost form factor is not automatically smaller than one any more, and the Gribov parameter $\beta$ becomes necessary. The value of $\beta^{\ast}$ is determined from the gap equations \eqref{beta gap eq}. After rewriting the integrand in partial fractions, the integral in the equation becomes of standard type, and we readily find the solution
\begin{equation}
  \beta^{\ast} = \frac{3g^2}{32} \left(\frac{g^2}{2\pi^2}-\nu^2\right)^2 \;.
\end{equation}
Mark that, in order to find this result, we had to take the square of both sides of the equation twice. One can easily verify that, in the region $g/\nu>3\pi/\sqrt2$ which concerns us, no spurious solutions were introduced when doing so.

Replacing this value of $\beta^{\ast}$ in the off-diagonal propagator \eqref{ww} one can immediately check that it
%\begin{multline}
%  \langle W_\mu^{+}(p)W_\nu^{-}(-p)\rangle = \frac{\pi/g^3}{\sqrt{\frac{g^2}{4\pi^2}-\nu^2}} \left(\frac{\frac{g^3}{2\pi}\sqrt{\frac{g^2}{4\pi^2}-\nu^2}-\frac i4\nu^2g^2}{p^2+\frac{\nu^2}4g^2+i\frac{g^3}{2\pi}\sqrt{\frac{g^2}{4\pi^2}-\nu^2}} + \frac{\frac{g^3}{2\pi}\sqrt{\frac{g^2}{4\pi^2}-\nu^2}+\frac i4\nu^2g^2}{p^2+\frac{\nu^2}4g^2-i\frac{g^3}{2\pi}\sqrt{\frac{g^2}{4\pi^2}-\nu^2}}\right) \\ \times \left(\delta_{\mu\nu}-\frac{p_\mu p_\nu}{p^2}\right) \;.
%\end{multline}
clearly displays two complex conjugate poles. As such, the off-diagonal propagator  cannot describe a physical excitation of the physical spectrum, being adequate for a confining phase. This means that the off-diagonal components of the gauge field are confined in the region $g/\nu>3\pi/\sqrt2$.
















\subsubsection{The diagonal $SU(2)$ boson and the photon field} \label{sect7}
The other two gauge bosons --- the $A^3_\mu$ and the $B_\mu$ --- have their propagators given by \eqref{a3a3}, \eqref{bb}, and \eqref{ba3} or equivalently --- the $Z_\mu$ and the $A_\mu$ --- by \eqref{zz}, \eqref{aa} and \eqref{az}. Here, $\omega$ is the only one of the two Gribov parameters present.

In the regime $g/\nu<3\sqrt2\pi/(1+\cos\theta_W)$ (regions I and II) this $\omega$ is not necessary to restrict the region of integration to within the first Gribov horizon. Due to this, the propagators are unmodified in comparison to the perturbative case.
%\begin{subequations} \label{propsrewrite} \begin{gather}
%\langle Z_{\mu}(p) Z_{\nu}(-p) \rangle = \frac{1}{p^{2} + \frac{\nu^{2}}{2}(g^{2} + g'^{2})} \left(\delta_{\mu\nu} - \frac{p_{\mu}p_{\nu}}{p^{2}}\right)\;, \\
%\langle A_{\mu}(p) A_{\nu}(-p) \rangle = \frac{1}{p^{2}} \left(\delta_{\mu\nu} - \frac{p_{\mu}p_{\nu}}{p^{2}}\right) \;.
%\end{gather} \end{subequations}

In the region $g/\nu>3\sqrt2\pi/(1+\cos\theta_W)$ (regions III and IV) the Gribov parameter $\omega$ does become necessary, and it has to be computed by solving its gap equation, eq. \eqref{omega gap eq}. Due to its complexity it seems impossible to do so analytically. Therefore we turn to numerical methods. Using Mathematica the gap equation can be straightforwardly solved for a list of values of the couplings. Then we determine the values where the propagators have poles. 

The denominators of the propagators are a polynomial which is of third order in $p^2$. There are two cases: there is a small region in parameter space where the polynomial has three real roots, and for all other values of the couplings there are one real and two complex conjugate roots. In \figurename\ \ref{regionsdiag1} these zones are labelled III and IV respectively. Let us analyze each region separately.

%Ordinarily, one would like to diagonalize the propagator matrix in order to separate the states present in the theory. In our case, however, doing so requires a nonlocal transformation, and the result will contain square roots containing the momentum of the fields. It seems to be more enlightening to, instead, perform a partial fraction decomposition. If we look at the two-point functions of the $A_3$ and $B$ fields \eqref{a3a3}, \eqref{bb}, and \eqref{ba3}, we can succinctly write those as\footnote{The projector $\delta_{\mu\nu}-\frac{p_\mu p_\nu}{p^2}$ will be ignored in this discussion, as it does not change anything nontrivial here.}
%\begin{equation}
%	\Delta_{ij} = \frac{f_{ij}(p^2)}{P(p^2)} \;.
%\end{equation}
%Here, the indices $i,j$ run over $A_3$ and $B$. The functions $f_{ij}(p^2)$ are polynomials of $p^2$ of at most second order, and the function $P(p^2)$ is a third-order polynomial of $p^2$. Furthermore, if we consider the functions $f_{ij}(p^2)$ to be the elements of $2\times2$ matrix, we can see that the determinant of this matrix is nothing but $p^2P(p^2)$.

%Let us assume that we know what the roots of $P(p^2)$ are, and call them $-m^2_n$ with $n=1,2,3$. It is then obvious that we can rewrite $P(p^2)$ as $(p^2+m_1^2)(p^2+m_2^2)(p^2+m_3^2)$. We can then perform a decomposition in partial fractions. We will have something of the form
%\begin{equation}
%	\frac{f_{ij}(p^2)}{P(p^2)} = \sum_{n=1}^3 \frac{\alpha_{ij,n}}{p^2+m_n^2} \;.
%\end{equation}
%The constants $\alpha_{ij,n}$ can be readily determined the usual way and we get
%\begin{equation}
%	\frac{f_{ij}(-m_1^2)}{(-m_1^2+m_2^2)(-m_1^2+m_3^2)} = \alpha_{ij,1}
%\end{equation}
%and analogously for $n=2,3$. In conclusion we find
%\begin{multline}
%	\Delta_{ij} = \frac{f_{ij}(p^2)}{P(p^2)} = \frac{f_{ij}(-m_1^2)}{(-m_1^2+m_2^2)(-m_1^2+m_3^2)} \frac1{p^2+m_1^2} \\ + \frac{f_{ij}(-m_2^2)}{(m_1^2-m_2^2)(-m_2^2+m_3^2)} \frac1{p^2+m_2^2} + \frac{f_{ij}(-m_3^2)}{(m_1^2-m_3^2)(m_2^2-m_3^2)} \frac1{p^2+m_3^2} \;.
%\end{multline}
%We can again interpret the constants $f_{ij}(-m_n^2)$ as elements of some $2\times2$ matrices, and we find that the determinants of these matrices are equal to $-m_n^2P(-m_n^2) = 0$, as the $-m_n^2$ are roots of the polynomial $P(p^2)$. Now it is obvious that a $2\times2$ matrix $\mx A$ with zero determinant can always be written in the form $\mx A = v v^T$ with $v$ some $2\times1$ matrix. Furthermore, this vector $v$ has norm $v^Tv = \tr\mx A$. This means that we can write our matrices in the form $\mx A = \tr\mx A \hat v \hat v^T$ where $\hat v$ is now the unit vector parallel to $v$. Therefore, let us write $f_{ij}(-m_n^2) = (f_{11}(-m_n^2)+f_{22}(-m_n^2)) \hat v_i^n \hat v_j^n$, resulting in
%\begin{multline} \label{partfracdec}
%	\Delta_{ij} = \frac{f_{11}(-m_1^2)+f_{22}(-m_1^2)}{(-m_1^2+m_2^2)(-m_1^2+m_3^2)} \frac1{p^2+m_1^2} \hat v_i^1\hat v_j^1 \\ + \frac{f_{11}(-m_2^2)+f_{22}(-m_2^2)}{(m_1^2-m_2^2)(-m_2^2+m_3^2)} \frac1{p^2+m_2^2} \hat v_i^2\hat v_j^2 + \frac{f_{11}(-m_3^2)+f_{22}(-m_3^2)}{(m_1^2-m_3^2)(m_2^2-m_3^2)} \frac1{p^2+m_3^2} \hat v_i^3\hat v_j^3 \;.
%\end{multline}
%The vectors $v_i^n$ can be interpreted as linear combinations of the $A_3$ and $B$ fields. Decomposing the two-point functions in this way, we thus find three ``states'' $v_1^n A_3 + v_2^n B$. These states are not orthogonal to each other (which would be impossible for three vectors in two dimensions). The coefficients in front of the Yukawa propagators will be the residues of the poles, and they have to be positive for a pole to correspond to a physical excitation.  The poles can be extracted from the zeros at $p^2_\ast$ of $P(p^2)=p^6+\frac{\nu^2}{2}p^4(g^2+g'^2)+\frac{g^2\omega}{3}(2p^2+\nu^2g'^2)$, viz.
%\begin{subequations} \begin{eqnarray}
% p^2_\ast&=& \frac{1}{6}\left\{ (g^{2}+g'^{2})\nu^{2} + \left[ (g^{2}+g'^{2})^{2}\nu^{4} - 8g^{2}\omega\right]\left[ (g^{2}+g'^{2})^{3}\nu^{6} - 12g^{2}(g^{2}-2g^{2})\nu^{2}\omega +  \right. \right. \nonumber \\
%&{}&+ \left. 2\sqrt{2}\sqrt{ g^{2}\omega \left( 9g'^{2}(g^{2}+g'^{2})^{3}\nu^{8} - 6g^{2}(g^{4}+20g^{2}g'^{2}-8g'^{4})\nu^{4}\omega + 64g^{4}\omega^{2}  \right)} \right]^{-1/3} + \nonumber \\
%&{}&+ \left[ (g^{2}+g'^{2})^{3}\nu^{6} - 12g^{2}(g^{2}-2g^{2})\nu^{2}\omega + 2\sqrt{2}\left( g^{2}\omega \left( 9g'^{2}(g^{2}+g'^{2})^{3}\nu^{8} - \right. \right. \right. \nonumber \\
%&{}&- \left. \left. \left. \left. 6g^{2}(g^{4}+20g^{2}g'^{2}-8g'^{4})\nu^{4}\omega + 64g^{4}\omega^{2}  \right)\right)^{1/2}  \right]^{1/3} \right\}
%\label{m1}
%\end{eqnarray}
%\begin{eqnarray}
% p^2_\ast&=& \frac{1}{6}\left\{  (g^{2}+g'^{2})\nu^{2} - \frac{1}{2} \left[ (g^{2}+g'^{2})^{2}\nu^{4} - 8g^{2}\omega\right]\left[ (g^{2}+g'^{2})^{3}\nu^{6} - 12g^{2}(g^{2}-2g^{2})\nu^{2}\omega +   \right. \right. \nonumber \\
%&{}&+ \left. 2\sqrt{2}\sqrt{ g^{2}\omega \left( 9g'^{2}(g^{2}+g'^{2})^{3}\nu^{8} - 6g^{2}(g^{4}+20g^{2}g'^{2}-8g'^{4})\nu^{4}\omega + 64g^{4}\omega^{2}  \right)} \right]^{-1/3} - \nonumber \\
%&{}&- \frac{1}{2}\left[ (g^{2}+g'^{2})^{3}\nu^{6} - 12g^{2}(g^{2}-2g^{2})\nu^{2}\omega + 2\sqrt{2}\left( g^{2}\omega \left( 9g'^{2}(g^{2}+g'^{2})^{3}\nu^{8} - \right. \right. \right. \nonumber \\
%&{}&- \left. \left. \left. \left. 6g^{2}(g^{4}+20g^{2}g'^{2}-8g'^{4})\nu^{4}\omega + 64g^{4}\omega^{2}  \right)\right)^{1/2}  \right]^{1/3} \right. \nonumber \\
%&{}&+ i\frac{\sqrt{3}}{2}\left[ (g^{2}+g'^{2})^{2}\nu^{4} - 8g^{2}\omega\right]\left[ (g^{2}+g'^{2})^{3}\nu^{6} - 12g^{2}(g^{2}-2g^{2})\nu^{2}\omega +  \right. \nonumber \\
%&{}&+ \left. 2\sqrt{2}\sqrt{ g^{2}\omega \left( 9g'^{2}(g^{2}+g'^{2})^{3}\nu^{8} - 6g^{2}(g^{4}+20g^{2}g'^{2}-8g'^{4})\nu^{4}\omega + 64g^{4}\omega^{2}  \right)} \right]^{-1/3} + \nonumber \\
%&{}&+ i\frac{\sqrt{3}}{2}\left[ (g^{2}+g'^{2})^{3}\nu^{6} - 12g^{2}(g^{2}-2g^{2})\nu^{2}\omega + 2\sqrt{2}\left( g^{2}\omega \left( 9g'^{2}(g^{2}+g'^{2})^{3}\nu^{8} - \right. \right. \right. \nonumber \\
%&{}&- \left. \left. \left. \left. 6g^{2}(g^{4}+20g^{2}g'^{2}-8g'^{4})\nu^{4}\omega + 64g^{4}\omega^{2}  \right)\right)^{1/2}  \right]^{1/3} \right\}\;,
%\label{m2}
%\end{eqnarray}
%and
%\begin{eqnarray}
% p^2_\ast &=& \frac{1}{6}\left\{  (g^{2}+g'^{2})\nu^{2} - \frac{1}{2} \left[ (g^{2}+g'^{2})^{2}\nu^{4} - 8g^{2}\omega\right]\left[ (g^{2}+g'^{2})^{3}\nu^{6} - 12g^{2}(g^{2}-2g^{2})\nu^{2}\omega +   \right. \right. \nonumber \\
%&{}&+ \left. 2\sqrt{2}\sqrt{ g^{2}\omega \left( 9g'^{2}(g^{2}+g'^{2})^{3}\nu^{8} - 6g^{2}(g^{4}+20g^{2}g'^{2}-8g'^{4})\nu^{4}\omega + 64g^{4}\omega^{2}  \right)} \right]^{-1/3} - \nonumber \\
%&{}&- \frac{1}{2}\left[ (g^{2}+g'^{2})^{3}\nu^{6} - 12g^{2}(g^{2}-2g^{2})\nu^{2}\omega + 2\sqrt{2}\left( g^{2}\omega \left( 9g'^{2}(g^{2}+g'^{2})^{3}\nu^{8} - \right. \right. \right. \nonumber \\
%&{}&- \left. \left. \left. \left. 6g^{2}(g^{4}+20g^{2}g'^{2}-8g'^{4})\nu^{4}\omega + 64g^{4}\omega^{2}  \right)\right)^{1/2}  \right]^{1/3} \right. \nonumber \\
%&{}&- i\frac{\sqrt{3}}{2}\left[ (g^{2}+g'^{2})^{2}\nu^{4} - 8g^{2}\omega\right]\left[ (g^{2}+g'^{2})^{3}\nu^{6} - 12g^{2}(g^{2}-2g^{2})\nu^{2}\omega +  \right. \nonumber \\
%&{}&+ \left. 2\sqrt{2}\sqrt{ g^{2}\omega \left( 9g'^{2}(g^{2}+g'^{2})^{3}\nu^{8} - 6g^{2}(g^{4}+20g^{2}g'^{2}-8g'^{4})\nu^{4}\omega + 64g^{4}\omega^{2}  \right)} \right]^{-1/3} + \nonumber \\
%&{}&- i\frac{\sqrt{3}}{2}\left[ (g^{2}+g'^{2})^{3}\nu^{6} - 12g^{2}(g^{2}-2g^{2})\nu^{2}\omega + 2\sqrt{2}\left( g^{2}\omega \left( 9g'^{2}(g^{2}+g'^{2})^{3}\nu^{8} - \right. \right. \right. \nonumber \\
%&{}&- \left. \left. \left. \left. 6g^{2}(g^{4}+20g^{2}g'^{2}-8g'^{4})\nu^{4}\omega + 64g^{4}\omega^{2}  \right)\right)^{1/2}  \right]^{1/3} \right\}\;,
%\label{m3}
%\end{eqnarray} \end{subequations}

%It is obviously not possible to disentangle the three (vector-like) degrees of freedom corresponding to these different masses using only two fields. Nonetheless, it is possible to derive a diagonalization of the $1PI$ propagator matrix. After the saddle point approximation and using the $A_\mu$ and $Z_\mu$ field variables, it is not difficult to see that the tree level $(A_\mu,Z_\mu)$ sector of the action arising from eq.~(\ref{Zquad1}) can be reformulated as
%\begin{multline}\label{dd1}
%  \int d^4p \left(\frac{1}{2}Z_\mu(p)(p^2+g^2\nu^2)Z_\mu(-p)+\frac{1}{2}A_\mu(p)p^2A_\mu(-p)+i\frac{1}{2}\sqrt{\frac{2}{3}\omega_\ast}\left((g'A_\mu(p)-gZ_\mu(p))V_\mu(-p)+(p\leftrightarrow-p)\right)\right. \\ \left.+ \frac{1}{2}V_\mu(p)p^2V_\mu(-p)\right)
%\end{multline}
%while working immediately on-shell, viz.~using $\p_\mu A_\mu=\p_\mu Z_\mu=0$. The equivalence with the original action can be straightforwardly established by integrating over the $V_\mu$ field. Here, we introduced the latter auxiliary field by hand, but it can be shown in general --- at least for the pure Yang--Mills case; for the current Yang--Mills--Higgs generalization this deserved further investigation at a later stage --- that the all-order no pole condition can be brought in local form by introducing a suitable set of boson and fermion auxiliary fields\footnote{These ghost fields are necessary to eliminate the determinant when integrating over the extra fields.}, see e.g.~\cite{Vandersickel:2012tz,Zwanziger:1989mf,Baulieu:2009ha,Capri:2012wx}.

%Having now 3 fields at our disposal with still 3 masses, there is better hope to diagonalize the previous action. First of all, the special limits $g'\to0$ and/or $\omega\to0$ are simply clear at the level of the action (\ref{dd1}). Secondly, the $1PI$ propagator matrix of (\ref{dd1}) only displays a $p^2$-dependence on the diagonal, each time of the form\footnote{In the formulation without the $V_\mu$ field this is not the case. One of the consequences is the appearance of the aforementioned momentum dependent square roots when a diagonalization in terms of the two fields $A_\mu$ and $Z_\mu$ is attempted.} $p^2+\ldots$. As such, the 3 eigenvalues will be of the form $p^2+m_i^2$. Upon using the associated eigenvectors, the action (\ref{dd1}) can then be simply diagonalized to
%\begin{equation}\label{dd2}
%  \int d^4p \left(\frac{1}{2}\lambda_\mu(p)(p^2+m_1^2)\lambda_\mu(-p)+\frac{1}{2}\eta_\mu(p)(p^2+m_2^2)\eta_\mu(-p)+\frac{1}{2}\kappa_\mu(p)(p^2+m_3^2)\kappa_\mu(-p)\right)\,,
%\end{equation}
%where the $\lambda_\mu,\eta_\mu,\kappa_\mu$ are the ``generalized $i$-particles'' of the current model, adopting the language of \cite{Baulieu:2009ha}. They are related to the original fields $A_\mu$, $Z_\mu$ and $V_\mu$ by momentum-independent linear transformations. The quadratic form appearing in \eqref{dd2} displays a standard propagator structure, with the possibility that two of the mass poles can be complex conjugate.






\subsubsection{Three real roots (region III)}



%\begin{figure}\begin{center}
%\includegraphics[width=.5\textwidth]{threems.pdf}
%\caption{The mass-squareds of the massive excitations found in the region where there are three massive poles (region III). \label{threems}}
%\end{center}\end{figure}

Region III is defined by the polynomial in the denominators of \eqref{a3a3}, \eqref{bb}, and \eqref{ba3} having three real roots. This region is sketched in \figurename\ \ref{regionsdiag1}. (Mark that the tip of the region is distorted due to the difficulty in accessing this part numerically.) 

%The square of the masses corresponding to these three roots are plotted in \figurename\ \ref{threems}.

The residues of related to these poles were computed numerically. Only the two of the three roots have positive residue and can correspond to physical states. Those are the one with highest and the one with lowest mass squared. The third of the roots, the one of intermediate value, has negative residue and thus belongs to some negative-norm state, which cannot be physical.

All three states have non-zero mass for non-zero values of the electromagnetic coupling $g'$, with the lightest of the states becoming massless in the limit $g'\to0$. In this limit we recover the behaviour found in this regime in the pure $SU(2)$ case \cite{Capri:2012cr} (the $Z$-boson field having one physical and one negative-norm pole in the propagator) with a massless fermion decoupled from the non-Abelian sector.






\subsubsection{One real root (region IV)}
In the remaining part of the parameter space, there is only one state with real mass-squared.
The two other roots of the polynomial in the denominators of \eqref{a3a3}, \eqref{bb}, and
\eqref{ba3} have non-zero imaginary part and are complex conjugate to each other. In order to
determine whether the pole coming from the real root corresponds to a physical particle
excitation, we computed its residue, which can be read off in the partial fraction
decomposition (the result is plotted in \figurename\ \ref{resrealmass}). It turns out the
residue is always positive, meaning that this excitation has positive norm and can thus be
interpreted as a physical, massive contributions. The poles coming from the complex roots
cannot, of course, correspond to such physical contributions.

%\begin{figure}\begin{center}
%\includegraphics[width=.5\textwidth]{realmass.pdf}
%\caption{The mass-squared of the one physical massive excitation found in region IV. \label{realmass}}
%\end{center}\end{figure}

\begin{figure} \begin{center}
\includegraphics[width=.5\textwidth]{res.pdf}
\caption{The residue of the pole of the photon propagator. It turns out to be positive for all values of the couplings within the region IV. \label{resrealmass}}
\end{center} \end{figure}

%\begin{figure}\begin{center}
%\includegraphics[width=.5\textwidth]{realpartccmass.pdf}\includegraphics[width=.5\textwidth]{impartccmass.pdf}
%\caption{The real (left) and imaginary (right) parts of the mass-squared of the other two, complex conjugate, poles. \label{ccmass}}
%\end{center}\end{figure}

In the limit $g'\to0$ we once more recover the corresponding results already found in the pure $SU(2)$ case \cite{Capri:2012cr} (two complex conjugate poles in the propagator of the non-Abelian boson field) plus a massless photon not influenced by the non-Abelian sector.



We shall emphasise here the complexity of the found ``phase spectrum'' in the $3d$ case. For the most part of the $(g'/\nu, g/\nu)$ plane we found the diagonal component of the bosonic field displaying a mix of physical and non-physical contributions, regarding the regions III and IV. The off-diagonal component was found to have physical meaning only in the region I. The transition between those regions was found to be continuous with respect to the effective perturbative parameter $\sim g/\nu$.


















\subsection{The $d=4$ case}
\label{d=4}


%In this section we define the notation and conventions used throughout the paper. The system we study is the electroweak sector of the $4d$ standard model, thereby generalizing our earlier works \cite{Capri:2012cr,Capri:2012ah,Capri:2013gha}. In the following sections we consider in detail the possible effects of Gribov copies in such a setting, based on the original Gribov no-pole analysis \cite{Gribov:1977wm,Sobreiro:2005ec}. A summary of our results can be found in section 6.












%\subsubsection{The vacuum energy}
%Looking at the above propagators, beside the decoupling of the $U(1)$ gauge field from the $SU(2)$ gauge field, one should note the likeness between the diagonal and off-diagonal propagators, though in general the two Gribov parameters, $\omega^\ast$ and $\beta^\ast$, differ. Therefore, given the important role played by the gap equations, it seems to be worth to analyze the two gap equations \eqref{beta gap eq} and \eqref{omega gap eq} in the limit $g'\to 0$. In this limit, the gap equations \eqref{beta gap eq} and \eqref{omega gap eq} become
%\begin{equation}
%\label{omega gapeq g'zero}
%\frac{3}{2}g^{2} \int\!\!\frac{d^{4} p}{(2\pi)^{4}}\, \frac{1}{p^{4}+\frac{\nu^{2}}{2}g^{2}p^{2}+\frac{\omega^{\ast}}{2}g^{2}} = 1\;, \qquad
%\frac{3}{2}g^{2} \int\!\! \frac{d^{4}p}{(2\pi)^{4}}\, \frac{1}{p^{4}+\frac{\nu^{2}}{2}g^{2}p^{2}+\frac{g^{2}}{2}\beta^{\ast}} =1\;.
%\end{equation}
%It is clear that these two equations are identical. Thus, when $g' \to 0$, there is only one gap equation and, therefore, only one Gribov parameter. Consequently, the diagonal and off-diagonal propagators of \eqref{propgzro} are identical. These results coincide with what was found in \cite{Capri:2012ah} in the case of Higgs field in the fundamental representation.

%Furthermore, from equation \eqref{Zf eq}, we get for the vacuum energy,
%\begin{equation}
%\mathcal{E}_{\omega^{\ast}} = \frac{3}{2}\omega^{\ast} - \frac{9}{2}g^2 \int\!\! \frac{d^{4}p}{(2\pi)^{4}}\log \! \left(p^{4} + p^{2}\frac{\nu^{2}}{2}g^{2} + \frac{\omega^{\ast}}{2}g^{2}\right)\;,
%\end{equation}
%which, again, is in agreement with the expression for the vacuum energy for the case of $SU(2)$ in the fundamental representation, upon redefining  $\frac{\omega^{\ast}}{2} = \frac{\beta^{\ast}}{3}$.

















%\subsubsection{About $\sigma_\text{off}(0)$ and $\sigma_\text{diag}(0)$ without the Gribov parameters} 
%\label{sect3}
%Before trying to solve the gap equations, it seems to be worthwhile to study what happens with $\sigma_\text{off}(0)$ and $\sigma_\text{diag}(0)$ in the absence of the Gribov parameters, which will allow us to search for regions where the Gribov parameters $\omega$ and $\beta$ are unnecessary, which happens whenever $\sigma_\text{off}(0)$ and/or $\sigma_\text{diag}(0)$ are less than one.

%Thus, we have
%\begin{equation}
%\label{sgoff} \langle \sigma_\text{off}(0) \rangle = \frac{3g^{2}}{4}\int\!\!\frac{d^{4}p}{(2\pi)^{4}}\,\frac{1}{p^{2}}\left(\frac{1}{p^{2}+\frac{\nu^{2}}{2}g^{2}}+\frac{1}{p^{2}+\frac{\nu^{2}}{2}(g^{2}+g'^{2})}\right)\,,\qquad
%\langle \sigma_\text{diag}(0) \rangle = \frac{3g^{2}}{2}\int\!\!\frac{d^{4}p}{(2\pi)^{4}}\,\frac{1}{p^{2}}\frac{1}{\left(p^{2}+\frac{\nu^{2}}{2}g^{2}\right)}\;.
%\end{equation}


In the $4$-dimensional case the diagonal and off-diagonal ghost form factors read, using the standard $\MSbar$ renormalization procedure,
\begin{equation}
\label{constas} 
\langle \sigma_\text{off}(0) \rangle = 1 - \frac{3g^{2}}{32\pi^{2}}\ln\frac{2a}{\cos(\theta_{W})}
\,, \qquad
\langle \sigma_\text{diag}(0) \rangle = 1-\frac{3g^{2}}{32\pi^{2}}\ln(2a)\;,
\end{equation}
where
\begin{equation}
\label{consta} a = \frac{\nu^{2}g^{2}}{4\bar{\mu}^{2}\,e^{1-\frac{32 \pi^{2}}{3g^{2}}}}\,, \qquad a' = \frac{\nu^{2}(g^{2}+g'^{2})}{4\bar{\mu}^{2}\,e^{1-\frac{32 \pi^{2}}{3g^{2}}}} = a \frac{g^{2}+g'^{2}}{g^{2}} = \frac{a}{\cos^{2}(\theta_{W})}
\end{equation}
and $\theta_{W}$ stands for the Weinberg angle. With expression \eqref{constas} we are able to identify three possible regions, depicted in \figurename\ \ref{regionsdiag}:
\begin{itemize}
\item Region I, where $\langle \sigma_\text{diag}(0)\rangle < 1$ and $\langle \sigma_\text{off}(0)\rangle < 1$, meaning $2a > 1$. In this case the Gribov parameters are both zero so that we have the massive $W^{\pm}$ and $Z$, and a massless photon. That region can be identified with the ``Higgs phase''.
\item Region II, where $\langle \sigma_\text{diag}(0)\rangle > 1$ and $\langle \sigma_\text{off}(0)\rangle < 1$, or equivalently $\cos \theta_{W} < 2a < 1$. In this region we have $\omega = 0$ while $\beta \neq 0$, leading to a modified $W^{\pm}$ propagator, and a free photon and a massive $Z$ boson.
\item The remaining parts of parameter space, where $\langle \sigma_\text{diag}(0)\rangle > 1$ and $\langle \sigma_\text{off}(0)\rangle > 1$, or $0 < 2a < \cos\theta_{W}$. In this regime we have both $\beta \neq 0$ and $\omega \neq 0$, which modifies the $W^{\pm}$, $Z$ and photon propagators. Furthermore this region will fall apart in two separate regions III and IV due to different behaviour of the propagators (see \figurename\ \ref{regionsdiag}).
\end{itemize}

















\subsubsection{The off-diagonal gauge bosons} \label{sect4}
Let us first look at the behaviour of the off-diagonal bosons under the influence of the Gribov horizon. The propagator \eqref{ww} only contains the $\beta$ Gribov parameter, meaning $\omega$ does not need be considered here.

As found in the previous section, this $\beta$ is not necessary in the regime $a>1/2$, due to the ghost form factor $\langle\sigma_\text{diag}(0)\rangle$ always being smaller than one. In this case, the off-diagonal boson propagator is simply of the massive type.
%\begin{equation}
%	\langle W^{+}_{\mu}(p)W^{-}_{\nu}(-p) \rangle = \frac{1}{p^{2} + \frac{\nu^{2}}{2}g^{2}}T_{\mu\nu}(p^2) \;.
%\end{equation}

In the case that $a<1/2$, the relevant ghost form factor is not automatically smaller than one anymore, and the Gribov parameter $\beta$ becomes necessary. The value of $\beta$ is given by the gap equations \eqref{beta gap eq}, which has exactly the same form as in the case without electromagnetic sector. Therefore the results will also be analogous. As the analysis is quite involved, we just quote the results here.

For $1/e<a<1/2$ the off-diagonal boson field has two real massive poles in its two-point function. One of these has a negative residue, however. This means we find one physical massive excitation, and one unphysical mode in this regime. When $a<1/e$ the two poles acquire a non-zero imaginary part and there are no poles with real mass-squared left. In this region the off-diagonal boson propagator is of Gribov type, and the $W$ boson is completely removed from the spectrum. More details can be found in \cite{Capri:2012ah}.


\begin{figure}\begin{center}
\parbox{.5\textwidth}{\includegraphics[width=.5\textwidth]{regions3d.pdf}} \quad \parbox{.4\textwidth}{\includegraphics[width=.4\textwidth]{regionsslice.pdf}}
\caption{Left is a plot of the region $a'<1/2$ (the region $a'>1/2$ covers all points with higher $\nu$). In red are points where the polynomial in the denominator of \eqref{a3a3} - \eqref{ba3} has three real roots, and in blue are the points where it has one real and two complex conjugate roots. At the right is a slice of the phase diagram for $g=10$. The region $a>1/2$ and $a'>1/2$ is labelled I, the region $a<1/2$ and $a'>1/2$ is II, and the region $a<1/2$ and $a'<1/2$ is split into the regions III (polynomial in the denominator of \eqref{a3a3} - \eqref{ba3} has three real roots, red dots in the diagram at the left) and IV (one real and two complex conjugate roots, blue dots in the diagram at the left). The dashed line separates the different regimes for off-diagonal SU(2) bosons (two real massive poles above the line, two complex conjugate poles below). \label{regionsdiag}}
\end{center}\end{figure}















\subsubsection{The diagonal SU(2) boson and the photon} \label{zandgamma}
The two other gauge bosons --- the diagonal SU(2) boson and the photon, $Z_\mu$ and the $A_\mu$ --- have their propagators given by \eqref{zz}, \eqref{aa} and \eqref{az}. Here, $\omega$ is the only of the two Gribov parameters present.

In the regime $a'>1/2$, $\omega$ is not necessary to restrict the region of integration to $\Omega$. Due to this, the propagators are unmodified in comparison to the perturbative case.
%\begin{equation} \label{propsrewrite}
%	\langle Z_{\mu}(p) Z_{\nu}(-p) \rangle = \frac{1}{p^{2} + \frac{\nu^{2}}{2}(g^{2} + g'^{2})} T_{\mu\nu}(p^2)\;, \quad
%	\langle A_{\mu}(p) A_{\nu}(-p) \rangle = \frac{1}{p^{2}}T_{\mu\nu}(p^2) \;.
%\end{equation}

In the region $a'<1/2$ the Gribov parameter $\omega$ does become necessary, and it has to be computed by solving its gap equation. Due to its complexity it seems impossible to compute analytically. Therefore we turn to numerical methods. 
%Using Mathematica the gap equation can be straighforwardly solved for a list of values for the couplings. In order to do this, we regularize the momentum integration by subtracting a term designed to cancel the large-$p^2$ divergence (as in the Pauli--Villars procedure):
%\begin{equation}
%	\frac{3}{2}g^{2}\int \frac{d^{4}p}{(2\pi)^{4}} \left(\frac{p^{2} + \frac{\nu^{2}}{2}g'^{2}}{p^{6} + \frac{\nu^{2}}{2}(g^{2} + g'^{2})p^{4} + \frac{\omega^{\ast}}{2}g^{2}p^{2} + \frac{\omega^{\ast}}{4}\nu^{2}g^{2}g'^{2}} - \frac1{(p^2+M^2)^2}\right) 	+ \frac{3}{2}g^{2}\int \frac{d^{4}p}{(2\pi)^{4}}\frac1{(p^2+M^2)^2} = 1 \;,
%\end{equation}
%where $M^2$ is an arbitrary mass scale. The second integral is readily computed by hand, whereas the first one converges and can be determined numerically. 
Once the parameter $\omega$ has been (numerically) determined, we look at the propagators to investigate the nature of the spectrum.

%The denominators of the propagators are a polynomial which is of third order in $p^2$. There are two cases: there is a small region in parameter space where the polynomial has three real roots, and for all other values of the couplings there are one real and two complex conjugate roots. In \figurename\ \ref{regionsdiag} these zones are labeled III and IV respectively.

%Just as in \cite{Capri:2013gha} we can decompose the propagator matrix $\Delta_{ij}$ (were $i$ and $j$ run over the fields $A_\mu^3$ and $B_\mu$) as
%\begin{eqnarray}
%	\Delta_{ij} &=& \frac{f_{11}(-m_1^2)+f_{22}(-m_1^2)}{(-m_1^2+m_2^2)(-m_1^2+m_3^2)} \frac1{p^2+m_1^2} \hat v_i^1\hat v_j^1 \nonumber \\
%	&+& \frac{f_{11}(-m_2^2)+f_{22}(-m_2^2)}{(m_1^2-m_2^2)(-m_2^2+m_3^2)} \frac1{p^2+m_2^2} \hat v_i^2\hat v_j^2 + \frac{f_{11}(-m_3^2)+f_{22}(-m_3^2)}{(m_1^2-m_3^2)(m_2^2-m_3^2)} \frac1{p^2+m_3^2} \hat v_i^3\hat v_j^3 \;. \label{partfracdec}
%\end{eqnarray}
%The vectors $v_i^n$ can be interpreted as linear combinations of the $A_3$ and $B$ fields. Decomposing the two-point functions in this way, we thus find three ``states'' $v_1^n A_3 + v_2^n B$. These states are not orthogonal to each other (which would be impossible for three vectors in two dimensions). The coefficients in front of the Yukawa propagators will be the residues of the poles, and they have to be positive for a pole to correspond to a physical excitation. From the analysis in \cite{Capri:2013gha} it also follows that $f_{11}(-m_n^2)+f_{22}(-m_n^2)$ will always be positive if the mass squared is real, which helps to determine the sign of the residue without having to explicitly compute it. The same results can also be attained by introducing ``generalized $i$-particles'' (see \cite{Baulieu:2009ha,Sorella:2010it}).

As the model under consideration depends on three dimensionless parameters ($g$, $g'$ and $\nu/\bar\mu$), it is not possible to plot the parameter dependence of these masses in a visually comprehensible way. Therefore we limit ourselves to discussing the behaviour we observed.

In region III, when there are three real poles in the full two-point function, it turns out that only the two of the three roots we identified have a positive residue and can correspond to physical states, being the one with highest and the one with lowest mass squared. The third one, the root of intermediate value, has negative residue and thus belongs to some negative-norm state, which cannot be physical. All three states have non-zero mass for non-zero values of the electromagnetic coupling $g'$, with the lightest of the states becoming massless in the limit $g'\to0$. In this limit we recover the behaviour found in this regime in the pure $SU(2)$ case \cite{Capri:2012ah} (the $Z$-boson field having one physical and one negative-norm pole in the propagator) with a massless boson decoupled from the non-Abelian sector.

In region IV there is only one state with real mass squared --- the other two having complex mass squared, conjugate to each other --- and from the partial fraction decomposition follows that it has positive residue. This means that, in this region, the diagonal-plus-photon sector contains one physical massive state (becoming massless in the limit $g'\to0$), and two states that can, at best, be interpreted as confined.























%-------------------------------------------
\section{Discussions about the results}
\label{discussion}
%-------------------------------------------

In this chapter we presented results achieved during the study of Yang-Mills models, in the
Landau gauge, coupled to a Higgs field, taking into account non-perturbative effects. More
specifically, the $SU(2)$ and $SU(2)\times U(1)$ models were analysed in $3$- and
$4$-dimensional Euclidean space-time, while the Higgs field was considered in its fundamental
and adjoint representation. The non-perturbative effects were taken into account by considering
the existence of Gribov ambiguities, or Gribov copies, in the (Landau) gauge fixing process. In
order to get rid of those ambiguities we followed the procedure developed by Gribov in his
seminal work \cite{Gribov:1977wm}, which consists in restricting the configuration space of the
gauge field into the first Gribov region $\Omega$. As found by Gribov, that restriction of the
path integral domain leads to a modification of the gluon propagator in a way that it is not
possible to attach any physical particle interpretation to it. The gauge field
propagator develops, after the Gribov restriction, two complex conjugate poles, preventing
any physical particle interpretation, since it presents positivity violation, which is a
reality condition of Osterwalder-Schrader. This may be interpreted as a sign of confinement of
the gauge field. In the same sense we could observe similar modifications of the gluon
propagator in the Yang-Mills + Higgs models. In general, the poles of the gauge field
propagator are functions of the parameters in each Yang-Mills model (including the Gribov
parameter $\gamma$ and the Higgs self mass parameter $\nu$), so that we could identify regions
in the parameter space where the gauge propagator has contributions coming from Yukawa type
modes (with real poles) and/or contributions from Gribov-like modes (presenting {\it cc}
poles). Regions where only Yukawa contribution exists are called {\it Higgs-like regimes}. On
the other hand, regions where there is only Gribov type contributions are named {\it confined
regimes} or {\it confined-like} regions. Note that contributions with negative residue, despite
being of the Yukawa type, have no physical particle interpretation as well.


In general we could find that the Higgs-like regime corresponds to the region of weak coupling,
{\it i.e.} small $g$ and sufficiently large $\nu$, reached in the UV regime. In that region of
the parameter space (coupling constant and Higgs vacuum expectation value), where we expect
perturbation theory to works, we do recover the standard perturbative Yang-Mills-Higgs
propagators. This is an important observation, since it means that the Gribov ambiguity does
not spoils the physical vector boson interpretation of the gauge sector where it is relevant.
On the other side, the confined-like regime corresponds to the strong coupling region in the
parameter space, characterized by large values of $g$ and sufficiently small $\nu$, but still
keeping logarithmic divergences under control. For higher values of the non-Abelian coupling
constant the Gribov horizon lets its influence be felt and the propagators become modified. In
general, for the $SU(2)$ and for the $SU(2)\times U(1)$ cases we could find an intermediate
region where contributions from physical modes (with real poles) mix with contributions from
non-physical modes (with {\it cc} poles or negative residues). 

It is very important to emphasise that the whole analysis strongly depends on the group
representation of the Higgs field, just as in Fradkin \& Shenker's work \cite{Fradkin:1978dv}.

For the fundamental representation of the Higgs field, either in the $SU(2)$ or in the
$SU(2)\times U(1)$ model, we could find that the two detected regimes, Higgs- and
confined-like, may be continuously connected, in the sense that the parameters of the theory
are allowed to continuously vary from one region to another, without leading to any
discontinuity or singularity of the vacuum energy or the two-point Green function. However, we
have to be careful when talking about \emph{analyticity region} in our perturbative approach,
since we could not rigorously prove such property, since there exist a region of the parameter space where our perturbative approximation is not reliable.

Something quite different happens in the adjoint representation of the scalar field. There we
could explicitly show (\emph{cf.} the section \ref{Adjrep}) the existence of an specific
configuration in the parameter space where the vacuum energy develops a jump discontinuity.
However, at the very point of the jump our approximation is beyond its range of
validity, and we cannot make any statement for sure about analyticity.

In the adjoint representation of the Higgs field the scenario looks quite different. Besides
the confined-like regime, in which the gluon propagator is of the Gribov-type, our results
indicated the existence of what can be called a $U(1)$ confined-like regime for finite values
of the Higgs condensate. This is a regime in which the third component $A^3_\mu$ of the gauge
field  displays a propagator of the Gribov-type, while the remaining off-diagonal components
$A^{\alpha}_\mu$, $\alpha=1,2$, exhibit a propagator of the Yukawa type. Interestingly,
something similar to this has been already detected on lattice studies of the three-dimensional
Georgi-Glashow model \cite{Nadkarni:1989na,Hart:1996ac}. A second result is the absence of a
Coulomb regime for finite Higgs condensate. For an infinite value of the latter, we were able
to clearly reveal the existence of a massless photon, in agreement with the lattice suggestion,
\emph{e.g.} the \cite{Brower:1982yn}.

One should keep in mind that in our analytic non-perturbative model, where Gribov ambiguities
where taken into account, we cannot properly speak about physical phases, at all. The non-local
Gribov term leads to a soft breaking of the BRST symmetry, forbidding us from defining in the
usual sense physical states. At the same time, there is no obvious gauge invariant order
parameter available in our model, which would be useful for probing phase transition.

Keeping safe the due difference between Fradkin \& Shenker's approach to the Yang-Mills + Higgs
theory and ours perturbative approach, it is fair enough to acknowledge the matching of a
couple of remarkable results. Fradkin \& Shenker clearly say that in the fundamental
representation of the Higgs field there is no phase transition to occur, being the theory in
the symmetric (or ordered) phase in the (almost) entire parameter space, despite of the
particular case of null Higgs coupling constant, $\nu =0$, where the theory is found to be in
the disordered phase  \cite{Fradkin:1978dv}. Besides, they did show that there exist two
different regimes in the configuration space, called \emph{confinement-like regime} and
\emph{Higgs-like regime}, and that any point of these regions of the configuration space are
smoothly connected to each other. In other words, the system is allowed to smoothly hang from
one point in the \emph{confined-like regime} to another point in the \emph{Higgs-like regime}.
It should be emphasized that in their case a phase transition is properly defined, since they
work on the lattice, measuring the gauge invariant Wilson loop order parameter. They could also
prove the existence of the analyticity region.

%At the end, one has to acknowledge that, despite being unable to define phase transition, there
%exist clear indicative signs of agreement between both approaches.
%











