

%%%%%%%%%%%%%%%%%%%%%%%%%%%%%%%%%%%%%%%%%%%%%%%%%%%%%%%%%%%%%%%%%%%%
\chapter{Backgrounds: Gribov, Gribov-Zwanziger and a bit of Refined Gribov-Zwanziger} 
\chaptermark{Backgrounds}
\label{usefulbkground}
%%%%%%%%%%%%%%%%%%%%%%%%%%%%%%%%%%%%%%%%%%%%%%%%%%%%%%%%%%%%%%%%%%%%


The understanding of nonperturbative aspects of non-Abelian gauge theories is one of the main
challenging problems in quantum field theories. As an example, we may quote the transition
between the confined and the Higgs regimes in an Yang-Mills theory coupled to a scalar Higgs
field. See refs.\cite{Polyakov:1976fu,Cornwall:1998pt,Baulieu:2001vw} for analytical
investigations and
\cite{Fradkin:1978dv,Nadkarni:1989na,Hart:1996ac,Caudy:2007sf,Maas:2011yx,Maas:2010nc,
Greensite:2011zz} for results obtained through numerical lattice simulations.


Non-perturbative effects can be accounted perturbatively by considering ambiguities in the
gauge-fixing process, first noted by Gribov in \cite{Gribov:1977wm}. These ambiguities, also
referred to as Gribov copies, are unavoidably present in the Landau gauge, since the
(Hermitian) Faddeev-Popov operator admits the existence of zero modes. The widely accepted
mechanism to get rid of these ambiguities was firstly proposed by Gribov, in his famous work
\cite{Gribov:1977wm}, where the domain of integration of the gauge field should be restricted
to a closed region that satisfy specific requirements. As a consequence, the gauge propagator
does not belong to the physical spectrum of the theory anymore and the ghost propagator is
enhanced in the deep IR regime. This framework is an effective model of the Yang-Mills theory,
and as such is useful in analytical analysis of the gauge field theory.

This chapter is devoted to the introduction of the Gribov, Gribov-Zwanziger (GZ) and of the
refined version of the Gribov-Zwanziger (RGZ) frameworks, give the central role these
approaches play in this thesis. However, given the existence of a vast bibliography covering
this topic, hanging from scientific papers to pedagogical reviews
\cite{Gribov:1977wm,Sobreiro:2005ec,Vandersickel:2012tz,Dudal:2009bf}, this chapter is not meant to be one
more detailed pedagogical review, but rather it will provide the most important concepts and
equations that are useful to the comprehension of this thesis. For example, the main idea of
the Gribov mechanism to get rid of (infinitesimal) gauge copies and the consequent violation of
positivity by the gauge field propagator will be presented in the first section of this
chapter, while the horizon function, developed by Zwanziger, is presented in the section
\ref{gzintro} together with the refined version. At the end, the important concept of BRST
symmetry breaking will be introduced and a brief discussion will be developed.












%-------------------------------------------------------
\section{An introduction to Gribov's issue}
\label{introGribov}
%-------------------------------------------------------

This section is organized in a way to provide a brief introduction to Gribov's ambiguities by
following his seminal work \cite{Gribov:1977wm}. We do not intend to provide the final word on
this matter and as such we would refer to \cite{Vandersickel:2012tz} for a more complete and
pedagogical reference. The Faddeev-Popov quantization procedure is reproduced in the subsection
\ref{introductiontogribov1}, while the Gribov problem will be introduced and implemented on the
path integral formalism in the two subsequent subsections, \ref{Gribov's issue} and
\ref{Gribovimplementation}. At the end of this section we will present one of the most
important outcomes of the Gribov issue: a possible interpretation of gluon confinement, which
is encoded in the poles of the gluon propagators. We should say, to clarify matters, that our
present work concerns computations up to one-loop order in perturbation theory.






\subsection{Quantization of non-Abelian gauge field}
\label{introductiontogribov1}

The gauge-invariant action of a non-Abelian gauge field, or the Yang-Mills ($\YM$) action, is given by
\begin{eqnarray}
S_{\text{YM}} &=& \int \mathrm{d}^d x \frac{1}{4}  F_{\mu\nu}^a  F_{\mu\nu}^a \;,
\label{YMact}
\end{eqnarray}
with $
F^{a}_{\mu\nu} ~=~ \partial_{\mu}A^{a}_{\nu} - \partial_{\nu}A^{a}_{\mu} + gf^{abc}A^{b}_{\mu}A^{c}_{\nu}
$ 
being the field strength tensor. The action \eqref{YMact} enjoys the feature of being symmetric under gauge transformation, which is defined for the gauge field as
\begin{eqnarray}
A_{\mu}' ~=~ U^{\dagger}A_{\mu}U - \frac{\ii}{g} U^{\dagger}(\p_\mu U)\;,
\label{gaugetransf}
\end{eqnarray}
with $U(x) \in SU(N)$ and $N$ being the number of colors. A geometrical representation of this symmetry can be seen in Figure \ref{fig1}, where each orbit, representing equivalent gauge fields, is crossed by a gauge curve ${\cal F}$. It means that, the YM action is invariant under transformations that keep the transformed gauge field on the same gauge orbit.

As is well known, the gauge invariance of the action induces inconsistencies in the
quantization of the gauge field reflecting the existence of infinite physically equivalent
configurations. To get rid of those spurious configurations from the system one has to fix the
gauge, which is, in the geometrical view, to choose a convenient curve ${\cal F}$ that crosses
only once each gauge orbit. In the path integral formalism, the gauge-fixing procedure is
carried out by the Faddeev-Popov's procedure.


\begin{figure}[h]
\begin{center}
\includegraphics[width=8cm]{faddeevbetercrop.pdf}
\caption{Gauge orbits of a system with rotational symmetry in a plane and a function $\mathcal F$ which picks one representative from each gauge orbit.}
\label{fig1}
\end{center}
\end{figure}


The (Euclidean) gauge fixed partition function reads
\begin{eqnarray}
\label{genym}
Z_{\text{YM}}(J) ~=~ \int_{\cal F} [\d A]\;\; \e^{-S_{\text{YM}} + \int {\rm d} x J_\mu^a A_\mu^a}\;,
\label{YMaction0}
\end{eqnarray}
where $\int_{\cal F}$ denotes the path integral restricted to the curve $\cal F$, which can be recast in the form



%Naively, if one intends to compute the gluon propagator predicted by \eqref{YMact} then the following generating functional would be needed,
%\begin{eqnarray}
%\label{genym}
%Z_{\text{YM}}(J) ~=~ \int [\d A]\;\; \e^{-S_{\text{YM}} + \int {\rm d} x J_\mu^a A_\mu^a}\;.
%\label{YMaction0}
%\end{eqnarray}
%Let us report here the computation of the tree level gluon propagator, and for that propose only quadratic terms involving the quantized gauge field are needed. With that being said, the quadratic generating functional becomes
%\begin{eqnarray}
%\label{YM1}
%Z(J)_\quadr ~=~ \int [\d A] \;\;  \e^{ \frac{1}{2} \int \d x \d y  A_\mu^a(x)\left[ {\mathbb Q}^{ab}_{\mu\nu}(x,y) \right] A_\nu^b(y) + \int \d x J_\mu^a(x) A_\mu^a (x)}\;,
%\end{eqnarray}
%with
%\begin{eqnarray}
%{\mathbb Q}_{\mu\nu}^{ab}(x,y) ~=~ \delta^{ab} \delta (x -y)  ( \p^2 \delta_{\mu\nu} - \p_\mu \p_\nu )\;.
%\end{eqnarray}
%Performing the Gaussian integration it is not difficult to see that the operator ${\mathbb Q}_{\mu\nu}^{ab}(x,y)$ is not invertible, due to the existence of zero modes. By zero modes we mean eigenstates associated with zero eigenvalues. The existence of these zero modes solutions become evident by direct inspection of the eigenvalue equation,
%\begin{eqnarray}
%\int  \d y \;\; \mathbb{Q}_{\mu\nu}^{ab}(x,y) {\rm Y}_\nu(y)~=~ 0 \;,
%\qquad \text{if, for example,} \qquad {\rm Y}_\nu(y) ~=~ \p_\nu \theta (y) \;.
%\label{zeromode}
%\end{eqnarray}
%It is not difficult to link the eigenstate zero modes $\p_\nu \theta(y)$ with a given configuration of a gauge transformed field $A_{\nu}$ if the element of the gauge group $SU(N)$ would be written as
%\begin{eqnarray}
%U(x) ~=~ e^{-ig\theta^{a}(x)\tau^{a}}\;,
%\end{eqnarray}
%with $\tau^{a}$ being the infinitesimal generator of the $SU(N)$ group (with $a = 1,2, ..., N^{2}-1$) and $\theta^{a}$ the gauge transformation parameter. Therefore, if we chose an specific gauge field configuration so that its gauge transformation leads to 
%\begin{eqnarray}
%A_{\mu}' &=& - \frac{\ii}{g} (\p_\mu U) U^{\dagger} ~=~  \tau^a \p_\mu \theta^a\;,
%\end{eqnarray}
%which is called a ``pure gauge'' configuration, then we can see that the zero modes
%\eqref{zeromode} are closely related to a gauge transformation. The crucial point is that
%physically equivalent configurations of the gauge field $A_{\mu}$ --- $i.e.$ those on the same
%orbit in Figure \ref{fig1} --- are being summed up in \eqref{genym} and between them are the
%zero mode configurations. Due to the symmetry of the Yang-Mills action under gauge
%transformations, we need only to consider those physically distinct configurations, which
%means that we need to pick up only one gauge field configuration per orbit. That is the
%meaning of gauge-fixing: the definition of a curve ${\cal F}$ on the configuration space of
%the gauge field that crosses once each orbit. This definition of gauge-fixing should be kept in mind for further discussion.

%The implementation of the gauge-fixing restriction will be done according to Faddeev-Popov's procedure, to be described in what follows. 


%Let us consider two gauge configurations that are infinitesimally related by a gauge transformation, then belonging to the same gauge orbit. Namely,
%\begin{eqnarray}
%A'_{\mu} ~=~ A_{\mu} - D_{\mu}\theta(x)  \;,
%\label{inftgaugetransf}
%\end{eqnarray}
%where $\theta(x) = \theta^{a}(x)\tau^{a}$ should be just regarded, for a while, as the (infinitesimal) parameter responsible for the $SU(N)$ gauge transformation. Besides that, the restriction of the functional integration to the curve ${\cal F}(A)$ will be done by means of a delta function,
\begin{eqnarray}
Z_{\YM}(J) ~=~ V\int [\d A] \;\; \Delta_{\cal F}(A) \delta({\cal F}(A')) \mathrm{e}^{-S_{\text{YM}} + \int {\rm d} x J_\mu^a A_\mu^a}\;.
\label{genfp}
\end{eqnarray}
The $V$ factor accounts for the (infinite) orbit's volume, while $\Delta_{\cal F}(A)$ stands for the Jacobian of infinitesimal gauge transformations 
\begin{eqnarray}
A'_{\mu} ~=~ A_{\mu} - D_{\mu}\theta(x) \,.
\label{inftgaugetransf}
\end{eqnarray}
The Jacobian is there since we are working with the gauge transformed integration measure. The
$\theta(x) ~=~ \theta^{a}(x)\tau^{a}$ stands for the infinitesimal gauge transformation
parameter, while $\tau^{a}$ are the $SU(N)$ generators.  $\delta({\cal F}(A))$ is the delta
function ensuring the gauge condition ${\cal F(A)}=0$. It is worthwhile to emphasize that the
Jacobian of a given transformation ({\it e.g.} the infinitesimal gauge transformation) is
defined as the absolute value of the determinant of the derivative --- with respect to the
transformation parameter ($\theta$) ---  of the transformed field. In our gauge-fixing case we
have
\begin{equation}
\Delta_{\mathcal F} (A) ~=~  |\det \mathcal  M_{ab} (x,y) | \qquad \text{with} \qquad  \mathcal
M_{ab} (x,y) ~=~ \left. \frac{\delta \mathcal F^a (A'_\mu (x))  }{\delta \theta^b(y)}
\right|_{\mathcal F(A') =0} \;.
\label{absvalue}
\end{equation}
The delta function can be written as
\begin{eqnarray}
\delta({\cal F}) ~\propto~ \exp\left\{-\frac{i}{2\xi}\int d^{d}x\,{\cal F}^{2}\right\} \,,
\end{eqnarray}
so that performing a sort of functional Fourier transformation, with the introduction of an
auxiliary field named ``Nakanishi-Lautrup field''. Firstly, let us take the example of a real
function, in order to clarify things:
\begin{eqnarray}
\hat{f}(b) ~=~ \int d^{d}x \;\e^{-ib\,x}f(x) ~=~ \int d^{d}x\;
\e^{-ib\,x}\e^{-\frac{i}{2\xi}x^{2}}\,,
\end{eqnarray}
which leads to the following Fourier transformed function,
\begin{eqnarray}
\hat{f}(b) ~\propto~ \e^{i\frac{\xi b^{2}}{2}} \,.
\end{eqnarray}
Thus, returning to the gauge-fixing, one ends up with an equivalent expression,
\begin{eqnarray}
\delta({\cal F}) ~\propto~ \int \left[\d b^{a} \right] \exp\left\{ i\int d^{d}x\, b^{a}{\cal
F}^{a} \right\}   \exp\left\{\frac{i\xi}{2}\int d^{d}x\,b^{a}b^{a}\right\}\,.
\label{deltaF}
\end{eqnarray}
Note that the Nakanishi-Lautrup field works as a Lagrange multiplier field, ensuring the gauge
fixing condition. The Landau gauge is recovered in the limit $\xi \to 0$.

In order to introduce the Gribov issue in its original form, the Landau gauge condition
will be chosen. Namely,
\begin{eqnarray}
\mathcal{F}^a (A'_\mu (x)) ~=~ \p_{\mu} A_{\mu }^{\prime a} (x) \;.
\label{notideal}
\end{eqnarray}
The $A'_{\mu}$ stands for the infinitesimal gauge transformation, given by the equation
\eqref{inftgaugetransf}.
After choosing the gauge condition, the Jacobian operator, named the Faddeev-Popov operator,
reads 
\begin{eqnarray}\label{mab}
 \mathcal M_{ab} (x,y) ~=~  \left.    - \p_\mu  D_\mu^{ab} \delta (y-x)    \right|_{ \mathcal
F(A) = 0 } \;,
\label{fpoperator}
\end{eqnarray}
while the delta function \eqref{deltaF} amounts to
\begin{eqnarray}
\delta({\cal F}) ~\propto~ \int \left[\d b^{a} \right] \exp\left\{ ib^{a} \p_{\mu}A^{a}_{\mu} +
\frac{i\xi}{2}b^{a} b^{a} \right\}\;.
\end{eqnarray}
In order to obtain the final expression of the gauge fixed Yang-Mills partition function, the
Jacobian must be rewritten as ``the exponential of something'', in order to be added into the
action. This will be achieved by introducing a couple of anticommuting {\it real} Grassmann
variables, named the ``Faddeev-Popov ghosts'' $(\bar{c}^{a},\,c^{a})$. The point is that, the
integration rule of a {\it Gaussian-like functional of Grassmann variables} is given by
\begin{eqnarray}
\int \left[ \d\bar{c} \right] \left[ \d c \right] \exp\left\{ \bar{c}^{a} M^{ab} c^{b} \right\}
~=~ \det[M^{ab}]\,.
\end{eqnarray}
Therefore, replacing $M^{ab}$ for the Faddeev-Popov operator ${\cal M}^{ab}$, one ends up with
the following generating function,
\begin{eqnarray}
Z[J] &=&   \int [\d A][\d c] [\d \overline c]         \exp \left[- S_\YM + \int \d
x \left(   \overline c^a   \p_\mu  D_\mu^{ab}  c^b -  \frac{1}{2 \xi} (\p_\mu A_\mu^a )^2
\right) + \int \d x J_\mu^a A_\mu^a  \right]\,.
\label{genfuncfp}
\end{eqnarray}

Let us emphasize two important (not so clear) assumptions made in the process to obtain
\eqref{genfuncfp}:
\begin{itemize}
\item The gauge condition ${\cal F}^{a}$ is said to pick up only one field configuration from each gauge orbit, representing the physical equivalent configurations related by gauge transformations;

\item The determinant of $\mathcal M_{ab} (x,y)$ is supposed to be always positive.
\end{itemize}
These assumptions were considered to be true during the quantization procedure developed by
Faddeev-Popov and described above. Gribov showed in his work \cite{Gribov:1977wm} that the
failure of these assumptions are closed related to the existence of zero-modes of the
Faddeev-Popov operator ${\cal M}^{ab}$. The problem surrounding the failure of these
assumptions defines the Gribov issue. In the next subsection the Gribov issue will be
described and subsequently the mechanism proposed by Gribov to fix these these quantization
inconsistencies will be presented. Important consequences of such mechanism will be discussed
in the subsection \eqref{A sign of confinement from gluon propagator}.















\subsection{Gribov's issue}
\label{Gribov's issue}

As stated before, one of the most important hypotheses required to derive the gauge fixed
Yang-Mills partition function is that: {\it once the gauge-fixing condition
is chosen, one should be able to find out only one gauge field configuration $A_{\mu}$ that
fulfils the gauge condition, ${\cal F}=0$, and that is related to another configuration
$A'_{\mu}$ through a gauge transformation}. This situation is represented in Figure \ref{2fig1}
by the curve $\bf L$ (where the Landau gauge is chosen). In other words, it means that in
principle one should {\it not} be able to find out two gauge-equivalent configurations, let us
say $A_{\mu}(x)$ and $A'_{\mu}(x)$, that satisfy, both of them, the gauge condition. Such
hypothetically forbidden situation is graphically depicted in Figure \ref{2fig1} by the curve
${\bf L}'$. Another forbidden configuration is the one described in Figure \ref{2fig1} by the
curve ${\bf L}''$. This curve describes the situation where no gauge-equivalent configuration
satisfy the gauge condition, at all.




\begin{figure}[h]
\begin{center}
\includegraphics[width=8cm]{gribov1.pdf}
\caption{The gauge condition curve can cross each orbit of equivalence once, more then once and
at no point at all. The horizontal axes denotes the transversal gluon propagator, while the
vertical axis represents the longitudinal one. This picture was taken from Gribov's
seminal paper \cite{Gribov:1977wm}.}
\label{2fig1}
\end{center}
\end{figure}


While there is no examples of situations depicted by the curve $\bf L''$, the
situation described by the curve $\bf L'$ is quite typical in non-Abelian gauge theories and,
therefore, is worth analysing \cite{Gribov:1977wm,Sobreiro:2005ec,Vandersickel:2012tz,Dudal:2009bf}. To
this end, let us consider two gauge-equivalent configurations, $A'_{\mu}$ and $A_{\mu}$,
related by an {\it infinitesimal} gauge transformation \eqref{inftgaugetransf} and, both of
them, satisfying the Landau gauge condition. That is,
\begin{eqnarray}
A'_{\mu} ~=~ A_{\mu} - D_{\mu}\alpha(x)\;,  \qquad  \p_\mu A_\mu ~=~ 0 \quad \& \quad  \p_\mu A_\mu' ~=~ 0 \;.
\end{eqnarray}
Therefore, one should end up with
\begin{eqnarray}
- \p_\mu D_\mu \alpha &=& 0\;,
\label{FPeigenvlueqtion}
\end{eqnarray}
whence $D_{\mu}$ stands for the covariant derivative,
\begin{eqnarray}
D_{\mu} ~=~ \p_{\mu}\delta^{ab} - igA_{\mu}\,.
\end{eqnarray}
Such equation points to the existence of zero-modes, that are eigenstates of the FP operator
associated to null eigenvalues. Therefore, one may conclude that if there are (at least) two
infinitesimally gauge-equivalent fields satisfying the Landau gauge, defining gauge copies, the
Faddeev-Popov operator has zero-modes (eigenstates associated to zero eigenvalues). Note that
for the Abelian case the equation \eqref{FPeigenvlueqtion} reduces to the Laplace equation,
\begin{eqnarray}
\p^{2} \alpha &=& 0\;.
\label{abelianeingvalue}
\end{eqnarray}
Since it defines plane waves, which are not normalisable, we will not consider this case,
restricting ourselves to the analysis of fields that smoothly vanish at infinity. It becomes 
quite evident that a closer look at the space of eigenvalues of the FP operator is of great
importance for a better comprehension of the problem. Besides, let us regard that the FP
operator is Hermitian in the Landau gauge, which allows us to sort its eigenvalues in the real
axes. Therefore, let us start by considering a gauge configuration that is close enough to the
trivial vacuum. In this case the eigenvalue equation,
\begin{eqnarray}
- \p_\mu D_\mu \psi ~=~ \epsilon \psi\;,
\end{eqnarray}
reduces to
\begin{equation}
-\p_\mu^2 \psi = \epsilon \psi \;.
\end{equation}
Notice that this equation is solvable only for positive $\epsilon$, since in the momentum space
we have $p^{2} = \epsilon >0$. Then, for small enough field configurations there is no
zero-mode issues anymore. Otherwise, if $A_{\mu}$ turns out to be large, but still not too
large, one reaches the zero-mode solution $\epsilon = 0$, since a higher potential of the gauge
field tends to decrease the eigenvalue of the FP operator in the Landau gauge. Thus, for even
large amplitudes of $A_{\mu}$ the eigenvalue turns to be negative; if it keeps growing the
$\epsilon$ reaches zero again. Note that, in the Landau gauge, it is possible to identify a
threshold value for the squared norm of the gauge field, $\Vert A \Vert^{2}_{c}$, below
which the eigenvalue of the FP operator is positive, and for amplitudes whose squared norm
is greater than such critical value the eigenvalue is negative.

In his work \cite{Gribov:1977wm} Gribov showed that the domain of functional integration should
be restricted to the first region, named ``first Gribov region'' $\Omega$, where $\epsilon >0$,
in order to avoid gauge copies. It is, however, known that this region is actually not
completely free of copies. Besides, it has being analytically motivated that the existence of
those copies inside $\Omega$ do not influence physical results (see \cite{Dudal:2014rxa} and
references therein). Furthermore, until today there is no way to analytically implement the
restriction of the partition function to the region actually free of copies, known as the
Fundamental Modular Region, $\Lambda$.

The first Gribov region $\Omega$ enjoys some useful properties that have mathematical proof
still only in the Landau gauge, which justifies our interest in this gauge (see
\cite{Vandersickel:2012tz} and references therein) \footnote{{\it cf.}
\cite{Capri:2016aqq,Capri:2015nzw,Capri:2015pja,Capri:2015ixa} for recent developments on the
Gribov issue in the wider class of Linear Covariant Gauges.}. Namely,
\begin{itemize}
\item 
For every field configuration infinitesimally close to the border $\delta\Omega$  and belonging to the region immediately out side $\Omega$ (called $\Omega_{2}$), there exist a gauge-equivalent configuration belonging to $\Omega$ and infinitesimally close to the border $\delta\Omega$ as well \cite{Gribov:1977wm}. It was also proven that every gauge orbit intersects the first Gribov region $\Omega$ \cite{Capri:2005dy,Zwanziger:1982na}.

\item
The Gribov region is convex \cite{Zwanziger:2003cf}. This means that for two gluon fields $A_\mu^1$ and $A_\mu^2$ belonging to the Gribov region, also the gluon field $A_\mu = \alpha A_\mu^1 + \beta A_\mu^2$ with $\alpha, \beta \geq 0$ and $\alpha + \beta  =1$, is inside the Gribov region.

\item
One may also show that the Gribov region is bounded in every direction \cite{Zwanziger:2003cf}.
\end{itemize}

For more details concerning (proofs of) properties of Gribov regions see \cite{Vandersickel:2012tz} and references therein.
















\subsection{Implementing the restriction to the first Gribov region $\Omega$}
\label{Gribovimplementation}
As was discussed before, our aim is to restrict the domain of the functional integration to the region where the $\FP$ operator has positive eigenvalues. Therefore, let us define the first Gribov region as the region where the $\FP$ operator is positive definite. Namely,
\begin{eqnarray}
\Omega ~\equiv ~ \{ A^a_{\mu}, \, \p_{\mu} A^a_{\mu} ~=~  0, \, \mathcal{M}^{ab}  >0  \} \;.
\label{defgribovregion}
\end{eqnarray}
Once again, $\mathcal{M}^{ab}$ stands for the Faddeev-Popov operator, defined innumerable times
and given once more,
\begin{eqnarray}\label{mab2}
\mathcal M^{ab}(x,y) ~=~ - \p_\mu  D_\mu^{ab} \delta(x-y)  ~=~ 
-\p_\mu \left( \p_\mu \delta^{ab} -
gf^{abc} A_\mu^c \right) \delta(x-y)\;.
\end{eqnarray}
The condition defines a positive definite operator, which means that only gauge field
configurations associated with positive eigenvalues of the FP operator will be considered.

One should notice that, since the $\FP$ operator is the inverse of the ghost propagator, thus
the ghost propagator plays a central role in the Gribov issue. Therefore, the propagator of the
ghost field is worthwhile to compute, which will be done up to the first loop order, by
following \cite{Gribov:1977wm}. The positive definite condition imposed on the FP operator can
be implemented into the partition function by means of the ghost propagator, which can be read
as
\begin{eqnarray}
\left\langle  c^{a}(k) \overline{c}^{b}(-k)  \right\rangle ~=~ \mathcal G (k^2, A)_{ab}\;.
\label{ghostequation0}
\end{eqnarray}
The effective computation of the equation \eqref{ghostequation0} shall be performed with the
partition function \eqref{genfuncfp} and treating the gauge field as an external classical
field. Namely, one gets,
\begin{eqnarray}
\mathcal G (k^2, A) &=&  \frac{1}{N^2 -1}\delta_{ab} \mathcal G (k^2, A)_{ab} ~=~
\frac{1}{k^2} + \frac{1}{V}\frac{1}{k^4} \frac{N g^2}{N^2 - 1} \int\frac{ \d^d q}{(2 \pi)^d}
A_\mu^\ell(-q) A_\nu^{\ell} (q)  \frac{(k+q)_\mu  k_\nu}{(k+q)^2} \nonumber\\
&=& \frac{1}{k^2}\left( 1 + \sigma(k, A) \right)\;,
\label{coloraway}
\end{eqnarray}
whereby
\begin{eqnarray}
\sigma(k,A) ~=~ \frac{1}{V}\frac{1}{k^2} \frac{Ng^2}{N^2 - 1} \int\frac{ \d^d q}{(2 \pi)^d}
A_\mu^\ell(-q) A_\nu^{\ell} (q) \frac{ (k+q)_\mu  k_\nu }{(k+q)^2}\;.
\label{1ghostformfacto}
\end{eqnarray}
Making use of the property
\begin{eqnarray}
A_{\mu }^{a}(q)A_{\nu }^{a}(-q) &=&
\left( \delta _{\mu \nu }-\frac{q_{\mu }q_{\nu }}{q^{2}}\right) \omega (A)(q)   
\nonumber \\
&\Rightarrow &\omega (A)(q)=\frac{1}{d-1}A_{\lambda }^{a}(q)A_{\lambda }^{a}(-q)\,,
\end{eqnarray}
which follows from the transversality of the gauge field, $q_\mu A^a_\mu(q)=0$, 
we will have
\begin{eqnarray}
\sigma(k,A) &=& \frac{1}{V}\frac{1}{k^2} \frac{Ng^2}{(N^2 - 1)(d-1)} 
\int\frac{ \d^d q}{(2 \pi)^d}\;
A_\mu^\ell(-q) A_\mu^{\ell}(q)
\left( \delta _{\mu \nu }-\frac{q_{\mu }q_{\nu }}{q^{2}}\right) 
\frac{ (k+q)_\mu  k_\nu }{(k+q)^2}
\;,
\nonumber \\
&=& 
\frac{1}{V}\frac{1}{k^2} \frac{Ng^2}{(N^2 - 1)(d-1)} 
\int\frac{ \d^d q}{(2 \pi)^d}\;
\frac{A_\mu^\ell(q) A_\mu^{\ell}(-q)}{(k+q)^2}
\left( (k+q)_\mu  k_\mu -\frac{q_{\mu }(k+q)_\mu  k_\nu q_{\nu}}{q^{2}} 
\right)
\,.
\nonumber \\
\label{1ghostformfacto3}
\end{eqnarray}
Such expression may be rewritten as follows,
\begin{eqnarray}
\sigma(k,A) &=& \frac{1}{V}\frac{1}{k^2} \frac{Ng^2}{(N^2 - 1)(d-1)} 
\int\frac{ \d^d q}{(2 \pi)^d}\;
\frac{A_\mu^\ell(q) A_\mu^{\ell}(-q)}{(k+q)^2}
\left(k^{2}  - \frac{q_{\mu}k_\mu k_\nu q_{\nu} }{q^{2}} 
\right) 
\,,
\nonumber \\
&=&
\frac{1}{V} \frac{Ng^2}{(N^2 - 1)(d-1)} 
\int\frac{ \d^d q}{(2 \pi)^d}\;
\frac{A_\mu^\ell(q) A_\mu^{\ell}(-q)}{(k+q)^2}
\frac{k_{\mu}k_{\nu}}{k^2}
\left(\delta_{\mu\nu} - \frac{q_{\mu}q_{\nu} }{q^{2}} 
\right)  \,.
\label{1ghostformfacto3}
\end{eqnarray}
Now, reminding the property
\begin{equation}
\int \frac{d^{d}p}{(2\pi )^{d}}
\mathcal{F}(p^2) \left( \delta _{\mu \nu }-\frac{p_{\mu }p_{\nu }}{p^{2}}\right)  
~=~ \left( \frac{d-1}{d} \right)  
\int \frac{d^{d}p}{(2\pi )^{d}}\;\mathcal{F}(p^2) \delta_{\mu\nu} \,,
\label{a}
\end{equation}
the ghost form factor becomes
\begin{eqnarray}
\sigma(k,A) &=&
\frac{1}{V} \frac{Ng^2}{d(N^2 - 1)} 
\int\frac{ \d^d q}{(2 \pi)^d}\;
\frac{A_\mu^\ell(q) A_\mu^{\ell}(-q)}{(k+q)^2}  \,.
\label{ghostfrmfact1}
\end{eqnarray}
The ghost propagator \eqref{coloraway} can be perturbatively approximated by
\begin{eqnarray}
\mathcal G (k^2, A) ~\approx~  \frac{1}{k^2}\frac{1}{( 1 - \sigma(k, A) )}
\label{ghstprop00}
\end{eqnarray}
whereby we can see that for $ \sigma(k, A) < 1$ the domain of integration is safely restricted
to $\Omega$, characterising the no-pole condition. It is not so dificult to see that 
$\sigma(k,A)$ is a
decreasing function of $k$, from \eqref{ghostfrmfact1}, which means that the largest value of
$\sigma$ is obtained at $k=0$. Therefore, if the condition $ \sigma(0, A) < 1$ is ensured, the
system is (or would be) completely safe of gauge copies for any non zero value of $k$. Note
that the only allowed pole is at $k^2=0$, which has the meaning of approaching the boundary of
the region $\Omega$. At the end, the ghost form may be computed by taking the limit 
$k^{2} \to 0$, which reads
\begin{eqnarray}
\sigma(0,A) ~=~
\frac{1}{V} \frac{1}{d} \frac{Ng^2}{N^2 - 1} \int\frac{ \d^d q}{(2 \pi)^4} A_\alpha^\ell(-q) A_\alpha^{\ell} (q) \frac{ 1 }{q^2}\;.
\label{nopole}
\end{eqnarray}
The partition function restricted to $\Omega$ then becomes,
\begin{eqnarray}\label{ZJ}
Z_{G} &=& \int_\Omega [\d A]   \exp \left[- S_\FP   \right] 
\nonumber\\
&=& \int [\d A][\d c] [\d \overline c]    V(\Omega)     \exp \left[- S_\YM  -   \int \d x \left(  \overline c^a   \p_\mu  D_\mu^{ab}  c^b -  \frac{1}{2 \xi} (\p_\mu A_\mu^a )^2  \right)  \right] \;,
\end{eqnarray}
with
\begin{eqnarray}
V(\Omega) &=& \theta (1 - \sigma(0,A))\;,
\label{nopolestepfunct}
\end{eqnarray}
where $\theta (1 - \sigma(0,A))$ is the Heaviside step function ensuring the no-pole
condition. Considering the transversality of the gauge field in the Landau gauge and by making
use of the integral representation of the Heaviside step function, one gets the following
expression of the partition function restricted to the first Gribov region $\Omega$,
%\footnote{For more details concerning the derivation of the partition  function restricted to
%the first Gribov region take a look at
%\cite{Gribov:1977wm,Sobreiro:2005ec,Vandersickel:2012tz}.},
\begin{equation}
Z_{G} ~=~  \mathcal N \int \frac{\d \beta}{2\pi \ii \beta} \int {\cal D} A    \e^{\beta (1 -
\sigma(0,A))}    \e^{ - S_\FP }
\;.
\label{fullgenertfunct}
\end{equation}
The final expression for the gauge fixed Yang-Mills action accounting for the Gribov 
copies reads
\begin{eqnarray}
S_{G} ~=~ S_{\YM} + S_{\gf} +  S_{\beta}\;,
\label{fullgribovact}
\end{eqnarray}
with $S_{\beta} ~=~ \beta\left( \sigma(0,A) - 1\right)$.

Roughly speaking, the Gribov restriction to the first Gribov region has already been
implemented into the partition function. However, notice that a new parameter has been
introduced, named the Gribov parameter $\beta$, which still deserves some analysis. As will
become clear, this new parameter is not a free parameter of the theory, but rather it is
dynamically determined by its gap equation, which amounts to ensure the no-pole condition. Akin
to the mass gap equation, the Gribov parameter gap equation will be derived from the vacuum
energy of the theory, computed up to the first loop order in perturbation theory. To that end,
only terms quadratic in the fields, from the Gribov action \eqref{fullgribovact}, must be taken
into account. Doing so, one should get
\begin{equation}
Z_{quad} ~=~ \iint \frac{\d\beta e^{\beta }}{2\pi i\beta } [\d A] \; 
\exp \left\{-\frac{1}{2}  \int \frac{\d^{d}q}{(2\pi )^{d}}  \;  A_{\mu }^{a}(q)\mathcal{P}_{\mu \nu }^{ab }A_{\nu }^{b }(-q)    \right\}  \;,
\label{Zq0}
\end{equation}
with
\begin{eqnarray}
\mathcal{P}_{\mu \nu }^{ab} &=&  \delta^{ab } \left(  q^{2}\delta _{\mu \nu }   +  \left( \frac{1}{\xi } -1 \right) q_{\mu }q_{\nu } +  \frac{2Ng^{2}\beta}{(N^{2}-1)Vd} \frac{\delta _{\mu \nu }}{q^{2}}   \right)   \;.
\label{P0}
\end{eqnarray}
It is straightforward to compute the functional integration on the gauge field, since a
Gaussian integration, leading to the functional determinant of $\mathcal{P}_{\mu \nu }^{ab}$:
\begin{equation}
Z_{quad}  ~=~ \int {\frac{\d \beta}{2\pi i} }
e^{({\beta} -\ln\beta)} \left[\det\mathcal{P}^{ab}_{\mu\nu}\right]^{-\frac{1}{2}} \;.
\label{Zq2f00}
\end{equation}
This functional determinant may be exponentiated by making use of the relation
\begin{equation}
\left[ \det \mathcal{P}_{\mu\nu}^{ab} \right]^{-\frac{1}{2}} ~=~ e^{-\frac{1}{2} \ln \det
\mathcal{P}_{\mu\nu}^{ab}} ~=~ e^{-\frac{1}{2}Tr \ln \mathcal{P}_{\mu\nu}^{ab}} \;.
\label{functdeterminant}
\end{equation}
Thus, after taking the trace of $\ln\mathcal{P}_{\mu \nu }^{ab}$, whose technicalities is
detailed in \cite{Vandersickel:2012tz}, one may finally get
\begin{eqnarray}
Z_{quad} ~=~ \int_{- \ii \infty + \epsilon}^{+ \ii \infty + \epsilon}\frac{\d \beta}{2\pi \ii} \e^{f(\beta)} \;,
\label{ven}
\end{eqnarray}
with
\begin{eqnarray}
f(\beta) ~=~ \beta - \ln \beta  -\frac{(d-1)(N^2 - 1)}{2}  V \int \frac{\d^d q}{(2\pi)^d} \ln \left( q^2 + \frac{\beta N g^2 }{N^2 - 1} \frac{2}{d V}\frac{1}{q^2} \right)\;.
\label{minusfreenergy}
\end{eqnarray}
The factors $(N^{2}-1)$ and $(d-1)$ in front of the integral came from the trace over the
$SU(N)$ gauge group indices and from the trace over the Euclidean space-time indices,
respectively\footnote{ A careful computation of the functional trace of $\ln
\mathcal{P}_{\mu\nu}^{ab}$ can be found in \cite{Vandersickel:2012tz}}. Note that the factor
$(d-1)$ is obtained only after the Landau gauge limit is taken.

In the thermodynamic limit (when $V\to \infty$) the saddle-point approximation becomes exact,
and the integral \eqref{ven} can be easily computed, resulting in
\begin{eqnarray}
\e^{-V{\cal E}_{v}} ~=~  \e^{f(\beta^{\ast})} \;.
\label{relatvaccener}
\end{eqnarray}
One should notice now that the vacuum energy has effectively been computed up to first loop
order, since 
\begin{eqnarray}
{\cal E}_{v} ~=~ -\frac{1}{V}f(\beta^{\ast})\,,
\end{eqnarray}
within the thermodynamic limit. The saddle-point approximation, that becomes exact within the
thermodynamic limit, states that the integral \eqref{ven} equals the integrated function
evaluated at its maximum value. Thus, the stared parameter $\beta^{\ast}$ accounts for the
value of $\beta$ that maximizes the integrated function, which amounts to computing the Gribov
parameter gap equation,
\begin{eqnarray}
\left.
\frac{\partial f}{\partial \beta}\right|_{\beta = \beta^{\ast}}=0\;,
\label{gapeq}
\end{eqnarray}
which lead us to
\begin{eqnarray}
\frac{d-1}{d}N g^2    \int \frac{\d^d q}{(2\pi)^d}  \;   \frac{1}{ \left( q^4 + \frac{2\beta^{\ast} N g^2}{(N^2 - 1)dV} \right) }  ~=~ 1   \;.
\label{finalgapeq2}
\end{eqnarray}
Note that the Gribov parameter $\beta$, introduced to get rid of gauge ambiguities by
restricting the path integral to the first Gribov region $\Omega$, is not in fact a free
parameter of the theory. Otherwise, it is dynamically determined by its gap equation
\eqref{finalgapeq2}. Besides, it has dimension of $[mass]^{4}$ and is proportional to the
space-time volume $V$. Consequently, in the thermodynamic limit the logarithmic term of
equation \eqref{minusfreenergy} becomes zero, leading to the equation \eqref{finalgapeq2}.




















%--------------------------------------------------------
\subsection{The gauge propagator}
\label{A sign of confinement from gluon propagator}
%--------------------------------------------------------


In the present subsection we motivate that a possible sign of confinement could be read off
from the poles of the gluon propagator, putting this quantity at the centre of any further
discussion in the present work. At one-loop order only quadratic terms of the action
\eqref{fullgribovact} really matter, so that one can read off the two point function of the
gauge field from the inverse of the operator \eqref{P0}, setting $\xi \to 0$ at the very end of
the computation. Notice that the computation is performed within the thermodynamic limit, so
that the Gribov parameter must satisfy its gap equation. Namely,
\begin{equation}
\left\langle A_{\mu }^{a}(q)A_{\nu}^{b}(-q)\right\rangle ~=~ \frac{q^{2}}{q^{4}  +  \frac{2N
g^{2}\beta^{\ast} }{(N^{2}-1)dV} } \left( \delta _{\mu \nu }  -  \frac{q_{\mu }q_{\nu}}{q^{2}}
\right)\delta^{ab}  \;.
\label{Gribovprop0}
\end{equation}
Things become easier to analyze if we redefine the Gribov parameter as,
\begin{eqnarray}
\lambda^{4} = \frac{2\beta^{\ast} N g^2}{(N^2 - 1)dV}\;.
\end{eqnarray}
Consequently, the gauge propagator can be decomposed as,
\begin{eqnarray}
\left\langle A_{\mu }^{a}(q)A_{\nu}^{b}(-q)\right\rangle 
~=~ \frac{1}{2} \left( \frac{1}{q^{2}  +  i \lambda^{2} } +  \frac{1}{q^{2}  -  i \lambda^{2} }
\right)  \left( \delta _{\mu \nu }  -  \frac{q_{\mu }q_{\nu}}{q^{2}}  \right)\delta^{ab}\;.
\label{Gribovprop1}
\end{eqnarray}
Observe from \eqref{Gribovprop1} that the gluon propagator is suppressed in the infrared (IR)
regime, while displaying two complex conjugate poles, $m^{2}_{\pm} = \pm i\lambda^{2}$. That
feature does not allow us to attach the usual physical particle interpretation to the gluon
propagator, since such type of propagator is deprived of a spectral representation
\cite{Cucchieri:2007rg,Cucchieri:2008fc,Cucchieri:2011ig,Cucchieri:2004mf,Cucchieri:2014via}.
From the analytic point of view the gluon propagator \eqref{Gribovprop1} has not
a(n) (always) positive K\"{a}ll\'en-Lehmann spectral representation, which is necessary to
attach a probabilistic interpretation to the propagator\footnote{ See
\cite{Baulieu:2009ha,Sorella:2010it} and references therein for more details on the confinement
interpretation of gluons, $i$-particles and the existence of local composite operators, related
to these $i$-particles, displaying positive K\"{a}ll\'en-Lehmann spectral representation. For
lattice results pointing to the same confinement interpretation see
\cite{Cucchieri:2007rg,Cucchieri:2008fc,Cucchieri:2011ig,Cucchieri:2004mf,Cucchieri:2014via}.}.
These features lead us to interpret the gauge field as being confined.

As already mentioned in this thesis, our concept of confinement, throughout this work, will
always be concerned with the existence of a Gribov-kind of propagator. Particularly, not only
the gauge field will be susceptible to present such a Gribov-type propagator, but also the quark
field.















%------------------------------------------------------------------------------------
\section{A brief summary of the Gribov--Zwanziger framework}
\label{gzintro}
%------------------------------------------------------------------------------------


About ten years after Gribov's seminal paper has been published
\cite{Gribov:1977wm}, a generalization to the mechanism of getting rid of a leftover gauge
ambiguity after fixing the gauge was proposed by D. Zwanziger \cite{Zwanziger:1988jt}.
The main idea of his work is to take the trace of every positive eigenvalue of the
Faddeev-Popov operator,
\begin{eqnarray}
{\cal M} ~=~ - \p_{\mu}D_{\mu} ~=~ -\p_{\mu}\left(\p_{\mu} -igA_{\mu} \right)\,,
\end{eqnarray}
starting from the smallest eigenvalue. Regard that negative eigenvalues shall be avoided since
it is linked to the existence of gauge copies configurations --- and zero-modes ---, as was
presented in the previous section. Note that constant fields may also be eigenstates of the FP
operator in the Landau gauge related to zero eigenvalues. Since there is no gauge
configurations associated to {\it negative} eigenvalues  with constant eigenstates (the
constant fields), we will not consider such configurations.

Zwanziger did show that restricting the domain of integration of the gauge field to the first
Gribov region $\Omega$ is equivalent to take into account only gauge field configurations
that minimize the squared norm of the gauge field with respect to the gauge orbit
\cite{Zwanziger:1982na,Zwanziger:1988jt},
\begin{eqnarray}
 \left\Vert A\right\Vert^{2}_{min} ~=~ \min_{U\in SU(N)}\int d^{4}x \,
\left(A^{U}\right)^{2}\,.
\label{AAmin}
\end{eqnarray}
In other words, it means that the allegedly gauge physical configurations are those
that satisfy the (Landau) gauge condition and, furthermore, that minimizes the functional
\eqref{AAmin}. It should be clear that such minimized squared norm \eqref{AAmin} is, in 
fact, a gauge-invariant quantity, and that, at the same time, it is {\it nonlocal}
\footnote{The reader is pointed to a list of recent publications concerning the nonlocality of
such dimension 2 gauge field composite operator \eqref{AAmin},
\cite{Capri:2015pja,Capri:2015nzw,Capri:2016aqq,Fiorentini:2016rwx,Dudal:2006xd}.} (\emph{cf.}
\cite{Vandersickel:2012tz,Zwanziger:1982na,Zwanziger:1988jt,Zwanziger:1989mf,Zwanziger:1990tn,
Zwanziger:1992qr} and references therein).

For the sake of clarity, let us give once again the definition of the first Gribov region,
firstly introduced in Gribov's paper \cite{Gribov:1977wm} as
\begin{align}
\Omega \;= \; \{ A^a_{\mu}\;; \;\; \partial_\mu A^a_{\mu}=0\;; \;\; {\cal M}^{ab}=-(\partial^2
\delta^{ab} -g f^{abc}A^{c}_{\mu}\partial_{\mu})\; >0 \; \} \,.
\label{gr}
\end{align} 
Although we have already mentioned the important features of this region, let us state it
again, as a matter of completeness \cite{Vandersickel:2012tz,Zwanziger:1982na,Zwanziger:1988jt,Zwanziger:1989mf,Dell'Antonio:1989jn,Dell'Antonio:1991xt}:
\begin{itemize}
	\item[i.] $\Omega$  is convex and bounded in all direction in field space. Its
boundary, $\partial \Omega$, is the Gribov horizon, where the first vanishing eigenvalue of the
Faddeev-Popov operator shows up;
	\item[ii.] every gauge orbit crosses at least once the region $\Omega$. 
\end{itemize} 

In order to implement the restriction to this first Gribov's region, D. Zwanziger proposed an
all order procedure, by computing the FP operator's eigenvalue perturbatively, stating from the
lowest eigenvalue of the ``nonperturbative term'' of the FP operator,
\begin{eqnarray}
{\cal M}^{ab} ~=~ {\cal M}_{0}^{ab} + {\cal M}_{1}^{ab} ~=~ -\p^{2}\delta^{ab} +
gf^{abc}A^{c}_{\mu}\p_{\mu}\,.
\end{eqnarray}
Note that, it is straightforward to see that the lowest considered eigenvalue of the FP
operator must be always greater than zero. Thus, after perturbatively deriving the space of
positive eigenvalues, he took the trace over all of them obtaining, at the end, a positive
quantity, namely,
\begin{eqnarray}
dV(N^{2}-1) - H(A) ~>~ 0\,,
\label{positivitycondition}
\end{eqnarray}
where the functional $H(A)$ was identified with {\it horizon function} ({\it cf.}
\cite{Vandersickel:2012tz,Zwanziger:1982na,Zwanziger:1988jt,Zwanziger:1989mf}),
\begin{align}
H(A)  ~=~  {g^{2}}\int d^{4}x\;d^{4}y\; f^{abc}A_{\mu}^{b}(x)\left[ {\cal M}^{-1}\right]^{ad}
(x,y)f^{dec}A_{\mu}^{e}(y)   \;.
\label{hf2}
\end{align}
Therefore, the idea is to restrict the Yang-Mills path integral to the domain of integration of
the gauge field where the positivity condition \eqref{positivitycondition} is satisfied. It
amounts to make use of the following partition function, hereinafter called the
Gribov-Zwanziger partition function,
\begin{eqnarray}
Z_{GZ} ~=~ \int {\cal D} \Phi \theta (dV(N^{2}-1) - H(A))\e^{-S_{FP}}\,.
\label{gzpartition1}
\end{eqnarray}

The existence of such horizon function reflects the existence of a critical value for the
squared norm of the gauge field, $\left\Vert A \right\Vert_{c}$, beyond which the gauge
configuration corresponds to a negative eigenvalue of the FP operator.

The effect of the $\theta$-function into the Faddeev-Popov action will be derived in the
thermodynamic limit, where the $\theta$-function amounts to a $\delta$-function reflecting the
concept that in the limit $V\to \infty$ the volume of a $d$-dimensional sphere is directly
proportional to the surface of the border of this sphere. Thus, within the thermodynamic limit
the partition function \eqref{gzpartition1} can be rewritten as
\begin{eqnarray}
Z_{GZ} ~=~ \int {\cal D} \Phi \delta (dV(N^{2}-1) - H(A))\e^{-S_{FP}}\,.
\label{gzpartition2}
\end{eqnarray}
Finally, one  may use the same integral representation of the $\delta$-function, or even, may
use the equivalence between the microcanonical ensemble and the canonical ensemble in order to
obtain the GZ partition function,
\begin{equation}
 Z_{GZ} ~=~ \;
\int {\cal D}A\;{\cal D}c\;{\cal D}\bar{c} \; {\cal D} b \; e^{-\left[  S_{FP}+\gamma^4 H(A)
-V\gamma^4 4(N^2-1) \right]} \;, 
\label{zww2}
\end{equation}
The parameter $\gamma$ has the dimension of a mass
and is known as the Gribov parameter \footnote{Up to this point no relation exists between the
former Gribov parameter $\beta$ and the just derived $\gamma$ parameter. The authors of
\cite{Capri:2012wx} showed that the Gribov's mechanism amounts the Zwanziger's mechanism when
computed at first-order.}. It is not a free parameter of the theory; instead of
that, it is a dynamical quantity, being determined in a self-consistent way through a gap
equation called the \emph{horizon condition} \cite{Vandersickel:2012tz,Zwanziger:1982na,
Zwanziger:1988jt,Zwanziger:1989mf, Zwanziger:1990tn,Zwanziger:1992qr}, given by 
\begin{equation}
\left\langle H(A)   \right\rangle = 4V \left(  N^{2}-1\right) \;,   
\label{hc2}
\end{equation}
where the vacuum expectation value $\left\langle H(A)  \right\rangle$  has to be evaluated with
the measure defined in eq.\eqref{zww2}. The gap equation becomes exact due to the equivalence
between the microcanonical and canonical ensemble in the thermodynamic limit.

It is worth mentioning that most recently an all order proof has been published on the
equivalence between the Gribov's procedure and Zwanziger's one, \cite{Capri:2012wx}. Regarding
that the Gribov's approach relies on the perturbative expansion of the ghost propagator,
accounting up to the first non-null term of the expansion, and that the Zwanziger's one is an
all order computation of the FP operator's spectrum, the referred work computed the full
ghost propagator in perturbation theory, concluding at the end that both approaches are
equivalent at first order in perturbation theory.




\subsection{The local formulation of the Gribov-Zwanziger action}

Being able to construct a partition function for Yang-Mills theories that takes into account
the Gribov ambiguities, related to the gauge-fixing procedure, is a big achievement in the
direction of better comprehending the quantization procedure of non-Abelian fields. However,
equation \eqref{zww2} is not actually useful to compute physical quantities, not analytically
at least. The point is that one needs the action to be local in order to be able to compute
useful quantities, such as the two point function of the gauge field.

In this subsection we are going to present a localized version of the GZ action. Note, however,
that no details concerning its derivation will be provided, since such procedure has been
already extensively treated, \cite{Vandersickel:2012tz}. Rather, we will just mention the
mechanism with which one would obtain the same local expression.
%\footnote{For a pedagogical review concerning the derivation of the local-GZ action, the reader is pointed to }.

Although the horizon function $H(A)$ is a nonlocal quantity, it can be recast
in a local form by means of the introduction of a set of auxiliary fields
$(\bar{\omega}_\mu^{ab}, \omega_\mu^{ab}, \bar{\varphi}_\mu^{ab},\varphi_\mu^{ab})$, where
$(\bar{\varphi}_\mu^{ab},\varphi_\mu^{ab})$ are a pair of bosonic fields, and
$(\bar{\omega}_\mu^{ab}, \omega_\mu^{ab})$ are a pair of anti-commuting fields. It turns out
that the Gribov-Zwanziger partition function $Z_{GZ}$, in equation \eqref{zww2}, can be
rewritten as \cite{Vandersickel:2012tz,Zwanziger:1988jt,Zwanziger:1989mf,Zwanziger:1992qr}
\begin{equation}
 Z_{GZ} ~=~ \;
\int {\cal D}\phi \; e^{-S_{GZ}} \;, \label{lzww1}
\end{equation}
with $\phi$ accounting for every single field of the theory,
$\{A,\,c,\,\bar{c},\,b,\,\omega,\,\bar{\omega},\,\varphi,\,\bar{\varphi}\}$. The Faddeev-Popov
action $S_{GZ}$ is then given by the local expression 
\begin{equation} 
S_{GZ} = S_{YM} + S_{gf} + S_0+S_\gamma  \;, 
\label{sgz2}
\end{equation}
with
\begin{equation}
S_0 =\int d^{4}x \left( {\bar \varphi}^{ac}_{\mu} (\partial_\nu D^{ab}_{\nu} )
\varphi^{bc}_{\mu} - {\bar \omega}^{ac}_{\mu}  (\partial_\nu D^{ab}_{\nu} ) \omega^{bc}_{\mu}
- gf^{amb} (\partial_\nu  {\bar \omega}^{ac}_{\mu} ) (D^{mp}_{\nu}c^p) \varphi^{bc}_{\mu}
\right) \;, 
\label{s0}
\end{equation}
and 
\begin{equation}
S_\gamma =\; \gamma^{2} \int d^{4}x \left( gf^{abc}A^{a}_{\mu}(\varphi^{bc}_{\mu} + {\bar \varphi}^{bc}_{\mu})\right)-4 \gamma^4V (N^2-1)\;. 
\label{hfl}
\end{equation} 
Let us now make some comments on the terms of the above action and about the mechanism one
should follow to obtain such an action. We are not going to provide a step-by-step construction
of the localized action \eqref{sgz2}. Otherwise, we will provide a backward construction.
Note that the term $gf^{amb} (\partial_\nu  {\bar \omega}^{ac}_{\mu} ) (D^{mp}_{\nu}c^p)
\varphi^{bc}_{\mu} $ has no physical meaning, in the sense that it is not possible to construct
any Feynman diagram with entering $\bar{c}$ and $\omega$ fields, so that the vertex with
$\bar{\omega}$ and $c$ would influence. This term is introduced into the action by means of a
shift in the $\omega$ field with the aim of writing the action $S_{0}$ as an exact BRST
quantity. Namely,
\begin{eqnarray}
S_{0} ~=~ s\;\int d^{4}x \,\left( \bar{\omega}^{ac}_{\mu} \p_{\nu}D_{\nu}  \varphi^{bc}_{\mu}
\right)\,.
\label{S0}
\end{eqnarray}
To check that the contribution $S_{0}$, given in equation \eqref{S0}, to the GZ action is
indeed BRST exact, consider the following BRST transformation rule of the fields,
\begin{eqnarray}
\label{brst0}
sA^{a}_{\mu} &=& - D^{ab}_{\mu}c^{b}\;,\nonumber \\
s c^{a} &=& \frac{1}{2}gf^{abc}c^{b}c^{c} \;, \nonumber \\
s{\bar c}^{a} &=& b^{a}\;, \qquad \; \; 
sb^{a} = 0 \;, \nonumber \\
s{\bar \omega}^{ab}_\mu & = & {\bar \varphi}^{ab}_\mu \;, \qquad  s {\bar \varphi}^{ab}_\mu =0\;, \nonumber \\
s { \varphi}^{ab}_\mu&  = & {\omega}^{ab}_\mu  \;, \qquad s {\omega}^{ab}_\mu = 0 \;.
\end{eqnarray}
Therefore, an equivalent shift may be performed in order to remove the referred term.
After that, one should ends up with the expression
\begin{eqnarray}
\int d^{4}x \left[ 
{\bar \varphi}^{ac}_{\mu} (\partial_\nu D^{ab}_{\nu} )
\varphi^{bc}_{\mu} - {\bar \omega}^{ac}_{\mu}  (\partial_\nu D^{ab}_{\nu} ) \omega^{bc}_{\mu}
+\gamma^{2} gf^{abc}A^{a}_{\mu}(\varphi^{bc}_{\mu} + {\bar \varphi}^{bc}_{\mu})
\right]\;,
\label{act9}
\end{eqnarray}
where the $\omega$ field must be regarded as being the shifted one. The functional integration
of the fermionic fields $(\bar{\omega},\omega)$ can easily be computed, leading to 
$\det\left[ \p_{\nu}D_{\nu} \right]$. In order to integrate out the fields
$(\bar{\varphi},\varphi)$ one has to define the sources $\bar{J}$ and $J$ as
\begin{eqnarray}
\bar{J}^{bc}_{\mu} ~=~ J^{bc}_{\mu} ~=~ \gamma^{2}gf^{abc}A^{a}_{\mu} 
\,,
\label{sourcesJJ}
\end{eqnarray}
so that the integral \eqref{act9} may be rewritten as
\begin{eqnarray}
\int d^{4}x \left[ 
{\bar \varphi}^{ac}_{\mu} (\partial_\nu D^{ab}_{\nu} ) \varphi^{bc}_{\mu} +
\bar{J}^{bc}_{\mu}\,\varphi^{bc}_{\mu} + J^{bc}_{\mu}\bar{\varphi}^{bc}_{\mu}
\right]\;.
\end{eqnarray}
The couple of fields $(\bar{\omega},\omega)$ were already integrated in the above expression.
Summing and subtracting the term 
$\bar{J}^{bc}_{\mu}(\partial_\nu D^{ab}_{\nu})^{-1}J^{ac}_{\mu}$ we can rewrite this integral
as following,
\begin{eqnarray}
\int d^{4}x \left\{ \left[ 
{\bar \varphi}^{ac}_{\mu}  + (\partial_\nu D^{ab}_{\nu} )^{-1}\bar{J}^{bc}_{\mu} \right]
(\partial_\nu D^{ab}_{\nu} )
\left[ \varphi^{bc}_{\mu} + (\partial_\nu D^{ab}_{\nu})^{-1} J^{ac}_{\mu} \right] -
\bar{J}^{bc}_{\mu}(\partial_\nu D^{ab}_{\nu})^{-1}J^{ac}_{\mu}
\right\}
\,.
\end{eqnarray}
Performing the shifts $ {\bar\varphi}^{ac}_{\mu}  + (\partial_\nu D^{ab}_{\nu}
)^{-1}\bar{J}^{bc}_{\mu} \to \bar{\varphi}^{\prime\, ac}_{\mu}$ and
$ \varphi^{bc}_{\mu} + (\partial_\nu D^{ab}_{\nu})^{-1} J^{ac}_{\mu} \to \varphi^{\prime\,
bc}_{\mu}$ one ends up with
\begin{eqnarray}
\int d^{4}x \left[ 
{\bar \varphi}^{\prime\, ac}_{\mu} (\partial_\nu D^{ab}_{\nu} ) \varphi^{\prime\, bc}_{\mu}  
- \bar{J}^{bc}_{\mu}(\partial_\nu D^{ab}_{\nu} )^{-1}J^{ac}_{\mu}
\right]\;.
\label{act10}
\end{eqnarray}
After all, one ends up with a Gaussian integration of the bosonic fields
$(\bar{\varphi}^{\prime},\varphi^{\prime})$, whose integral leads to $\det\left[ \p_{\nu}D_{\nu} \right]^{-1}$,
and with the horizon function in its nonlocal version. Therefore, in order to obtain the
localized version of the GZ action, starting from the GZ action of equation
\eqref{gzpartition2}, one should perform the process just described in the backward direction.


\subsubsection{The gap equation, or horizon condition:}


Back to the local formulation of the Gribov-Zwanziger action, the horizon condition \eqref{hc2}
takes the simpler form 
\begin{equation}
 \frac{\partial \mathcal{E}_v}{\partial\gamma^2}=0\;,   
\label{ggap}
\end{equation}
where $\mathcal{E}_{v}(\gamma)$ is the vacuum energy defined by,
\begin{equation}
 e^{-V\mathcal{E}_{v}}=\;Z_{GZ}\;  
\label{vce2} \;.
\end{equation}
The local action $S_{GZ}$ in eq.\eqref{sgz2} is known as the Gribov-Zwanziger action. It has
been shown to be renormalizable to all orders
\cite{Zwanziger:1988jt,Zwanziger:1989mf,Zwanziger:1992qr}. 


\subsubsection{The gauge propagator:}

Finally, with the local version of the generating functional, the gluon and ghost propagator
could be computed. At first order in loop expansion, only quadratic terms in the fields of the
GZ action eq.\eqref{sgz2} have to be kept, while terms of great order will be ignored.
Therefore, performing the same step-by-step of the previous section, one would be able to
compute the gauge and ghost propagators, ending up with
\begin{eqnarray} 
\langle  A^a_\mu(k)  A^b_\nu(-k) \rangle  ~=~  \frac{k^2}{k^4 + 2Ng^2\gamma^4} \,\, \delta^{ab}
\left(\delta_{\mu\nu} - \frac{k_\mu k_\nu}{k^2}     \right)   \;, 
\label{glrgz1}
\end{eqnarray} 
for the gluon field, and 
\begin{eqnarray}
\left\langle  c^{a}(k) \overline{c}^{b}(-k)  \right\rangle ~=~ \mathcal G (k^2)_{ab}\;,
\end{eqnarray}
with
\begin{eqnarray}
\mathcal G (k^2) &=&
\frac{1}{k^2} + \frac{1}{V}\frac{1}{k^4} \frac{N g^2}{N^2 - 1} \int\frac{ \d^d q}{(2
\pi)^d} \frac{q^2}{q^4 + 2Ng^2\gamma^4} \left(\delta_{\mu\nu} - \frac{q_\mu q_\nu}{q^2}\right)
 \frac{(k-q)_\mu  k_\nu}{(k-q)^2} \nonumber\\
&=& \frac{1}{k^2}\left( 1 + \sigma(k) \right)\;,
\label{coloraway1}
\end{eqnarray}
for the ghost fields. Let us make, at this point, a brief analysis of \eqref{glrgz1} and
\eqref{coloraway1} in the IR regime. It is not difficult to see that in the deep IR regime the
gauge field propagator is strongly suppressed and tends to zero in the limit $k^2 \to 0$. As
can be checked in \eqref{Gribovprop0}, this behavior is shared by Gribov and Gribov-Zwanziger
approaches. 



\subsubsection{The ghost propagator:}

Since we are performing a perturbative computation up to one-loop order, one must follow the
same step-by-step of the previous chapter in order to compute the ghost form factor. Therefore,
one should ends up with
\begin{eqnarray}
\sigma(k) ~=~ \frac{1}{V}\frac{1}{k^2} \frac{N g^2}{N^2 - 1} \int\frac{ \d^d q}{(2
\pi)^d} \frac{q^2}{q^4 + 2Ng^2\gamma^4} \left(\delta_{\mu\nu} - \frac{q_\mu q_\nu}{q^2}\right)
 \frac{(k-q)_\mu  k_\nu}{(k-q)^2}\,.
\end{eqnarray}
Note that the term linear in $q_{\mu}$ is zero, due to the transversal projector. Making use
of the identity $\int d^{d}q f(q)q_{\mu}q_{\nu}/q^{2} = 1/d\,\int d^{d}q f(q)\delta_{\mu\nu}$,
one ends up with,
\begin{eqnarray}
\sigma(k) ~=~ \frac{1}{V} \frac{N g^2 (d-1)}{d(N^2 - 1)} \int\frac{
\d^d q}{(2 \pi)^d} \frac{q^2}{q^4 + 2Ng^2\gamma^4} \frac{1}{(k-q)^2}\,.
\end{eqnarray}
Taking the limit $k \to 0$, we have
\begin{eqnarray}
\sigma(0) ~=~ \frac{1}{V} \frac{N g^2 (d-1)}{d(N^2 - 1)} \int\frac{
\d^d q}{(2 \pi)^d} \frac{1}{q^4 + 2Ng^2\gamma^4}\,,
\end{eqnarray}
which is a divergent integral and the ghost propagator is enhanced, just as in the Gribov
approach. However, before effectively solving the integral one has to fix the
Gribov parameter $\gamma^{2}$ dynamically, through the horizon condition \eqref{hc2} computed
with the quadratic generating functional. 






\section{A brief introduction to the refined version of GZ}

Recently, a refinement of the Gribov-Zwanziger action has been worked out by the authors
\cite{Dudal:2007cw,Dudal:2008sp,Dudal:2011gd,Dudal:2008rm}, by taking into account the existence of certain
dimension two condensates\footnote{See \cite{Gracey:2010cg,Thelan:2014mza} for a recent
detailed investigation on  the structure of these condensates in color space.}.  The Refined
Gribov-Zwanziger (RGZ) action reads \cite{Dudal:2007cw,Dudal:2008sp,Dudal:2011gd,Dudal:2008rm}
\begin{equation}
S_{RGZ} = S_{GZ} + \int d^4x \left(  \frac{m^2}{2} A^a_\mu A^a_\mu  - \mu^2 \left( {\bar
\varphi}^{ab}_{\mu}  { \varphi}^{ab}_{\mu} -  {\bar \omega}^{ab}_{\mu}  { \omega}^{ab}_{\mu}
\right)   \right)  \;,  
\label{rgz}
\end{equation}
where $S_{GZ}$ stands for the Gribov-Zwanziger action,  eq.\eqref{sgz2}.  As much as the Gribov
parameter $\gamma^2$, the massive parameters $(m^2, \mu^2)$ have a dynamical origin, being
related to the existence of the dimension two condensates $\langle A^a_\mu A^a_\mu \rangle$ and
$\langle {\bar \varphi}^{ab}_{\mu}  { \varphi}^{ab}_{\mu} -  {\bar \omega}^{ab}_{\mu}  {
\omega}^{ab}_{\mu}  \rangle$, \cite{Dudal:2007cw,Dudal:2008sp,Dudal:2011gd,Dudal:2008rm}. 
The gluon propagator obtained from the RGZ action turns out to be suppressed in the infrared region, attaining a non-vanishing value at zero momentum, $k^2=0$, {\it i.e.}
\begin{eqnarray} 
\langle  A^a_\mu(k)  A^b_\nu(-k) \rangle  & = &  \delta^{ab}  \left(\delta_{\mu\nu} - \frac{k_\mu k_\nu}{k^2}     \right)   {\cal D}(k^2) \;, \label{glrgz} \\
{\cal D}(k^2) & = & \frac{k^2 +\mu^2}{k^4 + (\mu^2+m^2)k^2 + 2Ng^2\gamma^4 + \mu^2 m^2}  \;. \label{Dg}
\end{eqnarray} 
One should note that the gluon propagator obtained in the Gribov-Zwanziger approach differ from
the refined one by the terms proportional to $\mu^{2}$ and $m^{2}$. So, putting these
parameters to zero the GZ gluon propagator is recovered, with the well known suppressed
behavior in the IR regime, going to zero for $k^{2} \to 0$.
Also, unlike the case of the GZ action, the ghost propagator stemming from the Refined theory is not enhanced in the deep infrared:
\begin{equation}
{\cal G}^{ab}(k^2) = \langle  {\bar c}^{a} (k)  c^b(-k) \rangle \Big|_{k\sim 0} \; \sim \frac{\delta^{ab}}{k^2}   \;.\label{ghrgz} 
\end{equation}
The infrared behaviour of the  gluon and ghost propagators obtained from the RGZ  action turns
out to be in very good agreement with the most recent  numerical lattice simulations on large
lattices \cite{Cucchieri:2007rg,Cucchieri:2008fc,Cucchieri:2011ig}. Moreover, the numerical
estimates  \cite{Cucchieri:2011ig}  of the parameters $(m^2,\mu^2,\gamma^2)$ show that the RGZ
gluon propagator \eqref{glrgz} exhibits complex poles and violates  reflection positivity. This
kind of two-point function lacks the  K{\"a}ll{\'e}n-Lehmann spectral representation and cannot
be associated with the propagation of physical particles. Rather, it indicates that, in the
nonperturbative infrared region, gluons are not physical excitations of the spectrum of the
theory, {\it i.e.} they are confined.  It is worth mentioning here that the RGZ gluon
propagator has been employed in analytic calculation of the first glueball states
\cite{Dudal:2010cd,Dudal:2013wja}, yielding results which compare well with the available
numerical simulations as well as with other approaches, see \cite{Mathieu:2008me} for an
account on this topic. The RGZ gluon propagator has also been used in order to  study the
Casimir energy within the MIT bag model \cite{Canfora:2013zna}. The resulting energy has the
correct expected confining behaviour. Applications  of the RGZ theory at finite temperature can
be found in  \cite{Fukushima:2013xsa,Canfora:2013kma}. 










\section{The BRST breaking}

One important aspect of both GZ and RGZ theories is that they exhibit a soft breaking of the
BRST symmetry. Indeed, it has been extensively studied that the breaking of the BRST symmetry
is intimately connected with the restriction of the domain of integration of the gauge field to
the region inside the Gribov horizon
\cite{Baulieu:2008fy,Zwanziger:2009je,Sorella:2009vt,Zwanziger:1993dh,vonSmekal:2008en,Dudal:2010hj,Dudal:2009xh,Sorella:2010it,Capri:2010hb,Dudal:2012sb,Reshetnyak:2013bga}.


In fact, considering either the GZ action \eqref{sgz2} or the RGZ action \eqref{rgz}, one
should be able to prove that the BRST variation of both of these actions is not zero, but
rather it equals an integrated polynomial of order smaller than $4$ ({\it i.e.} the space-time
dimension) and proportional to $\gamma^{2}$ \cite{Dudal:2007cw,Dudal:2008sp,Dudal:2011gd,Dudal:2008rm}.
Namely, 
\begin{equation}
s S_{GZ} ~=~ s S_{RGZ} ~=~ \gamma^2 \Delta  \;, \label{brstbrr}
\end{equation}
with
\begin{equation}
\Delta = \int d^{4}x \left( - gf^{abc} (D_\mu^{am}c^m) (\varphi^{bc}_{\mu} + {\bar \varphi}^{bc}_{\mu})   + g f^{abc}A^a_\mu \omega^{bc}_\mu            \right)  \;. \label{brstb1}
\end{equation}
To check the above statement, one has to consider the BRST variation rule of each field as the
one given in \eqref{brst0}.
%
%\begin{eqnarray}
%\label{brst1}
%sA^{a}_{\mu} &=& - D^{ab}_{\mu}c^{b}\;,\nonumber \\
%s c^{a} &=& \frac{1}{2}gf^{abc}c^{b}c^{c} \;, \nonumber \\
%s{\bar c}^{a} &=& b^{a}\;, \qquad \; \; 
%sb^{a} = 0 \;, \nonumber \\
%s{\bar \omega}^{ab}_\mu & = & {\bar \varphi}^{ab}_\mu \;, \qquad  s {\bar \varphi}^{ab}_\mu =0\;, \nonumber \\
%s { \varphi}^{ab}_\mu&  = & {\omega}^{ab}_\mu  \;, \qquad s {\omega}^{ab}_\mu = 0 \;, 
%\end{eqnarray}
%it is immediately checked that the Gribov-Zwanziger action breaks the BRST symmetry, as summarized by the equation\footnote{A similar equation holds in the case of the RGZ action .}
%

Notice that the breaking term $\Delta$ is of dimension two in the fields and, as such, is said
to be a soft breaking. 
Equation \eqref{brstbrr} can be translated into a
set of softly broken Slavnov-Taylor  identities which ensure the all order renormalizability of
both GZ and RGZ actions. The presence of the soft breaking term $\Delta$ turns out to be
necessary in order to have a confining gluon propagator which attains a non-vanishing value at
zero momentum, eqs.\eqref{glrgz},\eqref{Dg}, in agreement with the lattice data
\cite{Cucchieri:2007rg,Cucchieri:2008fc,Cucchieri:2011ig}. It is worth underlining that this
property is deeply related to the soft breaking of the BRST symmetry. In fact, the
non-vanishing of the propagator at zero momentum relies on  the parameter $\mu^2$, which
reflects the existence of the   BRST-exact dimension-two condensate
\cite{Dudal:2007cw,Dudal:2008sp,Dudal:2011gd,Dudal:2008rm}.
Recently, the breaking of the BRST symmetry in the IR regime was firstly observed on the
lattice, as can be checked in \cite{Cucchieri:2014via}, by making use of the possibility of
fixing the (minimal) Landau gauge on the lattice. To that end, the authors investigated if the
so-called Bose-ghost propagator, at zero temperature, is zero or not. Such Bose-ghost
propagator can be read as
\begin{eqnarray}
{\cal Q}^{abcd}_{\;\;\mu\nu}(x,y) &=& \left\langle \omega^{ab}_{\mu}\bar{\omega}^{cd}_{\nu} +
\varphi^{ab}_{\mu} \bar{\varphi}^{cd}_{\nu}  \right\rangle 
\nonumber \\
&=&
\left\langle s \left( \varphi^{ab}_{\mu} \bar{\omega}^{cd}_{\nu} \right)  \right\rangle 
\;.
\label{boseghostprop}
\end{eqnarray}
Note that this is a BRST exact quantity and as
such shall be zero for a BRST symmetric theory. Otherwise, if the Bose-ghost propagator is not
zero, then it is an evidence that the BRST symmetry is broken. This Bose-ghost propagator has
been proposed as a carrier of long-range confining force in the minimal Landau gauge
\cite{Zwanziger:2009je}. In order to investigate the Bose-ghost propagator the authors of
\cite{Cucchieri:2014via} noticed that the quantity \eqref{boseghostprop} may be written as
\begin{eqnarray}
{\cal Q}^{abcd}_{\;\;\mu\nu}(x,y) ~=~  \left\langle {\cal R}^{ab}_{\;\;\mu}
{\cal R}^{cd}_{\;\;\nu}
\right\rangle \;,
\end{eqnarray}
where
\begin{eqnarray}
{\cal R}^{ab}_{\;\;\mu} ~=~ -g\int d^{4}z\; \left( {\cal M}^{-1}
\right)^{ab}f^{abc}A^{c}_{\mu}\;.
\end{eqnarray}
Such quantity may be accessed by taking the inverse of the FP operator for the gauge propagator
within the Gribov restriction. One must be careful to interpret these results: there is no
consistent proof of the equivalence of the minimal Landau gauge on the lattice and the usual
analytical Landau gauge, so far.













%%%%%%%%%%%%%%%%%%%%%%%%%%%%%%%%%%%%%%%%%%%%%%%%%%%%%%%%%%%%%%%%%%%%
