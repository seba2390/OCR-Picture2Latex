

%%%%%%%%%%%%%%%%%%%%%%%%%%%%%%%%%%%%%
\chapter{Final words
\label{finalwords}}
%%%%%%%%%%%%%%%%%%%%%%%%%%%%%%%%%%%%%


The present thesis was devoted to the study of aspects of the Gribov problem in Euclidean
Yang-Mills theories coupled to matter field.  We presented some evidences that point to the
existence of a possible interplay between the gauge sector and the
matter sector, in regimes of sufficiently low energy (known as the infrared (IR) regime). Our
work is analytic, based on the quantization procedure of Gribov where (infinitesimal) gauge
fixing ambiguities are taken into account by getting rid of nonzero modes of the Faddeev-Popov
operator. This framework was briefly introduced in the second chapter, where some fundamental
concepts were discussed, and some important quantities were derived. We presented that, to get
rid of such gauge ambiguities a nonlocal term must be added to the Faddeev-Popov action, which
leads to two important consequences: the soft BRST breaking and a deeply modified gauge
field propagator, whose expression displays complex conjugate poles. Such behaviour
of the gauge field may be interpreted as a sign of confinement, providing us an alternative
tool to analytically investigate gauge confinement.

Our study of Yang-Mills theories coupled to matter field started with the analysis of
Yang-Mills models coupled to the Higgs field, in Euclidean space-time with dimensionality $d=3$
and $d=4$. Two different representations of the scalar field was considered, the fundamental
one, which is an example of a nontrivial representation, and the adjoint representation, which
is an example of a trivial one. The scalar Higgs field was considered to be frozen in its
vacuum configuration. Our analysis concern the direct observation of the
propagator of the gauge field: if it presents or not complex conjugate poles, or negative
residues. We use to say that the gauge propagator is of the Gribov-type when it presents
\emph{cc} poles, and the Gribov's alternative confinement criterion applies. We could generally
notice that the representation of the scalar field is of great importance to the analysis.

The point we would like to emphasise here is that our work shares remarkable similarities with
the seminal paper of Fradkin \& Shenker \cite{Fradkin:1978dv} and others lattice works
\cite{Nadkarni:1989na,Hart:1996ac}, despite the existence of fundamental differences between
them. The authors of \cite{Fradkin:1978dv} made use of the lattice formalism to investigate the
structure of gauge theories coupled to the Higgs field. In particular, the Wilson loop was
measured, as a suitable order parameter of (static quark) confinement, in mainly two different
scenarios, for the scalar field in its fundamental configuration, and for the scalar field in
its adjoint representation. In the other side, our work was made in the continuum space-time,
by means of the Gribov effect model, to probe for gauge field confinement in Yang-Mills + Higgs
theories.

Therefore, when we say that both works share some similarities, we do not mean we are measuring
the same thing, or else, that we are obtaining the same results. But rather, we mean that the
structure of the confinement spectrum of the gauge field, in the light of Gribov's approach, is
quite similar to the structure of (static) quark confinement, \`a la Wilson loop. Namely, for
the Higgs field in the fundamental representation we could find: 1)
the existence of two distinct regimes, the Higgs-like and the confined-like, corresponding to
weak and strong coupling regions, respectively; 2) these two detected regimes, Higgs- and
confined-like, were found to be continuously connected, in the sense that the parameters of the
theory are allowed to continuously vary from one region to another, without leading to any
discontinuity or singularity of the vacuum energy or the two-point Green function. The
similarity with Fradkin \& Shenker indeed exists, although we could not properly talk about the
existence of an \emph{analyticity region} in our work. Besides being a perturbative work, we
could also find out a region in our model where perturbation theory is not trustworthy anymore.
Such unreliable region lies in between the Higgs- and confined-regions and prevents us from
proving the existence of an analyticity region.

Similarities are still present in the adjoint Higgs field case: 1) in both works, the
connection between the two distinct regimes is not a smooth connection anymore. In our work we
could find a point of discontinuity in the vacuum energy, although our perturbative
computations are not reliable at precisely this point. Perhaps, this feature may be a sign, in
the gauge sector, of the phase transition associated with the breaking of the center symmetry,
since this kind of phase transition is only possible for the matter field in the adjoint
representation (or in its absence);
 2) we could also find the existence of a third regime, besides the confined- and Higgs-like
ones. We could detect a kind of $U(1)$ confined-like regime, where the third component of the
gauge field has a propagator of the Gribov-type and the off-diagonal sector is massive.
Interestingly, something similar to this has been already detected on lattice studies of the
three-dimensional Georgi-Glashow model \cite{Nadkarni:1989na,Hart:1996ac}.

Finally, we may conclude that, by means of the Gribov's approach to the quantization of the
gauge field, leading to an alternative criterion of gauge field confinement, the structure of
the Yang-Mills + Higgs field's spectrum shares some resemblances with what is find out by works
on the lattice, where order parameters of (static quark) confinement is measured, such as the
Wilson loop. Thus, we obtained our first indicative sign that IR features of the gauge sector
may be reflected, in some sense, in the IR behaviour of the matter sector.


Subsequently, we proposed an effective model to the matter sector, inspired by the
Gribov-Zwanziger structure of the gauge sector that leads to the confinement interpretation of
the gauge field, where a kind of horizon-function is consistently plugged to the matter field.
By consistently we mean that we could show that such implementation does not lead to any new UV
divergences, but only to those already present in the standard, noneffective, Yang-Mills +
matter theory. Such UV safety was order by order analysed in the fifth chapter, by means of a
careful analysis of the Feynman diagrams. An all order renormalizability proof has been
provided in the Appendix, together with the example of the ${\cal N}=1$ supersymmetric
Yang-Mills model within the Gribov horizon.

The matter field was considered in its adjoint representation, so that we could show that it is
possible to construct an operator ${\cal R}^{ai}_{\;\;F}$ for matter field,
\begin{eqnarray}
{\cal R}^{ai}_{\;\;F}(x)  &=&  g \int d^4z\;  ({\cal M}^{-1})^{ab} (x,z)   \;(T^b)^{ij}
\;F^{j}(z)  
\,,
\end{eqnarray}
in analogy with the gauge field restricted to the first Gribov region.
Furthermore, we could also show that the correlation function $\langle{\cal R}_{\;\;F}{\cal
R}_{\;\;F}\rangle$ is nonvanishing and, from the available lattice data, seems to behave like
the Boson-ghost propagator in the IR regime, \eqref{RRgluon}, namely
\begin{eqnarray}
\langle \tilde {\cal R}^{ai}_{\;\;F}(k)\tilde {\cal R}^{bj}_{\;\;F}(-k)\rangle&\stackrel{k\to
0}{\sim}&\frac{1}{k^4} 
\,.
\end{eqnarray}
After constructing such effective nonlocal model to the matter sector we could show that, the
nonvanishing of $\langle{\cal R}_{\;\;F}{\cal R}_{\;\;F}\rangle$ indicates the soft breaking
of the BRST symmetry in the matter sector, since the \emph{vev} of ${\cal R}_{\;\;F}{\cal
R}_{\;\;F}$ can be written in terms the \emph{vev} of a BRST exact local operator. Therefore,
the correlation function $\langle{\cal R}_{\;\;F}{\cal R}_{\;\;F}\rangle$ could be regarded as
a direct signature for BRST breaking, being accessible both analytically as well as through
numerical lattice simulations.

Another important outcome of this effective model is that, by fitting our effective matter
propagator to the most recent lattice data we could find that, in this scenario, the matter
field is deprived of an asymptotic physical particle interpretation, since its propagator
displays positivity violation, so not satisfying every reality condition of
Osterwalder-Schrader (just as the gauge field). In this sense the adjoint scalar
propagators  consistently represent confined degrees of freedom, that do not exhibit a physical
propagating pole. We could also qualitatively show that our results are renormalization scheme
independent. A more quantitative analysis would require further simulations with improved
statistics and even larger lattices.



Subsequently, we studied the Gribov-Zwanziger (GZ) action for $SU(2)$ gauge theories
with the Polyakov loop coupled to it via the background field formalism. Doing so, we were able
to compute simultaneously the finite temperature value of the Polyakov loop and of the
Gribov mass parameter, up to the leading order. Within this formalism we could confirm the
existence of a second order deconfinement phase transition of static quarks. Besides, we could
also observe that the GZ mass parameter evidently feels the effects of quark confinement: 
such a mass parameter develops a cusp-like behaviour precisely at the critical temperature of
quark confinement, probed by the Polyakov loop parameter. It is perhaps worthwhile to stress
here that at temperatures above $T_{c}$, the Gribov mass is nonzero, indicating that the gluon
propagator still violates positivity and as such it rather describes a quasi- than a
\textquotedblleft free\textquotedblright\ observable particle. It would means that the gauge
sector is, indeed, sensible to IR effects of the matter sector. In a sense, it may corroborate
our feelings that the IR structure of the gauge sector may be transferred to the matter sector.

Besides that, we could also find that our model is plagued by the existence of an instability
region in the vicinity of the critical temperature. To investigate such region we computed the
pressure and the trace anomaly, so that we could find out a region of negative pressure. It is
worthwhile to emphasise that those results were obtained in the GZ formalism. The RGZ formalism
was adopted with the status of a first computation, while a full finite-temperature approach of
the RGZ in the presence of the Polyakov loop is currently being carried out. The main outcome
of the RGZ formalism is that the instability region is considerably smaller, possibly because
the RGZ framework provides an adequate description of the zero-temperature gauge dynamics. One
should also keep in mind that we are dealing with a perturbative effective theory, so that
higher loop order computations, within the RGZ framework, should be carried out in order to
make any reliable assertion about the existence of instability regions.



Finally, we close this thesis stating that a lot of work still has to be done in the direction
of a deeper understanding of nonperturbative effects of QCD. Precisely, the concept of gauge
confinement is not as clear as the one of quark confinement, regarding that a definite
understanding of confinement is lacking, at all. A step towards the reconciliation of Gribov's
mechanism and BRST breaking has been made, so that the possibility to define physical states
even in the (R)GZ framework still exists \cite{Capri:2015ixa,Capri:2015nzw,Capri:2016aqq}.
Concomitantly, the same construction of a nonlocal effective model of the matter field that
still keeps the action invariant under a nonperturbative BRST transformation has been worked
out \cite{Capri:2016aqq}. Equally, there are currently efforts of carrying on investigations 
on finite temperature Yang-Mills theory, within the RGZ framework, in order to better
understand our earlier results. It is becoming clear for us that a nonlocal horizon-like term
in the matter sector sufficiently describes some IR features of this sector. However,
fundamental arguments that would justify the existence of that horizon-like term in
matter are still lacking. We hope that further work would point us to the right answer.













