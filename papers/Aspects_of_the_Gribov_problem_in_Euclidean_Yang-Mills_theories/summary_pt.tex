


%-----------------------------------
\chapter*{Resumo}
\addcontentsline{toc}{chapter}{Portuguese Summary}
%-----------------------------------



A presente tese \'e baseada nos trabalhos
\cite{Capri:2013oja,Capri:2013gha,Capri:2012ah,Capri:2012cr,Capri:2014jqa,Canfora:2015yia,Capri:2015mna}
e destina-se ao estudo dos aspectos do problema de Gribov em teorias de Yang-Mills acoplada ao
campo de mat\'eria. Aqui apresentaremos algumas evid\^encias, matem\'aticas e f\'isicas, que
apontam para a exist\^encia de uma poss\'ivel influ\^encia entre os setores de calibre e de
mat\'eria, em regimes de baixa energia, conhecido como o regime infra-vermelho. Em outras
palavras, defenderemos que um certo efeito f\'isico no setor de bosons vetoriais da Natureza,
devido a intera\c{c}\~oes fortes na regi\~ao de baixas energias (cf. o Modelo Padr\~ao), pode
ser refletido no setor de mat\'eria, no mesmo regime de energia. 

%Especificamente, proporemos
%que o horizonte de Gribov, do setor de calibre da teoria, pode ser sentido pelo setor de
%mat\'eria, e que tal efeito pode ser descrito por uma teoria efetiva, na qual um termo massivo
%n\~ao-local seria introduzido ao campo de mat\'eria. 

%Este termo parece ser gerado dinamicamente
%pela teoria e descreve efeitos n\~ao-perturbativos do campo de mat\'eria.

Este trabalho est\'a dividido em tr\^es partes: na primeira parte, que compreende o cap\'itulo
\ref{The Yang-Mills $+$ Higgs field theory}, apresentaremos um estudo an\'itico das teorias
de Yang-Mills acopladas ao campo escalar de Higg, de acordo com o esquema de quantiza\c{c}\~ao
de Gribov. Neste esquema, proposto por Gribov para eliminar copias infinitesimais do campo de
calibre, o propagador do campo de calibre sofre profundas modifica\c{c}\~oes, apresentando
polos complexos conjugados, que o levam a violar a condi\c{c}\~ao de positividade, ou
condi\c{c}\~ao de realidade de Osterwalder-Schrader, n\~ao permitindo-nos interpret\'a-lo como
part\'icula f\'isica do espectro da teoria. Guardando as devidas diferen\c{c}as e
peculiaridades de cada modelo, faremos um paralelo entre o nosso modelo e os resultados obtidos
por Fradkin-Shenker no trabalho \cite{Fradkin:1978dv}.

Na segunda parte desta tese proporemos um modelo efetivo ao setor de mat\'eria no qual um termo
n\~ao-local ser\'a introduzido, provocando uma quebra suave da simetria BRST, equivalente \`a
fun\c{c}\~ao horizonte presente no formalismo de Gribov-Zwanziger no setor de calibre.
Mostraremos que a implementa\c{c}\~ao deste termo n\~ao-local, que provoca uma quebra suave da
simetria BRST, pode ser feita de maneira consistente. Como consequ\^encia, observamos que este
modelo se ajusta aos resultados mais recentes obtidos pelo m\'etodo da ``QCD na rede'', de
forma que os propagadores da mat\'eria violam a condi\c{c}\~ao de positividade. Al\'em disso,
mostraremos que este modelo efetivo de confinamento do campo de mat\'eria n\~ao acarreta em
novas diverg\^encias no regime ultravioleta, mas somente gera as diverg\^encias usuais do
modelo ``n\~ao-efetivo''. Esta segunda parte est\'a compreendida nos cap\'itulos
\ref{brstonmatter} e \ref{UVpropsofconfiningprop}. Algumas demonstra\c{c}\~oes est\~ao expostas
no Ap\^endice \ref{ARscalaraction} e \ref{algrenorm}.

A terceira e \'ultima parte desta tese, referente ao cap\'itulo \ref{Ploop1}, destina-se ao
ao estudo \`a temperatura finita do confinamento de \emph{quarks} est\'aticos, na teoria de calibre
n\~ao-Abeliana $SU(2)$, de acordo com o arcabou\c{c}o te\'orico de Gribov-Zwanziger. O estudo
ser\'a feito por meio do \emph{loop de Polyakov}, que ser\'a introduzido utilizando-se o
m\'etodo de campos-de-fundo, de forma que o confinamento dos \emph{quarks} est\'aticos ser\'a
analisado por apenas um \'unico par\^ametro do modelo. Estudaremos os efeitos deste
par\^ametro de ordem, associado ao \emph{loop de Polyakov}, sobre o par\^ametro de Gribov, e
para isso analisaremos o modelo proposto em v\'arias situa\c{c}\~oes distintas. Como
consequ\^encia, observamos que o par\^ametro de Gribov \'e claramente sens\'ivel \`a
transi\c{c}\~ao de fase dos \emph{quarks} est\'aticos, e que h\'a uma regi\~ao de instabilidade
nas visinhan\c{c}as da temperatura c\'itica. 
%Uma breve an\'alise \'e feita com o modelo refinado de Gribov-Zwanziger.
