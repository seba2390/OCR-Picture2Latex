%--------------------------------------------------------------------------------
\chapter{The UV safety of any Gribov-like confined theory}
\chaptermark{The UV safety of any Gribov-like theory}
\label{UVpropsofconfiningprop}
%--------------------------------------------------------------------------------

In the present Chapter we are going to continue the analysis started in the previous Chapter,
concerning general effective models presenting non-local terms in the action \emph{\`a la}
Gribov. Precisely, here we present some interesting observations about the UV behavior of such
models, guided by the already known UV safety of Yang-Mills models within the Gribov horizon,
and prove to all orders the renormalizability of such general models.

The feature that we want to explore is the fact that both the GZ and the RGZ
tree-level propagators hold the key for the good UV behavior of the theory. More precisely,
notice that the gluon propagator ${\cal D}(k^2)$ \eqref{Dg} can be rewritten as a sum of its UV
perturbative term plus an effective non-perturbative contribution,
 \begin{eqnarray} 
{\cal D}(k^2) & = & \frac{k^2 +\mu^2}{k^4 + (\mu^2+m^2)k^2 + 2Ng^2\gamma^4 + \mu^2 m^2}  \;.\nonumber\\ 
& = &  \frac{1}{k^2 +  m^2} - \frac{2Ng^2\gamma^4}{\left(k^2 +  m^2\right)\left(k^2 +  M_{+}^2\right)\left(k^2 +  M_{-}^2\right)}
\label{Dg2}
\end{eqnarray}
where
\begin{eqnarray} 
M^2_{\pm}= \frac{\mu^2 + m^2}{2} \pm \frac 12 \sqrt{\left(\mu^2 + m^2 \right)^2 -
8Ng^2\gamma^4} \,.
\label{masses1}
\end{eqnarray}
The first term in \eqref{Dg2} represents the usual propagator of a massive vector boson. The
second term is the contribution coming from the restriction to the Gribov region. Notice the
negative sign that points to an unphysical contribution that violates positivity requirements.
The important feature we want to emphasize is the subleading contribution of the second term in
the $UV$: it presents a $\sim 1/k^4$ suppression with respect to the standard first term, which
will always produce a UV convergent loop contribution in dimension 4. The renormalization of
the RGZ and GZ (which corresponds to $\mu = m =0$) follows from this important property and, as
already mentioned, it is well known that $\gamma$ does not renormalize independently and thus
cannot be considered as an independent dynamically generated scale.

One is thus led to conjecture that this is a general property of theories displaying such
confining propagators, with $\gamma$ standing for a general mass scale associated with
confinement of the fundamental fields; $\gamma$ must be understood as a scale determined by
other dynamically generated scales of the theory. More precisely, the second term in
\eqref{Dg2} cannot generate any new $UV$ divergences in the theory and therefore cannot change
the renormalization properties of the theory, which must be the same as with $\gamma = 0$. In a
diagrammatic approach, only positive powers of propagators appear, so that it is clear that the
highly-suppressed Gribov contribution (cf. \eqref{Dg2}, e.g.) will not influence the deep UV
behavior of the theory. Furthermore, it follows that if the theory with $\gamma = 0$ does not
generate a mass scale , then, since there can be no divergences proportional to $\gamma$, no
mass scale will be generated in the  $\gamma \neq 0$ theory. This in turn means that it is not
possible to assign a dynamical meaning to the $\gamma$ parameter in this case, i.e., the only
possible solution is to have  $\gamma = 0$ in these cases.

In the following sections we will study a variety of examples that support these claims. In section \ref{confscalar} we discuss the case of an interacting scalar field theory displaying a confining propagator. In section 
\ref{confscalarferm} we consider the inclusion of confined fermions interacting with the confined scalars through a Yukawa term. In section \ref{symN1} we discuss the case of Super Yang-Mills with ${\cal N} =1$ supersymmetries
and show to all orders via the algebraic renormalization approach that the adoption of Gribov-type propagators does not produce any new UV divergences, with the renormalization of the IR parameters being completely defined by the UV renormalization of the parameters of the original theory. 
%Section \ref{conc} collects our summary and conclusions.






\section{The confined scalar field}
\label{confscalar}

Let us begin with the theory of a scalar field $\phi$, whose action is given by \eqref{acs}
with a decoupled gauge field. In this situation we have
\begin{eqnarray} 
S ~=~ \int d^4 x \Biggl[ \frac 12  \left(\p{\phi}\cdot \p{\phi} \right)^{2} + 
\frac12 m_{\phi}^2 \; {\phi}\cdot{\phi}
 + \frac{\lambda}{4}  ({\phi}\cdot{\phi})^{2} + 
\frac{\sigma^4}{2}  \left(  \frac{{\phi}\cdot{\phi}}{-\p^2}\right)  \Biggr] \,,
\label{scalar}
\end{eqnarray}
%\begin{eqnarray} 
%S_s  &=& \int d^4 x \left[ \frac 12 \phi \left( -\partial^2 + m^2 \right) \phi \right]\label{scalar2}\\
%S_{int} &=&  \int d^4 x \; \left[\frac 14 \lambda \phi^4\right]\label{ints}\\
%S_{\gamma} &=& \int d^4 x \left[ \frac 12 \phi \left(  \frac{\gamma^4}{-\partial^2}\right) \phi \right]\,,
%\label{scalar3}
%\end{eqnarray}
The parameter $m_{\phi}^2$ is the mass of the scalar field in the deconfined ($\sigma \to 0$)
theory and $\lambda$ is the quartic coupling constant. Here,  $\sigma$ is the confining
parameter, or infrared parameter, that shall play a similar role for the scalars as the Gribov
mass does for the confined gluons, as detailed in the previous Chapter. Our claim in this case
is that for the case that the parameter $\sigma$ is non-zero the deep UV behavior of the theory
is not affected, at all.

One should notice that the model considered here follows the construction of the previous
Chapter, \ref{brstonmatter}. Since the action \eqref{scalar} is equivalent to the action
\eqref{acs} defined in the previous Chapter, for a decoupled gauge field, and we are
interested in computing the corresponding propagators, we have just to follow the same
procedure as developed therein: introduce a couple of auxiliary fields in
order to localize the action; change to the Fourier space, obtaining
%\begin{eqnarray}
%S^{\,quad} ~=~ \int d^{4}x\, \Bigg\{ \frac 12 \phi \left( -\partial^2 + m^2 \right) \phi ~+~
%\tilde{\eta}\p^{2}\eta ~-~ \tilde{\theta}\p^{2}\theta ~+~
%\gamma^{2}\phi(\eta + \tilde{\eta}) \Bigg\}\,.
%\end{eqnarray}
%In the Fourier space, we have
\begin{eqnarray}
S^{\,quad} ~=~ \int \frac{d^{4}k}{(2\pi)^{4}}\, 
\Bigg\{ 
\frac 12 \phi \left( k^2 + m_{\phi}^2 \right) \phi ~ - ~
\tilde{\eta} k^{2} \eta ~+~ \tilde{\theta} k^{2} \theta ~+~
\sigma^{2}\phi(\eta + \tilde{\eta}
\Bigg\}\,;
\end{eqnarray}
and integrating out the auxiliary fields. One should end up, afterwards, with
\begin{eqnarray}
S^{\,quad} ~=~ \int \frac{d^{4}k}{(2\pi)^{4}}\, 
\Bigg\{ 
\frac 12 \phi \Biggl[ \frac{ k^{4} +m_{\phi}^{2}k^{2} - \sigma^{4}}{k^{2}} \Biggr] \phi
\Bigg\}\,.
\label{scquadact}
\end{eqnarray}
Finally, from the functional generator, one can identify the inverse of the momentum dependent
factor of the quadratic term $\phi^{2}$ of equation \eqref{scquadact} as the \emph{tree-level
confining propagator} of the scalar field:
\begin{eqnarray} 
{\cal D}(k^2) & = & \frac{k^2}{k^4 +m_{\phi}^2k^2 + \sigma^4}  \;.\nonumber\\ 
& = &  \frac{1}{k^2 +  m_{\phi}^2} - \frac{\sigma^4 }{\left(k^2 +  m_{\phi}^2\right)\left(k^2 +
M_{+}^2\right)\left(k^2 +  M_{-}^2\right)}
\nonumber\\ 
& = &  \frac{1}{k^2 +  m_{\phi}^2} -
\sigma^4 \Delta(k^2)
\label{scalarprop}
\end{eqnarray}
where we have isolated the confining contribution to the scalar propagator, $\sigma^4\Delta$,
with
\begin{eqnarray} 
\Delta(k^2)&=& \frac{1}{\left(k^2 +  m_{\phi}^2\right)\left(k^2 +  M_{+}^2\right)\left(k^2 +
M_{-}^2\right)}\label{delta}\,,
\end{eqnarray}
which is highly suppressed in the UV: $\Delta\sim 1/k^6$. The mass parameters $M^2_{\pm}$ are
written in terms of $\sigma$ and $m_{\phi}$:
\begin{eqnarray} 
M^2_{\pm}&=& \frac{m_{\phi}^2}{2} \pm \frac 12 \sqrt{m_{\phi}^4 - 4\sigma^4} \,.
\label{scalarmasses}
\end{eqnarray}
Note that $M^2_{\pm}$ may become complex for large enough $\sigma/m_{\phi}$. The complexity of
these IR mass parameters is closely related to positivity violation and, then, with the absence
of a physical particle interpretation for these excitations, leading to the Gribov-kind
confinement interpretation.

It is not difficult to see that there are no new $UV$ divergences associated with the
non-local contribution (\emph{i.e.} proportional to $\sigma^{4}$) to the action \eqref{scalar}
by looking at the diagrams of primitive divergences of the
theory.

In fact, the one-loop scalar selfenergy is
\begin{eqnarray}
\Diagram{ 
&& c & \\
&&& \\
 f& & f &f
} \propto \int d^4 p{\cal D}(p) = \int d^4 p \frac{1}{p^2 +  m_{\phi}^2}  + \sigma^4 \int d^4p
\Delta (p^2)\nonumber\\
 = \int d^4 p \frac{1}{p^2 +  m_{\phi}^2}  + \sigma^4 (\text{UV finite})\,.
\end{eqnarray}
The correction to the quartic coupling at one loop reads:
\begin{eqnarray}
\Diagram{ 
fd \;\;\;\;\;\;\; & !{fl}{k-p} !{flu}{p}  & \;\;\;\;\;\;\; fu \\
fu \;\;\;\;\;\;\; &&  \;\;\;\;\;\;\; fd
} \propto \int d^4 p {\cal D}(k-p) {\cal D}(p) = \int d^4 p \frac{1}{p^2 +  m_{\phi}^2}
\frac{1}{(k-p)^2 +  m_{\phi}^2} +\nonumber\\
&&\hspace{-10cm}+ \,\sigma^4 \int d^4p \Delta (p^2) \frac{1}{(k-p)^2 +  m_{\phi}^2} 
 + \sigma^4 \int d^4p  \frac{1}{p^2 +  m_{\phi}^2}\Delta ((k-p)^2) \nonumber \\
 &&\hspace{-10cm}+\, \sigma^8  \int d^4p \Delta (p^2) \Delta ((k-p)^2) \nonumber\\
= \int d^4 p \frac{1}{p^2 +  m_{\phi}^2} \frac{1}{(k-p)^2 +  m_{\phi}^2} + {\cal O}(\sigma^4,
\sigma^8) (\text{UV finite }) 
\end{eqnarray}

As a representative example at two-loop order, we may look at the scalar selfenergy sunset
diagram:
\begin{eqnarray}
\Diagram{ 
& !{fl}{k-p-q} !{flu}{p}  &\\
&& \\
f f & !{f}{q} f
} &\propto& \int d^4 p \int d^4 q {\cal D}(k-p-q) {\cal D}(q){\cal D}(p)\nonumber\\
&=&  \int d^4 p \int d^4 q \frac{1}{p^2 +  m_{\phi}^2} \frac{1}{q^2 +
m_{\phi}^2}\frac{1}{(k-p-q)^2 +  m_{\phi}^2} + \nonumber\\
 && + \,\sigma^4 \int d^4 p \int d^4 q  \Delta (p^2)  \frac{1}{q^2 +
m_{\phi}^2}\frac{1}{(k-p-q)^2 +  m_{\phi}^2} +\nonumber\\
&& + \,\sigma^4     \int d^4 p \int d^4 q \frac{1}{p^2 +  m_{\phi}^2} \Delta (q^2)
\frac{1}{(k-p-q)^2 +  m_{\phi}^2}+\nonumber\\
 && + \,\sigma^4      \int d^4 p \int d^4 q \frac{1}{p^2 +  m_{\phi}^2} \frac{1}{q^2 +
m_{\phi}^2} \Delta ((k-p-q)^2) +{\cal O}(\sigma^ 8)\nonumber\\
&=&  \int d^4 p \frac{1}{p^2 +  m_{\phi}^2} \frac{1}{q^2 +  m_{\phi}^2}\frac{1}{(k-p-q)^2 +
m_{\phi}^2} +
\nonumber\\
&&
\phantom{\int d^4 p}
\ \ \ \ \ \ \ \ \ \ \ \ \ \ \  + {\cal O}(\sigma^4, \sigma^8, \sigma^{12}) (\text{UV
finite})\,.
\end{eqnarray}

In all  examples above, the appearance of a general form for the contributions of the confining
scale with increasingly UV convergent momentum integrals is clear.
It is straightforward to realize then that this pattern will spread throughout all orders of
the diagrammatic expansion, so that we are led to infer that  contributions proportional to
$\sigma$  cannot give rise to new primitive divergences, besides the ones coming from the
standard theory (that one with $\sigma =0$). 



%--------------------------------------------------------------------------
\section{The confined fermion and scalar fields interacting}
\sectionmark{The fermion and scalar fields interacting}
\label{confscalarferm}
%--------------------------------------------------------------------------

The same reasoning can be applied when Dirac fermions are added to the theory, with an Yukawa
coupling and a fermionic Gribov-type term rendering the fermionic excitations also confined.  

We consider here the theory in the absence of scalar condensates. In this case, the full action
reads 
\begin{eqnarray} 
S &=& \int d^4 x \Biggl[ \frac 12  {\phi}\left(  -\p^{2} +  m_{\phi}^2 \right){\phi} + 
 + \frac{\lambda}{4}  ({\phi}\cdot {\phi})^{2} + 
\frac{\sigma^4}{2}  \left(  \frac{{\phi}\cdot{\phi}}{-\p^2}\right)  +
\nonumber \\
&&
\phantom{\int d^4 x }
\bar{\psi} \left(\dslash + m_{\psi}\right) \psi + g\phi\bar{\psi} \psi + \frac 12 \phi \left(  \frac{\gamma^4}{-\partial^2}\right)
\phi  + \bar{\psi}\left(\frac{M^3}{-\partial^2}\right)\psi \Biggr]\,,
\label{scalarferm}
\end{eqnarray}
%where 
%\begin{eqnarray} 
%S_s  &=& \int d^4 x \left( \frac 12 \phi \left( -\partial^2 + m_{\phi}^2 \right) \phi \right)\\
%S_f &=&  \int d^4 x \;\bar{\psi} \left(\dslash + m_{\psi}\right) \psi \label{fermion}\\
%S_{int} &=&  \int d^4 x \; \left(g\phi\bar{\psi} \psi +\frac 14 \lambda
%\phi^4\right)\label{int}\\
%S_{\gamma, M} &=& \int d^4 x \left( \frac 12 \phi \left(  \frac{\gamma^4}{-\partial^2}\right)
%\phi  + \bar{\psi}\left(\frac{M^3}{-\partial^2}\right)\psi\right)
%\label{scalar3F}
%\end{eqnarray}
where $m_{\psi}$ is the mass of the original fermion field (i.e. for $M\to 0$) and $g$ is the
Yukawa coupling. In the fermionic sector the IR mass scale analogous to the Gribov parameter is
$M$, such as in the previous Chapter. 

Analogously to the previous purely scalar case, the \emph{tree-level propagators} of both the
scalar and fermion fields are obtained through the insertion of auxiliary fields,
different doublets to each sector, and the subsequent integration of such fields in the Fourier
space. After all, it is not difficult to see that there are no UV divergences associated to
the non-local terms proportional to $\sigma$ and $M$. The scalar excitations display the same
confining propagator as the one
derived in the last section,  \eqref{scalarprop}, while for fermion field we have
\begin{eqnarray} 
{\cal S}(k^2) & = & \frac{i\kslash + m_{\psi} + \frac{M^3}{k^2}}{k^2 + (m_{\psi} +
\frac{M^3}{k^2})^2}  \;.\nonumber\\ 
& = &  \frac{i\kslash + M }{k^2 + m_{\psi}^2} + M^3 \frac{(k^2+m_{\psi}^2)k^2 -(i\kslash +
m_{\psi})(2Mk^2 + M^3)  }{(k^6 + (m_{\psi} k^2 + M^3)^2)(k^2 + m_{\psi}^2)}
\nonumber\\ 
& = &  \frac{i\kslash + m_{\psi} }{k^2 + m_{\psi}^2} + M^3 \Sigma(k^2)\,,
\label{fermprop}
\end{eqnarray}
Again, the isolated confining contribution to the propagator is highly suppressed in the UV
with respect to the standard massive Dirac term ($\sim 1/k$):
\begin{eqnarray} 
\Sigma(k)& = &   \frac{(k^2+m_{\psi}^2)k^2 -(i\kslash + m_{\psi})(2m_{\psi}k^2 + M^3)  }{(k^6 +
(m_{\psi} k^2 + M^3)^2)(k^2 + m_{\psi}^2)}\sim 1/k^4
 \,,
\label{sigma1}
\end{eqnarray}
and we anticipate that the primitive divergences of the theory with confined propagators will
be exactly the ones coming from terms of the original (local) theory, since any contribution
proportional to $\sigma$ or $M$ will be strongly suppressed in the UV regime.

At one loop order, besides the diagrams already analyzed in the previous section, new diagrams contributing to primitive divergences appear, due to the presence of fermion lines (dashed ones):
\begin{figure}[h!]
   \centering
       \includegraphics[width=0.8\linewidth]{diags-yukawa.png}
               \caption{One-loop diagrams containing fermion (dashed) lines for the fermion and scalar selfenergies and cubic, quartic and Yukawa couplings, respectively. \label{fermiondiags}}
\end{figure}

It should be noticed that the Yukawa coupling breaks the discrete symmetry $\phi\to -\phi$
originally present in the scalar sector, generating at the quantum level a cubic scalar
interaction. This means that the renormalizable version of this theory requires a counterterm
for the cubic scalar interaction, even if the physical value of this coupling is set to zero.
In the case of a pseudoscalar Yukawa coupling (i.e. $g\phi\bar\psi\psi \to
g\phi\bar{\psi} \gamma^{5} \psi $), parity symmetry guarantees that the cubic terms vanish
identically. We emphasize, however, that our statement concerning the UV properties of
Gribov-type confining propagators remains valid in any case, as will be made explicit below via
the whole set of primitive divergences at one loop order. 

In order to investigate the influence of the confining propagators in the UV regime, we may
isolate the free fermion and scalar propagators from the confining contributions, namely
$\Sigma(k)\stackrel{UV}{\sim} 1/k^4$ and $\Delta(k^2)\stackrel{UV}{\sim} 1/k^6$, being both
highly suppressed in the UV. Writing down explicitly the momentum integrals in the
corresponding expressions for the one-loop diagrams in Fig. \ref{fermiondiags}, we have,
respectively:
\begin{enumerate}[label=(\alph*)]
\item the one-loop fermion self energy:
\begin{eqnarray}
 \int d^4 p {\cal D}(k-p) {\cal S}(p)
 &=&   \int d^4 p  \frac{1}{(k-p)^2 +  m_{\phi}^2} \frac{i\pslash +m_{\psi}}{p^2 +  m_{\psi}^2}
+ \sigma^4  \int d^4 p {\Delta}((k-p)^2) \frac{i\pslash +m_{\psi}}{p^2 +  m_{\psi}^2}
\nonumber\\&&
+ M^3 \int d^4 p  \frac{1}{(k-p)^2 +  m_{\phi}^2} \Sigma(p)
+ \sigma^4 M^3 \int d^4 p {\Delta}((k-p)^2)\Sigma(p) \nonumber
 \\
& =& \int d^4 p \frac{i\pslash +m_{\psi}}{p^2 +  m_{\psi}^2} \frac{1}{(k-p)^2 +  m_{\phi}^2} +
{\cal O}(\sigma^4, M^3, \sigma^4 M^3) (\text{UV finite}) 
\end{eqnarray}

\item  the fermion loop contributing to the scalar self energy:
\begin{equation}
 \int d^4 p\, {\rm Tr}[{\cal S}(p){\cal S}(k-p) ] ~=~ \int d^4 p {\rm Tr}\Big[\frac{i\pslash
+m_{\psi}}{p^2 +  m_{\psi}^2} \frac{i(\kslash -\pslash)+ m_{\psi}}{(k-p)^2 +  m_{\psi}^2}
\Big]+ {\cal O}( M^3, M^6) (\text{UV finite}) 
\end{equation}

\item the triangular diagram contributing to the scalar cubic interaction:
\begin{eqnarray}
 \int d^4 p\, {\rm Tr}[{\cal S}(p){\cal S}(p-k){\cal S}(p-k-k') ]
 &=& \int d^4 p {\rm Tr}\Big[ \frac{i\pslash +m_{\psi}}{p^2 +  m_{\psi}^2}
\frac{i(\pslash-\kslash) +m_{\psi}}{(p-k)^2 +  m_{\psi}^2} \frac{i(\pslash -\kslash -\kslash')+
m_{\psi}}{(p-k-k')^2 +  m_{\psi}^2}\Big] +\nonumber \\
 &&+ {\cal O}( M^3, M^6, M^{9}) (\text{UV finite}) 
\end{eqnarray}

\item the fermion loop correction to the $\phi^4$ vertex:
\begin{eqnarray}
 \int d^4 p\,{\rm Tr}\Big[{\cal S}(p){\cal S}(p-k){\cal S}(p-k-k'){\cal S}(p-k-k'-k'')  \Big]
 &=& \nonumber\\&&\hspace{-10cm}=
  \int d^4 p {\rm Tr}\Big[ \frac{i\pslash +m_{\psi}}{p^2 +  m_{\psi}^2}
\frac{i(\pslash-\kslash) +m_{\psi}}{(p-k)^2 +  m_{\psi}^2} \frac{i(\pslash -\kslash -\kslash')+
m_{\psi}}{(p-k-k')^2 +  m_{\psi}^2}
  \frac{i(\pslash -\kslash -\kslash'-\kslash'')+ m_{\psi}}{(p-k-k'-k'')^2 +  m_{\psi}^2}\Big]
+\nonumber \\
 &&
\hspace{-6cm}+ {\cal O}( M^3, M^6, M^{9},M^{12}) (\text{UV finite}) 
\end{eqnarray}

\item the modification of the Yukawa coupling:
\begin{eqnarray}
 \int d^4 p\,\Big[{\cal S}(p){\cal D}(p-k){\cal S}(p-k-k') \Big]
 &=& 
  \int d^4 p {\rm Tr}\Big[ \frac{i\pslash +m_{\psi}}{p^2 +  m_{\psi}^2}\frac{1}{(p-k)^2 +
m_{\phi}^2} \frac{i(\pslash-\kslash-\kslash') +m_{\psi}}{(p-k-k')^2 +  m_{\psi}^2} 
 \Big] +\nonumber \\
 &&+ {\cal O}( \sigma^4,M^3, M^6, \sigma^4 M^3, \sigma^4 M^6) (\text{UV finite}) 
\end{eqnarray}


\end{enumerate}

As already occurred for the confining scalar theory in the previous section, the highly
suppressed UV behavior of the confining pieces $\Sigma(k)\stackrel{UV}{\sim} 1/k^4$ and
$\Delta(k^2)\stackrel{UV}{\sim} 1/k^6$ enforces the convergence of all terms proportional to
the new massive parameters introduced ($\sigma$ and $M$).
The divergent integrals in all diagrams above are exactly the ones coming from the original
action, i.e. the one obtained in the limit $\sigma \to 0$ and $M \to 0$. 
In the theory including the confining quadratic non-local terms, the absence
of new primitive divergences then guarantees that the parameters $\sigma$ and $M$ can be
consistently related to dynamically generated scales and do not affect the UV regime of the
theory.

Realizing that any diagrammatic expression at higher loops will involve higher powers of the
propagators, it becomes straightforward to envision the generalization of our claim in the full
diagrammatic expansion of this general Yukawa theory. Therefore, given the renormalizability of
the original theory, one concludes that the resulting action with confining, Gribov-type
propagators is renormalizable and the IR confining parameters in both fermionic and bosonic
sectors do not display an independent renormalization, being thus consistent with dynamically
generated mass scales.















%--------------------------------------------
\section{$\mathcal N=1$ Super Yang--Mills within the Gribov--Zwanziger approach}
\sectionmark{$\mathcal N=1$ Super Yang--Mills within the GZ approach}
\label{symN1}
%--------------------------------------------

Let us now investigate a more intricate theory with confining propagators, including gauge interactions as well as Majorana fermions. We consider here Yang-Mills theory in $D=4$ spacetime dimensions with $\mathcal N=1$ supersymmetry in the presence of the Gribov horizon. We shall use this (most complicated) example to prove, to all-orders in the loop expansion, our claim concerning the good UV behavior of Gribov-type propagators. The IR parameters introduced will be shown to have renormalization parameters that are completely determined by the renormalization of the original theory.

This theory has already been put forward and investigated in \cite{Capri:2014xea}. There, the extension of the Gribov-Zwanziger framework to $\mathcal  N = 1$ 
Super-Yang-Mills (SYM) theories quantized in the Wess-Zumino gauge by imposing the Landau gauge condition was presented. The resulting 
effective action is
\begin{equation} 
S_{SGZ}^{N=1} = S_{SYM}^{N=1}  + Q \int d^4x \left( {\bar c}^a \partial_\mu A^a_\mu + {\bar \omega}^{ac}_{\mu}  (-\partial_\nu D^{ab}_{\nu} ) \varphi^{bc}_{\mu}  \right) +  S_{\gamma} + S_{\tilde{G}} \;, \label{sgzn1}
\end{equation} 
where $Q$ is the full transformation accounting for the supersymmetryc transformation and the
BSRT transformation, and is defined in the appendix so that the action \eqref{sgzn1} results in
\eqref{fnlact}; $S_{\gamma}$ is the horizon term in its local form, eq.\eqref{hfl}, namely 
\begin{equation}
S_\gamma =\; \gamma^{2} \int d^{4}x \left( gf^{abc}A^{a}_{\mu}(\varphi^{bc}_{\mu} + {\bar \varphi}^{bc}_{\mu})\right)-4 \gamma^4V (N^2-1)\;. \label{hfln1}
\end{equation} 
and the term $S_{\tilde{G}}$ is given by 
\begin{equation}
S_{\tilde{G}} = - \frac{1}{2}M^3\int d^{4}x \left( \bar{\lambda}^{a\alpha}\frac{\delta_{\alpha\beta}}{\partial^{2}}\lambda^{a\beta}\right) \;, \label{sslambda}
 \end{equation} 
which also has a new massive constant $M$. This quantum action takes into account the existence of Gribov copies in the path-integral quantization of the theory. It encodes the restriction to the first Gribov horizon while
keeping full compatibility with non-perturbative supersymmetric features, such as the exactly vanishing vacuum energy. 

Even though this non-perturbative framework has been constructed through the introduction of two massive parameters $\gamma, M$ which are not present in the classical action,
those new parameters are determined in a dynamical, self-consistent way via two
non-perturbative conditions: (i) the Gribov gap equation, that fixes $\gamma$ by imposing the
positivity of the Faddeev-Popov operator and eliminating a large set of Gribov copies from the
functional integral, and (ii) the vanishing of the vacuum energy, which determines the
parameter $M$ that plays the role of a supersymmetric counterpart of the Gribov parameter
$\gamma$, guaranteeing a consistent non-perturbative fermion sector. Interestingly, the
appearance of the dynamical fermionic scale $M$ has been shown to be directly related to the
formation of a gluino condensate, a well-known non-perturbative property of ${\cal N}=1$ SYM
theories. For further details, the reader is referred to \cite{Capri:2014xea}. A brief
summary of the notation adopted may also be found in the Appendix \ref{notations}.

The propagators of the theory \eqref{sgzn1} can be straightforwardly shown to be of the Gribov type. The gauge field propagator is:
%
\begin{equation}
\langle
A_{\mu}^a(p)A_{\nu}^b(-p)
\rangle
= \delta^{ab}\left(\delta_{\mu\nu}-\frac{p_{\mu}p_{\nu}}{p^2}\right)
\frac{p^2}{p^4+2Ng^2\gamma^4}\,,
\end{equation}
%
which, apart from the more complicated tensorial structure, is equivalent to the Gribov scalar propagator studied above in section \ref{confscalar}. The gauge field propagator in this Gribov-extended $\mathcal N=1$ SYM theory displays thus a confining contribution that is suppressed by an extra $1/p^4$ factor in the UV as compared to the free term.

For gluino fields we have:
%
%
%
\begin{eqnarray}
\langle
\bar{\lambda}_{\alpha}^a(p)
{\lambda}_{\beta}^b(-p)
\rangle
&=&
\delta^{ab}\frac{ip_{\mu}(\gamma_{\mu})_{\alpha\beta}+m(p^2)\delta_{\alpha\beta}}{p^2+m^2(p^2)}
\,,
\\
\langle \lambda^{a\rho}(p)\lambda^{b}_{\beta}(-p)\rangle 
&=& 
- \frac{\big( ip_{\mu}(\gamma_{\mu})_{\alpha\beta} + m(p^2)\delta_{\alpha\beta}\big)\delta^{ab}C^{\alpha\rho}}{p^{2} + m^{2}(p^2)}
\,,
\\
\langle \bar{\lambda}^{a}_{\alpha}(p)\bar{\lambda}^{b\tau}(-p)\rangle 
&=&
\frac{\big( ip_{\mu}(\gamma_{\mu})_{\alpha\beta} + m(p^2)\delta_{\alpha\beta}\big)\delta^{ab}C^{\beta\tau}}{p^{2} + m^{2}(p^2)}
\,,
\end{eqnarray}
%
where $C^{\alpha\beta}$ is the charge conjugation matrix and
%
\begin{equation}
m(p^2)=\frac{M^3}{p^2}\,.
\end{equation}
%
The presence of three two-point correlation functions involving gluino fields is a result of the lack of charge conservation for Majorana fermions. One verifies however that all of them have the form of Gribov propagators with $M$ playing an analogous role as the Gribov parameter in the gluino sector. In particular, one can easily check that the same structure observed for the Gribov fermion propagator
in the previous section (cf. Eq.\eqref{fermprop}) is found here:
\begin{eqnarray} 
\langle
\bar{\lambda}_{\alpha}^a(k)
{\lambda}_{\beta}^b(-k)
\rangle& = & \frac{i\kslash + \frac{M^3}{k^2}}{k^2 + \frac{M^6}{k^4}} 
% \;.\nonumber\\ 
%& = & 
= \frac{i\kslash }{k^2} + M^3 \Sigma_{\lambda}(k^2)\,,
\label{gluinoprop}
\end{eqnarray}
where the isolated confining contribution $\Sigma_{\lambda}$ to the gluino propagator is again highly suppressed in the UV with respect to the leading term ($\sim 1/k$):
\begin{eqnarray} 
\Sigma_{\lambda}(k^2)& = &   \frac{k^4 -i\kslash M^3  }{(k^6 + M^6)k^2}\stackrel{\rm UV}{\sim} 1/k^4
 \,.
\label{sigmag}
\end{eqnarray}

The same reasoning applied in the scalar and Yukawa theories above may be followed here in order to prove that the UV regime of the theory remains the same even after the inclusion of nonlocal confining terms in the propagators. One may compute the one-loop primitive divergences and show that the confining parameters $\gamma,M$ will not affect the UV divergent pieces, due to the high suppression observed in the Gribov-type propagators. We shall, however, use this most complicated theory analyzed in the current section to present an all-order algebraic proof of renormalizability and of the fact that the confining parameters $\gamma,M$ do not display independent renormalization.

\noindent The non-local action \eqref{sgzn1} is, however, not helpful in the algebraic renormalization procedure. Fortunately we are able to write its local form with the insertion of auxiliary fields. 

\noindent The whole action which describes our model can then be written in its local form as, 
\begin{eqnarray}
\label{fnlact1}
S &=& S_{SYM} + S_{gf} + S_{GZ'} + S_{L\tilde{G}}\nonumber \\
&&
=\int d^{4}x\; \left\{\frac{1}{4}F^{a}_{\mu \nu}F^{a}_{\mu\nu} 
+ \frac{1}{2} \bar{\lambda}^{a\alpha} (\gamma_{\mu})_{\alpha\beta} D^{ab}_{\mu}\lambda^{b\beta}
+ \frac{1}{2}\mathfrak{D}^a\mathfrak{D}^a 
+ b^{a}\partial_{\mu}A^{a}_{\mu} \right. \nonumber\\
&&
+\check{c}^{a}\left[\partial_{\mu}D^{ab}_{\mu}c^{b}
-\bar{\epsilon}^{\alpha}(\gamma_{\mu})_{\alpha\beta}\partial_{\mu}\lambda^{a\,\beta}\right]
+ \tilde{\varphi}^{ac}_{\mu}\partial_{\nu}D_{\nu}^{ab}\varphi^{bc}_{\mu} 
-\tilde{\omega}^{ac}_{\mu}\partial_{\nu}D_{\nu}^{ab}\omega^{bc}_{\mu} \nonumber \\
&&
-gf^{abc}(\partial_{\nu}\tilde{\omega}^{ad}_{\mu})(D^{bk}_{\nu}c^{k})\varphi^{cd}_{\mu}
+gf^{abc}(\partial_{\nu}\tilde{\omega}^{ad}_{\mu})(\bar{\epsilon}^\alpha(\gamma_\nu)_{\alpha\beta}\lambda^{\beta b})\varphi^{cd}_{\mu} \nonumber \\
&&
+\gamma^{2}gf^{abc}A^{a}_{\mu}(\varphi^{bc}_{\mu} + \tilde{\varphi}^{bc}_{\mu}) 
-\gamma^{4}4(N_{c}^{2}-1) 
+\hat{\zeta}^{a\alpha} (\partial^{2} - \mu^{2})\zeta^{a}_{~\alpha} \nonumber \\
&&
\left.
-\hat{\theta}^{a\alpha}(\partial^{2} - \mu^{2})\theta^{a}_{~\alpha} 
-M^{3/2}(\bar{\lambda}^{a\alpha}\theta^{a}_{~\alpha} 
+\hat{\theta}^{a\alpha}\lambda^{a}_{~\alpha})
\right\}\;.
\label{thSYM}
\end{eqnarray} 

Applying the algebraic renormalization procedure to the local action above we are able to prove that:
(i) the Gribov-extended SYM theory is renormalizable; and (ii) the massive parameters $\gamma,
M$ introduced in the infrared action do not renormalize independently, meaning that they are
consistent with dynamically generated mass scales, produced by nonperturbative interactions in
the original theory. All details of the proof were developed in the Appendix \eqref{algrenorm}.

The final results for the renormalization factors related to the confining parameters
$M,\gamma$ may be read off from the renormalization of external sources conveniently introduced
in the algebraic procedure (developed in the Appendix \eqref{algrenorm}). The renormalization of
the sources $m_{\psi}$ and $\tilde{m_{\psi}}$ give us the renormalization factor of the Gribov
parameter $\gamma^{2}$, while the renormalization of $V$ and $\hat{V}$ give us the
renormalization of $m_{\psi}^{3/2}$, when every source assumes its physical value stated at
\eqref{physval2}. We have:
\begin{eqnarray}
&&
Z_{\tilde{M}} =Z_{M} = Z^{-1/2}_{g}Z^{-1/4}_{A}\;, \nonumber \\
&&
Z_{\hat{V}} =Z_{V} = Z^{-1/2}_{\lambda}\;,
\end{eqnarray}
which clearly prove that the renormalization of the infrared parameters $m_{\psi},\gamma$ is
fixed by the renormalization factor of the original $SYM$ theory: the renormalization of the
gauge coupling, $Z_g$, the wave function renormalization of the gauge field, $Z_A$, and and the
wave function renormalization of the gluing field, $Z_{\lambda}$.

Therefore we conclude that this action is indeed a suitable nonperturbative infrared action for
${\cal N}=1$ SYM theories, reducing consistently to the ultraviolet original action. Moreover,
even in this very intricate non-Abelian gauge theory with matter fields, the good UV behavior
in the presence of confining propagators of the Gribov type shows up at all orders.











%-------------------------------
\section{Discussions about the results\label{conc5}}
%-------------------------------


Carrying on the analysis started on Chapter \ref{brstonmatter}, we have studied the UV
behavior of quantum field theory models in which the two-point correlation functions of the
elementary fields are described by confining propagators of the Gribov type. Our analysis was
not restricted to the gauge sector of the theory, but it concerns general properties in the UV
regime of any kind of fields that is said to be confined in the Gribov sense.


We could show that, order by order, the UV divergent behavior of the Feynman diagrams is not
affected by the infrared parameters of the theory. By infrared parameters we mean those
associated to the non-local term of the action, which accounts for non-perturbative
effects of the theory: $\sigma$ for the scalar field sector, and $M$ for the fermion field
sector. More precisely, we observed that contributions to Feynman diagrams stemming from the
non-local terms of the action, which are those proportional to $\sigma$ or $M$, are always
finite, being, thus, highly suppressed in the UV regime by the standard ultraviolet tree-level
propagator.

As a consequence, no new UV divergences in the infrared parameters can arise. Otherwise said,
the only UV divergences affecting the 1PI Green's functions of the theory are those present
when the infrared parameters are set to zero (and no non-perturbative effect is taken into
account). Therefore, the infrared parameters do not renormalize independently.

An all order proof was also presented, which can be checked in the Appendix \ref{algrenorm}, in the case of ${\cal N}=1$ Super Yang-Mills model within the Gribov horizon and with a \emph{horizon-like} term in the super-partner sector. We could explicitly show that both infrared parameters, $\gamma^{2}$, for the gauge sector, and $M^{3}$, for the fermion sector, do not renormalize independently, \emph{i.e.} their renormalization factor depends on the renormalization factor of fundamental fields and parameters of the original theory (out of Gribov horizon).  Namely, their renormalization factors can be read out of
\begin{eqnarray}
&&
Z_{\tilde{M}} =Z_{M} = Z^{-1/2}_{g}Z^{-1/4}_{A}\;, \nonumber \\
&&
Z_{\hat{V}} =Z_{V} = Z^{-1/2}_{\lambda}\;.
\end{eqnarray}























