

%%%%%%%%%%%%%%%%%%%%%%%%%%%%%%%%%%%%%%%%%%%%%%%%%%%%%%%%%%%%%%%%
\chapter{The Electroweak theory: $SU(2)\times U(1)+$Higgs field}
\label{The Electroweak theory}
%%%%%%%%%%%%%%%%%%%%%%%%%%%%%%%%%%%%%%%%%%%%%%%%%%%%%%%%%%%%%%%%

From now on in this work only the fundamental case of the Higgs field will be treated, for reasons relying on the physical relevance of the fundamental representation of this field. As a first step, we are going to present, as in the previous sections, general results for $d$-dimension. Afterwards, the $3$ and $4$-dimensional cases will be considered in the subsections \ref{d=3} and \ref{d=4}. The starting action of the $SU(2) \times U(1)+$Higgs field reads
\begin{eqnarray}
S ~=~ \int \d^{d}x  \;  \bigg(\frac{1}{4}  F_{\mu \nu }^{a} F_{\mu \nu }^{a}  +  \frac{1}{4} B_{\mu\nu} B_{\mu\nu} +{\bar c}^a \partial_\mu D^{ab}_\mu c^b - \frac{(\partial_\mu A^a_\mu)^{2}}{2\xi} 
 + {\bar c}\partial^2 c  - \frac{(\partial_\mu B_\mu)^{2}}{2\xi}  +
\nonumber \\
+
(D_{\mu }^{ij}\Phi^{j})^{\dagger}( D_{\mu }^{ik}\Phi^{k})+\frac{\lambda }{2}\left(\Phi^{\dagger}\Phi - \nu^{2}\right)^{2}   \bigg)  \;,
\label{Sf}
\end{eqnarray}
where the covariant derivative is defined by
\begin{equation}
D_{\mu }^{ij}\Phi^{j} =\partial _{\mu }\Phi^{i} - \frac{ig'}{2}B_{\mu}\Phi^{i} -   ig \frac{(\tau^a)^{ij}}{2}A_{\mu }^{a}\Phi^{j}  \;.
\end{equation}
and the vacuum expectation value (\textit{vev}) of the Higgs field is $\langle \Phi^{i} \rangle ~=~ \nu\delta^{2i}$.
%%\begin{equation}
%\langle \Phi \rangle  = \left( \begin{array}{ccc}
%                                          0  \\
%                                          \nu
%                                          \end{array} \right)  \;.
%\label{vevf}
%%\end{equation}
The indices $i,j=1,2$ refer to the fundamental representation of $SU(2)$ and $\tau^a, a=1,2,3$, are the Pauli matrices. The coupling constants $g$ and $g'$ refer to the groups $SU(2)$ and $U(1)$, respectively. The field strengths $F^a_{\mu\nu}$ and $B_{\mu\nu}$ are given by
\begin{equation}
F^a_{\mu\nu} = \partial_\mu A^a_\nu -\partial_\nu A^a_\mu + g \varepsilon^{abc} A^b_\mu A^c_\nu \;, \qquad B_{\mu \nu} = \partial_\mu B_\nu -\partial_\nu B_\mu  \;.
\label{fs}
\end{equation}

In order to obtain the boson propagators only quadratic terms of the starting action are required and, due to the new covariant derivative, this quadratic action is not diagonal any more. To diagonalize this action one could introduce a set of new fields, related to the standard ones by
%To obtain the  gauge boson propagators, we consider the quadratic part of the action (\ref{Sf}), given by
%\begin{multline}
%S_{quad} = \int\!\! \d^{d}x \; \frac{1}{2}A_{\mu}^{\alpha}\left[ \left(-\partial_{\mu}\partial_{\mu} + \frac{\nu^{2}}{2}g^{2}\right)\delta_{\mu\nu} + \partial_{\mu}\partial_{\nu} \right]A_{\nu}^{\alpha} + 
%\int\!\! \d^{d}x\, \frac{1}{2}B_{\mu}\left[ \left(-\partial_{\mu}\partial_{\mu} + \frac{\nu^{2}}{2}g'^{2}\right)\delta_{\mu\nu} + \partial_{\mu}\partial_{\nu}\right]B_{\nu} \\
%+ \int\!\! \d^{d}x\, \frac{1}{2}A_{\mu}^{3}\left[ \left(-\partial_{\mu}\partial_{\mu} + \frac{\nu^{2}}{2}g^{2}\right)\delta_{\mu\nu} + \partial_{\mu}\partial_{\nu}\right]A_{\nu}^{3} - \frac{1}{4}
%\int\!\! \d^{d}x\, \nu^{2}g\,g'\,A^{3}_{\mu}B_{\mu} - \frac{1}{4} 
%\int\!\! \d^{d}x\, \nu^{2}g\,g'\,B_{\mu}A^{3}_{\mu} \;.
%\label{Squad}
%\end{multline}
%In order to diagonalize expression \eqref{Squad} we introduce the following fields
\begin{subequations} \begin{gather}
W^+_\mu = \frac{1}{\sqrt{2}} \left( A^1_\mu + iA^2_\mu \right) \;, \qquad W^-_\mu = \frac{1}{\sqrt{2}} \left( A^1_\mu - iA^2_\mu \right)  \;,
\label{ws} \\
Z_\mu =\frac{1}{\sqrt{g^2+g'^2} } \left(  -g A^3_\mu + g' B_\mu \right) \qquad \text{and}\qquad A_\mu =\frac{1}{\sqrt{g^2+g'^2} } \left(  g' A^3_\mu + gB_\mu \right) \;.
\label{za}
\end{gather} 
\end{subequations}
The inverse relation can be easily obtained.
%Let us also give, for further use, the inverse combinations:
%\begin{subequations} \begin{gather}
%A^1_\mu = \frac{1}{\sqrt{2}} \left( W^+_\mu + W^-_\mu \right) \;, \qquad A^2_\mu = \frac{1}{i\sqrt{2}} \left( W^+_\mu - W^-_\mu \right) \;,
%\label{iw} \\
%B_\mu =\frac{1}{\sqrt{g^2+g'^2} } \left(  g A_\mu + g' Z_\mu \right) \qquad \text{and}\qquad A^3_\mu =\frac{1}{\sqrt{g^2+g'^2} } \left(  g' A_\mu - gZ_\mu \right) \;.
%\label{iza}
%\end{gather} \end{subequations}
With this new set of fields the quadratic part of the action reads,
\begin{eqnarray}
S_{quad} &=&  \int d^3 x   \left( \frac{1}{2} (\partial_\mu W^+_\nu - \partial_\nu W^+_\mu)(\partial_\mu W^-_\nu - \partial_\nu W^-_\mu)  + \frac{g^2\nu^2}{2}W^+_\mu W^-_\mu   \right) 
\nonumber \\
&+& \int d^3x  \left(  \frac{1}{4} (\partial_\mu Z_\nu - \partial_\nu Z_\mu)^2  + \frac{(g^2+g'^2)\nu^2}{4}Z_\mu Z _\mu  +    \frac{1}{4} (\partial_\mu A_\nu - \partial_\nu A_\mu)^2  \right)  \;,
\label{qd}
\end{eqnarray}
from which we can read off the masses of the fields $W^+$, $W^-$, and $Z$:
\begin{equation}
m^2_W = \frac{g^2\nu^2}{2} \;, \qquad m^2_Z =  \frac{(g^2+g'^2)\nu^2}{2}  \;. \label{ms}
\end{equation}

The restriction to the Gribov region $\Omega$ still is needed and the procedure here becomes quite similar to what was carried out in the subsection \ref{Adjrep}. Due to the breaking of the global gauge invariance, caused by the Higgs field (through the covariant derivatives), the ghost sector can be split up in two different sectors, diagonal and off-diagonal. Namely, the ghost propagator reads,



%\subsubsection{The Landau gauge fixing}

%The action \eqref{Sf} has to be supplemented by the gauge fixing term $S_{\text{gf}}$  to allow for a meaningful quantization. To that purpose we should emphasize that we are not going to apply the unitary gauge, as already stated in subsection \ref{gfwithbroksym}. Contrary to that we will adopt the Landau gauge condition that should be covered when $\xi \to 0$ takes place:  $f^{a} ~=~ \partial_\mu A^a_\mu - \xi b^{a}$ and $h^{a} ~=~   \partial_\mu B_\mu - \xi b $. Thus, just by following the gauge fixing procedure developed in the section \ref{introductiontogribov},
%\begin{equation}
%S_{\text{gf}} = \int d^3x \; \left(  {\bar c}^a \partial_\mu D^{ab}_\mu c^b - \frac{(\partial_\mu A^a_\mu)^{2}}{2\xi} 
% + {\bar c}\partial^2 c  - \frac{(\partial_\mu B_\mu)^{2}}{2\xi} \right)   \;,
%\label{cgf}
%\end{equation}
%where the limit $\xi \to 0$ enforce the conditions $\partial_\mu A^a_\mu=0$ and $\partial_\mu B_\mu=0 $, while $({\bar c}^a,\; c^a)$ and $({\bar c},\; c)$ are the $SU(2)$ and $U(1)$ Faddeev--Popov ghosts, respectively. The total action then reads
%\begin{equation}
%S_{f} = S + S_{\text{gf}}  \;. 
%\label{fact}
%\end{equation}
%The propagators of the gauge fields are given, in terms of $W^{\pm}_{\mu}$, $Z_{\mu}$ and $A_{\mu}$, as
%\begin{subequations} \label{WZA gluon prop} \begin{eqnarray}
%\langle W^{+}_{\mu}(p)W^{-}_{\nu}(-p) \rangle &=& \frac{1}{p^{2} + \frac{\nu^{2}}{2}g^{2}}\left(\delta_{\mu\nu} - \frac{p_{\mu}p_{\nu}}{p^{2}}\right) \;, \\
%\langle Z_{\mu}(p) Z_{\nu}(-p) \rangle &=& \frac{1}{p^{2} + \frac{\nu^{2}}{2}(g^{2} + g'^{2})} \left(\delta_{\mu\nu} - \frac{p_{\mu}p_{\nu}}{p^{2}}\right)\;, \\
%\langle A_{\mu}(p) A_{\nu}(-p) \rangle &=& \frac{1}{p^{2}} \left(\delta_{\mu\nu} - \frac{p_{\mu}p_{\nu}}{p^{2}}\right) \;, \\
%\langle A_\mu(p) Z_\nu(-p) \rangle & = & 0 \;.
%\end{eqnarray} \end{subequations}
%Taking into account the relation between the mass eigenstates and the fields $A^{\alpha}_{\mu}$, $A^{3}_{\mu}$ and $B_{\mu}$, we also have the following propagators,
%\begin{subequations} \label{AB gluon prop} \begin{eqnarray}
%\langle A^{\alpha}_{\mu}(p) A^{\beta}_{\nu}(-p) \rangle &=& \frac{1}{p^{2} + \frac{\nu^{2}}{2}g^{2}}\left(\delta_{\mu\nu} - \frac{p_{\mu}p_{\nu}}{p^{2}}\right)\delta^{\alpha \beta} \;, \\
%\langle A^{3}_{\mu}(p) A^{3}_{\nu}(-p) \rangle &=& \frac{1}{p^{2} + \frac{\nu^{2}}{2}(g^{2} + g'^{2})} \left(\delta_{\mu\nu} - \frac{p_{\mu}p_{\nu}}{p^{2}}\right)\;,  \\
%\langle B_{\mu}(p) B_{\nu}(-p) \rangle &=& \frac{1}{p^{2}} \left(\delta_{\mu\nu} - \frac{p_{\mu}p_{\nu}}{p^{2}}\right) \;,  \\
%\langle A^{3}_{\mu}(p) B_{\nu}(-p) \rangle &=& \frac{\nu^{2}}{4}(g^{2} + g'^{2}) \frac{g'}{g} \frac{1}{p^{2}\left(p^{2} + \frac{\nu^{2}}{2}(g^{2} + g'^{2}) \right)} \left(\delta_{\mu\nu} - \frac{p_{\mu}p_{\nu}}{p^{2}}\right)\;.
%\end{eqnarray} \end{subequations}




















%\subsubsection{The restriction to the Gribov region}
%\label{TRGR}

%As established in \cite{Gribov:1977wm}, the gauge fixing procedure is plagued by the existence of Gribov copies,  an observation later on shown to be a general feature of non-Abelian gauge fixing \cite{Singer:1978dk}. In the Landau gauge, it is easy to see that at least the infinitesimally connected gauge copies  can be dealt with by restricting the domain of integration in the path integral to the so called Gribov region $\Omega$ \cite{Gribov:1977wm,Sobreiro:2005ec,Vandersickel:2012tz}. A way to implement the restriction to the region $\Omega$ has been worked out by Gribov in his original work and developed in \ref{introductiontogribov}
%\footnote{We refer to the recent review \cite{Vandersickel:2012tz} for a complete list of references.}. 
%It amounts to impose the no-pole condition for the connected two-point ghost function $\mathcal{G}^{ab}(k;A) = \langle k | \left(  -\partial D^{ab}(A) \right)^{-1} |k\rangle $, which is nothing but the inverse of the Faddeev--Popov operator $-\partial D^{ab}(A)$. One requires that  $\mathcal{G}^{ab}(k;A)$ has no poles at finite non vanishing values of $k^2$, so that it stays always positive. That way one ensures that the Gribov horizon is not crossed, {\it i.e.}~one remains inside $\Omega$. The only allowed pole is at $k^2=0$, which has the meaning of approaching the boundary of the region $\Omega$. In a recent work \cite{Capri:2012wx}, the no-pole condition was linked in a more precise fashion to the Zwanziger implementation of the Gribov restriction \cite{Zwanziger:1989mf}, thereby making clear their equivalence.

%Following the procedure developed in section \ref{introductiontogribov}, for the connected two-point ghost function $\mathcal{G}^{ab}(k;A)$ at first order in the gauge fields,  one finds
%\begin{equation}
%\mathcal{G}^{ab}(k;A) ~=~ \frac{1}{k^{2}}\left[ \delta^{ab} - g^{2}\frac{k_{\mu}k_{\nu}}{k^{2}} \int\!\! \frac{\d^{d}p}{(2\pi)^{d}} \,\, \epsilon^{amc}\epsilon^{cnb} \frac{1}{(k-p)^{2}}A^{m}_{\mu}(p)A^{n}_{\nu}(-p)\right]\; ,
%\label{ghprop}
%\end{equation}
%which, owing to
%\begin{equation}
%\epsilon^{amc}\epsilon^{cnb} ~=~ \delta^{an} \delta^{mb}  - \delta^{ab} \delta^{mn} \;, \label{dd}
%\end{equation}
%can be written as
\begin{equation}
\mathcal{G}^{ab}(k;A) ~=~ \left(
  \begin{array}{ll}
   \delta^{\alpha \beta} \mathcal{G}_{off}(k;A) & \,\,\,\,\,\,\,\,0 \\
   \,\,\,\;\;\;\;\;\;0 & \mathcal{G}_{diag}(k;A)
  \end{array}
\right).
\label{gh prop offdiag}
\end{equation}
By expliciting the ghost form factor we have
\begin{eqnarray}
\mathcal{G}_{off}(k;A) 
~\simeq~  \frac{1}{k^{2}} \left( \frac{1}{1 - \sigma_{off}(k;A)} \right) \;,
\label{gh off}
\end{eqnarray}
and
\begin{eqnarray}
\mathcal{G}_{diag}(k;A) 
~\simeq ~ \frac{1}{k^{2}} \left( \frac{1}{1 - \sigma_{diag}(k;A)} \right)\;,
\label{gh diag}
\end{eqnarray}
where
\begin{subequations} \begin{equation}
\sigma_{off}(0;A) ~=~ \frac{g^{2}}{dV} \int\!\! \frac{\d^{d}p}{(2\pi)^{d}} \;  \frac{1}{p^{2}} \left( \frac{1}{2} A^{\alpha}_{\mu}(p)A^{\alpha}_{\mu}(-p) + A^{3}_{\mu}(p)A^{3}_{\mu}(-p)\right)  \;,
\label{sigma off}
\end{equation}
and
\begin{equation}
\sigma_{diag}(0;A) ~=~ \frac{g^{2}}{dV} \int\!\! \frac{\d^{d}p}{(2\pi)^{d}} \; \frac{1}{p^{2}} A^{\alpha}_{\mu}(p)A^{\alpha}_{\mu}(-p)\;.
\label{sigma diag}
\end{equation} \end{subequations}
In order to obtain expressions  \eqref{sigma off} and \eqref{sigma diag}, where $V$ denotes the (infinte) space-time volume, the transversality of the gluon field and the property that $\sigma(k;A)_{off}$ and $\sigma(k;A)_{diga}$ are decreasing functions of $k$ were used\footnote{For more details concerning the ghost computation see \cite{Capri:2013oja,Capri:2013gha,Capri:2012ah,Vandersickel:2012tz}}. From equations \eqref{gh off} and \eqref{sigma off} one can easily read off the two no-pole conditions. Namely,
\begin{subequations} \begin{equation}
\sigma_{off}(0;A) < 1  \;,
\label{sigmaoffnopole}
\end{equation}
and
\begin{equation}
\sigma_{diag}(0;A) < 1\;.
\label{sigmadiagnopole}
\end{equation} \end{subequations}






%Expressions  \eqref{sigma off} and \eqref{sigma diag} are obtained by taking the limit $k \to 0$ of equations \eqref{gh off},  \eqref{gh diag}, and by making use of the property
%\begin{eqnarray}
%A_{\mu }^{a}(q)A_{\nu }^{a}(-q) &=&\left( \delta _{\mu \nu }-\frac{q_{\mu }q_{\nu }}{q^{2}}\right) \omega (A)(q)  
%\nonumber \\
%&\Rightarrow &\omega (A)(q) ~=~ \frac{1}{(d-1)}A_{\lambda }^{a}(q)A_{\lambda }^{a}(-q)    \;,
%\label{p1}
%\end{eqnarray}
%which follows from the transversality of the gauge field, $q_\mu A^a_\mu(q)=0$. Also, it is useful to remind that, for an arbitrary function $\mathcal{F}(p^2)$, we have
%\begin{equation}
%\int \frac{\d^{d}p}{(2\pi )^{d}}\left( \delta _{\mu \nu }-\frac{p_{\mu }p_{\nu }}{p^{2}}\right) \mathcal{F}(p^2) ~=~ \mathcal{A}\;\delta _{\mu \nu }   \;,
%\label{p2}
%\end{equation}
%where, upon contracting both sides of equation \eqref{p2} with $\delta_{\mu\nu}$,
%\begin{equation}
%\mathcal{A} ~=~ \frac{d-1}{d}\int \frac{\d^{d}p}{(2\pi )^{d}}\mathcal{F}(p^2)\;.   
%\label{p3}
%\end{equation}

At the end, the partition function restricted to the first Gribov region $\Omega$ reads,
\begin{eqnarray}
Z &=&  \int \frac{\d \omega}{2\pi i \omega}\frac{\d \beta}{2\pi i \beta} [\d A] [\d B]  \; \; e^{\omega (1-\sigma_{off})} \, e^{\beta (1-\sigma_{diag})} e^{-S}\;.
\label{ptionfucnt22}
\end{eqnarray}










\section{The $d$-dimensional gluon propagator and the gap equation}

The perturbative computation at the semi-classical level requires only quadratic terms of the full action, defined in eq.\eqref{ptionfucnt22} (with $S$ given by eq.\eqref{Sf}), yielding a Gaussian integral over the fields. Inserting external fields to obtain the boson propagators, one gets, after taking the limit $\xi \to 0$, the following propagators,

%As explained in section \ref{introductiontogribov}, the implementation of the restriction to the first Gribov region $\Omega$ in the functional integral concerns the application of the no-pole conditions \eqref{sigmaoffnopole} and \eqref{sigmadiagnopole}, which is encoded into Heaviside step functions. By making use of its integral representation,
%\begin{equation}
%\theta(x) = \int_{-i\infty +\epsilon}^{i\infty +\epsilon} \frac{d\omega}{2\pi i \omega} \; e^{\omega x}   \;, 
%\label{rp}
%\end{equation}
%we could get for the functional integral,

%By computing up to one-loop order in perturbation theory, we have
%\begin{eqnarray}
%Z_{quad} &=& \mathcal{N'} \int \frac{\d \omega}{2\pi i \omega}\frac{\d \beta}{2\pi i \beta} [\d A^{\alpha}] [\d A^{3}] [\d B] \; e^{\omega}\, e^{\beta} \exp \left\{-\frac{1}{2} \int\!\! \frac{\d^{d}p}{(2\pi)^{d}}\, A^{\alpha}_{\mu}(p) \left[ \left(p^{2} + \frac{\nu^{2}}{2}g^{2} + \right. \right. \right. 
%\nonumber \\
% &{}& \left. \left. + \frac{2g^{2}}{dV}\left(\beta +\frac{\omega}{2}\right)\frac{1}{p^{2}}\right) \left(\delta_{\mu\nu} - \frac{p_{\mu}p_{\nu}}{p^{2}}\right)\right]A^{\alpha}_{\nu}(-p) + A^{3}_{\mu}(p) \left[ \left(p^{2} + \frac{\nu^{2}}{2}g^{2} + \frac{2g^{2}}{dV}\frac{\omega}{p^{2}}\right) \times \right.
%\nonumber \\
%&{}& \left. \times \left(\delta_{\mu\nu} - \frac{p_{\mu}p_{\nu}}{p^{2}}\right) \right]A^{3}_{\nu}(-p)  + B_{\mu}(p) \left[ \left(p^{2} + \frac{\nu^{2}}{2}g'^{2}\right)\left(\delta_{\mu\nu} - \frac{p_{\mu}p_{\nu}}{p^{2}}\right) \right] B_{\nu}(-p) - 
%\nonumber \\
%&{}& \left. - A^{3}_{\mu}(p) \left[\nu^{2}g\,g'\left(\delta_{\mu\nu} - \frac{p_{\mu}p_{\nu}}{p^{2}}\right) \right]B_{\nu}(-p) \right\}\;.
%\label{Zquad1}
%\end{eqnarray}
%It turns out to be convenient to perform the following change of variables
%\begin{subequations} \begin{equation}
%\beta \to \beta -\frac{\omega}{2}
%\end{equation}
%and
%\begin{equation}
%\omega \to \omega\;.
%\end{equation} \end{subequations}
%Therefore, equation \eqref{Zquad1} can be rewritten as,
%\begin{eqnarray}
%Z_{quad} &=& \mathcal{N'} \int \frac{\d \omega}{2\pi i}\frac{\d \beta}{2\pi i} [\d A^{\alpha}] [\d A^{3}] [\d B] \;\delta (\partial A^\alpha) \; \delta (\partial A^3)\; \delta({\partial B}) \;\; e^{-\ln\left(\beta\omega-\frac{\omega^{2}}{2}\right)} e^{\beta}\,e^{\frac{\omega}{2}}\;  \times 
%\nonumber \\
%&{ }& \times \exp \left[-\frac{1}{2} \int \frac{\d^{d}p}{(2\pi)^{d}} A^{\alpha}_{\mu}(p)\, Q^{\alpha \beta}_{\mu \nu}\, A^{\beta}_{\nu}(-p)\right] \times   \exp \left[-\frac{1}{2} \int \frac{\d^{d}p}{(2\pi)^{d}} \left(
% \begin{array}{cc}
%  A^{3}_{\mu}(p) & B_{\mu}(p)
% \end{array} \right)
%\mathcal{P}_{\mu \nu} \left(
%\begin{array}{ll}
%A^{3}_{\nu}(-p) \\
%B_{\nu}(-p)
%\end{array} \right) \right]  \nonumber \\
%\label{Zquad2}
%\end{eqnarray}
%where
%\begin{subequations} \begin{equation}
%Q^{\alpha \beta}_{\mu \nu} ~=~ \left[ p^{2} + \frac{\nu^{2} g^{2}}{2} + \frac{2g^{2}\beta}{dV} \frac{1}{p^{2}} \right] \delta^{\alpha \beta} \left( \delta_{\mu \nu} - \frac{p_{\mu} p_{\nu}}{p^{2}}\right) \label{Qmunu}
%\end{equation}
%and
%\begin{equation}
%\mathcal{P}_{\mu \nu} ~=~ \left(
%                        \begin{array}{cc}
%                         p^{2} + \frac{\nu^{2}}{2}g^{2} + \frac{2 g^{2} \omega }{dV} \frac{1}{p^{2}} & - \frac{\nu^{2}}{2}g\,g' \\
%                         - \frac{\nu^{2}}{2}g\,g' & p^{2} + \frac{\nu^{2}}{2}{g'}^2
%                        \end{array} \right) \left( \delta_{\mu \nu} - \frac{p_{\mu} p_{\nu}}{p^{2}} \right)\;.
%\label{P mu nu}
%\end{equation} \end{subequations}
%From expression \eqref{Zquad2} we can easily deduce the two-point correlation functions of the fields $A^\alpha_\mu$, $A^3_\mu$ and $B_\mu$, namely
\begin{subequations} \label{propsaandb} \begin{gather}
\langle  A^{\alpha}_\mu(p) A^{\beta}_\nu(-p) \rangle ~=~ \frac{p^2}{p^4 + \frac{\nu^2g^2}{2} p^2 + \frac{2g^2\beta}{dV}} \; \delta^{\alpha \beta} \left( \delta_{\mu\nu} - \frac{p_\mu p_\nu}{p^2} \right)  \;,   
\label{aalpha} \\
\langle  A^{3}_\mu(p) A^{3}_\nu(-p) \rangle ~=~ \frac{p^2 \left(p^2 +\frac{\nu^2}{2} g'^{2}\right)}{p^6 + \frac{\nu^2}{2} p^4 \left(g^2 +g'^2 \right)  + \frac{2\omega g^2}{dV} \left( p^2 + \frac{\nu^2 g'^2}{2} \right)} \;  \left( \delta_{\mu\nu} - \frac{p_\mu p_\nu}{p^2} \right)  \;,   \label{a3a3}
\\ %
\langle  B_\mu(p) B_\nu(-p) \rangle ~=~ \frac{ \left(p^4 +\frac{\nu^2}{2} g^{2} p^2+\frac{2\omega g^2}{dV}  \right)}{p^6 + \frac{\nu^2}{2} p^4 \left(g^2 +g'^2 \right)  + \frac{2\omega g^2}{dV} \left( p^2 +  \frac{\nu^2 g'^2}{2} \right)} \;  \left( \delta_{\mu\nu} - \frac{p_\mu p_\nu}{p^2} \right)  \;,   \label{bb}
\\
\langle  A^3_\mu(p) B_\nu(-p) \rangle ~=~  \frac{ \frac{\nu^2}{2} g g'  p^2}{p^6 + \frac{\nu^2}{2} p^4 \left(g^2 +g'^2 \right)  + \frac{2\omega g^2}{dV} \left( p^2 + \frac{\nu^2 g'^2}{2} \right)} \;  \left( \delta_{\mu\nu} - \frac{p_\mu p_\nu}{p^2} \right)  \;.   \label{ba3}
\end{gather} \end{subequations}
Moving to the fields $W^{+}_\mu, W^{-}_\mu, Z_\mu, A_\mu$, one obtains 
\begin{subequations} \label{propszandgamma} \begin{gather}
\langle  W^{+}_\mu(p) W^{-}_\nu(-p) \rangle ~=~ \frac{p^2}{p^4 + \frac{\nu^2g^2}{2} p^2 + \frac{2g^2\beta}{dV} } \;  \left( \delta_{\mu\nu} - \frac{p_\mu p_\nu}{p^2} \right)  \;,   \label{ww}
\\
\langle  Z_\mu(p) Z_\nu(-p) \rangle ~=~ \frac{\left( p^4 +\frac{2\omega}{dV} \frac{g^2 g'^2}{g^2+g'^2}  \right)}{p^6 + \frac{\nu^2}{2} p^4 \left(g^2 +g'^2 \right)  + \frac{2\omega g^2}{dV} \left( p^2 + \frac{\nu^2 g'^2}{2} \right) } \;  \left( \delta_{\mu\nu} - \frac{p_\mu p_\nu}{p^2} \right)  \;,   \label{zz}
\\
\langle  A_\mu(p) A_\nu(-p) \rangle ~=~ \frac{\left( p^4 +\frac{\nu^2}{2} p^2 (g^2+g'^2) +\frac{2\omega}{dV} \frac{g^4}{g^2+g'^2}\right)}{p^6 + \frac{\nu^2}{2} p^4 \left(g^2 +g'^2 \right)  + \frac{2\omega g^2}{dV} \left( p^2 + \frac{\nu^2 g'^2}{2} \right) } \;  \left( \delta_{\mu\nu} - \frac{p_\mu p_\nu}{p^2} \right)  \;,   \label{aa}
\\
\langle  A_\mu(p) Z_\nu(-p) \rangle ~=~ \frac{\frac{2\omega}{dV} \frac{g^3 g'}{g^2+g'^2} }{p^6 + \frac{\nu^2}{2} p^4 \left(g^2 +g'^2 \right)  + \frac{2\omega g^2}{dV} \left( p^2 + \frac{\nu^2 g'^2}{2} \right)} \;  \left( \delta_{\mu\nu} - \frac{p_\mu p_\nu}{p^2} \right)  \;.   \label{az}
\end{gather} \end{subequations}
As expected, all propagators get deeply modified in the IR by the presence of the Gribov parameters $\beta$ and $\omega$. Notice, in particular, that due to the parameter $\omega$ a mixing between the fields $A_\mu$ and $Z_\mu$ arises, eq.\eqref{az}. As such, the original photon and the boson $Z$ loose their distinct particle interpretation.  Moreover, it is straightforward to check that in the limit $\beta \rightarrow 0$ and $\omega \rightarrow 0$, the standards propagators are recovered.
%, {\it i.e.}
%\begin{subequations} \begin{gather}
%\langle  W^{+}_\mu(p) W^{-}_\nu(-p) \rangle {\big |}_{\beta=0} ~=~ \frac{1}{p^2 + \frac{\nu^2g^2}{2} } \;  \left( \delta_{\mu\nu} - \frac{p_\mu p_\nu}{p^2} \right)  \;,   \label{ww0}
%\\
%\langle  Z_\mu(p) Z_\nu(-p) \rangle  {\big |}_{\omega=0} ~=~ \frac{1}{p^2 + \frac{\nu^2}{2} \left(g^2 +g'^2 \right)} \;  \left( \delta_{\mu\nu} - \frac{p_\mu p_\nu}{p^2} \right)  \;,   \label{zz0}
%\\
%\langle  A_\mu(p) A_\nu(-p) \rangle{\big |}_{\omega=0}  ~=~ \frac{1}{p^2}   \;  \left( \delta_{\mu\nu} - \frac{p_\mu p_\nu}{p^2} \right)  \;,   \label{aa0}
%\\
%\langle  A_\mu(p) Z_\nu(-p) \rangle {\big |}_{\omega=0} = 0    \;.   \label{az0}
%\end{gather} 
%\end{subequations}

Let us now proceed by deriving the gap equations which will enable us to (dynamically) fix the Gribov parameters, $\beta$ and $\omega$, as function of $g$, $g'$ and $\nu^2$. Thus, performing the path integral of eq.\eqref{ptionfucnt22}, in the semi-classical level, we get
%\begin{equation}
%Z_{quad} ~=~ \mathcal{N'} \int \frac{\d \omega}{2\pi i}\frac{\d \beta}{2\pi i}\,e^{-\ln\left(\beta\omega -\frac{\omega^{2}}{2}\right)} e^{\frac{\omega}{2}}\,e^{\beta} \left[\det Q_{\mu \nu}^{\alpha \beta} \right]^{-1/2}\, \left[ \det \mathcal{P}_{\mu \nu} \right]^{-1/2},  \label{Zquad3}
%\end{equation}
%with
%\begin{subequations} \begin{equation}
%\left[\det Q_{\mu \nu}^{\alpha \beta} \right]^{-1/2} ~=~ \exp \left[ -\frac{2(d-1)}{2} \int \frac{\d^{d}p}{(2\pi)^{d}}  \;  \log \left(p^{2} + \frac{g^{2}\nu^{2}}{2} + \frac{2g^{2}\beta}{dV} \frac{1}{p^{2}}\right) \right] \label{calc_det1}
%\end{equation}
%and
%\begin{equation}
%\left[ \det \mathcal{P}_{\mu \nu} \right]^{-1/2} ~=~ \exp \left[ - \frac{(d-1)}{2} \int \frac{\d^{d}p}{(2\pi)^{d}} \log \lambda_{+}(p, \omega)\, \lambda_{-}(p, \omega) \right].   \label{calc_det2}
%\end{equation} \end{subequations}
%where $\lambda_{\pm}$ are the eigenvalues of the $2\times 2$ matrix of eq.\eqref{P mu nu}, {\it i.e.}
%\begin{equation}
%\lambda_{\pm} ~=~ \frac{\left( p^{4} + \frac{\nu^{2}}{4} p^{2}(g^{2} + g'^{2}) + \frac{g^{2}\omega}{dV} \right) \pm \sqrt{\left[ \frac{\nu^{2}}{4}(g^{2} + g'^{2})p^{2} + \frac{g^{2}\omega}{dV}\right]^{2} - \frac{\omega}{3}\nu^{2}g^{2}\,g'^{2}p^{2}}}{p^{2}}  \;. \label{ev}
%\end{equation}
%Thus,
%\begin{equation}
%Z_{quad} = \mathcal{N} \int \frac{\d \omega}{2\pi i}\frac{\d \beta}{2\pi i} e^{f(\omega, \beta)} \;, \label{Zf eq}
%\end{equation}
%where
\begin{eqnarray}
f(\omega, \beta) ~=~ \frac{\omega}{2} + \beta - \frac{2(d-1)}{2} \int \frac{\d^{d}p}{(2\pi)^{d}} \; \log \left(p^{2} + \frac{\nu^{2}}{2}g^{2} + \frac{2g^{2}\beta}{dV} \frac{1}{p^{2}}\right) -
\nonumber \\
- \frac{(d-1)}{2} \int \frac{\d^{d}p}{(2\pi)^{d}} \; \log \lambda_{+}(p, \omega)\, \lambda_{-}(p, \omega)\;. 
\label{f eq}
\end{eqnarray}
In eq.\eqref{f eq}, $f(\omega,\beta)$ is defined according to eq.\eqref{Zq3} and
\begin{equation}
\lambda_{\pm} ~=~ \frac{\left( p^{4} + \frac{\nu^{2}}{4} p^{2}(g^{2} + g'^{2}) + \frac{g^{2}\omega}{dV} \right) \pm \sqrt{\left[ \frac{\nu^{2}}{4}(g^{2} + g'^{2})p^{2} + \frac{g^{2}\omega}{dV}\right]^{2} - \frac{\omega}{3}\nu^{2}g^{2}\,g'^{2}p^{2}}}{p^{2}}  \;. 
\label{ev1}
\end{equation}
Making use of the thermodynamic limit, where the saddle point approximation takes place, we have the two gap equations given by\footnote{For more details see \cite{Capri:2013oja,Capri:2013gha,Capri:2012ah}.}

%Following Gribov's framework \cite{Gribov:1977wm,Sobreiro:2005ec,Vandersickel:2012tz}, expression \eqref{Zf eq} is evaluated in a saddle point approximation, {\it i.e.}
%\begin{equation}
%Z_{quad} \simeq e^{f(\omega^{\ast}, \beta^{\ast})}\;,
%\label{Zf eq1}
%\end{equation}
%where $(\beta^*, \omega^*)$ are determined by the stationarity conditions
%\begin{equation}
%\frac{\partial f(\omega, \beta)}{\partial \beta}\bigg{|}_{\beta^{\ast}, \omega^{\ast}} ~=~ \frac{\partial f(\omega, \beta)}{\partial \omega}\bigg{|}_{\beta^{\ast}, \omega^{\ast}} ~=~ 0 \;, \nonumber
%\label{saddle condition}
%\end{equation}
%from which we get the two gap equations: the first one, from the $\omega$ derivative,
%\begin{subequations} \begin{multline}
%\frac{2(d-1)}{2d} g^{2} \int \frac{\d^{d}p}{(2\pi)^{d}}\; \frac{1}{\left [ p^{4} + \frac{\nu^{2}}{4}(g^{2}+g'^{2})p^{2} + \frac{g^{2}\omega^{\ast}}{dV}\right]^{2} - \left[ \frac{\nu^{2}}{4}(g^{2}+g'^{2})p^{2} + \frac{g^{2}\omega^{\ast}}{dV} \right]^{2} + \frac{\nu^{2}g^{2}g'^{2}\omega^{\ast}}{dV} p^{2}} \times 
%\\
%\left\{ \left[ 1+ \frac{\frac{\nu^{2}}{4}(g^{2}+g'^{2})p^{2} + \frac{g^{2}\omega^{\ast}}{dV}  - \frac{\nu^{2}}{2}g'^{2}p^{2}}{\sqrt{\left[ \frac{\nu^{2}}{4}(g^{2}+g'^{2})p^{2} + \frac{g^{2}\omega^{\ast}}{dV} \right]^{2} - \frac{\nu^{2}g^{2}g'^{2}\omega^{\ast}}{dV}p^{2}}}\right] \right. \times 
%\\
%\left[ p^{4} + \frac{\nu^{2}}{4}(g^{2}+g'^{2})p^{2} +  \frac{g^{2}\omega^{\ast}}{dV} - \sqrt{\left[ \frac{\nu^{2}}{4}(g^{2}+g'^{2})p^{2} + \frac{g^{2}\omega^{\ast}}{dV}  \right]^{2} - \frac{\nu^{2}g^{2}g'^{2}\omega^{\ast}}{dV}  p^{2}}\right] + 
%\\
%\left[ 1-\frac{\frac{\nu^{2}}{4}(g^{2}+g'^{2})p^{2} + \frac{g^{4}\omega^{\ast}}{3}  -  \frac{\nu^{2}}{2}g'^{2}p^{2}}{\sqrt{\left[ \frac{\nu^{2}}{4}(g^{2}+g'^{2})p^{2} + \frac{g^{2}\omega^{\ast}}{dV}  \right]^{2} - \frac{\nu^{2}g^{2}g'^{2}\omega^{\ast}}{dV}  p^{2}}} \right] \times 
%\\
%\left. \left[ p^{4}+\frac{\nu^{2}}{4}(g^{2}+g'^{2})p^{2} + \frac{g^{2}\omega^{\ast}}{dV} + \sqrt{\left[ \frac{\nu^{2}}{4}(g^{2}+g'^{2})p^{2}  +  \frac{g^{2}\omega^{\ast}}{dV}  \right]^{2} - \frac{\nu^{2}g^{2}g'^{2}\omega^{\ast}}{dV} p^{2}} \right] \right\}  ~=~ 1\;,
%\label{omega gap eq1}
%\end{multline}
%and the second one, from the $\beta$ derivative,
\begin{equation}
\frac{4(d-1)}{2d}g^{2} \int \frac{\d^{d}p}{(2\pi)^{d}} \;  \frac{1}{p^{4}+\frac{g^{2}\nu^{2}}{2}p^{2} + \frac{2g^{2}\beta^{\ast}}{dV} } ~=~ 1   \;,
\label{beta gap eq}
\end{equation} 
and
%In particular, after a little algebra, eq.\eqref{omega gap eq1} can be considerably simplified, yielding
\begin{equation}
\frac{2(d-1)}{d}g^{2}\int\!\! \frac{\d^{d}p}{(2\pi)^{d}} \; \frac{p^{2} + \frac{\nu^{2}}{2}g'^{2}}{p^{6} + \frac{\nu^{2}}{2}(g^{2} + g'^{2})p^{4} + \frac{2\omega^{\ast}g^{2}}{dV}p^{2} + \frac{\nu^{2}g^{2}\,g'^{2}\omega^{\ast}}{dV} } ~=~ 1 \;.
\label{omega gap eq}
\end{equation}
%

















%\subsubsection{The limit $g' \to 0$.}
%An important check to be done is the case where $g'=0$, which must recover the results of \cite{Capri:2012ah} with the Higgs field in the fundamental representation, obtaining a $U(1)$ massless gauge field decoupled from the $SU(2)$ gauge sector. This decoupling can be easily seen just by setting $g'=0$ in the propagator expressions \eqref{a3a3} - \eqref{ba3} and \eqref{aalpha}:
%\begin{subequations} \label{propgzro} \begin{eqnarray}
%\langle A^{\alpha}_{\mu}(p)A^{\beta}_{\nu}(-p)\rangle &=& \frac{p^{2}}{p^{4} + \frac{\nu^{2}}{2}g^{2}p^{2} + \frac{\beta}{2}g^{2}} \delta^{\alpha\beta}T_{\mu\nu}(p^2)\;,\quad \langle A^{3}_{\mu}(p)A^{3}_{\nu}(-p)\rangle = \frac{p^{2}}{p^{4}+\frac{\nu^{2}}{2}g^{2}p^{2}+\frac{\omega^{2}}{2}g^{2}}T_{\mu\nu}(p^2)\;, \\
%\langle B_{\mu}(p)B_{\nu}(-p)\rangle &=&  \frac{1}{p^{2}} T_{\mu\nu}(p^2)\;,\quad \langle A^{3}_{\mu}(p)B_{\nu}(-p)\rangle = 0\;.
%\end{eqnarray} \end{subequations}
%Also, as in the last section, one should be able to write the propagators in terms of the fields $W^{\pm}$, $Z$ and $A$ obtaining
%\begin{subequations} \label{ppgzro} \begin{eqnarray}
%\langle W^{+}_{\mu}(p)W^{-}_{\nu}(-p)\rangle &=& \frac{p^{2}}{p^{4} + \frac{\nu^{2}}{2}g^{2}p^{2} + \frac{\beta}{2}g^{2}} \delta^{\alpha\beta}T_{\mu\nu}(p^2))\;, \label{ww1} \quad \langle Z_{\mu}(p)Z_{\nu}(-p)\rangle = \frac{p^{2}}{p^{4}+\frac{\nu^{2}}{2}g^{2}p^{2}+\frac{\omega^{2}}{2}g^{2}} T_{\mu\nu}(p^2)\;, \\
%\langle A_{\mu}(p)A_{\nu}(-p)\rangle &=& \frac{1}{p^{2}} T_{\mu\nu}(p^2)\;, \quad
%\langle A_{\mu}(p)Z_{\nu}(-p)\rangle = 0\;.
%\end{eqnarray} \end{subequations}
%These propagators, \eqref{propgzro} and \eqref{ppgzro}, could also be derived by taking $g'=0$ in the quadratic partition function, or even in the generating functional \eqref{Zquad1}, and following the steps of the last section.





Given the difficulties in solving the gap equations \eqref{beta gap eq} and \eqref{omega gap eq}, we propose an alternative approach to probe the gluon propagators in the parameter space $\nu$, $g$ and $g'$. Instead of explicitly solve the gap equations, let us search for the necessity to implement the Gribov restriction. For that we mean to compute $\langle \sigma_{off}(0) \rangle $ and $\langle \sigma_{diag}(0) \rangle$ with the gauge field propagators unchanged by the Gribov terms, {\it i.e.}, before applying the Gribov restriction. Therefore, if $\langle \sigma_{off}(0;A) \rangle  < 1$ and $\langle \sigma_{diag}(0;A) \rangle < 1$ already in this case (without Gribov restrictions), then we would say that there is no need to restrict the domain of integration to $\Omega$. In that case we have, immediately, $\beta^* = \omega^* =0$ and the standard Higgs procedure takes place. Namely, the expression of each ghost form factor is
\begin{eqnarray}
\langle \sigma_{off}(0) \rangle & = &  \frac{(d-1)g^{2}}{d}  \int\!\!  \frac{\d^d p}{(2\pi)^d} \frac{1}{p^{2}}\left(\frac{1}{p^{2} + \frac{\nu^{2}}{2}g^{2}} + \frac{1}{p^{2} + \frac{\nu^{2}}{2}(g^{2}+g'^{2})} \right)  \;.
\label{sgoff1}
\end{eqnarray}
and
\begin{equation}
\langle \sigma_{diag}(0) \rangle ~=~ \frac{2(d-1)g^{2}}{d} \int\!\!\frac{\d^{d}p}{(2\pi)^{d}}\frac{1}{p^{2}}\left(\frac{1}{p^{2} + \frac{\nu^{2}}{2}g^{2}}\right)  \;.
\label{sgdiag1}
\end{equation}





%\subsubsection{About $\sigma_\text{off}(0)$ and $\sigma_\text{diag}(0)$ without the Gribov parameters}
%\label{Evaluation of the ghost form factors}

%Specifically in the next two sections we propose a different approach to check if there exist values of the Higgs condensate $\nu$ and of the coupling $(g,\;g')$ for which both $\langle \sigma_{off}(0) \rangle $ and $\langle \sigma_{diag}(0) \rangle$  already satisfy the no-pole condition
%\begin{equation}
%\langle \sigma_{off}(0;A) \rangle  < 1  \;, \qquad   \langle \sigma_{diag}(0;A) \rangle < 1\;,
%\label{n-pole}
%\end{equation}
%in which case $\beta^\ast$ and/or $\omega^\ast$ could be immediately set equal to zero. Therefore, we present here the expression of both quantities for the $d$-dimensional case. Thus,

%Analogously, for $\langle \sigma_{diag}(0)\rangle$ one gets


%Now that we have in hands all expressions needed to analyze the gluon propagator, let us specialize them to the $d=3$ and $d=4$ cases. With that, we are able to provide a map of the parameter space displaying regions where we have sign of confinement, regions where the Higgs mechanism takes place unaltered and mixed regions where the propagators have ``confined'' and ``deconfined'' contributions in its expressions.




















\subsection{The $d=3$ case} 
\label{d=3}

In the three-dimensional case things become easier since there is no divergences to treat. Therefore, computing the ghost form factors \eqref{sgoff1} and \eqref{sgdiag1} we led to the following conditions 


%Before discussing the gap equations equations \eqref{beta gap eq} and \eqref{omega gap eq}, it is worthwhile to evaluate the vacuum expectation values of the ghost form factors  $\sigma_{off}(0)$ and $\sigma_{diag}(0)$, eqs. \eqref{sigma off} and \eqref{sigma diag}, without taking into account the restriction to the Gribov region, {\it i.e.}~without the presence of the two Gribov parameters $(\beta^{\ast},\omega^{\ast})$. This will enable us to verify if there exist values of the Higgs condensate $\nu$ and of the couplings $(g,g')$ for which both $\langle \sigma_{off}(0) \rangle $ and $\langle \sigma_{diag}(0) \rangle$  already satisfy the no-pole condition
%\begin{equation}
%\langle \sigma_{off}(0;A) \rangle  < 1  \;, \qquad   \langle \sigma_{diag}(0;A) \rangle < 1\;,
%\label{n-pole}
%\end{equation}
%in which case $\beta^\ast$ and/or $\omega^\ast$ could be immediately set equal to zero.

%Let us start by considering $\langle \sigma_{off}(0)\rangle$. From eqs.\eqref{sigma off} and \eqref{AB gluon prop} we easily obtain
%\begin{eqnarray}
%\langle \sigma_{off}(0) \rangle & = &  \frac{2g^{2}}{3}\int\!\!\frac{d^3 p}{(2\pi)^3} \frac{1}{p^{2}}\left(\frac{1}{p^{2} + \frac{\nu^{2}}{2}g^{2}} + \frac{1}{p^{2} + \frac{\nu^{2}}{2}(g^{2}+g'^{2})} \right)  \nonumber \\
% & =& \frac{g^{2}}{3\pi^{2}}\int_{0}^{\infty} \!\!dp\left(\frac{1}{p^{2} + \frac{\nu^{2}}{2}g^{2}} + \frac{1}{p^{2} + \frac{\nu^{2}}{2}(g^{2}+g'^{2})}\right)\;.
%\label{sgoff1}
%\end{eqnarray}
%Analogously, for $\langle \sigma_{diag}(0)\rangle$ one gets
%\begin{equation}
%\langle \sigma_{diag}(0) \rangle = \frac{4g^{2}}{3}\int\!\!\frac{d^{3}p}{(2\pi)^{3}}\frac{1}{p^{2}}\left(\frac{1}{p^{2} + \frac{\nu^{2}}{2}g^{2}}\right)
%= \frac{2g^{2}}{3\pi^{2}}\int_{0}^{\infty}\!\! \frac{dp}{p^{2} + \frac{\nu^{2}}{2}g^{2}}\;.
%\label{sgdiag1}
%\end{equation}
% As
%\begin{eqnarray}
%\int_{0}^{\infty} \frac{dp}{p^{2} + m^{2}} &=& \frac{\pi}{2m}\;,
%\label{decomp1}
%\end{eqnarray}
%we found for $\langle \sigma_{off}(0)\rangle$ and $\langle \sigma_{diag}(0)\rangle$
%\begin{subequations} \label{rmn1} \begin{eqnarray}
%\langle \sigma_{off}(0)\rangle &=& \frac{g}{3\sqrt{2}\pi\nu}(1 + \cos(\theta_{W})) \;, \\
%\langle \sigma_{diag}(0)\rangle &=& \frac{2g}{3\sqrt{2}\pi\nu}\;,
%\end{eqnarray} \end{subequations} 
%where
%\begin{equation}
%\cos(\theta_{W}) = \frac{g}{\sqrt{g^{2}+g'^{2}}}
%\label{thW}
%\end{equation}
%is the Weinberg angle. For the integration domain of the Yang--Mills field to be the configuration space inside the first Gribov horizon, we need that $\langle \sigma_{off}(0)\rangle$ and $\langle \sigma_{diag}(0)\rangle$  be less than one. Thus,


\begin{subequations} 
\label{conds} 
\begin{eqnarray}
(1+\cos(\theta_{W}))\frac{g}{\nu} &<& 3\sqrt{2}\pi \label{firstcond} \\
2\frac{g}{\nu} &<& 3\sqrt{2}\pi \label{secondcond} \;,
\end{eqnarray} 
\end{subequations}
where $\theta(W)$ stands for the Weinberg angle,
\begin{equation}
\cos(\theta_{W}) = \frac{g}{\sqrt{g^{2}+g'^{2}}}\;.
\label{thW}
\end{equation}
These two conditions make phase space fall apart in three regions, as depicted in \figurename\ \ref{regionsdiag1}.
\begin{itemize}
	\item If $g/\nu<3\pi/\sqrt2$, neither Gribov parameter is necessary to make the integration cut off at the Gribov horizon. In this regime the theory is unmodified from the usual perturbative electroweak theory.
	\item In the intermediate case $3\pi/\sqrt2<g/\nu<3\sqrt2\pi/(1+\cos\theta_W)$ only one of the two Gribov parameters,  $\beta$, is necessary. The off-diagonal ($W$) gauge bosons will see their propagators modified due to the presence of a non-zero $\beta$, while the $Z$ boson and the photon $A$ remain untouched.
	\item In the third phase, when $g/\nu>3\sqrt2\pi/(1+\cos\theta_W)$, both Gribov parameters are needed, and all propagators are influenced by them. The off-diagonal gauge bosons are confined. The behaviour of the diagonal gauge bosons depends on the values of the couplings, and the third phase falls apart into two parts, as detailed in section \ref{sect7}.
\end{itemize}
Note that here in the $3$-dimensional $SU(2)\times U(1)+$Higgs case, as well as in the $3d$ $SU(2)+$Higgs treated in section \ref{3dsu2}, an effective coupling constant becomes of utmost importance when discussing the trustworthiness of the our semi-classical results.

\begin{figure}\begin{center}
\includegraphics[width=.25\textwidth]{fourregions.pdf}
\caption{There appear to be four regions in phase space. The region I is defined by condition \eqref{secondcond} and is characterized by ordinary Yang--Mills--Higgs behaviour (massive $W$ and $Z$ bosons, massless photon). The region II is defined by \eqref{firstcond} while excluding all points of region I --- this region only has electrically neutral excitations, as the $W$ bosons are confined (see Section \ref{sect6}); the massive $Z$ and the massless photon are unmodified from ordinary Yang--Mills--Higgs behaviour. Region III has confined $W$ bosons, while both photon and $Z$ particles are massive due to influence from the Gribov horizon; furthermore there is a negative-norm state. In region IV all $SU(2)$ bosons are confined and only a massive photon is left. Mark that the tip of region III is hard to deal with numerically --- the discontinuity shown in the diagram is probably an artefact due to this difficulty.  Details are collected in Section \ref{sect7}. \label{regionsdiag1}}
\end{center}\end{figure}

















\subsubsection{The off-diagonal ($W$) gauge bosons} 
\label{sect6}
Let us first look at the behaviour of the off-diagonal bosons under the influence of the Gribov horizon. The propagator \eqref{ww}  only contains the $\beta$ Gribov parameter, meaning that $\omega$ need not be considered here.

In the regime $g/\nu<3\pi/\sqrt2$ (region I in \figurename\ \ref{regionsdiag1}) the parameter $\beta$ is not necessary, due to the ghost form factor $\langle\sigma_{diag}(0)\rangle$ always being smaller than one. In this case, the off-diagonal boson propagator is simply of massive type, with mass parameter $\frac{\nu^{2}}{2}g^{2}$.
%\begin{equation}
%  \langle W^{+}_{\mu}(p)W^{-}_{\nu}(-p) \rangle = \frac{1}{p^{2} + \frac{\nu^{2}}{2}g^{2}}\left(\delta_{\mu\nu} - \frac{p_{\mu}p_{\nu}}{p^{2}}\right) \;.
%\end{equation}

In the case that $g/\nu>3\pi/\sqrt2$ (regions II, III, and IV in \figurename\ \ref{regionsdiag1}), the relevant ghost form factor is not automatically smaller than one any more, and the Gribov parameter $\beta$ becomes necessary. The value of $\beta^{\ast}$ is determined from the gap equations \eqref{beta gap eq}. After rewriting the integrand in partial fractions, the integral in the equation becomes of standard type, and we readily find the solution
\begin{equation}
  \beta^{\ast} = \frac{3g^2}{32} \left(\frac{g^2}{2\pi^2}-\nu^2\right)^2 \;.
\end{equation}
Mark that, in order to find this result, we had to take the square of both sides of the equation twice. One can easily verify that, in the region $g/\nu>3\pi/\sqrt2$ which concerns us, no spurious solutions were introduced when doing so.

Replacing this value of $\beta^{\ast}$ in the off-diagonal propagator \eqref{ww} one can immediately check that it
%\begin{multline}
%  \langle W_\mu^{+}(p)W_\nu^{-}(-p)\rangle = \frac{\pi/g^3}{\sqrt{\frac{g^2}{4\pi^2}-\nu^2}} \left(\frac{\frac{g^3}{2\pi}\sqrt{\frac{g^2}{4\pi^2}-\nu^2}-\frac i4\nu^2g^2}{p^2+\frac{\nu^2}4g^2+i\frac{g^3}{2\pi}\sqrt{\frac{g^2}{4\pi^2}-\nu^2}} + \frac{\frac{g^3}{2\pi}\sqrt{\frac{g^2}{4\pi^2}-\nu^2}+\frac i4\nu^2g^2}{p^2+\frac{\nu^2}4g^2-i\frac{g^3}{2\pi}\sqrt{\frac{g^2}{4\pi^2}-\nu^2}}\right) \\ \times \left(\delta_{\mu\nu}-\frac{p_\mu p_\nu}{p^2}\right) \;.
%\end{multline}
clearly displays two complex conjugate poles. As such, the off-diagonal propagator  cannot describe a physical excitation of the physical spectrum, being adequate for a confining phase. This means that the off-diagonal components of the gauge field are confined in the region $g/\nu>3\pi/\sqrt2$.
















\subsubsection{The diagonal $SU(2)$ boson and the photon field} \label{sect7}
The other two gauge bosons --- the $A^3_\mu$ and the $B_\mu$ --- have their propagators given by \eqref{a3a3}, \eqref{bb}, and \eqref{ba3} or equivalently --- the $Z_\mu$ and the $A_\mu$ --- by \eqref{zz}, \eqref{aa} and \eqref{az}. Here, $\omega$ is the only one of the two Gribov parameters present.

In the regime $g/\nu<3\sqrt2\pi/(1+\cos\theta_W)$ (regions I and II) this $\omega$ is not necessary to restrict the region of integration to within the first Gribov horizon. Due to this, the propagators are unmodified in comparison to the perturbative case.
%\begin{subequations} \label{propsrewrite} \begin{gather}
%\langle Z_{\mu}(p) Z_{\nu}(-p) \rangle = \frac{1}{p^{2} + \frac{\nu^{2}}{2}(g^{2} + g'^{2})} \left(\delta_{\mu\nu} - \frac{p_{\mu}p_{\nu}}{p^{2}}\right)\;, \\
%\langle A_{\mu}(p) A_{\nu}(-p) \rangle = \frac{1}{p^{2}} \left(\delta_{\mu\nu} - \frac{p_{\mu}p_{\nu}}{p^{2}}\right) \;.
%\end{gather} \end{subequations}

In the region $g/\nu>3\sqrt2\pi/(1+\cos\theta_W)$ (regions III and IV) the Gribov parameter $\omega$ does become necessary, and it has to be computed by solving its gap equation, eq. \eqref{omega gap eq}. Due to its complexity it seems impossible to do so analytically. Therefore we turn to numerical methods. Using Mathematica the gap equation can be straightforwardly solved for a list of values of the couplings. Then we determine the values where the propagators have poles. 

The denominators of the propagators are a polynomial which is of third order in $p^2$. There are two cases: there is a small region in parameter space where the polynomial has three real roots, and for all other values of the couplings there are one real and two complex conjugate roots. In \figurename\ \ref{regionsdiag1} these zones are labelled III and IV respectively. Let us analyze each region separately.

%Ordinarily, one would like to diagonalize the propagator matrix in order to separate the states present in the theory. In our case, however, doing so requires a nonlocal transformation, and the result will contain square roots containing the momentum of the fields. It seems to be more enlightening to, instead, perform a partial fraction decomposition. If we look at the two-point functions of the $A_3$ and $B$ fields \eqref{a3a3}, \eqref{bb}, and \eqref{ba3}, we can succinctly write those as\footnote{The projector $\delta_{\mu\nu}-\frac{p_\mu p_\nu}{p^2}$ will be ignored in this discussion, as it does not change anything nontrivial here.}
%\begin{equation}
%	\Delta_{ij} = \frac{f_{ij}(p^2)}{P(p^2)} \;.
%\end{equation}
%Here, the indices $i,j$ run over $A_3$ and $B$. The functions $f_{ij}(p^2)$ are polynomials of $p^2$ of at most second order, and the function $P(p^2)$ is a third-order polynomial of $p^2$. Furthermore, if we consider the functions $f_{ij}(p^2)$ to be the elements of $2\times2$ matrix, we can see that the determinant of this matrix is nothing but $p^2P(p^2)$.

%Let us assume that we know what the roots of $P(p^2)$ are, and call them $-m^2_n$ with $n=1,2,3$. It is then obvious that we can rewrite $P(p^2)$ as $(p^2+m_1^2)(p^2+m_2^2)(p^2+m_3^2)$. We can then perform a decomposition in partial fractions. We will have something of the form
%\begin{equation}
%	\frac{f_{ij}(p^2)}{P(p^2)} = \sum_{n=1}^3 \frac{\alpha_{ij,n}}{p^2+m_n^2} \;.
%\end{equation}
%The constants $\alpha_{ij,n}$ can be readily determined the usual way and we get
%\begin{equation}
%	\frac{f_{ij}(-m_1^2)}{(-m_1^2+m_2^2)(-m_1^2+m_3^2)} = \alpha_{ij,1}
%\end{equation}
%and analogously for $n=2,3$. In conclusion we find
%\begin{multline}
%	\Delta_{ij} = \frac{f_{ij}(p^2)}{P(p^2)} = \frac{f_{ij}(-m_1^2)}{(-m_1^2+m_2^2)(-m_1^2+m_3^2)} \frac1{p^2+m_1^2} \\ + \frac{f_{ij}(-m_2^2)}{(m_1^2-m_2^2)(-m_2^2+m_3^2)} \frac1{p^2+m_2^2} + \frac{f_{ij}(-m_3^2)}{(m_1^2-m_3^2)(m_2^2-m_3^2)} \frac1{p^2+m_3^2} \;.
%\end{multline}
%We can again interpret the constants $f_{ij}(-m_n^2)$ as elements of some $2\times2$ matrices, and we find that the determinants of these matrices are equal to $-m_n^2P(-m_n^2) = 0$, as the $-m_n^2$ are roots of the polynomial $P(p^2)$. Now it is obvious that a $2\times2$ matrix $\mx A$ with zero determinant can always be written in the form $\mx A = v v^T$ with $v$ some $2\times1$ matrix. Furthermore, this vector $v$ has norm $v^Tv = \tr\mx A$. This means that we can write our matrices in the form $\mx A = \tr\mx A \hat v \hat v^T$ where $\hat v$ is now the unit vector parallel to $v$. Therefore, let us write $f_{ij}(-m_n^2) = (f_{11}(-m_n^2)+f_{22}(-m_n^2)) \hat v_i^n \hat v_j^n$, resulting in
%\begin{multline} \label{partfracdec}
%	\Delta_{ij} = \frac{f_{11}(-m_1^2)+f_{22}(-m_1^2)}{(-m_1^2+m_2^2)(-m_1^2+m_3^2)} \frac1{p^2+m_1^2} \hat v_i^1\hat v_j^1 \\ + \frac{f_{11}(-m_2^2)+f_{22}(-m_2^2)}{(m_1^2-m_2^2)(-m_2^2+m_3^2)} \frac1{p^2+m_2^2} \hat v_i^2\hat v_j^2 + \frac{f_{11}(-m_3^2)+f_{22}(-m_3^2)}{(m_1^2-m_3^2)(m_2^2-m_3^2)} \frac1{p^2+m_3^2} \hat v_i^3\hat v_j^3 \;.
%\end{multline}
%The vectors $v_i^n$ can be interpreted as linear combinations of the $A_3$ and $B$ fields. Decomposing the two-point functions in this way, we thus find three ``states'' $v_1^n A_3 + v_2^n B$. These states are not orthogonal to each other (which would be impossible for three vectors in two dimensions). The coefficients in front of the Yukawa propagators will be the residues of the poles, and they have to be positive for a pole to correspond to a physical excitation.  The poles can be extracted from the zeros at $p^2_\ast$ of $P(p^2)=p^6+\frac{\nu^2}{2}p^4(g^2+g'^2)+\frac{g^2\omega}{3}(2p^2+\nu^2g'^2)$, viz.
%\begin{subequations} \begin{eqnarray}
% p^2_\ast&=& \frac{1}{6}\left\{ (g^{2}+g'^{2})\nu^{2} + \left[ (g^{2}+g'^{2})^{2}\nu^{4} - 8g^{2}\omega\right]\left[ (g^{2}+g'^{2})^{3}\nu^{6} - 12g^{2}(g^{2}-2g^{2})\nu^{2}\omega +  \right. \right. \nonumber \\
%&{}&+ \left. 2\sqrt{2}\sqrt{ g^{2}\omega \left( 9g'^{2}(g^{2}+g'^{2})^{3}\nu^{8} - 6g^{2}(g^{4}+20g^{2}g'^{2}-8g'^{4})\nu^{4}\omega + 64g^{4}\omega^{2}  \right)} \right]^{-1/3} + \nonumber \\
%&{}&+ \left[ (g^{2}+g'^{2})^{3}\nu^{6} - 12g^{2}(g^{2}-2g^{2})\nu^{2}\omega + 2\sqrt{2}\left( g^{2}\omega \left( 9g'^{2}(g^{2}+g'^{2})^{3}\nu^{8} - \right. \right. \right. \nonumber \\
%&{}&- \left. \left. \left. \left. 6g^{2}(g^{4}+20g^{2}g'^{2}-8g'^{4})\nu^{4}\omega + 64g^{4}\omega^{2}  \right)\right)^{1/2}  \right]^{1/3} \right\}
%\label{m1}
%\end{eqnarray}
%\begin{eqnarray}
% p^2_\ast&=& \frac{1}{6}\left\{  (g^{2}+g'^{2})\nu^{2} - \frac{1}{2} \left[ (g^{2}+g'^{2})^{2}\nu^{4} - 8g^{2}\omega\right]\left[ (g^{2}+g'^{2})^{3}\nu^{6} - 12g^{2}(g^{2}-2g^{2})\nu^{2}\omega +   \right. \right. \nonumber \\
%&{}&+ \left. 2\sqrt{2}\sqrt{ g^{2}\omega \left( 9g'^{2}(g^{2}+g'^{2})^{3}\nu^{8} - 6g^{2}(g^{4}+20g^{2}g'^{2}-8g'^{4})\nu^{4}\omega + 64g^{4}\omega^{2}  \right)} \right]^{-1/3} - \nonumber \\
%&{}&- \frac{1}{2}\left[ (g^{2}+g'^{2})^{3}\nu^{6} - 12g^{2}(g^{2}-2g^{2})\nu^{2}\omega + 2\sqrt{2}\left( g^{2}\omega \left( 9g'^{2}(g^{2}+g'^{2})^{3}\nu^{8} - \right. \right. \right. \nonumber \\
%&{}&- \left. \left. \left. \left. 6g^{2}(g^{4}+20g^{2}g'^{2}-8g'^{4})\nu^{4}\omega + 64g^{4}\omega^{2}  \right)\right)^{1/2}  \right]^{1/3} \right. \nonumber \\
%&{}&+ i\frac{\sqrt{3}}{2}\left[ (g^{2}+g'^{2})^{2}\nu^{4} - 8g^{2}\omega\right]\left[ (g^{2}+g'^{2})^{3}\nu^{6} - 12g^{2}(g^{2}-2g^{2})\nu^{2}\omega +  \right. \nonumber \\
%&{}&+ \left. 2\sqrt{2}\sqrt{ g^{2}\omega \left( 9g'^{2}(g^{2}+g'^{2})^{3}\nu^{8} - 6g^{2}(g^{4}+20g^{2}g'^{2}-8g'^{4})\nu^{4}\omega + 64g^{4}\omega^{2}  \right)} \right]^{-1/3} + \nonumber \\
%&{}&+ i\frac{\sqrt{3}}{2}\left[ (g^{2}+g'^{2})^{3}\nu^{6} - 12g^{2}(g^{2}-2g^{2})\nu^{2}\omega + 2\sqrt{2}\left( g^{2}\omega \left( 9g'^{2}(g^{2}+g'^{2})^{3}\nu^{8} - \right. \right. \right. \nonumber \\
%&{}&- \left. \left. \left. \left. 6g^{2}(g^{4}+20g^{2}g'^{2}-8g'^{4})\nu^{4}\omega + 64g^{4}\omega^{2}  \right)\right)^{1/2}  \right]^{1/3} \right\}\;,
%\label{m2}
%\end{eqnarray}
%and
%\begin{eqnarray}
% p^2_\ast &=& \frac{1}{6}\left\{  (g^{2}+g'^{2})\nu^{2} - \frac{1}{2} \left[ (g^{2}+g'^{2})^{2}\nu^{4} - 8g^{2}\omega\right]\left[ (g^{2}+g'^{2})^{3}\nu^{6} - 12g^{2}(g^{2}-2g^{2})\nu^{2}\omega +   \right. \right. \nonumber \\
%&{}&+ \left. 2\sqrt{2}\sqrt{ g^{2}\omega \left( 9g'^{2}(g^{2}+g'^{2})^{3}\nu^{8} - 6g^{2}(g^{4}+20g^{2}g'^{2}-8g'^{4})\nu^{4}\omega + 64g^{4}\omega^{2}  \right)} \right]^{-1/3} - \nonumber \\
%&{}&- \frac{1}{2}\left[ (g^{2}+g'^{2})^{3}\nu^{6} - 12g^{2}(g^{2}-2g^{2})\nu^{2}\omega + 2\sqrt{2}\left( g^{2}\omega \left( 9g'^{2}(g^{2}+g'^{2})^{3}\nu^{8} - \right. \right. \right. \nonumber \\
%&{}&- \left. \left. \left. \left. 6g^{2}(g^{4}+20g^{2}g'^{2}-8g'^{4})\nu^{4}\omega + 64g^{4}\omega^{2}  \right)\right)^{1/2}  \right]^{1/3} \right. \nonumber \\
%&{}&- i\frac{\sqrt{3}}{2}\left[ (g^{2}+g'^{2})^{2}\nu^{4} - 8g^{2}\omega\right]\left[ (g^{2}+g'^{2})^{3}\nu^{6} - 12g^{2}(g^{2}-2g^{2})\nu^{2}\omega +  \right. \nonumber \\
%&{}&+ \left. 2\sqrt{2}\sqrt{ g^{2}\omega \left( 9g'^{2}(g^{2}+g'^{2})^{3}\nu^{8} - 6g^{2}(g^{4}+20g^{2}g'^{2}-8g'^{4})\nu^{4}\omega + 64g^{4}\omega^{2}  \right)} \right]^{-1/3} + \nonumber \\
%&{}&- i\frac{\sqrt{3}}{2}\left[ (g^{2}+g'^{2})^{3}\nu^{6} - 12g^{2}(g^{2}-2g^{2})\nu^{2}\omega + 2\sqrt{2}\left( g^{2}\omega \left( 9g'^{2}(g^{2}+g'^{2})^{3}\nu^{8} - \right. \right. \right. \nonumber \\
%&{}&- \left. \left. \left. \left. 6g^{2}(g^{4}+20g^{2}g'^{2}-8g'^{4})\nu^{4}\omega + 64g^{4}\omega^{2}  \right)\right)^{1/2}  \right]^{1/3} \right\}\;,
%\label{m3}
%\end{eqnarray} \end{subequations}

%It is obviously not possible to disentangle the three (vector-like) degrees of freedom corresponding to these different masses using only two fields. Nonetheless, it is possible to derive a diagonalization of the $1PI$ propagator matrix. After the saddle point approximation and using the $A_\mu$ and $Z_\mu$ field variables, it is not difficult to see that the tree level $(A_\mu,Z_\mu)$ sector of the action arising from eq.~(\ref{Zquad1}) can be reformulated as
%\begin{multline}\label{dd1}
%  \int d^4p \left(\frac{1}{2}Z_\mu(p)(p^2+g^2\nu^2)Z_\mu(-p)+\frac{1}{2}A_\mu(p)p^2A_\mu(-p)+i\frac{1}{2}\sqrt{\frac{2}{3}\omega_\ast}\left((g'A_\mu(p)-gZ_\mu(p))V_\mu(-p)+(p\leftrightarrow-p)\right)\right. \\ \left.+ \frac{1}{2}V_\mu(p)p^2V_\mu(-p)\right)
%\end{multline}
%while working immediately on-shell, viz.~using $\p_\mu A_\mu=\p_\mu Z_\mu=0$. The equivalence with the original action can be straightforwardly established by integrating over the $V_\mu$ field. Here, we introduced the latter auxiliary field by hand, but it can be shown in general --- at least for the pure Yang--Mills case; for the current Yang--Mills--Higgs generalization this deserved further investigation at a later stage --- that the all-order no pole condition can be brought in local form by introducing a suitable set of boson and fermion auxiliary fields\footnote{These ghost fields are necessary to eliminate the determinant when integrating over the extra fields.}, see e.g.~\cite{Vandersickel:2012tz,Zwanziger:1989mf,Baulieu:2009ha,Capri:2012wx}.

%Having now 3 fields at our disposal with still 3 masses, there is better hope to diagonalize the previous action. First of all, the special limits $g'\to0$ and/or $\omega\to0$ are simply clear at the level of the action (\ref{dd1}). Secondly, the $1PI$ propagator matrix of (\ref{dd1}) only displays a $p^2$-dependence on the diagonal, each time of the form\footnote{In the formulation without the $V_\mu$ field this is not the case. One of the consequences is the appearance of the aforementioned momentum dependent square roots when a diagonalization in terms of the two fields $A_\mu$ and $Z_\mu$ is attempted.} $p^2+\ldots$. As such, the 3 eigenvalues will be of the form $p^2+m_i^2$. Upon using the associated eigenvectors, the action (\ref{dd1}) can then be simply diagonalized to
%\begin{equation}\label{dd2}
%  \int d^4p \left(\frac{1}{2}\lambda_\mu(p)(p^2+m_1^2)\lambda_\mu(-p)+\frac{1}{2}\eta_\mu(p)(p^2+m_2^2)\eta_\mu(-p)+\frac{1}{2}\kappa_\mu(p)(p^2+m_3^2)\kappa_\mu(-p)\right)\,,
%\end{equation}
%where the $\lambda_\mu,\eta_\mu,\kappa_\mu$ are the ``generalized $i$-particles'' of the current model, adopting the language of \cite{Baulieu:2009ha}. They are related to the original fields $A_\mu$, $Z_\mu$ and $V_\mu$ by momentum-independent linear transformations. The quadratic form appearing in \eqref{dd2} displays a standard propagator structure, with the possibility that two of the mass poles can be complex conjugate.






\subsubsection{Three real roots (region III)}



%\begin{figure}\begin{center}
%\includegraphics[width=.5\textwidth]{threems.pdf}
%\caption{The mass-squareds of the massive excitations found in the region where there are three massive poles (region III). \label{threems}}
%\end{center}\end{figure}

Region III is defined by the polynomial in the denominators of \eqref{a3a3}, \eqref{bb}, and \eqref{ba3} having three real roots. This region is sketched in \figurename\ \ref{regionsdiag1}. (Mark that the tip of the region is distorted due to the difficulty in accessing this part numerically.) 

%The square of the masses corresponding to these three roots are plotted in \figurename\ \ref{threems}.

The residues of related to these poles were computed numerically. Only the two of the three roots have positive residue and can correspond to physical states. Those are the one with highest and the one with lowest mass squared. The third of the roots, the one of intermediate value, has negative residue and thus belongs to some negative-norm state, which cannot be physical.

All three states have non-zero mass for non-zero values of the electromagnetic coupling $g'$, with the lightest of the states becoming massless in the limit $g'\to0$. In this limit we recover the behaviour found in this regime in the pure $SU(2)$ case \cite{Capri:2012cr} (the $Z$-boson field having one physical and one negative-norm pole in the propagator) with a massless fermion decoupled from the non-Abelian sector.






\subsubsection{One real root (region IV)}
In the remaining part of parameter space, there is only one state with real mass-squared. The two other roots of the polynomial in the denominators of \eqref{a3a3}, \eqref{bb}, and \eqref{ba3} have non-zero imaginary part and are complex conjugate to each other. In order to determine whether the pole coming from the real root corresponds to a physical particle excitation, we computed its residue, which can be read off in the partial fraction decomposition (the result is plotted in \figurename\ \ref{resrealmass}). It turns out the residue is always positive, meaning that this excitation has positive norm and can thus be interpreted as a physical, massive particle. The poles coming from the complex roots cannot, of course, correspond to such physical excitations.

%\begin{figure}\begin{center}
%\includegraphics[width=.5\textwidth]{realmass.pdf}
%\caption{The mass-squared of the one physical massive excitation found in region IV. \label{realmass}}
%\end{center}\end{figure}

\begin{figure} \begin{center}
\includegraphics[width=.5\textwidth]{res.pdf}
\caption{The residue of the pole of the photon propagator. It turns out to be positive for all values of the couplings within the region IV. \label{resrealmass}}
\end{center} \end{figure}

%\begin{figure}\begin{center}
%\includegraphics[width=.5\textwidth]{realpartccmass.pdf}\includegraphics[width=.5\textwidth]{impartccmass.pdf}
%\caption{The real (left) and imaginary (right) parts of the mass-squared of the other two, complex conjugate, poles. \label{ccmass}}
%\end{center}\end{figure}

In the limit $g'\to0$ we once more recover the corresponding results already found in the pure $SU(2)$ case \cite{Capri:2012cr} (two complex conjugate poles in the propagator of the non-Abelian boson field) plus a massless photon not influenced by the non-Abelian sector.



We shall emphasise here the complexity of the found ``phase spectrum'' in the $3d$ case. For the most part of the $(g'/\nu, g/\nu)$ plane we found the diagonal component of the bosonic field displaying a mix of physical and non-physical contributions, regarding the regions III and IV. The off-diagonal component was found to have physical meaning only in the region I. The transition between those regions was found to be continuous with respect to the effective perturbative parameter $\sim g/\nu$.


















\subsection{The $d=4$ case}
\label{d=4}


%In this section we define the notation and conventions used throughout the paper. The system we study is the electroweak sector of the $4d$ standard model, thereby generalizing our earlier works \cite{Capri:2012cr,Capri:2012ah,Capri:2013gha}. In the following sections we consider in detail the possible effects of Gribov copies in such a setting, based on the original Gribov no-pole analysis \cite{Gribov:1977wm,Sobreiro:2005ec}. A summary of our results can be found in section 6.












%\subsubsection{The vacuum energy}
%Looking at the above propagators, beside the decoupling of the $U(1)$ gauge field from the $SU(2)$ gauge field, one should note the likeness between the diagonal and off-diagonal propagators, though in general the two Gribov parameters, $\omega^\ast$ and $\beta^\ast$, differ. Therefore, given the important role played by the gap equations, it seems to be worth to analyze the two gap equations \eqref{beta gap eq} and \eqref{omega gap eq} in the limit $g'\to 0$. In this limit, the gap equations \eqref{beta gap eq} and \eqref{omega gap eq} become
%\begin{equation}
%\label{omega gapeq g'zero}
%\frac{3}{2}g^{2} \int\!\!\frac{d^{4} p}{(2\pi)^{4}}\, \frac{1}{p^{4}+\frac{\nu^{2}}{2}g^{2}p^{2}+\frac{\omega^{\ast}}{2}g^{2}} = 1\;, \qquad
%\frac{3}{2}g^{2} \int\!\! \frac{d^{4}p}{(2\pi)^{4}}\, \frac{1}{p^{4}+\frac{\nu^{2}}{2}g^{2}p^{2}+\frac{g^{2}}{2}\beta^{\ast}} =1\;.
%\end{equation}
%It is clear that these two equations are identical. Thus, when $g' \to 0$, there is only one gap equation and, therefore, only one Gribov parameter. Consequently, the diagonal and off-diagonal propagators of \eqref{propgzro} are identical. These results coincide with what was found in \cite{Capri:2012ah} in the case of Higgs field in the fundamental representation.

%Furthermore, from equation \eqref{Zf eq}, we get for the vacuum energy,
%\begin{equation}
%\mathcal{E}_{\omega^{\ast}} = \frac{3}{2}\omega^{\ast} - \frac{9}{2}g^2 \int\!\! \frac{d^{4}p}{(2\pi)^{4}}\log \! \left(p^{4} + p^{2}\frac{\nu^{2}}{2}g^{2} + \frac{\omega^{\ast}}{2}g^{2}\right)\;,
%\end{equation}
%which, again, is in agreement with the expression for the vacuum energy for the case of $SU(2)$ in the fundamental representation, upon redefining  $\frac{\omega^{\ast}}{2} = \frac{\beta^{\ast}}{3}$.

















%\subsubsection{About $\sigma_\text{off}(0)$ and $\sigma_\text{diag}(0)$ without the Gribov parameters} 
%\label{sect3}
%Before trying to solve the gap equations, it seems to be worthwhile to study what happens with $\sigma_\text{off}(0)$ and $\sigma_\text{diag}(0)$ in the absence of the Gribov parameters, which will allow us to search for regions where the Gribov parameters $\omega$ and $\beta$ are unnecessary, which happens whenever $\sigma_\text{off}(0)$ and/or $\sigma_\text{diag}(0)$ are less than one.

%Thus, we have
%\begin{equation}
%\label{sgoff} \langle \sigma_\text{off}(0) \rangle = \frac{3g^{2}}{4}\int\!\!\frac{d^{4}p}{(2\pi)^{4}}\,\frac{1}{p^{2}}\left(\frac{1}{p^{2}+\frac{\nu^{2}}{2}g^{2}}+\frac{1}{p^{2}+\frac{\nu^{2}}{2}(g^{2}+g'^{2})}\right)\,,\qquad
%\langle \sigma_\text{diag}(0) \rangle = \frac{3g^{2}}{2}\int\!\!\frac{d^{4}p}{(2\pi)^{4}}\,\frac{1}{p^{2}}\frac{1}{\left(p^{2}+\frac{\nu^{2}}{2}g^{2}\right)}\;.
%\end{equation}


In the $4$-dimensional case the diagonal and off-diagonal ghost form factors read, using the standard $\MSbar$ renormalization procedure,
\begin{equation}
\label{constas} 
\langle \sigma_\text{off}(0) \rangle = 1 - \frac{3g^{2}}{32\pi^{2}}\ln\frac{2a}{\cos(\theta_{W})}
\,, \qquad
\langle \sigma_\text{diag}(0) \rangle = 1-\frac{3g^{2}}{32\pi^{2}}\ln(2a)\;,
\end{equation}
where
\begin{equation}
\label{consta} a = \frac{\nu^{2}g^{2}}{4\bar{\mu}^{2}\,e^{1-\frac{32 \pi^{2}}{3g^{2}}}}\,, \qquad a' = \frac{\nu^{2}(g^{2}+g'^{2})}{4\bar{\mu}^{2}\,e^{1-\frac{32 \pi^{2}}{3g^{2}}}} = a \frac{g^{2}+g'^{2}}{g^{2}} = \frac{a}{\cos^{2}(\theta_{W})}
\end{equation}
and $\theta_{W}$ stands for the Weinberg angle. With expression \eqref{constas} we are able to identify three possible regions, depicted in \figurename\ \ref{regionsdiag}:
\begin{itemize}
\item Region I, where $\langle \sigma_\text{diag}(0)\rangle < 1$ and $\langle \sigma_\text{off}(0)\rangle < 1$, meaning $2a > 1$. In this case the Gribov parameters are both zero so that we have the massive $W^{\pm}$ and $Z$, and a massless photon. That region can be identified with the ``Higgs phase''.
\item Region II, where $\langle \sigma_\text{diag}(0)\rangle > 1$ and $\langle \sigma_\text{off}(0)\rangle < 1$, or equivalently $\cos \theta_{W} < 2a < 1$. In this region we have $\omega = 0$ while $\beta \neq 0$, leading to a modified $W^{\pm}$ propagator, and a free photon and a massive $Z$ boson.
\item The remaining parts of parameter space, where $\langle \sigma_\text{diag}(0)\rangle > 1$ and $\langle \sigma_\text{off}(0)\rangle > 1$, or $0 < 2a < \cos\theta_{W}$. In this regime we have both $\beta \neq 0$ and $\omega \neq 0$, which modifies the $W^{\pm}$, $Z$ and photon propagators. Furthermore this region will fall apart in two separate regions III and IV due to different behaviour of the propagators (see \figurename\ \ref{regionsdiag}).
\end{itemize}

















\subsubsection{The off-diagonal gauge bosons} \label{sect4}
Let us first look at the behaviour of the off-diagonal bosons under the influence of the Gribov horizon. The propagator \eqref{ww} only contains the $\beta$ Gribov parameter, meaning $\omega$ does not need be considered here.

As found in the previous section, this $\beta$ is not necessary in the regime $a>1/2$, due to the ghost form factor $\langle\sigma_\text{diag}(0)\rangle$ always being smaller than one. In this case, the off-diagonal boson propagator is simply of the massive type.
%\begin{equation}
%	\langle W^{+}_{\mu}(p)W^{-}_{\nu}(-p) \rangle = \frac{1}{p^{2} + \frac{\nu^{2}}{2}g^{2}}T_{\mu\nu}(p^2) \;.
%\end{equation}

In the case that $a<1/2$, the relevant ghost form factor is not automatically smaller than one anymore, and the Gribov parameter $\beta$ becomes necessary. The value of $\beta$ is given by the gap equations \eqref{beta gap eq}, which has exactly the same form as in the case without electromagnetic sector. Therefore the results will also be analogous. As the analysis is quite involved, we just quote the results here.

For $1/e<a<1/2$ the off-diagonal boson field has two real massive poles in its two-point function. One of these has a negative residue, however. This means we find one physical massive excitation, and one unphysical mode in this regime. When $a<1/e$ the two poles acquire a non-zero imaginary part and there are no poles with real mass-squared left. In this region the off-diagonal boson propagator is of Gribov type, and the $W$ boson is completely removed from the spectrum. More details can be found in \cite{Capri:2012ah}.


\begin{figure}\begin{center}
\parbox{.5\textwidth}{\includegraphics[width=.5\textwidth]{regions3d.pdf}} \quad \parbox{.4\textwidth}{\includegraphics[width=.4\textwidth]{regionsslice.pdf}}
\caption{Left is a plot of the region $a'<1/2$ (the region $a'>1/2$ covers all points with higher $\nu$). In red are points where the polynomial in the denominator of \eqref{a3a3} - \eqref{ba3} has three real roots, and in blue are the points where it has one real and two complex conjugate roots. At the right is a slice of the phase diagram for $g=10$. The region $a>1/2$ and $a'>1/2$ is labelled I, the region $a<1/2$ and $a'>1/2$ is II, and the region $a<1/2$ and $a'<1/2$ is split into the regions III (polynomial in the denominator of \eqref{a3a3} - \eqref{ba3} has three real roots, red dots in the diagram at the left) and IV (one real and two complex conjugate roots, blue dots in the diagram at the left). The dashed line separates the different regimes for off-diagonal SU(2) bosons (two real massive poles above the line, two complex conjugate poles below). \label{regionsdiag}}
\end{center}\end{figure}















\subsubsection{The diagonal SU(2) boson and the photon} \label{zandgamma}
The two other gauge bosons --- the diagonal SU(2) boson and the photon, $Z_\mu$ and the $A_\mu$ --- have their propagators given by \eqref{zz}, \eqref{aa} and \eqref{az}. Here, $\omega$ is the only of the two Gribov parameters present.

In the regime $a'>1/2$, $\omega$ is not necessary to restrict the region of integration to $\Omega$. Due to this, the propagators are unmodified in comparison to the perturbative case.
%\begin{equation} \label{propsrewrite}
%	\langle Z_{\mu}(p) Z_{\nu}(-p) \rangle = \frac{1}{p^{2} + \frac{\nu^{2}}{2}(g^{2} + g'^{2})} T_{\mu\nu}(p^2)\;, \quad
%	\langle A_{\mu}(p) A_{\nu}(-p) \rangle = \frac{1}{p^{2}}T_{\mu\nu}(p^2) \;.
%\end{equation}

In the region $a'<1/2$ the Gribov parameter $\omega$ does become necessary, and it has to be computed by solving its gap equation. Due to its complexity it seems impossible to compute analytically. Therefore we turn to numerical methods. 
%Using Mathematica the gap equation can be straighforwardly solved for a list of values for the couplings. In order to do this, we regularize the momentum integration by subtracting a term designed to cancel the large-$p^2$ divergence (as in the Pauli--Villars procedure):
%\begin{equation}
%	\frac{3}{2}g^{2}\int \frac{d^{4}p}{(2\pi)^{4}} \left(\frac{p^{2} + \frac{\nu^{2}}{2}g'^{2}}{p^{6} + \frac{\nu^{2}}{2}(g^{2} + g'^{2})p^{4} + \frac{\omega^{\ast}}{2}g^{2}p^{2} + \frac{\omega^{\ast}}{4}\nu^{2}g^{2}g'^{2}} - \frac1{(p^2+M^2)^2}\right) 	+ \frac{3}{2}g^{2}\int \frac{d^{4}p}{(2\pi)^{4}}\frac1{(p^2+M^2)^2} = 1 \;,
%\end{equation}
%where $M^2$ is an arbitrary mass scale. The second integral is readily computed by hand, whereas the first one converges and can be determined numerically. 
Once the parameter $\omega$ has been (numerically) determined, we look at the propagators to investigate the nature of the spectrum.

%The denominators of the propagators are a polynomial which is of third order in $p^2$. There are two cases: there is a small region in parameter space where the polynomial has three real roots, and for all other values of the couplings there are one real and two complex conjugate roots. In \figurename\ \ref{regionsdiag} these zones are labeled III and IV respectively.

%Just as in \cite{Capri:2013gha} we can decompose the propagator matrix $\Delta_{ij}$ (were $i$ and $j$ run over the fields $A_\mu^3$ and $B_\mu$) as
%\begin{eqnarray}
%	\Delta_{ij} &=& \frac{f_{11}(-m_1^2)+f_{22}(-m_1^2)}{(-m_1^2+m_2^2)(-m_1^2+m_3^2)} \frac1{p^2+m_1^2} \hat v_i^1\hat v_j^1 \nonumber \\
%	&+& \frac{f_{11}(-m_2^2)+f_{22}(-m_2^2)}{(m_1^2-m_2^2)(-m_2^2+m_3^2)} \frac1{p^2+m_2^2} \hat v_i^2\hat v_j^2 + \frac{f_{11}(-m_3^2)+f_{22}(-m_3^2)}{(m_1^2-m_3^2)(m_2^2-m_3^2)} \frac1{p^2+m_3^2} \hat v_i^3\hat v_j^3 \;. \label{partfracdec}
%\end{eqnarray}
%The vectors $v_i^n$ can be interpreted as linear combinations of the $A_3$ and $B$ fields. Decomposing the two-point functions in this way, we thus find three ``states'' $v_1^n A_3 + v_2^n B$. These states are not orthogonal to each other (which would be impossible for three vectors in two dimensions). The coefficients in front of the Yukawa propagators will be the residues of the poles, and they have to be positive for a pole to correspond to a physical excitation. From the analysis in \cite{Capri:2013gha} it also follows that $f_{11}(-m_n^2)+f_{22}(-m_n^2)$ will always be positive if the mass squared is real, which helps to determine the sign of the residue without having to explicitly compute it. The same results can also be attained by introducing ``generalized $i$-particles'' (see \cite{Baulieu:2009ha,Sorella:2010it}).

As the model under consideration depends on three dimensionless parameters ($g$, $g'$ and $\nu/\bar\mu$), it is not possible to plot the parameter dependence of these masses in a visually comprehensible way. Therefore we limit ourselves to discussing the behaviour we observed.

In region III, when there are three real poles in the full two-point function, it turns out that only the two of the three roots we identified have a positive residue and can correspond to physical states, being the one with highest and the one with lowest mass squared. The third one, the root of intermediate value, has negative residue and thus belongs to some negative-norm state, which cannot be physical. All three states have non-zero mass for non-zero values of the electromagnetic coupling $g'$, with the lightest of the states becoming massless in the limit $g'\to0$. In this limit we recover the behaviour found in this regime in the pure $SU(2)$ case \cite{Capri:2012ah} (the $Z$-boson field having one physical and one negative-norm pole in the propagator) with a massless boson decoupled from the non-Abelian sector.

In region IV there is only one state with real mass squared --- the other two having complex mass squared, conjugate to each other --- and from the partial fraction decomposition follows that it has positive residue. This means that, in this region, the diagonal-plus-photon sector contains one physical massive state (becoming massless in the limit $g'\to0$), and two states that can, at best, be interpreted as confined.


































%\chapter{The Electroweak theory: $SU(2)\times U(1)+$Higgs field}
%\label{The Electroweak theory}
%%%%%%%%%%%%%%%%%%%%%%%%%%%%%%%%%%%%%%%%%%%%%%%%%%%%%%%%%%%%%%%%%
%
%From now on in this work only the fundamental case of the Higgs field will be treated, for reasons relying on the physical relevance of the fundamental representation of this field. In what follows we are going to consider the case of $SU(2) \times U(1)$ coupled to the Higgs field in its fundamental representation. As a first step, we are going to present, as in the previous sections, general results for $d$-dimension. The interesting cases of $d=3$ and $d=4$ will be considered in the subsections \ref{ d=3} and \ref{d=4} of this present section. With that being said, the starting action of the $SU(2) \times U(1)+$Higgs filed model is
%\begin{equation}
%S=\int \d^{d}x  \;  \left(\frac{1}{4}  F_{\mu \nu }^{a} F_{\mu \nu }^{a}  +  \frac{1}{4} B_{\mu\nu} B_{\mu\nu} +
%(D_{\mu }^{ij}\Phi^{j})^{\dagger}( D_{\mu }^{ik}\Phi^{k})+\frac{\lambda }{2}\left(\Phi^{\dagger}\Phi - \nu^{2}\right)^{2}\right)  \;,
%\label{Sf}
%\end{equation}
%where the covariant derivative is defined by
%\begin{equation}
%D_{\mu }^{ij}\Phi^{j} =\partial _{\mu }\Phi^{i} - \frac{ig'}{2}B_{\mu}\Phi^{i} -   ig \frac{(\tau^a)^{ij}}{2}A_{\mu }^{a}\Phi^{j}  \;.
%\end{equation}
%and the vacuum expectation value (\textit{vev}) of the Higgs field is
%\begin{equation}
%\langle \Phi \rangle  = \left( \begin{array}{ccc}
%                                          0  \\
%                                          \nu
%                                          \end{array} \right)  \;.
%\label{vevf}
%\end{equation}
%The indices $i,j=1,2$ refer to the fundamental representation of $SU(2)$ and $\tau^a, a=1,2,3$, are the Pauli matrices. The coupling constants $g$ and $g'$ refer to the groups $SU(2)$ and $U(1)$, respectively. The field strengths $F^a_{\mu\nu}$ and $B_{\mu\nu}$ are given by
%\begin{equation}
%F^a_{\mu\nu} = \partial_\mu A^a_\nu -\partial_\nu A^a_\mu + g \varepsilon^{abc} A^b_\mu A^c_\nu \;, \qquad B_{\mu \nu} = \partial_\mu B_\nu -\partial_\nu B_\mu  \;.
%\label{fs}
%\end{equation}
%To obtain the  gauge boson propagators, we consider the quadratic part of the action (\ref{Sf}), given by
%\begin{multline}
%S_{quad} = \int\!\! \d^{d}x \; \frac{1}{2}A_{\mu}^{\alpha}\left[ \left(-\partial_{\mu}\partial_{\mu} + \frac{\nu^{2}}{2}g^{2}\right)\delta_{\mu\nu} + \partial_{\mu}\partial_{\nu} \right]A_{\nu}^{\alpha} + 
%\int\!\! \d^{d}x\, \frac{1}{2}B_{\mu}\left[ \left(-\partial_{\mu}\partial_{\mu} + \frac{\nu^{2}}{2}g'^{2}\right)\delta_{\mu\nu} + \partial_{\mu}\partial_{\nu}\right]B_{\nu} \\
%+ \int\!\! \d^{d}x\, \frac{1}{2}A_{\mu}^{3}\left[ \left(-\partial_{\mu}\partial_{\mu} + \frac{\nu^{2}}{2}g^{2}\right)\delta_{\mu\nu} + \partial_{\mu}\partial_{\nu}\right]A_{\nu}^{3} - \frac{1}{4}
%\int\!\! \d^{d}x\, \nu^{2}g\,g'\,A^{3}_{\mu}B_{\mu} - \frac{1}{4} 
%\int\!\! \d^{d}x\, \nu^{2}g\,g'\,B_{\mu}A^{3}_{\mu} \;.
%\label{Squad}
%\end{multline}
%In order to diagonalize expression \eqref{Squad} we introduce the following fields
%\begin{subequations} \begin{gather}
%W^+_\mu = \frac{1}{\sqrt{2}} \left( A^1_\mu + iA^2_\mu \right) \;, \qquad W^-_\mu = \frac{1}{\sqrt{2}} \left( A^1_\mu - iA^2_\mu \right)  \;,
%\label{ws} \\
%Z_\mu =\frac{1}{\sqrt{g^2+g'^2} } \left(  -g A^3_\mu + g' B_\mu \right) \qquad \text{and}\qquad A_\mu =\frac{1}{\sqrt{g^2+g'^2} } \left(  g' A^3_\mu + gB_\mu \right) \;.
%\label{za}
%\end{gather} \end{subequations}
%Let us also give, for further use, the inverse combinations:
%\begin{subequations} \begin{gather}
%A^1_\mu = \frac{1}{\sqrt{2}} \left( W^+_\mu + W^-_\mu \right) \;, \qquad A^2_\mu = \frac{1}{i\sqrt{2}} \left( W^+_\mu - W^-_\mu \right) \;,
%\label{iw} \\
%B_\mu =\frac{1}{\sqrt{g^2+g'^2} } \left(  g A_\mu + g' Z_\mu \right) \qquad \text{and}\qquad A^3_\mu =\frac{1}{\sqrt{g^2+g'^2} } \left(  g' A_\mu - gZ_\mu \right) \;.
%\label{iza}
%\end{gather} \end{subequations}
%For the quadratic part of the gauge action, we easily get
%\begin{eqnarray}
%S_{quad} &=&  \int d^3 x   \left( \frac{1}{2} (\partial_\mu W^+_\nu - \partial_\nu W^+_\mu)(\partial_\mu W^-_\nu - \partial_\nu W^-_\mu)  + \frac{g^2\nu^2}{2}W^+_\mu W^-_\mu   \right) 
%\nonumber \\
%&+& \int d^3x  \left(  \frac{1}{4} (\partial_\mu Z_\nu - \partial_\nu Z_\mu)^2  + \frac{(g^2+g'^2)\nu^2}{4}Z_\mu Z _\mu  +    \frac{1}{4} (\partial_\mu A_\nu - \partial_\nu A_\mu)^2  \right)  \;,
%\label{qd}
%\end{eqnarray}
%from which we can read off the masses of the fields $W^+$, $W^-$, and $Z$:
%\begin{equation}
%m^2_W = \frac{g^2\nu^2}{2} \;, \qquad m^2_Z =  \frac{(g^2+g'^2)\nu^2}{2}  \;. \label{ms}
%\end{equation}
%
%
%
%
%
%
%
%
%
%
%
%
%
%
%\subsubsection{The Landau gauge fixing}
%
%The action \eqref{Sf} has to be supplemented by the gauge fixing term $S_{\text{gf}}$  to allow for a meaningful quantization. To that purpose we should emphasize that we are not going to apply the unitary gauge, as already stated in subsection \ref{gfwithbroksym}. Contrary to that we will adopt the Landau gauge condition that should be covered when $\xi \to 0$ takes place:  $f^{a} ~=~ \partial_\mu A^a_\mu - \xi b^{a}$ and $h^{a} ~=~   \partial_\mu B_\mu - \xi b $. Thus, just by following the gauge fixing procedure developed in the section \ref{introductiontogribov},
%\begin{equation}
%S_{\text{gf}} = \int d^3x \; \left(  {\bar c}^a \partial_\mu D^{ab}_\mu c^b - \frac{(\partial_\mu A^a_\mu)^{2}}{2\xi} 
% + {\bar c}\partial^2 c  - \frac{(\partial_\mu B_\mu)^{2}}{2\xi} \right)   \;,
%\label{cgf}
%\end{equation}
%where the limit $\xi \to 0$ enforce the conditions $\partial_\mu A^a_\mu=0$ and $\partial_\mu B_\mu=0 $, while $({\bar c}^a,\; c^a)$ and $({\bar c},\; c)$ are the $SU(2)$ and $U(1)$ Faddeev--Popov ghosts, respectively. The total action then reads
%\begin{equation}
%S_{f} = S + S_{\text{gf}}  \;. 
%\label{fact}
%\end{equation}
%The propagators of the gauge fields are given, in terms of $W^{\pm}_{\mu}$, $Z_{\mu}$ and $A_{\mu}$, as
%\begin{subequations} \label{WZA gluon prop} \begin{eqnarray}
%\langle W^{+}_{\mu}(p)W^{-}_{\nu}(-p) \rangle &=& \frac{1}{p^{2} + \frac{\nu^{2}}{2}g^{2}}\left(\delta_{\mu\nu} - \frac{p_{\mu}p_{\nu}}{p^{2}}\right) \;, \\
%\langle Z_{\mu}(p) Z_{\nu}(-p) \rangle &=& \frac{1}{p^{2} + \frac{\nu^{2}}{2}(g^{2} + g'^{2})} \left(\delta_{\mu\nu} - \frac{p_{\mu}p_{\nu}}{p^{2}}\right)\;, \\
%\langle A_{\mu}(p) A_{\nu}(-p) \rangle &=& \frac{1}{p^{2}} \left(\delta_{\mu\nu} - \frac{p_{\mu}p_{\nu}}{p^{2}}\right) \;, \\
%\langle A_\mu(p) Z_\nu(-p) \rangle & = & 0 \;.
%\end{eqnarray} \end{subequations}
%Taking into account the relation between the mass eigenstates and the fields $A^{\alpha}_{\mu}$, $A^{3}_{\mu}$ and $B_{\mu}$, we also have the following propagators,
%\begin{subequations} \label{AB gluon prop} \begin{eqnarray}
%\langle A^{\alpha}_{\mu}(p) A^{\beta}_{\nu}(-p) \rangle &=& \frac{1}{p^{2} + \frac{\nu^{2}}{2}g^{2}}\left(\delta_{\mu\nu} - \frac{p_{\mu}p_{\nu}}{p^{2}}\right)\delta^{\alpha \beta} \;, \\
%\langle A^{3}_{\mu}(p) A^{3}_{\nu}(-p) \rangle &=& \frac{1}{p^{2} + \frac{\nu^{2}}{2}(g^{2} + g'^{2})} \left(\delta_{\mu\nu} - \frac{p_{\mu}p_{\nu}}{p^{2}}\right)\;,  \\
%\langle B_{\mu}(p) B_{\nu}(-p) \rangle &=& \frac{1}{p^{2}} \left(\delta_{\mu\nu} - \frac{p_{\mu}p_{\nu}}{p^{2}}\right) \;,  \\
%\langle A^{3}_{\mu}(p) B_{\nu}(-p) \rangle &=& \frac{\nu^{2}}{4}(g^{2} + g'^{2}) \frac{g'}{g} \frac{1}{p^{2}\left(p^{2} + \frac{\nu^{2}}{2}(g^{2} + g'^{2}) \right)} \left(\delta_{\mu\nu} - \frac{p_{\mu}p_{\nu}}{p^{2}}\right)\;.
%\end{eqnarray} \end{subequations}
%
%
%
%
%
%
%
%
%
%
%
%
%
%
%
%
%
%
%
%
%\subsubsection{The restriction to the Gribov region} \label{sect3}
%\label{TRGR}
%
%As established in \cite{Gribov:1977wm}, the gauge fixing procedure is plagued by the existence of Gribov copies,  an observation later on shown to be a general feature of non-Abelian gauge fixing \cite{Singer:1978dk}. In the Landau gauge, it is easy to see that at least the infinitesimally connected gauge copies  can be dealt with by restricting the domain of integration in the path integral to the so called Gribov region $\Omega$ \cite{Gribov:1977wm,Sobreiro:2005ec,Vandersickel:2012tz}. A way to implement the restriction to the region $\Omega$ has been worked out by Gribov in his original work and developed in \ref{introductiontogribov}
%\footnote{We refer to the recent review \cite{Vandersickel:2012tz} for a complete list of references.}. 
%It amounts to impose the no-pole condition for the connected two-point ghost function $\mathcal{G}^{ab}(k;A) = \langle k | \left(  -\partial D^{ab}(A) \right)^{-1} |k\rangle $, which is nothing but the inverse of the Faddeev--Popov operator $-\partial D^{ab}(A)$. One requires that  $\mathcal{G}^{ab}(k;A)$ has no poles at finite non vanishing values of $k^2$, so that it stays always positive. That way one ensures that the Gribov horizon is not crossed, {\it i.e.}~one remains inside $\Omega$. The only allowed pole is at $k^2=0$, which has the meaning of approaching the boundary of the region $\Omega$. In a recent work \cite{Capri:2012wx}, the no-pole condition was linked in a more precise fashion to the Zwanziger implementation of the Gribov restriction \cite{Zwanziger:1989mf}, thereby making clear their equivalence.
%
%Following the procedure developed in section \ref{introductiontogribov}, for the connected two-point ghost function $\mathcal{G}^{ab}(k;A)$ at first order in the gauge fields,  one finds
%\begin{equation}
%\mathcal{G}^{ab}(k;A) ~=~ \frac{1}{k^{2}}\left[ \delta^{ab} - g^{2}\frac{k_{\mu}k_{\nu}}{k^{2}} \int\!\! \frac{\d^{d}p}{(2\pi)^{d}} \,\, \epsilon^{amc}\epsilon^{cnb} \frac{1}{(k-p)^{2}}A^{m}_{\mu}(p)A^{n}_{\nu}(-p)\right]\; ,
%\label{ghprop}
%\end{equation}
%which, owing to
%\begin{equation}
%\epsilon^{amc}\epsilon^{cnb} ~=~ \delta^{an} \delta^{mb}  - \delta^{ab} \delta^{mn} \;, \label{dd}
%\end{equation}
%can be written as
%\begin{equation}
%\mathcal{G}^{ab}(k;A) ~=~ \left(
%  \begin{array}{ll}
%   \delta^{\alpha \beta} \mathcal{G}_{off}(k;A) & \,\,\,\,\,\,\,\,0 \\
%   \,\,\,\;\;\;\;\;\;0 & \mathcal{G}_{diag}(k;A)
%  \end{array}
%\right),
%\label{gh prop offdiag}
%\end{equation}
%where
%\begin{eqnarray}
%\mathcal{G}_{off}(k;A) & = &  \frac{1}{k^{2}} \left[1+ \frac{g^{2}}{V}  \frac{k_{\mu} k_{\nu}}{k^{2}} \int\!\! \frac{\d^{d}p}{(2\pi)^{d}} \frac{1}{(k-p)^{2}} \frac{1}{2} \left(A^{\alpha}_{\mu}(p)A^{\alpha}_{\nu}(-p) + 2A^{3}_{\mu}(p)A^{3}_{\nu}(-p)\right)\right]  \nonumber \\
%& \simeq&  \frac{1}{k^{2}} \left( \frac{1}{1 - \sigma_{off}(k;A)} \right) \;,
%\label{gh off}
%\end{eqnarray}
%and
%\begin{eqnarray}
%\mathcal{G}_{diag}(k;A) & = & \frac{1}{k^{2}} \left[1+ \frac{g^{2}}{V}  \frac{k_{\mu} k_{\nu}}{k^{2}} \int\!\! \frac{\d^{d}p}{(2\pi)^{d}} \frac{1}{(k-p)^{2}} A^{\alpha}_{\mu}(p)A^{\alpha}_{\nu}(-p)\right] \nonumber \\
%& \simeq & \frac{1}{k^{2}} \left( \frac{1}{1 - \sigma_{diag}(k;A)} \right)\;,
%\label{gh diag}
%\end{eqnarray}
%where $V$ denotes, as usual in this work, the (infinity) space-time volume. As we know, the quantities $\sigma_{off}(k;A)$ and $\sigma_{diag}(k;A)$ turn out to be decreasing functions of $k$. Thus, the no-pole condition is implemented by requiring that
%\begin{subequations} \begin{equation}
%\sigma_{off}(0;A) < 1  \;,
%\label{sigmaoffnopole}
%\end{equation}
%and
%\begin{equation}
%\sigma_{diag}(0;A) < 1\;,
%\label{sigmadiagnopole}
%\end{equation} \end{subequations}
%where
%\begin{subequations} \begin{equation}
%\sigma_{off}(0;A) ~=~ \frac{g^{2}}{dV} \int\!\! \frac{\d^{d}p}{(2\pi)^{d}} \;  \frac{1}{p^{2}} \left( \frac{1}{2} A^{\alpha}_{\mu}(p)A^{\alpha}_{\mu}(-p) + A^{3}_{\mu}(p)A^{3}_{\mu}(-p)\right)  \;,
%\label{sigma off}
%\end{equation}
%and
%\begin{equation}
%\sigma_{diag}(0;A) ~=~ \frac{g^{2}}{dV} \int\!\! \frac{\d^{d}p}{(2\pi)^{d}} \; \frac{1}{p^{2}} A^{\alpha}_{\mu}(p)A^{\alpha}_{\mu}(-p)\;.
%\label{sigma diag}
%\end{equation} \end{subequations}
%Expressions  \eqref{sigma off} and \eqref{sigma diag} are obtained by taking the limit $k \to 0$ of equations \eqref{gh off},  \eqref{gh diag}, and by making use of the property
%\begin{eqnarray}
%A_{\mu }^{a}(q)A_{\nu }^{a}(-q) &=&\left( \delta _{\mu \nu }-\frac{q_{\mu }q_{\nu }}{q^{2}}\right) \omega (A)(q)  
%\nonumber \\
%&\Rightarrow &\omega (A)(q) ~=~ \frac{1}{(d-1)}A_{\lambda }^{a}(q)A_{\lambda }^{a}(-q)    \;,
%\label{p1}
%\end{eqnarray}
%which follows from the transversality of the gauge field, $q_\mu A^a_\mu(q)=0$. Also, it is useful to remind that, for an arbitrary function $\mathcal{F}(p^2)$, we have
%\begin{equation}
%\int \frac{\d^{d}p}{(2\pi )^{d}}\left( \delta _{\mu \nu }-\frac{p_{\mu }p_{\nu }}{p^{2}}\right) \mathcal{F}(p^2) ~=~ \mathcal{A}\;\delta _{\mu \nu }   \;,
%\label{p2}
%\end{equation}
%where, upon contracting both sides of equation \eqref{p2} with $\delta_{\mu\nu}$,
%\begin{equation}
%\mathcal{A} ~=~ \frac{d-1}{d}\int \frac{\d^{d}p}{(2\pi )^{d}}\mathcal{F}(p^2)\;.   
%\label{p3}
%\end{equation}
%
%
%
%
%
%
%
%
%
%
%
%
%\subsubsection{The two gap equations}
%
%As explained in section \ref{introductiontogribov}, the implementation of the restriction to the first Gribov region $\Omega$ in the functional integral concerns the application of the no-pole conditions \eqref{sigmaoffnopole} and \eqref{sigmadiagnopole}, which is encoded into Heaviside step functions. By making use of its integral representation,
%\begin{equation}
%\theta(x) = \int_{-i\infty +\epsilon}^{i\infty +\epsilon} \frac{d\omega}{2\pi i \omega} \; e^{\omega x}   \;, 
%\label{rp}
%\end{equation}
%we could get for the functional integral,
%\begin{eqnarray}
%Z_{quad} &=& \mathcal{N'} \int \frac{\d \omega}{2\pi i \omega}\frac{\d \beta}{2\pi i \beta} [\d A] [\d B] \; \delta (\p A) \delta({\partial B}) \; \; e^{\omega (1-\sigma_{off})} \, e^{\beta (1-\sigma_{diag})} e^{-S_{quad}}\;.
%\label{ptionfucnt22}
%\end{eqnarray}
%By computing up to one-loop order in perturbation theory, we have
%\begin{eqnarray}
%Z_{quad} &=& \mathcal{N'} \int \frac{\d \omega}{2\pi i \omega}\frac{\d \beta}{2\pi i \beta} [\d A^{\alpha}] [\d A^{3}] [\d B] \; e^{\omega}\, e^{\beta} \exp \left\{-\frac{1}{2} \int\!\! \frac{\d^{d}p}{(2\pi)^{d}}\, A^{\alpha}_{\mu}(p) \left[ \left(p^{2} + \frac{\nu^{2}}{2}g^{2} + \right. \right. \right. 
%\nonumber \\
% &{}& \left. \left. + \frac{2g^{2}}{dV}\left(\beta +\frac{\omega}{2}\right)\frac{1}{p^{2}}\right) \left(\delta_{\mu\nu} - \frac{p_{\mu}p_{\nu}}{p^{2}}\right)\right]A^{\alpha}_{\nu}(-p) + A^{3}_{\mu}(p) \left[ \left(p^{2} + \frac{\nu^{2}}{2}g^{2} + \frac{2g^{2}}{dV}\frac{\omega}{p^{2}}\right) \times \right.
%\nonumber \\
%&{}& \left. \times \left(\delta_{\mu\nu} - \frac{p_{\mu}p_{\nu}}{p^{2}}\right) \right]A^{3}_{\nu}(-p)  + B_{\mu}(p) \left[ \left(p^{2} + \frac{\nu^{2}}{2}g'^{2}\right)\left(\delta_{\mu\nu} - \frac{p_{\mu}p_{\nu}}{p^{2}}\right) \right] B_{\nu}(-p) - 
%\nonumber \\
%&{}& \left. - A^{3}_{\mu}(p) \left[\nu^{2}g\,g'\left(\delta_{\mu\nu} - \frac{p_{\mu}p_{\nu}}{p^{2}}\right) \right]B_{\nu}(-p) \right\}\;.
%\label{Zquad1}
%\end{eqnarray}
%It turns out to be convenient to perform the following change of variables
%\begin{subequations} \begin{equation}
%\beta \to \beta -\frac{\omega}{2}
%\end{equation}
%and
%\begin{equation}
%\omega \to \omega\;.
%\end{equation} \end{subequations}
%Therefore, equation \eqref{Zquad1} can be rewritten as,
%\begin{eqnarray}
%Z_{quad} &=& \mathcal{N'} \int \frac{\d \omega}{2\pi i}\frac{\d \beta}{2\pi i} [\d A^{\alpha}] [\d A^{3}] [\d B] \;\delta (\partial A^\alpha) \; \delta (\partial A^3)\; \delta({\partial B}) \;\; e^{-\ln\left(\beta\omega-\frac{\omega^{2}}{2}\right)} e^{\beta}\,e^{\frac{\omega}{2}}\;  \times 
%\nonumber \\
%&{ }& \times \exp \left[-\frac{1}{2} \int \frac{\d^{d}p}{(2\pi)^{d}} A^{\alpha}_{\mu}(p)\, Q^{\alpha \beta}_{\mu \nu}\, A^{\beta}_{\nu}(-p)\right] \times   \exp \left[-\frac{1}{2} \int \frac{\d^{d}p}{(2\pi)^{d}} \left(
% \begin{array}{cc}
%  A^{3}_{\mu}(p) & B_{\mu}(p)
% \end{array} \right)
%\mathcal{P}_{\mu \nu} \left(
%\begin{array}{ll}
%A^{3}_{\nu}(-p) \\
%B_{\nu}(-p)
%\end{array} \right) \right]  \nonumber \\
%\label{Zquad2}
%\end{eqnarray}
%where
%\begin{subequations} \begin{equation}
%Q^{\alpha \beta}_{\mu \nu} ~=~ \left[ p^{2} + \frac{\nu^{2} g^{2}}{2} + \frac{2g^{2}\beta}{dV} \frac{1}{p^{2}} \right] \delta^{\alpha \beta} \left( \delta_{\mu \nu} - \frac{p_{\mu} p_{\nu}}{p^{2}}\right) \label{Qmunu}
%\end{equation}
%and
%\begin{equation}
%\mathcal{P}_{\mu \nu} ~=~ \left(
%                        \begin{array}{cc}
%                         p^{2} + \frac{\nu^{2}}{2}g^{2} + \frac{2 g^{2} \omega }{dV} \frac{1}{p^{2}} & - \frac{\nu^{2}}{2}g\,g' \\
%                         - \frac{\nu^{2}}{2}g\,g' & p^{2} + \frac{\nu^{2}}{2}{g'}^2
%                        \end{array} \right) \left( \delta_{\mu \nu} - \frac{p_{\mu} p_{\nu}}{p^{2}} \right)\;.
%\label{P mu nu}
%\end{equation} \end{subequations}
%From expression \eqref{Zquad2} we can easily deduce the two-point correlation functions of the fields $A^\alpha_\mu$, $A^3_\mu$ and $B_\mu$, namely
%\begin{subequations} \label{propsaandb} \begin{gather}
%\langle  A^{\alpha}_\mu(p) A^{\beta}_\nu(-p) \rangle ~=~ \frac{p^2}{p^4 + \frac{\nu^2g^2}{2} p^2 + \frac{2g^2\beta}{dV}} \; \delta^{\alpha \beta} \left( \delta_{\mu\nu} - \frac{p_\mu p_\nu}{p^2} \right)  \;,   
%\label{aalpha} \\
%\langle  A^{3}_\mu(p) A^{3}_\nu(-p) \rangle ~=~ \frac{p^2 \left(p^2 +\frac{\nu^2}{2} g'^{2}\right)}{p^6 + \frac{\nu^2}{2} p^4 \left(g^2 +g'^2 \right)  + \frac{2\omega g^2}{dV} \left( p^2 + \frac{\nu^2 g'^2}{2} \right)} \;  \left( \delta_{\mu\nu} - \frac{p_\mu p_\nu}{p^2} \right)  \;,   \label{a3a3}
%\\ %
%\langle  B_\mu(p) B_\nu(-p) \rangle ~=~ \frac{ \left(p^4 +\frac{\nu^2}{2} g^{2} p^2+\frac{2\omega g^2}{dV}  \right)}{p^6 + \frac{\nu^2}{2} p^4 \left(g^2 +g'^2 \right)  + \frac{2\omega g^2}{dV} \left( p^2 +  \frac{\nu^2 g'^2}{2} \right)} \;  \left( \delta_{\mu\nu} - \frac{p_\mu p_\nu}{p^2} \right)  \;,   \label{bb}
%\\
%\langle  A^3_\mu(p) B_\nu(-p) \rangle ~=~  \frac{ \frac{\nu^2}{2} g g'  p^2}{p^6 + \frac{\nu^2}{2} p^4 \left(g^2 +g'^2 \right)  + \frac{2\omega g^2}{dV} \left( p^2 + \frac{\nu^2 g'^2}{2} \right)} \;  \left( \delta_{\mu\nu} - \frac{p_\mu p_\nu}{p^2} \right)  \;.   \label{ba3}
%\end{gather} \end{subequations}
%Moving to the fields $W^{+}_\mu, W^{-}_\mu, Z_\mu, A_\mu$, one obtains 
%\begin{subequations} \label{propszandgamma} \begin{gather}
%\langle  W^{+}_\mu(p) W^{-}_\nu(-p) \rangle ~=~ \frac{p^2}{p^4 + \frac{\nu^2g^2}{2} p^2 + \frac{2g^2\beta}{dV} } \;  \left( \delta_{\mu\nu} - \frac{p_\mu p_\nu}{p^2} \right)  \;,   \label{ww}
%\\
%\langle  Z_\mu(p) Z_\nu(-p) \rangle ~=~ \frac{\left( p^4 +\frac{2\omega}{dV} \frac{g^2 g'^2}{g^2+g'^2}  \right)}{p^6 + \frac{\nu^2}{2} p^4 \left(g^2 +g'^2 \right)  + \frac{2\omega g^2}{dV} \left( p^2 + \frac{\nu^2 g'^2}{2} \right) } \;  \left( \delta_{\mu\nu} - \frac{p_\mu p_\nu}{p^2} \right)  \;,   \label{zz}
%\\
%\langle  A_\mu(p) A_\nu(-p) \rangle ~=~ \frac{\left( p^4 +\frac{\nu^2}{2} p^2 (g^2+g'^2) +\frac{2\omega}{dV} \frac{g^4}{g^2+g'^2}\right)}{p^6 + \frac{\nu^2}{2} p^4 \left(g^2 +g'^2 \right)  + \frac{2\omega g^2}{dV} \left( p^2 + \frac{\nu^2 g'^2}{2} \right) } \;  \left( \delta_{\mu\nu} - \frac{p_\mu p_\nu}{p^2} \right)  \;,   \label{aa}
%\\
%\langle  A_\mu(p) Z_\nu(-p) \rangle ~=~ \frac{\frac{2\omega}{dV} \frac{g^3 g'}{g^2+g'^2} }{p^6 + \frac{\nu^2}{2} p^4 \left(g^2 +g'^2 \right)  + \frac{2\omega g^2}{dV} \left( p^2 + \frac{\nu^2 g'^2}{2} \right)} \;  \left( \delta_{\mu\nu} - \frac{p_\mu p_\nu}{p^2} \right)  \;.   \label{az}
%\end{gather} \end{subequations}
%As expected, all propagators get deeply modified in the infrared by the presence of the Gribov parameters $\beta$ and $\omega$. Notice in particular that, due to the parameter $\omega$, a  mixing between the fields $A_\mu$ and $Z_\mu$ arises, eq.\eqref{az}.  The original photon and $Z$ boson as such loose their distinct particle interpretation.  Moreover, it is straightforward to check that in the limit $\beta \rightarrow 0$ and $\omega \rightarrow 0$, the standards propagators are recovered, {\it i.e.}
%\begin{subequations} \begin{gather}
%\langle  W^{+}_\mu(p) W^{-}_\nu(-p) \rangle {\big |}_{\beta=0} ~=~ \frac{1}{p^2 + \frac{\nu^2g^2}{2} } \;  \left( \delta_{\mu\nu} - \frac{p_\mu p_\nu}{p^2} \right)  \;,   \label{ww0}
%\\
%\langle  Z_\mu(p) Z_\nu(-p) \rangle  {\big |}_{\omega=0} ~=~ \frac{1}{p^2 + \frac{\nu^2}{2} \left(g^2 +g'^2 \right)} \;  \left( \delta_{\mu\nu} - \frac{p_\mu p_\nu}{p^2} \right)  \;,   \label{zz0}
%\\
%\langle  A_\mu(p) A_\nu(-p) \rangle{\big |}_{\omega=0}  ~=~ \frac{1}{p^2}   \;  \left( \delta_{\mu\nu} - \frac{p_\mu p_\nu}{p^2} \right)  \;,   \label{aa0}
%\\
%\langle  A_\mu(p) Z_\nu(-p) \rangle {\big |}_{\omega=0} = 0    \;.   \label{az0}
%\end{gather} \end{subequations}
%Let us now proceed by deriving the gap equations which will enable us to compute the Gribov parameters $\beta$ and $\omega$ in terms of the parameters $g$, $g'$ and $\nu^2$. To that end we integrate out the fields in expression  \eqref{Zquad2}, obtaining
%\begin{equation}
%Z_{quad} ~=~ \mathcal{N'} \int \frac{\d \omega}{2\pi i}\frac{\d \beta}{2\pi i}\,e^{-\ln\left(\beta\omega -\frac{\omega^{2}}{2}\right)} e^{\frac{\omega}{2}}\,e^{\beta} \left[\det Q_{\mu \nu}^{\alpha \beta} \right]^{-1/2}\, \left[ \det \mathcal{P}_{\mu \nu} \right]^{-1/2},  \label{Zquad3}
%\end{equation}
%with
%\begin{subequations} \begin{equation}
%\left[\det Q_{\mu \nu}^{\alpha \beta} \right]^{-1/2} ~=~ \exp \left[ -\frac{2(d-1)}{2} \int \frac{\d^{d}p}{(2\pi)^{d}}  \;  \log \left(p^{2} + \frac{g^{2}\nu^{2}}{2} + \frac{2g^{2}\beta}{dV} \frac{1}{p^{2}}\right) \right] \label{calc_det1}
%\end{equation}
%and
%\begin{equation}
%\left[ \det \mathcal{P}_{\mu \nu} \right]^{-1/2} ~=~ \exp \left[ - \frac{(d-1)}{2} \int \frac{\d^{d}p}{(2\pi)^{d}} \log \lambda_{+}(p, \omega)\, \lambda_{-}(p, \omega) \right].   \label{calc_det2}
%\end{equation} \end{subequations}
%where $\lambda_{\pm}$ are the eigenvalues of the $2\times 2$ matrix of eq.\eqref{P mu nu}, {\it i.e.}
%\begin{equation}
%\lambda_{\pm} ~=~ \frac{\left( p^{4} + \frac{\nu^{2}}{4} p^{2}(g^{2} + g'^{2}) + \frac{g^{2}\omega}{dV} \right) \pm \sqrt{\left[ \frac{\nu^{2}}{4}(g^{2} + g'^{2})p^{2} + \frac{g^{2}\omega}{dV}\right]^{2} - \frac{\omega}{3}\nu^{2}g^{2}\,g'^{2}p^{2}}}{p^{2}}  \;. \label{ev}
%\end{equation}
%Thus,
%\begin{equation}
%Z_{quad} = \mathcal{N} \int \frac{\d \omega}{2\pi i}\frac{\d \beta}{2\pi i} e^{f(\omega, \beta)} \;, \label{Zf eq}
%\end{equation}
%where
%\begin{equation}
%f(\omega, \beta) = \frac{\omega}{2} + \beta - \frac{2(d-1)}{2} \int \frac{\d^{d}p}{(2\pi)^{d}} \; \log \left(p^{2} + \frac{\nu^{2}}{2}g^{2} + \frac{2g^{2}\beta}{dV} \frac{1}{p^{2}}\right) - \frac{(d-1)}{2} \int \frac{\d^{d}p}{(2\pi)^{d}} \; \log \lambda_{+}(p, \omega)\, \lambda_{-}(p, \omega)\;. \label{f eq}
%\end{equation}
%Following Gribov's framework \cite{Gribov:1977wm,Sobreiro:2005ec,Vandersickel:2012tz}, expression \eqref{Zf eq} is evaluated in a saddle point approximation, {\it i.e.}
%\begin{equation}
%Z_{quad} \simeq e^{f(\omega^{\ast}, \beta^{\ast})}\;,
%\label{Zf eq1}
%\end{equation}
%where $(\beta^*, \omega^*)$ are determined by the stationarity conditions
%\begin{equation}
%\frac{\partial f(\omega, \beta)}{\partial \beta}\bigg{|}_{\beta^{\ast}, \omega^{\ast}} ~=~ \frac{\partial f(\omega, \beta)}{\partial \omega}\bigg{|}_{\beta^{\ast}, \omega^{\ast}} ~=~ 0 \;, \nonumber
%\label{saddle condition}
%\end{equation}
%from which we get the two gap equations: the first one, from the $\omega$ derivative,
%\begin{subequations} \begin{multline}
%\frac{2(d-1)}{2d} g^{2} \int \frac{\d^{d}p}{(2\pi)^{d}}\; \frac{1}{\left [ p^{4} + \frac{\nu^{2}}{4}(g^{2}+g'^{2})p^{2} + \frac{g^{2}\omega^{\ast}}{dV}\right]^{2} - \left[ \frac{\nu^{2}}{4}(g^{2}+g'^{2})p^{2} + \frac{g^{2}\omega^{\ast}}{dV} \right]^{2} + \frac{\nu^{2}g^{2}g'^{2}\omega^{\ast}}{dV} p^{2}} \times 
%\\
%\left\{ \left[ 1+ \frac{\frac{\nu^{2}}{4}(g^{2}+g'^{2})p^{2} + \frac{g^{2}\omega^{\ast}}{dV}  - \frac{\nu^{2}}{2}g'^{2}p^{2}}{\sqrt{\left[ \frac{\nu^{2}}{4}(g^{2}+g'^{2})p^{2} + \frac{g^{2}\omega^{\ast}}{dV} \right]^{2} - \frac{\nu^{2}g^{2}g'^{2}\omega^{\ast}}{dV}p^{2}}}\right] \right. \times 
%\\
%\left[ p^{4} + \frac{\nu^{2}}{4}(g^{2}+g'^{2})p^{2} +  \frac{g^{2}\omega^{\ast}}{dV} - \sqrt{\left[ \frac{\nu^{2}}{4}(g^{2}+g'^{2})p^{2} + \frac{g^{2}\omega^{\ast}}{dV}  \right]^{2} - \frac{\nu^{2}g^{2}g'^{2}\omega^{\ast}}{dV}  p^{2}}\right] + 
%\\
%\left[ 1-\frac{\frac{\nu^{2}}{4}(g^{2}+g'^{2})p^{2} + \frac{g^{4}\omega^{\ast}}{3}  -  \frac{\nu^{2}}{2}g'^{2}p^{2}}{\sqrt{\left[ \frac{\nu^{2}}{4}(g^{2}+g'^{2})p^{2} + \frac{g^{2}\omega^{\ast}}{dV}  \right]^{2} - \frac{\nu^{2}g^{2}g'^{2}\omega^{\ast}}{dV}  p^{2}}} \right] \times 
%\\
%\left. \left[ p^{4}+\frac{\nu^{2}}{4}(g^{2}+g'^{2})p^{2} + \frac{g^{2}\omega^{\ast}}{dV} + \sqrt{\left[ \frac{\nu^{2}}{4}(g^{2}+g'^{2})p^{2}  +  \frac{g^{2}\omega^{\ast}}{dV}  \right]^{2} - \frac{\nu^{2}g^{2}g'^{2}\omega^{\ast}}{dV} p^{2}} \right] \right\}  ~=~ 1\;,
%\label{omega gap eq1}
%\end{multline}
%and the second one, from the $\beta$ derivative,
%\begin{equation}
%\frac{4(d-1)}{2d}g^{2} \int \frac{\d^{d}p}{(2\pi)^{d}} \;  \frac{1}{p^{4}+\frac{g^{2}\nu^{2}}{2}p^{2} + \frac{2g^{2}\beta^{\ast}}{dV} } ~=~ 1   \;.
%\label{beta gap eq}
%\end{equation} \end{subequations}
%In particular, after a little algebra, eq.\eqref{omega gap eq1} can be considerably simplified, yielding
%\begin{equation}
%\frac{2(d-1)}{d}g^{2}\int\!\! \frac{\d^{d}p}{(2\pi)^{d}} \; \frac{p^{2} + \frac{\nu^{2}}{2}g'^{2}}{p^{6} + \frac{\nu^{2}}{2}(g^{2} + g'^{2})p^{4} + \frac{2\omega^{\ast}g^{2}}{dV}p^{2} + \frac{\nu^{2}g^{2}\,g'^{2}\omega^{\ast}}{dV} } ~=~ 1 \;.
%\label{omega gap eq}
%\end{equation}
%%
%
%
%
%
%
%
%
%
%
%
%
%
%
%
%
%
%
%\subsubsection{The limit $g' \to 0$.}
%An important check to be done is the case where $g'=0$, which must recover the results of \cite{Capri:2012ah} with the Higgs field in the fundamental representation, obtaining a $U(1)$ massless gauge field decoupled from the $SU(2)$ gauge sector. This decoupling can be easily seen just by setting $g'=0$ in the propagator expressions \eqref{a3a3} - \eqref{ba3} and \eqref{aalpha}:
%\begin{subequations} \label{propgzro} \begin{eqnarray}
%\langle A^{\alpha}_{\mu}(p)A^{\beta}_{\nu}(-p)\rangle &=& \frac{p^{2}}{p^{4} + \frac{\nu^{2}}{2}g^{2}p^{2} + \frac{\beta}{2}g^{2}} \delta^{\alpha\beta}T_{\mu\nu}(p^2)\;,\quad \langle A^{3}_{\mu}(p)A^{3}_{\nu}(-p)\rangle = \frac{p^{2}}{p^{4}+\frac{\nu^{2}}{2}g^{2}p^{2}+\frac{\omega^{2}}{2}g^{2}}T_{\mu\nu}(p^2)\;, \\
%\langle B_{\mu}(p)B_{\nu}(-p)\rangle &=&  \frac{1}{p^{2}} T_{\mu\nu}(p^2)\;,\quad \langle A^{3}_{\mu}(p)B_{\nu}(-p)\rangle = 0\;.
%\end{eqnarray} \end{subequations}
%Also, as in the last section, one should be able to write the propagators in terms of the fields $W^{\pm}$, $Z$ and $A$ obtaining
%\begin{subequations} \label{ppgzro} \begin{eqnarray}
%\langle W^{+}_{\mu}(p)W^{-}_{\nu}(-p)\rangle &=& \frac{p^{2}}{p^{4} + \frac{\nu^{2}}{2}g^{2}p^{2} + \frac{\beta}{2}g^{2}} \delta^{\alpha\beta}T_{\mu\nu}(p^2))\;, \label{ww1} \quad \langle Z_{\mu}(p)Z_{\nu}(-p)\rangle = \frac{p^{2}}{p^{4}+\frac{\nu^{2}}{2}g^{2}p^{2}+\frac{\omega^{2}}{2}g^{2}} T_{\mu\nu}(p^2)\;, \\
%\langle A_{\mu}(p)A_{\nu}(-p)\rangle &=& \frac{1}{p^{2}} T_{\mu\nu}(p^2)\;, \quad
%\langle A_{\mu}(p)Z_{\nu}(-p)\rangle = 0\;.
%\end{eqnarray} \end{subequations}
%These propagators, \eqref{propgzro} and \eqref{ppgzro}, could also be derived by taking $g'=0$ in the quadratic partition function, or even in the generating functional \eqref{Zquad1}, and following the steps of the last section.
%
%
%
%
%
%
%
%
%
%
%
%
%
%\subsubsection{About $\sigma_\text{off}(0)$ and $\sigma_\text{diag}(0)$ without the Gribov parameters}
%\label{Evaluation of the ghost form factors}
%
%Specifically in the next two sections we propose a different approach to check if there exist values of the Higgs condensate $\nu$ and of the coupling $(g,\;g')$ for which both $\langle \sigma_{off}(0) \rangle $ and $\langle \sigma_{diag}(0) \rangle$  already satisfy the no-pole condition
%\begin{equation}
%\langle \sigma_{off}(0;A) \rangle  < 1  \;, \qquad   \langle \sigma_{diag}(0;A) \rangle < 1\;,
%\label{n-pole}
%\end{equation}
%in which case $\beta^\ast$ and/or $\omega^\ast$ could be immediately set equal to zero. Therefore, we present here the expression of both quantities for the $d$-dimensional case. Thus,
%\begin{eqnarray}
%\langle \sigma_{off}(0) \rangle & = &  \frac{(d-1)g^{2}}{d}  \int\!\!  \frac{\d^d p}{(2\pi)^d} \frac{1}{p^{2}}\left(\frac{1}{p^{2} + \frac{\nu^{2}}{2}g^{2}} + \frac{1}{p^{2} + \frac{\nu^{2}}{2}(g^{2}+g'^{2})} \right)  \;.
%\label{sgoff1}
%\end{eqnarray}
%Analogously, for $\langle \sigma_{diag}(0)\rangle$ one gets
%\begin{equation}
%\langle \sigma_{diag}(0) \rangle ~=~ \frac{2(d-1)g^{2}}{d} \int\!\!\frac{\d^{d}p}{(2\pi)^{d}}\frac{1}{p^{2}}\left(\frac{1}{p^{2} + \frac{\nu^{2}}{2}g^{2}}\right)  \;.
%\label{sgdiag1}
%\end{equation}
%
%Now that we have in hands all expressions needed to analyze the gluon propagator, let us specialize them to the $d=3$ and $d=4$ cases. With that, we are able to provide a map of the parameter space displaying regions where we have sign of confinement, regions where the Higgs mechanism takes place unaltered and mixed regions where the propagators have ``confined'' and ``deconfined'' contributions in its expressions.
%
%
%
%
%
%
%
%
%
%
%
%
%
%
%
%
%
%
%
%
%\subsection{Results of $d=3$} 
%\label{ d=3}
%
%
%Before discussing the gap equations equations \eqref{beta gap eq} and \eqref{omega gap eq}, it is worthwhile to evaluate the vacuum expectation values of the ghost form factors  $\sigma_{off}(0)$ and $\sigma_{diag}(0)$, eqs. \eqref{sigma off} and \eqref{sigma diag}, without taking into account the restriction to the Gribov region, {\it i.e.}~without the presence of the two Gribov parameters $(\beta^{\ast},\omega^{\ast})$. This will enable us to verify if there exist values of the Higgs condensate $\nu$ and of the couplings $(g,g')$ for which both $\langle \sigma_{off}(0) \rangle $ and $\langle \sigma_{diag}(0) \rangle$  already satisfy the no-pole condition
%\begin{equation}
%\langle \sigma_{off}(0;A) \rangle  < 1  \;, \qquad   \langle \sigma_{diag}(0;A) \rangle < 1\;,
%\label{n-pole}
%\end{equation}
%in which case $\beta^\ast$ and/or $\omega^\ast$ could be immediately set equal to zero.
%
%Let us start by considering $\langle \sigma_{off}(0)\rangle$. From eqs.\eqref{sigma off} and \eqref{AB gluon prop} we easily obtain
%\begin{eqnarray}
%\langle \sigma_{off}(0) \rangle & = &  \frac{2g^{2}}{3}\int\!\!\frac{d^3 p}{(2\pi)^3} \frac{1}{p^{2}}\left(\frac{1}{p^{2} + \frac{\nu^{2}}{2}g^{2}} + \frac{1}{p^{2} + \frac{\nu^{2}}{2}(g^{2}+g'^{2})} \right)  \nonumber \\
% & =& \frac{g^{2}}{3\pi^{2}}\int_{0}^{\infty} \!\!dp\left(\frac{1}{p^{2} + \frac{\nu^{2}}{2}g^{2}} + \frac{1}{p^{2} + \frac{\nu^{2}}{2}(g^{2}+g'^{2})}\right)\;.
%\label{sgoff1}
%\end{eqnarray}
%Analogously, for $\langle \sigma_{diag}(0)\rangle$ one gets
%\begin{equation}
%\langle \sigma_{diag}(0) \rangle = \frac{4g^{2}}{3}\int\!\!\frac{d^{3}p}{(2\pi)^{3}}\frac{1}{p^{2}}\left(\frac{1}{p^{2} + \frac{\nu^{2}}{2}g^{2}}\right)
%= \frac{2g^{2}}{3\pi^{2}}\int_{0}^{\infty}\!\! \frac{dp}{p^{2} + \frac{\nu^{2}}{2}g^{2}}\;.
%\label{sgdiag1}
%\end{equation}
% As
%\begin{eqnarray}
%\int_{0}^{\infty} \frac{dp}{p^{2} + m^{2}} &=& \frac{\pi}{2m}\;,
%\label{decomp1}
%\end{eqnarray}
%we found for $\langle \sigma_{off}(0)\rangle$ and $\langle \sigma_{diag}(0)\rangle$
%\begin{subequations} \label{rmn1} \begin{eqnarray}
%\langle \sigma_{off}(0)\rangle &=& \frac{g}{3\sqrt{2}\pi\nu}(1 + \cos(\theta_{W})) \;, \\
%\langle \sigma_{diag}(0)\rangle &=& \frac{2g}{3\sqrt{2}\pi\nu}\;,
%\end{eqnarray} \end{subequations} 
%where
%\begin{equation}
%\cos(\theta_{W}) = \frac{g}{\sqrt{g^{2}+g'^{2}}}
%\label{thW}
%\end{equation}
%is the Weinberg angle. For the integration domain of the Yang--Mills field to be the configuration space inside the first Gribov horizon, we need that $\langle \sigma_{off}(0)\rangle$ and $\langle \sigma_{diag}(0)\rangle$  be less than one. Thus,
%\begin{subequations} \label{conds} \begin{eqnarray}
%(1+\cos(\theta_{W}))\frac{g}{\nu} &<& 3\sqrt{2}\pi \label{firstcond} \\
%2\frac{g}{\nu} &<& 3\sqrt{2}\pi \label{secondcond} \;.
%\end{eqnarray} \end{subequations}
%These two conditions make phase space fall apart in three regions, as depicted in \figurename\ \ref{regionsdiag}.
%\begin{itemize}
%	\item If $g/\nu<3\pi/\sqrt2$, neither Gribov parameter is necessary to make the integration cut off at the Gribov horizon. In this regime the theory is unmodified from the usual perturbative electroweak theory.
%	\item In the intermediate case $3\pi/\sqrt2<g/\nu<3\sqrt2\pi/(1+\cos\theta_W)$ only one of the two Gribov parameters,  $\beta$, is necessary. The off-diagonal ($W$) gauge bosons will see their propagators modified due to the presence of a nonzero $\beta$, while the $Z$ boson and the photon $A$ remain untouched.
%	\item In the third phase, when $g/\nu>3\sqrt2\pi/(1+\cos\theta_W)$, both Gribov parameters are needed, and all propagators are influenced by them. The off-diagonal gauge bosons are confined. The behavior of the diagonal gauge bosons depends on the values of the couplings, and the third phase falls apart into two parts, as detailed in section \ref{zandgamma}.
%\end{itemize}
%
%\begin{figure}\begin{center}
%\includegraphics[width=.25\textwidth]{fourregions.pdf}
%\caption{There appear to be four regions in phase space. The region I is defined by condition \eqref{secondcond} and is characterized by ordinary Yang--Mills--Higgs behavior (massive $W$ and $Z$ bosons, massless photon). The region II is defined by \eqref{firstcond} while excluding all points of region I --- this region only has electrically neutral excitations, as the $W$ bosons are confined (see Section \ref{sect6}); the massive $Z$ and the massless photon are unmodified from ordinary Yang--Mills--Higgs behavior. Region III has confined $W$ bosons, while both photon and $Z$ particles are massive due to influence from the Gribov horizon; furthermore there is a negative-norm state. In region IV all $SU(2)$ bosons are confined and only a massive photon is left. Mark that the tip of region III is hard to deal with numerically --- the discontinuity shown in the diagram is probably an artefact due to this difficulty.  Details are collected in Section \ref{sect7}. \label{regionsdiag}}
%\end{center}\end{figure}
%
%
%
%
%
%
%
%
%
%
%
%
%
%
%
%
%
%\subsubsection{The off-diagonal ($W$) gauge bosons} 
%\label{sect6}
%Let us first look at the behavior of the off-diagonal bosons under the influence of the Gribov horizon. The propagator \eqref{ww}  only contains the $\beta$ Gribov parameter, meaning that $\omega$ need not be considered here.
%
%As found in the previous section, the parameter $\beta$ is not necessary in the regime $g/\nu<3\pi/\sqrt2$ (region I), due to the ghost form factor $\langle\sigma_{diag}(0)\rangle$ always being smaller than one. In this case, the off-diagonal boson propagator is simply of massive type:
%\begin{equation}
%  \langle W^{+}_{\mu}(p)W^{-}_{\nu}(-p) \rangle = \frac{1}{p^{2} + \frac{\nu^{2}}{2}g^{2}}\left(\delta_{\mu\nu} - \frac{p_{\mu}p_{\nu}}{p^{2}}\right) \;.
%\end{equation}
%
%In the case that $g/\nu>3\pi/\sqrt2$ (regions II, III, and IV), the relevant ghost form factor is not automatically smaller than one anymore, and the Gribov parameter $\beta$ becomes necessary. The value of $\beta^{\ast}$ is determined from the gap equations \eqref{beta gap eq}. After rewriting the integrand in partial fractions, the integral in the equation becomes of standard type, and we readily find the solution
%\begin{equation}
%  \beta^{\ast} = \frac{3g^2}{32} \left(\frac{g^2}{2\pi^2}-\nu^2\right)^2 \;.
%\end{equation}
%Mark that, in order to find this result, we had to take the square of both sides of the equation twice. One can easily verify that, in the region $g/\nu>3\pi/\sqrt2$ which concerns us, no spurious solutions were introduced when doing so.
%
%With this value of $\beta^{\ast}$, the off-diagonal propagator can be rewritten as
%\begin{multline}
%  \langle W_\mu^{+}(p)W_\nu^{-}(-p)\rangle = \frac{\pi/g^3}{\sqrt{\frac{g^2}{4\pi^2}-\nu^2}} \left(\frac{\frac{g^3}{2\pi}\sqrt{\frac{g^2}{4\pi^2}-\nu^2}-\frac i4\nu^2g^2}{p^2+\frac{\nu^2}4g^2+i\frac{g^3}{2\pi}\sqrt{\frac{g^2}{4\pi^2}-\nu^2}} + \frac{\frac{g^3}{2\pi}\sqrt{\frac{g^2}{4\pi^2}-\nu^2}+\frac i4\nu^2g^2}{p^2+\frac{\nu^2}4g^2-i\frac{g^3}{2\pi}\sqrt{\frac{g^2}{4\pi^2}-\nu^2}}\right) \\ \times \left(\delta_{\mu\nu}-\frac{p_\mu p_\nu}{p^2}\right) \;.
%\end{multline}
%It clearly displays two complex conjugate poles  because $g/\nu>3\pi/\sqrt2$. As such, the off-diagonal propagator  cannot describe a physical excitation of the physical spectrum, being adequate for a confining phase. This means that the off-diagonal components of the gauge field are confined in the region $g/\nu>3\pi/\sqrt2$.
%
%
%
%
%
%
%
%
%
%
%
%
%
%
%
%
%\subsubsection{The diagonal $SU(2)$ boson and the photon field} \label{zandgamma} \label{sect7}
%The other two gauge bosons --- the $A^3_\mu$ and the $B_\mu$ --- have their propagators given by \eqref{a3a3}, \eqref{bb}, and \eqref{ba3} or equivalently --- the $Z_\mu$ and the $A_\mu$ --- by \eqref{zz}, \eqref{aa} and \eqref{az}. Here, $\omega$ is the only of the two Gribov parameters present.
%
%In the regime $g/\nu<3\sqrt2\pi/(1+\cos\theta_W)$ (regions I and II) this $\omega$ is not necessary to restrict the region of integration to within the first Gribov horizon. Due to this, the propagators are unmodified in comparison to the perturbative case:
%\begin{subequations} \label{propsrewrite} \begin{gather}
%\langle Z_{\mu}(p) Z_{\nu}(-p) \rangle = \frac{1}{p^{2} + \frac{\nu^{2}}{2}(g^{2} + g'^{2})} \left(\delta_{\mu\nu} - \frac{p_{\mu}p_{\nu}}{p^{2}}\right)\;, \\
%\langle A_{\mu}(p) A_{\nu}(-p) \rangle = \frac{1}{p^{2}} \left(\delta_{\mu\nu} - \frac{p_{\mu}p_{\nu}}{p^{2}}\right) \;.
%\end{gather} \end{subequations}
%
%In the region $g/\nu>3\sqrt2\pi/(1+\cos\theta_W)$ (regions III and IV) the Gribov parameter $\omega$ does become necessary, and it has to be computed by solving its gap equation, eq. \eqref{omega gap eq}. Due to its complexity it seems impossible to do so analytically. Therefore we turn to numerical methods. Using Mathematica the gap equation can be straighforwardly solved for a list of values of the couplings. Then we determine the values where the propagators have poles. 
%
%The denominators of the propagators are a polynomial which is of third order in $p^2$. There are two cases: there is a small region in parameter space where the polynomial has three real roots, and for all other values of the couplings there are one real and two complex conjugate roots. In \figurename\ \ref{regionsdiag} these zones are labeled III and IV respectively.
%
%Ordinarily, one would like to diagonalize the propagator matrix in order to separate the states present in the theory. In our case, however, doing so requires a nonlocal transformation, and the result will contain square roots containing the momentum of the fields. It seems to be more enlightening to, instead, perform a partial fraction decomposition. If we look at the two-point functions of the $A_3$ and $B$ fields \eqref{a3a3}, \eqref{bb}, and \eqref{ba3}, we can succinctly write those as\footnote{The projector $\delta_{\mu\nu}-\frac{p_\mu p_\nu}{p^2}$ will be ignored in this discussion, as it does not change anything nontrivial here.}
%\begin{equation}
%	\Delta_{ij} = \frac{f_{ij}(p^2)}{P(p^2)} \;.
%\end{equation}
%Here, the indices $i,j$ run over $A_3$ and $B$. The functions $f_{ij}(p^2)$ are polynomials of $p^2$ of at most second order, and the function $P(p^2)$ is a third-order polynomial of $p^2$. Furthermore, if we consider the functions $f_{ij}(p^2)$ to be the elements of $2\times2$ matrix, we can see that the determinant of this matrix is nothing but $p^2P(p^2)$.
%
%Let us assume that we know what the roots of $P(p^2)$ are, and call them $-m^2_n$ with $n=1,2,3$. It is then obvious that we can rewrite $P(p^2)$ as $(p^2+m_1^2)(p^2+m_2^2)(p^2+m_3^2)$. We can then perform a decomposition in partial fractions. We will have something of the form
%\begin{equation}
%	\frac{f_{ij}(p^2)}{P(p^2)} = \sum_{n=1}^3 \frac{\alpha_{ij,n}}{p^2+m_n^2} \;.
%\end{equation}
%The constants $\alpha_{ij,n}$ can be readily determined the usual way and we get
%\begin{equation}
%	\frac{f_{ij}(-m_1^2)}{(-m_1^2+m_2^2)(-m_1^2+m_3^2)} = \alpha_{ij,1}
%\end{equation}
%and analogously for $n=2,3$. In conclusion we find
%\begin{multline}
%	\Delta_{ij} = \frac{f_{ij}(p^2)}{P(p^2)} = \frac{f_{ij}(-m_1^2)}{(-m_1^2+m_2^2)(-m_1^2+m_3^2)} \frac1{p^2+m_1^2} \\ + \frac{f_{ij}(-m_2^2)}{(m_1^2-m_2^2)(-m_2^2+m_3^2)} \frac1{p^2+m_2^2} + \frac{f_{ij}(-m_3^2)}{(m_1^2-m_3^2)(m_2^2-m_3^2)} \frac1{p^2+m_3^2} \;.
%\end{multline}
%We can again interpret the constants $f_{ij}(-m_n^2)$ as elements of some $2\times2$ matrices, and we find that the determinants of these matrices are equal to $-m_n^2P(-m_n^2) = 0$, as the $-m_n^2$ are roots of the polynomial $P(p^2)$. Now it is obvious that a $2\times2$ matrix $\mx A$ with zero determinant can always be written in the form $\mx A = v v^T$ with $v$ some $2\times1$ matrix. Furthermore, this vector $v$ has norm $v^Tv = \tr\mx A$. This means that we can write our matrices in the form $\mx A = \tr\mx A \hat v \hat v^T$ where $\hat v$ is now the unit vector parallel to $v$. Therefore, let us write $f_{ij}(-m_n^2) = (f_{11}(-m_n^2)+f_{22}(-m_n^2)) \hat v_i^n \hat v_j^n$, resulting in
%\begin{multline} \label{partfracdec}
%	\Delta_{ij} = \frac{f_{11}(-m_1^2)+f_{22}(-m_1^2)}{(-m_1^2+m_2^2)(-m_1^2+m_3^2)} \frac1{p^2+m_1^2} \hat v_i^1\hat v_j^1 \\ + \frac{f_{11}(-m_2^2)+f_{22}(-m_2^2)}{(m_1^2-m_2^2)(-m_2^2+m_3^2)} \frac1{p^2+m_2^2} \hat v_i^2\hat v_j^2 + \frac{f_{11}(-m_3^2)+f_{22}(-m_3^2)}{(m_1^2-m_3^2)(m_2^2-m_3^2)} \frac1{p^2+m_3^2} \hat v_i^3\hat v_j^3 \;.
%\end{multline}
%The vectors $v_i^n$ can be interpreted as linear combinations of the $A_3$ and $B$ fields. Decomposing the two-point functions in this way, we thus find three ``states'' $v_1^n A_3 + v_2^n B$. These states are not orthogonal to each other (which would be impossible for three vectors in two dimensions). The coefficients in front of the Yukawa propagators will be the residues of the poles, and they have to be positive for a pole to correspond to a physical excitation.  The poles can be extracted from the zeros at $p^2_\ast$ of $P(p^2)=p^6+\frac{\nu^2}{2}p^4(g^2+g'^2)+\frac{g^2\omega}{3}(2p^2+\nu^2g'^2)$, viz.
%\begin{subequations} \begin{eqnarray}
% p^2_\ast&=& \frac{1}{6}\left\{ (g^{2}+g'^{2})\nu^{2} + \left[ (g^{2}+g'^{2})^{2}\nu^{4} - 8g^{2}\omega\right]\left[ (g^{2}+g'^{2})^{3}\nu^{6} - 12g^{2}(g^{2}-2g^{2})\nu^{2}\omega +  \right. \right. \nonumber \\
%&{}&+ \left. 2\sqrt{2}\sqrt{ g^{2}\omega \left( 9g'^{2}(g^{2}+g'^{2})^{3}\nu^{8} - 6g^{2}(g^{4}+20g^{2}g'^{2}-8g'^{4})\nu^{4}\omega + 64g^{4}\omega^{2}  \right)} \right]^{-1/3} + \nonumber \\
%&{}&+ \left[ (g^{2}+g'^{2})^{3}\nu^{6} - 12g^{2}(g^{2}-2g^{2})\nu^{2}\omega + 2\sqrt{2}\left( g^{2}\omega \left( 9g'^{2}(g^{2}+g'^{2})^{3}\nu^{8} - \right. \right. \right. \nonumber \\
%&{}&- \left. \left. \left. \left. 6g^{2}(g^{4}+20g^{2}g'^{2}-8g'^{4})\nu^{4}\omega + 64g^{4}\omega^{2}  \right)\right)^{1/2}  \right]^{1/3} \right\}
%\label{m1}
%\end{eqnarray}
%\begin{eqnarray}
% p^2_\ast&=& \frac{1}{6}\left\{  (g^{2}+g'^{2})\nu^{2} - \frac{1}{2} \left[ (g^{2}+g'^{2})^{2}\nu^{4} - 8g^{2}\omega\right]\left[ (g^{2}+g'^{2})^{3}\nu^{6} - 12g^{2}(g^{2}-2g^{2})\nu^{2}\omega +   \right. \right. \nonumber \\
%&{}&+ \left. 2\sqrt{2}\sqrt{ g^{2}\omega \left( 9g'^{2}(g^{2}+g'^{2})^{3}\nu^{8} - 6g^{2}(g^{4}+20g^{2}g'^{2}-8g'^{4})\nu^{4}\omega + 64g^{4}\omega^{2}  \right)} \right]^{-1/3} - \nonumber \\
%&{}&- \frac{1}{2}\left[ (g^{2}+g'^{2})^{3}\nu^{6} - 12g^{2}(g^{2}-2g^{2})\nu^{2}\omega + 2\sqrt{2}\left( g^{2}\omega \left( 9g'^{2}(g^{2}+g'^{2})^{3}\nu^{8} - \right. \right. \right. \nonumber \\
%&{}&- \left. \left. \left. \left. 6g^{2}(g^{4}+20g^{2}g'^{2}-8g'^{4})\nu^{4}\omega + 64g^{4}\omega^{2}  \right)\right)^{1/2}  \right]^{1/3} \right. \nonumber \\
%&{}&+ i\frac{\sqrt{3}}{2}\left[ (g^{2}+g'^{2})^{2}\nu^{4} - 8g^{2}\omega\right]\left[ (g^{2}+g'^{2})^{3}\nu^{6} - 12g^{2}(g^{2}-2g^{2})\nu^{2}\omega +  \right. \nonumber \\
%&{}&+ \left. 2\sqrt{2}\sqrt{ g^{2}\omega \left( 9g'^{2}(g^{2}+g'^{2})^{3}\nu^{8} - 6g^{2}(g^{4}+20g^{2}g'^{2}-8g'^{4})\nu^{4}\omega + 64g^{4}\omega^{2}  \right)} \right]^{-1/3} + \nonumber \\
%&{}&+ i\frac{\sqrt{3}}{2}\left[ (g^{2}+g'^{2})^{3}\nu^{6} - 12g^{2}(g^{2}-2g^{2})\nu^{2}\omega + 2\sqrt{2}\left( g^{2}\omega \left( 9g'^{2}(g^{2}+g'^{2})^{3}\nu^{8} - \right. \right. \right. \nonumber \\
%&{}&- \left. \left. \left. \left. 6g^{2}(g^{4}+20g^{2}g'^{2}-8g'^{4})\nu^{4}\omega + 64g^{4}\omega^{2}  \right)\right)^{1/2}  \right]^{1/3} \right\}\;,
%\label{m2}
%\end{eqnarray}
%and
%\begin{eqnarray}
% p^2_\ast &=& \frac{1}{6}\left\{  (g^{2}+g'^{2})\nu^{2} - \frac{1}{2} \left[ (g^{2}+g'^{2})^{2}\nu^{4} - 8g^{2}\omega\right]\left[ (g^{2}+g'^{2})^{3}\nu^{6} - 12g^{2}(g^{2}-2g^{2})\nu^{2}\omega +   \right. \right. \nonumber \\
%&{}&+ \left. 2\sqrt{2}\sqrt{ g^{2}\omega \left( 9g'^{2}(g^{2}+g'^{2})^{3}\nu^{8} - 6g^{2}(g^{4}+20g^{2}g'^{2}-8g'^{4})\nu^{4}\omega + 64g^{4}\omega^{2}  \right)} \right]^{-1/3} - \nonumber \\
%&{}&- \frac{1}{2}\left[ (g^{2}+g'^{2})^{3}\nu^{6} - 12g^{2}(g^{2}-2g^{2})\nu^{2}\omega + 2\sqrt{2}\left( g^{2}\omega \left( 9g'^{2}(g^{2}+g'^{2})^{3}\nu^{8} - \right. \right. \right. \nonumber \\
%&{}&- \left. \left. \left. \left. 6g^{2}(g^{4}+20g^{2}g'^{2}-8g'^{4})\nu^{4}\omega + 64g^{4}\omega^{2}  \right)\right)^{1/2}  \right]^{1/3} \right. \nonumber \\
%&{}&- i\frac{\sqrt{3}}{2}\left[ (g^{2}+g'^{2})^{2}\nu^{4} - 8g^{2}\omega\right]\left[ (g^{2}+g'^{2})^{3}\nu^{6} - 12g^{2}(g^{2}-2g^{2})\nu^{2}\omega +  \right. \nonumber \\
%&{}&+ \left. 2\sqrt{2}\sqrt{ g^{2}\omega \left( 9g'^{2}(g^{2}+g'^{2})^{3}\nu^{8} - 6g^{2}(g^{4}+20g^{2}g'^{2}-8g'^{4})\nu^{4}\omega + 64g^{4}\omega^{2}  \right)} \right]^{-1/3} + \nonumber \\
%&{}&- i\frac{\sqrt{3}}{2}\left[ (g^{2}+g'^{2})^{3}\nu^{6} - 12g^{2}(g^{2}-2g^{2})\nu^{2}\omega + 2\sqrt{2}\left( g^{2}\omega \left( 9g'^{2}(g^{2}+g'^{2})^{3}\nu^{8} - \right. \right. \right. \nonumber \\
%&{}&- \left. \left. \left. \left. 6g^{2}(g^{4}+20g^{2}g'^{2}-8g'^{4})\nu^{4}\omega + 64g^{4}\omega^{2}  \right)\right)^{1/2}  \right]^{1/3} \right\}\;,
%\label{m3}
%\end{eqnarray} \end{subequations}
%
%It is obviously not possible to disentangle the three (vector-like) degrees of freedom corresponding to these different masses using only two fields. Nonetheless, it is possible to derive a diagonalization of the $1PI$ propagator matrix. After the saddle point approximation and using the $A_\mu$ and $Z_\mu$ field variables, it is not difficult to see that the tree level $(A_\mu,Z_\mu)$ sector of the action arising from eq.~(\ref{Zquad1}) can be reformulated as
%\begin{multline}\label{dd1}
%  \int d^4p \left(\frac{1}{2}Z_\mu(p)(p^2+g^2\nu^2)Z_\mu(-p)+\frac{1}{2}A_\mu(p)p^2A_\mu(-p)+i\frac{1}{2}\sqrt{\frac{2}{3}\omega_\ast}\left((g'A_\mu(p)-gZ_\mu(p))V_\mu(-p)+(p\leftrightarrow-p)\right)\right. \\ \left.+ \frac{1}{2}V_\mu(p)p^2V_\mu(-p)\right)
%\end{multline}
%while working immediately on-shell, viz.~using $\p_\mu A_\mu=\p_\mu Z_\mu=0$. The equivalence with the original action can be straightforwardly established by integrating over the $V_\mu$ field. Here, we introduced the latter auxiliary field by hand, but it can be shown in general --- at least for the pure Yang--Mills case; for the current Yang--Mills--Higgs generalization this deserved further investigation at a later stage --- that the all-order no pole condition can be brought in local form by introducing a suitable set of boson and fermion auxiliary fields\footnote{These ghost fields are necessary to eliminate the determinant when integrating over the extra fields.}, see e.g.~\cite{Vandersickel:2012tz,Zwanziger:1989mf,Baulieu:2009ha,Capri:2012wx}.
%
%Having now 3 fields at our disposal with still 3 masses, there is better hope to diagonalize the previous action. First of all, the special limits $g'\to0$ and/or $\omega\to0$ are simply clear at the level of the action (\ref{dd1}). Secondly, the $1PI$ propagator matrix of (\ref{dd1}) only displays a $p^2$-dependence on the diagonal, each time of the form\footnote{In the formulation without the $V_\mu$ field this is not the case. One of the consequences is the appearance of the aforementioned momentum dependent square roots when a diagonalization in terms of the two fields $A_\mu$ and $Z_\mu$ is attempted.} $p^2+\ldots$. As such, the 3 eigenvalues will be of the form $p^2+m_i^2$. Upon using the associated eigenvectors, the action (\ref{dd1}) can then be simply diagonalized to
%\begin{equation}\label{dd2}
%  \int d^4p \left(\frac{1}{2}\lambda_\mu(p)(p^2+m_1^2)\lambda_\mu(-p)+\frac{1}{2}\eta_\mu(p)(p^2+m_2^2)\eta_\mu(-p)+\frac{1}{2}\kappa_\mu(p)(p^2+m_3^2)\kappa_\mu(-p)\right)\,,
%\end{equation}
%where the $\lambda_\mu,\eta_\mu,\kappa_\mu$ are the ``generalized $i$-particles'' of the current model, adopting the language of \cite{Baulieu:2009ha}. They are related to the original fields $A_\mu$, $Z_\mu$ and $V_\mu$ by momentum-independent linear transformations. The quadratic form appearing in \eqref{dd2} displays a standard propagator structure, with the possibility that two of the mass poles can be complex conjugate.
%
%
%
%
%
%
%\subsubsection{Three real roots (region III)}
%\begin{figure}\begin{center}
%\includegraphics[width=.5\textwidth]{figures/threems.pdf}
%\caption{The mass-squareds of the massive excitations found in the region where there are three massive poles (region III). \label{threems}}
%\end{center}\end{figure}
%Region III is defined by the polynomial in the denominators of \eqref{a3a3}, \eqref{bb}, and \eqref{ba3} having three real roots. This region is sketched in \figurename\ \ref{regionsdiag}. (Mark that the tip of the region is distorted due to the difficulty in accessing this part numerically.) The square of the masses corresponding to these three roots are plotted in \figurename\ \ref{threems}.
%
%We computed the residues of these poles, the expression whereof can be read off in the partial fraction decomposition \eqref{partfracdec}. Only the two of the three roots we identified have a positive residue and can correspond to physical states, being the one with highest and the one with lowest mass squared. The third of the roots, the one of intermediate value, has negative residue and thus belongs to some negative-norm state, which cannot be physical.
%
%All three states have nonzero mass for nonzero values of the electromagnetic coupling $g'$, with the lightest of the states becoming massless in the limit $g'\to0$. In this limit we recover the behavior found in this regime in the pure $SU(2)$ case \cite{Capri:2012cr} (the $Z$-boson field having one physical and one negative-norm pole in the propagator) with a massless fermion decoupled from the non-Abelian sector.
%
%
%
%
%\subsubsection{One real root (region IV)}
%In the remaining part of parameter space, there is only one state with real mass-squared. The two other roots of the polynomial in the denominators of \eqref{a3a3}, \eqref{bb}, and \eqref{ba3} have nonzero imaginary part and are complex conjugate to each other. The square of the masses corresponding to these roots are plotted in \figurename\ \ref{realmass} for the real root, and in \figurename\ \ref{ccmass} (real and imaginary part) for the complex conjugate roots. In order to determine whether the pole coming from the real root corresponds to a physical particle excitation, we computed its residue, which can be read off in the partial fraction decomposition \eqref{partfracdec}. 
%
%The result is plotted in \figurename\ \ref{resrealmass}. It turns out the residue is always positive, meaning that this excitation has positive norm and can thus be interpreted as a physical, massive particle. The poles coming from the complex roots cannot, of course, correspond to such physical excitations.
%
%\begin{figure}\begin{center}
%\includegraphics[width=.5\textwidth]{figures/realmass.pdf}
%\caption{The mass-squared of the one physical massive excitation found in region IV. \label{realmass}}
%\end{center}\end{figure}
%
%\begin{figure} \begin{center}
%\includegraphics[width=.5\textwidth]{figures/res.pdf}
%\caption{The residue of the pole of the photon propagator which is depicted in \figurename\ \ref{realmass}. It turns out to be positive for all values of the couplings within the region IV. \label{resrealmass}}
%\end{center} \end{figure}
%
%\begin{figure}\begin{center}
%\includegraphics[width=.5\textwidth]{figures/realpartccmass.pdf}\includegraphics[width=.5\textwidth]{figures/impartccmass.pdf}
%\caption{The real (left) and imaginary (right) parts of the mass-squared of the other two, complex conjugate, poles. \label{ccmass}}
%\end{center}\end{figure}
%In the limit $g'\to0$ we once more recover the corresponding results already found in the pure $SU(2)$ case \cite{Capri:2012cr} (two complex conjugate poles in the propagator of the non-Abelian boson field) plus a massless photon not influenced by the non-Abelian sector.
%
%
%
%
%
%
%
%
%
%
%
%
%
%
%
%
%
%
%
%
%
%
%\subsection{Results of $d=4$}
%\label{d=4}
%
%
%In this section we define the notation and conventions used throughout the paper. The system we study is the electroweak sector of the $4d$ standard model, thereby generalizing our earlier works \cite{Capri:2012cr,Capri:2012ah,Capri:2013gha}. In the following sections we consider in detail the possible effects of Gribov copies in such a setting, based on the original Gribov no-pole analysis \cite{Gribov:1977wm,Sobreiro:2005ec}. A summary of our results can be found in section 6.
%
%
%
%
%
%
%
%
%
%
%
%
%\subsubsection{The vacuum energy}
%Looking at the above propagators, beside the decoupling of the $U(1)$ gauge field from the $SU(2)$ gauge field, one should note the likeness between the diagonal and off-diagonal propagators, though in general the two Gribov parameters, $\omega^\ast$ and $\beta^\ast$, differ. Therefore, given the important role played by the gap equations, it seems to be worth to analyze the two gap equations \eqref{beta gap eq} and \eqref{omega gap eq} in the limit $g'\to 0$. In this limit, the gap equations \eqref{beta gap eq} and \eqref{omega gap eq} become
%\begin{equation}
%\label{omega gapeq g'zero}
%\frac{3}{2}g^{2} \int\!\!\frac{d^{4} p}{(2\pi)^{4}}\, \frac{1}{p^{4}+\frac{\nu^{2}}{2}g^{2}p^{2}+\frac{\omega^{\ast}}{2}g^{2}} = 1\;, \qquad
%\frac{3}{2}g^{2} \int\!\! \frac{d^{4}p}{(2\pi)^{4}}\, \frac{1}{p^{4}+\frac{\nu^{2}}{2}g^{2}p^{2}+\frac{g^{2}}{2}\beta^{\ast}} =1\;.
%\end{equation}
%It is clear that these two equations are identical. Thus, when $g' \to 0$, there is only one gap equation and, therefore, only one Gribov parameter. Consequently, the diagonal and off-diagonal propagators of \eqref{propgzro} are identical. These results coincide with what was found in \cite{Capri:2012ah} in the case of Higgs field in the fundamental representation.
%
%Furthermore, from equation \eqref{Zf eq}, we get for the vacuum energy,
%\begin{equation}
%\mathcal{E}_{\omega^{\ast}} = \frac{3}{2}\omega^{\ast} - \frac{9}{2}g^2 \int\!\! \frac{d^{4}p}{(2\pi)^{4}}\log \! \left(p^{4} + p^{2}\frac{\nu^{2}}{2}g^{2} + \frac{\omega^{\ast}}{2}g^{2}\right)\;,
%\end{equation}
%which, again, is in agreement with the expression for the vacuum energy for the case of $SU(2)$ in the fundamental representation, upon redefining  $\frac{\omega^{\ast}}{2} = \frac{\vartheta^{\ast}}{3}$.
%
%
%
%
%
%
%
%
%
%
%
%
%
%
%
%
%
%\subsubsection{About $\sigma_\text{off}(0)$ and $\sigma_\text{diag}(0)$ without the Gribov parameters} 
%\label{sect3}
%Before trying to solve the gap equations, it seems to be worthwhile to study what happens with $\sigma_\text{off}(0)$ and $\sigma_\text{diag}(0)$ in the absence of the Gribov parameters, which will allow us to search for regions where the Gribov parameters $\omega$ and $\beta$ are unnecessary, which happens whenever $\sigma_\text{off}(0)$ and/or $\sigma_\text{diag}(0)$ are less than one.
%
%Thus, we have
%\begin{equation}
%\label{sgoff} \langle \sigma_\text{off}(0) \rangle = \frac{3g^{2}}{4}\int\!\!\frac{d^{4}p}{(2\pi)^{4}}\,\frac{1}{p^{2}}\left(\frac{1}{p^{2}+\frac{\nu^{2}}{2}g^{2}}+\frac{1}{p^{2}+\frac{\nu^{2}}{2}(g^{2}+g'^{2})}\right)\,,\qquad
%\langle \sigma_\text{diag}(0) \rangle = \frac{3g^{2}}{2}\int\!\!\frac{d^{4}p}{(2\pi)^{4}}\,\frac{1}{p^{2}}\frac{1}{\left(p^{2}+\frac{\nu^{2}}{2}g^{2}\right)}\;.
%\end{equation}
%Using standard techniques, this gives
%\begin{equation}
%\label{constas} \langle \sigma_\text{off}(0) \rangle = \frac{1}{2}\left(1-\frac{3g^{2}}{32\pi^{2}}\ln(2a)\right) + \frac{1}{2}\left(1-\frac{3g^{2}}{32\pi^{2}}\ln(2 a')\right)\,, \qquad
%\langle \sigma_\text{diag}(0) \rangle = 1-\frac{3g^{2}}{32\pi^{2}}\ln(2a)\;,
%\end{equation}
%where
%\begin{equation}
%\label{consta} a = \frac{\nu^{2}g^{2}}{4\bar{\mu}^{2}\,e^{1-\frac{32 \pi^{2}}{3g^{2}}}}\,, \qquad a' = \frac{\nu^{2}(g^{2}+g'^{2})}{4\bar{\mu}^{2}\,e^{1-\frac{32 \pi^{2}}{3g^{2}}}} = a \frac{g^{2}+g'^{2}}{g^{2}} = \frac{a}{\cos^{2}(\theta_{W})}\;,
%\end{equation}
%such that the off-diagonal ghost form factor can be rewritten as
%\begin{equation}
%\langle \sigma_\text{off}(0) \rangle = 1 - \frac{3g^{2}}{32\pi^{2}}\ln\frac{2a}{\cos(\theta_{W})}\;.
%\label{sigmaoff4}
%\end{equation}
%where $\theta_{W}$ is the Weinberg angle. With the 2nd expression of \eqref{sgoff} and \eqref{sigmaoff4} we are able to identify three possible regions:
%\begin{itemize}
%\item Region I, where $\langle \sigma_\text{diag}(0)\rangle < 1$ and $\langle \sigma_\text{off}(0)\rangle < 1$, meaning $2a > 1$. In this case the Gribov parameters are both zero so that we have the massive $W^{\pm}$ and $Z$, and a massless photon, as in \eqref{WZAgluonprop1}, in what we can call the Higgs phase.
%\item Region II, where $\langle \sigma_\text{diag}(0)\rangle > 1$ and $\langle \sigma_\text{off}(0)\rangle < 1$, or equivalently $\cos \theta_{W} < 2a < 1$. In this region we have $\omega = 0$ while $\beta \neq 0$, leading to a modified $W^{\pm}$ propagator, and a free photon and a massive $Z$ boson.
%\item The remaining parts of parameter space, where $\langle \sigma_\text{diag}(0)\rangle > 1$ and $\langle \sigma_\text{off}(0)\rangle > 1$, or $0 < 2a < \cos\theta_{W}$. In this regime we have both $\beta \neq 0$ and $\omega \neq 0$, which modifies the $W^{\pm}$, $Z$ and photon propagators. Furthermore this region will fall apart in two separate regions III and IV due to different behavior of the propagators.
%\end{itemize}
%
%
%
%
%
%
%
%
%
%
%
%
%
%
%
%
%
%\subsubsection{The off-diagonal gauge bosons} \label{sect4}
%Let us first look at the behavior of the off-diagonal bosons under the influence of the Gribov horizon. The propagator \eqref{ww} only contains the $\beta$ Gribov parameter, meaning $\omega$ does not need be considered here.
%
%As found in the previous section, this $\beta$ is not necessary in the regime $a>1/2$, due to the ghost form factor $\langle\sigma_\text{diag}(0)\rangle$ always being smaller than one. In this case, the off-diagonal boson propagator is simply of the massive type:
%\begin{equation}
%	\langle W^{+}_{\mu}(p)W^{-}_{\nu}(-p) \rangle = \frac{1}{p^{2} + \frac{\nu^{2}}{2}g^{2}}T_{\mu\nu}(p^2) \;.
%\end{equation}
%
%In the case that $a<1/2$, the relevant ghost form factor is not automatically smaller than one anymore, and the Gribov parameter $\beta$ becomes necessary. The value of $\beta$ is given by the gap equations \eqref{beta gap eq}, which has exactly the same form as in the case without electromagnetic sector. Therefore the results will also be analogous. As the analysis is quite involved, we just quote the results here.
%
%For $1/e<a<1/2$ the off-diagonal boson field has two real massive poles in its two-point function. One of these has a negative residue, however. This means we find one physical massive excitation, and one unphysical mode in this regime. When $a<1/e$ the two poles acquire a nonzero imaginary part and there are no poles with real mass-squared left. In this region the off-diagonal boson propagator is of Gribov type, and the $W$ boson is completely removed from the spectrum. More details can be found in \cite{Capri:2012ah}.
%
%
%\begin{figure}\begin{center}
%\parbox{.5\textwidth}{\includegraphics[width=.5\textwidth]{figures/regions3d.pdf}} \quad \parbox{.4\textwidth}{\includegraphics[width=.4\textwidth]{figures/regionsslice.pdf}}
%\caption{Left is a plot of the region $a'<1/2$ (the region $a'>1/2$ covers all points with higher $\nu$). In red are points where the polynomial in the denominator of \eqref{a3a3} - \eqref{ba3} has three real roots, and in blue are the points where it has one real and two complex conjugate roots. At the right is a slice of the phase diagram for $g=10$. The region $a>1/2$ and $a'>1/2$ is labeled I, the region $a<1/2$ and $a'>1/2$ is II, and the region $a<1/2$ and $a'<1/2$ is split into the regions III (polynomial in the denominator of \eqref{a3a3} - \eqref{ba3} has three real roots, red dots in the diagram at the left) and IV (one real and two complex conjugate roots, blue dots in the diagram at the left). The dashed line separates the different regimes for off-diagonal SU(2) bosons (two real massive poles above the line, two complex conjugate poles below). \label{regionsdiag}}
%\end{center}\end{figure}
%
%
%
%
%
%
%
%
%
%
%
%
%
%
%
%\subsubsection{The diagonal SU(2) boson and the photon} \label{zandgamma}
%The two other gauge bosons --- the diagonal SU(2) boson and the photon, $Z_\mu$ and the $A_\mu$ --- have their propagators given by \eqref{zz}, \eqref{aa} and \eqref{az}. Here, $\omega$ is the only of the two Gribov parameters present.
%
%In the regime $a'>1/2$, $\omega$ is not necessary to restrict the region of integration to $\Omega$. Due to this, the propagators are unmodified in comparison to the perturbative case:
%\begin{equation} \label{propsrewrite}
%	\langle Z_{\mu}(p) Z_{\nu}(-p) \rangle = \frac{1}{p^{2} + \frac{\nu^{2}}{2}(g^{2} + g'^{2})} T_{\mu\nu}(p^2)\;, \quad
%	\langle A_{\mu}(p) A_{\nu}(-p) \rangle = \frac{1}{p^{2}}T_{\mu\nu}(p^2) \;.
%\end{equation}
%
%In the region $a'<1/2$ the Gribov parameter $\omega$ does become necessary, and it has to be computed by solving its gap equation. Due to its complexity it seems impossible to do so analytically. Therefore we turn to numerical methods. Using Mathematica the gap equation can be straighforwardly solved for a list of values for the couplings. In order to do this, we regularize the momentum integration by subtracting a term designed to cancel the large-$p^2$ divergence (as in the Pauli--Villars procedure):
%\begin{equation}
%	\frac{3}{2}g^{2}\int \frac{d^{4}p}{(2\pi)^{4}} \left(\frac{p^{2} + \frac{\nu^{2}}{2}g'^{2}}{p^{6} + \frac{\nu^{2}}{2}(g^{2} + g'^{2})p^{4} + \frac{\omega^{\ast}}{2}g^{2}p^{2} + \frac{\omega^{\ast}}{4}\nu^{2}g^{2}g'^{2}} - \frac1{(p^2+M^2)^2}\right) 	+ \frac{3}{2}g^{2}\int \frac{d^{4}p}{(2\pi)^{4}}\frac1{(p^2+M^2)^2} = 1 \;,
%\end{equation}
%where $M^2$ is an arbitrary mass scale. The second integral is readily computed by hand, whereas the first one converges and can be determined numerically. Once the $\omega$ parameter has been determined, we look at the propagators to investigate the nature of the spectrum.
%
%The denominators of the propagators are a polynomial which is of third order in $p^2$. There are two cases: there is a small region in parameter space where the polynomial has three real roots, and for all other values of the couplings there are one real and two complex conjugate roots. In \figurename\ \ref{regionsdiag} these zones are labeled III and IV respectively.
%
%Just as in \cite{Capri:2013gha} we can decompose the propagator matrix $\Delta_{ij}$ (were $i$ and $j$ run over the fields $A_\mu^3$ and $B_\mu$) as
%\begin{eqnarray}
%	\Delta_{ij} &=& \frac{f_{11}(-m_1^2)+f_{22}(-m_1^2)}{(-m_1^2+m_2^2)(-m_1^2+m_3^2)} \frac1{p^2+m_1^2} \hat v_i^1\hat v_j^1 \nonumber \\
%	&+& \frac{f_{11}(-m_2^2)+f_{22}(-m_2^2)}{(m_1^2-m_2^2)(-m_2^2+m_3^2)} \frac1{p^2+m_2^2} \hat v_i^2\hat v_j^2 + \frac{f_{11}(-m_3^2)+f_{22}(-m_3^2)}{(m_1^2-m_3^2)(m_2^2-m_3^2)} \frac1{p^2+m_3^2} \hat v_i^3\hat v_j^3 \;. \label{partfracdec}
%\end{eqnarray}
%The vectors $v_i^n$ can be interpreted as linear combinations of the $A_3$ and $B$ fields. Decomposing the two-point functions in this way, we thus find three ``states'' $v_1^n A_3 + v_2^n B$. These states are not orthogonal to each other (which would be impossible for three vectors in two dimensions). The coefficients in front of the Yukawa propagators will be the residues of the poles, and they have to be positive for a pole to correspond to a physical excitation. From the analysis in \cite{Capri:2013gha} it also follows that $f_{11}(-m_n^2)+f_{22}(-m_n^2)$ will always be positive if the mass squared is real, which helps to determine the sign of the residue without having to explicitly compute it. The same results can also be attained by introducing ``generalized $i$-particles'' (see \cite{Baulieu:2009ha,Sorella:2010it}).
%
%As the model under consideration depends on three dimensionless parameters ($g$, $g'$ and $\nu/\bar\mu$), it is not possible to plot the parameter dependence of these masses in a visually comprehensible way. Therefore we limit ourselves to discussing the behavior we observed.
%
%In region III, when there are three real poles in the full two-point function, it turns out that only the two of the three roots we identified have a positive residue and can correspond to physical states, being the one with highest and the one with lowest mass squared. The third one, the root of intermediate value, has negative residue and thus belongs to some negative-norm state, which cannot be physical. All three states have nonzero mass for nonzero values of the electromagnetic coupling $g'$, with the lightest of the states becoming massless in the limit $g'\to0$. In this limit we recover the behavior found in this regime in the pure $SU(2)$ case \cite{Capri:2012ah} (the $Z$-boson field having one physical and one negative-norm pole in the propagator) with a massless boson decoupled from the non-Abelian sector.
%
%In region IV there is only one state with real mass squared --- the other two having complex mass squared, conjugate to each other --- and from the partial fraction decomposition follows that it has positive residue. This means that, in this region, the diagonal-plus-photon sector contains one physical massive state (becoming massless in the limit $g'\to0$), and two states that can, at best, be interpreted as confined.
%
%
