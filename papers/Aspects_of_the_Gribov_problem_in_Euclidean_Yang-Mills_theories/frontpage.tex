
\begin{titlepage}
\pagestyle{empty}
\setlength{\topmargin}{0cm}

\large

\begin{figure}
%\centering
\hspace{2.2cm}
%\begin{minipage}{.5\textwidth} DIEGO
  %\centering
% \includegraphics[width=30mm,left]{UGentLogo}  DIEGO
%  \captionof{figure}{A figure}
  %\label{fig:test1}
%\end{minipage}%  DIEGO
%\begin{minipage}{.3\textwidth} DIEGO
%\hspace{0.2cm}
% \includegraphics[width=30mm]{logo_uerj_cor} DIEGO
%\end{minipage}% DIEGO
%  \captionof{figure}{A figure}
  %\label{fig:test1}
\subfigure{\includegraphics[width=30mm]{ruglogotest}}
\hspace{4.4cm}
\subfigure{\includegraphics[width=27mm]{logo_uerj_cor}}
\end{figure}


% first column
\begin{minipage}[t]{0.45\textwidth}
%This is the first column.\\
%This is still in the first column. 
\begin{center}
Faculty of Sciences \\
%Department of Physics and Astronomy 
\end{center}
\end{minipage}
%second column
\begin{minipage}[t]{0.5\textwidth}  
\begin{center}
Science and Technology\\ Center \\
%Department of Physics 
\end{center}
\end{minipage} 

\vspace{0.5cm}

\begin{minipage}[t]{0.45\textwidth}
%This is the first column.\\
%This is still in the first column. 
\begin{center}
%Faculty of Sciences \\
Department of Physics \\and Astronomy 
\end{center}
\end{minipage}
%second column
\begin{minipage}[t]{0.5\textwidth}  
\begin{center}
%Faculty of Science and Bio-Engineering Sciences \\
Armando Dias Tavares \\Physics Institute
\end{center}
\end{minipage}





\begin{center}


\vspace{1cm}

August, 2016 \\

\vspace*{2cm}

{\Huge\bf Aspects of the Gribov problem in Euclidean Yang-Mills theories} \\

\vspace*{1cm}

{\LARGE Igor F. \textsc{Justo}} % vul hier je naam in \LARGE

\vspace*{2cm}

Promotors at Ghent University: Prof.\ Dr.\ David Dudal and Prof.\ Dr.\ Henri Verschelde \\ 
Promotors at Rio de Janeiro State University: Prof.\ Dr.\ Silvio Paolo Sorella and \\ Prof.\ Dr.\ Marcio Andr\'e L\'opes Capri

\vspace*{2cm}

Thesis submitted in fulfillment of the requirements for the degree of  \\ 
\textsc{Doctor in Sciences: Physics} \\ 
at Ghent University, and  \\
\textsc{Doctor in Sciences} \\ 
at Rio de Janeiro State University

\end{center}

\end{titlepage}











%------------------------------------------------------------------------------------------------------------
%% Titlepage
%\thispagestyle{empty}
%
%
%\begin{center}
%\begin{minipage}{0.75\linewidth}
%    \centering
%% University logo
%%    \includegraphics[width=0.3\linewidth]{logo.pdf}
%%    \rule{0.4\linewidth}{0.15\linewidth}\par
%
%
%
%    \vspace{3cm}
%% Thesis title
%    {\uppercase{\Large Non-perturbative aspects of Yang-Mills theories\\ }}
%    \vspace{3cm}
%%Author's name
%    {\Large Igor F. Justo\par}
%    \vspace{3cm}
%%Degree
%    {\Large Thesis submitted in fulfillment of the requirements for the degree of\par}
%    {\Large \bf Doctor of Science: Physics\par}
%    {\Large at Ghent University and\par}
%    {\Large \bf Doctor in Science\par}
%    {\Large at State University of Rio de Janeiro\par}
%
%    \vspace{3cm}
%%Date
%    {\Large May 2016}
%\end{minipage}
%\end{center}
%\clearpage
