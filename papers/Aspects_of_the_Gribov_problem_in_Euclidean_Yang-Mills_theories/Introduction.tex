
%%%%%%%%%%%%%%%%%%%%%%%%%%%%%%%%%%%%%%%%%%%%%%%%%
\chapter{Introduction}


Up to the present day confinement still is one of the most intriguing features of strong
interactions: why do quarks and gluons, being the fundamental excitations of fields of the
theoretical model that describes the strong interactions, widely known as QCD (Quantum
Chromodynamics), not appear in the spectrum of asymptotic physical particles? Or else, the
mechanism by which confinement happens in nature is another not answered question concerning strong coupling effects.

The strong interaction is one of the four fundamental interactions of Nature, next to the
gravitational, electromagnetic and weak interaction. Together, these four interactions reside
at the heart of the Standard Model (SM), which ought to theoretically describe all known
elementary particle physics processes, but neglecting quantized gravitational effects. For a
recent pedagogical review about the Standard Model (also referred to as a `Standard
Theory') take a look at \cite{'tHooft:2007zza}.

Despite the existence of some open questions surrounding the Standard Model (SM), such as too
many free parameters to fix, the not yet explained dark matter and
the quantization of gravity, its success in describing and foreseeing innumerable physical
process and particles, in scales of 10$^{-15}$cm and smaller than that, makes the SM the most
accepted theory to describe the physics of the fundamental particles, up to the present day \cite{Agashe:2014kda}. 

In particular, the present thesis is devoted to the study of effects of strong interactions,
more precisely those related to the interaction of fundamental colored particles (or color
sources), by making use of the QCD framework. Quantum chromodynamics (QCD) is a theoretical
model based on the theory of quantized relativistic fields, more widely known as Quantum Field
Theory, where particles are described by scalar fields, such as the Higgs one, and by
fermionic fields, accounting for the quarks, while the interaction between those particles
is mediated by gauge particles, which are described by vector fields
belonging to the adjoint representation of non-Abelian gauge groups such as the $SU(3)$.

Physically we may cite two typical characteristic features of strong interactions: confinement
and chiral symmetry breaking. Let us give a close look into these features.



\section{The confinement problem}


The modern understanding of confinement, in its physical sense, developed historically from a
more strict view of \emph{quark confinement} to a more general sense of \emph{color
confinement}. By color we mean a charge carried by particles described by QCD, which has
nothing to do with visual color, due to global gauge symmetry. The point is that quarks cannot
be found as free particles, but only \emph{confined} in hadrons (= composite state of quarks).
At the same time, gluons, or gauge particles, which are responsible for the mediation of
strong interactions, can also not be found as free particles in the physical spectrum of the
theory but, instead of that, only trapped inside \emph{glueballs} (= composite states of
gluons)\footnote{Up to the day of closing this thesis, there is no a definite particle to be
called {\it glueball}. However, the authors of \cite{Brunner:2015yha} claim that the resonance
``$f0(1710)$'' is the prefered candidate for a glueball. Further experimental results are
still expected to confirm (or not) the ``$f0(1710)$'' as the glueball.}. A hybrid composition of quarks and gluons is also possible, leading to the state of
quark-gluon plasma. Then, the modern concept of \emph{confinement} arises as only particles, or
composition of particles, carrying neutral color charge can be seen as asymptotic free
particles. Examples of hadrons are easy to find out, such as protons and neutrons that belong
to the set of composite states made up of three quarks called baryons.


Describing confinement is one of the tricky points of QCD. The fundamental particles of nature
are described in this framework as small excitations of quantum fields around the vacuum, with
these fields obeying specific rules of (local) gauge transformation of non-Abelian groups, such
as the $SU(3)$. Such gauge transformation leaves invariant the QCD action, so that the theory
is said to be gauge invariant.

Theoretically, it is widely believed that the phase transition ``confinement/deconfinement'' is
intimately related to the spontaneous breaking of a symmetry, or else, due to the existence (or
not) of a remnant symmetry. A {\it spontaneously broken symmetry} is meant to be a symmetry of
the system that does not leave invariant a physical state of the theory, such as the vacuum
state. In other words, suppose that $s$ stands for the (infinitesimal) transformation of a
certain symmetry of the theory ({\it i.e.} $sS=0$, where $S$ is the classical action of the
theory, but also $s\Gamma[\varphi]=0$, where $\Gamma[\varphi]$ is the quantum action of the
theory, as a function of the fields); and also suppose that $\varphi_{0}$ stands for the vacuum
configuration of that system ({\it i.e.} $\Gamma[\varphi_{0}]$ assumes its minimal value).
Therefore, the $s$ symmetry is said to be {\it spontaneously broken} if $s\varphi_{0} \neq
\varphi_{0}$. That is, in this case $\Gamma[\varphi_{0}] \neq \Gamma[s\varphi_{0}]$: the vacuum
is said to be degenerated in such cases, since there are two distinct vacuum configurations,
$\varphi_{0}$ and $s\varphi_{0}$ \cite{Weinberg:1996kr,Peskin:1995ev,Ryder:1985wq}.

A famous and simple example of phase transition due to spontaneous symmetry breaking can be
found in the Linear Sigma model, where a continuous symmetry group $O(N)$ is broken down to
$O(N-1)$; a more complex example is the Yang-Mills theory coupled to the Higgs field, known as
the Higgs mechanism, where the framework of spontaneous symmetry breaking is applied to the
theory of gauge fields. Times before the proposal of the Higgs mechanism, a similar procedure
had been applied by Ginzburg and Landau to the study of superconductors, although being a
classic (or statistical) model, where they plugged an external magnetic field into the model so
that the electromagnetic field could penetrate into the material only down to $m_{A}^{-1}$
depth; $m_{A}$ is the acquired mass by the electromagnetic field due to the spontaneous
symmetry breaking \cite{Weinberg:1996kr,Peskin:1995ev,Ryder:1985wq,Greensite:2011zz}. 

Two important theorems concerning symmetry breaking should be discussed in order to
better understand the link between the phase transition ``confinement/deconfinement'' and the
spontaneous symmetry breaking: one of them is the Goldstone theorem, while the other one is
the Elitzur's theorem.

\begin{itemize}
\item The Goldstone theorem:

\emph{There exists a spinless massless particle for every spontaneously broken continuous
symmetry,} \cite{Weinberg:1996kr,Peskin:1995ev,Ryder:1985wq,Greensite:2011zz}.

That is, in the case of the $SO(N)$ global transformations, for instance, there are $N(N-1)/2$
independent symmetries. It means that, in a theory with $\varphi^{i}$ real scalar fields, with
$i=1,2,\cdots,N$ and obeying an specific rule of transformation of the $SO(N)$ group, leaving invariant the theory, there are $N(N-1)/2$ independent transformations. The number of massless particles is,
\begin{eqnarray}
\frac{N(N-1)}{2} - \frac{(N-1)(N-2)}{2} ~=~ \frac{2N -2}{2} ~=~ N-1\,,
\end{eqnarray}
which is exactly the number of broken symmetries; $N(N - 1)/2$ comes form the original $SO(N)$
group symmetry, whilst $(N -1)(N-2)/2$ comes form the remaining $SO(N-1)$ group of symmetry
after the breaking.

Still in the instance of real scalar fields, let us consider a particular configuration of its
potential energy, which is given by
\begin{eqnarray}
V[\varphi^{i}] ~=~ - \frac12 \mu^{2} \pmb{\varphi}\cdot\pmb{\varphi} +
\frac{\lambda}{4}\left[ \pmb{\varphi}\cdot\pmb{\varphi} \right]^{2}\,,
\label{mexianhat}
\end{eqnarray}
called the \emph{Mexican hat} potential. In this case the vacuum expectation value (\emph{vev})
of the scalar field is not zero, but rather is degenerated $\langle  \varphi^{i} \rangle
=\nu\delta^{i0}$ around one chosen direction.
Supposing that this theoretical model has a \emph{global} $SO(N)$ symmetry, one can easily see
that, by performing a perturbation of the scalar field around its vacuum configuration, in the
chosen direction $\delta^{i0}$
\cite{Weinberg:1996kr,Peskin:1995ev,Ryder:1985wq,Greensite:2011zz},
\begin{eqnarray}
\pmb{\varphi}(x) ~=~ (\pi^{k}(x),\, \nu + \sigma(x))\,,
\end{eqnarray}
the potential becomes,
\begin{eqnarray}
V[\varphi^{i}] ~=~ - \frac12 (2\mu^{2}) {\sigma}^{2} - \frac{\lambda}{2}\pi^{2}\sigma^{2} -
\mu\sqrt{\lambda}\pi^{2}\sigma - \sqrt{\lambda}\mu \sigma^{3} - \frac{\lambda}{4}(\sigma^{2})^{2}
- \frac{\lambda}{4}(\pi^{2})^{2}
\,,
\end{eqnarray}
regarding that $\nu^{2} = \mu^2/\lambda$. Gathering the kinetic therm, the Lagrangian reads,
\begin{eqnarray}
{\cal L} ~=~  (\p_{\mu}\pi)^{2} + (\p_{\mu}\sigma)^{2} - V[\pi^{k},\,\sigma]\,.
\end{eqnarray}
So, it is not difficult to see that, at the end one gets one massive mode, the $\sigma$ one,
and $N-1$ massless and spinless modes, the $\pi^{k}(x)$ fields, corresponding to the foreseen
$N-1$ Goldstone bosons.


\item Elitizur's theorem:

\emph{It is not possible to spontaneously break a local (gauge) symmetry},
\cite{Elitzur:1975im}.

In other words, the vacuum expectation value (\emph{vev}) of a gauge non-invariant local
operator must vanish, and one should be careful when dealing with gauge theories coupled to
Higgs fields.

Let us provide again, but in other words, the concept of spontaneous symmetry breaking. A
symmetry is said to be \emph{spontaneously} broken in the sense that the vacuum configuration
is not symmetric under such (global) transformation. That is, if $\varphi_{0}$ is the vacuum
configuration of the scalar field and $\delta_{gl}\varphi_{0} \neq \varphi_{0}$, with
$\delta_{gl}$ standing for the variation of a global transformation, then we say that the global symmetry
$\delta_{gl} \Gamma[\varphi]=0$ is spontaneously broken, since $\delta_{gl}\Gamma[\varphi_{0}]
\neq 0$ (again $\Gamma[\varphi]$ stands for the quantum action of the theory)
\cite{Weinberg:1996kr,Peskin:1995ev,Ryder:1985wq}.

In the above example of the scalar field, where the theory is symmetric under global $SO(N)$
group transformations, the vacuum configuration before breaking the symmetry was the trivial
vacuum, $\varphi_{0} =0$. But after, when the potential acquires the \emph{Mexican hat}
potential form, the vacuum configuration is not symmetric anymore, but rather it is degenerated
out of $\varphi_{0} =0$.

Even though Elitzur's theorem forbids spontaneous symmetry breaking of local symmetries, it
seems that there exist in Nature exceptions to this rule: \emph{e.g.} the mechanism of mass
generation of the gauge field, such as in the Electroweak model, called {\it Higgs mechanism}.
Frequently we use to say that the \emph{gauge symmetry is spontaneously broken}, but what
exactly we mean by this expression? In order to answer, let us analyze one easy example, known
as the Georgi-Glashow model.

In the Georgi-Glashow model the Lagrangian reads,
\begin{eqnarray}
{\cal L} ~=~ -\frac{1}{4}F^{a}_{\mu\nu}F^{a}_{\mu\nu} + D_{\mu}\pmb{\varphi} D_{\mu}\pmb{\varphi} +
\frac{\lambda}{4}(\pmb{\varphi}\cdot\pmb{\varphi} - \nu^{2})^{2} \,,
\label{GGmodel}
\end{eqnarray}
where $D_{\mu} = \p_{\mu} -igA^{a}_{\mu}t^{a}$ stands for the covariant derivative, and
$F^{a}_{\mu\nu}$ stands for the field strength tensor:
\begin{eqnarray}
F^{a}_{\mu\nu} ~=~ \p_{\mu}A^{a}_{\nu} - \p_{\nu}A^{a}_{\mu} +
g^{2}f^{abc}A^{b}_{\mu}A^{c}_{\nu}\,.
\end{eqnarray}
Regarding
the fact that the Lagrangian \eqref{GGmodel} is invariant under the gauge transformation
$SU(N)$, let us {\it choose the direction of the broken symmetry} as being
\begin{eqnarray}
\langle \varphi^{i} \rangle = \nu\delta^{iN}\,,
\end{eqnarray}
with the scalar field in the fundamental representation. Expanding once again around the vacuum
configuration
\begin{eqnarray}
\pmb{\varphi} ~=~ (\tilde{\varphi}^{k},\,\nu + \sigma)
\label{unitgauge}
\end{eqnarray}
 and defining $\tilde{\varphi}$ rotated, so that it is perpendicular to $t^{a}\langle \varphi^{i} \rangle$,
\begin{eqnarray}
\tilde{\varphi}^{k}(t^{a})_{ki}\langle \varphi^{i}\rangle~=~ 0\,,
\label{unitycondit}
\end{eqnarray}
one ends up with a massive term for the gauge field of the form
$\frac{1}{2}A^{a}_{\mu}A^{b}_{\mu}\mu^{2}_{ab}$,
with \cite{Weinberg:1996kr,Peskin:1995ev,Ryder:1985wq}
\begin{eqnarray}
\mu^{2}_{ab} ~=~ (t^{a})_{iN}(t^{b})_{iN}\nu^{2}\,.
\label{gaugemass}
\end{eqnarray}

Let us now make some important remarks concerning this massive term. The first remark concerns
the importance of the condition \eqref{unitycondit}: this is the called \emph{unitary gauge
condition} and is nothing more than a rotation of the scalar field so as to end up with
$\tilde{\varphi}^{k}$ perpendicular to the broken directions $t^{a}\langle \varphi^{i}
\rangle$. The second remark concerns the \emph{spontaneous gauge symmetry breaking}: the
realization of the spontaneous symmetry breaking, in the sense that $\langle \varphi^{i}
\rangle \neq 0$, leads to the breaking of the (local) gauge symmetry, through the appearance of
the quadratic term $\frac{1}{2}A^{a}_{\mu}A^{b}_{\mu}\mu^{2}_{ab}$. However, it is clear that
this induced breaking in the gauge sector does not lead to the appearance of a vector Goldstone
boson. Instead of that, there exist $2N-1$ \emph{massive} vector bosons. Note that the massive
term \eqref{gaugemass} depends on the modes of the scalar fields associated to the broken
symmetries, $t^{a}\langle \varphi^{i} \rangle$, and since there exist  
\begin{eqnarray}
(N^{2}-1) - [(N-1)^{2} -1] ~=~ 2N -1 
\end{eqnarray} 
broken symmetries, then that is the number of massive vector bosons.

The Georgi-Glashow model is recovered for $N=2$ and, as mentioned before, with the Higgs field
in the fundamental representation the gauge group is said to be completely broken, yielding to
$3$ massive gauge bosons; another example is the electroweak gauge theory, where the
$SU(2)\times U(1)$ gauge symmetry is broken down to $U(1)$, providing mass to the $W^{\pm}$ and
$Z^{0}$ gauge bosons and to the matter field, leaving massless the photon and an (approximate)
massless pion \cite{Weinberg:1996kr,Peskin:1995ev,Ryder:1985wq,Greensite:2011zz}.

The point here is that, the breaking of a local gauge symmetry happens when the unitary gauge
is applied, \eqref{unitgauge} and \eqref{unitycondit}, in the \emph{Mexican hat} potential
configuration of the scalar field, which induces the breaking. However, in this local symmetry
breaking, there is no Goldstone boson associated. Otherwise, massive excitations of the vector
boson field appear, accounting for the ``missing'' Goldstone bosons degrees of freedom.
\end{itemize}


In order to probe for the existence of such global symmetry one should make use of gauge
invariant local operators that shall be sensible to the realization (or not) of that
symmetry. We call this operator an order parameter. Two well known order parameters are the
Wilson and the Polyakov loops. Both of them are related to the potential energy between two
color static sources. More precisely, the Wilson loop can be seen as a measure of the 
process of creation-interaction-annihilation of a pair of \emph{static} quark-antiquarks,
\cite{Wilson:1974sk}. Namely, its continuum expression is given by \cite{Greensite:2011zz}
\begin{eqnarray}
W ~=~ \left\langle P\,\exp\left[ig \oint_{C}\;dx^{\mu}A_{\mu}(x)\right]  \right\rangle ~\sim~
\e^{-V(R)T}\,,
\label{wilsonloop}
\end{eqnarray}
whence $T$ stands for the length in the time direction, and $V(R)$ is the quark-antiquark
potential, depending on their spacial distance $R$. In its discrete space-time version,
\emph{i.e.} on the lattice, the existence of such quark-antiquark creation (annihilation)
operator is evident, together with the creation operator of a gauge flux tube, mediating the
interaction between the quark-antiquark pair. We should emphasise the fact that the
\emph{vev} \eqref{wilsonloop} may be computed in a pure Yang-Mills theory, that is, in the
absence of any matter field, at all. Physically, this is a gauge theory in the presence of
heavy quark matter, which is theoretically achieved in the limit of infinite quark mass.

The Wilson loop is sensible to the existence of three possible phases, concerning the potential
between the static quarks \cite{Greensite:2011zz}:
\begin{itemize}
\item ``{\bf Yukawa, or massive} phase'', where the potential is given by
\begin{eqnarray}
V(R) ~=~ -g^{2}\frac{\e^{-mR}}{R} + 2V_{0}\,,
\label{perimeterlaw}
\end{eqnarray}
where $V_{0}$ stands for the self-energy of the system. The Wilson loop exhibit a
\emph{perimeter law} falloff, for a sequence of non self-intersecting loops,
\begin{eqnarray}
W  ~\sim~ \exp [-V_{0}P(\Gamma)]\,,
\end{eqnarray}
for situations where $R$ is large enough, in front of $1/m$; $P(C)$ is the perimeter of the
loop $\Gamma$ \footnote{At this point do not mistake $\Gamma$ for the quantum action.}.

\item ``{\bf Coulomb, or massless} phase'' phase, where the potential is proportional to the Coulomb
potential,
\begin{eqnarray}
V(R) ~=~ -\frac{g^{2}}{R} + 2V_{0}\,.
\end{eqnarray}
Also in this phase, if the self-energy contribution is considerably greater then the $1/R$ rule
potential, \emph{i.e.} $R\gg V^{-1}_{0}$, then the Wilson loop will also fall-off as 
\begin{eqnarray}
W  ~\sim~ \exp [-V_{0}P(\Gamma)]\,,
\end{eqnarray}
also when the loops do not intersect with themselves.

\item ``{\bf Disordered, or Magnetic disordered}'' phase, where the potential goes as
\begin{eqnarray}
V(R) ~=~ \sigma R + 2V_{0}\,.
\end{eqnarray}
In this case, the Wilson loop exhibit an \emph{area law} fall-off, for non self-intersecting
loops,
\begin{eqnarray}
W ~=~ \exp[ -\sigma RT - 2V_{0}T]\,.
\end{eqnarray}
Note that the potential energy grows as $\sim R$, that is, the bigger the distance between the
static quarks, the greater is the self-energy of the system. We use to attach a
\emph{confinement-like} interpretation to this scenario. On the other hand, in the first two
cases the potential behaves as $\sim V_{0}$, at the best, configuring a short-distance
interaction behaviour, and charged free particles can be found, such as the $W^{\pm}$ vector
bosons in the gauge + Higgs theory.

The expression ``magnetic disordered'', of the third regime of the Wilson loop, stems from the
fact that we are considering large enough loops $\Gamma$, so that its \emph{vev} equals the
product of vacuum expectation value of $n$ smaller loops ($\Gamma_{1}$, $\Gamma_{2}$, $\cdots$,
$\Gamma_{n}$) that lies inside the biggest one $\Gamma$ (\emph{cf.} \cite{Greensite:2011zz}).
Then we say that these $n$ loops are \emph{uncorrelated}, as much the same as the disordered
phase of the Ising model, whence the term was inspired.
\end{itemize}

The Polyakov loop \cite{Polyakov:1978vu} is, in its turn, sensible to a very important global
symmetry, intrinsic to gauge theories, called the \emph{center symmetry}. This symmetry group
is an intrinsic subgroup of a gauge group, let us say $G$, and is a set of elements that
commute with every elements of $G$. Said in other way, if $z_{n}$ belongs to the center
symmetry of a gauge group $G$, then $z_{n}$ commutes with every single element of $G$ --- so,
it is an element of the center of the group $G$. Therefore, considering the $SU(N)$ gauge
group, its associated center symmetry $Z_{N}$ is composed by elements of the following type,
\cite{Greensite:2011zz,Fukushima:2003fw,Fukushima:thesis}
\begin{eqnarray}
z_{n} ~=~ \exp\biggl( \frac{2i n\pi}{N} \biggr)\mathbb{1}\,,
\end{eqnarray}
whence $\mathbb{1}$ stands for the unity $N\times N$ matrix; $n = 0,\,1,\,2,\,\cdots,\,N-1$.
Since $Z_{N}$ is a subgroup of $SU(N)$, then it is straightforward to see that a pure gauge
theory is invariant under $Z_{N}$ transformations. However, that is not true in the Yang-Mills
+ matter field theory, with the matter field in any of its non-trivial group representation,
such as the fundamental one: the center symmetry is explicitly broken in this case. For the
matter field in the adjoint representation, which is an example of a trivial representation of
the gauge group, this center symmetry is still preserved, so that it may be spontaneously
broken afterwards (\emph{cf.} \cite{Greensite:2011zz} for further details) \footnote{Take
a look at the Chapter \ref{The Yang-Mills $+$ Higgs field theory} to see that in the theory of
Yang-Mills + Higgs field in its fundamental representation two distinct regimes, namely
the \emph{confinement-like} and the \emph{Higg-like}, coexiste in the same phase of the
theory, corresponding to the explicitly broken center symmetry (\emph{cf.} Fradkin \& Shenker
\cite{Fradkin:1978dv}).}.

The Polyakov loop may be defined in the continuum space-time as
\begin{eqnarray}
{\cal P} ~=~ \left\langle  P\,\exp\left[ ig\oint_{C}\, A_{0}dx^{0}  \right] \right\rangle ~\sim~
\e^{-FT}\,,
\label{polyakovloop}
\end{eqnarray}
and can be interpreted as an Euclidean space-time finite temperature torus circling around, and
accounts for the temporal component of the Wilson loop,
\cite{Fukushima:2003fw,Fukushima:thesis}. In equation \eqref{polyakovloop}, $F$ is the free
energy between the static quarks, and $T$ is the temperature.  As can be seen from
\eqref{polyakovloop}, the Polyakov loop, akin to the Wilson loop, is a measure of the
self-energy between static quarks. It is straightforward to see that if ${\cal P}\neq0$, then
the free energy between static quarks is finite, while that when ${\cal P}=0$, there is an
infinite free energy between them. So, one may classify as \emph{confined} and
\emph{deconfined} regimes situations of infinite binding energy between static quarks (${\cal
P}=0$) and situation of finite binding energy between static quarks (${\cal P}\neq0$),
respectively.


The sensibility of the Polyakov loop to the realization of the center symmetry can be seen from
its transformation under an element $z_{n}$ of $Z_{N}$, intrinsic to the gauge group $SU(N)$.
The {\it vev} of the Polyakov loop is understood to be computed in a pure Yang-Mills theory, or
coupled to an adjoint matter field. Namely, one has \cite{Fukushima:2003fw,Fukushima:thesis}
\begin{eqnarray}
{\cal P} ~\to~ z_{n} \left\langle  P\,\exp\left[ ig\oint_{C}\, A_{0}dx^{0}  \right]
\right\rangle ~=~ z_{n}{\cal P} \,.
\end{eqnarray}
Thus, the Polyakov loop is clearly not invariant under the center symmetry transformation, in
the specified situation. The Polyakov loop may, then, be seen as a suitable order parameter for
the center symmetry breaking.
Then, one has:
\begin{itemize}
\item[i.] The symmetric phase, where
\begin{eqnarray}
{\cal P} ~=~ 0\,,
\end{eqnarray}
which corresponds, as mentioned above, to the confined phase, of infinite energy between static
quarks;

\item[ii.] The broken phase, where
\begin{eqnarray}
{\cal P} ~\neq~ 0\,,
\end{eqnarray}
which corresponds to the deconfined phase, with finite energy between the static quarks
\footnote{The reader is pointed to
\cite{Fukushima:2002bk,Fukushima:thesis,Fukushima:2010bq,Bazavov:2009zn,Fukushima:2003fw,Greensite:2011zz}
for a detailed study on the Polyakov loop and the center symmetry}.
\end{itemize}

In both expressions \eqref{wilsonloop} and \eqref{polyakovloop} $P$ accounts for the \emph{path
ordering} of the (gauge field) operator $A_{\mu}$ as it appears in the closed path. At first
order in perturbation theory this path ordering operator is meaningless.

What happens if one has a gauge field theory coupled to a matter field in the fundamental
representation? In this case, the center symmetry is explicitly broken and no phase transition
occurs, \cite{Fradkin:1978dv,Greensite:2011zz}.

In the specific case of Yang-Mills theories coupled to scalar fields, in the adjoint
representation, whose potential energy is of the \emph{Mexican hat} type, the gauge symmetry is
said to be spontaneously broken after fixing the unitary gauge. However, it is not fully
broken, by leaving intact the global center symmetry (which does not happen in the fundamental
representation of the scalar field). Thus, phase transition may still be probed by means of the
Polyakov loop.


Another point of great interest is the order of the phase transition. In general, it can be
probed by measuring derivatives of the free energy with respect to thermodynamic order
parameters: divergences on the first derivative
would correspond to first order phase transitions; divergences on the second derivative
corresponds to a second order phase transition (we refer to \cite{LeBellac:1991cq} for a
pedagogical approach to this matter). Precisely, for pure gauge field theory it has been found
that for the $SU(2)$ gauge theory a second order phase transition takes place at a temperature
of $T_{c} = \unit{295}{\mega\electronvolt}$; and for $SU(N)$ gauge theories, with $N \ge 3$, a
first order phase transition is found to occur at $T_{c} = \unit{270}{\mega\electronvolt}$,
\cite{Fukushima:2002bk,Fukushima:thesis,Fukushima:2010bq,Bazavov:2009zn,Fukushima:2003fw,Banks:1979yr}.



%A remarkable work concerning the spectrum of phase transition in the Yang-Mills + Higgs
%fields is due to E. Fradkin \& S. H. Shenker, \cite{Fradkin:1978dv}. Roughly speaking, it
%becomes clear that the representation of the scalar Higgs field is of great importance for the
%analysis. In section \eqref{The Yang-Mills $+$ Higgs field theory}, subsection
%\ref{FSresults}, a brief discussion on the work of Fradkin \& Shenker will be made.



But, what about the phase transition in the presence of dynamical quarks? In these cases things
get overcomplicated since the usual order parameters, \eqref{wilsonloop} and
\eqref{polyakovloop}, cannot be used anymore. Physically the scenario is that a threshold value
for the dynamical quark separation is reached, so that beyond this value the potential between
them goes flat, instead of growing linear with the separation length $R$, as happened in the
static quark scenario. It indicates a dynamical screening mechanism for the gauge field
known as the \emph{string breaking} effect. A possible interpretation is that the potential
energy between the quarks grows (linearly) up to a level high enough to create a pair of
quark-antiquarks. Theoretically the following happens: the traditional order parameter
\eqref{polyakovloop} work by measuring the existence of the center
symmetry, $Z(N)$, which is associated to the gauge symmetry. However, the presence of
dynamical quarks explicitly breaks the center symmetry, preventing the Polyakov
loop from measuring any phase transition (this is similar to what happens in the
Georgi-Glashow model with the Higgs field in the fundamental representation). Furthermore,
with dynamical quarks the Wilson loop is not sensible to the disordered phase anymore, since
due to the string breaking dynamical effect, at some point ({\it i.e.} at some distance $R$ from
the quarks) the potential energy between the quarks will not grows linear with $R$ but rather
becomes flat.

Something similar happens in the Yang-Mills + Higgs theory: for the scalar field in its
fundamental representation the global (center) symmetry is explicitly broken, so that no
Goldstone boson is present and no phase transition takes place; for the scalar field
in the adjoint representation the global (center) symmetry is not explicitly broken, and then a
phase transition is allowed to occur and to be measured (further details on subsection
\ref{FSresults} and in \cite{Greensite:2011zz}).


The present thesis is mainly devoted to add a small piece to this big puzzle called confinement
by attacking the problem with an alternative approach. Instead of searching for an order
parameter and analytically probing it --- which is far from being an easy task
--- we do apply the framework of Gribov (and Gribov-Zwanziger) to investigate the existence of
gauge field confinement. As it is introduced on chapter \ref{usefulbkground},
confinement in Gribov's framework is not (clearly) related to the breaking of the global
symmetry, neither with the potential energy between quarks. The way to probe for confinement,
however, relies on the alternative criterion firstly proposed by Gribov, where the existence of
complex conjugate poles in the gauge field propagator must indicate the gauge confinement. The
point is that in such cases the gauge field is deprived of any physical particle
interpretation. Inspired by what happens in the gauge sector and expecting a confinement
behaviour also for the quark sector, we will propose in this thesis an effective action for the
matter field where the matter propagator exhibit a similar non-physical interpretation. That is
not the first time that (gauge) confinement and the Gribov issue are linked to each other.
However, that {\it is} the first time that the Gribov issue is investigated in the presence of
matter fields, or else, that the quark sector presents a Gribov-type structure.


\section{Chiral symmetry breaking}

As mentioned before, the modern concept of fundamental physics is based on the
existence/break\-ing of symmetries. We have been discussing that confinement can be understood
as the symmetric, or magnetic disordered, phase of a gauge theory. However, nothing have been
said about the existence of \emph{approximate symmetries} in nature. One famous approximate
symmetry is the $SU(2)$ isotopic symmetry, related to the mass of quarks $u$ and $d$: from the
most recent Particle Data Group's (PDG) data \cite{Agashe:2014kda}, at the mass scale of
$\mu \approx \unit{2}{\giga\electronvolt}$ and in a
mass-independent subtraction scheme called $\msbar$, quark-$u$'s mass is about $1.8$ --
$\unit{3.0}{\mega\electronvolt}$, while quark-$d$'s mass is $4.5$ --
$\unit{5.3}{\mega\electronvolt}$, so the rate between their mass is $m_{u}/m_{d} = 0.35$ --
$0.58$. 

Therefore, it is clear that both of the quarks have masses of the same order of magnitude, 
allowing us to formulate a(n) (approximate) symmetric theory, where quarks $u$ and
$d$ belong to the same doublet,
\begin{eqnarray}
\psi ~=~ \left(
\begin{array}{ll}
u \\
d
\end{array}
\right)\,,
\end{eqnarray}
and with the corresponding action symmetric under $SU(2)$ gauge transformations,
\begin{eqnarray}
\left(
\begin{array}{ll}
u \\
d
\end{array}
\right)
~\to~ 
\exp\left\{ i\vartheta^{a} t^{a} \right\} 
\left(
\begin{array}{ll}
u \\
d
\end{array}
\right)\,.
\label{chiraltransf}
\end{eqnarray}
In the above equation $\vartheta$ is a real parameter; $t^{a}$ accounts for the three $SU(2)$
generators, \emph{i.e.} the Pauli matrices. We use to say that quarks $u$ and $d$
(approximately) belong to the same \emph{isospin}. The same approximation is not so good
regarding the quark $s$; his mass is about $90$ -- $\unit{100}{\mega\electronvolt}$, so it is
an order of magnitude grater than quarks $u$ and $d$ masses and such an approximation leads to
inaccurate results. For a realistic model, where the mass of quarks $u$, $d$ and $s$ are
considered as being different from each other, we say that the isotopic symmetry is explicitly
broken \cite{Weinberg:1996kr,Peskin:1995ev,Ryder:1985wq}. 

Besides the isotopic symmetry there may also be another approximate symmetry, if one notice that
the quarks $u$ and $d$ masses are relatively small enough to put it to zero, regarding that the
energy scale is $\mu \approx \unit{2}{\giga\electronvolt}$. One should also recall that the mass
of protons (composed of three quarks), $\sim \unit{938}{\mega\electronvolt}$, is much higher
than the mass of its constituents, as mentioned before (check for the most recent particle data
at \cite{Agashe:2014kda}). So, it is reasonable to make the quark massless approximation, and
one ends up with an $SU(2)\times SU(2)$ symmetric theory, known as the \emph{chiral symmetry},
$\chi$S. At this point, of massless quarks $u$ and $d$, the fermionic doublet $\psi$ must
transform as 
\begin{eqnarray}
\left(
\begin{array}{ll}
u \\
d
\end{array}
\right)
~\to~ 
\exp\left\{ i\vartheta^{a} t^{a} + i\gamma_{5}\theta^{a} t^{a} \right\} 
\left(
\begin{array}{ll}
u \\
d
\end{array}
\right)\,,
\end{eqnarray}
leaving invariant the action. In the equation above, \eqref{chiraltransf}, $\gamma_{5}$ is the
pseudo scalar Dirac matrix, defined in terms of the four Dirac matrices (take a look at Appendix
\ref{notations} for details),
\begin{eqnarray}
\gamma_{5} ~=~ i\gamma^{0}\gamma^{1}\gamma^{2}\gamma^{3} ~=~
\frac{i}{4!}\epsilon^{\mu\nu\rho\sigma}\gamma_{\mu}\gamma_{\nu}\gamma_{\rho}\gamma_{\sigma}\,,
\end{eqnarray}
and $\theta^{a}$ stands for a global real parameter related to the conserved axial-vector
current
\begin{eqnarray}
J_{\gamma_{5}}^{a\;\mu} ~=~ i \bar{\psi} \gamma^{\,\,\mu}\gamma_{5}t^{a}\psi\,.
\label{veccurrent}
\end{eqnarray}
Of course, the isotopic (approximate) symmetry does also have an associated conserved current,
which can be read as
\begin{eqnarray}
J^{a\; \mu} ~=~ i\bar{\psi}\gamma^{\,\,\mu}t^{a}\psi\,.
\end{eqnarray}

So, what we have seen is that for a massless $u$- and $d$-quark the theory enjoys an approximate
chiral symmetry, with conserved current given by \eqref{veccurrent} \footnote{For a detailed
analysis the reader is pointed to standard textbooks
\cite{Weinberg:1996kr,Peskin:1995ev,Ryder:1985wq} where this topic is fully covered.}. 


Despite the fact that $u$- and $d$-quarks appear as (almost) massless particles in the QCD
action, related to an (approximate) chiral symmetry, the composite states of quarks, such as
protons and neutrons, are not found as (almost) massless particles in Nature. Instead of
that they are considerably heavier ($m_{p} = 938.272046 \pm 2.1\times
\unit{10^{-5}}{\mega\electronvolt}$) in contrast to quarks,
\cite{Roberts:1994dr,Agashe:2014kda}. Thus we are forced to ask if the approximate chiral
symmetry is indeed a reasonable approximation. If it is so, the chiral symmetry
$SU(2)\times SU(2)$ must be spontaneously broken down to the isotopic symmetry $SU(2)$, by
means of a dynamical process of mass generation for quarks, and with the rising of
massless Goldstone (we point the reader to \cite{'tHooft:1979bh,Banks:1979yr}
for a historical reference on this subject). Indeed, the pion meson ($\pi$) seems to
(approximately) fulfill these requirements, displaying the smallest mass of the known
particles, thus being identified with an (approximate) Goldstone boson and, then, pointing to
the effective existence of an spontaneous chiral symmetry breaking (S$\chi$SB),
\cite{Weinberg:1996kr,Peskin:1995ev,Ryder:1985wq}. To prove that nature really undergoes an
spontaneous chiral symmetry breaking is not an easy task, and as such has not been done, yet.
Despite this difficulty, it became clear that we do not need to fully comprehend the whole
process by which the chiral symmetry is broken to $SU(2)$, but rather that interesting process
of nature can be analyzed by just considering the existence of an approximate symmetry that is
spontaneously broken down to $SU(2)$,
\cite{'tHooft:1979bh,Banks:1979yr,Alexandru:2012sd,Weinberg:1996kr,Peskin:1995ev,Ryder:1985wq}.

In order to probe the breaking/restoration of the chiral symmetry one should measure the
existence, or not, of a non-zero \emph{chiral condensate} $\langle \bar{\psi}\psi \rangle$,
which can be done analytically, up to a certain accuracy (or energy level) in perturbation
theory. As effective theories we may cite some well known models, such as the Nambu-Jona-Lasino
model \cite{Nambu:1961tp,Volkov:2005kw,Palhares:2012fv,Fukushima:2003fw}; the MIT bag model
\cite{Palhares:2012fv,Canfora:2013zna,Bellac:2011kqa,Chodos:1974pn,DeTar:1979vb}; and also a quite new proposal by D. Dudal,
\emph{et al.} \cite{Dudal:2013vha} of introducing into the quark sector a nonlocal structure
similar to the Gribov-Zwanziger \emph{horizon} of the gluon sector, leading to a
renormalizable, confining and broken chiral symmetric theory. General properties of introducing
such a nonlocal term in the matter sector will be discussed on chapter \ref{brstonmatter},
while issues concerning UV divergences of such effective model will be treated on chapter
\ref{UVpropsofconfiningprop}.









%
