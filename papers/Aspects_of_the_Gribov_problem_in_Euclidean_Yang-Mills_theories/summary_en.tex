%-----------------------------------
\chapter*{English Summary}
\addcontentsline{toc}{chapter}{English Summary}
%-----------------------------------


The content of the present thesis is based on the papers
\cite{Capri:2013oja,Capri:2013gha,Capri:2012ah,Capri:2012cr,Capri:2014jqa,Canfora:2015yia,Capri:2015mna}
and is devoted to the study of aspects of the Gribov problem in Euclidean Yang-Mills
theories coupled to matter fields. Here, we present some, mathematical and physical,
evidences that point to the existence of a possible interplay between the gauge sector
and the matter sector, in regimes of sufficiently low energy (known as the infrared regime). In
other words, we claim that an effect in the vector boson sector of Nature due to strong
interactions (\emph{cf.} the Standard Model), at low energies, may be reflected in
the matter sector, in the same regime. Specifically, we propose that the Gribov
horizon function of the gauge sector may be felt by the matter field, and that it would be
described by an effective non-local mass term attached to the matter field. Such a term seems
to be dynamically generated and accounts for non-perturbative aspects of the matter field.

To achieve our goal this study is divided in three parts: 
on the first part, comprehending chapter \ref{The Yang-Mills $+$ Higgs field
theory}, we present an analytical study of the Yang-Mills theory coupled to the scalar Higgs
field, in the framework of Gribov quantization scheme. In this framework the propagator of the
gauge field may be profoundly modified, depending on the values of the parameters of the
theory, presenting complex conjugate poles. Such scenario prevents us from
attaching any physical particle interpretation to gauge propagator, since it violates the
positivity principle of Osterwalder-Schrader. Our study, thus, concerns the analysis of
the gauge field propagator in the configuration space of parameters of the model. The scalar
field is accounted in its fundamental and adjoint representations, in a 3- and 4-dimensional
Euclidean space-time. At the end we try to make a parallel between our study, of gauge field
confinement, and the work by Fradkin-Shenker \cite{Fradkin:1978dv}, where the Wilson loop is
measured in a (mostly) equivalent scenario.


In the second part of this thesis we propose, and analyze, an effective model for the matter
sector that leads to a soft breaking of the BRST symmetry, by plugging in a non-local term to
this sector, equivalent to the Gribov-Zwanziger horizon of the gauge field.
We will show that such construction may be consistently implemented and that it leads to a
confinement interpretation of the matter field, according to the Gribov's conception of
confinement. By fitting our effective matter propagator, said to be of the Gribov-type, to the
most recent lattice data, we could verify that, indeed, the BRST symmetry is soft broken, by
measuring a local gauge invariant operator that is BRST exact. Furthermore, this new matter
field propagator is found to violate the positivity principle, according to the lattice
fit.  Besides, the UV safety of such effective model is studied: there we prove that such
confinement mechanism, which resembles Gribov's procedure, does not lead to new divergences
other than those from the original (non-effective) theory. This second part is comprehended in
chapters \ref{brstonmatter} and \ref{UVpropsofconfiningprop}; a proof of all order UV stability
is presented in the Appendex \ref{ARscalaraction}, and an example is developed in Appendex
\ref{algrenorm}.

\newpage
The third and last part of this thesis, concerning chapter \ref{Ploop1}, is devoted to the
analysis of the finite temperature theory of static quarks, within the Gribov-Zwanziger
framework. To probe for confinement phases transition in this model the Polyakov loop is
introduced by means of a background field framework, so that the quark phase transition can be
analysed by a single parameter of the theory. This gauge invariant order parameter, related to
the Polyakov loop, is accounted in different circumstances so that the interplay
between this one and the Gribov parameter could be probed. As an interesting outcome we could
see that the behavior of the Gribov parameter is clearly sensible to the quark phase
transition, while an unavoidable region of instability is present. A brief discussion/analysis
is made towards the refined-Gribov-Zwanziger approach to this model.

%, and the Polyakov loop is accounted within the Gribov-Zwanziger framework.
%chapter \ref{usefulbkground} contains some background knowledge that might be useful for some
%reader not used with the Gribov(-Zwanziger) framework procedure and also not with some
%important details concerning the phase spectrum of the Yang-Mills + Higgs theory. On chapter
%\ref{conclusion} we summarize the results and state some remarkable conclusions.

