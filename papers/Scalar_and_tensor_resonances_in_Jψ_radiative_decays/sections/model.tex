\section{Amplitude models}
\label{sec:model}
We describe here the sets of models used to fit the data. We consider several possible variations, in order to perform a thorough study of the systematic uncertainties of our results. In the analysis of $\eta^{(\prime)}\pi$ COMPASS data~\cite{JPAC:2018zyd}, we chose one model as the nominal one, and the differences with other models were quoted as systematic uncertainties. However, here the spread of the results is too large to permit this strategy, and we will simply list the results of each model without selecting a preferred one. 



We parametrize the partial wave amplitudes following the coupled-channel $N/D$
formalism~\cite{Chew:1960iv,Bjorken:1960zz,Aitchison:1972ay,Oller:1998hw},
\begin{equation}
\label{eq:amplitude}
 a^J_i(s) = E_\gamma\, p_i^{J} \, \sum_k n^J_k(s) \left[ {D^J(s)}^{-1} \right]_{ki}\,,
\end{equation}
with the index $i=\h\bar \h=\pi\pi$, $K\bar K$, and later $\rho\rho$; as customary, $s$ is the $\h\bar \h$ invariant mass squared, 
$p_i=\sqrt{s- 4m_{i}^2}/2$
is the breakup momentum in the $h\bar h$ rest frame. One power of photon energy
$E_\gamma=(m_{\jpsi}^2-s)/(2\sqrt{s})$ for E1 transitions is required by gauge invariance. The intensities are calculated as $I^J_i(s) = \mathcal{N} p_i \left|a^J_i(s)\right|^2$, with $\mathcal{N}$ a normalization factor. The $n^J_k(s)$ incorporate exchange forces ({\it cf.} the right diagram in 
  Fig.~\ref{fig:proces}) 
     in the production process
and are smooth functions of $s$ in the physical region. The matrix 
$D^J(s)$  represents the $\h \bar \h \to \h \bar \h$ final state
interactions, and contains cuts only on the real axis above
thresholds (right hand cuts), which are constrained by
unitarity. For the numerator $n^J_k(s)$, we use an effective polynomial expansion,
\begin{equation}
    n^J_k(s)=\sum_{n=0}^{n_\text{max}}{a^{J,k}_n  T_n\left[\omega(s)\right]},
    \label{eq:production}
\end{equation}

where $T_n$ are the Chebyshev polynomials of order $n$. For systematic studies, we consider three different choices of   $\omega(s)$,
\begin{subequations}
\begin{align}
\omega(s)_\text{pole}&=\frac{s}{s+s_0}\,, \label{eq:omegapole} \\
\omega(s)_\text{scaled}&=2\frac{s-s_\text{min}}{s_\text{max}-s_\text{min}}-1\,, \label{eq:omegascaled}\\
\omega(s)_\text{pole+scaled} &=2\frac{\omega(s)_\text{pole}-\omega(s_\text{min})_\text{pole}}{\omega(s_\text{min})_\text{pole}-\omega(s_\text{max})_\text{pole}}-1\,, \label{eq:omegapolescaled}
\end{align}
\end{subequations}
where $s_0 = 1\gev^2$ is an effective scale parameter that controls the position of the left-hand singularities in Eqs.~\eqref{eq:omegapole} and~\eqref{eq:omegapolescaled}, and reflects the short range nature of production. Instead, Eq.~\eqref{eq:omegascaled} has no singularity, which corresponds to neglecting completely the right diagram in Fig.~\ref{fig:proces}. 
Eqs.~\eqref{eq:omegascaled} and~\eqref{eq:omegapolescaled} exploit the orthogonality of Chebyshev polynomials in the $[-1,1]$ interval in order to reduce correlations, being $\left[s_\text{min},s_\text{max}\right] = \left[\left(1\gev\right)^2,\left(2.5\gev\right)^2\right]$ the fitting region.  

A customary parametrization of the denominator is given by~\cite{Aitchison:1972ay}
\begin{equation}
\label{eq:Dsol}
D^J_{ki}(s) =  \left[ {K^J(s)}^{-1}\right]_{ki} - \frac{s}{\pi}\int_{4m_{k}^2}^{\infty}ds'\frac{\rho N^J_{ki}(s') }{s'(s'-s - i\epsilon)}, 
\end{equation}
where 
\begin{subequations}
\begin{align}
\rho N^J_{ki}(s')_\text{nominal}  &= \delta_{ki} \,\frac{(2p_i)^{2J+1}}{\left(s'+s_L\right)^{2J+\alpha}}, \label{eq:rhoN}
\end{align}
that is an effective description of the left-hand singularities in scattering 
controlled by the $s_L$ parameter, which we vary between $0$ and $1\gev^2$. The parameter $\alpha$ controls the asymptotic behavior of the integrand. 
As an alternative model, we consider the projection of a cross-channel exchange of mass squared $s_L$,
\begin{equation}
\label{eq:QCM}
\rho N^J_{ki}(s')_\text{$Q$-model} = \delta_{ki}\, \frac{Q_J(z_{s'})}{2p_i^2},
\end{equation}
\end{subequations}
where $Q_J(z_{s'})$ is the second kind Legendre function, and $z_{s'}=1+s_L/2p_i^2$. 
This function behaves asymptotically as $\log(s')/ s'$,  and has a left hand cut starting at $s'= 4m_i^2 - s_L$. For the $K$-matrix, we consider 
\begin{subequations}
\begin{align}
K^J_{ki}(s)_\text{nominal} &= \sum_R \frac{g^{J,R}_k g^{J,R}_i}{m_R^2 - s} + c^J_{ki} + d^J_{ki} \,s,\label{eq:Kmatrix}
\end{align}
with $c^J_{ki}= c^J_{ik}$ and $d^J_{ki}= d^J_{ik}$.
Alternatively, we parametrize the inverse of the $S$-wave $K$-matrix as a sum of CDD poles~\cite{Castillejo:1955ed,JPAC:2017dbi},
\begin{equation}
\left[K^J(s)^{-1}\right]_{ki}^\text{CDD} = c^J_{ki} - d^J_{ki} \,s - \sum_R \frac{g^{J,R}_k g^{J,R}_i}{m_R^2 - s} \, , \label{eq:CDD}
\end{equation}
\end{subequations}
where $c^J_{ki} = c^J_{ik}$ and $d^J_{ki} = d^J_{ik}$ are constrained to be positive.
For single channel, this choice ensures that no poles can appear on the first Riemann sheet. Even in the case of coupled channels, their occurrence is scarce, and when they do occur they are deep in the complex plane, far from the physical region. 
No CDD-like denominator will be used for the $D$-wave, as its structure looks much simpler. Ideally, the natural extension of the single channel CDD parametrization would be the inclusion of positive defined matrices for each  term in Eq.~\eqref{eq:CDD}, however this is expensive to compute from a numerical point of view, and not so simple to implement in our fits~\cite{Bedlinskiy:2014tvi}.
