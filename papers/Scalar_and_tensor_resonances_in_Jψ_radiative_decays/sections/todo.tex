First, the data will be fitted between 1 GeV and 2.5 GeV. The data below violates all we know about Watson's theorem, and it does not seem that a simple model assumption (like including the $\omega$) could alleviate this discrepancy. The data above 2.5 GeV does not show any interesting feature, however it shows some concerning behavior, particularly for the different magnetic moments in the $D$-wave. Of the 3 different multipoles measured for the $D$-waves we will fir $0++$ for simplicity. We consider fitting any of the other 2 would produce similar results, just a different numerator.

We always try to use the less possible number of poles to describe the data, we consider that a possible $f_0(1370)$ could be missing, and we will therefore include it later as a systematic check. But frankly speaking the data does not show a prominent peak or anything like that.

The fits have been performed in the following way:
\begin{enumerate}
    \item First we fit the $D$-waves intensities alone with a $K$-matrix, as these have a clear structure dictated by two well known resonances.
    \item This waves are then included together with the $S$-wave model we are using to fit the whole data with 2 coupled channels.
    \item We then extract a collection of the best fits from each trial and refit using these as starting points.
    \item These starting points will be used for a set of 3 channel fits including the $\rho \rho$ channel, where the $\rho$ will be considered elastic for simplicity. 
\end{enumerate}

The following list summarizes the main trials and different systematics we have already explored.

\begin{figure}
{\centering
\includegraphics[width=0.49\textwidth]{pipiintensity2.png} 
\includegraphics[width=0.49\textwidth]{kkintensity2.png}}
\caption{$\pi \pi$ (left) and $KK$ (right) $S$-wave (top), and $D$-wave (middle) interesting regions.}
\label{fig:regions}
\end{figure}

\begin{todolist}
  \item[\done] Try $K$-matrix:
  The $K$-matrix fits are the original ones, they offer simplicity and they are easy to fit, we also have some physics interpretation of some of the parameters used there. The main problem is that with such a big number of parameters nothing prevents us from generating $1^\text{st}$ sheet poles, and indeed they appear for both $S$ and $D$-waves. We have finished adding several checks to prevent us from selecting bad fits here.
  \item[\done] Try different number of poles:
  We have tested different numbers of poles, including or removing the heaviest ones do not change the fit by much, as expected. The other ones ($f_0(1500),f_0(1710),f_2,f'_2$) have a very high significance. The heavier possible $f_0(2020)$ and $f_2(1950)$ seem to be poorly determined, as expected because their shapes are not very pronounced.
  \item[\done] Try backgorund poles:
  Another try was to use light or heavy $K$-matrix poles as background, in order to use less linear background terms and stop producing $1^\text{st}$ sheet poles. This trial was unsuccessful however, and the $1^\text{st}$ sheet poles were still there.
  \item[\done] Try $K^{-1}$-matrix:
  This consists on parametrizing the inverse of the $K$-matrix as a smooth polynomial (HadSpec preferred method). The results are incredibly nice, easy and stable, as stable as the huge collection of $1^\text{st}$ sheet poles we find.
  \item[\done] Try $CDD$:
  It is a slight modification of the previous trial, but constraining the inverse to the linear terms, and adding some bound for the values of the parameters. This results in a $CDD$ like coupled channel formalism, which solves the problem of the $1^\text{st}$ sheet poles (they still appear, but very far). However it seems that due to the lack of elasticity in the $S$-wave the $D$-wave has to help, and thus $1^\text{st}$ sheet poles appear there now. It could be solver by also using a $CDD$ for the $D$-wave, it is the only thing I haven't tried yet. \arkaitz{UPDATE: This is no longer a concern, we have many different CDD fits without $1^\text{st}$ sheet poles close by.}
  \item[\done] Try different $CM$:
  The main problem regarding our original CM term is that it is indeed oversubtrated, which produces less of a phase space variation, but also includes a higher polynomial term. My guess was that due to this higher order $1^\text{st}$ sheet and spurious poles would have an easier time appearing close to the real axis. It seems that is the case and a non oversubtracted CM helps preventing these poles from appearing close to the real axis, although it does not fully solve the problem. These two, together with a more sophisticated term based on the second kind legendre functions $Q_\ell$ are our main systematics here. These are the same as for the hybrid paper. A different $Q_\ell$ second kind Legendre function can be used to describe an alternative CM, but I haven't found any difference.
 \item[\done] The main final trial is to use CDD parametrizations also in the $D$-wave, although this is expected to be a minor change, I bet there are people who would complain of we do not do it.
  
  \item[\done] Prepare the local $\chi^2$: see Fig. \ref{fig:regions}. My idea would be to create two local $\chi^2$, one for the intensities and anther one for the phases, as the later have huge uncertainties and averaging both $\chi^2$ would be missleading.
  {\small
  \begin{enumerate}
  \item From 1.4 to 2.2 GeV for $\pi \pi$ $S$-wave
  \item From 1.15 to 1.35 GeV for $\pi \pi$ $D$-wave
    \item From 1.3 to 2.3 GeV for $K K$ $S$-wave
  \item From 1.15 to 1.6 GeV for $K K$ $D$-wave
  \end{enumerate}}
  \item[\done] Prepare rectangles in the complex
  plane, lets forget about the third channel first, so all these are on the 4th Riemmann sheet
  {\small  
    \begin{enumerate}
    \item No first sheet poles in a big window.
    \item $[M=1400-1550,\Gamma=80-200]$ MeV for the $f_0(1500)$
    \item $[M=1680-1850,\Gamma=85-250]$ MeV for the $f_0(1710)$
    \item $[M=1850-2050,\Gamma=250-600]$ MeV for the $f_0(2020)$
    \item $[M=2150-2350,\Gamma=150-550]$ MeV for a possible $f_0(2200)$??? NOT CONFIRMED
    \item $[M=1250-1290,\Gamma=120-220]$ MeV for the $f_2(1270)$
    \item $[M=1480-1560,\Gamma=60-160]$ MeV for the $f_2'(1525)$
    \item $[M=1700-1900,\Gamma=90-400]$ MeV for a possible $f_2(1810)$???  NOT CONFIRMED
    \item $[M=1880-2050,\Gamma=250-650]$ MeV for a possible $f_2(1950)$???    NOT CONFIRMED  
    \end{enumerate}}
    
    \item[\done] Try more with the $Q_\ell$ CM \ale{check the code, check why first sheet pole windows don't work}


\item What remains now is to select the very best fits for each systematic, and run a bootstrap prioritizing those with larger deviations from the preferred fit. Alessandro is finishing updating the plotter so I can do this easier and faster.

\item Finally we should pay some little attention at least at the second solution by BESIII on their extraction. This solution was originally discarded because as seen in the original paper the different multipoles in the $D$-waves produce different phases, which violates Watson's theorem. We are not doing anything with it right now, but it should not very complicated to adapt some of our most flexible fits to this case and check the different pole positions on the $S$-wave mostly. 

  \end{todolist}