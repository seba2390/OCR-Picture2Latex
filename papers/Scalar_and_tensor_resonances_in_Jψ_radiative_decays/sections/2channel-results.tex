\section{2-channel Results}
\label{sec:2charesults}

\begin{figure}
\centering
\includegraphics[width=0.32\textwidth]{pipiSspread} \includegraphics[width=0.32\textwidth]{pipiDspread} \includegraphics[width=0.32\textwidth]{pipiPhspread} \\
\includegraphics[width=0.32\textwidth]{KKSspread} \includegraphics[width=0.32\textwidth]{KKDspread} \includegraphics[width=0.32\textwidth]{KKPhspread}
\caption{\label{fig:2channelfits} Best 2-channel fits to $\pi \pi$ (top) and $K \bar K$ (second row) final states. The intensities for the $S$- (left), $D$-wave (center), and their relative phase (right) are shown. The green and red lines denote the fit results. We remark that the model variations in the 2-channel fits are much larger than the statistical uncertainties, in particular for the phases. All these fits produce $\chi^2/\text{dof}\sim 1.7$--$2$. 
}
\end{figure}


We first explore the 2-channel fits. In total, considering the various possibilities discussed in Section~\ref{sec:model}, we could fit 27 different amplitude choices, without considering further variations of the fixed parameters (\eg the position of the left hand cut, the number of \KCDD, the order of the  polynomial in the numerator $n^J(s)$, \dots), which would amount to thousands of different possibilities. 
For phase space functions, we consider both Eq.~\eqref{eq:QCM} and Eq.~\eqref{eq:rhoN} for $\alpha\leq 1$. 
The choice of $\alpha$ is motivated by the asymptotic behavior of the phase space: if $\alpha> 1$, the integral is oversubtracted, making other subtractions redundant. Even if those fits produce similar results and fit quality, they tend to produce narrow unphysical $1^\text{st}$ sheet poles. Thus we restrict $\alpha=0$ for the final best fits. 
For the denominator, we vary the order of the background terms. We also tried to increase the \KCDD from the nominal 3 to 5 to see if extra resonances are produced. These fits do not produce noticeable differences and the additional poles produced are unstable and far from the fitted region, effectively merging with the background. 
Finally, we consider the different numerator variables listed in Eq.~\eqref{eq:production}. We also vary the order of the production polynomial between $2^\text{nd}$ and $3^\text{rd}$ order. Lower orders are not able to reproduce the data, in particular the relative phases would be heavily affected.

We select 14 models that do not produce noticeable unphysical behaviors, as  $1^\text{st}$-sheet poles narrower than $1\gev$. 
Summarizing, we have 3--4 parameters per wave per channel for the numerator polynomial, 2 couplings and a mass for the six bare resonances, 3--6 per wave for the background polynomial in the denominator.
Depending on the specific choices, they amount to 
40--44 parameters fitted to data, by performing a $\chi^2$ minimization with {\tt MINUIT}~\cite{minuit}.\footnote{This requires systematic uncertainties and correlations between partial waves to be negligible, as found in~\cite{Schluter:2012mep}. Correlations can actually be relevant, in particular in the high energy region, as shown in~\cite{Bibrzycki:2021rwh}.}
For each model, the fits are initialized by randomly choosing $O(10^5)$ different sets of values for the parameters. The best fits that do not produce any unphysical behavior have $\chi^2/\text{dof} \sim 1.7$--$2$. 
We show in Fig.~\ref{fig:2channelfits} the various 2-channel fits selected as best choices. Notice that none of them can reproduce the bump at $\sim 2.4\gev$ in the $S$-wave. Moreover, some of the dips between the peaks in the intensities are poorly described, with some local  $\chi^2/\text{bins}\gtrsim 4$.
As we anticipated in the \nameref{sec:intro}, some of the resonances in the fitted region have sizeable coupling to a $4\pi$ channel, which is not included in the two channel fits. Absence of a channel may be responsible 
 of producing tension between model and the data.  In particular, most of the 2-channel fits fail to describe the $f_2'(1525)$ lineshape properly, and our assumption that this state is saturated by $\pi\pi$ and $K\bar K$ only seems far too rigid, in particular considering that its coupling to $\pi\pi$ is negligible. This is the main reason why we expect the opening a third channel to improve the description of data. .

In these exploratory 2-channel studies no detailed statistical analysis is  performed. Nevertheless, we discuss the results on the pole positions. 
We show in Fig.~\ref{fig:poles2channel} the poles that appear on the Riemann sheets closest to the physical axis. Firstly, it is worth noting that not all fits produce the same number of resonant poles. Secondly, some of the fits produce additional ``spurious'' poles nearby, unstable upon variations of the model.  As mentioned in the \nameref{sec:intro}, the PDG lists five   $S$- and seven  $D$-wave resonances in this energy region, respectively ({\it cf.} Table~\ref{tab:pdgpoles}).
Grouping in clusters the poles obtained from the fits of the various models  that can be identified with physical resonances is not a simple task, especially for the heavier broad resonances that have large  uncertainties. 
Out of the 12 PDG resonances, we can identify only 6. As said, increasing the number of \KCDD does not help. 
The four lower mass clusters do not spread much and can be easily recognized. We note that the $f_0(1500)$ is clearly lighter than what is listed in the PDG average, whereas the $f_0(1710)$ is systematically heavier. Both $f_2(1270)$ and $f'_2(1525)$ seem to have masses close to those of the PDG, although the latter's width is not very well determined in  the fits.
Two heavier mass clusters seem to exist, each spreading over at least two states listed in the PDG. We identify them as the $f_0(2020)$ and the $f_2(1950)$. Some models produce a fourth narrow $S$-wave pole at around $\sim 2\gev$. One might wonder whether this cluster should be identified as the $f_0(2020)$, or as an additional state with almost the same mass. Most of the models produce the broader pole only. For those parametrizations that produce both, the narrower state has a much smaller total coupling, and decays preferably to the $K \bar K$ state. As we will see later, when including a third  channel this narrow pole disappears. 
Finally, we note that there is no pole that could be identified with the $f_0(1370)$, even when an {\em ad hoc} $K$-matrix$\big/$CDD pole is added. However, this is not unexpected, as the $f_0(1370)$ couples mostly to $4 \pi$. Phenomenologically, little mixing is expected between this resonance and the scalar glueball~\cite{Giacosa:2005zt,Giacosa:2005qr,Albaladejo:2008qa,Janowski:2014ppa}, which would additionally suppress its production in $\jpsi$ radiative decays. Its broad width would make its identification even more complicated. We conclude that, although we do not find evidence for this resonance in our analysis, its existence is not challenged.

The intervals of mass and width where the six resonances appear in the best models are shown in Fig.~\ref{fig:poles2channel} and summarized in Tab.~\ref{tab:poles2channel}. It is worth noting, as shown in Fig.~\ref{fig:poles2channel}, that the spreads for the heavier poles are compatible with several different resonances listed in the PDG. We remark again that we have intentionally conducted no statistical analysis here.
Summarizing, even though 2-channel fits describe data reasonably overall, they miss local features that affect the determination of some resonances.


\begin{table}[b]
\caption{Poles positions of the 2- and 3-channel fits. The intervals summarize the spread of results among the 15 best models. Statistical uncertainties are not taken into account. }
\begin{ruledtabular}
\begin{tabular}{l l  c  c  c c  c c c}
&  &$f_0(1500)$ & $f_0(1710)$ & $f_0(2020)$ & $f_0(2330)$ & $f_2(1270)$ & $f_2'(1525)$ & $f_2(1950)$ \\ \hline
\multirow{2}{*}{2-ch.} & Mass \mevp & $[1420,1456]$ & $[1739,1803]$ & $[1874,2098]$ & $-$ & $[1262,1282]$ & $[1471,1497]$ & $[1861,2139]$ \\ 
& Width \mevp & $[70,118]$ & $[109,215]$ & $[118,410]$ & $-$ & $[179,231]$ & $[51,103]$ & $[72,320]$ \\ \hline
\multirow{2}{*}{3-ch.} & Mass \mevp & $[1437,1471]$ & $[1756,1785]$ & $[1955,2098]$ & $[2313,2525]$ & $[1256,1279]$ & $[1488,1517]$ & $[1862,2084]$ \\ 
& Width \mevp & $[93,126]$ & $[139,171]$ & $[206,427]$ & $[180,411]$ & $[182,214]$ & $[68,101]$ & $[217,539]$ \\ \end{tabular}
\end{ruledtabular}
\label{tab:poles2channel}
\end{table}


\begin{figure}
{\centering
\includegraphics[width=0.49\textwidth]{polesS}  \includegraphics[width=0.49\textwidth]{polesD}} \\
{\centering
\includegraphics[width=0.49\textwidth]{polesS3c}  \includegraphics[width=0.49\textwidth]{polesD3c}}
\caption{\label{fig:poles2channel} Pole position of the various candidates for the 2- and 3-channel $\pi \pi, K \bar K$ fits, for the 14 systematics considered. We show in the left panels the $S$-wave with the $f_0(1500), f_0(1710)$, $f_0(2020)$ and a possible $f_0(2330)$ resonances. In the right panels the $D$-wave is shown, with the $f_2(1270), f_2'(1525)$ and a possible $f_2(1950)$. Identified poles are represented by colored markers, unidentified ones by gray ones. The colored rectangles represent the maximum spread in mass and width among the 14 models. For comparison, we show as gray rectangles the mass and widths (with uncertainties) of the 12 resonances listed in the PDG. We remark that the PDG lists mostly Breit-Wigner parameters, rather than pole positions in the complex plane. The 3-channel fits show a general improvement of the pole spreads. A new $f_0(2330)$ is found, while the $f_2(1950)$ is pushed deeper into the complex plane.
}
\end{figure}

