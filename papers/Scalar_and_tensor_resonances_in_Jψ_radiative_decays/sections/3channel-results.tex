\section{3-channel Results}
\label{sec:3charesults}
\begin{figure}
\centering
\includegraphics[width=0.32\textwidth]{pipiSspread3c} \includegraphics[width=0.32\textwidth]{pipiDspread3c} \includegraphics[width=0.32\textwidth]{pipiPhspread3c} \\
\includegraphics[width=0.32\textwidth]{KKSspread3c} \includegraphics[width=0.32\textwidth]{KKDspread3c} \includegraphics[width=0.32\textwidth]{KKPhspread3c} 
\caption{\label{fig:3channelfits} Best 3-channel fits to $\pi \pi$ (top) and $K \bar K$ (second row) final states. The intensities for the $S$- (left), $D$-wave (center), and their relative phase (right) are shown. The red lines denote the fit results. All these produce $\chi^2/\text{dof}\sim 1.1$--$1.2$.
 }
\end{figure}
We extend our model to include a third channel corresponding to an effective $4\pi$ final state. Since we are not sensitive to the details of the dynamics populating it, we approximate it as a stable $\rho\rho$ channel, with $m_\rho = 762\mev$~\cite{GarciaMartin:2011nna}.  Indeed, including the $\rho$ width does not improve the fit sizably, but makes the analytic continuation extremely complicated~\cite{JPAC:2018zwp}.\footnote{Alternatively, one could use approximate methods for analytic continuation, for example Pad\'e approximants, as in~\cite{Ropertz:2018stk}.} 
Restarting the fits from scratch with an additional unconstrained channel is unfeasible.
Instead we use the best 2-channel fits of the models of Section~\ref{sec:2charesults}, and use their parameters as starting point for the new 3-channel fits, to obtain more stable results. Since the 2-channel fits have reasonable quality already, we expect the contribution of the third channel to be small. To reduce the number of parameters, the numerator coefficients $a^{J,\rho\rho}_n$ are set to zero. Moreover, in the coefficients of the polynomial in $K^J(s)^{(-1)}$, 
we set the cross terms between the first two and the third channel to zero. The total number of parameters increases to 53--56, depending on the specific model.



\begin{figure}
\centering
\includegraphics[width=0.32\textwidth]{pipiS-inputcddcm5newc3_bootstrap-out} \includegraphics[width=0.32\textwidth]{pipiD-inputcddcm5newc3_bootstrap-out} \includegraphics[width=0.32\textwidth]{pipiPh-inputcddcm5newc3_bootstrap-out} \\
\includegraphics[width=0.32\textwidth]{KKS-inputcddcm5newc3_bootstrap-out} \includegraphics[width=0.32\textwidth]{KKD-inputcddcm5newc3_bootstrap-out} \includegraphics[width=0.32\textwidth]{KKPh-inputcddcm5newc3_bootstrap-out}
\caption{One of the final 3-channel fits, with statistical uncertainties included. The solid line and green band show the central value and the $1\sigma$ confidence level provided by the bootstrap analysis, calculated for $O(10^4)$ samples. }
\label{fig:bootstrapbestfit}
\end{figure}


The full list of plots and fit parameters for the 14 models is available in the \nameref{sup:supp-material}. 
In Fig.~\ref{fig:3channelfits} we show the results for the 14 best  3-channel models. These can be compared to the 2-channel fits in Fig.~\ref{fig:2channelfits}. It is evident that the fits improve: the average $\chi^2/\text{dof}$ drops from $\sim 1.7$--$2$ to $\sim 1.1$--$1.2$. 
More importantly, the local description of the relative phases and of the regions around the peaks are much more accurate. The effect can be seen in Fig.~\ref{fig:poles2channel},  where it is evident that poles are determined more precisely when the new channel is added. 


Most of the models lead to similar results, except for some deviation of the $K \bar K$ phase close to threshold. By construction our models respect Watson's theorem, which means that at the $K \bar K$ threshold  the $\pi \pi$ and the $K \bar K$ phases are identical. However, in some of our fits the $K \bar K$ phase moves rapidly just above threshold, because of peculiar cancellations between large numerators. Another interesting feature is the ``quasi-zero'' behavior on the $\pi \pi$ $D$-wave around 1.5\gev, which is evident in the intensity and seems to produce a sharp motion in the relative phase. A simple interpretation is that, if the $D$-waves are almost elastic, one expects a zero to appear between two resonances. If one assumes the coupling of the $f_2'(1525)$ to $\pi \pi$ to be almost zero, then this behavior could be explained by the interference between the $f_2(1270)$ and a heavier resonance coupling strongly to $\pi \pi$. This matches the behavior shown in the $D$-wave intensity, where the $f_2(1950)$ candidate produces a small peak. Moreover, the $K \bar K$ $D$-wave does not show any rapid motion, suggesting that, were a heavy resonance to exist, it would couple mostly to the other channels.

The structure of the $S$-waves is much richer. There are four clearly visible peaks in $\pi \pi$, and three  in $K \bar K$. It is worth noticing how different the values at the peak intensities look when comparing the same resonance in both final states. In particular, in $K \bar K$ the peak associated to the $f_0(1710)$ is roughly six times stronger than the $f_0(1500)$ one. We will show later that this is reflected in a much larger coupling of the $f_0(1710)$ to this channel. As can be seen in Fig.~\ref{fig:3channelfits}, 
our best fit reproduces all intensity peaks with high accuracy. There is a slightly larger local $\chi^2$ value around the $1.5\gev$ region in the $K \bar K$ $S$-wave. This region is below the $\rho\rho$ open channel, which seems to prevent our fit from fully reproducing the peak and interference. We are aware that the $f_0(1500)$ resonance couples to $4\pi$, but the local description is nevertheless reasonable. We thus conclude that a third channel is not strictly needed to describe such behavior. Ideally, the $\rho \rho$ channel should include  both $\ell =0$ and $\ell =2$  contributions. However, the latter is suppressed at threshold, and having no data to fit makes it impossible to distinguish the two. Nonetheless, we performed some alternative fits including just an $\ell =2$ channel to asses our systematics. We get $\chi^2/\text{dof}\sim 1.4$, not as good as in the $\ell =0$ case, being the channel suppressed as  mentioned. The pole positions calculated this way are compatible with the $\ell =0$ models (see below), and we do not see much variation in the $S$-wave, as expected. We do not consider these fits any further.
Even for these 3-channel fits, there is no evidence for more resonances than the 
seven ones  discussed above. Fits with additional \KCDD do not improve the data description, and the additional poles are far and unstable. 

The statistical uncertainties are determined via bootstrap~\cite{recipes,EfroTibs93,Landay:2016cjw}. We generate $O(10^4)$
pseudodatasets: each data point is resampled from a gaussian distribution having by mean and standard deviation its value and uncertainty; to avoid unphysical negative intensities, data points compatible with zero within $2\sigma$ are instead resampled from a Gamma distribution (see Appendix~\ref{app:gamma}). Each 
pseudodataset is refitted to the original model, and the (co)variance of the population of the fit parameters provides an estimate of their statistical uncertainties and correlations. In Fig.~\ref{fig:bootstrapbestfit} we show as an example the uncertainties for one of the models. 
