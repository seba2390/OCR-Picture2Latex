\section{Introduction}
\label{sec:intro}

\subsection{Motivation}

Reliable estimation of a signal (or image) from nonlinear observations is of fundamental interest to several signal processing and machine learning applications. However, such an estimation is confounded by cases where the nonlinearity in each observation is well-modeled by a \emph{periodic} function such as a sinusoidal function, or sawtooth function, or a square-wave function. Periodic functions are many-to-one mappings, and inverting them can be challenging.

Our focus in this paper is a special kind of periodic nonlinear observation model encountered in high-dynamic range (HDR) imaging. It is well known that real world scenes contain a large range of brightness levels. However, due to hardware limitations, not all brightness levels can be accurately captured using conventional photography; if tuned incorrectly, most scene intensity levels can lie in the saturation region of the image sensors, causing loss of scene information. Similar problems arise in the case of multiplexed imaging systems, such as lensless and coded aperture imaging~\cite{codedaperture,asif2017flatcam}.

%While increasing the dynamic range can solve this problem, it is an impractical strategy since imaging sensors need to have infinite dynamic range, which is infeasible. 
One solution is to increase the dynamic range of the image sensors, but this can lead to expensive hardware. An alternative solution is to deploy a special type of image sensor that {wraps} the observed intensity value at a pixel over a given dynamic range. This is analogous to the familiar \emph{modulo} operation with respect to a parameter $R$, and we call this stylized imaging system a \emph{modulo camera}~\cite{ICCP15_Zhao}.  
Fig.~\ref{fig:func}(a) (black) depicts the modulo nonlinearity, and a major challenge is to undo the effect of this transformation for each observed pixel.

An added challenge in HDR imaging arises due to \emph{quantization}. In fact, the ``true" observations in a modulo camera are quantized versions of the (idealized) modulo observation, and the errors caused in the quantization propagates into the estimation process. Loss of information in the quantization process is unavoidable in principle, and the effect of quantization is magnified with fewer quantization levels. In acquisition systems with low bit-depth, such estimation errors can be very pronounced. Fig.~\ref{fig:func}(a) (cyan) depicts the quantization nonlinearity incurred during the observation process.

\subsection{Setup}

We formalize the above discussion as follows. Assume $\mathcal{X} \subseteq \R^{n}$ to be a given (known) subset in the data space, and consider a signal (or image) $x \in \mathcal{X}$. We model (possible) multiplexing operations and gain adjustments as linear transformations, denoted by $A\in\mathbb{R}^{p\times n}$ and $C\in\R^{m\times p}$ respectively. The composite observation model becomes:
\begin{equation}
\label{quan_obs}
u=f(Ax),~y=Q(Cu),
\end{equation}
where $f(\cdot)=\mod(\cdot,R)$ denotes the modulo function with respect to a range parameter $R$ and $Q(\cdot)$ denotes a quantization function. In this paper, we consider a 1-bit quantization function with only two levels, $0$ and $1$. A representative example is shown in Fig.\ \ref{fig:func} where $A$ and $C$ are identity operators. In  Figs~\ref{fig:func}(c)  and~\ref{fig:func}(d), the outputs of the functions $f$ and $Q$ are displayed when a test grayscale image (Fig.\ \ref{fig:func}(a)) is used in the input. Our overall objective is to estimate the original signal $x$ from the set of measurements $y$. 

%%%%%%%%%%%%%%%%%%%%%%%%%%%%%%%%%%%%%
\begin{figure}[t]
	
	\begin{center}
		\begingroup
		\setlength{\tabcolsep}{0.1pt} % Default value: 6pt
		\renewcommand{\arraystretch}{.1} % Default value: 1
		\begin{tabular}{ccc}      %{c@{\hskip .1pt}c@{\hskip .1pt}c}
			\multicolumn{3}{c}{\begin{tikzpicture}
				\draw[<->,thick] (-3,0)--(3,0) node[anchor=north]{$t$};
				\draw (0,0) node[anchor=north]{$0$};
				\draw (0,1.1) node[anchor=west] {$R$};
				\draw (1,0) node[anchor=north]{$R$};
				\draw (2,0) node[anchor=north] {$2R$};
					\draw (-1,0) node[anchor=north]{$-R$};
				\draw (-2,0) node[anchor=north] {$-2R$};
				\draw[] (2,2) node[anchor=west] {{$Qof(t)$}};
				\draw[cyan,thick] (1.6,2) -- (2,2);
				%\draw [densely dotted,thick] (-2.5,1)--(3,1);
				\draw[->,thick] (0,0)--(0,2);
				\draw[] (2,1.5) node[anchor=west] {{$f(t)$}};
				\draw[thick] (1.6,1.5) -- (2,1.5);
				\draw[thick] (-2,0) --(-1,1)-| (-1,0) -- (0,1) -| (0,0) --(1,1)-| (1,0) -- (2,1) -| (2,0);
				\draw[densely dotted,thick] (2,0)--(2.5,0.5);
				\draw[densely dotted,thick] (-2,0)|-(-2,1) -- (-2.5,0.5);
				\draw[thick, cyan] (-2,0) -- ++(0.5,0)-| ++(0,0.5) -- ++(0.5,0) -| ++(0,-0.5) -- ++(0.5,0)-| ++(0,0.5) -- ++(0.5,0) -| ++(0,-0.5) -- ++(0.5,0)-| ++(0,0.5) -- ++(0.5,0) -| ++(0,-0.5) -- ++(0.5,0)-| ++(0,0.5) -- ++(0.5,0) -| ++(0,-0.5);
				\end{tikzpicture}}\\
			\multicolumn{3}{c}{(a)}\\
			\includegraphics[trim = 10mm 60mm 25mm 40mm,clip, width = 0.32\linewidth]{./orgimg.pdf}&
			\includegraphics[trim = 10mm 60mm 25mm 40mm,clip, width = 0.32\linewidth]{./modimg.pdf}&
			\includegraphics[trim = 10mm 60mm 25mm 40mm,clip,width = 0.32\linewidth]{./quantimg.pdf} \\
			(b) & (c) & (d)
		\end{tabular}
		\endgroup
	\end{center}
	\caption{\small{\emph{ (a) Modulo function, $f(t) = \mod(t,R)$ and quantized modulo function, $Qof(t)$; (b,c,d) Depiction of forward model. An input image (b) is transformed via a modulo function $f(t) = \mod(t,R)$, to (c). Such a ``modulo" image is further quantized to obtain (d).}}}
	\label{fig:func}
\end{figure}
%%%%%%%%%%%%%%%%%%%%%%%%%%%%%%%%%%%%%%%%%%%%%%%%%%

\subsection{Our contributions}

Clearly, the above estimation procedure is challenging due to the highly non-invertible nature of the observation model. In this paper, we design a systematic approach that takes some initial steps towards resolving this challenge. Our overarching assumption is that the measurement operations $A$ and $C$ are part of the design space. The core idea in our approach is that a very small, but carefully designed, non-adaptive set of measurements can support efficient estimation of the unknown signal.

Our approach follows stagewise. First, we consider the problem of inverting the quantization function, i.e., recovering $u$ from $y = Q(Cu)$. We demonstrate the existence of a linear operator $C$ (together with an efficient reconstruction algorithm) that supports such an inversion. Specifically, our operator $C$ obeys a particular block-diagonal form with weights chosen according to a harmonic progression; see Section~\ref{sec:Model} for details. We only consider 1-bit quantization functions, but similar ideas can presumably be extended for a higher number of quantization levels. In addition, our method supports the criterion of \emph{consistent reconstruction} as defined in \cite{jacques2011dequantizing}.

Next, we consider the problem of inverting the modulo operation, i.e., recovering $x$ from $u = f(Ax)$.  We propose an algorithm that builds upon the approach proposed in \cite{SoltaniHegde_ICASSP16}. In particular, we show that if the operator $A$ satisfies a certain \emph{factorization} $A = DB$, then $f$ can be stably inverted. To enable efficient inversion, the matrix $D$ must also be block-diagonal with weights chosen either randomly, or according to a geometric progression. In the former case, the reconstruction algorithm is an extension of the approach of~\cite{SoltaniHegde_ICASSP16}, while in the latter case the reconstruction follows the approach of~\cite{ICCP15_Zhao}.

The above two-stage procedure can be easily adapted to the case where we have some prior knowledge of the original signal $x$. This enables our approach to be used in conjunction with compressive imaging architectures. Common priors used in compressive imaging include \emph{sparsity} in some known orthonormal basis~\cite{foucart2013}. Note that our measurements are highly quantized and the total ``bit" complexity of our observations is far smaller than conventional techniques. Therefore, within our framework, one can choose to increase the number of quantizer measurements (rows of $C$) and/or modulo measurements (rows of $D$) in order to achieve better estimation performance.


Fig.~\ref{fig:demo} displays some representative results using our approach. We begin with a standard ``Peppers" image, compute a modulo transformation with three multiplexed measurements per pixel, and further modulate it with a sequence of three harmonic multipliers per measurement before passing it through a 1-bit quantizer. (In words, the overall ``oversampling factor" in our method is $9\times$.) The final binary measurements displayed in Fig.\ \ref{fig:demo}(a) are given as inputs to our reconstruction algorithm. The results from the first and second stages are displayed as images in Fig.\ \ref{fig:demo}(b). As is visually evident, our method is able to successfully reconstruct the image, as displayed in Fig.\ \ref{fig:demo}(c). 


\begin{figure}[t]
	\begin{center}
		\begingroup
		\setlength{\tabcolsep}{1pt} % Default value: 6pt
		\renewcommand{\arraystretch}{.1} % Default value: 1
		{\setlength{\tabcolsep}{1mm}
		\begin{tabular}{ccc|c|c}      %{c@{\hskip .1pt}c@{\hskip .1pt}c}
			\centering
			\includegraphics[trim = 30mm 60mm 40mm 65mm,clip, width = 0.15\linewidth]{./quant11.pdf}&
			\includegraphics[trim = 30mm 60mm 40mm 65mm,clip, width = 0.15\linewidth]{./quant12.pdf}&
			\includegraphics[trim = 30mm 60mm 40mm 65mm,clip, width = 0.15\linewidth]{./quant13.pdf}&
			\includegraphics[trim = 90mm 125mm 90mm 120mm,clip, width = 0.18\linewidth]{./mod11.pdf}&
				\multirow{3}{20mm}{\includegraphics[trim = 90mm 85mm 90mm 120mm,clip, width = \linewidth]{./dms_img.pdf}}\\
			\includegraphics[trim = 30mm 60mm 40mm 65mm,clip, width = 0.15\linewidth]{./quant21.pdf}& 
			\includegraphics[trim = 30mm 60mm 40mm 65mm,clip, width = 0.15\linewidth]{./quant22.pdf}&
			\includegraphics[trim = 30mm 60mm 40mm 65mm,clip, width = 0.15\linewidth]{./quant23.pdf}&
			\includegraphics[trim = 90mm 125mm 90mm 120mm,clip, width = 0.18\linewidth]{./mod21.pdf}&\\
			\includegraphics[trim = 30mm 50mm 40mm 65mm,clip, width = 0.15\linewidth]{./quant31.pdf}& 
			\includegraphics[trim = 30mm 50mm 40mm 65mm,clip, width = 0.15\linewidth]{./quant32.pdf}&
			\includegraphics[trim = 30mm 50mm 40mm 65mm,clip, width = 0.15\linewidth]{./quant33.pdf}& 
			\includegraphics[trim = 90mm 125mm 90mm 120mm,clip, width = 0.18\linewidth]{./mod31.pdf}&\\[1pt]
			\multicolumn{3}{c|}{(a)} &(b)&{\centering(c)}
		\end{tabular}}
		\endgroup
	\end{center}
	\caption{\small{\emph{Illustration of our approach. A given input image is modulated pixel-wise with three pre-chosen weights, passed through a modulo sensor, modulated again pixel-wise with three weights, and quantized to binary images. The resulting observations are shown in (a). The images in (b) and (c) represent the reconstruction of the modulo images, $\widehat{u}$ and the final image, $\widehat{x}$, respectively.}}}
	
	\label{fig:demo}
\end{figure}


