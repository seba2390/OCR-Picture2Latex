\section{Prior Work}
\label{sec:Prior}
\vspace{-0.5em} %EXTRASPACE
The recovery problem considered in this paper is a confluence of 3 sub-problems --- dequantization, modulo inversion, and sparse recovery. While each of them have been separately considered in considerable detail in signal processing, to the best of our knowledge our work is the first to combine all three sub-problems. The first and third subproblems fall under the purview of quantized compressive sensing and has been the topic of extensive study, dating back to the work of~\cite{DBLP:journals/corr/JacquesC16} and \cite{boufounos20081}. While \cite{DBLP:journals/corr/JacquesC16} uses additive, random dither to compensate the effect of quantization, our method proposes a different, multiplicative approach. Moreover, \cite{boufounos20081} introduces sparse recovery from 1-bit measurements, but strictly assumes that the signal is normalized to have unit Euclidean norm. In \cite{kamilov2012message}, a similar problem of recovering $x$ from $Q(Ax)$ has been studied, but contrary to our setup, it pre-supposes the components of $z=Ax$ to be correlated. 

The modulo inversion subproblem is also known in the literature as \emph{phase unwrapping}. The algorithm proposed in \cite{bioucas2007phase} is specialized to images, and employs graph cuts for phase unwrapping from a single modulo measurement per pixel. However, the inherent assumption there is that the input image has very few sharp discontinuities, and this makes it unsuitable for practical situations with textured images. Our main motivation for this paper is the work of \cite{ICCP15_Zhao} on HDR imaging using a modulo camera sensor. For image reconstruction using multiple measurements, it proposes the method called the multi-shot UHDR recovery algorithm; below, we show that this method can be effectively used in conjunction with quantizers as well as multiplexing mechanisms such as compressive imaging systems. 

Our approach can be viewed as an application of the ``decoupled" measurements idea described in \cite{SoltaniHegde_ICASSP16}. This earlier work did not take into account the effect of quantization in the reconstruction procedure. We resolve this issue using our proposed harmonic multipliers method that we describe in detail below.
