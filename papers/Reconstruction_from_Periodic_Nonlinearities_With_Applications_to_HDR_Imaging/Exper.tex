




\section{Experimental Results}
\label{sec:Results}
\begin{figure*}[t]
	\begin{center}
		\begingroup
		\setlength{\tabcolsep}{1pt} % Default value: 6pt
		\renewcommand{\arraystretch}{.1} % Default value: 1
		\begin{tabular}{ccc}      %{c@{\hskip .1pt}c@{\hskip .1pt}c}
			\includegraphics[trim = 30mm 80mm 35mm 80mm, clip, width=0.25\linewidth]{./hm_err.pdf}&
			\includegraphics[trim = 30mm 80mm 35mm 80mm, clip, width=0.25\linewidth]{./rqm_rqmms17.pdf}&
			\includegraphics[trim = 30mm 80mm 35mm 80mm, clip, width=0.25\linewidth]{./rqms_rqmmss17.pdf}
			\\
			(a) & (b) & (c)
		\end{tabular}
		\endgroup
	\end{center}
	\caption{\emph{ Normalized error vs number of quantized measurements $(k)$ for: (a) dequantization using HM algorithm; (b) reconstruction from quantized modulo measurements using RQM and RQM multi-shot; (c) reconstruction from quantized modulo measurements of sparse input using RQM and RQM multi-shot.}}
	\label{fig:results}
\end{figure*}


\begin{figure*}[t]
	\begin{center}
		\begingroup
		\setlength{\tabcolsep}{0.1pt} % Default value: 6pt
		%\renewcommand{\arraystretch}{-1} % Default value: 1
		\begin{tabular}{cccccc}      %{c@{\hskip .1pt}c@{\hskip .1pt}c}

				%\includegraphics[trim = 35mm 75mm 45mm 79mm, clip, width=0.14\linewidth]{../figs/orgimg.pdf}&
				\includegraphics[trim = 35mm 75mm 45mm 77mm, clip, width=0.12\linewidth]{./hm_k_5.pdf}& \hspace{15pt}
				\includegraphics[trim = 89mm 127.25mm 92mm 119.15mm, clip, width=0.12\linewidth]{./dmf_k_5.pdf}&
					\includegraphics[trim = 89mm 126.5mm 92mm 120mm, clip, width=0.12\linewidth]{./dms_k_5.pdf}& \hspace{15pt}
						\includegraphics[trim = 70mm 109mm 75mm 103.5mm, clip, width=0.12\linewidth]{./dmfs_imgorg.pdf}&
							\includegraphics[trim = 70mm 109mm 75mm 103mm, clip, width=0.12\linewidth]{./dmfs_k_10.pdf}&
								\includegraphics[trim = 70mm 110mm 75mm 100.5mm, clip, width=0.12\linewidth]{./dmss_k_10.pdf} \\
								
			
		
			 (a) & \multicolumn{2}{c}{\hspace{20pt}(b)} & \multicolumn{3}{c}{\hspace{22pt}(c)}
		\end{tabular}
		\endgroup
	\end{center}
	\caption{\emph{Image reconstruction results: (a) image reconstructed from quantized measurements using HM, $k=5$; (b) image reconstructed from quantized modulo measurements using RQM (left) and RQM multi-shot (right), $k=5$; (c) sparse input image (left), image reconstructed from quantized modulo measurements of sparse input using RQM-sparse (centre) and RQM multi-shot-sparse (right), $k=10$.}}
	\vspace{-0.4em}%EXTRASPACE
	\label{fig:imgresult}
\end{figure*}
	\vspace{-0.5em}%EXTRASPACE
In this section, we provide some representative numerical experiments for our proposed algorithm. 
First, we provide results describing our proposed dequantization procedure on a real test image. Further, we also provide results for the combined task of de-quantization and modulo recovery, with and without sparsity priors on the underlying signal. We employ two different algorithms for modulo recovery, and therefore we have following four combinations for our experiments: (i) reconstruction from de-quantized modulo observations using RQM, with and without sparsity priors, (ii) de-quantization using our HM algorithm followed by modulo recovery using the multi-shot UHDR recovery algorithm (we refer this whole procedure as \emph{RQM-multi-shot}), with and without sparsity priors.
	\vspace{-0.7em}%EXTRASPACE
\subsection{Dequantization}
	\vspace{-0.4em}%EXTRASPACE
In this experiment, we only focus on the first stage, i.e., we attempt to recover the image $w$ from set of $k$ quantized measurements $y^j$, $j=0,1,2,...,k-1$, using the \textsc{HMDequantization} method for different values of $k$. We record the normalized estimation error defined as $\frac{\|\widehat{w}-w\|_2}{\|w\|_2}$, with $\widehat{w}$ being the estimate of $w$. Here, $w$ is the grayscale form of an 8-bit, 3-channel RGB image of the size $512 \times 512$. The 1-bit quantizer described in Eq \eqref{eq:quantfunc} is used with $\Delta = 2^7$ to calculate $y$. Based on values of $y^0$, the coefficients $c_j$s are decided for each element of $w$. Subsequently, $(k-1)$ measurements are obtained according to Eq \eqref{eq:hm}, and the \textsc{HMDequantization} method is used to obtain $\widehat{w}$. The normalized estimation error is plotted against the number of measurements $k$ in Fig.~\ref{fig:results}(a). As we observe from the plot, our algorithm can recover $w$ within $10\%$ of error with as low as $5$ measurements. Increasing the value of $k$ improves the recovery performance rapidly in this regime, and less than $5\%$ error can be achieved with $k\geq9$. %sNumber of measurements can be easily decided by the percentage of accuracy required.



	\vspace{-0.7em}%EXTRASPACE
\subsection{Experiment: No sparsity priors}
	\vspace{-0.3em}%EXTRASPACE
We take an 8-bit, 3-channel RGB image of size $256 \times 256$, convert it to grayscale, and scale the dynamic range to [0,1]. Since there are no sparsity priors assumed here, we let $B$ be the identity matrix. We consider two cases for $D$. In the forward model specified by the RQM algorithm, the vectorized image $x$ is first multiplied by the block diagonal matrix $D_{mf}$. The size of $D_{mf}$ is set $(k'n) \times n$ as it contains $k'$ blocks of size $n \times n$ each. Diagonal of each block contains uniformly distributed random variables in the range $[-T,T]$.  Similarly, in the forward model specified by for RQM-multi-shot, $x$ is multiplied by the block diagonal matrix $D_{ms}$; here, all diagonal elements for $r^{th}$ block are same and equal to $2^{9-r}$; for $r = 1,2, \ldots, k'$ \cite{ICCP15_Zhao}. 
%The product $(DBx)$ is then passed through a modulo function to obtain $u$. We multiply $u$ with $C$, again a block diagonal matrix of size $k\m_1 \times m_1$, calculated as described in Section \ref{sec:Model}. $y$ is output measured with $Cu$ as the input of Quantization function $Q$. The final measurement matrix $y$ is of size $(kk'n) \times 1$.


To recover $\widehat{z} = \widehat{x}$, the estimation of $z =x$, from measurement $y$, we employ both the RQM as well as the RQM-multi-shot algorithms in two separate experiments.
In Fig.~\ref{fig:results}(b), we plot the normalized estimation error in recovered $x$ in case of RQM-multi-shot by varying $k$, while $k'$ is fixed to 3. As we can see, we are able to recover the original image within $5\%$ error only with $k=3$ quantized measurements. Fig.~\ref{fig:results}(b) also shows the variation of normalized estimation error for the RQM algorithm with $k'$ fixed to 4. To recover the original image within $5\%$ error, RQM requires $k=15$ quantized measurements. 

	\vspace{-0.7em}%EXTRASPACE
\subsection{Experiment: Sparsity priors}
	\vspace{-0.3em}%EXTRASPACE
We now evaluate the performance of the proposed method in scenarios where the input signal is $s$-sparse. We use the same $256 \times 256$ RGB image, convert to grayscale, and after obtaining its 2D Haar wavelet decomposition, retain the $s=1000$ largest coefficients to sparsify the image. We further multiply the sparse test image by a subsampled Fourier matrix with $q=8000$ multiplied with a diagonal matrix with random $\pm1$ entries to get $z=Bx$. The rest of the observation process is identical to the experiment described above.
%The only difference is the size of $z$ is reduced to $q \times 1$ from $n \times 1$ after introducing sparsity. Subsequently, the size of $D$ (either $D_{mf}$ or $D_{ms}$) reduced to $(k'q)\times q$. As a result, we get measurement matrix $y$ with size $(kk'1)\times 1$.

%The estimate of $u$, $\widehat{u}$ is computed from the measurement matrix $y$ using the \textsc{HMDequantization}. 
Again, two separate experiments are performed with using RQM in one and RQM-multi-shot algorithm in another to recover $\widehat{z}$ from $y$. The final step is to compute the estimate of high dimensional signal $\widehat{x} \in \mathbb{R}^n$ from $\widehat{z} \in \mathbb{R}^q$, which we achieve using the CoSamP algorithm~\cite{cosamp}.
%We use CoSaMP algorithm\cite{cosamp}. Other recovery algorithms like structured-sparse recovery can also be well used.
For the RQM-multi-shot-sparse algorithm, we fix $k' = 3$, and obtain the plot of relative error by varying the value of $k$, which is shown in Fig.~\ref{fig:results}(c). As we can see from the plot, we are able to recover the original image within $5\%$ error only with the use of $7$ quantized measurements. Similar to the experiment without sparsity, we fixed $k'=4$ for RQM-sparse algorithm, and measure the normalized estimation error in $\hat{x}$ for different values of $k$. Corresponding plot is in Fig.~\ref{fig:results}(c). To estimate the original image within $5\%$ relative error, $k=25$ quantized measurements are used.

Considering that the input is sparse and the measurements $y$ are binary, the storage requirements for $y$ are considerably smaller compared to the case without sparsity. In essence, leveraging the sparsity prior can reduce the sample complexity of the algorithm drastically. The tradeoff is to choose a higher value of $k$ or $k'$, which will affect the running time only marginally but improves recovery performance by a significant amount.



