% Template for SSP-2016 paper; to be used with:
%          spconf.sty  - ICASSP/ICIP LaTeX style file, and
%          IEEEbib.bst - IEEE bibliography style file.
% --------------------------------------------------------------------------
\documentclass{article}
\usepackage{spconf,amsmath,epsfig}
\usepackage{amssymb}
\usepackage{amsthm}
\usepackage{algorithm}
\usepackage{algpseudocode}
%\usepackage{fullpage}
\usepackage{tikz}
\usepackage{graphicx}
\usepackage{subfig}
\usepackage{epstopdf}
\usepackage{epsfig}
\usetikzlibrary{tikzmark,calc}
\usepackage{caption}
\usepackage{float}
\usepackage{multirow}
\usepackage{todonotes}
\usepackage{microtype}

\numberwithin{equation}{section} 
\newtheorem{theorem}{Theorem}[section]                   %[section]   %numberes automatically
\newtheorem{lemma}[theorem]{Lemma}         %[theorem]  in the middle
\newtheorem{corollary}[theorem]{Corollary}     %{corollary}{Corollary}[theorem]   
\newtheorem{proposition}[theorem]{Proposition}
\newtheorem{claim}[theorem]{Claim}
\newtheorem{ass}[theorem]{Assumption}
\newtheorem{fact}[theorem]{Fact}
\newtheorem{heuristic}[theorem]{Heuristic}
\newtheorem{definition}[theorem]{Definition}
\newtheorem{remark}[theorem]{Remark}
\newtheorem{exmp}[theorem]{Example} 
% Example definitions.
% --------------------
\def\x{{\mathbf x}}
\def\R{{\mathbb{R}}}
\def\L{{\cal L}}
\DeclareMathOperator*{\argmin}{arg\,min}
\DeclareMathOperator*{\argmax}{arg\,max}

\renewcommand{\mod}{\mathrm{mod}}

\newcommand{\red}[1]{{\textcolor{red}{#1}}}  %\textcolor{red}


% latex spacing issues
\setlength{\textfloatsep}{5pt plus 1.0pt minus 2.0pt}
\setlength{\floatsep}{5pt plus 1.0pt minus 2.0pt}
\setlength{\abovecaptionskip}{3pt plus 1pt minus 2pt}
\setlength{\abovedisplayskip}{3pt}
\setlength{\belowdisplayskip}{3pt}

%\usepackage{titlesec}


%\ninept
% Title.
% ------
\title{Reconstruction from Periodic Nonlinearities, \\ With Applications to HDR Imaging}
%
% Single address.
% ---------------
\name{Viraj Shah, Mohammadreza Soltani, Chinmay Hegde\thanks{This work was supported in part by grants from the National Science Foundation (NSF CCF-1566281) and NVIDIA.}
}
\address{ECpE Department, Iowa State University, Ames, IA, 50010}
 
%
% For example:
% ------------
%\address{School\\
%	Department\\
%	Address}
%
% Two addresses (uncomment and modify for two-address case).
% ----------------------------------------------------------
%\twoauthors
%  {A. Author-one, B. Author-two\sthanks{Thanks to XYZ agency for funding.}}
%	{School A-B\\
%	Department A-B\\
%	Address A-B}
%  {C. Author-three, D. Author-four\sthanks{The fourth author performed the work
%	while at ...}}
%	{School C-D\\
%	Department C-D\\
%	Address C-D}
%
\begin{document}
\ninept
%
\maketitle
%
\begin{abstract}
We consider the problem of reconstructing signals and images from periodic nonlinearities. For such problems, we design a measurement scheme that supports efficient reconstruction; moreover, our method can be adapted to extend to compressive sensing-based signal and image acquisition systems. Our techniques can be potentially useful for reducing the measurement complexity of high dynamic range (HDR) imaging systems, with little loss in reconstruction quality. Several numerical experiments on real data demonstrate the effectiveness of our approach.
\end{abstract}
%
%\begin{keywords}
%, sparse recovery, nonlinear measurements, fast algorithms.
%\end{keywords}

%%%%%%%%%%%%%%%%%%%%%%%%%%%%%%%%%%%%%%%%%%%%%%%%%%%%%%%%%%%%%%%%%%%%%%%%%%%%%%%%%%%%%%

\section{Introduction}
\label{sec: intro}

% Operating safely in dynamic environments is crucial for autonomous robots in real-world scenarios. Existing control barrier functions for obstacle avoidance often assume point or circular robots, limiting their applicability to robots with more complex geometries. In this paper, we address this limitation by presenting an analytic approach to compute the distance between a polygonal robot and moving elliptical obstacles in a 2D environment. This distance computation is utilized in constructing a control barrier function for safe control synthesis, enabling the operation of a robot with a more intricate shape. Our proposed approach offers real-time tight elliptical obstacle avoidance for polygon-shaped robots. 

Obstacle avoidance in static and dynamic environments is a central challenge for safe mobile robot autonomy. 

At the planning level, several motion planning algorithms have been developed to provide a feasible path that ensures obstacle avoidance, including prominent approaches like A$^*$~\cite{A_star_planning}, RRT$^*$~\cite{RRT_star}, and their variants~\cite{informed_rrt_star, neural_rrt_star}. These algorithms typically assume that a low-level tracking controller can execute the planned path. However, in dynamic environments where obstacles and conditions change rapidly, reliance on such a controller can be limiting. A significant contribution to the field was made by Khatib \cite{potential-field}, who introduced artificial potential fields to enable collision avoidance during not only the motion planning stage but also the real-time control of a mobile robot. Later, Rimon and Koditschek \cite{navigation-function} developed navigation functions, a particular form of artificial potential functions that guarantees simultaneous collision avoidance and stabilization to a goal configuration.
%. These functions strive to ensure collision avoidance and stabilization towards a goal configuration simultaneously. 
% Meanwhile, Fox \cite{Fox1997TheDW} introduced the dynamics window concept, an influential approach to obstacle avoidance that proactively filters out unsafe control actions. 
In recent years, research has delved into the domain of trajectory generation and optimization, with innovative algorithms proposed for quadrotor safe navigation \cite{mellinger_snap_2011, zhou2019robust, tordesillas2019faster}. In parallel, the rise of learning-based approaches \cite{michels2005high, pfeiffer2018reinforced, loquercio2021learning} has added a new direction to the field, utilizing machine learning to facilitate both planning and real-time obstacle avoidance. Despite their promise, these methods often face challenges in dynamic environments and in providing safety guarantees.


In the field of safe control synthesis, integrating control Lyapunov functions (CLFs) and control barrier functions (CBFs) into a quadratic program (QP) has proven to be a reliable and efficient strategy for formulating safe stabilizing controls across a wide array of robotic tasks \cite{glotfelter2017nonsmooth, grandia_2021_legged, wang2017_aerial}. While CBF-based methodologies have been deployed for obstacle avoidance \cite{srinivasan2020synthesis, Long_learningcbf_ral21, almubarak2022safety, dawson2022learning, abdi2023safe}, such strategies typically simplify the robot as a point or circle and assume static environments when constructing CBFs for control synthesis. Some recent advances have also explored the use of time-varying CBFs to facilitate safe control in dynamic environments \cite{he2021rule, molnar2022safety, hamdipoor2023safe}. However, this concept has yet to be thoroughly investigated in the context of obstacle avoidance for rigid-body robots. For the safe autonomy of robot arms, Koptev \textit{et al}. \cite{Koptev2023_neural_joint_control} introduced a neural network approach to approximate the signed distance function of a robot arm and use it for safe reactive control in dynamic environments. In \cite{Hamatani2020arm}, a CBF construction formula is proposed for a robot arm with a static and circular obstacle. A configuration-aware control approach for the robot arm was proposed in \cite{ding2022configurationaware} by integrating geometric restrictions with CBFs. Thirugnanam \textit{et al}. \cite{discrete_polytope_cbf} introduced a discrete CBF constraint between polytopes and further incorporated the constraint in a model predictive control to enable safe navigation. The authors also extended the formulation for continuous-time systems in \cite{polytopic_cbf} but the CBF computation between polytopes is numerical, requiring a duality-based formulation with non-smooth CBFs. 

\subsubsection*{Notations}

The sets of non-negative real and natural numbers are denoted $\bbR_{\geq 0}$ and $\bbN$. For $N \in \bbN$, $[N] := \{1,2, \dots N\}$. The orientation of a 2D body is denoted by $0 \leq \theta < 2\pi$ for counter-clockwise rotation. We denote the corresponding rotation matrix as 
% \begin{equation}
% \label{eq: rotation}
    $\bfR(\theta) = \begin{bmatrix} \cos \theta & -\sin \theta \\ \sin \theta & \cos \theta \end{bmatrix}.$
% \end{equation}
The configuration of a 2D rigid-body is described by position and orientation, and the space of the positions and orientations in 2D is called the special Euclidean group, denoted as $SE(2)$. Also, we use $\|\bfx\|$ to denote the $L_2$ norm for a vector $\bfx$ and $\otimes$ to denote the Kronecker product. The gradient of a differentiable function $V$ is denoted by $\nabla V$, and its Lie derivative along a vector field $f$ by $\calL_f V  = \nabla V \cdot f$. A continuous function $\alpha: [0,a)\rightarrow [0,\infty )$ is of class $\calK$ if it is strictly increasing and $\alpha(0) = 0$. A continuous function $\alpha:\mathbb{R} \rightarrow \mathbb{R}$ is of extended class $\calK_{\infty}$ if it is of class $\calK$ and $\lim_{r \rightarrow \infty} \alpha(r) = \infty$. Lastly, consider the body-fixed frame of the ellipse $\calE'$. The signed distance function (SDF) of the ellipse $\psi_\calE: \mathbb{R}^2 \to \mathbb{R}$ is defined as 
\begin{equation}
%\label{eq: SDF}
    \psi_\calE(\bfp') 
    = \left\{
    \begin{array}{ll}
        d(\calE',\bfp'), & \text{if } \bfp' \in \calE^c,  \\
        -d(\calE',\bfp'), & \text{if } \bfp' \in \calE,
    \end{array} 
    \right. \notag
\end{equation}
where $d$ is the Euclidean distance. In addition, $\|\nabla \psi_\calE (\bfp')\| = 1$ for all $\bfp'$ except on the boundary of the ellipse and its center of mass, the origin.


\textbf{Contributions}: (i) We present an analytic distance formula in $SE(2)$ for elliptical and polygonal objects, enabling closed-form calculations for distance and its gradient. (ii) We introduce a novel time-varying control barrier function, specifically for rigid-body robots described by one or multiple $SE(2)$ configurations. Its efficacy of ensuring safe autonomy is demonstrated in ground robot navigation and multi-link robot arm problems. 



% This file was created by matlab2tikz.
%
%The latest updates can be retrieved from
%  http://www.mathworks.com/matlabcentral/fileexchange/22022-matlab2tikz-matlab2tikz
%where you can also make suggestions and rate matlab2tikz.
%
\definecolor{mycolor1}{rgb}{0.00000,0.44706,0.74118}%
\definecolor{mycolor2}{rgb}{0.85098,0.32549,0.09804}%
%
\begin{tikzpicture}

\begin{axis}[%
width=0.4\columnwidth,
height = 0.25\columnwidth,
at={(0.758in,0.455in)},
scale only axis,
xmin=-5,
xmax=5,
xlabel style={font=\color{white!15!black}},
xlabel style={font=\footnotesize},
xlabel={Input ${\rho}_*$},
ymin=-1.75,
ymax=1.75,
ylabel style={font=\color{white!15!black}},
ylabel={Output $f({\rho}_*)$},
ylabel style={font=\footnotesize},
axis background/.style={fill=white}
]
\addplot [color=white!80!black, line width=1.3pt, forget plot,dotted]
  table[row sep=crcr]{%
-5	0.798503975621518\\
-4.99	0.809761238521288\\
-4.98	0.821160199113556\\
-4.97	0.832701149142376\\
-4.96	0.844378509385361\\
-4.95	0.856184683122152\\
-4.94	0.868116318918629\\
-4.93	0.880166245485379\\
-4.92	0.892329526045555\\
-4.91	0.904604704415608\\
-4.9	0.916978305943714\\
-4.89	0.929452254058218\\
-4.88	0.94201277745546\\
-4.87	0.954659899253512\\
-4.86	0.967384861508899\\
-4.85	0.980180460820653\\
-4.84	0.993044188103078\\
-4.83	1.00596428830677\\
-4.82	1.01894016150486\\
-4.81	1.03195851811094\\
-4.8	1.04501563599461\\
-4.79	1.05810767845245\\
-4.78	1.07122672210662\\
-4.77	1.08436391746965\\
-4.76	1.09751387845623\\
-4.75	1.11066936546925\\
-4.74	1.12382604723801\\
-4.73	1.13697478675607\\
-4.72	1.15010734036384\\
-4.71	1.16322056737889\\
-4.7	1.17630423410402\\
-4.69	1.18935486214533\\
-4.68	1.2023654443878\\
-4.67	1.21532754290429\\
-4.66	1.22823353173572\\
-4.65	1.24108285290641\\
-4.64	1.25386380342904\\
-4.63	1.26657221705217\\
-4.62	1.27920011011427\\
-4.61	1.29174406645145\\
-4.6	1.30419565011023\\
-4.59	1.31655026963864\\
-4.58	1.32880082085519\\
-4.57	1.34094226089516\\
-4.56	1.35296757589098\\
-4.55	1.36487470795715\\
-4.54	1.37665448728607\\
-4.53	1.38830323542926\\
-4.52	1.39981611636116\\
-4.51	1.41118852276395\\
-4.5	1.42241166250034\\
-4.49	1.43348686440862\\
-4.48	1.44440471950987\\
-4.47	1.45516050828466\\
-4.46	1.46575095392792\\
-4.45	1.47617714624414\\
-4.44	1.48642635489358\\
-4.43	1.49650096231939\\
-4.42	1.50639373000849\\
-4.41	1.51610127946909\\
-4.4	1.5256234655702\\
-4.39	1.53495392353486\\
-4.38	1.54408952839831\\
-4.37	1.5530298079127\\
-4.36	1.5617691244087\\
-4.35	1.57030713915622\\
-4.34	1.57863758542734\\
-4.33	1.58676520340661\\
-4.32	1.59467832728247\\
-4.31	1.60238357015206\\
-4.3	1.6098764180179\\
-4.29	1.61715249218038\\
-4.28	1.62421441887011\\
-4.27	1.63105858223566\\
-4.26	1.63768256459097\\
-4.25	1.64408973693603\\
-4.24	1.65027507120295\\
-4.23	1.65623953086216\\
-4.22	1.66198396205276\\
-4.21	1.66750462082435\\
-4.2	1.67280278442306\\
-4.19	1.67788150096994\\
-4.18	1.68273742614128\\
-4.17	1.68736949321341\\
-4.16	1.69178424855315\\
-4.15	1.6959732609898\\
-4.14	1.69994392037952\\
-4.13	1.70369667650188\\
-4.12	1.70722610334923\\
-4.11	1.71054040723485\\
-4.1	1.71363848756613\\
-4.09	1.71652255764032\\
-4.08	1.71919198936545\\
-4.07	1.72164584674339\\
-4.06	1.72389023495989\\
-4.05	1.72592638819506\\
-4.04	1.72775499562783\\
-4.03	1.72937983465306\\
-4.02	1.73079716731858\\
-4.01	1.73201508794583\\
-4	1.73303152862056\\
-3.99	1.73385054855379\\
-3.98	1.73447613849955\\
-3.97	1.73490829053265\\
-3.96	1.7351484411932\\
-3.95	1.73520392508336\\
-3.94	1.73507312555508\\
-3.93	1.73475818887298\\
-3.92	1.73426524932232\\
-3.91	1.733591777328\\
-3.9	1.73274367123886\\
-3.89	1.73172390474077\\
-3.88	1.73053691451151\\
-3.87	1.72918127819923\\
-3.86	1.7276635820552\\
-3.85	1.72598555673043\\
-3.84	1.72414874397617\\
-3.83	1.72215765695392\\
-3.82	1.72001154389155\\
-3.81	1.71771935925826\\
-3.8	1.71528055203239\\
-3.79	1.71270080342331\\
-3.78	1.70997806786249\\
-3.77	1.7071167372472\\
-3.76	1.70412656164513\\
-3.75	1.70100155462986\\
-3.74	1.6977485140194\\
-3.73	1.6943697768614\\
-3.72	1.69087122721681\\
-3.71	1.68724935345202\\
-3.7	1.68351639669007\\
-3.69	1.67966474095572\\
-3.68	1.67570288768729\\
-3.67	1.6716329084357\\
-3.66	1.6674591599759\\
-3.65	1.66318265722534\\
-3.64	1.65880560270395\\
-3.63	1.65433135407193\\
-3.62	1.64976464706148\\
-3.61	1.64510363204714\\
-3.6	1.64035583515226\\
-3.59	1.63552014131667\\
-3.58	1.63060213259583\\
-3.57	1.62559920533363\\
-3.56	1.62052209842123\\
-3.55	1.61536414839302\\
-3.54	1.61013359770101\\
-3.53	1.60483354989217\\
-3.52	1.59946390942201\\
-3.51	1.59402404841409\\
-3.5	1.58852171818223\\
-3.49	1.5829544187937\\
-3.48	1.57732687689941\\
-3.47	1.57164474505412\\
-3.46	1.56590391410254\\
-3.45	1.56010822433583\\
-3.44	1.55426022069861\\
-3.43	1.54836120640022\\
-3.42	1.54241503497523\\
-3.41	1.53642452832913\\
-3.4	1.5303857611902\\
-3.39	1.5243064827052\\
-3.38	1.51818499699686\\
-3.37	1.51202257229185\\
-3.36	1.50582275499008\\
-3.35	1.49959057892132\\
-3.34	1.49332095640522\\
-3.33	1.48702269521549\\
-3.32	1.48068769585645\\
-3.31	1.47432620819923\\
-3.3	1.4679353726389\\
-3.29	1.46151709392\\
-3.28	1.45507368958145\\
-3.27	1.44860600960066\\
-3.26	1.44211837795689\\
-3.25	1.43560598935923\\
-3.24	1.42907404679935\\
-3.23	1.42252451652241\\
-3.22	1.41595608221881\\
-3.21	1.40937227224872\\
-3.2	1.40277039025249\\
-3.19	1.3961562036384\\
-3.18	1.38952993561555\\
-3.17	1.38288957047114\\
-3.16	1.37623884736403\\
-3.15	1.36958128763005\\
-3.14	1.3629124680971\\
-3.13	1.35623428035724\\
-3.12	1.34954882390927\\
-3.11	1.34285808389947\\
-3.1	1.33616600833164\\
-3.09	1.32946446920697\\
-3.08	1.32276167321247\\
-3.07	1.31605566415392\\
-3.06	1.30934947354255\\
-3.05	1.30263906703518\\
-3.04	1.29592965777357\\
-3.03	1.28921909790001\\
-3.02	1.28251041151256\\
-3.01	1.2758053887043\\
-3	1.26910092663149\\
-2.99	1.26240082269776\\
-2.98	1.25570204519039\\
-2.97	1.24900957169441\\
-2.96	1.24232154812472\\
-2.95	1.23564025750399\\
-2.94	1.22896152240602\\
-2.93	1.22229040826965\\
-2.92	1.21562644925054\\
-2.91	1.2089694816316\\
-2.9	1.20231979613358\\
-2.89	1.19568101759967\\
-2.88	1.18904859628058\\
-2.87	1.18242326839157\\
-2.86	1.17581006852236\\
-2.85	1.16920600855842\\
-2.84	1.16260864521099\\
-2.83	1.15602252080257\\
-2.82	1.14944746036882\\
-2.81	1.14288292554528\\
-2.8	1.1363283654464\\
-2.79	1.12978428670912\\
-2.78	1.1232515362192\\
-2.77	1.11672791847967\\
-2.76	1.11021599633458\\
-2.75	1.10371645035215\\
-2.74	1.09722623828558\\
-2.73	1.09074699492132\\
-2.72	1.08427691871393\\
-2.71	1.07782078964884\\
-2.7	1.07137406614026\\
-2.69	1.06493532532687\\
-2.68	1.05850947347032\\
-2.67	1.05209293236049\\
-2.66	1.04568711414699\\
-2.65	1.03928823089326\\
-2.64	1.0328996728621\\
-2.63	1.02651951379206\\
-2.62	1.02014647083115\\
-2.61	1.01378429190168\\
-2.6	1.00742862543566\\
-2.59	1.00107838783864\\
-2.58	0.994735170468446\\
-2.57	0.988397586864339\\
-2.56	0.982066656854121\\
-2.55	0.975741426018671\\
-2.54	0.969414323645617\\
-2.53	0.963093795493109\\
-2.52	0.956779527845439\\
-2.51	0.950462492783566\\
-2.5	0.944148606024582\\
-2.49	0.937836159580524\\
-2.48	0.931522901565682\\
-2.47	0.925206257208095\\
-2.46	0.918889304745879\\
-2.45	0.912568203951841\\
-2.44	0.906243756982647\\
-2.43	0.899912959364605\\
-2.42	0.893575109807699\\
-2.41	0.88723347815918\\
-2.4	0.880882261457502\\
-2.39	0.874520425924668\\
-2.38	0.868148649791961\\
-2.37	0.861764130769861\\
-2.36	0.855368482841749\\
-2.35	0.848957983148779\\
-2.34	0.842532477873376\\
-2.33	0.836092971751531\\
-2.32	0.829632754122422\\
-2.31	0.823153550672019\\
-2.3	0.816655473385142\\
-2.29	0.810135284391483\\
-2.28	0.80359337295789\\
-2.27	0.797026322559428\\
-2.26	0.790434206357572\\
-2.25	0.783814086637904\\
-2.24	0.777166558197675\\
-2.23	0.77049017604118\\
-2.22	0.76378107126786\\
-2.21	0.757038008189732\\
-2.2	0.750264323537655\\
-2.19	0.743452525301761\\
-2.18	0.736605771654106\\
-2.17	0.729719566960712\\
-2.16	0.722793615875244\\
-2.15	0.715829014721271\\
-2.14	0.708818631626796\\
-2.13	0.701765826216212\\
-2.12	0.694667401597498\\
-2.11	0.687520054660993\\
-2.1	0.680328751083318\\
-2.09	0.673084221776624\\
-2.08	0.665790791825416\\
-2.07	0.658443578862746\\
-2.06	0.651044080360668\\
-2.05	0.643586414572063\\
-2.04	0.636072143522906\\
-2.03	0.628501976558661\\
-2.02	0.620871358520758\\
-2.01	0.613179329677222\\
-2	0.605428729645835\\
-1.99	0.597613084827818\\
-1.98	0.589733164424291\\
-1.97	0.581784312693161\\
-1.96	0.573770390410049\\
-1.95	0.565692451541706\\
-1.94	0.55754114713622\\
-1.93	0.54932123843082\\
-1.92	0.541028561312105\\
-1.91	0.532662200529788\\
-1.9	0.524225316762637\\
-1.89	0.515712401167977\\
-1.88	0.507128092539294\\
-1.87	0.498460563533196\\
-1.86	0.489719540565386\\
-1.85	0.480899038207187\\
-1.84	0.472003887517146\\
-1.83	0.463025545524985\\
-1.82	0.453965411686743\\
-1.81	0.444827610690617\\
-1.8	0.435604960267299\\
-1.79	0.426302389810722\\
-1.78	0.416920792574285\\
-1.77	0.407450400860782\\
-1.76	0.397896617613481\\
-1.75	0.388261502245183\\
-1.74	0.378539934129942\\
-1.73	0.368733885216992\\
-1.72	0.358843207118091\\
-1.71	0.348868664734168\\
-1.7	0.338809071011197\\
-1.69	0.328662750402235\\
-1.68	0.318430184145727\\
-1.67	0.308113808616918\\
-1.66	0.297715157223426\\
-1.65	0.287226691451466\\
-1.64	0.276655449956899\\
-1.63	0.266001775686683\\
-1.62	0.255259435718644\\
-1.61	0.244437751951862\\
-1.6	0.2335298244808\\
-1.59	0.222539478771345\\
-1.58	0.211469221238967\\
-1.57	0.200315135188084\\
-1.56	0.189081020560431\\
-1.55	0.177766658218117\\
-1.54	0.166372854281097\\
-1.53	0.154899284504029\\
-1.52	0.143348439414903\\
-1.51	0.131723312779668\\
-1.5	0.120020074276663\\
-1.49	0.108246109266694\\
-1.48	0.0963949349407187\\
-1.47	0.0844741480951388\\
-1.46	0.0724842870021624\\
-1.45	0.0604218340184262\\
-1.44	0.0482935814234559\\
-1.43	0.036098091312223\\
-1.42	0.0238405919781161\\
-1.41	0.0115172962524074\\
-1.4	-0.000865132694310547\\
-1.39	-0.0133096437315603\\
-1.38	-0.0258124969025519\\
-1.37	-0.0383684730526917\\
-1.36	-0.0509779978046293\\
-1.35	-0.0636408269870493\\
-1.34	-0.0763544232661175\\
-1.33	-0.0891148354547377\\
-1.32	-0.101922291022636\\
-1.31	-0.114770452306616\\
-1.3	-0.127662898942434\\
-1.29	-0.140591645087792\\
-1.28	-0.153559597329213\\
-1.27	-0.166559036561594\\
-1.26	-0.179588924841047\\
-1.25	-0.192650589705457\\
-1.24	-0.205737112629504\\
-1.23	-0.21884838228518\\
-1.22	-0.231981512271963\\
-1.21	-0.245128976520626\\
-1.2	-0.258293337722532\\
-1.19	-0.271471718628495\\
-1.18	-0.284658798797216\\
-1.17	-0.297852643643263\\
-1.16	-0.311052403874486\\
-1.15	-0.324249778461384\\
-1.14	-0.337445149515211\\
-1.13	-0.350637986399644\\
-1.12	-0.363817855666088\\
-1.11	-0.376990132435513\\
-1.1	-0.390145045689556\\
-1.09	-0.403283318006071\\
-1.08	-0.416398934353246\\
-1.07	-0.429488461271778\\
-1.06	-0.4425531119975\\
-1.05	-0.45558318583098\\
-1.04	-0.468580068981554\\
-1.03	-0.481537582993864\\
-1.02	-0.494454378549588\\
-1.01	-0.507323269555588\\
-1	-0.520143947912837\\
-0.99	-0.532914157508764\\
-0.98	-0.545629963918486\\
-0.97	-0.558282445152377\\
-0.96	-0.570871607078345\\
-0.95	-0.583396496582993\\
-0.94	-0.595850859910709\\
-0.93	-0.608230650079971\\
-0.92	-0.620537010753353\\
-0.91	-0.632756900148659\\
-0.9	-0.644894567923895\\
-0.89	-0.656943750238026\\
-0.88	-0.668900722531347\\
-0.87	-0.680764212188115\\
-0.86	-0.692528114228149\\
-0.85	-0.704186567801471\\
-0.84	-0.715742038252149\\
-0.83	-0.727186570276945\\
-0.82	-0.738516388212104\\
-0.81	-0.749731429193839\\
-0.8	-0.760823943187095\\
-0.79	-0.771793743087262\\
-0.78	-0.782636382990627\\
-0.77	-0.793347448942755\\
-0.76	-0.803925436104387\\
-0.75	-0.814362974199851\\
-0.74	-0.824659886287619\\
-0.73	-0.834812768267449\\
-0.72	-0.844817783380598\\
-0.71	-0.854672676739526\\
-0.7	-0.864373642116721\\
-0.69	-0.873914961213885\\
-0.68	-0.883294044254748\\
-0.67	-0.892511099530508\\
-0.66	-0.901561914184902\\
-0.649999999999999	-0.910442609325384\\
-0.64	-0.919150096467233\\
-0.63	-0.927681531963208\\
-0.62	-0.936033540940578\\
-0.61	-0.944204670482494\\
-0.6	-0.952191985775367\\
-0.59	-0.959992518483509\\
-0.58	-0.967602053603757\\
-0.57	-0.975019903354143\\
-0.56	-0.982243874234075\\
-0.55	-0.989272832022991\\
-0.54	-0.996101326249132\\
-0.53	-1.00272605507552\\
-0.52	-1.00914970900615\\
-0.51	-1.01536820280239\\
-0.5	-1.02137400342114\\
-0.49	-1.02717424179103\\
-0.48	-1.03276254751875\\
-0.47	-1.03813809796514\\
-0.46	-1.04329852341611\\
-0.45	-1.0482395402378\\
-0.44	-1.05296807840168\\
-0.43	-1.0574729401129\\
-0.42	-1.06175891778768\\
-0.41	-1.06582280242503\\
-0.399999999999999	-1.06966482815531\\
-0.39	-1.07328011137636\\
-0.38	-1.0766728479958\\
-0.37	-1.07984295698082\\
-0.36	-1.08278283977608\\
-0.35	-1.08549711898519\\
-0.34	-1.08798546637077\\
-0.33	-1.09024543059656\\
-0.32	-1.0922765557698\\
-0.31	-1.09408202810461\\
-0.3	-1.09565695541108\\
-0.29	-1.09700852878744\\
-0.28	-1.09813028089712\\
-0.27	-1.09902510567854\\
-0.26	-1.09969255201166\\
-0.25	-1.10013452023465\\
-0.24	-1.10035032471444\\
-0.23	-1.10034225536748\\
-0.22	-1.10011054147156\\
-0.21	-1.09965541453979\\
-0.2	-1.09897808015442\\
-0.19	-1.09808141723822\\
-0.18	-1.09696853433051\\
-0.17	-1.09563562741147\\
-0.16	-1.09408902464314\\
-0.149999999999999	-1.09232281952632\\
-0.14	-1.09034990450179\\
-0.13	-1.08816475688427\\
-0.12	-1.08577384059437\\
-0.11	-1.08317459217972\\
-0.0999999999999996	-1.08037023092821\\
-0.0899999999999999	-1.07736839206371\\
-0.0800000000000001	-1.07416265071785\\
-0.0700000000000003	-1.07076685544625\\
-0.0599999999999996	-1.06717506674725\\
-0.0499999999999998	-1.06339081953685\\
-0.04	-1.05942266499923\\
-0.0300000000000002	-1.05526993359539\\
-0.0199999999999996	-1.05093149755253\\
-0.00999999999999979	-1.04641929313326\\
0	-1.04173025799163\\
0.00999999999999979	-1.03687042754833\\
0.0199999999999996	-1.03184056119751\\
0.0300000000000002	-1.02664900057208\\
0.04	-1.02129617792539\\
0.0499999999999998	-1.01578634229229\\
0.0599999999999996	-1.01012664636645\\
0.0700000000000003	-1.00431477064073\\
0.0800000000000001	-0.998358309249981\\
0.0899999999999999	-0.992262198933153\\
0.0999999999999996	-0.986029691610229\\
0.11	-0.979663683108811\\
0.12	-0.973173825458004\\
0.13	-0.966558302075374\\
0.14	-0.959821949764786\\
0.149999999999999	-0.952974637693248\\
0.16	-0.946015072605429\\
0.17	-0.938951157855454\\
0.18	-0.931786749425471\\
0.19	-0.924525985214861\\
0.2	-0.917175007436513\\
0.21	-0.909736800912135\\
0.22	-0.902219066422774\\
0.23	-0.89462262918089\\
0.24	-0.886955883212841\\
0.25	-0.879223100502304\\
0.26	-0.871425999786943\\
0.27	-0.863576013888266\\
0.28	-0.855672131629191\\
0.29	-0.847722529102819\\
0.3	-0.839729459544502\\
0.31	-0.831704011311603\\
0.32	-0.823644754985051\\
0.33	-0.815558710625523\\
0.34	-0.807451511497219\\
0.35	-0.799328168651928\\
0.36	-0.791193881689374\\
0.37	-0.783053585345444\\
0.38	-0.774912730933932\\
0.39	-0.766775436617227\\
0.399999999999999	-0.758643820721187\\
0.41	-0.750531342167238\\
0.42	-0.74243241147015\\
0.43	-0.734359009357179\\
0.44	-0.726313197409059\\
0.45	-0.718302756081355\\
0.46	-0.710327480423093\\
0.47	-0.702396081605626\\
0.48	-0.694510723063973\\
0.49	-0.686679670612957\\
0.5	-0.678900874225193\\
0.51	-0.671185684115615\\
0.52	-0.66353635476725\\
0.53	-0.655956369717432\\
0.54	-0.648451741070001\\
0.55	-0.641022937553353\\
0.56	-0.633677986131089\\
0.57	-0.626419322766597\\
0.58	-0.619252795869882\\
0.59	-0.612179273027791\\
0.6	-0.605207119305399\\
0.61	-0.598337854391778\\
0.62	-0.591573879253503\\
0.63	-0.584917367825927\\
0.64	-0.578377429150589\\
0.649999999999999	-0.571956633506191\\
0.66	-0.565654110699807\\
0.67	-0.559475829915397\\
0.68	-0.553426255352215\\
0.69	-0.547504550963361\\
0.7	-0.541719965115348\\
0.71	-0.53607137838481\\
0.72	-0.530562753118533\\
0.73	-0.525196547473359\\
0.74	-0.519975527702062\\
0.75	-0.514905469082734\\
0.76	-0.509981192533419\\
0.77	-0.50521361198888\\
0.78	-0.500598556061554\\
0.79	-0.496143251916271\\
0.8	-0.491850609644567\\
0.81	-0.487716426207921\\
0.82	-0.483749976465299\\
0.83	-0.479948732570063\\
0.84	-0.476314977011607\\
0.85	-0.472853248590843\\
0.86	-0.469562206091617\\
0.87	-0.46644320652241\\
0.88	-0.463501948482515\\
0.89	-0.460736520347383\\
0.9	-0.458144834125411\\
0.91	-0.455734631509756\\
0.92	-0.453503942095085\\
0.93	-0.451455085687004\\
0.94	-0.449588596404892\\
0.95	-0.447902819082414\\
0.96	-0.44640121625005\\
0.97	-0.445087000368398\\
0.98	-0.443952957688496\\
0.99	-0.443011072055298\\
1	-0.442252715242583\\
1.01	-0.441679908765369\\
1.02	-0.441294792719944\\
1.03	-0.441095624334084\\
1.04	-0.441084226601764\\
1.05	-0.441262190579774\\
1.06	-0.441624435103408\\
1.07	-0.442175478059888\\
1.08	-0.442916296586152\\
1.09	-0.443840028565892\\
1.1	-0.444951279328964\\
1.11	-0.446251870283251\\
1.12	-0.447736362622136\\
1.13	-0.449404672883303\\
1.14	-0.451257273025686\\
1.15	-0.453297974640013\\
1.16	-0.455520244396749\\
1.17	-0.457924129750133\\
1.18	-0.460511431065743\\
1.19	-0.463279906977389\\
1.2	-0.46622852347451\\
1.21	-0.469356543477495\\
1.22	-0.472657702968562\\
1.23	-0.476139871970821\\
1.24	-0.479796456963326\\
1.25	-0.483627046891381\\
1.26	-0.48763190631733\\
1.27	-0.491805672345527\\
1.28	-0.496149390030907\\
1.29	-0.500665536188378\\
1.3	-0.505343957063577\\
1.31	-0.510189226578714\\
1.32	-0.515197574999416\\
1.33	-0.520368048654081\\
1.34	-0.525697578127628\\
1.35	-0.531187287846648\\
1.36	-0.53682993094028\\
1.37	-0.542628130034944\\
1.38	-0.548578766367987\\
1.39	-0.554677782170115\\
1.4	-0.56092640873903\\
1.41	-0.567319760123011\\
1.42	-0.573855216750154\\
1.43	-0.580533373136323\\
1.44	-0.587350937928141\\
1.45	-0.594304105489868\\
1.46	-0.601390089145565\\
1.47	-0.608604473251608\\
1.48	-0.615948565625002\\
1.49	-0.623421638241491\\
1.5	-0.631018101882943\\
1.51	-0.638732074050795\\
1.52	-0.646563979573092\\
1.53	-0.654508590595563\\
1.54	-0.662567184141812\\
1.55	-0.670736535235872\\
1.56	-0.679011214819922\\
1.57	-0.687387322542779\\
1.58	-0.69586456157006\\
1.59	-0.704434757556931\\
1.6	-0.713102242259034\\
1.61	-0.721858194090065\\
1.62	-0.730702032277213\\
1.63	-0.739626842847688\\
1.64	-0.748634804777695\\
1.65	-0.757716083245848\\
1.66	-0.766873833827812\\
1.67	-0.776098277581716\\
1.68	-0.785390591208983\\
1.69	-0.794743101707104\\
1.7	-0.804157006790474\\
1.71	-0.813624841478437\\
1.72	-0.823144370606124\\
1.73	-0.832710111300277\\
1.74	-0.842320638050648\\
1.75	-0.851970799825533\\
1.76	-0.861654902770644\\
1.77	-0.871371423365046\\
1.78	-0.881118117720467\\
1.79	-0.890884526467072\\
1.8	-0.900673574611391\\
1.81	-0.910476233803825\\
1.82	-0.920294483217941\\
1.83	-0.930117590820188\\
1.84	-0.939942512363757\\
1.85	-0.949767780903943\\
1.86	-0.959586655646973\\
1.87	-0.969397706300889\\
1.88	-0.979194155445712\\
1.89	-0.988973412106878\\
1.9	-0.998730134065084\\
1.91	-1.00845850129569\\
1.92	-1.01815722538279\\
1.93	-1.02782046334047\\
1.94	-1.0374430465925\\
1.95	-1.04702345499614\\
1.96	-1.05655483579847\\
1.97	-1.06603105944815\\
1.98	-1.07545163209747\\
1.99	-1.0848128954538\\
2	-1.09410549840382\\
2.01	-1.10333024929237\\
2.02	-1.11247607292712\\
2.03	-1.12154502009942\\
2.04	-1.13053071144902\\
2.05	-1.13942793036604\\
2.06	-1.14823476217707\\
2.07	-1.15694419703961\\
2.08	-1.16555251295488\\
2.09	-1.17405497515753\\
2.1	-1.18244960546418\\
2.11	-1.1907312231469\\
2.12	-1.19889332706277\\
2.13	-1.20693602700356\\
2.14	-1.21485282195047\\
2.15	-1.22263749198156\\
2.16	-1.23029009072063\\
2.17	-1.23780624611992\\
2.18	-1.24517976953638\\
2.19	-1.25240741017222\\
2.2	-1.25948387766972\\
2.21	-1.26640867284053\\
2.22	-1.27317698798364\\
2.23	-1.27978464181241\\
2.24	-1.28622751922118\\
2.25	-1.29250143018874\\
2.26	-1.29860562175732\\
2.27	-1.30453636807409\\
2.28	-1.3102878377152\\
2.29	-1.3158574711145\\
2.3	-1.32124440092978\\
2.31	-1.32644122453029\\
2.32	-1.33144949212791\\
2.33	-1.33626217837081\\
2.34	-1.3408812614018\\
2.35	-1.34530012214747\\
2.36	-1.34951380558893\\
2.37	-1.35352914877656\\
2.38	-1.35733384820535\\
2.39	-1.36092764208161\\
2.4	-1.36430998254247\\
2.41	-1.36748155793862\\
2.42	-1.37043245711519\\
2.43	-1.37316629611818\\
2.44	-1.3756788472094\\
2.45	-1.37796994568942\\
2.46	-1.38003685669003\\
2.47	-1.38187902770559\\
2.48	-1.38349013821701\\
2.49	-1.38487433853189\\
2.5	-1.38602974663952\\
2.51	-1.38695019236116\\
2.52	-1.38764057886538\\
2.53	-1.3880967473759\\
2.54	-1.38831646170582\\
2.55	-1.38830033506149\\
2.56	-1.38804634396345\\
2.57	-1.3875574494103\\
2.58	-1.38682904751712\\
2.59	-1.38586205212276\\
2.6	-1.38465826030178\\
2.61	-1.38321315207499\\
2.62	-1.38152942687205\\
2.63	-1.37960679920545\\
2.64	-1.37744520500877\\
2.65	-1.37504407400483\\
2.66	-1.37240314630298\\
2.67	-1.36952460170287\\
2.68	-1.36640488434602\\
2.69	-1.36305168606616\\
2.7	-1.35945886695077\\
2.71	-1.35563073700289\\
2.72	-1.35156417307231\\
2.73	-1.3472643028076\\
2.74	-1.34273299472128\\
2.75	-1.33796586436057\\
2.76	-1.33296984598564\\
2.77	-1.32774053066465\\
2.78	-1.32228159944425\\
2.79	-1.31659899080661\\
2.8	-1.3106889774197\\
2.81	-1.3045547784405\\
2.82	-1.29819444863088\\
2.83	-1.29161531012978\\
2.84	-1.28481590446299\\
2.85	-1.27780047911101\\
2.86	-1.27056958478386\\
2.87	-1.26312435297289\\
2.88	-1.25546803857\\
2.89	-1.24760286501551\\
2.9	-1.2395293847901\\
2.91	-1.23125221411482\\
2.92	-1.22277060286827\\
2.93	-1.21409194658507\\
2.94	-1.2052128551219\\
2.95	-1.19614193630919\\
2.96	-1.18687339881819\\
2.97	-1.17741914768252\\
2.98	-1.16777650387911\\
2.99	-1.15794960471296\\
3	-1.1479407978164\\
3.01	-1.13775283898861\\
3.02	-1.12738859167416\\
3.03	-1.11684895613825\\
3.04	-1.10614087805782\\
3.05	-1.09526528019844\\
3.06	-1.08422602304191\\
3.07	-1.07302548127386\\
3.08	-1.06166147640111\\
3.09	-1.05014692486156\\
3.1	-1.03847846958806\\
3.11	-1.02666202069265\\
3.12	-1.01469647910102\\
3.13	-1.00259111184894\\
3.14	-0.990342468357359\\
3.15	-0.977962623549444\\
3.16	-0.965443520215047\\
3.17	-0.95279648649851\\
3.18	-0.940021763340627\\
3.19	-0.92712446874196\\
3.2	-0.914108789677701\\
3.21	-0.90097271388834\\
3.22	-0.887719001136834\\
3.23	-0.874359180836266\\
3.24	-0.860892749344532\\
3.25	-0.847319151222937\\
3.26	-0.833645501875529\\
3.27	-0.819877966495473\\
3.28	-0.806011387432125\\
3.29	-0.792057686242146\\
3.3	-0.778011051797234\\
3.31	-0.763882467602888\\
3.32	-0.749671676276405\\
3.33	-0.735382367766609\\
3.34	-0.721021048152921\\
3.35	-0.706587565789054\\
3.36	-0.692084399072036\\
3.37	-0.677516943074175\\
3.38	-0.662888176452201\\
3.39	-0.648200678854054\\
3.4	-0.633459061096338\\
3.41	-0.61866354508524\\
3.42	-0.603821904196693\\
3.43	-0.588931735616155\\
3.44	-0.573999935554782\\
3.45	-0.559029092988531\\
3.46	-0.544021898549421\\
3.47	-0.528982404176083\\
3.48	-0.513912394709861\\
3.49	-0.498817155564874\\
3.5	-0.483698319135437\\
3.51	-0.468556620450072\\
3.52	-0.453403164299818\\
3.53	-0.438232540589736\\
3.54	-0.423051328161779\\
3.55	-0.40786085307783\\
3.56	-0.392669233388612\\
3.57	-0.377473007482351\\
3.58	-0.36227856061906\\
3.59	-0.347090119918145\\
3.6	-0.331907375223771\\
3.61	-0.316735997273316\\
3.62	-0.301575835736221\\
3.63	-0.286438314485469\\
3.64	-0.271316676146312\\
3.65	-0.256216463154614\\
3.66	-0.24114442555175\\
3.67	-0.226100374421196\\
3.68	-0.211088535269081\\
3.69	-0.196108621876267\\
3.7	-0.181168042904911\\
3.71	-0.166267422283835\\
3.72	-0.151411066578926\\
3.73	-0.136603162539274\\
3.74	-0.121838130393576\\
3.75	-0.107131769972964\\
3.76	-0.0924765990668149\\
3.77	-0.0778789362396664\\
3.78	-0.0633444980394732\\
3.79	-0.0488729225960722\\
3.8	-0.0344665286183141\\
3.81	-0.0201313276682043\\
3.82	-0.00586799059136035\\
3.83	0.0083185999305582\\
3.84	0.0224308880750632\\
3.85	0.0364615424730344\\
3.86	0.0504104306938162\\
3.87	0.0642713864896718\\
3.88	0.0780447674733922\\
3.89	0.0917249345867606\\
3.9	0.105314018918415\\
3.91	0.118804714200176\\
3.92	0.132194821888225\\
3.93	0.145484403252788\\
3.94	0.158666988700237\\
3.95	0.171744064582893\\
3.96	0.1847123752423\\
3.97	0.197565918939051\\
3.98	0.210306146553519\\
3.99	0.222930425715884\\
4	0.235431454157851\\
4.01	0.247812370172164\\
4.02	0.260066909985922\\
4.03	0.27219739395607\\
4.04	0.284197050368048\\
4.05	0.296062565118112\\
4.06	0.307796320552286\\
4.07	0.319393695417661\\
4.08	0.330849575061935\\
4.09	0.342167834053527\\
4.1	0.353342711965728\\
4.11	0.364373977503549\\
4.12	0.375256338306098\\
4.13	0.385989072505257\\
4.14	0.396572514902229\\
4.15	0.407003695236002\\
4.16	0.417278314243733\\
4.17	0.427395388716002\\
4.18	0.43736050120831\\
4.19	0.44715907515442\\
4.2	0.456799713264159\\
4.21	0.466277884905989\\
4.22	0.475588860623007\\
4.23	0.484733054097044\\
4.24	0.493707990143001\\
4.25	0.502515637021961\\
4.26	0.511153601645866\\
4.27	0.519619751822205\\
4.28	0.527914726129112\\
4.29	0.536035436851563\\
4.3	0.543978978300173\\
4.31	0.551744985180101\\
4.32	0.55933681708585\\
4.33	0.566753226061588\\
4.34	0.573986643273978\\
4.35	0.581042341731302\\
4.36	0.587916959578766\\
4.37	0.594613383855574\\
4.38	0.601126703271707\\
4.39	0.607460480522579\\
4.4	0.613612766669658\\
4.41	0.619583555662701\\
4.42	0.625369638677191\\
4.43	0.630978131542273\\
4.44	0.636402098583937\\
4.45	0.641641246077299\\
4.46	0.646703684227608\\
4.47	0.651581282637654\\
4.48	0.65627667127225\\
4.49	0.660794967671975\\
4.5	0.665130996687239\\
4.51	0.669287144222714\\
4.52	0.673263902283063\\
4.53	0.677061118589386\\
4.54	0.680686152237768\\
4.55	0.684130906257507\\
4.56	0.687401739578508\\
4.57	0.690496580787613\\
4.58	0.693418078435746\\
4.59	0.696170474716513\\
4.6	0.698747999559822\\
4.61	0.701160683220895\\
4.62	0.703403042745423\\
4.63	0.705481358564256\\
4.64	0.707395852709782\\
4.65	0.70914805106085\\
4.66	0.710738129102126\\
4.67	0.712170818747135\\
4.68	0.713444155896821\\
4.69	0.714565922814316\\
4.7	0.71553935596285\\
4.71	0.716358466803034\\
4.72	0.717031899741878\\
4.73	0.717558042266008\\
4.74	0.717944239940765\\
4.75	0.718187926481235\\
4.76	0.718296065198587\\
4.77	0.718269448592081\\
4.78	0.718110578245749\\
4.79	0.717823749952165\\
4.8	0.717410813444256\\
4.81	0.716875483886893\\
4.82	0.716222757758952\\
4.83	0.715453290271106\\
4.84	0.714566684310104\\
4.85	0.713573628099149\\
4.86	0.712472678599949\\
4.87	0.711267593535005\\
4.88	0.70996361669679\\
4.89	0.708562047155737\\
4.9	0.707071821978017\\
4.91	0.705489406839346\\
4.92	0.703823735882706\\
4.93	0.702073305723494\\
4.94	0.700246639906203\\
4.95	0.698346737156261\\
4.96	0.696375127671015\\
4.97	0.694339387671182\\
4.98	0.692237127754173\\
4.99	0.690080949965642\\
5	0.687866122230017\\
};
\addplot [color=white!80!black, line width=1.3pt, forget plot,dotted]
  table[row sep=crcr]{%
-5	-0.145340871016341\\
-4.99	-0.149846985488889\\
-4.98	-0.154414144537871\\
-4.97	-0.159032352984443\\
-4.96	-0.163707017220306\\
-4.95	-0.168426248170787\\
-4.94	-0.173192167519646\\
-4.93	-0.177999173500337\\
-4.92	-0.182843643800942\\
-4.91	-0.187722784153966\\
-4.9	-0.1926355072798\\
-4.89	-0.19757631237417\\
-4.88	-0.202540296509571\\
-4.87	-0.207527721782197\\
-4.86	-0.212532657333716\\
-4.85	-0.217551810031582\\
-4.84	-0.222583349849231\\
-4.83	-0.227624018986225\\
-4.82	-0.232669252846188\\
-4.81	-0.237713972439646\\
-4.8	-0.242758248425876\\
-4.79	-0.247796502309387\\
-4.78	-0.252825702293404\\
-4.77	-0.257842397676787\\
-4.76	-0.262846109339992\\
-4.75	-0.267824903193645\\
-4.74	-0.272786362753611\\
-4.73	-0.277720169395774\\
-4.72	-0.282627480036554\\
-4.71	-0.287497898477798\\
-4.7	-0.2923343374078\\
-4.69	-0.297131597266422\\
-4.68	-0.30188521171988\\
-4.67	-0.306592337873641\\
-4.66	-0.311250557342263\\
-4.65	-0.315860162507751\\
-4.64	-0.320409231573877\\
-4.63	-0.324900781176302\\
-4.62	-0.329329055730161\\
-4.61	-0.333693106520889\\
-4.6	-0.33798733821324\\
-4.59	-0.342209929196567\\
-4.58	-0.346357781121689\\
-4.57	-0.350426004014256\\
-4.56	-0.354414109271499\\
-4.55	-0.358316153210863\\
-4.54	-0.36213313302327\\
-4.53	-0.365855466816556\\
-4.52	-0.369488785565258\\
-4.51	-0.373023378956958\\
-4.5	-0.376460818913058\\
-4.49	-0.379794490828018\\
-4.48	-0.383021020182884\\
-4.47	-0.386141741833888\\
-4.46	-0.38915333095158\\
-4.45	-0.392048779508751\\
-4.44	-0.394830907196926\\
-4.43	-0.39749601309084\\
-4.42	-0.400037548157575\\
-4.41	-0.402453136418251\\
-4.4	-0.404746574636373\\
-4.39	-0.406913715314402\\
-4.38	-0.408945432522318\\
-4.37	-0.410844123086013\\
-4.36	-0.412610949081128\\
-4.35	-0.41423929618578\\
-4.34	-0.415727125052573\\
-4.33	-0.417075972333095\\
-4.32	-0.418279027115675\\
-4.31	-0.419334227961654\\
-4.3	-0.420248042957275\\
-4.29	-0.421009329797244\\
-4.28	-0.421620021226334\\
-4.27	-0.422078396657569\\
-4.26	-0.422382035709337\\
-4.25	-0.422531746061428\\
-4.24	-0.422524534580991\\
-4.23	-0.422357799787014\\
-4.22	-0.42203464705532\\
-4.21	-0.421547475768778\\
-4.2	-0.420896151876873\\
-4.19	-0.420087005501052\\
-4.18	-0.419110760756177\\
-4.17	-0.417969980043393\\
-4.16	-0.416663604810669\\
-4.15	-0.415191026891835\\
-4.14	-0.413552067891706\\
-4.13	-0.411743573181994\\
-4.12	-0.409767610031216\\
-4.11	-0.407622939089875\\
-4.1	-0.405308495341652\\
-4.09	-0.402826021953309\\
-4.08	-0.400174970905362\\
-4.07	-0.397353319950127\\
-4.06	-0.394360526576032\\
-4.05	-0.391198685941256\\
-4.04	-0.387869887435493\\
-4.03	-0.384371965066078\\
-4.02	-0.380704710012542\\
-4.01	-0.376866782029695\\
-4	-0.372862485392265\\
-3.99	-0.368690917717081\\
-3.98	-0.364352254155896\\
-3.97	-0.359849076391435\\
-3.96	-0.355179882905463\\
-3.95	-0.350345824956808\\
-3.94	-0.345348527870499\\
-3.93	-0.340186631965977\\
-3.92	-0.334869219472166\\
-3.91	-0.32938796175358\\
-3.9	-0.323748293712019\\
-3.89	-0.31795367250042\\
-3.88	-0.312002696462493\\
-3.87	-0.305895091031308\\
-3.86	-0.299637105380341\\
-3.85	-0.293226523815362\\
-3.84	-0.286668428692331\\
-3.83	-0.279961057629391\\
-3.82	-0.273108362920927\\
-3.81	-0.266115323939743\\
-3.8	-0.258975143253335\\
-3.79	-0.251701310172259\\
-3.78	-0.244285218790689\\
-3.77	-0.236735874653983\\
-3.76	-0.229053133549484\\
-3.75	-0.221238066585395\\
-3.74	-0.213299700747863\\
-3.73	-0.205232395507575\\
-3.72	-0.197042726310724\\
-3.71	-0.188729221495847\\
-3.7	-0.180299360171656\\
-3.69	-0.171754448001014\\
-3.68	-0.163099330236287\\
-3.67	-0.154331310724545\\
-3.66	-0.145455632915061\\
-3.65	-0.13647649060365\\
-3.64	-0.127397111007839\\
-3.63	-0.11821557130833\\
-3.62	-0.108942695315741\\
-3.61	-0.0995744729800261\\
-3.6	-0.0901150754334118\\
-3.59	-0.0805703231604001\\
-3.58	-0.0709442303936402\\
-3.57	-0.0612350037114586\\
-3.56	-0.051447733016052\\
-3.55	-0.0415903224881422\\
-3.54	-0.0316574410455984\\
-3.53	-0.0216566615574774\\
-3.52	-0.0115926827615331\\
-3.51	-0.00146598016265431\\
-3.5	0.0087174615123542\\
-3.49	0.0189552207531185\\
-3.48	0.0292464222751884\\
-3.47	0.0395889428235465\\
-3.46	0.0499735035794564\\
-3.45	0.0604025390036287\\
-3.44	0.0708676993022199\\
-3.43	0.0813688504505954\\
-3.42	0.0919024516552749\\
-3.41	0.102468446295024\\
-3.4	0.113055497087707\\
-3.39	0.123671766767767\\
-3.38	0.134304964520961\\
-3.37	0.144953999215102\\
-3.36	0.155618198239356\\
-3.35	0.166290838156451\\
-3.34	0.17696925054903\\
-3.33	0.187655095321074\\
-3.32	0.198340010529721\\
-3.31	0.209025952454864\\
-3.3	0.219704559058783\\
-3.29	0.230376174221356\\
-3.28	0.24103601212916\\
-3.27	0.251682028785229\\
-3.26	0.262313373141824\\
-3.25	0.27292362897178\\
-3.24	0.283514734976756\\
-3.23	0.294077587443034\\
-3.22	0.30461360923036\\
-3.21	0.315120450144959\\
-3.2	0.325595872403471\\
-3.19	0.336033006891846\\
-3.18	0.346434139035377\\
-3.17	0.356792790070794\\
-3.16	0.367111687510443\\
-3.15	0.377383693167695\\
-3.14	0.387608563181775\\
-3.13	0.397784573579643\\
-3.12	0.407908163053489\\
-3.11	0.417977821188997\\
-3.1	0.427995505513639\\
-3.09	0.437950820123746\\
-3.08	0.447848685436964\\
-3.07	0.457684558195389\\
-3.06	0.467454514355636\\
-3.05	0.477160259506673\\
-3.04	0.486797403036422\\
-3.03	0.496367303808308\\
-3.02	0.505867068244687\\
-3.01	0.515294589854302\\
-3	0.524649092993524\\
-2.99	0.533928296744921\\
-2.98	0.543129634358224\\
-2.97	0.552255425542271\\
-2.96	0.561304595014732\\
-2.95	0.57026989305526\\
-2.94	0.579154821509226\\
-2.93	0.587958899787367\\
-2.92	0.596680207193367\\
-2.91	0.605316184288076\\
-2.9	0.613867725928954\\
-2.89	0.622333667321512\\
-2.88	0.630715143060954\\
-2.87	0.639008113731594\\
-2.86	0.647211644593902\\
-2.85	0.655327945195572\\
-2.84	0.66335393493775\\
-2.83	0.671294910686854\\
-2.82	0.679142542194061\\
-2.81	0.686900768429352\\
-2.8	0.694569211830079\\
-2.79	0.702144687401232\\
-2.78	0.709630623502354\\
-2.77	0.717026503398144\\
-2.76	0.724329642882135\\
-2.75	0.731539409707817\\
-2.74	0.738660428177987\\
-2.73	0.745690281558075\\
-2.72	0.752629072183872\\
-2.71	0.759474749407324\\
-2.7	0.766230052096368\\
-2.69	0.772893424278076\\
-2.68	0.779467216947609\\
-2.67	0.78595031593829\\
-2.66	0.792340867544502\\
-2.65	0.798644844075409\\
-2.64	0.804854013810113\\
-2.63	0.810977559445387\\
-2.62	0.817010426142758\\
-2.61	0.822957075491707\\
-2.6	0.828811519021144\\
-2.59	0.834584166359375\\
-2.58	0.840262442522962\\
-2.57	0.84585564320018\\
-2.56	0.851366596039063\\
-2.55	0.856786420761646\\
-2.54	0.862122114779533\\
-2.53	0.867373434073959\\
-2.52	0.872538378286576\\
-2.51	0.877619323719085\\
-2.5	0.882617880709664\\
-2.49	0.887533593902935\\
-2.48	0.892367626108659\\
-2.47	0.897119085363438\\
-2.46	0.901788195001461\\
-2.45	0.906377910657226\\
-2.44	0.910888764766709\\
-2.43	0.915316459417764\\
-2.42	0.919666818096102\\
-2.41	0.923936591121784\\
-2.4	0.928130993210254\\
-2.39	0.932243500303188\\
-2.38	0.936281294368051\\
-2.37	0.940240800917687\\
-2.36	0.944124477075473\\
-2.35	0.947935195272347\\
-2.34	0.951665705729432\\
-2.33	0.95532229777546\\
-2.32	0.958906082928432\\
-2.31	0.962412611165741\\
-2.3	0.965846809942999\\
-2.29	0.969207058367997\\
-2.28	0.972491789074906\\
-2.27	0.975704882116488\\
-2.26	0.978847603385526\\
-2.25	0.98191380298692\\
-2.24	0.984909925271745\\
-2.23	0.987833867406699\\
-2.22	0.990684758848086\\
-2.21	0.993463832246553\\
-2.2	0.996173164546681\\
-2.19	0.998806889895664\\
-2.18	1.00137425901768\\
-2.17	1.00386420252088\\
-2.16	1.00628617523658\\
-2.15	1.00863720550407\\
-2.14	1.01091678478166\\
-2.13	1.01312278970234\\
-2.12	1.015261700545\\
-2.11	1.01732650992987\\
-2.1	1.01931925356241\\
-2.09	1.02124108959752\\
-2.08	1.02309161227656\\
-2.07	1.02487036643088\\
-2.06	1.02657810277642\\
-2.05	1.02821350617399\\
-2.04	1.02977610723494\\
-2.03	1.03126751881158\\
-2.02	1.03268543233813\\
-2.01	1.03403226029328\\
-2	1.03530765875796\\
-1.99	1.0365085032742\\
-1.98	1.03763530562488\\
-1.97	1.03869057962699\\
-1.96	1.03967169303699\\
-1.95	1.04057910451703\\
-1.94	1.04141306753905\\
-1.93	1.04217490012772\\
-1.92	1.04285952212413\\
-1.91	1.04347205592028\\
-1.9	1.04401021796421\\
-1.89	1.04447137991065\\
-1.88	1.04485824616908\\
-1.87	1.04517341092696\\
-1.86	1.04541143975313\\
-1.85	1.04557346712109\\
-1.84	1.04566312417408\\
-1.83	1.04567532846371\\
-1.82	1.0456118996567\\
-1.81	1.04547352907989\\
-1.8	1.04525962008362\\
-1.79	1.0449694598641\\
-1.78	1.04460931892935\\
-1.77	1.04416464310483\\
-1.76	1.04365152824014\\
-1.75	1.04306098751104\\
-1.74	1.04239382669397\\
-1.73	1.04165402596527\\
-1.72	1.04083928085164\\
-1.71	1.0399504964223\\
-1.7	1.03898586500144\\
-1.69	1.03794856754457\\
-1.68	1.03683891249505\\
-1.67	1.03565284984033\\
-1.66	1.03439466043021\\
-1.65	1.03306308443132\\
-1.64	1.03166293926532\\
-1.63	1.03018781405904\\
-1.62	1.0286451948094\\
-1.61	1.02702908667593\\
-1.6	1.025343902094\\
-1.59	1.02358909422704\\
-1.58	1.02176606642127\\
-1.57	1.01987410035623\\
-1.56	1.01791791101256\\
-1.55	1.01589274289756\\
-1.54	1.01380422140146\\
-1.53	1.01164726266579\\
-1.52	1.0094307840545\\
-1.51	1.00714831098921\\
-1.5	1.0048038908879\\
-1.49	1.00240056517468\\
-1.48	0.999937432146401\\
-1.47	0.997415468221986\\
-1.46	0.994835893099522\\
-1.45	0.992197607830255\\
-1.44	0.989504265094761\\
-1.43	0.986762314316582\\
-1.42	0.983960016651506\\
-1.41	0.981108367924619\\
-1.4	0.978207799720187\\
-1.39	0.975255121263861\\
-1.38	0.972256052209331\\
-1.37	0.969208782427145\\
-1.36	0.966118368248246\\
-1.35	0.962981687110416\\
-1.34	0.959804899750427\\
-1.33	0.956585487850189\\
-1.32	0.953325147838808\\
-1.31	0.950027322751067\\
-1.3	0.946691806861011\\
-1.29	0.943321593071852\\
-1.28	0.939915412683744\\
-1.27	0.936478036546545\\
-1.26	0.933006957792383\\
-1.25	0.929508338181091\\
-1.24	0.925978949367331\\
-1.23	0.922420159987079\\
-1.22	0.918839385146545\\
-1.21	0.915233213406684\\
-1.2	0.911603664422851\\
-1.19	0.907950640451561\\
-1.18	0.90427939352927\\
-1.17	0.900586972397765\\
-1.16	0.896879030964829\\
-1.15	0.893151077279013\\
-1.14	0.889409251993883\\
-1.13	0.885652918538143\\
-1.12	0.88188366239743\\
-1.11	0.878102732418184\\
-1.1	0.874311141254197\\
-1.09	0.870506984402019\\
-1.08	0.866695021326847\\
-1.07	0.862878792450656\\
-1.06	0.859053469477969\\
-1.05	0.855220261177948\\
-1.04	0.851384322235155\\
-1.03	0.847546131800029\\
-1.02	0.843699415667564\\
-1.01	0.83985412703485\\
-1	0.836003541555136\\
-0.99	0.832154790834778\\
-0.98	0.828304337837145\\
-0.97	0.824453006771933\\
-0.96	0.82060383413953\\
-0.95	0.816753292044355\\
-0.94	0.812906483313147\\
-0.93	0.809060286514763\\
-0.92	0.805214354152797\\
-0.91	0.801372230521166\\
-0.9	0.797531975924658\\
-0.89	0.793695252051139\\
-0.88	0.789861029869021\\
-0.87	0.786028822992401\\
-0.86	0.782198697153663\\
-0.85	0.778370515765889\\
-0.84	0.774544800998226\\
-0.83	0.770724644300721\\
-0.82	0.766906144376525\\
-0.81	0.763086845604182\\
-0.8	0.75927002126654\\
-0.79	0.755456043855029\\
-0.78	0.751638852832077\\
-0.77	0.747825691875918\\
-0.76	0.744009046439898\\
-0.75	0.740196022445988\\
-0.74	0.736379552186278\\
-0.73	0.732562028744555\\
-0.72	0.728740956718849\\
-0.71	0.724915421942409\\
-0.7	0.72109073707123\\
-0.69	0.717257371072258\\
-0.68	0.71342101647563\\
-0.67	0.709579087134559\\
-0.66	0.705729439495764\\
-0.649999999999999	0.7018707133715\\
-0.64	0.698005254436749\\
-0.63	0.694130108286825\\
-0.62	0.690246368159122\\
-0.61	0.686351280782992\\
-0.6	0.682441212019177\\
-0.59	0.678520091342787\\
-0.58	0.674586683724973\\
-0.57	0.670637754430564\\
-0.56	0.66667381947461\\
-0.55	0.662691846643912\\
-0.54	0.658693127432462\\
-0.53	0.654678082178336\\
-0.52	0.65064497596348\\
-0.51	0.646590605848201\\
-0.5	0.642515345140648\\
-0.49	0.638419072654636\\
-0.48	0.634299966677849\\
-0.47	0.630161233528459\\
-0.46	0.625997665674878\\
-0.45	0.621808870463788\\
-0.44	0.617596913488198\\
-0.43	0.613359229889789\\
-0.42	0.609096822816718\\
-0.41	0.604808826739943\\
-0.399999999999999	0.600489411629475\\
-0.39	0.596147211427819\\
-0.38	0.591776819881614\\
-0.37	0.587377000823686\\
-0.36	0.582951307336839\\
-0.35	0.578498211386635\\
-0.34	0.574014137599272\\
-0.33	0.56950108286204\\
-0.32	0.564961873566743\\
-0.31	0.560395192330541\\
-0.3	0.555798047935707\\
-0.29	0.551175929813517\\
-0.28	0.54652300641738\\
-0.27	0.541844872053695\\
-0.26	0.537138160116145\\
-0.25	0.532407183270212\\
-0.24	0.527646462806208\\
-0.23	0.522866294699188\\
-0.22	0.518056856044005\\
-0.21	0.51322510085169\\
-0.2	0.508368581922378\\
-0.19	0.503492640408362\\
-0.18	0.498594438815302\\
-0.17	0.493674721079257\\
-0.16	0.488739129507351\\
-0.149999999999999	0.483786435084123\\
-0.14	0.478815486735044\\
-0.13	0.473828639268718\\
-0.12	0.468829964650347\\
-0.11	0.463818367345442\\
-0.0999999999999996	0.458796822904037\\
-0.0899999999999999	0.453762992683846\\
-0.0800000000000001	0.448725030003111\\
-0.0700000000000003	0.443681379145852\\
-0.0599999999999996	0.43863473574718\\
-0.0499999999999998	0.433585324656064\\
-0.04	0.428535675006909\\
-0.0300000000000002	0.423485706420656\\
-0.0199999999999996	0.418442018618472\\
-0.00999999999999979	0.413407373813416\\
0	0.408377526163242\\
0.00999999999999979	0.403358897508452\\
0.0199999999999996	0.398352556793346\\
0.0300000000000002	0.393365680733175\\
0.04	0.38838929977745\\
0.0499999999999998	0.383435839032462\\
0.0599999999999996	0.378504817028408\\
0.0700000000000003	0.37359878163368\\
0.0800000000000001	0.368717224962426\\
0.0899999999999999	0.363867462112258\\
0.0999999999999996	0.359048799724303\\
0.11	0.354263536782255\\
0.12	0.349513497043499\\
0.13	0.344806480258339\\
0.14	0.340138783073401\\
0.149999999999999	0.335517799826917\\
0.16	0.330939553743643\\
0.17	0.326413346069102\\
0.18	0.321937584558959\\
0.19	0.317514538950209\\
0.2	0.31314986488215\\
0.21	0.308842224459159\\
0.22	0.304601164761047\\
0.23	0.300419411600652\\
0.24	0.29630534374484\\
0.25	0.292258952715865\\
0.26	0.288284746946665\\
0.27	0.284383454049939\\
0.28	0.280554628964704\\
0.29	0.27680366499032\\
0.3	0.27313499369388\\
0.31	0.269547764779617\\
0.32	0.266040053927565\\
0.33	0.26261864791829\\
0.34	0.259286024606235\\
0.35	0.256039713565595\\
0.36	0.252884893744791\\
0.37	0.249823816389471\\
0.38	0.246854146784116\\
0.39	0.24398115391669\\
0.399999999999999	0.241202728129304\\
0.41	0.238524145537805\\
0.42	0.235943428158711\\
0.43	0.23346320938646\\
0.44	0.231084882177423\\
0.45	0.228807884918136\\
0.46	0.226630923373733\\
0.47	0.224563647762885\\
0.48	0.222594341290812\\
0.49	0.220731284431058\\
0.5	0.218977225097367\\
0.51	0.217324553065171\\
0.52	0.215777412456517\\
0.53	0.214339229109286\\
0.54	0.213008115748206\\
0.55	0.211780560888893\\
0.56	0.210659393436732\\
0.57	0.209641708777349\\
0.58	0.208730718086395\\
0.59	0.20792641043999\\
0.6	0.207225631559366\\
0.61	0.206627995343987\\
0.62	0.206132737868096\\
0.63	0.20573784852806\\
0.64	0.205446083306509\\
0.649999999999999	0.205255484665114\\
0.66	0.205163073447363\\
0.67	0.205167036299712\\
0.68	0.205267769174823\\
0.69	0.205464007369832\\
0.7	0.20575249152992\\
0.71	0.206135193061352\\
0.72	0.206606464582148\\
0.73	0.207167126303738\\
0.74	0.207813638091249\\
0.75	0.208546125077007\\
0.76	0.20936005268555\\
0.77	0.210258537992703\\
0.78	0.211232030657074\\
0.79	0.212285059361473\\
0.8	0.213408591849545\\
0.81	0.214610165733128\\
0.82	0.215878022750956\\
0.83	0.217215997295298\\
0.84	0.218617410699778\\
0.85	0.22008411108881\\
0.86	0.221609137111482\\
0.87	0.223193325325329\\
0.88	0.224832362744185\\
0.89	0.226524103640659\\
0.9	0.228269177327905\\
0.91	0.230061835009124\\
0.92	0.231898407759757\\
0.93	0.233777594557065\\
0.94	0.23569703834545\\
0.95	0.237654687877412\\
0.96	0.239650951101477\\
0.97	0.241675869784925\\
0.98	0.243730662138132\\
0.99	0.245813745078758\\
1	0.247922277766395\\
1.01	0.250053700205094\\
1.02	0.252207834348115\\
1.03	0.254375691568844\\
1.04	0.256558730416406\\
1.05	0.258756258915468\\
1.06	0.26096363812119\\
1.07	0.263182587559934\\
1.08	0.26540713435161\\
1.09	0.267635011369795\\
1.1	0.269863067755894\\
1.11	0.272094661995653\\
1.12	0.274321762730577\\
1.13	0.276545961456027\\
1.14	0.278763392407233\\
1.15	0.280975179103226\\
1.16	0.283176589061461\\
1.17	0.285368390794857\\
1.18	0.287547045348056\\
1.19	0.289712697033698\\
1.2	0.29186141615913\\
1.21	0.293997334710533\\
1.22	0.296111820727156\\
1.23	0.298210971139102\\
1.24	0.300287996993461\\
1.25	0.302344710484015\\
1.26	0.304378596706936\\
1.27	0.306392682505562\\
1.28	0.308384138849833\\
1.29	0.310352231129269\\
1.3	0.312294424602257\\
1.31	0.314210689541271\\
1.32	0.316104723642589\\
1.33	0.317974654942577\\
1.34	0.319817827094559\\
1.35	0.321634857448646\\
1.36	0.323428627055263\\
1.37	0.325197696235391\\
1.38	0.326939925652013\\
1.39	0.328658123009774\\
1.4	0.330352319063406\\
1.41	0.332024554496366\\
1.42	0.333672778239639\\
1.43	0.335297362153714\\
1.44	0.336899508148424\\
1.45	0.338483352252938\\
1.46	0.340045355921224\\
1.47	0.341588832607884\\
1.48	0.343115600888902\\
1.49	0.344622930112066\\
1.5	0.346113606284086\\
1.51	0.347594099274371\\
1.52	0.349055337698049\\
1.53	0.350506521298389\\
1.54	0.351947893301642\\
1.55	0.353376412482048\\
1.56	0.354799965443473\\
1.57	0.356215427942555\\
1.58	0.35762857282424\\
1.59	0.359035716697945\\
1.6	0.36044047252604\\
1.61	0.36184351192195\\
1.62	0.363250789714049\\
1.63	0.364658567475261\\
1.64	0.366071159907801\\
1.65	0.367488100048103\\
1.66	0.368912879665643\\
1.67	0.370350421470719\\
1.68	0.371797105516384\\
1.69	0.373255265511454\\
1.7	0.374729551419534\\
1.71	0.376218589347963\\
1.72	0.377723098322948\\
1.73	0.379247875698455\\
1.74	0.380796053939607\\
1.75	0.382363689079535\\
1.76	0.383954438556874\\
1.77	0.38556972756655\\
1.78	0.387213481580378\\
1.79	0.388883981145515\\
1.8	0.390580744605192\\
1.81	0.392312723425038\\
1.82	0.394072993964624\\
1.83	0.395866845671295\\
1.84	0.397691955462332\\
1.85	0.399556650723769\\
1.86	0.401453841507383\\
1.87	0.403385991148336\\
1.88	0.405357292480191\\
1.89	0.407365445096113\\
1.9	0.40941369289554\\
1.91	0.411500797168614\\
1.92	0.413626677923332\\
1.93	0.415792986011334\\
1.94	0.417999571309936\\
1.95	0.420244441063158\\
1.96	0.422533214091386\\
1.97	0.424859125591268\\
1.98	0.427228444681999\\
1.99	0.429634642281254\\
2	0.432084740428891\\
2.01	0.434571949984176\\
2.02	0.4370992303607\\
2.03	0.439662836986087\\
2.04	0.442267114972062\\
2.05	0.444905307740833\\
2.06	0.447581071575862\\
2.07	0.450291641558578\\
2.08	0.45303702146194\\
2.09	0.455815613241796\\
2.1	0.458623230658145\\
2.11	0.461461993630207\\
2.12	0.464329881100705\\
2.13	0.467223019822222\\
2.14	0.470144801165207\\
2.15	0.473088079229747\\
2.16	0.476053226626576\\
2.17	0.479036324394964\\
2.18	0.482036435285177\\
2.19	0.485053417535227\\
2.2	0.488084486964769\\
2.21	0.491123943851019\\
2.22	0.494173638336113\\
2.23	0.497227134950299\\
2.24	0.500285613172821\\
2.25	0.503343478822416\\
2.26	0.506400182817325\\
2.27	0.509448770119525\\
2.28	0.512494733065188\\
2.29	0.515525847319032\\
2.3	0.518545066005457\\
2.31	0.521549068167964\\
2.32	0.524529942386577\\
2.33	0.527490745247072\\
2.34	0.530422796230115\\
2.35	0.533328380827664\\
2.36	0.536199811813196\\
2.37	0.539035318137908\\
2.38	0.54183168008751\\
2.39	0.544586898975068\\
2.4	0.54729383260917\\
2.41	0.549952633139967\\
2.42	0.552561270906036\\
2.43	0.555112322758172\\
2.44	0.557604372321845\\
2.45	0.560032718454871\\
2.46	0.56239630902372\\
2.47	0.564688348647981\\
2.48	0.566908819188711\\
2.49	0.569055243808434\\
2.5	0.571119169109741\\
2.51	0.573103074886242\\
2.52	0.574998013302642\\
2.53	0.576806421403514\\
2.54	0.578521048114268\\
2.55	0.580142526136805\\
2.56	0.581663826923183\\
2.57	0.583084464844159\\
2.58	0.58439869251791\\
2.59	0.585606463681867\\
2.6	0.586703843942131\\
2.61	0.587691234850208\\
2.62	0.588560797377735\\
2.63	0.589312035313738\\
2.64	0.589941572267884\\
2.65	0.590448125499076\\
2.66	0.59083187268107\\
2.67	0.591085694667439\\
2.68	0.591207804023121\\
2.69	0.591202455958016\\
2.7	0.591056641781493\\
2.71	0.590779877724198\\
2.72	0.590364236160234\\
2.73	0.58980826481688\\
2.74	0.589110167724677\\
2.75	0.588271849783011\\
2.76	0.58728815368527\\
2.77	0.586160192903249\\
2.78	0.58488501997826\\
2.79	0.583464026550596\\
2.8	0.58189299973923\\
2.81	0.580172673834835\\
2.82	0.578306079670912\\
2.83	0.576287940497754\\
2.84	0.574119216766926\\
2.85	0.571796720683109\\
2.86	0.569327576227316\\
2.87	0.566703946855268\\
2.88	0.563933004411102\\
2.89	0.561010080278931\\
2.9	0.557937009452716\\
2.91	0.554715260533638\\
2.92	0.551339645380922\\
2.93	0.547820121139628\\
2.94	0.54415317406937\\
2.95	0.540336808412474\\
2.96	0.536375973647318\\
2.97	0.532269744869568\\
2.98	0.528023223316652\\
2.99	0.5236327102974\\
3	0.519103999902094\\
3.01	0.514433338870842\\
3.02	0.509627781775649\\
3.03	0.504688274008283\\
3.04	0.499615464866803\\
3.05	0.494410285587016\\
3.06	0.489078975061128\\
3.07	0.48361956462729\\
3.08	0.478038729822709\\
3.09	0.472333889125484\\
3.1	0.466511401453478\\
3.11	0.460571085790043\\
3.12	0.454517011133879\\
3.13	0.448351727247546\\
3.14	0.442084501123751\\
3.15	0.435707621415449\\
3.16	0.429231319181374\\
3.17	0.422653798871797\\
3.18	0.415980184767857\\
3.19	0.409215749496411\\
3.2	0.402361812938386\\
3.21	0.395420469778743\\
3.22	0.388401488983825\\
3.23	0.381299257235966\\
3.24	0.374124361739071\\
3.25	0.36687649028054\\
3.26	0.359559332641163\\
3.27	0.352180488838541\\
3.28	0.344737966946138\\
3.29	0.337236255305884\\
3.3	0.329683718244569\\
3.31	0.322080353061827\\
3.32	0.31442911137714\\
3.33	0.306736595423591\\
3.34	0.299005528387762\\
3.35	0.291237336171252\\
3.36	0.283438888248584\\
3.37	0.275613085261005\\
3.38	0.267762220381749\\
3.39	0.259889361598399\\
3.4	0.252002214974723\\
3.41	0.244096781550086\\
3.42	0.236185796681154\\
3.43	0.228268202563468\\
3.44	0.220347227788863\\
3.45	0.21242657841075\\
3.46	0.204508252503463\\
3.47	0.196601567651713\\
3.48	0.188702685712028\\
3.49	0.180819845556337\\
3.5	0.172956065534833\\
3.51	0.165110046142672\\
3.52	0.157291545838753\\
3.53	0.14949702663009\\
3.54	0.141737547773522\\
3.55	0.134008526322157\\
3.56	0.12631383371508\\
3.57	0.118662314037671\\
3.58	0.111053638723843\\
3.59	0.103482516688265\\
3.6	0.0959651006471387\\
3.61	0.0884950606920034\\
3.62	0.0810802175816995\\
3.63	0.0737200301698835\\
3.64	0.0664161816858378\\
3.65	0.0591724659440616\\
3.66	0.0519913998951942\\
3.67	0.0448725236231911\\
3.68	0.0378209008638\\
3.69	0.0308397223945284\\
3.7	0.0239257572776376\\
3.71	0.0170874386091773\\
3.72	0.0103209546425565\\
3.73	0.00362787401437649\\
3.74	-0.00298502643340505\\
3.75	-0.00951979876828658\\
3.76	-0.0159753815243343\\
3.77	-0.0223501744564234\\
3.78	-0.0286396350172134\\
3.79	-0.0348435184905625\\
3.8	-0.0409674303451983\\
3.81	-0.0470022976828764\\
3.82	-0.0529488243168927\\
3.83	-0.0588105937557668\\
3.84	-0.0645824812721941\\
3.85	-0.0702632119189364\\
3.86	-0.0758574327249956\\
3.87	-0.0813575838189066\\
3.88	-0.0867675211031086\\
3.89	-0.0920831553786641\\
3.9	-0.097309842165916\\
3.91	-0.102443382502126\\
3.92	-0.107483411956773\\
3.93	-0.112430851841922\\
3.94	-0.117282821520372\\
3.95	-0.122044793479102\\
3.96	-0.126713171109309\\
3.97	-0.131285742177281\\
3.98	-0.135766754629426\\
3.99	-0.140153935985371\\
4	-0.144450216520337\\
4.01	-0.148653956749487\\
4.02	-0.152762824229915\\
4.03	-0.156781316165275\\
4.04	-0.160707928546489\\
4.05	-0.164543873984615\\
4.06	-0.168287687893923\\
4.07	-0.171942103294052\\
4.08	-0.175506871885591\\
4.09	-0.178982425190785\\
4.1	-0.182371971365609\\
4.11	-0.185667441337654\\
4.12	-0.188878676702244\\
4.13	-0.192003744168085\\
4.14	-0.195038921373127\\
4.15	-0.197991472464182\\
4.16	-0.200860503580754\\
4.17	-0.203645581822044\\
4.18	-0.206346058626462\\
4.19	-0.208964588934192\\
4.2	-0.211499187446941\\
4.21	-0.213954486145167\\
4.22	-0.216327288896492\\
4.23	-0.218624872578235\\
4.24	-0.220839870223537\\
4.25	-0.222978046547607\\
4.26	-0.225039674965239\\
4.27	-0.227025551846626\\
4.28	-0.228932250041852\\
4.29	-0.230764237942491\\
4.3	-0.232523930347627\\
4.31	-0.234209362137863\\
4.32	-0.235824190739996\\
4.33	-0.237363266069179\\
4.34	-0.238831012161825\\
4.35	-0.240229260101117\\
4.36	-0.241557731539633\\
4.37	-0.242816729472367\\
4.38	-0.244005421952221\\
4.39	-0.245128561945255\\
4.4	-0.246183511405738\\
4.41	-0.247171779298794\\
4.42	-0.248091817732284\\
4.43	-0.248947844767946\\
4.44	-0.249735581263858\\
4.45	-0.250461702339451\\
4.46	-0.251121192041944\\
4.47	-0.251721529967732\\
4.48	-0.252254444051721\\
4.49	-0.25272622926526\\
4.5	-0.253133147352726\\
4.51	-0.253479641103644\\
4.52	-0.253760974576662\\
4.53	-0.253983484017372\\
4.54	-0.254144584190093\\
4.55	-0.254246732920562\\
4.56	-0.254290040567304\\
4.57	-0.254267415822708\\
4.58	-0.254188087449096\\
4.59	-0.254048832233825\\
4.6	-0.2538473082512\\
4.61	-0.253593976876301\\
4.62	-0.253275066535945\\
4.63	-0.252901348756147\\
4.64	-0.252467037139757\\
4.65	-0.251972823712224\\
4.66	-0.25142069634948\\
4.67	-0.250812393864431\\
4.68	-0.250145816568714\\
4.69	-0.249417737816301\\
4.7	-0.248635011175479\\
4.71	-0.247792959368349\\
4.72	-0.246893207297464\\
4.73	-0.245935509215832\\
4.74	-0.244918565800381\\
4.75	-0.243847025260413\\
4.76	-0.242716541131442\\
4.77	-0.241528882769894\\
4.78	-0.240283170274859\\
4.79	-0.238980959295538\\
4.8	-0.237620908049153\\
4.81	-0.236203273536337\\
4.82	-0.234731891411252\\
4.83	-0.233197206582042\\
4.84	-0.231608555782541\\
4.85	-0.229962905332723\\
4.86	-0.22825928435105\\
4.87	-0.22650084335696\\
4.88	-0.22468522210016\\
4.89	-0.222810675185371\\
4.9	-0.220879907486988\\
4.91	-0.218893650428205\\
4.92	-0.216851656495182\\
4.93	-0.21475444605272\\
4.94	-0.212597235703171\\
4.95	-0.210386259117472\\
4.96	-0.208118154160007\\
4.97	-0.205795928488486\\
4.98	-0.203418778479738\\
4.99	-0.200984641745064\\
5	-0.198496402269436\\
};
\addplot [color=mycolor1, line width=1.3pt, forget plot]
  table[row sep=crcr]{%
-5	0\\
-4.99	0\\
-4.98	0\\
-4.97	0\\
-4.96	0\\
-4.95	0\\
-4.94	0\\
-4.93	0\\
-4.92	0\\
-4.91	0\\
-4.9	0\\
-4.89	0\\
-4.88	0\\
-4.87	0\\
-4.86	0\\
-4.85	0\\
-4.84	0\\
-4.83	0\\
-4.82	0\\
-4.81	0\\
-4.8	0\\
-4.79	0\\
-4.78	0\\
-4.77	0\\
-4.76	0\\
-4.75	0\\
-4.74	0\\
-4.73	0\\
-4.72	0\\
-4.71	0\\
-4.7	0\\
-4.69	0\\
-4.68	0\\
-4.67	0\\
-4.66	0\\
-4.65	0\\
-4.64	0\\
-4.63	0\\
-4.62	0\\
-4.61	0\\
-4.6	0\\
-4.59	0\\
-4.58	0\\
-4.57	0\\
-4.56	0\\
-4.55	0\\
-4.54	0\\
-4.53	0\\
-4.52	0\\
-4.51	0\\
-4.5	0\\
-4.49	0\\
-4.48	0\\
-4.47	0\\
-4.46	0\\
-4.45	0\\
-4.44	0\\
-4.43	0\\
-4.42	0\\
-4.41	0\\
-4.4	0\\
-4.39	0\\
-4.38	0\\
-4.37	0\\
-4.36	0\\
-4.35	0\\
-4.34	0\\
-4.33	0\\
-4.32	0\\
-4.31	0\\
-4.3	0\\
-4.29	0\\
-4.28	0\\
-4.27	0\\
-4.26	0\\
-4.25	0\\
-4.24	0\\
-4.23	0\\
-4.22	0\\
-4.21	0\\
-4.2	0\\
-4.19	0\\
-4.18	0\\
-4.17	0\\
-4.16	0\\
-4.15	0\\
-4.14	0\\
-4.13	0\\
-4.12	0\\
-4.11	0\\
-4.1	0\\
-4.09	0\\
-4.08	0\\
-4.07	0\\
-4.06	0\\
-4.05	0\\
-4.04	0\\
-4.03	0\\
-4.02	0\\
-4.01	0\\
-4	0\\
-3.99	0\\
-3.98	0\\
-3.97	0\\
-3.96	0\\
-3.95	0\\
-3.94	0\\
-3.93	0\\
-3.92	0\\
-3.91	0\\
-3.9	0\\
-3.89	0\\
-3.88	0\\
-3.87	0\\
-3.86	0\\
-3.85	0\\
-3.84	0\\
-3.83	0\\
-3.82	0\\
-3.81	0\\
-3.8	0\\
-3.79	0\\
-3.78	0\\
-3.77	0\\
-3.76	0\\
-3.75	0\\
-3.74	0\\
-3.73	0\\
-3.72	0\\
-3.71	0\\
-3.7	0\\
-3.69	0\\
-3.68	0\\
-3.67	0\\
-3.66	0\\
-3.65	0\\
-3.64	0\\
-3.63	0\\
-3.62	0\\
-3.61	0\\
-3.6	0\\
-3.59	0\\
-3.58	0\\
-3.57	0\\
-3.56	0\\
-3.55	0\\
-3.54	0\\
-3.53	0\\
-3.52	0\\
-3.51	0\\
-3.5	0\\
-3.49	0\\
-3.48	0\\
-3.47	0\\
-3.46	0\\
-3.45	0\\
-3.44	0\\
-3.43	0\\
-3.42	0\\
-3.41	0\\
-3.4	0\\
-3.39	0\\
-3.38	0\\
-3.37	0\\
-3.36	0\\
-3.35	0\\
-3.34	0\\
-3.33	0\\
-3.32	0\\
-3.31	0\\
-3.3	0\\
-3.29	0\\
-3.28	0\\
-3.27	0\\
-3.26	0\\
-3.25	0\\
-3.24	0\\
-3.23	0\\
-3.22	0\\
-3.21	0\\
-3.2	0\\
-3.19	0\\
-3.18	0\\
-3.17	0\\
-3.16	0\\
-3.15	0\\
-3.14	0\\
-3.13	0\\
-3.12	0\\
-3.11	0\\
-3.1	0\\
-3.09	0\\
-3.08	0\\
-3.07	0\\
-3.06	0\\
-3.05	0\\
-3.04	0\\
-3.03	0\\
-3.02	0\\
-3.01	0\\
-3	0\\
-2.99	0\\
-2.98	0\\
-2.97	0\\
-2.96	0\\
-2.95	0\\
-2.94	0\\
-2.93	0\\
-2.92	0\\
-2.91	0\\
-2.9	0\\
-2.89	0\\
-2.88	0\\
-2.87	0\\
-2.86	0\\
-2.85	0\\
-2.84	0\\
-2.83	0\\
-2.82	0\\
-2.81	0\\
-2.8	0\\
-2.79	0\\
-2.78	0\\
-2.77	0\\
-2.76	0\\
-2.75	0\\
-2.74	0\\
-2.73	0\\
-2.72	0\\
-2.71	0\\
-2.7	0\\
-2.69	0\\
-2.68	0\\
-2.67	0\\
-2.66	0\\
-2.65	0\\
-2.64	0\\
-2.63	0\\
-2.62	0\\
-2.61	0\\
-2.6	0\\
-2.59	0\\
-2.58	0\\
-2.57	0\\
-2.56	0\\
-2.55	0\\
-2.54	0\\
-2.53	0\\
-2.52	0\\
-2.51	0\\
-2.5	0\\
-2.49	0\\
-2.48	0\\
-2.47	0\\
-2.46	0\\
-2.45	0\\
-2.44	0\\
-2.43	0\\
-2.42	0\\
-2.41	0\\
-2.4	0\\
-2.39	0\\
-2.38	0\\
-2.37	0\\
-2.36	0\\
-2.35	0\\
-2.34	0\\
-2.33	0\\
-2.32	0\\
-2.31	0\\
-2.3	0\\
-2.29	0\\
-2.28	0\\
-2.27	0\\
-2.26	0\\
-2.25	0\\
-2.24	0\\
-2.23	0\\
-2.22	0\\
-2.21	0\\
-2.2	0\\
-2.19	0\\
-2.18	0\\
-2.17	0\\
-2.16	0\\
-2.15	0\\
-2.14	0\\
-2.13	0\\
-2.12	0\\
-2.11	0\\
-2.1	0\\
-2.09	0\\
-2.08	0\\
-2.07	0\\
-2.06	0\\
-2.05	0\\
-2.04	0\\
-2.03	0\\
-2.02	0\\
-2.01	0\\
-2	0\\
-1.99	0\\
-1.98	0\\
-1.97	0\\
-1.96	0\\
-1.95	0\\
-1.94	0\\
-1.93	0\\
-1.92	0\\
-1.91	0\\
-1.9	0\\
-1.89	0\\
-1.88	0\\
-1.87	0\\
-1.86	0\\
-1.85	0\\
-1.84	0\\
-1.83	0\\
-1.82	0\\
-1.81	0\\
-1.8	0\\
-1.79	0\\
-1.78	0\\
-1.77	0\\
-1.76	0\\
-1.75	0\\
-1.74	0\\
-1.73	0\\
-1.72	0\\
-1.71	0\\
-1.7	0\\
-1.69	0\\
-1.68	0\\
-1.67	0\\
-1.66	0\\
-1.65	0\\
-1.64	0\\
-1.63	0\\
-1.62	0\\
-1.61	0\\
-1.6	0\\
-1.59	0\\
-1.58	0\\
-1.57	0\\
-1.56	0\\
-1.55	0\\
-1.54	0\\
-1.53	0\\
-1.52	0\\
-1.51	0\\
-1.5	0\\
-1.49	0\\
-1.48	0\\
-1.47	0\\
-1.46	0\\
-1.45	0\\
-1.44	0\\
-1.43	0\\
-1.42	0\\
-1.41	0\\
-1.4	0\\
-1.39	0\\
-1.38	0\\
-1.37	0\\
-1.36	0\\
-1.35	0\\
-1.34	0\\
-1.33	0\\
-1.32	0\\
-1.31	0\\
-1.3	0\\
-1.29	0\\
-1.28	0\\
-1.27	0\\
-1.26	0\\
-1.25	0\\
-1.24	0\\
-1.23	0\\
-1.22	0\\
-1.21	0\\
-1.2	0\\
-1.19	0\\
-1.18	0\\
-1.17	0\\
-1.16	0\\
-1.15	0\\
-1.14	0\\
-1.13	0\\
-1.12	0\\
-1.11	0\\
-1.1	0\\
-1.09	0\\
-1.08	0\\
-1.07	0\\
-1.06	0\\
-1.05	0\\
-1.04	0\\
-1.03	0\\
-1.02	0\\
-1.01	0\\
-1	0\\
-0.99	0\\
-0.98	0\\
-0.97	0\\
-0.96	0\\
-0.95	0\\
-0.94	0\\
-0.93	0\\
-0.92	0\\
-0.91	0\\
-0.9	0\\
-0.89	0\\
-0.88	0\\
-0.87	0\\
-0.86	0\\
-0.85	0\\
-0.84	0\\
-0.83	0\\
-0.82	0\\
-0.81	0\\
-0.8	0\\
-0.79	0\\
-0.78	0\\
-0.77	0\\
-0.76	0\\
-0.75	0\\
-0.74	0\\
-0.73	0\\
-0.72	0\\
-0.71	0\\
-0.7	0\\
-0.69	0\\
-0.68	0\\
-0.67	0\\
-0.66	0\\
-0.649999999999999	0\\
-0.64	0\\
-0.63	0\\
-0.62	0\\
-0.61	0\\
-0.6	0\\
-0.59	0\\
-0.58	0\\
-0.57	0\\
-0.56	0\\
-0.55	0\\
-0.54	0\\
-0.53	0\\
-0.52	0\\
-0.51	0\\
-0.5	0\\
-0.49	0\\
-0.48	0\\
-0.47	0\\
-0.46	0\\
-0.45	0\\
-0.44	0\\
-0.43	0\\
-0.42	0\\
-0.41	0\\
-0.399999999999999	0\\
-0.39	0\\
-0.38	0\\
-0.37	0\\
-0.36	0\\
-0.35	0\\
-0.34	0\\
-0.33	0\\
-0.32	0\\
-0.31	0\\
-0.3	0\\
-0.29	0\\
-0.28	0\\
-0.27	0\\
-0.26	0\\
-0.25	0\\
-0.24	0\\
-0.23	0\\
-0.22	0\\
-0.21	0\\
-0.2	0\\
-0.19	0\\
-0.18	0\\
-0.17	0\\
-0.16	0\\
-0.149999999999999	0\\
-0.14	0\\
-0.13	0\\
-0.12	0\\
-0.11	0\\
-0.0999999999999996	0\\
-0.0899999999999999	0\\
-0.0800000000000001	0\\
-0.0700000000000003	0\\
-0.0599999999999996	0\\
-0.0499999999999998	0\\
-0.04	0\\
-0.0300000000000002	0\\
-0.0199999999999996	0\\
-0.00999999999999979	0\\
0	0\\
0.00999999999999979	0\\
0.0199999999999996	0\\
0.0300000000000002	0\\
0.04	0\\
0.0499999999999998	0\\
0.0599999999999996	0\\
0.0700000000000003	0\\
0.0800000000000001	0\\
0.0899999999999999	0\\
0.0999999999999996	0\\
0.11	0\\
0.12	0\\
0.13	0\\
0.14	0\\
0.149999999999999	0\\
0.16	0\\
0.17	0\\
0.18	0\\
0.19	0\\
0.2	0\\
0.21	0\\
0.22	0\\
0.23	0\\
0.24	0\\
0.25	0\\
0.26	0\\
0.27	0\\
0.28	0\\
0.29	0\\
0.3	0\\
0.31	0\\
0.32	0\\
0.33	0\\
0.34	0\\
0.35	0\\
0.36	0\\
0.37	0\\
0.38	0\\
0.39	0\\
0.399999999999999	0\\
0.41	0\\
0.42	0\\
0.43	0\\
0.44	0\\
0.45	0\\
0.46	0\\
0.47	0\\
0.48	0\\
0.49	0\\
0.5	0\\
0.51	0\\
0.52	0\\
0.53	0\\
0.54	0\\
0.55	0\\
0.56	0\\
0.57	0\\
0.58	0\\
0.59	0\\
0.6	0\\
0.61	0\\
0.62	0\\
0.63	0\\
0.64	0\\
0.649999999999999	0\\
0.66	0\\
0.67	0\\
0.68	0\\
0.69	0\\
0.7	0\\
0.71	0\\
0.72	0\\
0.73	0\\
0.74	0\\
0.75	0\\
0.76	0\\
0.77	0\\
0.78	0\\
0.79	0\\
0.8	0\\
0.81	0\\
0.82	0\\
0.83	0\\
0.84	0\\
0.85	0\\
0.86	0\\
0.87	0\\
0.88	0\\
0.89	0\\
0.9	0\\
0.91	0\\
0.92	0\\
0.93	0\\
0.94	0\\
0.95	0\\
0.96	0\\
0.97	0\\
0.98	0\\
0.99	0\\
1	0\\
1.01	0\\
1.02	0\\
1.03	0\\
1.04	0\\
1.05	0\\
1.06	0\\
1.07	0\\
1.08	0\\
1.09	0\\
1.1	0\\
1.11	0\\
1.12	0\\
1.13	0\\
1.14	0\\
1.15	0\\
1.16	0\\
1.17	0\\
1.18	0\\
1.19	0\\
1.2	0\\
1.21	0\\
1.22	0\\
1.23	0\\
1.24	0\\
1.25	0\\
1.26	0\\
1.27	0\\
1.28	0\\
1.29	0\\
1.3	0\\
1.31	0\\
1.32	0\\
1.33	0\\
1.34	0\\
1.35	0\\
1.36	0\\
1.37	0\\
1.38	0\\
1.39	0\\
1.4	0\\
1.41	0\\
1.42	0\\
1.43	0\\
1.44	0\\
1.45	0\\
1.46	0\\
1.47	0\\
1.48	0\\
1.49	0\\
1.5	0\\
1.51	0\\
1.52	0\\
1.53	0\\
1.54	0\\
1.55	0\\
1.56	0\\
1.57	0\\
1.58	0\\
1.59	0\\
1.6	0\\
1.61	0\\
1.62	0\\
1.63	0\\
1.64	0\\
1.65	0\\
1.66	0\\
1.67	0\\
1.68	0\\
1.69	0\\
1.7	0\\
1.71	0\\
1.72	0\\
1.73	0\\
1.74	0\\
1.75	0\\
1.76	0\\
1.77	0\\
1.78	0\\
1.79	0\\
1.8	0\\
1.81	0\\
1.82	0\\
1.83	0\\
1.84	0\\
1.85	0\\
1.86	0\\
1.87	0\\
1.88	0\\
1.89	0\\
1.9	0\\
1.91	0\\
1.92	0\\
1.93	0\\
1.94	0\\
1.95	0\\
1.96	0\\
1.97	0\\
1.98	0\\
1.99	0\\
2	0\\
2.01	0\\
2.02	0\\
2.03	0\\
2.04	0\\
2.05	0\\
2.06	0\\
2.07	0\\
2.08	0\\
2.09	0\\
2.1	0\\
2.11	0\\
2.12	0\\
2.13	0\\
2.14	0\\
2.15	0\\
2.16	0\\
2.17	0\\
2.18	0\\
2.19	0\\
2.2	0\\
2.21	0\\
2.22	0\\
2.23	0\\
2.24	0\\
2.25	0\\
2.26	0\\
2.27	0\\
2.28	0\\
2.29	0\\
2.3	0\\
2.31	0\\
2.32	0\\
2.33	0\\
2.34	0\\
2.35	0\\
2.36	0\\
2.37	0\\
2.38	0\\
2.39	0\\
2.4	0\\
2.41	0\\
2.42	0\\
2.43	0\\
2.44	0\\
2.45	0\\
2.46	0\\
2.47	0\\
2.48	0\\
2.49	0\\
2.5	0\\
2.51	0\\
2.52	0\\
2.53	0\\
2.54	0\\
2.55	0\\
2.56	0\\
2.57	0\\
2.58	0\\
2.59	0\\
2.6	0\\
2.61	0\\
2.62	0\\
2.63	0\\
2.64	0\\
2.65	0\\
2.66	0\\
2.67	0\\
2.68	0\\
2.69	0\\
2.7	0\\
2.71	0\\
2.72	0\\
2.73	0\\
2.74	0\\
2.75	0\\
2.76	0\\
2.77	0\\
2.78	0\\
2.79	0\\
2.8	0\\
2.81	0\\
2.82	0\\
2.83	0\\
2.84	0\\
2.85	0\\
2.86	0\\
2.87	0\\
2.88	0\\
2.89	0\\
2.9	0\\
2.91	0\\
2.92	0\\
2.93	0\\
2.94	0\\
2.95	0\\
2.96	0\\
2.97	0\\
2.98	0\\
2.99	0\\
3	0\\
3.01	0\\
3.02	0\\
3.03	0\\
3.04	0\\
3.05	0\\
3.06	0\\
3.07	0\\
3.08	0\\
3.09	0\\
3.1	0\\
3.11	0\\
3.12	0\\
3.13	0\\
3.14	0\\
3.15	0\\
3.16	0\\
3.17	0\\
3.18	0\\
3.19	0\\
3.2	0\\
3.21	0\\
3.22	0\\
3.23	0\\
3.24	0\\
3.25	0\\
3.26	0\\
3.27	0\\
3.28	0\\
3.29	0\\
3.3	0\\
3.31	0\\
3.32	0\\
3.33	0\\
3.34	0\\
3.35	0\\
3.36	0\\
3.37	0\\
3.38	0\\
3.39	0\\
3.4	0\\
3.41	0\\
3.42	0\\
3.43	0\\
3.44	0\\
3.45	0\\
3.46	0\\
3.47	0\\
3.48	0\\
3.49	0\\
3.5	0\\
3.51	0\\
3.52	0\\
3.53	0\\
3.54	0\\
3.55	0\\
3.56	0\\
3.57	0\\
3.58	0\\
3.59	0\\
3.6	0\\
3.61	0\\
3.62	0\\
3.63	0\\
3.64	0\\
3.65	0\\
3.66	0\\
3.67	0\\
3.68	0\\
3.69	0\\
3.7	0\\
3.71	0\\
3.72	0\\
3.73	0\\
3.74	0\\
3.75	0\\
3.76	0\\
3.77	0\\
3.78	0\\
3.79	0\\
3.8	0\\
3.81	0\\
3.82	0\\
3.83	0\\
3.84	0\\
3.85	0\\
3.86	0\\
3.87	0\\
3.88	0\\
3.89	0\\
3.9	0\\
3.91	0\\
3.92	0\\
3.93	0\\
3.94	0\\
3.95	0\\
3.96	0\\
3.97	0\\
3.98	0\\
3.99	0\\
4	0\\
4.01	0\\
4.02	0\\
4.03	0\\
4.04	0\\
4.05	0\\
4.06	0\\
4.07	0\\
4.08	0\\
4.09	0\\
4.1	0\\
4.11	0\\
4.12	0\\
4.13	0\\
4.14	0\\
4.15	0\\
4.16	0\\
4.17	0\\
4.18	0\\
4.19	0\\
4.2	0\\
4.21	0\\
4.22	0\\
4.23	0\\
4.24	0\\
4.25	0\\
4.26	0\\
4.27	0\\
4.28	0\\
4.29	0\\
4.3	0\\
4.31	0\\
4.32	0\\
4.33	0\\
4.34	0\\
4.35	0\\
4.36	0\\
4.37	0\\
4.38	0\\
4.39	0\\
4.4	0\\
4.41	0\\
4.42	0\\
4.43	0\\
4.44	0\\
4.45	0\\
4.46	0\\
4.47	0\\
4.48	0\\
4.49	0\\
4.5	0\\
4.51	0\\
4.52	0\\
4.53	0\\
4.54	0\\
4.55	0\\
4.56	0\\
4.57	0\\
4.58	0\\
4.59	0\\
4.6	0\\
4.61	0\\
4.62	0\\
4.63	0\\
4.64	0\\
4.65	0\\
4.66	0\\
4.67	0\\
4.68	0\\
4.69	0\\
4.7	0\\
4.71	0\\
4.72	0\\
4.73	0\\
4.74	0\\
4.75	0\\
4.76	0\\
4.77	0\\
4.78	0\\
4.79	0\\
4.8	0\\
4.81	0\\
4.82	0\\
4.83	0\\
4.84	0\\
4.85	0\\
4.86	0\\
4.87	0\\
4.88	0\\
4.89	0\\
4.9	0\\
4.91	0\\
4.92	0\\
4.93	0\\
4.94	0\\
4.95	0\\
4.96	0\\
4.97	0\\
4.98	0\\
4.99	0\\
5	0\\
};
\addplot [color=mycolor2, dashed, line width=1.3pt, forget plot]
  table[row sep=crcr]{%
-5	0.770472187846317\\
-4.99	0.774313119954837\\
-4.98	0.778061676413651\\
-4.97	0.781717566593223\\
-4.96	0.785280509431497\\
-4.95	0.788750233460293\\
-4.94	0.792126476830734\\
-4.93	0.79540898733771\\
-4.92	0.798597522443363\\
-4.91	0.801691849299611\\
-4.9	0.804691744769688\\
-4.89	0.807596995448709\\
-4.88	0.810407397683262\\
-4.87	0.813122757590007\\
-4.86	0.815742891073304\\
-4.85	0.818267623841853\\
-4.84	0.820696791424341\\
-4.83	0.823030239184112\\
-4.82	0.825267822332842\\
-4.81	0.827409405943223\\
-4.8	0.829454864960657\\
-4.79	0.831404084213958\\
-4.78	0.83325695842506\\
-4.77	0.835013392217728\\
-4.76	0.836673300125273\\
-4.75	0.838236606597277\\
-4.74	0.839703246005315\\
-4.73	0.841073162647678\\
-4.72	0.842346310753106\\
-4.71	0.843522654483514\\
-4.7	0.844602167935725\\
-4.69	0.845584835142204\\
-4.68	0.84647065007079\\
-4.67	0.847259616623431\\
-4.66	0.847951748633924\\
-4.65	0.848547069864651\\
-4.64	0.849045614002323\\
-4.63	0.849447424652721\\
-4.62	0.849752555334442\\
-4.61	0.849961069471655\\
-4.6	0.850073040385849\\
-4.59	0.8500885512866\\
-4.58	0.850007695261338\\
-4.57	0.849830575264117\\
-4.56	0.849557304103404\\
-4.55	0.849188004428869\\
-4.54	0.848722808717192\\
-4.53	0.848161859256879\\
-4.52	0.847505308132096\\
-4.51	0.846753317205515\\
-4.5	0.845906058100179\\
-4.49	0.844963712180391\\
-4.48	0.843926470531616\\
-4.47	0.842794533939416\\
-4.46	0.841568112867402\\
-4.45	0.840247427434221\\
-4.44	0.838832707389569\\
-4.43	0.837324192089238\\
-4.42	0.835722130469199\\
-4.41	0.834026781018723\\
-4.4	0.832238411752541\\
-4.39	0.830357300182049\\
-4.38	0.828383733285557\\
-4.37	0.826318007477594\\
-4.36	0.824160428577256\\
-4.35	0.821911311775617\\
-4.34	0.819570981602196\\
-4.33	0.817139771890486\\
-4.32	0.814618025742554\\
-4.31	0.812006095492701\\
-4.3	0.809304342670203\\
-4.29	0.806513137961127\\
-4.28	0.803632861169229\\
-4.27	0.800663901175928\\
-4.26	0.797606655899387\\
-4.25	0.794461532252664\\
-4.24	0.791228946100982\\
-4.23	0.787909322218082\\
-4.22	0.784503094241697\\
-4.21	0.781010704628124\\
-4.2	0.777432604605923\\
-4.19	0.773769254128723\\
-4.18	0.77002112182716\\
-4.17	0.766188684959946\\
-4.16	0.762272429364066\\
-4.15	0.758272849404117\\
-4.14	0.754190447920794\\
-4.13	0.75002573617852\\
-4.12	0.745779233812235\\
-4.11	0.741451468773344\\
-4.1	0.737042977274829\\
-4.09	0.732554303735537\\
-4.08	0.727986000723637\\
-4.07	0.723338628899264\\
-4.06	0.718612756956355\\
-4.05	0.713808961563671\\
-4.04	0.708927827305024\\
-4.03	0.703969946618709\\
-4.02	0.698935919736146\\
-4.01	0.693826354619749\\
-4	0.688641866900002\\
-3.99	0.683383079811788\\
-3.98	0.678050624129934\\
-3.97	0.67264513810402\\
-3.96	0.667167267392418\\
-3.95	0.661617664995605\\
-3.94	0.65599699118873\\
-3.93	0.650305913453457\\
-3.92	0.644545106409079\\
-3.91	0.638715251742921\\
-3.9	0.632817038140036\\
-3.89	0.626851161212192\\
-3.88	0.620818323426173\\
-3.87	0.614719234031388\\
-3.86	0.608554608986798\\
-3.85	0.602325170887175\\
-3.84	0.596031648888697\\
-3.83	0.589674778633876\\
-3.82	0.583255302175845\\
-3.81	0.576773967902\\
-3.8	0.570231530457006\\
-3.79	0.563628750665175\\
-3.78	0.556966395452227\\
-3.77	0.550245237766437\\
-3.76	0.543466056499182\\
-3.75	0.536629636404888\\
-3.74	0.529736768020393\\
-3.73	0.522788247583729\\
-3.72	0.515784876952331\\
-3.71	0.50872746352069\\
-3.7	0.501616820137441\\
-3.69	0.494453765021909\\
-3.68	0.487239121680127\\
-3.67	0.479973718820303\\
-3.66	0.472658390267786\\
-3.65	0.465293974879508\\
-3.64	0.457881316457928\\
-3.63	0.450421263664478\\
-3.62	0.442914669932518\\
-3.61	0.435362393379825\\
-3.6	0.427765296720601\\
-3.59	0.420124247177028\\
-3.58	0.412440116390366\\
-3.57	0.404713780331616\\
-3.56	0.396946119211742\\
-3.55	0.389138017391476\\
-3.54	0.381290363290698\\
-3.53	0.373404049297422\\
-3.52	0.365479971676372\\
-3.51	0.357519030477184\\
-3.5	0.349522129442213\\
-3.49	0.341490175913989\\
-3.48	0.333424080742294\\
-3.47	0.325324758190907\\
-3.46	0.31719312584399\\
-3.45	0.309030104512155\\
-3.44	0.300836618138203\\
-3.43	0.29261359370255\\
-3.42	0.284361961128351\\
-3.41	0.276082653186326\\
-3.4	0.267776605399309\\
-3.39	0.259444755946514\\
-3.38	0.251088045567541\\
-3.37	0.242707417466127\\
-3.36	0.234303817213641\\
-3.35	0.22587819265236\\
-3.34	0.217431493798504\\
-3.33	0.208964672745057\\
-3.32	0.200478683564389\\
-3.31	0.191974482210669\\
-3.3	0.183453026422099\\
-3.29	0.174915275622969\\
-3.28	0.166362190825541\\
-3.27	0.157794734531774\\
-3.26	0.149213870634911\\
-3.25	0.140620564320915\\
-3.24	0.132015781969783\\
-3.23	0.123400491056743\\
-3.22	0.114775660053337\\
-3.21	0.106142258328413\\
-3.2	0.0975012560490237\\
-3.19	0.0888536240812423\\
-3.18	0.0802003338909209\\
-3.17	0.0715423574443809\\
-3.16	0.0628806671090572\\
-3.15	0.0542162355541045\\
-3.14	0.0455500356509757\\
-3.13	0.0368830403739791\\
-3.12	0.0282162227008277\\
-3.11	0.0195505555131899\\
-3.1	0.010887011497253\\
-3.09	0.0022265630443051\\
-3.08	-0.00642981784865092\\
-3.07	-0.0150811596782414\\
-3.06	-0.0237264915340156\\
-3.05	-0.0323648431976774\\
-3.04	-0.0409952452422467\\
-3.03	-0.049616729131141\\
-3.02	-0.0582283273171638\\
-3.01	-0.0668290733413922\\
-3	-0.0754180019319548\\
-2.99	-0.0839941491026903\\
-2.98	-0.0925565522516706\\
-2.97	-0.101104250259589\\
-2.96	-0.109636283587994\\
-2.95	-0.118151694377362\\
-2.94	-0.126649526545009\\
-2.93	-0.135128825882804\\
-2.92	-0.143588640154715\\
-2.91	-0.152028019194136\\
-2.9	-0.160446015001018\\
-2.89	-0.168841681838773\\
-2.88	-0.17721407633095\\
-2.87	-0.18556225755768\\
-2.86	-0.193885287151861\\
-2.85	-0.202182229395094\\
-2.84	-0.210452151313347\\
-2.83	-0.218694122772343\\
-2.82	-0.226907216572665\\
-2.81	-0.235090508544555\\
-2.8	-0.243243077642418\\
-2.79	-0.251364006039005\\
-2.78	-0.25945237921927\\
-2.77	-0.267507286073898\\
-2.76	-0.275527818992485\\
-2.75	-0.283513073956363\\
-2.74	-0.291462150631074\\
-2.73	-0.299374152458456\\
-2.72	-0.307248186748362\\
-2.71	-0.315083364769982\\
-2.7	-0.322878801842765\\
-2.69	-0.33063361742694\\
-2.68	-0.33834693521361\\
-2.67	-0.346017883214427\\
-2.66	-0.353645593850827\\
-2.65	-0.36122920404282\\
-2.64	-0.368767855297335\\
-2.63	-0.376260693796091\\
-2.62	-0.383706870483001\\
-2.61	-0.391105541151104\\
-2.6	-0.398455866528995\\
-2.59	-0.40575701236677\\
-2.58	-0.413008149521455\\
-2.57	-0.420208454041928\\
-2.56	-0.427357107253313\\
-2.55	-0.434453295840847\\
-2.54	-0.441496211933203\\
-2.53	-0.448485053185264\\
-2.52	-0.455419022860346\\
-2.51	-0.462297329911854\\
-2.5	-0.469119189064357\\
-2.49	-0.475883820894094\\
-2.48	-0.482590451908882\\
-2.47	-0.489238314627434\\
-2.46	-0.495826647658058\\
-2.45	-0.502354695776764\\
-2.44	-0.508821710004726\\
-2.43	-0.515226947685135\\
-2.42	-0.521569672559406\\
-2.41	-0.527849154842738\\
-2.4	-0.534064671299029\\
-2.39	-0.54021550531513\\
-2.38	-0.546300946974427\\
-2.37	-0.552320293129757\\
-2.36	-0.558272847475634\\
-2.35	-0.564157920619793\\
-2.34	-0.569974830154033\\
-2.33	-0.575722900724359\\
-2.32	-0.581401464100416\\
-2.31	-0.5870098592442\\
-2.3	-0.592547432378051\\
-2.29	-0.598013537051911\\
-2.28	-0.603407534209844\\
-2.27	-0.608728792255818\\
-2.26	-0.613976687118726\\
-2.25	-0.619150602316658\\
-2.24	-0.624249929020402\\
-2.23	-0.629274066116182\\
-2.22	-0.634222420267615\\
-2.21	-0.639094405976881\\
-2.2	-0.643889445645114\\
-2.19	-0.648606969631988\\
-2.18	-0.653246416314507\\
-2.17	-0.657807232144989\\
-2.16	-0.662288871708231\\
-2.15	-0.666690797777864\\
-2.14	-0.671012481371872\\
-2.13	-0.675253401807295\\
-2.12	-0.679413046754084\\
-2.11	-0.683490912288125\\
-2.1	-0.687486502943406\\
-2.09	-0.691399331763347\\
-2.08	-0.695228920351257\\
-2.07	-0.698974798919944\\
-2.06	-0.702636506340446\\
-2.05	-0.706213590189901\\
-2.04	-0.709705606798532\\
-2.03	-0.713112121295757\\
-2.02	-0.716432707655407\\
-2.01	-0.719666948740062\\
-2	-0.722814436344482\\
-1.99	-0.725874771238151\\
-1.98	-0.728847563206901\\
-1.97	-0.731732431093648\\
-1.96	-0.734529002838197\\
-1.95	-0.737236915516147\\
-1.94	-0.739855815376865\\
-1.93	-0.742385357880544\\
-1.92	-0.744825207734325\\
-1.91	-0.747175038927502\\
-1.9	-0.749434534765774\\
-1.89	-0.751603387904579\\
-1.88	-0.753681300381466\\
-1.87	-0.755667983647541\\
-1.86	-0.757563158597955\\
-1.85	-0.759366555601439\\
-1.84	-0.761077914528896\\
-1.83	-0.762696984781023\\
-1.82	-0.764223525314984\\
-1.81	-0.765657304670117\\
-1.8	-0.766998100992671\\
-1.79	-0.768245702059586\\
-1.78	-0.769399905301295\\
-1.77	-0.770460517823554\\
-1.76	-0.771427356428306\\
-1.75	-0.772300247633556\\
-1.74	-0.77307902769228\\
-1.73	-0.773763542610346\\
-1.72	-0.77435364816345\\
-1.71	-0.774849209913081\\
-1.7	-0.775250103221487\\
-1.69	-0.775556213265658\\
-1.68	-0.775767435050325\\
-1.67	-0.775883673419965\\
-1.66	-0.775904843069814\\
-1.65	-0.77583086855589\\
-1.64	-0.775661684304023\\
-1.63	-0.775397234617891\\
-1.62	-0.775037473686056\\
-1.61	-0.774582365588014\\
-1.6	-0.774031884299237\\
-1.59	-0.773386013695228\\
-1.58	-0.772644747554574\\
-1.57	-0.771808089561002\\
-1.56	-0.770876053304435\\
-1.55	-0.769848662281054\\
-1.54	-0.768725949892361\\
-1.53	-0.767507959443241\\
-1.52	-0.766194744139028\\
-1.51	-0.764786367081573\\
-1.5	-0.76328290126432\\
-1.49	-0.761684429566375\\
-1.48	-0.759991044745585\\
-1.47	-0.758202849430624\\
-1.46	-0.756319956112079\\
-1.45	-0.754342487132544\\
-1.44	-0.752270574675723\\
-1.43	-0.750104360754534\\
-1.42	-0.747843997198237\\
-1.41	-0.745489645638554\\
-1.4	-0.743041477494815\\
-1.39	-0.740499673958112\\
-1.38	-0.737864425974467\\
-1.37	-0.735135934227021\\
-1.36	-0.732314409117232\\
-1.35	-0.729400070745109\\
-1.34	-0.726393148888449\\
-1.33	-0.723293882981115\\
-1.32	-0.720102522090329\\
-1.31	-0.716819324893\\
-1.3	-0.713444559651082\\
-1.29	-0.709978504185959\\
-1.28	-0.706421445851877\\
-1.27	-0.702773681508405\\
-1.26	-0.699035517491942\\
-1.25	-0.695207269586266\\
-1.24	-0.691289262992133\\
-1.23	-0.687281832295923\\
-1.22	-0.683185321437339\\
-1.21	-0.679000083676164\\
-1.2	-0.674726481558079\\
-1.19	-0.670364886879541\\
-1.18	-0.665915680651727\\
-1.17	-0.661379253063552\\
-1.16	-0.656756003443761\\
-1.15	-0.652046340222088\\
-1.14	-0.647250680889515\\
-1.13	-0.642369451957599\\
-1.12	-0.637403088916898\\
-1.11	-0.632352036194487\\
-1.1	-0.627216747110574\\
-1.09	-0.621997683834216\\
-1.08	-0.616695317338145\\
-1.07	-0.611310127352695\\
-1.06	-0.605842602318863\\
-1.05	-0.60029323934047\\
-1.04	-0.594662544135461\\
-1.03	-0.58895103098633\\
-1.02	-0.583159222689677\\
-1.01	-0.57728765050491\\
-1	-0.571336854102093\\
-0.99	-0.565307381508935\\
-0.98	-0.559199789056942\\
-0.97	-0.553014641326733\\
-0.96	-0.54675251109251\\
-0.95	-0.540413979265713\\
-0.94	-0.533999634837844\\
-0.93	-0.527510074822477\\
-0.92	-0.520945904196457\\
-0.91	-0.514307735840293\\
-0.9	-0.507596190477751\\
-0.89	-0.500811896614657\\
-0.88	-0.493955490476909\\
-0.87	-0.487027615947709\\
-0.86	-0.480028924504021\\
-0.85	-0.47296007515226\\
-0.84	-0.465821734363222\\
-0.83	-0.458614576006252\\
-0.82	-0.451339281282667\\
-0.81	-0.44399653865844\\
-0.8	-0.436587043796142\\
-0.79	-0.429111499486159\\
-0.78	-0.421570615577184\\
-0.77	-0.413965108905999\\
-0.76	-0.406295703226548\\
-0.75	-0.3985631291383\\
-0.74	-0.390768124013933\\
-0.73	-0.382911431926317\\
-0.72	-0.374993803574834\\
-0.71	-0.367015996211005\\
-0.7	-0.358978773563475\\
-0.69	-0.350882905762322\\
-0.68	-0.342729169262729\\
-0.67	-0.334518346768011\\
-0.66	-0.326251227152002\\
-0.649999999999999	-0.317928605380824\\
-0.64	-0.309551282434034\\
-0.63	-0.301120065225159\\
-0.62	-0.292635766521626\\
-0.61	-0.284099204864108\\
-0.6	-0.275511204485268\\
-0.59	-0.266872595227938\\
-0.58	-0.258184212462718\\
-0.57	-0.249446897005017\\
-0.56	-0.24066149503154\\
-0.55	-0.231828857996231\\
-0.54	-0.222949842545673\\
-0.53	-0.214025310433963\\
-0.52	-0.205056128437071\\
-0.51	-0.196043168266679\\
-0.5	-0.18698730648352\\
-0.49	-0.177889424410228\\
-0.48	-0.168750408043694\\
-0.47	-0.159571147966949\\
-0.46	-0.150352539260579\\
-0.45	-0.141095481413682\\
-0.44	-0.131800878234368\\
-0.43	-0.122469637759829\\
-0.42	-0.113102672165962\\
-0.41	-0.103700897676583\\
-0.399999999999999	-0.0942652344722136\\
-0.39	-0.0847966065984764\\
-0.38	-0.0752959418740781\\
-0.37	-0.0657641717984184\\
-0.36	-0.0562022314588161\\
-0.35	-0.0466110594373657\\
-0.34	-0.0369915977174405\\
-0.33	-0.0273447915898365\\
-0.32	-0.0176715895585846\\
-0.31	-0.00797294324642656\\
-0.3	0.00175019270002497\\
-0.29	0.0114968607054478\\
-0.28	0.0212661003612652\\
-0.27	0.0310569485213988\\
-0.26	0.0408684393983004\\
-0.25	0.050699604659266\\
-0.24	0.0605494735230053\\
-0.23	0.0704170728564711\\
-0.22	0.0803014272719287\\
-0.21	0.0902015592242686\\
-0.2	0.100116489108537\\
-0.19	0.110045235357685\\
-0.18	0.119986814540523\\
-0.17	0.129940241459875\\
-0.16	0.139904529250912\\
-0.149999999999999	0.14987868947967\\
-0.14	0.159861732241727\\
-0.13	0.169852666261044\\
-0.12	0.17985049898895\\
-0.11	0.189854236703259\\
-0.0999999999999996	0.199862884607525\\
-0.0899999999999999	0.209875446930401\\
-0.0800000000000001	0.21989092702512\\
-0.0700000000000003	0.229908327469061\\
-0.0599999999999996	0.239926650163407\\
-0.0499999999999998	0.249944896432881\\
-0.04	0.259962067125552\\
-0.0300000000000002	0.26997716271269\\
-0.0199999999999996	0.27998918338868\\
-0.00999999999999979	0.289997129170959\\
0	0.3\\
0.00999999999999979	0.309996795839292\\
0.0199999999999996	0.319986516775345\\
0.0300000000000002	0.329968163117682\\
0.04	0.33994073549882\\
0.0499999999999998	0.349903234974237\\
0.0599999999999996	0.359854663122295\\
0.0700000000000003	0.369794022144127\\
0.0800000000000001	0.379720314963466\\
0.0899999999999999	0.389632545326423\\
0.0999999999999996	0.399529717901181\\
0.11	0.409410838377609\\
0.12	0.419274913566789\\
0.13	0.429120951500434\\
0.14	0.438947961530199\\
0.149999999999999	0.448754954426867\\
0.16	0.458540942479404\\
0.17	0.468304939593867\\
0.18	0.478045961392171\\
0.19	0.487763025310685\\
0.2	0.49745515069866\\
0.21	0.507121358916468\\
0.22	0.516760673433667\\
0.23	0.526372119926847\\
0.24	0.535954726377275\\
0.25	0.545507523168312\\
0.26	0.55502954318261\\
0.27	0.56451982189906\\
0.28	0.573977397489493\\
0.29	0.583401310915119\\
0.3	0.592790606022704\\
0.31	0.60214432964046\\
0.32	0.611461531673651\\
0.33	0.6207412651999\\
0.34	0.629982586564188\\
0.35	0.639184555473536\\
0.36	0.648346235091364\\
0.37	0.657466692131506\\
0.38	0.666544996951887\\
0.39	0.675580223647846\\
0.399999999999999	0.684571450145086\\
0.41	0.693517758292263\\
0.42	0.702418233953178\\
0.43	0.711271967098592\\
0.44	0.72007805189763\\
0.45	0.728835586808779\\
0.46	0.73754367467046\\
0.47	0.746201422791188\\
0.48	0.754807943039271\\
0.49	0.763362351932088\\
0.5	0.771863770724886\\
0.51	0.780311325499136\\
0.52	0.788704147250402\\
0.53	0.797041371975731\\
0.54	0.805322140760554\\
0.55	0.813545599865087\\
0.56	0.821710900810226\\
0.57	0.829817200462922\\
0.58	0.837863661121029\\
0.59	0.845849450597629\\
0.6	0.853773742304802\\
0.61	0.861635715336855\\
0.62	0.869434554552984\\
0.63	0.87716945065938\\
0.64	0.884839600290749\\
0.649999999999999	0.892444206091254\\
0.66	0.899982476794866\\
0.67	0.907453627305108\\
0.68	0.914856878774207\\
0.69	0.922191458681614\\
0.7	0.929456600911907\\
0.71	0.936651545832068\\
0.72	0.943775540368112\\
0.73	0.950827838081078\\
0.74	0.957807699242358\\
0.75	0.964714390908368\\
0.76	0.971547186994555\\
0.77	0.978305368348714\\
0.78	0.984988222823637\\
0.79	0.991595045349057\\
0.8	0.998125138002903\\
0.81	1.00457781008184\\
0.82	1.01095237817112\\
0.83	1.01724816621367\\
0.84	1.0234645055785\\
0.85	1.02960073512832\\
0.86	1.03565620128653\\
0.87	1.0416302581033\\
0.88	1.04752226732103\\
0.89	1.05333159843899\\
0.9	1.05905762877722\\
0.91	1.06469974353961\\
0.92	1.07025733587628\\
0.93	1.07572980694508\\
0.94	1.08111656597238\\
0.95	1.08641703031303\\
0.96	1.09163062550949\\
0.97	1.09675678535017\\
0.98	1.101794951927\\
0.99	1.10674457569211\\
1	1.1116051155137\\
1.01	1.11637603873112\\
1.02	1.12105682120905\\
1.03	1.12564694739088\\
1.04	1.13014591035122\\
1.05	1.13455321184756\\
1.06	1.13886836237111\\
1.07	1.14309088119667\\
1.08	1.14722029643175\\
1.09	1.15125614506476\\
1.1	1.1551979730123\\
1.11	1.15904533516561\\
1.12	1.16279779543611\\
1.13	1.16645492680005\\
1.14	1.17001631134225\\
1.15	1.17348154029895\\
1.16	1.17685021409977\\
1.17	1.18012194240872\\
1.18	1.18329634416431\\
1.19	1.18637304761879\\
1.2	1.18935169037637\\
1.21	1.19223191943061\\
1.22	1.1950133912008\\
1.23	1.19769577156747\\
1.24	1.20027873590694\\
1.25	1.20276196912491\\
1.26	1.20514516568909\\
1.27	1.20742802966098\\
1.28	1.20961027472657\\
1.29	1.21169162422619\\
1.3	1.2136718111833\\
1.31	1.21555057833247\\
1.32	1.2173276781462\\
1.33	1.21900287286097\\
1.34	1.22057593450219\\
1.35	1.22204664490821\\
1.36	1.2234147957534\\
1.37	1.22468018857021\\
1.38	1.22584263477025\\
1.39	1.22690195566444\\
1.4	1.22785798248211\\
1.41	1.22871055638915\\
1.42	1.2294595285052\\
1.43	1.23010475991982\\
1.44	1.23064612170765\\
1.45	1.23108349494263\\
1.46	1.23141677071121\\
1.47	1.23164585012454\\
1.48	1.2317706443297\\
1.49	1.23179107451991\\
1.5	1.23170707194379\\
1.51	1.23151857791352\\
1.52	1.23122554381214\\
1.53	1.23082793109971\\
1.54	1.2303257113186\\
1.55	1.22971886609766\\
1.56	1.2290073871555\\
1.57	1.22819127630267\\
1.58	1.22727054544291\\
1.59	1.22624521657335\\
1.6	1.22511532178377\\
1.61	1.22388090325477\\
1.62	1.22254201325499\\
1.63	1.22109871413737\\
1.64	1.21955107833432\\
1.65	1.21789918835195\\
1.66	1.21614313676326\\
1.67	1.2142830262004\\
1.68	1.21231896934583\\
1.69	1.21025108892258\\
1.7	1.20807951768345\\
1.71	1.20580439839923\\
1.72	1.20342588384595\\
1.73	1.20094413679109\\
1.74	1.19835932997883\\
1.75	1.19567164611432\\
1.76	1.19288127784693\\
1.77	1.18998842775254\\
1.78	1.18699330831479\\
1.79	1.18389614190546\\
1.8	1.18069716076372\\
1.81	1.17739660697451\\
1.82	1.17399473244593\\
1.83	1.17049179888557\\
1.84	1.166888077776\\
1.85	1.16318385034916\\
1.86	1.15937940755987\\
1.87	1.15547505005835\\
1.88	1.15147108816172\\
1.89	1.14736784182468\\
1.9	1.14316564060905\\
1.91	1.13886482365252\\
1.92	1.13446573963632\\
1.93	1.12996874675207\\
1.94	1.12537421266754\\
1.95	1.12068251449159\\
1.96	1.11589403873814\\
1.97	1.11100918128914\\
1.98	1.10602834735672\\
1.99	1.1009519514443\\
2	1.09578041730688\\
2.01	1.09051417791034\\
2.02	1.08515367538985\\
2.03	1.07969936100736\\
2.04	1.07415169510823\\
2.05	1.06851114707685\\
2.06	1.0627781952915\\
2.07	1.05695332707821\\
2.08	1.05103703866378\\
2.09	1.04502983512788\\
2.1	1.03893223035434\\
2.11	1.03274474698149\\
2.12	1.02646791635167\\
2.13	1.02010227845984\\
2.14	1.01364838190142\\
2.15	1.00710678381913\\
2.16	1.00047804984913\\
2.17	0.993762754066227\\
2.18	0.986961478928241\\
2.19	0.980074815219604\\
2.2	0.973103361994067\\
2.21	0.966047726516613\\
2.22	0.958908524204558\\
2.23	0.951686378567827\\
2.24	0.944381921148438\\
2.25	0.936995791459185\\
2.26	0.929528636921526\\
2.27	0.921981112802689\\
2.28	0.914353882152\\
2.29	0.906647615736431\\
2.3	0.898862991975389\\
2.31	0.891000696874741\\
2.32	0.883061423960087\\
2.33	0.87504587420928\\
2.34	0.86695475598422\\
2.35	0.858788784961896\\
2.36	0.850548684064718\\
2.37	0.842235183390119\\
2.38	0.833849020139446\\
2.39	0.82539093854615\\
2.4	0.816861689803273\\
2.41	0.808262031990244\\
2.42	0.799592729998995\\
2.43	0.790854555459394\\
2.44	0.782048286664016\\
2.45	0.773174708492244\\
2.46	0.764234612333726\\
2.47	0.755228796011176\\
2.48	0.746158063702541\\
2.49	0.737023225862536\\
2.5	0.727825099143556\\
2.51	0.718564506315972\\
2.52	0.709242276187817\\
2.53	0.699859243523882\\
2.54	0.690416248964203\\
2.55	0.680914138941987\\
2.56	0.67135376560094\\
2.57	0.661735986712049\\
2.58	0.652061665589787\\
2.59	0.642331671007782\\
2.6	0.632546877113933\\
2.61	0.622708163345003\\
2.62	0.612816414340676\\
2.63	0.602872519857108\\
2.64	0.592877374679961\\
2.65	0.582831878536945\\
2.66	0.572736936009864\\
2.67	0.562593456446186\\
2.68	0.552402353870132\\
2.69	0.542164546893313\\
2.7	0.531880958624895\\
2.71	0.521552516581336\\
2.72	0.511180152595673\\
2.73	0.500764802726383\\
2.74	0.490307407165831\\
2.75	0.4798089101483\\
2.76	0.469270259857627\\
2.77	0.458692408334438\\
2.78	0.448076311383016\\
2.79	0.437422928477775\\
2.8	0.426733222669392\\
2.81	0.416008160490565\\
2.82	0.405248711861443\\
2.83	0.394455849994703\\
2.84	0.383630551300318\\
2.85	0.372773795289995\\
2.86	0.361886564481312\\
2.87	0.350969844301556\\
2.88	0.340024622991271\\
2.89	0.329051891507533\\
2.9	0.318052643426947\\
2.91	0.307027874848392\\
2.92	0.295978584295519\\
2.93	0.284905772618994\\
2.94	0.273810442898532\\
2.95	0.262693600344692\\
2.96	0.251556252200472\\
2.97	0.240399407642702\\
2.98	0.229224077683241\\
2.99	0.218031275069998\\
3	0.20682201418778\\
3.01	0.195597310958976\\
3.02	0.184358182744096\\
3.03	0.173105648242158\\
3.04	0.161840727390957\\
3.05	0.150564441267197\\
3.06	0.139277811986523\\
3.07	0.127981862603445\\
3.08	0.116677617011175\\
3.09	0.105366099841374\\
3.1	0.0940483363638348\\
3.11	0.0827253523860977\\
3.12	0.0713981741530196\\
3.13	0.0600678282462957\\
3.14	0.0487353414839503\\
3.15	0.0374017408198073\\
3.16	0.0260680532429496\\
3.17	0.0147353056771734\\
3.18	0.00340452488045106\\
3.19	-0.00792326265558602\\
3.2	-0.0192470308061365\\
3.21	-0.0305657539136822\\
3.22	-0.0418784068883931\\
3.23	-0.0531839653084726\\
3.24	-0.0644814055204339\\
3.25	-0.0757697047393015\\
3.26	-0.0870478411487235\\
3.27	-0.0983147940009853\\
3.28	-0.109569543716914\\
3.29	-0.120811071985668\\
3.3	-0.132038361864398\\
3.31	-0.143250397877768\\
3.32	-0.154446166117332\\
3.33	-0.165624654340749\\
3.34	-0.176784852070836\\
3.35	-0.187925750694438\\
3.36	-0.199046343561118\\
3.37	-0.210145626081639\\
3.38	-0.221222595826252\\
3.39	-0.23227625262276\\
3.4	-0.243305598654354\\
3.41	-0.254309638557221\\
3.42	-0.265287379517897\\
3.43	-0.276237831370374\\
3.44	-0.287160006692932\\
3.45	-0.298052920904704\\
3.46	-0.308915592361951\\
3.47	-0.319747042454051\\
3.48	-0.330546295699174\\
3.49	-0.341312379839653\\
3.5	-0.352044325937026\\
3.51	-0.362741168466753\\
3.52	-0.373401945412582\\
3.53	-0.384025698360574\\
3.54	-0.394611472592762\\
3.55	-0.405158317180444\\
3.56	-0.415665285077091\\
3.57	-0.426131433210877\\
3.58	-0.436555822576799\\
3.59	-0.446937518328408\\
3.6	-0.457275589869103\\
3.61	-0.467569110943021\\
3.62	-0.477817159725479\\
3.63	-0.488018818912977\\
3.64	-0.498173175812757\\
3.65	-0.508279322431891\\
3.66	-0.518336355565904\\
3.67	-0.52834337688692\\
3.68	-0.53829949303132\\
3.69	-0.548203815686903\\
3.7	-0.558055461679546\\
3.71	-0.567853553059345\\
3.72	-0.577597217186242\\
3.73	-0.587285586815119\\
3.74	-0.596917800180347\\
3.75	-0.606493001079799\\
3.76	-0.616010338958304\\
3.77	-0.625468968990529\\
3.78	-0.634868052163302\\
3.79	-0.644206755357346\\
3.8	-0.653484251428432\\
3.81	-0.662699719287927\\
3.82	-0.671852343982743\\
3.83	-0.680941316774676\\
3.84	-0.68996583521912\\
3.85	-0.698925103243159\\
3.86	-0.707818331223012\\
3.87	-0.71664473606085\\
3.88	-0.72540354126095\\
3.89	-0.734093977005196\\
3.9	-0.742715280227912\\
3.91	-0.751266694690022\\
3.92	-0.759747471052532\\
3.93	-0.768156866949314\\
3.94	-0.776494147059211\\
3.95	-0.78475858317742\\
3.96	-0.792949454286181\\
3.97	-0.801066046624746\\
3.98	-0.809107653758615\\
3.99	-0.81707357664805\\
4	-0.824963123715854\\
4.01	-0.832775610914398\\
4.02	-0.840510361791902\\
4.03	-0.848166707557956\\
4.04	-0.855743987148282\\
4.05	-0.863241547288719\\
4.06	-0.870658742558439\\
4.07	-0.877994935452371\\
4.08	-0.885249496442844\\
4.09	-0.892421804040431\\
4.1	-0.899511244853991\\
4.11	-0.906517213649907\\
4.12	-0.913439113410506\\
4.13	-0.920276355391667\\
4.14	-0.927028359179595\\
4.15	-0.933694552746776\\
4.16	-0.940274372507083\\
4.17	-0.946767263370056\\
4.18	-0.953172678794321\\
4.19	-0.959490080840165\\
4.2	-0.965718940221253\\
4.21	-0.971858736355478\\
4.22	-0.977908957414954\\
4.23	-0.983869100375123\\
4.24	-0.989738671062995\\
4.25	-0.995517184204503\\
4.26	-1.00120416347097\\
4.27	-1.00679914152468\\
4.28	-1.01230166006358\\
4.29	-1.01771126986503\\
4.3	-1.02302753082871\\
4.31	-1.02825001201855\\
4.32	-1.03337829170382\\
4.33	-1.03841195739926\\
4.34	-1.04335060590429\\
4.35	-1.04819384334128\\
4.36	-1.05294128519296\\
4.37	-1.05759255633881\\
4.38	-1.06214729109057\\
4.39	-1.0666051332268\\
4.4	-1.07096573602649\\
4.41	-1.07522876230171\\
4.42	-1.07939388442934\\
4.43	-1.08346078438185\\
4.44	-1.08742915375706\\
4.45	-1.09129869380706\\
4.46	-1.09506911546601\\
4.47	-1.09874013937716\\
4.48	-1.10231149591873\\
4.49	-1.10578292522894\\
4.5	-1.10915417723002\\
4.51	-1.11242501165122\\
4.52	-1.11559519805093\\
4.53	-1.11866451583774\\
4.54	-1.12163275429053\\
4.55	-1.1244997125776\\
4.56	-1.12726519977486\\
4.57	-1.12992903488289\\
4.58	-1.1324910468432\\
4.59	-1.13495107455333\\
4.6	-1.13730896688108\\
4.61	-1.1395645826777\\
4.62	-1.14171779079005\\
4.63	-1.14376847007185\\
4.64	-1.14571650939386\\
4.65	-1.14756180765311\\
4.66	-1.14930427378107\\
4.67	-1.15094382675094\\
4.68	-1.15248039558378\\
4.69	-1.15391391935378\\
4.7	-1.15524434719248\\
4.71	-1.15647163829192\\
4.72	-1.15759576190694\\
4.73	-1.15861669735633\\
4.74	-1.15953443402306\\
4.75	-1.16034897135348\\
4.76	-1.16106031885556\\
4.77	-1.16166849609605\\
4.78	-1.16217353269673\\
4.79	-1.16257546832958\\
4.8	-1.16287435271102\\
4.81	-1.1630702455951\\
4.82	-1.1631632167657\\
4.83	-1.16315334602776\\
4.84	-1.16304072319748\\
4.85	-1.16282544809157\\
4.86	-1.16250763051544\\
4.87	-1.16208739025042\\
4.88	-1.16156485704008\\
4.89	-1.16094017057537\\
4.9	-1.16021348047898\\
4.91	-1.15938494628853\\
4.92	-1.15845473743891\\
4.93	-1.15742303324359\\
4.94	-1.15629002287488\\
4.95	-1.15505590534335\\
4.96	-1.15372088947612\\
4.97	-1.15228519389431\\
4.98	-1.15074904698941\\
4.99	-1.14911268689875\\
5	-1.14737636147996\\
};
\end{axis}

\end{tikzpicture}%

\section{Mathematical Model}
\label{sec:Model}
\vspace{-0.2em} %EXTRASPACE
Let us introduce some notation. Let $x_i$ denote the $i^{\mathrm{th}}$ entry of the vector $x\in\mathbb{R}^n$. Moreover, $x(i:q:(k-1)q+i)$ denotes the sub-vector in $\mathbb{R}^k$ formed by the entries of $x$ starting at index $i + qr$, where $r = 0, 1, \ldots, k-1$ and $i \in [q]$. In addition, $A(i:q:(k-1)q,l)$ represents the sub-vector constructed by selecting the $l^{\mathrm{th}}$ column of matrix $A$ and choosing the entries of the selected column as above. 


In~\eqref{quan_obs}, we assume that the matrix $A$ can be factorized as $A=DB$ where $D\in\R^{p\times n}$ is obtained by stacking $k'$ diagonal matrices with size $q\times q$. Moreover, $B\in\R^{q\times n}$ is a matrix which depends on the prior used for $x$. For instance, if $x$ is assumed to be a sparse vector in some orthobasis, then $B$ can be any matrix that supports sparse recovery; for instance, $B$ can satisfy the restricted isometry property (RIP)~\cite{CandesRIP}, or the null-space property (NSP)~\cite{foucart2013}. 

Also, in~\eqref{quan_obs} we assume that $C$ is another block diagonal matrix of size $m\times p$ formed by $k$ diagonal matrices with size $p\times p$ such that the diagonal entries in each of these blocks are specially chosen (see Section~\ref{sec:harmonic} below for more details). We assume that $p$ and $m$ are multiples of $q$ and $p$, respectively. 

For brevity, let us denote $\mod(\cdot,R)$ as $\mod(\cdot)$. Then, the model~\eqref{quan_obs} can be expanded as:
\begin{align}\label{Mainmodel}
y = Q\left(
\begin{bmatrix}
C^0 \\
C^1 \\
\vdots  \\
C^{k-1}
\end{bmatrix}
\mod\left(
\begin{bmatrix}
D^1 \\
D^2 \\
 \vdots  \\
D^{k'}
\end{bmatrix}
Bx
\right)\right) .
\end{align}

We define the following intermediate variables:
\begin{align}\label{mainabr}
u = \mod(DBx)\in\R^p, \ \ z = Bx\in\R^q,
\end{align}
As discussed in~\cite{SoltaniHegde_ICASSP16}, the block diagonal structure of $D$ and $C$ allows the signal reconstruction problem to be reduced to a sequence of \emph{decoupled} scalar estimation problems; such a decoupling enables the estimation of each entry of $z$ and $u$ independently of all other entries.

We also assume that the analog values of the input signal lies within a range known \emph{a priori}. 
We define $\Delta$ as the reference point for the quantization, and assume that the inputs to the quantizer are bounded within the region $[0,2\Delta]$. 
The function under consideration is the 1-bit quantization function, 
defined as follows:
\begin{align}\label{eq:quantfunc}
Q(u_i)= 
\begin{cases}
0,& \text{if } u_i\leq {\Delta}\\
1,              & \text{if } {\Delta} < u_i \leq {2\Delta} 
\end{cases}
\end{align}

For every element of the input to the quantizer, $u_i$, we measure multiple outputs $y_{i,0},y_{i,1}, \cdots ,y_{i,k-1}$, given by:
\begin{align}\label{eq:hmshort}
y_{i,j} = Q(c_{i,j}u_i) , ~~~~ j = 0,1,2,...,k-1.
\end{align}
with $c_0 =1$, and each subsequent $c_{i,j}$ is defined as :
\begin{align}\label{eq:hm}
c_{i,j} = 
\begin{cases}
\frac{k}{k-j},& \text{if } y_{i,0} = 0,\\
\frac{k}{k+j},              & \text{if } y_{i,0} = 1,
\end{cases} ~~~~ j = 1,2,...,k-1.
\end{align}
The underlying idea is to increase or decrease the value of $c_{i,j}u_i$ gradually and to detect the index $j^*$ for which $y_{i,j}$ changes its value for the first time. Using $j^*$, Interval containing $u_i$ can be determined. Here, the reason for choosing the values of $c_{i,j}$ in harmonic progression is to ensure that all such intervals corresponding to the different values of $j^*$ have equal widths. This fact is made clearer in Section~\ref{sec:harmonic}.

As the multipliers $c_{i,j}$ form a harmonic progression for both the cases, we call the proposed measurement scheme the \emph{harmonic multipliers method}. In order to express it as a linear transformation, these multipliers can be arranged into a block diagonal matrix $C$ of size $(kp) \times p$, for which the diagonal entry in $i^{th}$ row in block $C^j$ will be $c_{i,j}$, the multiplier corresponding to the input signal element $u_i$. 
%Now, the final measurement matrix $y$ is $(kp) \times 1$, containing $k$ blocks of size $p\times 1$, each representing the measurement vector $y^j$. 
The forward model can be written in the form of following equation:
\begin{align}
y = Q(Cu),
\end{align}
where $u$ is defined in~\eqref{mainabr}.


\section{Reconstruction Procedure}
%\vspace{-0.5em}%EXTRASPACE
To solve the inverse problem in~\eqref{Mainmodel}, we propose a three stage procedure that we call \textit{reconstruction from de-quantized modulo observations}, or \textit{RQM} for short. In the first stage, RQM estimates $u= mod(Ax)$ from vector $y$. Next, it uses the estimate, $\widehat{u}$ to produce an estimate $z$ in the second stage (say $\widehat{z}$). Finally, this estimate $\widehat{z}$ is used for recovery of the original $x$. The pseudocode of RQM is given as Alg.~\ref{alg:DMF}.  We now describe each of these stages in detail. 

\begin{algorithm}[t]
\caption{\textsc{RQM}}
\label{alg:DMF}
\begin{algorithmic}
\State\textbf{Inputs:} $y$, $D$, $B$, $C$, $k$, $k'$, $\Omega$, $s$
\State\textbf{Output:}  $\widehat{x}$
%\State $ k \leftarrow m/q$
\State \textbf{Stage 1: Harmonic dequantization}
\State $\widehat{u}\leftarrow \textsc{HMDequantization}(y,C,k)$
\State \textbf{Stage 2: Modulo recovery}
\State $\theta\leftarrow  \exp(i \widehat{u})$
\For {$l =1:q$}
\State $t \leftarrow D(l:q:(k'-1)q+l,l)$
\State $\phi \leftarrow \theta(l:q:(k'-1)q+l)$
%\State estimate $z_l$ from $u = \exp(jtz_l)$ using 
\State $\widehat{z_l} = \argmax_{\omega\in\Omega}|\langle y,\psi_{\omega}\rangle|$
\EndFor
\State $\widehat{z} \leftarrow [\widehat{z_1},\widehat{z_2}\ldots,\widehat{z_q}]^T$
\State \textbf{Stage 3: Sparse recovery}
%\State $\varepsilon \leftarrow \mathcal{O}(\frac{1}{|T|})$
\State $\widehat{x} \leftarrow \textsc{CoSaMP}(\widehat{z},B,s)$
\end{algorithmic}
\end{algorithm}

\subsection{Harmonic dequantization}
\label{sec:harmonic}
We first attempt to recover $u$. Based on the value of $Q(u_i)$, we can know the interval in which $u_i$ lies, which can either be $[0,\Delta]$ or $[\Delta,2\Delta]$. For $u_i \in [0,\Delta]$, the multipliers are designed in such a way that with each multiplication, we gradually increase the value of $c_{i,j}u_i$ until it becomes greater than $\Delta$ at $j=j^*$ to give $Q(c_{i,j^*}u_i)=1$. 

Using $j^*$, we can decide the interval of $u_i$ as follows. %For $u_i$ such that $y_{i,0}=0$, if $y_{i,j}$ changes from $0$ to $1$ at $j = j^*$ for the first time, 
Equation \eqref{eq:hmshort} gives:
\[
y_{i,{j^*-1}} = Q(c_{i,j^*-1} u_i) = \left \lfloor{\dfrac{k u_i}{(k-j^*+1) \Delta}}\right \rfloor = 0.
\]
%$$\implies u < \Delta \dfrac{(k-j^*+1)}{k}$$
Similarly, 
\[
y_{i,{j^*}} = Q(c_{i,j^*} u_i) = \left \lfloor{\dfrac{k u_i}{(k-j^*) \Delta}}\right \rfloor = 1.
\]
%$$\implies u \geq \Delta \dfrac{(k-j^*)}{k}$$
Combining the above relations, we infer that:
\begin{align}
\label{eq:hminter1}
\Delta \dfrac{(k-j^*)}{k} \leq u_i < \Delta \dfrac{(k-j^*+1)}{k}.
\end{align}
Similarly, for $u_i$ with $y_{i,0}=1$, through each multiplication with $c_{i,j}$, the value of $c_{i,j} u_i$ decreases gradually and becomes less than $\Delta$ for $j=j^*$ for the first time. 
\begin{align}
\label{eq:hminter2}
\Delta \dfrac{(k+j^*-1)}{k} \leq u_i < \Delta \dfrac{(k+j^*)}{k}.
\end{align}
\begin{algorithm}[t]
	\caption{\textsc{HMDequantization}}
	\label{alg:HM}
	\begin{algorithmic}
		\State\textbf{Inputs:} $y$, $C$, $k$
		\State\textbf{Output:}  $\widehat{u}$\\
		$n \leftarrow length(y)/k$
		%\State $ k \leftarrow m/q$
		\For {$l =1:n$}
		\If {$y_l = 0$}
		\State $t \leftarrow y(l+n:n:(k-1)n+l,1)$
		\State $j^* \leftarrow \min_{j \in \{1,2,...,k-1\}} \text{such that } t_j = 1$
		\State $\widehat{u}_{l} \leftarrow v \sim U[\Delta\frac{k-j^*}{k},\Delta
		\frac{k-j^*+1}{k}]$
		\ElsIf {$y_l = 1$}
		\State $t \leftarrow y(l+n:n:(k-1)n+l,1)$
		\State $j^* \leftarrow \min_{j \in \{1,2,...,k-1\}} \text{such that } t_j = 0$
		\State $\widehat{u}_{l} \leftarrow v \sim U[\Delta\frac{k+j^*-1}{k},\Delta\frac{k+j^*}{k}]$
		\EndIf
		\EndFor
		
	\end{algorithmic}
\end{algorithm}
Equations \eqref{eq:hminter1} and \eqref{eq:hminter2} provide us the interval on the real line that contains $u_i$. To remove bias, a random real number is chosen from this interval as the final estimate $\widehat{u_i}$. The width of this interval is $\delta = \frac{\Delta}{k}$, which is same for every interval corresponding to different values of $j^*$ owing to harmonic design of the multipliers. The value of $\delta$ (and consequently, the estimation error) can be made sufficiently small by increasing the value of $k$. For the estimate $\widehat{u_i}$ to lie within an $\epsilon \Delta$ neighborhood of $u_i$, the minimum value required for $k$ can be calculated as 
%\begin{align}
$k_{\text{req}} = \left \lceil{\frac{1}{\epsilon}}\right \rceil$.
%\end{align}
%As the quantized measurements occupy only 1 bit of storage for each pixel, increasing number of measurements doesn't affect the sample complexity by large.
%For the case when the range of the input is $[\Delta - \alpha, \Delta + \beta]$, %the number of measurements required is different for the points lying on the either side of the reference point $\Delta$. 
%an analogous procedure can be developed by defining two separate $k$ values $k_{\alpha}, k_{\beta}$, one for each interval. 


In this paper, we assume that the first multiplier $c_0 =1$, and it is possible to decide the appropriate $c_j$'s by looking at the value of the first measurement $y_{i,0}$ in Eq\ \eqref{eq:hm}. 
In case purely non-adaptive measurements are desired, a similar approach can be followed by acquiring $2k-1$ measurements with each possible value of $c_j$ specified by both cases of~\eqref{eq:hm}. The pseudocode of this stage is given as Alg~\ref{alg:HM}.

\vspace{-1.2em}%EXTRASPACE
\subsection{Modulo recovery}\label{ToneEst}
	\vspace{-0.5em}%EXTRASPACE
The output of \textsc{HMDequantization} acts as input for the modulo recovery stage. The goal of this stage is to find an estimate for the vector $z$. There are several different ways of doing this, including the multi-shot UHDR method of~\cite{ICCP15_Zhao}. Here, we describe a novel approach, based on the MF-Sparse algorithm of~\cite{SoltaniHegde_ICASSP16}. We assume that the entries of $z$ belong to some bounded set $\Omega \in R$. Fix $l \in [q]$ and form $\theta =  \exp(i \widehat{u})$. Let {$t = D(l:q:(k'-1)q+l,l)$ and $\phi = \theta(l:q:(k'-1)q+l)$}, 
%h = e(l:q:(k-1)q+l)$} 
which are vectors in $\mathbb{R}^{k'}$. Thus, we have the following model:
\begin{align}\label{tonmodel}
\phi = \exp(i z_l t) \, .
\end{align}
In the above model, $\phi$ can be interpreted as a set of time samples of a complex-valued signal with frequencies $z_l \in \Omega$, measured at time locations $t$. As a result, we can independently recover $z_l$ for $l=1, \ldots, q$ by solving a least-squares problem~\cite{eftekhari2013matched}:
\begin{align}
\label{OptmExp}
\widehat{z_l} = \underset{v \in \Omega}{\argmin}~\|\phi - \exp(i \, vt )\|_2^2 = \underset{v \in \Omega}{\argmax}~\left|\langle \phi,\psi_{v}\rangle\right|,
 \end{align}
for all $l=1,\ldots,q$, where  $\psi_{v}\in\mathbb{R}^{k'}$ denotes a \emph{template vector} given by $\psi_{v} = \exp(j t v)$ for any $v \in \Omega$. The solution of this optimization problem is equivalent to performing a \emph{matched filter} from irregularly spaced samples. Numerically, the optimization problem in~\eqref{OptmExp} can be solved using a grid search over the set $\Omega$, and the resolution of this grid search controls the running time of the algorithm. For fine enough resolutions, the estimation of $z_l$ is more accurate, at the expense of increased running time. This issue is also discussed in~\cite{eftekhari2013matched} and~\cite{eldarxampling,tangcsoffgrid,chioffgrid} have proposed more sophisticated spectral estimation techniques. %After obtaining all the estimates $\widehat{z_l}$'s, we stack them in a vector $\widehat{z}$. 

In~\eqref{tonmodel}, the vector $\theta$ is modeled in terms of complex exponentials. As discussed in~\cite{SoltaniHegde_ICASSP16}, we can equivalently use a real-valued sine function. That is, the vector $\phi$ can be defined as:
\begin{align}
\label{eq:realsine_obs}
\phi = \sin(i z_l t),
\end{align}
Similar to the complex case, we estimate $z$ by solving~\cite{SoltaniHegde_ICASSP16}:
\begin{align*}
\label{OptmSin}
\widehat{z_l} &= \underset{v \in\Omega}{\argmin}~\|\phi - \sin(v t)\|_2^2 = \underset{v \in \Omega}{\argmax}~\left( 2\left|\langle \phi,\psi_{v}\rangle\right| - \|\psi_{v}\|_2^2\right),
\end{align*}
for $l=1,\ldots,q$ and $\phi$ as defined above and $\psi_v =\sin(tv)$. 
%{Also, $\psi_{v} = \sin(tv)$ for any $v \in \Omega$}. Now it remain to estimate the original signal $x$ from $\widehat{z}$ obtained in the second stage. This is done by the final stage, Sparse recovery.
	\vspace{-1.2em}%EXTRASPACE
\subsection{Sparse recovery}
	\vspace{-0.5em}%EXTRASPACE
Finally, we estimate the original signal $x$ from $\widehat{z}$ obtained as the output of the second stage.
Note that the use of sparse recovery here is generic, and we could in principle use any other prior model of relevance to the specific imaging application. Since we assume that matrix $B$ in~\eqref{Mainmodel} supports stable sparse recovery and the underlying signal $x$ is $s$-sparse, we can use any generic sparse recovery algorithm to estimate $x$. In our experiments, we chose to use the CoSaMP algorithm \cite{cosamp} due to its ease and speed. Hence, we take $\widehat{z}$ from previous stage and run CoSaMP to obtain the final estimation, $\widehat{x}$.










\section{Experimental Results}
\label{sec:Results}
\begin{figure*}[t]
	\begin{center}
		\begingroup
		\setlength{\tabcolsep}{1pt} % Default value: 6pt
		\renewcommand{\arraystretch}{.1} % Default value: 1
		\begin{tabular}{ccc}      %{c@{\hskip .1pt}c@{\hskip .1pt}c}
			\includegraphics[trim = 30mm 80mm 35mm 80mm, clip, width=0.25\linewidth]{./hm_err.pdf}&
			\includegraphics[trim = 30mm 80mm 35mm 80mm, clip, width=0.25\linewidth]{./rqm_rqmms17.pdf}&
			\includegraphics[trim = 30mm 80mm 35mm 80mm, clip, width=0.25\linewidth]{./rqms_rqmmss17.pdf}
			\\
			(a) & (b) & (c)
		\end{tabular}
		\endgroup
	\end{center}
	\caption{\emph{ Normalized error vs number of quantized measurements $(k)$ for: (a) dequantization using HM algorithm; (b) reconstruction from quantized modulo measurements using RQM and RQM multi-shot; (c) reconstruction from quantized modulo measurements of sparse input using RQM and RQM multi-shot.}}
	\label{fig:results}
\end{figure*}


\begin{figure*}[t]
	\begin{center}
		\begingroup
		\setlength{\tabcolsep}{0.1pt} % Default value: 6pt
		%\renewcommand{\arraystretch}{-1} % Default value: 1
		\begin{tabular}{cccccc}      %{c@{\hskip .1pt}c@{\hskip .1pt}c}

				%\includegraphics[trim = 35mm 75mm 45mm 79mm, clip, width=0.14\linewidth]{../figs/orgimg.pdf}&
				\includegraphics[trim = 35mm 75mm 45mm 77mm, clip, width=0.12\linewidth]{./hm_k_5.pdf}& \hspace{15pt}
				\includegraphics[trim = 89mm 127.25mm 92mm 119.15mm, clip, width=0.12\linewidth]{./dmf_k_5.pdf}&
					\includegraphics[trim = 89mm 126.5mm 92mm 120mm, clip, width=0.12\linewidth]{./dms_k_5.pdf}& \hspace{15pt}
						\includegraphics[trim = 70mm 109mm 75mm 103.5mm, clip, width=0.12\linewidth]{./dmfs_imgorg.pdf}&
							\includegraphics[trim = 70mm 109mm 75mm 103mm, clip, width=0.12\linewidth]{./dmfs_k_10.pdf}&
								\includegraphics[trim = 70mm 110mm 75mm 100.5mm, clip, width=0.12\linewidth]{./dmss_k_10.pdf} \\
								
			
		
			 (a) & \multicolumn{2}{c}{\hspace{20pt}(b)} & \multicolumn{3}{c}{\hspace{22pt}(c)}
		\end{tabular}
		\endgroup
	\end{center}
	\caption{\emph{Image reconstruction results: (a) image reconstructed from quantized measurements using HM, $k=5$; (b) image reconstructed from quantized modulo measurements using RQM (left) and RQM multi-shot (right), $k=5$; (c) sparse input image (left), image reconstructed from quantized modulo measurements of sparse input using RQM-sparse (centre) and RQM multi-shot-sparse (right), $k=10$.}}
	\vspace{-0.4em}%EXTRASPACE
	\label{fig:imgresult}
\end{figure*}
	\vspace{-0.5em}%EXTRASPACE
In this section, we provide some representative numerical experiments for our proposed algorithm. 
First, we provide results describing our proposed dequantization procedure on a real test image. Further, we also provide results for the combined task of de-quantization and modulo recovery, with and without sparsity priors on the underlying signal. We employ two different algorithms for modulo recovery, and therefore we have following four combinations for our experiments: (i) reconstruction from de-quantized modulo observations using RQM, with and without sparsity priors, (ii) de-quantization using our HM algorithm followed by modulo recovery using the multi-shot UHDR recovery algorithm (we refer this whole procedure as \emph{RQM-multi-shot}), with and without sparsity priors.
	\vspace{-0.7em}%EXTRASPACE
\subsection{Dequantization}
	\vspace{-0.4em}%EXTRASPACE
In this experiment, we only focus on the first stage, i.e., we attempt to recover the image $w$ from set of $k$ quantized measurements $y^j$, $j=0,1,2,...,k-1$, using the \textsc{HMDequantization} method for different values of $k$. We record the normalized estimation error defined as $\frac{\|\widehat{w}-w\|_2}{\|w\|_2}$, with $\widehat{w}$ being the estimate of $w$. Here, $w$ is the grayscale form of an 8-bit, 3-channel RGB image of the size $512 \times 512$. The 1-bit quantizer described in Eq \eqref{eq:quantfunc} is used with $\Delta = 2^7$ to calculate $y$. Based on values of $y^0$, the coefficients $c_j$s are decided for each element of $w$. Subsequently, $(k-1)$ measurements are obtained according to Eq \eqref{eq:hm}, and the \textsc{HMDequantization} method is used to obtain $\widehat{w}$. The normalized estimation error is plotted against the number of measurements $k$ in Fig.~\ref{fig:results}(a). As we observe from the plot, our algorithm can recover $w$ within $10\%$ of error with as low as $5$ measurements. Increasing the value of $k$ improves the recovery performance rapidly in this regime, and less than $5\%$ error can be achieved with $k\geq9$. %sNumber of measurements can be easily decided by the percentage of accuracy required.



	\vspace{-0.7em}%EXTRASPACE
\subsection{Experiment: No sparsity priors}
	\vspace{-0.3em}%EXTRASPACE
We take an 8-bit, 3-channel RGB image of size $256 \times 256$, convert it to grayscale, and scale the dynamic range to [0,1]. Since there are no sparsity priors assumed here, we let $B$ be the identity matrix. We consider two cases for $D$. In the forward model specified by the RQM algorithm, the vectorized image $x$ is first multiplied by the block diagonal matrix $D_{mf}$. The size of $D_{mf}$ is set $(k'n) \times n$ as it contains $k'$ blocks of size $n \times n$ each. Diagonal of each block contains uniformly distributed random variables in the range $[-T,T]$.  Similarly, in the forward model specified by for RQM-multi-shot, $x$ is multiplied by the block diagonal matrix $D_{ms}$; here, all diagonal elements for $r^{th}$ block are same and equal to $2^{9-r}$; for $r = 1,2, \ldots, k'$ \cite{ICCP15_Zhao}. 
%The product $(DBx)$ is then passed through a modulo function to obtain $u$. We multiply $u$ with $C$, again a block diagonal matrix of size $k\m_1 \times m_1$, calculated as described in Section \ref{sec:Model}. $y$ is output measured with $Cu$ as the input of Quantization function $Q$. The final measurement matrix $y$ is of size $(kk'n) \times 1$.


To recover $\widehat{z} = \widehat{x}$, the estimation of $z =x$, from measurement $y$, we employ both the RQM as well as the RQM-multi-shot algorithms in two separate experiments.
In Fig.~\ref{fig:results}(b), we plot the normalized estimation error in recovered $x$ in case of RQM-multi-shot by varying $k$, while $k'$ is fixed to 3. As we can see, we are able to recover the original image within $5\%$ error only with $k=3$ quantized measurements. Fig.~\ref{fig:results}(b) also shows the variation of normalized estimation error for the RQM algorithm with $k'$ fixed to 4. To recover the original image within $5\%$ error, RQM requires $k=15$ quantized measurements. 

	\vspace{-0.7em}%EXTRASPACE
\subsection{Experiment: Sparsity priors}
	\vspace{-0.3em}%EXTRASPACE
We now evaluate the performance of the proposed method in scenarios where the input signal is $s$-sparse. We use the same $256 \times 256$ RGB image, convert to grayscale, and after obtaining its 2D Haar wavelet decomposition, retain the $s=1000$ largest coefficients to sparsify the image. We further multiply the sparse test image by a subsampled Fourier matrix with $q=8000$ multiplied with a diagonal matrix with random $\pm1$ entries to get $z=Bx$. The rest of the observation process is identical to the experiment described above.
%The only difference is the size of $z$ is reduced to $q \times 1$ from $n \times 1$ after introducing sparsity. Subsequently, the size of $D$ (either $D_{mf}$ or $D_{ms}$) reduced to $(k'q)\times q$. As a result, we get measurement matrix $y$ with size $(kk'1)\times 1$.

%The estimate of $u$, $\widehat{u}$ is computed from the measurement matrix $y$ using the \textsc{HMDequantization}. 
Again, two separate experiments are performed with using RQM in one and RQM-multi-shot algorithm in another to recover $\widehat{z}$ from $y$. The final step is to compute the estimate of high dimensional signal $\widehat{x} \in \mathbb{R}^n$ from $\widehat{z} \in \mathbb{R}^q$, which we achieve using the CoSamP algorithm~\cite{cosamp}.
%We use CoSaMP algorithm\cite{cosamp}. Other recovery algorithms like structured-sparse recovery can also be well used.
For the RQM-multi-shot-sparse algorithm, we fix $k' = 3$, and obtain the plot of relative error by varying the value of $k$, which is shown in Fig.~\ref{fig:results}(c). As we can see from the plot, we are able to recover the original image within $5\%$ error only with the use of $7$ quantized measurements. Similar to the experiment without sparsity, we fixed $k'=4$ for RQM-sparse algorithm, and measure the normalized estimation error in $\hat{x}$ for different values of $k$. Corresponding plot is in Fig.~\ref{fig:results}(c). To estimate the original image within $5\%$ relative error, $k=25$ quantized measurements are used.

Considering that the input is sparse and the measurements $y$ are binary, the storage requirements for $y$ are considerably smaller compared to the case without sparsity. In essence, leveraging the sparsity prior can reduce the sample complexity of the algorithm drastically. The tradeoff is to choose a higher value of $k$ or $k'$, which will affect the running time only marginally but improves recovery performance by a significant amount.






%\section{Acknowledgements}
%This work was supported in part by the National Science Foundation under the grants CCF-1566281 and IIP-1632116.

% -------------------------------------------------------------------------
{{
\footnotesize
\bibliographystyle{IEEEbib}
\bibliography{../Common/chinbiblio,../Common/csbib,../Common/mrsbiblio,../Common/vsbib,../Common/kernels}
}
}

\end{document}
