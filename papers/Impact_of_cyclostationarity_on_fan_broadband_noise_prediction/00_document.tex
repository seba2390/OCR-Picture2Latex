%% This is file `elsarticle-template-1a-num.tex',
%%
%% Copyright 2009 Elsevier Ltd
%%
%% This file is part of the 'Elsarticle Bundle'.
%% ---------------------------------------------
%%
%% It may be distributed under the conditions of the LaTeX Project Public
%% License, either version 1.2 of this license or (at your option) any
%% later version.  The latest version of this license is in
%%    http://www.latex-project.org/lppl.txt
%% and version 1.2 or later is part of all distributions of LaTeX
%% version 1999/12/01 or later.
%%
%% The list of all files belonging to the 'Elsarticle Bundle' is
%% given in the file `manifest.txt'.
%%
%% Template article for Elsevier's document class `elsarticle'
%% with numbered style bibliographic references
%%
%% $Id: elsarticle-template-1a-num.tex 151 2009-10-08 05:18:25Z rishi $
%% $URL: http://lenova.river-valley.com/svn/elsbst/trunk/elsarticle-template-1a-num.tex $
%%

% \documentclass[preprint,12pt,twocolumn]{elsarticle}
% \documentclass[final,3p,11pt,times]{elsarticle}

%% Use the option review to obtain double line spacing
\documentclass[preprint,3p,11pt,times]{elsarticle}

%% Use the options 1p,twocolumn; 3p; 3p,twocolumn; 5p; or 5p,twocolumn
%% for a journal layout:
%% \documentclass[final,1p,times]{elsarticle}
% \documentclass[final,1p,times,twocolumn]{elsarticle}
%% \documentclass[final,3p,times]{elsarticle}
% \documentclass[final,3p,times,twocolumn]{elsarticle}
%% \documentclass[final,5p,times]{elsarticle}
% \documentclass[final,5p,times,twocolumn]{elsarticle}
% \documentclass[preprint,5p,times,twocolumn]{elsarticle}
\usepackage[T1]{fontenc}
\usepackage[utf8]{inputenc}
%% if you use PostScript figures in your article
%% use the graphics package for simple commands
%% \usepackage{graphics}
%% or use the graphicx package for more complicated commands
%% \usepackage{graphicx}
%% or use the epsfig package if you prefer to use the old commands
%% \usepackage{epsfig}

%%%%%%%%%%%%%%%%%%%%%%%%%%%%%%%%%%%%%%%%%%%%%%%%%%%%%%%%%%%%%%%%
% Verlinkung des Dokuments + nützliche Befehle fürs PDF, siehe hypersetup
% weiter unten - empfehlenswert! Sollte nach allen anderen Packages einge-
% bunden werden.
\usepackage{hyperref}
% Eigenschaften der Verlinkung
\hypersetup{
%  pdftitle={Fan broadband noise prediction by extension of the random particle mesh method to cyclostationarity},% Titel (intern)
%  pdftitle={Method to investigate the impact of wake cyclostationarity on fan broadband noise},% Titel (intern)
    pdftitle={Impact of cyclostationarity on fan broadband noise prediction},% Titel (intern)
  pdfauthor={Attila Wohlbrandt, Carolin Kissner, Sebastien Guerin},% Author (intern)
  pdfkeywords={Modeling, stochastic sources, fan broadband noise, CFD, CAA, cyclostationarity},%
% Stichtwörter (intern)
  colorlinks=false,% Verlinkung farbig
%   urlcolor=red,% Farben für Links festlegen
%   filecolor=red,
  linkcolor=black,
  citecolor=black,
  pdfborder={0 0 0},% Rähmchen bei Links ausschalten
  breaklinks=true, % Links umbrechen
  naturalnames=false, %für korrekte Verlinkung (warum???)
  hypertexnames=true,
  bookmarksnumbered=true,
  bookmarksdepth=subsection,
  pdfstartview={Fit}
}
\pdfcompresslevel=9
\setcounter{tocdepth}{2}
\renewcommand*{\appendixname}{}
%Warnungen von hyperref abschalten
\usepackage{etoolbox}
\makeatletter
\patchcmd{\@decl@short}{\bbl@info}{\@gobble}{}{}
\patchcmd{\@decl@short}{\bbl@info}{\@gobble}{}{}
\makeatother
%Waernungen entfernen
\usepackage{scrhack}
%%%%%%%%%%%%%%%%%%%%%%%%%%%%%%%%%%%%%%%%%%%%%%%%%%%%%%%%%%%%%%%%%%%

\usepackage{chngcntr}
% \counterwithout{figure}{chapter}
\counterwithout{equation}{section}


\usepackage{amsmath}
%% The amssymb package provides various useful mathematical symbols
\usepackage{amssymb}
%% The amsthm package provides extended theorem environments
%% \usepackage{amsthm}

\usepackage{multicol}

\usepackage[per-mode = symbol]{siunitx}  % used to plot all units in SI units use 
% \sisetup{output-exponent-marker=\ensuremath{\mathrm{e}}} %get 1e-10 instead of 1x10^(-10)
%Number only: \num{1e-10}
%Number with units: \SI{1e-10}{\meter\per\second}

%production packages, should be removed later
\usepackage{color}
% 
\usepackage{soul}

%% The lineno packages adds line numbers. Start line numbering with
%% \begin{linenumbers}, end it with \end{linenumbers}. Or switch it on
%% for the whole article with \linenumbers after \end{frontmatter}.
%% \usepackage{lineno}

%% natbib.sty is loaded by default. However, natbib options can be
%% provided with \biboptions{...} command. Following options are
%% valid:

%%   round  -  round parentheses are used (default)
%%   square -  square brackets are used   [option]
%%   curly  -  curly braces are used      {option}
%%   angle  -  angle brackets are used    <option>
%%   semicolon  -  multiple citations separated by semi-colon
%%   colon  - same as semicolon, an earlier confusion
%%   comma  -  separated by comma
%%   numbers-  selects numerical citations
%%   super  -  numerical citations as superscripts
%%   sort   -  sorts multiple citations according to order in ref. list
%%   sort&compress   -  like sort, but also compresses numerical citations
%%   compress - compresses without sorting
%%
%% \biboptions{comma,round}

% \biboptions{}

% own newcommands
\newcommand{\bfm}[1]{\boldsymbol{#1}}
\newcommand{\pp}[2]{\frac{\partial{#1}}{\partial{#2}}}
\newcommand{\ppsd}[2]{\frac{\partial^2 #1}{\partial #2^2}}
% \newcommand{\bm}[1]{\mbox{\boldsy $ #1 $ \unboldmath} \!\!}
\newcommand{\nop}{\bfm \nabla}
\usepackage{xspace}
\newcommand{\Karman}{K\'arm\'an\xspace}
% \newcommand{\von}{\textsc{von} }
% \newcommand{\Von}{\textsc{Von} }
% \newcommand{\Liepmann}{\textsc{Liepmann} }
% \newcommand{\Modified}{\textsc{Modified} }
% \newcommand{\modified}{\textsc{modified} }
% \newcommand{\Gauss}{\textsc{Gauss} }
% \newcommand{\Gaussian}{\textsc{Gaussian} }


\usepackage[english]{varioref}
	\labelformat{equation}{(#1)}

\usepackage{mathrsfs}
\usepackage{amsbsy}
\DeclareMathOperator\sgn{sgn}
\DeclareMathOperator\const{const}
\newcommand{\abl}{\text{d}}
% \newcommand{\const}{\text{const}}
\newcommand{\nmax}{{n_\text{max}}}
\newcommand{\nmin}{{n_\text{min}}}
%\usepackage{dsfont}
%\newcommand{\N}{\mathds{N}}  %menge der natürlichen zahlen
\newcommand{\ordnung}{\mathscr{O}} %Ordnung
\newcommand{\U}{\mathscr{U}} %White noise field
\newcommand{\R}{\mathscr{R}} %cross covariance
\renewcommand{\L}{\mathcal{L}} %lift function
\newcommand{\G}{\mathcal{G}} %gaussian transform

\newcommand{\x}{{\bfm x}}
\newcommand{\T}{{\bfm T}}
\renewcommand{\r}{{\bfm r}}
\renewcommand{\u}{{\bfm u}}
\newcommand{\e}{{\bfm e}}

\newcommand{\pAbl}[2]{\frac{\partial #1}{\partial #2}}
\newcommand{\dAbl}[2]{\frac{\abl #1}{\abl #2}}
\newcommand{\uline}[1]{\underline{#1}}
\newcommand{\uuline}[1]{\underline{\underline{#1}}}

\newcommand{\bpm}{\begin{pmatrix}}
\newcommand{\epm}{\end{pmatrix}}
\newcommand{\bit}{\begin{itemize}}
\newcommand{\eit}{\end{itemize}}
%
\usepackage{mathrsfs}%\mathscr{Altdeutsche Buchstaben}
\usepackage{subfigure}
\usepackage{subfigmat}% matrices of similar subfigures, aka small mulitples
\usepackage{pdflscape}
\usepackage{multirow}

\renewcommand{\thefootnote}{\fnsymbol{footnote}}
% \renewcommand{\cite}{\citep}

\usepackage{enumitem} % Less space in itemize use: \begin{itemize}[noitemsep,nolistsep]

\usepackage{tikz} %Für Diagramme und Baumdiagramme
\usetikzlibrary{fit,shapes}
\usetikzlibrary{trees}

\usepackage{wrapfig}

\usepackage{overpic}
\usepackage{xargs}
% %%%%%%%%final%%%%%%%%%%%%

% \newcommand{\highlight}[1]{#1}
\usepackage[disable]{todonotes}

% %%%%%%%%review%%%%%%%%%%%%
% \usepackage[colorinlistoftodos,prependcaption,textsize=tiny]{todonotes}
% \usepackage[xcolor=dvipsnames]{xcolor}
% \newcommandx{\todoT}[3][1=]{\todo[linecolor=blue,backgroundcolor=blue!25,bordercolor=blue,#1]{{\bf Reviewer\#1:} #2 \break{\bf Comment: }#3 }}
% \newcommandx{\todoU}[3][1=]{\todo[linecolor=blue,backgroundcolor=blue!25,bordercolor=blue,#1]{{\bf Rewiever\#2:} #2 \break{\bf Comment: }#3 }}
% \newcommand{\highlight}[1]{{\color{blue} #1}}
\setlength{\marginparwidth}{3cm}
\newcommandx{\unsure}[2][1=]{\todo[linecolor=red,backgroundcolor=red!25,bordercolor=red,#1]{#2}}
\newcommandx{\change}[2][1=]{\todo[linecolor=blue,backgroundcolor=blue!25,bordercolor=blue,#1]{#2}}
\newcommandx{\info}[2][1=]{\todo[linecolor=green,backgroundcolor=green!25,bordercolor=green,#1]{#2}}
\newcommandx{\improvement}[2][1=]{\todo[linecolor=violet,backgroundcolor=violet!25,bordercolor=violet,#1]{#2}}
\newcommandx{\missing}[2][1=]{\todo[linecolor=red,backgroundcolor=red!25,bordercolor=red,#1]{#2}}


  % Define commands to assure consistent treatment throughout document
 \newcommand{\eqnref}[1]{(\ref{#1})}
 \newcommand{\class}[1]{\texttt{#1}}
 \newcommand{\package}[1]{\texttt{#1}}
 \newcommand{\file}[1]{\texttt{#1}}
 \newcommand{\BibTeX}{\textsc{Bib}\TeX}
\newcommand{\bbf}[1]{\mbox{\boldmath $#1$}}

\journal{Journal of Sound and Vibration}

\begin{document}


% \todo[inline]{SG: Der Abstract sollte mehr auf die Erkenntnisse angehen }
% \todo[inline]{SG: Intro ist noch teilweise konfuse oder nicht genug präzise}
%\todo[inline]{SG: Es fehlt einen Teil über die Berechnung der TLS für den X-XC Fälle. TLS= Einfache Umfangsmittelung der TLS Werte ohne Wichtung mit TKE  }

%%%%
%todo
\setcounter{page}{1}
\pagenumbering{roman}
 \listoftodos
% 

\clearpage
\begin{frontmatter}
\setcounter{page}{1}
\pagenumbering{arabic}

%% Title, authors and addresses

%% use the tnoteref command within \title for footnotes;
%% use the tnotetext command for the associated footnote;
%% use the fnref command within \author or \address for footnotes;
%% use the fntext command for the associated footnote;
%% use the corref command within \author for corresponding author footnotes;
%% use the cortext command for the associated footnote;
%% use the ead command for the email address,
%% and the form \ead[url] for the home page:
%%
%% \title{Title\tnoteref{label1}}
%% \tnotetext[label1]{}
%% \author{Name\corref{cor1}\fnref{label2}}
%% \ead{email address}
%% \ead[url]{home page}
%% \fntext[label2]{}
%% \cortext[cor1]{}
%% \address{Address\fnref{label3}}
%% \fntext[label3]{}

% \title{Extension of the Random Particle Mesh method to \\periodic turbulent flows for \\fan broadband noise prediction}
% \title{Fan broadband noise prediction by extension of \\ 
%       the random particle mesh method to cyclostationarity}
%  \title{   Method to investigate the impact of wake cyclostationarity on fan
%  broadband noise    }
\title{Impact of cyclostationarity on \\fan broadband noise prediction}

\author{A. Wohlbrandt\footnote{Corresponding Author: attila.wohlbrandt@dlr.de,\\ Telephone +49 30 310006-21, Fax +49 30 310006-39 }}
\author{C. Kissner}
\author{S. Gu\'erin}
% \author[asta]{R. Ewert}
\address{Institute of Propulsion Technology, Engine Acoustics Department\\ German Aerospace Center (DLR), M\"uller-Breslau-Str.8, 10623 Berlin, Germany}
% \address[asta]{Institute of Aerodynamics and Flow Technology, Technical Acoustics\\ German Aerospace Center (DLR), Lilienthalplatz 7, 38108 Braunschweig, Germany}


\begin{abstract}
%% Text of abstract
One of the dominant noise sources of modern Ultra High Bypass Ratio (UHBR) engines is the interaction of the rotor wakes with the leading edges of the stator vanes in the fan stage.  While the tonal components of this noise generation mechanism are fairly well understood by now, the broadband components are not.  This calls to further the understanding of the broadband noise generation in the fan stage.
This article introduces the cyclostationary stochastic hybrid (CSH) method, which accommodates in-depth studies of the impact of cyclostationary wake characteristics on the broadband noise in the fan stage.  The Random Particle Mesh (RPM) method is used to synthesize a turbulence field in the stator domain using a URANS simulation characterized by time-periodic turbulence and mean flow. The rotor-stator interaction noise is predicted by a two-dimensional CAA computation of the stator cascade. The impact of cyclostationarity is decomposed into various effects investigated separately. This leads to the finding that the periodic turbulent kinetic energy (TKE) and periodic flow have only a negligible effect on the radiated sound power. The impact of the periodic integral length scale (TLS) is, however, substantial. The limits of a stationary representation of the TLS are demonstrated making the CSH method indispensable when background and wake TKE are of comparable level. Good agreement of the CSH method with measurements obtained from the 2015 AIAA Fan Broadband Noise Prediction Workshop are also shown. 

%For AIAA: 100 to 200 words (maximum) in one paragraph, no numerical references, acronyms or abbreviations.

\end{abstract}

\begin{keyword}
%% keywords here, in the form: keyword \sep keyword

%% MSC codes here, in the form: \MSC code \sep code
%% or \MSC[2008] code \sep code (2000 is the default)
Isotropic Turbulence \sep
Fan Broadband Noise Simulation \sep 
Computational Aeroacoustics \sep
Cyclostationary Turbulence \sep   
Fast Random Particle Mesh Method
\end{keyword}
\end{frontmatter}

% \input{01_highlights}
% \improvement[inline]{Seb: Unterschied zu AIAA 2015 und die Konferenz ausreichend erwähnt. Vorbeugen von Plagiatsvorwürfen. Evtl. doch bei AIAA einreichen.}
% \improvement[inline]{Überprüfe: Abkürzung RPM, fRPM oder (F)RPM verwenden. Ich präferiere fRPM für das Tool und RPM für die Methode. \\CK: Im Methodenteil finde ich es manchmal recht schwer zwischen Tool und Methode zu differenzieren.} 
% \improvement[inline]{Überprüfe: Haben wir erwähnt, wie wir die zeitlich und örtlich veränderliche Längenskala mit Filtern für kosntante Längenskalen erzeugen? - Ja. Intro (9. Absatz) und im Methodenteil auch} 

\section{Introduction}
%%%%%%%%%%%%%%%%%%%%%%%%%%%%%%%%%%%%%%%%%%%%%%%%%%%%%%
%   SITUATION
%   1. What is the background of the topic?
%   2. Why is the topic important?
%%%%%%%%%%%%%%%%%%%%%%%%%%%%%%%%%%%%%%%%%%%%%%%%%%%%%%

% \todo[inline]{Insgesamt finde ich, dass die Anzahl der Referenzen ausreicht.  Ich würde die Einleitung ungern aufblasen, indem man detaillierter auf weitere Paper eingeht.  Auflistungen sind aber vielleicht nicht schlecht (mögliche Stellen in Fett).  Eventuell dabei den Fokus auf Journal Artikel setzen.}

%CITATIONS

Current and future engines used in civil aviation have large bypass ratios meaning that the fan plays an ever increasing role as a noise source. Fan noise, in particular rotor-stator-interaction (RSI) noise, is one of the most dominant noise sources of an ultra-high bypass ratio (UHBR) engine.  It has the largest contribution during the approach phase and is only surpassed by jet noise during the take-off phase. Its prediction is of an increasing importance for the development of new technologies in light of the overall growth in air traffic and progressively more stringent noise regulations.

%\todo[inline]{Seb: References, was ist mit Linern? \\ CK: added liners to the list, although I'm not if the list needs to be exhaustive as the beginning of the sentence "These approaches include...".  I also wouldn't use references unless we just list relevant paper after each point.}
The tonal components of this noise generation mechanism have been researched extensively.  As a result, different methods were successfully applied to reduce the tonal RSI noise.  These approaches include reducing the tip circumferential speed to subsonic speeds, using acoustic liners in the engine duct, increasing the rotor-stator gap, choosing certain blade counts to strategically use acoustic cut-off effects, and modifying the blade geometry to e.g. reinforce destructive radial interferences.  Due to a reduction of tonal noise, the relative contribution of the broadband noise has significantly increased.  Hence, a greater understanding of the broadband noise generation mechanism in the fan stage is required to further reduce the RSI noise. 


%%%%%%%%%%%%%%%%%%%%%%%%%%%%%%%%%%%%%%%%%%%%%%%%%%%%%%
%   COMPLICATION
%		-limitation, short coming, lack of understanding of current methods, approaches
%%%%%%%%%%%%%%%%%%%%%%%%%%%%%%%%%%%%%%%%%%%%%%%%%%%%%%
%\todo{Mention \citet{shur_effect_2016} here? No - didn't cite for analytical methods, so let's not cite anything for scale-resolving methods.}
However, the prediction of fan broadband noise is still considered to be a challenge:  On one hand, analytical models are restrictive as they require strong assumptions.  On the other hand, CFD computations that fully resolve turbulent scales are exceedingly demanding in computational resources.   To advance the current understanding of broadband noise generation in a fan stage, methods are needed that are both fast and affordable without being overly restrictive.

Hybrid approaches fill the gap, if it is deemed possible to divide the physical problem into individual phenomena which can be calculated sequentially. This results in a process chain where every task is completed by the most efficient method respectively.  Hybrid approaches can combine numerical, analytical and empirical methods. In broadband noise predictions the two most prominent ones are Large Eddy Simulations (LES) coupled to an acoustic analogy and stochastic methods coupled to a Computational AeroAcoustics (CAA) method. This paper focuses on the latter method. \citet{allan_comparison_2014} compared the two mentioned methods and concluded that the hybrid approach relying on a stochastic method yields satisfactory noise results at a fraction of the cost of LES.

For RSI noise, a hybrid approach can be divided into three main tasks: Firstly, the sound generation mechanisms are modeled either directly or by synthesizing a turbulent field which impinges on the blade row.  Secondly, the sound is propagated considering complex duct geometries and flow. Lastly, the sound is radiated into the far field, i.e. to an observer.  The two last mentioned parts can be realized by a CAA simulation applying the Linearized Euler Equations (LEE). %As an alternative for the last step, an analytical far field extrapolation method is sufficient and even more efficient. \todo[inline]{Wenn die Alternative besser ist wie es dieser Satz impliziert, dann kann man sich als Leser fragen, warum wir das nicht so gemacht haben.  Ich würde es generell weglassen.  Das macht einen ohnehin komplizierten Absatz noch unübersichtlicher.} 
The first part, however, has proven to be the crux of the matter. For many years, the only way of modeling the sound sources was to use discrete harmonic gusts to generate the turbulent field~\cite{amiet_high_1976,scott_finite-difference_1995,peake_influence_2004,glegg_panel_2010}.  This method is still in use to model RSI noise. In fact, \citet{lau_effect_2013} have recently investigated the impingement of harmonic gusts in a three-dimensional (3D) CAA simulation to investigate the influence of wavy leading edges.

Aside from this analytically motivated method, two classes of stochastic methods are used to model broadband noise:  the Stochastic Noise Generation and Radiation (SNGR) method and the Random Particle Mesh (RPM) method.

The SNGR methods apply a random set of superposed Fourier modes to realize a target model spectrum, e.g. a von \Karman or a Liepmann spectrum. The SNGR methods can be traced back to the work of \citet{kraichnan_diffusion_1970}, who proposed the theoretical framework, and to \citet{bechara_stochastic_1994}, who was the first to apply it to predict noise generated by free turbulence. \citet{clair_experimental_2013} predicted the effects of wavy leading edges (LE) of isolated airfoils, while \citet{gill_reduced_2014} investigated real symmetric airfoils at zero angle of attack. To predict RSI noise \citet{polacsek_numerical_2015} have presented an approach simulating only one stator vane with periodic boundary conditions in circumferential direction. The far-field signature is obtained by extrapolating the instantaneous pressure on the blade surface.

The RPM method by \citet{ewert_caa_2011} synthesizes the turbulent fluctuations by spatially filtering white noise. The mean turbulent quantities are taken from a preceding RANS simulation. This method is now established and has been successfully applied to model different sources such as jet noise~\cite{ewert_three-parameter_2012}, slat noise~\cite{ewert_broadband_2008}, haystacking~\cite{siefert_sweeping_2009} or airfoil self noise~\cite{cozza_broadband_2012}. For the prediction of fan noise, the method is also applied. \citet{kim_advanced_2015} applied it to investigate the turbulence interaction with a flat plate in two- and three-dimensional space. They generated a von \Karman model spectrum by an optimization technique utilizing a set of Gaussian and Mexican hat filters. The divergence-free turbulence is coupled into a CAA domain by a sponge-layer technique. For centrifugal fans \citet{heo_unsteady_2015} have applied the RPM method to time-periodic flow with cyclostationary turbulence. The synthetic fluctuations are used as sources in an acoustic analogy solved by a boundary element method. A sufficient agreement with measurements is only achieved if cyclostationarity is considered.

% In previous years, we have investigated various methods for coupling turbulent
% sources into the CAA domain\cite{wohlbrandt_simultaneous_2013}.  However, the
% most reliable method proved to be to impose turbulent fluctuations upstream of
% the stator vanes. Three predominant methods have emerged: \textbf{(1) a
% manipulation of Tam's radiation boundary condition~\cite{tam_dispersion-relation-preserving_1993}, (2) the sponge approach, and (3) the LEE-relaxation method~\cite{ewert_linear-_2014}.} In this paper, the latter is used.  The advantage is that it generates turbulent fluctuations without influencing the acoustic radiation and thus, the seeding region can be positioned close to the source region.

To compare the results to experimental data, it is crucial to use realistic model spectra. \citet{dieste_random_2012} have derived complex filter stencils to model von \Karman spectra directly. This turns out to be computationally intensive. A more efficient solution consists in empirically weighting Gaussian filters with different length scales and amplitudes~\cite{kim_advanced_2015,hainaut_caa_2016}. \citet{wohlbrandt_analytical_2016} have derived analytical weighting functions in order to realize typical isotropic turbulence spectra by superposition of Gaussian spectra. They also showed that the reconstruction with five discrete realizations is sufficiently accurate to cover a frequency range with a change of one order of magnitude when the spectra are logarithmically distributed.  The current article will show how this technique can be used to simulate length scales varying in space and time while realizing temporally and spatially constant Gaussian filtered fields.

%%%%%%%%%%%%%%%%%%%%%%%%%%%%%%%%%%%%%%%%%%%%%%%%%%%%%%
%   RESEARCH QUESTION
%   - which question is this paper trying to answer to address complication
%%%%%%%%%%%%%%%%%%%%%%%%%%%%%%%%%%%%%%%%%%%%%%%%%%%%%%

Broadband noise in the fan stage is caused by the interaction of the turbulence in the rotor wakes with the surfaces at the leading edges of the stator vanes in the presence of a time-periodic mean flow.  %The focus of the paper is to develop a method, which facilitates the study of time-periodic parameters and their impact on the broadband noise levels.  In particular, the method allows to separately study the effects of time-periodic variations of length scale, turbulent kinetic energy, and mean flow. 
%%%%%%%%%%%%%%%%%%%%%%%%%%%%%%%%%%%%%%%%%%%%%%%%%%%%%%
%   SOLUTION 1
%   - how are we going to answer the research question in this paper
%%%%%%%%%%%%%%%%%%%%%%%%%%%%%%%%%%%%%%%%%%%%%%%%%%%%%%
%%%%%%%%%%%%%%%%%%%%%%%%%%%%%%%%%%%%%%%%%%%%%%%%%%%%%%
%   SOLUTION 2
%   - how is the discussed S1 contributing to science
%%%%%%%%%%%%%%%%%%%%%%%%%%%%%%%%%%%%%%%%%%%%%%%%%%%%%%
The first objective of the current paper is to utilize the RPM method to develop the cyclostationary stochastic hybrid (CSH) method to include time-periodic turbulence variations and mean flow, which are essential in studying broadband noise generation of rotating parts. 
This method allows for an in-depth study of cyclostationary turbulence and therefore contributes to a greater understanding of broadband noise generation in fans. This is especially important for the development and improvement of analytical tools. The second objective is to separately study the impact of the different effects due to cyclostationarity. A preliminary study utilizing this method was presented by \citet{wohlbrandt_extension_2015}. This article consolidates the method and applies it to another fan, for which measurement data have been made available.
% \todo[inline]{Ich denke, dass das der richtige Punkt ist um Unterschiede ganz deutlich auszuführen.  Ich denke nicht, dass Gleiches auf uns zutrifft wie auf das von Sebastien erwähnte Paper.  Die Liste der Unterschiede ist lang.  Die Formulierungen, Struktur, Bilder und Erkenntnisse sind auch unterschiedlich.  Ich fände es gut, das 2015 Paper tatsächlich als Vorstudie darzustellen.} 

%%%%%%%%%%%%%%%%%%%%%%%%%%%%%%%%%%%%%%%%%%%%%%%%%%%%%%
%   short outline of paper
%%%%%%%%%%%%%%%%%%%%%%%%%%%%%%%%%%%%%%%%%%%%%%%%%%%%%%

This paper is structured as follows:  The used hybrid approach, the extensions for including cyclostationarity, the general procedure for the setup of such a computation as well as evaluation methods are discussed in Section~\ref{sec:method}.  In Section~\ref{sec:Application}, the CSH method is demonstrated by applying it to the NASA Source Diagnostic Test (SDT) fan.  The effects of cyclostationary wake characteristics on the fan broadband noise are discussed in Section ~\ref{sec:results}.  Additionally, the numerical sound power spectra are compared to experimental data.  Key features of the method as well as significant findings are summarized in Section ~\ref{sec:conclusion}.      

% \todo[inline]{Outline}
% State-of-the-art:  give a short time line of broadband noise simulation (if we decide to not clutter the introduction with that)
% Method:  what needs to go in there?  LEE formulation for periodic background flow, application of VanVliet, Von \Karman... ? 
% Research setup: Q3D URANS, design of patches, modified geometry…
% Results:  show some pretty pictures, maybe discuss freq. peaks, parameter study?
% Discussion:  compare obtained numerical results with experimental data, discuss why simulating 1/11 of total fan is a justified simplification
% Conclusion/ Outlook:  give an idea of how we want to apply this method in the future,  what we want to study with itThe objective of the paper is to present

% \subsection*{Journals only (sorted by year)}
% \bit
% 	\item Ewert\cite{ewert_broadband_2008} 
% 		\bit
% 			\item Slat-noise with RPM
% 		\eit
% 	\item Jurdic et al.\cite{jurdic_investigation_2009}
% 	\item Ewert et al.\cite{ewert_caa_2011}
% 		\bit
% 			\item
% 		\eit
% 	\item Kim et al.\cite{kim_proposed_2010}
% 		\bit
% 			\item Optimization of sponge zone technique for gust-airfoil interaction
% 			\item Innovation 1: Forcing pressure instead of total energy (as we do anyways?)
% 			\item innovation 2: introducting an additional factor for impulse equations, which forces incomming harmonic gusts only upstream of 2D plate.
% 			\item also use 2L for sponge depth and forcing of 4$\sigma$
% 		\eit		
% 	\item Dieste and Gabard\cite{dieste_random_2012}
% 		\bit
% 			\item Alternatve Filters 
% 			\item No reference to cyclostationarity. Periodic TKE only mentioned in Disseration
% 			\item Evolving turbulence negligible for RSI.
% 			\item Restricted to flat plates
% 		\eit
% 	\item Cozza et al.\cite{cozza_broadband_2012}
% 		\bit
% 			\item RPM for TE-noise
% 		\eit
% 	\item Lau et al.\cite{lau_effect_2013}
% 		\bit
% 			\item Harmonic gusts for 3D-CAA to investigate wavy leading edges.
% 		\eit
% 	\item Clair et al.\cite{clair_experimental_2013}
% 		\bit
% 			\item 3D Non-linear LEE with HIT at inflow (Tam's BC)
% 			\item Investigating LE serrations. Could show positive influence of serations
% 			\item Fourier-modes with von \Karman spectrum to realise wall-normal component only
% 			\item Spanwise gust contribution neglected
% 			\item Far field  data optained by FWH with unsteady wall pressure and porous formulation 
% 		\eit
% % 	\item Haeri et al.~\cite{aeri_calculations_2014} %conference proceeding
% % 		\bit
% % 			\item Extension of investigation by Lau et al.\cite{lau_effect_2013} to Fourier-Modes
% % 		\eit
% 	\item Gabard~\cite{gabard_noise_2014}
% 		\bit
% 			\item builds on method by Polacsek et al.\cite{polacsek_equivalent-source_2009}
% 			\item Generates in duct broadband noise field characterised by its modal using the duct shape function.
% 			\item Generates stochastic noise source able to produce multimode, broadband sound fields
% 			\item truely random.
% 		\eit
% 	\item Kim and Haeri\cite{kim_advanced_2015}
% 		\bit
% 			\item synthetic eddy method (SEM) derived from taking the curl of a vector potential function \todo{isn't that a stream function then?} to create divergence-free velocity field.
% 			\item Full 3D
% 			\item Mixture of Gaussian filters and mexican hat filters to empirically generate von \Karman spectrum by using optimisation strategies
% 			\item Upstream coupling by usage of Sponge-layer
% 		\eit
% 	\item Heo et al.\cite{heo_unsteady_2015}
% 		\bit
% 			\item Couples FRPM to BEM 
% 			\item uses URANS for turbulent statistics
% 			\item Incorporating unsteady flow field into FRPM (U-FRPM)
% 			\item Applies time-variant turbulence kinetic energy and turbulence dissipation extracted from unsteady RANS solutions.
% 			\item See tonal and broadband noise as result.
% 			\item No unsteady CAA as BEM is used
% 		\eit
% 	\item Ayton and Peake\cite{ayton_high-frequency_2015}
% 	\item Ju et al.\cite{ju_investigation_2015}
% 	\item Polacsek et al.\cite{polacsek_numerical_2015}
% \eit
% \subsection*{Conferences 2014-2016 (sorted by year)}
% \bit
% 	\item Allan and Darbyshire~\cite{allan_comparison_2014}
% 		\bit
% 			\item Compare LES-Acoustic-Analogy to RPM-CAA 
% 			\item Also applicable in  maritime acoustics
% 			\item conclude that RPM-CAA can provide satisfactory noise results at a fraction of the costs.
% 			\item no full paper access		
% 		\eit
% 	\item Gill et al.\cite{gill_reduced_2014}	
% 		\bit
% 			\item 
% 		\eit
% 	\item Shur et al.\cite{shur_effect_2016}	
% 		\bit
% 			\item IDDES for a full ventilator
% 		\eit
% 	\item Hainaut et al.\cite{hainaut_caa_2015,hainaut_caa_2016}	
% 		\bit
% 			\item
% 		\eit
% 	\item Haeri et al.\cite{haeri_3d_2014}
% 		\bit
% 			\item
% 		\eit
% 	\item Santana et al.\cite{santana_boundary_2014}
% 		\bit
% 			\item
% 		\eit
% 	\item Ayton et al.\cite{ayton_comparison_2015}
% 		\bit
% 			\item unsteady airfoil interaction comparing experiment, analytical and numerical
% 			\item Method by Gill et al.\cite{gill_symmetric_2013}
% 			\item Improve match between analytical and numerical solution if singularity at TE is incorporated into analytical solution
% 		\eit
% 	\item Bouley et al.\cite{bouley_mode-matching_2015}
% 		\bit
% 			\item
% 		\eit
% 	\item Gea-Aguilera et al.\cite{gea-aguilera_synthetic_2015,gea-aguilera_leading_2016}
% 		\bit
% 			\item
% 		\eit
% 	\item Grace\cite{grace_further_2015}
% 		\bit
% 			\item
% 		\eit
% 	\item Kim et al.\cite{kim_mechanisms_2015}
% 		\bit
% 			\item
% 		\eit
% 	\item Paruchuri et al.\cite{paruchuri_aerofoil_2015}
% 		\bit
% 			\item
% 		\eit
% 	\item Casalino et al.\cite{casalino_turbofan_2016}
% 		\bit
% 			\item
% 		\eit
% 	\item Geyer et al.\cite{geyer_noise_2016}
% 		\bit
% 			\item
% 		\eit
% \eit
% \subsection*{Our pubs}
% \bit
% 	\item \cite{wohlbrandt_simultaneous_2013}
% 		\bit
% 			\item
% 		\eit
% 	\item \cite{wohlbrandt_extension_2015}
% 		\bit
% 			\item
% 		\eit
% 	\item \cite{wohlbrandt_analytical_2016}
% 		\bit
% 			\item
% 		\eit
% \eit


Fig.~\ref{fig:method} presents the pipeline of our framework. Given an image that contains a person and a mirror, our goal is to recover the human mesh considering the mirror geometry. The key insight is that the person and his/her mirror image can be treated as two people, and we reconstruct them together with the mirror symmetry constraints. This section will be organized as follows. First, the formulation of single-person mesh recovery is introduced (Sec.~\ref{sec:spmr}). Then the mirror symmetry constraints that relate the two people will be elaborated (Sec.~\ref{sec:mi_geo}). Finally, the objective functions and the whole optimization are described (Sec.~\ref{sec:opt}).

\subsection{Human mesh recovery with SMPL model}
\label{sec:spmr}

We adopt the SMPL model~\cite{SMPL:2015} as our human representation. The SMPL model is a differentiable function $\bm M(\bm \theta, \bm \beta) \in \mathbb R^{3\times N_v}$ mapping the pose parameters $\bm \theta \in \mathbb R^{72}$ and the shape parameters $\bm{\beta} \in \mathbb R^{10}$ to a triangulated mesh with $N_v = 6890$ vertices. The 3D body joints $\bm J(\bm\theta, \bm\beta)$ of the model can be defined as a linear combination of the mesh vertices. Hence for $N_j$ joints, we defined the body joints $\bm J(\bm\theta, \bm\beta) \in \mathbb{R}^{3\times N_j} = \mathcal{J}(\bm M(\bm\theta, \bm\beta))$, where $\mathcal{J}$ is a pre-trained linear regressor. Let $\bm R \in SO(3)$ and $\bm T \in \mathbb R^3$ denote the global rotation and translation, respectively.

Given an image and the detected 2D bounding boxes, the 2D human keypoints $\bm W$ can be estimated with the cropped regions. The objective function for human mesh recovery generally consists of a reprojection term $L_{2d}$ and a prior term $L_p$ with respect to variables $\bm \theta$, $\bm \beta$, $\bm R$ and $\bm T$.

The reprojection term penalizes the weighted 2D distance between the estimated 2D keypoints $\bm{W}$ with the confidence $c$, and the corresponding projected SMPL joints:
\begin{equation}
    L_{2d} = \sum_i c_i\rho(\bm{W}_i - \Pi_K(\bm{R}\bm{J}(\bm\theta, \bm\beta)_i + \bm{T})),
\end{equation}
where $\Pi_K$ is the projection from 3D to 2D through the intrinsic parameter $K$. $\rho$ denotes the Geman-McClure robust error function for suppressing noisy detections. 

The human body priors are used to encourage realistic 3D human mesh results. Since the pose and shape parameters ($ \bm{\Tilde{\theta}}, \bm{\Tilde{\beta}}$) estimated by a neural network can be viewed as learned prior, the final results are supposed to be close to them:
\begin{equation}
    L_{p} = ||\bm{\theta} - \bm{\Tilde{\theta}}||_2^2  + \lambda_{\beta}|| \bm{\beta} - \bm{\Tilde{\beta}}||_2^2,
\end{equation}
where $\lambda_{\beta}$ is a weight.

\subsection{Mirror-induced constraints}
\label{sec:mi_geo}
If there is a mirror in the image, the relation between the person and the mirrored person can be used to enhance the reconstruction performance. This relation is a simple reflection transformation if the mirror geometry is known, which however is impracticable for an arbitrary image from the Internet. To tackle this problem and take advantage of the characteristic of the mirror, the following mirror-induced constraints are introduced, as illustrated in Fig.~\ref{fig:mirrorsym}. Note that all symbols with the superscript prime refer to variables related to the mirrored person unless specifically mentioned.

\paragraph{Mirror symmetry constraints:}
Since the adopted human representation disentangles the orientation $\bm R$, pose parameters $\bm \theta$ and shape parameters $\bm \beta$, $\bm \beta$ can be shared by the person and the mirrored person, and $\bm \theta$ is related to $\bm \theta'$ by a simple reflection operation as follows:
\begin{equation}
\label{eq:param}
    \bm{\beta}' = \bm{\beta}, ~\bm{\theta}' = \mathcal{S}(\bm \theta),
\end{equation}
where $\mathcal{S}(\cdot)$ denotes the reflection operation on axis angles. 
\begin{figure}[t]
\centering
\includegraphics[trim=3cm 18.5cm 10cm 3.5cm, width=0.8\linewidth,clip]{figures/mirrorloss.pdf}
\caption{\textbf{An illustration of mirror-induced constraints.} The line segment connecting the joint $\bm{J}_i$ and its mirrored joint $\bm{J}_i'$ has the direction $\bm n_i$ and the middle point $\bm p_i$. Theoretically, $\bm{n}_i // \bm{n}_j$, and $\bm{n}_i \perp \overline{\bm{p}_i\bm{p}_j}$. If the mirror normal $\bm{n}$ (red arrow) is known, $\bm{n} // \bm{n}_i$ and $\bm{n} // \bm{n}_j$ should be satisfied as well.}
\label{fig:mirrorsym}
\end{figure}

As Eq.~\ref{eq:param} does not take $\bm R$ and $\bm{T}$ into consideration, the constraint on 3D keypoints can be imposed to estimate the human orientation and position better. We abbreviate the global coordinates of the $i$-th joint $\bm{R}\bm{J}(\bm\theta, \bm\beta)_i + \bm{T}$ as $\bm{J}_i$. Given a pair of body joints $i, j$, we denote the direction of the line segment $\overline{\bm{J}_i\bm{J}_i'}$, $\overline{\bm{J}_j\bm{J}_j'}$ as $\bm{n}_i$, $\bm{n}_j$ and the middle point of them as $\bm p_i$, $\bm p_j$, respectively. Ideally, $\bm n_i$ should be parallel to $\bm n_j$ and $\bm p_i, \bm{p}_j$ are supposed to be on the mirror plane. Despite the fact that the mirror geometry is unknown, it needs to be satisfied that $\bm{n}_i$ is perpendicular to the line  $\overline{\bm{p}_i\bm{p}_j}$. So for 
any pair of joints, we minimize the sum of the L2 norm of the cross product between $\bm{n}_i$ and $\bm{n}_j$, and the inner product between $\bm{n}_i$ and $\bm{p}_j - \bm{p}_i$: 
\begin{equation}\label{eq:mirrorsym}
    L_{s} = \sum_{(i, j)}(||\bm{n}_i \times \bm{n}_j||_2 + || \bm{n}_i\cdot (\bm{p}_j - \bm{p}_i) ||_2).
\end{equation}

\paragraph{Mirror normal constraint:}
A mirror can be represented as a plane, parameterized as its normal and position. If its normal $\bm{n}$ is known, the geometric properties of the mirror can thus be utilized explicitly by constraining $\bm n_i$ and $\bm n$ to be parallel with the following loss function:
\begin{equation}\label{eq:mirrorgt}
    L_{n} =  \sum_i||\bm{n} \times \bm{n}_i||_2. 
\end{equation}
\vspace{-0.5cm}
\begin{figure}[t]
	\centering
	\includegraphics[width=1\linewidth,trim={8cm 8cm 8cm 7.5cm},clip]{figures/vp_small.pdf}
	\includegraphics[width=\linewidth, trim={2.5cm 2cm 1cm 3cm},clip]{figures/vpdemo.pdf}
	\vspace{-0.8cm}
	\caption{\textbf{Vanishing points in an image containing a person and a mirror.} In most cases at least two vanishing points can be found, where $\bm v_0$ comes from 2D human keypoints, and $\bm v_1$ comes from the annotated mirror edges. $O_c$ denotes the camera center. Note that $\overline{O_c v_0} // \bm n$ and $\overline{O_c v_1} \perp \bm n$, where $\bm n$ is the mirror normal.
	}
	\label{fig:vp}
\end{figure}
\paragraph{Mirror normal estimation:}
Though the mirror normal is not directly available, the vanishing points can be used to estimate it. The vanishing point of lines with direction $\bm n$ in 3D space is the intersection $\bm v$ of the image plane with a ray through the camera center with direction $\bm n$~\cite{hartley2003multiple}:
\begin{equation}
\bm v=K \bm n,
\label{eq:vanish}
\end{equation}
where the vanishing point $\bm v\in \mathbb{R}^3$ is in the form of homogeneous coordinates and $K$ is the camera intrinsic matrix. 

Eq.~\ref{eq:vanish} reveals that obtaining the mirror normal $\bm n$ requires both $K$ and $\bm v$. As the parallel lines connecting points on the real object and corresponding points on the mirrored object are perpendicular to the mirror, the vanishing point $\bm v$ with this direction can be estimated through their 2D positions.
To get such correspondences, some previous works require additional inputs such as masks \cite{Hu2005MultipleView3R}, which is infeasible for images from the Internet.
Fortunately, since 2D human keypoints provide robust semantic correspondences, \eg the left ankle of the real person and the right ankle of the mirrored person, this vanishing point can be acquired naturally and automatically in our setting ($\bm v_0$ in Fig.~\ref{fig:vp}).

Note that if the intrinsic matrix $K$ is provided, the mirror normal can thus be solved easily through $\bm n=K^{-1} \bm v_0$, otherwise $K$ should be calibrated from a single image if possible. From the projective geometry~\cite{hartley2003multiple}, we know that it is possible to calibrate the camera intrinsic parameters from a single image. Suppose the camera has zero skew and square pixels. The intrinsic matrix $K$ can be computed via three orthogonal vanishing points. Additionally, if the principal point is assumed to be in the image center (only the focal length is unknown), $K$ can be computed via only two orthogonal vanishing points. Please refer to the supplementary material for more details. 

As we have stated, one vanishing point $\bm v_0$ has been acquired based on reliable 2D human keypoints. Different from the general scene where finding orthogonal relations may be difficult, our setting contains richer information. Fig.~\ref{fig:vp} shows that if we annotate the mirror edges, at least one vanishing point $\bm v_1$ orthogonal to $\bm v_0$ can be obtained. With these vanishing points, the calibration can be performed. Note that images from the same video share the same intrinsic matrix $K$, thus the annotation process is not laborious.

The mirror normal constraint is optional, which depends on how easy it is to find mirror edges. In the experiment, we will show that our method can still achieve satisfactory performance without the mirror normal constraint.

\subsection{Objective function and optimization}
\label{sec:opt}
Combining all discussed above, the final objective function to optimize can be written as:
\begin{equation}
\label{eq:loss0}
\begin{split}
    \min_{\substack{\Theta, \Theta'}}~L_{2d}+L_{2d}' + \lambda_p (L_{p}&+L_{p}') + \lambda_s L_s + \lambda_n L_n \\
     s.t.~~ \bm{\beta}' = \bm{\beta}&, ~\bm{\theta}' = \mathcal{S}(\bm \theta),
\end{split}
\end{equation}
where $\Theta=\{\bm\theta, \bm\beta, \bm R, \bm T\}$ and $\Theta'=\{\bm\theta', \bm\beta', \bm R', \bm T'\}$. $L_{2d}'$ and $L_p'$ refer to the reprojection term and the prior term of the mirrored person, respectively. $\lambda_p$, $\lambda_s$ and $\lambda_n$ are weights. $\lambda_n$ is set to zero whenever the mirror normal is unavailable. If there are two or more people, the optimization can be done for each subject separately.

We optimize Eq.~\ref{eq:loss0} with respect to all parameters using L-BFGS and PyTorch. An off-the-shelf model~\cite{kolotouros2019spin} is adopted to generate the initial estimation. Given the 2D keypoints~\cite{sun2019deep, cao2017realtime}, $\bm R$ and $\bm T$ are further optimized by aligning the initial SMPL model to the 2D keypoints. To improve the robustness of the initialization, we select the person with smaller reprojection error and apply the selected pose parameter to the other person after a reflection operation. 

























































\section{Application}\label{sec:Application}


The CSH method described in the previous section was applied to NASA's 22-in Source Diagnostic Test (SDT) fan at approach condition.  The effects of cyclostationary parameters were examined and the numerical data were compared to experimental data presented at the Fan Broadband Noise Prediction Workshop organized in the framework of the AIAA 2014 and 2015 Aeroacoustics Conferences.  The experimental setup of the Realistic Test Case 2 (RC2) was described in detail by \citet{nallasamy_computation_2005}.  The operating conditions are given in table~\ref{tab:operatingPoint} and the input specifications, in the workshop problem statement~\cite{envia_panel_2015}.    

% \todo[inline]{CK: Brauchen wir sowohl Approach und Design?  Mit Design haben wir nicht gerechnet.  Und falls wir Design da lassen, müssten wir vielleicht alle punkte angeben.
% \\AW: Habe einen Satz in die Tabelle eingefügt, \\AW@SG: was sagst du dazu?} 

\begin{table}
		\centering
		\caption[Fan characteristics]{Fan characteristics and operating points of the NASA-SDT fan. Design point is not used, but shown here for reference. \label{tab:operatingPoint}}%~\cite{hughes_aerodynamic_2002,moreau_measurements_2013,moreau_unified_2016} 
		\begin{tabular}{|l|r|r|}
		\hline
		Fan diameter   		& \multicolumn{2}{ c|}{\SI{0.56}{m}}	\\
		Rotor blade count $N_B$
							& \multicolumn{2}{ c|}{22} 			\\
		Stator vane count $N_V$ 
							& \multicolumn{2}{ c|}{54} 			\\
		\hline
		\hline
			&\textit{Design} &\textit{Approach} \\
		\hline
		Fan-Pressure ratio $\Pi$
							& 1.48			& 1.15		\\
		Axial Mach number $M_x$ in rotor plane
							& 0.59			& 	0.31	\\
		Relativ tip Mach number $M_\text{tip,rel}$
							&	1.39		&   0.80		\\ %M_tip= Omega*R/c_0; Mrel= sqrt(M_x^2+Mtip^2)
		\hline
		\end{tabular}
		\end{table}
% \afterpage{
% 		\begin{table}
% 		\caption[Randbedingungen für die beiden Fangeometrien]{Randbedingungen für den Betriebspunkt \textit{Approach} der beiden Fangeometrien. Alle Werte sind bei 50\% Kanalhöhe aus der Simulation extrahiert mit Ausnahme der turbulenten Eigenschaften zur Rotor-Stator-Interaktion-Simulation (RSI) des RC2-Fans, hier werden direkt die Werte aus der Hitzdrahtmessung an der Messebene 2 (Statorvorderkante) verwendet.\label{tab:operatingPointsBC}}
% 		\centering
% 		\begin{tabular}{|c|r|r|r|}
% 		\hline
% 		&					& RC2-Fan     		& UHBR-Fan \\
% % 		& Betriebspunkt		&\multicolumn{2}{c|}{\textit{approach}}\\
% 		\cline{2-4}
% 		& Drehgeschwindigkeit $n_\text{RPM}$	
% 							& 7808~rpm			& 3187~rpm\\
% 		\multirow{3}{*}[-2mm]{\parbox[c]{2mm}{\rotatebox[origin=c]{90}{RANS-Einlass}}}
% 		& Massenstrom $\dot m$
% 							& $\SI{26.92}{\kg\per\second}$ 
% 												& $\SI{47.33}{\kg\per\second}$ \\
% 		& Totaldruck $p_t$ 
% 							& 
% 							$\SI{101332.5}{\Pa}$					& $\SI{101325}{\Pa}$ \\
% 		& Totaltemperatur $T_t$ 
% 							& $\SI{288.16}{\K}$
% 												& $\SI{288.15}{\K}$ \\ 
% 		\cline{2-4}
% 		& Mach-Zahl $M_I$ 
% 							&  $\num{0.312}$ 	&$\num{0.22}$\\
% 		& TKE $k_{t,I}$ 
% 							& $\SI{0.11}{\meter^2\per\second^2}$
% 												& $\SI{2.94}{\meter^2\per\second^2}$\\
% 		& TLS $\Lambda_I$ 
% 							& $\SI{1.3e-3}{\meter} $
% 												&$\SI{4.5e-4}{\meter} $\\ 
% 		\hline
% 		\multirow{3}{*}[0mm]{\parbox[c]{2mm}{\rotatebox[origin=c]{90}{Für RSI}}} 
% % % 		Mach-Zahl $M$ (RSI)        
% % 							& 
% % 												&\\
% 		& TKE $k_{t}$
% 							& $\SI{32.57 }{\meter^2\per\second^2}$
% 												&$\SI{9.24}{\meter^2\per\second^2}$\\
% 		& TLS $\Lambda$
% 							& $\SI{5.1e-3}{\meter}$
% 												&$\SI{3.3e-3}{\meter}$,\\
% 		&					&					&($l_p     = \SI{6.9e-4}{\meter}$)\footnotemark\\ 
% 		\hline
% 		\multirow{3}{*}[0mm]{\parbox[c]{2mm}{\rotatebox[origin=c]{90}{q3D}}} 
% 		& Rotorblattabstand\footnotemark  $s_B$
% 								& $\SI{0.055}{\meter}$	& $\SI{0.0981}{\meter}  $\\
% 		& Statorblattabstand\addtocounter{footnote}{-1}\footnotemark $s_V$
% 								&   $\SI{0.024}{\meter}$	& $\SI{0.059}{\meter} $ \\
% 		& Statorsehnenlänge $c$	
% 								& 	$\SI{0.039}{\meter}$ 	& $\SI{0.081}{\meter}  $\\
% 		& Stator\textit{solidity} $c/s$	&	1.63					&	1.37				\\
% 		\hline
% 		\end{tabular}
% 		 \end{table}
% \addtocounter{footnote}{-1}
%  \footnotetext{In der Untersuchung am UHBR-Fan wird fälschlicher Weise die Pseudolängenskala $l_p$verwendet, nicht die integrale Längenskala $\Lambda$.}
% \addtocounter{footnote}{+1}
%  \footnotetext{An der radialen Position der Mischungsebene des q3D-Gitters.}
% }


\subsection{Description of applied procedure}

The CAA simulations were performed on two-dimensional cascade mesh at midspan of the annular duct at the stator leading edge. It is computed in two-dimensional space to reduce simulation cost and to, therefore, allow for parameter variations. 
%The 3D application is assumed straight forward, but has not been in the focus of the current investigation.
The solidity of the vanes is greater than one. Thus the cascade effect cannot be neglected~\cite{blandeau_comparison_2011} and all stator vanes had to be considered to correctly predict the sound propagation.  

\begin{figure}[h]
	\centering
%wohlbrandt@wood:~/RPM/fanBBN_workshop_AIAA2015/plots4JSV/lay_RC2_3DTo2D_JSV.lpk
	\begin{overpic}[width=\textwidth]{figures/eps_transform3Dto2D.pdf}
	\put( 5,55){\fcolorbox{black}{white}{1}}
	\put(45,55){\fcolorbox{black}{white}{2}}
	\put( 5,25){\fcolorbox{black}{white}{3}}
	\put(45,25){\fcolorbox{black}{white}{4}}
	\put(82,25){\fcolorbox{black}{white}{5}}
	\end{overpic}
	\caption{Transfomation of the fan geometry (1) into a two-dimensional cascade. Instantaneous axial component of the background flow in [m/s] is shown.\label{fig:RC2_transform}}
\end{figure}

% \todo[inline]{Zeige RPM patch in 4 and 5? Wäre natürlich vollständiger. }

To prepare the two-dimensional cascade domain for the CAA, a series of steps was followed as illustrated in Figure~\ref{fig:RC2_transform}. The numbers in the subsequent description correspond to the numbers in the figure. Text that appears in italics refers to topics which are further explained below the enumeration.
\begin{enumerate}
\item Firstly, the CAD geometry of the SDT fan with the baseline stator configuration has to be obtained.  For this case, the number of stator vanes was increased from 54 to 55.  The new number of vanes allowed for a reduced computation with fully periodic boundary conditions using only two rotor blades and five stator vanes.  This made the URANS and CAA computations more efficient.  The increase in the number of stator vanes was not expected to significantly change the broadband noise characteristics of the fan stage.       

\item Secondly, the CAD geometry was used to set up a three-dimensional RANS simulation for one blade passage. 

\item A q3D domain was generated by extracting streamlines at 49\%, 50\%, and 51\% of the duct height at the stator leading edge from the RANS computation.  From that a q3D URANS computation was set up with two cells in the radial direction for two rotor and five stator passages to achieve circumferential periodicity. 

\item In the next step, the streamline at \textit{50\% of the duct radius} was extracted from the q3D URANS in the stator domain only and expanded to either the full annular duct or to a fraction of the annular duct. The \textit{rotor was ignored}. The Fourier coefficients of flow and turbulent variables were then interpolated from this extracted and expanded CFD mesh onto the CAA mesh and the fRPM patch.  In areas where the CAA mesh lay outside of the  CFD stator domain, an \textit{extrapolation} was applied.  

\item Lastly, a transformation into \textit{stream-surface coordinates} was applied to the flow and geometry to produce a 2D cascade.               
\end{enumerate}
Now the keywords in italics are detailed.

\begin{figure}
% ~/RPM/fanBBN_workshop_AIAA2015/hotWireTurb/lay_hotWire_JSV.lay
		\subfigure[Turbulent kinetic energy (TKE).\label{fig:RC2_hotwire_TKE}]{
			\includegraphics[width=0.45\textwidth]{figures/eps_hotWire_TKE}}
		\subfigure[Turbulent integral lengthscale (TLS).\label{fig:RC2_hotwire_TLS}]{
			\includegraphics[width=0.45\textwidth]{figures/eps_hotWire_TLS}}
% 	\vspace{-1em}
	\caption{Turbulent characteristics determined through hotwire measurements by \citet{podboy_steady_2003} for the benchmark testcase~\cite{envia_panel_2015} at two axial positions: (1) half-way between rotor and stator and (2) in front of the stator LE. The dashed vertical line indicates the position of CAA simulation 50\%. (Reproduced with permission) \label{fig:hotwire}}
\end{figure}

\begin{description}
\item{Using a streamline at \textit{50\% of the duct radius}}
	for the q3D simulation is considered representative for the whole duct. For fan acoustic power measurements with omnidirectional microphones, this position is representative according to ISO 5136:1990~\cite{barret_noise_1960,arnold_experimentelle_1999}. Repeating the simulation at differing duct heights would increase the accuracy. The hotwire measurements reproduced in Fig.~\ref{fig:hotwire} show small variations of the turbulent characteristics in a considerably large area.  The maximum at the outer rim is due to a flow detachment in the rotor tip region. This effect cannot be accounted for here.
\item{\textit{The rotor was ignored}}
	as its consideration is computationally expensive. The error in sound radiation must be kept in mind but seems acceptable for subsonic flow without shocks \cite{moreau_impact_2016}. The main effect of the rotor is to block the acoustic waves propagating in the upstream direction. This shielding effect increases with the relative  Mach number of the rotor.    
\item{\textit{The Extrapolation}}
	of the flow quantities is used to allow for the use of a larger CAA domain, especially for the damping zones at the inflow and outflow boundaries. A nearest-neighbor interpolation was used. For the complex coefficients of the non-stationary flow data of the P-PP configurations, the phase cannot be considered as can be seen in the time-reconstructed flow at Steps 4 and 5 of Fig.~\ref{fig:RC2_transform}. The solution is nevertheless continuous and does not have a noticeable influence on the acoustic radiation.
\item{A \textit{stream-surface coordinate}}
	transformation from the q3D-grid into 2D-grid coordinates $m'$ and $\vartheta$ was applied as described in the Appendix~\ref{app:mtheta}. 
	 The benefits are that the flow quantities are transformational invariants and the circumferential distance is independent of the axially changing duct radius of the streamline. The latter allows for the application of periodic boundary conditions. For convenience, we use for the 2D grid coordinates
	\begin{align}
	x &= R_\text{LE} m', &&& y &= R_\text{LE} \vartheta,
	\end{align}
	with the radius at the leading edge $R_\text{LE}$. Hence, the chord length and pitch are invariant to the transformation.
\end{description}


\subsection{Definition of test matrix}
\label{sec:confRealised}
Eight different configurations were realized.  An overview of those configurations is given in Table~\ref{tab:conf}.  The abbreviations used to denote the test cases were introduced in Section~\ref{sec:config}.  The shown test matrix consists of five primary test cases investigating the effect of cyclostationarity and three secondary test cases confirming supplementary aspects. The same CAA grid resolution was used for all test cases.  

\begin{table}
\caption{Test matrix of simulated configurations \label{tab:conf}}
\centering\setlength\extrarowheight{2pt}
\begin{tabular}{|c|l|l|m{1cm}|m{1cm}|m{1cm}|m{1cm}|}
\hline
& type 			& specifications &
\rotatebox[origin=b]{90}{number} 
\rotatebox[origin=b]{90}{of vanes} &	
\rotatebox[origin=b]{90}{periodic}
\rotatebox[origin=b]{90}{mean flow}	& 
\rotatebox[origin=b]{90}{periodic}
\rotatebox[origin=b]{90}{TKE}          & 
\rotatebox[origin=b]{90}{periodic}
\rotatebox[origin=b]{90}{TLS} \\ \hline\hline
\multirow{5}{*}{\parbox[c]{2mm}{\rotatebox[origin=c]{90}{primary}}}
	& P-PP	&	        & 5	& 		yes		&	yes	&	yes \\ \cline{2-7}
	& C-PP	&	        & 5	& 		no		&	yes	&	yes \\ \cline{2-7}
	& C-PC	&	        & 5	& 		no		&	yes	&	no \\ \cline{2-7}
	& C-CC	&	        & 5	& 		no		&	no	&	no \\ \cline{2-7}
	& C-CC-tls	& alternative TLS	        & 5	& 		no		&	no	&	no \\ \hline \hline
\multirow{3}{*}{\parbox[c]{2mm}{\rotatebox[origin=c]{90}{secondary}}}
	& C-CC-55 & full cascade
						& 55	& 		no		&	no	&	no \\ \cline{2-7}
	& C-CC-red  & patch with reduced damping area	
						& 5	& 		no &			no	&	no \\ \cline{2-7}
	& C-CC-double  & double patch		        
						& 5	& 		no		&	no	&	no \\ \hline
\end{tabular}
\end{table}

The five primary test cases were simulated with only five stator vanes in order to reduce simulation times.  The boundary conditions for the reduced number of blades were still fully periodic, since the number of total vanes was increased to 55. For the primary test cases, the fRPM patch remained unchanged so that the influence of every parameter could be investigated.  The cases have been described in section~\ref{sec:config}. Figures~\ref{fig:PPPinCAA} and \subref{fig:CCCinCAA} show the instantaneous vorticity fields for the P-PP and the C-CC cases, respectively. The P-PP case is the most realistic test case realizing full cyclostationarity. The structure of the wake is reproduced in the mean flow, as shown by the contour of the axial velocity, and in the wake turbulence, as shown by the contour of the vorticity magnitude. The primary C-CC case uses a constant TKE, a constant TLS in fRPM domain, the so-called patch, and a constant mean flow in the CAA and fRPM domains.  Figure~\ref{fig:CCCinCAA} shows that the rotor wake structures are neglected and averages are used.  The C-CC-tls case also uses constant turbulent and flow variables but the TLS was determined using a different technique, which will be discussed in detail in Section~\ref{sec:results}.

\begin{figure}
%/usr/kiely/kissner/simulations/RC2/piano/TestMatrix_Final/output_P_PP_test/lay_VidMeanFlow_Vorticity.lay
		\subfigure[P-PP test case\label{fig:PPPinCAA}]{
			\includegraphics[width=0.50\textwidth,trim=4 4 4 4,clip]{figures/eps_PPP_MeanFlowVort.pdf}}
%/usr/kiely/kissner/simulations/RC2/piano/TestMatrix_Final/output_C_CC_test/lay_VidMeanFlow_Vorticity.lay		
		\subfigure[C-CC test case\label{fig:CCCinCAA}]{
			\includegraphics[width=0.50\textwidth,trim=4 4 4 4,clip]{figures/eps_CCC_MeanFlowVort.pdf}}
% 	\vspace{-1em}
	\caption{CAA domain of P-PP and C-CC test cases.  Contours show background axial velocity and turbulence vorticity magnitude.\label{fig:PPPCCCinCAA}}
\end{figure}

% \todo[inline]{Attila: Ich finde es komisch in der Vergangenheitsform von noch nicht gezeigten Ergebnissen zu sprechen. ist das nicht irgendwie vorgegriffen? \\ CK: Mag sich komisch anfühlen, ist aber richtig.  Ich habe versucht folgende Regeln konsequent anzuwenden (kann aber noch Schnitzer geben): Past tense - describing completed work; present tense: statements of general truth, ongoing stuff, in descriptions of pictures and tables and so on}
The three secondary test cases were done in order to substantiate the findings of the four primary test cases.  The C-CC-55 simulation was done using a full cascade with 55 vanes.  Furthermore, two test cases were simulated with modified patches and constant cyclostationary characteristics.  The seeding area of the patches remained unchanged.  Details regarding the fRPM patch generation can be found in subsection~\ref{subsec:SetupFRPMCAA}.
For case C-CC-red, the initial patch was modified by significantly reducing the safety margins in the lateral direction.  In these safety margins, the turbulent kinetic energy was set to zero.  This test case confirmed that the Young-Van-Vliet filter does not require safety margins. For case C-CC-double, the seeding area was increased to span two pitches instead of one. This was performed in order to confirm that the acoustic radiation of all blades can be assumed to be uncorrelated.



 %This simulation was compared to the primary C-CC case to investigate the consequences of only using a portion of the full annular duct for the primary test matrix. 
% \unsure[inline]{Ich finde das wir das nicht noch einmal zeigen müssen, das ist schon oft gezeigt worden: A different random seeding to generate the white noise field was also used on this patch.  This test case was done in order to prove that the results are the same regardless of the used random variable fields if the calculation times are appropriately long.  } 
%  This patch also used a reduced filter damping zone as the patch would not fit into the CAA domain otherwise.  This was performed in order to confirm that the acoustic radiation of all blades are uncorrelated, i.e. that exciting one blade and multiplying the acoustic power level by the number of vanes does result in a correct overall acoustic power level.  


%\multirow{4}{*}{\parbox[c]{2mm}{\rotatebox[origin=c]{90}{full cascade}}}

\subsection{Setup of CFD computation}

TRACE was used to perform the q3D URANS simulation at 50\% of the stator height of the baseline SDT fan. Periodic boundary conditions were used for a reduced duct consisting of two rotor blades and five stator vanes.  The investigated operating condition was approach.  However, the static pressure at the stator outlet needed to be slightly modified in order to avoid a separation of flow.  %That way, the flow variables of the harmonics in the stator domain could surely be attributed to cyclostationarity rather than to an unsteadiness due to flow separation.     

In the stator domain, unsteady solutions of the first 15 harmonics of the rotor blade passing frequency were used to accurately reconstruct the rotor wakes in the absolute frame of reference.  The number of harmonics required mainly depends on the flow gradients in the wake.  The higher the gradients, the more harmonics are required for an accurate representation of the wake.  At mean flow, i.e. at the 0th harmonic, the rotor wakes were averaged via flux averaging at the mixing plane between the moving frame of reference in the rotor block and the absolute frame of reference in the stator block.  When constant turbulence and flow characteristics were considered, the 0th harmonic of the URANS computation rather than a RANS computation was used in order to guarantee consistency.

The Hellsten explicit algebraic Reynolds stress model as implemented by \citet{franke_turbulence_2010} was used to sufficiently reproduce the turbulent characteristics in the leading edge and wake regions. The turbulence in the blade boundary layer was fully resolved by the used mesh.

The spatial discretization was done via a MUSCL (Monotonic Upstream Scheme for Conservation Laws) method of second order accuracy based on Fromm's scheme, while the time discretization was done using an Euler Backward scheme of second order accuracy.  The grid contained nearly 900,000 cells. % and ca.~50 passage revolutions were computed in total to ensure a converged solution.

%897712 grid cells, computation times?, nearly fifty rounds 	

%\begin{equation}
%l_s=0.4\frac{\sqrt{|k|}}{0.09 \omega}
%\end{equation}
	

%/usr/kiely/kissner/simulations/RC2/piano/TestMatrix_Final/lay_setup.lay
\begin{figure}
\centering
\includegraphics[width=\textwidth,trim=4 4 4 4,clip]{figures/eps_CAASetup.pdf}
\caption{CAA setup for five stator vanes showing the sponge zones, the vortex source, and the vortex sink.  The white dots indicate sensor positions and the contour shows the pressure of the background flow, i.e. the mean flow. \label{fig:CAA_Setup} }
\end{figure}

\subsection{fRPM patch generation and setup of CAA computation} \label{subsec:SetupFRPMCAA}
Figure~\ref{fig:CAA_Setup} displays the typical CAA setup for the computation of fan broadband noise with the fRPM method for a reduced duct containing only five stator vanes.  The setup for the full cascade case C-CC-55 was done analogously.  The \textit{vortex source} was generated via a fRPM patch. In the \textit{vortex sink} region, the vorticity is filtered out.  \textit{Sponge zones} were applied at the in- and outlet boundaries of the CAA domain and the white dots indicate the positions of the used \textit{sensors}.  In this subsection, the CAA mesh generation will discussed before examining the italic keywords in detail.

For high-order spatial discretization schemes, a high grid quality of the CAA domain is essential.  The used grid resolution was determined by two factors: acoustics and turbulence.  The CAA mesh was designed to enable the propagation of sound waves up to a frequency $f$ of \SI{20}{kHz} without significant dissipation.  The lowest acoustic grid resolution was chosen to be 10 points per wavelength (PPW), which is approximately twice the theoretical resolution limit of 5.4 PPW for a DRP scheme \cite{de_roeck_overview_2004}. This leads to the following acoustic grid resolution: $dx_f\leq\frac{\lambda}{\text{PPW}}=\SI{1.7e-3}{m}$.  In regions of the CAA domain, where synthesized turbulence is injected into and convected in the domain, a smaller grid resolution is required.  To fully resolve this turbulence, vortical structures have to be resolved with at least 4 points per length scale.  In this case,  the determined grid resolution was given from the smallest Gaussian length scale needed to reconstruct a von \Karman turbulence spectrum through a superposition of ten analytically weighted Gaussian spectra. The analytical weighting function and these best-practice rules are derived by \citet{wohlbrandt_analytical_2016} to realize smooth von \Karman spectra.
The von \Karman turbulence spectrum was determined from experimental data of \citet{podboy_steady_2003} to realize an integral turbulent length scale of $\Lambda\approx\SI{5.1e-3}{m}$ at 50\% of the stator height at the stator LE.  The resulting turbulent grid resolution was  $dx_v\leq\SI{0.51e-3}{m}$.  Furthermore, the boundary layer of the background flow was considered but not fully resolved in the CAA domain. %\todo[inline]{Do we need to show ''previous studies''? \\Attila: I find it interesting for the community. Maybe this should be moved to the results. \\CK: Definitely interesting but hard to fit into this paper.  Let me know if you find a way.  Another thing: don't have pics and spectra for this exact setup.} 
Previous studies have shown that artificial sound is generated at a blunt stator trailing edge (TE) if the boundary layer is neglected.  The vortices that move along the blade surface create sound when interacting with a blunt TE.  This is not problematic for pointed TEs but for blunt TEs as is the case for the baseline SDT stator.  The boundary layer prevents the direct interaction between the stator TE and the vortices as it pushes the vortices further away from the blade surface.  The final CAA grid contained 43,324 grid cells per stator passage.     

%45680 particles for initial patch
%/usr/kiely/kissner/simulations/RC2/piano/TestMatrix_Final/lay_PatchComparison.lay
\begin{figure}
\centering
		\subfigure[Initial patch\label{fig:PatchIni}]{
			\includegraphics[width=0.25\textwidth,trim=0 6 4 4,clip]{figures/eps_Patches_1.pdf}}
		\subfigure[Patch with reduced safety margins\label{fig:PatchRedDamp}]{
			\includegraphics[width=0.25\textwidth,trim=0 6 4 4,clip]{figures/eps_Patches_2.pdf}}
		\subfigure[Patch spanning two pitches\label{fig:Patch2Blades}]{
			\includegraphics[width=0.25\textwidth,trim=0 6 4 4,clip]{figures/eps_Patches_3.pdf}}
% 	\vspace{-1em}
	\caption{Used patches for C-CC configuration.  The contour shows the TKE of the mean flow.\label{fig:Patches}}
\end{figure}

\begin{description}
\item{The \textit{vortex source}} produces the synthesized turbulence using inputs from the (U)RANS computation for TKE, TLS, and mean flow and is coupled into the domain using the LEE-relaxation method.  The LE of the source region was located approximately two chord lengths upstream of the stator LE.  The mesh resolution was kept constant to the CAA mesh resolution and five particles per grid cell were used.  As mentioned in the preceding paragraph, a von \Karman spectrum was chosen to generate the synthetic turbulence as a von \Karman spectrum closely emulates realistic turbulence spectra found in turbomachinery. 

Assuming uncorrelated vanes, the RSI is only simulated for a single vane.  All other vanes ensure the correct acoustic radiation only.  Figure~\ref{fig:PatchIni} shows the initial patch used for the primary test matrix.  The contour shows the TKE levels.  The region of the patch containing TKE spans exactly one pitch in order to interact with exactly one stator vane.  The rest of the patch area does not contribute to the synthesized turbulence but acts as safety margin and accounts for lateral convection.  Since the Young-Van-Vliet filter does theoretically not require safety margin as it works on a ghost layer instead, the safety margin was significantly reduced for configuration C-CC-red [see Figure~\ref{fig:PatchRedDamp}].  For configuration C-CC-double, the patch spanned two pitches and has no safety margins [see Figure~\ref{fig:Patch2Blades}].      	



\item{The \textit{vortex sink}} cancels out vorticity downstream of the stator trailing edge utilizing the LEE-relaxation method by \citet{ewert_linear-_2014}.  The vortices in the CAA domain downstream of the stator TE interact with the stator wakes creating hydrodynamic pressure disturbances.  To get a clean acoustic pressure signal at the sensor position, the vortices were removed from the domain and only acoustic perturbations, which are of interest, remained.  This was done by setting the target vorticity $\bfm\Omega^\text{ref}$ to zero in the whole \textit{vortex sink} region as marked in Fig.~\ref{fig:CAA_Setup} and as alluded to in subsection~\ref{sec:coupling}.  

\item{The \textit{sponge zones}} were needed to avoid reflections at the in- and outlet boundary conditions of the CAA domain.  The sponges were 85 cells deep and a small cell stretching in axial direction - not exceeding a value of 1.1 -  was also introduced.

\item{\textit{Sensors}} were equally spaced up- and downstream of the source region to compute the emitted sound power $P$ of an equivalent 3D duct from the 2D cascade, as derived in the Appendix~\ref{sec:PWL}. As 2D turbulence is generated by the 2D RPM method the correction in Eq.~\ref{eq:2D3DGeschwSpektrum} was used, assuming that the lateral velocity component is the major noise source. The integration surface $S$ resulted from a line of the considered 2D cascades discretized in the y-direction by 275 probes for the full-cascade configuration and 25 probes for the five-vane configuration. As the turbulence only impinged on a single vane, the sound power was multiplied by the number of vanes to get the overall sound power.
\end{description}
The setup for the different configurations remained exactly the same.  Only the number of considered harmonics for the TKE, TLS, and mean flow varied.  Periodic variables were realized by 15 harmonics, while constant variables were realized by the 0th harmonic, i. e. the mean variables.  The computation for the C-CC test case lasted approximately four days on five Intel(R) Xeon(R) CPU E5-2630 v3 CPUs.  The same setup for the P-PP took about eight days to compute.  Realizing cyclostationarity is computationally more expensive as summing up the harmonics at each time step takes more time and more information has to be stored.        


\section{Results and discussion}\label{sec:results}

In the following section, the results of the test matrix are discussed.  At first, the primary test matrix is examined to study the effects of cyclostationarity.  Next, an analytical test case is analyzed to substantiate and augment the findings of the first subsection.  The comparison to the measurements can be found in subsection~\ref{sec:meas}. In the last two subsections, elemental assumptions made during the design of the primary test matrix are checked: 1.) The authors assumed that results of a simulation on a reduced duct containing only five stator blades are equivalent to results of a simulation on the full duct containing all 55 stator blades. 2.)  The authors assumed that the RSI noise generation mechanism of each blade is uncorrelated to that of all other blades in the duct and that it is therefore sufficient to only study one blade.

\subsection{Analysis of the primary test cases}
%/usr/kiely/kissner/simulations/RC2/piano/TestMatrix_Final/lay_powerspectra_all.lay
\begin{figure}
\centering
\includegraphics[width=0.9\textwidth,trim=4 4 1 1,clip]{figures/eps_PowerSpectra_all.pdf} 
\caption{Comparison of sound power spectra of the primary test matrix. The greyed-out regions indicate frequencies outside the range, for which the CAA mesh was designed. \label{fig:PowerSpectra} }
\end{figure}

The primary test matrix was designed to systematically study the effect of cyclostationarity in the turbulence and in the mean flow on the broadband RSI noise.  The study was conducted for a well-known fan, i.e. the NASA SDT fan in its baseline configuration.

In Figure~\ref{fig:PowerSpectra} the sound power level  $L_W$ 
\begin{align}
L_W = 10 \log_{10}\left( P\right/P_\text{ref}) \label{eq:soundPowerLevel}
\end{align}
are shown for all primary test cases with $P$ in Eq.~\ref{eq:P} and $P_\text{ref} = \SI{1e-12}{\watt}$ calculated up- and downstream of the stator.  The grayed-out areas in the graph indicate frequencies outside the range of the target CAA mesh resolution. As the mesh resolution is twice as high as recommended in the literature, results up to \SI{40}{kHz} are deemed to be trustworthy.

When neglecting the periodic nature of the mean flow, the difference seems to be negligible for this configuration as can be concluded when comparing the P-PP and the C-PP primary test cases.  However, when the TLS is taken to be constant over the stator pitch (C-PC) instead of periodic (C-PP), it results in a notable offset in sound power. The offset is large at lower frequencies and disappears at very high frequencies.  Lastly, there is little to no difference in the sound power spectra between the C-CC and C-PC test cases indicating that the cyclostationarity of the TKE does not influence the sound power levels.  To sum up, only the cyclostationarity of the TLS influences the RSI noise for the baseline SDT configuration at approach conditions.

%/usr/kiely/kissner/simulations/RC2/piano/TestMatrix_Final/lay_patchspectra_analytical.lay
\begin{figure}
\centering
\includegraphics[width=0.9\textwidth,trim=4 4 1 1,clip]{figures/eps_PatchSpectra.pdf} 
\caption{Upwash velocity spectra shown for constant and fully periodic patches.  Comparisons to the respective analytical 2D velocity spectra computed with Eq.~\ref{eq:E22_Karman_2D} and Eq.~\ref{eq:Sii} are shown. \label{fig:PatchSpectra} }
\end{figure}
\subsection{Alternative averaging to realize correct TLS}
% \todo[inline]{I focused exclusively on the upwash velocity freq. spectrum here since I eliminated it from the graphic.  Recheck to see if you are in agreement.}
In order to examine the effect of the TLS in more detail, the authors took a closer look at the synthesized turbulence in the fRPM vortex source patch.  In Figure~\ref{fig:PatchSpectra}, the lateral 2D velocity frequency spectrum $S^{2D}_{22}(f)$, which is most relevant for the broadband noise generation at the stator LE, realized by the patches for configurations C-CC and C-PP along with their respective analytical solutions are shown. For the constant case, the analytical solutions are calculated using the Eq.~\ref{eq:E22_Karman_2D} and Eq.~\ref{eq:Sii} for the 2D von \Karman turbulence spectrum. For this purpose the turbulent kinetic energy, the turbulent specific dissipation rate and the flow velocity are circumferentially averaged: 
\begin{align}\label{eq:LambdaC}
k_t^C = \frac{1}{2\pi}\int\limits_0^{2\pi} k_t(\vartheta)\abl \vartheta, &&
\omega_t^C = \frac{1}{2\pi}\int\limits_0^{2\pi} \omega_t(\vartheta)\abl \vartheta, &&
u_0^C = \frac{1}{2\pi}\int\limits_0^{2\pi} u_0(\vartheta)\abl \vartheta.
\end{align}
The TLS was determined by Eq.~\ref{eq:Lambda}. For the fully periodic case (P-PP), the analytical velocity spectrum results from integrating over the analytical spectra at each circumferential point of the downstream patch border:
\begin{align}\label{eq:Spii}
S^\text{P}_{ii}(f) = \frac{1}{2\pi}\int\limits_0^{2\pi}S_{ii}(f,\vartheta) \abl \vartheta.
\end{align}  
The TLS is then determined by fitting the averaged velocity frequency spectrum to a standard von \Karman velocity spectrum.  

It can be noted that the numerical spectra match the respective analytical spectra well, particularly at high frequencies.  There is an offset at lower frequencies, which may indicate that the turbulence is lacking energy at those frequencies.  The pronounced offset between the fully periodic and fully constant spectra is most significant.  When a fit is performed for the PP spectra,  the averaged TLS was $\SI{0.0023}{m}$.  In contrast, the averaged TLS for the CC spectra was $\SI{0.0039}{m}$, while the averaged TKE and flow velocity were nearly equivalent.  These findings indicate that the method used for averaging plays a significant role in determining the turbulence characteristics.  Averaging the TKE, TLS, and mean flow over the circumference before calculating velocity frequency spectra (CC) gives different results than calculating spectra for each TKE, TLS, and mean flow before averaging the spectra over the circumference (PP).  
%This could explain part of the offset observed in Figure~\ref{fig:PowerSpectra}. 
%/usr/kiely/kissner/simulations/RC2/piano/TestMatrix_Final/lay_powerspectra_ls.lay
\begin{figure}
\centering
\includegraphics[width=0.9\textwidth,trim=4 4 1 1,clip]{figures/eps_PowerSpectra_ls.pdf} 
\caption{C-CC test case with adjusted turbulent length scale compared to primary test cases C-CC and P-PP \label{fig:PowerSpectra_ls} }
\end{figure}

To further test these findings, the C-CC-tls configuration was simulated.  Instead of using an averaged TLS of $\SI{0.0039}{m}$, a constant TLS of $\SI{0.0023}{m}$ as determined by the PP upwash velocity frequency spectrum was imposed onto the patch.  If the hypothesis that the averaging technique is essential when considering cyclostationary processes is true, the simulation with the modified TLS should produce the same results as the fully periodic simulation (P-PP).  In fact, this is confirmed by the power spectra shown in Figure~\ref{fig:PowerSpectra_ls}.                 

\subsection{Impact of cyclostationarity for an analytic test case}
% \todo[inline]{Text wrap this image?}
%/usr/kiely/kissner/m-files/analyticalSpectra/lay_wakeTKE.lay
\begin{figure}[htb]
\parbox{0.39\textwidth}{
% \begin{wrapfigure}{r}{0.4\textwidth,trim=4 4 1 1,clip}
\centering
\includegraphics[width=0.39\textwidth,trim=4 4 1 1,clip]{figures/eps_TKEwake.pdf} 
\caption{Turbulent kinetic energy of extracted wake.  Background turbulence modified to demonstrate impact on cyclostationarity.\label{fig:TKEWake_analytical} }
% \end{wrapfigure}
% \end{figure}
%/usr/kiely/kissner/m-files/analyticalSpectra/lay_SpectraTKE.lay
}\hfill
\parbox{0.59\textwidth}{
% \begin{figure}
% \centering
\includegraphics[width=0.59\textwidth,trim=4 4 1 1,clip]{figures/eps_TKESpectra.pdf} 
\caption{Analytical 2D velocity spectra calculated for modified wake using different averaging techniques. \label{fig:TKESpectra_analytical} }
}
\end{figure}
In the previous subsection, results of the fully periodic test case were reproduced using a fully constant simulation with a TLS determined by a different circumferential averaging technique.  This finding is convenient, since it allows for the use of a less computationally expensive technique granted that the TLS is chosen accordingly. However, for all we know, this finding is only valid for this particular fan configuration at this particular operating point.  To test this, the authors aimed at finding an analytical test case, for which the P-PP case cannot be correctly reproduced by a C-CC case.

%Initial: TKE=0.056603, v ~ 156, TI = 0.12452%
%Modified: TKE=5, v~156, TI = 1.17%

In order to achieve this, we slightly manipulated the initial TKE.  Figure~\ref{fig:TKEWake_analytical} shows the TKE of two wakes over one rotor passage.  The initial wake was extracted from the (U)RANS simulation.  The TKE of the background turbulence was set to a constant value to clearly show the difference between the initial and modified cases.  The constant TKE value of the background turbulence is equivalent to the extracted, slightly fluctuating values and results in the same velocity frequency spectra as before.  For the modified case, the TKE in the wake remained the same.  Only the turbulence intensity of the background turbulence was increased from 0.1\% to 1\%.  All other variables remained unchanged.  The resulting velocity frequency spectra by applying the different averaging techniques for the modified wake are shown in Figure~\ref{fig:TKESpectra_analytical}.  As observed in the previous subsection, the periodic variation of the TKE and the mean flow have no impact.  Though the difference in the shape of the velocity frequency spectra due to the periodicity of the TLS is compelling, particularly when considering the direction perpendicular to the flow.  The fully periodic velocity frequency spectrum now exhibits two pronounced bumps instead of just one.  The bump at the lower frequency is due the background turbulence, while the bump at a higher frequency can be attributed to the wake turbulence.  For this hypothetical case, test cases using a constant value for the TLS will never be able to reproduce the shape of the spectrum correctly.


The analytical case with an increased background turbulence intensity proved that there are cases for which the best averaging technique is of no use and cyclostationarity must be simulated in order to synthesize realistic turbulence.      



\subsection{Comparison to measurements}
\label{sec:meas}
\begin{figure}
\centering
\includegraphics[width=0.9\textwidth,trim=4 4 1 1,clip]{figures/eps_PowerSpectra_experiment.pdf} 
\caption{Comparison of numerical and experimental sound power level spectra. \label{fig:PowerSpectraExp} }
\end{figure}
The fully periodic case (P-PP) most accurately reproduces the actual physics.  
For this case the sound power levels are compared to the measurements\footnote{The rotor-stator noise contribution was obtained by subtracting the rotor-alone results from the overall noise results (rotor+stator).} in Figure~\ref{fig:PowerSpectraExp}. The overall trend and levels are reproduced.  While the experimental data differs from the numerical results up- and downstream of the fan in regions below \SI{10}{kHz}, the high-frequency fall off is well predicted. An under-prediction of the sound power at lower frequencies has also been shown by \citet{nallasamy_computation_2005}, who used a RANS-informed, analytical method, at both inlet (upstream) and exhaust (downstream). %\todo[inline]{SG: Mit welcher Methode}

The differences can be explained twofold: (1) The measurements may have contributions from additional noise sources (e.g. rotor trailing edge, jet noise).
(2) Some simplifications and assumptions were made in the numerical hybrid approach. The TKE, TLS, and mean flow were taken from a (U)RANS simulation and a locally isotropic von \Karman spectrum was assumed.  The data was used "as is" and no adjustments were made to achieve a better agreement with experimental results.  Additionally, the used approach is two-dimensional and can only simulate broadband noise resulting from the interaction of turbulence with the blade surfaces.  Any other sound sources that may have been captured by the measurements cannot be considered by this approach.  The method also neglects the rotor, i.e. transmission losses or reflections at the rotor are disregarded.  %Another issue is that the stator was never measured by itself.  Instead, experiments were conducted for a rotor-alone and a rotor-stator configuration.  To determine sound power spectra up- and downstream of the stator, the sound power of the rotor-stator configuration was subtracted by the rotor-alone configuration.  It is uncertain if the thusly determined sound power spectra is truly correct as it assumes that sound sources of rotor and stator are uncorrelated and can therefore be summed up to calculate the total sound.         
% Note that a seemingly better fit of the original C-CC test case to the experimental data is considered to be a coincidence by the authors. The P-PP configuration reproduces the actual physics more accurately.

\subsection{Comparison of full and reduced duct computations}
%/usr/kiely/kissner/simulations/RC2/piano/TestMatrix_Final/lay_powerspectra_fullred.lay
\begin{figure}
\centering
\includegraphics[width=0.9\textwidth,trim=4 4 1 1,clip]{figures/eps_PowerSpectra_fullred.pdf} 
\caption{Comparison of the reduced and full cascade using the C-CC configuration.  Cut-on frequencies for azimuthal modes $m=1$ and $m=-1$ are shown. \label{fig:PowerSpectraFullRed} }
\end{figure}
The first elemental assumption that the authors made in the design of the primary test matrix was to assume that simulating a reduced, periodic duct is equivalent to simulating the full duct.  To test this assumption, the C-CC-55 simulation was performed containing all 55 stator vanes.  The resulting narrow-band sound power spectra are shown in Figure~\ref{fig:PowerSpectraFullRed}.  The power spectra are, in fact, nearly equivalent.  Only at lower frequencies, the power spectra of the full cascade is smooth while there are peaks in the power spectra of the reduced cascade. Peaks in fan power spectra are often indicative of where a new acoustic mode of azimuthal order $m$ suddenly becomes cut-on.  The equations for calculating cut-on frequencies of acoustic modes are listed in the Appendix~\ref{app:cuton}.  Aside from the azimuthal mode order, the cut-on frequency depends on the flow speeds and the geometry.  In this case, the duct geometries differ:  The reduced duct with only five stator blades has a smaller circumference than the full duct. For these cases, the relevant Mach numbers are: 
\begin{align*}
	\text{upstream} &&&M_x = 0.40,	&& M_y =-0.21,\text{ and}\\
	\text{downstream} &&&M_x = 0.44,	&& M_y = 0.00.
\end{align*} 
The resulting characteristic frequencies for the first azimuthal mode orders are:

% first column
\begin{minipage}[t]{0.5\textwidth}
\begin{itemize}
	\item 5-vane configuration 
	\begin{itemize}
		\item upstream
		\begin{itemize}
			\item[] $f_{m=-1} = 3169.6$~Hz, 
			\item[] $f_{m=+1} = 1968.7$~Hz, and
		\end{itemize}
		\item downstream 
		\begin{itemize}
			\item[] $f_{|m|=1} = 2508.2$~Hz,
		\end{itemize}
	\end{itemize}
\end{itemize}
\end{minipage}
%second column
\begin{minipage}[t]{0.5\textwidth}
\begin{itemize}
	\item 55-vane configuration 
	\begin{itemize}
		\item upstream
		\begin{itemize}
			\item[] $f_{m=-1} = 303.9$~Hz, 
			\item[] $f_{m=+1} = 178.4$~Hz, and
		\end{itemize}
		\item downstream 
		\begin{itemize}
			\item[] $f_{|m|=1} = 237.5$~Hz.
		\end{itemize}
	\end{itemize}
\end{itemize}
\end{minipage}
\vspace{0.5em}  


For the full cascade, the cut-on frequencies are very low meaning that the acoustic modes of the first and subsequent azimuthal mode orders are cut-on for most of the frequency range.  The determined cut-on frequencies for the reduced cascade are higher and align well with the peaks in the power spectra - both up- and downstream of the stator [see Figure~\ref{fig:PowerSpectraFullRed}].  Aside from the peaks due to the cut-on frequencies, the assumption of the authors was correct and the reduced duct does reproduce the sound power correctly.     


\subsection{Acoustic correlation of vane blades}

%/usr/kiely/kissner/simulations/RC2/piano/TestMatrix_Final/lay_powerspectra_2pitches.lay

The second elemental assumption that the authors made was to assume that the investigated sound generation mechanism of each blade is uncorrelated to that of the other blades.  This therefore allows for investigating the impingement of turbulence on only one blade and for multiplying the determined sound power of one blade by the number of blades to receive the total sound power.  In order to confirm this assumption, configuration C-CC-double was investigated with a patch spanning two pitches and therefore turbulence impinging onto two stator blades.

\begin{figure}
\centering
\includegraphics[width=0.9\textwidth,trim=4 4 1 1,clip]{figures/eps_PowerSpectra_2Pitches.pdf}
\caption{Comparison of different patches to check assumption of uncorrelated blades. \label{fig:PowerSpectra_2Pitches} }
\end{figure}
	
An interim step was necessary due to the fact that a patch spanning over two vanes with a lateral safety margin of the same size as the original patch did not fit into the CAA domain for the reduced cascade.  For this interim step, configuration C-CC-red used a patch without a damping zone in lateral direction [see subsection~\ref{subsec:SetupFRPMCAA}].  Figure~\ref{fig:PowerSpectra_2Pitches} shows that the resulting power spectra for the initial patch, used for the primary test matrix, and this patch are identical.  This confirms that the Young-Van-Vliet Filter does, in fact, not need a safety margin in the lateral direction.  Since a different random variable field also had to be used, it also confirms that the solution is independent of the used random variable field.  Since the results were the same, the initial patch spanning one pitch can directly be compared to the patch spanning two pitches.  Both patches yield exactly the same sound power levels [see Figure~\ref{fig:PowerSpectra_2Pitches}].  This confirms that there is no significant acoustic correlation of the vane blades. This result has been anticipated as the TLS is much smaller than the pitch.

%\todo{AW: There is a gray border on all of your pics. 
%\\I have added a trim and clip to remove them, but
%\\these could be removed by removing the alpha channel (transparency) of the original image, I guess... }
\section{Conclusion}
We have presented a novel tokenization scheme for Vision Transformers, replacing the standard uniform patch grid with a mixed-resolution sequence of tokens, where each token represents a patch of arbitrary size. We integrated the Quadtree algorithm with a novel feature-based saliency scorer to create mixed-resolution patch mosaics, making this work the first to use the Quadtree representations of RGB images as inputs for a neural network.

Through experiments in image classification, we have shown the capacity of standard Vision Transformer models to adapt to mixed-resolution tokenization via fine-tuning. Our Quadformer models achieve substantial accuracy gains compared to vanilla ViTs when controlling for the number of patches or GMACs. Although we do not use dedicated tools for accelerated inference, Quadformers also show gains when controlling for inference speed.

We believe that future work could successfully apply mixed-resolution ViTs to other computer vision tasks, especially those that involve large images with heterogeneous information densities, including tasks with dense outputs such as image generation and segmentation.

\appendix
\renewcommand\thefigure{\arabic{figure}}  
\section{Appendix}
% \addcontentsline{toc}{section}{Appendix}
\appendix

\renewcommand{\thetable}{A\arabic{table}}
\setcounter{table}{0}


\begin{center}
\LARGE
\textbf{Supplementary Material}
\end{center}

\hfill \break

\section{Full results}
We report ImageNet-1k top-1 accuracy and various cost indicators for every model configuration that appears in the figures of the main text (see Table \ref{table:results_vit_small}, Table \ref{table:results_vit_base}, Table \ref{table:results_vit_large}). Throughput is measured on a single GeForce RTX 3090 GPU in mixed precision.


\section{More implementation details}
\paragraph{Hyperparameters.} We train all of our models using the timm library~\cite{rw2019timm} with the following hyperparameters: learning rate warmup for 5 epochs, learning rate cooldown for 10 epochs, cosine learning rate scheduler~\cite{Loshchilov2016SGDRSG}, weight decay 0.025, DropPath~\cite{Huang2016DeepNW} rate 0.1, AdamW~\cite{Loshchilov2017DecoupledWD} optimizer with epsilon $1\text{e-}8$, AutoAugment~\cite{Cubuk2018AutoAugmentLA} image augmentations with configuration \verb|rand-m9-mstd0.5-inc1|, mixup~\cite{Zhang2017mixupBE} alpha 0.8, cutmix~\cite{Yun2019CutMixRS} alpha 1.0, label smoothing 0.1. Unless otherwise specified, we use base learning rate $5\text{e-}5$.

We fine-tune ViT-Small models for 130 epochs with batch size 1024, ViT-Base models for 60 epochs with batch size 400, and ViT-Large models for 20 epochs with batch size 192. For evaluation, we use exponential moving average (EMA)~\cite{Polyak1992AccelerationOS} with decay 0.99996. We use the default values in timm for all other hyperparameters.


\hfill \break

\begin{table}[h!]
\centering
\textbf{ViT-Small}
\resizebox{0.95\linewidth}{!}{
\begingroup
\renewcommand{\arraystretch}{1.1}
\begin{tabular}{ | c | c c c c | c | }
\hline
\multirow{2}{*}{Method}   &   \multirow{2}{*}{\#Patches}  &  \multirow{2}{*}{GMACs}   &  Throughput  & Runtime  & ImageNet-1k   \\
   &  &  & ims/sec   & $\mu$-secs/im   & Top-1 Acc.   \\
\hline
\multirow{5}{*}{Vanilla ViT}    & 64    & 1.44   & 6489   & 154   & 74.55   \\
 & 81    & 1.83   & 5208   & 192   & 76.36   \\
 & 100   & 2.28   & 4212   & 237   & 77.55   \\
 & 121   & 2.78   & 3460   & 289   & 78.26   \\
 & 169   & 3.94   & 2315   & 432   & 79.84   \\
 & 196   & 4.62   & 1975   & 506   & 80.28   \\
\hline
\multirow{5}{*}{\shortstack[c]{Quadformer \\ \small{Feature-based scorer}}}   & 64    & 1.54   & 3611   & 277   & 76.53   \\
 & 79    & 1.88   & 3204   & 312   & 77.53   \\
 & 100   & 2.37   & 2766   & 362   & 78.64   \\
 & 121   & 2.87   & 2419   & 413   & 79.35   \\
 & 169   & 4.04   & 1792   & 558   & 80.43   \\
 & 196   & 4.71   & 1576   & 635   & 80.84   \\
\hline
\multirow{5}{*}{\shortstack[c]{Quadformer \\ \small{Pixel-blur scorer}}}    & 64    & 1.45   & 5150   & 194   & 74.97   \\
 & 79    & 1.79   & 4362   & 229   & 76.27   \\
 & 100   & 2.28   & 3590   & 279   & 77.47   \\
 & 121   & 2.78   & 3022   & 331   & 78.58   \\
 & 169   & 3.95   & 2104   & 475   & 80.01   \\
 & 196   & 4.62   & 1813   & 552   & 80.4   \\
\hline
\end{tabular}
\endgroup
}% close resizebox
\vspace*{0.2cm}
\caption{Full results - ViT Small.}
\label{table:results_vit_small}
\end{table}

\hfill \break

\hfill \break

\hfill \break

\hfill \break


\begin{table}[h!]
\centering
\textbf{ViT-Base}
\resizebox{0.95\linewidth}{!}{
\begingroup
\renewcommand{\arraystretch}{1.1}
\begin{tabular}{ | c | c c c c | c | }
\hline
\multirow{2}{*}{Method}   &   \multirow{2}{*}{\#Patches}  &  \multirow{2}{*}{GMACs}   &  Throughput  & Runtime  & ImageNet-1k   \\
   &  &  & ims/sec   & $\mu$-secs/im   & Top-1 Acc.   \\
\hline
\multirow{5}{*}{Vanilla ViT}   & 64  &  5.6 & 2676   & 374   & 80.78   \\
   & 81  &  7.2  & 2155   & 464    & 81.73   \\
   & 100 &  8.8  & 1739   & 575    & 82.31   \\
   & 121 &  10.7  & 1429   & 700    & 82.71   \\
   & 169 &  15.1  & 966    & 1035   & 83.74   \\
   & 196 &  17.6  & 823    & 1215   & 84.07   \\
\hline
\multirow{5}{*}{\shortstack[c]{Quadformer \\ \small{Feature-based scorer}}}   &   64    & 5.7  &   2019   &   495   &   81.52   \\
   &   79   & 7.1  &   1732   &   577   &   82.34   \\
   &   100  & 8.9  &   1435   &   697   &   83.05   \\
   &   121  & 10.8 &   1218   &   821   &   83.50   \\
   &   169  & 15.2 &   864    &   1157  &   84.23   \\
   &   196  & 17.7 &   750    &   1333  &   84.38   \\
\hline
\multirow{5}{*}{\shortstack[c]{Quadformer \\ \small{Pixel-blur scorer}}}   & 64 &  5.7  & 2424   & 413   & 80.78   \\
   & 79  & 7.0  & 2021   & 495   & 81.68  \\
   & 100 & 8.8  & 1630   & 613   & 82.57  \\
   & 121 & 10.7 & 1354   & 739   & 83.06   \\
   & 169 & 15.1 & 931    & 1074  & 83.87   \\
   & 196 & 17.6 & 800    & 1250  & 84.23   \\
\hline
\multirow{5}{*}{\shortstack[c]{Quadformer \\ \small{Oracle scorer}}}   & 64   &  ---   &  ---   & ---   & 84.76   \\
   & 79    &  ---   &  ---   & ---   & 85.19   \\
   & 100   &  ---   &  ---   & ---   & 85.40   \\
   & 121   &  ---   &  ---   & ---   & 85.67   \\
   & 169   &  ---   &  ---   & ---   & 85.40   \\
   & 196   &  ---   &  ---   & ---   & 85.25   \\
\hline
\end{tabular}
\endgroup
}% close resizebox
\vspace*{0.2cm}
\caption{Full results - ViT Base.}
\label{table:results_vit_base}
\end{table}



\begin{table}[h!]
% \vspace*{10pt}
\centering
\textbf{ViT-Large}
\resizebox{0.95\linewidth}{!}{
\begingroup
\renewcommand{\arraystretch}{1.1}
\begin{tabular}{ | c | c c c c | c | }
\hline
\multirow{2}{*}{Method}   &   \multirow{2}{*}{\#Patches}  &  \multirow{2}{*}{GMACs}   &  Throughput  & Runtime  & ImageNet-1k   \\
   &  &  & ims/sec   & $\mu$-secs/im   & Top-1 Acc.   \\
\hline
\multirow{5}{*}{Vanilla ViT}   & 64 &  19.9   & 900   & 1111   & 82.00   \\
   & 81  & 25.2   & 720   & 1389   & 83.02   \\
   & 100 & 31.1   & 580   & 1724   & 83.86   \\
   & 121 & 37.7   & 478   & 2092   & 84.46   \\
   & 169 & 53.0   & 323   & 3096   & 85.42   \\
   & 196 & 61.7   & 277   & 3610   & 85.74   \\
\hline
\multirow{5}{*}{\shortstack[c]{Quadformer \\ \small{Feature-based scorer}}}   & 64   & 20.1   & 777   & 1287   & 82.88   \\
   & 79  &  24.7  & 649   & 1541   & 83.67   \\
   & 100 &  31.3  & 527   & 1898   & 84.41   \\
   & 121 &  37.9  & 440   & 2273   & 85.03   \\
   & 169 &  53.1  & 306   & 3268   & 85.65   \\
   & 196 &  61.8  & 265   & 3774   & 85.79   \\
\hline
\multirow{5}{*}{\shortstack[c]{Quadformer \\ \small{Pixel-blur scorer}}}  & 64  & 19.9   & 869   & 1151  & 81.66  \\
 & 79  & 24.6   & 712   & 1404  & 82.69  \\
 & 100 & 31.1   & 568   & 1761  & 83.61  \\
 & 121 & 37.7   & 470   & 2128  & 84.3   \\
 & 169 & 53.0   & 320   & 3125  & 85.22  \\
 & 196 & 61.7   & 275   & 3636  & 85.56  \\
\hline
\multirow{5}{*}{\shortstack[c]{Quadformer \\ \small{Oracle scorer}}}   & 64   &  ---   &  ---   & ---   & 85.89   \\
   & 79    &  ---   &  ---   & ---   & 86.33   \\
   & 100   &  ---   &  ---   & ---   & 86.5   \\
   & 121   &  ---   &  ---   & ---   & 86.7   \\
   & 169   &  ---   &  ---   & ---   & 86.52   \\
   & 196   &  ---   &  ---   & ---   & 86.54   \\
\hline
\end{tabular}
\endgroup
}% close resizebox
\vspace*{0.2cm}
\caption{Full results - ViT-Large.}
\label{table:results_vit_large}
\end{table}


\bibliographystyle{model1a-num-names}
% \bibliographystyle{elsarticle} %numerisches Zitate
% \bibliographystyle{model2-names.bst}\biboptions{authoryear} %Harvard style
\bibliography{bibliography}
\addcontentsline{toc}{section}{References}
%% Authors are advised to submit their bibtex database files. They are
%% requested to list a bibtex style file in the manuscript if they do
%% not want to use model1a-num-names.bst.


\end{document}
