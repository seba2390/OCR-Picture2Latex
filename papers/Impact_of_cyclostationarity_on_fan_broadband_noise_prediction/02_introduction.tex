% \improvement[inline]{Seb: Unterschied zu AIAA 2015 und die Konferenz ausreichend erwähnt. Vorbeugen von Plagiatsvorwürfen. Evtl. doch bei AIAA einreichen.}
% \improvement[inline]{Überprüfe: Abkürzung RPM, fRPM oder (F)RPM verwenden. Ich präferiere fRPM für das Tool und RPM für die Methode. \\CK: Im Methodenteil finde ich es manchmal recht schwer zwischen Tool und Methode zu differenzieren.} 
% \improvement[inline]{Überprüfe: Haben wir erwähnt, wie wir die zeitlich und örtlich veränderliche Längenskala mit Filtern für kosntante Längenskalen erzeugen? - Ja. Intro (9. Absatz) und im Methodenteil auch} 

\section{Introduction}
%%%%%%%%%%%%%%%%%%%%%%%%%%%%%%%%%%%%%%%%%%%%%%%%%%%%%%
%   SITUATION
%   1. What is the background of the topic?
%   2. Why is the topic important?
%%%%%%%%%%%%%%%%%%%%%%%%%%%%%%%%%%%%%%%%%%%%%%%%%%%%%%

% \todo[inline]{Insgesamt finde ich, dass die Anzahl der Referenzen ausreicht.  Ich würde die Einleitung ungern aufblasen, indem man detaillierter auf weitere Paper eingeht.  Auflistungen sind aber vielleicht nicht schlecht (mögliche Stellen in Fett).  Eventuell dabei den Fokus auf Journal Artikel setzen.}

%CITATIONS

Current and future engines used in civil aviation have large bypass ratios meaning that the fan plays an ever increasing role as a noise source. Fan noise, in particular rotor-stator-interaction (RSI) noise, is one of the most dominant noise sources of an ultra-high bypass ratio (UHBR) engine.  It has the largest contribution during the approach phase and is only surpassed by jet noise during the take-off phase. Its prediction is of an increasing importance for the development of new technologies in light of the overall growth in air traffic and progressively more stringent noise regulations.

%\todo[inline]{Seb: References, was ist mit Linern? \\ CK: added liners to the list, although I'm not if the list needs to be exhaustive as the beginning of the sentence "These approaches include...".  I also wouldn't use references unless we just list relevant paper after each point.}
The tonal components of this noise generation mechanism have been researched extensively.  As a result, different methods were successfully applied to reduce the tonal RSI noise.  These approaches include reducing the tip circumferential speed to subsonic speeds, using acoustic liners in the engine duct, increasing the rotor-stator gap, choosing certain blade counts to strategically use acoustic cut-off effects, and modifying the blade geometry to e.g. reinforce destructive radial interferences.  Due to a reduction of tonal noise, the relative contribution of the broadband noise has significantly increased.  Hence, a greater understanding of the broadband noise generation mechanism in the fan stage is required to further reduce the RSI noise. 


%%%%%%%%%%%%%%%%%%%%%%%%%%%%%%%%%%%%%%%%%%%%%%%%%%%%%%
%   COMPLICATION
%		-limitation, short coming, lack of understanding of current methods, approaches
%%%%%%%%%%%%%%%%%%%%%%%%%%%%%%%%%%%%%%%%%%%%%%%%%%%%%%
%\todo{Mention \citet{shur_effect_2016} here? No - didn't cite for analytical methods, so let's not cite anything for scale-resolving methods.}
However, the prediction of fan broadband noise is still considered to be a challenge:  On one hand, analytical models are restrictive as they require strong assumptions.  On the other hand, CFD computations that fully resolve turbulent scales are exceedingly demanding in computational resources.   To advance the current understanding of broadband noise generation in a fan stage, methods are needed that are both fast and affordable without being overly restrictive.

Hybrid approaches fill the gap, if it is deemed possible to divide the physical problem into individual phenomena which can be calculated sequentially. This results in a process chain where every task is completed by the most efficient method respectively.  Hybrid approaches can combine numerical, analytical and empirical methods. In broadband noise predictions the two most prominent ones are Large Eddy Simulations (LES) coupled to an acoustic analogy and stochastic methods coupled to a Computational AeroAcoustics (CAA) method. This paper focuses on the latter method. \citet{allan_comparison_2014} compared the two mentioned methods and concluded that the hybrid approach relying on a stochastic method yields satisfactory noise results at a fraction of the cost of LES.

For RSI noise, a hybrid approach can be divided into three main tasks: Firstly, the sound generation mechanisms are modeled either directly or by synthesizing a turbulent field which impinges on the blade row.  Secondly, the sound is propagated considering complex duct geometries and flow. Lastly, the sound is radiated into the far field, i.e. to an observer.  The two last mentioned parts can be realized by a CAA simulation applying the Linearized Euler Equations (LEE). %As an alternative for the last step, an analytical far field extrapolation method is sufficient and even more efficient. \todo[inline]{Wenn die Alternative besser ist wie es dieser Satz impliziert, dann kann man sich als Leser fragen, warum wir das nicht so gemacht haben.  Ich würde es generell weglassen.  Das macht einen ohnehin komplizierten Absatz noch unübersichtlicher.} 
The first part, however, has proven to be the crux of the matter. For many years, the only way of modeling the sound sources was to use discrete harmonic gusts to generate the turbulent field~\cite{amiet_high_1976,scott_finite-difference_1995,peake_influence_2004,glegg_panel_2010}.  This method is still in use to model RSI noise. In fact, \citet{lau_effect_2013} have recently investigated the impingement of harmonic gusts in a three-dimensional (3D) CAA simulation to investigate the influence of wavy leading edges.

Aside from this analytically motivated method, two classes of stochastic methods are used to model broadband noise:  the Stochastic Noise Generation and Radiation (SNGR) method and the Random Particle Mesh (RPM) method.

The SNGR methods apply a random set of superposed Fourier modes to realize a target model spectrum, e.g. a von \Karman or a Liepmann spectrum. The SNGR methods can be traced back to the work of \citet{kraichnan_diffusion_1970}, who proposed the theoretical framework, and to \citet{bechara_stochastic_1994}, who was the first to apply it to predict noise generated by free turbulence. \citet{clair_experimental_2013} predicted the effects of wavy leading edges (LE) of isolated airfoils, while \citet{gill_reduced_2014} investigated real symmetric airfoils at zero angle of attack. To predict RSI noise \citet{polacsek_numerical_2015} have presented an approach simulating only one stator vane with periodic boundary conditions in circumferential direction. The far-field signature is obtained by extrapolating the instantaneous pressure on the blade surface.

The RPM method by \citet{ewert_caa_2011} synthesizes the turbulent fluctuations by spatially filtering white noise. The mean turbulent quantities are taken from a preceding RANS simulation. This method is now established and has been successfully applied to model different sources such as jet noise~\cite{ewert_three-parameter_2012}, slat noise~\cite{ewert_broadband_2008}, haystacking~\cite{siefert_sweeping_2009} or airfoil self noise~\cite{cozza_broadband_2012}. For the prediction of fan noise, the method is also applied. \citet{kim_advanced_2015} applied it to investigate the turbulence interaction with a flat plate in two- and three-dimensional space. They generated a von \Karman model spectrum by an optimization technique utilizing a set of Gaussian and Mexican hat filters. The divergence-free turbulence is coupled into a CAA domain by a sponge-layer technique. For centrifugal fans \citet{heo_unsteady_2015} have applied the RPM method to time-periodic flow with cyclostationary turbulence. The synthetic fluctuations are used as sources in an acoustic analogy solved by a boundary element method. A sufficient agreement with measurements is only achieved if cyclostationarity is considered.

% In previous years, we have investigated various methods for coupling turbulent
% sources into the CAA domain\cite{wohlbrandt_simultaneous_2013}.  However, the
% most reliable method proved to be to impose turbulent fluctuations upstream of
% the stator vanes. Three predominant methods have emerged: \textbf{(1) a
% manipulation of Tam's radiation boundary condition~\cite{tam_dispersion-relation-preserving_1993}, (2) the sponge approach, and (3) the LEE-relaxation method~\cite{ewert_linear-_2014}.} In this paper, the latter is used.  The advantage is that it generates turbulent fluctuations without influencing the acoustic radiation and thus, the seeding region can be positioned close to the source region.

To compare the results to experimental data, it is crucial to use realistic model spectra. \citet{dieste_random_2012} have derived complex filter stencils to model von \Karman spectra directly. This turns out to be computationally intensive. A more efficient solution consists in empirically weighting Gaussian filters with different length scales and amplitudes~\cite{kim_advanced_2015,hainaut_caa_2016}. \citet{wohlbrandt_analytical_2016} have derived analytical weighting functions in order to realize typical isotropic turbulence spectra by superposition of Gaussian spectra. They also showed that the reconstruction with five discrete realizations is sufficiently accurate to cover a frequency range with a change of one order of magnitude when the spectra are logarithmically distributed.  The current article will show how this technique can be used to simulate length scales varying in space and time while realizing temporally and spatially constant Gaussian filtered fields.

%%%%%%%%%%%%%%%%%%%%%%%%%%%%%%%%%%%%%%%%%%%%%%%%%%%%%%
%   RESEARCH QUESTION
%   - which question is this paper trying to answer to address complication
%%%%%%%%%%%%%%%%%%%%%%%%%%%%%%%%%%%%%%%%%%%%%%%%%%%%%%

Broadband noise in the fan stage is caused by the interaction of the turbulence in the rotor wakes with the surfaces at the leading edges of the stator vanes in the presence of a time-periodic mean flow.  %The focus of the paper is to develop a method, which facilitates the study of time-periodic parameters and their impact on the broadband noise levels.  In particular, the method allows to separately study the effects of time-periodic variations of length scale, turbulent kinetic energy, and mean flow. 
%%%%%%%%%%%%%%%%%%%%%%%%%%%%%%%%%%%%%%%%%%%%%%%%%%%%%%
%   SOLUTION 1
%   - how are we going to answer the research question in this paper
%%%%%%%%%%%%%%%%%%%%%%%%%%%%%%%%%%%%%%%%%%%%%%%%%%%%%%
%%%%%%%%%%%%%%%%%%%%%%%%%%%%%%%%%%%%%%%%%%%%%%%%%%%%%%
%   SOLUTION 2
%   - how is the discussed S1 contributing to science
%%%%%%%%%%%%%%%%%%%%%%%%%%%%%%%%%%%%%%%%%%%%%%%%%%%%%%
The first objective of the current paper is to utilize the RPM method to develop the cyclostationary stochastic hybrid (CSH) method to include time-periodic turbulence variations and mean flow, which are essential in studying broadband noise generation of rotating parts. 
This method allows for an in-depth study of cyclostationary turbulence and therefore contributes to a greater understanding of broadband noise generation in fans. This is especially important for the development and improvement of analytical tools. The second objective is to separately study the impact of the different effects due to cyclostationarity. A preliminary study utilizing this method was presented by \citet{wohlbrandt_extension_2015}. This article consolidates the method and applies it to another fan, for which measurement data have been made available.
% \todo[inline]{Ich denke, dass das der richtige Punkt ist um Unterschiede ganz deutlich auszuführen.  Ich denke nicht, dass Gleiches auf uns zutrifft wie auf das von Sebastien erwähnte Paper.  Die Liste der Unterschiede ist lang.  Die Formulierungen, Struktur, Bilder und Erkenntnisse sind auch unterschiedlich.  Ich fände es gut, das 2015 Paper tatsächlich als Vorstudie darzustellen.} 

%%%%%%%%%%%%%%%%%%%%%%%%%%%%%%%%%%%%%%%%%%%%%%%%%%%%%%
%   short outline of paper
%%%%%%%%%%%%%%%%%%%%%%%%%%%%%%%%%%%%%%%%%%%%%%%%%%%%%%

This paper is structured as follows:  The used hybrid approach, the extensions for including cyclostationarity, the general procedure for the setup of such a computation as well as evaluation methods are discussed in Section~\ref{sec:method}.  In Section~\ref{sec:Application}, the CSH method is demonstrated by applying it to the NASA Source Diagnostic Test (SDT) fan.  The effects of cyclostationary wake characteristics on the fan broadband noise are discussed in Section ~\ref{sec:results}.  Additionally, the numerical sound power spectra are compared to experimental data.  Key features of the method as well as significant findings are summarized in Section ~\ref{sec:conclusion}.      

% \todo[inline]{Outline}
% State-of-the-art:  give a short time line of broadband noise simulation (if we decide to not clutter the introduction with that)
% Method:  what needs to go in there?  LEE formulation for periodic background flow, application of VanVliet, Von \Karman... ? 
% Research setup: Q3D URANS, design of patches, modified geometry…
% Results:  show some pretty pictures, maybe discuss freq. peaks, parameter study?
% Discussion:  compare obtained numerical results with experimental data, discuss why simulating 1/11 of total fan is a justified simplification
% Conclusion/ Outlook:  give an idea of how we want to apply this method in the future,  what we want to study with itThe objective of the paper is to present

% \subsection*{Journals only (sorted by year)}
% \bit
% 	\item Ewert\cite{ewert_broadband_2008} 
% 		\bit
% 			\item Slat-noise with RPM
% 		\eit
% 	\item Jurdic et al.\cite{jurdic_investigation_2009}
% 	\item Ewert et al.\cite{ewert_caa_2011}
% 		\bit
% 			\item
% 		\eit
% 	\item Kim et al.\cite{kim_proposed_2010}
% 		\bit
% 			\item Optimization of sponge zone technique for gust-airfoil interaction
% 			\item Innovation 1: Forcing pressure instead of total energy (as we do anyways?)
% 			\item innovation 2: introducting an additional factor for impulse equations, which forces incomming harmonic gusts only upstream of 2D plate.
% 			\item also use 2L for sponge depth and forcing of 4$\sigma$
% 		\eit		
% 	\item Dieste and Gabard\cite{dieste_random_2012}
% 		\bit
% 			\item Alternatve Filters 
% 			\item No reference to cyclostationarity. Periodic TKE only mentioned in Disseration
% 			\item Evolving turbulence negligible for RSI.
% 			\item Restricted to flat plates
% 		\eit
% 	\item Cozza et al.\cite{cozza_broadband_2012}
% 		\bit
% 			\item RPM for TE-noise
% 		\eit
% 	\item Lau et al.\cite{lau_effect_2013}
% 		\bit
% 			\item Harmonic gusts for 3D-CAA to investigate wavy leading edges.
% 		\eit
% 	\item Clair et al.\cite{clair_experimental_2013}
% 		\bit
% 			\item 3D Non-linear LEE with HIT at inflow (Tam's BC)
% 			\item Investigating LE serrations. Could show positive influence of serations
% 			\item Fourier-modes with von \Karman spectrum to realise wall-normal component only
% 			\item Spanwise gust contribution neglected
% 			\item Far field  data optained by FWH with unsteady wall pressure and porous formulation 
% 		\eit
% % 	\item Haeri et al.~\cite{aeri_calculations_2014} %conference proceeding
% % 		\bit
% % 			\item Extension of investigation by Lau et al.\cite{lau_effect_2013} to Fourier-Modes
% % 		\eit
% 	\item Gabard~\cite{gabard_noise_2014}
% 		\bit
% 			\item builds on method by Polacsek et al.\cite{polacsek_equivalent-source_2009}
% 			\item Generates in duct broadband noise field characterised by its modal using the duct shape function.
% 			\item Generates stochastic noise source able to produce multimode, broadband sound fields
% 			\item truely random.
% 		\eit
% 	\item Kim and Haeri\cite{kim_advanced_2015}
% 		\bit
% 			\item synthetic eddy method (SEM) derived from taking the curl of a vector potential function \todo{isn't that a stream function then?} to create divergence-free velocity field.
% 			\item Full 3D
% 			\item Mixture of Gaussian filters and mexican hat filters to empirically generate von \Karman spectrum by using optimisation strategies
% 			\item Upstream coupling by usage of Sponge-layer
% 		\eit
% 	\item Heo et al.\cite{heo_unsteady_2015}
% 		\bit
% 			\item Couples FRPM to BEM 
% 			\item uses URANS for turbulent statistics
% 			\item Incorporating unsteady flow field into FRPM (U-FRPM)
% 			\item Applies time-variant turbulence kinetic energy and turbulence dissipation extracted from unsteady RANS solutions.
% 			\item See tonal and broadband noise as result.
% 			\item No unsteady CAA as BEM is used
% 		\eit
% 	\item Ayton and Peake\cite{ayton_high-frequency_2015}
% 	\item Ju et al.\cite{ju_investigation_2015}
% 	\item Polacsek et al.\cite{polacsek_numerical_2015}
% \eit
% \subsection*{Conferences 2014-2016 (sorted by year)}
% \bit
% 	\item Allan and Darbyshire~\cite{allan_comparison_2014}
% 		\bit
% 			\item Compare LES-Acoustic-Analogy to RPM-CAA 
% 			\item Also applicable in  maritime acoustics
% 			\item conclude that RPM-CAA can provide satisfactory noise results at a fraction of the costs.
% 			\item no full paper access		
% 		\eit
% 	\item Gill et al.\cite{gill_reduced_2014}	
% 		\bit
% 			\item 
% 		\eit
% 	\item Shur et al.\cite{shur_effect_2016}	
% 		\bit
% 			\item IDDES for a full ventilator
% 		\eit
% 	\item Hainaut et al.\cite{hainaut_caa_2015,hainaut_caa_2016}	
% 		\bit
% 			\item
% 		\eit
% 	\item Haeri et al.\cite{haeri_3d_2014}
% 		\bit
% 			\item
% 		\eit
% 	\item Santana et al.\cite{santana_boundary_2014}
% 		\bit
% 			\item
% 		\eit
% 	\item Ayton et al.\cite{ayton_comparison_2015}
% 		\bit
% 			\item unsteady airfoil interaction comparing experiment, analytical and numerical
% 			\item Method by Gill et al.\cite{gill_symmetric_2013}
% 			\item Improve match between analytical and numerical solution if singularity at TE is incorporated into analytical solution
% 		\eit
% 	\item Bouley et al.\cite{bouley_mode-matching_2015}
% 		\bit
% 			\item
% 		\eit
% 	\item Gea-Aguilera et al.\cite{gea-aguilera_synthetic_2015,gea-aguilera_leading_2016}
% 		\bit
% 			\item
% 		\eit
% 	\item Grace\cite{grace_further_2015}
% 		\bit
% 			\item
% 		\eit
% 	\item Kim et al.\cite{kim_mechanisms_2015}
% 		\bit
% 			\item
% 		\eit
% 	\item Paruchuri et al.\cite{paruchuri_aerofoil_2015}
% 		\bit
% 			\item
% 		\eit
% 	\item Casalino et al.\cite{casalino_turbofan_2016}
% 		\bit
% 			\item
% 		\eit
% 	\item Geyer et al.\cite{geyer_noise_2016}
% 		\bit
% 			\item
% 		\eit
% \eit
% \subsection*{Our pubs}
% \bit
% 	\item \cite{wohlbrandt_simultaneous_2013}
% 		\bit
% 			\item
% 		\eit
% 	\item \cite{wohlbrandt_extension_2015}
% 		\bit
% 			\item
% 		\eit
% 	\item \cite{wohlbrandt_analytical_2016}
% 		\bit
% 			\item
% 		\eit
% \eit