\section{Conlusion and Outlook} \label{sec:conclusion}
% \subsection*{Conclusion}
The cyclostationary stochastic hybrid (CSH) method has been developed based on the RPM method. It enables to account for the effect of turbulence cyclostationarity in fan broadband noise. Since each parameter relevant for the wake turbulence description can be activated individually, it is possible to assess their respective impacts  on broadband RSI noise.  Those parameters are the turbulence kinetic energy, the integral turbulence length scale, and the mean flow.     
% and their impact on the broadband RSI noise levels.  It expands the fRPM formulation to include time-periodic turbulence variations and time-periodic mean flow.  
%A high flexibility to switch single aspects of cyclostationarity on and off allows to separately study the effects and role of time-periodic variations of length scale, turbulent kinetic energy, and mean flow.
%The method enhances the fast Random Particle Mesh (fRPM) formulation to include time-periodic turbulence variations and time-periodic mean flow. 

The periodic mean flow and periodic turbulence statistics are extracted from a URANS solution that makes use of a Hellsten $k$-$\omega$ turbulence model. The values are passed as Fourier-coefficients to the CAA and fRPM domains. The data can therefore be reconstructed at any arbitrary time step. 

The benefit of using the fRPM-method is a local realization of von \Karman spectra of arbitrary integral length scale and variance in the time domain via analytical weighting of Gaussian spectra. This yields a stable formulation for time variations, even at the steep spatial gradients of TKE and TLS between wake and background turbulence.


% \begin{itemize}
%  \item A new method based on the RPM approach has been developed to model and investigate the effect of cyclostationarity on rotor--stator interaction broadband noise.
%  \item Was it a big challenge? What was modified/extended in the method?
%  \item Advantage of RPM: local realisation of the turbulence in the time domain (via weighting of the Gauss spectra!)
%  \item A URANS solution with a $k-\omega$ provides the periodic mean flow and the periodic turbulence statistics
%  \item Flexibility to switch on/off the variation of mean flow and turbulence statitics and enhance our understanding of the role of those parameters
%  \item Possibility to identify the role of each parameter 
%  \item Restricted here to 2D and ``local isotropy'' with a von \Karman spectral distribution of the velocity fluctuation
%   \item 2D--3D correction enables to reproduce the measurements fairly well based on data at 50\%. 
% \end{itemize}

One main aspect that motivated this study and the development of CSH method was to investigate which error is made by circumferentially averaging the turbulence and the mean flow before calculating the aeroacoustic blade response.  This averaging technique is commonly used in analytical and most hybrid approaches.  In the investigations presented in this paper, we have observed the following:   
\begin{itemize}
 \item The turbulent length scale is a key parameter. The result depends on the way it is calculated. A simple averaging is not adequate. We propose an alternative method: averaging of the energy/velocity spectrum and fitting that averaged spectrum for the determination of the length scale. With this averaging technique, we can reproduce the results of the fully periodic test case with a constant assumption.
 \item The influence of the cyclostationarity of the TKE and of the mean flow on the sound power levels is negligible for the investigated fan simulation at approach condition.
 \item An analytical investigation shows that the consideration of cyclostationarity is imperative when increasing the background turbulence intensity from 0.1\% to a still realistic 1.0\%. The resulting upwash velocity frequency spectrum differs significantly from a stationary isotropic model spectrum.  This will have a direct impact on the emitted sound.
 \item Furthermore, we have verified that using a reduced number of vanes (5 instead of 55) is a reasonable approach to reduce the computational effort without compromising the results.  Slight differences in the results can only be observed in the low frequency range below the cutoff frequency of the first higher oder mode. Due to the small length scales, it is sufficient to trigger only one blade and consider the signal emitted by all blades to be uncorrelated.
\end{itemize}

% \subsection*{Outlook}
The CSH method allows for future, in-depth study of cyclostationary turbulence and will therefore further the understanding of broadband noise generation in fans.  This is especially important for the development and improvement of analytical tools. 

By using stationary spectral analysis to investigate the cyclostationary signals, interesting features are removed from the signal. To look at the intermittency or ''noise events'' of each wake, a cyclostationary analysis~\cite{gardner_cyclostationarity:_2006} has to be utilized. This would highlight variations in the lift coefficient and in the pressure distribution over the time.

% Extension to 3D and anisotropy of the turbulence are also necessary to come nearer to the reality. 
The CSH method is currently applied only in two spatial dimensions and uses locally isotropic von \Karman spectra. A 2D-3D correction of the resulting spectra makes it possible to reproduce the measurements fairly well based on numerical data at 50\% duct height at the stator LE. An extension to 3D is contained inherently in the method and will be in focus of future studies.  In addition, modeling anisotropic turbulence by utilizing higher order turbulence models in the URANS solutions is envisioned.


% Carolin
% \begin{itemize}
%  \item Extension of (F)RPM works 
%  \item Agreement with experimental data reasonable
%  \item For this case:  cyclostationarity of TLS has most impact
%  \item For this case:  averaging technique has major impact on TLS
%  \item In general: there are cases where neglecting the cyclostationarity will result in incorrect velocity freq. spectra of turbulence
% \end{itemize}


\section{Acknowledgement}
The authors would like to thank Roland Ewert, Jürgen Dierke, and Nils Reiche (DLR) for their advice, enlightening discussions, and valuable time.
Furthermore we would like to acknowledge \citet{envia_panel_2015} for the organization of the fan broadband noise workshop held at the AIAA Aviation conference since 2014. A special thanks goes to Edmane Envia at NASA Glenn Research Center for setting up this test case and providing the extensive data for the SDT fan.



% The authors acknowledge the financial support of the European Commission via the FP7 Collaborative Project IDEALVENT (Grant Agreement no 314066).

