
\section{Results and discussion}\label{sec:results}

In the following section, the results of the test matrix are discussed.  At first, the primary test matrix is examined to study the effects of cyclostationarity.  Next, an analytical test case is analyzed to substantiate and augment the findings of the first subsection.  The comparison to the measurements can be found in subsection~\ref{sec:meas}. In the last two subsections, elemental assumptions made during the design of the primary test matrix are checked: 1.) The authors assumed that results of a simulation on a reduced duct containing only five stator blades are equivalent to results of a simulation on the full duct containing all 55 stator blades. 2.)  The authors assumed that the RSI noise generation mechanism of each blade is uncorrelated to that of all other blades in the duct and that it is therefore sufficient to only study one blade.

\subsection{Analysis of the primary test cases}
%/usr/kiely/kissner/simulations/RC2/piano/TestMatrix_Final/lay_powerspectra_all.lay
\begin{figure}
\centering
\includegraphics[width=0.9\textwidth,trim=4 4 1 1,clip]{figures/eps_PowerSpectra_all.pdf} 
\caption{Comparison of sound power spectra of the primary test matrix. The greyed-out regions indicate frequencies outside the range, for which the CAA mesh was designed. \label{fig:PowerSpectra} }
\end{figure}

The primary test matrix was designed to systematically study the effect of cyclostationarity in the turbulence and in the mean flow on the broadband RSI noise.  The study was conducted for a well-known fan, i.e. the NASA SDT fan in its baseline configuration.

In Figure~\ref{fig:PowerSpectra} the sound power level  $L_W$ 
\begin{align}
L_W = 10 \log_{10}\left( P\right/P_\text{ref}) \label{eq:soundPowerLevel}
\end{align}
are shown for all primary test cases with $P$ in Eq.~\ref{eq:P} and $P_\text{ref} = \SI{1e-12}{\watt}$ calculated up- and downstream of the stator.  The grayed-out areas in the graph indicate frequencies outside the range of the target CAA mesh resolution. As the mesh resolution is twice as high as recommended in the literature, results up to \SI{40}{kHz} are deemed to be trustworthy.

When neglecting the periodic nature of the mean flow, the difference seems to be negligible for this configuration as can be concluded when comparing the P-PP and the C-PP primary test cases.  However, when the TLS is taken to be constant over the stator pitch (C-PC) instead of periodic (C-PP), it results in a notable offset in sound power. The offset is large at lower frequencies and disappears at very high frequencies.  Lastly, there is little to no difference in the sound power spectra between the C-CC and C-PC test cases indicating that the cyclostationarity of the TKE does not influence the sound power levels.  To sum up, only the cyclostationarity of the TLS influences the RSI noise for the baseline SDT configuration at approach conditions.

%/usr/kiely/kissner/simulations/RC2/piano/TestMatrix_Final/lay_patchspectra_analytical.lay
\begin{figure}
\centering
\includegraphics[width=0.9\textwidth,trim=4 4 1 1,clip]{figures/eps_PatchSpectra.pdf} 
\caption{Upwash velocity spectra shown for constant and fully periodic patches.  Comparisons to the respective analytical 2D velocity spectra computed with Eq.~\ref{eq:E22_Karman_2D} and Eq.~\ref{eq:Sii} are shown. \label{fig:PatchSpectra} }
\end{figure}
\subsection{Alternative averaging to realize correct TLS}
% \todo[inline]{I focused exclusively on the upwash velocity freq. spectrum here since I eliminated it from the graphic.  Recheck to see if you are in agreement.}
In order to examine the effect of the TLS in more detail, the authors took a closer look at the synthesized turbulence in the fRPM vortex source patch.  In Figure~\ref{fig:PatchSpectra}, the lateral 2D velocity frequency spectrum $S^{2D}_{22}(f)$, which is most relevant for the broadband noise generation at the stator LE, realized by the patches for configurations C-CC and C-PP along with their respective analytical solutions are shown. For the constant case, the analytical solutions are calculated using the Eq.~\ref{eq:E22_Karman_2D} and Eq.~\ref{eq:Sii} for the 2D von \Karman turbulence spectrum. For this purpose the turbulent kinetic energy, the turbulent specific dissipation rate and the flow velocity are circumferentially averaged: 
\begin{align}\label{eq:LambdaC}
k_t^C = \frac{1}{2\pi}\int\limits_0^{2\pi} k_t(\vartheta)\abl \vartheta, &&
\omega_t^C = \frac{1}{2\pi}\int\limits_0^{2\pi} \omega_t(\vartheta)\abl \vartheta, &&
u_0^C = \frac{1}{2\pi}\int\limits_0^{2\pi} u_0(\vartheta)\abl \vartheta.
\end{align}
The TLS was determined by Eq.~\ref{eq:Lambda}. For the fully periodic case (P-PP), the analytical velocity spectrum results from integrating over the analytical spectra at each circumferential point of the downstream patch border:
\begin{align}\label{eq:Spii}
S^\text{P}_{ii}(f) = \frac{1}{2\pi}\int\limits_0^{2\pi}S_{ii}(f,\vartheta) \abl \vartheta.
\end{align}  
The TLS is then determined by fitting the averaged velocity frequency spectrum to a standard von \Karman velocity spectrum.  

It can be noted that the numerical spectra match the respective analytical spectra well, particularly at high frequencies.  There is an offset at lower frequencies, which may indicate that the turbulence is lacking energy at those frequencies.  The pronounced offset between the fully periodic and fully constant spectra is most significant.  When a fit is performed for the PP spectra,  the averaged TLS was $\SI{0.0023}{m}$.  In contrast, the averaged TLS for the CC spectra was $\SI{0.0039}{m}$, while the averaged TKE and flow velocity were nearly equivalent.  These findings indicate that the method used for averaging plays a significant role in determining the turbulence characteristics.  Averaging the TKE, TLS, and mean flow over the circumference before calculating velocity frequency spectra (CC) gives different results than calculating spectra for each TKE, TLS, and mean flow before averaging the spectra over the circumference (PP).  
%This could explain part of the offset observed in Figure~\ref{fig:PowerSpectra}. 
%/usr/kiely/kissner/simulations/RC2/piano/TestMatrix_Final/lay_powerspectra_ls.lay
\begin{figure}
\centering
\includegraphics[width=0.9\textwidth,trim=4 4 1 1,clip]{figures/eps_PowerSpectra_ls.pdf} 
\caption{C-CC test case with adjusted turbulent length scale compared to primary test cases C-CC and P-PP \label{fig:PowerSpectra_ls} }
\end{figure}

To further test these findings, the C-CC-tls configuration was simulated.  Instead of using an averaged TLS of $\SI{0.0039}{m}$, a constant TLS of $\SI{0.0023}{m}$ as determined by the PP upwash velocity frequency spectrum was imposed onto the patch.  If the hypothesis that the averaging technique is essential when considering cyclostationary processes is true, the simulation with the modified TLS should produce the same results as the fully periodic simulation (P-PP).  In fact, this is confirmed by the power spectra shown in Figure~\ref{fig:PowerSpectra_ls}.                 

\subsection{Impact of cyclostationarity for an analytic test case}
% \todo[inline]{Text wrap this image?}
%/usr/kiely/kissner/m-files/analyticalSpectra/lay_wakeTKE.lay
\begin{figure}[htb]
\parbox{0.39\textwidth}{
% \begin{wrapfigure}{r}{0.4\textwidth,trim=4 4 1 1,clip}
\centering
\includegraphics[width=0.39\textwidth,trim=4 4 1 1,clip]{figures/eps_TKEwake.pdf} 
\caption{Turbulent kinetic energy of extracted wake.  Background turbulence modified to demonstrate impact on cyclostationarity.\label{fig:TKEWake_analytical} }
% \end{wrapfigure}
% \end{figure}
%/usr/kiely/kissner/m-files/analyticalSpectra/lay_SpectraTKE.lay
}\hfill
\parbox{0.59\textwidth}{
% \begin{figure}
% \centering
\includegraphics[width=0.59\textwidth,trim=4 4 1 1,clip]{figures/eps_TKESpectra.pdf} 
\caption{Analytical 2D velocity spectra calculated for modified wake using different averaging techniques. \label{fig:TKESpectra_analytical} }
}
\end{figure}
In the previous subsection, results of the fully periodic test case were reproduced using a fully constant simulation with a TLS determined by a different circumferential averaging technique.  This finding is convenient, since it allows for the use of a less computationally expensive technique granted that the TLS is chosen accordingly. However, for all we know, this finding is only valid for this particular fan configuration at this particular operating point.  To test this, the authors aimed at finding an analytical test case, for which the P-PP case cannot be correctly reproduced by a C-CC case.

%Initial: TKE=0.056603, v ~ 156, TI = 0.12452%
%Modified: TKE=5, v~156, TI = 1.17%

In order to achieve this, we slightly manipulated the initial TKE.  Figure~\ref{fig:TKEWake_analytical} shows the TKE of two wakes over one rotor passage.  The initial wake was extracted from the (U)RANS simulation.  The TKE of the background turbulence was set to a constant value to clearly show the difference between the initial and modified cases.  The constant TKE value of the background turbulence is equivalent to the extracted, slightly fluctuating values and results in the same velocity frequency spectra as before.  For the modified case, the TKE in the wake remained the same.  Only the turbulence intensity of the background turbulence was increased from 0.1\% to 1\%.  All other variables remained unchanged.  The resulting velocity frequency spectra by applying the different averaging techniques for the modified wake are shown in Figure~\ref{fig:TKESpectra_analytical}.  As observed in the previous subsection, the periodic variation of the TKE and the mean flow have no impact.  Though the difference in the shape of the velocity frequency spectra due to the periodicity of the TLS is compelling, particularly when considering the direction perpendicular to the flow.  The fully periodic velocity frequency spectrum now exhibits two pronounced bumps instead of just one.  The bump at the lower frequency is due the background turbulence, while the bump at a higher frequency can be attributed to the wake turbulence.  For this hypothetical case, test cases using a constant value for the TLS will never be able to reproduce the shape of the spectrum correctly.


The analytical case with an increased background turbulence intensity proved that there are cases for which the best averaging technique is of no use and cyclostationarity must be simulated in order to synthesize realistic turbulence.      



\subsection{Comparison to measurements}
\label{sec:meas}
\begin{figure}
\centering
\includegraphics[width=0.9\textwidth,trim=4 4 1 1,clip]{figures/eps_PowerSpectra_experiment.pdf} 
\caption{Comparison of numerical and experimental sound power level spectra. \label{fig:PowerSpectraExp} }
\end{figure}
The fully periodic case (P-PP) most accurately reproduces the actual physics.  
For this case the sound power levels are compared to the measurements\footnote{The rotor-stator noise contribution was obtained by subtracting the rotor-alone results from the overall noise results (rotor+stator).} in Figure~\ref{fig:PowerSpectraExp}. The overall trend and levels are reproduced.  While the experimental data differs from the numerical results up- and downstream of the fan in regions below \SI{10}{kHz}, the high-frequency fall off is well predicted. An under-prediction of the sound power at lower frequencies has also been shown by \citet{nallasamy_computation_2005}, who used a RANS-informed, analytical method, at both inlet (upstream) and exhaust (downstream). %\todo[inline]{SG: Mit welcher Methode}

The differences can be explained twofold: (1) The measurements may have contributions from additional noise sources (e.g. rotor trailing edge, jet noise).
(2) Some simplifications and assumptions were made in the numerical hybrid approach. The TKE, TLS, and mean flow were taken from a (U)RANS simulation and a locally isotropic von \Karman spectrum was assumed.  The data was used "as is" and no adjustments were made to achieve a better agreement with experimental results.  Additionally, the used approach is two-dimensional and can only simulate broadband noise resulting from the interaction of turbulence with the blade surfaces.  Any other sound sources that may have been captured by the measurements cannot be considered by this approach.  The method also neglects the rotor, i.e. transmission losses or reflections at the rotor are disregarded.  %Another issue is that the stator was never measured by itself.  Instead, experiments were conducted for a rotor-alone and a rotor-stator configuration.  To determine sound power spectra up- and downstream of the stator, the sound power of the rotor-stator configuration was subtracted by the rotor-alone configuration.  It is uncertain if the thusly determined sound power spectra is truly correct as it assumes that sound sources of rotor and stator are uncorrelated and can therefore be summed up to calculate the total sound.         
% Note that a seemingly better fit of the original C-CC test case to the experimental data is considered to be a coincidence by the authors. The P-PP configuration reproduces the actual physics more accurately.

\subsection{Comparison of full and reduced duct computations}
%/usr/kiely/kissner/simulations/RC2/piano/TestMatrix_Final/lay_powerspectra_fullred.lay
\begin{figure}
\centering
\includegraphics[width=0.9\textwidth,trim=4 4 1 1,clip]{figures/eps_PowerSpectra_fullred.pdf} 
\caption{Comparison of the reduced and full cascade using the C-CC configuration.  Cut-on frequencies for azimuthal modes $m=1$ and $m=-1$ are shown. \label{fig:PowerSpectraFullRed} }
\end{figure}
The first elemental assumption that the authors made in the design of the primary test matrix was to assume that simulating a reduced, periodic duct is equivalent to simulating the full duct.  To test this assumption, the C-CC-55 simulation was performed containing all 55 stator vanes.  The resulting narrow-band sound power spectra are shown in Figure~\ref{fig:PowerSpectraFullRed}.  The power spectra are, in fact, nearly equivalent.  Only at lower frequencies, the power spectra of the full cascade is smooth while there are peaks in the power spectra of the reduced cascade. Peaks in fan power spectra are often indicative of where a new acoustic mode of azimuthal order $m$ suddenly becomes cut-on.  The equations for calculating cut-on frequencies of acoustic modes are listed in the Appendix~\ref{app:cuton}.  Aside from the azimuthal mode order, the cut-on frequency depends on the flow speeds and the geometry.  In this case, the duct geometries differ:  The reduced duct with only five stator blades has a smaller circumference than the full duct. For these cases, the relevant Mach numbers are: 
\begin{align*}
	\text{upstream} &&&M_x = 0.40,	&& M_y =-0.21,\text{ and}\\
	\text{downstream} &&&M_x = 0.44,	&& M_y = 0.00.
\end{align*} 
The resulting characteristic frequencies for the first azimuthal mode orders are:

% first column
\begin{minipage}[t]{0.5\textwidth}
\begin{itemize}
	\item 5-vane configuration 
	\begin{itemize}
		\item upstream
		\begin{itemize}
			\item[] $f_{m=-1} = 3169.6$~Hz, 
			\item[] $f_{m=+1} = 1968.7$~Hz, and
		\end{itemize}
		\item downstream 
		\begin{itemize}
			\item[] $f_{|m|=1} = 2508.2$~Hz,
		\end{itemize}
	\end{itemize}
\end{itemize}
\end{minipage}
%second column
\begin{minipage}[t]{0.5\textwidth}
\begin{itemize}
	\item 55-vane configuration 
	\begin{itemize}
		\item upstream
		\begin{itemize}
			\item[] $f_{m=-1} = 303.9$~Hz, 
			\item[] $f_{m=+1} = 178.4$~Hz, and
		\end{itemize}
		\item downstream 
		\begin{itemize}
			\item[] $f_{|m|=1} = 237.5$~Hz.
		\end{itemize}
	\end{itemize}
\end{itemize}
\end{minipage}
\vspace{0.5em}  


For the full cascade, the cut-on frequencies are very low meaning that the acoustic modes of the first and subsequent azimuthal mode orders are cut-on for most of the frequency range.  The determined cut-on frequencies for the reduced cascade are higher and align well with the peaks in the power spectra - both up- and downstream of the stator [see Figure~\ref{fig:PowerSpectraFullRed}].  Aside from the peaks due to the cut-on frequencies, the assumption of the authors was correct and the reduced duct does reproduce the sound power correctly.     


\subsection{Acoustic correlation of vane blades}

%/usr/kiely/kissner/simulations/RC2/piano/TestMatrix_Final/lay_powerspectra_2pitches.lay

The second elemental assumption that the authors made was to assume that the investigated sound generation mechanism of each blade is uncorrelated to that of the other blades.  This therefore allows for investigating the impingement of turbulence on only one blade and for multiplying the determined sound power of one blade by the number of blades to receive the total sound power.  In order to confirm this assumption, configuration C-CC-double was investigated with a patch spanning two pitches and therefore turbulence impinging onto two stator blades.

\begin{figure}
\centering
\includegraphics[width=0.9\textwidth,trim=4 4 1 1,clip]{figures/eps_PowerSpectra_2Pitches.pdf}
\caption{Comparison of different patches to check assumption of uncorrelated blades. \label{fig:PowerSpectra_2Pitches} }
\end{figure}
	
An interim step was necessary due to the fact that a patch spanning over two vanes with a lateral safety margin of the same size as the original patch did not fit into the CAA domain for the reduced cascade.  For this interim step, configuration C-CC-red used a patch without a damping zone in lateral direction [see subsection~\ref{subsec:SetupFRPMCAA}].  Figure~\ref{fig:PowerSpectra_2Pitches} shows that the resulting power spectra for the initial patch, used for the primary test matrix, and this patch are identical.  This confirms that the Young-Van-Vliet Filter does, in fact, not need a safety margin in the lateral direction.  Since a different random variable field also had to be used, it also confirms that the solution is independent of the used random variable field.  Since the results were the same, the initial patch spanning one pitch can directly be compared to the patch spanning two pitches.  Both patches yield exactly the same sound power levels [see Figure~\ref{fig:PowerSpectra_2Pitches}].  This confirms that there is no significant acoustic correlation of the vane blades. This result has been anticipated as the TLS is much smaller than the pitch.

%\todo{AW: There is a gray border on all of your pics. 
%\\I have added a trim and clip to remove them, but
%\\these could be removed by removing the alpha channel (transparency) of the original image, I guess... }