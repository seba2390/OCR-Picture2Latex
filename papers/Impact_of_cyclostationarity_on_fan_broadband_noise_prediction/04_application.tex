\section{Application}\label{sec:Application}


The CSH method described in the previous section was applied to NASA's 22-in Source Diagnostic Test (SDT) fan at approach condition.  The effects of cyclostationary parameters were examined and the numerical data were compared to experimental data presented at the Fan Broadband Noise Prediction Workshop organized in the framework of the AIAA 2014 and 2015 Aeroacoustics Conferences.  The experimental setup of the Realistic Test Case 2 (RC2) was described in detail by \citet{nallasamy_computation_2005}.  The operating conditions are given in table~\ref{tab:operatingPoint} and the input specifications, in the workshop problem statement~\cite{envia_panel_2015}.    

% \todo[inline]{CK: Brauchen wir sowohl Approach und Design?  Mit Design haben wir nicht gerechnet.  Und falls wir Design da lassen, müssten wir vielleicht alle punkte angeben.
% \\AW: Habe einen Satz in die Tabelle eingefügt, \\AW@SG: was sagst du dazu?} 

\begin{table}
		\centering
		\caption[Fan characteristics]{Fan characteristics and operating points of the NASA-SDT fan. Design point is not used, but shown here for reference. \label{tab:operatingPoint}}%~\cite{hughes_aerodynamic_2002,moreau_measurements_2013,moreau_unified_2016} 
		\begin{tabular}{|l|r|r|}
		\hline
		Fan diameter   		& \multicolumn{2}{ c|}{\SI{0.56}{m}}	\\
		Rotor blade count $N_B$
							& \multicolumn{2}{ c|}{22} 			\\
		Stator vane count $N_V$ 
							& \multicolumn{2}{ c|}{54} 			\\
		\hline
		\hline
			&\textit{Design} &\textit{Approach} \\
		\hline
		Fan-Pressure ratio $\Pi$
							& 1.48			& 1.15		\\
		Axial Mach number $M_x$ in rotor plane
							& 0.59			& 	0.31	\\
		Relativ tip Mach number $M_\text{tip,rel}$
							&	1.39		&   0.80		\\ %M_tip= Omega*R/c_0; Mrel= sqrt(M_x^2+Mtip^2)
		\hline
		\end{tabular}
		\end{table}
% \afterpage{
% 		\begin{table}
% 		\caption[Randbedingungen für die beiden Fangeometrien]{Randbedingungen für den Betriebspunkt \textit{Approach} der beiden Fangeometrien. Alle Werte sind bei 50\% Kanalhöhe aus der Simulation extrahiert mit Ausnahme der turbulenten Eigenschaften zur Rotor-Stator-Interaktion-Simulation (RSI) des RC2-Fans, hier werden direkt die Werte aus der Hitzdrahtmessung an der Messebene 2 (Statorvorderkante) verwendet.\label{tab:operatingPointsBC}}
% 		\centering
% 		\begin{tabular}{|c|r|r|r|}
% 		\hline
% 		&					& RC2-Fan     		& UHBR-Fan \\
% % 		& Betriebspunkt		&\multicolumn{2}{c|}{\textit{approach}}\\
% 		\cline{2-4}
% 		& Drehgeschwindigkeit $n_\text{RPM}$	
% 							& 7808~rpm			& 3187~rpm\\
% 		\multirow{3}{*}[-2mm]{\parbox[c]{2mm}{\rotatebox[origin=c]{90}{RANS-Einlass}}}
% 		& Massenstrom $\dot m$
% 							& $\SI{26.92}{\kg\per\second}$ 
% 												& $\SI{47.33}{\kg\per\second}$ \\
% 		& Totaldruck $p_t$ 
% 							& 
% 							$\SI{101332.5}{\Pa}$					& $\SI{101325}{\Pa}$ \\
% 		& Totaltemperatur $T_t$ 
% 							& $\SI{288.16}{\K}$
% 												& $\SI{288.15}{\K}$ \\ 
% 		\cline{2-4}
% 		& Mach-Zahl $M_I$ 
% 							&  $\num{0.312}$ 	&$\num{0.22}$\\
% 		& TKE $k_{t,I}$ 
% 							& $\SI{0.11}{\meter^2\per\second^2}$
% 												& $\SI{2.94}{\meter^2\per\second^2}$\\
% 		& TLS $\Lambda_I$ 
% 							& $\SI{1.3e-3}{\meter} $
% 												&$\SI{4.5e-4}{\meter} $\\ 
% 		\hline
% 		\multirow{3}{*}[0mm]{\parbox[c]{2mm}{\rotatebox[origin=c]{90}{Für RSI}}} 
% % % 		Mach-Zahl $M$ (RSI)        
% % 							& 
% % 												&\\
% 		& TKE $k_{t}$
% 							& $\SI{32.57 }{\meter^2\per\second^2}$
% 												&$\SI{9.24}{\meter^2\per\second^2}$\\
% 		& TLS $\Lambda$
% 							& $\SI{5.1e-3}{\meter}$
% 												&$\SI{3.3e-3}{\meter}$,\\
% 		&					&					&($l_p     = \SI{6.9e-4}{\meter}$)\footnotemark\\ 
% 		\hline
% 		\multirow{3}{*}[0mm]{\parbox[c]{2mm}{\rotatebox[origin=c]{90}{q3D}}} 
% 		& Rotorblattabstand\footnotemark  $s_B$
% 								& $\SI{0.055}{\meter}$	& $\SI{0.0981}{\meter}  $\\
% 		& Statorblattabstand\addtocounter{footnote}{-1}\footnotemark $s_V$
% 								&   $\SI{0.024}{\meter}$	& $\SI{0.059}{\meter} $ \\
% 		& Statorsehnenlänge $c$	
% 								& 	$\SI{0.039}{\meter}$ 	& $\SI{0.081}{\meter}  $\\
% 		& Stator\textit{solidity} $c/s$	&	1.63					&	1.37				\\
% 		\hline
% 		\end{tabular}
% 		 \end{table}
% \addtocounter{footnote}{-1}
%  \footnotetext{In der Untersuchung am UHBR-Fan wird fälschlicher Weise die Pseudolängenskala $l_p$verwendet, nicht die integrale Längenskala $\Lambda$.}
% \addtocounter{footnote}{+1}
%  \footnotetext{An der radialen Position der Mischungsebene des q3D-Gitters.}
% }


\subsection{Description of applied procedure}

The CAA simulations were performed on two-dimensional cascade mesh at midspan of the annular duct at the stator leading edge. It is computed in two-dimensional space to reduce simulation cost and to, therefore, allow for parameter variations. 
%The 3D application is assumed straight forward, but has not been in the focus of the current investigation.
The solidity of the vanes is greater than one. Thus the cascade effect cannot be neglected~\cite{blandeau_comparison_2011} and all stator vanes had to be considered to correctly predict the sound propagation.  

\begin{figure}[h]
	\centering
%wohlbrandt@wood:~/RPM/fanBBN_workshop_AIAA2015/plots4JSV/lay_RC2_3DTo2D_JSV.lpk
	\begin{overpic}[width=\textwidth]{figures/eps_transform3Dto2D.pdf}
	\put( 5,55){\fcolorbox{black}{white}{1}}
	\put(45,55){\fcolorbox{black}{white}{2}}
	\put( 5,25){\fcolorbox{black}{white}{3}}
	\put(45,25){\fcolorbox{black}{white}{4}}
	\put(82,25){\fcolorbox{black}{white}{5}}
	\end{overpic}
	\caption{Transfomation of the fan geometry (1) into a two-dimensional cascade. Instantaneous axial component of the background flow in [m/s] is shown.\label{fig:RC2_transform}}
\end{figure}

% \todo[inline]{Zeige RPM patch in 4 and 5? Wäre natürlich vollständiger. }

To prepare the two-dimensional cascade domain for the CAA, a series of steps was followed as illustrated in Figure~\ref{fig:RC2_transform}. The numbers in the subsequent description correspond to the numbers in the figure. Text that appears in italics refers to topics which are further explained below the enumeration.
\begin{enumerate}
\item Firstly, the CAD geometry of the SDT fan with the baseline stator configuration has to be obtained.  For this case, the number of stator vanes was increased from 54 to 55.  The new number of vanes allowed for a reduced computation with fully periodic boundary conditions using only two rotor blades and five stator vanes.  This made the URANS and CAA computations more efficient.  The increase in the number of stator vanes was not expected to significantly change the broadband noise characteristics of the fan stage.       

\item Secondly, the CAD geometry was used to set up a three-dimensional RANS simulation for one blade passage. 

\item A q3D domain was generated by extracting streamlines at 49\%, 50\%, and 51\% of the duct height at the stator leading edge from the RANS computation.  From that a q3D URANS computation was set up with two cells in the radial direction for two rotor and five stator passages to achieve circumferential periodicity. 

\item In the next step, the streamline at \textit{50\% of the duct radius} was extracted from the q3D URANS in the stator domain only and expanded to either the full annular duct or to a fraction of the annular duct. The \textit{rotor was ignored}. The Fourier coefficients of flow and turbulent variables were then interpolated from this extracted and expanded CFD mesh onto the CAA mesh and the fRPM patch.  In areas where the CAA mesh lay outside of the  CFD stator domain, an \textit{extrapolation} was applied.  

\item Lastly, a transformation into \textit{stream-surface coordinates} was applied to the flow and geometry to produce a 2D cascade.               
\end{enumerate}
Now the keywords in italics are detailed.

\begin{figure}
% ~/RPM/fanBBN_workshop_AIAA2015/hotWireTurb/lay_hotWire_JSV.lay
		\subfigure[Turbulent kinetic energy (TKE).\label{fig:RC2_hotwire_TKE}]{
			\includegraphics[width=0.45\textwidth]{figures/eps_hotWire_TKE}}
		\subfigure[Turbulent integral lengthscale (TLS).\label{fig:RC2_hotwire_TLS}]{
			\includegraphics[width=0.45\textwidth]{figures/eps_hotWire_TLS}}
% 	\vspace{-1em}
	\caption{Turbulent characteristics determined through hotwire measurements by \citet{podboy_steady_2003} for the benchmark testcase~\cite{envia_panel_2015} at two axial positions: (1) half-way between rotor and stator and (2) in front of the stator LE. The dashed vertical line indicates the position of CAA simulation 50\%. (Reproduced with permission) \label{fig:hotwire}}
\end{figure}

\begin{description}
\item{Using a streamline at \textit{50\% of the duct radius}}
	for the q3D simulation is considered representative for the whole duct. For fan acoustic power measurements with omnidirectional microphones, this position is representative according to ISO 5136:1990~\cite{barret_noise_1960,arnold_experimentelle_1999}. Repeating the simulation at differing duct heights would increase the accuracy. The hotwire measurements reproduced in Fig.~\ref{fig:hotwire} show small variations of the turbulent characteristics in a considerably large area.  The maximum at the outer rim is due to a flow detachment in the rotor tip region. This effect cannot be accounted for here.
\item{\textit{The rotor was ignored}}
	as its consideration is computationally expensive. The error in sound radiation must be kept in mind but seems acceptable for subsonic flow without shocks \cite{moreau_impact_2016}. The main effect of the rotor is to block the acoustic waves propagating in the upstream direction. This shielding effect increases with the relative  Mach number of the rotor.    
\item{\textit{The Extrapolation}}
	of the flow quantities is used to allow for the use of a larger CAA domain, especially for the damping zones at the inflow and outflow boundaries. A nearest-neighbor interpolation was used. For the complex coefficients of the non-stationary flow data of the P-PP configurations, the phase cannot be considered as can be seen in the time-reconstructed flow at Steps 4 and 5 of Fig.~\ref{fig:RC2_transform}. The solution is nevertheless continuous and does not have a noticeable influence on the acoustic radiation.
\item{A \textit{stream-surface coordinate}}
	transformation from the q3D-grid into 2D-grid coordinates $m'$ and $\vartheta$ was applied as described in the Appendix~\ref{app:mtheta}. 
	 The benefits are that the flow quantities are transformational invariants and the circumferential distance is independent of the axially changing duct radius of the streamline. The latter allows for the application of periodic boundary conditions. For convenience, we use for the 2D grid coordinates
	\begin{align}
	x &= R_\text{LE} m', &&& y &= R_\text{LE} \vartheta,
	\end{align}
	with the radius at the leading edge $R_\text{LE}$. Hence, the chord length and pitch are invariant to the transformation.
\end{description}


\subsection{Definition of test matrix}
\label{sec:confRealised}
Eight different configurations were realized.  An overview of those configurations is given in Table~\ref{tab:conf}.  The abbreviations used to denote the test cases were introduced in Section~\ref{sec:config}.  The shown test matrix consists of five primary test cases investigating the effect of cyclostationarity and three secondary test cases confirming supplementary aspects. The same CAA grid resolution was used for all test cases.  

\begin{table}
\caption{Test matrix of simulated configurations \label{tab:conf}}
\centering\setlength\extrarowheight{2pt}
\begin{tabular}{|c|l|l|m{1cm}|m{1cm}|m{1cm}|m{1cm}|}
\hline
& type 			& specifications &
\rotatebox[origin=b]{90}{number} 
\rotatebox[origin=b]{90}{of vanes} &	
\rotatebox[origin=b]{90}{periodic}
\rotatebox[origin=b]{90}{mean flow}	& 
\rotatebox[origin=b]{90}{periodic}
\rotatebox[origin=b]{90}{TKE}          & 
\rotatebox[origin=b]{90}{periodic}
\rotatebox[origin=b]{90}{TLS} \\ \hline\hline
\multirow{5}{*}{\parbox[c]{2mm}{\rotatebox[origin=c]{90}{primary}}}
	& P-PP	&	        & 5	& 		yes		&	yes	&	yes \\ \cline{2-7}
	& C-PP	&	        & 5	& 		no		&	yes	&	yes \\ \cline{2-7}
	& C-PC	&	        & 5	& 		no		&	yes	&	no \\ \cline{2-7}
	& C-CC	&	        & 5	& 		no		&	no	&	no \\ \cline{2-7}
	& C-CC-tls	& alternative TLS	        & 5	& 		no		&	no	&	no \\ \hline \hline
\multirow{3}{*}{\parbox[c]{2mm}{\rotatebox[origin=c]{90}{secondary}}}
	& C-CC-55 & full cascade
						& 55	& 		no		&	no	&	no \\ \cline{2-7}
	& C-CC-red  & patch with reduced damping area	
						& 5	& 		no &			no	&	no \\ \cline{2-7}
	& C-CC-double  & double patch		        
						& 5	& 		no		&	no	&	no \\ \hline
\end{tabular}
\end{table}

The five primary test cases were simulated with only five stator vanes in order to reduce simulation times.  The boundary conditions for the reduced number of blades were still fully periodic, since the number of total vanes was increased to 55. For the primary test cases, the fRPM patch remained unchanged so that the influence of every parameter could be investigated.  The cases have been described in section~\ref{sec:config}. Figures~\ref{fig:PPPinCAA} and \subref{fig:CCCinCAA} show the instantaneous vorticity fields for the P-PP and the C-CC cases, respectively. The P-PP case is the most realistic test case realizing full cyclostationarity. The structure of the wake is reproduced in the mean flow, as shown by the contour of the axial velocity, and in the wake turbulence, as shown by the contour of the vorticity magnitude. The primary C-CC case uses a constant TKE, a constant TLS in fRPM domain, the so-called patch, and a constant mean flow in the CAA and fRPM domains.  Figure~\ref{fig:CCCinCAA} shows that the rotor wake structures are neglected and averages are used.  The C-CC-tls case also uses constant turbulent and flow variables but the TLS was determined using a different technique, which will be discussed in detail in Section~\ref{sec:results}.

\begin{figure}
%/usr/kiely/kissner/simulations/RC2/piano/TestMatrix_Final/output_P_PP_test/lay_VidMeanFlow_Vorticity.lay
		\subfigure[P-PP test case\label{fig:PPPinCAA}]{
			\includegraphics[width=0.50\textwidth,trim=4 4 4 4,clip]{figures/eps_PPP_MeanFlowVort.pdf}}
%/usr/kiely/kissner/simulations/RC2/piano/TestMatrix_Final/output_C_CC_test/lay_VidMeanFlow_Vorticity.lay		
		\subfigure[C-CC test case\label{fig:CCCinCAA}]{
			\includegraphics[width=0.50\textwidth,trim=4 4 4 4,clip]{figures/eps_CCC_MeanFlowVort.pdf}}
% 	\vspace{-1em}
	\caption{CAA domain of P-PP and C-CC test cases.  Contours show background axial velocity and turbulence vorticity magnitude.\label{fig:PPPCCCinCAA}}
\end{figure}

% \todo[inline]{Attila: Ich finde es komisch in der Vergangenheitsform von noch nicht gezeigten Ergebnissen zu sprechen. ist das nicht irgendwie vorgegriffen? \\ CK: Mag sich komisch anfühlen, ist aber richtig.  Ich habe versucht folgende Regeln konsequent anzuwenden (kann aber noch Schnitzer geben): Past tense - describing completed work; present tense: statements of general truth, ongoing stuff, in descriptions of pictures and tables and so on}
The three secondary test cases were done in order to substantiate the findings of the four primary test cases.  The C-CC-55 simulation was done using a full cascade with 55 vanes.  Furthermore, two test cases were simulated with modified patches and constant cyclostationary characteristics.  The seeding area of the patches remained unchanged.  Details regarding the fRPM patch generation can be found in subsection~\ref{subsec:SetupFRPMCAA}.
For case C-CC-red, the initial patch was modified by significantly reducing the safety margins in the lateral direction.  In these safety margins, the turbulent kinetic energy was set to zero.  This test case confirmed that the Young-Van-Vliet filter does not require safety margins. For case C-CC-double, the seeding area was increased to span two pitches instead of one. This was performed in order to confirm that the acoustic radiation of all blades can be assumed to be uncorrelated.



 %This simulation was compared to the primary C-CC case to investigate the consequences of only using a portion of the full annular duct for the primary test matrix. 
% \unsure[inline]{Ich finde das wir das nicht noch einmal zeigen müssen, das ist schon oft gezeigt worden: A different random seeding to generate the white noise field was also used on this patch.  This test case was done in order to prove that the results are the same regardless of the used random variable fields if the calculation times are appropriately long.  } 
%  This patch also used a reduced filter damping zone as the patch would not fit into the CAA domain otherwise.  This was performed in order to confirm that the acoustic radiation of all blades are uncorrelated, i.e. that exciting one blade and multiplying the acoustic power level by the number of vanes does result in a correct overall acoustic power level.  


%\multirow{4}{*}{\parbox[c]{2mm}{\rotatebox[origin=c]{90}{full cascade}}}

\subsection{Setup of CFD computation}

TRACE was used to perform the q3D URANS simulation at 50\% of the stator height of the baseline SDT fan. Periodic boundary conditions were used for a reduced duct consisting of two rotor blades and five stator vanes.  The investigated operating condition was approach.  However, the static pressure at the stator outlet needed to be slightly modified in order to avoid a separation of flow.  %That way, the flow variables of the harmonics in the stator domain could surely be attributed to cyclostationarity rather than to an unsteadiness due to flow separation.     

In the stator domain, unsteady solutions of the first 15 harmonics of the rotor blade passing frequency were used to accurately reconstruct the rotor wakes in the absolute frame of reference.  The number of harmonics required mainly depends on the flow gradients in the wake.  The higher the gradients, the more harmonics are required for an accurate representation of the wake.  At mean flow, i.e. at the 0th harmonic, the rotor wakes were averaged via flux averaging at the mixing plane between the moving frame of reference in the rotor block and the absolute frame of reference in the stator block.  When constant turbulence and flow characteristics were considered, the 0th harmonic of the URANS computation rather than a RANS computation was used in order to guarantee consistency.

The Hellsten explicit algebraic Reynolds stress model as implemented by \citet{franke_turbulence_2010} was used to sufficiently reproduce the turbulent characteristics in the leading edge and wake regions. The turbulence in the blade boundary layer was fully resolved by the used mesh.

The spatial discretization was done via a MUSCL (Monotonic Upstream Scheme for Conservation Laws) method of second order accuracy based on Fromm's scheme, while the time discretization was done using an Euler Backward scheme of second order accuracy.  The grid contained nearly 900,000 cells. % and ca.~50 passage revolutions were computed in total to ensure a converged solution.

%897712 grid cells, computation times?, nearly fifty rounds 	

%\begin{equation}
%l_s=0.4\frac{\sqrt{|k|}}{0.09 \omega}
%\end{equation}
	

%/usr/kiely/kissner/simulations/RC2/piano/TestMatrix_Final/lay_setup.lay
\begin{figure}
\centering
\includegraphics[width=\textwidth,trim=4 4 4 4,clip]{figures/eps_CAASetup.pdf}
\caption{CAA setup for five stator vanes showing the sponge zones, the vortex source, and the vortex sink.  The white dots indicate sensor positions and the contour shows the pressure of the background flow, i.e. the mean flow. \label{fig:CAA_Setup} }
\end{figure}

\subsection{fRPM patch generation and setup of CAA computation} \label{subsec:SetupFRPMCAA}
Figure~\ref{fig:CAA_Setup} displays the typical CAA setup for the computation of fan broadband noise with the fRPM method for a reduced duct containing only five stator vanes.  The setup for the full cascade case C-CC-55 was done analogously.  The \textit{vortex source} was generated via a fRPM patch. In the \textit{vortex sink} region, the vorticity is filtered out.  \textit{Sponge zones} were applied at the in- and outlet boundaries of the CAA domain and the white dots indicate the positions of the used \textit{sensors}.  In this subsection, the CAA mesh generation will discussed before examining the italic keywords in detail.

For high-order spatial discretization schemes, a high grid quality of the CAA domain is essential.  The used grid resolution was determined by two factors: acoustics and turbulence.  The CAA mesh was designed to enable the propagation of sound waves up to a frequency $f$ of \SI{20}{kHz} without significant dissipation.  The lowest acoustic grid resolution was chosen to be 10 points per wavelength (PPW), which is approximately twice the theoretical resolution limit of 5.4 PPW for a DRP scheme \cite{de_roeck_overview_2004}. This leads to the following acoustic grid resolution: $dx_f\leq\frac{\lambda}{\text{PPW}}=\SI{1.7e-3}{m}$.  In regions of the CAA domain, where synthesized turbulence is injected into and convected in the domain, a smaller grid resolution is required.  To fully resolve this turbulence, vortical structures have to be resolved with at least 4 points per length scale.  In this case,  the determined grid resolution was given from the smallest Gaussian length scale needed to reconstruct a von \Karman turbulence spectrum through a superposition of ten analytically weighted Gaussian spectra. The analytical weighting function and these best-practice rules are derived by \citet{wohlbrandt_analytical_2016} to realize smooth von \Karman spectra.
The von \Karman turbulence spectrum was determined from experimental data of \citet{podboy_steady_2003} to realize an integral turbulent length scale of $\Lambda\approx\SI{5.1e-3}{m}$ at 50\% of the stator height at the stator LE.  The resulting turbulent grid resolution was  $dx_v\leq\SI{0.51e-3}{m}$.  Furthermore, the boundary layer of the background flow was considered but not fully resolved in the CAA domain. %\todo[inline]{Do we need to show ''previous studies''? \\Attila: I find it interesting for the community. Maybe this should be moved to the results. \\CK: Definitely interesting but hard to fit into this paper.  Let me know if you find a way.  Another thing: don't have pics and spectra for this exact setup.} 
Previous studies have shown that artificial sound is generated at a blunt stator trailing edge (TE) if the boundary layer is neglected.  The vortices that move along the blade surface create sound when interacting with a blunt TE.  This is not problematic for pointed TEs but for blunt TEs as is the case for the baseline SDT stator.  The boundary layer prevents the direct interaction between the stator TE and the vortices as it pushes the vortices further away from the blade surface.  The final CAA grid contained 43,324 grid cells per stator passage.     

%45680 particles for initial patch
%/usr/kiely/kissner/simulations/RC2/piano/TestMatrix_Final/lay_PatchComparison.lay
\begin{figure}
\centering
		\subfigure[Initial patch\label{fig:PatchIni}]{
			\includegraphics[width=0.25\textwidth,trim=0 6 4 4,clip]{figures/eps_Patches_1.pdf}}
		\subfigure[Patch with reduced safety margins\label{fig:PatchRedDamp}]{
			\includegraphics[width=0.25\textwidth,trim=0 6 4 4,clip]{figures/eps_Patches_2.pdf}}
		\subfigure[Patch spanning two pitches\label{fig:Patch2Blades}]{
			\includegraphics[width=0.25\textwidth,trim=0 6 4 4,clip]{figures/eps_Patches_3.pdf}}
% 	\vspace{-1em}
	\caption{Used patches for C-CC configuration.  The contour shows the TKE of the mean flow.\label{fig:Patches}}
\end{figure}

\begin{description}
\item{The \textit{vortex source}} produces the synthesized turbulence using inputs from the (U)RANS computation for TKE, TLS, and mean flow and is coupled into the domain using the LEE-relaxation method.  The LE of the source region was located approximately two chord lengths upstream of the stator LE.  The mesh resolution was kept constant to the CAA mesh resolution and five particles per grid cell were used.  As mentioned in the preceding paragraph, a von \Karman spectrum was chosen to generate the synthetic turbulence as a von \Karman spectrum closely emulates realistic turbulence spectra found in turbomachinery. 

Assuming uncorrelated vanes, the RSI is only simulated for a single vane.  All other vanes ensure the correct acoustic radiation only.  Figure~\ref{fig:PatchIni} shows the initial patch used for the primary test matrix.  The contour shows the TKE levels.  The region of the patch containing TKE spans exactly one pitch in order to interact with exactly one stator vane.  The rest of the patch area does not contribute to the synthesized turbulence but acts as safety margin and accounts for lateral convection.  Since the Young-Van-Vliet filter does theoretically not require safety margin as it works on a ghost layer instead, the safety margin was significantly reduced for configuration C-CC-red [see Figure~\ref{fig:PatchRedDamp}].  For configuration C-CC-double, the patch spanned two pitches and has no safety margins [see Figure~\ref{fig:Patch2Blades}].      	



\item{The \textit{vortex sink}} cancels out vorticity downstream of the stator trailing edge utilizing the LEE-relaxation method by \citet{ewert_linear-_2014}.  The vortices in the CAA domain downstream of the stator TE interact with the stator wakes creating hydrodynamic pressure disturbances.  To get a clean acoustic pressure signal at the sensor position, the vortices were removed from the domain and only acoustic perturbations, which are of interest, remained.  This was done by setting the target vorticity $\bfm\Omega^\text{ref}$ to zero in the whole \textit{vortex sink} region as marked in Fig.~\ref{fig:CAA_Setup} and as alluded to in subsection~\ref{sec:coupling}.  

\item{The \textit{sponge zones}} were needed to avoid reflections at the in- and outlet boundary conditions of the CAA domain.  The sponges were 85 cells deep and a small cell stretching in axial direction - not exceeding a value of 1.1 -  was also introduced.

\item{\textit{Sensors}} were equally spaced up- and downstream of the source region to compute the emitted sound power $P$ of an equivalent 3D duct from the 2D cascade, as derived in the Appendix~\ref{sec:PWL}. As 2D turbulence is generated by the 2D RPM method the correction in Eq.~\ref{eq:2D3DGeschwSpektrum} was used, assuming that the lateral velocity component is the major noise source. The integration surface $S$ resulted from a line of the considered 2D cascades discretized in the y-direction by 275 probes for the full-cascade configuration and 25 probes for the five-vane configuration. As the turbulence only impinged on a single vane, the sound power was multiplied by the number of vanes to get the overall sound power.
\end{description}
The setup for the different configurations remained exactly the same.  Only the number of considered harmonics for the TKE, TLS, and mean flow varied.  Periodic variables were realized by 15 harmonics, while constant variables were realized by the 0th harmonic, i. e. the mean variables.  The computation for the C-CC test case lasted approximately four days on five Intel(R) Xeon(R) CPU E5-2630 v3 CPUs.  The same setup for the P-PP took about eight days to compute.  Realizing cyclostationarity is computationally more expensive as summing up the harmonics at each time step takes more time and more information has to be stored.        
