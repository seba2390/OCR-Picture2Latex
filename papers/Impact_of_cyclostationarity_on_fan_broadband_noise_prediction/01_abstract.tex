One of the dominant noise sources of modern Ultra High Bypass Ratio (UHBR) engines is the interaction of the rotor wakes with the leading edges of the stator vanes in the fan stage.  While the tonal components of this noise generation mechanism are fairly well understood by now, the broadband components are not.  This calls to further the understanding of the broadband noise generation in the fan stage.
This article introduces the cyclostationary stochastic hybrid (CSH) method, which accommodates in-depth studies of the impact of cyclostationary wake characteristics on the broadband noise in the fan stage.  The Random Particle Mesh (RPM) method is used to synthesize a turbulence field in the stator domain using a URANS simulation characterized by time-periodic turbulence and mean flow. The rotor-stator interaction noise is predicted by a two-dimensional CAA computation of the stator cascade. The impact of cyclostationarity is decomposed into various effects investigated separately. This leads to the finding that the periodic turbulent kinetic energy (TKE) and periodic flow have only a negligible effect on the radiated sound power. The impact of the periodic integral length scale (TLS) is, however, substantial. The limits of a stationary representation of the TLS are demonstrated making the CSH method indispensable when background and wake TKE are of comparable level. Good agreement of the CSH method with measurements obtained from the 2015 AIAA Fan Broadband Noise Prediction Workshop are also shown. 

%For AIAA: 100 to 200 words (maximum) in one paragraph, no numerical references, acronyms or abbreviations.