\subsection{Stream-surface coordinates}
\label{app:mtheta}
The q3D-grid is transformed into a 2D-grid by means of the stream-surface coordinates $m'$ and $\vartheta$.
With the turbo-machine axis defined in positive $x$-direction and the radius defined as $r = \sqrt{y^2+z^2}$, the transformation into stream surface coordinates is given by:
\begin{align}
m' &= \int \sqrt{\abl x^2 +\abl r^2} & \vartheta &= \tan^{-1}(z/y)
\end{align}
In practice this integral is solved by using $m'_0 = 0$ and 
\begin{align}
m'_i &= m'_{i-1} + \frac{2}{r_i+r_{i-1}}
	\sqrt{(r_i+r_{i-1})^2+(x_i+x_{i-1})^2}
\end{align}
This transformation leads to non-dimensional coordinates. 
% For convenience we use for the 2D grid:
% \begin{align}
% x &= R_\text{LE} m', &&& y &= R_\text{LE} \vartheta,
% \end{align}
% with the radius at the leading edge $R_\text{LE}$. In this way the chord length and pitch are invarant to the transformation.

\subsection{Sound power level}
\label{sec:PWL}
% \begin{multicols}{2}
The sound power $P$ is defined as
\begin{align}
P = \int_S n_i I_i \abl S,
\end{align}
with $S$ the integration surface, $n_i$ the surface normal vector, and $I_i$ the net acoustic intensity defined by \citet{Morfey_Sound_1971} as
% \end{multicols}
\begin{align}
I_i = <pu_i> + \frac{u^0_i }{\rho_0 c_0} <pp> + \frac{u^0_i u^0_j}{c_0^2} <pu_j> + \rho_0 u^0_j <u_i u_j >.
\end{align}
% \begin{multicols}{2}
Instead of using the rms value indicated by $<ab>$, the cross-spectral density function $S_{ab}$ is used
\begin{align}
<ab> \mapsto  S_{ab}.
\end{align}

For the 2D-cascade the integration surface $\abl S$ is replaced by a discrete segment $\Delta S$. This is shown in Fig.~\ref{fig:powerSketch}. In the upper left corner, the cascade and the microphone positions are shown. Along a line in the $y$-direction, which corresponds to the circumference of the duct $r\theta$, the axial intensity is determined. A correction of 2D to 3D turbulence must be considered assuming the main sound source is the transversal velocity component\footnote{An additional correction for the difference of 2D and 3D propagation does not need to be considered for computations of sound power}. For the von \Karman turbulence spectrum it is given by a quotient $Q_{2D\rightarrow3D}$ of Eq.~\ref{eq:E22_Karman_3D} to Eq.~\ref{eq:E22_Karman_2D}. Assuming that the intensity at position $m$ is representative for a circle segment $\Delta S_m$, the sound power in a comparable duct can be determined by 2D axial intensities of a cascade as
\begin{align}\label{eq:P}
P = Q_{2D\rightarrow3D} \sum\limits_{m=1}^{N_m} I_{x,m} \Delta S_m
\end{align}
with
\begin{align}\label{eq:2D3DGeschwSpektrum}
Q_{2D\rightarrow3D} = \frac{E_{22}^{3D}(k_1)}{E_{22}^{2D}(k1)} = \frac{1}{10}\left(3\hat k_1^{-2}+8\right).
\end{align}
For further definitions refer to the next section.

 %Appendix~\ref{app:VelSpectra}.
% 
% \end{multicols}

\begin{figure}
	\begin{minipage}{0.7\textwidth}
		\centering
		 
% \documentclass[border=10pt]{standalone} 
% 
% \usepackage{verbatim}
% \usepackage[utf8x]{inputenc}
% \usepackage[T1]{fontenc}
% 
% \usepackage{tikz}
% \usetikzlibrary{fit}
% \usetikzlibrary{shapes.misc}

% \begin{document}


\tikzset{cross/.style={cross out, draw, 
          minimum size=2*(#1-\pgflinewidth), 
          inner sep=0pt, outer sep=0pt}}
\newcommand{\dist}{3}
\newcommand{\radi}{2}
\newcommand{\coordx}{-5.5}
\newcommand{\coordy}{\dist+0.5}
\newcommand{\coordxD}{0}
\newcommand{\coordyD}{0}

\begin{tikzpicture}[very thick,x=0.8cm,y=0.8cm]
  %contour of caa domain
  \draw (-6,\dist-1) -- (-5,\dist-1);
  \draw[out=0, in=180] (-5,\dist-1) to (-2, \dist);
  \draw (-2,\dist) -- (-1,\dist);
  \draw (-1,\dist) -- (-1,\dist+1.5);

  %blades
  \draw (-4,\dist-0.5) -- (-3,\dist);
  \draw (-4,\dist ) -- (-3,\dist+0.5);
  \draw (-4,\dist+0.5) -- (-3,\dist+1);
  \draw (-4,\dist+1  ) -- (-3,\dist+1.5);

  %microphones
   \draw[dotted,blue] (-2,\dist) -- (-2,\dist+1.5);

  %circle
   \draw[rotate=0 ] (0,0) -- (\radi,0);
   \draw[rotate=15 ] (0.625*\radi,0) node[cross=2pt,blue,thick] {};
   \draw[rotate=30] (0,0) -- (\radi,0);
   \draw[rotate=45] (0.625*\radi,0) node[cross=2pt,blue,thick] {};
   \draw[rotate=60] (0,0) -- (\radi,0);
   \draw[rotate=75] (0.625*\radi,0) node[cross=2pt,blue,thick] {};
   \draw[rotate=90] (0,0) -- (\radi,0);
   \draw (0,0) circle (\radi);

	%connect
	\draw[->, out=-90, in = 135] (-2,\dist)  to  node[above right]{$I_x$} (0.2588190451*0.625*\radi,0.96*0.625*\radi);

    \coordinate[<-, label={right:$\Delta S$}] (S) at (0.6*\radi,\radi+0.5);
	\draw[->,out=-180,in=45] (S) to (0.2588190451*3,0.96*\radi);

%coordinate system cascade
   \draw [->,rotate around={0:(\coordx,\coordy)}](\coordx,\coordy) -- (\coordx+0.5,\coordy)node[anchor=west] {$x$};
   \draw [->,rotate around={90:(\coordx,\coordy)}](\coordx,\coordy) -- (\coordx+0.5,\coordy)node[anchor=west] {$y$};
%coordinate system duct
%    \draw [<-,rotate around={225:(\coordxD,\coordyD)}](\coordxD,\coordyD) -- (\coordxD+0.5,\coordyD)node[anchor=west] {$x$};
   \draw (\coordxD,\coordyD) node[cross=4pt,thick] {};
   \draw (\coordxD,\coordyD) node[anchor=north] {$x$};
   \draw [->,rotate around={90:(\coordxD,\coordyD)}](\coordxD,\coordyD) -- (\coordxD+1,\coordyD)node[anchor=east] {$y$};  
   \draw [->,rotate around={0:(\coordxD,\coordyD)}](\coordxD,\coordyD) -- (\coordxD+1,\coordyD)node[anchor=north] {$z$};  

\end{tikzpicture}
% \end{document}
	\end{minipage}\hfill
	\begin{minipage}{0.3\textwidth}
		\caption{Sketch describing the computation of the sound power from a 2D-cascade simulation. In the top left corner, the 2D-cascade and the microphone positions (blue dots) are shown. By rolling up the cascade of radially constant  acoustic intensities on each microphone segment $\Delta S$, a sound power for the whole duct can be determined.\label{fig:powerSketch}}
	\end{minipage}
\end{figure}


\subsection{Velocity Spectra}\label{app:VelSpectra}
In this investigation, the RPM method generates isotropic turbulence of von Karman shape. For validation, the synthesized spectra are compared to the analytical solution. The velocity one-dimensional wavenumber autospectra in flow direction $\bfm e_1$ and in the direction perpendicular to the flow $\bfm e_2$ are given as:
\begin{align}
E_{11}(k_1) = \frac{u_t^2\Lambda}{\pi} \frac{1}{(1+\hat k_1^2)^{5/6}}  \label{eq:E11_Karman}\\
E_{22}(k_1) = \frac{u_t^2\Lambda}{2 \pi} \frac{1+\frac{8}{3} \hat k_1^2}{(1+\hat k_1^2)^{11/6}} \label{eq:E22_Karman_3D}
\end{align}
with the convective wavenumber $k_1 = \omega/u_0$, the integral length scale (TLS) $\Lambda$, the turbulence velocity variance $u_t$ and the reduced wavenumber $\hat k = k /k_e$ with $k_e = \frac{\sqrt{\pi}\Gamma(5/6)}{\Lambda\Gamma(1/3)}$. For 2D turbulence the lateral velocity one-dimensional wavenumber autospectrum differs and is given as:
\begin{align}\label{eq:E22_Karman_2D}
E_{22}^{2D}(k_1) &= \frac{5 u_t^2\Lambda}{3\pi}  \frac{\hat k_1^2}{\left(1+\hat k_1^2\right)^{11/6}}.
\end{align}
Using the Taylor hypothesis all these can be transformed into frequency space by
\begin{equation}\label{eq:Sii}
S_{ii}(f)=2 E_{ii}(k_1)\frac{2\pi}{u_0}.
\end{equation}
% \end{multicols}


\subsection{Cut-on frequencies}\label{app:cuton}
Often fan spectra calculated with the (linear) wave equations exhibit peaks at frequencies where a new acoustic mode of azimuthal order $m$ and $n$  suddenly becomes cut-on. This is particularly true in the low frequency range where the number of cut-on modes is small. The cut-on frequency in a two dimensional annular duct (infinitely thin) depends on the azimuthal mode order only, the geometry and the flow speed. For flows including swirl, there is a difference for positive and negative azimuthal mode orders $m$. The cut-on frequencies are given by
\begin{align}
f_c(m<0) &= \
%frac{|m|c_0}{2\pi R} \sqrt{1-\left|M_x^2+M_y^2\right|} &= 
\frac{|m| c_0}{N_V s_V} \left(M_y - \sqrt{1-M_x^2}\right), \\
f_c(m>0) &= 
%&=   
\frac{|m| c_0}{N_V s_V} \left(M_y + \sqrt{1-M_x^2}\right)
\end{align}
with $c_0$ the speed of sound and $M_x$ and $M_y$  the axial and circumferential flow Mach number components, respectively.
