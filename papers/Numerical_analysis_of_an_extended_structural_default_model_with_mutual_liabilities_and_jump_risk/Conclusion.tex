\section{Conclusion}
In this paper we considered a structural default model of interlinkage in the banking system. In particular, we studied a simplified setting of two banks numerically. This paper contains several new results. First, we developed a finite-difference method,
an extension of the Hundsdorfer-Verwer scheme, for the resulting partial integro-differential equation (PIDE), studied its stability and consistency. To deal with the integral component, we used the idea of its iterative computation from \cite{LiptonSepp}. The method gives second order convergence in both time and space variables and is unconditionally stable.

Second, by applying the finite-difference method, we computed various model characteristics, such as joint and marginal survival probabilities, CDS and FTD spreads, as well as CVA and DVA, and estimated the impact of jumps on the results. For a more sophisticated analysis, we calibrated the model to the market, and demonstrated a sizeable impact of jumps on joint and marginal survival probabilities in the case of two banks.
The development of numerical methods which are feasible for larger systems of banks appears to be an important future research direction.

From a numerical analysis perspective, we have extended the stability analysis of \cite{intHoutStability} to include an integral term arising from a jump-diffusion process with one-sided exponential jump size distribution.
By Fourier analysis, we were able to show that the scheme is stable in the $l_2$-sense when considering probability densities on an infinite domain. An interesting open question is the stability analysis in the presence of absorbing boundary conditions, such that the individual matrices involved in the splitting do not commute and the eigenvectors and eigenvalues of the combined operator cannot directly be computed.
We are planning to address this in future research.