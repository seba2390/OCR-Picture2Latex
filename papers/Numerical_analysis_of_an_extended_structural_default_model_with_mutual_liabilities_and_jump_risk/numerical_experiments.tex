\section{Numerical experiments}
\label{numerical_experiments}
In this section, we analyze the model characteristics and the impact of jumps. Specifically, we compute joint and marginal survival probabilities, CDS and FTD spreads as well as CVA and DVA depending on initial asset values. We also compute the difference between the solution with and without jumps.

Consider the parameters in Table \ref{table:params}.
\begin{table}[H]
	\begin{center}
		\begin{tabular}{| c | c | c | c | c | c | c | c | c | c | c | c |}
			\hline
			$L_{1,0}$ & $L_{2, 0}$ & $L_{12, 0}$ & $L_{21, 0}$ & $R_1$ & $R_2$ & $T$ & $\sigma_1$ & $\sigma_2$ & $\rho$ & $\varsigma_1$ & $\varsigma_2$ \\ 
			\hline
			60 & 70 & 10 & 15 & 0.4 & 0.45 & 1 & 1 & 1  & 0.5 & 1 & 1 \\
			\hline
		\end{tabular}
	\caption{Model parameters.\label{table:params}}		
	\end{center}
\end{table}
For the model with jumps, we further consider the parameters in Table \ref{table:jumps}.
\begin{table}[H]
	\begin{center}
		\begin{tabular}{| c | c | c | }
			\hline
			 $\lambda_1$& $\lambda_2$ & $\lambda_{12}$ \\
			\hline
			0.5 & 0.5 & 0.3 \\
			\hline
		\end{tabular}
		\caption{Jump intensities.\label{table:jumps}}
	\end{center}
\end{table}

We compute all tests using a $100\times100$ spatial grid with the maximum values $X_1^{100} = X_2^{100} = 10$ and constant time step $\Delta \tau = 0.01$. As the parameters of the HV scheme, we choose $\sigma = \frac{1}{2}$ and $\theta = \frac{3}{4}$. 

In Figures \ref{jointSurvProb1}--\ref{CVA1} we present various model characteristics and compare the results with and without jumps. From these figures, we can observe that jumps can have a significant impact, especially near the default boundaries:

\begin{itemize}
\item
in Figure \ref{jointSurvProb1} for the joint survival probability,
the biggest impact of jumps is around the default boundaries for both $x_1$ and $x_2$;
\item
in Figure \ref{marginalSurvProb1} for the marginal survival probability of the first bank,
we can observe that the biggest impact of jumps is near the default boundary of the first bank;
\item
for the CDS spread, in Figure \ref{CDSPrice1}, (b), the biggest impact of jumps is also seen near the default boundary, but it has the opposite direction, because jumps can only increase the CDS spread;
\item
in Figure \ref{FTDPrice1}, (d) for FTD the spread, the biggest impact of jumps is near both default boundaries, and it has a positive impact;
\item
finally, for CVA, (f), the highest impact of jumps is near the default boundary of the first bank, see Figure \ref{CVA1}.
\end{itemize}

\begin{figure}%[H]
	\begin{center}
		\subfloat[]{\includegraphics[width=0.5\textwidth]{joint_jumps.png}}
		\subfloat[]{\includegraphics[width=0.5\textwidth]{joint_diff_jumps.png}}\\
	\end{center}
	\vspace{-20pt}
	\caption{The joint survival probability: (a) value, (b) difference between model with and without jumps.}
 	\label{jointSurvProb1}
\end{figure}

 \begin{figure}%[H]
	\begin{center}
		\subfloat[]{\includegraphics[width=0.5\textwidth]{marginal_jumps.png}}
		\subfloat[]{\includegraphics[width=0.5\textwidth]{marginal_diff_jumps.png}}\\
	\end{center}
	\vspace{-20pt}
	\caption{The marginal survival probability: (a) value, (b)  difference between model with and without jumps.}
 	\label{marginalSurvProb1}
\end{figure}


 \begin{figure}%[H]
	\begin{center}
		\subfloat[]{\includegraphics[width=0.49\textwidth]{cds_jumps.png}}
		\subfloat[]{\includegraphics[width=0.49\textwidth]{cds_diff_jumps.png}}\\
		\subfloat[]{\includegraphics[width=0.49\textwidth]{ftd_jumps.png}}
		\subfloat[]{\includegraphics[width=0.49\textwidth]{ftd_diff_jumps.png}}\\
		\subfloat[]{\includegraphics[width=0.49\textwidth]{cva_jumps.png}}
		\subfloat[]{\includegraphics[width=0.49\textwidth]{cva_diff_jumps.png}}
	\end{center}
	\vspace{-20pt}
	\caption{
	Values of different credit products with left the value and right the difference between model with and without jumps.
	Top row: Credit Default Swap spread, written on the first bank.
	Middle row: First-to-Default spread.
	Bottom row: CVA of CDS, where the first bank is Reference name (RN) and the second bank is Protection Seller (PS).
	}
 	\label{CDSPrice1}
	\label{FTDPrice1}
	\label{CVA1}
\end{figure}




% \begin{figure}[H]
%	\begin{center}
%		\subfloat[]{\includegraphics[width=0.55\textwidth]{cds_jumps.png}}
%		\subfloat[]{\includegraphics[width=0.55\textwidth]{cds_diff_jumps.png}}\\
%	\end{center}
%	\vspace{-20pt}
%	\caption{Credit Default Swap spread, written on the first bank: (a) value, (b)  difference between model with and without jumps.}
% 	\label{CDSPrice1}
%\end{figure}
%
% \begin{figure}[H]
%	\begin{center}
%		\subfloat[]{\includegraphics[width=0.55\textwidth]{ftd_jumps.png}}
%		\subfloat[]{\includegraphics[width=0.55\textwidth]{ftd_diff_jumps.png}}\\
%	\end{center}
%	\vspace{-20pt}
%	\caption{First-to-Default spread: (a) value, (b)  difference between model with and without jumps.}
% 	\label{FTDPrice1}
%\end{figure}
%
% \begin{figure}[H]
%	\begin{center}
%		\subfloat[]{\includegraphics[width=0.55\textwidth]{cva_jumps.png}}
%		\subfloat[]{\includegraphics[width=0.55\textwidth]{cva_diff_jumps.png}}\\
%	\end{center}
%	\vspace{-20pt}
%	\caption{CVA of CDS, where the first bank is Reference name (RN) and the second bank is Protection Seller (PS) : (a) value, (b)  difference between model with and without jumps.}
% 	\label{CVA1}
%\end{figure}

%From the figures above, we can see that jumps have the most impact near the default boundaries, especially with large intensities. This might have a significant impact in model characteristics.

