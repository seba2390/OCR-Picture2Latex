\section{Model}

We consider %a special case of 
the model in \cite{Lipton2015} %for interconnected banking network 
for two banks. Assume that the banks have external assets and liabilities, $A_i$ and $L_i$ respectively, for $i = 1, 2$, and interbank mutual liabilities $L_{12}$ and $L_{21}$, where $L_{ij}$ is the amount the $i$-th bank owes to the $j$-th bank. Then, the total assets and liabilities for banks 1 and 2 are
\begin{equation}
	\begin{aligned}
		& \tilde{A}_1 = A_1 + L_{21}, \quad \tilde{L}_1 = L_1 + L_{12}, \\
		& \tilde{A}_2 = A_2 + L_{12}, \quad \tilde{L}_2 = L_2 + L_{21}.
	\end{aligned}
\end{equation}

\subsection{Dynamics of assets and liabilities}

As in \cite{Lipton2015}, we assume that the firms' asset values before default are governed by
\begin{equation}
	\label{assets_dynamics}
	\frac{d A_i}{A_i} = (\mu - \kappa_i \lambda_i(t)) \, d t + \sigma_i\, d W_i(t) + (e^{J_i} - 1) \, d N_i(t), \quad i = 1, 2,
\end{equation}
where $\mu$ is the deterministic growth rate, and, for $i=1,2$, $\sigma_i$ are the corresponding volatilities, $W_i$ are correlated standard Brownian motions,
\begin{equation}
	d W_1(t) d W_2(t) = \rho \, d t, 
\end{equation}
with correlation $\rho$, $N_i$ are Poisson processes independent of $W_i$, $\lambda_i$ are the intensities of jump arrivals, $J_i$ are random negative exponentially distributed jump sizes with probability density function
\begin{equation}
	\tilde{\omega}_i(s) = \left\{
	\begin{aligned}
		& 0, & s > 0, \\
		& \vartheta_i e^{\vartheta_i s}, & s \le 0,
	\end{aligned}
	\right.
\end{equation} 
with parameters $\vartheta_i > 0$, and $\kappa_i$ are jump compensators
\begin{equation}
	\kappa_i = \mathbb{E} [e^{J_i} - 1] = -\frac{1}{\varsigma_i + 1}.
\end{equation}
The jump processes are correlated in the spirit of \cite{MarshallOlkin}. Consider independent Poisson processes $N_{\{1\}}(t), N_{\{2\}}(t)$ and $N_{\{12\}}(t)$, with the corresponding intensities $\lambda_{\{1\}}, \lambda_{\{2\}}$ and $\lambda_{\{12\}}$. Then, we define the processes $N_1(t)$ and $N_2(t)$ as
\begin{equation}
	\begin{aligned}
		N_i(t) &= N_{\{i\}}(t) + N_{\{12\}}(t), \qquad i=1,2,\\
		\lambda_i &= \lambda_{\{i\}} + \lambda_{\{12\}},
	\end{aligned}
\end{equation} 
%\begin{equation}
%	\begin{aligned}
%		& N_1(t) = N_{\{1\}}(t) + N_{\{12\}}(t), \\
%		& N_2(t) = N_{\{2\}}(t) + N_{\{12\}}(t), \\	
%	\end{aligned}
%\end{equation}
%with
%\begin{equation}
%	\begin{aligned}
%		& \lambda_1 = \lambda_{\{1\}} + \lambda_{\{12\}}, \\
%		& \lambda_2 = \lambda_{\{2\}} + \lambda_{\{12\}}. \\	
%	\end{aligned}
%\end{equation}
%Using this jump definition, we assume that 
i.e., there are both systemic and idiosyncratic sources of jumps.

We assume that the liabilities are deterministic and have the following dynamics
\begin{equation}
	\frac{d L_i}{L_i} = \mu \, d t, \qquad \frac{d L_{ij}}{L_{ij}} = \mu \, d t,
\end{equation}
where $\mu$ is the same growth rate as defined in \eqref{assets_dynamics}. For pricing purposes, under the risk-neutral measure, we consider $\mu$ as a risk-free short rate. In the following, we take for simplicity $\mu = 0$, but the analysis would not change significantly for $\mu\neq 0$.
\subsection{Default boundaries}

Following \cite{Lipton2015}, we introduce time-dependent default boundaries $\Lambda_i(t)$. 
Bank $i$ is assumed defaulted if its asset value process crosses its default boundary, such that the default time for bank $i$ is
%Once, we know the default boundaries, we can introduce the default time for each bank
\begin{equation}
	\tau_i = \inf\{t |\, A_i(t) \le \Lambda_i\}, \quad i = 1, 2,
\end{equation}
and we define $\tau = \min(\tau_1, \tau_2)$.



Before any of the banks $i=1,2$ has defaulted, $t<\tau$, 
\begin{equation}
	\Lambda_i = \left\{
	\begin{aligned}
	& R_i (L_i + L_{i \bar{i}}) - L_{\bar{i} i} \equiv \Lambda_i^{<},  \quad & t < T, \\
	&  L_i + L_{i \bar{i}} - L_{\bar{i} i}  \equiv \Lambda_i^{=},  & t = T,
	\end{aligned}
	\right.
\end{equation}
where $0 \le R_i \le 1$ is the recovery rate and $\bar{i} = 3 - i$.


If the $k$-th bank defaults at intermediate time $t$, then for the surviving bank $\bar{k} = 3 -k$ the default boundary changes to
$\Lambda_{\bar{k}}(t+) = \tilde{\Lambda}_{\bar{k}}(t)$, where
\begin{equation}
	\tilde{\Lambda}_{\bar{k}} = \left\{
	\begin{aligned}
	& R_{\bar{k}} (L_{\bar{k}} + L_{\bar{k} k} - R_k L_{k \bar{k}})  \equiv \tilde{\Lambda}_k^{<},  \quad & t < T,\\
	& L_{\bar{k}} + L_{\bar{k} k} - R_k L_{k \bar{k}}   \equiv \tilde{\Lambda}_i^{=},  & t = T.
	\end{aligned}
	\right.
\end{equation}
It is clear that for $\Delta \Lambda_k(t) \equiv \Lambda_k(t+)-\Lambda_k(t)$ we have
\begin{equation}
	\Delta \Lambda_k \equiv  \tilde{\Lambda}_k - \Lambda_k = 
	\begin{cases}
		 (1 - R_{\bar{k}} R_k) L_{k \bar{k}},  &t < T, \\
		 (1 - R_k) L_{k \bar{k}},  &t = T.
	\end{cases}
\end{equation}
Thus, $\Delta  \Lambda_k > 0$ and the corresponding default boundaries move to the right.
This mechanism can therefore trigger cascades of defaults.

\subsection{Terminal conditions}
We need to specify the settlement process at time $t = T$. We shall do this in the spirit of \cite{Eisenberg}. Since at time $T$ full settlement is expected, we assume that bank $i$ will pay the fraction $\omega_i$ of its total liabilities to creditors. This implies that if $\omega_i = 1$ the bank pays all liabilities (both external and interbank) and survives. On the other hand, if $0 < \omega_i < 1$, bank $i$ defaults, and pays only a fraction of its liabilities. Thus, we can describe the terminal condition as a system of equations
\begin{equation}
	\label{term_cond}
	\min \left\{A_i(T) + \omega_{\bar{i}} L_{\bar{i}i}, L_i + L_{i \bar{i}}  \right\} = \omega_i \left(L_i + L_{i \bar{i}} \right).
\end{equation}
There is a unique vector $\omega = (\omega_1, \omega_2)^T$ such that the condition (\ref{term_cond}) is satisfied. See \cite{Lipton2015}, \cite{LiptonItkin2015} for details.



\subsection{Formulation of backward Kolmogorov equation}
For convenience, we introduce normalized dimensionless variables
\begin{equation}
	\bar{t} = \Sigma^2 t, \quad X_i = \frac{\Sigma}{\sigma_i} \ln \left(\frac{A_i}{\Lambda_i^{<}}\right), \quad \bar{\lambda}_i = \frac{\lambda_i}{\Sigma^2},
\end{equation}
where
\begin{equation*}
	\Sigma = \sqrt{\sigma_1 \sigma_2}.
\end{equation*}
Denote also
\begin{equation}
	\xi_i = -\left( \frac{\sigma_i}{2 \Sigma} + \kappa_i \bar{\lambda}_i\right), \quad \zeta_i = \frac{\Sigma}{\sigma_i}.
\end{equation}
Applying Itô's formula to $X_i$, we find its dynamics
\begin{equation}
	d X_i = \xi_i \, d \bar{t} + d W_i(\bar{t}) + \zeta_i J_i \, d N_i(\bar{t}).
\end{equation}
In the following, we omit bars for simplicity.

The default boundaries change to 
\begin{equation}
	\mu_i =
	\begin{cases}
		\mu_i^{<} = 0, & t < T, \\
		\mu_i^{=} = \frac{\Sigma}{\sigma_i} \ln \left(\frac{\Lambda_i^{=}(t)}{\Lambda_i^{<}(t)} \right), & t = T.
	\end{cases}
\end{equation}

Assume that the terminal payoff for a contract is $\psi(X_T)$. Then, the value function is given by
\begin{multline}
	V(x, t) = \mathbb{E} \left[\psi(X_T) \cdot \mathbbm{1}_{\{\tau \ge T \}}  + \int_t^T \chi(s, X_s) \cdot \mathbbm{1}_{\{\tau > s \}}  \, d s \, + \right. \\
	\left. + \phi_{1, 0}(\tau_1, X_2(\tau_1)) \cdot \mathbbm{1}_{\{\tau_1 < T \}} +  \phi_{2, 0}(\tau_2, X_1(\tau_2)) \cdot \mathbbm{1}_{\{\tau_2 < T \}} | \, X(t) = x \right],
\end{multline}
where $\chi(\tau, x)$ is the contract payment at an intermediate time $t \le s \le T$ (for example, coupon payment), and $\phi_{1, 0}(t, X_2(t))$ and  $\phi_{2, 0}(t, X_1(t))$ are the payoffs in case of intermediate default of bank 1 or 2, respectively.

Then, according to the Feynman--Kac formula, the corresponding pricing equation is the Kolmogorov backward equation
\begin{align}
		\label{kolm_1}
		\frac{\partial V}{\partial t} + \mathcal{L} V &= \chi(t, x), \\
		V(t, 0, x_2) &= \phi_{2, 0}(t, x_1), \quad V(t, x_1, x_2) \underset{x_1 \to +\infty}{\longrightarrow} \phi_{2, \infty}(t, x_2), \\
		V(t, x_1, 0) &= \phi_{1, 0}(t, x_2), \quad V(t, x_1, x_2) \underset{x_2 \to +\infty}{\longrightarrow} \phi_{1, \infty}(t, x_1), \\
		\label{kolm_2} V(T, x) &= \psi(x),
\end{align}
where Kolmogorov backward operator
\begin{multline}
	\label{kolmogorov_backward}
	\mathcal{L} f = \frac{1}{2}  f_{x_1 x_1} + \rho f_{x_1 x_2} + \frac{1}{2}  f_{x_2 x_2} +  \xi_1 f_{x_1} + \xi_2 f_{x_2} +  \lambda_{1}  \mathcal{J}_1 f + \lambda_{2}  \mathcal{J}_2 f + \lambda_{12}  \mathcal{J}_{12} f - v  f \\
	= \Delta_{\rho} f + \xi \cdot \nabla f + \mathcal{J} f - v f,
\end{multline}
where $v = \lambda_1 + \lambda_2 + \lambda_{12}$ and
\begin{align}
\mathcal{J}_1 f(x) &= \varsigma_1 \int_0^{x_1} f(x_1 - u,  x_2) e^{-\varsigma_1 u} d u, \label{j1_eq}\\
\mathcal{J}_2 f(x) &= \varsigma_2 \int_0^{x_2} f(x_1,  x_2 - u) e^{-\varsigma_2 u} d u, \label{j2_eq}\\
\mathcal{J}_{12} f(x) &=  \mathcal{J}_1 \mathcal{J}_2 f(x) =  \varsigma_1  \varsigma_2  \int_0^{x_1} \int_0^{x_2} f(x_1 - u, x_2 - v) e^{-\varsigma_1 u - \varsigma_2 v} d u d v \label{j12_eq},
\end{align}
$\varsigma_i = \sigma_i \vartheta_i / \Sigma$, and $\phi_{i, 0}, \phi_{i, \infty}$ are given.

In the following, we formulate the Kolmogorov backward equation for specific quantities.

\subsection{Joint and marginal survival probabilities}
\label{section:joint}
The joint survival probability is the probability that both banks do not default by the terminal time $T$ and given by
\begin{equation}
	Q(t, x) = \mathbb{E}[\mathbbm{1}_{\{\tau \ge T, X_1(T) \ge \mu_1^{=}, X_2(T) \ge \mu_2^{=}\}} \, | X(t) = x].
\end{equation}
Then, applying (\ref{kolm_1})--(\ref{kolm_2}) with $\psi(x) =\mathbbm{1}_{\{x_1 \ge \mu_1^{=}, x_2 \ge \mu_2^{=}\}}$ and $\chi(t, x) = 0$, we get
\begin{equation}
\label{joint_surv_prob}
\begin{aligned}
		& \frac{\partial Q}{\partial t} + \mathcal{L} Q = 0, \\
		& Q(t, x_{1}, 0) = 0, \quad Q(t, 0, x_2) = 0, \\
		& Q(T, x) = \mathbbm{1}_{\{x_1 \ge \mu_1^{=}, x_2 \ge \mu_2^{=}\}}.
\end{aligned}
\end{equation}

The marginal survival probability for the first bank is the probability that the first bank does not default by the terminal time $T$,
\begin{equation}
	q_1(t, x) = \mathbb{E}[\mathbbm{1}_{\{\tau \ge T, X_T \in D_1 \cup D_{12})\}} +  \Xi(\tau_2, X_1(\tau_2)) \cdot \mathbbm{1}_{\{\tau_2 < T \}}| \, X(t) = x],
\end{equation}
where $D_{12}$ is the set where both banks survive at the terminal time, $D_1$ is the set where only the first bank survives, and $ \Xi(\tau_2, X_1(\tau_2)) $ is the one-dimensional survival probability with the modified boundaries.

Then, applying (\ref{kolm_1})--(\ref{kolm_2}) with $\psi(x) = \mathbbm{1}_{\{x \in D_1 \cup D_{12})\}}, \chi(t, x) = 0$, we get
\begin{equation}
\begin{aligned}
		& \frac{\partial}{\partial t} q_1(t, x) + \mathcal{L} q_1(t, x) = 0, \\
		& q_1(t, 0, x_2) = 0, \quad
		q_1(t, x_1, 0) = \Xi(t, x_1) = 
		\begin{cases}
			\chi_{1,0}(t, x_1), & x_1 \ge \tilde{\mu}_1^{<}, \\
			0, & x_1 < \tilde{\mu}_1^{<},
		\end{cases} \\
		& q_1(t, \infty, x_2) = 1, \quad
		q_1(t, x_1, \infty) = 
			\chi_{1,\infty}(t, x_1), \\
		& q_1(T, x) = \mathbbm{1}_{\{x \in D_1 \cup D_{12} \}}.
\end{aligned}
\end{equation}
The function $\chi_{1, 0}(t, x_1)$ is the 1D survival probability, which solves the following boundary value problem
\begin{equation}
	\begin{aligned}
		& \frac{\partial}{\partial t} \chi_{1, 0}(t, x_1) + \mathcal{L}_1 \chi_{1, 0}(t, x_1)= 0, \\
		& \chi_{1, 0}(t, \tilde{\mu}_1^{<}) = 0, \quad \chi_{1, 0}(t, \infty) = 1, \\
		& \chi_{1, 0}(T, x_1) = \mathbbm{1}_{\{x_1 > \tilde{\mu}_1^{=}\}},
	\end{aligned}
\end{equation}
where
\begin{equation*}
	\mathcal{L}_1 f = \frac{1}{2} \frac{\partial^2}{\partial x_1^2} f + \xi_1 \frac{\partial}{\partial x_1} f + \lambda_1 \mathcal{J}_1 f - \lambda_1 f.
\end{equation*}

Accordingly, $\chi_{1, \infty}(t, x_1)$ is the 1D survival probability that solves the following boundary value problem
\begin{equation}
	\begin{aligned}
		& \frac{\partial}{\partial t} \chi_{1, \infty}(t, x_1) + \mathcal{L}_1 \chi_{1, \infty}(t, x_1)= 0, \\
		& \chi_{1, \infty}(t,  0) = 0, \quad \chi_{1, \infty}(t, \infty) = 1, \\
		& \chi_{1, \infty}(T, x_1) = \mathbbm{1}_{\{x_1 > \mu_1^{=}\}}.
	\end{aligned}
\end{equation}
We formulate the pricing problems for CDS, FTD, CVA and DVA in Appendix A.
