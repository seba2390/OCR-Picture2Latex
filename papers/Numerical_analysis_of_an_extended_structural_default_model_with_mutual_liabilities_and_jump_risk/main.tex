\documentclass[a4paper,11pt]{article}

\usepackage{cmap}     
\usepackage{amsmath,amssymb}
\usepackage[utf8]{inputenc}
\usepackage[T2A]{fontenc}  
\usepackage{graphicx}    
\usepackage[margin=1in]{geometry}
\usepackage{fancyhdr}    
\usepackage{setspace}
\usepackage{wrapfig}
\usepackage{subfig}
\usepackage{listings}
\usepackage{color}
\usepackage{setspace}
\usepackage{textcomp}
\usepackage{float}
\usepackage{listings}
\usepackage{amsthm}
\usepackage{algorithm}
\usepackage{algpseudocode}
\usepackage{afterpage}
\usepackage[toc,page]{appendix}
\usepackage{natbib}
\usepackage{bbm}
\usepackage{footnote}

\makesavenoteenv{tabular}

\newtheorem{theorem}{Theorem}
\newtheorem{lemma}{Lemma}
\newtheorem{corollary}{Corollary}
\newtheorem{definition}{Definition}

%\doublespacing

\DeclareMathOperator{\cov}{cov}
\DeclareMathOperator{\diag}{diag}
\DeclareMathOperator{\fft}{fft}
\DeclareMathOperator{\ifft}{ifft}
\DeclareMathOperator{\Area}{Area}
\DeclareMathOperator{\inter}{int}
\DeclareMathOperator{\rank}{rank}
\DeclareMathOperator{\vect}{vec}
\DeclareMathOperator{\corr}{corr}

\begin{document}    
\title{Numerical analysis of an extended structural default model with mutual liabilities and jump risk} 
\author{Vadim Kaushansky\thanks{The first author gratefully acknowledges support from the Economic and Social Research Council and Bank of America Merrill Lynch}
\footnote{Mathematical Institute \& Oxford-Man Institute, University of Oxford, UK, E-mail: vadim.kaushansky@maths.ox.ac.uk},  Alexander Lipton\footnote{Massachusetts Institute of Technology, Connection Science, Cambridge, MA, USA, E-mail: alexlipt@mit.edu}, Christoph Reisinger\footnote{Mathematical Institute  \& Oxford-Man Institute, University of Oxford, UK, E-mail: christoph.reisinger@maths.ox.ac.uk}}    
\date{}
   
\maketitle 
\begin{abstract}
We consider a structural default model in an interconnected banking network as in \cite{Lipton2015}, with mutual obligations between each pair of banks. We analyse the model numerically for two banks with jumps in their asset value processes. Specifically, we develop a finite difference method for the resulting two-dimensional partial integro-differential equation, and study its stability and consistency. We then compute joint and marginal survival probabilities, as well as prices of credit default swaps (CDS), first-to-default swaps (FTD), credit and debt value adjustments (CVA and DVA). Finally, we calibrate the model to market data and assess the impact of jump risk.
\end{abstract}

\noindent
{\bf Keywords:} structural default model; mutual liabilities; jump-diffusion; finite-difference and splitting methods; calibration. \\
\medskip

%\noindent
%{\bf Highlights:}
%%The novel results of this paper are as follows:
%\begin{itemize}
%\item We analyze a two-dimensional structural default model with interbank liabilities and negative exponential jumps; in particular, we calibrate the model to the market and analyze the impact of jumps on joint and marginal survival probabilities; 
%\item we develop a new finite-difference method to solve the multidimensional PIDE, which is of second order consistent in both time and space variables; 
%\item we prove the von Neumann and $l_2$ stability of the method; % extending \cite{intHoutStability} from PDEs to PIDEs;
%%to our knowledge, this is the first result on stability of a splitting scheme for this type of multi-dimensional PIDE taking into account Dirichlet boundary conditions;
%\item we demonstrate empirically that in the presence of discontinuous terminal and boundary conditions, second order of convergence can be maintained by local averaging of the data and suitable refinement of the timestep close to maturity.
%\end{itemize}


\section{Introduction}
\label{sec:Introduction}


The goal in top-$\size$ recommendation is to recommend to each
consumer a small set of $\size$ items from a large collection of
items~\cite{cremonesi2010performance}.  For example, Netflix may want
to recommend $\size$ appealing movies to each consumer.  Collaborative
Filtering (CF)~\cite{herlocker2002empirical,lee2012comparative} is a
common top-$\size$ recommendation method.  CF infers user interests by
analyzing partially observed user-item interaction data, such as user
ratings on movies or historical purchase
logs~\cite{kanagal2012supercharging}. The main assumption in CF is that
users with similar interaction patterns have similar interests.


Standard CF methods for top-$\size$ recommendation focus on making  suggestions  that accurately reflect the user's preference history. However, as  observed in previous work,  CF recommendations are generally biased toward  popular items, leading to a rich get richer effect~\cite{vargas2014improving,steck2011item}.  The major reasons for this are \textit{popularity bias} and \textit{sparsity} of CF interaction data (detailed in Section~\ref{sec:related-work}). In a nutshell, to maintain  accuracy, recommendations are generated from the dense regions of the data,  where the popular items lie.  

However,  accurately suggesting popular items, may not be satisfactory for the consumers. For example, in Netflix, an accuracy-focused movie recommender may recommend ``Star Wars: The Force Awakens'' to users who have seen ``Star Wars: Rogue One''.  But, those users are probably already aware of ``The Force Awakens''. Considering additional factors, such as novelty of recommendations,  can lead to more effective suggestions~\cite{cremonesi2010performance,Castells2015,zhang2008avoiding,ziegler2005improving,zhang2012auralist}. 
%Second, accuracy-focused models typically achieve a   overall item-space coverage across their recommendations,  whereas high item-space coverage helps providers of the items increase revenue
%, users satisfaction since they are  likely already aware of or can find these items on their own.  

Focusing on popular items also adversely affects the satisfaction of  the providers of the items. This is because  accuracy-focused models typically achieve a  low overall item space coverage across their recommendations, whereas   high item space coverage helps providers of the items increase their revenue~\cite{vargas2014improving,Castells2015,adomavicius2011maximizing,anderson2006thelongtail, yin2012challenging,adomavicius2012improving}.
%accuracy-focused models typically achieve a

In contrast to the relatively small number of popular items, there are copious  {\it long-tail\/} items that have fewer observations (e.g., ratings) available. More precisely,  using the Pareto  principle (i.e.,~the $80/20$ rule),  long-tail items can be defined as items that generate the lower $20\%$ of observations~\cite{yin2012challenging}. Experimentally we found that these items correspond to almost $85\%$ of the items in several datasets (Sections~\ref{sec:Notation} and \ref{sec:Experiments}). %Table~\ref{tab:DatasetStatsticsSmall})


As previously shown, one way to improve the novelty of top-$\size$ sets is to recommend interesting long-tail items~\cite{cremonesi2010performance,ge2010beyond}.  The intuition  is that since they have fewer observations available,  they are more likely to be unseen~\cite{Kaminskas:2016:DSN:3028254.2926720}.  
 %For example, in online commerce,  newly added items are long-tail items that are yet to be discovered.  
Moreover, long-tail item promotion also results in higher overall coverage of the item space%, which increases profits for providers of the items
~\cite{vargas2014improving,Castells2015,zhang2008avoiding,zhang2012auralist,adomavicius2011maximizing,anderson2006thelongtail,yin2012challenging,jambor2010optimizing}. Because long-tail promotion reduces accuracy~\cite{steck2011item}, there are trade-offs to be explored.


%original submitted to ICDE
%This work studies three aspects of top-$\size$ recommendation: accuracy, novelty, and item-space coverage, and examines their trade-offs. In most previous work, predictions of a base recommendation system are re-ranked to handle their trade-offs~\cite{adomavicius2012improving,jambor2010optimizing,zhang2013personalize,wang2009portfolio}. Due to performance considerations, however, these techniques are not customized per user. For example,  parameters that balance the trade-off between novelty and accuracy are cross-validated at a global level.  This can be detrimental since users have varying preferences for  objectives such as long-tail novelty. We explore how to  automatically infer  user  preference for long-tail novelty, and how to leverage  it to correct  the popularity bias in standard recommender models. Our work does not rely on any additional contextual data, although such data, if available, can help promote newly-added long-tail items~\cite{agarwal2009regression,Saveski:2014:ICR:2645710.2645751}.

This work studies three aspects of top-$\size$ recommendation: accuracy, novelty, and item space coverage, and examines their trade-offs. In most previous work, predictions of a base recommendation algorithm are \textit{re-ranked} to handle these trade-offs~\cite{adomavicius2012improving,jambor2010optimizing,zhang2013personalize,wang2009portfolio}. The re-ranking models are computationally efficient but suffer from two drawbacks. First, due to performance considerations,  parameters that balance the trade-off between novelty and accuracy  are not customized per user. Instead they are cross-validated at a global level.  This can be detrimental since users have varying preferences for  objectives such as long-tail novelty. Second,  the re-ranking methods are often limited to a specific base recommender  that may be sensitive to dataset density. 
As a result, the datasets are pruned and the problem is studied in dense settings~\cite{adomavicius2012improving,ho2014likes}; but real world  scenarios are often sparse~\cite{kanagal2012supercharging,liu2017experimental}.   
% Because  dataset density can impact the performance of most base recommenders (like R-SVD), which in turn affects the performance of the re-ranking model, 

\iffalse
We address these limitations by directly inferring  user  preference for long-tail novelty  from interaction data.  This  allows us to customize the re-ranking  per user, and design a \textit{generic} framework, which resolves the second problem. In particular, since the long-tail novelty preferences are estimated independently of any base  recommender model, we can  plug-in an appropriate base recommender w.r.t. the dataset sparsity.% including ones that are more suitable for sparse settings.  

Modelling  user  preference for  long-tail novelty using only item popularity statistics, e.g., the average popularity of rated items as in~\cite{jugovac2017efficient}, disregards additional information like whether the user found the item interesting and the long-tail preferences of other users  of the items. \iffalse To incorporate them, we introduce the notion of  \emph{item long-tail importance}. Both  user long-tail preferences and item long-tail importance are dependent:  a user has high preference for discovering long-tail items if she is interested in important long-tail items, and an item that is associated with many of these kinds of users is likely to be more important.  We propose a joint optimization framework to directly learn,  from interaction data, both the users' long-tail preferences and the  items' long-tail importance. \fi
We propose an optimization approach that  incorporates  this information and  directly learns,  from interaction data, the users' long-tail novelty preferences.

Next, we use these learned preferences  to design a  top-$\size$ recommendation framework thats is generic, and provides customized balance between accuracy, novelty, and coverage. We refer to it as framework as GANC.  Using GANC, we design a novel algorithm, {\it Ordered Sampling-based Locally Greedy (OSLG)\/}, that relies on the learned long-tail novelty preferences  to scalably correct for popularity bias. Our work does not rely on any additional contextual data, although such data, if available, can help promote newly-added long-tail items~\cite{agarwal2009regression,Saveski:2014:ICR:2645710.2645751}. In summary:
\fi

We address the first limitation by directly inferring  user  preference for long-tail novelty  from interaction data.   Estimating these  preferences  using only item popularity statistics, e.g., the average popularity of rated items as in~\cite{jugovac2017efficient}, disregards additional information, like whether the user found the item interesting or the long-tail preferences of other users  of the items. We propose an approach that  incorporates  this information and  learns the users' long-tail novelty preferences from interaction data.

This approach allows us to customize the re-ranking  per user, and  design a \textit{generic} re-ranking framework, which resolves the second limitation of prior work. In particular, since the long-tail novelty preferences are estimated independently of any base recommender, we can  plug-in an appropriate one w.r.t. different factors, such as the dataset sparsity.

Our top-$\size$ recommendation framework, \textbf{GANC}, is \textbf{G}eneric, and provides customized balance between \textbf{A}ccuracy, \textbf{N}ovelty, and \textbf{C}overage. % Moreover, based on the learned long-tail novelty preferences, we also design a novel algorithm, {\it Ordered Sampling-based Locally Greedy (OSLG)\/}, that relies on the learned long-tail novelty preferences  to scalably correct for popularity bias. 
Our work does not rely on any additional contextual data, although such data, if available, can help promote newly-added long-tail items~\cite{agarwal2009regression,Saveski:2014:ICR:2645710.2645751}. In summary:

%Consider  the following toy example:
\vspace{-0.2cm}
\begin{table}[htb]
\centering
\scriptsize
%\small
\begin{tabular}{ccccccc} 
%\toprule
%&\multirow{2}{*}{}&\multicolumn{7}{c}{Ratings}\\
& & \cellcolor{blue!35}$w_1$ &\cellcolor{blue!18} $w_2$ & $\dots$ &\cellcolor{blue!8} $w_{89}$  &\cellcolor{blue!8} $w_{99}$   
\\
&   &$i_1$&$i_2$&$\dots$&$i_{89}$&$i_{90}$\\ 
\cmidrule(r){3-7} 	 
%\midrule
\cellcolor{red!35}$\theta_1$  &$u_1 $   &5 &   & $\dots$ &  &   \\
\cellcolor{red!28}$\theta_2$  &$u_2$     &5 &    & $\dots$ &  &  \\
 $\theta_3=?$  &$\bf u_3$  &5 &  &   $\dots$ &  &  \\
\cellcolor{red!10}$\theta_4$ & $u_4$  &  &5   & $\dots$ & &\\ 
\cellcolor{red!10}$\theta_5$ & $u_5$  &  & 5  & $\dots$ & &\\ 
$\theta_6=?$  & $\bf u_6$ & &5  &      $\dots$& &  \\ 
 & & $\hdots$  &$\hdots$   &$\hdots$   &$\hdots$   &$\hdots$  \\
%\midrule 
\cmidrule(r){3-7} 	 
\multicolumn{2}{c}{item pop.}  & 3  & 3  & $\dots$ &50&60\\  
%\bottomrule
%$ f_i$    &3  &3  &1  &3  &1  &2  \\  \hline
\end{tabular}
%#.
\caption{Simplified user-item interaction data. The user long-tail novelty preference ($\theta_u$), item long-tail importance weight ($w_i$) are highlighted. Darker colors indicate larger values. } \label{tab:example}
\end{table} 
\vspace{-0.2cm}
\begin{example}  
In Table~\ref{tab:example}, we are interested in estimating $\theta_3$ and $\theta_6$,  the long-tail preference of users $u_3$ and $u_6$ who have each rated a single movie. Additional ratings for other users  are not included here.  Considering only rating information, we observe $i_1$ and $i_2$ are  equally popular $|\mathcal{U}_{i_1}^{\trainset}| = |\mathcal{U}_{i_2}^{\trainset}|=3$, and $r_{31}=5$ and $r_{62}=5$. Using Eq.~\ref{eq:tfidf-risk}  we have $\theta_3 = \theta_6$. However, if we were given the long-tail preferences of the each item's user set, specifically that $u_1$ and $u_2$ have high long-tail preference (darker red), while $u_4$ and $u_5$ have lower long-tail preference (lighter red), we could conclude $i_1$ is a more important long-tail item compared to $i_2$ (indicated by a darker blue shade for $w_1$), and we expect  $\theta_3 \geq \theta_6$.

% On the other hand, if we knew that $u_4$ and $u_5$ have lower long-tail preference, we could conclude $i_2$ is a  less significant long-tail item. Therefore, However, if we  consider the long-tail preferences of other users, we may reason differently.    We need another variable $w_i$ which captures this information. 
%we would conclude that $u_3$ has higher long-tail preference compared to $u_6$, since the users $i_1$ is a more prominent long-tail item. 

% Relying only  on item popularity information, we would  conclude   $u_3$ and $u_6$ have equal long-tail preference, since $i_1$ and $i_2$ are  equally popular. However, considering  the second column,  long-tail preference of users,  long-tail importance for each item,  which captures the long-tail preference of its users. Since  that  both users of $i_1$ have high long-tail preference while  the users of $i_2$ have lower preference,  we may conclude $i_1$ is a more important long-tail item compared to $i_2$. Therefore, $u_3$'s long-tail preference should be at least as large as $u_6$'s preference. Specifically, consider two  items $i_1$ and $i_2$, with the following rating data: $i_1=\{u_1:5, u_2:5, u_3:5 \}$, $i_2=\{u_4:5, u_5:5, u_6:5\}$.  

%Table~\ref{tab:example} shows  simplified rating data. We want an estimate of the long-tail preference of $u_3$ and $u_6$, who have each  rated a single movie.  Relying only  on movie popularity information, we would  conclude   $u_3$ and $u_6$ have similar long-tail preference, since $m_1$ and $m_2$ are  equally popular. However, considering the long-tail preferences of other users of those movies, we may reason differently: since $u_1$ and $u_2$ have high long-tail preference, and $u_4$ and $u_5$ have low long-tail preference, $m_1$ is a more prominent long-tail item compared to $m_2$. Therefore, it is likely that $u_3$ has higher long-tail preference compared to $u_6$.considering the long-tail preferences of other users of those movies, we may reason differently.  For example, 
\label{ex:running}
\end{example}



%------------------------------

\iffalse
\begin{example}
Table~\ref{tab:example} shows rating data for a simplified system. %Note the user-item interaction matrix is sparse.
For this example, we define popular movies as those that have received  three or more ratings; $\{m_1, m_2, m_4\}$ are popular and  $\{m_3, m_5, m_6\}$ are niche movies. We observe $u_1$ and $u_3$  have rated relatively popular movies (risk-averse) while $u_2$ and $u_4$ have rated niche movies (risk-loving). 
\label{ex:running}
\end{example}

\begin{table}[htb]
\centering
\scriptsize
\begin{tabular}{ccccccc} 
\toprule
			&$m_1$ &$m_2$   &$m_3$    &$m_4$   &$m_5$ &$m_6$  \\ \hline 
$u_1 $ &5  &4  & - &-  &-  &-   \\
$u_2$  &-  &-  &-  &-  &5  &5   \\
$u_3$  &-  &4  &-  &5  &-  &-   \\
$u_4$  &-  &-  &3  &-  &-  &4   \\ 
$u_5$  &5  &-  &-  &3  &-  &-   \\ 
$u_6$  &4  &2  &-  &4  &-  &-   \\ 
\bottomrule
%$ f_i$    &3  &3  &1  &3  &1  &2  \\  \hline
\end{tabular}
\caption{User-Movie rating data} \label{tab:example}
\end{table}

It is essential to consider consumer characteristics in designing recommender systems so that they promote long-tail items to the right group of users and spread demand evenly between hit and niche items.  

\fi





%------------------------------
\iffalse
\begin{table}[htb]
\centering
\scriptsize
\begin{tabular}{ccccccc} 
\toprule
			&$m_1$ &$m_2$   &$m_3$    &$m_4$   &$m_5$ &$m_6$  \\ \hline 
$u_1 $ &\textbf{5}  & \textbf{4}  &\textcolor{gray}{ 1.2} &-  &-  &-   \\
$u_2$  &-  &-  &-  &-  & \textbf{5}  &\textbf{5}   \\
$u_3$  &-  &\textbf{4}  &-  &\textbf{5}  &-  &-   \\
$u_4$  &-  &-  &\textbf{3}  &-  &-  &\textbf{4}   \\ 
$u_5$  &\textbf{5}  &-  &-  &\textbf{3}  &-  &-   \\ 
$u_6$  &\textbf{4}  &\textbf{2}  &-  &\textbf{4}  &-  &-   \\ 
\bottomrule
%$ f_i$    &3  &3  &1  &3  &1  &2  \\  \hline
\end{tabular}
\caption{User-Movie rating data} \label{tab:example}
\end{table}
% $\mathcal{P}^1= \{ \mathcal{P}_1^1 \{i_1,i_2,i_3\}, \mathcal{P}_2^1:\{i_2,i_3,i_5\}  \}$
 %$\mathcal{P}^2= \{ \mathcal{P}_1^2: \{i_1,i_2,i_3\}, \mathcal{P}_2^2:\{i_2,i_5,i_6\}  \}$
 %$\mathcal{P}^3= \{ \mathcal{P}_1^3: \{i_7,i_8,i_9\}, \mathcal{P}_2^3:\{i_{10},i_{11},i_{12}\}  \}$
\begin{table}[htb]
\centering
\tiny
\begin{tabular}{ccc} 
\toprule
		&$u_1$&$u_2$  \\ \hline 
$\mathcal{P}^1 $ & $\{i_1,i_2,i_3\}$ & $\{i_2,i_3,i_5\} $ \\
$\mathcal{P}^2$ & $\{i_1,i_2,i_3\}$ & $\{i_2,i_5,i_6\} $ \\
$\mathcal{P}^3$ & $\{i_7,i_8,i_9\}$ & $\{i_{10},i_{11},i_{12} \}$ \\
\bottomrule
%$ f_i$    &3  &3  &1  &3  &1  &2  \\  \hline
\end{tabular}
\caption{Top-$\size$ allocations to users.} \label{tab:paretoExamples}
\end{table}
\fi


\iffalse
When considering long-tail items, it is important to consider consumers' willingness  to explore niche or unpopular items and their propensity towards similar items. In particular, they can be characterized by their  {\it risk degree\/} and {\it focusing degree\/}, respectively.  We compute these estimates  based on historical rating information. The following example further describes these notions in the context of movie rating data. 

\begin{example}  
Table~\ref{tab:example} shows rating data for a simplified system with $6$ users, $6$ movies, and $3$ genres. $m_i^{j}$ implies that movie $m_i$ belongs to genre $j$. Note the user-item interaction matrix is sparse. 
  For this setting, we define popular movies as those that have received  three or more ratings; $\{m_1, m_2, m_4\}$ are popular and  $\{m_3, m_5, m_6\}$ are niche movies. We now profile the users according to their risk and focusing degree. E.g., $u_1$ has rated relatively popular movies belonging to the same genre (risk-averse, high focusing degree); $u_2$ has rated niches movies in the same genre (risk-loving, high focusing degree); $u_3$ has rated popular movies in two different genres (risk-averse, low focusing degree), and $u_4$ has rated niches movies in two different genres (risk-loving, low focusing degree). 
\label{ex:running}
\end{example}
\begin{table}[htb]
\centering
\tiny
\begin{tabular}{ccccccc} 
\toprule
			&$m_1^{1}$ &$m_2^{1}$   &$m_3^{2}$    &$m_4^{3}$   &$m_5^{3}$ &$m_6^{3}$  \\ \hline 
$u_1 $ &5  &4  &-  &-  &-  &-   \\
$u_2$  &-  &-  &-  &-  &5  &5   \\
$u_3$  &-  &4  &-  &5  &-  &-   \\
$u_4$  &-  &-  &3  &-  &-  &4   \\ 
$u_5$  &5  &-  &-  &3  &-  &-   \\ 
$u_6$  &4  &2  &-  &4  &-  &-   \\ 
\bottomrule
%$ f_i$    &3  &3  &1  &3  &1  &2  \\  \hline
\end{tabular}
\caption{User-Movie rating data} \label{tab:example}
\end{table}
It is essential to consider these consumer characteristics in designing recommender systems so that they promote long-tail items to the right group of users and spread demand evenly between the hit and niche items.  
\fi
\iffalse
\begin{center}
\begin{figure*}[tp]
%\scalebox{0.5}{%
\resizebox{1\textwidth}{!}{%
%\small%\addtolength{\tabcolsep}{5pt}% below sums to 8
\begin{tabularx}{1.5\textwidth}{>{\hsize=2.5\hsize}X>{\hsize=2.5\hsize}X>{\hsize=0.5\hsize}X>{\hsize=0.5\hsize}X>{\hsize=0.5\hsize}X>{\hsize=0.5\hsize}X>{\hsize=0.5\hsize}X>{\hsize=0.5\hsize}X}
    \multirow{12}{*}{\includegraphics[scale=0.3]{codeForExample/popularity-movie.png}} & \multirow{12}{*}{\includegraphics[scale=0.3]{codeForExample/scatterplot.png}} & & & & & & \\
%   & &               &       &       &       &       &       \\
    & &\multicolumn{1}{l|}{}               &$m_1^{g1}$   	&$m_2^{g1}$    	&$m_3^{g2}$    &$m_4^{g2}$      &$m_5^{g3}$    \\ \cline{3-8}%\hline
    & &\multicolumn{1}{l|}{u1}          &5  &5  &-  &-   &-  \\
    & &\multicolumn{1}{l|}{u2}    		&-  &-  &4  &4  &5  \\
    & &\multicolumn{1}{l|}{u3}   			&1  &2  &1  &-  &-   \\
    & &\multicolumn{1}{l|}{u4}     		&1  &-  &-  &-  &-  \\
    & &               &       &       &       &       &       \\
    & &               &       &       &       &       &       \\
    & &               &       &       &       &       &       \\
    & &               &       &       &       &       &	\\
    \\
\end{tabularx}}
\caption{User-Movie interaction data a) Popularity-Movie histogram b)Movie genres/clusters c) User-Movie rating data} \label{fig:example}
\end{figure*}
\end{center}
\fi



%We propose a novel approach that allows us to  promote long-tail items in a targeted manner, thereby improving the novelty of top-$\size$ sets, the overall item-space coverage across recommendations, while maintaining reasonable levels of accuracy.

%Next, we integrate these learned preferences  in a generic  top-$\size$ recommendation framework to provide customized balance between accuracy and coverage.

%sequentially make recommendations, while adjusting its parameters with regard to the set of top-$\size$ recommendations made so far. However, since  sequential parameter updates  cause  scalability issues, we propose a sampling based algorithm. This variant of our framework, called {\it Ordered Sampling-based Locally Greedy (OSLG)\/},  allows us to  correct for the popularity bias in recommendations with regard to individual user long-tail preferences. 

%ICDE submission
%Our framework differs with  prior work in the following aspects:  unlike~\cite{adomavicius2011maximizing,adomavicius2012improving,zhang2013personalize,ho2014likes},  the long-tail preference personalization in our framework is learned rather than optimized using cross-validation or parameter tuning. In other words, our personalization method is independent of the underlying base  recommendation models.  Moreover, our framework is  generic. This enables us to  plug-in several base recommenders, and evaluate their  effectiveness without requiring  extensive tuning for the accuracy and coverage trade-off. 


%\vspace{-2.8pt}
\begin{itemize}

\item  We examine various measures for estimating user long-tail novelty preference in Section~\ref{sec:lt-pref} and formulate an optimization problem  to directly learn users' preferences for long-tail  items from interaction data in Section~\ref{sec:learning-lt-pref}. %In addition, we introduce several heuristics for measuring the user preference for less common items from historical rating data.% 

\item  We integrate the user preference estimates into GANC %, a generic re-ranking framework that provides customized balance between accuracy, novelty, and coverage 
(Section~\ref{sec:RiskbasedReranking}), and  introduce {\it Ordered Sampling-based Locally Greedy (OSLG)\/}, a scalable algorithm that relies  on user long-tail preferences to correct the popularity bias (Section~\ref{sec:optimizationAlgorithm}).
%We introduce OSLG, a scalable algorithm that relies  on user long-tail preferences to  maximize item space coverage \textcolor{red}{while maintaining acceptable levels of accuracy} (Section~\ref{sec:optimizationAlgorithm}).

\item   We conduct an extensive empirical study and evaluate performance from  accuracy, novelty, and coverage perspectives (Section~\ref{sec:Experiments}).  We use five  datasets with varying density and difficulty levels. %:  Netflix, MovieTweetings, and MovieLens (100K, 1M, 10M). 
  In contrast to most related work,  our evaluation considers realistic settings that include a large number of infrequent  items and users. %This enables us to study the impact of  data density on the performance trade-offs of several  state of the art top-$\size$ recommendation algorithms. %   %,  and use the all-items ranking protocol~\cite{steck2013evaluation,vargas2014improving}, where performance is measured using all items with train data. to evaluate the performance of several  state of the art top-$\size$ recommendation algorithms 
 
\item Our empirical results confirm that the performance of re-ranking models is impacted by the underlying   base recommender and the dataset density. Our generic approach enables us to easily incorporate a suitable base recommender to devise an effective solution for both dense and sparse settings. In dense settings, we use the same base recommender as existing re-ranking approaches, and we outperform them in accuracy and coverage metrics. For sparse settings, we plug-in a more suitable base recommender, and devise an effective solution that is competitive with existing top-$\size$ recommendation methods in accuracy and novelty. 

%Directly estimating the long-tail novelty preferences allows us to customize re-ranking per user, and  devise a generic framework.   
 
\end{itemize}

Section~\ref{sec:related-work} describes related work. Section~\ref{sec:conclusion} concludes.

\section{The \MakeLowercase{i}W\MakeLowercase{inr}NFL model}
\label{sec:model}

In this section we are going to present the data we used to develop our in-game probability model as well as the design details of {\method}. 

{\bf Data: }In order to perform our analysis we utilize a dataset collected from NFL's Game Center for all the regular season games between the seasons 2009 and 2016. 
We access the data using the Python {\tt nflgame} API \cite{nflgame}. 
The dataset includes detailed play-by-play information for every game that took place during these seasons. 
This information is used to obtain the state of the game that will drive the design of {\method}. 
In total, we collected information for 2,048 regular season games and a total of 338,294 snaps/plays. 

{\bf Model: }
{\method} is based on a logistic regression model that calculates the probability of the home team winning given the current status of the game as: 

\begin{equation}
\Pr(H=1| \mathbf{x})= \frac{\exp(\mathbf{\weight}^T\cdot\mathbf{x})}{1+\exp(\mathbf{\weight}^T\cdot\mathbf{x})}
\label{eq:reg}
\end{equation}
where $H$ is the dependent random variable of our model representing whether the home team wins or not, $\mathbf{x}$ is the vector with the independent variables, while the coefficient vector $\mathbf{\weight}$ includes the weights for each independent variable and is estimated using the corresponding data.  
For a game of infinite duration a linear model could be a very good approximation.  
However, the boundary effects from the finite duration of a game create several non-linearities \cite{winston2012mathletics}.  
For this reason, we enhance our model - using the same set of features - with a Support Vector Machine classifier with radial kernel for the last three minutes of regulation.  
In order to obtain a probability output from the SVM classifier, we further use Platt's scaling \cite{platt1999probabilistic}: 

\begin{equation}
\Pr(H=1| \mathbf{x})= \frac{1}{1+\exp{(Af(x)+B)}}
\label{eq:platt}
\end{equation}
where $f(x)$ is the uncalibrated value produced by the SVM classifier: 

\begin{equation}
f(x) = \sum_{i} (\alpha_i y_i k(\mathbf{x}_i\cdot\mathbf{x}))+ b
\label{eq:svm}
\end{equation}
where $k(\mathbf{x},\mathbf{x}')$ is the kernel used for the SVM.   
Figure \ref{fig:iwinrNFL} depicts the simple flow chart of {\method}. 


\begin{figure}[t]
\begin{center}
\includegraphics[scale=0.35]{plots/iwinrNFL.pdf}%\vspacecap
 \caption{{\method} includes a linear and a non-linear component.}
 \label{fig:iwinrNFL}
\end{center}
\end{figure}

In order to describe the status of the game we use the following variables:

\begin{enumerate}
\item {\bf Ball Possession Team:} This binary feature captures whether the home or the visiting team has the ball possession
\item {\bf Score Differential:} This feature captures the current score differential (home - visiting)
\item {\bf Timeouts Remaining:} This feature is represented by two independent variables - one for the home and one for the away team - and they capture the number of timeouts remaining for each of the teams
%\item {\bf Quarter:} This feature captures the current quarter of the game
%\item {\bf Time Remaining:} This feature captures the time (in seconds) remaining for the current quarter to end
\item {\bf Time Elapsed: } This feature captures the time elapsed since the beginning of the game
\item {\bf Down:} This feature represents the down of the team in possession
\item {\bf Field Position:} This feature captures the distance covered by the team in possession from their own yard line
\item {\bf Yards-to-go:} This variables represents the number of yards needed for a first down
\item {\bf Ball Possession Time: } This variable captures the time that the offensive unit of the home team is on the field 
\item {\bf Ranking Differential: } This variable represents the difference of the win percentage for the two team (home - visiting)
\end{enumerate}

The last independent variable is representative of the power ranking difference between the two teams. 
Most of the existing models that include such a variable are using the Vegas line spread for each game.  
We choose not to do so for the following reason.  
The objective of the Vegas line is not to predict game outcomes but rather distribute money across the different bets.  
Exactly because of this objective the line is changing during the week before the game.  
While this line can change due to new information for the competing teams (e.g., injury updates), the line is mainly changing when a particular team has accumulated the majority of the bets. 
In this case it will also be hard to choose which line to use (e.g., the opening, the closing or some average of them).  
Therefore, we choose to use the win percentage differential of the two teams as an indicator of their strength (even though this has its own issues given the uneven schedule in NFL).  
However, note that if one would like to use the point spread as a variable this can be easily incorporated in the model. 
Table \ref{tab:iwinrnfl} presents the coefficients of the logistic regression model of {\method} with standardized independent variables for better comparisons. 


\begin{table}[ht]
\begin{center}
\def\sym#1{\ifmmode^{#1}\else\(^{#1}\)\fi}
\begin{tabular}{l*{1}{c}}
\toprule
                    &\multicolumn{1}{c}{(1)}\\
                    &\multicolumn{1}{c}{Winner}\\
\midrule
Possession Team (H)         &      0.41\sym{***}\\
                    &     (49.19)         \\
\addlinespace
Score Differential           &      3.59\sym{***}\\
                    &    (247.34)         \\
\addlinespace
Home Timeouts           &     0.12\sym{***}\\
                    &      (8.74)         \\
\addlinespace
Away Timeouts           &     -0.11\sym{***}\\
                    &    (-12.47)         \\
\addlinespace
Ball Possession Time  &     -0.05.\\
                    &    (-1.66)         \\
\addlinespace
Time Lapsed       &   -0.05.\\
                    &      (-1.66)         \\
\addlinespace
Down                &   -0.01         \\
                    &      (0.04)         \\
\addlinespace
Field Position            &   0.02\sym{**} \\
                    &      (2.71)         \\
\addlinespace
Yards-to-go                &  -0.01         \\
                    &      (0.23)         \\
\addlinespace
Rating differential         &       0.75\sym{***}\\
                    &     (80.47)         \\
\addlinespace
Intercept            &       0.57\sym{*}\\
                    &    (2.09)         \\
\midrule
Observations        &      338,294         \\
\bottomrule
\multicolumn{2}{l}{\footnotesize \textit{t} statistics in parentheses}\\
\multicolumn{2}{l}{\footnotesize \sym{$_.$} \(p<0.1\), \sym{*} \(p<0.05\), \sym{**} \(p<0.01\), \sym{***} \(p<0.001\)}\\
\end{tabular}
\end{center}
\caption{Standardized logisitic regression coefficients for {\method}.}
\label{tab:iwinrnfl}
\end{table}


As we can see, as one might have expected the current scoring differential exhibits the strongest correlation with the in-game win probability.  
The only factors that do not appear to be statistically significant predictors of the dependent variable are the down and the yards-to-go. 
Even though the corresponding coefficients are negative as one might have expected (e.g., being at an earlier down gives you more chances to advance the ball), they are not significant in estimating the win probability. 
On the contrary, all else being equal timeouts appear to be quiet important since they can help a team stop the clock, while teams with better win percentage appear to have an advantage as well, since this can be a sign of a better team. 
In the following section we provide a detailed evaluation of {\method}.

\section{Numerical scheme}

We shall solve the PIDE \eqref{kolm_1}--\eqref{kolm_2} numerically with an Alternating Direction Implicit (ADI) method. The scheme is a modification of \cite{LiptonSepp} that is unconditionally stable and has second order of convergence in both time and space step.

In order to deal with a forward equation instead of a backward equation, we change the time variable to $\tau = T - t$, so that
\begin{equation}
	\label{pide_forward}
	\begin{aligned}
		& \frac{\partial V}{\partial \tau} = \mathcal{L} V(\tau, x_1, x_2) - \chi(\tau, x_1, x_2), \\
		& V(\tau, x_1, 0) = \phi_{0, 1}(\tau, x_1), \quad V(\tau, 0, x_2) =  \phi_{0, 2}(\tau, x_2), \\
		& V(\tau, x_1, x_2)  \underset{x_2 \to +\infty}{\longrightarrow} \phi_{\infty, 1}(\tau, x_1), \quad V(\tau, x_2, x_2)  \underset{x_1 \to +\infty}{\longrightarrow}  \phi_{\infty, 2}(\tau, x_2), \\
		& V(0, x_1, x_2) = \psi(x_1, x_2).
	\end{aligned}
\end{equation}

We consider the same grid for integral and differential part of the equation
\begin{equation}
	\begin{aligned}
		0 = x_1^0 < x_1^1 < \ldots < x_1^{m_1}, \\
		0 = x_2^0 < x_2^1 < \ldots < x_2^{m_2},
	\end{aligned}
\end{equation}
where $x_1^{m_1}$ and $x_2^{m_2}$ are large positive numbers.

The grid is non-uniform, and is chosen such that relatively many points lie near the default boundaries for better precision. We use a method similar to \cite{itkin2011jumps} to construct the grid.
\subsection{Discretization of the integral part of the PIDE}
In this section, we shall show how to deal with the integral part of the PIDE, and develop an iterative algorithm for the fast computation of the integral operator on the grid. To this end, we outline the scheme from \cite{LiptonSepp} and then give a new method.

The first approach is to deal with the integral operators directly. After the approximation of the integral, we get (\cite{LiptonSepp})
\begin{align}
	&\mathcal{J}_1 V(x_1 + h, x_2) = e^{-\varsigma_1 h} \mathcal{J}_1 V(x_1, x_2) +  \omega_0(\varsigma_1, h) V(x_1, x_2) + \omega_1(\varsigma_1, h) V(x_1 + h, x_2) + O(h^3), \label{J_1_approx}\\
	& \mathcal{J}_2 V(x_1, x_2 + h) = e^{-\varsigma_2 h} \mathcal{J}_2 V(x_1, x_2) +  \omega_0(\varsigma_2, h) V(x_1, x_2) + \omega_1(\varsigma_2, h) V(x_1, x_2 + h) + O(h^3) \label{J_2_approx},
\end{align}
where
\begin{equation*}
	\omega_0(\varsigma, h) = \frac{1 - (1 + \varsigma h) e^{-\varsigma h}}{\varsigma h}, \quad \omega_1(\varsigma, h) = \frac{-1 + \varsigma h + e^{-\varsigma h}}{\varsigma h}.
\end{equation*}

We can also approximate $\mathcal{J}_{12} V = \mathcal{J}_1 \mathcal{J}_2 V$ by applying above approximations for $\mathcal{J}_1$ and $\mathcal{J}_2$ consecutively.


Consider the grid
\begin{equation}
	\begin{aligned}
		0 = x_1^0 < x_1^1 < \ldots < x_1^{m_1}, \\
		0 = x_2^0 < x_2^1 < \ldots < x_2^{m_2},
	\end{aligned}
\end{equation}
where $x_1^{m_1}$ and $x_2^{m_2}$ are large positive numbers.\\

Then, we can write recurrence formulas for computing the integral operator on the grid. Denote $J_1^{i, j}, J_2^{i, j}$, $J_{12}^{i, j}$ the corresponding approximations of $\mathcal{J}_{1}V(x_1^i, x_2^j)$ , $ \mathcal{J}_{2}V(x_1^i, x_2^j)$, $\mathcal{J}_{12}V(x_1^i, x_2^j)$ on the grid. Applying (\ref{J_1_approx}) and (\ref{J_2_approx}) we get

\begin{align}
	&J_1^{i+1, j} = e^{-\varsigma_1 h^1_{i+1}} J_1^{i, j} + \omega_0(\varsigma_1, h^1_{i+1}) V(x_1^i, x_2^j) + \omega_1(\varsigma_1, h_{i+1}^1) V(x_1^{i+1}, x_2^{j}), \label{J_1_rec}\\
	&J_2^{i, j+1} = e^{-\varsigma_2 h^2_{j+1}} J_2^{i, j} + \omega_0(\varsigma_2, h^2_{j+1}) V(x_1^i, x_2^j) + \omega_1(\varsigma_2, h_{j+1}^2) V(x_1^{i}, x_2^{j+1}), \label{J_2_rec}
\end{align}
where $h_{i+1}^1 = x_1^{i+1} - x_1^{i}, h_{j+1}^2 = x_2^{j+1} - x_2^{j}$.

For an alternative method, we rewrite the integral operator as a differential equation
\begin{align}
		& \frac{\partial}{\partial x_1} \left(\mathcal{J}_1 V(x_1, x_2)e^{\varsigma_1 x_1} \right) =  \varsigma_1 V(x_1, x_2) e^{\varsigma_1 x_1}, \label{J1_ode}\\
		& \frac{\partial}{\partial x_2} \left(\mathcal{J}_2 V(x_1, x_2)e^{\varsigma_2 x_2} \right) =  \varsigma_2 V(x_1, x_2) e^{\varsigma_2 x_2}, \label{J2_ode}\\
		& \frac{\partial^2}{\partial x_1 \partial x_2} \left(\mathcal{J}_{12} V(x_1, x_2)e^{\varsigma_1 x_1 + \varsigma_2 x_2} \right) =  \varsigma_1  \varsigma_2 V(x_1, x_2) e^{\varsigma_1 x_1 + \varsigma_2 x_2}. \label{J12_pde}
\end{align}
Then, we apply the Adams-Moulton method of second order which gives us third order of accuracy locally (\cite{butcher2008numerical})
\begin{align}
	&J_1^{i+1, j} = e^{-\varsigma_1 h^1_{i+1}} J_1^{i, j} + \frac{1}{2} h^1_{i+1} e^{-\varsigma_1 h^1_{i+1}} \varsigma_1  V(x_1^i, x_2^j) + \frac{1}{2} h^1_{i+1} \varsigma_1 V(x_1^{i+1}, x_2^{j}), \label{J_1_rec_adams}\\
	&J_2^{i, j+1} = e^{-\varsigma_2 h^2_{j+1}} J_2^{i, j} +\frac{1}{2} h^2_{j+1} e^{-\varsigma_2 h^2_{j+1}} \varsigma_2  V(x_1^i, x_2^j) + \frac{1}{2} h^2_{j+1} \varsigma_2 V(x_1^{i}, x_2^{j+1}), \label{J_2_rec_adams}
\end{align}
where $h_{i+1}^1 = x_1^{i+1} - x_1^{i}, h_{j+1}^2 = x_2^{j+1} - x_2^{j}$, and is equivalent to the trapezoidal rule.

We can rewrite (\ref{J_1_rec_adams})--(\ref{J_2_rec_adams}) in the same notation as  (\ref{J_1_rec})--(\ref{J_2_rec}) by defining
\begin{equation*}
	\omega_0(\varsigma, h) = \frac{1}{2} h e^{-\varsigma h} \varsigma, \quad \omega_1(\varsigma, h) = \frac{1}{2} h \varsigma.
\end{equation*}
So,
\begin{align*}
	&J_1^{i+1, j} = e^{-\varsigma_1 h^1_{i+1}} J_1^{i, j} + \omega_0(\varsigma_1, h^1_{i+1}) V(x_1^i, x_2^j) + \omega_1(\varsigma_1, h_{i+1}^1) V(x_1^{i+1}, x_2^{j}),\\
	&J_2^{i, j+1} = e^{-\varsigma_2 h^2_{j+1}} J_2^{i, j} + \omega_0(\varsigma_2, h^2_{j+1}) V(x_1^i, x_2^j) + \omega_1(\varsigma_2, h_{j+1}^2) V(x_1^{i}, x_2^{j+1}). 
\end{align*}

As a result we get explicit recursive formulas for approximations of $\mathcal{J}_1 V$ and $\mathcal{J}_2 V$ that can be computed for all grid points via $O(m_1 m_2)$ operations. Both methods give the same order of accuracy. As was discussed above, in order to compute the approximation of $\mathcal{J}_{12} V$ we can apply consecutively the approximations of $\mathcal{J}_2 V$ and $\mathcal{J}_1 (\mathcal{J}_2 V)$. So, we have the two-step procedure:
\begin{equation}
	I_{12}^{i+1, j} = e^{-\varsigma_1 h^1_{i+1}} I_{12}^{i, j} + \omega_0(\varsigma_1, h^1_{i+1}) V(x_1^i, x_2^j) + \omega_1(\varsigma_1, h_{i+1}^1) V(x_1^{i+1}, x_2^{j}), \label{I_12_rec}\\
\end{equation}
and 
\begin{equation}
	J_{12}^{i, j+1} = e^{-\varsigma_2 h^2_{j+1}} J_{12}^{i, j} + \omega_0(\varsigma_2, h^2_{j+1}) I_{12}^{i, j} + \omega_1(\varsigma_2, h_{j+1}^2) I_{12}^{i, j+1}. \label{J_12_rec}\\
\end{equation}
Using this two-step procedure, we can also compute an approximation of $\mathcal{J}_{12} V$ on the grid in complexity $O(m_1 m_2)$.

 We shall subsequently analyze the stability of the second method and use it in the numerical tests. The results for the first method would be very similar.
 
% \paragraph{Eigenvalues of discretized jump operator.}
%\label{jump_eigs}
For the implementation, computing and storing a matrix representation of the jump operator is not necessary, since the operator can be computed iteratively as described above, but we shall use matrix notation for the analysis. We henceforth denote $J_1, J_2$, and $J_{12}$ the matrices of the discretized jump operators. From (\ref{J_1_rec})--({\ref{J_2_rec}) we can find that the matrices $J_1$ and $J_2$ are lower-triangular with diagonal elements $w_1 = \omega_1(\varsigma_1, h_1)$ and $w_2 =  \omega_1(\varsigma_2, h_2)$. Then, $J_{12} = J_1 J_2$ is also a lower-triangular matrix with diagonal elements $w_1 w_2$. To illustrate, in Figure \ref{jump_matrices} we plot the sparsity patterns in $J_1, J_2$, and $J_{12}$.
 \begin{figure}[H]
	\begin{center}
				\subfloat[%Sparsity pattern of 
				$J_1$.]{\includegraphics[width=0.32\textwidth,trim={2cm 0 2cm 0.5cm},clip]{jump_matrix_1.png}}
				\subfloat[%Sparsity pattern of 
				$J_2$.]{\includegraphics[width=0.32\textwidth,trim={2cm 0 2cm 0.5cm},clip]{jump_matrix_2.png}}
				\subfloat[%Sparsity pattern of 
				$J_{12}$.]{\includegraphics[width=0.32\textwidth,trim={2cm 0 2cm 0.5cm},clip]{jump_matrix_12.png}}
	\end{center}		
	\vspace{-20pt}
	\caption{Sparsity pattern of $J_1$, $J_2$, and $J_{12}$. Here, $m_1=m_2=20$ and $nz$ is the number of non-zero elements of the matrices.}
 	\label{jump_matrices}
\end{figure}
%Since the matrices are lower-triangular, the eigenvalues are the diagonal elements. Thus, matrix $J_1$ has all eigenvalues equal to $w_1$, matrix $J_2$ has all eigenvalues equal to $w_2$, and matrix $J_{12}$ has all eigenvalues equal to $w_{12} = w_1 w_2$.

\subsection{Discretization of the differential part of the PIDE}
Now consider the approximation of derivatives in the differential operator on a non-uniform grid. We use the standard derivative approximation (\cite{kluge2002}, \cite{in2010adi}). For the first derivative over each variable consider right-sided, central, and left-sided schemes. So, for the derivative over $x_1$ we have:
\begin{align}
&	\frac{\partial V}{\partial x_1}(x_1^i, x_2^j) \approx \alpha^1_{i, -2} V(x_1^{i-2}, x_2^j) + \alpha^1_{i, -1} V(x_1^{i-1}, x_2^j)+ \alpha^1_{i, 0} V(x_1^i, x_2^j), \label{D_x1_1}\\
&	\frac{\partial V}{\partial x_1}(x_1^i, x_2^j) \approx \beta^1_{i, -1} V(x_1^{i-1}, x_2^j) + \beta^1_{i, 0} V(x_1^{i}, x_2^j)+ \beta^1_{i, 1} V(x_1^{i+1}, x_2^j), \label{D_x1_center}\\
&	\frac{\partial V}{\partial x_1}(x_1^i, x_2^j) \approx \gamma^1_{i, 0} V(x_1^{i}, x_2^j) + \gamma^1_{i, 1} V(x_1^{i+1}, x_2^j)+ \gamma^1_{i, 2} V(x_1^{i+2}, x_2^j) \label{D_x1_2},
\end{align}
while for derivative over $x_2$ we have:
\begin{align}
&	\frac{\partial V}{\partial x_2}(x_1^i, x_2^j) \approx \alpha^2_{j, -2} V(x_1^i, x_2^{j-2}) + \alpha^2_{j, -1} V(x_1^i, x_2^{j-1})+ \alpha^2_{j, 0} V(x_1^i, x_2^j), \label{D_x2_1} \\
&	\frac{\partial V}{\partial x_2}(x_1^i, x_2^j) \approx \beta^2_{j, -1} V(x_1^i, x_2^{j-1}) + \beta^2_{j, 0} V(x_1^{i}, x_2^j)+ \beta^2_{j, 1} V(x_1^j, x_2^{j+1}), \label{D_x2_center}\\
&	\frac{\partial V}{\partial x_2}(x_1^i, x_2^j) \approx \gamma^2_{j, 0} V(x_1^{i}, x_2^j) + \gamma^2_{j, 1} V(x_1^i, x_2^{j+1}, x_2^j)+ \gamma^2_{j, 2} V(x_1^i, x_2^{j+2}), \label{D_x2_2}
\end{align}
with coefficients
\begin{equation*}
	\begin{aligned}
		\alpha^k_{i, -2} &= \frac{\Delta x_k^i}{\Delta x_k^{i-1} (\Delta x_k^{i-1} + \Delta x_k^i)},  & \alpha^k_{i, -1} &= \frac{-\Delta x_k^{i-1} - \Delta x_k^i}{\Delta x_k^{i-1} \Delta x_k^i},  &\alpha^k_{i, 0} &= \frac{\Delta x_k^{i-1} + 2 \Delta x_k^i}{\Delta x_k^i (\Delta x_k^{i-1} + \Delta x_k^i)} , \\
		\beta^k_{i, -1} &= \frac{-\Delta x_k^{i+1}}{\Delta x_k^{i} (\Delta x_k^{i} + \Delta x_k^{i+1})}, & \beta^k_{i, 0} &= \frac{\Delta x_k^{i+1} - \Delta x_k^i}{\Delta x_k^{i} \Delta x_k^{i+1}},  &\beta^k_{i, 1} &= \frac{\Delta x_k^{i}}{\Delta x_k^{i+1} (\Delta x_k^{i} + \Delta x_k^{i+1})} , \\
		\gamma^k_{i, 0} &= \frac{-2\Delta x_k^{i+1} - \Delta x_k^{i+2}}{\Delta x_k^{i+1} (\Delta x_k^{i+1} + \Delta x_k^{i+2})}, & \gamma^k_{i, 1} &= \frac{\Delta x_k^{i+1} + \Delta x_k^{i+2}}{\Delta x_k^{i+1} \Delta x_k^{i+2}},  & \gamma^k_{i, 2} &= \frac{-\Delta x_k^{i+1}}{\Delta x_k^{i+2} (\Delta x_k^{i+1} + \Delta x_k^{i+2})} .
	\end{aligned}
\end{equation*}
For the boundaries at $0$ we use the schemes \eqref{D_x1_1} and \eqref{D_x2_1}, for the right boundaries at $x_1^{m_1}$ and $x_2^{m_2}$ we use the schemes \eqref{D_x1_2} and \eqref{D_x2_2}, and for other points we use the central schemes \eqref{D_x1_center} and \eqref{D_x2_center}.

To approximate the second derivative we use the central scheme:
\begin{align}
	&	\frac{\partial^2 V}{\partial x_1^2}(x_1^i, x_2^j) \approx \delta^1_{i, -1} V(x_1^{i-1}, x_2^j) + \delta^1_{i, 0} V(x_1^{i}, x_2^j)+ \delta^1_{i, 1} V(x_1^{i+1}, x_2^j), \label{D2_x1} \\
&	\frac{\partial^2 V}{\partial x_2^2}(x_1^i, x_2^j) \approx \delta^2_{j, -1} V(x_1^i, x_2^{j-1}) + \delta^2_{j, 0} V(x_1^{i}, x_2^j)+ \delta^2_{j, 1} V(x_1^j, x_2^{j+1}) \label{D2_x2},
\end{align}
with coefficients
\begin{equation*}
		\delta^k_{i, -1} = \frac{2}{\Delta x_k^{i} (\Delta x_k^{i} + \Delta x_k^{i+1})}, \quad \delta^k_{i, 0} = \frac{-2}{\Delta x_k^{i} \Delta x_k^{i+1}}, \quad \delta^k_{i, 1} = \frac{2}{\Delta x_k^{i+1} (\Delta x_k^{i} + \Delta x_k^{i+1})},
\end{equation*}	
and to approximate the second mixed derivative we use the scheme:
\begin{equation}
	\frac{\partial^2 V}{\partial x_1 \partial x_2} (x_1^i, x_2^j) \approx \sum_{k, l = -1}^1 \beta_{i, k}^1 \beta_{j, l}^2 V(x_1^{i+k}, x_2^{j+l}). \label{D_x1x2}	
\end{equation}

As a result, we can approximate the differential operator $\mathcal{D} V$ by a discrete operator
\begin{equation}
	D V = D_1 V + D_2 V + D_{12} V,
\end{equation}
where $D_1 V$ contains the discretized derivatives over $x_1$ defined in (\ref{D_x1_1})--(\ref{D_x1_2}) and (\ref{D2_x1}), $D_2 V$ contains the discretized derivatives over $x_2$ defined in (\ref{D_x2_1})--(\ref{D_x2_2}) and (\ref{D2_x2}), and $D_{12} V$ contains the discretized mixed derivative defined in (\ref{D_x1x2}).

By straightforward but lengthy Taylor expansion of the expression in (\ref{D_x1_1})--(\ref{D_x1x2}), the scheme (\ref{HV_scheme}) has  second order truncation error in variables $x_1$ and $x_2$ for meshes which are either uniform or smooth transformations of such meshes, as we shall consider later.

\subsection{Time discretization: ADI scheme}
After discretization over $(x_1, x_2)$ we can rewrite PIDE (\ref{pide_forward}) as a system of ordinary (linear) differential equations. Consider the vector $U(t) \in \mathbb{R}^{m_1m_2 \times 1}$ whose elements correspond to $V(t, x_1^i, x_2^j)$. Then
\begin{equation}
	\begin{aligned}
		& U'(t) = \tilde{A} U(t) + b(t), \\
		& U(0) = U_0,
	\end{aligned}
\end{equation}
where $\tilde{A} = D_1 + D_2 + D_{12} + \lambda_1 J_1 + \lambda_2 J_2 + \lambda_{12} J_{12} - (\lambda_1 + \lambda_2 + \lambda_{12}) I$, and $b(t)$ is determined from boundary conditions and the right-hand side.

To solve this system, we apply an ADI scheme for the time discretization. Consider, for simplicity, a uniform time mesh with time step $\Delta t: t_n = n \Delta t, n = 0, \ldots, N-1$. 

We decompose the matrix $\tilde{A}$  into three matrices, $\tilde{A} = \tilde{A}_0 + \tilde{A}_1 + \tilde{A}_2$, where
\begin{align*}
	& \tilde{A}_0 =  D_{12} + \lambda_1 J_1 + \lambda_2 J_2 + \lambda_{12} J_{12},  \\
  	& \tilde{A}_1 = D_1 - \left(\lambda_1 + \frac{\lambda_{12}}{2} \right) I, \\
	  & \tilde{A}_2 = D_2 - \left(\lambda_2 + \frac{\lambda_{12}}{2} \right) I,
\end{align*}
and $b(t) = b_0(t) + b_1(t) + b_2(t)$, where $b_0(t)$ corresponds to the right-hand side and the FD discretization of the mixed derivatives on the boundary, $b_1(t)$ and $b_2(t)$ correspond to the FD discretization of the derivatives over $x_1$ and $x_2$ on the boundary.

Now we can apply a traditional ADI scheme with matrices $\tilde{A}_0, \tilde{A}_1$, and $\tilde{A}_2$. We choose the Hundsdorfer--Verwer (HV) scheme (\cite{HV}) in order to have second order accuracy in the time variable, and unconditional stability, as we shall prove below. For convenience, denote
\begin{align}
	& F_j(t, x) = \tilde{A}_j x + b_j(t), \quad j = 0, 1, 2, \label{F_j}\\
	& F(t, x) = (\tilde{A}_0 + \tilde{A}_1 + \tilde{A}_2 ) x + (b_0(t) + b_1(t) + b_2(t)),
\end{align}
and apply the Hundsdorfer--Verwer (HV) scheme:
\begin{equation}
	\label{HV_scheme}
	\left\{
	\begin{aligned}
	&	Y_0 = U_{n-1} + \Delta t F(t_{n-1}, U_{n-1}), \\
	&	Y_j = Y_{j-1} + \theta \Delta t (F_j(t_n, Y_j) - F_j(t_n, U_{n-1})), \quad j = 1, 2, \\
	&	\tilde{Y}_0 = Y_0 + \sigma \Delta t (F(t_n, Y_2) - F(t_{n-1}, U_{n-1})), \\
	&	\tilde{Y}_j = \tilde{Y}_{j-1} + \theta \Delta t (F_j(t_n, \tilde{Y}_j - F_j(t_n, Y_2)), \quad j = 1, 2, \\
	&	U_n = \tilde{Y}_2.
	\end{aligned}
	\right.
\end{equation}

In this scheme, parts that contain $F_1$ and $F_2$ are treated implicitly. The matrix $\tilde{A}_1$ is tridiagonal and $\tilde{A}_2$ is block-tridiagonal and can be inverted via $O(m_1 m_2)$ operations.  As a result, the overall complexity is $O(m_1 m_2)$ for a single time step or $O(N m_1 m_2)$ for the whole procedure.

Moreover, the scheme has second order of consistency in both $(x_1, x_2)$ and $t$ for any given $\theta$ and $\sigma = \frac{1}{2}$. 

\subsection{Stability analysis}
In this section, we consider the PIDE \eqref{pide_forward} with zero boundary conditions at $0$ in both directions and on a uniform grid,
such that $F_j(t, x) = \tilde{A}_j x$ and
\begin{equation}
	\label{HV_nobound}
	\left\{
	\begin{aligned}
	&	Y_0 = U_{n-1} + \Delta t \tilde{A} U_{n-1}, \\
	&	Y_j = Y_{j-1} + \theta \Delta t (\tilde{A}_j Y_j - \tilde{A}_j U_{n-1}),\quad j = 1, 2, \\
	&	\tilde{Y}_0 = Y_0 + \sigma \Delta t (\tilde{A} Y_2- \tilde{A} U_{n-1}), \\
	&	\tilde{Y}_j = \tilde{Y}_{j-1} + \theta \Delta t (\tilde{A}_j\tilde{Y}_j - \tilde{A}_j Y_2), \quad j = 1, 2 \\
	&	U_n = \tilde{Y}_2.
	\end{aligned}
	\right.
\end{equation}

For convenience, we denote by $F: U_n = F U_{n-1}$.

We further consider the PDE on $\mathbb{R}^2$, i.e., without default boundaries. Hence, we assume that diffusion and jump operators are discretized on 
 an infinite, uniform mesh $\{(j_1 h_1, j_2 h_2), (j_1, j_2) \in \mathbb{Z}^2\}$, such that, e.g.\ $D_1, D_2, D_{12}, J_1, J_2$ are infinite matrices.
 This is different to \cite{intHoutStability}, where finite matrices and periodic boundary conditions (without integral terms) are considered.

We use von Neumann stability analysis, as first introduced by \cite{charney1950numerical}, by expanding the solution into a Fourier series.
Hence, we shall show that the proposed scheme (\ref{HV_nobound}) is unconditionally stable,
 i.e.\ we will show that all eigenvalues of the operator  $F$ have moduli bounded by 1 plus an $O(\Delta t)$ term,
 where the corresponding eigenfunctions are given by $\exp(i \phi_1 j_1) \exp(i \phi_2 j_2)$, with $\phi_1$ and $\phi_2$ the wave numbers and
 $j_1$ and $j_2$ the grid coordinates.
 
%\paragraph{Stability analysis of scheme (\ref{HV_nobound})}
 \cite{intHoutStability} show that when all matrices commute
(as in the PDE case with periodic boundary conditions), 
the eigenvalues for $F$ are given by %this leads to the condition %(\cite{intHoutStability})
%\begin{equation}
%	\label{stab_eq}
%	|T(\tilde{z}_0, \tilde{z}_1, \tilde{z}_2)| \le 1,
%\end{equation}
\begin{eqnarray}
\label{defT}
T(\tilde{z}_0, \tilde{z}_1, \tilde{z}_2) &=& 1 + 2 \frac{\tilde{z}_0 + \tilde{z}}{p} - \frac{\tilde{z}_0 + \tilde{z}}{p^2} + \sigma \frac{(\tilde{z}_0 + \tilde{z})^2}{p^2} \quad \text{with} \\
	p &=& (1 - \theta \tilde{z}_1) (1 - \theta \tilde{z}_2), \nonumber
\end{eqnarray}
where $\tilde{z}_j = \tilde{\mu}_j \Delta t$, where $\tilde{\mu}_j$ is an eigenvalue of $\tilde{A}_j$, $j = 0, 1, 2$, $\tilde{z} = \tilde{z}_1 + \tilde{z}_2$, $\theta \ge 0$.


The analysis is made slightly more complicated in our case through the presence of the  jump operators.
In the remainder of this section, %we first show that for our case \eqref{stab_eq} can still be applied, and then 
we show that stability is still given under the same conditions on $\theta$ and $\sigma$ as in the purely diffusive case. For the correspondence of notation with \cite{intHoutStability}, we denote $A = A_0 + A_1 + A_2$, where $A_0 = D_{12}, A_1 = D_1$, $A_2 = D_2$ and $\mu_0, \mu_1$, and $\mu_2$ are the eigenvalues of the corresponding matrices. Similar to $\tilde{z}_0, \tilde{z}_1,$ and $\tilde{z}_2$, we define scaled eigenvalues $z_0 = \mu_0 \Delta t, z_1 = \mu_1 \Delta t, z_2 = \mu_2 \Delta t$.



%Using properties of lower-triangular matrices, we can easily see that the matrix $F$ from \eqref{HV_nobound} can be represented as $F = U T_F U^{*}$, where $T_F$ is a lower-triangular matrix, whose eigenvalues are equal to \eqref{defT}.




%\begin{lemma}
%\label{lemma_commute}
%The following identities are satisfied
We have the eigenvalues $\tilde{\mu}_j$ of $\tilde{A}_j$ given by
\begin{align}
	& \tilde{\mu}_0 = \mu_0 + \lambda_1 w_1 + \lambda_2 w_2 + \lambda_{12} w_{12}, \label{mu_0_eq} \\
	& \tilde{\mu}_1 = \mu_1 - \left(\lambda_1 + \frac{\lambda_{12}}{2}\right), \label{mu_1_eq}\\
	& \tilde{\mu}_2 = \mu_2 - \left(\lambda_2 + \frac{\lambda_{12}}{2}\right) \label{mu_2_eq},
\end{align}
where $\mu_j$ is  an eigenvalue of $A_j$, and $w_1, w_2$, and $w_{12}$ are eigenvalues of $J_1, J_2$, and $J_{12}$.
%\end{lemma}

Denote % $z_j = \mu_j \Delta t$ with $\mu_j$  an eigenvalue of $A_j$, 
$z = z_1 + z_2$, $s_1 = w_1  \Delta t, s_2 = w_2 \Delta t, s_{12} = w_{12} \Delta t$, where $w_1, w_2, w_{12}$ are eigenvalues of $J_1, J_2, J_{12}$ respectively, and $s_0 = \lambda_1 s_1 + \lambda_2 s_2 + \lambda_{12} s_{12}$.

	Multiplying (\ref{mu_0_eq})--(\ref{mu_2_eq}) by $\Delta t$, we have 
	\begin{align} 
		& \tilde{z}_0 = z_0 + s_0, \label{tilde_z0} \\
		& \tilde{z}_1 = z_1 - \left(\lambda_1 + \frac{\lambda_{12}}{2}\right) \Delta t, \\
		& \tilde{z}_2 = z_2 - \left(\lambda_2 + \frac{\lambda_{12}}{2}\right) \Delta t.  \label{tilde_z2} 		
	\end{align}
\begin{theorem}[\cite{intHoutStability}, Theorem 3.2]
	\label{theor_inthout}
	Assume $\Re({z}_1) \le 0, \Re({z}_2) \le 0$, $|{z}_0| \le 2\sqrt{\Re({z}_1) \Re({z}_2)}$, where ${z}_0, {z}_1$, and ${z}_2$ are the eigenvalues of ${A}_0, {A}_1$, and ${A}_2$, %in \eqref{F_j} 
	and
	\begin{equation*}
		\frac{1}{2} \le \sigma \le \left(1 + \frac{\sqrt{2}}{2} \right) \theta.
	\end{equation*}
	Then,
	\begin{equation*}
		|T({z}_0, {z}_1, {z}_2)| \le 1,
	\end{equation*}
	and the Hundsdorfer--Verwer scheme \eqref{HV_nobound} is stable in the purely diffusive case.
\end{theorem}

\begin{lemma}
The scaled eigenvalues of $A_0$, $A_1$, $A_2$, $J_1$, $J_2$, $J_{12}$ can be expressed as
\begin{eqnarray}
\label{z0}
	z_0 &=& -\rho b [\sin{\phi_1} \sin{\phi_2}],  \\
	\label{z1}
	z_1 &=& -a_1 (1 - \cos{\phi_1}) + i \xi_1 q_1 \sin{\phi_1}, \\
	\label{z2}
	z_2 &=& -a_2 (1 - \cos{\phi_2}) + i \xi_2 q_2 \sin{\phi_2}, \\
%	s_1 &=& \Delta t \, \zeta_1 h_1 \left(\frac{1}{2} + \frac{\exp(-h_1 (\zeta_1 + i \phi_1))}{1-\exp(-h_1 (\zeta_1 + i \phi_1))} \right), \\
%	s_2 &=& \Delta t \, \zeta_2 h_2 \left(\frac{1}{2} + \frac{\exp(-h_2 (\zeta_2 + i \phi_2))}{1-\exp(-h_2 (\zeta_2 + i \phi_2))} \right), \\
	s_1 &=& \Delta t \, \zeta_1 h_1 \left(\frac{1}{2} + \frac{\exp(-h_1 \zeta_1 + i \phi_1)}{1-\exp(-h_1 \zeta_1 + i \phi_1)} \right), \\
	s_2 &=& \Delta t \, \zeta_2 h_2 \left(\frac{1}{2} + \frac{\exp(-h_2 \zeta_2 + i \phi_2)}{1-\exp(-h_2 \zeta_2 + i \phi_2)} \right), \\
	s_{12} &=& s_1 s_2/\Delta t,
\end{eqnarray}
where
\begin{equation*}
	q_1 = \frac{\Delta t}{h_1}, \quad q_2 = \frac{\Delta t}{h_2}, \quad a_1 = \frac{\Delta t}{h_1^2}, \quad a_2 = \frac{\Delta t}{h_2^2}, \quad b= \frac{\Delta t}{h_1 h_2},
\end{equation*}
and $\phi_j \in [0, 2 \pi]$ for $j = 1, 2$.

Moreover,
\begin{equation}
\label{karelineq}
	|z_0| \le 2 \sqrt{\Re(z_1) \Re(z_2)}.
\end{equation}	
\end{lemma}
\begin{proof}
	All six eigenvalues follow by insertion of the ansatz $U=\exp(i \phi_1 j_1) \exp(i \phi_2 j_2)$.
	For instance, %for $U(j,k)=$
	\[
	(J_1 U)(j_1,j_2) = \zeta_1 h_1 \left(\frac{1}{2} U(j_1,j_2) +  \sum_{k=1}^\infty \exp(-\zeta_1 h_1 k) U(j_1-k,j_2) \right),
	\]
	and the result follows by using the special form of $U$ and evaluating the geometric series.
	
	Alternatively, the first three equations follow immediately from the eigenvalues for finite matrices (\cite{intHoutStability}, p.29),
	which are given by (\ref{z0})--(\ref{z2}) where $\phi_j = 2 l \pi/m_j$, $l=1,\ldots,m_j$.
	In the infinite mesh case, the spectrum is the continuous limit and (\ref{karelineq}) still holds.
%	 then the eigenvalues of the semi-infinite (banded Toeplitz) matrix are given by the Schmidt and Spitzer Theorem (see Theorem 11.17 in \cite{bottcher2005spectral}) as precisely the limits of sequences of eigenvalues of the finite-dimensional matrices.
\end{proof}


%\begin{lemma}
%\label{lemma_z0ineq}
%For $\tilde{z}_0, \tilde{z}_1, \tilde{z}_2$ from \eqref{tilde_z0}--\eqref{tilde_z2},
%\begin{equation}
%	\label{z_0_ineq}
%	|\tilde{z}_0|^2 \le 4 \Re(\tilde{z}_1) \Re(\tilde{z}_2) + c \Delta t,
%\end{equation}
%for some constant $c$ that does not depend on $\Delta t, h_1,$ and $h_2$.
%\end{lemma}
%\begin{proof}
%Define, 
%\begin{gather*}
%	\tilde{\lambda}_1 = \left(\lambda_1 + \frac{\lambda_{12}}{2} \right) \Delta t, \\
%	\tilde{\lambda}_2 = \left(\lambda_2 + \frac{\lambda_{12}}{2} \right) \Delta t.
%\end{gather*}
%Then,
%\begin{align}
%	|\tilde{z}_0|^2 = |z_0 + s_0|^2 = |z_0|^2 + 2 z_0 s_0 + s_0^2 &\le 4 \Re(z_1) \Re(z_2) + 2 z_0 s_0 + s_0^2  \\
%	& = 4 (\Re(\tilde{z}_1) + \tilde{\lambda}_1)( \Re(\tilde{z}_2) + \tilde{\lambda}_2) + 2 z_0 s_0 + s_0^2 .
%\end{align}
%We can rewrite the first term as
%\begin{equation*}
%	 (\Re(\tilde{z}_1) + \tilde{\lambda}_1)( \Re(\tilde{z}_2) + \tilde{\lambda}_2) = \Re(\tilde{z}_1) \Re(\tilde{z}_2) + \Re(z_1) \tilde{\lambda}_2 + \Re(z_2) \tilde{\lambda}_1 - \tilde{\lambda}_1 \tilde{\lambda}_2.
%\end{equation*}
%Thus, we have
%\begin{equation*}
%	|z_0 + s_0|^2 \le 4\Re(\tilde{z}_1) \Re(\tilde{z}_2) + 4 \Re(z_1) \tilde{\lambda}_2 + 4 \Re(z_2) \tilde{\lambda}_1 - 4\tilde{\lambda}_1 \tilde{\lambda}_2 + 2 z_0 s_0 + s_0^2.
%\end{equation*}
%From \eqref{z0}--\eqref{z2}, we can see that $z_0, z_1$, and $z_2$ does not depend on $\Delta t$ (they depend on $\frac{\Delta t}{h_i}$, which is assumed to be constant), while $\tilde{\lambda_1}, \tilde{\lambda_2},$ and $s_0$ are proportional to $\Delta t$. Thus
%\begin{equation}
%	4 \Re(z_1) \tilde{\lambda}_2 + 4 \Re(z_2) \tilde{\lambda}_1 - 4\tilde{\lambda}_1 \tilde{\lambda}_2 + 2 z_0 s_0 + s_0^2 \le c \Delta t,
%\end{equation}
%for some constant $c$.
%
%Thus, we have proved the inequality (\ref{z_0_ineq}).
%\end{proof}

\begin{theorem}
	Consider $\frac{1}{2} \le \sigma \le \left(1 + \frac{\sqrt{2}}{2} \right) \theta$. Then there exists $c>0$, independent of $\Delta t\le 1$, $h_1$ and $h_2$, such that
	\begin{enumerate}
	\item
	\begin{equation}
	|T(\tilde{z}_0, \tilde{z}_1, \tilde{z}_2)| \le 1 + c \Delta t, \qquad \forall \phi_1, \phi_2 \in [0,2\pi],
	\end{equation}
	i.e., the scheme is von Neumann stable;
	\item
	\label{part2}
	\begin{equation}
	|U_n|_2 %:= \sum_{j=-\infty}^{\infty} U_{n}(j)^2 
	\le {\rm e}^{c n \Delta t} |U_0|_2, \qquad \forall n\ge 0,
	\end{equation}
	for $|U_n|_2 = h_1 h_2 \left(\sum_{j_1,j_2=-\infty}^\infty |U_n(j_1,j_2)|^2\right)^{\scriptsize 1/2}$, i.e., the scheme is $l_2$ stable.
	\end{enumerate}
\end{theorem}
\begin{proof}


%The first inequality is a weak version of Theorem 3.2 of  \cite{intHoutStability} and can be proved in the similar way by 
%adding the term $c \Delta t$.

First, we have that
\begin{eqnarray*}
|T({z}_0, \tilde{z}_1, \tilde{z}_2)| = \left|1 + 2 \frac{{z}_0 + \tilde{z}}{p} - \frac{{z}_0 + \tilde{z}}{p^2} + \sigma \frac{({z}_0 + \tilde{z})^2}{p^2}\right|
\le 1,
\end{eqnarray*}
where as before $p = (1 - \theta \tilde{z}_1) (1 - \theta \tilde{z}_2)$ and $\tilde{z} = \tilde{z}_1 + \tilde{z}_2$. 
This follows from Theorem \ref{theor_inthout} because $\lambda_1$, $\lambda_2$ and $\lambda_{12}$ are positive and therefore (\ref{karelineq}) is still satisfied with $z_1$ and $z_2$ replaced by $\tilde{z}_1$ and $\tilde{z}_2$.

We have
\begin{eqnarray*}
T(\tilde{z}_0, \tilde{z}_1, \tilde{z}_2) &=&
T({z}_0, \tilde{z}_1, \tilde{z}_2) +
2 \frac{s_0}{p} - \frac{s_0}{p^2} + \sigma \frac{2 s_0 (z_0 + \tilde{z}) + s_0^2}{p^2}.
\end{eqnarray*}
A simple calculation shows that $|s_0|\le c_0 \, \Delta t$ for a constant $c_0$ (independent of $\Delta t, h_1, h_2, \phi_1$, $\phi_2$;
indeed, $c_0=2 \lambda_1 + 2 \lambda_2 + 4 \lambda_{12}$
works for small enough $h_1$, $h_2$).
Therefore, and because $|p|\ge 1$, $|z_0 + \tilde{z}|/|p|\le c_1$ for a constant $c_1$,
\[
\left|2 \frac{s_0}{p} - \frac{s_0}{p^2} + \sigma \frac{2 s_0 (z_0 + \tilde{z}) + s_0^2}{p^2} \right|
\le c \Delta t,
\]
for any $c\ge (3 + 2 \sigma c_1 + c_0 \sigma) c_0$.
From this the first statement follows.

We can now deduce part \ref{part2} by a standard argument. For the discrete-continuous Fourier transform
\[
%\mathcal{F}: 
l_2(\mathbb{Z}^2) \rightarrow L_2(-\pi,\pi)^2, \qquad U \rightarrow \widehat{U}, \qquad \widehat{U}(\phi_1,\phi_2) = h_1 h_2 \sum_{j,k \in \mathbb{Z}} U(j,k) {\rm e}^{-i (j \phi_1 + k \phi_2)},
\]
we have
\[
\widehat{U}_{n+1}(\phi_1,\phi_2) = T(\tilde{z}_0, \tilde{z}_1, \tilde{z}_2) \, \widehat{U}_{n}(\phi_1,\phi_2), \qquad \forall n\ge 0.
\]
Then, by Parseval,
\begin{eqnarray*}
|U_n|_2^2 &=& \frac{1}{4 \pi^2} |\widehat{U}_n|^2  \\
&=& \frac{1}{4 \pi^2} 
\frac{1}{h_1^2 h_2^2} \int_{-\pi}^\pi |\widehat{U}_n(\phi_1,\phi_2)|^2 
\, {\rm d} \phi_1 \, {\rm d} \phi_2 \\
&\le& \frac{1}{4 \pi^2} 
\frac{1}{h_1^2 h_2^2} \int_{-\pi}^\pi (1+c \Delta t)^{2n} |\widehat{U}_0(\phi_1,\phi_2)|^2 
\, {\rm d} \phi_1 \, {\rm d} \phi_2 \\
&\le& {\rm e}^{2 c n \Delta t}  \frac{1}{4 \pi^2} 
\frac{1}{h_1^2 h_2^2} \int_{-\pi}^\pi |\widehat{U}_0(\phi_1,\phi_2)|^2 
\, {\rm d} \phi_1 \, {\rm d} \phi_2 \\ 
&=& {\rm e}^{2 c n \Delta t}  |U_0|_2^2.
\end{eqnarray*}
\end{proof}



%\begin{remark}
This ($l_2$-)stability result together with second order consistency implies ($l_2$-)convergence of second order for all solutions which
are sufficiently smooth that the truncation error is defined and bounded. In our setting, where the initial condition is discontinuous, this is not given. Since the step function lies in the ($l_2$-)closure of smooth functions, convergence is guaranteed, but usually not of second order. We show this empirically in the next section and demonstrate how second order convergence can be restored practically.
%\end{remark}




\subsection{Discontinuous boundary and terminal conditions}

It is well documented (see, e.g.\ \cite{pooley2003}) that the spatial convergence order of central finite difference schemes is generally reduced to one for discontinuous payoffs. Moreover, the time convergence order of the Crank-Nicolson scheme is reduced to one due to the lack of damping of high-frequency components of the error, and this behaviour is inherited by the HV scheme. We address these two issues in the following way.

First, we smooth the terminal condition by the method of local averaging from \cite{pooley2003}, i.e., instead of using nodal values of $\phi$ directly, we use the approximation
\[
\phi(x_1^i,x_2^j) \approx \frac{1}{h_1 h_2} \int_{x_2^i-h_2/2}^{x_2^i+h_2/2} \int_{x_1^j-h_1/2}^{x_1^j+h_1/2}
\phi(\xi_1,\xi_2) \, d\xi_1 d\xi_2.
\]
For step functions with values of 0 and 1, this procedure attaches to each node the fraction of the area where the payoff is 1, in a cell of of size $h
_1 \times h_2$ centred at this point.

We illustrate the convergence improvement on the example of joint survival probabilities. Other quantities show a similar behaviour.
The model parameters in the following tests are the same as in the next section, specifically Table \ref{table:params}.

We choose $\sigma = \frac{1}{2}$ and $\theta = \frac{3}{4}$ in the HV scheme. 

The observed convergence with and without this smoothing procedure is shown in Figure \ref{fig_conv1}.
We choose the $l_2$-norm for its closeness to the stability analysis -- in the periodic case, Fourier analysis gives convergence results in $l_2$ -- and the  $l_\infty$-norm for its relevance to the problem at hand, where we are interested in the solution pointwise.
The behaviour in the $l_1$-norm is very similar.

Hereby, for a method of order $p\ge 1$ we estimate the error by extrapolation as
\begin{equation*}
|Q^{nX}(x_1, x_2) - Q(x_1, x_2)| \approx \frac{1}{2^p-1}  |Q^{nX}(x_1, x_2) - Q^{nX/2}(x_1, x_2)|,
\end{equation*}
where $Q$ is the exact solution, $Q^{nX}$ the solution with $nX$ mesh points,
and the norms are computed by either taking the maximum over mesh points or numerical quadrature.
Here, $nT=1000$ is fixed.



\begin{figure}[H]
	\begin{center}
%				\subfloat[$l_1$-norm.]{\includegraphics[width=0.33\textwidth]{conv_analysis2.png}}
%				\subfloat[$l_2$-norm.]{\includegraphics[width=0.33\textwidth]{conv_analysis1.png}}
%				\subfloat[$l_{\infty}$-norm.]{\includegraphics[width=0.33\textwidth]{conv_analysis3.png}}\\
%				\subfloat[$l_1$-norm.]{\includegraphics[width=0.45\textwidth]{conv_analysis2.png}}
				\subfloat[$l_2$-norm.]{\includegraphics[width=0.49\textwidth]{conv_analysis1.png}} \hfill
				\subfloat[$l_{\infty}$-norm.]{\includegraphics[width=0.49\textwidth]{conv_analysis3.png}}\\
	\end{center}		
	\vspace{-20pt}
	\caption{Convergence analysis for $l_2$- and $l_{\infty}$-norms of the error depending on the mesh size with fixed time-step.}
 	\label{fig_conv1}
\end{figure}

The convergence is clearly of first order without averaging and of second order with averaging.


Second, we modify the scheme using the idea from \cite{reisinger2013} by changing the time variable $\tilde{t} = \sqrt{t}$. This change of variables 
leads to the new PDE
\[
		\frac{\partial V}{\partial \tilde{t}} + 2 \tilde{t} \mathcal{L} V = 2 \tau \chi(\tilde{t}^2, x),
\]
instead of (\ref{kolm_1}), to which we apply the numerical scheme. %, but improves the convergence rate. 
%From numerical results in Section \ref{numerical_experiments}, we can observe second order of convergence.


In Figure \ref{fig_conv2}, we show the convergence with and without time change, estimating the errors in a similar way to above, with $nX=800$ fixed.
 \begin{figure}[H]
	\begin{center}
%				\subfloat[$l_1$-norm.]{\includegraphics[width=0.33\textwidth]{conv_analysis5.png}}
%				\subfloat[$l_2$-norm.]{\includegraphics[width=0.33\textwidth]{conv_analysis4.png}}
%				\subfloat[$l_{\infty}$-norm.]{\includegraphics[width=0.33\textwidth]{conv_analysis6.png}}\\
%				\subfloat[$l_1$-norm.]{\includegraphics[width=0.45\textwidth]{conv_analysis5.png}}
				\subfloat[$l_2$-norm.]{\includegraphics[width=0.49\textwidth]{conv_analysis4.png}} \hfill
				\subfloat[$l_{\infty}$-norm.]{\includegraphics[width=0.49\textwidth]{conv_analysis6.png}}\\

	\end{center}		
	\vspace{-20pt}
	\caption{Convergence analysis for $l_2$- and $l_{\infty}$-norms of the error depending on time-step with fixed mesh size.}
 	\label{fig_conv2}
\end{figure}

The convergence is clearly of first order without time change and of second order with time change. We took here $T=5$ to illustrate the effect more clearly.

%To justify second order convergence rate, in Figure \ref{fig_conv} we present the $l_2$-norm of error depending on the mesh size computed for joint survival probability. %We choose the number of time steps $n = 2m$, where $m$ is the mesh size in each direction.
% \begin{figure}[H]
%	\begin{center}
%		\includegraphics[width=0.9\textwidth]{conv_analysis.png}
%	\end{center}
%	\vspace{-20pt}
%	\caption{$l_2$ norm of error depending on the mesh size}
% 	\label{fig_conv}
%\end{figure}
%

\section{Numerical experiments}
\label{numerical_experiments}
In this section, we analyze the model characteristics and the impact of jumps. Specifically, we compute joint and marginal survival probabilities, CDS and FTD spreads as well as CVA and DVA depending on initial asset values. We also compute the difference between the solution with and without jumps.

Consider the parameters in Table \ref{table:params}.
\begin{table}[H]
	\begin{center}
		\begin{tabular}{| c | c | c | c | c | c | c | c | c | c | c | c |}
			\hline
			$L_{1,0}$ & $L_{2, 0}$ & $L_{12, 0}$ & $L_{21, 0}$ & $R_1$ & $R_2$ & $T$ & $\sigma_1$ & $\sigma_2$ & $\rho$ & $\varsigma_1$ & $\varsigma_2$ \\ 
			\hline
			60 & 70 & 10 & 15 & 0.4 & 0.45 & 1 & 1 & 1  & 0.5 & 1 & 1 \\
			\hline
		\end{tabular}
	\caption{Model parameters.\label{table:params}}		
	\end{center}
\end{table}
For the model with jumps, we further consider the parameters in Table \ref{table:jumps}.
\begin{table}[H]
	\begin{center}
		\begin{tabular}{| c | c | c | }
			\hline
			 $\lambda_1$& $\lambda_2$ & $\lambda_{12}$ \\
			\hline
			0.5 & 0.5 & 0.3 \\
			\hline
		\end{tabular}
		\caption{Jump intensities.\label{table:jumps}}
	\end{center}
\end{table}

We compute all tests using a $100\times100$ spatial grid with the maximum values $X_1^{100} = X_2^{100} = 10$ and constant time step $\Delta \tau = 0.01$. As the parameters of the HV scheme, we choose $\sigma = \frac{1}{2}$ and $\theta = \frac{3}{4}$. 

In Figures \ref{jointSurvProb1}--\ref{CVA1} we present various model characteristics and compare the results with and without jumps. From these figures, we can observe that jumps can have a significant impact, especially near the default boundaries:

\begin{itemize}
\item
in Figure \ref{jointSurvProb1} for the joint survival probability,
the biggest impact of jumps is around the default boundaries for both $x_1$ and $x_2$;
\item
in Figure \ref{marginalSurvProb1} for the marginal survival probability of the first bank,
we can observe that the biggest impact of jumps is near the default boundary of the first bank;
\item
for the CDS spread, in Figure \ref{CDSPrice1}, (b), the biggest impact of jumps is also seen near the default boundary, but it has the opposite direction, because jumps can only increase the CDS spread;
\item
in Figure \ref{FTDPrice1}, (d) for FTD the spread, the biggest impact of jumps is near both default boundaries, and it has a positive impact;
\item
finally, for CVA, (f), the highest impact of jumps is near the default boundary of the first bank, see Figure \ref{CVA1}.
\end{itemize}

\begin{figure}%[H]
	\begin{center}
		\subfloat[]{\includegraphics[width=0.5\textwidth]{joint_jumps.png}}
		\subfloat[]{\includegraphics[width=0.5\textwidth]{joint_diff_jumps.png}}\\
	\end{center}
	\vspace{-20pt}
	\caption{The joint survival probability: (a) value, (b) difference between model with and without jumps.}
 	\label{jointSurvProb1}
\end{figure}

 \begin{figure}%[H]
	\begin{center}
		\subfloat[]{\includegraphics[width=0.5\textwidth]{marginal_jumps.png}}
		\subfloat[]{\includegraphics[width=0.5\textwidth]{marginal_diff_jumps.png}}\\
	\end{center}
	\vspace{-20pt}
	\caption{The marginal survival probability: (a) value, (b)  difference between model with and without jumps.}
 	\label{marginalSurvProb1}
\end{figure}


 \begin{figure}%[H]
	\begin{center}
		\subfloat[]{\includegraphics[width=0.49\textwidth]{cds_jumps.png}}
		\subfloat[]{\includegraphics[width=0.49\textwidth]{cds_diff_jumps.png}}\\
		\subfloat[]{\includegraphics[width=0.49\textwidth]{ftd_jumps.png}}
		\subfloat[]{\includegraphics[width=0.49\textwidth]{ftd_diff_jumps.png}}\\
		\subfloat[]{\includegraphics[width=0.49\textwidth]{cva_jumps.png}}
		\subfloat[]{\includegraphics[width=0.49\textwidth]{cva_diff_jumps.png}}
	\end{center}
	\vspace{-20pt}
	\caption{
	Values of different credit products with left the value and right the difference between model with and without jumps.
	Top row: Credit Default Swap spread, written on the first bank.
	Middle row: First-to-Default spread.
	Bottom row: CVA of CDS, where the first bank is Reference name (RN) and the second bank is Protection Seller (PS).
	}
 	\label{CDSPrice1}
	\label{FTDPrice1}
	\label{CVA1}
\end{figure}




% \begin{figure}[H]
%	\begin{center}
%		\subfloat[]{\includegraphics[width=0.55\textwidth]{cds_jumps.png}}
%		\subfloat[]{\includegraphics[width=0.55\textwidth]{cds_diff_jumps.png}}\\
%	\end{center}
%	\vspace{-20pt}
%	\caption{Credit Default Swap spread, written on the first bank: (a) value, (b)  difference between model with and without jumps.}
% 	\label{CDSPrice1}
%\end{figure}
%
% \begin{figure}[H]
%	\begin{center}
%		\subfloat[]{\includegraphics[width=0.55\textwidth]{ftd_jumps.png}}
%		\subfloat[]{\includegraphics[width=0.55\textwidth]{ftd_diff_jumps.png}}\\
%	\end{center}
%	\vspace{-20pt}
%	\caption{First-to-Default spread: (a) value, (b)  difference between model with and without jumps.}
% 	\label{FTDPrice1}
%\end{figure}
%
% \begin{figure}[H]
%	\begin{center}
%		\subfloat[]{\includegraphics[width=0.55\textwidth]{cva_jumps.png}}
%		\subfloat[]{\includegraphics[width=0.55\textwidth]{cva_diff_jumps.png}}\\
%	\end{center}
%	\vspace{-20pt}
%	\caption{CVA of CDS, where the first bank is Reference name (RN) and the second bank is Protection Seller (PS) : (a) value, (b)  difference between model with and without jumps.}
% 	\label{CVA1}
%\end{figure}

%From the figures above, we can see that jumps have the most impact near the default boundaries, especially with large intensities. This might have a significant impact in model characteristics.


\section{Calibration}

In this section we present calibration results of the model. There are eight unknown parameters, see \eqref{kolmogorov_backward}--\eqref{j12_eq}: $\sigma_1, \sigma_2, \rho, \varsigma_1, \varsigma_2, \lambda_1, \lambda_2, \lambda_{12}$. We use CDS and equity put option prices (with different strikes) as market data. If FTD contracts are available, one can use them to estimate $\rho$ and $\lambda_{12}$. Otherwise, historical estimation with share prices time series can be used. 

The data for external liabilities can be found in banks' balance sheets, which are publicly available. Usually, mutual liabilities data are not public information, thus we made an assumption that they are a fixed proportion of the total liabilities, which coincides with \cite{DavidLehar}. In particular, we fix the mutual liabilities as 5\% of total liabilities.

The asset's value is the sum of liabilities and equity price.

We choose Unicredit Bank as the first bank and Santander as the second bank. In Table \ref{data_table} we provide their equity price $E_i$, assets $A_i$ and liabilities $L_i$. As in \cite{LiptonSepp}, the liabilities are computed as a ratio of total liabilities and shares outstanding.

\begin{table}[H]
	\begin{center}
		\begin{tabular}{| c | c | c | c | c | c |}
			\hline
			$E_1(0)$ & $L_1(0)$ & $A_1(0)$ & $E_2(0)$ & $L_2(0)$ & $A_2(0)$  \\ 
			\hline
			6.02&  137.70& 143.72& 6.23 & 86.41 & 92.64 \\
			\hline
		\end{tabular}
		\caption{Assets and liabilities on 30/06/2015 (Bloomberg).}
		\label{data_table}	
	\end{center}
\end{table}
For the calibration we choose 1-year at-the-money, in-the-money, and out-of-the-money equity put options on the banks, and 1-year CDS contracts. Since the spreads of CDS are usually significantly lower than the option prices, we scale them by some weight $w_i$ in the objective function. As a result, we have the following 6-dimensional minimization problem:
\begin{multline}
	\label{calibration_eq}
	\min_{\theta} \{ w_1 (V^{CDS}_1(\theta) - \bar{V}^{CDS}_1)^2 + \sum \limits_{i = 1}^3 (V^{opt}_1(K_{i, 1}, \theta) - \bar{V}^{opt}_1(K_{i, 1}))^2 + \\
	+ w_2 (V^{CDS}_2(\theta) - \bar{V}^{CDS}_2)^2  + \sum \limits_{i = 1}^3 (V^{opt}_2(K_{i, 2}, \theta) - \bar{V}^{opt}_2(K_{i, 2}))^2  \},
\end{multline}
 where $\theta = (\sigma_1, \sigma_2, \lambda_1, \lambda_2, \varsigma_1, \varsigma_2)$, $V^{CDS}_i(\theta)$ is the model CDS spread on the $i$-th bank and $\bar{V}^{CDS}_i$ is the market CDS spread on the $i$-th bank, $V^{opt}_1(K, \theta)$ is the model price of the equity put option on the $i$-th bank with the strike $K$ and $\bar{V}^{opt}_i(K)$ is the market price of the equity put option on the $i$-th bank with strike $K$. Strikes $K_{1, j}, K_{2, j}$, and $K_{3, j}$ are chosen in such a way to take into account the smile. In particular, we choose $K_{1, j} = 1.1 E_j, K_{2, j} = E_j, K_{3, j} = 0.9 E_j$.

In order to find the global minimum of \eqref{calibration_eq} by a Newton-type method, we need to find a good starting point, otherwise an optmization procedure might finish in local minima which are not global minima. To choose the starting point, we calibrate one-dimensional models for each bank without mutual liabilities
\begin{multline}
	\label{calibration_eq1d}
	\min_{\theta_j} \{ w_j (V^{CDS}_j(\theta_j) - \bar{V}^{CDS}_j)^2 + (V^{opt}_j(K_{1, j}, \theta_j) - \bar{V}^{opt}_j(K_{1, j}))^2 + \\
	+(V^{opt}_j(K_{2, j}, \theta_j) - \bar{V}^{opt}_i(K_{2, j}))^2  + (V^{opt}_j(K_{3, j}, \theta_j) - \bar{V}^{opt}_j(K_{3, j}))^2  \},
\end{multline}
where $\theta_j = (\sigma_j, \lambda_j, \varsigma_j)$ for $j = 1, 2$.

The global minima of \eqref{calibration_eq1d} can be found via the {\bf{chebfun toolbox}} (\cite{Trefethen}) that uses Chebyshev polynomials to approximate the function, and then the global minima can be easily found. The calibration results of the one-dimensional model for the first and the second banks are presented in Table \ref{table:params_1d}. We note that the global minima of \eqref{calibration_eq} cannot be found via the chebfun toolbox, since it works with functions up to three variables. There are also more fundamental complexity issues for higher-dimensional tensor product interpolation.

\begin{table}[H]
	\begin{center}
		\begin{tabular}{| c | c | c | c | c | c |}
			\hline
			$\sigma_1$ & $\lambda_1$ & $\varsigma_1$ & $\sigma_2$ & $\lambda_2$ & $\varsigma_2$  \\ 
			\hline
			 0.0117&  0.1001& 0.3661& 0.0154 & 0.0160 & 0.0545\\
			\hline
		\end{tabular}
		\caption{Calibrated parameters of one-dimensional models on 30/06/2015 for $T = 1$.}
		\label{table:params_1d}	
	\end{center}
\end{table}

Similar to \cite{LiptonSepp}, for simplicity, we further assume that 
\begin{equation}
	\lambda_{\{12\}} = \rho \cdot \min(\lambda_1, \lambda_2).
	\label{lambda_assumption}
\end{equation}
 Then, we estimate $\rho$ from historical data. We take one year daily  equity prices $E_i(t)$ by time series (from Bloomberg) and estimate the covariance of asset returns $r_t^i = \frac{\Delta A_i(t)}{A_i(t)}$
\begin{equation}
	\widehat{\cov}(A_1, A_2) = \sum \limits_{i = 1}^n \left(r_{i, 1} - \bar{r_1} \right)\left(r_{i, 2} - \bar{r}_2  \right),
	\label{cov_est}
\end{equation}
where $\bar{r}_1$ and $\bar{r}_2$ are the sample mean of asset returns.

Using \eqref{assets_dynamics}, we can see that \eqref{cov_est} converges to
\begin{equation}
	\widehat{\cov}(A_1, A_2) \underset{n \to +\infty}{\longrightarrow} \sigma_1 \sigma_2 \left( \rho+ \lambda_{\{12\}} /(\varsigma_1 \varsigma_2) \right).
\end{equation}
Using the last equation and \eqref{lambda_assumption}, we can extract the estimated values of $\rho$ and $\lambda_{\{12\}}$. The estimation results are in Table \ref{table:corr_params}.
\begin{table}[H]
	\begin{center}
		\begin{tabular}{| c | c | c | }
			\hline
			& $\rho$ & $\lambda_{\{12\}} $ \\
			\hline
			Estimated value & 0.510 & 0.0188 \\
			\hline
			Confidence interval \footnotemark & (0.500, 0.526)& (0.0182, 0.0194) \\
			\hline
		\end{tabular}
		\caption{Historically estimated correlation coefficients on 30/06/2015 with 1 year window.}
		\label{table:corr_params}	
	\end{center}
\end{table}
\footnotetext{We use a $3 \sigma$ confidence interval.}

Finally, we perform a six-dimensional (constrained) optimization of \eqref{calibration_eq} with the starting point from Table \ref{table:params_1d} and correlation parameters from Table \ref{table:corr_params}. We choose different alternatives of mutual liabilities to have a clear picture how mutual liabilities influence on model parameters. We use the {\bf lsqnonlin} method in Matlab that uses a Trust Region Reflective algorithm \cite{conn2000trust} (with the gradient computed numerically). The model CDS spreads are computed using the method in Section \ref{CDS_pricing}, while equity option prices are computed in the usual finite-difference manner (see \cite{LiptonSepp} for details).  Results are presented in Table \ref{table:params_2d}.
%\begin{table}[H]
%	\begin{center}
%		\begin{tabular}{|c | c | c | c | c | c | c | c |}
%			\hline
%			$L_{12}$ & $L_{21}$ & $\sigma_1$ & $\lambda_1$ & $\varsigma_1$ & $\sigma_2$ & $\lambda_2$ & $\varsigma_2$  \\ 
%			\hline
%			0.0 & 0.0 & 0.0117&  0.1001& 0.3661& 0.0154 & 0.0160 & 0.0545 \\
%			2.0 & 3.0 & 0.0119 & 0.1012 & 0.3968 & 0.0153 & 0.0153 & 0.0517 \\
%			3.0 & 2.0 & 0.0119 & 0.0976 & 0.3841 & 0.0156 & 0.0154 & 0.0522 \\
%			5.0 & 6.0 & 0.0122 & 0.1021 & 0.4233 & 0.0154 & 0.0149 & 0.0491 \\
%			5.0 & 4.0 & 0.0120 & 0.1079 & 0.4212 & 0.0155 & 0.0149 & 0.0496 \\
%			5.0 & 0.0 & 0.0117 & 0.0989 & 0.3627 & 0.0160 & 0.0151 & 0.0527 \\
%			0.0 & 4.0 & 0.0117 & 0.0993 & 0.3796 & 0.0154 & 0.0145 & 0.0522 \\
%			\hline
%		\end{tabular}
%		\caption{Calibrated parameters of two-dimensional model with mutual liabilities on 30/06/2015 for $T = 1$.}
%		\label{table:params_2d}	
%	\end{center}
%\end{table}

\begin{table}[H]
	\begin{center}
		\begin{tabular}{|c | c | c | c | c | c | c | c |}
			\hline
			 Model & $\sigma_1$ & $\lambda_1$ & $\varsigma_1$ & $\sigma_2$ & $\lambda_2$ & $\varsigma_2$  \\ 
			\hline
			With jumps & 0.0122&  0.0950& 0.3958& 0.0160 & 0.0148 & 0.0505 \\
			Without jumps & 0.0206 & -- & -- & 0.0317 & -- & -- \\
			\hline
		\end{tabular}
		\caption{Calibrated parameters of two-dimensional model with mutual liabilities on 30/06/2015 for $T = 1$.}
		\label{table:params_2d}	
	\end{center}
\end{table}

In Table \ref{table:results} we present joint and marginal survival probabilities computed using the equations from Section \ref{section:joint}. From these results, we can conclude that jumps play an important role in the model.
\begin{table}[H]
	\begin{center}
		\begin{tabular}{|c | c | c | c | c | c | c | c |}
			\hline
			Model &Joint s/p & Marginal s/p   \\ 
			\hline
			With jumps & 0.9328 & 0.9666 \\
			Without jumps & 0.9717 & 0.9801 \\
			\hline
		\end{tabular}
		\caption{Joint and marginal survival probabilities for the calibrated models.}
		\label{table:results}	
	\end{center}
\end{table}


%\begin{table}[H]
%	\begin{center}
%		\begin{tabular}{|c | c | c | c | c | c | c | c |}
%			\hline
%			Model &Joint s/p & Marginal s/p   \\ 
%			\hline
%			0.0 & 0.0 & 0.8879 \\
%			2.0 & 3.0 &  0.8869 \\
%			3.0 & 2.0 &  0.8868\\
%			5.0 & 6.0 &  0.8861 \\
%			5.0 & 4.0 &  0.8810\\
%			5.0 & 0.0 &  0.8869\\
%			0.0 & 4.0 &  0.8866\\
%			\hline
%		\end{tabular}
%		\caption{Marginal survival probabilities for the calibrated models.}
%	\end{center}
%\end{table}

%
%K>> resid
%
%resid =
%
%    0.0167
%   -0.2809
%   -0.6476
%    1.0280
%
%K>> result
%
%result =
%
%    0.0180
%    0.0366
%    4.2949

%result =
%
%    0.0119
%    0.1089
%   30.5082
%
%K>> resid
%
%resid =
%
%   -0.0177
%    1.0682
%    0.6890
%   -1.6769


\section{Conclusion}
\label{sec:conclusion}
This paper presents a generic top-$\size$ recommendation framework for  trading-off accuracy, novelty, and coverage. To achieve this, we profile the users according to their preference for long-tail novelty. We examine various measures, and formulate an optimization problem to learn these user preferences from interaction data.  We then integrate the user preference estimates in our generic framework, GANC.  Extensive experiments on several datasets confirm that there are trade-offs between accuracy, coverage, and novelty. Almost all re-ranking models increase coverage and novelty at the cost of accuracy. However, existing re-ranking models typically rely on rating prediction models, and are hence more effective in dense settings. Using a generic approach, we can easily incorporate a suitable base accuracy recommender to devise an effective solution for both sparse and dense settings.  %Our results  also indicate there is no single method that outperforms other methods in all metrics. However our techniques obtain a significant improvement in coverage, while  . 
Although we integrated the  long-tail novelty preference estimates into a re-ranking framework, their use-case is not limited to these frameworks. In  the future, we intend to explore the temporal and topical dynamics of long-tail novelty preference, particularly in settings where contextual information is  available.  
%We achieve these objectives without using any additional contextual information.


\iffalse
While we focused on promoting long-tail items across users, we did not consider diversity of individual top-$\size$ recommendations, a factor that has been shown to positively affect consumer satisfaction. This is one direction for future work. Moreover, the sequential setting  in our work, creates a dependency between different batches, where,  the items recommended to a batch of users, depends on those recommended to previous batches. This dependency is created through the parameter $\mathbf{f}$, that is updated every time a top-$\size$ set  is allocated to a batch of users. A future direction for our work is to estimate a distribution over $\mathbf{f}$ that allows us to independently solve the problem for each user, leading to improvements across all performance metrics, including recommendation time. 

We design algorithms that take advantage of the structure in the value functions to obtain both efficient and scalable solutions. 
We design an algorithm that takes advantage of the structure in the value functions to obtain both efficient and scalable solutions. 

\textcolor{red}{Our  sequential  algorithms can be applied for batch recommendation contexts,~e.g., personalized email marketing, where based on prior interaction data between users and items,  a new round of recommendations must be sent to all users in the system.  However, the independent coverage algorithms lift the sequential setting restrictions and allow it be applied for re-ranking the output of base recommender in any setting. }A future direction for our work is to incorporate explicit diversity metrics in the framework. 
\fi


%We have a presented a submodular maximization framework to systematically trade-off relevance and diversity in recommendations to individual users and coverage across the item-space. This ensures both consumer and producer satisfaction. We model users according to their risk and focusing degrees and promote long-tail items to the right group of consumers. Consequently, we obtain a significant improvement in coverage while maintaining reasonable levels of user satisfaction. Furthermore, our methods are able to achieve a more balanced distribution across the set of recommended items. In the future, we plan to investigate the effect of using alternative base recommender systems. 

%Future Work
%However most of these methods assume that the ratings are missing at random (MAR). Since our method of generating recommendations is based on the completed matrix, assuming MAR might introduce additional bias, we will use methods which assume that the ratings at missing not at random (MNAR),explored in~\cite{steck2010training, icml2014c2_hernandez-lobatob14}. 	 
%Long Tail %Recently, authors in~\cite{cremonesi2010performance} conducted extensive experiments to evaluate the performances of various matrix factorization-based algorithms and neighborhood models on the task of recommending long tail items. Their experimental results show that long tail recommendation leads to a decrease in accuracy for all algorithms. They also showed that for this task, SVD outperforms other state-of-the-art algorithms. 

\newpage
\section{Dataset Visualizations}
\label{sec:app_dataset_visuals}

%%%%%%
%%
%%
\subsection{Examples of each view class}
\newcommand{\BC}{0.33}
\setlength{\tabcolsep}{0.1cm}
\begin{figure}[!h]
\begin{tabular}{c c c c}
    PLAX  & PSAX & OTHER 
    \\
    \includegraphics[width=\BC\textwidth]{figures/small_appendix/Appendix_PLAX1.jpg}
    &
    \includegraphics[width=\BC\textwidth]{figures/small_appendix/Appendix_PSAX1.jpg}
    &
    \includegraphics[width=\BC\textwidth]{figures/small_appendix/Appendix_Other1.jpg}
    &
   
    \\
    
    \includegraphics[width=\BC\textwidth]{figures/small_appendix/Appendix_PLAX2.jpg}
    &
    \includegraphics[width=\BC\textwidth]{figures/small_appendix/Appendix_PSAX2.jpg}
    &
    \includegraphics[width=\BC\textwidth]{figures/small_appendix/Appendix_Other2.jpg}
    &
   
     \\
     
     \includegraphics[width=\BC\textwidth]{figures/small_appendix/Appendix_PLAX3.jpg}
    &
    \includegraphics[width=\BC\textwidth]{figures/small_appendix/Appendix_PSAX3.jpg}
    &
    \includegraphics[width=\BC\textwidth]{figures/small_appendix/Appendix_Other3.jpg}
    &
   
     \\
     
     \includegraphics[width=\BC\textwidth]{figures/small_appendix/Appendix_PLAX4.jpg}
    &
    \includegraphics[width=\BC\textwidth]{figures/small_appendix/Appendix_PSAX4.jpg}
    &
    \includegraphics[width=\BC\textwidth]{figures/small_appendix/Appendix_Other4.jpg}
    &
   
    \end{tabular}	
    \caption{Examples of images for each possible view label in our dataset. \emph{From left to right:} Four examples of peristernal long axis (PLAX) view, four examples of peristernal short axis (PSAX) view, and four examples of other kinds of view in our ``Other'' class. }
    \label{fig:VIEW_SAMPLES_APPENDIX}
\end{figure}

%%%%%%
%%
%%
\newpage
\subsection{Examples of each view for a Severe AS patient}
\newcommand{\BA}{0.33}
\setlength{\tabcolsep}{0.1cm}
\begin{figure}[!h]
\begin{tabular}{c c c c}
    PLAX  & PSAX & OTHER 
    \\
    \includegraphics[width=\BA\textwidth]{figures/small_appendix/SevereAS_11112007_PLAX1.jpg}
    &
    \includegraphics[width=\BA\textwidth]{figures/small_appendix/SevereAS_11112007_PSAX1.jpg}
    &
    \includegraphics[width=\BA\textwidth]{figures/small_appendix/SevereAS_11112007_Other1.jpg}
    &
    
    \\
    
    \includegraphics[width=\BA\textwidth]{figures/small_appendix/SevereAS_11112007_PLAX2.jpg}
    &
    \includegraphics[width=\BA\textwidth]{figures/small_appendix/SevereAS_11112007_PSAX2.jpg}
    &
    \includegraphics[width=\BA\textwidth]{figures/small_appendix/SevereAS_11112007_Other2.jpg}
    &
   
     \\
     
     \includegraphics[width=\BA\textwidth]{figures/small_appendix/SevereAS_11112007_PLAX3.jpg}
    &
    \includegraphics[width=\BA\textwidth]{figures/small_appendix/SevereAS_11112007_PSAX3.jpg}
    &
    \includegraphics[width=\BA\textwidth]{figures/small_appendix/SevereAS_11112007_Other3.jpg}
    &
  
    \end{tabular}	
    \caption{Examples of images from a patient with Severe AS in our dataset. \emph{From left to right:} Three examples of parasternal long axis (PLAX) view, three examples of parasternal short axis (PSAX) view, and three examples of other kinds of view in our ``Other'' class. }
    \label{fig:PatientSevereAS}
\end{figure}


%%%%%%
%%
%%
\newpage
\subsection{Examples of each view for a No AS patient}
\newcommand{\BB}{0.33}
\setlength{\tabcolsep}{0.1cm}
\begin{figure}[!h]
\begin{tabular}{c c c c}
    PLAX  & PSAX & OTHER 
    \\
    \includegraphics[width=\BB\textwidth]{figures/small_appendix/NoAS_1996889_PLAX1.jpg}
    &
    \includegraphics[width=\BB\textwidth]{figures/small_appendix/NoAS_1996889_PSAX1.jpg}
    &
    \includegraphics[width=\BB\textwidth]{figures/small_appendix/NoAS_1996889_Other1.jpg}
    &
    
    \\
    
    \includegraphics[width=\BB\textwidth]{figures/small_appendix/NoAS_1996889_PLAX2.jpg}
    &
    \includegraphics[width=\BB\textwidth]{figures/small_appendix/NoAS_1996889_PSAX2.jpg}
    &
    \includegraphics[width=\BB\textwidth]{figures/small_appendix/NoAS_1996889_Other2.jpg}
    &
   
     \\
     
     \includegraphics[width=\BB\textwidth]{figures/small_appendix/NoAS_1996889_PLAX3.jpg}
    &
    \includegraphics[width=\BB\textwidth]{figures/small_appendix/NoAS_1996889_PSAX3.jpg}
    &
    \includegraphics[width=\BB\textwidth]{figures/small_appendix/NoAS_1996889_Other3.jpg}
    &
  
    \end{tabular}	
    \caption{Examples of images from a patient with No AS in our dataset. \emph{From left to right:} Three examples of parasternal long axis (PLAX) view, three examples of parasternal short axis (PSAX) view, and three examples of other kinds of view in our ``Other'' class. }
    \label{fig:PatientNoAS}
\end{figure}



\newpage 
\section{Further Results}

\subsection{Assessment of ensembling}

Table~\ref{tab:best_single_checkpoint_VS_ensemble_FS_echo260} compares using a single checkpoint (one point estimate of neural network weight vector $\theta$) to using an ensemble of parameters aggregated from the last 25 checkpoints (one per epoch).

\begin{table}[!h]
    \centering
    \begin{tabular}{c|cccc|c}
    \textit{Diagnosis classification} & Split 1  & Split 2 & Split 3 & Split 4 & Average\\
    \hline
    Best single checkpoint  & 61.81 & 59.79 & 56.05 & 64.21 & 60.46\\
    Ensemble  & 62.95 & 61.03 & 56.58 & 63.84 & \textbf{61.13}
	\\ \hline
    \textit{View classification}  &   &  &  &  & 
    \\ \hline
    Best single checkpoint  & 93.03 & 93.24 & 92.39 & 93.79 & 93.11\\
    Ensemble  & 92.37 & 93.24 & 93.72 & 93.87 & \textbf{93.30}\\
    \end{tabular}
    \caption{Comparing best single checkpoint performance with ensemble performance on \textbf{Full-size \datasetName-156-52}}
    \label{tab:best_single_checkpoint_VS_ensemble_FS_echo260}
\end{table}


%%%%%%
%%
%%
\subsection{Patient-level diagnosis performance on bonus heldout set}

Table~\ref{tab:diagnosis classification patient unlabeled_heldout_174} examines the performance of the best labeled-set-only methods and MixMatch methods on the 174 patient studies that have diagnosis but no view labels.
 While the images used here were originally included in the unlabeled training set (which was used to train SSL methods like MixMatch), the diagnosis labels were not provided at all during training time. 
 We thus still believe this is an authentic test of generalization given the scarcity of labeled data available for our task.
 Of course, additional independent evaluation (especially from another institution) is needed.

\begin{table}[!h]
    \centering
    \begin{tabular}{l l l|rrrr|c}
    Pretrain & Method & Voting
    & Split 1  & Split 2 & Split 3 & Split 4 & average\\
    \hline
    & Basic WRN & Simple average & 76.73 & 75.25 & 76.87 & 81.88 & 77.68\\
    & Basic WRN & View-prioritized & 73.63 & 83.21 & 79.70 & 80.08 & 79.18\\
    %SSL & FS & MixMatch & Priority view + confidence & 94.58 & 84.17 & 77.50 & 92.5 & 87.19\\
    \hline
    & MixMatch & Simple average & 85.32 & 76.29 & 74.14 & 79.95 & 78.93\\
    view & MixMatch & Simple average & 83.36 & 77.96 & 75.61 & 81.37 & 79.58\\
    & MixMatch & View-prioritized & 83.27 & 83.76 & 82.34 & 82.83 & \textbf{83.05}\\
    view & MixMatch & View-prioritized & 82.53 & 86.15 & 79.62 & 83.27 & 82.89\\
    %view & MixMatch & LR with view-priority & 80.42 & 84.24 & 76.58 & 80.67 & 80.48\\
    %(MixMatch transfered) + MysteryMethod & NA & NA & NA\\ 
    \end{tabular}
    \caption{Patient-level AS Severity Diagnosis Classification on the \textbf{bonus heldout set} of 174 patients for whom we have diagnosis labels only (no view labels). We show balanced accuracy on models trained on each of the four folds on four \textbf{full-size \datasetName-156-52} dataset.
    }%endcaption
    \label{tab:diagnosis classification patient unlabeled_heldout_174}
\end{table}


%%%%%%
%%
%%
\subsection{Assessment of MixMatch hyperparameter sensitivity}

In Table~\ref{tab:MixMatch hyperparameters ablation study}, we consider four possible strategies for setting the hyperparameters of MixMatch, varying two  key settings for the weight on unlabeled loss $\lambda$. First, we vary whether the final value of $\lambda$ is set to its \emph{best} value among a grid of candidates (based on validation set performance), or \emph{fixed} to a constant.
Second, we vary whether $\lambda$ remains fixed over iterations throughout a training run, or is updated over iterations on a linear ramp schedule from 0 to its final target value. 

From this comparison, we see we consistent gains across splits (average gain across splits of over 1.6\% balanced accuracy) for using a delayed ramp up schedule with target value selected via grid search.

\begin{table}[!h]
    \centering
    \begin{tabular}{l l| rrrr | r}
    Final $\lambda$ value & $\lambda$ update schedule & Split 1  & Split 2 & Split 3 & Split 4 & Average\\
    \hline
    best on val & Delayed ramp-up  & 65.57 & 62.69 & 60.87 & 66.29 & 63.86\\
    best on val & Immediate ramp-up & 65.07 & 61.87 & 60.82 & 65.37 & 63.28\\
    best on val & Constant  & 65.03 & 61.52 & 58.87 & 65.22 & 62.66\\
    100 (fixed) & Constant & 63.94 & 61.79 & 58.87 & 64.35 & 62.24\\
    \end{tabular}
    \caption{Ablation study of different settings of the unlabeled loss weight $\lambda$ for MixMatch. AS severity diagnosis classification for individual images on the \textbf{full-size \datasetName-156-52} dataset. showing balanced accuracy averaged over the test sets from multiple folds (each fold’s test set contains all images from 52 patients). }%endcaption
    \label{tab:MixMatch hyperparameters ablation study}
\end{table}



%%%%%%
%%
%%
\subsection{Assessment of alternative view prioritization strategy using thresholding}


An anonymous reviewer suggested an alternative strategy for prioritizing images of relevant view.
The alternative strategy works as follows: for each image, we compute the predicted probability that the image is a ``relevant view'' (either PLAX and PSAX) by summing the probabilities of each view type.
However, instead of using this raw probability as a weight (as our chosen method does), we use a \emph{cutoff threshold} and simply average the diagnosis predictions of images whose relevant view probability is above the cutoff.
For each patient, we use the majority vote prediction of the diagnosis from the images of relevant views.
The value of the cutoff threshold is selected using the validation set to maximize balanced accuracy.

Table~\ref{tab:Suggested_Aggregation_Ablation} shows the performance of this strategy (``threshold-then-average'') on the full-size dataset.
Using this alternative prioritization strategy together with our suggested methodology for patient-level diagnosis (using MixMatch, pretraining on view), we find the average test set balanced accuracy is around 85.8\%, while the weighted average strategy in the main paper achieves over 90\% balanced accuracy. We take this as reasonably decisive evidence that a weighted average (rather than a simple cutoff) should be preferred.

\begin{table}[!h]
    \centering
    \begin{tabular}{l l l|rrrr|c}
    Pretrain & Method & Aggregation across images
    & Split 1  & Split 2 & Split 3 & Split 4 & average\\
    \hline
    & Basic WRN & Threshold-then-Average & 85.42 & 86.25 & 79.17 & 92.50 & 85.84 \\
    %SSL & FS & MixMatch & Priority view + confidence & 94.58 & 84.17 & 77.50 & 92.5 & 87.19\\
    & MixMatch & Threshold-then-Average & 83.33 & 84.17 & 77.50 & 94.58 & 84.90 \\
    view & MixMatch & Threshold-then-Averagen & 86.67 & 80.00 & 82.50 & 94.17 & 85.84\\
    %view & MixMatch & LR with view-priority & 87.08 & 82.08 & 85.00 & 88.75 & 85.73\\
    %(MixMatch transfered) + MysteryMethod & NA & NA & NA\\ 
    \end{tabular}
    \caption{Alternative view-prioritizing strategy for patient-level AS severity diagnosis classification on the \textbf{full-size \datasetName-156-52} dataset, showing balanced accuracy on the test set across multiple folds (each fold’s test set contains 52 patients).}
    %endcaption
    \label{tab:Suggested_Aggregation_Ablation}
\end{table}



%%%%%%
%%
%%
\subsection{ROC Curve of patient-level diagnosis: no AS vs. mild/moderate/severe AS}

Fig.~\ref{fig: No AS vs Some AS} shows receiver operating curves for several methods for the task of distinguishing no AS vs Some AS (which aggregates both the mild/moderate and severe levels in the 3-level diagnosis task of the main paper).

\begin{figure}[!h]
\begin{tabular}{c c}
	\includegraphics[width=0.43\textwidth]{figures/fold0_multitask_PatientLevel_NoVSSome_NormalizedPriorityStrategyClassProbabilityScore.pdf}
	&
    \includegraphics[width=0.43\textwidth]{figures/fold1_multitask_PatientLevel_NoVSSome_NormalizedPriorityStrategyClassProbabilityScore.pdf}
	\\
	(a) Split 1 & (b) Split 2
	\\
	\includegraphics[width=0.43\textwidth]{figures/fold2_multitask_PatientLevel_NoVSSome_NormalizedPriorityStrategyClassProbabilityScore.pdf}
	&
    \includegraphics[width=0.43\textwidth]{figures/fold3_multitask_PatientLevel_NoVSSome_NormalizedPriorityStrategyClassProbabilityScore.pdf}
	\\
	(c) Split 3 & (d) Split 4
\end{tabular}
    
\caption{ROC curves for binary diagnosis task (no AS vs ``mild/moderate/severe AS'') on \textbf{full-size \datasetName-156-52}.
    }%endcaption
    \label{fig: No AS vs Some AS}
\end{figure}

\section{Methodological Details}

\subsection{Image processing details}
\label{sec:removing_doppler}

\paragraph{Removing doppler images.}
In the raw data of all imagery available for an echocardiogram study, 
we obtained TIFF files that represent both cineloops and Doppler images.

We verified in our labeled set that all Doppler images have one of the following landscape aspect ratio $(831, 323)$, $(901, 384)$, $(901, 390)$, $(704, 305)$, $(831, 421)$, $(901, 469)$ or $(563, 294)$. Only the Dopplers have these aspect ratios. We thus filtered out Doppler completely via these aspect ratios. 

\paragraph{Downsizing}
The original images are provided as high-resolution TIFF format images (hundreds of pixels per side) of varying aspect ratios. Generally, we can expect that both view and diagnosis classifiers would perform better given higher-resolution input (and holding other factors the same). The main trade-off of processing higher-resolution images is increased runtime and memory requirements. In our preliminary experiments, we compared downsizing all images to a standard square aspect ratio at 3 possible sizes: 32x32, 64x64 and 128x128. We found that 64x64 achieves a good balance between model performance and computation cost. 
A prior study by \citet{madaniDeepEchocardiographyDataefficient2018} provides a more extensive study of optimal resolution size. The interested reader can refer to their work for more details. 


\subsection{Architecture Settings and Hyperparameters}
\label{sec:arch_and_hyperparameters}

\paragraph{Weighted cross-entropy for labeled loss}
To counteract the effect of class imbalance in the dataset, we use weighted cross-entropy for the labeled loss. For an input image $x$ whose true label $y$ indicates it belongs to class $c$, the weighted cross-entropy assumes the following form:
\begin{align}
\mathcal{L}^L(\theta, x) = - w_{c} \log \hat{p}_{c}(\theta, x),
\end{align}
where $\hat{p}_{c}$ is the predicted probability of class $c$. The weight $w_{c}$ is calculated using the training set statistics as follow:
\begin{align}
w_{c} = \frac{\prod_{k\neq c}{N_{k}}}{\sum_{j}\prod_{k \neq j}{N_{k}}}
\end{align}
where $N_{k}$ is the number of images of class $k$ in the training set.

\paragraph{Common architecture.}
Following~\citet{oliverRealisticEvaluationDeep2018}, for all considered methods, we use the \emph{same} backbone neural network architecture: a wide residual network~\citep{zagoruykoWideResidualNetworks2017} with 28 layers (WRN-28), which has total of 5,931,683 parameters.
This same network architecture is used in the original MixMatch evaluation~\citep{berthelotMixmatchHolisticApproach2019} with promising results.

\paragraph{Common training protocol.}
All SSL methods we consider follow the loss minimization framework with two primary losses (one for ``labeled'' data and one for ``unlabeled'' data) in Eq.~\eqref{eq:standard-SSL-loss-template}.
We allow every method to train for 32 epochs (where each epoch processes $2^{16}$ images, as in \citet{berthelotMixmatchHolisticApproach2019}).
Our preliminary experiments suggest that after 30 epochs all methods effectively converge in terms of validation balanced accuracy. 

\paragraph{Common regularization.}
For all methods, we expect performance will be vulnerable to overfitting, so we impose an L2-norm penalty on the weights $\theta$, also known as weight decay. Each method selects its preferred value of this penalty strength hyperparameter. We searched values in [0.0002, 0.002, 0.02].

\paragraph{Common optimization.}
We use ADAM \citep{kingma2014adam} to optimize each model.
Each method selects the value of the step size (learning rate) as a hyperparameter. We experimented with 0.002 and 0.0007
%HZ: 'performance being sensitive to learning rate' is very reasonable. But we don't have an ablation to back it. 
%We find performance is sensitive to the step size (learning rate) hyperparameter, so we perform a grid search and select the value that maximizes balanced accuracy on the validation set.

\paragraph{Hyperparameters for Pseudo-Label.}
Beyond the usual hyperparameters for our loss-minimization SSL framework, another important hyperparameter for pseudo-label is the threshold $\tau$. We find that performance is not very sensitive to the chosen $\tau$ value as long as it is within a certain range. We set $\tau$ to 0.95, as done in past literature that evaluates Pseudo-Label as an SSL method ~\citep{oliverRealisticEvaluationDeep2018,berthelotMixmatchHolisticApproach2019, berthelotRemixmatchSemisupervisedLearning2019, sohnFixmatchSimplifyingSemisupervised2020}.


\paragraph{Hyperparameters for VAT.}
Beyond the usual hyperparameters for our SSL framework, for VAT we need to select a value for $\epsilon$.
In \citet{miyatoVirtualAdversarialTraining2019}, the authors claimed that they can achieve superior performance by tuning only $\epsilon$ and fixing $\lambda$ to 1. In our experiment, we used the default $\lambda$ as in \cite{berthelotMixmatchHolisticApproach2019} and searched the value of $\epsilon$ in [2, 6, 18], together with learning rate and weight decay. We select the best hyperparameters using validation set performance. 


\paragraph{Hyperparameters for MixMatch.}
Beyond the usual hyperparameters for our SSL framework, the key hyperparameters for MixMatch include the number of augmentations $K$, the temperature $T>0$ used for sharpening, interpolation hyperparameter $\alpha$ and unlabeled loss coefficient $\lambda$. We set $K=2$, $T=0.5$, and $\alpha=0.75$ as done in \citet{berthelotMixmatchHolisticApproach2019}, and search for $\lambda$ in the range [10, 30, 75, 100, 130] using validation set. 

\paragraph{Hyperparameters for Multitask training.}
We searched $\gamma$, the hyperparameter that control the strength of the auxilliary view loss in Eq.~\eqref{eq:multitask}, in the range [10, 3, 1, 0.3, 0.1]. The best $\alpha$ is selected together with other hyperparameters on validation set. 

\bibliographystyle{apalike}
\bibliography{lit_bib} 



\end{document}
