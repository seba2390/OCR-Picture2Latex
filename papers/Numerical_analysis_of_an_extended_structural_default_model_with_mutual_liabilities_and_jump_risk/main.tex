\documentclass[a4paper,11pt]{article}

\usepackage{cmap}     
\usepackage{amsmath,amssymb}
\usepackage[utf8]{inputenc}
\usepackage[T2A]{fontenc}  
\usepackage{graphicx}    
\usepackage[margin=1in]{geometry}
\usepackage{fancyhdr}    
\usepackage{setspace}
\usepackage{wrapfig}
\usepackage{subfig}
\usepackage{listings}
\usepackage{color}
\usepackage{setspace}
\usepackage{textcomp}
\usepackage{float}
\usepackage{listings}
\usepackage{amsthm}
\usepackage{algorithm}
\usepackage{algpseudocode}
\usepackage{afterpage}
\usepackage[toc,page]{appendix}
\usepackage{natbib}
\usepackage{bbm}
\usepackage{footnote}

\makesavenoteenv{tabular}

\newtheorem{theorem}{Theorem}
\newtheorem{lemma}{Lemma}
\newtheorem{corollary}{Corollary}
\newtheorem{definition}{Definition}

%\doublespacing

\DeclareMathOperator{\cov}{cov}
\DeclareMathOperator{\diag}{diag}
\DeclareMathOperator{\fft}{fft}
\DeclareMathOperator{\ifft}{ifft}
\DeclareMathOperator{\Area}{Area}
\DeclareMathOperator{\inter}{int}
\DeclareMathOperator{\rank}{rank}
\DeclareMathOperator{\vect}{vec}
\DeclareMathOperator{\corr}{corr}

\begin{document}    
\title{Numerical analysis of an extended structural default model with mutual liabilities and jump risk} 
\author{Vadim Kaushansky\thanks{The first author gratefully acknowledges support from the Economic and Social Research Council and Bank of America Merrill Lynch}
\footnote{Mathematical Institute \& Oxford-Man Institute, University of Oxford, UK, E-mail: vadim.kaushansky@maths.ox.ac.uk},  Alexander Lipton\footnote{Massachusetts Institute of Technology, Connection Science, Cambridge, MA, USA, E-mail: alexlipt@mit.edu}, Christoph Reisinger\footnote{Mathematical Institute  \& Oxford-Man Institute, University of Oxford, UK, E-mail: christoph.reisinger@maths.ox.ac.uk}}    
\date{}
   
\maketitle 
\begin{abstract}
We consider a structural default model in an interconnected banking network as in \cite{Lipton2015}, with mutual obligations between each pair of banks. We analyse the model numerically for two banks with jumps in their asset value processes. Specifically, we develop a finite difference method for the resulting two-dimensional partial integro-differential equation, and study its stability and consistency. We then compute joint and marginal survival probabilities, as well as prices of credit default swaps (CDS), first-to-default swaps (FTD), credit and debt value adjustments (CVA and DVA). Finally, we calibrate the model to market data and assess the impact of jump risk.
\end{abstract}

\noindent
{\bf Keywords:} structural default model; mutual liabilities; jump-diffusion; finite-difference and splitting methods; calibration. \\
\medskip

%\noindent
%{\bf Highlights:}
%%The novel results of this paper are as follows:
%\begin{itemize}
%\item We analyze a two-dimensional structural default model with interbank liabilities and negative exponential jumps; in particular, we calibrate the model to the market and analyze the impact of jumps on joint and marginal survival probabilities; 
%\item we develop a new finite-difference method to solve the multidimensional PIDE, which is of second order consistent in both time and space variables; 
%\item we prove the von Neumann and $l_2$ stability of the method; % extending \cite{intHoutStability} from PDEs to PIDEs;
%%to our knowledge, this is the first result on stability of a splitting scheme for this type of multi-dimensional PIDE taking into account Dirichlet boundary conditions;
%\item we demonstrate empirically that in the presence of discontinuous terminal and boundary conditions, second order of convergence can be maintained by local averaging of the data and suitable refinement of the timestep close to maturity.
%\end{itemize}


\IEEEraisesectionheading{\section{Introduction}}

\IEEEPARstart{V}{ision} system is studied in orthogonal disciplines spanning from neurophysiology and psychophysics to computer science all with uniform objective: understand the vision system and develop it into an integrated theory of vision. In general, vision or visual perception is the ability of information acquisition from environment, and it's interpretation. According to Gestalt theory, visual elements are perceived as patterns of wholes rather than the sum of constituent parts~\cite{koffka2013principles}. The Gestalt theory through \textit{emergence}, \textit{invariance}, \textit{multistability}, and \textit{reification} properties (aka Gestalt principles), describes how vision recognizes an object as a \textit{whole} from constituent parts. There is an increasing interested to model the cognitive aptitude of visual perception; however, the process is challenging. In the following, a challenge (as an example) per object and motion perception is discussed. 



\subsection{Why do things look as they do?}
In addition to Gestalt principles, an object is characterized with its spatial parameters and material properties. Despite of the novel approaches proposed for material recognition (e.g.,~\cite{sharan2013recognizing}), objects tend to get the attention. Leveraging on an object's spatial properties, material, illumination, and background; the mapping from real world 3D patterns (distal stimulus) to 2D patterns onto retina (proximal stimulus) is many-to-one non-uniquely-invertible mapping~\cite{dicarlo2007untangling,horn1986robot}. There have been novel biology-driven studies for constructing computational models to emulate anatomy and physiology of the brain for real world object recognition (e.g.,~\cite{lowe2004distinctive,serre2007robust,zhang2006svm}), and some studies lead to impressive accuracy. For instance, testing such computational models on gold standard controlled shape sets such as Caltech101 and Caltech256, some methods resulted $<$60\% true-positives~\cite{zhang2006svm,lazebnik2006beyond,mutch2006multiclass,wang2006using}. However, Pinto et al.~\cite{pinto2008real} raised a caution against the pervasiveness of such shape sets by highlighting the unsystematic variations in objects features such as spatial aspects, both between and within object categories. For instance, using a V1-like model (a neuroscientist's null model) with two categories of systematically variant objects, a rapid derogate of performance to 50\% (chance level) is observed~\cite{zhang2006svm}. This observation accentuates the challenges that the infinite number of 2D shapes casted on retina from 3D objects introduces to object recognition. 

Material recognition of an object requires in-depth features to be determined. A mineralogist may describe the luster (i.e., optical quality of the surface) with a vocabulary like greasy, pearly, vitreous, resinous or submetallic; he may describe rocks and minerals with their typical forms such as acicular, dendritic, porous, nodular, or oolitic. We perceive materials from early age even though many of us lack such a rich visual vocabulary as formalized as the mineralogists~\cite{adelson2001seeing}. However, methodizing material perception can be far from trivial. For instance, consider a chrome sphere with every pixel having a correspondence in the environment; hence, the material of the sphere is hidden and shall be inferred implicitly~\cite{shafer2000color,adelson2001seeing}. Therefore, considering object material, object recognition requires surface reflectance, various light sources, and observer's point-of-view to be taken into consideration.


\subsection{What went where?}
Motion is an important aspect in interpreting the interaction with subjects, making the visual perception of movement a critical cognitive ability that helps us with complex tasks such as discriminating moving objects from background, or depth perception by motion parallax. Cognitive susceptibility enables the inference of 2D/3D motion from a sequence of 2D shapes (e.g., movies~\cite{niyogi1994analyzing,little1998recognizing,hayfron2003automatic}), or from a single image frame (e.g., the pose of an athlete runner~\cite{wang2013learning,ramanan2006learning}). However, its challenging to model the susceptibility because of many-to-one relation between distal and proximal stimulus, which makes the local measurements of proximal stimulus inadequate to reason the proper global interpretation. One of the various challenges is called \textit{motion correspondence problem}~\cite{attneave1974apparent,ullman1979interpretation,ramachandran1986perception,dawson1991and}, which refers to recognition of any individual component of proximal stimulus in frame-1 and another component in frame-2 as constituting different glimpses of the same moving component. If one-to-one mapping is intended, $n!$ correspondence matches between $n$ components of two frames exist, which is increased to $2^n$  for one-to-any mappings. To address the challenge, Ullman~\cite{ullman1979interpretation} proposed a method based on nearest neighbor principle, and Dawson~\cite{dawson1991and} introduced an auto associative network model. Dawson's network model~\cite{dawson1991and} iteratively modifies the activation pattern of local measurements to achieve a stable global interpretation. In general, his model applies three constraints as it follows
\begin{inlinelist}
	\item \textit{nearest neighbor principle} (shorter motion correspondence matches are assigned lower costs)
	\item \textit{relative velocity principle} (differences between two motion correspondence matches)
	\item \textit{element integrity principle} (physical coherence of surfaces)
\end{inlinelist}.
According to experimental evaluations (e.g.,~\cite{ullman1979interpretation,ramachandran1986perception,cutting1982minimum}), these three constraints are the aspects of how human visual system solves the motion correspondence problem. Eom et al.~\cite{eom2012heuristic} tackled the motion correspondence problem by considering the relative velocity and the element integrity principles. They studied one-to-any mapping between elements of corresponding fuzzy clusters of two consecutive frames. They have obtained a ranked list of all possible mappings by performing a state-space search. 



\subsection{How a stimuli is recognized in the environment?}

Human subjects are often able to recognize a 3D object from its 2D projections in different orientations~\cite{bartoshuk1960mental}. A common hypothesis for this \textit{spatial ability} is that, an object is represented in memory in its canonical orientation, and a \textit{mental rotation} transformation is applied on the input image, and the transformed image is compared with the object in its canonical orientation~\cite{bartoshuk1960mental}. The time to determine whether two projections portray the same 3D object
\begin{inlinelist}
	\item increase linearly with respect to the angular disparity~\cite{bartoshuk1960mental,cooperau1973time,cooper1976demonstration}
	\item is independent from the complexity of the 3D object~\cite{cooper1973chronometric}
\end{inlinelist}.
Shepard and Metzler~\cite{shepard1971mental} interpreted this finding as it follows: \textit{human subjects mentally rotate one portray at a constant speed until it is aligned with the other portray.}



\subsection{State of the Art}

The linear mapping transformation determination between two objects is generalized as determining optimal linear transformation matrix for a set of observed vectors, which is first proposed by Grace Wahba in 1965~\cite{wahba1965least} as it follows. 
\textit{Given two sets of $n$ points $\{v_1, v_2, \dots v_n\}$, and $\{v_1^*, v_2^* \dots v_n^*\}$, where $n \geq 2$, find the rotation matrix $M$ (i.e., the orthogonal matrix with determinant +1) which brings the first set into the best least squares coincidence with the second. That is, find $M$ matrix which minimizes}
\begin{equation}
	\sum_{j=1}^{n} \vert v_j^* - Mv_j \vert^2
\end{equation}

Multiple solutions for the \textit{Wahba's problem} have been published, such as Paul Davenport's q-method. Some notable algorithms after Davenport's q-method were published; of that QUaternion ESTimator (QU\-EST)~\cite{shuster2012three}, Fast Optimal Attitude Matrix \-(FOAM)~\cite{markley1993attitude} and Slower Optimal Matrix Algorithm (SOMA)~\cite{markley1993attitude}, and singular value decomposition (SVD) based algorithms, such as Markley’s SVD-based method~\cite{markley1988attitude}. 

In statistical shape analysis, the linear mapping transformation determination challenge is studied as Procrustes problem. Procrustes analysis finds a transformation matrix that maps two input shapes closest possible on each other. Solutions for Procrustes problem are reviewed in~\cite{gower2004procrustes,viklands2006algorithms}. For orthogonal Procrustes problem, Wolfgang Kabsch proposed a SVD-based method~\cite{kabsch1976solution} by minimizing the root mean squared deviation of two input sets when the determinant of rotation matrix is $1$. In addition to Kabsch’s partial Procrustes superimposition (covers translation and rotation), other full Procrustes superimpositions (covers translation, uniform scaling, rotation/reflection) have been proposed~\cite{gower2004procrustes,viklands2006algorithms}. The determination of optimal linear mapping transformation matrix using different approaches of Procrustes analysis has wide range of applications, spanning from forging human hand mimics in anthropomorphic robotic hand~\cite{xu2012design}, to the assessment of two-dimensional perimeter spread models such as fire~\cite{duff2012procrustes}, and the analysis of MRI scans in brain morphology studies~\cite{martin2013correlation}.

\subsection{Our Contribution}

The present study methodizes the aforementioned mentioned cognitive susceptibilities into a cognitive-driven linear mapping transformation determination algorithm. The method leverages on mental rotation cognitive stages~\cite{johnson1990speed} which are defined as it follows
\begin{inlinelist}
	\item a mental image of the object is created
	\item object is mentally rotated until a comparison is made
	\item objects are assessed whether they are the same
	\item the decision is reported
\end{inlinelist}.
Accordingly, the proposed method creates hierarchical abstractions of shapes~\cite{greene2009briefest} with increasing level of details~\cite{konkle2010scene}. The abstractions are presented in a vector space. A graph of linear transformations is created by circular-shift permutations (i.e., rotation superimposition) of vectors. The graph is then hierarchically traversed for closest mapping linear transformation determination. 

Despite of numerous novel algorithms to calculate linear mapping transformation, such as those proposed for Procrustes analysis, the novelty of the presented method is being a cognitive-driven approach. This method augments promising discoveries on motion/object perception into a linear mapping transformation determination algorithm.



Online convex optimization with memory has emerged as an important and challenging area with a wide array of applications, see \citep{lin2012online,anava2015online,chen2018smoothed,goel2019beyond,agarwal2019online,bubeck2019competitively} and the references therein.  Many results in this area have focused on the case of online optimization with switching costs (movement costs), a form of one-step memory, e.g., \citep{chen2018smoothed,goel2019beyond,bubeck2019competitively}, though some papers have focused on more general forms of memory, e.g., \citep{anava2015online,agarwal2019online}. In this paper we, for the first time, study the impact of feedback delay and nonlinear switching cost in online optimization with switching costs. 

An instance consists of a convex action set $\mathcal{K}\subset\mathbb{R}^d$, an initial point $y_0\in\mathcal{K}$, a sequence of non-negative convex cost functions $f_1,\cdots,f_T:\mathbb{R}^d\to\mathbb{R}_{\ge0}$, and a switching cost $c:\mathbb{R}^{d\times(p+1)}\to\mathbb{R}_{\ge0}$. To incorporate feedback delay, we consider a situation where the online learner only knows the geometry of the hitting cost function at each round, i.e., $f_t$, but that the minimizer of $f_t$ is revealed only after a delay of $k$ steps, i.e., at time $t+k$.  This captures practical scenarios where the form of the loss function or tracking function is known by the online learner, but the target moves over time and measurement lag means that the position of the target is not known until some time after an action must be taken. 
To incorporate nonlinear (and potentially nonconvex) switching costs, we consider the addition of a known nonlinear function $\delta$ from $\mathbb{R}^{d\times p}$ to $\mathbb{R}^d$ to the structured memory model introduced previously.  Specifically, we have
\begin{align}
c(y_{t:t-p}) = \frac{1}{2}\|y_t-\delta(y_{t-1:t-p})\|^2,    \label{e.newswitching}
\end{align}
where we use $y_{i:j}$ to denote either $\{y_i, y_{i+1}, \cdots, y_j\}$ if $i\leq j$, or  $\{y_i, y_{i-1}, \cdots, y_j\}$ if $i > j$ throughout the paper. Additionally, we use $\|\cdot\|$ to denote the 2-norm of a vector or the spectral norm of a matrix.

In summary, we consider an online agent that interacts with the environment as follows:
% \begin{inparaenum}[(i)] 
\begin{enumerate}%[leftmargin=*]
    \item The adversary reveals a function $h_t$, which is the geometry of the $t^\mathrm{th}$ hitting cost, and a point $v_{t-k}$, which is the minimizer of the $(t-k)^\mathrm{th}$ hitting cost. Assume that $h_t$ is $m$-strongly convex and $l$-strongly smooth, and that $\arg\min_y h_t(y)=0$.
    \item The online learner picks $y_t$ as its decision point at time step $t$ after observing $h_t,$  $v_{t-k}$.
    \item The adversary picks the minimizer of the hitting cost at time step $t$: $v_t$. 
    \item The learner pays hitting cost $f_t(y_t)=h_t(y_t-v_t)$ and switching cost $c(y_{t:t-p})$ of the form \eqref{e.newswitching}.
\end{enumerate}

The goal of the online learner is to minimize the total cost incurred over $T$ time steps, $cost(ALG)=\sum_{t=1}^Tf_t(y_t)+c(y_{t:t-p})$, with the goal of (nearly) matching the performance of the offline optimal algorithm with the optimal cost $cost(OPT)$. The performance metric used to evaluate an algorithm is typically the \textit{competitive ratio} because the goal is to learn in an environment that is changing dynamically and is potentially adversarial. Formally, the competitive ratio (CR) of the online algorithm is defined as the worst-case ratio between the total cost incurred by the online learner and the offline optimal cost: $CR(ALG)=\sup_{f_{1:T}}\frac{cost(ALG)}{cost(OPT)}$.

It is important to emphasize that the online learner decides $y_t$ based on the knowledge of the previous decisions $y_1\cdots y_{t-1}$, the geometry of cost functions $h_1\cdots h_t$, and the delayed feedback on the minimizer $v_1\cdots v_{t-k}$. Thus, the learner has perfect knowledge of cost functions $f_1\cdots f_{t-k}$, but incomplete knowledge of $f_{t-k+1}\cdots f_t$ (recall that $f_t(y)=h_t(y-v_t)$).

Both feedback delay and nonlinear switching cost add considerable difficulty for the online learner compared to versions of online optimization studied previously. Delay hides crucial information from the online learner and so makes adaptation to changes in the environment more challenging. As the learner makes decisions it is unaware of the true cost it is experiencing, and thus it is difficult to track the optimal solution. This is magnified by the fact that nonlinear switching costs increase the dependency of the variables on each other. It further stresses the influence of the delay, because an inaccurate estimation on the unknown data, potentially magnifying the mistakes of the learner. 

The impact of feedback delay has been studied previously in online learning settings without switching costs, with a focus on regret, e.g., \citep{joulani2013online,shamir2017online}.  However, in settings with switching costs the impact of delay is magnified since delay may lead to not only more hitting cost in individual rounds, but significantly larger switching costs since the arrival of delayed information may trigger a very large chance in action.  To the best of our knowledge, we give the first competitive ratio for delayed feedback in online optimization with switching costs. 

We illustrate a concrete example application of our setting in the following.

\begin{example}[Drone tracking problem]
\label{example:drone} \emph{
Consider a drone with vertical speed $y_t\in\mathbb{R}$. The goal of the drone is to track a sequence of desired speeds $y^d_1,\cdots,y^d_T$ with the following tracking cost:}
\begin{equation}
    \sum_{t=1}^T \frac{1}{2}(y_t-y^d_t)^2 + \frac{1}{2}(y_t-y_{t-1}+g(y_{t-1}))^2,
\end{equation}
\emph{where $g(y_{t-1})$ accounts for the gravity and the aerodynamic drag. One example is $g(y)=C_1+C_2\cdot|y|\cdot y$ where $C_1,C_2>0$ are two constants~\cite{shi2019neural}. Note that the desired speed $y_t^d$ is typically sent from a remote computer/server. Due to the communication delay, at time step $t$ the drone only knows $y_1^d,\cdots,y_{t-k}^d$.}

\emph{This example is beyond the scope of existing results in online optimization, e.g.,~\cite{shi2020online,goel2019beyond,goel2019online}, because of (i) the $k$-step delay in the hitting cost $\frac{1}{2}(y_t-y_t^d)$ and (ii) the nonlinearity in the switching cost $\frac{1}{2}(y_t-y_{t-1}+g(y_{t-1}))^2$ with respective to $y_{t-1}$. However, in this paper, because we directly incorporate the effect of delay and nonlinearity in the algorithm design, our algorithms immediately provide constant-competitive policies for this setting.}
\end{example}


\subsection{Related Work}
This paper contributes to the growing literature on online convex optimization with memory.  
Initial results in this area focused on developing constant-competitive algorithms for the special case of 1-step memory, a.k.a., the Smoothed Online Convex Optimization (SOCO) problem, e.g., \citep{chen2018smoothed,goel2019beyond}. In that setting, \citep{chen2018smoothed} was the first to develop a constant, dimension-free competitive algorithm for high-dimensional problems.  The proposed algorithm, Online Balanced Descent (OBD), achieves a competitive ratio of $3+O(1/\beta)$ when cost functions are $\beta$-locally polyhedral.  This result was improved by \citep{goel2019beyond}, which proposed two new algorithms, Greedy OBD and Regularized OBD (ROBD), that both achieve $1+O(m^{-1/2})$ competitive ratios for $m$-strongly convex cost functions.  Recently, \citep{shi2020online} gave the first competitive analysis that holds beyond one step of memory.  It holds for a form of structured memory where the switching cost is linear:
$
    c(y_{t:t-p})=\frac{1}{2}\|y_t-\sum_{i=1}^pC_iy_{t-i}\|^2,
$
with known $C_i\in\mathbb{R}^{d\times d}$, $i=1,\cdots,p$. If the memory length $p = 1$ and $C_1$ is an identity matrix, this is equivalent to SOCO. In this setting, \citep{shi2020online} shows that ROBD has a competitive ratio of 
\begin{align}
    \frac{1}{2}\left( 1 + \frac{\alpha^2 - 1}{m} + \sqrt{\Big( 1 + \frac{\alpha^2 - 1}{m}\Big)^2 + \frac{4}{m}} \right),
\end{align}
when hitting costs are $m$-strongly convex and $\alpha=\sum_{i=1}^p\|C_i\|$. 


Prior to this paper, competitive algorithms for online optimization have nearly always assumed that the online learner acts \emph{after} observing the cost function in the current round, i.e., have zero delay.  The only exception is \citep{shi2020online}, which considered the case where the learner must act before observing the cost function, i.e., a one-step delay.  Even that small addition of delay requires a significant modification to the algorithm (from ROBD to Optimistic ROBD) and analysis compared to previous work. 

As the above highlights, there is no previous work that addresses either the setting of nonlinear switching costs nor the setting of multi-step delay. However, the prior work highlights that ROBD is a promising algorithmic framework and our work in this paper extends the ROBD framework in order to address the challenges of delay and non-linear switching costs. Given its importance to our work, we describe the workings of ROBD in detail in Algorithm~\ref{robd}. 

\begin{algorithm}[t!]
  \caption{ROBD \citep{goel2019beyond}}
  \label{robd}
\begin{algorithmic}[1]
  \STATE {\bfseries Parameter:} $\lambda_1\ge0,\lambda_2\ge0$
  \FOR{$t=1$ {\bfseries to} $T$}
  \STATE {\bfseries Input:} Hitting cost function $f_t$, previous decision points $y_{t-p:t-1}$
  \STATE $v_t\leftarrow\arg\min_yf_t(y)$
  \STATE $y_t\leftarrow\arg\min_yf_t(y)+\lambda_1c(y,y_{t-1:t-p})+\frac{\lambda_2}{2}\|y-v_t\|^2_2$
  \STATE {\bfseries Output:} $y_t$
  \ENDFOR
   
\end{algorithmic}
\end{algorithm}

Another line of literature that this paper contributes to is the growing understanding of the connection between online optimization and adaptive control. The reduction from adaptive control to online optimization with memory was first studied in \citep{agarwal2019online} to obtain a sublinear static regret guarantee against the best linear state-feedback controller, where the approach is to consider a disturbance-action policy class with some fixed horizon.  Many follow-up works adopt similar reduction techniques \citep{agarwal2019logarithmic, brukhim2020online, gradu2020adaptive}. A different reduction approach using control canonical form is proposed by \citep{li2019online} and further exploited by \citep{shi2020online}. Our work falls into this category.  The most general results so far focus on Input-Disturbed Squared Regulators, which can be reduced to online convex optimization with structured memory (without delay or nonlinear switching costs).  As we show in \Cref{Control}, the addition of delay and nonlinear switching costs leads to a significant extension of the generality of control models that can be reduced to online optimization. 

\section{Numerical scheme}

We shall solve the PIDE \eqref{kolm_1}--\eqref{kolm_2} numerically with an Alternating Direction Implicit (ADI) method. The scheme is a modification of \cite{LiptonSepp} that is unconditionally stable and has second order of convergence in both time and space step.

In order to deal with a forward equation instead of a backward equation, we change the time variable to $\tau = T - t$, so that
\begin{equation}
	\label{pide_forward}
	\begin{aligned}
		& \frac{\partial V}{\partial \tau} = \mathcal{L} V(\tau, x_1, x_2) - \chi(\tau, x_1, x_2), \\
		& V(\tau, x_1, 0) = \phi_{0, 1}(\tau, x_1), \quad V(\tau, 0, x_2) =  \phi_{0, 2}(\tau, x_2), \\
		& V(\tau, x_1, x_2)  \underset{x_2 \to +\infty}{\longrightarrow} \phi_{\infty, 1}(\tau, x_1), \quad V(\tau, x_2, x_2)  \underset{x_1 \to +\infty}{\longrightarrow}  \phi_{\infty, 2}(\tau, x_2), \\
		& V(0, x_1, x_2) = \psi(x_1, x_2).
	\end{aligned}
\end{equation}

We consider the same grid for integral and differential part of the equation
\begin{equation}
	\begin{aligned}
		0 = x_1^0 < x_1^1 < \ldots < x_1^{m_1}, \\
		0 = x_2^0 < x_2^1 < \ldots < x_2^{m_2},
	\end{aligned}
\end{equation}
where $x_1^{m_1}$ and $x_2^{m_2}$ are large positive numbers.

The grid is non-uniform, and is chosen such that relatively many points lie near the default boundaries for better precision. We use a method similar to \cite{itkin2011jumps} to construct the grid.
\subsection{Discretization of the integral part of the PIDE}
In this section, we shall show how to deal with the integral part of the PIDE, and develop an iterative algorithm for the fast computation of the integral operator on the grid. To this end, we outline the scheme from \cite{LiptonSepp} and then give a new method.

The first approach is to deal with the integral operators directly. After the approximation of the integral, we get (\cite{LiptonSepp})
\begin{align}
	&\mathcal{J}_1 V(x_1 + h, x_2) = e^{-\varsigma_1 h} \mathcal{J}_1 V(x_1, x_2) +  \omega_0(\varsigma_1, h) V(x_1, x_2) + \omega_1(\varsigma_1, h) V(x_1 + h, x_2) + O(h^3), \label{J_1_approx}\\
	& \mathcal{J}_2 V(x_1, x_2 + h) = e^{-\varsigma_2 h} \mathcal{J}_2 V(x_1, x_2) +  \omega_0(\varsigma_2, h) V(x_1, x_2) + \omega_1(\varsigma_2, h) V(x_1, x_2 + h) + O(h^3) \label{J_2_approx},
\end{align}
where
\begin{equation*}
	\omega_0(\varsigma, h) = \frac{1 - (1 + \varsigma h) e^{-\varsigma h}}{\varsigma h}, \quad \omega_1(\varsigma, h) = \frac{-1 + \varsigma h + e^{-\varsigma h}}{\varsigma h}.
\end{equation*}

We can also approximate $\mathcal{J}_{12} V = \mathcal{J}_1 \mathcal{J}_2 V$ by applying above approximations for $\mathcal{J}_1$ and $\mathcal{J}_2$ consecutively.


Consider the grid
\begin{equation}
	\begin{aligned}
		0 = x_1^0 < x_1^1 < \ldots < x_1^{m_1}, \\
		0 = x_2^0 < x_2^1 < \ldots < x_2^{m_2},
	\end{aligned}
\end{equation}
where $x_1^{m_1}$ and $x_2^{m_2}$ are large positive numbers.\\

Then, we can write recurrence formulas for computing the integral operator on the grid. Denote $J_1^{i, j}, J_2^{i, j}$, $J_{12}^{i, j}$ the corresponding approximations of $\mathcal{J}_{1}V(x_1^i, x_2^j)$ , $ \mathcal{J}_{2}V(x_1^i, x_2^j)$, $\mathcal{J}_{12}V(x_1^i, x_2^j)$ on the grid. Applying (\ref{J_1_approx}) and (\ref{J_2_approx}) we get

\begin{align}
	&J_1^{i+1, j} = e^{-\varsigma_1 h^1_{i+1}} J_1^{i, j} + \omega_0(\varsigma_1, h^1_{i+1}) V(x_1^i, x_2^j) + \omega_1(\varsigma_1, h_{i+1}^1) V(x_1^{i+1}, x_2^{j}), \label{J_1_rec}\\
	&J_2^{i, j+1} = e^{-\varsigma_2 h^2_{j+1}} J_2^{i, j} + \omega_0(\varsigma_2, h^2_{j+1}) V(x_1^i, x_2^j) + \omega_1(\varsigma_2, h_{j+1}^2) V(x_1^{i}, x_2^{j+1}), \label{J_2_rec}
\end{align}
where $h_{i+1}^1 = x_1^{i+1} - x_1^{i}, h_{j+1}^2 = x_2^{j+1} - x_2^{j}$.

For an alternative method, we rewrite the integral operator as a differential equation
\begin{align}
		& \frac{\partial}{\partial x_1} \left(\mathcal{J}_1 V(x_1, x_2)e^{\varsigma_1 x_1} \right) =  \varsigma_1 V(x_1, x_2) e^{\varsigma_1 x_1}, \label{J1_ode}\\
		& \frac{\partial}{\partial x_2} \left(\mathcal{J}_2 V(x_1, x_2)e^{\varsigma_2 x_2} \right) =  \varsigma_2 V(x_1, x_2) e^{\varsigma_2 x_2}, \label{J2_ode}\\
		& \frac{\partial^2}{\partial x_1 \partial x_2} \left(\mathcal{J}_{12} V(x_1, x_2)e^{\varsigma_1 x_1 + \varsigma_2 x_2} \right) =  \varsigma_1  \varsigma_2 V(x_1, x_2) e^{\varsigma_1 x_1 + \varsigma_2 x_2}. \label{J12_pde}
\end{align}
Then, we apply the Adams-Moulton method of second order which gives us third order of accuracy locally (\cite{butcher2008numerical})
\begin{align}
	&J_1^{i+1, j} = e^{-\varsigma_1 h^1_{i+1}} J_1^{i, j} + \frac{1}{2} h^1_{i+1} e^{-\varsigma_1 h^1_{i+1}} \varsigma_1  V(x_1^i, x_2^j) + \frac{1}{2} h^1_{i+1} \varsigma_1 V(x_1^{i+1}, x_2^{j}), \label{J_1_rec_adams}\\
	&J_2^{i, j+1} = e^{-\varsigma_2 h^2_{j+1}} J_2^{i, j} +\frac{1}{2} h^2_{j+1} e^{-\varsigma_2 h^2_{j+1}} \varsigma_2  V(x_1^i, x_2^j) + \frac{1}{2} h^2_{j+1} \varsigma_2 V(x_1^{i}, x_2^{j+1}), \label{J_2_rec_adams}
\end{align}
where $h_{i+1}^1 = x_1^{i+1} - x_1^{i}, h_{j+1}^2 = x_2^{j+1} - x_2^{j}$, and is equivalent to the trapezoidal rule.

We can rewrite (\ref{J_1_rec_adams})--(\ref{J_2_rec_adams}) in the same notation as  (\ref{J_1_rec})--(\ref{J_2_rec}) by defining
\begin{equation*}
	\omega_0(\varsigma, h) = \frac{1}{2} h e^{-\varsigma h} \varsigma, \quad \omega_1(\varsigma, h) = \frac{1}{2} h \varsigma.
\end{equation*}
So,
\begin{align*}
	&J_1^{i+1, j} = e^{-\varsigma_1 h^1_{i+1}} J_1^{i, j} + \omega_0(\varsigma_1, h^1_{i+1}) V(x_1^i, x_2^j) + \omega_1(\varsigma_1, h_{i+1}^1) V(x_1^{i+1}, x_2^{j}),\\
	&J_2^{i, j+1} = e^{-\varsigma_2 h^2_{j+1}} J_2^{i, j} + \omega_0(\varsigma_2, h^2_{j+1}) V(x_1^i, x_2^j) + \omega_1(\varsigma_2, h_{j+1}^2) V(x_1^{i}, x_2^{j+1}). 
\end{align*}

As a result we get explicit recursive formulas for approximations of $\mathcal{J}_1 V$ and $\mathcal{J}_2 V$ that can be computed for all grid points via $O(m_1 m_2)$ operations. Both methods give the same order of accuracy. As was discussed above, in order to compute the approximation of $\mathcal{J}_{12} V$ we can apply consecutively the approximations of $\mathcal{J}_2 V$ and $\mathcal{J}_1 (\mathcal{J}_2 V)$. So, we have the two-step procedure:
\begin{equation}
	I_{12}^{i+1, j} = e^{-\varsigma_1 h^1_{i+1}} I_{12}^{i, j} + \omega_0(\varsigma_1, h^1_{i+1}) V(x_1^i, x_2^j) + \omega_1(\varsigma_1, h_{i+1}^1) V(x_1^{i+1}, x_2^{j}), \label{I_12_rec}\\
\end{equation}
and 
\begin{equation}
	J_{12}^{i, j+1} = e^{-\varsigma_2 h^2_{j+1}} J_{12}^{i, j} + \omega_0(\varsigma_2, h^2_{j+1}) I_{12}^{i, j} + \omega_1(\varsigma_2, h_{j+1}^2) I_{12}^{i, j+1}. \label{J_12_rec}\\
\end{equation}
Using this two-step procedure, we can also compute an approximation of $\mathcal{J}_{12} V$ on the grid in complexity $O(m_1 m_2)$.

 We shall subsequently analyze the stability of the second method and use it in the numerical tests. The results for the first method would be very similar.
 
% \paragraph{Eigenvalues of discretized jump operator.}
%\label{jump_eigs}
For the implementation, computing and storing a matrix representation of the jump operator is not necessary, since the operator can be computed iteratively as described above, but we shall use matrix notation for the analysis. We henceforth denote $J_1, J_2$, and $J_{12}$ the matrices of the discretized jump operators. From (\ref{J_1_rec})--({\ref{J_2_rec}) we can find that the matrices $J_1$ and $J_2$ are lower-triangular with diagonal elements $w_1 = \omega_1(\varsigma_1, h_1)$ and $w_2 =  \omega_1(\varsigma_2, h_2)$. Then, $J_{12} = J_1 J_2$ is also a lower-triangular matrix with diagonal elements $w_1 w_2$. To illustrate, in Figure \ref{jump_matrices} we plot the sparsity patterns in $J_1, J_2$, and $J_{12}$.
 \begin{figure}[H]
	\begin{center}
				\subfloat[%Sparsity pattern of 
				$J_1$.]{\includegraphics[width=0.32\textwidth,trim={2cm 0 2cm 0.5cm},clip]{jump_matrix_1.png}}
				\subfloat[%Sparsity pattern of 
				$J_2$.]{\includegraphics[width=0.32\textwidth,trim={2cm 0 2cm 0.5cm},clip]{jump_matrix_2.png}}
				\subfloat[%Sparsity pattern of 
				$J_{12}$.]{\includegraphics[width=0.32\textwidth,trim={2cm 0 2cm 0.5cm},clip]{jump_matrix_12.png}}
	\end{center}		
	\vspace{-20pt}
	\caption{Sparsity pattern of $J_1$, $J_2$, and $J_{12}$. Here, $m_1=m_2=20$ and $nz$ is the number of non-zero elements of the matrices.}
 	\label{jump_matrices}
\end{figure}
%Since the matrices are lower-triangular, the eigenvalues are the diagonal elements. Thus, matrix $J_1$ has all eigenvalues equal to $w_1$, matrix $J_2$ has all eigenvalues equal to $w_2$, and matrix $J_{12}$ has all eigenvalues equal to $w_{12} = w_1 w_2$.

\subsection{Discretization of the differential part of the PIDE}
Now consider the approximation of derivatives in the differential operator on a non-uniform grid. We use the standard derivative approximation (\cite{kluge2002}, \cite{in2010adi}). For the first derivative over each variable consider right-sided, central, and left-sided schemes. So, for the derivative over $x_1$ we have:
\begin{align}
&	\frac{\partial V}{\partial x_1}(x_1^i, x_2^j) \approx \alpha^1_{i, -2} V(x_1^{i-2}, x_2^j) + \alpha^1_{i, -1} V(x_1^{i-1}, x_2^j)+ \alpha^1_{i, 0} V(x_1^i, x_2^j), \label{D_x1_1}\\
&	\frac{\partial V}{\partial x_1}(x_1^i, x_2^j) \approx \beta^1_{i, -1} V(x_1^{i-1}, x_2^j) + \beta^1_{i, 0} V(x_1^{i}, x_2^j)+ \beta^1_{i, 1} V(x_1^{i+1}, x_2^j), \label{D_x1_center}\\
&	\frac{\partial V}{\partial x_1}(x_1^i, x_2^j) \approx \gamma^1_{i, 0} V(x_1^{i}, x_2^j) + \gamma^1_{i, 1} V(x_1^{i+1}, x_2^j)+ \gamma^1_{i, 2} V(x_1^{i+2}, x_2^j) \label{D_x1_2},
\end{align}
while for derivative over $x_2$ we have:
\begin{align}
&	\frac{\partial V}{\partial x_2}(x_1^i, x_2^j) \approx \alpha^2_{j, -2} V(x_1^i, x_2^{j-2}) + \alpha^2_{j, -1} V(x_1^i, x_2^{j-1})+ \alpha^2_{j, 0} V(x_1^i, x_2^j), \label{D_x2_1} \\
&	\frac{\partial V}{\partial x_2}(x_1^i, x_2^j) \approx \beta^2_{j, -1} V(x_1^i, x_2^{j-1}) + \beta^2_{j, 0} V(x_1^{i}, x_2^j)+ \beta^2_{j, 1} V(x_1^j, x_2^{j+1}), \label{D_x2_center}\\
&	\frac{\partial V}{\partial x_2}(x_1^i, x_2^j) \approx \gamma^2_{j, 0} V(x_1^{i}, x_2^j) + \gamma^2_{j, 1} V(x_1^i, x_2^{j+1}, x_2^j)+ \gamma^2_{j, 2} V(x_1^i, x_2^{j+2}), \label{D_x2_2}
\end{align}
with coefficients
\begin{equation*}
	\begin{aligned}
		\alpha^k_{i, -2} &= \frac{\Delta x_k^i}{\Delta x_k^{i-1} (\Delta x_k^{i-1} + \Delta x_k^i)},  & \alpha^k_{i, -1} &= \frac{-\Delta x_k^{i-1} - \Delta x_k^i}{\Delta x_k^{i-1} \Delta x_k^i},  &\alpha^k_{i, 0} &= \frac{\Delta x_k^{i-1} + 2 \Delta x_k^i}{\Delta x_k^i (\Delta x_k^{i-1} + \Delta x_k^i)} , \\
		\beta^k_{i, -1} &= \frac{-\Delta x_k^{i+1}}{\Delta x_k^{i} (\Delta x_k^{i} + \Delta x_k^{i+1})}, & \beta^k_{i, 0} &= \frac{\Delta x_k^{i+1} - \Delta x_k^i}{\Delta x_k^{i} \Delta x_k^{i+1}},  &\beta^k_{i, 1} &= \frac{\Delta x_k^{i}}{\Delta x_k^{i+1} (\Delta x_k^{i} + \Delta x_k^{i+1})} , \\
		\gamma^k_{i, 0} &= \frac{-2\Delta x_k^{i+1} - \Delta x_k^{i+2}}{\Delta x_k^{i+1} (\Delta x_k^{i+1} + \Delta x_k^{i+2})}, & \gamma^k_{i, 1} &= \frac{\Delta x_k^{i+1} + \Delta x_k^{i+2}}{\Delta x_k^{i+1} \Delta x_k^{i+2}},  & \gamma^k_{i, 2} &= \frac{-\Delta x_k^{i+1}}{\Delta x_k^{i+2} (\Delta x_k^{i+1} + \Delta x_k^{i+2})} .
	\end{aligned}
\end{equation*}
For the boundaries at $0$ we use the schemes \eqref{D_x1_1} and \eqref{D_x2_1}, for the right boundaries at $x_1^{m_1}$ and $x_2^{m_2}$ we use the schemes \eqref{D_x1_2} and \eqref{D_x2_2}, and for other points we use the central schemes \eqref{D_x1_center} and \eqref{D_x2_center}.

To approximate the second derivative we use the central scheme:
\begin{align}
	&	\frac{\partial^2 V}{\partial x_1^2}(x_1^i, x_2^j) \approx \delta^1_{i, -1} V(x_1^{i-1}, x_2^j) + \delta^1_{i, 0} V(x_1^{i}, x_2^j)+ \delta^1_{i, 1} V(x_1^{i+1}, x_2^j), \label{D2_x1} \\
&	\frac{\partial^2 V}{\partial x_2^2}(x_1^i, x_2^j) \approx \delta^2_{j, -1} V(x_1^i, x_2^{j-1}) + \delta^2_{j, 0} V(x_1^{i}, x_2^j)+ \delta^2_{j, 1} V(x_1^j, x_2^{j+1}) \label{D2_x2},
\end{align}
with coefficients
\begin{equation*}
		\delta^k_{i, -1} = \frac{2}{\Delta x_k^{i} (\Delta x_k^{i} + \Delta x_k^{i+1})}, \quad \delta^k_{i, 0} = \frac{-2}{\Delta x_k^{i} \Delta x_k^{i+1}}, \quad \delta^k_{i, 1} = \frac{2}{\Delta x_k^{i+1} (\Delta x_k^{i} + \Delta x_k^{i+1})},
\end{equation*}	
and to approximate the second mixed derivative we use the scheme:
\begin{equation}
	\frac{\partial^2 V}{\partial x_1 \partial x_2} (x_1^i, x_2^j) \approx \sum_{k, l = -1}^1 \beta_{i, k}^1 \beta_{j, l}^2 V(x_1^{i+k}, x_2^{j+l}). \label{D_x1x2}	
\end{equation}

As a result, we can approximate the differential operator $\mathcal{D} V$ by a discrete operator
\begin{equation}
	D V = D_1 V + D_2 V + D_{12} V,
\end{equation}
where $D_1 V$ contains the discretized derivatives over $x_1$ defined in (\ref{D_x1_1})--(\ref{D_x1_2}) and (\ref{D2_x1}), $D_2 V$ contains the discretized derivatives over $x_2$ defined in (\ref{D_x2_1})--(\ref{D_x2_2}) and (\ref{D2_x2}), and $D_{12} V$ contains the discretized mixed derivative defined in (\ref{D_x1x2}).

By straightforward but lengthy Taylor expansion of the expression in (\ref{D_x1_1})--(\ref{D_x1x2}), the scheme (\ref{HV_scheme}) has  second order truncation error in variables $x_1$ and $x_2$ for meshes which are either uniform or smooth transformations of such meshes, as we shall consider later.

\subsection{Time discretization: ADI scheme}
After discretization over $(x_1, x_2)$ we can rewrite PIDE (\ref{pide_forward}) as a system of ordinary (linear) differential equations. Consider the vector $U(t) \in \mathbb{R}^{m_1m_2 \times 1}$ whose elements correspond to $V(t, x_1^i, x_2^j)$. Then
\begin{equation}
	\begin{aligned}
		& U'(t) = \tilde{A} U(t) + b(t), \\
		& U(0) = U_0,
	\end{aligned}
\end{equation}
where $\tilde{A} = D_1 + D_2 + D_{12} + \lambda_1 J_1 + \lambda_2 J_2 + \lambda_{12} J_{12} - (\lambda_1 + \lambda_2 + \lambda_{12}) I$, and $b(t)$ is determined from boundary conditions and the right-hand side.

To solve this system, we apply an ADI scheme for the time discretization. Consider, for simplicity, a uniform time mesh with time step $\Delta t: t_n = n \Delta t, n = 0, \ldots, N-1$. 

We decompose the matrix $\tilde{A}$  into three matrices, $\tilde{A} = \tilde{A}_0 + \tilde{A}_1 + \tilde{A}_2$, where
\begin{align*}
	& \tilde{A}_0 =  D_{12} + \lambda_1 J_1 + \lambda_2 J_2 + \lambda_{12} J_{12},  \\
  	& \tilde{A}_1 = D_1 - \left(\lambda_1 + \frac{\lambda_{12}}{2} \right) I, \\
	  & \tilde{A}_2 = D_2 - \left(\lambda_2 + \frac{\lambda_{12}}{2} \right) I,
\end{align*}
and $b(t) = b_0(t) + b_1(t) + b_2(t)$, where $b_0(t)$ corresponds to the right-hand side and the FD discretization of the mixed derivatives on the boundary, $b_1(t)$ and $b_2(t)$ correspond to the FD discretization of the derivatives over $x_1$ and $x_2$ on the boundary.

Now we can apply a traditional ADI scheme with matrices $\tilde{A}_0, \tilde{A}_1$, and $\tilde{A}_2$. We choose the Hundsdorfer--Verwer (HV) scheme (\cite{HV}) in order to have second order accuracy in the time variable, and unconditional stability, as we shall prove below. For convenience, denote
\begin{align}
	& F_j(t, x) = \tilde{A}_j x + b_j(t), \quad j = 0, 1, 2, \label{F_j}\\
	& F(t, x) = (\tilde{A}_0 + \tilde{A}_1 + \tilde{A}_2 ) x + (b_0(t) + b_1(t) + b_2(t)),
\end{align}
and apply the Hundsdorfer--Verwer (HV) scheme:
\begin{equation}
	\label{HV_scheme}
	\left\{
	\begin{aligned}
	&	Y_0 = U_{n-1} + \Delta t F(t_{n-1}, U_{n-1}), \\
	&	Y_j = Y_{j-1} + \theta \Delta t (F_j(t_n, Y_j) - F_j(t_n, U_{n-1})), \quad j = 1, 2, \\
	&	\tilde{Y}_0 = Y_0 + \sigma \Delta t (F(t_n, Y_2) - F(t_{n-1}, U_{n-1})), \\
	&	\tilde{Y}_j = \tilde{Y}_{j-1} + \theta \Delta t (F_j(t_n, \tilde{Y}_j - F_j(t_n, Y_2)), \quad j = 1, 2, \\
	&	U_n = \tilde{Y}_2.
	\end{aligned}
	\right.
\end{equation}

In this scheme, parts that contain $F_1$ and $F_2$ are treated implicitly. The matrix $\tilde{A}_1$ is tridiagonal and $\tilde{A}_2$ is block-tridiagonal and can be inverted via $O(m_1 m_2)$ operations.  As a result, the overall complexity is $O(m_1 m_2)$ for a single time step or $O(N m_1 m_2)$ for the whole procedure.

Moreover, the scheme has second order of consistency in both $(x_1, x_2)$ and $t$ for any given $\theta$ and $\sigma = \frac{1}{2}$. 

\subsection{Stability analysis}
In this section, we consider the PIDE \eqref{pide_forward} with zero boundary conditions at $0$ in both directions and on a uniform grid,
such that $F_j(t, x) = \tilde{A}_j x$ and
\begin{equation}
	\label{HV_nobound}
	\left\{
	\begin{aligned}
	&	Y_0 = U_{n-1} + \Delta t \tilde{A} U_{n-1}, \\
	&	Y_j = Y_{j-1} + \theta \Delta t (\tilde{A}_j Y_j - \tilde{A}_j U_{n-1}),\quad j = 1, 2, \\
	&	\tilde{Y}_0 = Y_0 + \sigma \Delta t (\tilde{A} Y_2- \tilde{A} U_{n-1}), \\
	&	\tilde{Y}_j = \tilde{Y}_{j-1} + \theta \Delta t (\tilde{A}_j\tilde{Y}_j - \tilde{A}_j Y_2), \quad j = 1, 2 \\
	&	U_n = \tilde{Y}_2.
	\end{aligned}
	\right.
\end{equation}

For convenience, we denote by $F: U_n = F U_{n-1}$.

We further consider the PDE on $\mathbb{R}^2$, i.e., without default boundaries. Hence, we assume that diffusion and jump operators are discretized on 
 an infinite, uniform mesh $\{(j_1 h_1, j_2 h_2), (j_1, j_2) \in \mathbb{Z}^2\}$, such that, e.g.\ $D_1, D_2, D_{12}, J_1, J_2$ are infinite matrices.
 This is different to \cite{intHoutStability}, where finite matrices and periodic boundary conditions (without integral terms) are considered.

We use von Neumann stability analysis, as first introduced by \cite{charney1950numerical}, by expanding the solution into a Fourier series.
Hence, we shall show that the proposed scheme (\ref{HV_nobound}) is unconditionally stable,
 i.e.\ we will show that all eigenvalues of the operator  $F$ have moduli bounded by 1 plus an $O(\Delta t)$ term,
 where the corresponding eigenfunctions are given by $\exp(i \phi_1 j_1) \exp(i \phi_2 j_2)$, with $\phi_1$ and $\phi_2$ the wave numbers and
 $j_1$ and $j_2$ the grid coordinates.
 
%\paragraph{Stability analysis of scheme (\ref{HV_nobound})}
 \cite{intHoutStability} show that when all matrices commute
(as in the PDE case with periodic boundary conditions), 
the eigenvalues for $F$ are given by %this leads to the condition %(\cite{intHoutStability})
%\begin{equation}
%	\label{stab_eq}
%	|T(\tilde{z}_0, \tilde{z}_1, \tilde{z}_2)| \le 1,
%\end{equation}
\begin{eqnarray}
\label{defT}
T(\tilde{z}_0, \tilde{z}_1, \tilde{z}_2) &=& 1 + 2 \frac{\tilde{z}_0 + \tilde{z}}{p} - \frac{\tilde{z}_0 + \tilde{z}}{p^2} + \sigma \frac{(\tilde{z}_0 + \tilde{z})^2}{p^2} \quad \text{with} \\
	p &=& (1 - \theta \tilde{z}_1) (1 - \theta \tilde{z}_2), \nonumber
\end{eqnarray}
where $\tilde{z}_j = \tilde{\mu}_j \Delta t$, where $\tilde{\mu}_j$ is an eigenvalue of $\tilde{A}_j$, $j = 0, 1, 2$, $\tilde{z} = \tilde{z}_1 + \tilde{z}_2$, $\theta \ge 0$.


The analysis is made slightly more complicated in our case through the presence of the  jump operators.
In the remainder of this section, %we first show that for our case \eqref{stab_eq} can still be applied, and then 
we show that stability is still given under the same conditions on $\theta$ and $\sigma$ as in the purely diffusive case. For the correspondence of notation with \cite{intHoutStability}, we denote $A = A_0 + A_1 + A_2$, where $A_0 = D_{12}, A_1 = D_1$, $A_2 = D_2$ and $\mu_0, \mu_1$, and $\mu_2$ are the eigenvalues of the corresponding matrices. Similar to $\tilde{z}_0, \tilde{z}_1,$ and $\tilde{z}_2$, we define scaled eigenvalues $z_0 = \mu_0 \Delta t, z_1 = \mu_1 \Delta t, z_2 = \mu_2 \Delta t$.



%Using properties of lower-triangular matrices, we can easily see that the matrix $F$ from \eqref{HV_nobound} can be represented as $F = U T_F U^{*}$, where $T_F$ is a lower-triangular matrix, whose eigenvalues are equal to \eqref{defT}.




%\begin{lemma}
%\label{lemma_commute}
%The following identities are satisfied
We have the eigenvalues $\tilde{\mu}_j$ of $\tilde{A}_j$ given by
\begin{align}
	& \tilde{\mu}_0 = \mu_0 + \lambda_1 w_1 + \lambda_2 w_2 + \lambda_{12} w_{12}, \label{mu_0_eq} \\
	& \tilde{\mu}_1 = \mu_1 - \left(\lambda_1 + \frac{\lambda_{12}}{2}\right), \label{mu_1_eq}\\
	& \tilde{\mu}_2 = \mu_2 - \left(\lambda_2 + \frac{\lambda_{12}}{2}\right) \label{mu_2_eq},
\end{align}
where $\mu_j$ is  an eigenvalue of $A_j$, and $w_1, w_2$, and $w_{12}$ are eigenvalues of $J_1, J_2$, and $J_{12}$.
%\end{lemma}

Denote % $z_j = \mu_j \Delta t$ with $\mu_j$  an eigenvalue of $A_j$, 
$z = z_1 + z_2$, $s_1 = w_1  \Delta t, s_2 = w_2 \Delta t, s_{12} = w_{12} \Delta t$, where $w_1, w_2, w_{12}$ are eigenvalues of $J_1, J_2, J_{12}$ respectively, and $s_0 = \lambda_1 s_1 + \lambda_2 s_2 + \lambda_{12} s_{12}$.

	Multiplying (\ref{mu_0_eq})--(\ref{mu_2_eq}) by $\Delta t$, we have 
	\begin{align} 
		& \tilde{z}_0 = z_0 + s_0, \label{tilde_z0} \\
		& \tilde{z}_1 = z_1 - \left(\lambda_1 + \frac{\lambda_{12}}{2}\right) \Delta t, \\
		& \tilde{z}_2 = z_2 - \left(\lambda_2 + \frac{\lambda_{12}}{2}\right) \Delta t.  \label{tilde_z2} 		
	\end{align}
\begin{theorem}[\cite{intHoutStability}, Theorem 3.2]
	\label{theor_inthout}
	Assume $\Re({z}_1) \le 0, \Re({z}_2) \le 0$, $|{z}_0| \le 2\sqrt{\Re({z}_1) \Re({z}_2)}$, where ${z}_0, {z}_1$, and ${z}_2$ are the eigenvalues of ${A}_0, {A}_1$, and ${A}_2$, %in \eqref{F_j} 
	and
	\begin{equation*}
		\frac{1}{2} \le \sigma \le \left(1 + \frac{\sqrt{2}}{2} \right) \theta.
	\end{equation*}
	Then,
	\begin{equation*}
		|T({z}_0, {z}_1, {z}_2)| \le 1,
	\end{equation*}
	and the Hundsdorfer--Verwer scheme \eqref{HV_nobound} is stable in the purely diffusive case.
\end{theorem}

\begin{lemma}
The scaled eigenvalues of $A_0$, $A_1$, $A_2$, $J_1$, $J_2$, $J_{12}$ can be expressed as
\begin{eqnarray}
\label{z0}
	z_0 &=& -\rho b [\sin{\phi_1} \sin{\phi_2}],  \\
	\label{z1}
	z_1 &=& -a_1 (1 - \cos{\phi_1}) + i \xi_1 q_1 \sin{\phi_1}, \\
	\label{z2}
	z_2 &=& -a_2 (1 - \cos{\phi_2}) + i \xi_2 q_2 \sin{\phi_2}, \\
%	s_1 &=& \Delta t \, \zeta_1 h_1 \left(\frac{1}{2} + \frac{\exp(-h_1 (\zeta_1 + i \phi_1))}{1-\exp(-h_1 (\zeta_1 + i \phi_1))} \right), \\
%	s_2 &=& \Delta t \, \zeta_2 h_2 \left(\frac{1}{2} + \frac{\exp(-h_2 (\zeta_2 + i \phi_2))}{1-\exp(-h_2 (\zeta_2 + i \phi_2))} \right), \\
	s_1 &=& \Delta t \, \zeta_1 h_1 \left(\frac{1}{2} + \frac{\exp(-h_1 \zeta_1 + i \phi_1)}{1-\exp(-h_1 \zeta_1 + i \phi_1)} \right), \\
	s_2 &=& \Delta t \, \zeta_2 h_2 \left(\frac{1}{2} + \frac{\exp(-h_2 \zeta_2 + i \phi_2)}{1-\exp(-h_2 \zeta_2 + i \phi_2)} \right), \\
	s_{12} &=& s_1 s_2/\Delta t,
\end{eqnarray}
where
\begin{equation*}
	q_1 = \frac{\Delta t}{h_1}, \quad q_2 = \frac{\Delta t}{h_2}, \quad a_1 = \frac{\Delta t}{h_1^2}, \quad a_2 = \frac{\Delta t}{h_2^2}, \quad b= \frac{\Delta t}{h_1 h_2},
\end{equation*}
and $\phi_j \in [0, 2 \pi]$ for $j = 1, 2$.

Moreover,
\begin{equation}
\label{karelineq}
	|z_0| \le 2 \sqrt{\Re(z_1) \Re(z_2)}.
\end{equation}	
\end{lemma}
\begin{proof}
	All six eigenvalues follow by insertion of the ansatz $U=\exp(i \phi_1 j_1) \exp(i \phi_2 j_2)$.
	For instance, %for $U(j,k)=$
	\[
	(J_1 U)(j_1,j_2) = \zeta_1 h_1 \left(\frac{1}{2} U(j_1,j_2) +  \sum_{k=1}^\infty \exp(-\zeta_1 h_1 k) U(j_1-k,j_2) \right),
	\]
	and the result follows by using the special form of $U$ and evaluating the geometric series.
	
	Alternatively, the first three equations follow immediately from the eigenvalues for finite matrices (\cite{intHoutStability}, p.29),
	which are given by (\ref{z0})--(\ref{z2}) where $\phi_j = 2 l \pi/m_j$, $l=1,\ldots,m_j$.
	In the infinite mesh case, the spectrum is the continuous limit and (\ref{karelineq}) still holds.
%	 then the eigenvalues of the semi-infinite (banded Toeplitz) matrix are given by the Schmidt and Spitzer Theorem (see Theorem 11.17 in \cite{bottcher2005spectral}) as precisely the limits of sequences of eigenvalues of the finite-dimensional matrices.
\end{proof}


%\begin{lemma}
%\label{lemma_z0ineq}
%For $\tilde{z}_0, \tilde{z}_1, \tilde{z}_2$ from \eqref{tilde_z0}--\eqref{tilde_z2},
%\begin{equation}
%	\label{z_0_ineq}
%	|\tilde{z}_0|^2 \le 4 \Re(\tilde{z}_1) \Re(\tilde{z}_2) + c \Delta t,
%\end{equation}
%for some constant $c$ that does not depend on $\Delta t, h_1,$ and $h_2$.
%\end{lemma}
%\begin{proof}
%Define, 
%\begin{gather*}
%	\tilde{\lambda}_1 = \left(\lambda_1 + \frac{\lambda_{12}}{2} \right) \Delta t, \\
%	\tilde{\lambda}_2 = \left(\lambda_2 + \frac{\lambda_{12}}{2} \right) \Delta t.
%\end{gather*}
%Then,
%\begin{align}
%	|\tilde{z}_0|^2 = |z_0 + s_0|^2 = |z_0|^2 + 2 z_0 s_0 + s_0^2 &\le 4 \Re(z_1) \Re(z_2) + 2 z_0 s_0 + s_0^2  \\
%	& = 4 (\Re(\tilde{z}_1) + \tilde{\lambda}_1)( \Re(\tilde{z}_2) + \tilde{\lambda}_2) + 2 z_0 s_0 + s_0^2 .
%\end{align}
%We can rewrite the first term as
%\begin{equation*}
%	 (\Re(\tilde{z}_1) + \tilde{\lambda}_1)( \Re(\tilde{z}_2) + \tilde{\lambda}_2) = \Re(\tilde{z}_1) \Re(\tilde{z}_2) + \Re(z_1) \tilde{\lambda}_2 + \Re(z_2) \tilde{\lambda}_1 - \tilde{\lambda}_1 \tilde{\lambda}_2.
%\end{equation*}
%Thus, we have
%\begin{equation*}
%	|z_0 + s_0|^2 \le 4\Re(\tilde{z}_1) \Re(\tilde{z}_2) + 4 \Re(z_1) \tilde{\lambda}_2 + 4 \Re(z_2) \tilde{\lambda}_1 - 4\tilde{\lambda}_1 \tilde{\lambda}_2 + 2 z_0 s_0 + s_0^2.
%\end{equation*}
%From \eqref{z0}--\eqref{z2}, we can see that $z_0, z_1$, and $z_2$ does not depend on $\Delta t$ (they depend on $\frac{\Delta t}{h_i}$, which is assumed to be constant), while $\tilde{\lambda_1}, \tilde{\lambda_2},$ and $s_0$ are proportional to $\Delta t$. Thus
%\begin{equation}
%	4 \Re(z_1) \tilde{\lambda}_2 + 4 \Re(z_2) \tilde{\lambda}_1 - 4\tilde{\lambda}_1 \tilde{\lambda}_2 + 2 z_0 s_0 + s_0^2 \le c \Delta t,
%\end{equation}
%for some constant $c$.
%
%Thus, we have proved the inequality (\ref{z_0_ineq}).
%\end{proof}

\begin{theorem}
	Consider $\frac{1}{2} \le \sigma \le \left(1 + \frac{\sqrt{2}}{2} \right) \theta$. Then there exists $c>0$, independent of $\Delta t\le 1$, $h_1$ and $h_2$, such that
	\begin{enumerate}
	\item
	\begin{equation}
	|T(\tilde{z}_0, \tilde{z}_1, \tilde{z}_2)| \le 1 + c \Delta t, \qquad \forall \phi_1, \phi_2 \in [0,2\pi],
	\end{equation}
	i.e., the scheme is von Neumann stable;
	\item
	\label{part2}
	\begin{equation}
	|U_n|_2 %:= \sum_{j=-\infty}^{\infty} U_{n}(j)^2 
	\le {\rm e}^{c n \Delta t} |U_0|_2, \qquad \forall n\ge 0,
	\end{equation}
	for $|U_n|_2 = h_1 h_2 \left(\sum_{j_1,j_2=-\infty}^\infty |U_n(j_1,j_2)|^2\right)^{\scriptsize 1/2}$, i.e., the scheme is $l_2$ stable.
	\end{enumerate}
\end{theorem}
\begin{proof}


%The first inequality is a weak version of Theorem 3.2 of  \cite{intHoutStability} and can be proved in the similar way by 
%adding the term $c \Delta t$.

First, we have that
\begin{eqnarray*}
|T({z}_0, \tilde{z}_1, \tilde{z}_2)| = \left|1 + 2 \frac{{z}_0 + \tilde{z}}{p} - \frac{{z}_0 + \tilde{z}}{p^2} + \sigma \frac{({z}_0 + \tilde{z})^2}{p^2}\right|
\le 1,
\end{eqnarray*}
where as before $p = (1 - \theta \tilde{z}_1) (1 - \theta \tilde{z}_2)$ and $\tilde{z} = \tilde{z}_1 + \tilde{z}_2$. 
This follows from Theorem \ref{theor_inthout} because $\lambda_1$, $\lambda_2$ and $\lambda_{12}$ are positive and therefore (\ref{karelineq}) is still satisfied with $z_1$ and $z_2$ replaced by $\tilde{z}_1$ and $\tilde{z}_2$.

We have
\begin{eqnarray*}
T(\tilde{z}_0, \tilde{z}_1, \tilde{z}_2) &=&
T({z}_0, \tilde{z}_1, \tilde{z}_2) +
2 \frac{s_0}{p} - \frac{s_0}{p^2} + \sigma \frac{2 s_0 (z_0 + \tilde{z}) + s_0^2}{p^2}.
\end{eqnarray*}
A simple calculation shows that $|s_0|\le c_0 \, \Delta t$ for a constant $c_0$ (independent of $\Delta t, h_1, h_2, \phi_1$, $\phi_2$;
indeed, $c_0=2 \lambda_1 + 2 \lambda_2 + 4 \lambda_{12}$
works for small enough $h_1$, $h_2$).
Therefore, and because $|p|\ge 1$, $|z_0 + \tilde{z}|/|p|\le c_1$ for a constant $c_1$,
\[
\left|2 \frac{s_0}{p} - \frac{s_0}{p^2} + \sigma \frac{2 s_0 (z_0 + \tilde{z}) + s_0^2}{p^2} \right|
\le c \Delta t,
\]
for any $c\ge (3 + 2 \sigma c_1 + c_0 \sigma) c_0$.
From this the first statement follows.

We can now deduce part \ref{part2} by a standard argument. For the discrete-continuous Fourier transform
\[
%\mathcal{F}: 
l_2(\mathbb{Z}^2) \rightarrow L_2(-\pi,\pi)^2, \qquad U \rightarrow \widehat{U}, \qquad \widehat{U}(\phi_1,\phi_2) = h_1 h_2 \sum_{j,k \in \mathbb{Z}} U(j,k) {\rm e}^{-i (j \phi_1 + k \phi_2)},
\]
we have
\[
\widehat{U}_{n+1}(\phi_1,\phi_2) = T(\tilde{z}_0, \tilde{z}_1, \tilde{z}_2) \, \widehat{U}_{n}(\phi_1,\phi_2), \qquad \forall n\ge 0.
\]
Then, by Parseval,
\begin{eqnarray*}
|U_n|_2^2 &=& \frac{1}{4 \pi^2} |\widehat{U}_n|^2  \\
&=& \frac{1}{4 \pi^2} 
\frac{1}{h_1^2 h_2^2} \int_{-\pi}^\pi |\widehat{U}_n(\phi_1,\phi_2)|^2 
\, {\rm d} \phi_1 \, {\rm d} \phi_2 \\
&\le& \frac{1}{4 \pi^2} 
\frac{1}{h_1^2 h_2^2} \int_{-\pi}^\pi (1+c \Delta t)^{2n} |\widehat{U}_0(\phi_1,\phi_2)|^2 
\, {\rm d} \phi_1 \, {\rm d} \phi_2 \\
&\le& {\rm e}^{2 c n \Delta t}  \frac{1}{4 \pi^2} 
\frac{1}{h_1^2 h_2^2} \int_{-\pi}^\pi |\widehat{U}_0(\phi_1,\phi_2)|^2 
\, {\rm d} \phi_1 \, {\rm d} \phi_2 \\ 
&=& {\rm e}^{2 c n \Delta t}  |U_0|_2^2.
\end{eqnarray*}
\end{proof}



%\begin{remark}
This ($l_2$-)stability result together with second order consistency implies ($l_2$-)convergence of second order for all solutions which
are sufficiently smooth that the truncation error is defined and bounded. In our setting, where the initial condition is discontinuous, this is not given. Since the step function lies in the ($l_2$-)closure of smooth functions, convergence is guaranteed, but usually not of second order. We show this empirically in the next section and demonstrate how second order convergence can be restored practically.
%\end{remark}




\subsection{Discontinuous boundary and terminal conditions}

It is well documented (see, e.g.\ \cite{pooley2003}) that the spatial convergence order of central finite difference schemes is generally reduced to one for discontinuous payoffs. Moreover, the time convergence order of the Crank-Nicolson scheme is reduced to one due to the lack of damping of high-frequency components of the error, and this behaviour is inherited by the HV scheme. We address these two issues in the following way.

First, we smooth the terminal condition by the method of local averaging from \cite{pooley2003}, i.e., instead of using nodal values of $\phi$ directly, we use the approximation
\[
\phi(x_1^i,x_2^j) \approx \frac{1}{h_1 h_2} \int_{x_2^i-h_2/2}^{x_2^i+h_2/2} \int_{x_1^j-h_1/2}^{x_1^j+h_1/2}
\phi(\xi_1,\xi_2) \, d\xi_1 d\xi_2.
\]
For step functions with values of 0 and 1, this procedure attaches to each node the fraction of the area where the payoff is 1, in a cell of of size $h
_1 \times h_2$ centred at this point.

We illustrate the convergence improvement on the example of joint survival probabilities. Other quantities show a similar behaviour.
The model parameters in the following tests are the same as in the next section, specifically Table \ref{table:params}.

We choose $\sigma = \frac{1}{2}$ and $\theta = \frac{3}{4}$ in the HV scheme. 

The observed convergence with and without this smoothing procedure is shown in Figure \ref{fig_conv1}.
We choose the $l_2$-norm for its closeness to the stability analysis -- in the periodic case, Fourier analysis gives convergence results in $l_2$ -- and the  $l_\infty$-norm for its relevance to the problem at hand, where we are interested in the solution pointwise.
The behaviour in the $l_1$-norm is very similar.

Hereby, for a method of order $p\ge 1$ we estimate the error by extrapolation as
\begin{equation*}
|Q^{nX}(x_1, x_2) - Q(x_1, x_2)| \approx \frac{1}{2^p-1}  |Q^{nX}(x_1, x_2) - Q^{nX/2}(x_1, x_2)|,
\end{equation*}
where $Q$ is the exact solution, $Q^{nX}$ the solution with $nX$ mesh points,
and the norms are computed by either taking the maximum over mesh points or numerical quadrature.
Here, $nT=1000$ is fixed.



\begin{figure}[H]
	\begin{center}
%				\subfloat[$l_1$-norm.]{\includegraphics[width=0.33\textwidth]{conv_analysis2.png}}
%				\subfloat[$l_2$-norm.]{\includegraphics[width=0.33\textwidth]{conv_analysis1.png}}
%				\subfloat[$l_{\infty}$-norm.]{\includegraphics[width=0.33\textwidth]{conv_analysis3.png}}\\
%				\subfloat[$l_1$-norm.]{\includegraphics[width=0.45\textwidth]{conv_analysis2.png}}
				\subfloat[$l_2$-norm.]{\includegraphics[width=0.49\textwidth]{conv_analysis1.png}} \hfill
				\subfloat[$l_{\infty}$-norm.]{\includegraphics[width=0.49\textwidth]{conv_analysis3.png}}\\
	\end{center}		
	\vspace{-20pt}
	\caption{Convergence analysis for $l_2$- and $l_{\infty}$-norms of the error depending on the mesh size with fixed time-step.}
 	\label{fig_conv1}
\end{figure}

The convergence is clearly of first order without averaging and of second order with averaging.


Second, we modify the scheme using the idea from \cite{reisinger2013} by changing the time variable $\tilde{t} = \sqrt{t}$. This change of variables 
leads to the new PDE
\[
		\frac{\partial V}{\partial \tilde{t}} + 2 \tilde{t} \mathcal{L} V = 2 \tau \chi(\tilde{t}^2, x),
\]
instead of (\ref{kolm_1}), to which we apply the numerical scheme. %, but improves the convergence rate. 
%From numerical results in Section \ref{numerical_experiments}, we can observe second order of convergence.


In Figure \ref{fig_conv2}, we show the convergence with and without time change, estimating the errors in a similar way to above, with $nX=800$ fixed.
 \begin{figure}[H]
	\begin{center}
%				\subfloat[$l_1$-norm.]{\includegraphics[width=0.33\textwidth]{conv_analysis5.png}}
%				\subfloat[$l_2$-norm.]{\includegraphics[width=0.33\textwidth]{conv_analysis4.png}}
%				\subfloat[$l_{\infty}$-norm.]{\includegraphics[width=0.33\textwidth]{conv_analysis6.png}}\\
%				\subfloat[$l_1$-norm.]{\includegraphics[width=0.45\textwidth]{conv_analysis5.png}}
				\subfloat[$l_2$-norm.]{\includegraphics[width=0.49\textwidth]{conv_analysis4.png}} \hfill
				\subfloat[$l_{\infty}$-norm.]{\includegraphics[width=0.49\textwidth]{conv_analysis6.png}}\\

	\end{center}		
	\vspace{-20pt}
	\caption{Convergence analysis for $l_2$- and $l_{\infty}$-norms of the error depending on time-step with fixed mesh size.}
 	\label{fig_conv2}
\end{figure}

The convergence is clearly of first order without time change and of second order with time change. We took here $T=5$ to illustrate the effect more clearly.

%To justify second order convergence rate, in Figure \ref{fig_conv} we present the $l_2$-norm of error depending on the mesh size computed for joint survival probability. %We choose the number of time steps $n = 2m$, where $m$ is the mesh size in each direction.
% \begin{figure}[H]
%	\begin{center}
%		\includegraphics[width=0.9\textwidth]{conv_analysis.png}
%	\end{center}
%	\vspace{-20pt}
%	\caption{$l_2$ norm of error depending on the mesh size}
% 	\label{fig_conv}
%\end{figure}
%

%!TEX root = SISC_elastic_3d.tex
\section{Numerical Experiments}
We present four numerical experiments. 
 In Sec.~\ref{convergence_study}, we verify the order of the convergence of the proposed scheme (\ref{elastic_semi_c}, \ref{fine_scheme}, \ref{continuous_sol}, \ref{continuous_traction}).  In  Sec.~\ref{iterative_section}, we present three iterative methods for solving the linear systems (\ref{traction_gamma_pre}) and (\ref{traction_gamma_corr}). The efficiency of the iterative methods is investigated and a comparison with the LU-factorization method is conducted. Next, in Sec.~\ref{gaussian_source} we show that our schemes generate little reflection at the mesh refinement interface. Finally, the energy conservation property is verified in Sec.~\ref{conserved_energy} with heterogeneous and discontinuous material properties.
\section{Calibration}

In this section we present calibration results of the model. There are eight unknown parameters, see \eqref{kolmogorov_backward}--\eqref{j12_eq}: $\sigma_1, \sigma_2, \rho, \varsigma_1, \varsigma_2, \lambda_1, \lambda_2, \lambda_{12}$. We use CDS and equity put option prices (with different strikes) as market data. If FTD contracts are available, one can use them to estimate $\rho$ and $\lambda_{12}$. Otherwise, historical estimation with share prices time series can be used. 

The data for external liabilities can be found in banks' balance sheets, which are publicly available. Usually, mutual liabilities data are not public information, thus we made an assumption that they are a fixed proportion of the total liabilities, which coincides with \cite{DavidLehar}. In particular, we fix the mutual liabilities as 5\% of total liabilities.

The asset's value is the sum of liabilities and equity price.

We choose Unicredit Bank as the first bank and Santander as the second bank. In Table \ref{data_table} we provide their equity price $E_i$, assets $A_i$ and liabilities $L_i$. As in \cite{LiptonSepp}, the liabilities are computed as a ratio of total liabilities and shares outstanding.

\begin{table}[H]
	\begin{center}
		\begin{tabular}{| c | c | c | c | c | c |}
			\hline
			$E_1(0)$ & $L_1(0)$ & $A_1(0)$ & $E_2(0)$ & $L_2(0)$ & $A_2(0)$  \\ 
			\hline
			6.02&  137.70& 143.72& 6.23 & 86.41 & 92.64 \\
			\hline
		\end{tabular}
		\caption{Assets and liabilities on 30/06/2015 (Bloomberg).}
		\label{data_table}	
	\end{center}
\end{table}
For the calibration we choose 1-year at-the-money, in-the-money, and out-of-the-money equity put options on the banks, and 1-year CDS contracts. Since the spreads of CDS are usually significantly lower than the option prices, we scale them by some weight $w_i$ in the objective function. As a result, we have the following 6-dimensional minimization problem:
\begin{multline}
	\label{calibration_eq}
	\min_{\theta} \{ w_1 (V^{CDS}_1(\theta) - \bar{V}^{CDS}_1)^2 + \sum \limits_{i = 1}^3 (V^{opt}_1(K_{i, 1}, \theta) - \bar{V}^{opt}_1(K_{i, 1}))^2 + \\
	+ w_2 (V^{CDS}_2(\theta) - \bar{V}^{CDS}_2)^2  + \sum \limits_{i = 1}^3 (V^{opt}_2(K_{i, 2}, \theta) - \bar{V}^{opt}_2(K_{i, 2}))^2  \},
\end{multline}
 where $\theta = (\sigma_1, \sigma_2, \lambda_1, \lambda_2, \varsigma_1, \varsigma_2)$, $V^{CDS}_i(\theta)$ is the model CDS spread on the $i$-th bank and $\bar{V}^{CDS}_i$ is the market CDS spread on the $i$-th bank, $V^{opt}_1(K, \theta)$ is the model price of the equity put option on the $i$-th bank with the strike $K$ and $\bar{V}^{opt}_i(K)$ is the market price of the equity put option on the $i$-th bank with strike $K$. Strikes $K_{1, j}, K_{2, j}$, and $K_{3, j}$ are chosen in such a way to take into account the smile. In particular, we choose $K_{1, j} = 1.1 E_j, K_{2, j} = E_j, K_{3, j} = 0.9 E_j$.

In order to find the global minimum of \eqref{calibration_eq} by a Newton-type method, we need to find a good starting point, otherwise an optmization procedure might finish in local minima which are not global minima. To choose the starting point, we calibrate one-dimensional models for each bank without mutual liabilities
\begin{multline}
	\label{calibration_eq1d}
	\min_{\theta_j} \{ w_j (V^{CDS}_j(\theta_j) - \bar{V}^{CDS}_j)^2 + (V^{opt}_j(K_{1, j}, \theta_j) - \bar{V}^{opt}_j(K_{1, j}))^2 + \\
	+(V^{opt}_j(K_{2, j}, \theta_j) - \bar{V}^{opt}_i(K_{2, j}))^2  + (V^{opt}_j(K_{3, j}, \theta_j) - \bar{V}^{opt}_j(K_{3, j}))^2  \},
\end{multline}
where $\theta_j = (\sigma_j, \lambda_j, \varsigma_j)$ for $j = 1, 2$.

The global minima of \eqref{calibration_eq1d} can be found via the {\bf{chebfun toolbox}} (\cite{Trefethen}) that uses Chebyshev polynomials to approximate the function, and then the global minima can be easily found. The calibration results of the one-dimensional model for the first and the second banks are presented in Table \ref{table:params_1d}. We note that the global minima of \eqref{calibration_eq} cannot be found via the chebfun toolbox, since it works with functions up to three variables. There are also more fundamental complexity issues for higher-dimensional tensor product interpolation.

\begin{table}[H]
	\begin{center}
		\begin{tabular}{| c | c | c | c | c | c |}
			\hline
			$\sigma_1$ & $\lambda_1$ & $\varsigma_1$ & $\sigma_2$ & $\lambda_2$ & $\varsigma_2$  \\ 
			\hline
			 0.0117&  0.1001& 0.3661& 0.0154 & 0.0160 & 0.0545\\
			\hline
		\end{tabular}
		\caption{Calibrated parameters of one-dimensional models on 30/06/2015 for $T = 1$.}
		\label{table:params_1d}	
	\end{center}
\end{table}

Similar to \cite{LiptonSepp}, for simplicity, we further assume that 
\begin{equation}
	\lambda_{\{12\}} = \rho \cdot \min(\lambda_1, \lambda_2).
	\label{lambda_assumption}
\end{equation}
 Then, we estimate $\rho$ from historical data. We take one year daily  equity prices $E_i(t)$ by time series (from Bloomberg) and estimate the covariance of asset returns $r_t^i = \frac{\Delta A_i(t)}{A_i(t)}$
\begin{equation}
	\widehat{\cov}(A_1, A_2) = \sum \limits_{i = 1}^n \left(r_{i, 1} - \bar{r_1} \right)\left(r_{i, 2} - \bar{r}_2  \right),
	\label{cov_est}
\end{equation}
where $\bar{r}_1$ and $\bar{r}_2$ are the sample mean of asset returns.

Using \eqref{assets_dynamics}, we can see that \eqref{cov_est} converges to
\begin{equation}
	\widehat{\cov}(A_1, A_2) \underset{n \to +\infty}{\longrightarrow} \sigma_1 \sigma_2 \left( \rho+ \lambda_{\{12\}} /(\varsigma_1 \varsigma_2) \right).
\end{equation}
Using the last equation and \eqref{lambda_assumption}, we can extract the estimated values of $\rho$ and $\lambda_{\{12\}}$. The estimation results are in Table \ref{table:corr_params}.
\begin{table}[H]
	\begin{center}
		\begin{tabular}{| c | c | c | }
			\hline
			& $\rho$ & $\lambda_{\{12\}} $ \\
			\hline
			Estimated value & 0.510 & 0.0188 \\
			\hline
			Confidence interval \footnotemark & (0.500, 0.526)& (0.0182, 0.0194) \\
			\hline
		\end{tabular}
		\caption{Historically estimated correlation coefficients on 30/06/2015 with 1 year window.}
		\label{table:corr_params}	
	\end{center}
\end{table}
\footnotetext{We use a $3 \sigma$ confidence interval.}

Finally, we perform a six-dimensional (constrained) optimization of \eqref{calibration_eq} with the starting point from Table \ref{table:params_1d} and correlation parameters from Table \ref{table:corr_params}. We choose different alternatives of mutual liabilities to have a clear picture how mutual liabilities influence on model parameters. We use the {\bf lsqnonlin} method in Matlab that uses a Trust Region Reflective algorithm \cite{conn2000trust} (with the gradient computed numerically). The model CDS spreads are computed using the method in Section \ref{CDS_pricing}, while equity option prices are computed in the usual finite-difference manner (see \cite{LiptonSepp} for details).  Results are presented in Table \ref{table:params_2d}.
%\begin{table}[H]
%	\begin{center}
%		\begin{tabular}{|c | c | c | c | c | c | c | c |}
%			\hline
%			$L_{12}$ & $L_{21}$ & $\sigma_1$ & $\lambda_1$ & $\varsigma_1$ & $\sigma_2$ & $\lambda_2$ & $\varsigma_2$  \\ 
%			\hline
%			0.0 & 0.0 & 0.0117&  0.1001& 0.3661& 0.0154 & 0.0160 & 0.0545 \\
%			2.0 & 3.0 & 0.0119 & 0.1012 & 0.3968 & 0.0153 & 0.0153 & 0.0517 \\
%			3.0 & 2.0 & 0.0119 & 0.0976 & 0.3841 & 0.0156 & 0.0154 & 0.0522 \\
%			5.0 & 6.0 & 0.0122 & 0.1021 & 0.4233 & 0.0154 & 0.0149 & 0.0491 \\
%			5.0 & 4.0 & 0.0120 & 0.1079 & 0.4212 & 0.0155 & 0.0149 & 0.0496 \\
%			5.0 & 0.0 & 0.0117 & 0.0989 & 0.3627 & 0.0160 & 0.0151 & 0.0527 \\
%			0.0 & 4.0 & 0.0117 & 0.0993 & 0.3796 & 0.0154 & 0.0145 & 0.0522 \\
%			\hline
%		\end{tabular}
%		\caption{Calibrated parameters of two-dimensional model with mutual liabilities on 30/06/2015 for $T = 1$.}
%		\label{table:params_2d}	
%	\end{center}
%\end{table}

\begin{table}[H]
	\begin{center}
		\begin{tabular}{|c | c | c | c | c | c | c | c |}
			\hline
			 Model & $\sigma_1$ & $\lambda_1$ & $\varsigma_1$ & $\sigma_2$ & $\lambda_2$ & $\varsigma_2$  \\ 
			\hline
			With jumps & 0.0122&  0.0950& 0.3958& 0.0160 & 0.0148 & 0.0505 \\
			Without jumps & 0.0206 & -- & -- & 0.0317 & -- & -- \\
			\hline
		\end{tabular}
		\caption{Calibrated parameters of two-dimensional model with mutual liabilities on 30/06/2015 for $T = 1$.}
		\label{table:params_2d}	
	\end{center}
\end{table}

In Table \ref{table:results} we present joint and marginal survival probabilities computed using the equations from Section \ref{section:joint}. From these results, we can conclude that jumps play an important role in the model.
\begin{table}[H]
	\begin{center}
		\begin{tabular}{|c | c | c | c | c | c | c | c |}
			\hline
			Model &Joint s/p & Marginal s/p   \\ 
			\hline
			With jumps & 0.9328 & 0.9666 \\
			Without jumps & 0.9717 & 0.9801 \\
			\hline
		\end{tabular}
		\caption{Joint and marginal survival probabilities for the calibrated models.}
		\label{table:results}	
	\end{center}
\end{table}


%\begin{table}[H]
%	\begin{center}
%		\begin{tabular}{|c | c | c | c | c | c | c | c |}
%			\hline
%			Model &Joint s/p & Marginal s/p   \\ 
%			\hline
%			0.0 & 0.0 & 0.8879 \\
%			2.0 & 3.0 &  0.8869 \\
%			3.0 & 2.0 &  0.8868\\
%			5.0 & 6.0 &  0.8861 \\
%			5.0 & 4.0 &  0.8810\\
%			5.0 & 0.0 &  0.8869\\
%			0.0 & 4.0 &  0.8866\\
%			\hline
%		\end{tabular}
%		\caption{Marginal survival probabilities for the calibrated models.}
%	\end{center}
%\end{table}

%
%K>> resid
%
%resid =
%
%    0.0167
%   -0.2809
%   -0.6476
%    1.0280
%
%K>> result
%
%result =
%
%    0.0180
%    0.0366
%    4.2949

%result =
%
%    0.0119
%    0.1089
%   30.5082
%
%K>> resid
%
%resid =
%
%   -0.0177
%    1.0682
%    0.6890
%   -1.6769


\section{Conclusion}
We have presented a neural performance rendering system to generate high-quality geometry and photo-realistic textures of human-object interaction activities in novel views using sparse RGB cameras only. 
%
Our layer-wise scene decoupling strategy enables explicit disentanglement of human and object for robust reconstruction and photo-realistic rendering under challenging occlusion caused by interactions. 
%
Specifically, the proposed implicit human-object capture scheme with occlusion-aware human implicit regression and human-aware object tracking enables consistent 4D human-object dynamic geometry reconstruction.
%
Additionally, our layer-wise human-object rendering scheme encodes the occlusion information and human motion priors to provide high-resolution and photo-realistic texture results of interaction activities in the novel views.
%
Extensive experimental results demonstrate the effectiveness of our approach for compelling performance capture and rendering in various challenging scenarios with human-object interactions under the sparse setting.
%
We believe that it is a critical step for dynamic reconstruction under human-object interactions and neural human performance analysis, with many potential applications in VR/AR, entertainment,  human behavior analysis and immersive telepresence.




\newpage
\appendix
\section{Pricing equations}
\subsection{Credit default swap}
\label{CDS_pricing}
A credit default swap (CDS) is a contract designed to exchange credit risk of a Reference Name (RN) between a Protection Buyer (PB) and a Protection Seller (PS). PB makes periodic coupon payments to PS conditional on no default of RN, up to the nearest payment date, in the exchange for receiving from PS the loss given RN's default.

Consider a CDS contract written on the first bank (RN), denote its price $C_1(t, x)$.\footnote{For the CDS contracts written on the second bank, the similar expression could be provided by analogy.} We assume that the coupon is paid continuously and equals to $c$. Then, the value of a standard CDS contract can be given (\cite{BieleckiRutkowski}) by the solution of  (\ref{kolm_1})--(\ref{kolm_2})  with $\chi(t, x) = c$ and terminal condition
\begin{equation*}
	\psi(x) = 
	\begin{cases}
		1 - \min(R_1, \tilde{R}_1(1)), \quad (x_1, x_2) \in D_2, \\
		1 - \min(R_1, \tilde{R}_1(\omega_2)), \quad (x_1, x_2) \in D_{12}, \\		
	\end{cases}
\end{equation*}
where $\omega_2 = \omega_2(x)$ is defined in (\ref{term_cond}) and 
\begin{equation*}
	\tilde{R}_1(\omega_2) = \min \left[1, \frac{A_1(T) +  \omega_2 L_{2 1}(T)}{L_1(T) + \omega_2 L_{12}(T)}\right].
\end{equation*}
Thus, the pricing problem for CDS contract on the first bank is
\begin{equation}
\begin{aligned}
		& \frac{\partial}{\partial t} C_1(t, x) + \mathcal{L} C_1(t, x) = c, \\
		& C_1(t, 0, x_2) = 1 - R_1, \quad C_1(t, \infty, x_2) = -c(T-t), \\
		& C_1(t, x_1, 0) = \Xi(t, x_1) = 
		\begin{cases}
			c_{1,0}(t, x_1), & x_1 \ge \tilde{\mu}_1, \\
			1-R_1, & x_1 < \tilde{\mu}_i,
		\end{cases} \quad C_1(t, x_1, \infty) = c_{1,\infty}(t, x_1),\\
		& C_1(T, x) = \psi(x) = 
	\begin{cases}
		1 - \min(R_1, \tilde{R}_1(1)), \quad (x_1, x_2) \in D_2, \\
		1 - \min(R_1, \tilde{R}_1(\omega_2)), \quad (x_1, x_2) \in D_{12}, \\		
	\end{cases}
\end{aligned}
\end{equation}
where $c_{1,0}(t, x_1)$ is the solution of the following boundary value problem:
\begin{equation}
\begin{aligned}
		& \frac{\partial}{\partial t} c_{1, 0}(t, x_1) + \mathcal{L}_1 c_{1, 0}(t, x_1) = c, \\
		& c_{1, 0}(t, \tilde{\mu}_1^{<}) = 1 - R_1, \quad c_{1, 0}(t, \infty) = -c(T-t), \\
		& c_{1, 0}(T, x_1) = (1 - R_1) \mathbbm{1}_{\{\tilde{\mu}_1^{<} \le x_1 \le \tilde{\mu}_1^{=}\}}, 
\end{aligned}
\end{equation}
and $c_{1,\infty}(t, x_1)$ is the solution of the following boundary value problem
\begin{equation}
\begin{aligned}
		& \frac{\partial}{\partial t} c_{1, \infty}(t, x_1) + \mathcal{L}_1 c_{1, \infty}(t, x_1) = c, \\
		& c_{1, \infty}(t, 0) = 1 - R_1, \quad c_{1, \infty}(t, \infty) = -c(T-t), \\
		& c_{1, \infty}(T, x_1) = (1 - R_1) \mathbbm{1}_{\{x_1 \le \mu_1^{=}\}}.
\end{aligned}
\end{equation}

\subsection{First-to-default swap}
An FTD contract refers to a basket of reference names (RN). Similar to a regular CDS, the Protection Buyer (PB) pays a regular coupon payment $c$ to the Protection Seller (PS) up to the first default of any of the RN in the basket or maturity time $T$. In return, PS compensates PB the loss caused by the first default.

Consider the FTD contract referenced on $2$ banks, and denote its price $F(t, x)$. We assume that the coupon is paid continuously and equals to $c$. Then, the value of FTD contract can be given (\cite{LiptonItkin2015}) by the solution of  (\ref{kolm_1})--(\ref{kolm_2})  with $\chi(t, x) = c$ and terminal condition
\begin{equation*}
	\psi(x) = \beta_0  \mathbbm{1}_{\{x \in D_{12}\}} + \beta_1 \mathbbm{1}_{\{x \in D_{1}\}} + \beta_2 \mathbbm{1}_{\{x \in D_{2}\}},
\end{equation*}
where
\begin{equation*}
	\begin{aligned}
		\beta_0 = 1 - \min[\min(R_1, \tilde{R}_1(\omega_2), \min(R_2, \tilde{R}_2(\omega_1)], \\
		\beta_1 = 1 - \min(R_2, \tilde{R}_2(1)), \quad \beta_2 = 1 - \min(R_1, \tilde{R}_1(1)),
	\end{aligned}
\end{equation*}
and
\begin{equation*}
	\tilde{R}_1(\omega_2) = \min \left[1, \frac{A_1(T) +  \omega_2 L_{2 1}(T)}{L_1(T) + \omega_2 L_{12}(T)}\right], \quad \tilde{R}_2(\omega_1) = \min \left[1, \frac{A_2(T) +  \omega_1 L_{1 2}(T)}{L_2(T) + \omega_1 L_{21}(T)}\right].
\end{equation*}
with $\omega_1 = \omega_1(x)$ and $\omega_2 = \omega_2(x)$ defined in (\ref{term_cond}).

Thus, the pricing problem for a FTD contract is
\begin{equation}
\begin{aligned}
		& \frac{\partial}{\partial t} F(t, x) + \mathcal{L} F(t, x) = c, \\
		& F(t, x_1, 0) = 1 - R_2,  \quad F(t, 0, x_2) = 1 - R_1, \\
		& F(t, x_1, \infty) = f_{2,\infty}(t, x_1), \quad F(t, \infty, x_2) = f_{1,\infty}(t, x_2), \\
		& F(T, x) = \beta_0  \mathbbm{1}_{\{x \in D_{12}\}} + \beta_1 \mathbbm{1}_{\{x \in D_{1}\}} + \beta_2 \mathbbm{1}_{\{x \in D_{2}\}},
\end{aligned}
\end{equation}
where $f_{1,\infty}(t, x_1)$ and $f_{2,\infty}(t, x_2)$ are the solutions of the following boundary value problems
\begin{equation}
\begin{aligned}
		& \frac{\partial}{\partial t} f_{i, \infty}(t, x_i) + \mathcal{L}_i f_{i, \infty}(t, x_i) = c, \\
		& f_{i, \infty}(t, 0) = 1 - R_i, \quad f_{i, \infty}(t, \infty) = -c(T-t), \\
		& f_{1, \infty}(T, x_i) = (1 - R_i) \mathbbm{1}_{\{x_i \le \mu_i^{=}\}}.
\end{aligned}
\end{equation}

\subsection{Credit and Debt Value Adjustments for CDS}

Credit Value Adjustment and Debt Value Adjustment can be considered either unilateral or bilateral. For unilateral counterparty risk, we need to consider only two banks (RN, and PS for CVA and PB for DVA), and a two-dimensional problem can be formulated, while bilateral counterparty risk requires a three-dimensional problem, where Reference Name, Protection Buyer, and Protection Seller are all taken into account. We follow \cite{LiptonSav} for the pricing problem formulation but include jumps and mutual liabilities, which affects the boundary conditions.

\paragraph{Unilateral CVA and DVA}
The Credit Value Adjustment represents the additional price associated with the possibility of a counterparty's default. Then, CVA can be defined as
\begin{equation}
	V^{CVA} = (1- R_{PS}) \mathbb{E}[\mathbbm{1}_{\{\tau^{PS} < \min(T, \tau^{RN}) \}} (V_{\tau^{PS}}^{CDS})^{+} \, | \mathcal{F}_t],
\end{equation}
where $R_{PS}$ is the recovery rate of PS, $\tau^{PS}$ and $\tau^{RN}$ are the default times of PS and RN, and $V_t^{CDS}$ is the price of a CDS without counterparty credit risk.

We associate $x_1$ with the Protection Seller and $x_2$ with the Reference Name, then CVA can be given by the solution of  (\ref{kolm_1})--(\ref{kolm_2})  with $\chi(t, x) = 0$ and $\psi(x) = 0$. Thus,
\begin{equation}
\begin{aligned}
		& \frac{\partial}{\partial t} V^{CVA}+ \mathcal{L} V^{CVA} = 0, \\
		& V^{CVA}(t, 0, x_2) = (1 - R_{PS}) V^{CDS}(t, x_2)^{+}, \quad V^{CVA}(t, x_1, 0) = 0, \\
		& V^{CVA}(T, x_1, x_2) = 0.
\end{aligned}
\end{equation}

Similar, Debt Value Adjustment represents the additional price associated with the default and defined as
\begin{equation}
	V^{DVA} = (1- R_{PB}) \mathbb{E}[\mathbbm{1}_{\{\tau^{PB} < \min(T, \tau^{RN}) \}} (V_{\tau^{PB}}^{CDS})^{-} \, | \mathcal{F}_t],
\end{equation}
where $R_{PB}$ and $\tau^{PB}$ are the recovery rate and default time of the protection buyer.

Here, we associate $x_1$ with the Protection Buyer and $x_2$ with the Reference Name, then, similar to CVA,  DVA can be given by the solution of  (\ref{kolm_1})--(\ref{kolm_2}),
\begin{equation}
\begin{aligned}
		& \frac{\partial}{\partial t} V^{DVA}+ \mathcal{L} V^{DVA} = 0, \\
		& V^{DVA}(t, 0, x_2) = (1 - R_{PB}) V^{CDS}(t, x_2)^{-}, \quad V^{DVA}(t, x_1, 0) = 0, \\
		& V^{DVA}(T, x_1, x_2) = 0.
\end{aligned}
\end{equation}

\paragraph{Bilateral CVA and DVA}

When we defined unilateral CVA and DVA, we assumed that either protection  buyer, or protection seller are risk-free. Here we assume that they are both risky. Then, 
The Credit Value Adjustment represents the additional price associated with the possibility of counterparty's default and defined as
\begin{equation}
	V^{CVA} = (1 - R_{PS}) \mathbb{E}[\mathbbm{1}_{\{\tau^{PS} < \min(\tau^{PB}, \tau^{RN}, T)\}} (V^{CDS}_{\tau^{PS}})^{+} \, | \mathcal{F}_t],
\end{equation} 

Similar, for DVA
\begin{equation}
	V^{DVA} = (1 - R_{PB}) \mathbb{E}[\mathbbm{1}_{\{\tau^{PB} < \min(\tau^{PS}, \tau^{RN}, T)\}} (V^{CDS}_{\tau^{PB}})^{-} \, | \mathcal{F}_t],
\end{equation} 


We associate $x_1$ with protection seller, $x_2$ with protection buyer, and $x_3$ with reference name. Here, we have a three-dimensional process. Applying three-dimensional version of (\ref{kolm_1})--(\ref{kolm_2}) with $\psi(x) = 0, \chi(t, x) = 0$, we get
\begin{equation}
	\label{CVA_pde}
\begin{aligned}
		& \frac{\partial}{\partial t} V^{CVA} + \mathcal{L}_3 V^{CVA} = 0, \\
		& V^{CVA}(t, 0, x_2, x_3) = (1 - R_{PS}) V^{CDS}(t, x_3)^{+}, \\
		& V^{CVA}(t, x_1, 0, x_3 ) = 0, \quad V^{CVA}(t, x_1, x_2, 0)  = 0, \\
		& V^{CVA}(T, x_1, x_2, x_3) = 0,
\end{aligned}
\end{equation}
and
\begin{equation}
\label{DVA_pde}
\begin{aligned}
		& \frac{\partial}{\partial t} V^{DVA} + \mathcal{L}_3 V^{DVA} = 0, \\
		& V^{DVA}(t, 0, x_2, x_3) = (1 - R_{PB}) V^{CDS}(t, x_3)^{-}, \\
		& V^{DVA}(t, x_1, 0, x_3 ) = 0, \quad V^{DVA}(t, x_1, x_2, 0)  = 0, \\
		& V^{DVA}(T, x_1, x_2, x_3) = 0,
\end{aligned}
\end{equation}
where $\mathcal{L}_3 f$ is the three-dimensional infinitesimal generator.



\bibliographystyle{apalike}
\bibliography{lit_bib} 



\end{document}
