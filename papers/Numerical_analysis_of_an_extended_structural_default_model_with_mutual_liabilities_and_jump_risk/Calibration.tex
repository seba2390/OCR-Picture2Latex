\section{Calibration}

In this section we present calibration results of the model. There are eight unknown parameters, see \eqref{kolmogorov_backward}--\eqref{j12_eq}: $\sigma_1, \sigma_2, \rho, \varsigma_1, \varsigma_2, \lambda_1, \lambda_2, \lambda_{12}$. We use CDS and equity put option prices (with different strikes) as market data. If FTD contracts are available, one can use them to estimate $\rho$ and $\lambda_{12}$. Otherwise, historical estimation with share prices time series can be used. 

The data for external liabilities can be found in banks' balance sheets, which are publicly available. Usually, mutual liabilities data are not public information, thus we made an assumption that they are a fixed proportion of the total liabilities, which coincides with \cite{DavidLehar}. In particular, we fix the mutual liabilities as 5\% of total liabilities.

The asset's value is the sum of liabilities and equity price.

We choose Unicredit Bank as the first bank and Santander as the second bank. In Table \ref{data_table} we provide their equity price $E_i$, assets $A_i$ and liabilities $L_i$. As in \cite{LiptonSepp}, the liabilities are computed as a ratio of total liabilities and shares outstanding.

\begin{table}[H]
	\begin{center}
		\begin{tabular}{| c | c | c | c | c | c |}
			\hline
			$E_1(0)$ & $L_1(0)$ & $A_1(0)$ & $E_2(0)$ & $L_2(0)$ & $A_2(0)$  \\ 
			\hline
			6.02&  137.70& 143.72& 6.23 & 86.41 & 92.64 \\
			\hline
		\end{tabular}
		\caption{Assets and liabilities on 30/06/2015 (Bloomberg).}
		\label{data_table}	
	\end{center}
\end{table}
For the calibration we choose 1-year at-the-money, in-the-money, and out-of-the-money equity put options on the banks, and 1-year CDS contracts. Since the spreads of CDS are usually significantly lower than the option prices, we scale them by some weight $w_i$ in the objective function. As a result, we have the following 6-dimensional minimization problem:
\begin{multline}
	\label{calibration_eq}
	\min_{\theta} \{ w_1 (V^{CDS}_1(\theta) - \bar{V}^{CDS}_1)^2 + \sum \limits_{i = 1}^3 (V^{opt}_1(K_{i, 1}, \theta) - \bar{V}^{opt}_1(K_{i, 1}))^2 + \\
	+ w_2 (V^{CDS}_2(\theta) - \bar{V}^{CDS}_2)^2  + \sum \limits_{i = 1}^3 (V^{opt}_2(K_{i, 2}, \theta) - \bar{V}^{opt}_2(K_{i, 2}))^2  \},
\end{multline}
 where $\theta = (\sigma_1, \sigma_2, \lambda_1, \lambda_2, \varsigma_1, \varsigma_2)$, $V^{CDS}_i(\theta)$ is the model CDS spread on the $i$-th bank and $\bar{V}^{CDS}_i$ is the market CDS spread on the $i$-th bank, $V^{opt}_1(K, \theta)$ is the model price of the equity put option on the $i$-th bank with the strike $K$ and $\bar{V}^{opt}_i(K)$ is the market price of the equity put option on the $i$-th bank with strike $K$. Strikes $K_{1, j}, K_{2, j}$, and $K_{3, j}$ are chosen in such a way to take into account the smile. In particular, we choose $K_{1, j} = 1.1 E_j, K_{2, j} = E_j, K_{3, j} = 0.9 E_j$.

In order to find the global minimum of \eqref{calibration_eq} by a Newton-type method, we need to find a good starting point, otherwise an optmization procedure might finish in local minima which are not global minima. To choose the starting point, we calibrate one-dimensional models for each bank without mutual liabilities
\begin{multline}
	\label{calibration_eq1d}
	\min_{\theta_j} \{ w_j (V^{CDS}_j(\theta_j) - \bar{V}^{CDS}_j)^2 + (V^{opt}_j(K_{1, j}, \theta_j) - \bar{V}^{opt}_j(K_{1, j}))^2 + \\
	+(V^{opt}_j(K_{2, j}, \theta_j) - \bar{V}^{opt}_i(K_{2, j}))^2  + (V^{opt}_j(K_{3, j}, \theta_j) - \bar{V}^{opt}_j(K_{3, j}))^2  \},
\end{multline}
where $\theta_j = (\sigma_j, \lambda_j, \varsigma_j)$ for $j = 1, 2$.

The global minima of \eqref{calibration_eq1d} can be found via the {\bf{chebfun toolbox}} (\cite{Trefethen}) that uses Chebyshev polynomials to approximate the function, and then the global minima can be easily found. The calibration results of the one-dimensional model for the first and the second banks are presented in Table \ref{table:params_1d}. We note that the global minima of \eqref{calibration_eq} cannot be found via the chebfun toolbox, since it works with functions up to three variables. There are also more fundamental complexity issues for higher-dimensional tensor product interpolation.

\begin{table}[H]
	\begin{center}
		\begin{tabular}{| c | c | c | c | c | c |}
			\hline
			$\sigma_1$ & $\lambda_1$ & $\varsigma_1$ & $\sigma_2$ & $\lambda_2$ & $\varsigma_2$  \\ 
			\hline
			 0.0117&  0.1001& 0.3661& 0.0154 & 0.0160 & 0.0545\\
			\hline
		\end{tabular}
		\caption{Calibrated parameters of one-dimensional models on 30/06/2015 for $T = 1$.}
		\label{table:params_1d}	
	\end{center}
\end{table}

Similar to \cite{LiptonSepp}, for simplicity, we further assume that 
\begin{equation}
	\lambda_{\{12\}} = \rho \cdot \min(\lambda_1, \lambda_2).
	\label{lambda_assumption}
\end{equation}
 Then, we estimate $\rho$ from historical data. We take one year daily  equity prices $E_i(t)$ by time series (from Bloomberg) and estimate the covariance of asset returns $r_t^i = \frac{\Delta A_i(t)}{A_i(t)}$
\begin{equation}
	\widehat{\cov}(A_1, A_2) = \sum \limits_{i = 1}^n \left(r_{i, 1} - \bar{r_1} \right)\left(r_{i, 2} - \bar{r}_2  \right),
	\label{cov_est}
\end{equation}
where $\bar{r}_1$ and $\bar{r}_2$ are the sample mean of asset returns.

Using \eqref{assets_dynamics}, we can see that \eqref{cov_est} converges to
\begin{equation}
	\widehat{\cov}(A_1, A_2) \underset{n \to +\infty}{\longrightarrow} \sigma_1 \sigma_2 \left( \rho+ \lambda_{\{12\}} /(\varsigma_1 \varsigma_2) \right).
\end{equation}
Using the last equation and \eqref{lambda_assumption}, we can extract the estimated values of $\rho$ and $\lambda_{\{12\}}$. The estimation results are in Table \ref{table:corr_params}.
\begin{table}[H]
	\begin{center}
		\begin{tabular}{| c | c | c | }
			\hline
			& $\rho$ & $\lambda_{\{12\}} $ \\
			\hline
			Estimated value & 0.510 & 0.0188 \\
			\hline
			Confidence interval \footnotemark & (0.500, 0.526)& (0.0182, 0.0194) \\
			\hline
		\end{tabular}
		\caption{Historically estimated correlation coefficients on 30/06/2015 with 1 year window.}
		\label{table:corr_params}	
	\end{center}
\end{table}
\footnotetext{We use a $3 \sigma$ confidence interval.}

Finally, we perform a six-dimensional (constrained) optimization of \eqref{calibration_eq} with the starting point from Table \ref{table:params_1d} and correlation parameters from Table \ref{table:corr_params}. We choose different alternatives of mutual liabilities to have a clear picture how mutual liabilities influence on model parameters. We use the {\bf lsqnonlin} method in Matlab that uses a Trust Region Reflective algorithm \cite{conn2000trust} (with the gradient computed numerically). The model CDS spreads are computed using the method in Section \ref{CDS_pricing}, while equity option prices are computed in the usual finite-difference manner (see \cite{LiptonSepp} for details).  Results are presented in Table \ref{table:params_2d}.
%\begin{table}[H]
%	\begin{center}
%		\begin{tabular}{|c | c | c | c | c | c | c | c |}
%			\hline
%			$L_{12}$ & $L_{21}$ & $\sigma_1$ & $\lambda_1$ & $\varsigma_1$ & $\sigma_2$ & $\lambda_2$ & $\varsigma_2$  \\ 
%			\hline
%			0.0 & 0.0 & 0.0117&  0.1001& 0.3661& 0.0154 & 0.0160 & 0.0545 \\
%			2.0 & 3.0 & 0.0119 & 0.1012 & 0.3968 & 0.0153 & 0.0153 & 0.0517 \\
%			3.0 & 2.0 & 0.0119 & 0.0976 & 0.3841 & 0.0156 & 0.0154 & 0.0522 \\
%			5.0 & 6.0 & 0.0122 & 0.1021 & 0.4233 & 0.0154 & 0.0149 & 0.0491 \\
%			5.0 & 4.0 & 0.0120 & 0.1079 & 0.4212 & 0.0155 & 0.0149 & 0.0496 \\
%			5.0 & 0.0 & 0.0117 & 0.0989 & 0.3627 & 0.0160 & 0.0151 & 0.0527 \\
%			0.0 & 4.0 & 0.0117 & 0.0993 & 0.3796 & 0.0154 & 0.0145 & 0.0522 \\
%			\hline
%		\end{tabular}
%		\caption{Calibrated parameters of two-dimensional model with mutual liabilities on 30/06/2015 for $T = 1$.}
%		\label{table:params_2d}	
%	\end{center}
%\end{table}

\begin{table}[H]
	\begin{center}
		\begin{tabular}{|c | c | c | c | c | c | c | c |}
			\hline
			 Model & $\sigma_1$ & $\lambda_1$ & $\varsigma_1$ & $\sigma_2$ & $\lambda_2$ & $\varsigma_2$  \\ 
			\hline
			With jumps & 0.0122&  0.0950& 0.3958& 0.0160 & 0.0148 & 0.0505 \\
			Without jumps & 0.0206 & -- & -- & 0.0317 & -- & -- \\
			\hline
		\end{tabular}
		\caption{Calibrated parameters of two-dimensional model with mutual liabilities on 30/06/2015 for $T = 1$.}
		\label{table:params_2d}	
	\end{center}
\end{table}

In Table \ref{table:results} we present joint and marginal survival probabilities computed using the equations from Section \ref{section:joint}. From these results, we can conclude that jumps play an important role in the model.
\begin{table}[H]
	\begin{center}
		\begin{tabular}{|c | c | c | c | c | c | c | c |}
			\hline
			Model &Joint s/p & Marginal s/p   \\ 
			\hline
			With jumps & 0.9328 & 0.9666 \\
			Without jumps & 0.9717 & 0.9801 \\
			\hline
		\end{tabular}
		\caption{Joint and marginal survival probabilities for the calibrated models.}
		\label{table:results}	
	\end{center}
\end{table}


%\begin{table}[H]
%	\begin{center}
%		\begin{tabular}{|c | c | c | c | c | c | c | c |}
%			\hline
%			Model &Joint s/p & Marginal s/p   \\ 
%			\hline
%			0.0 & 0.0 & 0.8879 \\
%			2.0 & 3.0 &  0.8869 \\
%			3.0 & 2.0 &  0.8868\\
%			5.0 & 6.0 &  0.8861 \\
%			5.0 & 4.0 &  0.8810\\
%			5.0 & 0.0 &  0.8869\\
%			0.0 & 4.0 &  0.8866\\
%			\hline
%		\end{tabular}
%		\caption{Marginal survival probabilities for the calibrated models.}
%	\end{center}
%\end{table}

%
%K>> resid
%
%resid =
%
%    0.0167
%   -0.2809
%   -0.6476
%    1.0280
%
%K>> result
%
%result =
%
%    0.0180
%    0.0366
%    4.2949

%result =
%
%    0.0119
%    0.1089
%   30.5082
%
%K>> resid
%
%resid =
%
%   -0.0177
%    1.0682
%    0.6890
%   -1.6769

