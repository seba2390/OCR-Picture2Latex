\section{Introduction}




The estimation and mitigation of counterparty credit risk %and the estimation 
has become a pillar of financial risk management.
The impact of such risks on financial derivatives is explicitly acknowledged by a valuation adjustment.
For an exposition of the background and mathematical models we refer the reader to \cite{gregory2012counterparty}.
Although reduced-from models provide for a more direct simulation of default events and are commonly used in financial institutions, we follow here a structural approach which maps the capital structure of a bank into stochastic processes for equity and debt, and models default as the hitting of a lower barrier, as in \cite{BlackCox}.
An extensive literature review of further developments of this model is given in \cite{LiptonSepp}.

A particular concern to regulators and central banks is the impact of default of an entity on the financial system -- credit contagion. Of the various channels of such systemic risk described in \cite{hurd2016contagion}, we focus here on dependencies through asset correlation and interbank liabilities.
Specifically, we consider the extended structural default model introduced in \cite{Lipton2015}, where asset values are assumed to follow stochastic processes with correlated diffusion and jump components, and where mutual liabilities can lead to default cascades.
%There are two main approaches for credit risk modeling: reduced-form and structural. Reduced-form approach does not infer any information about the default event and assume that it comes by ``surprise''.  While in structural approach the default event is modeled based on the capital structure of the banks.
%In classical structural models, banks consider only their own assets and liabilities, and, as a result, their capital structures (and thus the default events) are independent on each other.  \cite{Lipton2015} extended the structural default framework by taking into account the fact that banks have mutual liabilities to each other.


%The structural default framework is widely used for assessing credit risk of corporate debt. Introduced in its simplest form in the seminal work of \cite{Merton}, this framework was further extended in various papers, see a survey in \cite{LiptonSepp} and references therein. In contrast to reduced-form models, structural default models suffer from the curse of dimensionality when the number of counterparties grows; however, these models provide a more natural financial description of the default event for a typical firm.
%
%Inspired by \cite{webber2011systemic},  \cite{Lipton2015} extended the structural default framework by taking into account the fact that banks have mutual liabilities to each other. Taking this effect into consideration is very important in order to accurately analyze credit worthiness of individual banks and the banking network as a whole. For instance, large mutual liabilities imply that an adverse shock to a bank is rapidly transmitted to the entire system, with severe implications for its stability.  \cite{DavidLehar} indicate that renegotiations between highly interconnected banks facilitate mutual private sector bailouts to lower the need for government bailouts. The relative size of mutual liabilities compared to total liabilities is quite significant. For instance, the relative fraction of interbank loans is 12\% in the EU and 8.5\% for Canada (\cite{DavidLehar}), and 4.5\% for US (as per Economic Research website of the Federal Reserve Bank of St. Louis).

\cite{LiptonItkin2015} consider the model without jumps and obtain explicit expressions for several quantities of interest including the joint and marginal survival probabilities as well as CDSs and FTD prices.
They demonstrate that mutual liabilities can have a large impact on the survival probabilities of banks. Thus, a shock of one bank can cause ripples through the whole banking system. 

We focus here on the numerical computation of survival probabilities and credit products in the extended model with jumps, where closed-from expressions are no longer available.
Our work is therefore most similar to \cite{LiptonItkin2014}, who 
develop a finite difference method for the resulting partial integro-differential equation (PIDE) where the integral term results from a fairly general correlated Levy process in the jump component. % using a special splitting scheme.
By Strang splitting into the diffusion and jump operators, overall second order consistency in the timestep is obtained.
% even though these operators due not commute.
Hereby, the multi-dimensional diffusion operator is itself split dimensionwise using the Hundsdorfer-Verwer (HV) scheme, and the jump operator is 
treated as a pseudo-differential operator, which allows efficient evaluation of the discretised operator by an iterative procedure.
Stability of each of the steps is guaranteed under standard conditions.

Our approach is more straightforward in that we apply a modification of the HV scheme directly, where we treat the jump term in the same way as the cross-derivative term in the classical HV scheme.
For the analysis we consider infinite meshes, i.e.\ ignoring the boundaries,
%, where a zero Dirichlet boundary condition is applied at the default boundary, 
such that the discrete operators are also infinite-dimensional.
In our analysis we build heavily on the results in \cite{intHoutStability} on stability of the PDE with cross-derivatives (but no integral term) and periodic boundary conditions on a finite mesh.

We show that the (unconditional) von Neumann stability of the scheme is not materially affected by the jump operator, as its contribution to the symbol of the scheme is of a lower order in the mesh size.
For concreteness, we restrict ourselves to the model with negative exponential jumps described in \cite{Lipton2015}.
This allows a simple recursive computation of the discrete jump operator and gives an explicit form of its eigenvalues. However, the analysis can in principle be extended to other jump size distributions.

%However, as the discretised operators do not commute -- through the presence of absorbing boundary conditions -- the operators
%are not simultaneously diagonalisable and the problem remains that the eigenvalues of the combined splitting operator are not directly obtained from the eigenvalues of the individual operators. By showing and using that the commutator is of low rank, we are still able to establish that the operators are simultaneously triangularisable. We thereby obtain the eigenvalues of the combined operator and can show that the scheme is stable in the von Neumann sense.\footnote{Incidentally, this idea can also be applied to the schemes in  \cite{LiptonItkin2014}.}

A survey of splitting methods in finance is found in \cite{toivanen2015application}. These are roughly arranged in two groups: splitting by dimension (for multi-dimensional PDEs; such as in \cite{intHoutStability}), and splitting by operator type 
(for PIDEs, diffusion and jumps; such as in \cite{andersen2000jump}). 
\cite{LiptonItkin2014} perform these two splittings successively as described above.
To our knowledge, the present paper is the first to perform and analyse splitting into dimensions and jumps simultaneously.
%Even though this is a special case of the model considered in \cite{LiptonItkin2014}, it requires a special treatment that can be done in more sophisticated way. 

The scheme is constructed to be second order consistent with the continuous integro-differential operator applied to smooth functions. However, the discontinuities in the data lead to empirically observed convergence of only first order in both space and time step.
To address this, we apply a spatial smoothing technique discussed in  \cite{pooley2003} for discontinuous option payoffs in the one-dimensional setting, and a change of the time variable to square-root time (see \cite{reisinger2013}), equivalent to a quadratically refined time mesh close to maturity, in order to restore second order convergence.
We emphasise that the presence of %absorbing boundary conditions and 
discontinuous initial data is essential to the nature of P(I)DE models of credit risk. Hence this approach improves on previous works in a key aspect of the numerical solution.

Similar to \cite{LiptonItkin2015}, we restrict the analysis to the two-dimensional case, but there is no fundamental problem in extending the method to multiple dimensions. However, due to the curse of dimensionality, for more than three-dimensional problems, standard finite-difference methods are computationally too expensive. The two-dimensional case already allows us to investigate various important model characteristics, such as joint and marginal survival probabilities, prices of credit derivatives, Credit and Debt Value Adjustments, and specifically the impact of mutual obligations.

In this paper, we consider both unilateral and bilateral counterparty risk as discussed in \cite{LiptonSav}. For the unilateral case, the model with two banks is considered, where one is a reference name and the other is either a protection buyer or a protection seller, while for the bilateral case, reference name, protection seller, and protection buyer are considered together, which leads to a three-dimensional problem. We give the equations in the Appendix, but do not include computations.

Moreover, we provide a calibration of the model with negative exponential jumps to market data, and for this calibrated model assess the impact of mutual obligations on survival probabilities.

%can be calibrated much easier than for general Levy processes. 
%\noindent
%{\bf Highlights:}
The novel results of this paper are as follows:
\begin{itemize}
\item We analyze a two-dimensional structural default model with interbank liabilities and negative exponential jumps; in particular, we calibrate the model to the market and analyze the impact of jumps on joint and marginal survival probabilities; 
\item we develop a new finite-difference method to solve the multidimensional PIDE, which is second order consistent in both time and space variables; 
\item we prove the von Neumann and $l_2$ stability of the method; % extending \cite{intHoutStability} from PDEs to PIDEs;
%to our knowledge, this is the first result on stability of a splitting scheme for this type of multi-dimensional PIDE taking into account Dirichlet boundary conditions;
\item we demonstrate empirically that in the presence of discontinuous terminal and boundary conditions, second order of convergence can be maintained by local averaging of the data and suitable refinement of the timestep close to maturity.
\end{itemize}


The rest of the paper is organized as follows. In Section 2, we formulate the model for two banks with jumps, which is a simplified formulation of \cite{Lipton2015} for two banks only. Then, we briefly discuss how to compute various model characteristics. In Section 3 we propose a numerical scheme for a general pricing problem; we further prove its stability and consistency. In Section 4 we provide numerical results for various model characteristics computed with the numerical scheme from Section 3. In Section 5 we calibrate the model to the market, and in Section 6 we conclude.