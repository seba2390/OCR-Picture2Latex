% !TEX root = WeeBac17.tex
\section{Omitted proofs}\label{appendix:omitted}
\subsection{Proof of Proposition~\ref{prop:wp_dyadic_bound}}
We begin by giving an informal outline of the idea of the proof.

Consider a partition $\{Q_i\}_{i \in \cI}$ of $X$, for some index set $\cI$.
The measures $\mu$ and $\nu$ both induce measures on each set in the partition.
We will transport $\mu$ to $\nu$ by first moving mass \emph{between} sets in this partition, and then moving mass \emph{within} each set in the partition.
If $\mu(Q_i) \neq \nu(Q_i)$ for one of the sets $Q_i$, we we need to transport an amount of mass equal to $|\mu(Q_i) - \nu(Q_i)|$ into or out of $Q_i$.
In total, we can transport the mass that $\mu$ assigns to each set in the partition to its proper set under $\nu$ for a total cost of
\begin{equation*}
\sum_{i \in \cI} |\mu(Q_i) - \nu(Q_i)| \diam(S) \leq \sum_{i \in \cI} |\mu(Q_i) - \nu(Q_i)|\,,
\end{equation*}
where we use the fact that $\diam(S) \leq \diam(X) \leq 1$ by assumption.

After the first step of the transport plan, $\mu$ has been transported so that each set in the partition contains the correct total amount of mass.
It therefore suffices in the second step to properly arrange the mass \emph{within} each set.
Moving the mass within $Q_i$ cannot cost more than $\diam(Q_i)$, so the total cost of arranging the mass within each set is at most
\begin{equation*}
\sum_{i \in \cI} \nu(Q_i) \diam(Q_i) \leq \max_{i \in \cI} \diam(Q_i)\,.
\end{equation*}

We have obtained a transport of $\mu$ to $\nu$ for a total cost of approximately
\begin{equation*}
\max_{i \in \cI} \diam(Q_i) + \sum_{i \in \cI} |\mu(Q_i) - \nu(Q_i)|\,.
\end{equation*}

This ``single scale'' bound is generally not tight, but a more refined bound can be obtained by applying the above argument recursively: instead of naively bounding the cost of moving the mass within $Q_i$ by the quantity $\diam(Q_i)$, we can partition $Q_i$ into smaller sets  and estimate the cost of moving the mass within $Q_i$ by first moving it between the sets of the partition before moving it within each smaller set.
Iterating the argument $k^*$ times yields the bound.

We now show how to make the above argument precise.
Given two measures $\mu$ and $\nu$ on $X$, write $\cC(\mu, \nu)$ for the set of couplings between~$\mu$ and~$\nu$; that is, for the set of measures on $X \times X$ whose projection onto the first and second coordinate correspond to $\mu$ and $\nu$ respectively.

Fix a $k^* \geq 1$.
We will define two sequences of measure $\pi_k$ and $\rho_k$ on $X$ for $1 \leq k \leq k^*$ such that $\sum_{k=1}^{k^*} \pi_k \leq \mu$ and $\sum_{k=1}^{k^*} \rho_k \leq \nu$.
Given such a sequence, we set $\mu_1 = \mu$ and $\nu_1 = \nu$ and write
\begin{align*}
\mu_k  & = \mu - \sum_{\ell=1}^{k-1} \pi_\ell \\
\nu_k & = \nu - \sum_{\ell =1}^{k-1} \rho_\ell
\end{align*}
for $k \leq k^*+1$.

Note that if $\gamma_k \in \cC(\pi_k, \rho_k)$ for $1 \leq k \leq k^*$ and $\gamma_{k^*+1} \in \cC(\mu_{k^*+1}, \nu_{k^*+1})$, then
\begin{equation*}
\sum_{k=1}^{k^*+1} \gamma_k \in \cC\left(\sum_{k=1}^{k^*} \pi_k + \mu_{k^*+1}, \sum_{k=1}^{k^*} \rho_k + \nu_{k^*+1}\right) = \cC(\mu, \nu)\,,
\end{equation*}
therefore
\begin{equation*}
W_p^p (\mu, \nu) \leq \sum_{k=1}^{k^*} W_p^p(\pi_k, \rho_k) + W_p^p(\mu_{k^*+1}, \nu_{k^*+1})\,.
\end{equation*}

For $k \geq 1$, define
\begin{align*}
\pi_k &= \sum_{\substack{Q_i^k \in \cQ^k\\\mu_k(Q_i^k) > 0}} \left(1 - \frac{\nu_k(Q_i^k)}{\mu_k(Q_i^k)} \right)_+ \mu_k|_{Q_i^k}\,, \\
\rho_k &= \sum_{\substack{Q_i^k \in \cQ^k\\\nu_k(Q_i^k) > 0}} \left(1 - \frac{\mu_k(Q_i^k)}{\nu_k(Q_i^k)} \right)_+ \nu_k|_{Q_i^k}\,.
\end{align*}

Note that $0 \leq \pi_k \leq \mu_k$ and $0 \leq \rho_k \leq \nu_k$ for all $k$, hence $0 \leq \mu_k \leq \mu$ and $0 \leq \nu_k \leq \nu$ for all $k$ as well.

\begin{applemma}\label{lem:same_mass_pi_rho}
If $Q \in \cQ^{k-1}$, then
\begin{equation*}
\pi_{k}(Q) = \rho_{k}(Q)\,.
\end{equation*}
Moreover,
\begin{equation*}
\pi_k(S) = \rho_k(S) \leq \sum_{Q_i^k \in \cQ^k} |\mu(Q_i^k) - \nu(Q_i^k)|\,.
\end{equation*}
\end{applemma}

%\begin{applemma}\label{lem:total_mass}
%\begin{equation*}
%\pi_k(S) = \rho_k(S) \leq \sum_{Q_i^k \in \cQ^k} |\mu(Q_i^k) - \nu(Q_i^k)|
%\end{equation*}
%\end{applemma}

\begin{applemma}\label{lem:diameter_bound}
If $\alpha$ and $\beta$ are two measures on $X$ such that
\begin{equation*}
\alpha(Q) = \beta(Q)
\end{equation*}
for all $Q \in \cQ^{k}$, then
\begin{equation*}
W_p^p(\alpha, \beta) \leq \delta^{kp} \alpha(S)\,.
\end{equation*}
\end{applemma}
We can now obtain the final bound.
By Lemmas~\ref{lem:same_mass_pi_rho} and~\ref{lem:diameter_bound},
\begin{equation*}
W_p^p(\pi_k, \rho_k) \leq \delta^{(k-1)p} \sum_{Q_i^k \in \cQ^k} |\mu(Q_i^k) - \nu(Q_i^k)|
\end{equation*}
and
\begin{equation*}
W_p^p(\mu_{k^*+1}, \mu_{k^* + 1}) \leq \delta^{k^* p} \mu_{k^*+1}(S) \leq \delta^{k^* p} \mu(S) \leq \delta^{k^* p}\,.
\end{equation*}
The bound follows.
\qed

\subsection{Proof of Proposition~\ref{prop:ordering}}
We prove the inequalities in order.
If $d < d_H(\mu)$, then by~\cite[Proposition~10.3]{falconer1997techniques} there exists a compact set $K$ with positive mass and a $r_0 > 0$ such that
\begin{equation*}
\mu(B(x, r)) \leq r^d
\end{equation*}
for all $r \leq r_0$ and all $x \in K$.
(See also the proof of~\cite[Corollary~12.16]{GraLus07}.)
Let~$\tau < \mu(K)/2$.
If $S$ is any set with $\mu(S) \geq 1- \tau$, then $\mu(S \cap K) > \mu(K)/2$.
If $\cN_\ep(S) = N$, then in particular there exists a covering of $S \cap K$ by at most $N$ balls of radius $\ep$ whose centers all lie in $K$.
Indeed, any set of diameter at most $\ep$ which intersects $S \cap K$ is contained in a ball of radius $\ep$ whose center is in $K$.
If $\ep \leq r_0$, then each such ball satisfies $\mu(B(x, r)) \leq \ep^d$, so
\begin{equation*}
N \geq \ep^{-d} \mu(K)/2\,.
\end{equation*}
We therefore have for all $\tau$ sufficiently small,
\begin{equation*}
\liminf_{\ep \to 0} \frac{\log \cN_\ep'(\mu, \tau)}{- \log \ep} \geq d\,.
\end{equation*}
Thus $d_*(\mu) \geq d$.
Since $d < d_H(\mu)$ was arbitrary, we have $d_H(\mu) \leq d_*(\mu)$, as desired.

That $d_*(\mu) \leq d^*_p(\mu)$ follows from the simple observation that for all positive $\alpha$ and $\tau$,
\begin{equation*}
\liminf_{\ep \to 0} d_\ep(\mu, \tau) \leq \liminf_{\ep \to 0} d_\ep(\mu, \ep^{\alpha})\,.
\end{equation*}

Finally, if $d_M(\mu) \geq 2p$, then setting $s > d_M(\mu)$ yields
\begin{equation*}
\limsup_{\ep \to 0} d_\ep(\mu, \ep^{\frac{sp}{s - 2p}}) \leq \limsup_{\ep \to 0} d_\ep(\mu) = d_M(\mu) < s\,,
\end{equation*}
so $d_p^*(\mu) \leq s$.
Since $s > d_M(\mu)$ was arbitrary, we obtain $d_p^*(\mu) \leq d_M(\mu)$.
\qed
\subsection{Proof of Proposition~\ref{prop:covering_to_dyadic}}
Write $N_k = \cN_{3^{-(k+1)}}(S_k)$.
For $1 \leq k \leq k^*$, let $C^{k} = \{C^{k}_1, \dots\}$ be a finite covering of $S$ by balls of diameter $3^{-(k+1)}$ such that $C^{k}_1, \dots, C^{k}_{N_k}$ covers $S_k$.
Such a covering can always be found by choosing an optimal covering of $S_k$ and extending this covering to a covering of all of $S$.
Since $\cN_{3^{-(k^* + 1)}}(S) < \infty$, this requires only a finite number of additional balls.

We begin by constructing $\cQ^{k^*}$.
Let $\cQ^{k^*}_1 = C^{k^*}_1$, and for $1 < \ell \leq |C^{k^*}|$ let
\begin{equation*}
\cQ^{k^*}_\ell = C^{k^*}_\ell \setminus \left(\bigcup_{n = 1}^{\ell -1} \cQ^{k^*}_n \right)\,.
\end{equation*}
Let $\cQ^{k^*} = \{\cQ^{k^*}_1, \dots \}$.
Note that $\diam(\cQ^{k^*}_\ell) \leq \diam(C^{k^*}_\ell) = 3^{-(k^*+1)} < 3^{-k^*}$, that $\cQ^{k^*}$ forms a partition of $S$, and that at most $N_{k^*}$ elements of $\cQ^{k^*}$ intersect $S^{k^*}$.

We now show how to construct $\cQ^{k}$ from $\cQ^{k+1}$ and $C^k$.
Let
\begin{equation*}
\cQ^k_1 = \bigcup_{\substack{Q \in \cQ^{k+1} \\ Q \cap C^k_1 \neq \emptyset}} Q\,,
\end{equation*}
and for $1 < \ell \leq |C^{k^*}|$ let
\begin{equation*}
\cQ^k_\ell = \Big(\bigcup_{\substack{Q \in \cQ^{k+1} \\ Q \cap C^k_\ell \neq \emptyset}} Q\Big) \setminus \Big(\bigcup_{n = 1}^{\ell -1} \cQ^{k}_n \Big)\,.
\end{equation*}
Let $\cQ^{k} = \{\cQ^{k}_1, \dots \}$.

The sets in $\cQ^k$ clearly form a partition of $S$, and by construction at most $N_k$ elements of $\cQ^k$ intersect $S^k$
Moreover, since $\diam(C^k_\ell) \leq 3^{-(k+1)}$ for all $\ell$ and $\diam(Q) \leq 3^{-(k+1)}$ for all $Q \in \cQ^{k+1}$, the distance between any two points in $\cQ^k_\ell$ is at most $3 \cdot 3^{-(k+1)} = 3^{-k}$, so each element of $\cQ^k$ has diameter at most $3^{-k}$.
Finally, since each set in $\cQ^k$ is the union of sets in $\cQ^{k+1}$, the partition $\cQ^{k+1}$ refines $\cQ^k$, as desired.
\qed

\subsection{Proof of Proposition~\ref{prop:abs_cont}}
The only inequality that does not follow from Proposition~\ref{prop:ordering} is the first.
By absolute continuity, for all $\tau > 0$ there exists a $\sigma > 0$ such that any set $T$ for which $\mu(T) \geq 1- \tau$ satisfies $\cH^d(T) \geq \sigma$.
If $\cH^d(T) \geq \sigma$, then in particular for any covering $\{B(x_i, \ep)\}$ of $T$ by balls of radius $\ep$, we must have $\sum_{i} \ep^d \geq \sigma$.
Therefore such a covering contains at least $\sigma \ep^{-d}$ balls, so
\begin{equation*}
\frac{\log \cN_\ep(\mu, \tau)}{- \log \ep} \geq d + \frac{\log \sigma}{- \log \ep}\,,
\end{equation*}
and taking limits yields that $d_*(\mu) \geq d$, as desired.
\qed



\subsection{Proof of Proposition~\ref{prop:finite_sample_converse}}
For all integers $k \geq 0$, denote by $N_k$ the smallest positive integer $n$ such that $n$ is a power of two and $\delta_{n} \leq 2^{-k}$.
Such an integer always exists because the sequence $\delta_n$ decreases to $0$.
We require the following lemma, whose proof is deferred to Appendix~\ref{sec:proofs}.
\begin{applemma}\label{lem:converse_m_bound}
The sequence $N_{k+1}/N_k$ is bounded.
\end{applemma}
Let $m$ be an integer large enough that $N_{k+1}/N_k \leq 2^m$ for all $n$.
Let $\cQ$ be the standard dyadic partition of $[0, 1]$, with $\cQ^k$ being a partition of $[0, 1]^m$ consisting of~$2^{km}$ cubes of side length~$2^{-k}$.

Our measure $\mu$ will satisfy $\cN_{2^{-k}}(\mu) = N_{k-2}$ for all $k \geq 2$.
We will define a sequence of measures $\{\mu_k\}_{k=2}^{\infty}$ iteratively and construct $\mu$ as their limit in the weak topology.

Let $\mu_2$ be the uniform distribution on $[0, 1/4]^m$.
For each positive integer $k$, the measure $\mu_k$ will be supported on $N_{k-2}$ cubes in $\cQ$, and will be uniform on its support.
We will call a cube $Q_i \in \cQ^k$ \emph{live} if $\mu_k(Q_i) \neq 0$.

Fix an ordering $x_0, \dots, x_{2^m-1}$ of the $2^m$ elements of $\{0, 1\}^m$.
To produce $\mu_{k+1}$ from $\mu_k$, divide each live cube of $\mu_k$ into $2^m$ cubes of side length $2^{-(k+1)}$.
The ordering of $\{0, 1\}^m$ induces an order on these $2^m$ subcubes.

Given a live $Q \in \cQ^k$, define the restriction $\mu_{k+1}|_{Q}$ by requiring that $\mu_{k+1}(Q) = \mu_k(Q)$ and that $\mu_{k+1}|_Q$ be uniform on the union of the first $N_{k+1}/N_k$ subcubes of $Q$.
Note that $N_{k+1}/N_k$ is an integer because both $N_{k+1}$ and $N_k$ are powers of $2$, and by assumption $N_{k+1}/N_k \leq 2^m$, the total number of subcubes of $Q$.
Since $\cQ^k$ forms a partition of $[0, 1]^m$, combining the measures $\mu_{k+1}|_Q$ for $Q \in \cQ^k$ yields a probability measure $\mu_{k+1}$ on $[0, 1]^m$.
By Prokhorov's theorem, this sequence of measures $\mu_k$ possesses a subsequence converging in distribution to some measure $\mu$.

The following lemma collects necessary properties of $\mu$.
Its proof appears in Appendix~\ref{sec:proofs}.
\begin{applemma}\label{lem:mu_facts}
If $N_k \leq n < N_{k+1}$, then
\begin{equation*}
\cN_{2^{-k-4}}(\mu, 1/2) > n
\end{equation*}
Moreover,
\begin{equation*}
2^{-(k+1)} \leq \delta_n \leq 2^{-k}
\end{equation*}
and
\begin{equation*}
2^{-k -4} \leq n^{-1/d_n} \leq 2^{-k}\,.
\end{equation*}
\end{applemma}

We can now obtain the lower bound.
Let $\nu$ be any measure supported on at most $n$ points.
If $N_k \leq n < N_{k+1}$, then by Lemma~\ref{lem:mu_facts}, if $X \sim \mu$, then
\begin{equation*}
\p[\min_{y \in \supp(\nu)} \|X - y\|_\infty \leq 2^{-k-5}] < 1/2\,.
\end{equation*}
Markov's inequality therefore implies for any coupling $(X, Y)$ of $\mu$ and $\nu$ that
\begin{equation*}
\E[\|X - Y\|_\infty^p]^{1/p} \geq 2^{-k-4}\p[\min_{y \in \supp(\nu)} \|X - y\|_\infty > 2^{-k-5}]^{1/p} \geq 2^{-k-6} \geq 2^{-6}n^{-1/d_n}\,,
\end{equation*}
as claimed.
\qed



\subsection{Proof of Proposition~\ref{prop:duality}}
Both claims are standard, and details can be found in~\cite[Theorem~5.10]{Vil09}.
The first follows follows from the assumption that $X$ is a bounded Polish space.
For the second, we use the fact that the supremum is achieved by an $f$ satisfying
\begin{equation}\label{eqn:c_convexity}
f(x) = \inf_{y \in X} f^c(y) + D(x, y)^p \quad \forall x \in X\,.
\end{equation}
Let $f$ be a function achieving the supremum in~\eqref{eqn:dual_def} and satisfying~\eqref{eqn:c_convexity}.
By adding a constant to $f$ and $f^c$, we can assume that $\sup_{x \in X} f(x) = 1$.
Then for all $y \in X$,
\begin{equation*}
f^c(y) = \sup_{x \in X} f(x) - D(x, y)^p \geq 0\,,
\end{equation*}
and~\eqref{eqn:c_convexity} then implies
\begin{equation*}
f(x) \geq 0 \quad \forall x \in X\,,
\end{equation*}
as claimed.
\qed

\subsection{Proof of Lemma~\ref{lem:gaussian_tail}}
Let $\Sigma = \sum_{i=1}^d \lambda_i v_i v_i^\top$ be an eigendecomposition of $\Sigma$ with $\lambda_1 \geq \dots \geq \lambda_d \geq 0$.
Then $\|Z\|_2^2$ has the same distribution as $\sum_{i=1}^d \lambda_i \xi_i^2$, where $\xi_1, \dots, \xi_d$ are i.i.d. standard Gaussian random variables. 

By~\cite[Lemma~1]{LauMas00}, for any positive $t$, we have
\begin{equation*}
\p\left[\sum_{i=1}^d \lambda_i (\xi_i^2-1) \geq 2 \sqrt{ t \sum_{i=1}^d \lambda_i^2} + 2 \lambda_{1} t\right] \leq \exp(-t)\,.
\end{equation*}
Bounding both $\left(\sum_{i=1}^d \lambda_i^2\right)^{1/2}$ and $\lambda_1$ by $\Tr(\Sigma)$ yields
\begin{equation*}
\p\left[\|Z\|_2^2 \geq (1 + 2\sqrt t + 2 t)\Tr(\Sigma)\right] \leq \exp(-t)\,.
\end{equation*}
Finally, setting $c^2 = 2(\sqrt t + 1)^2$, we obtain
\begin{equation*}
\p\left[\|Z\|_2^2 \geq c^2\Tr(\Sigma)\right] \leq \exp(-(c - \sqrt 2)^2/2) \leq \exp(-c^2/4)
\end{equation*}
for all $c \geq 5$, as desired.
\qed

\section{Additional lemmas}\label{sec:proofs}
\subsection{Proof of Lemma~\ref{lem:same_mass_pi_rho}}
\begin{proof}
We first show that for any $\ell < k$, if $Q \in \cQ^{\ell}$, then
\begin{equation*}
\mu_k(Q) = \nu_k(Q)\,.
\end{equation*}

Suppose first that $Q \in \cQ^{k-1}$.
By definition, $\mu_k = \mu_{k-1} - \pi_{k-1}$.
We obtain
\begin{equation*}
\mu_k(Q) = (\mu_{k-1} - \pi_{k-1})(Q) = \min\{\mu_{k-1}(Q), \nu_{k-1}(Q)\}\,,
\end{equation*}
and likewise
\begin{equation*}
\nu_k(Q) = \min\{\mu_{k-1}(Q), \nu_{k-1}(Q)\}\,.
\end{equation*}

Since $\cQ$ is a dyadic partition, any $Q \in \cQ^\ell$ for $\ell < k$ can be written as a disjoint union of $Q_1, \dots, Q_m \in \cQ^{k-1}$.
Hence
\begin{equation*}
\mu_k(Q) = \sum_{i=1}^m \mu_k(Q_i) = \sum_{i=1}^m \nu_k(Q_i) = \nu_k(Q)\,,
\end{equation*}
as claimed.

Note that this also implies
\begin{equation*}
\pi_k(Q) = \mu_k(Q) - \mu_{k+1}(Q) = \nu_k(Q) - \nu_{k+1}(Q) = \rho_k(Q)\,.
\end{equation*}

We now prove the bound on $\pi_K(S)$.
By definition, 
\begin{equation*}
\rho_k(S) = \sum_{Q_i^k \in \cQ^k} (\nu_k(Q_i^k) - \mu_k(Q_i^k))_+ = \frac 1 2 \sum_{Q_i^k \in \cQ^k} |\nu_k(Q_i^k) - \mu_k(Q_i^k)|\,.
\end{equation*}

We now show that, for any $P \in \cQ^{k-1}$, there exist scalars $c_1, c_2 \in [0, 1]$ depending on $P$ such that
\begin{align*}
\mu_k|_P & = c_1 \mu|_P \\
\nu_k|_P & = c_2 \nu|_P\,.
\end{align*}

We proceed by induction on $k$.
By symmetry, it suffices to prove the claim for $\mu_k$ and $\mu$.
Since $\mu_1 = \mu$, it holds for $k = 1$.
Now assume $\mu_{k-1}|_P = c_1 \mu|_P$.
We have
\begin{equation*}
\mu_k|_P = \mu_{k-1}|_P - \pi_{k-1}|_P = \min\left\{\frac{\nu_{k-1}(P)}{\mu_{k-1}(P)}, 1\right\} \mu_{k-1}|_P = c_1' \mu|_P\,,
\end{equation*}
where $c_1' = \min\left\{\frac{\nu_{k-1}(P)}{\mu_{k-1}(P)}, 1\right\} c_1$.
This proves the claim.


Now, given such a $P \in \cQ^{k-1}$ and $c_1, c_2 \in [0, 1]$, we have $\mu_k(P) = \nu_k(P)$, so
\begin{equation*}
c_1 \mu(P) = c_2 \nu(P)\,.
\end{equation*}

Summing over the elements of $\cQ^k$ contained in $P$, we obtain
\begin{align*}
\sum_{Q_i^k \subset P} |\mu_k(Q_i^k) - \nu_k(Q_i^k)| & =\sum_{Q_i^k \subset P} |c_1 \mu(Q_i^k) - c_2\nu(Q_i^k)| \\
& \leq \sum_{Q_i^k \subset P} c_1 |\mu(Q_i^k) - \nu(Q_i^k)| + \sum_{Q_i^k \subset P} \nu(Q_i^k) |c_1 - c_2| \\
& = \sum_{Q_i^k \subset P} c_1 |\mu(Q_i^k) - \nu(Q_i^k)| + c_2 |\mu(P) - \nu(P)| \\
& \leq \sum_{Q_i^k \subset P} (c_1 + c_2) |\mu(Q_i^k) - \nu(Q_i^k)| \\
& \leq 2 \sum_{Q_i^k \subset P}|\mu(Q_i^k) - \nu(Q_i^k)|\,.
\end{align*}
Finally, summing over all $P \in \cQ^{k-1}$ yields
\begin{equation*}
\rho_k(S) = \frac 1 2 \sum_{Q_i^k \in \cQ^k} |\nu_k(Q_i^k) - \mu_k(Q_i^k)| \leq \sum_{Q_i^k \in \cQ^k}|\mu(Q_i^k) - \nu(Q_i^k)|\,,
\end{equation*}
as claimed.
\qed

\subsection{Proof of Lemma~\ref{lem:diameter_bound}}
Let
\begin{equation*}
\gamma = \sum_{\substack{Q_i^{k} \in \cQ^{k}\\\alpha(Q_i^{k})> 0}} \frac{\alpha \otimes \beta}{\alpha(Q_i^k)}\,.
\end{equation*}
Note that $\gamma \in C(\alpha, \beta)$.
Indeed, for any measurable $U \subset S$, since $\cQ^{k-1}$ is a partition of $S$, we have
\begin{equation*}
\gamma(S, U) = \sum_{Q_i^{k} \in \cQ^{k}} \frac{\alpha(Q_i^{k}) \beta(Q_i^{k} \cap U)}{\alpha(Q_i^{k-1})} = \beta(U)\,.
\end{equation*}
On the other hand, by assumption, $\alpha(Q_i^{k}) = \beta(Q_k^{k})$, so
\begin{equation*}
\gamma_k(U, S) = \sum_{Q_i^{k} \in \cQ^{k}} \frac{\alpha(Q_i^{k-1} \cap U) \beta(Q_i^{k-1})}{\beta(Q_k^{k-1})} = \alpha(U)\,.
\end{equation*}

We have
\begin{align*}
\int D(x, y)^p \mathrm{d}\gamma(x, y) & = \sum_{Q_i^{k} \in \cQ^{k}} \frac{1}{\alpha(Q_i^{k})}\int_{Q_i^{k}} D(x, y)^p \mathrm{d}\alpha(x) \mathrm{d}\beta(y) \\
&\leq \sum_{Q_i^{k} \in \cQ^{k}} \beta(Q_i^{k}) \diam(Q_i^{k})^p \\
&\leq  \alpha(S) \delta^{kp}\,.
\end{align*}
\end{proof}

\subsection{Proof of Lemma~\ref{lem:converse_m_bound}}
By assumption, there exist constants $c$ and $\alpha$ such that $\frac 1 c n^{\alpha} \leq \delta_n \leq c n^{\alpha}$ for all $n$ sufficiently large.
Let $M = (2 c^2)^{-1/\alpha}$.
Then for $n$ sufficiently large, 
\begin{equation*}
\delta_{M n} \leq c {M n}^{\alpha} = \frac{1}{2c} n^{\alpha} \leq \frac 1 2 \delta_n\,.
\end{equation*}
This implies that for $k$ sufficiently large, $\delta_{N_k} \leq 2^{-k}$ implies that $\delta_{M N_k} \leq 2^{-k-1}$, so that $N_{k+1} \leq M N_k$.
Hence $N_{k+1}/N_k \leq M$ for all $k$ sufficiently large, so $N_{k+1}/N_k$ is bounded for all $k$.
\qed

\subsection{Proof of Lemma~\ref{lem:mu_facts}}
We first show the key property of $\mu$.
For any $x \in [0, 1]^m$ and $r > 0$, denote by $B(x, r)$ the open $\ell_\infty$ ball of radius $r$ around $x$.
We claim that for any $x \in [0, 1]^m$ and $\ell \geq 2$,
\begin{equation*}
\mu(B(x, 2^{-\ell-1})) \leq \frac{1}{N_{\ell-2}}\,.
\end{equation*}

The claim certainly holds when $B(x, 2^{-\ell-1})$ exactly coincides with one of the cubes in $\cQ^\ell$, since each live cube in $\cQ^\ell$ has mass exactly $1/N_{\ell-2}$ by construction.

For all other $x$, note that the restriction of $\mu$ to each live cube in $\cQ^\ell$ is the same measure. In general, the cube $B(x, 2^{-\ell-1})$ intersects $2^m$ cubes cubes in $\cG_\ell$, so we can partition $B(x, 2^{-\ell-1})$ into $2^m$ pieces which, via translation, exactly cover a cube of~$\cQ^\ell$. Each piece has mass at most the mass of the corresponding piece in a live cube, hence the measure is at most the measure of a live cube.

This property immediately implies a bound on the number of balls needed to cover any set $S$ such that $\mu(S) \geq 1/2$.
Since each ball of diameter $2^{-\ell}$ has mass at most $1/N_{\ell-2}$, to cover a set of mass $1/2$ requires at least $N_{\ell-2}/2$ balls.
Therefore for all  $\ell \geq 2$,
\begin{equation}\label{eqn:ball_bound}
\cN_{2^{-\ell}}(\mu, 1/2) \geq N_{\ell-2}/2\,.
\end{equation}

Since $n \leq N_{k+1}$, we have by definition $\delta_n > 2^{-(k+1)}$.
Because $\frac{\log n}{- \log \delta_n}$ is nondecreasing and at least $1$ for all $n \geq 2$, we have as long as $n \geq 2$ that
\begin{equation*}
\frac{\log 2n}{- \log \delta_{2n}} \geq \frac{\log n}{- \log \delta_n} \geq \frac{\log 2n}{- \log \delta_n/2}\,,
\end{equation*}
and therefore $\delta_{2n} \geq \frac 1 2 \delta_n > 2^{-(k+2)}$.
This implies $N_{k+2}/2 > n$.

Choosing $\ell = k+4$ in~\eqref{eqn:ball_bound} yields
\begin{equation*}
\cN_{2^{-k - 4}}(\mu, 1/2) > n\,.
\end{equation*}
This proves the first claim.

If $N_k \leq n < N_{k+1}$, then by definition of $N_k$ and $N_{k+1}$ and the fact that $\delta_n$ is nonincreasing in $n$,
\begin{equation*}
\delta_n \leq \delta_{N_k} \leq 2^{-k}
\end{equation*}
and
\begin{equation*}
\delta_n > 2^{-k-1}\,.
\end{equation*}
This proves the second claim.

To prove the third claim, we first note that the definition of $d_n$ implies that
\begin{equation*}
n^{-1/d_n}
\end{equation*}
is nonincreasing as $n$ increases.
We can therefore prove an upper bound on $n^{-1/d_n}$ by proving an upper bound on $N_k^{-1/d_{N_k}}$.

Recall that
\begin{align*}
d_{N_k} = \inf_{\ep > 0} \max \left\{d_{\geq \ep}(\mu, \ep^p), \frac{\log N_k}{- \log \ep}\right\}\,.
\end{align*}
Choosing $\ep = 2^{-(k+2)}$ yields
\begin{equation*}
d_{N_k} \leq \max \{d_{\geq 2^{-(k+2)}}(\mu), \frac{\log_2 N_k}{k+2}\}\,.
\end{equation*}
To bound the first term, note that if $\ep' \in [2^{-\ell}, 2^{-\ell+1})$, then $\cN_{\ep'}(\mu) \leq \cN_{2^{-\ell}}(\mu) = \cN_{\ell-2}$.
Therefore $d_\ep' = \frac{\log \cN_{\ep'}(\mu)}{- \log \ep'} \leq \frac{\log N_{\ell -2}}{\ell - 1}$.

As above, since $\frac{\log n}{- \log \delta_n}$ is non decreasing and at least $1$ for all $n \geq 2$, we have that $\delta_{N_{\ell-2}} \geq \frac 1 2 \delta_{N_{\ell-2}/2} > 2^{-\ell + 1}$.
Combining this with the above bound implies that if $\ep' \in [2^{-\ell}, 2^{-\ell +1})$, then
\begin{equation*}
d_\ep' \leq \frac{\log N_{\ell - 2}}{- \log \delta_{N_{\ell -2}}}\,.
\end{equation*}
The assumption that $\frac{\log n}{- \log \delta_n}$ is nonincreasing therefore implies
\begin{equation*}
d_{\geq 2^{-{k+2}}}(\mu) \leq \max_{2 \leq \ell \leq k+2} \frac{\log N_{\ell - 2}}{- \log \delta_{N_{\ell -2}}} \leq \frac{\log N_k}{- \log \delta_{N_k}} \leq \frac{\log_2 N_k}{k}\,.
\end{equation*}
We obtain
\begin{equation*}
d_{N_k} \leq \frac{\log_2 N_k}{k}\,,
\end{equation*}
so $n^{-1/d_n} \leq N_k^{-1/d_{N_k}} \leq 2^{-k}$.

To obtain the lower bound, note that if $\ep \leq 2^{-(k+4)}$, then
\begin{equation*}
d_{\geq \ep}(\mu, \ep^p) \geq d_{2^{-(k+4)}}(\mu, 1/2) > \frac{\log_2 n}{k+4}\,,
\end{equation*}
where we have used the fact proved above that $\cN_{2^{-(k+4)}}(\mu, 1/2) > n$.
If $\ep > 2^{-(k+4)}$, then
\begin{equation*}
\frac{\log n}{- \log \ep} > \frac{\log_2 n}{k+4}\,.
\end{equation*}
Combining these bounds yields
\begin{equation*}
d_n = \inf_{\ep > 0} \max \left\{d_{\geq \ep}(\mu, \ep^p), \frac{\log n}{- \log \ep}\right\} > \frac{\log_2 n}{k+4}\,,
\end{equation*}
so
\begin{equation*}
n^{-1/d_n} > 2^{-(k+4)}\,,
\end{equation*}
as claimed.
\qed
