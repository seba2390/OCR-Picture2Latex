A 69-year-old Caucasian woman presents with confusion, headache, generalized weakness, ataxia, and increased agitation on the second day following a CT myelogram during which she received L2-L3 interspace lumbar puncture with fluoroscopic guided injection of 15 cc Omnipaque (Iohexol) 300, a nonionic water-soluble contrast material, in the prone position using a 20-gauge spinal needle. There were no complications during the procedure and after an adequate observation period the patient was discharged in a stable condition.
Her past medical history includes chronic back pain with spinal stenosis, diabetes, coronary artery disease, hyperlipidemia, colitis, gout, atrial fibrillation, and sick sinus syndrome. Her surgical history includes pacemaker placement, hysterectomy, and tonsillectomy. She has no known allergies. Her medication includes aspirin, allopurinol, amlodipine, colchicine, furosemide, gabapentin, metformin, pravastatin, metoprolol, pregabalin, and spironolactone.
On physical examination, the patient's vitals were stable. She was noted to be restless, agitated, and disoriented to place and had hyperactive deep tendon reflexes. She had no focal neurological deficits. Except for a mild elevation in WBC of 13.0, her laboratory values including CBC, CMP, urine analysis, and toxicity screen were all within normal limits.
A head CT revealed new supratentorial bilateral vasogenic edema with loss of all sulci and with poor differentiation of white and gray matter (). There was no evidence of any compression of the quadrigeminal plate cistern or fourth ventricle on imaging. These findings were not visualized on previous imaging. Patient was admitted to the neurology intensive care unit (ICU) and was treated with mannitol, corticosteroids, and seizure prophylaxis. She was discharged in a stable condition after a 5-day admission with resolution of both clinical symptoms and radiographic findings of cerebral edema on head CT ().