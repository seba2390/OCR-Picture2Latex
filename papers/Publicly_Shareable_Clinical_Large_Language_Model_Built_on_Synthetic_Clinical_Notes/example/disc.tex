Discharge Summary

Patient Name: [Name]

Date of Admission: [Date]

Date of Discharge: [Date]

Admitting Diagnosis: Cerebral Edema

This is a discharge summary for [Name], a 69-year-old Caucasian woman who was admitted with confusion, headache, generalized weakness, ataxia, and increased agitation. She was diagnosed with cerebral edema and was treated with mannitol, corticosteroids, and seizure prophylaxis. She was discharged in a stable condition after a 5-day admission with resolution of both clinical symptoms and radiographic findings of cerebral edema on head CT.

Past Medical History:

- Chronic back pain with spinal stenosis
- Diabetes
- Coronary artery disease
- High cholesterol
- Colitis
- Gout
- Atrial fibrillation
- Sick sinus syndrome

Surgical History:

- Pacemaker placement
- Hysterectomy
- Tonsillectomy

Medications:

- Aspirin
- Allopurinol
- Amlodipine
- Colchicine
- Furosemide
- Gabapentin
- Metformin
- Pravastatin
- Metoprolol
- Pregabalin
- Spironolactone

Admission:

The patient presented with confusion, headache, generalized weakness, ataxia, and increased agitation on the second day following a CT myelogram. She received L2-L3 interspace lumbar puncture with fluoroscopic guided injection of 15 cc Omnipaque (Iohexol) 300 during the procedure.

Physical Examination:

On physical examination, the patient's vitals were stable. She was noted to be restless, agitated, and disoriented to place and had hyperactive deep tendon reflexes. She had no focal neurological deficits.

Diagnostic Studies:

A head CT revealed new supratentorial bilateral vasogenic edema with loss of all sulci and with poor differentiation of white and gray matter. There was no evidence of any compression of the quadrigeminal plate cistern or fourth ventricle on imaging.

Hospital Course:

The patient was treated with mannitol, corticosteroids, and seizure prophylaxis and was discharged in a stable condition after a 5-day admission with resolution of both clinical symptoms and radiographic findings of cerebral edema on head CT.

Follow-up:

The patient will follow up with her primary care physician and neurologist after discharge.