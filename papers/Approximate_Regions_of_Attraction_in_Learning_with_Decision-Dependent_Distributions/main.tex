\documentclass[twoside]{article}

\usepackage[accepted]{aistats2023}
% If your paper is accepted, change the options for the package
% aistats2023 as follows:
%
%\usepackage[accepted]{aistats2023}
%
% This option will print headings for the title of your paper and
% headings for the authors names, plus a copyright note at the end of
% the first column of the first page.

% If you set papersize explicitly, activate the following three lines:
%\special{papersize = 8.5in, 11in}
%\setlength{\pdfpageheight}{11in}
%\setlength{\pdfpagewidth}{8.5in}
\usepackage{amsthm}

% If you use natbib package, activate the following three lines:
\usepackage[round]{natbib}
\renewcommand{\bibname}{References}
\renewcommand{\bibsection}{\subsubsection*{\bibname}}

% If you use BibTeX in apalike style, activate the following line:
\bibliographystyle{apalike}

%%%%%%%%%%%%%%%%%%%%%%%%%%%%%%%%%%%%%%%%%%%%%%%%%%%%%%%%%%%%%%%%%%

\usepackage{graphicx} % DO NOT CHANGE THIS

%%%%%%%%%%%%%%%%%%%%%%%%%%%%%%%%%%%%%%%%%%%%%%%%%%%%%%%%%%%%%%%%%%

% Commonly used symbols and commands.

\usepackage{amsmath}
\usepackage{amssymb}

\newcommand{\eps}{\epsilon}

\newcommand{\mc}{\mathcal}
\newcommand{\mb}{\mathbb}
\newcommand{\mf}{\mathbf}

%%%%%%%%%%%%%%%%%%%%%%%%%%%%%%%%%%%%%%%%%%%%%%%%%%%%%%%%%%%%%%%%%%

% Matrix operations

\newcommand{\T}{\intercal}

%%%%%%%%%%%%%%%%%%%%%%%%%%%%%%%%%%%%%%%%%%%%%%%%%%%%%%%%%%%%%%%%%%

% Probability things

% Equal in distribution
\def\eqd{\,{\buildrel d \over =}\,} 

% Essential supremum
\DeclareMathOperator*{\esssup}{ess\,sup}

%%%%%%%%%%%%%%%%%%%%%%%%%%%%%%%%%%%%%%%%%%%%%%%%%%%%%%%%%%%%%%%%%%

% Environments

\newtheorem{definition}{Definition}
\newtheorem{assumption}{Assumption}
\newtheorem{proposition}{Proposition}
\newtheorem{lemma}{Lemma}
\newtheorem{theorem}{Theorem}
\newtheorem{corollary}{Corollary}
\newtheorem{remark}{Remark}
\newtheorem{example}{Example}

%%%%%%%%%%%%%%%%%%%%%%%%%%%%%%%%%%%%%%%%%%%%%%%%%%%%%%%%%%%%%%%%%%
% following loops stolen from djhsu
\def\ddefloop#1{\ifx\ddefloop#1\else\ddef{#1}\expandafter\ddefloop\fi}
\def\ddef#1{\expandafter\def\csname bb#1\endcsname{\ensuremath{\mathbb{#1}}}}
\ddefloop ABCDEFGHIJKLMNOPQRSTUVWXYZ\ddefloop
\def\ddef#1{\expandafter\def\csname bf#1\endcsname{\ensuremath{\mathbf{#1}}}}
\ddefloop ABCDEFGHIJKLMNOPQRSTUVWXYZabcdefghijklmnopqrstuvwxyz\ddefloop
\def\ddef#1{\expandafter\def\csname bs#1\endcsname{\ensuremath{\boldsymbol{#1}}}}
\ddefloop ABCDEFGHIJKLMNOPQRSTUVWXYZabcdefghijklmnopqrstuvwxyz\ddefloop
\def\ddef#1{\expandafter\def\csname sf#1\endcsname{\ensuremath{\mathsf{#1}}}}
\ddefloop ABCDEFGHIJKLMNOPQRSTUVWXYZ\ddefloop
\def\ddef#1{\expandafter\def\csname c#1\endcsname{\ensuremath{\mathcal{#1}}}}
\ddefloop ABCDEFGHIJKLMNOPQRSTUVWXYZ\ddefloop

\DeclareMathOperator*{\argmax}{argmax}
\DeclareMathOperator*{\argmin}{argmin}

% mark steps in equations
\newcommand\meq[2]{\stackrel{\mathclap{\normalfont\mbox\tiny{#1}}}{#2}}



% Other
\newcommand{\xnom}{x_{PR}}
\newcommand{\dxnom}{\dot x_{PR}}
\newcommand{\xpert}{x_{RR}}
\newcommand{\dxpert}{\dot x_{RR}}

\newcommand{\fnom}{f_{PR}}
\newcommand{\fpert}{f_{RR}}

\newcommand{\phinom}{\varphi_{PR}}
\newcommand{\phipert}{\varphi_{RR}}

\newcommand{\Vnom}{V_{PR}}
\usepackage{enumitem}

\begin{document}

% If your paper is accepted and the title of your paper is very long,
% the style will print as headings an error message. Use the following
% command to supply a shorter title of your paper so that it can be
% used as headings.
%
%\runningtitle{I use this title instead because the last one was very long}

% If your paper is accepted and the number of authors is large, the
% style will print as headings an error message. Use the following
% command to supply a shorter version of the authors names so that
% they can be used as headings (for example, use only the surnames)
%
%\runningauthor{Surname 1, Surname 2, Surname 3, ...., Surname n}

\twocolumn[

\aistatstitle{Approximate Regions of Attraction in Learning with Decision-Dependent Distributions}

\aistatsauthor{ Roy Dong \And Heling Zhang \And  Lillian J. Ratliff }

\aistatsaddress{ UIUC \And  UIUC \And UW } ]

\begin{abstract}
  As data-driven methods are deployed in real-world settings, the processes that generate the observed data will often react to the decisions of the learner. For example, a data source may have some incentive for the algorithm to provide a particular label (e.g. approve a bank loan), and manipulate their features accordingly. Work in strategic classification and decision-dependent distributions seeks to characterize the closed-loop behavior of deploying learning algorithms by explicitly considering the effect of the classifier on the underlying data distribution. More recently, works in performative prediction seek to classify the closed-loop behavior by considering general properties of the mapping from classifier to data distribution, rather than an explicit form. Building on this notion, we analyze repeated risk minimization as the perturbed trajectories of the gradient flows of performative risk minimization. We consider the case where there may be multiple local minimizers of performative risk, motivated by situations where the initial conditions may have significant impact on the long-term behavior of the system. We provide sufficient conditions to characterize the region of attraction for the various equilibria in this settings. Additionally, we introduce the notion of performative alignment, which provides a geometric condition on the convergence of repeated risk minimization to performative risk minimizers.
\end{abstract}

%%%%%%%%%%%%%%%%%%%%%%%%%%%%%%%%%%%%%%%%%%%%%%%%%%%%%%%%%%%%%%%%

\section{INTRODUCTION}
\label{sec:intro}
Reinforcement learning has achieved great success in areas such as Game-playing \citep{silver2018general,vinyals2019grandmaster}, robotics \cite{kober2013reinforcement}, large language models \citep{ouyang2022training}, etc.
However, due to safety concerns or physical limitations, in some real-world reinforcement learning problems, we must consider additional constraints that may influence the optimal policy and the learning process \citep{garcia2015comprehensive}.
% For example, a robotic arm must not take actions that may cause harm to itself or the environments.
A standard framework to handle such cases is the constrained Markov Decision Process (CMDP) \citep{altman1999constrained}.
Within the CMDP framework, the agent has to maximize
the expected cumulative reward while
obeying a finite number of constraints, which are usually in the form of expected cumulative cost criteria.

However, we are sometimes concerned with the problem with a continuum of constraints.
For example,
the constraints we meet might be time-evolving or subject to uncertain parameters, which
cannot be formulated as an ordinary CMDP
(see Examples \ref{Example_Time_Evolving} and  \ref{Example_Uncertain}).
In this paper we would study a generalized CMDP  
to address the above problem.  Because the constraints are not only infinite-number but also lie
in a continuous set,
the generalization is not trivial. Fortunately, we find that we can borrow the idea behind semi-infinite programming (SIP) \citep{remez1934determination, hettich1993semi} to deal with the semi-infinite constraints.
Accordingly, we propose \emph{semi-infinitely constrained Markov decision processes} (SICMDPs)
as a novel complement to the ordinary CMDP framework.
%More specifically,  an SICMDP model %, we consider 
%contains a continuum of constraints whereas an ordinary CMDP contains a finite number of constraints. 

%This generalization is natural but not trivial. However, we can brows the idea  
%The idea is quite natural and can be backtracked
%to the practice of extending linear programming to linear semi-infinite programming (LSIP) %\cite{remez1934determination, GobernaLSIO1998}.
%In addition, 
%As a complementary approach to the ordinary CMDP framework, 
%SICMDP can be used to model these problems  which cannot be described by a finite number of constraints
%that are not covered by .
%For example,
%the restrictions we consider can be time-evolving or subject to uncertain parameters
%, thus
%cannot be described by a finite number of constraints but a continuum of constraints 
%(see Examples \ref{Example_Time_Evolving} and  \ref{Example_Uncertain}).

We also present two reinforcement learning algorithms to solve SICMDPs called SI-CRL and SI-CPO, respectively.
SI-CRL is a model-based reinforcement learning algorithm designed for tabular cases, and SI-CPO is a policy optimization algorithm for non-tabular cases.
% and analyze its performance both theoretically and empirically.
The main challenge is that we need to deal with a continuum of constraints, thus reinforcement learning algorithms for ordinary CMDPs do not work anymore.
In SI-CRL, we tackle this difficulty by first transforming the reinforcement learning problem to an equivalent LSIP problem, which can then be solved using methods in the LSIP literature like the dual exchange methods \citep{Hu1990,reemtsen1998numerical}.
In SI-CPO, we resort to the idea of cooperative stochastic approximation developed in \cite{lan2020algorithms, wei2020comirror}.
As far as we know, we are the first to introduce tools from semi-infinitely programming (SIP) into the reinforcement learning community for solving constrained reinforcement learning problems.

% To the best of our knowledge, we are the first to apply tools from semi-infinitely programming (SIP) to solve reinforcement learning problems.
Furthermore, we give theoretical analysis for both SI-CRL and SI-CPO.
We decompose the error of SI-CRL into two parts: the statistical error from approximating the true SICMDP with an offline dataset and the optimization error due to the fact that the solution of the LSIP problem obtained by the dual exchange method is inexact.
On the optimization side, we show that the iteration complexity of SI-CRL is $O\left(\left\{\mathrm{diam}(Y)L\sqrt{|\gS|^2|\gA|m}/\left[(1-\gamma)\epsilon\right]\right\}^m\right)$.
On the statistical side, we show that the sample complexity of SI-CRL is $\widetilde O\left(\frac{|S|^2|A|^2}{\epsilon^2(1-\gamma)^3}\right)$ if the offline dataset is generated by a generative model, and $\widetilde O\left(\frac{|S||A|}{\nu_{\min} \epsilon^2(1-\gamma)^3}\right)$ if the dataset is generated by a probability measure $\nu$ as considered in \cite{chen2019information}.
Here $\widetilde O$ means that all logarithm terms are discarded.
For SI-CPO, things become a little more complicated because other than the statistical error and the optimization error, we also need to consider the function approximation error, which comes from imperfect policy parametrizations.
It is shown if the function approximation error can be controlled to $O(\epsilon)$ order, the iteration complexity of SI-CPO is $\widetilde{O}\left(\frac{1}{\epsilon^2(1-\gamma)^6}\right)$ and the sample complexity of SI-CPO is $\widetilde{O}(\frac{1}{\epsilon^4(1-\gamma)^{10}})$.
Here our iteration complexity bound is equivalent to a typical $\widetilde O(1/\sqrt{T})$ global convergence rate.

We perform a set of numerical experiments to illustrate the SICMDP model and validate our proposed algorithms.
Specifically, we examine two numerical examples, namely the discharge of sewage and ship route planning.
Through the discharge of sewage example, we show the advantage of the SICMDP framework over the CMDP baseline obtained by naive discretization in modeling realistic sequential decision-making problems.
Moreover, we demonstrate the effectiveness of the SI-CRL and SI-CPO algorithms in such tabular environments. 
In the ship route planning example, we illustrate the benefits of the SICMDP framework and the ability of the SI-CPO algorithm to address complex continuous control tasks involving continuous state spaces with modern deep reinforcement learning techniques.

% In summary, our contributions are listed as follows.
% First, we present the SICMDP model, which can be viewed as a generalization of the ordinary CMDP model.
% Second, we propose an algorithm to perform reinforcement learning for SICMDPs, which is called SI-CRL, and we believe that we are the first to apply tools from SIP
% to solve reinforcement learning problems.
% Third, we give a theoretical analysis of SI-CRL and identify both its sample complexity and iteration complexity.
% In addition, we perform numerical experiments to illustrate the SICMDP model and validate the SI-CRL algorithm.
% \{This paragraph can be removed!!! \}






%%%%%%%%%%%%%%%%%%%%%%%%%%%%%%%%%%%%%%%%%%%%%%%%%%%%%%%%%%%%%%%%

\section{BACKGROUND}
\label{sec:background}

% Panoptic segmentation

% 3D segmentation

% Multi-object tracking

% Online 3D panoptic:

% PanopticFusion: (IROS 2019)
% https://arxiv.org/pdf/1903.01177.pdf
%
% - most similar to ours
% - PSPNet + M-RCNN + 2D fusion
% - volumetric mapping, 
% - greedy matching with IoU -> optimal only with 0.5 threshold
% - voxel & class weighting
% - CRF regularisation
%
% - good:
%
% - bad:
%  - CRF post-processing step
%  - greedy data-association
%    - can't be tuned for lower overlap ratios -> has to have high framerate, large changes in viewpoint could break this
%    - IoU: sensitive to 2D labels projecting over object borders (CRF and voxel weighting seem to alleviate this)

% Voxblox++: (Robotics & automation letters 2019)
% https://arxiv.org/pdf/1903.00268.pdf
% https://github.com/ethz-asl/voxblox-plusplus
%
% - M-RCNN + geometric segmentation + fusion 
% - data association of geometric segments with 3D overlap (no. points inside volume), fixed threshold for min number of points
% - instance label is assigned to a segment based on highest overlap
% - only one detected segment per reference label, as in PanopticFusion and Ours
% - TSDF Integration 
%
% good: 
% - because of geometric segmentation objects with no associated semantic class can also be segmented
% bad:
% - two different object segment types -> confusing, overly complicated ?
% - quite inaccurate (fixed below)

% Reconstructing Interactive 3D Scenes by Panoptic Mapping and CAD Model Alignments (ICRA 2021)
% https://arxiv.org/pdf/2103.16095.pdf
% https://github.com/hmz-15/Interactive-Scene-Reconstruction
%
% - based heavily on Voxblox++, much more accurate
% - Scene-graph ("contact graph") for mapping object relations
% - Search & replace voxels with CAD models, with geometrical and physical constraints
% - Object 6D pose
% - Format for robot interaction
%
% - Segmentation: bilateral fusion of geomatric and semantic segments -> reduce segmentation noise compared to Voxblox++
% - Fusion: triplet count improves consistency over Voxblox++ pairwise count strategy (take semantic label into account in addition to instance and geometry)
% - Fusion: instance labels are also combined if there is enough overlap with common geometric label for long enough time
%   - this means multiple detections can match the same reference unlike ours, voxblox++ and PanopticFusion ?
%

% Panoptic-MOPE: (ROBOTICS AND AUTOMATION LETTERS 2020)
% https://ieeexplore.ieee.org/stamp/stamp.jsp?tp=&arnumber=8977356
% https://github.com/hoangcuongbk80/Object-RPE/tree/panoptic-mope
%
% - novel RGB-D semantic segmentation model + M-RCNN
% - camera tracking based on "addaptively weighted optimization of geometric, appearance, and semantic cues"
% - surfel map: 
%   - how does it scale ? authors satate they tested on room-sized environments, but could be applied in larger scale as well ...
%     - could maybe be applied as VO in a SLAM algorithm ...
%   - demo only on a small pallet + surroundings, might not be applicable in large-scale SLAM

% US VS THEM:
%
% - based heavily on PanopticFusion, with modifications:
%   - instead of greedy data-association (which seems to be the case in others as well), we solve LAP (JPDA?)
%     - overlap threshold can be tuned, which renders the algorithm more flexible
%     - could be extended to dynamic tracking ?
%   - multiple options for association likelihood
%   - outlier rejection (either clustering or probabilistic)
%   - test different options for decreasing processing time
%   - no post-processing
%
% - model-agnostic:
%   - completely separated from segmentation
%   - does not care how point clouds are obtained -> applicable for LIDAR segmentation (e.g. EfficientLPS) as well
%
% - also agnostic to localisation method
%   - could, however, be utilised to find landmark locations / poses

% More compact version of this paragraph to introduction to save space?
%Panoptic segmentation -- proposed in \cite{panoptic_segmentation} -- aims to solve the unified task of semantic- and instance segmentation. Semantic classes are separated to \textit{stuff} -- amorphous, unquantifiable regions like sky, road or floor -- and \textit{things} -- quantifiable objects. The distinction between the two can vary depending on the application, but a semantic class can only belong to one or another. The article also proposes a unified panoptic evaluation metric, coined \textbf{Panoptic Quality} (PQ). Many 2D approaches to panoptic segmentation -- \textit{e.g.} \cite{panopticfpn,seamless,panoptic_deeplab,efficientps} -- have since been proposed. Deep neural networks for performing semantic- or instance segmentation directly on the 3D reconstruction -- \textit{e.g.} on \cite{scannet,s3dis,paris_lille_3d} -- have also been proposed, but since they require the reconstructed 3D scene, they are mostly offline approaches and therefore out of scope for this work. Some recent works also apply panoptic segmentation to point clouds -- \textit{e.g.} methods in the SemanticKITTI panoptic segmentation competition \cite{semantic_kitti} -- mostly aimed at segmenting LiDAR output. They are suitable for online processing, but similar to RGB-D images require a method for tracking object instances persistent in both time and space. In fact, our proposed method, as well as some others mentioned in this work, could use segmented LiDAR point clouds as an input similarly to RGB-D images.

PanopticFusion \cite{panopticfusion} is the first work to propose online integration of panoptic image segmentations to a 3D reconstruction. They integrate point clouds generated from segmented images to a TSDF voxel volume \cite{tsdf,voxblox} by greedily matching detected segments with the reconstruction and regulating each voxel's corresponding instance with a weighting function. Semantic labels are inferred in a bayesian manner based on confidence scores provided by the segmentation model. They also apply a Conditional Random Field (CRF) to regularise the reconstruction, improving results significantly. Voxblox++ \cite{voxblox++} -- introduced later the same year -- is a similar approach that also integrates segmented RGB-D images into a TSDF volume. It leverages geometric segmentation of depth images to improve instance segmentation accuracy. Both geometric and semantic segments are used to compute a pair-wise weight, which is used to greedily match them with segments in the reconstruction. Because of the geometric segmentation, the method allows segmentation of objects with no known semantic class in addition to objects recognised by the instance segmentation model. 

Recently, \cite{interactive_3d_scenes} built upon the idea of Voxblox++. They apply Voxblox++ for 3D instance integration, with two small but effective modifications: the pair-wise weight is replaced by a triplet weight that also takes semantic labels into account in the fusion, and -- in addition to geometric segments -- instance segments are fused if they overlap by a significant amount. The article introduces a method for searching and aligning CAD models to reconstructed objects based on geometry and semantic class, as well as geometrical and physical rules. With the CAD models, a contact graph and interactive virtual scene are reconstructed to allow a robot to simulate its interaction with the environment. SceneGraphFusion \cite{scenegraphfusion} is another approach that forms a scene graph online from a stream of RGB-D images, but unlike the above-mentioned approach, it generates the graph with a deep neural network, after which the panoptic labels for geometrically segmented portions of the 3D reconstruction are produced a side product.

Panoptic-MOPE \cite{panoptic_mope} is another recent approach, which integrates sequences of RGB-D images into a surfel reconstruction. Unlike other mentioned approaches -- which assume the camera pose either known or estimated elsewhere -- it also tracks camera movements based on geometric-, appearance- and semantic cues. The method also applies a novel RGB-D panoptic segmentation model. Although it is only tested on room-sized environments, the authors claim it could be scaled to larger environments as well.

%%%%%%%%%%%%%%%%%%%%%%%%%%%%%%%%%%%%%%%%%%%%%%%%%%%%%%%%%%%%%%%%

\section{PERFORMATIVE PREDICTION, FLOWS, AND PERTURBATIONS}
\label{sec:model}
\section{The \MakeLowercase{i}W\MakeLowercase{inr}NFL model}
\label{sec:model}

In this section we are going to present the data we used to develop our in-game probability model as well as the design details of {\method}. 

{\bf Data: }In order to perform our analysis we utilize a dataset collected from NFL's Game Center for all the regular season games between the seasons 2009 and 2016. 
We access the data using the Python {\tt nflgame} API \cite{nflgame}. 
The dataset includes detailed play-by-play information for every game that took place during these seasons. 
This information is used to obtain the state of the game that will drive the design of {\method}. 
In total, we collected information for 2,048 regular season games and a total of 338,294 snaps/plays. 

{\bf Model: }
{\method} is based on a logistic regression model that calculates the probability of the home team winning given the current status of the game as: 

\begin{equation}
\Pr(H=1| \mathbf{x})= \frac{\exp(\mathbf{\weight}^T\cdot\mathbf{x})}{1+\exp(\mathbf{\weight}^T\cdot\mathbf{x})}
\label{eq:reg}
\end{equation}
where $H$ is the dependent random variable of our model representing whether the home team wins or not, $\mathbf{x}$ is the vector with the independent variables, while the coefficient vector $\mathbf{\weight}$ includes the weights for each independent variable and is estimated using the corresponding data.  
For a game of infinite duration a linear model could be a very good approximation.  
However, the boundary effects from the finite duration of a game create several non-linearities \cite{winston2012mathletics}.  
For this reason, we enhance our model - using the same set of features - with a Support Vector Machine classifier with radial kernel for the last three minutes of regulation.  
In order to obtain a probability output from the SVM classifier, we further use Platt's scaling \cite{platt1999probabilistic}: 

\begin{equation}
\Pr(H=1| \mathbf{x})= \frac{1}{1+\exp{(Af(x)+B)}}
\label{eq:platt}
\end{equation}
where $f(x)$ is the uncalibrated value produced by the SVM classifier: 

\begin{equation}
f(x) = \sum_{i} (\alpha_i y_i k(\mathbf{x}_i\cdot\mathbf{x}))+ b
\label{eq:svm}
\end{equation}
where $k(\mathbf{x},\mathbf{x}')$ is the kernel used for the SVM.   
Figure \ref{fig:iwinrNFL} depicts the simple flow chart of {\method}. 


\begin{figure}[t]
\begin{center}
\includegraphics[scale=0.35]{plots/iwinrNFL.pdf}%\vspacecap
 \caption{{\method} includes a linear and a non-linear component.}
 \label{fig:iwinrNFL}
\end{center}
\end{figure}

In order to describe the status of the game we use the following variables:

\begin{enumerate}
\item {\bf Ball Possession Team:} This binary feature captures whether the home or the visiting team has the ball possession
\item {\bf Score Differential:} This feature captures the current score differential (home - visiting)
\item {\bf Timeouts Remaining:} This feature is represented by two independent variables - one for the home and one for the away team - and they capture the number of timeouts remaining for each of the teams
%\item {\bf Quarter:} This feature captures the current quarter of the game
%\item {\bf Time Remaining:} This feature captures the time (in seconds) remaining for the current quarter to end
\item {\bf Time Elapsed: } This feature captures the time elapsed since the beginning of the game
\item {\bf Down:} This feature represents the down of the team in possession
\item {\bf Field Position:} This feature captures the distance covered by the team in possession from their own yard line
\item {\bf Yards-to-go:} This variables represents the number of yards needed for a first down
\item {\bf Ball Possession Time: } This variable captures the time that the offensive unit of the home team is on the field 
\item {\bf Ranking Differential: } This variable represents the difference of the win percentage for the two team (home - visiting)
\end{enumerate}

The last independent variable is representative of the power ranking difference between the two teams. 
Most of the existing models that include such a variable are using the Vegas line spread for each game.  
We choose not to do so for the following reason.  
The objective of the Vegas line is not to predict game outcomes but rather distribute money across the different bets.  
Exactly because of this objective the line is changing during the week before the game.  
While this line can change due to new information for the competing teams (e.g., injury updates), the line is mainly changing when a particular team has accumulated the majority of the bets. 
In this case it will also be hard to choose which line to use (e.g., the opening, the closing or some average of them).  
Therefore, we choose to use the win percentage differential of the two teams as an indicator of their strength (even though this has its own issues given the uneven schedule in NFL).  
However, note that if one would like to use the point spread as a variable this can be easily incorporated in the model. 
Table \ref{tab:iwinrnfl} presents the coefficients of the logistic regression model of {\method} with standardized independent variables for better comparisons. 


\begin{table}[ht]
\begin{center}
\def\sym#1{\ifmmode^{#1}\else\(^{#1}\)\fi}
\begin{tabular}{l*{1}{c}}
\toprule
                    &\multicolumn{1}{c}{(1)}\\
                    &\multicolumn{1}{c}{Winner}\\
\midrule
Possession Team (H)         &      0.41\sym{***}\\
                    &     (49.19)         \\
\addlinespace
Score Differential           &      3.59\sym{***}\\
                    &    (247.34)         \\
\addlinespace
Home Timeouts           &     0.12\sym{***}\\
                    &      (8.74)         \\
\addlinespace
Away Timeouts           &     -0.11\sym{***}\\
                    &    (-12.47)         \\
\addlinespace
Ball Possession Time  &     -0.05.\\
                    &    (-1.66)         \\
\addlinespace
Time Lapsed       &   -0.05.\\
                    &      (-1.66)         \\
\addlinespace
Down                &   -0.01         \\
                    &      (0.04)         \\
\addlinespace
Field Position            &   0.02\sym{**} \\
                    &      (2.71)         \\
\addlinespace
Yards-to-go                &  -0.01         \\
                    &      (0.23)         \\
\addlinespace
Rating differential         &       0.75\sym{***}\\
                    &     (80.47)         \\
\addlinespace
Intercept            &       0.57\sym{*}\\
                    &    (2.09)         \\
\midrule
Observations        &      338,294         \\
\bottomrule
\multicolumn{2}{l}{\footnotesize \textit{t} statistics in parentheses}\\
\multicolumn{2}{l}{\footnotesize \sym{$_.$} \(p<0.1\), \sym{*} \(p<0.05\), \sym{**} \(p<0.01\), \sym{***} \(p<0.001\)}\\
\end{tabular}
\end{center}
\caption{Standardized logisitic regression coefficients for {\method}.}
\label{tab:iwinrnfl}
\end{table}


As we can see, as one might have expected the current scoring differential exhibits the strongest correlation with the in-game win probability.  
The only factors that do not appear to be statistically significant predictors of the dependent variable are the down and the yards-to-go. 
Even though the corresponding coefficients are negative as one might have expected (e.g., being at an earlier down gives you more chances to advance the ball), they are not significant in estimating the win probability. 
On the contrary, all else being equal timeouts appear to be quiet important since they can help a team stop the clock, while teams with better win percentage appear to have an advantage as well, since this can be a sign of a better team. 
In the following section we provide a detailed evaluation of {\method}.

%%%%%%%%%%%%%%%%%%%%%%%%%%%%%%%%%%%%%%%%%%%%%%%%%%%%%%%%%%%%%%%%

\subsection{EXAMPLES}
\label{sec:example}


\section{An example}
\label{sec:6}
\begin{figure}[H]

\includegraphics[scale=.6]{schematic}
\centering
\caption{ A sketch of the example constructed in this section. The manifold decomposes into three pieces, left and right caps and a middle region. The loop drawn on the boundary of the left cap only bounds surfaces of high topological complexity that are at least partly contained in the right cap. This ensures the isoperimetric ratio of that loop is very small.}
\label{fig:example_sketch}
\end{figure}

The aim of this section is to show that the first positive eigenvalue of the 1-form Laplacian can vanish exponentially fast in relation to volume. This contrasts the behaviour of the first positive eigenvalue of the Laplacian on functions.

Our construction is similar to that in \cite{BD}. Essentially, we choose a hyperbolic 3-manifold with totally geodesic boundary and glue it to itself using a particular psuedoAnosov with several useful properties. By \cite{BMNS}, this family has geometry that up to bounded error can be understood in terms of a simple model family. Using this model family, we show that one can find curves with uniformly bounded length whose stable commutator length grows exponentially in the volume. We then use the spectral gap upper bound in Theorem A to conclude the first positive eigenvalue vanishes exponentially fast.

Throughout this section, we need to compare geodesic lengths in different submanifolds of a given manifold $M$. Let $|\cdot|_{X}$ denote the geodesic length of a homotopy class of curves relative endpoints in a manifold $X$ and $\text{length}(\cdot)$ be the length in $M$ of the curve. Similarly, when we compute stable commutator length for the fundamental group of a manifold $X$, which may or may not be a submanifold of $M$, we denote it $\scl_X$.

We will need that for certain curves, $\scl$ is comparable to length. We begin with a simple but essential technical lemma.
\begin{figure}[H]
\labellist
\small\hair 2pt
 \pinlabel {$a_0$} [ ] at 830 1100
 \pinlabel {$a_1$} [ ] at 1300 1100
 \pinlabel {$b_0$} [ ] at 830 1250
 \pinlabel {$b_1$} [ ] at 1300 1250
 \pinlabel {$t_1$} [ ] at 380 1290
 \pinlabel {$t_0$} [ ] at 380 1120
\endlabellist
\centering
\includegraphics[width = 15cm]{lemdiagram}
\caption{ Illustrating Lemma \ref{lem:6.1} , the rectangular base of the figure is part of the totally geodesic surface $S$ and the box is the corresponding part of the tubular neighborhood $N_\e(S)$ foliated by surfaces $S_t$ parallel to $S$. Drawn in the box is the surface $\Sigma$, which is transverse the foliation except at isolated points. The multicurve $a_0\cup a_1$ is part of a single level set $S_{t_0} \cap \Sigma$, but only $a_0$ is part of the curve $c_{t_0}$ described in the lemma, whereas the multicurve $b_0\cup b_1$ forms the multicurve $c_{t_1}$ in the lemma.}
\label{fig:box}
\end{figure}


\begin{lem} \label{lem:6.1} Let $M$ be a compact hyperbolic 3-manifold with totally geodesic boundary $ \d M = S$. Let $\e$ be smaller than the injectivity radius of $M$ and such that $N_\e(S)$ is an embedded tubular neighborhood. Let $\{S_t\}$ be the leaves of the foliation of $N_\e(S)$ by surfaces equidistant from $S$.  Let $\Sigma$ be a smooth incompressible proper not necessarily immersed surface in $M$ that is transverse to the foliation $\{S_t\}$ except at isolated points. Let $c = \d \Sigma$. By transversality, for generic $t$ the multicurve $c_t$ given by the part of $S_t\cap \Sigma$ that cobounds a subsurface of $\Sigma$ with $c = c_0$ is a smooth multicurve.  Let $T$ be the set (of full measure) of all $t\in[0,\e)$ such that $c_t$ is a smooth multicurve. Since $S$ is totally geodesic, each multicurve $c_t$ is homotopic to a possibly degenerate geodesic multicurve $\gamma_t$ in $S$. Let $\Sigma_\e$ be the part of $\Sigma$ contained in $N_\e(S)$. Then for $C = 1/\e$, we have that $$\inf\limits_{t\in T} |\gamma_t|_{S} \leq C \emph{Area}(\Sigma_\e).$$
\end{lem}

\begin{proof} The coarea formula implies the inequality $\inf\limits_{t\in T} |c_t|_{S_t} \leq C \text{Area}(\Sigma_\e)$.  Since $S$ is totally geodesic, for all $t\in T$, we have $|\gamma_t|_S\leq |c_t|_{S_t}.$  \end{proof}
Note that in the previous lemma, when $\inf\limits_{t\in T} |\gamma_t|_S$ is zero, because $\Sigma$ is incompressible and any loop with length less than $\inj(M)$ bounds a disk, every component of $\Sigma$ can either be homotoped to be disjoint from $N_\e(S)$ or be contained in $S$.

The next proposition requires a notion of geometric complexity for homology classes. For any compact Riemannian manifold $M$ one can define the stable norm on the first homology of $M$ (see \cite{Gromovmetric} Section 4C). The mass of a Lipschitz 1-chain $\alpha = \sum_i t_i\alpha_i$ in $M$ is defined to be $\text{mass}(\alpha) = \sum_i |t_i|\text{length}(\alpha_i).$ The mass of a class $a\in H_1(M)$ is then the infimal value of the mass of a chain $\alpha$ representing $a$.
For a class $a\in H_1(M)$, the stable norm of $a$ is then given by $$||a||_{s,M} = \inf\limits_{m>0}\frac{\text{mass}(m a)}{m}.$$

Stable commutator length can also be generalized to geodesic multicurves (see Section 2.6 of \cite{Calegari}), which can naturally be viewed as Lipschitz chains. Suppose $\gamma_i\in \pi_1 M$ and $ \sum_i n [\gamma_i] = 0$ in $H_1(M)$. Let $\gamma$ be the geodesic multicurve, which is not necessarily simple, consisting of the geodesic loops determined by $\gamma_i$. Say a surface $f:S\to M$ is admissible of degree $n(S)$ if it has no closed components and $\d S$ is a union of circles $S^1_i$ with $f|_{S^1_i}$ a degree $n(S)$ cover of $\gamma_i$.
Then we define stable commutator length of $\gamma$ to be $$\scl(\gamma) = \inf\limits_{S \text{ admissible}} \frac{\chi_-(S)}{2n(S)}.$$ When $\gamma$ is a single loop, this definition agrees with the usual definition of stable commutator length.

 \begin{prop} \label{prop:6.2} Let $M$ be a compact oriented hyperbolic 3-manifold with totally geodesic boundary $\d M = S$. Let $\gamma$ be a geodesic multicurve in $S$ that is rationally nullhomologous in $M$. Then there is a constant $D> 0$ depending only on $M$ such that $$||[\gamma]||_{s,S}\leq D\scl_M(\gamma).$$ \end{prop}

 \begin{proof} If $\gamma$ is nullhomologous in $S$, then the left hand side is zero and the inequality holds. Assume now that $[\gamma]\neq 0\in H_1(S)$. Fix $\delta>0$. Let $\Sigma_m$ be an incompressible admissible surface for $\gamma$ of degree $m = n(S)$ such that $\chi_-(\Sigma_m)/2m -\scl(\gamma) < \delta$. We can triangulate $\Sigma_m$ so that there is a single vertex on each boundary component. This triangulation has $4g +3b - 4$ faces, where $g$ is the genus of $\Sigma_m$ and $b$ the number of boundary components. We can then straighten this triangulation to obtain a piecewise totally geodesic triangulated surface. Replace $\Sigma_m$ with this surface. Since every face of this triangulation of $\Sigma_m$ is geodesic, every face has area at most $\pi$. Since there are $4g+3b - 4$ faces and $\chi_-(\Sigma_m) = 2g-2 + b$, we can estimate  $$\text{Area}(\Sigma_m) \leq 3\pi\chi_-(\Sigma_m).$$

We can perturb $\Sigma_m$ to obtain a smooth surface $\Sigma’_m$ that it is transverse the foliation of $N_\e(S)$ except at isolated points and in doing so increase the area by less than $\delta$. Let $\gamma_t$ be the family of multicurves in Lemma \ref{lem:6.1}  applied to $\Sigma’_m$. Since each curve $\gamma_t$ cobounds a surface in $S$ with $\d \Sigma_m$, they are homologous, thus $||[\gamma_t]||_{s,S} = m||[\gamma]||_{s,S}$. Since $||[\gamma_t]||_{s,S}\leq |\gamma_t|_S$,
Lemma \ref{lem:6.1}  implies that $$m||[\gamma]||_{s,S}\leq C\text{Area}(\Sigma_m’) \leq C \text{Area}(\Sigma_m) + C\delta \leq 3C\pi\chi_-(\Sigma_m) + C\delta.$$
From this we get  $$||[\gamma]||_{s,S}\leq 6C\pi\chi_-(\Sigma_m)/2m + C\delta/m\leq 6C\pi\scl_M(\gamma) + 6C\pi\delta +C\delta/m.$$
Since the stable commutator length of a nontrivial rational commutator is bounded away from zero by a constant only depending on $M$, by Theorem 3.9 in \cite{Calegari}, we can replace $C$ with a larger constant $D$ such that $$||[\gamma]||_{s,S}\leq D\scl_M(\gamma),$$ as desired. \end{proof}

 We now introduce the family of manifolds that we use in our construction. The family $\{W_n\}$ of manifolds we study are easily understood using the model manifold theory of \cite{BMNS}. In particular, there is a $K$-biLipschitz map between $W_n$ and a model manifold $M_n$, where $K$ is independent of $n$. The base of the construction is Thurston’s tripus manifold $W$ (see \cite{thurstonbook}, Section 3.3.12), a hyperbolic manifold with totally geodesic boundary, and a psuedoAnosov homeomorphism $f$ of the boundary surface $\d W$. The model manifold $ M_n$ is a degree $n$ cyclic cover of the mapping torus $M_{f}$ cut open along a fiber with two oppositely oriented copies of $W$, denoted $W^+$ and $W^-$ glued as described in \cite{BMNS} Section 2.15 to the two boundary components of the cut open mapping torus. This decomposes $W_n$ into three pieces, a product region $S\times [0, n]$ and the caps $W^+$ and $W^-$ in a metrically controlled way. It will be convenient to set $M^+ = W^+\subset M_n$ and $M^- = W^-\subset M_n$ when talking about the caps of the model manifold $M_n$ for fixed $n$, and to let $W^+$ and $W^-$ denote the images of these spaces under the natural inclusion into $W_n$.

 \vspace{1cm}
\begin{figure}[H]
\labellist
\small\hair 2pt
 \pinlabel {$M^+$} [ ] at 950 1100
 \pinlabel {$M^-$} [ ] at 2000 1100
 \pinlabel {$S\times[0,n]$} [ ] at 1500 1600
\endlabellist
\centering
\includegraphics[scale=.15]{modelscheme}
\caption{ A schematic picture of the model manifold $M_n$ with caps $M^+$ and $M^-$ two oppositely oriented copies of the tripus manifold.}
\label{fig:caps_schematic}
\end{figure}

Given a multicurve $c$ in $M^{\pm}$, we say $c$ \textbf{bounds on both sides} if there are incompressible surfaces $S^+$ in $M^+$ and $S^-$ in $M^-$ both with boundary homotopic to $c$.

We encode the construction and its essential properties in the following proposition.

\begin{prop}  \label{prop:6.3}There is a family $\{W_n\}$ of closed hyperbolic 3-manifolds with injectivity radius uniformly bounded below and volume growing linearly in $n$ constructed from the tripus and a pseudoAnosov $f$ as described above. Each manifold $W_n$ is $K$-biLipschitz equivalent to the model manifold $M_n$ for some constant $K$ independent of $n$. Any homologically nontrivial loop in $H_1(\d W^{\pm})$ that bounds a surface in $M^{\pm}$ cannot bound on both sides. The pseudoAnosov $f$ is such that for any nonzero class $a\in H_1(\d W^+)$, the stable norm of $f_*^n(a)$ grows exponentially.
\end{prop}

\begin{proof}
Let $W$ be Thurston’s tripus manifold, a compact hyperbolic 3-manifold with totally geodesic boundary a genus 2 surface for which the inclusion map $H_1(\d W;\Z)\to H_1(W;\Z)$ is onto.
The homology of the boundary $\d W$ decomposes as the direct sum of rank 2 submodules $U$ and $V$, where $V\subset H_1(\d W)$ is the image of the boundary map $\d : H_2(W,\d W;\Z)\to H_1(\d W;\Z)$ (which is also the kernel of the inclusion $H_1(\d W)\to H_1(W)$) and $U$ is a compliment of $V$ (note that the inclusion map $H_1(\d W)\to H_1(W)$ restricted to $U$ is an isomorphism).
Let $S$ be a genus 2 surface, which we will use to mark the boundaries of $W^{+}$ and $W^-$. Assume $H_1(S;\Z)$ is generated by $e_1,~e_2,~e_3,~e_4$. Choose a marking $S\to \d W^+$ so in $W^+$ one has $U = \langle e_1,e_2\rangle $ and $V = \langle e_3, e_4\rangle$.
Similarly, choose a marking $S\to \d W^-$ so that in $W^-$ one has $V = \langle e_1,e_2\rangle $ and $U = \langle e_3, e_4\rangle$. We then define $$W_n = W^+\cup_{f^n}W^-$$ where $f:S\to S$ is a pseudo-Anosov that acts on $H_1(S)$ by the symplectic matrix\[ F = \begin{pmatrix}
 2 &  1 & 0 & 0 \\
 1 & 1 & 0 & 0 \\
 0 & 0 & 1 & -1\\
 0 & 0 & -1 & 2
\end{pmatrix} \] For the existence of such a pseudoAnosov mapping class, see the proof Lemma 7.1 in \cite{BD}. This matrix preserves the subspace decomposition above, and so ensures that every curve in $\d W^{\pm}$ that is not nullhomologous in $\d W^{\pm}$ but bounds a surface in $M^{\pm}$ cannot bound on both sides.

The mapping class $f$ acts as an Anosov matrix on $U$ and $V$. This ensures the standard Euclidean $\ell^2$-norm $||F^n(a)||_E$ of an element $a\in H_1(S)$ grows exponentially in $n$ (indeed, for our choice of $F$, it grows like $(\frac{3+\sqrt{5}}{2})^n$). Since norms on finite dimensional real vector spaces are comparable, there is a constant comparing the stable norm induced by the metric inherited from $W$ to the standard Euclidean $\ell^2$-norm on $H_1(S)$.

Lemma 7.3 in \cite{BD} explains how Theorem 8.1 in \cite{BMNS} implies that for large $n$ the manifolds $W_n$ admit a $K$-biLipschitz diffeomorphism $\mu$ from the model manifold $ M_n$ as described above. After increasing $K$, we can drop the large $n$ condition. This then also implies the linear volume growth and injectivity radius bounds.
\end{proof}

\begin{remark}
Using the model manifold, one can easily estimate the Cheeger constant of $W_n$, which will decay like $1/n$.
\end{remark}


\begin{mainthm}  \label{thm:C} The family $W_n$ of closed hyperbolic 3-manifolds from Proposition \ref{prop:6.3}  has 1-form Laplacian spectral gap that vanishes exponentially fast in relation to volume:
$$\sqrt{\lambda(W_n)}\leq B\vol(W_n)e^{-r\vol(W_n)},$$
where $r$ and $B$ are positive positive constants and $\lambda(W_n)$ is the first positive eigenvalue of the 1-form Laplacian on $W_n$.
\end{mainthm}


\begin{proof}
We continue using notation introduced in the previous propositions. Take $\gamma$ in $\d M^+\subset M_n$ to be an embedded geodesic loop representing the class $e_1 \in U \subset H_1(\d W^+)$. Recall from the proof of Proposition \ref{prop:6.3} that $\gamma\subset \d M^+$ does not bound a surface in $M^+$ but that $f^n(\gamma)\subset \d M^-$ bounds a surface in $M^-$. Let $\alpha_n = f^n(\gamma)\subset \d M^- \subset M_n$. Note that $\alpha_n$ and $\gamma$ are isotopic in $M_n$.

Fix $\delta>0$. Consider some positive integer $m$ and  incompressible surface $\Sigma_m$ bounding $\alpha_n^m$ in $M_n$ with $\chi_-(\Sigma_m)/2m - \scl_{M_n}(\gamma) < \delta$ and which minimizes $\chi_-$ among surfaces with boundary $\alpha_n^m$. We can then replace $\Sigma_m$ with a homotopic surface that pushes the boundary of $\Sigma_m$ into the interior of $M^-$ and which intersects $\d M^-$ transversely and essentially in both $\d M^-$ and $\Sigma_m$.
We can then attach an annulus to $\Sigma_m$ cobounding $\alpha_n^m$ and the boundary of the modified surface $\Sigma_m$. This new $\Sigma_m$ bounds $\alpha_n^m$ with a collar neighborhood of the boundary contained entirely in $M^-$ and intersects $\d M^-$ transversely in a union of loops essential in both $\Sigma_m$ and $\d M^-$.

We focus on the portion of $\Sigma_m$ that lies in $M^-$. Define $\Sigma^-_m =\Sigma_m\cap M^-$. If $\Sigma_m$ is contained in $M^-$, then Proposition \ref{prop:6.2}  applied to $\alpha_n^m$ in $M^-$ implies that
$$||[\alpha_n^m]||_{s,\d M^-} = m||[\alpha_n]||_{s,\d M^-} \leq m\scl_{M^-}(\alpha_n)\leq D\chi_-(\Sigma_m^-) = D\chi_-(\Sigma_m) ,$$
where $||\cdot||_{s, \d M^-}$ is the stable norm of $H_1(\d M^-)$. Since $\chi_-(\Sigma_m)/m- \scl_{M_n}(\gamma)\leq \delta$, we conclude $$||[\alpha_n]||_{s,\d M^-}\leq D\scl_{M_n}(\gamma) + D\delta.$$

Our goal now is to get this same estimate for the other possible ways $\Sigma_m$ sits in $M_n$.
\vspace{1cm}
\begin{figure}[H]
\labellist
\small\hair 2pt
 \pinlabel {$M^+$} [ ] at 900 1560
 \pinlabel {$S\times[0,n]$} [ ] at 1480 1650
 \pinlabel {$M^-$} [ ] at 2000 1560
 \pinlabel {$\alpha$} [ ] at 1750 1050
\endlabellist
\centering
\includegraphics[scale=.2]{schematic5}
\caption{ A schematic picture of the simplest case of a surface bounding $\alpha$ in $M^-$.}
\label{fig:alpha_bounds}
\end{figure}

Consider the case that $\Sigma_m$ does not lie entirely in $M^-$. There are two possibilities. The first involves the surface $\Sigma_m$ passing into the product region but not intersecting $M^+$. In this case the surface can be homotoped to lie in $M^-$, so that Proposition \ref{prop:6.2} applies, giving the desired estimate as in the previous case.
\vspace{2cm}
\begin{figure}[H]
\labellist
\small\hair 2pt
 \pinlabel {$M^+$} [ ] at 900 1560
 \pinlabel {$S\times[0,n]$} [ ] at 1480 1650
 \pinlabel {$M^-$} [ ] at 2000 1560
 \pinlabel {$\alpha$} [ ] at 1750 850
\endlabellist
\centering

\includegraphics[scale=.2]{schematic3}
\caption{ A schematic picture of a surface bounding $\alpha$ that passes back into $M^-$ but does not pass into $M^+$.}
\label{fig:pass_back_schematic}
\end{figure}

\vspace{2cm}
\begin{figure}[H]
\labellist
\small\hair 2pt
 \pinlabel {$M^+$} [ ] at 900 1560
 \pinlabel {$S\times[0,n]$} [ ] at 1480 1650
 \pinlabel {$M^-$} [ ] at 2000 1560
 \pinlabel {$\alpha$} [ ] at 1750 850
 \pinlabel {$c_1$} [ ] at 1750 1285
 \pinlabel {$c_0$} [ ] at 1750 1070

\endlabellist
\centering

\includegraphics[scale=.2]{schematic6}
\caption{ A schematic picture of a surface bounding $\alpha$ that passes back into $M^+$. Notice the multicurve $c^- = c_0\cup c_1$ bounds surfaces in $M^+$ and $M^-$, so is homologically trivial. }
\label{fig:bounds_both_sides}
\end{figure}



The second possibility concerns the surface $\Sigma_m$ crossing through the product region into $M^+$ with an essential intersection with $\d \Sigma^+$. In this case, we will see that the surface $\Sigma_m^-$ has boundary homologous to $\alpha_n^m$, which will allow us to apply Proposition \ref{prop:6.2} to obtain the desired estimate. By construction, a sufficiently small collar $C$ of the boundary $\d \Sigma_m$ in $\Sigma_m$ maps into $M^-$, so in particular, a subsurface of $\Sigma_m^-$ has some boundary component that maps to $\alpha_n^m$. That boundary component can be closed by attaching a surface $S^-$ that bounds $\alpha_n^m$ in $M^-$ to $\Sigma_m$.
From this, we see that the multicurve $c^- = \d \Sigma^-_m -\alpha_n $ bounds surfaces in $M^+$ and $M^-$.
Thus by Proposition \ref{prop:6.3}, $c^-$ must be homologically trivial in $\d M^{-}$. Let $x = \d\Sigma_m^- = c^- + \alpha_n^m$. Since $c^-$ is nullhomologous, $||[x]||_{s,\d M^-} = ||[\alpha_n^m]||_{s,\d M^-} = m||[\alpha_n]||_{s,\d M^-}.$
By Proposition \ref{prop:6.2} , $||[x]||_{s,\d M^-}\leq D\scl_{M^-}(x).$ Since $x$ is essential in $\Sigma_m$, we get that $\chi_-(\Sigma_m^-) \leq \chi_-(\Sigma_m)$, then using that $\chi_-(\Sigma_m)/2m - \delta \leq \scl_{M_n}(\alpha_n)$,
we obtain $\scl_{M^-}(x) \leq \chi_-(\Sigma_m^-)/2 \leq m\scl_{M_n}(\alpha_n) + \delta m$.
Putting this all together and dividing by $m$, we get that $$||[\alpha_n]||_{s,\d M^-}\leq D\scl_{M_n}(\alpha_n) + D\delta.$$

We therefore have in each case that there is a constant $D$ independent of $n$ such that $$||[\alpha_n]||_{s,\d M^-}\leq D\scl_{M_n}(\alpha_n) + D\delta.$$ By Proposition \ref{prop:6.3} , $||[\alpha_n]||_{s,\d M^-} = ||[f^n(\gamma)]||_{s,\d M^+}$ grows exponentially in $n$.
Thus for some constants $B>0$ and $r >0$, we have $$Be^{rn}\leq D\scl_{M_n}(\gamma) + D\delta,$$ where we use that $\gamma$ and $\alpha_n$ are homotopic in $M_n$. Using the injectivity radius lower bound and Theorem 3.9 of \cite{Calegari}, we can increase $D$ and drop the additive constant in this inequality. By Proposition \ref{prop:6.3} , the volume growth of the $W_n$ is proportional to $n$ so there is a constant $C$ such that $\vol(W_n)\leq Cn$.
Additionally, using the $K$-biLipschitz comparison of Proposition \ref{prop:6.3} , the length of $\gamma$ in $W_n$ is bounded from above by $2K|\gamma|_{W}$, where $W$ is the tripus. As a result, Theorem A implies that the spectral gap for the 1-form Laplacian of the manifolds $W_n$ vanishes exponentially fast in $n$.
In particular, we have \[\sqrt{\lambda(W_n)} \leq A\vol(W_n)\frac{|\gamma|_{W_n}}{\scl_{W_n}(\gamma)} \leq 2KACB^{-1}D|\gamma|_Wne^{-rn},\] so the result holds after redefining $B$ to be $2KACB^{-1}D|\gamma|_W.$

\end{proof}


%%%%%%%%%%%%%%%%%%%%%%%%%%%%%%%%%%%%%%%%%%%%%%%%%%%%%%%%%%%%%%%%

\section{ANALYSIS OF PERFORMATIVE RISK MINIMIZING GRADIENT FLOW}
\label{sec:analysis_prm}
% !TEX root = main.tex

In this section, we consider PRM gradient flow, defined by Equation~\eqref{eq:prm_flow}. We observe that gradient flows provide complete vector fields, and that trajectories will converge to local performative risk minimizers under very mild conditions.

First, we state a proposition guaranteeing that flow is well-defined. The compact sublevel sets ensure that trajectories of Equation~\eqref{eq:prm_flow} remain bounded, which is sufficient to guarantee existence and uniqueness of solutions globally. For proof of the following proposition, we refer the reader to either~{\citet[Section 3.1]{Khalil:2001wj}} or~{\citet[Section 9.3]{Hirsch:2012tx}}.

\begin{proposition}[Existence and uniqueness of gradient flows]

Suppose the performative risk $PR(\cdot)$ is continuously differentiable, and its sublevel sets $\{ x : PR(x) \le c \}$ are compact for every $c \in \mb{R}$. Then for any initial condition $\xnom(0) = x_0$, there exists a unique solution to the differential equation in Equation~\eqref{eq:prm_flow}, defined for all $t \ge 0$.

\end{proposition}

Next, we note that gradient flows have nice properties from the perspective of optimization. Namely: every isolated local minima is locally asymptotically stable, and we can provide sufficient conditions to characterize a subset of the region of convergence.

\begin{proposition}[Convergence of gradient flows]

Suppose the performative risk $PR(\cdot)$ is twice continuously differentiable, and $x^*$ is an isolated local performative risk minimizer. Then $x^*$ is a locally asymptotically stable equilibrium of Equation~\eqref{eq:prm_flow}. 
Furthermore, take any $c$ such that $PR(x^*) \le c$. Let $A \subseteq \{ x : PR(x) \le c \}$ denote the connected component of $\{ x : PR(x) \le c \}$ that contains $x^*$. If $x^*$ is the only local performative minimizer in $A$, then all solutions with initial conditions in $A$ converge to $x^*$.

\end{proposition}

\begin{proof}
Since $x^*$ is an isolated local minimizer and the performative risk is twice continuously differentiable, there exists a neighborhood $U \ni x^*$ such that $\nabla PR(\cdot)$ is non-zero for all $x \neq x^*$. By continuity, there exists some constant $\eps$ such that the connected component of $\{ x : PR(x) \le PR(x^*) + \eps \}$ containing $x^*$ is contained in $U$. Since it is a sublevel set of $PR(\cdot)$ and $\mc{L}_{\fnom}PR(x) < 0$ on its boundary, it is positively invariant. Furthermore, since $\mc{L}_{\fnom}(x) < 0$ for all $x \neq x^*$ on this set, $x^*$ is locally asymptotically stable by standard Lyapunov arguments (see, e.g.~\citet[Section 4]{Khalil:2001wj}).
\end{proof}

The sublevel sets of the performative risk are positively invariant with respect to the PRM gradient flow. Furthermore, because of the continuity of trajectories, each connected component will also be positively invariant. This, in tandem with the fact that trajectories must either converge to a local minima or go off to infinity, also implies the previous proposition.

With minimal assumptions, isolated local performative risk minimizers are all locally attractive in the PRM gradient flow. In Section~\ref{sec:analysis_RGD}, we will view the PRM gradient flow as the nominal dynamics. From this perspective, we analyze the RGD flow as a perturbation from these nominal dynamics. To be able to do any perturbation-based analysis, we will need some stronger conditions on the convergence of the gradient flow associated with performative risk minimization. We note these assumptions here.

\begin{assumption}[Sufficient curvature of the PR]
\label{ass:exist_V}

Fix some isolated local performative risk minimizer $x^*$. 
We assume there exists positive constants $c_1$, $c_2$, $c_3$, $c_4$ and $\delta$ such that the following holds in a neighborhood of $x^*$:
\begin{equation}
\label{eq:v_ineq1}
c_1 |x-x^*|^2 \le PR(x) - PR(x^*) \le c_2 |x-x^*|^2
\end{equation}
\begin{equation}
\label{eq:v_ineq2}
c_3 |x - x^*| - \delta \le | \nabla PR(x) | \le c_4 |x - x^*| + \delta
\end{equation}

We will let $r$ denote the radius of this neighborhood, so the above inequalities are valid on the set $\{ x : |x - x^*| \le r \}$.

\end{assumption}
Assumption~\ref{ass:exist_V} provides conditions on which $V(x) = PR(x) - PR(x^*)$ can be used as a Lyapunov function locally. Next, we provide conditions directly on the loss $\ell(\cdot)$ and the decision-dependent distribution shift $\mc{D}(\cdot)$ which can ensure that Assumption 1 holds. First, we provide sufficient conditions for the bounds in Equation~\eqref{eq:v_ineq1}.

\begin{proposition}[Performative risk bounds]
\label{prop:pr_bnds}
Let $x^*$ be a performative risk minimizer and fix any $x$. If:
\begin{enumerate}
    \item $\ell(\cdot,x)$ is $L_1$ Lipschitz continuous
    \item $\mc{W}_1(\mc{D}(x),\mc{D}(x^*)) \le L_2 |x-x^*|^2$
    \item $\ell(z,\cdot)$ is $m$-strongly convex and $L_3$-smooth for every $z$
\end{enumerate}
Then: $(m/2 - L_1 L_2) |x-x^*|^2 \le PR(x) - PR(x^*) \le (L_1 L_2 + L_3/2) |x-x^*|^2$.
\end{proposition}

\begin{proof}
First, we can break up the performative risk into two parts: $PR(x) - PR(x^*) = R(x,x) - R(x^*,x^*) = [R(x,x) - R(x,x^*)] + [R(x,x^*) - R(x^*,x^*)]$. 
Note that $R(x,x) - R(x,x^*) = \mb{E}_{Z \sim \mc{D}(x)} [\ell(Z,x)] - \mb{E}_{Z \sim \mc{D}(x^*)} [\ell(Z,x)]$. Conditions (1) and (2), along with Kantorovich-Rubenstein duality~\citep{Villani:2003th}, implies this quantity is bounded in absolute value: $|R(x,x) - R(x,x^*)| \le L_1 L_2 |x-x^*|^2$. 
On the other hand, $R(x,x^*) - R(x^*,x^*) = \mb{E}_{Z \sim \mc{D}(x^*)} [\ell(Z,x) - \ell(Z,x^*)]$. By convexity and $L_3$-smoothness, $\ell(z,x) - \ell(z,x^*) \le \langle \nabla_x \ell(z,x^*), x-x^* \rangle + \frac{L_3}{2} |x-x^*|^2$ for any $z$; taking the expectation and noting that $\nabla PR(x^*) = 0$, we have $R(x,x^*) - R(x^*,x^*) \le \frac{L_3}{2} |x-x^*|^2$. In the other direction, using strong convexity and similar arguments, we get: $R(x,x^*) - R(x^*,x^*) \ge \frac{m}{2} |x-x^*|^2$. Combining these results yields the desired results.
\end{proof}
Note that Condition (2) in Proposition~\ref{prop:pr_bnds} is a variation on the typical $\eps$-sensitivity definition. Recall that $\eps$-sensitivity states that for any $x$ and $y$, $\mc{W}_1(\mc{D}(x),\mc{D}(y)) \le \eps|x-y|$~\citep{Perdomo:2020tz}. In contrast, Condition (2) only requires this condition to hold around the point $x^*$, but requires a stricter bound for $x$ close to $x^*$. This bound is also more lax than $\eps$-sensitivity farther away from $x^*$.

Next, we provide sufficient conditions for a bound on the absolute value of the gradient of the performative risk. 
% We fixed this issue: 
%This does not exactly recover Assumption~\ref{ass:exist_V}, as there are additive constants on the upper and lower bounds which do not scale with $|x-x^*|$. However, these additive constants will be small for decision-dependent distribution shifts $\mc{D}(\cdot)$ with sufficiently small sensitivity parameter $\eps$.

\begin{proposition}[Gradient bounds of the performative risk]
\label{prop:grad_bounds}
Let $x^*$ be a performative risk minimizer and fix any $x$. If:
\begin{enumerate}
    \item $\ell(\cdot,x)$ and $\ell(\cdot, x^*)$ are both $L_1$ Lipschitz continuous
    \item $\ell(z,\cdot)$ is $m$-strongly convex and $L_3$-smooth for every $z$
    \item $\mc{D}(\cdot)$ is $\eps$-sensitive, i.e. $\mc{W}_1(\mc{D}(x),\mc{D}(y)) \le \eps |x-y|$
    \item $\nabla_x \ell(\cdot,x)$ is $L_4$ Lipschitz continuous
\end{enumerate}
Then: $(m- \eps L_4)|x-x^*| - 2 \eps L_1 \le |\nabla PR(x)| \le (L_3 + \eps L_4) |x-x^*| + 2 \eps L_1$.
\end{proposition}

\begin{proof}
Similar to the previous proposition, we break apart this gradient. Note that $\nabla PR(x^*) = 0$, so:
$|\nabla PR(x)| = |\nabla PR(x) - \nabla PR(x^*)| =
|\nabla_{x_1} R(x,x) - \nabla_{x_1} R(x^*,x^*) +
\nabla_{x_2} R(x,x) - \nabla_{x_2} R(x^*,x^*)|$. For the $\nabla_{x_1}$ terms, we have: $m|x-x^*| \le |\nabla_{x_1} R(x,x^*) - \nabla_{x_1}R(x^*,x^*)| \le L_3|x-x^*|$ by standard convexity arguments, and $|\nabla_{x_1} R(x,x) - \nabla_{x_1}R(x,x^*)| \le \eps L_4 |x-x^*|$ by the same Kantorovich-Rubenstein duality argument as the previous proposition. For the $\nabla_{x_2}$ terms, note that the mapping $x_2 \mapsto R(x,x_2)$ is $\eps L_1$ Lipschitz continuous. Thus, $|\nabla_{x_2} R(x,x)| \le \eps L_1$ and similarly $|\nabla_{x_2} R(x^*,x^*)|$. Combining these inequalities yields the desired result.
\end{proof}

Depending on the situation, we may be able to directly verify Assumption~\ref{ass:exist_V}, although, for more complex settings, this is likely to be very difficult. Propositions~\ref{prop:pr_bnds} and~\ref{prop:grad_bounds} provide a set of sufficient conditions for this assumption to hold, but checking the conditions on the decision-dependent distribution shift $\mc{D}(\cdot)$ may be difficult in practice as well. This is one limitation of this current work, and we believe it is an interesting future research direction to identify conditions which are easy to verify, even in settings with limited information about the distribution shift itself.


%%%%%%%%%%%%%%%%%%%%%%%%%%%%%%%%%%%%%%%%%%%%%%%%%%%%%%%%%%%%%%%%

\section{ANALYSIS OF REPEATED RISK MINIMIZING FLOW}
\label{sec:analysis_RGD}
% !TEX root = main.tex

In the previous section, we consider the PRM gradient flow and showed that the trajectories converge to local performative risk minimizers in very general settings. In this section, we will consider the RGD flow, defined by Equation~\eqref{eq:RGD_flow}. 
The RGD flow is not necessarily a gradient flow, and generally will not inherit the nice properties we saw in Section~\ref{sec:analysis_prm}.

The following theorem provides conditions on the transient response and steady-state behavior of the RGD flow. Prior to $T$, the trajectories converge exponentially quickly. After $T$, we have an ultimate bound that holds.

\begin{theorem}[Ultimate bounds for RGD flow]
\label{th:perturb1}

Fix any isolated performative risk minimizer $x^*$ and suppose the conditions of Assumption~\ref{ass:exist_V} hold. Let $(c_i)_{i=1}^4$ and $\delta$ denote the constants from Assumption~\ref{ass:exist_V} and $r > 0$ denote the radius where the inequalities are valid.

Suppose that there exists positive constants $\eps < c_3^2/c_4$ such that the following holds on $U = \{ x : |x - x^*| \le r \}$:
\begin{equation}
\label{eq:ass_V1}
|\nabla_{x_2}R(x,x)| \le \eps |x-x^*| + \delta
\end{equation}
Additionally, suppose the initial condition satisfies:
\[
|x_0 - x^*| \le \sqrt{\frac{c_1}{c_2}}r
\]
Take any $\theta \in (0,1)$ such that:
\[
\delta \le \sqrt{\frac{c_1}{c_2}} 
\frac{(1 - \theta) r (c_3^2/c_4 - \eps)}{c_4 + 2c_3 + \epsilon}
\]
Then, there exists a $T \ge 0$ such that:
\begin{itemize}
\item For all $t \le T$:
\[
\begin{aligned}
&|\varphi_{\fpert}(t;x_0) - x^*| \le \\
&\qquad\sqrt{\frac{c_2}{c_1}} \exp(-t\theta (c_3^2 - c_4\eps)/2c_2) |x_0 - x^*|
\end{aligned}
\]
\item For all $t \ge T$:
\[
\begin{aligned}
    &|\varphi_{\fpert}(t;x_0) - x^*| 
    \le \\
    &\qquad\sqrt{\frac{c_2}{c_1}} 
    \max
    \left\{
    \frac{\delta (c_4 + 2c_3 + \epsilon)}{ (1-\theta) (c_3^2 - c_4 \eps)},
    \frac{\delta}{c_3}
    \right\}.
\end{aligned}
\]
\end{itemize}

\end{theorem}

\begin{proof}
See Appendix~\ref{app:proof_perturb1}.
\end{proof}

Note that, in the special case where $\delta = \lambda = 0$, we have that the RGD flow converges exponentially quickly to $x^*$ locally. Similarly, in the special case where Assumption~\ref{ass:exist_V} holds everywhere (i.e. $r = \infty$), then there is only one minimizer $x^*$, and all initial conditions converge to a neighborhood of $x^*$ exponentially fast. In addition, if every condition in Assumption 1 holds with equality, the bounds in Theorem 1 also hold with equality. Please see Appendix A for an example where this occurs.

Additionally, note that locally performatively stable points are equilibria of the RGD flow. This result provides constraints on where performatively stable points can be located. Consider the special case where Assumption~\ref{ass:exist_V} holds globally (i.e. $r = \infty$) and, consequently, there exists only one minimizer $x^*$. In this special case, Theorem~\ref{th:perturb1} shows that all performatively stable points must be close to $x^*$. The phenomena that, under certain conditions, performatively stable points are near performative risk minimizers, was first noted in~\citet{Perdomo:2020tz}. Our results here provide another set of conditions under which the same result holds.

Furthermore, in the presence of Propisition 3 and 4, Theorem 1 can be restated in terms of $\ell(\cdot)$ and $\cD(\cdot)$ as follows:

\begin{corollary}
    Let $x^*$ be a performative risk minimizer and fix any $x$. If:
    \begin{enumerate}
    \item $\ell(\cdot,x)$ and $\ell(\cdot, x^*)$ are both $L_1$ Lipschitz continuous
    \item $\ell(z,\cdot)$ is $m$-strongly convex and $L_3$-smooth for every $z$
    \item $\mc{D}(\cdot)$ is $\eps$-sensitive, i.e. $\mc{W}_1(\mc{D}(x),\mc{D}(y)) \le \eps |x-y|$
    \item $\mc{W}_1(\mc{D}(x),\mc{D}(x^*)) \le L_2 |x-x^*|^2$
    \item $\nabla_x \ell(\cdot,x)$ is $L_4$ Lipschitz continuous
\end{enumerate}
Suppose the initial condition satisfies:
\[
|x_0 - x^*| \le \sqrt{\frac{m - 2L_1L_2}{L_3 + 2L_1L_2}}r
\]
Then, there exists a $T \ge 0$ such that:
\begin{itemize}
\item For all $t \le T$:
\[
|\varphi_{\fpert}(t;x_0) - x^*| \le 
\]
\[
\sqrt{\frac{L_3 + 2L_1L_2}{m - 2L_1L_2}} \exp\left(\frac{-t\theta (m-\epsilon L_4)^2}{L_3 + 2L_1L_2}\right) |x_0 - x^*|
\]
\item For all $t \ge T$:
\[
\begin{aligned}
    &|\varphi_{\fpert}(t;x_0) - x^*| \le 
\sqrt{\frac{L_3 + 2L_1L_2}{m - 2L_1L_2}} \\
&\qquad\cdot\max
\left\{
    \frac{4\epsilon L_1 (L_3 + 2m - \epsilon L_4)}{ (m-\epsilon L_4)^2},
    \frac{2 \epsilon L_1}{m - \epsilon L_4}
\right\}.
\end{aligned}
\]
\end{itemize}
\end{corollary}
\begin{proof}
This follows immediately by combining Theorem~\ref{th:perturb1} with Propositions~\ref{prop:pr_bnds} and~\ref{prop:grad_bounds}.
\end{proof}


%%%%%%%%%%%%%%%%%%%%%%%%%%%%%%%%%%%%%%%%%%%%%%%%%%%%%%%%%%%%%%%%

\subsection{Performative alignment}
% !TEX root = main.tex

From the previous analysis, we also identify conditions on the directions of the performative perturbations that are sufficient to show the convergence of Equation~\eqref{eq:RGD_flow}, the RGD flow, to performative risk minimizers.

\begin{theorem}[Performative alignment]
\label{th:perf_align}
Suppose $x^*$ is a isolated local performative risk minimizer and the following holds for all $x$ in a neighborhood of $x^*$:
\begin{equation}
\label{eq:perf_align}
|\nabla_{x_2}R(x,x)|^2 \le \langle -\nabla_{x_1}R(x,x), \nabla_{x_2}R(x,x) \rangle
\end{equation}
Then $x^*$ is a locally asymptotically stable equilibrium point of the RGD flow, given by Equation~\eqref{eq:RGD_flow}. 
Note that this does \textbf{not} require Assumption~\ref{ass:exist_V}.
\end{theorem}

\begin{proof}
Let $V(x) = PR(x) - PR(x^*)$. 
Since $x^*$ is a locally asymptotically equilibria of the PRM flow, we have: $V(x^*) = 0$, $V(x) > 0$ for $x \neq 0$, and $\mc{L}_{\fnom} V(x) < 0$ for $x \neq 0$. The performative alignment condition ensures that $\mc{L}_{\fnom + g} V(x) < 0$ as well, and the desired result follows.
\end{proof}

We refer to Equation~\eqref{eq:perf_align} as the \textbf{performative alignment} condition. This condition states that the performative perturbation never increases the performative risk, and the convergence of performative risk minimization is sufficient to guarantee convergence of repeated risk minimization. In other words, the perturbation is pointing in the correct direction to ensure that $PR(\cdot) - PR(x^*)$ can still act as a Lyapunov function.

Another perspective on performative alignment is to consider the performative risk as a bilinear form whose arguments are parameterized by $x$. In particular, consider the decoupled performative risk $R(\cdot,\cdot)$. Let $\ell_{x} := \ell(\cdot, x)$ and let $\mu_x$ denote the probability distribution associated with $\mc{D}(x)$. Then, we can write $R(x_1,x_2) = \langle \mu_{x_2}, \ell_{x_1} \rangle$. From this perspective, $R(\cdot,\cdot)$ is a bilinear form in $\ell_x$ and $\mu_x$. As such, the performative alignment condition becomes a condition on the way in which $\ell$ and $\mu$ are \textit{parameterized} by $x$.

In Appendix~\ref{sec:perf_align_ex}, we apply Theorem~\ref{th:perf_align} to the example outlined in Section~\ref{sec:simple_ex}. It provides insight into one of the ways to use Theorem~\ref{th:perf_align}: when we fix a loss $\ell(\cdot)$, we can view the performative alignment condition as specifying a class of decision-dependent distribution shifts which do not hamper the convergence of RGD to performative risk minimizers.


%%%%%%%%%%%%%%%%%%%%%%%%%%%%%%%%%%%%%%%%%%%%%%%%%%%%%%%%%%%%%%%%

% \section{Numerical examples}
% \label{sec:num_results}
% % !TEX root = main.tex

\begin{figure*}[t!]
  \centering
\includegraphics[width=0.325\textwidth]{./figs/0_simp_c1c2.pdf}
\includegraphics[width=0.315\textwidth]{./figs/0_simp_c3c4.pdf}
\includegraphics[width=0.26\textwidth]{./figs/0_simp_quad_bnds.pdf}
  \caption{We verify that the performative risk bounds in Assumption~\ref{ass:exist_V} are satisfied in the example discussed in Section~\ref{sec:example}. (a) As a function of $r$ (the radius of the domain where the inequalities hold), we show the tightest constants $c_1$ and $c_2$ for the bound. We also plot $\sqrt{c_1/c_2}r$, which is the radius of a neighborhood of $x=0$ to which Theorem~\ref{th:perturb1} can be applied. (b) As a function of $r$, we show the tightest constants for $c_3$ and $c_4$. (c) Choosing the $c_1$ and $c_2$ constants for $r = 0.5$, we visualize how the quadratic bounds hold for the performative risk locally.}
  \label{fig:simp_ex_demo}
\end{figure*}
In this section, we revisit the models introduced in Section~\ref{sec:example}. We demonstrate how the results of Sections~\ref{sec:analysis_prm} and~\ref{sec:analysis_RGD} can be applied. First, we show that the example satisfies Assumption~\ref{ass:exist_V} and we calculate its corresponding constants. Second, we apply Theorem~\ref{th:perturb1} and show the theoretical convergence rates match simulated trajectories. 
Finally, we also apply Theorem~\ref{th:perf_align} to the example from Section~\ref{sec:simple_ex} and characterize the class of distribution shifts satisfy the performative alignment condition.

\subsection{Checking the curvature of the performative risk and region of convergence}

Recall the example from Section~\ref{sec:simple_ex}, where $x$ was a scalar, the loss function was the squared error, and the decision-dependent distribution was a Bernoulli random variable whose distribution was determined by $p(\cdot)$. In this section, we consider the specific decision-dependent distribution shift $p = \varphi$, which is defined in Equation~\eqref{eq:varphi_def}.

When we consider this example, we can see that the bounds on Assumption~\ref{ass:exist_V} cannot hold globally, which matches our previous observation that there are multiple isolated performative risk minimizers. However, these bounds may hold locally: we can view the constants $(c_i)_{i=1}^4$ from Assumption~\ref{ass:exist_V} as a function of the size of the domain $r$.

For concreteness, let us focus on the equilibrium point $x = 0$. Recall that Assumption~\ref{ass:exist_V} must hold locally, on the domain $\{ x : |x-x^*| \le r\}$. As we increase $r$, the constants will worsen; we visualize this in Figure~\ref{fig:simp_ex_demo}(a)--(b). Note that these bounds only have to hold locally around the equilibria, as visualized in Figure~\ref{fig:simp_ex_demo}(c). Furthermore, the gradient bounds in Assumption~\ref{ass:exist_V} cannot hold beyond $r > 0.40$, since $\nabla PR(x) = 0$ at that point.

Recall that the convergence results of Theorem~\ref{th:perturb1} can only apply to all initial conditions satisfying $|x_0 - x^*| < \sqrt{c_1/c_2}r$; we visualize this as well in Figure~\ref{fig:simp_ex_demo}(a). 
On the set $(0,0.40]$, we can see the quantity $\sqrt{c_1/c_2}r$ is the largest at $r = 0.4$, with constants $c_1 = 0.50$ and $c_2 = 1.78$. 
Thus, around the equilibrium $x = 0$, the theorem can be applied to all points in the set $\{ x : |x| \le 0.21 \}$, with $\delta = 0$. Thus, our theorem shows that all points in this neighborhood of $x = 0$ will converge. This under-approximates the true region of attraction, which we numerically saw to be $\{ x : x < 0.23 \}$.



\subsection{Performative alignment with squared error and Bernoulli distributions}
\label{sec:perf_align_ex}

We again consider the example from Section~\ref{sec:simple_ex}. However, in this section, we consider a general decision-dependent distribution shift $p(\cdot)$. 
% Recall the example from Section~\ref{sec:simple_ex}, where $x$ was a scalar, the loss function was the squared error, and the decision-dependent distribution was a Bernoulli random variable whose distribution was determined by $p(\cdot)$.  
We suppose that $p(0) = 0$ and $p(1) = 1$, so we have two performative risk minimizers as in our previous example. We have $\nabla_{x_1}R(x,x) = x - p(x)$ and $\nabla_{x_2}R(x,x) = (1/2 - x) p'(x)$. 
The performative alignment condition becomes:
\begin{equation}
    \label{eq:perf_align_ex}
    |1/2 - x|^2 |p'(x)|^2 \le (p(x)-x)(1/2 - x)p'(x)
\end{equation}
Theorem~\ref{th:perf_align} states that if this condition holds for all $x \in (0,c)$, then any initial conditions $x_0 \in (0,c)$ will converge to $x = 0$. Similarly, if this condition holds for all $x \in (c,1)$, then all initial conditions in $(c,1)$ will converge to $x = 1$. Theorem~\ref{th:perf_align} also implies that this condition cannot be satisfied for all $x \in (0,1)$, as then these initial conditions would converge to \textit{both} $x = 0$ and $x = 1$.

If we suppose that $p(\cdot)$ is monotonic on $(0,1)$, i.e. $p'(x) \ge 0$, we can also interpret the performative alignment condition as follows. For $x \in (1/2,1)$, the performative alignment condition becomes $p(x) - x \ge (1/2 - x)p'(x)$. In this regime, $(1/2 - x)p'(x) \le 0$. In this setting, if $p(x) - x$ is too negative, the RGD flow will push $x$ away from the nearby minimizer $x = 1$. Similarly, for $x \in (0,1/2)$, the condition becomes $p(x) - x \le (1/2 - x)p'(x)$. In this regime, $(1/2 - x)p'(x) \ge 0$, and the condition states that $p(x) - x$ cannot be too large, or the RGD flow will push $x$ away from the minimizer $x = 0$.

In this section, we used Theorem~\ref{th:perf_align} to identify conditions on the decision-dependent distribution shift $p(\cdot)$ which ensure that the performative risk does not increase even when the dynamics follow repeated gradient descent.
For this example, the condition is that $p$ satisfies Equation~\eqref{eq:perf_align_ex} for all $x \in (0,c)$. 
More generally, the performative alignment condition allow us to specify a class of distribution shifts which behave well with respect to performative risk minimization.



%%%%%%%%%%%%%%%%%%%%%%%%%%%%%%%%%%%%%%%%%%%%%%%%%%%%%%%%%%%%%%%%

\section{CLOSING REMARKS}
\label{sec:conclusion}

In this paper, we analyzed the problem of performative prediction in settings where multiple isolated equilibria may be of interest. We analyzed the gradient flow of performative risk minimization, and identified regions of attraction for various equilibria. We viewed repeated gradient descent flow as a perturbation of the PRM gradient flow. In particular, we used a Lyapunov function for the PRM gradient flow to analyze the trajectories of the RGD flow. We found conditions on which RGD flow will converge to the local PRM minimizers, and conditions on which they will converge to a neighborhood of PRM minimizers. 

These results provide a method to analyze the regions of attraction for various equilibria under repeated risk minimization. In real-world settings with decision-dependent distributions, we expect many situations where the initialization may have a significant outcome on the trajectories and final outcomes. 

%%%%%%%%%%%%%%%%%%%%%%%%%%%%%%%%%%%%%%%%%%%%%%%%%%%%%%%%%%%%%%%%

\subsubsection*{Acknowledgements}
Lillian J. Ratliff is supported by NSF CAREER Award No.1844729.

\bibliography{main}

\appendix
\onecolumn

\section{PROOF OF THEOREM~\ref{th:perturb1}}
\label{app:proof_perturb1}
% !TEX root = main.tex

Let $V(x) = R(x,x) - R(x^*,x^*)$. Note that $V(x) \ge 0$ on $U = \{ x : |x - x^*| \le r \}$ and $V(x) = 0$ if and only if $x = x^*$. Furthermore, note that $\frac{\partial V}{\partial x}(x) = [\nabla_{x_1}R(x,x) + \nabla_{x_2}R(x,x)]^\T$.

Consider the function $t \mapsto V(\varphi_{\fpert}(t;x_0))$ and its time derivative. Also, let $\xpert(t) = \varphi_{\fpert}(t;x_0)$. 
When $|\xpert-x^*| \ge \delta/c_3$, taking the derivative along trajectories of the repeated risk minimization flow and using Equations~\eqref{eq:v_ineq2} and~\eqref{eq:ass_V1}:
\[
\begin{aligned}
    \mc{L}_{\fnom + g}V 
    &=
    \frac{\partial V}{\partial x} (\fnom(x) + g) 
    =
    -|\nabla_{x_1}R + \nabla_{x_2}R|^2 + \langle \nabla_{x_1}R + \nabla_{x_2}R, \nabla_{x_2}R \rangle \\
    &\le
    -\left( c_3 |\xpert-x^*|-\delta \right)^2
    +
    \left( c_4 |\xpert-x^*|+\delta \right) |\nabla_{x_2}R| \\
    &\le
    -\left( c_3 |\xpert-x^*|-\delta \right)^2
    +
    \left( c_4 |\xpert-x^*|+\delta \right) (\epsilon|\xpert-x^*|+\delta)\\
    &=
    -\left(c_4\epsilon-c_3^2\right)|\xpert-x^*|^2
    +\left(c_4\delta + 2c_3\delta+\delta\epsilon\right)|\xpert-x^*|
\end{aligned}
\]
These inequalities are valid so long as $\xpert(t)$ stays within $U$, which we will ensure later in the proof. Note that $\eps$ is sufficiently small (by assumption) to ensure that $-c_3^2 + c_4 \eps < 0$.

Let $\alpha := c_3^2 - c_4 \eps > 0$. Take any $\theta \in (0,1)$ and note that:
\[
\mc{L}_{\fnom + g}V(\xpert) \le - \theta \alpha | \xpert - x^* |^2 - (1 - \theta) \alpha | \xpert - x^* |^2 +\left(c_4\delta + 2c_3\delta+\delta\epsilon\right)|\xpert-x^*|
\]
Now, let 
\[
    \mu(\theta)
    :=
    \max
    \left\{
    \frac
    {
        \left(c_4\delta + 2c_3\delta+\delta\epsilon\right)
    }
    {\alpha(1 - \theta)},
    \frac{\delta}{c_3}
    \right\}.
\]
If $|\xpert - x^*| \ge \mu(\theta)$, then:
\[
- \theta \alpha | \xpert - x^* |^2 - (1 - \theta) \alpha | \xpert - x^* |^2 +\left(c_4\delta + 2c_3\delta+\delta\epsilon\right)|\xpert-x^*| \le 0,
\]
and thus
\[
\mc{L}_{\fnom + g}V(\xpert) \le - \theta \alpha | \xpert - x^* |^2.
\]
Trajectories of Equation~\eqref{eq:RGD_flow} has two stages: a transient due to its initial condition, and then an ultimate bound due to the perturbation. Let $T(\theta) = \inf~\{ t \ge 0 : |\xpert(t) - x^*| \le \mu(\theta) \}$. Prior to $T(\theta)$, we have:
\[
\frac{d}{dt} V(\xpert(t)) \le - \theta \alpha |\xpert(t) - x^*|^2 \le - \frac{\theta \alpha}{c_2} V(\xpert(t))
\]
The latter follows from Equation~\eqref{eq:v_ineq1}. 
By the comparison principle (see, e.g.~\citep[Lemma 3.4]{Khalil:2001wj}), we have $V(\xpert(t)) \le \exp(-t\theta \alpha / c_2) V(x_0)$. Again using Equation~\eqref{eq:v_ineq1}, this yields the following inequality, valid for all $t \le T(\theta)$:
\[
|\xpert(t) - x^*| \le \sqrt{\frac{c_2}{c_1}} \exp(-t\theta \alpha/2c_2) |x_0 - x^*|
\]
Note that this inequality also provides an upper bound on $T(\theta)$. Additionally, note that this implies the bound $|\xpert(t) - x^*| \le r$, by our assumption on the initial condition. Prior to $T(\theta)$, our trajectory stays in $U$, where our inequalities are valid.

At time $T(\theta)$, we have $|\xpert(t) - x^*| \le \mu(\theta)$. Note that this inequality implies $V(\xpert(t)) \le c_2 \mu^2(\theta)$. Since $\mc{L}_{\fnom + g}V < 0$ on the boundary of $\Omega(\theta) := \{ x : V(x) \le c_2 \mu^2(\theta) \}$, we have that $\Omega(\theta)$ is a positively invariant set. So, for all $t \ge T(\theta)$, we have $\xpert(t) \in \Omega(\theta)$. Using Equation~\eqref{eq:v_ineq1}, we have the following for all $t \ge T(\theta)$:
\[
|\xpert(t) - x^*| \le 
\sqrt{\frac{c_2}{c_1}} \mu(\theta) 
\]
The condition on $\theta$ ensures that this quantity is bounded by $r$, and the trajectory stays in $U$ for $t \ge T(\theta)$. 
This proves our desired result.

Additionally, we can show that this bound is tight by considering the following example. 
Suppose $\cD(x_2)$ is the point mass distribution (i.e. $p(z) = \delta(z - x_2)$) and $l(z, x_1) = 1/2 |z|^2 + 1/2|x_1|^2$. Then the performative risk is given by $R(x_1, x_2) = 1/2 |x_1|^2 + 1/2|x_2|^2$. It follows that $x^* 0$ is the performative risk minimizer. Following the arguments in Appendix A, one would find that the dynamics of $V(x) = R(x, x) - R(x^*, x^*)$ follows $\frac{d}{dt} V(x(t)) = -2 |x(t) - x^*|^2 = -2 V(x(t))$, which yields 
    $|x(t) - x^*| = \exp(-2t)|x_0 - x^*|$.


\section{NUMERICAL EXAMPLES}
\label{sec:num_results}
% !TEX root = main.tex

\begin{figure*}[t!]
  \centering
\includegraphics[width=0.325\textwidth]{./figs/0_simp_c1c2.pdf}
\includegraphics[width=0.315\textwidth]{./figs/0_simp_c3c4.pdf}
\includegraphics[width=0.26\textwidth]{./figs/0_simp_quad_bnds.pdf}
  \caption{We verify that the performative risk bounds in Assumption~\ref{ass:exist_V} are satisfied in the example discussed in Section~\ref{sec:example}. (a) As a function of $r$ (the radius of the domain where the inequalities hold), we show the tightest constants $c_1$ and $c_2$ for the bound. We also plot $\sqrt{c_1/c_2}r$, which is the radius of a neighborhood of $x=0$ to which Theorem~\ref{th:perturb1} can be applied. (b) As a function of $r$, we show the tightest constants for $c_3$ and $c_4$. (c) Choosing the $c_1$ and $c_2$ constants for $r = 0.5$, we visualize how the quadratic bounds hold for the performative risk locally.}
  \label{fig:simp_ex_demo}
\end{figure*}
In this section, we revisit the models introduced in Section~\ref{sec:example}. We demonstrate how the results of Sections~\ref{sec:analysis_prm} and~\ref{sec:analysis_RGD} can be applied. First, we show that the example satisfies Assumption~\ref{ass:exist_V} and we calculate its corresponding constants. Second, we apply Theorem~\ref{th:perturb1} and show the theoretical convergence rates match simulated trajectories. 
Finally, we also apply Theorem~\ref{th:perf_align} to the example from Section~\ref{sec:simple_ex} and characterize the class of distribution shifts satisfy the performative alignment condition.

\subsection{Checking the curvature of the performative risk and region of convergence}

Recall the example from Section~\ref{sec:simple_ex}, where $x$ was a scalar, the loss function was the squared error, and the decision-dependent distribution was a Bernoulli random variable whose distribution was determined by $p(\cdot)$. In this section, we consider the specific decision-dependent distribution shift $p = \varphi$, which is defined in Equation~\eqref{eq:varphi_def}.

When we consider this example, we can see that the bounds on Assumption~\ref{ass:exist_V} cannot hold globally, which matches our previous observation that there are multiple isolated performative risk minimizers. However, these bounds may hold locally: we can view the constants $(c_i)_{i=1}^4$ from Assumption~\ref{ass:exist_V} as a function of the size of the domain $r$.

For concreteness, let us focus on the equilibrium point $x = 0$. Recall that Assumption~\ref{ass:exist_V} must hold locally, on the domain $\{ x : |x-x^*| \le r\}$. As we increase $r$, the constants will worsen; we visualize this in Figure~\ref{fig:simp_ex_demo}(a)--(b). Note that these bounds only have to hold locally around the equilibria, as visualized in Figure~\ref{fig:simp_ex_demo}(c). Furthermore, the gradient bounds in Assumption~\ref{ass:exist_V} cannot hold beyond $r > 0.40$, since $\nabla PR(x) = 0$ at that point.

Recall that the convergence results of Theorem~\ref{th:perturb1} can only apply to all initial conditions satisfying $|x_0 - x^*| < \sqrt{c_1/c_2}r$; we visualize this as well in Figure~\ref{fig:simp_ex_demo}(a). 
On the set $(0,0.40]$, we can see the quantity $\sqrt{c_1/c_2}r$ is the largest at $r = 0.4$, with constants $c_1 = 0.50$ and $c_2 = 1.78$. 
Thus, around the equilibrium $x = 0$, the theorem can be applied to all points in the set $\{ x : |x| \le 0.21 \}$, with $\delta = 0$. Thus, our theorem shows that all points in this neighborhood of $x = 0$ will converge. This under-approximates the true region of attraction, which we numerically saw to be $\{ x : x < 0.23 \}$.



\subsection{Performative alignment with squared error and Bernoulli distributions}
\label{sec:perf_align_ex}

We again consider the example from Section~\ref{sec:simple_ex}. However, in this section, we consider a general decision-dependent distribution shift $p(\cdot)$. 
% Recall the example from Section~\ref{sec:simple_ex}, where $x$ was a scalar, the loss function was the squared error, and the decision-dependent distribution was a Bernoulli random variable whose distribution was determined by $p(\cdot)$.  
We suppose that $p(0) = 0$ and $p(1) = 1$, so we have two performative risk minimizers as in our previous example. We have $\nabla_{x_1}R(x,x) = x - p(x)$ and $\nabla_{x_2}R(x,x) = (1/2 - x) p'(x)$. 
The performative alignment condition becomes:
\begin{equation}
    \label{eq:perf_align_ex}
    |1/2 - x|^2 |p'(x)|^2 \le (p(x)-x)(1/2 - x)p'(x)
\end{equation}
Theorem~\ref{th:perf_align} states that if this condition holds for all $x \in (0,c)$, then any initial conditions $x_0 \in (0,c)$ will converge to $x = 0$. Similarly, if this condition holds for all $x \in (c,1)$, then all initial conditions in $(c,1)$ will converge to $x = 1$. Theorem~\ref{th:perf_align} also implies that this condition cannot be satisfied for all $x \in (0,1)$, as then these initial conditions would converge to \textit{both} $x = 0$ and $x = 1$.

If we suppose that $p(\cdot)$ is monotonic on $(0,1)$, i.e. $p'(x) \ge 0$, we can also interpret the performative alignment condition as follows. For $x \in (1/2,1)$, the performative alignment condition becomes $p(x) - x \ge (1/2 - x)p'(x)$. In this regime, $(1/2 - x)p'(x) \le 0$. In this setting, if $p(x) - x$ is too negative, the RGD flow will push $x$ away from the nearby minimizer $x = 1$. Similarly, for $x \in (0,1/2)$, the condition becomes $p(x) - x \le (1/2 - x)p'(x)$. In this regime, $(1/2 - x)p'(x) \ge 0$, and the condition states that $p(x) - x$ cannot be too large, or the RGD flow will push $x$ away from the minimizer $x = 0$.

In this section, we used Theorem~\ref{th:perf_align} to identify conditions on the decision-dependent distribution shift $p(\cdot)$ which ensure that the performative risk does not increase even when the dynamics follow repeated gradient descent.
For this example, the condition is that $p$ satisfies Equation~\eqref{eq:perf_align_ex} for all $x \in (0,c)$. 
More generally, the performative alignment condition allow us to specify a class of distribution shifts which behave well with respect to performative risk minimization.



%%%%%%%%%%%%%%%%%%%%%%%%%%%%%%%%%%%%%%%%%%%%%%%%%%%%%%%%%%%%%%%%

\end{document}
