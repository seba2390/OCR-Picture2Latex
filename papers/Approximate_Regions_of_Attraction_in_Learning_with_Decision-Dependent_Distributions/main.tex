\documentclass[twoside]{article}

\usepackage[accepted]{aistats2023}
% If your paper is accepted, change the options for the package
% aistats2023 as follows:
%
%\usepackage[accepted]{aistats2023}
%
% This option will print headings for the title of your paper and
% headings for the authors names, plus a copyright note at the end of
% the first column of the first page.

% If you set papersize explicitly, activate the following three lines:
%\special{papersize = 8.5in, 11in}
%\setlength{\pdfpageheight}{11in}
%\setlength{\pdfpagewidth}{8.5in}
\usepackage{amsthm}

% If you use natbib package, activate the following three lines:
\usepackage[round]{natbib}
\renewcommand{\bibname}{References}
\renewcommand{\bibsection}{\subsubsection*{\bibname}}

% If you use BibTeX in apalike style, activate the following line:
\bibliographystyle{apalike}

%%%%%%%%%%%%%%%%%%%%%%%%%%%%%%%%%%%%%%%%%%%%%%%%%%%%%%%%%%%%%%%%%%

\usepackage{graphicx} % DO NOT CHANGE THIS

%%%%%%%%%%%%%%%%%%%%%%%%%%%%%%%%%%%%%%%%%%%%%%%%%%%%%%%%%%%%%%%%%%

% Commonly used symbols and commands.

\usepackage{amsmath}
\usepackage{amssymb}

\newcommand{\eps}{\epsilon}

\newcommand{\mc}{\mathcal}
\newcommand{\mb}{\mathbb}
\newcommand{\mf}{\mathbf}

%%%%%%%%%%%%%%%%%%%%%%%%%%%%%%%%%%%%%%%%%%%%%%%%%%%%%%%%%%%%%%%%%%

% Matrix operations

\newcommand{\T}{\intercal}

%%%%%%%%%%%%%%%%%%%%%%%%%%%%%%%%%%%%%%%%%%%%%%%%%%%%%%%%%%%%%%%%%%

% Probability things

% Equal in distribution
\def\eqd{\,{\buildrel d \over =}\,} 

% Essential supremum
\DeclareMathOperator*{\esssup}{ess\,sup}

%%%%%%%%%%%%%%%%%%%%%%%%%%%%%%%%%%%%%%%%%%%%%%%%%%%%%%%%%%%%%%%%%%

% Environments

\newtheorem{definition}{Definition}
\newtheorem{assumption}{Assumption}
\newtheorem{proposition}{Proposition}
\newtheorem{lemma}{Lemma}
\newtheorem{theorem}{Theorem}
\newtheorem{corollary}{Corollary}
\newtheorem{remark}{Remark}
\newtheorem{example}{Example}

%%%%%%%%%%%%%%%%%%%%%%%%%%%%%%%%%%%%%%%%%%%%%%%%%%%%%%%%%%%%%%%%%%
% following loops stolen from djhsu
\def\ddefloop#1{\ifx\ddefloop#1\else\ddef{#1}\expandafter\ddefloop\fi}
\def\ddef#1{\expandafter\def\csname bb#1\endcsname{\ensuremath{\mathbb{#1}}}}
\ddefloop ABCDEFGHIJKLMNOPQRSTUVWXYZ\ddefloop
\def\ddef#1{\expandafter\def\csname bf#1\endcsname{\ensuremath{\mathbf{#1}}}}
\ddefloop ABCDEFGHIJKLMNOPQRSTUVWXYZabcdefghijklmnopqrstuvwxyz\ddefloop
\def\ddef#1{\expandafter\def\csname bs#1\endcsname{\ensuremath{\boldsymbol{#1}}}}
\ddefloop ABCDEFGHIJKLMNOPQRSTUVWXYZabcdefghijklmnopqrstuvwxyz\ddefloop
\def\ddef#1{\expandafter\def\csname sf#1\endcsname{\ensuremath{\mathsf{#1}}}}
\ddefloop ABCDEFGHIJKLMNOPQRSTUVWXYZ\ddefloop
\def\ddef#1{\expandafter\def\csname c#1\endcsname{\ensuremath{\mathcal{#1}}}}
\ddefloop ABCDEFGHIJKLMNOPQRSTUVWXYZ\ddefloop

\DeclareMathOperator*{\argmax}{argmax}
\DeclareMathOperator*{\argmin}{argmin}

% mark steps in equations
\newcommand\meq[2]{\stackrel{\mathclap{\normalfont\mbox\tiny{#1}}}{#2}}



% Other
\newcommand{\xnom}{x_{PR}}
\newcommand{\dxnom}{\dot x_{PR}}
\newcommand{\xpert}{x_{RR}}
\newcommand{\dxpert}{\dot x_{RR}}

\newcommand{\fnom}{f_{PR}}
\newcommand{\fpert}{f_{RR}}

\newcommand{\phinom}{\varphi_{PR}}
\newcommand{\phipert}{\varphi_{RR}}

\newcommand{\Vnom}{V_{PR}}
\usepackage{enumitem}

\begin{document}

% If your paper is accepted and the title of your paper is very long,
% the style will print as headings an error message. Use the following
% command to supply a shorter title of your paper so that it can be
% used as headings.
%
%\runningtitle{I use this title instead because the last one was very long}

% If your paper is accepted and the number of authors is large, the
% style will print as headings an error message. Use the following
% command to supply a shorter version of the authors names so that
% they can be used as headings (for example, use only the surnames)
%
%\runningauthor{Surname 1, Surname 2, Surname 3, ...., Surname n}

\twocolumn[

\aistatstitle{Approximate Regions of Attraction in Learning with Decision-Dependent Distributions}

\aistatsauthor{ Roy Dong \And Heling Zhang \And  Lillian J. Ratliff }

\aistatsaddress{ UIUC \And  UIUC \And UW } ]

\begin{abstract}
  As data-driven methods are deployed in real-world settings, the processes that generate the observed data will often react to the decisions of the learner. For example, a data source may have some incentive for the algorithm to provide a particular label (e.g. approve a bank loan), and manipulate their features accordingly. Work in strategic classification and decision-dependent distributions seeks to characterize the closed-loop behavior of deploying learning algorithms by explicitly considering the effect of the classifier on the underlying data distribution. More recently, works in performative prediction seek to classify the closed-loop behavior by considering general properties of the mapping from classifier to data distribution, rather than an explicit form. Building on this notion, we analyze repeated risk minimization as the perturbed trajectories of the gradient flows of performative risk minimization. We consider the case where there may be multiple local minimizers of performative risk, motivated by situations where the initial conditions may have significant impact on the long-term behavior of the system. We provide sufficient conditions to characterize the region of attraction for the various equilibria in this settings. Additionally, we introduce the notion of performative alignment, which provides a geometric condition on the convergence of repeated risk minimization to performative risk minimizers.
\end{abstract}

%%%%%%%%%%%%%%%%%%%%%%%%%%%%%%%%%%%%%%%%%%%%%%%%%%%%%%%%%%%%%%%%

\section{INTRODUCTION}
\label{sec:intro}
% !TEX root = ../arxiv.tex

Unsupervised domain adaptation (UDA) is a variant of semi-supervised learning \cite{blum1998combining}, where the available unlabelled data comes from a different distribution than the annotated dataset \cite{Ben-DavidBCP06}.
A case in point is to exploit synthetic data, where annotation is more accessible compared to the costly labelling of real-world images \cite{RichterVRK16,RosSMVL16}.
Along with some success in addressing UDA for semantic segmentation \cite{TsaiHSS0C18,VuJBCP19,0001S20,ZouYKW18}, the developed methods are growing increasingly sophisticated and often combine style transfer networks, adversarial training or network ensembles \cite{KimB20a,LiYV19,TsaiSSC19,Yang_2020_ECCV}.
This increase in model complexity impedes reproducibility, potentially slowing further progress.

In this work, we propose a UDA framework reaching state-of-the-art segmentation accuracy (measured by the Intersection-over-Union, IoU) without incurring substantial training efforts.
Toward this goal, we adopt a simple semi-supervised approach, \emph{self-training} \cite{ChenWB11,lee2013pseudo,ZouYKW18}, used in recent works only in conjunction with adversarial training or network ensembles \cite{ChoiKK19,KimB20a,Mei_2020_ECCV,Wang_2020_ECCV,0001S20,Zheng_2020_IJCV,ZhengY20}.
By contrast, we use self-training \emph{standalone}.
Compared to previous self-training methods \cite{ChenLCCCZAS20,Li_2020_ECCV,subhani2020learning,ZouYKW18,ZouYLKW19}, our approach also sidesteps the inconvenience of multiple training rounds, as they often require expert intervention between consecutive rounds.
We train our model using co-evolving pseudo labels end-to-end without such need.

\begin{figure}[t]%
    \centering
    \def\svgwidth{\linewidth}
    \input{figures/preview/bars.pdf_tex}
    \caption{\textbf{Results preview.} Unlike much recent work that combines multiple training paradigms, such as adversarial training and style transfer, our approach retains the modest single-round training complexity of self-training, yet improves the state of the art for adapting semantic segmentation by a significant margin.}
    \label{fig:preview}
\end{figure}

Our method leverages the ubiquitous \emph{data augmentation} techniques from fully supervised learning \cite{deeplabv3plus2018,ZhaoSQWJ17}: photometric jitter, flipping and multi-scale cropping.
We enforce \emph{consistency} of the semantic maps produced by the model across these image perturbations.
The following assumption formalises the key premise:

\myparagraph{Assumption 1.}
Let $f: \mathcal{I} \rightarrow \mathcal{M}$ represent a pixelwise mapping from images $\mathcal{I}$ to semantic output $\mathcal{M}$.
Denote $\rho_{\bm{\epsilon}}: \mathcal{I} \rightarrow \mathcal{I}$ a photometric image transform and, similarly, $\tau_{\bm{\epsilon}'}: \mathcal{I} \rightarrow \mathcal{I}$ a spatial similarity transformation, where $\bm{\epsilon},\bm{\epsilon}'\sim p(\cdot)$ are control variables following some pre-defined density (\eg, $p \equiv \mathcal{N}(0, 1)$).
Then, for any image $I \in \mathcal{I}$, $f$ is \emph{invariant} under $\rho_{\bm{\epsilon}}$ and \emph{equivariant} under $\tau_{\bm{\epsilon}'}$, \ie~$f(\rho_{\bm{\epsilon}}(I)) = f(I)$ and $f(\tau_{\bm{\epsilon}'}(I)) = \tau_{\bm{\epsilon}'}(f(I))$.

\smallskip
\noindent Next, we introduce a training framework using a \emph{momentum network} -- a slowly advancing copy of the original model.
The momentum network provides stable, yet recent targets for model updates, as opposed to the fixed supervision in model distillation \cite{Chen0G18,Zheng_2020_IJCV,ZhengY20}.
We also re-visit the problem of long-tail recognition in the context of generating pseudo labels for self-supervision.
In particular, we maintain an \emph{exponentially moving class prior} used to discount the confidence thresholds for those classes with few samples and increase their relative contribution to the training loss.
Our framework is simple to train, adds moderate computational overhead compared to a fully supervised setup, yet sets a new state of the art on established benchmarks (\cf \cref{fig:preview}).


%%%%%%%%%%%%%%%%%%%%%%%%%%%%%%%%%%%%%%%%%%%%%%%%%%%%%%%%%%%%%%%%

\section{BACKGROUND}
\label{sec:background}
\section{Background and Motivation}

\subsection{IBM Streams}

IBM Streams is a general-purpose, distributed stream processing system. It
allows users to develop, deploy and manage long-running streaming applications
which require high-throughput and low-latency online processing.

The IBM Streams platform grew out of the research work on the Stream Processing
Core~\cite{spc-2006}.  While the platform has changed significantly since then,
that work established the general architecture that Streams still follows today:
job, resource and graph topology management in centralized services; processing
elements (PEs) which contain user code, distributed across all hosts,
communicating over typed input and output ports; brokers publish-subscribe
communication between jobs; and host controllers on each host which
launch PEs on behalf of the platform.

The modern Streams platform approaches general-purpose cluster management, as
shown in Figure~\ref{fig:streams_v4_v6}. The responsibilities of the platform
services include all job and PE life cycle management; domain name resolution
between the PEs; all metrics collection and reporting; host and resource
management; authentication and authorization; and all log collection. The
platform relies on ZooKeeper~\cite{zookeeper} for consistent, durable metadata
storage which it uses for fault tolerance.

Developers write Streams applications in SPL~\cite{spl-2017} which is a
programming language that presents streams, operators and tuples as
abstractions. Operators continuously consume and produce tuples over streams.
SPL allows programmers to write custom logic in their operators, and to invoke
operators from existing toolkits. Compiled SPL applications become archives that
contain: shared libraries for the operators; graph topology metadata which tells
both the platform and the SPL runtime how to connect those operators; and
external dependencies. At runtime, PEs contain one or more operators. Operators
inside of the same PE communicate through function calls or queues. Operators
that run in different PEs communicate over TCP connections that the PEs
establish at startup. PEs learn what operators they contain, and how to connect
to operators in other PEs, at startup from the graph topology metadata provided
by the platform.

We use ``legacy Streams'' to refer to the IBM Streams version 4 family. The
version 5 family is for Kubernetes, but is not cloud native. It uses the
lift-and-shift approach and creates a platform-within-a-platform: it deploys a
containerized version of the legacy Streams platform within Kubernetes.

\subsection{Kubernetes}

Borg~\cite{borg-2015} is a cluster management platform used internally at Google
to schedule, maintain and monitor the applications their internal infrastructure
and external applications depend on. Kubernetes~\cite{kube} is the open-source
successor to Borg that is an industry standard cloud orchestration platform.

From a user's perspective, Kubernetes abstracts running a distributed
application on a cluster of machines. Users package their applications into
containers and deploy those containers to Kubernetes, which runs those
containers in \emph{pods}. Kubernetes handles all life cycle management of pods,
including scheduling, restarting and migration in case of failures.

Internally, Kubernetes tracks all entities as \emph{objects}~\cite{kubeobjects}.
All objects have a name and a specification that describes its desired state.
Kubernetes stores objects in etcd~\cite{etcd}, making them persistent,
highly-available and reliably accessible across the cluster. Objects are exposed
to users through \emph{resources}. All resources can have
\emph{controllers}~\cite{kubecontrollers}, which react to changes in resources.
For example, when a user changes the number of replicas in a
\code{ReplicaSet}, it is the \code{ReplicaSet} controller which makes sure the
desired number of pods are running. Users can extend Kubernetes through
\emph{custom resource definitions} (CRDs)~\cite{kubecrd}. CRDs can contain
arbitrary content, and controllers for a CRD can take any kind of action.

Architecturally, a Kubernetes cluster consists of nodes. Each node runs a
\emph{kubelet} which receives pod creation requests and makes sure that the
requisite containers are running on that node. Nodes also run a
\emph{kube-proxy} which maintains the network rules for that node on behalf of
the pods. The \emph{kube-api-server} is the central point of contact: it
receives API requests, stores objects in etcd, asks the scheduler to schedule
pods, and talks to the kubelets and kube-proxies on each node. Finally,
\emph{namespaces} logically partition the cluster. Objects which should not know
about each other live in separate namespaces, which allows them to share the
same physical infrastructure without interference.

\subsection{Motivation}
\label{sec:motivation}

Systems like Kubernetes are commonly called ``container orchestration''
platforms. We find that characterization reductive to the point of being
misleading; no one would describe operating systems as ``binary executable
orchestration.'' We adopt the idea from Verma et al.~\cite{borg-2015} that
systems like Kubernetes are ``the kernel of a distributed system.'' Through CRDs
and their controllers, Kubernetes provides state-as-a-service in a distributed
system. Architectures like the one we propose are the result of taking that view 
seriously.

The Streams legacy platform has obvious parallels to the Kubernetes
architecture, and that is not a coincidence: they solve similar problems.
Both are designed to abstract running arbitrary user-code across a distributed
system.  We suspect that Streams is not unique, and that there are many
non-trivial platforms which have to provide similar levels of cluster
management.  The benefits to being cloud native and offloading the platform
to an existing cloud management system are: 
\begin{itemize}
    \item Significantly less platform code.
    \item Better scheduling and resource management, as all services on the cluster are 
        scheduled by one platform.
    \item Easier service integration.
    \item Standardized management, logging and metrics.
\end{itemize}
The rest of this paper presents the design of replacing the legacy Streams 
platform with Kubernetes itself.



%%%%%%%%%%%%%%%%%%%%%%%%%%%%%%%%%%%%%%%%%%%%%%%%%%%%%%%%%%%%%%%%

\section{PERFORMATIVE PREDICTION, FLOWS, AND PERTURBATIONS}
\label{sec:model}
Online convex optimization with memory has emerged as an important and challenging area with a wide array of applications, see \citep{lin2012online,anava2015online,chen2018smoothed,goel2019beyond,agarwal2019online,bubeck2019competitively} and the references therein.  Many results in this area have focused on the case of online optimization with switching costs (movement costs), a form of one-step memory, e.g., \citep{chen2018smoothed,goel2019beyond,bubeck2019competitively}, though some papers have focused on more general forms of memory, e.g., \citep{anava2015online,agarwal2019online}. In this paper we, for the first time, study the impact of feedback delay and nonlinear switching cost in online optimization with switching costs. 

An instance consists of a convex action set $\mathcal{K}\subset\mathbb{R}^d$, an initial point $y_0\in\mathcal{K}$, a sequence of non-negative convex cost functions $f_1,\cdots,f_T:\mathbb{R}^d\to\mathbb{R}_{\ge0}$, and a switching cost $c:\mathbb{R}^{d\times(p+1)}\to\mathbb{R}_{\ge0}$. To incorporate feedback delay, we consider a situation where the online learner only knows the geometry of the hitting cost function at each round, i.e., $f_t$, but that the minimizer of $f_t$ is revealed only after a delay of $k$ steps, i.e., at time $t+k$.  This captures practical scenarios where the form of the loss function or tracking function is known by the online learner, but the target moves over time and measurement lag means that the position of the target is not known until some time after an action must be taken. 
To incorporate nonlinear (and potentially nonconvex) switching costs, we consider the addition of a known nonlinear function $\delta$ from $\mathbb{R}^{d\times p}$ to $\mathbb{R}^d$ to the structured memory model introduced previously.  Specifically, we have
\begin{align}
c(y_{t:t-p}) = \frac{1}{2}\|y_t-\delta(y_{t-1:t-p})\|^2,    \label{e.newswitching}
\end{align}
where we use $y_{i:j}$ to denote either $\{y_i, y_{i+1}, \cdots, y_j\}$ if $i\leq j$, or  $\{y_i, y_{i-1}, \cdots, y_j\}$ if $i > j$ throughout the paper. Additionally, we use $\|\cdot\|$ to denote the 2-norm of a vector or the spectral norm of a matrix.

In summary, we consider an online agent that interacts with the environment as follows:
% \begin{inparaenum}[(i)] 
\begin{enumerate}%[leftmargin=*]
    \item The adversary reveals a function $h_t$, which is the geometry of the $t^\mathrm{th}$ hitting cost, and a point $v_{t-k}$, which is the minimizer of the $(t-k)^\mathrm{th}$ hitting cost. Assume that $h_t$ is $m$-strongly convex and $l$-strongly smooth, and that $\arg\min_y h_t(y)=0$.
    \item The online learner picks $y_t$ as its decision point at time step $t$ after observing $h_t,$  $v_{t-k}$.
    \item The adversary picks the minimizer of the hitting cost at time step $t$: $v_t$. 
    \item The learner pays hitting cost $f_t(y_t)=h_t(y_t-v_t)$ and switching cost $c(y_{t:t-p})$ of the form \eqref{e.newswitching}.
\end{enumerate}

The goal of the online learner is to minimize the total cost incurred over $T$ time steps, $cost(ALG)=\sum_{t=1}^Tf_t(y_t)+c(y_{t:t-p})$, with the goal of (nearly) matching the performance of the offline optimal algorithm with the optimal cost $cost(OPT)$. The performance metric used to evaluate an algorithm is typically the \textit{competitive ratio} because the goal is to learn in an environment that is changing dynamically and is potentially adversarial. Formally, the competitive ratio (CR) of the online algorithm is defined as the worst-case ratio between the total cost incurred by the online learner and the offline optimal cost: $CR(ALG)=\sup_{f_{1:T}}\frac{cost(ALG)}{cost(OPT)}$.

It is important to emphasize that the online learner decides $y_t$ based on the knowledge of the previous decisions $y_1\cdots y_{t-1}$, the geometry of cost functions $h_1\cdots h_t$, and the delayed feedback on the minimizer $v_1\cdots v_{t-k}$. Thus, the learner has perfect knowledge of cost functions $f_1\cdots f_{t-k}$, but incomplete knowledge of $f_{t-k+1}\cdots f_t$ (recall that $f_t(y)=h_t(y-v_t)$).

Both feedback delay and nonlinear switching cost add considerable difficulty for the online learner compared to versions of online optimization studied previously. Delay hides crucial information from the online learner and so makes adaptation to changes in the environment more challenging. As the learner makes decisions it is unaware of the true cost it is experiencing, and thus it is difficult to track the optimal solution. This is magnified by the fact that nonlinear switching costs increase the dependency of the variables on each other. It further stresses the influence of the delay, because an inaccurate estimation on the unknown data, potentially magnifying the mistakes of the learner. 

The impact of feedback delay has been studied previously in online learning settings without switching costs, with a focus on regret, e.g., \citep{joulani2013online,shamir2017online}.  However, in settings with switching costs the impact of delay is magnified since delay may lead to not only more hitting cost in individual rounds, but significantly larger switching costs since the arrival of delayed information may trigger a very large chance in action.  To the best of our knowledge, we give the first competitive ratio for delayed feedback in online optimization with switching costs. 

We illustrate a concrete example application of our setting in the following.

\begin{example}[Drone tracking problem]
\label{example:drone} \emph{
Consider a drone with vertical speed $y_t\in\mathbb{R}$. The goal of the drone is to track a sequence of desired speeds $y^d_1,\cdots,y^d_T$ with the following tracking cost:}
\begin{equation}
    \sum_{t=1}^T \frac{1}{2}(y_t-y^d_t)^2 + \frac{1}{2}(y_t-y_{t-1}+g(y_{t-1}))^2,
\end{equation}
\emph{where $g(y_{t-1})$ accounts for the gravity and the aerodynamic drag. One example is $g(y)=C_1+C_2\cdot|y|\cdot y$ where $C_1,C_2>0$ are two constants~\cite{shi2019neural}. Note that the desired speed $y_t^d$ is typically sent from a remote computer/server. Due to the communication delay, at time step $t$ the drone only knows $y_1^d,\cdots,y_{t-k}^d$.}

\emph{This example is beyond the scope of existing results in online optimization, e.g.,~\cite{shi2020online,goel2019beyond,goel2019online}, because of (i) the $k$-step delay in the hitting cost $\frac{1}{2}(y_t-y_t^d)$ and (ii) the nonlinearity in the switching cost $\frac{1}{2}(y_t-y_{t-1}+g(y_{t-1}))^2$ with respective to $y_{t-1}$. However, in this paper, because we directly incorporate the effect of delay and nonlinearity in the algorithm design, our algorithms immediately provide constant-competitive policies for this setting.}
\end{example}


\subsection{Related Work}
This paper contributes to the growing literature on online convex optimization with memory.  
Initial results in this area focused on developing constant-competitive algorithms for the special case of 1-step memory, a.k.a., the Smoothed Online Convex Optimization (SOCO) problem, e.g., \citep{chen2018smoothed,goel2019beyond}. In that setting, \citep{chen2018smoothed} was the first to develop a constant, dimension-free competitive algorithm for high-dimensional problems.  The proposed algorithm, Online Balanced Descent (OBD), achieves a competitive ratio of $3+O(1/\beta)$ when cost functions are $\beta$-locally polyhedral.  This result was improved by \citep{goel2019beyond}, which proposed two new algorithms, Greedy OBD and Regularized OBD (ROBD), that both achieve $1+O(m^{-1/2})$ competitive ratios for $m$-strongly convex cost functions.  Recently, \citep{shi2020online} gave the first competitive analysis that holds beyond one step of memory.  It holds for a form of structured memory where the switching cost is linear:
$
    c(y_{t:t-p})=\frac{1}{2}\|y_t-\sum_{i=1}^pC_iy_{t-i}\|^2,
$
with known $C_i\in\mathbb{R}^{d\times d}$, $i=1,\cdots,p$. If the memory length $p = 1$ and $C_1$ is an identity matrix, this is equivalent to SOCO. In this setting, \citep{shi2020online} shows that ROBD has a competitive ratio of 
\begin{align}
    \frac{1}{2}\left( 1 + \frac{\alpha^2 - 1}{m} + \sqrt{\Big( 1 + \frac{\alpha^2 - 1}{m}\Big)^2 + \frac{4}{m}} \right),
\end{align}
when hitting costs are $m$-strongly convex and $\alpha=\sum_{i=1}^p\|C_i\|$. 


Prior to this paper, competitive algorithms for online optimization have nearly always assumed that the online learner acts \emph{after} observing the cost function in the current round, i.e., have zero delay.  The only exception is \citep{shi2020online}, which considered the case where the learner must act before observing the cost function, i.e., a one-step delay.  Even that small addition of delay requires a significant modification to the algorithm (from ROBD to Optimistic ROBD) and analysis compared to previous work. 

As the above highlights, there is no previous work that addresses either the setting of nonlinear switching costs nor the setting of multi-step delay. However, the prior work highlights that ROBD is a promising algorithmic framework and our work in this paper extends the ROBD framework in order to address the challenges of delay and non-linear switching costs. Given its importance to our work, we describe the workings of ROBD in detail in Algorithm~\ref{robd}. 

\begin{algorithm}[t!]
  \caption{ROBD \citep{goel2019beyond}}
  \label{robd}
\begin{algorithmic}[1]
  \STATE {\bfseries Parameter:} $\lambda_1\ge0,\lambda_2\ge0$
  \FOR{$t=1$ {\bfseries to} $T$}
  \STATE {\bfseries Input:} Hitting cost function $f_t$, previous decision points $y_{t-p:t-1}$
  \STATE $v_t\leftarrow\arg\min_yf_t(y)$
  \STATE $y_t\leftarrow\arg\min_yf_t(y)+\lambda_1c(y,y_{t-1:t-p})+\frac{\lambda_2}{2}\|y-v_t\|^2_2$
  \STATE {\bfseries Output:} $y_t$
  \ENDFOR
   
\end{algorithmic}
\end{algorithm}

Another line of literature that this paper contributes to is the growing understanding of the connection between online optimization and adaptive control. The reduction from adaptive control to online optimization with memory was first studied in \citep{agarwal2019online} to obtain a sublinear static regret guarantee against the best linear state-feedback controller, where the approach is to consider a disturbance-action policy class with some fixed horizon.  Many follow-up works adopt similar reduction techniques \citep{agarwal2019logarithmic, brukhim2020online, gradu2020adaptive}. A different reduction approach using control canonical form is proposed by \citep{li2019online} and further exploited by \citep{shi2020online}. Our work falls into this category.  The most general results so far focus on Input-Disturbed Squared Regulators, which can be reduced to online convex optimization with structured memory (without delay or nonlinear switching costs).  As we show in \Cref{Control}, the addition of delay and nonlinear switching costs leads to a significant extension of the generality of control models that can be reduced to online optimization. 

%%%%%%%%%%%%%%%%%%%%%%%%%%%%%%%%%%%%%%%%%%%%%%%%%%%%%%%%%%%%%%%%

\subsection{EXAMPLES}
\label{sec:example}


\section{An example}
\label{sec:6}
\begin{figure}[H]

\includegraphics[scale=.6]{schematic}
\centering
\caption{ A sketch of the example constructed in this section. The manifold decomposes into three pieces, left and right caps and a middle region. The loop drawn on the boundary of the left cap only bounds surfaces of high topological complexity that are at least partly contained in the right cap. This ensures the isoperimetric ratio of that loop is very small.}
\label{fig:example_sketch}
\end{figure}

The aim of this section is to show that the first positive eigenvalue of the 1-form Laplacian can vanish exponentially fast in relation to volume. This contrasts the behaviour of the first positive eigenvalue of the Laplacian on functions.

Our construction is similar to that in \cite{BD}. Essentially, we choose a hyperbolic 3-manifold with totally geodesic boundary and glue it to itself using a particular psuedoAnosov with several useful properties. By \cite{BMNS}, this family has geometry that up to bounded error can be understood in terms of a simple model family. Using this model family, we show that one can find curves with uniformly bounded length whose stable commutator length grows exponentially in the volume. We then use the spectral gap upper bound in Theorem A to conclude the first positive eigenvalue vanishes exponentially fast.

Throughout this section, we need to compare geodesic lengths in different submanifolds of a given manifold $M$. Let $|\cdot|_{X}$ denote the geodesic length of a homotopy class of curves relative endpoints in a manifold $X$ and $\text{length}(\cdot)$ be the length in $M$ of the curve. Similarly, when we compute stable commutator length for the fundamental group of a manifold $X$, which may or may not be a submanifold of $M$, we denote it $\scl_X$.

We will need that for certain curves, $\scl$ is comparable to length. We begin with a simple but essential technical lemma.
\begin{figure}[H]
\labellist
\small\hair 2pt
 \pinlabel {$a_0$} [ ] at 830 1100
 \pinlabel {$a_1$} [ ] at 1300 1100
 \pinlabel {$b_0$} [ ] at 830 1250
 \pinlabel {$b_1$} [ ] at 1300 1250
 \pinlabel {$t_1$} [ ] at 380 1290
 \pinlabel {$t_0$} [ ] at 380 1120
\endlabellist
\centering
\includegraphics[width = 15cm]{lemdiagram}
\caption{ Illustrating Lemma \ref{lem:6.1} , the rectangular base of the figure is part of the totally geodesic surface $S$ and the box is the corresponding part of the tubular neighborhood $N_\e(S)$ foliated by surfaces $S_t$ parallel to $S$. Drawn in the box is the surface $\Sigma$, which is transverse the foliation except at isolated points. The multicurve $a_0\cup a_1$ is part of a single level set $S_{t_0} \cap \Sigma$, but only $a_0$ is part of the curve $c_{t_0}$ described in the lemma, whereas the multicurve $b_0\cup b_1$ forms the multicurve $c_{t_1}$ in the lemma.}
\label{fig:box}
\end{figure}


\begin{lem} \label{lem:6.1} Let $M$ be a compact hyperbolic 3-manifold with totally geodesic boundary $ \d M = S$. Let $\e$ be smaller than the injectivity radius of $M$ and such that $N_\e(S)$ is an embedded tubular neighborhood. Let $\{S_t\}$ be the leaves of the foliation of $N_\e(S)$ by surfaces equidistant from $S$.  Let $\Sigma$ be a smooth incompressible proper not necessarily immersed surface in $M$ that is transverse to the foliation $\{S_t\}$ except at isolated points. Let $c = \d \Sigma$. By transversality, for generic $t$ the multicurve $c_t$ given by the part of $S_t\cap \Sigma$ that cobounds a subsurface of $\Sigma$ with $c = c_0$ is a smooth multicurve.  Let $T$ be the set (of full measure) of all $t\in[0,\e)$ such that $c_t$ is a smooth multicurve. Since $S$ is totally geodesic, each multicurve $c_t$ is homotopic to a possibly degenerate geodesic multicurve $\gamma_t$ in $S$. Let $\Sigma_\e$ be the part of $\Sigma$ contained in $N_\e(S)$. Then for $C = 1/\e$, we have that $$\inf\limits_{t\in T} |\gamma_t|_{S} \leq C \emph{Area}(\Sigma_\e).$$
\end{lem}

\begin{proof} The coarea formula implies the inequality $\inf\limits_{t\in T} |c_t|_{S_t} \leq C \text{Area}(\Sigma_\e)$.  Since $S$ is totally geodesic, for all $t\in T$, we have $|\gamma_t|_S\leq |c_t|_{S_t}.$  \end{proof}
Note that in the previous lemma, when $\inf\limits_{t\in T} |\gamma_t|_S$ is zero, because $\Sigma$ is incompressible and any loop with length less than $\inj(M)$ bounds a disk, every component of $\Sigma$ can either be homotoped to be disjoint from $N_\e(S)$ or be contained in $S$.

The next proposition requires a notion of geometric complexity for homology classes. For any compact Riemannian manifold $M$ one can define the stable norm on the first homology of $M$ (see \cite{Gromovmetric} Section 4C). The mass of a Lipschitz 1-chain $\alpha = \sum_i t_i\alpha_i$ in $M$ is defined to be $\text{mass}(\alpha) = \sum_i |t_i|\text{length}(\alpha_i).$ The mass of a class $a\in H_1(M)$ is then the infimal value of the mass of a chain $\alpha$ representing $a$.
For a class $a\in H_1(M)$, the stable norm of $a$ is then given by $$||a||_{s,M} = \inf\limits_{m>0}\frac{\text{mass}(m a)}{m}.$$

Stable commutator length can also be generalized to geodesic multicurves (see Section 2.6 of \cite{Calegari}), which can naturally be viewed as Lipschitz chains. Suppose $\gamma_i\in \pi_1 M$ and $ \sum_i n [\gamma_i] = 0$ in $H_1(M)$. Let $\gamma$ be the geodesic multicurve, which is not necessarily simple, consisting of the geodesic loops determined by $\gamma_i$. Say a surface $f:S\to M$ is admissible of degree $n(S)$ if it has no closed components and $\d S$ is a union of circles $S^1_i$ with $f|_{S^1_i}$ a degree $n(S)$ cover of $\gamma_i$.
Then we define stable commutator length of $\gamma$ to be $$\scl(\gamma) = \inf\limits_{S \text{ admissible}} \frac{\chi_-(S)}{2n(S)}.$$ When $\gamma$ is a single loop, this definition agrees with the usual definition of stable commutator length.

 \begin{prop} \label{prop:6.2} Let $M$ be a compact oriented hyperbolic 3-manifold with totally geodesic boundary $\d M = S$. Let $\gamma$ be a geodesic multicurve in $S$ that is rationally nullhomologous in $M$. Then there is a constant $D> 0$ depending only on $M$ such that $$||[\gamma]||_{s,S}\leq D\scl_M(\gamma).$$ \end{prop}

 \begin{proof} If $\gamma$ is nullhomologous in $S$, then the left hand side is zero and the inequality holds. Assume now that $[\gamma]\neq 0\in H_1(S)$. Fix $\delta>0$. Let $\Sigma_m$ be an incompressible admissible surface for $\gamma$ of degree $m = n(S)$ such that $\chi_-(\Sigma_m)/2m -\scl(\gamma) < \delta$. We can triangulate $\Sigma_m$ so that there is a single vertex on each boundary component. This triangulation has $4g +3b - 4$ faces, where $g$ is the genus of $\Sigma_m$ and $b$ the number of boundary components. We can then straighten this triangulation to obtain a piecewise totally geodesic triangulated surface. Replace $\Sigma_m$ with this surface. Since every face of this triangulation of $\Sigma_m$ is geodesic, every face has area at most $\pi$. Since there are $4g+3b - 4$ faces and $\chi_-(\Sigma_m) = 2g-2 + b$, we can estimate  $$\text{Area}(\Sigma_m) \leq 3\pi\chi_-(\Sigma_m).$$

We can perturb $\Sigma_m$ to obtain a smooth surface $\Sigma’_m$ that it is transverse the foliation of $N_\e(S)$ except at isolated points and in doing so increase the area by less than $\delta$. Let $\gamma_t$ be the family of multicurves in Lemma \ref{lem:6.1}  applied to $\Sigma’_m$. Since each curve $\gamma_t$ cobounds a surface in $S$ with $\d \Sigma_m$, they are homologous, thus $||[\gamma_t]||_{s,S} = m||[\gamma]||_{s,S}$. Since $||[\gamma_t]||_{s,S}\leq |\gamma_t|_S$,
Lemma \ref{lem:6.1}  implies that $$m||[\gamma]||_{s,S}\leq C\text{Area}(\Sigma_m’) \leq C \text{Area}(\Sigma_m) + C\delta \leq 3C\pi\chi_-(\Sigma_m) + C\delta.$$
From this we get  $$||[\gamma]||_{s,S}\leq 6C\pi\chi_-(\Sigma_m)/2m + C\delta/m\leq 6C\pi\scl_M(\gamma) + 6C\pi\delta +C\delta/m.$$
Since the stable commutator length of a nontrivial rational commutator is bounded away from zero by a constant only depending on $M$, by Theorem 3.9 in \cite{Calegari}, we can replace $C$ with a larger constant $D$ such that $$||[\gamma]||_{s,S}\leq D\scl_M(\gamma),$$ as desired. \end{proof}

 We now introduce the family of manifolds that we use in our construction. The family $\{W_n\}$ of manifolds we study are easily understood using the model manifold theory of \cite{BMNS}. In particular, there is a $K$-biLipschitz map between $W_n$ and a model manifold $M_n$, where $K$ is independent of $n$. The base of the construction is Thurston’s tripus manifold $W$ (see \cite{thurstonbook}, Section 3.3.12), a hyperbolic manifold with totally geodesic boundary, and a psuedoAnosov homeomorphism $f$ of the boundary surface $\d W$. The model manifold $ M_n$ is a degree $n$ cyclic cover of the mapping torus $M_{f}$ cut open along a fiber with two oppositely oriented copies of $W$, denoted $W^+$ and $W^-$ glued as described in \cite{BMNS} Section 2.15 to the two boundary components of the cut open mapping torus. This decomposes $W_n$ into three pieces, a product region $S\times [0, n]$ and the caps $W^+$ and $W^-$ in a metrically controlled way. It will be convenient to set $M^+ = W^+\subset M_n$ and $M^- = W^-\subset M_n$ when talking about the caps of the model manifold $M_n$ for fixed $n$, and to let $W^+$ and $W^-$ denote the images of these spaces under the natural inclusion into $W_n$.

 \vspace{1cm}
\begin{figure}[H]
\labellist
\small\hair 2pt
 \pinlabel {$M^+$} [ ] at 950 1100
 \pinlabel {$M^-$} [ ] at 2000 1100
 \pinlabel {$S\times[0,n]$} [ ] at 1500 1600
\endlabellist
\centering
\includegraphics[scale=.15]{modelscheme}
\caption{ A schematic picture of the model manifold $M_n$ with caps $M^+$ and $M^-$ two oppositely oriented copies of the tripus manifold.}
\label{fig:caps_schematic}
\end{figure}

Given a multicurve $c$ in $M^{\pm}$, we say $c$ \textbf{bounds on both sides} if there are incompressible surfaces $S^+$ in $M^+$ and $S^-$ in $M^-$ both with boundary homotopic to $c$.

We encode the construction and its essential properties in the following proposition.

\begin{prop}  \label{prop:6.3}There is a family $\{W_n\}$ of closed hyperbolic 3-manifolds with injectivity radius uniformly bounded below and volume growing linearly in $n$ constructed from the tripus and a pseudoAnosov $f$ as described above. Each manifold $W_n$ is $K$-biLipschitz equivalent to the model manifold $M_n$ for some constant $K$ independent of $n$. Any homologically nontrivial loop in $H_1(\d W^{\pm})$ that bounds a surface in $M^{\pm}$ cannot bound on both sides. The pseudoAnosov $f$ is such that for any nonzero class $a\in H_1(\d W^+)$, the stable norm of $f_*^n(a)$ grows exponentially.
\end{prop}

\begin{proof}
Let $W$ be Thurston’s tripus manifold, a compact hyperbolic 3-manifold with totally geodesic boundary a genus 2 surface for which the inclusion map $H_1(\d W;\Z)\to H_1(W;\Z)$ is onto.
The homology of the boundary $\d W$ decomposes as the direct sum of rank 2 submodules $U$ and $V$, where $V\subset H_1(\d W)$ is the image of the boundary map $\d : H_2(W,\d W;\Z)\to H_1(\d W;\Z)$ (which is also the kernel of the inclusion $H_1(\d W)\to H_1(W)$) and $U$ is a compliment of $V$ (note that the inclusion map $H_1(\d W)\to H_1(W)$ restricted to $U$ is an isomorphism).
Let $S$ be a genus 2 surface, which we will use to mark the boundaries of $W^{+}$ and $W^-$. Assume $H_1(S;\Z)$ is generated by $e_1,~e_2,~e_3,~e_4$. Choose a marking $S\to \d W^+$ so in $W^+$ one has $U = \langle e_1,e_2\rangle $ and $V = \langle e_3, e_4\rangle$.
Similarly, choose a marking $S\to \d W^-$ so that in $W^-$ one has $V = \langle e_1,e_2\rangle $ and $U = \langle e_3, e_4\rangle$. We then define $$W_n = W^+\cup_{f^n}W^-$$ where $f:S\to S$ is a pseudo-Anosov that acts on $H_1(S)$ by the symplectic matrix\[ F = \begin{pmatrix}
 2 &  1 & 0 & 0 \\
 1 & 1 & 0 & 0 \\
 0 & 0 & 1 & -1\\
 0 & 0 & -1 & 2
\end{pmatrix} \] For the existence of such a pseudoAnosov mapping class, see the proof Lemma 7.1 in \cite{BD}. This matrix preserves the subspace decomposition above, and so ensures that every curve in $\d W^{\pm}$ that is not nullhomologous in $\d W^{\pm}$ but bounds a surface in $M^{\pm}$ cannot bound on both sides.

The mapping class $f$ acts as an Anosov matrix on $U$ and $V$. This ensures the standard Euclidean $\ell^2$-norm $||F^n(a)||_E$ of an element $a\in H_1(S)$ grows exponentially in $n$ (indeed, for our choice of $F$, it grows like $(\frac{3+\sqrt{5}}{2})^n$). Since norms on finite dimensional real vector spaces are comparable, there is a constant comparing the stable norm induced by the metric inherited from $W$ to the standard Euclidean $\ell^2$-norm on $H_1(S)$.

Lemma 7.3 in \cite{BD} explains how Theorem 8.1 in \cite{BMNS} implies that for large $n$ the manifolds $W_n$ admit a $K$-biLipschitz diffeomorphism $\mu$ from the model manifold $ M_n$ as described above. After increasing $K$, we can drop the large $n$ condition. This then also implies the linear volume growth and injectivity radius bounds.
\end{proof}

\begin{remark}
Using the model manifold, one can easily estimate the Cheeger constant of $W_n$, which will decay like $1/n$.
\end{remark}


\begin{mainthm}  \label{thm:C} The family $W_n$ of closed hyperbolic 3-manifolds from Proposition \ref{prop:6.3}  has 1-form Laplacian spectral gap that vanishes exponentially fast in relation to volume:
$$\sqrt{\lambda(W_n)}\leq B\vol(W_n)e^{-r\vol(W_n)},$$
where $r$ and $B$ are positive positive constants and $\lambda(W_n)$ is the first positive eigenvalue of the 1-form Laplacian on $W_n$.
\end{mainthm}


\begin{proof}
We continue using notation introduced in the previous propositions. Take $\gamma$ in $\d M^+\subset M_n$ to be an embedded geodesic loop representing the class $e_1 \in U \subset H_1(\d W^+)$. Recall from the proof of Proposition \ref{prop:6.3} that $\gamma\subset \d M^+$ does not bound a surface in $M^+$ but that $f^n(\gamma)\subset \d M^-$ bounds a surface in $M^-$. Let $\alpha_n = f^n(\gamma)\subset \d M^- \subset M_n$. Note that $\alpha_n$ and $\gamma$ are isotopic in $M_n$.

Fix $\delta>0$. Consider some positive integer $m$ and  incompressible surface $\Sigma_m$ bounding $\alpha_n^m$ in $M_n$ with $\chi_-(\Sigma_m)/2m - \scl_{M_n}(\gamma) < \delta$ and which minimizes $\chi_-$ among surfaces with boundary $\alpha_n^m$. We can then replace $\Sigma_m$ with a homotopic surface that pushes the boundary of $\Sigma_m$ into the interior of $M^-$ and which intersects $\d M^-$ transversely and essentially in both $\d M^-$ and $\Sigma_m$.
We can then attach an annulus to $\Sigma_m$ cobounding $\alpha_n^m$ and the boundary of the modified surface $\Sigma_m$. This new $\Sigma_m$ bounds $\alpha_n^m$ with a collar neighborhood of the boundary contained entirely in $M^-$ and intersects $\d M^-$ transversely in a union of loops essential in both $\Sigma_m$ and $\d M^-$.

We focus on the portion of $\Sigma_m$ that lies in $M^-$. Define $\Sigma^-_m =\Sigma_m\cap M^-$. If $\Sigma_m$ is contained in $M^-$, then Proposition \ref{prop:6.2}  applied to $\alpha_n^m$ in $M^-$ implies that
$$||[\alpha_n^m]||_{s,\d M^-} = m||[\alpha_n]||_{s,\d M^-} \leq m\scl_{M^-}(\alpha_n)\leq D\chi_-(\Sigma_m^-) = D\chi_-(\Sigma_m) ,$$
where $||\cdot||_{s, \d M^-}$ is the stable norm of $H_1(\d M^-)$. Since $\chi_-(\Sigma_m)/m- \scl_{M_n}(\gamma)\leq \delta$, we conclude $$||[\alpha_n]||_{s,\d M^-}\leq D\scl_{M_n}(\gamma) + D\delta.$$

Our goal now is to get this same estimate for the other possible ways $\Sigma_m$ sits in $M_n$.
\vspace{1cm}
\begin{figure}[H]
\labellist
\small\hair 2pt
 \pinlabel {$M^+$} [ ] at 900 1560
 \pinlabel {$S\times[0,n]$} [ ] at 1480 1650
 \pinlabel {$M^-$} [ ] at 2000 1560
 \pinlabel {$\alpha$} [ ] at 1750 1050
\endlabellist
\centering
\includegraphics[scale=.2]{schematic5}
\caption{ A schematic picture of the simplest case of a surface bounding $\alpha$ in $M^-$.}
\label{fig:alpha_bounds}
\end{figure}

Consider the case that $\Sigma_m$ does not lie entirely in $M^-$. There are two possibilities. The first involves the surface $\Sigma_m$ passing into the product region but not intersecting $M^+$. In this case the surface can be homotoped to lie in $M^-$, so that Proposition \ref{prop:6.2} applies, giving the desired estimate as in the previous case.
\vspace{2cm}
\begin{figure}[H]
\labellist
\small\hair 2pt
 \pinlabel {$M^+$} [ ] at 900 1560
 \pinlabel {$S\times[0,n]$} [ ] at 1480 1650
 \pinlabel {$M^-$} [ ] at 2000 1560
 \pinlabel {$\alpha$} [ ] at 1750 850
\endlabellist
\centering

\includegraphics[scale=.2]{schematic3}
\caption{ A schematic picture of a surface bounding $\alpha$ that passes back into $M^-$ but does not pass into $M^+$.}
\label{fig:pass_back_schematic}
\end{figure}

\vspace{2cm}
\begin{figure}[H]
\labellist
\small\hair 2pt
 \pinlabel {$M^+$} [ ] at 900 1560
 \pinlabel {$S\times[0,n]$} [ ] at 1480 1650
 \pinlabel {$M^-$} [ ] at 2000 1560
 \pinlabel {$\alpha$} [ ] at 1750 850
 \pinlabel {$c_1$} [ ] at 1750 1285
 \pinlabel {$c_0$} [ ] at 1750 1070

\endlabellist
\centering

\includegraphics[scale=.2]{schematic6}
\caption{ A schematic picture of a surface bounding $\alpha$ that passes back into $M^+$. Notice the multicurve $c^- = c_0\cup c_1$ bounds surfaces in $M^+$ and $M^-$, so is homologically trivial. }
\label{fig:bounds_both_sides}
\end{figure}



The second possibility concerns the surface $\Sigma_m$ crossing through the product region into $M^+$ with an essential intersection with $\d \Sigma^+$. In this case, we will see that the surface $\Sigma_m^-$ has boundary homologous to $\alpha_n^m$, which will allow us to apply Proposition \ref{prop:6.2} to obtain the desired estimate. By construction, a sufficiently small collar $C$ of the boundary $\d \Sigma_m$ in $\Sigma_m$ maps into $M^-$, so in particular, a subsurface of $\Sigma_m^-$ has some boundary component that maps to $\alpha_n^m$. That boundary component can be closed by attaching a surface $S^-$ that bounds $\alpha_n^m$ in $M^-$ to $\Sigma_m$.
From this, we see that the multicurve $c^- = \d \Sigma^-_m -\alpha_n $ bounds surfaces in $M^+$ and $M^-$.
Thus by Proposition \ref{prop:6.3}, $c^-$ must be homologically trivial in $\d M^{-}$. Let $x = \d\Sigma_m^- = c^- + \alpha_n^m$. Since $c^-$ is nullhomologous, $||[x]||_{s,\d M^-} = ||[\alpha_n^m]||_{s,\d M^-} = m||[\alpha_n]||_{s,\d M^-}.$
By Proposition \ref{prop:6.2} , $||[x]||_{s,\d M^-}\leq D\scl_{M^-}(x).$ Since $x$ is essential in $\Sigma_m$, we get that $\chi_-(\Sigma_m^-) \leq \chi_-(\Sigma_m)$, then using that $\chi_-(\Sigma_m)/2m - \delta \leq \scl_{M_n}(\alpha_n)$,
we obtain $\scl_{M^-}(x) \leq \chi_-(\Sigma_m^-)/2 \leq m\scl_{M_n}(\alpha_n) + \delta m$.
Putting this all together and dividing by $m$, we get that $$||[\alpha_n]||_{s,\d M^-}\leq D\scl_{M_n}(\alpha_n) + D\delta.$$

We therefore have in each case that there is a constant $D$ independent of $n$ such that $$||[\alpha_n]||_{s,\d M^-}\leq D\scl_{M_n}(\alpha_n) + D\delta.$$ By Proposition \ref{prop:6.3} , $||[\alpha_n]||_{s,\d M^-} = ||[f^n(\gamma)]||_{s,\d M^+}$ grows exponentially in $n$.
Thus for some constants $B>0$ and $r >0$, we have $$Be^{rn}\leq D\scl_{M_n}(\gamma) + D\delta,$$ where we use that $\gamma$ and $\alpha_n$ are homotopic in $M_n$. Using the injectivity radius lower bound and Theorem 3.9 of \cite{Calegari}, we can increase $D$ and drop the additive constant in this inequality. By Proposition \ref{prop:6.3} , the volume growth of the $W_n$ is proportional to $n$ so there is a constant $C$ such that $\vol(W_n)\leq Cn$.
Additionally, using the $K$-biLipschitz comparison of Proposition \ref{prop:6.3} , the length of $\gamma$ in $W_n$ is bounded from above by $2K|\gamma|_{W}$, where $W$ is the tripus. As a result, Theorem A implies that the spectral gap for the 1-form Laplacian of the manifolds $W_n$ vanishes exponentially fast in $n$.
In particular, we have \[\sqrt{\lambda(W_n)} \leq A\vol(W_n)\frac{|\gamma|_{W_n}}{\scl_{W_n}(\gamma)} \leq 2KACB^{-1}D|\gamma|_Wne^{-rn},\] so the result holds after redefining $B$ to be $2KACB^{-1}D|\gamma|_W.$

\end{proof}


%%%%%%%%%%%%%%%%%%%%%%%%%%%%%%%%%%%%%%%%%%%%%%%%%%%%%%%%%%%%%%%%

\section{ANALYSIS OF PERFORMATIVE RISK MINIMIZING GRADIENT FLOW}
\label{sec:analysis_prm}
% !TEX root = main.tex

In this section, we consider PRM gradient flow, defined by Equation~\eqref{eq:prm_flow}. We observe that gradient flows provide complete vector fields, and that trajectories will converge to local performative risk minimizers under very mild conditions.

First, we state a proposition guaranteeing that flow is well-defined. The compact sublevel sets ensure that trajectories of Equation~\eqref{eq:prm_flow} remain bounded, which is sufficient to guarantee existence and uniqueness of solutions globally. For proof of the following proposition, we refer the reader to either~{\citet[Section 3.1]{Khalil:2001wj}} or~{\citet[Section 9.3]{Hirsch:2012tx}}.

\begin{proposition}[Existence and uniqueness of gradient flows]

Suppose the performative risk $PR(\cdot)$ is continuously differentiable, and its sublevel sets $\{ x : PR(x) \le c \}$ are compact for every $c \in \mb{R}$. Then for any initial condition $\xnom(0) = x_0$, there exists a unique solution to the differential equation in Equation~\eqref{eq:prm_flow}, defined for all $t \ge 0$.

\end{proposition}

Next, we note that gradient flows have nice properties from the perspective of optimization. Namely: every isolated local minima is locally asymptotically stable, and we can provide sufficient conditions to characterize a subset of the region of convergence.

\begin{proposition}[Convergence of gradient flows]

Suppose the performative risk $PR(\cdot)$ is twice continuously differentiable, and $x^*$ is an isolated local performative risk minimizer. Then $x^*$ is a locally asymptotically stable equilibrium of Equation~\eqref{eq:prm_flow}. 
Furthermore, take any $c$ such that $PR(x^*) \le c$. Let $A \subseteq \{ x : PR(x) \le c \}$ denote the connected component of $\{ x : PR(x) \le c \}$ that contains $x^*$. If $x^*$ is the only local performative minimizer in $A$, then all solutions with initial conditions in $A$ converge to $x^*$.

\end{proposition}

\begin{proof}
Since $x^*$ is an isolated local minimizer and the performative risk is twice continuously differentiable, there exists a neighborhood $U \ni x^*$ such that $\nabla PR(\cdot)$ is non-zero for all $x \neq x^*$. By continuity, there exists some constant $\eps$ such that the connected component of $\{ x : PR(x) \le PR(x^*) + \eps \}$ containing $x^*$ is contained in $U$. Since it is a sublevel set of $PR(\cdot)$ and $\mc{L}_{\fnom}PR(x) < 0$ on its boundary, it is positively invariant. Furthermore, since $\mc{L}_{\fnom}(x) < 0$ for all $x \neq x^*$ on this set, $x^*$ is locally asymptotically stable by standard Lyapunov arguments (see, e.g.~\citet[Section 4]{Khalil:2001wj}).
\end{proof}

The sublevel sets of the performative risk are positively invariant with respect to the PRM gradient flow. Furthermore, because of the continuity of trajectories, each connected component will also be positively invariant. This, in tandem with the fact that trajectories must either converge to a local minima or go off to infinity, also implies the previous proposition.

With minimal assumptions, isolated local performative risk minimizers are all locally attractive in the PRM gradient flow. In Section~\ref{sec:analysis_RGD}, we will view the PRM gradient flow as the nominal dynamics. From this perspective, we analyze the RGD flow as a perturbation from these nominal dynamics. To be able to do any perturbation-based analysis, we will need some stronger conditions on the convergence of the gradient flow associated with performative risk minimization. We note these assumptions here.

\begin{assumption}[Sufficient curvature of the PR]
\label{ass:exist_V}

Fix some isolated local performative risk minimizer $x^*$. 
We assume there exists positive constants $c_1$, $c_2$, $c_3$, $c_4$ and $\delta$ such that the following holds in a neighborhood of $x^*$:
\begin{equation}
\label{eq:v_ineq1}
c_1 |x-x^*|^2 \le PR(x) - PR(x^*) \le c_2 |x-x^*|^2
\end{equation}
\begin{equation}
\label{eq:v_ineq2}
c_3 |x - x^*| - \delta \le | \nabla PR(x) | \le c_4 |x - x^*| + \delta
\end{equation}

We will let $r$ denote the radius of this neighborhood, so the above inequalities are valid on the set $\{ x : |x - x^*| \le r \}$.

\end{assumption}
Assumption~\ref{ass:exist_V} provides conditions on which $V(x) = PR(x) - PR(x^*)$ can be used as a Lyapunov function locally. Next, we provide conditions directly on the loss $\ell(\cdot)$ and the decision-dependent distribution shift $\mc{D}(\cdot)$ which can ensure that Assumption 1 holds. First, we provide sufficient conditions for the bounds in Equation~\eqref{eq:v_ineq1}.

\begin{proposition}[Performative risk bounds]
\label{prop:pr_bnds}
Let $x^*$ be a performative risk minimizer and fix any $x$. If:
\begin{enumerate}
    \item $\ell(\cdot,x)$ is $L_1$ Lipschitz continuous
    \item $\mc{W}_1(\mc{D}(x),\mc{D}(x^*)) \le L_2 |x-x^*|^2$
    \item $\ell(z,\cdot)$ is $m$-strongly convex and $L_3$-smooth for every $z$
\end{enumerate}
Then: $(m/2 - L_1 L_2) |x-x^*|^2 \le PR(x) - PR(x^*) \le (L_1 L_2 + L_3/2) |x-x^*|^2$.
\end{proposition}

\begin{proof}
First, we can break up the performative risk into two parts: $PR(x) - PR(x^*) = R(x,x) - R(x^*,x^*) = [R(x,x) - R(x,x^*)] + [R(x,x^*) - R(x^*,x^*)]$. 
Note that $R(x,x) - R(x,x^*) = \mb{E}_{Z \sim \mc{D}(x)} [\ell(Z,x)] - \mb{E}_{Z \sim \mc{D}(x^*)} [\ell(Z,x)]$. Conditions (1) and (2), along with Kantorovich-Rubenstein duality~\citep{Villani:2003th}, implies this quantity is bounded in absolute value: $|R(x,x) - R(x,x^*)| \le L_1 L_2 |x-x^*|^2$. 
On the other hand, $R(x,x^*) - R(x^*,x^*) = \mb{E}_{Z \sim \mc{D}(x^*)} [\ell(Z,x) - \ell(Z,x^*)]$. By convexity and $L_3$-smoothness, $\ell(z,x) - \ell(z,x^*) \le \langle \nabla_x \ell(z,x^*), x-x^* \rangle + \frac{L_3}{2} |x-x^*|^2$ for any $z$; taking the expectation and noting that $\nabla PR(x^*) = 0$, we have $R(x,x^*) - R(x^*,x^*) \le \frac{L_3}{2} |x-x^*|^2$. In the other direction, using strong convexity and similar arguments, we get: $R(x,x^*) - R(x^*,x^*) \ge \frac{m}{2} |x-x^*|^2$. Combining these results yields the desired results.
\end{proof}
Note that Condition (2) in Proposition~\ref{prop:pr_bnds} is a variation on the typical $\eps$-sensitivity definition. Recall that $\eps$-sensitivity states that for any $x$ and $y$, $\mc{W}_1(\mc{D}(x),\mc{D}(y)) \le \eps|x-y|$~\citep{Perdomo:2020tz}. In contrast, Condition (2) only requires this condition to hold around the point $x^*$, but requires a stricter bound for $x$ close to $x^*$. This bound is also more lax than $\eps$-sensitivity farther away from $x^*$.

Next, we provide sufficient conditions for a bound on the absolute value of the gradient of the performative risk. 
% We fixed this issue: 
%This does not exactly recover Assumption~\ref{ass:exist_V}, as there are additive constants on the upper and lower bounds which do not scale with $|x-x^*|$. However, these additive constants will be small for decision-dependent distribution shifts $\mc{D}(\cdot)$ with sufficiently small sensitivity parameter $\eps$.

\begin{proposition}[Gradient bounds of the performative risk]
\label{prop:grad_bounds}
Let $x^*$ be a performative risk minimizer and fix any $x$. If:
\begin{enumerate}
    \item $\ell(\cdot,x)$ and $\ell(\cdot, x^*)$ are both $L_1$ Lipschitz continuous
    \item $\ell(z,\cdot)$ is $m$-strongly convex and $L_3$-smooth for every $z$
    \item $\mc{D}(\cdot)$ is $\eps$-sensitive, i.e. $\mc{W}_1(\mc{D}(x),\mc{D}(y)) \le \eps |x-y|$
    \item $\nabla_x \ell(\cdot,x)$ is $L_4$ Lipschitz continuous
\end{enumerate}
Then: $(m- \eps L_4)|x-x^*| - 2 \eps L_1 \le |\nabla PR(x)| \le (L_3 + \eps L_4) |x-x^*| + 2 \eps L_1$.
\end{proposition}

\begin{proof}
Similar to the previous proposition, we break apart this gradient. Note that $\nabla PR(x^*) = 0$, so:
$|\nabla PR(x)| = |\nabla PR(x) - \nabla PR(x^*)| =
|\nabla_{x_1} R(x,x) - \nabla_{x_1} R(x^*,x^*) +
\nabla_{x_2} R(x,x) - \nabla_{x_2} R(x^*,x^*)|$. For the $\nabla_{x_1}$ terms, we have: $m|x-x^*| \le |\nabla_{x_1} R(x,x^*) - \nabla_{x_1}R(x^*,x^*)| \le L_3|x-x^*|$ by standard convexity arguments, and $|\nabla_{x_1} R(x,x) - \nabla_{x_1}R(x,x^*)| \le \eps L_4 |x-x^*|$ by the same Kantorovich-Rubenstein duality argument as the previous proposition. For the $\nabla_{x_2}$ terms, note that the mapping $x_2 \mapsto R(x,x_2)$ is $\eps L_1$ Lipschitz continuous. Thus, $|\nabla_{x_2} R(x,x)| \le \eps L_1$ and similarly $|\nabla_{x_2} R(x^*,x^*)|$. Combining these inequalities yields the desired result.
\end{proof}

Depending on the situation, we may be able to directly verify Assumption~\ref{ass:exist_V}, although, for more complex settings, this is likely to be very difficult. Propositions~\ref{prop:pr_bnds} and~\ref{prop:grad_bounds} provide a set of sufficient conditions for this assumption to hold, but checking the conditions on the decision-dependent distribution shift $\mc{D}(\cdot)$ may be difficult in practice as well. This is one limitation of this current work, and we believe it is an interesting future research direction to identify conditions which are easy to verify, even in settings with limited information about the distribution shift itself.


%%%%%%%%%%%%%%%%%%%%%%%%%%%%%%%%%%%%%%%%%%%%%%%%%%%%%%%%%%%%%%%%

\section{ANALYSIS OF REPEATED RISK MINIMIZING FLOW}
\label{sec:analysis_RGD}
% !TEX root = main.tex

In the previous section, we consider the PRM gradient flow and showed that the trajectories converge to local performative risk minimizers in very general settings. In this section, we will consider the RGD flow, defined by Equation~\eqref{eq:RGD_flow}. 
The RGD flow is not necessarily a gradient flow, and generally will not inherit the nice properties we saw in Section~\ref{sec:analysis_prm}.

The following theorem provides conditions on the transient response and steady-state behavior of the RGD flow. Prior to $T$, the trajectories converge exponentially quickly. After $T$, we have an ultimate bound that holds.

\begin{theorem}[Ultimate bounds for RGD flow]
\label{th:perturb1}

Fix any isolated performative risk minimizer $x^*$ and suppose the conditions of Assumption~\ref{ass:exist_V} hold. Let $(c_i)_{i=1}^4$ and $\delta$ denote the constants from Assumption~\ref{ass:exist_V} and $r > 0$ denote the radius where the inequalities are valid.

Suppose that there exists positive constants $\eps < c_3^2/c_4$ such that the following holds on $U = \{ x : |x - x^*| \le r \}$:
\begin{equation}
\label{eq:ass_V1}
|\nabla_{x_2}R(x,x)| \le \eps |x-x^*| + \delta
\end{equation}
Additionally, suppose the initial condition satisfies:
\[
|x_0 - x^*| \le \sqrt{\frac{c_1}{c_2}}r
\]
Take any $\theta \in (0,1)$ such that:
\[
\delta \le \sqrt{\frac{c_1}{c_2}} 
\frac{(1 - \theta) r (c_3^2/c_4 - \eps)}{c_4 + 2c_3 + \epsilon}
\]
Then, there exists a $T \ge 0$ such that:
\begin{itemize}
\item For all $t \le T$:
\[
\begin{aligned}
&|\varphi_{\fpert}(t;x_0) - x^*| \le \\
&\qquad\sqrt{\frac{c_2}{c_1}} \exp(-t\theta (c_3^2 - c_4\eps)/2c_2) |x_0 - x^*|
\end{aligned}
\]
\item For all $t \ge T$:
\[
\begin{aligned}
    &|\varphi_{\fpert}(t;x_0) - x^*| 
    \le \\
    &\qquad\sqrt{\frac{c_2}{c_1}} 
    \max
    \left\{
    \frac{\delta (c_4 + 2c_3 + \epsilon)}{ (1-\theta) (c_3^2 - c_4 \eps)},
    \frac{\delta}{c_3}
    \right\}.
\end{aligned}
\]
\end{itemize}

\end{theorem}

\begin{proof}
See Appendix~\ref{app:proof_perturb1}.
\end{proof}

Note that, in the special case where $\delta = \lambda = 0$, we have that the RGD flow converges exponentially quickly to $x^*$ locally. Similarly, in the special case where Assumption~\ref{ass:exist_V} holds everywhere (i.e. $r = \infty$), then there is only one minimizer $x^*$, and all initial conditions converge to a neighborhood of $x^*$ exponentially fast. In addition, if every condition in Assumption 1 holds with equality, the bounds in Theorem 1 also hold with equality. Please see Appendix A for an example where this occurs.

Additionally, note that locally performatively stable points are equilibria of the RGD flow. This result provides constraints on where performatively stable points can be located. Consider the special case where Assumption~\ref{ass:exist_V} holds globally (i.e. $r = \infty$) and, consequently, there exists only one minimizer $x^*$. In this special case, Theorem~\ref{th:perturb1} shows that all performatively stable points must be close to $x^*$. The phenomena that, under certain conditions, performatively stable points are near performative risk minimizers, was first noted in~\citet{Perdomo:2020tz}. Our results here provide another set of conditions under which the same result holds.

Furthermore, in the presence of Propisition 3 and 4, Theorem 1 can be restated in terms of $\ell(\cdot)$ and $\cD(\cdot)$ as follows:

\begin{corollary}
    Let $x^*$ be a performative risk minimizer and fix any $x$. If:
    \begin{enumerate}
    \item $\ell(\cdot,x)$ and $\ell(\cdot, x^*)$ are both $L_1$ Lipschitz continuous
    \item $\ell(z,\cdot)$ is $m$-strongly convex and $L_3$-smooth for every $z$
    \item $\mc{D}(\cdot)$ is $\eps$-sensitive, i.e. $\mc{W}_1(\mc{D}(x),\mc{D}(y)) \le \eps |x-y|$
    \item $\mc{W}_1(\mc{D}(x),\mc{D}(x^*)) \le L_2 |x-x^*|^2$
    \item $\nabla_x \ell(\cdot,x)$ is $L_4$ Lipschitz continuous
\end{enumerate}
Suppose the initial condition satisfies:
\[
|x_0 - x^*| \le \sqrt{\frac{m - 2L_1L_2}{L_3 + 2L_1L_2}}r
\]
Then, there exists a $T \ge 0$ such that:
\begin{itemize}
\item For all $t \le T$:
\[
|\varphi_{\fpert}(t;x_0) - x^*| \le 
\]
\[
\sqrt{\frac{L_3 + 2L_1L_2}{m - 2L_1L_2}} \exp\left(\frac{-t\theta (m-\epsilon L_4)^2}{L_3 + 2L_1L_2}\right) |x_0 - x^*|
\]
\item For all $t \ge T$:
\[
\begin{aligned}
    &|\varphi_{\fpert}(t;x_0) - x^*| \le 
\sqrt{\frac{L_3 + 2L_1L_2}{m - 2L_1L_2}} \\
&\qquad\cdot\max
\left\{
    \frac{4\epsilon L_1 (L_3 + 2m - \epsilon L_4)}{ (m-\epsilon L_4)^2},
    \frac{2 \epsilon L_1}{m - \epsilon L_4}
\right\}.
\end{aligned}
\]
\end{itemize}
\end{corollary}
\begin{proof}
This follows immediately by combining Theorem~\ref{th:perturb1} with Propositions~\ref{prop:pr_bnds} and~\ref{prop:grad_bounds}.
\end{proof}


%%%%%%%%%%%%%%%%%%%%%%%%%%%%%%%%%%%%%%%%%%%%%%%%%%%%%%%%%%%%%%%%

\subsection{Performative alignment}
% !TEX root = main.tex

From the previous analysis, we also identify conditions on the directions of the performative perturbations that are sufficient to show the convergence of Equation~\eqref{eq:RGD_flow}, the RGD flow, to performative risk minimizers.

\begin{theorem}[Performative alignment]
\label{th:perf_align}
Suppose $x^*$ is a isolated local performative risk minimizer and the following holds for all $x$ in a neighborhood of $x^*$:
\begin{equation}
\label{eq:perf_align}
|\nabla_{x_2}R(x,x)|^2 \le \langle -\nabla_{x_1}R(x,x), \nabla_{x_2}R(x,x) \rangle
\end{equation}
Then $x^*$ is a locally asymptotically stable equilibrium point of the RGD flow, given by Equation~\eqref{eq:RGD_flow}. 
Note that this does \textbf{not} require Assumption~\ref{ass:exist_V}.
\end{theorem}

\begin{proof}
Let $V(x) = PR(x) - PR(x^*)$. 
Since $x^*$ is a locally asymptotically equilibria of the PRM flow, we have: $V(x^*) = 0$, $V(x) > 0$ for $x \neq 0$, and $\mc{L}_{\fnom} V(x) < 0$ for $x \neq 0$. The performative alignment condition ensures that $\mc{L}_{\fnom + g} V(x) < 0$ as well, and the desired result follows.
\end{proof}

We refer to Equation~\eqref{eq:perf_align} as the \textbf{performative alignment} condition. This condition states that the performative perturbation never increases the performative risk, and the convergence of performative risk minimization is sufficient to guarantee convergence of repeated risk minimization. In other words, the perturbation is pointing in the correct direction to ensure that $PR(\cdot) - PR(x^*)$ can still act as a Lyapunov function.

Another perspective on performative alignment is to consider the performative risk as a bilinear form whose arguments are parameterized by $x$. In particular, consider the decoupled performative risk $R(\cdot,\cdot)$. Let $\ell_{x} := \ell(\cdot, x)$ and let $\mu_x$ denote the probability distribution associated with $\mc{D}(x)$. Then, we can write $R(x_1,x_2) = \langle \mu_{x_2}, \ell_{x_1} \rangle$. From this perspective, $R(\cdot,\cdot)$ is a bilinear form in $\ell_x$ and $\mu_x$. As such, the performative alignment condition becomes a condition on the way in which $\ell$ and $\mu$ are \textit{parameterized} by $x$.

In Appendix~\ref{sec:perf_align_ex}, we apply Theorem~\ref{th:perf_align} to the example outlined in Section~\ref{sec:simple_ex}. It provides insight into one of the ways to use Theorem~\ref{th:perf_align}: when we fix a loss $\ell(\cdot)$, we can view the performative alignment condition as specifying a class of decision-dependent distribution shifts which do not hamper the convergence of RGD to performative risk minimizers.


%%%%%%%%%%%%%%%%%%%%%%%%%%%%%%%%%%%%%%%%%%%%%%%%%%%%%%%%%%%%%%%%

% \section{Numerical examples}
% \label{sec:num_results}
% \section{Numerical implementation and solution}
In this section, we explain how to implement and solve the minimization problem \eqref{eq:dynamic_recon} numerically which, depending on the amount of time steps $T$, can be challenging. 
We derive a general primal-dual algorithm for its solution, before we line out some strategies to reduce computational costs and speed up the implementation at the end of this section. 

\subsection{Gradients and sampling operators}
\label{subsec:implementation of operators}

In order to use the discrete total variation already defined in Section \ref{subsubsec:TV}, we need a discrete gradient operator that maps an image $u \in \C^N$ to its gradient $\nabla u \in \C^{N \times 2}$. 
Following \cite{ChambollePock}, we implement the gradient by standard forward differences. Moreover, we  will use its discrete adjoint, the negative divergence $-\diverg$, defined by the identity $\langle \nabla u, w \rangle_{\C^{N \times 2}} = - \langle u, \diverg(w) \rangle_{\C^N}$.
The inner product on the gradient space $\C^{N\times 2}$ is defined in a straightforward way as 
\begin{align*}
	\langle v,w \rangle_{\C^{N \times 2}} = \Real(v_1^* w_1) + \Real(v_2^* w_2),
\end{align*}
%
for $v,w \in \C^{N \times 2}$.
For $v \in \C^{N \times 2}$, the (isotropic) 1-norm is defined by 
\begin{align*}
	\|v \|_1 := \sum_{i=1}^N \sqrt{|(v_i)_1|^2 + |(v_i)_2|^2},
\end{align*}
and accordingly the dual $\infty$-norm for $w \in \C^{N \times 2}$ is given by 
\begin{align*}
	\| w \|_\infty := \max_{i=1,\cdots,N} |w_i| = \max_{i=1,\cdots,N} ~ \sqrt{|(w_i)_1|^2 + |(w_i)_2|^2}.
\end{align*}

The sampling operators $\Kcal_t \colon \C^N \to \C^{M_t}$ (and analogously for $\Kcal_0 \colon \C^{N_0} \to \C^{M_0}$) we consider are either a standard fast Fourier transform (FFT) on a Cartesian grid, followed by a projection onto the sampled frequencies, or a non-uniform fast Fourier transform (NUFFT), in case the sampled frequencies are not located on a Cartesian grid \cite{Fessler:NUFFT}. \\

\noindent {\it Fourier transform on a Cartesian grid - the simulated data case}\\
\noindent
In the numerical study on artificial data we use a simple version of the Fourier transform and sampling operator (the same can also be found in e.g. \cite{Ehrhardt2016,Rasch2017}). 
We discretize the image domain on the unit square using an (equi-spaced) Cartesian grid with $N_1 \times N_2$ pixels such that the discrete grid points are given by 
\begin{align*}
 \Omega_N = \left\{ \left(\frac{n_1}{N_1-1}, \frac{n_2}{N_2-1} \right) ~\Big|~ n_1 = 0, \dots, N_1-1, \; n_2 = 0, \dots N_2-1 \right\}.
\end{align*}
We proceed analogously with the $k$-space, i.e. the location of the $(m_1,m_2)$-th Fourier coefficient is given by $(m_1/(N_1-1), m_2/(N_2-1))$. Then, we arrive at the following formula for the (standard) Fourier transform $\Fcal$ applied to $u \in \C^{N_1 \times N_2}$:
\begin{align*}
 (\Fcal u)_{m_1,m_2} = \frac{1}{N_1 N_2} \sum_{n_1=0}^{N_1-1} \sum_{n_2=0}^{N_2-1} u_{n_1,n_2} e^{-2\pi i \left(\frac{n_1 m_1}{N_1} + \frac{n_2 m_2}{N_2} \right)},
\end{align*}
where $ m_1 = 0, \dots, N_1-1, m_2 = 0, \dots, N_2 -1$.
For simplicity, we use a vectorized version such that $\Fcal \colon \C^N \to \C^N$ with $N = N_1 \cdot N_2$.
We then employ a simple sampling operator $\Scal_t \colon \C^N \to \C^{M_t}$ which discards all Fourier frequencies which are not located on the desired sampling geometry at time $t$ (i.e. the chosen spokes). 
More precisely, following \cite{Ehrhardt2016}, if we let $\Pcal_t \colon \{1,\dots,M_t \} \to \{1,\dots,N\}$ be an injective mapping which chooses $M_t$ Fourier coefficients from the $N$ coefficients available, we can define the sampling operator $\Scal$ applied to $f \in \C^N$ as
\begin{align*}
 \Scal_t \colon \C^N \to \C^{M_t}, \quad (\Scal_t f)_k = f_{\Pcal_t(k)}.
\end{align*}
The full forward operator $\Kcal_t$ can hence be expressed as 
\begin{align}\label{eq:forward_op_art}
 \Kcal_t \colon \C^N \xrightarrow{\Fcal} \C^N \xrightarrow{\Scal_t} \C^{M_t}.
\end{align}
The corresponding adjoint operator of $\Kcal_t$ is given by  
\begin{align*}
 \Kcal_t^* \colon \C^{M_t} \xrightarrow{\Scal_t^*} \C^N \xrightarrow{\Fcal^{-1}} \C^N,
\end{align*}
where $\Fcal^{-1}$ denotes the standard inverse Fourier transform and $\Scal_t^*$ `fills' the missing frequencies with zeros, i.e. 
\begin{align*}
 (\Scal_t^*z)_l = \sum_{k=1}^{M_t} z_k \delta_{l,\Pcal_t(k)}, \qquad \text{for } l = 1, \dots, N.
\end{align*}
For the prior $u_0$, we choose a full Cartesian sampling, which corresponds to $\Pcal_0$ being the identity. For the subsequent dynamic scan, we set up $\Pcal_t$ such that it chooses the frequencies located on (discrete) spokes through the center of the $k$-space.     
It is important to notice that this implies that the locations of the (discretized) spokes are still located on a Cartesian grid, which allows to employ a standard fast Fourier transform (FFT) followed by the above projection onto the desired frequencies. 
This is not the case for the operators we use for real data. \\

\noindent {\it Non-uniform Fourier transform - the real data case}\\ 
\noindent 
In contrast to the above (simplified) setup for artificial data, in many real world application the measured $k$-space frequencies $\xi_m$ in \eqref{eq:fourier_transform} are {\it not} located on a Cartesian grid. 
While this is not a problem with respect to the formula itself, it however excludes the possibility to employ a fast Fourier transform, 
%numerically
which usually reduces the computational costs of an $N$-point Fourier transform from an order of $O(N^2)$ to $O(N \log N)$.
To get to a similar order of convergence also for non-Cartesian samplings, it is necessary to employ the concept of non-uniform fast Fourier transforms (NUFFT) \cite{Fessler:NUFFT,Fessler:code,Matej2004,Nguyen:1999,Strohmer2000}. 
We only give a quick intuition here and for further information we refer the reader to the literature listed above. 
The main idea is to use a (weighted) and oversampled standard Cartesian $K$-point FFT $\Fcal$, $K \geq N$ followed by an interpolation $\Scal$ in $k$-space onto the desired frequencies $\xi_m$. 
Note that the oversampling takes place in $k$-space.
The operator $\Kcal_t$ for time $t$ can hence again be expressed as a concatenation of a $K$-point FFT and a sampling operator 
\textbf{\begin{align*}
 \Kcal_t \colon \C^N \xrightarrow{\Fcal} \C^N \xrightarrow{\Scal_t} \C^{M_t}.
\end{align*}}
For our numerical experiments with the experimental DCE-MRI data, the sampling operator $\Scal_t$ and its adjoint were taken from the NUFFT package \cite{Fessler:code}.

\subsection{Numerical solution}
%
Due to the nondifferentiablity and the involved operators we apply a primal-dual method \cite{ChambollePock} to solve the minimization problem \eqref{eq:dynamic_recon}. 
We first line out how to solve the (simple) TV-regularized problem for the prior (\ref{tvu0}) and then extend the approach to the dynamic problem. 
Interestingly, the problem for the prior already provides all the ingredients needed for the numerical solution of the dynamic problem, which can then be done in a very straightforward way. 
We consider the problem 
\begin{equation} \label{tvu0}
	\min_{u_0} ~ \frac{\alpha_0}{2} \| \Kcal_0 u_0 - f_0 \|_{\C^{M_0}}^2 + \| \nabla u_0 \|_1, 
\end{equation}
with $u_0 \in \C^{N_0}$.
Dualizing both terms leads to its primal-dual formulation 
\begin{align}\label{eq:tv_pd}
	\min_{u_0} \max_{y_1,y_2} ~ \langle y_1, \Kcal_0 u_0 - f_0 \rangle_{C^{M_0}} - \frac{1}{2 \alpha_0} \|y_1 \|_{\C^{M_0}}^2 + \langle y_2, \nabla u_0 \rangle_{\C^{N_0 \times 2}} + \chi_{C}(y_2),
\end{align}
where $y_1 \in \C^{M_0}$ and $\chi_{C}$ denotes the characteristic function of the set  
\begin{align*}
	C := \{ y \in \C^{N_0 \times 2} ~|~ \|y \|_\infty \leq 1 \}.
\end{align*}
% 
The primal-dual algorithm in \cite{ChambollePock} now essentially consists in performing a proximal gradient descent on the primal variable $u_0$ and a proximal gradient ascent on the dual variables $y_1$ and $y_2$, where the gradients are taken with respect to the linear part, the proximum with respect to the nonlinear part. 
We hence need to compute the proximal operators for the nonlinear parts in \eqref{eq:tv_pd} to obtain the update steps for $u_0$ and $y_1,y_2$. 
It is easy to see that the proximal operator for $\phi(y_1) = \frac{1}{2 \alpha} \|y_1\|_{\C^{M_0}}^2$ is given by 
\begin{align}\label{eq:prox_dual_l2}
	y_1 = \prox_{\sigma \phi} (r) \Leftrightarrow y_1 = \frac{\alpha r}{\alpha + \sigma}.
\end{align}
The proximal operators for the update of $y_2$ are given by a simple projection onto the set $C$, i.e. 
\begin{align}\label{eq:prox_proj}
	y_2 = \proj_C (r) \Leftrightarrow (y_2)_i = r_i / \max (|r_i|,1) \quad \text{for all } i.
\end{align}
Putting everything together leads to Algorithm \ref{alg:prior}.
\begin{algorithm}[t!] 
\caption{\textbf{Reconstruction of the prior}}
{
\begin{algorithmic}[1]
\Require step sizes $\tau,\sigma > 0$, data $f_0$, parameter $\alpha_0$
\Ensure $u_0^0 = \bar{u}_0^0 = \Kcal_0^*f_0, ~ y_1^0 = y_2^0 = 0$
	\While{$\sim$ stop crit}
    	\State {\it Dual updates}
        \State $y_1^{k+1} = (\alpha_0 \left[ y_1^k + \sigma (\Kcal_0 \bar{u}_0^k - f_0)\right]) / (\alpha_0 + \sigma)$
          \State $y_2^{k+1} = \proj_C \left(y_2^k + \sigma \nabla \bar{u}_0^k\right)$
          \State {\it Primal updates}
          \State $u_0^{k+1} =  u_0^k - \tau \left[ \Kcal_0^* y_1^{k+1} - \diverg(y_2^{k+1}) \right]$
          \State {\it Overrelaxation}
          \State $\bar{u}_0^{k+1}= 2 u_0^{k+1} - u_0^k$
	\EndWhile\\
\Return $u_0 = u_0^k$
\end{algorithmic}
}
\label{alg:prior}
\end{algorithm}
\ \\

The numerical realization of the dynamic problem is now straightforward.
In order to deal with the infimal convolution, we use its definition and introduce an additional auxiliary variable yielding 
\begin{alignat*}{4}
	&\min_{\ubold}&& ~ &&\sum_{t=1}^T \frac{\alpha_t}{2} \| \Kcal_t u_t - f_t \|_{\C^{M_t}}^2 + \sum_{t=1}^{T-1} \frac{\gamma_t}{2} \|u_{t+1} - u_t \|_{\C_N}^2 + \sum_{t=1}^T w_t \TV(u_t) \\
	& && + && \sum_{t=1}^T (1-w_t) \ICBTV^{p_0}(u_t,u_0) \\    
   = &\min_{\ubold,\zbold}&& ~ &&\sum_{t=1}^T \frac{\alpha_t}{2} \| \Kcal_t u_t - f_t \|_{\C^{M_t}}^2 + \sum_{t=1}^{T-1} \frac{\gamma_t}{2} \|u_{t+1} - u_t \|_{\C_N}^2 +\sum_{t=1}^T w_t \| \nabla u_t \|_1\\
   & && + &&\sum_{t=1}^T (1-w_t) \left[ \| \nabla (u_t - z_t) \|_1 + \| \nabla z_t \|_1  - \langle p_0,u_t \rangle_{\C^N} + \langle 2 p_0, z_t \rangle_{\C^N} \right]
\end{alignat*}
where $\ubold = [u_1, \dots, u_T] \in \C^{N \times T}$ and $\zbold = [z_1, \dots, z_T] \in \C^{N \times T}$.
Introducing a dual variable $\ybold$ for all the terms containing an operator, leads to the primal-dual formulation 
\begin{alignat}{4}
\label{eq:primal_dual}
	&\min_{\ubold,\zbold} \max_{\ybold} && ~ && \sum_{t=1}^T \left(\langle y_{t,1}, \Kcal_t u_t - f_t \rangle_{\C^{M_t}} - \frac{1}{2 \alpha_t} \|y_{t,1} \|_{\C^M}^2 \right) + \sum_{t=1}^{T-1} \frac{\gamma_t}{2} \| u_{t+1} - u_t \|_{\C^N}^2 \notag \\
    & && + && \sum_{t=1}^T \left(\langle y_{t,2}, \nabla u_t \rangle_{\C^{N \times 2}} + \langle y_{t,3}, \nabla (u_t - z_t) _{\C^{N \times 2}} + \langle y_{t,4}, \nabla z_t \rangle_{\C^{N \times 2}}\right) \notag \\
    & && - && \sum_{t=1}^T \langle (1-w_t)p_0,u_t \rangle_{\C^N} + \sum_{t=1}^T \langle 2(1-w_t)p_0,z_t \rangle_{\C^N}  \notag \\
    & && + && \sum_{t=1}^T \left(\chi_{C_{t,2}}(y_{t,2}) + \chi_{C_{t,3}}(y_{t,3}) + \chi_{C_{t,4}}(y_{t,4})\right)
\end{alignat}
where $\ybold = [\ybold_1, \dots, \ybold_T]$, $\ybold_t = [y_{t,1}, \dots, y_{t,4}]$, and for all $t = 1, \dots, T$, $u_{t} \in \C^{M_t}$ and
\begin{align*}
	&C_{t,2} := \{ y \in \C^{M \times 2} ~|~ \|y \|_{\infty} \leq w_t \}, \\
    &C_{t,3} := \{ y \in \C^{M \times 2} ~|~ \|y \|_{\infty} \leq (1-w_t) \}, \\
    &C_{t,4} := \{ y \in \C^{M \times 2} ~|~ \|y \|_{\infty} \leq (1-w_t) \}. \\
\end{align*}
%
\begin{algorithm}[t!]
\caption{\textbf{Dynamic reconstruction with structural prior}}
{
\begin{algorithmic}[1]
\Require step sizes $\tau,\sigma > 0$, subgradient $p_0$, for all $t=1,\dots,T$: data $f_t$, parameters $\alpha_t$, $w_t$, $\gamma_t$
\Ensure for all $t=1,\dots,T$: $u_t^0 = \bar{u}_t^0 = \Kcal_t^*f_t, ~ z_t^0 = \bar{z}_t^0 = 0, ~ y_{t,1}^0 = y_{t,2}^0 = y_{t,3}^0 = y_{t,4}^0 = 0$
	\While{$\sim$ stop crit}
    	\For{t=1,\dots,T} 
          \State {\it Dual updates}
          \State $y_{t,1}^{k+1} = \frac{\alpha_t \left[ y_{t,1} + \sigma (\Kcal_t \bar{u}_t^k - f_t)\right]}{\alpha_t + \sigma}$
          \State $y_{t,2}^{k+1} = \proj_{C_2}\left(y_{t,2}^k + \sigma \nabla \bar{u}_t^k\right)$
          \State $y_{t,3}^{k+1} = \proj_{C_3}\left(y_{t,3}^k + \sigma \nabla (\bar{u}_t^k - \bar{z}_t^k) \right)$
          \State $y_{t,4}^{k+1} = \proj_{C_4}\left(y_{t,4}^k + \sigma \nabla \bar{z}_t^k \right)$
          \State {\it Primal updates}
          \State $u_t^{k+1} =  \frac{u_t^k - \tau \left[ \Kcal_t^* y_{t,1}^{k+1} - \diverg(y_{t,2}^{k+1}) - \diverg(y_{t,3}^{k+1}) - (1-w_t)p_0 \right] + \tau \gamma_t u_{t+1}^k + \tau \gamma_{t-1} u_{t-1}^k}{\tau (\gamma_t + \gamma_{t+1}) +1}$
          \State $z_t^{k+1} - \tau \left[ 2(1-w_t) p_0 + \diverg(y_{t,3}^{k+1}) - \diverg(y_{t,4}^{k+1}) \right]$
          \State {\it Overrelaxation}
          \State $(\bar{u}_t^{k+1}, \bar{z}_t^{k+1}) = 2 (u_t^{k+1}, z_t^{k+1}) - (u_t^k,z_t^k)$
    	\EndFor
	\EndWhile\\
\Return for all $t = 1,\dots,T$: $u_t = u_t^k$
\end{algorithmic}
}
\label{alg:fmri}
\end{algorithm}
%
To solve the problem, we again perform a proximal gradient descent on the primal variables $\ubold$ and $\zbold$, and a proximal gradient ascent on the dual variables $\ybold$, where the gradients are taken with respect to the linear parts, the proximum with respect to the nonlinear parts. 
We hence need to compute the proximal operators for the nonlinear parts in \eqref{eq:primal_dual} to obtain the update steps for $u_t,z_t$ and $\ybold_t$ for every $t = 1, \dots, T$. 
The proximal operators for $\phi_t(y_{t,1}) = \frac{1}{2 \alpha_t} \| y_{t,1} \|_{\C^{M_t}}^2$ can be computed exactly as in \eqref{eq:prox_dual_l2}.
The proximal operators for the updates of $y_{t,j}$, $j = 2,3,4$, are given by projections onto the sets $C_{t,j}$ similar to \eqref{eq:prox_proj}.
For the squared norm related to the time regularization, we notice that for every $1< t < T$, $u_t$ only interacts with the previous and the following time step, i.e. $u_{t-1}$ and  $u_{t+1}$. 
Hence, analogously to $\phi_t$, the proximum for 
\begin{align*}
	\psi_t(u_t) = \frac{\gamma_{t-1}}{2} \|u_t - u_{t-1}\|_{\C^N}^2 + \frac{\gamma_t}{2} \|u_{t+1} - u_t\|_{\C^N}^2
\end{align*}
is given by 
\begin{align*}
	u_t = \prox_{\tau \psi_t} (r) \Leftrightarrow u_t = \frac{r + \tau \gamma_t u_{t+1} + \tau \gamma_{t-1} u_{t-1}}{\tau (\gamma_t + \gamma_{t-1}) + 1}. 
\end{align*}
The two odd updates for $t = 1$ and $t = T$ can be obtained by the same formula by simply setting $\gamma_0 = 0$ and $\gamma_T = 0$, respectively.
Putting everything together, we obtain Algorithm \ref{alg:fmri}.\\

\subsection{Step sizes and stopping criteria}
We quickly discuss the choice of the step sizes $\tau, \sigma$ and stopping criteria for Algorithm \ref{alg:fmri}. 
In most standard applications it stands to reason to choose the step sizes according to the condition $\tau \sigma \| L \|^2 < 1$ ($L$ denotes the collection of all operators) such that convergence of the algorithm is guaranteed \cite{ChambollePock}.
However, depending on $T$, i.e. the number of time frames we consider, the norm of the operator $L$ 
can be very costly to compute, or too large such that the condition $\tau \sigma \| L \|^2 < 1$ only permits extremely small step sizes. 
For practical use, we instead simply choose $\tau$ and $\sigma$ reasonably ''small`` and track both the energy of the problem and the primal-dual residual \cite{Goldstein:Adaptive} to monitor convergence. 
For the sake of brevity, we do not write down the primal-dual residual for Algorithm \ref{alg:fmri} and instead refer the reader to \cite{Goldstein:Adaptive} for its definition. 
The implementation is then straightforward.
We hence stop the algorithm if both, the relative change in energy between consecutive iterates and the primal-dual residual, have dropped below a certain threshold.

\subsection{Practical considerations}
It is clear that for a large number of time frames $T$ Algorithm \ref{alg:fmri} starts to require an increasing amount of time to return reliable results and for reasonably ''large`` step sizes $\tau$ and $\sigma$ it is even doubtful whether we can obtain convergence. 
In practice, it is hence necessary to divide the time series $\Tbold = \{1, \dots, T\}$ into $l$ smaller bits of consecutive time frames. More precisely, choose numbers $1 \leq T_1 < \dots < T_l = T$ such that $\Tbold = \Tbold_1 \cup \Tbold_2 \cup \dots \cup \Tbold_l$ with $\Tbold = \{1, \dots, T_1, T_1+1, \dots, T_2, \dots, T_{l-1}+1, \dots, T_l\}$.
We can then perform the reconstruction separately for all $\Tbold_i$. 
In order to keep the ''continuity`` between $\Tbold_i$ and $\Tbold_{i+1}$, we can include the last frame of $\Tbold_i$ into the reconstruction of $\Tbold_{i+1}$ by letting $\gamma_{T_i} \neq 0$ and choosing $u_{T_i}$ as the respective last frame of $\Tbold_i$.
This divides the overall problem into smaller and easier subproblems, which can be solved faster.
In practice, we observed that a size of five to ten frames per subset $\Tbold_i$ is a reasonable choice, which essentially gives very similar results as doing a reconstruction for the entire time series $\Tbold$.








%%%%%%%%%%%%%%%%%%%%%%%%%%%%%%%%%%%%%%%%%%%%%%%%%%%%%%%%%%%%%%%%

\section{CLOSING REMARKS}
\label{sec:conclusion}

In this paper, we analyzed the problem of performative prediction in settings where multiple isolated equilibria may be of interest. We analyzed the gradient flow of performative risk minimization, and identified regions of attraction for various equilibria. We viewed repeated gradient descent flow as a perturbation of the PRM gradient flow. In particular, we used a Lyapunov function for the PRM gradient flow to analyze the trajectories of the RGD flow. We found conditions on which RGD flow will converge to the local PRM minimizers, and conditions on which they will converge to a neighborhood of PRM minimizers. 

These results provide a method to analyze the regions of attraction for various equilibria under repeated risk minimization. In real-world settings with decision-dependent distributions, we expect many situations where the initialization may have a significant outcome on the trajectories and final outcomes. 

%%%%%%%%%%%%%%%%%%%%%%%%%%%%%%%%%%%%%%%%%%%%%%%%%%%%%%%%%%%%%%%%

\subsubsection*{Acknowledgements}
Lillian J. Ratliff is supported by NSF CAREER Award No.1844729.

\bibliography{main}

\appendix
\onecolumn

\section{PROOF OF THEOREM~\ref{th:perturb1}}
\label{app:proof_perturb1}
% !TEX root = main.tex

Let $V(x) = R(x,x) - R(x^*,x^*)$. Note that $V(x) \ge 0$ on $U = \{ x : |x - x^*| \le r \}$ and $V(x) = 0$ if and only if $x = x^*$. Furthermore, note that $\frac{\partial V}{\partial x}(x) = [\nabla_{x_1}R(x,x) + \nabla_{x_2}R(x,x)]^\T$.

Consider the function $t \mapsto V(\varphi_{\fpert}(t;x_0))$ and its time derivative. Also, let $\xpert(t) = \varphi_{\fpert}(t;x_0)$. 
When $|\xpert-x^*| \ge \delta/c_3$, taking the derivative along trajectories of the repeated risk minimization flow and using Equations~\eqref{eq:v_ineq2} and~\eqref{eq:ass_V1}:
\[
\begin{aligned}
    \mc{L}_{\fnom + g}V 
    &=
    \frac{\partial V}{\partial x} (\fnom(x) + g) 
    =
    -|\nabla_{x_1}R + \nabla_{x_2}R|^2 + \langle \nabla_{x_1}R + \nabla_{x_2}R, \nabla_{x_2}R \rangle \\
    &\le
    -\left( c_3 |\xpert-x^*|-\delta \right)^2
    +
    \left( c_4 |\xpert-x^*|+\delta \right) |\nabla_{x_2}R| \\
    &\le
    -\left( c_3 |\xpert-x^*|-\delta \right)^2
    +
    \left( c_4 |\xpert-x^*|+\delta \right) (\epsilon|\xpert-x^*|+\delta)\\
    &=
    -\left(c_4\epsilon-c_3^2\right)|\xpert-x^*|^2
    +\left(c_4\delta + 2c_3\delta+\delta\epsilon\right)|\xpert-x^*|
\end{aligned}
\]
These inequalities are valid so long as $\xpert(t)$ stays within $U$, which we will ensure later in the proof. Note that $\eps$ is sufficiently small (by assumption) to ensure that $-c_3^2 + c_4 \eps < 0$.

Let $\alpha := c_3^2 - c_4 \eps > 0$. Take any $\theta \in (0,1)$ and note that:
\[
\mc{L}_{\fnom + g}V(\xpert) \le - \theta \alpha | \xpert - x^* |^2 - (1 - \theta) \alpha | \xpert - x^* |^2 +\left(c_4\delta + 2c_3\delta+\delta\epsilon\right)|\xpert-x^*|
\]
Now, let 
\[
    \mu(\theta)
    :=
    \max
    \left\{
    \frac
    {
        \left(c_4\delta + 2c_3\delta+\delta\epsilon\right)
    }
    {\alpha(1 - \theta)},
    \frac{\delta}{c_3}
    \right\}.
\]
If $|\xpert - x^*| \ge \mu(\theta)$, then:
\[
- \theta \alpha | \xpert - x^* |^2 - (1 - \theta) \alpha | \xpert - x^* |^2 +\left(c_4\delta + 2c_3\delta+\delta\epsilon\right)|\xpert-x^*| \le 0,
\]
and thus
\[
\mc{L}_{\fnom + g}V(\xpert) \le - \theta \alpha | \xpert - x^* |^2.
\]
Trajectories of Equation~\eqref{eq:RGD_flow} has two stages: a transient due to its initial condition, and then an ultimate bound due to the perturbation. Let $T(\theta) = \inf~\{ t \ge 0 : |\xpert(t) - x^*| \le \mu(\theta) \}$. Prior to $T(\theta)$, we have:
\[
\frac{d}{dt} V(\xpert(t)) \le - \theta \alpha |\xpert(t) - x^*|^2 \le - \frac{\theta \alpha}{c_2} V(\xpert(t))
\]
The latter follows from Equation~\eqref{eq:v_ineq1}. 
By the comparison principle (see, e.g.~\citep[Lemma 3.4]{Khalil:2001wj}), we have $V(\xpert(t)) \le \exp(-t\theta \alpha / c_2) V(x_0)$. Again using Equation~\eqref{eq:v_ineq1}, this yields the following inequality, valid for all $t \le T(\theta)$:
\[
|\xpert(t) - x^*| \le \sqrt{\frac{c_2}{c_1}} \exp(-t\theta \alpha/2c_2) |x_0 - x^*|
\]
Note that this inequality also provides an upper bound on $T(\theta)$. Additionally, note that this implies the bound $|\xpert(t) - x^*| \le r$, by our assumption on the initial condition. Prior to $T(\theta)$, our trajectory stays in $U$, where our inequalities are valid.

At time $T(\theta)$, we have $|\xpert(t) - x^*| \le \mu(\theta)$. Note that this inequality implies $V(\xpert(t)) \le c_2 \mu^2(\theta)$. Since $\mc{L}_{\fnom + g}V < 0$ on the boundary of $\Omega(\theta) := \{ x : V(x) \le c_2 \mu^2(\theta) \}$, we have that $\Omega(\theta)$ is a positively invariant set. So, for all $t \ge T(\theta)$, we have $\xpert(t) \in \Omega(\theta)$. Using Equation~\eqref{eq:v_ineq1}, we have the following for all $t \ge T(\theta)$:
\[
|\xpert(t) - x^*| \le 
\sqrt{\frac{c_2}{c_1}} \mu(\theta) 
\]
The condition on $\theta$ ensures that this quantity is bounded by $r$, and the trajectory stays in $U$ for $t \ge T(\theta)$. 
This proves our desired result.

Additionally, we can show that this bound is tight by considering the following example. 
Suppose $\cD(x_2)$ is the point mass distribution (i.e. $p(z) = \delta(z - x_2)$) and $l(z, x_1) = 1/2 |z|^2 + 1/2|x_1|^2$. Then the performative risk is given by $R(x_1, x_2) = 1/2 |x_1|^2 + 1/2|x_2|^2$. It follows that $x^* 0$ is the performative risk minimizer. Following the arguments in Appendix A, one would find that the dynamics of $V(x) = R(x, x) - R(x^*, x^*)$ follows $\frac{d}{dt} V(x(t)) = -2 |x(t) - x^*|^2 = -2 V(x(t))$, which yields 
    $|x(t) - x^*| = \exp(-2t)|x_0 - x^*|$.


\section{NUMERICAL EXAMPLES}
\label{sec:num_results}
\section{Numerical implementation and solution}
In this section, we explain how to implement and solve the minimization problem \eqref{eq:dynamic_recon} numerically which, depending on the amount of time steps $T$, can be challenging. 
We derive a general primal-dual algorithm for its solution, before we line out some strategies to reduce computational costs and speed up the implementation at the end of this section. 

\subsection{Gradients and sampling operators}
\label{subsec:implementation of operators}

In order to use the discrete total variation already defined in Section \ref{subsubsec:TV}, we need a discrete gradient operator that maps an image $u \in \C^N$ to its gradient $\nabla u \in \C^{N \times 2}$. 
Following \cite{ChambollePock}, we implement the gradient by standard forward differences. Moreover, we  will use its discrete adjoint, the negative divergence $-\diverg$, defined by the identity $\langle \nabla u, w \rangle_{\C^{N \times 2}} = - \langle u, \diverg(w) \rangle_{\C^N}$.
The inner product on the gradient space $\C^{N\times 2}$ is defined in a straightforward way as 
\begin{align*}
	\langle v,w \rangle_{\C^{N \times 2}} = \Real(v_1^* w_1) + \Real(v_2^* w_2),
\end{align*}
%
for $v,w \in \C^{N \times 2}$.
For $v \in \C^{N \times 2}$, the (isotropic) 1-norm is defined by 
\begin{align*}
	\|v \|_1 := \sum_{i=1}^N \sqrt{|(v_i)_1|^2 + |(v_i)_2|^2},
\end{align*}
and accordingly the dual $\infty$-norm for $w \in \C^{N \times 2}$ is given by 
\begin{align*}
	\| w \|_\infty := \max_{i=1,\cdots,N} |w_i| = \max_{i=1,\cdots,N} ~ \sqrt{|(w_i)_1|^2 + |(w_i)_2|^2}.
\end{align*}

The sampling operators $\Kcal_t \colon \C^N \to \C^{M_t}$ (and analogously for $\Kcal_0 \colon \C^{N_0} \to \C^{M_0}$) we consider are either a standard fast Fourier transform (FFT) on a Cartesian grid, followed by a projection onto the sampled frequencies, or a non-uniform fast Fourier transform (NUFFT), in case the sampled frequencies are not located on a Cartesian grid \cite{Fessler:NUFFT}. \\

\noindent {\it Fourier transform on a Cartesian grid - the simulated data case}\\
\noindent
In the numerical study on artificial data we use a simple version of the Fourier transform and sampling operator (the same can also be found in e.g. \cite{Ehrhardt2016,Rasch2017}). 
We discretize the image domain on the unit square using an (equi-spaced) Cartesian grid with $N_1 \times N_2$ pixels such that the discrete grid points are given by 
\begin{align*}
 \Omega_N = \left\{ \left(\frac{n_1}{N_1-1}, \frac{n_2}{N_2-1} \right) ~\Big|~ n_1 = 0, \dots, N_1-1, \; n_2 = 0, \dots N_2-1 \right\}.
\end{align*}
We proceed analogously with the $k$-space, i.e. the location of the $(m_1,m_2)$-th Fourier coefficient is given by $(m_1/(N_1-1), m_2/(N_2-1))$. Then, we arrive at the following formula for the (standard) Fourier transform $\Fcal$ applied to $u \in \C^{N_1 \times N_2}$:
\begin{align*}
 (\Fcal u)_{m_1,m_2} = \frac{1}{N_1 N_2} \sum_{n_1=0}^{N_1-1} \sum_{n_2=0}^{N_2-1} u_{n_1,n_2} e^{-2\pi i \left(\frac{n_1 m_1}{N_1} + \frac{n_2 m_2}{N_2} \right)},
\end{align*}
where $ m_1 = 0, \dots, N_1-1, m_2 = 0, \dots, N_2 -1$.
For simplicity, we use a vectorized version such that $\Fcal \colon \C^N \to \C^N$ with $N = N_1 \cdot N_2$.
We then employ a simple sampling operator $\Scal_t \colon \C^N \to \C^{M_t}$ which discards all Fourier frequencies which are not located on the desired sampling geometry at time $t$ (i.e. the chosen spokes). 
More precisely, following \cite{Ehrhardt2016}, if we let $\Pcal_t \colon \{1,\dots,M_t \} \to \{1,\dots,N\}$ be an injective mapping which chooses $M_t$ Fourier coefficients from the $N$ coefficients available, we can define the sampling operator $\Scal$ applied to $f \in \C^N$ as
\begin{align*}
 \Scal_t \colon \C^N \to \C^{M_t}, \quad (\Scal_t f)_k = f_{\Pcal_t(k)}.
\end{align*}
The full forward operator $\Kcal_t$ can hence be expressed as 
\begin{align}\label{eq:forward_op_art}
 \Kcal_t \colon \C^N \xrightarrow{\Fcal} \C^N \xrightarrow{\Scal_t} \C^{M_t}.
\end{align}
The corresponding adjoint operator of $\Kcal_t$ is given by  
\begin{align*}
 \Kcal_t^* \colon \C^{M_t} \xrightarrow{\Scal_t^*} \C^N \xrightarrow{\Fcal^{-1}} \C^N,
\end{align*}
where $\Fcal^{-1}$ denotes the standard inverse Fourier transform and $\Scal_t^*$ `fills' the missing frequencies with zeros, i.e. 
\begin{align*}
 (\Scal_t^*z)_l = \sum_{k=1}^{M_t} z_k \delta_{l,\Pcal_t(k)}, \qquad \text{for } l = 1, \dots, N.
\end{align*}
For the prior $u_0$, we choose a full Cartesian sampling, which corresponds to $\Pcal_0$ being the identity. For the subsequent dynamic scan, we set up $\Pcal_t$ such that it chooses the frequencies located on (discrete) spokes through the center of the $k$-space.     
It is important to notice that this implies that the locations of the (discretized) spokes are still located on a Cartesian grid, which allows to employ a standard fast Fourier transform (FFT) followed by the above projection onto the desired frequencies. 
This is not the case for the operators we use for real data. \\

\noindent {\it Non-uniform Fourier transform - the real data case}\\ 
\noindent 
In contrast to the above (simplified) setup for artificial data, in many real world application the measured $k$-space frequencies $\xi_m$ in \eqref{eq:fourier_transform} are {\it not} located on a Cartesian grid. 
While this is not a problem with respect to the formula itself, it however excludes the possibility to employ a fast Fourier transform, 
%numerically
which usually reduces the computational costs of an $N$-point Fourier transform from an order of $O(N^2)$ to $O(N \log N)$.
To get to a similar order of convergence also for non-Cartesian samplings, it is necessary to employ the concept of non-uniform fast Fourier transforms (NUFFT) \cite{Fessler:NUFFT,Fessler:code,Matej2004,Nguyen:1999,Strohmer2000}. 
We only give a quick intuition here and for further information we refer the reader to the literature listed above. 
The main idea is to use a (weighted) and oversampled standard Cartesian $K$-point FFT $\Fcal$, $K \geq N$ followed by an interpolation $\Scal$ in $k$-space onto the desired frequencies $\xi_m$. 
Note that the oversampling takes place in $k$-space.
The operator $\Kcal_t$ for time $t$ can hence again be expressed as a concatenation of a $K$-point FFT and a sampling operator 
\textbf{\begin{align*}
 \Kcal_t \colon \C^N \xrightarrow{\Fcal} \C^N \xrightarrow{\Scal_t} \C^{M_t}.
\end{align*}}
For our numerical experiments with the experimental DCE-MRI data, the sampling operator $\Scal_t$ and its adjoint were taken from the NUFFT package \cite{Fessler:code}.

\subsection{Numerical solution}
%
Due to the nondifferentiablity and the involved operators we apply a primal-dual method \cite{ChambollePock} to solve the minimization problem \eqref{eq:dynamic_recon}. 
We first line out how to solve the (simple) TV-regularized problem for the prior (\ref{tvu0}) and then extend the approach to the dynamic problem. 
Interestingly, the problem for the prior already provides all the ingredients needed for the numerical solution of the dynamic problem, which can then be done in a very straightforward way. 
We consider the problem 
\begin{equation} \label{tvu0}
	\min_{u_0} ~ \frac{\alpha_0}{2} \| \Kcal_0 u_0 - f_0 \|_{\C^{M_0}}^2 + \| \nabla u_0 \|_1, 
\end{equation}
with $u_0 \in \C^{N_0}$.
Dualizing both terms leads to its primal-dual formulation 
\begin{align}\label{eq:tv_pd}
	\min_{u_0} \max_{y_1,y_2} ~ \langle y_1, \Kcal_0 u_0 - f_0 \rangle_{C^{M_0}} - \frac{1}{2 \alpha_0} \|y_1 \|_{\C^{M_0}}^2 + \langle y_2, \nabla u_0 \rangle_{\C^{N_0 \times 2}} + \chi_{C}(y_2),
\end{align}
where $y_1 \in \C^{M_0}$ and $\chi_{C}$ denotes the characteristic function of the set  
\begin{align*}
	C := \{ y \in \C^{N_0 \times 2} ~|~ \|y \|_\infty \leq 1 \}.
\end{align*}
% 
The primal-dual algorithm in \cite{ChambollePock} now essentially consists in performing a proximal gradient descent on the primal variable $u_0$ and a proximal gradient ascent on the dual variables $y_1$ and $y_2$, where the gradients are taken with respect to the linear part, the proximum with respect to the nonlinear part. 
We hence need to compute the proximal operators for the nonlinear parts in \eqref{eq:tv_pd} to obtain the update steps for $u_0$ and $y_1,y_2$. 
It is easy to see that the proximal operator for $\phi(y_1) = \frac{1}{2 \alpha} \|y_1\|_{\C^{M_0}}^2$ is given by 
\begin{align}\label{eq:prox_dual_l2}
	y_1 = \prox_{\sigma \phi} (r) \Leftrightarrow y_1 = \frac{\alpha r}{\alpha + \sigma}.
\end{align}
The proximal operators for the update of $y_2$ are given by a simple projection onto the set $C$, i.e. 
\begin{align}\label{eq:prox_proj}
	y_2 = \proj_C (r) \Leftrightarrow (y_2)_i = r_i / \max (|r_i|,1) \quad \text{for all } i.
\end{align}
Putting everything together leads to Algorithm \ref{alg:prior}.
\begin{algorithm}[t!] 
\caption{\textbf{Reconstruction of the prior}}
{
\begin{algorithmic}[1]
\Require step sizes $\tau,\sigma > 0$, data $f_0$, parameter $\alpha_0$
\Ensure $u_0^0 = \bar{u}_0^0 = \Kcal_0^*f_0, ~ y_1^0 = y_2^0 = 0$
	\While{$\sim$ stop crit}
    	\State {\it Dual updates}
        \State $y_1^{k+1} = (\alpha_0 \left[ y_1^k + \sigma (\Kcal_0 \bar{u}_0^k - f_0)\right]) / (\alpha_0 + \sigma)$
          \State $y_2^{k+1} = \proj_C \left(y_2^k + \sigma \nabla \bar{u}_0^k\right)$
          \State {\it Primal updates}
          \State $u_0^{k+1} =  u_0^k - \tau \left[ \Kcal_0^* y_1^{k+1} - \diverg(y_2^{k+1}) \right]$
          \State {\it Overrelaxation}
          \State $\bar{u}_0^{k+1}= 2 u_0^{k+1} - u_0^k$
	\EndWhile\\
\Return $u_0 = u_0^k$
\end{algorithmic}
}
\label{alg:prior}
\end{algorithm}
\ \\

The numerical realization of the dynamic problem is now straightforward.
In order to deal with the infimal convolution, we use its definition and introduce an additional auxiliary variable yielding 
\begin{alignat*}{4}
	&\min_{\ubold}&& ~ &&\sum_{t=1}^T \frac{\alpha_t}{2} \| \Kcal_t u_t - f_t \|_{\C^{M_t}}^2 + \sum_{t=1}^{T-1} \frac{\gamma_t}{2} \|u_{t+1} - u_t \|_{\C_N}^2 + \sum_{t=1}^T w_t \TV(u_t) \\
	& && + && \sum_{t=1}^T (1-w_t) \ICBTV^{p_0}(u_t,u_0) \\    
   = &\min_{\ubold,\zbold}&& ~ &&\sum_{t=1}^T \frac{\alpha_t}{2} \| \Kcal_t u_t - f_t \|_{\C^{M_t}}^2 + \sum_{t=1}^{T-1} \frac{\gamma_t}{2} \|u_{t+1} - u_t \|_{\C_N}^2 +\sum_{t=1}^T w_t \| \nabla u_t \|_1\\
   & && + &&\sum_{t=1}^T (1-w_t) \left[ \| \nabla (u_t - z_t) \|_1 + \| \nabla z_t \|_1  - \langle p_0,u_t \rangle_{\C^N} + \langle 2 p_0, z_t \rangle_{\C^N} \right]
\end{alignat*}
where $\ubold = [u_1, \dots, u_T] \in \C^{N \times T}$ and $\zbold = [z_1, \dots, z_T] \in \C^{N \times T}$.
Introducing a dual variable $\ybold$ for all the terms containing an operator, leads to the primal-dual formulation 
\begin{alignat}{4}
\label{eq:primal_dual}
	&\min_{\ubold,\zbold} \max_{\ybold} && ~ && \sum_{t=1}^T \left(\langle y_{t,1}, \Kcal_t u_t - f_t \rangle_{\C^{M_t}} - \frac{1}{2 \alpha_t} \|y_{t,1} \|_{\C^M}^2 \right) + \sum_{t=1}^{T-1} \frac{\gamma_t}{2} \| u_{t+1} - u_t \|_{\C^N}^2 \notag \\
    & && + && \sum_{t=1}^T \left(\langle y_{t,2}, \nabla u_t \rangle_{\C^{N \times 2}} + \langle y_{t,3}, \nabla (u_t - z_t) _{\C^{N \times 2}} + \langle y_{t,4}, \nabla z_t \rangle_{\C^{N \times 2}}\right) \notag \\
    & && - && \sum_{t=1}^T \langle (1-w_t)p_0,u_t \rangle_{\C^N} + \sum_{t=1}^T \langle 2(1-w_t)p_0,z_t \rangle_{\C^N}  \notag \\
    & && + && \sum_{t=1}^T \left(\chi_{C_{t,2}}(y_{t,2}) + \chi_{C_{t,3}}(y_{t,3}) + \chi_{C_{t,4}}(y_{t,4})\right)
\end{alignat}
where $\ybold = [\ybold_1, \dots, \ybold_T]$, $\ybold_t = [y_{t,1}, \dots, y_{t,4}]$, and for all $t = 1, \dots, T$, $u_{t} \in \C^{M_t}$ and
\begin{align*}
	&C_{t,2} := \{ y \in \C^{M \times 2} ~|~ \|y \|_{\infty} \leq w_t \}, \\
    &C_{t,3} := \{ y \in \C^{M \times 2} ~|~ \|y \|_{\infty} \leq (1-w_t) \}, \\
    &C_{t,4} := \{ y \in \C^{M \times 2} ~|~ \|y \|_{\infty} \leq (1-w_t) \}. \\
\end{align*}
%
\begin{algorithm}[t!]
\caption{\textbf{Dynamic reconstruction with structural prior}}
{
\begin{algorithmic}[1]
\Require step sizes $\tau,\sigma > 0$, subgradient $p_0$, for all $t=1,\dots,T$: data $f_t$, parameters $\alpha_t$, $w_t$, $\gamma_t$
\Ensure for all $t=1,\dots,T$: $u_t^0 = \bar{u}_t^0 = \Kcal_t^*f_t, ~ z_t^0 = \bar{z}_t^0 = 0, ~ y_{t,1}^0 = y_{t,2}^0 = y_{t,3}^0 = y_{t,4}^0 = 0$
	\While{$\sim$ stop crit}
    	\For{t=1,\dots,T} 
          \State {\it Dual updates}
          \State $y_{t,1}^{k+1} = \frac{\alpha_t \left[ y_{t,1} + \sigma (\Kcal_t \bar{u}_t^k - f_t)\right]}{\alpha_t + \sigma}$
          \State $y_{t,2}^{k+1} = \proj_{C_2}\left(y_{t,2}^k + \sigma \nabla \bar{u}_t^k\right)$
          \State $y_{t,3}^{k+1} = \proj_{C_3}\left(y_{t,3}^k + \sigma \nabla (\bar{u}_t^k - \bar{z}_t^k) \right)$
          \State $y_{t,4}^{k+1} = \proj_{C_4}\left(y_{t,4}^k + \sigma \nabla \bar{z}_t^k \right)$
          \State {\it Primal updates}
          \State $u_t^{k+1} =  \frac{u_t^k - \tau \left[ \Kcal_t^* y_{t,1}^{k+1} - \diverg(y_{t,2}^{k+1}) - \diverg(y_{t,3}^{k+1}) - (1-w_t)p_0 \right] + \tau \gamma_t u_{t+1}^k + \tau \gamma_{t-1} u_{t-1}^k}{\tau (\gamma_t + \gamma_{t+1}) +1}$
          \State $z_t^{k+1} - \tau \left[ 2(1-w_t) p_0 + \diverg(y_{t,3}^{k+1}) - \diverg(y_{t,4}^{k+1}) \right]$
          \State {\it Overrelaxation}
          \State $(\bar{u}_t^{k+1}, \bar{z}_t^{k+1}) = 2 (u_t^{k+1}, z_t^{k+1}) - (u_t^k,z_t^k)$
    	\EndFor
	\EndWhile\\
\Return for all $t = 1,\dots,T$: $u_t = u_t^k$
\end{algorithmic}
}
\label{alg:fmri}
\end{algorithm}
%
To solve the problem, we again perform a proximal gradient descent on the primal variables $\ubold$ and $\zbold$, and a proximal gradient ascent on the dual variables $\ybold$, where the gradients are taken with respect to the linear parts, the proximum with respect to the nonlinear parts. 
We hence need to compute the proximal operators for the nonlinear parts in \eqref{eq:primal_dual} to obtain the update steps for $u_t,z_t$ and $\ybold_t$ for every $t = 1, \dots, T$. 
The proximal operators for $\phi_t(y_{t,1}) = \frac{1}{2 \alpha_t} \| y_{t,1} \|_{\C^{M_t}}^2$ can be computed exactly as in \eqref{eq:prox_dual_l2}.
The proximal operators for the updates of $y_{t,j}$, $j = 2,3,4$, are given by projections onto the sets $C_{t,j}$ similar to \eqref{eq:prox_proj}.
For the squared norm related to the time regularization, we notice that for every $1< t < T$, $u_t$ only interacts with the previous and the following time step, i.e. $u_{t-1}$ and  $u_{t+1}$. 
Hence, analogously to $\phi_t$, the proximum for 
\begin{align*}
	\psi_t(u_t) = \frac{\gamma_{t-1}}{2} \|u_t - u_{t-1}\|_{\C^N}^2 + \frac{\gamma_t}{2} \|u_{t+1} - u_t\|_{\C^N}^2
\end{align*}
is given by 
\begin{align*}
	u_t = \prox_{\tau \psi_t} (r) \Leftrightarrow u_t = \frac{r + \tau \gamma_t u_{t+1} + \tau \gamma_{t-1} u_{t-1}}{\tau (\gamma_t + \gamma_{t-1}) + 1}. 
\end{align*}
The two odd updates for $t = 1$ and $t = T$ can be obtained by the same formula by simply setting $\gamma_0 = 0$ and $\gamma_T = 0$, respectively.
Putting everything together, we obtain Algorithm \ref{alg:fmri}.\\

\subsection{Step sizes and stopping criteria}
We quickly discuss the choice of the step sizes $\tau, \sigma$ and stopping criteria for Algorithm \ref{alg:fmri}. 
In most standard applications it stands to reason to choose the step sizes according to the condition $\tau \sigma \| L \|^2 < 1$ ($L$ denotes the collection of all operators) such that convergence of the algorithm is guaranteed \cite{ChambollePock}.
However, depending on $T$, i.e. the number of time frames we consider, the norm of the operator $L$ 
can be very costly to compute, or too large such that the condition $\tau \sigma \| L \|^2 < 1$ only permits extremely small step sizes. 
For practical use, we instead simply choose $\tau$ and $\sigma$ reasonably ''small`` and track both the energy of the problem and the primal-dual residual \cite{Goldstein:Adaptive} to monitor convergence. 
For the sake of brevity, we do not write down the primal-dual residual for Algorithm \ref{alg:fmri} and instead refer the reader to \cite{Goldstein:Adaptive} for its definition. 
The implementation is then straightforward.
We hence stop the algorithm if both, the relative change in energy between consecutive iterates and the primal-dual residual, have dropped below a certain threshold.

\subsection{Practical considerations}
It is clear that for a large number of time frames $T$ Algorithm \ref{alg:fmri} starts to require an increasing amount of time to return reliable results and for reasonably ''large`` step sizes $\tau$ and $\sigma$ it is even doubtful whether we can obtain convergence. 
In practice, it is hence necessary to divide the time series $\Tbold = \{1, \dots, T\}$ into $l$ smaller bits of consecutive time frames. More precisely, choose numbers $1 \leq T_1 < \dots < T_l = T$ such that $\Tbold = \Tbold_1 \cup \Tbold_2 \cup \dots \cup \Tbold_l$ with $\Tbold = \{1, \dots, T_1, T_1+1, \dots, T_2, \dots, T_{l-1}+1, \dots, T_l\}$.
We can then perform the reconstruction separately for all $\Tbold_i$. 
In order to keep the ''continuity`` between $\Tbold_i$ and $\Tbold_{i+1}$, we can include the last frame of $\Tbold_i$ into the reconstruction of $\Tbold_{i+1}$ by letting $\gamma_{T_i} \neq 0$ and choosing $u_{T_i}$ as the respective last frame of $\Tbold_i$.
This divides the overall problem into smaller and easier subproblems, which can be solved faster.
In practice, we observed that a size of five to ten frames per subset $\Tbold_i$ is a reasonable choice, which essentially gives very similar results as doing a reconstruction for the entire time series $\Tbold$.








%%%%%%%%%%%%%%%%%%%%%%%%%%%%%%%%%%%%%%%%%%%%%%%%%%%%%%%%%%%%%%%%

\end{document}
