% !TEX root = main.tex

In the previous section, we consider the PRM gradient flow and showed that the trajectories converge to local performative risk minimizers in very general settings. In this section, we will consider the RGD flow, defined by Equation~\eqref{eq:RGD_flow}. 
The RGD flow is not necessarily a gradient flow, and generally will not inherit the nice properties we saw in Section~\ref{sec:analysis_prm}.

The following theorem provides conditions on the transient response and steady-state behavior of the RGD flow. Prior to $T$, the trajectories converge exponentially quickly. After $T$, we have an ultimate bound that holds.

\begin{theorem}[Ultimate bounds for RGD flow]
\label{th:perturb1}

Fix any isolated performative risk minimizer $x^*$ and suppose the conditions of Assumption~\ref{ass:exist_V} hold. Let $(c_i)_{i=1}^4$ and $\delta$ denote the constants from Assumption~\ref{ass:exist_V} and $r > 0$ denote the radius where the inequalities are valid.

Suppose that there exists positive constants $\eps < c_3^2/c_4$ such that the following holds on $U = \{ x : |x - x^*| \le r \}$:
\begin{equation}
\label{eq:ass_V1}
|\nabla_{x_2}R(x,x)| \le \eps |x-x^*| + \delta
\end{equation}
Additionally, suppose the initial condition satisfies:
\[
|x_0 - x^*| \le \sqrt{\frac{c_1}{c_2}}r
\]
Take any $\theta \in (0,1)$ such that:
\[
\delta \le \sqrt{\frac{c_1}{c_2}} 
\frac{(1 - \theta) r (c_3^2/c_4 - \eps)}{c_4 + 2c_3 + \epsilon}
\]
Then, there exists a $T \ge 0$ such that:
\begin{itemize}
\item For all $t \le T$:
\[
\begin{aligned}
&|\varphi_{\fpert}(t;x_0) - x^*| \le \\
&\qquad\sqrt{\frac{c_2}{c_1}} \exp(-t\theta (c_3^2 - c_4\eps)/2c_2) |x_0 - x^*|
\end{aligned}
\]
\item For all $t \ge T$:
\[
\begin{aligned}
    &|\varphi_{\fpert}(t;x_0) - x^*| 
    \le \\
    &\qquad\sqrt{\frac{c_2}{c_1}} 
    \max
    \left\{
    \frac{\delta (c_4 + 2c_3 + \epsilon)}{ (1-\theta) (c_3^2 - c_4 \eps)},
    \frac{\delta}{c_3}
    \right\}.
\end{aligned}
\]
\end{itemize}

\end{theorem}

\begin{proof}
See Appendix~\ref{app:proof_perturb1}.
\end{proof}

Note that, in the special case where $\delta = \lambda = 0$, we have that the RGD flow converges exponentially quickly to $x^*$ locally. Similarly, in the special case where Assumption~\ref{ass:exist_V} holds everywhere (i.e. $r = \infty$), then there is only one minimizer $x^*$, and all initial conditions converge to a neighborhood of $x^*$ exponentially fast. In addition, if every condition in Assumption 1 holds with equality, the bounds in Theorem 1 also hold with equality. Please see Appendix A for an example where this occurs.

Additionally, note that locally performatively stable points are equilibria of the RGD flow. This result provides constraints on where performatively stable points can be located. Consider the special case where Assumption~\ref{ass:exist_V} holds globally (i.e. $r = \infty$) and, consequently, there exists only one minimizer $x^*$. In this special case, Theorem~\ref{th:perturb1} shows that all performatively stable points must be close to $x^*$. The phenomena that, under certain conditions, performatively stable points are near performative risk minimizers, was first noted in~\citet{Perdomo:2020tz}. Our results here provide another set of conditions under which the same result holds.

Furthermore, in the presence of Propisition 3 and 4, Theorem 1 can be restated in terms of $\ell(\cdot)$ and $\cD(\cdot)$ as follows:

\begin{corollary}
    Let $x^*$ be a performative risk minimizer and fix any $x$. If:
    \begin{enumerate}
    \item $\ell(\cdot,x)$ and $\ell(\cdot, x^*)$ are both $L_1$ Lipschitz continuous
    \item $\ell(z,\cdot)$ is $m$-strongly convex and $L_3$-smooth for every $z$
    \item $\mc{D}(\cdot)$ is $\eps$-sensitive, i.e. $\mc{W}_1(\mc{D}(x),\mc{D}(y)) \le \eps |x-y|$
    \item $\mc{W}_1(\mc{D}(x),\mc{D}(x^*)) \le L_2 |x-x^*|^2$
    \item $\nabla_x \ell(\cdot,x)$ is $L_4$ Lipschitz continuous
\end{enumerate}
Suppose the initial condition satisfies:
\[
|x_0 - x^*| \le \sqrt{\frac{m - 2L_1L_2}{L_3 + 2L_1L_2}}r
\]
Then, there exists a $T \ge 0$ such that:
\begin{itemize}
\item For all $t \le T$:
\[
|\varphi_{\fpert}(t;x_0) - x^*| \le 
\]
\[
\sqrt{\frac{L_3 + 2L_1L_2}{m - 2L_1L_2}} \exp\left(\frac{-t\theta (m-\epsilon L_4)^2}{L_3 + 2L_1L_2}\right) |x_0 - x^*|
\]
\item For all $t \ge T$:
\[
\begin{aligned}
    &|\varphi_{\fpert}(t;x_0) - x^*| \le 
\sqrt{\frac{L_3 + 2L_1L_2}{m - 2L_1L_2}} \\
&\qquad\cdot\max
\left\{
    \frac{4\epsilon L_1 (L_3 + 2m - \epsilon L_4)}{ (m-\epsilon L_4)^2},
    \frac{2 \epsilon L_1}{m - \epsilon L_4}
\right\}.
\end{aligned}
\]
\end{itemize}
\end{corollary}
\begin{proof}
This follows immediately by combining Theorem~\ref{th:perturb1} with Propositions~\ref{prop:pr_bnds} and~\ref{prop:grad_bounds}.
\end{proof}
