\subsection{Micromagnetic modeling}
  Berkov \cite{Berkov:98a,Berkov:98b,Berkov:98c}
developed a numerical method to evaluate   the distribution of energy barriers between
 metastable states in many-particle systems which determines the optimal path
between the two given metastable states by minimizing
the action in the Onsager�Machlup functional \cite{Onsager:53}
for the transition probability.
This method essentially performs a local saddle-point search in an high-dimensional
energy landscape.
Mathematically similar problems
exist in several disciplines of
physics and chemistry.
In the last years, several new methods to locate saddle-points
 have been developed in these fields \cite{Henkelman:00a,Henkelman:00b,Olsen:04}.
Based on such algorithms, an improved elastic band technique
for micromagnetics was presented by \cite{Dittrich:02}.

The main problem in energy-barrier computation is that micromagnetic structures $m$ are
described by many variables and accordingly
energy $E$ is a function of $m$.
Minimizing $E(m)$ requires sophisticated algorithms, but for
energy-barrier calculations it is even necessary to
determine saddle points in this high-dimensional energy landscape.

Several approaches are available for this task, but because saddle-point calculation
is equivalent to minimizing $(\nabla E(m))^2$ all rapidly converging
methods require second derivatives of $E$.
This however is rather to be avoided if the calculation should be performed
effectively.

The present study develops a combination of several of the above cited techniques to efficiently
calculate energy barriers in micromagnetic models.
\subsection{Action minimization}

Berkov \cite{Berkov:98a} introduced action minimization as a tool
for finding optimal transition paths in thermally driven
micromagnetic systems.
He discretized the time dependent action of the
the magnetic particle system and used a numerical quadrature
representation for direct numerical minimization.
This rigorous approach is complicated by its explicit dependence upon
transition time.
However, transition paths turn out to be geodesics of the energy surface
in the limit of infinite transition time, where
 energy barriers are lowest.

Dittrich~{\em et al.} \cite{Dittrich:02}
make use of this fact by directly searching for geodesic paths using a
modification of the nudged elastic band (NEB) algorithm
of Henkelman~{\em et al.}\cite{Henkelman:00a,Henkelman:00b}.

A problem of this algorithm is that it involves the numerical solution of a large
system of ordinary differential equations. Moreover, there is a tendency of
the NEB algorithm to produce spurious up-down-up movements
along the gradient (kinks) which cannot be completely removed in all cases.

Here, we combine both approaches by designing a
 path relaxation algorithm
similar to NEB, but constraint to decrease the action at each step.
The algorithm performs a fast gradient-like relaxation from an initial path
towards the optimal transition path. It
  detects and avoids the development of kinks, and does not involve
numerical solutions of differential equations.

The important problem of finding an
initial path which is likely to lie in the basin of attraction
of the optimal transition path under the proposed relaxation scheme
is also investigated.


\subsection{Geometric action}
Here we define the {\em geometric action} for a path $p$ as the minimal action for
any transition along this path.


In a general mechanic system the action of a transition process $x(t)$ from state
$x(0)=x_0$ into $x(t_{end})=x_1$ is defined
by
\begin{equation}\label{action-1}
    S(x(t)) ~:=~ \int \limits_0^{t_{end}} \langle \dot{x} + \nabla E, \dot{x} + \nabla E, \rangle\,dt.
\end{equation}
The probability that this transition process occurs depends monotonously on
$\exp(-S(x(t)))$.
In the next section we will be looking for the optimal transition path
in the energy landscape determined by $E$.
The quality of any given path $p$ is defined as its geometric action:
the action of the transition process $x_p(t)$
along   $p$ which minimizes (\ref{action-1}).

We start with a canonical parametrisation of $p$ by  arc length $s$
and try  to find  a reparametrisation $s(t)$ which minimizes $S$ along $p$.
For this optimal transition process we then have
\begin{equation}\label{action-2}
    S_{\rm min}(p)~=~S(x_p(t)) ~=~ \int \limits_0^{L(p)}
    \langle \frac{dx}{ds}\, v + \nabla E, \frac{dx}{ds}\, v  + \nabla E, \rangle\,\frac{ds}{v},
\end{equation}
where $L(p)$ is the arc length and $v(s)~=~ \frac{ds}{dt}(s)$ is the local velocity of the
optimal transition at arc length $s$.
Finding $s(t)$  thus is reduced to the  variational problem of finding
the function $v(s)$ which minimizes (\ref{action-2}).
The corresponding Euler-Lagrange equation is
\begin{equation}\label{Euler-1}
   \frac{d}{dv} \left( \frac{1}{v}\,
   \langle \frac{dx}{ds}\, v + \nabla E, \frac{dx}{ds}\, v  + \nabla E, \rangle \right) ~=~0.
\end{equation}
A short calculation confirms that it's solution is
\begin{equation}\label{Euler-2}
  v~=~ \|\dot{x}\|~=~ \|\nabla E\| \,\left\|\frac{dx}{ds}\right\|^{-1} ~=~\|\nabla E\|.
\end{equation}
The last equality uses the fact that for the arc length parametrisation
$\| {dx}/{ds}\|~=~1$. Inserting this result into (\ref{action-2}) yields
\begin{equation}\label{action-3}
    S_{\rm min}(p)  ~=~ 2\,\int \limits_0^{L(p)}
    \|\nabla E\|  \, \left\|\frac{dx}{ds}\right\| + \left\langle\frac{dx}{ds},\nabla E  \right\rangle~ds
    ~=~ 2\,\int \limits_0^{L(p)}
    \|\nabla E\|    + \left\langle\frac{dx}{ds},\nabla E  \right\rangle~ds.
\end{equation}
This integral can be simplified further by noting that
\begin{equation}\label{action-4}
 \int \limits_0^{L(p)} \left\langle\frac{dx}{ds},\nabla E  \right\rangle~ds ~=~
 \int \limits_{E(x_0)}^{E(x_1)}  dE  ~=~ E(x_1) -E(x_0)~=:~\Delta E.
\end{equation}
Accordingly, one obtains the geometric action of $p$
as
\begin{equation}\label{action-5}
    S_{\rm min}(p)  ~=~  2\,\Delta E~+~2\,\int \limits_0^{L(p)}
    \|\nabla E\| \,ds.
\end{equation}

\subsection{Finding the optimal path by variation of the geometric action}
It is possible to find the Euler-Lagrange equations for the
optimal path by variation of the geometric action
$S_{\rm min}(p)$ with respect to $x$.

To this end we reparametrize (\ref{action-5}) by $w(s)=s/L(p)$ and obtain
\begin{equation}\label{action-6}
     S_{\rm min}(p)  ~=~  2\,\Delta E~+~ 2\,\int \limits_0^{1}
    \|\nabla E\| \, \left\|\frac{dx}{dw}\right\|  \,dw.
\end{equation}
The    Euler-Lagrange equation
of the variational problem $ \delta S_{\rm min}(p) ~=~0$
after some simplification
has the form
\begin{equation}\label{EL-2}
\frac{d^2x}{ds^2}~=~  \nabla\, \log\|\nabla E\|.
\end{equation}
The details of the calculation are given in the appendix.

\subsection{The optimal path is a geodesic}
A path along an energy surface which fulfills
 \begin{equation}\label{TP-1}
\dot{x}~=~  \pm  \nabla E
  \end{equation}
is a geodesic.
In the one-dimensional case (\ref{Euler-2}) directly  implies that
the optimal transition path is a geodesic.
In the multidimensional case this
not simply  follows from (\ref{Euler-2})
which is valid for any
geometric transition path.
%
Yet, by applying the Cauchy inequality to
(\ref{action-6}) one obtains  for the optimal  path
$p$
\begin{equation}\label{TP-3}
     S_{\rm min}(p)  ~\geq~  2\,\Delta E~+~ 2\,\int \limits_0^{1}
    \left|\left\langle \nabla E ,\, \frac{dx}{dw}\right\rangle\right|  \,dw.
\end{equation}
The integration interval $[0,1]$ can be divided into
finitely many parts $[w_k,w_{k+1}]$
with alternating constant sign of $ \langle \nabla E ,\, dx/dw \rangle$.
Accordingly, $\nabla E(x(w_{k}))~=~0$ and
\begin{equation}\label{TP-4}
     S_{\rm min}(p)  ~\geq~  2\,\Delta E~+~ 2\,\sum \limits_{k=0}^{K}
    \left| \, E(x(w_{k+1}))- E(x(w_{k})) \right|.
\end{equation}
Here the right hand side is a lower limit of $ S_{\rm min}$.
Accordingly, a geodesic  which fulfills (\ref{TP-1}) achieves equality
in (\ref{TP-3}). It therefore coincides with the least action path between
the prescribed endpoints.
Equality in (\ref{TP-4}) means that the least action depends only upon the energies at
the traversed  critical points.

\subsection{Morse theory}
Topologically different critical points in multi-dimensions are distinguished
by their {\em Morse index}, which is defined as the dimension $n_-$
of the sub-manifold on which the Hessian is negative definite.
Intuitively, the Morse index of the highest saddle point along the optimal transition path
should not be too large. This is  because , apart from singular cases,
  the action is minimized along an only one-dimensional manifold,
   i.e. an isolated path parallel to the gradient which connects initial and final minima.
Therefore, if  at the  highest saddle point the Morse index is  $n_- > 1$,
the other $n_- - 1$ descending directions should lead into different LEM states.
The choice for these LEM states should not be too large whenever the initial and final minima
are close to the global  energy minimum.

On the other hand, Morse theory \cite{Milnor:63M} implies that the total number of saddle-points
will be huge in realistic micromagnetic calculations.
%
For a two-dimensional  surface  with Euler characteristic
$\chi_{\rm Euler}$,
the numbers
$N_{\rm min}$ of minima,
 $N_{\rm max}$ of maxima, and
 $N_{\rm saddle}$ of  saddle points
 are connected by the relation
\begin{equation}\label{crit}
 N_{\rm min} + N_{\rm max} -N_{\rm saddle} ~=~ \chi_{\rm Euler}.
 \end{equation}
For a sphere is $\chi_{\rm Euler}=2$.
 %
The generalization of (\ref{crit}) to a finite-dimensional
compact manifold is the {\em Morse relation}
 \begin{equation}\label{Euler-N}
 \sum \limits_{k=0}^{N} (-1)^k N_{k} ~=~ \chi_{\rm Euler}.
 \end{equation}
Here $N_{k}$  is the number of critical points
with Morse index  $k$, i.e.
where the negative definite sub-manifold has dimension $n_- = k$.
In case of the micromagnetic energy depending on $N$ magnetization directions
  the manifold is a direct product of $N$ two-dimensional
spheres, therefore   $ \chi_{\rm Euler} = 2^N$.
%
On this $2 N$-dimensional manifold we get from (\ref{Euler-N})
 \begin{equation}\label{crit-N}
 N_{\rm min} + N_{\rm max}+ N_{{\rm even}} -N_{ {\rm odd}} ~=~ 2^N,
 \end{equation}
where $N_{ {\rm even}}$ and $N_{{\rm odd}}$ are the number of
true saddle points
with  even or odd Morse index, respectively.

If this manifold describes  an ensemble of
 interacting SD grains
 each grain has at least two critical points,
 minima or maxima,  in zero external field.
 If   interaction is weak, the total energy of the system   inherits almost
 all these minima and maxima as saddle-points.
  Thus, it is understandable that the total number of critical points exceeds even a huge figure
  like $2^N$.

  The situation is different if the $2 N$-dimensional manifold
  describes an exchange coupled grain with inhomogeneous magnetization structure,
  like in   most micromagnetic applications.
  In this case, it turns out that only a limited number of   minima and maxima,
  like flower  or  vortex states, exist.
  %
     According to (\ref{crit-N}), there must appear an enormous number
     of true saddle points  with even Morse index -- i.e. not one-dimensional lines.

     Thus, inevitably there exists a large number of paths,
     with different action $S$,
     which  connect the local minima and complicate the  search of the lowest action path.
     %
     Another circumstance causing difficulties to find path lines by numerical means,
     is that the field lines passing through the saddle points,
     form sub-manifolds of zero measure.
     In other words, the saddle points are unstable in the sense,
     that almost all field lines in their vicinity are deflected aside
     (except for those which go directly through the saddle point).

%To briefly review Morse theory, we consider a differentiable
%function F : X ---> R on a real manifold X. Points where the
%differential of F vanishes are called critical points of F. A
%critical point is called non-degenerate if the Hessian of F is
%non-degenerate at this point. F is called a Morse function if all
%its critical points are non-degenerate. In applications to
%variational problems, X is the space of trial maps, F is the
%functional to be varied, and the critical points of F are the
%solutions to the variational problem. The non-degeneracy condition
%guarantees that the character of each critical point - local
%minimum, local maximum, or saddle - is determined by the Hessian of
%F at this point. The index of the Hessian is called the Morse index
%of the critical point. It is defined as the maximal dimension of a
%subspace on which the Hessian is negative definite. At a local
%minimum the Morse index is zero, at a local maximum it is equal to
%the dimension of X.

%Morse theory was first worked out by Morse [229Jump To The Next
%Citation Point] for the case that X is finite-dimensional and
%compact (see Milnor [224Jump To The Next Citation Point] for a
%detailed exposition). The main result is the following. On a compact
%manifold X, for every Morse function the Morse inequalities Nk &gt;
%Bk, k = 0,1,2,..., (59) and the Morse relation oo oo sum k sum k (-
%1) Nk = (- 1) Bk (60) k=0 k=0 hold true. Here Nk denotes the number
%of critical points with Morse index k and Bk denotes the kth Betti
%number of X. Formally, B k is defined for each topological space X
%in terms of the kth singular homology space Hk(X ) with coefficients
%in a field F (see, e.g., [78], p. 32). (The results of Morse theory
%hold for any choice of F.) Geometrically, B0 counts the connected
%components of X and, for k > 1, Bk counts the �holes� in X that
%prevent a k-cycle with coefficients in F from being a boundary. In
%particular, if X is contractible to a point, then B = 0 k for k > 1.
%The right-hand side of Equation (60View Equation) is, by definition,
%the Euler characteristic of X. By compactness of X, all Nk and Bk
%are finite and in both sums of Equation (60View Equation) only
%finitely many summands are different from zero.

\subsection{Euler-Lagrange path relaxation}
To numerically calculate the optimal geometric transition path
one would like to start from an arbitrarily chosen initial
path $p_0$ which then is iteratively improved
by some updating scheme.
An heuristic procedure is known as 'nudged elastic band' technique.
Here we derive an optimal updating scheme which corresponds to
a gradient minimization of minimal path action.
To this end we follow the standard derivation of the Euler-Lagrange
equation to prove that the change $\delta S_{\rm min}$ in action
due to a variation $\delta x$ in the path is given by
\begin{equation}\label{EL-4}
\delta S_{\rm min}~=~
2\,\int \limits_0^L \left( \nabla \|\nabla E\| - \|\nabla E\| \, \frac{d^2x}{ds^2}
\right)
\,\delta x\,ds.
\end{equation}
The direction of maximal increase in $S_{\rm min}$ now is a function
$\delta^* x \in L_2([0,L])$ with $\|\delta^* x\|_2~=~1$ which maximizes
(\ref{EL-4}). By Cauchy's inequality
\begin{equation}\label{optstep}
\delta^* x~=~
\frac{   \nabla \|\nabla E\| - \|\nabla E\| \, \frac{d^2x}{ds^2}
 }{
\left\| \nabla \|\nabla E\| - \|\nabla E\| \, \frac{d^2x}{ds^2}
\right\|_2}.
\end{equation}
The optimal Euler-Lagrange relaxation towards a minimal
 action path accordingly is given by
\begin{equation}\label{EL-update}
    x_{n}~:=~ x_{n-1} - \alpha\,\delta^* x_{n-1},
\end{equation}
for sufficiently small  positive step size $\alpha$.
Two points are noteworthy. First, the updating depends only on $\|\nabla E\|$ and not
explicitly on $E$, and second, the term $d^2x/ds^2$ leads to reduction of local curvature
in regions where $\|\nabla E\|>0$. This prevents the updating scheme from producing
 kinks. Only at points with  $\|\nabla E\|=0$ the final
 solution may not be differentiable.


\subsection{Relation between action and path relaxation}
There exists a heuristic scheme to find a path between two minima
 across a saddle-point
in multidimensional energy landscapes, which
makes use of the fact that there is such a path which
 always runs along the energy gradient.

Starting from an arbitrary initial path $x_0(s)$ it is thus attempted to
adjust the path in the direction along the negative energy gradient, but perpendicular to the
local tangent.
This relaxation scheme can be analytically described as the solution of a boundary
value problem for a system of nonlinear partial differential equations
\begin{equation}\label{pde-1}
    \frac{\partial x}{\partial u}(u,s)~=~
    - \nabla E + \frac{\langle \frac{\partial x}{\partial s}, \nabla E \rangle}{\| \frac{\partial x}{\partial s}\|^2}
    \, \frac{\partial x}{\partial s},
\end{equation}
with the boundary conditions
\begin{equation}\label{bnd-1}
 ~~~x(0,s)~=~x_0(s),~x(u,0)~=~x_0(0),~x(u,L)~=~x_0(L).
\end{equation}



















%
