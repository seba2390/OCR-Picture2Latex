
\documentclass[a4paper]{article}
%\renewcommand{\baselinestretch}{1}
\usepackage{psfig}
%\newcommand{\scite}{\cite}
\def\bild#1{\def\thefigure{\arabic{figure}}%
   \refstepcounter{figure}\label{#1}~\thefigure}
\def\tabelle#1{\def\thetable{\arabic{table}}%
   \refstepcounter{table}\label{#1}~\thetable}
%\def\be{\begin{equation}}
%\def\ee{\end{equation}}
\def\eps{\varepsilon}
%\def\Cal{\frak}

\usepackage{amssymb}
\usepackage{amsmath}

\begin{document}

\title{
Energy barriers in three-dimensional micromagnetic models and the physics of thermo-viscous magnetization in multidomain particles}



\author{Karl Fabian \\
	Geological Survey of Norway,\\
Leiv Eirikssons vei 39,\\
7491 Trondheim,\\
Norway,\\
karl.fabian@ngu.no
	\and
	Valera P. Shcherbakov  \\
	Geophysical Observatory 'Borok',\\  Yaroslavskaja Oblast, 151742,\\ Russia \\
	}

\maketitle

\begin{abstract}
A first principle micromagnetic and statistical
calculation of
viscous remanent magnetization (VRM)  in an ensemble of cubic magnetite
pseudo single-domain particles
is presented.
This is achieved by developing
a fast relaxation algorithm for finding optimal
transition paths between micromagnetic local energy minima.
It combines a nudged elastic band technique with action minimization.
Initial paths are obtained by repetitive minimizations of
modified energy functions.
For a cubic pseudo-single domain particle, 60 different local energy minima are
identified and all optimal energy barriers between them
are numerically calculated for zero external field.
The results allow to  estimate also the energy barriers in
in weak external fields which are necessary to
construct the  time dependent  transition  matrices which describe  the continuous
homogeneous Markov processes of VRM acquisition and decay.
By   spherical averaging the
remanence acquisition in an isotropic PSD ensemble was calculated
over all time scales.
The modelled particle ensemble shows a physically meaningful overshooting
during VRM  acquisition.
The results also explain why
VRM acquisition  in PSD particles can occur much faster than
VRM decay and therefore can explain  for findings of extremely
stable VRM in some paleomagnetic studies.
\end{abstract}

 
\section{Introduction}
Reinforcement learning has achieved great success in areas such as Game-playing \citep{silver2018general,vinyals2019grandmaster}, robotics \cite{kober2013reinforcement}, large language models \citep{ouyang2022training}, etc.
However, due to safety concerns or physical limitations, in some real-world reinforcement learning problems, we must consider additional constraints that may influence the optimal policy and the learning process \citep{garcia2015comprehensive}.
% For example, a robotic arm must not take actions that may cause harm to itself or the environments.
A standard framework to handle such cases is the constrained Markov Decision Process (CMDP) \citep{altman1999constrained}.
Within the CMDP framework, the agent has to maximize
the expected cumulative reward while
obeying a finite number of constraints, which are usually in the form of expected cumulative cost criteria.

However, we are sometimes concerned with the problem with a continuum of constraints.
For example,
the constraints we meet might be time-evolving or subject to uncertain parameters, which
cannot be formulated as an ordinary CMDP
(see Examples \ref{Example_Time_Evolving} and  \ref{Example_Uncertain}).
In this paper we would study a generalized CMDP  
to address the above problem.  Because the constraints are not only infinite-number but also lie
in a continuous set,
the generalization is not trivial. Fortunately, we find that we can borrow the idea behind semi-infinite programming (SIP) \citep{remez1934determination, hettich1993semi} to deal with the semi-infinite constraints.
Accordingly, we propose \emph{semi-infinitely constrained Markov decision processes} (SICMDPs)
as a novel complement to the ordinary CMDP framework.
%More specifically,  an SICMDP model %, we consider 
%contains a continuum of constraints whereas an ordinary CMDP contains a finite number of constraints. 

%This generalization is natural but not trivial. However, we can brows the idea  
%The idea is quite natural and can be backtracked
%to the practice of extending linear programming to linear semi-infinite programming (LSIP) %\cite{remez1934determination, GobernaLSIO1998}.
%In addition, 
%As a complementary approach to the ordinary CMDP framework, 
%SICMDP can be used to model these problems  which cannot be described by a finite number of constraints
%that are not covered by .
%For example,
%the restrictions we consider can be time-evolving or subject to uncertain parameters
%, thus
%cannot be described by a finite number of constraints but a continuum of constraints 
%(see Examples \ref{Example_Time_Evolving} and  \ref{Example_Uncertain}).

We also present two reinforcement learning algorithms to solve SICMDPs called SI-CRL and SI-CPO, respectively.
SI-CRL is a model-based reinforcement learning algorithm designed for tabular cases, and SI-CPO is a policy optimization algorithm for non-tabular cases.
% and analyze its performance both theoretically and empirically.
The main challenge is that we need to deal with a continuum of constraints, thus reinforcement learning algorithms for ordinary CMDPs do not work anymore.
In SI-CRL, we tackle this difficulty by first transforming the reinforcement learning problem to an equivalent LSIP problem, which can then be solved using methods in the LSIP literature like the dual exchange methods \citep{Hu1990,reemtsen1998numerical}.
In SI-CPO, we resort to the idea of cooperative stochastic approximation developed in \cite{lan2020algorithms, wei2020comirror}.
As far as we know, we are the first to introduce tools from semi-infinitely programming (SIP) into the reinforcement learning community for solving constrained reinforcement learning problems.

% To the best of our knowledge, we are the first to apply tools from semi-infinitely programming (SIP) to solve reinforcement learning problems.
Furthermore, we give theoretical analysis for both SI-CRL and SI-CPO.
We decompose the error of SI-CRL into two parts: the statistical error from approximating the true SICMDP with an offline dataset and the optimization error due to the fact that the solution of the LSIP problem obtained by the dual exchange method is inexact.
On the optimization side, we show that the iteration complexity of SI-CRL is $O\left(\left\{\mathrm{diam}(Y)L\sqrt{|\gS|^2|\gA|m}/\left[(1-\gamma)\epsilon\right]\right\}^m\right)$.
On the statistical side, we show that the sample complexity of SI-CRL is $\widetilde O\left(\frac{|S|^2|A|^2}{\epsilon^2(1-\gamma)^3}\right)$ if the offline dataset is generated by a generative model, and $\widetilde O\left(\frac{|S||A|}{\nu_{\min} \epsilon^2(1-\gamma)^3}\right)$ if the dataset is generated by a probability measure $\nu$ as considered in \cite{chen2019information}.
Here $\widetilde O$ means that all logarithm terms are discarded.
For SI-CPO, things become a little more complicated because other than the statistical error and the optimization error, we also need to consider the function approximation error, which comes from imperfect policy parametrizations.
It is shown if the function approximation error can be controlled to $O(\epsilon)$ order, the iteration complexity of SI-CPO is $\widetilde{O}\left(\frac{1}{\epsilon^2(1-\gamma)^6}\right)$ and the sample complexity of SI-CPO is $\widetilde{O}(\frac{1}{\epsilon^4(1-\gamma)^{10}})$.
Here our iteration complexity bound is equivalent to a typical $\widetilde O(1/\sqrt{T})$ global convergence rate.

We perform a set of numerical experiments to illustrate the SICMDP model and validate our proposed algorithms.
Specifically, we examine two numerical examples, namely the discharge of sewage and ship route planning.
Through the discharge of sewage example, we show the advantage of the SICMDP framework over the CMDP baseline obtained by naive discretization in modeling realistic sequential decision-making problems.
Moreover, we demonstrate the effectiveness of the SI-CRL and SI-CPO algorithms in such tabular environments. 
In the ship route planning example, we illustrate the benefits of the SICMDP framework and the ability of the SI-CPO algorithm to address complex continuous control tasks involving continuous state spaces with modern deep reinforcement learning techniques.

% In summary, our contributions are listed as follows.
% First, we present the SICMDP model, which can be viewed as a generalization of the ordinary CMDP model.
% Second, we propose an algorithm to perform reinforcement learning for SICMDPs, which is called SI-CRL, and we believe that we are the first to apply tools from SIP
% to solve reinforcement learning problems.
% Third, we give a theoretical analysis of SI-CRL and identify both its sample complexity and iteration complexity.
% In addition, we perform numerical experiments to illustrate the SICMDP model and validate the SI-CRL algorithm.
% \{This paragraph can be removed!!! \}





 \section{Action and path integrals}
\subsection{Micromagnetic modeling}
  Berkov \cite{Berkov:98a,Berkov:98b,Berkov:98c}
developed a numerical method to evaluate   the distribution of energy barriers between
 metastable states in many-particle systems which determines the optimal path
between the two given metastable states by minimizing
the action in the Onsager�Machlup functional \cite{Onsager:53}
for the transition probability.
This method essentially performs a local saddle-point search in an high-dimensional
energy landscape.
Mathematically similar problems
exist in several disciplines of
physics and chemistry.
In the last years, several new methods to locate saddle-points
 have been developed in these fields \cite{Henkelman:00a,Henkelman:00b,Olsen:04}.
Based on such algorithms, an improved elastic band technique
for micromagnetics was presented by \cite{Dittrich:02}.

The main problem in energy-barrier computation is that micromagnetic structures $m$ are
described by many variables and accordingly
energy $E$ is a function of $m$.
Minimizing $E(m)$ requires sophisticated algorithms, but for
energy-barrier calculations it is even necessary to
determine saddle points in this high-dimensional energy landscape.

Several approaches are available for this task, but because saddle-point calculation
is equivalent to minimizing $(\nabla E(m))^2$ all rapidly converging
methods require second derivatives of $E$.
This however is rather to be avoided if the calculation should be performed
effectively.

The present study develops a combination of several of the above cited techniques to efficiently
calculate energy barriers in micromagnetic models.
\subsection{Action minimization}

Berkov \cite{Berkov:98a} introduced action minimization as a tool
for finding optimal transition paths in thermally driven
micromagnetic systems.
He discretized the time dependent action of the
the magnetic particle system and used a numerical quadrature
representation for direct numerical minimization.
This rigorous approach is complicated by its explicit dependence upon
transition time.
However, transition paths turn out to be geodesics of the energy surface
in the limit of infinite transition time, where
 energy barriers are lowest.

Dittrich~{\em et al.} \cite{Dittrich:02}
make use of this fact by directly searching for geodesic paths using a
modification of the nudged elastic band (NEB) algorithm
of Henkelman~{\em et al.}\cite{Henkelman:00a,Henkelman:00b}.

A problem of this algorithm is that it involves the numerical solution of a large
system of ordinary differential equations. Moreover, there is a tendency of
the NEB algorithm to produce spurious up-down-up movements
along the gradient (kinks) which cannot be completely removed in all cases.

Here, we combine both approaches by designing a
 path relaxation algorithm
similar to NEB, but constraint to decrease the action at each step.
The algorithm performs a fast gradient-like relaxation from an initial path
towards the optimal transition path. It
  detects and avoids the development of kinks, and does not involve
numerical solutions of differential equations.

The important problem of finding an
initial path which is likely to lie in the basin of attraction
of the optimal transition path under the proposed relaxation scheme
is also investigated.


\subsection{Geometric action}
Here we define the {\em geometric action} for a path $p$ as the minimal action for
any transition along this path.


In a general mechanic system the action of a transition process $x(t)$ from state
$x(0)=x_0$ into $x(t_{end})=x_1$ is defined
by
\begin{equation}\label{action-1}
    S(x(t)) ~:=~ \int \limits_0^{t_{end}} \langle \dot{x} + \nabla E, \dot{x} + \nabla E, \rangle\,dt.
\end{equation}
The probability that this transition process occurs depends monotonously on
$\exp(-S(x(t)))$.
In the next section we will be looking for the optimal transition path
in the energy landscape determined by $E$.
The quality of any given path $p$ is defined as its geometric action:
the action of the transition process $x_p(t)$
along   $p$ which minimizes (\ref{action-1}).

We start with a canonical parametrisation of $p$ by  arc length $s$
and try  to find  a reparametrisation $s(t)$ which minimizes $S$ along $p$.
For this optimal transition process we then have
\begin{equation}\label{action-2}
    S_{\rm min}(p)~=~S(x_p(t)) ~=~ \int \limits_0^{L(p)}
    \langle \frac{dx}{ds}\, v + \nabla E, \frac{dx}{ds}\, v  + \nabla E, \rangle\,\frac{ds}{v},
\end{equation}
where $L(p)$ is the arc length and $v(s)~=~ \frac{ds}{dt}(s)$ is the local velocity of the
optimal transition at arc length $s$.
Finding $s(t)$  thus is reduced to the  variational problem of finding
the function $v(s)$ which minimizes (\ref{action-2}).
The corresponding Euler-Lagrange equation is
\begin{equation}\label{Euler-1}
   \frac{d}{dv} \left( \frac{1}{v}\,
   \langle \frac{dx}{ds}\, v + \nabla E, \frac{dx}{ds}\, v  + \nabla E, \rangle \right) ~=~0.
\end{equation}
A short calculation confirms that it's solution is
\begin{equation}\label{Euler-2}
  v~=~ \|\dot{x}\|~=~ \|\nabla E\| \,\left\|\frac{dx}{ds}\right\|^{-1} ~=~\|\nabla E\|.
\end{equation}
The last equality uses the fact that for the arc length parametrisation
$\| {dx}/{ds}\|~=~1$. Inserting this result into (\ref{action-2}) yields
\begin{equation}\label{action-3}
    S_{\rm min}(p)  ~=~ 2\,\int \limits_0^{L(p)}
    \|\nabla E\|  \, \left\|\frac{dx}{ds}\right\| + \left\langle\frac{dx}{ds},\nabla E  \right\rangle~ds
    ~=~ 2\,\int \limits_0^{L(p)}
    \|\nabla E\|    + \left\langle\frac{dx}{ds},\nabla E  \right\rangle~ds.
\end{equation}
This integral can be simplified further by noting that
\begin{equation}\label{action-4}
 \int \limits_0^{L(p)} \left\langle\frac{dx}{ds},\nabla E  \right\rangle~ds ~=~
 \int \limits_{E(x_0)}^{E(x_1)}  dE  ~=~ E(x_1) -E(x_0)~=:~\Delta E.
\end{equation}
Accordingly, one obtains the geometric action of $p$
as
\begin{equation}\label{action-5}
    S_{\rm min}(p)  ~=~  2\,\Delta E~+~2\,\int \limits_0^{L(p)}
    \|\nabla E\| \,ds.
\end{equation}

\subsection{Finding the optimal path by variation of the geometric action}
It is possible to find the Euler-Lagrange equations for the
optimal path by variation of the geometric action
$S_{\rm min}(p)$ with respect to $x$.

To this end we reparametrize (\ref{action-5}) by $w(s)=s/L(p)$ and obtain
\begin{equation}\label{action-6}
     S_{\rm min}(p)  ~=~  2\,\Delta E~+~ 2\,\int \limits_0^{1}
    \|\nabla E\| \, \left\|\frac{dx}{dw}\right\|  \,dw.
\end{equation}
The    Euler-Lagrange equation
of the variational problem $ \delta S_{\rm min}(p) ~=~0$
after some simplification
has the form
\begin{equation}\label{EL-2}
\frac{d^2x}{ds^2}~=~  \nabla\, \log\|\nabla E\|.
\end{equation}
The details of the calculation are given in the appendix.

\subsection{The optimal path is a geodesic}
A path along an energy surface which fulfills
 \begin{equation}\label{TP-1}
\dot{x}~=~  \pm  \nabla E
  \end{equation}
is a geodesic.
In the one-dimensional case (\ref{Euler-2}) directly  implies that
the optimal transition path is a geodesic.
In the multidimensional case this
not simply  follows from (\ref{Euler-2})
which is valid for any
geometric transition path.
%
Yet, by applying the Cauchy inequality to
(\ref{action-6}) one obtains  for the optimal  path
$p$
\begin{equation}\label{TP-3}
     S_{\rm min}(p)  ~\geq~  2\,\Delta E~+~ 2\,\int \limits_0^{1}
    \left|\left\langle \nabla E ,\, \frac{dx}{dw}\right\rangle\right|  \,dw.
\end{equation}
The integration interval $[0,1]$ can be divided into
finitely many parts $[w_k,w_{k+1}]$
with alternating constant sign of $ \langle \nabla E ,\, dx/dw \rangle$.
Accordingly, $\nabla E(x(w_{k}))~=~0$ and
\begin{equation}\label{TP-4}
     S_{\rm min}(p)  ~\geq~  2\,\Delta E~+~ 2\,\sum \limits_{k=0}^{K}
    \left| \, E(x(w_{k+1}))- E(x(w_{k})) \right|.
\end{equation}
Here the right hand side is a lower limit of $ S_{\rm min}$.
Accordingly, a geodesic  which fulfills (\ref{TP-1}) achieves equality
in (\ref{TP-3}). It therefore coincides with the least action path between
the prescribed endpoints.
Equality in (\ref{TP-4}) means that the least action depends only upon the energies at
the traversed  critical points.

\subsection{Morse theory}
Topologically different critical points in multi-dimensions are distinguished
by their {\em Morse index}, which is defined as the dimension $n_-$
of the sub-manifold on which the Hessian is negative definite.
Intuitively, the Morse index of the highest saddle point along the optimal transition path
should not be too large. This is  because , apart from singular cases,
  the action is minimized along an only one-dimensional manifold,
   i.e. an isolated path parallel to the gradient which connects initial and final minima.
Therefore, if  at the  highest saddle point the Morse index is  $n_- > 1$,
the other $n_- - 1$ descending directions should lead into different LEM states.
The choice for these LEM states should not be too large whenever the initial and final minima
are close to the global  energy minimum.

On the other hand, Morse theory \cite{Milnor:63M} implies that the total number of saddle-points
will be huge in realistic micromagnetic calculations.
%
For a two-dimensional  surface  with Euler characteristic
$\chi_{\rm Euler}$,
the numbers
$N_{\rm min}$ of minima,
 $N_{\rm max}$ of maxima, and
 $N_{\rm saddle}$ of  saddle points
 are connected by the relation
\begin{equation}\label{crit}
 N_{\rm min} + N_{\rm max} -N_{\rm saddle} ~=~ \chi_{\rm Euler}.
 \end{equation}
For a sphere is $\chi_{\rm Euler}=2$.
 %
The generalization of (\ref{crit}) to a finite-dimensional
compact manifold is the {\em Morse relation}
 \begin{equation}\label{Euler-N}
 \sum \limits_{k=0}^{N} (-1)^k N_{k} ~=~ \chi_{\rm Euler}.
 \end{equation}
Here $N_{k}$  is the number of critical points
with Morse index  $k$, i.e.
where the negative definite sub-manifold has dimension $n_- = k$.
In case of the micromagnetic energy depending on $N$ magnetization directions
  the manifold is a direct product of $N$ two-dimensional
spheres, therefore   $ \chi_{\rm Euler} = 2^N$.
%
On this $2 N$-dimensional manifold we get from (\ref{Euler-N})
 \begin{equation}\label{crit-N}
 N_{\rm min} + N_{\rm max}+ N_{{\rm even}} -N_{ {\rm odd}} ~=~ 2^N,
 \end{equation}
where $N_{ {\rm even}}$ and $N_{{\rm odd}}$ are the number of
true saddle points
with  even or odd Morse index, respectively.

If this manifold describes  an ensemble of
 interacting SD grains
 each grain has at least two critical points,
 minima or maxima,  in zero external field.
 If   interaction is weak, the total energy of the system   inherits almost
 all these minima and maxima as saddle-points.
  Thus, it is understandable that the total number of critical points exceeds even a huge figure
  like $2^N$.

  The situation is different if the $2 N$-dimensional manifold
  describes an exchange coupled grain with inhomogeneous magnetization structure,
  like in   most micromagnetic applications.
  In this case, it turns out that only a limited number of   minima and maxima,
  like flower  or  vortex states, exist.
  %
     According to (\ref{crit-N}), there must appear an enormous number
     of true saddle points  with even Morse index -- i.e. not one-dimensional lines.

     Thus, inevitably there exists a large number of paths,
     with different action $S$,
     which  connect the local minima and complicate the  search of the lowest action path.
     %
     Another circumstance causing difficulties to find path lines by numerical means,
     is that the field lines passing through the saddle points,
     form sub-manifolds of zero measure.
     In other words, the saddle points are unstable in the sense,
     that almost all field lines in their vicinity are deflected aside
     (except for those which go directly through the saddle point).

%To briefly review Morse theory, we consider a differentiable
%function F : X ---> R on a real manifold X. Points where the
%differential of F vanishes are called critical points of F. A
%critical point is called non-degenerate if the Hessian of F is
%non-degenerate at this point. F is called a Morse function if all
%its critical points are non-degenerate. In applications to
%variational problems, X is the space of trial maps, F is the
%functional to be varied, and the critical points of F are the
%solutions to the variational problem. The non-degeneracy condition
%guarantees that the character of each critical point - local
%minimum, local maximum, or saddle - is determined by the Hessian of
%F at this point. The index of the Hessian is called the Morse index
%of the critical point. It is defined as the maximal dimension of a
%subspace on which the Hessian is negative definite. At a local
%minimum the Morse index is zero, at a local maximum it is equal to
%the dimension of X.

%Morse theory was first worked out by Morse [229Jump To The Next
%Citation Point] for the case that X is finite-dimensional and
%compact (see Milnor [224Jump To The Next Citation Point] for a
%detailed exposition). The main result is the following. On a compact
%manifold X, for every Morse function the Morse inequalities Nk &gt;
%Bk, k = 0,1,2,..., (59) and the Morse relation oo oo sum k sum k (-
%1) Nk = (- 1) Bk (60) k=0 k=0 hold true. Here Nk denotes the number
%of critical points with Morse index k and Bk denotes the kth Betti
%number of X. Formally, B k is defined for each topological space X
%in terms of the kth singular homology space Hk(X ) with coefficients
%in a field F (see, e.g., [78], p. 32). (The results of Morse theory
%hold for any choice of F.) Geometrically, B0 counts the connected
%components of X and, for k > 1, Bk counts the �holes� in X that
%prevent a k-cycle with coefficients in F from being a boundary. In
%particular, if X is contractible to a point, then B = 0 k for k > 1.
%The right-hand side of Equation (60View Equation) is, by definition,
%the Euler characteristic of X. By compactness of X, all Nk and Bk
%are finite and in both sums of Equation (60View Equation) only
%finitely many summands are different from zero.

\subsection{Euler-Lagrange path relaxation}
To numerically calculate the optimal geometric transition path
one would like to start from an arbitrarily chosen initial
path $p_0$ which then is iteratively improved
by some updating scheme.
An heuristic procedure is known as 'nudged elastic band' technique.
Here we derive an optimal updating scheme which corresponds to
a gradient minimization of minimal path action.
To this end we follow the standard derivation of the Euler-Lagrange
equation to prove that the change $\delta S_{\rm min}$ in action
due to a variation $\delta x$ in the path is given by
\begin{equation}\label{EL-4}
\delta S_{\rm min}~=~
2\,\int \limits_0^L \left( \nabla \|\nabla E\| - \|\nabla E\| \, \frac{d^2x}{ds^2}
\right)
\,\delta x\,ds.
\end{equation}
The direction of maximal increase in $S_{\rm min}$ now is a function
$\delta^* x \in L_2([0,L])$ with $\|\delta^* x\|_2~=~1$ which maximizes
(\ref{EL-4}). By Cauchy's inequality
\begin{equation}\label{optstep}
\delta^* x~=~
\frac{   \nabla \|\nabla E\| - \|\nabla E\| \, \frac{d^2x}{ds^2}
 }{
\left\| \nabla \|\nabla E\| - \|\nabla E\| \, \frac{d^2x}{ds^2}
\right\|_2}.
\end{equation}
The optimal Euler-Lagrange relaxation towards a minimal
 action path accordingly is given by
\begin{equation}\label{EL-update}
    x_{n}~:=~ x_{n-1} - \alpha\,\delta^* x_{n-1},
\end{equation}
for sufficiently small  positive step size $\alpha$.
Two points are noteworthy. First, the updating depends only on $\|\nabla E\|$ and not
explicitly on $E$, and second, the term $d^2x/ds^2$ leads to reduction of local curvature
in regions where $\|\nabla E\|>0$. This prevents the updating scheme from producing
 kinks. Only at points with  $\|\nabla E\|=0$ the final
 solution may not be differentiable.


\subsection{Relation between action and path relaxation}
There exists a heuristic scheme to find a path between two minima
 across a saddle-point
in multidimensional energy landscapes, which
makes use of the fact that there is such a path which
 always runs along the energy gradient.

Starting from an arbitrary initial path $x_0(s)$ it is thus attempted to
adjust the path in the direction along the negative energy gradient, but perpendicular to the
local tangent.
This relaxation scheme can be analytically described as the solution of a boundary
value problem for a system of nonlinear partial differential equations
\begin{equation}\label{pde-1}
    \frac{\partial x}{\partial u}(u,s)~=~
    - \nabla E + \frac{\langle \frac{\partial x}{\partial s}, \nabla E \rangle}{\| \frac{\partial x}{\partial s}\|^2}
    \, \frac{\partial x}{\partial s},
\end{equation}
with the boundary conditions
\begin{equation}\label{bnd-1}
 ~~~x(0,s)~=~x_0(s),~x(u,0)~=~x_0(0),~x(u,L)~=~x_0(L).
\end{equation}



















%

\section{A modified relaxation method to determine transition paths from micromagnetic models}

\subsection{Definitions}
A micromagnetic structure $m$ is determined by
$K$ magnetization vectors on a spacial grid over the particle.
Each magnetization vector is a unit vector determined by two
polar angles $\theta, \phi$.
The distance $d(m_1,m_2)$ between two magnetization structures is defined by
\begin{equation}\label{dist}
   d(m_1,m_2)~:=~ \left[ \frac{1}{V} \, \int \limits_V \arccos^2 ( m_1(r) \cdot m_2(r) ) \,dV\right]^{1/2}.
\end{equation}
The local direction vector from $m_1$ to $m_2$ is the gradient
\begin{equation}\label{dir}
  v(m_1,m_2)~:=~ -\nabla  d(\,.\,,m_2) (m_1).
\end{equation}

If two magnetization structures $m_0$ and $m_1$ contain no
opposite directions, i.e. for all $r\in V$ we have $ m_0(r) \neq -m_1(r)$,
then it is possible to linearly interpolate between $m_0$ and $m_1$ by
defining $ m_t(m_0,m_1) (r) $ as the intermediate vector
on the smaller  great circle segment connecting $m_0(r)$ and $m_1(r)$ which
has angular distance $t \arccos(m_0(r) \cdot m_1(r))$ from $m_0(r)$.

By minimization of $E(m)$
using gradient information $\nabla E (m)$
an initial minimum $m_A$ and a final minimum $m_B$
are found.


\subsection{Outline of the relaxation procedure}
In summary, the results of the previous section show that the transition probability
between $m_A$ and   $m_B$ is in very good approximation determined by the 
minimal energy barrier between them.
This barrier is achieved along some optimal geometrical least action path 
$m(s)$ with $m(0)~=~m_A$ and $m(1)~=~m_B$ which also represents the most 
likely transition path.
The state of maximal energy along this path is a saddle point of the 
total energy function $E(m)$, and the least action path is  everywhere parallel to 
$\nabla E(m)$.
%
To find this optimal transition path, we propose a relaxation method 
which combines the advantages of the NEB technique of \cite{Dittrich:02}
with the additional action minimization of \cite{Berkov:98b}. 


Two techniques are required for finding the optimal path by means of iterative 
relaxation: 
\begin{enumerate}
  \item  An updating scheme which determines an improved transition path $m^{k+1}(s)$
  from a previous path $m^k(s)$.
  \item A method of finding an initial transition path $m^0(s)$ within the basin of attraction of 
   the optimal path.
\end{enumerate}

The here proposed updating scheme starts from an
initial path $m^0(s)$ which is 
determined  by interpolating  $N$ intermediate states
$m^0(s_j)$ which correspond to the magnetization structures at the 
$s$-coordinates $0=s_1<s_2<\ldots<s_N=1$ for $j=1,\ldots, N$.
For the interpolation to be well-defined,  $N$ is required to be large enough 
to ensure that neighboring structures $m^k(s_j)$ and $m^k(s_{j+1})$ never contain opposite 
magnetization vectors. 

Similar to the NEB method, in step $k$ the path $m^k(s)$ is changed according to
\begin{equation}\label{relax}
    m^{k+1}(s) ~:=~  m^{k}(s) - \alpha_k \, \left[ \nabla E(m^k(s))-
          \left( \nabla E(m^k(s)) \cdot t^k(s)  \right)\,t^k(s)\right],
\end{equation}
where $t^k(s)$ is the tangent vector to the path $m^k(s)$  at $s$ and
$\alpha_k>0$ is a real number.
This updating scheme moves the path downward along the part of the energy gradient which
is perpendicular to path itself. This algorithm 
converges to a path which is almost everywhere 
parallel to $\nabla E$.
However,  (\ref{relax}) describes  not a true gradient 
descent for the action, and the final path may not achieve minimal action
  due to the formation of kinks during the minimization 
\cite{Henkelman:00a,Henkelman:00b,Dittrich:02}.

The here proposed method differs from previous NEB techniques  
in two details:
First,  
$\alpha_k$ is chosen such that $S( m^{k+1})< S( m^{k})$. This ensures that 
the action decreases in each step.
Second, $\alpha_k$ is dynamically adapted to achieve rapid convergence.
The following procedure to choose  $\alpha_k$  
fulfills both aims.
 
\subsubsection*{Adaption of $\alpha_k$}
Starting with the  initial value $\alpha_0=1$,
it is evaluated after each step whether the action of the updated path is decreased, i.e.
$ S( m^{k+1}(s)) < S( m^{k}(s))$. 
%
While this is true, the new value
$\alpha_{k+1}=\alpha_k$ is kept constant, but only for at most five steps. 
In this phase (\ref{relax}) performs a  quasi-gradient descent 'creeping' towards 
the optimal path. 

Afterwards,
if $S$ still decreases, we set $\alpha_{k+1}=2\,\alpha_k$ 
in each following step until
some $ S( m^{k+1}(s)) > S( m^{k}(s))$.
This phase can be interpreted as an 'accelerated steepest descent'.

If at any time $ S( m^{k+1}(s)) > S( m^{k}(s))$, 
the path $ m^{k+1}(s)$ is rejected and (\ref{relax})
is evaluated for 
a new  $\alpha_{k+1}$-value  of
$\alpha_{k+1}=1/4\,\alpha_k$.
This behavior avoids 'overshooting' of the gradient descent and 


All our tests show that this iterative adaption of $\alpha_k$ 
leads to a  much faster convergence than
choosing any fix value $\alpha_k=\alpha$.
 
By comparing the achieved action $S( m^{k}(s))$ to $ S_{\rm min}$ from
(\ref{TP-4})
during the relaxation, it is possible to detect  the 
formation of kinks and to decide when the minimization succeeds.
This action criterion is better than testing whether 
the final path is parallel to $\nabla E$, since
the latter is also true for paths with kinks.
 
\subsection{Determination of the initial path}
Since the relaxation scheme works similar to a gradient
minimization algorithm, it adjusts to the next local optimum of the
action function.
Therefore the choice of the initial path is crucial for obtaining the globally
optimal path.
Here we propose a  method to find a good initial path $m^0(s)$ by
a sequence of minimizations of modified energy functions.

For parameters $\mu, \beta, \eps$ and $\Delta$ we consider the
modified energy function
\begin{equation}\label{modEn}
   E^\ast_\Delta(m) ~=~
         E(m) + \mu \,[ d(m, m_A) -\Delta]^2 + \frac{\beta}{d(m,m_B)+\eps}.
\end{equation}
The parameters  $\mu, \beta, \eps$ are chosen such that
for $\Delta=d(m_B, m_A) $ the final state $m_B$ is a unique
optimum of $E^\ast_\Delta(m)$, while for $m$ with  $d(m , m_A) \ll d(m_B, m_A)$ the last term of
$E^\ast_\Delta(m)$ is small in comparison to $E(m)$.
The value of $\mu$ should be large enough to ensure that the distance between
$m_A$ and the minimum of
$E^\ast_\Delta$ is indeed close to $\Delta$.

Now a sequence $m^0_j,~j=1,\ldots, J$ of magnetization structures is iteratively
obtained by setting $m^0_0~=~m_A$ and
$m^0_{j}$ to the result of minimizing $E^\ast_{\Delta_j} (m)$, where
$\Delta_j~=~j/J\,d(m_B, m_A)$.

Interpolating between the  structures  $m^0_j$ determines an
initial path  which (1) starts in $m_A$ and ends in $m_B$,
(2) has relatively equally spaced
intermediate states $m^0_{j}$, and (3) prefers
 intermediate states $m^0_{j}$ at distance $\Delta_j$ with low energy $E(m)$.

\section{Energy barriers in a cubic pseudo-single domain particle}
\begin{table}[t!]
\centering
\caption{Voice conversion \& F0 manipulation results. MOS results are reported with 95\% confidence interval. VDE, and FFE are reported for F0 manipulation while PER, WER, EER, and MOS are reported for voice conversion. Notice, for VDE, and FFE higher is the better since F0 was flattened.}
\label{tab:conv}

\resizebox{1\columnwidth}{!}{
\begin{tabular}{c@{~} | c@{~} | c@{~}c@{~} | c@{~} | c@{~} ||  c@{~}c@{~} }
\toprule
\multirow{2}{*}{Dataset} & \multirow{2}{*}{Method} & \multicolumn{4}{c||}{Voice Conversion} & \multicolumn{2}{c}{F0 Manipulation} \\
\cmidrule{3-8}
& & PER~$\downarrow$ & WER~$\downarrow$ & EER~$\downarrow$ & MOS~$\uparrow$ & VDE~$\uparrow$ & FFE~$\uparrow$ \\
\midrule
VCTK & GT  & 17.16 & 4.32 & 3.25 & 4.11$\pm$0.29 & -- & -- \\
\midrule 
\multirow{3}{*}{LJ}
% & ASR-TTS   & 50.74  & --     & 66.08 & 32.96 & 1.46 \\
& CPC       & 22.22 	& 16.11 		& 0.46 		& 3.57$\pm$0.15 		& \bf 46.68 & \bf 48.71\\
& HuBERT    & \bf 19.09 & \bf 12.23 & \bf 0.31  & \bf 3.71$\pm$0.24 & 39.20 		& 48.42\\
& VQ-VAE    & 40.88 	& 36.96 		& 9.65 		& 2.90$\pm$0.17 		& 10.54 	& 12.08 \\
\midrule 
\multirow{3}{*}{VCTK} 
% & ASR-TTS   & 68.88  & --    & 41.77 & 13.55 & 6.48 \\
& CPC       &  23.58 		& 15.98 		& \bf 4.83  &  3.42 $\pm$ 0.24 		& \bf 25.29 & \bf 26.97 \\
& HuBERT    &  \bf 20.85 	& \bf 12.72 & 6.01  		& \bf  3.58 $\pm$ 0.28 	& 23.46 	& 26.67 \\
& VQ-VAE    & 36.88  		& 29.44 		& 11.56 		& 3.08 $\pm$ 0.34 		& 7.03  	& 7.80  \\
\bottomrule
\end{tabular}}
\vspace{-0.4cm}
\end{table}

\vspace{-0.1cm}
\section{Results}
\vspace{-0.1cm}
Our results cover
% We report results for 
three different settings: (i) speech reconstruction experiments; (ii) speaker conversion and F0 manipulation; (iii) bitrate analysis with subjective tests for speech codec evaluation. We employ two datasets: LJ~\cite{ljspeech17} single speaker dataset and VCTK~\cite{vctk} multi-speaker dataset. All datasets were resampled to a 16kHz sample rate.

% \paragraph*{Implementation Details.}
% \smallskip
\noindent{\bf Implementation Details\quad} 
\label{sec:impl}
We follow the same setup as in~\cite{lakhotia2021generative}. For CPC, we used the model from~\cite{Riviere2020}, which was trained on a ``clean'' 6k hour sub-sample of the LibriLight dataset~\cite{Kahn2020,Riviere2020}. We extract a downsampled representation from an intermediate layer with a 256-dimensional embedding and a hop size of 160 audio samples. For HuBERT we used a \textsc{Base} 12 transformer-layer model trained for two iterations~\cite{hsu2020hubert} on 960 hours of LibriSpeech corpus~\cite{Panayotov2015}. 
% This model encodes every 320 raw audio samples into a 768-dimensional vector. 
This model downsamples the raw audio $\times320$ into a sequence of 768-dimensional vectors. Similarly to~\cite{lakhotia2021generative}, activations were extracted from the sixth layer.

%CPC: We use a dictionary of 100 units, leading to a bitrate of 700bps.
%HuBERT: A dictionary of 100 units is used, leading to a bitrate of 350bps. 
%VQVE: The VQ-VAE discrete code operates at a bitrate of 800bps.
% For both CPC and HuBERT, the k-means algorithm is applied to convert continuous frames to discrete codes, using the LibriSpeech clean-100h~\cite{Panayotov2015} dataset. 
For CPC and HuBERT, the k-means algorithm is trained on LibriSpeech clean-100h~\cite{Panayotov2015} dataset to convert continuous frames to discrete codes. We quantize both learned representations with $K=100$ centroids. Leading to a bitrate of 700bps for CPC and 350bps for HuBERT.

% VQ-VAE
Similarly to CPC models, we trained the VQ-VAE content encoder model on the ``clean'' 6K hours subset from the LibriLight dataset. We use an encoder operating on the raw signal to extract discrete units, similar to~\cite{jukebox}. In addition, ``random restarts'' were performed when the mean usage of a codebook vector fell below a predetermined threshold. Finally, we used HiFiGAN (architecture and objective) as the decoder instead of a simple convolutional decoder, as it improved the overall audio quality. This model encodes the raw audio into a sequence of discrete tokens from 256 possible tokens~\cite{garbacea2019low} with a hop size of 160 raw audio samples. The VQ-VAE discrete code operates at a bitrate of 800bps. We additionally experimented with 100 discrete units for VQ-VAE, however results were the best for 256. This finding is consistent with~\cite{garbacea2019low}.

% verification model
The speaker verification network uses the architecture proposed in~\cite{heigold2016end}. It was trained on the VoxCeleb2~\cite{voxceleb2} dataset, achieving a 7.4\% Equal Error Rate (EER) for speaker verification on the test split of the VoxCeleb1~\cite{Nagrani17} dataset.

% pitch
Only a single F0 representation is considered across all evaluated models, trained on the VCTK dataset.
% The F0 is extracted from the raw audio using YAAPT~\cite{yaapt} algorithm, using a window size of 20ms and a 5ms hop. 
The F0 is extracted from the raw audio using a window size of 20ms and a 5ms hop. 
As a result, the F0 sequence is sampled at 200Hz. 
% We apply the quantization described at Sec.~\ref{sec:method}, using a pitch codebook of $K'=20$ tokens and an encoder that downsamples the pitch by $\times16$. 
The quantization described at Sec.~\ref{sec:method}, is applied using an F0 codebook of $K'=20$ tokens and an encoder that downsamples the signal by $\times16$. Hence, the discrete F0 representation is sampled at 12.5Hz, leading to a bitrate of 65bps. The final bitrate of the evaluated codecs is the sum of the pitch code bitrate with the content code bitrate.

% \paragraph*{Evaluation Metrics}
% \smallskip
\noindent{\bf Evaluation Metrics\quad} 
We consider both subjective and objective evaluation metrics. For subjective tests, we report the Mean Opinion Scores (MOS). In which human evaluators rate the naturalness of audio samples on a scale of 1--5. Each experiment, included 50 randomly selected samples rated by 30 raters. For objective evaluation, we consider: (i) Equal Error Rate~(EER) as an automatic speaker verification metric obtained using a pre-trained speaker verification network. We report EER between test utterances and enrolled speakers; (ii) Voicing Decision Error (VDE)~\cite{nakatani2008method}, which measures the portion of frames with voicing decision error; (iii) F0 Frame Error (FFE)~\cite{chu2009reducing}, measures the percentage of frames that contain a deviation of more than 20\% in pitch value or have a voicing decision error; (iv) Word Error Rate (WER) and Phoneme Error Rate (PER), proxy metrics to the intelligibility of the generated audio. We used a pre-trained ASR network~\cite{baevski2020wav2vec} on both reconstructed and converted samples to calculate both metrics. %To generate target phonemes, the g2p-en~\cite{g2pE2019} Grapheme2Phoneme module was used.

% \vspace{-0.1cm}
% \smallskip
\noindent{\bf Reconstruction \& Conversion}
% \vspace{-0.1cm}
We start by reporting the reconstruction performance. Results are summarized in Table~\ref{tab:recon}. When considering the intelligibility of the reconstructed signal HuBERT reaches the lowest PER and WER scores across all models, where both CPC and HuBERT are superior to VQ-VAE. However, when considering F0 reconstruction VQ-VAE outperforms both HuBERT and CPC by a significant margin. This results are somewhat intuitive, bearing in mind VQ-VAE objective is to fully reconstruct the input signal. In terms of subjective evaluation, all models reach similar MOS scores, with one exception of CPC on LJ. 

%Notice, since the same F0 units are used for each method, this result implies the VQ-VAE units contain some information about the F0 of the signal, enabling better reconstruction. Regarding speaker information, the CPC gets the lowest EER. 

To better evaluate the disentanglement properties of each method with respect to speaker identity and F0, we conducted an additional set of experiments aiming at speaker conversion and F0 manipulation. For voice conversion, we converted each test utterance into five random target speakers. Next, we employed a speaker verification network, which extracts \emph{d-vector} representation to evaluate speaker-converted utterances' similarity to real speaker utterances (low error-rate indicates good conversion), providing measurement to the speaker identity's disentanglement from the evaluated coding method. The error-rate is reported between converted test utterances and enrolled speakers. For the LJ speech single speaker dataset, we converted samples from the VCTK dataset to the single speaker and enrolled all VCTK speakers together with the single speaker. Results are summarized in Table~\ref{tab:conv} (left). Unlike resynthesis results, on voice conversion CPC and HuBERT outperform VQ-VAE on both LJ and VCTK datasets, indicating VQ-VAE contains more information about the speaker in the encoded units, hence producing more artifacts. Notice, this also affects WER, PER, and the overall subjective quality (MOS). 

Next, to evaluate the presence of F0 in the discrete units, we flattened the F0 units before synthesizing the signal and calculated VDE and FFE with respect to the original F0 values. F0 flattening was done by setting the speakers' mean F0 value across all voiced frames. In this experiment, we expected units that contain F0 information to be better at F0 reconstruction over disentangled units. Results are summarized in Table~\ref{tab:conv} (right). Notice VQ-VAE can still reconstruct the F0 almost at the same level as when using the original F0 as conditioning (5.2 vs 7.03, and 5.59 vs 7.8), in contrast to CPC and HuBERT.

\begin{figure}[t!]
\centering
\includegraphics[width=0.65\columnwidth, trim={50 20 70 20}]{figures/codec_2.pdf}
% \caption{MUSHRA subjective listening test results as a function of bitrate per second for various methods. Purple dots denote the baseline methods, and green dots the proposed SSL based method.} 
\caption{MUSHRA subjective quality results as a function of bitrate per second. Purple dots denote the baseline methods, and green dots the proposed SSL based method.} 
\label{fig:codec}
\vspace{-0.5cm}
\end{figure}

% \vspace{-0.1cm}
% \smallskip
\noindent{\bf Speech Codec}
Our final experiment evaluates the obtained speech units as a low bitrate speech codec. 
% Therefore, we evaluate how the performance varies as a function of the number of discrete units. Changing the number of units is equivalent to varying the bitrate of the encoded signal. 
We use a subjective MUSHRA-type listening test~\cite{series2014method} to measure the perceived quality of the proposed speech codec with regard to its bitrate constraints. In MUSHRA evaluations, listeners are presented with a labeled uncompressed signal for reference, a set of test samples to rate, a copy of the uncompressed reference, and a low-quality anchor. Listeners are asked to rate each test utterance and the copy of the uncompressed reference with respect to the labeled reference in a scale of 1-100.

The experiment is performed on the VCTK dataset~\cite{vctk}. For evaluation, we used 20 utterances from 5 speakers. The set of speakers in the test data is disjoint with those in the training data. For this experiment, HuBERT models with 50, 100, and 200 units were trained as described in Sec.~\ref{sec:impl}. For comparison, we included other speech codecs in our evaluation: Opus~\cite{valin2012definition} wideband at 9 kbps VBR, Codec2~\cite{rowe2011codec} at 2.4 kbps and LPCNet~\cite{valin2019real} operating at 1.6 kbps. The LPCNet model was trained from scratch on the VCTK dataset following the experimental setup in~\cite{valin2019real}. The VQ-VAE model employs the HiFiGAN decoder trained on the LibriLight dataset to match the amount of data reported in~\cite{garbacea2019low}. We compressed the anchor sample with Speex~\cite{valin2016speex} at 4 kbps as a low anchor. Fig.~\ref{fig:codec} depicts the results. HuBERT with 50 units reaches the best MUSHRA score while its bitrate is only 365bps, which is significantly lower than the baseline methods.
%\subsection{First principles calculation of magnetic viscosity}
\subsection{Statistical theory of MD VRM}
The first application of the above presented calculation of energy barriers is a complete
description of low-field viscous magnetization processes in a micromagnetically modelled
cubic particle.
%
%
Knowing the optimal transition paths between all  LEM structures $S_i$
of the investigated particle
allows for calculating the zero field temporal isothermal transition matrix $M(\Delta t)$,
which describes the continuous homogeneous Markov process of random thermally activated transitions
between all possible states:
\begin{equation}\label{transmat}
   M (\Delta t)~\equiv~\mathbb{P}\left[{ S(t)= S_j  \wedge S(t+\Delta t)= S_i }\right]~=~
   \,\exp\left[\mu \, \Delta t \right].
\end{equation}
Here the matrix elements $\mu_{ij}$ of the infinitesimal generator of the semigroup  $M(t)$ are given by
the relative outflow from $S_j$ to $S_i$ for $i\neq j$. The relative inflow from all other
states determines the diagonal element $\mu_{ii}$.
\begin{eqnarray}
% \nonumber to remove numbering (before each equation)
  \mu_{ij} &=& -\frac{\Delta E_{ij}}{k_B T\,\tau_0} ~~\mbox{for}~~i\neq j\\
  \mu_{ii}&=& -\sum \limits_{i\neq j}  \mu_{ji}
\end{eqnarray}
Once, the $ \mu_{ij}$ have been calculated, it is easily possible to determine the viscous decay
of any initial probability distribution
$\rho_i^0~\equiv~\mathbb{P}\left[{ S_0=  S_i }\right]$
by   multiplication with the time evolution matrix exponential
\begin{equation}\label{probdens}
   \rho( t) ~=~\exp\left[\mu \,  t \right]\, \rho_i^0
\end{equation}
Multiplication by the corresponding magnetizations $m_i$ of states $S_i$ yields the
viscous evolution of remanence
\begin{equation}\label{VRMt}
    m(t) ~=~ \sum \limits_i m_i\, \rho_i(t).
\end{equation}
%
When a small field $H$ is applied, the energy barrier
$E_b^{ij}$ in first order  changes according to
\begin{equation}\label{field-barr}
    E_b^{ij}(H)~=~  E_b^{ij}+ (m_j - m_{ij}^{\max})\, H,
\end{equation}
where $m_{ij}^{\max}$ denotes the magnetization  at the  maximum
energy state along the optimal transition path from $S_i$ to $S_j$.
The approximation used to obtain (\ref{field-barr}) assumes that $H$ is so small
that it does not change the magnetization structures of the LEM and saddle-point states noticeably.
Only the field induced energy is taken into account.
It is easily seen that all other energy changes are of second order in $H$.

Using the in-field energy barriers it is straightforward  to determine the
matrix exponential which governs VRM acquisition. By defining
\begin{eqnarray}
% \nonumber to remove numbering (before each equation)
  \mu_{ij}(H) &=& -\frac{E_b^{ij}(H)}{k_B T\,\tau_0} ~~\mbox{for}~~i\neq j\\
  \mu_{ii}(H)&=& -\sum \limits_{i\neq j}  \mu_{ji}(H),
\end{eqnarray} the above zero-field theory automatically  extends to the weak field case.

In   case of our cubic PSD particle, all matrices are of size $60\times60$
and the calculations have been performed by a Mathematica (\copyright Wolfram Research)
program.

\subsection{Viscous remanence acquisition and decay in an ensemble of cubic PSD magnetite}
Using the above mathematical methods it is possible to calculate
the statistics of viscous remanence acquisition and decay for our
single PSD particle
with respect to any field vector of sufficiently small length $H$.
In order to model an isotropic ensemble, it is necessary to average the
VRM properties over all possible field directions. This has been approximated by drawing 20
random directions  from an equi-distribution over the unit sphere and averaging the
modelled VRM acquisition and decay curves.
For room temperature this yielded   the ensemble curve as shown in Fig.~\ref{ViscAcq}.
~\\[3mm]%
\par
\vbox{ \centerline{\hbox{ \psfig{figure=ViscAcq-2.eps,width=120mm}
}} \footnotesize
%  \begin{center}
{\bf Figure \bild{ViscAcq} :}
Modelled acquisition of viscous magnetization in the
cubic particle with $\lambda = 5.0$. The initial state is
an
equi-distribution over all possible LEM states with zero net magnetization.
In a
small external field the first acquisition process is the
immediate decay from V-$110$ type states into V-$100$ type states,
which occurs within about $10^{-9}$s.
Due to the field induced asymmetry of the energy barriers, a remanence is acquired
during this process. The second process is a decay of
F-$111$ type states into V-$100$ type states. This occurs between about $10^2$s and $10^3$s
and shows an intermediate overshooting of remanence.
%  \end{center}
\normalsize }
~\\[0.2cm]%
The left hand side of Fig.~\ref{ViscAcq} shows the remanence acquisition in a modelled
field of $H=60\mu$T when starting from an initial state $\rho_0$ at $t=0$ which assigns equal probability
to all existing LEM states.
Already within $10^{-9}$s the remanence increases rapidly due to the immediate
depletion of the nearly unstable V-$110$ vortex states which decay into the
  stable
V-$100$ states (see Table~\ref{trans-barr-2}).
The remanence forms because in zero field there are two equally probable
transitions, e.g. V-$110~\rightarrow$~V-$100$ and V-$110~\rightarrow$~V-$010$.
Within the external field one of these decay paths becomes more probable which leads to
a relative overpopulation of the field aligned V-$100$ type states.
Nearly synchronously there occurs a two step process
V-$111~\rightarrow$~V-$110~\rightarrow$~V-$100$.It
is controlled by the somewhat slower transition V-$111~\rightarrow$~V-$110$, but still
both take place within the first few $10^{-9}$s.
The last VRM acquisition processes occurs only after a much longer waiting time of $10-10^3$s.
First the initial F-$111$ type states transform via F-$110$ type states into V-$110$ type states
which then immediately decay into V-$100$
(Fig.~\ref{state-scheme}).
This last process  produces an astonishing remanence overshooting as displayed in
Fig.~\ref{ViscAcq}: The remanence during the VRM acquisition process is for a certain time
higher than the finally obtained equilibrium VRM.
In the next section we will show that is is not an artifact of the modelling,
but can be explained by a  real physical process.

The right hand side of  Fig.~\ref{ViscAcq} shows that
when the field is switched off after VRM acquisition, the obtained remanence is carried
only by   extremely stable  V-$100$-type states which require a   theoretical
waiting time of $10^{15}$s to equilibrate into a zero remanence state. 
\section{Viscous magnetization}
\subsubsection{Viscous magnetization anomalies}
One of the most astonishing results of this study with respect to viscous magnetization is the predicted transient
increase of remanence during
the VRM acquisition. To understand this phenomenon more closely, we
give a physical explanation of this effect in terms of a simplified model.


The in field potential barrier is
\begin{equation}
E_s - E_F + (m_F - m_s) h.
\end{equation}
Here $E_s$ is the energy at the saddle point and $E_F$ is the zero field
energy of the flower state.
$W+$ and $W-$ are the number of grains (probability) in the vortex $V-100$
parallel and antiparallel to the external field $h$ in $x$-direction.

\subsubsection{High stability of PSD VRM}
Extremely stable VRM has been often observed in paleomagnetic studies.
The above mechanisms give a first theoretical explanation why
high stability of VRM should occur in PSD samples.
The basic process is the relaxation of naturally produced
metastable states into stable ones.

This typically occurs for TRM acquisition in PSD ensembles where the cooling rate is fast enough to
stabilize metastable flower states.
Then a long term VRM, acquired in the field after cooling and carried by newly formed
 vortex states is extremely
stable and can significantly bias any paleomagnetic measurement, especially paleointensity determinations.


Laboratory AF demagnetization  rather leads to a more stable
LEM state (perhaps even the GEM) because the magnetization structure is
provided with a lot of energy which is stepwise reduced.
Thus AF-demagnetization would rather end up in a vortex state for a PSD particle.

During natural chemo-viscous (VCRM) magnetization by crystal growth a grain
changes sequential from the  SP state into a stable SD and later a PSD state.
The first stable SD state is almost homogeneously magnetized along a $\langle 111\rangle$-axis.
It then transforms in a more developed flower state which then
becomes metastable as soon as the vortex has lower energy.
At this point,  the process of VRM acquisition will starts to
produce extremely stable remanences.

\section{Appendix}
\subsubsection*{Details of calculation}
This uses tensor calculus and Einstein's sum convention and for
comparison with classical physics interprets the variable $w$ as a time $t$ and $x$ as a
generalized variable $q$.
We have to apply the Euler-Lagrange operator
$\frac{\partial}{\partial q} -  \frac{d}{dt}\frac{\partial}{\partial \dot{q}}$
to the Lagrange function $L(q,\dot{q}) ~=~ \sqrt{g_i\,g_i } \,
 \sqrt{ \dot{q}_i \dot{q}_i}$, where $g_i ~=~ \partial_i E(q)$.
This needs the following expressions
\begin{equation}\label{NR1}
  \dot{\partial}_k \|\dot{q}\|~:=~\frac{\partial}{\partial \dot{q}_k}\, \sqrt{ \dot{q}_i \dot{q}_i}~=~\frac{\dot{q}_k}{\|\dot{q}\|}
\end{equation}
\begin{equation}\label{NR2}
 \frac{d}{dt} \,\dot{\partial}_k \|\dot{q}\| ~=~ \frac{d}{dt}\,\frac{\dot{q}_k}{\|\dot{q}\|}~=~
\frac{\ddot{q}_k}{\|\dot{q}\|}-\frac{\dot{q}_k\,(\dot{q}_j\,\ddot{q}_j)}{\|\dot{q}\|^3}~=~
\frac{\ddot{q}_k}{\|\dot{q}\|}.
\end{equation}
The last equation uses the fact that in the chosen parametrisation
 the tangent vector $\dot{q}$ has constant length along the path
and thus   is perpendicular to $\ddot{q}$.
Further,
\begin{equation}\label{NR3}
\partial_k (g_i\,g_i)^{1/2}~=~\frac{ g_j\,\partial_k g_j}{(g_i\,g_i)^{1/2}}~=~\frac{ g_j\,\partial_k g_j}{\|g\|}.
 \end{equation}
Putting this together results in
\begin{equation}\label{NR4}
\left( \frac{\partial}{\partial q} -  \frac{d}{dt}\frac{\partial}{\partial \dot{q}}\right)\,L(q,\dot{q})
 ~=~
 \|\dot{q}\|\,\partial_k\,\|g\| - \|g\|  \frac{d}{dt} \,\dot{\partial}_k \|\dot{q}\|~=~
 \frac{\|\dot{q}\|}{\|g\|}\, g_j\,\partial_k g_j - \frac{\|g\|}{\|\dot{q}\|}\,\ddot{q}_k.
  \end{equation}
Using the arc length parametrisation where $\|\dot{q}\|=1$,
this finally leads to the Euler-Lagrange equation
\begin{equation}\label{EL-3}
\ddot{q}~=~ \frac12 \,\frac{\nabla\, \|g\|^2}{\|g\|^2}~=~  \nabla\,\log \|g\|.
\end{equation}


\bibliography{kf,vrm,trm,hys}
\bibliographystyle{amsplain}

\end{document}
