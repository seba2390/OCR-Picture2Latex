\subsubsection{Viscous magnetization anomalies}
One of the most astonishing results of this study with respect to viscous magnetization is the predicted transient
increase of remanence during
the VRM acquisition. To understand this phenomenon more closely, we
give a physical explanation of this effect in terms of a simplified model.


The in field potential barrier is
\begin{equation}
E_s - E_F + (m_F - m_s) h.
\end{equation}
Here $E_s$ is the energy at the saddle point and $E_F$ is the zero field
energy of the flower state.
$W+$ and $W-$ are the number of grains (probability) in the vortex $V-100$
parallel and antiparallel to the external field $h$ in $x$-direction.

\subsubsection{High stability of PSD VRM}
Extremely stable VRM has been often observed in paleomagnetic studies.
The above mechanisms give a first theoretical explanation why
high stability of VRM should occur in PSD samples.
The basic process is the relaxation of naturally produced
metastable states into stable ones.

This typically occurs for TRM acquisition in PSD ensembles where the cooling rate is fast enough to
stabilize metastable flower states.
Then a long term VRM, acquired in the field after cooling and carried by newly formed
 vortex states is extremely
stable and can significantly bias any paleomagnetic measurement, especially paleointensity determinations.


Laboratory AF demagnetization  rather leads to a more stable
LEM state (perhaps even the GEM) because the magnetization structure is
provided with a lot of energy which is stepwise reduced.
Thus AF-demagnetization would rather end up in a vortex state for a PSD particle.

During natural chemo-viscous (VCRM) magnetization by crystal growth a grain
changes sequential from the  SP state into a stable SD and later a PSD state.
The first stable SD state is almost homogeneously magnetized along a $\langle 111\rangle$-axis.
It then transforms in a more developed flower state which then
becomes metastable as soon as the vortex has lower energy.
At this point,  the process of VRM acquisition will starts to
produce extremely stable remanences.
