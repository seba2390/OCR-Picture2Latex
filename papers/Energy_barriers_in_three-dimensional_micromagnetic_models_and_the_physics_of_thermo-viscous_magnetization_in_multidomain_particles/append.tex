\subsubsection*{Details of calculation}
This uses tensor calculus and Einstein's sum convention and for
comparison with classical physics interprets the variable $w$ as a time $t$ and $x$ as a
generalized variable $q$.
We have to apply the Euler-Lagrange operator
$\frac{\partial}{\partial q} -  \frac{d}{dt}\frac{\partial}{\partial \dot{q}}$
to the Lagrange function $L(q,\dot{q}) ~=~ \sqrt{g_i\,g_i } \,
 \sqrt{ \dot{q}_i \dot{q}_i}$, where $g_i ~=~ \partial_i E(q)$.
This needs the following expressions
\begin{equation}\label{NR1}
  \dot{\partial}_k \|\dot{q}\|~:=~\frac{\partial}{\partial \dot{q}_k}\, \sqrt{ \dot{q}_i \dot{q}_i}~=~\frac{\dot{q}_k}{\|\dot{q}\|}
\end{equation}
\begin{equation}\label{NR2}
 \frac{d}{dt} \,\dot{\partial}_k \|\dot{q}\| ~=~ \frac{d}{dt}\,\frac{\dot{q}_k}{\|\dot{q}\|}~=~
\frac{\ddot{q}_k}{\|\dot{q}\|}-\frac{\dot{q}_k\,(\dot{q}_j\,\ddot{q}_j)}{\|\dot{q}\|^3}~=~
\frac{\ddot{q}_k}{\|\dot{q}\|}.
\end{equation}
The last equation uses the fact that in the chosen parametrisation
 the tangent vector $\dot{q}$ has constant length along the path
and thus   is perpendicular to $\ddot{q}$.
Further,
\begin{equation}\label{NR3}
\partial_k (g_i\,g_i)^{1/2}~=~\frac{ g_j\,\partial_k g_j}{(g_i\,g_i)^{1/2}}~=~\frac{ g_j\,\partial_k g_j}{\|g\|}.
 \end{equation}
Putting this together results in
\begin{equation}\label{NR4}
\left( \frac{\partial}{\partial q} -  \frac{d}{dt}\frac{\partial}{\partial \dot{q}}\right)\,L(q,\dot{q})
 ~=~
 \|\dot{q}\|\,\partial_k\,\|g\| - \|g\|  \frac{d}{dt} \,\dot{\partial}_k \|\dot{q}\|~=~
 \frac{\|\dot{q}\|}{\|g\|}\, g_j\,\partial_k g_j - \frac{\|g\|}{\|\dot{q}\|}\,\ddot{q}_k.
  \end{equation}
Using the arc length parametrisation where $\|\dot{q}\|=1$,
this finally leads to the Euler-Lagrange equation
\begin{equation}\label{EL-3}
\ddot{q}~=~ \frac12 \,\frac{\nabla\, \|g\|^2}{\|g\|^2}~=~  \nabla\,\log \|g\|.
\end{equation}
