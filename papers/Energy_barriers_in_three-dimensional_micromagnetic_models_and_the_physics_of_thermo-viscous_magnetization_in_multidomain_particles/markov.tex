\subsection{Statistical theory of MD VRM}
The first application of the above presented calculation of energy barriers is a complete
description of low-field viscous magnetization processes in a micromagnetically modelled
cubic particle.
%
%
Knowing the optimal transition paths between all  LEM structures $S_i$
of the investigated particle
allows for calculating the zero field temporal isothermal transition matrix $M(\Delta t)$,
which describes the continuous homogeneous Markov process of random thermally activated transitions
between all possible states:
\begin{equation}\label{transmat}
   M (\Delta t)~\equiv~\mathbb{P}\left[{ S(t)= S_j  \wedge S(t+\Delta t)= S_i }\right]~=~
   \,\exp\left[\mu \, \Delta t \right].
\end{equation}
Here the matrix elements $\mu_{ij}$ of the infinitesimal generator of the semigroup  $M(t)$ are given by
the relative outflow from $S_j$ to $S_i$ for $i\neq j$. The relative inflow from all other
states determines the diagonal element $\mu_{ii}$.
\begin{eqnarray}
% \nonumber to remove numbering (before each equation)
  \mu_{ij} &=& -\frac{\Delta E_{ij}}{k_B T\,\tau_0} ~~\mbox{for}~~i\neq j\\
  \mu_{ii}&=& -\sum \limits_{i\neq j}  \mu_{ji}
\end{eqnarray}
Once, the $ \mu_{ij}$ have been calculated, it is easily possible to determine the viscous decay
of any initial probability distribution
$\rho_i^0~\equiv~\mathbb{P}\left[{ S_0=  S_i }\right]$
by   multiplication with the time evolution matrix exponential
\begin{equation}\label{probdens}
   \rho( t) ~=~\exp\left[\mu \,  t \right]\, \rho_i^0
\end{equation}
Multiplication by the corresponding magnetizations $m_i$ of states $S_i$ yields the
viscous evolution of remanence
\begin{equation}\label{VRMt}
    m(t) ~=~ \sum \limits_i m_i\, \rho_i(t).
\end{equation}
%
When a small field $H$ is applied, the energy barrier
$E_b^{ij}$ in first order  changes according to
\begin{equation}\label{field-barr}
    E_b^{ij}(H)~=~  E_b^{ij}+ (m_j - m_{ij}^{\max})\, H,
\end{equation}
where $m_{ij}^{\max}$ denotes the magnetization  at the  maximum
energy state along the optimal transition path from $S_i$ to $S_j$.
The approximation used to obtain (\ref{field-barr}) assumes that $H$ is so small
that it does not change the magnetization structures of the LEM and saddle-point states noticeably.
Only the field induced energy is taken into account.
It is easily seen that all other energy changes are of second order in $H$.

Using the in-field energy barriers it is straightforward  to determine the
matrix exponential which governs VRM acquisition. By defining
\begin{eqnarray}
% \nonumber to remove numbering (before each equation)
  \mu_{ij}(H) &=& -\frac{E_b^{ij}(H)}{k_B T\,\tau_0} ~~\mbox{for}~~i\neq j\\
  \mu_{ii}(H)&=& -\sum \limits_{i\neq j}  \mu_{ji}(H),
\end{eqnarray} the above zero-field theory automatically  extends to the weak field case.

In   case of our cubic PSD particle, all matrices are of size $60\times60$
and the calculations have been performed by a Mathematica (\copyright Wolfram Research)
program.

\subsection{Viscous remanence acquisition and decay in an ensemble of cubic PSD magnetite}
Using the above mathematical methods it is possible to calculate
the statistics of viscous remanence acquisition and decay for our
single PSD particle
with respect to any field vector of sufficiently small length $H$.
In order to model an isotropic ensemble, it is necessary to average the
VRM properties over all possible field directions. This has been approximated by drawing 20
random directions  from an equi-distribution over the unit sphere and averaging the
modelled VRM acquisition and decay curves.
For room temperature this yielded   the ensemble curve as shown in Fig.~\ref{ViscAcq}.
~\\[3mm]%
\par
\vbox{ \centerline{\hbox{ \psfig{figure=ViscAcq-2.eps,width=120mm}
}} \footnotesize
%  \begin{center}
{\bf Figure \bild{ViscAcq} :}
Modelled acquisition of viscous magnetization in the
cubic particle with $\lambda = 5.0$. The initial state is
an
equi-distribution over all possible LEM states with zero net magnetization.
In a
small external field the first acquisition process is the
immediate decay from V-$110$ type states into V-$100$ type states,
which occurs within about $10^{-9}$s.
Due to the field induced asymmetry of the energy barriers, a remanence is acquired
during this process. The second process is a decay of
F-$111$ type states into V-$100$ type states. This occurs between about $10^2$s and $10^3$s
and shows an intermediate overshooting of remanence.
%  \end{center}
\normalsize }
~\\[0.2cm]%
The left hand side of Fig.~\ref{ViscAcq} shows the remanence acquisition in a modelled
field of $H=60\mu$T when starting from an initial state $\rho_0$ at $t=0$ which assigns equal probability
to all existing LEM states.
Already within $10^{-9}$s the remanence increases rapidly due to the immediate
depletion of the nearly unstable V-$110$ vortex states which decay into the
  stable
V-$100$ states (see Table~\ref{trans-barr-2}).
The remanence forms because in zero field there are two equally probable
transitions, e.g. V-$110~\rightarrow$~V-$100$ and V-$110~\rightarrow$~V-$010$.
Within the external field one of these decay paths becomes more probable which leads to
a relative overpopulation of the field aligned V-$100$ type states.
Nearly synchronously there occurs a two step process
V-$111~\rightarrow$~V-$110~\rightarrow$~V-$100$.It
is controlled by the somewhat slower transition V-$111~\rightarrow$~V-$110$, but still
both take place within the first few $10^{-9}$s.
The last VRM acquisition processes occurs only after a much longer waiting time of $10-10^3$s.
First the initial F-$111$ type states transform via F-$110$ type states into V-$110$ type states
which then immediately decay into V-$100$
(Fig.~\ref{state-scheme}).
This last process  produces an astonishing remanence overshooting as displayed in
Fig.~\ref{ViscAcq}: The remanence during the VRM acquisition process is for a certain time
higher than the finally obtained equilibrium VRM.
In the next section we will show that is is not an artifact of the modelling,
but can be explained by a  real physical process.

The right hand side of  Fig.~\ref{ViscAcq} shows that
when the field is switched off after VRM acquisition, the obtained remanence is carried
only by   extremely stable  V-$100$-type states which require a   theoretical
waiting time of $10^{15}$s to equilibrate into a zero remanence state. 