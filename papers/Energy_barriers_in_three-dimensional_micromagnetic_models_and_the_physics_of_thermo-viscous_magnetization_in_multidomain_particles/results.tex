\subsection{Material constants}
The numerical calculations have been performed in terms of
the  reduced material parameters $Q$ and $\lambda$ \cite{Hubert:98M}.
The magnetic hardness $Q$, in the case of cubic magneto-crystalline anisotropy, is the numerical ratio of
 $Q=K_1/K_d$. Here $K_d$ is the characteristic magnetostatic self-energy, which in terms of the
 saturation magnetization $M_s$ is defined as
 $K_d= 1/2 \mu_0 M_s^2$. The exchange length $\lambda= \sqrt{A/K_d}$ determines the characteristic length scale
 above which magnetostatic self-energy is able to overcome exchange coupling, represented
 by the exchange constant $A$.
For magnetite these
material constants are $M_s=480$~kA/m, $A=1.32\cdot10^{-11}$~J/m, and
$K_1=-1.25\cdot10^4$~J/m$^3$.
Accordingly,
$K_d=145$~kJ/m$^3$,  $\lambda_{ex}=9.55$~nm and
magnetic hardness is $Q=-0.0863$. Unless otherwise stated, all results in the following
subsections are obtained for the magnetite-like value $Q=-0.1$. Lengths $\lambda$ are given
in units of $\lambda_{ex}$.
\subsection{Potential barriers in cubic particles}
~\\[3mm]%
\par
\vbox{ \centerline{\hbox{%
\begin{tabular}{ccccc}
  \hline
  $\lambda$&Initial state& Initial energy $E$ & Intermediate state& Barrier $\eps$\\
   \hline
  \hline
 4.0 & F-111& 0.286404 & F-110 & 0.007825 \\
 4.5 & F-111& 0.28308 & V-011 & 0.005417 \\
 4.5 & F-111& 0.28308 & F-011 & 0.008172 \\
 4.5 & F-111& 0.28308 & V-001 & 0.011041 \\
  \hline
\end{tabular}
}} \footnotesize
%  \begin{center}
{\bf Table \tabelle{trans-barr-1} :}
Energy barriers for numerically found optimal transitions between
listed initial and final states for $Q=-0.1$ and
different values of $\lambda$.
%  \end{center}
\normalsize }
~\\[0.2cm]%


If to compare the barriers   with the SD case, where $\eps    = 1/120 = 0.008333$,
the actual barriers are less, especially for $\lambda =4.5$.
But it must be kept in mind that for $\lambda =4.0$ the transition is quasi-SD as the
saddle point is F-110 type and almost homogeneous. For  $\lambda =4.5$, however,
the minimum energy saddle point is V-110 type,
and development of vortex evidently considerably reduces the barrier.
~\\[3mm]%
\par
\vbox{ \centerline{\hbox{ \psfig{figure=EBarr-lam4-F111-F111.eps,width=140mm}
}} \footnotesize
%  \begin{center}
{\bf Figure \bild{lam4-barr} :}
Energy variation across the optimal transition from a F-111 flower state to a
F-\={1}11 vortex state at $\lambda = 4.0$. Each circle represents an intermediate
 magnetization state used for the calculation. The dashed line corresponds to the
 absolute value of the energy gradient along the transition path. The maximum energy state along the
 path is the F-011 flower state. The three-dimensional micromagnetic calculation ($\eps_{FF}$)leads only to
 a minor decrease of the energy barrier with respect to coherent rotation ($\eps_{coherent}$).
This improvement results from the small spin deflections close to the particle surface.
%  \end{center}
\normalsize }
~\\[0.2cm]%

For  $\lambda =5.0$ the situation is most complex as there are a
number of competitive LEM-states  with similar energy.
V001 with E = 0.27079, V110 with E = 0. 276931, V111 with E = 0.277701,
F111 with E = 0.279527.
As the result, there are no direct jumps between topologically identical
states like V-001 and V-010.
Such transitions were the only ones  observed for lambda $\lambda =4.0$ and $\lambda =4.5$.
For $\lambda =5.0$  the state V-001 can change
  to V-010 only indirectly, either through V-111, or V-011, or F-111.
 Thus, the diagonal elements of the matrix below are empty.
~\\[3mm]%
\par
\vbox{ \centerline{\hbox{%
\begin{tabular}{ccccc}
  \hline
  &V-001&V-110&V-111&F-111\\
  \hline\hline
  V-001&0&0.000123&0.001389 & 0.00349\\
 V-110&0.006263&0&0.000598& 0.002925\\
 V-111 &   0.008297 & 0.001368&0   &     0.003836\\
 F-111  &  0.012117   & 0.005521   & 0.005665  &0\\
  \hline
\end{tabular}
}} \footnotesize
%  \begin{center}
{\bf Table \tabelle{trans-barr-2} :}
Energy barriers for optimal transitions between
a reduced set of initial and final states for $Q=-0.1$ and
  $\lambda=5.0$.
  Transitions with the same  initial state are listed in the same column.
Transitions with the same  final state are listed in the same line.
The corresponding complete set of 60 LEM states is obtained by taking into
account cubic symmetry leading to 8 states of class F-111 or V-111,
6 states of class V-001, and 12 states of class V-110.
In addition, all vortex states exist in two varieties of different helicity.
%  \end{center}
\normalsize }
~\\[0.2cm]%



~\\[3mm]%
\par
\vbox{ \centerline{\hbox{ \psfig{figure=EBarr-lam5-F111-V111.eps,width=140mm}
}} \footnotesize
%  \begin{center}
{\bf Figure \bild{lam5-barr} :}
Energy variation across the optimal transition from a F-111 flower state to a
V-111 vortex state at $\lambda = 5.0$. Each circle represents an intermediate
 magnetization state used for the calculation. The dashed line corresponds to the
 absolute value of the energy gradient along the transition path. Because the energies of
 F-111 and V-111 are different, also the energy barriers for a transition from
 F-111 to V-111 ($\eps_{FV}$) and from V-111 to F-111 ($\eps_{VF}$) differ.
%  \end{center}
\normalsize }
~\\[0.2cm]%

For  $\lambda \geq 6.0$ the situation becomes simple again as now the only stable states are
of type V-001.
~\\[3mm]%
\par
\vbox{ \centerline{\hbox{%
\begin{tabular}{ccccc}
  \hline
  $\lambda$&Initial state&   Intermediate state& Barrier $\eps$\\
   \hline
  \hline
 6.0 & V-100&   V-111 & 0.009486  \\
 7.0 & V-100&  V-111 & 0.006372 \\
 8.0 & V-100&   V-111 & 0.005876  \\
  \hline
\end{tabular}
}} \footnotesize
%  \begin{center}
{\bf Table \tabelle{trans-barr-3} :}
Energy barriers for optimal transitions between
vortex states for $Q=-0.1$ and
  $\lambda=6.0-8.0$.
While V-111 is a marginally stable LEM at $\lambda=6.0$,
it is unstable for $\lambda=7.0,8.0$.
Types of V-100 vortex states are the global energy minima at these grain sizes
and the energy barrier refers to a symmetric transition between
two adjacent states of this type, e.g. V-100 to V-010.
%  \end{center}
\normalsize }
~\\[0.2cm]%


~\\[3mm]%
\par
\vbox{ \centerline{\hbox{ \psfig{figure=EBarr-lam6-V111-V100.eps,width=140mm}
}} \footnotesize
%  \begin{center}
{\bf Figure \bild{lam6-barr} :}
Energy variation across the  transition from a V-111 flower state to a
V-100 vortex state at $\lambda = 6.0$.
The top figures show the magnetization structure of the V-111 state as seen along the $\langle111\rangle$-direction
and the V-100 vortex as seen along the $\langle100\rangle$-direction.
In the bottom diagram again each circle represents an intermediate
 magnetization state used for the calculation. The dashed line corresponds to the
 absolute value of the energy gradient along the transition path.
 The tiny energy barrier $\eps_{V111-V100}$ in relation to $\eps_{V100-V111}$
  indicates that the V-111 state is a very unstable LEM
 as compared to V-100. However, it is important for the transition between
 the more stable vortex states. E.g. the optimal transition
 from  V-001 to V-100 is a
 combination of the symmetric vortex rotations V-001 to V-111, and the shown
 transition  V111 to V100.
%  \end{center}
\normalsize }
~\\[0.2cm]%

Consider now the thermo-activation barrier $E_b = \eps \,\lambda^3$
for the most stable LEM states:
  F-111 for   $\lambda \leq 4.5$, and V-001 for  $\lambda \geq 5.0$.
~\\[3mm]%
\par
\vbox{ \centerline{\hbox{%
 \begin{tabular}{cccc}
    \hline
    $\lambda$& $\eps$ & $E_b$& $m$\\
      \hline
  \hline
4.0&  0.007825  &  0.5008 & 0.991297\\
4.5& 0.005417  &  0.49362& 0.986527\\
5.0  & 0.006263  &  0.782875  &  0.705928\\
6.0  & 0.009486 &   2.048976 &   0.509672\\
7.0  & 0.006372  &  2.185596 &   0.310451\\
8.0  & 0.005876 &    3.008517 &\\
    \hline
  \end{tabular}
}} \footnotesize
%  \begin{center}
{\bf Table \tabelle{trans-barr-4} :}
Minimal energy density $\eps$ or
absolute energy $E_b$ necessary
to leave the global energy minima for different
values of $\lambda$.
$m$ is the reduced magnetization at the global minimum.
%  \end{center}
\normalsize }
~\\[0.2cm]%
~\\[3mm]%
\par
\vbox{ \centerline{\hbox{ \psfig{figure=State-Scheme-2.eps,width=120mm}
}} \footnotesize
%  \begin{center}
{\bf Figure \bild{state-scheme} :}
Schematic representation of transition paths in magnetization space for a
cubic particle with $\lambda = 5.0$.
Each sphere corresponds to the magnetization of an LEM state. The cubic structure
reflects the cubic symmetry of the particle.
Grey arrows indicate two of the many possible transition paths (or decay modes):
(1) A direct decay form F-$\bar{1}\bar{1}\bar{1}$ to V-$\bar{1}\bar{1}\bar{1}$;
(2) Indirect decay from F-$\bar{1}\bar{1}\bar{1}$ over an instable intermediate
F-$\bar{1}0\bar{1}$ state into V-$\bar{1}0\bar{1}$ and
further to either V-$\bar{1}00 or V-00\bar{1}$.
Note, that each sphere in principle represents two vortex states of
inverse helicity (R and L). However, a transitions between
vortex states of different helicity
have large energy barriers and can be neglected.
%  \end{center}
\normalsize }
~\\[0.2cm]%


%
%
%Statelist:\\
%\footnotesize
%F$\bar{1}\bar{1}\bar{1}$, F$\bar{1}\bar{1}1$, F$\bar{1}1\bar{1}$,
%F$\bar{1}11$, F$1\bar{1}\bar{1}$, F$1\bar{1}1$, F$11\bar{1}$, F$111$,\\[3mm]
%V$_{-}00\bar{1}$, V$_{+}00\bar{1}$, V$_{-}001$, V$_{+}001$, V$_{-}0\bar{1}0$,
%V$_{+}0\bar{1}0$, V$_{-}010$, V$_{+}010$,\\[3mm]
% V$_{-}0\bar{1}\bar{1}$,
%V$_{+}0\bar{1}\bar{1}$, V$_{-}0\bar{1}1$, V$_{+}0\bar{1}1$, V$_{-}01\bar{1}$,
%V$_{+}01\bar{1}$, V$_{-}011$, V$_{+}011$,\\[3mm]
% V$_{-}\bar{1}00$, V$_{+}\bar{1}00$,
%V$_{-}100$, V$_{+}100$,\\[3mm]
% V$_{-}\bar{1}0\bar{1}$, V$_{+}\bar{1}0\bar{1}$,
%V$_{-}\bar{1}01$, V$_{+}\bar{1}01$, V$_{-}10\bar{1}$, V$_{+}10\bar{1}$,
%V$_{-}101$, V$_{+}101$,\\
% V$_{-}\bar{1}\bar{1}0$, V$_{+}\bar{1}\bar{1}0$,
%V$_{-}\bar{1}10$, V$_{+}\bar{1}10$, V$_{-}1\bar{1}0$, V$_{+}1\bar{1}0$,
%V$_{-}110$, V$_{+}110$,\\[3mm]
% V$_{-}\bar{1}\bar{1}\bar{1}$,
%V$_{+}\bar{1}\bar{1}\bar{1}$, V$_{-}\bar{1}\bar{1}1$, V$_{+}\bar{1}\bar{1}1$,
%V$_{-}\bar{1}1\bar{1}$, V$_{+}\bar{1}1\bar{1}$, V$_{-}\bar{1}11$,
%V$_{+}\bar{1}11$,\\
% V$_{-}1\bar{1}\bar{1}$, V$_{+}1\bar{1}\bar{1}$,
%V$_{-}1\bar{1}1$, V$_{+}1\bar{1}1$, V$_{-}11\bar{1}$, V$_{+}11\bar{1}$,
%V$_{-}111$, V$_{+}111$\\
%\normalsize 