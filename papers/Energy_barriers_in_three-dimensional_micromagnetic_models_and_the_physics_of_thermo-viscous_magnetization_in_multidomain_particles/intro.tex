\subsection{Aim and outline of the article}

The geomagnetic field has been perpetually recorded by magnetic remanence
carriers in newly formed rocks throughout the Earth's history.
%
Therefore,
crustal rocks form a paleomagnetic archive which is accessible  through
 rock magnetic measurements.
 Yet, their interpretation requires a thorough understanding of
the physical processes occuring during  remanence acquisition.
%
Thermo-viscous magnetization of natural pseudo-single or multidomain
particles is   the most abundant
remanence  in paleomagnetism, although
more reliable single domain (SD) remanence carriers are preferred, and theoretical interpretation
is based on paradigms developed from SD theory.
The main aim of this article is to propose new computational and conceptional
methods to obtain a physical understanding of the remanence acquisition in
multidomain particles.
The second section will introduce a new technique to determine energy barriers in
micromagnetic models of inhomogenously magnetized particles.
Although this is technically challenging, it is an essential prerequisite
for a quantitative study of thermo-viscous magnetization.
In the third section we use the new computational method to calculate all energy barriers for a
three-dimensional model of metastable flower and vortex-states in a cubic magnetite particle.
The fourth section introduces the theoretical background for a statistical analysis  of viscous
remanence acquisition and decay in weak fields, based on the computed
energy barriers between the metastable states.
In the fifth section this theory is applied to the energy barriers calculated in section~3, and
the physical meaning of the model results obtained is discussed.

\subsection{Micromagnetic modeling}
Micromagnetic modeling is now a standard technique to determine
stable and metastable magnetization states in small ferro- and ferrimagnetic particles. It is used to calculate and analyze  magnetization structures
          in natural and synthetic magnetic nanoparticles.
          This size range is of special interest in rock magnetism
    where the reconstruction of the Earth magnetic field depends critically
    on the reliability of remanence carriers in natural rocks.
    The grain size distribution of these remanence carriers
    rarely is confined to the relatively small SD-size range.
     Accordingly, detailed knowledge of the
    physical mechanism of magnetization change in larger nanoparticles is
    needed to assess and evaluate the magnetic measurement results from natural
    materials.
Because of its importance for understanding remanence acquisition in natural rocks, rock magnetic studies were among the first to apply numerical micromagnetic models.
The first approach to estimate barriers between single-domain and two-domain states
used a one-dimensional model of magnetization change \cite{Enkin:88}.
When three-dimensional models were developed
to understand inhomogenous remanence states \cite{Williams:89}, it was immediately a main interest
to obtain energy barriers to model the acquisition of thermoremanence \cite{Enkin:94,Thomson:94,Winklhofer:97,Muxworthy:03}.
Knowing the energy barriers between
different magnetization states within a single particle
also leads to a  quantitative prediction  of magnetic viscosity and
magnetic stability of remanence information, even over geological time scales.
%
An important result of early micromagnetic calculations was that beyond the regime where exchange
    forces
    dominate, i.e. beyond length scales of several exchange lengths $\sqrt{A/K_d}$,
    there exist a multitude of local energy minima corresponding to meta-stable magnetization
    structures \cite{Williams:89,Fabian:96,Rave:98}.
In the context of thermo-viscous remanence, the most important property of meta-stable magnetization
    structures $m$ is their
    residence time $\tau(m)$. It denotes the expectation value of the
    time during which the system remains in
    state $m$, if it initially is in this state at time $t=0$.
The residence time $\tau(m)$ directly depends on the transition probabilities
    $p(m,m')$ between $m$ and all other LEM $m'$, which in turn depend upon the
    possible transition pathways.
To determine the transition probability $p(m,m')$ in very good approximation,
    it is sufficient to find the   most likely transition path between $m$ and  $m'$,
    which is the path with  the lowest energy barrier.
    This path  runs across the saddle-point with lowest energy of all
    which connect  $m$ and $m'$.
Therefore, the problem of finding the
    transition probabilities $p(m,m')$ is closely related to finding saddle-points in high-dimensional
    micromagnetic energy-landscapes.
    
\subsection{Statistical theory}
Given these transition probabilities, the geologically important mechanism of
thermoremanence acquisition can be described as a
 stochastic process of
magnetization change in a temperature dependent energy landscape.
Its transition matrix is related to $p(m,m')$, determined by the energy barriers between the possible
states \cite{Fabian:03b}.

