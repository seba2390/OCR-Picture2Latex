
\subsection{Definitions}
A micromagnetic structure $m$ is determined by
$K$ magnetization vectors on a spacial grid over the particle.
Each magnetization vector is a unit vector determined by two
polar angles $\theta, \phi$.
The distance $d(m_1,m_2)$ between two magnetization structures is defined by
\begin{equation}\label{dist}
   d(m_1,m_2)~:=~ \left[ \frac{1}{V} \, \int \limits_V \arccos^2 ( m_1(r) \cdot m_2(r) ) \,dV\right]^{1/2}.
\end{equation}
The local direction vector from $m_1$ to $m_2$ is the gradient
\begin{equation}\label{dir}
  v(m_1,m_2)~:=~ -\nabla  d(\,.\,,m_2) (m_1).
\end{equation}

If two magnetization structures $m_0$ and $m_1$ contain no
opposite directions, i.e. for all $r\in V$ we have $ m_0(r) \neq -m_1(r)$,
then it is possible to linearly interpolate between $m_0$ and $m_1$ by
defining $ m_t(m_0,m_1) (r) $ as the intermediate vector
on the smaller  great circle segment connecting $m_0(r)$ and $m_1(r)$ which
has angular distance $t \arccos(m_0(r) \cdot m_1(r))$ from $m_0(r)$.

By minimization of $E(m)$
using gradient information $\nabla E (m)$
an initial minimum $m_A$ and a final minimum $m_B$
are found.


\subsection{Outline of the relaxation procedure}
In summary, the results of the previous section show that the transition probability
between $m_A$ and   $m_B$ is in very good approximation determined by the 
minimal energy barrier between them.
This barrier is achieved along some optimal geometrical least action path 
$m(s)$ with $m(0)~=~m_A$ and $m(1)~=~m_B$ which also represents the most 
likely transition path.
The state of maximal energy along this path is a saddle point of the 
total energy function $E(m)$, and the least action path is  everywhere parallel to 
$\nabla E(m)$.
%
To find this optimal transition path, we propose a relaxation method 
which combines the advantages of the NEB technique of \cite{Dittrich:02}
with the additional action minimization of \cite{Berkov:98b}. 


Two techniques are required for finding the optimal path by means of iterative 
relaxation: 
\begin{enumerate}
  \item  An updating scheme which determines an improved transition path $m^{k+1}(s)$
  from a previous path $m^k(s)$.
  \item A method of finding an initial transition path $m^0(s)$ within the basin of attraction of 
   the optimal path.
\end{enumerate}

The here proposed updating scheme starts from an
initial path $m^0(s)$ which is 
determined  by interpolating  $N$ intermediate states
$m^0(s_j)$ which correspond to the magnetization structures at the 
$s$-coordinates $0=s_1<s_2<\ldots<s_N=1$ for $j=1,\ldots, N$.
For the interpolation to be well-defined,  $N$ is required to be large enough 
to ensure that neighboring structures $m^k(s_j)$ and $m^k(s_{j+1})$ never contain opposite 
magnetization vectors. 

Similar to the NEB method, in step $k$ the path $m^k(s)$ is changed according to
\begin{equation}\label{relax}
    m^{k+1}(s) ~:=~  m^{k}(s) - \alpha_k \, \left[ \nabla E(m^k(s))-
          \left( \nabla E(m^k(s)) \cdot t^k(s)  \right)\,t^k(s)\right],
\end{equation}
where $t^k(s)$ is the tangent vector to the path $m^k(s)$  at $s$ and
$\alpha_k>0$ is a real number.
This updating scheme moves the path downward along the part of the energy gradient which
is perpendicular to path itself. This algorithm 
converges to a path which is almost everywhere 
parallel to $\nabla E$.
However,  (\ref{relax}) describes  not a true gradient 
descent for the action, and the final path may not achieve minimal action
  due to the formation of kinks during the minimization 
\cite{Henkelman:00a,Henkelman:00b,Dittrich:02}.

The here proposed method differs from previous NEB techniques  
in two details:
First,  
$\alpha_k$ is chosen such that $S( m^{k+1})< S( m^{k})$. This ensures that 
the action decreases in each step.
Second, $\alpha_k$ is dynamically adapted to achieve rapid convergence.
The following procedure to choose  $\alpha_k$  
fulfills both aims.
 
\subsubsection*{Adaption of $\alpha_k$}
Starting with the  initial value $\alpha_0=1$,
it is evaluated after each step whether the action of the updated path is decreased, i.e.
$ S( m^{k+1}(s)) < S( m^{k}(s))$. 
%
While this is true, the new value
$\alpha_{k+1}=\alpha_k$ is kept constant, but only for at most five steps. 
In this phase (\ref{relax}) performs a  quasi-gradient descent 'creeping' towards 
the optimal path. 

Afterwards,
if $S$ still decreases, we set $\alpha_{k+1}=2\,\alpha_k$ 
in each following step until
some $ S( m^{k+1}(s)) > S( m^{k}(s))$.
This phase can be interpreted as an 'accelerated steepest descent'.

If at any time $ S( m^{k+1}(s)) > S( m^{k}(s))$, 
the path $ m^{k+1}(s)$ is rejected and (\ref{relax})
is evaluated for 
a new  $\alpha_{k+1}$-value  of
$\alpha_{k+1}=1/4\,\alpha_k$.
This behavior avoids 'overshooting' of the gradient descent and 


All our tests show that this iterative adaption of $\alpha_k$ 
leads to a  much faster convergence than
choosing any fix value $\alpha_k=\alpha$.
 
By comparing the achieved action $S( m^{k}(s))$ to $ S_{\rm min}$ from
(\ref{TP-4})
during the relaxation, it is possible to detect  the 
formation of kinks and to decide when the minimization succeeds.
This action criterion is better than testing whether 
the final path is parallel to $\nabla E$, since
the latter is also true for paths with kinks.
 
\subsection{Determination of the initial path}
Since the relaxation scheme works similar to a gradient
minimization algorithm, it adjusts to the next local optimum of the
action function.
Therefore the choice of the initial path is crucial for obtaining the globally
optimal path.
Here we propose a  method to find a good initial path $m^0(s)$ by
a sequence of minimizations of modified energy functions.

For parameters $\mu, \beta, \eps$ and $\Delta$ we consider the
modified energy function
\begin{equation}\label{modEn}
   E^\ast_\Delta(m) ~=~
         E(m) + \mu \,[ d(m, m_A) -\Delta]^2 + \frac{\beta}{d(m,m_B)+\eps}.
\end{equation}
The parameters  $\mu, \beta, \eps$ are chosen such that
for $\Delta=d(m_B, m_A) $ the final state $m_B$ is a unique
optimum of $E^\ast_\Delta(m)$, while for $m$ with  $d(m , m_A) \ll d(m_B, m_A)$ the last term of
$E^\ast_\Delta(m)$ is small in comparison to $E(m)$.
The value of $\mu$ should be large enough to ensure that the distance between
$m_A$ and the minimum of
$E^\ast_\Delta$ is indeed close to $\Delta$.

Now a sequence $m^0_j,~j=1,\ldots, J$ of magnetization structures is iteratively
obtained by setting $m^0_0~=~m_A$ and
$m^0_{j}$ to the result of minimizing $E^\ast_{\Delta_j} (m)$, where
$\Delta_j~=~j/J\,d(m_B, m_A)$.

Interpolating between the  structures  $m^0_j$ determines an
initial path  which (1) starts in $m_A$ and ends in $m_B$,
(2) has relatively equally spaced
intermediate states $m^0_{j}$, and (3) prefers
 intermediate states $m^0_{j}$ at distance $\Delta_j$ with low energy $E(m)$.
