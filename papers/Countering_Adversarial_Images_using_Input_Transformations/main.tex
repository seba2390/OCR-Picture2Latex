\documentclass{article} % For LaTeX2e
\usepackage{iclr2018_conference,times}
\usepackage{hyperref}
\usepackage{url}
\usepackage{epsfig}
\usepackage{graphicx}
\usepackage{subcaption}

% Custom stuff
\usepackage{amsmath,amsthm,verbatim,amssymb,amsfonts,amscd,mathrsfs}
\usepackage[inline]{enumitem}
\usepackage{color}
\usepackage{wrapfig}
\usepackage{booktabs}
\usepackage{dsfont}
\usepackage{dirtytalk}
% --------------------------------------------------------------------
\usepackage[only,llbracket,rrbracket]{stmaryrd}
\SetSymbolFont{stmry}{bold}{U}{stmry}{m}{n}

\usepackage{amsmath,amsthm,amssymb,mathtools,nicefrac}
\usepackage[bb=boondox]{mathalfa}
\usepackage{xspace}
\usepackage{enumitem}
\usepackage{bbding}
\usepackage[sort&compress]{natbib}

\makeatletter
\def\NAT@spacechar{~}% NEW
\makeatother

% --------------------------------------------------------------------
\newcommand{\notation}[4][0]{\newcommand{#2}[#1]{#3}}

\newcommand{\rname}[1]{[\textsc{#1}]}

% --------------------------------------------------------------------
% TODOs

\usepackage{todonotes}
\usepackage{setspace}

\setlength{\marginparwidth}{3cm}

\newcounter{todocnt}
\newcommand{\todox}[3][]{%
\refstepcounter{todocnt}{%
\setstretch{0.7}%
\todo[color={red!100!green!33},size=\small,#1]{%
  \textbf{[\uppercase{#2}\thetodocnt]:}~#3}}}

\newcommand{\gb}[2][]{\todox[#1]{GB}{#2}}
\newcommand{\py}[2][]{\todox[#1]{PY}{#2}}
\newcommand{\jh}[2][]{\todox[#1]{JH}{#2}}
\newcommand{\te}[2][]{\todox[#1]{TE}{#2}}

% --------------------------------------------------------------------
% English

\def\ie{i.e.\xspace}
\def\eg{e.g.\xspace}

% --------------------------------------------------------------------
% Local labels

\newcounter{localc}

\newcommand*\lclabel[1]{\label{exer:\thelocalc:#1}}
\newcommand*\lcref[1]{\ref{exer:\thelocalc:#1}}
\newcommand*\lceqref[1]{\eqref{exer:\thelocalc:#1}}

\newcommand{\steplocal}{\stepcounter{localc}}

% --------------------------------------------------------------------
% Standard symbols

\newcommand{\eqdef}{\mathrel{\stackrel{\scriptscriptstyle \triangle}{=}}}
\newcommand{\iffdef}{\mathrel{\stackrel{\scriptscriptstyle \triangle}{\iff}}}

\newcommand{\inv}[1]{#1^{\raisebox{.2ex}{$\scriptscriptstyle-\!1$}}}

\newcommand{\proj}[1]{\pi_{#1}}
\newcommand{\fst}{\proj{1}}
\newcommand{\snd}{\proj{2}}

\newcommand{\rR}{\mathrel{\mathcal{R}}}
\newcommand{\rS}{\mathrel{\mathcal{S}}}

\newcommand{\oR}{\mathcal{R}}
\newcommand{\oS}{\mathcal{S}}

\newcommand{\simplies}{\mathrel{\Rightarrow}}

\newcommand{\true}{{\mathrel{\top}}}
\newcommand{\false}{{\mathrel{\perp}}}

\newcommand{\fun}[1]{\lambda\,{#1} .\,}
\newcommand{\card}[1]{{|{#1}|}}
\newcommand{\rcomp}[1]{\overline{#1}}

\newcommand{\relrestr}[3]{\mathrel{{#3}_{|_{{#1} \!\times\! {#2}}}}}

\newcommand{\qt}[2]{{#1}_{/_{\!#2}}}
\newcommand{\qtc}[2]{{[#1]}_{#2}}

\newcommand{\rrefl}[1]{\mathrel{#1^=}}
\newcommand{\rsym}[1]{\mathrel{\inv{#1}}}

\def\ssrc{\mathop{\top}}
\def\sdst{\mathop{\perp}}

\newcommand{\cset}[1]{\mathcal{E}(#1)}

% --------------------------------------------------------------------
% \bigtimes
\DeclareFontFamily{U}{mathx}{\hyphenchar\font45}
\DeclareFontShape{U}{mathx}{m}{n}{
      <5> <6> <7> <8> <9> <10>
      <10.95> <12> <14.4> <17.28> <20.74> <24.88>
      mathx10
      }{}
\DeclareSymbolFont{mathx}{U}{mathx}{m}{n}
\DeclareMathSymbol{\bigtimes}{1}{mathx}{"91}

% --------------------------------------------------------------------
% Standard sets

\newcommand{\rset}[1]{\ensuremath{\mathbb{#1}}}

\newcommand{\BB}{{\{0, 1\}}}
\newcommand{\NN}{\rset{N}}
\newcommand{\ZZ}{\rset{Z}}
\newcommand{\QQ}{\rset{Q}}
\newcommand{\RR}{\rset{R}}
\newcommand{\RRP}{\rset{R}^+}

\renewcommand{\setminus}{\mathrel{-}}

\newcommand{\I}[1]{\mathcal{I}_{#1}}

\newcommand{\intv}[2]{{[{#1}, {#2}]}}
\newcommand{\indicator}[1]{\raisebox{.25ex}{$\chi_{#1}$}}

% ---------------------------------------------------------------------
% Distributions

\newcommand{\DistOp}{{\mathbb{D}}}
\newcommand{\FDistOp}{{\DistOp^{\scriptscriptstyle =1}}}
\newcommand{\Exp}{\mathbb{E}}
\newcommand{\ExpD}{\mathbb{E}}
\newcommand{\PrS}{\mathbb{P}}

\newcommand{\Dist}{\DistOp}
\newcommand{\FDist}{\FDistOp}

\newcommand{\dnull}[1][]{{{\mathbb{0}}^{#1}}}
\newcommand{\dunit}[2][]{{{\mathbb{1}}^{#1}_{#2}}}
\newcommand{\dnormed}[1]{{#1}^{=1}}
\newcommand{\dlet}[3]{\ExpD_{{#1} \sim {#2}} [{#3}]}
\newcommand{\dslet}[2]{\ExpD_{#1} [{#2}]}
\newcommand{\dclet}[4]{\dlet {#1} {\drestr {#2} {#3}} {#4}}
\newcommand{\dsclet}[3]{\dslet {\drestr {#1} {#2}} {#3}}
\newcommand{\dmargin}[2]{{#1} \circ \inv{#2}}
\newcommand{\dlift}[1]{#1^{\sharp}}
\newcommand{\mlift}[1]{{\overline{#1}}} %FIXME
\newcommand{\dproj}[1]{\dlift{\proj{#1}}}
\newcommand{\dfst}{\dproj{1}}
\newcommand{\dsnd}{\dproj{2}}
\newcommand{\drestr}[2]{{#1}_{|{#2}}}
\newcommand{\rdrestr}[2]{\drestr {#2} {#1}}
\newcommand{\dlim}[2]{\lim_{{#1}\infty}\,#2}
\newcommand{\dprod}{\mathbin{\star}}
\newcommand{\dprodeq}{\mathbin{\overline{\star}}}
\newcommand{\djoin}[2]{{#1} \mathbin{\Join} {#2}}
\newcommand{\dswap}[1]{#1^{\leftrightarrow}}
\newcommand{\dleq}{\mathrel{\preceq}}
\newcommand{\dgeq}{\mathrel{\succeq}}
\newcommand{\dlt}{\mathrel{\prec}}
\newcommand{\dgt}{\mathrel{\succ}}
\newcommand{\lap}[1]{\mathcal{L}_{#1}}

\newcommand{\E}[3]{\Exp_{{#1} \sim {#2}} [{#3}]}
\newcommand{\cE}[4]{\Exp_{{#1} \sim {#2}} [{#3} \mid {#4}]}
\newcommand{\sE}[2]{\Exp_{{#1}} [{#2}]}

\renewcommand{\P}[3]{\PrS_{{#1} \sim {#2}} [{#3}]}
\newcommand{\sP}[2]{\PrS_{#1} [{#2}]}
\newcommand{\iP}[1]{\PrS [{#1}]}

\newcommand{\mass}[1]{|{#1}|}
\DeclareMathOperator{\supp}{supp}

\newcommand{\wtn}[2]{\langle {#1, #2} \rangle}

\newcommand{\dcoupled}[3]
  {{#1} \mathrel{\blacktriangleleft} \langle {#2} \mathrel{\&} {#3} \rangle}

\newcommand{\dlifted}[4]
  {{#1} \mathrel{\blacktriangleleft_{#2}} \langle {#3} \mathrel{\&} {#4} \rangle}

\newcommand{\dacoupled}[6]
  {\langle {#1, #2} \rangle \mathrel{\blacktriangleleft}_{#3,#4}
     \langle {#5} \mathrel{\&} {#6} \rangle}

\newcommand{\dalifted}[7]
  {\langle {#1, #2} \rangle \mathrel{\blacktriangleleft_{#3,#4}^{#5}}
     \langle {#6} \mathrel{\&} {#7} \rangle}

% --------------------------------------------------------------------
% Assertions

\newcommand{\pre}{\phi}
\newcommand{\post}{\psi}

% --------------------------------------------------------------------
% Sid'ed

\def\ms{\mspace{-1.5mu}}

\newcommand{\lmark}{{\scriptscriptstyle \vartriangleleft}}
\newcommand{\rmark}{{\scriptscriptstyle \vartriangleright}}
\newcommand{\pmark}{{\scriptscriptstyle \Join}}

\newcommand{\lside}{_{\ms\lmark}}
\newcommand{\rside}{_{\ms\rmark}}

\newcommand{\LPVars}[1][]{\PVars[#1]^{\ms\lmark}}
\newcommand{\RPVars}[1][]{\PVars[#1]^{\ms\rmark}}
\newcommand{\PPVars}[1][]{\PVars[#1]^{\ms\pmark}}

\newcommand{\LSExpr}[1][]{\SExpr[#1]^{\ms\lmark}}
\newcommand{\RSExpr}[1][]{\SExpr[#1]^{\ms\rmark}}
\newcommand{\PSExpr}[1][]{\SExpr[#1]^{\ms\pmark}}

% --------------------------------------------------------------------
% Semantics

\newcommand{\sem}[1]{{\llbracket {#1} \rrbracket}}
\newcommand{\semop}[1]{{\overline{#1}}}

% --------------------------------------------------------------------
% Logics

\newcommand{\lift}[1]{\mathrel{#1^\sharp}}
\newcommand{\alifttoplas}[2]{\mathrel{#1^{(1)}_{#2}}}
\newcommand{\alifticalp}[2]{\mathrel{#1^{(2)}_{#2}}}
\newcommand{\aliftnew}[2]{\mathrel{#1^{(\star)}_{#2}}}

\newcommand{\symlifttoplas}[2]{\mathrel{\overline{#1}^{(1)}_{#2}}}
\newcommand{\symliftnew}[2]{\mathrel{\overline{#1}^{(\star)}_{#2}}}

% --------------------------------------------------------------------
% Refs

% \iflipics
% \theoremstyle{plain}
% \newtheorem{conj}[theorem]{Conjecture}
% \theoremstyle{definition}
% \newtheorem{defprop}[theorem]{Defn.-Lemma}
% \newtheorem{nottion}[theorem]{Notation}
% \else
% \theoremstyle{plain}
% \newtheorem{theorem}{Theorem}
% \newtheorem{lemma}[theorem]{Lemma}
% \newtheorem{corollary}[theorem]{Corollary}
% \newtheorem{conj}[theorem]{Conjecture}
% \theoremstyle{definition}
% \newtheorem{definition}[theorem]{Definition}
% \newtheorem{defprop}[theorem]{Defn.-Lemma}
% \newtheorem{example}[theorem]{Example}
% \newtheorem{nottion}[theorem]{Notation}
% \theoremstyle{remark}
% \newtheorem*{remark}{Remark}
% \fi

\usepackage{hyperref}

\iflipics
\else
\usepackage[capitalise]{cleveref}

\crefname{section}{section}{sections}
\Crefname{section}{Section}{Sections}

\crefname{lemma}{lemma}{lemmas}
\Crefname{lemma}{Lemma}{Lemmas}

\crefname{theorem}{theorem}{theorems}
\Crefname{theorem}{Theorem}{Theorems}

\crefname{definition}{definition}{definitions}
\Crefname{definition}{Definition}{Definitions}
\fi

\renewcommand{\epsilon}{\varepsilon}

%%% Local Variables:
%%% mode: latex
%%% TeX-master: "main"
%%% End:

\theoremstyle{plain}
\newtheorem{definition}{Definition}
\newtheorem{theorem}{Theorem}
\newtheorem{corollary}{Corollary}
\newtheorem{lemma}{Lemma}
\newtheorem{proposition}{Proposition}
\newtheorem{claim}{Claim}
\newtheorem{conjecture}{Conjecture}
\newtheorem{problem}{Problem}

\theoremstyle{remark}
\newtheorem{remark}{\textbf{Remark}} 
\newtheorem{example}{\textbf{Example}} 
\newtheorem{question}{\textbf{Question}} 
\newtheorem*{note}{\textbf{Note}}
\newtheorem*{mistake}{\textbf{Mistake}}

\usepackage{csquotes}

\title{Countering Adversarial Images\\using Input Transformations}

% Authors must not appear in the submitted version. They should be hidden
% as long as the \iclrfinalcopy macro remains commented out below.
% Non-anonymous submissions will be rejected without review.

\author{Chuan Guo\thanks{This work was performed whilst Chuan Guo was at Facebook AI Research.}\\
Cornell University\\
%\texttt{cg563@cs.cornell.edu}\\
\And
Mayank Rana \& Moustapha Ciss{\'e} \& Laurens van der Maaten\\
Facebook AI Research\\
%\texttt{\{mayankrana, moustaphacisse, lvdmaaten\}@fb.com}
}

% The \author macro works with any number of authors. There are two commands
% used to separate the names and addresses of multiple authors: \And and \AND.
%
% Using \And between authors leaves it to \LaTeX{} to determine where to break
% the lines. Using \AND forces a linebreak at that point. So, if \LaTeX{}
% puts 3 of 4 authors names on the first line, and the last on the second
% line, try using \AND instead of \And before the third author name.

\newcommand{\fix}{\marginpar{FIX}}
\newcommand{\new}{\marginpar{NEW}}

\iclrfinalcopy % Uncomment for camera-ready version, but NOT for submission.

\begin{document}

\maketitle

\begin{abstract}
This paper investigates strategies that defend against adversarial-example attacks on image-classification systems by transforming the inputs before feeding them to the system. Specifically, we study applying image transformations such as bit-depth reduction, JPEG compression, total variance minimization, and image quilting before feeding the image to a convolutional network classifier. Our experiments on ImageNet show that total variance minimization and image quilting are very effective defenses in practice, in particular, when the network is trained on transformed images. The strength of those defenses lies in their non-differentiable nature and their inherent randomness, which makes it difficult for an adversary to circumvent the defenses. \emph{Our best defense eliminates $60\%$ of strong gray-box and $90\%$ of strong black-box attacks by a variety of major attack methods.}
\end{abstract}

\section{Introduction}
\label{introduction}
% \leavevmode
% \\
% \\
% \\
% \\
% \\
\section{Introduction}
\label{introduction}

AutoML is the process by which machine learning models are built automatically for a new dataset. Given a dataset, AutoML systems perform a search over valid data transformations and learners, along with hyper-parameter optimization for each learner~\cite{VolcanoML}. Choosing the transformations and learners over which to search is our focus.
A significant number of systems mine from prior runs of pipelines over a set of datasets to choose transformers and learners that are effective with different types of datasets (e.g. \cite{NEURIPS2018_b59a51a3}, \cite{10.14778/3415478.3415542}, \cite{autosklearn}). Thus, they build a database by actually running different pipelines with a diverse set of datasets to estimate the accuracy of potential pipelines. Hence, they can be used to effectively reduce the search space. A new dataset, based on a set of features (meta-features) is then matched to this database to find the most plausible candidates for both learner selection and hyper-parameter tuning. This process of choosing starting points in the search space is called meta-learning for the cold start problem.  

Other meta-learning approaches include mining existing data science code and their associated datasets to learn from human expertise. The AL~\cite{al} system mined existing Kaggle notebooks using dynamic analysis, i.e., actually running the scripts, and showed that such a system has promise.  However, this meta-learning approach does not scale because it is onerous to execute a large number of pipeline scripts on datasets, preprocessing datasets is never trivial, and older scripts cease to run at all as software evolves. It is not surprising that AL therefore performed dynamic analysis on just nine datasets.

Our system, {\sysname}, provides a scalable meta-learning approach to leverage human expertise, using static analysis to mine pipelines from large repositories of scripts. Static analysis has the advantage of scaling to thousands or millions of scripts \cite{graph4code} easily, but lacks the performance data gathered by dynamic analysis. The {\sysname} meta-learning approach guides the learning process by a scalable dataset similarity search, based on dataset embeddings, to find the most similar datasets and the semantics of ML pipelines applied on them.  Many existing systems, such as Auto-Sklearn \cite{autosklearn} and AL \cite{al}, compute a set of meta-features for each dataset. We developed a deep neural network model to generate embeddings at the granularity of a dataset, e.g., a table or CSV file, to capture similarity at the level of an entire dataset rather than relying on a set of meta-features.
 
Because we use static analysis to capture the semantics of the meta-learning process, we have no mechanism to choose the \textbf{best} pipeline from many seen pipelines, unlike the dynamic execution case where one can rely on runtime to choose the best performing pipeline.  Observing that pipelines are basically workflow graphs, we use graph generator neural models to succinctly capture the statically-observed pipelines for a single dataset. In {\sysname}, we formulate learner selection as a graph generation problem to predict optimized pipelines based on pipelines seen in actual notebooks.

%. This formulation enables {\sysname} for effective pruning of the AutoML search space to predict optimized pipelines based on pipelines seen in actual notebooks.}
%We note that increasingly, state-of-the-art performance in AutoML systems is being generated by more complex pipelines such as Directed Acyclic Graphs (DAGs) \cite{piper} rather than the linear pipelines used in earlier systems.  
 
{\sysname} does learner and transformation selection, and hence is a component of an AutoML systems. To evaluate this component, we integrated it into two existing AutoML systems, FLAML \cite{flaml} and Auto-Sklearn \cite{autosklearn}.  
% We evaluate each system with and without {\sysname}.  
We chose FLAML because it does not yet have any meta-learning component for the cold start problem and instead allows user selection of learners and transformers. The authors of FLAML explicitly pointed to the fact that FLAML might benefit from a meta-learning component and pointed to it as a possibility for future work. For FLAML, if mining historical pipelines provides an advantage, we should improve its performance. We also picked Auto-Sklearn as it does have a learner selection component based on meta-features, as described earlier~\cite{autosklearn2}. For Auto-Sklearn, we should at least match performance if our static mining of pipelines can match their extensive database. For context, we also compared {\sysname} with the recent VolcanoML~\cite{VolcanoML}, which provides an efficient decomposition and execution strategy for the AutoML search space. In contrast, {\sysname} prunes the search space using our meta-learning model to perform hyperparameter optimization only for the most promising candidates. 

The contributions of this paper are the following:
\begin{itemize}
    \item Section ~\ref{sec:mining} defines a scalable meta-learning approach based on representation learning of mined ML pipeline semantics and datasets for over 100 datasets and ~11K Python scripts.  
    \newline
    \item Sections~\ref{sec:kgpipGen} formulates AutoML pipeline generation as a graph generation problem. {\sysname} predicts efficiently an optimized ML pipeline for an unseen dataset based on our meta-learning model.  To the best of our knowledge, {\sysname} is the first approach to formulate  AutoML pipeline generation in such a way.
    \newline
    \item Section~\ref{sec:eval} presents a comprehensive evaluation using a large collection of 121 datasets from major AutoML benchmarks and Kaggle. Our experimental results show that {\sysname} outperforms all existing AutoML systems and achieves state-of-the-art results on the majority of these datasets. {\sysname} significantly improves the performance of both FLAML and Auto-Sklearn in classification and regression tasks. We also outperformed AL in 75 out of 77 datasets and VolcanoML in 75  out of 121 datasets, including 44 datasets used only by VolcanoML~\cite{VolcanoML}.  On average, {\sysname} achieves scores that are statistically better than the means of all other systems. 
\end{itemize}


%This approach does not need to apply cleaning or transformation methods to handle different variances among datasets. Moreover, we do not need to deal with complex analysis, such as dynamic code analysis. Thus, our approach proved to be scalable, as discussed in Sections~\ref{sec:mining}.

\section{Problem Definition}
\label{definition}
% !TeX root = ../main.tex

We study defenses against non-targeted adversarial examples for image-recognition systems. Let $\mathcal{X} = [0,1]^{H \times W \times C}$ be the image space. Given an image classifier
$h(\cdot)$ and a source image $\bx \in \mathcal{X}$, a \emph{non-targeted\footnote{Given a target class $c$, a \emph{targeted adversarial example} $\bx'$ is an example that satisfies $h(\bx') = c$. We do not consider targeted attacks in this study.} adversarial example} of
$\bx$ is a perturbed image $\bx' \in \mathcal{X}$ such that $h(\bx) \neq h(\bx')$ and
$d(\bx, \bx') \leq \rho$ for some dissimilarity function $d(\cdot, \cdot)$ and $\rho \geq 0$. Ideally, $d(\cdot, \cdot)$ measures the perceptual difference between $\bx$ and $\bx'$ but, in practice, the Euclidean distance $d(\bx, \bx') = \| \bx - \bx' \|_2$ or the Chebyshev distance
$d(\bx, \bx') = \| \bx - \bx' \|_\infty$ is most commonly used.

Given a set of $N$ images $\{\bx_1,\ldots,\bx_N\}$ and a target classifier $h(\cdot)$, an
\emph{adversarial attack} aims to generate $\{\bx_1',\ldots,\bx_N'\}$ such that each $\bx_n'$
is an adversarial example for $\bx_n$. The \emph{success rate} of an attack is measured by the proportion of predictions that was altered by an attack: $\frac{1}{N} \sum_{n=1}^N \mathds{1}[h(\bx_n) \neq h(\bx_n')]$. The success rate is generally measured as a function of the magnitude of the perturbations performed by the attack, using the \emph{normalized $L_2$-dissimilarity}:
\begin{equation}
\label{l2-dissim}
\frac{1}{N} \sum_{n=1}^N \frac{\| \bx_n - \bx_n' \|_2}{\| \bx_n \|_2}.
\end{equation}
A strong adversarial attack has a high success rate whilst its normalized $L_2$-dissimilarity is low.

In most practical settings, an adversary does not have direct access to the model $h(\cdot)$ and has to do a \emph{black-box} attack. However, prior work has shown successful attacks by transferring adversarial examples generated for a separately-trained model to an unknown target model \citep{liu2016delving}. Therefore, we investigate both the black-box and a more difficult \emph{gray-box} attack setting: in our gray-box setting, the adversary has access to the model architecture and the model parameters, but is unaware of the defense strategy that is being used.

A \emph{defense} is an approach that aims make the prediction on an adversarial example $h(\bx')$ equal to the prediction on the corresponding clean example $h(\bx)$. In this study, we focus on \emph{image-transformation defenses} $g(\bx)$ that perform prediction via $h(g(\bx'))$. Ideally, $g(\cdot)$ is a complex, non-differentiable, and potentially stochastic function: this makes it difficult for an adversary to attack the prediction model $h(g(\bx))$ even when the adversary knows both $h(\cdot)$ and $g(\cdot)$.


\section{Adversarial Attacks}
\label{attacks}
\section{Assumptions for security}
\label{sec:attacks}

The security of a quantum cryptographic protocol relies on assumptions about the physics of the devices that are employed to implement the protocol.  In this section, we discuss these assumptions. For concreteness, we focus on the case of QKD, for which we describe the full set of assumptions in \secref{sec:attacks:assumptionlist}.  We then explain why these assumptions are needed and to what extent they are justified in \secref{sec:attacks:necessity}. Experimental work in QKD has shown however that the assumptions are often very difficult to meet, and are actually not met in many cases. This fact can be exploited by quantum hacking attacks, which are described in \ref{sec:attacks:hacking}. Finally, in Section~\ref{sec:attacks:countermeasures}, we discuss countermeasures against these attacks. 

\subsection{Standard assumptions for QKD} \label{sec:attacks:assumptionlist}

The security of QKD protocols usually relies on the following assumptions.

\begin{enumerate}
  \item \label{item_qm} All devices used by Alice and Bob, as well as the communication channels connecting them, are correctly and completely\footnote{The completeness of quantum theory can be derived from their correctness;  see \secref{sec:completeness}.} described by quantum theory.
    \item \label{item_res} The channel that Alice and Bob use to exchange classical messages is authentic, i.e., it is impossible for an adversary to modify messages or insert new ones. 
  \item \label{item_conv} The devices that Alice and Bob use locally to execute the steps of the protocol, e.g., for preparing and measuring quantum systems, do exactly what they are instructed to do.
\end{enumerate}

As already indicated earlier, due to the lack of proof techniques, additional assumptions had been introduced in the past. A prominent example is the \emph{i.i.d.}\ assumption, which demands that the quantum channel connecting Alice and Bob be described by a sequence of identical and independently distributed maps. Physically, this means that an adversary's interception strategy is such that each signal sent from Alice to Bob is modified in the same manner and independently of the other signals. Security under the i.i.d.\ assumption is called security against \emph{collective attacks}~\cite[see also \secref{sec:qkd.other.models}]{BM97b,BBBvdGM02}. Another assumption, which  usually comes on top of the i.i.d.\ assumption, is that Eve only stores classical data, which she obtains by individually measuring the pieces of information she gained from each signal sent from Alice to Bob. Since it is difficult to argue why an adversary should be restricted in that particular way, the corresponding security guarantee is rather weak. It is usually referred to as security against \emph{individual attacks}~\cite[see \secref{sec:qkd.other.models}]{Fuchsetal1997,Lutkenhaus2000}. 

Most modern security proofs do however not require such additional assumptions, i.e., they are based entirely on Assumptions~\ref{item_qm}--\ref{item_conv} above. This means, in particular, that the quantum channel connecting Alice and Bob can be arbitrary, and may even be entirely controlled by Eve. In this case, one talks about security against \emph{general attacks}, \emph{coherent attacks}, or \emph{joint attacks}. Sometimes the term  \emph{unconditional security} appeared in the literature~\cite{SBCDLP09}, but it is important to keep in mind that the assumptions listed above are still necessary.

\subsection{Necessity and justification of assumptions} \label{sec:attacks:necessity}

Assumption~\ref{item_qm} is often implicit, for it is a prerequisite to even describe the cryptographic scheme. It justifies the use of the formalism of quantum theory to model the different systems, such as the communication channel, including any possible attacks on them. The assumption thus captures the main idea behind quantum cryptography, namely that an adversary is limited by the laws of quantum theory.  The other two assumptions ensure that the experimental implementation follows the theoretical prescription that enters the security definition (Definition~\ref{def:security}), namely the description of the protocol $\pi_{AB}$ and the used resources. In particular, Assumption~\ref{item_res} guarantees that the resources shared between Alice and Bob fulfil the theoretical specifications~$\aR$, which in the case of QKD includes the classical authentic communication channel. Assumption~\ref{item_conv} guarantees that the steps prescribed by the protocol~$\pi_{A B}$ are correctly executed.

Assumption~\ref{item_qm} is widely accepted \--- and proving it wrong would represent a major breakthrough in physics. Nevertheless, it has been shown that there exist QKD protocols that only rely on the weaker assumption of \emph{no-signalling}~\cite{BHK05}.  

Assumption~\ref{item_res} demands that an authentic communication channel is set up between Alice and Bob. There exist information-theoretically secure protocols that achieve this, provided that Alice and Bob share a weak secret key~\cite[see also \secref{sec:smt.auth}]{RW03,DW09,ACLV19}.  Assumption~\ref{item_res} can thus be met by the use of such authentication protocols (see also~\secref{sec:intro} as well as standard textbooks on classical cryptography)

Although Assumption~\ref{item_conv} sounds rather natural, and is in fact required for almost any cryptographic scheme, including any classical one, it is rather challenging to meet.  Numerous quantum hacking experiments, which have been conducted over the past few years, have shown that many implementations of QKD failed to satisfy this assumption. To illustrate this problem, we describe selected examples of such attacks in the following subsection.

\subsection{Quantum hacking attacks} \label{sec:attacks:hacking}          

We start with the \emph{photon number splitting attack}~\cite{Brassardetal2000}, which targets optical implementations of QKD that use individual photons as quantum information carriers. Suppose, for concreteness, that Alice and Bob implement the BB84 protocol~\cite{BB84} by encoding the qubits into the polarisation degree of freedom of individual photons. Specifically, Alice may use a single-photon source that emits photons with a polarisation that she can choose. The BB84 protocol\footnote{This protocol is explained in more detail in \secref{sec:securityproofs}, where a security proof is also sketched.} requires her to send in each round at random a state from one orthonormal basis, say $\{\ket{h}, \ket{v}\}$, where $\ket{h}$ may be realised by a horizontally polarised photon and $\ket{v}$ by a vertically polarised one, or from a complementary basis $\{\ket{d^+}, \ket{d^-}\}$, where $\ket{d^+} = \smash{\frac{1}{\sqrt{2}}} (\ket{h} + \ket{v})$ and $\ket{d^-} = \smash{\frac{1}{\sqrt{2}}} (\ket{h} - \ket{v})$. It may now happen that, in an experimental implementation, the source sometimes accidentally emits two photons at once, which then carry the same polarisation. The states emitted in the four cases are thus $\ket{h} \otimes \ket{h}$, $\ket{v} \otimes \ket{v}$, $\ket{d^+} \otimes \ket{d^+}$, and $\ket{d^-} \otimes \ket{d^-}$.

Before describing the actual attack, we first give a simple information-theoretic argument for why this is problematic. Note first that one single photon carries no information about the choice of the basis made by Alice. Indeed, for either of the basis choices, the density operator describing the photon is maximally mixed, i.e., $\frac{1}{2} \proj{h} + \frac{1}{2} \proj{v} = \frac{1}{2} \proj{d^+} + \frac{1}{2} \proj{d^-} =  \frac{1}{2} \mathbf{1}$. This is however no longer the case for a pulse consisting of two photons, i.e., 
\begin{align}
  \frac{1}{2} \proj{h}^{\otimes 2} + \frac{1}{2} \proj{v}^{\otimes 2} \neq \frac{1}{2} \proj{d^+}^{\otimes 2} + \frac{1}{2} \proj{d^-}^{\otimes 2} \ .
\end{align}
Hence, if the source accidentally emits two equally polarised photons instead of one, it reveals information about Alice's basis choice, which it shouldn't. 

It is therefore not surprising that such two-photon pulses can be exploited by an adversary to attack the system. Eve, who intercepts the channel, may split the two-photon pulse into two, keep one of the photons and send the other one to Bob. The latter thus receives photons in exactly the way prescribed by the protocol, and hence does not notice the interception. Eve, meanwhile, may measure the photons she captured. In principle, if Eve had quantum memory, she could even wait with the measurement until Alice announces the basis choice to Bob, and hence always gain full information about the polarisation state that Alice prepared. 

While the photon number splitting attack exploits an imperfection of the sender (namely that it sometimes emits two identically polarised photons instead of one), many quantum attacks are targeted towards the receiver. An example is the \emph{time-shift attack}~\cite{Makarovetal2006,qi2007time,Zhaoetal2008}, which exploits inaccuracies of the photon detectors. In order to avoid dark counts, the photon detectors are often set up such that they only count photons that arrive within a small time window around the time when a signal is expected to arrive. Furthermore, Bob's receiver device may consist of more than one detector, e.g., one for each possible polarisation state. The time windows of the different detectors are then never perfectly synchronised. This means that there are times at which the receiver is more sensitive to signals with respect to one polarisation than another. Eve may therefore, by appropriately delaying the signals sent from Alice and Bob, bias the detected signals towards one or the other polarisation, and thus gain information about what Bob measures. While this information may be partial, it can, together with the error correction information that is available to Eve, be sufficient to infer the final key. 

Another attack that is targeted towards the receiver is the \emph{detector blinding attack} ~\cite{Makarov2009,WKRFNW11,LWWESM10,GLLSKM11}, where the adversary tries to control the detectors by illuminating them with bright laser light.  In a QKD implementation that uses the encoding of information into the polarisation of individual photons, the detectors are usually configured such they can optimally detect single photon pulses. That is, they should click whenever the incoming pulse contains a photon, and not click if the pulse is empty.  However, the behaviour of such detectors may be rather different in a regime where the incoming pulses contain many photons. For example, it could be that they always click when they are exposed to bright light with a particular intensity, and they may never click for another intensity. Hence, by sending in light with appropriately chosen polarisation and intensity, Eve may gain immediate control over the clicks of Bob's detector. To exploit this for an attack, Eve may mimic Bob's receiver, i.e., intercept the photons sent from Alice and measure them in a randomly chosen basis, as Bob would do. She then sends bright light to Bob to ensure that he obtains the same detector clicks as if he had directly obtained Alice's photons. This works particularly well for implementations that use a \emph{passive basis choice}, i.e., where Bob's measurement basis is not  provided as an input, but rather made by the detection device itself. In this case, an adversary can essentially remote-control Bob and thus get hold of the entire key. 

Yet another hacking strategy are \emph{Trojan-horse attacks}~\cite{Vakhitov2001,GisinFaselKraus2006}. Here the idea is to send a bright laser pulse via the optical fibre into Alice or Bob's component to extract information about its internal settings. Depending on the sender and receiver hardware which is used, measuring the reflection of the pulse can allow Eve, for instance, to determine the basis choices made by Alice and Bob.

In some optical implementations of QKD, e.g., in the \emph{plug-and-play}~\cite{Muller97} or the \emph{circular-type}~\cite{Nishioka2002} system, Alice does not have a photon source but instead encodes information by modulating an incoming signal from Bob before sending it back to him. The signal thus travels twice in opposite directions through the same optical links, which helps reducing fluctuations due to birefringence  and environmental noise. The two-fold use of the (insecure) channel however opens additional possibilities of attacks~\cite{GisinFaselKraus2006}. A prominent example is the \emph{phase-remapping attack}~\cite{FQTL07,XQL10}. It exploits the fact that the modulator used by Alice to encode information into the signal coming from Bob acts on that signal during a particular time interval. In the attack, the adversary slightly advances or delays the signal on its way from Bob to Alice, so that it no longer lies fully within that time interval. The modulation by Alice will then be incomplete, which means that the encoding of the information in the signal differs from what is foreseen by the protocol. This can in turn be exploited by Eve in an intercept-and-resend attack on the signal returned from Alice to Bob. 


\subsection{Countermeasures against quantum hacking} \label{sec:attacks:countermeasures}

The attacks described here have in common that they all exploit a breakdown of Assumption~\ref{item_conv}. Specifically, in the case of the photon-splitting attack, the device used by Alice sends out more information than it is supposed to. In the case of the time-shift attack, it is Bob's measurement device  whose measurement operators are not constant over time and can even be partially controlled by Eve. Finally, in the case of the detector blinding attack on systems with passive basis choice, Eve even takes over control of the randomness used to choose the basis.

A seemingly obvious countermeasure to prevent such attacks is to manufacture sources and detectors that meet the theoretical specifications. That is, one would need a perfect single-photon source, as well as detectors that are perfectly efficient and only measure photon pulses in a specified parameter regime. Such requirements are however unrealistic --- the devices used in experiments will always, at least slightly, deviate from these specifications. 

The other possibility is to develop cryptographic protocols and  security proofs that tolerate imperfections of the devices~\cite{GLLP04}.  This has been done in particular for the attacks described above. To prevent photon number splitting attacks, an efficient countermeasure is the \emph{decoy-state} method~\cite{Hwang2003,Wang2005,Loetal2005}. The idea here is that Alice sometimes deliberately sends multi-photon pulses. Alice and Bob can  then check statistically whether an adversary captured them. Another possibility is to use protocols where Alice's encoding of information has the property that, even when one photon is extracted from a pulse, the information about what Alice sent is still partial~\cite{SARG,TamakiLo,SYK14}. In the case of time-shift attacks, it is sufficient to characterise the maximum bias in the detector efficiencies that can be introduced and account for it in the security proofs. Finally, for the detector blinding attacks, a possible countermeasure is to add tests to the protocol, such as a monitoring of the photocurrent, in order to detect those~\cite{Yuanetal2010}.  

The main problem with such countermeasures is however that the space of possible imperfections is hard to characterise. The above are just a few examples of attacks, and many others have been proposed, and sometimes even demonstrated to work successfully in experiments. For example, an adversary may exploit imperfections in the randomness that Alice and Bob use for choosing their measurement basis. To prevent such attacks, one may again extend the protocols such that they can tolerate imperfect randomness (see \secref{sec:alternative.randomness}). 



The last decade has thus seen an arms race between designers and attackers of quantum cryptographic schemes. A possible way out of this unsatisfactory situation is \emph{device-independent cryptography}. Here the idea is to replace Assumption~\ref{item_conv} by something much weaker. Namely, one requires that the devices used by Alice and Bob do not unintentionally send information out to an adversary, and that the classical processing of information done by Alice and Bob is correct. Crucially, however,  one does no longer demand that the sources and detectors used by Alice and Bob work according to their specifications. The way this can work is explained in \secref{sec:alternative.di}. 

%%% Local Variables:
%%% TeX-master: "main.tex"
%%% End:


\section{Defenses}
\label{defenses}

% !TeX root = ../main.tex

% One explanation for the principle behind adversarial perturbations is that they alter the
% behavior of lower level convolutional filters that detect edges and textures. This misrepresented
% information propagates forward through the network and combine at the upper layers to produce
% features that are indistinguishable from the class that the example is designed to mimic.
% In summary, despite the per-pixel perturbation being relatively small, they combine across
% channels and spatial coordinates to fool the network.

% The intuition behind our defense mechanisms is to remove or replace pixels from the input image so that combination of per-pixel perturbations cannot occur. We investigate the effect of cropping, random pixel dropping, and image quilting on adversarial examples. \autoref{ladybug} shows a sample image and a corresponding adversarial image under these transformations.

Adversarial attacks alter particular statistics of the input image in order to change the model prediction. Indeed, adversarial perturbations $\bx \!-\! \bx'$ have a particular structure, as illustrated by Figure~\ref{samples}. We design and experiment with image transformations that alter the structure of these perturbations, and investigate whether the alterations undo the effects of the adversarial attack. We investigate five image transformations: (1) image cropping and rescaling, (2) bit-depth reduction, (3) JPEG compression, (4) total variance minimization, and (5) image quilting.

\begin{wrapfigure}{r}{0.5\textwidth}
    \vspace{-2em}
    \includegraphics[width=0.5\textwidth]{figures/ladybug.pdf}
    \caption{Illustration of total variance minimization and image quilting applied to an original and an adversarial image (produced using I-FGSM with $\epsilon \!=\! 0.03$, corresponding to a normalized $L_2$-dissimilarity of 0.075). From left to right, the columns correspond to: (1) no transformation, (2) total variance minimization, and (3) image quilting. From top to bottom, rows correspond to: (1) the original image, (2) the corresponding adversarial image produced by I-FGSM, and (3) the absolute difference between the two images above. Difference images were multiplied by a constant scaling factor to increase visibility.}
    \label{ladybug}
    \vspace{-2em}
\end{wrapfigure}

\subsection{Image cropping-rescaling, bit-depth reduction, and compression}

% \begin{figure}[t!]
%     \centering
%     \begin{subfigure}[t]{0.5\columnwidth}
%         \centering
%         \includegraphics[width=\textwidth]{figures/ifgs_0_0025_resnet50_crop_top1.png}
%     \end{subfigure}%
%     \begin{subfigure}[t]{0.5\columnwidth}
%         \centering
%         \includegraphics[width=\textwidth]{figures/ifgs_0_0250_resnet50_crop_top1.png}
%     \end{subfigure}
%     \caption{Effect of different crop and downsample ratios on adversarial examples.}
%     \label{crop-fig}
% \end{figure}

We first introduce three simple image transformations: image cropping-rescaling, bit-depth reduction \citep{xu2017feature}, and JPEG compression and decompression \citep{dziugaite2016study}. \emph{Image cropping-rescaling} has the effect of altering the spatial positioning of the adversarial perturbation, which is important in making attacks successful. Following \citet{he2016residual}, we crop and rescale images at training time as part of the data augmentation. At test time, we average predictions over random image crops. \emph{Bit-depth reduction} \citep{xu2017feature} perform a simple type of quantization that can removes small (adversarial) variations in pixel values from an image; we reduce images to $3$ bits in our experiments. \emph{JPEG compression} \citep{dziugaite2016study} removes small perturbations in a similar way; we perform compression at quality level $75$ (out of $100$).



% LAURENS: If we want to show this, this should move to the experimental section.
% \autoref{crop-fig} Shows the effectiveness of the random cropping for various values of $r$. The adversarial examples are generated by I-FGSM at $\epsilon = 0.001$ (left)
% and $\epsilon = 0.01$ (right) on a pretrained ResNet-50 model. The model achieves a Top-1
% validation accuracy of 76.02\% on clean images. At $r=1$, the adversarial image is unmodified and the corresponding
% accuracy reflects the base model's accuracy without any defense. For smaller perturbation, cropping is very effective even at higher values of $r$, which shows that these adversarial
% perturbations are very unstable against a few dropped pixels. At larger perturbation,
% cropping at much lower $r$ is the most effective by a large margin.

% We suspect that having a lower crop ratio always strictly increases effectiveness against
% adversarial examples. However, this comes at a large cost of losing the object of interest
% in crops of smaller size. To counter this effect, we can ensemble the prediction of different
% crops by averaging their class probability vectors. By taking the ensemble of 30 random crops,
% we have obtain a much higher accuracy on adversarial examples when using smaller crop ratios.

% \subsection{Pixel dropout}
% \label{pixel_dropout}

% \begin{figure}[t!]
%     \centering
%     \begin{subfigure}[t]{0.5\columnwidth}
%         \centering
%         \includegraphics[width=\textwidth]{figures/ifgs_0_0025_resnet50_drop_top1.png}
%     \end{subfigure}%
%     \begin{subfigure}[t]{0.5\columnwidth}
%         \centering
%         \includegraphics[width=\textwidth]{figures/ifgs_0_0250_resnet50_drop_top1.png}
%     \end{subfigure}
%     \caption{Effect of different random pixel drop rates on adversarial examples.}
%     \label{pixel-drop}
% \end{figure}

% Another method that breaks the structure of adversarial perturbations whilst keeping semantic content intact is  pixel dropout. As images channels are normalized to be zero-mean before they are used as input into the model, we implement pixel dropout by sampling a Bernoulli random variable $X(i, j, k)$ for each pixel location $(i, j, k)$ and replace the corresponding pixel value by $0$ whenever the draw equals $X(i, j, k) = 1$.

% LAURENS: Idem. If we want to show this, it should go in the experimental section.
% \autoref{pixel-drop} demonstrates the effect of random pixel drops on the same adversarial examples
% as in \autoref{crop-fig}. This defense effectively mitigates adversarial examples even when
% the drop rate is very small (e.g. $p = 0.05$). However, since the model does not have exposure to this type of transformation at training time, accuracy sharply deteriorates at larger $p$.


\subsection{Total variance minimization}
An alternative way of removing adversarial perturbations is via a compressed sensing approach that combines pixel dropout with total variation minimization \citep{rudin1992tvm}. This approach randomly selects a small set of pixels, and reconstructs the ``simplest'' image that is consistent with the selected pixels. The reconstructed image does not contain the adversarial perturbations because these perturbations tend to be small and localized.

% Because adversarial perturbations tend to be small and localized, compressed-sensing approaches that combine pixel dropout with total variation minimization are likely to remove these fine-scale perturbations without affecting the coarser-scale information in the image that contains most of its semantic information. The key idea of this approach is to randomly remove the majority of the pixel values from the image as to remove the adversarial perturbation, and then to reconstruct the image from the non-removed pixels. The reconstructed image likely does not contain much of the adversarial image.

Specifically, we first select a random set of pixels by sampling a Bernoulli random variable $X(i, j, k)$ for each pixel location $(i, j, k)$; we maintain a pixel when $X(i, j, k) = 1$. Next, we use total variation minimization to constructs an image $\bz$ that is similar to the (perturbed) input image $\bx$ for the selected set of pixels, whilst also being ``simple'' in terms of total variation by solving:
\begin{equation}
\label{tvcs}
\min_\bz \| (1-X) \odot (\bz - \bx) \|_2 + \lambda_{\tv} \cdot \tv_p(\bz).
\end{equation}
Herein, $\odot$ denotes element-wise multiplication, and $\tv_p(\bz)$ represents the $L_p$-total variation of $\bz$:
\begin{equation}
\label{lptv}
\tv_p(\bz) = \sum_{k=1}^{K} \left[ \sum_{i=2}^{N} \|\bz(i,:,k) - \bz(i-1,:,k)\|_p + \sum_{j=2}^{N} \|\bz(:,j,k) - \bz(:,j-1,k)\|_p \right].
\end{equation}
The total variation (TV) measures the amount of fine-scale variation in the image $\bz$, as a result of which TV minimization encourages removal of small (adversarial) perturbations in the image. The objective function~(\ref{tvcs}) is convex in $\bz$, which makes solving for $\bz$ straightforward. In our implementation, we set $p \!=\! 2$ and employ a special-purpose solver based on the split Bregman method
\citep{goldstein2009split} to perform total variance minimization efficiently.

The effectiveness of TV minimization is illustrated by the images in the middle column of Figure~\ref{ladybug}: in particular, note that the adversarial perturbations that were present in the background for the non-transformed image (see bottom-left image) have nearly completely disappeared in the TV-minimized adversarial image (bottom-center image). As expected, TV minimization also changes image structure in non-homogeneous regions of the image, but as these perturbations were not adversarially designed we expect the negative effect of these changes to be limited.

%When defending against adversaries that minimize the $L_\infty$-dissimiarity
%(such as FGSM and I-FGSM), it seems intuitive that the $L_2$ reconstruction loss in
%\autoref{tvcs} should be replaced by $L_\infty$. However, since optimizing the $L_\infty$-norm
%directly is intractable, we solve the equivalent constrained optimization problem instead.
%\begin{align}
%\label{tvinf}
%&\min_\bz \tv_p(\bz) \nonumber \\
%\text{s.t. } \bx(i,j,k) - \tau &\leq \bz(i,j,k) \leq \bx(i,j,k) + \tau \\
%&\hspace{36pt} \text{for all } X(i,j,k) = 0. \nonumber
%\end{align}
%This gives rise to a box-constrained convex optimization problem. However, due to the large number
%of variables, we chose to use L-BFGS with box constraint \cite{byrd1995limited} instead.

% LAURENS: This should move to experiments, also.

% We apply TV compressed sensing to the same adversarial examples as in \autoref{crop-fig}.
% We use isotropic TV minimization and choose $\lambda_{\tv} $ to optimize the visual quality
% of reconstruction images while removing a large portion of adversarial perturbation. As shown
% in \autoref{pixel-drop}, applying total variation reconstruction significantly improves accuracy
% compared to random pixel drops, especially for larger drop rates (e.g. $p \geq 0.8$).

\subsection{Image quilting}

% \begin{figure}[t!]
%     \centering
%     \includegraphics[width=0.5\textwidth]{figures/ifgs_resnet50_quilt_top1.png}
%     \caption{Effect of different patch size for quilting on adversarial examples.}
%     \label{quilting}
% \end{figure}

Image quilting \citep{efros2001quilting} is a non-parametric technique that synthesizes images by piecing together small patches that are taken from a database of image patches. The algorithm places appropriate patches in the database for a predefined set of grid points, and computes minimum graph cuts \citep{boykov2001graphcuts} in all overlapping boundary regions to remove edge artifacts.

Image quilting can be used to remove adversarial perturbations by constructing a patch database that only contains patches from ``clean'' images (without adversarial perturbations); the patches used to create the synthesized image are selected by finding the $K$ nearest neighbors (in pixel space) of the corresponding patch from the adversarial image in the patch database, and picking one of these neighbors uniformly at random. The motivation for this defense is that the resulting image only consists of pixels that were not modified by the adversary --- the database of real patches is unlikely to contain the structures that appear in adversarial images.

The right-most column of Figure~\ref{ladybug} illustrates the effect of image quilting on adversarial images. Whilst interpretation of these images is more complicated due to the quantization errors that image quilting introduces, it is interesting to note that the absolute differences between quilted original and the quilted adversarial image appear to be smaller in non-homogeneous regions of the image. This suggests that TV minimization and image quilting lead to inherently different defenses.

% \autoref{quilting} shows the effect of image quilting with various patch sizes. For smaller $b$, the model's accuracy on clean images is very high, but is not as robust against adversarial images. This is likely due to adversarial perturbation affecting the choice of the nearest neighbor from $\mathcal{B}$. For larger patch size, this is more difficult, but at the cost of image fidelity and accuracy on clean images.


\section{Experiments}
\label{experiment}
\section{Experimental Evaluation}
\label{sec:experiment}
To demonstrate the viability of our modeling methodology, we show experimentally how through the deliberate combination and configuration of parallel FREEs, full control over 2DOF spacial forces can be achieved by using only the minimum combination of three FREEs.
To this end, we carefully chose the fiber angle $\Gamma$ of each of these actuators to achieve a well-balanced force zonotope (Fig.~\ref{fig:rigDiagram}).
We combined a contracting and counterclockwise twisting FREE with a fiber angle of $\Gamma = 48^\circ$, a contracting and clockwise twisting FREE with $\Gamma = -48^\circ$, and an extending FREE with $\Gamma = -85^\circ$.
All three FREEs were designed with a nominal radius of $R$ = \unit[5]{mm} and a length of $L$ = \unit[100]{mm}.
%
\begin{figure}
    \centering
    \includegraphics[width=0.75\linewidth]{figures/rigDiagram_wlabels10.pdf}
    \caption{In the experimental evaluation, we employed a parallel combination of three FREEs (top) to yield forces along and moments about the $z$-axis of an end effector.
    The FREEs were carefully selected to yield a well-balanced force zonotope (bottom) to gain full control authority over $F^{\hat{z}_e}$ and $M^{\hat{z}_e}$.
    To this end, we used one extending FREE, and two contracting FREEs which generate antagonistic moments about the end effector $z$-axis.}
    \label{fig:rigDiagram}
\end{figure}


\subsection{Experimental Setup}
To measure the forces generated by this actuator combination under a varying state $\vec{x}$ and pressure input $\vec{p}$, we developed a custom built test platform (Fig.~\ref{fig:rig}). 
%
\begin{figure}
    \centering
    \includegraphics[width=0.9\linewidth]{figures/photos/rig_labeled.pdf}
    \caption{\revcomment{1.3}{This experimental platform is used to generate a targeted displacement (extension and twist) of the end effector and to measure the forces and torques created by a parallel combination of three FREEs. A linear actuator and servomotor impose an extension and a twist, respectively, while the net force and moment generated by the FREEs is measured with a force load cell and moment load cell mounted in series.}}
    \label{fig:rig}
\end{figure}
%
In the test platform, a linear actuator (ServoCity HDA 6-50) and a rotational servomotor (Hitec HS-645mg) were used to impose a 2-dimensional displacement on the end effector. 
A force load cell (LoadStar  RAS1-25lb) and a moment load cell (LoadStar RST1-6Nm) measured the end-effector forces $F^{\hat{z_e}}$ and moments $M^{\hat{z_e}}$, respectively.
During the experiments, the pressures inside the FREEs were varied using pneumatic pressure regulators (Enfield TR-010-g10-s). 

The FREE attachment points (measured from the end effector origin) were measured to be:
\begin{align}
    \vec{d}_1 &= \bmx 0.013 & 0 & 0 \emx^T  \text{m}\\
    \vec{d}_2 &= \bmx -0.006 & 0.011 & 0 \emx^T  \text{m}\\
    \vec{d}_3 &= \bmx -0.006 & -0.011 & 0 \emx^T \text{m}
%    \vec{d}_i &= \bmx 0 & 0 & 0 \emx^T , && \text{for } i = 1,2,3
\end{align}
All three FREEs were oriented parallel to the end effector $z$-axis:
\begin{align}
    \hat{a}_i &= \bmx 0 & 0 & 1 \emx^T, \hspace{20pt} \text{for } i = 1,2,3
\end{align}
Based on this geometry, the transformation matrices $\bar{\mathcal{D}}_i$ were given by:
\begin{align}
    \bar{\mathcal{D}}_1 &= \bmx 0 & 0 & 1 & 0 & -0.013 & 0 \\ 0 & 0 & 0 & 0 & 0 & 1 \emx^T  \\
    \bar{\mathcal{D}}_2 &= \bmx 0 & 0 & 1 & 0.011 & 0.006 & 0 \\ 0 & 0 & 0 & 0 & 0 & 1 \emx^T  \\
    \bar{\mathcal{D}}_3 &= \bmx 0 & 0 & 1 & -0.011 & 0.006 & 0 \\ 0 & 0 & 0 & 0 & 0 & 1 \emx^T 
%    \bar{\mathcal{D}}_i &= \bmx 0 & 0 & 1 & 0 & 0 & 0 \\ 0 & 0 & 0 & 0 & 0 & 1 \emx^T , && \text{for } i = 1,2,3
\end{align}
These matrices were used in equation \eqref{eq:zeta} to yield the state-dependent fluid Jacobian $\bar{J}_x$ and to compute the resulting force zontopes.
%while using measured values of $\vec{\zeta}^{\,\text{meas}} (\vec{q}, \vec{P})$ and $\vec{\zeta}^{\,\text{meas}} (\vec{q}, 0)$ in \eqref{eq:fiberIso} yields the empirical measurements of the active force.



\subsection{Isolating the Active Force}
To compare our model force predictions (which focus only on the active forces induced by the fibers)
to those measured empirically on a physical system, we had to remove the elastic force components attributed to the elastomer. 
Under the assumption that the elastomer force is merely a function of the displacement $\vec{x}$ and independent of pressure $\vec{p}$ \cite{bruder2017model}, this force component can be approximated by the measured force at a pressure of $\vec{p}=0$. 
That is: 
\begin{align}
    \vec{f}_{\text{elast}} (\vec{x}) = \vec{f}_{\text{\,meas}} (\vec{x}, 0)
\end{align}
With this, the active generalized forces were measured indirectly by subtracting off the force generated at zero pressure:
\begin{align}
    \vec{f} (\vec{x}, \vec{p})  &= \vec{f}_{\text{meas}} (\vec{x}, \vec{p}) - \vec{f}_{\text{meas}} (\vec{x}, 0)     \label{eq:fiberIso}
\end{align}


%To validate our parallel force model, we compare its force predictions, $\vec{\zeta}_{\text{pred}}$, to those measured empirically on a physical system, $\vec{\zeta}_\text{meas}$. 
%From \eqref{eq:Z} and \eqref{eq:zeta}, the force at the end effector is given by:
%\begin{align}
%    \vec{\zeta}(\vec{q}, \vec{P}) &= \sum_{i=1}^n \bar{\mathcal{D}}_i \left( {\bar{J}_V}_i^T(\vec{q_i}) P_i + \vec{Z}_i^{\text{elast}} (\vec{q_i}) \right) \\
%    &= \underbrace{\sum_{i=1}^n \bar{\mathcal{D}}_i {\bar{J}_V}_i^T(\vec{q_i}) P_i}_{\vec{\zeta}^{\,\text{fiber}} (\vec{q}, \vec{P})} + \underbrace{\sum_{i=1}^n \bar{\mathcal{D}}_i \vec{Z}_i^{\text{elast}} (\vec{q_i})}_{\vec{\zeta}^{\text{elast}} (\vec{q})}   \label{eq:zetaSplit}
%     &= \vec{\zeta}^{\,\text{fiber}} (\vec{q}, \vec{P}) + \vec{\zeta}^{\text{elast}} (\vec{q})
%\end{align}
%\Dan{These will need to reflect changes made to previous section.}
%The model presented in this paper does not specify the elastomer forces, $\vec{\zeta}^{\text{elast}}$, therefore we only validate its predictions %of the fiber forces, $\vec{\zeta}^{\,\text{fiber}}$. 
%We isolate the fiber forces by noting that $\vec{\zeta}^{\text{elast}} (\vec{q}) = \vec{\zeta}(\vec{q}, 0)$ and rearranging \eqref{eq:zetaSplit}
%\begin{align}
%    \vec{\zeta}^{\,\text{fiber}} (\vec{q}, \vec{P})  &= \vec{\zeta}(\vec{q}, \vec{P}) - \vec{\zeta}(\vec{q}, 0)     \label{eq:fiberIso}
%%    \vec{\zeta}^{\,\text{fiber}}_{\text{emp}} (\vec{q}, \vec{P})  &= \vec{\zeta}_{\text{emp}}(\vec{q}, \vec{P}) - %\vec{\zeta}_{\text{emp}}(\vec{q}, 0)
%\end{align}
%Thus we measure the fiber forces indirectly by subtracting off the forces generated at zero pressure.  


\subsection{Experimental Protocol}
The force and moment generated by the parallel combination of FREEs about the end effector $z$-axis  was measured in four different geometric configurations: neutral, extended, twisted, and simultaneously extended and twisted (see Table \ref{table:RMSE} for the exact deformation amounts). 
At each of these configurations, the forces were measured at all pressure combinations in the set
\begin{align}
    \mathcal{P} &= \left\{ \bmx \alpha_1 & \alpha_2 & \alpha_3 \emx^T p^{\text{max}} \, : \, \alpha_i = \left\{ 0, \frac{1}{4}, \frac{1}{2}, \frac{3}{4}, 1 \right\} \right\}
\end{align}
with $p^{\text{max}}$ = \unit[103.4]{kPa}. 
\revcomment{3.2}{The experiment was performed twice using two different sets of FREEs to observe how fabrication variability might affect performance. The results from Trial 1 are displayed in Fig.~\ref{fig:results} and the error for both trials is given in Table \ref{table:RMSE}.}



\subsection{Results}

\begin{figure*}[ht]
\centering

\def\picScale{0.08}    % define variable for scaling all pictures evenly
\def\plotScale{0.2}    % define variable for scaling all plots evenly
\def\colWidth{0.22\linewidth}

\begin{tikzpicture} %[every node/.style={draw=black}]
% \draw[help lines] (0,0) grid (4,2);
\matrix [row sep=0cm, column sep=0cm, style={align=center}] (my matrix) at (0,0) %(2,1)
{
& \node (q1) {(a) $\Delta l = 0, \Delta \phi = 0$}; & \node (q2) {(b) $\Delta l = 5\text{mm}, \Delta \phi = 0$}; & \node (q3) {(c) $\Delta l = 0, \Delta \phi = 20^\circ$}; & \node (q4) {(d) $\Delta l = 5\text{mm}, \Delta \phi = 20^\circ$};

\\

&
\node[style={anchor=center}] {\includegraphics[width=\colWidth]{figures/photos/s0w0pic_colored.pdf}}; %\fill[blue] (0,0) circle (2pt);
&
\node[style={anchor=center}] {\includegraphics[width=\colWidth]{figures/photos/s5w0pic_colored.pdf}}; %\fill[blue] (0,0) circle (2pt);
&
\node[style={anchor=center}] {\includegraphics[width=\colWidth]{figures/photos/s0w20pic_colored.pdf}}; %\fill[blue] (0,0) circle (2pt);
&
\node[style={anchor=center}] {\includegraphics[width=\colWidth]{figures/photos/s5w20pic_colored.pdf}}; %\fill[blue] (0,0) circle (2pt);

\\

\node[rotate=90] (ylabel) {Moment, $M^{\hat{z}_e}$ (N-m)};
&
\node[style={anchor=center}] {\includegraphics[width=\colWidth]{figures/plots3/s0w0.pdf}}; %\fill[blue] (0,0) circle (2pt);
&
\node[style={anchor=center}] {\includegraphics[width=\colWidth]{figures/plots3/s5w0.pdf}}; %\fill[blue] (0,0) circle (2pt);
&
\node[style={anchor=center}] {\includegraphics[width=\colWidth]{figures/plots3/s0w20.pdf}}; %\fill[blue] (0,0) circle (2pt);
&
\node[style={anchor=center}] {\includegraphics[width=\colWidth]{figures/plots3/s5w20.pdf}}; %\fill[blue] (0,0) circle (2pt);

\\

& \node (xlabel1) {Force, $F^{\hat{z}_e}$ (N)}; & \node (xlabel2) {Force, $F^{\hat{z}_e}$ (N)}; & \node (xlabel3) {Force, $F^{\hat{z}_e}$ (N)}; & \node (xlabel4) {Force, $F^{\hat{z}_e}$ (N)};

\\
};
\end{tikzpicture}

\caption{For four different deformed configurations (top row), we compare the predicted and the measured forces for the parallel combination of three FREEs (bottom row). 
\revcomment{2.6}{Data points and predictions corresponding to the same input pressures are connected by a thin line, and the convex hull of the measured data points is outlined in black.}
The Trial 1 data is overlaid on top of the theoretical force zonotopes (grey areas) for each of the four configurations.
Identical colors indicate correspondence between a FREE and its resulting force/torque direction.}
\label{fig:results}
\end{figure*}






% & \node (a) {(a)}; & \node (b) {(b)}; & \node (c) {(c)}; & \node (d) {(d)};


For comparison, the measured forces are superimposed over the force zonotope generated by our model in Fig.~\ref{fig:results}a-~\ref{fig:results}d.
To quantify the accuracy of the model, we defined the error at the $j^{th}$ evaluation point as the difference between the modeled and measured forces
\begin{align}
%    \vec{e}_j &= \left( {\vec{\zeta}_{\,\text{mod}}} - {\vec{\zeta}_{\,\text{emp}}} \right)_j
%    e_j &= \left( F/M_{\,\text{mod}} - F/M_{\,\text{emp}} \right)_j
    e^F_j &= \left( F^{\hat{z}_e}_{\text{pred}, j} - F^{\hat{z}_e}_{\text{meas}, j} \right) \\
    e^M_j &= \left( M^{\hat{z}_e}_{\text{pred}, j} - M^{\hat{z}_e}_{\text{meas}, j} \right)
\end{align}
and evaluated the error across all $N = 125$ trials of a given end effector configuration.
% using the following metrics:
% \begin{align}
%     \text{RMSE} &= \sqrt{ \frac{\sum_{j=1}^{N} e_j^2}{N} } \\
%     \text{Max Error} &= \max \{ \left| e_j \right| : j = 1, ... , N \}
% \end{align}
As shown in Table \ref{table:RMSE}, the root-mean-square error (RMSE) is less than \unit[1.5]{N} (\unit[${8 \times 10^{-3}}$]{Nm}), and the maximum error is less than \unit[3]{N}  (\unit[${19 \times 10^{-3}}$]{Nm}) across all trials and configurations.

\begin{table}[H]
\centering
\caption{Root-mean-square error and maximum error}
\begin{tabular}{| c | c || c | c | c | c|}
    \hline
     & \rule{0pt}{2ex} \textbf{Disp.} & \multicolumn{2}{c |}{\textbf{RMSE}} & \multicolumn{2}{c |}{\textbf{Max Error}} \\ 
     \cline{2-6}
     & \rule{0pt}{2ex} (mm, $^\circ$) & F (N) & M (Nm) & F (N) & M (Nm) \\
     \hline
     \multirow{4}{*}{\rotatebox[origin=c]{90}{\textbf{Trial 1}}}
     & 0, 0 & 1.13 & $3.8 \times 10^{-3}$ & 2.96 & $7.8 \times 10^{-3}$ \\
     & 5, 0 & 0.74 & $3.2 \times 10^{-3}$ & 2.31 & $7.4 \times 10^{-3}$ \\
     & 0, 20 & 1.47 & $6.3 \times 10^{-3}$ & 2.52 & $15.6 \times 10^{-3}$\\
     & 5, 20 & 1.18 & $4.6 \times 10^{-3}$ & 2.85 & $12.4 \times 10^{-3}$ \\  
     \hline
     \multirow{4}{*}{\rotatebox[origin=c]{90}{\textbf{Trial 2}}}
     & 0, 0 & 0.93 & $6.0 \times 10^{-3}$ & 1.90 & $13.3 \times 10^{-3}$ \\
     & 5, 0 & 1.00 & $7.7 \times 10^{-3}$ & 2.97 & $19.0 \times 10^{-3}$ \\
     & 0, 20 & 0.77 & $6.9 \times 10^{-3}$ & 2.89 & $15.7 \times 10^{-3}$\\
     & 5, 20 & 0.95 & $5.3 \times 10^{-3}$ & 2.22 & $13.3 \times 10^{-3}$ \\  
     \hline
\end{tabular}
\label{table:RMSE}
\end{table}

\begin{figure}
    \centering
    \includegraphics[width=\linewidth]{figures/photos/buckling.pdf}
    \caption{At high fluid pressure the FREE with fiber angle of $-85^\circ$ started to buckle.  This effect was less pronounced when the system was extended along the $z$-axis.}
    \label{fig:buckling}
\end{figure}

%Experimental precision was limited by unmodeled material defects in the FREEs, as well as sensor inaccuracy. While the commercial force and moment sensors used have a quoted accuracy of 0.02\% for the force sensor and 0.2\% for the moment sensor (LoadStar Sensors, 2015), a drifting of up to 0.5 N away from zero was noticed on the force sensor during testing.

It should be noted, that throughout the experiments, the FREE with a fiber angle of $-85^\circ$ exhibited noticeable buckling behavior at pressures above $\approx$ \unit[50]{kPa} (Fig.~\ref{fig:buckling}). 
This behavior was more pronounced during testing in the non-extended configurations (Fig.~\ref{fig:results}a~and~\ref{fig:results}c). 
The buckling might explain the noticeable leftward offset of many of the points in Fig.~\ref{fig:results}a and Fig.~\ref{fig:results}c, since it is reasonable to assume that buckling reduces the efficacy of of the FREE to exert force in the direction normal to the force sensor. 

\begin{figure}
    \centering
    \includegraphics[width=\linewidth]{figures/zntp_vs_x4.pdf}
    \caption{A visualization of how the \emph{force zonotope} of the parallel combination of three FREEs (see Fig.~\ref{fig:rig}) changes as a function of the end effector state $x$. One can observe that the change in the zonotope ultimately limits the work-space of such a system.  In particular the zonotope will collapse for compressions of more than \unit[-10]{mm}.  For \revcomment{2.5}{scale and comparison, the convex hulls of the measured points from Fig.~\ref{fig:results}} are superimposed over their corresponding zonotope at the configurations that were evaluated experimentally.}
    % \marginnote{\#2.5}
    \label{fig:zntp_vs_x}
\end{figure}

\section{Discussion}
\label{conclusion}
% \vspace{-0.5em}
\section{Conclusion}
% \vspace{-0.5em}
Recent advances in multimodal single-cell technology have enabled the simultaneous profiling of the transcriptome alongside other cellular modalities, leading to an increase in the availability of multimodal single-cell data. In this paper, we present \method{}, a multimodal transformer model for single-cell surface protein abundance from gene expression measurements. We combined the data with prior biological interaction knowledge from the STRING database into a richly connected heterogeneous graph and leveraged the transformer architectures to learn an accurate mapping between gene expression and surface protein abundance. Remarkably, \method{} achieves superior and more stable performance than other baselines on both 2021 and 2022 NeurIPS single-cell datasets.

\noindent\textbf{Future Work.}
% Our work is an extension of the model we implemented in the NeurIPS 2022 competition. 
Our framework of multimodal transformers with the cross-modality heterogeneous graph goes far beyond the specific downstream task of modality prediction, and there are lots of potentials to be further explored. Our graph contains three types of nodes. While the cell embeddings are used for predictions, the remaining protein embeddings and gene embeddings may be further interpreted for other tasks. The similarities between proteins may show data-specific protein-protein relationships, while the attention matrix of the gene transformer may help to identify marker genes of each cell type. Additionally, we may achieve gene interaction prediction using the attention mechanism.
% under adequate regulations. 
% We expect \method{} to be capable of much more than just modality prediction. Note that currently, we fuse information from different transformers with message-passing GNNs. 
To extend more on transformers, a potential next step is implementing cross-attention cross-modalities. Ideally, all three types of nodes, namely genes, proteins, and cells, would be jointly modeled using a large transformer that includes specific regulations for each modality. 

% insight of protein and gene embedding (diff task)

% all in one transformer

% \noindent\textbf{Limitations and future work}
% Despite the noticeable performance improvement by utilizing transformers with the cross-modality heterogeneous graph, there are still bottlenecks in the current settings. To begin with, we noticed that the performance variations of all methods are consistently higher in the ``CITE'' dataset compared to the ``GEX2ADT'' dataset. We hypothesized that the increased variability in ``CITE'' was due to both less number of training samples (43k vs. 66k cells) and a significantly more number of testing samples used (28k vs. 1k cells). One straightforward solution to alleviate the high variation issue is to include more training samples, which is not always possible given the training data availability. Nevertheless, publicly available single-cell datasets have been accumulated over the past decades and are still being collected on an ever-increasing scale. Taking advantage of these large-scale atlases is the key to a more stable and well-performing model, as some of the intra-cell variations could be common across different datasets. For example, reference-based methods are commonly used to identify the cell identity of a single cell, or cell-type compositions of a mixture of cells. (other examples for pretrained, e.g., scbert)


%\noindent\textbf{Future work.}
% Our work is an extension of the model we implemented in the NeurIPS 2022 competition. Now our framework of multimodal transformers with the cross-modality heterogeneous graph goes far beyond the specific downstream task of modality prediction, and there are lots of potentials to be further explored. Our graph contains three types of nodes. while the cell embeddings are used for predictions, the remaining protein embeddings and gene embeddings may be further interpreted for other tasks. The similarities between proteins may show data-specific protein-protein relationships, while the attention matrix of the gene transformer may help to identify marker genes of each cell type. Additionally, we may achieve gene interaction prediction using the attention mechanism under adequate regulations. We expect \method{} to be capable of much more than just modality prediction. Note that currently, we fuse information from different transformers with message-passing GNNs. To extend more on transformers, a potential next step is implementing cross-attention cross-modalities. Ideally, all three types of nodes, namely genes, proteins, and cells, would be jointly modeled using a large transformer that includes specific regulations for each modality. The self-attention within each modality would reconstruct the prior interaction network, while the cross-attention between modalities would be supervised by the data observations. Then, The attention matrix will provide insights into all the internal interactions and cross-relationships. With the linearized transformer, this idea would be both practical and versatile.

% \begin{acks}
% This research is supported by the National Science Foundation (NSF) and Johnson \& Johnson.
% \end{acks}

\section*{Acknowledgements}
We thank Kilian Weinberger, Iasonas Kokkinos, Changhan Wang, and the entire Facebook AI Research team for helpful discussions and code support.

\bibliography{citations}
\bibliographystyle{iclr2018_conference}

\end{document}
