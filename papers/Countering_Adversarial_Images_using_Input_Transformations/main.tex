\documentclass{article} % For LaTeX2e
\usepackage{iclr2018_conference,times}
\usepackage{hyperref}
\usepackage{url}
\usepackage{epsfig}
\usepackage{graphicx}
\usepackage{subcaption}

% Custom stuff
\usepackage{amsmath,amsthm,verbatim,amssymb,amsfonts,amscd,mathrsfs}
\usepackage[inline]{enumitem}
\usepackage{color}
\usepackage{wrapfig}
\usepackage{booktabs}
\usepackage{dsfont}
\usepackage{dirtytalk}
\input com_notations.tex

\newcommand{\apriori}{\emph{a priori }}
\newcommand{\aposteriori}{\emph{a posteriori }}

% observations
\newcommand{\ALLobs}{\boldsymbol{\calY}}
\newcommand{\allobs}{\boldsymbol{\mathtt{z}}}
\newcommand{\MATobs}[1]{\bfZ_{#1}}
\newcommand{\Vobs}[1]{\bfz_{#1}}

% unknown image
\newcommand{\MATima}{\bfX}
\newcommand{\Vima}{\bfx}
\newcommand{\ima}[1]{x_{#1}}

% transformations
\newcommand{\ftrans}[2]{\calF_{#1}\left(#2\right)}
\newcommand{\MATtrans}[1]{\bfF_{#1}}
\newcommand{\kernel}[1]{\boldsymbol{\kappa}_{#1}}

% noise vector
\newcommand{\MATnoise}[1]{\bfE_{#1}}
\newcommand{\Vnoise}[1]{\bfe_{#1}}
\newcommand{\noisevar}[1]{{s^2_{#1}}}
\newcommand{\Vnoisevar}{\bss^2}

% dimensions
\newcommand{\noobs}{p}
\newcommand{\nbobs}{P}

\newcommand{\nbrowobs}[1]{n_{\mathrm{x},#1}}
\newcommand{\nbcolobs}[1]{n_{\mathrm{y},#1}}
\newcommand{\nbbandobs}[1]{n_{\lambda,#1}}
\newcommand{\nbpixobs}[1]{N_{#1}}

\newcommand{\nbrowima}{m_{\mathrm{x}}}
\newcommand{\nbcolima}{m_{\mathrm{y}}}
\newcommand{\nbbandima}{m_{\lambda}}
\newcommand{\nbpixima}{M}


\newcommand{\imam}[1]{\boldsymbol{\mu}_{#1}}
%\newcommand{\imamean}{\boldsymbol{\mu}_{\Vima}}
\newcommand{\meansub}{\imam{\bsu}}
\newcommand{\Covsub}{\boldsymbol{\Sigma}}
\newcommand{\imamall}{\imam{\bfu}^{\star}}
\newcommand{\imacovall}{\imacovmat{\bfu}^{\star}}
%\newcommand{\imam_whole}{\imam{\bfu}^{\star}}

\newcommand{\imacovmat}[1]{\boldsymbol{\Sigma}_{#1}}
\newcommand{\covmat}[1]{\boldsymbol{\Sigma}_{#1}} 
\newcommand{\CovEstHS}{\hat{\boldsymbol{\Sigma}}_{\bsz_{1}}}

%Parameters for GMM prior
\newcommand{\imacoefgm}{\alpha}
\newcommand{\imahid}{\boldsymbol{\zeta}}
\newcommand{\nei}{\boldsymbol{\nu}}

%% unknown (hyper)parameter vector
\newcommand{\hypervect}{\boldsymbol{\Phi}}
\newcommand{\paramvect}{\boldsymbol{\theta}}
%
%\newcommand{\Valpha}{\boldsymbol{\alpha}}
%\newcommand{\Vomega}{\boldsymbol{\omega}}
%
% samples
\newcommand{\sample}[2]{\tilde{#1}^{#2}}
\newcommand{\samplebis}[2]{{#1}^{\left(#2\right)}}
\newcommand{\samplenoisevar}[1]{{\widetilde{\sigma}}^{2(#1)}}
\newcommand{\sampleparamvect}[1]{\widetilde{\paramvect}^{(#1)}}


\newcommand{\MAP}[1]{\hat{#1}_{\mathrm{MAP}}}
\newcommand{\MMSE}[1]{\hat{#1}_{\mathrm{MMSE}}}
\newcommand{\argmax}{\mathrm{arg}\max}
\newcommand{\argmin}{\mathrm{arg}\min}

% norm
\newcommand{\norm}[1]{\left\|#1\right\|}

% R set
\newcommand{\R}{\mathds{R}}
%\newcommand{\R}{\mathbb{R}}
%\newcommand{\R}{\mbox{\rm I\kern-.25em R}}  % This is for use with cmr11
%\newcommand{\R}{\mbox{\rm I\kern-.2em R}}


\newcommand{\dirac}[1]{\delta\left({#1}\right)}

% inverse
\newcommand{\inv}{^{-1}}

% hermitian operator
\newcommand{\herm}{^{*}}

% transpose
\newcommand{\transp}{^T}

% exponential trace
\newcommand{\etr}{\mathrm{etr}}


\newcommand{\Ndistr}[1]{\mathcal{N}\left(#1\right)}

% vecteur un
\newcommand{\Vun}{{\boldsymbol{1}}}
% vecteur nul
\newcommand{\Vzero}[1]{\boldsymbol{0}_{#1}}
\newcommand{\Vzeros}[1]{\boldsymbol{0}_{#1}}
% identit�
\newcommand{\Id}[1]{\textbf{I}_{#1}}
% fonction indicatrice
%\newcommand{\Indicfun}[2]{\mathbb{I}_{#1}(#2)}
\newcommand{\Indicfun}[2]{\textbf{1}_{#1}\left(#2\right)}
\newcommand{\Diag}[2]{\left[#1\right]_{#2}}




% d�finition de l'environnement algo
\newenvironment{algogo}[1]{
\smallskip
\noindent \hrule\vspace{0.2\baselineskip} \hrule
\smallskip
\begin{small}
\refstepcounter{algo} \center{\bf \textsc{Algorithm \thealgo:}}
\\{\center{\bf #1}}
\smallskip
\flushleft
 } {
\end{small}
\bigskip
\hrule\vspace{0.2\baselineskip} \hrule
%\bigskip
\smallskip }

\newcounter{algo}
\renewcommand{\thealgo}{\arabic{algo}}
\theoremstyle{plain}
\newtheorem{definition}{Definition}
\newtheorem{theorem}{Theorem}
\newtheorem{corollary}{Corollary}
\newtheorem{lemma}{Lemma}
\newtheorem{proposition}{Proposition}
\newtheorem{claim}{Claim}
\newtheorem{conjecture}{Conjecture}
\newtheorem{problem}{Problem}

\theoremstyle{remark}
\newtheorem{remark}{\textbf{Remark}} 
\newtheorem{example}{\textbf{Example}} 
\newtheorem{question}{\textbf{Question}} 
\newtheorem*{note}{\textbf{Note}}
\newtheorem*{mistake}{\textbf{Mistake}}

\usepackage{csquotes}

\title{Countering Adversarial Images\\using Input Transformations}

% Authors must not appear in the submitted version. They should be hidden
% as long as the \iclrfinalcopy macro remains commented out below.
% Non-anonymous submissions will be rejected without review.

\author{Chuan Guo\thanks{This work was performed whilst Chuan Guo was at Facebook AI Research.}\\
Cornell University\\
%\texttt{cg563@cs.cornell.edu}\\
\And
Mayank Rana \& Moustapha Ciss{\'e} \& Laurens van der Maaten\\
Facebook AI Research\\
%\texttt{\{mayankrana, moustaphacisse, lvdmaaten\}@fb.com}
}

% The \author macro works with any number of authors. There are two commands
% used to separate the names and addresses of multiple authors: \And and \AND.
%
% Using \And between authors leaves it to \LaTeX{} to determine where to break
% the lines. Using \AND forces a linebreak at that point. So, if \LaTeX{}
% puts 3 of 4 authors names on the first line, and the last on the second
% line, try using \AND instead of \And before the third author name.

\newcommand{\fix}{\marginpar{FIX}}
\newcommand{\new}{\marginpar{NEW}}

\iclrfinalcopy % Uncomment for camera-ready version, but NOT for submission.

\begin{document}

\maketitle

\begin{abstract}
This paper investigates strategies that defend against adversarial-example attacks on image-classification systems by transforming the inputs before feeding them to the system. Specifically, we study applying image transformations such as bit-depth reduction, JPEG compression, total variance minimization, and image quilting before feeding the image to a convolutional network classifier. Our experiments on ImageNet show that total variance minimization and image quilting are very effective defenses in practice, in particular, when the network is trained on transformed images. The strength of those defenses lies in their non-differentiable nature and their inherent randomness, which makes it difficult for an adversary to circumvent the defenses. \emph{Our best defense eliminates $60\%$ of strong gray-box and $90\%$ of strong black-box attacks by a variety of major attack methods.}
\end{abstract}

\section{Introduction}
\label{introduction}
\section{Introduction}  \label{sec:introduction}

\newcommand\inexpIntro[3]{#1?(#2,#3).}
\newcommand\rinexpIntro[3]{*#1?(#2,#3).}
\newcommand\outexpIntro[3]{#1!(#2,#3).}
\newcommand\outatomIntro[3]{#1!(#2,#3)}

We propose a fully automated method for proving termination of \(\pi\)-calculus processes.
Although there have been a lot of studies on termination analysis for the \(\pi\)-calculus
and related calculi~\cite{Deng06IC,Demangeon07,SangiorgiTermination,KobayashiHybrid,Yoshida04IC,DBLP:journals/jlp/DemangeonHS10,Venet98SAS}, most of them have been rather theoretical,
and there have been surprisingly little efforts in developing  fully automated termination
verification methods and tools based on them. To our knowledge,
Kobayashi's \typical{}~\cite{TyPiCal,KobayashiHybrid} is the only exception that
can prove termination of \(\pi\)-calculus processes (extended with natural numbers)
fully automatically, but its termination analysis is quite limited (see Section~\ref{sec:relatedwork}).

Our method is based on a reduction to termination analysis for sequential programs:
we translate a \(\pi\)-calculus process \(P\) to a sequential program \(S_P\), so that
if \(S_P\) is terminating, so is \(P\). The reduction allows us to use
powerful, mature methods and tools
for termination analysis of sequential programs~\cite{heizmann2016ultimate,freqterm,DBLP:conf/lics/PodelskiR04,Kuwahara2014Termination,DBLP:journals/cacm/CookPR11}.

The idea of the translation is to convert a chain of communications on replicated input
channels to a chain of recursive function calls of the target sequential program.
Let us consider the following Fibonacci process:
\begin{align*}
    & \rinexpIntro{\fib}{n}{r}
        \ifexp{n<2}{ \soutatom{r}{1} \\ &\quad}
                   { \nuexp{s_1} \nuexp{s_2} (\outatomIntro{\fib}{n-1}{s_1} \PAR \outatomIntro{\fib}{n-2}{s_2} \PAR \sinexp{s_1}{x}\sinexp{s_2}{y}\soutatom{r}{x+y}) \\}
    & \PAR \outatomIntro{\fib}{m}{r}
\end{align*}
Here, the process
$\rinexpIntro{\fib}{n}{r} \ldots$ is a function server that computes the \(n\)-th Fibonacci number
in parallel and returns the result to \(r\),
and $\outatom{\fib}{m}{r}$ sends a request for computing the \(m\)-th Fibonacci number;
those who are not familiar with the syntax of the \(\pi\)-calculus may wish to consult
Section~\ref{sec:targetlanguage} first.
To prove that the process above is terminating for any integer \(m\),
it suffices to show that there is no infinite chain of communications on $\fib$:
\[
    \fib(m,r) \to \fib(m_1,r_1) \to \fib(m_2,r_2) \to \cdots.
\]
We convert the process above to the following program:\footnote{The actual translation
  given later is a little more complex.}
\begin{verbatim}
 let rec fib(n) = if n<2 then () else (fib(n-1) [] fib(n-2)) in
 fib(m)
\end{verbatim}
Here, \texttt{[]} represents the non-deterministic choice.
Note that, although the calculation of Fibonacci numbers is not preserved,
for each chain of communications on \texttt{fib}, there is a corresponding
sequence of recursive calls:
\[
\mathtt{fib}(m) \to \mathtt{fib}(m_1) \to \mathtt{fib}(m_2) \to \cdots.
\]
Thus, the termination of the sequential program above implies the termination of
the original process.
As shown in the example above, (i) each communication on a replicated input channel
is converted to a function call, (ii) each communication on a non-replicated input
channel is just removed (or, in the actual translation, replaced by a call of
a trivial function defined by \(f(\seq{x})=(\,)\)), and (iii) parallel composition
is replaced by a non-deterministic choice.
We formalize the translation outlined above and prove its correctness.

The basic translation sketched above sometimes loses too much information.
For example, consider the following process:
\begin{align*}
    & \rinexpIntro{\pre}{n}{r} \soutatom{r}{n-1} \\
    & \PAR \rinexpIntro{f}{n}{r} \ifexp{n<0}{ \soutatom{r}{1} }
                                       { \nuexp{s} (\outatomIntro{\pre}{n}{s} \PAR \sinexp{s}{x}\outatomIntro{f}{x}{r}) } \\
    & \PAR \outatomIntro{f}{m}{r}
\end{align*}
The translation sketched above would yield:
\begin{verbatim}
  let pred(n) = n-1 in
  let rec f(n) = if n<0 then () else (pred(n) [] f(*)) in
  f(m)
\end{verbatim}
Here, \texttt{*} represents a non-deterministic integer: since we have removed
the input $\sinatom{s}{x}$, we do not have information about the value of \( x \).
As a result, the sequential program above is non-terminating, although the original
process is terminating.
To remedy this problem, we also refine the basic translation above by using a refinement
type system for the \(\pi\)-calculus. Using the refinement type system,
we can infer that the value of \(x\) in the original process is less than \(n\),
so that we can refine the definition of \texttt{f} to:
\begin{verbatim}
 let rec f(n) = ... else (pred(n) [] let x=* in assume(x<n);f(x))
\end{verbatim}
The target program is now terminating, from which
we can deduce that the original process is also terminating.
We have implemented an automated tool based on the refined translation above.

The contributions of this paper are summarized as follows.
\begin{itemize}
\item The formalization of the basic translation from the \(\pi\)-calculus
  (extended with integers) to sequential programs, and a proof of its correctness.
\item The formalization of a refined translation based on a refinement type system.
\item An implementation of the refined translation, including automated refinement type
  inference based on CHC solving, and experiments to evaluate the effectiveness of
  our method.
\end{itemize}

The rest of this paper is structured as follows.
Section~\ref{sec:targetlanguage} introduces the source and target languages
of our translation.
Section~\ref{sec:approach} 
formalizes the basic translation, and proves its correctness.
Section~\ref{sec:refinement} refines the basic translation by using a refinement type system.
Section~\ref{sec:implementation} reports an implementation and experiments.
Section~\ref{sec:relatedwork} discusses related work,
and Section~\ref{sec:conclusion} concludes the paper.


\section{Problem Definition}
\label{definition}
\section{Preliminaries on Matroids and Submodular Functions}\label{sec:definition}

\paragraph{Continuous extensions and the correlation gap of submodular functions.}

Consider any set function $f : 2^{\A} \rightarrow \mathbb{R}_{\geq 0}$ over a ground set $\A$. Recall that $f$ is submodular, if $\forall S,T \subseteq \A$ we have $f(S \cup T) + f(S \cap T) \leq f(S) + f(T)$. For any point $\x \in [0,1]^k$, we denote by $S \sim \x$ the random set $S \subseteq \A$, such that $\Pro{i \in S} = x_i$. We consider two canonical continuous extensions of a set function: 

\begin{definition}[Continuous extensions]
For any set function $f$ the {\em multi-linear extension} is
\begin{align*}
    F(\x) = \Ex{S \sim \x}{f(S)} = \sum_{S \subseteq \A} f(S) \prod_{i \in S} x_i \prod_{i \notin S} (1-x_i).
\end{align*}
Moreover, the {\em concave closure} is defined as
\begin{align*}
    f^+(\x) = \max_{\alpha}\{\sum_{S \subseteq \A} \alpha_S f(S)~|~\sum_{S \subseteq \A} \alpha_S \bm{1}_S = \x, \sum_{S \subseteq \A} \alpha_S = 1, \alpha \succeq 0\}.
\end{align*}
\end{definition}

\begin{lemma}[Correlation gap \cite{CCPV07}] \label{lem:correlationgap}
Let $f: 2^k \rightarrow \mathbb{R}_{\geq 0}$ be a monotone (non-decreasing) submodular function. Then for any point $\x \in [0,1]^k$, we have
$
F(\x) \leq f^+(\x) \leq \left(1 - \frac{1}{e}\right)^{-1} F(\x).
$
\end{lemma}


\paragraph{Matroid polytope and the weighted rank function.} Consider a matroid $\M = (\A, \I)$, where $\A$ is the {\em ground set} and $\I$ is the family of {\em independent sets}
\footnote{Any subset of the ground set $\A$ that is not independent is called {\em dependent}. Any maximal independent set of a matroid, namely, a set $B \in \I$ such that for every $e \in \A \backslash B$, the set $B \cup \{e\}$ is dependent, is called a {\em basis}. Any minimal dependent set, that is, a set $C \notin \I$ such that for each $e \in C$ it holds $C\backslash \{e\} \in \I$ is called a {\em circuit}.}. Recall that in any matroid, the family $\I$ satisfies the following two properties: (i) Every subset of an independent set (including the empty set) is an independent set, namely, if $S' \subset S \subseteq \A$ and $S \in \I$, then $S' \in \I$ ({\em hereditary property}). (ii) Let $S, S' \subseteq \A$ be two independent sets with $|S| < |S'|$, then there exists some $e \in S' \backslash S$ such that $S \cup \{e\} \in \I$ ({\em augmentation property}). See \cite{schrijver03, oxley06} for more details on matroids.

We assume that access to $\M$ is given through an {\em independence oracle} \cite{hausmann81, robinson80}, namely, a black-box routine that, given a set $S \subseteq \A$, answers whether $S$ is an independent set of $\M$. 
For any set $R \subset \A$ we define the {\em restriction} of $\M$ to $R$, denoted by $\M | R$, to be the matroid $\M | R = (R, \{I \in \I~|~I \subseteq R\})$. Every matroid $\M$ is associated with a {\em rank} function\footnote{The rank is {\em monotone} non-decreasing, {\em submodular}, and satisfies $\rk(S) \leq |S|$, $\forall S \subseteq \A$ (see \cite{oxley06}).} $\rk: 2^{\A} \rightarrow \mathbb{N}$, such that for any $S\subseteq \A$, $\rk(S)$ denotes the maximum size of an independent set contained in $S$. Let $\x(S) = \sum_{e \in S} x_e$ for some vector $\x \in \mathbb{R}^k$. For any matroid $\M = (\A,\I)$, the {\em matroid polytope} is defined as

$$\mathcal{P}(\M) \equiv \left\{ \x(S) \leq \rk(S), \forall S \in 2^{\A}, S \neq \emptyset \text{ and } \x \succeq 0 \right\}.
$$
It can be proved \cite{schrijver03} that the above polytope is the convex hull of the indicator vectors of all independent sets. This fact immediately leads to the following lemma: 

\begin{lemma}\label{lem:characteristic}
For any matroid $\M=(\A, \I)$ and point $\x \in \mathcal{P}(\M)$, there exists a collection of $k = |\A|$ independent sets $\I(\x) = \{ I_1, \dots, I_k\} \subseteq \I$ and a probability distribution over $\I(\x)$ such that $\Prob{I \sim \I(\x)}{i \in I} = x_i$, i.e., an element $i$ belongs to a sampled set with marginal probability equal to $x_i$.
\end{lemma}
Given any non-negative linear {\em weight} vector $\w \in \mathbb{R}^k_{\geq 0}$, the problem of computing a maximum weight independent set can be solved optimally by the standard greedy algorithm: Starting from the empty set $S = \emptyset$, add each ground element $e \in \A$ to the set $S$ in a non-increasing order of weights, as long as the set $S \cup \{e\}$ does not contain a circuit. Given a matroid $\M=(\A,\I)$ and a weight vector $\w$, the function $f_{\M,\w}(S) = \max_{I \in \I, I \subseteq S}\{\w(I)\}$ is called the {\em weighted rank function} of $\M$ and returns the weight of the maximum independent set of the restriction $\M|S$.

\begin{lemma}[Weighted rank function \cite{CCPV07}] \label{lem:weightedrank}
For any matroid $\M$ and non-negative weight vector $\w$, the function $f_{\M, \w}(S) = \max_{I \in \I, I \subseteq S}\{\w(I)\}$ is monotone (non-decreasing) submodular.
\end{lemma}




%\begin{lemma}[Correlation Gap] %\label{lem:correlationggap}
%Let $f$ be a monotone (non-decreasing) submodular function. For the sets $S$ and $T_1, \dots, T_k$ constructed as described above, we have:
%\begin{align*}
%    \Ex{S \sim \x}{f(S)} \geq \left(1 - \frac{1}{e}\right) \Ex{T \sim \I(\x)}{f(T)}.
%\end{align*}
%\end{lemma}

 

\section{Adversarial Attacks}
\label{attacks}
\section{Assumptions for security}
\label{sec:attacks}

The security of a quantum cryptographic protocol relies on assumptions about the physics of the devices that are employed to implement the protocol.  In this section, we discuss these assumptions. For concreteness, we focus on the case of QKD, for which we describe the full set of assumptions in \secref{sec:attacks:assumptionlist}.  We then explain why these assumptions are needed and to what extent they are justified in \secref{sec:attacks:necessity}. Experimental work in QKD has shown however that the assumptions are often very difficult to meet, and are actually not met in many cases. This fact can be exploited by quantum hacking attacks, which are described in \ref{sec:attacks:hacking}. Finally, in Section~\ref{sec:attacks:countermeasures}, we discuss countermeasures against these attacks. 

\subsection{Standard assumptions for QKD} \label{sec:attacks:assumptionlist}

The security of QKD protocols usually relies on the following assumptions.

\begin{enumerate}
  \item \label{item_qm} All devices used by Alice and Bob, as well as the communication channels connecting them, are correctly and completely\footnote{The completeness of quantum theory can be derived from their correctness;  see \secref{sec:completeness}.} described by quantum theory.
    \item \label{item_res} The channel that Alice and Bob use to exchange classical messages is authentic, i.e., it is impossible for an adversary to modify messages or insert new ones. 
  \item \label{item_conv} The devices that Alice and Bob use locally to execute the steps of the protocol, e.g., for preparing and measuring quantum systems, do exactly what they are instructed to do.
\end{enumerate}

As already indicated earlier, due to the lack of proof techniques, additional assumptions had been introduced in the past. A prominent example is the \emph{i.i.d.}\ assumption, which demands that the quantum channel connecting Alice and Bob be described by a sequence of identical and independently distributed maps. Physically, this means that an adversary's interception strategy is such that each signal sent from Alice to Bob is modified in the same manner and independently of the other signals. Security under the i.i.d.\ assumption is called security against \emph{collective attacks}~\cite[see also \secref{sec:qkd.other.models}]{BM97b,BBBvdGM02}. Another assumption, which  usually comes on top of the i.i.d.\ assumption, is that Eve only stores classical data, which she obtains by individually measuring the pieces of information she gained from each signal sent from Alice to Bob. Since it is difficult to argue why an adversary should be restricted in that particular way, the corresponding security guarantee is rather weak. It is usually referred to as security against \emph{individual attacks}~\cite[see \secref{sec:qkd.other.models}]{Fuchsetal1997,Lutkenhaus2000}. 

Most modern security proofs do however not require such additional assumptions, i.e., they are based entirely on Assumptions~\ref{item_qm}--\ref{item_conv} above. This means, in particular, that the quantum channel connecting Alice and Bob can be arbitrary, and may even be entirely controlled by Eve. In this case, one talks about security against \emph{general attacks}, \emph{coherent attacks}, or \emph{joint attacks}. Sometimes the term  \emph{unconditional security} appeared in the literature~\cite{SBCDLP09}, but it is important to keep in mind that the assumptions listed above are still necessary.

\subsection{Necessity and justification of assumptions} \label{sec:attacks:necessity}

Assumption~\ref{item_qm} is often implicit, for it is a prerequisite to even describe the cryptographic scheme. It justifies the use of the formalism of quantum theory to model the different systems, such as the communication channel, including any possible attacks on them. The assumption thus captures the main idea behind quantum cryptography, namely that an adversary is limited by the laws of quantum theory.  The other two assumptions ensure that the experimental implementation follows the theoretical prescription that enters the security definition (Definition~\ref{def:security}), namely the description of the protocol $\pi_{AB}$ and the used resources. In particular, Assumption~\ref{item_res} guarantees that the resources shared between Alice and Bob fulfil the theoretical specifications~$\aR$, which in the case of QKD includes the classical authentic communication channel. Assumption~\ref{item_conv} guarantees that the steps prescribed by the protocol~$\pi_{A B}$ are correctly executed.

Assumption~\ref{item_qm} is widely accepted \--- and proving it wrong would represent a major breakthrough in physics. Nevertheless, it has been shown that there exist QKD protocols that only rely on the weaker assumption of \emph{no-signalling}~\cite{BHK05}.  

Assumption~\ref{item_res} demands that an authentic communication channel is set up between Alice and Bob. There exist information-theoretically secure protocols that achieve this, provided that Alice and Bob share a weak secret key~\cite[see also \secref{sec:smt.auth}]{RW03,DW09,ACLV19}.  Assumption~\ref{item_res} can thus be met by the use of such authentication protocols (see also~\secref{sec:intro} as well as standard textbooks on classical cryptography)

Although Assumption~\ref{item_conv} sounds rather natural, and is in fact required for almost any cryptographic scheme, including any classical one, it is rather challenging to meet.  Numerous quantum hacking experiments, which have been conducted over the past few years, have shown that many implementations of QKD failed to satisfy this assumption. To illustrate this problem, we describe selected examples of such attacks in the following subsection.

\subsection{Quantum hacking attacks} \label{sec:attacks:hacking}          

We start with the \emph{photon number splitting attack}~\cite{Brassardetal2000}, which targets optical implementations of QKD that use individual photons as quantum information carriers. Suppose, for concreteness, that Alice and Bob implement the BB84 protocol~\cite{BB84} by encoding the qubits into the polarisation degree of freedom of individual photons. Specifically, Alice may use a single-photon source that emits photons with a polarisation that she can choose. The BB84 protocol\footnote{This protocol is explained in more detail in \secref{sec:securityproofs}, where a security proof is also sketched.} requires her to send in each round at random a state from one orthonormal basis, say $\{\ket{h}, \ket{v}\}$, where $\ket{h}$ may be realised by a horizontally polarised photon and $\ket{v}$ by a vertically polarised one, or from a complementary basis $\{\ket{d^+}, \ket{d^-}\}$, where $\ket{d^+} = \smash{\frac{1}{\sqrt{2}}} (\ket{h} + \ket{v})$ and $\ket{d^-} = \smash{\frac{1}{\sqrt{2}}} (\ket{h} - \ket{v})$. It may now happen that, in an experimental implementation, the source sometimes accidentally emits two photons at once, which then carry the same polarisation. The states emitted in the four cases are thus $\ket{h} \otimes \ket{h}$, $\ket{v} \otimes \ket{v}$, $\ket{d^+} \otimes \ket{d^+}$, and $\ket{d^-} \otimes \ket{d^-}$.

Before describing the actual attack, we first give a simple information-theoretic argument for why this is problematic. Note first that one single photon carries no information about the choice of the basis made by Alice. Indeed, for either of the basis choices, the density operator describing the photon is maximally mixed, i.e., $\frac{1}{2} \proj{h} + \frac{1}{2} \proj{v} = \frac{1}{2} \proj{d^+} + \frac{1}{2} \proj{d^-} =  \frac{1}{2} \mathbf{1}$. This is however no longer the case for a pulse consisting of two photons, i.e., 
\begin{align}
  \frac{1}{2} \proj{h}^{\otimes 2} + \frac{1}{2} \proj{v}^{\otimes 2} \neq \frac{1}{2} \proj{d^+}^{\otimes 2} + \frac{1}{2} \proj{d^-}^{\otimes 2} \ .
\end{align}
Hence, if the source accidentally emits two equally polarised photons instead of one, it reveals information about Alice's basis choice, which it shouldn't. 

It is therefore not surprising that such two-photon pulses can be exploited by an adversary to attack the system. Eve, who intercepts the channel, may split the two-photon pulse into two, keep one of the photons and send the other one to Bob. The latter thus receives photons in exactly the way prescribed by the protocol, and hence does not notice the interception. Eve, meanwhile, may measure the photons she captured. In principle, if Eve had quantum memory, she could even wait with the measurement until Alice announces the basis choice to Bob, and hence always gain full information about the polarisation state that Alice prepared. 

While the photon number splitting attack exploits an imperfection of the sender (namely that it sometimes emits two identically polarised photons instead of one), many quantum attacks are targeted towards the receiver. An example is the \emph{time-shift attack}~\cite{Makarovetal2006,qi2007time,Zhaoetal2008}, which exploits inaccuracies of the photon detectors. In order to avoid dark counts, the photon detectors are often set up such that they only count photons that arrive within a small time window around the time when a signal is expected to arrive. Furthermore, Bob's receiver device may consist of more than one detector, e.g., one for each possible polarisation state. The time windows of the different detectors are then never perfectly synchronised. This means that there are times at which the receiver is more sensitive to signals with respect to one polarisation than another. Eve may therefore, by appropriately delaying the signals sent from Alice and Bob, bias the detected signals towards one or the other polarisation, and thus gain information about what Bob measures. While this information may be partial, it can, together with the error correction information that is available to Eve, be sufficient to infer the final key. 

Another attack that is targeted towards the receiver is the \emph{detector blinding attack} ~\cite{Makarov2009,WKRFNW11,LWWESM10,GLLSKM11}, where the adversary tries to control the detectors by illuminating them with bright laser light.  In a QKD implementation that uses the encoding of information into the polarisation of individual photons, the detectors are usually configured such they can optimally detect single photon pulses. That is, they should click whenever the incoming pulse contains a photon, and not click if the pulse is empty.  However, the behaviour of such detectors may be rather different in a regime where the incoming pulses contain many photons. For example, it could be that they always click when they are exposed to bright light with a particular intensity, and they may never click for another intensity. Hence, by sending in light with appropriately chosen polarisation and intensity, Eve may gain immediate control over the clicks of Bob's detector. To exploit this for an attack, Eve may mimic Bob's receiver, i.e., intercept the photons sent from Alice and measure them in a randomly chosen basis, as Bob would do. She then sends bright light to Bob to ensure that he obtains the same detector clicks as if he had directly obtained Alice's photons. This works particularly well for implementations that use a \emph{passive basis choice}, i.e., where Bob's measurement basis is not  provided as an input, but rather made by the detection device itself. In this case, an adversary can essentially remote-control Bob and thus get hold of the entire key. 

Yet another hacking strategy are \emph{Trojan-horse attacks}~\cite{Vakhitov2001,GisinFaselKraus2006}. Here the idea is to send a bright laser pulse via the optical fibre into Alice or Bob's component to extract information about its internal settings. Depending on the sender and receiver hardware which is used, measuring the reflection of the pulse can allow Eve, for instance, to determine the basis choices made by Alice and Bob.

In some optical implementations of QKD, e.g., in the \emph{plug-and-play}~\cite{Muller97} or the \emph{circular-type}~\cite{Nishioka2002} system, Alice does not have a photon source but instead encodes information by modulating an incoming signal from Bob before sending it back to him. The signal thus travels twice in opposite directions through the same optical links, which helps reducing fluctuations due to birefringence  and environmental noise. The two-fold use of the (insecure) channel however opens additional possibilities of attacks~\cite{GisinFaselKraus2006}. A prominent example is the \emph{phase-remapping attack}~\cite{FQTL07,XQL10}. It exploits the fact that the modulator used by Alice to encode information into the signal coming from Bob acts on that signal during a particular time interval. In the attack, the adversary slightly advances or delays the signal on its way from Bob to Alice, so that it no longer lies fully within that time interval. The modulation by Alice will then be incomplete, which means that the encoding of the information in the signal differs from what is foreseen by the protocol. This can in turn be exploited by Eve in an intercept-and-resend attack on the signal returned from Alice to Bob. 


\subsection{Countermeasures against quantum hacking} \label{sec:attacks:countermeasures}

The attacks described here have in common that they all exploit a breakdown of Assumption~\ref{item_conv}. Specifically, in the case of the photon-splitting attack, the device used by Alice sends out more information than it is supposed to. In the case of the time-shift attack, it is Bob's measurement device  whose measurement operators are not constant over time and can even be partially controlled by Eve. Finally, in the case of the detector blinding attack on systems with passive basis choice, Eve even takes over control of the randomness used to choose the basis.

A seemingly obvious countermeasure to prevent such attacks is to manufacture sources and detectors that meet the theoretical specifications. That is, one would need a perfect single-photon source, as well as detectors that are perfectly efficient and only measure photon pulses in a specified parameter regime. Such requirements are however unrealistic --- the devices used in experiments will always, at least slightly, deviate from these specifications. 

The other possibility is to develop cryptographic protocols and  security proofs that tolerate imperfections of the devices~\cite{GLLP04}.  This has been done in particular for the attacks described above. To prevent photon number splitting attacks, an efficient countermeasure is the \emph{decoy-state} method~\cite{Hwang2003,Wang2005,Loetal2005}. The idea here is that Alice sometimes deliberately sends multi-photon pulses. Alice and Bob can  then check statistically whether an adversary captured them. Another possibility is to use protocols where Alice's encoding of information has the property that, even when one photon is extracted from a pulse, the information about what Alice sent is still partial~\cite{SARG,TamakiLo,SYK14}. In the case of time-shift attacks, it is sufficient to characterise the maximum bias in the detector efficiencies that can be introduced and account for it in the security proofs. Finally, for the detector blinding attacks, a possible countermeasure is to add tests to the protocol, such as a monitoring of the photocurrent, in order to detect those~\cite{Yuanetal2010}.  

The main problem with such countermeasures is however that the space of possible imperfections is hard to characterise. The above are just a few examples of attacks, and many others have been proposed, and sometimes even demonstrated to work successfully in experiments. For example, an adversary may exploit imperfections in the randomness that Alice and Bob use for choosing their measurement basis. To prevent such attacks, one may again extend the protocols such that they can tolerate imperfect randomness (see \secref{sec:alternative.randomness}). 



The last decade has thus seen an arms race between designers and attackers of quantum cryptographic schemes. A possible way out of this unsatisfactory situation is \emph{device-independent cryptography}. Here the idea is to replace Assumption~\ref{item_conv} by something much weaker. Namely, one requires that the devices used by Alice and Bob do not unintentionally send information out to an adversary, and that the classical processing of information done by Alice and Bob is correct. Crucially, however,  one does no longer demand that the sources and detectors used by Alice and Bob work according to their specifications. The way this can work is explained in \secref{sec:alternative.di}. 

%%% Local Variables:
%%% TeX-master: "main.tex"
%%% End:


\section{Defenses}
\label{defenses}

% !TeX root = ../main.tex

% One explanation for the principle behind adversarial perturbations is that they alter the
% behavior of lower level convolutional filters that detect edges and textures. This misrepresented
% information propagates forward through the network and combine at the upper layers to produce
% features that are indistinguishable from the class that the example is designed to mimic.
% In summary, despite the per-pixel perturbation being relatively small, they combine across
% channels and spatial coordinates to fool the network.

% The intuition behind our defense mechanisms is to remove or replace pixels from the input image so that combination of per-pixel perturbations cannot occur. We investigate the effect of cropping, random pixel dropping, and image quilting on adversarial examples. \autoref{ladybug} shows a sample image and a corresponding adversarial image under these transformations.

Adversarial attacks alter particular statistics of the input image in order to change the model prediction. Indeed, adversarial perturbations $\bx \!-\! \bx'$ have a particular structure, as illustrated by Figure~\ref{samples}. We design and experiment with image transformations that alter the structure of these perturbations, and investigate whether the alterations undo the effects of the adversarial attack. We investigate five image transformations: (1) image cropping and rescaling, (2) bit-depth reduction, (3) JPEG compression, (4) total variance minimization, and (5) image quilting.

\begin{wrapfigure}{r}{0.5\textwidth}
    \vspace{-2em}
    \includegraphics[width=0.5\textwidth]{figures/ladybug.pdf}
    \caption{Illustration of total variance minimization and image quilting applied to an original and an adversarial image (produced using I-FGSM with $\epsilon \!=\! 0.03$, corresponding to a normalized $L_2$-dissimilarity of 0.075). From left to right, the columns correspond to: (1) no transformation, (2) total variance minimization, and (3) image quilting. From top to bottom, rows correspond to: (1) the original image, (2) the corresponding adversarial image produced by I-FGSM, and (3) the absolute difference between the two images above. Difference images were multiplied by a constant scaling factor to increase visibility.}
    \label{ladybug}
    \vspace{-2em}
\end{wrapfigure}

\subsection{Image cropping-rescaling, bit-depth reduction, and compression}

% \begin{figure}[t!]
%     \centering
%     \begin{subfigure}[t]{0.5\columnwidth}
%         \centering
%         \includegraphics[width=\textwidth]{figures/ifgs_0_0025_resnet50_crop_top1.png}
%     \end{subfigure}%
%     \begin{subfigure}[t]{0.5\columnwidth}
%         \centering
%         \includegraphics[width=\textwidth]{figures/ifgs_0_0250_resnet50_crop_top1.png}
%     \end{subfigure}
%     \caption{Effect of different crop and downsample ratios on adversarial examples.}
%     \label{crop-fig}
% \end{figure}

We first introduce three simple image transformations: image cropping-rescaling, bit-depth reduction \citep{xu2017feature}, and JPEG compression and decompression \citep{dziugaite2016study}. \emph{Image cropping-rescaling} has the effect of altering the spatial positioning of the adversarial perturbation, which is important in making attacks successful. Following \citet{he2016residual}, we crop and rescale images at training time as part of the data augmentation. At test time, we average predictions over random image crops. \emph{Bit-depth reduction} \citep{xu2017feature} perform a simple type of quantization that can removes small (adversarial) variations in pixel values from an image; we reduce images to $3$ bits in our experiments. \emph{JPEG compression} \citep{dziugaite2016study} removes small perturbations in a similar way; we perform compression at quality level $75$ (out of $100$).



% LAURENS: If we want to show this, this should move to the experimental section.
% \autoref{crop-fig} Shows the effectiveness of the random cropping for various values of $r$. The adversarial examples are generated by I-FGSM at $\epsilon = 0.001$ (left)
% and $\epsilon = 0.01$ (right) on a pretrained ResNet-50 model. The model achieves a Top-1
% validation accuracy of 76.02\% on clean images. At $r=1$, the adversarial image is unmodified and the corresponding
% accuracy reflects the base model's accuracy without any defense. For smaller perturbation, cropping is very effective even at higher values of $r$, which shows that these adversarial
% perturbations are very unstable against a few dropped pixels. At larger perturbation,
% cropping at much lower $r$ is the most effective by a large margin.

% We suspect that having a lower crop ratio always strictly increases effectiveness against
% adversarial examples. However, this comes at a large cost of losing the object of interest
% in crops of smaller size. To counter this effect, we can ensemble the prediction of different
% crops by averaging their class probability vectors. By taking the ensemble of 30 random crops,
% we have obtain a much higher accuracy on adversarial examples when using smaller crop ratios.

% \subsection{Pixel dropout}
% \label{pixel_dropout}

% \begin{figure}[t!]
%     \centering
%     \begin{subfigure}[t]{0.5\columnwidth}
%         \centering
%         \includegraphics[width=\textwidth]{figures/ifgs_0_0025_resnet50_drop_top1.png}
%     \end{subfigure}%
%     \begin{subfigure}[t]{0.5\columnwidth}
%         \centering
%         \includegraphics[width=\textwidth]{figures/ifgs_0_0250_resnet50_drop_top1.png}
%     \end{subfigure}
%     \caption{Effect of different random pixel drop rates on adversarial examples.}
%     \label{pixel-drop}
% \end{figure}

% Another method that breaks the structure of adversarial perturbations whilst keeping semantic content intact is  pixel dropout. As images channels are normalized to be zero-mean before they are used as input into the model, we implement pixel dropout by sampling a Bernoulli random variable $X(i, j, k)$ for each pixel location $(i, j, k)$ and replace the corresponding pixel value by $0$ whenever the draw equals $X(i, j, k) = 1$.

% LAURENS: Idem. If we want to show this, it should go in the experimental section.
% \autoref{pixel-drop} demonstrates the effect of random pixel drops on the same adversarial examples
% as in \autoref{crop-fig}. This defense effectively mitigates adversarial examples even when
% the drop rate is very small (e.g. $p = 0.05$). However, since the model does not have exposure to this type of transformation at training time, accuracy sharply deteriorates at larger $p$.


\subsection{Total variance minimization}
An alternative way of removing adversarial perturbations is via a compressed sensing approach that combines pixel dropout with total variation minimization \citep{rudin1992tvm}. This approach randomly selects a small set of pixels, and reconstructs the ``simplest'' image that is consistent with the selected pixels. The reconstructed image does not contain the adversarial perturbations because these perturbations tend to be small and localized.

% Because adversarial perturbations tend to be small and localized, compressed-sensing approaches that combine pixel dropout with total variation minimization are likely to remove these fine-scale perturbations without affecting the coarser-scale information in the image that contains most of its semantic information. The key idea of this approach is to randomly remove the majority of the pixel values from the image as to remove the adversarial perturbation, and then to reconstruct the image from the non-removed pixels. The reconstructed image likely does not contain much of the adversarial image.

Specifically, we first select a random set of pixels by sampling a Bernoulli random variable $X(i, j, k)$ for each pixel location $(i, j, k)$; we maintain a pixel when $X(i, j, k) = 1$. Next, we use total variation minimization to constructs an image $\bz$ that is similar to the (perturbed) input image $\bx$ for the selected set of pixels, whilst also being ``simple'' in terms of total variation by solving:
\begin{equation}
\label{tvcs}
\min_\bz \| (1-X) \odot (\bz - \bx) \|_2 + \lambda_{\tv} \cdot \tv_p(\bz).
\end{equation}
Herein, $\odot$ denotes element-wise multiplication, and $\tv_p(\bz)$ represents the $L_p$-total variation of $\bz$:
\begin{equation}
\label{lptv}
\tv_p(\bz) = \sum_{k=1}^{K} \left[ \sum_{i=2}^{N} \|\bz(i,:,k) - \bz(i-1,:,k)\|_p + \sum_{j=2}^{N} \|\bz(:,j,k) - \bz(:,j-1,k)\|_p \right].
\end{equation}
The total variation (TV) measures the amount of fine-scale variation in the image $\bz$, as a result of which TV minimization encourages removal of small (adversarial) perturbations in the image. The objective function~(\ref{tvcs}) is convex in $\bz$, which makes solving for $\bz$ straightforward. In our implementation, we set $p \!=\! 2$ and employ a special-purpose solver based on the split Bregman method
\citep{goldstein2009split} to perform total variance minimization efficiently.

The effectiveness of TV minimization is illustrated by the images in the middle column of Figure~\ref{ladybug}: in particular, note that the adversarial perturbations that were present in the background for the non-transformed image (see bottom-left image) have nearly completely disappeared in the TV-minimized adversarial image (bottom-center image). As expected, TV minimization also changes image structure in non-homogeneous regions of the image, but as these perturbations were not adversarially designed we expect the negative effect of these changes to be limited.

%When defending against adversaries that minimize the $L_\infty$-dissimiarity
%(such as FGSM and I-FGSM), it seems intuitive that the $L_2$ reconstruction loss in
%\autoref{tvcs} should be replaced by $L_\infty$. However, since optimizing the $L_\infty$-norm
%directly is intractable, we solve the equivalent constrained optimization problem instead.
%\begin{align}
%\label{tvinf}
%&\min_\bz \tv_p(\bz) \nonumber \\
%\text{s.t. } \bx(i,j,k) - \tau &\leq \bz(i,j,k) \leq \bx(i,j,k) + \tau \\
%&\hspace{36pt} \text{for all } X(i,j,k) = 0. \nonumber
%\end{align}
%This gives rise to a box-constrained convex optimization problem. However, due to the large number
%of variables, we chose to use L-BFGS with box constraint \cite{byrd1995limited} instead.

% LAURENS: This should move to experiments, also.

% We apply TV compressed sensing to the same adversarial examples as in \autoref{crop-fig}.
% We use isotropic TV minimization and choose $\lambda_{\tv} $ to optimize the visual quality
% of reconstruction images while removing a large portion of adversarial perturbation. As shown
% in \autoref{pixel-drop}, applying total variation reconstruction significantly improves accuracy
% compared to random pixel drops, especially for larger drop rates (e.g. $p \geq 0.8$).

\subsection{Image quilting}

% \begin{figure}[t!]
%     \centering
%     \includegraphics[width=0.5\textwidth]{figures/ifgs_resnet50_quilt_top1.png}
%     \caption{Effect of different patch size for quilting on adversarial examples.}
%     \label{quilting}
% \end{figure}

Image quilting \citep{efros2001quilting} is a non-parametric technique that synthesizes images by piecing together small patches that are taken from a database of image patches. The algorithm places appropriate patches in the database for a predefined set of grid points, and computes minimum graph cuts \citep{boykov2001graphcuts} in all overlapping boundary regions to remove edge artifacts.

Image quilting can be used to remove adversarial perturbations by constructing a patch database that only contains patches from ``clean'' images (without adversarial perturbations); the patches used to create the synthesized image are selected by finding the $K$ nearest neighbors (in pixel space) of the corresponding patch from the adversarial image in the patch database, and picking one of these neighbors uniformly at random. The motivation for this defense is that the resulting image only consists of pixels that were not modified by the adversary --- the database of real patches is unlikely to contain the structures that appear in adversarial images.

The right-most column of Figure~\ref{ladybug} illustrates the effect of image quilting on adversarial images. Whilst interpretation of these images is more complicated due to the quantization errors that image quilting introduces, it is interesting to note that the absolute differences between quilted original and the quilted adversarial image appear to be smaller in non-homogeneous regions of the image. This suggests that TV minimization and image quilting lead to inherently different defenses.

% \autoref{quilting} shows the effect of image quilting with various patch sizes. For smaller $b$, the model's accuracy on clean images is very high, but is not as robust against adversarial images. This is likely due to adversarial perturbation affecting the choice of the nearest neighbor from $\mathcal{B}$. For larger patch size, this is more difficult, but at the cost of image fidelity and accuracy on clean images.


\section{Experiments}
\label{experiment}
\newcommand{\twomoons}{{\tt Twomoons}}
\newcommand{\gauss}{{\tt Gauss}}
\newcommand{\sculpture}{{\tt Sculpture}}
\newcommand{\baseline}{{\tt Baseline}}
\newcommand{\MM}{{\tt MsgPassing}}
\newcommand{\blackboard}{{\tt Blackboard}}
\newcommand{\ncut}{\text{ncut}}
\newcommand{\chensays}[2][]{\textcolor{blue} {\textsc{Jiecao #1:} \emph{#2}}}

\section{Experiments}
In this section we present experimental results for  graph clustering in the message passing and blackboard models. We will compare the following three algorithms. (1) \baseline: each site sends all the data to the coordinator directly; (2) \MM: our algorithm in the message passing model (Section~\ref{sec:gcmessage}); (3) 
\blackboard: our algorithm in  the blackboard model (Section~\ref{sec:bb}).


%Since both of our algorithms are crucially based on the use of spectral scarification, our main focus in the experiments is to investigate to what extend the quality of the spectral clustering algorithms will be affected by using spectral sparsification, the saving of communication costs by using spectral sparsificaion, ...
%
%
%The goal of this experiment is not to demonstrate the effectiveness of the spectral clustering algorithm. We mainly want to investigate the following, 
%\begin{itemize}
%\item to what extend the quality of clustered results will be affected by using spectral sparsification.
%\item saving of communication costs by using spectral sparsifier.
%\item the affect of constants in algorithms of the message passing/blackboard model.
%\end{itemize}
%
%
%\subsection{The Setup}
%\paragraph{Reference Algorithms}
%We compare different algorithms in our experiment.

%Note that we can also run \MM~ in the blackboard model.

Besides giving the visualized results of these algorithms on various datasets, we also measure the qualities of the results via the {\em normalized cut}, defined as 
\[
\ncut(A_1, \ldots, A_{k}) = \frac{1}{2}\sum_{i\in[k]}\frac{w(A_i, V\backslash A_i)}{\vol(A_i)},
\]
 which is a standard objective function to be minimized for spectral clustering algorithms. 
%We will compare the communication costs of these algorithms in different settings.

%We also compare the total communication costs of different algorithms/models. As the unit does not matter in our case, we normalize all communication costs by the cost of \baseline.  Whenever possible, we will visualize the clustered results.

We implemented the algorithms using multiple languages, including Matlab, Python and C++. Our experiments were conducted on an IBM NeXtScale nx360 M4 server, which is equipped with 2 Intel Xeon E5-2652 v2 8-core processors, 32GB RAM and 250GB local storage.


\subsection{Datasets.}
We test the algorithms in the following real and synthetic datasets, which is visualized in \figref{visualization}.


\begin{figure}[h]
     \centering
     \subfigure[\twomoons]{\includegraphics[width=0.23\textwidth]{twomoons-14000-original.png}\label{fig:twomoons}}
     ~~
     \subfigure[\gauss]{\includegraphics[width=0.23\textwidth]{gauss-10000-original.png}\label{fig:gauss}}
     ~~
     \subfigure[\sculpture]{\includegraphics[width=0.13\textwidth,height=0.16\textwidth]{sculpture-11680-original.jpg}\label{fig:sculpture}}
     \caption{Visualization of the datasets for our experiments.}
     \label{fig:visualization}
\end{figure}



\vspace{-1mm}
\begin{itemize}
\item \twomoons : this dataset contains $n=14,000$ coordinates in $\mathbb{R}^2$. We consider each point to be a vertex. For any two vertices $u, v$, we add an edge with weight $w(u,v) = \exp\{-\|u-v\|_2^2/\sigma^2\}$ with $\sigma = 0.1$ when one vertex is among the $7000$-nearest points of the other.  This construction results in a graph with about $110,000,000$ edges.

\item  \gauss : this dataset contains $n = 10,000$ points in $\mathbb{R}^2$. There are $4$ clusters in this dataset, each generated using a Gaussian distribution. We construct a complete graph as the similarity graph.  For any two vertices $u, v$, we define the weight $w(u,v) = \exp\{-\|u-v\|_2^2/\sigma^2\}$ with $\sigma = 1$. The resulting graph has about $100,000,000$ edges.

\item \sculpture : a photo of \textit{The Greek Slave}~\footnote{Available in e.g., \url{http://artgallery.yale.edu/collections/objects/14794}}. We use an $80\times 150$ version of this photo where each pixel is viewed as a vertex. To construct a similarity graph, we map each pixel to a point in $\mathbb{R}^5$, i.e., $(x, y, r, g, b)$, where the latter three coordinates are the RGB values. For any two vertices $u, v$, we  put an edge between $u, v$ with weight $w(u,v) = \exp\{-\|u-v\|_2^2/\sigma^2\}$ with $\sigma = 0.5$ if one of $u, v$ is among the $5000$-nearest points of the other. This results in a graph with about $70,000,000$ edges.
\end{itemize}
\vspace{-1mm}
In the distributed model edges are randomly partitioned across $s$ sites. 

%\vspace{-1.5mm}



\subsection{Results on clustering quality}
%{\em Quality.} \
\begin{figure*}[ht]
     \centering
     \subfigure[\baseline]{\includegraphics[width=0.2\textwidth]{twomoons-14000-original-clustered.png}\label{fig:twomoons-clustered-original}}
     \subfigure[\MM]{\includegraphics[width=0.2\textwidth]{twomoons-14000-sparsify-clustered-15.png}\label{fig:twomoons-clustered-sparsify}}
     \subfigure[\blackboard]{\includegraphics[width=0.2\textwidth]{twomoons-14000-chain-clustered.png}\label{fig:twomoons-clustered-chain}}
     \caption*{\twomoons, $k = 2$;}

\subfigure[\baseline]{\includegraphics[width=0.2\textwidth]{gauss-10000-original-clustered.png}\label{fig:gauss-clustered-original}}
     \subfigure[\MM]{\includegraphics[width=0.2\textwidth]{gauss-10000-sparsify-clustered-15.png}\label{fig:gauss-clustered-sparsify}}
     \subfigure[\blackboard]{\includegraphics[width=0.2\textwidth]{gauss-10000-chain-clustered.png}\label{fig:gauss-clustered-chain}}
     \caption*{\gauss, $k = 4$}


     \subfigure[\baseline]{\includegraphics[width=0.2\textwidth,height=0.2\textwidth]{sculpture-11680-original-clustered.png}\label{fig:sculpture-clustered-original}}  
     \subfigure[\MM]{\includegraphics[width=0.2\textwidth,height=0.2\textwidth]{sculpture-11680-sparsify-clustered-15.png}\label{fig:sculpture-clustered-sparsify}}
     \subfigure[\blackboard]{\includegraphics[width=0.2\textwidth,height=0.2\textwidth]{sculpture-11680-chain-clustered.png}\label{fig:sculpture-clustered-chain}}
     \caption*{\sculpture, $k = 3$. }


     
     \caption{Visualization of the results on \twomoons, \gauss\ and \sculpture. In the message passing model each site samples $5 n$ edges; in the blackboard model all sites jointly sample $10n$ edges (in \twomoons~ and \gauss) or $20n$ edges (in \sculpture) and the chain has length $18$. $s = 15$.}
     \label{fig:quality-1}
\end{figure*}

We visualize the clustered results for 
the \twomoons, \gauss\ and \sculpture\ in Figure~\ref{fig:quality-1}.
% and visualize the clustered results for \gauss\ and \sculpture in Figure~\ref{fig:quality-2}.
It can be seen that \baseline, \MM\ and \blackboard\ give results of very similar qualities.  For simplicity, here we only present the visualization for $s=15$. Similar results were observed when we varied the values of $s$.  
%\he{To Qin: Do you plan to have two titles (Results \& Quality)?}


% \begin{figure*}[h]
%      \centering
% \subfigure[\baseline]{\includegraphics[width=0.3\textwidth]{gauss-10000-original-clustered.png}\label{fig:gauss-clustered-original}}
%      \subfigure[\MM]{\includegraphics[width=0.3\textwidth]{gauss-10000-sparsify-clustered-15.png}\label{fig:gauss-clustered-sparsify}}
%      \subfigure[\blackboard]{\includegraphics[width=0.3\textwidth]{gauss-10000-chain-clustered.png}\label{fig:gauss-clustered-chain}}
%      \caption*{\gauss, $k = 4$}


%      \subfigure[\baseline]{\includegraphics[width=0.2\textwidth]{sculpture-11680-original-clustered.png}\label{fig:sculpture-clustered-original}}  
%      \subfigure[\MM]{\includegraphics[width=0.2\textwidth]{sculpture-11680-sparsify-clustered-15.png}\label{fig:sculpture-clustered-sparsify}}
%      \subfigure[\blackboard]{\includegraphics[width=0.2\textwidth]{sculpture-11680-chain-clustered.png}\label{fig:sculpture-clustered-chain}}
%      \caption*{\sculpture, $k = 3$. }

%      \caption{Visualization of results on \gauss\ and \sculpture; in the message passing model each site samples $5 n$ edges; in the blackboard model all sites jointly sample $10n$ (in \gauss) or $20n$ (in \sculpture) edges and the chain has length $18$.}
%      \label{fig:quality-2}
% \end{figure*}


We also compare the normalized cut (ncut) values of the clustering results of different algorithms.  The results are presented in Figure \ref{fig:quality}. In all datasets, the ncut values of different algorithms are very close. The ncut value of \MM\ slightly decreases when we increase the value of $s$, while the ncut value of \blackboard\ is independent of $s$.
%We comment that in general, it is difficult to compare \MM\ and \blackboard\ directly because they are affected by different parameters.


\begin{figure*}[!ht]
  \centering
  \subfigure[\twomoons]{\includegraphics[width=0.33\textwidth]{twomoons-14000-ncut.png}\label{fig:twomoons-quality}}\hspace*{-1.1em}
  \subfigure[\gauss]{\includegraphics[width=0.31\textwidth]{gauss-10000-ncut.png}\label{fig:gauss-quality}}\hspace*{-1.1em}
  \subfigure[\sculpture]{\includegraphics[width=0.31\textwidth]{sculpture-11680-ncut.png}\label{fig:sculpture-quality}}\hspace*{-1.1em}
  \subfigure{\includegraphics[width=0.14\textwidth]{legend.png}}
     \caption{Comparisons on normalized cuts. In the message passing model, each site samples $5n$ edges; in each round of the algorithm in the blackboard model, all sites jointly sample $10n$ edges (in \twomoons~and \gauss) or $20n$ edges (in \sculpture) edges and the chain has length $18$.}
     \label{fig:quality}
\end{figure*}

%\textcolor{red}{To Jiecao: Can you put the color lines indicating baseline, message passing, and blackboard within one row in Pic 2? Withthis we can save some space.}

%\vspace{-1.5mm}

\subsection{Results on communication costs} 
\begin{figure*}[!ht]
     \centering
     \subfigure[\twomoons]{\includegraphics[width=0.3\textwidth]{twomoons-14000-communication.png}\label{fig:twomoons-communication}}
     \subfigure[\gauss]{\includegraphics[width=0.3\textwidth]{gauss-10000-communication.png}\label{fig:gauss-communication}}
     \subfigure[\sculpture]{\includegraphics[width=0.3\textwidth]{sculpture-11680-communication.png}\label{fig:sculpture-communication}}


     \subfigure[\twomoons]{\includegraphics[width=0.32\textwidth]{twomoons-14000-communication-2.png}\label{fig:twomoons-communication-2}}
     \subfigure[\gauss]{\includegraphics[width=0.32\textwidth]{gauss-10000-communication-2.png}\label{fig:gauss-communication-2}}
     \subfigure[\sculpture]{\includegraphics[width=0.32\textwidth]{sculpture-11680-communication-2.png}\label{fig:sculpture-communication-2}}
     \caption{Comparisons on communication costs. In the message passing model, each site samples $5n$ edges; in each round of the algorithm in the blackboard model, all sites jointly sample $10n$ (in \twomoons~and \gauss) or $20n$ (in \sculpture) edges and the chain has length $18$. }
     \label{fig:communication}
\end{figure*}

We compare the communication costs of different algorithms in Figure \ref{fig:communication}. We observe that while achieving similar clustering qualities as \baseline, both \MM\ and \blackboard\ are significantly more communication-efficient (by one or two orders of magnitudes in our experiments). We also notice that the value of $s$ does not affect the communication cost of \blackboard, while the communication cost of \MM\ grows almost linearly with $s$; when $s$ is large, \MM\ uses significantly more communication than \blackboard. These confirm our theory.  %In Figure~\ref{fig:mm-const} and Figure~\ref{fig:blackboard-const}   in Appendix~\ref{sec:parameters} we present how the performance of \MM\ and \blackboard\ are affected by their parameters.

%
%
%\vspace{-1.5mm}
%\paragraph{Summary.}  From our experimental results we conclude that \MM\ and \blackboard\ achieve similar clustering quality as the native algorithm \baseline, while significantly reduce the communication cost.  When the number of sites is large, \blackboard\ is more communication efficient than \MM, as predicted by our theory.



\subsection{Parameters in \MM\ and \blackboard}
\label{sec:parameters}

Figure \ref{fig:mm-const} shows in \MM how the value of ncut is affected by the number of sites and the number of edges sampled in each site. 
Here, each site samples $cn$ edges. 
When $c=3$ and $s=1$, the ncut value diverges in all datasets. This is because with such a small $c$, the algorithm does not generate a valid sparsifier. In general, increasing $c$ or $s$ will slightly decrease the ncut value. But once they are above some thresholds, the ncut values of \MM\ and \baseline\ become very close.

Figure \ref{fig:blackboard-const} shows in \blackboard  how the ncut value is affected by the number of iterations and the number of edges sampled. When the number of iterations is set to be $5$, ncut values diverge in all datasets. This is because we cannot expect to generate a valid sparsifier by using such few iterations. It can be seen from \ref{fig:bb-gauss-constant} that for a fixed $c$, performing more iterations will help to reduce ncut values. From the same figure, one can also conclude that for fixed iterations, increasing $c$ also helps to reduce the ncut values.



\begin{figure*}[h!t]
     \centering
     \subfigure[\twomoons]{\includegraphics[width=0.3\textwidth]{twomoons-c.png}\label{fig:mm-twomoons-constant}}
     \subfigure[\gauss~dataset]{\includegraphics[width=0.3\textwidth]{gauss-c.png}\label{fig:mm-gauss-constant}}
     \subfigure[\sculpture]{\includegraphics[width=0.3\textwidth]{sculpture-c.png}\label{fig:mm-sculpture-constant}}
     \caption{The pictures above show the $\ncut$ values with respect to the values of $c$ and $s$ for the \MM\ algorithm. Here  
 each site samples $c n$ edges.}
     \label{fig:mm-const}
\end{figure*}


\begin{figure*}[h!t]
     \centering
     \subfigure[\twomoons]{\includegraphics[width=0.3\textwidth]{twomoons-iter.png}\label{fig:bb-twomoons-constant}}
     \subfigure[\gauss]{\includegraphics[width=0.3\textwidth]{gauss-iter.png}\label{fig:bb-gauss-constant}}
     \subfigure[\sculpture]{\includegraphics[width=0.3\textwidth]{sculpture-iter.png}\label{fig:bb-sculpture-constant}}
     \caption{The pictures above show how the $\ncut$ values are affected by the number of iterations and the value of $c$ for the \blackboard\ algorithm. Here 
all sites jointly sample $c n$ edges. }
     \label{fig:blackboard-const}
\end{figure*}







\section{Discussion}
\label{conclusion}

\begin{comment}
\begin{figure}
\includegraphics[width=\linewidth]{figs/beyond_tss_lesion.pdf}
\caption[]{End-to-End runtime lesion study of the entire MNIST dataset and the FMA featurized music dataset. Each of DROP's contributions provides a runtime improvement.}
\label{fig:beyond_lesion}
\end{figure}
\end{comment}



\section{Conclusion}
\label{sec:conclusion}

Advanced data analytics techniques must scale to rising data volumes. 
DR techniques offer a powerful toolkit when processing these datasets, with PCA frequently outperforming popular techniques in exchange for high computational cost. 
In response, we propose DROP, a new dimensionality reduction optimizer. 
DROP combines progressive sampling, progress estimation, and online aggregation to identify high quality low dimensional bases via PCA without processing the entire dataset by balancing the runtime of downstream tasks and achieved dimensionality. 
Thus, DROP provides a first step in bridging the gap between quality and efficiency in end-to-end DR for downstream \red{analytics}. 

%We revisit canonical operators for time series dimensionality reduction and the measurement study of~\cite{keogh-study}, and show that PCA is more effective than popular alternatives in the data mining literature often by a margin of over $2\times$ on average on gold-standard time series benchmark data sets with respect to output data dimension. More surprisingly, we empirically demonstrate that a small number of samples are sufficient to accurately characterize directions of maximum variance and obtain a high-quality low-dimensional transformation.




\section*{Acknowledgements}
We thank Kilian Weinberger, Iasonas Kokkinos, Changhan Wang, and the entire Facebook AI Research team for helpful discussions and code support.

\bibliography{citations}
\bibliographystyle{iclr2018_conference}

\end{document}
