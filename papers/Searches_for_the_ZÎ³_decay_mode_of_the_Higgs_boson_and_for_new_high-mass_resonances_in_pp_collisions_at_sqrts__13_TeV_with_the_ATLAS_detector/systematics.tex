The experimental and theoretical uncertainties that are considered in the searches can be 
grouped into three classes:
uncertainties associated with the parameterisation of the signal and background distributions
(see Section~\ref{sec:uncshape}), experimental uncertainties in the efficiency and acceptance affecting
the expected event yields (see Section~\ref{sec:uncyield}), and theoretical uncertainties in the 
modelling of
the signal in the simulation (see Section~\ref{sec:unctheo}). The nuisance parameters in the likelihood 
function (see Section~\ref{sec:statistics}) represent the uncertainties which
are studied in each category using the simulated signal samples generated at $\mH =125$~\GeV\ for the 
$H\to Z\gamma$ search, and 
$\mX = [300-2500]$~\GeV\ at high-mass.
The main experimental sources of uncertainty are summarised in Table~\ref{tab:syst}.


\begin{table}
  \caption{The main sources of experimental uncertainty for the $H/X\to Z \gamma$ searches. The gluon--gluon fusion signal samples
produced at $\mh = 125~\GeV$ and
$\mX = $ [300--2500]~\GeV\ are used to estimate the systematic uncertainty.
The ranges for the uncertainties span the variations among different categories and different \mX~resonance masses. 
The uncertainty values are given as fractions of the total predictions, except for the spurious signal uncertainty, 
which is reported as the absolute number of events. Values are not listed if systematic sources are negligible or 
not applicable.
}
\label{tab:syst}
  \centering
  \begin{tabular}{lr@{--}l@{\hspace{1.5ex}}r@{--}l@{}}
    \hline
    \hline       
Sources & \alignUnderEnDash{\htoZg} &  & \alignUnderEnDash{\xtozg} & \\
\hline
     \hline

     \multicolumn{5}{c}{\em Luminosity $\mathrm{[\%]}$}                          \\
     \hline
     Luminosity                         & \alignUnderEnDash{3.2} & & \alignUnderEnDash{3.2} &                 \\
     \hline
    \multicolumn{5}{c}{\em Signal efficiency $\mathrm{[\%]}$}                          \\
     \hline
     Modelling of pile-up interactions    & 0.02 & 0.03 & $<0.01$ & 0.2     \\
     Photon identification efficiency     & 0.7 & 1.7   & 2.0 & 2.6     \\
     Photon isolation efficiency          & 0.07 & 0.4  & 0.6 & 0.6     \\
     Electron identification efficiency   & 0.0 & 1.6  & 0.0 & 2.6     \\
     Electron isolation efficiency        & 0.0 & 0.2  & 0.0 & 3.5     \\
     Electron reconstruction efficiency   & 0.0 & 0.4  & 0.0 & 1.0     \\
     Electron trigger efficiency          & 0.0 & 0.1  & 0.0 & 0.2     \\
     Muon selection efficiency            & 0.0 & 1.6  & 0.0 & 0.7     \\
     Muon trigger efficiency              & 0.0 & 3.5  & 0.0 & 4.2     \\
     MC statistical uncertainty           &  &         & 1.2 & 2.0      \\
     Jet energy scale, resolution, and pile-up   & 0.2 & 10  &  &   \\
    \hline
    Total (signal efficiency)              & 2.1 & 10 & 4.0 & 6.3\\
    \hline
    \multicolumn{5}{c}{\em Signal modelling on $\sigma_\mathrm{CB}$ $\mathrm{[\%]}$} \\
    \hline
    Electron and photon energy scale       &  0.6 & 3.5  & 1.0 & 4.0        \\
    Electron and photon energy resolution  &  1.1 & 4.0  & 4.0 & 30        \\   
    Muon momentum scale                    &  0.0 & 0.5  & 0.0 & 3.0         \\
    Muon ID resolution                     &  0.0 & 3.7  & 0.0 & 2.0         \\
    Muon MS resolution                     &  0.0 & 1.7  & 0.0 & 4.0         \\
\hline
    \multicolumn{5}{c}{\em Signal modelling on $\mu_\mathrm{CB}$ $\mathrm{[\%]}$} \\
\hline
    Electron and photon energy scale       &  0.1 & 0.2  & 0.2 & 0.6   \\
    Muon momentum scale                    &  0.0 & 0.03 & 0.0 & 0.03   \\
    Higgs mass                             &  \alignUnderEnDash{0.2}       &  &  &         \\
    \hline
    \multicolumn{5}{c}{\em Background modelling $\mathrm{[Events]}$}   \\
    \hline
     Spurious signal                      &   1.7 & 25     &  0.005 & 6.1 \\
     \hline \hline
  \end{tabular}
\end{table}


\subsection{Uncertainties from signal and background modelling}
\label{sec:uncshape}

The uncertainties in the lepton 
and photon momentum and energy scale and resolution
impact the modelling of the signal. 
Their impact is assessed by comparing the nominal $m_{Z\gamma}$ shape parameters with the 
$m_{Z\gamma}$ shape parameters after varying
the lepton and photon momentum and energy scale and resolution by their uncertainties.
Uncertainties in both  the  position ($\mu_\mathrm{CB}$) and width ($\sigma_\mathrm{CB}$) 
of the signal $m_{Z\gamma}$ distribution are considered.

The systematic uncertainties in the muon momentum scale and resolution were determined from
$Z\to\mu\mu$ and $J/\psi\to\mu\mu$ events using the techniques described in Ref.~\cite{Aad:2016jkr}.
At $m_H=125~\GeV$, the uncertainty in the muon momentum scale (resolution) varies 
$\sigma_\mathrm{CB}$ by up to 0.5\% (4.0\%).
In the high-mass search, the effect of the muon momentum scale (resolution) uncertainty is to change
$\sigma_\mathrm{CB}$ by up to 3.0\% (4.0\%). The typical effect of the muon momentum 
scale uncertainty is to change 
$\mu_\mathrm{CB}$ by $< 0.1\%$ of its nominal value.

The systematic uncertainties in the
electron and photon energy scale and resolution follow those in Refs.~\cite{Aad:2014nim, ATLAS-egammacalib}. The
overall energy scale factors and their uncertainties were determined using $Z\to ee$ events.
Compared to Ref.~\cite{ATLAS-egammacalib}, several systematic uncertainties were re-evaluated with 
the 13~\TeV\ data, including uncertainties related to the
observed LAr cell non-linearity, the material simulation, the intercalibration of the first and
second layer of the calorimeter, and the pedestal corrections.
At $m_H=125~\GeV$, the uncertainty in the electron and photon energy scale (resolution) results in
variation in $\sigma_\mathrm{CB}$ between 0.6\% and 3.5\% 
(1.1\% and 4.0\%) depending on the 
category. For a high-mass resonance, $\sigma_\mathrm{CB}$ varies between 
1.0\% and 4.0\% 
(4.0\% and 30\%) 
due to uncertainties
in the electron/photon momentum scale (momentum resolution).
The variation in $\mu_\mathrm{CB}$ is less than 0.2\% (0.6\%) at $m_H=125~\GeV$ (at high masses).

For the $H\to Z\gamma$ search, an additional uncertainty in the assumed Higgs mass $\mH = 125.09~\GeV$ is
added in the fit, reflecting the $0.24~\GeV$~\cite{higgs-mass} uncertainty in the measured Higgs boson mass.

The uncertainty due to the choice of background function is taken to be the signal yield 
(spurious signal) obtained
when fitting the $m_{Z\gamma}$ spectra reconstructed from background-only distributions
as discussed in Section~\ref{sec:signalBkgModel}.



\subsection{Experimental uncertainties affecting the signal efficiency and acceptance}
\label{sec:uncyield}

Experimental uncertainties affecting the signal efficiency and acceptance 
can be either correlated between all event
categories (yield uncertainties) or anticorrelated between some of the categories 
(migration uncertainties) when they are related to how the signal populates the event categories.

The uncertainty in the combined 2015+2016 integrated luminosity is 3.2\%, 
correlated between all categories. It is derived, 
following a methodology similar to that detailed in Ref.~\cite{Aaboud:2016hhf}, from a preliminary 
calibration 
of the luminosity scale using $x$--$y$ beam-separation scans performed in August 2015 and May 2016.

A variation in the pile-up reweighting of the simulation is included to cover the uncertainty 
in the ratio of the predicted and measured inelastic cross sections in the fiducial volume 
defined by $m > 13~\GeV$ where $m$ is the mass of the hadronic 
system~\cite{Aaboud:2016mmw}. The uncertainty in the signal efficiency is no more than 
0.03\% (0.2\%)
for the $H\to Z\gamma$ (high-mass resonance) search.

The uncertainties in the reconstruction, identification, isolation, and trigger efficiency measurements
for muons, electrons and photons (see Section~\ref{sec:eventSelection}) are treated as fully 
correlated between all categories. They are determined from control samples of $J/\psi\to\mu\mu$
and $Z\to\mu\mu$ for muons, $J/\psi\to ee$ and $Z\to ee$ for electrons, and $Z\to\ell\ell\gamma$,
$Z\to ee$, and inclusive photons for photons, using methods described in 
Refs.~\cite{Aad:2016jkr, ATLAS-electrons, ATLAS-photonid}.

For the $H\to Z\gamma$ search, the uncertainties in the signal efficiency from the 
photon identification and isolation are found to be no more than 1.7\% and 0.4\%, respectively. 
The uncertainties in the signal efficiency
from the electron reconstruction, identification, isolation, and trigger are
found to be no more than 0.4\%, 1.6\%, 0.2\%, and 0.1\%, respectively. 
The uncertainties in the signal efficiency from the muon
selection and trigger are determined to be no more than 1.6\% and 3.5\%, respectively.
For the high-mass search, the uncertainties in the signal efficiency from 
photon identification and isolation are found to be no more than 2.6\% and 0.6\%, respectively. 
The uncertainties in the signal efficiency
from the electron reconstruction, identification, isolation, and trigger are
found to be no more than 1.0\%, 2.6\%, 3.5\%, and 0.2\%, respectively. 
The uncertainties in the efficiency from the muon
selection and trigger are determined to be as large as 0.7\% and 4.2\%, respectively.
The uncertainty due to the limited size of the simulated event samples ranges from 1.2\% to 2.0\% 
for the search for high-mass resonances.

In the $H\to Z\gamma$ search, the expected signal yield in the VBF category is 
affected by the jet energy scale and resolution and the jet vertex tagging efficiency. The corresponding 
uncertainties are anticorrelated with the other categories. Uncertainties
in the jet energy scale and resolution are estimated from 
the transverse momentum balance in dijet, $\gamma+$jet and $Z$+jet 
events~\cite{Aaboud:2017jcu}.
Uncertainties in the efficiency of the jet vertex tagging are estimated by shifting the associated 
corrections
applied to the simulation by an amount allowed by the data. The uncertainties in the category 
acceptances are as large as 4.6\%, 6.9\%, and 4.8\% from the data-driven jet calibration, the 
impact
of the jet flavour composition on the calibration, and the jet vertex tagging.


\subsection{Theoretical and modelling uncertainties}
\label{sec:unctheo}

\begin{table}
  \caption{The main sources of theoretical and modelling uncertainties for the $H\to Z \gamma$ 
    search. 
    For the uncertainties in the total efficiency and the acceptance of the different categories, 
    the gluon--gluon
    fusion samples produced with \textsc{Powheg Box} v1 with and without MPI are used, as well as 
    the nominal 
    \textsc{Powheg Box} v2 gluon--gluon fusion signal sample along with the sample
    generated with \textsc{MadGraph5\_aMC@NLO}, as described in the text.
    The combined uncertainty on the total cross section and efficiency is given assuming the
    cross sections predicted by the SM.
    The ranges for the uncertainties cover the variations among different categories.
    The uncertainty values are given as relative uncertainties.}
\label{tab:thsyst}
  \centering
  \begin{tabular}{lr@{--}l@{}}
    \hline
    \hline       
\multicolumn{1}{c}{Sources} & \alignUnderEnDash{ } & \\
\hline
     \hline

     \multicolumn{3}{c}{\em Total cross section and efficiency $\mathrm{[\%]}$} \\
     \hline
     Underlying event                         & \alignUnderEnDash{5.3} & \\
     ggF perturbative order                   & \alignUnderEnDash{3.9} & \\
     ggF PDF and $\alpha_\mathrm{s}$          & \alignUnderEnDash{3.2} & \\
     VBF perturbative order                   & \alignUnderEnDash{0.4} & \\
     VBF PDF and $\alpha_\mathrm{s}$          & \alignUnderEnDash{2.1} & \\
     $WH$ ($ZH$) perturbative order           & \alignUnderEnDash{0.5~(3.8)} & \\
     $WH$ ($ZH$) PDF and $\alpha_\mathrm{s}$  & \alignUnderEnDash{1.9~(1.6)} & \\
     Interference                             & \alignUnderEnDash{5.0} & \\
     $B$($H\to Z\gamma$)            & \alignUnderEnDash{5.9} & \\\hline
     Total (total cross section and efficiency) & \alignUnderEnDash{10} & \\
     \hline
    \multicolumn{3}{c}{\em Category acceptance $\mathrm{[\%]}$}  \\
     \hline
     ggF $H+2$-jets in VBF-enriched category  & 0.5 & 45 \\
     ggF BDT variables                        & 0.2 & 15 \\
     ggF Higgs \pt                            & 8.4 & 22 \\
     PDF and $\alpha_\mathrm{s}$              & 0.2 & 2.0 \\
     Underlying event                         & 2.9 & 25 \\\hline
     Total (category acceptance)              & 9.5 & 49\\
     \hline \hline
  \end{tabular}
\end{table}


For the $H\to Z\gamma$ search, theoretical and modelling uncertainties in the 
SM predictions for
Higgs boson production and the decay to the $Z\gamma$ final state are taken into account and
are summarised in Table~\ref{tab:thsyst}. 
They fall into
two classes: uncertainties in the total predicted cross sections, the predicted decay 
branching ratio and the total efficiency, correlated between all categories; 
and uncertainties in the event fractions per category, anticorrelated between certain
categories.

Uncertainties related to the total acceptance and efficiency for $H\to Z\gamma$ events
affect the extraction of the signal strength, the branching ratio of $H\to Z\gamma$
assuming SM Higgs boson production, as well as the product of the Higgs boson production
cross section and the branching ratio of $H\to Z\gamma$ (see Section~\ref{sec:statistics}).
The uncertainty in the total efficiency due to the modelling of multiple-parton 
interactions
is estimated from the difference in efficiency with and without multiple-particle interactions 
for the gluon--gluon fusion simulation sample, and found to be 5.3\%.

The uncertainties related to the predicted Higgs boson production cross section affect
the extraction of the signal strength as well as the branching ratio of $H\to Z\gamma$
assuming SM Higgs boson production.
The uncertainties in the predicted total cross sections of the different Higgs 
boson production processes due to the perturbative order of the calculation
and the combined uncertainties in the PDFs and $\alpha_\mathrm{s}$ are 3.9\% and 3.2\% for gluon--gluon fusion
production, respectively, 
and range from 0.4\% to 3.8\% for the other production processes for a Higgs boson mass of 125.09~\GeV\ and
a centre-of-mass energy of 13~\TeV~\cite{deFlorian:2016spz}.  
An additional $5.0\%$~\cite{Dicus:2013lta} uncertainty accounts for the effect, in the
selected phase space of the $\ell\ell\gamma$ final state, of the
interfering $H\to\ell\ell\gamma$ decay amplitudes
that are neglected in the
calculation of Ref.~\cite{deFlorian:2016spz}.
They originate from internal
photon conversion in Higgs boson decays to diphotons
($H\to\gamma^*\gamma\to\ell\ell\gamma$) or from
Higgs boson decays to dileptons with an off-shell lepton
($H\to\ell\ell^*\to\ell\ell\gamma$)~\cite{Chen:2012ju,Firan:2007tp}. 

The uncertainty in the predicted Higgs boson branching ratio to $Z\gamma$ affects the
extraction of the signal strength. 
The relative theoretical uncertainty in the predicted
Higgs boson branching ratio is 5.9\%~\cite{deFlorian:2016spz}.


Uncertainties in the modelling of kinematic distributions in the simulation of Higgs boson 
production processes affect the predicted event fractions in the different categories.
The uncertainty in the modelling of the production of jets in gluon--gluon fusion production due to
the perturbative order in QCD is estimated by scale variations in MCFM~\cite{Campbell:2010ff}. It
accounts for the uncertainty in the overall normalisation of $H+2$-jets events as well as the 
uncertainty due to the use of $\Delta\phi_{Z\gamma,jj}$, which serves to apply an implicit third-jet
veto, in the VBF BDT. The estimation of this uncertainty uses 
an extension of the Stewart--Tackmann method~\cite{Stewart:2011cf, Gangal:2013nxa}. 
It corresponds to 45\%
of the ggF contribution to the VBF category.
Additional uncertainties are assigned to account for potential mismodelling of the variables that serve as input
to the VBF BDT (see Section~\ref{sec:eventSelection}). They are estimated by reweighting the simulated ggF
events to match the distributions in $m_{jj}$, $\Delta\eta_{jj}$, $\ptt$, $\Delta R^\mathrm{min}_{Z/\gamma, j}$, 
and $|\eta_{Z\gamma}-(\eta_{j1} + \eta_{j2})/2|$ obtained from \textsc{MadGraph5\_aMC@NLO}. The resulting uncertainty
in the ggF contribution to the VBF category is 15\%.
Uncertainties in the modelling of the Higgs boson \pt\ spectrum are taken to be the envelope of the
variations of the renormalisation, factorisation, and resummation scales 
obtained using \textsc{HRes} 2.3~\cite{Grazzini:2013mca} to simulate the \pt\ spectrum.
The resulting uncertainties are evaluated using 
the Stewart--Tackmann method~\cite{Stewart:2011cf, Gangal:2013nxa} for the high relative $\pt$ and the \ptt\ categorisation
and found to range from 8.4\% to 22\%. 
Uncertainties from the choice of PDF set and $\alpha_\mathrm{s}$ are evaluated using the combined error PDF
set, which takes into account 30 variations of NNLO PDFs and two variations of $\alpha_\mathrm{s}$, 
following the PDF4LHC recommendations~\cite{Butterworth:2015oua} and are found to range from 
0.2\% to 2.0\%.
The uncertainty in the acceptance due to the modelling of multiple-parton interactions is estimated 
from the difference in acceptance with and without multiple-particle interactions for the gluon--gluon
fusion simulation sample and ranges from 2.9\% to 25\%. 

