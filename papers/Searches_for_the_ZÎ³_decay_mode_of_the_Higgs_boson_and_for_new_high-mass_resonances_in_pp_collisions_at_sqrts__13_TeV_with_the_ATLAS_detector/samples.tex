Samples of simulated Monte Carlo (MC) events are used to optimise the search strategy, 
evaluate the selection
efficiency and to study the different background contributions. The 
generated event samples were processed with the detailed ATLAS detector simulation~\cite{ATLAS-SIM} 
based on \textsc{Geant4}~\cite{Geant4} (one exception
is noted below).

For the $H\to Z\gamma$ search, the mass
of the Higgs boson is chosen to be $\mh = 125$~\GeV\ and the corresponding width is 
$\Gamma_H = 4.1$~\MeV~\cite{deFlorian:2016spz}.
The SM Higgs boson production was simulated with 
\textsc{Powheg Box} v2~\cite{Nason:2004rx, Frixione:2007vw, Alioli:2010xd}
using the combined parton distribution function (PDF) set PDF4LHC following the
recommendations~\cite{Butterworth:2015oua} based on the CT14~\cite{Dulat:2015mca}, 
MMHT14~\cite{Harland-Lang:2014zoa} and NNPDF3.0~\cite{Ball:2014uwa} PDF sets and the
Hessian reduction method~\cite{Gao:2013bia, Carrazza:2015aoa, Watt:2012tq}.
The techniques used
for the simulation of gluon--gluon fusion (ggF) production, vector-boson fusion (VBF) production, and
production in association with a vector boson  ($WH$ and $ZH$, together referred to as $VH$) 
and the perturbative order achieved
are summarised in Table~\ref{tab:samples} and detailed below. 
Higgs boson production in association with a $t\bar{t}$ pair and other production
processes are not considered as their contributions to the total Higgs production cross section
are of the order of $0.1\%$ or less.

\begin{table}[htbp]
\caption{Higgs boson production processes produced with \textsc{Powheg Box} with the techniques
used and their precision in $\alpha_\mathrm{s}$ for the event generation (gen.). The total cross section
is known with higher precision in QCD and electroweak (norm.) than available in the event 
generation. The events
were reweighted to reproduce the more precise total cross section.}
\label{tab:samples}
\begin{center}
\begin{tabular}{lllll}
\hline\hline
Process & Technique        & QCD (gen.) & QCD (norm.) & EW (norm.)\\
\hline
ggF     & MiNLO \& NNLOPS & NNLO (incl.), NLO ($H + 1$-jet) & NNNLO      & NLO \\
VBF     & \textsc{Powheg}            & NLO                      & approx. NNLO & NLO \\
$VH$    & MiNLO            & NLO (incl. and $H + 1$-jet) & NNLO         & NLO \\
\hline\hline
\end{tabular}
\end{center}
\end{table}
 
Higgs boson production via ggF was simulated with \textsc{Powheg Box}, using the MiNLO
approach~\cite{Hamilton:2015nsa}, which achieves next-to-leading-order (NLO) precision for both 
the inclusive and the $H + 1$-jet process in quantum chromodynamics (QCD).
In addition, the NNLOPS approach~\cite{Hamilton:2013fea} was used to improve the precision for
inclusive observables to next-to-next-to-leading-order (NNLO) in QCD: 
the Higgs transverse momentum spectrum achieved by 
this technique was found to be in agreement 
with the result obtained using QCD  
resummation with next-to-next-to-leading logarithmic (NNLL)+NNLO precision 
from the HqT calculation~\cite{Bozzi:2005wk, deFlorian:2011xf}. Top
and bottom quark mass effects are included up to NLO precision in QCD. The central scale
choice for the nominal factorisation ($\mu_\mathrm{F}$) and 
renormalisation scales ($\mu_\mathrm{R}$) is $\mu_\mathrm{F} = \mu_\mathrm{R} = m_H/2$. 
The events were reweighted to reproduce the inclusive cross section
at next-to-next-to-next-to-leading-order (NNNLO) precision in QCD and NLO precision in
electroweak corrections~\cite{Anastasiou:2015ema, Anastasiou:2016cez, Actis:2008ug, Anastasiou:2008tj, Butterworth:2015oua, deFlorian:2016spz}.

Higgs boson production via VBF was simulated with \textsc{Powheg Box} at NLO precision in QCD~\cite{Nason:2009ai}. 
The events were reweighted to reproduce the inclusive cross section with approximate-NNLO precision in QCD and NLO 
precision in electroweak corrections~\cite{Butterworth:2015oua, deFlorian:2016spz, Ciccolini:2007jr, Ciccolini:2007ec, Bolzoni:2010xr}. 

Higgs boson production in association with a vector boson via quark--antiquark initial states
was simulated at 
NLO precision in QCD for inclusive events and $H + 1$-jet events using the MiNLO technique~\cite{Hamilton:2012rf}. 
The events were reweighted to reproduce 
the total $VH$ production cross section, including also production via gluon--gluon initial states, at NNLO precision in QCD with NLO electroweak 
corrections~\cite{Butterworth:2015oua, deFlorian:2016spz, Brein:2003wg, Altenkamp:2012sx, Denner:2011id}. 


The effects of parton showering, hadronisation and multiple parton interactions (MPI) were
simulated using \textsc{Pythia 8.186}~\cite{Pythia8} configured with the AZNLO set of 
parameters~\cite{Aad:2014xaa} and the CTEQ6L1~\cite{cteq6l1} PDF set.
The events were reweighted to 
reproduce the \htoZg~branching ratio calculated with 
\textsc{Hdecay}~\cite{deFlorian:2016spz, Djouadi:1997yw, Djouadi:2006bz}.

Three additional event samples of gluon--gluon fusion production are used for the studies of theoretical 
uncertainties. 
The first is an event sample generated with \textsc{MadGraph5\_aMC@NLO} version 5.2.3.3 with 
FxFx multijet merging~\cite{Alwall:2014hca,Frederix:2012ps} of
$H + 0$-jet and $H + 1$-jet at NLO precision in QCD, using the NNPDF 3.0 PDF set. 
The decay of the Higgs boson and the 
parton showering, hadronisation and MPI were provided by 
\textsc{Pythia 8.186} using the A14 set of parameters~\cite{ATLASA14} and the 
NNPDF2.3 PDF set~\cite{Ball:2012cx}. 
Two more samples were simulated with
\textsc{Powheg Box} v1~\cite{Nason:2004rx, Frixione:2007vw, Alioli:2010xd, Bagnaschi:2011tu}
 with the CT10 PDF set~\cite{ct10} and 
\textsc{Pythia 8.186} for parton showering and hadronisation using the AZNLO set of parameters and the
CTEQ6L1 PDF set. The two samples were produced with and
 without MPI to study the uncertainties in the signal acceptance related to the
modelling of non-perturbative effects.

Production of $CP$-even, high-mass spin-0 resonances $X$ in the mass range $m_X \in $ [300--2500]~\GeV\ 
was simulated for 
the gluon--gluon fusion and vector-boson fusion production processes and for an 
intrinsic resonance width of 4~\MeV,
which is much smaller than the experimental resolution (see Section~\ref{sec:signalBkgModel})
and referred to as narrow width assumption (NWA). 
Due to the assumed narrow width of
the resonance, the interference between the resonant signal and the non-resonant background
is neglected.
The ggF (VBF) process was simulated for $m_X = 300$, 500, 700, 750, 800, 1000, 1500, 2000 and 2500~\GeV\ 
($m_X =$ 300, 500, 1000 and 2500~\GeV). Both the ggF and VBF processes were 
produced with \textsc{Powheg Box} v1 with the CT10 PDF set.

Production of $CP$-even, high-mass spin-2 resonances $X$ with mass 
$m_X = 250$, 300, 500, 750, 1000, 1500, 2000
and 2500~\GeV\ for an intrinsic resonance width of 4~\MeV\ via gluon--gluon and
quark--antiquark initial states was simulated at LO in QCD in the Higgs Characterisation 
Model~\cite{Artoisenet:2013puc} 
with \textsc{MadGraph5\_aMC@NLO} 2.3.3~\cite{Alwall:2014hca}. 

For the high-mass spin-0 (spin-2) resonances, the parton showering,
hadronisation and MPI were simulated with \textsc{Pythia 8.186} using the AZNLO 
(A14) set of 
parameters and the CTEQ6L1 (NNPDF2.3) PDF set.

The signal shape and the reconstruction and selection efficiency of the studied high-mass
resonances are parameterised as a function of $m_X$. The parameterisation allows
the extraction of the signal shape and efficiency for any mass point at which no simulation sample is available.

The background mainly originates from non-resonant production of a a $Z$ boson and a prompt
photon ($Z$$+$$\gamma$), with a smaller contribution from production of $Z$
bosons in association with jets ($Z+$jets), with one jet misidentified as
a photon. $Z$$+$$\gamma$ production within the SM is primarily due to 
radiation of photons from final-state leptons (FSR) or initial-state quarks (ISR).
Both SM processes were simulated using the \sherpa
generator ~\cite{Gleisberg:2008ta} (version 2.1.1 for $Z$$+$$\gamma$ and version 2.2.0 for 
$Z$+jets), 
and the matrix elements were calculated using the \textsc{Comix}~\cite{Gleisberg:2008fv} and
\textsc{OpenLoops}~\cite{Cascioli:2011va} generators,
where $Z$$+$$\gamma$ production was calculated for real emission of up to two partons
at leading order (LO) in QCD and merged with the \sherpa parton
shower~\cite{Schumann:2007mg} using the ME+PS@LO
prescription~\cite{Hoeche:2009rj}. 
The process of $Z$+jets was calculated for up 
to two partons at next-to-leading-order (NLO)
and four partons at LO and merged with the parton shower using
the ME+PS@NLO prescription~\cite{Hoeche:2012yf}.
For the $Z$$+$$\gamma$ ($Z$+jets) samples, the CT10 (NNPDF3.0) PDF set was used in conjunction
with dedicated parton shower tuning developed by the \sherpa authors.
To study the background model in detail, a large sample of $Z$$+$$\gamma$ events was simulated 
using fast
simulation of the calorimeter response~\cite{fastsim}.

For all event samples, the additional inelastic $pp$ collisions per bunch crossing were simulated
with \textsc{Pythia 8.186}
using the \textit{A2} set of tuned parameters~\cite{Pythia8Tune} and the MRSTW2008LO PDF 
set~\cite{MRST}.
The MC events were reweighted to reproduce
the distribution of the average number of interactions per bunch crossing observed in the data.

Corrections derived from trigger, identification, reconstruction, and isolation efficiency
measurements for electrons and muons, from identification and isolation efficiency measurements
for photons, and from selection efficiency measurements for jets 
are applied to the simulated events to improve the description of the data.
Similarly, energy scale and resolution corrections for all simulated objects are also taken into 
account.

