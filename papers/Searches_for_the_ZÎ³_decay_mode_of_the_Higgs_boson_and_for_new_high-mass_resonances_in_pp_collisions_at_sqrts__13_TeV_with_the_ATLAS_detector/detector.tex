The ATLAS detector~\cite{ATLASDET-2008} at the LHC is a multipurpose particle detector with a forward-backward 
symmetric cylindrical geometry and a near 4$\pi$ coverage in solid angle.\footnote{The ATLAS experiment uses a 
right-handed coordinate system with its origin at the nominal interaction point (IP) in the centre 
of the detector and the $z$-axis along the beam pipe. The $x$-axis points from the IP to the centre of the 
LHC ring, and the $y$-axis points upward. Cylindrical coordinates $(r,\phi)$ are used in the transverse plane, 
$\phi$ being the azimuthal angle around the $z$-axis. The pseudorapidity is defined in terms of the polar 
angle $\theta$ as $\eta=-\ln\tan(\theta/2)$. The transverse energy is defined as $\et = E \sin(\theta)$.}
It consists of an inner tracking
detector, electromagnetic and hadronic calorimeters, and a muon spectrometer.
The inner detector (ID), immersed in a 2 T axial magnetic field provided by a thin superconducting
solenoid, includes silicon pixel and microstrip detectors,
which provide precision tracking in the pseudorapidity range  $|\eta| <2.5$, and a transition-radiation
tracker (TRT) providing additional tracking and information for electron 
identification for  $|\eta| <2.0$. For the \sqtt\
data-taking period, the ID was upgraded with a silicon-pixel insertable B-layer~\cite{ATLASDET-IBL},
providing additional tracking information from a new layer closest to the interaction point.
The solenoid is surrounded by sampling calorimeters: a lead/liquid-argon (LAr) 
electromagnetic calorimeter covering the region $|\eta| <3.2$, a hadronic calorimeter with a
steel/scintillator-tile barrel section for $|\eta| < 1.7$ and two copper/LAr endcaps
for $1.5 < |\eta| < 3.2$. 
The forward region is covered by additional coarser-granularity LAr calorimeters up to $|\eta| = 4.9$. 
The calorimeter is surrounded by the muon spectrometer (MS) consisting of three large 
superconducting toroids each containing eight coils.
Precise momentum measurements for muons with pseudorapidity up to $|\eta| = 2.7$
are provided by three layers of tracking chambers.
The muon spectrometer also includes separate 
trigger chambers covering $|\eta|$ up to 2.4.

A two-level trigger system~\cite{ATLAS:trigperf} was used during the \sqtt~data-taking period. 
The first-level trigger (L1) is implemented
in hardware and uses a subset of the detector information. This is followed by a software-based
level which runs algorithms similar to the offline reconstruction software, reducing the event rate to 
approximately 1~kHz from the maximum L1 rate of 100 kHz.

The $pp$ data collected by ATLAS in 2015 and 2016 were taken at a centre-of-mass energy of $\sqrt{s}$ = 13~\TeV\
and with a bunch spacing of 25~ns. After requiring that the full detector was operational 
during data-taking and application of requirements on the
data quality, the integrated luminosity corresponds to \lumithirteentev~\invfb, of which \intLumiFifteen~and 
\intLumiSixteen~\ifb\ were taken during 2015 and 2016, respectively. 
The average number of $pp$ interactions per bunch crossing (pile-up) ranged from about 13 in 2015 to about 25 in 2016,
with a peak instantaneous luminosity of \peakLumi\ achieved in 2016.

The events were collected with triggers requiring either one or two electrons or muons in the event.
The single-muon trigger has a transverse momentum (\pt) threshold of 26~\GeV\ and applies a 
requirement on the muon track isolation. The track isolation
is defined as the scalar sum of the transverse momenta of the ID tracks 
in a cone of $\Delta R = \sqrt{(\Delta\eta)^2 + (\Delta\phi)^2} = 0.3$ around the muon.
For the trigger used during 2016, the cone size was modified to be $\Delta R = 10 / (\pt/\GeV)$ for
muons with $\pt > 33.3~\GeV$.
The track isolation is computed from ID tracks with
$\pt > 1~\GeV$ and with a longitudinal impact parameter $z_0$ within 6~mm of the $z_0$ of the muon track,
excluding the muon track itself.
The track isolation is required to be less than 
6\% (7\%) of the muon's transverse momentum in the 2015 (2016) data set.
A second single-muon trigger with a \pt\ threshold of 50~\GeV\ has no requirement on the track isolation.
The dimuon trigger has \pt\ thresholds of 22~\GeV\ and 8~\GeV\ 
and does not apply track isolation criteria. Single-electron triggers with 
\pt\ thresholds at 24~\GeV\ (26~\GeV), 60~\GeV, and 120~\GeV\ (140~\GeV) are used, 
as well as a dielectron trigger with
a \pt\ threshold of 12~\GeV\ (17~\GeV) during the 2015 (2016) data taking. 
In the 2016 data-taking period, the lowest-threshold single-electron
trigger required the track isolation in a cone of $\Delta R = 0.2$ for $\pt < 50$~\GeV\ and 
$\Delta R = 10 / (\pt/\GeV)$ for $\pt > 50$~\GeV\ to be less than 10\% of the electron's 
transverse energy.
For all electron triggers, electron candidates must satisfy
identification criteria based on the properties of the energy cluster in the 
electromagnetic calorimeter and its associated track. The single-electron triggers with lower
thresholds use tighter criteria for the electron identification.
For $H\to Z\gamma$ events that pass the analysis preselection (see Section~\ref{sec:eventSelection}), 
the efficiency to pass the trigger selection is 92.9\%~(96.9\%) for $Z$ boson decays to
muon (electron) pairs.
For a high-mass resonance at 1~\TeV,
the corresponding efficiencies are 94.3\% and 99.8\% for muon and electron final states, respectively.
