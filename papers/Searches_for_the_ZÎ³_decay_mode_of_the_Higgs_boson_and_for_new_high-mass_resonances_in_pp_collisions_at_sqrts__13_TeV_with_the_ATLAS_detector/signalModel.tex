The signal and background yields are extracted from the
data $m_{Z\gamma}$ distribution by assuming analytical models.
The parameters that describe the shape of the signal are obtained from simulated signal samples. 
The analytical models used for the background shape are chosen using simulated background
samples and the values of their free parameters are determined from the fit to data.


\subsection{Signal modelling}
\label{sec:signalModel}

The signal mass distribution in the searches for both the Higgs boson and the high-mass resonance decay to $Z\gamma$ is well 
modelled with a double-sided
Crystal Ball (DSCB) function (a Gaussian function with power-law tails on both sides)~\cite{dscb1, dscb2}.
The peak position 
and width
of the Gaussian component are represented by $\mu_\mathrm{CB}$ and $\sigma_\mathrm{CB}$, respectively.


To determine the parameters of the DSCB for the $H\to Z\gamma$ search, a 
fit is performed to all the categories (see Section~\ref{sec:categorisation})
from the simulated signal samples produced via ggF, VBF and $VH$ processes at $\mH = 125~\GeV$. 
A shift of 90~\MeV\ is applied
to the peak position $\mu_\mathrm{CB}$
to build a signal model for $\mH = 125.09~\GeV$.
For the high-mass search, a simultaneous fit is performed to all signal samples,
$\mX = $ [300--2500]~\GeV\ ($\mX = $ [250--2500]~\GeV) for the spin-0 (spin-2) interpretation. 
This allows a parameterisation of
the signal shape for masses \mX\ for which no simulation sample is available.
The \mX\ dependence of the signal shape parameters is parameterised by polynomials, 
and their coefficients are determined during the simultaneous fit.
The parameterisation is done separately for each of the three models considered, a spin-0 resonance
and a spin-2 resonance produced via either gluon--gluon or quark--antiquark initial states.

\begin{figure}
\begin{center}
\subfigure[]{\includegraphics[width=.50\textwidth]{fitplot_125.pdf}}%
\subfigure[]{\includegraphics[width=.50\textwidth]{fitplot_1000.pdf}}
\end{center}
\caption{The differential distribution of the invariant $Z\gamma$ mass ($m_{Z\gamma}$) for (a) Higgs
bosons with $m_H=125~\GeV$ in the low \ptt\ 
categories and (b) high-mass spin-0 particles produced 
via gluon--gluon fusion and with $m_X=1000~\GeV$, using the narrow width assumption (NWA). 
The markers show the $m_{Z\gamma}$ distributions and 
the solid and dotted lines the fitted parameterisations used in the searches.
The bottom part of the figures shows the residuals between the markers and the 
parameterisation.}
\label{fig:sig-mmg-fit}
\end{figure}


Figure~\ref{fig:sig-mmg-fit} shows the MC-simulated $m_{Z\gamma}$ distribution at 
$\mH = 125$~\GeV\ for the low \ptt categories and at $\mX = 1000$~\GeV. 
Similar fit qualities are obtained for all the categories in both searches.

Additionally, the signal efficiency defined as the number of events satisfying all the selection 
criteria (as given in Section~\ref{sec:eventSelection}) normalised to the
total number of events is needed to extract \sigbr ~from the measured yield.
For the $H\to Z\gamma$ search,
the signal efficiency times the acceptance in each category are shown in 
Table~\ref{tab:cat-eff}.\footnote{The efficiency difference between $m_H = 125~\GeV$ and
$m_H = 125.09~\GeV$ is estimated to be smaller than 1\%.}  

For the search for high-mass resonances, the signal efficiency is parameterised as a function 
of the resonance mass with an exponentiated second-order polynomial. 
Figure~\ref{fig:efficiency}(a) shows the
reconstruction and selection efficiency for $X\to Z(\to\ell\ell)\gamma$ events for a spin-0 resonance
produced in gluon--gluon fusion, separately for $Z\to ee$ and $Z\to\mu\mu$. 
The efficiencies range from about 30\% to about 46\% in the mass range from
250~\GeV\ to 2.4~\TeV. For a spin-0
resonance produced via vector-boson fusion, the efficiency is larger by up to 4\% over the full
resonance mass range considered.
Figure~\ref{fig:efficiency}(b) shows the reconstruction and selection efficiency for 
spin-2 resonances produced via gluon--gluon and quark--antiquark initial states 
as a function of the resonance mass. 
For spin-2 resonances
produced in gluon--gluon (quark--antiquark) initial states, the efficiencies range from about 22\% (28\%) to about 
35\% (54\%) in the mass range from 250~\GeV\ to 2.4~\TeV.
The efficiency differences between the spin-0 resonance
produced via gluon--gluon fusion, the spin-2 resonance produced via gluon--gluon initial states
and the spin-2 resonance produced via quark--antiquark initial states are primarily related to
the different photon transverse momentum distributions between the different production mechanisms.

\begin{figure}
\begin{center}
\subfigure[]{\includegraphics[width=0.5\textwidth]{ggH-signaleffvsmH.pdf}}%
\subfigure[]{\includegraphics[width=0.5\textwidth]{high-mass-efficiency-spin-2.pdf}}
\end{center}
\caption{Reconstruction and selection efficiency (including kinematic acceptance) for the 
$X\to Z\gamma$ final state as a function of the resonance mass $m_X$ (a) for
a spin-0 resonance via gluon--gluon fusion, separately for the $ee$ and the 
$\mu\mu$ categories, and 
(b) for a spin-2 resonance produced via either the gluon--gluon or the quark--antiquark 
initial states.
The markers show the efficiencies obtained from simulation, while the curves represent the 
parameterisation used in the analysis. The efficiencies are given with respect to (a)
$X\to Z(\to ee)\gamma$ and $X\to Z(\to \mu\mu)\gamma$, respectively, and (b) 
$X\to Z(\to \ell\ell)\gamma$ where $\ell = e, \mu$.}
\label{fig:efficiency}
\end{figure}


