\documentclass[fleqn,10pt]{wlscirep_SI}
\usepackage[utf8]{inputenc}
\usepackage[T1]{fontenc}
% \title{Scientific Reports Title to see here}
\title{Supplementary Information for "Local impacts on road networks and access to critical locations during extreme floods"}

% \author[1,*]{Alice Author}
% \author[2]{Bob Author}
% \author[1,2,+]{Christine Author}
% \author[2,+]{Derek Author}
% \affil[1]{Affiliation, department, city, postcode, country}
% \affil[2]{Affiliation, department, city, postcode, country}
\author[1,2,*]{Simone Loreti}
\author[3]{Enrico Ser-Giacomi}
\author[1,2]{Andreas Zischg}
\author[1,2,4,5]{Margreth Keiler}
\author[6,7,*]{Marc Barthelemy} 
\affil[1]{University of Bern, Institute of Geography, Bern, 3012, Switzerland}
\affil[2]{University of Bern, Oeschger Centre for Climate
Change Research, Mobiliar Lab for Natural Risks, Bern, 3012, Switzerland}
\affil[3]{Massachusetts Institute of Technology, Department of Earth, Atmospheric and Planetary Sciences, Cambridge MA, 02139, United States}
\affil[4]{University of Innsbruck, Department of Geography, Innsbruck, 6020, Austria}
\affil[5]{Austrian Academy of Sciences, Institute of Interdisciplinary Mountain Research, Innsbruck, 6020, Austria}
\affil[6]{Institut de Physique Th\'eorique, CEA,
CNRS-URA 2306, Gif-surYvette, F-91191, France}
\affil[7]{Centre d’Analyse et de Math\'ematique Sociales (CNRS/EHESS), Paris, 75006, France}

% \affil[*]{corresponding.author@email.example}
\affil[*]{To whom correspondence should be addressed. E-mail: simone.loreti@giub.unibe.ch or marc.barthelemy@ipht.fr}
% \correspondingauthor{\textsuperscript{1}To whom correspondence should be addressed. E-mail: simone.loreti@giub.unibe.ch or marc.barthelemy@ipht.fr}







\begin{document}


% \flushbottom
\maketitle
% * <john.hammersley@gmail.com> 2015-02-09T12:07:31.197Z:
%
%  Click the title above to edit the author information and abstract
%
% \thispagestyle{empty}

% \noindent Please note: Abbreviations should be introduced at the first mention in the main text – no abbreviations lists. Suggested structure of main text (not enforced) is provided below.



\vspace{-0.5cm}
\noindent 
\textbf{Date:}
31 January 2022.\\
\textbf{Disclaimer:}
This research article has been published the 28th of January 2022 on \href{https://www.nature.com/articles/s41598-022-04927-3 }{Scientific Reports}.\\
\textbf{DOI:} https://doi.org/10.1038/s41598-022-04927-3.\\
\textbf{Cite this article:}
Download citation from the \href{https://www.nature.com/articles/s41598-022-04927-3#citeas}{Scientific Reports} citation tool.
\vspace{0.5cm}





\section*{Supplementary Text}
\subsection*{Generalized logistic function}

% use only for debug \cref{fig:map_giantcomponent_76_hour}

% Type or paste text here. This should be additional explanatory text such as an extended technical description of results, full details of mathematical models, etc.   
We consider the generalised Verhulst's equation \cite{Verhulst},
% $$
% (F-C_1)(t) = \frac{D_{1}}{(1+A_{1}e^{-\beta_{1}{\gamma_1}(t-t_0)})^{1/{\gamma_1}}} + C_{1}
% $$
\begin{align}
\frac{dF(t)}{dt} &= \beta_{1} \biggl(F-C_1\biggr) \biggl(1- \biggl(\frac{F-C_1}{D_{1}} \biggr)^{\gamma_1} \biggr)
\intertext{where $C_1$ is the lower asymptote, $D_1$ is the upper asymptote, $\gamma_1$ is an exponent which allows to vary the shape of the (solution) sigmoidal curve and $\beta_1$ is the \emph{intrinsic growth rate} \cite{Tsoularis} indicating the curve steepness. We then derive the 
corresponding generalised logistic function as follows:}
% corresponding extended logistic function as follows:}
\beta_{1} \int_{t_0}^t dt &= \int \frac{dF}{\biggl(F-C_1\biggr)\biggl(1- \biggl(\frac{F-C_1}{D_{1}} \biggr)^{\gamma_1} \biggr)} \nonumber
\\
\beta_{1} (t-t_0) &= ln\biggl|F-C_1\biggr| - \frac{ln\biggl|1-\biggl(\frac{F-C_1}{D_{1}}\biggr)^{\gamma_1}\biggr|}{{\gamma_1}} + \text{const.} \nonumber
\\
{\gamma_1} \beta_{1} (t-t_0) &= ln\biggl|F-C_1\biggr|^{\gamma_1} - ln\biggl|1-\biggl(\frac{F-C_1}{D_{1}}\biggr)^{\gamma_1}\biggr| + \text{const.} \nonumber
\\
- {\gamma_1} \beta_{1} (t-t_0) &= - ln\biggl|F-C_1\biggr|^{\gamma_1} + ln\biggl|1-\biggl(\frac{F-C_1} {D_{1}}\biggr)^{\gamma_1}\biggr| + \text{const.} \nonumber
\\
- \text{const} - {\gamma_1} \beta_{1} (t-t_0) &= ln\left| \frac{1-\biggl(\frac{F-C_1} {D_{1}}\biggr)^{\gamma_1}}{(F-C_1)^{\gamma_1}} \right| \nonumber
%
\\ \intertext{we use $W = \pm e^{- \text{const}}$,}
%
W e^{- {\gamma_1} \beta_{1} (t-t_0)} &= \frac{1-\biggl(\frac{F-C_1} {D_{1}}\biggr)^{\gamma_1}}{(F-C_1)^{\gamma_1}} \nonumber
\\
W e^{- {\gamma_1} \beta_{1} (t-t_0)} &= \biggl( 1-\biggl(\frac{F-C_1} {D_{1}}\biggr)^{\gamma_1} \biggr) \frac{1}{(F-C_1)^{\gamma_1}} \nonumber
\\
W e^{- {\gamma_1} \beta_{1} (t-t_0)} &= \frac{1}{(F-C_1)^{\gamma_1}} - \frac{1}{D_{1}^{\gamma_1}} \nonumber
\\
\frac{1}{D_{1}^{\gamma_1}} + W e^{- {\gamma_1} \beta_{1} (t-t_0)} &= \frac{1}{(F-C_1)^{\gamma_1}} \nonumber
\\
\frac{1 + D_{1}^{\gamma_1} W e^{- {\gamma_1} \beta_{1} (t-t_0)}}{D_{1}^{\gamma_1}} &= \frac{1}{(F-C_1)^{\gamma_1}} \nonumber
%
\\ \intertext{we use $A_{1} = D_{1}^{\gamma_1} W$,} 
%
(F-C_1)^{\gamma_1} &= \frac{D_{1}^{\gamma_1}}{1 + A_{1} e^{- {\gamma_1} \beta_{1} (t-t_0)}} \nonumber
\\
F-C_1 &= \frac{D_{1}}{\biggl( 1 + A_{1} e^{- {\gamma_1} \beta_{1} (t-t_0)}\biggr)^{1/{\gamma_1}}} \nonumber
%
% \\ \intertext{Finally, we obtain the extended logistic function:} 
\\ \intertext{Finally, we obtain the generalised logistic function:} 
%
F(t) &= \frac{D_{1}}{\biggl(1+A_{1}e^{-\beta_{1}{\gamma_1}(t-t_0)}\biggr)^{1/{\gamma_1}}} + C_{1}
\label{extended_logistic_function}
\end{align}
With lower asymptote equal to $C_1=0$, \cref{extended_logistic_function} would be solution of the Richards differential equation \cite{Nelder,Richards,Wang2012,Turner}.



% \section*{Heading}
% \subsection*{Subhead}
% Type or paste text here. You may break this section up into subheads as needed (e.g., one section on ``Materials'' and one on ``Methods'').

% \subsection*{Materials}
% Add a materials subsection if you need to.

% \subsection*{Methods}
% Add a methods subsection if you need to.

% %%% Each figure should be on its own page
% \begin{figure}
% \centering
% \includegraphics[width=\textwidth]{example-image}
% \caption{First figure}
% \end{figure}

% \begin{figure}
% \centering
% \includegraphics[width=\textwidth]{frog}
% \caption{Second figure}
% \end{figure}

% \begin{table}\centering
% \caption{This is a table}

% \begin{tabular}{lrrr}
% Species & CBS & CV & G3 \\
% \midrule
% 1. Acetaldehyde & 0.0 & 0.0 & 0.0 \\
% 2. Vinyl alcohol & 9.1 & 9.6 & 13.5 \\
% 3. Hydroxyethylidene & 50.8 & 51.2 & 54.0\\
% \bottomrule
% \end{tabular}
% \end{table}


% 
% ------ FIGURES -------------------------------------------------
%
\section*{Supplementary Figures}
%
% ------ PERCOLATION Map Giant Component -------------------------
%
\begin{figure}[H]
\centering
% \includegraphics[width=11.4cm]{figure1/map_giantcomponent_76_hour_crop5.png}
\includegraphics[width=12cm]{figure1/Sup_Fig_1.pdf}


% \includegraphics[width=\linewidth]{figure1/borders4_crop.png}
\caption{A representative visual comparison between the percolation approach and our proposed framework and metrics.
\textbf{(a)} Snapshot of the road network and of the three largest connected components during the maximum extension of a real-like flood \citep{Zischg}, at $t_{MFE}=76$ hours. The percolation approach shows that the mobility of people occurs within each of the clusters, but without the possibility of crossing from one cluster to another since they are disconnected. However, we do not have any information about the ``internal'' situation of each cluster. For example, it looks like that (apparently) vehicles can freely circulate without any obstacle or speed braking within the giant component.
\textbf{(b)} Illustration of the road network, with highlight on the maximum value of $|\zeta_c|$ over the entire flooding period (for each town). With our approach, differently by percolation, we are able to infer the internal dynamics of each cluster, identifying the pivotal towns where large variations occur, both in space and in time. In particular, \textbf{(b)} shows the ``re-routing mechanism'' which corresponds to the mutual exchange of nodes among two or more adjacent towns (or groups of towns). We can clearly observe this mechanism between Thun and Thierachern, where Thierachern acquires some of the nodes lost by Thun, and between Kirchenthurnen and Gelterfingen, where Kirchenthurnen acquires some of the nodes lost by Gelterfingen. If we extend further our vision, we could observe the mechanism of mutual exchange of nodes among other adjacent towns, i.e. between the two groups of (i) Kirchenthurnen-Mühledorf and (ii) Gelterfingen-Wichtrach-Jaberg. Exception is made for the town of Münsingen, which just loses many nodes but there are not adjacent towns which acquire its lost nodes. 
\textcolor{black}{This figure was produced with Matlab \cite{MATLAB2019}.}
}
\label{fig:map_giantcomponent_76_hour}
\end{figure}


% 
% ------ TABLES --------------------------------------------------
%
% 
\section*{Supplementary Tables}

% 
% ------ TABLE 1 -------------------------------------------------
%
\begin{table}[H]
\centering
% \caption{Fixed-width columns.}
% https://tex.stackexchange.com/questions/209802/footnote-in-table-environment
% \begin{threeparttable}
% \begin{tabular}[t]{ c c c c c c c c c c c}
\begin{tabular}[t]{ c c c c c c c c c c}
\toprule
% &Interurban Speed Limit&Urban Speed Limit\tnote{*}\\
% &$R^2$&$A$&$B$&$C$&$D$&$E$&$t_p$&$\alpha$&$\beta$&$\gamma$\\
&$R^2$&$A$&$B$&$C$&$D$&$E$&$t_p$&$\beta$&$\gamma$\\
\midrule
$\frac{D_1}{\big(B_1 + A_1 e^{-E_1 (t-t_p)}\big)^{1/\gamma_1} } + C_1$ 
% $\frac{D_1}{\big(B_1 + A_1 e^{-\beta_1 (t-t_0)}\big)^{1/\gamma_1} } + C_1$ 
% & $0.9958$ & $14.92$ & $1.02$ & $50.00$ & $864.50$ & $52.55$ & $0.2248$ & $1.5420$ \\
& $0.9959$ & $16.60$ & $0.97$ & $50.00$ & $834.70$ & $0.2295$ & $52.2$  & $0.145$ & $1.581$\\
$B_2 + A_2 e^{\beta_2 t}$ & $0.9975$ & $3.73$ & $35.89$ & & & & &  $0.0698$ \\
% $A_2 e^{\beta_2 t} + B_2$ & $0.9974$ & $3.72$ & $35.93$ & & & & $0.0699$ \\
% $A_3 e^{-\beta_3 e^{-\gamma_2 t}} + B_3$ & $0.9904$ & $1053$ & $67.31$ & & & & $345.2$ & $0.0959$ \\
$A_3 e^{-\gamma_3 t}$ & $0.9981$ & $2334$ & & & & & & &  $0.0134$ \\


$B_4 + A_4 e^{\beta_4 t}$ & $0.9800$ & $64.94$ & $416.30$ & & & & &  $0.0434$ \\


$B_5 + A_5 e^{-\gamma_5 t}$ & $0.9911$ & $3836$ & $800.60$ & & & & & &  $0.0158$ \\
% $A_4 e^{\gamma_3 t}$ & $0.9980$ & $2334$ & & & & & & $-0.0134$ \\
\bottomrule
\end{tabular}
%   \begin{tablenotes}
%   \item[*] Speed limits are lower for urban areas, for the road types indicated.
%   %\item[a] Another footnote.
%   \end{tablenotes}
%   \end{threeparttable}
  \caption{Best fit values.}
  \label{Tab:table_bestfit}
\end{table}


% ------ TABLE 2 -------------------------------------------------
%
\begin{table}[H]
\centering
% \caption{Fixed-width columns.}
% https://tex.stackexchange.com/questions/209802/footnote-in-table-environment
% \begin{threeparttable}
\begin{tabular}[t]{l>{\raggedright}p{0.35\linewidth}>{\raggedright\arraybackslash}p{0.3\linewidth}}
\toprule
% &Interurban Speed Limit&Urban Speed Limit\tnote{*}\\
&Interurban Speed Limit $[km/h]$&Urban Speed Limit $[km/h]$\\
\midrule
1 m road&-&50 \\
% 1 m way fragment&-&50 \\
2 m road&80&50 \\
% 2 m way fragment&80&50 \\
3 m road&80&50 \\
4 m road&80&50 \\
6 m road&80&50 \\
8 m road&80&50 \\
10 m road&80&50 \\
entrance&50&- \\
exit&50&- \\
highway&100&- \\
motorways&120&- \\
% car's train&0&- \\
% serves access&0&- \\
% ferry&0&- \\
% ferrata&0&- \\
% marked track&0&- \\
service area&20&- \\
service areas connection&20&- \\
service area entrance&40&- \\
square&-&20 \\
\bottomrule
\end{tabular}
%   \begin{tablenotes}
%   \item[*] Speed limits are lower for urban areas, for the road types indicated.
%   %\item[a] Another footnote.
%   \end{tablenotes}
%   \end{threeparttable}
  \caption{Speed limits adopted in this study for different road types and widths \cite{Finocchio}.}
  \label{Tab:speedlimits}
\end{table}%





%%% Add this line AFTER all your figures and tables
% \FloatBarrier
%% Loreti - 19 February 2021
% Forcing bibliography to the end [duplicate]
% https://tex.stackexchange.com/questions/28898/forcing-bibliography-to-the-end
\clearpage










% \bibliography{SciRep_sample}
\bibliography{supp_SciRep_Loreti_V2_SupplementaryInformation}

\end{document}