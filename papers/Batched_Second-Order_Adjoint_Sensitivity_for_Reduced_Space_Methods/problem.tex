\section{Reduced space problem}
In \refsec{sec:background:powerflow} we briefly introduce the power flow
nonlinear equations to motivate our application. We present in \refsec{sec:background:reducedspace}
the reduced space problem associated with the power flow problem, and recall
in \refsec{sec:background:adjoint} the first-order adjoint method, used to evaluate
efficiently the gradient in the reduced space, and later applied to compute the adjoint of the sensitivities.

\subsection{Presentation of the power flow problem.}
\label{sec:background:powerflow}
We present a brief overview of the steady-state solution of the power flow
problem. The power grid can be described as a graph $\mathcal{G} = \{V, E\}$
with $n_v$ vertices and $n_e$ edges. The steady state of the network is described
by the following nonlinear equations, holding at all nodes $i \in V$,
\begin{equation}
  \label{eq:powerflow}
  \left\{
  \begin{aligned}
  P_i^{inj} &= v_i \sum_{j \in A(i)} v_j (g_{ij}\cos{(\theta_i - \theta_j)} + b_{ij}\sin{(\theta_i - \theta_j})) \,,  \\
  Q_i^{inj} &= v_i \sum_{j \in A(i)} v_j (g_{ij}\sin{(\theta_i - \theta_j)} - b_{ij}\cos{(\theta_i - \theta_j})) \,,
  \end{aligned}
  \right.
\end{equation}
where at node $i$, $(P_i^{inj}$ and  $Q_i^{inj})$ are respectively the active and
reactive power injections; $v_i$ is the voltage magnitude; $\theta_i$ the
voltage angle; and $A(i) \subset V$ is the set of adjacent nodes:
for all $j \in A(i)$, there exists a line $(i, j)$ connecting node $i$ and node $j$.
The values $g_{ij}$ and $b_{ij}$ are associated with the physical characteristics
of the line $(i, j)$. Generally, we distinguish the (\emph{PV}) nodes --- associated
to the generators --- from the  (\emph{PQ}) nodes comprising only loads.
We note that the structure of the nonlinear equations~\refeqn{eq:powerflow} depends
on the structure of the underlying graph through the adjacencies~$A(\cdot)$.

We rewrite the nonlinear equations~\refeqn{eq:powerflow}
in the standard form~\refeqn{eq:nonlinearsystem}.
At all nodes the power injection $P^{inj}_i$
should match the net production $P_i^g$ minus the load $P_i^d$:
\begin{equation}
  \label{eq:powerflowvec}
  g(\bm{x}, \bm{p}) =
    \begin{bmatrix}
      \bm{P}_{pv}^{inj} - \bm{P}^{g} + \bm{P}_{pv}^d \\
      \bm{P}_{pq}^{inj} + \bm{P}_{pq}^d \\
      \bm{Q}_{pq}^{inj} + \bm{Q}_{pd}^d
    \end{bmatrix}
    = 0
    ~ \text{with} ~
    \bm{x} =
    \begin{bmatrix}
      \bm{\theta}^{pv} \\ \bm{\theta}^{pq} \\ \bm{v}^{pq}
    \end{bmatrix} \,.
\end{equation}
In~\refeqn{eq:powerflowvec}, we have selected only a subset of the power flow equations
\refeqn{eq:powerflow} to ensure that the nonlinear system $g(\bm{x}, \bm{p}) = 0$
is invertible with respect to the state $\bm{x}$. The unknown variable $\bm{x}$
corresponds to the voltage
angles at the PV and PQ nodes and the voltage magnitudes at the PQ nodes.
However, in contrast to the variable $\bm{x}$, we have some flexibility in
choosing the parameters $\bm{p}$.

In optimal power flow (OPF) applications, we are looking at minimizing
a given operating cost $f: \mathbb{R}^{n_x} \times \mathbb{R}^{n_p} \to \mathbb{R}$
(associated to the active power generations $\bm{P}^g$) while satisfying
the power flow equations~\eqref{eq:powerflowvec}. In that particular case,
$\bm{p}$ is a design variable associated to the active power generations
and the voltage magnitude at PV nodes: ~$\bm{p} = (\bm{P}^g, \bm{v}_{pv})$.
We define the OPF problem as
\begin{equation}
  \label{eq:nonlinearopt}
  \min_{\bm{x}, \bm{p}} \; f(\bm{x}, \bm{p}) ~ \text{ subject to }
  g(\bm{x}, \bm{p}) = 0 \; .
\end{equation}

\subsection{Projection in the reduced space.}
\label{sec:background:reducedspace}
We note that in Equation~\eqref{eq:powerflowvec}, the functional $g$ is continuous
and that the dimension
of the output space is equal to the dimension of the input variable $\bm{x}$. Thanks
to the particular network structure of the problem (encoded by the adjacencies $A(\cdot)$
in \eqref{eq:powerflow}), the Jacobian $\nabla_{\bm{x}} g$ is sparse.

Generally, the nonlinear system~\eqref{eq:powerflowvec} is solved iteratively
with a Newton-Raphson algorithm.
If at a fixed parameter $\bm{p}$ the Jacobian $\nabla_{\bm{x}}g$ is invertible,
we compute the solution $x(\bm p)$ iteratively, starting from an initial guess $\bm{x}_0$:
$\bm{x}_{k+1} = \bm{x}_k - (\nabla_{\bm{x}} g_k)^{-1} g(\bm{x}_k, \bm{p})$
for $k=1,\dots, K$.
We know that if $\bm{x}_0$ is close enough to the solution, then
the convergence of the algorithm is quadratic.

With the projection completed, the optimization problem~\eqref{eq:nonlinearopt}
rewrites in the reduced space as
\begin{equation}
  \label{eq:nonlinearoptreduced}
  \min_{\bm{p}} \; F(\bm p) := f\big(x(\bm{p}), \bm{p}\big) \; ,
\end{equation}
reducing the number of optimization variables from $n_x+n_p$ to $n_p$,
while at the same time eliminating all equality constraints in the formulation.


\subsection{First-Order Adjoint Method.}
\label{sec:background:adjoint}

With the reduced space problem~\eqref{eq:nonlinearoptreduced} defined,
we compute the reduced gradient $\nabla_{\bm{p}} F$ required for the reduced space optimization routine.
By definition, as $x(\bm{p})$ satisfies $g(x(\bm{p}), \bm{p}) = 0$, the chain rule yields
$\nabla_{\bm{p}} F = \nabla_{\bm{p}} f + \nabla_{\bm{x}} f \cdot \nabla_{\bm{p}} x$
with
$\nabla_{\bm{p}} x= - \big(\nabla_{\bm{x}}g)^{-1} \nabla_{\bm{p}}g$.
However, evaluating the full sensitivity matrix $\nabla_{\bm{p}} x$ involves the
resolution of $n_x$ linear system.
On the contrary, the \emph{adjoint method} requires solving a
\emph{single} linear system.
For every dual $\bm{\lambda} \in \mathbb{R}^{n_x}$, we introduce a
Lagrangian function defined as
\begin{equation}
  \label{eq:lagrangian}
 \ell(\bm{x}, \bm{p}, \bm{\lambda}) := f(\bm{x}, \bm{p}) + \bm{\lambda}^\top g(\bm{x}, \bm{p})
 \; .
\end{equation}
If $\bm{x}$ satisfies $g(\bm{x}, \bm{p}) = 0$, then the Lagrangian $\ell(\bm{x},
\bm{p}, \bm{\lambda})$ does not depend on $\bm{\lambda}$ and we
get $\ell(\bm{x}, \bm{p}, \bm{\lambda}) = F(\bm{p})$. By
using the chain rule, the total derivative of $\ell$
with relation to the parameter $\bm{p}$ satisfies
\begin{equation*}
  \begin{aligned}
    d_p \ell &= \big(\nabla_{\bm{x}} f \cdot \nabla_{\bm{p}} x +
    \nabla_{\bm{p}} f  \big) +
    \bm{\lambda}^\top \big(\nabla_{\bm{x}} g \cdot \nabla_{\bm u} x +
    \nabla_{\bm{p}} g \big)  \\
             &= \big(\nabla_{\bm{p}} f + \bm{\lambda}^\top \nabla_{\bm{p}} g \big) +
             \big(\nabla_{\bm{x}} f + \bm{\lambda}^\top\nabla_{\bm{x}} g \big) \nabla_{\bm{p}} x  \;.
  \end{aligned}
\end{equation*}
We observe that
by setting the first-order adjoint to $\bm{\lambda} = - (\nabla_{\bm{x}}g)^{-\top}\nabla_{\bm{x}} f^\top$,
the reduced gradient $\nabla_{\bm{p}} F$ satisfies
\begin{equation}
  \label{eq:reduced_gradient}
  \nabla_{\bm{p}} F =
  \nabla_{\bm{p}} \ell =
  \nabla_{\bm{p}} f +
  \bm{\lambda}^\top \nabla_{\bm{p}} g
  \; ,
\end{equation}
with $\bm{\lambda}$ evaluated by solving a single linear system.
