% Arxiv wants this to force pdflatex
\pdfoutput=1 
% %%%%%%%%%%%%%%%%%%%%%%%%%  ltexpprt_twocolumn.tex  %%%%%%%%%%%%%%%%%%%%%%%%%%%%%%%%
%
% This is ltexpprt_twocolumn.tex, an example file for use with the SIAM LaTeX2E
% Preprint Series macros. It is designed to provide two-column output.
% Please take the time to read the following comments, as they document
% how to use these macros. This file can be composed and printed out for
% use as sample output.

% Any comments or questions regarding these macros should be directed to:
%
%                 Donna Witzleben
%                 SIAM
%                 3600 University City Science Center
%                 Philadelphia, PA 19104-2688
%                 USA
%                 Telephone: (215) 382-9800
%                 Fax: (215) 386-7999
%                 e-mail: witzleben@siam.org


% This file is to be used as an example for style only. It should not be read
% for content.

%%%%%%%%%%%%%%% PLEASE NOTE THE FOLLOWING STYLE RESTRICTIONS %%%%%%%%%%%%%%%

%%  1. There are no new tags.  Existing LaTeX tags have been formatted to match
%%     the Preprint series style.
%%
%%  2. Do not change the margins or page size!  Do not change from the default
%%     text font!
%%
%%  3. You must use \cite in the text to mark your reference citations and
%%     \bibitem in the listing of references at the end of your chapter. See
%%     the examples in the following file. If you are using BibTeX, please
%%     supply the bst file with the manuscript file.
%%
%%  4. This macro is set up for two levels of headings (\section and
%%     \subsection). The macro will automatically number the headings for you.
%%
%%  5. No running heads are to be used for this volume.
%%
%%  6. Theorems, Lemmas, Definitions, Equations, etc. are to be double numbered,
%%     indicating the section and the occurrence of that element
%%     within that section. (For example, the first theorem in the second
%%     section would be numbered 2.1. The macro will
%%     automatically do the numbering for you.
%%
%%  7. Figures and Tables must be single-numbered.
%%     Use existing LaTeX tags for these elements.
%%     Numbering will be done automatically.
%%
%%  8. Page numbering is no longer included in this macro.
%%     Pagination will be set by the program committee.
%%
%%
%%%%%%%%%%%%%%%%%%%%%%%%%%%%%%%%%%%%%%%%%%%%%%%%%%%%%%%%%%%%%%%%%%%%%%%%%%%%%%%



\documentclass[twoside,leqno,twocolumn]{article}


% Comment out the line below if using A4 paper size
\usepackage[letterpaper]{geometry}

\usepackage{ltexpprt}
\usepackage{hyperref}

\usepackage{amsmath}
\usepackage{amssymb}
\usepackage[ruled,vlined,algo2e]{algorithm2e}
\usepackage[usebox=true]{jlcode}
\usepackage{bm}
\usepackage{tikz}
\usepackage[numbers]{natbib}
\usepackage{pgfplotstable}
\usepackage{xcolor}
\usepackage[normalem]{ulem}
\usepackage{multirow}
\usepackage{listings}
\pgfplotsset{compat=1.16}
\usetikzlibrary{arrows,shadows,positioning}
\definecolor{mygreen}{RGB}{80,200,120}

\tikzset{
  connection/.style={
    inner sep=0, outer sep=0,
  },
  frame/.style={
    rectangle, draw,
    text width=10em, text centered,
    minimum height=2.5em,
  },
  parallel/.style={
    rectangle, draw,
    text width=2em, text centered,
    minimum height=1.5em,drop shadow,fill=mygreen!100,
  },
  hframe/.style={
    rectangle, draw,
    text width=10em, text centered,fill=red!20,
    minimum height=2.5em,
  },
  input/.style={
    rectangle, draw,
    text width=2em, text centered,
    minimum height=1em,drop shadow,fill=red!20,
    rounded corners,
  },
  line/.style={
    draw, -latex',rounded corners=3mm,
  },
  seq/.style={
    rectangle, draw,
    minimum width=2em, text centered,
    minimum height=2em,drop shadow,fill=blue!20,
    rounded corners,
  },
  n/.style={circle, fill, minimum size=4pt,inner sep=0pt, outer sep=0pt},
}

% \usepackage{listing}
\newcommand{\REAL}{\mathbb{R}}

\newcommand{\dpart}[2]{\dfrac{\partial #1}{\partial #2}}
\newcommand{\deriv}[2]{\dfrac{d #1}{d #2}}
\newcommand{\grad}[2]{\nabla_{#2}{#1}}
\newcommand{\hess}[2]{\dfrac{d^2 #1}{d #2}}
\newcommand{\dhess}[2]{\dfrac{\partial^2 #1}{\partial #2}}
\newcommand{\hhess}[2]{\nabla^2_{#2}{#1}}

\newcommand{\CUDAjl}{{\tt CUDA.jl}}
\newcommand{\cusolver}{{\tt cuSOLVER}}
\newcommand{\cusolverrf}{{\tt cuSOLVER\_RF}}
\newcommand{\krylovjl}{{\tt Krylov.jl}}
\newcommand{\KA}{{\tt KernelAbstractions.jl}}
\newcommand{\bigo}[1]{\mathcal{O}\left( #1 \right)}

% various reference macros
\newcommand{\refalg}[1]{Algorithm~\ref{#1}}
\newcommand{\refsec}[1]{Section~\ref{#1}}
\newcommand{\reffig}[1]{Figure~\ref{#1}}
\newcommand{\refsubfig}[1]{Figure~\subref{#1}}
\newcommand{\reftab}[1]{Table~\ref{#1}}
\newcommand{\refeqn}[1]{(\ref{#1})}
\newcommand{\reflst}[1]{Listing~(\ref{#1})}

\newcommand{\ms}[1]{\textcolor{blue}{#1}}
\newcommand{\stms}[1]{\sout{\textcolor{blue}{#1}}}

\newcommand{\am}[1]{\textcolor{blue}{#1}}
\newcommand{\stam}[1]{\sout{\textcolor{blue}{#1}}}

\newcommand{\fp}[1]{\textcolor{blue}{#1}}
\newcommand{\stfp}[1]{\sout{\textcolor{blue}{#1}}}

\pgfplotstableread{data/J.dat}\matrixJ
\pgfplotstableread{data/Jc.dat}\matrixJc
\def\nrows{52}
\def\ncols{52}
\def\ncolors{14}

% to avoid overfull hbox due to \tt etc.
\tolerance=5000

\begin{document}
\newcommand\relatedversion{}

% \renewcommand\relatedversion{\thanks{The full version of the paper can be accessed at \protect\url{https://arxiv.org/abs/1902.09310}}} % Replace URL with link to full paper or comment out this line

%%
%% The "title" command has an optional parameter,
%% allowing the author to define a "short title" to be used in page headers.
% \title{\Large Revisiting Reduced Space Hessian Accumulation for SIMD Architectures \relatedversion}
% \title{\Large Batched Hessian Contractions of Nonlinear Systems of Equations for Reduced Space Methods\relatedversion}
\title{\Large Batched Second-Order Adjoint Sensitivity for Reduced Space Methods\relatedversion}

%%
%% The "author" command and its associated commands are used to define
%% the authors and their affiliations.
%% Of note is the shared affiliation of the first two authors, and the
%% "authornote" and "authornotemark" commands
%% used to denote shared contribution to the research.
\author{François Pacaud \thanks{Argonne National Laboratory}
  \and Michel Schanen \footnotemark[1]
  \and Daniel Adrian Maldonado \footnotemark[1]
\and Alexis Montoison \thanks{$GERAD$ and Polytechnique Montréal}
\and Valentin Churavy \thanks{Massachusetts Institute of Technology}
\and Julian Samaroo \footnotemark[3]
\and Mihai Anitescu \footnotemark[1]
}

\date{}
\maketitle

% Copyright Statement
% When submitting your final paper to a SIAM proceedings, it is requested that you include
% the appropriate copyright in the footer of the paper.  The copyright added should be
% consistent with the copyright selected on the copyright form submitted with the paper.
% Please note that "20XX" should be changed to the year of the meeting.

% Default Copyright Statement
\fancyfoot[R]{\scriptsize{Copyright \textcopyright\ 2021 by SIAM\\
Unauthorized reproduction of this article is prohibited}}

% Depending on which copyright you agree to when you sign the copyright form, the copyright
% can be changed to one of the following after commenting out the default copyright statement
% above.

%\fancyfoot[R]{\scriptsize{Copyright \textcopyright\ 20XX\\
%Copyright for this paper is retained by authors}}

%\fancyfoot[R]{\scriptsize{Copyright \textcopyright\ 20XX\\
%Copyright retained by principal author's organization}}

%\pagenumbering{arabic}
%\setcounter{page}{1}%Leve this line commented out.


\begin{abstract} \small\baselineskip=9pt
  This paper presents an efficient method for extracting the second-order
  sensitivities from a system of implicit nonlinear equations on upcoming graphical processing units (GPU) dominated computer systems.
  We design a custom automatic differentiation (AutoDiff) backend that
  targets highly parallel architectures by
  extracting the second-order information in batch. When the nonlinear
  equations are associated to a reduced space optimization problem, we leverage the
  parallel reverse-mode accumulation in a batched adjoint-adjoint algorithm
  to compute efficiently the reduced Hessian of the problem.
  We apply the method to extract the reduced Hessian associated to the balance equations
  of a power network, and show on the largest instances that a parallel GPU implementation is 30 times faster than a sequential CPU reference based on UMFPACK
\end{abstract}


\section{Introduction}  \label{sec:introduction}

\newcommand\inexpIntro[3]{#1?(#2,#3).}
\newcommand\rinexpIntro[3]{*#1?(#2,#3).}
\newcommand\outexpIntro[3]{#1!(#2,#3).}
\newcommand\outatomIntro[3]{#1!(#2,#3)}

We propose a fully automated method for proving termination of \(\pi\)-calculus processes.
Although there have been a lot of studies on termination analysis for the \(\pi\)-calculus
and related calculi~\cite{Deng06IC,Demangeon07,SangiorgiTermination,KobayashiHybrid,Yoshida04IC,DBLP:journals/jlp/DemangeonHS10,Venet98SAS}, most of them have been rather theoretical,
and there have been surprisingly little efforts in developing  fully automated termination
verification methods and tools based on them. To our knowledge,
Kobayashi's \typical{}~\cite{TyPiCal,KobayashiHybrid} is the only exception that
can prove termination of \(\pi\)-calculus processes (extended with natural numbers)
fully automatically, but its termination analysis is quite limited (see Section~\ref{sec:relatedwork}).

Our method is based on a reduction to termination analysis for sequential programs:
we translate a \(\pi\)-calculus process \(P\) to a sequential program \(S_P\), so that
if \(S_P\) is terminating, so is \(P\). The reduction allows us to use
powerful, mature methods and tools
for termination analysis of sequential programs~\cite{heizmann2016ultimate,freqterm,DBLP:conf/lics/PodelskiR04,Kuwahara2014Termination,DBLP:journals/cacm/CookPR11}.

The idea of the translation is to convert a chain of communications on replicated input
channels to a chain of recursive function calls of the target sequential program.
Let us consider the following Fibonacci process:
\begin{align*}
    & \rinexpIntro{\fib}{n}{r}
        \ifexp{n<2}{ \soutatom{r}{1} \\ &\quad}
                   { \nuexp{s_1} \nuexp{s_2} (\outatomIntro{\fib}{n-1}{s_1} \PAR \outatomIntro{\fib}{n-2}{s_2} \PAR \sinexp{s_1}{x}\sinexp{s_2}{y}\soutatom{r}{x+y}) \\}
    & \PAR \outatomIntro{\fib}{m}{r}
\end{align*}
Here, the process
$\rinexpIntro{\fib}{n}{r} \ldots$ is a function server that computes the \(n\)-th Fibonacci number
in parallel and returns the result to \(r\),
and $\outatom{\fib}{m}{r}$ sends a request for computing the \(m\)-th Fibonacci number;
those who are not familiar with the syntax of the \(\pi\)-calculus may wish to consult
Section~\ref{sec:targetlanguage} first.
To prove that the process above is terminating for any integer \(m\),
it suffices to show that there is no infinite chain of communications on $\fib$:
\[
    \fib(m,r) \to \fib(m_1,r_1) \to \fib(m_2,r_2) \to \cdots.
\]
We convert the process above to the following program:\footnote{The actual translation
  given later is a little more complex.}
\begin{verbatim}
 let rec fib(n) = if n<2 then () else (fib(n-1) [] fib(n-2)) in
 fib(m)
\end{verbatim}
Here, \texttt{[]} represents the non-deterministic choice.
Note that, although the calculation of Fibonacci numbers is not preserved,
for each chain of communications on \texttt{fib}, there is a corresponding
sequence of recursive calls:
\[
\mathtt{fib}(m) \to \mathtt{fib}(m_1) \to \mathtt{fib}(m_2) \to \cdots.
\]
Thus, the termination of the sequential program above implies the termination of
the original process.
As shown in the example above, (i) each communication on a replicated input channel
is converted to a function call, (ii) each communication on a non-replicated input
channel is just removed (or, in the actual translation, replaced by a call of
a trivial function defined by \(f(\seq{x})=(\,)\)), and (iii) parallel composition
is replaced by a non-deterministic choice.
We formalize the translation outlined above and prove its correctness.

The basic translation sketched above sometimes loses too much information.
For example, consider the following process:
\begin{align*}
    & \rinexpIntro{\pre}{n}{r} \soutatom{r}{n-1} \\
    & \PAR \rinexpIntro{f}{n}{r} \ifexp{n<0}{ \soutatom{r}{1} }
                                       { \nuexp{s} (\outatomIntro{\pre}{n}{s} \PAR \sinexp{s}{x}\outatomIntro{f}{x}{r}) } \\
    & \PAR \outatomIntro{f}{m}{r}
\end{align*}
The translation sketched above would yield:
\begin{verbatim}
  let pred(n) = n-1 in
  let rec f(n) = if n<0 then () else (pred(n) [] f(*)) in
  f(m)
\end{verbatim}
Here, \texttt{*} represents a non-deterministic integer: since we have removed
the input $\sinatom{s}{x}$, we do not have information about the value of \( x \).
As a result, the sequential program above is non-terminating, although the original
process is terminating.
To remedy this problem, we also refine the basic translation above by using a refinement
type system for the \(\pi\)-calculus. Using the refinement type system,
we can infer that the value of \(x\) in the original process is less than \(n\),
so that we can refine the definition of \texttt{f} to:
\begin{verbatim}
 let rec f(n) = ... else (pred(n) [] let x=* in assume(x<n);f(x))
\end{verbatim}
The target program is now terminating, from which
we can deduce that the original process is also terminating.
We have implemented an automated tool based on the refined translation above.

The contributions of this paper are summarized as follows.
\begin{itemize}
\item The formalization of the basic translation from the \(\pi\)-calculus
  (extended with integers) to sequential programs, and a proof of its correctness.
\item The formalization of a refined translation based on a refinement type system.
\item An implementation of the refined translation, including automated refinement type
  inference based on CHC solving, and experiments to evaluate the effectiveness of
  our method.
\end{itemize}

The rest of this paper is structured as follows.
Section~\ref{sec:targetlanguage} introduces the source and target languages
of our translation.
Section~\ref{sec:approach} 
formalizes the basic translation, and proves its correctness.
Section~\ref{sec:refinement} refines the basic translation by using a refinement type system.
Section~\ref{sec:implementation} reports an implementation and experiments.
Section~\ref{sec:relatedwork} discusses related work,
and Section~\ref{sec:conclusion} concludes the paper.

%!TEX root = main.tex
\section{Problem Definition and Notations}
\label{sec:problem}







% In this section, we will first describe key concepts and notations used in this paper, and formally define our problem. Then we will use a case study to make our idea of story tree more concrete.

% \subsection{Problem Definition and Notations}
% \label{subsec:problem-define}

We first present some definitions of key concepts in the top-down hierarchy: \textit{topic} $\rightarrow$ \textit{story} $\rightarrow$ \textit{event} to be used in this paper.

\begin{definition}
  \textit{Event}: an event $\mathcal{E}$ is a set of one or several documents that contain highly similar information.
\end{definition}

\begin{definition}
  \textit{Story}: a story $\mathcal{S}$ is a tree of events that revolve around a group of specific persons and happen at certain places during specific times. A directed edge from event $\mathcal{E}_1$ to $\mathcal{E}_2$ indicates a temporal evolution or a logical connection from $\mathcal{E}_1$ to $\mathcal{E}_2$.
\end{definition}

\begin{definition}
  \textit{Topic}: a topic consists of a set of stories that are highly correlated or similar to each other.
  \vspace{-1mm}
\end{definition}


Each topic may contain multiple story trees, and each story tree consists of multiple logically connected events.
In our work, events (instead of news documents) are the smallest atomic units. Each event is also assumed to belong to a single story and contains partial information about that story.
For instance, considering the topic \textit{American presidential election}, \textit{2016 U.S. presidential election} is a story within this topic, and  \textit{Trump and Hilary's first television debate} is an event within this story.


We now introduce some notations and describe our problem formally. Given a news document stream $D = \{ \mathcal{D}_1, \mathcal{D}_2, \ldots, \mathcal{D}_t,\ldots \}$, where $\mathcal{D}_t$ is the set of news documents collected on time period $t$, our objective is to: a) cluster all news documents $D$ into a set of events $E = \{ \mathcal{E}_1, \ldots, \mathcal{E}_{|E|} \}$, and b) connect the extracted events to form a set of stories $S = \{ \mathcal{S}_1, ..., \mathcal{S}_{|S|} \}$. Each story $\mathcal{S} = (E, L)$ contains a set of events $E$ and a set of links $L$, where $L_{i,j} := <\mathcal{E}_i, \mathcal{E}_j>$ denotes a directed link from event $\mathcal{E}_i$ to $\mathcal{E}_j$, which indicates a temporal evolution or logical connection relationship.

%We now illustrate our problem with an example. (A example Fig) Fig... shows ...
Furthermore, we require the events and story trees to be extracted in an online or incremental manner. That is, we extract events from each $\mathcal D_t$ individually when the news corpus $\mathcal D_t$ arrives in time period $t$, and \emph{merge} the discovered events into the existing story trees that were found at time $t-1$. This is a unique strength of our scheme as compared to prior work, since we do not need to repeatedly process older documents and can deliver  a set of evolving yet logically consistent story trees to users.  

% \subsection{Case Study}
% \label{subsec:case-study}

\begin{figure}
\includegraphics[width=3.4in]{figure/StoryStructures}
\caption{Different structures to characterize a story.}
\vspace{-2mm}
\label{fig:storyStructures}
\vspace{-2mm}
\end{figure}

For example, Fig.~\ref{fig:CaseStudy} illustrates the story tree of ``2016 U.S. presidential election''. The story contains $20$ nodes, where each node indicates an event in 2016 U.S. election, and each link indicates a temporal evolution or a logical connection between two events. %For example, event $19$ says America votes to elect new president, and event $20$ says Donald Trump is elected president. 
The index number on each node represents the event sequence over the timeline. There are $6$ paths within this story tree, where the path $1 \rightarrow 20$ indicates the whole presidential election process, branch $3 \rightarrow 6$ is about Hilary's health conditions, branch $7 \rightarrow 13$ talks about television debates, $14 \rightarrow 18$ depicts the investigation into Hilary's ``mail door'', etc. As we can see, by modeling the evolutionary and logical structure of a story into a story tree, users can easily grasp the logic of news stories and learn the main information quickly. 


Let us represent each story by an empty root node $s$ from which the story is originated, and denote each event by an event node $e$. The events in a story can be organized in one of the following four structures shown in Fig. \ref{fig:storyStructures}: a) a flat structure that does not include dependencies between events; b) a timeline structure that organizes events by their timestamps; c) a graph structure that checks the connection between all pairs of events and maintains a subset of most strong connections; d) a tree structure, which represents a story's evolving structure by a tree.  

Compared with a tree structure, sorting events by timestamps omits the logical connection between events, while using directed acyclic graphs to model event dependencies without considering the evolving consistency of the whole story can leads to unnecessary connections between events.
Through extensive user experience studies in Sec.~\ref{sec:eval}, we show that tree structures are the most effective way to represent breaking news stories as compared to other structures, including the more complex graph structures. 

To demonstrate that collecting network data can reduce the posterior uncertainty about the parameters of the population model,
we consider a population consisting of $K=3$ subpopulations.
The $K=3$ subpopulations correspond to 
\bi
\item a low-degree subpopulation of size 127 with degree parameter $\gamma_1 = -3.5$;
\item a moderate-degree subpopulation of size 50 with degree parameter $\gamma_2 = -1.5$;
\item a high-degree subpopulation of size 10 with degree parameter $\gamma_3 = .5$.
\ei
% The degree parameters of population members $i$ are given by $\theta_i = \bZ_i^\top\, \bgamma$ ($i = 1, \dots, N$).
We generate 1,000 ego-centric samples of sizes $n = 25$,\, $50$,\, $75$,\, $100$,\, $125$,\, $150$,\, $187$ from the population of size $N = 187$.
We then estimate the population model from each sample of contacts along with observations of the exposure, infectious, and removal times of infected population members.
In addition,
we estimate the population model without observations of contacts,
which corresponds to a sample size of $n=0$,
using observations of the exposure, infectious, and removal times of infected population members.
% As pointed out in Section 5.2,
% collecting network data sampling helps infer the unobserved sources of infections,
% which in turn helps infer the population model.
To assess how much the posterior uncertainty about the parameters of the population model is reduced by sampling contacts,
we use the mean squared error (MSE) of the posterior median and mean of the parameters.

\newcommand{\anoise}{{\mathcal{AN}}}
\newcommand{\pnoise}{{\mathcal{PN}}}
\section{Stochastic Games for V-Formation}
\label{sec:sgv}

We describe the specialization of the stochastic-game verification problem to
V-formation.  In particular, we present the AMPC-based control strategy for reaching a V-formation, and the various attacker strategies against which we evaluate the resilience of our controller.

The MDP $\M$ for V-formation was presented in Section~\ref{sec:background}. The state variables of the MDP are the positions and velocities of the birds, and the control variables (defining the actions) are the accelerations and displacements. In the transition relation given in equation~(\ref{eq:v}), the attacker chooses the displacement $\vec{d}(t)$ it needs to manipulate the position of the birds,
whereas the controller chooses the acceleration $\vec{a}(t)$ to apply. Together, the pair $(\vec{a}(t),\vec{d}(t))$ defines the action that transforms one MDP state to another. We now define the controller's and attacker's strategies.

\subsection{Controller's Adaptive Strategies}

Given current state $(\vec{x}(t),\vec{v}(t))$, the controller's strategy $\sigma_C$ returns a probability distribution on the space of all possible accelerations (for all birds).  As mentioned above, this probability distribution is specified implicitly via a randomized algorithm that returns an actual acceleration (again for all birds).  This randomized algorithm is the AMPC algorithm, which inherits its randomization from the randomized PSO procedure it deploys.  

When the controller computes an acceleration, it assumes that the attacker does {\em{not}} introduce any disturbances; i.e., the controller uses the following model:
\vspace*{-4mm}\begin{eqnarray}
 \xv_i(t + 1) &=& \xv_i(t) + \vv_i(t+1) \qquad \forall~i\,{\in}\,\{1,\ldots,B\}, \nonumber \\
 \vv_i(t + 1) &=& \vv_i(t) + \va_i(t), \label{eq:noattack} %\\[-6mm]
\end{eqnarray}
where $\va(t)$ is the only control variable. Note that the controller chooses its next action $\va(t)$ based on the current configuration $(\xv(t),\vv(t))$ of the flock using MPC. The current configuration may have been influenced by the disturbance $\vec{d}(t-1)$ introduced by the attacker in the previous time step.  Hence, the current state need not be the state predicted by the controller when performing MPC in step $t-1$. Moreover, depending on the severity of the attacker action $\vec{d}(t-1)$, the AMPC procedure dynamically adapts its behavior, i.e.\ the choice of horizon $h$, in order to enable the controller to pick the best control action $\vec{a}(t)$ in response.

\subsection{Attacker's Strategies}

We are interested in evaluating the resilience of our V-formation controller when it is threatened by an attacker that can remove a certain number of birds from the flock, or manipulate a certain number of birds by taking control of their actuators (modeled by the displacement term in equation~(\ref{eq:trans})).
We assume that the attack lasts for a limited amount of time, after which the controller attempts to bring the system back into the good set of states. When there is no attack, the system behavior is the one given by equation~(\ref{eq:noattack}).

Note that there can be many different criteria for evaluating the success of an attack,  %(see Remark~\ref{remark:criteria})
but in our experiments, the controller is declared the winner if it can bring the flock to V-formation.
We consider three attack strategies (but see the future work discussion in Section~\ref{sec:conclusion}), each of which defines a V-formation game.

\vspace*{-0.5mm}\paragraph{\bf Remove Birds Game.}
In an RBG, the attacker selects a subset of $R$ birds, where $R\,{\ll}\,B$, and removes them from the flock.  The removal of bird $i$ from the flock at time $t\,{=}\,0$ can be simulated in our framework by allowing the attacker to set the displacement $\vd_i(0)$ for bird $i$ to $\infty$.  We assume that the flock is in a V-formation at time $t\,{=}\,0$.  
Thus, the goal of the controller is to bring the flock back into a V-formation consisting of $B\,{-}\,R$ birds.
%he controller needs to find the best adjustments in velocity $a_i$ for all remaining birds $i \in N - R$ during its turn. %$i \in N \wedge i \notin R$.
%Essentially, this results in a single-move game for the adversary. 
In an RBG, the attacker plays only one move.
When picking birds, the attacker is able to decide which birds will have the greatest negative impact on the flock's fitness when removed from the flock. Apart from seeing if the controller can bring the flock back to a V-formation, we also analyze the time it takes the controller to do so. 
%return to a v-formation for $R \leq \lceil\log(N)\rceil$ and 

% \todo[inline]{SAS: I would only suggest that the size R of the subset of
% birds removed from the flock (of size N) be such that R << N.  O/w I am
% not sure how interesting this game is.  Jesse has simulation results for
% R=1 and N=7.  Also, we should consider this game with and without process
% noise (PN), as Jesse has shown that the resiliency of the flock to remain
% in a V is highly dependent on the magnitude of PN.  It does very well with
% no PN or small PN, but resilience seems to degrade with increasing PN.}
%
%\begin{theorem}
%For any birds picked by the attacker, where $\left\vert{N - R}\right\vert \geq 3$, the planner can find 
%accelerations for each remaining bird in $N$ that will finally lead to a state $s^{*}$ such that cost 
%$J(s^{*})\{\leqslant}\,\varphi$.
%\end{theorem}

\vspace*{-0.5mm}\paragraph{\bf Random Displacement Game.}
In an RDG, the attacker chooses the displacement vector for a fixed number $R$ of birds uniformly from the space $[0,M]\times[0,2\pi]$. This means that the magnitude of the displacement vector is picked from the interval $[0,M]$, and the direction of the displacement vector is picked from the interval $[0,2\pi]$. We vary $M$ in our experiments. The $R$ birds that are picked in different steps are not necessarily the same, as the attacker makes this choice uniformly at random at runtime as well.
%In our second game, each player has control over all birds in the flock. The flock starts in a V-formation. However, both players have different goals and strategies. While the controller wants to keep the flock in a V-formation, the adversarial player tries to disrupt the V. Both players use the same planning approach but the controller tries to minimize the fitness function while the adversary tries to maximize the fitness in each step.
%In our second game, the adversarial player introduces malicious birds into the flock. These birds are controlled by the other player and hence can perturb the flock. To do so, the adversary adds small amounts of noise to this bird to distract the flock and disturb the v-formation. If this alone is not successful, the adversary can use a greater amount of noise to achieve the goal. However, this allows the controller to identify the adversary and henceforth ignore the malicious bird. 
The game starts from an initial V-formation. The attacker is allowed a fixed number of moves, say $20$, after which the displacement vector is identically $0$ for all birds.  The controller, which has been running in parallel with the attacker, is then tasked with moving the flock back to a V-formation, if necessary.
%
\vspace*{-0.5mm}\paragraph{\bf{AMPC Game.}}
An AMPC game is similar to an RDG except that the attacker does not use a uniform distribution to determine the displacement vector. The attacker is advanced and calculates the displacement (that will be the worst for the controller) using the AMPC procedure. See Figure~\ref{fig:ampc}.  In detail, the attacker applies AMPC, but assumes the controller applies zero acceleration. Thus, the attacker uses the following model of the flock dynamics:
\vspace*{-1mm}\begin{eqnarray}
 \xv_i(t + 1) &=& \xv_i(t) + \vv_i(t+1) + \vd_i(t) \qquad \forall~i\,{\in}\,\{1,\ldots,B\}, \nonumber \\
 \vv_i(t + 1) &=& \vv_i(t). \label{eq:attack} %\\[-6mm]
\end{eqnarray}
Note that the attacker is still allowed to have $\vd_i(t)$ be nonzero for a small number $R$ of birds. However, it can choose which $R$ birds it picks in each step.  It uses the AMPC procedure to simultaneously pick the $R$ birds and their displacements.
%Being a fair game, both players have the same capabilities. This means the controller as well as the adversary are able to use receding horizons to try to predict the best moves for their individual birds.

%\begin{theorem}
%
%\end{theorem}

%\paragraph{\bf Game 3.}%: Interior Lines}
% In our third game the adversary has only access to a specific subset of the birds. One could consider the attacker to add a set of malicious birds $M$ to the existing flock $N$.  Additionally we assume the controller is able to detect the attacker and hence the adversarial player needs to wait for the opportune moment to perform the actual attack. This means, the adversarial player can disrupt the V-formation slightly but only has one single move to interrupt and perturb the V-formation permanently. 
% \todo[inline] {Lukas: some important questions: the ATTACKER-ARES only controls the malicious birds and the CTL-ARES only the 'good' birds. however, does the CTL-ARES consider the malicious birds in its planning as 'good' birds? same for the ATTACKER-ARES. To me it would make sense, that the ATTACKER-ARES knows which ones are malicious birds and which ones are 'good' birds, but the CTL-ARES does not. So the CTL-ARES would consider ALL birds ($M \cup N$) but only controls the 'good' ones ($N$) -- i hope this makes any sense.}
%The third game is very similar to the second. However, when performing the final move, the attacker can decide whether it is more beneficial to introduce noise with a great magnitude to the flock or simply remove a specific number of birds from the flock. Again, we consider this a fair game where both players are able to use receding horizons do identify potential moves. Furthermore, we allow the adversary to remove up to $\log(N)$ birds from the flock.
%\subsection{Implementation: the Game is on}
%\label{sec:implementation}
%
%\todo[inline]{The following section would be the new implementation of our algorithm that deals with stochastic MDP and two-player games.}
%
% For this work, we extended the original \emph{deterministic Markov Decision Process} presented by Lukina et al.~\cite{lukina2016arxiv} to a \emph{classical MDP}~\cite{russellnorvig} by adding noise to the transition relation of the MDP. By doing so, we improved the original model and made it more realistic.
%
%We added and analyzed two different types of noises, processing noise ($\pnoise$) and actuator noise ($\anoise$). $\pnoise$ is applied to the position of each bird in our flock and changes the transition relation as follows
%\vspace*{-1mm}\begin{eqnarray*}
%\label{eq:pnoise_model}
% \xv_i(t + 1) &=& \xv_i(t) + \vv_i(t+1) + \pnoise %\label{eq:x_anoise},\\
% \vv_i(t + 1) &=& \vv_i(t) + \va_i(t) \label{eq:v_anoise},\\[-6mm]
%\end{eqnarray*}
%where $\pnoise \sim \mathcal{N}(0, \sigma^2)$. Here, $\sigma$ 

%In contrast, actuator noise is added to the acceleration action of the transition relation.
%\vspace*{-1mm}\begin{eqnarray*}
%\label{eq:model}
 %\xv_i(t + 1) &=& \xv_i(t) + \vv_i(t+1)\label{eq:x_anoise},\\
 %\vv_i(t + 1) &=& \vv_i(t) + \va_i(t) + \anoise\label{eq:v_anoise},\\[-6mm]
%\end{eqnarray*}

%\noindent where $\anoise \sim \mathcal{N}(0, \sigma^2)$. For our experiments we tried different $\sigma$, i.e. $\sigma = 0.05, 0.1, 0.2, 0.25$ and $0.3$.

%\begin{remark}\label{remark:criteria}
%Even though we use reaching V-formation as our success criterion (for the controller), we could have also used other criteria to decide if the attacker has been successful. For example, one could have used following criteria.
%
%\begin{itemize}
%\item \emph{Energy attack} is considered successful when a flock is not traveling in a V-formation for a certain amount of time. 
%
%\vspace*{1mm}\item \emph{Collisions} occur when two birds are in dangerous proximity from each other. This may happen through spoofing of existing birds or adversarial birds deliberately trying to lead to collisions with the others.
%
%\vspace*{1mm}\item \emph{Heading change} brings success, when the entire flock is diverged from its original direction (mission target) by a certain degree. 
%\end{itemize}
%\end{remark}

\begin{theorem}[AMPC resilience in a C-A game]
\label{thm:resilience}
Given a controller-attacker game, there is a finite maximum horizon $h_{\mathit{max}}$ and a finite maximum number of game-execution steps $m$ such that AMPC controller will win the controller-attacker game in $m$ steps with probability one.
\end{theorem}

\begin{proof}
Since the flock MDP (defined by Equation~6) is controllable, the PSO algorithm we use is fair, and the attack has a bounded duration, the proof of the theorem follows from Theorem~\ref{thm:ampc}. 
\end{proof}

\begin{remark}
While Theorem~\ref{thm:resilience} states that the controller is expected to win with probability one, we expect winning probability to be possibly lower than one in many cases because: (1)~the maximum horizon $h_{\mathit{max}}$ is fixed in advance, and so is (2) the maximum number of execution steps $m$; (3) the underlying PSO algorithm is also run with bounded number of particles and time.
\end{remark}

\begin{table}[t!]
\centering
\caption{Voice conversion \& F0 manipulation results. MOS results are reported with 95\% confidence interval. VDE, and FFE are reported for F0 manipulation while PER, WER, EER, and MOS are reported for voice conversion. Notice, for VDE, and FFE higher is the better since F0 was flattened.}
\label{tab:conv}

\resizebox{1\columnwidth}{!}{
\begin{tabular}{c@{~} | c@{~} | c@{~}c@{~} | c@{~} | c@{~} ||  c@{~}c@{~} }
\toprule
\multirow{2}{*}{Dataset} & \multirow{2}{*}{Method} & \multicolumn{4}{c||}{Voice Conversion} & \multicolumn{2}{c}{F0 Manipulation} \\
\cmidrule{3-8}
& & PER~$\downarrow$ & WER~$\downarrow$ & EER~$\downarrow$ & MOS~$\uparrow$ & VDE~$\uparrow$ & FFE~$\uparrow$ \\
\midrule
VCTK & GT  & 17.16 & 4.32 & 3.25 & 4.11$\pm$0.29 & -- & -- \\
\midrule 
\multirow{3}{*}{LJ}
% & ASR-TTS   & 50.74  & --     & 66.08 & 32.96 & 1.46 \\
& CPC       & 22.22 	& 16.11 		& 0.46 		& 3.57$\pm$0.15 		& \bf 46.68 & \bf 48.71\\
& HuBERT    & \bf 19.09 & \bf 12.23 & \bf 0.31  & \bf 3.71$\pm$0.24 & 39.20 		& 48.42\\
& VQ-VAE    & 40.88 	& 36.96 		& 9.65 		& 2.90$\pm$0.17 		& 10.54 	& 12.08 \\
\midrule 
\multirow{3}{*}{VCTK} 
% & ASR-TTS   & 68.88  & --    & 41.77 & 13.55 & 6.48 \\
& CPC       &  23.58 		& 15.98 		& \bf 4.83  &  3.42 $\pm$ 0.24 		& \bf 25.29 & \bf 26.97 \\
& HuBERT    &  \bf 20.85 	& \bf 12.72 & 6.01  		& \bf  3.58 $\pm$ 0.28 	& 23.46 	& 26.67 \\
& VQ-VAE    & 36.88  		& 29.44 		& 11.56 		& 3.08 $\pm$ 0.34 		& 7.03  	& 7.80  \\
\bottomrule
\end{tabular}}
\vspace{-0.4cm}
\end{table}

\vspace{-0.1cm}
\section{Results}
\vspace{-0.1cm}
Our results cover
% We report results for 
three different settings: (i) speech reconstruction experiments; (ii) speaker conversion and F0 manipulation; (iii) bitrate analysis with subjective tests for speech codec evaluation. We employ two datasets: LJ~\cite{ljspeech17} single speaker dataset and VCTK~\cite{vctk} multi-speaker dataset. All datasets were resampled to a 16kHz sample rate.

% \paragraph*{Implementation Details.}
% \smallskip
\noindent{\bf Implementation Details\quad} 
\label{sec:impl}
We follow the same setup as in~\cite{lakhotia2021generative}. For CPC, we used the model from~\cite{Riviere2020}, which was trained on a ``clean'' 6k hour sub-sample of the LibriLight dataset~\cite{Kahn2020,Riviere2020}. We extract a downsampled representation from an intermediate layer with a 256-dimensional embedding and a hop size of 160 audio samples. For HuBERT we used a \textsc{Base} 12 transformer-layer model trained for two iterations~\cite{hsu2020hubert} on 960 hours of LibriSpeech corpus~\cite{Panayotov2015}. 
% This model encodes every 320 raw audio samples into a 768-dimensional vector. 
This model downsamples the raw audio $\times320$ into a sequence of 768-dimensional vectors. Similarly to~\cite{lakhotia2021generative}, activations were extracted from the sixth layer.

%CPC: We use a dictionary of 100 units, leading to a bitrate of 700bps.
%HuBERT: A dictionary of 100 units is used, leading to a bitrate of 350bps. 
%VQVE: The VQ-VAE discrete code operates at a bitrate of 800bps.
% For both CPC and HuBERT, the k-means algorithm is applied to convert continuous frames to discrete codes, using the LibriSpeech clean-100h~\cite{Panayotov2015} dataset. 
For CPC and HuBERT, the k-means algorithm is trained on LibriSpeech clean-100h~\cite{Panayotov2015} dataset to convert continuous frames to discrete codes. We quantize both learned representations with $K=100$ centroids. Leading to a bitrate of 700bps for CPC and 350bps for HuBERT.

% VQ-VAE
Similarly to CPC models, we trained the VQ-VAE content encoder model on the ``clean'' 6K hours subset from the LibriLight dataset. We use an encoder operating on the raw signal to extract discrete units, similar to~\cite{jukebox}. In addition, ``random restarts'' were performed when the mean usage of a codebook vector fell below a predetermined threshold. Finally, we used HiFiGAN (architecture and objective) as the decoder instead of a simple convolutional decoder, as it improved the overall audio quality. This model encodes the raw audio into a sequence of discrete tokens from 256 possible tokens~\cite{garbacea2019low} with a hop size of 160 raw audio samples. The VQ-VAE discrete code operates at a bitrate of 800bps. We additionally experimented with 100 discrete units for VQ-VAE, however results were the best for 256. This finding is consistent with~\cite{garbacea2019low}.

% verification model
The speaker verification network uses the architecture proposed in~\cite{heigold2016end}. It was trained on the VoxCeleb2~\cite{voxceleb2} dataset, achieving a 7.4\% Equal Error Rate (EER) for speaker verification on the test split of the VoxCeleb1~\cite{Nagrani17} dataset.

% pitch
Only a single F0 representation is considered across all evaluated models, trained on the VCTK dataset.
% The F0 is extracted from the raw audio using YAAPT~\cite{yaapt} algorithm, using a window size of 20ms and a 5ms hop. 
The F0 is extracted from the raw audio using a window size of 20ms and a 5ms hop. 
As a result, the F0 sequence is sampled at 200Hz. 
% We apply the quantization described at Sec.~\ref{sec:method}, using a pitch codebook of $K'=20$ tokens and an encoder that downsamples the pitch by $\times16$. 
The quantization described at Sec.~\ref{sec:method}, is applied using an F0 codebook of $K'=20$ tokens and an encoder that downsamples the signal by $\times16$. Hence, the discrete F0 representation is sampled at 12.5Hz, leading to a bitrate of 65bps. The final bitrate of the evaluated codecs is the sum of the pitch code bitrate with the content code bitrate.

% \paragraph*{Evaluation Metrics}
% \smallskip
\noindent{\bf Evaluation Metrics\quad} 
We consider both subjective and objective evaluation metrics. For subjective tests, we report the Mean Opinion Scores (MOS). In which human evaluators rate the naturalness of audio samples on a scale of 1--5. Each experiment, included 50 randomly selected samples rated by 30 raters. For objective evaluation, we consider: (i) Equal Error Rate~(EER) as an automatic speaker verification metric obtained using a pre-trained speaker verification network. We report EER between test utterances and enrolled speakers; (ii) Voicing Decision Error (VDE)~\cite{nakatani2008method}, which measures the portion of frames with voicing decision error; (iii) F0 Frame Error (FFE)~\cite{chu2009reducing}, measures the percentage of frames that contain a deviation of more than 20\% in pitch value or have a voicing decision error; (iv) Word Error Rate (WER) and Phoneme Error Rate (PER), proxy metrics to the intelligibility of the generated audio. We used a pre-trained ASR network~\cite{baevski2020wav2vec} on both reconstructed and converted samples to calculate both metrics. %To generate target phonemes, the g2p-en~\cite{g2pE2019} Grapheme2Phoneme module was used.

% \vspace{-0.1cm}
% \smallskip
\noindent{\bf Reconstruction \& Conversion}
% \vspace{-0.1cm}
We start by reporting the reconstruction performance. Results are summarized in Table~\ref{tab:recon}. When considering the intelligibility of the reconstructed signal HuBERT reaches the lowest PER and WER scores across all models, where both CPC and HuBERT are superior to VQ-VAE. However, when considering F0 reconstruction VQ-VAE outperforms both HuBERT and CPC by a significant margin. This results are somewhat intuitive, bearing in mind VQ-VAE objective is to fully reconstruct the input signal. In terms of subjective evaluation, all models reach similar MOS scores, with one exception of CPC on LJ. 

%Notice, since the same F0 units are used for each method, this result implies the VQ-VAE units contain some information about the F0 of the signal, enabling better reconstruction. Regarding speaker information, the CPC gets the lowest EER. 

To better evaluate the disentanglement properties of each method with respect to speaker identity and F0, we conducted an additional set of experiments aiming at speaker conversion and F0 manipulation. For voice conversion, we converted each test utterance into five random target speakers. Next, we employed a speaker verification network, which extracts \emph{d-vector} representation to evaluate speaker-converted utterances' similarity to real speaker utterances (low error-rate indicates good conversion), providing measurement to the speaker identity's disentanglement from the evaluated coding method. The error-rate is reported between converted test utterances and enrolled speakers. For the LJ speech single speaker dataset, we converted samples from the VCTK dataset to the single speaker and enrolled all VCTK speakers together with the single speaker. Results are summarized in Table~\ref{tab:conv} (left). Unlike resynthesis results, on voice conversion CPC and HuBERT outperform VQ-VAE on both LJ and VCTK datasets, indicating VQ-VAE contains more information about the speaker in the encoded units, hence producing more artifacts. Notice, this also affects WER, PER, and the overall subjective quality (MOS). 

Next, to evaluate the presence of F0 in the discrete units, we flattened the F0 units before synthesizing the signal and calculated VDE and FFE with respect to the original F0 values. F0 flattening was done by setting the speakers' mean F0 value across all voiced frames. In this experiment, we expected units that contain F0 information to be better at F0 reconstruction over disentangled units. Results are summarized in Table~\ref{tab:conv} (right). Notice VQ-VAE can still reconstruct the F0 almost at the same level as when using the original F0 as conditioning (5.2 vs 7.03, and 5.59 vs 7.8), in contrast to CPC and HuBERT.

\begin{figure}[t!]
\centering
\includegraphics[width=0.65\columnwidth, trim={50 20 70 20}]{figures/codec_2.pdf}
% \caption{MUSHRA subjective listening test results as a function of bitrate per second for various methods. Purple dots denote the baseline methods, and green dots the proposed SSL based method.} 
\caption{MUSHRA subjective quality results as a function of bitrate per second. Purple dots denote the baseline methods, and green dots the proposed SSL based method.} 
\label{fig:codec}
\vspace{-0.5cm}
\end{figure}

% \vspace{-0.1cm}
% \smallskip
\noindent{\bf Speech Codec}
Our final experiment evaluates the obtained speech units as a low bitrate speech codec. 
% Therefore, we evaluate how the performance varies as a function of the number of discrete units. Changing the number of units is equivalent to varying the bitrate of the encoded signal. 
We use a subjective MUSHRA-type listening test~\cite{series2014method} to measure the perceived quality of the proposed speech codec with regard to its bitrate constraints. In MUSHRA evaluations, listeners are presented with a labeled uncompressed signal for reference, a set of test samples to rate, a copy of the uncompressed reference, and a low-quality anchor. Listeners are asked to rate each test utterance and the copy of the uncompressed reference with respect to the labeled reference in a scale of 1-100.

The experiment is performed on the VCTK dataset~\cite{vctk}. For evaluation, we used 20 utterances from 5 speakers. The set of speakers in the test data is disjoint with those in the training data. For this experiment, HuBERT models with 50, 100, and 200 units were trained as described in Sec.~\ref{sec:impl}. For comparison, we included other speech codecs in our evaluation: Opus~\cite{valin2012definition} wideband at 9 kbps VBR, Codec2~\cite{rowe2011codec} at 2.4 kbps and LPCNet~\cite{valin2019real} operating at 1.6 kbps. The LPCNet model was trained from scratch on the VCTK dataset following the experimental setup in~\cite{valin2019real}. The VQ-VAE model employs the HiFiGAN decoder trained on the LibriLight dataset to match the amount of data reported in~\cite{garbacea2019low}. We compressed the anchor sample with Speex~\cite{valin2016speex} at 4 kbps as a low anchor. Fig.~\ref{fig:codec} depicts the results. HuBERT with 50 units reaches the best MUSHRA score while its bitrate is only 365bps, which is significantly lower than the baseline methods.

\begin{comment}
\begin{figure}
\includegraphics[width=\linewidth]{figs/beyond_tss_lesion.pdf}
\caption[]{End-to-End runtime lesion study of the entire MNIST dataset and the FMA featurized music dataset. Each of DROP's contributions provides a runtime improvement.}
\label{fig:beyond_lesion}
\end{figure}
\end{comment}



\section{Conclusion}
\label{sec:conclusion}

Advanced data analytics techniques must scale to rising data volumes. 
DR techniques offer a powerful toolkit when processing these datasets, with PCA frequently outperforming popular techniques in exchange for high computational cost. 
In response, we propose DROP, a new dimensionality reduction optimizer. 
DROP combines progressive sampling, progress estimation, and online aggregation to identify high quality low dimensional bases via PCA without processing the entire dataset by balancing the runtime of downstream tasks and achieved dimensionality. 
Thus, DROP provides a first step in bridging the gap between quality and efficiency in end-to-end DR for downstream \red{analytics}. 

%We revisit canonical operators for time series dimensionality reduction and the measurement study of~\cite{keogh-study}, and show that PCA is more effective than popular alternatives in the data mining literature often by a margin of over $2\times$ on average on gold-standard time series benchmark data sets with respect to output data dimension. More surprisingly, we empirically demonstrate that a small number of samples are sufficient to accurately characterize directions of maximum variance and obtain a high-quality low-dimensional transformation.





\section*{Acknowledgments}
This research was supported by the Exascale Computing Project (17-SC-20-SC),
a joint project of the U.S. Department of Energy’s Office of Science and
National Nuclear Security Administration, responsible for delivering a
capable exascale ecosystem, including software, applications, and hardware
technology, to support the nation’s exascale computing imperative.

\small
\bibliographystyle{abbrv}
\bibliography{abbrv,paper.bib}


% Argonne Licence.
\vfill
\begin{flushright}
%
{\footnotesize
  \framebox{\parbox{0.5\textwidth}{
The submitted manuscript has been created by UChicago Argonne, LLC,
Operator of Argonne National Laboratory (``Argonne"). Argonne, a
U.S. Department of Energy Office of Science laboratory, is operated
under Contract No. DE-AC02-06CH11357. The U.S. Government retains for
itself, and others acting on its behalf, a paid-up nonexclusive,
irrevocable worldwide license in said article to reproduce, prepare
derivative works, distribute copies to the public, and perform
publicly and display publicly, by or on behalf of the Government.
The Department of
Energy will provide public access to these results of federally sponsored research in accordance
with the DOE Public Access Plan. http://energy.gov/downloads/doe-public-access-plan. }}
\normalsize
}
\end{flushright}

\end{document}
