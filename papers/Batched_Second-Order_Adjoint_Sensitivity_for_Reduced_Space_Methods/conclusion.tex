\section{Conclusion}
In this paper we have devised and implemented a practical batched algorithm (see \refalg{algo:batchreduction})
to extract on SIMD architectures the second-order sensitivities from a system
of nonlinear equations. Our implementation on NVIDIA
GPUs leverages the programming language Julia to generate portable kernels
and differentiated code. We have observed that on the largest cases the batch
code is 30x faster than a reference CPU implementation using
UMFPACK. This is important for upcoming large-scale computer systems where availability of general purpose CPUs is very limited.
We have illustrated the interest of the reduced Hessian when used inside
a real-time tracking algorithm.
%
Our solution adheres to the paradigm of differential and composable
programming, leveraging the built-in metaprogramming capabilities of Julia.
In the future, we will investigate extending the method
to other classes of problems (such as uncertainty
quantification, optimal control, trajectory optimization, or PDE-constrained optimization).


