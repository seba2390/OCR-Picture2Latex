% File LSUthesis.tex Created by Lucius Schoenbaum January 4, 2017



\documentclass[12pt,letterpaper]{lsuetd}

%
\usepackage{amsthm,amsmath,amssymb,amsxtra,latexsym,mathrsfs,url}
%\def\mypath{/Users/lucius/mytex/}
\def\pic{\includegraphics}
%\usepackage{\mypath ab,\mypath platon,\mypath weyl,\mypath gauss}%,\mypath npm}
\usepackage{ab}
\usepackage{thesis_coms}
\usepackage{verbatim} % \begin{comment} environment
\usepackage{cite} % bibtex 
%\usepackage{natbib} % another way to call bibtex


%\usepackage{titlesec}
%\titlespacing{\chapter}{0pt}{50pt}{200pt} 

%\usepackage{newclude}
%\usepackage{mathtools} % for \mathrlap
%\usepackage{ulem} % for strikethrough. Type \sout{blablabla}. never mind, this screws with \em. You can fix this by 
	% loading \usepackage[normalem]{ulem}, but instead I will just use {soul}.
%\usepackage{soul} % for strikethrough. Type \st{blablabla}. never mind, this gives error messages. 
%\usepackage[normalem]{ulem} % (see just above)

%\usepackage{\mypath npmxe} % for \samediagram

\usepackage{setspace} % for single spacing of quotations: add \onehalfspacing to the quote environment every time you use it
\usepackage[all]{nowidow} % eliminate widows 

%\clubpenalty=10000000 % orphans are penalized (10000 is high penalty)

\usepackage{graphicx} % for \includegraphics
\usepackage{tikz-cd} % for commutative diagrams using tikz
\usepackage{wrapfig} % for the wrapfigure environment, 
%QUICK GRAPHICS TEMPLATES:
%\begin{figure}[t] %  figure placement: here, top, bottom, or page   \centering   \pic[width=3in]{example.jpg}   \caption{Caption goes here.}   \label{f.example}\end{figure}
%\begin{wrapfigure}{L}{3cm}\pic[width=3cm, trim = 0cm 1.2cm 0cm 0cm, clip]{example.jpg}\end{wrapfigure} 

%\DeclareMathOperator{\lim}{lim}
%\newcommand{
%\newcommand{\sous}[1]{{}_{#1}} % argument in presubscript ("sousscript"). --> see platon.sty

%\raggedbottom % stop TeX from putting in weird vertical whitespace in order to make text cover the entirety of every full page (which isn't necessary). 
%\usepackage{mathdots} % for the \iddots command (inverse diagonal dots) 
%\usetikzlibrary{decorations.pathreplacing} % for "brace" decoration of paths (in tikz)

\usepackage{scalerel,stackengine}
\stackMath
\newcommand\verywidehat[1]{%
\savestack{\tmpbox}{\stretchto{%
  \scaleto{%
    \scalerel*[\widthof{\ensuremath{#1}}]{\kern-.6pt\bigwedge\kern-.6pt}%
    {\rule[-\textheight/2]{1ex}{\textheight}}%WIDTH-LIMITED BIG WEDGE
  }{\textheight}% 
}{0.5ex}}%
\stackon[1pt]{#1}{\tmpbox}%
}

\usepackage{scalerel,stackengine,amsxtra} % amsxtra contains \sptilde 
\stackMath % 
\newcommand\verywidetilde[1]{ % \verywidetilde
\savestack{\tmpbox}{\stretchto{ %
  \scaleto{ %
    \scalerel*[\widthof{\ensuremath{#1}}]{\kern-0.1pt \sptilde \kern-0.3pt} %
    {\rule[-\textheight/2]{1ex}{\textheight}} %WIDTH-LIMITED BIG WEDGE
  }{\textheight} % 
}{1.5ex}} %
\stackon[-2pt]{#1}{\tmpbox} %
}


%\DeclareMathOperator{\lim}{lim}
%\newcommand{
%\newcommand{\namegoeshere}{stringgoeshere} % name, string



\allowdisplaybreaks % split up multalign environments (especially in genm.tex) 




\usepackage{setspace,graphics,dsfont,verbatim,paralist,indentfirst}
\setlength{\topmargin}{-0.5in}
\setlength{\textheight}{9.0in}
\addtolength{\evensidemargin}{-0.50in}
\addtolength{\oddsidemargin}{-0.50in}
\addtolength{\textwidth}{1.00in}
\setlength{\parindent}{1.75em}
\setlength{\parskip}{0ex}
\setcounter{tocdepth}{2}

\makeatletter

\begin{document}
\renewcommand\@pnumwidth{1.55em}
\renewcommand\@tocrmarg{9.55em}
\renewcommand*\l@chapter{\@dottedtocline{0}{1.5em}{2.3em}}
\renewcommand*\l@figure{\@dottedtocline{1}{0em}{3.1em}}
\let\l@table\l@figure

\pagenumbering{roman}
\thispagestyle{empty}
\begin{center}
%The title page is first created.
%$\phantom{v}$
%\vfill
TOWARDS THEORY AND APPLICATIONS OF GENERALIZED~CATEGORIES TO AREAS OF TYPE~THEORY AND CATEGORICAL~LOGIC\\

\vfill
\doublespacing
A Dissertation \\
\singlespacing
Submitted to the Graduate Faculty of the \\
Louisiana State University and \\
Agricultural and Mechanical College \\
in partial fulfillment of the \\
requirements for the degree of \\
Doctor of Philosophy \\
\doublespacing
in \\
                                       
The Department of Mathematics \\
\singlespacing
\vfill

by \\
Lucius Traylor Schoenbaum \\
B.S., University of Georgia, 2005 \\
M.A., University of Georgia, 2008 \\
M.S., Louisiana State University, 2012 \\
%(Bachelor degree abbreviation), (Name of Previous University), (Year of Graduation with Bachelor's Degree)  \\
%If necessary, copy and paste the previous line here to include a master's degree.
%\vspace{2pt}
\vspace{1pt}
May 2017
\end{center}
\pagebreak
%The Copyright Page and Dedication sections can be added here, if desired.

%\chapter*{Copyright Page}
%\doublespacing
%\vspace{0.55ex}
%Insert the appropriate text for the copyright page here.
%\addcontentsline{toc}{chapter}{\hspace{-1.5em} {COPYRIGHT PAGE} \vspace{12pt}}
%\pagebreak

%\chapter*{Dedication}
%\doublespacing
%\vspace{0.55ex}
%Insert the appropriate text for the dedication or epigraph page here.  This part of the ETD must not exceed one page.
%\addcontentsline{toc}{chapter}{\hspace{-1.5em} {DEDICATION} \vspace{12pt}}
%\pagebreak


%The code below adds the Acknowledgments section to the Table of Contents.
\addcontentsline{toc}{chapter}{\hspace{-1.5em} {{\bf \large Acknowledgements}} \vspace{12pt}}
%\pagebreak
%The Preface section can now be added, if desired.

%\chapter*{Preface}
%\doublespacing
%\vspace{0.55ex}
%Insert the appropriate text for the preface here.
%\addcontentsline{toc}{chapter}{\hspace{-1.5em} {PREFACE} \vspace{12pt}}
%\pagebreak

\chapter*{Acknowledgments}
\doublespacing
\vspace{0.55ex}
\vspace{5ex}
This work clearly 
shows the scars of an existence on earth. %
I can only hope that for this reason, the work has something of the flavor of life, even though it is really only mathematics. %
%The story of this work is an interesting one, but this is not the place to tell it. %

I'm grateful to many people for helping me along the way. %during this journey. %
%
I would like to thank my Mom and Dad for their love and support. I love you Mom. I love you Dad.
%
I would like to thank Cecilia, Kelsey, Adrian, Mandy, and Melanie for their support while I was working on this, and experiencing all the highs and lows of life as a graduate student. I love you all. 
%

I would like to thank the mentors I have had over the years, especially Robert Varley, Robert Rumely, David Edwards, Ed Azoff, Akos Magyar, and Ted Shifrin at UGA mathematics for supporting and encouraging me and teaching me so much. I also thank everyone at UGA philosophy, especially my advisor for my Master's degree in philosophy, Brad Bassler, who first exposed me %my young eyes 
to the closely 
coupled vision of logic, philosophy, mathematics, and physics % all as one 
that has been the foundation of my adult life. %I extend thanks to everyone for their help, and for their continued support. 
I owe a debt of gratitude to Gestur Olafsson, Jimmie Lawson, and Daniel Sage for their mentorship and support during my time in Baton Rouge, and for often giving me the chance  %opportunity 
to pursue questions of my own. %I thank each of them from the bottom of my heart. % I wanted answers for. 
%It has been a great privilege to study advanced mathematics at an American university. %What a long, strange trip it's been---and 
%Thanks especially to my thesis advisor, Daniel Sage, who supervised %gave me a fruitful, supportive environment for %asking questions, finding answers, in short, doing 
%my research. %and helped get it into its final form, 
%along with ever-generous assistance and sage advice from 
%along with my coadvisor 
%Jimmie Lawson. 
Several other LSU professors helped me in ways big and small, and I am especially grateful to James Madden, Milen Yakimov, Charles Delzell, and Ricardo Estrada for their help and their knowledge. I'm also grateful to the LSU math department staff, LSU math's talented IT, and the custodians and campus employees who kept my office clean and safe. %spick and span. % so I didn't have to. % during my comings and goings. % at all hours. 
%And thanks to Ojo and Natalie for .
% of the day and night. %, for taking care of me and pretty much everyone else in this neck of the woods who is lucky enough to be blessed with time and space for doing mathematics. 

There are many others whose work is reflected in mine. % of many years. % in a city by the river. 
I am grateful to all the researchers who have inspired me and lifted me through their work. 
I would like to take the opportunity to thank all those who have been reviewers of my papers, especially the one based on my master's thesis, for their critiques  that led to many improvements. %
%Looking back, I am particularly struck by what a change of direction my research took after the Topos {\`a} l'IH{\'E}S conference. %
%I am grateful to all the participants for a memorable experience and for their hospitality. % a warm, collegial, and supportive atmosphere. %
I am grateful to % everyone and to 
the people of Louisiana %, of all colors and creeds, 
for hosting me, and who by way of a three year Board of Regents Fellowship made the early part of my studies possible. 
They (most of the time%, rarely on game days
) provided an environment conducive to study and reflection. 
For that, I would also thank all the places and spaces of South Louisiana I have haunted, if only I could. 




\singlespacing
\tableofcontents
\pagebreak

%The code below generates the List of Tables and adds it to the Table of Contents.
\renewcommand\@pnumwidth{1.55em}
\renewcommand\@tocrmarg{8.55em}
%\addcontentsline{toc}{chapter}{\hspace{-1.5em} LIST OF TABLES \vspace{12pt}}
%\listoftables
%\pagebreak
%The code below generates the List of Figures and adds it to the Table of Contents.
%\addcontentsline{toc}{chapter}{\hspace{-1.5em} LIST OF FIGURES \vspace{12pt}}
%\listoffigures
%\pagebreak
%The List of Nomenclature may be included here, if desired.

%\chapter*{List of Nomenclature}
%\doublespacing
%\vspace{0.55ex}
%Provide the definitions of the symbols used in your thesis or dissertation here.
%\addcontentsline{toc}{chapter}{\hspace{-1.5em} LIST OF NOMENCLATURE \vspace{12pt}}
%\pagebreak

%The code below adds the Abstract and places it within the Table of Contents.
\renewenvironment{abstract}{{\hspace{-2.2em} \huge \textbf{\abstractname}} \par}{\pagebreak}
\addcontentsline{toc}{chapter}{\hspace{-1.5em} {\bf \large Abstract}}
\begin{abstract}
\vspace{0.55ex}
\vspace{5ex}
\doublespacing
%Insert the text of your abstract here.  Make sure there is one blank line between the end of the Abstract text and the ``end'' command below to maintain double--spaced lines.

% File LSUabstract.tex Created by Lucius Schoenbaum Monday, January 16, 2017



%
Motivated by potential applications to theoretical computer science, in particular those areas where the Curry-Howard correspondence plays an important role, as well as by the ongoing search in pure mathematics for feasible approaches to higher category theory, we undertake a detailed study of a new mathematical abstraction, the generalized category. 
%
It is a partially defined monoid equipped with endomorphism maps defining sources and targets on arbitrary elements, possibly allowing a proximal behavior with respect to composition. 
% gencat
We first present a formal introduction to the theory of generalized categories. We describe functors, equivalences, natural transformations, adjoints, and limits in the generalized setting. 
% genm
Next we indicate how the theory of monads extends to generalized categories, and discuss applications to computer science. In particular we discuss implications for the functional programming paradigm, and discuss how to extend categorical semantics to the generalized setting. 
% th
Next, we present a variant of the calculus of deductive systems developed in \cite{LaK1c,LaK2}, 
and give a generalization of the Curry-Howard-Lambek theorem giving an equivalence between the category of typed lambda-calculi and the category of cartesian closed categories and exponential-preserving morphisms that leverages the theory of generalized categories. %We discuss potential applications and extensions. 
% it 
Next, we develop elementary topos theory in the generalized setting of ideal toposes, utilizing the formalism developed for the Curry-Howard-Lambek theorem. In particular, we prove that ideal toposes possess the same Heyting algebra structure and squares of adjoints that ordinary toposes do. 
Finally, we develop generalized sheaves, and show that such categories form ideal toposes. We extend Lawvere and Tierney's theorem relating $j$-sheaves and sheaves in the sense of Grothendieck to the generalized setting. 

\end{abstract}

\pagenumbering{arabic}
\addtocontents{toc}{\vspace{12pt} \hspace{-1.7em} {\bf \large Chapter} \vspace{-0.2em}}

\singlespacing

\setlength{\textfloatsep}{12pt plus 2pt minus 2pt}
\setlength{\intextsep}{6pt plus 2pt minus 2pt}
\chapter{Prelude: The 2-Category of Categories}\label{c.prelude}
\doublespacing
\vspace{10ex}
%\input{d0(prelude)}
% File diss_1(pre).tex Created by Lucius Schoenbaum September 4, 2016
% prelude

%\chapter{Prelude: The 2-Category of Categories}\label{c.prelude}

The following is a brief review of perhaps the most important elementary construction in category theory: the strict 2-category of categories.

Let $\sC, \sD$ be categories. Two natural transformations $\beta: G \natto H, \alpha: F \natto G$ between functors $F,G: \sC \to \sD$ may be composed via the rule
$$\beta \vertof \alpha (X) := \beta(X) \x \alpha(X)$$
where $(\x)$ denotes composition in $\sD$. This gives a category $\Nat(\sC, \sD)$. Identities in $\Nat(\sC, \sD)$ are given by $\id_F (X) := \id_X$. 

Given natural transformations $\alpha: F \natto G$ between functors $\sC \to \sD$, and $\beta: F' \natto G'$ between functors $\sD \to \sE$, we obtain a well-defined function $\Ob(\sC) \to \Mor(\sE)$ via
$$\beta \star \alpha (X) := \alpha ( \hat \beta (X) ) \x \bar \alpha ( \beta (X)),$$
where hats and bars are used as defined in section \ref{s.gencat} below. This can also be written
$$\beta \star \alpha = (\alpha \of \hat\beta) \vertof (\bar\alpha \of \beta)$$
Note that 
$$\bar \alpha (X) = \overline{\alpha(X)},$$
$$\hat \alpha (X) = \widehat{\alpha(X)}.$$

\prop[The Five Facts]
In the notation above, whenever expressions on both sides of the formula are defined, we have:
\enu
	% Fact 1:
	\item $\beta \star \alpha = (\hat\alpha \of \beta) \vertof (\alpha \of \bar\beta).$ %$\beta \star \alpha (X) = \alpha (\hat \beta (X)) \x \bar \alpha( \beta (X)).$
	% Fact 2:
	\item $\beta \star \alpha$ is a natural transformation $G \of F \natto G' \of F'.$ 
	% Fact 3:
	\item $(\gamma \star \beta) \star \alpha = \gamma \star (\beta \star \alpha).$
	% Fact 4:
	\item If
		$$\begin{rightbrace}
			\alpha: F \natto G \\
			\beta: G \natto H 
		\end{rightbrace} : \sC \to \sD, $$
		$$\begin{rightbrace}
			\alpha' : F' \natto G' \\
			\beta' : G' \natto H' 
		\end{rightbrace} : \sD \to \sE, $$
		then
		$$(\beta' \vertof \alpha') \star (\beta \vertof \alpha) = (\beta' \star \beta) \vertof (\alpha' \star \alpha).$$
	% Fact 5:
	\item If $\id_F^{\vertof}$ is the identity of $F$ with respect to the product $\vertof$ in $\Nat(\sC, \sD)$, then 
		$$\alpha \star \id^{\vertof}_F = \alpha,$$
		$$\id^{\vertof}_F \star \beta = \beta,$$
		whenever both sides are defined.
\Enu
\Prop
\prf
(1) 
\begin{align*}
	(\beta \star \alpha) (X)		&= (\alpha \of \hat \beta) \vertof (\bar \alpha \of \beta) (X) \\
						&= \alpha(\hat \beta(X)) \x \bar \alpha (\beta(X)) \\
						&= \alpha(\widehat{\beta(X)}) \x F(\beta(X)) \\
						&= G(\beta(X)) \x \alpha(\overline{\beta(X)}) \\
						&= \hat \alpha (\beta(X)) \x \alpha(\bar \beta (X)) \\
						&= (\hat \alpha \of \beta) \vertof (\alpha \of \bar \beta) (X).
\end{align*}

(2) by Fact 1.

(3) Apply the definition.

(4) by Fact 1 and since $\hat \alpha = \bar \beta, \widehat{\alpha'} = \overline{\beta'}$.

(5) direct calculation.
\Prf

{\em Remarks.} 
\enu
	\item We may write simply $\id_F$ or $1_F$ in light of Fact (5).
	\item Fact 4 is often referred to as the {\em interchange law}.
\Enu

An immediate consequence of the Five Facts is the following: {\em The category of categories is a strict two-category.} %
By ``the category of categories'' is meant the set of small categories, functors, and natural transformations in a fixed universe $\sU_{univ}$. 

$$\scriptscriptstyle{\blacksquare \quad \blacksquare \quad \blacksquare}$$
\vspace{-5pt}

%

%

%

\pagebreak

\singlespacing
\chapter{Introduction}
\doublespacing
\vspace{10ex}
%\input{d1(i)}
% File diss9(i).tex Created by Lucius Schoenbaum February 16, 2016
% introduction to dissertation


%\chapter{Introduction}\label{c.i}

%``IT SHOULD BE POSSIBLE TO READ ONLY THE INTRODUCTION---THE REST OF THE PAPER SHOULD CONTAIN ONLY THE DETAILS''
% TEST: Does the introduction tell me everything I need to know? If I read only the introduction, do I understand everything important? (Everything conceptually important, as well as all the major results and major steps/difficulties in the proofs?)

\section{Overview and Motivation}\label{s.overviewandmotivation}

Category theory \cite{MacCW,KaSc1,BaWe1} has its origins in mathematics, and has since become a well-established area of foundations, with a rich interaction with computer science. It begins with the insight that diagrams and morphisms have a mathematical life unto themselves, independent of function theory, and independent of any use of points as arguments. %
%The fact that this is indeed the basic insight of category theory, as well as, in many instances, the arguments that one makes to establish basic results of category theory, suggests an approach to category theory that regards a category as a set equipped with algebraic structure. 
The more one works with categories, in fact, %
the more one becomes cognizant of the view that a category is in fact a set of morphisms that are in some way algebraically structured. %
%
%For various reasons, this isn't how authors who write about categories tend to proceed, nor what mathematicians and computer scientists who use categories tend to treat category theory. A major reason is doubtlessly the fact that many constructions rely on a two step procedure of defining objects, then defining morphisms between them---though this dependence is somewhat illusory, as we shall see. We won't speculate further on what gave rise to this general consensus, but we will make note of one case in which the authors themselves give an explicit reason. 
In  \cite{EiMa1}, the paper on natural transformations in which the elementary notions of category theory are introduced for the first time, Eilenberg and MacLane write 
%\singlespacing
\begin{quote}
\onehalfspacing 
It is thus clear that the objects play a secondary role, and could be entirely omitted from the definition of a category. However, the manipulation of the applications would be slightly less convenient were this done. 
%\doublespacing
\end{quote}
Thus two views have been known to category theorists since the beginning of the subject. % to be somewhat at odds with one another. % fix this later
The two approaches, the one-sorted definition describing a universe of pure maps, and the two-sorted definition including the objects that are in applications prior to the maps that they inspire, pull against one another in a way that seems, in practice, like a natural, irresolvable tension. %
%by their existence in applications, %is well known to category theory specialists. 
The latter approach has proven to be the dominant one, while the former approach has made occasional appearances, for example in work by Ehresmann \cite{Ehresmann1}, Street \cite{StT1}, and %. %as noted in \cite{MacCW}, it was used in lectures by Freyd. 
%It appears in a famous paper by Street \cite{StreetOriented}. 
more recently, in work by R. % his initials
Cockett \cite{CockettConstellations}. %
% Nevertheless, it has existed only in somewhat esoteric literature, and it is---perhaps, or perhaps not---for this reason that the generalization we study here was slow in coming. For while the object-arrow dichotomy present in the two-sorted formulation obscures the possibility of leaping into the general setting, in the one-sorted formulation, the step that can be taken immediately presents itself. 

%The generalization studied here
%Categories are developed in many standard treatments (e.g., \cite{MacCW,KaSc1,BaWe1}) using a two-sorted language. A one-sorted language is less commonly used, but it has appeared as well, for example, in \cite{StreetOriented}. %
The potential for generalization begins with the less-often-used one-sorted formulation, which necessitates an axiom requiring the source and target maps $s$ and $t$ to be trivial upon iteration: $ss = st = s, tt = ts = t$. This condition, however, is extraneous. Dropping it gives rise to a rather general notion, %This is the starting point for the notion of a generalized category, which is 
which may be %
weakened further via replacing some equalities with inequalities, as suggested by some kinds of applications \cite{SmPl1}. %, which is motivated by applications to computer science we will come to later. %
This is the jumping-off-point of our work in Chapter \ref{c.gencat}. 

There exists no literature on this generalization of category theory. %which is possible only via the less-dominant one-sorted approach. %, has never been studied in the literature prior to this work. %
This lack of attention to the abstraction is likely due in some part to a lack of knowledge about the robustness of the categorical theory existing in the general case. %
%As the project took flight, 
Having noticed signs that such a theory might be sufficiently strong to have potential applications, the author undertook the investigations that appear in this dissertation. %
% applications of such a generalized notion of category might exist in the realm of computer science, specifically in the area of type theory, an active subdomain of the theory of programming languages. %
With much literature in computer science devoted to subjects such as metaprogramming, dependent types, and other generalizations of type theory based on Church's typed lambda calculus (see for example \cite{Barendregt1}), there exists no lack of potential applications of this work to categorical semantics and other areas of theoretical computer science. %
However, the author, not having been trained in computer science himself, does not venture far outside of the mathematical parts of category theory in this work. %
%We have established 
We find in this theoretical investigation that in the generalized setting a considerable part of the fundamental theory of categories and categorical logic persists in a remarkably robust form. During our investigation a wealth of unforeseen new abstractions have arisen, and there is much at the time of this writing that is still unexplored, both in the realm of theory and in the realm of applications. 
%and opened new areas that suggest that promise still exists for the discovery of productive applications of generalized categorical tools to mathematics and computer science. %

\begin{comment}
%\section{Generalization: Motivation}
outline:
\enu
	\item A generalized category {\em isn't} actually a category
	\item This is important to understand in order to understand how this relates to higher category theory
	\item higher categories become generalized categories that are flat (dimension is one). 
	\item Motivation No. 1 --- you don't {\em lose} anything from CaT when you go to gCaT. 
	\item Motivation No. 1' --- interesting things happen when you go to gCaT. You end up in doing a fun kind of math. (That {\em is} important!!!)
	\item Motivation No. 2 --- study higher categories via the flattening map, and obtain in this way a simplification or supporting structure for the notoriously onerous task of doing calculations in higher categories. Analogy: like working with cheap, clunky tools versus working with a really nice set of tools. It doesn't let you do anything you couldn't already do............but it does make a difference. (Ask anyone who has ever done any kind of work whether providing a better set of tools is not ``something''.)
	\item Motivation No. 3 --- extend Curry-Howard and get to proof theory.
	\item Motivation No. 4 --- extend Curry-Howard and get to computer science applications (graduated types). 
\Enu








% March 2016: why do this? 
		% what i like about this naturality/nonnaturality distinction is that it sheds light on a concept in category theory that i think is not well understood, and not easy to understand. 
		% but the real reason is that adjoint functors are crucial in the theory of topoi. Hence, we must have a clear, unmistakable sense of what the appropriate notion should be in the setting of these ``generalized topoi''. 
		
		% with nonnatural adjoint pairs, you get goodies. Just not all of them. 
				% they commute with limits as you expect. 
				% ()()() [task right now is to fill in here.]

% question, 4-8-16: we have yoneda only with an ugly hypothesis. WHAT KNOWN CASES IS THIS SATISFIED? (simplicial sets? globular sets? etc.) 


Generalized categories share something in common with the ``yoga'' of topos theory. In the introduction to \cite{sga4} an analogy is given: topological spaces are something like a group given along with a set of generators and relations, while topos theory is, by analogy, something like the abstract theory of groups, in which a set of generators is optional. If we think of the objects (\ref{d.morphism}) of a generalized category as being like ``points'' or zero-dimensional elements, a general generalized category need not have points, or need only be grounded with points in an incomplete way. This is analogous to the situation in topos theory. 
\end{comment}


% 5-2-16
%-mention enriched and the opposing way of generality given by the study of generalized categories. 

% 4-20?-16
%The focus of the following article is not on any of the many widespread applications of category theory, but on the fundamental framework of category theory and its close relative, topos theory. 

%Category theory, first introduced in \cite{EiMa1}, has seen many efforts at generalization of its set of tools. Using impressive ingenuity and creative tools of almost every shape and size, researchers from around the world have invented calculi mirroring or extending that of ordinary category theory in order to fit the needs of applications across a range of disciplines. By and large, these have all fallen somewhere beneath the broad umbrellas of topology \cite{}, algebraic geometry \cite{}, and physics \cite{}. % be sure to mention Kelly, Bousfield/Kan, Baez, Grothendieck, ...

%An approach that has long been influential is the enriched framework, see for example \cite{KellyEnriched1}. An enriched category is not in fact a category at all, but a structure equipped with an abstract hom map (on pairs) into a monoidal category. The multiplication (tensor product)  abstracts the composition operation in a normal category. % 
%In enriched category theory one must settle for generalized elements in place of morphisms, since hom ``sets'' are not necessarily sets in general. Thus certain arguments of category theory cannot be carried all the way to completion. As noted by \cite{ShulmanEnrichedHomotopyLimits} this includes Quillen's small object argument. 

%Higher category theory, as a coherent if still young subject, may be roughly viewed as having arisen out of the search for a satisfactory notion of weak $n$-category. The challenge has been to develop ``models'' of the weak case which best facilitate computations. The interest in these structures is driven to a large extent by coherence laws. These relations can sometimes offer insight into physical models. 

%The approach taken here, which we cannot possibly recommend except as a foundation for a more perfect approach along similar lines, is in a sense the ``opposite'' of the enriched approach: instead of general hom maps, we take general domain and codomain maps as our starting point. %As we will see,  stands in relation to higher category theory as the skeletal beginnings of a complete generalization (the missing ingredient being the weakening of the general associativity and identity laws). Our foray into topos theory could be placed alongside of the relatively recent development of $\infty$-topos theory, creating a sort of as-yet-incomplete diamond which suggests, perhaps, a common generalization lies unseen somewhere in the offing. 

\section{Summary of Contents}\label{s.outline}

\indent
In Chapter \ref{c.gencat} we introduce the main abstraction of our work, the notion of generalized category. We show that the standard tools of category theory carry over to the generalized setting, and we also present some negative results that may perhaps indicate to the interested reader some of the obstacles to further %illustrate the challenge that would be faced in attempting to carry 
generalization along the same lines. % that we have encountered. 
%any further than we have carried it here. 

In Chapter \ref{c.genm} we give a treatment of the theory of monads. We show that the theory of the Kleisli triple carries over to the generalized setting, however only by a more intricate construction than in the ordinary categorical setting. We also investigate the theory of algebras and the Tripleability theorems. Although the theory of $T$-algebras for monads seems to fit comfortably in the setting of generalized categories, the results of an investigation into a theory of generalized Eilenberg-Moore category were disappointing. However, the Tripleability theorems and other aspects of the theory of $T$-algebras can nevertheless be carried out via the usual one-categorical construction. 

In Chapter \ref{c.th} we treat cartesian closed categories and the work of Lambek \cite{LaK1c,LaK2} on the Curry-Howard Correspondence. The main result shown is that the correspondence carries over to the generalized setting. This involves developing a new syntactic abstraction, a generalized typed lambda calculus, to coincide with the generalized semantics of Chapter \ref{c.gencat}. This result lays the groundwork for applications of generalized categories in type theory, particularly for models of subtyping and higher-kinded type systems. As we also discuss, the generalized result also proves instructive when interpreting the usual Curry-Howard Correspondence in the one-categorical setting. 

In Chapter \ref{c.it} we use the results of Chapter \ref{c.th} develop the theory of elementary toposes in the setting of generalized categories, following \cite{sga4,MaMo1,JoN1,FrD1,LaSc1}. The abstraction studied is referred to as an ideal topos, as it extends the notion of ideal cartesian closed category introduced in Chapter \ref{c.th}. 
%We introduce generalized type theories, developing them as an extension of the generalized typed lambda calculus of Chapter \ref{c.th}. 
The theory of toposes has an important (perhaps decisively so) relationship with the theory of sheaves. We have therefore undertaken an investigation into a generalization of sheaves and show that they, too, possess a robust theory, one which is in certain respects different than both the theory of ideal elementary toposes and the theory of one-categorical sheaves that it extends. 
The flavor of the subject is a departure from previous chapters due to the obstacles that arise to setting up the full toolkit of sheaf theory in the generalized setting. This impasse is overcome using bipartite (generalized) categories, or what we call pointed profunctors. 
%although at the time of this writing the generalization of the slice theorem is still an important objective. 
As a byproduct of our work, we obtain a working definition of the category of generalized sets. %
While we discuss many fundamental topics in topos theory (basic topos theory, sheaf theory, and a few remarks on 2-sheaves) we do not include a discussion of the internal language of generalized categories, as this would burden the reader with still more lengthy preliminaries along the lines of the treatment in Chapter \ref{c.th}. 

The work done in Chapter \ref{c.it} shows that the topos theoretic foundations for logic, topology, and geometry laid down by some of the greatest minds of the past one hundred years has a wide new realm of applicability. 
Our work in this Chapter is lacking, however, for it states theory only, and gives no applications. 
We conclude this introduction by voicing the hope that such applications might someday be forthcoming. %

\begin{comment}
In Chapter \ref{c.pt}, we conclude by introducing a notion of topos system and using it to extend a result on topological spaces to topos theory in the generalized setting. 
We hope this short chapter will be an antidote to the abstraction of the previous chapters, 
as well as allow the beautiful geometric side of topos theory to make a brief entrance in this work. 
The main result, a classification of the points of a topological topos system with respect to a node, is also meaningful in 
the setting of topological spaces, 
where it is a classification of points with respect to their convergence relative to an arbitrary subset. 
Even in this classical setting, the result is meaningful, for it illuminates the nature of $T_0$ spaces 
by isolating a small set of features (namely, rare point types) that set them apart from 
spaces of separation $T_1$ or higher. 
\end{comment}


%______________________________

%

%

%

%\section{Preliminaries}\label{s.prelims}

%STUFF THAT GOES SOMEWHERE APPROPRIATE:

%Although according to this terminology any element of a generalized category can be called a morphism, we will tend to use the term ``element'' to refer to a generically chosen, well, element of a generalized category. The term ``morphism'' will be used in the way it is normally used in category theory. 

%(after discussing the concept of an ``evaluation generalized category'' or ``urelement generalized category'') The fact that the notion of generalized category absorbs the evaluation operation in an elementary way---not the way via cartesian closed structure which is quite unruly since it cartesian closed structure is rare in general and difficult to construct---is encouraging from the point of view of applications of category theory to computer science. Indeed, many computer scientists who use categories assume an evaluation operation is present, even though one is not specifically given by the abstract notion of a category. 
% my point is that Wadler, Moggi, ?????(I don't know anyone else specifically) come out championing all this fancy monad stuff and then, when they actually write down what they are doing, they are treating their morphisms as though they are functions that can accept arguments! Which is unbelievable from a category theory point of view, but understandable given the examples they are interested in---it's understandable, yes, but also...rather um, not quite using the definition correctly I have to say. But I wouldn't complain about that --- my point is just to say LOOK AT THIS! If we go to generalized categories, we don't have to fib any longer! We **have** our evaluation operation. My point is...I may be getting ahead of myself but...don't you think that the generalized category concept really fits computer science like a glove? It's almost as though category theory is this block with a shape, and computer science is this, you know, hold that needs a block, and the hold is **almost** the right shape for category theory...and so people have been shoving it in for decades now, just cramming it in and hammering it down when it won't go all the way down, you know what I mean? But if we merely change the definition ever so slightly, we have this notion of a generalized category, and....it just slips in like a hand in a velvet glove. 

%We assume in this writing that all sets are small unless explicitly specified. 






























%The notion of an $n$-category with well-defined (strict) composition and identity operations, defined on the nose, is generally viewed as a well-understood case. An observation that to the best of our knowledge has not been made before, however, is that ordinary category theory carries over not only to the strict $n$-categorical setting, but in fact even more broadly, to a theory of what can be called ``non-well-founded'' categories. This involves a certain conceptual shift: it amounts to a unification of the two normally distinct concepts of object and morphism into what we call below a {\em subject} (Definition \ref{d.morphism}). A detailed development of this general setting is here for the reader who wishes to see it for him or herself, in chapter \ref{c.gencat}. By passing to the general setting, which we study here under the heading {\em generalized category} (Definition \ref{d.gencat}), we are able to refine our understanding of some the basic concepts of category theory: functor, natural transformation, and adjoint pair in ways that are explained in sections \ref{s.gencat}, \ref{ss.naturality}. % should still work in those Kan extensions...
%The theory of limits also extends to generalized categories, in a way that permits new types of diagrams (at essentially no extra cost), see section \ref{ss.limits}. %though we know of no application for the new limits that are possible.

%In section \ref{s.sht}, which chapter \ref{c.gencat} prepares the way for, we show that the theory of sheaves carries over to the setting of generalized categories. %
%This involves more than replacing words in definitions: the generalized category of sheaves over a generalized category has a structure that is not precisely parallel to the manner in which the category of sheaves are normally defined in terms of objects and maps. %
%By showing that sheaf generalized categories have the desired properties we expect, we develop, as a side effect so to speak, a machine for generating generalized categories with strong exactness properties. %
%In section \ref{ss.internal} we develop internal generalized categories, and give precise statements for a few basic results about them. %
%In section \ref{s.ncat} we discuss the project of extending our work to higher category theory. 
% old and bad:
% the weak setting and attempt to place our work in context of the higher category theory literature. %

%In section \ref{s.tt}, we turn to the topological side of topos theory, the point of view championed by Grothendieck. Our work relies on an abstraction derived from the categorical data of a topos, which we call a {\em topos system}. This abstraction is rather well described, though not explicitly axiomatized, in the original work on topos theory \cite{sga4}, but it has received little attention in the form we have it here. Most work along the lines of topology without points has instead been devoted to advancing the study of locales \cite{???????}. %This is so much the case, in fact, that the author was quite surprised to find topos systems after finally finding the courage to read the first half of \cite{sga4}. Perhaps they have not had their due. In any case, topos systems are a convenient setting for our result on points, which we believe sheds some light on the topological theory within topos theory, so we are glad to make use of them. 
%We are proud to offer a modest result in what is surely (to us, at least) a difficult area of mathematics: a classification of the points of a general topos, relative to a subtopos. We provide a graphical way of illustrating the principles contained in the statement. 

%Note: the notion of monoidal category, bicategory, etc. (just take a monoidal category). You have Isbell's argument that shows that you cannot just pass to the skeleton and thus avoid the messy calculations. 

%This argument could also be important for generalized category theory. Think about Mr. Doubful, who says, ``a generalized category has an associated category, the isoskeleton, and dom/cod are not canonical, i.e. they clearly do not lift to the isoskeleton. Therefore the generalized category structure is an illusion and irrelevant.'' You can argue back to Mr. Doubtful using Isbell's argument: not so fast, Mr. Doubtful, you cannot even make that argument for {\em categories}.
























%

%

%

%

%

% old shit:::

\begin{comment}
Outline:

% background

1. First there was analysis.
% In the beginning, there was analysis. ()()(). During the nineteenth century, ()()(), the ideas of analysis were abstracted. ()()() This gave rise to the subject known as functional analysis, and ()()(), setting the stage for the modern mathematical fields based on set theory and mathematical logic, and soon, category theory. 

2. Then there was algebraic analysis. Kashiwara and D-modules and this had lots of good consequences. Interesting results. Activity.

3. Meanwhile, in analysis, the subject of multilinear analysis and nonlinear functional analysis began to grow ()()()()(). Control theory. Inclusions.

4. So there are these two trends. 

5. The basic ideas of algebraic analysis grew out of the ideas of Grothendieck and his school, introduced in \cite{sga4} (more precisely, the second edition ()()()()). If the central motivation for this work was the Weil conjectures on the cohomology of ()()(), the central idea of this work, according to Grothendieck, was the concept of a topos. [talk about topos theory and what it is supposed to be. Discuss the subsequent evolution of the subject of topos theory.]

6. Topos theory is category theory plus order theory % I want to emphasize this...
\end{comment}



% my work

%1. The first part of this thesis is a preliminary investigation into how these trends might be part of a broader, conjectural area: a generalization of algebraic analysis to the study of multi-valued sheaves. As is very natural for the subject matter at hand, will approach the subject in the style of the Grothendieck school, working on a rigorous basis in a general language. As we will see, there is motivation ``from within'' to carry out this investigation: namely, much of the basic machinery of presheaves and sheaves
%carries over to a multilinear setting that we will specify. 

%2. The second part of this thesis is an investigation into the structure of Grothendieck's topos theory. We will attempt to elucidate the topological intuition that is at the heart of topos theory by classifying the points of a Grothendieck topos with respect to convergence properties relative to a fixed subtopos. 

%3. In an appendix, we present a simplified axiomatization of the subtopos lattice of a topos, which will, we hope, further elucidate the structure and motivate the continued pursuit of Grothendieck's fifty-year-old vision of a new general framework for topology. 




%\subsection{Summary}\label{ss.sum} 

% summary goes here















% stuff from primex days...
\begin{comment}

For better or for worse, this thesis exists thanks to a certain vision of the foundations of mathematics. We will now explain that vision. It is the purpose of the work that follows to simply flesh out the picture we describe here. 

We take the following as basic principles:
\enu
	\item the fundamental data-type in mathematics is the set. 
	\item sets are not enough: mathematics wants to grow beyond the boundaries of set theory. \label{i.setsnotenough}
\Enu
Experience over the decades since set theory was invented have, we think, established \ref{i.setsnotenough}. ()()()()() [explain this; survey the literature; ring theory, duality, multisets, fuzzy sets, probability, domains, toposes...]

In this work, we would like to survey these set-like data-types, focusing attention on a particular one that appears in this work for the first time, the primex. 

Category theory is very important for our point of view.
We use a flavor of category theory that is purely external (that is, built on top of set theory) as opposed to internal, which is category theory as a language built on top of some choice of deductive system (say, first order logic). 

()()()(). % 
In other words, our main goal is to argue for a view of foundations that looks like this, which could be associated with the names Grothendieck, Lawvere, Tierney:
$$
pic % OrT + CaT -->--> ToT
$$ 
to one that looks like this:
$$
pic2 % SeT --> OrT + CaT -->--> ToT + PxT
$$
We wish to make the case (this thesis cannot be said to have fulfilled this wish, but it is a start) that this is a fruitful way to view foundations: topos theory informs primex theory, and vice versa, and virtually all the structure and substance of mathematics fits nicely into the picture presented by figure 2. Thus, we finally arrive at figure 3:
$$
pic3 % SeT --> OrT + CaT -->-->-->--> ToT + PxT --> MaS
$$
With this picture, we close the book on foundations (with the exception of logic, of course, which we have purposefully omitted from this thesis in order to streamline the ideas.) The point of our point of view is the following: if you were to start from absolute beginnings in mathematics, and you were absolutely certain that you wanted to do everything, and do it ``the right way'', and had all the resources to pursue that aim---then you would start by progressing step by step through these phases (set theory, to order theory and category theory, to topos theory and primex theory) and then, finally, you would start to build mathematics, using tools freely whenever they suggested themselves. 

With that, we have the entire idea, so there is nothing left but to begin, after a brief summary of the contents. 

\end{comment}












% ___________________________
\begin{comment}
%The basic notions of category theory generalize to a much larger class of structures, {\em generalized categories}, which can be thought of as non-well-founded categories. We develop the basic theory of generalized categories (limits, adjoints, natural equivalences), discuss examples of generalized categories which are not categories, construct generalized categories from categories (in particular,  the generalized category kSet from the category Set), discuss the relationship to other concepts in higher category theory, and begin down the road leading to sheaf theory. In the next version of the draft, we will discuss homological algebra and descent theory.

% weighted limits in generalized categories? Riehl, Joyal


As we have many things to recall for the reader, we take the opportunity to make what may be considered a small contribution to the literature. Namely, the observation that significant amounts of category theory generalizes to the setting of {\em generalized categories}, structures which offer a window into higher category theory which, as far as we know, is newly introduced here.\footnote{As noted below, the authors \cite{EiMa1} note that ()()().} We will pause for examples and breaks to discuss intuition during the formal development that follows, but the quickly scanning reader should note the essential idea of a generalized category: a generalized category may be thought of as a non-well-founded category. This explanation is probably help even if the reader has no familiarity at all with the theory of non-well-founded sets \cite{Barwise1,Kunen1,???}.

We apologize up front for the dearth of examples (there are a few). We hope attentive and adroit readers will be piqued to dream of what the observation that category theory generalizes to generalized categories may perhaps unlock. Our basic point: having read this, the reader who happens to encounter a generalized category will be ready. 

% ok, enough, let's get started. 

In this section we will develop the basic category theory that we need. We include this section for the sake of completion, and in order to show that category theory can be developed in the more general setting of {\em generalized categories}.\footnote{The reader might wish to call generalized categories {\em generalized categories}, a word currently used in biology, but we have opted to follow Johnstone in using {\em toposes} in place of {\em topoi} (which is itself a word currently used---in literature!) so we do the same with respect to our generalized categories.} %
Generalized categories are preferable to categories at the level of abstract nonsense. In concrete situations a generalized category notion can at times get in the way (we leave it to the reader to discover this), but generalized categories have the advantage of great generality (greater than category theory) in spite of striking simplicity that makes abstract nonsense flow along nicely. 

The observation been made already long ago in \cite[p. 238]{EiMa1} %in \cite[p. 238]{EM1945}, immediately following the first axiomatic definition of a category, the authors point out that 
\footnote{``the objects play a secondary role, and could be entirely omitted from the definition [...] However, the manipulation of the applications would be slightly less convenient were this done.''}
that objects are subordinate to arrows in category theory, but at the time of this writing we are not aware of any development of generalized categories in the literature. 

Generalized categories can be thought of as non-wellfounded categories, or the analog for categories to non-wellfounded sets \cite{???}. 

%This class of structures is singled out for study because it is highly general, and yet it supports a good theory: of natural equivalence, of adjoints and limits, of sheaves and representability. Our main goal in this note is to indicate/verify that this is indeed the case. 

\end{comment}

\pagebreak

\singlespacing
\chapter{Generalized Categories}\label{c.gencat}
\doublespacing
\vspace{10ex}
			%\input{gencat1(i)} % introduction
%\input{gencat2(gencat)} % definition
% file gencat2(gencat).tex Created by Lucius Schoenbaum, November 20, 2016





\section{Beginnings}\label{s.gencat}

%\subsection{First Approach}\label{s.firstapproach}

{\em Preliminaries.} %After we begin, 
We use notation $\source(f),\target(f)$, $\dom(f),\cod(f)$, and $\bar f, \hat f$, more or less interchangeably, to denote the source and target of an element of a generalized category. The lattermost notation may be used when it improves readability of formulas. %
We write composition $G \of F := (f \mapsto G(F(f)))$ and in general, for mappings $F$ and $G$ with common domain and codomain (in which concatenation is meaningful) we define the operation
$$G \vertof F := (f \mapsto G(f)F(f)),$$
%For natural transformations $\beta$ from $F$ to $G$, $\alpha$ from $F'$ to $G'$ between between categories or generalized categories $\sC$, $\sD$, and $\sE$ we set
%$$\beta \hozof \alpha := \hat{\alpha} \of \beta \vertof \alpha \of \bar{\beta},$$
the standard vertical composition operation \cite{MacCW}. %
In any context where it is meaningful, we use the standard arrow notation $f:a \to b$ to mean that an element $f$ is given, the source of $f$ is $a$, and the target of $f$ is $b$. %
%We will encounter planar binary trees at several places. For such a tree $\ft$ we define 
%\begin{align*}
%\troot(\ft) &\text{ is the root of $\ft$.} \\
%\tleft(\ft) &\text{ is the tree given by the left descendant of the root, and its descendants.} \\
%\tright(\ft) &\text{ is the tree given by the right descendant of the root, and its descendants.}
%\end{align*}
The notation $\downarrow$ %``a.d.'' %used throughout the article 
%abbreviates, ``and defined'', ``are defined'', ``all defined'', or ``and both sides are defined,'' in other words, 
indicates that all composed pairs of elements in the expression or relation are in fact composable pairs. 

\subsection{Definition}\label{ss.gencat}

\dfn\label{d.gencat}
A {\em generalized category} is a structure $(\sC, \dleq, \source,\target, \cdot)$ where $\sC$ is a set, $\dleq$ is a relation on $\sC$, $\source$ and $\target$ are mappings $\sC \to \sC$, and $(\cdot)$ is a partially defined mapping $\sC \times \sC \to \sC$, denoted $a \cdot b$ or $ab$. These are required to satisfy
\enu
	\item $(\sC, \dleq)$ is a partially ordered set, % with bottom element $\flr$.
			\label{ax.gencat-po}
	\item $ab$ $\downarrow$ if and only if $\source(a) \dleq \target(b)$. \label{ax.gencat-po-comp}
	\item If $(ab)c$ $\downarrow$ or $a(bc)$ $\downarrow$ then $(ab)c = a(bc)$. \label{ax.gencat-assoc}
	\item If $ab$ $\downarrow$ then $\source(ab) = \source(b)$ and $\target(ab) = \target(a)$. \label{ax.gencat-comp-st}
	\item (Element-Identity) For all $a \in \sC$, there exists $b \in \sC$ such that \label{ax.gencat-element-id}
		\enu
			\item $\source(b) = \target(b) = a$,
			\item if $bc$ $\downarrow$ then $bc = c$,
			\item if $cb$ $\downarrow$ then $cb = c$,
%			\stuffgoeshere % what about when it's a proximal composition? ---> 11/21/16 It should be fine, for example i = id_N and f: \R \to \R should give fi=f.
		\Enu
	\item (Object-Identity) Let $a \in \sC$ and $\source(a) = \target(a) = a$. Then \label{ax.gencat-object-id}
		\enu
			\item if $ba$ $\downarrow$ then $ba = b$. 
			\item If $ab$ $\downarrow$ then $ab = b$.
		\Enu
	\item (Order Congruences\footnote{These axioms are needed for the Kleisli construction in Chapter \ref{c.genm}.})
		\enu
			\item If $a \dleq b$ then $\source(a) \dleq \source(b)$ and $\target(a) \dleq \target(b)$. \label{ax.gencat-order1}
			\item $a \dleq b$ and $c \dleq d$ and $ac,bd$ $\downarrow$ implies $ac \dleq bd.$ \label{ax.gencat-order2}
						% ---> 8/21/16 p. 21
			\item $a \dleq b$ implies $1_a \dleq 1_b$. \label{ax.order3}
		\Enu
\Enu
%\stuffgoeshere
%A generalized category is said to be equipped {\em with identities} if for every $a \in \sC$, if there exists $b \in \sC$ such that $\source(b) = a$ or $\target(b) = a$, then there exists $c \in \sC$ such that $c b$ $\downarrow$ implies $c b = b$, and $b c$ $\downarrow$ implies $b c = b$.%\label{x.subjectidentity} \label{i.subjectidentity}
The element $c$ of axiom (\ref{ax.gencat-element-id}) %
is unique, and is denoted $1_a$ or $\id_a$, and called the {\em identity} on $a$. %
%A {\em morphism} $\rho$ in $\sC$ is simply any object of $\sC$, that is, the set-theoretic complement in $\sC$ of the set of objects. %
%An {\em arrow} or {\em proper arrow} $\ksi$ of $\sC$ is a non-subject of $\sC$, that is, the set-theoretic complement in $\sC$ of the set of subjects. %
\Dfn

As a partially ordered set a generalized category resembles, but is weaker than, a domain \cite{GrZ+}, indeed motivation for the ordering comes from domain theory \cite{SmPl1, WaD1}. If $a \dleq b$, we say that $a$ {\em approximates} $b$, and $b$ {\em sharpens} $a$. %We may think of bottom as {\em failure}, {\em abort}, {\em no value}, or {\em bottom}. 
When the ordering $\dleq$ is nontrivial, one may call $\sC$ a {\em proximal} generalized category. %
We often think of proximal categories as having at least a bottom element $\bottom$, but we do not assume this in the definition, since we would like, as a special case, for an ordinary one-category to be a generalized category. %
If the order given by $\dleq$ is discrete, we might say that the generalized category is {\em discrete}, and similarly for other order-theoretic attributes, but as this may lead to confusion with the notion of a discrete category (one with essentially no morphisms), we shall say instead that such a generalized category is a {\em sharp} generalized category. We allow ourselves to refer to a {\em proximal} generalized category whenever we wish to emphasize that we refer to a generalized category that is not assumed to be sharp. 

%\dfn\label{d.morphism}
%The following terminology is convenient and fits with the standard terminology for categories. Let $\sC$ be a generalized category.
%
An {\em element} $f \in \sC$ is an element $f$ of the underlying set $\sC$. 
%
An {\em object} $a$ in $\sC$ is an element $a$ of $\sC$ such that $\source(a) = \target(a) = a$. 
%
We write $\Ob(\sC)$ for the set of objects. %We inevitably use the term ``arrow'' with its normal informal meaning. 
%
For $a \in \sC$, we define the {\em height} of $a$, denoted $\height(a),$
to be the maximum of the set of nonnegative integers $n$ such that there exists a sequence $\vec s$ of source and target operations of length $n$ such that $\vec s (i)$ is an object, %
%If there is no such $n$, we say that $\height(a) = \infty$. 
unless there is an infinite sequence $\vec s$ of source and target operations such that no subsequence yields an object. In that case, we say that $\height(a) = \infty$. 
%A {\em 1-cell} is an element whose domain and codomain are objects. {\em 2-cells}, {\em 3-cells}, etc. are defined similarly in terms of proximity to objects, if there are any. %Note that 1-cells are 2-cells are 3-cells, and so on: these classes are nested. 
%\Dfn

With this terminology, Definition \ref{d.gencat} says that in a generalized category with identities, every element $a$ has an identity $1_a$, and that if the element is an object, this identity is $a$ itself. %
If $a \in \sC$ has identity $1_a$ and is not an object, then $a \neq 1_a$. %
%\footnote{We assume these elements $1_a$ to exist because otherwise the structure of adjoints and natural transformations is not as well-behaved.}

The maps $\source$ and $\target$ of the definition are called the  {\em source} or {\em domain} and {\em target} or {\em codomain} maps, respectively. We may sometimes denote the map $\source(a)$ by $\bar a$, and the map $\target(a)$ by $\hat a$. %The notation $\source(a)$ and $\target(a)$, respectively, may also be used. %This is because $\source(\cdot)$ and $\target(\cdot)$ commonly arises as the domain/codomain of a function. %[here: good place for an example of something, not functions, which forms a generalized category which is not a category.]

%(4) The axiom (\ref{x.objectidentity}) says that objects (defined below) are their own identities. This is one of the principles of generalized category theory that takes some getting used to when approached from category theory. 

%(5) [Rough Edge, March 2014] The axioms stating that all subjects in a generalized category possess identities is perhaps the one that is most easily tinkered with. There are two extremes: this is the first extreme, the other extreme is that {\em only} the objects have identities, and in general, one could study a generalized category in which some of the subjects have identities, and others do not. We do not know at this time what is the best definition to use. We assume that all subjects have identities in order to make adjoints and limits behave in a convenient manner (see below). 
% March 31, 2014: My view on this is that it would be crazy to subject yourself to the lack of identities over subjects! The theory of natural equivalence wouldn't work---the very first things proved in the following manuscript wouldn't work--->the material on the isoskeleton. Without the isoskeleton, you really don't have an interesting structure to think about. So generalized categories should have identities over subjects.  


%\footnote{Here, and for the rest of this note, we shall refer to \cite{MacCW} as shorthand for ``the many good references available on category theory,'' which generally tend to follow \cite{MacCW} in their conventions. See, for instance, \cite{KaSc,WeL}, and many other excellent references.} 

%It may be worthwhile to delay further progress at this elementary stage by developing a simple example. 

%The converse does not hold: from a generalized category it is not possible in general to construct a category this way. A simple example is seen by taking the generalized category $\sC$ to be ()()()().

Given a generalized category $\sC$, any element of $\sC$ may be composed with other compatible elements, and it is equipped with a ``tail'' of fellow elements, defined by the $\source$ and $\target$ maps. We think of the product %{\em semitically}, that is, 
as developing from right to left, and we may write $c:a \to b$
when $\source(a) = b$, $\target(a) = c$. %
Note as an aside that if one pictures instead a representation $a = \sous{c}a_b$ of $a$, one has a picture of composition $\sous{c}a_b \,\sous{b}d_e = \sous{c}(ad)_e$. %In a generalized category, we tend to write $\sous{c}a_b$ or $a = \sous{c}a_b$ instead of $a:b \to c$ (this notation is fitting, however, for morphisms---see below). 
This notation can be iterated to 
$$a = \sous{\sous{g}c_f}a_{\sous{e}b_d}$$ 
In this manner one can visualize a binary tree. %When the product $ab$ is not defined, we write $ab = \nll$, and we may write $ab \,\downarrow$ to indicate that the product $ab$ exists (or in other words, $\target(b) = \source(a)$). %The notation $f:X \to Y$ or $f = \sous{Y}f_X$ express the same fact that $\source(f) = X$ and $\target(f) = Y$. % and serve to remind us that in a generalized category domains and codomain are not necessarily objects (though they can be) and elements having domain and codomain are not necessarily morphisms in the usual sense. 

%Considering a concrete example, such as the generalized category generated by the category of all groups, we may write $\id_X$ for $X$, with the identification $\id_X = X$ being understood. %it being understood that we are identifying the element being denoted with the element $X$ itself---depending on whether we wish to take an ``object'' or an ``arrow'' perspective (both perspectives are useful in different situations). 
%This does not cause any confusion, unless you aren't used to it, and it is not hard to adjust. This step is not new: A similar definition can be found in \cite{MacCW}. In fact 

% I like this, but I am trying to ween myself off this habit...
%Duality (we follow the terminology of \cite{MacCW}) has extraordinary power, in ways both good and bad. It can save time---sometimes cutting it in half---but it can also be disorienting. It can cause interesting phenomena to emerge, as well as tedious redudancies. 

% I LIKE THIS PARAGRAPH BUT IT IS TOO PEDAGOGICAL FOR WHAT WE HAVE RIGHT NOW:
%The monics in $\sC$ are those elements which cancel on the left, and the epics are the elements in $\sC$ which cancel on the right. In many concrete examples, the injective (surjective) maps are monic (epic, respectively), but there may exist monics which are not injective, and epics which are not surjective. In spite of this, it is still helpful to think of injective and surjective maps when calculating with monics and epics. In fact, the etymology of these terms is rooted in this intuition: monic from Greek prefix {\em mono} for 1 (as in 1-to-1), and epic from Greek prefix {\em epi} for on/over (as in onto). 
%The reader should keep in mind the triples {\em inj., monic, left}, and {\em surj., epic, right}. 

%We now make note of the glossary \ref{s.glossary} at the end of this section which the reader may wish to glance over, and refer back to as needed. We take as given much of the standard terminology of category theory (as in \cite{MacCW}), brought over to the setting of generalized categories. 

%We saw above that a category can generate a generalized category. 

\subsection{An Alternative Approach}\label{ss.secondapproach}

We pause to make note of an alternative approach, and discuss why we choose the approach of Definition \ref{d.gencat}. %

\dfn\label{d.gencatalt}
A {\em generalized category} is a structure $(\sC, \dleq, \source,\target, \cdot)$ where $\sC$ is a set, $\dleq$ is a relation on $\sC$, $\source$ and $\target$ are operators (mappings) $\sC \to \sC$, and $(\cdot)$ is a partially defined binary operation $\sC \times \sC \to \sC$, denoted $a \cdot b$ or $ab$. These are required to satisfy
\enu
	\item $(\sC, \dleq)$ is a partially ordered set, % with bottom element $\flr$.
	\item If $(ab)c$ $\downarrow$ or $a(bc)$ $\downarrow$ then $(ab)c = a(bc)$. 
	\item If $ab$ $\downarrow$ then $\source(ab) = \source(b)$ and $\target(ab) = \target(a)$.
	\item $ab$ $\downarrow$ if and only if $\source(a) \dleq \target(b)$.
	\item (Object-Identity) Let $a \in \sC$ and $\source(a) = \target(a) = a$. Then \label{ax.gencatalt-objid}
		\enu
			\item if $ba$ $\downarrow$ then $ba = b$. 
			\item If $ab$ $\downarrow$ then $ab = b$.
		\Enu
	\item (Order Congruences)
		\enu
			\item If $a \dleq b$ then $\source(a) \dleq \source(b)$ and $\target(a) \dleq \target(b)$. \label{ax.gencatalt-order1}
			\item $a \dleq b$ and $c \dleq d$ and $ac,bd$ $\downarrow$ implies $ac \dleq bd.$ \label{ax.gencatalt-order2}
				% ---> 8/21/16 p. 21
			\item $a \dleq b$ implies $1_a \dleq 1_b$. \label{ax.gencatalt-order3}
		\Enu
\Enu
%\stuffgoeshere
A generalized category is said to be equipped {\em with identities} if for every $a \in \sC$, if there exists $b \in \sC$ such that $\source(b) = a$ or $\target(b) = a$, then there exists $c \in \sC$ such that $c b$ $\downarrow$ implies $c b = b$, and $b c$ $\downarrow$ implies $b c = b$.%\label{x.subjectidentity} \label{i.subjectidentity}
The element $c$ %of axiom (\ref{x.subjectidentity}) 
is unique, and is denoted $1_a$ or $\id_a$, and called the {\em identity} on $a$. %
An {\em element} $f \in \sC$ is an element $f$ of the underlying set $\sC$. %
An {\em object} $a$ in $\sC$ is an element $a$ of $\sC$ such that $\source(a) = \target(a) = a$. %
A {\em subject} $U$ in $\sC$ is an element $U$ of $\sC$ such that there exists $f \in \sC$ such that $\source(f) = U$, or there exists $f \in \sC$ such that $\target(f) = U$. %
%A {\em morphism} $\rho$ in $\sC$ is simply any object of $\sC$, that is, the set-theoretic complement in $\sC$ of the set of objects. %
%An {\em arrow} or {\em proper arrow} $\ksi$ of $\sC$ is a non-subject of $\sC$, that is, the set-theoretic complement in $\sC$ of the set of subjects. %
\Dfn

The approach of Definition \ref{d.gencat} has the advantage of having fewer basic concepts than Definition \ref{d.gencatalt}. %
All elements are subjects and all elements have identities. %
This makes many steps of the development go smoothly. %
On the other hand, Definition \ref{d.gencat} creates so many identities that one sometimes wonders if they are better avoided after all. %
%by the presence of so many identities. (For example, an identity $1_{1_{a}}$ for every identity, and so on.) %
% can be an inconvenience at times. %
%Indeed, at times it simply seems strange. %
Thus one might seem to be at an impasse concerning whether Definition \ref{d.gencat} or Definition \ref{d.gencatalt} is more preferable. % 
This ambivalence is resolved by the notion of ideal category introduced in Chapter \ref{c.th}. %
Ideal categories arise naturally in categorical logic. %
In such categories, and in particular in the generalized category of contexts $\boC\Lambda$, there are identities present %
just as Definition \ref{d.gencat} requires. %
This tipping of the scales is the reason why we favor Definition \ref{d.gencat} over Definition \ref{d.gencatalt}. %
In order to facilitate discussions about generalized categories in the sense of Definition \ref{d.gencat}, we say that {\em closing over $1_{()}$} is the obvious operation of ensuring (via free generation where needed) that axiom (\ref{ax.gencat-element-id}) is satisfied. 

%In particular all generalized categories are assumed to be small and locally small except where specified. ???????? % i have no idea...

%%%

\subsection{Resuming, from Definition \ref{d.gencat}}

\prop\label{p.duality}
Up to reversal of $\,\dleq$, Definition \ref{d.gencat} is symmetric in the source and target maps $\source$ and $\target$. Therefore every proof $\Phi$ about a generalized category $\sC$ continues to hold when, in all assumptions, definitions, and deduction steps, composition, the order $\dleq$, and the role of source and target are reversed. 
\Prop%\exa()()()\Exa

Such a proof $\Phi'$ is said to be obtained from $\Phi$ ``by duality'' \cite{MacCW}. %
This simple fact has a profound effect on the entire subject. %
The generalized category formed by the operation of Proposition \ref{p.duality} is called the {\em opposite generalized category} $\sC^{op}$ of $\sC$. %
%


\begin{example}
Let $\sC$ be a category \cite{MacCW}. %
Then {\em the generalized category generated by $\sC$} is obtained from $\sC$ by identifying the identity $1_X$ of each object $X \in \sC$ with $X$, and closing over $1_{()}$. %
Considering a concrete example, such as the generalized category generated by the category of all groups, we may write $\id_X$ for $X$, with the identification $\id_X = X$ being understood. %
More formally, we define: %

\dfn\label{d.cat}
A generalized category $\sC$ is a {\em category} or {\em one-category} if the source and target of every nonidentity $f$ in $\sC$ is an object in $\sC$. 
\Dfn

%In summary, 
We now have a rough ontology:

$$
\begin{tabular}{c|c}
\begin{tabular}{c}
sharp category \\
= category 
\end{tabular} 
	& 
\begin{tabular}{c}
proximal category
\end{tabular} 
			\\ \hline
\begin{tabular}{c}
sharp generalized category 
\end{tabular} 
	& 
\begin{tabular}{c}
proximal generalized category \\
= generalized category \\
\end{tabular}
\end{tabular}
$$
\end{example}

\begin{example}
In some instances it is possible to write down a generalized category explicitly. There is an empty generalized category, and $\sC = \set{a:a \to a}$, the trivial generalized category. %
More generally, any set $S$ is a generalized category after setting $\source(a) = \target(a) = a$ for $a \in S$, we say that the generalized category is {\em discrete} or a {\em zero-category}, or simply that it is a set. %
(Thus, sets and categories are examples of generalized categories.) 
Because of the identity axiom, other than finite sets there are no finite generalized categories. 
To amend language, we therefore define:

\dfn\label{d.figen}
A generalized category $\sC$ is {\em finitely generated} if there is a finite set $\sC$ such that the remainder of $\sC$ consists only of identities. 
\Dfn

%that are finite one-categories. 
There are many examples of generalized categories that are not ordinary categories, the simplest perhaps being
%$\sC = \set{\sous{a}a_a, \sous{b}b_b, \sous{c}c_c},$ 
%$\sD = \set{\sous{a}a_a, \sous{b}b_b, \sous{b}c_a, \sous{b}d_a},$  
%$\sE = \set{\sous{a}a_a, \sous{a} b_b}$ 
$\sC = \set{a: a \to a, b: a \to b}$.
%(notice that $b$ is a subject, but it does not need an identity). 
Another simple example is 
%$\set{\sous{a} b_a, \sous{b} a_b}$ 
$\sC = \set{a: b \to b, b: a \to a}$.
This generalized category is finite, but does not possess objects, moreover every element is a subject. %If a generalized category is infinite, on the other hand, it may exhibit other pathological behaviors, for example, it 
A generalized category may also lack objects due to infinite descent, for example
$\sC = \set{a_n : a_{n-1} \to a_{n-1} \mid n \in \Z}.$ %, or it may consist entirely of subjects due to infinite ascent. %Sets in ZFC with the axiom of Foundation, or non-well-founded sets, . In this sense, a generalized category can be thought of loosely as a non-well-founded category. 
% Dimitris is right about this: the resemblance is merely a vague one that has no bearing on the development.
\end{example}

\begin{example}
Besides the aforementioned sources in domain theory, 
motivation for the proximal relation $\dleq$ in a proximal generalized category comes, via categorical semantics, from %
the subtyping relation in some type theoretical systems \cite{Reynolds1980,Pierce1}, a feature characteristically found in object-oriented languages. %
Subtyped type systems are often preorders, thus, we can access the semantics given by a generalized category by, for one thing, equating mutual subtypes. %
As in domain theoretic orders, subtyped type systems often include a bottom type; %
they may also include a global maximum type. %
Such a structuring of types creates a comfortable intuitive environment for type theory, and makes the type checker behave less rigidly. %
However, representation of data in such systems can demand trade-offs that make such systems less suitable for some kinds of applications. %
Moreover, in an industry-level type system, problems and subtleties may arise due to the need for subtyping rules to %
interact coherently with rules that govern other advantageous type features, such as records, recursive types, and polymorphism. %
In practice, therefore, subtyping produces both benefits as well as costs, 
and has been the focus of much research and discussion in computer science. %

One approach to implementing subtyping involves data type {\em coercion,} or the automated physical modification of stored data at run-time. %
Type-theoretically, condition (\ref{ax.gencat-comp-st}) of Definition \ref{d.gencat} corresponds to a type system in which there exists a coercive evaluation mechanism. %
%Thus, for example, a term $f$ of function type $A \to B$, is able to be evaluated at a value $x$ of type $A_0$ if $A_0$ is a subtype of the domain type $A$ of $f$. %
%If the semantics of our type system is {\em generalized} category theory, however, we must interpret these as {\em graduated types}, or types in a system in which there is no distinction at all between types and values. This additional freedom allows new kinds of type safety mechanisms based on graduated approximations, as opposed to the usual, two-sorted term/type paradigm. 
\end{example}

%{\em Remarks on the definition.} (1) Categories arise from a dichotomy: objects (which are not morphisms) and morphisms (which are not objects). We merge the two by identifying the object with the identity on the object: $X = \id_X$. %This makes abstract study go more smoothly, and makes it easier to develop and study generalizations of categories (such as bicategories and 2-categories). %Perhaps counterintuitively, this seemingly roundabout approach to categories has many benefits. In particular, it is helpful (as we will see) to understand which principles of category theory are native to category theory, and which are really coming from the fact that you are working with a generalized category. %The intuition for generalized categories is also useful later when one is confronted with higher categories. 

%Another way to approach the concept of a category is as follows. 

%Thus there are two notions one might adopt as that of a {\em morphism}: as something that ``goes between objects'', or, of something that ``nothing ever goes between''. A category is precisely that sort of generalized category in which these two notions coincide. 
%Or again, say that Then a category is a generalized category composed of near-objects. 

%(3) We invite the reader to think of their favorite concretely presented category, say, Ab, the category of abelian groups. The basic idea of approaching a category like Ab generalized-category-theoretically is that we wish to throw off the objects, viewing them as a nuisance, and focus on the morphisms alone. We do this by thinking of domains and codomains in a category as being maps, too---namely, the identity maps on the objects. %Once you do this, it is not hard to see that there is no reason, a priori, why any map could not play the role of domain or codomain.

%(3) (Some opinions.) This definition of a category doesn't tell the whole story. In fact, one cannot (we believe) ignore category theory in its usual axiomatization. If we call this the ``category-theoretic'' perspective, then we can say that the category-theoretic perspective (of objects and morphisms) and the generalized-category-theoretic perspective (of morphisms only) interact with one another, and one can never entirely take one and ignore the other. Another way of looking at it is to say that in studying mathematics there is initially the ``set theoretical perspective'', which becomes the ``category-theoretic perspective'', which finally goes over entirely to the generalized-category-theoretic perspective. As we see it, all three play an important role. Category theory and set theory by itself are not enough because certain structures that are degenerate in a category only emerge in a generalized category, and this strengthens the intuition of a category. Generalized category theory is also a convenient foundation for higher category theory, which plays an unavoidable role in current research in many fields.

%There are many very well-known categories, all arising from specifying a system of things and a system of maps between them. For example, the familiar structures of basic algebra and of topology are categories, often denoted Ab, Top, $R$-Mod, etc. A good discussion of these examples can be found, for example, in \cite{MacCW}. 

%\exa
%A {\em discrete} generalized category consists only of objects. After ignoring the trivial domain and codomain map structure, a category $\sC$ with one object is a semigroup with identity. If, in addition, all elements are invertible (i.e., isomorphisms) then it is a group. Thus all semigroups with 1 and all groups are examples of categories. (Thus generalized category theory gives rise to a notion of generalized groupoids.) 
%\Exa

\begin{example}\label{ex.binarytree}
%An important class of examples is given by planar binary trees. 
Let $\sC$ be a generalized category, and consider 
%A given generalized category $\sC$ may be viewed as an abstraction rather like a binary tree by adding 
the condition on $\sC$ that hom sets should contain a unique element or else be empty. 
To obtain a (possibly infinite) planar binary tree one adds the condition that source and target may not loop except trivially, that is, for every element $a \in \sC$, and for every finite sequence $(x_1, \dots, x_n)$ where $x_i$ is either $\source$ or $\target$ (source or target) if $x_n x_{n-1} \dots x_1 a = a$ then it is required that $\source a = \target a = a$, that is, or (using the terminology of trees) that $a$ is a leaf.  % (The reader might enjoy finding what they are.) 
%
Presheaves on such trees arise for example in database theory, see for example \cite{SpK3}. %
\end{example}

\exa
A {\em generalized (directed) graph} (cf. Chapter \ref{c.th}) is simply a triple $(\sA, \source, \target)$, where $\sA$ is a carrier set, %
and $\source, \target$ are maps $\sA \to \sA$. %
An element of $\sA$ is (synonymously) an {\em edge}. %
%An element of $\sI_\sA$ is an {\em identity} in $\sA$. % 
An {\em object} in a generalized graph is an element $a \in \sA$ such that $\source a = \target a = a$, %
that is, a common fixed point of the endomorphisms $\source$ and $\target$. 
%We say that a {\em subject} in a generalized graph is an element $a \in \sA$ such that there is an element $f \in \sA$ such that either $sf = a$ or $tf = a$. %
%A {\em generalized graph with identities} is a generalized graph with an identity over every subject. %a generalized graph $\sA$ equipped with a set $\sI \lies \sA$. %every subject $a$ in $\sA$ is equipped with a designated element $1_a$. %
%We write $\Ob(\sA)$, $\Sb(\sA)$ for the set of objects and subjects of $\sA$, respectively. %
%Same as in section \ref{s.gencat}, an {\em object} of a generalized graph $\sA$ is $a \in \sA$ such that $sa = ta = a$. 
Ordinary graphs correspond bijectively with 1-dimensional generalized graphs, 
where we say that generalized graph is {\em 1-dimensional} if $\source \source = \source \text{ and } \target \target = \target.$
With the obvious composition via compound paths, a generalized graph becomes a (sharp) generalized category. 

There are plentiful settings where generalized graphs may arise. For example, suppose that there is a system of goods $\sA_0$. %
The edges of $\sA$ are certificates (issued, say perhaps, by different governing bodies) that say that a good $a \in \sA_0$ may be exchanged for another good $b \in \sA_0$. %
Suppose it is accepted that a good is always exchangeable for itself. %
Now let's suppose that such certificates themselves may be exchanged, % either for goods, or for other certificates. %
but that this requires that one has a higher-level certificate for this higher-level trade. %
If we imagine a certain impetus exists among those we imagine making the exchanges, we can expect that there will next arise trading for these certificates as well, 
giving rise to a generalized graph (in fact, a generalized deductive system, via a simple extension of Kolmogorov's reasoning about intuitionistic logic in \cite{Kolmogorov1}). 
\Exa

\exa\label{x.setf}
For a planar binary tree $\ft$, let
\begin{align*}
\troot(\ft) &\text{ is the root of $\ft$.} \\
\tleft(\ft) &\text{ is the tree given by the left descendant of the root, and its descendants.} \\
\tright(\ft) &\text{ is the tree given by the right descendant of the root, and its descendants.}
\end{align*}
From any category $\sC$ we can form a sharp generalized category $\sC f$ as follows: take the set $\sC f$ to be the set of all planar binary trees of morphisms in $\sC$, subject to a source-and-target condition %we will state in a moment. % 
$$\dom \troot(\dom \ff) = \dom \troot(\ff) ,$$
and
$$\cod \troot(\cod \ff) = \cod \troot(\ff),$$
where if $\ff$ be such a tree,
$$\cod \ff = \tleft(\ff),$$
the left descendent tree of $\ff$, and 
$$\dom \ff = \tright(\ff),$$
the right descendent tree of $\ff$. 
These conditions set up a recursive condition on elements of $\sC f$.
For $\fg, \ff \in \sC f$, we define $\fg \cdot \ff$ to be the tree $\fh$ with left descendent $\troot(\ff)$, right descendent $\troot(\fg)$ and root $\troot(g)\cdot \troot(f)$. 
%where $\vec{g \of f}$ denotes the tree with root $g \of f$ and left descendent tree that of $\fg$, and right descendent tree that of $\ff$. 
This is a well-defined product, by the source-and-target condition above. It is checked that this is a (sharp) generalized category. An element of $\sC f$ may be visualized as
$$
\begin{tikzcd}
															& \bullet \ar[d, dashed, "\hat \ff"]					 		\\
	X \ar[d, dashed, "\bar \ff"] \ar[r, dashed, "\ff"] 							& Y 		\\
	\bullet															&
\end{tikzcd}
$$
%or as
%$$
%pic
%\begin{tikz}
%	
%\end{tikz}
%$$
Constructions on the original $\sC$ can be carried over to $\sC f$, for example, if $\sC$ has products (equalizers, coproducts, coequalizers), then so (respectively) does $\sC f$. If $\sC$ is (co)complete, however, it does not imply that $\sC f$ is (co)complete. 

\begin{comment}
Lawvere's comma category construction \cite{LwV63, MacCW} may also be observed to yield generalized categories, even when the input data is an ordinary category. %
Fix two generalized categories $\sC$, $\sD$, and $\sE$, and functors $S:\sD \to \sC$ and $T: \sE \to \sC$. %
Let 
$$
(S,T) := \set{(d,e,\ff) \mid d \in \sD, e \in \sE, \ff \text{ is a planar binary tree of pairs $(f,g)$, $f,g \in \sC$ }}
$$
Set
$$\overline{(d,e,\ff)} = (\bar d, \bar e, \tright(\ff)),$$
$$\widehat{(d,e,\ff)} = (\hat d, \hat e, \tleft(\ff)).$$
Composition in $(S,T)$ is defined as in $\sC f$.
\end{comment}

\Exa






%--------------------------------------------------------------------
%HIGHER CATEGORY THEORY

\begin{comment}
\dfn
%EEE % this isn't going to fly that well. You would be much wiser to introduce the ***usual*** definition and ***then*** introduce the generalized category version. Otherwise it's going to raise eyebrows.

Let $\sC$ be a generalized category (keeping all the notation of Definition \ref{d.gencat}) on which a second partially-defined operation $(\star):\sC \cross \sC \to \sC$ is defined. Then $\sC$ is a {\em 2-generalized category} if, for all $a,b,c \in \sC$,
\enu
	\item $(a \star b) \star c = a \star (b \star c),$
	\item $a \star b \,\downarrow$ if and only if $\source \target(a) = \source \source(a) = \target \target(b) = \source \target(b)$, 
	\item %(Subject-Identity) 
	For all $a \in \Sb(\sC),$ there exists an identity $e_a$ (necessarily unique) with respect to $\star$, defined in the manner identical to $1_a$ with respect to the first operation of $\sC$, \label{ax.subjectidentity2}
	\item %(Object-Identity) 
	For all $a \in \Ob(\sC),$ $e_a = 1_a = a$. \label{ax.objectidentity2}
	\item %(Cellular Axioms) 
		\enu
			\item $\source(a \star b) = \source(a) \source(b),$
			\item $\target(a \star b) = \target(a) \target(b),$
			\item $1_a \star 1_b = 1_{ab},$
			\item $(a \star b) (c \star d) = (ac) \star (bd)$, whenever either side is defined (interchange law).
		\Enu \label{ax.cellular2}
\Enu
\Dfn

An immediate consequence of the definition is that elements which satisfy $\source \source (a) = \target \source(a)$ and $\target \target (a) = \source \target (a)$, called {\em 2-cells}, play a special role in any 2-generalized category. A 2-cell can be viewed as an element with a ``simplified'' binary tree structure:
$$a = \sous{\sous{e}c}a_{b_d}$$
It can be visualized using arrows: %(in what is known as a pasting diagram):
$$
pic
$$
Similarly, $n$-categories can be viewed as generalized categories equipped with $n-1$ additional operations satisfying properties analogous to these, giving rise to a class of {\em $k$-cells} for every $k \leq n$. %!!!!!!!!!!!!!!!!!! % cheng-lauda, baez, ...

%2-categories play an important role in category theory, and they also arise in topology, logic, and theoretical physics (see \cite{Baez1,JoN2,???}).

A fibered category can be interpreted as a generalized category that is not a category. [we add details for this example later.] EEE

% FIBERED CATEGORY (WAIT UNTIL PRESHEAVES?)
% NERVE OF A CATEGORY
% SIMPLICIAL SET N CATEGORY INFINITY CATEGORY ETC

Higher category in general is concerned with the study of $k$-morphisms for $k$ between $0$ and $\infty$. A $0$-morphism is an object, a $1$-morphism is a morphism, a $2$-morphism is a 2-cell, a $3$-morphism is a transformation or mapping between $2$-cells, and so on: a $k$-morphism is a transformation of $k-1$-morphisms. Many models have arisen for the study of such higher-categorical systems in recent years. Always in demand in these theories is the need for an approachable calculus to manage the increasing complexity brought on by climbing higher and higher up the chain of $k$-cells. Generalized categories, as we hope to show, provide an enlarged system that parallels and provides supporting concepts for ordinary category theory. %that is, much of ordinary category theory as has existed now for some time carries over to the setting of generalized categories, some of it almost entirely unchanged. 
%The notion of equivalence that naturally suggests itself for generalized categories \ref{ReferenceTheDefinition}, however, creates the need to branch some of the traditional concepts, and modify some ideas, hence it is not entirely trivial task to set up generalized categories, and get them going. We will see this better as we go.
\end{example}
\end{comment}

































%\input{gencat3(elem)} % nonnat
% file gencat3(elem).tex Created by Lucius Schoenbaum, November 20, 2016





\section{Elementary Theory, Category of Invertibles}\label{s.elem}

We now define functors and hom sets: 

\dfn\label{d.functor}
A mapping $\sC \to \sC'$ between generalized categories is {\em functorial} or a {\em functor} if 
\enu
	\item $a \dleq b \eimplies F(a) \dleq F(b),$
	\item $F(\bar a) = \overline{F(a)},$
	\item $F(\hat a) = \widehat{F(a)},$
	\item $F(ab) = F(a)F(b), \text{ if } ab \,\downarrow,$
	\item $F(1_a) = 1_{F(a)}.$
\Enu
We thus have a category ${\bf GenCat}$ of generalized categories and functors. %
\Dfn

%We say $F$ is {\em unital} if for all $a \in \Sb(\sC)$, 
%From now on we always assume that a functor between generalized categories is unital. 
Functors are also called {\em covariant functors}. A {\em contravariant functor} from $\sC$ to $\sC'$ is a unital map satisfying
\enu
	\item if $a \dleq b$ then $F(b) \dleq F(a)$,
	\item $F(\bar a) = \widehat{F(a)},$
	\item $F(\hat a) = \overline{F(a)},$
	\item $F(ab) = F(b)F(a) \text{ if } ab \,\downarrow,$
\Enu
instead of the corresponding covariant relations. %

\dfn\label{d.homsets}
The sets 
$$\hom(a,b) = \set{c \in \sC \mid \bar c = a, \hat c = b},$$ 
for $a,b \in \sC$, are called the {\em hom sets} of $\sC$. 
\Dfn

\dfn
A {\em subcategory} of a generalized category $\sC$ is a subset $\sC'$ of $\sC$ whose order is inherited from $\sC$ closed under source, target, composition, and identities: if $a \in \sC'$, then $1_a \in \sC'$. %Let $\sC'$ be a subgeneralized category of a generalized category $\sC$. %If $\sC'$ is a category, it is said to be a {\em categorial} subgeneralized category. 
%If $\sC$ is a category, then so is $\sC'$, and $\sC'$ is called a {\em subcategory} of $\sC$.
%full, faithful, fully faithful, essentially surjective. ----> in the section on functors
%full subgeneralized category ---> just after defining hom sets.
A subcategory $\sC'$ is {\em full} if $a,b \in \sC'$ implies $\hom(a,b)$ is contained in $\sC'$. 
\Dfn

%The conditions can be better understood if one thinks of the previously mentioned example in which one considers a functor $F$ between generalized categories which are planar binary trees. 
\begin{comment}
The unital condition follows from the others in the case of categories. In general, it need not follow. Consider the example %(not the simplest possible) 
$\sC = \set{x: x \to x, a: x \to x, c_1 : a \to a, c_2: a \to a, c_3: a \to a}$ with multiplication given by: $xx = x,$ and
\[
\begin{tabular}{c|ccc}
	$\cdot$		 	& $c_1$ 		& $c_2$ 		& $c_3$		\\ \hline
	$c_1$			& $c_1$		& $c_2$		& $c_3$		\\
	$c_2$			& $c_2$		& $c_2$		& $c_3$		\\
	$c_3$			& $c_3$		& $c_3$		& $c_3$		
	\end{tabular}
\]
A nonunital functor can be defined from $\sC' = \set{y: y \to y , \alpha: y \to y, d_1: \alpha \to \alpha, d_2: \alpha \to \alpha}$ by mapping 
$$y \mapsto x, \alpha \mapsto a, d_1 \mapsto c_2, d_2 \mapsto c_3.$$
\end{comment}

%The conditions described are a little stronger than those found in many books on category theory. They require that $ab = \nll$ implies $F(a)F(b) = \nll$. This is equivalent (essentially) to requiring that the preimage of a chain complex under $F$ is a set of chain complexes. 

%For the rest of this discussion, $F$ is a functor from a generalized category $\sC$ to a generalized category $\sC'$ unless specified otherwise.

%\prop\label{p.functorbasics}
The composition of two functors is a functor, and functors send objects to objects. 

\dfn
Two generalized categories $\sC$ and $\sC'$ are {\em isomorphic} if there is an invertible functor (i.e., invertible as a mapping) $F$ from $\sC$ to $\sC'$. 
\Dfn

\prop\label{p.flattening}
%The following is not really an example, but rather a general construction, but we include as an example anyway. 
There is a functor, flattening, from the category of generalized categories to the category of categories. 
\Prop
\prf
Let $\sC$ be a generalized category with identities. %Define
Let $\Ob(\sC_{flat})$ be $\set{[f] \mid f \in \sC}$, the objects of $\sC$ indexed by the elements of $\sC$. 
Let $\Mor(\sC_F)$ again be a set $\set{(f) \mid f \in \sC}$ indexed by the elements of $\sC$, and define source and target 
$s((f)) = [s(f)],$ 
$t((f)) = [t(f)].$ 
Then $\sC_{flat}$ is a category whose composition and identities are
$$(g) \cdot (f) := (gf),$$
$$
1_{[f]} = (1_f).
$$
Given a functor $F: \sC \to \sD$ in ${\bf GenCat}$, we immediately obtain a functor $\sC_{flat} \to \sD_{flat}$. 
\Prf

Note that $\sC_{flat}$ contains a flattening of the identity structure, even in cases where $\hom(a,a) = \set{1_a}$.  

There is also a category $flat \sC$, the further flattening of $\sC$ to a zero-category. %
%It is obtained simply by forgetting the source, target, and composition structure. %
It is defined by:
$$flat(f) := \begin{cases} 	(f), 	& \text{if $f = 1_g$ for some $g \in \sC$, } \\ 
					[f]	& \text{otherwise,}
	\end{cases}
$$
where $[f]$ is defined by $\source([f]) = \target([f]) = [f]$, and $(f): flat(\source(f)) \to flat(\target(f)).$
%For $a \in \sC$, $\hom(a,a)$ is a semigroup with identity $1_a$. Later, we will encounter situations where the hom sets are rings under an additional additive operation, as occurs for example in $R$-Mod.

%A generalized category can be visualized as in Figure \ref{f.gencatpicture}.\begin{figure}[t]    \centering   \pic[width=3in]{gencatpicture.jpg}   \caption{a generalized category}   \label{f.gencatpicture}\end{figure}

%%%%%%%%%%%%%%%%%%%%%%%%%%%%%%%%%%%%%%%%

%Up until now we have not made any mention of any of the notion of isomorphism in category theory. The generalized-category-theoretic perspective we have developed allows an intuitive approach which provides motivation for the notions of morphism and isomorphism of functors. %(also known as natural transformation and natural equivalence, respectively).

% REVISION, MARCH 13, 2014: 
% OLD APPROACH: "One of the awkward points, however, is that isomorphism contains the term {\em morphism}, and not all nonobjects of a generalized category are morphisms in general." -----> THEREFORE DEFINE ISOMIC AND SAY AN ISOMIC IS ISOMORPHISM IF IT IS A MORPHISM
% NEW APPROACH: 
% MOTIVATION: This is absolutely ridiculous and no one is going to like it one bit.
% NEW APPROACH: just give in and say that a morphism is another term for a non-object. 

\dfn\label{d.invertible}
If $\sC$ is a generalized category, an element $a \in \sC$ is {\em invertible} if there exists $b \in \sC$ such that $ab = 1_{\hat a}$ and $ba = 1_{\bar a}$. %An invertible morphism is an {\em isomorphism}. 
%The element $b$ is unique, and it is called the {\em inverse} of $a$. 
\Dfn

\prop\label{p.invertible}
$\phantom{v}$
\enu
	\item The inverse $a^{-1}$ of an element $a$ of $\sC$ is unique if it exists. 
	\item $\widehat{a^{-1}} = \bar a$ and $\overline{a^{-1}} = \hat a$. (Even if $\sC$ is proximal.) 
	\item All objects $a$ are invertible: $a^{-1} = a$. 
	\item Functors send invertibles to invertibles: $F(\theta^{-1}) = F(\theta)^{-1}$.
%	\item $a$ and $a^{-1}$ are coproximate.
\Enu
\Prop

There are a few ways a generalized category may be partitioned into equivalence classes:

\dfn\label{d.epimonicisoclass}
For $a, b \in \sC,$ we have the following equivalence relations:
\enu
	\item $a$ and $b$ are in the same {\em monic class}, or {\em subobject}, $a \sim_m b$, if there exists invertible element $\theta \in \sC$ such that $a\theta = b$. 
	\item $a$ and $b$ are in the same {\em epic class}, or {\em quotient}, $a \sim_e b$, if there exists invertible element $\theta \in \sC$ such that $\theta a = b$; 
	\item $a$ and $b$ are in the same {\em iso class}, $a \sim b$, if there exist invertible elements $\theta_1, \theta_2 \in \sC$ such that $\theta_1 a = b \theta_2$. 
\Enu
%We say that $a$ and $b$ belong to the same {\em monic class}, {\em epic class}, {\em iso class}, respectively.
\Dfn

Let $\Theta$ denote the set of all invertible elements in $\sC$. Define the symbol 
$$a\Theta := \set{a\cdot\theta \mid \theta \in \Theta \eand a\cdot\theta \downarrow},$$
and define the symbols $\Theta a, \Theta a\Theta$, etc. similarly. Then for $a,b \in \sC$,
$b$ belongs to the monic class of $a$ if and only if $b \in a\Theta$, %
$b$ belongs to the epic class of $a$ if and only if $b \in \Theta a$, %
and 
$b$ belongs to the iso class of $a$ if and only if $b \in \Theta a \Theta$. %
This notation is useful for back-of-the-envelope calculations, but it can be misleading: it need not be true that $\Theta f \Theta = \Theta g \Theta$, even if $f$ and $g$ are invertible. %

%This is the easiest way to write the monic, epic, and iso class of $a$ in practice. 

\dfn\label{d.epimoniciso}
An element $m$ of a generalized category $\sC$ is {\em monic} if $mf,mg \downarrow$ and $mf = mg$ implies $f = g$. %
An element $e$ in $\sC$ is {\em epi} if $fe,ge \downarrow$ and $fe = ge$ implies $f = g$. %
We say $a$ is {\em isomorphic} to $b$, denoted 
$$a \iso b,$$
if there exists an invertible element $\theta$ with $\bar\theta = a, \hat\theta = b$. %
\Dfn

If $a$ is monic and $a \sim_m b$, then $b$ is monic, and the $\theta$ given by the definition is unique. Similarly, if $a$ is epic and $a \sim_e b$. 

%\dfn
%The {\em isomorphism class} of $a$ in $\sC$ is the set of all $b \in \sC$ such that $a \simeq b$. The {\em monic class} of $a$ in $\sC$ is the set of all $b \in \sC$ such that $a \sim_m b$. Dually, one may define the {\em epic class} of $a$. Finally, the {\em isomic class} of $a \in \sC$ is the set of all $b \in \sC$ such that $a \sim b$. 
%Two elements $a,b$ are {\em isomorphic} if $a \simeq b$, {\em monically equivalent} if $a \sim_m b$, {\em epically equivalent} if $a \sim_e b$, and {\em isomically equivalent} if $a \sim b$. 
%\Dfn

For every $a,b \in \sC$, $a$ is isomorphic to $b$ iff $1_a$ is in the same iso class as $1_b$, that is,
$$a \iso b \quad \ifaoif \quad 1_a \sim 1_b.$$
 %(provided $1_a$ and $1_b$ exist, but they do, if we adopt the definition above.......)
For $a,b$ objects, this becomes: 
$$a \iso b \quad \ifaoif \quad a \sim b.$$

% The reader can verify that the monic classes partition $\sC$, and so do the epic classes.


%\dfn $S$ is an {\em isounique set} if it is either $\nll$ or it is an isomic class in $\sC$. It is {\em unique} if it is either $\nll$ or an element in $\sC$.\footnote{We use a set theory in which singleton set $\set{x}$ is equal to $x$, see \cite{Sch2008}.} This language is normally applied in the case that $S$ is defined by means of a property or set of properties. \Dfn

%\dfn A {\em monic-skeletal} generalized category is one in which every monic class consists of a single element. {\em epic-skeletal} generalized category is one in which every epic class consists of a single element. An {\em isomic-skeletal} or {\em isoskeletal} generalized category is one in which every isomic class consists of a single element. Finally, an {\em isomorphic-skeletal} or simply {\em skeletal} generalized category is one in which every isomorphism class consists of a single element.
%and {\em two-sided-skeletal} generalized category are defined similarly. [some or all these notions may be trivial.]
%An {\em isomorphism-skeletal} generalized category is one in which every isomorphism class consists of a single object.
%\Dfn Here, we are simply pointing out the possible ways of viewing a generalized category $\sC$. 

\prop\label{p.isoskel}
Let $\sC$ be a generalized category. Then the set of iso classes forms a sharp category. %
%Let $\tilde \sC$ be the set of iso classes of $\sC$. Then the composition and domain/codomain operations descend to well-defined composition and domain/codomain operations on $\tilde \sC$, making $\tilde \sC$ a category in which the objects are precisely the iso classes of invertible elements of $\sC$.
The objects of this category are the iso classes of invertible elements of $\sC$. 
\Prop
\prf
Let $\tilde \sC$ be the set of iso classes of $\sC$, %
let $\tilde a, \tilde b$, ... denote elements in $\tilde \sC$. 
Define
$$\tilde a \cdot \tilde b := %\Theta a \Theta b \Theta = 
\set{\theta_1 a \theta_2 b \theta_3 \mid \theta_1,\theta_2,\theta_3 \text{ invertible, and $\theta_1 a \theta_2 b \theta_3 \, \downarrow$}}.$$
This is a partially defined map $\tilde \sC \cross \tilde \sC \to \tilde \sC$. For $a \in \sC$, let 
$$\tilde \source (\tilde a) := \widetilde{1_{\source a}},$$
$$\tilde \target (\tilde a) := \widetilde{1_{\target a}}.$$
These operations are well-defined: if $a = \theta_1 b \theta_2$, then $\bar a$ is isomorphic to $\bar b$, so, say, $\theta 1_{\bar b} \theta^{-1} = 1_{\bar a}$, so $\widetilde{1_{\source a}} = \widetilde{1_{\source b}}$, and similarly for $\tilde \target$. %

We take the order $\dleq$ on $\tilde \sC$ to be trivial, and we check Definition \ref{d.gencat}. %
The first four conditions are immediate: for (\ref{ax.gencat-comp-st}),  %
if $\tilde a, \tilde b \in \tilde \sC$, then $ \tilde a \tilde b \,\downarrow$. This occurs 
if and only if $ \set{\theta \in \sC \mid \theta: \bar a \to \hat b \text{ is invertible}}$ is nonempty, 
if and only if $ 1_{\bar a} \sim 1_{\hat b}$, 
if and only if $\tilde \source(\tilde a) = \tilde \target(\tilde b)$. %
Next, we observe that if $\tilde a$ is an element of the form $\tilde\source \tilde b$ or $\tilde \target \tilde b$ 
in $\tilde \sC$, then it must be of the form $\widetilde{1_b}$ for some $b \in \sC$, and 
$$\tilde\target(\widetilde{1_b}) = \tilde \source(\widetilde{1_b}) = \widetilde{1_b},$$
so $\tilde{1_b}$ is an object. Next, we have
$$\widetilde{1_a} \cdot \tilde b = \set{\theta_1 1_a \theta_2 b \theta_3 } = \set{\theta_4 b \theta_3} = \tilde b,$$
and similarly, $\tilde b \widetilde{1_a} = \tilde b$ whenever the product is defined. %
So $\tilde \sC$ is a sharp generalized category, in fact a one-category, after closing over $1_{()}$. %
The second statement is merely the observation that $a$ is invertible if and only if $\tilde a = \widetilde{1_{\source a}} = \widetilde{1_{\target a}}.$
\Prf

\dfn\label{d.catinvertibles}
We refer to the category $\tilde \sC$ of Proposition \ref{p.isoskel} as the {\em category of invertibles} of $\sC$. %
\Dfn

The {\em skeleton} of a generalized category $\sC$ is any full subcategory such that each element of $\sC$ is isomorphic in $\sC$ to exactly one element of the subcategory. %
Skeletons are unique up to isomorphism \cite{MacCW}. %The same notion carries over to our present setting (by replacing the word category with the word generalized category) so generalized categories, too, can be said to have skeletons. 
In the case of a category $\sC$, %(but only in that case) % 
the category of invertibles expresses exactly the same data as a skeleton, but in a different way: any iso class that is an object in the category of invertibles contains not a set of invertibles in $\sC$ that are pairwise isomorphic, but instead, the set of all the isomorphisms that relate them pairwise to one another. %
On the other hand, an iso class that is an arrow in the category of invertibles is a noninvertible arrow $f \in \sC$ well-defined up to a commutative square with invertible columns. 

%The category of invertibles of a generalized category $\sC$ may be viewed as ``the category of invertibles'' of the generalized category. Axiomatically, 
Since every element has an identity, thus taking the category of invertibles is the same as the operation of flattening (Proposition \ref{p.flattening})
% make sure you define it above
followed by taking the skeleton, yielding the description just made in the previous paragraph. Thus it is perhaps natural to think of it as the ``category of identities'' of the generalized category. 

 %However, it is not the same notion as what we have called the category of invertibles. That is why we have added the prefix. %We are thus maintain the terminology used in \cite{MacCW}.

%[note for draft: I haven't verified that the same sort of construction gives you a generalized category if you repeat it for the isomorphism classes, the monic classes, the epic classes. I expect it does. But I only know that I will need the category of invertibles later on.]

% "unisomic" --- interesting and I like the word, but I know of no practical use for it...it's not a generalized-category-theoretic notion, it exists only when you are looking at categories.
%\dfn If $a$ and $b$ are isomic, and the isomic element $\sous{b} \theta_a$ is unique, we may write further $$a \unisomic b,$$ and say that $a$ and $b$ are {\em unisomic}. \Dfn In applications, it is common to say that $a$ and $b$ are {\em unique up to unique isomorphism}.

%In any generalized category, nonsubjects are unique. 
%Here is an example illustrating the role of isomorphism and isomic classes.

%Monic and epic classes allow us to define the following notions. 

%Since subelements have monic representatives, they consist entirely of monics, and similarly, quotients consist entirely of epics. If $\sC$ is a category, a subelement is  a {\em subobject}---but note that, even there, it is technically not an object, nor a class of objects, but rather a class of monics. %The reason to use such language for these classes is the following.

%\prop Domains of subelements and codomains of quotients in $\sC$ are isounique. epimonic classes\Prop

%\exa Later, we will study generalized categories (such as, for example, abelian categories like $R$-Mod) where every element has a kernel and a cokernel. By definition, these kernels will be certain subelements, and cokernels will be certain quotients.  \Exa

%\section{Diagrams} For a monic element $a$, we may write $$\hat a \overset{a}{\leftarrowtail} \bar a,$$ and for an epic element $a$, we may write $$\hat a \overset{a}{\twoheadleftarrow} \bar a.$$ In general we may write $$\hat a \overset{a}{\leftarrow} \bar a,$$ or we may drop the semitic orientation and write $$\bar a \overset{a}{\rightarrow} \hat a.$$ We may shorten these to $ \cdot \overset{a}{\leftarrow} \cdot,$ $\cdot \overset{a}{\leftarrowtail} \cdot,$ etc. If $\nll \neq ab,$ we may write something like: $$ \cdot \overset{a}{\leftarrow} \cdot \overset{b}{\leftarrow} \cdot$$ If $\nll \neq ab = cd,$ we may indicate this by writing something like: $$\begin{array}{c} \cdot \overset{b}{\leftarrow} \cdot \\ \downarrow {\scriptstyle a} \quad \downarrow {\scriptstyle d} \\ \cdot \overset{c}{\leftarrow} \cdot \\ \end{array}$$  Such figures are known as {\em diagrams.} In the lattermost case, it is traditional to say that the diagram {\em commutes}. This language is also applied to diagrams which are compositions of such squares triangles, or any other shape. The intended meaning is (almost always) that given any two points of the diagram, the course through the diagram in the direction indicated by the arrows is independent of the path chosen, where we think of an element of $\sC$ depicted as an arrow as a channel directing flow between the connected vertices. 

%The notion of functor for a category could be generalized in the following ways (not intended to be an exhaustive list):

%\dfn A {\em textbook functor} is ()()() A {\em good inverse functor} is ()()() A {\em floating functor} is ()()() A {\em composition-only functor} is ()()(). A {\em presheaf functor}, a notion we use in section \ref{s.presheaf} (see below). \Dfn

%We adopt the following definition. % floating functors for limits? Are we going to use them, or no? Still need to test that.

It is also the case that a functor $F$ lifts to a functorial map $\tilde F$ on the category of invertibles.
%\Prop
%\prf
Indeed, define 
%EEE %CHECK AND EXPLAIN THIS
$$\tilde F : \tilde \sC \to \tilde \sC',$$
via 
$$\tilde F(\tilde a) := \widetilde{F(a)}.$$
This is well-defined, as a consequence of (2) (which depends on the unital property of $F$):
$$\tilde F (\theta_1 a \theta_2) = \verywidetilde{F(\theta_1) F(a) F(\theta_2)} = \widetilde{F(a)}.$$
So we check functoriality: we have
$$\widetilde{1_{\source(F(a))}} = \widetilde{1_{F(\source(a))}} = \widetilde{F(1_{\source(a)})} = \tilde{F}(\widetilde{1_{\source(a)}}) = \tilde{F}(\source(\tilde a)),$$
and
$$\widetilde{1_{\source(F(a))}} = \source(\widetilde{F(a)}) = \source(\tilde{F}(\tilde{a})).$$
Similarly, 
$$\target(\tilde{F}(\tilde{a})) = \tilde{F}(\target(\tilde{a})).$$
And
$$\tilde{F}(\tilde{a}\tilde{b}) = \tilde{F}(\widetilde{a \theta b}) = \widetilde{F(a\theta b)} = \verywidetilde{F(a) \theta' F(b)} = \tilde{F}(\tilde{a}) \tilde{F}(\tilde{b}).$$
Finally, $\tilde{F}$ is unital since $\tilde \sC$ and $\tilde{\sC'}$ are categories.

\begin{comment}
% faithful criterion
\prop
Let $F:\sC \to \sC'$ be a functor. The following are equivalent provided $\sC$ and $\sC'$ are {\em well-founded}, %[artinian?...], 
that is, every sequence of domain/codomain operations applied to elements of $\sC$ terminates after finitely many steps.
\enu
	\item $F$ is injective,
	\item For every $a,b$ with $\hat a = \hat b, \bar a = \bar b$, $F(a) = F(b)$ implies $a=b$, that is, $F$ is {\em faithful}.
\Enu
\Prop
\prf
Assume the apparently weaker condition, and let $a, b$ belong to $\sC$ with $F(a) = F(b)$. Then $F(\hat a) = F(\hat b)$, and $F(\bar a) = F(\bar b)$. The same reasoning applies down any chain of domain/codomain operations, so without loss of generality, we may assume that $a$ and $b$ are objects, and in this case, the faithfulness condition implies that $a = b$. 
\Prf
\end{comment}

%\section{Equivalence of Generalized categories}

%As we saw above in the case of terminal, initial, and zero objects, 
%We often wish to consider notions only defined up to isomorphism. The analog of this notion for a generalized category is that of an element defined only up to isomic class. %In other words, what we wish to do is study the generalized category alongside its category of invertibles. 
%Hence our first task is to relate functors in the two settings. 

A notion weaker than isomorphism arises from considering the categories of invertibles. 

\dfn\label{d.equivalent}
Generalized categories $\sC$ and $\sC'$ are {\em equivalent} if their categories of invertibles are isomorphic. 
\Dfn

This definition appeals directly to a comparison of the categories of invertibles. 
%Let $F,G:\sC \to \sC'$ be two functors. We say they are {\em isomorphic functors} if they define the same functor from the category of invertibles of $\sC$ to the category of invertibles of $\sC'$. %We denote this $F \iso G.$
%Unpacking this definition, we see that what is really being defined is a notion of {\em isomorphism} $\theta$ between $F$ and $G$. Specifically, 
%
Now consider two functors $F,G: \sC \to \sC'$ that both define the same functor $\tilde{\sC} \to \widetilde{\sC'}$ on the categories of invertibles of $\sC$ and $\sC'$. This can only mean that there exist a pair of functions $\theta_1,\theta_2:\sC \to \sC'$ such that $\call{a \in \sC} \theta_i(a)$ is invertible for $i=1,2$, and for all $a \in \sC$, %(we put $\theta_1$ on the left-hand side for the sake of convenience later on):
$$ \theta_1 (a) F(a) = G(a) \theta_2 (a) \,\downarrow.$$
%We invert the first isomorphism for convenience.  
If this holds we may write 
$$F \iso G.$$

\prop\label{p.equivalent}
Two generalized categories $\sC$ and $\sC'$ are equivalent if either of the following two equivalent conditions are satisfied. 
\enu
	\item Their categories of invertibles are isomorphic via a pair $\tilde F, \tilde G$, where $\tilde G = \tilde F^{-1}$, that come from functors
	 $F: \sC \to \sC'$ and $G: \sC' \to \sC$. 
	\item There exist two functors $F, G$ from $\sC \to \sC'$ ($\sC' \to \sC$, respectively) satisfying 
		$$F \of G \iso \id_{\sC'},$$
		$$G \of F \iso \id_{\sC}.$$
\Enu
\Prop
%EEE % prove it (check carefully)

We can consider properties that a functor $\tilde F$ on the category of invertibles has as an ordinary functor, and view them as properties of the underlying functor $F$:

\dfn\label{d.essinjsurj}
A functor $F:\sC \to \sC'$ is {\em essentially injective} if it satisfies one of the following equivalent conditions, 
\enu
	\item $\tilde F$ is injective. 
	\item For $a,b \in \sC$, $F(a) = F(b)$ implies $a \sim b$.
\Enu
and $F$ is {\em essentially surjective} if it satisfies one of the following equivalent conditions: 
\enu
	\item $\tilde F$ is surjective.
	\item For $\alpha \in \sC'$, there exists $a \in \sC$ with $F(a) \sim \alpha$.
\Enu
\Dfn

%We close this section by introducing two important topics: the functor category and adjoints. 

%\section{The Functor Category}

%We are not done using the notion of isomorphism of functors. Since it gives us a notion of isomorphism, we have motivation for defining a category, which can be achieved in the following way:

From our initial investigation of equivalences between generalized categories, we arrived at the notion of equivalence via a pair of functors $F$ and $G$. We could, however, view this machinery (the pair ($\theta_1, \theta_2$)) as instead relating the two functors, and extend it:

\dfn\label{d.mof}
Let $\sC, \sC'$ be generalized categories, let $F,G:\sC \to \sC'$ be two functors. We say that a {\em morphism of functors} \cite{KaSc1} from $F$ to $G$ is a pair $(\theta_1,\theta_2)$ of maps $\sC \to \sC'$ satisfying, for all $a \in \sC$,
\begin{equation}\label{eq.mof}
\theta_1 (a) F(a) = G(a) \theta_2 (a) \, \downarrow
\end{equation}
Note that here, $\theta_1$ and $\theta_2$ are no longer presumed to be invertible. 
We may write the morphism of functors with the notation 
$(\theta_1, \theta_2): F \Rightarrow G$. 
% QUESTION: could you cleverly find a way to define a morphism of functors when the functors are between four random generalized categories??????? If so, you could set 2-categories free from all the cruddiness of the Cellular axioms.
\Dfn

%Note that the values for the map $\sC \to \sC'$ are irrelevant for the arrows, so we could just as well think of $\theta_1$ and $\theta_2$ as maps $\Sb(\sC) \to \sC'$. 
Note that the maps $\theta_1$ and $\theta_2$ are maps from $\sC$ to $\sC'$, not from $\Ob(\sC)$ to $\sC'$ (cf. \cite{MacCW}). %as the usual definition \cite{MacCW} would suggest. 

\begin{example}
%This kind of morphism, a morphism of functors, indeed arises in practice. For example, they arise in the study of sheaves (\ref{s.sheaf}), or consider the following example: 
Let $A = (a_{ij})$ be a matrix with coefficients in a ring $R$, and let $f:R \to S$ be a ring homomorphism. One naturally sets $f(A) = (f(a_{ij}))$, and doing this, one sees that
\eqn\label{eq.fdet}
\det(f(A)) = f(\det(A)).
\Eqn
This relation can be interpreted by observing that $GL_n$ is a functor from the category of rings to the category of groups, and likewise for the mapping that sends a ring to its group of units, and a ring homomorphism to the pointwise-identical homomorphism on the respective groups of units. So if $f:R \to S$, and writing $F(f)$ for the map defined above extending $f$ to a map on $GL_n (R)$, and $G(f)$ for the map changing $f$ to a map on the group of units, we have
$$\det() \of F(f) = G(f) \of \det()$$
by rewriting equation (\ref{eq.fdet}). From this expression we can read off the morphism of functors: 
$$\theta_1(f) = \det : GL_n(S) \to S^{\cross},$$
$$\theta_2(f) = \det : GL_n(R) \to R^{\cross}.$$
We see that in this example, $\theta_1$ and $\theta_2$ come from a single map $\theta$ on the objects (rings). This is not only typical of categories, it is guaranteed to happen. Indeed, if we return to the general situation of Definition \ref{d.mof}, inserting $a = 1_b$ into equation (\ref{eq.mof}) gives 
$$ \theta_1 (1_b) = \theta_2 (1_b)$$
for $b \in \sC$, so in particular, for all objects $b$,
$$ \theta_1 (b) = \theta_2 (b).$$
Thus $\theta_1$ and $\theta_2$ are identical on objects, and since one-categories have no higher morphisms, this single map on objects completely characterizes $(\theta_1, \theta_2)$. %if $\sC$ is a one-categories, as a single mapping $\Ob(\sC) \to \sD$, where $\sC$ and $\sD$.
\end{example}

In the terminology of section \ref{s.nat} that follows, this means that a morphism of functors between functors relating categories is always {\em natural}. %
In the setting of generalized categories, we might suppose that this naturality property is a condition special to one-categories, %
since it does not appear to have any a priori motivation. %
%since it does not appear linked directly to the notion of equivalence (of categories, and of general generalized categories). 
%We will see, however, that naturality (in this sense) is nevertheless used at key steps in proofs of many landmark results in category theory. 
However, %we encounter a sort of {\em a posteriori} motivation
the theory that results from dropping the naturality condition appears to be significantly weaker: %
%Specifically, we note: 
\enu
	\item There is no strict 2-category of non-natural transformations, functors, and generalized categories. 
		Here, the wheel turns on the tiniest of pedestals: in the notation of Chapter \ref{c.prelude}, the relations
		$$\bar \alpha (X) = \overline{\alpha(X)},$$
		$$\hat \alpha (X) = \widehat{\alpha(X)}$$
		hold only in the natural setting. So we do not prove Fact 1. 
	\item While there is a notion of non-natural adjunction, there is no hom set bijection. %
		A key step in the proof uses the naturality of the unit and counit maps. %
		This in turn is used to prove that left adjoints are right exact. %
	\item Because there is no adjoint hom set bijection, some theorems relating equivalences of categories with 
		%sometimes-observable 
		properties of functors %(full, faithful, essentially surjective) 
		no longer hold. In particular a full, faithful, essentially surjective functor might not define an equivalence. 
\Enu
For these reasons, we do not take the development any further until we introduce naturality in the next section. 

%By carefully making the distinction in this writing, we hope to help put the matter to rest once and for all. % Hence, we will avoid naturality for a little longer, before introducing it in the next subsection. %since (as we will see) some of the properties we expect in fact do not depend on naturality. 


%Next, we check whether we have a two-category when we look at a case in which ()()(). % Notes of August---generalized monads stuff
%\stuffgoeshere

%The operations we defined in this proof can be written formulaically in the form
%\begin{align*}
%	\psi \of \theta &= \psi\theta, 			\\
%	\psi \star \theta &= (\hat \psi \of \theta)(\psi \of \bar \theta \,). 
%\end{align*}
%But you have to be careful with these formulas because there is ambiguity with what the $\of$ operation means until after you have switched to $\theta_1, \theta_2$ from the morphism of functors $\theta$.
%The morphisms of functors are collectively sometimes called the {\em functor category}. 

%\section{Adjoints}

%Another spin on the notion of equivalence of categories gives rise to the framework of adjoints, which was first rigorously developed in \cite{Kan58}. In this paragraph, we will only point out that there is a {\em naive} notion of adjoints, coming straight from the abstract definitions. In the next section, we will see that there is stronger notion of adjoint that is important for applications. %It helps to consider the abstract notion first.



%

%

%

% adjoint pairs (non-natural case)


\begin{comment}
Next, we turn to adjoint pairs. For this notion, we follow the model of Definition \ref{p.equivalent} of an equivalence via functor $F: \sC \to \sC'$, but we must add the usual conditions, the triangle laws, in order to establish the rich theory of adjoint functors in the generalized setting. 

\dfn\label{d.adjoint}
Let $X, A$ be generalized categories, and let $F:X \to A$, $G:A \to X$ be functors. 
%$F$ and $G$ are an {\em adjoint pair} and that $F$ is the {\em left adjoint} of $G$, and that $G$ is the {\em right adjoint} of $F$, if
An {\em adjunction} $(F,G,\eta,\epsilon)$ is a pair of morphisms of functors 
$$\eta: F \of G \Rightarrow \id_{A},$$
$$\epsilon: \id_X \Rightarrow G \of F,$$
which satisfy the following identities: for all $x \in \Sb(X)$,
\begin{equation}\label{e.T1}
\epsilon_2 (F(x)) \of F(\eta_2(x)) = 1_{F(x)},
\end{equation}
and for all $a \in \Sb(X)$,
\begin{equation}\label{e.T2}
G(\epsilon_1 (a)) \of \eta_1 (G(a)) = 1_{G(a)}.
\end{equation}
In short, $\epsilon_2 F \of F \eta_2 = 1_F$, and $G \epsilon_1 \of \eta_1 G = 1_G$.
\Dfn

%The extra condition expressed by the triangular laws \ref{e.T1} and \ref{e.T2} imply an important bijection:

Thus, we have decomposed an adjunction, as defined in e.g. \cite{MacCW}, into three pieces: the unit and counit, the triangular laws, and finally, the naturality of the unit and counit. The question is whether all of these pieces are needed. If we measure this question by whether or not the important bijection 
\begin{equation}\label{eq.adjbij}
\hom(F(x), a) \iso \hom(x,G(a))
\end{equation}
can be established, then the answer is no. In the proof of this bijection (see the proof of Proposition \ref{t.nathombij} below) a key step uses the naturality of the unit and counit maps. 

It is safe to regard the bijection \ref{eq.adjbij} as a property that cannot be removed from the notion of adjunction---it is used, for example, in the proof that functors with left (right) adjoints are left (right) exact. In topos theory, where adjunctions play a key role, we must therefore take natural adjunctions, not the non-natural adjunctions of Definition \ref{d.adjoint}, as our starting point. 

%So we see that a ``nonnatural'' adjunction can have useful properties. 

Let a functor $G:A \to X$ {\em have a left adjoint} if it fits into a (nonnatural) adjunction $(F,G,\eta,\epsilon)$ as in Definition \ref{d.adjoint}.
\end{comment}





%The forgetful functor to Set from a category $\sC$, when defined, is an example of a right adjoint functor. Its left adjoint, which takes a set to an object of $\sC$, is usually just the ``free'' construction that you would naively expect. For example, the left adjoint of the forgetful functor $F$ from Vect to Set, sending a vector space to its underlying set of vectors, and a linear map to the underlying function, is the functor $G$ given by sending a set to the vector space freely generated by the set (take the underlying set as a basis) and sending functions to the obvious linear maps. %This is in fact a left/right adjoint pair in the stronger sense we introduce in the next section. %(It is a good exercise to check this!) 



%\input{gencat4(nat)} % naturality
% file gencat4(nat).tex Created by Lucius Schoenbaum, November 20, 2016




\section{Naturality}\label{s.nat}

%In the previous sections, we have introduced essentially all the basic notions of category theory (with the exception of limits, see section \ref{ss.limits}). %as they appear in \cite{MacCW,EM1945,Kan58}. %With this section, we begin the study of extra features which breathe life into the basic framework. The first of these is naturality. 
%In this section, we study an extra condition on equivalence of generalized categories, that the morphisms $\theta_1$ and $\theta_2$ should depend only on the domain and codomain. This extra condition arises naturally in applications. With it, many new things can be done and statements can often be phrased in many different ways. %, giving category theory almost the quality of sorcery. %or black magic. %Naturality is one of the notions that breathes life into category theory. 

%[The basic observation is that you have weak and strong equivalence; thus weaker is going to let you prove that more categories are equivalent....... but it has the downside that it doesn't have the nice theory of adjoints that strong equivalence has. What does this mean precisely? I don't know yet, but I'm curious.]

In this section we establish the second of the two notions of equivalence we consider, namely
natural equivalence. %
As already noted, {\em the distinction between natural and non-natural vanishes in the case of categories.} %
Under natural equivalence, we obtain a 2-category of generalized categories, and in particular, an interchange law (Theorem \ref{t.functorcategory}). We can also establish, using the final lynchpin that naturality provides, the hom set bijection associated with adjoint pairs (Theorem \ref{t.nathombij}). Consequently the familiar rule that an equivalence between categories is given by a fully faithful essentially surjective functor carries over to generalized categories (Theorem \ref{t.ffes}). The full and faithful properties are tied to the naturality condition, which gives rise to maps not only on individual elements, but on entire hom sets. 

\dfn\label{d.nattrans}
Let $\sC, \sD$ be generalized categories, let $F,G: \sC \to \sD$. %
Let $(\theta_1, \theta_2) :F \Rightarrow G$ be a morphism of functors. %
We say that $(\theta_1, \theta_2)$ is {\em natural} or that $(\theta_1, \theta_2)$ is a {\em natural transformation} if, 
for every $a,b \in \sC$, %
$$\theta_1(a) = \theta_1 (b)$$
whenever $\hat a = \hat b$, and
$$\theta_2 (a) = \theta_2 (b)$$ 
whenever $\bar a = \bar b$.
\Dfn

Thus, naturality means that the function $\theta_1(a)$ can be replaced with the function $\hat a \mapsto \theta_1(1_{\hat a})$ of the element $\hat a$, and $\theta_2$ can be replaced with the function $\bar a \mapsto \theta_2(1_{\bar a})$ of the element $\bar a$. But, as noted in section \ref{s.elem}, 
$\theta_1(1_b) = \theta_2(1_b)$ 
for all elements $b$. 
Hence a natural transformation reduces to a single map $\theta: \sC \to \sC'$, from which $\theta_1$ and $\theta_2$ are immediately derived: 
$$\theta_1 (a) := \theta(1_{\hat a}),$$
$$\theta_2 (a) := \theta(1_{\bar a}).$$
%We will call this $\theta$ the {\em natural map} of the morphism of functors in this case. 
We refer to a natural transformation $(\theta_1, \theta_2)$ by referring to this map $\theta$. In terms of $\theta$ the defining relation of a morphism of functors becomes
$$\theta(\hat f \hspace{0.8pt}) \cdot F(f) = G(f) \cdot \theta(\bar f \hspace{0.8pt}) \, \downarrow.$$

\dfn\label{d.naturaleq}
Two generalized categories $\sC$ and $\sC'$ are {\em naturally equivalent} %or (adding an extra word) {\em naturally equivalent} % hush
 if they are equivalent via natural transformations
$$\theta:F\of G \iso \id_{\sC'},$$
$$\theta':G\of F \iso \id_\sC.$$
\Dfn

Naturally equivalent generalized categories are, in particular, equivalent (Definition \ref{d.equivalent}). %
With the extra condition of naturality, 
the way is clear to extend many 
justly well-known results of one-category theory \cite{MacCW} to the generalized setting: 

\thm\label{t.functorcategory}
The system given by all of the 
generalized categories, 
functors, 
and natural tranformations
forms a strict 2-category. % in which the morphisms of functors are the 2-cells, 1-cells are the functors, and the objects are the generalized categories.
\Thm
\prf
We define the products 
$$\theta_1 \vertof \theta_2,$$
$$\theta_1 \star \theta_2$$ 
just as in Chapter \ref{c.prelude}, and proceed as in the one-categorical case. %
\Prf

We include the naturality condition when defining adjoints: %

\dfn\label{d.adjunction}
Let $\sC$ and $\sD$ be generalized categories. An %(order-preserving) 
{\em adjunction} %$(F,G,\phi)$ or 
$(F,G,\eta,\varepsilon)$ is a pair of functors 
\vspace{-1ex}
$$
\begin{tikzcd}
	\sC \arrow[r, "F", shift left] \arrow[r, leftarrow, shift right, "G" below] & \sD \\
\end{tikzcd}\vspace{-3.5ex} 
$$
%Then $(F, G)$ is an {\em adjoint pair} if for every $f$ in $\sC$ and $g$ in $\sD$, there is a bijection of sets
%$$\hom(F(f), g) \iso \hom(f, G(g)),$$
%that is natural in $f$ and $g$. This means that ()()()
together with %(order preserving) 
natural transformations
$$
\eta: \id_\sC \to G\of F, \quad \varepsilon: F \of G \to \id_\sD,
$$
satisfying the identities
\begin{align}\label{eq.triid}
(G \of \varepsilon) \vertof (\eta \of G) &= 1_G, \\
(\varepsilon \of F) \vertof (F \of \eta) &= 1_F,
\end{align}
where $1_F$ is the mapping $f \mapsto 1_{F(f)}$. %
Given an adjunction $(F,G,\eta,\varepsilon)$, $\eta$ is called the {\em unit} and $\varepsilon$ is called the {\em counit} of the adjunction. 
A natural equivalence $(\theta, \theta')$ is an {\em adjoint equivalence} if $\theta$ and $\theta'$ are the unit and counit of an adjunction. 
\Dfn

\thm\label{t.nathombij}
Let $\sC, \sD$ be generalized categories, and let $F,G: \sC \to \sD$ be functors. The following are equivalent:
\enu
	\item $(F,G,\eta,\varepsilon)$ forms an adjunction 
$
\begin{tikzcd}
	\sC \arrow[r, "F", shift left] \arrow[r, leftarrow, shift right, "G" below] & \sD \\
\end{tikzcd}\vspace{-1.5ex} 
$.
	\item %Equivalent, given such an $F$ and $G$, 
%Then $(F, G)$ is an {\em adjoint pair} if 
For every $f$ in $\sC$ and $g$ in $\sD$, there is a %n order-preserving 
bijection of sets
\begin{equation}\label{eq.adjbij}
\hom(F(f), g) \iso \hom(f, G(g)),
\end{equation}
that is natural in $f$ and $g$. This means that if $\phi_{f,g}$ is the bijection (\ref{eq.adjbij}), then 
for every $k: g \to g'$, and $h:f' \to f$, the following diagrams commute: 
$$
\begin{tikzcd}
\hom(F(f),g) \arrow[r, "\phi_{f,g}"] \arrow[d, "k_{*}"] & 
\hom(f,G(g)) \arrow[d, "G(k)_{*}"] \\
\hom(F(f),g') \arrow[r, "\phi_{f,g'}"] &
\hom(f,G(g')) 
\end{tikzcd}
\quad\quad
\begin{tikzcd}
\hom(F(f),g) \arrow[r, "\phi_{f,g}"] \arrow[d, "F(h)^{*}"] &
\hom(f, G(g)) \arrow[d, "h^{*}"] \\
\hom(F(f'), g) \arrow[r, "\phi_{f',g}"] &
\hom(f', G(g))
\end{tikzcd}
$$
Equivalently $\phi$ satisfies
%$$f \cdot F(g) \text{ $\downarrow$} \eimplies \phi(f \cdot F(g)) = \phi(f) g,$$
%$$g' \cdot g \text{ $\downarrow$}, \eimplies \phi(g' \cdot g) = G(g') \phi(g)$$
%for all $f \in \sC, g,g' \in \sD$ such that $\bar g = F(x)$ for some $x \in \sD$. %
\[
u \cdot F(v): F(f) \to g \eimplies \phi(u \cdot F(v)) = \phi(u) \cdot v,
\]
\[
v' \cdot v: F(f) \to g \eimplies \phi(v' \cdot v) = G(v') \cdot \phi(v).
\]
%When $\phi$ is given, because of axiom \ref{ax.order1s} of definition \ref{d.gencat}, the unit and counit $\eta$ and $\varepsilon$ are both order-preserving. This implies that $\flr$ must be an object, if we require that $\eta(\flr) = \flr$ and similarly for $\varepsilon$. 
\Enu
\Thm
\prf
The proof is formally the same as in the one-categorical case (see \cite{MacCW}). 
\Prf
%\prf
%\stuffgoeshere
%are commutative. This reduces to $\phi(kf) = $ and $\phi(f Fh) = $. Suppose that $\sC$ and $\sC'$ are categories. Then, as noted above, $\eta_1 = \eta_2$ and $\varepsilon_1 = \varepsilon_2$. Hence, in the case of categories $X$ and $A$, an adjoint pair defines a bijection between hom sets that is natural in both $x$ and $a$. (Apply the triangular laws.) Conversely, every bijection between hom sets, natural in $x$ and $a$, defines an adjoint pair $(F,G,\eta,\varepsilon)$ in which both $\varepsilon$ and $\varepsilon$ have $\varepsilon_1 = \varepsilon_2$, $\eta_1 = \eta_2$. 
%\stuffgoeshere
%\dfn
%{\em universal pair of arrows} $(\lambda, u_1, u_2)$ ()()()
%\stuffgoeshere
%\Dfn
%\stuffgoeshere
%\Prf

\dfn\label{d.faithfulfull}
Let $\sC,\sD$ be generalized categories, $F: \sC \to \sD$ a functor. 
For $a,b \in \sC$, let $F_{a,b}$ be the mapping on the domain $\hom(a,b)$ given by $f \mapsto F(f)$. %
We say that $F$ is {\em faithful} if %
for all $a,b$, $F_{a,b}$ is injective, 
and 
we say that $F$ is {\em full} if %
for all $a,b$, $F_{a,b}$ is surjective. 
\Dfn

Thus for example full means: if $\alpha, \beta$ in $\sD$ are of the form $F(a), F(b)$, for $a,b \in \sC$, and if $\gamma: \alpha \to \beta$, then $\gamma$ is of the form $F(c)$ for $c \in \sC$. 

\thm\label{t.ffes}
Let $\sC, \sD$ be generalized categories, and let $F: \sC \to \sD$ be a functor. The following are equivalent:
\enu
	\item $F$ is a natural equivalence,
	\item $F$ is a natural adjoint equivalence,
	\item $F$ is full, faithful, and essentially surjective.
\Enu
\Thm
\prf
The proof, much the same as in the one-categorical case, is left to the reader. %\cite{MacCW}. 
\Prf


























%\input{gencat5(lim)} % limits
% file gencat5(lim).tex Created by Lucius Schoenbaum, November 20, 2016





\section{Limits}\label{s.lim}

In this section we establish the elements of the theory of limits and colimits in sharp generalized categories. 
%The reader can consult, for example, \cite{MacCW,KaSc1} for further details (without, of course, the extra generality adopted here). We note that this entire section contains no surprises for the reader who is already familiar with basic facts about limits in categories. Such a reader might like to skim over this section and go on to section \ref{c.presheaf} on presheaves over generalized categories. 
We consider limits with respect to mappings $I \to \sC$ as in Definition \ref{d.limit} that are weaker than functors. This, for example, allows us to form the shape of a product or coproduct of any set of elements in a generalized category. %but we will have no need of this in this work. %

\dfn\label{d.limitfunctor}
Let $\sC, \sC'$ be generalized categories. A {\em functor up to objects} from $\sC$ to $\sC'$ is a map $F:\sC \to \sC'$ satisfying, for every $a,b \in \sC$,
\enu
	\item $F(ab) = F(a)(b)$,
	\item $F(a)$ is an identity in $\sC'$ if and only if $a$ is an identity in $\sC$,
	\item $F(\source(a)) = \source(F(a))$ unless $a$ is an object of $\sC$,
	\item $F(\target(a)) = \target(F(a))$ unless $a$ is an object of $\sC$. 
\Enu
\Dfn
%\exa
%The map defined above from a generalized category to its isoskeleton is a limit functor.
%\Exa
%The name limit functor is just meant to be short for, ``functor used for limits''. An example of a limit functor is an ordinary functor. In particular, the notion of limit functor and ordinary functor coincide if $\sC$ and $\sC'$ are categories. %On a first reading, the reader can view the material in this section replacing limit functors everywhere with ordinary functors. The enlarged class of limit functors is (essentially) the class which behaves just like a functor, except it need not send objects to objects---more precisely, it need not send ``half-objects'' (like $\sous{b}a_a$ or $\sous{a}a_b$) to half-objects. 
%\end{comment}

\dfn\label{d.cone}
Let $\sC$ be a generalized category, $I$ a generalized category (the index of a cone needs only be a set, but in practice it is always a (generalized) category). A {\em cone} in $\sC$ with index $I$ is a map $\sigma:I \to \sC$ such that 
%$$\bar \sigma$, defined as $\set{\overline{\sigma(i)}}_{i \in I}$ is a well-defined element of $\sC$. 
$$\text{for all $i,j \in I$, } \overline{\sigma(i)} = \overline{\sigma(j)}.$$ %
Dually, {\em cocone} in $\sC$ with index $I$ is a map $\sigma:I \to \sC$ such that %$\hat \sigma$ is defined, in the manner similar to cones. 
for all $i,j \in I$, $\widehat{\sigma(i)} = \widehat{\sigma(j)}.$ %
%A {\em bicone} is a cone $\sigma:I \to \sC$ which is also a cocone. %
A cone or cocone is {\em finitely generated} if the index set $I$ is finitely generated (Definition \ref{d.figen}). %
This common source is the {\em vertex} of the cone, and the {\em vertex} of a cocone is the common target. %
%Finite cones, cocones, or bicones may sometimes be denoted $(a_1, a_2, \dots, a_n)$, for all the $a_i$ in $\sC$. %
Given a cone or cocone $\pi$, we may refer to $\pi(i)$ for some $i \in I$ as a {\em member} of the cone. %
\Dfn

\dfn\label{d.limit}
Let $\sC, I$ be generalized categories. Let $\alpha: I \to \sC$ be a functor, possibly only a functor up to objects. %
A cone is said to be {\em over (or below) the base $\alpha$} if %
\enu
	\item $\widehat{\pi(i)} = \alpha(i)$, for all $i \in I$, %
	\item for all $i \in I$, $\pi(\hat i) = \alpha(i) \pi(\bar i)$. %
\Enu
A {\em limit} of $\alpha$ is a cone $\pi:I \to \sC$ below the base $\alpha$ such that for any cone $\tilde \pi:I \to \sC$ over the same base $\alpha$, there is a unique $\lambda \in \sC$ such that $\tilde \pi = \pi \vertof \lambda$. (Here, $\pi \vertof \lambda$ is the map defined by $(\pi \vertof \lambda)(i) = \pi(i) \cdot \lambda.$) %\footnote{(((here we are using $\evalu$, the evaluation operation, but not the one I discuss below---not the formal evaluation map introduced in a category or generalized category such as $R$-Mod, but rather, a part of our foundational syntax/language, namely, {\em evaluation} (!) of a function. (Cf. section \ref{c.rmod})}

Dually, a cocone is said to be {\em over (or below) the base $\alpha$} if 
\enu
	\item $\overline{\pi(i)} = \alpha(i)$, for all $i \in I$,
	\item for all $i \in I$, $\pi(\bar i) = \pi(\hat i) \alpha(i)$.
\Enu
A {\em colimit} of $\alpha$ is a cocone $\pi:I \to \sC$ such that for any cone $\tilde \pi:I \to \sC$ over the base $\alpha$, there is a unique $\lambda \in \sC$ such that $\tilde \pi = \lambda \vertof \pi$. Here, $\lambda \vertof \pi$ is the map defined by $(\lambda \vertof \pi)(i) = \lambda \cdot \pi(i)$, as before. 
\Dfn

Thus a cone fits a pattern as in the following Figure:
$$
\begin{tikzcd}[column sep=1.5em]
\alpha(\bar i) \arrow{rr}{\alpha(i)} && \alpha(\hat i) \\
 & \text{(vertex)} \arrow{ul}{\pi(\bar i)} \arrow{ur}[swap]{\pi(\hat i)} 
\end{tikzcd}
$$

%The cone/cocone structure can be visualized, in general, as in Figure (\ref{f.limit1}). We have added an extra curved line connecting the members of the cone/cocone. This merely emphasizes that they are grouped together. This symbolism reduces to a symbolism that is used by many authors, sometimes formally, to denote a pullback diagram (see below) in that particular case. 

The word limit is often used to refer to the domain of the cone, and similarly colimit is used to refer to the codomain of the cocone. %On the other hand, there are many situations where consternation is instead caused by using the other convention, referring to the limit as the cone. It is best to rely primarily on visualizations of the limit structure (along with the logical relations spelled out in the definition) instead of terminology. However, 
%When we refer to limits/colimits in this section, the term {\em limit}/{\em colimit} refers always to the cone/cocone $\pi$. 
The terms {\em product}, {\em equalizer}, {\em coproduct}, {\em coequalizer}, etc. retain their meaning from ordinary categories, referring to limits based on diagrams $\alpha: I \to \sC$ of the same shape as in the one-categorical case, and where $\alpha$ may be a functor only up to objects. %
We follow standard terminology and say that a generalized category {\em has finite limits} if there is a limit cone for every finitely generated diagram $\alpha: I \to \sC$, and dually for colimits. 
%$$pic$$

%\begin{figure}[t]\label{f.limit1}
%\centering
   %\pic[width=3in]{limit1.jpg}
%\end{figure}

We denote the set of limits of the functor $\alpha:I \to \sC$ by $\lim(\alpha,I)$ or just $\lim \alpha$. We denote the colimit $\colim(\alpha,I)$ or simply $\colim(\alpha)$. %
%\section{Examples of Limits in General generalized categories: Terminology}
% include?
%Let $a \cross b$ denote (up to isomorphism) a domain of $\prod(a,b) \,\downarrow$. If $a \cross (b \cross c) \,\downarrow$, then $a \cross (b \cross c)$ is isomorphic to $(a \cross b) \cross c$. This isomorphism is natural in $a,b,c$.

If $\sC$ is a generalized category, there exist (finitely generated) diagrams $J \to \sC$ that cannot be defined and do not exist in an ordinary category. 
However, we still have:
%for example
%$$ pic$$
%For this reason, the following result is surprising, though its proof is a direct repetition of the proof in the one-categorical case:

\thm\label{t.prodeqsuffice}
Let $\sC$ be a generalized category. For $\sC$ to have all finite limits, it suffices that $\sC$ has all finite products and equalizers. 
\Thm
\prf
%We proceed by induction on the height of elements of the diagram $I$ (section \ref{s.gencat}). %since $I$ is finite. 
%For $i \in I$, 
%let $\alpha: I \to \sC$ be a finitely generated diagram on $\sC$. %
We proceed by induction on the height of finitely generated diagrams $\alpha: I \to \sC$. 
A finitely generated diagram of height $0$ is a finite product, hence it has a limit cone in $\sC$ by hypothesis. %
%For such a diagram $\alpha$ and for $k = 0,1,\dots$, let $\alpha^k$ be $\alpha$ restricted to elements of $I$ of height $k$. %
%Then $\alpha^0$ is a finite product in $\sC$, thus has a limit cone. %
%Suppose that $\alpha^{k}$ has a limit cone $\sigma^{\leq k}$. %we show that $\alpha^{k+1}$ also has a limit cone. %
Suppose that all finitely generated diagrams of height $k \geq 0$ have a limit cone, and 
let $\alpha: I \to \sC$ be a diagram of height $k + 1$. %
Define 
$$\alpha^{\leq k}$$
to be $\alpha$ restricted to the generalized category $I^{\leq k}$ formed by taking the collection of all elements of $I$ of height $\leq k$, along with all identities of $I$. 
It is easy to see that $I^{\leq k}$ is closed under composition, thus it is a generalized category. Therefore $\alpha^{\leq k}$ is a diagram on $\sC$, and by hypothesis, has a limit cone $\sigma^{\leq k}$ with vertex, say, $L^{\leq k}$. %
%Let $\sigma^{k}$ be the limit cone formed by taking the product on the set $I^k$ of elements of $I$ of height precisely $k$. %
%Then since there is a subcone of the cone $\sigma^{\leq k}$ that is a cone for $I^k$, there is a universal arrow %
%$u_1$ in $\sC$ from $\overline{\sigma^{\leq k}}$ to $\overline{\sigma^{k}}$, the vertices of the respective cones. %
%Now let $\sigma^{k+1}$ be the limit cone formed by taking the product on the set $I^{k+1}$ of elements of $I$ of height precisely $k+1$. 
%
% back up, one little adjustment...because the source/target of something of height k+1 is not necessarily something of height k.
%
%Now let $\sigma^{\leq k, prod}$ be the limit cone of the set-flattening (section \ref{s.elem}) $flat(I^{\leq k})$ of $I^{\leq k}$. 
%Then using $\sigma^{\leq k}: I^{\leq k} \to \sC$, a cone may be constructed $\tilde \sigma^{\leq k}: $
% ()()() lost my train of thought....
%
Consider $flat(I^{\leq k})$, the flattening of $I^{\leq k}$ to a zero-category (section \ref{s.elem}). %
The diagram $flat(\alpha^{\leq k}): flat(I^{\leq k}) \to \sC$ induced by $\alpha^{\leq k}$ %
is a diagram of height zero, so it has a limit cone $\sigma^{\leq k, flat}$, with vertex, say, $L^{\leq k, flat}$. 
The cone $\sigma^{\leq k}$ on $I^{\leq k}$ induces a cone on $flat(I^{\leq k})$, 
so there exists a universal arrow 
$$u_1: L^{\leq k} \to L^{\leq k, flat}.$$
Now let $I^{k+1, flat}$ be the flattened (to a zero category) elements of $I$ of height $k+1$. %
The diagram $\alpha$ induces a diagram $\alpha^{k+1, flat}$ on $I^{k+1, flat}$, %
defined by 
$$\alpha^{k+1, flat} (i) := \target(\alpha(i)).$$
This diagram (of height zero) has a limit cone $\sigma^{k+1, flat}$ 
with vertex, say, $L^{k+1, flat}$. %
For $i \in I$ of height $k+1$, let $\pi_i$ be the element in $\sC$ which is the projection 
$$\pi_i : L^{\leq k, flat} \to \target(\alpha(i)),$$
coming from the diagram $\sigma^{\leq k, flat}$ on $I^{\leq k, flat}$ (where our notation hides this fact about $\pi_i$). 

The previous cone $\sigma^{\leq k, flat}$ with vertex $L^{\leq k , flat}$ itself has projection arrows to the elements $\target(\alpha(i))$ 
as $i$ ranges over $\alpha^{k+1, flat}$. 
Therefore, there is a universal arrow 
$$u_2: L^{\leq k, flat} \to L^{k+1, flat}.$$
Moreover, for each $i$ of height $k+1$, there is also a projection arrow to the element $\source(\alpha(i))$, and 
composing each of these projection arrows with $\alpha(i)$ gives a second cone with the same vertex $L^{\leq k, flat}$ %
on the diagram $\alpha^{k+1, flat}$. 
So we may again find a universal arrow
$$u_3: L^{\leq k, flat} \to L^{k+1, flat},$$
by applying the universal property of the limit with vertex $L^{k+1, flat}$ a second time. %
%Since $u_2$ and $u_3$ are parallel, we compose them with 
We compose $u_2$ and $u_3$ with $u_1$ to form parallel arrows, and take the equalizer:
\[
\begin{tikzcd}
L \arrow[r, "e"] &
L^{\leq k} \arrow[r, "u_{1}"] &
L^{\leq k {,} flat}
\arrow[r, shift left, "u_{2}"]
\arrow[r, shift right, swap, "u_{3}"] &
L^{k+1 {,} flat}
\end{tikzcd}
\]
Now we define, for $i$ in $I$ of height $\leq k+1$,
$$\sigma^{\leq k+1} (i) := \pi_i \cdot u_1 \cdot e.$$
We claim that this is a limit cone for the diagram $\alpha^{\leq k+1}: I^{\leq k+1} \to \sC$. 
%IS A CONE: 
Since we pass through $e$ to reach $L^{\leq k+1}$, $\sigma^{\leq k+1}$ satisfies 
$\sigma^{\leq k+1} (\hat i) = \alpha^{\leq k+1} (i) \cdot \sigma^{\leq k+1} (\hat i)$, 
hence is a limit cone. 
%UARROW: 
Suppose that 
$\tilde \sigma^{\leq k+1}: I^{\leq k+1} \to \sC$ is a diagram with vertex, say, $\tilde L$ satisfying 
$\tilde \sigma^{\leq k+1} (\hat i) = \alpha^{\leq k+1} (i) \tilde \sigma^{\leq k+1} (\bar i).$ 
Then $\tilde \sigma^{\leq k+1}$ restricts to a cone on $\alpha^{\leq k}$, hence there is a universal arrow
$$\tilde e: \tilde L \to L^{\leq k}.$$
Because $\tilde \sigma^{\leq k+1}$ has the limit property even at the height $k+1$, $\tilde \sigma^{\leq k+1}$ satisfies $u_2 \cdot u_1 \cdot \tilde e = u_3 \cdot u_1 \cdot \tilde e$, and thus $\tilde e$ factors through $e$ uniquely, as desired. 
\Prf


\dfn
Let $F: C \to C'$ be a functor. Then $F$ {\em preserves limits} or is {\em left exact} if for every functor $\alpha:I \to C$, 
$$F(\lim(\alpha)) \lies \lim(F\of \alpha).$$
Dually, $F$ {\em preserves colimits} or is {\em right exact} if for every functor $\alpha:I \to C$,
$$F(\colim(\alpha)) \lies \colim(F\of \alpha).$$
$F$ is said to {\em create limits} if for every element $\pi \in \lim(F \of \alpha)$, there exists a unique $\pi' \in \lim(\alpha)$ such that $F(\pi') = \pi$. Dually, $F$ is said to {\em create colimits} if for every element $\pi \in \colim(F \of \alpha)$, there exists a unique $\pi' \in \colim(\alpha)$ such that $F(\pi') = \pi$.%\footnote{In \cite{MacCW} $F$ is said to {\em create limits/colimits}. This terminology is just too spooky.}
\Dfn

For example, the hom functor 
$$b \mapsto \hom(-,b)$$
preserves limits. %Indeed, ()()(). 
Dually, the contravariant hom functor 
$$a \mapsto \hom(a,-)$$
preserves colimits. These functors may be extended to generalized categories.  %\cite{ScMd}. 

%We close this paragraph by recording another basic statement about Set for reference purposes. It will be explained/proven in the next paragraph.

% "Switching the Order in Set"
%\prop\label{p.setlimitswitch}
%Let $P$ be finite, $J$ filtrant. Then for every functor $P \cross J \to \text{Set}$, the canonical arrow $\kappa$ is an isomorphism. Consequentially, for every finite limit and filtrant colimit in Set, the order of the limits can be switched. [need to work on this] \qedhere
%\Prop

%\subsection{More Statements About Limits}

%In this paragraph, we study the existence and preservation of limits, and when their order can be switched.

%It is natural to be interested in whether, in any situation, the limit/colimit can always be guaranteed to exist. 

% "Pair-Finite"
%\prop
%Let $C$ be a generalized category, $S \lies C$. If $C$ has a terminal object and the product $\prod(a,b)$ is defined for all $a,b \in S$, then $C$ has all finite product $\prod(a_1,\dots,a_n)$ for $a_1,\dots,a_n \in S$. 
%\Prop
%\prf
%()()()
%\Lem
%()()()
%\Prf

%Intuition suggests that a useful property of a generalized category is that it always has limits. Thus one could investigate the {\em completeness} of generalized categories, that is, the property that all (small) limits exist. The following argument due to Freyd shows that this is a strong condition. Suppose that $C$ is a complete category, that is, it has all small limits (in a suitably ``large'' universe). Choose an index $I$ with the cardinality strictly greater than the cardinality of $C$. Suppose that there exist $f,g \in C$ with ()((). Then ()()() This is a contradiction. Therefore $C$ must be a {\em preorder}, that is, hom sets in $C$ must be either empty or contain only one element. 

%Another natural question is whether, given some functor, it should preserve limits or not.
\thm\label{t.hlalex}
%\stuffgoeshere % write statement
Let $F: \sC \to \sD$ be a functor between generalized categories $\sC$ and $\sD$. 
Then if $F$ has a left adjoint $G: \sD \to \sC$, then it is left exact. 
%Has natural right adjoint, preserves colimits.
\Thm
\prf
Like the proof for categories, the proof for generalized categories relies on naturality of the adjoints via the bijection (\ref{eq.adjbij}). 
\Prf

The dual statement to \ref{t.hlalex} is immediate: a functor with a right adjoint is right exact. %
% Freyd adjoint functor theorem???? talk about it????

\begin{comment}
In the following statement, we say that an {\em algebraic category} is a category defined via unary, binary, ..., $n$-ary operations defined on a set, and structure-preserving maps (homomorphisms). 

\prop
Let $A$ be an algebraic category. The forgetful functor to $\Set$ defines filtrant colimits. Therefore colimits exist in $A$.
\Prop
\prf
\stuffgoeshere
\Prf
\end{comment}

\begin{comment}
\thm
Given a cofinal functor $L:J \to I$ and a functor $\alpha:I \to C$, if 
$$\colim(\alpha \of L) \,\downarrow,$$
then 
$$\colim(\alpha) \,\downarrow,$$
and 
$$\colim(\alpha) \iso \colim(\alpha)$$
via an explicit isomorphism.
\Thm
\prf
\stuffgoeshere
\stuffgoeshere
\Prf

\end{comment}

%\section{Exponentials and Cartesian Closed Structure}

%This subject is treated in chapter \ref{c.th}. %, section \ref{s.th}. 

%\section{Ends, Coends, Kan Extensions}\label{s.kanx}

%say that Kan extensions are basically just universal arrows
%\stuffgoeshere




%\input{gencat1(globular)} % globular sets
% File gencat1(globular).tex Created by Lucius Schoenbaum November 22, 2016
% globular sets and gencats








\section{Globular Sets}\label{s.globularsets}

It is worthwhile to remark on the relationship between generalized categories and globular sets. Recall that
a globular set is a presheaf of shape $\mathbb{G}$ (that is, a functor $\mathbb{G}^{\text{op}} \to \Set$), where $\mathbb{G}$ is the category of natural numbers $n \geq 0$ together with maps
$$
\begin{tikzcd}
0 \ar[r, shift left=1ex, "\sigma_{0}"] \ar[r, shift right=1ex, "\tau_{0}"']
	& 1 \ar[r, shift left=1ex, "\sigma_{1}"] \ar[r, shift right=1ex, "\tau_{1}"']	
	& 2 \ar[r, shift left=1ex, "\sigma_{2}"] \ar[r, shift right=1ex, "\tau_{2}"']
	& ...
\end{tikzcd}
$$
subject to the relations 
$
\sigma_{i+1} \of \sigma_{i} = \tau_{i+1} \sigma_{i}, \quad \tau_{i+1} \of \tau_{i} = \sigma_{i+1} \of \tau_{i},
$
for $i \geq 0$.

%Next, we define:

\dfn
Let $\sC$ be a generalized category. A {\em $k$-cell} in $\sC$ is an element $f$ of $\sC$ such that for every $k$-element sequence $\vec s$ of operations $\source$ and $\target$ that satisfy when applied to $f$, %where $\source$ and $\target$ are the source and target operators of $\sC$, one has
%$$sSf = Sf, \quad tSf = Sf.$$
\enu
	\item $\source^k f$ and $\target^k$ are objects, and $\source^{k-n} f$ and $\target^{k-n}$ are not objects, for all $0 \leq n \leq k$, 
	\item $\source \target f = \source \source f$ and $\target \source f = \target \target f$,
	\item $\source f$ and $\target f$ is are $(k-1)$-cells.
\Enu
% ALTERNATIVE: TAKE THE SET OF ALL GENERALIZED CATEGORIES THAT SATISFY THESE AXIOMS, THESE ARE CELLULAR GENCATS: 
%\enu
%	\item $\source\target = \source\source, \target\target = \target\source$ %(where \source,\target denote source and target, usual prefix notation)
%	\item for all elements $a$, there is $n \geq 0$ such that $\source^n a = \source^{n+1} a$. 
%	\item for all $n$, $\source^n = \source^{n+1}$ if and only if $\target^n = \target^{n+1}$,
%	\item for all $m,n$, $\source^{m+n} = \source^n$ implie\source $\source^{n+1} = \source^n$.
%\Enu
%For example, in a 1-dimensional category, all elements are 1-cells, and some elements are also 0-cells. 
An element $f$ of a generalized category $\sC$ is {\em cellular} if $f$ is a $k$-cell for some $k \geq 1$, %
and a generalized category $\sC$ is {\em cellular} if every element of $\sC$ is cellular. 
%!!!!!!!!!!!!!!!!!!!! % I think you want to insist that you have a kind of balanced situation in the cellular case. That is, the recursion is triggered at a well-defined height in the descending tree. But is this really necessary, or am I just biased? 
\Dfn

%We can now state:

\prop
There is an equivalence (given by a forgetful-free adjunction) between sharp, cellular generalized categories and the category of globular sets. 
\Prop
\prf
%\prop\label{p.adjunction} 
%There is an adjunction between a subcategory of the category of GCs and globular sets.
%\Prop
To prove this, we must be sure clarify the statement: when referring to sharp, cellular generalized categories, we refer not to the full subcategory but to the category whose morphisms $F: \sC \to \sD$ are subject to the extra condition 
%Consider the conditions on functors $F: \sC \to \sD$
\enu
	\item for all $a \in \sC$, $s(F(a)) = F(a)$ implies $s a = a$.
\Enu
This says we cannot map $k$-cells for $k > 0$ to $0$-cells. %
Then let $\dim(a) := \min\set{n \mid \source^n a = \source^{n+1} a }$. %
%Let $\sG$ be the category of globular sets. %
Define a mapping 
$$\sC \mapsto (n \mapsto \set{a \in \sC \mid \dim a = n}).$$ %
to the category of globular sets, for a sharp cellular generalized category $\sC$. This is the desired equivalence. 
%Then this mapping goes from $\Ob(For(\fD))$ to $\Ob(\sG)$, where $For(-)$ is the forgetful functor dropping all composition operations. 
%\stuffgoeshere
\Prf

Examples of noncellular generalized categories are abundant, for example arising from the theory of trees and related notions, see for example \cite{ElBlTi1}. 
 %!!!!!!!! say more - parse trees, iterative equations, ....


























%\input{gencat6(conc)} % conclusion
% file gencat6(conc).tex Created by Lucius Schoenbaum, November 20, 2016





\section{Conclusion}\label{s.gencat-conc}

%We have carried out an investigation of assumptions about the basic notions in category theory, 
%with the motivation being to find a stable and sufficiently rich generalization. %
%From now on, we will adopt the generalization that this work suggests, and when we refer to left or right adjoints
%This concludes our investigation of the foundations of category theory, though we have not made mention of some of the more advanced parts of basic theory, such as ends, coends, and Kan extensions. %
There are numerous advanced notions of category theory that have not yet made an appearance in our development, 
for example, ends, coends, monads, Kan extensions, to name only a few. %Some of these are treated in the generalized setting in \cite{ScMd}. %Some of these will be approached in subsequent work, while others shall remain a task for future work. 
%
Our investigation has yielded the following observations: %
There exists a generalization of category theory. %he devices and results of category theory apply. 
More precisely, there exists a theory of functors, natural transformations, adjoint pairs, limits, and colimits for generalized categories. %
Still more precisely, there are two generalizations that are combined into one larger one:  %The generalization can be carried out along two dimensions: by introducing an order and 
First, by allowing an approximate operation of composition (i.e., proximal categories),
and second, by allowing generalized higher cells. 
We have seen that the structure of limits and natural transformations is similar to the structure as it arises in ordinary one-categories, 
so that, surprisingly, perhaps, the proximal structure has little effect on aspects of the theory. %
%(For an instance where this is less the case, see \cite{ScM2}). %
%			\item The main structures of category theory generalize (natural transformations, monads, 
%	\item Attempts have been made to push this generalization further, but they result in algebraic systems that appear too weak to be of interest. 
We have investigated a notion of non-natural transformation suggested by the one-categorical case where naturality is not a necessary assumption, and we have found that the device of non-natural equivalence is not sufficiently strong. Therefore, we have argued that naturality must be an explicit assumption in the generalized setting. 
%	\item The device of \stuffgoeshere does not give rise to a strict 2-category. 
Thus, we have both extended the boundaries of category theory, and made note of some limits to further extensions of the new boundaries we have drawn, which strengthens the case for our particular approach. %
With the foundations developed here, it is possible to go further. 
% the generalized setting of topics such as, for example, the algebraic theory of monads, the logic-driven theory of cartesian closed categories, and the sheaf-theoretic domain of topos theory. 




























%\appendix
%\input{gencat1(prelude)} % prelude
\pagebreak
\singlespacing
\chapter{Monads and Generalized Categories}\label{c.genm}
\doublespacing
\vspace{10ex}
			%\input{genm}
%\input{genm1(i)} % introduction
% File genm1(i).tex Created by Lucius Schoenbaum August 23, 2016
% genm - introduction






















\section{Introduction}\label{s.genm-i}

%The main purpose of this  is intended to serve as a brief introduction to generalized categories. 
The monad abstraction, which arose out of pure mathematics, has proven to be an enduring and versatile notion through a wealth of %ways it has found
 applications to algebra, logic, and computer science in recent %the past several 
 decades. %
Generalized categories have been introduced due to their inherent interest as mathematical objects---they are a class of abstractions far more general than ordinary categories, and yet, ordinary category theory continues to apply to them---and due to their own potential applications to computer science, where they are suggested as a model for higher order programming, in particular higher-kinded types. %
Therefore it is natural to place these two notions, the monad abstraction and the generalized category, side by side to see what can be said about their relationship. %That is the purpose of this short report. %

%\subsection{Related Work}
Order-enriched categories (suitable ones, with the property of $\omega$-completeness) are the foundation of the categorical domain theory developed by Smyth, Plotkin, Wall and others \cite{SmPl1,WaD1}. A special case of a generalized category is an order-enriched category, thus their work carries over to the present setting; one of our goals is to further clarify this relationship. The motivation for going to a further level of generality is twofold: first, the theoretical material is in fact well-behaved in the wider setting, indicating that it is appropriate to do so---in other words, generalized categories have an associated mathematical theory---and second, the observation that combining type coercion with the order of approximation (the ordering motivating the study of order-enriched categories in the context of domain theory) yields a transtive relation, opening a door to applications. Another door to applications is via the Curry-Howard-Lambek correspondence and the categorical semantics of Moggi. %, which we will only briefly mention here. 


%The relationship between generalized categories and domain theory is, at present, not well understood, and not well developed. In the context of the applications envisioned for generalized categories, in particular the model for thought provided by the theory of graduated types (Section \ref{ss.gtyt}), it can be said that domain theory are to generalized categories as affine spaces are to manifolds. (Whether that analogy has any punch to it, however, remains to be seen.) 











































 

			%\input{genm2(gencat)} % gencat
%\input{genm3(mon)} % monads
% File genm3(mon).tex Created by Lucius Schoenbaum August 23, 2016
% monads





















\section{Monads in Generalized Categories}\label{s.mon}

\dfn\label{d.monad}
Let $\sC$ be a generalized category. A {\em monad} on $\sC$ is a structure $(T,\eta,\mu)$, where $T: \sC \to \sC$ is a functor, and $\eta$ and $\mu$ are (order-preserving) natural transformations $\id_\sC \to T$ and $T^2 \to T$, respectively, such that the following hold:
\enu
	\item $\mu \vertof (T \of \mu) = \mu \vertof (\mu \of T)$ \label{ax.monmagic1}
	\item $\mu \vertof (T \of \eta) = \mu \vertof (\eta \of T) = 1_T$, \label{ax.monmagic2}
\Enu
where $1_T$ denotes the mapping $f \mapsto 1_{T(f)}$. %
A monad is said to satisfy the {\em monic condition} \cite{MoI1} if
for all $x,y$ in $\sC$, 
$$\eta(\hat x) x \dleq \eta(\hat y) y \eimplies x \dleq y.$$
It follows that $\eta(x)$ is a monic for all $x \in \sC$. %
\Dfn

A monad is often referred to %({\em par abus}) 
simply by the symbol for the functor $T$, with $\eta = \eta_T$, $\mu = \mu_T$ understood. 

\exa\label{x.mon}
Consider the generalized category $\Set f$, that is, the construction $\sC f$ of Example \ref{x.setf} applied to the category of sets. If we fix a group $G$, the usual monad on $\Set$ generated by $G$ extends directly to $\Set f$, giving a monad on $\Set f$: 
$$T(\ff) = \id_G \times \ff$$
$$
\eta(\ff) = \text{ the tree $\fh$ with $\tleft(\fh) = T(\ff)$, $\tright(\fh) = \ff$, $\troot(\fh) = x \mapsto (1_G, x)$ }
$$
$$
\mu(\ff) = \text{ the tree $\fh$ with $\tleft(\fh) = T^2 (\ff)$, $\tright(\fh) = T(\ff)$, $\troot(\fh) = (g,h,x) \mapsto (gh, x)$ }.
$$
%\stuffgoeshere
\Exa

%\subsection{The Kleisli Construction}\label{ss.mon}

Now we establish some background needed for the Kleisli contruction: %is still present in the generalized setting: 
a generalized category equipped with an adjunction is also equipped with a monad. 

\prop\label{p.monadj}
Let $\sC$ be a generalized category, and let $(F,G,\eta,\epsilon)$ be an (order-preserving) adjunction on $\sC$. Then there is a monad on $\sC$ given by $T = G \of F$ with $\eta_T = \eta, \mu_T = G \of \epsilon \of F$.
\Prop
\prf
The proof is standard, but the definitions are not. %
The unit $\eta$ and counit $\epsilon$ are indeed order-preserving natural transformations. Hence so is $\eta_T$ and $\mu_T$, which we denote $\eta$ and $\mu$ for the rest of the proof. 
We still have horizontal composition $\hozof$ given by $\beta \star \alpha = (\beta \of \bar{\alpha}) \vertof (\hat{\beta} \of \alpha)$, and the interchange law. These imply (a sharp expression)
$$\epsilon \hozof \epsilon = \epsilon \vertof (F \of G \of \epsilon) = \epsilon \vertof (\epsilon \of F \of G).$$
So applying $G$ and $F$ on the left and right side respectively, 
\begin{align*}
G \of (\epsilon \vertof (F \of G \of \epsilon)) \of F 
	&= G \of (\epsilon \vertof (\epsilon \of F \of G) ) \of F
\end{align*}
and so
\begin{align*}
(G \of \epsilon \of F) \vertof (G \of F \of G \of \epsilon \of F)
	&= (G \of \epsilon \of F) \vertof (G \of \epsilon \of F \of G \of F),
\end{align*}
or $\mu \vertof (T \of \mu) = \mu \vertof (\mu \of T).$ %
We need to check if $\mu \vertof (T \of \eta) = \mu \vertof (\eta \of T) = 1_T.$
This follows similarly from the triangular identities. 
%\stuffgoeshere 
% (G \of \epsilon \of F) \vertof (G \of F \of \eta) = G \of 1_F = 1_{G \of F}
% (G \of \epsilon \of F) \vertof (\eta \of G \of F) = 1_G \of F = 1_{G \of F}.
%check that the triangular identities follow if you define via $\phi$
%\stuffgoeshere
Therefore $(T, \eta, \mu)$ is an (order-preserving) monad. 
\Prf

We will now verify that, as in the ungeneralized case, every monad comes from an adjunction. First, the Kleisli construction may be extended to the generalized setting:

\dfn[Kleisli Construction]\label{d.kleisliconst}
If $\sC$ is a generalized category, and $T = (T,\eta, \mu)$ is a monad on $\sC$, set
$$\sC_T := \set{(y,f) \mid T(y) = \hat f, \,(\star)\,}$$
where $(\star)$ is a condition (detailed in the proof) that may be safely ignored on a first reading, and
\begin{equation*}
\begin{cases}
	\, \overline{(y,f)} = (\hat{\bar{f}}, \eta(\hat{\bar{f}})\bar f), &\\
	\, \widehat{(y,f)} = (\hat y , \eta(\hat y) \cdot y), &\\
	\, (z,g) \cdot (y,f) = (z, \mu(z) \cdot T(g) \cdot f), \hspace{20pt}\text{ if } \bar g \dleq y.  &\\
\end{cases}
\end{equation*}
%$$(z,g) \cdot (y,f) = 
%	\begin{cases} 
%		\flr, 					& \bar g \not\dleq y, \\
%		\mu(z) \cdot T(g) \cdot f,	& \bar g \dleq y.
%	\end{cases}
%$$
%\end{align*}
\Dfn

\thm\label{t.kleisli}
Let $\sC$ be a generalized category with identities. Then if $T$ satisfies the monic condition, then $\sC_T$ is a generalized category (not necessarily with identities). There exists an order-preserving adjunction
$$
\begin{tikzcd}
	\sC \arrow[r, "F", shift left] \arrow[r, leftarrow, shift right, "G" below] & \sC_T^0 \\
\end{tikzcd}\vspace{-3.5ex}
$$
where $\sC_T^0$ is the image of $F$ in $\sC_T$, such that the monad generated by $(F,G,\eta, \epsilon)$ is $T$. 
\Thm
\prf
Define a poset structure by setting
$$(z,g) \dleq (y,f) := (z \dleq y, g \dleq f).$$
%This gives a poset. % with bottom $(\flr, \eta(\flr)\flr)$. 
Now suppose that $(z,g), (y,f)$ are two elements of $\sC_T$ such that $\overline{(z,g)} = \widehat{(z,g)}.$ Then 
$$(\hat{\bar{g}}, \eta(\hat{\bar{g}}) \bar g) = (\hat y, \eta(\hat y) y).$$
So $\eta(\hat z) \bar g = \eta(\hat y) y.$ By the monic condition, 
$$\bar g = y.$$ 
So we have
\begin{align*}
((w,h)\cdot (z,g)) \cdot (y,f) 
	&= (w,\mu(w)T(h)\cdot g) \cdot (y,f) \\
	&= (w,\mu(w) \cdot T(\mu(w)) T^2 (h) \cdot T(g)\cdot f) \\
	&= (w, \mu(w) \cdot \mu(T(w))\cdot T^2 (h) \cdot T(g) \cdot f), \text{ and $T(w) = \hat h$} \\
	&= (w, \mu(w) \cdot T(h) \cdot \mu(\bar h) \cdot T(g) \cdot f) \text{ and $\bar h = z$} \\
	&= (w, \mu(w) \cdot T(h) \cdot (\mu(z) \cdot T(g) \cdot f)) \\
	&= (w,h) \cdot (z, \mu(z) \cdot T(g) \cdot f) \\
	&= (w,h) \cdot ((z,g) \cdot (y,f)), 
\end{align*}
after making the choice of $y$ to represent $(y,f)$. 

Next,
\begin{align*}
\overline{(z,g) \cdot (y,f)}
	&= \overline{(z, \mu(z) \cdot T(g) \cdot f)} \\
	&= (\verywidehat{\overline{\mu(z) \cdot T(g) \cdot f}}, \eta( \widehat{\overline{\text{same}}}) \cdot \overline{\text{same}}) \\
	&= (\widehat{T(z)}, \eta(\widehat{T(z)})\cdot T(z)) \\
	&= (\hat{\bar{g}}, \eta(\hat{\bar{g}}) \bar{g}) = \overline{(z,g)}, 
\end{align*}
and
\begin{align*}
\verywidehat{(z,g) \cdot (y,f)} 
	&= \verywidehat{(z,\mu(z) \cdot T(g) \cdot f) } \\
	&= (\hat z, \eta(\hat z) \cdot z) \\
	&= \widehat{(z,g)}.
\end{align*}

Now if $\overline{(z,g)} \not\dleq \widehat{(y,f)},$ they do not compose in $\sC_T$ by definition. On the other hand, if $\overline{(z,g)} \dleq \widehat{(y,f)},$ we must guarantee that they may be composed. If we have
$$
\overline{(z,g)} \dleq \widehat{(y,f)},
$$
then
$$
\eta(\hat{\bar{g}}) \cdot \bar{g} \dleq \eta(\hat{y}) \cdot y$$
so 
$$T(\eta(\hat{\bar{g}})) \cdot T(\bar{g}) \dleq T(\eta(\hat{y})) \cdot T(y)$$
so 
$$\mu(\hat{\bar{g}}) \cdot T(\eta(\hat{\bar{g}})) \cdot T(\bar{g}) \dleq \mu(\hat{y}) \cdot T(\eta(\hat{y}) \cdot T(y)$$
so $T(\bar g) \dleq \hat f$, as desired, using monotonicity of $\mu$. 

Now if $(y,f) \in \sC_T$ is an object, then $(y,f) = (y, \eta(y))$ for some $y \in \Ob(\sC)$. For then 
$$(y,f) = (\hat{\bar{f}}, \eta(\hat{\bar{f}}) \cdot \bar{f}) = (\hat{y}, \eta(\hat{y})\cdot y)$$
So 1. $y = \hat{\bar{f}}$ and 2. $f = \eta(\hat{\bar{f}}) \bar{f}$ and 3. $y = \hat{y}$ and 4. $f = \eta(\hat y) \hat y$. 3. and 4. with $\eta$-monotonicity gives $\bar f = y$. Taking source in 2. gives $y = \bar y$. So with 3., $y$ is an object in $\sC$. Now 4. shows that $f = \eta(y)$. %
Now we can check that, using $\eta$-monotonicity (twice!), $(y, \eta(y))$ satisfies axiom (\ref{ax.gencat-object-id}) of Definition \ref{d.gencat}.

%should get $1_{F(f)} = (f,\eta(f))$ % [4] 8/23/16
%\stuffgoeshere
Now, if $(z,g) \dleq (y,f),$ then $\hat{\bar{g}} \dleq \hat{\bar{f}}$. So $\eta(\hat{\bar{g}}) \dleq \eta( \hat{\bar{f}}).$ And $\bar{g} \dleq \bar{f},$ so $\eta(\hat{\bar{g}}) \bar g \dleq \eta(\hat{\bar{f}}) \bar{f}$. Similarly, $\eta(\hat z) z \dleq \eta(\hat y) y.$ So $\overline{(z,g)} \dleq \overline{(y,f)}$, and $\widehat{(z,g)} \dleq \widehat{(y,f)}.$ 

If $(z,g) \dleq (y,f),$ and $(u,k) \dleq (w,h),$ it is similarly verified that $(z,g) \cdot (u,k) \dleq (y,f) \cdot (w,h)$. 

The last axiom to check is $a \dleq b$ then $1_a \dleq 1_b$. The condition $(\star)$ in Definition \ref{d.kleisliconst} is the following: 
$$(\star): (y,f) = 1_{(x,u)} \eimplies \there v \text{ such that } (x,u) = (\hat{v}, \eta(\hat{v}) \cdot v).$$
%If we peek ahead to the definition of $F$, 
This condition specifies that we throw away all identities in $\sC_T$ that are not in the image of $F_T$, defined below. We will use this now, and we will use it again below. 
Let $(y,f) \dleq (z,g)$, and suppose that both $(y,f)$ and $(z,g)$ have identities $1_{(y,f)}$ and $1_{(z,g)}$, respectively. By the condition $(\star)$ we may write $(y,f) = (\hat u, \eta(\hat u) u)$ and $(z,g) = (\hat v, \eta(\hat v) v)$ for some $u,v \in \sC$. Then it is not difficult to see that $1_{(y,f)} = (u,\eta(u))$ and $1_{(z,g)} = (v, \eta(v))$. Now %, only once, 
we apply the full strength of the monic condition to conclude that $u \dleq v$, from which axiom (Identities) of Definition \ref{d.gencat} follows. 
So $\sC_T$ is a generalized category. %and that the subcategory $\sC_T^0$ is a generalized category with identities. 

Next, define mappings
$$F_T (f) := (\hat f, \eta(\hat f) \cdot f),$$
$$G_T(y,f) := \mu(y) \cdot T(f).$$
We have 
\begin{align*}
F_T (g \cdot f) 
	&= (\widehat{g \cdot f}, \eta(\widehat{g \cdot f}) \cdot g \cdot f) \\
	&= (\hat g, \eta(\hat g) \cdot g \cdot f) \\
	&= (\hat g, \id_{T(\hat g)} T(g) \cdot \eta(\bar g) \cdot f), \text{ and $\bar g = \hat f$} \\
	&= (\hat g, \mu(\hat g) \cdot T(\eta(\hat g)) \cdot T(g) \cdot \eta(\hat f) f) \\
	&= (\hat g, \mu(\hat g) \cdot T(\eta(\hat g) \cdot g) \cdot \eta(\hat f)\cdot f) \\
	&= (\hat g, \eta(\hat g) \cdot g) \cdot (\hat f, \eta(\hat f) \cdot f) \\
	&= F_T(g) \cdot F_T(f).
\end{align*}
Now if $f \dleq g$, then $\hat f \dleq \hat g$, and by substitutivity (Definition \ref{d.gencat}) and monotonicity of $\eta$, % 
$(\hat f, \eta(\hat f)f) \dleq (\hat g, \eta(\hat g) g).$ So $F_T$ is monotonic. We can also check that 
$$F_T (\hat f) = (\hat{\hat{f}}, \eta(\hat{\hat{f}}) \cdot \hat{f}) = \verywidehat{(\hat{f}, \eta(\hat{f}) \cdot f)} = \widehat{F_T (f)}$$
and 
$$F_T (\bar f) = (\hat{\bar{f}}, \eta( \hat{\bar{f}}) \bar{f}) = \overline{F_T (f)}.$$

%Next, $F_T(1_f) = (\widehat{1_f}, \eta(\widehat{1_f}) \cdot 1_f) = (f, \eta(f))$

%And $1_{F_T(f)} = 1_{(\hat f, \eta(\hat f) \cdot f)}.$ 
Now suppose that $f \in \sC$ is a subject. Then it has an identity $1_f$. The image $F(1_f)$ of this element is $(f,\eta(f))$, the identity of $F(f)$. 

So $F_T$ is a functor $\sC \to \sC_T$. The image $\sC_T^0$ of $\sC$ under $F$ is a generalized category with identities, %by condition $(\star)$. 
for it is easy to see in general that the image under a functor of an element that is not a subject is not a subject.

We also have 
\begin{align*}
G_T ((z,g) \cdot (y,f)) 
	&= G_T (z, \mu(z)T(g) f) \\
	&= \mu(z) \cdot T(\mu(z) \cdot T(g) \cdot f) \\
	&= \mu(z) \cdot T(g) \cdot \mu(\bar g) \cdot T(f) \text{ and $\bar g = y$} \\
	&= G_T (z,g) \cdot G_T (y,f).
\end{align*}
And
\begin{align*}
G_T (\overline{(y,f)} 
	&= G_T (\hat{\bar{f}}, \eta(\hat{\bar{f}}) \bar{f}) \\
	&= \mu (\hat{\bar{f}}) \cdot T(\eta ( \hat{\bar{f}}) \cdot \bar{f}) \\
	&= T(\bar{f}) \\
	&= \overline{T(f)} \\
	&= \overline{\mu(y) T(f)} \\
	&= \overline{G_T (y,f)}.
\end{align*}
Similarly
\begin{align*}
G_T (\widehat{(y,f)}) 
	&= \mu(\hat y) \cdot T(\eta(\hat y))T(y) \\
	&= T(y) \\
	&= \verywidehat{\mu(y)T(f)} \\
	&= \verywidehat{G_T (y,f)}.
\end{align*}
For monotonicity, if $(y,f) \dleq (z,g)$,
$$G_T (y,f) = \mu(y) T(f) \dleq \mu(z) T(g) = G_T (z,g).$$
%identities $G_T (1_{(y,f)}) = 1_{G_T (y,f)}$
%$G_T (\flr) = \flr$
%\stuffgoeshere
For identities, we apply $(\star)$ again:
$$G_T (1_{(y,f)}) = G_T (v, \eta(v)) = \mu(v)T(\eta(v)) = 1_{T(v)}$$
and indeed, $T(v) = G_T (\hat v, \eta(\hat v) v) = G_T (y,f).$ So $F_T$ and $G_T$ are functors. We shall make no distinction between these and their appropriate restrictions to $\sC_T^0$.

Next, define a map
$$\phi: \hom(F(f), (z,g)) \to \hom(f, G(z,g))$$
by $\phi(u,k) := k$. Clearly, $\phi$ is order-preserving. 
%check $\flr$
%\stuffgoeshere
Next, $\phi$ is injective, for if $\phi(u,k) = \phi(u', k)$, then $\eta(z) \cdot u = g$ and $\eta(z) \cdot u' = g$. So $\eta(z) \cdot u = \eta(z) \cdot u'$. Hence $u = u'$ by the monic condition. 

It is surjective if it can be shown that for every $k \in \hom(f,G(z,g))$ there is a $u$ such that $\phi(u,k) = k$, or in other words, there is a $u$ such that %
$\eta(z) \cdot u = g$. % same as $\there u \st \mu(z) \cdot T(g) = T(u).$
If $(z,g)$ is in the image of $F$, then we obtain the desired $u$. So $\phi$ is surjective onto $\sC_T^0$. 

Now we check naturality of $\phi$. We have
\begin{align*}
\phi((w,h) \cdot (u,k)) 
	&= \phi(w, \mu(w T(h) \cdot k) \\
	&= \mu(w) T(h) \cdot k \\
	&= G(w,h) \cdot \phi(u,k),
\end{align*}
and
\begin{align*}
\phi(w,l_1) \cdot F(l_2)) 
	&= \phi((v,l_1) \cdot (\hat{l_2}, \eta( \hat{l_2}) l_2) \\
	&= \phi((v, \mu(v) \cdot T(l_1) \cdot \eta(\hat{l_2}) l_2)) \text{ and $\bar{l_1} = \hat{l_2}$} \\
	&= \mu(v) \cdot T(l_1) \cdot \eta(\bar{l_1}) \cdot l_2 \\
	&= \mu(v) \eta(\hat{l_1}) \cdot l_1 \cdot l_2 \text{ and $\hat{l_1} = T(v)$} \\
	&= l_1 l_2 \\
	&= \phi(v, l_1) \cdot l_2.
\end{align*}
So we have an order-preserving adjunction between $\sC$ and $\sC_T^0$. Applying Proposition \ref{p.monadj}, we obtain a monad $(T^\phi, \eta^\phi, \mu^\phi)$ on $\sC$, and it remains to show that this monad is precisely the original monad $T$ given on $\sC$. %
First, we have
\begin{align*}
T^\phi (f) 
	&= G_T \of F_T (f) \\
	&= G_T (\hat f, \eta(\hat f) f) \\
	&= \mu(\hat f) \cdot T (\eta(\hat f) \cdot f) \\
	&= T(f).
\end{align*}
Next,
$$\mu^\phi = G \of \epsilon \of F = \phi^{-1}(1_{\mu(y) T(f)}),$$
so
\begin{align*}
\mu^\phi(f) 
	&= G(\epsilon(F(f))) \\
	&= G(\epsilon(\hat f, \eta(\hat f) f)) \\
	&= G(\phi^{-1}(1_{G(\hat f, \eta(\hat f) f)})) \\
	&= G(\phi^{-1}(1_{T(f)})),
\end{align*}
and $\phi^{-1}(1_{T(f)}) = (f, 1_{T(f)})$ because $\phi$ is injective and $\phi(f, 1_{T(f)}) = 1_{T(f)}.$ Therefore this is
\begin{align*}
	&= G(\phi^{-1} (1_{T(f)})) \\
	&= G(f, 1_{T(f)}) \\ 
	&= \mu(f).
\end{align*}
Finally,
\begin{align*}
\eta^\phi (f)
	&= \phi(1_{F(f)}) \\
	&= \phi(1_{(\hat f, \eta(\hat f) f)}) \\
	&= \eta(f)
\end{align*}
again since $\phi$ is injective.
\Prf


\subsection{Monads and Triples}\label{ss.montriple}

For applications, e.g. \cite{MoI1, WaR1, AcAdMiVe1}, a monad is often thought of as a {\em triple} in the sense we now define. We will verify that the usual interchangability between the two notions holds, but only in a certain sense. 

\dfn\label{d.triple}
$\sC$ a generalized category. A {\em triple}, or {\em Kleisli triple}, on $\sC$ is a pair $(\eta, ()^*)$ %
where $\eta$ and $()^*$ are monotonic %, bottom-preserving 
maps $\sC \to \sC$. 
For $f \in \sC$, set the abbreviation
$$T(f) := (\eta(\hat f) \cdot f)^*.$$
The mappings $\eta, ()^*$ are required to satisfy
\enu
	\item[$(0^1)$] $\widehat{f^*} = \hat f, \,\, \overline{f^*} = T(\bar f)$
	\item[$(0^2)$] $\widehat{\eta(f)} = T(f), \,\, \overline{\eta(f)} = f$
	\item[$(1)$] $\eta(f)^* = 1_{T(f)}$ \label{ax.etastar}
	\item[$(2)$] $f = f^* \cdot \eta(\bar f)$ \label{ax.stareta}
	\item[$(3)$] $(g^* \cdot f)^* = g^* \cdot f^*$
\Enu
\Dfn
Note that by axiom (1), $T$ restricts to a mapping $\Ob(\sC) \to \Ob(\sC)$. 

\exa
Triples appear in applications to programming language theory, where the mapping $T()$ is considered to send a type $A$ to its corresponding type $T(A)$ of {\em computations} of that type, in a system where programs are regarded not as pure functions, but as functions with complications such as failure to terminate, indeterminacy, continuations, or side effects. Correspondingly, $\eta: A \to T(A)$ is regarded as the inclusion of values into computations of type $A$, and $(f)^*:T(A) \to T(B)$ is thought of as the extension of $f:A \to T(B)$ to $T(A)$. 
\Exa

\prop\label{p.montriple}
In the generalized setting, a Kleisli triple gives rise to a monad, and conversely (though not on the Kleisli category $\sC_T$) assuming the following {\em hypothesis}: if $T(a) = T(b)$ then $\mu(a) = \mu(b)$. 
\Prop
\prf
Let $\sC$ be equipped with the triple $(\eta, ()^*)$. Then 
\begin{align*}
T(g \cdot f) 
	&= (\eta(\widehat{gf}) gf )^* \\
	&= (\eta(\hat g) \cdot g \cdot f)^* \\
	&= (\eta(\hat g) \cdot g \cdot f)^* \\
	&= ((\eta(\hat g) \cdot g)^* \cdot \eta(\overline{\eta(\hat g)\cdot g}) \cdot f)^* \\
	&= ((\eta(\hat g) \cdot g)^* \eta(\hat f) \cdot f)^* \\
	&= (\eta(\hat g) g)^* (\eta(\hat f) f)^* \\
	&= T(g) T(f).
\end{align*}
and
\begin{align*}
\overline{T(f)} 
	&= \overline{(\eta(\hat f) \cdot f)^*} \\
	&= T(\overline{\eta(\hat f) \cdot f}) \\
	&= T(\bar f).
\end{align*}
and
\begin{align*}
\widehat{T(f)} 
	&= \verywidehat{(\eta(\hat f) \cdot f)^*} \\
	&= \verywidehat{\eta(\hat f) \cdot f)} \\
	&= \widehat{\eta(\hat f)} = T(\hat f).
\end{align*}
and
$$ T(1_f) = (\eta(\widehat{1_f})1_f)^* = \eta(f)^* = 1_{T(f)}.$$
Now if $f \dleq g$, then 
$$T(f) = \eta(\hat f)\cdot f \dleq \eta(\hat g) \cdot g = T(g),$$
using monotonicity of $\eta$. %And since $\eta$ and $()^*$ are bottom preserving, $T(\flr) = \flr$. 
So $T$ is a functor. %
Now for $f \in \sC$ define
$$\mu(f) = (1_{T(f)})^*.$$ %
To show that $(T,\eta,\mu)$ is a monad on $\sC$, it remains to check that $\eta$ and $\mu$ are natural transformations, and axioms (\ref{ax.monmagic1}) and (\ref{ax.monmagic2}) of Definition \ref{d.monad}. %
We leave this to the reader. %
We use the monotonicity of $()^*$ to prove the monotonicity of $\mu$. % 

% qed.

Now, %
let $(T,\eta, \mu)$ be a monad on $\sC$. %
We define a very abstract generalized category that contains some things that can be easily constructed, and some things that cannot be. Let
%
$$\sC'_T = \set{(f,\vec y) \mid \vec y = (y_0, y_1, \dots), \widehat{s^i(f)} = T(y_i) }.$$
%(We can think of elements (if we like) as planar binary trees.) 
Put
\begin{equation}
\begin{cases}
	\, \widehat{(f, \vec y)} = (T(y_0), \hat{y_0}, \hat{\bar{y_0}}, \hat{\bar{\bar{y_0}}}, \dots) & \\
	\, \overline{(f, \vec y)} = (\bar f, y_1, y_2, \dots) & \\
	\, (g, \vec z) \cdot (f, \vec y) = (g \cdot f, z_0, y_1, \dots) & 
\end{cases}
\end{equation}
and
%$$T(f,\vec y) := (T(f), T(y_0), T(y_1), \dots)$$
\begin{equation}
\begin{cases}
	\, \eta(f, \vec y) = (\eta(f), f, \vec y) & \\
	\, (f, \vec y)^* = (\mu(y_0)T(f), y_0, T(y_1), T(y_2), \dots). & 
\end{cases}
\end{equation}
Let $(f, \vec y) \dleq (g, \vec z)$ if $f \dleq g$ and for all $i \geq 0$, $y_i \dleq z_i$, so $\sC'_T$ is a poset. % with bottom $\flr = (\flr, \flr, \dots)$. %
Composition is associative since
\begin{align*}
(h, \vec v) \cdot ((g, \vec z) \cdot (f, \vec y))
	&= (h,\vec v) \cdot (g \cdot f, z_0, y_1, \dots) \\
	&= (h \cdot g \cdot f, v_0, y_1, \dots) \\
	&= (h \cdot g, v_0, z_1, \dots) \cdot (f, \vec y) \\
	&= ((h, \vec v) \cdot (g, \vec z)) \cdot (f, \vec y) 
\end{align*}
and indeed $\overline{(g,\vec z) \cdot (f, \vec y)} = (\bar f, y_1, \dots) = \overline{(f,\vec y)},$ and $\widebar{(g,\vec z) \cdot (f, \vec y)} = \widehat{(g,\vec z)}.$ %
If $\overline{(g,\vec z)} \dleq \widehat{(f,\vec y)},$ then 
$$(\bar g, y_1, \dots) \dleq (T(y_0), \hat{y_0}, \hat{\bar{y_0}}, \dots),$$
so $\bar g \dleq \hat f$, and hence $(g,\vec z) \cdot (f, \vec y)$ is defined. (The converse may be imposed in the definition of $\sC'_T$.) %
Axioms (\ref{ax.gencat-order1}) and (\ref{ax.gencat-order2}) of Definition \ref{d.gencat} are trivial. %

It can be checked that objects of $\sC'_T$ are in one-to-one correspondence with objects of $\sC$ in the image of $T$. These in turn have well-behaved identities, so axiom (\ref{ax.gencat-object-id}) of Definition \ref{d.gencat} is satisfied. %
%
So $\sC'_T$ is a generalized category. %
%
It can also be easily verified by the reader that an element $(f, \vec y)$ of $\sC'_T$ has an identity iff it is of the form $(T(u), \hat u, \hat{\bar{u}}, \dots)$ for some $u \in \sC$. In other words, there is a one-to-one correspondence between elements of $\sC$ possessing an identity and elements of $\sC$ in the image of $T$.\footnote{%
In fact, the ``ingoing'' hom set of an element {\em not} of this form is empty. Therefore these elements may be thought of as copresheaves over a base.} %

Now, we check that we have a triple on $\sC'_T$. First, we check that 
$$T(f,\vec y) = (T(f), T(y_0), T(y_1), \dots)$$
Next,
\begin{align*}
\widehat{(f,\vec y)^*} 
	&= \verywidehat{(\mu(y_0)T(f), y_0, T(\dots))} \\
	&= (T(y_0), \hat{y_0}, \hat{\bar{y_0}}, \dots) \\
	&= \widehat{(f, \vec y )}
\end{align*}
and
\begin{align*}
\overline{(f, \vec y)^*} 
	&= \overline{(\mu(y_0)T(f), y_0, T(\dots))} \\
	&= (\overline{\mu(y_0)T(f)}, T(y_1), T(y_2), \dots) \\
	&= (T(\bar f), T(y_1), \dots) \\
	&= T(\bar f, y_1, \dots) \\
	&= T(\overline{(f, \vec y)}) 
\end{align*} 
and 
\begin{align*}
\widehat{\eta(f, \vec y)} 
	&= \verywidehat{(\eta(f), f, \vec y)} \\
	&= (T(f), \hat f, \hat{\bar{f}}, \hat{\bar{\bar{f}}}, \dots) \\
	&= (T(f), T(y_0), T(y_1), \dots) \\
	&= T(f, \vec y)
\end{align*}
and similarly, $\overline{\eta(f,\vec y)} = (f, \vec y)$. Next, we have
\begin{align*}
\eta(f, \vec y)
	&= (\eta(f), f, \vec y)^* \\
	&= (\mu(f)T(\eta(f)), f, T(\vec y)) \\
	&= (1_{T(f)} , f, T(y_0), T(y_2), \dots) \\
	&= (1_{T(f)}, f, \hat f, \hat{\bar{f}}, \dots) \\
	&= 1_{(T(f), \hat f, \hat{\bar{f}}, \dots)} \\
	&= 1_{T(f, y_0, y_1, \dots)}.
\end{align*}
and
\begin{align*}
(f,\vec y)^* \cdot \eta(\overline{(f, \vec y)})
	&= (\mu(y_0)T(f), y_0, T(y_1), T(y_2), \dots) \cdot (\eta(\bar f), \bar f, y_1, y_2, \dots) \\
	&= (\mu(y_0) \cdot T(f) \cdot \eta(\bar f), y_0, y_1, \dots) \\
	&= (f, \vec y).
\end{align*}
and finally
\begin{align*}
((g, \vec z)^* \cdot (f, \vec y))^*
	&= ((\mu(z_0)T(g), z_0, T(z_1), T(z_2), \dots) \cdot (f, y_0, y_1, \dots))^* \\
	&= ((\mu(z_0)T(g)f, z_0, y_1, \dots))^* \\
	&= (\mu(z_0) \cdot T(\mu(z_0)\cdot T(g) \cdot f), z_0, T(y_1), T(y_2), \dots) \\
	&= (\mu(z_0) \cdot T(\mu(z_0) \cdot T^2 (g) \cdot T(f), \text{same}) \\
	&= (\mu(z_0) \cdot \mu(T(z_0)) \cdot T^2 (g) \cdot T(f), \text{same}), \text{ and $T(z_0) = \hat g$} \\
	&= (\mu(z_0) \cdot T(g) \cdot \mu(\bar g) \cdot T(f), z_0, T(y_1), T(y_2), \dots).
\end{align*}
Now	applying the hypothesis in the statement of the Proposition, we conclude that $\mu(\bar g) = \mu(y_0)$. Hence this expression is $(g,\vec z)^*(f,\vec y)^*$, and $(\eta, ()^*)$ is a triple on $\sC'_T$, as desired. 
\Prf

Can the hypothesis be removed from Proposition \ref{p.montriple}? %

Is there also a triple defined on $\sC_T$? %

What is $(M(\sC'_T))'_T$, where $M()$ is the construction of the first half of the proof?

It can be verified that, via the proof of \cite{MacCW} with few changes, there is a unique comparison functor from $\sC_T$ to $\sD$, given any adjunction 
\begin{tikzcd}
	\sC \arrow[r, "F", shift left] \arrow[r, leftarrow, shift right, "G"'] & \sD \\
\end{tikzcd}
that induces the monad $T$. 






















































%\input{genm4(alg)} % algebras
% File genm4(alg).tex Created by Lucius Schoenbaum September 10, 2016
% algebras




















\begin{comment}
\section{Algebras}\label{s.alg}

Now that we have established the basic theory of monads in the generalized setting, we shall delve a little deeper in this section. We omit details. %
%All generalized categories in this section are assumed to be sharp. %
Let $\sC$ be a (sharp, for simplicity) generalized category, and let $T$ be a monad on $\sC$. 

We can approach the notion of 

\dfn\label{d.usualemcat}
Let $\sC^T_0$ be the two-sorted category of $T$-algebras
\Dfn

\dfn
Then let 
$\sC^T$ be the set of all planar binary trees $\ff$ with nodes in $\sC$ that satisfy
%These are subject to the following conditions:
\enu
	\item $\troot(\tleft \ff) \cdot T(\troot(\ff)) = \troot(\ff) \cdot \troot(\tright \ff), \quad\text{a.d.}$
	\item 
\Enu

\stuffgoeshere
\Dfn
\end{comment}


\subsection{Eilenberg-Moore Algebras}
Once we have investigated Kleisli categories, it is natural to ask what occurs when we look at Eilenberg-Moore algebras (as well as ordinary algebras and coalgebras) in the generalized setting. %
The results are so far inconclusive. %
Several key steps in results on Eilenberg-Moore algebras depend on the %character %the category of algebras for a monad, which can be viewed as a generalization of the category of representations of a group or similar structure, has 
observation that objects in $\sC^T$ possess a shared object-morphism character. %
Thus, it might appear that the theory of algebras is in some sense already a native of the generalized setting, % %
and that (interestingly) the category of algebras is a noncellular generalized category, manifesting as a sort of ``refraction,'' if you like, of the input category. %---this is the definition $\sC'^T$. %
%fits remarkably well in the generalized setting. %
%The meaning of this remark will be . %
%(The reader might only need to reflect a little while in order to perceive the idea for him or herself.) 
%[comment for draft.] 
However, the latter point of view runs into difficulties that the author could not redeem. At several points, uniqueness results apparently break down, resulting in a weaker theory than the theory of algebras based on the 1-dimensional category $\sC^T$. % (Definition \ref{d.usualemcat}. %Let the reader see for him or herself. 

On the other hand, the %(rather more boring!) 
1-dimensional definition succeeds, as usual, and takes its usual role as the terminal object in the category of those adjunctions that define $T$ in $\sC$, and supplies the usual Monadicity (Tripleability) theorems. %(Though the author has not checked all the details carefully.) %
%We invite the reader to weigh these considerations, and somewhat the author's unease. 



























%%%%%%%%%%
\begin{comment}
A comparison functor is {\em monadic} if it is invertible as a mapping. 

\thm%[Beck's theorem]
Let $\sC,\sD$ be (sharp) generalized categories, and let $(F,G,\eta,\epsilon)$ be an adjunction 
$$
\begin{tikzcd}
	\sC \arrow[r, "F", shift left] \arrow[r, leftarrow, shift right, "G" below] & \sD \\
\end{tikzcd}\vspace{-3.5ex}
$$
Let $T$ be the monad defined on $\sC$ by $(F,G,\eta,\epsilon)$. Let the notation $\sC^T, F^T, G^T, \epsilon^T, \eta^T, K$ (the Eilenberg-Moore construction) be as before. Then the following are equivalent:
\enu
	\item The comparison functor $K$ is invertible. 
	\item $G$ creates coequalizers for parallel pairs $f,g$ in $\sD$ for which $Gf, Gg$ has an absolute coequalizer in $\sC$.
	\item $G$ creates coequalizers for parallel pairs $f,g$ in $\sD$ for which $Gf, Gg$ has a split coequalizer in $\sC$. 
\Enu
\Thm
\prf
\stuffgoeshere

\Prf
\end{comment}
%%%%%%%%%%





















			%\input{genm5(app)} % applications
%\input{genm6(conc)} % conclusion
% File genm6(conc).tex Created by Lucius Schoenbaum August 23, 2016
% conclusion































\section{Conclusion and Future Work}\label{s.conc}

In this chapter we have %that the framework of category theory carries over to a much wider setting, 
presented the theory of monads as evidence of the existence of a robust mathematical framework generalizing category theory. %
We have not mentioned several topics, e.g., strengths, Linton's correspondence, and distributive laws, %
but we have given indication of some applications of this framework. 

It may be noticed that the mathematics of generalized categories is in some ways conceptually simpler, and in other ways conceptually more challenging compared to ordinary category theory. The inherent emphasis on recursion and corecursion in the framework suggests potential connections between generalized categories, iterative equations, and categorical fixpoint theory. 

%Much more work can be done to clarify the picture this work presents, some of which the author has done in \cite{ScMd}, some of which the author would like to do, and some the author imagines to be far beyond his reckoning. 

%For further details on applications of generalized categories, see \cite{ScMd}.






















































































\pagebreak
\singlespacing
\chapter{A Generalization of the Curry-Howard Correspondence}\label{c.th}
\doublespacing
\vspace{10ex}
			%\input{th}
%\input{th1(i)} % introduction
% File th1(i).tex Created by Lucius Schoenbaum September 27, 2016
% th - introduction






















\section{Introduction}\label{s.th-i}

In a series of papers \cite{LaK1c,LaK2}, Lambek developed an extension of the Curry-Howard correspondence \cite{Howard1969} to the domain of categorical logic. %
Lambek's extension has since become a cornerstone of programming language theory, particularly in the functional programming paradigm. It has also been influential in logic. % 
This paper %, an extension of \cite{ScM2}, 
is devoted to a generalization of the Curry-Howard-Lambek correspondence which makes use of the tools provided by generalized categories. %
%It will make use of a slight modification of the notion of type in which types are characterized as very general path spaces and behave formally in a way analogous to an ideal inside of the (generalized) category on the categorical side of the correspondence. %
% could also mention that it gives a notion of generalized topos.
Those who agree with Philip Wadler \cite{WaR5} that, as a general rule, semantics should guide development in logic and programming language theory may take interest in this product of a generalization on the semantic side. %
Those with a pure interest in category theory might note some features of our approach, for example, we show (section \ref{s.cat}) that using the framework of generalized categories, a cartesian functor between cartesian closed categories may be ``promoted'' to a cartesian closed functor. To the best of our knowledge this construction is at least somewhat new. 

Lambek in his work makes extensive use of deductive systems \cite{LaK1c}. %
A short discussion of the intuition for this notion (which may be unfamiliar) affords the opportunity to provide some intuition for the notion of generalized category. However, the reader is free to ignore this discussion if he or she wishes; nothing in the main body of the paper depends on it. %
A deductive system is just enough machinery to allow the question: from a given point $a$ of the deductive system $\sA$, can I travel to another point $b \in \sA$ via a valid path? %
A conceptual picture of this is the following. Suppose that there is a system of goods $\sA_0$. %
The edges of $\sA$ are certificates (issued, say perhaps, by different governing bodies) that say that a good $a \in \sA_0$ may be exchanged for another good $b \in \sA_0$. %
It is accepted that a good is always exchangeable for itself. %
Now let's suppose that such certificates themselves may be exchanged, % either for goods, or for other certificates. %
but that this requires that one has a higher-level certificate for this higher-level trade. %
If we imagine a certain impetus exists among those we imagine making the exchanges, we can expect that there will next arise trading for these certificates as well. % 
Let us make two simple observations:
\enu
	\item The resulting deductive system is not necessarily {\em cellular}, in the sense that the economy is liberalized to the extent that certificates may be good for exchange of different {\em kinds} of goods and certificates. For example, a certificate may be for a good, in return for a certificate good for a certificate in return for a good.
	\item There need not be, in the abstract, any {\em goods} at all. The system could be one of certificates for certificates for certificates, and so on. This observation may be utilized to clean up the abstract formalism: a system with no atoms is conceptually simpler and the easiest one to work with while developing elementary principles. 
\Enu
These two observations suggest, via the intuition, a generalization of category theory that we outline in section \ref{s.gencat}. 
%Now, finally, suppose that there were no goods in the first place. %
%In fact, all the goods were in reality certificates, which allowed exchange of certificates for other certificates, and so on, ad infinitum. %
%This gives an intuition for a deductive system in general.\footnote{% 
%This example is recognizable: it could be said to be instantiated by the framework of pure mathematics. Mathematics is generally recognized to be the study of structure. But what are structures? Structure, any pure mathematician will say, arises from...the study of structure. And so on, and so on, ad infinitum. To supply a concrete example, a group gives rise to a field via the field of fractions of the group ring. A field gives rise to a group via the group of units. You can repeat as often as you like. And so on, and so on---there is no end to it. FIX} 
%
%Of course, the general case includes all the more basic cases, as well as other, very general structures. 

%We can ask the question: why don't we allow every every path to be valid? %One answer: we don't wish to do this, because we will want to work with {\em types.} 
%This would make every type valid, something we do not wish to do, since we wish to have a system that functions in conformity with the Curry-Howard correspondence for ordinary categories and type theory \cite{Howard,Curry}. %
%A mathematical model of such a system is, for example, the fundamental $\infinity$-groupoid of a topological space, an example which is cellular and in which several monoids are present. A general framework in which such a rich algebraic structure can be placed. 

Some work during intermediate stages is necessary in order to accomplish our aim. %
Under the usual Curry-Howard correspondence, types are interpreted as propositions which are true only when they are inhabited by a term. %(meaning that there exists a term of that type). %
%An innovation, one which sits in agreement with the recent introduction of homotopy theory to type theory \cite{HoTT}, that we take is to %develop types, not as target (objects), but as {\em path spaces} or (if preferred) %(depending on how you interpret the symbolism) 
%{\em blueprints of paths}. %
%This means that in our thinking, if we wish, we may identify types with abstract hom sets. %
%This interpretation, in turn, is prompted by the consideration of generalized categories: to define the internal language of a generalized category, we must abandon the view that types are objects. This leaves us with no choice but to consider alternatives. %
It is based on the types-as-targets view of categorical semantics, which limits the applicability of generalized categories to type theory. %
If we consider the alternative {\em types-as-paths} view, in which a proposition depends on both a source and a target, we find a calculus that is not only amenable to the generalized setting, but also fits well with the Lambek equational theory of cartesian closed categories \cite{LaK2}. %
%To obtain this, we simply add structure to the deductive system, closely following Lambek. 
%Because the notation that we use is used by many different authors in many different ways, % let us state the main point once more. %
The types-as-path view is motivated by the notion that a type is like the blueprint of a bridge between two points, or (in the logical intuition), a {\em conjecture}. %
Using the intuition from programs, on the other hand, the type-as-path is an {\em approximation} or {\em abstraction} of any choice of concrete transformation between two different kinds of data. This supports our approach, since this is how types are often viewed in applications, see for example \cite{Pierce1}. %
%However we wish to think of it, it is this bridge that we wish to ``inhabit'' with a valid path. 
The types, which we write $a \vdash b$, when viewed categorically, assume the role of exponential objects. %
We are able to give this description a precise formal treatment by combining (1) the contributions of Lambek and (2) the framework of generalized category theory. 

%bridge. We would like to be able to consider systems in which there is a type present between {\em any} two edges, and we do not wish for these types to be valid in all instances.\footnote{ %
%One might ask further: why do we wish to consider such types as members of the underlying set of the (generalized) graph? %Why not just treat them as abstractions living somewhere outside of the underlying system? The answer is that 
%We will eventually want to consider elements of the graph that have these types as source and target, or in other words to provide what are known as {\em higher kinded types}, and the motivation for this is to provide language support for dependent type theory, polymorphism, and other uses---ultimately in order to achieve the goals of programming language design (to allow code reuse, for example) \cite{PierceTyTVol2}.}
%However, we do not wish for these types to always be valid! %We can introduce these conjectures:
%We would then not have access to the cartesian closed machinery that plays an important role in deductive systems. 
%Formally, we have:







%The purpose of this note is to establish an extension of the work of Lambek and others \cite{LaK1c,LaK2,LaSc1,SeY1} on the relationship between cartesian closed categories and higher order logic. %
%Our work relies on generalized category theory \cite{ScMd}. %

%We have chosen our main theorem due to its compelling beauty and apparent interest to the field of categorical logic. However, there are 
%other contributions, however not all are of the nature of results: % not all of which we assume to have a long, illustrious future but which seem to suggest themselves as interesting: %
%\enu
%	\item The extension of Curry-Howard-Lambek to generalized categories. 
%	\item The types-as-paths approach to type theory.
%	\item The construction of the ideal of types.
%	\item A notion of generalized cartesian closed category. 
%	\item The theorem above
%	\item types-as-paths.
%\Enu
%The extension of the Curry-Howard-Lambek theorem to generalized categories is a result whose significance has yet to be determined, due to the paucity of examples of generalized categories. %
%The types-as-paths approach to type theory adopts the view that (ignoring the equational structure of categories and $\lambda$-calculi) types amount to queries in a generalized graph. The crucial extra structure provided by cartesian closed structure is the {\em name bijection} (Lawvere's terminology)
%$$f:X \to Y \,\,\, \longleftrightarrow \,\,\, f: 1 \to Y^X, $$
%allowing a function to be associated with a unique {\em term}, an element whose source is the terminal object. 

%The generalized cartesian closed category we propose %
%amounts merely to an interpretation of Lambek's equational theory of \cite{Lambek1}. %as a generalization of ccc. 
%The observation is this: when we add the ideal of types---in other words, when we take the types-as-paths view---we are forced to depart with the usual adjunction. In spite of this, Lambek's axioms continue to hold! % Curry-Howard correspondence. 
%This leads us to conjecture that a theory of generalized elementary toposes should also be based on the equational axiomatization of Lambek.

%In section \ref{s.gencat}, we introduce generalized categories \cite{ScMd}. 
In section \ref{s.th}, we develop ideal cartesian closed  categories, the notion we take of cartesian closed category in the generalized setting. These come equipped with an ideal of types, in the sense discussed above. In section \ref{s.poly}, we introduce polynomial categories, by closely following Lambek \cite{LaK2}, and in section \ref{s.lam} we define a notion of generalized type theoretic system (lambda calculus) corresponding to the semantics we have introduced, and verify that the anticipated equivalence holds. %
In all that we have done we have closely followed %only slightly modified 
the well-established work of Lambek and others. However, our work lays the foundation for many possible avenues for further development in areas such as proof theory, programming language semantics, topos theory, and homotopy type theory. We discuss some topics for future work in section \ref{s.conc}. %We carry somewhat further the discussion of this apparently rich set of possibilities in section \ref{s.conc}. 







%Conflicting points of view. 
%1. A type is...simply the exponential. 
%		? idea: you're wrong; what you're doing is manipulating exponentials 
%		? this doesn't permit the niceties at the base: objects can no longer be 
%	? no, this cannot be correct. Consider the exponential of two objects. Then

%2. The exponential of a pair of types is the turnstile of the two. 
%		? however, for terms, it works differently.
%		? the exponential of term/type pair gives you a type (Abstraction-Beats-Concreteness Principle.)


% stupid shit from early on:
%The details of the correspondence are the main body of the paper. In this introduction, we will discuss the significance of the generalization, and summarize the proof, which can be (depending on a reader's state of mind) hard to read for its essence. 
%The main insight of this paper is the main step that leads to the generalization. It is this:
%$$ A \vdash A = A.$$
%This relation symbolizes, succinctly though cryptically, the hiding place where is found the generalized correspondence we present here. For it is only when we pass to generalized categories that we understand that the object-type correspondence in the internal language functor should be understood differently: a type should be thought of, not as an object, but as a {\em vacuous path between} two {\em objects}. This observation is really what this entire paper is about; the rest, though important, merely amounts to a verification of details. 

%entirety of the idea that is brought out once one passes to the generalized correspondence we present here, but hidden in the ordinary Curry-Howard correspondence, because of the 1-dimensionality \cite{SchMd} of the structures involved in 



%This solves the following problems:
%\enu
%	\item The identifications between types in the definition of \cite{LaSc1}. % what was I intending by this????
%	\item The principle of inhabitance  % ???? ay me...
%	\item The principle of functional ontology.  % ???? no idea...
%	\item The demands of the ``Barendregt cube'': dependent types, polymorphism, and the general higher type theory of the Calculus of Constructions \cite{CoquandCC} are supplied with a semantic basis in the generalized setting, where there is less. ????? bhtitr???? 
%	\item The constrained set of arrows in Lambek's category of cartesian closed categories. We are able to loosen this constraint. ()()()
%\Enu







 

			%\input{th2(gencat)} % gencat
%\input{th3(th)} % genth
% File th3(th).tex Created by Lucius Schoenbaum August 23, 2016
% generalized theories




















\vspace{12pt}
\section[Generalized Deductive Systems and Ideal Cartesian Closed Categories]{\onehalfspacing Generalized~Deductive~Systems and
Ideal~Cartesian~Closed~Categories}\label{s.th}

\subsection{Generalized Deductive Systems and Generalized Graphs}\label{ss.gendedsys}

\dfn\label{d.gengraph}
A {\em generalized graph} is a triple $(\sA, s, t)$, where $\sA$ is a set, %
$s, t$ are maps $\sA \to \sA$. %
%and $\sI \lies \sA$ is a subset of $\sA$ satisfying
%\enu
%	\item $a \in \sI$ implies $sa = ta$.
%	\item $a, b \in \sI$ implies $sa \neq sb$.
%	\item If $sa = ta = a$, then $a \in \sI$.
%\Enu
%We may write $sa = \bar a, ta = \hat a$ for the maps $s$ and $t$.

A {\em morphism} $\Phi: \sA \to \sB$ of generalized graphs is a mapping $\Phi$ from $\sA$ to $\sB$ such that 
\enu
	\item $\Phi(s(a)) = s(\Phi(a))$
	\item $\Phi(t(a)) = t(\Phi(a))$
\Enu
This gives a category $\bf{Graph}$ of generalized graphs.
\Dfn

An element of $\sA$ is (synonymously) an {\em edge}. %
%An element of $\sI_\sA$ is an {\em identity} in $\sA$. % 
An {\em object} in a generalized graph is an element $a \in \sA$ such that $sa = ta = a$. %
We say that a {\em subject} in a generalized graph is an element $a \in \sA$ such that there is an element $f \in \sA$ such that either $sf = a$ or $tf = a$. %
%A {\em generalized graph with identities} is a generalized graph with an identity over every subject. %a generalized graph $\sA$ equipped with a set $\sI \lies \sA$. %every subject $a$ in $\sA$ is equipped with a designated element $1_a$. %
We write $\Ob(\sA)$, $\Sb(\sA)$ for the set of objects and subjects of $\sA$, respectively. %
We say that generalized graph is {\em 1-dimensional} if 
$$ss = s \text{ and } tt = t.$$
%Same as in section \ref{s.gencat}, an {\em object} of a generalized graph $\sA$ is $a \in \sA$ such that $sa = ta = a$. 
Ordinary graphs correspond bijectively with 1-dimensional generalized graphs. %

Recall that in an algebraic system $(A, f)$ in which $A$ is a carrier set where equality $(=)$ is defined and a unary operation $f$ is defined (a mapping $A \to A$), we say that $f$ is {\em substitutive} if for all $a,b \in A$
$$a = b \eimplies fa = fb.$$
%It is normal in abstract algebra and many other subjects to {\em assume} that all operations are substitutive without saying so. %
%However, 
(The word {\em congruence} also arises frequently in connection with this property.) %
The source and target operations in a generalized graph are not assumed to be substitutive. %
(In fact, there is no notion of equality defined in the language of generalized graphs until we come to Definition \ref{d.icat}.) %
This comes with the advantage that we can apply inductive pattern-matching in proofs about elements in a generalized graph (and we may even do so constructively, if they are finitely generated in some finite language), %
though yet another hypothesis is needed if these patterns matchings are to be exhaustive in $\sA$. (Such a hypothesis will apply to polynomials in section \ref{s.poly}.) %
%However, the reader (depending on their background) might take the presence of an equality relation (and associated substitutivity for operations) for granted, so it bears mentioning
%One such language is the following:

\dfn\label{d.dedsys}
A {\em generalized typed deductive system} or simply a {\em generalized deductive system} is a structure 
$$(\sA, s, t, \cdot, \vdash, \sV),$$ 
where $(\sA, s, t)$ is a generalized graph, %$\sI$ is a subset of $\sA$, 
($\cdot$) is a partially-defined operation $\sA \times \sA \to \sA$ on $\sA$,
$(\vdash)$ is an operation $\sA \times \sA \to \sA$, %on the set of subjects\footnote{If $a \in \sA$ is not a subject, then by axiom \ref{ax.vdashst} $\vdash$ cannot be defined for argument $a$.} 
%of $\sA$, 
and $\sV$ is a subset of $\sA$, satisfying 
\enu
	\item for all $a,b \in \sA$, $b \cdot a \text{ is defined iff } ta = sb$
%	\item if $a \in \sI$, then $sa = ta$
%	\item if $a,b \in \sI$, then $sa = sb$ (or $ta = tb$) implies $a = b$
%	\item for all $a \in \sA$, there exists $b \in \sI$ such that $sb = tb = a$, \label{ax.allhaveidentities}
%	\item $a \in \sI$ for all objects
	\item $s(ab) = s(b)$ and $t(ab) = t(a)$
	\item $s(a \vdash b) = a$ and $t(a \vdash b) = b$. \label{ax.vdashst}
%	\item $(f \vdash \hat g) \cdot g = f \vdash \bar g,$ and $f \cdot (\bar f \vdash g) = f \vdash g$. 
	\item for every $a \in \sA$, $a \vdash a \in \sV$. \label{ax.identityvalid}
	\item if $a,b \in \sV$, and $a \cdot b$ $\downarrow$, then $a \cdot b \in \sV$. \label{ax.compositionvalid}
	%\item if $a,b \in \sV$, then $a \vdash b \in \sV$. \label{ax.pft}
	\item for every $a,b \in \sA$, if there exists $u \in \sV$ with $\bar u = a$ and $\hat u = b$, then $a \vdash b \in \sV$. \label{ax.inhabitance}
\Enu
%and $\sV$ is a subset of $\sA$ that contains $\sI$. %satisfying $\sI \lies \sV \lies \sA$. 
A morphism of generalized typed deductive systems $\phi: \sA \to \sB$ is a morphism of generalized graphs satisfying
\enu
	\item $\phi(a \cdot b) = \phi(a) \cdot \phi(b),$
	\item $\phi(a \vdash b) = \phi(a) \vdash \phi(b),$
	\item if $a \in \sV$, then $\phi(a) \in \sV$.
%	\item if $a \in \sI$, then $\phi(a) \in \sI$.
\Enu
This gives a category ${\bf DedSys}$ of generalized (typed) deductive systems.
\Dfn

Since we can now compose edges, we shall refer elements $a$ of a deductive system $\sA$ as edges or {\em paths} (there is no actual distinction between the two terms, except in case products in $(\cdot)$ are freely generated on a basis in $\sA$.) %
The elements of $\sV$ may be thought of as {\em valid paths} of $\sA$. %Thus, axiom (\ref{ax.identityvalid}) says that an identity is always a valid path. %
%An element in 
In the set of edges going from $a$ to $b$, the unique edge $a \vdash b$ is called the {\em type} with source $a$ and target $b$. %
We may use the notation $a \dashv b$ interchangeably to denote $b \vdash a$, thus $a\dashv b \threeline b \vdash a$. 
%
Finally, when using axiom (\ref{ax.inhabitance}) we call $u$ a {\em witness} and say that the type $a \vdash b$ is {\em inhabited} if there is found such a $u$. %
We may write $1_a$ in place of $a \vdash a$. %

%Since we allow $\vdash$ to be a mapping from all of $\sC \times \sC$, all elements of a generalized typed deductive system are subjects by axiom (\ref{ax.vdashst}), a point we will henceforth not comment on, though it often affects the forms of definitions. For example, it is %even though it has a considerable effect on the development. 
%for this reason that the principle that all subjects have identities (a convenience) becomes axiom (\ref{ax.allhaveidentities}) stating that all elements are equipped with identities. %Because of this, axiom (\ref{s.allhaveidentities}) simply reads that all elements have identities, even though it should read that all {\em subjects} have identities. (This, at least, is our point of view.) 

In our work it is possible to ignore the role of $\sV$, but its presence suggests generalizations of the calculus, for example $\sV$ %is viewed dynamically, it 
might be useful in a model of concurrency, or be impacted by modal operators.

\dfn\label{d.pidedsys}
An {\em positive intuitionistic generalized deductive system} is a generalized deductive system 
$$(\sA, s, t, \cdot, \vdash, \sV)$$ 
equipped with the additional structure
$$(\trut, \wedge, \langle,\rangle, ()^*)$$
consisting of:
\enu
	\item A distinguished element $\trut \in \sA$,
%	\item A designated element $\fals \in \sA$,
%	\item A designated element $ter_a \in \sA$ for each subject $a \in \sA$,        % <----- these can be types...
%	\item A designated element $ini_a \in _\sA$ for each subject $a \in \sA$,
	\item A mapping $\wedge: \sA \times \sA \to \sA$,
%	\item A mapping $\vee: \sA \times \sA \to \sA$,
%	\item For each pair of subjects $a,b$, distinguished elements $\pi_{a,b}, \pi'_{a,b} \sA$, %\epsilon: \Sb(\sA) \times \Sb(\sA) \to \sA$,
	\item A partially defined mapping $\langle,\rangle : \sA \times \sA \to \sA$,
%	\item A mapping $\epsilon: \Sb(\sA) \times \Sb(\sA) \to \sA$,
	\item A partially defined mapping $()^* : \sA \to \sA$
%	\item A pair of mappings $\iota_1, \iota_2: \Sb(\sA) \times \Sb(\sA) \to \sA$
%	\item A mapping $()^{()} \,: \sA \times \sA \to \sA$.
%	\item A partially defined binary operation $\langle , \rangle \,: \sA \times \sA \to \sA$.
%	\item A partially defined binary operation $[ , ] \,: \sA \times \sA \to \sA$.
%	\item A partially defined ternary operation $* : \sA \times \sA \times \sA \to \sA$, denoted $a*(b,c)$.
\Enu
subject to the following axioms,
\enu
%	\item at least one pi for each pair of subjects
%	\item epsilon defined when  		never mind
	\item $\langle a,b \rangle$ is defined if and only if the source of $a$ and $b$ are identical.
	\item $a^*$ is defined if and only if the source of $a$ is the wedge of two subjects in $\sA$.
\Enu
the following source and target conditions:
\enu
	\item $\hat\trut = \bar\trut = \trut$, %$\hat{\trut} = \bar{\trut} = \trut.$ % $\trut$ is an object.
%	\item $\hat{\fals} = \bar{\fals} = \fals.$ % $\fals$ is an object.
	\item $s(a \wedge b) = s(a) \wedge s(b)$ and $t(a \wedge b) = t(a) \wedge t(b)$, 
%	\item st for pi 1
%	\item st for pi 2
	\item $s(\langle a , b \rangle) = s(a)$ and $t(\langle a, b \rangle) = t(a) \wedge t(b)$
%	\item st for epsilon
	\item $s(a^*) = proj_1(s(a))$ and $t(a^*) = proj_2 (s(a)) \vdash t(a)$, where $proj_1$ and $proj_2$ are the projections on wedge ($\wedge$) products.
\Enu
and the following {\em rules}, or {\em validities:} For all $a,b \in \sV$,
\enu
	\item $\trut \in \sV$,
%	\item for every subject $a \in \sA$, $\trut \vdash a \in \sV$, % ter_a
	\item $a \wedge b$ is valid if $a$ and $b$ are valid,
	\item $\langle a, b \rangle$ is valid if $a$ and $b$ are valid,
%	\item epsilon is valid
	\item $a^*$ is valid if $a$ is valid,
	\item for every pair of subjects $a,b \in \sA$, the following types are valid: \label{ru.validatoms}
		\enu
			\item $\trut \vdash a$
			\item $a \wedge b \vdash a$ 
			\item $a \wedge b \vdash b$ % \pi_{a,b},\pi'_{a,b}
			\item $(a \vdash b) \wedge a \vdash b$ 
		\Enu
\Enu
%	\item $\trut \in \sV$. 						----> follows because \trut is an object
%	\item $\fals \vdash a \in \sV$.
%	\item $\pi'_{a,b}, \pi_{a,b}, \iota_{a,b}, \iota'_{a,b} \in \sV$.
%	\item If $a*(b,c)$ a.d. and $a \in \sV$, then $a*(b,c) \in \sV$.
%	\item If $\langle a, b \rangle$ a.d.~and $a, b \in \sV$, then $\langle a,b \rangle \in \sV$.
%	\item If $[a,b]$ a.d.~and $a,b \in \sV$, then $[a,b] \in \sV$.
%	\item $\epsilon(a,b)$ a.d.~implies $\epsilon(a,b) \in \sV$.
%	\item If $f \wedge g \in \sV$.
%	\item $f \vee g \in \sV$.
%	\item 
%\Enu 
A morphism $f: \sA \to \sB$ of positive intuitionistic generalized deductive systems is a morphism of generalized deductive systems satisfying 
\enu
	\item $F(\trut) = \trut,$
%	\item $F(ter_a) = ter_{F(a)},$
	\item $F(a \wedge b) = F(a) \wedge F(b),$
%	\item $F(\pi_{a,b}) = \pi_{F(a),F(b)},$
	\item $F(\langle f, g \rangle ) = \langle F(f), F(g) \rangle,$
	\item $F(f^*) = F(f)^*.$
\Enu
This gives a category ${\bf p.i.DedSys}$ of positive intuitionistic generalized deductive systems. 
\Dfn

In order to form complex expressions out of simple ones, it is convenient to have names for individual elements of $\sA$. % 
For example, we choose (applying rule \ref{ru.validatoms}) valid elements of $\sA$
	$$ter_a : a \to \trut$$
	$$\pi_{a,b} : a \wedge b \to a$$
	$$\pi_{a,b} : a \wedge b \to b$$
	$$\epsilon_{a,b} : (a \vdash b) \wedge a \to b$$
Note that these elements may themselves be types, even though we usually think of the types as valid due to the existence of a witness and use of axiom \ref{ax.inhabitance} of Definition \ref{d.dedsys}. %
By {\em term} (or {\em global element}) of a deductive system we refer to any element of a deductive system whose source is $\trut$. 

One may use the deductive system to show the validity of Heyting's axioms for intuitionistic logic (those that do not contain the $\lor$ and $\bot$ connectives), showing that any type that may be interpreted as a valid proposition of intuitionistic logic has a witness. The following types, for example, are inhabited.
\enu
	\item $a \vdash a \wedge a$
	\item $a \vdash a \wedge \trut$
	\item $((a \wedge b) \vdash c) \vdash (a \vdash (b \vdash c))$
	\item $(a \vdash (b \wedge c)) \dashv \vdash ((a \vdash b) \wedge (a \vdash c))$
\Enu
where $\dashv \vdash$ denotes that the type is bi-inhabited (or there is a valid path going in either direction). %
%None of these properties depend on the axiom (\ref{ax.pft}); it apparently has no correlate in the theory of cartesian closed categories. %or in the intuitionistic propositional calculus. %
%It indicates a proof-theoretic extension of intuitionistic propositional calculus, one which (if we apply the BHK interpretation) is constructively valid. %
%We use axiom (\ref{ax.pft}) in Theorem \ref{t.dedt}.

%WAIT. IF SOURCE AND TARGET ARE NOT SUBSTITUTIVE OPERATIONS, THEN WE CAN MODIFY THIS STATEMENT, CAN'T WE? FOR EXAMPLE WE CAN LET PI AND PI' NOT DEPEND ON ANY PARAMETERS.
%$\to$ NO, IT'S NOT CORRECT. 
%$\to$ BUT...IF $(AB)C = A(BC)$ IS A VALID PATH...SHOULDN'T I BE ABLE TO APPLY THAT AND REWRITE THINGS I FIND IN MY DEDUCTIVE SYSTEM??????
%Because of axiom \ref{ax.????}, we can think of the types as sitting in an ``ideal'' within the graph. If we think of addition and multiplication.
%It is important that source and target are not substitutive in a deductive system. Therefore 
%We can write $a^*$ in place of $a*(b,c)$ if the meaning is clear from the context.
%If we pause to reflect for a moment, this definition tells us a few things, for example: 
%\enu
%	\item Every well-formed expression in the structures ()()() produces a witness for a type.
%	\item
%\Enu

%We have written the source and target axioms down explicitly is because of the added difficulty that generalized categories present over ordinary categories. 

%The BHK interpretation is the following.
%\enu
%	\item There is a certificate in $\sA$ to pass from anything % including another certificate...
%	to $\trut$. (Or: a solution of something can be transformed to a solution of $\trut$, cf. \cite{KolmogorovBHK}.)
%	\item There is a certificate to pass from $a \wedge y$ to $a$, as well as to $y$.
%	\item If there is a certificate to pass from $z$ to $a$ and from $z$ to $y$, then there is a certificate to pass from $z$ to $a \wedge y$. 
%	\item ...
%\Enu

%Finally, we add the aforementioned conjectures, or {\em types}:

%\subsection{Categories}\label{ss.cat}

%We first make an obvious definition that will clarify the presentation.

\subsection{Categories Equationally Defined}\label{s.cat}
%Categories form the semantic model of deductive systems. %
Lambek \cite{LaK1c,LaSc1} observed that categories are obtained from deductive systems via a set of equational axioms. %
%In order to have a language for discussion, we modify the definition of a category. 
In this section we will develop Lambek's formalization in the setting of generalized categories. %
It is clear that any ordinary category (or generalized category) can be made into a ``typed deductive'' category. %
Simply take all arrows to be valid and introduce $\vdash$ as a free operation %(you may think of the element $a \vdash b$ if you wish as being something like the hom set of $a$ and $b$.) %The following may or may not be helpful: 
Observe that if composition $(\cdot)$ is viewed as multiplication and $\wedge$ is viewed as an additive product on the subjects of $\sC$, the set of elements of the form $a \vdash b$ behaves like a (ring-theoretic) ideal in the category. Thus if we are thinking of a category, we may think next of introducing an ``ideal of types'' to the category. %
This demands we introduce a further technicality, a set of constants. %
%Because there remains an ordinary notion of type for a terms, however, we will call the elements not types, but {\em bitypes}, to avoid confusion. 

\dfn\label{d.icat}
A {\em ideal category} or {\em ideal generalized category} is a structure $(\sC, \vdash, \sV)$ consisting of a generalized category $\sC$ (section \ref{s.gencat}), %
a distinguished subset of elements $\sV \lies \sC$, %that includes all identities of $\sC$, %
a distinguished subset of elements $\sK \lies \sC$,
and an operation $\vdash: \sC \times \sC \to \sC$ such that 
\enu
	\item $\verywidehat{f \vdash g} = f,$
	\item $\overline{f \vdash g} = g.$
%	\item for all $f,g \in \sC$, $f \vdash g$ is not in $\sK$, \label{ax.kone}
%	\item for all $k \in \sK$, $f \in \sC$, if $k \cdot a \downarrow$ then $k \cdot a \in \sK$, \label{ax.ktwo}
%	\item for all $k \in \sK$, $f \in \sC$, if $a \cdot k \downarrow$ then $a \cdot k \in \sK$ unless there exists $g,h \in \sC$ such that $a = g \vdash h$, \label{ax.kthree} \stuffgoeshere % fix, cf. page 12 of 11/1/16 draft
	\item $f \cdot (\bar f \vdash g) = \hat f \vdash g,$ unless $\bar f = g$, in which case $f \cdot (\bar f \vdash g) = f$, or unless $f \in \sK$ or $\bar f \vdash g \in \sK$. \label{ax.vdash1}
	\item $(f \vdash \hat g) \cdot g = f \vdash \bar g,$ unless $f = \hat g$, in which case $(f \vdash \hat g) \cdot g = g$, \label{ax.vdash2}
	\item if $g \cdot f$ $\downarrow$, and $g,f \in \sV$, then $g \cdot f \in \sV$,
	\item $f \vdash f \in \sV$ for all $f \in \sC$,
	\item (witnesses) $u \in \sV$ implies $\hat u \vdash \bar u \in \sV$.
	\item $\vdash$ is substitutive (Section \ref{ss.gendedsys}) in both arguments. 
\Enu
A functor $F: \sC \to \sD$ between generalized ideal categories is an ordinary functor (section \ref{s.gencat}) which preserves validity and $\vdash$:
\enu
	\item $f \in \sV_\sC \eimplies F(f) \in \sV_\sD,$
	\item $F(f \vdash g) = F(f) \vdash F(g)$. \label{ax.deductivefunctor2}
\Enu
%Hereafter we call such a functor an {\em ideal functor}. %
This defines a category ${\bf IdealCat}$. 
\Dfn

%Axioms \ref{ax.kone}-\ref{ax.vdash2} say that both $\sK$ and the set of elements of the form $f \vdash g$ are absorbtive on the right, while on the left, both are absorbtive up to the limitation in place due to the stronger absorbtivity of the opposite class on the right. 
By axioms \ref{ax.vdash1} and \ref{ax.vdash2}, for $f \in \sC$, $f \vdash f$ is the identity of $f$, which may be denoted $1_f$. %
In particular, all elements (including identities) of an ideal category have identities. 
The identities, types, and constants figuring here will arise again in Section \ref{s.lam}, where we encounter the symbols $\settl{x}{x}$ and $\settl{x}{y}$. %

%This latter condition on functors can be compared to the condition arising in Lambek's version of the Curry-Howard correspondence \cite{LaK1c}: 
%Lambek shows that there is an equivalence between the category of simply typed $\lambda$-calculi (with translations as arrows, \cite{LaSc1}) and the category of cartesian closed categories, with arrows the functors that preserve the cartesian closed structure on the nose. 
%$$F(\trut) = \trut$$
%$$F(ter_X) = ter_{F(X)}$$
%$$F(X \wedge Y) = F(X) \wedge F(Y)$$
%$$F(\langle a, b \rangle) = \langle F(a), F(b) \rangle$$
%$$F(\pi_{X,Y}) = \pi_{F(X),F(Y)}$$
%$$F(\pi'_{X,Y}) = \pi'_{F(X), F(Y)}$$
%$$F(X^Y) = F(X)^{F(Y)}$$ 
%$$F(\epsilon_{X,Y}) = \epsilon_{F(X), F(Y)}$$
%$$F(a^*) = F(a)^*$$
%for objects $X,Y$, and arrows $a,b$.

%The following definition is essentially that of a cartesian closed category, except that we have incorporated our ideal of types as defined in definition \ref{d.idcat}. In order to 

\dfn\label{d.iccc}
An {\em ideal cartesian closed category} is a ideal category with identities $\sC$ that is equipped with a structure
$$(\trut, \wedge, \langle,\rangle, ()^*)$$
where
\enu
	\item $\trut$ is a distinguished valid element in $\sC$,
	\item $\wedge$ is an operation $\sC \times \sC \to \sC$,
%	\item $\set{ter_a}$ is a set of valid elements 
%		$$ter_a : a \to \trut$$
%	of $\sC$ indexed by $\Sb(\sC)$, 
%	\item $\set{\pi_{a,b}}$ and $\set{\pi'_{a,b}}$ are sets of valid elements
%		$$\pi_{a,b} : a \wedge b \to a, \quad\quad\quad \pi'_{a,b} : a \wedge b \to a$$
%		indexed by pairs in $\Sb(\sC) \times \Sb(\sC)$.
	\item $\langle, \rangle$ is a partially defined operation $\sC \times \sC \to \sC$
%	\item $\set{\epsilon_{a,b}}$ is a set of valid elements
%		$$\epsilon_{a,b} : (a \vdash b) \wedge a \to b$$
	\item $()^*$ is a partially defined operation $\sC \to \sC$
\Enu
%with validities 					% these are taken care of since blablabla defines a positive intuitionistic deductive system on \sC (see below)
%\enu
%	\item $\trut \in \sV$,
%	\item if $g,f \in \sV$, then $g \wedge f \in \sV$,
%	\item $
%\Enu
which satisfies the conditions:
\enu
	\item $\trut \in \sK$, and $\sK$ is closed under $\wedge, \langle,\rangle, $ and $()^*$,
	\item the structure
		$$(s,t,\sV,\vdash, \cdot, \trut, \wedge, \langle,\rangle, ()^*)$$
		defines a positive intuitionistic deductive system on $\sC$.
	\item for all $a \in \sC$, if $f: a \to \trut$ then $f = (a \vdash \trut)$. \label{ax.terminalar}
%	\item for all $a \in \sC$, there is a $k \in \sK$ such that $k: a \to \trut$, and if $f: a \to \trut$, either $f = a \vdash \trut$ or $f = k$.
	%for all $f:a \to \trut$, $f = a \vdash \trut$. 
%	\item pi and pi'
%	\item epsilon
%	\item For each pair of subjects $a,b \in \sC$ there exists a set !!!!!!!!!!!! FIX---CAN'T
%		$$\set{\pi_{a,b}:a \wedge b \to a, \quad \pi'_{a,b} : a \wedge b \to b, \quad \epsilon_{a,b}: (a \vdash b) \wedge a \to b}$$ 
%		that together satisfy:
	\item For every pair $(a,b)$ of subjects of $\sC$, there exists a good pair $(\pi, \pi')$ for $(a,b)$. 
	\item For every good pair $(\pi, \pi')$ for any pair of subjects $(a,b)$, there is a good evaluation $\epsilon = \epsilon_{\pi, \pi'}$ for $(\pi, \pi')$. 
\Enu
Here, if $(a,b)$ is a pair of subjects of $\sC$, then a pair $(\pi, \pi')$ of elements of $\sC$ are a {\em good pair for $(a,b)$} if
			\enu
				\item $\pi$ and $\pi'$ are valid,
				\item $\pi: a \wedge b \to a$, and $\pi': a \wedge b \to b$,
				\item if $\pi \langle f, g \rangle$ $\downarrow$ then $\pi \langle f, g \rangle = f,$
				\item if $\pi' \langle f, g \rangle$ $\downarrow$ then $\pi' \langle f, g \rangle = g,$
				\item if $\langle \pi f, \pi' f \rangle$ $\downarrow$ then $\langle \pi f, \pi' f \rangle = f$, %for all $f: c \to a$, $g: c \to b$, $h: c \to (a \wedge b)$.
				\item if $f \cdot \pi$ and $g \cdot \pi'$ $\downarrow$, then $\langle f \cdot \pi, g \cdot \pi' \rangle = f \wedge g$. 
			\Enu
and a {\em good evaluation} for a good pair $(\pi, \pi')$ for a pair of subjects $(a,b)$ is an element $\epsilon = \epsilon_{\pi, \pi'}$ of $\sC$ that satisfies, for every $c \in \sC$ and every good pair $(\pi_{c,a}, \pi'_{c,a})$ for $(c,a)$,
			\enu
				\item $\epsilon$ is valid,
				\item $\epsilon: (a \vdash b) \wedge a \to b$,
				\item if $\epsilon \cdot \langle f^* \cdot \pi_{c,a}, \pi'_{c,a} \rangle$ $\downarrow$ then $\epsilon \cdot \langle f^* \cdot \pi_{c,a}, \pi'_{c,a} \rangle = f,$ 
				\item if $(\epsilon \cdot \langle f \cdot \pi_{c,a}, \pi'_{c,a} \rangle )^*$ $\downarrow$ then $(\epsilon \cdot \langle f \cdot \pi_{c,a}, \pi'_{c,a} \rangle )^* = f$. %for all $h: c \wedge b \to a$ and $k: c \to (a \dashv b).$
			\Enu
A morphism $F: \sC \to \sD$ between ideal cartesian closed categories $\sC$ and $\sD$ is a functor of ideal categories satisfying
\enu
	\item $F(\trut) = \trut$,
%	\item $F(ter_a) = ter_{F(a)}$,
	\item $F(a \wedge b) = F(a) \wedge F(b),$
	\item $F(\langle a , b \rangle) = \langle F(a), F(b) \rangle,$
	\item $F$ sends a good pair in $\sC$ to a good pair in $\sD$. %For every $(a,b)$ of subjects and for every good pair $(\pi, \pi')$ for $(a,b)$, 
%	$$F(\pi_{a,b}) = \pi_{F(a),F(b)}, \text{ and } F(\pi'_{a,b}) = \pi'_{F(a), F(b)}.$$
%		for every good pair $(\pi, \pi')$ of
%	\item $F(\epsilon_{a,b}) = \epsilon_{F(a),F(b)},$
%	\item $F(a^*) = F(a)^*.$
\Enu
Thus we have a category ${\bf ICCC}$ of ideal cartesian closed categories.
\Dfn

Axiom \ref{ax.terminalar} is relevant when the possibility exists that the element $a \vdash \trut$ might be a constant. 
%
%In this writing 
We continue to use the notation of deductive systems in a category $\sC$. Note that many authors write $\times$ for the binary product, which we continue to denote $\wedge$, and $1$ for the terminal object, which we continue to denote $\trut$. This seems appropriate as we will never stray far from the point of view provided by deductive systems and the lambda calculus. 

Note that morphisms of ideal cartesian closed categories are stronger maps than ordinary functors between categories that happen to be cartesian closed. For ordinary categories, these functors are sometimes called {\em cartesian functors}. %
%
It is easy to see that a good pair $(\pi, \pi')$ for a pair of subjects $(a,b)$ is unique if it exists. Hence a good evaluation $\epsilon \threelines \epsilon_{\pi, \pi'}$ depends only on $(a,b)$ and may be denoted $\epsilon_{a,b}$. %, 
Similarly, we often write $\pi \threelines \pi_{a,b}$ and $\pi' \threelines \pi'_{a,b}$. It follows that $F(\pi_{a,b}) = \pi_{F(a), F(b)}$, and similarly for $\pi'$.

\prop\label{p.basiciccc}
The following hold in ideal cartesian closed categories:
\enu
%	\item $\trut = (\trut \vdash \trut) = 1_\trut$.
	\item $\langle f, g \rangle \cdot h = \langle f \cdot h , g \cdot h \rangle$
	\item $1_a \wedge 1_b = 1_{a \wedge b}$
	\item $(f \wedge g) \cdot (f' \wedge g') = (f \cdot f') \wedge (g \cdot g')$
	\item $\epsilon_{a,b}^* = 1_{a \vdash b}$
%	\item $h^* \cdot k = (h \cdot \langle k \cdot \pi_{d,b}, \pi'_{d,b}\rangle)^*$
	\item $f^* \cdot g$ $\downarrow$ implies $f^* \cdot g = (f \cdot \langle g \cdot \pi, \pi' \rangle)^*$, where $(\pi, \pi')$ is the obvious good pair.
\Enu
\Prop

\prop
A morphism $F: \sC \to \sD$ between ideal cartesian closed categories preserves the evaluation $\epsilon$ and adjoint operation $()^*$. %
\Prop
\prf
%The idea is that star is preserved automatically because it's a functor, and therefore although epsilon is not preserved on the nose, you get something on the other side that satisfies the axioms, and therefore, you can rename things and get a morphism that is preserving things ''on the nose''. 
By functoriality, we have $F((f: a \wedge b \to c)^*) = F(f^*) : F(a) \to F(b \vdash c) = F(f^*) : F(a) \to (F(b) \vdash F(c))$. But this latter expression is $F(f)^*$, so $F(f^*) = F(f)^*$. %
%
It follows that evaluations $\epsilon$ are also preserved. Indeed, if $(a,b)$ are chosen and $(\pi, \pi')$ is a good pair for $(a,b)$, then choose a good evaluation $\epsilon = \epsilon_{a,b}$ for $(\pi, \pi')$. %
Then 
$$F(\epsilon)^* = F(\epsilon^*) = F(1_{a \vdash b}) = 1_{F(a) \vdash F(b)}.$$
Hence 
\begin{align*}
\epsilon_{F(a), F(b)} 	&= \epsilon_{F(a), F(b)} \cdot \langle 1_{F(a) \vdash F(b)} \pi_{F(a) \vdash F(b) ,a}, \pi'_{F(a) \vdash F(b),a} \rangle  \\
				&= \epsilon_{F(a), F(b)} \cdot \langle F(\epsilon_{a,b})^* \pi_{c,a}, \pi'_{c,a}\rangle  \\
				&= F(\epsilon_{a,b}),
\end{align*}
by the good evaluation properties of $\epsilon_{a,b}$. 
%and check that $F(\epsilon \cdot \langle f^* \cdot \pi_{c,b}, \pi'_{c,b} \rangle) = F(f) = F(\epsilon) \cdot \langle F(f)^* \pi_{F(c), F(b)}, \pi'_{F(c), F(b)} \rangle$. The other axiom is similar. 
%Then if you define it the clever way, voila, everything works... but is it too clever by half? 
%\stuffgoeshere
%\stuffgoeshere
\Prf

%We motivate definitions \ref{d.icat} and \ref{d.iccc} by the following:

Next we present a few ways to produce ideal cartesian closed categories:

\prop\label{p.cattoicat}
%Let $\sC$ be a cartesian closed category. Then there is an ideal cartesian closed category $\tilde \sC$ 
There is an (in general, nonconstructive) functor from the category ${\bf CCC}$ of cartesian closed categories to the category {\bf ICCC}. 
%The category of cartesian closed categories may be embedded into the category of ideal cartesian closed categories, in such a manner that it becomes a reflective subcategory. 
\Prop
\prf
Let $F: \sC \to \sD$ be a functor in the category of cartesian closed categories (of the ordinary sort). We carry out the following construction on both $\sC$ and $\sD$; first take $\sC$. Take any new pair of identifiers $\vdash$ and $\wedge$. For each object $X$ of $\sC$, form, via recursion, the collections of triples
$$\sA_X = \set{(Y_1, Z_1, \vdash) \mid \text{ there exists } Y, Z \in \Ob(\sC) \text{ such that } X = Z^Y \eand Y_1 \in \sC_Y, Z_1 \in \sC_Z}$$
$$\sB_X = \set{(Y_1, Z_1, \wedge) \mid \text{ there exists } Y, Z \in \Ob(\sC) \text{ such that } X = Y \wedge Z \eand Y_1 \in \sC_Y, Z_1 \in \sC_Z}$$
$$\sC_X = \sA_X \cup \sB_X.$$
We take
$$\Ob(\tilde{\sC}) = \Ob(\sC) \cup \bigcup_{X \in \Ob(\sC)} \sC_X,$$
and for each $V \in \Ob(\tilde{\sC})$ we assume given from the construction of the $\sC_X$'s a function $\deflate(V)$ defined by
$$\deflate(V) = \begin{cases} V, & \eif V \in \sC, \\ Z^Y, &\eif V \in \sA_X \text{ for some $X$}, \\ Y \wedge Z, & \eif V \in \sB_X \text{ for some $X$}. \end{cases}$$
For every $U,V \in \tilde \sC$, define
$$\hom(U,V) := \hom(\deflate(U), \deflate(V)),$$
with composition and identities defined in the obvious way, in particular 
$$\deflate(f \cdot g) := \deflate(f) \cdot \deflate(g),$$
where $\deflate(f)$ for a morphism $f$ is defined in the obvious way analogous to $\deflate()$ on objects. 
The reader can now check that the symbols in Definition \ref{d.iccc} may be introduced and the axioms verified, and that we may extend $F$ to a functor $\tilde F: \tilde \sC \to \tilde \sD$ that satisfies the conditions of Definitions \ref{d.icat} and \ref{d.iccc}. 
\Prf

Another result that gives examples of ideal cartesian closed categories is:

\prop
Let $\sE$ be a generalized category of generalized presheaves over a generalized category $\sC$. Then $\sE$ is an ideal cartesian closed category. 
\Prop
\prf 
See Chapter \ref{c.it}. 
\Prf

The adjunction that holds in a cartesian closed category, because the mappings $- \times X$ and $-^X$ are no longer functors in the generalized setting. However, we do have:

\prop
\enu
	\item there is a bijection  
		$$\hom(c \wedge b, a) \bij \hom(c, a^b)$$
	\item there is a bijection
		$$\hom(a,b) \bij \hom(\trut, b^a)$$
%	\item 
\Enu
Let $a \iso b$ denote that there exists a pair of elements $f:a \to b$ and $g:b \to a$ such that $fg = 1_b$ and $gf = 1_a$. Then in an ideal cartesian closed category
\enu
	\item $(a \wedge b) \vdash c \iso (a \vdash b) \vdash c$
	\item $a \vdash (b \wedge c) \iso (a \vdash b) \wedge (a \vdash c)$
\Enu
\Prop
\prf
See \cite{LaSc1}. 
\Prf

Given $f:a \to b$ we write 
$$\name f $$ % name f 
for the induced term $1 \to a \vdash b$, called (Lawvere's terminology) the {\em name} of $f$. 
%
%Given a term $g:\trut \to (a \vdash b),$ we write
%$$\unname g$$ % unname f
%for the induced edge $g: a \to b$.

Finally, we relate deductive systems to categories as follows:

\prop
Every deductive system $\sA$ on which there is defined an equivalence relation denoted $=$, and a distinguished subset $\sK$ of constants in $\sA$, with respect to which the following statements are satisfied:
\enu
	\item $f \cdot (\bar f \vdash g) = \hat f \vdash g,$ unless $\bar f = g$, in which case $f \cdot (\bar f \vdash g) = f$, unless $f$ is constant or $\bar f \vdash g$ is constant,
	\item $(f \vdash \hat g) \cdot g = f \vdash \bar g,$ unless $f = \hat g$, in which case $(f \vdash \hat g) \cdot g = g$, 
%	\item $f 1_{\bar f} = f$, if $\bar f$ has an identity $1_{\bar f} \in \sI,$
%	\item $1_{\hat f} f = f$, if $\hat f$ has an identity $1_{\hat f} \in \sI,$
	\item $(hg)f = h(gf)$ for all composable $f,g,h \in \sA$,
	\item $a = b$ implies $s(a) = s(b)$,
	\item $a = b$ implies $t(a) = t(b)$,
	\item $a = b$ implies $ca = cb$ and $ac = bc$, for all composable $c$,
	\item $a = b$ implies $a \vdash c = b \vdash c$ and $c \vdash a = c \vdash b$, for all $c$ in $\sA$,
%	\item $a = b$ implies $a$ has an identity iff $b$ has an identity, and $1_a = 1_b$.
	%$1_a = 1_b$
\Enu
is an ideal generalized category (in particular, a generalized category), taking $\sV$ to be the valid paths in $\sA$. 
\Prop
\prf
We check the axioms of Definition \ref{d.gencat} and see that they may be verified using axioms and rules of Definitions \ref{d.dedsys} and \ref{d.icat}. %
\Prf

% this statement is redundant because we **define** ccc in terms of Lambek's equational axioms:
\begin{comment}
\prop
Every positive intuitionistic d-system $\sA$ in which
\enu
	\item unique ter
	\item pi and univ
		\enu
			\item pi of univ is first
			\item pi of univ is second
			\item univ of pi's is itself
		\Enu
	\item epsilon of univ of h star of pi, pi is h
	\item star of epsilon of univ of k of pi, pi' is k. if $k:C \to a^b$.
\Enu
is a ideal cartesian closed category. 
\Prop
\prf
\stuffgoeshere
\Prf
\end{comment}


%Unlike in the case of ordinary categories, in the generalized categorical setting, these basic properties do not imply that a cartesian closed structure as we have defined it gives rise to a cartesian closed structure on the underlying category $\sC$, if the latter notion is defined in the obvious way using the generalized notions of product, terminal, and adjunction of section \ref{s.gencat}. %
%Indeed, 
The notion of a cartesian closed category cannot be extended to the generalized setting: the mapping $X \mapsto X \times Y$ is a functor only when $X$ is an object. %
Our approach is to allow %respond to this by allowing 
the mapping on the other side, $Z \mapsto Z^Y$, to fail to be a functor as well. %
This is possible thanks to Lambek's formalization: %
We are able, by following Lambek, to derive a calculus of cartesian closed categories in the generalized setting, in spite of the weaker underlying structure. %


%When we do so, we preserve the Lambek formalization, which is strong enough for applications as our derivations (following Lambek) have shown. %
%The Lord giveth, and the Lord taketh away! %

%Finally, to address the apparent abstraction of an ideal cartesian closed category (definition \ref{d.iccc}), we show that such structures exist in relative abundance:
%\prop
%Given an (ordinary) cartesian closed category $\sC$, there exists an ideal cartesian closed category $\tilde \sC$ that is ()()()()????
%Given functor $F : \sC \to \sD$ between cartesian closed categories, there is an ideal functor $\tilde F: \sC \to \sD$ such that ()()())(????
%\Prop
%\prf
%\stuffgoeshere
%\Prf
































%\input{th4(poly)} % polynomials
% File th4(poly).tex Created by Lucius Schoenbaum October 18, 2016
% polynomials




















\section{Polynomials and Lambda-Calculi}\label{s.poly}

%The most important idea contained in the Curry-Howard correspondence is the observation that, 
Adding variables to a deductive system with a positive intuitionistic structure reduces, by the Deduction Theorem (Theorem \ref{t.dedt}), validity of all paths to the validity of paths from a terminal object. Therefore the focus shifts from the space to the {\em polynomials over the space}, in the sense we now define. %
% canonical source, which can be anything we wish. 
% comment: CuHoT becomes like a Nullstellensatz.

\subsection{Polynomials Systems and Polynomial Categories}\label{ss.poly}
The notion of indeterminate may be applied in this setting just as it may be applied in the setting of groups, rings, and fields. However, %(as takes some getting used to) 
we must assign a source and target to each new indeterminate. It is convenient to let the source of every indeterminate be $1$, the (fixed choice of) terminal object. This does not mean we cannot substitute a variable with a different source for the indeterminate---substitution of, say, $a$ for $x$ in $\phi(x)$ is allowed whenever $x$ and $a$ have the same target; the source of $a$ is irrelevant. %
In this sense, it is more correct (but less convenient) to say that an indeterminate simply does not have a source. %
We denote an indeterminate over a deduction system $\sA$ by symbols $x, y, z, $ etc. 
% POLYNOMIALS CAN'T BE SOURCE AND TARGET??????:
For now, we require that the target of $x, y, \dots$ is in $\sA$. (In particular, it cannot itself be a polynomial). %though this might be possible in a more elaborate development, leading to a very general form of polymorphism.) %
A more general system might allow indeterminates over polynomials and make use of the notion of {\em telescope} \cite{deBruijn1}, but we will have no need for this added generality. 

\begin{comment}
% TELESCOPE STUFF
\dfn
A polynomial {\em depends on} an indeterminate $x$ if it or anything in its s/t tree depends on $x$.

A polynomial {\em strongly depends on} an indeterminate $x$ if, ignoring its s/t structure, it depends on $x$, that is, it has $x$ in an expression that defines it.  (This notion is invariant under equality.) 

Define {\em does not depend on}, {\em does not strongly depend on}, etc. in the obvious way. 
\Dfn

By iterating the construction we may define general polynomial systems $\sA[\vec{x}]$ and general polynomial categories $\sC(\vec{x})$. %By using this notation it is always understood that $\vec x$ denotes a telescope of indeterminates over $\sC$. 
Finally, we say that a {\em polynomial over $\sC$} is an element of $\sC(\vec x)$ for any finite sequence of indeterminates $\vec x$. %

%ITERATING: DEFINE 
%$$WF$$
\dfn\label{d.wf}
A finite sequence $(x_1, x_2, \dots, x_n)$ of indeterminates is {\em well-formed} if for all $i \geq 1$, for all $j < i$, $x_j$ does not depend on $x_i$. 

Let $WF$ be the set of all well-formed sequences of indeterminates. 
\Dfn

%Hence, given a list of indeterminates $x_1, \dots, x_n$ pulled from the air, we must require that they be well-formed, or in other words, a {\em telescope} \cite{GETCITEFROMNORELLTHESIS}. A finite sequence $x_1, \dots, x_n$ of indeterminates is a {\em telescope} if for all $i \geq 1$, $j > i$, $x_j$ does not depend on $x_i$. Note this condition applies in the strong sense that neither the target of $x_j$, nor the target or source of it, nor...and so on depend on $x_i$. 

% // TELESCOPE STUFF
\end{comment}

\dfn\label{d.poly}
Let $\sA$ be a positive intuitionistic deductive system. %
Let $x$ be an indeterminate with target $\hat x$ in $\sA$.
We write $\sA[x]$ for the positive intuitionistic deduction system freely % NB
generated on the set $\sA \cup \set{x}$. %
This means that
\enu
	\item Operations on $\sA$ of Definition \ref{d.pidedsys} are extended from $\sA$ to $\sA[x]$ by free generation on expressions $\phi$ containing any instance of $x$: %also adding identities: 
		$$\phi ::= f \,\mid\, x \,\mid\, %1_\phi \,\mid\, 
		\phi \vdash \phi \,\mid\, \phi \wedge \phi \,\mid\, \langle \phi, \phi \rangle \,\mid\, \phi^* $$
	where $f$ can be any element of $\sA$, and $x$ is any indeterminate. Expressions so generated that do not contain any instance of $x$ are thrown out, and the set of all elements of $\sA$ is then added back in. 
	\item The valid elements of $\sA[x]$ are $x$, those of $\sA$, and those generated from $x$ and those of $\sA$ using the validities of Definition \ref{d.pidedsys}.
\Enu
There is an obvious embedding of $\sA$ in $\sA[x]$, via which we will usually view $\sA$ as a subset of  $\sA[x]$. 
%\stuffgoeshere
%We also impose that $x$ itself is valid in $\sA[x]$. [[[[[[[[[ How does validity extend to $\sA[x]$???
\Dfn

We call elements of $\sA[x]$ synonymously {\em polynomials over $\sA$}. %
We write $\phi, \psi, \dots$ to denote polynomials in $\sA[x]$. %
We do not normally write the variable $x$ as in $\phi(x),$ etc. as many authors do, but this should not lead to any confusion as long as it is understood what may depend on $x$.
% 
When we iterate to form $\sA[x][y]$, etc., we again require that the source and target of indeterminates be in $\sA$. Given indeterminates $x_1, x_2, \dots, x_n$, we denote by $\sA[x_1, \dots, x_n]$ or $\sA[\vec x]$ the iterated construction $(\dots((\sA[x_1])[x_2]) \dots [x_n])$. 


%\dfn
%A {\em proof by assumption} is a valid element in $\sA[x]$. 
%\Dfn

We could define a ``proof'' to be a valid path from the terminal object $\trut$ in a positive intuitionistic deductive system (say). Then we could ask what structure might allow us to ``discharge'' assumptions, as is done in natural deduction systems (see for example \cite{TrSc1}). %In order to avoid confusion with the notion of proof that we use in the usual metalanguage of mathematics, we refer to the formal notion as a {\em proofpath} (in \cite{LaSc1} it is simply called a {\em proof}): 
%\dfn
%A {\em proofpath} of $f \in \sA$ is a valid path from $\trut$ to $f$. 
%\Dfn
To refine the question, one may consider a proof $\phi$ of $f \in \sA[x]$, for $f \in \sA$. 
This would be a path through the deductive system that is allowed to ``use'' the ``assumption'' $x$. %
%Since $\phi$ is a path beginning at $\trut$, we may infer that $x$ has source $\trut$, whenever we consider elements of $\sA[x]$ as proofpaths (for some choice of $\sA$ and $x$).
%BUT WAIT HOW DO I KNOW THAT THE INDETERMINATE STARTS AT $\trut$? DO I ALWAYS KNOW THIS?????
%\stuffgoeshere
%\Dfn
In logic, the following result is, by long tradition, known as the Deduction Theorem. It is interpreted as an introduction rule when the construction of polynomials is interpreted as establishing a context. Note that polynomials do not necessarily have an element of $\sA$ as source and target, so the quantifiers on $a$ and $b$ are a significant part of the statement. %(These ``higher'' polynomials arise in \cite{ScMd}.)
%

\thm\label{t.dedt}
Let $\sA$ be a positive intuitionistic deductive system. 
Then for all $a,b \in \sA$, $a \vdash b$ is valid in $\sA[x]$ if and only if $\hat x \wedge a \vdash b$ is valid in $\sA$. %CORRECT THE STATEMENT SLIGHTLY? IT'S A FUNNY ISSUE THAT ARISES.
%there exists a proofpath from $\hat x \wedge \overline{\phi(x)}$ to $\widehat{\phi(x)}$ in $\sA$. 
\Thm
\prf
The proof is just as in \cite{LaSc1}, except that we must add clauses for the operations $\wedge$ and $\vdash$. %and a recursion rule to account for source and target. 
Note that several steps depend on the existence of identities on the subjects of $\sA$, as assumed in definition \ref{d.pidedsys}. 
First, % one direction is easy 
let $f$ be a valid path from $\hat x \wedge a$ to $b$ in $\sA$. Then since $\phi := \langle (x \cdot (a \vdash \trut) , 1_a \rangle $ is a valid path from $a$ to $\hat x \wedge a$ in $\sA[x]$, we obtain a witness $f \cdot \phi$ of the type $a \vdash b$ in $\sA[x]$, as desired. 

Now suppose $\phi$ is a valid path from $a$ to $b$ in $\sA[x]$. Suppose that for all polynomials in $x$ $\phi_<$ of length strictly less than $\phi$, there is a witness of $\hat x \wedge \overline{\phi_<} \vdash \widehat{\phi_<}$, denoted %derived by induction from $\phi$ in the second half of the previous proof by 
$$\kappa_x (\phi_<).$$
Now we proceed by cases:
\enu
	\item if $\phi \in \sA$, then $\phi \cdot \pi'_{\hat x, a}$ validates $\hat x \wedge a \vdash b$. \label{case.constant}
	\item if $\phi = x$, then $\pi_{\hat x, a}$ validates $\hat x \wedge a \vdash b$. 
	\item if $\phi = \psi \vdash \chi$ for some $\psi, \chi \in \sA[x]$, then 
			$a$ is identical to $\psi$ and $b$ is identical to $\chi$, hence this case reduces to case (\ref{case.constant}).
	\item if $\phi = \psi \cdot \chi$ for some $\psi, \chi \in \sA[x]$, then 
			$$\kappa_x \psi \cdot \langle \pi_{\hat x, a}, \kappa_x \chi \rangle$$
			is the desired witness. ($\chi \cdot \kappa_x \psi$ doesn't work, because $x$ is still not eliminated.)	
	\item if $\phi = \psi \wedge \chi$ for some $\psi, \chi \in \sA[x]$, then 
			$$\langle \kappa_x (\psi) \cdot \pi_{\bar{\psi}, \bar{\chi}}, \kappa_x (\chi) \cdot \pi'_{\bar{\psi}, \bar{\chi}} \rangle $$
			is the desired witness. (The alternative witness $\kappa_x \psi \wedge \chi_x \cdot \lambda$, where $\lambda$ is a munging factor, gives a definition of $\kappa_x$ under which one does not prove Theorem \ref{t.fCompT}.)
	\item if $\phi = \langle \psi, \chi \rangle$ for some $\psi, \chi \in \sA[x]$, then
			$$\langle \kappa_x \psi, \kappa_x \chi \rangle$$
			is the desired witness.
	\item if $\phi = \psi^*$ for some $\psi \in \sA[x]$, then
			$$(\kappa_x (\psi )\cdot \alpha)^*$$
			is the desired witness, where $\alpha$ is the associator.
\Enu
Proceeding by induction on the length of polynomials $\phi$ in $\sA[x]$ if necessary, we obtain in each case the desired witness of $\hat x \wedge a \vdash b$. 
%Now, we have to show that source and target are ok.
%\stuffgoeshere
%draw a picture?
%\stuffgoeshere
\Prf

We denote the witness of $\hat x \wedge a \vdash b$ derived by pattern matching on $\phi: \bar \phi \to \hat \phi$ in the second half of the preceding proof by
$$\kappa_x (\phi): \hat x \wedge \bar \phi \to \hat \phi.$$
Now we pass from deductive systems to (ideal) categories. When we do so, it is necessary to ensure that the polynomial system over an indeterminate remains in our category. Hence we fix the following definition: 

\dfn\label{d.polyicat}
Let $\sC$ be an ideal cartesian closed category. %
Let $x$ be an indeterminate in $\sC$. 
To define the symbol
$$\sC(x),$$ 
%proceed as follows. First, viewing $\sC$ as a generalized graph, 
observe that $\sC$ is equipped with the structure 
$$(s, t, \cdot, \vdash, \sI, \sV)$$
of a positive intuitionistic deductive system, when regarded as a generalized graph. %For example, take $\sI$ to be all elements of the form $1_a$ for some $a$. 
Take $\sK_{\sC(x)}$ to be the set of constant polynomials.\footnote{This definition restricts behavior of terminal arrows $\phi \vdash \psi$ for polynomials $\phi$ and $\psi$, but it will not make a difference for our purposes.} 
Now take the polynomial system $\sC[x]$ of Definition \ref{d.poly}, %
and then take the smallest equivalence relation $=_x$ of paths in $\sC[x]$ satisfying the conditions:
\enu
%	\item For all $f,g,h \in \sC$, if $gf = h$ a.d. in $\sC$, then $gf = h$ in $\sC[x]$ when $g,f,h$ are taken as elements in $\sC[x]$. 
	\item If $f = g$ in $\sC$, then $f =_x g$ in $\sC(x)$,
	\item $(\phi \vdash \hat \psi ) \cdot \psi =_x (\phi \vdash \bar \phi)$ unless $\phi =_x \bar \psi$, in which case $(\phi \vdash \hat \psi) \cdot \psi = \psi$,
	\item $\psi \cdot (\bar \psi \vdash \phi) =_x (\hat \psi \vdash \phi) $ unless $\phi =_x \hat \psi$, in which case $\psi \cdot (\bar \psi \vdash \phi) =_x (\hat \psi \vdash \phi),$ unless $\psi \in \sK$ or $\bar \psi \vdash \phi \in \sK$, 
%	\stuffgoeshere % constants - figure out what they do (now that we know that they don't do very much it should be easy) and then state that here. 
%	\item $\phi \cdot 1_{\bar \phi} =_x 1_{\hat \phi} \cdot \phi =_x \phi$ for $\phi \in \sC(x)$.
	\item For all $\phi, \psi \in \sC[x]$, if $(\chi \cdot \psi) \cdot \phi$ is defined, then 
		 $$(\chi \cdot \psi) \cdot \phi =_x \chi \cdot (\psi \cdot \phi),$$
	\item Composition $(\cdot)$, combination $\langle, \rangle$, and the turnstile $(\vdash)$ in $\sC(x)$ is substitutive in both arguments:
	% for example, if $\phi =_x \psi$ and $\phi$ $\downarrow$ then $\phi \vdash \chi =_x \psi \vdash \chi$. 
		\enu
			\item if $\phi =_x \psi$ then $\phi \vdash \chi =_x \psi \vdash \chi$ and $\chi \vdash \phi =_x \chi \vdash \psi$,
			\item if $\phi =_x \psi$ and $\phi \cdot \chi$ $\downarrow$ then $\phi \cdot \chi =_x \psi \cdot \chi$ and if $\chi' \cdot \phi$ $\downarrow$ then $\chi' \cdot \phi =_x \chi' \cdot \psi$,
			\item if $\phi =_x \psi$ and $\langle \phi, \chi \rangle$ $\downarrow$ then $\langle \phi, \chi \rangle =_x \langle \psi, \chi \rangle$ and $\langle \chi, \phi \rangle =_x \langle \chi, \psi \rangle$,
		\Enu 
%	\item $\langle , \rangle$ is substituve in both arguments in $\sC(x)$.
%	\item $(\wedge)$ is substitutive in both arguments in $\sC(x)$.
%	\item $\vdash$ is substitutive in both arguments in $\sC(x)$. 
%	\item The axioms of an ideal cartesian closed category hold in $\sC(x)$. (Definition \ref{d.idealccc})
	\item For all $\phi: a \to \trut$, $f =_x a \vdash \trut$,
	\item For all pairs $(a,b) \in \sC$ (viewed as a deductive system), if the unique good pair for $(a,b)$ is $(\pi_{a,b}, \pi'_{a,b})$ and any good evaluation $\epsilon_{a,b}$ is taken, then these are required to satisfy their usual equational properties in expressions involving $x$: 
		\enu
			\item if $\pi_{a,b} \langle \phi, \psi \rangle$ $\downarrow$ then $\pi_{a,b} \langle \phi, \psi \rangle =_x \phi$,
			\item if $\pi'_{a,b} \langle \phi, \psi \rangle$ $\downarrow$ then $\pi'_{a,b} \langle \phi, \psi \rangle =_x \psi$,
			\item if $\langle \pi_{a,b} \cdot \phi, \pi'_{a,b} \cdot \phi \rangle$ $\downarrow$ then $\langle \pi_{a,b} \cdot \phi, \pi'_{a,b} \cdot \phi \rangle =_x \phi$,
			\item if $\langle \phi \cdot \pi_{a,b} , \psi \cdot \pi'_{a,b} \rangle$ $\downarrow$ then $\langle \phi \cdot \pi_{a,b}, \psi \cdot \pi'_{a,b} \rangle = \phi \wedge \psi$,
			\item if $\epsilon \langle \phi^* \cdot \pi_{c,a}, \pi'_{c,a} \rangle$ $\downarrow$ then $\epsilon \langle \phi^* \cdot \pi_{c,a}, \pi'_{c,a} \rangle = \phi$, 
			\item if $(\epsilon \langle \phi \cdot \pi_{c,a}, \pi'_{c,a} \rangle )^*$ $\downarrow$ then $(\epsilon \langle \phi \cdot \pi_{c,a}, \pi'_{c,a} \rangle )^* = \phi$. %for all $h: c \wedge b \to a$ and $k: c \to (a \dashv b).$		\Enu
%	\item A similar axiom to the preceding axiom holds for each good evaluation $\epsilon_{a,b}$. 
%	\item There exists a good evaluation for every good pair.
		\Enu
\Enu
%Define composition and sources and targets in $\sC(x)$:
%$$\overline{\phi(x)} = \bar{x},$$
%$$\widehat{\phi(x)} = 
%\begin{cases}
%	\phi(x) = k \in \sC, 				& \hat{k}, \\
%	\phi(x) = x,					& \hat{x}, \\
%	\phi(x) = \psi(x) \times \chi(x),		& \hat {\psi(x)} \times \hat{\chi(x)}, \\
%	...							& 			\\	
%\end{cases}
%It must be shown that the operations of $\sC[x]$ remain well-defined in $\sC(x)$. We leave this to the reader.
\Dfn

The construction of $\sC(x)$ is thus carried out closely following Lambek. 
%
By iterating the construction of Definition \ref{d.polyicat} we may define general polynomial systems $\sA[\vec{x}]$ and general polynomial categories $\sC(\vec{x})$. %By using this notation it is always understood that $\vec x$ denotes a telescope of indeterminates over $\sC$. %
A {\em polynomial over $\sC$} is an element of $\sC(\vec x)$ for any sequence of indeterminates $\vec x$. %

The following properties are established in \cite{LaSc1} for ordinary cartesian closed categories. The proof in our setting is similar when source and target do not depend on $x$, but in general requires a recursive step:

\lem\label{l.fCompT-iccc}
Let $\sC$ be an ideal cartesian closed category. Then $\sC(x)$ is an ideal closed category, and moreover:
\enu
	\item For every ideal cartesian closed category $\sD$, for every $F: \sC \to \sD$, and for every $a: F(\bar x) \to F(\hat x)$ in $\sD$, %
		there exists a unique functor $\theta: \sC(x) \to \sD$ satisfying %
			$$ \theta(x) = a, \quad \theta(f) = F(f) \,\, \text{ for all } f \in \sC.$$
%\stuffgoeshere
%where the map $H_x : \sC \to \sC(x)$ is defined by
%$$H_x (f) = f.$$
	\item As a consequence of (1), for every $a \in \sC$, there is a unique functor $S_x^a: \sC(x) \to \sC$ (called {\em substitution of $a$ for $x$}) satisfying
			$$S_x^a (x) = a, \qquad S_x^a (f) = f \,\, \text{ for all } f \in \sC.$$
\Enu
\Lem

%Show that $\sA[x][y] = \sA[y][x]$
%\stuffgoeshere
%\stuffgoeshere

%\subsection{Functional Completeness}\label{s.fCompT}

\thm\label{t.fCompT}
Let $\sC$ be an ideal cartesian closed category, %
let $\phi \in \sC(x),$ where $\phi: \trut \to \hat \phi$. 
%For every proofpath $\phi(x) \in \sC(x)$ of, say, $a \in \sC$, 
Then %
there exists a unique element $g:\hat x \to \hat{\phi}$ in $\sC$, such that
$$\phi = g \cdot x$$
in $\sC(x)$.
\Thm
\prf
The proof we give, following Lambek, proceeds by passing through $\sC[x]$, the polynomial generalized positive deductive system over $\sC$, and then verifying that one is able to mod out by $=_x$. 
%In $\sC[x]$, let the desired element $g$ (for $\phi$ fixed) be denoted $\ksi(\phi)$. Then clearly we may choose to define
%$$\ksi(\phi) := \kappa_x (\phi) \cdot \lambda,$$
%where $\kappa_x$ was defined after Theorem \ref{t.dedt}, and $\lambda$ is an obvious munging term (in fact $\lambda \threeline \langle 1_{\hat \phi}, \hat \phi \vdash \trut \rangle$).
%
First we show that $\kappa_x \phi$ has a new behavior because of $=_x$:

\lem\label{l.fCompT}
$\kappa_x \phi$ is a well-defined element of $\sC(x)$, satisfies
$$\kappa_x \phi \cdot \langle x , \trut \rangle = \phi,$$
and is the unique element of $\sC(x)$ that does so. 
\Lem
\prf 
One must check that
$$\eif \phi =_x \psi, \ethen \kappa_x \phi =_x \kappa_x \psi.$$
%and therefore 
%By the definition of $=_x$, this will imply that $
This requires checking each of the relations
We need only check the new case created by $\wedge$; the other cases can be checked as in \cite{LaSc1}. This follows from the definition of $\kappa_x$: for any $\phi, \psi$ in $\sC(x)$ we have $\phi \wedge \psi =_x \langle \phi \cdot \pi , \psi \cdot \pi' \rangle$. We verify that
\begin{align*}
\kappa_x (\phi \wedge \psi) 		&= \langle \kappa_x (\phi) \pi , \kappa_x (\psi) \pi' \rangle \\
						&= \langle \kappa_x (\phi \cdot \pi) , \kappa_x (\psi \cdot \pi') \rangle \\ %\text{ (easy to check)} \\
						&= \kappa_x (\langle \phi \cdot \pi, \psi \cdot \pi' \rangle).     %\kappa_x (\phi) \wedge \kappa_x (\psi).
\end{align*}
from the definition of $\kappa_x$ for this case. 
The uniqueness of the choice of $\ksi(\phi)$ is the result of the following calculation in $\sC(x)$ \cite{LaSc1, LaK2}:
\begin{align*}
\hspace{170pt}
\kappa_x \phi 	&=_x \kappa_x (\tilde f \cdot \langle x , \trut \rangle) \\
			&=_x \tilde f \cdot \kappa_x ( \langle x , \trut ) \\
			&=_x \tilde f \cdot \langle \kappa_x x , \kappa_x \trut \rangle \\
			&=_x \tilde f \cdot \langle \pi_{\hat x, \trut}, \trut \cdot \pi'_{\hat x, \trut} \rangle \\
			&=_x \tilde f. \hspace{230pt}\qedhere
\end{align*}
\Prf

Now we finish the proof of Theorem \ref{t.fCompT}. %
We define the element $g$ in $\sC(x)$ to be
$$g := \kappa_x \phi \cdot \beta,$$
where $\beta$ is just the obvious munging term, in fact $\beta \threeline \langle 1_{\hat x}, \hat{x} \vdash \trut \rangle$. %
Indeed, we have
\begin{align*}
g \cdot x 	
		&= \kappa_x \phi \cdot \langle 1_{\hat x}, \hat{x} \vdash \trut \rangle \\
		&= \kappa_x \phi \cdot \langle x, \trut \vdash \trut \rangle \\
		&= \kappa_x \phi \cdot \langle x, \trut \rangle \\
		&= \phi
\end{align*}
by Lemma \ref{l.fCompT}. %
For uniqueness of $g$, suppose that $\tilde g \in \sC$ satisfies $\tilde g \cdot x = \phi$ in $\sC(x)$. We calculate
\begin{align*}
\kappa_x (\phi) \cdot \beta 	
		&= \kappa_x (\tilde g \cdot x) \cdot \beta \\
		&= \kappa_x (\tilde g \cdot x) \cdot \langle 1_{\hat x}, \hat x \vdash \trut \rangle \\
		&= \kappa_x (\tilde g) \cdot \langle \pi_{\hat x, \trut}, \kappa_x x \rangle \cdot \langle 1_{\hat x} , \hat x \vdash \trut \rangle \\
		&= \tilde g \cdot \pi'_{\hat x, \hat x} \langle \pi_{\hat x, \trut} , \pi_{\hat x, \trut} \rangle \cdot \langle 1_{\hat x} , \hat x \vdash \trut \rangle \\
		&= \tilde g \cdot \pi_{\hat x, \trut} \cdot \langle 1_{\hat x}, \hat x \vdash \trut \rangle \\
		&= \tilde g \cdot 1_{\hat x} \\
		&= \tilde g.
\end{align*}
But $\kappa_x (\phi) \cdot \beta = g$ by definition of $g$. So $g = \tilde g$, and $g$ is unique. 
\Prf

%Theorem \ref{t.fCompT} is true for a cartesian generalized category \cite{LaSc1}, but we have no need for this level of generality.

From Theorem \ref{t.fCompT} we define notation (to resemble a counit) $\varepsilon_x \phi: \hat x \to \hat \phi$ by
$$\varepsilon_x \phi := g = \kappa_x (\phi) \cdot \beta.$$
Theorem \ref{t.fCompT} has the following corollary:

\cor\label{c.fCompT}
Let $\sC$ be an ideal cartesian closed category, %
and let $\phi \in \sC(x)$ have source $\trut$. %
Then there exists a unique element $h: \trut \to (\hat x \vdash \hat \phi)$ such that 
$$\phi =_x \epsilon \cdot \langle h, x \rangle$$
in $\sC(x)$. %
\Cor
\prf
This is obtained by taking the name of the element $g$ of Theorem \ref{t.fCompT}: that is, take
\[
h = \name{g}. \qedhere
\]
%\vspace{-10pt}
\Prf

From Corollary \ref{c.fCompT} we define notation $\lambda_x \phi : \trut \to (\hat x \vdash \hat \phi)$ by
\[
\lambda_x \phi := h = \name{\kappa_x (\phi) \cdot \langle 1_{\hat x} , \hat x \vdash \trut \rangle} 
\]
%for the element $h$. 
% will be important in section \ref{s.lam}.

%Theorem \ref{t.fCompT} is known as {\em Functional Completeness}. It is the application of the Deduction Theorem to the categorical setting. 

%it is interesting that
As an aside, we observe from the proofs of Theorem \ref{t.dedt} and \ref{t.fCompT} that $\wedge$'s identity in categories suggests whether the symbol may be sugared out of generalized deduction systems entirely. This would mean $\langle, \rangle$ would be defined as a basic operation subject to an equational axiom: %(where equality would, in the preceding formalism, be interpreted at the metalevel):
$$t(\langle a, b \rangle) = \langle \hat a \pi, \hat b \pi' \rangle.$$
In this case $\pi$ and $\pi'$ must satisfy a self-referential axiom:
$$s(\pi) = s(\pi') = \langle a \cdot \pi, b \cdot \pi' \rangle.$$
%$$s(\pi') = \langle a \cdot \pi, 































%\input{th5(lam)} % lambek's theorem
% File th5(lam).tex Created by Lucius Schoenbaum September 30, 2016
% lambek's theorem



















%

%

%

% lambda calculus
\section[Typed Lambda Calculus and the Main Correspondence]{\onehalfspacing Typed Lambda Calculus and the Main Correspondence}\label{s.lam}

In this section we will finally observe what happens on the syntactic side of the correspondence after generalizing semantics. As it turns out,  types acquire a richer structure and simultaneously assume the role of function constants. %
By a {\em generalized lambda calculus} (Definition \ref{d.lam}) we refer to the simplest such type system possible: %one that resembles the polynomial category over an ideal cartesian closed category. 
we do not make mention of natural numbers objects (see \cite{LaSc1}), Boolean types, or other features that may appear in applications of lambda calculus. %We leave the task of introducing them % in the equivalence \ref{t.chl} 
%to the motivated reader. 
%We anticipate further study on how several of these may be added to the system. 

% 
%
%
%
%
%   			~~~~Typed Lambda Calculus~~~~
%
%
%
%
%
The next definition is not used in the sequel. It is included in order to establish a basis for defining variables before making Definition \ref{d.lam}.

\dfn\label{d.prelam}
A {\em pre-generalized typed lambda calculus} is a structure
$$(\Lambda, \sT_\Lambda, \sS_\Lambda, s,t, \cdot, \vdash, %\oneterm_{()}, 
\trut, \wedge, \ty, \name{}, *, ()^\cdot, \pi, \pi', \wr, \langle,\rangle, \lambda, \sV_\Lambda)$$
where
\enu
	\item $\Lambda$ is a set,
	\item $\sT_\Lambda$ and $\sS_\Lambda$ are disjoint subsets of $\Lambda$ and $\sT_\Lambda \cup \sS_\Lambda = \Lambda$,
	\item $\sV_\Lambda$ is a subset of $\Lambda$,	
	\item the system
		$$(\sT_\Lambda, s,t,\cdot,\vdash, \sV')$$
		is an ideal category, where $\sV' = \sV_\Lambda \cap \sT_\Lambda$,
%	\item $s$ and $t$ are mappings $\sT_\Lambda \to \sT_\Lambda$,
%	\item $(\cdot)$ is a partially defined mapping $\sT_\Lambda \times \sT_\Lambda \to \sT_\Lambda$,
%	\item \vdash
	\item $\trut$ is a designated element of $\sT_\Lambda$,
	\item $\wedge$ is a mapping $\sT_\Lambda \times \sT_\Lambda \to \sT_\Lambda$,
	\item $\name{}$ is a mapping $\sT_\Lambda \to \sS_\Lambda$,
	\item $\ty$ is a mapping $\sS_\Lambda \to \sT_\Lambda$,
%	\item $\oneterm_{()}$ is a mapping $\sT_{\Lambda} \to \sS_{\Lambda}$, 
	\item and in $\sS_\Lambda$:
	\enu
		\item $*$ is a designated element of $\sS_\Lambda$,
		\item $()^\cdot$ is a mapping $\Lambda \to \sS_\Lambda$,
		\item $\pi$, $\pi'$ are partially defined mappings $\sS_\Lambda \to \sS_\Lambda$,
		\item $\wr$ and $\langle,\rangle$ are partially defined mappings $\sS_\Lambda \times \sS_\Lambda \to \sS_\Lambda$,
	%	\item $\langle, \rangle$ is a partially defined mapping $
		\item $\lambda$ is a mapping $\sX \times \sS \to \sS$, where $\sX$ is defined below,
	\Enu
\Enu
subject to the conditions
\enu
	\item % trut
		$\hat \trut = \bar \trut = \trut,$
%	\item % wedge
%	\item % ty
%	\item % name/oneterm
%		$\name{1_A} = \oneterm_A,$ where $1_A$ is the identity on $A$, if it exists.
% 	\item % *
% 	\item % ^cdot
 	\item % pi, pi'
		for all $s \in \sS_\Lambda$, $\pi(s)$ $\downarrow$ iff $\pi'(s)$ $\downarrow$ iff there exist $A,B \in \sT_\Lambda$ such that $\ty(s) = A \wedge B$, 
	\item % wr
		$s \wr t$ $\downarrow$ iff there exist $A,B \in \sT_\Lambda$ such that $\ty(s) = A \vdash B$ and $\ty(t) = A$,
	\item % langle,rangle
		$\langle s,t \rangle$ $\downarrow$ iff $\ty(s) = \ty(t)$,
%	\item % \lambda 
\Enu
typing conditions
\enu
%	\item % trut
%	\item % wedge
%	\item % ty
	\item % name
		$\ty(\name{A}) = \bar A \vdash \hat A$,
%	\item % oneterm
%		$\ty(\oneterm_A) = A \vdash A$,
 	\item % *
		$\ty(*) = \trut$,
 	\item % ^cdot
		for all $\alpha \in \Lambda$, $\ty(\alpha^\cdot) = \ty(\alpha),$
 	\item % pi, pi'
		if $s \in \sS_\Lambda$ and $\ty(s) = A \wedge B$, then $\ty(\pi(s)) = A$ and $\ty(\pi'(s)) = B$,
	\item % wr
		if $s \wr t$ $\downarrow$, then $\ty(s \wr t) = \widehat{\ty(s)}$,
	\item % langle,rangle
		if $\langle s, t \rangle$ $\downarrow$, then $\ty(\langle s, t \rangle) = \ty(s) \wedge \ty(t)$,
	\item % \lambda 
		if $\lambda(x, s)$ $\downarrow$, then $\ty(\lambda(x, s)) = \ty(x) \vdash \ty(s)$,
\Enu
%$$\sV_\Lambda := \set{T \in \sT_\Lambda \mid \text{ there is } s \in \sS_\Lambda \text{ such that } \ty(s) = T},$$
%the {\em standard} valid types of $\Lambda$. We may also define a {\em nonstandard} set of valid types (see Definition \ref{s.adjoinx}, so that $\sV_\Lambda$ is part of the definition of $\Lambda$, but we usually understand $\sV_\Lambda$ to be standard unless specified otherwise. %
and the validities
\enu
	\item % trut
		$* \in \sV_\Lambda$,
	\item % wedge
		if $A,B \in \sV_\Lambda$, then $A \wedge B \in \sV_\Lambda$,
	\item % ty
		(witnesses, propositions-as-types) if $s \in \sV_\Lambda$, then $\ty(s) \in \sV_\Lambda$,
	\item % name
		If $A \in \sV_\Lambda$, then $\name{A} \in \sV_\Lambda$.
 	\item % *
		$* \in \sV_\Lambda$,
 %	\item % ^cdot		% definitely no validities here		
 	\item % pi, pi'
		if $c \in \sV_\Lambda$ and $\pi(c), \pi'(c)$ $\downarrow$, then $\pi(c), \pi'(c) \in \sV_\Lambda$,
	\item % wr
		if $a,f \in \sV$, then $f \wr a \in \sV$,
	\item % langle,rangle
		$a,b \in \sV_\Lambda$ implies $\langle a, b \rangle \in \sV_\Lambda$,
	\item % \lambda 
		if $s \in \sV_\Lambda$, then $\lambda(x,s) \in \sV_\Lambda$. %where $\Lambda_x$ is defined below (Definition \ref{d.indeterminatevariable}).
\Enu
\begin{comment}
%%% We don't need to define a morphism of these gadgets. The morphism notion for GTLC's isn't an extension of this notion anyway, so it only complicates things completely unnecessarily. 
A {\em morphism} $\Phi: \Lambda \to \Mu$ between pre-generalized typed lambda calculi $\Lambda$ and $\Mu$, also called a {\em translation}, is a mapping 
	$$\Phi: \Lambda \to \Mu$$
that satisfies
\enu
	\item for all $A \in \sT_\Lambda, s \in \sS_\Lambda$, $\Phi(A) \in \sT_\Mu$ and $\Phi(s) \in \sS_\Mu,$
	\item the restriction of $\Phi$ to $\sT_\Lambda$ is a morphism of ideal categories that satisfies
		$$\Phi(\trut_\Lambda) = \trut_\Mu,$$
		$$\Phi(A \wedge B) = \Phi(A) \wedge \Phi(B),$$
	\item % ty
		$\Phi(\ty(s)) = \ty(\Phi(s)),$
	\item % name
		$\Phi(\name{A}) = \name{\Phi(A)}$
	\item % *
		$\Phi(*) = *,$
	\item % ^cdot
		for all $\alpha \in \Lambda$, $\Phi(\alpha^\cdot) = \Phi(\alpha)^\cdot,$
	\item % pi, pi'
		$\Phi(\pi(c)) = \pi(\Phi(c),$ and $\Phi(\pi'(c)) = \pi'(\Phi(c)),$
	\item % wr
		
	\item % langle,rangle
		ff
	\item % lambda
		ff
\Enu
This gives a category $\prelambdaCalc$ of pre-generalized typed lambda calculi.
\end{comment}
\Dfn

Note that many type theories, e.g. \cite{MoI1}, include {\em function constants} $f: A \to B$ as well as terms and types; in this formalism (guided by the new semantics) function constants are indistinguishable from types, and together with objects they form a category. Types behave as function constants via the derived operation $A \star s := \name{A} \wr s.$ The operation $\name{}$ is used not only here but also in the construction of $\boC \Lambda$ in Definition \ref{d.catcon}. 

%A generalized typed lambda calculus is a pre-generalized typed lambda calculus on which there is an equality relation defined. 
%to an equality relation on $\sS_\Lambda$, but 
%In order to introduce the equality relation, we must introduce variables. %
Elements of $\sT_\Lambda$ are called {\em types}, and elements of $\sS_\Lambda$ are called {\em terms}. For a term $s$, the element $\ty(s)$ of $\sT_\Lambda$ is called the {\em type of $s$.} %
We may write $s:T$ to denote the relation $\ty(s) = T$.
A term of the form $\alpha^\cdot$ for some $\alpha$ (which may be a type or a term) is called a {\em variable}. %
We may iterate the operation $()^\cdot$, %
and we do not allow $()^\cdot$ to be substitutive in its argument. %
Therefore we may assume that the symbol $x_i$ %is used to denote a variable of type $T$, say, then $x_i$ 
unpacks to %
$((\dots ((A)^\cdot )^\cdot \dots )^\cdot )^\cdot$. %
In this way, we have a countable stock $x_1, x_2, \dots$ of distinct ``standard'' variables for each type $A$. %
For technical reasons (see below, before Definition \ref{d.lam}), we take these standard variables to be the only variables of $\Lambda$, 
and we place the obvious (total) ordering on variables of each type. %
%In general, we define the set of {\em variables} of $\Lambda$ to be
%$$\sX := \set{\alpha^\cdot \mid \alpha \in \Lambda}.$$
%The operation $\lambda$ is called {\em lambda abstraction}. 
A variable $x_i$ is {\em free} in a term if it appears in the term, unless it appears but only within a well-formed expression of the form $\lambda(x_i, s)$. In this case we say it appears {\em captured} or {\em bound}. 
We define the mapping on terms
$$\FV(s) = \set{x \in \sX \mid x \text{ appears free in $s$, and } x \nin \sV_\Lambda},$$
where the phrase ``appears free'' has its usual meaning, except that we assume that no variable {\em appears free} in any type. So, for example, for all types $A$, $\FV(\name{A})$ is empty. %
If $s$ is a term, $x$ is a variable, and $t$ is a term whose type is the same as the type of $x$ %and satisfies : for all $y \in \FV(t)$, $y$ does not appear captured in $s$, 
we define notation
$$s[x/t]$$
to be the term $s$ with the variable $x$ replaced by $t'$ in each instance where it does not appear bound in $s$, %
where $t'$ is $t$ with any variable $y \in \FV(t)$ that appears captured in $s$, that is, 
$$y \in \FV(t) \cap \CAP(s),$$
where $\CAP(s)$ is the set of variables appearing captured in $s$, %
replaced by a variable of the same type that is not in the set $\VAR(s) \cup \VAR(t)$ of variables appearing in either $s$ or $t$. %
These choices are made in {\em the simplest order-preserving way}, by which is meant that once the set of variables to be changed is found, the entire set is incremented by the smallest positive integer such that the set of variables so generated is not in $\VAR(s) \cup \VAR(t)$. %
These incrementing operations are associative, as is the substitution operation itself. Hence we have
$$s[x/t][y/r] = s[x/t[y/r]]$$
for all terms $s,r,t$ and variables $x,y$. %
We may often ignore the extra step involving $t'$, for it is only necessary because we have not set terms $s$ and $s'$ equal in $\Lambda$ which are the same up to one or more free variables (a form of $\alpha$-conversion) in $\prelambdaCalc$ or in the category $\lambdaCalc$ defined next. %
Note that a morphism in $\prelambdaCalc$ sends closed terms to closed terms. %

\dfn\label{d.lam}
A {\em generalized typed lambda calculus} is %
a pre-generalized typed lambda calculus %
on which there is an equality relation on the terms $\sS_\Lambda$ of $\Lambda$ defined as follows: %
Let $\sP$ be the finite power set $\sP_{fi} (\sX)$ of $\sX$. %
For each finite set $\bar x = \set{x_1, \dots, x_n}$ in $\sP$, %
let 
$$\sR(\Lambda, \bar x) := \set{s \in \sS_\Lambda \mid \FV(s) \lies \bar x}.$$
We define the relation $=_{\bar x}$ on $\sR(\Lambda, \bar x)$ 
%$$(=_{\bar x}) : \sR(\Lambda, \bar x) \times \sR(\Lambda, \bar x) \to {\bf 2} $$
to be the smallest equivalence relation that satisfies
\enu
	\item $=_{\bar x}$ is reflexive, symmetric, and transitive,
%	\item substitutive wrt ``all term forming operations'' (Lambek):
	\item Substitutivity conditions:
	\enu
%		\item % name
			% (no substitutivity condition)
%		\item % *
			% (no substitutivity condition)
%		\item % ^cdot
			% (no substitutivity condition)
		\item % pi,pi'
			if $s =_{\bar x} t$, and $\pi(s)$ $\downarrow$, then $\pi(s) =_{\bar x} \pi(t)$, and $\pi'(s) =_{\bar x} \pi'(t)$,
		\item % wr 
			if $s =_{\bar x} t$, then $s \wr r =_{\bar x} t \wr r$ and $u \wr s =_{\bar x} u \wr t$ whenever these expressions are well-defined, 
		\item % langle
			if $s =_{\bar x} t$, and $\langle s,r \rangle$ $\downarrow$, then $\langle s,r \rangle =_{\bar x} \langle t,r \rangle$, and similarly in the second argument,
		\item % lambda
			if $s =_{\bar x} t$, then $s \wr r =_{\bar x} t \wr r$ and $u \wr s =_{\bar x} u \wr t$ whenever these expressions are well-defined, 
	\Enu
	\item for all $s:\trut$, $s =_{\bar x} *$,
	\item for all $a : A, b: B,$ 
		$$\pi(\langle a, b \rangle) =_{\bar x} a,$$
		$$\pi'(\langle a, b \rangle) =_{\bar x} b,$$
	\item for all $c: A \wedge B$,
		$$\langle \pi(c), \pi'(c) \rangle =_{\bar x} c,$$
	\item For all terms $s \in \sS_\Lambda$, terms $a \in \sS_\Lambda$, and variable $x$ that may appear in $\bar x$,
	\enu
%		\item $\oneterm_A \wr s =_{\bar x} s,$
		\item $(\lambda(x,s)) \wr a =_{\bar x} s[x/a],$
		\item $\lambda(x, s \wr x) =_{\bar x} s$,
		\item if $\FV(s) = \set{x}$, there exists a unique $A \in \sT_\Lambda$ such that $s =_{\set{x}} \name{A} \wr x$. \label{ax.unname}
		\item ($\alpha$-conversion for lambda terms) 
		$$\lambda(y, s) =_{\bar x} \lambda(y', s[y/y'])$$
		if $\ty(y) = \ty(y')$ and $y' \nin \FV(s)$.
	\Enu
\Enu
We observe that $\FV()$ is still well-defined. %
%For convenience, 
We denote by 
$$s^\natural$$ 
the type $A$ given by Axiom \ref{ax.unname}. %(Note this notation is distinct from the notation introduced in section \ref{s.th}.)
We impose the condition on the $=_{\bar x}$'s that: 
\enu
	\item if $\bar x \lies \bar y$ then for all $s,t \in \sR(\Lambda, \bar y)$, $s =_{\bar x} t$ implies $s =_{\bar y} t$.
\Enu
Because %
(1) $s =_{\FV s} s$, and %
(2) if $s =_{\bar x} t$ and $s =_{\bar y} t$, then there exists a finite set $\bar z$ such that $s =_{\bar z} t$ and $\bar x, \bar y \lies \bar z$, %
we may define an equivalence relation {\em equality in $\sS_\Lambda$} on the set $\sS_\Lambda$ of terms of $\Lambda$ by
$$s = t \,\,\,\text{ if }\,\, s =_{\bar x} t \text{ for some $\bar x$ in $\sP$}.$$
%
A {\em morphism} $\Phi: \Lambda \to \Mu$ of generalized typed lambda calculi, also called a {\em translation}, is a mapping 
$$\Phi: \Lambda \to \Mu$$
that satisfies the following, where equalities between terms are interpreted as equality in $\sS_\Lambda$:
\enu
	\item for all $A \in \sT_\Lambda, s \in \sS_\Lambda$, $\Phi(A) \in \sT_\Mu$ and $\Phi(s) \in \sS_\Mu,$
	\item the restriction of $\Phi$ to $\sT_\Lambda$ is a morphism of ideal categories that satisfies
		$$\Phi(\trut_\Lambda) = \trut_\Mu,$$
		$$\Phi(A \wedge B) = \Phi(A) \wedge \Phi(B),$$
	\item if $s =_{\bar x} t$, then $\Phi(s) =_{\Phi(\bar x)} \Phi(t)$. \label{ax.preservesequalities}
	\item % ty
		$\Phi(\ty(s)) = \ty(\Phi(s)),$
%	\item $\Phi$ is a morphism of pre-generalized typed lambda calculi, except that some properties are only satisfied up to equality. Specifically, %$\Phi$ is no longer required to satisfy ()()() on the nose but can satisfy up to equality
%	\item $\Phi(\trut) = \trut,$
%	\item $\Phi(A \wedge B) = \Phi(A) \wedge \Phi(B),$
	\item % name
		$\Phi(\name{A}) = \name{\Phi(A)},$
	\item % *
		$\Phi(*) = *$, %where $=$ is equality in $\sS_\Lambda$,
	\item % ^cdot
		for all $\alpha \in \Lambda$, $\Phi(\alpha^\cdot) = \Phi(\alpha)^\cdot,$
	\item % pi, pi'
		$\Phi(\pi(c)) = \pi(\Phi(c),$ and $\Phi(\pi'(c)) = \pi'(\Phi(c)),$
	\item % wr
		$\Phi(s \wr t) = \Phi(s) \wr \Phi(t),$
	\item % langle,rangle
		$\Phi(\langle s, t \rangle) = \langle \Phi(s), \Phi(t) \rangle,$
	\item % lambda
		$\Phi(\lambda(x,s)) = \lambda(\Phi(x), \Phi(s)).$
\Enu
As a consequence of (\ref{ax.preservesequalities}), $\Phi$ preserves equalities in $\Lambda$:
$$s = t \eimplies \Phi(s) = \Phi(t).$$
This gives a category $\lambdaCalc$ of generalized typed lambda calculi.
\Dfn

Given a generalized typed lambda calculus $\Lambda$, we can construct an ideal cartesian closed category using Theorem \ref{t.fCompT}:

\dfn\label{d.catcon}
Let $\Lambda$ be a typed lambda calculus. %
Let $\sB_\Lambda$ be the set of {\em bulletins} in $\Lambda$, that is, the set of terms in $\Lambda$ that have only one free variable. %
Also for $A \in \sT_\Lambda$, let 
$$\settl{\bullet}{A} := \settl{x}{\name{A}\wr x}, \quad x : \bar A,$$
that is, a symbol $\settl{x}{s}$ where $x$ is a variable of type $\bar A$, and $s$ is the term $\name{A} \wr x$. %
By Axiom (\ref{ax.unname}) of Definition \ref{d.lam} we may identify these symbols with types in $\Lambda$. 
We define $\boC\Lambda$ to be the set
$$\boC\Lambda := \set{\settl{x}{s} \mid s \in \sB_\Lambda, \text{ $x$ a variable}},$$ 
of symbols $\settl{x}{s}$ for variable $x$ and bulletin $s$, equipped with the structure
\begin{align*}
%	\item % s/t
		\overline{\settl{x}{s}} &:= \settl{\bullet}{\ty(x)}, \\ %\settl{\overline{\ty(x)}^\cdot}{\name{\ty(x)} \wr \overline{\ty(x)}^\cdot}, \\
		\widehat{\settl{x}{s}} &:= \settl{\bullet}{\ty(s)}, \\ %\settl{\overline{\ty(s)}^\cdot}{\name{\ty(s)} \wr \overline{\ty(s)}^\cdot}, \\
%	\item %$\cdot$
		\settl{x}{s} \cdot \settl{y}{t} &:= \settl{y'}{s[x/t]}, \\
%	\item %$\vdash$
		\settl{x}{s} \vdash \settl{y}{t} &:= \settl{u}{v}, \quad u:\ty(s), v:\ty(t), 
%	\item % identities ----> these simply exist
\end{align*}
where
$$y' = 
	\begin{cases} 	
				y 
						& \text{ if $\FV(t)$ is empty}, \\ 
				inc_n(y)	
						& \text{ if $\FV(t) = \set{u}$ and $\FV(s[x/t]) = \set{inc_n (u)}$,}
	\end{cases}
$$
where $inc_n$ is the modification of the variable described after Definition \ref{d.prelam}. %
Let $\sK_{\boC\Lambda}$ be the set of symbols $\settl{x}{k}$ where $k$ is a constant in $\Lambda$, that is, $\FV(k) = \nll$. %
Let equality of symbols in $\boC\Lambda$ be defined by
\enu
	\item $\settl{x}{s} = \settl{y}{t}$ if $\ty(x) = \ty(y), \ty(s) = \ty(t),$ and there is $z:\ty(x)$ such that $s[x/z] = t$, 
%	\item $\settl{x}{s} = \settl{y}{t} \,\,\eif\, s[x/z] = t[y/z] \text{ in $\sS_\Lambda$,}$
	\item $\settl{x}{s} = \settl{x}{u}, \quad u:\ty{s}, \,\,$ \text{ if $\FV(s) = \set{y}$ and $y \neq x$}, 
	\item if $\ty(s) = \trut$, then $\settl{x}{s} = \settl{x}{*}$, 
	\item for all bulletins $s$ and all variables $x,y$ of the same type, $\settl{x}{s} = \settl{y}{s[x/y]}$. 
\Enu
%where $z$ does not appear in $s$ or $t$. %
This gives an ideal category $\boC\Lambda$, where %where the type $\settl{x}{s} \vdash \settl{y}{t}$ is
%$$\settl{u}{f}
the identity of $\settl{x}{s}$ is 
$$1_{\settl{x}{s}} = \settl{y}{y}, \quad y:s^{\natural},$$
terminal arrows are of the form 
$$\settl{y}{*},$$
and types (in the sense of section \ref{s.th}) are of the form
$$\settl{x}{y}, \quad x \neq y.$$
%(in fact all identities of $\boC\Lambda$ are of the form $\settl{x}{x}$) and 
Validities defining $\boC\Lambda$ are the evident ones based on Definition \ref{d.icat}. 
\Dfn

We have an ideal category $\boC\Lambda$, but we have not directly made any assumptions about the category $\sT_\Lambda$. Nevertheless, we have:

%We remark that it demands a little effort to show that in fact $\boC\Lambda$ has an associative composition operation. The difficulty is that given bulletins (terms in one variable) $s,t,r$ in variables $x,y,z$,
%$$s[x/t][y/r]$$
%and
%$$s[x/(t[y/r])]$$
%are equal only if one assumes that variables are linearly ordered, and that a certain ``sweeping'' operation occurs before the substitutions denoted by $[\cdot / \cdot]$, that assures that no unintended binding of variables occurs. The linear ordering is necessary to insure that a ``fresh'' variable is chosen when such a choice is required. This is necessary because $\alpha$-substitution with respect to free variables is not presumed to be possible in $\Lambda$ terms (that is, terms that are equal up to substitutions of the form $[x/y]$ for variables $x$ and $y$ of the same type are not assumed to be equal). We omit the details of this device, because it would require introducing new notation and will not arise again. 

\prop\label{p.catcon}
$\boC\Lambda$ is an ideal cartesian closed category. 
\Prop
\prf
Set
\begin{align*}
%	\item % trut
		\trut &:= \settl{u}{*}, \quad u:\trut_\Lambda,\\
%	\item % wedge
		\settl{x}{s} \wedge \settl{y}{t} &:= \settl{z}{\langle s \wr \pi(z), t \wr \pi'(z) \rangle}, \\
%	\item % langle rangle
		\langle \settl{x}{s}, \settl{y}{t} \rangle &:= \settl{z}{\langle s[x/z], t[y/z] \rangle}, \\
%	\item % star
		\settl{z}{s}^* &:= \settl{x}{\lambda(y,s \wr \langle x,y \rangle) }, \quad \text{ where $z:A \times B, x:A$}, \\
%	\item % pi, pi'
		\pi &:= \settl{z}{\pi(z)}, \\
		\pi' &:= \settl{z}{\pi'(z)},\\
%	\item % epsilon
		\epsilon &:= \settl{z}{\pi(z) \wr \pi'(z)}, 
\end{align*}
with validities as needed (Definition \ref{d.iccc}). 
%\stuffgoeshere
%\stuffgoeshere
\Prf



% 
%
%
%
%
%   			~~~~Internal Language~~~~
%
%
%
%
%

We can also construct a typed lambda calculus from the data of a cartesian closed category:

\dfn\label{d.internallanguage}
Let $\sC$ be an ideal cartesian closed category. We define the symbol $\boL\sC$ as follows:
\enu
	\item The set of types of $\boL\sC$ is the set of symbols $A_f$ indexed by elements $f \in \sC$:
		$$\sT_{\boL\sC} := \set{A_f \mid f \in \sC},$$
		in fact we set $A_f = f$ and take $\sC$ itself as the set of types (this is needed for the proof of Theorem \ref{t.chl}), however, we use the notation $A_f$ at times when it seems to lessen the potential for confusion.
	\item The set of terms of $\boL\sC$ is the set of polynomials $\phi$ over $\sC$ sourced at $\trut$, that is, %the polynomial system over $\sC$,
		$$\sS_{\boL\sC} := \set{\phi \mid \phi \in \sC[\vec x] \text{ for some $\vec x$, } \hat{\phi} \text{ is in $\sC$, and } \bar{\phi} = \trut}, $$
		where we assume %(for the sake of neatness) 
		that indeterminates have internal structure given by the syntax $()^\cdot$. 
	\item Define
		\begin{align*}
			\ty(\phi) &:= A_{\hat \phi}, \\
		 	s(A_f) &:= A_{sf}, \\
			t(A_f) &:= A_{tf}, \\
			A_f \cdot A_g &:= A_{f \cdot g}, \\
			A_f \wedge A_g &:= A_{f \wedge g}, \\
			A_f \vdash A_g &:= A_{f \vdash g}, \\
%			1_{A_f} &:= A_{1_f}, \\
			\name{A_f} &:= \name{f}, \quad \text{ the name of $f$,} \\
			\trut_{\boL\sC} &:= \trut_\sC \vdash \trut_\sC, \\
			* &:= \trut_\sC, \\
			\sK_{\boL\sC} &\text{ is the set of constant polynomials.} %
		\end{align*}
	\item if $\phi$ is a bulletin in $x$ over $\sC$, then let 
		$$\phi^\natural := A_{\epsilon_x \phi}.$$
%	\item in $\boL\sC$, let \label{ax.namecondition} %we require that
%		$$\name{A_f} = \tilde f$$
%		for every $f \in \sC$,
	\item $\sV_{\boL\sC}$ is the set $\set{A_f  \mid f \in \sV_\sC}$ %
	joined with the set of valid constant terms, %
	joined with the set of polynomials valid according to Definition \ref{d.poly}. %
	\item Define
		\begin{align*}	
			\trut_{\boL\sC} 	&:= \trut \vdash \trut, \\
			*_{\boL\sC} 	&:= \trut,
		\end{align*}
%	\stuffgoeshere
\Enu
Then we have a pre-generalized typed lambda calculus. %
%Finally, 
We make from this a generalized typed lambda calculus by imposing the equality relation on terms inherited from equality in $\sC(\vec x)$:
the equality relation $=_{\vec x}$ is defined to be equality in $\sC(\vec x)$, along with the usual inclusions of polynomial systems in one another. 
\Dfn

%\prop\label{p.internallanguage}
%$\boL\sC$ is a generalized typed lambda calculus.
%\Prop
%\prf
%\stuffgoeshere
%\Prf

%We can now check the axioms of Definition \ref{d.lam}. %
$\boL\sC$ is called the {\em internal language} of the ideal cartesian closed category $\sC$. %
% 
%
%
%
%
%   			~~~~Category of Contexts~~~~
%
%
%
%
%
Next, we verify that these constructions are functorial: 

\prop\label{p.functor}
We have the following:
\enu
	\item $\boC$ is a functor from $\lambdaCalc$ to ${\bf ICCC}$. % 
	\item $\boL$ is a functor from ${\bf ICCC}$ to $\lambdaCalc$. %
\Enu
\Prop
\prf
Given $\Phi: \Lambda \to \Lambda'$, we define $\boC\Phi : \boC\Lambda \to \boC\Lambda'$ by
%$$\boC\Phi(f_A) := f_{\Phi(A)},$$
$$\boC\Phi \settl{x}{s} := \settl{\Phi(x)}{\Phi(s)}$$
for %$f_A \in \sT_{\boC\Lambda}$ and 
$\settl{x}{s} \in \boC\Lambda.$ 
%
Now we check that $\boC \Phi$ is a morphism in ${\bf ICCC}$, %
and that %
$\boC$ is a functor (Definition \ref{d.functor}). %
%$\boC(\Phi \of \Phi') = \boC(\Phi) \of \boC(\Phi')$ and $\boC(1_\Lambda) = 1_{\boC(\Lambda)}.$
%\stuffgoeshere

Let $F: \sC \to \sD$ in ${\bf ICCC}$. Define a mapping $\boL F: \boL(\sC) \to \boL(\sD)$ by
\begin{align*}
\boL F(A_f) &:= A_{F(f)}, \\
%\boL F (\tilde a) &:= \widetilde{F(a)}, \\
\boL F ( \alpha^\cdot) &:= (\boL F(\alpha))^\cdot, 
\end{align*}
and extend $F$ from $\sC$ to polynomials over $\sC$ in the most straightforward way. % 
Now we check that $\boL F$ is a morphism in $\lambdaCalc$, and that $\boL$ is indeed a functor. %
%$\boL\dom(F) = \dom(\boL F),$ 
%$\boL\cod(F) = \cod(\boL F),$
%$\boL(G \of F) = \boL G \of \boL F,$
%$\boL(\id_\sC) = \id_{\boL\sC},$
%for all $\sC \in {\bf ICCC}$.
%\stuffgoeshere
\Prf

% 
%
%
%
%
%   			~~~~CHL Correspondence~~~~
%
%
%
%
%

%To define $\eta$, 
%we need, for fixed generalized typed lambda calculus $\Lambda$, 
%Before  must define the counit. 

\dfn\label{d.eta}
Let $\Lambda$ be a generalized typed lambda calculus. Define a mapping $\Lambda$ to $\boL\boC\Lambda$ %, or roughly speaking, the polynomials over $\sC\Lambda$, %
by defining, in the pre-generalized typed lambda calculus $\Lambda_0$ obtained by ignoring equalities in $\sS_\Lambda$, 
\begin{align*}
\eta_\Lambda(A) &:= A_{\settl{\bullet}{A}}	& A \in \sT_\Lambda, % the first $A$ is just a formal symbol
		 			\\
\eta_\Lambda(k) &:= \settl{x}{k}, 			& k \in \sS_\Lambda, \FV(k) = \nll, \ty(x) = \trut_{\boL\boC\Lambda},	\\
% * (case of k)
\eta_\Lambda(\alpha^\cdot) &:= (\eta_\Lambda(\alpha))^\cdot,					& \alpha \in \Lambda,		\\
\eta_\Lambda(\pi(\phi)) &:= \pi(\eta_\Lambda \phi ), 												& \\
\eta_\Lambda(\pi'(\phi)) &:= \pi'(\eta_\Lambda \phi ),												& \\
\eta_\Lambda(\langle \phi, \psi \rangle) &:= \langle \eta_\Lambda (\phi) , \eta_\Lambda( \psi) \rangle 			& \\
\eta_\Lambda(\phi \wr \psi) &:= \eta_\Lambda (\phi) \wr \eta_\Lambda (\psi)								& \\
\eta_\Lambda(\lambda(x, \phi)) &:= \lambda(\eta_\Lambda(x), \eta_\Lambda (\phi))			& x \in \sX_\Lambda 
\end{align*}
The map $\eta_\Lambda$ is well-defined upon passage to $\Lambda$, since analogous equalities between polynomials hold in both $\Lambda$ and $\boL\boC\Lambda$. 
\Dfn

An alternative approach (really the same) to Definition \ref{d.eta} is via an isomorphism with a lambda calculus with parameter \cite{LaSc1}: 

%Next, to show that $\boC$ and $\boL$ form an adjunction, we just need a lemma to help with the unit:

\dfn\label{d.indeterminatevariable}
Let $\Lambda$ be a generalized typed lambda calculus, and let $x \in \sX_\Lambda$ be a variable. We define the symbol 
$$\Lambda_x$$
to be the generalized typed lambda calculus is defined exactly as $\Lambda$, except that
$$\sV_{\Lambda_x} := \set{x} \cup \sV,$$
that is, $x$ is taken to be a validating term in $\Lambda_x$. 
%\stuffgoeshere
\Dfn

Intuitively, $\Lambda_x$ is $\Lambda$ with $x$ treated as a constant instead of as a variable. %
%In the following lemma, for the sake of neatness we treat the symbol $x$ as both an indeterminate and a variable. 

\lem\label{l.indeterminatevariable}
Let $\sC$ be an ideal cartesian closed category, and let $x$ be an indeterminate (with the variable syntax). Then the polynomial category 
$\boC\Lambda(x)$ over $\boC\Lambda$ is isomorphic to $\boC\Lambda_x$ in ${\bf ICCC}$. %where $x$ denotes an indeterminate, and $\tilde x$ is a lambda term $\Lambda$ with type the term in $\Lambda$ corresponding to the target $x$ in $\sC$. 
%CHECK TARGET OF X AS INDET AND TYPE OF X AS TERM
\Lem
\prf
By Proposition \ref{l.fCompT}, we need only check that $\boC\Lambda_{x}$ has the desired universal property of $\boC\Lambda(x)$. %
See \cite{LaSc1}. %
\Prf

Using Lemma \ref{l.indeterminatevariable}, we can identify polynomials $\tilde \phi$ over $\boC\Lambda$ with the corresponding symbol $\settl{u:\trut}{\phi(\vec x)}$ in $\boC\Lambda_{\vec x}$, where $\vec x = \FV(\phi)$ corresponds to the free variables $\ksi_1, \dots, \ksi_n$ of $\tilde \phi$ over $\boC\Lambda$ via the isomorphism. 

%From Lemma \ref{l.indeterminatevariable} we define, for any element $\settl{y}{\phi(x_1,\dots,x_n)}$ in $\sC(\Lambda_{x_1, x_2, \dots, x_n})$, where $x_1, \dots, x_n$ are variables-treated-as-constants,
%$$\rho \settl{y}{\phi(x_1, \dots, x_n)}$$
%to be the corresponding polynomial in the variables $x_1, \dots, x_n$ over $\sC\Lambda$. % in the variables $x_1, \dots, x_n$. 
%It is a long expression, but it may be written down explicitly. 

%
Finally, we have an extension of Lambek's equivalence between simply typed lambda calculi and cartesian closed categories:

\thm\label{t.chl}
The functors $\boC$ and $\boL$ form an equivalence
$$
\begin{tikzcd}
	\lambdaCalc \arrow[r, "\boC", shift left] \arrow[r, leftarrow, shift right, "\boL" below] & {\bf ICCC} \\
\end{tikzcd}\vspace{-3.5ex} 
$$
between $\lambdaCalc$ and ${\bf ICCC}$. 
\Thm
\prf
%Show that $\varepsilon$ is a natural transformation
For $\sD$ in ${\bf ICCC}$, define $\varepsilon_\sD: \boC\boL \sD \to \sD$ to be the map 
$$\varepsilon_\sD : \settl{x}{\phi} \mapsto 
		\begin{cases} 
			\varepsilon_x \phi, & \text{if $\FV(\phi) = \set{x},$ or $\FV(\phi)$ is empty, or $\ty(\phi) = \trut_\Lambda$,} \\
			\hat x \vdash \hat \phi \,\,\text{ in $\sC$, } & \text{otherwise.} 
		\end{cases}
$$
This map is well-defined since if $\settl{x}{\phi} = \settl{y}{\psi}$, then $\phi[x/z] = \psi[y/z]$, where $z$ does not appear in $\phi$ or $\psi$. Let these be $\phi(z), \psi(z)$. Then $\varepsilon_z \phi(z) = \varepsilon_z \psi(z)$. But $z$ is eliminated by evaluation, so $\varepsilon_x \phi = \varepsilon_z \phi(z) = \varepsilon_z \psi(z) = \varepsilon_x \psi.$
%
Let $F: \sC \to \sD$ in ${\bf ICCC}$. Then to check that $\varepsilon: \sD \mapsto \varepsilon_\sD$ is a natural transformation, that is,
$$\varepsilon(\sD) \of \boC\boL(F) = F \of \varepsilon(\sC),$$
we check that for every $\settl{x}{\phi}$ in $\boC\boL \sC$, where $\phi$ is a bulletin in $x$ over $\sC$,
$$\varepsilon_\sD (\boC\boL F(\settl{x}{\phi})) = F(\varepsilon_\sC (\settl{x}{\phi})).$$
If $\phi$ is a non-constant bulletin in a variable different than the variable appearing in the symbol, then
\begin{align*}
\varepsilon_\sD (\boC\boL F \settl{x}{\phi}) 
	&= \varepsilon_\sD (\boC\boL F \settl{x}{y} ) \\
	&= \varepsilon_\sD \settl{x' : F(\ty(x))}{y' : F(\ty(y))} \\
	&= F(\ty(x)) \vdash F(\ty(y)) \\
	&= F( \ty(x) \vdash \ty(y) ) \\
	&= F(\varepsilon_\sC \settl{x}{\phi}).
\end{align*}
In the other cases, %
this reduces to checking that %
$$F(\varepsilon_x \phi) = \varepsilon_z \boL F \phi,$$
where $\boL F(x) \threeline z.$ We proceed by cases as in the proof of Theorem \ref{t.dedt}: % induction on the length of $\phi$. %
if $\phi$ is a constant $k: \trut \to \hat k$, then 
\begin{align*}
F \varepsilon_x \phi 
	&= F(k) \cdot F(\pi'_{\trut, \trut} \cdot \langle 1_\trut , \trut \vdash \trut \rangle ) \\
	&= F(k) \cdot 1_\trut \\
	&= F(k) \\
	&= \boL F(k) \\
	&= \epsilon_z \boL F(k). 
\end{align*}
If $\phi$ is a variable $x: \trut \to \hat x$ equal to the variable captured by the symbol, then
\begin{align*}
F \varepsilon_x \phi
	&= F \varepsilon_x x \\
	&= F( \pi_{\trut, \hat x} \cdot \langle 1_\trut, \hat x \vdash \trut \rangle ) \\
	&= \pi_{\trut, \widehat{F(x)}} \cdot \langle 1_\trut, \widehat{F(x)} \vdash \trut \rangle ) \\
	&= \varepsilon_{\boL F (x)} \,\boL F(x). 
\end{align*}
The other cases are similar. 

%
For a generalized typed lambda calculus $\Lambda$ in $\lambdaCalc$, define $\eta(\Lambda) := \eta_\Lambda$ of Definition \ref{d.eta}.
%\[
%\eta(\Lambda) : 
%	\begin{cases} 
%		types: & A \mapsto A, \\ 
%		terms: & s \mapsto r(s), 
%	\end{cases}
%\]
%where $r$ is the map of Lemma \ref{l.indeterminatevariable}. %
To show that $\eta$ is a natural transformation, let $\Phi: \Lambda \to \Mu$ in $\lambdaCalc$. Then
$$\eta(\Mu) \of \Phi = \boL \boC(\Phi) \of \eta(\Lambda)$$
becomes, for types, 
$$\eta_\Mu (\Phi(A)) = \boL\boC\Phi (\eta_\Lambda (A)),$$
which is easily verified. Indeed,
\begin{align*}
\boL\boC\Phi(\eta_\Lambda (A))
	&= \boL\boC\Phi(A_{\settl{\bullet}{A}}) \\
	&= A_{\boC\Phi{\settl{\bullet}{A}}} \\
	&= A_{\settl{z}{\Phi(\name{A}) \wr z}}, \quad \ty z = \Phi(\bar A) = \overline{\Phi(A)}, \\
	&= A_{\settl{z}{\name{\Phi(A)} \wr z}} \\
	&= \eta_\Mu (\Phi(A)).
\end{align*}
For terms, we proceed by induction on the length of a term $s$ of $\Lambda$. %(Constant terms have length zero.)
If $s = k$ is a constant term (of length zero), 
\begin{align*}
\eta_\Mu (\Phi(k)) 
	&= \settl{u}{ \Phi k } \quad u:\trut \\
	&= \settl{u }{ \Phi k } \quad u: \Phi(\trut) \text{ since $\Phi(\trut) = \trut$} \\
	&= \boC \Phi \settl{x}{k} \\
	&= \boL \boC \Phi \settl{x}{k} \\
	&= \boL\boC\Phi (\eta_\Lambda(k)).
\end{align*}
If $s = x$, a variable of type $A$, then
\begin{align*}
\boL\boC\Phi(\eta_\Lambda(x)). 
	&= \boL \boC \Phi(\ksi), \quad \ksi: \settl{\bullet}{A} \\
	&= \Phi\ksi, \quad \Phi\ksi : \settl{\bullet}{\Phi A} \\
	&= \eta_\Mu \Phi (x). 
\end{align*}
We can similarly check the other cases $\pi(t), \pi'(t), t \wr r, \langle t, r \rangle, \lambda(y, t)$. 
%this is true because ()()() 

%Show that they are invertible at all points as mappings
Both $\eta_\Lambda$ and $\varepsilon_\Lambda$ are invertible as maps. %
Indeed, by Theorem \ref{t.fCompT}, $\varepsilon$ is injective, and also surjective (since $g \cdot y$ is itself a polynomial). %
%So $\varepsilon$ is invertible. %
%We omit the proof that $\eta$ is invertible, but note that it is much simpler to prove using the approach of \cite{LaSc1}, defining $\eta$ in terms of a universal property. %
%If $\eta(s) = \eta(t)$, then 
%Moreover $\eta$, too, is invertible. Indeed, it is the identity from $\sT_\Lambda$ to $\sT_{\boL\boC\Lambda}$, and on $\sS_\Lambda$, ()()()
%\stuffgoeshere
%Show that the triangle laws are satisfied
To show that $\eta_\Lambda$ is invertible, we use Lemma \ref{l.indeterminatevariable}: if $\phi$ is a polynomial over $\boC\Lambda$ in variables $x_1, \dots, x_n$, we pass via the isomorphism of Lemma \ref{l.indeterminatevariable} from $\phi$ to an element $\phi'$ in $\boC\Lambda_{x_1, \dots, x_n}$ of the form $\settl{y}{t}$. Now note that $\eta_\Lambda(t) = \phi$, so $\eta_\Lambda$ is surjective. On the other hand if $\eta_\Lambda s = \eta_\Lambda t$, for two terms $s,t \in \sS_\Lambda$, then 
$\settl{u:\trut}{s} = \settl{u:\trut}{t} \quad \text{ in $\boC\Lambda_{x_1, \dots, x_n}$.} $ %
so $s = t$ as terms over $\Lambda$, by definition of equality in $\sS_{\boC\Lambda_{x_1, \dots, x_n}}$. %

Next we check (cf. Definition \ref{d.adjunction}) that the triangle laws hold. 
%$$(\boL \of \varepsilon) \vertof (\eta \of \boL) = 1_\boL$$
%becomes $\boL(\varepsilon(\sC)) \of \eta(\boL(\sC)) = \id_{\boL(\sC)}$, which becomes in $\boL\sC$
%$$type: \quad \boL (\varepsilon(\sC)) (\eta(\boL(\sC))(A_f)) = A_f				\quad f \in \sC$$
%$$term: \quad \boL(\varepsilon(\sC)) (\eta(\boL(\sC))(\tilde \phi)) = \tilde \phi		\quad \phi \in \sC(\vec x) \text{ for some $\vec x$}$$
Let $\sC$ be in ${\bf ICCC}$. For a type $A_f$ in $\boL\sC$, 
\begin{align*}
\boL\varepsilon_\sC (\eta_{\boL\sC} (A_f)) 
	&= \boL\varepsilon_\sC (A_{\settl{\bullet}{A_f}}) \\
	&= A_{\varepsilon_\sC (\settl{\bullet}{A_f}) } \\
	&= A_{\varepsilon_z \name{f} \wr z}, \quad z:A_{\bar f}, \\
	&= A_{\varepsilon_z f\cdot z} \\
	&= A_f.
\end{align*}
Next, let $\phi$ be a term of $\boL\sC$, that is, a polynomial over $\sC$ in variables $x_1, \dots, x_n$, say. Then 
\begin{align*}
\boL\varepsilon_\sC (\eta_{\boL\sC} (\phi) ) 
	&= \varepsilon_{\sC(x_1, \dots, x_n)} (\settl{u:\trut}{\phi})  & \text{by Lemma \ref{l.indeterminatevariable}} \\
	&= \varepsilon_{u} \phi 		\\ %	& \text{where $\varepsilon$ is evaluation in $\sC(x_1, \dots, x_n)$} \\
	&= \phi.
\end{align*}
Next, if $\settl{x}{s}$ is an element of $\boC\Lambda$, then we must verify: %
$$\varepsilon_{\boC\Lambda} ( \boC\eta_{\Lambda} \settl{x}{s} ) = \settl{x}{s}.$$
The first case we check is that where $s$ is a bulletin in a variable not equal to that appearing in the symbol. %
Then 
$\settl{x}{s} = \settl{x}{y} = \ty(x) \vdash \ty(y)$ 
for some variable $y$, with $\ty(y) = \ty(x)$. We have
\begin{align*}
\varepsilon_{\boC\Lambda} (\boC \eta_\Lambda \settl{x}{s} ) 
	&= \varepsilon_{\boC\Lambda} ( \settl{\ksi}{\ksi'} ) , \quad \text{where $\ksi:\settl{\bullet}{\ty(s)}$, $\ksi':\settl{\bullet}{\ty(x)}$} \\
	&= \hat{\ksi} \vdash \hat{\ksi'} \\
	&= \settl{x}{y} \\
	&= \settl{x}{s}.
\end{align*}
Next, we check when %
$s = k$ is a constant term of type $B$ in $\Lambda$, and $x$ is a variable of type $A$ in $\Lambda$. Then 
\begin{align*}
\varepsilon_{\boC\Lambda} (\boC\eta_\Lambda \settl{x}{k} )
	&= \varepsilon_{\boC\Lambda} \settl{\eta_\Lambda x}{\eta_\Lambda k} \\
	&= \varepsilon_{\boC\Lambda} \settl{\ksi}{\settl{u}{k}}, 
		\text{ where $\ksi$ has type $\eta_\Lambda \ty(x)  = \settl{\bullet}{A} = \settl{v:\trut}{A \wr v}$, and $u:\trut_\Lambda$} \\
	&= \varepsilon_\ksi \settl{u}{k} \\
	&= \settl{u}{k} \cdot \arter_{\hat \ksi} \\
	&= \settl{u}{k} \cdot \settl{w:A}{*} \\
	&= \settl{w:A}{k[x/*]} \\
	&= \settl{w}{k} \\
	&= \settl{x}{k}.
\end{align*}
% since $\settl{x}{k} \cdot \ksi = \settl{*^\cdot}{k}$. ???? %
Next, if $s = x$ is a variable of type $A$ and is the same variable as that appearing in the symbol, then
\begin{align*}
\varepsilon_{\boC\Lambda} (\boC\eta_\Lambda \settl{x}{x} )
	&= \varepsilon_{\boC\Lambda} \settl{\eta_\Lambda x}{\eta_\Lambda x} \\
	&= \varepsilon_{\boC\Lambda} \settl{\ksi}{\ksi}, \quad \text{ where $\ksi$ has type $\settl{\bullet}{A} = A$},\\
	&= \varepsilon_\ksi \ksi \\
	&= \settl{\ksi}{\ksi} \\
	&= \settl{x}{x}.
\end{align*}
The other cases are proved similarly. Hence the triangle laws hold, and the theorem is proved. 
%$$(\varepsilon \of \boC) \vertof (\boC \of \eta) = 1_\boC$$
%becomes 
%$\varepsilon(\boC(\Lambda) \of \boC(\eta(\Lambda)) = \id_{\boC(\Lambda)}$, %
%becomes in $\boC\Lambda$ %
%$$\varepsilon(\boC(\Lambda)) (\boC(\eta(\Lambda)) (f_A)) = f_A 					\quad f_A \in \boC\Lambda \text{ coming from a type $A$}$$
%$$\varepsilon(\boC(\Lambda)) (\boC(\eta(\Lambda)) (\settl{x}{s})) = \settl{x}{s}	\quad \settl{x}{s} \in \boC\Lambda \text{ coming from a term $s$}$$
%\stuffgoeshere
% ayd
\Prf

%The category $\boC(\boL(\sC))$ is sometimes called the {\em syntactic category} of $\sC$. %


% \\ end










































































































%\input{th6(conc)} % conclusion
% File th6(conc).tex Created by Lucius Schoenbaum September 27, 2016
% conclusion and future work































\section{Conclusion and Future Work}\label{s.th-conc}

%PSEUDO-ONE-DIMENSIONAL % referring to this in the paper "ideal toposes" (it.tex), so define it here.

In this chapter we have shown that cartesian closed structure can be modified to include mappings on the set of objects that recover the base and the power of an exponential. %(it is a common misconception, often assumed implicitly, that such mappings exist in general). 
We have indicated that the mathematics of cartesian closed categories is not affected by this addition, and moreover, by making this modification, we widen further the class of admissible functors (for some purposes relevant to categorical logic and type theory) %involved in the Curry-Howard-Lambek correspondence and similar things
to include arbitrary cartesian functors. We have also shown that this calculus extends beyond categories, to generalized categories. % of Chapter \ref{s.gencat}. 
We have also presented a lambda calculus which permits the extension of the Curry-Howard-Lambek correspondence to the general case. % rather only the pseudo-one-dimensional case that arises from one-dimensional categories \ref{t.????}. 
%In future work we hope to carry this extension all the way through. This will yield a strongly-featured type system on the language side, possessing polymorphic types and more besides.
Our work suggests that polynomials over categories and terms over types are in fact essentially the same thing. This can also be seen also in the ordinary categorical case, but in the generalized setting, the observation is made unavoidable. % Our work also suggests that 
The fundamental insight of the Curry-Howard correspondence is thus that the cartesian closed structure on a cartesian closed category can be expressed almost entirely in terms of properties of objects in the space of polynomials. This seems (to the author) to be the mathematical content of the theorem. 

Because of the rich variety of subject matter in categorical logic and related subjects, there are a number of directions in which this work can be continued. %for instance, %We conclude with a brief,  incomplete survey. 
%Moggi's metalanguage. %}\label{ss.moggi}
%
For example, the work of Moggi on computational effects \cite{MoI1} has had an influence on much subsequent work (see for example Wadler\cite{WaR2}, Mulry, \cite{MuY1,MuY2}, Kobayashi \cite{Kobayashi1}, Semmelroth and Sabry \cite{SemmelrothSabry1}). In \cite{MoI1}, an extension of the lambda calculus is introduced and it is shown that it is possible to provide categorical semantics for computational effects by making use of monads. %and the Curry-Howard correspondence. %extended in the work above. % 
In fact, two constructions are presented. The first relates a cartesian closed category equipped with a monad to a monadic equational theory (one in which contexts consist of a unique typed variable) extended by a computational effect (he calls this the {\em simple metalanguage}), and the second relates a {\em strong} monad to a general equational theory (what Moggi calls an {\em algebraic} equational theory, this one called the {\em metalanguage}), i.e., one in which contexts may be arbitrary finite lists of typed variables. %
%Moggi's first construction is enough to 

Let $\sC$ be a category with a monad $T = (T, \eta, \mu)$. Then $T$ is a {\em strong monad} if it is equipped with a natural transformation $t$ from the functor $(-) \times T(-):  \sC \times \sC \to \sC \times \sC$ to the functor $T(- \times -) :  \sC \times \sC \to \sC \times \sC$ (where $\times$ denotes both the product in Cat and the product in $\sC$). This $t$, called a {\em strength}, must additionally satisfy the identities:
$$(T \of \pi_{1,A}) \vertof t_{1, A} = \pi'_{1, TA},$$
$$(T \of \alpha_{A,B,C} ) \vertof t_{A \times B, C} = t_{A, B \times C} \vertof (1_A \times t_{B, C}) \vertof \alpha_{A, B, TC}$$
$$t_{A,B} \vertof (1_A \times \eta_B) = \eta_{A \times B}$$
$$t_{A,B} \vertof (1_A \times \mu_B) = \mu_{A \times B} \vertof (T \of t_{A,B}) \vertof t_{A, TB}$$
where notation is the same as in the preceding sections, except $\times$ denotes the product in $\sC$. 
%
%which he calls the {\em simple metalanguage}, consists of 
%may be extended to the generalized setting in the following way. 
%
In Chapter \ref{c.genm}, the fundamental parts of the theory of monads are extended to the setting of generalized categories in two ways, one via a generalized triple, and the other via a generalized Kleisli construction. %(It is unclear to the author which is more useful.) 
%Future work is to relate this work to the simple metalanguage. As of this writing the task seems very near to completion, however, 
Questions remain about how the present work is connected to Moggi's, since the Kleisli category in the generalized setting is a more subtle construction than in the one-categorical setting. 
%
%Once this has been achieved, the way will be open to the substantial literature on descendants of Moggi's type theory  %, Fluet \cite{Fluet1}). 


%\subsection{Generalized Type Theories}\label{ss.gentyt}
%dependent type theory
%homotopy type theory
%hyperdoctrines and locally cartesian closed categories


%Future work should be devoted to ()()(). 
%We should also make mention of the program launched in the book \cite{hott1}. Homotopy type theory, as a system based on dependent type theory, should be studied in order to find common ground and a better harmony with the ideas and methods demanded by generalized category theory. 

%The identity types in homotopy type theory are at the heart of the innovations it brings to Martin-L{\"o}f dependent type theory. Recall that %
%\stuffgoeshere
%We can use the ideal of types to simplify this set of ideas as follows. 
%\stuffgoeshere

%HYPERDOCTRINES, LOCALLY CARTESIAN CLOSED CATEGORIES, SEMANTICS OF ML TYPE THEORY 

%STUFF TO SAY ABOUT DEPENDENT TYPES
%\enu
%	\item There they use telescopes, that corresponds nicely
%	\item In our framework we have more, for example, GIVE AN EXAMPLE
%	\item This leads to several open avenues of research, for example, is it possible to construct a type-checking algorithm based on generalized theories? REFINE THIS LIST OF POSSIBLE THINGS TO DO
%	\item Dependent types can be modeled in the system by DO THIS
%\Enu



%

%

%

% PfT
%\subsection{Sequent Calculi}\label{ss.pft}
%The generalization of the notion of a sequent permits the extension of generalized categories to generalized trees with selected binary partitions of children. This observation suggests a path for developing categorical models for proof-theoretical systems. Several other systems that all may be analyzed in a similar manner. This is a task for future investigations. 


%This work might be viewed as arising from a critique of the ordinary approach to type theory and proof theory, which distinguishes between the notion of {\em type} and the notion of {\em sequent}. The latter term


%goal: do LaSc1 p. 128ff and include the solution of Russell's paradox and all of that.
%``We will also show that this generality allows us to ...do things... by providing semantics for dependent types. In a future work we will do the same for polymorphism and calculus of constructions.''









%

%

%

% ToT
%\subsection{Topos Theory}\label{s.tot}

%We answer this by showing that any generalized category may be extended to a cartesian closed structure via the usual topos-theoretic constructions, suitably generalized. %
It is also possible to extend our work in this paper to the setting of topos theory. We may define:

%\subsection{Sheaves over Generalized Base}\label{ss.sht}

\dfn
An {\em ideal elementary topos} %or {\em generalized elementary topos} 
is an ideal cartesian closed category with 
\enu
	\item all finite limits and colimits,
%	\item ideal cartesian closed,
	\item a subobject classifier.
\Enu
\Dfn

\noindent An investigation into topos theory in the generalized setting (including several sheaf theoretical constructions) is made in Chapter \ref{c.it}. % begun, but it remains still at the time of this writing a work in progress. 

%We will not continue the development of elementary topos theory here, however we note that 
%\stuffgoeshere

\begin{comment}
%BUT WHAT ABOUT gTOT? $\longleftarrow$
Toposes were originally introduced as categories equivalent to categories of sheaves in \cite{sga4}, and later \cite{LaTi1} the definition was simplified. %
Nevertheless, the sheaf-theoretic framework continues to be important in topos theory and related subjects. %
One reason for this is {\em the problem of examples}: where in nature are the general structures we call toposes be found? How can they be recognized? %
Sheaf theory provides the answer to both of these questions, due to the importance and prevalence of sheaf categories (and their many variants) throughout most areas of mathematics. %

We have shown that a way to construct a ``nontrivial'' (in the sense of: not 1-dimensional, not cellular) generalized categories is by constructing an ideal cartesian closed category from a given cartesian closed category. %
But what about the still more general notion? %
%We can find such structures in the domian of sheaves, well-known to be a categorical paradise. %
%In order to achieve such generality, however, we must abandon the functorial approach in favor of that of profunctors. Our preference regarding profunctors is to rely upon the following abstraction: 
This motivates the study of generalized sheaves. %

\dfn
A {\em bipartite generalized category} is a generalized category carrying the structure of 
\Dfn

\dfn
A {\em presheaf} is a bipartite generalized category whose 
\Dfn

%``I think it should be a functor.'' 
It shouldn't be a functor because a functor does not give a sufficiently general construction. % 

\prop
%ideally??????
Let $\sE$ be a generalized presheaf category over a base $\sC$. Then $\sE$ is an 
ideal elementary topos. 
\Prop
\prf
\stuffgoeshere
\stuffgoeshere
\Prf

%TASK: CHECK THIS IS TRUE 
If we define a {\em generalized topos} to be a cartesian closed generalized category with finite limits and a subobject classifier, then we obtain in this way a large class of rich generalized categories. The ideal of types studied in section \ref{s.th} can be added to such structures as well, if desired. 
\end{comment}











































































\pagebreak
\singlespacing
\chapter{Ideal Toposes}\label{c.it}
\doublespacing
\vspace{10ex}
%\input{it1(i)}
% File it1(i).tex Created by Lucius Schoenbaum October 14, 2016
% introduction































\section{Introduction}\label{s.it-i}

In Chapter \ref{c.th} it was shown that the framework of generalized categories may be used introduce an ideal of types to an ordinary category. % giving a formal foundation to commonplace pattern-matching operations in functional programming. 
Thus generalized categories (of which ordinary categories are a special case) can be given a structure possessing all the desirable properties of a cartesian closed structure, in spite of there being no way to put a true cartesian closed structure on a generalized category, since the mapping $- \times X$ %ordinarily a functor, 
is not a functor in general. The structure that arises in its stead is an ideal cartesian closed category: %
a generalized category equipped with cartesian structure and an evaluation operation, along with types and a set of constants. %
In this Chapter, we prepare the way for the study of topos theory in the generalized setting. %
We study both the elementary case and the case in which the topos structure arises from the existence of a site of definition. %

%In section \ref{s.gencat} we review the basic theory of generalized categories \cite{ScMd}. Then 
In section \ref{s.it} we define ideal elementary toposes and develop basic facts and constructions in that setting, following \cite{sga4,MaMo1,JoN1,FrD1,LaSc1}. An ideal topos is an ideal cartesian closed category equipped with finite limits, finite colimits, and a subobject classifier. After establishing preliminary results we show that the slice theorem can be generalized, and that for this reason, the generalized topos theory has, perhaps surprisingly, much of the same character as the usual one-categorical topos theory, arising from the Heyting algebra structure present and the rich system of adjoints generated by an arrow. %
%In section \ref{s.ty} we develop the notion of a generalized type theory by extending the notion of a generalized typed lambda calculus \cite{ScM3}. 
A major feature of ordinary topos theory is the relationship with sheaf theory through which there arises a rich world of examples and applications. %
In section \ref{s.sht}, we develop a generalization of sheaf theory, and show that the connection between the categorical theory of sheaves developed by Grothendieck and his school \cite{sga4} and the theory of toposes developed by Lawvere, Tierney and others \cite{LaTi1} are in relation with one another in the generalized setting much as in the one-categorical case. %This involves some new constructions: a new Grothendieck category, a new notion of comma category, and an approach to sheaves that extends beyond the functorial approach, necessitated by the need for constructions related to 
An approach to sheaves that differs from the usual functorial approach is necessary in order to (in particular) carry %
the Yoneda embedding through in the generalized setting. % This change in approach leads to a generalized notion of sets. 
The approach we adopt uses pointed profunctors. 
%atop of this ``pseudo'' cartesian closed structure (it was called an {\em ideal} cartesian closed structure in \cite{ScM3}). 

%()()()

Aside from yielding information about the extent of applicability of topos theory, a result shown by our work is that it is often easier to work in the generalized setting than in the one-categorical setting. In part this may be due to the same phenomenon that occurs with other kinds of generalizations: in the general setting there are sometimes fewer possible directions for an argument to go in, thus reasoning in the general setting is psychologically more straightforward than in settings that arise from applications. 

The style in which this chapter is written is a departure from the style of most works in category theory. In part, this is a response to the needs of the subject matter. The author, much of whose training in mathematics was in analysis, %
has experienced first hand the reality that %
category theory is far better appreciated and far better understood in some circles than in others. %
He believes that an understanding and appreciation of the power and concreteness of categorical methods is possible among all mathematicians. %
There is, in the author's view, more than enough room for a wide variety of stylistic approaches to category theory. 






























































%\input{it3(it)}
% File it3(it).tex Created by Lucius Schoenbaum October 14, 2016
% basic ideal elementary toposes







































\section{Basic Ideal Topos Theory}\label{s.it}

In a cartesian closed category, corresponding to a fragment of intuitionistic logic, there is one truth value, $\trut$, possibly accompanied by a symbol for absurdity $\bottom$. %
In a topos, this single truth value $\trut$ is expanded via a monic $\trut \armonic \Obsoc$. %
The intuition for the role of this monic, coming from sheaf theory, is {\em truth-in-place}, or localized truth. %
This means that what is true in a topos $\sE$ (or in the internal language of $\sE$) depends in general both on {\em location} in the topos and on {\em how far one can see} in the topos. Such an intuition might be a useful guide as we build the formalism.

\dfn\label{d.soc}
A {\em subobject classifier} for a generalized category $\sE$ is a pair $(\arsoc, \Obsoc)$ where $\Obsoc$ is an object in $\sE$ and where $\arsoc: \trut \armonic \Omega$ is in $\sE$, such that for every monic $m:a \armonic b$ in $\sE$, there is a unique arrow $p: b \to \Obsoc$ such that the following square is a pullback:
\[
\begin{tikzcd}
a \arrow[tail]{d}[swap]{m} \arrow{r}{\arter} 		& \trut \arrow[tail]{d}{\arsoc} \\
b \arrow{r}[swap]{p} 					& \Obsoc 
\end{tikzcd}
\]
\Dfn

\dfn\label{d.it}
An {\em ideal elementary topos} is an ideal cartesian closed category $\sE$ with
\enu
	\item all finite limits and colimits,\footnote{As in the one-category case, it suffices to have a terminal/equalizers/products, or a terminal/pullbacks (dually for colimits) \cite{ScMi}.} 
	\item a subobject classifier. 
\Enu
\Dfn

%In Definition \ref{d.it}, 
%In \cite{ScM3}, we used the notation $\wedge$ for the product in a category and $\trut$ for the terminal. %We will write $a \sqcup b$ for the coproduct. 
%We will use notation $A \times B$ and $1$. 

%Until specified otherwise {\em ideal topos} will refer to an elementary ideal topos (Definition \ref{d.it}).

An ideal elementary topos $\sE$ has equalizers: for every pair of elements $f,g : a \to b$ in $\sE$, there is an arrow $e:e_0 \to a$ which has the property that $f \cdot e = g \cdot e$, and which is universal with respect to this property \cite{MacCW}. 
%Similarly, for every pair of elements
An equalizer is denoted $\Eq(f,g)$. This notation denotes an arrow well-defined in $\sE$ only up to isomorphism. 
%
It is also convenient to define, for element $a \in \sE$,  
$$\arsoc_a \threeline \arsoc \cdot \arter_a.$$
%\dfn
%If $\sC$ is an ideal category, an element of the form $a \vdash b$, for $a,b \in \sC$ is a {\em type-as-path} or a {\em type} of $\sC$ \cite{ScM3}. %
%A {\em term} $t$ in an ideal category with terminal object is an element with source $1$, where 1 is the (choice of) terminal object. %
%The {\em type} of a term $t$ is (synonymously) the target of $t$. %
%Let $\sE$ be an ideal elementary topos. %
%$\Omega$ is the {\em truth object} of $\sE$. %
%If $a \in \sE$, the type $a \vdash \Omega$ is the {\em power} of $a$, %
%also denoted $\sP a $. %

Because an ideal elementary topos is an ideal cartesian closed category, it carries the bijection:
\begin{equation}\label{eq.cccbij}
\hom(a,b) \bij \hom(1, a \vdash b).
\end{equation}
The element on the right, in terms of the element on the left, is denoted \cite{ScM3,LaSc1} 
$$\name{f}$$
This is called the {\em name} of $f$. 
%\stuffgoeshere % is it natural? does idealness cause something new to happen?
%
%Let $\sE$ be an ideal elementary topos. 
In the case $b = \Omega$ the bijection (\ref{eq.cccbij}) becomes:
\begin{equation}\label{eq.namebijpower}
\hom(a,\Obsoc) \bij \hom(1, a \vdash \Omega).
\end{equation}
The element $a \vdash \Omega$ is viewed as a power of $a$ (cf. Theorem \ref{t.sliceT}). %
For example, in the category of sets, it reduces to the set of maps from a set $A$ to the two-element set $\set{0,1}$. % 
The bijection (\ref{eq.namebijpower}) is augmented in a topos. Let $\Subobj (a)$ denote the subobjects of an element $a \in \sE$, that is, the monics of target $a$ taken only up to isomorphisms between sources \cite{MacCW}. 

\prop\label{p.predbij}
For every $a \in \sE$, there is a bijection
\begin{equation}\label{eq.predbij}
\Subobj (a) \bij \hom(a, \Obsoc) %\bij \hom (1, \sP a)$$
\end{equation}
\Prop
\prf
Consider the map which sends a monic $m$ to the element in the lower row of the diagram of Definition \ref{d.soc}. %
It lifts to a well-defined map on subobjects, by a simple diagram chase. %
It is equally clear that it is injective, and if $\phi:a \to \Obsoc$ is given, then since $\arsoc$ is monic (Definition \ref{d.it}), the pullback of $\phi$ and $\arsoc$ is also monic, so it is also surjective. %
\Prf

The bijections (\ref{eq.namebijpower},\ref{eq.predbij}) are natural in $a$. %
A {\em predicate} in $\sE$ is an element $\phi$ in an ideal elementary topos $\sE$ with target $\Obsoc$. As in the proof of Proposition \ref{p.predbij}, we let 
$$\pred(m), \quad \pred([m]), \quad \pred(s)$$
denote the predicate associated to monics $m: \bar m \rightarrowtail \hat m$, subobjects $[m]$, and elements $s: (a \vdash \Omega) \to \trut$ via Proposition \ref{p.predbij} and Equation \ref{eq.namebijpower} (it does not lead to any confusion to overload the symbol $\pred$ in this way). The predicate $\pred(m)$ is also called the {\em characteristic function}, the {\em characteristic map}, or the {\em classifying map} of $m$, or of the subobject defined by $m$, or of $s: 1 \to (a \vdash \Omega)$. %
%
Given a predicate, we can also apply the bijection (\ref{eq.predbij}) in the other direction to obtain a subobject from a predicate. %We denote by
%$$\subob(\phi).$$
%the subobject obtained this way from a predicate $\phi$. 
In general, however, we cannot obtain a canonical monic given a predicate. We can do so in some cases, for example, in the category of sets we may use subsets. Analogous canonical monics are given more generally in generalized presheaf categories by subpointed profunctors (see section \ref{s.sht}). %
% Beyond that, however, we may be stuck with subobjects. 
For many purposes, however, such monics, though appealing, are not really necessary. 

Because we will make limited use of elementary ideal toposes% in this section% and in section \ref{s.ty}
, we introduce only Lambek's {\em strict} logical morphisms \cite{LaSc1}: % (logical functors are weaker in general \cite{JoN1,MaMo1}):

\dfn\label{d.catidealtop}
A {\em morphism} $F:\sE \to \sE'$ of ideal elementary toposes is a functor \cite{ScM3} of ideal cartesian closed categories $F:\sE \to \sE'$ such that
such that 
$$F(\Obsoc) = \Obsoc,$$
$$F(\pred(m)) = \pred(F(m))$$
for all monics $m$ in $\sE$. %
This gives a category $\idealTop$ of elementary ideal toposes. %
\Dfn

%Applying \ref{p.mainbijections}, characteristic predicate of subobject (or monic) $m$, and of an atom:
%$$\pred(m)$$
%$$\pred(s)$$


% alphabet:
% a,b,c,f,g,h 	elements of \sE
% phi psi		polynomials over \sE
% s,t,r 		terms of abstract type theory
% A,B,C		types of abstract type theory
% m,[a],...		monics/subobjects
% p,q,r 		predicates

\prop\label{p.epimoniciso}
Let $\sE$ be an ideal elementary topos. Then every monic in $\sE$ is an equalizer, and every epi monic is an isomorphism. 
\Prop
\prf
Because $\sE$ has a subobject classifier, every monic in $\sE$ is an equalizer. Indeed, 
$$\pred(m) \cdot m = \arsoc \cdot \arter_a = \arsoc \cdot \arter_b \cdot m,$$
and from this observation it is not difficult to see that a monic $m:a \to b$ is the equalizer of $\pred(m)$ and $\arsoc_b$. %
Therefore a monic is an equalizer of, say, $f$ and $g$. If it is also epi, then it follows that $f = g$. Therefore the identity $1_{\hat m}$ also has the equalizing property, giving a universal arrow $k: \hat m \to \bar m$. So $k\cdot m = 1_{\hat m}$. Moreover $m \cdot k = 1_{\bar m}$, since 
$$m \cdot k \cdot m = m = m \cdot 1_{\bar m}$$
and $m$ is monic. %
\Prf

Let $a \in \sE$. Let $\Delta_a$ be the diagonal map
$$\Delta_a := \langle 1_a, 1_a \rangle.$$
This map is monic, hence it has an associated predicate
$$\delta_a := \pred (\Delta_a).$$
A functor of ideal cartesian closed categories $F: \sE \to \sE'$ between ideal elementary toposes preserves $\pred()$ if and only if it preserves the $\delta$:
$$F(\delta_a) = \delta_{F(a)},$$
for all $a \in \sE$. %
A pullback diagram
$$
\begin{tikzcd}
\cdot \arrow[r, "\pullb'_{f,g}"] \arrow[d, swap, "\pullb_{f,g}"] & 
b \arrow[d, "g"] 
\\
a \arrow[r, "f"] &
c
\end{tikzcd}
$$
gives rise to another pullback diagram
$$
\begin{tikzcd}
\cdot \arrow[d, swap, "\langle \pullb_{f,g} {,} \pullb'_{f,g} \rangle"] \arrow[r, "f \cdot \pullb_{f,g} %= g \cdot \pullb'_{f,g}
	"]  & 
c \arrow[d, "\Delta_c"]
\\
a \wedge b \arrow[r, "f \wedge g"] &
c \wedge c,
\end{tikzcd}
$$
and conversely. Here (as in 
previous chapters) $a \wedge b$ denotes the product of $a$ and $b$. %(the author finds the notation $a \times b$ confusing except in cases where the limit is created by limits in the category of sets). 

We can also define 
$$\set{\cdot}_a := (\delta_a)^*,$$
where $()^*$ is the exponential transpose in $\sE$. %
We also take the predicate
$$\sigma_a := \pred(\set{\cdot}_a)$$
once we show that: %$\set{\cdot}_a$ is monic: 

\prop\label{p.singletonmonic}
$\set{\cdot}_a$ is monic.
\Prop
\prf
Suppose that $\set{\cdot}_a \cdot b = \set{\cdot}_a \cdot b'$. Then, in the notation of \cite{ScM3},
\[
(\delta_a \cdot \langle b \cdot \pi_{\bar b, a} , \pi'_{\bar b, a} \rangle )^* = (\delta_a \cdot \langle b' \pi_{\bar b, a} , \pi'_{\bar b, a} \rangle )^* 
\]
so
\[
\delta_a \cdot \langle b \cdot \pi_{\bar b, a} , \pi'_{\bar b, a} \rangle = \delta_a \cdot \langle b' \pi_{\bar b, a} , \pi'_{\bar b, a} \rangle.
\]
So 
\begin{equation}\label{eq.deltawedge1}
\delta_{a} \cdot (b \wedge 1_a) = \delta_a \cdot (b' \wedge 1_a).
\end{equation}
In the commutative diagram
$$
\begin{tikzcd}
\bar b \arrow[r, "b"] \arrow[d, swap, "\langle 1_{\bar b} {,} b \rangle"] & 
a \arrow[r, "ter_{a}"] \arrow[d, "\Delta_{a}"] & 
\trut \arrow[d, "\arsoc"] 
\\
\bar b \wedge a \arrow[r, "b \wedge 1_{a}"] &
a \wedge a \arrow[r, "\delta_{a}"] &
\Obsoc
\end{tikzcd}
$$
both squares are pullbacks. %
Hence the square formed by combining the two squares is also a pullback square. %
By equation (\ref{eq.deltawedge1}), 
the pair $(\langle 1_{\bar b}, b \rangle, ter_a \cdot b)$ is also a pullback for the diagram in which the bottom row is replaced with 
$\bar b \wedge a \overset{b' \wedge 1_a}{\to} a \wedge a \overset{\delta_a}{\to} \Obsoc$. %
So there is an invertible $\theta$ such that $\langle 1_{\bar b}, b \rangle \cdot \theta = \langle 1_{\bar b}, b' \rangle$. %
So $\langle \theta, b \cdot \theta \rangle = \langle 1_{\bar b}, b' \rangle$. %
So $\theta = 1_{\bar b}$ and $b \cdot \theta = b'$, and so $b = b'$, as desired. 
\Prf

Hence $\sigma_a$ is well-defined. The predicates $\sigma_a$ and $\delta_a$ may be interpreted, respectively, 
as tests: 
$$\text{$f: c \to (a \vdash \Omega)$ is a singleton} \quad \becomes \quad \sigma_a \cdot f = \trut_{c} $$
$$\text{$b,c :d \to a$ are equal} \quad \becomes \quad \delta_a \langle b,c \rangle = \trut_{d} $$
These interpretations can be extended to polynomials over $\sE$ \cite{ScM3}. %in section \ref{s.ty}. 
In the latter case, the interpretation is justified in the formalism of $\sE$:
%along with evaluation interpreted as
%$$\text{$foo$}$$
%\prop
%There are the following maps:
%\enu
%	\item $\wr_b$
%	\item $\set{\cdot}_b$ % MaMo1 p. 166

\prop
Let $\sE$ be an ideal elementary topos, and let $b,c: d \to a$ in $\sE$. Then 
$$b = c \eifaoif \delta_a \langle b,c \rangle = \trut_d.$$
\Prop
\prf
We follow the proof found in \cite{LaSc1}, which applies without change in the generalized setting. 
If $b = c$, then 
\begin{align*}
\delta_a \cdot \langle b, c \rangle 
&= \delta_a \cdot \langle b, b \rangle \\
&= \delta_a \langle 1_a, 1_a \rangle b \\
&= \arsoc_a \cdot b = \arsoc_{\bar b}.
\end{align*}
Conversely, if $\delta_a \cdot \langle b,c \rangle = \arsoc_a$, 
then the pair $(\langle b, c \rangle, ter_a)$ form a cone on the diagram 
$a \wedge a \overset{\delta_a}{\to} \Obsoc \overset{\arsoc}{\leftarrow} \trut$. 
But this diagram has pullback $a$. So there is a unique $k: d \to a$ such that 
$$
\langle b,c \rangle = \Delta_a \cdot k = \langle k, k \rangle.
$$
By taking projections, we have $b = c = k$. % ayd
\Prf

Another important result, supplying monic ``images'' in toposes, continues to hold in the generalized setting:

\prop\label{p.imagee}
Let $\sE$ be an ideal elementary topos. For every $f \in \sE$, there is a factorization of $f$
$$f = m \cdot e$$
where $e$ is epi and $m$ is a monic that is universal in the sense that for every monic $\tilde m$ such that there exists $e$ such that $f = \tilde m \cdot e$, there exists $e'$ such that $m = \tilde m \cdot e'$. 
\Prop
\prf
Using the existence of finite colimits, take the pushout of the diagram $\hat f \overset{f}{\leftarrow} \bar f \overset{f}{\to} \hat f$ and call the injections $x$ and $y$. 
These form a diagram 
$
\begin{tikzcd}
\hat f \arrow[r, shift right, swap, "y"] \arrow[r, shift left, "x"] &
\cdot
\end{tikzcd}
$; 
let $m$ denote their equalizer. 
Since $f$ has the equalizing property (i.e., $x \cdot f = y \cdot f$), $f$ factors through $m$. 
Call this factorization $f = m \cdot e$. 
See the figures:
\[
\begin{tikzcd}
\cdot \arrow[r, "f"] \arrow[d, swap, "f"] &
\cdot \arrow[d, "y"] \\
\cdot \arrow[r, "x"] &
\cdot 
\end{tikzcd}
\quad\quad
\begin{tikzcd}
\cdot \arrow[d, swap, "e"] \arrow[dr, "f"] \\
\cdot \arrow[r, swap, "m"] &
\cdot \arrow[r, shift right, swap, "y"] \arrow[r, shift left, "x"] &
\cdot 
\end{tikzcd}
\]
Say that $m$ has the {\em image property} if
$$\text{for all } \tilde m, \tilde e, f = \tilde m \cdot \tilde e, \tilde m \text{ monic} \eimplies \text{ there exists } u \text{ such that } m = \tilde m \cdot u.$$
We claim that $m$ has the image property. Indeed, if 
$f = \tilde m \cdot \tilde e$, $\tilde m$ monic, then by Proposition \ref{p.epimoniciso} $\tilde m$ is an equalizer, say, of $s$ and $t$ in $\sE$. But if 
$s \cdot \tilde m = t \cdot \tilde m$, then 
$s \cdot \tilde m \cdot \tilde e = t \cdot \tilde m \cdot \tilde e$, so $s \cdot f = t \cdot f$. 
This gives a unique $u$ such that $s = u \cdot x, t = u \cdot y$, see figures:
\[
\begin{tikzcd}
\cdot \arrow[r, "f"] \arrow[d, swap, "f"] &
\cdot \arrow[d, "y"] \arrow[ddr, "t"] \\
\cdot \arrow[r, swap, "x"] \arrow[drr, swap, "s"] &
\cdot \arrow[dr, "u" near start] \\
&
&
\cdot 
\end{tikzcd}
\quad\quad
\begin{tikzcd}
\cdot \arrow[d, swap] \arrow[dr, "m"] \\
\cdot \arrow[r, swap, "\tilde m"] &
\cdot \arrow[r, shift right, swap, "s"] \arrow[r, shift left, "t"] &
\cdot 
\end{tikzcd}
\]
Therefore $u \cdot x \cdot m = u \cdot y \cdot m$, so $s \cdot m = t \cdot m$,
so $m$ is an imposter cone for $\tilde m$. So $m$ factors through $\tilde m$, as desired. 

Now $m$ is monic since it is an equalizer; it remains to show that $e$ is epi. 
%Consider the factorization $f = m \cdot e$ and 
Suppose that we repeat the construction 
(equalizer of pushout) on the arrow $e$ instead of $f$; %
this yields a factorization $e = m' \cdot e'$ of $e$. The composition 
$m \cdot m'$ is monic, and $f$ factors through it, hence $m$ factors through it 
as well, say, $m = m \cdot m' \cdot v$. Therefore $1_{\hat{m'}} = m' \cdot v$. 
Therefore the monic $m'$ is epi, hence $m'$ is an isomorphism, 
again by using Proposition \ref{p.epimoniciso}. 
Now if the equalizer of the pushout of $\cdot \overset{e'}{\leftarrow} \cdot \overset{e'}{\to} \cdot$ is an isomorphism, then the projections of the pushout are equal. %
It can be checked that $e'$ is therefore epi. %
It follows immediately that $e$ is epi as well, as was to be proved. 
\Prf

The $m$ of Proposition \ref{p.imagee} (the equalizer of the pushout of $\cdot \overset{f}{\leftarrow} \cdot \overset{f}{\to} \cdot$) is called the {\em image} of $f$ and we denote it by $\im(f)$, following \cite{FrD1}. %Below (section \ref{s.slicecon}) we learn more about $\im(\cdot)$ after introducing the subobject. 

\cor\label{c.imagee}
The subobjects below $a \in \sE$ form a lattice. 
\Cor
\prf
For monics $m,n$ below $a$, define:
$$[m] \cap [n] := [n \cdot \pullb_{m,n}],$$
$$[m] \cup [n] := [\im [m,n]],$$
where $[m,n]$ is the universal arrow with respect to the coproduct $\bar m + \bar n$, following \cite{LaSc1}. %
These operations are well-defined, and provide a greatest lower bound (least upper bound, respectively). %
%Finally, $ini_a : \nlld \to a$, $1_a : a \to a$ provide the minimum and maximum, respectively. %
\Prf

The lattice $\Sub(a)$ has maximal element given by $1_a: a \to a$. In fact, in toposes $\Sub(a)$ has still more order-theoretic structure, but this takes time to prove (cf. Corollary \ref{c.sliceT}). 
%
Moreover, there is a morphism of lattices (i.e., an order-preserving map) $\Sub(a) \to \Sub(b)$ given an arrow $k: b \to a$, given by the pullback
$$k^{-1}([m]) := [\pullb_{k,m}].$$
%
Next we make the following observation:

\prop\label{p.subterob}
% element $a \in \sE$ is {\em open} if 
Let $a$ be an element of an ideal elementary topos $\sE$. 
The following are equivalent: 
 \enu
 	\item The terminal arrow $ter_a: a \to \trut$ is monic. 
	\item For all $b \in \sE$, there is at most arrow $b \to a$. 
\Enu
\Prop

If $a \in \sE$ has the property of Proposition \ref{p.subterob}, it is said to be {\em open}. 




% // s.it





















\begin{comment}

\thm\label{t.sliceT}
Let $\sE$ be an ideal elementary topos, let $u \in \sE$. Then the slice $\sE/u$ is also an ideal elementary topos. 
\Thm
\prf
We first cut the distance down by showing that parts of Definition \ref{d.it} are redundant: %using an alternative characterization of an ideal elementary topos is the following:

\prop
Let $\sE$ be an ideal category with
\enu
	\item all pullbacks,
	\item terminal object,
	\item subobject classifier,
	\item {\em (ideal) power objects}: To each subject $b \in \sE$ there exists an ()()() EEE % isn't this a problem bc the power object is attached to the type structure?
\Enu
then $\sE$ is an ideal elementary topos. Conversely, every ideal elementary topos satisfies these conditions. 
\Prop
\prf
We construct $\epsilon$ and $()^*$:
\stuffgoeshere
We show that $\sE$ has arbitrary colimits. % This is hard!!!
\stuffgoeshere
\Prf


has terminal
\stuffgoeshere
has binary product
\stuffgoeshere
has equalizer
\stuffgoeshere
has initial
\stuffgoeshere
has binary coproduct 
\stuffgoeshere
has coequalizer
\stuffgoeshere
To obtain the evaluation map on $\sE/u$, we use

\lem
To show that an ideal category with subobject classifier (???) is an ideal cartesian closed category, it suffices to show 
\enu
	\item power object
	\item membership map (like evaluation map)
\Enu
In this case, we say that $\sE$ has {\em powers}. 
\Lem
\prf
()()()
\Prf

$\sE/u$ has powers. 

Therefore $\sE/u$ is an ideal elementary topos. 
\Prf % // sliceT

\end{comment}












\begin{comment}
TASKS FOR FUTURE:
\enu
	\item direct image of a monic as an arrow on powers % MaMo1 �4
	\item monads, Beck's theorem, finite colimits % MaMo1 �5
	\item subobjects is a Heyting algebra, arrow gives functor % Mamo Prop 3 p. 186, 
	\item Sub 1 and Open sE % Mamo1 Prop 4 p. 189
	\item Beck-Chevalley condition % Mamo1 �9 
	\item injective objects % MaMo1 �10
%	\item Lawvere Tierney topologies
	\item Ch. V section 5,6,7,8,9 (coalgebras, internal, filter-quotient construction)
	\item setlike (Lawvere axioms)
\Enu
\end{comment}




































































































































%\input{it6(slicecon)}
% File it6(slicecons).tex Created by Lucius Schoenbaum January 1, 2017
% slice theorem and consequences




























%\section{Slice Categories and Presheaves}\label{s.slicecon}

A result %for its wide-reaching consequences, 
that more than any other single result gives a special character to toposes among categories says that the slice category of a topos is again a topos. 
The ``Fundamental Theorem of Topos Theory'' \cite{FrD1}, also called the Slice Theorem, is next to be proved in the generalized setting. 
%In this section ()()()
%which first appeared in \cite{sga4} as a result on Grothendieck toposes, is a significant result in basic topos theory,  %
%
We first introduce the slice category, and reintroduce subobjects, this time as a subcategory of the slice category. % This will give us the convenience of viewing subobjects as elements of a category. 

%We also show that a generalized presheaf is the colimit of representable presheaves (in the sense of the generalized Yoneda lemma of section \ref{s.sht}). This brings our development of generalized topos theory to a state in which it is possible to explore topics in topos theory at the generalized level. In other words, what we show is that in a certain light, topos theory is ``really'' a theory about generalized toposes: its truths are extended from the one-categorical level to the general level with relatively little difficulty. In fact, in the author's opinion, many observations of topos theory are made more percievable when viewed in the relative ``openness'' of generalized categories, but this is surely a matter of opinion. 

\dfn\label{d.subandslice}
Let $b$ be an element in an ideal elementary topos $\sE$. We define the {\em slice category} above or below $b$ to be the set of all triples 
$$(f,s,t)$$
that satisfy: $s: \bar s \to b$, $t: \bar t \to b$, $f: \bar s \to \bar t$, and $t \cdot f = s$. 
Composition and source/target are defined by:
$$(g,t,r) \cdot (f,s,t) := (g \cdot f,s,r),$$
$$\overline{(f,s,t)} := (1_{\bar s}, s, s),$$
$$\widehat{(f,s,t)} := (1_{\bar t}, t, t).$$
These operations define the structure of a one-category on $\sE/b$. 
\Dfn

Let $b$ be an element of a generalized category $\sE$. 
Consider the set of all monics with target $b$, along with a choice of map $f: m_1 \to m_2$ between each pair $m_1, m_2$ of monics with target $b$, whenever such a choice can be made (that is, whenever $\hom(m_1, m_2)$ is nonempty). In this way, the monics are taken as objects of a preorder-category and a subcategory of the slice category, denoted $\Sub_\sE (b)$ and called (herein) the {\em subobject category}. %
We must be somewhat careful to keep in mind that the (semi-)lattice structure of Corollary \ref{c.imagee} is only defined up to isomorphism in the subobject {\em category} defined in this section. As usual, this does not lead to any difficulty. 

The slice category is the category-theoretic analog of descending sets in preorders. In that case, there is nothing that distinguishes the subobject category from the slice category, though in general they are distinct (for example even in the case of the category of sets). 

\prop\label{p.subandslice}
%basic facts relating to sub and slice
Let $\sE$ be an ideal category or generalized category, and let $b \in \sE$.
\enu
	\item $\sE/b$ has terminal object $1_{\sE/b} := (1_b, 1_b, 1_b)$.
	\item If $\sE$ has finite limits, then $\sE/b$ has finite limits. 
	\item If $\sE$ has products and a subobject classifier, then $\sE/b$ has a subobject classifier. 
	\item $(m,s,t)$ is monic in $\sE/b$ if and only if $m$ is monic in $\sE$. \label{i.monicslice}
	\item The inclusion 
		$$i: \Sub(b) \hookrightarrow \sE/b$$
		of the subobject category in the slice category has left adjoint given by $\im(-)$. 
	\item There is an isomorphism of lattices
		$$\Sub_{\sE} (b) \bij \Sub_{\sE/b} (1_{\sE/b}).$$ \label{i.latticeiso}
\Enu
\Prop
\prf
(\ref{i.latticeiso}) follows from (\ref{i.monicslice}). %
The product arrows in $\sE/b$ are
$$\pi_{(1,s,s), (1,t,t)} := (\pullb_{s,t}, t \cdot \pullb_{s,t}, s)$$
$$\pi'_{(1,s,s), (1,t,t)} := (\pullb'_{s,t}, t \cdot \pullb'_{s,t}, s)$$
Also set
$$\Omega_{\sE/b} := (1_{\Omega \wedge b}, \pi'_{\Omega, b}, \pi'_{\Omega, b}),$$
\[
\arsoc_{\sE/b} := (\langle \delta_b \cdot \Delta_b, 1_b \rangle, 1_b, \pi'_{\Omega,b}). \qedhere
\]
\Prf

Note that the Grothendieck construction, or ``category of elements'' of a presheaf, is also a slice category. % over the section object of a pointed profunctor. 
More precisely, using the language of section \ref{s.sht}, let $(P, \S_P)$ be a pointed profunctor over a generalized category $\sC$. Then the Grothendieck construction $\el(P)$ is $P/\S_P$, that is, it is the one-category whose underlying set is the set of triples $(f,s,t)$ with $f \in \sC$, $s,t$ in $P$ with target $\S_P$ (i.e., sections), and satisfying $f: \bar s \to \bar t$, and $t \cdot f = s$. Composition in $\el(P)$ is given by 
$$(g,t,r) \cdot (f,s,t) = (g \cdot f, s, r).$$
and source and target $\overline{(f,s,t)} = (1_{\bar s}, s, s)$, $\widehat{(f,s,t)} = (1_{\bar t}, t, t)$. 
%
As with any slice, the map $(f, s, t) \mapsto f$ defines a functor $\pi = \pi_P$ from $\el(P)$ to $\sC$:
$$\pi(f,s,t) := f.$$
Indeed, 
$$\pi( \overline{(f,s,t)}) = \pi((1_{\bar s}, s, s)) = \overline{1_{\bar s}} = \bar s = \bar f = \overline{\pi((f,s,t))},$$
and similarly for target/composition. 

%\subsection{The Main Theorem}\label{ss.sliceT}

%The reader might find grounds to disagree, but the author finds it quite remarkable that 
%The slice theorem remains a theorem in the setting generalized categories. %It is striking, though 
%There is nothing deep involved, that 
The ``sub and slice'' structure, i.e., the structure of the subobject lattice/category and the slice category, take on their usual role familiar from the one-categorical case. However, it is even more fitting in the generalized situation than in the one-categorical situation to call the slice category a ``slice'': it is a one-dimensional structure within the generalized category (by ``one-dimensional'' we mean it is a one-category). This one-category has in it %, in the usual way, 
the subobject lattice (a true lattice, for a topos) represented as a system of monics. %This dramatically simplifies the topos-theoretic structure overlaying the basic generalized category theory relative to the one-categorical situation, but of course, we must verify. 
The fundamental result is: %the following:

\thm\label{t.sliceT}
Let $\sE$ be an ideal elementary topos, $b \in \sE$. Then the slice category 
$\sE/b$ 
is a cartesian closed category. 
\Thm
\prf
%In fact, under the definitions we have, we may proceed just as in the case of elementary topos. % 
%For the sake of completeness, we give a sketch of the proof. %
First observe that, unlike in the one-categorical case, the power object operator $P(-)$ is not a functor. Rather, it is given by two separate operators, which we denote by $P(-)$ and $\sP(-)$, defined for $a \in \sE$ by
$$P(a) := a \vdash \Omega,$$
$$\sP(a) := (\varepsilon_{\Omega, \hat a} \langle \pi_{\hat a \vdash \Omega, \bar a}, a \cdot \pi'_{\hat a \vdash \Omega, \bar a} \rangle )^*.$$
Then $\sP(a): a \to \Omega$, and $\sP(a) : P(\hat a) \to P(\bar a).$ Moreover $\sP(a)$ is the unique element in $\sE$ that satisfies (omitting the subscripts of projections)
$$\varepsilon_{\Omega, \bar a} \cdot \langle \sP(a) \cdot \pi, \pi' \rangle = \varepsilon_{\Omega, \hat a} \cdot \langle \pi, a \cdot \pi' \rangle.$$
or what is the same again
$$ \varepsilon_{\Omega, \bar a} \cdot (\sP(a) \wedge 1) = \varepsilon_{\Omega, \hat a} \cdot (a \wedge 1). $$
The rest of the proof is to verify that this change in the nature of the $P(-)$ operator does not invalidate the proof in the one-categorical case. 
%affect the situation as it arises in the one-categorical case. 

Now we wish to prove that there is a natural bijection between two hom sets, as in:
\begin{equation}\label{e.powerslice}
\hom_{\sE/b}(\fs \wedge \ft, \Omega_{\sE/b} ) \bij \hom_{\sE/b} (\ft, P \fs) 
\end{equation}
where (provisionally) we use the letters $\fs, \ft,$ etc. for objects $(1_{\bar a},a,a)$ of the slice, and where $P(-)$, the power object operator in the category $\sE/b$, is still to be defined. %
In order to do this, we convert hom sets to subobject lattices, as we may always do in toposes. 
For the rest of the proof of (\ref{e.powerslice}), let $\fs = (1_{\bar s}, s, s)$ and $\ft = (1_{\bar t}, t, t)$, for $s,t \in \sE$ with target $b$. %

First, $\fs \wedge \ft$ is the pullback $(1, t \pullb_{s,t}, t \pullb_{s,t})$ with source $\bar s \times_b \bar t$. %
By Proposition \ref{p.subandslice}, $\Omega_{\sE/b}$ is $(1_{\Omega \wedge b}, \pi', \pi')$. 
Therefore, by inspection, we have a bijection
$$\hom_{\sE/b} (\fs \wedge \ft, \Omega_{\sE/b}) \bij \hom_\sE ( \bar s \times_b \bar t , \Omega)$$
that is natural in $\fs, \ft$. 
Now in the topos $\sE$ we use the natural bijection 
$$\hom(\bar s \times_b \bar t, \Omega) \bij \Sub(\bar s \times_b \bar t).$$
The fiber product $\bar s \times_b \bar t$ is a subobject of $\bar s \wedge \bar t$, whence we obtain
$$\Sub(\bar s \times_b \bar t) \bij \set{[m] \mid [m] \text{ is a subobject of } \bar s \wedge \bar t \text{ and } [m] \lies [ s \cdot \pullb_{s,t}]}$$
Next we translate to predicates and obtain 
$$\hom_{\sE/b}(\fs \wedge \ft, \Omega_{\sE/b} ) \bij \set{h: \bar s \wedge \bar t \to \Omega \mid h \wedge_b \pred( \langle \pullb_{s,t} , \pullb_{s,t}' \rangle ) = h}$$
where here we write an infix operator $\wedge_b$ for the arrow $\hom(b, \Omega \wedge \Omega) \to \hom(b, \Omega)$ 
arising from the meet $\cap$ in $\Sub(b)$ via $\hom(b, \Omega) \iso \Sub(b)$, 
and where we use the standard expression of the fiber product as a subobject of the product:
$$\bar s \times_b \bar t \overset{\langle \pullb_{s,t} , \pullb_{s,t}' \rangle}{\armonic} \bar s \wedge \bar t.$$
Now we take star (transpose) of the $h$'s to obtain 
$$\hom_{\sE/b}(\fs \wedge \ft, \Omega_{\sE/b}) \bij \set{k: \bar t \to P(\bar s) \mid k \wedge_{int} \sP(s) \cdot \set{\cdot}_b \cdot t = k}$$
where $\wedge_{int}$ (``internal meet'') denotes the infix operator defined by the arrow $P(b) \wedge P(b) \to P(b)$ again arising from the meet $\cap$ in $\Sub(b)$, 
and where we have used the fact (result of inspection) that 
$$\pred(\langle \pullb_{s,t} , \pullb'_{s,t} \rangle) = \varepsilon_{\Omega, c} (1 \wedge (\sP(s) \cdot \set{\cdot}_b \cdot t)).$$
This now becomes (writing $\wedge_{int}$ for the internal meet arrow, not the infix operator) 
$$\hom_{\sE/b}(\fs \wedge \ft, \Omega_{\sE/b}) \bij \set{k: \bar t \to P(\bar s) \mid \wedge_{int} \cdot (1 \wedge (\sP(s) \cdot \set{\cdot}_b \cdot t)) \cdot \langle k,t \rangle = \pi_{k,t} \cdot \langle k,t \rangle}$$
Now let 
$$t_1 := \wedge_{int} \cdot (1 \wedge (\sP(s) \cdot \set{\cdot}_b \cdot t).$$
and define, for objects $(1,s,s)$ in $\sE/b$,
$$P_{\sE/b} ( (1,s,s)) := (1, \pi'_{P(s), b} \cdot \Eq(t_1, \pi_{k,t}), \pi'_{P(s), b} \cdot \Eq(t_1, \pi_{k,t}) ),$$
we have the desired bijection (\ref{e.powerslice}) using the universal property of the equalizer $\Eq(t_1, \pi_{k,t})$ of $t_1$ and the first projection $\pi_{k,t}$. 
%
%
%We first define an element $\phi$ to be
%$$\phi := \varepsilon_{\Omega, \bar s} \langle \pi, \sP(s) \cdot \set{\cdot}_b \cdot t \pi' \rangle.$$ 
%$w: \bar t \to P(\bar s)$. 
%\stuffgoeshere
%\stuffgoeshere
\Prf

We now make note of some of the most significant consequences of the slice theorem (for more see, for example, \cite{JoN1,MaMo1}). The proofs of these results now follow closely the one-categorical case. 

\cor\label{c.sliceT}
Let $\sE$ be an ideal elementary topos. 
\enu
	\item The slice $\sE/b$ over an element $b \in \sE$ is an elementary topos. \label{i.slicetopos}
	\item The subobject category $\Sub_a (\sE)$ below every element $a \in \sE$ is (up to isomorphism) a Heyting algebra. \label{i.subheyting}
	\item If $k: b \to a$, then the arrow $k^*: \sE/a \to \sE/b$ given by 
		$$k^*(f,s,t) := (\langle t \cdot \pullb_{k,t} , \pullb'_{k,t} \rangle_{s,k} , \pullb'_{k,s} , \pullb'_{k,t} )$$
		has a left adjoint $\Sigma_k$ and a right adjoint $\Pi_k$. 
	\label{i.pisigma}
\Enu
\Cor
%\prf
%(\ref{i.slicetopos}) 

%(\ref{i.subheyting}) 

%(\ref{i.pisigma}) 

%\stuffgoeshere
%\stuffgoeshere

%\Prf

There are many other consequences of Theorem \ref{t.sliceT}, see the references for others. %
In particular, the slice theorem is used in the proof of the theorem of Giraud \cite{sga4}, relating Grothendieck toposes to a list of conditions on an ordinary one-category. % 
%It may be possible to generalize this result based on what we have done so far, along with what we will do in the next section. 
%However
In order to extend this result it must be determined whether a generalized sheaf (to be defined below) is the colimit of representable ones. %a  fact we have not been able to verify or falsify in the generalized case. 



%\enu
%	\item 
		%Pullbacks preserve epis. % 
%	\item 
		%Coproducts preserve pullbacks. % goldblatt, freyd, ...
%	\item 
		%In a topos, every arrow whose target is the initial object $\nlld$ is an isomorphism, and every arrow whose source is $\nlld$ is monic.
%\Enu




































%\input{it5(sht)}
% File it4(sht).tex Created by Lucius Schoenbaum November 3, 2016
% 





\section{Sheaves over Generalized Categories}\label{s.sht}

%In this section we establish sheaf theory in the generalized setting. 
%\stuffgoeshere 
% it is interesting. There is actually interesting stuff in this section. 
%First, we establish a language that may be used in the setting of section \ref{s.it} (the setting of ideal elementary toposes): 
If we shunt away the task of providing motivation, it is remarkably easy, due to the work of Lawvere and Tierney, to introduce a notion of sheaf:

\dfn\label{d.latitop}
Let $\sE$ be an ideal elementary topos. A {\em Lawvere-Tierney topology,} or {\em topology} on $\sE$ is a choice of element $j \in \sE$ such that
\enu
	\item[(0)] $j : \Obsoc \to \Obsoc$,
	\item[(1)] $j \cdot \arsoc = \arsoc$,
	\item[(2)] $j \cdot j = j$,
	\item[(3)] $j \cdot \delta_{\arsoc} = \delta_{\arsoc} \cdot (j \wedge j).$
\Enu
A monic $m$ is {\em dense} with respect to $j$ if %
$$[m] = \subob (j\cdot \pred(m)),$$
where $[m]$ is the subobject defined by $m$ and $\subob(\phi)$ is the subobject defined by a predicate $\phi$. Dense subobjects are defined similarly. 
\Dfn

\dfn\label{d.sheaflatitop}
Let $(\sE,j)$ be an elementary ideal topos $\sE$ equipped with a Lawvere-Tierney topology $j$. %
%
A {\em sheaf} for $j$ on $\sE$ is an element $P$ of $\sE$ such that %
for every monic $m$ in $\sE$, and for every $f:\bar m \to P$ in $\sE$, if $m$ is dense, there exists a unique $g \in \sE$ such that %
$$g \cdot m = f.$$
\Dfn

%There is challenge is motivating these definitions. %
To motivate these definitions, however, we must relate them to a generalization of geometric sheaf theory. %
This is the task for the rest of this section. 
%Is there such a generalization? %
%We will see that the answer is yes. %
%We have now defined a sheaf-theoretic language in the setting of ideal elementary toposes. We will return to this language after developing generalized sheaves in the sense of \cite{sga4}. % in section \ref{s.sht}. 

\subsection{Pointed Profunctors}\label{ss.pprofunctor}
The profunctor abstraction \cite{BeU2} provides a convenient language for beginning a study of generalized sheaves. 

\dfn\label{d.pointedprofunctor}
A {\em pointed profunctor} is a structure $(P, \S)$ consisting of a generalized category $P$ equipped with a distinguished object $\S$ of $P$ satisfying: for all elements $f$ in $P$,
$$\dom(f) = \S \eimplies f = \id_\S.$$
\Dfn

The full generalized category $\sC$ on the objects $U \in P$, $U \neq \S$ is the {\em base category} of the pointed profunctor $P$. %
In this case $P$ is said to be defined {\em over $\sC$}. %
The element $\S$ is the {\em section object}. %
If $U$ is an element of $\sC$, the base category of $P$, the set of elements $\hom(U, \S)$ are called {\em sections over $U$} and the set $\hom(U, \S)$ is also called the {\em fiber over $U$}, denoted $\Fib_P (U)$ or $P(U)$. %

\dfn
A {\em morphism} of pointed profunctors $P$ and $Q$ is a functor $\phi: P \to Q$ satisfying: %
\enu
	\item $\phi(\S_P) = \S_Q,$
	\item for all elements $U \in \sC$, $\phi(U) = \S \eimplies U = \S.$
\Enu
If the induced map between base categories is the identity, then $F$ is said to be {\em base-preserving}. 
\Dfn

%There is a category of pointed profunctors; there is a category of pointed profunctors preserving the base category. 

%Let $\sE$ be the category of pointed profunctors over a fixed base category $\sC$.

%it has products

%it has coproducts

%it has finite limits

%it is cartesian closed

%it has subobject classifier

A pointed profunctor over a category is not a presheaf, because it may have fibers over morphisms in the base. %{\em morphisms} (in the usual terminology, or, from a different viewpoint, the terminology of section \ref{c.gencat}) as well as over {\em objects} (in the usual viewpoint, or in the terminology of section \ref{c.gencat}, {\em subjects}). 

\prop\label{p.presheafiso}
Let $\sC$ be a generalized category. 
The category of presheaves over $\sC$ is isomorphic to the category of pointed profunctors and base-preserving morphisms over $\sC$ with the property that
\begin{equation}\label{eq.presheafcondition}
\text{proper arrows of $\sC$ remain proper arrows of $P$.}
\end{equation}
\Prop
\prf
Fix a generalized category $\sC$ and let $(A,\S)$ be a pointed profunctor over $\sC$ with the property \ref{eq.presheafcondition}. %
Define %
$$F:\sC^{op} \to \text{Set}$$
by sending $a \in \sC$ to the map $s \mapsto sa$ from the set $\set{s \in A \mid \bar s = \hat a}$ to the set $\set{s \in A \mid \bar s = \bar a}$. Then we can check that $F(ab) = F(b) \of F(a),$ and $F(a)$ is an identity in the category of sets if and only if $a$ is an identity in $\sC$. %
So $F$ is a presheaf on $\sC$. %Note %
%$$\overline{F(a)} = 1_{F(1_{\bar a})} = F(1_{\bar a}),$$
%$$\widehat{F(a)} = 1_{F(1_{\hat a})} = F(1_{\hat a}).$$
Now let $F:\sC^{op} \to \text{Set}$ be a limit functor. For $U \in \Ob(\sC),$ let $\Sect(U)$ %
be the set which is the domain (and codomain) of $F(1_a)$, and let $\Sect(\sC) := \bigcup_{U \in \Ob(\sC)} \Sect(a)$.
Let 
$$P := \sC \sqcup \Sect(\sC) \sqcup \set{\S},$$
where $\sqcup$ is disjoint union, and extend the operations of $\sC$ to $P$: for $a \in \sC$, $s \in \Sect(\sC),$ 
\begin{align*}
	\S &\in \Ob(A),			\\
	\cod(s) &:= \S,			\\
	\dom(s) &:= a, \quad \call{s \in \Sect(a)}, \\
	s\cdot a &:= F(a)(s), \text{ if} \cod(s) = \dom(a), \text{ not def. otherwise.}
\end{align*}
%We claim that this is a presheaf. First, it is a generalized category (since only products of the form $sa$ with $s,a$ as above are nontrivial, we can restrict our attention to these):
Then $P$ is a category, and moreover a pointed profunctor over $\sC$. %
%Axiom 1: $(sa)b = F(ab)(s) = (F(b) \of F(a))(s) = F(b) \of (F(a)(s)) = F(b) \of (sa) = (sa)b.$

%Axiom 2,3: $\dom(sa) = d(a)$ and $\cod(sa) = \cod(s) = \S.$

%Axiom 4,5, and axioms for presheaf: The extra presheaf axioms are satisfied. Thus all subjects are in $\sC$, and hence have identities from $\sC$. The same applies to objects; the only new object is $\S$.

We thus obtain a presheaf, and this construction is clearly inverse to the preceding one. %
%Let $\Phi(P) := F$. 
Now we have to show that this bijection $\Phi$ extends to maps. Let $A,B$ be two presheaves, and let $\phi:A \to B$ be a morphism of presheaves. Then because
$$\phi(sa) = \phi(s)\phi(a),$$
if we let $\Phi(P) := F$, $\Phi(Q) := G$, we have $\phi(F(a)(s)) = G(\phi(a)) (\phi(s)).$ %
We are given that $\phi$ leaves $\sC$ fixed, so we have, for all $s \in \dom(F(a))$,
$$(\phi \of F(a))(s) = (G(a) \of \phi) (s).$$
For $U \in \sC$, let $\phi_{nat}(U) = \phi|_{\Sect(U)}$. Then for all $a \in \sC$, 
$$\phi_{nat}(\hat a) \of F(a) = G(a) \of \phi_{nat} (\bar a) \,\downarrow.$$
So we have a natural transformation $F \Rightarrow G$.

Conversely, suppose $\phi_{nat}$ is a natural map for a natural transformation $\phi:F \to G$, where $F,G$ are two presheaves over $\sC$. Then for all $a \in \sC$, %
$$F(a): \Sect_F (\bar a) \to \Sect_F (\hat a),$$
$$G(a): \Sect_G (\bar a) \to \Sect_G (\hat a).$$
So $\phi_{nat} (a)$ is a function 
$$ \phi_{nat}(a):\Sect_F (a) \to \Sect_G (a).$$
We can therefore define a function
$$\phi: \Sect_F(\sC) \to \Sect_G (\sC)$$
by mapping $s$ (over $a$) to $\phi_{nat}(a)(s)$. Extend this map to $A$ by setting $\phi(U) = U$ for $U \in \sC$, and $\phi(\S) = \S$. %
%From this definition, we see that the observations made previously can be reversed. %
We obtain a base-preserving morphism of pointed profunctors. %
This is $\Phi(\phi)$, and $\Phi$ is still invertible after extending to maps. %
Functoriality of $\Phi$ is easily verified. %
So we obtain the desired isomorphism. %
\Prf

This result can be interpretted as indication that the category of sets is not a natural player in the generalized setting. % 
This result can be extended to the two-categorical level as well, where the same phenomenon recurs. 
We omit the details, but make mention of the following, whose proof is a more involved rendition of the steps in the preceding one. The bicategorical notion coinciding with pointed profunctor is a {\em coweighted bicategory}: 

\thm\label{t.bicat}
Let $\sC$ be a category. Then the category of (normalized) fibered categories and optransformations is isomorphic to the category of coweighted bicategories over the categorical base $\sC$ and base-preserving opmorphisms. 
\Thm

\subsection{Generalized Sheaves}\label{ss.gensheaf}
We have obtained a category of pointed profunctors in the preceding section. We could apply the construction of \cite{ScM3} to obtain an ideal cartesian closed category. %
%
We could then pursue the investigation further and ask: %is it an ideal cartesian closed category? %
is it an ideal elementary topos? %
However, this line of inquiry is likely only to yield a dead letter, because there is no hope of establishing a Yoneda embedding in the generalized setting via this construction. %
%Plan: 
%- from now on we use the language of presheaves, but the notation of pointed profunctors.
%- now we define a generalized category structure on the presheaves. 
%- then we go through some hoops in order to get the yoneda embedding as a full-fledged embedding of the generalized category structure of the site in the presheaf generalized category. This involves establishing a presheaf that acts like a morphism of presheaves, hence a
%- but then we have the yoneda embedding, and we can proceed to prove that the presheaf generalized category is ccc, and so on. 
%- also prove (???) that every presheaf is a colimit of representables. 
%\subsection{Profunctors, or Pointed Profunctor Categories}
%Indeed, if we wish to extend the framework of sheaf theory to generalized categories, it is crucial to extend the Yoneda lemma. 
The Yoneda embedding is a full-fledged functorial embedding of the base generalized category provided that there is a way to assign domains and codomains to presheaves (pointed profunctors). To develop such machinery, we must modify our approach. We turn to the language of bipartite categories. %This motivates an approach like that of this section.
%This language is perhaps the most convenient, for example, it avoids making explicit reference to the category of sets, which has the effect of pulling us out of our universe $\sU$. 

% this is another "captain obvious" remark from back in the less mature days:
%The letter $P$ a pointed profunctor to denote the section object $\S_P$. In fact, from a certain standpoint it makes sense to view the section object as being {\em precisely} the pointed profunctor/presheaf. This standpoint is made formal with the following definition:

\dfn\label{d.bipartitegencat}
A {\em bipartite generalized category} is a generalized category $\sE$ equipped with a subcategory $\sC$, the {\em base} %or {\em site} 
of the generalized category, such that every element $\rho$ of $\sE$ such that $\cod(\rho) \in \sC$ satisfies $\rho \in \sC$. 
\Dfn

We say that $\sE$ is {\em over the base $\sC$} in this case. 
%
The elements of $\sE$ not in $\sC$ whose target is in $\sC$ is called a {\em section} of $\sC$. 
%
The set of elements of $\sE$ that are neither base elements nor sections is called, collectively, the {\em upper region} of $\sE$. 

%Intuitively, the generalized category is divided into two separate ``regions'', the base and the compliment of the base, connected by arrows that are always pointing out of the base. 
%The arrows \stuffgoeshere are the {\em sections} of $\sE$. The hom sets of sections are the {\em fibers} of $\sE$ over elements of the base.

Let $\sE$ be a bipartite generalized category. It is easy to see that fixing a choice of element $P \in \sE$ in the upper region produces, by a direct construction, a pointed profunctor $\tilde P$ over the base of $\sE$. Moreover, a choice of element $P \in \sE$ in the upper region produces a morphism of pointed profunctors $\bar P : \tilde P_1 \to \tilde P_2$ from its source to its target. %, let us denote it $\phi$ this time and let $E_1, E_2$ be the domain and codomain of $\phi$, also produces a pointed profunctor homomorphism. 

\dfn\label{d.maximalbipartitegencat}
A bipartite generalized category $\sE$ is {\em maximal} over its base if every construction of a pointed profunctor yields an element of $\sE$, and if every construction of a pointed profunctor homomorphism also produces an element of $\sE$. 
%If a generalized category $\sC$ is given, an element of the upper region of a maximal generalized category $\sE$ over the base $\sC$ is called a {\em generalized presheaf} over $\sC$. 
\Dfn

\prop\label{p.maxunique}
There exists a unique maximal bipartite generalized category over a given category $\sC$. 
\Prop
\prf
Suppose that there were two distinct maximal bipartite categories $\sE, \sE'$ over a base $\sC$. Then 
$$\sE'' := \sC \sqcup \Sect(\sE) \sqcup \Sect(\sE') \sqcup \UR(\sE) \sqcup \UR(\sE')$$
is a bipartite category over $\sC$ extending both, contradicting the maximality of $\sE$ and $\sE'$. 
\Prf

This proof is (classically) valid, assuming we work in a universe $\sU_{univ}$ closed under the usual set theoretical operations. %

%This definition  allows us to assume that a generalized category of presheaves exists without necessarily fully specifying individual presheaves as they are defined---in particular, without necessarily specifying the multiplication and domain/codomain structure of a presheaf. The domain and codomain values of elements, as objects, in any maximal bipartite generalized category are not canonical, similar to such values in the skeleton of a generalized category. 
% So for ``many purposes'', one might say, such information does not matter. %Still, it is a question whether a better definition might exist. 
%For representable presheaves (more precisely, for {\em represented} ones), we do have this information, from the Yoneda lemma:

\thm\label{l.yoneda}
Let $\sE$ be the maximal bipartite generalized category over a base generalized category $\sC$. Then there exists a functor $\yoneda:\sC \to \sE$ with the property that for every element $U \in \sC$ there is a bijection between the fiber of any element $P$ and the homomorphisms from $\yoneda (U)$ to $P$.
\Thm
\prf
The proof is by construction: we produce a copy of $\sC$ (an ``exact carbon copy'') along with all sections of $\sE$, and a third copy of each element $f: a \to b$ in $\sC$, changing the target of the third copy to the second copy of the target. By maximality, this system of pointed profunctors is already in $\sE$. 
\Prf

%Something that this neat and tidy proof masks is that 
The same construction, carried out by copying the base and moving the copy to the upper region, gives an extension of any bipartite generalized category. %
In some sense, this is the meaning of Yoneda's lemma. %
%
We can go further on the basis we have now laid out, but instead, we shall continue on the basis of a notion of presheaf in the generalized setting that is in agreement with the notion of ideal elementary topos used in section \ref{s.it}. %
The reason we must shift is if $P \vdash Q$ is an ideal element \cite{ScM3} of the upper region of a bipartite category, %
the composition of $P \vdash Q$ with a section element yields a section element that must itself be ideal. %
Therefore we define: % 

\dfn\label{d.imbgencat}
An {\em ideal maximal bipartite generalized category} $\sE$ is a bipartite category that satisfies:
\enu
	\item $\sE$ has the maximality property of Definition \ref{d.maximalbipartitegencat},
	\item For every $P$ in the upper region of $\sE$ and for every $U$ in the base of $\sE$, there is a section $s_U :U \to P$ that has the property that for all $f, g \in \sE$ that are not identities, if $s \cdot f$ $\downarrow$ then $s_U \cdot f = s_{\bar f}$, and if $g \cdot s$ $\downarrow$ then $g \cdot s = s_U$. 
\Enu
\Dfn

%
By the same argument as proved Proposition \ref{p.maxunique}, there is a unique ideal maximal bipartite generalized category over a base $\sC$. The $s_U$ of Definition \ref{d.imbgencat} is the ``ideal element'' in the fiber (cf. \cite{ScM3}). 
%We define: %

\dfn\label{d.gencatpresheaf}
Let $\sC$ be a generalized category. The ideal maximal bipartite generalized category $\sE$ over $\sC$ is called the {\em generalized category of generalized presheaves over $\sC$}. %
An element $P$ of the upper region of $\sE$ %, or the pointed profunctor $\tilde P$ defined by that element, 
is called a {\em generalized presheaf} over $\sC$. 
\Dfn

Thus a generalized presheaf is never considered ``in a vacuum'', but always in the context of a system of presheaves. 

%\dfn\label{d.representable}
%Let $\sC$ be a generalized category. A generalized presheaf $P$ over $\sC$ is {\em representable} if it is isomorphic in $\sE$ to an generalized presheaf of the form $\yoneda(U)$ for some $U \in \sC$. 
%\Dfn

%An interesting task for future work is the question whether, or whether not, every generalized presheaf over a base $\sC$ is a colimit of representable generalized presheaves in the generalized setting. %
%This result is used, for example, during the construction of the sheafification functor. %
%The proof in the ordinary one-categorical case relies on the following construction: %
%Let $P$ be a generalized presheaf. We define a generalized category, the {\em Grothendieck category} \cite{JoN1}, denoted $\el(P)$, by taking binary trees $\fs$ of sections $s$ of $P$. %
%We compose these trees as we do the elements of the comma category of Example \ref{ex.setf}. %
% example of the composition? 
%We add to this system elements $\tilde\rho : \fs \to \fs'$ corresponding to elements $\rho:U \to V$ of $\sC$, where 
%$$\troot s(\tilde\rho) = s(\rho),$$
%$$\troot t(\tilde\rho) = t(\rho),$$
%and define:
%$$\tilde \rho: \fs \to \fs',$$
%$$\tilde\rho_1 \cdot \tilde{\rho_2} := \widetilde{\rho_1 \cdot \rho_2}.$$
%We call this system $\el(P)$.
%This system $\el(P)$ is equipped with a functor $\pi_P$ projecting to $\sC$:
%$$\pi_P : \el(P) \to \sC.$$

%In greater generality, we have:
%\lem
%Let $\sE$ be a cocomplete generalized category, and let $F: \sC \to \sE$ be a functor. Then the functor 
%$$R_F: \sE \to \PSh(\sC)$$
%given by sending $P \in \sE$ to the presheaf whose sections over $U \in \sC$ is 
%$\Fib_{F(P)} (U) := \hom_\sE ( F(U), 
%The diagram for the colimit is $\el(P)$, defined to be
%$$\el(P) := \set{foo}$$
%\stuffgoeshere
%\stuffgoeshere
%Thus we have mappings $\theta$ from (generalized) presheaves to the set of functors between the categories  such that for all $\phi: P \to Q$, 
%$$\theta_Q
%$$\theta: P \iso \colim (\yoneda \of \pi_P).$$
%\Prf

Next, we have a result that gives a construction of an ideal cartesian closed category \cite{ScM3} from the data of a generalized category:

\thm\label{t.pshiccc}
The generalized presheaf category over a base $\sC$ % d.gencatpresheaf
is an ideal cartesian closed category. 
\Thm
\prf
We check the definition \cite{ScM3}. %
Let $\sE$ be the generalized presheaf category over $\sC$. %
It is clearly a generalized category, and we can for simplicity (not being concerned with any logical interpretation at present) assume that all elements of $\sE$ are valid. %
We have an element $\trut$ in $\sE$ given by the presheaf ${\bf pt}$ with itself as source and target and with a single element $pt$ in each fiber. %
We take the constants of $\sE$ to be the terminal arrows $\arter_P$ that send every section of a generalized presheaf $P$ to $pt$. %
(This element $pt$ must be the ideal section in each fiber.) %
The binary product operation $\times$, as noted above, is given by fiber-wise union and component-wise composition, recursively defined on sources and targets. %
% ideal category
The operation $\langle P,Q \rangle$ is defined when $\bar P = \bar Q$ by sending $s$ in $\bar P = \bar Q$ to $( P\cdot s, Q \cdot s )$. 

Now we can no longer put off the ideal element $P \vdash Q$ for given $P,Q$ in $\sE$. 
%
We must construct given $P$ and $Q$ an element 
$$P \vdash Q$$
sourced at $P$ and targeted at $Q$. Let $\S_{P \vdash Q}$ be a fresh symbol, and consider the set of arrows $s$,
$$s:P \times \yoneda(U) \to Q$$
For each such $s$ we define an element 
$$\tilde s : U \to \S_{P \vdash Q}.$$
This defines a pointed profunctor with section object $\S_{P \vdash Q}$. %
We make a generalized presheaf out of this pointed profunctor by setting %
$$s(P \vdash Q) = P,$$
$$t(P \vdash Q) = Q.$$
If $\phi: R \to P$, $\psi: Q \to S$ in $\sE$, we define 
$$(P \vdash Q) \cdot \phi := R \vdash Q,$$
$$\psi \cdot (P \vdash Q) := P \vdash S,$$
unless $\phi$ is a constant in $\sK$. In that case, $\phi$ is a terminal, hence $P$ is ${\bf pt}$. In this case, we may define $(P \vdash Q) \cdot \phi$ on sections to be be the constant map $s \mapsto s_U$, where $s_U$ is the ideal section, on each fiber. 
%$P \vdash Q$ cooperates well with the ideal section $\set{s_U}$
%\stuffgoeshere
We can also check that if $(P,Q)$ is a pair of elements in $\sE$, then we have the projections $\pi_{P,Q}, \pi'{P,Q}: P \times Q \to P, P \times Q \to Q$, respectively, and $(\pi_{P,Q}, \pi'_{P,Q})$ satisfy the axioms for a good pair for $(P,Q)$. 

% % %

The operation $()^*$ must now be defined. Let $\phi: R \times P \to Q$ in $\sE$. We define a pointed profunctor homomorphism $\phi^*$ from $R$ to $P \vdash Q$ (which we identify with $\hom(\yoneda(-) \times P , Q)$) now containing an ideal element $\yoneda(-))$ by defining $\phi^*(s)$, for $s:U \to R$, to be the the map sending a section $(f,x)$ of $\yoneda(U) \times P$, where $f$ is identified with an element of $\hom(-,U)$, to be
$$\phi \cdot \langle s \cdot f, t \rangle.$$
This defines a pointed profunctor homomorphism. %
This establishes that $\sE$ defines a positive intuitionistic generalized deductive system. 
Before checking good evaluations, we pause to observe that there is a unique $f: a \to \trut$, hence $f$ must be $a \vdash \trut$. 
In particular, for $a = \trut$, this shows that $\trut \vdash \trut = \trut$. 
This shows that $\sK$ is closed under 
$\sK$ closed under $\wedge, \langle,\rangle,$ and $()^*$, as desired. %

It remains only to show that there are good evaluations. 
%GOOD EVALUATIONS:
Let $P,Q$ be a pair of elements in $\sE$. Let $(\pi, \pi')$ be a good pair. Since good pairs are unique if they exist, these are the usual projections on $P \times Q$. %NOTE WE CAN'T ASSUME THEY ARE THE PROJECTIONS. RESUMING: 
%We wish to find an evaluation. 
For $U$ in $\sC$, for $\theta$ a section of $P \vdash Q$ over $U$, and $s$ a section of $P$ over $U$, define 
$$\epsilon_U (\theta, s) = \theta_U ( 1_U, s),$$
where we again identify a section $\theta$ of $P \vdash Q$ over $U$ with its associated element $\yoneda(U) \times P \to Q$. %
We pass to this morphism the identity $1_U$ on $U$ along with the given section $s$ of $P$. %
Finally, we check that Lambek's axioms
$$\epsilon \cdot \langle f^* \cdot \pi_{c,a}, \pi'_{c,a} \rangle = f,$$
$$(\epsilon \cdot \langle f \cdot \pi_{c,a}, \pi'_{c,a} \rangle)^* = f.$$
%ayd
%\stuffgoeshere
follow as a consequence of the definitions of $\epsilon$ and $()^*$. 
\Prf

%We can now complete the basic set of sheaf-theoretic tools inside of a maximal bipartite generalized category, by showing:

%The generalized presheaves are not only an ideal cartesian closed category, but they  form a topos. 
We will prove that the generalized presheaves form an ideal elementary topos after introducing sieves. % and subfunctors. 

%We will go now beyond ideal cartesian closed categories and establish the well-known fact in the one-categorical setting that the presheaves form an ideal elementary topos. 

%PLAN: DEFINE SIEVE DEFINE TOPOLOGY DEFINE SUBFUNCTOR ETC. DEFINE IDEAL GROTHENDIECK TOPOS PROVE IDEAL ELEMENTARY TOPOS

\dfn\label{d.sieve}
A {\em sieve} on a generalized category $\sC$ is a subset $\sS$ of $\sC$ such that there exists $U \in \sC$ such that for every $s \in \sS$, $\hat s = U$. We say that $\sS$ is a {\em sieve at $U$}. We write $\bar \sS = U$. 
\Dfn

The notation $\bar \sS$ is intended to evoke the definition $\sS: U \to \Omega$, where the target $\Omega$ will be defined below. Note the possible source of confusion: $\overline{\sS}$ is the common {\em target} of elements of $s$. 

\prop\label{p.sieve}
We have the following:
\enu
	\item The set $\set{f \in \sC \mid \hat f = U}$ is a sieve, called the {\em maximal sieve} at $U$. 
	\item If $\sS$ is a sieve at $U$ and $\rho$ is an element of $\sC$ with $\hat \rho = U$, the set
$$\stem{\sS}{\rho} := \set{\sigma \in \sC \mid \rho \cdot \sigma \downarrow \eand \rho \cdot \sigma \in \sS}$$ 
		is a sieve at $\bar \rho$. 
	\item If $\sS$ is sieve, $f \in \sC$, and $\hat f = \bar \sS$, then $f \in \sS$ if and only if $\stem{\sS}{f}$ is maximal. 
\Enu
\Prop

%Let $\sE$ be a maximal bipartite generalized category. %
%To begin the proof, we first make a definition. %
The order-theoretic notion corresponding to a sieve is a lower set, sometimes called a downset or descending set. %
%In this respect it would be better to call a maximal sieve a principle sieve, since it corresponds to a  ----> no, 
%
The symbol $\stem{\sS}{\rho}$, denoted $\rho^*(\sS)$ in \cite{MaMo1}, denotes a generalization of the notion of restriction of a cover: $\set{U_\alpha}|_U := \set{U_\alpha \cap U}$, where $\set{U_\alpha}$ denotes an open cover in a topological space. 
%
Sieves are closely related to the notion of a subpointed profunctor: %

\dfn\label{d.subpointedprofunctor}
A {\em subpointed profunctor} $P'$ of a pointed profunctor $P$ on a generalized category $\sC$ is a generalized category of $P$ with the same section object $\S$ and the same base generalized category $\sC$. 
\Dfn

The subpointed profunctor is closed under composition, hence $s \cdot \rho$ is in $P'$ for every $\rho$ in $\sC$ and $s$ in $\Sect (P')$. 
It can be explicitly observed that the subpointed profunctors of a given pointed profunctor $P$ have the structure of a Heyting algebra, in fact, a frame. Indeed, the meet and join of subpointed profunctors $P_1$ and $P_2$ are given by 
$$P_1 \vee P_2 = P_1 \cup P_2,$$
$$P_1 \wedge P_2 = P_1 \cap P_2,$$
where the symbols $\cup$ and $\cap$ are used to denote the corresponding sets, along with the obvious pointed profunctor structure. %
The maximum element of the Heyting algebra is $P$ itself, and the minimum is the subpointed profunctor $(\sC, \S)$ with no sections. The difference or residual of $P_1$ and $P_2$ is
$$P_1 \Leftarrow P_2 = \set{s \in P_1 \mid \text{for all } f \in P, s \cdot f \,\downarrow, s \cdot f \in P_2 \eimplies s \cdot f \in P_1}.$$
This structure entails a {\em pseudocomplement} to any subpointed profunctor:
$$\lnot P := \set{s \in P \mid s \nin P, \eand s\cdot f \nin P \text{ for all } f \text{ such that } s \cdot f \,\downarrow}.$$
The data of a sieve at $U \in \sC$ is also given by a subpointed profunctor of $\yoneda(U)$. %where $y$ is the yoneda functor.
%For example, a sieve is produced from any family or arrows $\set{\rho_i}_{i \in I}$ with common codomain by taking the set of all 
%In an elementary topos (or ideal elementary topos) 

\dfn\label{d.gtop}
A {\em topology on $\sC$} is a collection $J$ of subsets in $\sC$, satisfying:
\enu
	\item Every $\sS$ in $J$ is a sieve in $\sC$.
	\item For every $U \in \sC$, the maximal sieve at $U$ is in $J$. 
	\item If $\sS$ is in $J$ and $\rho \in \sC$ with $\hat \rho = \bar{\sS}$, then $\stem{\sS}{\rho}$ is in $J$. 
	\item If $\sS$ is in $J$ and $\sR$ is any sieve satisfying $\bar \sR = \bar \sS$, % sharing a common target with $\sS$, 
	and if for every $f \in \sS$, $\stem{f}{\sR}$ is in $J$, then $\sR$ is in $J$. 
\Enu
Elements of the subcollection of $J$ consisting of sieves at $U \in \sC$ are called the sieves that {\em cover $U$} or {\em covering sieves} for $U$.  
A generalized category $\sC$ equipped with a topology $J$, denoted $\sC$ or $(\sC, J)$, is a {\em site}. 
\Dfn

\prop\label{p.gtop}
Let $(\sC, J)$ be a site. 
\enu
	\item If $\sS, \sS'$ are covering sieves at a common target $U \in \sC$, then the intersection $\sS \cap \sS'$ of $\sS$ and $\sS'$ (regarded as subsets of $\sE$) is a covering sieve. 
	\item If $\sS$ is a covering sieve and if $\sR$ is a sieve such that $\sR \rise \sS$ (that is, $\sR$ is larger than $\sS$), then $\sR$ is a covering sieve. 
	\item If $\sS$ is a covering sieve and if for every $f \in \sS$ there is given a covering sieve $\sS_f$ at $\bar f$,
	then the ``sum''
	$$\underset{f \in \sS}{\fS} \sS_f := \set{f \cdot g \mid f \in \sS, g \in \sS_f}$$
	is a covering sieve. 
\Enu
\Prop

%EEE % establish ccc, establish soc, establish limits, colimits, establish generators.


%\subsection{Topologies}\label{s.gtoy}

%In this section we introduce sheaves over a generalized category in a straightforward manner. 
%Now that we have introduced our ideas related to presheaves and what we called pointed profunctors above, we will constrain ourselves to making reference to presheaves. %but as you might guess our preference is to think and use the notation for pointed profunctors. 
%\stuffgoeshere


%These axioms are illustrated in Figure \ref{f.topology}.
%$$pic$$
%\begin{float}
%\centering
%\pic[width = 400pt]{grothtop.jpg}
%\end{float}

%Note that a sieve $\sS$ might be a covering sieve for $U$, $\sS \in J(U)$, while not being a covering sieve for some other subject $U'$, $\sS \nin J(U')$.

%This terminology was also used also to refer to the lower part of a bipartite generalized category, but this will not cause any confusion. 

%The basic properties for a Grothendieck topology carry over to the generalized setting essentially unchanged. 
We can repeat the same constructions as usual, leading to well-known examples of topologies \cite{sga4,MaMo1}. % : the trivial topology, the atomic topology, the topology given by epimorphic families, ...
%\stuffgoeshere

\dfn\label{d.closedsieve}
Let $(\sC, J)$ be a site. A {\em closed sieve} $\sS$ on $\sC$ is a sieve $\sS$ with the property that for all elements $f \in \sC$,
$$\stem{\sS}{f} \in J \eimplies \stem{\sS}{f} \text{ is the maximal sieve}. $$
In particular, $\sS$ is maximal, thus to be closed is somehow to be super-maximal. 
The {\em closure} of a sieve $\sS$ is the sieve
$$\closure{\sS} := \set{f \in \sC \mid \hat f = \overline{\sS} \eand \stem{\sS}{f} \text{ is in } J}.$$
\Dfn

\prop\label{p.closedsieve} $\phantom{v}$
\enu
	\item The closure of a sieve $\sS$ is a closed sieve. The closure of a sieve is the smallest closed sieve containing it. %addition, the closure operation is stable with respect to the induced sieve operation: 
	\item If $\sS$ is closed, then for all composable $f$, $\stem{\sS}{f}$ is closed. 
	\item $\stem{\closure(\sS)}{f} = \closure(\stem{\sS}{f})$.
	\item A closed covering sieve is maximal, and conversely. 
\Enu
\Prop

Closed sieves need not be covers, and covering sieves need not be closed sieves. %
Authors usually write $\overline{\sS}$ for the closure, following the usual convention in topology for the same terminology (for a different notion). We prefer to write $\overline{\sS}$ for the element of $\sC$ that $\sS$ covers, since this emphasizes its role in the subobject classifier: $\sS : \overline{\sS} \to \Omega$, even though this unfortunately prevents us from using the commonplace notation for the closure (in the sense of sieves). In the ordinary localic case in which $\sC$ is the category of elements $U$ of a locale, and covers are epimorphic families, the closed sieves are precisely the principal lower sets $\downarrow U$. 
These are precisely those lower sets which are closed under joins of their elements. This is also a degenerate case in the sense that these closed sieves are (covering, hence) maximal. %(essentially  there is no ``parallel arrow'' structure).  %
%this is perhaps the source of the ``closed'', ``closure'' terminology. 
%The notation and terminology \cite{MaMo1}: 
%The notion of closure of a sieve 

\dfn\label{d.matchingfamily}
Let $\sC$ be a generalized category and let $J$ be a topology on $\sC$. Let $P$ be a generalized presheaf on $\sC$, and let $\sS$ be a covering sieve in $\sC$. %
A {\em matching family in $P$ with respect to $\sS$} is %
a mapping $x:\sS \to \Sect(P),$ where $\Sect(P)$ is the set of all sections of $P$, that satisfies:
\enu
	\item $x$ assigns a section of $P$ to each element $\rho$ in $\sS$,
	\item (compatibility) For all $\sigma$ in $\sC$, $x(\rho) \cdot \sigma \downarrow$ implies
		$$x(\rho) \cdot \sigma = x(\rho \cdot \sigma).$$ %
\Enu
If $x = \set{x_\rho}$ is a matching family, an {\em amalgamation} of $x$ is an element $s \in \overline{\sS}$ such that
% is \el(P) as I defined it above the right notion? 
such that 
$$ \text{For every $\rho \in \sS$, } x(\rho) = s \cdot \rho.$$ %
that is, $x$ is completely determined by $s$. %
An amalgamation need not exist, or be uniquely defined if it exists. %
A {\em generalized sheaf} on $\sC$ or on $(\sC, J)$ is a generalized presheaf $P$ such that for every element $U$ of $\sC$, and for every cover $\sS$ of $U$, every matching family for $\sS$ has a unique amalgamation. %

The {\em generalized category of generalized sheaves on a site $\sC$} %by considering the generalized category of generalized presheaves $\sE$ on $\sC$, and taking 
is the subcategory $\Sheaf(\sC,J)$ of the generalized category of generalized presheaves consisting of generalized presheaves $P$ that satisfy: %the sheaves, and morphisms to be the presheaf morphisms. 
$$P \text{ is a sheaf with respect to $J$}, \eand \source(P), \target(P) \text{ is in $\Sheaf(\sC,J)$.}$$
along with composition induced from the category of generalized presheaves. 
\Dfn

\thm\label{t.gensh}
The generalized category of generalized sheaves over a site $(\sC, J)$ is an ideal cartesian closed category, and has a subobject classifier. 
\Thm
\prf
%show cartesian closed works: in sheaf, sheaf, out sheaf % p. 136
Let $P, Q$ be sheaves; we claim that $P \vdash Q$ is also a sheaf, which is all we need in order to show that $\Sh(\sC,J)$ is an ideal cartesian closed category in light of Theorem \ref{t.pshiccc}. 
%\stuffgoeshere
But indeed, we can show that $P \vdash Q$ is separated, and that 
$P \vdash Q$ amalgamates, much as in the one-categorical setting, see for example \cite{MaMo1}. %
% soc
Let $\S_\Omega$ be the target of a pointed profunctor $\Omega$ over $\sC$, and define sections by setting
$$\Fib_\Omega (U) := \set{\sS \mid \sS \text{ is a closed sieve at $U$}}$$
and complete the categorical structure by setting
$$\sS \cdot \rho := \stem{\sS}{\rho}$$
for $\rho: U \to V$ and $\sS$ a closed sieve at $U$. %
We incorporate $\Omega$ into the system of presheaves by making $\Omega$ an object: 
$$\hat\Omega = \bar\Omega = \Omega.$$
The terminal element $\trut$ is the subpointed profunctor whose fibers consist only of maximal sieves. %
An arrow $\arsoc: \trut \to \Omega$ in $\Sh(\sC, J)$ is given by inclusion. %
%We claim that $\Obsoc$ is a subobject classifier for the generalized presheaves on $\sC$. 
%and that $\Obsoc$ is a sheaf. 
%SHOW THAT OMEGA IS A SOC FOR PSH
%\stuffgoeshere
%We claim that $\Omega$ is a subobject classifier for $\Sh(\sC)$. %
%\stuffgoeshere
%SHOW THAT OMEGA IS A SHEAF
%\stuffgoeshere
%Now we claim that $\Omega$ is a sheaf, from which it follows that $\Omega$ is a subobject classifier for $\Sh(\sC,J)$. 
%\stuffgoeshere
As in the one-categorical case, we can now show that $\Obsoc$ is a sheaf, and that $(\Obsoc, \arsoc)$ is the desired subobject classifier of $\Sh(\sC, J)$. 
\Prf

%We finish this after ()()()()
%\stuffgoeshere
%What we have, then, is the observation that it is possible to lace together subobject lattices in a strict higher category in exactly the same way as in the 1-categorical setting. From here, it is now appears possible to develop any topos-theoretic tool one wishes to have. 
%This shows:

\cor\label{t.genpsh}
Every generalized category of generalized presheaves is an ideal elementary topos. 
\Cor
\prf
We have finite limits and colimits in this case, exactly as in the one-categorical case. 
\Prf

%\dfn
%A topology is {\em subcanonical} if every representable presheaf is a sheaf. 
%\Dfn

%We complete this picture somewhat by defining:

%\dfn\label{d.igrtopos}
%An {\em ideal Grothendieck topos} is a generalized category that is 
%equivalent \stuffgoeshere % does this have meaning? write down what it means. what about the ideal stuff? if you don't get answers to these questions, just remove this definition and it will be fine. 
%to the category of generalized sheaves over a generalized category $\sC$. 
%\Dfn

%We relate Grothendieck topologies and those of Lawvere-Tierney. 
The following result extending \cite{LaTi1} says that in the setting of generalized categories, Lawvere-Tierney topologies on ideal elementary toposes extend the notion of Grothendieck topology to settings in which there is no site of definition. 

\thm\label{t.latiT}
Let $\sC$ be a generalized category, and let $\sE = \PSh(\sC)$ be the generalized category of generalized presheaves over $\sC$. 
%\enu
%\item 
For every Grothendieck topology $J$ on $\sC$, there is a Lawvere-Tierney topology $j$ on $\sE$ such that $P \in \sE$ is a generalized sheaf with respect to $J$ if and only if $P$ is a generalized sheaf with respect to $j$. Conversely, given a Lawvere-Tierney topology $j$ on a generalized category of generalized sheaves, there is a Grothendieck topology $J$ on $\sE$ such that the same property holds. 
%\item Conversely, every Lawvere-Tierney topology $j$ on $\sE$ determines a 
%\Enu
\Thm
\prf
%We prove this following \cite{MaMo1}. 
Essentially no changes are needed to the proof in the generalized setting. %
%
First, let $(\sC, J)$ be a site and let $\sE$ be as in the statement of \ref{t.latiT}. %
Define %
$$j = \set{g \mid \stem{\sS}{g} \in J}.$$
%SHOW j LATI TOP
We can check from the definitions that $j$ is a Lawvere-Tierney topology. 
%\stuffgoeshere
Let $j$ be a Lawvere-Tierney topology on an ideal elementary topos $\sE$, and assume that $\sE$ is $\PSh(\sC)$ for some base generalized category $\sC$. %
In this case, $\Omega$ is a particular pointed profunctor, and we can consider subpointed profunctors of it. 
Consider the subpointed profunctor of $\Omega$ given by 
$$J = \set{\text{sections } \sS \text{ of } \Omega \mid j(\sS) \text{ is the maximal sieve at $\overline{\sS}$} }.$$
We can check from the definitions that $J$ is a Grothendieck topology. 
%SHOW J GROTHENDIECK TOPOLOGY
%\stuffgoeshere
The procedures yielding $j$ from $J$ and vice versa are inverse to one another. Moreover, a presheaf $P$ is a sheaf with respect to $J$ if and only if it is a sheaf with respect to $j$. 
%$j \to J, J \to j$
%\stuffgoeshere
%As $j$ commutes with meets, it must be order-preserving. 
%SHEAF WRT J IFF SHEAF WRT j 
%\stuffgoeshere
%Consider the category $\Sh_j (\sE)$ of sheaves on $\sE$ with respect to $j$. 
%\stuffgoeshere
\Prf































%

%

%

% old shit::::


\begin{comment}


%

%

%

% ss.yo
%\section{Yoneda Embedding, and Related Facts}\label{s.yo}


% YoL
\prop
yoneda
\Prop


% Grothendieck construction, ...
\prop


colimit of representables
\Prop


()()()




%\chapter{Presheaves}\label{c.presheaf}



%\dfn Let $C, C'$ be generalized categories. A {\em presheaf functor} from $C$ to $C'$ is a function $F:C \to C'$ satisfying, for all $a,b \in C$,$$F(ab) = F(a)F(b),$$$$\bar{F(a)} = F(1_{\bar{a}}),$$$$\hat{F(a)} = F(1_{\hat{a}}).$$\Dfn

For limit functors, we can define natural transformations and morphisms of functors just as for ordinary functors. A preliminary result about limit functors (definition \ref{d.limitfunctor}) is an easy extension of what has been previously verified. 

\prop\label{p.functorcategory3}
Let $C$, $C'$ be two generalized categories. 
\enu
	\item Natural transformations between limit functors from $C$ to $C'$ form a well-defined functor category.
	\item The natural transformations between all limit functors, along with all functors, define a 2-category in which the natural transformations are the 2-cells and the objects are the generalized categories.
\Enu
\Prop

\newcommand{\one}{${}_1$\,}
\newcommand{\two}{${}_2$\,}

\dfn
Let $C$ be a generalized category. A {\em presheaf\one} on $C$ is a limit functor from $C^{op}$ into Set. %is there a m.o.f. notion for presheaf functors???????? is that the same as a homomorphism between presheaves???????
\Dfn

By proposition \ref{p.functorcategory3}, the presheaves\one over $C$ form a category if we take morphisms to be natural transformations.

\dfn
Let $A$ be a generalized category, and let $\S \in A$. Then $(A,\S)$ is said to be a {\em presheaf\two} if 
\enu
	\item $a \in \Ob(A)$,
	\item $\bar a = \S$ implies $a = \S$,
	\item $\hat a = \S$ implies $a \nin \Sb(A)$.
\Enu
The element $\S$ is called the {\em codomain of sections} for $A$. 
\Dfn

\dfn
Let $A,A'$ be presheaves\two with codomains of sections $\S_A, \S_{A'}$, respectively. A {\em morphism of presheaves\two} $F:A \to A'$ is a functor $F$ satisfying
$$ F(\S_A) = \S_{A'}.$$
\Dfn

Clearly, the presheaves\two now form a category. 

For a presheaf\two $A$, we can consider the set $C_A$,
$$C_A := \set{a \in A \mid \hat a \neq \S}.$$
This forms a full subcategory of $A$. If we take the set of all presheaves $A$ with $C_A = C$, for some fixed generalized category $C$, we can define maps between these to be morphisms of presheaves\two which restrict to an identity map on $C_A$, where $A$ is the morphism's domain. In this way, we obtain a category which we call {\em the category of presheaves\two over $C$}.

\prop\label{p.presheafiso}
The category of presheaves\one over $C$ and the category of presheaves\two over $C$ are isomorphic.
\Prop
\prf
Fix a generalized category $C$. Suppose you have a presheaf\two $(A,\S)$, with $C = C_A$ (we treat it as identical to $C$). Define 
$$F:C^{op} \to \text{Set}$$
by sending $a \in C$ to the map $s \mapsto sa$ from the set $\set{s \in A \mid \bar s = \hat a}$ to the set $\set{s \in A \mid \bar s = \bar a}$. Then we can check:
$$F(ab) = F(b) \of F(a),$$
$$F(a) \in identities(Set) \text{ if and only if } a \in identities(C).$$
So we have a limit functor, and hence, a presheaf. We pause to note (this is not needed for the proof)
$$\overline{F(a)} = 1_{F(1_{\bar a})} = F(1_{\bar a}),$$
$$\widehat{F(a)} = 1_{F(1_{\hat a})} = F(1_{\hat a}).$$
Now let $C$ be the same, and let $F:C^{op} \to \text{Set}$ be a limit functor. Let, for $a \in C,$ 
$$\Sect(a)$$
be the set which is the domain (and codomain) of $F(1_a)$. Set 
$$\Sect(C) := \bigcup_{a \in \Sb(C)} \Sect(a).$$
Let 
$$A := C \sqcup \Sect(C) \sqcup \set{\S}.$$
We extend the operations of $C$ to $A$ as follows: for $a \in C$, $s \in \Sect(C),$
\begin{align*}
	\S &\in \Ob(A),			\\
	\cod(s) &:= \S,			\\
	\dom(s) &:= a, \quad \call{s \in \Sect(a)}, \\
	sa &:= \begin{cases} \nll & \cod(s) \neq \dom(a), \\ F(a)(s), & \cod(s) = \dom(a). \end{cases}
\end{align*}
We claim that this is a presheaf\two. First, it is a generalized category (since only products of the form $sa$ with $s,a$ as above are nontrivial, we can restrict our attention to these):

Axiom 1: $(sa)b = F(ab)(s) = (F(b) \of F(a))(s) = F(b) \of (F(a)(s)) = F(b) \of (sa) = (sa)b.$

Axiom 2,3: $\dom(sa) = d(a)$ and $\cod(sa) = \cod(s) = \S.$

Axiom 4,5, and axioms for presheaf\two : The extra presheaf axioms are satisfied. Thus all subjects are in $C$, and hence have identities from $C$. The same applies to objects; the only new object is $\S$.

We thus obtain a presheaf, and this construction is clearly an inverse to the previous one. We define $\Phi(A) = F$ via this construction.

Now we have to show that this bijection extends to maps. Let $A,B$ be two presheaves\two, and let $\phi:A \to B$ be a presheaf\two morphism. Then because
$$\phi(sa) = \phi(s)\phi(a),$$
if we let $\Phi(A) = F$, $\Phi(B) = G$, we have
$$\phi(F(a)(s)) = G(\phi(a)) (\phi(s)).$$
If we now assume that $\phi$ leaves $C$ fixed, we have, for all $s \in \dom(F(a))$,
$$(\phi \of F(a))(s) = (G(a) \of \phi) (s).$$
For $x \in C$, let $\phi_{nat}(x) = \phi|_{\Sect(x)}$. Then for all $a \in C$,
$$\nll \neq \phi_{nat}(\hat a) \of F(a) = G(a) \of \phi_{nat} (\bar a).$$
So we have a natural map, defining a natural transformation $F \Rightarrow G$.

Conversely, suppose $\phi_{nat}$ is a natural map for a natural transformation $\phi:F \to G$, where $F,G$ are two presheaves\one over $C$. Then for all $a \in C$,
$$F(a): \Sect_F (\bar a) \to \Sect_F (\hat a),$$
$$G(a): \Sect_G (\bar a) \to \Sect_G (\hat a).$$
So $\phi_{nat} (a)$ is a function 
$$ \phi_{nat}(a):\Sect_F (a) \to \Sect_G (a).$$
We can thus define a function
$$\phi: \Sect_F(C) \to \Sect_G (C)$$
by mapping $s$ (over $a$) to $\phi_{nat}(a)(s)$. Extend this map to $A$ by setting $\phi(c) = c$ for $c \in C$, and $\phi(\S) = \S$. From this definition, we see that the observations made previously can be reversed: we obtain a presheaf\two morphism fixing $C$. This is $\Phi(\phi)$, and $\Phi$ is still bijective. 

Functoriality of $\Phi$ is easily verified. So we obtain the desired isomorphism.
\Prf

In light of Prop. \ref{p.presheafiso}, from now on, we drop the notations presheaf\one and presheaf\two , and refer to a presheaf $A$ interchangeably in the sense of both definitions.

%A heuristic illustration of a presheaf along with its codomain of sections is the following: $$\pic{presheaf1.jpg}$$

Let $C$ be a generalized category and let $a \in C$. Consider the following constructions. 

Construction 1. The intuition is that we wish to take an arbitrary element $a \in C$ and and make a copy $\S_a$ of $a$, and then make a copy of $\hom(b,a)$ for all $b$, and make those copies serve as sections for a presheaf with $\S = \S_a$. This can clearly be made formal and yields a presheaf over $C$. Denote this $A_a$.

Construction 2. Now, let $b = \bar a$, $c = \hat a$, and consider the presheaves $A_b$ and $A_c$. For the sake of discussion, write $\tilde s$ if we wish to speak of the section in $A_b$ or $A_c$ that is obtained by making a copy of $s \in C$, and otherwise, of course, just write $s \in C$. Then we can construct a morphism of presheaves over $C$ by sending a section $\tilde s \in A_b$ over $x \in C$ to $\widetilde{as} \in A_c$ over $x \in C$, and leaving all the nonsections fixed. Denote this map $H(a)$; it is clearly a morphism of presheaves. [later] %It is clearly a morphism of presheaves: $H(sa) = H(s)H(a) = H(s)a$. [fix later]

Let $\PSh(C)$ denote the category of presheaves over $C$.

\prop
$H:C \to \PSh(C)$ is a limit functor.
\Prop
\prf
The map $H$ sends $a$ to an identity map if and only if $a$ is an identity of $C$, and $H(ab) = H(a)H(b)$, so $H$ is a limit functor.
\Prf

\prop
For all $a,b \in \Sb(C)$,
$$\hom_{\PSh(C)} ( H(a),H(b) ) \underset{bij}{\to} \hom_C ( a,b ).$$
\Prop
\prf
We note that taking $a,b \nin \Sb(C)$ gives only nonsense.
Let $a,b \in \Sb(C)$. $\hom(H(a), H(b))$ is in bijection with $\hom(A_a,A_b)$, presheaf morphisms $\phi$ between presheaves $A_a$ and $A_b$. %If $a \nin \Sb(C)$, then $\hom(a,b) = \nll$. On the other hand, there are no sections in $A_a$, and 
%If $a \in \Sb(C)$, then 
Let $\phi \in \hom(A_a,A_b)$. Consider the image of $\widetilde{1_a}$ under $\phi$. This will be a section $\tilde s$ corresponding to an element $s \in \hom(a,b)$.   %Such a presheaf morphism acts on the elements $\tilde s$ which correspond bijectively with .
Conversely, taking arbitrary $s \in \hom(a,b)$ gives a map $\phi_s$, in fact this is just $H(s)$. We see that $\phi_{H(s)} = s$, so the two maps are bijections. [later]
\Prf

We can express this by saying that {\em $H$ is a fully faithful limit functor from the generalized category $C$ to the category $\PSh(C)$.}

%Now we put this construction into a more general framework.

%\dfn
%Let $C$ be a category. A {\em k-algorithm} for $C$ is an invertible map from some subset $S$ of objects $C$ into the set of morphisms, satisfying the following conditions.
%\enu
%	\item ()()() 
%\Enu
%\Dfn

%\exa
%The {\em interpretation k-algorithm} is the usual one applied to Set: we can obtain a function from a set by interpreting the function as a set, provided
%\Exa

%\exa
%The {\em Yoneda k-algorithm} is the one just described above for the category of presheaves. 
%\Exa

%We can now continue along the usual line of development for presheaves (see, for example, \cite{SGA4,MacMoer,KaSc1}).

%\chapter{Addendum} We see from these reflections that by the study of generalized categories one finds the importance of the map $$\Phi:generalized categories \to Categories$$ given by making the identities $1_a$ of the generalized category into the objects of a category. Considering the inversion of this map leads one to consider all maps $$\Psi:Categories \to generalized categories$$ given by sending the objects of a category $C$ to morphisms of the same category $C$. Thus one can characterize an arbitrary generalized category by giving (1) a category, and (2) a map like this one. This sort of map we call a {\em k-algorithm}. An example of such a k-algorithm is the trivial one which transforms Set into kSet (see example \ref{e.set}). We conjecture that the study of such k-algorithms for various categories could be an interesting line of development. 


%

%

%

\end{comment}

















% // s.sht





































































































































































































%\input{it1(i)}
%\input{it1(i)}
%\input{it1(i)}
\pagebreak
\singlespacing
%To insert additional chapters, copy the previous five lines, using chapterX as the argument of the 
%\input command for Chapter X, where X=6,7,8,...
\addtocontents{toc}{\vspace{12pt}}
\addcontentsline{toc}{chapter}{\hspace{-1.6em} {\bf \large References }}
%$\phantom{v}$\\ 
%\vspace{0.9em} \\
%\begin{thebibliography}{999}
%\vspace{0.9em}
%%merlin.mbs apsrmp4-1.bst 2010-07-25 4.21a (PWD, AO, DPC) hacked
%Control: key (0)
%Control: author (3) reversed first dotless
%Control: editor formatted (0) differently from author
%Control: production of article title (0) allowed
%Control: page (1) range
%Control: year (0) verbatim
%Control: production of eprint (0) enabled
\begin{thebibliography}{312}%
\makeatletter
\providecommand \@ifxundefined [1]{%
 \@ifx{#1\undefined}
}%
\providecommand \@ifnum [1]{%
 \ifnum #1\expandafter \@firstoftwo
 \else \expandafter \@secondoftwo
 \fi
}%
\providecommand \@ifx [1]{%
 \ifx #1\expandafter \@firstoftwo
 \else \expandafter \@secondoftwo
 \fi
}%
\providecommand \natexlab [1]{#1}%
\providecommand \enquote  [1]{``#1''}%
\providecommand \bibnamefont  [1]{#1}%
\providecommand \bibfnamefont [1]{#1}%
\providecommand \citenamefont [1]{#1}%
\providecommand \href@noop [0]{\@secondoftwo}%
\providecommand \href [0]{\begingroup \@sanitize@url \@href}%
\providecommand \@href[1]{\@@startlink{#1}\@@href}%
\providecommand \@@href[1]{\endgroup#1\@@endlink}%
\providecommand \@sanitize@url [0]{\catcode `\\12\catcode `\$12\catcode
  `\&12\catcode `\#12\catcode `\^12\catcode `\_12\catcode `\%12\relax}%
\providecommand \@@startlink[1]{}%
\providecommand \@@endlink[0]{}%
\providecommand \url  [0]{\begingroup\@sanitize@url \@url }%
\providecommand \@url [1]{\endgroup\@href {#1}{\urlprefix }}%
\providecommand \urlprefix  [0]{URL }%
\providecommand \Eprint [0]{\href }%
\providecommand \doibase [0]{http://dx.doi.org/}%
\providecommand \selectlanguage [0]{\@gobble}%
\providecommand \bibinfo  [0]{\@secondoftwo}%
\providecommand \bibfield  [0]{\@secondoftwo}%
\providecommand \translation [1]{[#1]}%
\providecommand \BibitemOpen [0]{}%
\providecommand \bibitemStop [0]{}%
\providecommand \bibitemNoStop [0]{.\EOS\space}%
\providecommand \EOS [0]{\spacefactor3000\relax}%
\providecommand \BibitemShut  [1]{\csname bibitem#1\endcsname}%
\let\auto@bib@innerbib\@empty
%</preamble>
\bibitem [{\citenamefont {Aaronson}\ \emph {et~al.}(2019)\citenamefont
  {Aaronson}, \citenamefont {Cojocaru}, \citenamefont {Gheorghiu},\ and\
  \citenamefont {Kashefi}}]{ACGK19}%
  \BibitemOpen
  \bibfield  {author} {\bibinfo {author} {\bibnamefont {Aaronson},
  \bibfnamefont {Scott}}, \bibinfo {author} {\bibfnamefont {Alexandru}\
  \bibnamefont {Cojocaru}}, \bibinfo {author} {\bibfnamefont {Alexandru}\
  \bibnamefont {Gheorghiu}}, \ and\ \bibinfo {author} {\bibfnamefont {Elham}\
  \bibnamefont {Kashefi}}} (\bibinfo {year} {2019}),\ \bibfield  {title}
  {\enquote {\bibinfo {title} {Complexity-theoretic limitations on blind
  delegated quantum computation},}\ }in\ \href {\doibase
  10.4230/LIPIcs.ICALP.2019.6} {\emph {\bibinfo {booktitle} {46th International
  Colloquium on Automata, Languages, and Programming (ICALP 2019)}}},\ \bibinfo
  {series} {LIPIcs}, Vol.\ \bibinfo {volume} {132},\ \bibinfo {editor} {edited
  by\ \bibinfo {editor} {\bibfnamefont {Christel}\ \bibnamefont {Baier}},
  \bibinfo {editor} {\bibfnamefont {Ioannis}\ \bibnamefont {Chatzigiannakis}},
  \bibinfo {editor} {\bibfnamefont {Paola}\ \bibnamefont {Flocchini}}, \ and\
  \bibinfo {editor} {\bibfnamefont {Stefano}\ \bibnamefont {Leonardi}}}\
  (\bibinfo  {publisher} {Schloss Dagstuhl})\ pp.\ \bibinfo {pages}
  {6:1--6:13},\ \Eprint {http://arxiv.org/abs/arXiv:1704.08482}
  {arXiv:1704.08482} \BibitemShut {NoStop}%
\bibitem [{\citenamefont {Ac\'{\i}n}\ \emph {et~al.}(2007)\citenamefont
  {Ac\'{\i}n}, \citenamefont {Brunner}, \citenamefont {Gisin}, \citenamefont
  {Massar}, \citenamefont {Pironio},\ and\ \citenamefont {Scarani}}]{ABGMPS07}%
  \BibitemOpen
  \bibfield  {author} {\bibinfo {author} {\bibnamefont {Ac\'{\i}n},
  \bibfnamefont {Antonio}}, \bibinfo {author} {\bibfnamefont {Nicolas}\
  \bibnamefont {Brunner}}, \bibinfo {author} {\bibfnamefont {Nicolas}\
  \bibnamefont {Gisin}}, \bibinfo {author} {\bibfnamefont {Serge}\ \bibnamefont
  {Massar}}, \bibinfo {author} {\bibfnamefont {Stefano}\ \bibnamefont
  {Pironio}}, \ and\ \bibinfo {author} {\bibfnamefont {Valerio}\ \bibnamefont
  {Scarani}}} (\bibinfo {year} {2007}),\ \bibfield  {title} {\enquote {\bibinfo
  {title} {Device-independent security of quantum cryptography against
  collective attacks},}\ }\href {\doibase 10.1103/PhysRevLett.98.230501}
  {\bibfield  {journal} {\bibinfo  {journal} {Phys. Rev. Lett.}\ }\textbf
  {\bibinfo {volume} {98}},\ \bibinfo {pages} {230501}}\BibitemShut {NoStop}%
\bibitem [{\citenamefont {Ac\'{\i}n}\ \emph {et~al.}(2012)\citenamefont
  {Ac\'{\i}n}, \citenamefont {Massar},\ and\ \citenamefont {Pironio}}]{AMP12}%
  \BibitemOpen
  \bibfield  {author} {\bibinfo {author} {\bibnamefont {Ac\'{\i}n},
  \bibfnamefont {Antonio}}, \bibinfo {author} {\bibfnamefont {Serge}\
  \bibnamefont {Massar}}, \ and\ \bibinfo {author} {\bibfnamefont {Stefano}\
  \bibnamefont {Pironio}}} (\bibinfo {year} {2012}),\ \bibfield  {title}
  {\enquote {\bibinfo {title} {Randomness versus nonlocality and
  entanglement},}\ }\href {\doibase 10.1103/PhysRevLett.108.100402} {\bibfield
  {journal} {\bibinfo  {journal} {Phys. Rev. Lett.}\ }\textbf {\bibinfo
  {volume} {108}},\ \bibinfo {pages} {100402}},\ \Eprint
  {http://arxiv.org/abs/arXiv:1107.2754} {arXiv:1107.2754} \BibitemShut
  {NoStop}%
\bibitem [{\citenamefont {Aggarwal}\ \emph {et~al.}(2019)\citenamefont
  {Aggarwal}, \citenamefont {Chung}, \citenamefont {Lin},\ and\ \citenamefont
  {Vidick}}]{ACLV19}%
  \BibitemOpen
  \bibfield  {author} {\bibinfo {author} {\bibnamefont {Aggarwal},
  \bibfnamefont {Divesh}}, \bibinfo {author} {\bibfnamefont {Kai-Min}\
  \bibnamefont {Chung}}, \bibinfo {author} {\bibfnamefont {Han-Hsuan}\
  \bibnamefont {Lin}}, \ and\ \bibinfo {author} {\bibfnamefont {Thomas}\
  \bibnamefont {Vidick}}} (\bibinfo {year} {2019}),\ \bibfield  {title}
  {\enquote {\bibinfo {title} {A quantum-proof non-malleable extractor},}\ }in\
  \href {\doibase 10.1007/978-3-030-17656-3_16} {\emph {\bibinfo {booktitle}
  {Advances in Cryptology -- EUROCRYPT 2019}}},\ \bibinfo {editor} {edited by\
  \bibinfo {editor} {\bibfnamefont {Yuval}\ \bibnamefont {Ishai}}\ and\
  \bibinfo {editor} {\bibfnamefont {Vincent}\ \bibnamefont {Rijmen}}}\
  (\bibinfo  {publisher} {Springer})\ pp.\ \bibinfo {pages} {442--469},\
  \Eprint {http://arxiv.org/abs/arXiv:1710.00557} {arXiv:1710.00557}
  \BibitemShut {NoStop}%
\bibitem [{\citenamefont {Aharonov}\ \emph {et~al.}(2010)\citenamefont
  {Aharonov}, \citenamefont {{Ben-Or}},\ and\ \citenamefont {Eban}}]{ABE10}%
  \BibitemOpen
  \bibfield  {author} {\bibinfo {author} {\bibnamefont {Aharonov},
  \bibfnamefont {Dorit}}, \bibinfo {author} {\bibfnamefont {Michael}\
  \bibnamefont {{Ben-Or}}}, \ and\ \bibinfo {author} {\bibfnamefont {Elad}\
  \bibnamefont {Eban}}} (\bibinfo {year} {2010}),\ \bibfield  {title} {\enquote
  {\bibinfo {title} {Interactive proofs for quantum computations},}\ }in\
  \href@noop {} {\emph {\bibinfo {booktitle} {Proceedings of Innovations in
  Computer Science, ICS 2010}}}\ (\bibinfo  {publisher} {Tsinghua University
  Press})\ pp.\ \bibinfo {pages} {453--469},\ \Eprint
  {http://arxiv.org/abs/arXiv:0810.5375} {arXiv:0810.5375} \BibitemShut
  {NoStop}%
\bibitem [{\citenamefont {Ahlswede}\ and\ \citenamefont
  {Csisz\'ar}(1993)}]{AC93}%
  \BibitemOpen
  \bibfield  {author} {\bibinfo {author} {\bibnamefont {Ahlswede},
  \bibfnamefont {Rudolph}}, \ and\ \bibinfo {author} {\bibfnamefont {Imre}\
  \bibnamefont {Csisz\'ar}}} (\bibinfo {year} {1993}),\ \bibfield  {title}
  {\enquote {\bibinfo {title} {Common randomness in information theory and
  cryptography---{Part I}: Secret sharing},}\ }\href {\doibase
  10.1109/18.243431} {\bibfield  {journal} {\bibinfo  {journal} {IEEE Trans.
  Inf. Theory}\ }\textbf {\bibinfo {volume} {39}}~(\bibinfo {number} {4}),\
  \bibinfo {pages} {1121--1132}}\BibitemShut {NoStop}%
\bibitem [{\citenamefont {Alagic}\ \emph {et~al.}(2016)\citenamefont {Alagic},
  \citenamefont {Broadbent}, \citenamefont {Fefferman}, \citenamefont
  {Gagliardoni}, \citenamefont {Schaffner},\ and\ \citenamefont
  {St.~Jules}}]{ABFGSSJ16}%
  \BibitemOpen
  \bibfield  {author} {\bibinfo {author} {\bibnamefont {Alagic}, \bibfnamefont
  {Gorjan}}, \bibinfo {author} {\bibfnamefont {Anne}\ \bibnamefont
  {Broadbent}}, \bibinfo {author} {\bibfnamefont {Bill}\ \bibnamefont
  {Fefferman}}, \bibinfo {author} {\bibfnamefont {Tommaso}\ \bibnamefont
  {Gagliardoni}}, \bibinfo {author} {\bibfnamefont {Christian}\ \bibnamefont
  {Schaffner}}, \ and\ \bibinfo {author} {\bibfnamefont {Michael}\ \bibnamefont
  {St.~Jules}}} (\bibinfo {year} {2016}),\ \bibfield  {title} {\enquote
  {\bibinfo {title} {Computational security of quantum encryption},}\ }in\
  \href {\doibase 10.1007/978-3-319-49175-2_3} {\emph {\bibinfo {booktitle}
  {Proceedings of the 9th International Conference on Information Theoretic
  Security, ICITS 2016}}},\ \bibinfo {editor} {edited by\ \bibinfo {editor}
  {\bibfnamefont {Anderson~C.A.}\ \bibnamefont {Nascimento}}\ and\ \bibinfo
  {editor} {\bibfnamefont {Paulo}\ \bibnamefont {Barreto}}}\ (\bibinfo
  {publisher} {Springer})\ pp.\ \bibinfo {pages} {47--71},\ \Eprint
  {http://arxiv.org/abs/arXiv:1602.01441} {arXiv:1602.01441} \BibitemShut
  {NoStop}%
\bibitem [{\citenamefont {Alagic}\ \emph {et~al.}(2018)\citenamefont {Alagic},
  \citenamefont {Gagliardoni},\ and\ \citenamefont {Majenz}}]{AGM18}%
  \BibitemOpen
  \bibfield  {author} {\bibinfo {author} {\bibnamefont {Alagic}, \bibfnamefont
  {Gorjan}}, \bibinfo {author} {\bibfnamefont {Tommaso}\ \bibnamefont
  {Gagliardoni}}, \ and\ \bibinfo {author} {\bibfnamefont {Christian}\
  \bibnamefont {Majenz}}} (\bibinfo {year} {2018}),\ \bibfield  {title}
  {\enquote {\bibinfo {title} {Unforgeable quantum encryption},}\ }in\ \href
  {\doibase 10.1007/978-3-319-78372-7_16} {\emph {\bibinfo {booktitle}
  {Advances in Cryptology -- {EUROCRYPT} 2018, Proceedings, Part {III}}}},\
  \bibinfo {series} {LNCS}, Vol.\ \bibinfo {volume} {10822},\ \bibinfo {editor}
  {edited by\ \bibinfo {editor} {\bibfnamefont {Jesper~B.}\ \bibnamefont
  {Nielsen}}\ and\ \bibinfo {editor} {\bibfnamefont {Vincent}\ \bibnamefont
  {Rijmen}}}\ (\bibinfo  {publisher} {Springer})\ pp.\ \bibinfo {pages}
  {489--519},\ \Eprint {http://arxiv.org/abs/arXiv:1709.06539}
  {arXiv:1709.06539} \BibitemShut {NoStop}%
\bibitem [{\citenamefont {Alagic}\ and\ \citenamefont {Majenz}(2017)}]{AM17}%
  \BibitemOpen
  \bibfield  {author} {\bibinfo {author} {\bibnamefont {Alagic}, \bibfnamefont
  {Gorjan}}, \ and\ \bibinfo {author} {\bibfnamefont {Christian}\ \bibnamefont
  {Majenz}}} (\bibinfo {year} {2017}),\ \bibfield  {title} {\enquote {\bibinfo
  {title} {Quantum non-malleability and authentication},}\ }in\ \href {\doibase
  10.1007/978-3-319-63715-0_11} {\emph {\bibinfo {booktitle} {Advances in
  Cryptology -- CRYPTO 2017, Proceedings, Part II}}},\ \bibinfo {series}
  {LNCS}, Vol.\ \bibinfo {volume} {10402},\ \bibinfo {editor} {edited by\
  \bibinfo {editor} {\bibfnamefont {Jonathan}\ \bibnamefont {Katz}}\ and\
  \bibinfo {editor} {\bibfnamefont {Hovav}\ \bibnamefont {Shacham}}}\ (\bibinfo
   {publisher} {Springer})\ pp.\ \bibinfo {pages} {310--341},\ \Eprint
  {http://arxiv.org/abs/arXiv:1610.04214} {arXiv:1610.04214} \BibitemShut
  {NoStop}%
\bibitem [{\citenamefont {Alicki}\ and\ \citenamefont {Fannes}(2004)}]{AF04}%
  \BibitemOpen
  \bibfield  {author} {\bibinfo {author} {\bibnamefont {Alicki}, \bibfnamefont
  {Robert}}, \ and\ \bibinfo {author} {\bibfnamefont {Mark}\ \bibnamefont
  {Fannes}}} (\bibinfo {year} {2004}),\ \bibfield  {title} {\enquote {\bibinfo
  {title} {Continuity of quantum conditional information},}\ }\href {\doibase
  10.1088/0305-4470/37/5/L01} {\bibfield  {journal} {\bibinfo  {journal} {J.
  Phys. A}\ }\textbf {\bibinfo {volume} {37}},\ \bibinfo {pages}
  {L55--L57}}\BibitemShut {NoStop}%
\bibitem [{\citenamefont {Alon}\ \emph {et~al.}(2020)\citenamefont {Alon},
  \citenamefont {Chung}, \citenamefont {Chung}, \citenamefont {Huang},
  \citenamefont {Lee},\ and\ \citenamefont {Shen}}]{ACCHLS21}%
  \BibitemOpen
  \bibfield  {author} {\bibinfo {author} {\bibnamefont {Alon}, \bibfnamefont
  {Bar}}, \bibinfo {author} {\bibfnamefont {Hao}\ \bibnamefont {Chung}},
  \bibinfo {author} {\bibfnamefont {Kai-Min}\ \bibnamefont {Chung}}, \bibinfo
  {author} {\bibfnamefont {Mi-Ying}\ \bibnamefont {Huang}}, \bibinfo {author}
  {\bibfnamefont {Yi}~\bibnamefont {Lee}}, \ and\ \bibinfo {author}
  {\bibfnamefont {Yu-Ching}\ \bibnamefont {Shen}}} (\bibinfo {year} {2020}),\
  \href@noop {} {\enquote {\bibinfo {title} {Round efficient secure multiparty
  quantum computation with identifiable abort},}\ }\bibinfo {howpublished} {to
  appear at CRYPTO 2021},\ \bibinfo {note} {e-Print
  \href{http://eprint.iacr.org/2020/1464}{IACR 2020/1464}}\BibitemShut
  {NoStop}%
\bibitem [{\citenamefont {Ambainis}\ \emph {et~al.}(2009)\citenamefont
  {Ambainis}, \citenamefont {Bouda},\ and\ \citenamefont {Winter}}]{ABW09}%
  \BibitemOpen
  \bibfield  {author} {\bibinfo {author} {\bibnamefont {Ambainis},
  \bibfnamefont {Andris}}, \bibinfo {author} {\bibfnamefont {Jan}\ \bibnamefont
  {Bouda}}, \ and\ \bibinfo {author} {\bibfnamefont {Andreas}\ \bibnamefont
  {Winter}}} (\bibinfo {year} {2009}),\ \bibfield  {title} {\enquote {\bibinfo
  {title} {Non-malleable encryption of quantum information},}\ }\href {\doibase
  10.1063/1.3094756} {\bibfield  {journal} {\bibinfo  {journal} {J. Math.
  Phys.}\ }\textbf {\bibinfo {volume} {50}}~(\bibinfo {number} {4}),\ \bibinfo
  {pages} {042106}},\ \Eprint {http://arxiv.org/abs/arXiv:0808.0353}
  {arXiv:0808.0353} \BibitemShut {NoStop}%
\bibitem [{\citenamefont {Ambainis}\ \emph {et~al.}(2000)\citenamefont
  {Ambainis}, \citenamefont {Mosca}, \citenamefont {Tapp},\ and\ \citenamefont
  {de~Wolf}}]{AMTW00}%
  \BibitemOpen
  \bibfield  {author} {\bibinfo {author} {\bibnamefont {Ambainis},
  \bibfnamefont {Andris}}, \bibinfo {author} {\bibfnamefont {Michele}\
  \bibnamefont {Mosca}}, \bibinfo {author} {\bibfnamefont {Alain}\ \bibnamefont
  {Tapp}}, \ and\ \bibinfo {author} {\bibfnamefont {Ronald}\ \bibnamefont
  {de~Wolf}}} (\bibinfo {year} {2000}),\ \bibfield  {title} {\enquote {\bibinfo
  {title} {Private quantum channels},}\ }in\ \href@noop {} {\emph {\bibinfo
  {booktitle} {Proceedings of the 41st Symposium on Foundations of Computer
  Science, FOCS~'00}}}\ (\bibinfo  {publisher} {IEEE})\ p.\ \bibinfo {pages}
  {547},\ \Eprint {http://arxiv.org/abs/arXiv:quant-ph/0003101}
  {arXiv:quant-ph/0003101} \BibitemShut {NoStop}%
\bibitem [{\citenamefont {Ambainis}\ and\ \citenamefont {Smith}(2004)}]{AS04}%
  \BibitemOpen
  \bibfield  {author} {\bibinfo {author} {\bibnamefont {Ambainis},
  \bibfnamefont {Andris}}, \ and\ \bibinfo {author} {\bibfnamefont {Adam}\
  \bibnamefont {Smith}}} (\bibinfo {year} {2004}),\ \bibfield  {title}
  {\enquote {\bibinfo {title} {Small pseudo-random families of matrices:
  Derandomizing approximate quantum encryption},}\ }in\ \href {\doibase
  10.1007/978-3-540-27821-4_23} {\emph {\bibinfo {booktitle} {Proceedings of
  the 8th International Workshop on Randomization and Computation, RANDOM
  2004}}}\ (\bibinfo  {publisher} {Springer})\ pp.\ \bibinfo {pages}
  {249--260},\ \Eprint {http://arxiv.org/abs/arXiv:quant-ph/0404075}
  {arXiv:quant-ph/0404075} \BibitemShut {NoStop}%
\bibitem [{\citenamefont {Arnon-Friedman}(2018)}]{ArnonThesis}%
  \BibitemOpen
  \bibfield  {author} {\bibinfo {author} {\bibnamefont {Arnon-Friedman},
  \bibfnamefont {Rotem}}} (\bibinfo {year} {2018}),\ \emph {\bibinfo {title}
  {Reductions to IID in Device-independent Quantum Information Processing}},\
  \href@noop {} {Ph.D. thesis}\ (\bibinfo  {school} {Swiss Federal Institute of
  Technology (ETH) Zurich}),\ \Eprint {http://arxiv.org/abs/arXiv:1812.10922}
  {arXiv:1812.10922} \BibitemShut {NoStop}%
\bibitem [{\citenamefont {Arnon-Friedman}\ \emph {et~al.}(2018)\citenamefont
  {Arnon-Friedman}, \citenamefont {Dupuis}, \citenamefont {Fawzi},
  \citenamefont {Renner},\ and\ \citenamefont {Vidick}}]{ADFRV18}%
  \BibitemOpen
  \bibfield  {author} {\bibinfo {author} {\bibnamefont {Arnon-Friedman},
  \bibfnamefont {Rotem}}, \bibinfo {author} {\bibfnamefont {Fr{\'e}d{\'e}ric}\
  \bibnamefont {Dupuis}}, \bibinfo {author} {\bibfnamefont {Omar}\ \bibnamefont
  {Fawzi}}, \bibinfo {author} {\bibfnamefont {Renato}\ \bibnamefont {Renner}},
  \ and\ \bibinfo {author} {\bibfnamefont {Thomas}\ \bibnamefont {Vidick}}}
  (\bibinfo {year} {2018}),\ \bibfield  {title} {\enquote {\bibinfo {title}
  {Practical device-independent quantum cryptography via entropy
  accumulation},}\ }\href@noop {} {\bibfield  {journal} {\bibinfo  {journal}
  {Nat. Commun.}\ }\textbf {\bibinfo {volume} {9}}~(\bibinfo {number} {1}),\
  \bibinfo {pages} {1--11}}\BibitemShut {NoStop}%
\bibitem [{\citenamefont {Arnon-Friedman}\ \emph {et~al.}(2016)\citenamefont
  {Arnon-Friedman}, \citenamefont {Portmann},\ and\ \citenamefont
  {Scholz}}]{AFPS16}%
  \BibitemOpen
  \bibfield  {author} {\bibinfo {author} {\bibnamefont {Arnon-Friedman},
  \bibfnamefont {Rotem}}, \bibinfo {author} {\bibfnamefont {Christopher}\
  \bibnamefont {Portmann}}, \ and\ \bibinfo {author} {\bibfnamefont
  {Volkher~B.}\ \bibnamefont {Scholz}}} (\bibinfo {year} {2016}),\ \bibfield
  {title} {\enquote {\bibinfo {title} {Quantum-proof multi-source randomness
  extractors in the {Markov} model},}\ }in\ \href {\doibase
  10.4230/LIPIcs.TQC.2016.2} {\emph {\bibinfo {booktitle} {11th Conference on
  the Theory of Quantum Computation, Communication and Cryptography (TQC
  2016)}}},\ \bibinfo {series} {LIPIcs}, Vol.~\bibinfo {volume} {61}\ (\bibinfo
   {publisher} {Schloss Dagstuhl})\ pp.\ \bibinfo {pages} {2:1--2:34},\ \Eprint
  {http://arxiv.org/abs/arXiv:1510.06743} {arXiv:1510.06743} \BibitemShut
  {NoStop}%
\bibitem [{\citenamefont {Arnon-Friedman}\ \emph {et~al.}(2019)\citenamefont
  {Arnon-Friedman}, \citenamefont {Renner},\ and\ \citenamefont
  {Vidick}}]{AFRV19}%
  \BibitemOpen
  \bibfield  {author} {\bibinfo {author} {\bibnamefont {Arnon-Friedman},
  \bibfnamefont {Rotem}}, \bibinfo {author} {\bibfnamefont {Renato}\
  \bibnamefont {Renner}}, \ and\ \bibinfo {author} {\bibfnamefont {Thomas}\
  \bibnamefont {Vidick}}} (\bibinfo {year} {2019}),\ \bibfield  {title}
  {\enquote {\bibinfo {title} {Simple and tight device-independent security
  proofs},}\ }\href {\doibase 10.1137/18M1174726} {\bibfield  {journal}
  {\bibinfo  {journal} {SIAM J. Comput.}\ }\textbf {\bibinfo {volume}
  {48}}~(\bibinfo {number} {1}),\ \bibinfo {pages} {181--225}},\ \Eprint
  {http://arxiv.org/abs/arXiv:1607.01797} {arXiv:1607.01797} \BibitemShut
  {NoStop}%
\bibitem [{\citenamefont {Aspect}\ \emph {et~al.}(1982)\citenamefont {Aspect},
  \citenamefont {Dalibard},\ and\ \citenamefont {Roger}}]{Aspect82}%
  \BibitemOpen
  \bibfield  {author} {\bibinfo {author} {\bibnamefont {Aspect}, \bibfnamefont
  {Alain}}, \bibinfo {author} {\bibfnamefont {Jean}\ \bibnamefont {Dalibard}},
  \ and\ \bibinfo {author} {\bibfnamefont {G\'erard}\ \bibnamefont {Roger}}}
  (\bibinfo {year} {1982}),\ \bibfield  {title} {\enquote {\bibinfo {title}
  {Experimental test of {Bell}'s inequalities using time-varying analyzers},}\
  }\href {\doibase 10.1103/PhysRevLett.49.1804} {\bibfield  {journal} {\bibinfo
   {journal} {Phys. Rev. Lett.}\ }\textbf {\bibinfo {volume} {49}},\ \bibinfo
  {pages} {1804--1807}}\BibitemShut {NoStop}%
\bibitem [{\citenamefont {Aspect}\ \emph {et~al.}(1981)\citenamefont {Aspect},
  \citenamefont {Grangier},\ and\ \citenamefont {Roger}}]{Aspect81}%
  \BibitemOpen
  \bibfield  {author} {\bibinfo {author} {\bibnamefont {Aspect}, \bibfnamefont
  {Alain}}, \bibinfo {author} {\bibfnamefont {Philippe}\ \bibnamefont
  {Grangier}}, \ and\ \bibinfo {author} {\bibfnamefont {G\'erard}\ \bibnamefont
  {Roger}}} (\bibinfo {year} {1981}),\ \bibfield  {title} {\enquote {\bibinfo
  {title} {Experimental tests of realistic local theories via {Bell}'s
  theorem},}\ }\href {\doibase 10.1103/PhysRevLett.47.460} {\bibfield
  {journal} {\bibinfo  {journal} {Phys. Rev. Lett.}\ }\textbf {\bibinfo
  {volume} {47}},\ \bibinfo {pages} {460--463}}\BibitemShut {NoStop}%
\bibitem [{\citenamefont {Backes}\ \emph {et~al.}(2004)\citenamefont {Backes},
  \citenamefont {Pfitzmann},\ and\ \citenamefont {Waidner}}]{BPW04}%
  \BibitemOpen
  \bibfield  {author} {\bibinfo {author} {\bibnamefont {Backes}, \bibfnamefont
  {Michael}}, \bibinfo {author} {\bibfnamefont {Birgit}\ \bibnamefont
  {Pfitzmann}}, \ and\ \bibinfo {author} {\bibfnamefont {Michael}\ \bibnamefont
  {Waidner}}} (\bibinfo {year} {2004}),\ \bibfield  {title} {\enquote {\bibinfo
  {title} {A general composition theorem for secure reactive systems},}\ }in\
  \href {\doibase 10.1007/978-3-540-24638-1_19} {\emph {\bibinfo {booktitle}
  {Theory of Cryptography, Proceedings of TCC 2004}}},\ \bibinfo {series}
  {LNCS}, Vol.\ \bibinfo {volume} {2951}\ (\bibinfo  {publisher} {Springer})\
  pp.\ \bibinfo {pages} {336--354}\BibitemShut {NoStop}%
\bibitem [{\citenamefont {Backes}\ \emph {et~al.}(2007)\citenamefont {Backes},
  \citenamefont {Pfitzmann},\ and\ \citenamefont {Waidner}}]{BPW07}%
  \BibitemOpen
  \bibfield  {author} {\bibinfo {author} {\bibnamefont {Backes}, \bibfnamefont
  {Michael}}, \bibinfo {author} {\bibfnamefont {Birgit}\ \bibnamefont
  {Pfitzmann}}, \ and\ \bibinfo {author} {\bibfnamefont {Michael}\ \bibnamefont
  {Waidner}}} (\bibinfo {year} {2007}),\ \bibfield  {title} {\enquote {\bibinfo
  {title} {The reactive simulatability ({RSIM}) framework for asynchronous
  systems},}\ }\href {\doibase 10.1016/j.ic.2007.05.002} {\bibfield  {journal}
  {\bibinfo  {journal} {Inform. and Comput.}\ }\textbf {\bibinfo {volume}
  {205}}~(\bibinfo {number} {12}),\ \bibinfo {pages} {1685--1720}},\ \bibinfo
  {note} {extended version of~\textcite{PW01}, e-Print
  \href{http://eprint.iacr.org/2004/082}{IACR 2004/082}}\BibitemShut {NoStop}%
\bibitem [{\citenamefont {Badertscher}\ \emph {et~al.}(2020)\citenamefont
  {Badertscher}, \citenamefont {Cojocaru}, \citenamefont {Colisson},
  \citenamefont {Kashefi}, \citenamefont {Leichtle}, \citenamefont {Mantri},\
  and\ \citenamefont {Wallden}}]{BCCKLMW20}%
  \BibitemOpen
  \bibfield  {author} {\bibinfo {author} {\bibnamefont {Badertscher},
  \bibfnamefont {Christian}}, \bibinfo {author} {\bibfnamefont {Alexandru}\
  \bibnamefont {Cojocaru}}, \bibinfo {author} {\bibfnamefont {L{\'e}o}\
  \bibnamefont {Colisson}}, \bibinfo {author} {\bibfnamefont {Elham}\
  \bibnamefont {Kashefi}}, \bibinfo {author} {\bibfnamefont {Dominik}\
  \bibnamefont {Leichtle}}, \bibinfo {author} {\bibfnamefont {Atul}\
  \bibnamefont {Mantri}}, \ and\ \bibinfo {author} {\bibfnamefont {Petros}\
  \bibnamefont {Wallden}}} (\bibinfo {year} {2020}),\ \bibfield  {title}
  {\enquote {\bibinfo {title} {Security limitations of classical-client
  delegated quantum computing},}\ }in\ \href {\doibase
  10.1007/978-3-030-64834-3_23} {\emph {\bibinfo {booktitle} {Advances in
  Cryptology -- ASIACRYPT 2020, Proceedings, Part {II}}}},\ \bibinfo {series}
  {LNCS}, Vol.\ \bibinfo {volume} {12492},\ \bibinfo {editor} {edited by\
  \bibinfo {editor} {\bibfnamefont {Shiho}\ \bibnamefont {Moriai}}\ and\
  \bibinfo {editor} {\bibfnamefont {Huaxiong}\ \bibnamefont {Wang}}}\ (\bibinfo
   {publisher} {Springer},\ \bibinfo {address} {Cham})\ pp.\ \bibinfo {pages}
  {667--696},\ \Eprint {http://arxiv.org/abs/arXiv:2007.01668}
  {arXiv:2007.01668} \BibitemShut {NoStop}%
\bibitem [{\citenamefont {Banfi}\ \emph {et~al.}(2019)\citenamefont {Banfi},
  \citenamefont {Maurer}, \citenamefont {Portmann},\ and\ \citenamefont
  {Zhu}}]{BMPZ19}%
  \BibitemOpen
  \bibfield  {author} {\bibinfo {author} {\bibnamefont {Banfi}, \bibfnamefont
  {Fabio}}, \bibinfo {author} {\bibfnamefont {Ueli}\ \bibnamefont {Maurer}},
  \bibinfo {author} {\bibfnamefont {Christopher}\ \bibnamefont {Portmann}}, \
  and\ \bibinfo {author} {\bibfnamefont {Jiamin}\ \bibnamefont {Zhu}}}
  (\bibinfo {year} {2019}),\ \bibfield  {title} {\enquote {\bibinfo {title}
  {Composable and finite computational security of quantum message
  transmission},}\ }in\ \href {\doibase 10.1007/978-3-030-36030-6_12} {\emph
  {\bibinfo {booktitle} {Theory of Cryptography, Proceedings of {TCC} 2019,
  Part {I}}}},\ \bibinfo {series} {LNCS}, Vol.\ \bibinfo {volume} {11891}\
  (\bibinfo  {publisher} {Springer})\ pp.\ \bibinfo {pages} {282--311},\
  \Eprint {http://arxiv.org/abs/arXiv:1908.03436} {arXiv:1908.03436}
  \BibitemShut {NoStop}%
\bibitem [{\citenamefont {Barnum}\ \emph {et~al.}(2002)\citenamefont {Barnum},
  \citenamefont {Cr{\'e}peau}, \citenamefont {Gottesman}, \citenamefont
  {Smith},\ and\ \citenamefont {Tapp}}]{BCGST02}%
  \BibitemOpen
  \bibfield  {author} {\bibinfo {author} {\bibnamefont {Barnum}, \bibfnamefont
  {Howard}}, \bibinfo {author} {\bibfnamefont {Claude}\ \bibnamefont
  {Cr{\'e}peau}}, \bibinfo {author} {\bibfnamefont {Daniel}\ \bibnamefont
  {Gottesman}}, \bibinfo {author} {\bibfnamefont {Adam}\ \bibnamefont {Smith}},
  \ and\ \bibinfo {author} {\bibfnamefont {Alain}\ \bibnamefont {Tapp}}}
  (\bibinfo {year} {2002}),\ \bibfield  {title} {\enquote {\bibinfo {title}
  {Authentication of quantum messages},}\ }in\ \href {\doibase
  10.1109/SFCS.2002.1181969} {\emph {\bibinfo {booktitle} {Proceedings of the
  43rd Symposium on Foundations of Computer Science, FOCS~'02}}}\ (\bibinfo
  {publisher} {IEEE})\ pp.\ \bibinfo {pages} {449--458},\ \Eprint
  {http://arxiv.org/abs/arXiv:quant-ph/0205128} {arXiv:quant-ph/0205128}
  \BibitemShut {NoStop}%
\bibitem [{\citenamefont {Barrett}\ \emph {et~al.}(2013)\citenamefont
  {Barrett}, \citenamefont {Colbeck},\ and\ \citenamefont {Kent}}]{BCK13}%
  \BibitemOpen
  \bibfield  {author} {\bibinfo {author} {\bibnamefont {Barrett}, \bibfnamefont
  {Jonathan}}, \bibinfo {author} {\bibfnamefont {Roger}\ \bibnamefont
  {Colbeck}}, \ and\ \bibinfo {author} {\bibfnamefont {Adrian}\ \bibnamefont
  {Kent}}} (\bibinfo {year} {2013}),\ \bibfield  {title} {\enquote {\bibinfo
  {title} {Memory attacks on device-independent quantum cryptography},}\ }\href
  {\doibase 10.1103/PhysRevLett.110.010503} {\bibfield  {journal} {\bibinfo
  {journal} {Phys. Rev. Lett.}\ }\textbf {\bibinfo {volume} {110}},\ \bibinfo
  {pages} {010503}},\ \Eprint {http://arxiv.org/abs/arXiv:1201.4407}
  {arXiv:1201.4407} \BibitemShut {NoStop}%
\bibitem [{\citenamefont {Barrett}\ \emph {et~al.}(2005)\citenamefont
  {Barrett}, \citenamefont {Hardy},\ and\ \citenamefont {Kent}}]{BHK05}%
  \BibitemOpen
  \bibfield  {author} {\bibinfo {author} {\bibnamefont {Barrett}, \bibfnamefont
  {Jonathan}}, \bibinfo {author} {\bibfnamefont {Lucien}\ \bibnamefont
  {Hardy}}, \ and\ \bibinfo {author} {\bibfnamefont {Adrian}\ \bibnamefont
  {Kent}}} (\bibinfo {year} {2005}),\ \bibfield  {title} {\enquote {\bibinfo
  {title} {No signaling and quantum key distribution},}\ }\href {\doibase
  10.1103/PhysRevLett.95.010503} {\bibfield  {journal} {\bibinfo  {journal}
  {Phys. Rev. Lett.}\ }\textbf {\bibinfo {volume} {95}}~(\bibinfo {number}
  {1}),\ \bibinfo {pages} {1--4}}\BibitemShut {NoStop}%
\bibitem [{\citenamefont {Baumgratz}\ \emph {et~al.}(2014)\citenamefont
  {Baumgratz}, \citenamefont {Cramer},\ and\ \citenamefont {Plenio}}]{BCP14}%
  \BibitemOpen
  \bibfield  {author} {\bibinfo {author} {\bibnamefont {Baumgratz},
  \bibfnamefont {Tillmann}}, \bibinfo {author} {\bibfnamefont {Marcus}\
  \bibnamefont {Cramer}}, \ and\ \bibinfo {author} {\bibfnamefont {Martin~B.}\
  \bibnamefont {Plenio}}} (\bibinfo {year} {2014}),\ \bibfield  {title}
  {\enquote {\bibinfo {title} {Quantifying coherence},}\ }\href {\doibase
  10.1103/PhysRevLett.113.140401} {\bibfield  {journal} {\bibinfo  {journal}
  {Phys. Rev. Lett.}\ }\textbf {\bibinfo {volume} {113}},\ \bibinfo {pages}
  {140401}},\ \Eprint {http://arxiv.org/abs/arxiv:1311.0275} {arxiv:1311.0275}
  \BibitemShut {NoStop}%
\bibitem [{\citenamefont {Beaver}(1992)}]{Bea92}%
  \BibitemOpen
  \bibfield  {author} {\bibinfo {author} {\bibnamefont {Beaver}, \bibfnamefont
  {Donald}}} (\bibinfo {year} {1992}),\ \bibfield  {title} {\enquote {\bibinfo
  {title} {Foundations of secure interactive computing},}\ }in\ \href {\doibase
  10.1007/3-540-46766-1_31} {\emph {\bibinfo {booktitle} {Advances in
  Cryptology -- CRYPTO~'91}}},\ \bibinfo {series} {LNCS}, Vol.\ \bibinfo
  {volume} {576}\ (\bibinfo  {publisher} {Springer})\ pp.\ \bibinfo {pages}
  {377--391}\BibitemShut {NoStop}%
\bibitem [{\citenamefont {Bell}(1964)}]{Bell64}%
  \BibitemOpen
  \bibfield  {author} {\bibinfo {author} {\bibnamefont {Bell}, \bibfnamefont
  {John~Stewart}}} (\bibinfo {year} {1964}),\ \bibfield  {title} {\enquote
  {\bibinfo {title} {On the {E}instein-{P}odolsky-{R}osen paradox},}\
  }\href@noop {} {\bibfield  {journal} {\bibinfo  {journal} {Physics}\ }\textbf
  {\bibinfo {volume} {1}}~(\bibinfo {number} {3}),\ \bibinfo {pages}
  {195--200}}\BibitemShut {NoStop}%
\bibitem [{\citenamefont {Bell}(1966)}]{Bell66}%
  \BibitemOpen
  \bibfield  {author} {\bibinfo {author} {\bibnamefont {Bell}, \bibfnamefont
  {John~Stewart}}} (\bibinfo {year} {1966}),\ \bibfield  {title} {\enquote
  {\bibinfo {title} {On the problem of hidden variables in quantum
  mechanics},}\ }\href {\doibase 10.1103/RevModPhys.38.447} {\bibfield
  {journal} {\bibinfo  {journal} {Rev. Mod. Phys.}\ }\textbf {\bibinfo {volume}
  {38}},\ \bibinfo {pages} {447--452}}\BibitemShut {NoStop}%
\bibitem [{\citenamefont {Bell}\ and\ \citenamefont {Aspect}(2004)}]{BellFree}%
  \BibitemOpen
  \bibfield  {author} {\bibinfo {author} {\bibnamefont {Bell}, \bibfnamefont
  {John~Stewart}}, \ and\ \bibinfo {author} {\bibfnamefont {Alain}\
  \bibnamefont {Aspect}}} (\bibinfo {year} {2004}),\ \enquote {\bibinfo {title}
  {Free variables and local causality},}\ in\ \href {\doibase
  10.1017/CBO9780511815676.014} {\emph {\bibinfo {booktitle} {Speakable and
  Unspeakable in Quantum Mechanics: Collected Papers on Quantum Philosophy}}},\
  Chap.~\bibinfo {chapter} {12},\ \bibinfo {edition} {2nd}\ ed.\ (\bibinfo
  {publisher} {Cambridge University Press})\ pp.\ \bibinfo {pages}
  {100--104}\BibitemShut {NoStop}%
\bibitem [{\citenamefont {Bellare}\ \emph {et~al.}(1997)\citenamefont
  {Bellare}, \citenamefont {Desai}, \citenamefont {Jokipii},\ and\
  \citenamefont {Rogaway}}]{BDJR97}%
  \BibitemOpen
  \bibfield  {author} {\bibinfo {author} {\bibnamefont {Bellare}, \bibfnamefont
  {Mihir}}, \bibinfo {author} {\bibfnamefont {Anand}\ \bibnamefont {Desai}},
  \bibinfo {author} {\bibfnamefont {Eron}\ \bibnamefont {Jokipii}}, \ and\
  \bibinfo {author} {\bibfnamefont {Phillip}\ \bibnamefont {Rogaway}}}
  (\bibinfo {year} {1997}),\ \bibfield  {title} {\enquote {\bibinfo {title} {A
  concrete security treatment of symmetric encryption},}\ }in\ \href {\doibase
  10.1109/SFCS.1997.646128} {\emph {\bibinfo {booktitle} {Proceedings of the
  38th Annual Symposium on Foundations of Computer Science, FOCS~'97}}}\
  (\bibinfo  {publisher} {IEEE})\ pp.\ \bibinfo {pages} {394--403}\BibitemShut
  {NoStop}%
\bibitem [{\citenamefont {Bellare}\ \emph {et~al.}(1998)\citenamefont
  {Bellare}, \citenamefont {Desai}, \citenamefont {Pointcheval},\ and\
  \citenamefont {Rogaway}}]{BDPR98}%
  \BibitemOpen
  \bibfield  {author} {\bibinfo {author} {\bibnamefont {Bellare}, \bibfnamefont
  {Mihir}}, \bibinfo {author} {\bibfnamefont {Anand}\ \bibnamefont {Desai}},
  \bibinfo {author} {\bibfnamefont {David}\ \bibnamefont {Pointcheval}}, \ and\
  \bibinfo {author} {\bibfnamefont {Phillip}\ \bibnamefont {Rogaway}}}
  (\bibinfo {year} {1998}),\ \bibfield  {title} {\enquote {\bibinfo {title}
  {Relations among notions of security for public-key encryption schemes},}\
  }in\ \href {\doibase 10.1007/BFb0055718} {\emph {\bibinfo {booktitle}
  {Advances in Cryptology -- CRYPTO~'98}}}\ (\bibinfo  {publisher} {Springer})\
  pp.\ \bibinfo {pages} {26--45}\BibitemShut {NoStop}%
\bibitem [{\citenamefont {Bellare}\ and\ \citenamefont {Rogaway}(2006)}]{BR06}%
  \BibitemOpen
  \bibfield  {author} {\bibinfo {author} {\bibnamefont {Bellare}, \bibfnamefont
  {Mihir}}, \ and\ \bibinfo {author} {\bibfnamefont {Phillip}\ \bibnamefont
  {Rogaway}}} (\bibinfo {year} {2006}),\ \bibfield  {title} {\enquote {\bibinfo
  {title} {The security of triple encryption and a framework for code-based
  game-playing proofs},}\ }in\ \href {\doibase 10.1007/11761679_25} {\emph
  {\bibinfo {booktitle} {Advances in Cryptology -- EUROCRYPT 2006}}},\ \bibinfo
  {series} {LNCS}, Vol.\ \bibinfo {volume} {4004},\ \bibinfo {editor} {edited
  by\ \bibinfo {editor} {\bibfnamefont {Serge}\ \bibnamefont {Vaudenay}}}\
  (\bibinfo  {publisher} {Springer})\ pp.\ \bibinfo {pages} {409--426},\
  \bibinfo {note} {e-Print \href{http://eprint.iacr.org/2004/331}{IACR
  2004/331}}\BibitemShut {NoStop}%
\bibitem [{\citenamefont {{Ben-Aroya}}\ and\ \citenamefont
  {{Ta-Shma}}(2012)}]{BT12}%
  \BibitemOpen
  \bibfield  {author} {\bibinfo {author} {\bibnamefont {{Ben-Aroya}},
  \bibfnamefont {Avraham}}, \ and\ \bibinfo {author} {\bibfnamefont {Amnon}\
  \bibnamefont {{Ta-Shma}}}} (\bibinfo {year} {2012}),\ \bibfield  {title}
  {\enquote {\bibinfo {title} {Better short-seed quantum-proof extractors},}\
  }\href {\doibase 10.1016/j.tcs.2011.11.036} {\bibfield  {journal} {\bibinfo
  {journal} {Theoretical Computer Science}\ }\textbf {\bibinfo {volume}
  {419}},\ \bibinfo {pages} {17--25}},\ \Eprint
  {http://arxiv.org/abs/arXiv:1004.3737} {arXiv:1004.3737} \BibitemShut
  {NoStop}%
\bibitem [{\citenamefont {{Ben-Or}}\ \emph {et~al.}(2006)\citenamefont
  {{Ben-Or}}, \citenamefont {Cr\'epeau}, \citenamefont {Gottesman},
  \citenamefont {Hassidim},\ and\ \citenamefont {Smith}}]{BCGHS06}%
  \BibitemOpen
  \bibfield  {author} {\bibinfo {author} {\bibnamefont {{Ben-Or}},
  \bibfnamefont {Michael}}, \bibinfo {author} {\bibfnamefont {Claude}\
  \bibnamefont {Cr\'epeau}}, \bibinfo {author} {\bibfnamefont {Daniel}\
  \bibnamefont {Gottesman}}, \bibinfo {author} {\bibfnamefont {Avinatan}\
  \bibnamefont {Hassidim}}, \ and\ \bibinfo {author} {\bibfnamefont {Adam}\
  \bibnamefont {Smith}}} (\bibinfo {year} {2006}),\ \bibfield  {title}
  {\enquote {\bibinfo {title} {Secure multiparty quantum computation with
  (only) a strict honest majority},}\ }in\ \href {\doibase
  10.1109/FOCS.2006.68} {\emph {\bibinfo {booktitle} {Proceedings of the 47th
  Symposium on Foundations of Computer Science, FOCS~'06}}},\ pp.\ \bibinfo
  {pages} {249--260},\ \Eprint {http://arxiv.org/abs/arXiv:0801.1544}
  {arXiv:0801.1544} \BibitemShut {NoStop}%
\bibitem [{\citenamefont {{Ben-Or}}\ \emph {et~al.}(2005)\citenamefont
  {{Ben-Or}}, \citenamefont {Horodecki}, \citenamefont {Leung}, \citenamefont
  {Mayers},\ and\ \citenamefont {Oppenheim}}]{BHLMO05}%
  \BibitemOpen
  \bibfield  {author} {\bibinfo {author} {\bibnamefont {{Ben-Or}},
  \bibfnamefont {Michael}}, \bibinfo {author} {\bibfnamefont {Micha\l{}}\
  \bibnamefont {Horodecki}}, \bibinfo {author} {\bibfnamefont {Debbie}\
  \bibnamefont {Leung}}, \bibinfo {author} {\bibfnamefont {Dominic}\
  \bibnamefont {Mayers}}, \ and\ \bibinfo {author} {\bibfnamefont {Jonathan}\
  \bibnamefont {Oppenheim}}} (\bibinfo {year} {2005}),\ \bibfield  {title}
  {\enquote {\bibinfo {title} {The universal composable security of quantum key
  distribution},}\ }in\ \href {\doibase 10.1007/978-3-540-30576-7_21} {\emph
  {\bibinfo {booktitle} {Theory of Cryptography, Proceedings of TCC 2005}}},\
  \bibinfo {series} {LNCS}, Vol.\ \bibinfo {volume} {3378}\ (\bibinfo
  {publisher} {Springer})\ pp.\ \bibinfo {pages} {386--406},\ \Eprint
  {http://arxiv.org/abs/arXiv:quant-ph/0409078} {arXiv:quant-ph/0409078}
  \BibitemShut {NoStop}%
\bibitem [{\citenamefont {{Ben-Or}}\ and\ \citenamefont {Mayers}(2004)}]{BM04}%
  \BibitemOpen
  \bibfield  {author} {\bibinfo {author} {\bibnamefont {{Ben-Or}},
  \bibfnamefont {Michael}}, \ and\ \bibinfo {author} {\bibfnamefont {Dominic}\
  \bibnamefont {Mayers}}} (\bibinfo {year} {2004}),\ \href@noop {} {\enquote
  {\bibinfo {title} {General security definition and composability for quantum
  \& classical protocols},}\ }\bibinfo {howpublished} {e-Print},\ \Eprint
  {http://arxiv.org/abs/arXiv:quant-ph/0409062} {arXiv:quant-ph/0409062}
  \BibitemShut {NoStop}%
\bibitem [{\citenamefont {Bennett}\ \emph
  {et~al.}(1996{\natexlab{a}})\citenamefont {Bennett}, \citenamefont
  {Bernstein}, \citenamefont {Popescu},\ and\ \citenamefont
  {Schumacher}}]{Bennett96}%
  \BibitemOpen
  \bibfield  {author} {\bibinfo {author} {\bibnamefont {Bennett}, \bibfnamefont
  {Charles~H}}, \bibinfo {author} {\bibfnamefont {Herbert~J.}\ \bibnamefont
  {Bernstein}}, \bibinfo {author} {\bibfnamefont {Sandu}\ \bibnamefont
  {Popescu}}, \ and\ \bibinfo {author} {\bibfnamefont {Benjamin}\ \bibnamefont
  {Schumacher}}} (\bibinfo {year} {1996}{\natexlab{a}}),\ \bibfield  {title}
  {\enquote {\bibinfo {title} {Concentrating partial entanglement by local
  operations},}\ }\href {\doibase 10.1103/PhysRevA.53.2046} {\bibfield
  {journal} {\bibinfo  {journal} {Phys. Rev. A}\ }\textbf {\bibinfo {volume}
  {53}},\ \bibinfo {pages} {2046--2052}}\BibitemShut {NoStop}%
\bibitem [{\citenamefont {Bennett}\ \emph
  {et~al.}(1992{\natexlab{a}})\citenamefont {Bennett}, \citenamefont
  {Bessette}, \citenamefont {Brassard}, \citenamefont {Salvail},\ and\
  \citenamefont {Smolin}}]{BBBSS92}%
  \BibitemOpen
  \bibfield  {author} {\bibinfo {author} {\bibnamefont {Bennett}, \bibfnamefont
  {Charles~H}}, \bibinfo {author} {\bibfnamefont {Fran{\c{c}}ois}\ \bibnamefont
  {Bessette}}, \bibinfo {author} {\bibfnamefont {Gilles}\ \bibnamefont
  {Brassard}}, \bibinfo {author} {\bibfnamefont {Louis}\ \bibnamefont
  {Salvail}}, \ and\ \bibinfo {author} {\bibfnamefont {John}\ \bibnamefont
  {Smolin}}} (\bibinfo {year} {1992}{\natexlab{a}}),\ \bibfield  {title}
  {\enquote {\bibinfo {title} {Experimental quantum cryptography},}\ }\href
  {\doibase 10.1007/BF00191318} {\bibfield  {journal} {\bibinfo  {journal} {J.
  Crypt.}\ }\textbf {\bibinfo {volume} {5}}~(\bibinfo {number} {1}),\ \bibinfo
  {pages} {3--28}}\BibitemShut {NoStop}%
\bibitem [{\citenamefont {Bennett}\ and\ \citenamefont
  {Brassard}(1984)}]{BB84}%
  \BibitemOpen
  \bibfield  {author} {\bibinfo {author} {\bibnamefont {Bennett}, \bibfnamefont
  {Charles~H}}, \ and\ \bibinfo {author} {\bibfnamefont {Gilles}\ \bibnamefont
  {Brassard}}} (\bibinfo {year} {1984}),\ \bibfield  {title} {\enquote
  {\bibinfo {title} {Quantum cryptography: Public key distribution and coin
  tossing},}\ }in\ \href@noop {} {\emph {\bibinfo {booktitle} {Proceedings of
  IEEE International Conference on Computers, Systems, and Signal
  Processing}}},\ pp.\ \bibinfo {pages} {175--179}\BibitemShut {NoStop}%
\bibitem [{\citenamefont {Bennett}\ \emph {et~al.}(1995)\citenamefont
  {Bennett}, \citenamefont {Brassard}, \citenamefont {Cr{\'e}peau},\ and\
  \citenamefont {Maurer}}]{BBCM95}%
  \BibitemOpen
  \bibfield  {author} {\bibinfo {author} {\bibnamefont {Bennett}, \bibfnamefont
  {Charles~H}}, \bibinfo {author} {\bibfnamefont {Gilles}\ \bibnamefont
  {Brassard}}, \bibinfo {author} {\bibfnamefont {Claude}\ \bibnamefont
  {Cr{\'e}peau}}, \ and\ \bibinfo {author} {\bibfnamefont {Ueli}\ \bibnamefont
  {Maurer}}} (\bibinfo {year} {1995}),\ \bibfield  {title} {\enquote {\bibinfo
  {title} {Generalized privacy amplification},}\ }\href {\doibase
  10.1109/18.476316} {\bibfield  {journal} {\bibinfo  {journal} {IEEE Trans.
  Inf. Theory}\ }\textbf {\bibinfo {volume} {41}}~(\bibinfo {number} {6}),\
  \bibinfo {pages} {1915--1923}}\BibitemShut {NoStop}%
\bibitem [{\citenamefont {Bennett}\ \emph
  {et~al.}(1992{\natexlab{b}})\citenamefont {Bennett}, \citenamefont
  {Brassard}, \citenamefont {Cr{\'{e}}peau},\ and\ \citenamefont
  {Skubiszewska}}]{BBCS92}%
  \BibitemOpen
  \bibfield  {author} {\bibinfo {author} {\bibnamefont {Bennett}, \bibfnamefont
  {Charles~H}}, \bibinfo {author} {\bibfnamefont {Gilles}\ \bibnamefont
  {Brassard}}, \bibinfo {author} {\bibfnamefont {Claude}\ \bibnamefont
  {Cr{\'{e}}peau}}, \ and\ \bibinfo {author} {\bibfnamefont
  {Marie{-}H{\'{e}}l{\`{e}}ne}\ \bibnamefont {Skubiszewska}}} (\bibinfo {year}
  {1992}{\natexlab{b}}),\ \bibfield  {title} {\enquote {\bibinfo {title}
  {Practical quantum oblivious transfer},}\ }in\ \href {\doibase
  10.1007/3-540-46766-1_29} {\emph {\bibinfo {booktitle} {Advances in
  Cryptology -- CRYPTO~'91}}},\ \bibinfo {series} {LNCS}, Vol.\ \bibinfo
  {volume} {576}\ (\bibinfo  {publisher} {Springer})\ pp.\ \bibinfo {pages}
  {351--366}\BibitemShut {NoStop}%
\bibitem [{\citenamefont {Bennett}\ \emph
  {et~al.}(1992{\natexlab{c}})\citenamefont {Bennett}, \citenamefont
  {Brassard},\ and\ \citenamefont {Mermin}}]{BBM92}%
  \BibitemOpen
  \bibfield  {author} {\bibinfo {author} {\bibnamefont {Bennett}, \bibfnamefont
  {Charles~H}}, \bibinfo {author} {\bibfnamefont {Gilles}\ \bibnamefont
  {Brassard}}, \ and\ \bibinfo {author} {\bibfnamefont {N.~David}\ \bibnamefont
  {Mermin}}} (\bibinfo {year} {1992}{\natexlab{c}}),\ \bibfield  {title}
  {\enquote {\bibinfo {title} {Quantum cryptography without {Bell}'s
  theorem},}\ }\href {\doibase 10.1103/PhysRevLett.68.557} {\bibfield
  {journal} {\bibinfo  {journal} {Phys. Rev. Lett.}\ }\textbf {\bibinfo
  {volume} {68}},\ \bibinfo {pages} {557--559}}\BibitemShut {NoStop}%
\bibitem [{\citenamefont {Bennett}\ \emph
  {et~al.}(1996{\natexlab{b}})\citenamefont {Bennett}, \citenamefont
  {Brassard}, \citenamefont {Popescu}, \citenamefont {Schumacher},
  \citenamefont {Smolin},\ and\ \citenamefont {Wootters}}]{Benett96b}%
  \BibitemOpen
  \bibfield  {author} {\bibinfo {author} {\bibnamefont {Bennett}, \bibfnamefont
  {Charles~H}}, \bibinfo {author} {\bibfnamefont {Gilles}\ \bibnamefont
  {Brassard}}, \bibinfo {author} {\bibfnamefont {Sandu}\ \bibnamefont
  {Popescu}}, \bibinfo {author} {\bibfnamefont {Benjamin}\ \bibnamefont
  {Schumacher}}, \bibinfo {author} {\bibfnamefont {John~A.}\ \bibnamefont
  {Smolin}}, \ and\ \bibinfo {author} {\bibfnamefont {William~K.}\ \bibnamefont
  {Wootters}}} (\bibinfo {year} {1996}{\natexlab{b}}),\ \bibfield  {title}
  {\enquote {\bibinfo {title} {Purification of noisy entanglement and faithful
  teleportation via noisy channels},}\ }\href {\doibase
  10.1103/PhysRevLett.76.722} {\bibfield  {journal} {\bibinfo  {journal} {Phys.
  Rev. Lett.}\ }\textbf {\bibinfo {volume} {76}},\ \bibinfo {pages}
  {722--725}}\BibitemShut {NoStop}%
\bibitem [{\citenamefont {Bennett}\ \emph {et~al.}(1988)\citenamefont
  {Bennett}, \citenamefont {Brassard},\ and\ \citenamefont {Robert}}]{BBR88}%
  \BibitemOpen
  \bibfield  {author} {\bibinfo {author} {\bibnamefont {Bennett}, \bibfnamefont
  {Charles~H}}, \bibinfo {author} {\bibfnamefont {Gilles}\ \bibnamefont
  {Brassard}}, \ and\ \bibinfo {author} {\bibfnamefont {Jean-Marc}\
  \bibnamefont {Robert}}} (\bibinfo {year} {1988}),\ \bibfield  {title}
  {\enquote {\bibinfo {title} {Privacy amplification by public discussion},}\
  }\href {\doibase 10.1137/0217014} {\bibfield  {journal} {\bibinfo  {journal}
  {SIAM J. Comput.}\ }\textbf {\bibinfo {volume} {17}}~(\bibinfo {number}
  {2}),\ \bibinfo {pages} {210--229}}\BibitemShut {NoStop}%
\bibitem [{\citenamefont {Berta}\ \emph {et~al.}(2010)\citenamefont {Berta},
  \citenamefont {Christandl}, \citenamefont {Colbeck}, \citenamefont {Renes},\
  and\ \citenamefont {Renner}}]{Berta10}%
  \BibitemOpen
  \bibfield  {author} {\bibinfo {author} {\bibnamefont {Berta}, \bibfnamefont
  {Mario}}, \bibinfo {author} {\bibfnamefont {Matthias}\ \bibnamefont
  {Christandl}}, \bibinfo {author} {\bibfnamefont {Roger}\ \bibnamefont
  {Colbeck}}, \bibinfo {author} {\bibfnamefont {Joseph~M.}\ \bibnamefont
  {Renes}}, \ and\ \bibinfo {author} {\bibfnamefont {Renato}\ \bibnamefont
  {Renner}}} (\bibinfo {year} {2010}),\ \bibfield  {title} {\enquote {\bibinfo
  {title} {The uncertainty principle in the presence of quantum memory},}\
  }\href {\doibase 10.1038/nphys1734} {\bibfield  {journal} {\bibinfo
  {journal} {Nat. Phys.}\ }\textbf {\bibinfo {volume} {6}}~(\bibinfo {number}
  {9}),\ \bibinfo {pages} {659--662}}\BibitemShut {NoStop}%
\bibitem [{\citenamefont {Berta}\ \emph {et~al.}(2017)\citenamefont {Berta},
  \citenamefont {Fawzi},\ and\ \citenamefont {Scholz}}]{BFS17}%
  \BibitemOpen
  \bibfield  {author} {\bibinfo {author} {\bibnamefont {Berta}, \bibfnamefont
  {Mario}}, \bibinfo {author} {\bibfnamefont {Omar}\ \bibnamefont {Fawzi}}, \
  and\ \bibinfo {author} {\bibfnamefont {Volkher~B.}\ \bibnamefont {Scholz}}}
  (\bibinfo {year} {2017}),\ \bibfield  {title} {\enquote {\bibinfo {title}
  {Quantum-proof randomness extractors via operator space theory},}\ }\href
  {\doibase 10.1109/TIT.2016.2627531} {\bibfield  {journal} {\bibinfo
  {journal} {IEEE Trans. Inf. Theory}\ }\textbf {\bibinfo {volume}
  {63}}~(\bibinfo {number} {4}),\ \bibinfo {pages} {2480--2503}},\ \Eprint
  {http://arxiv.org/abs/arxiv:1409.3563} {arxiv:1409.3563} \BibitemShut
  {NoStop}%
\bibitem [{\citenamefont {Biham}\ \emph {et~al.}(2000)\citenamefont {Biham},
  \citenamefont {Boyer}, \citenamefont {Boykin}, \citenamefont {Mor},\ and\
  \citenamefont {Roychowdhury}}]{BBBMR00}%
  \BibitemOpen
  \bibfield  {author} {\bibinfo {author} {\bibnamefont {Biham}, \bibfnamefont
  {Eli}}, \bibinfo {author} {\bibfnamefont {Michel}\ \bibnamefont {Boyer}},
  \bibinfo {author} {\bibfnamefont {P.~Oscar}\ \bibnamefont {Boykin}}, \bibinfo
  {author} {\bibfnamefont {Tal}\ \bibnamefont {Mor}}, \ and\ \bibinfo {author}
  {\bibfnamefont {Vwani}\ \bibnamefont {Roychowdhury}}} (\bibinfo {year}
  {2000}),\ \bibfield  {title} {\enquote {\bibinfo {title} {A proof of the
  security of quantum key distribution (extended abstract)},}\ }in\ \href
  {\doibase 10.1145/335305.335406} {\emph {\bibinfo {booktitle} {Proceedings of
  the 32nd Symposium on Theory of Computing, STOC~'00}}}\ (\bibinfo
  {publisher} {ACM})\ pp.\ \bibinfo {pages} {715--724},\ \Eprint
  {http://arxiv.org/abs/arXiv:quant-ph/9912053} {arXiv:quant-ph/9912053}
  \BibitemShut {NoStop}%
\bibitem [{\citenamefont {Biham}\ \emph {et~al.}(2006)\citenamefont {Biham},
  \citenamefont {Boyer}, \citenamefont {Boykin}, \citenamefont {Mor},\ and\
  \citenamefont {Roychowdhury}}]{BBBMR06}%
  \BibitemOpen
  \bibfield  {author} {\bibinfo {author} {\bibnamefont {Biham}, \bibfnamefont
  {Eli}}, \bibinfo {author} {\bibfnamefont {Michel}\ \bibnamefont {Boyer}},
  \bibinfo {author} {\bibfnamefont {P.~Oscar}\ \bibnamefont {Boykin}}, \bibinfo
  {author} {\bibfnamefont {Tal}\ \bibnamefont {Mor}}, \ and\ \bibinfo {author}
  {\bibfnamefont {Vwani}\ \bibnamefont {Roychowdhury}}} (\bibinfo {year}
  {2006}),\ \bibfield  {title} {\enquote {\bibinfo {title} {A proof of the
  security of quantum key distribution},}\ }\href {\doibase
  10.1007/s00145-005-0011-3} {\bibfield  {journal} {\bibinfo  {journal} {J.
  Crypt.}\ }\textbf {\bibinfo {volume} {19}}~(\bibinfo {number} {4}),\ \bibinfo
  {pages} {381--439}},\ \bibinfo {note} {full version of \textcite{BBBMR00}},\
  \Eprint {http://arxiv.org/abs/arXiv:quant-ph/0511175}
  {arXiv:quant-ph/0511175} \BibitemShut {NoStop}%
\bibitem [{\citenamefont {Biham}\ \emph {et~al.}(2002)\citenamefont {Biham},
  \citenamefont {Boyer}, \citenamefont {Brassard}, \citenamefont {van~de
  Graaf},\ and\ \citenamefont {Mor}}]{BBBvdGM02}%
  \BibitemOpen
  \bibfield  {author} {\bibinfo {author} {\bibnamefont {Biham}, \bibfnamefont
  {Eli}}, \bibinfo {author} {\bibfnamefont {Michel}\ \bibnamefont {Boyer}},
  \bibinfo {author} {\bibfnamefont {Gilles}\ \bibnamefont {Brassard}}, \bibinfo
  {author} {\bibfnamefont {Jeroen}\ \bibnamefont {van~de Graaf}}, \ and\
  \bibinfo {author} {\bibfnamefont {Tal}\ \bibnamefont {Mor}}} (\bibinfo {year}
  {2002}),\ \bibfield  {title} {\enquote {\bibinfo {title} {Security of quantum
  key distribution against all collective attacks},}\ }\href {\doibase
  10.1007/s00453-002-0973-6} {\bibfield  {journal} {\bibinfo  {journal}
  {Algorithmica}\ }\textbf {\bibinfo {volume} {34}}~(\bibinfo {number} {4}),\
  \bibinfo {pages} {372--388}},\ \Eprint
  {http://arxiv.org/abs/quant-ph/9801022} {quant-ph/9801022} \BibitemShut
  {NoStop}%
\bibitem [{\citenamefont {Biham}\ and\ \citenamefont {Mor}(1997)}]{BM97b}%
  \BibitemOpen
  \bibfield  {author} {\bibinfo {author} {\bibnamefont {Biham}, \bibfnamefont
  {Eli}}, \ and\ \bibinfo {author} {\bibfnamefont {Tal}\ \bibnamefont {Mor}}}
  (\bibinfo {year} {1997}),\ \bibfield  {title} {\enquote {\bibinfo {title}
  {Security of quantum cryptography against collective attacks},}\ }\href
  {\doibase 10.1103/PhysRevLett.78.2256} {\bibfield  {journal} {\bibinfo
  {journal} {Phys. Rev. Lett.}\ }\textbf {\bibinfo {volume} {78}},\ \bibinfo
  {pages} {2256--2259}},\ \Eprint {http://arxiv.org/abs/arXiv:quant-ph/9605007}
  {arXiv:quant-ph/9605007} \BibitemShut {NoStop}%
\bibitem [{\citenamefont {Blum}(1983)}]{Blu83}%
  \BibitemOpen
  \bibfield  {author} {\bibinfo {author} {\bibnamefont {Blum}, \bibfnamefont
  {Manuel}}} (\bibinfo {year} {1983}),\ \bibfield  {title} {\enquote {\bibinfo
  {title} {Coin flipping by telephone a protocol for solving impossible
  problems},}\ }\href {\doibase 10.1145/1008908.1008911} {\bibfield  {journal}
  {\bibinfo  {journal} {ACM SIGACT News}\ }\textbf {\bibinfo {volume}
  {15}}~(\bibinfo {number} {1}),\ \bibinfo {pages} {23--27}}\BibitemShut
  {NoStop}%
\bibitem [{\citenamefont {Boileau}\ \emph {et~al.}(2005)\citenamefont
  {Boileau}, \citenamefont {Tamaki}, \citenamefont {Batuwantudawe},
  \citenamefont {Laflamme},\ and\ \citenamefont {Renes}}]{Boileau}%
  \BibitemOpen
  \bibfield  {author} {\bibinfo {author} {\bibnamefont {Boileau}, \bibfnamefont
  {J-C}}, \bibinfo {author} {\bibfnamefont {Kiyoshi}\ \bibnamefont {Tamaki}},
  \bibinfo {author} {\bibfnamefont {Jamie}\ \bibnamefont {Batuwantudawe}},
  \bibinfo {author} {\bibfnamefont {Raymond}\ \bibnamefont {Laflamme}}, \ and\
  \bibinfo {author} {\bibfnamefont {Joseph~M.}\ \bibnamefont {Renes}}}
  (\bibinfo {year} {2005}),\ \bibfield  {title} {\enquote {\bibinfo {title}
  {Unconditional security of a three state quantum key distribution
  protocol},}\ }\href {\doibase 10.1103/PhysRevLett.94.040503} {\bibfield
  {journal} {\bibinfo  {journal} {Phys. Rev. Lett.}\ }\textbf {\bibinfo
  {volume} {94}},\ \bibinfo {pages} {040503}}\BibitemShut {NoStop}%
\bibitem [{\citenamefont {Born}(1926)}]{Born26}%
  \BibitemOpen
  \bibfield  {author} {\bibinfo {author} {\bibnamefont {Born}, \bibfnamefont
  {Max}}} (\bibinfo {year} {1926}),\ \bibfield  {title} {\enquote {\bibinfo
  {title} {Zur {Q}uantenmechanik der {S}to{\ss}vorg{\"a}nge},}\ }\href@noop {}
  {\bibfield  {journal} {\bibinfo  {journal} {Zeitschrift f{\"u}r Physik}\
  }\textbf {\bibinfo {volume} {37}}~(\bibinfo {number} {12}),\ \bibinfo {pages}
  {863--867}}\BibitemShut {NoStop}%
\bibitem [{\citenamefont {Boykin}\ and\ \citenamefont
  {Roychowdhury}(2003)}]{BR03}%
  \BibitemOpen
  \bibfield  {author} {\bibinfo {author} {\bibnamefont {Boykin}, \bibfnamefont
  {P~Oscar}}, \ and\ \bibinfo {author} {\bibfnamefont {Vwani}\ \bibnamefont
  {Roychowdhury}}} (\bibinfo {year} {2003}),\ \bibfield  {title} {\enquote
  {\bibinfo {title} {Optimal encryption of quantum bits},}\ }\href {\doibase
  10.1103/PhysRevA.67.042317} {\bibfield  {journal} {\bibinfo  {journal} {Phys.
  Rev. A}\ }\textbf {\bibinfo {volume} {67}},\ \bibinfo {pages} {042317}},\
  \Eprint {http://arxiv.org/abs/arXiv:quant-ph/0003059}
  {arXiv:quant-ph/0003059} \BibitemShut {NoStop}%
\bibitem [{\citenamefont {Bozzio}\ \emph {et~al.}(2019)\citenamefont {Bozzio},
  \citenamefont {Diamanti},\ and\ \citenamefont {Grosshans}}]{BDG19}%
  \BibitemOpen
  \bibfield  {author} {\bibinfo {author} {\bibnamefont {Bozzio}, \bibfnamefont
  {Mathieu}}, \bibinfo {author} {\bibfnamefont {Eleni}\ \bibnamefont
  {Diamanti}}, \ and\ \bibinfo {author} {\bibfnamefont {Fr\'ed\'eric}\
  \bibnamefont {Grosshans}}} (\bibinfo {year} {2019}),\ \bibfield  {title}
  {\enquote {\bibinfo {title} {Semi-device-independent quantum money with
  coherent states},}\ }\href {\doibase 10.1103/PhysRevA.99.022336} {\bibfield
  {journal} {\bibinfo  {journal} {Phys. Rev. A}\ }\textbf {\bibinfo {volume}
  {99}},\ \bibinfo {pages} {022336}},\ \Eprint
  {http://arxiv.org/abs/arXiv:1812.09256} {arXiv:1812.09256} \BibitemShut
  {NoStop}%
\bibitem [{\citenamefont {Branciard}\ \emph {et~al.}(2012)\citenamefont
  {Branciard}, \citenamefont {Cavalcanti}, \citenamefont {Walborn},
  \citenamefont {Scarani},\ and\ \citenamefont {Wiseman}}]{BCWSW12}%
  \BibitemOpen
  \bibfield  {author} {\bibinfo {author} {\bibnamefont {Branciard},
  \bibfnamefont {Cyril}}, \bibinfo {author} {\bibfnamefont {Eric~G.}\
  \bibnamefont {Cavalcanti}}, \bibinfo {author} {\bibfnamefont {Stephen~P.}\
  \bibnamefont {Walborn}}, \bibinfo {author} {\bibfnamefont {Valerio}\
  \bibnamefont {Scarani}}, \ and\ \bibinfo {author} {\bibfnamefont {Howard~M.}\
  \bibnamefont {Wiseman}}} (\bibinfo {year} {2012}),\ \bibfield  {title}
  {\enquote {\bibinfo {title} {One-sided device-independent quantum key
  distribution: Security, feasibility, and the connection with steering},}\
  }\href {\doibase 10.1103/PhysRevA.85.010301} {\bibfield  {journal} {\bibinfo
  {journal} {Phys. Rev. A}\ }\textbf {\bibinfo {volume} {85}},\ \bibinfo
  {pages} {010301}}\BibitemShut {NoStop}%
\bibitem [{\citenamefont {Brand{\~a}o}\ \emph {et~al.}(2016)\citenamefont
  {Brand{\~a}o}, \citenamefont {Ramanathan}, \citenamefont {Grudka},
  \citenamefont {Horodecki}, \citenamefont {Horodecki}, \citenamefont
  {Horodecki}, \citenamefont {Szarek},\ and\ \citenamefont
  {Wojew{\'o}dka}}]{BRGHHHSW16}%
  \BibitemOpen
  \bibfield  {author} {\bibinfo {author} {\bibnamefont {Brand{\~a}o},
  \bibfnamefont {Fernando G S~L}}, \bibinfo {author} {\bibfnamefont
  {Ravishankar}\ \bibnamefont {Ramanathan}}, \bibinfo {author} {\bibfnamefont
  {Andrzej}\ \bibnamefont {Grudka}}, \bibinfo {author} {\bibfnamefont {Karol}\
  \bibnamefont {Horodecki}}, \bibinfo {author} {\bibfnamefont {Micha\l{}}\
  \bibnamefont {Horodecki}}, \bibinfo {author} {\bibfnamefont {Pawe\l{}}\
  \bibnamefont {Horodecki}}, \bibinfo {author} {\bibfnamefont {Tomasz}\
  \bibnamefont {Szarek}}, \ and\ \bibinfo {author} {\bibfnamefont {Hanna}\
  \bibnamefont {Wojew{\'o}dka}}} (\bibinfo {year} {2016}),\ \bibfield  {title}
  {\enquote {\bibinfo {title} {Realistic noise-tolerant randomness
  amplification using finite number of devices},}\ }\href {\doibase
  10.1038/ncomms11345} {\bibfield  {journal} {\bibinfo  {journal} {Nat.
  Commun.}\ }\textbf {\bibinfo {volume} {7}},\ \bibinfo {pages} {11345}},\
  \Eprint {http://arxiv.org/abs/arXiv:1310.4544} {arXiv:1310.4544} \BibitemShut
  {NoStop}%
\bibitem [{\citenamefont {Brassard}\ \emph {et~al.}(1998)\citenamefont
  {Brassard}, \citenamefont {Cr\'epeau}, \citenamefont {Mayers},\ and\
  \citenamefont {Salvail}}]{BCMS98}%
  \BibitemOpen
  \bibfield  {author} {\bibinfo {author} {\bibnamefont {Brassard},
  \bibfnamefont {Gilles}}, \bibinfo {author} {\bibfnamefont {Claude}\
  \bibnamefont {Cr\'epeau}}, \bibinfo {author} {\bibfnamefont {Dominic}\
  \bibnamefont {Mayers}}, \ and\ \bibinfo {author} {\bibfnamefont {Louis}\
  \bibnamefont {Salvail}}} (\bibinfo {year} {1998}),\ \href@noop {} {\enquote
  {\bibinfo {title} {Defeating classical bit commitments with a quantum
  computer},}\ }\bibinfo {howpublished} {e-print},\ \Eprint
  {http://arxiv.org/abs/arXiv:quant-ph/9806031} {arXiv:quant-ph/9806031}
  \BibitemShut {NoStop}%
\bibitem [{\citenamefont {Brassard}\ \emph {et~al.}(2000)\citenamefont
  {Brassard}, \citenamefont {L\"utkenhaus}, \citenamefont {Mor},\ and\
  \citenamefont {Sanders}}]{Brassardetal2000}%
  \BibitemOpen
  \bibfield  {author} {\bibinfo {author} {\bibnamefont {Brassard},
  \bibfnamefont {Gilles}}, \bibinfo {author} {\bibfnamefont {Norbert}\
  \bibnamefont {L\"utkenhaus}}, \bibinfo {author} {\bibfnamefont {Tal}\
  \bibnamefont {Mor}}, \ and\ \bibinfo {author} {\bibfnamefont {Barry~C.}\
  \bibnamefont {Sanders}}} (\bibinfo {year} {2000}),\ \bibfield  {title}
  {\enquote {\bibinfo {title} {Limitations on practical quantum
  cryptography},}\ }\href {\doibase 10.1103/PhysRevLett.85.1330} {\bibfield
  {journal} {\bibinfo  {journal} {Phys. Rev. Lett.}\ }\textbf {\bibinfo
  {volume} {85}},\ \bibinfo {pages} {1330--1333}}\BibitemShut {NoStop}%
\bibitem [{\citenamefont {Braunstein}\ and\ \citenamefont
  {Pirandola}(2012)}]{BP12}%
  \BibitemOpen
  \bibfield  {author} {\bibinfo {author} {\bibnamefont {Braunstein},
  \bibfnamefont {Samuel~L}}, \ and\ \bibinfo {author} {\bibfnamefont {Stefano}\
  \bibnamefont {Pirandola}}} (\bibinfo {year} {2012}),\ \bibfield  {title}
  {\enquote {\bibinfo {title} {Side-channel-free quantum key distribution},}\
  }\href {\doibase 10.1103/PhysRevLett.108.130502} {\bibfield  {journal}
  {\bibinfo  {journal} {Phys. Rev. Lett.}\ }\textbf {\bibinfo {volume} {108}},\
  \bibinfo {pages} {130502}}\BibitemShut {NoStop}%
\bibitem [{\citenamefont {Broadbent}\ \emph {et~al.}(2009)\citenamefont
  {Broadbent}, \citenamefont {Fitzsimons},\ and\ \citenamefont
  {Kashefi}}]{BFK09}%
  \BibitemOpen
  \bibfield  {author} {\bibinfo {author} {\bibnamefont {Broadbent},
  \bibfnamefont {Anne}}, \bibinfo {author} {\bibfnamefont {Joseph}\
  \bibnamefont {Fitzsimons}}, \ and\ \bibinfo {author} {\bibfnamefont {Elham}\
  \bibnamefont {Kashefi}}} (\bibinfo {year} {2009}),\ \bibfield  {title}
  {\enquote {\bibinfo {title} {Universal blind quantum computation},}\ }in\
  \href {\doibase 10.1109/FOCS.2009.36} {\emph {\bibinfo {booktitle}
  {Proceedings of the 50th Symposium on Foundations of Computer Science,
  FOCS~'09}}}\ (\bibinfo  {publisher} {IEEE Computer Society})\ pp.\ \bibinfo
  {pages} {517--526},\ \Eprint {http://arxiv.org/abs/arXiv:0807.4154}
  {arXiv:0807.4154} \BibitemShut {NoStop}%
\bibitem [{\citenamefont {Broadbent}\ \emph {et~al.}(2013)\citenamefont
  {Broadbent}, \citenamefont {Gutoski},\ and\ \citenamefont {Stebila}}]{BGS13}%
  \BibitemOpen
  \bibfield  {author} {\bibinfo {author} {\bibnamefont {Broadbent},
  \bibfnamefont {Anne}}, \bibinfo {author} {\bibfnamefont {Gus}\ \bibnamefont
  {Gutoski}}, \ and\ \bibinfo {author} {\bibfnamefont {Douglas}\ \bibnamefont
  {Stebila}}} (\bibinfo {year} {2013}),\ \bibfield  {title} {\enquote {\bibinfo
  {title} {Quantum one-time programs},}\ }in\ \href {\doibase
  10.1007/978-3-642-40084-1_20} {\emph {\bibinfo {booktitle} {Advances in
  Cryptology -- CRYPTO 2013}}},\ \bibinfo {series} {LNCS}, Vol.\ \bibinfo
  {volume} {8043}\ (\bibinfo  {publisher} {Springer})\ pp.\ \bibinfo {pages}
  {344--360},\ \Eprint {http://arxiv.org/abs/arXiv:1211.1080} {arXiv:1211.1080}
  \BibitemShut {NoStop}%
\bibitem [{\citenamefont {Broadbent}\ and\ \citenamefont
  {Jeffery}(2015)}]{BJ15}%
  \BibitemOpen
  \bibfield  {author} {\bibinfo {author} {\bibnamefont {Broadbent},
  \bibfnamefont {Anne}}, \ and\ \bibinfo {author} {\bibfnamefont {Stacey}\
  \bibnamefont {Jeffery}}} (\bibinfo {year} {2015}),\ \bibfield  {title}
  {\enquote {\bibinfo {title} {Quantum homomorphic encryption for circuits of
  low t-gate complexity},}\ }in\ \href {\doibase 10.1007/978-3-662-48000-7_30}
  {\emph {\bibinfo {booktitle} {Advances in Cryptology -- CRYPTO 2015}}},\
  \bibinfo {editor} {edited by\ \bibinfo {editor} {\bibfnamefont {Rosario}\
  \bibnamefont {Gennaro}}\ and\ \bibinfo {editor} {\bibfnamefont {Matthew}\
  \bibnamefont {Robshaw}}}\ (\bibinfo  {publisher} {Springer})\ pp.\ \bibinfo
  {pages} {609--629},\ \Eprint {http://arxiv.org/abs/arXiv:1412.8766}
  {arXiv:1412.8766} \BibitemShut {NoStop}%
\bibitem [{\citenamefont {Broadbent}\ and\ \citenamefont
  {Schaffner}(2016)}]{BS16}%
  \BibitemOpen
  \bibfield  {author} {\bibinfo {author} {\bibnamefont {Broadbent},
  \bibfnamefont {Anne}}, \ and\ \bibinfo {author} {\bibfnamefont {Christian}\
  \bibnamefont {Schaffner}}} (\bibinfo {year} {2016}),\ \bibfield  {title}
  {\enquote {\bibinfo {title} {Quantum cryptography beyond quantum key
  distribution},}\ }\href {\doibase 10.1007/s10623-015-0157-4} {\bibfield
  {journal} {\bibinfo  {journal} {Des. Codes Cryptogr.}\ }\textbf {\bibinfo
  {volume} {78}}~(\bibinfo {number} {1}),\ \bibinfo {pages} {351--382}},\
  \Eprint {http://arxiv.org/abs/arXiv:1510.06120} {arXiv:1510.06120}
  \BibitemShut {NoStop}%
\bibitem [{\citenamefont {Broadbent}\ and\ \citenamefont
  {Wainewright}(2016)}]{BW16}%
  \BibitemOpen
  \bibfield  {author} {\bibinfo {author} {\bibnamefont {Broadbent},
  \bibfnamefont {Anne}}, \ and\ \bibinfo {author} {\bibfnamefont {Evelyn}\
  \bibnamefont {Wainewright}}} (\bibinfo {year} {2016}),\ \bibfield  {title}
  {\enquote {\bibinfo {title} {Efficient simulation for quantum message
  authentication},}\ }in\ \href {\doibase 10.1007/978-3-319-49175-2_4} {\emph
  {\bibinfo {booktitle} {Proceedings of the 9th International Conference on
  Information Theoretic Security, ICITS 2016}}}\ (\bibinfo  {publisher}
  {Springer})\ pp.\ \bibinfo {pages} {72--91},\ \Eprint
  {http://arxiv.org/abs/arXiv:1607.03075} {arXiv:1607.03075} \BibitemShut
  {NoStop}%
\bibitem [{\citenamefont {Brunner}\ \emph {et~al.}(2014)\citenamefont
  {Brunner}, \citenamefont {Cavalcanti}, \citenamefont {Pironio}, \citenamefont
  {Scarani},\ and\ \citenamefont {Wehner}}]{BCPSW14}%
  \BibitemOpen
  \bibfield  {author} {\bibinfo {author} {\bibnamefont {Brunner}, \bibfnamefont
  {Nicolas}}, \bibinfo {author} {\bibfnamefont {Daniel}\ \bibnamefont
  {Cavalcanti}}, \bibinfo {author} {\bibfnamefont {Stefano}\ \bibnamefont
  {Pironio}}, \bibinfo {author} {\bibfnamefont {Valerio}\ \bibnamefont
  {Scarani}}, \ and\ \bibinfo {author} {\bibfnamefont {Stephanie}\ \bibnamefont
  {Wehner}}} (\bibinfo {year} {2014}),\ \bibfield  {title} {\enquote {\bibinfo
  {title} {Bell nonlocality},}\ }\href {\doibase 10.1103/RevModPhys.86.419}
  {\bibfield  {journal} {\bibinfo  {journal} {Rev. Mod. Phys.}\ }\textbf
  {\bibinfo {volume} {86}},\ \bibinfo {pages} {419--478}},\ \Eprint
  {http://arxiv.org/abs/arXiv:1303.2849} {arXiv:1303.2849} \BibitemShut
  {NoStop}%
\bibitem [{\citenamefont {Buhrman}\ \emph {et~al.}(2014)\citenamefont
  {Buhrman}, \citenamefont {Chandran}, \citenamefont {Fehr}, \citenamefont
  {Gelles}, \citenamefont {Goyal}, \citenamefont {Ostrovsky},\ and\
  \citenamefont {Schaffner}}]{BCFGGOS14}%
  \BibitemOpen
  \bibfield  {author} {\bibinfo {author} {\bibnamefont {Buhrman}, \bibfnamefont
  {Harry}}, \bibinfo {author} {\bibfnamefont {Nishanth}\ \bibnamefont
  {Chandran}}, \bibinfo {author} {\bibfnamefont {Serge}\ \bibnamefont {Fehr}},
  \bibinfo {author} {\bibfnamefont {Ran}\ \bibnamefont {Gelles}}, \bibinfo
  {author} {\bibfnamefont {Vipul}\ \bibnamefont {Goyal}}, \bibinfo {author}
  {\bibfnamefont {Rafail}\ \bibnamefont {Ostrovsky}}, \ and\ \bibinfo {author}
  {\bibfnamefont {Christian}\ \bibnamefont {Schaffner}}} (\bibinfo {year}
  {2014}),\ \bibfield  {title} {\enquote {\bibinfo {title} {Position-based
  quantum cryptography: Impossibility and constructions},}\ }\href {\doibase
  10.1137/130913687} {\bibfield  {journal} {\bibinfo  {journal} {SIAM J.
  Comput.}\ }\textbf {\bibinfo {volume} {43}}~(\bibinfo {number} {1}),\
  \bibinfo {pages} {150--178}},\ \bibinfo {note} {a preliminary version
  appeared at CRYPTO 2011},\ \Eprint {http://arxiv.org/abs/arXiv:1009.2490}
  {arXiv:1009.2490} \BibitemShut {NoStop}%
\bibitem [{\citenamefont {Calderbank}\ and\ \citenamefont {Shor}(1996)}]{CS96}%
  \BibitemOpen
  \bibfield  {author} {\bibinfo {author} {\bibnamefont {Calderbank},
  \bibfnamefont {A~R}}, \ and\ \bibinfo {author} {\bibfnamefont {Peter~W.}\
  \bibnamefont {Shor}}} (\bibinfo {year} {1996}),\ \bibfield  {title} {\enquote
  {\bibinfo {title} {Good quantum error-correcting codes exist},}\ }\href
  {\doibase 10.1103/PhysRevA.54.1098} {\bibfield  {journal} {\bibinfo
  {journal} {Phys. Rev. A}\ }\textbf {\bibinfo {volume} {54}},\ \bibinfo
  {pages} {1098--1105}}\BibitemShut {NoStop}%
\bibitem [{\citenamefont {Canetti}(2000)}]{Can00}%
  \BibitemOpen
  \bibfield  {author} {\bibinfo {author} {\bibnamefont {Canetti}, \bibfnamefont
  {Ran}}} (\bibinfo {year} {2000}),\ \bibfield  {title} {\enquote {\bibinfo
  {title} {Security and composition of multiparty cryptographic protocols},}\
  }\href {\doibase 10.1007/s001459910006} {\bibfield  {journal} {\bibinfo
  {journal} {J. Crypt.}\ }\textbf {\bibinfo {volume} {13}}~(\bibinfo {number}
  {1}),\ \bibinfo {pages} {143--202}},\ \bibinfo {note} {e-Print
  \href{http://eprint.iacr.org/1998/018}{IACR 1998/018}}\BibitemShut {NoStop}%
\bibitem [{\citenamefont {Canetti}(2001)}]{Can01}%
  \BibitemOpen
  \bibfield  {author} {\bibinfo {author} {\bibnamefont {Canetti}, \bibfnamefont
  {Ran}}} (\bibinfo {year} {2001}),\ \bibfield  {title} {\enquote {\bibinfo
  {title} {Universally composable security: A new paradigm for cryptographic
  protocols},}\ }in\ \href {\doibase 10.1109/SFCS.2001.959888} {\emph {\bibinfo
  {booktitle} {Proceedings of the 42nd Symposium on Foundations of Computer
  Science, FOCS~'01}}}\ (\bibinfo  {publisher} {IEEE})\ pp.\ \bibinfo {pages}
  {136--145}\BibitemShut {NoStop}%
\bibitem [{\citenamefont {Canetti}(2020)}]{Can20}%
  \BibitemOpen
  \bibfield  {author} {\bibinfo {author} {\bibnamefont {Canetti}, \bibfnamefont
  {Ran}}} (\bibinfo {year} {2020}),\ \href@noop {} {\enquote {\bibinfo {title}
  {Universally composable security: A new paradigm for cryptographic
  protocols},}\ }\bibinfo {howpublished} {e-Print
  \href{http://eprint.iacr.org/2000/067}{IACR 2000/067}},\ \bibinfo {note}
  {updated version of~\textcite{Can01}}\BibitemShut {NoStop}%
\bibitem [{\citenamefont {Canetti}\ \emph
  {et~al.}(2006{\natexlab{a}})\citenamefont {Canetti}, \citenamefont {Cheung},
  \citenamefont {Kaynar}, \citenamefont {Liskov}, \citenamefont {Lynch},
  \citenamefont {Pereira},\ and\ \citenamefont {Segala}}]{CCKLLPS06a}%
  \BibitemOpen
  \bibfield  {author} {\bibinfo {author} {\bibnamefont {Canetti}, \bibfnamefont
  {Ran}}, \bibinfo {author} {\bibfnamefont {Ling}\ \bibnamefont {Cheung}},
  \bibinfo {author} {\bibfnamefont {Dilsun~Kirli}\ \bibnamefont {Kaynar}},
  \bibinfo {author} {\bibfnamefont {Moses}\ \bibnamefont {Liskov}}, \bibinfo
  {author} {\bibfnamefont {Nancy~A.}\ \bibnamefont {Lynch}}, \bibinfo {author}
  {\bibfnamefont {Olivier}\ \bibnamefont {Pereira}}, \ and\ \bibinfo {author}
  {\bibfnamefont {Roberto}\ \bibnamefont {Segala}}} (\bibinfo {year}
  {2006}{\natexlab{a}}),\ \bibfield  {title} {\enquote {\bibinfo {title}
  {Task-structured probabilistic {I/O} automata},}\ }in\ \href {\doibase
  10.1109/WODES.2006.1678432} {\emph {\bibinfo {booktitle} {Proceedings of the
  8th International Workshop on Discrete Event Systems, {WODES} 2006}}}\
  (\bibinfo  {publisher} {IEEE})\ pp.\ \bibinfo {pages} {207--214},\ \bibinfo
  {note} {extended version available at
  \url{http://theory.csail.mit.edu/~lcheung/papers/task-PIOA-TR.pdf}}\BibitemShut
  {NoStop}%
\bibitem [{\citenamefont {Canetti}\ \emph
  {et~al.}(2006{\natexlab{b}})\citenamefont {Canetti}, \citenamefont {Cheung},
  \citenamefont {Kaynar}, \citenamefont {Liskov}, \citenamefont {Lynch},
  \citenamefont {Pereira},\ and\ \citenamefont {Segala}}]{CCKLLPS06b}%
  \BibitemOpen
  \bibfield  {author} {\bibinfo {author} {\bibnamefont {Canetti}, \bibfnamefont
  {Ran}}, \bibinfo {author} {\bibfnamefont {Ling}\ \bibnamefont {Cheung}},
  \bibinfo {author} {\bibfnamefont {Dilsun~Kirli}\ \bibnamefont {Kaynar}},
  \bibinfo {author} {\bibfnamefont {Moses}\ \bibnamefont {Liskov}}, \bibinfo
  {author} {\bibfnamefont {Nancy~A.}\ \bibnamefont {Lynch}}, \bibinfo {author}
  {\bibfnamefont {Olivier}\ \bibnamefont {Pereira}}, \ and\ \bibinfo {author}
  {\bibfnamefont {Roberto}\ \bibnamefont {Segala}}} (\bibinfo {year}
  {2006}{\natexlab{b}}),\ \bibfield  {title} {\enquote {\bibinfo {title}
  {Time-bounded task-{PIOAs}: {A} framework for analyzing security
  protocols},}\ }in\ \href {\doibase 10.1007/11864219_17} {\emph {\bibinfo
  {booktitle} {Proceedings of the 20th International Symposium on Distributed
  Computing, {DISC} 2006}}},\ pp.\ \bibinfo {pages} {238--253}\BibitemShut
  {NoStop}%
\bibitem [{\citenamefont {Canetti}\ \emph {et~al.}(2007)\citenamefont
  {Canetti}, \citenamefont {Dodis}, \citenamefont {Pass},\ and\ \citenamefont
  {Walfish}}]{CDPW07}%
  \BibitemOpen
  \bibfield  {author} {\bibinfo {author} {\bibnamefont {Canetti}, \bibfnamefont
  {Ran}}, \bibinfo {author} {\bibfnamefont {Yevgeniy}\ \bibnamefont {Dodis}},
  \bibinfo {author} {\bibfnamefont {Rafael}\ \bibnamefont {Pass}}, \ and\
  \bibinfo {author} {\bibfnamefont {Shabsi}\ \bibnamefont {Walfish}}} (\bibinfo
  {year} {2007}),\ \bibfield  {title} {\enquote {\bibinfo {title} {Universally
  composable security with global setup},}\ }in\ \href {\doibase
  10.1007/978-3-540-70936-7_4} {\emph {\bibinfo {booktitle} {Theory of
  Cryptography, Proceedings of TCC 2007}}},\ \bibinfo {series} {LNCS}, Vol.\
  \bibinfo {volume} {4392}\ (\bibinfo  {publisher} {Springer})\ pp.\ \bibinfo
  {pages} {61--85},\ \bibinfo {note} {e-Print
  \href{http://eprint.iacr.org/2006/432}{IACR 2006/432}}\BibitemShut {NoStop}%
\bibitem [{\citenamefont {Canetti}\ and\ \citenamefont
  {Fischlin}(2001)}]{CF01}%
  \BibitemOpen
  \bibfield  {author} {\bibinfo {author} {\bibnamefont {Canetti}, \bibfnamefont
  {Ran}}, \ and\ \bibinfo {author} {\bibfnamefont {Marc}\ \bibnamefont
  {Fischlin}}} (\bibinfo {year} {2001}),\ \bibfield  {title} {\enquote
  {\bibinfo {title} {Universally composable commitments},}\ }in\ \href
  {\doibase 10.1007/3-540-44647-8_2} {\emph {\bibinfo {booktitle} {Advances in
  Cryptology --- CRYPTO 2001}}},\ \bibinfo {editor} {edited by\ \bibinfo
  {editor} {\bibfnamefont {Joe}\ \bibnamefont {Kilian}}}\ (\bibinfo
  {publisher} {Springer})\ pp.\ \bibinfo {pages} {19--40},\ \bibinfo {note}
  {e-Print \href{http://eprint.iacr.org/2001/055}{IACR 2001/055}}\BibitemShut
  {NoStop}%
\bibitem [{\citenamefont {Canetti}\ \emph {et~al.}(2003)\citenamefont
  {Canetti}, \citenamefont {Krawczyk},\ and\ \citenamefont {Nielsen}}]{CKN03}%
  \BibitemOpen
  \bibfield  {author} {\bibinfo {author} {\bibnamefont {Canetti}, \bibfnamefont
  {Ran}}, \bibinfo {author} {\bibfnamefont {Hugo}\ \bibnamefont {Krawczyk}}, \
  and\ \bibinfo {author} {\bibfnamefont {Jesper~B.}\ \bibnamefont {Nielsen}}}
  (\bibinfo {year} {2003}),\ \bibfield  {title} {\enquote {\bibinfo {title}
  {Relaxing chosen-ciphertext security},}\ }in\ \href {\doibase
  10.1007/978-3-540-45146-4_33} {\emph {\bibinfo {booktitle} {Advances in
  Cryptology -- CRYPTO 2003}}},\ \bibinfo {editor} {edited by\ \bibinfo
  {editor} {\bibfnamefont {Dan}\ \bibnamefont {Boneh}}}\ (\bibinfo  {publisher}
  {Springer})\ pp.\ \bibinfo {pages} {565--582}\BibitemShut {NoStop}%
\bibitem [{\citenamefont {Canetti}\ \emph {et~al.}(2002)\citenamefont
  {Canetti}, \citenamefont {Lindell}, \citenamefont {Ostrovsky},\ and\
  \citenamefont {Sahai}}]{CLOS02}%
  \BibitemOpen
  \bibfield  {author} {\bibinfo {author} {\bibnamefont {Canetti}, \bibfnamefont
  {Ran}}, \bibinfo {author} {\bibfnamefont {Yehuda}\ \bibnamefont {Lindell}},
  \bibinfo {author} {\bibfnamefont {Rafail}\ \bibnamefont {Ostrovsky}}, \ and\
  \bibinfo {author} {\bibfnamefont {Amit}\ \bibnamefont {Sahai}}} (\bibinfo
  {year} {2002}),\ \bibfield  {title} {\enquote {\bibinfo {title} {Universally
  composable two-party and multi-party secure computation},}\ }in\ \href
  {\doibase 10.1145/509907.509980} {\emph {\bibinfo {booktitle} {Proceedings of
  the 34th Symposium on Theory of Computing, STOC~'02}}}\ (\bibinfo
  {publisher} {ACM})\ p.\ \bibinfo {pages} {494–503},\ \bibinfo {note}
  {e-Print \href{http://eprint.iacr.org/2002/140}{IACR 2002/140}}\BibitemShut
  {NoStop}%
\bibitem [{\citenamefont {Carter}\ and\ \citenamefont {Wegman}(1979)}]{CW79}%
  \BibitemOpen
  \bibfield  {author} {\bibinfo {author} {\bibnamefont {Carter}, \bibfnamefont
  {Larry}}, \ and\ \bibinfo {author} {\bibfnamefont {Mark~N.}\ \bibnamefont
  {Wegman}}} (\bibinfo {year} {1979}),\ \bibfield  {title} {\enquote {\bibinfo
  {title} {Universal classes of hash functions},}\ }\href {\doibase
  10.1016/0022-0000(79)90044-8} {\bibfield  {journal} {\bibinfo  {journal} {J.
  Comput. Syst. Sci.}\ }\textbf {\bibinfo {volume} {18}}~(\bibinfo {number}
  {2}),\ \bibinfo {pages} {143--154}}\BibitemShut {NoStop}%
\bibitem [{\citenamefont {Chandran}\ \emph {et~al.}(2009)\citenamefont
  {Chandran}, \citenamefont {Goyal}, \citenamefont {Moriarty},\ and\
  \citenamefont {Ostrovsky}}]{CGMO09}%
  \BibitemOpen
  \bibfield  {author} {\bibinfo {author} {\bibnamefont {Chandran},
  \bibfnamefont {Nishanth}}, \bibinfo {author} {\bibfnamefont {Vipul}\
  \bibnamefont {Goyal}}, \bibinfo {author} {\bibfnamefont {Ryan}\ \bibnamefont
  {Moriarty}}, \ and\ \bibinfo {author} {\bibfnamefont {Rafail}\ \bibnamefont
  {Ostrovsky}}} (\bibinfo {year} {2009}),\ \bibfield  {title} {\enquote
  {\bibinfo {title} {Position based cryptography},}\ }in\ \href {\doibase
  10.1007/978-3-642-03356-8_23} {\emph {\bibinfo {booktitle} {Advances in
  Cryptology -- CRYPTO 2009}}},\ \bibinfo {editor} {edited by\ \bibinfo
  {editor} {\bibfnamefont {Shai}\ \bibnamefont {Halevi}}}\ (\bibinfo
  {publisher} {Springer})\ pp.\ \bibinfo {pages} {391--407}\BibitemShut
  {NoStop}%
\bibitem [{\citenamefont {Chen}\ \emph {et~al.}(2017)\citenamefont {Chen},
  \citenamefont {Chung}, \citenamefont {Lai}, \citenamefont {Vadhan},\ and\
  \citenamefont {Wu}}]{CCLVW17}%
  \BibitemOpen
  \bibfield  {author} {\bibinfo {author} {\bibnamefont {Chen}, \bibfnamefont
  {Yi-Hsiu}}, \bibinfo {author} {\bibfnamefont {Kai-Min}\ \bibnamefont
  {Chung}}, \bibinfo {author} {\bibfnamefont {Ching-Yi}\ \bibnamefont {Lai}},
  \bibinfo {author} {\bibfnamefont {Salil~P.}\ \bibnamefont {Vadhan}}, \ and\
  \bibinfo {author} {\bibfnamefont {Xiaodi}\ \bibnamefont {Wu}}} (\bibinfo
  {year} {2017}),\ \href@noop {} {\enquote {\bibinfo {title} {Computational
  notions of quantum min-entropy},}\ }\bibinfo {howpublished} {e-print},\
  \Eprint {http://arxiv.org/abs/arXiv:1704.07309} {arXiv:1704.07309}
  \BibitemShut {NoStop}%
\bibitem [{\citenamefont {Childs}(2005)}]{Chi05}%
  \BibitemOpen
  \bibfield  {author} {\bibinfo {author} {\bibnamefont {Childs}, \bibfnamefont
  {Andrew~M}}} (\bibinfo {year} {2005}),\ \bibfield  {title} {\enquote
  {\bibinfo {title} {Secure assisted quantum computation},}\ }\href@noop {}
  {\bibfield  {journal} {\bibinfo  {journal} {Quantum Inf. Comput.}\ }\textbf
  {\bibinfo {volume} {5}}~(\bibinfo {number} {6}),\ \bibinfo {pages}
  {456--466}},\ \Eprint {http://arxiv.org/abs/arXiv:quant-ph/0111046}
  {arXiv:quant-ph/0111046} \BibitemShut {NoStop}%
\bibitem [{\citenamefont {Chiribella}\ \emph {et~al.}(2009)\citenamefont
  {Chiribella}, \citenamefont {D'Ariano},\ and\ \citenamefont
  {Perinotti}}]{CDP09}%
  \BibitemOpen
  \bibfield  {author} {\bibinfo {author} {\bibnamefont {Chiribella},
  \bibfnamefont {Giulio}}, \bibinfo {author} {\bibfnamefont {Giacomo~Mauro}\
  \bibnamefont {D'Ariano}}, \ and\ \bibinfo {author} {\bibfnamefont {Paolo}\
  \bibnamefont {Perinotti}}} (\bibinfo {year} {2009}),\ \bibfield  {title}
  {\enquote {\bibinfo {title} {Theoretical framework for quantum networks},}\
  }\href {\doibase 10.1103/PhysRevA.80.022339} {\bibfield  {journal} {\bibinfo
  {journal} {Phys. Rev. A}\ }\textbf {\bibinfo {volume} {80}},\ \bibinfo
  {pages} {022339}},\ \Eprint {http://arxiv.org/abs/arXiv:0904.4483}
  {arXiv:0904.4483} \BibitemShut {NoStop}%
\bibitem [{\citenamefont {Chitambar}\ and\ \citenamefont {Gour}(2019)}]{CG19}%
  \BibitemOpen
  \bibfield  {author} {\bibinfo {author} {\bibnamefont {Chitambar},
  \bibfnamefont {Eric}}, \ and\ \bibinfo {author} {\bibfnamefont {Gilad}\
  \bibnamefont {Gour}}} (\bibinfo {year} {2019}),\ \bibfield  {title} {\enquote
  {\bibinfo {title} {Quantum resource theories},}\ }\href {\doibase
  10.1103/RevModPhys.91.025001} {\bibfield  {journal} {\bibinfo  {journal}
  {Rev. Mod. Phys.}\ }\textbf {\bibinfo {volume} {91}},\ \bibinfo {pages}
  {025001}}\BibitemShut {NoStop}%
\bibitem [{\citenamefont {Christandl}\ \emph {et~al.}(2007)\citenamefont
  {Christandl}, \citenamefont {Ekert}, \citenamefont {Horodecki}, \citenamefont
  {Horodecki}, \citenamefont {Oppenheim},\ and\ \citenamefont
  {Renner}}]{christandl2007unifying}%
  \BibitemOpen
  \bibfield  {author} {\bibinfo {author} {\bibnamefont {Christandl},
  \bibfnamefont {Matthias}}, \bibinfo {author} {\bibfnamefont {Artur}\
  \bibnamefont {Ekert}}, \bibinfo {author} {\bibfnamefont {Micha{\l}}\
  \bibnamefont {Horodecki}}, \bibinfo {author} {\bibfnamefont {Pawe{\l}}\
  \bibnamefont {Horodecki}}, \bibinfo {author} {\bibfnamefont {Jonathan}\
  \bibnamefont {Oppenheim}}, \ and\ \bibinfo {author} {\bibfnamefont {Renato}\
  \bibnamefont {Renner}}} (\bibinfo {year} {2007}),\ \bibfield  {title}
  {\enquote {\bibinfo {title} {Unifying classical and quantum key
  distillation},}\ }in\ \href {\doibase 10.1007/978-3-540-70936-7_25} {\emph
  {\bibinfo {booktitle} {Theory of Cryptography Conference, Proceedings of
  {TCC} 2007}}},\ \bibinfo {series} {LNCS}, Vol.\ \bibinfo {volume} {4392},\
  \bibinfo {editor} {edited by\ \bibinfo {editor} {\bibfnamefont {Salil~P.}\
  \bibnamefont {Vadhan}}}\ (\bibinfo  {publisher} {Springer})\ pp.\ \bibinfo
  {pages} {456--478},\ \Eprint {http://arxiv.org/abs/arXiv:quant-ph/0608199}
  {arXiv:quant-ph/0608199} \BibitemShut {NoStop}%
\bibitem [{\citenamefont {Christandl}\ \emph {et~al.}(2009)\citenamefont
  {Christandl}, \citenamefont {K\"onig},\ and\ \citenamefont {Renner}}]{CKR09}%
  \BibitemOpen
  \bibfield  {author} {\bibinfo {author} {\bibnamefont {Christandl},
  \bibfnamefont {Matthias}}, \bibinfo {author} {\bibfnamefont {Robert}\
  \bibnamefont {K\"onig}}, \ and\ \bibinfo {author} {\bibfnamefont {Renato}\
  \bibnamefont {Renner}}} (\bibinfo {year} {2009}),\ \bibfield  {title}
  {\enquote {\bibinfo {title} {Postselection technique for quantum channels
  with applications to quantum cryptography},}\ }\href {\doibase
  10.1103/PhysRevLett.102.020504} {\bibfield  {journal} {\bibinfo  {journal}
  {Phys. Rev. Lett.}\ }\textbf {\bibinfo {volume} {102}},\ \bibinfo {pages}
  {020504}},\ \Eprint {http://arxiv.org/abs/arXiv:0809.3019} {arXiv:0809.3019}
  \BibitemShut {NoStop}%
\bibitem [{\citenamefont {Christandl}\ \emph {et~al.}(2004)\citenamefont
  {Christandl}, \citenamefont {Renner},\ and\ \citenamefont {Ekert}}]{CRE04}%
  \BibitemOpen
  \bibfield  {author} {\bibinfo {author} {\bibnamefont {Christandl},
  \bibfnamefont {Matthias}}, \bibinfo {author} {\bibfnamefont {Renato}\
  \bibnamefont {Renner}}, \ and\ \bibinfo {author} {\bibfnamefont {Artur}\
  \bibnamefont {Ekert}}} (\bibinfo {year} {2004}),\ \href@noop {} {\enquote
  {\bibinfo {title} {A generic security proof for quantum key distribution},}\
  }\bibinfo {howpublished} {e-Print},\ \Eprint
  {http://arxiv.org/abs/arXiv:quant-ph/0402131} {arXiv:quant-ph/0402131}
  \BibitemShut {NoStop}%
\bibitem [{\citenamefont {Christensen}\ \emph {et~al.}(2013)\citenamefont
  {Christensen}, \citenamefont {McCusker}, \citenamefont {Altepeter},
  \citenamefont {Calkins}, \citenamefont {Gerrits}, \citenamefont {Lita},
  \citenamefont {Miller}, \citenamefont {Shalm}, \citenamefont {Zhang},
  \citenamefont {Nam}, \citenamefont {Brunner}, \citenamefont {Lim},
  \citenamefont {Gisin},\ and\ \citenamefont {Kwiat}}]{Christensen}%
  \BibitemOpen
  \bibfield  {author} {\bibinfo {author} {\bibnamefont {Christensen},
  \bibfnamefont {Bradley~G}}, \bibinfo {author} {\bibfnamefont {Kevin~T.}\
  \bibnamefont {McCusker}}, \bibinfo {author} {\bibfnamefont {J.~B.}\
  \bibnamefont {Altepeter}}, \bibinfo {author} {\bibfnamefont {Brice}\
  \bibnamefont {Calkins}}, \bibinfo {author} {\bibfnamefont {Thomas}\
  \bibnamefont {Gerrits}}, \bibinfo {author} {\bibfnamefont {Adriana~E.}\
  \bibnamefont {Lita}}, \bibinfo {author} {\bibfnamefont {Aaron}\ \bibnamefont
  {Miller}}, \bibinfo {author} {\bibfnamefont {L.~K.}\ \bibnamefont {Shalm}},
  \bibinfo {author} {\bibfnamefont {Y.}~\bibnamefont {Zhang}}, \bibinfo
  {author} {\bibfnamefont {S.~W.}\ \bibnamefont {Nam}}, \bibinfo {author}
  {\bibfnamefont {Nicolas}\ \bibnamefont {Brunner}}, \bibinfo {author}
  {\bibfnamefont {Charles Ci~Wen}\ \bibnamefont {Lim}}, \bibinfo {author}
  {\bibfnamefont {Nicolas}\ \bibnamefont {Gisin}}, \ and\ \bibinfo {author}
  {\bibfnamefont {Paul~G.}\ \bibnamefont {Kwiat}}} (\bibinfo {year} {2013}),\
  \bibfield  {title} {\enquote {\bibinfo {title} {Detection-loophole-free test
  of quantum nonlocality, and applications},}\ }\href {\doibase
  10.1103/PhysRevLett.111.130406} {\bibfield  {journal} {\bibinfo  {journal}
  {Phys. Rev. Lett.}\ }\textbf {\bibinfo {volume} {111}},\ \bibinfo {pages}
  {130406}}\BibitemShut {NoStop}%
\bibitem [{\citenamefont {Chung}\ \emph
  {et~al.}(2014{\natexlab{a}})\citenamefont {Chung}, \citenamefont {Li},\ and\
  \citenamefont {Wu}}]{CLW14}%
  \BibitemOpen
  \bibfield  {author} {\bibinfo {author} {\bibnamefont {Chung}, \bibfnamefont
  {Kai-Min}}, \bibinfo {author} {\bibfnamefont {Xin}\ \bibnamefont {Li}}, \
  and\ \bibinfo {author} {\bibfnamefont {Xiaodi}\ \bibnamefont {Wu}}} (\bibinfo
  {year} {2014}{\natexlab{a}}),\ \href@noop {} {\enquote {\bibinfo {title}
  {Multi-source randomness extractors against quantum side information, and
  their applications},}\ }\bibinfo {howpublished} {e-Print},\ \Eprint
  {http://arxiv.org/abs/arXiv:1411.2315} {arXiv:1411.2315} \BibitemShut
  {NoStop}%
\bibitem [{\citenamefont {Chung}\ \emph
  {et~al.}(2014{\natexlab{b}})\citenamefont {Chung}, \citenamefont {Shi},\ and\
  \citenamefont {Wu}}]{CSW14}%
  \BibitemOpen
  \bibfield  {author} {\bibinfo {author} {\bibnamefont {Chung}, \bibfnamefont
  {Kai-Min}}, \bibinfo {author} {\bibfnamefont {Yaoyun}\ \bibnamefont {Shi}}, \
  and\ \bibinfo {author} {\bibfnamefont {Xiaodi}\ \bibnamefont {Wu}}} (\bibinfo
  {year} {2014}{\natexlab{b}}),\ \href@noop {} {\enquote {\bibinfo {title}
  {Physical randomness extractors: Generating random numbers with minimal
  assumptions},}\ }\bibinfo {howpublished} {e-Print},\ \Eprint
  {http://arxiv.org/abs/arXiv:1402.4797} {arXiv:1402.4797} \BibitemShut
  {NoStop}%
\bibitem [{\citenamefont {Clauser}\ \emph {et~al.}(1969)\citenamefont
  {Clauser}, \citenamefont {Horne}, \citenamefont {Shimony},\ and\
  \citenamefont {Holt}}]{CHSH69}%
  \BibitemOpen
  \bibfield  {author} {\bibinfo {author} {\bibnamefont {Clauser}, \bibfnamefont
  {John}}, \bibinfo {author} {\bibfnamefont {Michael}\ \bibnamefont {Horne}},
  \bibinfo {author} {\bibfnamefont {Abner}\ \bibnamefont {Shimony}}, \ and\
  \bibinfo {author} {\bibfnamefont {Richard}\ \bibnamefont {Holt}}} (\bibinfo
  {year} {1969}),\ \bibfield  {title} {\enquote {\bibinfo {title} {Proposed
  experiment to test local hidden-variable theories},}\ }\href {\doibase
  10.1103/PhysRevLett.23.880} {\bibfield  {journal} {\bibinfo  {journal} {Phys.
  Rev. Lett.}\ }\textbf {\bibinfo {volume} {23}}~(\bibinfo {number} {15}),\
  \bibinfo {pages} {880--884}}\BibitemShut {NoStop}%
\bibitem [{\citenamefont {Coffman}\ \emph {et~al.}(2000)\citenamefont
  {Coffman}, \citenamefont {Kundu},\ and\ \citenamefont
  {Wootters}}]{Coffman00}%
  \BibitemOpen
  \bibfield  {author} {\bibinfo {author} {\bibnamefont {Coffman}, \bibfnamefont
  {Valerie}}, \bibinfo {author} {\bibfnamefont {Joydip}\ \bibnamefont {Kundu}},
  \ and\ \bibinfo {author} {\bibfnamefont {William~K.}\ \bibnamefont
  {Wootters}}} (\bibinfo {year} {2000}),\ \bibfield  {title} {\enquote
  {\bibinfo {title} {Distributed entanglement},}\ }\href {\doibase
  10.1103/PhysRevA.61.052306} {\bibfield  {journal} {\bibinfo  {journal} {Phys.
  Rev. A}\ }\textbf {\bibinfo {volume} {61}},\ \bibinfo {pages}
  {052306}}\BibitemShut {NoStop}%
\bibitem [{\citenamefont {Colbeck}(2006)}]{Col06}%
  \BibitemOpen
  \bibfield  {author} {\bibinfo {author} {\bibnamefont {Colbeck}, \bibfnamefont
  {Roger}}} (\bibinfo {year} {2006}),\ \emph {\bibinfo {title} {Quantum And
  Relativistic Protocols For Secure Multi-Party Computation}},\ \href@noop {}
  {Ph.D. thesis}\ (\bibinfo  {school} {University of Cambridge}),\ \Eprint
  {http://arxiv.org/abs/arXiv:0911.3814} {arXiv:0911.3814} \BibitemShut
  {NoStop}%
\bibitem [{\citenamefont {Colbeck}\ and\ \citenamefont {Renner}(2011)}]{CR11}%
  \BibitemOpen
  \bibfield  {author} {\bibinfo {author} {\bibnamefont {Colbeck}, \bibfnamefont
  {Roger}}, \ and\ \bibinfo {author} {\bibfnamefont {Renato}\ \bibnamefont
  {Renner}}} (\bibinfo {year} {2011}),\ \bibfield  {title} {\enquote {\bibinfo
  {title} {No extension of quantum theory can have improved predictive
  power},}\ }\href {\doibase 10.1038/ncomms1416} {\bibfield  {journal}
  {\bibinfo  {journal} {Nat. Commun.}\ }\textbf {\bibinfo {volume} {2}},\
  \bibinfo {pages} {411}},\ \Eprint {http://arxiv.org/abs/arXiv:1005.5173}
  {arXiv:1005.5173} \BibitemShut {NoStop}%
\bibitem [{\citenamefont {Colbeck}\ and\ \citenamefont {Renner}(2012)}]{CR12}%
  \BibitemOpen
  \bibfield  {author} {\bibinfo {author} {\bibnamefont {Colbeck}, \bibfnamefont
  {Roger}}, \ and\ \bibinfo {author} {\bibfnamefont {Renato}\ \bibnamefont
  {Renner}}} (\bibinfo {year} {2012}),\ \bibfield  {title} {\enquote {\bibinfo
  {title} {Free randomness can be amplified},}\ }\href {\doibase
  10.1038/nphys2300} {\bibfield  {journal} {\bibinfo  {journal} {Nat. Phys.}\
  }\textbf {\bibinfo {volume} {8}}~(\bibinfo {number} {6}),\ \bibinfo {pages}
  {450--454}},\ \Eprint {http://arxiv.org/abs/arXiv:1105.3195}
  {arXiv:1105.3195} \BibitemShut {NoStop}%
\bibitem [{\citenamefont {Coles}\ \emph {et~al.}(2017)\citenamefont {Coles},
  \citenamefont {Berta}, \citenamefont {Tomamichel},\ and\ \citenamefont
  {Wehner}}]{Coles}%
  \BibitemOpen
  \bibfield  {author} {\bibinfo {author} {\bibnamefont {Coles}, \bibfnamefont
  {Patrick~J}}, \bibinfo {author} {\bibfnamefont {Mario}\ \bibnamefont
  {Berta}}, \bibinfo {author} {\bibfnamefont {Marco}\ \bibnamefont
  {Tomamichel}}, \ and\ \bibinfo {author} {\bibfnamefont {Stephanie}\
  \bibnamefont {Wehner}}} (\bibinfo {year} {2017}),\ \bibfield  {title}
  {\enquote {\bibinfo {title} {Entropic uncertainty relations and their
  applications},}\ }\href {\doibase 10.1103/RevModPhys.89.015002} {\bibfield
  {journal} {\bibinfo  {journal} {Rev. Mod. Phys.}\ }\textbf {\bibinfo {volume}
  {89}},\ \bibinfo {pages} {015002}}\BibitemShut {NoStop}%
\bibitem [{\citenamefont {Conway}\ and\ \citenamefont
  {Kochen}(2006)}]{Conway2006}%
  \BibitemOpen
  \bibfield  {author} {\bibinfo {author} {\bibnamefont {Conway}, \bibfnamefont
  {John}}, \ and\ \bibinfo {author} {\bibfnamefont {Simon}\ \bibnamefont
  {Kochen}}} (\bibinfo {year} {2006}),\ \bibfield  {title} {\enquote {\bibinfo
  {title} {The free will theorem},}\ }\href {\doibase
  10.1007/s10701-006-9068-6} {\bibfield  {journal} {\bibinfo  {journal} {Found.
  Phys.}\ }\textbf {\bibinfo {volume} {36}}~(\bibinfo {number} {10}),\ \bibinfo
  {pages} {1441--1473}}\BibitemShut {NoStop}%
\bibitem [{\citenamefont {Coretti}\ \emph {et~al.}(2013)\citenamefont
  {Coretti}, \citenamefont {Maurer},\ and\ \citenamefont {Tackmann}}]{CMT13}%
  \BibitemOpen
  \bibfield  {author} {\bibinfo {author} {\bibnamefont {Coretti}, \bibfnamefont
  {Sandro}}, \bibinfo {author} {\bibfnamefont {Ueli}\ \bibnamefont {Maurer}}, \
  and\ \bibinfo {author} {\bibfnamefont {Bj\"orn}\ \bibnamefont {Tackmann}}}
  (\bibinfo {year} {2013}),\ \bibfield  {title} {\enquote {\bibinfo {title}
  {Constructing confidential channels from authenticated channels---public-key
  encryption revisited},}\ }in\ \href {\doibase 10.1007/978-3-642-42033-7_8}
  {\emph {\bibinfo {booktitle} {Advances in Cryptology -- ASIACRYPT 2013}}},\
  \bibinfo {series} {LNCS}, Vol.\ \bibinfo {volume} {8269}\ (\bibinfo
  {publisher} {Springer})\ pp.\ \bibinfo {pages} {134--153},\ \bibinfo {note}
  {e-Print \href{http://eprint.iacr.org/2013/719}{IACR 2013/719}}\BibitemShut
  {NoStop}%
\bibitem [{\citenamefont {Cover}\ and\ \citenamefont {Thomas}(2012)}]{CT12}%
  \BibitemOpen
  \bibfield  {author} {\bibinfo {author} {\bibnamefont {Cover}, \bibfnamefont
  {Thomas~M}}, \ and\ \bibinfo {author} {\bibfnamefont {Joy~A.}\ \bibnamefont
  {Thomas}}} (\bibinfo {year} {2012}),\ \href@noop {} {\emph {\bibinfo {title}
  {Elements of information theory}}}\ (\bibinfo  {publisher} {John Wiley \&
  Sons})\BibitemShut {NoStop}%
\bibitem [{\citenamefont {Cramer}\ \emph {et~al.}(2015)\citenamefont {Cramer},
  \citenamefont {Damg{\aa}rd},\ and\ \citenamefont {Nielsen}}]{CDN15}%
  \BibitemOpen
  \bibfield  {author} {\bibinfo {author} {\bibnamefont {Cramer}, \bibfnamefont
  {Ronald}}, \bibinfo {author} {\bibfnamefont {Ivan~B.}\ \bibnamefont
  {Damg{\aa}rd}}, \ and\ \bibinfo {author} {\bibfnamefont {Jesper~B.}\
  \bibnamefont {Nielsen}}} (\bibinfo {year} {2015}),\ \href {\doibase
  10.1017/CBO9781107337756} {\emph {\bibinfo {title} {Secure Multiparty
  Computation and Secret Sharing}}}\ (\bibinfo  {publisher} {Cambridge
  University Press})\BibitemShut {NoStop}%
\bibitem [{\citenamefont {Cr\'{e}peau}\ \emph {et~al.}(2002)\citenamefont
  {Cr\'{e}peau}, \citenamefont {Gottesman},\ and\ \citenamefont
  {Smith}}]{CGS02}%
  \BibitemOpen
  \bibfield  {author} {\bibinfo {author} {\bibnamefont {Cr\'{e}peau},
  \bibfnamefont {Claude}}, \bibinfo {author} {\bibfnamefont {Daniel}\
  \bibnamefont {Gottesman}}, \ and\ \bibinfo {author} {\bibfnamefont {Adam}\
  \bibnamefont {Smith}}} (\bibinfo {year} {2002}),\ \bibfield  {title}
  {\enquote {\bibinfo {title} {Secure multi-party quantum computation},}\ }in\
  \href {\doibase 10.1145/509907.510000} {\emph {\bibinfo {booktitle}
  {Proceedings of the 34th Symposium on Theory of Computing, STOC~'02}}}\
  (\bibinfo  {publisher} {ACM})\ pp.\ \bibinfo {pages} {643--652},\ \Eprint
  {http://arxiv.org/abs/arXiv:quant-ph/0206138} {arXiv:quant-ph/0206138}
  \BibitemShut {NoStop}%
\bibitem [{\citenamefont {Cr\'e{}peau}\ and\ \citenamefont
  {Kilian}(1988)}]{CK88}%
  \BibitemOpen
  \bibfield  {author} {\bibinfo {author} {\bibnamefont {Cr\'e{}peau},
  \bibfnamefont {Claude}}, \ and\ \bibinfo {author} {\bibfnamefont {Joe}\
  \bibnamefont {Kilian}}} (\bibinfo {year} {1988}),\ \bibfield  {title}
  {\enquote {\bibinfo {title} {Achieving oblivious transfer using weakened
  security assumptions},}\ }in\ \href {\doibase 10.1109/SFCS.1988.21920} {\emph
  {\bibinfo {booktitle} {Proceedings of the 29th Symposium on Foundations of
  Computer Science, FOCS~'88}}},\ pp.\ \bibinfo {pages} {42--52}\BibitemShut
  {NoStop}%
\bibitem [{\citenamefont {Curty}\ \emph {et~al.}(2014)\citenamefont {Curty},
  \citenamefont {Xu}, \citenamefont {Cui}, \citenamefont {Lim}, \citenamefont
  {Tamaki},\ and\ \citenamefont {Lo}}]{CXCLTL14}%
  \BibitemOpen
  \bibfield  {author} {\bibinfo {author} {\bibnamefont {Curty}, \bibfnamefont
  {Marcos}}, \bibinfo {author} {\bibfnamefont {Feihu}\ \bibnamefont {Xu}},
  \bibinfo {author} {\bibfnamefont {Wei}\ \bibnamefont {Cui}}, \bibinfo
  {author} {\bibfnamefont {Charles Ci~Wen}\ \bibnamefont {Lim}}, \bibinfo
  {author} {\bibfnamefont {Kiyoshi}\ \bibnamefont {Tamaki}}, \ and\ \bibinfo
  {author} {\bibfnamefont {Hoi-Kwong}\ \bibnamefont {Lo}}} (\bibinfo {year}
  {2014}),\ \bibfield  {title} {\enquote {\bibinfo {title} {Finite-key analysis
  for measurement-device-independent quantum key distribution},}\ }\href
  {\doibase 10.1038/ncomms4732} {\bibfield  {journal} {\bibinfo  {journal}
  {Nat. Commun.}\ }\textbf {\bibinfo {volume} {5}},\ \bibinfo {pages} {3732}},\
  \Eprint {http://arxiv.org/abs/arXiv:1307.1081} {arXiv:1307.1081} \BibitemShut
  {NoStop}%
\bibitem [{\citenamefont {Damg{\aa}rd}\ \emph {et~al.}(2007)\citenamefont
  {Damg{\aa}rd}, \citenamefont {Fehr}, \citenamefont {Salvail},\ and\
  \citenamefont {Schaffner}}]{DFSS07}%
  \BibitemOpen
  \bibfield  {author} {\bibinfo {author} {\bibnamefont {Damg{\aa}rd},
  \bibfnamefont {Ivan~B}}, \bibinfo {author} {\bibfnamefont {Serge}\
  \bibnamefont {Fehr}}, \bibinfo {author} {\bibfnamefont {Louis}\ \bibnamefont
  {Salvail}}, \ and\ \bibinfo {author} {\bibfnamefont {Christian}\ \bibnamefont
  {Schaffner}}} (\bibinfo {year} {2007}),\ \bibfield  {title} {\enquote
  {\bibinfo {title} {Secure identification and {QKD} in the
  bounded-quantum-storage model},}\ }in\ \href {\doibase
  10.1007/978-3-540-74143-5_19} {\emph {\bibinfo {booktitle} {Advances in
  Cryptology -- CRYPTO 2007}}},\ \bibinfo {editor} {edited by\ \bibinfo
  {editor} {\bibfnamefont {Alfred}\ \bibnamefont {Menezes}}}\ (\bibinfo
  {publisher} {Springer})\ pp.\ \bibinfo {pages} {342--359}\BibitemShut
  {NoStop}%
\bibitem [{\citenamefont {Damg{\aa}rd}\ \emph {et~al.}(2008)\citenamefont
  {Damg{\aa}rd}, \citenamefont {Fehr}, \citenamefont {Salvail},\ and\
  \citenamefont {Schaffner}}]{DFSS08}%
  \BibitemOpen
  \bibfield  {author} {\bibinfo {author} {\bibnamefont {Damg{\aa}rd},
  \bibfnamefont {Ivan~B}}, \bibinfo {author} {\bibfnamefont {Serge}\
  \bibnamefont {Fehr}}, \bibinfo {author} {\bibfnamefont {Louis}\ \bibnamefont
  {Salvail}}, \ and\ \bibinfo {author} {\bibfnamefont {Christian}\ \bibnamefont
  {Schaffner}}} (\bibinfo {year} {2008}),\ \bibfield  {title} {\enquote
  {\bibinfo {title} {Cryptography in the bounded-quantum-storage model},}\
  }\href {\doibase 10.1137/060651343} {\bibfield  {journal} {\bibinfo
  {journal} {SIAM J. Comput.}\ }\textbf {\bibinfo {volume} {37}}~(\bibinfo
  {number} {6}),\ \bibinfo {pages} {1865--1890}},\ \bibinfo {note} {a
  preliminary version appeared at FOCS '05},\ \Eprint
  {http://arxiv.org/abs/arXiv:quant-ph/0508222} {arXiv:quant-ph/0508222}
  \BibitemShut {NoStop}%
\bibitem [{\citenamefont {De}\ \emph {et~al.}(2012)\citenamefont {De},
  \citenamefont {Portmann}, \citenamefont {Vidick},\ and\ \citenamefont
  {Renner}}]{DPVR12}%
  \BibitemOpen
  \bibfield  {author} {\bibinfo {author} {\bibnamefont {De}, \bibfnamefont
  {Anindya}}, \bibinfo {author} {\bibfnamefont {Christopher}\ \bibnamefont
  {Portmann}}, \bibinfo {author} {\bibfnamefont {Thomas}\ \bibnamefont
  {Vidick}}, \ and\ \bibinfo {author} {\bibfnamefont {Renato}\ \bibnamefont
  {Renner}}} (\bibinfo {year} {2012}),\ \bibfield  {title} {\enquote {\bibinfo
  {title} {Trevisan's extractor in the presence of quantum side information},}\
  }\href {\doibase 10.1137/100813683} {\bibfield  {journal} {\bibinfo
  {journal} {SIAM J. Comput.}\ }\textbf {\bibinfo {volume} {41}}~(\bibinfo
  {number} {4}),\ \bibinfo {pages} {915--940}},\ \Eprint
  {http://arxiv.org/abs/arXiv:0912.5514} {arXiv:0912.5514} \BibitemShut
  {NoStop}%
\bibitem [{\citenamefont {Demay}\ and\ \citenamefont {Maurer}(2013)}]{DM13}%
  \BibitemOpen
  \bibfield  {author} {\bibinfo {author} {\bibnamefont {Demay}, \bibfnamefont
  {Gregory}}, \ and\ \bibinfo {author} {\bibfnamefont {Ueli}\ \bibnamefont
  {Maurer}}} (\bibinfo {year} {2013}),\ \bibfield  {title} {\enquote {\bibinfo
  {title} {Unfair coin tossing},}\ }in\ \href {\doibase
  10.1109/ISIT.2013.6620488} {\emph {\bibinfo {booktitle} {Proceedings of the
  2013 IEEE International Symposium on Information Theory, ISIT 2013}}}\
  (\bibinfo  {publisher} {IEEE})\ pp.\ \bibinfo {pages}
  {1556--1560}\BibitemShut {NoStop}%
\bibitem [{\citenamefont {Devetak}\ and\ \citenamefont {Winter}(2005)}]{DW05}%
  \BibitemOpen
  \bibfield  {author} {\bibinfo {author} {\bibnamefont {Devetak}, \bibfnamefont
  {Igor}}, \ and\ \bibinfo {author} {\bibfnamefont {Andreas}\ \bibnamefont
  {Winter}}} (\bibinfo {year} {2005}),\ \bibfield  {title} {\enquote {\bibinfo
  {title} {Distillation of secret key and entanglement from quantum states},}\
  }\href {\doibase 10.1098/rspa.2004.1372} {\bibfield  {journal} {\bibinfo
  {journal} {Proc. R. Soc. London, Ser. A}\ }\textbf {\bibinfo {volume}
  {461}}~(\bibinfo {number} {2053}),\ \bibinfo {pages} {207--235}},\ \Eprint
  {http://arxiv.org/abs/arXiv:quant-ph/0306078} {arXiv:quant-ph/0306078}
  \BibitemShut {NoStop}%
\bibitem [{\citenamefont {Dickinson}\ and\ \citenamefont {Nayak}(2006)}]{DN06}%
  \BibitemOpen
  \bibfield  {author} {\bibinfo {author} {\bibnamefont {Dickinson},
  \bibfnamefont {Paul}}, \ and\ \bibinfo {author} {\bibfnamefont {Ashwin}\
  \bibnamefont {Nayak}}} (\bibinfo {year} {2006}),\ \bibfield  {title}
  {\enquote {\bibinfo {title} {Approximate randomization of quantum states with
  fewer bits of key},}\ }in\ \href {\doibase 10.1063/1.2400876} {\emph
  {\bibinfo {booktitle} {AIP Conference Proceedings}}},\ Vol.\ \bibinfo
  {volume} {864},\ pp.\ \bibinfo {pages} {18--36},\ \Eprint
  {http://arxiv.org/abs/arXiv:quant-ph/0611033} {arXiv:quant-ph/0611033}
  \BibitemShut {NoStop}%
\bibitem [{\citenamefont {DiVincenzo}\ \emph {et~al.}(2004)\citenamefont
  {DiVincenzo}, \citenamefont {Horodecki}, \citenamefont {Leung}, \citenamefont
  {Smolin},\ and\ \citenamefont {Terhal}}]{DHLST04}%
  \BibitemOpen
  \bibfield  {author} {\bibinfo {author} {\bibnamefont {DiVincenzo},
  \bibfnamefont {David}}, \bibinfo {author} {\bibfnamefont {Micha\l{}}\
  \bibnamefont {Horodecki}}, \bibinfo {author} {\bibfnamefont {Debbie}\
  \bibnamefont {Leung}}, \bibinfo {author} {\bibfnamefont {John}\ \bibnamefont
  {Smolin}}, \ and\ \bibinfo {author} {\bibfnamefont {Barbara}\ \bibnamefont
  {Terhal}}} (\bibinfo {year} {2004}),\ \bibfield  {title} {\enquote {\bibinfo
  {title} {Locking classical correlation in quantum states},}\ }\href@noop {}
  {\bibfield  {journal} {\bibinfo  {journal} {Phys. Rev. Lett.}\ }\textbf
  {\bibinfo {volume} {92}},\ \bibinfo {pages} {067902}},\ \Eprint
  {http://arxiv.org/abs/arXiv:quant-ph/0303088} {arXiv:quant-ph/0303088}
  \BibitemShut {NoStop}%
\bibitem [{\citenamefont {Dodis}\ and\ \citenamefont {Wichs}(2009)}]{DW09}%
  \BibitemOpen
  \bibfield  {author} {\bibinfo {author} {\bibnamefont {Dodis}, \bibfnamefont
  {Yevgeniy}}, \ and\ \bibinfo {author} {\bibfnamefont {Daniel}\ \bibnamefont
  {Wichs}}} (\bibinfo {year} {2009}),\ \bibfield  {title} {\enquote {\bibinfo
  {title} {Non-malleable extractors and symmetric key cryptography from weak
  secrets},}\ }in\ \href {\doibase 10.1145/1536414.1536496} {\emph {\bibinfo
  {booktitle} {Proceedings of the 41st Symposium on Theory of Computing,
  STOC~'09}}}\ (\bibinfo  {publisher} {ACM})\ pp.\ \bibinfo {pages}
  {601--610},\ \bibinfo {note} {e-Print
  \href{http://eprint.iacr.org/2008/503}{IACR 2008/503}}\BibitemShut {NoStop}%
\bibitem [{\citenamefont {Dulek}\ \emph {et~al.}(2020)\citenamefont {Dulek},
  \citenamefont {Grilo}, \citenamefont {Jeffery}, \citenamefont {Majenz},\ and\
  \citenamefont {Schaffner}}]{DGJMS20}%
  \BibitemOpen
  \bibfield  {author} {\bibinfo {author} {\bibnamefont {Dulek}, \bibfnamefont
  {Yfke}}, \bibinfo {author} {\bibfnamefont {Alex~B.}\ \bibnamefont {Grilo}},
  \bibinfo {author} {\bibfnamefont {Stacey}\ \bibnamefont {Jeffery}}, \bibinfo
  {author} {\bibfnamefont {Christian}\ \bibnamefont {Majenz}}, \ and\ \bibinfo
  {author} {\bibfnamefont {Christian}\ \bibnamefont {Schaffner}}} (\bibinfo
  {year} {2020}),\ \bibfield  {title} {\enquote {\bibinfo {title} {Secure
  multi-party quantum computation with a dishonest majority},}\ }in\ \href
  {\doibase 10.1007/978-3-030-45727-3_25} {\emph {\bibinfo {booktitle}
  {Advances in Cryptology -- EUROCRYPT 2020}}},\ \bibinfo {editor} {edited by\
  \bibinfo {editor} {\bibfnamefont {Anne}\ \bibnamefont {Canteaut}}\ and\
  \bibinfo {editor} {\bibfnamefont {Yuval}\ \bibnamefont {Ishai}}}\ (\bibinfo
  {publisher} {Springer})\ pp.\ \bibinfo {pages} {729--758},\ \Eprint
  {http://arxiv.org/abs/arXiv:1909.13770} {arXiv:1909.13770} \BibitemShut
  {NoStop}%
\bibitem [{\citenamefont {Dunjko}\ \emph {et~al.}(2014)\citenamefont {Dunjko},
  \citenamefont {Fitzsimons}, \citenamefont {Portmann},\ and\ \citenamefont
  {Renner}}]{DFPR14}%
  \BibitemOpen
  \bibfield  {author} {\bibinfo {author} {\bibnamefont {Dunjko}, \bibfnamefont
  {Vedran}}, \bibinfo {author} {\bibfnamefont {Joseph}\ \bibnamefont
  {Fitzsimons}}, \bibinfo {author} {\bibfnamefont {Christopher}\ \bibnamefont
  {Portmann}}, \ and\ \bibinfo {author} {\bibfnamefont {Renato}\ \bibnamefont
  {Renner}}} (\bibinfo {year} {2014}),\ \bibfield  {title} {\enquote {\bibinfo
  {title} {Composable security of delegated quantum computation},}\ }in\ \href
  {\doibase 10.1007/978-3-662-45608-8_22} {\emph {\bibinfo {booktitle}
  {Advances in Cryptology -- ASIACRYPT 2014, Proceedings, Part II}}},\ \bibinfo
  {series} {LNCS}, Vol.\ \bibinfo {volume} {8874}\ (\bibinfo  {publisher}
  {Springer})\ pp.\ \bibinfo {pages} {406--425},\ \Eprint
  {http://arxiv.org/abs/arXiv:1301.3662} {arXiv:1301.3662} \BibitemShut
  {NoStop}%
\bibitem [{\citenamefont {Dunjko}\ and\ \citenamefont {Kashefi}(2016)}]{DK16}%
  \BibitemOpen
  \bibfield  {author} {\bibinfo {author} {\bibnamefont {Dunjko}, \bibfnamefont
  {Vedran}}, \ and\ \bibinfo {author} {\bibfnamefont {Elham}\ \bibnamefont
  {Kashefi}}} (\bibinfo {year} {2016}),\ \href@noop {} {\enquote {\bibinfo
  {title} {Blind quantum computing with two almost identical states},}\
  }\bibinfo {howpublished} {e-Print},\ \Eprint
  {http://arxiv.org/abs/arXiv:1604.01586} {arXiv:1604.01586} \BibitemShut
  {NoStop}%
\bibitem [{\citenamefont {Dupuis}\ \emph {et~al.}(2020)\citenamefont {Dupuis},
  \citenamefont {Fawzi},\ and\ \citenamefont {Renner}}]{DFR20}%
  \BibitemOpen
  \bibfield  {author} {\bibinfo {author} {\bibnamefont {Dupuis}, \bibfnamefont
  {Fr{\'e}d{\'e}ric}}, \bibinfo {author} {\bibfnamefont {Omar}\ \bibnamefont
  {Fawzi}}, \ and\ \bibinfo {author} {\bibfnamefont {Renato}\ \bibnamefont
  {Renner}}} (\bibinfo {year} {2020}),\ \bibfield  {title} {\enquote {\bibinfo
  {title} {Entropy accumulation},}\ }\href {\doibase
  10.1007/s00220-020-03839-5} {\bibfield  {journal} {\bibinfo  {journal}
  {Commun. Math. Phys.}\ }\textbf {\bibinfo {volume} {379}}~(\bibinfo {number}
  {3}),\ \bibinfo {pages} {867--913}},\ \Eprint
  {http://arxiv.org/abs/arXiv:1607.01796} {arXiv:1607.01796} \BibitemShut
  {NoStop}%
\bibitem [{\citenamefont {Dupuis}\ \emph {et~al.}(2012)\citenamefont {Dupuis},
  \citenamefont {Nielsen},\ and\ \citenamefont {Salvail}}]{DNS12}%
  \BibitemOpen
  \bibfield  {author} {\bibinfo {author} {\bibnamefont {Dupuis}, \bibfnamefont
  {Fr{\'e}d{\'e}ric}}, \bibinfo {author} {\bibfnamefont {Jesper~B.}\
  \bibnamefont {Nielsen}}, \ and\ \bibinfo {author} {\bibfnamefont {Louis}\
  \bibnamefont {Salvail}}} (\bibinfo {year} {2012}),\ \bibfield  {title}
  {\enquote {\bibinfo {title} {Actively secure two-party evaluation of any
  quantum operation},}\ }in\ \href {\doibase 10.1007/978-3-642-32009-5_46}
  {\emph {\bibinfo {booktitle} {Advances in Cryptology -- CRYPTO 2012}}},\
  \bibinfo {series} {LNCS}, Vol.\ \bibinfo {volume} {7417},\ \bibinfo {editor}
  {edited by\ \bibinfo {editor} {\bibfnamefont {Reihaneh}\ \bibnamefont
  {Safavi-Naini}}\ and\ \bibinfo {editor} {\bibfnamefont {Ran}\ \bibnamefont
  {Canetti}}}\ (\bibinfo  {publisher} {Springer})\ pp.\ \bibinfo {pages}
  {794--811},\ \bibinfo {note} {e-Print
  \href{http://eprint.iacr.org/2012/304}{IACR 2012/304}}\BibitemShut {NoStop}%
\bibitem [{\citenamefont {Einstein}\ \emph {et~al.}(1935)\citenamefont
  {Einstein}, \citenamefont {Podolsky},\ and\ \citenamefont {Rosen}}]{EPR35}%
  \BibitemOpen
  \bibfield  {author} {\bibinfo {author} {\bibnamefont {Einstein},
  \bibfnamefont {Albert}}, \bibinfo {author} {\bibfnamefont {Boris}\
  \bibnamefont {Podolsky}}, \ and\ \bibinfo {author} {\bibfnamefont {Nathan}\
  \bibnamefont {Rosen}}} (\bibinfo {year} {1935}),\ \bibfield  {title}
  {\enquote {\bibinfo {title} {Can quantum-mechanical description of physical
  reality be considered complete?}}\ }\href {\doibase 10.1103/PhysRev.47.777}
  {\bibfield  {journal} {\bibinfo  {journal} {Phys. Rev.}\ }\textbf {\bibinfo
  {volume} {47}},\ \bibinfo {pages} {777--780}}\BibitemShut {NoStop}%
\bibitem [{\citenamefont {Ekert}(1991)}]{Eke91}%
  \BibitemOpen
  \bibfield  {author} {\bibinfo {author} {\bibnamefont {Ekert}, \bibfnamefont
  {Artur}}} (\bibinfo {year} {1991}),\ \bibfield  {title} {\enquote {\bibinfo
  {title} {Quantum cryptography based on {Bell}'s theorem},}\ }\href {\doibase
  10.1103/PhysRevLett.67.661} {\bibfield  {journal} {\bibinfo  {journal} {Phys.
  Rev. Lett.}\ }\textbf {\bibinfo {volume} {67}},\ \bibinfo {pages}
  {661--663}}\BibitemShut {NoStop}%
\bibitem [{\citenamefont {Ekert}\ and\ \citenamefont {Renner}(2014)}]{ER14}%
  \BibitemOpen
  \bibfield  {author} {\bibinfo {author} {\bibnamefont {Ekert}, \bibfnamefont
  {Artur}}, \ and\ \bibinfo {author} {\bibfnamefont {Renato}\ \bibnamefont
  {Renner}}} (\bibinfo {year} {2014}),\ \bibfield  {title} {\enquote {\bibinfo
  {title} {The ultimate physical limits of privacy},}\ }\href {\doibase
  10.1038/nature13132} {\bibfield  {journal} {\bibinfo  {journal} {Nature}\
  }\textbf {\bibinfo {volume} {507}}~(\bibinfo {number} {7493}),\ \bibinfo
  {pages} {443--447}},\ \bibinfo {note} {perspectives}\BibitemShut {NoStop}%
\bibitem [{\citenamefont {Elkouss}\ \emph {et~al.}(2009)\citenamefont
  {Elkouss}, \citenamefont {Leverrier}, \citenamefont {Alleaume},\ and\
  \citenamefont {Boutros}}]{ELAB09}%
  \BibitemOpen
  \bibfield  {author} {\bibinfo {author} {\bibnamefont {Elkouss}, \bibfnamefont
  {David}}, \bibinfo {author} {\bibfnamefont {Anthony}\ \bibnamefont
  {Leverrier}}, \bibinfo {author} {\bibfnamefont {Romain}\ \bibnamefont
  {Alleaume}}, \ and\ \bibinfo {author} {\bibfnamefont {Joseph~J.}\
  \bibnamefont {Boutros}}} (\bibinfo {year} {2009}),\ \bibfield  {title}
  {\enquote {\bibinfo {title} {Efficient reconciliation protocol for
  discrete-variable quantum key distribution},}\ }in\ \href {\doibase
  10.1109/ISIT.2009.5205475} {\emph {\bibinfo {booktitle} {Proceedings of the
  2009 IEEE International Symposium on Information Theory, ISIT 2009}}}\
  (\bibinfo  {publisher} {IEEE})\ pp.\ \bibinfo {pages}
  {1879--1883}\BibitemShut {NoStop}%
\bibitem [{\citenamefont {Elkouss}\ \emph {et~al.}(2011)\citenamefont
  {Elkouss}, \citenamefont {Martinez-Mateo},\ and\ \citenamefont
  {Martin}}]{EMM11}%
  \BibitemOpen
  \bibfield  {author} {\bibinfo {author} {\bibnamefont {Elkouss}, \bibfnamefont
  {David}}, \bibinfo {author} {\bibfnamefont {Jesus}\ \bibnamefont
  {Martinez-Mateo}}, \ and\ \bibinfo {author} {\bibfnamefont {Vicente}\
  \bibnamefont {Martin}}} (\bibinfo {year} {2011}),\ \bibfield  {title}
  {\enquote {\bibinfo {title} {Information reconciliation for quantum key
  distribution},}\ }\href@noop {} {\bibfield  {journal} {\bibinfo  {journal}
  {Quantum Inf. Comput.}\ }\textbf {\bibinfo {volume} {11}}~(\bibinfo {number}
  {3}),\ \bibinfo {pages} {226--238}}\BibitemShut {NoStop}%
\bibitem [{\citenamefont {Fehr}\ and\ \citenamefont {Schaffner}(2008)}]{FS08}%
  \BibitemOpen
  \bibfield  {author} {\bibinfo {author} {\bibnamefont {Fehr}, \bibfnamefont
  {Serge}}, \ and\ \bibinfo {author} {\bibfnamefont {Christian}\ \bibnamefont
  {Schaffner}}} (\bibinfo {year} {2008}),\ \bibfield  {title} {\enquote
  {\bibinfo {title} {Randomness extraction via $\delta$-biased masking in the
  presence of a quantum attacker},}\ }in\ \href {\doibase
  10.1007/978-3-540-78524-8_26} {\emph {\bibinfo {booktitle} {Theory of
  Cryptography, Proceedings of TCC 2008}}},\ \bibinfo {series} {LNCS}, Vol.\
  \bibinfo {volume} {4948}\ (\bibinfo  {publisher} {Springer})\ pp.\ \bibinfo
  {pages} {465--481},\ \Eprint {http://arxiv.org/abs/arXiv:0706.2606}
  {arXiv:0706.2606} \BibitemShut {NoStop}%
\bibitem [{\citenamefont {Fitzsimons}\ and\ \citenamefont
  {Kashefi}(2017)}]{FK17}%
  \BibitemOpen
  \bibfield  {author} {\bibinfo {author} {\bibnamefont {Fitzsimons},
  \bibfnamefont {Joseph~F}}, \ and\ \bibinfo {author} {\bibfnamefont {Elham}\
  \bibnamefont {Kashefi}}} (\bibinfo {year} {2017}),\ \bibfield  {title}
  {\enquote {\bibinfo {title} {Unconditionally verifiable blind computation},}\
  }\href {\doibase 10.1103/PhysRevA.96.012303} {\bibfield  {journal} {\bibinfo
  {journal} {Phys. Rev. A}\ }\textbf {\bibinfo {volume} {96}},\ \bibinfo
  {pages} {012303}},\ \Eprint {http://arxiv.org/abs/arXiv:1203.5217}
  {arXiv:1203.5217} \BibitemShut {NoStop}%
\bibitem [{\citenamefont {Freedman}\ and\ \citenamefont
  {Clauser}(1972)}]{FreedmanClauser}%
  \BibitemOpen
  \bibfield  {author} {\bibinfo {author} {\bibnamefont {Freedman},
  \bibfnamefont {Stuart~J}}, \ and\ \bibinfo {author} {\bibfnamefont {John~F.}\
  \bibnamefont {Clauser}}} (\bibinfo {year} {1972}),\ \bibfield  {title}
  {\enquote {\bibinfo {title} {Experimental test of local hidden-variable
  theories},}\ }\href {\doibase 10.1103/PhysRevLett.28.938} {\bibfield
  {journal} {\bibinfo  {journal} {Phys. Rev. Lett.}\ }\textbf {\bibinfo
  {volume} {28}},\ \bibinfo {pages} {938--941}}\BibitemShut {NoStop}%
\bibitem [{\citenamefont {Fuchs}(1998)}]{Fuchs98}%
  \BibitemOpen
  \bibfield  {author} {\bibinfo {author} {\bibnamefont {Fuchs}, \bibfnamefont
  {Christopher~A}}} (\bibinfo {year} {1998}),\ \bibfield  {title} {\enquote
  {\bibinfo {title} {Information gain vs.\ state disturbance in quantum
  theory},}\ }\href@noop {} {\bibfield  {journal} {\bibinfo  {journal}
  {Fortschritte der Physik: Progress of Physics}\ }\textbf {\bibinfo {volume}
  {46}}~(\bibinfo {number} {4-5}),\ \bibinfo {pages} {535--565}},\ \Eprint
  {http://arxiv.org/abs/arXiv:quant-ph/9611010} {arXiv:quant-ph/9611010}
  \BibitemShut {NoStop}%
\bibitem [{\citenamefont {Fuchs}\ \emph {et~al.}(1997)\citenamefont {Fuchs},
  \citenamefont {Gisin}, \citenamefont {Griffiths}, \citenamefont {Niu},\ and\
  \citenamefont {Peres}}]{Fuchsetal1997}%
  \BibitemOpen
  \bibfield  {author} {\bibinfo {author} {\bibnamefont {Fuchs}, \bibfnamefont
  {Christopher~A}}, \bibinfo {author} {\bibfnamefont {Nicolas}\ \bibnamefont
  {Gisin}}, \bibinfo {author} {\bibfnamefont {Robert~B.}\ \bibnamefont
  {Griffiths}}, \bibinfo {author} {\bibfnamefont {Chi-Sheng}\ \bibnamefont
  {Niu}}, \ and\ \bibinfo {author} {\bibfnamefont {Asher}\ \bibnamefont
  {Peres}}} (\bibinfo {year} {1997}),\ \bibfield  {title} {\enquote {\bibinfo
  {title} {Optimal eavesdropping in quantum cryptography. i. information bound
  and optimal strategy},}\ }\href {\doibase 10.1103/PhysRevA.56.1163}
  {\bibfield  {journal} {\bibinfo  {journal} {Phys. Rev. A}\ }\textbf {\bibinfo
  {volume} {56}},\ \bibinfo {pages} {1163--1172}}\BibitemShut {NoStop}%
\bibitem [{\citenamefont {Fuchs}\ and\ \citenamefont {Van
  De~Graaf}(1999)}]{FuchsvanGraaf}%
  \BibitemOpen
  \bibfield  {author} {\bibinfo {author} {\bibnamefont {Fuchs}, \bibfnamefont
  {Christopher~A}}, \ and\ \bibinfo {author} {\bibfnamefont {Jeroen}\
  \bibnamefont {Van De~Graaf}}} (\bibinfo {year} {1999}),\ \bibfield  {title}
  {\enquote {\bibinfo {title} {Cryptographic distinguishability measures for
  quantum-mechanical states},}\ }\href@noop {} {\bibfield  {journal} {\bibinfo
  {journal} {IEEE Trans. Inf. Theory}\ }\textbf {\bibinfo {volume}
  {45}}~(\bibinfo {number} {4}),\ \bibinfo {pages} {1216--1227}}\BibitemShut
  {NoStop}%
\bibitem [{\citenamefont {Fung}\ \emph {et~al.}(2007)\citenamefont {Fung},
  \citenamefont {Qi}, \citenamefont {Tamaki},\ and\ \citenamefont
  {Lo}}]{FQTL07}%
  \BibitemOpen
  \bibfield  {author} {\bibinfo {author} {\bibnamefont {Fung}, \bibfnamefont
  {Chi-Hang~Fred}}, \bibinfo {author} {\bibfnamefont {Bing}\ \bibnamefont
  {Qi}}, \bibinfo {author} {\bibfnamefont {Kiyoshi}\ \bibnamefont {Tamaki}}, \
  and\ \bibinfo {author} {\bibfnamefont {Hoi-Kwong}\ \bibnamefont {Lo}}}
  (\bibinfo {year} {2007}),\ \bibfield  {title} {\enquote {\bibinfo {title}
  {Phase-remapping attack in practical quantum-key-distribution systems},}\
  }\href {\doibase 10.1103/PhysRevA.75.032314} {\bibfield  {journal} {\bibinfo
  {journal} {Phys. Rev. A}\ }\textbf {\bibinfo {volume} {75}}~(\bibinfo
  {number} {3}),\ \bibinfo {pages} {032314}},\ \Eprint
  {http://arxiv.org/abs/arXiv:quant-ph/0601115} {arXiv:quant-ph/0601115}
  \BibitemShut {NoStop}%
\bibitem [{\citenamefont {Garg}\ \emph {et~al.}(2017)\citenamefont {Garg},
  \citenamefont {Yuen},\ and\ \citenamefont {Zhandry}}]{GYZ17}%
  \BibitemOpen
  \bibfield  {author} {\bibinfo {author} {\bibnamefont {Garg}, \bibfnamefont
  {Sumegha}}, \bibinfo {author} {\bibfnamefont {Henry}\ \bibnamefont {Yuen}}, \
  and\ \bibinfo {author} {\bibfnamefont {Mark}\ \bibnamefont {Zhandry}}}
  (\bibinfo {year} {2017}),\ \bibfield  {title} {\enquote {\bibinfo {title}
  {New security notions and feasibility results for authentication of quantum
  data},}\ }in\ \href {\doibase 10.1007/978-3-319-63715-0_12} {\emph {\bibinfo
  {booktitle} {Advances in Cryptology -- CRYPTO 2017}}},\ \bibinfo {series}
  {LNCS}, Vol.\ \bibinfo {volume} {10402},\ \bibinfo {editor} {edited by\
  \bibinfo {editor} {\bibfnamefont {Jonathan}\ \bibnamefont {Katz}}\ and\
  \bibinfo {editor} {\bibfnamefont {Hovav}\ \bibnamefont {Shacham}}}\ (\bibinfo
   {publisher} {Springer})\ pp.\ \bibinfo {pages} {342--371},\ \Eprint
  {http://arxiv.org/abs/arXiv:1607.07759} {arXiv:1607.07759} \BibitemShut
  {NoStop}%
\bibitem [{\citenamefont {Gavinsky}\ \emph {et~al.}(2007)\citenamefont
  {Gavinsky}, \citenamefont {Kempe}, \citenamefont {Kerenidis}, \citenamefont
  {Raz},\ and\ \citenamefont {de~Wolf}}]{GKKRD07}%
  \BibitemOpen
  \bibfield  {author} {\bibinfo {author} {\bibnamefont {Gavinsky},
  \bibfnamefont {Dmitry}}, \bibinfo {author} {\bibfnamefont {Julia}\
  \bibnamefont {Kempe}}, \bibinfo {author} {\bibfnamefont {Iordanis}\
  \bibnamefont {Kerenidis}}, \bibinfo {author} {\bibfnamefont {Ran}\
  \bibnamefont {Raz}}, \ and\ \bibinfo {author} {\bibfnamefont {Ronald}\
  \bibnamefont {de~Wolf}}} (\bibinfo {year} {2007}),\ \bibfield  {title}
  {\enquote {\bibinfo {title} {Exponential separations for one-way quantum
  communication complexity, with applications to cryptography},}\ }in\ \href
  {\doibase 10.1145/1250790.1250866} {\emph {\bibinfo {booktitle} {Proceedings
  of the 39th Symposium on Theory of Computing, STOC~'07}}}\ (\bibinfo
  {publisher} {ACM})\ pp.\ \bibinfo {pages} {516--525},\ \Eprint
  {http://arxiv.org/abs/arXiv:quant-ph/0611209} {arXiv:quant-ph/0611209}
  \BibitemShut {NoStop}%
\bibitem [{\citenamefont {Gerhardt}\ \emph {et~al.}(2011)\citenamefont
  {Gerhardt}, \citenamefont {Liu}, \citenamefont {Lamas-Linares}, \citenamefont
  {Skaar}, \citenamefont {Kurtsiefer},\ and\ \citenamefont
  {Makarov}}]{GLLSKM11}%
  \BibitemOpen
  \bibfield  {author} {\bibinfo {author} {\bibnamefont {Gerhardt},
  \bibfnamefont {Ilja}}, \bibinfo {author} {\bibfnamefont {Qin}\ \bibnamefont
  {Liu}}, \bibinfo {author} {\bibfnamefont {Ant{\'\i}a}\ \bibnamefont
  {Lamas-Linares}}, \bibinfo {author} {\bibfnamefont {Johannes}\ \bibnamefont
  {Skaar}}, \bibinfo {author} {\bibfnamefont {Christian}\ \bibnamefont
  {Kurtsiefer}}, \ and\ \bibinfo {author} {\bibfnamefont {Vadim}\ \bibnamefont
  {Makarov}}} (\bibinfo {year} {2011}),\ \bibfield  {title} {\enquote {\bibinfo
  {title} {Full-field implementation of a perfect eavesdropper on a quantum
  cryptography system},}\ }\href {\doibase 10.1038/ncomms1348} {\bibfield
  {journal} {\bibinfo  {journal} {Nat. Commun.}\ }\textbf {\bibinfo {volume}
  {2}},\ \bibinfo {pages} {349}},\ \Eprint
  {http://arxiv.org/abs/arXiv:1011.0105} {arXiv:1011.0105} \BibitemShut
  {NoStop}%
\bibitem [{\citenamefont {Gheorghiu}\ and\ \citenamefont
  {Vidick}(2019)}]{GV19}%
  \BibitemOpen
  \bibfield  {author} {\bibinfo {author} {\bibnamefont {Gheorghiu},
  \bibfnamefont {Alexandru}}, \ and\ \bibinfo {author} {\bibfnamefont {Thomas}\
  \bibnamefont {Vidick}}} (\bibinfo {year} {2019}),\ \bibfield  {title}
  {\enquote {\bibinfo {title} {Computationally-secure and composable remote
  state preparation},}\ }in\ \href {\doibase 10.1109/FOCS.2019.00066} {\emph
  {\bibinfo {booktitle} {Proceedings of the 60th Symposium on Foundations of
  Computer Science, FOCS~'19}}},\ pp.\ \bibinfo {pages} {1024--1033},\ \Eprint
  {http://arxiv.org/abs/arXiv:1904.06320} {arXiv:1904.06320} \BibitemShut
  {NoStop}%
\bibitem [{\citenamefont {Gisin}\ \emph {et~al.}(2006)\citenamefont {Gisin},
  \citenamefont {Fasel}, \citenamefont {Kraus}, \citenamefont {Zbinden},\ and\
  \citenamefont {Ribordy}}]{GisinFaselKraus2006}%
  \BibitemOpen
  \bibfield  {author} {\bibinfo {author} {\bibnamefont {Gisin}, \bibfnamefont
  {Nicolas}}, \bibinfo {author} {\bibfnamefont {Sylvain}\ \bibnamefont
  {Fasel}}, \bibinfo {author} {\bibfnamefont {Barbara}\ \bibnamefont {Kraus}},
  \bibinfo {author} {\bibfnamefont {Hugo}\ \bibnamefont {Zbinden}}, \ and\
  \bibinfo {author} {\bibfnamefont {Gr\'egoire}\ \bibnamefont {Ribordy}}}
  (\bibinfo {year} {2006}),\ \bibfield  {title} {\enquote {\bibinfo {title}
  {Trojan-horse attacks on quantum-key-distribution systems},}\ }\href
  {\doibase 10.1103/PhysRevA.73.022320} {\bibfield  {journal} {\bibinfo
  {journal} {Phys. Rev. A}\ }\textbf {\bibinfo {volume} {73}},\ \bibinfo
  {pages} {022320}},\ \Eprint {http://arxiv.org/abs/arXiv:quant-ph/0507063}
  {arXiv:quant-ph/0507063} \BibitemShut {NoStop}%
\bibitem [{\citenamefont {Giustina}\ \emph {et~al.}(2013)\citenamefont
  {Giustina}, \citenamefont {Mech}, \citenamefont {Ramelow}, \citenamefont
  {Wittmann}, \citenamefont {Kofler}, \citenamefont {Beyer}, \citenamefont
  {Lita}, \citenamefont {Calkins}, \citenamefont {Gerrits}, \citenamefont
  {Nam}, \citenamefont {Ursin},\ and\ \citenamefont {Zeilinger}}]{Giustina13}%
  \BibitemOpen
  \bibfield  {author} {\bibinfo {author} {\bibnamefont {Giustina},
  \bibfnamefont {Marissa}}, \bibinfo {author} {\bibfnamefont {Alexandra}\
  \bibnamefont {Mech}}, \bibinfo {author} {\bibfnamefont {Sven}\ \bibnamefont
  {Ramelow}}, \bibinfo {author} {\bibfnamefont {Bernhard}\ \bibnamefont
  {Wittmann}}, \bibinfo {author} {\bibfnamefont {Johannes}\ \bibnamefont
  {Kofler}}, \bibinfo {author} {\bibfnamefont {J{\"o}rn}\ \bibnamefont
  {Beyer}}, \bibinfo {author} {\bibfnamefont {Adriana}\ \bibnamefont {Lita}},
  \bibinfo {author} {\bibfnamefont {Brice}\ \bibnamefont {Calkins}}, \bibinfo
  {author} {\bibfnamefont {Thomas}\ \bibnamefont {Gerrits}}, \bibinfo {author}
  {\bibfnamefont {Sae~Woo}\ \bibnamefont {Nam}}, \bibinfo {author}
  {\bibfnamefont {Rupert}\ \bibnamefont {Ursin}}, \ and\ \bibinfo {author}
  {\bibfnamefont {Anton}\ \bibnamefont {Zeilinger}}} (\bibinfo {year} {2013}),\
  \bibfield  {title} {\enquote {\bibinfo {title} {Bell violation using
  entangled photons without the fair-sampling assumption},}\ }\href {\doibase
  10.1038/nature12012} {\bibfield  {journal} {\bibinfo  {journal} {Nature}\
  }\textbf {\bibinfo {volume} {497}}~(\bibinfo {number} {7448}),\ \bibinfo
  {pages} {227--230}}\BibitemShut {NoStop}%
\bibitem [{\citenamefont {Giustina}\ \emph {et~al.}(2015)\citenamefont
  {Giustina}, \citenamefont {Versteegh}, \citenamefont {Wengerowsky},
  \citenamefont {Handsteiner}, \citenamefont {Hochrainer}, \citenamefont
  {Phelan}, \citenamefont {Steinlechner}, \citenamefont {Kofler}, \citenamefont
  {Larsson}, \citenamefont {Abell\'an}, \citenamefont {Amaya}, \citenamefont
  {Pruneri}, \citenamefont {Mitchell}, \citenamefont {Beyer}, \citenamefont
  {Gerrits}, \citenamefont {Lita}, \citenamefont {Shalm}, \citenamefont {Nam},
  \citenamefont {Scheidl}, \citenamefont {Ursin}, \citenamefont {Wittmann},\
  and\ \citenamefont {Zeilinger}}]{Giustina15}%
  \BibitemOpen
  \bibfield  {author} {\bibinfo {author} {\bibnamefont {Giustina},
  \bibfnamefont {Marissa}}, \bibinfo {author} {\bibfnamefont {Marijn A.~M.}\
  \bibnamefont {Versteegh}}, \bibinfo {author} {\bibfnamefont {S\"oren}\
  \bibnamefont {Wengerowsky}}, \bibinfo {author} {\bibfnamefont {Johannes}\
  \bibnamefont {Handsteiner}}, \bibinfo {author} {\bibfnamefont {Armin}\
  \bibnamefont {Hochrainer}}, \bibinfo {author} {\bibfnamefont {Kevin}\
  \bibnamefont {Phelan}}, \bibinfo {author} {\bibfnamefont {Fabian}\
  \bibnamefont {Steinlechner}}, \bibinfo {author} {\bibfnamefont {Johannes}\
  \bibnamefont {Kofler}}, \bibinfo {author} {\bibfnamefont {Jan-\AA{}ke}\
  \bibnamefont {Larsson}}, \bibinfo {author} {\bibfnamefont {Carlos}\
  \bibnamefont {Abell\'an}}, \bibinfo {author} {\bibfnamefont {Waldimar}\
  \bibnamefont {Amaya}}, \bibinfo {author} {\bibfnamefont {Valerio}\
  \bibnamefont {Pruneri}}, \bibinfo {author} {\bibfnamefont {Morgan~W.}\
  \bibnamefont {Mitchell}}, \bibinfo {author} {\bibfnamefont {J\"orn}\
  \bibnamefont {Beyer}}, \bibinfo {author} {\bibfnamefont {Thomas}\
  \bibnamefont {Gerrits}}, \bibinfo {author} {\bibfnamefont {Adriana~E.}\
  \bibnamefont {Lita}}, \bibinfo {author} {\bibfnamefont {Lynden~K.}\
  \bibnamefont {Shalm}}, \bibinfo {author} {\bibfnamefont {Sae~Woo}\
  \bibnamefont {Nam}}, \bibinfo {author} {\bibfnamefont {Thomas}\ \bibnamefont
  {Scheidl}}, \bibinfo {author} {\bibfnamefont {Rupert}\ \bibnamefont {Ursin}},
  \bibinfo {author} {\bibfnamefont {Bernhard}\ \bibnamefont {Wittmann}}, \ and\
  \bibinfo {author} {\bibfnamefont {Anton}\ \bibnamefont {Zeilinger}}}
  (\bibinfo {year} {2015}),\ \bibfield  {title} {\enquote {\bibinfo {title}
  {Significant-loophole-free test of {Bell}'s theorem with entangled
  photons},}\ }\href {\doibase 10.1103/PhysRevLett.115.250401} {\bibfield
  {journal} {\bibinfo  {journal} {Phys. Rev. Lett.}\ }\textbf {\bibinfo
  {volume} {115}},\ \bibinfo {pages} {250401}}\BibitemShut {NoStop}%
\bibitem [{\citenamefont {Goldreich}(2004)}]{Gol04}%
  \BibitemOpen
  \bibfield  {author} {\bibinfo {author} {\bibnamefont {Goldreich},
  \bibfnamefont {Oded}}} (\bibinfo {year} {2004}),\ \href@noop {} {\emph
  {\bibinfo {title} {Foundations of Cryptography: Volume 2, Basic
  Applications}}}\ (\bibinfo  {publisher} {Cambridge University Press},\
  \bibinfo {address} {New York, NY, USA})\BibitemShut {NoStop}%
\bibitem [{\citenamefont {Goldreich}\ \emph {et~al.}(1987)\citenamefont
  {Goldreich}, \citenamefont {Micali},\ and\ \citenamefont
  {Wigderson}}]{GMW87}%
  \BibitemOpen
  \bibfield  {author} {\bibinfo {author} {\bibnamefont {Goldreich},
  \bibfnamefont {Oded}}, \bibinfo {author} {\bibfnamefont {Silvia}\
  \bibnamefont {Micali}}, \ and\ \bibinfo {author} {\bibfnamefont {Avi}\
  \bibnamefont {Wigderson}}} (\bibinfo {year} {1987}),\ \bibfield  {title}
  {\enquote {\bibinfo {title} {How to play any mental game},}\ }in\ \href
  {\doibase 10.1145/28395.28420} {\emph {\bibinfo {booktitle} {Proceedings of
  the 19th Symposium on Theory of Computing, STOC~'87}}}\ (\bibinfo
  {publisher} {ACM})\ pp.\ \bibinfo {pages} {218--–229}\BibitemShut {NoStop}%
\bibitem [{\citenamefont {Goldreich}\ \emph {et~al.}(1986)\citenamefont
  {Goldreich}, \citenamefont {Micali},\ and\ \citenamefont
  {Wigderson}}]{GMW86}%
  \BibitemOpen
  \bibfield  {author} {\bibinfo {author} {\bibnamefont {Goldreich},
  \bibfnamefont {Oded}}, \bibinfo {author} {\bibfnamefont {Silvio}\
  \bibnamefont {Micali}}, \ and\ \bibinfo {author} {\bibfnamefont {Avi}\
  \bibnamefont {Wigderson}}} (\bibinfo {year} {1986}),\ \bibfield  {title}
  {\enquote {\bibinfo {title} {Proofs that yield nothing but their validity and
  a methodology of cryptographic protocol design},}\ }in\ \href {\doibase
  10.1109/SFCS.1986.47} {\emph {\bibinfo {booktitle} {Proceedings of the 27th
  Symposium on Foundations of Computer Science, FOCS~'86}}}\ (\bibinfo
  {publisher} {IEEE})\ pp.\ \bibinfo {pages} {174--187}\BibitemShut {NoStop}%
\bibitem [{\citenamefont {Gottesman}\ and\ \citenamefont {Lo}(2003)}]{GL03}%
  \BibitemOpen
  \bibfield  {author} {\bibinfo {author} {\bibnamefont {Gottesman},
  \bibfnamefont {Daniel}}, \ and\ \bibinfo {author} {\bibfnamefont {Hoi-Kwong}\
  \bibnamefont {Lo}}} (\bibinfo {year} {2003}),\ \bibfield  {title} {\enquote
  {\bibinfo {title} {Proof of security of quantum key distribution with two-way
  classical communications},}\ }\href {\doibase 10.1109/TIT.2002.807289}
  {\bibfield  {journal} {\bibinfo  {journal} {IEEE Trans. Inf. Theory}\
  }\textbf {\bibinfo {volume} {49}}~(\bibinfo {number} {2}),\ \bibinfo {pages}
  {457--475}},\ \Eprint {http://arxiv.org/abs/arXiv:quant-ph/0105121}
  {arXiv:quant-ph/0105121} \BibitemShut {NoStop}%
\bibitem [{\citenamefont {Gottesman}\ \emph {et~al.}(2004)\citenamefont
  {Gottesman}, \citenamefont {Lo}, \citenamefont {L\"{u}tkenhaus},\ and\
  \citenamefont {Preskill}}]{GLLP04}%
  \BibitemOpen
  \bibfield  {author} {\bibinfo {author} {\bibnamefont {Gottesman},
  \bibfnamefont {Daniel}}, \bibinfo {author} {\bibfnamefont {Hoi-Kwong}\
  \bibnamefont {Lo}}, \bibinfo {author} {\bibfnamefont {Norbert}\ \bibnamefont
  {L\"{u}tkenhaus}}, \ and\ \bibinfo {author} {\bibfnamefont {John}\
  \bibnamefont {Preskill}}} (\bibinfo {year} {2004}),\ \bibfield  {title}
  {\enquote {\bibinfo {title} {Security of quantum key distribution with
  imperfect devices},}\ }\href@noop {} {\bibfield  {journal} {\bibinfo
  {journal} {Quantum Inf. Comput.}\ }\textbf {\bibinfo {volume} {4}}~(\bibinfo
  {number} {5}),\ \bibinfo {pages} {325--360}},\ \Eprint
  {http://arxiv.org/abs/arXiv:quant-ph/0212066} {arXiv:quant-ph/0212066}
  \BibitemShut {NoStop}%
\bibitem [{\citenamefont {Goyal}\ \emph {et~al.}(2010)\citenamefont {Goyal},
  \citenamefont {Ishai}, \citenamefont {Sahai}, \citenamefont {Venkatesan},\
  and\ \citenamefont {Wadia}}]{GISVW10}%
  \BibitemOpen
  \bibfield  {author} {\bibinfo {author} {\bibnamefont {Goyal}, \bibfnamefont
  {Vipul}}, \bibinfo {author} {\bibfnamefont {Yuval}\ \bibnamefont {Ishai}},
  \bibinfo {author} {\bibfnamefont {Amit}\ \bibnamefont {Sahai}}, \bibinfo
  {author} {\bibfnamefont {Ramarathnam}\ \bibnamefont {Venkatesan}}, \ and\
  \bibinfo {author} {\bibfnamefont {Akshay}\ \bibnamefont {Wadia}}} (\bibinfo
  {year} {2010}),\ \bibfield  {title} {\enquote {\bibinfo {title} {Founding
  cryptography on tamper-proof hardware tokens},}\ }in\ \href {\doibase
  10.1007/978-3-642-11799-2_19} {\emph {\bibinfo {booktitle} {Theory of
  Cryptography, Proceedings of TCC 2010}}},\ \bibinfo {series} {LNCS}, Vol.\
  \bibinfo {volume} {5978}\ (\bibinfo  {publisher} {Springer})\ pp.\ \bibinfo
  {pages} {308--326},\ \bibinfo {note} {e-Print
  \href{http://eprint.iacr.org/2010/153}{IACR 2010/153}}\BibitemShut {NoStop}%
\bibitem [{\citenamefont {Gutoski}(2012)}]{Gut12}%
  \BibitemOpen
  \bibfield  {author} {\bibinfo {author} {\bibnamefont {Gutoski}, \bibfnamefont
  {Gus}}} (\bibinfo {year} {2012}),\ \bibfield  {title} {\enquote {\bibinfo
  {title} {On a measure of distance for quantum strategies},}\ }\href {\doibase
  10.1063/1.3693621} {\bibfield  {journal} {\bibinfo  {journal} {J. Math.
  Phys.}\ }\textbf {\bibinfo {volume} {53}}~(\bibinfo {number} {3}),\ \bibinfo
  {pages} {032202}},\ \Eprint {http://arxiv.org/abs/arXiv:1008.4636}
  {arXiv:1008.4636} \BibitemShut {NoStop}%
\bibitem [{\citenamefont {Gutoski}\ and\ \citenamefont {Watrous}(2007)}]{GW07}%
  \BibitemOpen
  \bibfield  {author} {\bibinfo {author} {\bibnamefont {Gutoski}, \bibfnamefont
  {Gus}}, \ and\ \bibinfo {author} {\bibfnamefont {John}\ \bibnamefont
  {Watrous}}} (\bibinfo {year} {2007}),\ \bibfield  {title} {\enquote {\bibinfo
  {title} {Toward a general theory of quantum games},}\ }in\ \href {\doibase
  10.1145/1250790.1250873} {\emph {\bibinfo {booktitle} {Proceedings of the
  39th Symposium on Theory of Computing, STOC~'07}}}\ (\bibinfo  {publisher}
  {ACM})\ pp.\ \bibinfo {pages} {565--574},\ \Eprint
  {http://arxiv.org/abs/arXiv:quant-ph/0611234} {arXiv:quant-ph/0611234}
  \BibitemShut {NoStop}%
\bibitem [{\citenamefont {Hardy}(2005)}]{Har05}%
  \BibitemOpen
  \bibfield  {author} {\bibinfo {author} {\bibnamefont {Hardy}, \bibfnamefont
  {Lucien}}} (\bibinfo {year} {2005}),\ \href@noop {} {\enquote {\bibinfo
  {title} {Probability theories with dynamic causal structure: A new framework
  for quantum gravity},}\ }\bibinfo {howpublished} {e-Print},\ \Eprint
  {http://arxiv.org/abs/arXiv:gr-qc/0509120} {arXiv:gr-qc/0509120} \BibitemShut
  {NoStop}%
\bibitem [{\citenamefont {Hardy}(2007)}]{Har07}%
  \BibitemOpen
  \bibfield  {author} {\bibinfo {author} {\bibnamefont {Hardy}, \bibfnamefont
  {Lucien}}} (\bibinfo {year} {2007}),\ \bibfield  {title} {\enquote {\bibinfo
  {title} {Towards quantum gravity: a framework for probabilistic theories with
  non-fixed causal structure},}\ }\href {\doibase 10.1088/1751-8113/40/12/S12}
  {\bibfield  {journal} {\bibinfo  {journal} {J. Phys. A}\ }\textbf {\bibinfo
  {volume} {40}}~(\bibinfo {number} {12}),\ \bibinfo {pages} {3081}},\ \Eprint
  {http://arxiv.org/abs/arXiv:gr-qc/0608043} {arXiv:gr-qc/0608043} \BibitemShut
  {NoStop}%
\bibitem [{\citenamefont {Hardy}(2011)}]{Har11}%
  \BibitemOpen
  \bibfield  {author} {\bibinfo {author} {\bibnamefont {Hardy}, \bibfnamefont
  {Lucien}}} (\bibinfo {year} {2011}),\ \href@noop {} {\enquote {\bibinfo
  {title} {Reformulating and reconstructing quantum theory},}\ }\bibinfo
  {howpublished} {e-print},\ \Eprint {http://arxiv.org/abs/arXiv:1104.2066}
  {arXiv:1104.2066} \BibitemShut {NoStop}%
\bibitem [{\citenamefont {Hardy}(2012)}]{Har12}%
  \BibitemOpen
  \bibfield  {author} {\bibinfo {author} {\bibnamefont {Hardy}, \bibfnamefont
  {Lucien}}} (\bibinfo {year} {2012}),\ \bibfield  {title} {\enquote {\bibinfo
  {title} {The operator tensor formulation of quantum theory},}\ }\href
  {\doibase 10.1098/rsta.2011.0326} {\bibfield  {journal} {\bibinfo  {journal}
  {Philos. Trans. R. Soc. London, Ser. A}\ }\textbf {\bibinfo {volume}
  {370}}~(\bibinfo {number} {1971}),\ \bibinfo {pages} {3385--3417}},\ \Eprint
  {http://arxiv.org/abs/arXiv:1201.4390} {arXiv:1201.4390} \BibitemShut
  {NoStop}%
\bibitem [{\citenamefont {Hardy}(2015)}]{Har15}%
  \BibitemOpen
  \bibfield  {author} {\bibinfo {author} {\bibnamefont {Hardy}, \bibfnamefont
  {Lucien}}} (\bibinfo {year} {2015}),\ \bibfield  {title} {\enquote {\bibinfo
  {title} {Quantum theory with bold operator tensors},}\ }\href {\doibase
  10.1098/rsta.2014.0239} {\bibfield  {journal} {\bibinfo  {journal} {Philos.
  Trans. R. Soc. London, Ser. A}\ }\textbf {\bibinfo {volume} {373}}~(\bibinfo
  {number} {2047}),\ 10.1098/rsta.2014.0239}\BibitemShut {NoStop}%
\bibitem [{\citenamefont {Hayashi}\ and\ \citenamefont
  {Tsurumaru}(2012)}]{HT12}%
  \BibitemOpen
  \bibfield  {author} {\bibinfo {author} {\bibnamefont {Hayashi}, \bibfnamefont
  {Masahito}}, \ and\ \bibinfo {author} {\bibfnamefont {Toyohiro}\ \bibnamefont
  {Tsurumaru}}} (\bibinfo {year} {2012}),\ \bibfield  {title} {\enquote
  {\bibinfo {title} {Concise and tight security analysis of the
  {Bennett}{\textendash}{Brassard} 1984 protocol with finite key lengths},}\
  }\href {\doibase 10.1088/1367-2630/14/9/093014} {\bibfield  {journal}
  {\bibinfo  {journal} {New J. Phys.}\ }\textbf {\bibinfo {volume}
  {14}}~(\bibinfo {number} {9}),\ \bibinfo {pages} {093014}},\ \Eprint
  {http://arxiv.org/abs/arXiv:1107.0589} {arXiv:1107.0589} \BibitemShut
  {NoStop}%
\bibitem [{\citenamefont {Hayden}\ \emph {et~al.}(2011)\citenamefont {Hayden},
  \citenamefont {Leung},\ and\ \citenamefont {Mayers}}]{HLM11}%
  \BibitemOpen
  \bibfield  {author} {\bibinfo {author} {\bibnamefont {Hayden}, \bibfnamefont
  {Patrick}}, \bibinfo {author} {\bibfnamefont {Debbie}\ \bibnamefont {Leung}},
  \ and\ \bibinfo {author} {\bibfnamefont {Dominic}\ \bibnamefont {Mayers}}}
  (\bibinfo {year} {2011}),\ \href@noop {} {\enquote {\bibinfo {title} {The
  universal composable security of quantum message authentication with key
  recycling},}\ }\bibinfo {howpublished} {presented at QCrypt 2011, e-Print},\
  \Eprint {http://arxiv.org/abs/arXiv:1610.09434} {arXiv:1610.09434}
  \BibitemShut {NoStop}%
\bibitem [{\citenamefont {Hayden}\ \emph {et~al.}(2004)\citenamefont {Hayden},
  \citenamefont {Leung}, \citenamefont {Shor},\ and\ \citenamefont
  {Winter}}]{HLSW04}%
  \BibitemOpen
  \bibfield  {author} {\bibinfo {author} {\bibnamefont {Hayden}, \bibfnamefont
  {Patrick}}, \bibinfo {author} {\bibfnamefont {Debbie}\ \bibnamefont {Leung}},
  \bibinfo {author} {\bibfnamefont {Peter~W.}\ \bibnamefont {Shor}}, \ and\
  \bibinfo {author} {\bibfnamefont {Andreas}\ \bibnamefont {Winter}}} (\bibinfo
  {year} {2004}),\ \bibfield  {title} {\enquote {\bibinfo {title} {Randomizing
  quantum states: Constructions and applications},}\ }\href {\doibase
  10.1007/s00220-004-1087-6} {\bibfield  {journal} {\bibinfo  {journal}
  {Commun. Math. Phys.}\ }\textbf {\bibinfo {volume} {250}},\ \bibinfo {pages}
  {371--391}},\ \Eprint {http://arxiv.org/abs/arXiv:quant-ph/0307104v3}
  {arXiv:quant-ph/0307104v3} \BibitemShut {NoStop}%
\bibitem [{\citenamefont {Helstrom}(1976)}]{Hel76}%
  \BibitemOpen
  \bibfield  {author} {\bibinfo {author} {\bibnamefont {Helstrom},
  \bibfnamefont {Carl~W}}} (\bibinfo {year} {1976}),\ \href@noop {} {\emph
  {\bibinfo {title} {Quantum Detection and Estimation Theory}}},\ \bibinfo
  {series} {Mathematics in science and engineering}, Vol.\ \bibinfo {volume}
  {123}\ (\bibinfo  {publisher} {Academic Press})\BibitemShut {NoStop}%
\bibitem [{\citenamefont {Hensen}\ \emph {et~al.}(2015)\citenamefont {Hensen},
  \citenamefont {Bernien}, \citenamefont {Dr{\'e}au}, \citenamefont {Reiserer},
  \citenamefont {Kalb}, \citenamefont {Blok}, \citenamefont {Ruitenberg},
  \citenamefont {Vermeulen}, \citenamefont {Schouten}, \citenamefont
  {Abell{\'a}n}, \citenamefont {Amaya}, \citenamefont {Pruneri}, \citenamefont
  {Mitchell}, \citenamefont {Markham}, \citenamefont {Twitchen}, \citenamefont
  {Elkouss}, \citenamefont {Wehner}, \citenamefont {Taminiau},\ and\
  \citenamefont {Hanson}}]{Hensen}%
  \BibitemOpen
  \bibfield  {author} {\bibinfo {author} {\bibnamefont {Hensen}, \bibfnamefont
  {B}}, \bibinfo {author} {\bibfnamefont {H.}~\bibnamefont {Bernien}}, \bibinfo
  {author} {\bibfnamefont {A.~E.}\ \bibnamefont {Dr{\'e}au}}, \bibinfo {author}
  {\bibfnamefont {A.}~\bibnamefont {Reiserer}}, \bibinfo {author}
  {\bibfnamefont {N.}~\bibnamefont {Kalb}}, \bibinfo {author} {\bibfnamefont
  {M.~S.}\ \bibnamefont {Blok}}, \bibinfo {author} {\bibfnamefont
  {J.}~\bibnamefont {Ruitenberg}}, \bibinfo {author} {\bibfnamefont {R.~F.~L.}\
  \bibnamefont {Vermeulen}}, \bibinfo {author} {\bibfnamefont {R.~N.}\
  \bibnamefont {Schouten}}, \bibinfo {author} {\bibfnamefont {C.}~\bibnamefont
  {Abell{\'a}n}}, \bibinfo {author} {\bibfnamefont {W.}~\bibnamefont {Amaya}},
  \bibinfo {author} {\bibfnamefont {V.}~\bibnamefont {Pruneri}}, \bibinfo
  {author} {\bibfnamefont {M.~W.}\ \bibnamefont {Mitchell}}, \bibinfo {author}
  {\bibfnamefont {M.}~\bibnamefont {Markham}}, \bibinfo {author} {\bibfnamefont
  {D.~J.}\ \bibnamefont {Twitchen}}, \bibinfo {author} {\bibfnamefont
  {D.}~\bibnamefont {Elkouss}}, \bibinfo {author} {\bibfnamefont
  {S.}~\bibnamefont {Wehner}}, \bibinfo {author} {\bibfnamefont {T.~H.}\
  \bibnamefont {Taminiau}}, \ and\ \bibinfo {author} {\bibfnamefont
  {R.}~\bibnamefont {Hanson}}} (\bibinfo {year} {2015}),\ \bibfield  {title}
  {\enquote {\bibinfo {title} {Loophole-free bell inequality violation using
  electron spins separated by 1.3 kilometres},}\ }\href {\doibase
  10.1038/nature15759} {\bibfield  {journal} {\bibinfo  {journal} {Nature}\
  }\textbf {\bibinfo {volume} {526}}~(\bibinfo {number} {7575}),\ \bibinfo
  {pages} {682--686}}\BibitemShut {NoStop}%
\bibitem [{\citenamefont {Hofheinz}\ \emph {et~al.}(2006)\citenamefont
  {Hofheinz}, \citenamefont {M\"{u}ller-Quade},\ and\ \citenamefont
  {Unruh}}]{HMU06}%
  \BibitemOpen
  \bibfield  {author} {\bibinfo {author} {\bibnamefont {Hofheinz},
  \bibfnamefont {Dennis}}, \bibinfo {author} {\bibfnamefont {J\"{o}rn}\
  \bibnamefont {M\"{u}ller-Quade}}, \ and\ \bibinfo {author} {\bibfnamefont
  {Dominique}\ \bibnamefont {Unruh}}} (\bibinfo {year} {2006}),\ \bibfield
  {title} {\enquote {\bibinfo {title} {On the (im)possibility of extending coin
  toss},}\ }in\ \href@noop {} {\emph {\bibinfo {booktitle} {Advances in
  Cryptology -- EUROCRYPT 2006}}},\ \bibinfo {series} {LNCS}, Vol.\ \bibinfo
  {volume} {4004}\ (\bibinfo  {publisher} {Springer})\ pp.\ \bibinfo {pages}
  {504--521},\ \bibinfo {note} {e-Print
  \href{http://eprint.iacr.org/2006/177}{IACR 2006/177}}\BibitemShut {NoStop}%
\bibitem [{\citenamefont {Hofheinz}\ and\ \citenamefont {Shoup}(2013)}]{HS13}%
  \BibitemOpen
  \bibfield  {author} {\bibinfo {author} {\bibnamefont {Hofheinz},
  \bibfnamefont {Dennis}}, \ and\ \bibinfo {author} {\bibfnamefont {Victor}\
  \bibnamefont {Shoup}}} (\bibinfo {year} {2013}),\ \bibfield  {title}
  {\enquote {\bibinfo {title} {{GNUC}: A new universal composability
  framework},}\ }\href {\doibase 10.1007/s00145-013-9160-y} {\bibfield
  {journal} {\bibinfo  {journal} {J. Crypt.}\ ,\ \bibinfo {pages}
  {1--86}}}\bibinfo {note} {E-Print \href{http://eprint.iacr.org/2011/303}{IACR
  2011/303}}\BibitemShut {NoStop}%
\bibitem [{\citenamefont {Horodecki}\ \emph {et~al.}(2008)\citenamefont
  {Horodecki}, \citenamefont {Horodecki}, \citenamefont {Horodecki},
  \citenamefont {Leung},\ and\ \citenamefont {Oppenheim}}]{HHHLO08}%
  \BibitemOpen
  \bibfield  {author} {\bibinfo {author} {\bibnamefont {Horodecki},
  \bibfnamefont {Karol}}, \bibinfo {author} {\bibfnamefont {Micha\l{}}\
  \bibnamefont {Horodecki}}, \bibinfo {author} {\bibfnamefont {Pave\l{}}\
  \bibnamefont {Horodecki}}, \bibinfo {author} {\bibfnamefont {Debbie}\
  \bibnamefont {Leung}}, \ and\ \bibinfo {author} {\bibfnamefont {Jonathan}\
  \bibnamefont {Oppenheim}}} (\bibinfo {year} {2008}),\ \bibfield  {title}
  {\enquote {\bibinfo {title} {Quantum key distribution based on private
  states: Unconditional security over untrusted channels with zero quantum
  capacity},}\ }\href {\doibase 10.1109/TIT.2008.921870} {\bibfield  {journal}
  {\bibinfo  {journal} {IEEE Trans. Inf. Theory}\ }\textbf {\bibinfo {volume}
  {54}}~(\bibinfo {number} {6}),\ \bibinfo {pages} {2604--2620}}\BibitemShut
  {NoStop}%
\bibitem [{\citenamefont {Horodecki}\ and\ \citenamefont
  {Stankiewicz}(2020)}]{HS20}%
  \BibitemOpen
  \bibfield  {author} {\bibinfo {author} {\bibnamefont {Horodecki},
  \bibfnamefont {Karol}}, \ and\ \bibinfo {author} {\bibfnamefont {Maciej}\
  \bibnamefont {Stankiewicz}}} (\bibinfo {year} {2020}),\ \bibfield  {title}
  {\enquote {\bibinfo {title} {Semi-device-independent quantum money},}\ }\href
  {\doibase 10.1088/1367-2630/ab6872} {\bibfield  {journal} {\bibinfo
  {journal} {New J. Phys.}\ }\textbf {\bibinfo {volume} {22}}~(\bibinfo
  {number} {2}),\ \bibinfo {pages} {023007}},\ \Eprint
  {http://arxiv.org/abs/arxiv:1811.10552} {arxiv:1811.10552} \BibitemShut
  {NoStop}%
\bibitem [{\citenamefont {Horodecki}\ \emph {et~al.}(1998)\citenamefont
  {Horodecki}, \citenamefont {Horodecki},\ and\ \citenamefont
  {Horodecki}}]{HHH98}%
  \BibitemOpen
  \bibfield  {author} {\bibinfo {author} {\bibnamefont {Horodecki},
  \bibfnamefont {Micha\l{}}}, \bibinfo {author} {\bibfnamefont {Pawe\l{}}\
  \bibnamefont {Horodecki}}, \ and\ \bibinfo {author} {\bibfnamefont {Ryszard}\
  \bibnamefont {Horodecki}}} (\bibinfo {year} {1998}),\ \bibfield  {title}
  {\enquote {\bibinfo {title} {Mixed-state entanglement and distillation: Is
  there a ``bound'' entanglement in nature?}}\ }\href {\doibase
  10.1103/PhysRevLett.80.5239} {\bibfield  {journal} {\bibinfo  {journal}
  {Phys. Rev. Lett.}\ }\textbf {\bibinfo {volume} {80}},\ \bibinfo {pages}
  {5239--5242}}\BibitemShut {NoStop}%
\bibitem [{\citenamefont {Hwang}(2003)}]{Hwang2003}%
  \BibitemOpen
  \bibfield  {author} {\bibinfo {author} {\bibnamefont {Hwang}, \bibfnamefont
  {Won-Young}}} (\bibinfo {year} {2003}),\ \bibfield  {title} {\enquote
  {\bibinfo {title} {Quantum key distribution with high loss: Toward global
  secure communication},}\ }\href {\doibase 10.1103/PhysRevLett.91.057901}
  {\bibfield  {journal} {\bibinfo  {journal} {Phys. Rev. Lett.}\ }\textbf
  {\bibinfo {volume} {91}},\ \bibinfo {pages} {057901}}\BibitemShut {NoStop}%
\bibitem [{\citenamefont {Inamori}\ \emph {et~al.}(2007)\citenamefont
  {Inamori}, \citenamefont {L{\"u}tkenhaus},\ and\ \citenamefont
  {Mayers}}]{ILM07}%
  \BibitemOpen
  \bibfield  {author} {\bibinfo {author} {\bibnamefont {Inamori}, \bibfnamefont
  {Hitoshi}}, \bibinfo {author} {\bibfnamefont {Norbert}\ \bibnamefont
  {L{\"u}tkenhaus}}, \ and\ \bibinfo {author} {\bibfnamefont {Dominic}\
  \bibnamefont {Mayers}}} (\bibinfo {year} {2007}),\ \bibfield  {title}
  {\enquote {\bibinfo {title} {Unconditional security of practical quantum key
  distribution},}\ }\href {\doibase 10.1140/epjd/e2007-00010-4} {\bibfield
  {journal} {\bibinfo  {journal} {Eur. Phys. J. D}\ }\textbf {\bibinfo {volume}
  {41}}~(\bibinfo {number} {3}),\ \bibinfo {pages} {599--627}},\ \Eprint
  {http://arxiv.org/abs/arXiv:quant-ph/0107017} {arXiv:quant-ph/0107017}
  \BibitemShut {NoStop}%
\bibitem [{\citenamefont {Inoue}\ \emph {et~al.}(2002)\citenamefont {Inoue},
  \citenamefont {Waks},\ and\ \citenamefont {Yamamoto}}]{IWY02}%
  \BibitemOpen
  \bibfield  {author} {\bibinfo {author} {\bibnamefont {Inoue}, \bibfnamefont
  {Kyo}}, \bibinfo {author} {\bibfnamefont {Edo}\ \bibnamefont {Waks}}, \ and\
  \bibinfo {author} {\bibfnamefont {Yoshihisa}\ \bibnamefont {Yamamoto}}}
  (\bibinfo {year} {2002}),\ \bibfield  {title} {\enquote {\bibinfo {title}
  {Differential phase shift quantum key distribution},}\ }\href {\doibase
  10.1103/PhysRevLett.89.037902} {\bibfield  {journal} {\bibinfo  {journal}
  {Phys. Rev. Lett.}\ }\textbf {\bibinfo {volume} {89}},\ \bibinfo {pages}
  {037902}}\BibitemShut {NoStop}%
\bibitem [{\citenamefont {Ishai}\ \emph {et~al.}(2014)\citenamefont {Ishai},
  \citenamefont {Ostrovsky},\ and\ \citenamefont {Zikas}}]{IOZ14}%
  \BibitemOpen
  \bibfield  {author} {\bibinfo {author} {\bibnamefont {Ishai}, \bibfnamefont
  {Yuval}}, \bibinfo {author} {\bibfnamefont {Rafail}\ \bibnamefont
  {Ostrovsky}}, \ and\ \bibinfo {author} {\bibfnamefont {Vassilis}\
  \bibnamefont {Zikas}}} (\bibinfo {year} {2014}),\ \bibfield  {title}
  {\enquote {\bibinfo {title} {Secure multi-party computation with identifiable
  abort},}\ }in\ \href {\doibase 10.1007/978-3-662-44381-1_21} {\emph {\bibinfo
  {booktitle} {Advances in Cryptology -- CRYPTO 2014}}},\ \bibinfo {editor}
  {edited by\ \bibinfo {editor} {\bibfnamefont {Juan~A.}\ \bibnamefont
  {Garay}}\ and\ \bibinfo {editor} {\bibfnamefont {Rosario}\ \bibnamefont
  {Gennaro}}}\ (\bibinfo  {publisher} {Springer})\ pp.\ \bibinfo {pages}
  {369--386},\ \bibinfo {note} {e-Print
  \href{http://eprint.iacr.org/2015/325}{IACR 2015/325}}\BibitemShut {NoStop}%
\bibitem [{\citenamefont {Ishai}\ \emph {et~al.}(2008)\citenamefont {Ishai},
  \citenamefont {Prabhakaran},\ and\ \citenamefont {Sahai}}]{IPS08}%
  \BibitemOpen
  \bibfield  {author} {\bibinfo {author} {\bibnamefont {Ishai}, \bibfnamefont
  {Yuval}}, \bibinfo {author} {\bibfnamefont {Manoj}\ \bibnamefont
  {Prabhakaran}}, \ and\ \bibinfo {author} {\bibfnamefont {Amit}\ \bibnamefont
  {Sahai}}} (\bibinfo {year} {2008}),\ \bibfield  {title} {\enquote {\bibinfo
  {title} {Founding cryptography on oblivious transfer - efficiently},}\ }in\
  \href {\doibase 10.1007/978-3-540-85174-5_32} {\emph {\bibinfo {booktitle}
  {Advances in Cryptology -- CRYPTO 2008}}},\ \bibinfo {series} {LNCS}, Vol.\
  \bibinfo {volume} {5157}\ (\bibinfo  {publisher} {Springer})\ pp.\ \bibinfo
  {pages} {572--591}\BibitemShut {NoStop}%
\bibitem [{\citenamefont {Jost}\ and\ \citenamefont {Maurer}(2018)}]{JM18}%
  \BibitemOpen
  \bibfield  {author} {\bibinfo {author} {\bibnamefont {Jost}, \bibfnamefont
  {Daniel}}, \ and\ \bibinfo {author} {\bibfnamefont {Ueli}\ \bibnamefont
  {Maurer}}} (\bibinfo {year} {2018}),\ \bibfield  {title} {\enquote {\bibinfo
  {title} {Security definitions for hash functions: Combining {UCE} and
  {I}ndifferentiability},}\ }in\ \href@noop {} {\emph {\bibinfo {booktitle}
  {International Conference on Security and Cryptography for Networks -- SCN
  2018}}},\ \bibinfo {series} {LNCS}, Vol.\ \bibinfo {volume} {11035},\
  \bibinfo {editor} {edited by\ \bibinfo {editor} {\bibfnamefont {Dario}\
  \bibnamefont {Catalano}}\ and\ \bibinfo {editor} {\bibfnamefont {Roberto}\
  \bibnamefont {De~Prisco}}}\ (\bibinfo  {publisher} {Springer})\ pp.\ \bibinfo
  {pages} {83--101},\ \bibinfo {note} {e-Print
  \href{http://eprint.iacr.org/2006/281}{IACR 2006/281}}\BibitemShut {NoStop}%
\bibitem [{\citenamefont {Jouguet}\ and\ \citenamefont
  {Kunz-Jacques}(2014)}]{JK14}%
  \BibitemOpen
  \bibfield  {author} {\bibinfo {author} {\bibnamefont {Jouguet}, \bibfnamefont
  {Paul}}, \ and\ \bibinfo {author} {\bibfnamefont {Sebastien}\ \bibnamefont
  {Kunz-Jacques}}} (\bibinfo {year} {2014}),\ \bibfield  {title} {\enquote
  {\bibinfo {title} {High performance error correction for quantum key
  distribution using polar codes},}\ }\href@noop {} {\bibfield  {journal}
  {\bibinfo  {journal} {Quantum Inf. Comput.}\ }\textbf {\bibinfo {volume}
  {14}}~(\bibinfo {number} {3-4}),\ \bibinfo {pages} {329--338}}\BibitemShut
  {NoStop}%
\bibitem [{\citenamefont {Kaniewski}(2015)}]{Kan15}%
  \BibitemOpen
  \bibfield  {author} {\bibinfo {author} {\bibnamefont {Kaniewski},
  \bibfnamefont {J\c{e}drzej}}} (\bibinfo {year} {2015}),\ \emph {\bibinfo
  {title} {Relativistic quantum cryptography}},\ \href@noop {} {Ph.D. thesis}\
  (\bibinfo  {school} {National University of Singapore}),\ \Eprint
  {http://arxiv.org/abs/arXiv:1512.00602} {arXiv:1512.00602} \BibitemShut
  {NoStop}%
\bibitem [{\citenamefont {Kaniewski}\ \emph {et~al.}(2013)\citenamefont
  {Kaniewski}, \citenamefont {Tomamichel}, \citenamefont {H\"anggi},\ and\
  \citenamefont {Wehner}}]{KTHW13}%
  \BibitemOpen
  \bibfield  {author} {\bibinfo {author} {\bibnamefont {Kaniewski},
  \bibfnamefont {J\c{e}drzej}}, \bibinfo {author} {\bibfnamefont {Marco}\
  \bibnamefont {Tomamichel}}, \bibinfo {author} {\bibfnamefont {Esther}\
  \bibnamefont {H\"anggi}}, \ and\ \bibinfo {author} {\bibfnamefont
  {Stephanie}\ \bibnamefont {Wehner}}} (\bibinfo {year} {2013}),\ \bibfield
  {title} {\enquote {\bibinfo {title} {Secure bit commitment from relativistic
  constraints},}\ }\href {\doibase 10.1109/TIT.2013.2247463} {\bibfield
  {journal} {\bibinfo  {journal} {IEEE Transactions on Information Theory}\
  }\textbf {\bibinfo {volume} {59}}~(\bibinfo {number} {7}),\ \bibinfo {pages}
  {4687--4699}},\ \Eprint {http://arxiv.org/abs/arXiv:1206.1740}
  {arXiv:1206.1740} \BibitemShut {NoStop}%
\bibitem [{\citenamefont {Katz}\ and\ \citenamefont {Yung}(2006)}]{KY06}%
  \BibitemOpen
  \bibfield  {author} {\bibinfo {author} {\bibnamefont {Katz}, \bibfnamefont
  {Jonathan}}, \ and\ \bibinfo {author} {\bibfnamefont {Moti}\ \bibnamefont
  {Yung}}} (\bibinfo {year} {2006}),\ \bibfield  {title} {\enquote {\bibinfo
  {title} {Characterization of security notions for probabilistic private-key
  encryption},}\ }\href {\doibase 10.1007/s00145-005-0310-8} {\bibfield
  {journal} {\bibinfo  {journal} {J. Crypt.}\ }\textbf {\bibinfo {volume}
  {19}}~(\bibinfo {number} {1}),\ \bibinfo {pages} {67--95}}\BibitemShut
  {NoStop}%
\bibitem [{\citenamefont {Kent}(1999)}]{Ken99}%
  \BibitemOpen
  \bibfield  {author} {\bibinfo {author} {\bibnamefont {Kent}, \bibfnamefont
  {Adrian}}} (\bibinfo {year} {1999}),\ \bibfield  {title} {\enquote {\bibinfo
  {title} {Unconditionally secure bit commitment},}\ }\href {\doibase
  10.1103/PhysRevLett.83.1447} {\bibfield  {journal} {\bibinfo  {journal}
  {Phys. Rev. Lett.}\ }\textbf {\bibinfo {volume} {83}},\ \bibinfo {pages}
  {1447--1450}},\ \Eprint {http://arxiv.org/abs/arXiv:quant-ph/9810068}
  {arXiv:quant-ph/9810068} \BibitemShut {NoStop}%
\bibitem [{\citenamefont {Kent}(2012)}]{Ken12}%
  \BibitemOpen
  \bibfield  {author} {\bibinfo {author} {\bibnamefont {Kent}, \bibfnamefont
  {Adrian}}} (\bibinfo {year} {2012}),\ \bibfield  {title} {\enquote {\bibinfo
  {title} {Unconditionally secure bit commitment by transmitting measurement
  outcomes},}\ }\href {\doibase 10.1103/PhysRevLett.109.130501} {\bibfield
  {journal} {\bibinfo  {journal} {Phys. Rev. Lett.}\ }\textbf {\bibinfo
  {volume} {109}},\ \bibinfo {pages} {130501}},\ \Eprint
  {http://arxiv.org/abs/arXiv:1108.2879} {arXiv:1108.2879} \BibitemShut
  {NoStop}%
\bibitem [{\citenamefont {Kessler}\ and\ \citenamefont
  {Arnon-Friedman}(2020)}]{KAF20}%
  \BibitemOpen
  \bibfield  {author} {\bibinfo {author} {\bibnamefont {Kessler}, \bibfnamefont
  {Max}}, \ and\ \bibinfo {author} {\bibfnamefont {Rotem}\ \bibnamefont
  {Arnon-Friedman}}} (\bibinfo {year} {2020}),\ \bibfield  {title} {\enquote
  {\bibinfo {title} {Device-independent randomness amplification and
  privatization},}\ }\href {\doibase 10.1109/JSAIT.2020.3012498} {\bibfield
  {journal} {\bibinfo  {journal} {IEEE J. Sel. Areas Inf. Theory}\ }\textbf
  {\bibinfo {volume} {1}}~(\bibinfo {number} {2}),\ \bibinfo {pages}
  {568--584}},\ \Eprint {http://arxiv.org/abs/arXiv:1705.04148}
  {arXiv:1705.04148} \BibitemShut {NoStop}%
\bibitem [{\citenamefont {Koashi}(2004)}]{Koashi04}%
  \BibitemOpen
  \bibfield  {author} {\bibinfo {author} {\bibnamefont {Koashi}, \bibfnamefont
  {Masato}}} (\bibinfo {year} {2004}),\ \bibfield  {title} {\enquote {\bibinfo
  {title} {Unconditional security of coherent-state quantum key distribution
  with a strong phase-reference pulse},}\ }\href {\doibase
  10.1103/PhysRevLett.93.120501} {\bibfield  {journal} {\bibinfo  {journal}
  {Phys. Rev. Lett.}\ }\textbf {\bibinfo {volume} {93}},\ \bibinfo {pages}
  {120501}}\BibitemShut {NoStop}%
\bibitem [{\citenamefont {Koashi}(2009)}]{Koa09}%
  \BibitemOpen
  \bibfield  {author} {\bibinfo {author} {\bibnamefont {Koashi}, \bibfnamefont
  {Masato}}} (\bibinfo {year} {2009}),\ \bibfield  {title} {\enquote {\bibinfo
  {title} {Simple security proof of quantum key distribution based on
  complementarity},}\ }\href {\doibase 10.1088/1367-2630/11/4/045018}
  {\bibfield  {journal} {\bibinfo  {journal} {New J. Phys.}\ }\textbf {\bibinfo
  {volume} {11}}~(\bibinfo {number} {4}),\ \bibinfo {pages}
  {045018}}\BibitemShut {NoStop}%
\bibitem [{\citenamefont {Koashi}\ and\ \citenamefont
  {Winter}(2004)}]{KoashiWinter04}%
  \BibitemOpen
  \bibfield  {author} {\bibinfo {author} {\bibnamefont {Koashi}, \bibfnamefont
  {Masato}}, \ and\ \bibinfo {author} {\bibfnamefont {Andreas}\ \bibnamefont
  {Winter}}} (\bibinfo {year} {2004}),\ \bibfield  {title} {\enquote {\bibinfo
  {title} {Monogamy of quantum entanglement and other correlations},}\ }\href
  {\doibase 10.1103/PhysRevA.69.022309} {\bibfield  {journal} {\bibinfo
  {journal} {Phys. Rev. A}\ }\textbf {\bibinfo {volume} {69}},\ \bibinfo
  {pages} {022309}}\BibitemShut {NoStop}%
\bibitem [{\citenamefont {Kochen}\ and\ \citenamefont
  {Specker}(1967)}]{KocSpe67}%
  \BibitemOpen
  \bibfield  {author} {\bibinfo {author} {\bibnamefont {Kochen}, \bibfnamefont
  {Simon~B}}, \ and\ \bibinfo {author} {\bibfnamefont {Ernst~P.}\ \bibnamefont
  {Specker}}} (\bibinfo {year} {1967}),\ \bibfield  {title} {\enquote {\bibinfo
  {title} {The problem of hidden variables in quantum mechanics},}\ }\href@noop
  {} {\bibfield  {journal} {\bibinfo  {journal} {J. Math. Mech.}\ }\textbf
  {\bibinfo {volume} {17}},\ \bibinfo {pages} {59--87}}\BibitemShut {NoStop}%
\bibitem [{\citenamefont {K\"onig}\ \emph {et~al.}(2005)\citenamefont
  {K\"onig}, \citenamefont {Maurer},\ and\ \citenamefont {Renner}}]{KMR05}%
  \BibitemOpen
  \bibfield  {author} {\bibinfo {author} {\bibnamefont {K\"onig}, \bibfnamefont
  {Robert}}, \bibinfo {author} {\bibfnamefont {Ueli}\ \bibnamefont {Maurer}}, \
  and\ \bibinfo {author} {\bibfnamefont {Renato}\ \bibnamefont {Renner}}}
  (\bibinfo {year} {2005}),\ \bibfield  {title} {\enquote {\bibinfo {title} {On
  the power of quantum memory},}\ }\href {\doibase 10.1109/TIT.2005.850087}
  {\bibfield  {journal} {\bibinfo  {journal} {IEEE Trans. Inf. Theory}\
  }\textbf {\bibinfo {volume} {51}}~(\bibinfo {number} {7}),\ \bibinfo {pages}
  {2391--2401}},\ \Eprint {http://arxiv.org/abs/quant-ph/0305154}
  {quant-ph/0305154} \BibitemShut {NoStop}%
\bibitem [{\citenamefont {K\"onig}\ \emph {et~al.}(2007)\citenamefont
  {K\"onig}, \citenamefont {Renner}, \citenamefont {Bariska},\ and\
  \citenamefont {Maurer}}]{KRBM07}%
  \BibitemOpen
  \bibfield  {author} {\bibinfo {author} {\bibnamefont {K\"onig}, \bibfnamefont
  {Robert}}, \bibinfo {author} {\bibfnamefont {Renato}\ \bibnamefont {Renner}},
  \bibinfo {author} {\bibfnamefont {Andor}\ \bibnamefont {Bariska}}, \ and\
  \bibinfo {author} {\bibfnamefont {Ueli}\ \bibnamefont {Maurer}}} (\bibinfo
  {year} {2007}),\ \bibfield  {title} {\enquote {\bibinfo {title} {Small
  accessible quantum information does not imply security},}\ }\href {\doibase
  10.1103/PhysRevLett.98.140502} {\bibfield  {journal} {\bibinfo  {journal}
  {Phys. Rev. Lett.}\ }\textbf {\bibinfo {volume} {98}},\ \bibinfo {pages}
  {140502}},\ \Eprint {http://arxiv.org/abs/arXiv:quant-ph/0512021}
  {arXiv:quant-ph/0512021} \BibitemShut {NoStop}%
\bibitem [{\citenamefont {K\"onig}\ and\ \citenamefont {Terhal}(2008)}]{KT08}%
  \BibitemOpen
  \bibfield  {author} {\bibinfo {author} {\bibnamefont {K\"onig}, \bibfnamefont
  {Robert}}, \ and\ \bibinfo {author} {\bibfnamefont {Barbara~M.}\ \bibnamefont
  {Terhal}}} (\bibinfo {year} {2008}),\ \bibfield  {title} {\enquote {\bibinfo
  {title} {The bounded-storage model in the presence of a quantum adversary},}\
  }\href {\doibase 10.1109/TIT.2007.913245} {\bibfield  {journal} {\bibinfo
  {journal} {IEEE Trans. Inf. Theory}\ }\textbf {\bibinfo {volume}
  {54}}~(\bibinfo {number} {2}),\ \bibinfo {pages} {749--762}},\ \Eprint
  {http://arxiv.org/abs/arXiv:quant-ph/0608101} {arXiv:quant-ph/0608101}
  \BibitemShut {NoStop}%
\bibitem [{\citenamefont {K\"onig}\ \emph {et~al.}(2012)\citenamefont
  {K\"onig}, \citenamefont {Wehner},\ and\ \citenamefont
  {Wullschleger}}]{KWW12}%
  \BibitemOpen
  \bibfield  {author} {\bibinfo {author} {\bibnamefont {K\"onig}, \bibfnamefont
  {Robert}}, \bibinfo {author} {\bibfnamefont {Stephanie}\ \bibnamefont
  {Wehner}}, \ and\ \bibinfo {author} {\bibfnamefont {J\"urg}\ \bibnamefont
  {Wullschleger}}} (\bibinfo {year} {2012}),\ \bibfield  {title} {\enquote
  {\bibinfo {title} {Unconditional security from noisy quantum storage},}\
  }\href {\doibase 10.1109/TIT.2011.2177772} {\bibfield  {journal} {\bibinfo
  {journal} {IEEE Transactions on Information Theory}\ }\textbf {\bibinfo
  {volume} {58}}~(\bibinfo {number} {3}),\ \bibinfo {pages} {1962--1984}},\
  \Eprint {http://arxiv.org/abs/arXiv:0906.1030} {arXiv:0906.1030} \BibitemShut
  {NoStop}%
\bibitem [{\citenamefont {Kraus}\ \emph {et~al.}(2005)\citenamefont {Kraus},
  \citenamefont {Gisin},\ and\ \citenamefont {Renner}}]{PhysRevLett.95.080501}%
  \BibitemOpen
  \bibfield  {author} {\bibinfo {author} {\bibnamefont {Kraus}, \bibfnamefont
  {Barbara}}, \bibinfo {author} {\bibfnamefont {Nicolas}\ \bibnamefont
  {Gisin}}, \ and\ \bibinfo {author} {\bibfnamefont {Renner}\ \bibnamefont
  {Renner}}} (\bibinfo {year} {2005}),\ \bibfield  {title} {\enquote {\bibinfo
  {title} {Lower and upper bounds on the secret-key rate for quantum key
  distribution protocols using one-way classical communication},}\ }\href
  {\doibase 10.1103/PhysRevLett.95.080501} {\bibfield  {journal} {\bibinfo
  {journal} {Phys. Rev. Lett.}\ }\textbf {\bibinfo {volume} {95}},\ \bibinfo
  {pages} {080501}}\BibitemShut {NoStop}%
\bibitem [{\citenamefont {K{\"u}sters}(2006)}]{Kus06}%
  \BibitemOpen
  \bibfield  {author} {\bibinfo {author} {\bibnamefont {K{\"u}sters},
  \bibfnamefont {Ralf}}} (\bibinfo {year} {2006}),\ \bibfield  {title}
  {\enquote {\bibinfo {title} {Simulation-based security with inexhaustible
  interactive turing machines},}\ }in\ \href {\doibase 10.1109/CSFW.2006.30}
  {\emph {\bibinfo {booktitle} {Proceedings of the 19th IEEE workshop on
  Computer Security Foundations, CSFW~'06}}}\ (\bibinfo  {publisher} {IEEE})\
  pp.\ \bibinfo {pages} {309--320}\BibitemShut {NoStop}%
\bibitem [{\citenamefont {Laneve}\ and\ \citenamefont {del Rio}(2021)}]{LdR21}%
  \BibitemOpen
  \bibfield  {author} {\bibinfo {author} {\bibnamefont {Laneve}, \bibfnamefont
  {Lorenzo}}, \ and\ \bibinfo {author} {\bibfnamefont {L\'idia}\ \bibnamefont
  {del Rio}}} (\bibinfo {year} {2021}),\ \href@noop {} {\enquote {\bibinfo
  {title} {Impossibility of composable oblivious transfer in relativistic
  quantum cryptography},}\ }\bibinfo {howpublished} {e-Print},\ \Eprint
  {http://arxiv.org/abs/arXiv:2106.11200} {arXiv:2106.11200} \BibitemShut
  {NoStop}%
\bibitem [{\citenamefont {Leverrier}\ \emph {et~al.}(2008)\citenamefont
  {Leverrier}, \citenamefont {All\'eaume}, \citenamefont {Boutros},
  \citenamefont {Z\'emor},\ and\ \citenamefont {Grangier}}]{LABZG08}%
  \BibitemOpen
  \bibfield  {author} {\bibinfo {author} {\bibnamefont {Leverrier},
  \bibfnamefont {Anthony}}, \bibinfo {author} {\bibfnamefont {Romain}\
  \bibnamefont {All\'eaume}}, \bibinfo {author} {\bibfnamefont {Joseph}\
  \bibnamefont {Boutros}}, \bibinfo {author} {\bibfnamefont {Gilles}\
  \bibnamefont {Z\'emor}}, \ and\ \bibinfo {author} {\bibfnamefont {Philippe}\
  \bibnamefont {Grangier}}} (\bibinfo {year} {2008}),\ \bibfield  {title}
  {\enquote {\bibinfo {title} {Multidimensional reconciliation for a
  continuous-variable quantum key distribution},}\ }\href {\doibase
  10.1103/PhysRevA.77.042325} {\bibfield  {journal} {\bibinfo  {journal} {Phys.
  Rev. A}\ }\textbf {\bibinfo {volume} {77}},\ \bibinfo {pages}
  {042325}}\BibitemShut {NoStop}%
\bibitem [{\citenamefont {Lim}\ \emph {et~al.}(2014)\citenamefont {Lim},
  \citenamefont {Curty}, \citenamefont {Walenta}, \citenamefont {Xu},\ and\
  \citenamefont {Zbinden}}]{LCWXZ14}%
  \BibitemOpen
  \bibfield  {author} {\bibinfo {author} {\bibnamefont {Lim}, \bibfnamefont
  {Charles Ci~Wen}}, \bibinfo {author} {\bibfnamefont {Marcos}\ \bibnamefont
  {Curty}}, \bibinfo {author} {\bibfnamefont {Nino}\ \bibnamefont {Walenta}},
  \bibinfo {author} {\bibfnamefont {Feihu}\ \bibnamefont {Xu}}, \ and\ \bibinfo
  {author} {\bibfnamefont {Hugo}\ \bibnamefont {Zbinden}}} (\bibinfo {year}
  {2014}),\ \bibfield  {title} {\enquote {\bibinfo {title} {Concise security
  bounds for practical decoy-state quantum key distribution},}\ }\href
  {\doibase 10.1103/PhysRevA.89.022307} {\bibfield  {journal} {\bibinfo
  {journal} {Phys. Rev. A}\ }\textbf {\bibinfo {volume} {89}},\ \bibinfo
  {pages} {022307}}\BibitemShut {NoStop}%
\bibitem [{\citenamefont {Lim}\ \emph {et~al.}(2013)\citenamefont {Lim},
  \citenamefont {Portmann}, \citenamefont {Tomamichel}, \citenamefont
  {Renner},\ and\ \citenamefont {Gisin}}]{LPTRG13}%
  \BibitemOpen
  \bibfield  {author} {\bibinfo {author} {\bibnamefont {Lim}, \bibfnamefont
  {Charles Ci~Wen}}, \bibinfo {author} {\bibfnamefont {Christopher}\
  \bibnamefont {Portmann}}, \bibinfo {author} {\bibfnamefont {Marco}\
  \bibnamefont {Tomamichel}}, \bibinfo {author} {\bibfnamefont {Renato}\
  \bibnamefont {Renner}}, \ and\ \bibinfo {author} {\bibfnamefont {Nicolas}\
  \bibnamefont {Gisin}}} (\bibinfo {year} {2013}),\ \bibfield  {title}
  {\enquote {\bibinfo {title} {Device-independent quantum key distribution with
  local {Bell} test},}\ }\href {\doibase 10.1103/PhysRevX.3.031006} {\bibfield
  {journal} {\bibinfo  {journal} {Phys. Rev. X}\ }\textbf {\bibinfo {volume}
  {3}},\ \bibinfo {pages} {031006}},\ \Eprint
  {http://arxiv.org/abs/arXiv:1208.0023} {arXiv:1208.0023} \BibitemShut
  {NoStop}%
\bibitem [{\citenamefont {Lipinska}\ \emph {et~al.}(2020)\citenamefont
  {Lipinska}, \citenamefont {Ribeiro},\ and\ \citenamefont {Wehner}}]{LRW20}%
  \BibitemOpen
  \bibfield  {author} {\bibinfo {author} {\bibnamefont {Lipinska},
  \bibfnamefont {Victoria}}, \bibinfo {author} {\bibfnamefont {J\'er\'emy}\
  \bibnamefont {Ribeiro}}, \ and\ \bibinfo {author} {\bibfnamefont {Stephanie}\
  \bibnamefont {Wehner}}} (\bibinfo {year} {2020}),\ \bibfield  {title}
  {\enquote {\bibinfo {title} {Secure multiparty quantum computation with few
  qubits},}\ }\href {\doibase 10.1103/PhysRevA.102.022405} {\bibfield
  {journal} {\bibinfo  {journal} {Phys. Rev. A}\ }\textbf {\bibinfo {volume}
  {102}},\ \bibinfo {pages} {022405}},\ \Eprint
  {http://arxiv.org/abs/arXiv:2004.10486} {arXiv:2004.10486} \BibitemShut
  {NoStop}%
\bibitem [{\citenamefont {Liu}\ \emph {et~al.}(2013)\citenamefont {Liu},
  \citenamefont {Chen}, \citenamefont {Wang}, \citenamefont {Liang},
  \citenamefont {Shentu}, \citenamefont {Wang}, \citenamefont {Cui},
  \citenamefont {Yin}, \citenamefont {Liu}, \citenamefont {Li}, \citenamefont
  {Ma}, \citenamefont {Pelc}, \citenamefont {Fejer}, \citenamefont {Peng},
  \citenamefont {Zhang},\ and\ \citenamefont {Pan}}]{Liu13}%
  \BibitemOpen
  \bibfield  {author} {\bibinfo {author} {\bibnamefont {Liu}, \bibfnamefont
  {Yang}}, \bibinfo {author} {\bibfnamefont {Teng-Yun}\ \bibnamefont {Chen}},
  \bibinfo {author} {\bibfnamefont {Liu-Jun}\ \bibnamefont {Wang}}, \bibinfo
  {author} {\bibfnamefont {Hao}\ \bibnamefont {Liang}}, \bibinfo {author}
  {\bibfnamefont {Guo-Liang}\ \bibnamefont {Shentu}}, \bibinfo {author}
  {\bibfnamefont {Jian}\ \bibnamefont {Wang}}, \bibinfo {author} {\bibfnamefont
  {Ke}~\bibnamefont {Cui}}, \bibinfo {author} {\bibfnamefont {Hua-Lei}\
  \bibnamefont {Yin}}, \bibinfo {author} {\bibfnamefont {Nai-Le}\ \bibnamefont
  {Liu}}, \bibinfo {author} {\bibfnamefont {Li}~\bibnamefont {Li}}, \bibinfo
  {author} {\bibfnamefont {Xiongfeng}\ \bibnamefont {Ma}}, \bibinfo {author}
  {\bibfnamefont {Jason~S.}\ \bibnamefont {Pelc}}, \bibinfo {author}
  {\bibfnamefont {M.~M.}\ \bibnamefont {Fejer}}, \bibinfo {author}
  {\bibfnamefont {Cheng-Zhi}\ \bibnamefont {Peng}}, \bibinfo {author}
  {\bibfnamefont {Qiang}\ \bibnamefont {Zhang}}, \ and\ \bibinfo {author}
  {\bibfnamefont {Jian-Wei}\ \bibnamefont {Pan}}} (\bibinfo {year} {2013}),\
  \bibfield  {title} {\enquote {\bibinfo {title} {Experimental
  measurement-device-independent quantum key distribution},}\ }\href {\doibase
  10.1103/PhysRevLett.111.130502} {\bibfield  {journal} {\bibinfo  {journal}
  {Phys. Rev. Lett.}\ }\textbf {\bibinfo {volume} {111}},\ \bibinfo {pages}
  {130502}},\ \Eprint {http://arxiv.org/abs/arXiv:1209.6178} {arXiv:1209.6178}
  \BibitemShut {NoStop}%
\bibitem [{\citenamefont {Liu}(2014)}]{Liu14}%
  \BibitemOpen
  \bibfield  {author} {\bibinfo {author} {\bibnamefont {Liu}, \bibfnamefont
  {Yi-Kai}}} (\bibinfo {year} {2014}),\ \bibfield  {title} {\enquote {\bibinfo
  {title} {Single-shot security for one-time memories in the isolated qubits
  model},}\ }in\ \href {\doibase 10.1007/978-3-662-44381-1_2} {\emph {\bibinfo
  {booktitle} {Advances in Cryptology -- CRYPTO 2014}}},\ \bibinfo {editor}
  {edited by\ \bibinfo {editor} {\bibfnamefont {Juan~A.}\ \bibnamefont
  {Garay}}\ and\ \bibinfo {editor} {\bibfnamefont {Rosario}\ \bibnamefont
  {Gennaro}}}\ (\bibinfo  {publisher} {Springer})\ pp.\ \bibinfo {pages}
  {19--36},\ \Eprint {http://arxiv.org/abs/arXiv:1402.0049} {arXiv:1402.0049}
  \BibitemShut {NoStop}%
\bibitem [{\citenamefont {Liu}(2015)}]{Liu15}%
  \BibitemOpen
  \bibfield  {author} {\bibinfo {author} {\bibnamefont {Liu}, \bibfnamefont
  {Yi-Kai}}} (\bibinfo {year} {2015}),\ \bibfield  {title} {\enquote {\bibinfo
  {title} {Privacy amplification in the isolated qubits model},}\ }in\ \href
  {\doibase 10.1007/978-3-662-46803-6_26} {\emph {\bibinfo {booktitle}
  {Advances in Cryptology -- EUROCRYPT 2015}}},\ \bibinfo {editor} {edited by\
  \bibinfo {editor} {\bibfnamefont {Elisabeth}\ \bibnamefont {Oswald}}\ and\
  \bibinfo {editor} {\bibfnamefont {Marc}\ \bibnamefont {Fischlin}}}\ (\bibinfo
   {publisher} {Springer})\ pp.\ \bibinfo {pages} {785--814},\ \Eprint
  {http://arxiv.org/abs/arxiv:1410.3918} {arxiv:1410.3918} \BibitemShut
  {NoStop}%
\bibitem [{\citenamefont {Lo}(2003)}]{Lo03}%
  \BibitemOpen
  \bibfield  {author} {\bibinfo {author} {\bibnamefont {Lo}, \bibfnamefont
  {Hoi-Kwong}}} (\bibinfo {year} {2003}),\ \bibfield  {title} {\enquote
  {\bibinfo {title} {Method for decoupling error correction from privacy
  amplification},}\ }\href {\doibase 10.1088/1367-2630/5/1/336} {\bibfield
  {journal} {\bibinfo  {journal} {New J. Phys.}\ }\textbf {\bibinfo {volume}
  {5}},\ \bibinfo {pages} {36--36}}\BibitemShut {NoStop}%
\bibitem [{\citenamefont {Lo}\ and\ \citenamefont {Chau}(1999)}]{LoChau99}%
  \BibitemOpen
  \bibfield  {author} {\bibinfo {author} {\bibnamefont {Lo}, \bibfnamefont
  {Hoi-Kwong}}, \ and\ \bibinfo {author} {\bibfnamefont {Hoi~Fung}\
  \bibnamefont {Chau}}} (\bibinfo {year} {1999}),\ \bibfield  {title} {\enquote
  {\bibinfo {title} {Unconditional security of quantum key distribution over
  arbitrarily long distances},}\ }\href {\doibase
  10.1126/science.283.5410.2050} {\bibfield  {journal} {\bibinfo  {journal}
  {Science}\ }\textbf {\bibinfo {volume} {283}}~(\bibinfo {number} {5410}),\
  \bibinfo {pages} {2050--2056}}\BibitemShut {NoStop}%
\bibitem [{\citenamefont {Lo}\ \emph {et~al.}(2012)\citenamefont {Lo},
  \citenamefont {Curty},\ and\ \citenamefont {Qi}}]{LCQ12}%
  \BibitemOpen
  \bibfield  {author} {\bibinfo {author} {\bibnamefont {Lo}, \bibfnamefont
  {Hoi-Kwong}}, \bibinfo {author} {\bibfnamefont {Marcos}\ \bibnamefont
  {Curty}}, \ and\ \bibinfo {author} {\bibfnamefont {Bing}\ \bibnamefont {Qi}}}
  (\bibinfo {year} {2012}),\ \bibfield  {title} {\enquote {\bibinfo {title}
  {Measurement-device-independent quantum key distribution},}\ }\href {\doibase
  10.1103/PhysRevLett.108.130503} {\bibfield  {journal} {\bibinfo  {journal}
  {Phys. Rev. Lett.}\ }\textbf {\bibinfo {volume} {108}},\ \bibinfo {pages}
  {130503}},\ \Eprint {http://arxiv.org/abs/arXiv:1109.1473} {arXiv:1109.1473}
  \BibitemShut {NoStop}%
\bibitem [{\citenamefont {Lo}\ \emph {et~al.}(2005)\citenamefont {Lo},
  \citenamefont {Ma},\ and\ \citenamefont {Chen}}]{Loetal2005}%
  \BibitemOpen
  \bibfield  {author} {\bibinfo {author} {\bibnamefont {Lo}, \bibfnamefont
  {Hoi-Kwong}}, \bibinfo {author} {\bibfnamefont {Xiongfeng}\ \bibnamefont
  {Ma}}, \ and\ \bibinfo {author} {\bibfnamefont {Kai}\ \bibnamefont {Chen}}}
  (\bibinfo {year} {2005}),\ \bibfield  {title} {\enquote {\bibinfo {title}
  {Decoy state quantum key distribution},}\ }\href {\doibase
  10.1103/PhysRevLett.94.230504} {\bibfield  {journal} {\bibinfo  {journal}
  {Phys. Rev. Lett.}\ }\textbf {\bibinfo {volume} {94}},\ \bibinfo {pages}
  {230504}}\BibitemShut {NoStop}%
\bibitem [{\citenamefont {Lucamarini}\ \emph {et~al.}(2018)\citenamefont
  {Lucamarini}, \citenamefont {Yuan}, \citenamefont {Dynes},\ and\
  \citenamefont {Shields}}]{Lucamarinietal}%
  \BibitemOpen
  \bibfield  {author} {\bibinfo {author} {\bibnamefont {Lucamarini},
  \bibfnamefont {Marco}}, \bibinfo {author} {\bibfnamefont {Zhiliang~L.}\
  \bibnamefont {Yuan}}, \bibinfo {author} {\bibfnamefont {James~F.}\
  \bibnamefont {Dynes}}, \ and\ \bibinfo {author} {\bibfnamefont {Andrew~J.}\
  \bibnamefont {Shields}}} (\bibinfo {year} {2018}),\ \bibfield  {title}
  {\enquote {\bibinfo {title} {Overcoming the rate--distance limit of quantum
  key distribution without quantum repeaters},}\ }\href@noop {} {\bibfield
  {journal} {\bibinfo  {journal} {Nature}\ }\textbf {\bibinfo {volume}
  {557}}~(\bibinfo {number} {7705}),\ \bibinfo {pages} {400--403}}\BibitemShut
  {NoStop}%
\bibitem [{\citenamefont {L\"utkenhaus}(2000)}]{Lutkenhaus2000}%
  \BibitemOpen
  \bibfield  {author} {\bibinfo {author} {\bibnamefont {L\"utkenhaus},
  \bibfnamefont {Norbert}}} (\bibinfo {year} {2000}),\ \bibfield  {title}
  {\enquote {\bibinfo {title} {Security against individual attacks for
  realistic quantum key distribution},}\ }\href {\doibase
  10.1103/PhysRevA.61.052304} {\bibfield  {journal} {\bibinfo  {journal} {Phys.
  Rev. A}\ }\textbf {\bibinfo {volume} {61}},\ \bibinfo {pages}
  {052304}}\BibitemShut {NoStop}%
\bibitem [{\citenamefont {Lydersen}\ \emph {et~al.}(2010)\citenamefont
  {Lydersen}, \citenamefont {Wiechers}, \citenamefont {Wittmann}, \citenamefont
  {Elser}, \citenamefont {Skaar},\ and\ \citenamefont {Makarov}}]{LWWESM10}%
  \BibitemOpen
  \bibfield  {author} {\bibinfo {author} {\bibnamefont {Lydersen},
  \bibfnamefont {Lars}}, \bibinfo {author} {\bibfnamefont {Carlos}\
  \bibnamefont {Wiechers}}, \bibinfo {author} {\bibfnamefont {Christoffer}\
  \bibnamefont {Wittmann}}, \bibinfo {author} {\bibfnamefont {Dominique}\
  \bibnamefont {Elser}}, \bibinfo {author} {\bibfnamefont {Johannes}\
  \bibnamefont {Skaar}}, \ and\ \bibinfo {author} {\bibfnamefont {Vadim}\
  \bibnamefont {Makarov}}} (\bibinfo {year} {2010}),\ \bibfield  {title}
  {\enquote {\bibinfo {title} {Hacking commercial quantum cryptography systems
  by tailored bright illumination},}\ }\href {\doibase
  10.1038/nphoton.2010.214} {\bibfield  {journal} {\bibinfo  {journal} {Nat.
  Photonics}\ }\textbf {\bibinfo {volume} {4}}~(\bibinfo {number} {10}),\
  \bibinfo {pages} {686--689}},\ \Eprint {http://arxiv.org/abs/arXiv:1008.4593}
  {arXiv:1008.4593} \BibitemShut {NoStop}%
\bibitem [{\citenamefont {Ma}\ \emph {et~al.}(2019)\citenamefont {Ma},
  \citenamefont {Zhou}, \citenamefont {Yuan},\ and\ \citenamefont
  {Ma}}]{MZYM19}%
  \BibitemOpen
  \bibfield  {author} {\bibinfo {author} {\bibnamefont {Ma}, \bibfnamefont
  {Jiajun}}, \bibinfo {author} {\bibfnamefont {You}\ \bibnamefont {Zhou}},
  \bibinfo {author} {\bibfnamefont {Xiao}\ \bibnamefont {Yuan}}, \ and\
  \bibinfo {author} {\bibfnamefont {Xiongfeng}\ \bibnamefont {Ma}}} (\bibinfo
  {year} {2019}),\ \bibfield  {title} {\enquote {\bibinfo {title} {Operational
  interpretation of coherence in quantum key distribution},}\ }\href {\doibase
  10.1103/PhysRevA.99.062325} {\bibfield  {journal} {\bibinfo  {journal} {Phys.
  Rev. A}\ }\textbf {\bibinfo {volume} {99}},\ \bibinfo {pages} {062325}},\
  \Eprint {http://arxiv.org/abs/arXiv:1810.03267} {arXiv:1810.03267}
  \BibitemShut {NoStop}%
\bibitem [{\citenamefont {Ma}\ and\ \citenamefont {Razavi}(2012)}]{MR12}%
  \BibitemOpen
  \bibfield  {author} {\bibinfo {author} {\bibnamefont {Ma}, \bibfnamefont
  {Xiongfeng}}, \ and\ \bibinfo {author} {\bibfnamefont {Mohsen}\ \bibnamefont
  {Razavi}}} (\bibinfo {year} {2012}),\ \bibfield  {title} {\enquote {\bibinfo
  {title} {Alternative schemes for measurement-device-independent quantum key
  distribution},}\ }\href {\doibase 10.1103/PhysRevA.86.062319} {\bibfield
  {journal} {\bibinfo  {journal} {Phys. Rev. A}\ }\textbf {\bibinfo {volume}
  {86}},\ \bibinfo {pages} {062319}},\ \Eprint
  {http://arxiv.org/abs/arxiv:1204.4856} {arxiv:1204.4856} \BibitemShut
  {NoStop}%
\bibitem [{\citenamefont {Makarov}(2009)}]{Makarov2009}%
  \BibitemOpen
  \bibfield  {author} {\bibinfo {author} {\bibnamefont {Makarov}, \bibfnamefont
  {Vadim}}} (\bibinfo {year} {2009}),\ \bibfield  {title} {\enquote {\bibinfo
  {title} {Controlling passively quenched single photon detectors by bright
  light},}\ }\href {\doibase 10.1088/1367-2630/11/6/065003} {\bibfield
  {journal} {\bibinfo  {journal} {New J. Phys.}\ }\textbf {\bibinfo {volume}
  {11}}~(\bibinfo {number} {6}),\ \bibinfo {pages} {065003}}\BibitemShut
  {NoStop}%
\bibitem [{\citenamefont {Makarov}\ \emph {et~al.}(2006)\citenamefont
  {Makarov}, \citenamefont {Anisimov},\ and\ \citenamefont
  {Skaar}}]{Makarovetal2006}%
  \BibitemOpen
  \bibfield  {author} {\bibinfo {author} {\bibnamefont {Makarov}, \bibfnamefont
  {Vadim}}, \bibinfo {author} {\bibfnamefont {Andrey}\ \bibnamefont
  {Anisimov}}, \ and\ \bibinfo {author} {\bibfnamefont {Johannes}\ \bibnamefont
  {Skaar}}} (\bibinfo {year} {2006}),\ \bibfield  {title} {\enquote {\bibinfo
  {title} {Effects of detector efficiency mismatch on security of quantum
  cryptosystems},}\ }\href {\doibase 10.1103/PhysRevA.74.022313} {\bibfield
  {journal} {\bibinfo  {journal} {Phys. Rev. A}\ }\textbf {\bibinfo {volume}
  {74}},\ \bibinfo {pages} {022313}}\BibitemShut {NoStop}%
\bibitem [{\citenamefont {Mateus}\ \emph {et~al.}(2003)\citenamefont {Mateus},
  \citenamefont {Mitchell},\ and\ \citenamefont {Scedrov}}]{MMS03}%
  \BibitemOpen
  \bibfield  {author} {\bibinfo {author} {\bibnamefont {Mateus}, \bibfnamefont
  {Paulo}}, \bibinfo {author} {\bibfnamefont {John~C.}\ \bibnamefont
  {Mitchell}}, \ and\ \bibinfo {author} {\bibfnamefont {Andre}\ \bibnamefont
  {Scedrov}}} (\bibinfo {year} {2003}),\ \bibfield  {title} {\enquote {\bibinfo
  {title} {Composition of cryptographic protocols in a probabilistic
  polynomial-time process calculus},}\ }in\ \href {\doibase
  10.1007/978-3-540-45187-7_22} {\emph {\bibinfo {booktitle} {{CONCUR} 2003 --
  Concurrency Theory}}},\ \bibinfo {series} {LNCS}, Vol.\ \bibinfo {volume}
  {2761}\ (\bibinfo  {publisher} {Springer})\ pp.\ \bibinfo {pages}
  {327--349}\BibitemShut {NoStop}%
\bibitem [{\citenamefont {Mauerer}\ \emph {et~al.}(2012)\citenamefont
  {Mauerer}, \citenamefont {Portmann},\ and\ \citenamefont {Scholz}}]{MPS12}%
  \BibitemOpen
  \bibfield  {author} {\bibinfo {author} {\bibnamefont {Mauerer}, \bibfnamefont
  {Wolfgang}}, \bibinfo {author} {\bibfnamefont {Christopher}\ \bibnamefont
  {Portmann}}, \ and\ \bibinfo {author} {\bibfnamefont {Volkher~B.}\
  \bibnamefont {Scholz}}} (\bibinfo {year} {2012}),\ \href@noop {} {\enquote
  {\bibinfo {title} {A modular framework for randomness extraction based on
  trevisan's construction},}\ }\bibinfo {howpublished} {e-Print},\ \Eprint
  {http://arxiv.org/abs/arXiv:1212.0520} {arXiv:1212.0520} \BibitemShut
  {NoStop}%
\bibitem [{\citenamefont {Maurer}(1993)}]{Mau93}%
  \BibitemOpen
  \bibfield  {author} {\bibinfo {author} {\bibnamefont {Maurer}, \bibfnamefont
  {Ueli}}} (\bibinfo {year} {1993}),\ \bibfield  {title} {\enquote {\bibinfo
  {title} {Secret key agreement by public discussion},}\ }\href {\doibase
  10.1109/18.256484} {\bibfield  {journal} {\bibinfo  {journal} {IEEE Trans.
  Inf. Theory}\ }\textbf {\bibinfo {volume} {39}}~(\bibinfo {number} {3}),\
  \bibinfo {pages} {733--742}},\ \bibinfo {note} {a preliminary version
  appeared at CRYPTO~'92}\BibitemShut {NoStop}%
\bibitem [{\citenamefont {Maurer}(1994)}]{Mau94}%
  \BibitemOpen
  \bibfield  {author} {\bibinfo {author} {\bibnamefont {Maurer}, \bibfnamefont
  {Ueli}}} (\bibinfo {year} {1994}),\ \enquote {\bibinfo {title} {The strong
  secret key rate of discrete random triples},}\ in\ \href {\doibase
  10.1007/978-1-4615-2694-0_27} {\emph {\bibinfo {booktitle} {Communications
  and Cryptography: Two Sides of One Tapestry}}},\ \bibinfo {series} {The
  Springer International Series in Engineering and Computer Science}, Vol.\
  \bibinfo {volume} {276}\ (\bibinfo  {publisher} {Springer})\ pp.\ \bibinfo
  {pages} {271--285}\BibitemShut {NoStop}%
\bibitem [{\citenamefont {Maurer}(2002)}]{Mau02}%
  \BibitemOpen
  \bibfield  {author} {\bibinfo {author} {\bibnamefont {Maurer}, \bibfnamefont
  {Ueli}}} (\bibinfo {year} {2002}),\ \bibfield  {title} {\enquote {\bibinfo
  {title} {Indistinguishability of random systems},}\ }in\ \href {\doibase
  10.1007/3-540-46035-7_8} {\emph {\bibinfo {booktitle} {Advances in Cryptology
  -- EUROCRYPT 2002}}},\ \bibinfo {series} {LNCS}, Vol.\ \bibinfo {volume}
  {2332}\ (\bibinfo  {publisher} {Springer})\ pp.\ \bibinfo {pages}
  {110--132}\BibitemShut {NoStop}%
\bibitem [{\citenamefont {Maurer}(2012)}]{Mau12}%
  \BibitemOpen
  \bibfield  {author} {\bibinfo {author} {\bibnamefont {Maurer}, \bibfnamefont
  {Ueli}}} (\bibinfo {year} {2012}),\ \bibfield  {title} {\enquote {\bibinfo
  {title} {Constructive cryptography---a new paradigm for security definitions
  and proofs},}\ }in\ \href {\doibase 10.1007/978-3-642-27375-9_3} {\emph
  {\bibinfo {booktitle} {Proceedings of Theory of Security and Applications,
  TOSCA 2011}}},\ \bibinfo {series} {LNCS}, Vol.\ \bibinfo {volume} {6993}\
  (\bibinfo  {publisher} {Springer})\ pp.\ \bibinfo {pages}
  {33--56}\BibitemShut {NoStop}%
\bibitem [{\citenamefont {Maurer}\ \emph {et~al.}(2007)\citenamefont {Maurer},
  \citenamefont {Pietrzak},\ and\ \citenamefont {Renner}}]{MPR07}%
  \BibitemOpen
  \bibfield  {author} {\bibinfo {author} {\bibnamefont {Maurer}, \bibfnamefont
  {Ueli}}, \bibinfo {author} {\bibfnamefont {Krzysztof}\ \bibnamefont
  {Pietrzak}}, \ and\ \bibinfo {author} {\bibfnamefont {Renato}\ \bibnamefont
  {Renner}}} (\bibinfo {year} {2007}),\ \bibfield  {title} {\enquote {\bibinfo
  {title} {Indistinguishability amplification},}\ }in\ \href {\doibase
  10.1007/978-3-540-74143-5_8} {\emph {\bibinfo {booktitle} {Advances in
  Cryptology -- CRYPTO 2007}}},\ \bibinfo {series} {LNCS}, Vol.\ \bibinfo
  {volume} {4622}\ (\bibinfo  {publisher} {Springer})\ pp.\ \bibinfo {pages}
  {130--149}\BibitemShut {NoStop}%
\bibitem [{\citenamefont {Maurer}\ and\ \citenamefont {Renner}(2011)}]{MR11}%
  \BibitemOpen
  \bibfield  {author} {\bibinfo {author} {\bibnamefont {Maurer}, \bibfnamefont
  {Ueli}}, \ and\ \bibinfo {author} {\bibfnamefont {Renato}\ \bibnamefont
  {Renner}}} (\bibinfo {year} {2011}),\ \bibfield  {title} {\enquote {\bibinfo
  {title} {Abstract cryptography},}\ }in\ \href@noop {} {\emph {\bibinfo
  {booktitle} {Proceedings of Innovations in Computer Science, ICS 2011}}}\
  (\bibinfo  {publisher} {Tsinghua University Press})\ pp.\ \bibinfo {pages}
  {1--21}\BibitemShut {NoStop}%
\bibitem [{\citenamefont {Maurer}\ and\ \citenamefont {Renner}(2016)}]{MR16}%
  \BibitemOpen
  \bibfield  {author} {\bibinfo {author} {\bibnamefont {Maurer}, \bibfnamefont
  {Ueli}}, \ and\ \bibinfo {author} {\bibfnamefont {Renato}\ \bibnamefont
  {Renner}}} (\bibinfo {year} {2016}),\ \bibfield  {title} {\enquote {\bibinfo
  {title} {From indifferentiability to constructive cryptography (and back)},}\
  }in\ \href {\doibase 10.1007/978-3-662-53641-4_1} {\emph {\bibinfo
  {booktitle} {Theory of Cryptography, Proceedings of {TCC} 2016-B, Part
  {I}}}},\ \bibinfo {series} {LNCS}, Vol.\ \bibinfo {volume} {9985}\ (\bibinfo
  {publisher} {Springer})\ pp.\ \bibinfo {pages} {3--24},\ \bibinfo {note}
  {e-Print \href{http://eprint.iacr.org/2016/903}{IACR 2016/903}}\BibitemShut
  {NoStop}%
\bibitem [{\citenamefont {Maurer}\ \emph {et~al.}(2012)\citenamefont {Maurer},
  \citenamefont {R\"uedlinger},\ and\ \citenamefont {Tackmann}}]{MRT12}%
  \BibitemOpen
  \bibfield  {author} {\bibinfo {author} {\bibnamefont {Maurer}, \bibfnamefont
  {Ueli}}, \bibinfo {author} {\bibfnamefont {Andreas}\ \bibnamefont
  {R\"uedlinger}}, \ and\ \bibinfo {author} {\bibfnamefont {Bj\"orn}\
  \bibnamefont {Tackmann}}} (\bibinfo {year} {2012}),\ \bibfield  {title}
  {\enquote {\bibinfo {title} {Confidentiality and integrity: A constructive
  perspective},}\ }in\ \href {\doibase 10.1007/978-3-642-28914-9_12} {\emph
  {\bibinfo {booktitle} {Theory of Cryptography, Proceedings of TCC 2012}}},\
  \bibinfo {series} {LNCS}, Vol.\ \bibinfo {volume} {7194},\ \bibinfo {editor}
  {edited by\ \bibinfo {editor} {\bibfnamefont {Ronald}\ \bibnamefont
  {Cramer}}}\ (\bibinfo  {publisher} {Springer})\ pp.\ \bibinfo {pages}
  {209--229}\BibitemShut {NoStop}%
\bibitem [{\citenamefont {Maurer}\ and\ \citenamefont {Wolf}(2000)}]{MW00}%
  \BibitemOpen
  \bibfield  {author} {\bibinfo {author} {\bibnamefont {Maurer}, \bibfnamefont
  {Ueli}}, \ and\ \bibinfo {author} {\bibfnamefont {Stefan}\ \bibnamefont
  {Wolf}}} (\bibinfo {year} {2000}),\ \bibfield  {title} {\enquote {\bibinfo
  {title} {Information-theoretic key agreement: From weak to strong secrecy for
  free},}\ }in\ \href {\doibase 10.1007/3-540-45539-6_24} {\emph {\bibinfo
  {booktitle} {Advances in Cryptology -- EUROCRYPT 2000}}},\ \bibinfo {series}
  {LNCS}, Vol.\ \bibinfo {volume} {1807}\ (\bibinfo  {publisher} {Springer})\
  pp.\ \bibinfo {pages} {351--368}\BibitemShut {NoStop}%
\bibitem [{\citenamefont {Mayers}(1996)}]{May96}%
  \BibitemOpen
  \bibfield  {author} {\bibinfo {author} {\bibnamefont {Mayers}, \bibfnamefont
  {Dominic}}} (\bibinfo {year} {1996}),\ \bibfield  {title} {\enquote {\bibinfo
  {title} {Quantum key distribution and string oblivious transfer in noisy
  channels},}\ }in\ \href {\doibase 10.1007/3-540-68697-5_26} {\emph {\bibinfo
  {booktitle} {Advances in Cryptology -- CRYPTO~'96}}},\ \bibinfo {series}
  {LNCS}, Vol.\ \bibinfo {volume} {1109}\ (\bibinfo  {publisher} {Springer})\
  pp.\ \bibinfo {pages} {343--357},\ \Eprint
  {http://arxiv.org/abs/arXiv:quant-ph/9606003} {arXiv:quant-ph/9606003}
  \BibitemShut {NoStop}%
\bibitem [{\citenamefont {Mayers}(2001)}]{May01}%
  \BibitemOpen
  \bibfield  {author} {\bibinfo {author} {\bibnamefont {Mayers}, \bibfnamefont
  {Dominic}}} (\bibinfo {year} {2001}),\ \bibfield  {title} {\enquote {\bibinfo
  {title} {Unconditional security in quantum cryptography},}\ }\href {\doibase
  10.1145/382780.382781} {\bibfield  {journal} {\bibinfo  {journal} {J. ACM}\
  }\textbf {\bibinfo {volume} {48}}~(\bibinfo {number} {3}),\ \bibinfo {pages}
  {351--406}},\ \Eprint {http://arxiv.org/abs/arXiv:quant-ph/9802025}
  {arXiv:quant-ph/9802025} \BibitemShut {NoStop}%
\bibitem [{\citenamefont {Micali}\ and\ \citenamefont {Rogaway}(1992)}]{MR92}%
  \BibitemOpen
  \bibfield  {author} {\bibinfo {author} {\bibnamefont {Micali}, \bibfnamefont
  {Silvio}}, \ and\ \bibinfo {author} {\bibfnamefont {Phillip}\ \bibnamefont
  {Rogaway}}} (\bibinfo {year} {1992}),\ \bibfield  {title} {\enquote {\bibinfo
  {title} {Secure computation (abstract)},}\ }in\ \href {\doibase
  10.1007/3-540-46766-1_32} {\emph {\bibinfo {booktitle} {Advances in
  Cryptology -- CRYPTO~'91}}},\ \bibinfo {series} {LNCS}, Vol.\ \bibinfo
  {volume} {576}\ (\bibinfo  {publisher} {Springer})\ pp.\ \bibinfo {pages}
  {392--404}\BibitemShut {NoStop}%
\bibitem [{\citenamefont {Miller}\ and\ \citenamefont {Shi}(2014)}]{MS14}%
  \BibitemOpen
  \bibfield  {author} {\bibinfo {author} {\bibnamefont {Miller}, \bibfnamefont
  {Carl}}, \ and\ \bibinfo {author} {\bibfnamefont {Yaoyun}\ \bibnamefont
  {Shi}}} (\bibinfo {year} {2014}),\ \bibfield  {title} {\enquote {\bibinfo
  {title} {Robust protocols for securely expanding randomness and distributing
  keys using untrusted quantum devices},}\ }in\ \href {\doibase
  10.1145/2591796.2591843} {\emph {\bibinfo {booktitle} {Proceedings of the
  46th Symposium on Theory of Computing, STOC~'14}}}\ (\bibinfo  {publisher}
  {ACM})\ pp.\ \bibinfo {pages} {417--426},\ \Eprint
  {http://arxiv.org/abs/arXiv:1402.0489} {arXiv:1402.0489} \BibitemShut
  {NoStop}%
\bibitem [{\citenamefont {Mitchell}\ \emph {et~al.}(2006)\citenamefont
  {Mitchell}, \citenamefont {Ramanathan}, \citenamefont {Scedrov},\ and\
  \citenamefont {Teague}}]{MRST06}%
  \BibitemOpen
  \bibfield  {author} {\bibinfo {author} {\bibnamefont {Mitchell},
  \bibfnamefont {John~C}}, \bibinfo {author} {\bibfnamefont {Ajith}\
  \bibnamefont {Ramanathan}}, \bibinfo {author} {\bibfnamefont {Andre}\
  \bibnamefont {Scedrov}}, \ and\ \bibinfo {author} {\bibfnamefont {Vanessa}\
  \bibnamefont {Teague}}} (\bibinfo {year} {2006}),\ \bibfield  {title}
  {\enquote {\bibinfo {title} {A probabilistic polynomial-time process calculus
  for the analysis of cryptographic protocols},}\ }\href {\doibase
  10.1016/j.tcs.2005.10.044} {\bibfield  {journal} {\bibinfo  {journal} {Theor.
  Comput. Sci.}\ }\textbf {\bibinfo {volume} {353}}~(\bibinfo {number}
  {1–3}),\ \bibinfo {pages} {118--164}}\BibitemShut {NoStop}%
\bibitem [{\citenamefont {Muller}\ \emph {et~al.}(1997)\citenamefont {Muller},
  \citenamefont {Herzog}, \citenamefont {Huttner}, \citenamefont {Tittel},
  \citenamefont {Zbinden},\ and\ \citenamefont {Gisin}}]{Muller97}%
  \BibitemOpen
  \bibfield  {author} {\bibinfo {author} {\bibnamefont {Muller}, \bibfnamefont
  {Antoine}}, \bibinfo {author} {\bibfnamefont {Thomas}\ \bibnamefont
  {Herzog}}, \bibinfo {author} {\bibfnamefont {Bruno}\ \bibnamefont {Huttner}},
  \bibinfo {author} {\bibfnamefont {Woflgang}\ \bibnamefont {Tittel}}, \bibinfo
  {author} {\bibfnamefont {Hugo}\ \bibnamefont {Zbinden}}, \ and\ \bibinfo
  {author} {\bibfnamefont {Nicolas}\ \bibnamefont {Gisin}}} (\bibinfo {year}
  {1997}),\ \bibfield  {title} {\enquote {\bibinfo {title} {``plug and play''
  systems for quantum cryptography},}\ }\href {\doibase
  https://doi.org/10.1063/1.118224} {\bibfield  {journal} {\bibinfo  {journal}
  {Appl. Phys. Lett.}\ }\textbf {\bibinfo {volume} {70}}~(\bibinfo {number}
  {7}),\ \bibinfo {pages} {793--795}}\BibitemShut {NoStop}%
\bibitem [{\citenamefont {M\"uller-Quade}\ and\ \citenamefont
  {Renner}(2009)}]{MR09}%
  \BibitemOpen
  \bibfield  {author} {\bibinfo {author} {\bibnamefont {M\"uller-Quade},
  \bibfnamefont {J\"orn}}, \ and\ \bibinfo {author} {\bibfnamefont {Renato}\
  \bibnamefont {Renner}}} (\bibinfo {year} {2009}),\ \bibfield  {title}
  {\enquote {\bibinfo {title} {Composability in quantum cryptography},}\ }\href
  {\doibase 10.1088/1367-2630/11/8/085006} {\bibfield  {journal} {\bibinfo
  {journal} {New J. Phys.}\ }\textbf {\bibinfo {volume} {11}}~(\bibinfo
  {number} {8}),\ \bibinfo {pages} {085006}},\ \Eprint
  {http://arxiv.org/abs/arXiv:1006.2215} {arXiv:1006.2215} \BibitemShut
  {NoStop}%
\bibitem [{\citenamefont {Nielsen}\ and\ \citenamefont
  {Chuang}(2010)}]{nielsen2010quantum}%
  \BibitemOpen
  \bibfield  {author} {\bibinfo {author} {\bibnamefont {Nielsen}, \bibfnamefont
  {Michael~A}}, \ and\ \bibinfo {author} {\bibfnamefont {Isaac~L}\ \bibnamefont
  {Chuang}}} (\bibinfo {year} {2010}),\ \href@noop {} {\emph {\bibinfo {title}
  {Quantum Computation and Quantum Information}}}\ (\bibinfo  {publisher}
  {Cambridge University Press})\BibitemShut {NoStop}%
\bibitem [{\citenamefont {Nishioka}\ \emph {et~al.}(2002)\citenamefont
  {Nishioka}, \citenamefont {Ishizuka}, \citenamefont {Toshio},\ and\
  \citenamefont {Abe}}]{Nishioka2002}%
  \BibitemOpen
  \bibfield  {author} {\bibinfo {author} {\bibnamefont {Nishioka},
  \bibfnamefont {Tsuyoshi}}, \bibinfo {author} {\bibfnamefont {Hirokazu}\
  \bibnamefont {Ishizuka}}, \bibinfo {author} {\bibnamefont {Toshio}}, \ and\
  \bibinfo {author} {\bibfnamefont {Junichi}\ \bibnamefont {Abe}}} (\bibinfo
  {year} {2002}),\ \bibfield  {title} {\enquote {\bibinfo {title} {``circular
  type'' quantum key distribution},}\ }\href {\doibase 10.1109/68.992616}
  {\bibfield  {journal} {\bibinfo  {journal} {IEEE Photonics Technol. Lett.}\
  }\textbf {\bibinfo {volume} {14}}~(\bibinfo {number} {4}),\ \bibinfo {pages}
  {576--578}},\ \Eprint {http://arxiv.org/abs/arXiv:quant-ph/0106083}
  {arXiv:quant-ph/0106083} \BibitemShut {NoStop}%
\bibitem [{\citenamefont {Ohya}\ and\ \citenamefont {Petz}(1993)}]{OP93}%
  \BibitemOpen
  \bibfield  {author} {\bibinfo {author} {\bibnamefont {Ohya}, \bibfnamefont
  {Masanori}}, \ and\ \bibinfo {author} {\bibfnamefont {D\'enes}\ \bibnamefont
  {Petz}}} (\bibinfo {year} {1993}),\ \href@noop {} {\emph {\bibinfo {title}
  {Quantum Entropy and Its Use}}}\ (\bibinfo  {publisher}
  {Springer})\BibitemShut {NoStop}%
\bibitem [{\citenamefont {Paw\l{}owski}\ and\ \citenamefont
  {Brunner}(2011)}]{PB11}%
  \BibitemOpen
  \bibfield  {author} {\bibinfo {author} {\bibnamefont {Paw\l{}owski},
  \bibfnamefont {Marcin}}, \ and\ \bibinfo {author} {\bibfnamefont {Nicolas}\
  \bibnamefont {Brunner}}} (\bibinfo {year} {2011}),\ \bibfield  {title}
  {\enquote {\bibinfo {title} {Semi-device-independent security of one-way
  quantum key distribution},}\ }\href {\doibase 10.1103/PhysRevA.84.010302}
  {\bibfield  {journal} {\bibinfo  {journal} {Phys. Rev. A}\ }\textbf {\bibinfo
  {volume} {84}},\ \bibinfo {pages} {010302}},\ \Eprint
  {http://arxiv.org/abs/arXiv:1103.4105} {arXiv:1103.4105} \BibitemShut
  {NoStop}%
\bibitem [{\citenamefont {Peres}\ and\ \citenamefont
  {Terno}(2004)}]{Peres_Terno_RMP}%
  \BibitemOpen
  \bibfield  {author} {\bibinfo {author} {\bibnamefont {Peres}, \bibfnamefont
  {Asher}}, \ and\ \bibinfo {author} {\bibfnamefont {Daniel~R.}\ \bibnamefont
  {Terno}}} (\bibinfo {year} {2004}),\ \bibfield  {title} {\enquote {\bibinfo
  {title} {Quantum information and relativity theory},}\ }\href {\doibase
  10.1103/RevModPhys.76.93} {\bibfield  {journal} {\bibinfo  {journal} {Rev.
  Mod. Phys.}\ }\textbf {\bibinfo {volume} {76}},\ \bibinfo {pages}
  {93--123}}\BibitemShut {NoStop}%
\bibitem [{\citenamefont {Pfitzmann}\ and\ \citenamefont
  {Waidner}(2000)}]{PW00}%
  \BibitemOpen
  \bibfield  {author} {\bibinfo {author} {\bibnamefont {Pfitzmann},
  \bibfnamefont {Birgit}}, \ and\ \bibinfo {author} {\bibfnamefont {Michael}\
  \bibnamefont {Waidner}}} (\bibinfo {year} {2000}),\ \bibfield  {title}
  {\enquote {\bibinfo {title} {Composition and integrity preservation of secure
  reactive systems},}\ }in\ \href {\doibase 10.1145/352600.352639} {\emph
  {\bibinfo {booktitle} {Proceedings of the 7th ACM Conference on Computer and
  Communications Security, CSS~'00}}}\ (\bibinfo  {publisher} {ACM})\ pp.\
  \bibinfo {pages} {245--254}\BibitemShut {NoStop}%
\bibitem [{\citenamefont {Pfitzmann}\ and\ \citenamefont
  {Waidner}(2001)}]{PW01}%
  \BibitemOpen
  \bibfield  {author} {\bibinfo {author} {\bibnamefont {Pfitzmann},
  \bibfnamefont {Birgit}}, \ and\ \bibinfo {author} {\bibfnamefont {Michael}\
  \bibnamefont {Waidner}}} (\bibinfo {year} {2001}),\ \bibfield  {title}
  {\enquote {\bibinfo {title} {A model for asynchronous reactive systems and
  its application to secure message transmission},}\ }in\ \href {\doibase
  10.1109/SECPRI.2001.924298} {\emph {\bibinfo {booktitle} {IEEE Symposium on
  Security and Privacy}}}\ (\bibinfo  {publisher} {IEEE})\ pp.\ \bibinfo
  {pages} {184--200}\BibitemShut {NoStop}%
\bibitem [{\citenamefont {Pirandola}\ \emph {et~al.}(2015)\citenamefont
  {Pirandola}, \citenamefont {Ottaviani}, \citenamefont {Spedalieri},
  \citenamefont {Weedbrook}, \citenamefont {Braunstein}, \citenamefont {Lloyd},
  \citenamefont {Gehring}, \citenamefont {Jacobsen},\ and\ \citenamefont
  {Andersen}}]{Pirandola2015}%
  \BibitemOpen
  \bibfield  {author} {\bibinfo {author} {\bibnamefont {Pirandola},
  \bibfnamefont {Stefano}}, \bibinfo {author} {\bibfnamefont {Carlo}\
  \bibnamefont {Ottaviani}}, \bibinfo {author} {\bibfnamefont {Gaetana}\
  \bibnamefont {Spedalieri}}, \bibinfo {author} {\bibfnamefont {Christian}\
  \bibnamefont {Weedbrook}}, \bibinfo {author} {\bibfnamefont {Samuel~L.}\
  \bibnamefont {Braunstein}}, \bibinfo {author} {\bibfnamefont {Seth}\
  \bibnamefont {Lloyd}}, \bibinfo {author} {\bibfnamefont {Tobias}\
  \bibnamefont {Gehring}}, \bibinfo {author} {\bibfnamefont {Christian~S.}\
  \bibnamefont {Jacobsen}}, \ and\ \bibinfo {author} {\bibfnamefont {Ulrik~L.}\
  \bibnamefont {Andersen}}} (\bibinfo {year} {2015}),\ \bibfield  {title}
  {\enquote {\bibinfo {title} {High-rate measurement-device-independent quantum
  cryptography},}\ }\href {\doibase 10.1038/nphoton.2015.83} {\bibfield
  {journal} {\bibinfo  {journal} {Nat. Photonics}\ }\textbf {\bibinfo {volume}
  {9}},\ \bibinfo {pages} {397}}\BibitemShut {NoStop}%
\bibitem [{\citenamefont {Pironio}\ \emph {et~al.}(2009)\citenamefont
  {Pironio}, \citenamefont {Ac\'in}, \citenamefont {Brunner}, \citenamefont
  {Gisin}, \citenamefont {Massar},\ and\ \citenamefont {Scarani}}]{PABGMS09}%
  \BibitemOpen
  \bibfield  {author} {\bibinfo {author} {\bibnamefont {Pironio}, \bibfnamefont
  {Stefano}}, \bibinfo {author} {\bibfnamefont {Antonio}\ \bibnamefont
  {Ac\'in}}, \bibinfo {author} {\bibfnamefont {Nicolas}\ \bibnamefont
  {Brunner}}, \bibinfo {author} {\bibfnamefont {Nicolas}\ \bibnamefont
  {Gisin}}, \bibinfo {author} {\bibfnamefont {Serge}\ \bibnamefont {Massar}}, \
  and\ \bibinfo {author} {\bibfnamefont {Valerio}\ \bibnamefont {Scarani}}}
  (\bibinfo {year} {2009}),\ \bibfield  {title} {\enquote {\bibinfo {title}
  {Device-independent quantum key distribution secure against collective
  attacks},}\ }\href {\doibase 10.1088/1367-2630/11/4/045021} {\bibfield
  {journal} {\bibinfo  {journal} {New J. Phys.}\ }\textbf {\bibinfo {volume}
  {11}}~(\bibinfo {number} {4}),\ \bibinfo {pages} {045021}},\ \Eprint
  {http://arxiv.org/abs/arXiv:0903.4460} {arXiv:0903.4460} \BibitemShut
  {NoStop}%
\bibitem [{\citenamefont {Pironio}\ \emph {et~al.}(2010)\citenamefont
  {Pironio}, \citenamefont {Ac\'in}, \citenamefont {Massar}, \citenamefont
  {de~La~Giroday}, \citenamefont {Matsukevich}, \citenamefont {Maunz},
  \citenamefont {Olmschenk}, \citenamefont {Hayes}, \citenamefont {Luo},\ and\
  \citenamefont {Manning}}]{PAM10}%
  \BibitemOpen
  \bibfield  {author} {\bibinfo {author} {\bibnamefont {Pironio}, \bibfnamefont
  {Stefano}}, \bibinfo {author} {\bibfnamefont {Antonio}\ \bibnamefont
  {Ac\'in}}, \bibinfo {author} {\bibfnamefont {Serge}\ \bibnamefont {Massar}},
  \bibinfo {author} {\bibfnamefont {A~Boyer}\ \bibnamefont {de~La~Giroday}},
  \bibinfo {author} {\bibfnamefont {Dzimitry~N}\ \bibnamefont {Matsukevich}},
  \bibinfo {author} {\bibfnamefont {Peter}\ \bibnamefont {Maunz}}, \bibinfo
  {author} {\bibfnamefont {Steven}\ \bibnamefont {Olmschenk}}, \bibinfo
  {author} {\bibfnamefont {David}\ \bibnamefont {Hayes}}, \bibinfo {author}
  {\bibfnamefont {Le}~\bibnamefont {Luo}}, \ and\ \bibinfo {author}
  {\bibfnamefont {T~Andrew}\ \bibnamefont {Manning}}} (\bibinfo {year}
  {2010}),\ \bibfield  {title} {\enquote {\bibinfo {title} {Random numbers
  certified by {Bell}'s theorem},}\ }\href {\doibase 10.1038/nature09008}
  {\bibfield  {journal} {\bibinfo  {journal} {Nature}\ }\textbf {\bibinfo
  {volume} {464}}~(\bibinfo {number} {7291}),\ \bibinfo {pages} {1021--1024}},\
  \Eprint {http://arxiv.org/abs/arXiv:0911.3427} {arXiv:0911.3427} \BibitemShut
  {NoStop}%
\bibitem [{\citenamefont {Portmann}(2014)}]{Por14}%
  \BibitemOpen
  \bibfield  {author} {\bibinfo {author} {\bibnamefont {Portmann},
  \bibfnamefont {Christopher}}} (\bibinfo {year} {2014}),\ \bibfield  {title}
  {\enquote {\bibinfo {title} {Key recycling in authentication},}\ }\href
  {\doibase 10.1109/TIT.2014.2317312} {\bibfield  {journal} {\bibinfo
  {journal} {IEEE Trans. Inf. Theory}\ }\textbf {\bibinfo {volume}
  {60}}~(\bibinfo {number} {7}),\ \bibinfo {pages} {4383--4396}},\ \Eprint
  {http://arxiv.org/abs/arXiv:1202.1229} {arXiv:1202.1229} \BibitemShut
  {NoStop}%
\bibitem [{\citenamefont {Portmann}(2017{\natexlab{a}})}]{Por17}%
  \BibitemOpen
  \bibfield  {author} {\bibinfo {author} {\bibnamefont {Portmann},
  \bibfnamefont {Christopher}}} (\bibinfo {year} {2017}{\natexlab{a}}),\
  \bibfield  {title} {\enquote {\bibinfo {title} {Quantum authentication with
  key recycling},}\ }in\ \href {\doibase 10.1007/978-3-319-56617-7_12} {\emph
  {\bibinfo {booktitle} {Advances in Cryptology -- {EUROCRYPT} 2017,
  Proceedings, Part {III}}}},\ \bibinfo {series} {LNCS}, Vol.\ \bibinfo
  {volume} {10212}\ (\bibinfo  {publisher} {Springer})\ pp.\ \bibinfo {pages}
  {339--368},\ \Eprint {http://arxiv.org/abs/arXiv:1610.03422}
  {arXiv:1610.03422} \BibitemShut {NoStop}%
\bibitem [{\citenamefont {Portmann}(2017{\natexlab{b}})}]{Portmann2017}%
  \BibitemOpen
  \bibfield  {author} {\bibinfo {author} {\bibnamefont {Portmann},
  \bibfnamefont {Christopher}}} (\bibinfo {year} {2017}{\natexlab{b}}),\
  \href@noop {} {\enquote {\bibinfo {title} {({Quantum}) {Min}-entropy
  resources},}\ }\bibinfo {howpublished} {e-Print},\ \Eprint
  {http://arxiv.org/abs/arXiv:1705.10595} {arXiv:1705.10595} \BibitemShut
  {NoStop}%
\bibitem [{\citenamefont {Portmann}\ \emph {et~al.}(2017)\citenamefont
  {Portmann}, \citenamefont {Matt}, \citenamefont {Maurer}, \citenamefont
  {Renner},\ and\ \citenamefont {Tackmann}}]{PMMRT17}%
  \BibitemOpen
  \bibfield  {author} {\bibinfo {author} {\bibnamefont {Portmann},
  \bibfnamefont {Christopher}}, \bibinfo {author} {\bibfnamefont {Christian}\
  \bibnamefont {Matt}}, \bibinfo {author} {\bibfnamefont {Ueli}\ \bibnamefont
  {Maurer}}, \bibinfo {author} {\bibfnamefont {Renato}\ \bibnamefont {Renner}},
  \ and\ \bibinfo {author} {\bibfnamefont {Bj\"orn}\ \bibnamefont {Tackmann}}}
  (\bibinfo {year} {2017}),\ \bibfield  {title} {\enquote {\bibinfo {title}
  {Causal boxes: Quantum information-processing systems closed under
  composition},}\ }\href {\doibase 10.1109/TIT.2017.2676805} {\bibfield
  {journal} {\bibinfo  {journal} {IEEE Transactions on Information Theory}\
  }\textbf {\bibinfo {volume} {63}}~(\bibinfo {number} {5}),\ \bibinfo {pages}
  {3277--3305}},\ \Eprint {http://arxiv.org/abs/arXiv:1512.02240}
  {arXiv:1512.02240} \BibitemShut {NoStop}%
\bibitem [{\citenamefont {Prokop}(2020)}]{Pro20}%
  \BibitemOpen
  \bibfield  {author} {\bibinfo {author} {\bibnamefont {Prokop}, \bibfnamefont
  {Mil\v{o}s}}} (\bibinfo {year} {2020}),\ \href
  {https://project-archive.inf.ed.ac.uk/ug4/20201685/ug4_proj.pdf} {\enquote
  {\bibinfo {title} {Composable security of quantum bit commitment protocol},}\
  }\bibinfo {howpublished} {e-print}\BibitemShut {NoStop}%
\bibitem [{\citenamefont {Qi}\ \emph {et~al.}(2007)\citenamefont {Qi},
  \citenamefont {Fung}, \citenamefont {Lo},\ and\ \citenamefont
  {Ma}}]{qi2007time}%
  \BibitemOpen
  \bibfield  {author} {\bibinfo {author} {\bibnamefont {Qi}, \bibfnamefont
  {Bing}}, \bibinfo {author} {\bibfnamefont {Chi-Hang~Fred}\ \bibnamefont
  {Fung}}, \bibinfo {author} {\bibfnamefont {Hoi-Kwong}\ \bibnamefont {Lo}}, \
  and\ \bibinfo {author} {\bibfnamefont {Xiongfeng}\ \bibnamefont {Ma}}}
  (\bibinfo {year} {2007}),\ \bibfield  {title} {\enquote {\bibinfo {title}
  {Time-shift attack in practical quantum cryptosystems},}\ }\href@noop {}
  {\bibfield  {journal} {\bibinfo  {journal} {Quantum Inf. Comput.}\ }\textbf
  {\bibinfo {volume} {7}}~(\bibinfo {number} {1}),\ \bibinfo {pages}
  {73--82}}\BibitemShut {NoStop}%
\bibitem [{\citenamefont {Reichardt}\ \emph {et~al.}(2013)\citenamefont
  {Reichardt}, \citenamefont {Unger},\ and\ \citenamefont {Vazirani}}]{RUV13}%
  \BibitemOpen
  \bibfield  {author} {\bibinfo {author} {\bibnamefont {Reichardt},
  \bibfnamefont {Ben~W}}, \bibinfo {author} {\bibfnamefont {Falk}\ \bibnamefont
  {Unger}}, \ and\ \bibinfo {author} {\bibfnamefont {Umesh}\ \bibnamefont
  {Vazirani}}} (\bibinfo {year} {2013}),\ \bibfield  {title} {\enquote
  {\bibinfo {title} {Classical command of quantum systems},}\ }\href {\doibase
  10.1038/nature12035} {\bibfield  {journal} {\bibinfo  {journal} {Nature}\
  }\textbf {\bibinfo {volume} {496}},\ \bibinfo {pages} {456--460}},\ \bibinfo
  {note} {full version available on arXiv},\ \Eprint
  {http://arxiv.org/abs/arXiv:1209.0448} {arXiv:1209.0448} \BibitemShut
  {NoStop}%
\bibitem [{\citenamefont {Renes}(2013)}]{Renes2013}%
  \BibitemOpen
  \bibfield  {author} {\bibinfo {author} {\bibnamefont {Renes}, \bibfnamefont
  {Joseph~M}}} (\bibinfo {year} {2013}),\ \bibfield  {title} {\enquote
  {\bibinfo {title} {The physics of quantum information: Complementarity,
  uncertainty, and entanglement},}\ }\href@noop {} {\bibfield  {journal}
  {\bibinfo  {journal} {Int. J. Quantum Inf.}\ }\textbf {\bibinfo {volume}
  {11}}~(\bibinfo {number} {08}),\ \bibinfo {pages} {1330002}}\BibitemShut
  {NoStop}%
\bibitem [{\citenamefont {Renes}\ and\ \citenamefont
  {Renner}(2012)}]{RenRen12}%
  \BibitemOpen
  \bibfield  {author} {\bibinfo {author} {\bibnamefont {Renes}, \bibfnamefont
  {Joseph~M}}, \ and\ \bibinfo {author} {\bibfnamefont {Renato}\ \bibnamefont
  {Renner}}} (\bibinfo {year} {2012}),\ \bibfield  {title} {\enquote {\bibinfo
  {title} {One-shot classical data compression with quantum side information
  and the distillation of common randomness or secret keys},}\ }\href {\doibase
  10.1109/TIT.2011.2177589} {\bibfield  {journal} {\bibinfo  {journal} {IEEE
  Trans. Inf. Theory}\ }\textbf {\bibinfo {volume} {58}}~(\bibinfo {number}
  {3}),\ \bibinfo {pages} {1985--1991}}\BibitemShut {NoStop}%
\bibitem [{\citenamefont {Renes}\ and\ \citenamefont
  {Renner}(2020)}]{RenesRenner2020}%
  \BibitemOpen
  \bibfield  {author} {\bibinfo {author} {\bibnamefont {Renes}, \bibfnamefont
  {Joseph~M}}, \ and\ \bibinfo {author} {\bibfnamefont {Renato}\ \bibnamefont
  {Renner}}} (\bibinfo {year} {2020}),\ \href@noop {} {\enquote {\bibinfo
  {title} {Are quantum cryptographic security claims vacuous?}}\ }\bibinfo
  {howpublished} {e-Print},\ \Eprint {http://arxiv.org/abs/arXiv:2010.11961}
  {arXiv:2010.11961} \BibitemShut {NoStop}%
\bibitem [{\citenamefont {Renner}(2005)}]{Ren05}%
  \BibitemOpen
  \bibfield  {author} {\bibinfo {author} {\bibnamefont {Renner}, \bibfnamefont
  {Renato}}} (\bibinfo {year} {2005}),\ \emph {\bibinfo {title} {Security of
  Quantum Key Distribution}},\ \href@noop {} {Ph.D. thesis}\ (\bibinfo
  {school} {Swiss Federal Institute of Technology (ETH) Zurich}),\ \Eprint
  {http://arxiv.org/abs/arXiv:quant-ph/0512258} {arXiv:quant-ph/0512258}
  \BibitemShut {NoStop}%
\bibitem [{\citenamefont {Renner}(2007)}]{Ren07}%
  \BibitemOpen
  \bibfield  {author} {\bibinfo {author} {\bibnamefont {Renner}, \bibfnamefont
  {Renato}}} (\bibinfo {year} {2007}),\ \bibfield  {title} {\enquote {\bibinfo
  {title} {Symmetry of large physical systems implies independence of
  subsystems},}\ }\href {\doibase 10.1038/nphys684} {\bibfield  {journal}
  {\bibinfo  {journal} {Nat. Phys.}\ }\textbf {\bibinfo {volume} {3}}~(\bibinfo
  {number} {9}),\ \bibinfo {pages} {645--649}},\ \Eprint
  {http://arxiv.org/abs/arXiv:quant-ph/0703069} {arXiv:quant-ph/0703069}
  \BibitemShut {NoStop}%
\bibitem [{\citenamefont {Renner}\ \emph {et~al.}(2005)\citenamefont {Renner},
  \citenamefont {Gisin},\ and\ \citenamefont {Kraus}}]{RGK05}%
  \BibitemOpen
  \bibfield  {author} {\bibinfo {author} {\bibnamefont {Renner}, \bibfnamefont
  {Renato}}, \bibinfo {author} {\bibfnamefont {Nicolas}\ \bibnamefont {Gisin}},
  \ and\ \bibinfo {author} {\bibfnamefont {Barbara}\ \bibnamefont {Kraus}}}
  (\bibinfo {year} {2005}),\ \bibfield  {title} {\enquote {\bibinfo {title}
  {Information-theoretic security proof for quantum-key-distribution
  protocols},}\ }\href {\doibase 10.1103/PhysRevA.72.012332} {\bibfield
  {journal} {\bibinfo  {journal} {Phys. Rev. A}\ }\textbf {\bibinfo {volume}
  {72}},\ \bibinfo {pages} {012332}},\ \Eprint
  {http://arxiv.org/abs/arXiv:quant-ph/0502064} {arXiv:quant-ph/0502064}
  \BibitemShut {NoStop}%
\bibitem [{\citenamefont {Renner}\ and\ \citenamefont {K\"onig}(2005)}]{RK05}%
  \BibitemOpen
  \bibfield  {author} {\bibinfo {author} {\bibnamefont {Renner}, \bibfnamefont
  {Renato}}, \ and\ \bibinfo {author} {\bibfnamefont {Robert}\ \bibnamefont
  {K\"onig}}} (\bibinfo {year} {2005}),\ \bibfield  {title} {\enquote {\bibinfo
  {title} {Universally composable privacy amplification against quantum
  adversaries},}\ }in\ \href {\doibase 10.1007/978-3-540-30576-7_22} {\emph
  {\bibinfo {booktitle} {Theory of Cryptography, Proceedings of TCC 2005}}},\
  \bibinfo {series} {LNCS}, Vol.\ \bibinfo {volume} {3378},\ \bibinfo {editor}
  {edited by\ \bibinfo {editor} {\bibfnamefont {Joe}\ \bibnamefont {Kilian}}}\
  (\bibinfo  {publisher} {Springer})\ pp.\ \bibinfo {pages} {407--425},\
  \Eprint {http://arxiv.org/abs/arXiv:quant-ph/0403133}
  {arXiv:quant-ph/0403133} \BibitemShut {NoStop}%
\bibitem [{\citenamefont {Renner}\ and\ \citenamefont {Wolf}(2003)}]{RW03}%
  \BibitemOpen
  \bibfield  {author} {\bibinfo {author} {\bibnamefont {Renner}, \bibfnamefont
  {Renato}}, \ and\ \bibinfo {author} {\bibfnamefont {Stefan}\ \bibnamefont
  {Wolf}}} (\bibinfo {year} {2003}),\ \bibfield  {title} {\enquote {\bibinfo
  {title} {Unconditional authenticity and privacy from an arbitrarily weak
  secret},}\ }in\ \href {\doibase 10.1007/978-3-540-45146-4_5} {\emph {\bibinfo
  {booktitle} {Advances in Cryptology -- CRYPTO 2003}}},\ \bibinfo {series}
  {LNCS}, Vol.\ \bibinfo {volume} {2729}\ (\bibinfo  {publisher} {Springer})\
  pp.\ \bibinfo {pages} {78--95}\BibitemShut {NoStop}%
\bibitem [{\citenamefont {Renner}\ and\ \citenamefont {Wolf}(2005)}]{RW05}%
  \BibitemOpen
  \bibfield  {author} {\bibinfo {author} {\bibnamefont {Renner}, \bibfnamefont
  {Renato}}, \ and\ \bibinfo {author} {\bibfnamefont {Stefan}\ \bibnamefont
  {Wolf}}} (\bibinfo {year} {2005}),\ \bibfield  {title} {\enquote {\bibinfo
  {title} {Simple and tight bounds for information reconciliation and privacy
  amplification},}\ }in\ \href {\doibase 10.1007/11593447_11} {\emph {\bibinfo
  {booktitle} {Advances in Cryptology -- ASIACRYPT 2005}}},\ \bibinfo {series}
  {LNCS}, Vol.\ \bibinfo {volume} {3788},\ \bibinfo {editor} {edited by\
  \bibinfo {editor} {\bibfnamefont {Bimal}\ \bibnamefont {Roy}}}\ (\bibinfo
  {publisher} {Springer})\ pp.\ \bibinfo {pages} {199--216}\BibitemShut
  {NoStop}%
\bibitem [{\citenamefont {Renner}\ and\ \citenamefont {Cirac}(2009)}]{RC09}%
  \BibitemOpen
  \bibfield  {author} {\bibinfo {author} {\bibnamefont {Renner}, \bibfnamefont
  {Renner}}, \ and\ \bibinfo {author} {\bibfnamefont {J.~Ignacio}\ \bibnamefont
  {Cirac}}} (\bibinfo {year} {2009}),\ \bibfield  {title} {\enquote {\bibinfo
  {title} {{de Finetti} representation theorem for infinite-dimensional quantum
  systems and applications to quantum cryptography},}\ }\href {\doibase
  10.1103/PhysRevLett.102.110504} {\bibfield  {journal} {\bibinfo  {journal}
  {Phys. Rev. Lett.}\ }\textbf {\bibinfo {volume} {102}},\ \bibinfo {pages}
  {110504}},\ \Eprint {http://arxiv.org/abs/arXiv:0809.2243} {arXiv:0809.2243}
  \BibitemShut {NoStop}%
\bibitem [{\citenamefont {Rivest}\ \emph {et~al.}(1978)\citenamefont {Rivest},
  \citenamefont {Shamir},\ and\ \citenamefont {Adleman}}]{RSA78}%
  \BibitemOpen
  \bibfield  {author} {\bibinfo {author} {\bibnamefont {Rivest}, \bibfnamefont
  {Ronald~L}}, \bibinfo {author} {\bibfnamefont {Adi}\ \bibnamefont {Shamir}},
  \ and\ \bibinfo {author} {\bibfnamefont {Leonard}\ \bibnamefont {Adleman}}}
  (\bibinfo {year} {1978}),\ \bibfield  {title} {\enquote {\bibinfo {title} {A
  method for obtaining digital signatures and public-key cryptosystems},}\
  }\href@noop {} {\bibfield  {journal} {\bibinfo  {journal} {Commun. ACM}\
  }\textbf {\bibinfo {volume} {21}}~(\bibinfo {number} {2}),\ \bibinfo {pages}
  {120--126}}\BibitemShut {NoStop}%
\bibitem [{\citenamefont {Rogaway}(2006)}]{Rog06}%
  \BibitemOpen
  \bibfield  {author} {\bibinfo {author} {\bibnamefont {Rogaway}, \bibfnamefont
  {Phillip}}} (\bibinfo {year} {2006}),\ \bibfield  {title} {\enquote {\bibinfo
  {title} {Formalizing human ignorance},}\ }in\ \href {\doibase
  10.1007/11958239_14} {\emph {\bibinfo {booktitle} {Progress in Cryptology --
  VIETCRYPT 2006}}},\ \bibinfo {series} {LNCS}, Vol.\ \bibinfo {volume} {4341}\
  (\bibinfo  {publisher} {Springer})\ pp.\ \bibinfo {pages} {211--228},\
  \bibinfo {note} {e-Print \href{http://eprint.iacr.org/2006/281}{IACR
  2006/281}}\BibitemShut {NoStop}%
\bibitem [{\citenamefont {Rosenfeld}\ \emph {et~al.}(2017)\citenamefont
  {Rosenfeld}, \citenamefont {Burchardt}, \citenamefont {Garthoff},
  \citenamefont {Redeker}, \citenamefont {Ortegel}, \citenamefont {Rau},\ and\
  \citenamefont {Weinfurter}}]{Rosenfeld}%
  \BibitemOpen
  \bibfield  {author} {\bibinfo {author} {\bibnamefont {Rosenfeld},
  \bibfnamefont {Wenjamin}}, \bibinfo {author} {\bibfnamefont {Daniel}\
  \bibnamefont {Burchardt}}, \bibinfo {author} {\bibfnamefont {Robert}\
  \bibnamefont {Garthoff}}, \bibinfo {author} {\bibfnamefont {Kai}\
  \bibnamefont {Redeker}}, \bibinfo {author} {\bibfnamefont {Norbert}\
  \bibnamefont {Ortegel}}, \bibinfo {author} {\bibfnamefont {Markus}\
  \bibnamefont {Rau}}, \ and\ \bibinfo {author} {\bibfnamefont {Harald}\
  \bibnamefont {Weinfurter}}} (\bibinfo {year} {2017}),\ \bibfield  {title}
  {\enquote {\bibinfo {title} {Event-ready bell test using entangled atoms
  simultaneously closing detection and locality loopholes},}\ }\href {\doibase
  10.1103/PhysRevLett.119.010402} {\bibfield  {journal} {\bibinfo  {journal}
  {Phys. Rev. Lett.}\ }\textbf {\bibinfo {volume} {119}},\ \bibinfo {pages}
  {010402}}\BibitemShut {NoStop}%
\bibitem [{\citenamefont {Rowe}\ \emph {et~al.}(2001)\citenamefont {Rowe},
  \citenamefont {Kielpinski}, \citenamefont {Meyer}, \citenamefont {Sackett},
  \citenamefont {Itano}, \citenamefont {Monroe},\ and\ \citenamefont
  {Wineland}}]{Rowe}%
  \BibitemOpen
  \bibfield  {author} {\bibinfo {author} {\bibnamefont {Rowe}, \bibfnamefont
  {M~A}}, \bibinfo {author} {\bibfnamefont {David}\ \bibnamefont {Kielpinski}},
  \bibinfo {author} {\bibfnamefont {V.}~\bibnamefont {Meyer}}, \bibinfo
  {author} {\bibfnamefont {Charles~A.}\ \bibnamefont {Sackett}}, \bibinfo
  {author} {\bibfnamefont {Wayne~M.}\ \bibnamefont {Itano}}, \bibinfo {author}
  {\bibfnamefont {C.}~\bibnamefont {Monroe}}, \ and\ \bibinfo {author}
  {\bibfnamefont {D.~J.}\ \bibnamefont {Wineland}}} (\bibinfo {year} {2001}),\
  \bibfield  {title} {\enquote {\bibinfo {title} {Experimental violation of a
  bell's inequality with efficient detection},}\ }\href {\doibase
  10.1038/35057215} {\bibfield  {journal} {\bibinfo  {journal} {Nature}\
  }\textbf {\bibinfo {volume} {409}}~(\bibinfo {number} {6822}),\ \bibinfo
  {pages} {791--794}}\BibitemShut {NoStop}%
\bibitem [{\citenamefont {Sasaki}\ \emph {et~al.}(2014)\citenamefont {Sasaki},
  \citenamefont {Yamamoto},\ and\ \citenamefont {Koashi}}]{SYK14}%
  \BibitemOpen
  \bibfield  {author} {\bibinfo {author} {\bibnamefont {Sasaki}, \bibfnamefont
  {Toshihiko}}, \bibinfo {author} {\bibfnamefont {Yoshihisa}\ \bibnamefont
  {Yamamoto}}, \ and\ \bibinfo {author} {\bibfnamefont {Masato}\ \bibnamefont
  {Koashi}}} (\bibinfo {year} {2014}),\ \bibfield  {title} {\enquote {\bibinfo
  {title} {Practical quantum key distribution protocol without monitoring
  signal disturbance},}\ }\href {\doibase 10.1038/nature13303} {\bibfield
  {journal} {\bibinfo  {journal} {Nature}\ }\textbf {\bibinfo {volume} {509}},\
  \bibinfo {pages} {475}}\BibitemShut {NoStop}%
\bibitem [{\citenamefont {Scarani}(2013)}]{Sca13}%
  \BibitemOpen
  \bibfield  {author} {\bibinfo {author} {\bibnamefont {Scarani}, \bibfnamefont
  {Valerio}}} (\bibinfo {year} {2013}),\ \href@noop {} {\enquote {\bibinfo
  {title} {The device-independent outlook on quantum physics (lecture notes on
  the power of {Bell}'s theorem)},}\ }\bibinfo {howpublished} {e-Print},\
  \Eprint {http://arxiv.org/abs/arXiv:1303.3081} {arXiv:1303.3081} \BibitemShut
  {NoStop}%
\bibitem [{\citenamefont {Scarani}\ \emph {et~al.}(2004)\citenamefont
  {Scarani}, \citenamefont {Ac\'{\i}n}, \citenamefont {Ribordy},\ and\
  \citenamefont {Gisin}}]{SARG}%
  \BibitemOpen
  \bibfield  {author} {\bibinfo {author} {\bibnamefont {Scarani}, \bibfnamefont
  {Valerio}}, \bibinfo {author} {\bibfnamefont {Antonio}\ \bibnamefont
  {Ac\'{\i}n}}, \bibinfo {author} {\bibfnamefont {Gr\'egoire}\ \bibnamefont
  {Ribordy}}, \ and\ \bibinfo {author} {\bibfnamefont {Nicolas}\ \bibnamefont
  {Gisin}}} (\bibinfo {year} {2004}),\ \bibfield  {title} {\enquote {\bibinfo
  {title} {Quantum cryptography protocols robust against photon number
  splitting attacks for weak laser pulse implementations},}\ }\href {\doibase
  10.1103/PhysRevLett.92.057901} {\bibfield  {journal} {\bibinfo  {journal}
  {Phys. Rev. Lett.}\ }\textbf {\bibinfo {volume} {92}},\ \bibinfo {pages}
  {057901}}\BibitemShut {NoStop}%
\bibitem [{\citenamefont {Scarani}\ \emph {et~al.}(2009)\citenamefont
  {Scarani}, \citenamefont {Bechmann-Pasquinucci}, \citenamefont {Cerf},
  \citenamefont {Du\ifmmode~\check{s}\else \v{s}\fi{}ek}, \citenamefont
  {L\"utkenhaus},\ and\ \citenamefont {Peev}}]{SBCDLP09}%
  \BibitemOpen
  \bibfield  {author} {\bibinfo {author} {\bibnamefont {Scarani}, \bibfnamefont
  {Valerio}}, \bibinfo {author} {\bibfnamefont {Helle}\ \bibnamefont
  {Bechmann-Pasquinucci}}, \bibinfo {author} {\bibfnamefont {Nicolas~J.}\
  \bibnamefont {Cerf}}, \bibinfo {author} {\bibfnamefont {Miloslav}\
  \bibnamefont {Du\ifmmode~\check{s}\else \v{s}\fi{}ek}}, \bibinfo {author}
  {\bibfnamefont {Norbert}\ \bibnamefont {L\"utkenhaus}}, \ and\ \bibinfo
  {author} {\bibfnamefont {Momtchil}\ \bibnamefont {Peev}}} (\bibinfo {year}
  {2009}),\ \bibfield  {title} {\enquote {\bibinfo {title} {The security of
  practical quantum key distribution},}\ }\href {\doibase
  10.1103/RevModPhys.81.1301} {\bibfield  {journal} {\bibinfo  {journal} {Rev.
  Mod. Phys.}\ }\textbf {\bibinfo {volume} {81}},\ \bibinfo {pages}
  {1301--1350}},\ \Eprint {http://arxiv.org/abs/arXiv:0802.4155}
  {arXiv:0802.4155} \BibitemShut {NoStop}%
\bibitem [{\citenamefont {Scarani}\ and\ \citenamefont {Renner}(2008)}]{SR08}%
  \BibitemOpen
  \bibfield  {author} {\bibinfo {author} {\bibnamefont {Scarani}, \bibfnamefont
  {Valerio}}, \ and\ \bibinfo {author} {\bibfnamefont {Renato}\ \bibnamefont
  {Renner}}} (\bibinfo {year} {2008}),\ \bibfield  {title} {\enquote {\bibinfo
  {title} {Quantum cryptography with finite resources: Unconditional security
  bound for discrete-variable protocols with one-way postprocessing},}\ }\href
  {\doibase 10.1103/PhysRevLett.100.200501} {\bibfield  {journal} {\bibinfo
  {journal} {Phys. Rev. Lett.}\ }\textbf {\bibinfo {volume} {100}},\ \bibinfo
  {pages} {200501}},\ \Eprint {http://arxiv.org/abs/arXiv:0708.0709}
  {arXiv:0708.0709} \BibitemShut {NoStop}%
\bibitem [{\citenamefont {Schaffner}\ \emph {et~al.}(2009)\citenamefont
  {Schaffner}, \citenamefont {Terhal},\ and\ \citenamefont {Wehner}}]{STW09}%
  \BibitemOpen
  \bibfield  {author} {\bibinfo {author} {\bibnamefont {Schaffner},
  \bibfnamefont {Christian}}, \bibinfo {author} {\bibfnamefont {Barbara}\
  \bibnamefont {Terhal}}, \ and\ \bibinfo {author} {\bibfnamefont {Stephanie}\
  \bibnamefont {Wehner}}} (\bibinfo {year} {2009}),\ \bibfield  {title}
  {\enquote {\bibinfo {title} {Robust cryptography in the noisy-quantum-storage
  model},}\ }\href@noop {} {\bibfield  {journal} {\bibinfo  {journal} {Quantum
  Inf. Comput.}\ }\textbf {\bibinfo {volume} {9}}~(\bibinfo {number} {11}),\
  \bibinfo {pages} {963--996}},\ \Eprint {http://arxiv.org/abs/arXiv:0807.1333}
  {arXiv:0807.1333} \BibitemShut {NoStop}%
\bibitem [{\citenamefont {Seiler}\ and\ \citenamefont {Maurer}(2016)}]{SM16}%
  \BibitemOpen
  \bibfield  {author} {\bibinfo {author} {\bibnamefont {Seiler}, \bibfnamefont
  {Gregor}}, \ and\ \bibinfo {author} {\bibfnamefont {Ueli}\ \bibnamefont
  {Maurer}}} (\bibinfo {year} {2016}),\ \bibfield  {title} {\enquote {\bibinfo
  {title} {On the impossibility of information-theoretic composable coin toss
  extension},}\ }in\ \href {\doibase 10.1109/ISIT.2016.7541861} {\emph
  {\bibinfo {booktitle} {Proceedings of the 2016 IEEE International Symposium
  on Information Theory, ISIT 2016}}}\ (\bibinfo  {publisher} {IEEE})\ pp.\
  \bibinfo {pages} {3058--3061}\BibitemShut {NoStop}%
\bibitem [{\citenamefont {Shalm}\ \emph {et~al.}(2015)\citenamefont {Shalm},
  \citenamefont {Meyer-Scott}, \citenamefont {Christensen}, \citenamefont
  {Bierhorst}, \citenamefont {Wayne}, \citenamefont {Stevens}, \citenamefont
  {Gerrits}, \citenamefont {Glancy}, \citenamefont {Hamel}, \citenamefont
  {Allman}, \citenamefont {Coakley}, \citenamefont {Dyer}, \citenamefont
  {Hodge}, \citenamefont {Lita}, \citenamefont {Verma}, \citenamefont
  {Lambrocco}, \citenamefont {Tortorici}, \citenamefont {Migdall},
  \citenamefont {Zhang}, \citenamefont {Kumor}, \citenamefont {Farr},
  \citenamefont {Marsili}, \citenamefont {Shaw}, \citenamefont {Stern},
  \citenamefont {Abell\'an}, \citenamefont {Amaya}, \citenamefont {Pruneri},
  \citenamefont {Jennewein}, \citenamefont {Mitchell}, \citenamefont {Kwiat},
  \citenamefont {Bienfang}, \citenamefont {Mirin}, \citenamefont {Knill},\ and\
  \citenamefont {Nam}}]{Shalm}%
  \BibitemOpen
  \bibfield  {author} {\bibinfo {author} {\bibnamefont {Shalm}, \bibfnamefont
  {Lynden~K}}, \bibinfo {author} {\bibfnamefont {Evan}\ \bibnamefont
  {Meyer-Scott}}, \bibinfo {author} {\bibfnamefont {Bradley~G.}\ \bibnamefont
  {Christensen}}, \bibinfo {author} {\bibfnamefont {Peter}\ \bibnamefont
  {Bierhorst}}, \bibinfo {author} {\bibfnamefont {Michael~A.}\ \bibnamefont
  {Wayne}}, \bibinfo {author} {\bibfnamefont {Martin~J.}\ \bibnamefont
  {Stevens}}, \bibinfo {author} {\bibfnamefont {Thomas}\ \bibnamefont
  {Gerrits}}, \bibinfo {author} {\bibfnamefont {Scott}\ \bibnamefont {Glancy}},
  \bibinfo {author} {\bibfnamefont {Deny~R.}\ \bibnamefont {Hamel}}, \bibinfo
  {author} {\bibfnamefont {Michael~S.}\ \bibnamefont {Allman}}, \bibinfo
  {author} {\bibfnamefont {Kevin~J.}\ \bibnamefont {Coakley}}, \bibinfo
  {author} {\bibfnamefont {Shellee~D.}\ \bibnamefont {Dyer}}, \bibinfo {author}
  {\bibfnamefont {Carson}\ \bibnamefont {Hodge}}, \bibinfo {author}
  {\bibfnamefont {Adriana~E.}\ \bibnamefont {Lita}}, \bibinfo {author}
  {\bibfnamefont {Varun~B.}\ \bibnamefont {Verma}}, \bibinfo {author}
  {\bibfnamefont {Camilla}\ \bibnamefont {Lambrocco}}, \bibinfo {author}
  {\bibfnamefont {Edward}\ \bibnamefont {Tortorici}}, \bibinfo {author}
  {\bibfnamefont {Alan~L.}\ \bibnamefont {Migdall}}, \bibinfo {author}
  {\bibfnamefont {Yanbao}\ \bibnamefont {Zhang}}, \bibinfo {author}
  {\bibfnamefont {Daniel~R.}\ \bibnamefont {Kumor}}, \bibinfo {author}
  {\bibfnamefont {William~H.}\ \bibnamefont {Farr}}, \bibinfo {author}
  {\bibfnamefont {Francesco}\ \bibnamefont {Marsili}}, \bibinfo {author}
  {\bibfnamefont {Matthew~D.}\ \bibnamefont {Shaw}}, \bibinfo {author}
  {\bibfnamefont {Jeffrey~A.}\ \bibnamefont {Stern}}, \bibinfo {author}
  {\bibfnamefont {Carlos}\ \bibnamefont {Abell\'an}}, \bibinfo {author}
  {\bibfnamefont {Waldimar}\ \bibnamefont {Amaya}}, \bibinfo {author}
  {\bibfnamefont {Valerio}\ \bibnamefont {Pruneri}}, \bibinfo {author}
  {\bibfnamefont {Thomas}\ \bibnamefont {Jennewein}}, \bibinfo {author}
  {\bibfnamefont {Morgan~W.}\ \bibnamefont {Mitchell}}, \bibinfo {author}
  {\bibfnamefont {Paul~G.}\ \bibnamefont {Kwiat}}, \bibinfo {author}
  {\bibfnamefont {Joshua~C.}\ \bibnamefont {Bienfang}}, \bibinfo {author}
  {\bibfnamefont {Richard~P.}\ \bibnamefont {Mirin}}, \bibinfo {author}
  {\bibfnamefont {Emanuel}\ \bibnamefont {Knill}}, \ and\ \bibinfo {author}
  {\bibfnamefont {Sae~Woo}\ \bibnamefont {Nam}}} (\bibinfo {year} {2015}),\
  \bibfield  {title} {\enquote {\bibinfo {title} {Strong loophole-free test of
  local realism},}\ }\href {\doibase 10.1103/PhysRevLett.115.250402} {\bibfield
   {journal} {\bibinfo  {journal} {Phys. Rev. Lett.}\ }\textbf {\bibinfo
  {volume} {115}},\ \bibinfo {pages} {250402}}\BibitemShut {NoStop}%
\bibitem [{\citenamefont {Shaltiel}(2004)}]{Shaltiel04}%
  \BibitemOpen
  \bibfield  {author} {\bibinfo {author} {\bibnamefont {Shaltiel},
  \bibfnamefont {Ronen}}} (\bibinfo {year} {2004}),\ \bibfield  {title}
  {\enquote {\bibinfo {title} {Recent developments in explicit constructions of
  extractors},}\ }in\ \href {\doibase 10.1142/9789812562494_0013} {\emph
  {\bibinfo {booktitle} {Current Trends in Theoretical Computer Science: The
  Challenge of the New Century, Vol 1: Algorithms and Complexity}}}\ (\bibinfo
  {publisher} {World Scientific})\ pp.\ \bibinfo {pages} {189--228}\BibitemShut
  {NoStop}%
\bibitem [{\citenamefont {Shannon}(1949)}]{Shannon49}%
  \BibitemOpen
  \bibfield  {author} {\bibinfo {author} {\bibnamefont {Shannon}, \bibfnamefont
  {Claude~E}}} (\bibinfo {year} {1949}),\ \bibfield  {title} {\enquote
  {\bibinfo {title} {Communication theory of secrecy systems},}\ }\href@noop {}
  {\bibfield  {journal} {\bibinfo  {journal} {Bell system technical journal}\
  }\textbf {\bibinfo {volume} {28}}~(\bibinfo {number} {4}),\ \bibinfo {pages}
  {656--715}}\BibitemShut {NoStop}%
\bibitem [{\citenamefont {Sheridan}\ \emph {et~al.}(2010)\citenamefont
  {Sheridan}, \citenamefont {Thinh},\ and\ \citenamefont {Scarani}}]{STS10}%
  \BibitemOpen
  \bibfield  {author} {\bibinfo {author} {\bibnamefont {Sheridan},
  \bibfnamefont {Lana}}, \bibinfo {author} {\bibfnamefont {Phuc~Le}\
  \bibnamefont {Thinh}}, \ and\ \bibinfo {author} {\bibfnamefont {Valerio}\
  \bibnamefont {Scarani}}} (\bibinfo {year} {2010}),\ \bibfield  {title}
  {\enquote {\bibinfo {title} {Finite-key security against coherent attacks in
  quantum key distribution},}\ }\href {\doibase 10.1088/1367-2630/12/12/123019}
  {\bibfield  {journal} {\bibinfo  {journal} {New J. Phys.}\ }\textbf {\bibinfo
  {volume} {12}}~(\bibinfo {number} {12}),\ \bibinfo {pages} {123019}},\
  \Eprint {http://arxiv.org/abs/arXiv:1008.2596} {arXiv:1008.2596} \BibitemShut
  {NoStop}%
\bibitem [{\citenamefont {Shor}(1997)}]{Shor97}%
  \BibitemOpen
  \bibfield  {author} {\bibinfo {author} {\bibnamefont {Shor}, \bibfnamefont
  {Peter~W}}} (\bibinfo {year} {1997}),\ \bibfield  {title} {\enquote {\bibinfo
  {title} {Polynomial-time algorithms for prime factorization and discrete
  logarithms on a quantum computer},}\ }\href {\doibase
  10.1137/S0097539795293172} {\bibfield  {journal} {\bibinfo  {journal} {SIAM
  J. Comput.}\ }\textbf {\bibinfo {volume} {26}}~(\bibinfo {number} {5}),\
  \bibinfo {pages} {1484--1509}}\BibitemShut {NoStop}%
\bibitem [{\citenamefont {Shor}\ and\ \citenamefont {Preskill}(2000)}]{SP00}%
  \BibitemOpen
  \bibfield  {author} {\bibinfo {author} {\bibnamefont {Shor}, \bibfnamefont
  {Peter~W}}, \ and\ \bibinfo {author} {\bibfnamefont {John}\ \bibnamefont
  {Preskill}}} (\bibinfo {year} {2000}),\ \bibfield  {title} {\enquote
  {\bibinfo {title} {Simple proof of security of the {BB84} quantum key
  distribution protocol},}\ }\href {\doibase 10.1103/PhysRevLett.85.441}
  {\bibfield  {journal} {\bibinfo  {journal} {Phys. Rev. Lett.}\ }\textbf
  {\bibinfo {volume} {85}},\ \bibinfo {pages} {441--444}},\ \Eprint
  {http://arxiv.org/abs/arXiv:quant-ph/0003004} {arXiv:quant-ph/0003004}
  \BibitemShut {NoStop}%
\bibitem [{\citenamefont {Simmons}(1985)}]{Sim85}%
  \BibitemOpen
  \bibfield  {author} {\bibinfo {author} {\bibnamefont {Simmons}, \bibfnamefont
  {Gustavus~J}}} (\bibinfo {year} {1985}),\ \bibfield  {title} {\enquote
  {\bibinfo {title} {Authentication theory/coding theory},}\ }in\ \href
  {\doibase 10.1007/3-540-39568-7_32} {\emph {\bibinfo {booktitle} {Advances in
  Cryptology -- CRYPTO~'84}}},\ \bibinfo {series} {LNCS}, Vol.\ \bibinfo
  {volume} {196}\ (\bibinfo  {publisher} {Springer})\ pp.\ \bibinfo {pages}
  {411--431}\BibitemShut {NoStop}%
\bibitem [{\citenamefont {Simmons}(1988)}]{Sim88}%
  \BibitemOpen
  \bibfield  {author} {\bibinfo {author} {\bibnamefont {Simmons}, \bibfnamefont
  {Gustavus~J}}} (\bibinfo {year} {1988}),\ \bibfield  {title} {\enquote
  {\bibinfo {title} {A survey of information authentication},}\ }\href
  {\doibase 10.1109/5.4445} {\bibfield  {journal} {\bibinfo  {journal} {Proc.
  IEEE}\ }\textbf {\bibinfo {volume} {76}}~(\bibinfo {number} {5}),\ \bibinfo
  {pages} {603--620}}\BibitemShut {NoStop}%
\bibitem [{\citenamefont {Steane}(1996)}]{Steane96}%
  \BibitemOpen
  \bibfield  {author} {\bibinfo {author} {\bibnamefont {Steane}, \bibfnamefont
  {Andrew}}} (\bibinfo {year} {1996}),\ \bibfield  {title} {\enquote {\bibinfo
  {title} {Multiple-particle interference and quantum error correction},}\
  }\href@noop {} {\bibfield  {journal} {\bibinfo  {journal} {Proc. R. Soc.
  London, Ser. A}\ }\textbf {\bibinfo {volume} {452}}~(\bibinfo {number}
  {1954}),\ \bibinfo {pages} {2551--2577}}\BibitemShut {NoStop}%
\bibitem [{\citenamefont {Stinson}(1990)}]{Sti90}%
  \BibitemOpen
  \bibfield  {author} {\bibinfo {author} {\bibnamefont {Stinson}, \bibfnamefont
  {Douglas~R}}} (\bibinfo {year} {1990}),\ \bibfield  {title} {\enquote
  {\bibinfo {title} {The combinatorics of authentication and secrecy codes},}\
  }\href {\doibase 10.1007/BF02252868} {\bibfield  {journal} {\bibinfo
  {journal} {J. Crypt.}\ }\textbf {\bibinfo {volume} {2}}~(\bibinfo {number}
  {1}),\ \bibinfo {pages} {23--49}}\BibitemShut {NoStop}%
\bibitem [{\citenamefont {Stinson}(1994)}]{Sti94}%
  \BibitemOpen
  \bibfield  {author} {\bibinfo {author} {\bibnamefont {Stinson}, \bibfnamefont
  {Douglas~R}}} (\bibinfo {year} {1994}),\ \bibfield  {title} {\enquote
  {\bibinfo {title} {Universal hashing and authentication codes},}\ }\href
  {\doibase 10.1007/BF01388651} {\bibfield  {journal} {\bibinfo  {journal}
  {Des. Codes Cryptogr.}\ }\textbf {\bibinfo {volume} {4}}~(\bibinfo {number}
  {3}),\ \bibinfo {pages} {369--380}},\ \bibinfo {note} {a preliminary version
  appeared at CRYPTO~'91}\BibitemShut {NoStop}%
\bibitem [{\citenamefont {Streltsov}\ \emph {et~al.}(2017)\citenamefont
  {Streltsov}, \citenamefont {Adesso},\ and\ \citenamefont {Plenio}}]{SAP17}%
  \BibitemOpen
  \bibfield  {author} {\bibinfo {author} {\bibnamefont {Streltsov},
  \bibfnamefont {Alexander}}, \bibinfo {author} {\bibfnamefont {Gerardo}\
  \bibnamefont {Adesso}}, \ and\ \bibinfo {author} {\bibfnamefont {Martin~B.}\
  \bibnamefont {Plenio}}} (\bibinfo {year} {2017}),\ \bibfield  {title}
  {\enquote {\bibinfo {title} {Colloquium: Quantum coherence as a resource},}\
  }\href {\doibase 10.1103/RevModPhys.89.041003} {\bibfield  {journal}
  {\bibinfo  {journal} {Rev. Mod. Phys.}\ }\textbf {\bibinfo {volume} {89}},\
  \bibinfo {pages} {041003}}\BibitemShut {NoStop}%
\bibitem [{\citenamefont {Stucki}\ \emph {et~al.}(2005)\citenamefont {Stucki},
  \citenamefont {Brunner}, \citenamefont {Gisin}, \citenamefont {Scarani},\
  and\ \citenamefont {Zbinden}}]{SBGSZ05}%
  \BibitemOpen
  \bibfield  {author} {\bibinfo {author} {\bibnamefont {Stucki}, \bibfnamefont
  {Damien}}, \bibinfo {author} {\bibfnamefont {Nicolas}\ \bibnamefont
  {Brunner}}, \bibinfo {author} {\bibfnamefont {Nicolas}\ \bibnamefont
  {Gisin}}, \bibinfo {author} {\bibfnamefont {Valerio}\ \bibnamefont
  {Scarani}}, \ and\ \bibinfo {author} {\bibfnamefont {Hugo}\ \bibnamefont
  {Zbinden}}} (\bibinfo {year} {2005}),\ \bibfield  {title} {\enquote {\bibinfo
  {title} {Fast and simple one-way quantum key distribution},}\ }\href
  {\doibase 10.1063/1.2126792} {\bibfield  {journal} {\bibinfo  {journal}
  {Appl. Phys. Lett.}\ }\textbf {\bibinfo {volume} {87}}~(\bibinfo {number}
  {19}),\ \bibinfo {pages} {194108}},\ \Eprint
  {http://arxiv.org/abs/arXiv:quant-ph/0506097} {arXiv:quant-ph/0506097}
  \BibitemShut {NoStop}%
\bibitem [{\citenamefont {Tamaki}\ \emph {et~al.}(2003)\citenamefont {Tamaki},
  \citenamefont {Koashi},\ and\ \citenamefont {Imoto}}]{Tamaki03}%
  \BibitemOpen
  \bibfield  {author} {\bibinfo {author} {\bibnamefont {Tamaki}, \bibfnamefont
  {Kiyoshi}}, \bibinfo {author} {\bibfnamefont {Masato}\ \bibnamefont
  {Koashi}}, \ and\ \bibinfo {author} {\bibfnamefont {Nobuyuki}\ \bibnamefont
  {Imoto}}} (\bibinfo {year} {2003}),\ \bibfield  {title} {\enquote {\bibinfo
  {title} {Unconditionally secure key distribution based on two nonorthogonal
  states},}\ }\href {\doibase 10.1103/PhysRevLett.90.167904} {\bibfield
  {journal} {\bibinfo  {journal} {Phys. Rev. Lett.}\ }\textbf {\bibinfo
  {volume} {90}},\ \bibinfo {pages} {167904}}\BibitemShut {NoStop}%
\bibitem [{\citenamefont {Tamaki}\ and\ \citenamefont {Lo}(2006)}]{TamakiLo}%
  \BibitemOpen
  \bibfield  {author} {\bibinfo {author} {\bibnamefont {Tamaki}, \bibfnamefont
  {Kiyoshi}}, \ and\ \bibinfo {author} {\bibfnamefont {Hoi-Kwong}\ \bibnamefont
  {Lo}}} (\bibinfo {year} {2006}),\ \bibfield  {title} {\enquote {\bibinfo
  {title} {Unconditionally secure key distillation from multiphotons},}\ }\href
  {\doibase 10.1103/PhysRevA.73.010302} {\bibfield  {journal} {\bibinfo
  {journal} {Phys. Rev. A}\ }\textbf {\bibinfo {volume} {73}},\ \bibinfo
  {pages} {010302}}\BibitemShut {NoStop}%
\bibitem [{\citenamefont {Tamaki}\ \emph {et~al.}(2012)\citenamefont {Tamaki},
  \citenamefont {Lo}, \citenamefont {Fung},\ and\ \citenamefont
  {Qi}}]{TamakiLo12}%
  \BibitemOpen
  \bibfield  {author} {\bibinfo {author} {\bibnamefont {Tamaki}, \bibfnamefont
  {Kiyoshi}}, \bibinfo {author} {\bibfnamefont {Hoi-Kwong}\ \bibnamefont {Lo}},
  \bibinfo {author} {\bibfnamefont {Chi-Hang~Fred}\ \bibnamefont {Fung}}, \
  and\ \bibinfo {author} {\bibfnamefont {Bing}\ \bibnamefont {Qi}}} (\bibinfo
  {year} {2012}),\ \bibfield  {title} {\enquote {\bibinfo {title} {Phase
  encoding schemes for measurement-device-independent quantum key distribution
  with basis-dependent flaw},}\ }\href {\doibase 10.1103/PhysRevA.85.042307}
  {\bibfield  {journal} {\bibinfo  {journal} {Phys. Rev. A}\ }\textbf {\bibinfo
  {volume} {85}},\ \bibinfo {pages} {042307}}\BibitemShut {NoStop}%
\bibitem [{\citenamefont {Tan}\ \emph {et~al.}(2020)\citenamefont {Tan},
  \citenamefont {Lim},\ and\ \citenamefont {Renner}}]{TLR20}%
  \BibitemOpen
  \bibfield  {author} {\bibinfo {author} {\bibnamefont {Tan}, \bibfnamefont
  {Ernest Y-Z}}, \bibinfo {author} {\bibfnamefont {Charles Ci~Wen}\
  \bibnamefont {Lim}}, \ and\ \bibinfo {author} {\bibfnamefont {Renato}\
  \bibnamefont {Renner}}} (\bibinfo {year} {2020}),\ \bibfield  {title}
  {\enquote {\bibinfo {title} {Advantage distillation for device-independent
  quantum key distribution},}\ }\href {\doibase 10.1103/PhysRevLett.124.020502}
  {\bibfield  {journal} {\bibinfo  {journal} {Phys. Rev. Lett.}\ }\textbf
  {\bibinfo {volume} {124}},\ \bibinfo {pages} {020502}},\ \Eprint
  {http://arxiv.org/abs/arXiv:1903.10535} {arXiv:1903.10535} \BibitemShut
  {NoStop}%
\bibitem [{\citenamefont {Tang}\ \emph {et~al.}(2014)\citenamefont {Tang},
  \citenamefont {Yin}, \citenamefont {Chen}, \citenamefont {Liu}, \citenamefont
  {Zhang}, \citenamefont {Jiang}, \citenamefont {Zhang}, \citenamefont {Wang},
  \citenamefont {You}, \citenamefont {Guan}, \citenamefont {Yang},
  \citenamefont {Wang}, \citenamefont {Liang}, \citenamefont {Zhang},
  \citenamefont {Zhou}, \citenamefont {Ma}, \citenamefont {Chen}, \citenamefont
  {Zhang},\ and\ \citenamefont {Pan}}]{Tang2014}%
  \BibitemOpen
  \bibfield  {author} {\bibinfo {author} {\bibnamefont {Tang}, \bibfnamefont
  {Yan-Lin}}, \bibinfo {author} {\bibfnamefont {Hua-Lei}\ \bibnamefont {Yin}},
  \bibinfo {author} {\bibfnamefont {Si-Jing}\ \bibnamefont {Chen}}, \bibinfo
  {author} {\bibfnamefont {Yang}\ \bibnamefont {Liu}}, \bibinfo {author}
  {\bibfnamefont {Wei-Jun}\ \bibnamefont {Zhang}}, \bibinfo {author}
  {\bibfnamefont {Xiao}\ \bibnamefont {Jiang}}, \bibinfo {author}
  {\bibfnamefont {Lu}~\bibnamefont {Zhang}}, \bibinfo {author} {\bibfnamefont
  {Jian}\ \bibnamefont {Wang}}, \bibinfo {author} {\bibfnamefont {Li-Xing}\
  \bibnamefont {You}}, \bibinfo {author} {\bibfnamefont {Jian-Yu}\ \bibnamefont
  {Guan}}, \bibinfo {author} {\bibfnamefont {Dong-Xu}\ \bibnamefont {Yang}},
  \bibinfo {author} {\bibfnamefont {Zhen}\ \bibnamefont {Wang}}, \bibinfo
  {author} {\bibfnamefont {Hao}\ \bibnamefont {Liang}}, \bibinfo {author}
  {\bibfnamefont {Zhen}\ \bibnamefont {Zhang}}, \bibinfo {author}
  {\bibfnamefont {Nan}\ \bibnamefont {Zhou}}, \bibinfo {author} {\bibfnamefont
  {Xiongfeng}\ \bibnamefont {Ma}}, \bibinfo {author} {\bibfnamefont {Teng-Yun}\
  \bibnamefont {Chen}}, \bibinfo {author} {\bibfnamefont {Qiang}\ \bibnamefont
  {Zhang}}, \ and\ \bibinfo {author} {\bibfnamefont {Jian-Wei}\ \bibnamefont
  {Pan}}} (\bibinfo {year} {2014}),\ \bibfield  {title} {\enquote {\bibinfo
  {title} {Measurement-device-independent quantum key distribution over 200
  km},}\ }\href {\doibase 10.1103/PhysRevLett.113.190501} {\bibfield  {journal}
  {\bibinfo  {journal} {Phys. Rev. Lett.}\ }\textbf {\bibinfo {volume} {113}},\
  \bibinfo {pages} {190501}}\BibitemShut {NoStop}%
\bibitem [{\citenamefont {Terhal}(2004)}]{Terhal04}%
  \BibitemOpen
  \bibfield  {author} {\bibinfo {author} {\bibnamefont {Terhal}, \bibfnamefont
  {Barbara~M}}} (\bibinfo {year} {2004}),\ \bibfield  {title} {\enquote
  {\bibinfo {title} {Is entanglement monogamous?}}\ }\href {\doibase
  10.1147/rd.481.0071} {\bibfield  {journal} {\bibinfo  {journal} {IBM J. Res.
  Dev.}\ }\textbf {\bibinfo {volume} {48}}~(\bibinfo {number} {1}),\ \bibinfo
  {pages} {71--78}}\BibitemShut {NoStop}%
\bibitem [{\citenamefont {Thorisson}(2000)}]{Tho00}%
  \BibitemOpen
  \bibfield  {author} {\bibinfo {author} {\bibnamefont {Thorisson},
  \bibfnamefont {Hermann}}} (\bibinfo {year} {2000}),\ \href@noop {} {\emph
  {\bibinfo {title} {Coupling, Stationarity, and Regeneration}}},\ Probability
  and its Applications (New York)\ (\bibinfo  {publisher}
  {Springer})\BibitemShut {NoStop}%
\bibitem [{\citenamefont {Tittel}\ \emph {et~al.}(1998)\citenamefont {Tittel},
  \citenamefont {Brendel}, \citenamefont {Gisin}, \citenamefont {Herzog},
  \citenamefont {Zbinden},\ and\ \citenamefont {Gisin}}]{Tittel}%
  \BibitemOpen
  \bibfield  {author} {\bibinfo {author} {\bibnamefont {Tittel}, \bibfnamefont
  {Wolfgang}}, \bibinfo {author} {\bibfnamefont {Jurgen}\ \bibnamefont
  {Brendel}}, \bibinfo {author} {\bibfnamefont {Bernard}\ \bibnamefont
  {Gisin}}, \bibinfo {author} {\bibfnamefont {Thomas}\ \bibnamefont {Herzog}},
  \bibinfo {author} {\bibfnamefont {Hugo}\ \bibnamefont {Zbinden}}, \ and\
  \bibinfo {author} {\bibfnamefont {Nicolas}\ \bibnamefont {Gisin}}} (\bibinfo
  {year} {1998}),\ \bibfield  {title} {\enquote {\bibinfo {title} {Experimental
  demonstration of quantum correlations over more than 10 km},}\ }\href
  {\doibase 10.1103/PhysRevA.57.3229} {\bibfield  {journal} {\bibinfo
  {journal} {Phys. Rev. A}\ }\textbf {\bibinfo {volume} {57}},\ \bibinfo
  {pages} {3229--3232}}\BibitemShut {NoStop}%
\bibitem [{\citenamefont {Tomamichel}\ and\ \citenamefont
  {Leverrier}(2017)}]{TL17}%
  \BibitemOpen
  \bibfield  {author} {\bibinfo {author} {\bibnamefont {Tomamichel},
  \bibfnamefont {Marco}}, \ and\ \bibinfo {author} {\bibfnamefont {Anthony}\
  \bibnamefont {Leverrier}}} (\bibinfo {year} {2017}),\ \bibfield  {title}
  {\enquote {\bibinfo {title} {A largely self-contained and complete security
  proof for quantum key distribution},}\ }\href {\doibase
  10.22331/q-2017-07-14-14} {\bibfield  {journal} {\bibinfo  {journal}
  {Quantum}\ }\textbf {\bibinfo {volume} {1}},\ \bibinfo {pages} {14}},\
  \Eprint {http://arxiv.org/abs/arXiv:1506.08458} {arXiv:1506.08458}
  \BibitemShut {NoStop}%
\bibitem [{\citenamefont {Tomamichel}\ \emph {et~al.}(2012)\citenamefont
  {Tomamichel}, \citenamefont {Lim}, \citenamefont {Gisin},\ and\ \citenamefont
  {Renner}}]{TLGR12}%
  \BibitemOpen
  \bibfield  {author} {\bibinfo {author} {\bibnamefont {Tomamichel},
  \bibfnamefont {Marco}}, \bibinfo {author} {\bibfnamefont {Charles Ci~Wen}\
  \bibnamefont {Lim}}, \bibinfo {author} {\bibfnamefont {Nicolas}\ \bibnamefont
  {Gisin}}, \ and\ \bibinfo {author} {\bibfnamefont {Renato}\ \bibnamefont
  {Renner}}} (\bibinfo {year} {2012}),\ \bibfield  {title} {\enquote {\bibinfo
  {title} {Tight finite-key analysis for quantum cryptography},}\ }\href
  {\doibase 10.1038/ncomms1631} {\bibfield  {journal} {\bibinfo  {journal}
  {Nat. Commun.}\ }\textbf {\bibinfo {volume} {3}},\ \bibinfo {pages} {634}},\
  \Eprint {http://arxiv.org/abs/arXiv:1103.4130} {arXiv:1103.4130} \BibitemShut
  {NoStop}%
\bibitem [{\citenamefont {Tomamichel}\ and\ \citenamefont
  {Renner}(2011)}]{TR11}%
  \BibitemOpen
  \bibfield  {author} {\bibinfo {author} {\bibnamefont {Tomamichel},
  \bibfnamefont {Marco}}, \ and\ \bibinfo {author} {\bibfnamefont {Renato}\
  \bibnamefont {Renner}}} (\bibinfo {year} {2011}),\ \bibfield  {title}
  {\enquote {\bibinfo {title} {Uncertainty relation for smooth entropies},}\
  }\href {\doibase 10.1103/PhysRevLett.106.110506} {\bibfield  {journal}
  {\bibinfo  {journal} {Phys. Rev. Lett.}\ }\textbf {\bibinfo {volume} {106}},\
  \bibinfo {pages} {110506}},\ \Eprint {http://arxiv.org/abs/arXiv:1009.2015}
  {arXiv:1009.2015} \BibitemShut {NoStop}%
\bibitem [{\citenamefont {Tomamichel}\ \emph {et~al.}(2010)\citenamefont
  {Tomamichel}, \citenamefont {Schaffner}, \citenamefont {Smith},\ and\
  \citenamefont {Renner}}]{TSSR10}%
  \BibitemOpen
  \bibfield  {author} {\bibinfo {author} {\bibnamefont {Tomamichel},
  \bibfnamefont {Marco}}, \bibinfo {author} {\bibfnamefont {Christian}\
  \bibnamefont {Schaffner}}, \bibinfo {author} {\bibfnamefont {Adam}\
  \bibnamefont {Smith}}, \ and\ \bibinfo {author} {\bibfnamefont {Renato}\
  \bibnamefont {Renner}}} (\bibinfo {year} {2010}),\ \bibfield  {title}
  {\enquote {\bibinfo {title} {Leftover hashing against quantum side
  information},}\ }in\ \href {\doibase 10.1109/ISIT.2010.5513652} {\emph
  {\bibinfo {booktitle} {Proceedings of the 2010 IEEE International Symposium
  on Information Theory, ISIT 2010}}}\ (\bibinfo  {publisher} {IEEE})\ pp.\
  \bibinfo {pages} {2703--2707},\ \Eprint
  {http://arxiv.org/abs/arXiv:1002.2436} {arXiv:1002.2436} \BibitemShut
  {NoStop}%
\bibitem [{\citenamefont {Unruh}(2004)}]{Unr04}%
  \BibitemOpen
  \bibfield  {author} {\bibinfo {author} {\bibnamefont {Unruh}, \bibfnamefont
  {Dominique}}} (\bibinfo {year} {2004}),\ \href@noop {} {\enquote {\bibinfo
  {title} {Simulatable security for quantum protocols},}\ }\bibinfo
  {howpublished} {e-Print},\ \Eprint
  {http://arxiv.org/abs/arXiv:quant-ph/0409125} {arXiv:quant-ph/0409125}
  \BibitemShut {NoStop}%
\bibitem [{\citenamefont {Unruh}(2010)}]{Unr10}%
  \BibitemOpen
  \bibfield  {author} {\bibinfo {author} {\bibnamefont {Unruh}, \bibfnamefont
  {Dominique}}} (\bibinfo {year} {2010}),\ \bibfield  {title} {\enquote
  {\bibinfo {title} {Universally composable quantum multi-party computation},}\
  }in\ \href {\doibase 10.1007/978-3-642-13190-5_25} {\emph {\bibinfo
  {booktitle} {Advances in Cryptology -- EUROCRYPT 2010}}},\ \bibinfo {series}
  {LNCS}, Vol.\ \bibinfo {volume} {6110}\ (\bibinfo  {publisher} {Springer})\
  pp.\ \bibinfo {pages} {486--505},\ \Eprint
  {http://arxiv.org/abs/arXiv:0910.2912} {arXiv:0910.2912} \BibitemShut
  {NoStop}%
\bibitem [{\citenamefont {Unruh}(2011)}]{Unr11}%
  \BibitemOpen
  \bibfield  {author} {\bibinfo {author} {\bibnamefont {Unruh}, \bibfnamefont
  {Dominique}}} (\bibinfo {year} {2011}),\ \bibfield  {title} {\enquote
  {\bibinfo {title} {Concurrent composition in the bounded quantum storage
  model},}\ }in\ \href {\doibase 10.1007/978-3-642-20465-4_26} {\emph {\bibinfo
  {booktitle} {Advances in Cryptology -- EUROCRYPT 2011}}},\ \bibinfo {series}
  {LNCS}, Vol.\ \bibinfo {volume} {6632}\ (\bibinfo  {publisher} {Springer})\
  pp.\ \bibinfo {pages} {467--486},\ \bibinfo {note} {e-Print
  \href{http://eprint.iacr.org/2010/229}{IACR 2010/229}}\BibitemShut {NoStop}%
\bibitem [{\citenamefont {Unruh}(2013)}]{Unr13}%
  \BibitemOpen
  \bibfield  {author} {\bibinfo {author} {\bibnamefont {Unruh}, \bibfnamefont
  {Dominique}}} (\bibinfo {year} {2013}),\ \bibfield  {title} {\enquote
  {\bibinfo {title} {Everlasting multi-party computation},}\ }in\ \href
  {\doibase 10.1007/978-3-642-40084-1_22} {\emph {\bibinfo {booktitle}
  {Advances in Cryptology -- CRYPTO 2013}}},\ \bibinfo {series} {LNCS}, Vol.\
  \bibinfo {volume} {8043}\ (\bibinfo  {publisher} {Springer})\ pp.\ \bibinfo
  {pages} {380--397},\ \bibinfo {note} {e-Print
  \href{http://eprint.iacr.org/2012/177}{IACR 2012/177}}\BibitemShut {NoStop}%
\bibitem [{\citenamefont {Unruh}(2014)}]{Unr14}%
  \BibitemOpen
  \bibfield  {author} {\bibinfo {author} {\bibnamefont {Unruh}, \bibfnamefont
  {Dominique}}} (\bibinfo {year} {2014}),\ \bibfield  {title} {\enquote
  {\bibinfo {title} {Quantum position verification in the random oracle
  model},}\ }in\ \href {\doibase 10.1007/978-3-662-44381-1_1} {\emph {\bibinfo
  {booktitle} {Advances in Cryptology -- CRYPTO 2014}}},\ \bibinfo {series}
  {LNCS}, Vol.\ \bibinfo {volume} {8617}\ (\bibinfo  {publisher} {Springer})\
  pp.\ \bibinfo {pages} {1--18},\ \bibinfo {note} {e-Print
  \href{http://eprint.iacr.org/2014/118}{IACR 2014/118}}\BibitemShut {NoStop}%
\bibitem [{\citenamefont {Vakhitov}\ \emph {et~al.}(2001)\citenamefont
  {Vakhitov}, \citenamefont {Makarov},\ and\ \citenamefont
  {Hjelme}}]{Vakhitov2001}%
  \BibitemOpen
  \bibfield  {author} {\bibinfo {author} {\bibnamefont {Vakhitov},
  \bibfnamefont {Artem}}, \bibinfo {author} {\bibfnamefont {Vadim}\
  \bibnamefont {Makarov}}, \ and\ \bibinfo {author} {\bibfnamefont {Dag~R.}\
  \bibnamefont {Hjelme}}} (\bibinfo {year} {2001}),\ \bibfield  {title}
  {\enquote {\bibinfo {title} {Large pulse attack as a method of conventional
  optical eavesdropping in quantum cryptography},}\ }\href {\doibase
  10.1080/09500340108240904} {\bibfield  {journal} {\bibinfo  {journal} {J.
  Mod. Opt.}\ }\textbf {\bibinfo {volume} {48}}~(\bibinfo {number} {13}),\
  \bibinfo {pages} {2023--2038}}\BibitemShut {NoStop}%
\bibitem [{\citenamefont {Vazirani}\ and\ \citenamefont {Vidick}(2012)}]{VV12}%
  \BibitemOpen
  \bibfield  {author} {\bibinfo {author} {\bibnamefont {Vazirani},
  \bibfnamefont {Umesh}}, \ and\ \bibinfo {author} {\bibfnamefont {Thomas}\
  \bibnamefont {Vidick}}} (\bibinfo {year} {2012}),\ \bibfield  {title}
  {\enquote {\bibinfo {title} {Certifiable quantum dice: or, true random number
  generation secure against quantum adversaries},}\ }in\ \href {\doibase
  10.1145/2213977.2213984} {\emph {\bibinfo {booktitle} {Proceedings of the
  44th Symposium on Theory of Computing, STOC~'12}}}\ (\bibinfo  {publisher}
  {ACM})\ pp.\ \bibinfo {pages} {61--76},\ \Eprint
  {http://arxiv.org/abs/arXiv:1111.6054} {arXiv:1111.6054} \BibitemShut
  {NoStop}%
\bibitem [{\citenamefont {Vazirani}\ and\ \citenamefont {Vidick}(2014)}]{VV14}%
  \BibitemOpen
  \bibfield  {author} {\bibinfo {author} {\bibnamefont {Vazirani},
  \bibfnamefont {Umesh}}, \ and\ \bibinfo {author} {\bibfnamefont {Thomas}\
  \bibnamefont {Vidick}}} (\bibinfo {year} {2014}),\ \bibfield  {title}
  {\enquote {\bibinfo {title} {Fully device-independent quantum key
  distribution},}\ }\href {\doibase 10.1103/PhysRevLett.113.140501} {\bibfield
  {journal} {\bibinfo  {journal} {Phys. Rev. Lett.}\ }\textbf {\bibinfo
  {volume} {113}},\ \bibinfo {pages} {140501}},\ \Eprint
  {http://arxiv.org/abs/arXiv:1210.1810} {arXiv:1210.1810} \BibitemShut
  {NoStop}%
\bibitem [{\citenamefont {Vernam}(1926)}]{Vernam26}%
  \BibitemOpen
  \bibfield  {author} {\bibinfo {author} {\bibnamefont {Vernam}, \bibfnamefont
  {Gilbert~S}}} (\bibinfo {year} {1926}),\ \bibfield  {title} {\enquote
  {\bibinfo {title} {Cipher printing telegraph systems for secret wire and
  radio telegraphic communications},}\ }\href@noop {} {\bibfield  {journal}
  {\bibinfo  {journal} {Trans. Am. Inst. Electr. Eng.}\ }\textbf {\bibinfo
  {volume} {XLV}},\ \bibinfo {pages} {295--301}}\BibitemShut {NoStop}%
\bibitem [{\citenamefont {Vilasini}\ \emph {et~al.}(2019)\citenamefont
  {Vilasini}, \citenamefont {Portmann},\ and\ \citenamefont {del
  Rio}}]{VPdR19}%
  \BibitemOpen
  \bibfield  {author} {\bibinfo {author} {\bibnamefont {Vilasini},
  \bibfnamefont {V}}, \bibinfo {author} {\bibfnamefont {Christopher}\
  \bibnamefont {Portmann}}, \ and\ \bibinfo {author} {\bibfnamefont {L\'idia}\
  \bibnamefont {del Rio}}} (\bibinfo {year} {2019}),\ \bibfield  {title}
  {\enquote {\bibinfo {title} {Composable security in relativistic quantum
  cryptography},}\ }\href {\doibase 10.1088/1367-2630/ab0e3b} {\bibfield
  {journal} {\bibinfo  {journal} {New J. Phys.}\ }\textbf {\bibinfo {volume}
  {21}}~(\bibinfo {number} {4}),\ \bibinfo {pages} {043057}},\ \Eprint
  {http://arxiv.org/abs/arXiv:1708.00433} {arXiv:1708.00433} \BibitemShut
  {NoStop}%
\bibitem [{\citenamefont {Wang}(2005)}]{Wang2005}%
  \BibitemOpen
  \bibfield  {author} {\bibinfo {author} {\bibnamefont {Wang}, \bibfnamefont
  {Xiang-Bin}}} (\bibinfo {year} {2005}),\ \bibfield  {title} {\enquote
  {\bibinfo {title} {Beating the photon-number-splitting attack in practical
  quantum cryptography},}\ }\href {\doibase 10.1103/PhysRevLett.94.230503}
  {\bibfield  {journal} {\bibinfo  {journal} {Phys. Rev. Lett.}\ }\textbf
  {\bibinfo {volume} {94}},\ \bibinfo {pages} {230503}},\ \Eprint
  {http://arxiv.org/abs/arxiv:quant-ph/0410075} {arxiv:quant-ph/0410075}
  \BibitemShut {NoStop}%
\bibitem [{\citenamefont {Watrous}(2018)}]{Wat18}%
  \BibitemOpen
  \bibfield  {author} {\bibinfo {author} {\bibnamefont {Watrous}, \bibfnamefont
  {John}}} (\bibinfo {year} {2018}),\ \href {\doibase 10.1017/9781316848142}
  {\emph {\bibinfo {title} {The Theory of Quantum Information}}}\ (\bibinfo
  {publisher} {Cambridge University Press})\ \bibinfo {note} {available at
  \url{http://cs.uwaterloo.ca/~watrous/TQI/}}\BibitemShut {NoStop}%
\bibitem [{\citenamefont {Webb}(2015)}]{Web15}%
  \BibitemOpen
  \bibfield  {author} {\bibinfo {author} {\bibnamefont {Webb}, \bibfnamefont
  {Zak}}} (\bibinfo {year} {2015}),\ \bibfield  {title} {\enquote {\bibinfo
  {title} {The {Clifford} group forms a unitary 3-design},}\ }\href@noop {}
  {\bibfield  {journal} {\bibinfo  {journal} {Quantum Inf. Comput.}\ }\textbf
  {\bibinfo {volume} {16}}~(\bibinfo {number} {15{\&}16}),\ \bibinfo {pages}
  {1379--1400}},\ \Eprint {http://arxiv.org/abs/arXiv:1510.02769}
  {arXiv:1510.02769} \BibitemShut {NoStop}%
\bibitem [{\citenamefont {Wegman}\ and\ \citenamefont {Carter}(1981)}]{WC81}%
  \BibitemOpen
  \bibfield  {author} {\bibinfo {author} {\bibnamefont {Wegman}, \bibfnamefont
  {Mark~N}}, \ and\ \bibinfo {author} {\bibfnamefont {Larry}\ \bibnamefont
  {Carter}}} (\bibinfo {year} {1981}),\ \bibfield  {title} {\enquote {\bibinfo
  {title} {New hash functions and their use in authentication and set
  equality},}\ }\href@noop {} {\bibfield  {journal} {\bibinfo  {journal} {J.
  Comput. Syst. Sci.}\ }\textbf {\bibinfo {volume} {22}}~(\bibinfo {number}
  {3}),\ \bibinfo {pages} {265--279}}\BibitemShut {NoStop}%
\bibitem [{\citenamefont {Wehner}\ \emph {et~al.}(2008)\citenamefont {Wehner},
  \citenamefont {Schaffner},\ and\ \citenamefont {Terhal}}]{WST08}%
  \BibitemOpen
  \bibfield  {author} {\bibinfo {author} {\bibnamefont {Wehner}, \bibfnamefont
  {Stephanie}}, \bibinfo {author} {\bibfnamefont {Christian}\ \bibnamefont
  {Schaffner}}, \ and\ \bibinfo {author} {\bibfnamefont {Barbara~M.}\
  \bibnamefont {Terhal}}} (\bibinfo {year} {2008}),\ \bibfield  {title}
  {\enquote {\bibinfo {title} {Cryptography from noisy storage},}\ }\href
  {\doibase 10.1103/PhysRevLett.100.220502} {\bibfield  {journal} {\bibinfo
  {journal} {Phys. Rev. Lett.}\ }\textbf {\bibinfo {volume} {100}},\ \bibinfo
  {pages} {220502}},\ \Eprint {http://arxiv.org/abs/arXiv:0711.2895}
  {arXiv:0711.2895} \BibitemShut {NoStop}%
\bibitem [{\citenamefont {Weier}\ \emph {et~al.}(2011)\citenamefont {Weier},
  \citenamefont {Krauss}, \citenamefont {Rau}, \citenamefont {F\"{u}rst},
  \citenamefont {Nauerth},\ and\ \citenamefont {Weinfurter}}]{WKRFNW11}%
  \BibitemOpen
  \bibfield  {author} {\bibinfo {author} {\bibnamefont {Weier}, \bibfnamefont
  {Henning}}, \bibinfo {author} {\bibfnamefont {Harald}\ \bibnamefont
  {Krauss}}, \bibinfo {author} {\bibfnamefont {Markus}\ \bibnamefont {Rau}},
  \bibinfo {author} {\bibfnamefont {Martin}\ \bibnamefont {F\"{u}rst}},
  \bibinfo {author} {\bibfnamefont {Sebastian}\ \bibnamefont {Nauerth}}, \ and\
  \bibinfo {author} {\bibfnamefont {Harald}\ \bibnamefont {Weinfurter}}}
  (\bibinfo {year} {2011}),\ \bibfield  {title} {\enquote {\bibinfo {title}
  {Quantum eavesdropping without interception: an attack exploiting the dead
  time of single-photon detectors},}\ }\href {\doibase
  10.1088/1367-2630/13/7/073024} {\bibfield  {journal} {\bibinfo  {journal}
  {New J. Phys.}\ }\textbf {\bibinfo {volume} {13}}~(\bibinfo {number} {7}),\
  \bibinfo {pages} {073024}},\ \Eprint {http://arxiv.org/abs/arXiv:1101.5289}
  {arXiv:1101.5289} \BibitemShut {NoStop}%
\bibitem [{\citenamefont {Weihs}\ \emph {et~al.}(1998)\citenamefont {Weihs},
  \citenamefont {Jennewein}, \citenamefont {Simon}, \citenamefont
  {Weinfurter},\ and\ \citenamefont {Zeilinger}}]{Weihs}%
  \BibitemOpen
  \bibfield  {author} {\bibinfo {author} {\bibnamefont {Weihs}, \bibfnamefont
  {Gregor}}, \bibinfo {author} {\bibfnamefont {Thomas}\ \bibnamefont
  {Jennewein}}, \bibinfo {author} {\bibfnamefont {Christoph}\ \bibnamefont
  {Simon}}, \bibinfo {author} {\bibfnamefont {Harald}\ \bibnamefont
  {Weinfurter}}, \ and\ \bibinfo {author} {\bibfnamefont {Anton}\ \bibnamefont
  {Zeilinger}}} (\bibinfo {year} {1998}),\ \bibfield  {title} {\enquote
  {\bibinfo {title} {Violation of {Bell}'s inequality under strict einstein
  locality conditions},}\ }\href {\doibase 10.1103/PhysRevLett.81.5039}
  {\bibfield  {journal} {\bibinfo  {journal} {Phys. Rev. Lett.}\ }\textbf
  {\bibinfo {volume} {81}},\ \bibinfo {pages} {5039--5043}}\BibitemShut
  {NoStop}%
\bibitem [{\citenamefont {Wiesner}(1983)}]{Wie83}%
  \BibitemOpen
  \bibfield  {author} {\bibinfo {author} {\bibnamefont {Wiesner}, \bibfnamefont
  {Stephen}}} (\bibinfo {year} {1983}),\ \bibfield  {title} {\enquote {\bibinfo
  {title} {Conjugate coding},}\ }\href@noop {} {\bibfield  {journal} {\bibinfo
  {journal} {SIGACT news}\ }\textbf {\bibinfo {volume} {15}}~(\bibinfo {number}
  {1}),\ \bibinfo {pages} {78--88}},\ \bibinfo {note} {original manuscript
  written circa 1969}\BibitemShut {NoStop}%
\bibitem [{\citenamefont {Winter}(2017)}]{Win17}%
  \BibitemOpen
  \bibfield  {author} {\bibinfo {author} {\bibnamefont {Winter}, \bibfnamefont
  {Andreas}}} (\bibinfo {year} {2017}),\ \bibfield  {title} {\enquote {\bibinfo
  {title} {Weak locking capacity of quantum channels can be much larger than
  private capacity},}\ }\href {\doibase 10.1007/s00145-015-9215-3} {\bibfield
  {journal} {\bibinfo  {journal} {J. Crypt.}\ }\textbf {\bibinfo {volume}
  {30}}~(\bibinfo {number} {1}),\ \bibinfo {pages} {1--21}},\ \Eprint
  {http://arxiv.org/abs/arXiv:1403.6361} {arXiv:1403.6361} \BibitemShut
  {NoStop}%
\bibitem [{\citenamefont {Wolf}(1999)}]{Wol99}%
  \BibitemOpen
  \bibfield  {author} {\bibinfo {author} {\bibnamefont {Wolf}, \bibfnamefont
  {Stefan}}} (\bibinfo {year} {1999}),\ \emph {\bibinfo {title}
  {Information-Theoretically and Computationally Secure Key Agreement in
  Cryptography}},\ \href@noop {} {Ph.D. thesis}\ (\bibinfo  {school} {Swiss
  Federal Institute of Technology (ETH) Zurich})\BibitemShut {NoStop}%
\bibitem [{\citenamefont {Wootters}\ and\ \citenamefont
  {Zurek}(1982)}]{Wootters82}%
  \BibitemOpen
  \bibfield  {author} {\bibinfo {author} {\bibnamefont {Wootters},
  \bibfnamefont {William~K}}, \ and\ \bibinfo {author} {\bibfnamefont
  {Wojciech~H.}\ \bibnamefont {Zurek}}} (\bibinfo {year} {1982}),\ \bibfield
  {title} {\enquote {\bibinfo {title} {A single quantum cannot be cloned},}\
  }\href {\doibase 10.1038/299802a0} {\bibfield  {journal} {\bibinfo  {journal}
  {Nature}\ }\textbf {\bibinfo {volume} {299}}~(\bibinfo {number} {5886}),\
  \bibinfo {pages} {802--803}}\BibitemShut {NoStop}%
\bibitem [{\citenamefont {Xu}\ \emph {et~al.}(2010)\citenamefont {Xu},
  \citenamefont {Qi},\ and\ \citenamefont {Lo}}]{XQL10}%
  \BibitemOpen
  \bibfield  {author} {\bibinfo {author} {\bibnamefont {Xu}, \bibfnamefont
  {Feihu}}, \bibinfo {author} {\bibfnamefont {Bing}\ \bibnamefont {Qi}}, \ and\
  \bibinfo {author} {\bibfnamefont {Hoi-Kwong}\ \bibnamefont {Lo}}} (\bibinfo
  {year} {2010}),\ \bibfield  {title} {\enquote {\bibinfo {title} {Experimental
  demonstration of phase-remapping attack in a practical quantum key
  distribution system},}\ }\href {\doibase 10.1088/1367-2630/12/11/113026}
  {\bibfield  {journal} {\bibinfo  {journal} {New J. Phys.}\ }\textbf {\bibinfo
  {volume} {12}}~(\bibinfo {number} {11}),\ \bibinfo {pages}
  {113026}}\BibitemShut {NoStop}%
\bibitem [{\citenamefont {Yin}\ \emph {et~al.}(2016)\citenamefont {Yin},
  \citenamefont {Chen}, \citenamefont {Yu}, \citenamefont {Liu}, \citenamefont
  {You}, \citenamefont {Zhou}, \citenamefont {Chen}, \citenamefont {Mao},
  \citenamefont {Huang}, \citenamefont {Zhang}, \citenamefont {Chen},
  \citenamefont {Li}, \citenamefont {Nolan}, \citenamefont {Zhou},
  \citenamefont {Jiang}, \citenamefont {Wang}, \citenamefont {Zhang},
  \citenamefont {Wang},\ and\ \citenamefont {Pan}}]{Yin2016}%
  \BibitemOpen
  \bibfield  {author} {\bibinfo {author} {\bibnamefont {Yin}, \bibfnamefont
  {Hua-Lei}}, \bibinfo {author} {\bibfnamefont {Teng-Yun}\ \bibnamefont
  {Chen}}, \bibinfo {author} {\bibfnamefont {Zong-Wen}\ \bibnamefont {Yu}},
  \bibinfo {author} {\bibfnamefont {Hui}\ \bibnamefont {Liu}}, \bibinfo
  {author} {\bibfnamefont {Li-Xing}\ \bibnamefont {You}}, \bibinfo {author}
  {\bibfnamefont {Yi-Heng}\ \bibnamefont {Zhou}}, \bibinfo {author}
  {\bibfnamefont {Si-Jing}\ \bibnamefont {Chen}}, \bibinfo {author}
  {\bibfnamefont {Yingqiu}\ \bibnamefont {Mao}}, \bibinfo {author}
  {\bibfnamefont {Ming-Qi}\ \bibnamefont {Huang}}, \bibinfo {author}
  {\bibfnamefont {Wei-Jun}\ \bibnamefont {Zhang}}, \bibinfo {author}
  {\bibfnamefont {Hao}\ \bibnamefont {Chen}}, \bibinfo {author} {\bibfnamefont
  {Ming~Jun}\ \bibnamefont {Li}}, \bibinfo {author} {\bibfnamefont {Daniel}\
  \bibnamefont {Nolan}}, \bibinfo {author} {\bibfnamefont {Fei}\ \bibnamefont
  {Zhou}}, \bibinfo {author} {\bibfnamefont {Xiao}\ \bibnamefont {Jiang}},
  \bibinfo {author} {\bibfnamefont {Zhen}\ \bibnamefont {Wang}}, \bibinfo
  {author} {\bibfnamefont {Qiang}\ \bibnamefont {Zhang}}, \bibinfo {author}
  {\bibfnamefont {Xiang-Bin}\ \bibnamefont {Wang}}, \ and\ \bibinfo {author}
  {\bibfnamefont {Jian-Wei}\ \bibnamefont {Pan}}} (\bibinfo {year} {2016}),\
  \bibfield  {title} {\enquote {\bibinfo {title}
  {Measurement-device-independent quantum key distribution over a 404 km
  optical fiber},}\ }\href {\doibase 10.1103/PhysRevLett.117.190501} {\bibfield
   {journal} {\bibinfo  {journal} {Phys. Rev. Lett.}\ }\textbf {\bibinfo
  {volume} {117}},\ \bibinfo {pages} {190501}},\ \Eprint
  {http://arxiv.org/abs/arXiv:1606.06821} {arXiv:1606.06821} \BibitemShut
  {NoStop}%
\bibitem [{\citenamefont {Yin}\ and\ \citenamefont {Chen}(2019)}]{YinChen}%
  \BibitemOpen
  \bibfield  {author} {\bibinfo {author} {\bibnamefont {Yin}, \bibfnamefont
  {Hua-Lei}}, \ and\ \bibinfo {author} {\bibfnamefont {Zeng-Bing}\ \bibnamefont
  {Chen}}} (\bibinfo {year} {2019}),\ \bibfield  {title} {\enquote {\bibinfo
  {title} {Finite-key analysis for twin-field quantum key distribution with
  composable security},}\ }\href@noop {} {\bibfield  {journal} {\bibinfo
  {journal} {Sci. Rep.}\ }\textbf {\bibinfo {volume} {9}}~(\bibinfo {number}
  {1}),\ \bibinfo {pages} {1--9}}\BibitemShut {NoStop}%
\bibitem [{\citenamefont {Yuan}\ \emph {et~al.}(2010)\citenamefont {Yuan},
  \citenamefont {Dynes},\ and\ \citenamefont {Shields}}]{Yuanetal2010}%
  \BibitemOpen
  \bibfield  {author} {\bibinfo {author} {\bibnamefont {Yuan}, \bibfnamefont
  {Zhiliang}}, \bibinfo {author} {\bibfnamefont {James~F.}\ \bibnamefont
  {Dynes}}, \ and\ \bibinfo {author} {\bibfnamefont {Andrew~J.}\ \bibnamefont
  {Shields}}} (\bibinfo {year} {2010}),\ \bibfield  {title} {\enquote {\bibinfo
  {title} {Avoiding the blinding attack in {QKD}},}\ }\href {\doibase
  10.1038/nphoton.2010.269} {\bibfield  {journal} {\bibinfo  {journal} {Nat.
  Photonics}\ }\textbf {\bibinfo {volume} {4}},\ \bibinfo {pages}
  {800}}\BibitemShut {NoStop}%
\bibitem [{\citenamefont {Zhandry}(2012)}]{Zha12}%
  \BibitemOpen
  \bibfield  {author} {\bibinfo {author} {\bibnamefont {Zhandry}, \bibfnamefont
  {Mark}}} (\bibinfo {year} {2012}),\ \bibfield  {title} {\enquote {\bibinfo
  {title} {How to construct quantum random functions},}\ }in\ \href {\doibase
  10.1109/FOCS.2012.37} {\emph {\bibinfo {booktitle} {Proceedings of the 53rd
  Symposium on Foundations of Computer Science, FOCS~'12}}}\ (\bibinfo
  {publisher} {IEEE})\ pp.\ \bibinfo {pages} {679--687},\ \bibinfo {note}
  {e-Print \href{http://eprint.iacr.org/2012/182}{IACR 2012/182}}\BibitemShut
  {NoStop}%
\bibitem [{\citenamefont {Zhao}\ \emph {et~al.}(2008)\citenamefont {Zhao},
  \citenamefont {Fung}, \citenamefont {Qi}, \citenamefont {Chen},\ and\
  \citenamefont {Lo}}]{Zhaoetal2008}%
  \BibitemOpen
  \bibfield  {author} {\bibinfo {author} {\bibnamefont {Zhao}, \bibfnamefont
  {Yi}}, \bibinfo {author} {\bibfnamefont {Chi-Hang~Fred}\ \bibnamefont
  {Fung}}, \bibinfo {author} {\bibfnamefont {Bing}\ \bibnamefont {Qi}},
  \bibinfo {author} {\bibfnamefont {Christine}\ \bibnamefont {Chen}}, \ and\
  \bibinfo {author} {\bibfnamefont {Hoi-Kwong}\ \bibnamefont {Lo}}} (\bibinfo
  {year} {2008}),\ \bibfield  {title} {\enquote {\bibinfo {title} {Quantum
  hacking: Experimental demonstration of time-shift attack against practical
  quantum-key-distribution systems},}\ }\href {\doibase
  10.1103/PhysRevA.78.042333} {\bibfield  {journal} {\bibinfo  {journal} {Phys.
  Rev. A}\ }\textbf {\bibinfo {volume} {78}},\ \bibinfo {pages}
  {042333}}\BibitemShut {NoStop}%
\bibitem [{\citenamefont {Zhu}(2017)}]{Zhu17}%
  \BibitemOpen
  \bibfield  {author} {\bibinfo {author} {\bibnamefont {Zhu}, \bibfnamefont
  {Huangjun}}} (\bibinfo {year} {2017}),\ \bibfield  {title} {\enquote
  {\bibinfo {title} {Multiqubit {Clifford} groups are unitary 3-designs},}\
  }\href {\doibase 10.1103/PhysRevA.96.062336} {\bibfield  {journal} {\bibinfo
  {journal} {Phys. Rev. A}\ }\textbf {\bibinfo {volume} {96}},\ \bibinfo
  {pages} {062336}},\ \Eprint {http://arxiv.org/abs/arXiv:1510.02619}
  {arXiv:1510.02619} \BibitemShut {NoStop}%
\bibitem [{\citenamefont {Zuckerman}(1990)}]{Zuc90}%
  \BibitemOpen
  \bibfield  {author} {\bibinfo {author} {\bibnamefont {Zuckerman},
  \bibfnamefont {David}}} (\bibinfo {year} {1990}),\ \bibfield  {title}
  {\enquote {\bibinfo {title} {General weak random sources},}\ }in\ \href
  {\doibase 10.1109/FSCS.1990.89574} {\emph {\bibinfo {booktitle} {Proceedings
  of the 31st Symposium on Foundations of Computer Science, FOCS~'90}}}\
  (\bibinfo  {publisher} {IEEE})\ pp.\ \bibinfo {pages} {534--543}\BibitemShut
  {NoStop}%
\end{thebibliography}%

%\end{thebibliography}
%%%
% Recall how to use Bibtex: Within your file, the commands below are all you need, along with the line \usepackage{bibtex} in the preamble. use cite{} or citet{} (or...) to cite things. Use the labels you give to things in BibDesk. Then when you wish to populate the paper's bibliography, first run latex, then run bibtex, then run latex a few times. To run bibtex, go to the command line and type bibtex <filename>.
%A few more options for the \cite command are available. Here are some: \citet{key}	Jones et al. (1990) \citet*{key}	Jones, Baker, and Smith (1990) \citep{key}	(Jones et al. 1990) \citep*{key}	(Jones, Baker, and Smith 1990) \citep[p.~99]{key}	(Jones et al., 1990, p. 99) \citep[e.g.][]{key}	(e.g. Jones et al., 1990) \citep[e.g.][p.~99]{key}	(e.g. Jones et al., 1990, p. 99) \citeauthor{key}	Jones et al. \citeauthor*{key}	Jones, Baker, and Smith \citeyear{key}	1990 
\bibliographystyle{abbrv}
\bibliography{mathmain}


%\pagebreak
%\singlespacing
%\addtocontents{toc}{\vspace{12pt} \hspace{-1.8em} APPENDIX \vspace{-1em}}
%\appendix
%\chapter{Title of Appendix A}
%\vspace{0.5em}
%
\section[]{The effects of other parameters}

We discussed the effect of the bulge and disk masses on the
development of bars and spiral arms in the main text. Here we briefly
summarize the effects of the other parameters we investigated.


\subsection{The halo spin}

The spin of the halo is known to be an important parameter that 
affects the bar's secular evolution. 
\citet{2014ApJ...783L..18L} showed that a co-rotating disk and halo 
speed up the bar formation, but decrease its final length. This 
is due to the angular momentum transfer between the disk and halo.
If the halo does not spin it absorbs the bar's angular momentum, 
which slows down the bar and increases its length. 
A co-rotating halo, however, returns angular momentum to the disk instead of 
just absorbing it. 
This stabilizes the angular momentum transfer, and the bar evolution ceases.

We setup a few models, based on model md1mb1, but now with a rotating halo. 
In order to give spin to the halo we change the sign of the angular momentum $z$ component, $L_{\rm z}$.
For models md1mb1s0.65 and md1mb1s0.8, 65 and 85\,\% of the halo particles are rotating in the same 
direction as the disk. For models without rotation, this value is 50\,\%. 

To compare our results with previous studies, we measure the spin 
parameter~\citet{1969ApJ...155..393P,1971A&A....11..377P}:
\begin{eqnarray}
\lambda = \frac{J|E|^{1/2}}{GM_{\rm h}^{5/2}},
\end{eqnarray}
where $J$ is the magnitude of the angular momentum vector, and $E$ is the total 
energy.
In our models, $\alpha_{\rm h}=0.65$ (0.8) correspond 
to $\lambda\sim0.03$ (0.06).


In Fig.~\ref{fig:snapshots_spin_b} we present the effect that the halo spin
has on models md1mb1s0.65 and md1mb1s0.8. 
The results indicate that  
the bar is shorter for the models with a stronger halo spin.

In Fig.~\ref{fig:A2_max_spin} we show the length and maximum amplitude of
the resulting bars.
These results are consistent with previous results which show that
the length of the bar and its amplitude decay when the halo spin increases.
However, in contrast to~\citet{2013MNRAS.434.1287S} and \citet{2014ApJ...783L..18L} ,
we find that the epoch of bar formation in our models is similar, 
whereas a faster formation was expected based on the larger halo spin. 
In order to rule out the effect of run-to-run variations~\citep{2009MNRAS.398.1279S},
we performed four additional simulations for each of models md1mb1, md1mb1s0.65 and md1mb1s0.8.
For the bar formation epochs we calculated the average and standard deviation. 

The average bar formation-epoch is $0.674 \pm 0.053$,
$0.691\pm 0.083$, and $0.610\pm 0.069$\,Gyr for models md1mb1, md1mb1s0.65 and md1mb1s0.8, respectively.
This may be caused by the relatively early bar formation (within $\sim0.8$\,Gyr)
compared to the previous
studies; 1--2\,Gyr for \citet{2014ApJ...783L..18L} and 3--4\,Gyr for
\citet{2013MNRAS.434.1287S}.
Indeed, in~\citet{2014ApJ...783L..18L} the bar formation epoch starts slightly earlier when 
a moderate spin parameter ($\lambda=0.045$ and 0.06) is introduced. The
dependence of the bar formation-epoch on the halo spin is even clearer in
\citet{2013MNRAS.434.1287S}, where the formation time is longer
than in~\citet{2014ApJ...783L..18L}.
We therefore argue that the rapid bar formation in our models may hide the sequential delay
of the bar formation as caused by the halo spin.

%
\begin{figure}
\begin{center}
  \includegraphics[width=40mm]{figures/md1_mb1_a0.65_100M_1024c.pdf}
  \includegraphics[width=40mm]{figures/md1_mb1_a0.8_100M_1024c.pdf}\\
    \caption{Snapshots for models md1mb1s0.65 (left) and md1mb1s0.8 (right), which are the same as model md1mb1 (Fig.~\ref{fig:snapshots_10Gyr},far most right panel) but now including halo spin.}\label{fig:snapshots_spin_b}
\end{center}
\end{figure}



\begin{figure*}
\includegraphics[width=\columnwidth]{figures/mode_A2max_spin.pdf}\includegraphics[width=\columnwidth]{figures/mode_Dbar_mb1s.pdf}
\caption{Time evolution of the maximum amplitude for $m=2$ (left) and the bar length (averaged for every $\sim 0.1$\,Gyr) for models md1mb1, md1mb1s0.65, and md1mb1s0.8. For each model, we performed four simulations changing the random seed (varying positions and velocities of the particles) when generating the initial realizations.
\label{fig:A2_max_spin}}
\end{figure*}

In addition to the bar forming models above, we also added halo spin to a model that 
shows no bar formation within 10\, Gyr. This model, md0.5Rd1.5s, is based on md0.5Rd1.5
but now with a halo spin of 0.8. 
In Fig.~\ref{fig:snapshots_spin_sp}, we present the snapshots of the above models at $t=10$\,Gyr. 
In contrast to the barred galaxies, their spiral structures look quite similar. 
To quantitatively compare the spiral amplitudes we use the total amplitude of the spiral arms 
given by $\sum ^{10}_{m=1} |A_m|^2$, where $A_m$ is the Fourier amplitude (Eq.~\ref{eq:Fourier}).
Instead of the bar amplitude, we measured the spirals total amplitude at 
$2.2R_{\rm d}$ and at $4.5R_{\rm d}$ (for this model 9.5 and 19.5\,kpc, respectively), 
the results are shown in Fig.~\ref{fig:mode_spin_sp}. 
The evolution of the spiral amplitudes are quite similar for both models, 
just like the pitch angle  $24^{\circ}$--$29^{\circ}$ (with)
$24^{\circ}$--$26^{\circ}$ (without halo spin) and the number of 
spiral arms $m=7$--8 for $R=10$--14\,kpc (see Table~\ref{tb:pitch_angle}).


In addition, in Fig.~\ref{fig:AM} we investigate the angular-momentum flow for 
the disk and halo as a function of time and cylindrical radius.
Following \citet{2014ApJ...783L..18L} and \citet{2009ApJ...707..218V},
we measure the change in angular momentum of the $z$-component at
every $\sim 10$\,Myr.
For the halos (top panels) there is no continuous angular
momentum transfer from the disk to the halo, but we only discern random variations
in the angular momentum. These fluctuations look stronger at outer
radii, but this is because the angular momentum changes are normalized by
the disk' angular momentum, which is smaller in the outer regions.

The angular momentum of the disks vary with time (see the red and blue
stripes in the bottom panels), but overall the disk loses only 1.9\,\%
of its initial angular momentum for models with spin and 1.7\,\% for
models without.
The amplitude of the stripes for the disks roughly corresponds to the
amplitude of the spiral pattern. In Fig.~\ref{fig:amplitude_ev}, we show the
total power as a function of cylindrical radius and time for models
md0.5Rd1.5 (left) and md0.5Rd1.5s (right).
From this we conclude that for spiral arms the angular momentum transfer between the disk and 
the halo is not efficient.
On the other hand, for barred galaxies the angular momentum flow 
from the disk to the halo is considerably smaller for models with 
a larger halo spin \citep[see Fig. 3 in][]{2014ApJ...783L..18L}.


\begin{figure}
\includegraphics[width=40mm]{figures/md0.5_Rd1.5_110M_1024c.pdf}\includegraphics[width=40mm]{figures/md0.5_Rd1.5_a0.8_110M_1024c.pdf}\\
\caption{Snapshots for models md0.5Rd1.5 (left) and md0.5Rd1.5s (right). \label{fig:snapshots_spin_sp}}
\end{figure}

\begin{figure*}
  \includegraphics[width=\columnwidth]{figures/mode_rot_spiral_R10.pdf}
  \includegraphics[width=\columnwidth]{figures/mode_rot_spiral_R19.pdf}\\
\caption{Total power for models md0.5Rd1.5 and md0.5Rd1.5s at $R=9.5$ kpc (left) and 19.5 kpc (right). \label{fig:mode_spin_sp}}
\end{figure*}

\begin{figure*}
  \includegraphics[width=\columnwidth]{figures/AM_evolution_halo_md0.5mb1Rd1.5.pdf}\includegraphics[width=\columnwidth]{figures/AM_evolution_halo_md0.5mb1Rd1.5s.pdf}\\
  \includegraphics[width=\columnwidth]{figures/AM_evolution_disk_md0.5mb1Rd1.5.pdf}\includegraphics[width=\columnwidth]{figures/AM_evolution_disk_md0.5mb1Rd1.5s.pdf}\\
\caption{Angular momentum flow of the halo (top) and the disk (bottom) as a function of cylindrical radius and time for models md0.5mb1Rd1.5 (left) and md0.5mb1Rd1.5s (right). The angular momentum flow is calculated from the angular momentum's change in the $z$-component for every $\sim 10$\,Myr. The value (color) is scaled to the initial angular momentum of the disk at each radius for both the disks and halos.}
\label{fig:AM}
\end{figure*}


\begin{figure*}
  \includegraphics[width=\columnwidth]{figures/amplitude_evolution_md0.5mb1Rd1.5.pdf}
  \includegraphics[width=\columnwidth]{figures/amplitude_evolution_md0.5mb1Rd1.5s.pdf}\\
\caption{Total power as a function of cylindrical radius and time for models md0.5Rd1.5 (left) and md0.5Rd1.5s (right).}\label{fig:amplitude_ev}
\end{figure*}



\subsection{Initial Q value}

To verify the expectation that the initial value of Toomre's $Q$ parameter 
($Q_0$) influences the bar and spiral structure, we created a set of models in 
which we varied this parameter. 

The models are based on md0.5mb0, with one having an initially unstable disk
(md0.5mb0Q0.5) and 
the other having a large $Q_0$, in which no spiral arms develop (md0.5mb0Q2.0).
The time evolution of the bar's amplitude and length is presented in
Fig.~\ref{fig:A2_max_Q} 
and the surface densities are shown in Fig.~\ref{fig:snapshots_Q}.
For md0.5mb0Q2.0 there is no sign of spiral or bar structure until $\sim 5$\,Gyr, but 
a bar develops shortly after that (left panel of Fig.~\ref{fig:A2_max_Q}).
This matches with the 
expectation that $Q_0$ influences the bar formation epoch,
the smaller the $Q_0$ value the faster the bar forms.
The peak amplitude just after the bar formation is higher for
the larger $Q_0$, but the final amplitude is similar
(see the left panel of Fig.~\ref{fig:A2_max_Q}).
We also confirmed that the final bar length does not depend on $Q_0$
(see the right panel of Fig.~\ref{fig:A2_max_Q}). 
However, the radius that gives the maximum amplitude is different for 
the models with a large or a small value of $Q$.
The radius for $A_{\rm 2, max}$ is 2.6 and 4.9\,kpc for models with $Q_0=0.5$
and $2.0$, respectively. This result is qualitatively consistent with 
\citet{2012PASJ...64....5H} where an initially colder disk forms
a weaker and more compact bar due to the smaller velocity dispersion of the disk
(although they stopped their simulation just after the first amplitude peak).

This further proves (as discussed in Section~3.3) that the growth 
rate of swing amplification governs the bar formation timescale.
The growth rate decreases 
as $Q$ increases \citep{1981seng.proc..111T} which is confirmed by our simulations. 
With $Q_0=2.0$, the disk is initially stable and hence the spiral structure has 
to be induced by the bar. These ring-like spiral arms are sometimes seen in SB0--SBa 
galaxies such as NGC\,5101 \citep{2011ApJS..197...21H}.


\begin{figure*}
\includegraphics[width=\columnwidth]{figures/mode_A2max_Q.pdf}\includegraphics[width=\columnwidth]{figures/mode_Dbar_Q.pdf}
\caption{Same as Fig.~\ref{fig:A2_max_mdisk}, but for models md0.5mb0Q0.5 and md0.5mb0Q2.0.
\label{fig:A2_max_Q}}
\end{figure*}


\begin{figure}
\begin{center}
\includegraphics[width=40mm]{figures/md0.5_mb0_Q0.5_1536c.pdf}\includegraphics[width=40mm]{figures/md0.5_mb0_Q2.0_1536c.pdf}\\
    \caption{Snapshots for models md0.5mb0Q0.5 (left) and md0.5mb0Q2.0 (right).\label{fig:snapshots_Q}}
\end{center}
\end{figure}

\subsection{Disk scale length}

We further examine models md1mb1Rd1.5 and md0.5mb1Rd1.5, which
have a larger disk length scale. For these models the total disk mass is the same 
as that of models md1mb1 and md0.5mb1, but the disk scale length 
is larger. The changed disk scale length results in different rotation 
curves (see Fig.~\ref{fig:snapshots_Rdisk}). Given  Eq.~\ref{eq:mX} we expect 
that this leads to fewer spiral arms. The top views of these models are presented in
Fig.~\ref{fig:snapshots_Rdisk} (right panels) and the evolution of 
the bar's amplitude and length in Fig.~\ref{fig:A2_max_Rd}. 
The bar formation epoch of model md1mb1Rd1.5 (2\,Gyr) is
later than that of model mdmb1 (1\,Gyr). Model md0.5mb1Rd1.5 did not form 
a bar within 10\,Gyr, although model md0.5mb1 formed a bar at $\sim6$\,Gyr.
The difference
between these models is that the disk mass fraction ($f_{\rm d}$) for model
md1mb1R1.5 and md0.5mb1R1.5 is smaller than those for model md1mb1 
and md0.5mb1 (see Table~\ref{tb:bar_crit}).
Although the bar formation starts later for model md1mb1Rd1.5, the bar grows
faster, and 
the final bar length at 10\,Gyr is comparable for these models.
The bar's secular evolution, however, may continue further. 
In order to understand what decides the final bar length further simulations
are required. 


\begin{figure}
\includegraphics[width=50mm]{figures/rotation_curve_md1_Rd1.5.pdf}\includegraphics[width=38mm]{figures/md1_Rd1.5_80M_1024c.pdf}\\
\includegraphics[width=50mm]{figures/rotation_curve_md0.5_Rd1.5.pdf}\includegraphics[width=38mm]{figures/md0.5_Rd1.5_110M_1024c.pdf}\\
    \caption{Rotation curves (left) and snapshots at 10 Gyr (right) for models md1mb1Rd1.5 (top) and md0.5mb1Rd1.5 (bottom). 
    The gray dashed curve is the same as the one in Fig.~\ref{fig:snapshots_mb10}. \label{fig:snapshots_Rdisk}}
\end{figure}


\begin{figure*}
\includegraphics[width=\columnwidth]{figures/mode_A2max_Rd.pdf}\includegraphics[width=\columnwidth]{figures/mode_Dbar_Rd.pdf}
\caption{Same as Fig.~\ref{fig:A2_max_mdisk}, but now for models md1mb1Rd1.5 and md0.5mb1Rd1.5 with md1mb1 shown as reference.
\label{fig:A2_max_Rd}}
\end{figure*}


%AppendixA
%\pagebreak
%\chapter{Title of Appendix B}
%\vspace{0.5em}
%%!TEX root = hopfwright.tex

%%%%%%%%%%%%%%%%%%
%%% Appendix B %%%
%%%%%%%%%%%%%%%%%%

\section{Appendix: Endomorphism on a Compact Domain}
\label{sec:CompactDomain}



In order to construct the Newton-like map $T$ we defined operators $ A =  DF(\bar{x}_\epsilon) + \cO(\epsilon^2)$ and $A^{\dagger} = A^{-1} + \cO(\epsilon^2)$. 
However, as $(\bar{\alpha}_\epsilon,\bar{\omega}_\epsilon,\bar{c}_\epsilon) = (\pp,\pp,\bar{c}_\epsilon) + \cO(\epsilon^2)$,  the map $A$ can be better thought of as an $\cO(\epsilon^2)$ approximation of $DF(\pp,\pp,\bar{c}_\epsilon)$. 
Thus, when working with the map $T$ and considering points $ x \in  B_\epsilon(r,\rho)$ in its domain, we will often have to measure the distances of $ \alpha$ and $ \omega $ from $ \pp$. 
To that end, we define the following variables which will be used throughout the rest of the appendices. 
\begin{definition}
	\label{def:DeltaDef}
For $ \epsilon \geq 0$, and $r_\alpha,r_\omega,r_c >0$ we define 
\begin{alignat*}{2}
	\da^0 	&:= \tfrac{\epsilon^2}{5} ( 3 \pp -1) & \qquad\qquad
	\da 	&:= \da^0 + r_\alpha \\
	\dw^0 &:=  \tfrac{\epsilon^2}{5} &
	\dw &:=  \dw^0 + r_{\omega} \\ 
	\dc^0 &:=  \tfrac{2 \epsilon}{\sqrt{5}} &
	\dc &:=  \dc^0 + r_c . 
	% \\
	% \dt^0  &:= \dw^0 + \tfrac{1}{2} (\dw^0)^2 &
	% \dt  &:= \dw + \tfrac{1}{2} \dw^2 \\
	% \dtt^0  &:= 2 \dw^0 + \tfrac{1}{2} (2\dw^0)^2 &
	% \dtt  &:= 2 \dw + \tfrac{1}{2} (2\dw)^2  .
\end{alignat*}
\end{definition}


% \note[J]{
% 	I believe that we can replace the bounds $\dt$ by $\dw$  and $\dtt$ by $2 \dw$.In short, this follows from the following estimate.
% 	\[
% 	| e^{-i \omega }+i| \leq \int_{\pp}^\omega |\tfrac{\partial}{\partial \omega}  e^{-i \omega} | d\omega \leq  \int_{\pp}^\omega |1| d\omega = |\omega - \pp| .
% 	\]
% 	I have not gone through and done this yet. }
% \note[JB]{I think you are right. I think it also follows from $|e^{-i(\pp+\dw)}+i|^2=|e^{-i\dw}-1|^2 = (\cos \dw -1)^2+(\sin \dw)^2=2(1-\cos\dw) \leq 2 \cdot \frac{1}{2} \dw^2$.}
%
When considering an element $ ( \alpha , \omega, c)$ for our $\cO(\epsilon^2)$ analysis, we are often concerned with the 
 distances $|\alpha - \pp|$, $|\omega - \pp|$ and $ \| c - \bar{c}_\epsilon\|$, each of which is of order $\epsilon^2$.  
To create some  notational consistency in these definitions, $\da^0$ and $\dw^0$ are of order $\epsilon^2$, whereas $\dc^0$ is not capitalized as it is of order $\epsilon$. 
Using these definitions, it follows that for any $\rho>0$ and all  $(\alpha, \omega, c ) \in B_\epsilon(r,\rho)$ we have: 
\begin{alignat*}{1}
| \alpha - \pp | & \leq  \da       \\ 
	 | \omega - \pp| & \leq  \dw   \\
	\|c \| &\leq  \dc  .
	%  \\
	% | e^{- i \omega} + i| &\leq  \dt \\
	% | e^{-2 i \omega } +1| &\leq \dtt  .
\end{alignat*}
In this interpretation the superscript $0$ simply refers to $r=0$, i.e., the center of the ball $(\alpha,\omega,c) = \bx_\epsilon$.

The following elementary lemma will be used frequently in the estimates. 
\begin{lemma}\label{lem:deltatheta}
For all $x\in \R$ we have $|e^{ix}-1| \leq |x|$.
Furthermore, for all $|\omega - \bomega_\epsilon  | \leq r_\omega$  
%\note[JB]{I think this should be $|\omega - \bomega_\epsilon| \leq r_\omega$, no?} \note[J]{Yes, that is correct } 
we have 
$ |e^{- i \omega} + i| \leq  \dw$ and
$ | e^{-2 i \omega } +1| \leq 2 \dw $ .
\end{lemma}
\begin{proof}
We start with
\[
  |e^{ix}-1|^2 = (\cos x -1)^2+(\sin x)^2=2(1-\cos x) \leq 2 \cdot \tfrac{1}{2} x^2 = x^2.
\]
% Let $w = \omega - \pp$. Then $|w| \leq \dw$ and, using the previous inequality,
% \[
% | e^{- i \omega} + i|^2=
% |e^{-i(\pp+w)}+i|^2=|e^{-i w}-1|^2 \leq  w^2 =  \dw^2.
% \]
% \note[J]{To avoid using $w$ and $\omega$ in the same line, I propose we switch $ w \mapsto \theta$, as below. Also the last equality should be an inequality.}

Let $\theta = \omega - \pp$. Then $|\theta| \leq \dw$ and, using the previous inequality,
\[
| e^{- i \omega} + i|^2=
|e^{-i(\pp+\theta)}+i|^2=|e^{-i\theta}-1|^2 \leq  \theta^2 \leq  \dw^2.
\]
The final asserted inequality follows from an analogous argument.
\end{proof}


While the operators $U_\omega$ and $L_\omega$ are not continuous in $ \omega$ on all of $ \ell^1_0$, they are within the compact set $ B_\epsilon(r,\rho)$. 
To denote the derivative of these operators, we  define
\begin{alignat}{1}
	U_{\omega}' &:=  - i K^{-1} U_{\omega} \nonumber \\
	L_{\omega}' &:= - i \sigma^+( e^{- i \omega} I + K^{-1} U_{\omega}) + i \sigma^-(e^{i \omega} I - K^{-1} U_{\omega})  , \label{e:Lomegaprime}
\end{alignat}
and we derive Lipschitz bounds on $U_\omega$ and $L_\omega$ in the following proposition.
 
\begin{proposition}
	\label{prop:OmegaDerivatives}
	For the definitions above, $ \frac{\partial }{\partial  \omega} U_\omega = U_{\omega}' $ and $ \frac{\partial }{\partial  \omega}  L_\omega= L_{\omega}' $. 
	Furthermore,  for any $ (\alpha, \omega,c) \in B_\epsilon(r,\rho)$, we have the norm estimates
	\begin{alignat}{1}
	\| (U_{\omega} - U_{\omega_0} )c \| &\leq   \dw  \rho \nonumber  \\
	\|( L_{\omega} - L_{\omega_0} )c \| &\leq  2  \dw (  \dc +  \rho) .
	\label{e:LomegaLip}
	\end{alignat}
\end{proposition}
% \note[J]{ There was a mistake in the statement of this proposition. I changed the estimate $\| U_{\omega} - U_{\omega_0}  \| $ to $ \| (U_{\omega} - U_{\omega_0} )c \| $. Likewise for $ L_\omega$. }

\begin{proof}
One easily calculates that $ \frac{\partial U_\omega}{\partial  \omega} =  U_{\omega}'$,  whereby
$
	\| (U_{\omega} - U_{\omega_0} )c \| \leq \int_{\omega_0}^\omega \| \tfrac{\partial}{\partial \omega} U_\omega c \|  \leq    \dw  \rho  
$. 
Calculating $ \frac{\partial }{\partial  \omega}  L_{\omega} $, we obtain the following:
\begin{alignat*}{1}
 \frac{\partial }{\partial  \omega}  L_{\omega} 
&=  \frac{\partial }{\partial  \omega} \left[  \sigma^+( e^{- i \omega} I + U_{\omega}) + \sigma^-(e^{i \omega} I + U_{\omega}) \right] \\
&= - i \sigma^+( e^{- i \omega} I + K^{-1} U_{\omega}) + i \sigma^-(e^{i \omega} I - K^{-1} U_{\omega}) ,
\end{alignat*}
thus proving $ \frac{\partial L_\omega}{\partial  \omega} =  L_{\omega}'$,
and 
$\|( L_{\omega} - L_{\omega_0} )c \| \leq  \int_{\omega_0}^\omega \| \tfrac{\partial}{\partial \omega} L_\omega c \|  \leq   \dw ( 2  \dc + 2 \rho)$.
\end{proof}

\begin{proposition}
	Let $\epsilon\geq 0$ and  $r=(r_\alpha,r_\omega,r_c) \in \R^3_+$. 
	For any $ \rho > 0$ the map 
	 $T:B_{\epsilon}(r,\rho) \to \R^2 \times \ell^K_0 $ is well defined. 	
	We define functions 
% \note[J]{New definitions for $C_0$ and $C_1$ as the old ones did not quite match the estimates proven below. }
	\begin{alignat*}{1}
%	C_0 &:=  \frac{2 \epsilon^2}{\pi} 
%	\left[
%		\frac{8}{5},\frac{8}{5\sqrt{5}} \sqrt{\left(1-3 \pi /4 \right)^2+(2+\pi )^2},\frac{5 \pi }{2} 
%	\right]
%	\cdot \overline{A_0^{-1} A_1} \cdot [ \da , \dw , \dc ]^T ,
%%	\\
	% C_0 &:=  \frac{2 \epsilon^2}{\pi}
	% \left[
	% 	\frac{8}{5},\frac{2}{5} \sqrt{16+ 8\pi + 5 \pi^2},\frac{5 \pi }{2}
	% \right]
	% \cdot \overline{A_0^{-1} A_1} \cdot [ \da , \dw , \dc ]^T ,
	% \\
	C_0 &:=  \frac{2 \epsilon^2}{\pi} 
	\left[
	\frac{8}{5},\frac{2}{5} \sqrt{16+ 8\pi + 5 \pi^2},\frac{5 \pi }{2} 
	\right]
	\cdot \overline{A_0^{-1} A_1} \cdot [0,0 , \dc ]^T ,
	\\
%	C_1 &:= \frac{5 }{2 \pi} \left(1 +   \frac{4 \epsilon  }{5} \left(2+\sqrt{5}\right) \right) , \\
	% C_1 &:= \frac{5 }{2 \pi} + 2  \epsilon   \left(2+\sqrt{5}\right)  , \\
	C_1 &:= \frac{5 }{2 \pi} + \frac{\epsilon \sqrt{10}}{\pi}, \\
	C_2 &:= \dw  \left[  (1 + \pp) + \epsilon \pi  \right] , \\
	C_3 &:=  
	\da (2+ \dc) +	2 \dw (1+\pp) 
		+ \epsilon \left[ \pi + 2\da  + 4 \dc \da + \pi \dw \dc  + (\pp + \da ) \dc^2 \right] ,
	\end{alignat*}
where the expression for $C_0$ should be read as a product of a row vector, a $(3 \times 3)$ matrix and a column vector.
Furthermore we define, for any $\epsilon,r_\omega$ such that $C_1 C_2 <1$,
	\begin{equation}
		C(\epsilon,r_\alpha,r_\omega,r_c) := \frac{C_0+ C_1 C_3}{1 - C_1 C_2}
		 \, .
		\label{eq:RhoConstant}
	\end{equation}
	All of the functions $C_0,C_1,C_2,C_3$ and $C$ are nonnegative and monotonically increasing in their arguments $\epsilon$ and~$r$. 
	Furthermore, if  $C_1 C_2 < 1$ and $	C(\epsilon,r_\alpha,r_\omega,r_c) \leq \rho $
	then $\| K^{-1} \pi_c  T( x) \| \leq \rho $
	for $x \in B_{\epsilon}(r,\rho)$. 
	\label{prop:DerivativeEndo}
\end{proposition}

% \marginpar{This proposition is vague about the actual spaces being used, ie. $\ell_1,\ell^K_0$, etc.}

\begin{proof}
	Given their definitions, it is straightforward to check that the functions $C_i$ and $C$ are monotonically increasing in their arguments.  
	To prove the second half of the proposition, we split 
	$K^{-1} \pi_c  T(x)$ into several pieces. 
%\note[JB]{$\pi_c$ and $\pi_{\ge 2}$ added. Jonathan, could you please go through this and check?}
%\note[JB]{This did not work, since we do not have that $x$ is bounded by $[\da,\dw,\dc]^T$. Jonathan: what you probably meant was what I introduce as $\pi_c^0 x$, but could you please check?} 
	We define the projection $\pi_c^0 x = (0,0,\pi_c x)$.
We then obtain
	\begin{alignat*}{1}
	K^{-1} \pi_c  T(x)  &= K^{-1} \pi_c   [ x - A^{\dagger} F(x) ]   \\
	&= K^{-1} \pi_c  [ I \pi_c^0 x -    A^{\dagger} ( A \pi_c^0 x + F(x) - A \pi_c^0 x)]  \\
	&= \epsilon^2 K^{-1} \pi_c (A_0^{-1}A_{1})^2 \pi_c^0 x + K^{-1} \pi_c A^{\dagger} (F(x) - A \pi_c^0 x) \nonumber \\
	&=  \frac{2 \epsilon^2}{i\pi} \hat{U} \pi_{\ge 2} A_1 A_0^{-1}A_{1} \pi_c^0 x +\frac{2 }{i\pi} \hat{U}  \pi_{\ge 2} (I-\epsilon A_1 A_0^{-1}) (F(x) - A\pi_c^0 x)  ,
\end{alignat*}
where we have used that $K^{-1} \pi_c A_0^{-1} = \frac{2}{i\pi} \hat{U} \pi_{\ge 2}$, with the projection $\pi_{\ge 2}$ defined in~\eqref{e:pige2}.
By using $\| \hat{U} \| \leq \frac{5}{4}$, see Proposition~\ref{p:severalnorms}, we obtain the estimate
%\note[J]{Changed $[\da,\dw,\dc ]^T$ to $[0,0,\dc ]^T$}
\begin{equation}
	\| K^{-1} \pi_c T(x) \| \leq   \frac{2 \epsilon^2}{\pi} \overline{\hat{U}\pi_{\ge 2} A_1} \cdot  \overline{ A_0^{-1}A_{1}}  \cdot
	[0,0,\dc ]^T +\frac{5 }{2 \pi} \left(1 + \epsilon \| A_1 A_0^{-1} \| \right) \|F(x) - A\pi_c^0 x \| .
	\label{eq:DerivativeEndo}
\end{equation}
Here the $(1 \times 3)$ row vector $\overline{\hat{U}\pi_{\ge 2} A_1}$ is an upper bound on $\hat{U}\pi_{\ge 2} A_1$ interpreted as a linear operator from $\R^2 \times \ell^1_0$ to $\ell^1_0$, thus extending in a straightforward manner the definition of upper bounds given in  Definition~\ref{def:upperbound}.
	
	
	We have already calculated  an expression for
	 $ \overline{ A_0^{-1}A_{1}}$ in Proposition~\ref{prop:A0A1},  and  $  \| A_1 A_0^{-1}\| =\frac{2\sqrt{10}}{5}$ by Proposition~\ref{prop:A1A0}.  In order to finish the calculation of the right hand side of Equation \eqref{eq:DerivativeEndo}, we need to  estimate  $\| F(x) - A\pi_c^0 x \|$ and $\overline{\hat{U} \pi_{\ge 2} A_1} $. 
	We first calculate a bound on $\hat{U} \pi_{\ge 2} A_1 $. 
	We note that $ \hat{U} \pi_{\ge 2} A_1  =  \hat{U} \e_2 ( i_\C A_{1,2} \pi_{\alpha,\omega})+ \hat{U} \pi_{\ge 2}A_{1,*} \pi_c$.	
As $\|\hat{U} e_2\| = \| \tfrac{4-2i}{5} \e_2\|$,
it follows from the definition of $A_{1,2}$ 
that 
\[
	 \left| i_\C  A_{1,2}
	 \left( \!\!\begin{array}{c}\alpha \\ \omega \end{array} \!\!\right) \right|  
	 \cdot \| \hat{U} \e_2 \| 
	 \leq 
	 \left(\frac{\sqrt{20}}{5} |\alpha| +  \frac{\sqrt{(2-3 \pi/2)^2 +4(2+\pi)^2}}{5} |\omega| \right)  \cdot \frac{4}{\sqrt{5}}.
\]
	To calculate $ \| \hat{U} \pi_{\ge 2} A_{1,*} \|$ we note that $ \| \hat{U}\| \leq \frac{5}{4}$ and $ \|A_{1,*}\| = \pp \| L_{\omega_0} \| \leq 2 \pi$. 
	Hence $ \| \hat{U} \pi_{\ge 2} A_{1,*} \| \leq \frac{5 \pi}{2}$. 
	Combining these results, we obtain  that
%\note[JB]{I think the second, rearranged version, of the root looks ``nicer''. }
%	\[
%	\overline{\hat{U} \pi_{\ge 2}  A_1 } = \left[\frac{8}{5},\frac{8}{5\sqrt{5}} \sqrt{\left(1-3 \pi /4 \right)^2+(2+\pi )^2},\frac{5 \pi }{2} \right].
%	\] 
	\[
	\overline{\hat{U} \pi_{\ge 2}  A_1 } = \left[\frac{8}{5},\frac{2}{5} \sqrt{16 + 8 \pi + 5 \pi^2},\frac{5 \pi }{2} \right].
	\] 
Thereby, it follows from~\eqref{eq:DerivativeEndo} that 
\begin{equation}\label{e:C0C1}
	\| K^{-1} \pi_c T(x) \| \leq C_0 + C_1 \| F(x) - A \pi_c^0 x\|. 
\end{equation}
We now calculate
	\begin{alignat*}{1}
	F(x) - A \pi_c^0 x &= 
	(i \omega + \alpha e^{-i \omega} ) \e_1 + 
	( i \omega K^{-1} + \alpha U_{\omega}) c + 
	\epsilon \alpha e^{-i \omega} \e_2  +
	\alpha \epsilon L_\omega c + 
	\alpha \epsilon [ U_{\omega} c] * c  
	\\ &\qquad 
	- \pp (i K^{-1} + U_{\omega_0} + \epsilon L_{\omega_0} ) c \\
	&= i ( \omega - \pp) K^{-1} c + ( \alpha - \pp) U_{\omega} c +  \pp ( U_{\omega} - U_{\omega_0})c  \nonumber \\
	&\qquad  + \left[i ( \omega - \pp ) + ( \alpha - \pp) e^{-i \omega} + \pp( e^{- i \omega }+ i)\right] \e_1  \nonumber
	\\ 
	&\qquad  +\epsilon  \alpha   e^{-i \omega}  \e_2  
+  ( \alpha- \pp)  \epsilon L_{\omega} c + \pp \epsilon ( L_{\omega} - L_{\omega_0}) c + \alpha \epsilon [ U_{\omega} c ] * c .
	\end{alignat*}
Taking norms and using~\eqref{e:LomegaLip} and Lemma~\ref{lem:deltatheta}, we obtain 
	\begin{alignat*}{1}
	\| F(x) - A \pi_c^0 x\|& \leq  
	 \dw \rho + \da \dc + \pp \dw \rho
    +	2 (\dw + \da + \pp \dw)  
	   \\
	&\qquad + \epsilon \left[ 2(\pp + \da ) + 4 \dc \da + \pi  \dw (  \dc + \rho) + (\pp + \da ) \dc^2 \right]  \\
		&= \dw [ (1+\pp) +   \epsilon \pi ] \rho \nonumber \\ 
	&\qquad +  \da (2 + \dc)
	+	2 \dw (1+\pp) 
	+ \epsilon \left[ \pi + 2\da  + 4 \dc \da + \pi \dw \dc  + (\pp + \da ) \dc^2 \right].  
	\end{alignat*}


	We have now computed all of the necessary constants. Thus $ \| F(x) - A \pi_c^0 x \| \leq C_2 \rho + C_3$, and from~\eqref{e:C0C1}   we obtain 
	\begin{eqnarray*}
	\| K^{-1} \pi_c T(c) \|
	&\leq & C_0 +  C_1 ( C_2  \rho + C_3),
	\end{eqnarray*}
with the constants defined in the statement of the proposition.
We would like to select values of $\rho$ for which 
	\[
	\| K^{-1} \pi_c T(c) \| \leq \rho
	\]
	This is true if  
	$	C_0 +  C_1 ( C_2  \rho + C_3) \leq \rho$, 
	or equivalently 
	\[
	\frac{C_0 + C_1 C_3 }{1 - C_1 C_2} \leq \rho.
	\]
	This proves the theorem.
\end{proof}

%%%%%%%%%%%%%%%%%%
%%% Appendix C %%%
%%%%%%%%%%%%%%%%%%


%AppendixB
%\pagebreak
%If you need to insert additional appendices, copy the previous four lines, using appendixY as the
%argument of the \input commnd for Appendix Y, for Y=C,D,E,...

%Finally, the vita section is created and included in the Table of Contents.
\chapter*{Vita}
\doublespacing
\setlength{\parindent}{1.75em}
\vspace{0.2em}
\vspace{2ex}
\addtocontents{toc}{\vspace{12pt}}
\addcontentsline{toc}{chapter}{\hspace{-1.5em} {\bf \large Vita}}
Lucius Schoenbaum was born in New Orleans, Louisiana. He finished his undergraduate degree at the University of Georgia in 2005 with the major of Mathematics, and finished his Master's degree in Philosophy, also at the University of Georgia, under the supervision of O.B. Bassler in 2008, with a thesis in philosophical logic entitled ``Philosophy, Mathematics, and Proof Theory'' containing an axiomatization of typed set theory and a critique and attempted reconciliation of the contentious debate over foundations between Hilbert and Brouwer. Wishing to improve his understanding of the philosophical issues he wrestled with, he came to Louisiana State University to  continue his studies in mathematics in August 2009. He earned a Master's degree in Mathematics studying harmonic analysis under the supervision of Gestur Olafsson in 2012. He is currently a candidate for the degree of Doctor of Philosophy in Mathematics, which will be awarded in Spring 2017. His dissertation research has been carried out under the supervision of Daniel S. Sage and Jimmie D. Lawson. 

\end{document}





