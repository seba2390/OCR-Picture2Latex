\section{Proof for Blind Curing} 
In this section, we prove that in the Blind Curing setting, we cannot cure a complete binary tree in polynomial time with budget $\OO(W^\alpha)$, where W is the {\sc CutWidth}, and $\alpha$ is any constant. A complete binary tree has {\sc CutWidth} smaller than $\log(N)$ (Proposition \ref{cl:Wtree} in the appendix). Therefore, in the rest of the paper, we set $r = \log^\alpha(N)$.

We focus on the last moments of the curing, when only $\frac{N}{r^4}$ nodes remain infected. The proof relies on the fact that by the time we cure these last $\frac{N}{r^4}$ nodes, a new set of $\frac{N}{r^4}$  nodes will have become infected in another part of the graph with probability superpolynomially close to 1.

\subsection{The infection cannot be controlled when the cut is too high} \label{sec:step1}
We start by proving that without loss of generality, we can suppose the cut between the infected set and the susceptible set is less than $r^4$ when $\frac{N}{r^4}$  nodes remain infected. If the cut is above $r^4$, the infection becomes uncontrollable with high probability, and we end up with at least as many nodes infected \footnote{It is actually more likely that a large number of new nodes will become infected. However, our proof only requires the total number of infected nodes to not decrease, so this is what we prove.}, but a cut below $r^3$ after some time steps. Therefore, supposing the cut is below $r^4$ only reduces the expected time of curing. 

Intuitively, if the budget is much smaller than the cut, the leading term in the drift of the infection process will be driven by the new infections taking place, regardless of the policy in use. Trying to eradicate, or even contain, an epidemic in these conditions would be like fighting an avalanche with a flamethrower: some snow will melt, but it will not stop the avalanche - which will only stop by itself. Similarly, we can only hope to regain some control over the infection process when the budget is at least of the same order of magnitude as the cut. 

To prove this result, we introduce a random walk $G_t$ which stochastically dominates the curing process (Lemma \ref{lem:RW}). We define a stopping time, $T_{\rm SmallCut}$, which corresponds to the first time the cut reaches $r^3$.  We prove that by the time we reach this stopping point, many infections must have taken place (Lemma \ref{lem:Wald} and \ref{cl:GmanyInfection}), which implies that many time steps must have gone by. We can then use concentration inequalities to prove there are at least as many infected nodes at $T_{\rm SmallCut}$ as there were at the beginning of the random walk (Lemma \ref{lem:MoreCures}).

\begin{definition}
Let $A_t \sim \mathcal{B}(r,\delta)$, a binomial random variable with $r$ trials and probability $\delta$, and $B_t \sim \mathcal{B}(\frac{r^3}{3},\mu)$, a binomial with $\frac{r^3}{3}$ trials and probability $\mu$. 

We define the random walk $G_t$:
\[ G_t = \sum_{t'=t_0}^t  A_{t'} - B_{t'}.\]
\end{definition}

We are especially interested in the sign of the random variable 
$$G_{T_{\rm Small Cut}} =\displaystyle \sum_{t'=t_0}^{T_{\rm Small Cut}} A_{t'} - B_{t'}.$$ 
\begin{definition}
We call the \textbf{increase in susceptible} (uninfected) \textbf{nodes} since $t_0$ the random variable $I_{t_0} - I_t$, for $t > t_0$. This is the difference between the total number of infected nodes at time $t_0$, and the total number of infected nodes at time $t > t_0$. In other words, it corresponds to the difference between the number of nodes we successfully cured and the number of newly infected nodes between the times $t_0$ and $t$. Note that if more infections than curings have happened since $t_0$, the increase in susceptible nodes is negative.
\end{definition}

\begin{definition}
A random variable $X_1$ is \textbf{stochastically dominated} by a random variable $X_2$, if $\p[X_1 \geq x] \leq \p[X_2 \geq x]$ for all $x$. 
\end{definition}

\begin{lemma} \label{lem:RW} 
Let $t_0$ be the first time such that $I_{t_0} = \frac{N}{r^4}$ and the cut is above $r^4$. The random walk $G_t$, defined above, stochastically dominates the quantity $I_{t_0} - I_t$ (the increase in susceptible nodes since $t_0$) for any $t \leq T_{\rm Small Cut}$,  for every strategy.
\proof At each time step $t$, each node $i$ is assigned a budget $r_i^t$, with $r_i^t \leq r$, and gets cured with probability $\delta_i = 1- e^{-r_i^t \cdot \tau} \leq \delta = 1 - e^{-r \cdot \tau} $. By assumption of our model, there are at most $r$ nodes being cured, among which at most $r$ are infected (we do not know for sure if the nodes we are curing are infected or not since we are in the Blind Curing setting). Each of these infected nodes can therefore return to the susceptible state with probability at best $\delta$. In other words, the number of cured nodes is stochastically dominated by a binomial variable with $r$ trials and probability $\delta$, \textit{i.e.,} it is stochastically dominated by $A_t$.

Before the stopping time, the cut is at least as big as $r^3$. The maximum degree in a tree is 3, so 3 of these edges could lead to the same node. Therefore, there are at least $\frac{r^3}{3}$ potential infections happening with probability $\mu$. $B_t$ is therefore stochastically dominated by the number of new infections in the curing process, for any strategy. 

Thus, $G_t$ stochastically dominates $I_{t_0} - I_t$, for any $t \leq T_{\rm Small Cut}$, for every strategy.
\qed 
\end{lemma}

We use random walks properties to exponentially bound the probability that $G_{T_{\rm Small Cut}}$ is positive, which correponds to more cures than infections. We recall Wald's Inequality for random walks, whose proof appears in Section 9.4 of \cite{Gallager2013}.

\begin{theorem} {Wald's identity for 2 thresholds}  \label{th:wald2} \\
Let $X_i$, $i \geq 1$ be i.i.d. and let $\gamma(r) = \log(\E[e^{rX}])$ be the Moment Generating Function (MGF) of $X_1$. Let ${\rm Int}(X)$ be the interval of $r$ over which $\gamma(r)$ exists. For each $n \geq 1$, let $S_n = X_1 + \dots +X_n$. Let $\epsilon > 0$ and $\beta < 0$ be arbitrary, and let $J$ be the smallest $n$ for which either $S_n \geq \epsilon$ or $S_n \leq \beta$. Then for each $r \in {\rm Int}(X)$:
\[E[\exp(rS_J -J{\gamma}(r))] = 1. \]
\end{theorem}

\begin{corollary} \label{cor:wald} Under the conditions of Theorem \ref{th:wald2}, assume that $\E[X] < 0 $ and that $r^* > 0$ exists such that $\gamma(r^*) = 0$. Then:
 \[\p[S_J \geq \epsilon] \leq \exp(-r^*\epsilon). \]
\end{corollary}

We now use Wald's Inequality to prove $I_{t_0} - I_t$ cannot be very large.

\begin{lemma} \label{lem:Wald} 
If the cut is above $r^3$, the probability that the increase in susceptible nodes $I_{t_0} - I_t$ is higher than $K$ is exponentially small in $K$.
\proof
The curing process is stochastically dominated by the random walk described above.  Let $P_{\rm curingK}$ be the probability that $G_t$ reaches the value $K$ before stopping. 
Using Wald's Inequality (Corollary \ref{cor:wald}): 
$$ P_{\rm curingK} \leq e^{-x^*\cdot K}. $$
where $x^*$ is a value for which the MGF of $G_t$ is 1. We prove in Proposition \ref{cl:MGF} in the appendix that there exists such a $x^* > 0$, and in Proposition \ref{cl:MGFtau0}  in the appendix that $x^*$ converges to $\log(\frac{r}{3})$ when $\tau \to 0$.   \qed 
\end{lemma}

\begin{corollary} \label{cor:polylog}
The increase in susceptible nodes  $I_{t_0} - I_t$ is bounded above by $\frac{\log^2(N)}{x^*}$ with probability at least $1 - e^{-\log^2(N)}$.
\proof
Using Lemma \ref{lem:Wald}, we have:
\begin{align*}
e^{-x^*\cdot K} &\geq  e^{-\log^2(N)} \implies K \leq \frac{\log^2(N)}{x^*}.
\end{align*}
We conclude with setting $K = I_{t_0} - I_t$.
\qed 
\end{corollary}

We deduce from the previous result that many infections must have taken place.

\begin{proposition} \label{cl:GmanyInfection} 
At $T_{\rm SmallCut}$ (when the cut reaches $r^3$), at least $\frac{ r^4 }{7}$ infections will have taken place.
\proof Let C be the number of nodes cured between $t_0$ and $T_{\rm Small Cut}$ , and I be the number of new infections in the same time period. Any curing or infection reduces the cut by at most 3, since the graph is a binary tree. Therefore:
$$ 3C + 3I \geq r^4 - r^3.$$
On the other hand, using Corollary \ref{cor:polylog}, we can bound the increase in susceptible nodes:
$$ C - I \leq \frac{\log^2(N)}{x^*} .$$
Combining the two inequalities:
\begin{align*}
 I &\geq \frac{ r^4 - r^3}{6} - \frac{\log^2(N)}{x^*} \geq_{N \gg 1} \frac{ r^4}{7}.
\end{align*} 
\qed 
\end{proposition}

The previous Proposition proved that many infections happened. We now show this implies that many time steps must have passed by, which allows us to use concentration inequalities. To prove the next Lemma, we recall Hoeffding's Inequality:
\begin{theorem}

(Hoeffding's Inequality for general bounded random variables).

 Let $X_1,\dots,X_k$ be independent random variables. Assume that $X_t \in [m_t,M_t]$ almost surely for every $i$. Then, for any $\epsilon > 0$, we have \begin{align*}
 \p\left(\displaystyle\sum_{t=1}^{k} (X_t - \E[X_t]) \geq \epsilon\right) &\leq e^{{-\frac{2\epsilon^2}{\sum_{t=1}^{k}(M_t - m_t)^2}}}.
 \end{align*}
\end{theorem}

\begin{lemma}\label{lem:MoreCures}
The probability that the random walk reaches the stopping time with $I_{t_0} - I_t < 0$ tends to 0 as $\tau \to 0$.
\proof Using Hoeffding's Inequality: 

\begin{align*}  
\p \left(\sum_{t=1}^{k} A_t - B_t \geq 0 \right) &= \p \left.(\sum_{t=1}^{k} A_t - B_t - \E[A_t - B_t] \right. \geq  \left.- k\E[A_t - B_t] \right)  \\
&\leq \exp  \left(\frac{2\cdot ( k\E[A_t - B_t])^2}{\sum_{t=1}^{k} (r - (-\frac{r^3}{3}))^2} \right) \\
&\leq e^{-k  \frac{2(r\delta - \frac{r^3}{3}\mu)^2}{(r + \frac{r^3}{3})^2} }.
\end{align*}
Let $MoreCuring$ be the event that the increase in susceptible nodes at time $T_{\rm SmallCut}$ ($I_{t_0} - I_{T_{\rm SmallCut}}$)  is non-negative. We use Hoeffding's Inequality to bound $\p \left(\sum_{t=1}^{k} A_t - B_t \geq 0 \right)$. Then, b Proposition \ref{cl:GmanyInfection}, we know that at least $I = \frac{r^4}{7}$ infections must have taken place. To simplify the notations for this proof, we introduce two new stopping times, $T_{\rm Many Infections}$ and $T_{\rm Neg Binomial RW}$. $T_{\rm Small Cut}$ stochastically dominates $T_{\rm Many Infections}$, the number of time steps it takes for the random walk to infect $I$ new nodes. Since the infection rate is at least $\frac{r^3}{3}$,  $T_{\rm Many Infections}$ in turn stochastically dominates $T_{\rm Neg Binomial RW}$, a negative binomial distribution of parameter $\frac{3I}{r^3}$ and probability of failure $\mu$. We can therefore replace $T_{\rm Small Cut}$ by the simpler quantity $T_{\rm Neg Binomial RW}$ in the following calculations: 
\begin{align*}
\p(MoreCuring) &= \p \left(\sum_{t=1}^{T_{\rm Small Cut}} A_t - B_t \geq 0 \right) \\
&= \sum_{k=0}^{\infty} \p \left(\sum_{t=1}^{k} A_t - B_t \geq 0 \right) \cdot \p \left(T_{\rm Small Cut} = k \right) \\
&\leq \sum_{k=0}^{\infty} e^{-k  \frac{2(r^3\mu - r\delta)^2}{(r^3 + r)^2} } \cdot \p \left(T_{\rm Neg Binomial RW} = k \right) \\
&\leq e^{-\frac{I}{r^3}  \frac{6(r^3\mu - r\delta)^2}{(r^3 + r)^2} } \sum_{k=0}^{\infty} e^{-k  \frac{2(r^3\mu - r\delta)^2}{(r^3 + r)^2} } \cdot \p \left(T_{\rm Neg Binomial RW} = \frac{3I}{r^3} + k \right) \\
&\leq e^{-\frac{I}{r^3}  \frac{6(r^3\mu - r\delta)^2}{(r^3 + r)^2} } \left( \frac{\mu}{1-\mu e^{- \frac{2(r^3\mu - r\delta)^2}{(r^3 + r)^2} }} \right)^\frac{3I}{r^3} \\
&\to_{\tau \to 0} 0,
\end{align*}
where we have used that the MGF of a negative binomial of parameter M, probability of success $p$, evaluated at $u$, is $\left( \frac{1-p}{1-e^up} \right)^M $.
\qed
\end{lemma}

\subsection{There exists an infected node close to the root}  \label{sec:step2}
%\begin{definition}
%While we try to cure half of the last $\frac{N}{r^4}$ infected nodes of the tree, we call an \textbf{escape} the event that $\frac{N}{r^4}$ new nodes become infected in some other part of the graph. 
%\end{definition}

%The graph can only be totally cured if no $Escape$ happens. We prove that $Escape$s are, however, very likely. 
From the moment we start curing the last $\frac{N}{r^4}$ nodes, to the moment we have cured half of them and only $\frac{N}{2r^4}$ of these nodes remain infected, we show in this section that there exists an infected node at distance $\OO(\log\log(N))$ from the root (Proposition \ref{cl:rootclose}). This node stays infected for a high number of steps (Proposition \ref{cl:timeEndgame}). 

\begin{proposition} \label{cl:rootclose} 
If we select a set of $\frac{N}{2r^4}$ nodes in a tree such that the cut of this set is lower than $r^4$, then there is at least one node from this set at distance  $9\log(r) = 9\alpha\log\log(N) = \OO(\log\log(N))$ from the root.
\proof  We prove the contrapositive: if all the nodes of this set are at distance greater than $9\alpha\log\log(N)$ from the root, then the cut is higher than $r^4$.

Any subtree rooted at distance $9\alpha\log\log(N)$ from the root contains $\frac{N}{r^9}$ nodes, and has a cut of at least 1. Suppose all the $\frac{N}{2r^4}$ nodes of the selected set are at distance $9\alpha\log\log(N)$ or more from the root. We therefore need at least $\frac{N}{2r^4}/\frac{N}{r^9} = \frac{r^5}{2} $ such subtrees, for a total cut of at least $\frac{r^5}{2} > r^4 $.
Hence, the closest node is at distance at most $9\alpha\log\log(N) = \OO(\log\log(N))$ from the root. \qed 
\end{proposition}

We now show it takes many time steps to cure $\frac{N}{2r^4}$ nodes, regardless of the policy.

\begin{proposition} \label{cl:timeEndgame}
Curing half of the last $\frac{N}{r^4}$ nodes requires more than $ \frac{N}{2r^4} \cdot \frac{1}{\delta}$ time steps in expectation.
\proof If we ignore any potential infections, the time needed to cure $\frac{N}{2r^4}$ nodes is at least the sum of $\frac{N}{2r^4}$ geometric random variables of parameter $\delta$. The result follows by linearity of expectation. \qed 
\end{proposition}

\begin{proposition} 
Let $T_{\frac{N}{2r^4}}$ be the random variable representing the time to cure half of the $\frac{N}{r^4}$ last nodes. Then: 
$$\p\left(T_{\frac{N}{2r^4}} \leq \frac{N}{4r^5\delta}\right) \leq e^{-\frac{N}{8r^5}}.$$
\proof The proof can be found in the Appendix, Proposition \ref{cl:chernoffEndgame}.
\end{proposition}

Therefore, there exists an infected node close to exponentially many uninfected nodes, during at least $\frac{N}{4r^5 \delta}$ time steps. We now establish a lower bound on the probability of reinfecting $\frac{N}{r^4}$ new nodes in some other part of the graph, starting from this node.

\subsection{Low-cut case} \label{sec:BlindCuring}
We prove in this section that the probability of infecting $\frac{N}{r^4}$ new nodes in some other part of the graph, by the time it takes to cure half of the $\frac{N}{r^4}$ last infected nodes, is superpolynomially close to 1 for every strategy (Lemma \ref{lem:Pescape}). The graph can only be cured if this does not  happen.

The following Lemma is key to understanding why no strategy can prevent the reinfection. In the Blind Curing setting, we do not know which nodes are infected. Since there are exponentially many infection routes from the root of the tree, spreading the budget means there will always be a subtree on which very small budget is allocated. If the infection reaches this tree, reinfecting a lot of nodes becomes very likely.
\begin{lemma} \label{lem:minimalTree} 
For every time $t_0$, there exist $r$ subtrees containing $\frac{N}{r^3}$ nodes for which less than $\frac{t_3}{r}$ budget is used in the interval $[t_0, t_0 + t_3]$. We call any of these trees a \textbf{minimal tree} for $[t_0, t_0 + t_3]$.
\proof By the pigeonhole principle, since the total budget during this interval is $t_3 \cdot r$, and there are at least $r^3$ disjoint subtrees containing $\frac{N}{r^3}$ nodes (Proposition \ref{cl:Nsubtrees} in the appendix), there are at least $r$ subtrees that contain less than $ r \cdot \frac{t_3 \cdot r}{r^3} = \frac{t_3}{r}$ budget on this interval. \qed 
\end{lemma}

\begin{figure}[H]
\centering\includegraphics[width=8cm]{p123.pdf}
\caption{Visual representation of $P_{\rm root}(t_1)$, $P_{\rm mintree}(t_2)$, and $P_{\rm 3r}(t_3)$.}
\end{figure}

\begin{definition}
From Proposition \ref{cl:rootclose}, we know there exists an infected node close to the root. We call an \textbf{Escape$(t_0,t_1, t_2, t_3)$} the conjunction of the following events:
\begin{enumerate}
	\item At time $t_0$, this node infects its parent.
	\item The infection propagates from the parent node to the root in time $t_1$, without any node being cured.
	\item The infection propagates from the root of the tree to the root of a minimal tree for $[t_0+t_1+t_2, t_0+t_1+t_2+t_3]$ in time $t_2$, without any node being cured.
	\item $3r$ new nodes are infected in a minimal tree for $[t_0+t_1+t_2, t_0+t_1+t_2+t_3]$ in time $t_3$, without any node in a minimal tree being cured.
	\item The number of newly infected nodes reaches $\frac{N}{r^4}$ before it reaches $r$.
\end{enumerate}
\end{definition}

We notice that if an $Escape$ happens, then $\frac{N}{r^4}$ new nodes in some other part of the graph were reinfected. However, it is possible to reinfect $\frac{N}{r^4}$ new nodes without any $Escape$ happening. 

If $\{t_0, t_1, t_2, t_3\} \neq \{t_0', t_1', t_2', t_3'\}$, then $Escape(t_0, t_1, t_2, t_3)$ and
\\ $Escape(t_0', t_1', t_2', t_3')$  are disjoint events.

%We notice that if there exist $t_1, t_2$ and $t_3$ such that the infection first reaches the root in time $t_1$, then reaches a minimal tree for $[t_1 + t_2, \, t_1 + t_2 + t_3]$ in time $t_2$, then infects $3\cdot r$ nodes  in time $t_3$, and finally spreads to $\frac{N}{r^4}$ new nodes, then an $Escape$ must have happened. We use this fact to establish a lower bound on the probability of an $Escape$. 
We notice that the probability of all the events defined above is independent of $t_0$. To simplify notations, we set $t_0=0$ for the following definitions. 
\begin{itemize}
\item $P_{\rm root}(t_1)$, the probability that the infection reaches the root in time exactly $t_1$.
\item $P_{\rm mintree}(t_2)$, the probability that the infection reaches a minimal tree for $[t_1 + t_2, \, t_1 + t_2 + t_3]$ in time $t_2$, conditioned on the fact that the root of the tree is infected. Interestingly, by symmetry of the binary tree (all potential minimal trees are at the same distance from the root of the tree), this quantity does not depend on $t_1$ or $t_3$.
\item $P_{\rm 3r}(t_3)$, the probability that $3\cdot r$ nodes are reinfected in a minimal tree in time $t_3$, conditioned on the fact that the root of a minimal tree is infected. 
\item $P_{\rm spread}$, the probability that the increase in susceptible nodes since time $t_1 + t_2 + t_3$ reaches $-\frac{N}{r^4} + 3\cdot r$ before it reaches $2r$, conditioned on the fact that $3r$ new nodes are infected at time $t_1 + t_2 + t_3$.
\end{itemize}

The proof relies on the fact that \textit{no strategy can adapt to the infection moving towards a minimal tree}. In the Blind Curing setting, most of the budget is wasted covering nodes which are not infected or about to be infected, while most of the graph is left unprotected.


We now bound the probabilities defined above.

\begin{proposition}
\begin{align*} 
  \bullet \,\,P_{\rm root}(t_1) \,\quad&\geq   {t_1 \choose 9\alpha\log\log(N)}  \cdot \mu^{9\alpha\log\log(N)+1} (1-\mu)^{t_1 - 9\alpha\log\log(N)} (1-\delta)^{t_1}, \\
  \bullet \,\,P_{\rm mintree}(t_2) &\geq    {t_2 \choose 3\alpha\log\log(N)}    \cdot  \mu^{3\alpha\log\log(N)+1} (1-\mu)^{t_2 - 3\alpha\log\log(N)} (1-\delta)^{t_2}, \\
    \bullet \,\, P_{\rm 3r}(t_3) \,\,\,\,\,\quad&\geq  {t_3 \choose 3 r}  e^{-\frac{t_3}{r}\cdot \tau}  \cdot  \mu^{3 r+1} (1-\mu)^{t_3 - 3 r} e^{-\frac{t_3}{r}\cdot \tau}.
    \end{align*}
\proof 
\begin{itemize}
\item Straightforward combinatorics result.
\item Same, noticing $\log(r^3) = 3\alpha\log\log(N)$.
\item Since we are in a minimal tree for $[t_1 + t_2, \, t_1 + t_2 + t_3]$, the total budget that can be spread among all nodes during this time is $\frac{t_3}{r}$ (Lemma \ref{lem:minimalTree}). Let $r_i^{t}$ be the budget spent on node $i$ at time $t$, and let $\delta_i^{t}$ be the probability that node $i$ is cured at time $t$.
\begin{align*}
P_{\rm 3r}(t_3) &\geq   {t_3 \choose 3 r}  \mu^{3 r+1} (1-\mu)^{t_3 - 3 r} \prod_{\text{time} \, t=1}^{t_3} \prod_{\substack{\text{node } i \text{ in} \\ \text{minimal tree}}} (1 - \delta_i^{t} ) \\ 
&= {t_3 \choose 3 r}  \mu^{3 r+1} (1-\mu)^{t_3 - 3 r} \prod_{\text{time} \, t=1}^{t_3} \prod_{\substack{\text{node } i \text{ in} \\ \text{minimal tree}}} e^{-r_i^{t} \tau} \\
&= {t_3 \choose 3 r}  \mu^{3 r+1} (1-\mu)^{t_3 - 3 r}  e^{-\frac{t_3}{r}\cdot \tau}.
\end{align*}
\end{itemize}
\qed 
\end{proposition}

\begin{proposition} \label{cl:Pspread} 
Conditioned on the cut of the infected set being at least $3r$, the probability that the increase in susceptible nodes since time $t_1 + t_2 + t_3$ reaches $-\frac{N}{r^4} + 3\cdot r$ before it reaches $2r$, is at least $\frac{1 - \frac{1}{2}^{3\cdot r}}{1 - \frac{1}{2}^{\frac{N}{r^4}}} \geq \frac{1}{2}$.
\proof This is a classic Gambler's Ruin problem, with low boundary $2r$ and high boundary $\frac{N}{r^4}$. During the infection process, which starts with $3r$ infected nodes, the cut is always higher than $2r$, so the infection rate is always higher than $2r$, while the curing rate is $r$. The proof can be found in ~\cite{GrimmettGeoffreyStirzaker2001}.
\qed 
\end{proposition}

We now combine the previous results to bound the probability of escaping in one time step.

\begin{lemma} \label{lem:Pescape}
Let $P_{\rm escapeOneStep}$ be the probability that an $Escape$ starts at a given time step. Then:
\begin{align*}
P_{\rm escapeOneStep} &\geq \left(\mu \left(\frac{\mu (1-\delta)}{\delta + \mu (1-\delta)} \right)^{9\alpha\log\log(N)} \right) \cdot \left(\left(\frac{\mu e^{-\frac{1}{r}}}{(1-e^{-\frac{1}{r}}) + \mu (e^{-\frac{1}{r}})} \right)^{3r}\right)\cdot \left(\frac{1}{2} \right).
\end{align*}
Therefore, for $\tau$ sufficiently small (and in particular, as $\tau \to 0$),
$$P_{\rm escapeOneStep} \geq  \frac{\tau}{2e^3e^{12\alpha^2\log^2\log(N)}} +o(\tau).  $$
\proof We notice $P_{\rm root}(t_1)$ only depends on $t_1$, $P_{\rm mintree}(t_2)$ only depends on $t_2$, $P_{\rm 3r}(t_3)$ only depends on $t_3$, while $P_{\rm spread}$ is independent of $t_1$, $t_2$, and $t_3$. Therefore, using results from Lemma \ref{lem:Ppath} and Corollary \ref{cor:Ppath} in the Appendix which give the values of $P^{\rm startPath}$ and $P^{\rm path}$, we have:

\begin{align*}
P_{\rm escapeOneStep} &= \displaystyle\sum_{t_1, t_2, t_3 = 0}^{\infty} P_{\rm root}(t_1) \cdot  P_{\rm mintree}(t_2)  \cdot  P_{\rm 3r}(t_3) \cdot  P_{\rm spread}  \\
&= \left(\sum_{t_1= 0}^{\infty} P_{\rm root}(t_1) \right)\cdot \left(\sum_{t_2= 0}^{\infty} P_{\rm mintree}(t_2) \right) \cdot \left(\sum_{t_3 = 0}^{\infty}  P_{\rm 3r}(t_3)\right)\cdot \left(P_{\rm spread} \right) \\
&\geq \left(P^{\rm startPath}_{9\alpha\log\log(N)} \right)\cdot \left(P^{\rm path}_{3\alpha\log\log(N)} \right)\cdot \left(\left(\frac{\mu e^{-\frac{\tau}{r}}}{(1-e^{-\frac{\tau}{r}}) + \mu (e^{-\frac{\tau}{r}})} \right)^{3r}\right) \cdot \left(P_{\rm spread} \right) \\
&\geq \left(\mu\left(\frac{\mu (1-\delta)}{\delta + \mu (1-\delta)} \right)^{12\alpha\log\log(N)} \right)\cdot \left(\left(\frac{\mu e^{-\frac{\tau}{r}}}{(1-e^{-\frac{\tau}{r}}) + \mu (e^{-\frac{\tau}{r}})} \right)^{3r}\right)\cdot \left(\frac{1}{2} \right).
\end{align*}

 As $\tau \to 0$:

\begin{align*}
P_{\rm escapeOneStep} &\sim_{\tau \to 0}  \left(\tau \left(\frac{\tau }{(r+1)\tau} \right)^{12\alpha\log\log(N)} \right)\cdot \left(\frac{\tau}{(\frac{1}{r} + 1) \tau} \right)^{3r}\cdot \frac{1}{2}  \\
&\geq_{\tau \to 0} \tau  \left(e^{-12\alpha^2\log^2\log(N)} \right)\cdot \frac{e^{-\log(1+ \frac{1}{r})\cdot 3r}}{2} +o(\tau) \\
&\geq_{\tau \to 0} \tau  \left(e^{-12\alpha^2\log^2\log(N)} \right) \cdot \frac{e^{-3}}{2} +o(\tau) \\
&\geq_{\tau \to 0} \frac{\tau}{2e^3e^{12\alpha^2\log^2\log(N)}} +o(\tau).
\end{align*} 
\qed 
\end{lemma}

We therefore deduce the probability that no $Escape$ happens by the time we cure half of the $\frac{N}{r^4}$ infected nodes.

\begin{lemma} \label{lem:pNoescape}
Let $NoEscape$ be the event that no $Escape$ happens by the time we cure half of the $\frac{N}{r^4}$ infected nodes. Then:

\[\p \left(NoEscape \right) \leq_{N \gg 1} e^{-\frac{N}{e^{24\alpha^2\log^2\log(N)}}}. \]
\proof Since there are always more than $\frac{N}{2r^4}$ nodes infected, there is at least one infected node at distance $9\alpha\log\log(N)$ from the root (Lemma \ref{cl:rootclose}), which means the bound for $P_{\rm escapeOneStep}$ established in Lemma \ref{lem:Pescape} holds. Using Proposition \ref{cl:chernoffEndgame}, we split the analysis into two cases: whether we can cure  $\frac{N}{2r^4}$ in less than $\frac{N}{4r^5\delta}$ time steps or not. The probability of one $Escape$ starting at time $t$ being independent from the probability of an $Escape$ starting at any other time step $t'$:
\begin{align*}
\p \left(NoEscape \right) &\leq  \left(1- P_{\rm escapeOneStep} \right)^{T_{\frac{N}{2r^4}}} \\
&\leq  \p(T_{\frac{N}{2r^4}} \leq \frac{N}{4r^5\delta})\cdot  1 + \p(T_{\frac{N}{2r^4}} \geq \frac{N}{4r^5\delta})\cdot \left(1- P_{\rm escapeOneStep} \right)^{\frac{N}{4r^5\delta}} \\
&\leq  e^{-\frac{N}{8r^5}} + \left(1- P_{\rm escapeOneStep} \right)^{\frac{N}{4r^5\delta}}.
\end{align*}

Using Lemma \ref{lem:Pescape} to get an equivalent when $\tau \to 0$:

\begin{align*}
\p \left(NoEscape \right) &\leq e^{-\frac{N}{8r^5}} +  \left(1- \left(\mu\frac{\mu (1-\delta)}{\delta + \mu (1-\delta)} \right)^{12\alpha\log\log(N)} \right)^{\frac{N}{4 r^5\delta}} \\
&\leq_{\tau \to 0} e^{-\frac{N}{8r^5}} + \left(1 - \frac{\tau}{2e^3e^{12\alpha^2\log^2\log(N)}} + o(\tau) \right)^{\frac{N}{2r^5r\tau}}  \\ 
&\leq_{N \gg 1} e^{-\frac{N\tau}{4e^3r^6e^{12\log^2(r)}\tau}} \\
&\leq_{N \gg 1} e^{-\frac{N}{e^{24\alpha^2\log^2\log(N)}}}.
\end{align*}
\qed
\end{lemma}




\subsection{A Blind Curing result}
From Sections \ref{sec:step1} and \ref{sec:BlindCuring}, we know the graph can only be cured if we are in one of these two cases:
\begin{enumerate}
	\item The cut was above $r^4$, but we cured the whole graph anyway, which happens with probability less than $e^{-\log^2(N)}$ (Proposition \ref{cor:polylog})
	\item The cut was below $r^4$, but no $Escape$ happens by the time it takes to cure half of $\frac{N}{r^4}$ infected nodes, which happens with probability less than $e^{-\frac{N}{e^{24\alpha^2\log^2\log(N)}}}$ (Lemma \ref{lem:pNoescape}). 
\end{enumerate}

We can therefore obtain a bound on the expected time it takes to cure the whole graph.

\begin{theorem} \label{th:BlindCuring} 
In the Blind Curing setting, curing a complete binary tree takes $\Omega \left(   e^{\log^2(N)}  \right)  $ time in expectation with any budget polynomial in the {\sc CutWidth}. Therefore, no polynomial expected time curing strategy exists for budget $r=\OO(W^\alpha) = \OO(\log^\alpha(N))$, for all $\alpha$ constant.
\proof Let $CureLastNodes$ be the event that we are in case (1) or (2) described above. By union bound:
\begin{align*}
\p \left(CureLastNodes\right) &\leq e^{-\log^2(N)} + e^{-\frac{N}{e^{24\alpha^2\log^2\log(N)}}} \\
	&\leq 2e^{-\log^2(N)}.
\end{align*}
Using Proposition \ref{cl:taudelta} of the appendix, we have $\frac{1}{r} \leq \frac{\tau}{\delta}$. The number of times we try to cure the last $\frac{N}{r^4}$ is stochastically bounded below by a geometric variable of parameter $\p \left(CureLastNodes\right)$. Following Proposition \ref{cl:timeEndgame}, curing $\frac{N}{2r^4}$ lasts at least $\frac{N}{2r^4} \cdot \frac{1}{\delta}$ time steps, so $\frac{N}{2r^4}\cdot \frac{\tau}{\delta}$ time. Therefore, the expectation of the length of the curing process is the number of times we try to cure the last infected nodes, multiplied by the time it takes to cure them.

\begin{align*}
\mathbb{E}(\text{Length}) 
&\geq \underbrace{\frac{1}{\p \left(CureLastNodes\right) }}_{\substack{\text{expected number} \\ \text{of times we try}  \\ \text{to cure $\frac{N}{r^4}$ nodes}}} \cdot  \underbrace{\frac{N}{2r^4\delta}}_{\substack{\text{minimal number of } \\ \text{time steps to}  \\ \text{cure $\frac{N}{r^4}$ nodes}}} \cdot \quad \underbrace{\tau}_{\substack{\text{size of a} \\ \text{time step}}} \\
&\geq \frac{e^{\log^2(N)}}{2}  \cdot \frac{N}{2r^4} \cdot \frac{1}{r} \\
&= \Omega \left( e^{\log^2(N)} \right).
\end{align*} 
Hence, it is not curable in polynomial time for budget $r=\OO(W^\alpha) = \OO(\log^\alpha(N))$, for all $\alpha$ constant. \qed 
\end{theorem}


