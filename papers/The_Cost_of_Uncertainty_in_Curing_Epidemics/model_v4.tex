\section{Model and main contributions}
The key elements that define our model are the dynamics of the spreading process and the controlled curing process, and then the stochastic process that defines the degradation from perfect information. We describe these in detail, in this order. We then provide a few basic definitions that appear repeatedly throughout the paper, and then finally outline the main contributions of this work. 

\subsection{The SI + curing model}
\label{model}
In a standard SI (susceptible $\rightarrow$ infected) model, an epidemic spreads along edges from infected nodes to their neighbors according to an exponential spreading model: when a node becomes infected, it infects each uninfected neighbor according to an exponential random variable. SIS models are SI models where infected nodes also transition to susceptible, again at an exponential rate. Here, we consider the setting where the rate at which nodes transition from infected to susceptible is under our control, subject to a budget. How to optimally use this budget is the main question at hand. We prefer to call this a {\em controlled SI} process rather than a SIS process, because we are interested in the regime where our total curing budget is $o(N)$, where $N$ is the number of nodes. A SIS process typically has transitions from susceptible to infected of the same order as the infection rates; in our setting, this would correspond to a budget of at least $O(N)$. We note that much work has considered this setting, and has characterized the absorption time (into the ``all cured'' state) as a function of the topology of the network \cite{ganesh2005effect}.
%We consider an extended model where we have a limited curing budget that allows us to "cure" infected nodes, thus transitioning then back to susceptible. 

In the sequel, we consider a discrete, Bernoulli approximation to these exponential rate models, by considering the dynamics evolving with discrete time steps $\tau$; {\em we then take the time step $\tau$ to zero}, hence recovering the continuous time dynamics. In particular, this model is a discretization of the exponential model of \cite{Drakopoulos2015a}. As $\tau \to 0$, the models become equivalent. This discretization and the subsequent limit as $\tau \rightarrow 0$ facilitate our quantification of uncertainty, i.e., how much information we receive about the state of each node, in a given time interval. This is defined precisely below.

The dynamics of this controlled stochastic process evolve as follows. At each time $t$, for all $N$ nodes of the graph, the decision-maker assigns a budget $r_i^t$, subject to the constraints $\sum_{i =1}^{N} r_i^t = r$. During a time step of length $\tau$, each node $i$ is cured with probability $\delta_i^t = 1 - e^{-r_i^t \tau}$ if it was infected, and nothing happens otherwise -- the budget is wasted. Then, for every edge between an infected and a susceptible node, an infection occurs with probability $\mu = 1 - e^{-\tau}$. The number of infected nodes at time $t$ is given by $I_t$. In particular, since the graph is completely infected at the beginning, we have $I_0 = N$. We summarize the notation in Table \ref{notation-table}.  

\begin{figure}
	\centering\includegraphics[width=7cm]{tikz1.pdf}
	\caption{Visual representation of the different parameters -- see Table 1 for more details}
\end{figure}

\begin{table*}
  \caption{Notations}
  \label{notation-table}
  \begin{tabular}{cl}
    \toprule
$N$  & Number of nodes in the graph \\
$I_t$ & Number of infected nodes at time $t$ \\
$\tau$ & Size of a time step \\
$r_i^t$  & Budget spent on node $i$ at time $t$\\
$r = \sum_{i = 1}^{N} r_i^t$  & Total budget for each time step \\
$\mu =  1 - e^{- \tau}$ & Probability of an infection along an edge between a susceptible and an infected node \\
$\delta_{i}^{t} = 1 - e^{-r_{i}^{t} \cdot \tau}$ & Probability that node $i$ gets cured at time $t$ (if already infected)\\
$\delta = 1 - e^{-r \cdot \tau}$ & Maximum probability of being cured for a node \\
$p$ & $\mathbb{P}(\text{node $i$ raises a flag at time t } | \text{ node $i$ is infected})$ \\
 $q$ & $\mathbb{P}(\text{node $i$ raises a flag at time t } | \text{ node $i$ is susceptible})$  \\
    \bottomrule
  \end{tabular}
\end{table*}

%\begin{figure}
%\centering\includegraphics[width=10cm]{outcome4.jpg}
%\caption{4 possible outcome after one time step}
%\end{figure}

We now give a few definitions related to the above quantities, that we use throughout this paper.

\begin{definition}
We call \textbf{curing process} the stochastic process of cures and infections according to the model described in section \ref{model}. This process has a deterministic part (how much of the budget is assigned to which nodes at each time step), and a stochastic part (curing and infection follow geometric laws).
\end{definition}

\begin{definition}
We call a \textbf{strategy} the set of budgets assigned for each node at each time: $\{ r_i^t,\, i\in [N],\, t = k\cdot \tau, \, k \in \mathbb{N}\}$. We note that in the Partial Information setting that we introduce below, the actions taken at time $t_1$ may depend on the information accumulated until time $t_1 - \tau$. 
%However, in the Blind Curing setting, we have no information, and can therefore decide the whole strategy in advance.
\end{definition}

In the rest of the paper, we refer to the set of infected nodes (resp. susceptible nodes) as the \textbf{infected set} (resp. \textbf{susceptible set}). We may also refer to the cut between the infected set and the susceptible set as the \textbf{cut}. When we use the word \textbf{distance} between two nodes in a graph, we refer to the number of nodes in the shortest path between these two nodes. The distance between a node and a set is the shortest distance between this node and any node of the set.



\subsection{Partial Information/Blind Curing}
In the Complete Information setting, we assume that the status (infected or susceptible) of each node is known at each point in time. In what we call the Blind Curing model, we never have any information about the status of each node. The Blind Curing model is a technical tool we use en route to the final result. We introduce a Partial Information model that interpolates between these two extremes, and indeed is our main object of interest. Our model of partial information provides a stark tradeoff for the decision-maker: allocate resources to nodes whose status is very uncertain, and thus significantly raise the probability of wasting curing resources, or wait to collect more information and hence more certainty about the status of a node, running the risk that an infected node was allowed to infect neighbors unfettered. 
%The crux of our model is that at any time we have many false positives and negatives about which nodes are infected. The Partial Information model is a more realistic model of our knowledge of an epidemic's status. The Blind Curing model is a technical tool we use en route to the final result.

Our motivation for our partial information model comes from zero-day {\em behavioral} malware detectors, often called {\em Local Detectors} \cite{bose2008,gregoire2008}, where anti-malware software raises alerts of ``suspicious behavior'' that are then related to a central authority. We refer to these alerts as ``flags.'' Thus, in the Complete Information model, an infected node would raise a flag at each instant with probability $1$, and an uninfected node would never raise a flag. In the \textbf{Partial Information} model, at each time step, each node, independently of all others, raises a flag with some probability. The probability of getting a flag is $p$ if the node is infected, $q$ if the node is susceptible, with $p>q$. By aggregating the information about a node over multiple time steps, we can use basic concentration inequalities to deduce its state, and thus more observation time corresponds to higher certainty about a node's state.

As noted above, $p=1, q=0$ recovers the \textbf{Complete Information} setting, and $p=q$ the \textbf{Blind Curing} setting. 

In order to recover the continuous time dynamics, we let $\tau \rightarrow 0$. The key quantity that measures the amount of information per fixed unit time, is given by the rate function from Sanov's theorem, normalized by the time step: $\frac{\mathcal{D}(p||q)}{\tau}$, where $\mathcal{D}(p||q)$ is the Kullback-Leibler distance between $p$ and $q$ \cite{cover2012elements}. To understand this intuitively, this says that when $\frac{\mathcal{D}(p||q)}{\tau}$ is a constant, observing a node for a fixed period of time corresponds to administering a test with a nonzero false positive and false negative probability. That is, we can know the state of a node with constant probability of error by observing this node over a constant amount of time, which is what one expects from a real-world source of information. Note that as $\tau \rightarrow 0$, if $p-q$ is constant (or, more generally, if $D(p||q)$ goes to zero sublinearly) then we recover the Complete Information setting. Hence, the setting of interest is where $(p - q) \rightarrow 0$ as $\tau \rightarrow 0$, and the critical scaling is controlled by $\mathcal{D}(p||q)/\tau$.


\subsection{Main contributions}
Our main result consists of two parts. First, we show that there exist graphs that cannot be cured in polynomial time in the Blind Curing model. We then use this result to get a lower bound for the cost of lack of information in the Partial Information model. We obtain an expression for the lower bound that shows the required tradeoff between $\frac{\mathcal{D}(p||q)}{\tau}$ (the information available per unit of time), and the budget, $r$. 


\begin{theorem}{A Partial Information impossibility result.} \label{th:partialInfo} \\
	We consider the task of curing a fully infected complete balanced binary tree with $N$ nodes. Let $\frac{\mathcal{D}(p||q)}{\tau}$ be a measure of the amount of information we get per time step, and $r$ be the budget (curing rate) of our curing process. If 
\begin{equation}
\frac{\mathcal{D}(p||q)}{\tau} \leq  \OO\left(\frac{\log(N)\sqrt{\log(r)}}{r }\right),\label{eq:mainresult}
\end{equation}
as $\tau \to 0$, then it is fundamentally impossible for any algorithm (of any computational complexity) to cure the complete binary tree in polynomial expected time with budget $r = \OO(W^{\alpha})$, where $W$ is the {\sc CutWidth} of the graph and $\alpha$ is any constant.
\end{theorem}

For the Blind Curing case, we also have the following {\em upper bound}.
\begin{theorem}\label{thm:new_upperbound} For all $c> 0$, we can always cure the binary tree in expected linear time with budget $\OO(e^{4/c}N^c)$. In particular, our strategy does not require any information about the state of the nodes.
\end{theorem}
%We prove in Section \ref{sec:BlindCuring} that no polynomial curing is possible with no information, if our budget is polynomial in $\log(N)$. Therefore, the gap between our lower bound and our higher bound is not very wide in the Blind Curing setting.


{\bf Interpreting the result}. Suppose that if a node is observed for a fixed period of time, we can estimate its state (infected or not) with probability $1-\delta$ for $\delta$ some constant. Our results say that regardless of what this constant is, e.g., even if we have a test that takes 1 minute (or other time unit) to implement and returns a result that is $99\%$ (or any other constant quantity) accurate, then polynomial time curing is impossible, for budget any multiple of the {\sc CutWidth}. Indeed, as explained above, a constant-error estimate in a fixed unit of time corresponds to $D(p||q)/\tau$, the left-hand side of (\ref{eq:mainresult}), being a constant. On the other hand, if the budget is any multiple of the {\sc CutWidth}, the right-hand side of (\ref{eq:mainresult}) grows like $\sqrt{\log\log(N)}$, and in particular is larger than any constant. In contrast, with complete (and instantaneous) certainty of the state of each node (here the left-hand side of (\ref{eq:mainresult}) can be infinite), ~\cite{Drakopoulos2014} proves that every graph can be cured in linear expected time with budget higher than the {\sc CutWidth}.

For the blind setting, Theorem \ref{th:partialInfo} says that for budget of any polynomial of $\log(N)$, curing takes superpolynomial time. Theorem \ref{thm:new_upperbound} gives an upper bound that shows that this lower bound is not too far off; it says that a budget of $N^c$ is sufficient, for any $c > 0$. This theorem is proved in Appendix \ref{sec:upperBound}.

Our result focuses on the binary tree. Since our main result is a {\em lower bound}, this specific example is sufficient to resolve the question of whether the {\sc CutWidth} (or something proportional to it) is the right quantity to focus on to build a curing strategy robust to noise in our node estimates. In addition to this, we note that many graphs contain trees as subgraphs. Since adding nodes and edges only makes curing more difficult, our results can be seen to apply to any graph structure with a binary tree as a subgraph (as long as adding edges does not dramatically change the {\sc CutWidth} of the graph).
%\begin{corollary} 
%	We consider the task of curing a completely infected complete binary tree with $N$ nodes. 
%	
%	There exist constants $c, C \in \mathbb{R}$, such that if the measure of information $\frac{\mathcal{D}(p||q)}{\tau} \leq c$, no strategy with budget $r \leq C \cdot W$ can achieve polynomial time curing for the complete binary tree in the Partial Information setting, where $W$ is the {\sc CutWidth} of the binary tree.
%\end{corollary}

{\bf Proof Idea}. Our proof focuses on bottlenecks of the curing process: events that {\em must happen with high probability, regardless of the policy used, en route to curing an infection.} Specifically, our proof hinges on showing two such bottlenecks. First, we show that regardless of the policy, regardless of the stochastics of the curing and infection process, with high probability the last nodes to be cured cannot all be far from the root node. As we discuss below, the intuitive reason for this relies on our graph topology, and the fact that the cut between the set of infected nodes and the set of uninfected nodes must remain low if we hope to control the infection. On a binary tree, a simple calculation (Proposition \ref{cl:rootclose}) shows that any $\frac{N}{r^4}$-node set with low cut must contain nodes close to the root. The significance of this result is that at all times that matter (namely, at all points where the curing policy might be close to succeeding), there will be infected nodes that are not far from (exponentially) many uninfected nodes. Next, we show that in any interval of time, there must be many uninfected nodes {\em that are also unprotected} by the curing policy, regardless of what the curing policy is doing (Lemma \ref{lem:minimalTree}). In Theorem \ref{th:BlindCuring}, we combine these results to show that the probability that an infection begins, travels through the root to the unprotected subset of nodes and infects them before the remaining nodes are cured, is very close to 1.


