\appendix
\section{Complementary results}

\subsection{Properties of the binary tree}
We first establish a few properties of the complete binary tree.

\begin{proposition} \label{cl:Wtree} 
The {\sc CutWidth} of the complete binary tree is smaller than $\log(N)$.
\proof Consider the crusade \cite{Drakopoulos2015} implied by a Deep First Search over the tree. This crusade has a maximal cut of $\log(N) - 1$. Thus, by definition, the {\sc CutWidth} is lower than $\log(N) - 1$.

\end{proposition}

\begin{proposition} \label{cl:Nsubtrees} 
There are $r^3$ subtrees containing $\frac{N}{r^3}$ nodes, and they are at distance $\OO(\log\log(N))$ from the root.
\proof In the complete binary tree, there are $2^k$ subtrees at distance $k$ from the root that contain $\frac{N}{2^k}$ nodes. The results follows for $k = \frac{3\log(r)}{\log(2)}$. \\\qed 
\end{proposition}




\subsection{Some probabilities}

\subsubsection{Geometric variables}
\begin{proposition} \label{cl:minGeo} 
The minimum of $i$ independent geometric random variables of parameter $\mu$ is a geometric random variable of parameter $1 - (1-\mu)^i$.
\proof Let $Geo(i,\mu)$ be the minimum of $i$ independent geometric random variables. Then:
\begin{align*}
\p(Geo(i,\mu) \geq k) &= ((1 - \mu)^{k-1})^i \\
&= ((1 - \mu)^i)^{k-1}
\end{align*}
We recognize the probability distribution of a geometric variable with parameter $1 - (1-\mu)^i$.
\end{proposition}

\begin{lemma} \label{lem:infectionTime} 
As $\tau \to 0$, it takes less than $\frac{\log(k)}{\tau} $ time steps in expectation to infect $k$ new nodes.
\proof Every new infection increases the cut by 1. Let $Geo(i, \mu)$ be the minimum of $i$ geometric random variables of parameter $\mu$, and let $T_k$ be the time it takes to infect the $k$ new nodes. We have:
\begin{align*}
T_k &= \sum_{i=1}^{k-1} Geo(i, \mu)
\end{align*}
Using Claim \ref{cl:minGeo}, $Geo(i,\mu)$ is a geometric variable of parameter $1- (1-\mu)^i$. Therefore:
\begin{align*}
\E(T_k) &= \E\left(\sum_{i=1}^{k-1} Geo(i, \mu)\right) \\
&= \sum_{i=1}^{k-1} \frac{1}{1-(1-\mu)^i} \\
&=_{\tau \to 0} \sum_{i=1}^{k-1} \frac{1}{i\tau} \\ 
&\leq_{\tau \to 0} \frac{\log(k)}{\tau}
\end{align*}
\\\qed 
\end{lemma}

\subsubsection{Some curing probabilities}

\begin{proposition} \label{cl:curingDelta} 
	If all the budget at a given time step is spent, the probability that no nodes are cured in this time step is $1-\delta$.
	\proof Let $r_i$ be the budget attributed to the node $i$. Then:
	\begin{align*}
	P_{\rm NoCuring} &= \prod_{i=1}^{N} 1 - \delta_i = \prod_{i=1}^{N} e^{-r_i \tau} \\
	&= e^{-\left(\displaystyle\sum_{i=1}^{N} r_i \right) \tau} = e^{-r \tau} \\
	&= 1 - \delta.
	\end{align*}
	\qed 
\end{proposition}

\begin{lemma}\label{lem:Ppath} 
	The probability $P^{\rm path}_{\rm m}$ that $m$ nodes are reinfected along a path, such that no node on the $m$-length path is cured before they all become infected, is lower bounded by $\left(\frac{\mu (1-\delta)}{\delta + \mu (1-\delta)} \right)^{m+1}$.
	\proof Using Proposition \ref{cl:curingDelta}, 
	
	\begin{align*}
	P^{\rm path}_{\rm m} &\geq \sum_{t=0}^{\infty}  {m + t \choose m}  \mu^{m+1} (1-\mu)^t (1-\delta)^{m + t} \\
	&\geq (\mu (1-\delta))^{\rm m} \cdot \mu \cdot \sum_{t=0}^{\infty}  {m + t \choose m}   \left((1-\mu)(1-\delta)\right)^t \\
	&\geq (\mu (1-\delta))^{m} \cdot \mu \cdot \frac{1}{\left(1- (1-\mu)(1-\delta))\right)^{m+1}} \\
	&\geq \left(\frac{\mu (1-\delta)}{\delta + \mu (1-\delta)} \right)^{m+1}.
	\end{align*}
	\qed 
\end{lemma}

\begin{corollary} \label{cor:Ppath} 
	The probability $P^{\rm startPath}_{\rm m}$ that $m$ nodes are reinfected along a path, such that no node on the $m$-length path is cured before they all become infected, and such that there is an infection on the first time step, is lower bounded by $\mu\cdot \left(\frac{\mu (1-\delta)}{\delta + \mu (1-\delta)} \right)^{m}$.
	\proof Taking into account that the first time step is an infection:
	\begin{align*}
	P^{\rm startPath}_{\rm m} &\geq \sum_{t=0}^{\infty}  {m - 1 + t \choose m - 1}  \mu^{m+1} (1-\mu)^t (1-\delta)^{m + t} \\
	&\geq \mu \cdot\left(\frac{\mu (1-\delta)}{\delta + \mu (1-\delta)} \right)^{m}.
	\end{align*}
	\qed 
\end{corollary}

\begin{proposition}\label{cl:chernoffEndgame} 
	Let $T_{\frac{N}{2r^4}}$ be the random variable representing the time to cure half of the $\frac{N}{r^4}$ last nodes. Then: 
	$$\p\left(T_{\frac{N}{2r^4}} \leq \frac{N}{4r^5\delta}\right) \leq e^{-\frac{N}{8r^5}}.$$
	\proof The difficulty here lies in the fact that we want to obtain exponential concentration inequalities on a sum of geometric variables, which are unbounded. Therefore, we cannot directly use a Chernoff's bound. Following an idea from ~\cite{Brown}, we represent geometric variables as the sum of Bernoulli variables. Each variable is then bounded, which makes the analysis possible.
	
	Let $X_i^t$ be 1 if node $i$ was cured at time $t$, and 0 otherwise. Let $X^t$ be $r$ with probability $\delta$, and 0 otherwise. We notice $\p(X_i^t = 1) \leq \delta$, and $\forall t, \sum_{i=1}^{N} X_i^t \leq r$. By using Chernoff's bound on a sum of $\frac{N}{4r^5\delta}$ Bernoulli variables of parameter $\delta$, we can therefore bound the probability that curing $\frac{N}{2r^4}$ nodes happens in a short time (here less than $\frac{N}{4r^5\delta}$ time):
	\begin{align*}
	\p(T_{\frac{N}{2r^4}} \leq \frac{N}{4r^5\delta}) &= \p(\sum_{t=1}^{\frac{N}{4r^5\delta}} \sum_{i=1}^{N} X_i^t \geq \frac{N}{2r^4}) \\
	&\leq \p(\sum_{t=1}^{ \frac{N}{4r^5\delta}} X^t \geq \frac{N}{2r^4}) \\
	&\leq \p(\sum_{t=1}^{ \frac{N}{4r^5\delta}} \frac{X^t}{r} \geq \frac{N}{2r^5}) \\
	&\leq \p\left(\sum_{t=1}^{ \frac{N}{4r^5\delta}} \frac{X^t}{r} \geq \frac{N}{4r^5\delta}\cdot \delta \cdot (1+1)\right) \\
	&\leq \p\left(\sum_{t=1}^{ \frac{N}{4r^5\delta}} \frac{X^t}{r} \geq \E\left[\sum_{t=1}^{ \frac{N}{4r^5\delta}} \frac{X^t}{r} \right]\cdot (1+1)\right) \\
	&\leq e^{-\frac{ \frac{N}{4r^5}\cdot 1^2}{3}} \\
	&\leq e^{-\frac{N}{12r^5}}.
	\end{align*}
	\qed
\end{proposition}

\begin{proposition} \label{cl:noOtherInfection} 
	Conditioned on reaching a cut of $3r$ in a minimal tree in less than $\frac{30\log(r)}{\tau}$ time steps, the probability of not infecting any nodes outside of the $escape$ $P_{\rm NoOtherInfections}$ is bounded by:
	\begin{align*}
	P_{\rm NoOtherInfections} &\leq e^{-\frac{360\log^2(r)\mu}{\tau}} \\
	&\leq_{\tau \to 0} e^{-360\log^2(r)}.
	\end{align*}
	\proof
	For an infection to not be part of the $Escape$, it has to happen because of a node which is either on the path to the root, or on the path to a minimal tree. As calculated before, there are $12\log(r)$ such nodes, which all have at most one edge not on the path (the two others were used either to get infected, or to infect the next node on the path). What's more, each of these nodes was infected for at most $\frac{30\log(r)}{\tau}$ time steps. The probability of not infecting any node along those edges during all these time steps is therefore:
	\begin{align*}
	P_{\rm NoOtherInfections} &\leq \left((1-\mu)^{12\log(r)} \right)^{\frac{30\log(r)}{\tau}} \\
	&\leq (1-\mu)^{\frac{360\log^2(r)}{\tau}} \\
	&\leq e^{-\frac{360\log^2(r)\mu}{\tau}}.
	\end{align*}
	As $\tau$ goes to 0:
	\begin{align*}
	P_{\rm NoOtherInfections} &\leq  e^{-\frac{360\log^2(r)\mu}{\tau}} \\
	&\leq_{\tau \to 0} e^{-360\log^2(r)}.
	\end{align*}
	\qed
\end{proposition}

\subsubsection{Moment generating function of the random walk}

\begin{proposition} \label{cl:MGF}
There exists $x^*>0$ such that the Moment Generating Function (MGF) of $G_t$ evaluated at $x^*$ is 1.
\proof Since $G_t$ is a sum of independent random variables, and since the MGF of a Bernouilli random variable of parameter $p$ is equal to $MGF(x) = pe^x +(1-p)$:
\begin{align*}
MGF_{G_t}(x) &= MGF_{C_t}(x) \cdot MGF_{I_t}(x) \\
&= (\delta e^x + (1 - \delta))^r\cdot (\mu e^{-x} + (1 - \mu))^\frac{r^3}{3}
\end{align*}
We can see that:
 $$MGF(0) = 1, \qquad MGF'(0) < 0, \qquad MGF(r) \to_{r \to \infty} \infty$$
Therefore, by the Intermediate Value Theorem:
$$ \exists x^* > 0, MGF(x^*) = 0 $$
There is no closed form solution for $x^*$, but we can get an approximation when $\tau \to 0$. \\\qed 
\end{proposition}


\begin{proposition} \label{cl:MGFtau0} 
When $\tau \to 0$, we have a closed form solution: $x^* = \log(r) - \log(3)$.
\proof When $\tau \to 0$:

\begin{align*}
1 &= MGF_{G_t}(x^*) = (\delta e^{x^*} + (1 - \delta))^r\cdot (\mu e^{-x^*} + (1 - \mu))^\frac{r^3}{3} \\
&=_{\tau \to 0} (r\tau e^{x^*} + (1 - r\tau))^r\cdot (\tau e^{-x^*} + (1 - \tau))^\frac{r^3}{3} \\
&=_{\tau \to 0} (1 + r\tau (e^{x^*} - 1))^r (1 + \tau(e^{-x^*} - 1))^\frac{r^3}{3} \\
&=_{\tau \to 0} (1 + r^2\tau (e^{x^*} - 1)) (1 + \tau \frac{r^3}{3} (e^{-x^*} - 1)) \\
&=_{\tau \to 0} 1 + \tau \left( r^2(e^{x^*} - 1) - \frac{r^3}{3} (1 - e^{-x^*}) \right) \\
&=_{\tau \to 0} 1 + \tau \left( r^2 \cdot e^{-x^*}(e^{2x^*} - e^{x^*} - \frac{r\cdot e^{x^*}}{3}  + \frac{r}{3}) \right)
\end{align*}
If we want to nullify the first order in $\tau$, we need:

$$  (e^{x^*})^2  - (1 + \frac{r}{3}) (e^{x^*}) + \frac{r}{3} = 0 $$
This is a second order polynomial, which gives us the solution $e^{x^*} = 1$ (trivial solution for $x^* = 0$), and $e^{x^*} = \frac{r}{3}$, which gives us a non-trivial solution:

$$x^* = \log(r) - \log(3) > 0 $$ \\\qed 
\end{proposition}





\subsection{Some calculus}
\subsubsection{monotonicity results}

\begin{proposition} \label{cl:taudelta} 
The function $k(x) = \frac{x}{1-e^{-rx}}$ is increasing in x. In particular, for all $x \leq 0$, we have $k(x) \geq k(0) = \frac{1}{r}$.
\proof $x \to x$ and $x \to \frac{1}{1-e^{-rx}}$ are both increasing functions of $x$, so $k(x)$ is also increasing. \\\qed 
\end{proposition}

\section{A policy achieving the upper bound} \label{sec:upperBound}
The main contribution of this paper proves a lower bound on the budget in the Partial information setting. We prove that for budget $r = \OO(\log(N))$, there exists no strategy which allows polynomial expected curing time, unless $D(p||q)/\tau$ goes to infinity. Moreover, our result implies that if $D(p||q)/\tau = 0$, then for budget $r = \OO({\rm poly}(\log(N)))$, there exists no strategy which allows polynomial expected curing time. \\\\
We now study the converse problem. In this section we exhibit a policy which:
\begin{itemize}
	\item Does not require any knowledge of the state (works even in the Blind Curing setting);
	\item Achieves linear expected curing time;
	\item Needs $r \sim \OO(e^{\frac{4}{c}} \cdot N^c)$ budget, for any $c>0$.
\end{itemize}

\subsection{Description of the policy}
We consider the ordering $O$ of the nodes given by a Depth First Search on a binary tree. We split the graph into 3 sets: $A_{{\rm sus}}$, $A_{{\rm inf}}$ and $A_{{\rm buff}}$. Intuitively, these sets respectively represent the set of the nodes we believe are cured, the set of nodes we believe are infected, and the buffer zone in the middle. 

We run through the following algorithm. As we show in Section \ref{ssec:combining_upperbd}, the probability that we fail to cure the graph in one pass of the algorithm below (what we call one {\em iteration}), is at most $2/N$, and hence the expected time to cure, given our budget, is linear. 

To initialize each pass of the algorithm, we set $t=0$, and also initialize the sets $A_{{\rm sus}}^0 = A_{{\rm buff}}^0 = \emptyset$, $A_{{\rm inf}}^0 = V$.\\
Every $\frac{1}{\tau}$ time steps, we:
\begin{itemize}
	\item move a node from $A_{{\rm inf}}$ to $A_{{\rm buff}}$, following the ordering $O$
	\item remove all the nodes from $A_{{\rm buff}}$ which are at distance greater than $c\log(N)$ from any node of $A_{{\rm inf}}$, and place them in $A_{{\rm sus}}$
\end{itemize}
Then, during $\frac{1}{\tau}$ time steps, we:
\begin{itemize}
	\item cure all the nodes of $A_{{\rm buff}}$ with constant budget $c_1$;
	\item cure the new node with budget $(1 + c_2) \log(N)$, where $c_2$ constant.
\end{itemize}

This gives a total budget of $2^{c\log(N)}\cdot c_1 + c_2\log(N) = N^{c\log(2)} \cdot c_1 + c_2 \log(N)$.

At time step $t=\frac{N}{\tau}$, when $A_{{\rm inf}}^{\frac{N}{\tau}} = \emptyset$, we keep curing $A_{{\rm buff}}^{\frac{N}{\tau}}$ for an additional $\frac{c\log(N)}{\tau}$ time steps. 

One pass through the set of actions described above is called an \textbf{iteration}. We show below that the probability of failing to cure the entire graph in one iteration is bounded by $2/N$, and hence we can cure the graph in linear expected time. Equivalently, in time $\gamma \cdot N$, one can get a $(1-\epsilon)$-probability guarantee that the graph is cured, for any $\epsilon > 0$. 

%If at the end of one iteration there exists a node which is still infected, we start a new iteration from the beginning. Otherwise, we managed to cure the graph, which is what we wanted.

\subsection{Properties of the policy}
\begin{proposition}{Every node of the graph spends at least $\frac{c\log(N)}{\tau}$ time steps in $A_{{\rm buff}}$.}
\proof We notice $A_{{\rm inf}}$ is connected at all time. Therefore, when a node $i$ is removed from $A_{{\rm inf}}$ and added to $A_{{\rm buff}}$, it is at distance 1 from a node of $A_{{\rm inf}}$. Every subsequent node transferred from $A_{{\rm inf}}$ to $A_{{\rm buff}}$ can only increase the distance between $i$ and $A_{{\rm inf}}$ by 1. Since a new node is transferred every $\frac{1}{\tau}$ time steps, and all the nodes at distance no greater than $c\log(N)$ from $A_{{\rm inf}}$ are kept in $A_{{\rm buff}}$, every node $i$ of the graph spends at least $\frac{c\log(N)}{\tau}$ time steps in $A_{{\rm buff}}$.
\\\qed 
\end{proposition}

\begin{proposition}{Let $T_{{\rm cured}}$ be the time it takes to cure the graph, and $P_{{\rm OneIteration}}$ be an lower bound on the probability that the graph is cured in one iteration. Then:
	$$\E[T_{{\rm cured}}] \leq \frac{N + c\log(N)}{P_{{\rm OneIteration}}}.$$}
\proof $T_{{\rm cured}}$ is stochastically dominated by an exponential variable with parameter $P_{{\rm OneIteration}}$, which in turn has expectation $\frac{1}{P_{{\rm OneIteration}}}$. One iteration lasts exactly $\frac{N}{\tau} + \frac{c\log(N)}{\tau}$ time steps, and one time step lasts $\tau$ time, so an iteration lasts $N + c\log(N)$ time.
\\\qed 
\end{proposition}

\subsection{Analysis}
\begin{definition}
	We call an \textbf{epoch} $\frac{1}{\tau}$ consecutive time steps.
\end{definition}

If there is at least one infected node at the end of the policy, then either one of the following events must have happen:
\begin{enumerate}
	\item One node was not cured when it entered the buffer zone, and then proceeds to make its way to $A_{{\rm sus}}$.
	\item There was a path of infection from a node of $A_{{\rm inf}}$ to a node of $A_{{\rm sus}}$.
\end{enumerate}
We calculate the probability of the two events above happening during one epoch:

\subsubsection{Case 1: One node was not cured when it entered the buffer zone, and then proceeds to make its way to $A_{{\rm sus}}$.}
The probability of this event is lower than the probability that one node was not cured during one epoch when it entered the buffer zone:
\begin{align*}
\p(\rm{Case 1}) &\leq (1 - (1- e^{-(1+c_2)\log(N) \tau}))^{\frac{1}{\tau}} \\
&\leq e^{-(1+c_2)\log(N)} \\
&\leq \frac{1}{N^{1+c_2}}.
\end{align*}

\subsubsection{Case 2: There was a path of infection from a node of $A_{{\rm inf}}$ to a node of $A_{{\rm sus}}$.}
In the case 2:
\begin{enumerate}
	\item One node $n_s$ of $A_{buf}^{t_0}$ needs to become infected at time step $t_0$.
	\item One node $n_e$ of $A_{{\rm sus}}^{t_0 + t}$ becomes infected after $t$ time steps.
	\item Every $\frac{1}{\tau}$ time steps, the nodes of $A_{{\rm sus}}$ can become closer to where the infected node by a distance $1$. Therefore, $c\log(N) -\lfloor \tau\cdot t \rfloor - 1$ additional infections need to happen along the unique path between $n_s$ and $n_e$.
\end{enumerate}
Let us calculate the probability $p_1$ that $b$ becomes infected at time $t_0$, and then proceed to infect $c\log(N) -\lfloor \tau\cdot t \rfloor$ additional nodes along a cured path in  $t$ time steps, with the head of the infection not being cured:
\begin{align*}
p_1 = \mu \cdot {t \choose \max(0, c\log(N) -\lfloor \tau\cdot t \rfloor  - 1)} \mu^{c\log(N) -\lfloor \tau\cdot t \rfloor}(1-\beta)^{t}.
\end{align*}
Let us now sum over all time steps $t$, to get the probability $p_2$ that an infection reaches $A_{{\rm sus}}$ with exactly $d$ infections, starting from one time step:

\begin{align*}
p_2 &= \mu \cdot \displaystyle\sum_{t=\frac{c\log(N)}{1+\tau}}^{\infty}  {t \choose \max(0, c\log(N) -\lfloor \tau\cdot t \rfloor - 1)}  \mu^{c\log(N) -\lfloor \tau\cdot t \rfloor}(1-\beta)^{t} \\
&\leq \mu \cdot \displaystyle\sum_{t=\frac{c\log(N)}{1+\tau}}^{\infty}   {t \choose c\log(N) - 1}\frac{(c\log(N))!}{(c\log(N) -\lfloor \tau\cdot t \rfloor)!}\frac{t!}{(t+ \tau t)!} \\
&\qquad \cdot \mu^{c\log(N) -\lfloor \tau\cdot t \rfloor}(1-\beta)^{t} \\
&\leq \mu \cdot \displaystyle\sum_{t=\frac{c\log(N)}{1+\tau}}^{\infty}  {t \choose c\log(N) - 1}\left(\frac{c\log(N)}{t+ \lfloor \tau t \rfloor}\right)^{\lfloor \tau t\rfloor } \mu^{c\log(N) -\lfloor \tau\cdot t \rfloor}(1-\beta)^{t} \\
&\leq \mu \cdot \displaystyle\sum_{t=c\log(N)}^{\infty}  {t \choose c\log(N) - 1} \mu^{c\log(N) - \tau\cdot t}(1-\beta)^{t} \\
&= \mu \cdot \displaystyle\sum_{t'=0}^{\infty} {c\log(N)-1+t' \choose c\log(N)-1} \mu^{c\log(N)}\left(\frac{1-\beta}{\mu^\tau}\right)^{c\log(N) - 1 + t'} \\
&= \mu^{c\log(N) + 1}  \cdot \left(\frac{1-\beta}{\mu^\tau}\right)^{c\log(N) - 1} \frac{1}{\left(1-\left(\frac{1-\beta}{\mu^\tau}\right)\right)^{c\log(N)+1}} \\
&\sim_{\tau \to 0} \tau^{c\log(N) + 1}  \cdot 1^{c\log(N) - 1}  \frac{1}{\left(1-\left(\frac{1-c_1\cdot \tau}{\tau^{\tau}}\right)\right)^{c\log(N)}} + o(\tau)\\
&\sim_{\tau \to 0} \frac{\tau^{c\log(N) + 1}}{(c_1 \tau)^{c\log(N)}} + o(\tau) \\
&\sim_{\tau \to 0} \frac{\tau}{c_1^{c\log(N)}} + o(\tau)\\
&\sim_{\tau \to 0} \frac{\tau}{N^{c\cdot\log(c_1)}} + o(\tau),
\end{align*}
where we have used that $\left(\frac{c\log(N)}{t+ \lfloor \tau t \rfloor}\right) < 1$, that $\mu < 1$, so $\mu^{-\lfloor \tau\cdot t \rfloor} \leq \mu^{- \tau\cdot t}$, that $\displaystyle\sum_{k=0}^{\infty} {m + k \choose k} a^{k} = \frac{1}{(1-a)^{m+1}}$ when $|a| < 1$, and that $\tau^\tau \to_{\tau \to 0} 1$.

Now, if we select a starting node $n_s$ and an end node $n_e$, there is only one path between them in a tree. Such an infection can start $\frac{1}{\tau}$ times during one epoch. We can therefore apply a union bound:

\begin{align*}
\p(\rm Case 2) &\leq \sum_{t_0 = 1}^{\frac{1}{\tau}} \sum_{n_s, n_e} p_2 \\
&\leq \frac{N^2}{\tau} \cdot p_2 \\
&\leq \frac{1}{N^{c\cdot\log(c_1) - 2}}.
\end{align*}

\subsection{Combining the results for all time steps}
\label{ssec:combining_upperbd}
At each epoch, the probability of failure is upper bounded by $\p(\rm Case 1) + \p(\rm Case 2)$. The probability of failing during one iteration, which lasts $N + c\log(N)$ epochs, is therefore:
\begin{align*}
\p(\rm {\rm OneIteration}Fail) &\leq  (N + c\log(N))\cdot (\p(\rm Case 1)  + \p(\rm Case 2)) \\
&\leq 2N \cdot (\frac{1}{N^{c\cdot\log(c_1) - 2}} + \frac{1}{N^{1+c_2}}).
\end{align*}

Therefore, if we choose $c_2 = 1$, and $c_1 = e^{\frac{4}{c}}$, we have:

\begin{align*}
\p(\rm {\rm OneIteration}Fail) &\leq 2N \cdot (\frac{1}{N^{4 - 2}} + \frac{1}{N^{1+1}}) \\
&\leq  \frac{2}{N}.
\end{align*}


We have, therefore, an upper bound, as stated in Theorem \ref{thm:new_upperbound}. We repeat here, and complete the proof.
\vspace{0.2cm}

\noindent
{\sc Theorem} \ref{thm:new_upperbound}. {\em
	In the Blind Curing setting, for all $c>0$, we can cure the binary tree in expected linear time with budget $\OO(e^{\frac{4}{c}}\cdot N^{c} )$.
	\proof If we choose $c_2 = 1$, and $c_1 = e^{\frac{4}{c}}$, we have:
	
	\begin{align*}
	\p(\rm {\rm OneIteration}Fail) &\leq 2N \cdot (\frac{1}{N^{4 - 2}} + \frac{1}{N^{1+1}}) \\
	&\leq  \frac{2}{N}.
	\end{align*}
	
	Therefore, with budget $e^{\frac{4}{c}}\cdot N^{c} + \log(N)$, for all $c>0$:
	\begin{align*}
	\E[T_{{\rm cured}}] &\leq \frac{N + c\log(N)}{1 - \frac{2}{N}} \\
	&\leq 4N.
	\end{align*}
	}
	

\section{Numerical experiments} \label{sec:experiments}
In this section, we add some numerical experiments to illustrate the difficulty of the problem. We introduce two curing strategies: Naive Curing, which cures randomly a subset of nodes which signal themselves as infected (they raise a flag), and the strategy from Section \ref{sec:upperBound}, Blind Protection, which prevents the infection from spreading by curing every node near the infected set. We hope these two strategies can provide insight into the difficulty of curing the complete binary tree in our model. 

It is important to understand that the results we present are strategy-specific, which means better results could possibly be achieved with better strategies. Devising optimal strategies is however outside of the scope of this work.

\subsection{Impact of the lack of information}
In this section, we illustrate the dramatic impact of the lack of information on the Naive Curing strategy. In the following experiment, we consider a binary tree on 31 nodes. We use a budget $r = 16>\frac{N}{2}$. If $p_\epsilon$ is the probability of error, at each time step, an infected node raises a flag with probability $1 - p_\epsilon$, and a susceptible node raises a flag with probability $p_\epsilon$. We set the size of a time step to be $\tau = 0.1$.

\begin{figure}[H]
	\centering\includegraphics[width=8cm]{naive_cure_uncertainty.jpeg}
	\caption{Time to cure as a function of the probability of error for the Naive Curing strategy.}
	\label{fig:naive}
\end{figure}

The results can be seen in Figure \ref{fig:naive}. The time to cure increases faster than exponentially with the probability of error. We can see that even with 10\% of error, it takes more than 3500 time steps with budget $r= \frac{N}{2}$ on 31 nodes.

\subsection{Impact of size of the graph}
We now consider the strategy described in Appendix \ref{sec:upperBound}. For the purpose of these experiments, keeping the same notation as the previous section, we set $c_1 = 10$ (this is the budget for every node which we "protect"), and $c_2 = 1$ (we cure any new node with budget $(1+c_2)\log(N)$). We still have $\tau = 0.1$. For this experiment, we investigate the time it takes to cure the graph for a budget equal to different exponents of the number of nodes.

\begin{figure}[H]
	\centering\includegraphics[width=8cm]{time_to_cure_vs_budget.png}
	\caption{Time to cure as a function of the number of nodes for the Blind Protection strategy. The plots are the average of 20 runs.}
	\label{fig:Npower}
\end{figure}

The results are shown in Figure \ref{fig:Npower}. As theory predicts, the time to cure increases more slowly than $4\cdot N$, where $N$ is the number of nodes for budget $r = \OO(N^c)$, for all $c>0$ constant. 

As a reminder, in the Blind Curing setting, it is impossible to cure the complete binary tree in less than superpolynomial time for budget $r=\OO(\log^\alpha(N))$, for all $\alpha > 0$ constant.

