\section{Proof sketch}

\begin{figure*}
\centering\includegraphics[width=16cm]{proof-concept-4}
\caption{Visual representation of the main steps of the proof: when only $\frac{N}{r^4}$ nodes remain infected, no strategy can prevent the reinfection of $\frac{N}{r^4}$ new nodes in some other part of the graph. The graph can only be cured if the cycle is broken, a rare event which takes superpolynomial time in expectation.}
\end{figure*}
We first prove that polynomial curing is impossible in the Blind Curing setting if the budget is polynomial in the {\sc CutWidth}. We then show that in the Partial Information setting, we do not obtain enough information to detect threats of reinfection, and thus cannot prevent them: we are "blind" to the threats until it is too late.

Our proof in the Blind Curing setting focuses on a subprocess which is bound to happen for any curing strategy. We consider the last $\frac{N}{r^4}$ infected nodes. We show that by the time we cure these last remaining infected nodes, a new set of $\frac{N}{r^4}$ nodes becomes infected with high probability. Trying to cure the whole graph is then similar to playing a very long game of whack-a-mole with superpolynomial expected end time.

\subsection{Blind Curing setting}
\begin{description}
\item[Step 1 (Section \ref{sec:step1}):] We first show that if a strategy allows the cut between the infected and susceptible set to be much higher than the available budget $r$, the infection becomes uncontrollable. In this case, the infection rate exceeds the curing rate, and the reinfection would be inevitable even if we had complete knowledge about the infection state of each node at each time (\textit{i.e.} this happens even in the Complete Information setting). In particular, if $\frac{N}{r^4}$ nodes are infected and the cut is above $r^4$, the drift of the curing process is dominated by the infections. We can then use random walks results, such as Wald's Inequality, to prove that after a few time steps, we end up with at least as many infected nodes, but a cut below $r^3$ (we actually end up with many more infected nodes, but as many is enough for the proof). We can therefore focus on analyzing the situation in which $\frac{N}{r^4}$ nodes remain infected with a cut lower than $r^4$.

\item[Step 2 (Section \ref{sec:step2}):] Due to the topology of the binary tree, a cut below $r^4$ implies that there exists an infected node which is close to the root. This makes it easy for the infection to escape through the root, and reach a large number of susceptible nodes. One key point of the proof is that this node will remain infected (and therefore potentially infecting) for a very long time, and an infection can start at any time step during this period.

\item[Step 3 (Section \ref{sec:BlindCuring}):] Since the infection escapes through the root, the number of uninfected nodes easily accessible is very large, and specifically, larger than the budget. This makes it impossible to cover all the potential escape routes. Notice that this is very specific to the Blind Curing setting: if we knew in which direction the infection was escaping, we could prevent it as in ~\cite{Drakopoulos2015a}. It is because the number of potential infected nodes is exponentially higher than the number of actual nodes infected, and because we do not know where the infection actually is, that we end up wasting considerable curing budget on uninfected nodes. Therefore, the infection is very likely to escape, and a new set of $\frac{N}{r^4}$ nodes becomes infected again.
\end{description}


\subsection{Partial Information setting}
To extend this result to the Partial Information setting, we notice that as soon as the cut of the new infection reaches $3r$, we can use Gambler's Ruin results to show that at least $\frac{N}{r^4}$ nodes will become infected with constant probability. If we cannot detect the infection escaping until a cut of $3r$ is reached, we therefore cannot prevent the reinfection with constant probability. Using Sanov's Theorem, we show that the uncertainty in our state estimation for any node does not resolve itself quickly enough (in particular, with respect to how fast the neighborhoods of the binary tree grow). Specifically, the infection remains undetectable with constant probability until a cut of $3r$ is attained. This allows us to extend the result from the Blind Curing Setting to the Partial Information setting.
