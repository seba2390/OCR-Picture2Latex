\section{Conclusion}
We have shown that unless we know the state of each node with perfect accuracy, and instantaneously, then the {\sc CutWidth} of the graph is no longer the sole quantity which determines the budget required to cure an infection in polynomial time. Practically, this means that quickly obtaining signature-based diagnostic tools, even if expensive, is critical. On the theoretical side, our work shows that the interplay between stochastic processes and combinatorial properties of graphs needs to be better understood. Indeed, resolving the gap between our upper and lower bounds as a function of general topological graph quantities remains an important question. Similarly, extending our understanding of upper and lower bounds to other infection models is important. This work demonstrates the important connection between budget for control, and budget for estimation, as for many interesting problems, these two are inextricably intertwined. 