\section{ALC System and Adversarial Attacks}
\label{sec:sys_model}

The Automated Lane Centering (ALC) system, one of the Level 2 autonomous driving systems, are widely deployed in modern commercial vehicles.  %Generally, the autonomous driving or driving assistance system consist of sensing, perception, planning and control modules. 
In the ALC system, the perception module collects vision and distance input from cameras and radars, and outputs the perception of the environment to the planning and control module, which generates a desired trajectory and controls vehicle steering and acceleration.  
%The planning and control module will generate a desired trajectory by using the information of perception module, and then control both steering and acceleration. 
In the following, we will take the open-source software Openpilot (Fig.~\ref{fig:openpilot}) as an example to illustrate ALC's architecture.

%\begin{figure}[htbp]
%\centerline{\includegraphics[scale=0.25]{figures/system_model1.pdf}}
%\caption{Driving assistance system pipeline}
%\label{fig}
%\end{figure}
\begin{figure}[htbp]
\centerline{\includegraphics[width=\columnwidth]{figures/IV_system_pipeline_3.pdf}}
\caption{OpenPilot pipeline: the ALC system consists of a DNN-based perception module, a trajectory planning module, and an MPC controller. %The deep learning model will output predicted path, detected lane lines and corresponding confidence scores 
%More details are introduced in \cite{liang2021endtoend}.
%\takami{``Desire'' input is not used in v0.7 and v0.6. We do not need to mention it. ``fully connected layers'' might be good instead of hidden layers? Is there connection from Conv to hidden layers in OpenPilot Model?}
}
\vspace{-12pt}
\label{fig:openpilot}
\end{figure}

\subsection{DNN-based Perception Module}
Traditional perception modules often use edge detection or Hough transform to detect lane lines based on camera input~\cite{low2014simple}.
%after it reads the image from the front camera \cite{low2014simple}. 
Recently, DNN-based lane detection models achieve the state-of-the-art performance~\cite{wang2018lanenet} and are widely adopted in production-level ALC systems today such as Tesla AutoPilot~\cite{ingle2016tesla} and OpenPilot. The DNN-based perception module can detect lane lines and objects and provide necessary information for the planning and control module. 

In ALC tasks, since the lane lines are continuous across consecutive frames, recurrent neural network (RNN) is typically applied to utilize the time-series information to make the prediction more stable~\cite{zou2019robust}. 
%\junjie{do we have any citation for this argument?}.  
A convolutional neural network (CNN) processes the current frame and  feeds the combination of the CNN's and RNN's outputs into a fully-connect network. The concatenated output includes a predicted path (an estimation of the path to follow based on perception history), the detected right and left lane lines, and corresponding probabilities to indicate the detection confidence of each lane line. Note that the lane detection model first predicts lane points and then fits them into polynomial curves in post-processing for denoise and data compression.



\subsection{Planning and Control Modules}
%The planning modules, including mission planning and motion planning, are to generate the path and trajectory for autonomous vehicles. In current driving assistance systems, the planning module mainly focus on motion planning. Common motion planning frameworks in driving assistance system\cite{qian2016motion} can be divided into sampling based approaches and Model Predictive Control (MPC) based approaches. The sampling based methods sample directly the state space to obtain a set of feasible trajectories, and then select the best one, subject to some cost function\cite{ma2015efficient}. For instance, In Baidu Apollo version 2.5, a sample based method (the lattice algorithm) is applied to carry out path planning and speed planning for simple scenarios. Model predictive control (MPC) is an optimal control technique which deals with constraints on the states and the inputs \cite{bujarbaruah2018adaptive}.MPC-based approach permits high-precision planning and a certain degree of robustness. In commercial driving assistance systems like OpenPilot, they uses the MPC to follow the desired path and complete the motion planning. Specifically, in the planning and control module of OpenPilot, by calculating the weighted average of the detected left and right lane lines, the planner generates the desired path. Besides, the predicted path generated directly by the perception neural network will also be taken into consideration. 

The planning module generates a trajectory for the vehicle to follow. Sampling based methods, model based methods and deep learning models are often applied to the trajectory planning~\cite{qian2016motion}. For ALC, the desired path generated by the planner should be located in the middle between the left lane line and the right lane line. %and keep following this middle line. 
The planner module in OpenPilot generates an estimation of this desired path by calculating a weighted average of the detected left lane line, right lane line and predicted path, with the weights being the confidence scores outputted by the perception module. Intuitively, if the perception module is less confident on the predicted lanes, the generated desired path relies more on the predicted path; otherwise, it will be closer to the weighted average of the predicted left lane line and right lane line~\cite{liang2021endtoend,openpilot}. 

Given the desired path to follow, the lower-level controller calculates the vehicle maneuver and generates commands to control throttle, brake, and steering angle. In OpenPilot ALC, model predictive control (MPC)~\cite{bujarbaruah2018adaptive} is used to calculate the steering angle based on simplified system dynamics, vehicle heading constraints, and maximum steering angle constraints.
%under the constraints including \begin{enumerate*}
%    \item simplified system dynamics,
%    \item vehicle heading constraints, and
%    \item maximum steering angle constraints.
%\end{enumerate*}
MPC-based approach permits high-precision planning and a certain degree of robustness. For longitudinal control, the acceleration is generated by a PID controller by setting an appropriate reference speed.

\subsection{Attack Model %\junjie{``Attack Model'' or ``Threat Model'' are more common in Security.}
}
%\junjie{Better to have some justification, e.g., via reasoning or citation, on each of the assumptions.} 
In this paper, we assume a similar attack model as prior work~\cite{sato2020hold}. We focus on  attacks achieved through external physical world. In particular, we assume that the attacker cannot hack through software interfaces nor modify the victim vehicle. However, the attacker can deliberately change the physical environment that is perceived by the on-board sensors of the victim vehicle (e.g., cameras). Furthermore, we assume that the attacker is able to get knowledge of the victim's ADAS/autonomous driving system and can drive the same model vehicle to collect necessary data. This can be achieved by obtaining a victim vehicle model and conduct reverse engineering, as demonstrated in~\cite{tecent2019experimental,checkoway2011comprehensive}.  The attacker's goal is to design the appearance of certain object (in our case, a dirty patch) on the road such that
\begin{enumerate*}
    \item the adversarial object appears as normal/seemingly-benign for human drivers, and
    %hardly be noticed by human driver
    \item it can cause the ADAS/autonomous driving system to deviate from the driver's intended trajectory.
\end{enumerate*}
% In particular, we assume that attacker can not hack into the victim vehicle and should not be able to modify the software of the victim vehicle. However, the attacker has the ability to modify the physical environment that can be captured by the sensors equipped on the victim vehicle (for example, camera). Moreover, we assume that the attacker has knowledge of the victim's autonomous driving system and can drive vehicle of the same model and use the same autonomous driving system to collect data. The attacker's goal is to carefully change the appearance of certain object such that 
% \begin{enumerate*}
%     \item the object can hardly be noticed by human driver
%     \item and it can cause the autonomous driving system to drive off the road.
% \end{enumerate*}
Some example attack scenarios are discussed in~\cite{zhou2020deep,sato2020hold}. The work in~\cite{zhou2020deep}  generates an adversarial billboard to cause steering angle error. \cite{sato2020hold} generates a gray scale dirty patch on the road such that vehicles passing through the patch will deviate from its original lane, which is the first attack systematically designed for the ALC system and reaches state-of-art attack effect on production-level ADAS. Through our experiments, we will use the dirty road patch attack~\cite{sato2020hold} as a case study, but we believe that our approach can be extended to other similar physical environment attacks. As discussed in our preliminary work~\cite{liang2021endtoend}, these physical attacks typically will render abnormal behavior in the perception output and then propagate through the entire pipeline.
% if
% \begin{enumerate*}
%     \item the abnormal behavior of the perception module under attack can be identified and
%     \item the perception uncertainties can be estimated.
% \end{enumerate*}

%As an illustrating example, we use OpenPilot to drive on a straight road where a carefully designed dirty patch is placed. We observe that the confidence score of the perception module drops significantly when the vehicle is approaching the dirty patch, as indicated in figure~\ref{fig:prob_drop}. By looking more deep into the whole control pipeline, we find that the dirty patch actually deviates the predicted path towards left a lot and also degrades the confidence of the predicted lanes. Under the impact of the dirty patch, the path generator rends a final desired path that also lean towards left. Then in the lower-level controller will compute the corresponding steering angle based on this wrong desired path. 
