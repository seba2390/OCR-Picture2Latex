% \documentclass[conference]{IEEEtran}
\documentclass[letterpaper, 10 pt, conference]{ieeeconf}
\IEEEoverridecommandlockouts 
\overrideIEEEmargins 
% The preceding line is only needed to identify funding in the first footnote. If that is unneeded, please comment it out.
\usepackage{cite}
\usepackage{amsmath,amssymb,amsfonts}
\usepackage{algorithmic}
%\usepackage[linesnumbered]{algorithm2e}
\usepackage[algo2e]{algorithm2e} 
\usepackage{graphicx}
\usepackage{textcomp}
\let\labelindent\relax
\usepackage[inline]{enumitem}
\usepackage{xcolor}
\usepackage{caption}
\usepackage{graphicx}
\usepackage{algorithm}  

\usepackage{diagbox,slashbox}
\usepackage{subcaption}
\usepackage{hyperref}
\usepackage{verbatim}

\def\BibTeX{{\rm B\kern-.05em{\sc i\kern-.025em b}\kern-.08em
    T\kern-.1667em\lower.7ex\hbox{E}\kern-.125emX}}
    
% Space saving List environment for itemizing
\newenvironment{myitemize}{\begin{list}{$\bullet$}
{\setlength{\topsep}{1mm}
\setlength{\itemsep}{0.25mm}
\setlength{\parsep}{0.25mm}
\setlength{\itemindent}{0mm}
\setlength{\partopsep}{0mm}
\setlength{\labelwidth}{15mm}
\setlength{\leftmargin}{4mm}}}{\end{list}}

    
\newcommand{\hengyi}[1]{{{\color{blue} \textbf{(Hengyi: #1)}}}}
\newcommand{\ruochen}[1]{{{\color{blue} \textbf{(Ruochen: #1)}}}}
\newcommand{\qi}[1]{{{\color{blue} \textbf{(Qi: #1)}}}}
\newcommand{\alfred}[1]{{{\color{blue} \textbf{(Alfred: #1)}}}}
\newcommand{\junjie}[1]{{{\color{blue} \textbf{(Junjie: #1)}}}}
\newcommand{\takami}[1]{{{\color{blue} \textbf{(takami: #1)}}}}

\begin{document}

\title{\LARGE \bf End-to-end Uncertainty-based Mitigation of Adversarial Attacks to Automated Lane Centering\\
%{\footnotesize \textsuperscript{*}Note: Sub-titles are not captured in %Xplore and
%should not be used}
}

% \author{\IEEEauthorblockN{1\textsuperscript{st} Given Name Surname}
% \IEEEauthorblockA{\textit{dept. name of organization (of Aff.)} \\
% \textit{name of organization (of Aff.)}\\
% City, Country \\
% email address}
% \and
% \IEEEauthorblockN{2\textsuperscript{nd} Given Name Surname}
% \IEEEauthorblockA{\textit{dept. name of organization (of Aff.)} \\
% \textit{name of organization (of Aff.)}\\
% City, Country \\
% email address}
% }
% \author{
% \IEEEauthorblockN{Ruochen Jiao\IEEEauthorrefmark{1}\thanks{\IEEEauthorrefmark{1} These two authors contribute equally to the work.}, Hengyi Liang\IEEEauthorrefmark{1}, Takami Sato\IEEEauthorrefmark{2}, Junjie Shen\IEEEauthorrefmark{2}, Qi Alfred Chen\IEEEauthorrefmark{2}, Qi Zhu}
% \IEEEauthorblockA{Northwestern University \qquad \IEEEauthorrefmark{2} University of California, Irvine}
% }

\author{Ruochen Jiao$^{1,2}$, Hengyi Liang$^{1,2}$, Takami Sato$^{3}$, Junjie Shen$^{3}$, Qi Alfred Chen$^{3}$, Qi Zhu$^{2}$%
\thanks{$^{1}$ These two authors contribute equally to the work.}%
\thanks{$^{2}$ Ruochen Jiao, Hengyi Liang and Qi Zhu are with the Department of Electrical and Computer Engineering, Northwestern University, IL, USA.}%
\thanks{$^{3}$ Takami  Sato, Junjie Shen and Qi Alfred Chen are with the Department of Electrical Engineering and Computer Science, University of California, Irvine, CA, USA.}%
}

%\author{Xiangguo Liu$^{1}$, Neda Masoud$^{2}$, Qi Zhu$^{1}$% <-this % stops a space
%\thanks{*This work was not supported by any organization}% <-this % stops a space
%\thanks{$^{1}$Xiangguo Liu and Qi Zhu are with the Department of Electrical and Computer Engineering, Northwestern University, Evanston, IL 60201, USA.
%        {\tt\small xg.liu@u.northwestern.edu, qzhu@northwestern.edu.}}%
%\thanks{$^{2}$Neda Masoud is with the Department of Civil and Environmental Engineering, University of Michigan, Ann Arbor, MI 48109, USA.
%        {\tt\small nmasoud@umich.edu.}}%
%}

%author{Ruochen Jiao\IEEEauthorrefmark{1}\thanks{\IEEEauthorrefmark{1} These two authors contribute equally to the work.}, Hengyi Liang\IEEEauthorrefmark{1}, Takami Sato\IEEEauthorrefmark{2}, Junjie Shen\IEEEauthorrefmark{2}, Qi Alfred Chen\IEEEauthorrefmark{2}, Qi Zhu}
\maketitle

\begin{abstract}
In the development of advanced driver-assistance systems (ADAS) and autonomous vehicles, machine learning techniques that are based on deep neural networks (DNNs) have been widely used for vehicle perception. These techniques offer significant improvement on average perception accuracy over traditional methods, however have been shown to be susceptible to adversarial attacks, where small perturbations in the input may cause significant errors in the perception results and lead to system failure. Most prior works addressing such adversarial attacks focus only on the sensing and perception modules. In this work, we propose an end-to-end approach that addresses the impact of adversarial attacks throughout perception, planning, and control modules. In particular, we choose a target ADAS application, the automated lane centering system in OpenPilot, quantify the perception uncertainty under adversarial attacks, and design a robust planning and control module accordingly based on the uncertainty analysis.  
%However, these techniques often lack of the consideration of potential uncertainties in the environment and can be leveraged malicious attackers. In this work, we conduct an end-to-end analysis of a target ADAS application, automated lane centering, to shed light on how an adversarial attack can affect the result of perception module, propagate through the pipeline and finally cause the steering angle pointing to a wrong direction. We propose to design an end-to-end mitigation strategy, i.e. explicitly considering uncertainties in perception, planning and control modules, to alleviate the impact of potential adversarial attacks. 
We evaluate our proposed approach using both public dataset and production-grade autonomous driving simulator. The experiment results demonstrate that our approach can effectively mitigate the impact of adversarial attack and 
can achieve $55\%\sim90\%$ improvement over the original OpenPilot.
\end{abstract}

%\begin{IEEEkeywords}
%Autonomous driving, ADAS, adversarial attacks
%\end{IEEEkeywords}

\section{Introduction}
Machine learning techniques have been widely adopted in the development of autonomous vehicles and advanced driver-assistance systems (ADAS). 
%Last decades witnessed great progress in autonomous vehicle and automotive driving assistance systems (ADAS) with the development of deep learning. 
% The  autonomous vehicle (e.g. Baidu Apollo, Autoware etc. \cite{raju2019performance}) and ADAS (e.g. Tesla AutoPilot, OpenPilot \cite{openpilot} etc.) 
Most autonomous driving and ADAS software stacks, such as Baidu Apollo~\cite{baiduapollo} and OpenPilot~\cite{openpilot}, are generally composed of four-layered modules: sensing, perception, planning, and control~\cite{yurtsever2020survey}. The sensing and perception modules collect data from the surrounding environment via a variety of sensors such as cameras, LiDAR, radar, GPS and IMU, and use learning-based perception algorithms to process the collected data and understand the environment. The planning and control modules leverage the perception results to propose a feasible trajectory and generate detailed commands for the vehicle to track the trajectory. In those systems, deep neural networks (DNNs) are widely used for sensing and perception in transportation scenarios~\cite{gao2018object,xu2020data,siam2017deep}
%chen2017multi,buyval2018realtime,zhou2018voxelnet}
, such as semantic segmentation, object detection and tracking, as they often provide significantly better average perception accuracy over traditional feature-based methods. For planning and control, there are also increasing interests in applying neural networks with techniques such as reinforcement learning and imitation learning~\cite{rhinehart2018deep}, due to their capabilities of automatically learning a strategy within complex environment. 
%In sensing stage, cameras, LiDAR, GPS and IMU are widely applied to collect data from surrounding environment. 
%and then multi sensor fusion technologies can synthesize and calibrate the data produced by these sensors\cite{gao2018object}. 
%For perception and prediction, the deep learning based models\cite{gao2018object,xu2020data} process these inputs to give predictions and they achieved impressive performance. The planner will propose feasible trajectory according to the perceived environment and low-level controller will produce detailed command to track the trajectory. 

However, the adoption of DNN-based techniques in ADAS and autonomous driving also bring significant challenges to vehicle safety and security, given the ubiquitous uncertainties of the dynamic environment, the disturbances from environment interference, transient faults, and malicious attacks, and the lack of methodologies for predicting DNN behavior~\cite{zhu2020know}. In particular, extensive studies have shown that DNN-based perception tasks, such as image classification and object detection, may be susceptible to adversarial attacks~\cite{eykholt2018robust,chen2018shapeshifter}, where small perturbations to sensing input could result in drastically different perception results. There are also recent works on attacking DNN-based perception in ADAS and autonomous vehicles by adding adversarial perturbations to the physical environment in a stealthy way~\cite{zhou2020deep,sato2020hold,song2018physical,eykholt2018robust,sitawarin2018darts}. For instance, \cite{sato2020hold} generates a dirty road patch with carefully-designed adversarial patterns, which can appear as normal dirty patterns for human drivers while leading to significant perception errors and causing vehicles to deviate from their lanes within as short as 1 second. %\alfred{modified a bit as ``hardly noticeable by human'' is not true}
% sitawarin2018darts,boloor2019simple
%Although cyber-physical systems benefit greatly from deep learning components, the learning-enabled systems are also more vulnerable to new types of interference and attacks. The adversarial attacks on tasks such as image classification, object detection has been well studied \cite{eykholt2018robust,chen2018shapeshifter}. In recent years, there are some works focusing on the security of autonomous driving systems. In \cite{sato2020hold}, the authors propose an approach to generate a patch to attack deep learning based lane keeping systems. In \cite{cai2020real}, they focus on detecting out-of-distribution input and evaluate their impact on the system.  


%However, the safety and security of these learning-based ADAS and autonomous systems can hardly be guaranteed or even quantitatively analyzed, given the ubiquitous uncertainties in surrounding environment, complexity of multi-module systems, and the unpredictability of the DNN behaviors\cite{zhu2020know}. Some works propose methods to detect out-of-distribution inputs\cite{cai2020real} and design probabilistic deep learning based perception models\cite{sun2018probabilistic} and planning models\cite{hruschka2019uncertainty}, while few works focus on analyzing and mitigating the affect of attacks or noises in a system level.

The prior works addressing adversarial attacks mostly focus on detecting anomaly in the input  data~\cite{kimin2018simple,zheng2018robust,lu2017safetynet,yin2020adversarial}
%,yin2020adversarial,metzen2017detecting,} 
or making the perception neural networks themselves more robust against input perturbations~\cite{goodfellow2015explaining,madry2019deep,carlini2017towards}. In ADAS and autonomous driving, however, the impact of adversarial attacks on system safety and performance is eventually reflected through vehicle movement, taking into account of planning and control decisions. Thus, we believe that for those systems, it is important to take a holistic and end-to-end approach that addresses adversarial attacks throughout the sensing, perception, planning and control pipeline.
%papernot2016distillation,miyato2016distributional

In our preliminary work recently published in a work-in-progress paper~\cite{liang2021endtoend}, we studied the automated lane centring (ALC) system in OpenPilot~\cite{openpilot}, a popular open-source ADAS implementation, and investigated how the dirty road patch attack from~\cite{sato2020hold} could affect perception, planning and control modules. We discovered that a \emph{confidence score} generated by the perception module could serve as a sensitive signal for detecting such attack, however it does not quantitatively measure the extent of the attack and cannot be effectively used for mitigation. In this paper, motivated by the findings from~\cite{liang2021endtoend}, we propose a novel end-to-end approach for \emph{detecting and mitigating} the adversarial attacks on the ALC system. Our approach quantitatively estimates the \emph{uncertainty} of the perception results, and develops an adaptive planning and control method based on the uncertainty analysis to improve system safety and robustness.  
%In our previous related work~\cite{liang2021endtoend}, to study the system robustness and safety, we first analyze the pipeline of commercial automated lane centering systems and identify how different modules interact with each other under attacks. 
%In this paper, we propose a system level design and framework with uncertainty estimation, risk awareness, and adaptive planner and controller. Through the analyses and observations, we utilize the perception confidence as a signal to indicate risks and a threshold to adjust modes. In addition to qualitative analysis of risks, our work focus on explicitly quantify the uncertainty and the conduct adaptive planning and control to improve the system's safety and robustness.     

In the literature, methods have been proposed to address the uncertainties of various modules in the ADAS and autonomous driving pipeline. For instance, the method proposed in~\cite{nakashima2020uncertainty} utilizes estimated uncertainty as a threshold to decide which sensor is reliable. %The work in~\cite{hu2018probabilistic} uses mixture density network to estimate the intention, location and time information of surrounding vehicles. 
In the OpenPilot implementation, Multiple Hypothesis Prediction~\cite{rupprecht2017learning} is utilized to estimate the prediction confidence. Some works propose methods to detect out-of-distribution inputs~\cite{cai2020real} and design probabilistic deep learning based perception models~\cite{sun2018probabilistic} and planning models~\cite{hruschka2019uncertainty}. Different from these prior methods, our approach takes a system-level view and addresses the uncertainty from adversarial attacks throughout sensing, perception, planning and control. While this work focuses on the dirty road patch attack, we believe that our methodology can be applied to other adversarial attacks that cause perception uncertainties, and may be extended to address more general uncertainties (e.g., those caused by environment interference or transient faults).  
Specifically, our work makes the following contributions:
\begin{myitemize}
    \item We analyzed the impact of dirty road patch attack across the ADAS pipeline, and developed a method to quantitatively measure the perception uncertainty under attack, based on the analysis of both model and data uncertainties in the perception neural network. 
    \item We developed an uncertainty-aware adaptive planning and control method to improve system safety and robustness under adversarial attacks. 
    \item We conducted experiments on both public dataset and LGSVL~\cite{rong2020lgsvl}, a production-grade autonomous driving simulator. The results demonstrate that our approach can significantly improve the system robustness over the original OpenPilot implementation when under adversarial attacks, reducing the deviation of lateral deviation by $55\%\sim90\%$.
\end{myitemize}
%1. We analyze the safety of deep learning based automated lane centering systems and identify proper interfaces between modules that can be used to improve the system robustness.
%2. We proposes and implement a holistic frameworks to estimate the uncertainty from deep learning based perception modules and design robust planner and controller based on the uncertainty bound and perception confidence.

%3. We experimentally show that our approach outperform (more robust than) current commercial driving assistance systems against malicious physical attacks.




%Last decades witnessed great progress in autonomous vehicle and automotive driving assistance systems (ADAS) with the development of deep learning. The autonomous vehicle (e.g. Baidu Apollo, Autoware etc.) and ADAS (e.g. Tesla AutoPilot, OpenPilot etc.) are generally composed of four layered modules: sensing, perception, planning and control \cite{yurtsever2020survey}. Sensing modules detect physical presences and convert the physical signals into data which can be processed by the perception module. The camera, Lidar, Radar, and GPS are commonly deployed in driving assistance systems \cite{badue2020self}.

The rest of the paper is organized as follows. Section~\ref{sec:sys_model} introduces the ALC system in OpenPilot and the adversarial attack model to this system. Section~\ref{sec:our_approach} presents our uncertainty-based mitigation approach to address such adversarial attacks. Section~\ref{sec:experiments} shows the experimental results.



\begin{comment}
\section{Background}
\subsection{Deep Learning based Driving Assistance Systems}
 Recently, researchers applied various algorithms, especially deep learning algorithms, to transportation scenarios. In \cite{siam2017deep,chen2017multi,buyval2018realtime}, semantic segmentation, object detection and tracking are used to understand the environment in roads. In \cite{zhou2018voxelnet}, Lidar-based 3D object detection algorithms achieve a promising performance to detect pedestrians and vehicles. For planning and control modules, reinforcement learning and imitation learning also show considerable performance in path planning and control \cite{rhinehart2018deep}. However, few researches illustrate how deep learning based modules interact with other parts and influence the whole system, which is crucial for the system's safety. In this paper, we conduct end-to-end safety analysis and propose a holistic robust mitigation strategy with general significance.  

\subsection{Adversarial Attack Against Driving Assistance Systems}
Although cyber-physical systems benefit greatly from deep learning components, the learning-enabled systems are also more vulnerable to new types of interference and attacks. The adversarial attacks on tasks such as image classification, object detection has been well studied \cite{eykholt2018robust,chen2018shapeshifter}. In recent years, there are some works focusing on the security of autonomous driving systems. In \cite{sato2020hold}, the authors propose an approach to generate a patch to attack deep learning based lane keeping systems. In \cite{cai2020real}, they focus on detecting out-of-distribution input and evaluate their impact on the system.   

\subsection{Uncertainty Aware Autonomous Systems}
A robust design for autonomous systems must consider various uncertainty such as the inherent uncertainties from the dynamic environment, the disturbances to system operations from environment interference, transient errors, and malicious attacks ~\cite{zhu2021safe}. In sensing and perception stage, the work\cite{nakashima2020uncertainty} propose a method to utilize estimated uncertainty as a threshold to decide which sensor is reliable. In prediction and planning stage, the work\cite{hu2018probabilistic} proposed a method using mixture density network to estimate the intentions, location and time information for surrounding vehicles. In original OpenPilot, Multiple Hypothesis Prediction\cite{rupprecht2017learning} is utilized to estimate the confidence of prediction. There are also works design stochastic motion controller considering uncertainty\cite{suh2018stochastic}. In our work, we combine the different methods to estimate confidence and quantify uncertainty, and then utilize the  uncertainty to design adaptive planner and controller in a cross-layer way. 
\end{comment}


\section{ALC System and Adversarial Attacks}
\label{sec:sys_model}

The Automated Lane Centering (ALC) system, one of the Level 2 autonomous driving systems, are widely deployed in modern commercial vehicles.  %Generally, the autonomous driving or driving assistance system consist of sensing, perception, planning and control modules. 
In the ALC system, the perception module collects vision and distance input from cameras and radars, and outputs the perception of the environment to the planning and control module, which generates a desired trajectory and controls vehicle steering and acceleration.  
%The planning and control module will generate a desired trajectory by using the information of perception module, and then control both steering and acceleration. 
In the following, we will take the open-source software Openpilot (Fig.~\ref{fig:openpilot}) as an example to illustrate ALC's architecture.

%\begin{figure}[htbp]
%\centerline{\includegraphics[scale=0.25]{figures/system_model1.pdf}}
%\caption{Driving assistance system pipeline}
%\label{fig}
%\end{figure}
\begin{figure}[htbp]
\centerline{\includegraphics[width=\columnwidth]{figures/IV_system_pipeline_3.pdf}}
\caption{OpenPilot pipeline: the ALC system consists of a DNN-based perception module, a trajectory planning module, and an MPC controller. %The deep learning model will output predicted path, detected lane lines and corresponding confidence scores 
%More details are introduced in \cite{liang2021endtoend}.
%\takami{``Desire'' input is not used in v0.7 and v0.6. We do not need to mention it. ``fully connected layers'' might be good instead of hidden layers? Is there connection from Conv to hidden layers in OpenPilot Model?}
}
\vspace{-12pt}
\label{fig:openpilot}
\end{figure}

\subsection{DNN-based Perception Module}
Traditional perception modules often use edge detection or Hough transform to detect lane lines based on camera input~\cite{low2014simple}.
%after it reads the image from the front camera \cite{low2014simple}. 
Recently, DNN-based lane detection models achieve the state-of-the-art performance~\cite{wang2018lanenet} and are widely adopted in production-level ALC systems today such as Tesla AutoPilot~\cite{ingle2016tesla} and OpenPilot. The DNN-based perception module can detect lane lines and objects and provide necessary information for the planning and control module. 

In ALC tasks, since the lane lines are continuous across consecutive frames, recurrent neural network (RNN) is typically applied to utilize the time-series information to make the prediction more stable~\cite{zou2019robust}. 
%\junjie{do we have any citation for this argument?}.  
A convolutional neural network (CNN) processes the current frame and  feeds the combination of the CNN's and RNN's outputs into a fully-connect network. The concatenated output includes a predicted path (an estimation of the path to follow based on perception history), the detected right and left lane lines, and corresponding probabilities to indicate the detection confidence of each lane line. Note that the lane detection model first predicts lane points and then fits them into polynomial curves in post-processing for denoise and data compression.



\subsection{Planning and Control Modules}
%The planning modules, including mission planning and motion planning, are to generate the path and trajectory for autonomous vehicles. In current driving assistance systems, the planning module mainly focus on motion planning. Common motion planning frameworks in driving assistance system\cite{qian2016motion} can be divided into sampling based approaches and Model Predictive Control (MPC) based approaches. The sampling based methods sample directly the state space to obtain a set of feasible trajectories, and then select the best one, subject to some cost function\cite{ma2015efficient}. For instance, In Baidu Apollo version 2.5, a sample based method (the lattice algorithm) is applied to carry out path planning and speed planning for simple scenarios. Model predictive control (MPC) is an optimal control technique which deals with constraints on the states and the inputs \cite{bujarbaruah2018adaptive}.MPC-based approach permits high-precision planning and a certain degree of robustness. In commercial driving assistance systems like OpenPilot, they uses the MPC to follow the desired path and complete the motion planning. Specifically, in the planning and control module of OpenPilot, by calculating the weighted average of the detected left and right lane lines, the planner generates the desired path. Besides, the predicted path generated directly by the perception neural network will also be taken into consideration. 

The planning module generates a trajectory for the vehicle to follow. Sampling based methods, model based methods and deep learning models are often applied to the trajectory planning~\cite{qian2016motion}. For ALC, the desired path generated by the planner should be located in the middle between the left lane line and the right lane line. %and keep following this middle line. 
The planner module in OpenPilot generates an estimation of this desired path by calculating a weighted average of the detected left lane line, right lane line and predicted path, with the weights being the confidence scores outputted by the perception module. Intuitively, if the perception module is less confident on the predicted lanes, the generated desired path relies more on the predicted path; otherwise, it will be closer to the weighted average of the predicted left lane line and right lane line~\cite{liang2021endtoend,openpilot}. 

Given the desired path to follow, the lower-level controller calculates the vehicle maneuver and generates commands to control throttle, brake, and steering angle. In OpenPilot ALC, model predictive control (MPC)~\cite{bujarbaruah2018adaptive} is used to calculate the steering angle based on simplified system dynamics, vehicle heading constraints, and maximum steering angle constraints.
%under the constraints including \begin{enumerate*}
%    \item simplified system dynamics,
%    \item vehicle heading constraints, and
%    \item maximum steering angle constraints.
%\end{enumerate*}
MPC-based approach permits high-precision planning and a certain degree of robustness. For longitudinal control, the acceleration is generated by a PID controller by setting an appropriate reference speed.

\subsection{Attack Model %\junjie{``Attack Model'' or ``Threat Model'' are more common in Security.}
}
%\junjie{Better to have some justification, e.g., via reasoning or citation, on each of the assumptions.} 
In this paper, we assume a similar attack model as prior work~\cite{sato2020hold}. We focus on  attacks achieved through external physical world. In particular, we assume that the attacker cannot hack through software interfaces nor modify the victim vehicle. However, the attacker can deliberately change the physical environment that is perceived by the on-board sensors of the victim vehicle (e.g., cameras). Furthermore, we assume that the attacker is able to get knowledge of the victim's ADAS/autonomous driving system and can drive the same model vehicle to collect necessary data. This can be achieved by obtaining a victim vehicle model and conduct reverse engineering, as demonstrated in~\cite{tecent2019experimental,checkoway2011comprehensive}.  The attacker's goal is to design the appearance of certain object (in our case, a dirty patch) on the road such that
\begin{enumerate*}
    \item the adversarial object appears as normal/seemingly-benign for human drivers, and
    %hardly be noticed by human driver
    \item it can cause the ADAS/autonomous driving system to deviate from the driver's intended trajectory.
\end{enumerate*}
% In particular, we assume that attacker can not hack into the victim vehicle and should not be able to modify the software of the victim vehicle. However, the attacker has the ability to modify the physical environment that can be captured by the sensors equipped on the victim vehicle (for example, camera). Moreover, we assume that the attacker has knowledge of the victim's autonomous driving system and can drive vehicle of the same model and use the same autonomous driving system to collect data. The attacker's goal is to carefully change the appearance of certain object such that 
% \begin{enumerate*}
%     \item the object can hardly be noticed by human driver
%     \item and it can cause the autonomous driving system to drive off the road.
% \end{enumerate*}
Some example attack scenarios are discussed in~\cite{zhou2020deep,sato2020hold}. The work in~\cite{zhou2020deep}  generates an adversarial billboard to cause steering angle error. \cite{sato2020hold} generates a gray scale dirty patch on the road such that vehicles passing through the patch will deviate from its original lane, which is the first attack systematically designed for the ALC system and reaches state-of-art attack effect on production-level ADAS. Through our experiments, we will use the dirty road patch attack~\cite{sato2020hold} as a case study, but we believe that our approach can be extended to other similar physical environment attacks. As discussed in our preliminary work~\cite{liang2021endtoend}, these physical attacks typically will render abnormal behavior in the perception output and then propagate through the entire pipeline.
% if
% \begin{enumerate*}
%     \item the abnormal behavior of the perception module under attack can be identified and
%     \item the perception uncertainties can be estimated.
% \end{enumerate*}

%As an illustrating example, we use OpenPilot to drive on a straight road where a carefully designed dirty patch is placed. We observe that the confidence score of the perception module drops significantly when the vehicle is approaching the dirty patch, as indicated in figure~\ref{fig:prob_drop}. By looking more deep into the whole control pipeline, we find that the dirty patch actually deviates the predicted path towards left a lot and also degrades the confidence of the predicted lanes. Under the impact of the dirty patch, the path generator rends a final desired path that also lean towards left. Then in the lower-level controller will compute the corresponding steering angle based on this wrong desired path. 

\section{Our Approach}\label{sec:our_approach}
%[maybe prefacing line about IL] towards our IL approach
We begin with the reverse KL as our divergence of choice, since, noting as others have \cite{kostrikov2019imitation,hazan2019provably,kim2021imitation,camacho2021sparsedice}, its minimization may be viewed as an RL problem with rewards being the log distribution ratios:
\begin{align}
    \argmin_{\pi}\,D_{KL}(p_\pi||p_e) =& \argmax_{\pi}\,\mathbb{E}_{p_\pi(s,a)}\left[\log \frac{p_e(s,a)}{p_\pi(s,a)} \right] \nonumber \\=& 
    \argmax_{\pi}J(\pi, r{=}\log \frac{p_e}{p_\pi}).\label{eq:rKL}
\end{align}
Thus permitting the use of any RL algorithm for solving the IL objective, provided one has an appropriate estimate of the ratio.

%briefly
Before continuing however, we first motivate our coupled approach for such estimation, by illustrating the failure of what is perhaps a more natural next step: using two independent density estimators—say flows—for each of the densities $p_e$ and $p_\pi$ directly. Practically, this would mean alternating between learning the flows %(like Equation \ref{eq:regularization} below)
and using their log-ratio as reward in an RL algorithm. The table in Section \ref{section:ablation} concisely showcases a clear failure of this approach on all the standard benchmarks we later evaluate with.


\begin{figure}[t]
\vskip 0.1in
\centering
\includegraphics[width=0.95\columnwidth]{figures/BC_analysis_bottom_cropped.png} % Reduce the figure size so that it is slightly narrower than the column. Don't use precise values for figure width.This setup will avoid overfull boxes.
\caption{Left: The BC graph of an uncoupled flow for the HalfCheetah-v2 environment. Right: The BC graph of a coupled flow for the HalfCheetah-v2 environment. BC graphs for an estimator are generated by updating the estimator analogously to an RL run using $N$ saved BC rollouts. This yields $N$ estimators corresponding to $N$ intermediate BC agents. The BC graph is then, for all $i$, the scatter of the $i$’th estimator’s evaluation of an $i$’th BC agent's trajectory against its true environment reward.}
\label{fig:ind_flows_failure_and_bc}
\vskip -0.1in
\end{figure} 



The failure can be further understood through analyzing what we term the BC graph corresponding to an estimator of the log distribution ratio. That is, we first train a behavioral cloning agent once on sufficient expert trajectories, while rolling it out every few iterations. We then train the estimator analogously to an RL run, using a single expert trajectory along with the $N$ saved BC rollouts. This quick process yields $N$ estimators corresponding to $N$ intermediate BC agents. The BC graph is then, for all $i$, the $i$’th estimator’s evaluation of an $i$’th BC agent's trajectory, scattered against the true environment reward of that same trajectory. \footnote{Meaningfulness of the BC graph depends on how erratic the training was and how iteration number, loss and true environment reward line up during training. Our choice of Cheetah for illustration is motivated due to all these measures mostly aligning.} 
Intuitively, a non increasing graph, means a potential RL learner with an analogously trained reward may struggle to overcome those misleading behaviors ranked high by the synthetic reward, thus preventing them from reaching expert level. In practice of course, one would want some form of approximate monotonicity or upward trend. Though importantly, a BC graph's monotonicity by no means implies the success of an RL learner with the correspondingly constructed reward. This is more than a theoretical idiosyncrasy: Many estimators will emit a perfect BC graph while completely failing all attempts in RL (see Appendix \ref{section:appendix_regular_net_bc}). Only the reverse is true: its complete lack of an upward trend will usually imply an agent's failure. 


Loosely formalized, given BC's objective in Equation \ref{eq:bc_objective} and assuming a stationary (time independent) reward function $r$, a monotonically increasing BC graph essentially\footnote{Once again assumes alignment of environment reward with loss.} means that for all policies $ \pi_1, \pi_2$: $ J(\pi_1,r) > J(\pi_2,r)$ if $ \mathbb{E}_{p_e}[\log \pi_1(a|s)]  > \mathbb{E}_{p_e}[\log \pi_2(a|s)] $. Thus, further assuming continuity, a non monotonically increasing graph implies it either monotonically decreases or the existence of a local maximum. In both cases an RL learner with objective $\argmax_\pi{J(\pi,r)}$ may converge on a policy with suboptimal BC loss. Since only at optimality BC recovers the expert policy, this would guarantee the agent will not meet its truly intended goal of expert mimicry. 
Of course, in reality, RL can overcome a certain lack of an upward trend. Moreover, the rewards are neither stationary nor identical between the BC and RL runs, only analogously constructed, so such graphs are only loosely representative. Nonetheless, we find they can be highly insightful.





As Figure \ref{fig:ind_flows_failure_and_bc} suggests, the independently trained flows' BC graph is quite lacking. An agent would have no incentive according to the synthetic reward to make any progress, which is precisely what occurs as the table in Section \ref{section:ablation} demonstrates. This poor BC graph is due in part to each flow being evaluated on data completely out of its distribution (OOD), which flows are known to struggle with \cite{kirichenko2020normalizing}. Since the two flows estimates lack true meaning when evaluated on each others data, we need to tie them together somehow: They must be \textit{coupled}.

To perform our coupling, we employ the Donsker-Varadhan \cite{donsker1976asymptotic} form of the KL divergence:
%wanted to align but too big... how can it be done?
\begin{align}
D_{KL}&(p_\pi||p_{e}) =
\nonumber \\
&\sup_{x: S \times A \to \mathbb{R}} \mathbb{E}_{p_\pi(s,a)}\left[x(s,a)\right] - \log \mathbb{E}_{p_e(s,a)}\left[e^{x(s,a)} \right].\label{eq:Donsker}
\end{align}
 %note the "&" in the KL helps properly align equations
In the above, optimality occurs with  $x^*= \log \frac{p_\pi}{p_e} + C$ for any $C\in \mathbb{R}$ \cite{kostrikov2019imitation, gangwani2020harnessing,belghazi2018mine}. Thus after computing $x^*$, one recovers the log distribution ratio by simple negation, enabling use of $-x^*$ as reward in an RL algorithm to optimize our IL objective. This leads directly to our proposed approach for estimating the log distribution ratio, by coupling two flows through $x(s,a)$. That is, instead of training two flows independently, we propose to do so through maximization of Equation \ref{eq:Donsker}. More specifically, we inject the following inductive bias, modeling $x$ as $x_{\psi,\phi}(s,a) = \log p_\psi(s,a) - \log q_\phi(s,a),$ where  $p_\psi$ and $q_\phi$ are normalizing flows. %[so by maximizing \ref{eq:Donsker} one recovers x...] 


This coupling guarantees more meaningful values when the flows are evaluated on each others data, since it has already occurred during the maximization phase, hence sidestepping the OOD issue described earlier. Figure \ref{fig:ind_flows_failure_and_bc} illustrates this advantage. The right shows the coupled flows' BC graph, clearly remedying the issue with their uncoupled counterparts: A potential learner will now have the proper incentive to reproduce the expert's behavior. 

The drop in synthetic reward (i.e. $-x^*$) towards the end of the BC graph may seem daunting, but it actually expresses how well our estimator captures the expert's behavior: The drop occurs precisely beyond expert level, where the agent, while good in the environment, diverges from the true expert's behavior.\footnote{ This both illustrates the tendency of BC to overfit \cite{li2022rethinking}, while also raising concerns about the tendency for IL papers to report a table showing slight advantage in asymptotic reward as being meaningful. In truth, once at expert level, an advantage should not be claimed for slightly higher performance, unless the stated goal of the work is to do so, but that would no longer be imitation.}



Given this improved estimator, our full IL objective can then be written as:
\begin{equation}\label{eq:maxmin} 
    \argmax_{\pi} \min_{p_\psi, q_\phi}
    \log \mathbb{E}_{p_e(s,a)}\left[e^{\log \frac{p_\psi}{q_\phi}} \right] - \mathbb{E}_{p_\pi(s,a)}\left[\log \frac{p_\psi}{q_\phi}\right].
\end{equation}
As is commonplace in adversarial like approaches \cite{goodfellow2014generative, ho2016generative,kostrikov2019imitation}, the max-min objective above is trained in an alternating fashion, switching between learning $x$ and using $-x$ as reward in an RL algorithm. Moreover, we find it useful to use a squashing function on $x$, avoiding a potentially exploding KL, due to a lack of common support (subtly different then the earlier issue of OOD). 

%maybe mention the shift as well: thanks to RL's invariance to constant reward shift $\argmax_\pi J(\pi, r) = \argmax_\pi J(\pi, r+C)$

Our approach still enables training each flow independently along the way. We call this flow regularization and importantly, our method succeeds without such regularization. More specifically, for expert and agent batches of size $M$, this
regularization involves incorporating the following additional loss function into the minimization step of Equation \ref{eq:maxmin}: 
\begin{equation}\label{eq:regularization}
\mathcal{L} = - \frac{1}{M} \sum_{i=1}^M \log  q_\phi(s_e^i,a_e^i) + \log  p_\psi(s^i,a^i),
\end{equation}
with $\mathcal{L}$ to be weighted by a coefficient $\alpha$.

Noting that training flows is a delicate process, our approach further benefits from—though again does not require—use of a smoothing akin to the dequantization used when training normalizing flows on discrete data \cite{papamakarios2017masked}. More specifically, since our input is unnormalized, we smooth each dimension with uniform noise scaled to its value. That is, if $(s,a)$ is the vector to be smoothed, we sample uniform noise with dimension $dim((s,a))$, multiply them element-wise and add that to the original vector:
\begin{equation}
    {(s,a) \mathrel{{+}{=}} \beta \cdot (s,a) \odot u, \hspace{0.5em} u \sim Uniform(-\frac{1}{2},\frac{1}{2})^{dim((s,a))}},
\end{equation}
with weight $\beta$ controlling the smoothing level. Note if regularization is also present, smoothing still applies within the additional loss $\mathcal{L}$.


Finally, combining  all the above is our resulting algorithm, Coupled Flow Imitation Learning (CFIL). It is summarized in Algorithm \ref{alg:CFIL}, with only the number of batches per density update omitted. 
%algorithm notes:
%
\begin{algorithm}[tb]
\caption{CFIL}
\label{alg:CFIL}
\textbf{Input}: Expert demos $\mathcal{R}_E = \{(s_e,a_e)\}_{t=1}^N$; parameterized flow pair $p_\psi,q_\phi$; off-policy RL algorithm $\mathcal{A}$; density update rate $k$; squashing function $\sigma$; regularization and smoothing coefficients $\alpha, \beta$.\\
\textbf{Define}: $x_{\psi,\phi} = \sigma(\log p_\psi - \log q_\phi)$
\begin{algorithmic}[1] %[1] enables line numbers
\FOR{timestep $t=0,1,\dots,$}
\STATE Take a step in $\mathcal{A}$ with reward $r=-x_{\psi,\phi}$, while filling agent buffer $\mathcal{R}_A$ and potentially updating the policy and value networks according to $\mathcal{A}$'s settings.
\IF {$t\mod k =0$}
\STATE Sample expert and agent batches:
\STATE $\{(s_e^t,a_e^t)\}_{t=1}^M \sim \mathcal{R}_E$ and  $\{(s^t,a^t)\}_{t=1}^M \sim \mathcal{R}_A$
\IF{smooth}
\STATE ${(s,a) \mathrel{{+}{=}} \beta \cdot (s,a) \odot u, \hspace{0.5em} u \sim U(-\frac{1}{2},\frac{1}{2})^{dim((s,a))}}$ 
\ENDIF
\STATE Compute loss: %[possibly reference equation in paper]:
\STATE $\mathcal{J} = \log \frac{1}{M} \sum_{i=1}^M e^{x(s_e^i,a_e^i)} - \frac{1}{M} \sum_{i=1}^M x(s^i,a^i)$
\IF {flow reg}
\STATE Compute regularization loss:
\STATE $\mathcal{L} = - \frac{1}{M} \sum_{i=1}^M \log  q_\phi(s_e^i,a_e^i) + \log  p_\psi(s^i,a^i)$
\STATE $\mathcal{J} = \mathcal{J} + \alpha \mathcal{L}$
\ENDIF
\STATE Update $\psi \leftarrow \psi - \eta \nabla_{\psi}\mathcal{J}$
\STATE Update $\phi \leftarrow \phi - \eta \nabla_{\phi}\mathcal{J}$
\ENDIF
\ENDFOR

%"curly brackets turn an equation into a math atom and prevent it from breaking up"

\end{algorithmic}
\end{algorithm}
As in ValueDICE \cite{kostrikov2019imitation}, we found the bias due to the log-exp over the mini-batches did not hurt performance and was therefore left unhandled.

Another setting of interest is learning from observations (LFO) alone \cite{zhu2020off,torabi2018generative,torabi2021dealio}. That is, attempting to minimize: \begin{equation}
\argmin_{\pi} \, D_{KL}(d_\pi(s,s')||d_e(s,s')).
\end{equation}
While this objective is clearly underspecified in a non injective and deterministic MDP, in practice, recovering the expert's behavior is highly feasible \cite{zhu2020off}.
%injective meaning all actions lead to different states and deterministic meaning actions always lead to a state
%(since different actions may lead to the same state, hence multiple policies can induce identical state next-state distributions)
%so not one to one.
Seeing as none of our description above is specific to the domain of states and actions, CFIL naturally extends to LFO with no need of modification. This is in stark contrast to previous works in the LFO setting which have been highly tailored \cite{zhu2020off}. We shall demonstrate CFIL's utility in the LFO setting in the following section, where remarkably, we even find success when the flows model the single state distribution $d(s)$.



In this section we conduct comprehensive experiments to emphasise the effectiveness of DIAL, including evaluations under white-box and black-box settings, robustness to unforeseen adversaries, robustness to unforeseen corruptions, transfer learning, and ablation studies. Finally, we present a new measurement to test the balance between robustness and natural accuracy, which we named $F_1$-robust score. 


\subsection{A case study on SVHN and CIFAR-100}
In the first part of our analysis, we conduct a case study experiment on two benchmark datasets: SVHN \citep{netzer2011reading} and CIFAR-100 \cite{krizhevsky2009learning}. We follow common experiment settings as in \cite{rice2020overfitting, wu2020adversarial}. We used the PreAct ResNet-18 \citep{he2016identity} architecture on which we integrate a domain classification layer. The adversarial training is done using 10-step PGD adversary with perturbation size of 0.031 and a step size of 0.003 for SVHN and 0.007 for CIFAR-100. The batch size is 128, weight decay is $7e^{-4}$ and the model is trained for 100 epochs. For SVHN, the initial learinnig rate is set to 0.01 and decays by a factor of 10 after 55, 75 and 90 iteration. For CIFAR-100, the initial learning rate is set to 0.1 and decays by a factor of 10 after 75 and 90 iterations. 
%We compared DIAL to \cite{madry2017towards} and TRADES \citep{zhang2019theoretically}. 
%The evaluation is done using Auto-Attack~\citep{croce2020reliable}, which is an ensemble of three white-box and one black-box parameter-free attacks, and various $\ell_{\infty}$ adversaries: PGD$^{20}$, PGD$^{100}$, PGD$^{1000}$ and CW$_{\infty}$ with step size of 0.003. 
Results are averaged over 3 restarts while omitting one standard deviation (which is smaller than 0.2\% in all experiments). As can be seen by the results in Tables~\ref{black-and_white-svhn} and \ref{black-and_white-cifar100}, DIAL presents consistent improvement in robustness (e.g., 5.75\% improved robustness on SVHN against AA) compared to the standard AT 
%under variety of attacks 
while also improving the natural accuracy. More results are presented in Appendix \ref{cifar100-svhn-appendix}.


\begin{table}[!ht]
  \caption{Robustness against white-box, black-box attacks and Auto-Attack (AA) on SVHN. Black-box attacks are generated using naturally trained surrogate model. Natural represents the naturally trained (non-adversarial) model.
  %and applied to the best performing robust models.
  }
  \vskip 0.1in
  \label{black-and_white-svhn}
  \centering
  \small
  \begin{tabular}{l@{\hspace{1\tabcolsep}}c@{\hspace{1\tabcolsep}}c@{\hspace{1\tabcolsep}}c@{\hspace{1\tabcolsep}}c@{\hspace{1\tabcolsep}}c@{\hspace{1\tabcolsep}}c@{\hspace{1\tabcolsep}}c@{\hspace{1\tabcolsep}}c@{\hspace{1\tabcolsep}}c@{\hspace{1\tabcolsep}}c}
    \toprule
    & & \multicolumn{4}{c}{White-box} & \multicolumn{4}{c}{Black-Box}  \\
    \cmidrule(r){3-6} 
    \cmidrule(r){7-10}
    Defense Model & Natural & PGD$^{20}$ & PGD$^{100}$  & PGD$^{1000}$  & CW$^{\infty}$ & PGD$^{20}$ & PGD$^{100}$ & PGD$^{1000}$  & CW$^{\infty}$ & AA \\
    \midrule
    NATURAL & 96.85 & 0 & 0 & 0 & 0 & 0 & 0 & 0 & 0 & 0 \\
    \midrule
    AT & 89.90 & 53.23 & 49.45 & 49.23 & 48.25 & 86.44 & 86.28 & 86.18 & 86.42 & 45.25 \\
    % TRADES & 90.35 & 57.10 & 54.13 & 54.08 & 52.19 & 86.89 & 86.73 & 86.57 & 86.70 &  49.50 \\
    $\DIAL_{\kl}$ (Ours) & 90.66 & \textbf{58.91} & \textbf{55.30} & \textbf{55.11} & \textbf{53.67} & 87.62 & 87.52 & 87.41 & 87.63 & \textbf{51.00} \\
    $\DIAL_{\ce}$ (Ours) & \textbf{92.88} & 55.26  & 50.82 & 50.54 & 49.66 & \textbf{89.12} & \textbf{89.01} & \textbf{88.74} & \textbf{89.10} &  46.52  \\
    \bottomrule
  \end{tabular}
\end{table}


\begin{table}[!ht]
  \caption{Robustness against white-box, black-box attacks and Auto-Attack (AA) on CIFAR100. Black-box attacks are generated using naturally trained surrogate model. Natural represents the naturally trained (non-adversarial) model.
  %and applied to the best performing robust models.
  }
  \vskip 0.1in
  \label{black-and_white-cifar100}
  \centering
  \small
  \begin{tabular}{l@{\hspace{1\tabcolsep}}c@{\hspace{1\tabcolsep}}c@{\hspace{1\tabcolsep}}c@{\hspace{1\tabcolsep}}c@{\hspace{1\tabcolsep}}c@{\hspace{1\tabcolsep}}c@{\hspace{1\tabcolsep}}c@{\hspace{1\tabcolsep}}c@{\hspace{1\tabcolsep}}c@{\hspace{1\tabcolsep}}c}
    \toprule
    & & \multicolumn{4}{c}{White-box} & \multicolumn{4}{c}{Black-Box}  \\
    \cmidrule(r){3-6} 
    \cmidrule(r){7-10}
    Defense Model & Natural & PGD$^{20}$ & PGD$^{100}$  & PGD$^{1000}$  & CW$^{\infty}$ & PGD$^{20}$ & PGD$^{100}$ & PGD$^{1000}$  & CW$^{\infty}$ & AA \\
    \midrule
    NATURAL & 79.30 & 0 & 0 & 0 & 0 & 0 & 0 & 0 & 0 & 0 \\
    \midrule
    AT & 56.73 & 29.57 & 28.45 & 28.39 & 26.6 & 55.52 & 55.29 & 55.26 & 55.40 & 24.12 \\
    % TRADES & 58.24 & 30.10 & 29.66 & 29.64 & 25.97 & 57.05 & 56.71 & 56.67 & 56.77 & 24.92 \\
    $\DIAL_{\kl}$ (Ours) & 58.47 & \textbf{31.19} & \textbf{30.50} & \textbf{30.42} & \textbf{26.91} & 57.16 & 56.81 & 56.80 & 57.00 & \textbf{25.87} \\
    $\DIAL_{\ce}$ (Ours) & \textbf{60.77} & 27.87 & 26.66 & 26.61 & 25.98 & \textbf{59.48} & \textbf{59.06} & \textbf{58.96} & \textbf{59.20} & 23.51  \\
    \bottomrule
  \end{tabular}
\end{table}


% \begin{table}[!ht]
%   \caption{Robustness comparison of DIAL to Madry et al. and TRADES defense models on the SVHN dataset under different PGD white-box attacks and the ensemble Auto-Attack (AA).}
%   \label{svhn}
%   \centering
%   \begin{tabular}{llllll|l}
%     \toprule
%     \cmidrule(r){1-5}
%     Defense Model & Natural & PGD$^{20}$ & PGD$^{100}$ & PGD$^{1000}$ & CW$_{\infty}$ & AA\\
%     \midrule
%     $\DIAL_{\kl}$ (Ours) & $\mathbf{90.66}$ & $\mathbf{58.91}$ & $\mathbf{55.30}$ & $\mathbf{55.12}$ & $\mathbf{53.67}$  & $\mathbf{51.00}$  \\
%     Madry et al. & 89.90 & 53.23 & 49.45 & 49.23 & 48.25 & 45.25  \\
%     TRADES & 90.35 & 57.10 & 54.13 & 54.08 & 52.19 & 49.50 \\
%     \bottomrule
%   \end{tabular}
% \end{table}


\subsection{Performance comparison on CIFAR-10} \label{defence-settings}
In this part, we evaluate the performance of DIAL compared to other well-known methods on CIFAR-10. 
%To be consistent with other methods, 
We follow the same experiment setups as in~\cite{madry2017towards, wang2019improving, zhang2019theoretically}. When experiment settings are not identical between tested methods, we choose the most commonly used settings, and apply it to all experiments. This way, we keep the comparison as fair as possible and avoid reporting changes in results which are caused by inconsistent experiment settings \citep{pang2020bag}. To show that our results are not caused because of what is referred to as \textit{obfuscated gradients}~\citep{athalye2018obfuscated}, we evaluate our method with same setup as in our defense model, under strong attacks (e.g., PGD$^{1000}$) in both white-box, black-box settings, Auto-Attack ~\citep{croce2020reliable}, unforeseen "natural" corruptions~\citep{hendrycks2018benchmarking}, and unforeseen adversaries. To make sure that the reported improvements are not caused by \textit{adversarial overfitting}~\citep{rice2020overfitting}, we report best robust results for each method on average of 3 restarts, while omitting one standard deviation (which is smaller than 0.2\% in all experiments). Additional results for CIFAR-10 as well as comprehensive evaluation on MNIST can be found in Appendix \ref{mnist-results} and \ref{additional_res}.
%To further keep the comparison consistent, we followed the same attack settings for all methods.


\begin{table}[ht]
  \caption{Robustness against white-box, black-box attacks and Auto-Attack (AA) on CIFAR-10. Black-box attacks are generated using naturally trained surrogate model. Natural represents the naturally trained (non-adversarial) model.
  %and applied to the best performing robust models.
  }
  \vskip 0.1in
  \label{black-and_white-cifar}
  \centering
  \small
  \begin{tabular}{cccccccc@{\hspace{1\tabcolsep}}c}
    \toprule
    & & \multicolumn{3}{c}{White-box} & \multicolumn{3}{c}{Black-Box} \\
    \cmidrule(r){3-5} 
    \cmidrule(r){6-8}
    Defense Model & Natural & PGD$^{20}$ & PGD$^{100}$ & CW$^{\infty}$ & PGD$^{20}$ & PGD$^{100}$ & CW$^{\infty}$ & AA \\
    \midrule
    NATURAL & 95.43 & 0 & 0 & 0 & 0 & 0 & 0 &  0 \\
    \midrule
    TRADES & 84.92 & 56.60 & 55.56 & 54.20 & 84.08 & 83.89 & 83.91 &  53.08 \\
    MART & 83.62 & 58.12 & 56.48 & 53.09 & 82.82 & 82.52 & 82.80 & 51.10 \\
    AT & 85.10 & 56.28 & 54.46 & 53.99 & 84.22 & 84.14 & 83.92 & 51.52 \\
    ATDA & 76.91 & 43.27 & 41.13 & 41.01 & 75.59 & 75.37 & 75.35 & 40.08\\
    $\DIAL_{\kl}$ (Ours) & 85.25 & $\mathbf{58.43}$ & $\mathbf{56.80}$ & $\mathbf{55.00}$ & 84.30 & 84.18 & 84.05 & \textbf{53.75} \\
    $\DIAL_{\ce}$ (Ours)  & $\mathbf{89.59}$ & 54.31 & 51.67 & 52.04 &$ \mathbf{88.60}$ & $\mathbf{88.39}$ & $\mathbf{88.44}$ & 49.85 \\
    \midrule
    $\DIAL_{\awp}$ (Ours) & $\mathbf{85.91}$ & $\mathbf{61.10}$ & $\mathbf{59.86}$ & $\mathbf{57.67}$ & $\mathbf{85.13}$ & $\mathbf{84.93}$ & $\mathbf{85.03}$  & \textbf{56.78} \\
    $\TRADES_{\awp}$ & 85.36 & 59.27 & 59.12 & 57.07 & 84.58 & 84.58 & 84.59 & 56.17 \\
    \bottomrule
  \end{tabular}
\end{table}



\paragraph{CIFAR-10 setup.} We use the wide residual network (WRN-34-10)~\citep{zagoruyko2016wide} architecture. %used in the experiments of~\cite{madry2017towards, wang2019improving, zhang2019theoretically}. 
Sidelong this architecture, we integrate a domain classification layer. To generate the adversarial domain dataset, we use a perturbation size of $\epsilon=0.031$. We apply 10 of inner maximization iterations with perturbation step size of 0.007. Batch size is set to 128, weight decay is set to $7e^{-4}$, and the model is trained for 100 epochs. Similar to the other methods, the initial learning rate was set to 0.1, and decays by a factor of 10 at iterations 75 and 90. 
%For being consistent with other methods, the natural images are padded with 4-pixel padding with 32-random crop and random horizontal flip. Furthermore, all methods are trained using SGD with momentum 0.9. For $\DIAL_{\kl}$, we balance the robust loss with $\lambda=6$ and the domains loss with $r=4$. For $\DIAL_{\ce}$, we balance the robust loss with $\lambda=1$ and the domains loss with $r=2$. 
%We also introduce a version of our method that incorporates the AWP double-perturbation mechanism, named DIAL-AWP.
%which is trained using the same learning rate schedule used in ~\cite{wu2020adversarial}, where the initial 0.1 learning rate decays by a factor of 10 after 100 and 150 iterations. 
See Appendix \ref{cifar10-additional-setup} for additional details.

\begin{table}[ht]
  \caption{Black-box attack using the adversarially trained surrogate models on CIFAR-10.}
  \vskip 0.1in
  \label{black-box-cifar-adv}
  \centering
  \small
  \begin{tabular}{ll|c}
    \toprule
    \cmidrule(r){1-2}
    Surrogate (source) model & Target model & robustness \% \\
    % \midrule
    \midrule
    TRADES & $\DIAL_{\ce}$ & $\mathbf{67.77}$ \\
    $\DIAL_{\ce}$ & TRADES & 65.75 \\
    \midrule
    MART & $\DIAL_{\ce}$ & $\mathbf{70.30}$ \\
    $\DIAL_{\ce}$ & MART & 64.91 \\
    \midrule
    AT & $\DIAL_{\ce}$ & $\mathbf{65.32}$ \\
    $\DIAL_{\ce}$ & AT  & 63.54 \\
    \midrule
    ATDA & $\DIAL_{\ce}$ & $\mathbf{66.77}$ \\
    $\DIAL_{\ce}$ & ATDA & 52.56 \\
    \bottomrule
  \end{tabular}
\end{table}

\paragraph{White-box/Black-box robustness.} 
%We evaluate all defense models using Auto-Attack, PGD$^{20}$, PGD$^{100}$, PGD$^{1000}$ and CW$_{\infty}$ with step size 0.003. We constrain all attacks by the same perturbation $\epsilon=0.031$. 
As reported in Table~\ref{black-and_white-cifar} and Appendix~\ref{additional_res}, our method achieves better robustness compared to the other methods. Specifically, in the white-box settings, our method improves robustness over~\citet{madry2017towards} and TRADES by 2\% 
%using the common PGD$^{20}$ attack 
while keeping higher natural accuracy. We also observe better natural accuracy of 1.65\% over MART while also achieving better robustness over all attacks. Moreover, our method presents significant improvement of up to 15\% compared to the the domain invariant method suggested by~\citet{song2018improving} (ATDA).
%in both natural and robust accuracy. 
When incorporating 
%the double-perturbation mechanism of 
AWP, our method improves the results of $\TRADES_{\awp}$ by almost 2\%.
%and reaches state-of-the-art results for robust models with no additional data. 
% Additional results are available in Appendix~\ref{additional_res}.
When tested on black-box settings, $\DIAL_{\ce}$ presents a significant improvement of more than 4.4\% over the second-best performing method, and up to 13\%. In Table~\ref{black-box-cifar-adv}, we also present the black-box results when the source model is taken from one of the adversarially trained models. %Then, we compare our model to each one of them both as the source model and target model. 
In addition to the improvement in black-box robustness, $\DIAL_{\ce}$ also manages to achieve better clean accuracy of more than 4.5\% over the second-best performing method.
% Moreover, based on the auto-attack leader-board \footnote{\url{https://github.com/fra31/auto-attack}}, our method achieves the 1st place among models without additional data using the WRN-34-10 architecture.

% \begin{table}
%   \caption{White-box robustness on CIFAR-10 using WRN-34-10}
%   \label{white-box-cifar-10}
%   \centering
%   \begin{tabular}{lllll}
%     \toprule
%     \cmidrule(r){1-2}
%     Defense Model & Natural & PGD$^{20}$ & PGD$^{100}$ & PGD$^{1000}$ \\
%     \midrule
%     TRADES ~\cite{zhang2019theoretically} & 84.92  & 56.6 & 55.56 & 56.43  \\
%     MART ~\cite{wang2019improving} & 83.62  & 58.12 & 56.48 & 56.55  \\
%     Madry et al. ~\cite{madry2017towards} & 85.1  & 56.28 & 54.46 & 54.4  \\
%     Song et al. ~\cite{song2018improving} & 76.91 & 43.27 & 41.13 & 41.02  \\
%     $\DIAL_{\ce}$ (Ours) & $ \mathbf{90}$  & 52.12 & 48.88 & 48.78  \\
%     $\DIAL_{\kl}$ (Ours) & 85.25 & $\mathbf{58.43}$ & $\mathbf{56.8}$ & $\mathbf{56.73}$ \\
%     \midrule
%     $\DIAL_{\kl}$+AWP (Ours) & $\mathbf{85.91}$ & $\mathbf{61.1}$ & - & -  \\
%     TRADES+AWP \cite{wu2020adversarial} & 85.36 & 59.27 & 59.12 & -  \\
%     % MART + AWP & 84.43 & 60.68 & 59.32 & -  \\
%     \bottomrule
%   \end{tabular}
% \end{table}


% \begin{table}
%   \caption{White-box robustness on MNIST}
%   \label{white-box-mnist}
%   \centering
%   \begin{tabular}{llllll}
%     \toprule
%     \cmidrule(r){1-2}
%     Defense Model & Natural & PGD$^{40}$ & PGD$^{100}$ & PGD$^{1000}$ \\
%     \midrule
%     TRADES ~\cite{zhang2019theoretically} & 99.48 & 96.07 & 95.52 & 95.22 \\
%     MART ~\cite{wang2019improving} & 99.38  & 96.99 & 96.11 & 95.74  \\
%     Madry et al. ~\cite{madry2017towards} & 99.41  & 96.01 & 95.49 & 95.36 \\
%     Song et al. ~\cite{song2018improving}  & 98.72 & 96.82 & 96.26 & 96.2  \\
%     $\DIAL_{\kl}$ (Ours) & 99.46 & 97.05 & 96.06 & 95.99  \\
%     $\DIAL_{\ce}$ (Ours) & $\mathbf{99.49}$  & $\mathbf{97.38}$ & $\mathbf{96.45}$ & $\mathbf{96.33}$ \\
%     \bottomrule
%   \end{tabular}
% \end{table}


% \paragraph{Attacking MNIST.} For consistency, we use the same perturbation and step sizes. For MNIST, we use $\epsilon=0.3$ and step size of $0.01$. The natural accuracy of our surrogate (source) model is 99.51\%. Attacks results are reported in Table~\ref{black-and_white-mnist}. It is worth noting that the improvement margin is not conclusive on MNIST as it is on CIFAR-10, which is a more complex task.

% \begin{table}
%   \caption{Black-box robustness on MNIST and CIFAR-10 using naturally trained surrogate model and best performing robust models}
%   \label{black-box-mnist-cifar}
%   \centering
%   \begin{tabular}{lllllll}
%     \toprule
%     & \multicolumn{3}{c}{MNIST} & \multicolumn{3}{c}{CIFAR-10} \\
%     \cmidrule(r){2-4} 
%     \cmidrule(r){5-7}  
%     Defense Model & PGD$^{40}$ & PGD$^{100}$ & PGD$^{1000}$ & PGD$^{20}$ & PGD$^{100}$ & PGD$^{1000}$ \\
%     \midrule
%     TRADES ~\cite{zhang2019theoretically} & 98.12 & 97.86 & 97.81 & 84.08 & 83.89 & 83.8 \\
%     MART ~\cite{wang2019improving} & 98.16 & 97.96 & 97.89  & 82.82 & 82.52 & 82.47 \\
%     Madry et al. ~\cite{madry2017towards}  & 98.05 & 97.73 & 97.78 & 84.22 & 84.14 & 83.96 \\
%     Song et al. ~\cite{song2018improving} & 97.74 & 97.28 & 97.34 & 75.59 & 75.37 & 75.11 \\
%     $\DIAL_{\kl}$ (Ours) & 98.14 & 97.83 & 97.87  & 84.3 & 84.18 & 84.0 \\
%     $\DIAL_{\ce}$ (Ours)  & $\mathbf{98.37}$ & $\mathbf{98.12}$ & $\mathbf{98.05}$  & $\mathbf{89.13}$ & $\mathbf{88.89}$ & $\mathbf{88.78}$ \\
%     \bottomrule
%   \end{tabular}
% \end{table}



% \subsubsection{Ensemble attack} In addition to the white-box and black-box settings, we evaluate our method on the Auto-Attack ~\citep{croce2020reliable} using $\ell_{\infty}$ threat model with perturbation $\epsilon=0.031$. Auto-Attack is an ensemble of parameter-free attacks. It consists of three white-box attacks: APGD-CE which is a step size-free version of PGD on the cross-entropy ~\citep{croce2020reliable}. APGD-DLR which is a step size-free version of PGD on the DLR loss ~\citep{croce2020reliable} and FAB which  minimizes the norm of the adversarial perturbations, and one black-box attack: square attack which is a query-efficient black-box attack ~\citep{andriushchenko2020square}. Results are presented in Table~\ref{auto-attack}. Based on the auto-attack leader-board \footnote{\url{https://github.com/fra31/auto-attack}}, our method achieves the 1st place among models without additional data using the WRN-34-10 architecture.

%Additional results can be found in Appendix ~\ref{additional_res}.

% \begin{table}
%   \caption{Auto-Attack (AA) on CIFAR-10 with perturbation size $\epsilon=0.031$ with $\ell_{\infty}$ threat model}
%   \label{auto-attack}
%   \centering
%   \begin{tabular}{lll}
%     \toprule
%     \cmidrule(r){1-2}
%     Defense Model & AA \\
%     \midrule
%     TRADES ~\cite{zhang2019theoretically} & 53.08  \\
%     MART ~\cite{wang2019improving} & 51.1  \\
%     Madry et al. ~\cite{madry2017towards} & 51.52    \\
%     Song et al.   ~\cite{song2018improving} & 40.18 \\
%     $\DIAL_{\ce}$ (Ours) & 47.33  \\
%     $\DIAL_{\kl}$ (Ours) & $\mathbf{53.75}$ \\
%     \midrule
%     DIAL-AWP (Ours) & $\mathbf{56.78}$ \\
%     TRADES-AWP \cite{wu2020adversarial} & 56.17 \\
%     \bottomrule
%   \end{tabular}
% \end{table}


% \begin{table}[!ht]
%   \caption{Auto-Attack (AA) Robustness (\%) on CIFAR-10 with $\epsilon=0.031$ using an $\ell_{\infty}$ threat model}
%   \label{auto-attack}
%   \centering
%   \begin{tabular}{cccccc|cc}
%     \toprule
%     % \multicolumn{8}{c}{Defence Model}  \\
%     % \cmidrule(r){1-8} 
%     TRADES & MART & Madry & Song & $\DIAL_{\ce}$ & $\DIAL_{\kl}$ & DIAL-AWP  & TRADES-AWP\\
%     \midrule
%     53.08 & 51.10 & 51.52 &  40.08 & 47.33  & $\mathbf{53.75}$ & $\mathbf{56.78}$ & 56.17 \\

%     \bottomrule
%   \end{tabular}
% \end{table}

% \begin{table}[!ht]
% \caption{$F_1$-robust measurement using PGD$^{20}$ attack in white-box and black-box settings on CIFAR-10}
%   \label{f1-robust}
%   \centering
%   \begin{tabular}{ccccccc|cc}
%     \toprule
%     % \multicolumn{8}{c}{Defence Model}  \\
%     % \cmidrule(r){1-8} 
%     Defense Model & TRADES & MART & Madry & Song & $\DIAL_{\kl}$ & $\DIAL_{\ce}$ & DIAL-AWP  & TRADES-AWP\\
%     \midrule
%     White-box & 0.659 & 0.666 & 0.657 & 0.518 & $\mathbf{0.675}$  & 0.643 & $\mathbf{0.698}$ & 0.682 \\
%     Black-box & 0.844 & 0.831 & 0.846 & 0.761 & 0.847 & $\mathbf{0.895}$ & $\mathbf{0.854}$ &  0.849 \\
%     \bottomrule
%   \end{tabular}
% \end{table}

\subsubsection{Robustness to Unforeseen Attacks and Corruptions}
\paragraph{Unforeseen Adversaries.} To further demonstrate the effectiveness of our approach, we test our method against various adversaries that were not used during the training process. We attack the model under the white-box settings with $\ell_{2}$-PGD, $\ell_{1}$-PGD, $\ell_{\infty}$-DeepFool and $\ell_{2}$-DeepFool \citep{moosavi2016deepfool} adversaries using Foolbox \citep{rauber2017foolbox}. We applied commonly used attack budget 
%(perturbation for PGD adversaries and overshot for DeepFool adversaries) 
with 20 and 50 iterations for PGD and DeepFool, respectively.
Results are presented in Table \ref{unseen-attacks}. As can be seen, our approach  gains an improvement of up to 4.73\% over the second best method under the various attack types and an average improvement of 3.7\% over all threat models.


\begin{table}[ht]
  \caption{Robustness on CIFAR-10 against unseen adversaries under white-box settings.}
  \vskip 0.1in
  \label{unseen-attacks}
  \centering
%   \small
  \begin{tabular}{c@{\hspace{1.5\tabcolsep}}c@{\hspace{1.5\tabcolsep}}c@{\hspace{1.5\tabcolsep}}c@{\hspace{1.5\tabcolsep}}c@{\hspace{1.5\tabcolsep}}c@{\hspace{1.5\tabcolsep}}c@{\hspace{1.5\tabcolsep}}c}
    \toprule
    Threat Model & Attack Constraints & $\DIAL_{\kl}$ & $\DIAL_{\ce}$ & AT & TRADES & MART & ATDA \\
    \midrule
    \multirow{2}{*}{$\ell_{2}$-PGD} & $\epsilon=0.5$ & 76.05 & \textbf{80.51} & 76.82 & 76.57 & 75.07 & 66.25 \\
    & $\epsilon=0.25$ & 80.98 & \textbf{85.38} & 81.41 & 81.10 & 80.04 & 71.87 \\\midrule
    \multirow{2}{*}{$\ell_{1}$-PGD} & $\epsilon=12$ & 74.84 & \textbf{80.00} & 76.17 & 75.52 & 75.95 & 65.76 \\
    & $\epsilon=7.84$ & 78.69 & \textbf{83.62} & 79.86 & 79.16 & 78.55 & 69.97 \\
    \midrule
    $\ell_{2}$-DeepFool & overshoot=0.02 & 84.53 & \textbf{88.88} & 84.15 & 84.23 & 82.96 & 76.08 \\\midrule
    $\ell_{\infty}$-DeepFool & overshoot=0.02 & 68.43 & \textbf{69.50} & 67.29 & 67.60 & 66.40 & 57.35 \\
    \bottomrule
  \end{tabular}
\end{table}


%%%%%%%%%%%%%%%%%%%%%%%%% conference version %%%%%%%%%%%%%%%%%%%%%%%%%%%%%%%%%%%%%
\paragraph{Unforeseen Corruptions.}
We further demonstrate that our method consistently holds against unforeseen ``natural'' corruptions, consists of 18 unforeseen diverse corruption types proposed by \citet{hendrycks2018benchmarking} on CIFAR-10, which we refer to as CIFAR10-C. The CIFAR10-C benchmark covers noise, blur, weather, and digital categories. As can be shown in Figure \ref{corruption}, our method gains a significant and consistent improvement over all the other methods. Our method leads to an average improvement of 4.7\% with minimum improvement of 3.5\% and maximum improvement of 5.9\% compared to the second best method over all unforeseen attacks. See Appendix \ref{corruptions-apendix} for the full experiment results.


\begin{figure}[h]
 \centering
  \includegraphics[width=0.4\textwidth]{figures/spider_full.png}
%   \caption{Summary of accuracy over all unforeseen corruptions compared to the second and third best performing methods.}
  \caption{Accuracy comparison over all unforeseen corruptions.}
  \label{corruption}
\end{figure}


%%%%%%%%%%%%%%%%%%%%%%%%% conference version %%%%%%%%%%%%%%%%%%%%%%%%%%%%%%%%%%%%%

%%%%%%%%%%%%%%%%%%%%%%%%% Arxiv version %%%%%%%%%%%%%%%%%%%%%%%%%%%%%%%%%%%%%
% \newpage
% \paragraph{Unforeseen Corruptions.}
% We further demonstrate that our method consistently holds against unforeseen "natural" corruptions, consists of 18 unforeseen diverse corruption types proposed by \cite{hendrycks2018benchmarking} on CIFAR-10, which we refer to as CIFAR10-C. The CIFAR10-C benchmark covers noise, blur, weather, and digital categories. As can be shown in Figure  \ref{spider-full-graph}, our method gains a significant and consistent improvement over all the other methods. Our approach leads to an average improvement of 4.7\% with minimum improvement of 3.5\% and maximum improvement of 5.9\% compared to the second best method over all unforeseen attacks. Full accuracy results against unforeseen corruptions are presented in Tables \ref{corruption-table1} and \ref{corruption-table2}. 

% \begin{table}[!ht]
%   \caption{Accuracy (\%) against unforeseen corruptions.}
%   \label{corruption-table1}
%   \centering
%   \tiny
%   \begin{tabular}{lcccccccccccccccccc}
%     \toprule
%     Defense Model & brightness & defocus blur & fog & glass blur & jpeg compression & motion blur & saturate & snow & speckle noise  \\
%     \midrule
%     TRADES & 82.63 & 80.04 & 60.19 & 78.00 & 82.81 & 76.49 & 81.53 & 80.68 & 80.14 \\
%     MART & 80.76 & 78.62 & 56.78 & 76.60 & 81.26 & 74.58 & 80.74 & 78.22 & 79.42 \\
%     AT &  83.30 & 80.42 & 60.22 & 77.90 & 82.73 & 76.64 & 82.31 & 80.37 & 80.74 \\
%     ATDA & 72.67 & 69.36 & 45.52 & 64.88 & 73.22 & 63.47 & 72.07 & 68.76 & 72.27 \\
%     DIAL (Ours)  & \textbf{87.14} & \textbf{84.84} & \textbf{66.08} & \textbf{81.82} & \textbf{87.07} & \textbf{81.20} & \textbf{86.45} & \textbf{84.18} & \textbf{84.94} \\
%     \bottomrule
%   \end{tabular}
% \end{table}


% \begin{table}[!ht]
%   \caption{Accuracy (\%) against unforeseen corruptions.}
%   \label{corruption-table2}
%   \centering
%   \tiny
%   \begin{tabular}{lcccccccccccccccccc}
%     \toprule
%     Defense Model & contrast & elastic transform & frost & gaussian noise & impulse noise & pixelate & shot noise & spatter & zoom blur \\
%     \midrule
%     TRADES & 43.11 & 79.11 & 76.45 & 79.21 & 73.72 & 82.73 & 80.42 & 80.72 & 78.97 \\
%     MART & 41.22 & 77.77 & 73.07 & 78.30 & 74.97 & 81.31 & 79.53 & 79.28 & 77.8 \\
%     AT & 43.30 & 79.58 & 77.53 & 79.47 & 73.76 & 82.78 & 80.86 & 80.49 & 79.58 \\
%     ATDA & 36.06 & 67.06 & 62.56 & 70.33 & 64.63 & 73.46 & 72.28 & 70.50 & 67.31 \\
%     DIAL (Ours) & \textbf{48.84} & \textbf{84.13} & \textbf{81.76} & \textbf{83.76} & \textbf{78.26} & \textbf{87.24} & \textbf{85.13} & \textbf{84.84} & \textbf{83.93}  \\
%     \bottomrule
%   \end{tabular}
% \end{table}


% \begin{figure}[!ht]
%   \centering
%   \includegraphics[width=9cm]{figures/spider_full.png}
%   \caption{Accuracy comparison with all tested methods over unforeseen corruptions.}
%   \label{spider-full-graph}
% \end{figure}
% %%%%%%%%%%%%%%%%%%%%%%%%% Arxiv version %%%%%%%%%%%%%%%%%%%%%%%%%%%%%%%%%%%%%
%%%%%%%%%%%%%%%%%%%%%%%%% Arxiv version %%%%%%%%%%%%%%%%%%%%%%%%%%%%%%%%%%%%%

\subsubsection{Transfer Learning}
Recent works \citep{salman2020adversarially,utrera2020adversarially} suggested that robust models transfer better on standard downstream classification tasks. In Table \ref{transfer-res} we demonstrate the advantage of our method when applied for transfer learning across CIFAR10 and CIFAR100 using the common linear evaluation protocol. see Appendix \ref{transfer-learning-settings} for detailed settings.

\begin{table}[H]
  \caption{Transfer learning results comparison.}
  \vskip 0.1in
  \label{transfer-res}
  \centering
  \small
\begin{tabular}{c|c|c|c}
\toprule

\multicolumn{2}{l}{} & \multicolumn{2}{c}{Target} \\
\cmidrule(r){3-4}
Source & Defence Model & CIFAR10 & CIFAR100 \\
\midrule
\multirow{3}{*}{CIFAR10} & DIAL & \multirow{3}{*}{-} & \textbf{28.57} \\
 & AT &  & 26.95  \\
 & TRADES &  & 25.40  \\
 \midrule
\multirow{3}{*}{CIFAR100} & DIAL & \textbf{73.68} & \multirow{3}{*}{-} \\
 & AT & 71.41 & \\
 & TRADES & 71.42 &  \\
%  \midrule
% \multirow{3}{}{SVHN} & DIAL &  &  & \multirow{3}{}{-} \\
%  & Madry et al. &  &  &  \\
%  & TRADES &  &  &  \\ 
\bottomrule
\end{tabular}
\end{table}


\subsubsection{Modularity and Ablation Studies}

We note that the domain classifier is a modular component that can be integrated into existing models for further improvements. Removing the domain head and related loss components from the different DIAL formulations results in some common adversarial training techniques. For $\DIAL_{\kl}$, removing the domain and related loss components results in the formulation of TRADES. For $\DIAL_{\ce}$, removing the domain and related loss components results in the original formulation of the standard adversarial training, and for $\DIAL_{\awp}$ the removal results in $\TRADES_{\awp}$. Therefore, the ablation studies will demonstrate the effectiveness of combining DIAL on top of different adversarial training methods. 

We investigate the contribution of the additional domain head component introduced in our method. Experiment configuration are as in \ref{defence-settings}, and robust accuracy is based on white-box PGD$^{20}$ on CIFAR-10 dataset. We remove the domain head from both $\DIAL_{\kl}$, $\DIAL_{\awp}$, and $\DIAL_{\ce}$ (equivalent to $r=0$) and report the natural and robust accuracy. We perform 3 random restarts and omit one standard deviation from the results. Results are presented in Figure \ref{ablation}. All DIAL variants exhibits stable improvements on both natural accuracy and robust accuracy. $\DIAL_{\ce}$, $\DIAL_{\kl}$, and $\DIAL_{\awp}$ present an improvement of 1.82\%, 0.33\%, and 0.55\% on natural accuracy and an improvement of 2.5\%, 1.87\%, and 0.83\% on robust accuracy, respectively. This evaluation empirically demonstrates the benefits of incorporating DIAL on top of different adversarial training techniques.
% \paragraph{semi-supervised extensions.} Since the domain classifier does not require the class labels, we argue that additional unlabeled data can be leveraged in future work.
%for improved results. 

\begin{figure}[ht]
  \centering
  \includegraphics[width=0.35\textwidth]{figures/ablation_graphs3.png}
  \caption{Ablation studies for $\DIAL_{\kl}$, $\DIAL_{\ce}$, and $\DIAL_{\awp}$ on CIFAR-10. Circle represent the robust-natural accuracy without using DIAL, and square represent the robust-natural accuracy when incorporating DIAL.
  %to further investigate the impact of the domain head and loss on natural and robust accuracy.
  }
  \label{ablation}
\end{figure}

\subsubsection{Visualizing DIAL}
To further illustrate the superiority of our method, we visualize the model outputs from the different methods on both natural and adversarial test data.
% adversarial test data generated using PGD$^{20}$ white-box attack with step size 0.003 and $\epsilon=0.031$ on CIFAR-10. 
Figure~\ref{tsne1} shows the embedding received after applying t-SNE ~\citep{van2008visualizing} with two components on the model output for our method and for TRADES. DIAL seems to preserve strong separation between classes on both natural test data and adversarial test data. Additional illustrations for the other methods are attached in Appendix~\ref{additional_viz}. 

\begin{figure}[h]
\centering
  \subfigure[\textbf{DIAL} on natural logits]{\includegraphics[width=0.21\textwidth]{figures/domain_ce_test.png}}
  \hspace{0.03\textwidth}
  \subfigure[\textbf{DIAL} on adversarial logits]{\includegraphics[width=0.21\textwidth]{figures/domain_ce_adversarial.png}}
  \hspace{0.03\textwidth}
    \subfigure[\textbf{TRADES} on natural logits]{\includegraphics[width=0.21\textwidth]{figures/trades_test.png}}
    \hspace{0.03\textwidth}
    \subfigure[\textbf{TRADES} on adversarial logits]{\includegraphics[width=0.21\textwidth]{figures/trades_adversarial.png}}
  \caption{t-SNE embedding of model output (logits) into two-dimensional space for DIAL and TRADES using the CIFAR-10 natural test data and the corresponding PGD$^{20}$ generated adversarial examples.}
  \label{tsne1}
\end{figure}


% \begin{figure}[ht]
% \centering
%   \begin{subfigure}{4cm}
%     \centering\includegraphics[width=3.3cm]{figures/domain_ce_test.png}
%     \caption{\textbf{DIAL} on nat. examples}
%   \end{subfigure}
%   \begin{subfigure}{4cm}
%     \centering\includegraphics[width=3.3cm]{figures/domain_ce_adversarial.png}
%     \caption{\textbf{DIAL} on adv. examples}
%   \end{subfigure}
  
%   \begin{subfigure}{4cm}
%     \centering\includegraphics[width=3.3cm]{figures/trades_test.png}
%     \caption{\textbf{TRADES} on nat. examples}
%   \end{subfigure}
%   \begin{subfigure}{4cm}
%     \centering\includegraphics[width=3.3cm]{figures/trades_adversarial.png}
%     \caption{\textbf{TRADES} on adv. examples}
%   \end{subfigure}
%   \caption{t-SNE embedding of model output (logits) into two-dimensional space for DIAL and TRADES using the CIFAR-10 natural test data and the corresponding adversarial examples.}
%   \label{tsne1}
% \end{figure}



% \begin{figure}[ht]
% \centering
%   \begin{subfigure}{6cm}
%     \centering\includegraphics[width=5cm]{figures/domain_ce_test.png}
%     \caption{\textbf{DIAL} on nat. examples}
%   \end{subfigure}
%   \begin{subfigure}{6cm}
%     \centering\includegraphics[width=5cm]{figures/domain_ce_adversarial.png}
%     \caption{\textbf{DIAL} on adv. examples}
%   \end{subfigure}
  
%   \begin{subfigure}{6cm}
%     \centering\includegraphics[width=5cm]{figures/trades_test.png}
%     \caption{\textbf{TRADES} on nat. examples}
%   \end{subfigure}
%   \begin{subfigure}{6cm}
%     \centering\includegraphics[width=5cm]{figures/trades_adversarial.png}
%     \caption{\textbf{TRADES} on adv. examples}
%   \end{subfigure}
%   \caption{t-SNE embedding of model output (logits) into two-dimensional space for DIAL and TRADES using the CIFAR-10 natural test data and the corresponding adversarial examples.}
%   \label{tsne1}
% \end{figure}



\subsection{Balanced measurement for robust-natural accuracy}
One of the goals of our method is to better balance between robust and natural accuracy under a given model. For a balanced metric, we adopt the idea of $F_1$-score, which is the harmonic mean between the precision and recall. However, rather than using precision and recall, we measure the $F_1$-score between robustness and natural accuracy,
using a measure we call
%We named it
the
\textbf{$\mathbf{F_1}$-robust} score.
\begin{equation}
F_1\text{-robust} = \dfrac{\text{true\_robust}}
{\text{true\_robust}+\frac{1}{2}
%\cdot
(\text{false\_{robust}}+
\text{false\_natural})},
\end{equation}
where $\text{true\_robust}$ are the adversarial examples that were correctly classified, $\text{false\_{robust}}$ are the adversarial examples that were miss-classified, and $\text{false\_natural}$ are the natural examples that were miss-classified.
%We tested the proposed $F_1$-robust score using PGD$^{20}$ on CIFAR-10 dataset in white-box and black-box settings. 
Results are presented in Table~\ref{f1-robust} and demonstrate that our method achieves the best $F_1$-robust score in both settings, which supports our findings from previous sections.

% \begin{table}[!ht]
%   \caption{$F_1$-robust measurement using PGD$^{20}$ attack in white and black box settings on CIFAR-10}
%   \label{f1-robust}
%   \centering
%   \begin{tabular}{lll}
%     \toprule
%     \cmidrule(r){1-2}
%     Defense Model & White-box & Black-box \\
%     \midrule
%     TRADES & 0.65937  & 0.84435 \\
%     MART & 0.66613  & 0.83153  \\
%     Madry et al. & 0.65755 & 0.84574   \\
%     Song et al. & 0.51823 & 0.76092  \\
%     $\DIAL_{\ce}$ (Ours) & 0.65318   & $\mathbf{0.88806}$  \\
%     $\DIAL_{\kl}$ (Ours) & $\mathbf{0.67479}$ & 0.84702 \\
%     \midrule
%     \midrule
%     DIAL-AWP (Ours) & $\mathbf{0.69753}$  & $\mathbf{0.85406}$  \\
%     TRADES-AWP & 0.68162 & 0.84917 \\
%     \bottomrule
%   \end{tabular}
% \end{table}

\begin{table}[ht]
\small
  \caption{$F_1$-robust measurement using PGD$^{20}$ attack in white and black box settings on CIFAR-10.}
  \vskip 0.1in
  \label{f1-robust}
  \centering
%   \small
  \begin{tabular}{c
  @{\hspace{1.5\tabcolsep}}c @{\hspace{1.5\tabcolsep}}c @{\hspace{1.5\tabcolsep}}c @{\hspace{1.5\tabcolsep}}c
  @{\hspace{1.5\tabcolsep}}c @{\hspace{1.5\tabcolsep}}c @{\hspace{1.5\tabcolsep}}|
  @{\hspace{1.5\tabcolsep}}c
  @{\hspace{1.5\tabcolsep}}c}
    \toprule
    % \cmidrule(r){8-9}
     & TRADES & MART & AT & ATDA & $\DIAL_{\ce}$ & $\DIAL_{\kl}$ & $\DIAL_{\awp}$ & $\TRADES_{\awp}$ \\
    \midrule
    White-box & 0.659 & 0.666 & 0.657 & 0.518 & 0.660 & \textbf{0.675} & \textbf{0.698} & 0.682 \\
    Black-box & 0.844 & 0.831 & 0.845 & 0.761 & \textbf{0.890} & 0.847 & \textbf{0.854} & 0.849 \\ 
    \bottomrule
  \end{tabular}
\end{table}


\section{Conclusions}

In this work, we proposed a novel end-to-end uncertainty-based mitigation approach for adversarial attacks to the automated lane centering system. Our approach includes an uncertainty estimation method considering both data and model uncertainties, an uncertainty-aware trajectory planner, and an uncertainty-aware adaptive controller. Experiments on public dataset ad a production-grade simulator demonstrate the effectiveness of our approach in mitigating the attack effect. We believe that our methodology can be applied to other ADAS and autonomous driving functions, and will explore them in future work.


%In this work, we quantitatively analyze the impact of adversarial attacks throughout the pipeline of automated lane centering system. Based on the analysis, we design an end-to-end uncertainty aware mitigation method that includes  

%adaptive mitigation strategy that explicitly considers uncertainties. We test our proposed approach with public dataset and also conduct experiments with production-grade simulator. In experiments, our method mitigates the affect of adversarial attacks greatly. In the future, we will extend our end-to-end robust design to other ADAS applications and we can continuously improve the modules in the pipeline e.g. designing an uncertainty aware reinforcement learning based planner and controller.
%\begin{thebibliography}{00}

%\end{thebibliography}
%\bibliographystyle{abbrv}
\bibliographystyle{IEEEtran}
\bibliography{reference}

\end{document}
