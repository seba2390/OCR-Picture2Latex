\section{Experimental Results}
\label{sec:experiments}

We implement our design within the open-source driving assistance system OpenPilot. We test our proposed approach using both open dataset (e.g., comma2k19 dataset~\cite{schafer2018commute}) and
synthetic scenarios with an autonomous driving simulator.
%\begin{enumerate*}
%    \item  open dataset (e.g., comma2k19 dataset~\cite{schafer2018commute}) and
%    \item synthetic scenarios with an autonomous driving simulator. 
%\end{enumerate*}

For the open dataset, we adopt the kinematic bicycle model~\cite{kong2015kinematic} as the motion model for the ego vehicle. The kinematic bicycle model is commonly used to simulate how vehicles move according to speed and the front steering angle. The kinematic model can produce the vehicle trajectory and obtain the lateral deviation to measure the effectiveness of our proposed mitigation strategy. Moreover, as the trajectory calculated by the motion model may deviate from the original trace, we adopt the motion model based input generation from~\cite{sato2020hold} that combines motion model with perspective transformation to dynamically calculate camera frame updates due to attack-influenced vehicle control. %\alfred{modified a bit as this technique is inherent from the DRP paper}


%perspective transformation technique~\cite{tanaka2011vehicle} to dynamically calculate camera frame updates. \alfred{directly cite the DRP attack paper for these techniques? we}

For synthetic scenarios, we combine OpenPilot with LGSVL~\cite{rong2020lgsvl} to set up a closed-loop simulation environment. LGSVL is a Unity-based multi-robot simulator developed by LG Electronics America R\&D Center.  We reuse the LGSVL-OpenPilot bridge~\cite{sato2020hold} to transfer sensor data and driving command between the LGSVL simulator and OpenPilot. For both simulation methods, we test the original OpenPilot and ofur proposed design under different attacks as well as in benign situation.


% \subsection{Simulation methods}
% In our experiment, we adopt kinematic bicycle model \cite{kong2015kinematic} as the motion model for the ego vehicle. The kinematic bicycle model is commonly used to simulate how vehicles move according to speed and the front steering angle. 
% % In our experiment, the MPC will give optimized speed and steering angle. 
% The kinematic model can produce the actual trajectory and obtain the lateral deviation to measure the effectiveness of our proposed mitigation strategy. For simulation of ADAS, we need to ensure that the vehicle and the environment can interact, which means that this is a closed-loop system. Therefore, we also apply perspective transformation to dynamically synthesize camera frame according to the driving trajectory produced by the motion model\cite{sato2020hold}.

% %LGSVL is a Unity-based multi-robot simulator developed by LG Electronics America R&D Center.
% Although the kinematic model can simulate the trajectory with very small forecast error, more convincing closed-loop experiments are conducted in the production-grade autonomous vehicle simulator, LGSVL.  LGSVL  is an open-source autonomous driving simulator build based on the game engine Unity. It can produce photo-realistic simulation environment. The default vehicle dynamic model used in LGSVL is based on Unity's PhysX physics engine. It has realistic physical model for vehicles and high definition render pipeline from photorealistic environmental inputs to sensors. LGSVL has already supported open-source AD systems such as Baidu Apollo\cite{apollolg} and Autoware\cite{autowarelg}. We reuse the LGSVL-OpenPilot bridge~\cite{sato2020hold} to transfer sensor data and driving command between the LGSVL simulator and OpenPilot. 

% For both simulation methods, we test original OpenPilot and our proposed design under different attacks as well as in benign situation.

\subsection{Adversarial Attacks}
 %Most DNN-based adversarial examples are generated to attack specific tasks such as object detection and image classification.
In our work, as mentioned in previous sections, we applied the optimization-based physical attack in \cite{sato2020hold}. %which is the first attack systematically designed for Automated Lane Centering system. 
The attack works by placing an optimized patch on the road and it can lead the vehicle out of the lane within 1 second (which means the lateral deviation is larger than 0.735m in highway). The attack against OpenPilot shows high success rate and significant attack effect. To analyze the safety of the system and evaluate our proposed design, we conduct experiments under different settings (perturbation area, stealthiness levels, etc). The benign and attacked inputs are shown in Fig.~\ref{fig:input}.

\begin{figure}[htbp]
\centering
\begin{minipage}[t]{0.4\columnwidth}

\includegraphics[width=1\linewidth]{figures/t1_input.png}

\end{minipage}
\begin{minipage}[t]{0.4\columnwidth}
\centering
\includegraphics[width=1\linewidth]{figures/t16_input.png}

\end{minipage}
\caption{Input images in benign scenario and under attack. %\takami{more explanation?}\takami{there is no reference to this image.}
}
\label{fig:input}
\end{figure}

\subsection{Results for Uncertainty Estimation}
For each input frame, our system will calculate the data uncertainty by distributional parameter estimation and model uncertainty by Monte Carlo Dropout methods. The distributional parameter estimation is embedded in the original OpenPilot perception neural network. For the Monte Carlo Dropout, we only activate dropout in inference. There is always a trade-off between inference speed and estimation accuracy: larger number of samples results in higher accuracy and longer time. According to the analysis in~\cite{loquercio2020general} and our experiments, we set the dropout rate as 0.2 and the number of samples as 20. In this setting, the system can obtain precise variance estimation and reach a processing frequency of about 20Hz. The system will estimate uncertainties for 192 points of the predicted lane line in every frame. In experiments, we find that in most cases, data uncertainty and model uncertainty are at roughly the same order of magnitude (left subgraph in Fig.~\ref{fig:uncertainty estimation}). The uncertainty is relatively large when the vehicle is approaching the adversarial patch, which is consistent with the observations in confidence estimation. An example for our uncertainty estimation is shown in Fig.~\ref{fig:uncertainty estimation}. In a single frame, the uncertainties increase with the distance from the current position. This observation facilitates the planner to estimate a safe bound. As shown in Fig.~\ref{fig:demo_bound_predict_desire}, the uncertainty bound will form a triangle-like safe area and we will get a desired path from our planner, which is less affected by the attacks.  



\begin{figure}[htbp]
\centering
\begin{minipage}[b]{0.49\columnwidth}

\includegraphics[width=1\linewidth]{figures/uncertainty_bar1.pdf}

\end{minipage}
\begin{minipage}[b]{0.49\columnwidth}
\centering
\includegraphics[width=1\linewidth]{figures/uncertainty_t10_1.pdf}

\end{minipage}
\caption{Data uncertainty and model uncertainty. The left figure shows the overall uncertainties in different frames. The right figure shows the uncertainties increase with the distance. %\alfred{Make the x and y axis label bigger?} 
}
\label{fig:uncertainty estimation}
\end{figure}

%\begin{figure}[!ht]
%    \centering
%    \includegraphics[width=\columnwidth]{figures/uncertainty_t10_2.pdf}
%    \caption{
%    }
%    \label{fig:uncertainty estimation}
%\end{figure}

\begin{figure}[!ht]
    \centering
    \includegraphics[width=\columnwidth]{figures/demo_path_and_bound_fill_between.pdf}
    \caption{The shaded gray area indicates the safe area bounded by our estimation of uncertainties. The bounded area doesn't deviate to left under attacks.  %: positive dashed red points and negative dashed blue points are the original prediction; the solid curves are the fitting results after compensating the original points with estimated errors. The shaded gray area indicates the safe region bounded by our estimation of uncertainties. %\alfred{may be we can use shaded area (with certain degree of transparency) to highlight the uncertainty-bound safe area?} %\takami{I think the color can be used for lane type (e.g. red for left, blue for right)}
    }
    \label{fig:demo_bound_predict_desire}
\end{figure}


\subsection{Mitigation Results for Open Dataset}
We first test our proposed pipeline with highway image dataset and kinematic model. The input is consecutive image frames captured in a straight four-lane highway and the vehicle's speed is 20 m/s. In the benign case, our proposed systems and the original OpenPilot can both drive in the center of the lane. Fig.~\ref{fig:lateral dev} shows that the adversarial attacks can lead the vehicle with the original OpenPilot out of road and the deviation is more than 1.5 meters. In contrast, the system with our proposed uncertainty estimation and adaptive uncertainty-aware planner and controller can significantly mitigate the attack's effect, reducing the lateral deviation by about $66.8\%$. Besides, the experiments specifically demonstrate that the adversarial attack's effect can be further mitigated by utilizing the temporal locality with state cache. However, in this simulation setting, it is difficult for us to change the vehicle's speed and perception sampling frequency since they were determined by the driving scenario and sensor configurations in the dataset, which motivates us to conduct the following closed-loop simulation in LGSVL.


\begin{figure}[htbp]
\centerline{\includegraphics[width=\columnwidth]{figures/lat_dev_1.pdf}}
\caption{Comparison of our proposed method (with or without state cache) with the original OpenPilot and reference benign situation in terms of lateral deviation. The y-axis positive direction means left. 
%\takami{more explanation?}
}
\label{fig:lateral dev}
\end{figure}




\subsection{Mitigation Results in LGSVL-OpenPilot Environment}
%\takami{Should we include demo videos and include snapshot of simulator in this paper?}
In the closed-loop LGSVL simulation, we set the vehicle to follow a reference speed (20m/s) and control the throttle using a PID controller.  Fig.~\ref{fig:lgsvl_speed_adapt} compares our proposed approach with the original OpenPilot under adversarial scenario. The baseline is the solid blue curve which is the trajectory of the original OpenPilot driving under benign scenario. The solid green curve is the trajectory of the original OpenPilot driving under adversarial scenario where the patch is placed at 40 meters. As we can see, the original OpenPilot does not take prediction uncertainties under consideration and can deviate significantly under adversarial attack. Our proposed is also affected by the patch on the road but can significantly alleviate the adversarial impact.
\begin{figure}[!htbp]
    \centering
    \includegraphics[width=\columnwidth]{figures/lgsvl_speed_adaptation_v2.pdf}
    \caption{Ego vehicle trajectory under different scenarios. %\junjie{since ``attack: our approach'' ends at $\sim$400 meters, we can cut the figure at 400m to prevent any confusion.}
    }
    \label{fig:lgsvl_speed_adapt}
\end{figure}

To further evaluate our proposed approach, we conduct the experiments with different settings of the patch. We place the patch on different positions along the road (40m, 80m and 120m along the longitudinal direction) and adapt the perturbation area ratio (PAR) of the patch. The perturbation area ratio denotes the percentage of the pixels on the patch that are allowed to be perturbed. For our experiment, we generate dirty road patch with PAR ranging from $25\%$ to $100\%$. The maximum lateral deviation varies depending on different experiment settings. Throughout our experiments, the lateral deviation of the original OpenPilot ALC ranges from $0.8$ to $1.2$ meters, which is large enough to drive the victim vehicle out of the lane boundary. In contrast, the lateral deviation of our proposed approach ranges from $0.1$ to $0.4$ meter (still within the lane boundary). Overall, our approach can reduce the lateral deviation by $55.34\%\sim 90.76\%$, under various patch settings.
% To compare our much improvement of our proposed approach over the original OpenPilot ALC, we define $\frac{d_{base}-d_{our}}{d_{base}}$ as the improvement metric, here $d_{base}$ is the maximum lateral deviation of the original OpenPilot ALC and $d_{our}$ is the maximum lateral deviation of our proposed approach. Larger positive value of the improvement metric indicates smaller lateral deviate. Throughout our experiments, the lateral deviation of the original OpenPilot ALC ranges from $0.8$ to $1.2$ meters, which is larger enough to drive the victim vehicle across the lane boundary. In all the cases we test, the lateral deviation of our proposed approach ranges from $0.1$ to $0.4$ meter.
% The result is shown in Table~\ref{tab:varing_par_pos}. In all of the experiments, the maximum lateral deviation of original OpenPilot ALC ranges from $0.8$ to $1.2$ meters and it will drive the vehicle across the lane line. 
% Our mitigation strategy can reduce the the maximum deviation within 0.4 meters, which means that the vehicle can stay within its lane. Notice that lower PAR does not necessarily imply smaller lateral deviation, especially for Monte Carlo Dropout methods that introduce certain level of randomness during the perception process.

% \begin{table}[!h]
% \centering
% \caption{Improvement of our proposed approach tested in LGSVL-OpenPilot environment under different experiment settings: the perturbation area ratio (PAR) denotes the percentage of pixels that are perturbed and the patch is put on 40, 80, 120 meters along the longitudinal direction separately. \alfred{Describe how to interpret the result numbers? Now it's unclear what they mean and whether it is the higher the better or the lower the better}}            
% \label{tab:varing_par_pos}
% \begin{tabular}{|c|c|c|c|}
% \hline
% \diagbox{PAR ($\%$)}{Position (m) }    & 40   & 80   & 120  \\ \hline
% 100 & $59.33\%$  & $66.96\%$  & $76.15\%$ \\ \hline
% 75  &  $66.44\%$ & $90.76\%$  & $82.24\%$ \\ \hline
% 50  & $55.34\%$ & $76.72\%$  & $71.93\%$ \\ \hline
% 25  & $61.07\%$ & $66.07\%$  & $65.74\%$ \\ \hline
% \end{tabular}
% \end{table}


% \begin{table}[!h]
% \centering
% \caption{Maximum lateral deviation of our approach with LGSVL-OpenPilot environment under different experiment settings: the perturbation area ratio (PAR) denotes the percentage of pixels that are perturbed and the patch is put on 40, 80, 120 meters along the longitudinal direction.}            
% \label{tab:varing_par_pos}
% \begin{tabular}{|c|c|c|c|}
% \hline
% \diagbox{PAR ($\%$)}{Patch position (m) }    & 40   & 80   & 120  \\ \hline
% 100 & 0.34 m & 0.38 m & 0.26 m\\ \hline
% 75  & 0.30 m & 0.11 m & 0.19 m\\ \hline
% 50  & 0.40 m & 0.27 m & 0.32 m\\ \hline
% 25  & 0.37 m & 0.38 m & 0.37 m\\ \hline
% \end{tabular}
% \end{table}


% \begin{table}[!h]
% \centering
% \caption{Maximum lateral deviation of the original OpenPilot driving under adversarial scenario.}            
% \begin{tabular}{|c|c|c|c|}
% \hline
% \diagbox{PAR ($\%$)}{Patch position (m) }    & 40   & 80   & 120  \\ \hline
% 100 & 0.836 m & 1.15 m & 1.09 m\\ \hline
% 75  & 0.894 m & 1.19 m & 1.07 m\\ \hline
% 50  & 0.896 m & 1.16 m & 1.14 m\\ \hline
% 25  & 0.9505 m & 1.12 m & 1.08 m\\ \hline
% \end{tabular}
% \end{table}