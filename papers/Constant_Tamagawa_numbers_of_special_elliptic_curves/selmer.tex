

\documentclass{amsart}



%package
\usepackage{amsmath,amscd,amssymb,amsthm}
\usepackage{latexsym,enumerate,graphicx,geometry,rotating,braket,mathrsfs,url,graphicx}
\usepackage[all]{xy}
\usepackage[sort&compress,numbers]{natbib}
\usepackage[pdftex]{hyperref}
\usepackage{extarrows}
\usepackage{stmaryrd}
\usepackage{subfigure}
\usepackage{setspace}
\usepackage{float}


%package settings
%\hypersetup{CJKbookmarks,%
%       bookmarksnumbered,%
 %             colorlinks,%
              % linkcolor=blue,%
 %             citecolor=blue,%
  %            plainpages=false,%
   %         pdfstartview=FitH}


%def
\newcommand{\thebibfilename}{bib}

%thm
\renewcommand{\theequation}{\thechapter.\arabic{equation}}
\numberwithin{equation}{section}

\theoremstyle{plain}
\newtheorem{thm_}[equation]{Theorem}

\newtheorem{lemma_}[equation]{Lemma}
\newtheorem{prop_}[equation]{Proposition}
\newtheorem{cor_}[equation]{Corollary}
\newtheorem{eg_}[equation]{Example}
\newtheorem{con_}[equation]{Conjecture}
\newtheorem*{cons_}{Conjecture}

\theoremstyle{definition}
\newtheorem{thmu_}[equation]{Theorem}
\newtheorem*{thmus_}{Theorem}
\newtheorem{propu_}[equation]{Proposition}
\newtheorem*{propus_}{Proposition}
\newtheorem{coru_}[equation]{Corollary}
\newtheorem{lemu_}[equation]{Lemma}
\newtheorem{quesp_}[equation]{Question}
\newtheorem*{egus_}{Example}
\newtheorem{def_}[equation]{Definition}
\newtheorem*{defs_}{Definition}
\newtheorem{ex_}[equation]{Example}

%\theoremstyle{remark}
\renewcommand{\proofname}{\bf Proof}

\newcommand{\thm}[1]{\begin{thm_}#1\end{thm_}}
\newcommand{\thmu}[1]{\begin{thmu_}#1\end{thmu_}}
\newcommand{\thmus}[1]{\begin{thmus_}#1\end{thmus_}}
\newcommand{\lemm}[1]{\begin{lemma_}#1\end{lemma_}}
\newcommand{\lemu}[1]{\begin{lemu_}#1\end{lemu_}}
\newcommand{\lemus}[1]{\begin{lemus_}#1\end{lemus_}}
\newcommand{\eg}[1]{\begin{eg_}#1\end{eg_}}
\newcommand{\egu}[1]{\begin{egu_}#1\end{egu_}}
\newcommand{\egus}[1]{\begin{egus_}#1\end{egus_}}
\newcommand{\prop}[1]{\begin{prop_}#1\end{prop_}}
\newcommand{\propu}[1]{\begin{propu_}#1\end{propu_}}
\newcommand{\propus}[1]{\begin{propus_}#1\end{propus_}}
\newcommand{\defi}[1]{\begin{def_}#1\end{def_}}
\newcommand{\defis}[1]{\begin{defs_}#1\end{defs_}}
\newcommand{\rk}[1]{\begin{rk_}#1\end{rk_}}
\newcommand{\ex}[1]{\begin{ex_}#1\end{ex_}}
\newcommand{\cor}[1]{\begin{cor_}#1\end{cor_}}
\newcommand{\coru}[1]{\begin{coru_}#1\end{coru_}}
\newcommand{\con}[1]{\begin{con_}#1\end{con_}}
\newcommand{\cons}[1]{\begin{cons_}#1\end{cons_}}
\newcommand{\pf}[1]{\begin{proof}#1\end{proof}}

%%%%%%%%%%%%%%%%%%%%%%%%%%%%%%%%%%%%%%%%%%%%%%%%%%%%%%%%%%%%%%%%%%
\newcommand{\tabincell}[2]{\begin{tabular}{@{}#1@{}}#2\end{tabular}}
%%%%%%%%%%%%%%%%%%%%%%%%%%%%%%%%%%%%%%%%%%%%%%%%%%%%%%%%%%%%%%%%%%%%%%%%

%math operator
\DeclareMathOperator{\ord}{ord}
\DeclareMathOperator{\sgn}{sgn}
\DeclareMathOperator{\GL}{GL}
\DeclareMathOperator{\SL}{SL}
\DeclareMathOperator{\Gal}{Gal}
\DeclareMathOperator{\Hom}{Hom}
\DeclareMathOperator{\Ext}{Ext}
\DeclareMathOperator{\Tor}{Tor}
\DeclareMathOperator{\End}{End}
\DeclareMathOperator{\Ind}{Ind}
\DeclareMathOperator{\Spec}{Spec}
\DeclareMathOperator{\Stab}{Stab}
\DeclareMathOperator{\Aut}{Aut}
\DeclareMathOperator{\im}{im}
\DeclareMathOperator{\Ob}{Ob}
\DeclareMathOperator{\Br}{Br}
\DeclareMathOperator{\Pic}{Pic}
\DeclareMathOperator{\infl}{inf}
\DeclareMathOperator{\res}{res}
\DeclareMathOperator{\cores}{cor}
\DeclareMathOperator{\inv}{inv}
\DeclareMathOperator{\Frob}{Frob}
\DeclareMathOperator{\rank}{rank}
\DeclareMathOperator{\ch}{char}
%sym (changed defs with % ended)
\newcommand{\bA}{\mathbb A}%
\newcommand{\CC}{\mathbb C}
\newcommand{\DD}{\mathbb D}%
\newcommand{\FF}{\mathbb F}
\newcommand{\II}{\mathbb I}
\newcommand{\NN}{\mathbb N}
\newcommand{\QQ}{\mathbb Q}
\newcommand{\RR}{\mathbb R}
\newcommand{\ZZ}{\mathbb Z}

%eqn
\newcommand{\eq}[1]{\begin{equation}#1\end{equation}}
\newcommand{\eqn}[1]{\begin{equation*}#1\end{equation*}}
\newcommand{\ga}[1]{\begin{gather}#1\end{gather}}
\newcommand{\gan}[1]{\begin{gather*}#1\end{gather*}}
\newcommand{\al}[1]{\begin{align}#1\end{align}}
\newcommand{\aln}[1]{\begin{align*}#1\end{align*}}
\newcommand{\spl}[1]{\begin{split}#1\end{split}}
\newcommand{\cs}[1]{\begin{cases}#1\end{cases}}
\newcommand{\ml}[1]{\begin{multline}#1\end{multline}}
\newcommand{\mln}[1]{\begin{multline*}#1\end{multline*}}
%blk
\newcommand{\vs}{\vspace{5mm}}
\newcommand{\itmz}[1]{\begin{itemize}#1\end{itemize}}
\newcommand{\enmt}[1]{\begin{enumerate}#1\end{enumerate}}
%\newcommand{\it}{\item}
\newcommand{\clmn}[1]{\begin{columns}#1\end{columns}}
\newcommand{\cl}{\column{.5\textwidth}}
\newcommand{\tabl}[3]{\begin{center}\small{#1\\}\begin{tabular}{#2}#3\end{tabular}\end{center}}
\newcommand{\pb}[2]{\parbox[t]{#1}{#2}}
\newcommand{\rt}[1]{\begin{rotate}{90}#1\end{rotate}}
\newcommand{\itc}[1]{\item[\upshape{(#1)}]}%
\newcommand{\newnoindbf}[1]{\vspace{2mm}\noindent\textbf{#1}}

%this tex
\newcommand{\roright}[1]{\rotatebox{270}{#1}}
\newcommand{\simright}{\roright{$\sim$}}
\newcommand{\chapauthor}[1]{\begin{center}#1\bigskip\end{center}}
\newcommand{\Lt}{{L\{\tau\}}}
\newcommand{\Ct}{{\bfC\{\tau\}}}
\renewcommand{\mod}[1]{\ \mathrm{mod}\ #1}  % renew !!!
\DeclareMathOperator{\lc}{lc}




\newcommand{\ifof}{if and only if }


\usepackage[OT2,T1]{fontenc}

\DeclareSymbolFont{cyrletters}{OT2}{wncyr}{m}{n}

\DeclareMathSymbol{\Sha}{\mathalpha}{cyrletters}{"58}

\begin{document}
%%%%%%%%%%%%%%%%%%%%%%%%%%%%%%
%%%%%%%%%%%%%%%%%%%%%%%%%%%%%%

%title & author
\title{Constant Tamagawa numbers of special elliptic curves}
%\author{Luying Li, Chang Lv}
\author[L. Li]{Luying Li}
\address{State Key Laboratory of Information Security\\
Institute of Information Engineering\\
Chinese Academy of Sciences\\
Beijing 100093, P.R. China}
\email{lilouying16@mails.ucas.ac.cn}

\author[C. Lv]{Chang Lv}
\address{State Key Laboratory of Information Security\\
Institute of Information Engineering\\
Chinese Academy of Sciences\\
Beijing 100093, P.R. China}
\email{lvchang@amss.ac.cn}

\thanks{This work was supported by National Natural Science Foundation of China (Grant No.11701552)}

\maketitle



%%%%%%%%%%%%%%%%%%%%%%%%%%%%%%
%%%%%%%%%%%%%%%%%%%%%%%%%%%%%%


\begin{abstract}
For the elliptic curves $E_{\sigma 2D} : y^2 = x^3 + \sigma 2Dx$ , which has 2-isogeny curve $E'_{\sigma 2D} : y^2 = x^3 -\sigma 8Dx$,  $\sigma = \pm 1,\ D = p_1^{e_1}p_2^{e_2}\cdots p_n^{e_n}$, where $p_i$ are different odd prime numbers and $e_i = 1 \text{ or } 3$, we demonstrate that Tamagawa numbers of these elliptic curves are always one or zero by the use of matrix in finite field $\FF_2$. The specific number depends on the value of $\sigma$. By our proofs of these results, we find a method to quickly sieve a part of the elliptic curves with Mordell-Weil rank zero or rank one in this form as an application.
\keywords{Elliptic curves, Selmer groups, Tamagawa numbers}
\end{abstract}


\section{Introduction}
The Tamagawa numbers for $E_A: y^2 = x^3-A x$ with $A \in \QQ^\times$ are
$$T_{A} = \# S^{(\phi)}(E_A)/\# S^{(\phi)}(E_{-4A}) = 2^{-t_A}.$$ 
$S^{(\phi)}(E_A)$ is the $\phi$-Selmer group of $E_A$, $\phi$ is the morphism from $E_A$ to its 2-isogeny has the following form.
$$
\phi : E \rightarrow E', \quad (x,y)\mapsto (\frac{x^2}{y^2},\frac{y(-A-x^2)}{x^2})
$$

In this paper, we mainly concern the $\phi$-Selmer groups $S^{({\phi})}({E_{\pm}})$ of following elliptic curves:
$$
E = E_{-\sigma2D}: y^2 = x^3 + \sigma2D x, \ \sigma = \pm 1, \  D = p_1^{e_1}p_2^{e_2}\cdots p_n^{e_n} \in \QQ^\times ,\ e_i = 1,3.
$$
We show the Tamagawa numbers $t_{-\sigma 2 D}$ in this subgroup are uniform distribution of $\{0,1\}$.

Let $E' = E'_{\sigma 2D}: y^2 = x^3 -\sigma 8D x$ denote the 2-isogeny of $E$, and let $E_+ = y^2 = x^3 + 2 D x$, $E_+' = y^2 = x^3 - 8 D x$ and $E_- = y^2 = x^3 - 2 D x$, $E_-' = y^2 = x^3 + 8 D x$ for simplification.
Our main purpose is to prove the following theorems:
\thm{[Theorem \ref{thm_n1}] \label{thm_1} Let $E_-$, $E_-'$ be the elliptic curves defined above. Then     $$2^{-t_{2D}} = T_{2D} = \# S^{(\phi)}(E_-)/\# S^{(\phi)}(E'_{-}) = 1/2.$$ 
}
\thm{[Theorem \ref{thm_n2}] \label{thm_2} Let $E_+$, $E_+'$ be the elliptic curves defined above. Then     $$2^{-t_{-2D}} = T_{-2D} = \# S^{(\phi)}(E_+)/\# S^{(\phi)}(E'_{+}) = 1.$$
}

Furthermore, we claim that for all D, the parity of $\dim_2 S_2(E_-)$ is odd and the parity of $\dim_2 S_2(E_+)$ is even.
The definition of $\phi$-Selmer groups can be found in, for example \cite{gtm106}. From the proof of our main theory, we find a method to quickly sieve a part of the elliptic curves with Mordell-Weil rank zero or rank one in this form.

For $D$ which has quite few prime factors, we can enumerate all the possible cases of $S^{(\phi)}(E_\pm)$ and $S^{({\phi})}({E_\pm'})$ to verify the above two theorems. For example, if $D = p_{1}p_{2}$, we have Table 1 for $\sigma = -1$, and Table 2 for $\sigma = 1$, from which, it is easy to obtain our theorems.

\begin{table}[H]
\begin{center}

\begin{tabular}{cccc|l|l}
	$p \mod 8$ & $q \mod 8$ & $(\frac{p}{q})$ &($\frac{q}{p})$ &$S^{(\phi)}(E_-)$ &$S^{({\phi})}({E_-'})$ \\ \hline
	1 & 1 & -1 & -1 & 1,pq,2,2pq& 1,-1,pq,-pq,2,-2,2pq,-2pq\\
	1 & 1 & 1 & 1 & 1,p,q,pq,2,2p,2q,2pq & 1,-1,p,-p,q,-q,pq,-pq,2,-2,2p,-2p,2q,-2q,2pq,-2pq\\
	1 & 3 & -1 & -1 & 1,2pq & 1,pq,-2,-2pq\\
	1 & 3 & 1 & 1 & 1,p,2q,2pq & 1,p,q,pq,-pq,-2,-2p,-2q,-2pq\\
	1 & 5 & -1 & -1 & 1,2pq & 1,-1,2pq,-2pq\\
	1 & 5 & 1 & 1 & 1,p,2q,2pq & 1,-1,p,-p,2q,-2q,2pq,-2pq\\
	1 & 7 & -1 & -1 & 1,2pq & 1,-pq,2,-2pq\\
	1 & 7 & 1 & 1 & 1,p,2q,2pq & 1,p,-q,-pq,2,2p,-2q,-2pq\\
	3 & 1 & -1 & -1 & 1,2pq & 1,pq,-2,-2pq\\
	3 & 1 & 1 & 1 & 1,q,2p,2pq & 1,p,q,pq,-2,-2p,-2q,-2pq\\
	3 & 3 & -1 & 1 & 1,2pq &1,pq,-2,-2pq\\
	3 & 3 & 1 & -1 & 1,2pq &1,pq,-2,-2pq\\
	3 & 5 & -1 & -1 & 1,2pq &1,-q,2p,-2pq\\
	3 & 5 & 1 & 1 & 1,2pq &1,p,-2q,-2pq\\
	3 & 7 & -1 & 1 & 1,2pq &1,q,-2p,-2pq\\
	3 & 7 & 1 & -1 & 1,2pq &1,-q,2p,-2pq\\
	5 & 1 & -1 & -1 & 1,2pq &1,-1,2pq,-2pq\\
	5 & 1 & 1 & 1 & 1,q,2p,2pq &1,-1,q,-q,2p,-2p,2pq,-2pq\\
	5 & 3 & -1 & -1 & 1,2pq &1,-p,2q,-2pq\\
	5 & 3 & 1 & 1 & 1,2pq &1,q,-2p,-2pq\\
	5 & 5 & -1 & -1 & 1,2pq &1,-1,2pq,-2pq \\
	5 & 5 & 1 & 1 & 1,2pq &1,-1,2pq,-2pq\\
	5 & 7 & -1 & -1 & 1,2pq &1,-p,2q,-2pq\\
	5 & 7 & 1 & 1 & 1,2pq &1,-q,2p,-2pq\\
	7 & 1 & -1 & -1 & 1,2pq &1,-pq,2,-2pq\\
	7 & 1 & 1 & 1 & 1,q,2p,2pq &1,q,-p,-pq,2,-2p,2q,-2pq\\
	7 & 3 & -1 & 1 & 1,2pq &1,-p,2q,-2pq\\
	7 & 3 & 1 & -1 & 1,2pq &1,p,-2q,-2pq\\
	7 & 5 & -1 & -1 & 1,2pq &1,-q,2p,-2pq\\
	7 & 5 & 1 & 1 & 1,2pq &1,-p,2q,-2pq\\
	7 & 7 & -1 & 1 & 1,pq,2,2pq &1,q,-p,-pq,2,-2p,2q,-2pq\\
	7 & 7 & 1 & -1 & 1,pq,2,2pq &1,p,-q,-pq,2,2p,-2q-2pq\\
	\hline
\end{tabular}
	\caption{table 1}
\end{center}
\end{table}
\newpage
\begin{table}[h]
\begin{center}
\begin{tabular}{cccc|l|l}
	$p \mod 8$ & $q \mod 8$ & $(\frac{p}{q})$ &($\frac{q}{p})$ &$S^{({\phi})}({E_+'})$ &$S^{(\phi)}(E_+)$ \\ \hline
	1 & 1 & -1 & -1 & 1,pq,2,2pq& 1,pq,-2,-2pq\\
	1 & 1 & 1 & 1 & 1,p,q,pq,2,2p,2q,2pq & 1,p,q,pq,-2,-2p,-2q,-2pq\\
	1 & 3 & -1 & -1 & 1,2pq & 1,-2pq\\
	1 & 3 & 1 & 1 & 1,p,2q,2pq & 1,p,-2q,-2pq\\
	1 & 5 & -1 & -1 & 1,2pq & 1,-2pq\\
	1 & 5 & 1 & 1 & 1,p,2q,2pq & 1,p,-2q,-2pq\\
	1 & 7 & -1 & -1 & 1,pq,2,2pq & 1,-pq,2,-2pq\\
	1 & 7 & 1 & 1 & 1,p,q,pq,2,2p,2q,2pq & 1,p,-q,-pq,2,2p,-2q,-2pq\\
	3 & 1 & -1 & -1 & 1,2pq & 1,-2pq\\
	3 & 1 & 1 & 1 & 1,q,2p,2pq & 1,q,-2p,-2pq\\
	3 & 3 & -1 & 1 & 1,q,2p,2pq &1,pq,-2,-2pq\\
	3 & 3 & 1 & -1 & 1,p.2q,2pq &1,pq,-2,-2pq\\
	3 & 5 & -1 & -1 & 1,2pq &1,-2pq\\
	3 & 5 & 1 & 1 & 1,2pq &1,-2pq\\
	3 & 7 & -1 & 1 & 1,2pq &1,-2pq\\
	3 & 7 & 1 & -1 & 1,p,2q,2pq &1,-q,2p,-2pq\\
	5 & 1 & -1 & -1 & 1,2pq &1,-2pq\\
	5 & 1 & 1 & 1 & 1,q,2p,2pq &1,q,-2p,-2pq\\
	5 & 3 & -1 & -1 & 1,2pq &1,-2pq\\
	5 & 3 & 1 & 1 & 1,2pq &1,-2pq\\
	5 & 5 & -1 & -1 & 1,2pq &1,-2pq \\
	5 & 5 & 1 & 1 & 1,2pq &1,-2pq\\
	5 & 7 & -1 & -1 & 1,2pq &1,-2pq\\
	5 & 7 & 1 & 1 & 1,q,2p,2pq &1,-q,2p,-2pq\\
	7 & 1 & -1 & -1 & 1,pq,2,2pq &1,-pq,2,-2pq\\
	7 & 1 & 1 & 1 & 1,p,q,pq,2,2p,2q,2pq &1,q,-p,-pq,2,-2p,2q,-2pq\\
	7 & 3 & -1 & 1 & 1,q,2p,2pq &1,-p,2q,-2pq\\
	7 & 3 & 1 & -1 & 1,2pq &1,-2pq\\
	7 & 5 & -1 & -1 & 1,2pq &1,-2pq\\
	7 & 5 & 1 & 1 & 1,p,2q,2pq &1,-p,2q,-2pq\\
	7 & 7 & -1 & 1 & 1,pq,2,2pq &1,-p,2q,-2pq\\
	7 & 7 & 1 & -1 & 1,pq,2,2pq &1,-q,2p,-2pq\\
	\hline
\end{tabular}
	\caption{table 2}
\end{center}
\end{table}


The paper is organized as follows: in Section 2, we introduce some basic facts from the literature; in Section 3 we prove Theorems \ref{thm_1} and \ref{thm_2} by using the matrix in $\FF_2$. Finally, in Section 4, we show a method
which can quickly sieve a part of the elliptic curves with Mordell-Weil rank zero or rank one in this form.
\section{Notations and Lemmas}\label{sec_2}

\[
\begin{tabular}{l|l}
\text{Symbols} & \text{Meanings} \\ \hline
$D$  &  $D = p_1^{e_1}p_2^{e_2}\cdots p_n^{e_n}$, $p_i$ are different odd prime numbers and $e_i = 1,3$\\
$n$  &  number of odd prime factors of $D$\\
$p_{n+1}$ & $p_{n+1} = -1$ \\
$S$ & $S = \{ \infty, 2, p_i \}$ \\
$\mathbb{Q}(S,2)$ & $\mathbb{Q}(S,2) = <-1,2,p_i>\subseteq \mathbb{Q}^{\times}/\QQ^{\times2} $, $p_i$ is prime factor of D  \\
$\QQ_{2D}$ & $\QQ_{2D} = <p_i> \subseteq \QQ^{\times}/\QQ^{\times2} $\\
$\QQ_{-2D}$ & $\QQ_{-2D} = <-1, p_i> \subseteq \QQ^{\times}/\QQ^{\times2}$\\
$d$ & an element of $\mathbb{Q}(S,2)$, usually write as $d = p_{i_1}p_{i_2}\cdots p_{i_k} , 1 \le i_1 < i_2 < \cdots < i_k \le n+1$ \\

$\delta (d)$ & the sum of $y_{i_j}(-1)$, $p_{i_j}$ are different prime factors of $d$. The definition of $y_{i_j}(-1)$ can be found at \ref{eq:array2}\\
$(\frac{\ }{\ })$ & the Legendre  Symbol\\
\hline
\end{tabular}
\]
For $d\in \QQ(S,2)$, define the following curves: $ C_{d-}: W^2 = d + \frac{8DZ^4}{d},$ $ C'_{d-}: W^2 = d - \frac{2DZ^4}{d},$ $ C_{d+}: W^2 = d - \frac{8DZ^4}{d},$ $ C'_{d+}: W^2 = d + \frac{2DZ^4}{d}. $
According to \cite{gtm106}, we have
$$
\begin{aligned}
 S^{(\phi)}(E_-) \cong \{d \in \mathbb{Q}(S,2): C_{d-}(\mathbb{Q}_v) \neq \emptyset , \forall v \in S \} \subseteq \QQ^{\times}/\QQ^{\times2 }, \\
S^{({\phi})}(E_-') \cong \{d \in \mathbb{Q}(S,2): C_{d-}'(\mathbb{Q}_v) \neq \emptyset , \forall v \in S \}\subseteq \QQ^{\times}/\QQ^{\times2 }. \\
S^{(\phi)}(E_+) \cong \{d \in \mathbb{Q}(S,2): C_{d+}(\mathbb{Q}_v) \neq \emptyset , \forall v \in S \} \subseteq \QQ^{\times}/\QQ^{\times2 },\\
S^{({\phi})}(E_+') \cong \{d \in \mathbb{Q}(S,2): C_{d+}'(\mathbb{Q}_v) \neq \emptyset , \forall v \in S \}\subseteq \QQ^{\times}/\QQ^{\times2 }, \\
\end{aligned}
$$
and we have the following inclusions:
$\{1, \ 2D\} \subseteq S^{(\phi)}(E_-) $, $\{1, -2D\} \subseteq S^{({\phi})}({E_-'})$ and $\{1, \ -2D\} \subseteq S^{(\phi)}(E_+) $, $\{1, 2D\} \subseteq S^{({\phi})}({E_+'})$. Since $S^{(\phi)}(E_\pm) $ and $S^{({\phi})}({E_\pm'})$ have above properties, and are
multiplicative abelian groups, when we calculate $S^{(\phi)}(E_\pm)$ and $S^{({\phi})}({E_\pm'})$,
we only need to concern ourselves with $d \in \QQ_{2D}$ and $\QQ_{-2D}$.


\defi{Let

\begin{itemize}
\item $s_-(\phi) \triangleq  S^{(\phi)}(E_-) \bigcap \QQ_{2D}$,
\item $s_-'(\phi) \triangleq S^{({\phi})}({E_-'}) \bigcap \QQ_{-2D}$,
\item $s_+(\phi) \triangleq  S^{(\phi)}(E_+) \bigcap \QQ_{-2D}$,
\item $s_+'(\phi) \triangleq S^{({\phi})}({E_+'}) \bigcap \QQ_{2D}$.
\end{itemize}
}
Apparently, there is a nature map
$$
\begin{aligned}
s_-(\phi) &\rightarrow S^{(\phi)}(E_-) - s_-(\phi)\\
d &\mapsto d \times 2D (\mod \QQ^{\times}/\QQ^{\times 2})  \\
\end{aligned}
$$
and it is easy to check that it is a one-to-one map. Thus we have $ \#S^{(\phi)}(E_-) = 2\# s_-(\phi)$. Similarly, $ \# S^{(\phi)}(E_+) = 2\# s_+(\phi)$, $\#S^{({\phi})}({E_\pm'}) = 2 \# s_\pm'(\phi)$.

The following lemmas give the method to calculate $ s_\pm(\phi)$ and $s_\pm'(\phi) $ by using the Hensel's Lemma. The proof of these lemmas can be found in \cite{zeng}.

\lemm{\label{lem1}
Let $C_{d-}$ be defined as above. Then\vspace{1.5ex}

\begin{enumerate}
\item[(1)] $C_{d-}(\mathbb{Q}_2) \neq \emptyset \Leftrightarrow d  \equiv 1 (\mod 8)$, \vspace{1.5ex}
\item[(2)] $C_{d-}(\mathbb{Q}_{p_i}) \neq \emptyset \Leftrightarrow (\frac{d}{p_i}) \ =\ 1 , \forall p_i \nmid d$,
\item[(3)] $C_{d-}(\mathbb{Q}_{p_i}) \neq \emptyset \Leftrightarrow (\frac{2D/(dp_i^{e^i-1})}{p_i}) \ =\ 1 , \forall  p_i \mid d$.
\end{enumerate}
}

\lemm{\label{lem2}
	Let $C_{d-}'$ be defined as above. Then\vspace{1.5ex}

\begin{enumerate}
\item[(1)] $C_{d-}'(\mathbb{Q}_2) \neq \emptyset \Leftrightarrow d  \equiv 1 ( \mod 8) \ \ or \ d - 2D/d\equiv 1 ( \mod 8)  $, \vspace{1ex}
\item[(2)] $C_{d-}'(\mathbb{Q}_{p_i}) \neq \emptyset \Leftrightarrow (\frac{d}{p_i}) \ =\ 1 , \forall p_i \nmid d$,
\item[(3)] $C_{d-}'(\mathbb{Q}_{p_i}) \neq \emptyset \Leftrightarrow (\frac{-2D/(dp_i^{e_i-1})}{p_i}) \ =\ 1 , \forall  p_i \mid d$.
\end{enumerate}
}

\lemm{\label{lem3}
Let $C_{d+}$ be defined as above. Then\vspace{1.5ex}

\begin{enumerate}
\item[(1)] $C_{d+}(\mathbb{Q}_2) \neq \emptyset \Leftrightarrow d  \equiv 1 (\mod 8)$, \vspace{1.5ex}
\item[(2)] $C_{d+}(\mathbb{Q}_{p_i}) \neq \emptyset \Leftrightarrow (\frac{d}{p_i}) \ =\ 1 , \forall p_i \nmid d$,
\item[(3)] $C_{d+}(\mathbb{Q}_{p_i}) \neq \emptyset \Leftrightarrow (\frac{-2D/(d p_i^{e^i-1})}{p_i}) \ =\ 1 , \forall  p_i \mid d$.
\end{enumerate}
}

\lemm{\label{lem4}
	Let $C_{d+}'$ be defined as above. Then\vspace{1.5ex}

\begin{enumerate}
\item[(1)] $C_{d+}'(\mathbb{Q}_2) \neq \emptyset \Leftrightarrow d \equiv 1( \mod 8) \ \ or \ d + 2D/d\equiv\ 1\ ( \mod\ 8)  $, \vspace{1ex}
\item[(2)] $C_{d+}'(\mathbb{Q}_{p_i}) \neq \emptyset \Leftrightarrow (\frac{d}{p_i}) \ =\ 1 , \forall p_i \nmid d$,
\item[(3)] $C_{d+}'(\mathbb{Q}_{p_i}) \neq \emptyset \Leftrightarrow (\frac{2D/(dp_i^{e_i-1})}{p_i}) \ =\ 1 , \forall  p_i \mid  d$.
\end{enumerate}
}
Since the Legendre symbol has the properties $(\frac{q^2}{p}) = 1$ and $p^2 \equiv 1 (\mod 8)$ for any $q \in \QQ$ and $p$ odd prime satisfying $\gcd (p,q) = 1$, there is no difference between $e_i = 1$ or $3$ in Lemmas \ref{lem1} to \ref{lem4}. Thus we assume $e_i = 1$, $\forall 1 \le i \le n$, or in other words, $D$ is squrefree, in the following part.


\section{Proof of the relation}\label{sec_pr}

From the Lemmas \ref{lem1} to \ref{lem4}, we see that whether an element is in $s_{\pm}(\phi)$ and $s'_{\pm}(\phi)$ or not can be determined by the congruence conditions and the Legendre symbol conditions. In this section, we give the proof of the main theorems.
In our proof, we show that the congruence conditions here can switch to some other Legendre symbol conditions. With the property of Legendre symbol we mentioned in section \ref{sec_2}, we will use $d$ instead of $\widetilde{d} = d \mod  \QQ^{\times 2}$ for convenience. Similarly, when we write $d_1d_2$ in the following statement, actually we talk about $\widetilde{d_1d_2} = d_1*d_2 \mod  \QQ^{\times 2}$.

For any $d \in \mathbb{Q}_{2D}$, let $g(d) = (g_{1}(d),g_{2}(d),\cdots,g_{n}(d))$. Similarly, for any $d' \in \mathbb{Q}_{-2D}$,
let $f(d') = (f_{1}(d'),f_{2}(d'),\cdots,f_{n}(d'))$, where
\begin{equation}\label{eq:array}
g_{i}(d) =\left\{
\begin{array}{ll}
(\frac{d}{p_i})        &\forall p_{i} \nmid d,\\
(\frac{2D/d}{p_i})      &\forall p_{i} \mid d.\\
\end{array}
\right.
\hspace{2em}
f_{i}(d') =\left\{\begin{array}{ll}
(\frac{d'}{p_i})        &\forall p_{i} \nmid d',\\
(\frac{-2D/d'}{p_i})      &\forall p_{i} \mid d'.\\
\end{array}\right.
\end{equation}
Particularly, we have $f_{i}(-1) = (\frac{-1}{p_i})=(-1)^{\frac{p_i - 1}{2}}$, $f_{i}(-D) = (\frac{2}{p_i}) = (-1)^{\frac{p_i^2 - 1}{8}}$, and $f_{i}(p_j) = (\frac{p_j}{p_i})$. It follows that 
\begin{equation} \label{3.2}
  f_{i}(p_j)f_{j}(p_i) =(\frac{p_j}{p_i})(\frac{p_i}{p_j}) = (-1)^{\frac{ p_i - 1}{2}\cdot \frac{ p_j - 1}{2}}.  
\end{equation}

Besides, we give the  following lemma to show that $g$ and $f$ are group homomorphisms:

\lemm{\label{lemma_1} For any $d_1,d_2 \in \QQ_{2D}$, and $d'_1,d'_2 \in \QQ_{-2D}$,
$$g(d_1)g(d_2) = g(d_1d_2),\  \forall d_1,d_2 \in \mathbb{Q}_{2D},$$
$$f(d'_1)f(d'_2) = f(d'_1d'_2),\ \forall d'_1,d'_2 \in \mathbb{Q}_{-2D}.$$
}
\pf{For any $d_1,d_2 \in \mathbb{Q}_{2D}$, we only need to verify it in $i$-th component. Now we divide the proof into four cases.

(1) For any $i$ such that $p_i \nmid d_1, d_2$, since $(\frac{d^2}{p_i}) = 1$,
it is obvious that
 $$g_{i}(d_1)g_{i}(d_2) =(\frac{d_1}{p_i})(\frac{d_2}{p_i}) = (\frac{d_1d_2}{p_i}) = g_{i}(d_1d_2).$$

(2) For any $i$ such that $p_i \mid d_1, d_2,$ we have $p_i \nmid {d_1d_2}$, then
$$g_{i}(d_1)g_{i}(d_2) =(\frac{2D/d_1}{p_i})(\frac{2D/d_2}{p_i}) = (\frac{4D^2/d_1d_2}{p_i}) =(\frac{d_1d_2 \mod \QQ^{\times 2}}{p_i}) = g_{i}(d_1d_2).$$

(3) For any $i$ such that $p_i \mid d_1, p_i \nmid d_2,$ we have $p_i \mid {d_1d_2}$.
From case(2) we get
$$g_{i}(d_1)g_{i}(d_1d_2) = g_{i}(d_1d_1d_2) =  g_{i}(d_2),$$
$$g_{i}(d_1d_2) =g^{2}_{i}(d_1)g_{i}(d_1d_2) = g_{i}(d_1)g_{i}(d_2).$$

(4) Similarly, we get the conclusion when  $p_i \nmid d_1, p_i \mid d_2$.

To summarize,
$$g_{i}(d_1)g_{i}(d_2) = g_{i}(d_1d_2).$$

Similarly, we have
$$f_{i}(d'_1)f_{i}(d'_2) = f_{i}(d'_1d'_2) .$$
}
In order to simplify the proof, we introduce $x(d) $ and $ y(d')$ here.

\defi{
Define $x(d) = (x_{1}(d),x_{2}(d),\cdots,x_{n}(d))$ and $y(d') = (y_{1}(d'),y_{2}(d'),\cdots,y_{n}(d')) \in \FF_2^n$,  such that $g_{i}(d) = (-1)^{x_{i}(d)}, f_{i}(d') = (-1)^{y_{i}(d')}, \ \forall d \in \QQ_{2D}, d'\in \QQ_{-2D}$. In other words,
\begin{equation}\label{eq:array2}
x_{i}(d) =\left\{
\begin{array}{ll}
0       &\forall g_{i}(d)  = 1,\\
1      &\forall g_{i}(d) = -1.\\
\end{array}
\right.
\hspace{2em}
y_{i}(d') =\left\{\begin{array}{ll}
0       &\forall f_{i}(d')  = 1,\\
1      &\forall f_{i}(d') = -1.\\
\end{array}\right.
\end{equation}
}

Then by Lemma \ref{lemma_1}, we have
$$x_{i}(d_1) + x_{i}(d_2) = x_{i}(d_1d_2),  \forall d_1,d_2 \in \mathbb{Q}_{2D},$$
and
$$y_{i}(d'_1)+ y_{i}(d'_2) = y_{i}(d'_1d'_2),  \forall d'_1,d'_2 \in \mathbb{Q}_{-2D}.$$
Thus $x$ and $y$ are group homomorphisms as well.

By the properties of Legendre symbols and the definition of $x_{i}(d_1)$ and $y_{i}(d'_1)$, the following proposition is straightforward. For example, the third one is straight forward corollary by \ref{3.2}.
\prop{\label{prop1} We have

(1) $x_{i}(d) = y_{i}(d),\quad \forall d \in \QQ_{2D}, 1 \leq i \leq n,$ such that $ p_i \nmid d$,

(2) $x_{i}(d) = y_{i}(d) + y_{i}(-1) =  y_{i}(-d),\quad \forall d \in \QQ_{2D}, 1 \leq i \leq n$, such that $p_i \mid d$,

(3) $x_{j}(p_i) + x_{i}(p_j) = y_{j}(p_i) + y_{i}(p_j) =y_{i}(-1)y_{j}(-1) ,\quad \forall 1 \leq i \neq j \leq n$.
}

\lemm{\label{lemma_2} Assume  $d = p_{i_1}p_{i_2} \cdots p_{i_k}$, then we have the following results:
$$ y_{i_1}(-1)+ \cdots + y_{i_k}(-1) = 0 \Leftrightarrow (\frac{-1}{p_{i_1}})\cdots (\frac{-1}{p_{i_k}}) = 1 \Leftrightarrow d \equiv 1,5 \mod 8,$$
$$y_{i_1}(-D)+ \cdots + y_{i_k}(-D) = 0 \Leftrightarrow (\frac{2}{p_{i_1}})\cdots (\frac{2}{p_{i_k}}) = 1 \Leftrightarrow d \equiv 1,7 \mod 8,$$
$$y_{i_1}(D)+ \cdots + y_{i_k}(D) = 0 \Leftrightarrow (\frac{-2}{p_{i_1}})\cdots (\frac{-2}{p_{i_k}})= 1 \Leftrightarrow d \equiv 1,3 \mod 8.$$
}




\pf{ Let $n_{i} = \# \{ p \mid p \text{ is a factor of } d,\  p \equiv i (\mod 8)\},\  i = 1,3,5,7$. Since $3^2 \equiv 5^2 \equiv 7^2 \equiv 1 \mod 8$, we can know the congruent number of d module 8 through the result for  $n_i$ module 2.
It's easy to know that
$$
\begin{aligned}
d \equiv 1 (\mod 8) & \Leftrightarrow n_3  \equiv n_5 \equiv n_7 (\mod 2), \\
d \equiv 3 (\mod 8) & \Leftrightarrow n_5 \equiv n_7 \equiv n_3+1(\mod 2), \\
d \equiv 5 (\mod 8) & \Leftrightarrow n_3 \equiv n_7 \equiv n_5+1 (\mod 2), \\
d \equiv 7 (\mod 8) & \Leftrightarrow n_3 \equiv n_5 \equiv n_7 +1(\mod 2). \\
\end{aligned}
$$
With the fact that
$$
\begin{aligned}
y_{i}(-1) = 0   &\Leftrightarrow \ (\frac{-1}{p_i}) = 1 &\Leftrightarrow p_i \equiv 1,5 (\mod 8), \\
y_{i}(-D) = 0   &\Leftrightarrow\ \ (\frac{2}{p_i}) = 1 &\Leftrightarrow p_i \equiv 1,7 (\mod 8), \\
y_{i}(D) = 0   &\Leftrightarrow\ (\frac{-2}{p_i}) = 1 &\Leftrightarrow p_i \equiv 1,3 (\mod 8),  \\
\end{aligned}
$$
we can get the result easily.
}

%%%%%%%%%%%%%%%%%prepare for proof of E-
\thm{\label{thm1}Let $d = p_{i_1}p_{i_2} \cdots p_{i_k}$, $1 \le i_1 < i_2 \cdots < i_k \le n$. Then the condition $d \in s_-(\phi)$ is equivalent to $ \sum\limits_{j=1}^k y_{i_j}(p_m) = 0,\quad \forall 1 \le m \le n+1$.}
\pf{From Lemma \ref{lem1} we know
$$
\begin{aligned}
d = p_{i_1}p_{i_2} \cdots p_{i_k}  \in s_-(\phi) & \Leftrightarrow d \equiv 1 (\mod 8) \text{ and }  g_{m}(d) = 1, \forall 1\leq m \leq n, \\
&\Leftrightarrow d \equiv 1 (\mod 8) \text{ and } x_{m}(d) = 0, \forall 1\leq m \leq n .
\end{aligned}
$$
From Lemma \ref{lemma_2}, we have $d \equiv 1 \mod 8 \Leftrightarrow \delta(d) = \sum\limits_{j = 1}^k y_{i_j}(-1) = 0,$ and $\sum\limits_{j = 1}^k y_{i_j}(D) = 0$.
If we focus on the difference between $x_m(d)$ and the sum of $y_{i_j}(p_m)$, we have:
when $m \neq i_j, \ \forall 1 \le j \le k$:
	$$
	\begin{aligned}
	x_m(d) - \sum\limits_{j=1}^k y_{i_j}(p_m) & = \sum\limits_{j=1}^k (x_m(p_{i_j}) - y_{i_j}(p_m))\\
	& = \sum\limits_{j=1}^k (x_m(p_{i_j}) - x_{i_j}(p_m))\\
	& = \sum\limits_{j=1}^k (x_m(p_{i_j}) + x_{i_j}(p_m))\\
	& = \sum\limits_{j=1}^k (y_m(-1) y_{i_j}(-1))\\
	& = y_m(-1)\delta(d),
	\end{aligned}
	$$
when $ m = i_j, \ \exists 1 \le j \le k $:
	$$
	\begin{aligned}
	x_m(d) - \sum\limits_{j=1}^k y_{i_j}(p_m) & = \sum\limits_{j=1}^k (x_m(p_{i_j}) - y_{i_j}(p_m))\\
	& = \sum\limits_{1 \le j \le n ,m \neq i_j} (x_m(p_{i_j}) - x_{i_j}(p_m)) + y_{m}(-1)\\
	& = \sum\limits_{1 \le j \le n ,m \neq i_j} (x_m(p_{i_j}) + x_{i_j}(p_m))+ y_{m}(-1)\\
	& = \sum\limits_{1 \le j \le n ,m \neq i_j} (y_m(-1) y_{i_j}(-1)) + y_{m}(-1)\\
	& = y_m(-1)(\delta(d) - y_m(-1) + 1).
	\end{aligned}
	$$
It follows that
$$
\begin{aligned}
x_m(d) = \sum\limits_{j=1}^k y_{i_j}(p_m),\  \forall 1 \le m \le n & \Leftrightarrow \delta(d) = 0 \text{ or } y_m(-1) = 0,\  \forall 1 \le m \le n \\
& \Leftrightarrow \delta(d) = 0,
\end{aligned}
$$
and $\delta(d) = 0$ can be deduced from the conditions $d \equiv 1 \mod 8$ and $\sum\limits_{j=1}^k y_{i_j}(d') = 0, \ \forall d' \in \QQ_{-2D}$. Thus under the condition  $d \equiv 1 \mod 8$ or $\sum\limits_{j=1}^k y_{i_j}(p_{n+1}) = 0$, we have $x_m(d) = 0 \Leftrightarrow \sum\limits_{j=1}^k y_{i_j}(p_m) = 0,\  \forall 1 \le m \le n$.
And if $\sum\limits_{j=1}^k y_{i_j}(p_m) = 0,\  \forall 1 \le m \le n$, we have $$\sum\limits_{j = 1}^k y_{i_j}(D) = \sum\limits_{j = 1}^k \sum\limits_{m = 1}^n y_{i_j}(p_m) = \sum\limits_{m = 1}^n \sum\limits_{j = 1}^k y_{i_j}(p_m)= 0 .$$
Thus $d \in s_-(\phi)$ is equivalent to $ \sum\limits_{j=1}^k y_{i_j}(p_m) = 0, \forall 1 \le m \le n+1$.
}


\thm{\label{thm2}For any $d' \in \QQ_{-2D}$, $d' = p_{i_1}p_{i_2}\cdots p_{i_k} , \ 1\le i_1 < i_2 < \cdots < i_k \le n+1, \ k \ge 1$, we have $d' \in s_-'(\phi)$ if and only if $y_{j}(d') =  y_{j}(p_{i_1}) + y_{j}(p_{i_2}) + \cdots + y_{j}(p_{i_k}) = 0 \ \forall 1\le j \le n$.}

\pf{
	From Lemma \ref{lem2}, we have
$$
\begin{aligned}
d' = p_{i_1}p_{i_2} \cdots p_{i_k}  \in s'_-(\phi) & \Leftrightarrow d' \equiv 1 (\mod 8) \text{ or } d' - 2D/d' \equiv 1 (\mod 8)\text{ and }  f_{m}(d') = 1, \forall 1\leq m \leq n, \\
& \Leftrightarrow d' \equiv 1 (\mod 8) \text{ or } d' - 2D/d' \equiv 1 (\mod 8)\text{ and } y_{m}(d') = 0, \forall 1\leq m \leq n .
\end{aligned}
$$
Thus the forward direction is obvious, let us turn to the backward direction.

(\romannumeral1) If $i_k \le n$, without loss of generality, we may assume $d' = p_{1}p_{2}\cdots p_{k}$. We have
	$$ \sum\limits_{i = 1}^{k} y_{j}(p_i) = 0,\ \forall  k < j \le n,$$
	and
	$$  y_{i}(D) +\sum\limits_{j = k+1}^{n} y_{i}(p_j) = 0, \   \forall 1 \le i \le k. $$
	Then we get
	$$
	\begin{aligned}
	0 &= \sum\limits_{i = 1}^{k} (y_{i}(D) +\sum\limits_{j = k+1}^{n} y_{i}(p_j) ) + \sum\limits_{j = k+1}^{n}\sum\limits_{i = 1}^{k}y_{j}(p_i)  \\
	 &= \sum\limits_{j = k+1}^{n}\sum\limits_{i = 1}^{k}(y_{j}(p_i) + y_{i}(p_j)) + \sum\limits_{i = 1}^{k} y_{i}(D)\\
	 &= \sum\limits_{j = k+1}^{n}\sum\limits_{i = 1}^{k} y_{i}(-1)y_{j}(-1) + \sum\limits_{i = 1}^{k} y_{i}(D)\\
	 &= (\sum\limits_{j = k+1}^{n} y_{j}(-1))(\sum\limits_{i = 1}^{k}y_{i}(-1)) + \sum\limits_{i = 1}^{k} y_{i}(D).
	\end{aligned}
	$$

	 Let $e = p_{k+1}p_{k+2}\cdots p_{n}$ , $de = D$. Then we distinguish among three cases.

	Case 1: $\sum\limits_{i = 1}^{k}y_{i}(-1) = \sum\limits_{i = 1}^{k} y_{i}(D) = 0 $. By lemma\ref{lemma_2}, in this case we have $d'$ satisfies $d' \equiv 1,5 (\mod 8)$ and $d' \equiv 1,3 (\mod 8)$, so $d' \equiv 1 (\mod 8) $.

	Case 2: $\sum\limits_{i = 1}^{k}y_{i}(-1) = 1, \sum\limits_{j = k+1}^{n} y_{j}(-1)  = \sum\limits_{i = 1}^{k} y_{i}(D) = 0 $. By lemma\ref{lemma_2}, in this case we have $d'$ satisfies $d' \equiv 3,7 (\mod 8)$ and $d' \equiv 1,3 (\mod 8)$, $e$ satisfies  $e \equiv 1,5 (\mod 8)$. Hence $d' - 2D/d' \equiv d - 2e \equiv 1 (\mod 8)$.

	Case 3: $\sum\limits_{j = k+1}^{n} y_{j}(-1) = \sum\limits_{i = 1}^{k}y_{i}(-1) = \sum\limits_{i = 1}^{k} y_{i}(D) = 1$.  By lemma\ref{lemma_2}, in this case we have $d'$ satisfies  $d' \equiv 3,7 (\mod 8)$ and $d' \equiv 5,7 (\mod 8)$, $e$ satisfies  $e \equiv 3,7 (\mod 8)$. Hence $d' - 2D/d' \equiv d' - 2e \equiv 1 (\mod 8)$.

	(\romannumeral2) If $i_k = n+1$, without loss of generality, assume $d' = -p_{1}p_{2}\cdots p_{k-1}$. We have
	$$ y_{j}(-1) + \sum\limits_{i = 1}^{k-1} y_{j}(p_i) = 0, \ \forall  k \le j \le n,$$
	and
	$$  y_{i}(-D) +\sum\limits_{j = k}^{n} y_{i}(p_j) = 0 , \   \forall 1 \le i \le k-1. $$
Then we get
	$$
	\begin{aligned}
	0 &= \sum\limits_{i = 1}^{k-1} (y_{i}(-D) +\sum\limits_{j = k}^{n} y_{i}(p_j) ) + \sum\limits_{j = k}^{n}(\sum\limits_{i = 1}^{k} y_{j}(p_i) + y_{j}(-1))  \\
	 &= \sum\limits_{j = k}^{n}\sum\limits_{i = 1}^{k-1}(y_{j}(p_i) + y_{i}(p_j)) + \sum\limits_{i = 1}^{k-1} y_{i}(-D) + \sum\limits_{j = k}^{n} y_{j}(-1)\\
	 &= \sum\limits_{j = k}^{n}\sum\limits_{i = 1}^{k-1} y_{i}(-1)y_{j}(-1) + \sum\limits_{i = 1}^{k-1} y_{i}(-D)+ \sum\limits_{j = k}^{n} y_{j}(-1)\\
	&= (\sum\limits_{j = k}^{n} y_{j}(-1))(1+\sum\limits_{i = 1}^{k-1}y_{i}(-1)) + \sum\limits_{i = 1}^{k-1} y_{i}(-D).
	\end{aligned}
	$$
	Let $e = p_{k}p_{k+1}\cdots p_{n}$ , $de = -D$. Also, we distinguish among three cases.

	Case 1: $\sum\limits_{i = 1}^{k-1}y_{i}(-1) = 1$ and  $\sum\limits_{i = 1}^{k} y_{i}(-D) = 0 $. By lemma\ref{lemma_2}, in this case we have $-d' \equiv 3,7 (\mod 8)$ and $d' \equiv 1,7 (\mod 8)$, so $-d' \equiv 7 (\mod 8) $ and $d' \equiv 1 (\mod 8) $

	Case 2: $\sum\limits_{i = 1}^{k-1}y_{i}(-1) = 0, \sum\limits_{j = k}^{n} y_{j}(-1)  = \sum\limits_{i = 1}^{k-1} y_{i}(-D) = 0 $. By lemma\ref{lemma_2}, in this case we have $-d' \equiv 1,5 (\mod 8)$ and $-d' \equiv 1,7 (\mod 8)$, $e$ satisfies  $e \equiv 1,5 (\mod 8)$. Hence $d' - 2D/d' \equiv d + 2e \equiv 1 (\mod 8)$.

	Case 3: $\sum\limits_{i = 1}^{k-1}y_{i}(-1) = 0$, $\sum\limits_{j = k}^{n} y_{j}(-1) =  \sum\limits_{i = 1}^{k} y_{i}(D) = 1$. By lemma\ref{lemma_2}, in this case we have $-d' \equiv 1,5 (\mod 8)$ and $-d' \equiv 3,5 (\mod 8)$, $e$ satisfies  $e \equiv 3,7 (\mod 8)$. Hence $d' - 2D/d' \equiv d + 2e \equiv 1 (\mod 8)$.

	To sum up, $d' \in s_-'(\phi)$ is equivalent to $y_{j}(d') = 0 \ \forall 1\le j \le n$.
}
%%%%%%%%%%%%%%%%%prepare for proof of E+
\thm{\label{thm3}For a fixed $d_1 = p_{i_1}p_{i_2} \cdots p_{i_k}, 1\le i_1 < \cdots < i_k \le n$, then one and only one $d = \sigma d_1 \in s_+(\phi)$ is equivalent to $ \sum\limits_{j=1}^k x_{i_j}(p_m) = 0, \forall 1 \le m \le n$.
}
\pf{From Lemma \ref{lem1} we know
$$
\begin{aligned}
d = \sigma p_{i_1}p_{i_2} \cdots p_{i_k}  \in s_+(\phi) & \Leftrightarrow d \equiv 1 (\mod 8) \ and \  f_{m}(d) = 1, \forall 1\leq m \leq n, \\
&\Leftrightarrow d \equiv 1 (\mod 8) \ and  \ y_{m}(d) = 0, \forall 1\leq m \leq n .
\end{aligned}
$$
By Lemma \ref{lemma_2}, we have $d_1 \equiv \pm 1 \mod 8 \Leftrightarrow \sum\limits_{j = 1}^k x_{i_j}(D) = 0$.
If we focus on the difference between $y_m(d)$ and the sum of $x_{i_j}(p_m)$, we have:
when $m \neq i_j, \ \forall 1 \le j \le k$:
	$$
	\begin{aligned}
	y_m(d) - \sum\limits_{j=1}^k x_{i_j}(p_m) & = y_m(\sigma) + \sum\limits_{j=1}^k (y_m(p_{i_j}) - x_{i_j}(p_m))\\
	& =  y_m(\sigma) +\sum\limits_{j=1}^k (x_m(p_{i_j}) - x_{i_j}(p_m))\\
	& =  y_m(\sigma) +\sum\limits_{j=1}^k (x_m(p_{i_j}) + x_{i_j}(p_m))\\
	& =  y_m(\sigma) +\sum\limits_{j=1}^k (y_m(-1) y_{i_j}(-1))\\
	& =  y_m(\sigma) +y_m(-1)\delta(d_1),
	\end{aligned}
	$$
when $ m = i_j, \ \exists 1 \le j \le k $:
	$$
	\begin{aligned}
	y_m(d) - \sum\limits_{j=1}^k x_{i_j}(p_m) & =  y_m(\sigma) + \sum\limits_{j=1}^k (y_m(p_{i_j}) - x_{i_j}(p_m))\\
	& =  y_m(\sigma) + \sum\limits_{1 \le j \le n ,m \neq i_j} (x_m(p_{i_j}) - x_{i_j}(p_m)) + y_{m}(-1)\\
	& = y_m(\sigma) +  \sum\limits_{1 \le j \le n ,m \neq i_j} (x_m(p_{i_j}) + x_{i_j}(p_m))+ y_{m}(-1)\\
	& =  y_m(\sigma) + \sum\limits_{1 \le j \le n ,m \neq i_j} (y_m(-1) y_{i_j}(-1)) + y_{m}(-1)\\
	& =  y_m(\sigma) + y_m(-1)(\delta(d_1) - y_m(-1) + 1).
	\end{aligned}
	$$

First we prove the forward direction.
In fact if $\sigma = 1$, then by $d = \sigma d_1 \equiv 1 \mod 8$, we have $y_m(\sigma) = \delta(d_1) = 0$. As a result, $0 = y_m(d) = \sum\limits_{j=1}^k x_{i_j}(p_m),\  \forall 1\le m \le n$. Otherwise, we know $\sigma = -1$,
similarly, we have $y_m(\sigma) =  y_m(-1)$ and $\delta(d_1) = 1$. We can also get $0 = y_m(d) = \sum\limits_{j=1}^k x_{i_j}(p_m),\  \forall 1\le m \le n$. Our result follows.

For the backward direction, if $\sum\limits_{j=1}^k x_{i_j}(p_m) = 0, \forall 1 \le m \le n$, we have:
$$0 = \sum\limits_{m=1}^n \sum\limits_{j=1}^k x_{i_j}(p_m) = \sum\limits_{j=1}^k\sum\limits_{m=1}^n x_{i_j}(p_m) =\sum\limits_{j=1}^k x_{i_j}(D) = \sum\limits_{j=1}^k y_{i_j}(-D). $$
By Lemma \ref{lemma_2}, we have $d_1 \equiv 1,7 (\mod 8)$. Select $d = d_1$ when $d_1 \equiv 1 (\mod 8)$, and $d = -d_1$ when $d_1 \equiv 7 (\mod 8)$. Then $d \equiv 1 (\mod 8)$. Also we have $y_m(\sigma) = 0,\  \delta(d_1) = 0$ or $y_m(\sigma) =  y_m(-1),\ \delta(d_1) = 1$. Thus $0 = \sum\limits_{j=1}^k x_{i_j}(p_m)= y_m(d),\  \forall 1\le m \le n$. Since $d \equiv 1 (\mod 8)$ and $y_m(d) = 0, \forall 1\le m \le n$, we know $d \in s_+(\phi)$.
}

\thm{\label{thm4}For any $d \in \QQ_{2D}$, $d = p_{i_1}p_{i_2}\cdots p_{i_k} , \ 1\le i_1 < i_2 < \cdots < i_k \le n, \ k \ge 1$, we have $d \in s_+'(\phi)$ if and only if $x_{j}(d) =  x_{j}(p_{i_1}) + x_{j}(p_{i_2}) + \cdots + x_{j}(p_{i_k}) = 0 ,\ \forall 1\le j \le n$.}

\pf{
	From Lemma\ref{lem2}, we have
$$
\begin{aligned}
d = p_{i_1}p_{i_2} \cdots p_{i_k}  \in s_+'(\phi) & \Leftrightarrow d \equiv 1 (\mod 8) \text{ or } d + 2D/d \equiv 1 (\mod 8)\text{, and }  g_{m}(d) = 1, \forall 1\leq m \leq n, \\
& \Leftrightarrow d \equiv 1 (\mod 8) \text{ or } d + 2D/d \equiv 1 (\mod 8)\text{, and } x_{m}(d) = 0, \forall 1\leq m \leq n .
\end{aligned}
$$
Thus the forward direction is obvious, let us turn to the backward direction. Without loss of generality, assume $d = p_{1}p_{2}\cdots p_{k}$. We have
	$$ \sum\limits_{i = 1}^{k} x_{j}(p_i) = 0,\ \forall  k < j \le n,$$
	and
	$$  x_{i}(D) +\sum\limits_{j = k+1}^{n} x_{i}(p_j) = 0, \   \forall 1 \le i \le k. $$
	Then we get
	$$
	\begin{aligned}
	0 &= \sum\limits_{i = 1}^{k} (x_{i}(D) +\sum\limits_{j = k+1}^{n} x_{i}(p_j) ) + \sum\limits_{j = k+1}^{n}\sum\limits_{i = 1}^{k}x_{j}(p_i)  \\
	 &= \sum\limits_{j = k+1}^{n}\sum\limits_{i = 1}^{k}(y_{j}(p_i) + y_{i}(p_j)) + \sum\limits_{i = 1}^{k} x_{i}(D)\\
	 &= \sum\limits_{j = k+1}^{n}\sum\limits_{i = 1}^{k} y_{i}(-1)y_{j}(-1) + \sum\limits_{i = 1}^{k} y_{i}(-D)\\
	 &= (\sum\limits_{j = k+1}^{n} y_{j}(-1))(\sum\limits_{i = 1}^{k}y_{i}(-1)) + \sum\limits_{i = 1}^{k} y_{i}(-D)
	\end{aligned}
	$$

	 Let $e = p_{k+1}p_{k+2}\cdots p_{n}$ , $de = D$. There are three cases to check.

	Case 1: $\sum\limits_{i = 1}^{k}y_{i}(-1) = \sum\limits_{i = 1}^{k} y_{i}(-D) = 0 $. By Lemma \ref{lemma_2}, in this case we have $d$ satisfies $d \equiv 1,5 (\mod 8)$ and $d \equiv 1,7 (\mod 8)$, so $d \equiv 1 (\mod 8) $.

	Case 2: $\sum\limits_{i = 1}^{k}y_{i}(-1) = 1, \sum\limits_{j = k+1}^{n} y_{j}(-1)  = \sum\limits_{i = 1}^{k} y_{i}(-D) = 0 $. By Lemma \ref{lemma_2}, in this case we have $d$ satisfies $d \equiv 3,7 (\mod 8)$ and $d \equiv 1,7 (\mod 8)$, $e$ satisfies  $e \equiv 1,5 (\mod 8)$. Hence $d + 2D/d \equiv d + 2e \equiv 1 (\mod 8)$.

	Case 3: $\sum\limits_{j = k+1}^{n} y_{j}(-1) = \sum\limits_{i = 1}^{k}y_{i}(-1) = \sum\limits_{i = 1}^{k} y_{i}(-D) = 1$.  By Lemma \ref{lemma_2}, in this case we have $d$ satisfies  $d \equiv 3,7 (\mod 8)$ and $d \equiv 3,5 (\mod 8)$, $e$ satisfies  $e \equiv 3,7 (\mod 8)$. Hence $d + 2D/d \equiv d + 2e \equiv 1 (\mod 8)$.

	To sum up, $d \in s_+'(\phi)$ is equivalent to $x_{j}(d) = 0 \ \forall 1\le j \le n$.
}

\defi{\label{def3}Let $l(1) = 0$, and $l(d) = \max\{1 \le i \le n\mid p_i \text{ divides } d\}$, for any $d \in \QQ_{-2D}$, $d \neq 1$. Similarly, let $l'(1) = 0$, $p_{n+1} = -1, $ and $l'(d) = i_k,$ for $d' = p_{i_1}p_{i_2}\cdots p_{i_k} \in \QQ_{-2D}$, $1 \le i_1 < i_2 < \cdots < i_k \le n+1 ,$ $ k \geq 1$. }

In the following, we give the proof of Theorem \ref{thm_1}.
\thm{ \label{thm_n1} Let $E_-$, $E_-'$ be the elliptic curves defined above. Then     $$2^{-t_{2D}} = T_{2D} = \# S^{(\phi)}(E_-)/\# S^{(\phi)}(E'_{-}) = 1/2.$$ 
}
\pf{
 Let us define a matrix $Y = [m'_{i,j}]_{(n+1)\times n}$, where $m'_{i,j} = y_{j}(p_i)$.
 Since $y$ is a homomorphism, $Y$ encodes the information of $\{y_{i}(d)\}$. Let $r'_{i},1 \le i \le n+1$ be the row vectors of $Y$ and $c'_{j},1\le j \le n$ the column ones as below.
\begin{equation}\label{eq:array3}
\bordermatrix{%
       & c'_1       & c'_2     &\cdots     &c'_n\cr
r'_1    &y_{1}(p_1) & y_{2}(p_1) & \cdots  &y_{n}(p_1)\cr
%u_2    & ?         & ?       &\cdots     & ?\cr
\vdots & \vdots    &\vdots   &\cdots     &\vdots\cr
r'_n 		& y_{1}(p_n) & y_{2}(p_n) & \cdots  &y_{n}(p_n)\cr
r'_{n+1} 		&y_{1}(p_{n+1}) & y_{2}(p_{n+1}) & \cdots  &y_{n}(p_{n+1})
}
%\hspace{2em}
\begin{array}{c}
      \longleftarrow  \text{corresponding to $d' = p_1$} \\
%      \longleftarrow  \text{corresponding to $d' = p_2$} \\
		\vdots  \\
	  \longleftarrow  \text{corresponding to $d' = p_n$}\\
	  \longleftarrow  \text{corresponding to $d' = -1$}\\
    \end{array}
\end{equation}

Since $y$ is a homomorphism, to fulfill the condition $y_{i_1}(d')+y_{i_2}(d')+\cdots +y_{i_k}(d') = 0, \ \forall d' \in \QQ_{-2D}$ in Theorem \ref{thm1}, all we need is $y_{i_1}(d')+y_{i_2}(d')+\cdots +y_{i_k}(d') = 0, \quad \forall d' = p_{i}, \quad 1 \le i \le n+1,$ which is equivalent to the linear dependence condition $c'_{i_1}+c'_{i_2}+\cdots+c'_{i_k} = 0$.

Similarly, for a fixed $d' = p_{i_1}p_{i_2}\cdots p_{i_k} \neq 1, 1\le i_1 < i_2 < \cdots < i_k \le n+1$,  from Theorem \ref{thm2}, we have
$$
	\begin{aligned}
	d' \in s'(\phi) &\Leftrightarrow y(d') = 0 \\
					 &\Leftrightarrow r'_{i_1}+r'_{i_2}+\cdots +r'_{i_k}=0 \quad (y \text{ is a homomorphism}).
	\end{aligned}
$$
 Let $l(d), l'(d')$ be as in the Definition \ref{def3}, $R' = \{i>0 \mid \exists d \in s_-(\phi) , i = l(d) \}$ and $C' = \{i>0 \mid \exists d' \in s'_-(\phi) , i = l'(d') \}$.
 As the rank of $Y$ is a certain number for a given $Y$, we have $\# C' = \# R' + 1 = n+1-\rank(Y)$. Besides, since for any $ 0 \le i \le n$, either
 $$\#\{ d \mid d \in s_-(\phi) ,  l(d) \le i + 1\} = \#\{ d \mid d \in s_-(\phi) , l(d) \le i \}  $$ or
$$\#\{ d \mid d \in s_-(\phi) , l(d) \le i+1 \} = 2 \#\{ d \mid d \in s_-(\phi) , l(d) \le i\} , $$
we have $\# R' = \log_{2}(\# s_-(\phi))$, and similarly, $\# C' = \log_{2}(\# s'_-(\phi))$. In this way, we have  $\# s'_-(\phi) = 2\# s_-(\phi)$ immediately. By the fact $\#  S^{(\phi)}(E_-) = 2 \# s_-(\phi)$ and $\# S^{({\phi})}({E_-'}) = 2 \# s_-'(\phi)$, we have $\# S^{({\phi})}({E_-'}) = 2 \#  S^{(\phi)}(E_-)$. It follows that 
$$2^{-t_{2D}} = T_{2D} = \# S^{(\phi)}(E_-)/\# S^{(\phi)}(E'_{-}) = 1/2.$$
}

Similarly, we have the proof of Theorem \ref{thm_2}.

\thm{ \label{thm_n2} Let $E_+$, $E_+'$ be the elliptic curves defined above. Then     $$2^{-t_{-2D}} = T_{-2D} = \# S^{(\phi)}(E_+)/\# S^{(\phi)}(E'_{+}) = 1.$$
}
\pf{
 Let us define Matrix $X = [m_{i,j}]_{n\times n}$, where $m_{i,j} = x_{j}(p_i)$, which encodes information of $\{x_{i}(d)\}$. Let $r_{i},1 \le i \le n+1$ be the row vectors of $X$ and $c_{j},1\le j \le n$ be the column vectors.
\begin{equation}\label{eq:array4}
\bordermatrix{%
       & c_1       & c_2     &\cdots     &c_n\cr
r_1    &x_{1}(p_1) & x_{2}(p_1) & \cdots  &x_{n}(p_1)\cr
r_2    &x_{1}(p_2) & x_{2}(p_2) & \cdots  &x_{n}(p_2)\cr
\vdots & \vdots    &\vdots   &\cdots     &\vdots\cr
r_n 		& x_{1}(p_n) & x_{2}(p_n) & \cdots  &x_{n}(p_n)
}
%\hspace{2em}
\begin{array}{c}
      \longleftarrow  \text{corresponding to $d = p_1$} \\
     \longleftarrow  \text{corresponding to $d = p_2$} \\
		\vdots  \\
	  \longleftarrow  \text{corresponding to $d = p_n$}\\
    \end{array}
\end{equation}
Similarly, we have the condition $d = p_{i_1}p_{i_2} \cdots p_{i_k}  \in s_+(\phi)$ is equivalent to $r_{i_1}+r_{i_2}+\cdots +r_{i_k}=0$, And one of $ d $ and $-d \in s'_+(\phi)$ is equivalent to $c_{i_1}+c_{i_2}+\cdots +c_{i_k}=0$.

 Let $l(d)$ be as in the Definition \ref{def3}. Define $R= \{i>0\mid \exists d \in s_+(\phi) , i = l(d) \}$ and $C = \{i>0 \mid \exists d' \in s'_+(\phi) , i = l(d') \}$. Same as the Proof of Theorem \ref{thm_1}, we have
$$\# R = \# C = n-\rank(X), \quad \# R = \log_{2}(\# s_+(\phi)) \quad \# C = \log_{2}(\# s'_+(\phi)).$$ Similarly, we have  $\# s'_+(\phi) = \# s_+(\phi)$ immediately. By the fact $\#  S^{(\phi)}(E_+) = 2 \# s_+(\phi)$ and $\# S^{({\phi})}({E_+'}) = 2 \# s_+'(\phi)$, we have $\# S^{({\phi})}({E_+'}) = \#  S^{(\phi)}(E_+)$. Thus
$$2^{-t_{2D}} = T_{2D} = \# S^{(\phi)}(E_-)/\# S^{(\phi)}(E'_{-}) = 1.$$
}

\section{Application}

In this section, we give a method to quickly get the Mordell-Weil rank of part of Elliptic Curves in these forms.

In traditional method, determine the structure of $\phi$-Selmer group needs to traverse $\QQ(S,2)$, Which follows exponential computational complexity.
In the proof of our main theorem, the problem calculating $s_\pm(\phi)$ and $s'_\pm(\phi)$ are translated into the calculation of linear dependence of rows and columns in matrix $X$, $Y$ in $\FF_2$. Moreover, if we want to calculate the sizes of $s_\pm(\phi)$ and $s'_\pm(\phi)$, we only need to compute the ranks of $X$ and $Y$. Since by Gaussian elimination, we can determine the rank and the linear dependence relation of matrix with computational complexity $O(n^3)$ at most. We can quickly figure out the structure of $\phi$-Selmer groups of $E$.

In particular, if $\rank(X)= n$ (resp. $\rank(Y)= n$), we have $\# s'_+(\phi) = \# s_+(\phi) = 1$ (resp. $\# s'_-(\phi) =2 \# s_-(\phi) = 2$). It follows that $S^{(\phi)}(E_+)\cong S^{({\phi})}({E'_+}) \cong \ZZ/2\ZZ$ (and resp. $S^{({\phi})}({E'_-}) \cong (\ZZ/2\ZZ)^2 $, $s^{(\phi)}(E_+)\cong \ZZ/2\ZZ$.)
In addition, from, for example, \cite{gtm106}, we get
$$
\begin{aligned}
r_{E_{2D}} + \dim(\Sha(E_-)[{\phi}]) + \dim(\Sha(E'_-)[{\phi}]) &= \dim(S^{(\phi)}(E_-)) + \dim(S^{({\phi})}({E'_-})) - 2, \\
&= \dim(s_-(\phi)) + \dim(s'_-(\phi)),\\
&= \# R' + \# C',\\
&= 2n + 1 - 2\rank(Y).
\end{aligned}
$$
Similarly,
$$
\begin{aligned}
r_{E_{-2D}} + \dim(\Sha(E_+)[{\phi}]) + \dim(\Sha(E'_+)[{\phi}]) &= \dim(S^{(\phi)}(E_+)) + \dim(S^{({\phi})}({E'_+})) - 2, \\
&= \dim(s_+(\phi)) + \dim(s'_+(\phi)),\\
&= \# R + \# C,\\
&= 2n  - 2\rank(X).
\end{aligned}
$$
The $\dim$ here means $\dim_2$. Therefore, if $\rank(X) = n$, we have $r_{E_{+2D}} = 0$.
%%%%
Besides, by Birch–Stephens \cite{parity}, the parity of $t_A$ is the same as that of the root number of $E_A$. 
$$(-1)^{t_A} = w(E_A).$$ 
Let $L_E(s)$ be the $L$ function associated to E, an elliptic curve over $\QQ$ of conductor $N$ and
$$
\Lambda_{E}(s) = (2\pi)^{-s}N(E)^{\frac{s}{2}}L_{E}(s).
$$
By Modularity theorem [3], we have $\Lambda_{E}(s) = W(E)\Lambda_{E}(2-s)$,
Then by $W(E_{-2D}) = -1$, we have $L_E(1) = 0$. By Birch and Swinnerton-Dyer conjecture \cite{gtm106}, we have $r_{E_{2D}} \geq 1$. Therefore, if $\rank(Y) = n$, we have $r_{E_{2D}} = 1$ and $\Sha_{E}[\phi] = \Sha_{E'}[{\phi}] = \emptyset$. Hence $E_{-2D}(\QQ) \cong E_{tors}\oplus\ZZ \cong \ZZ/2\ZZ \oplus \ZZ$.
Overall, we can quickly sieve some of the elliptic curves in this form which has Mordell-Weil rank $0$ or $1$.










%%%%%%%%%%%%%%%%%%%%%%%%%%%%%%%%%%%%%%%%%%%%%%%%%%%%%%%%%%%%%%%%%%%%%%%
















%%%%%%%%%%%%%%%%%%%%%%%%%%%%%%


%This is the Tate-Shafarevich group:  $\Sha_{E/K}$.

%Here it is as a subscript:  $A_\Sha$.



% \appendix
% \chapter{Supplementary Material}
\label{appendix}

In this appendix, we present supplementary material for the techniques and
experiments presented in the main text.

\section{Baseline Results and Analysis for Informed Sampler}
\label{appendix:chap3}

Here, we give an in-depth
performance analysis of the various samplers and the effect of their
hyperparameters. We choose hyperparameters with the lowest PSRF value
after $10k$ iterations, for each sampler individually. If the
differences between PSRF are not significantly different among
multiple values, we choose the one that has the highest acceptance
rate.

\subsection{Experiment: Estimating Camera Extrinsics}
\label{appendix:chap3:room}

\subsubsection{Parameter Selection}
\paragraph{Metropolis Hastings (\MH)}

Figure~\ref{fig:exp1_MH} shows the median acceptance rates and PSRF
values corresponding to various proposal standard deviations of plain
\MH~sampling. Mixing gets better and the acceptance rate gets worse as
the standard deviation increases. The value $0.3$ is selected standard
deviation for this sampler.

\paragraph{Metropolis Hastings Within Gibbs (\MHWG)}

As mentioned in Section~\ref{sec:room}, the \MHWG~sampler with one-dimensional
updates did not converge for any value of proposal standard deviation.
This problem has high correlation of the camera parameters and is of
multi-modal nature, which this sampler has problems with.

\paragraph{Parallel Tempering (\PT)}

For \PT~sampling, we took the best performing \MH~sampler and used
different temperature chains to improve the mixing of the
sampler. Figure~\ref{fig:exp1_PT} shows the results corresponding to
different combination of temperature levels. The sampler with
temperature levels of $[1,3,27]$ performed best in terms of both
mixing and acceptance rate.

\paragraph{Effect of Mixture Coefficient in Informed Sampling (\MIXLMH)}

Figure~\ref{fig:exp1_alpha} shows the effect of mixture
coefficient ($\alpha$) on the informed sampling
\MIXLMH. Since there is no significant different in PSRF values for
$0 \le \alpha \le 0.7$, we chose $0.7$ due to its high acceptance
rate.


% \end{multicols}

\begin{figure}[h]
\centering
  \subfigure[MH]{%
    \includegraphics[width=.48\textwidth]{figures/supplementary/camPose_MH.pdf} \label{fig:exp1_MH}
  }
  \subfigure[PT]{%
    \includegraphics[width=.48\textwidth]{figures/supplementary/camPose_PT.pdf} \label{fig:exp1_PT}
  }
\\
  \subfigure[INF-MH]{%
    \includegraphics[width=.48\textwidth]{figures/supplementary/camPose_alpha.pdf} \label{fig:exp1_alpha}
  }
  \mycaption{Results of the `Estimating Camera Extrinsics' experiment}{PRSFs and Acceptance rates corresponding to (a) various standard deviations of \MH, (b) various temperature level combinations of \PT sampling and (c) various mixture coefficients of \MIXLMH sampling.}
\end{figure}



\begin{figure}[!t]
\centering
  \subfigure[\MH]{%
    \includegraphics[width=.48\textwidth]{figures/supplementary/occlusionExp_MH.pdf} \label{fig:exp2_MH}
  }
  \subfigure[\BMHWG]{%
    \includegraphics[width=.48\textwidth]{figures/supplementary/occlusionExp_BMHWG.pdf} \label{fig:exp2_BMHWG}
  }
\\
  \subfigure[\MHWG]{%
    \includegraphics[width=.48\textwidth]{figures/supplementary/occlusionExp_MHWG.pdf} \label{fig:exp2_MHWG}
  }
  \subfigure[\PT]{%
    \includegraphics[width=.48\textwidth]{figures/supplementary/occlusionExp_PT.pdf} \label{fig:exp2_PT}
  }
\\
  \subfigure[\INFBMHWG]{%
    \includegraphics[width=.5\textwidth]{figures/supplementary/occlusionExp_alpha.pdf} \label{fig:exp2_alpha}
  }
  \mycaption{Results of the `Occluding Tiles' experiment}{PRSF and
    Acceptance rates corresponding to various standard deviations of
    (a) \MH, (b) \BMHWG, (c) \MHWG, (d) various temperature level
    combinations of \PT~sampling and; (e) various mixture coefficients
    of our informed \INFBMHWG sampling.}
\end{figure}

%\onecolumn\newpage\twocolumn
\subsection{Experiment: Occluding Tiles}
\label{appendix:chap3:tiles}

\subsubsection{Parameter Selection}

\paragraph{Metropolis Hastings (\MH)}

Figure~\ref{fig:exp2_MH} shows the results of
\MH~sampling. Results show the poor convergence for all proposal
standard deviations and rapid decrease of AR with increasing standard
deviation. This is due to the high-dimensional nature of
the problem. We selected a standard deviation of $1.1$.

\paragraph{Blocked Metropolis Hastings Within Gibbs (\BMHWG)}

The results of \BMHWG are shown in Figure~\ref{fig:exp2_BMHWG}. In
this sampler we update only one block of tile variables (of dimension
four) in each sampling step. Results show much better performance
compared to plain \MH. The optimal proposal standard deviation for
this sampler is $0.7$.

\paragraph{Metropolis Hastings Within Gibbs (\MHWG)}

Figure~\ref{fig:exp2_MHWG} shows the result of \MHWG sampling. This
sampler is better than \BMHWG and converges much more quickly. Here
a standard deviation of $0.9$ is found to be best.

\paragraph{Parallel Tempering (\PT)}

Figure~\ref{fig:exp2_PT} shows the results of \PT sampling with various
temperature combinations. Results show no improvement in AR from plain
\MH sampling and again $[1,3,27]$ temperature levels are found to be optimal.

\paragraph{Effect of Mixture Coefficient in Informed Sampling (\INFBMHWG)}

Figure~\ref{fig:exp2_alpha} shows the effect of mixture
coefficient ($\alpha$) on the blocked informed sampling
\INFBMHWG. Since there is no significant different in PSRF values for
$0 \le \alpha \le 0.8$, we chose $0.8$ due to its high acceptance
rate.



\subsection{Experiment: Estimating Body Shape}
\label{appendix:chap3:body}

\subsubsection{Parameter Selection}
\paragraph{Metropolis Hastings (\MH)}

Figure~\ref{fig:exp3_MH} shows the result of \MH~sampling with various
proposal standard deviations. The value of $0.1$ is found to be
best.

\paragraph{Metropolis Hastings Within Gibbs (\MHWG)}

For \MHWG sampling we select $0.3$ proposal standard
deviation. Results are shown in Fig.~\ref{fig:exp3_MHWG}.


\paragraph{Parallel Tempering (\PT)}

As before, results in Fig.~\ref{fig:exp3_PT}, the temperature levels
were selected to be $[1,3,27]$ due its slightly higher AR.

\paragraph{Effect of Mixture Coefficient in Informed Sampling (\MIXLMH)}

Figure~\ref{fig:exp3_alpha} shows the effect of $\alpha$ on PSRF and
AR. Since there is no significant differences in PSRF values for $0 \le
\alpha \le 0.8$, we choose $0.8$.


\begin{figure}[t]
\centering
  \subfigure[\MH]{%
    \includegraphics[width=.48\textwidth]{figures/supplementary/bodyShape_MH.pdf} \label{fig:exp3_MH}
  }
  \subfigure[\MHWG]{%
    \includegraphics[width=.48\textwidth]{figures/supplementary/bodyShape_MHWG.pdf} \label{fig:exp3_MHWG}
  }
\\
  \subfigure[\PT]{%
    \includegraphics[width=.48\textwidth]{figures/supplementary/bodyShape_PT.pdf} \label{fig:exp3_PT}
  }
  \subfigure[\MIXLMH]{%
    \includegraphics[width=.48\textwidth]{figures/supplementary/bodyShape_alpha.pdf} \label{fig:exp3_alpha}
  }
\\
  \mycaption{Results of the `Body Shape Estimation' experiment}{PRSFs and
    Acceptance rates corresponding to various standard deviations of
    (a) \MH, (b) \MHWG; (c) various temperature level combinations
    of \PT sampling and; (d) various mixture coefficients of the
    informed \MIXLMH sampling.}
\end{figure}


\subsection{Results Overview}
Figure~\ref{fig:exp_summary} shows the summary results of the all the three
experimental studies related to informed sampler.
\begin{figure*}[h!]
\centering
  \subfigure[Results for: Estimating Camera Extrinsics]{%
    \includegraphics[width=0.9\textwidth]{figures/supplementary/camPose_ALL.pdf} \label{fig:exp1_all}
  }
  \subfigure[Results for: Occluding Tiles]{%
    \includegraphics[width=0.9\textwidth]{figures/supplementary/occlusionExp_ALL.pdf} \label{fig:exp2_all}
  }
  \subfigure[Results for: Estimating Body Shape]{%
    \includegraphics[width=0.9\textwidth]{figures/supplementary/bodyShape_ALL.pdf} \label{fig:exp3_all}
  }
  \label{fig:exp_summary}
  \mycaption{Summary of the statistics for the three experiments}{Shown are
    for several baseline methods and the informed samplers the
    acceptance rates (left), PSRFs (middle), and RMSE values
    (right). All results are median results over multiple test
    examples.}
\end{figure*}

\subsection{Additional Qualitative Results}

\subsubsection{Occluding Tiles}
In Figure~\ref{fig:exp2_visual_more} more qualitative results of the
occluding tiles experiment are shown. The informed sampling approach
(\INFBMHWG) is better than the best baseline (\MHWG). This still is a
very challenging problem since the parameters for occluded tiles are
flat over a large region. Some of the posterior variance of the
occluded tiles is already captured by the informed sampler.

\begin{figure*}[h!]
\begin{center}
\centerline{\includegraphics[width=0.95\textwidth]{figures/supplementary/occlusionExp_Visual.pdf}}
\mycaption{Additional qualitative results of the occluding tiles experiment}
  {From left to right: (a)
  Given image, (b) Ground truth tiles, (c) OpenCV heuristic and most probable estimates
  from 5000 samples obtained by (d) MHWG sampler (best baseline) and
  (e) our INF-BMHWG sampler. (f) Posterior expectation of the tiles
  boundaries obtained by INF-BMHWG sampling (First 2000 samples are
  discarded as burn-in).}
\label{fig:exp2_visual_more}
\end{center}
\end{figure*}

\subsubsection{Body Shape}
Figure~\ref{fig:exp3_bodyMeshes} shows some more results of 3D mesh
reconstruction using posterior samples obtained by our informed
sampling \MIXLMH.

\begin{figure*}[t]
\begin{center}
\centerline{\includegraphics[width=0.75\textwidth]{figures/supplementary/bodyMeshResults.pdf}}
\mycaption{Qualitative results for the body shape experiment}
  {Shown is the 3D mesh reconstruction results with first 1000 samples obtained
  using the \MIXLMH informed sampling method. (blue indicates small
  values and red indicates high values)}
\label{fig:exp3_bodyMeshes}
\end{center}
\end{figure*}

\clearpage



\section{Additional Results on the Face Problem with CMP}

Figure~\ref{fig:shading-qualitative-multiple-subjects-supp} shows inference results for reflectance maps, normal maps and lights for randomly chosen test images, and Fig.~\ref{fig:shading-qualitative-same-subject-supp} shows reflectance estimation results on multiple images of the same subject produced under different illumination conditions. CMP is able to produce estimates that are closer to the groundtruth across different subjects and illumination conditions.

\begin{figure*}[h]
  \begin{center}
  \centerline{\includegraphics[width=1.0\columnwidth]{figures/face_cmp_visual_results_supp.pdf}}
  \vspace{-1.2cm}
  \end{center}
	\mycaption{A visual comparison of inference results}{(a)~Observed images. (b)~Inferred reflectance maps. \textit{GT} is the photometric stereo groundtruth, \textit{BU} is the Biswas \etal (2009) reflectance estimate and \textit{Forest} is the consensus prediction. (c)~The variance of the inferred reflectance estimate produced by \MTD (normalized across rows).(d)~Visualization of inferred light directions. (e)~Inferred normal maps.}
	\label{fig:shading-qualitative-multiple-subjects-supp}
\end{figure*}


\begin{figure*}[h]
	\centering
	\setlength\fboxsep{0.2mm}
	\setlength\fboxrule{0pt}
	\begin{tikzpicture}

		\matrix at (0, 0) [matrix of nodes, nodes={anchor=east}, column sep=-0.05cm, row sep=-0.2cm]
		{
			\fbox{\includegraphics[width=1cm]{figures/sample_3_4_X.png}} &
			\fbox{\includegraphics[width=1cm]{figures/sample_3_4_GT.png}} &
			\fbox{\includegraphics[width=1cm]{figures/sample_3_4_BISWAS.png}}  &
			\fbox{\includegraphics[width=1cm]{figures/sample_3_4_VMP.png}}  &
			\fbox{\includegraphics[width=1cm]{figures/sample_3_4_FOREST.png}}  &
			\fbox{\includegraphics[width=1cm]{figures/sample_3_4_CMP.png}}  &
			\fbox{\includegraphics[width=1cm]{figures/sample_3_4_CMPVAR.png}}
			 \\

			\fbox{\includegraphics[width=1cm]{figures/sample_3_5_X.png}} &
			\fbox{\includegraphics[width=1cm]{figures/sample_3_5_GT.png}} &
			\fbox{\includegraphics[width=1cm]{figures/sample_3_5_BISWAS.png}}  &
			\fbox{\includegraphics[width=1cm]{figures/sample_3_5_VMP.png}}  &
			\fbox{\includegraphics[width=1cm]{figures/sample_3_5_FOREST.png}}  &
			\fbox{\includegraphics[width=1cm]{figures/sample_3_5_CMP.png}}  &
			\fbox{\includegraphics[width=1cm]{figures/sample_3_5_CMPVAR.png}}
			 \\

			\fbox{\includegraphics[width=1cm]{figures/sample_3_6_X.png}} &
			\fbox{\includegraphics[width=1cm]{figures/sample_3_6_GT.png}} &
			\fbox{\includegraphics[width=1cm]{figures/sample_3_6_BISWAS.png}}  &
			\fbox{\includegraphics[width=1cm]{figures/sample_3_6_VMP.png}}  &
			\fbox{\includegraphics[width=1cm]{figures/sample_3_6_FOREST.png}}  &
			\fbox{\includegraphics[width=1cm]{figures/sample_3_6_CMP.png}}  &
			\fbox{\includegraphics[width=1cm]{figures/sample_3_6_CMPVAR.png}}
			 \\
	     };

       \node at (-3.85, -2.0) {\small Observed};
       \node at (-2.55, -2.0) {\small `GT'};
       \node at (-1.27, -2.0) {\small BU};
       \node at (0.0, -2.0) {\small MP};
       \node at (1.27, -2.0) {\small Forest};
       \node at (2.55, -2.0) {\small \textbf{CMP}};
       \node at (3.85, -2.0) {\small Variance};

	\end{tikzpicture}
	\mycaption{Robustness to varying illumination}{Reflectance estimation on a subject images with varying illumination. Left to right: observed image, photometric stereo estimate (GT)
  which is used as a proxy for groundtruth, bottom-up estimate of \cite{Biswas2009}, VMP result, consensus forest estimate, CMP mean, and CMP variance.}
	\label{fig:shading-qualitative-same-subject-supp}
\end{figure*}

\clearpage

\section{Additional Material for Learning Sparse High Dimensional Filters}
\label{sec:appendix-bnn}

This part of supplementary material contains a more detailed overview of the permutohedral
lattice convolution in Section~\ref{sec:permconv}, more experiments in
Section~\ref{sec:addexps} and additional results with protocols for
the experiments presented in Chapter~\ref{chap:bnn} in Section~\ref{sec:addresults}.

\vspace{-0.2cm}
\subsection{General Permutohedral Convolutions}
\label{sec:permconv}

A core technical contribution of this work is the generalization of the Gaussian permutohedral lattice
convolution proposed in~\cite{adams2010fast} to the full non-separable case with the
ability to perform back-propagation. Although, conceptually, there are minor
differences between Gaussian and general parameterized filters, there are non-trivial practical
differences in terms of the algorithmic implementation. The Gauss filters belong to
the separable class and can thus be decomposed into multiple
sequential one dimensional convolutions. We are interested in the general filter
convolutions, which can not be decomposed. Thus, performing a general permutohedral
convolution at a lattice point requires the computation of the inner product with the
neighboring elements in all the directions in the high-dimensional space.

Here, we give more details of the implementation differences of separable
and non-separable filters. In the following, we will explain the scalar case first.
Recall, that the forward pass of general permutohedral convolution
involves 3 steps: \textit{splatting}, \textit{convolving} and \textit{slicing}.
We follow the same splatting and slicing strategies as in~\cite{adams2010fast}
since these operations do not depend on the filter kernel. The main difference
between our work and the existing implementation of~\cite{adams2010fast} is
the way that the convolution operation is executed. This proceeds by constructing
a \emph{blur neighbor} matrix $K$ that stores for every lattice point all
values of the lattice neighbors that are needed to compute the filter output.

\begin{figure}[t!]
  \centering
    \includegraphics[width=0.6\columnwidth]{figures/supplementary/lattice_construction}
  \mycaption{Illustration of 1D permutohedral lattice construction}
  {A $4\times 4$ $(x,y)$ grid lattice is projected onto the plane defined by the normal
  vector $(1,1)^{\top}$. This grid has $s+1=4$ and $d=2$ $(s+1)^{d}=4^2=16$ elements.
  In the projection, all points of the same color are projected onto the same points in the plane.
  The number of elements of the projected lattice is $t=(s+1)^d-s^d=4^2-3^2=7$, that is
  the $(4\times 4)$ grid minus the size of lattice that is $1$ smaller at each size, in this
  case a $(3\times 3)$ lattice (the upper right $(3\times 3)$ elements).
  }
\label{fig:latticeconstruction}
\end{figure}

The blur neighbor matrix is constructed by traversing through all the populated
lattice points and their neighboring elements.
% For efficiency, we do this matrix construction recursively with shared computations
% since $n^{th}$ neighbourhood elements are $1^{st}$ neighborhood elements of $n-1^{th}$ neighbourhood elements. \pg{do not understand}
This is done recursively to share computations. For any lattice point, the neighbors that are
$n$ hops away are the direct neighbors of the points that are $n-1$ hops away.
The size of a $d$ dimensional spatial filter with width $s+1$ is $(s+1)^{d}$ (\eg, a
$3\times 3$ filter, $s=2$ in $d=2$ has $3^2=9$ elements) and this size grows
exponentially in the number of dimensions $d$. The permutohedral lattice is constructed by
projecting a regular grid onto the plane spanned by the $d$ dimensional normal vector ${(1,\ldots,1)}^{\top}$. See
Fig.~\ref{fig:latticeconstruction} for an illustration of the 1D lattice construction.
Many corners of a grid filter are projected onto the same point, in total $t = {(s+1)}^{d} -
s^{d}$ elements remain in the permutohedral filter with $s$ neighborhood in $d-1$ dimensions.
If the lattice has $m$ populated elements, the
matrix $K$ has size $t\times m$. Note that, since the input signal is typically
sparse, only a few lattice corners are being populated in the \textit{slicing} step.
We use a hash-table to keep track of these points and traverse only through
the populated lattice points for this neighborhood matrix construction.

Once the blur neighbor matrix $K$ is constructed, we can perform the convolution
by the matrix vector multiplication
\begin{equation}
\ell' = BK,
\label{eq:conv}
\end{equation}
where $B$ is the $1 \times t$ filter kernel (whose values we will learn) and $\ell'\in\mathbb{R}^{1\times m}$
is the result of the filtering at the $m$ lattice points. In practice, we found that the
matrix $K$ is sometimes too large to fit into GPU memory and we divided the matrix $K$
into smaller pieces to compute Eq.~\ref{eq:conv} sequentially.

In the general multi-dimensional case, the signal $\ell$ is of $c$ dimensions. Then
the kernel $B$ is of size $c \times t$ and $K$ stores the $c$ dimensional vectors
accordingly. When the input and output points are different, we slice only the
input points and splat only at the output points.


\subsection{Additional Experiments}
\label{sec:addexps}
In this section, we discuss more use-cases for the learned bilateral filters, one
use-case of BNNs and two single filter applications for image and 3D mesh denoising.

\subsubsection{Recognition of subsampled MNIST}\label{sec:app_mnist}

One of the strengths of the proposed filter convolution is that it does not
require the input to lie on a regular grid. The only requirement is to define a distance
between features of the input signal.
We highlight this feature with the following experiment using the
classical MNIST ten class classification problem~\cite{lecun1998mnist}. We sample a
sparse set of $N$ points $(x,y)\in [0,1]\times [0,1]$
uniformly at random in the input image, use their interpolated values
as signal and the \emph{continuous} $(x,y)$ positions as features. This mimics
sub-sampling of a high-dimensional signal. To compare against a spatial convolution,
we interpolate the sparse set of values at the grid positions.

We take a reference implementation of LeNet~\cite{lecun1998gradient} that
is part of the Caffe project~\cite{jia2014caffe} and compare it
against the same architecture but replacing the first convolutional
layer with a bilateral convolution layer (BCL). The filter size
and numbers are adjusted to get a comparable number of parameters
($5\times 5$ for LeNet, $2$-neighborhood for BCL).

The results are shown in Table~\ref{tab:all-results}. We see that training
on the original MNIST data (column Original, LeNet vs. BNN) leads to a slight
decrease in performance of the BNN (99.03\%) compared to LeNet
(99.19\%). The BNN can be trained and evaluated on sparse
signals, and we resample the image as described above for $N=$ 100\%, 60\% and
20\% of the total number of pixels. The methods are also evaluated
on test images that are subsampled in the same way. Note that we can
train and test with different subsampling rates. We introduce an additional
bilinear interpolation layer for the LeNet architecture to train on the same
data. In essence, both models perform a spatial interpolation and thus we
expect them to yield a similar classification accuracy. Once the data is of
higher dimensions, the permutohedral convolution will be faster due to hashing
the sparse input points, as well as less memory demanding in comparison to
naive application of a spatial convolution with interpolated values.

\begin{table}[t]
  \begin{center}
    \footnotesize
    \centering
    \begin{tabular}[t]{lllll}
      \toprule
              &     & \multicolumn{3}{c}{Test Subsampling} \\
       Method  & Original & 100\% & 60\% & 20\%\\
      \midrule
       LeNet &  \textbf{0.9919} & 0.9660 & 0.9348 & \textbf{0.6434} \\
       BNN &  0.9903 & \textbf{0.9844} & \textbf{0.9534} & 0.5767 \\
      \hline
       LeNet 100\% & 0.9856 & 0.9809 & 0.9678 & \textbf{0.7386} \\
       BNN 100\% & \textbf{0.9900} & \textbf{0.9863} & \textbf{0.9699} & 0.6910 \\
      \hline
       LeNet 60\% & 0.9848 & 0.9821 & 0.9740 & 0.8151 \\
       BNN 60\% & \textbf{0.9885} & \textbf{0.9864} & \textbf{0.9771} & \textbf{0.8214}\\
      \hline
       LeNet 20\% & \textbf{0.9763} & \textbf{0.9754} & 0.9695 & 0.8928 \\
       BNN 20\% & 0.9728 & 0.9735 & \textbf{0.9701} & \textbf{0.9042}\\
      \bottomrule
    \end{tabular}
  \end{center}
\vspace{-.2cm}
\caption{Classification accuracy on MNIST. We compare the
    LeNet~\cite{lecun1998gradient} implementation that is part of
    Caffe~\cite{jia2014caffe} to the network with the first layer
    replaced by a bilateral convolution layer (BCL). Both are trained
    on the original image resolution (first two rows). Three more BNN
    and CNN models are trained with randomly subsampled images (100\%,
    60\% and 20\% of the pixels). An additional bilinear interpolation
    layer samples the input signal on a spatial grid for the CNN model.
  }
  \label{tab:all-results}
\vspace{-.5cm}
\end{table}

\subsubsection{Image Denoising}

The main application that inspired the development of the bilateral
filtering operation is image denoising~\cite{aurich1995non}, there
using a single Gaussian kernel. Our development allows to learn this
kernel function from data and we explore how to improve using a \emph{single}
but more general bilateral filter.

We use the Berkeley segmentation dataset
(BSDS500)~\cite{arbelaezi2011bsds500} as a test bed. The color
images in the dataset are converted to gray-scale,
and corrupted with Gaussian noise with a standard deviation of
$\frac {25} {255}$.

We compare the performance of four different filter models on a
denoising task.
The first baseline model (`Spatial' in Table \ref{tab:denoising}, $25$
weights) uses a single spatial filter with a kernel size of
$5$ and predicts the scalar gray-scale value at the center pixel. The next model
(`Gauss Bilateral') applies a bilateral \emph{Gaussian}
filter to the noisy input, using position and intensity features $\f=(x,y,v)^\top$.
The third setup (`Learned Bilateral', $65$ weights)
takes a Gauss kernel as initialization and
fits all filter weights on the train set to minimize the
mean squared error with respect to the clean images.
We run a combination
of spatial and permutohedral convolutions on spatial and bilateral
features (`Spatial + Bilateral (Learned)') to check for a complementary
performance of the two convolutions.

\label{sec:exp:denoising}
\begin{table}[!h]
\begin{center}
  \footnotesize
  \begin{tabular}[t]{lr}
    \toprule
    Method & PSNR \\
    \midrule
    Noisy Input & $20.17$ \\
    Spatial & $26.27$ \\
    Gauss Bilateral & $26.51$ \\
    Learned Bilateral & $26.58$ \\
    Spatial + Bilateral (Learned) & \textbf{$26.65$} \\
    \bottomrule
  \end{tabular}
\end{center}
\vspace{-0.5em}
\caption{PSNR results of a denoising task using the BSDS500
  dataset~\cite{arbelaezi2011bsds500}}
\vspace{-0.5em}
\label{tab:denoising}
\end{table}
\vspace{-0.2em}

The PSNR scores evaluated on full images of the test set are
shown in Table \ref{tab:denoising}. We find that an untrained bilateral
filter already performs better than a trained spatial convolution
($26.27$ to $26.51$). A learned convolution further improve the
performance slightly. We chose this simple one-kernel setup to
validate an advantage of the generalized bilateral filter. A competitive
denoising system would employ RGB color information and also
needs to be properly adjusted in network size. Multi-layer perceptrons
have obtained state-of-the-art denoising results~\cite{burger12cvpr}
and the permutohedral lattice layer can readily be used in such an
architecture, which is intended future work.

\subsection{Additional results}
\label{sec:addresults}

This section contains more qualitative results for the experiments presented in Chapter~\ref{chap:bnn}.

\begin{figure*}[th!]
  \centering
    \includegraphics[width=\columnwidth,trim={5cm 2.5cm 5cm 4.5cm},clip]{figures/supplementary/lattice_viz.pdf}
    \vspace{-0.7cm}
  \mycaption{Visualization of the Permutohedral Lattice}
  {Sample lattice visualizations for different feature spaces. All pixels falling in the same simplex cell are shown with
  the same color. $(x,y)$ features correspond to image pixel positions, and $(r,g,b) \in [0,255]$ correspond
  to the red, green and blue color values.}
\label{fig:latticeviz}
\end{figure*}

\subsubsection{Lattice Visualization}

Figure~\ref{fig:latticeviz} shows sample lattice visualizations for different feature spaces.

\newcolumntype{L}[1]{>{\raggedright\let\newline\\\arraybackslash\hspace{0pt}}b{#1}}
\newcolumntype{C}[1]{>{\centering\let\newline\\\arraybackslash\hspace{0pt}}b{#1}}
\newcolumntype{R}[1]{>{\raggedleft\let\newline\\\arraybackslash\hspace{0pt}}b{#1}}

\subsubsection{Color Upsampling}\label{sec:color_upsampling}
\label{sec:col_upsample_extra}

Some images of the upsampling for the Pascal
VOC12 dataset are shown in Fig.~\ref{fig:Colour_upsample_visuals}. It is
especially the low level image details that are better preserved with
a learned bilateral filter compared to the Gaussian case.

\begin{figure*}[t!]
  \centering
    \subfigure{%
   \raisebox{2.0em}{
    \includegraphics[width=.06\columnwidth]{figures/supplementary/2007_004969.jpg}
   }
  }
  \subfigure{%
    \includegraphics[width=.17\columnwidth]{figures/supplementary/2007_004969_gray.pdf}
  }
  \subfigure{%
    \includegraphics[width=.17\columnwidth]{figures/supplementary/2007_004969_gt.pdf}
  }
  \subfigure{%
    \includegraphics[width=.17\columnwidth]{figures/supplementary/2007_004969_bicubic.pdf}
  }
  \subfigure{%
    \includegraphics[width=.17\columnwidth]{figures/supplementary/2007_004969_gauss.pdf}
  }
  \subfigure{%
    \includegraphics[width=.17\columnwidth]{figures/supplementary/2007_004969_learnt.pdf}
  }\\
    \subfigure{%
   \raisebox{2.0em}{
    \includegraphics[width=.06\columnwidth]{figures/supplementary/2007_003106.jpg}
   }
  }
  \subfigure{%
    \includegraphics[width=.17\columnwidth]{figures/supplementary/2007_003106_gray.pdf}
  }
  \subfigure{%
    \includegraphics[width=.17\columnwidth]{figures/supplementary/2007_003106_gt.pdf}
  }
  \subfigure{%
    \includegraphics[width=.17\columnwidth]{figures/supplementary/2007_003106_bicubic.pdf}
  }
  \subfigure{%
    \includegraphics[width=.17\columnwidth]{figures/supplementary/2007_003106_gauss.pdf}
  }
  \subfigure{%
    \includegraphics[width=.17\columnwidth]{figures/supplementary/2007_003106_learnt.pdf}
  }\\
  \setcounter{subfigure}{0}
  \small{
  \subfigure[Inp.]{%
  \raisebox{2.0em}{
    \includegraphics[width=.06\columnwidth]{figures/supplementary/2007_006837.jpg}
   }
  }
  \subfigure[Guidance]{%
    \includegraphics[width=.17\columnwidth]{figures/supplementary/2007_006837_gray.pdf}
  }
   \subfigure[GT]{%
    \includegraphics[width=.17\columnwidth]{figures/supplementary/2007_006837_gt.pdf}
  }
  \subfigure[Bicubic]{%
    \includegraphics[width=.17\columnwidth]{figures/supplementary/2007_006837_bicubic.pdf}
  }
  \subfigure[Gauss-BF]{%
    \includegraphics[width=.17\columnwidth]{figures/supplementary/2007_006837_gauss.pdf}
  }
  \subfigure[Learned-BF]{%
    \includegraphics[width=.17\columnwidth]{figures/supplementary/2007_006837_learnt.pdf}
  }
  }
  \vspace{-0.5cm}
  \mycaption{Color Upsampling}{Color $8\times$ upsampling results
  using different methods, from left to right, (a)~Low-resolution input color image (Inp.),
  (b)~Gray scale guidance image, (c)~Ground-truth color image; Upsampled color images with
  (d)~Bicubic interpolation, (e) Gauss bilateral upsampling and, (f)~Learned bilateral
  updampgling (best viewed on screen).}

\label{fig:Colour_upsample_visuals}
\end{figure*}

\subsubsection{Depth Upsampling}
\label{sec:depth_upsample_extra}

Figure~\ref{fig:depth_upsample_visuals} presents some more qualitative results comparing bicubic interpolation, Gauss
bilateral and learned bilateral upsampling on NYU depth dataset image~\cite{silberman2012indoor}.

\subsubsection{Character Recognition}\label{sec:app_character}

 Figure~\ref{fig:nnrecognition} shows the schematic of different layers
 of the network architecture for LeNet-7~\cite{lecun1998mnist}
 and DeepCNet(5, 50)~\cite{ciresan2012multi,graham2014spatially}. For the BNN variants, the first layer filters are replaced
 with learned bilateral filters and are learned end-to-end.

\subsubsection{Semantic Segmentation}\label{sec:app_semantic_segmentation}
\label{sec:semantic_bnn_extra}

Some more visual results for semantic segmentation are shown in Figure~\ref{fig:semantic_visuals}.
These include the underlying DeepLab CNN\cite{chen2014semantic} result (DeepLab),
the 2 step mean-field result with Gaussian edge potentials (+2stepMF-GaussCRF)
and also corresponding results with learned edge potentials (+2stepMF-LearnedCRF).
In general, we observe that mean-field in learned CRF leads to slightly dilated
classification regions in comparison to using Gaussian CRF thereby filling-in the
false negative pixels and also correcting some mis-classified regions.

\begin{figure*}[t!]
  \centering
    \subfigure{%
   \raisebox{2.0em}{
    \includegraphics[width=.06\columnwidth]{figures/supplementary/2bicubic}
   }
  }
  \subfigure{%
    \includegraphics[width=.17\columnwidth]{figures/supplementary/2given_image}
  }
  \subfigure{%
    \includegraphics[width=.17\columnwidth]{figures/supplementary/2ground_truth}
  }
  \subfigure{%
    \includegraphics[width=.17\columnwidth]{figures/supplementary/2bicubic}
  }
  \subfigure{%
    \includegraphics[width=.17\columnwidth]{figures/supplementary/2gauss}
  }
  \subfigure{%
    \includegraphics[width=.17\columnwidth]{figures/supplementary/2learnt}
  }\\
    \subfigure{%
   \raisebox{2.0em}{
    \includegraphics[width=.06\columnwidth]{figures/supplementary/32bicubic}
   }
  }
  \subfigure{%
    \includegraphics[width=.17\columnwidth]{figures/supplementary/32given_image}
  }
  \subfigure{%
    \includegraphics[width=.17\columnwidth]{figures/supplementary/32ground_truth}
  }
  \subfigure{%
    \includegraphics[width=.17\columnwidth]{figures/supplementary/32bicubic}
  }
  \subfigure{%
    \includegraphics[width=.17\columnwidth]{figures/supplementary/32gauss}
  }
  \subfigure{%
    \includegraphics[width=.17\columnwidth]{figures/supplementary/32learnt}
  }\\
  \setcounter{subfigure}{0}
  \small{
  \subfigure[Inp.]{%
  \raisebox{2.0em}{
    \includegraphics[width=.06\columnwidth]{figures/supplementary/41bicubic}
   }
  }
  \subfigure[Guidance]{%
    \includegraphics[width=.17\columnwidth]{figures/supplementary/41given_image}
  }
   \subfigure[GT]{%
    \includegraphics[width=.17\columnwidth]{figures/supplementary/41ground_truth}
  }
  \subfigure[Bicubic]{%
    \includegraphics[width=.17\columnwidth]{figures/supplementary/41bicubic}
  }
  \subfigure[Gauss-BF]{%
    \includegraphics[width=.17\columnwidth]{figures/supplementary/41gauss}
  }
  \subfigure[Learned-BF]{%
    \includegraphics[width=.17\columnwidth]{figures/supplementary/41learnt}
  }
  }
  \mycaption{Depth Upsampling}{Depth $8\times$ upsampling results
  using different upsampling strategies, from left to right,
  (a)~Low-resolution input depth image (Inp.),
  (b)~High-resolution guidance image, (c)~Ground-truth depth; Upsampled depth images with
  (d)~Bicubic interpolation, (e) Gauss bilateral upsampling and, (f)~Learned bilateral
  updampgling (best viewed on screen).}

\label{fig:depth_upsample_visuals}
\end{figure*}

\subsubsection{Material Segmentation}\label{sec:app_material_segmentation}
\label{sec:material_bnn_extra}

In Fig.~\ref{fig:material_visuals-app2}, we present visual results comparing 2 step
mean-field inference with Gaussian and learned pairwise CRF potentials. In
general, we observe that the pixels belonging to dominant classes in the
training data are being more accurately classified with learned CRF. This leads to
a significant improvements in overall pixel accuracy. This also results
in a slight decrease of the accuracy from less frequent class pixels thereby
slightly reducing the average class accuracy with learning. We attribute this
to the type of annotation that is available for this dataset, which is not
for the entire image but for some segments in the image. We have very few
images of the infrequent classes to combat this behaviour during training.

\subsubsection{Experiment Protocols}
\label{sec:protocols}

Table~\ref{tbl:parameters} shows experiment protocols of different experiments.

 \begin{figure*}[t!]
  \centering
  \subfigure[LeNet-7]{
    \includegraphics[width=0.7\columnwidth]{figures/supplementary/lenet_cnn_network}
    }\\
    \subfigure[DeepCNet]{
    \includegraphics[width=\columnwidth]{figures/supplementary/deepcnet_cnn_network}
    }
  \mycaption{CNNs for Character Recognition}
  {Schematic of (top) LeNet-7~\cite{lecun1998mnist} and (bottom) DeepCNet(5,50)~\cite{ciresan2012multi,graham2014spatially} architectures used in Assamese
  character recognition experiments.}
\label{fig:nnrecognition}
\end{figure*}

\definecolor{voc_1}{RGB}{0, 0, 0}
\definecolor{voc_2}{RGB}{128, 0, 0}
\definecolor{voc_3}{RGB}{0, 128, 0}
\definecolor{voc_4}{RGB}{128, 128, 0}
\definecolor{voc_5}{RGB}{0, 0, 128}
\definecolor{voc_6}{RGB}{128, 0, 128}
\definecolor{voc_7}{RGB}{0, 128, 128}
\definecolor{voc_8}{RGB}{128, 128, 128}
\definecolor{voc_9}{RGB}{64, 0, 0}
\definecolor{voc_10}{RGB}{192, 0, 0}
\definecolor{voc_11}{RGB}{64, 128, 0}
\definecolor{voc_12}{RGB}{192, 128, 0}
\definecolor{voc_13}{RGB}{64, 0, 128}
\definecolor{voc_14}{RGB}{192, 0, 128}
\definecolor{voc_15}{RGB}{64, 128, 128}
\definecolor{voc_16}{RGB}{192, 128, 128}
\definecolor{voc_17}{RGB}{0, 64, 0}
\definecolor{voc_18}{RGB}{128, 64, 0}
\definecolor{voc_19}{RGB}{0, 192, 0}
\definecolor{voc_20}{RGB}{128, 192, 0}
\definecolor{voc_21}{RGB}{0, 64, 128}
\definecolor{voc_22}{RGB}{128, 64, 128}

\begin{figure*}[t]
  \centering
  \small{
  \fcolorbox{white}{voc_1}{\rule{0pt}{6pt}\rule{6pt}{0pt}} Background~~
  \fcolorbox{white}{voc_2}{\rule{0pt}{6pt}\rule{6pt}{0pt}} Aeroplane~~
  \fcolorbox{white}{voc_3}{\rule{0pt}{6pt}\rule{6pt}{0pt}} Bicycle~~
  \fcolorbox{white}{voc_4}{\rule{0pt}{6pt}\rule{6pt}{0pt}} Bird~~
  \fcolorbox{white}{voc_5}{\rule{0pt}{6pt}\rule{6pt}{0pt}} Boat~~
  \fcolorbox{white}{voc_6}{\rule{0pt}{6pt}\rule{6pt}{0pt}} Bottle~~
  \fcolorbox{white}{voc_7}{\rule{0pt}{6pt}\rule{6pt}{0pt}} Bus~~
  \fcolorbox{white}{voc_8}{\rule{0pt}{6pt}\rule{6pt}{0pt}} Car~~ \\
  \fcolorbox{white}{voc_9}{\rule{0pt}{6pt}\rule{6pt}{0pt}} Cat~~
  \fcolorbox{white}{voc_10}{\rule{0pt}{6pt}\rule{6pt}{0pt}} Chair~~
  \fcolorbox{white}{voc_11}{\rule{0pt}{6pt}\rule{6pt}{0pt}} Cow~~
  \fcolorbox{white}{voc_12}{\rule{0pt}{6pt}\rule{6pt}{0pt}} Dining Table~~
  \fcolorbox{white}{voc_13}{\rule{0pt}{6pt}\rule{6pt}{0pt}} Dog~~
  \fcolorbox{white}{voc_14}{\rule{0pt}{6pt}\rule{6pt}{0pt}} Horse~~
  \fcolorbox{white}{voc_15}{\rule{0pt}{6pt}\rule{6pt}{0pt}} Motorbike~~
  \fcolorbox{white}{voc_16}{\rule{0pt}{6pt}\rule{6pt}{0pt}} Person~~ \\
  \fcolorbox{white}{voc_17}{\rule{0pt}{6pt}\rule{6pt}{0pt}} Potted Plant~~
  \fcolorbox{white}{voc_18}{\rule{0pt}{6pt}\rule{6pt}{0pt}} Sheep~~
  \fcolorbox{white}{voc_19}{\rule{0pt}{6pt}\rule{6pt}{0pt}} Sofa~~
  \fcolorbox{white}{voc_20}{\rule{0pt}{6pt}\rule{6pt}{0pt}} Train~~
  \fcolorbox{white}{voc_21}{\rule{0pt}{6pt}\rule{6pt}{0pt}} TV monitor~~ \\
  }
  \subfigure{%
    \includegraphics[width=.18\columnwidth]{figures/supplementary/2007_001423_given.jpg}
  }
  \subfigure{%
    \includegraphics[width=.18\columnwidth]{figures/supplementary/2007_001423_gt.png}
  }
  \subfigure{%
    \includegraphics[width=.18\columnwidth]{figures/supplementary/2007_001423_cnn.png}
  }
  \subfigure{%
    \includegraphics[width=.18\columnwidth]{figures/supplementary/2007_001423_gauss.png}
  }
  \subfigure{%
    \includegraphics[width=.18\columnwidth]{figures/supplementary/2007_001423_learnt.png}
  }\\
  \subfigure{%
    \includegraphics[width=.18\columnwidth]{figures/supplementary/2007_001430_given.jpg}
  }
  \subfigure{%
    \includegraphics[width=.18\columnwidth]{figures/supplementary/2007_001430_gt.png}
  }
  \subfigure{%
    \includegraphics[width=.18\columnwidth]{figures/supplementary/2007_001430_cnn.png}
  }
  \subfigure{%
    \includegraphics[width=.18\columnwidth]{figures/supplementary/2007_001430_gauss.png}
  }
  \subfigure{%
    \includegraphics[width=.18\columnwidth]{figures/supplementary/2007_001430_learnt.png}
  }\\
    \subfigure{%
    \includegraphics[width=.18\columnwidth]{figures/supplementary/2007_007996_given.jpg}
  }
  \subfigure{%
    \includegraphics[width=.18\columnwidth]{figures/supplementary/2007_007996_gt.png}
  }
  \subfigure{%
    \includegraphics[width=.18\columnwidth]{figures/supplementary/2007_007996_cnn.png}
  }
  \subfigure{%
    \includegraphics[width=.18\columnwidth]{figures/supplementary/2007_007996_gauss.png}
  }
  \subfigure{%
    \includegraphics[width=.18\columnwidth]{figures/supplementary/2007_007996_learnt.png}
  }\\
   \subfigure{%
    \includegraphics[width=.18\columnwidth]{figures/supplementary/2010_002682_given.jpg}
  }
  \subfigure{%
    \includegraphics[width=.18\columnwidth]{figures/supplementary/2010_002682_gt.png}
  }
  \subfigure{%
    \includegraphics[width=.18\columnwidth]{figures/supplementary/2010_002682_cnn.png}
  }
  \subfigure{%
    \includegraphics[width=.18\columnwidth]{figures/supplementary/2010_002682_gauss.png}
  }
  \subfigure{%
    \includegraphics[width=.18\columnwidth]{figures/supplementary/2010_002682_learnt.png}
  }\\
     \subfigure{%
    \includegraphics[width=.18\columnwidth]{figures/supplementary/2010_004789_given.jpg}
  }
  \subfigure{%
    \includegraphics[width=.18\columnwidth]{figures/supplementary/2010_004789_gt.png}
  }
  \subfigure{%
    \includegraphics[width=.18\columnwidth]{figures/supplementary/2010_004789_cnn.png}
  }
  \subfigure{%
    \includegraphics[width=.18\columnwidth]{figures/supplementary/2010_004789_gauss.png}
  }
  \subfigure{%
    \includegraphics[width=.18\columnwidth]{figures/supplementary/2010_004789_learnt.png}
  }\\
       \subfigure{%
    \includegraphics[width=.18\columnwidth]{figures/supplementary/2007_001311_given.jpg}
  }
  \subfigure{%
    \includegraphics[width=.18\columnwidth]{figures/supplementary/2007_001311_gt.png}
  }
  \subfigure{%
    \includegraphics[width=.18\columnwidth]{figures/supplementary/2007_001311_cnn.png}
  }
  \subfigure{%
    \includegraphics[width=.18\columnwidth]{figures/supplementary/2007_001311_gauss.png}
  }
  \subfigure{%
    \includegraphics[width=.18\columnwidth]{figures/supplementary/2007_001311_learnt.png}
  }\\
  \setcounter{subfigure}{0}
  \subfigure[Input]{%
    \includegraphics[width=.18\columnwidth]{figures/supplementary/2010_003531_given.jpg}
  }
  \subfigure[Ground Truth]{%
    \includegraphics[width=.18\columnwidth]{figures/supplementary/2010_003531_gt.png}
  }
  \subfigure[DeepLab]{%
    \includegraphics[width=.18\columnwidth]{figures/supplementary/2010_003531_cnn.png}
  }
  \subfigure[+GaussCRF]{%
    \includegraphics[width=.18\columnwidth]{figures/supplementary/2010_003531_gauss.png}
  }
  \subfigure[+LearnedCRF]{%
    \includegraphics[width=.18\columnwidth]{figures/supplementary/2010_003531_learnt.png}
  }
  \vspace{-0.3cm}
  \mycaption{Semantic Segmentation}{Example results of semantic segmentation.
  (c)~depicts the unary results before application of MF, (d)~after two steps of MF with Gaussian edge CRF potentials, (e)~after
  two steps of MF with learned edge CRF potentials.}
    \label{fig:semantic_visuals}
\end{figure*}


\definecolor{minc_1}{HTML}{771111}
\definecolor{minc_2}{HTML}{CAC690}
\definecolor{minc_3}{HTML}{EEEEEE}
\definecolor{minc_4}{HTML}{7C8FA6}
\definecolor{minc_5}{HTML}{597D31}
\definecolor{minc_6}{HTML}{104410}
\definecolor{minc_7}{HTML}{BB819C}
\definecolor{minc_8}{HTML}{D0CE48}
\definecolor{minc_9}{HTML}{622745}
\definecolor{minc_10}{HTML}{666666}
\definecolor{minc_11}{HTML}{D54A31}
\definecolor{minc_12}{HTML}{101044}
\definecolor{minc_13}{HTML}{444126}
\definecolor{minc_14}{HTML}{75D646}
\definecolor{minc_15}{HTML}{DD4348}
\definecolor{minc_16}{HTML}{5C8577}
\definecolor{minc_17}{HTML}{C78472}
\definecolor{minc_18}{HTML}{75D6D0}
\definecolor{minc_19}{HTML}{5B4586}
\definecolor{minc_20}{HTML}{C04393}
\definecolor{minc_21}{HTML}{D69948}
\definecolor{minc_22}{HTML}{7370D8}
\definecolor{minc_23}{HTML}{7A3622}
\definecolor{minc_24}{HTML}{000000}

\begin{figure*}[t]
  \centering
  \small{
  \fcolorbox{white}{minc_1}{\rule{0pt}{6pt}\rule{6pt}{0pt}} Brick~~
  \fcolorbox{white}{minc_2}{\rule{0pt}{6pt}\rule{6pt}{0pt}} Carpet~~
  \fcolorbox{white}{minc_3}{\rule{0pt}{6pt}\rule{6pt}{0pt}} Ceramic~~
  \fcolorbox{white}{minc_4}{\rule{0pt}{6pt}\rule{6pt}{0pt}} Fabric~~
  \fcolorbox{white}{minc_5}{\rule{0pt}{6pt}\rule{6pt}{0pt}} Foliage~~
  \fcolorbox{white}{minc_6}{\rule{0pt}{6pt}\rule{6pt}{0pt}} Food~~
  \fcolorbox{white}{minc_7}{\rule{0pt}{6pt}\rule{6pt}{0pt}} Glass~~
  \fcolorbox{white}{minc_8}{\rule{0pt}{6pt}\rule{6pt}{0pt}} Hair~~ \\
  \fcolorbox{white}{minc_9}{\rule{0pt}{6pt}\rule{6pt}{0pt}} Leather~~
  \fcolorbox{white}{minc_10}{\rule{0pt}{6pt}\rule{6pt}{0pt}} Metal~~
  \fcolorbox{white}{minc_11}{\rule{0pt}{6pt}\rule{6pt}{0pt}} Mirror~~
  \fcolorbox{white}{minc_12}{\rule{0pt}{6pt}\rule{6pt}{0pt}} Other~~
  \fcolorbox{white}{minc_13}{\rule{0pt}{6pt}\rule{6pt}{0pt}} Painted~~
  \fcolorbox{white}{minc_14}{\rule{0pt}{6pt}\rule{6pt}{0pt}} Paper~~
  \fcolorbox{white}{minc_15}{\rule{0pt}{6pt}\rule{6pt}{0pt}} Plastic~~\\
  \fcolorbox{white}{minc_16}{\rule{0pt}{6pt}\rule{6pt}{0pt}} Polished Stone~~
  \fcolorbox{white}{minc_17}{\rule{0pt}{6pt}\rule{6pt}{0pt}} Skin~~
  \fcolorbox{white}{minc_18}{\rule{0pt}{6pt}\rule{6pt}{0pt}} Sky~~
  \fcolorbox{white}{minc_19}{\rule{0pt}{6pt}\rule{6pt}{0pt}} Stone~~
  \fcolorbox{white}{minc_20}{\rule{0pt}{6pt}\rule{6pt}{0pt}} Tile~~
  \fcolorbox{white}{minc_21}{\rule{0pt}{6pt}\rule{6pt}{0pt}} Wallpaper~~
  \fcolorbox{white}{minc_22}{\rule{0pt}{6pt}\rule{6pt}{0pt}} Water~~
  \fcolorbox{white}{minc_23}{\rule{0pt}{6pt}\rule{6pt}{0pt}} Wood~~ \\
  }
  \subfigure{%
    \includegraphics[width=.18\columnwidth]{figures/supplementary/000010868_given.jpg}
  }
  \subfigure{%
    \includegraphics[width=.18\columnwidth]{figures/supplementary/000010868_gt.png}
  }
  \subfigure{%
    \includegraphics[width=.18\columnwidth]{figures/supplementary/000010868_cnn.png}
  }
  \subfigure{%
    \includegraphics[width=.18\columnwidth]{figures/supplementary/000010868_gauss.png}
  }
  \subfigure{%
    \includegraphics[width=.18\columnwidth]{figures/supplementary/000010868_learnt.png}
  }\\[-2ex]
  \subfigure{%
    \includegraphics[width=.18\columnwidth]{figures/supplementary/000006011_given.jpg}
  }
  \subfigure{%
    \includegraphics[width=.18\columnwidth]{figures/supplementary/000006011_gt.png}
  }
  \subfigure{%
    \includegraphics[width=.18\columnwidth]{figures/supplementary/000006011_cnn.png}
  }
  \subfigure{%
    \includegraphics[width=.18\columnwidth]{figures/supplementary/000006011_gauss.png}
  }
  \subfigure{%
    \includegraphics[width=.18\columnwidth]{figures/supplementary/000006011_learnt.png}
  }\\[-2ex]
    \subfigure{%
    \includegraphics[width=.18\columnwidth]{figures/supplementary/000008553_given.jpg}
  }
  \subfigure{%
    \includegraphics[width=.18\columnwidth]{figures/supplementary/000008553_gt.png}
  }
  \subfigure{%
    \includegraphics[width=.18\columnwidth]{figures/supplementary/000008553_cnn.png}
  }
  \subfigure{%
    \includegraphics[width=.18\columnwidth]{figures/supplementary/000008553_gauss.png}
  }
  \subfigure{%
    \includegraphics[width=.18\columnwidth]{figures/supplementary/000008553_learnt.png}
  }\\[-2ex]
   \subfigure{%
    \includegraphics[width=.18\columnwidth]{figures/supplementary/000009188_given.jpg}
  }
  \subfigure{%
    \includegraphics[width=.18\columnwidth]{figures/supplementary/000009188_gt.png}
  }
  \subfigure{%
    \includegraphics[width=.18\columnwidth]{figures/supplementary/000009188_cnn.png}
  }
  \subfigure{%
    \includegraphics[width=.18\columnwidth]{figures/supplementary/000009188_gauss.png}
  }
  \subfigure{%
    \includegraphics[width=.18\columnwidth]{figures/supplementary/000009188_learnt.png}
  }\\[-2ex]
  \setcounter{subfigure}{0}
  \subfigure[Input]{%
    \includegraphics[width=.18\columnwidth]{figures/supplementary/000023570_given.jpg}
  }
  \subfigure[Ground Truth]{%
    \includegraphics[width=.18\columnwidth]{figures/supplementary/000023570_gt.png}
  }
  \subfigure[DeepLab]{%
    \includegraphics[width=.18\columnwidth]{figures/supplementary/000023570_cnn.png}
  }
  \subfigure[+GaussCRF]{%
    \includegraphics[width=.18\columnwidth]{figures/supplementary/000023570_gauss.png}
  }
  \subfigure[+LearnedCRF]{%
    \includegraphics[width=.18\columnwidth]{figures/supplementary/000023570_learnt.png}
  }
  \mycaption{Material Segmentation}{Example results of material segmentation.
  (c)~depicts the unary results before application of MF, (d)~after two steps of MF with Gaussian edge CRF potentials, (e)~after two steps of MF with learned edge CRF potentials.}
    \label{fig:material_visuals-app2}
\end{figure*}


\begin{table*}[h]
\tiny
  \centering
    \begin{tabular}{L{2.3cm} L{2.25cm} C{1.5cm} C{0.7cm} C{0.6cm} C{0.7cm} C{0.7cm} C{0.7cm} C{1.6cm} C{0.6cm} C{0.6cm} C{0.6cm}}
      \toprule
& & & & & \multicolumn{3}{c}{\textbf{Data Statistics}} & \multicolumn{4}{c}{\textbf{Training Protocol}} \\

\textbf{Experiment} & \textbf{Feature Types} & \textbf{Feature Scales} & \textbf{Filter Size} & \textbf{Filter Nbr.} & \textbf{Train}  & \textbf{Val.} & \textbf{Test} & \textbf{Loss Type} & \textbf{LR} & \textbf{Batch} & \textbf{Epochs} \\
      \midrule
      \multicolumn{2}{c}{\textbf{Single Bilateral Filter Applications}} & & & & & & & & & \\
      \textbf{2$\times$ Color Upsampling} & Position$_{1}$, Intensity (3D) & 0.13, 0.17 & 65 & 2 & 10581 & 1449 & 1456 & MSE & 1e-06 & 200 & 94.5\\
      \textbf{4$\times$ Color Upsampling} & Position$_{1}$, Intensity (3D) & 0.06, 0.17 & 65 & 2 & 10581 & 1449 & 1456 & MSE & 1e-06 & 200 & 94.5\\
      \textbf{8$\times$ Color Upsampling} & Position$_{1}$, Intensity (3D) & 0.03, 0.17 & 65 & 2 & 10581 & 1449 & 1456 & MSE & 1e-06 & 200 & 94.5\\
      \textbf{16$\times$ Color Upsampling} & Position$_{1}$, Intensity (3D) & 0.02, 0.17 & 65 & 2 & 10581 & 1449 & 1456 & MSE & 1e-06 & 200 & 94.5\\
      \textbf{Depth Upsampling} & Position$_{1}$, Color (5D) & 0.05, 0.02 & 665 & 2 & 795 & 100 & 654 & MSE & 1e-07 & 50 & 251.6\\
      \textbf{Mesh Denoising} & Isomap (4D) & 46.00 & 63 & 2 & 1000 & 200 & 500 & MSE & 100 & 10 & 100.0 \\
      \midrule
      \multicolumn{2}{c}{\textbf{DenseCRF Applications}} & & & & & & & & &\\
      \multicolumn{2}{l}{\textbf{Semantic Segmentation}} & & & & & & & & &\\
      \textbf{- 1step MF} & Position$_{1}$, Color (5D); Position$_{1}$ (2D) & 0.01, 0.34; 0.34  & 665; 19  & 2; 2 & 10581 & 1449 & 1456 & Logistic & 0.1 & 5 & 1.4 \\
      \textbf{- 2step MF} & Position$_{1}$, Color (5D); Position$_{1}$ (2D) & 0.01, 0.34; 0.34 & 665; 19 & 2; 2 & 10581 & 1449 & 1456 & Logistic & 0.1 & 5 & 1.4 \\
      \textbf{- \textit{loose} 2step MF} & Position$_{1}$, Color (5D); Position$_{1}$ (2D) & 0.01, 0.34; 0.34 & 665; 19 & 2; 2 &10581 & 1449 & 1456 & Logistic & 0.1 & 5 & +1.9  \\ \\
      \multicolumn{2}{l}{\textbf{Material Segmentation}} & & & & & & & & &\\
      \textbf{- 1step MF} & Position$_{2}$, Lab-Color (5D) & 5.00, 0.05, 0.30  & 665 & 2 & 928 & 150 & 1798 & Weighted Logistic & 1e-04 & 24 & 2.6 \\
      \textbf{- 2step MF} & Position$_{2}$, Lab-Color (5D) & 5.00, 0.05, 0.30 & 665 & 2 & 928 & 150 & 1798 & Weighted Logistic & 1e-04 & 12 & +0.7 \\
      \textbf{- \textit{loose} 2step MF} & Position$_{2}$, Lab-Color (5D) & 5.00, 0.05, 0.30 & 665 & 2 & 928 & 150 & 1798 & Weighted Logistic & 1e-04 & 12 & +0.2\\
      \midrule
      \multicolumn{2}{c}{\textbf{Neural Network Applications}} & & & & & & & & &\\
      \textbf{Tiles: CNN-9$\times$9} & - & - & 81 & 4 & 10000 & 1000 & 1000 & Logistic & 0.01 & 100 & 500.0 \\
      \textbf{Tiles: CNN-13$\times$13} & - & - & 169 & 6 & 10000 & 1000 & 1000 & Logistic & 0.01 & 100 & 500.0 \\
      \textbf{Tiles: CNN-17$\times$17} & - & - & 289 & 8 & 10000 & 1000 & 1000 & Logistic & 0.01 & 100 & 500.0 \\
      \textbf{Tiles: CNN-21$\times$21} & - & - & 441 & 10 & 10000 & 1000 & 1000 & Logistic & 0.01 & 100 & 500.0 \\
      \textbf{Tiles: BNN} & Position$_{1}$, Color (5D) & 0.05, 0.04 & 63 & 1 & 10000 & 1000 & 1000 & Logistic & 0.01 & 100 & 30.0 \\
      \textbf{LeNet} & - & - & 25 & 2 & 5490 & 1098 & 1647 & Logistic & 0.1 & 100 & 182.2 \\
      \textbf{Crop-LeNet} & - & - & 25 & 2 & 5490 & 1098 & 1647 & Logistic & 0.1 & 100 & 182.2 \\
      \textbf{BNN-LeNet} & Position$_{2}$ (2D) & 20.00 & 7 & 1 & 5490 & 1098 & 1647 & Logistic & 0.1 & 100 & 182.2 \\
      \textbf{DeepCNet} & - & - & 9 & 1 & 5490 & 1098 & 1647 & Logistic & 0.1 & 100 & 182.2 \\
      \textbf{Crop-DeepCNet} & - & - & 9 & 1 & 5490 & 1098 & 1647 & Logistic & 0.1 & 100 & 182.2 \\
      \textbf{BNN-DeepCNet} & Position$_{2}$ (2D) & 40.00  & 7 & 1 & 5490 & 1098 & 1647 & Logistic & 0.1 & 100 & 182.2 \\
      \bottomrule
      \\
    \end{tabular}
    \mycaption{Experiment Protocols} {Experiment protocols for the different experiments presented in this work. \textbf{Feature Types}:
    Feature spaces used for the bilateral convolutions. Position$_1$ corresponds to un-normalized pixel positions whereas Position$_2$ corresponds
    to pixel positions normalized to $[0,1]$ with respect to the given image. \textbf{Feature Scales}: Cross-validated scales for the features used.
     \textbf{Filter Size}: Number of elements in the filter that is being learned. \textbf{Filter Nbr.}: Half-width of the filter. \textbf{Train},
     \textbf{Val.} and \textbf{Test} corresponds to the number of train, validation and test images used in the experiment. \textbf{Loss Type}: Type
     of loss used for back-propagation. ``MSE'' corresponds to Euclidean mean squared error loss and ``Logistic'' corresponds to multinomial logistic
     loss. ``Weighted Logistic'' is the class-weighted multinomial logistic loss. We weighted the loss with inverse class probability for material
     segmentation task due to the small availability of training data with class imbalance. \textbf{LR}: Fixed learning rate used in stochastic gradient
     descent. \textbf{Batch}: Number of images used in one parameter update step. \textbf{Epochs}: Number of training epochs. In all the experiments,
     we used fixed momentum of 0.9 and weight decay of 0.0005 for stochastic gradient descent. ```Color Upsampling'' experiments in this Table corresponds
     to those performed on Pascal VOC12 dataset images. For all experiments using Pascal VOC12 images, we use extended
     training segmentation dataset available from~\cite{hariharan2011moredata}, and used standard validation and test splits
     from the main dataset~\cite{voc2012segmentation}.}
  \label{tbl:parameters}
\end{table*}

\clearpage

\section{Parameters and Additional Results for Video Propagation Networks}

In this Section, we present experiment protocols and additional qualitative results for experiments
on video object segmentation, semantic video segmentation and video color
propagation. Table~\ref{tbl:parameters_supp} shows the feature scales and other parameters used in different experiments.
Figures~\ref{fig:video_seg_pos_supp} show some qualitative results on video object segmentation
with some failure cases in Fig.~\ref{fig:video_seg_neg_supp}.
Figure~\ref{fig:semantic_visuals_supp} shows some qualitative results on semantic video segmentation and
Fig.~\ref{fig:color_visuals_supp} shows results on video color propagation.

\newcolumntype{L}[1]{>{\raggedright\let\newline\\\arraybackslash\hspace{0pt}}b{#1}}
\newcolumntype{C}[1]{>{\centering\let\newline\\\arraybackslash\hspace{0pt}}b{#1}}
\newcolumntype{R}[1]{>{\raggedleft\let\newline\\\arraybackslash\hspace{0pt}}b{#1}}

\begin{table*}[h]
\tiny
  \centering
    \begin{tabular}{L{3.0cm} L{2.4cm} L{2.8cm} L{2.8cm} C{0.5cm} C{1.0cm} L{1.2cm}}
      \toprule
\textbf{Experiment} & \textbf{Feature Type} & \textbf{Feature Scale-1, $\Lambda_a$} & \textbf{Feature Scale-2, $\Lambda_b$} & \textbf{$\alpha$} & \textbf{Input Frames} & \textbf{Loss Type} \\
      \midrule
      \textbf{Video Object Segmentation} & ($x,y,Y,Cb,Cr,t$) & (0.02,0.02,0.07,0.4,0.4,0.01) & (0.03,0.03,0.09,0.5,0.5,0.2) & 0.5 & 9 & Logistic\\
      \midrule
      \textbf{Semantic Video Segmentation} & & & & & \\
      \textbf{with CNN1~\cite{yu2015multi}-NoFlow} & ($x,y,R,G,B,t$) & (0.08,0.08,0.2,0.2,0.2,0.04) & (0.11,0.11,0.2,0.2,0.2,0.04) & 0.5 & 3 & Logistic \\
      \textbf{with CNN1~\cite{yu2015multi}-Flow} & ($x+u_x,y+u_y,R,G,B,t$) & (0.11,0.11,0.14,0.14,0.14,0.03) & (0.08,0.08,0.12,0.12,0.12,0.01) & 0.65 & 3 & Logistic\\
      \textbf{with CNN2~\cite{richter2016playing}-Flow} & ($x+u_x,y+u_y,R,G,B,t$) & (0.08,0.08,0.2,0.2,0.2,0.04) & (0.09,0.09,0.25,0.25,0.25,0.03) & 0.5 & 4 & Logistic\\
      \midrule
      \textbf{Video Color Propagation} & ($x,y,I,t$)  & (0.04,0.04,0.2,0.04) & No second kernel & 1 & 4 & MSE\\
      \bottomrule
      \\
    \end{tabular}
    \mycaption{Experiment Protocols} {Experiment protocols for the different experiments presented in this work. \textbf{Feature Types}:
    Feature spaces used for the bilateral convolutions, with position ($x,y$) and color
    ($R,G,B$ or $Y,Cb,Cr$) features $\in [0,255]$. $u_x$, $u_y$ denotes optical flow with respect
    to the present frame and $I$ denotes grayscale intensity.
    \textbf{Feature Scales ($\Lambda_a, \Lambda_b$)}: Cross-validated scales for the features used.
    \textbf{$\alpha$}: Exponential time decay for the input frames.
    \textbf{Input Frames}: Number of input frames for VPN.
    \textbf{Loss Type}: Type
     of loss used for back-propagation. ``MSE'' corresponds to Euclidean mean squared error loss and ``Logistic'' corresponds to multinomial logistic loss.}
  \label{tbl:parameters_supp}
\end{table*}

% \begin{figure}[th!]
% \begin{center}
%   \centerline{\includegraphics[width=\textwidth]{figures/video_seg_visuals_supp_small.pdf}}
%     \mycaption{Video Object Segmentation}
%     {Shown are the different frames in example videos with the corresponding
%     ground truth (GT) masks, predictions from BVS~\cite{marki2016bilateral},
%     OFL~\cite{tsaivideo}, VPN (VPN-Stage2) and VPN-DLab (VPN-DeepLab) models.}
%     \label{fig:video_seg_small_supp}
% \end{center}
% \vspace{-1.0cm}
% \end{figure}

\begin{figure}[th!]
\begin{center}
  \centerline{\includegraphics[width=0.7\textwidth]{figures/video_seg_visuals_supp_positive.pdf}}
    \mycaption{Video Object Segmentation}
    {Shown are the different frames in example videos with the corresponding
    ground truth (GT) masks, predictions from BVS~\cite{marki2016bilateral},
    OFL~\cite{tsaivideo}, VPN (VPN-Stage2) and VPN-DLab (VPN-DeepLab) models.}
    \label{fig:video_seg_pos_supp}
\end{center}
\vspace{-1.0cm}
\end{figure}

\begin{figure}[th!]
\begin{center}
  \centerline{\includegraphics[width=0.7\textwidth]{figures/video_seg_visuals_supp_negative.pdf}}
    \mycaption{Failure Cases for Video Object Segmentation}
    {Shown are the different frames in example videos with the corresponding
    ground truth (GT) masks, predictions from BVS~\cite{marki2016bilateral},
    OFL~\cite{tsaivideo}, VPN (VPN-Stage2) and VPN-DLab (VPN-DeepLab) models.}
    \label{fig:video_seg_neg_supp}
\end{center}
\vspace{-1.0cm}
\end{figure}

\begin{figure}[th!]
\begin{center}
  \centerline{\includegraphics[width=0.9\textwidth]{figures/supp_semantic_visual.pdf}}
    \mycaption{Semantic Video Segmentation}
    {Input video frames and the corresponding ground truth (GT)
    segmentation together with the predictions of CNN~\cite{yu2015multi} and with
    VPN-Flow.}
    \label{fig:semantic_visuals_supp}
\end{center}
\vspace{-0.7cm}
\end{figure}

\begin{figure}[th!]
\begin{center}
  \centerline{\includegraphics[width=\textwidth]{figures/colorization_visuals_supp.pdf}}
  \mycaption{Video Color Propagation}
  {Input grayscale video frames and corresponding ground-truth (GT) color images
  together with color predictions of Levin et al.~\cite{levin2004colorization} and VPN-Stage1 models.}
  \label{fig:color_visuals_supp}
\end{center}
\vspace{-0.7cm}
\end{figure}

\clearpage

\section{Additional Material for Bilateral Inception Networks}
\label{sec:binception-app}

In this section of the Appendix, we first discuss the use of approximate bilateral
filtering in BI modules (Sec.~\ref{sec:lattice}).
Later, we present some qualitative results using different models for the approach presented in
Chapter~\ref{chap:binception} (Sec.~\ref{sec:qualitative-app}).

\subsection{Approximate Bilateral Filtering}
\label{sec:lattice}

The bilateral inception module presented in Chapter~\ref{chap:binception} computes a matrix-vector
product between a Gaussian filter $K$ and a vector of activations $\bz_c$.
Bilateral filtering is an important operation and many algorithmic techniques have been
proposed to speed-up this operation~\cite{paris2006fast,adams2010fast,gastal2011domain}.
In the main paper we opted to implement what can be considered the
brute-force variant of explicitly constructing $K$ and then using BLAS to compute the
matrix-vector product. This resulted in a few millisecond operation.
The explicit way to compute is possible due to the
reduction to super-pixels, e.g., it would not work for DenseCRF variants
that operate on the full image resolution.

Here, we present experiments where we use the fast approximate bilateral filtering
algorithm of~\cite{adams2010fast}, which is also used in Chapter~\ref{chap:bnn}
for learning sparse high dimensional filters. This
choice allows for larger dimensions of matrix-vector multiplication. The reason for choosing
the explicit multiplication in Chapter~\ref{chap:binception} was that it was computationally faster.
For the small sizes of the involved matrices and vectors, the explicit computation is sufficient and we had no
GPU implementation of an approximate technique that matched this runtime. Also it
is conceptually easier and the gradient to the feature transformations ($\Lambda \mathbf{f}$) is
obtained using standard matrix calculus.

\subsubsection{Experiments}

We modified the existing segmentation architectures analogous to those in Chapter~\ref{chap:binception}.
The main difference is that, here, the inception modules use the lattice
approximation~\cite{adams2010fast} to compute the bilateral filtering.
Using the lattice approximation did not allow us to back-propagate through feature transformations ($\Lambda$)
and thus we used hand-specified feature scales as will be explained later.
Specifically, we take CNN architectures from the works
of~\cite{chen2014semantic,zheng2015conditional,bell2015minc} and insert the BI modules between
the spatial FC layers.
We use superpixels from~\cite{DollarICCV13edges}
for all the experiments with the lattice approximation. Experiments are
performed using Caffe neural network framework~\cite{jia2014caffe}.

\begin{table}
  \small
  \centering
  \begin{tabular}{p{5.5cm}>{\raggedright\arraybackslash}p{1.4cm}>{\centering\arraybackslash}p{2.2cm}}
    \toprule
		\textbf{Model} & \emph{IoU} & \emph{Runtime}(ms) \\
    \midrule

    %%%%%%%%%%%% Scores computed by us)%%%%%%%%%%%%
		\deeplablargefov & 68.9 & 145ms\\
    \midrule
    \bi{7}{2}-\bi{8}{10}& \textbf{73.8} & +600 \\
    \midrule
    \deeplablargefovcrf~\cite{chen2014semantic} & 72.7 & +830\\
    \deeplabmsclargefovcrf~\cite{chen2014semantic} & \textbf{73.6} & +880\\
    DeepLab-EdgeNet~\cite{chen2015semantic} & 71.7 & +30\\
    DeepLab-EdgeNet-CRF~\cite{chen2015semantic} & \textbf{73.6} & +860\\
  \bottomrule \\
  \end{tabular}
  \mycaption{Semantic Segmentation using the DeepLab model}
  {IoU scores on the Pascal VOC12 segmentation test dataset
  with different models and our modified inception model.
  Also shown are the corresponding runtimes in milliseconds. Runtimes
  also include superpixel computations (300 ms with Dollar superpixels~\cite{DollarICCV13edges})}
  \label{tab:largefovresults}
\end{table}

\paragraph{Semantic Segmentation}
The experiments in this section use the Pascal VOC12 segmentation dataset~\cite{voc2012segmentation} with 21 object classes and the images have a maximum resolution of 0.25 megapixels.
For all experiments on VOC12, we train using the extended training set of
10581 images collected by~\cite{hariharan2011moredata}.
We modified the \deeplab~network architecture of~\cite{chen2014semantic} and
the CRFasRNN architecture from~\cite{zheng2015conditional} which uses a CNN with
deconvolution layers followed by DenseCRF trained end-to-end.

\paragraph{DeepLab Model}\label{sec:deeplabmodel}
We experimented with the \bi{7}{2}-\bi{8}{10} inception model.
Results using the~\deeplab~model are summarized in Tab.~\ref{tab:largefovresults}.
Although we get similar improvements with inception modules as with the
explicit kernel computation, using lattice approximation is slower.

\begin{table}
  \small
  \centering
  \begin{tabular}{p{6.4cm}>{\raggedright\arraybackslash}p{1.8cm}>{\raggedright\arraybackslash}p{1.8cm}}
    \toprule
    \textbf{Model} & \emph{IoU (Val)} & \emph{IoU (Test)}\\
    \midrule
    %%%%%%%%%%%% Scores computed by us)%%%%%%%%%%%%
    CNN &  67.5 & - \\
    \deconv (CNN+Deconvolutions) & 69.8 & 72.0 \\
    \midrule
    \bi{3}{6}-\bi{4}{6}-\bi{7}{2}-\bi{8}{6}& 71.9 & - \\
    \bi{3}{6}-\bi{4}{6}-\bi{7}{2}-\bi{8}{6}-\gi{6}& 73.6 &  \href{http://host.robots.ox.ac.uk:8080/anonymous/VOTV5E.html}{\textbf{75.2}}\\
    \midrule
    \deconvcrf (CRF-RNN)~\cite{zheng2015conditional} & 73.0 & 74.7\\
    Context-CRF-RNN~\cite{yu2015multi} & ~~ - ~ & \textbf{75.3} \\
    \bottomrule \\
  \end{tabular}
  \mycaption{Semantic Segmentation using the CRFasRNN model}{IoU score corresponding to different models
  on Pascal VOC12 reduced validation / test segmentation dataset. The reduced validation set consists of 346 images
  as used in~\cite{zheng2015conditional} where we adapted the model from.}
  \label{tab:deconvresults-app}
\end{table}

\paragraph{CRFasRNN Model}\label{sec:deepinception}
We add BI modules after score-pool3, score-pool4, \fc{7} and \fc{8} $1\times1$ convolution layers
resulting in the \bi{3}{6}-\bi{4}{6}-\bi{7}{2}-\bi{8}{6}
model and also experimented with another variant where $BI_8$ is followed by another inception
module, G$(6)$, with 6 Gaussian kernels.
Note that here also we discarded both deconvolution and DenseCRF parts of the original model~\cite{zheng2015conditional}
and inserted the BI modules in the base CNN and found similar improvements compared to the inception modules with explicit
kernel computaion. See Tab.~\ref{tab:deconvresults-app} for results on the CRFasRNN model.

\paragraph{Material Segmentation}
Table~\ref{tab:mincresults-app} shows the results on the MINC dataset~\cite{bell2015minc}
obtained by modifying the AlexNet architecture with our inception modules. We observe
similar improvements as with explicit kernel construction.
For this model, we do not provide any learned setup due to very limited segment training
data. The weights to combine outputs in the bilateral inception layer are
found by validation on the validation set.

\begin{table}[t]
  \small
  \centering
  \begin{tabular}{p{3.5cm}>{\centering\arraybackslash}p{4.0cm}}
    \toprule
    \textbf{Model} & Class / Total accuracy\\
    \midrule

    %%%%%%%%%%%% Scores computed by us)%%%%%%%%%%%%
    AlexNet CNN & 55.3 / 58.9 \\
    \midrule
    \bi{7}{2}-\bi{8}{6}& 68.5 / 71.8 \\
    \bi{7}{2}-\bi{8}{6}-G$(6)$& 67.6 / 73.1 \\
    \midrule
    AlexNet-CRF & 65.5 / 71.0 \\
    \bottomrule \\
  \end{tabular}
  \mycaption{Material Segmentation using AlexNet}{Pixel accuracy of different models on
  the MINC material segmentation test dataset~\cite{bell2015minc}.}
  \label{tab:mincresults-app}
\end{table}

\paragraph{Scales of Bilateral Inception Modules}
\label{sec:scales}

Unlike the explicit kernel technique presented in the main text (Chapter~\ref{chap:binception}),
we didn't back-propagate through feature transformation ($\Lambda$)
using the approximate bilateral filter technique.
So, the feature scales are hand-specified and validated, which are as follows.
The optimal scale values for the \bi{7}{2}-\bi{8}{2} model are found by validation for the best performance which are
$\sigma_{xy}$ = (0.1, 0.1) for the spatial (XY) kernel and $\sigma_{rgbxy}$ = (0.1, 0.1, 0.1, 0.01, 0.01) for color and position (RGBXY)  kernel.
Next, as more kernels are added to \bi{8}{2}, we set scales to be $\alpha$*($\sigma_{xy}$, $\sigma_{rgbxy}$).
The value of $\alpha$ is chosen as  1, 0.5, 0.1, 0.05, 0.1, at uniform interval, for the \bi{8}{10} bilateral inception module.


\subsection{Qualitative Results}
\label{sec:qualitative-app}

In this section, we present more qualitative results obtained using the BI module with explicit
kernel computation technique presented in Chapter~\ref{chap:binception}. Results on the Pascal VOC12
dataset~\cite{voc2012segmentation} using the DeepLab-LargeFOV model are shown in Fig.~\ref{fig:semantic_visuals-app},
followed by the results on MINC dataset~\cite{bell2015minc}
in Fig.~\ref{fig:material_visuals-app} and on
Cityscapes dataset~\cite{Cordts2015Cvprw} in Fig.~\ref{fig:street_visuals-app}.


\definecolor{voc_1}{RGB}{0, 0, 0}
\definecolor{voc_2}{RGB}{128, 0, 0}
\definecolor{voc_3}{RGB}{0, 128, 0}
\definecolor{voc_4}{RGB}{128, 128, 0}
\definecolor{voc_5}{RGB}{0, 0, 128}
\definecolor{voc_6}{RGB}{128, 0, 128}
\definecolor{voc_7}{RGB}{0, 128, 128}
\definecolor{voc_8}{RGB}{128, 128, 128}
\definecolor{voc_9}{RGB}{64, 0, 0}
\definecolor{voc_10}{RGB}{192, 0, 0}
\definecolor{voc_11}{RGB}{64, 128, 0}
\definecolor{voc_12}{RGB}{192, 128, 0}
\definecolor{voc_13}{RGB}{64, 0, 128}
\definecolor{voc_14}{RGB}{192, 0, 128}
\definecolor{voc_15}{RGB}{64, 128, 128}
\definecolor{voc_16}{RGB}{192, 128, 128}
\definecolor{voc_17}{RGB}{0, 64, 0}
\definecolor{voc_18}{RGB}{128, 64, 0}
\definecolor{voc_19}{RGB}{0, 192, 0}
\definecolor{voc_20}{RGB}{128, 192, 0}
\definecolor{voc_21}{RGB}{0, 64, 128}
\definecolor{voc_22}{RGB}{128, 64, 128}

\begin{figure*}[!ht]
  \small
  \centering
  \fcolorbox{white}{voc_1}{\rule{0pt}{4pt}\rule{4pt}{0pt}} Background~~
  \fcolorbox{white}{voc_2}{\rule{0pt}{4pt}\rule{4pt}{0pt}} Aeroplane~~
  \fcolorbox{white}{voc_3}{\rule{0pt}{4pt}\rule{4pt}{0pt}} Bicycle~~
  \fcolorbox{white}{voc_4}{\rule{0pt}{4pt}\rule{4pt}{0pt}} Bird~~
  \fcolorbox{white}{voc_5}{\rule{0pt}{4pt}\rule{4pt}{0pt}} Boat~~
  \fcolorbox{white}{voc_6}{\rule{0pt}{4pt}\rule{4pt}{0pt}} Bottle~~
  \fcolorbox{white}{voc_7}{\rule{0pt}{4pt}\rule{4pt}{0pt}} Bus~~
  \fcolorbox{white}{voc_8}{\rule{0pt}{4pt}\rule{4pt}{0pt}} Car~~\\
  \fcolorbox{white}{voc_9}{\rule{0pt}{4pt}\rule{4pt}{0pt}} Cat~~
  \fcolorbox{white}{voc_10}{\rule{0pt}{4pt}\rule{4pt}{0pt}} Chair~~
  \fcolorbox{white}{voc_11}{\rule{0pt}{4pt}\rule{4pt}{0pt}} Cow~~
  \fcolorbox{white}{voc_12}{\rule{0pt}{4pt}\rule{4pt}{0pt}} Dining Table~~
  \fcolorbox{white}{voc_13}{\rule{0pt}{4pt}\rule{4pt}{0pt}} Dog~~
  \fcolorbox{white}{voc_14}{\rule{0pt}{4pt}\rule{4pt}{0pt}} Horse~~
  \fcolorbox{white}{voc_15}{\rule{0pt}{4pt}\rule{4pt}{0pt}} Motorbike~~
  \fcolorbox{white}{voc_16}{\rule{0pt}{4pt}\rule{4pt}{0pt}} Person~~\\
  \fcolorbox{white}{voc_17}{\rule{0pt}{4pt}\rule{4pt}{0pt}} Potted Plant~~
  \fcolorbox{white}{voc_18}{\rule{0pt}{4pt}\rule{4pt}{0pt}} Sheep~~
  \fcolorbox{white}{voc_19}{\rule{0pt}{4pt}\rule{4pt}{0pt}} Sofa~~
  \fcolorbox{white}{voc_20}{\rule{0pt}{4pt}\rule{4pt}{0pt}} Train~~
  \fcolorbox{white}{voc_21}{\rule{0pt}{4pt}\rule{4pt}{0pt}} TV monitor~~\\


  \subfigure{%
    \includegraphics[width=.15\columnwidth]{figures/supplementary/2008_001308_given.png}
  }
  \subfigure{%
    \includegraphics[width=.15\columnwidth]{figures/supplementary/2008_001308_sp.png}
  }
  \subfigure{%
    \includegraphics[width=.15\columnwidth]{figures/supplementary/2008_001308_gt.png}
  }
  \subfigure{%
    \includegraphics[width=.15\columnwidth]{figures/supplementary/2008_001308_cnn.png}
  }
  \subfigure{%
    \includegraphics[width=.15\columnwidth]{figures/supplementary/2008_001308_crf.png}
  }
  \subfigure{%
    \includegraphics[width=.15\columnwidth]{figures/supplementary/2008_001308_ours.png}
  }\\[-2ex]


  \subfigure{%
    \includegraphics[width=.15\columnwidth]{figures/supplementary/2008_001821_given.png}
  }
  \subfigure{%
    \includegraphics[width=.15\columnwidth]{figures/supplementary/2008_001821_sp.png}
  }
  \subfigure{%
    \includegraphics[width=.15\columnwidth]{figures/supplementary/2008_001821_gt.png}
  }
  \subfigure{%
    \includegraphics[width=.15\columnwidth]{figures/supplementary/2008_001821_cnn.png}
  }
  \subfigure{%
    \includegraphics[width=.15\columnwidth]{figures/supplementary/2008_001821_crf.png}
  }
  \subfigure{%
    \includegraphics[width=.15\columnwidth]{figures/supplementary/2008_001821_ours.png}
  }\\[-2ex]



  \subfigure{%
    \includegraphics[width=.15\columnwidth]{figures/supplementary/2008_004612_given.png}
  }
  \subfigure{%
    \includegraphics[width=.15\columnwidth]{figures/supplementary/2008_004612_sp.png}
  }
  \subfigure{%
    \includegraphics[width=.15\columnwidth]{figures/supplementary/2008_004612_gt.png}
  }
  \subfigure{%
    \includegraphics[width=.15\columnwidth]{figures/supplementary/2008_004612_cnn.png}
  }
  \subfigure{%
    \includegraphics[width=.15\columnwidth]{figures/supplementary/2008_004612_crf.png}
  }
  \subfigure{%
    \includegraphics[width=.15\columnwidth]{figures/supplementary/2008_004612_ours.png}
  }\\[-2ex]


  \subfigure{%
    \includegraphics[width=.15\columnwidth]{figures/supplementary/2009_001008_given.png}
  }
  \subfigure{%
    \includegraphics[width=.15\columnwidth]{figures/supplementary/2009_001008_sp.png}
  }
  \subfigure{%
    \includegraphics[width=.15\columnwidth]{figures/supplementary/2009_001008_gt.png}
  }
  \subfigure{%
    \includegraphics[width=.15\columnwidth]{figures/supplementary/2009_001008_cnn.png}
  }
  \subfigure{%
    \includegraphics[width=.15\columnwidth]{figures/supplementary/2009_001008_crf.png}
  }
  \subfigure{%
    \includegraphics[width=.15\columnwidth]{figures/supplementary/2009_001008_ours.png}
  }\\[-2ex]




  \subfigure{%
    \includegraphics[width=.15\columnwidth]{figures/supplementary/2009_004497_given.png}
  }
  \subfigure{%
    \includegraphics[width=.15\columnwidth]{figures/supplementary/2009_004497_sp.png}
  }
  \subfigure{%
    \includegraphics[width=.15\columnwidth]{figures/supplementary/2009_004497_gt.png}
  }
  \subfigure{%
    \includegraphics[width=.15\columnwidth]{figures/supplementary/2009_004497_cnn.png}
  }
  \subfigure{%
    \includegraphics[width=.15\columnwidth]{figures/supplementary/2009_004497_crf.png}
  }
  \subfigure{%
    \includegraphics[width=.15\columnwidth]{figures/supplementary/2009_004497_ours.png}
  }\\[-2ex]



  \setcounter{subfigure}{0}
  \subfigure[\scriptsize Input]{%
    \includegraphics[width=.15\columnwidth]{figures/supplementary/2010_001327_given.png}
  }
  \subfigure[\scriptsize Superpixels]{%
    \includegraphics[width=.15\columnwidth]{figures/supplementary/2010_001327_sp.png}
  }
  \subfigure[\scriptsize GT]{%
    \includegraphics[width=.15\columnwidth]{figures/supplementary/2010_001327_gt.png}
  }
  \subfigure[\scriptsize Deeplab]{%
    \includegraphics[width=.15\columnwidth]{figures/supplementary/2010_001327_cnn.png}
  }
  \subfigure[\scriptsize +DenseCRF]{%
    \includegraphics[width=.15\columnwidth]{figures/supplementary/2010_001327_crf.png}
  }
  \subfigure[\scriptsize Using BI]{%
    \includegraphics[width=.15\columnwidth]{figures/supplementary/2010_001327_ours.png}
  }
  \mycaption{Semantic Segmentation}{Example results of semantic segmentation
  on the Pascal VOC12 dataset.
  (d)~depicts the DeepLab CNN result, (e)~CNN + 10 steps of mean-field inference,
  (f~result obtained with bilateral inception (BI) modules (\bi{6}{2}+\bi{7}{6}) between \fc~layers.}
  \label{fig:semantic_visuals-app}
\end{figure*}


\definecolor{minc_1}{HTML}{771111}
\definecolor{minc_2}{HTML}{CAC690}
\definecolor{minc_3}{HTML}{EEEEEE}
\definecolor{minc_4}{HTML}{7C8FA6}
\definecolor{minc_5}{HTML}{597D31}
\definecolor{minc_6}{HTML}{104410}
\definecolor{minc_7}{HTML}{BB819C}
\definecolor{minc_8}{HTML}{D0CE48}
\definecolor{minc_9}{HTML}{622745}
\definecolor{minc_10}{HTML}{666666}
\definecolor{minc_11}{HTML}{D54A31}
\definecolor{minc_12}{HTML}{101044}
\definecolor{minc_13}{HTML}{444126}
\definecolor{minc_14}{HTML}{75D646}
\definecolor{minc_15}{HTML}{DD4348}
\definecolor{minc_16}{HTML}{5C8577}
\definecolor{minc_17}{HTML}{C78472}
\definecolor{minc_18}{HTML}{75D6D0}
\definecolor{minc_19}{HTML}{5B4586}
\definecolor{minc_20}{HTML}{C04393}
\definecolor{minc_21}{HTML}{D69948}
\definecolor{minc_22}{HTML}{7370D8}
\definecolor{minc_23}{HTML}{7A3622}
\definecolor{minc_24}{HTML}{000000}

\begin{figure*}[!ht]
  \small % scriptsize
  \centering
  \fcolorbox{white}{minc_1}{\rule{0pt}{4pt}\rule{4pt}{0pt}} Brick~~
  \fcolorbox{white}{minc_2}{\rule{0pt}{4pt}\rule{4pt}{0pt}} Carpet~~
  \fcolorbox{white}{minc_3}{\rule{0pt}{4pt}\rule{4pt}{0pt}} Ceramic~~
  \fcolorbox{white}{minc_4}{\rule{0pt}{4pt}\rule{4pt}{0pt}} Fabric~~
  \fcolorbox{white}{minc_5}{\rule{0pt}{4pt}\rule{4pt}{0pt}} Foliage~~
  \fcolorbox{white}{minc_6}{\rule{0pt}{4pt}\rule{4pt}{0pt}} Food~~
  \fcolorbox{white}{minc_7}{\rule{0pt}{4pt}\rule{4pt}{0pt}} Glass~~
  \fcolorbox{white}{minc_8}{\rule{0pt}{4pt}\rule{4pt}{0pt}} Hair~~\\
  \fcolorbox{white}{minc_9}{\rule{0pt}{4pt}\rule{4pt}{0pt}} Leather~~
  \fcolorbox{white}{minc_10}{\rule{0pt}{4pt}\rule{4pt}{0pt}} Metal~~
  \fcolorbox{white}{minc_11}{\rule{0pt}{4pt}\rule{4pt}{0pt}} Mirror~~
  \fcolorbox{white}{minc_12}{\rule{0pt}{4pt}\rule{4pt}{0pt}} Other~~
  \fcolorbox{white}{minc_13}{\rule{0pt}{4pt}\rule{4pt}{0pt}} Painted~~
  \fcolorbox{white}{minc_14}{\rule{0pt}{4pt}\rule{4pt}{0pt}} Paper~~
  \fcolorbox{white}{minc_15}{\rule{0pt}{4pt}\rule{4pt}{0pt}} Plastic~~\\
  \fcolorbox{white}{minc_16}{\rule{0pt}{4pt}\rule{4pt}{0pt}} Polished Stone~~
  \fcolorbox{white}{minc_17}{\rule{0pt}{4pt}\rule{4pt}{0pt}} Skin~~
  \fcolorbox{white}{minc_18}{\rule{0pt}{4pt}\rule{4pt}{0pt}} Sky~~
  \fcolorbox{white}{minc_19}{\rule{0pt}{4pt}\rule{4pt}{0pt}} Stone~~
  \fcolorbox{white}{minc_20}{\rule{0pt}{4pt}\rule{4pt}{0pt}} Tile~~
  \fcolorbox{white}{minc_21}{\rule{0pt}{4pt}\rule{4pt}{0pt}} Wallpaper~~
  \fcolorbox{white}{minc_22}{\rule{0pt}{4pt}\rule{4pt}{0pt}} Water~~
  \fcolorbox{white}{minc_23}{\rule{0pt}{4pt}\rule{4pt}{0pt}} Wood~~\\
  \subfigure{%
    \includegraphics[width=.15\columnwidth]{figures/supplementary/000008468_given.png}
  }
  \subfigure{%
    \includegraphics[width=.15\columnwidth]{figures/supplementary/000008468_sp.png}
  }
  \subfigure{%
    \includegraphics[width=.15\columnwidth]{figures/supplementary/000008468_gt.png}
  }
  \subfigure{%
    \includegraphics[width=.15\columnwidth]{figures/supplementary/000008468_cnn.png}
  }
  \subfigure{%
    \includegraphics[width=.15\columnwidth]{figures/supplementary/000008468_crf.png}
  }
  \subfigure{%
    \includegraphics[width=.15\columnwidth]{figures/supplementary/000008468_ours.png}
  }\\[-2ex]

  \subfigure{%
    \includegraphics[width=.15\columnwidth]{figures/supplementary/000009053_given.png}
  }
  \subfigure{%
    \includegraphics[width=.15\columnwidth]{figures/supplementary/000009053_sp.png}
  }
  \subfigure{%
    \includegraphics[width=.15\columnwidth]{figures/supplementary/000009053_gt.png}
  }
  \subfigure{%
    \includegraphics[width=.15\columnwidth]{figures/supplementary/000009053_cnn.png}
  }
  \subfigure{%
    \includegraphics[width=.15\columnwidth]{figures/supplementary/000009053_crf.png}
  }
  \subfigure{%
    \includegraphics[width=.15\columnwidth]{figures/supplementary/000009053_ours.png}
  }\\[-2ex]




  \subfigure{%
    \includegraphics[width=.15\columnwidth]{figures/supplementary/000014977_given.png}
  }
  \subfigure{%
    \includegraphics[width=.15\columnwidth]{figures/supplementary/000014977_sp.png}
  }
  \subfigure{%
    \includegraphics[width=.15\columnwidth]{figures/supplementary/000014977_gt.png}
  }
  \subfigure{%
    \includegraphics[width=.15\columnwidth]{figures/supplementary/000014977_cnn.png}
  }
  \subfigure{%
    \includegraphics[width=.15\columnwidth]{figures/supplementary/000014977_crf.png}
  }
  \subfigure{%
    \includegraphics[width=.15\columnwidth]{figures/supplementary/000014977_ours.png}
  }\\[-2ex]


  \subfigure{%
    \includegraphics[width=.15\columnwidth]{figures/supplementary/000022922_given.png}
  }
  \subfigure{%
    \includegraphics[width=.15\columnwidth]{figures/supplementary/000022922_sp.png}
  }
  \subfigure{%
    \includegraphics[width=.15\columnwidth]{figures/supplementary/000022922_gt.png}
  }
  \subfigure{%
    \includegraphics[width=.15\columnwidth]{figures/supplementary/000022922_cnn.png}
  }
  \subfigure{%
    \includegraphics[width=.15\columnwidth]{figures/supplementary/000022922_crf.png}
  }
  \subfigure{%
    \includegraphics[width=.15\columnwidth]{figures/supplementary/000022922_ours.png}
  }\\[-2ex]


  \subfigure{%
    \includegraphics[width=.15\columnwidth]{figures/supplementary/000025711_given.png}
  }
  \subfigure{%
    \includegraphics[width=.15\columnwidth]{figures/supplementary/000025711_sp.png}
  }
  \subfigure{%
    \includegraphics[width=.15\columnwidth]{figures/supplementary/000025711_gt.png}
  }
  \subfigure{%
    \includegraphics[width=.15\columnwidth]{figures/supplementary/000025711_cnn.png}
  }
  \subfigure{%
    \includegraphics[width=.15\columnwidth]{figures/supplementary/000025711_crf.png}
  }
  \subfigure{%
    \includegraphics[width=.15\columnwidth]{figures/supplementary/000025711_ours.png}
  }\\[-2ex]


  \subfigure{%
    \includegraphics[width=.15\columnwidth]{figures/supplementary/000034473_given.png}
  }
  \subfigure{%
    \includegraphics[width=.15\columnwidth]{figures/supplementary/000034473_sp.png}
  }
  \subfigure{%
    \includegraphics[width=.15\columnwidth]{figures/supplementary/000034473_gt.png}
  }
  \subfigure{%
    \includegraphics[width=.15\columnwidth]{figures/supplementary/000034473_cnn.png}
  }
  \subfigure{%
    \includegraphics[width=.15\columnwidth]{figures/supplementary/000034473_crf.png}
  }
  \subfigure{%
    \includegraphics[width=.15\columnwidth]{figures/supplementary/000034473_ours.png}
  }\\[-2ex]


  \subfigure{%
    \includegraphics[width=.15\columnwidth]{figures/supplementary/000035463_given.png}
  }
  \subfigure{%
    \includegraphics[width=.15\columnwidth]{figures/supplementary/000035463_sp.png}
  }
  \subfigure{%
    \includegraphics[width=.15\columnwidth]{figures/supplementary/000035463_gt.png}
  }
  \subfigure{%
    \includegraphics[width=.15\columnwidth]{figures/supplementary/000035463_cnn.png}
  }
  \subfigure{%
    \includegraphics[width=.15\columnwidth]{figures/supplementary/000035463_crf.png}
  }
  \subfigure{%
    \includegraphics[width=.15\columnwidth]{figures/supplementary/000035463_ours.png}
  }\\[-2ex]


  \setcounter{subfigure}{0}
  \subfigure[\scriptsize Input]{%
    \includegraphics[width=.15\columnwidth]{figures/supplementary/000035993_given.png}
  }
  \subfigure[\scriptsize Superpixels]{%
    \includegraphics[width=.15\columnwidth]{figures/supplementary/000035993_sp.png}
  }
  \subfigure[\scriptsize GT]{%
    \includegraphics[width=.15\columnwidth]{figures/supplementary/000035993_gt.png}
  }
  \subfigure[\scriptsize AlexNet]{%
    \includegraphics[width=.15\columnwidth]{figures/supplementary/000035993_cnn.png}
  }
  \subfigure[\scriptsize +DenseCRF]{%
    \includegraphics[width=.15\columnwidth]{figures/supplementary/000035993_crf.png}
  }
  \subfigure[\scriptsize Using BI]{%
    \includegraphics[width=.15\columnwidth]{figures/supplementary/000035993_ours.png}
  }
  \mycaption{Material Segmentation}{Example results of material segmentation.
  (d)~depicts the AlexNet CNN result, (e)~CNN + 10 steps of mean-field inference,
  (f)~result obtained with bilateral inception (BI) modules (\bi{7}{2}+\bi{8}{6}) between
  \fc~layers.}
\label{fig:material_visuals-app}
\end{figure*}


\definecolor{city_1}{RGB}{128, 64, 128}
\definecolor{city_2}{RGB}{244, 35, 232}
\definecolor{city_3}{RGB}{70, 70, 70}
\definecolor{city_4}{RGB}{102, 102, 156}
\definecolor{city_5}{RGB}{190, 153, 153}
\definecolor{city_6}{RGB}{153, 153, 153}
\definecolor{city_7}{RGB}{250, 170, 30}
\definecolor{city_8}{RGB}{220, 220, 0}
\definecolor{city_9}{RGB}{107, 142, 35}
\definecolor{city_10}{RGB}{152, 251, 152}
\definecolor{city_11}{RGB}{70, 130, 180}
\definecolor{city_12}{RGB}{220, 20, 60}
\definecolor{city_13}{RGB}{255, 0, 0}
\definecolor{city_14}{RGB}{0, 0, 142}
\definecolor{city_15}{RGB}{0, 0, 70}
\definecolor{city_16}{RGB}{0, 60, 100}
\definecolor{city_17}{RGB}{0, 80, 100}
\definecolor{city_18}{RGB}{0, 0, 230}
\definecolor{city_19}{RGB}{119, 11, 32}
\begin{figure*}[!ht]
  \small % scriptsize
  \centering


  \subfigure{%
    \includegraphics[width=.18\columnwidth]{figures/supplementary/frankfurt00000_016005_given.png}
  }
  \subfigure{%
    \includegraphics[width=.18\columnwidth]{figures/supplementary/frankfurt00000_016005_sp.png}
  }
  \subfigure{%
    \includegraphics[width=.18\columnwidth]{figures/supplementary/frankfurt00000_016005_gt.png}
  }
  \subfigure{%
    \includegraphics[width=.18\columnwidth]{figures/supplementary/frankfurt00000_016005_cnn.png}
  }
  \subfigure{%
    \includegraphics[width=.18\columnwidth]{figures/supplementary/frankfurt00000_016005_ours.png}
  }\\[-2ex]

  \subfigure{%
    \includegraphics[width=.18\columnwidth]{figures/supplementary/frankfurt00000_004617_given.png}
  }
  \subfigure{%
    \includegraphics[width=.18\columnwidth]{figures/supplementary/frankfurt00000_004617_sp.png}
  }
  \subfigure{%
    \includegraphics[width=.18\columnwidth]{figures/supplementary/frankfurt00000_004617_gt.png}
  }
  \subfigure{%
    \includegraphics[width=.18\columnwidth]{figures/supplementary/frankfurt00000_004617_cnn.png}
  }
  \subfigure{%
    \includegraphics[width=.18\columnwidth]{figures/supplementary/frankfurt00000_004617_ours.png}
  }\\[-2ex]

  \subfigure{%
    \includegraphics[width=.18\columnwidth]{figures/supplementary/frankfurt00000_020880_given.png}
  }
  \subfigure{%
    \includegraphics[width=.18\columnwidth]{figures/supplementary/frankfurt00000_020880_sp.png}
  }
  \subfigure{%
    \includegraphics[width=.18\columnwidth]{figures/supplementary/frankfurt00000_020880_gt.png}
  }
  \subfigure{%
    \includegraphics[width=.18\columnwidth]{figures/supplementary/frankfurt00000_020880_cnn.png}
  }
  \subfigure{%
    \includegraphics[width=.18\columnwidth]{figures/supplementary/frankfurt00000_020880_ours.png}
  }\\[-2ex]



  \subfigure{%
    \includegraphics[width=.18\columnwidth]{figures/supplementary/frankfurt00001_007285_given.png}
  }
  \subfigure{%
    \includegraphics[width=.18\columnwidth]{figures/supplementary/frankfurt00001_007285_sp.png}
  }
  \subfigure{%
    \includegraphics[width=.18\columnwidth]{figures/supplementary/frankfurt00001_007285_gt.png}
  }
  \subfigure{%
    \includegraphics[width=.18\columnwidth]{figures/supplementary/frankfurt00001_007285_cnn.png}
  }
  \subfigure{%
    \includegraphics[width=.18\columnwidth]{figures/supplementary/frankfurt00001_007285_ours.png}
  }\\[-2ex]


  \subfigure{%
    \includegraphics[width=.18\columnwidth]{figures/supplementary/frankfurt00001_059789_given.png}
  }
  \subfigure{%
    \includegraphics[width=.18\columnwidth]{figures/supplementary/frankfurt00001_059789_sp.png}
  }
  \subfigure{%
    \includegraphics[width=.18\columnwidth]{figures/supplementary/frankfurt00001_059789_gt.png}
  }
  \subfigure{%
    \includegraphics[width=.18\columnwidth]{figures/supplementary/frankfurt00001_059789_cnn.png}
  }
  \subfigure{%
    \includegraphics[width=.18\columnwidth]{figures/supplementary/frankfurt00001_059789_ours.png}
  }\\[-2ex]


  \subfigure{%
    \includegraphics[width=.18\columnwidth]{figures/supplementary/frankfurt00001_068208_given.png}
  }
  \subfigure{%
    \includegraphics[width=.18\columnwidth]{figures/supplementary/frankfurt00001_068208_sp.png}
  }
  \subfigure{%
    \includegraphics[width=.18\columnwidth]{figures/supplementary/frankfurt00001_068208_gt.png}
  }
  \subfigure{%
    \includegraphics[width=.18\columnwidth]{figures/supplementary/frankfurt00001_068208_cnn.png}
  }
  \subfigure{%
    \includegraphics[width=.18\columnwidth]{figures/supplementary/frankfurt00001_068208_ours.png}
  }\\[-2ex]

  \subfigure{%
    \includegraphics[width=.18\columnwidth]{figures/supplementary/frankfurt00001_082466_given.png}
  }
  \subfigure{%
    \includegraphics[width=.18\columnwidth]{figures/supplementary/frankfurt00001_082466_sp.png}
  }
  \subfigure{%
    \includegraphics[width=.18\columnwidth]{figures/supplementary/frankfurt00001_082466_gt.png}
  }
  \subfigure{%
    \includegraphics[width=.18\columnwidth]{figures/supplementary/frankfurt00001_082466_cnn.png}
  }
  \subfigure{%
    \includegraphics[width=.18\columnwidth]{figures/supplementary/frankfurt00001_082466_ours.png}
  }\\[-2ex]

  \subfigure{%
    \includegraphics[width=.18\columnwidth]{figures/supplementary/lindau00033_000019_given.png}
  }
  \subfigure{%
    \includegraphics[width=.18\columnwidth]{figures/supplementary/lindau00033_000019_sp.png}
  }
  \subfigure{%
    \includegraphics[width=.18\columnwidth]{figures/supplementary/lindau00033_000019_gt.png}
  }
  \subfigure{%
    \includegraphics[width=.18\columnwidth]{figures/supplementary/lindau00033_000019_cnn.png}
  }
  \subfigure{%
    \includegraphics[width=.18\columnwidth]{figures/supplementary/lindau00033_000019_ours.png}
  }\\[-2ex]

  \subfigure{%
    \includegraphics[width=.18\columnwidth]{figures/supplementary/lindau00052_000019_given.png}
  }
  \subfigure{%
    \includegraphics[width=.18\columnwidth]{figures/supplementary/lindau00052_000019_sp.png}
  }
  \subfigure{%
    \includegraphics[width=.18\columnwidth]{figures/supplementary/lindau00052_000019_gt.png}
  }
  \subfigure{%
    \includegraphics[width=.18\columnwidth]{figures/supplementary/lindau00052_000019_cnn.png}
  }
  \subfigure{%
    \includegraphics[width=.18\columnwidth]{figures/supplementary/lindau00052_000019_ours.png}
  }\\[-2ex]




  \subfigure{%
    \includegraphics[width=.18\columnwidth]{figures/supplementary/lindau00027_000019_given.png}
  }
  \subfigure{%
    \includegraphics[width=.18\columnwidth]{figures/supplementary/lindau00027_000019_sp.png}
  }
  \subfigure{%
    \includegraphics[width=.18\columnwidth]{figures/supplementary/lindau00027_000019_gt.png}
  }
  \subfigure{%
    \includegraphics[width=.18\columnwidth]{figures/supplementary/lindau00027_000019_cnn.png}
  }
  \subfigure{%
    \includegraphics[width=.18\columnwidth]{figures/supplementary/lindau00027_000019_ours.png}
  }\\[-2ex]



  \setcounter{subfigure}{0}
  \subfigure[\scriptsize Input]{%
    \includegraphics[width=.18\columnwidth]{figures/supplementary/lindau00029_000019_given.png}
  }
  \subfigure[\scriptsize Superpixels]{%
    \includegraphics[width=.18\columnwidth]{figures/supplementary/lindau00029_000019_sp.png}
  }
  \subfigure[\scriptsize GT]{%
    \includegraphics[width=.18\columnwidth]{figures/supplementary/lindau00029_000019_gt.png}
  }
  \subfigure[\scriptsize Deeplab]{%
    \includegraphics[width=.18\columnwidth]{figures/supplementary/lindau00029_000019_cnn.png}
  }
  \subfigure[\scriptsize Using BI]{%
    \includegraphics[width=.18\columnwidth]{figures/supplementary/lindau00029_000019_ours.png}
  }%\\[-2ex]

  \mycaption{Street Scene Segmentation}{Example results of street scene segmentation.
  (d)~depicts the DeepLab results, (e)~result obtained by adding bilateral inception (BI) modules (\bi{6}{2}+\bi{7}{6}) between \fc~layers.}
\label{fig:street_visuals-app}
\end{figure*}



%%%%%%%%%%%%%%%%%%%%%%%%%%%%%%
%\backmatter


	\bibliography{ref1}
	\bibliographystyle{amsplain}

\end{document}



