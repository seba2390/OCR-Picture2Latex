% Created May 2002 by Arjan Egges, Editorial Assistant
% of Computer Animation and Virtual Worlds

\documentclass[11pt,twocolumn]{scrartcl}

% Packages

\usepackage{casa_conf}	% The CASA style
\usepackage{graphicx}	% A package for graphics use (see figures)

% Added by us
\usepackage{amsmath}
\usepackage{tabularx, makecell}
\renewcommand\tabularxcolumn[1]{m{#1}}% for vertical centering text in X column
\usepackage{xcolor}
\usepackage{ulem}
\usepackage{wrapfig}
\usepackage[format=plain]{caption}
\usepackage{hyperref}

%\newcommand{\modify}[2]{{\color{blue} \sout{#1} #2}}
% without redacted content
%\newcommand{\modify}[2]{{\color{blue}#2}}
% without blue
\newcommand{\modify}[2]{#2}

\newcommand{\todo}[1]{{\color{red} TODO: #1}}
\newcommand{\leon}[1]{{\color{blue} #1}}
\newcommand{\alex}[1]{{\color{green} #1}}

\newcommand{\paragraphtitle}[1]{\vspace{5mm}\textbf{#1} \\}
\newcommand\etal{\textit{et al.~}}

\newenvironment{modified}
    {
    %\color{blue}
    }
    {}

% Define the title and complete author addresses here, please!

\title{Learning-based pose edition for efficient and interactive design}

% Remove author and institutional information for the review process
\author{Léon Victor$^{1,3}$, 
        Alexandre Meyer$^{1,2}$, 
        Saïda Bouakaz$^{1,2}$
        
        \\
        leon.victor@insa-lyon.fr, \{alexandre.meyer, saida.bouakaz\}@univ-lyon1.fr
       }
       
\date{%
    $^1$ Univ Lyon, LIRIS, UMR CNRS 5205\\%
    $^2$ Université Claude Bernard Lyon 1\\%
    $^3$ INSA Lyon\\%
    \vspace{5pt}
}

% Main document
\usepackage{capt-of, etoolbox}
\begin{document}
\makeatletter
\let\@oldmaketitle\@maketitle% Store \@maketitle
\renewcommand{\@maketitle}{\@oldmaketitle% Update \@maketitle to insert...
  \vspace{-10mm}
  \centering
  \includegraphics[width=0.8\linewidth]{teaser-casa.png}
  \captionof{figure}{Sample results of our method in various configuration. Our method is able to generate plausible poses given a starting pose (on the left) and some targets (in red), respecting skeleton constraints without having to explicitly specify them.}
    \bigskip}% ... an image
\makeatother
\vspace{-5pt}
\maketitle
\vspace{4pt}
\begin{abstract}
Authoring an appealing animation for a virtual character is a challenging task. In computer-aided keyframe animation artists define the key poses of a character by manipulating its underlying skeletons.
To look plausible, a character pose must respect many ill-defined constraints, and so the resulting realism greatly depends on the animator's skill and knowledge.
Animation software provide tools to help in this matter, relying on various algorithms to automatically enforce some of these constraints.
The increasing availability of motion capture data has raised interest in data-driven approaches to pose design, with the potential of shifting more of the task of assessing realism from the artist to the computer, and to provide easier access to non-experts. In this paper, we propose such a method, relying on neural networks to automatically learn the constraints from the data. We describe an efficient tool for pose design, allowing naïve users to intuitively manipulate a pose to create character animations.
\footnote{The code is available at \url{https://github.com/leonvictor/neural-pose-edition}}
\linebreak
\linebreak
\keywords{Character Pose Design, Machine Learning for Animation}
\end{abstract}


\section{Introduction}
Character animation is an essential part of computer-generated imagery industries such as feature films, cartoons or video games which make use of on-screen characters to tell stories, convey emotions and appeal to their audiences.
These characters are represented by 3-dimensional meshes whose motion is driven by an underlying skeleton. A common method to design and edit animations is keyframing: animator pose the character at desired time stamps (the key frames) and the computer interpolates between them to fill in the gaps.
Most animation software such as Blender or Maya provide interactive tools allowing users to pose a character by manipulating its underlying skeleton.
We propose an innovative solution that makes the pose editing process more affordable without compromising the quality of the results. The presented method leverages neural networks to implicitly learn the intricacies of a (human) skeleton and provide simple controls.
Our main goal is to create an intuitive real-time system that can produce appealing poses even for a novice user.

%Simple description of the method
Our framework relies on a few small networks requiring reasonable resources to train, with the added advantage of running quite fast at inference time. The core of our approach is an encoder-decoder trained on skeleton pose data, the task of which is to build a latent representation of the pose space, alleviating some of the limitations of the former. We then train a family of solver networks to work on this latent space in order to generate a pose satisfying user-defined target positions.
%%%%%%%%%%%%%%%%%%%%%%%%%%%%%%%%%%%%%%%%%%%%%%%%%%%%%%%%%%%%%%%%%%%%%%%%%%%%%%%%%%%%%%%%%%%%%%%%%%%%%%%%%%%%%%%%%%%%%%%%%%%%%%%%%%%%%%%%%%%%%%%%%%
\section{Related work}
\label{sect:related_work}
The industry standard for pose edition is to create rigs, a collection of pieces of software designed to manipulate a character's skeleton. The rig describes the skeleton's bones, how they relate to each other, are constrained in their possible motion and are deformed. These rules are loosely specified and creating a good rig requires a detailed understanding of physics and anatomy, as well as technical and artistic skills. Rigging is thus a time consuming task even for experienced animators, and even more so in large scale productions which often require a different in-depth rig for each character in the cast.
Previous work has helped alleviate this difficulty by providing efficient tools to speed up/and or ease the rigging process, relying on inverse kinematics or data-driven methods.
\subsection{Character pose design}
\subsubsection{Inverse Kinematics (IK)}
IK solvers are a family of methods commonly used in robotics, engineering and computer graphics, in which the parameterization of a kinematic chain is determined from the position of its end effector.
They are a staple tool in pose design software, ensuring the respect of elementary constraints during pose edition. Their de-facto role is to guarantee the length of the limbs, and in some cases to enforce the orientation angle range of a joint.
Many IK solutions have been studied over the years \cite{aristidou_inverse_2018}; usually revolving around approximated linearizations or heuristics. 

Numerical methods require a set of iterations to achieve a satisfactory solution formulated by a cost function to be minimized.
IK solutions can generally be divided into three sub-categories: Jacobian \cite{Siciliano_Handbook_Robot_2007}, Newtonians \cite{cohen_ik_1996} and Heuristics. Most software implement heuristic methods such as Cyclic Coordinate Descent (CCD) \cite{wang_ccd_1991} or 
Forward-Backward Reaching IK (FABRIK) \cite{aristidou_fabrik:_2011} due to their simplicity and extensibility. 

The main drawback of 
these solvers is that they manipulate kinematic chains without taking into account many morphological aspects that make a pose more or less plausible. They offer a first level of help to users but are not sufficient to guarantee a realistic pose. Many joints constraints are dependent on each other and require subjective, human-made approximations.

\subsubsection{Data-driven pose edition}
Data-driven methods offer promising opportunities to solve these approximations. Using real-life data can help in modelling the complex inter-dependencies of skeletons and providing users with smarter edition tools.
While it is still an early field of research, some solutions have been studied. Wu \etal \cite{wu_posing_2009} propose a method for natural character posing from a large motion database. It employs adaptive KD-clustering to select a representative frame from a database and sparse approximations to accelerate training and posing. 
Huang \etal in \cite{Huang_IK_MGDM_2017} present a method based on the formulation of multi-variate Gaussian distribution models (MGDMs), which learn the joint constraints of a kinematic skeleton from motion capture data. 

Some work has also been dedicated to finding new editing interfaces. \modify{}{Instead of the usual setup manipulating joints directly, Guay \etal \cite{guay_line_2013} articulate a framework based on the conceptual "line of action" which describes the overall pose dynamics. They provide a mathematical definition of the line of action, and a interface in which the software modifies the pose to follow a user-provided line. In the same line of though} Garcia \etal \cite{garcia_sketching_2019} propose \modify{a method transforming doodle of trajectories (position and orientation over time) }{a virtual reality-based interface where the user's hands motion (position and orientation over time) are transformed} into sequences of actions and then into detailed character animations using a dataset of parametrized motion clips automatically fitted to the trajectory. 

% ==> DL et Latent Space. 
\subsection{Neural modelling of human motion}
Neural networks have received a great amount of attention over the last decade and shown impressive result in modelling complex data. Human motion has not been spared and deep learning methods have proven their capability of generating realistic motion in a number of difficult cases. 

The literature in neural-based animation include example in user-controlled character navigation \cite{Holden2017} and interactions with the environment \cite{starke_neural_2019}. 
Holden \etal \cite{Holden2020} also show that neural networks can be used to replace parts of existing data-driven methods, improving their scalability potential.
More recently, some work has also focused on improving smaller parts of the animation pipeline rather than replacing it completely. Berson et al. \cite{berson_intuitive_2020} leverage neural networks to provide an interactive system to edit facial animation. 

% Wrap up
Data-driven IK and pose editing can relieve animators from time-consuming, back-and-forth pose adjustments by applying constraints extracted from real-world data. Recently, neural-network-based approaches have demonstrated their ability to model the intricacies of human motion while scaling to large amount of data and retaining a fast inference time. In this paper we seek to take advantage of these properties to create an efficient posing tool, intuitively usable even by a inexperienced user.
    
\section{Proposed method}
\label{sect:proposedMethod}
\label{sect:proposed_method}
\subsection{Method overview}

\begin{figure*}[h]
    \centering
    \includegraphics[width=0.75\linewidth]{nn_ik.png}
    \caption{High level overview of the generation setup. The target joint's positions (yellow) are matched as closely as possible, while the other joints (green) should be as close as possible to the starting pose (blue).}
    \label{fig:generation_setup}
\end{figure*}

We propose a method to solve a high level pose design problem in which a pose is modified to reach desired target positions for some of its joints. We leverage the modelling power of neural networks to implicitly learn skeleton constraints from a pre-existing pose database.
Our method, illustrated in Fig. \ref{fig:generation_setup}, relies on two models: an auto-encoder to build an alternative latent pose space, and a solver model operating on this space to solve the pose design problem. 
We also describe an optional post-processing step to smooth out the remaining errors, and \modify{}{outline} a methodology using multiple instances of the solver model at once to work with a varying amount of targets.

\subsection{Data}
\label{sect:materiel}
    
We train the models using a dataset of human poses, obtained by processing multiple available motion-capture datasets from the literature: Emilya \cite{fourati_emilya_2014}, CMU \cite{CMU_BVH}, and the clips from Edinburgh university \cite{holden_deep_2016}. Each animation clip is retargeted to a standard skeleton following the scheme proposed by \cite{HoldenAE2015}. The global translation is removed, and each joint's position is calculated relative to the root joint, which is the projection of the pelvis on the floor. The unified skeleton is composed of 21 joints; using the joints' positions in space, a pose is described by $ 3 \times 21 = 63 $ float values concatenated in a single vector. The dataset is then formed by the individual poses in each clip. Before feeding them to the network we also normalize each pose by subtracting the mean and dividing by the standard deviation of each feature. \modify{}{With a few jittery clips manually removed, the final dataset used in the following experiments is composed of about 1,5 million poses.}

\subsection{Models description}
\subsubsection{Autoencoder}
Auto-encoders are made up of two neural networks tasked to learn efficient encodings of complex data. The encoder maps real data points to a learned, usually more compact, latent space; and the \modify{encoder}{decoder} maps them back to the original data space.
We build such an auto-encoder of poses in order to build a common operating space for the following solvers. Generating points in the latent space allows us to ensure that the output is always a plausible pose, as the decoder is trained to turn any and all latent point into them.

The encoder network is composed of two fully connected layers with \modify{195}{200} neurons and ReLU \cite{relu_2010} activations, followed by an output layer with no activation. The output layer's size is based on the number of
dimensions $d$ in which the latent representations are encoded. We empirically find that $d=64$ yields a good balance of representation accuracy and inference speed. The decoder is the exact reversed replica and uses the same set of weights. 

% alex: 63, un peu étrange de ne pas avoir 64, les infos aiment bien les puissance de 2
%Saida : d'accord avec Alex
The autoencoder's weights are optimized by minimizing the mean squared error (MSE) between the input pose $x$ and its reconstructed equivalent $\hat x$ (Eq. \ref{eq:loss-ae}). In the following sections we refer to the encoder as $E$, the decoder as $D$ and a latent encoding as $z$, i.e. $z = E(x)$ and $\hat x = D(z)$. 

\begin{equation}
    \label{eq:loss-ae}
    L_{ae} = MSE(x, \hat x) = \frac{1}{d} \sum_{i=1}^{d}(x_i - \hat x_i)^2
\end{equation}

The autoencoder is trained for 20 epoch with batches of 256 poses, using the Adam optimizer \cite{kingma_adam_2017} with a learning rate of $0.0001$.

\subsubsection{Pose solver}

An instance of the solver model $S_t$ is specialized to solve the IK problem for $n$ specific targets $t$ and is trained to generate a new pose from an input pose and the desired targets locations. 
As it operates on the latent space built by the autoencoder, it more precisely accepts and outputs a latent pose vector, i.e. with $p_t$ the \modify{}{concatenated} target positions, $\hat z = S_t(z, p_t)$.

The network is composed of three fully connected layers with 126 neurons and ReLu activations, and an output layer with $d$ neurons.

During training, we randomly sample an input pose $x$ from the dataset and feed it to the network. \modify{The targets are generated by taking the positions of the considered joints on a random pose in the same animation clip as the input pose.}{We also sample a second pose $x'$ from the same source clip to use as target.} We found that this association helped the network learning by not relying on random (and possibly unreachable) target positions.

%alex: the loss function in Equation 2 (le 'in')
\modify{Its}{The network's} weights are optimized to minimize the loss function in Eq.\ref{eq:loss-ik} designed to represent its high level objective: \modify{reaching the targets with the associated joints while staying as close to the starting pose as possible}{reaching the targets with the associated joints while retaining a realistic pose}. We guide the network toward this objective by using a modified mean squared error function \modify{Eq.\ref{eq:loss-ik}}{, separating the poses ($x$ in this example) in two sets of joints: $x_{target}$ the joints associated to the targets $t$, and $x_{rest}$ the others} \modify{The generated pose's target joints $\hat x_{target}$ should be close to the input targets $p_t$, while its other joints $\hat x_{rest}$ should minimize their motion}{}.
We introduce a constant $k$ to give more relative importance to the target term of the function\modify{}{, so that the non-targets joints of $x'$ are only used to nudge the final result toward a plausible pose}. In our experiments $k$ is set to $0.01$.\modify{A side effect of our loss function is that the target positions are not an absolute truth to be reached at all cost. The solver is rather encouraged to use them as guides, only reaching them precisely when the starting pose would not require too much of a change.}{}
%alex: k is set to 0.01 => avec un chiffre aussi petit ca a de l'effet quand même ?

\begin{equation}
    \label{eq:loss-ik}
    L_{s} = MSE(\hat x_{target}, x'_{target}) + k \cdot MSE(\hat x _{rest}, x'_{rest})
\end{equation}



An instance of the solver model is trained for \modify{15}{5} epochs with the Adam optimizer using a learning rate of 0.0001 and a batch size of 256.

\subsection{Post processing}

It is a common observation with neural networks working with joints position that the generated positions can be jittery, and the resulting poses can suffer from slight variations in bone lengths. 
Our model is no exception, and while the variation is not visually detectable most of the time, computing the total bone length difference between the input skeleton and the generated pose shows that it is present. These variations are naturally undesirable and can result in visual discomfort on the spectator's end. In order to alleviate the problem we apply an optional post-processing step to the resulting poses to ensure constant bone lengths. We use the backward step from FABRIK as it is very lightweight computation-wise. Our experiments show that following this process  lends better results at a small cost in computing time (see table \ref{table:results}).

\subsection{Solving other targets configurations}
\label{sect:multi-solver}
Even though our solvers are designed to generate a pose considering one to two targets at once, it is possible to use multiple instances side by side and to switch to the correct one with regard to the selected targets. In cases where the user desires to use an arbitrary number of targets
(to suggest a position for a fixed joint for example) we can combine the multiple instances by running them in sequence, i.e. $\hat z = (S_{t3} \circ S_{t2} \circ S_{t1})(z)$ for $t1, t2, t3$ various targets and $S_{ti}$ the solvers trained to reach them.
%%%%%%%%%%%%%%%%%%%%%%%%%%%%%%%%%%%%%%%%%%%%%%%%%%%%%%%%%%%%%%%%%%%%%%%%%%%%%%%%%%%%%%%%%%%%%%%%%%%%%%%%%%%%%%%%%%%%%%%%%%%%%%%%%%%%%%
\section{Results}
\label{sect:results}
\begin{table}[t!]
\centering
\caption{Voice conversion \& F0 manipulation results. MOS results are reported with 95\% confidence interval. VDE, and FFE are reported for F0 manipulation while PER, WER, EER, and MOS are reported for voice conversion. Notice, for VDE, and FFE higher is the better since F0 was flattened.}
\label{tab:conv}

\resizebox{1\columnwidth}{!}{
\begin{tabular}{c@{~} | c@{~} | c@{~}c@{~} | c@{~} | c@{~} ||  c@{~}c@{~} }
\toprule
\multirow{2}{*}{Dataset} & \multirow{2}{*}{Method} & \multicolumn{4}{c||}{Voice Conversion} & \multicolumn{2}{c}{F0 Manipulation} \\
\cmidrule{3-8}
& & PER~$\downarrow$ & WER~$\downarrow$ & EER~$\downarrow$ & MOS~$\uparrow$ & VDE~$\uparrow$ & FFE~$\uparrow$ \\
\midrule
VCTK & GT  & 17.16 & 4.32 & 3.25 & 4.11$\pm$0.29 & -- & -- \\
\midrule 
\multirow{3}{*}{LJ}
% & ASR-TTS   & 50.74  & --     & 66.08 & 32.96 & 1.46 \\
& CPC       & 22.22 	& 16.11 		& 0.46 		& 3.57$\pm$0.15 		& \bf 46.68 & \bf 48.71\\
& HuBERT    & \bf 19.09 & \bf 12.23 & \bf 0.31  & \bf 3.71$\pm$0.24 & 39.20 		& 48.42\\
& VQ-VAE    & 40.88 	& 36.96 		& 9.65 		& 2.90$\pm$0.17 		& 10.54 	& 12.08 \\
\midrule 
\multirow{3}{*}{VCTK} 
% & ASR-TTS   & 68.88  & --    & 41.77 & 13.55 & 6.48 \\
& CPC       &  23.58 		& 15.98 		& \bf 4.83  &  3.42 $\pm$ 0.24 		& \bf 25.29 & \bf 26.97 \\
& HuBERT    &  \bf 20.85 	& \bf 12.72 & 6.01  		& \bf  3.58 $\pm$ 0.28 	& 23.46 	& 26.67 \\
& VQ-VAE    & 36.88  		& 29.44 		& 11.56 		& 3.08 $\pm$ 0.34 		& 7.03  	& 7.80  \\
\bottomrule
\end{tabular}}
\vspace{-0.4cm}
\end{table}

\vspace{-0.1cm}
\section{Results}
\vspace{-0.1cm}
Our results cover
% We report results for 
three different settings: (i) speech reconstruction experiments; (ii) speaker conversion and F0 manipulation; (iii) bitrate analysis with subjective tests for speech codec evaluation. We employ two datasets: LJ~\cite{ljspeech17} single speaker dataset and VCTK~\cite{vctk} multi-speaker dataset. All datasets were resampled to a 16kHz sample rate.

% \paragraph*{Implementation Details.}
% \smallskip
\noindent{\bf Implementation Details\quad} 
\label{sec:impl}
We follow the same setup as in~\cite{lakhotia2021generative}. For CPC, we used the model from~\cite{Riviere2020}, which was trained on a ``clean'' 6k hour sub-sample of the LibriLight dataset~\cite{Kahn2020,Riviere2020}. We extract a downsampled representation from an intermediate layer with a 256-dimensional embedding and a hop size of 160 audio samples. For HuBERT we used a \textsc{Base} 12 transformer-layer model trained for two iterations~\cite{hsu2020hubert} on 960 hours of LibriSpeech corpus~\cite{Panayotov2015}. 
% This model encodes every 320 raw audio samples into a 768-dimensional vector. 
This model downsamples the raw audio $\times320$ into a sequence of 768-dimensional vectors. Similarly to~\cite{lakhotia2021generative}, activations were extracted from the sixth layer.

%CPC: We use a dictionary of 100 units, leading to a bitrate of 700bps.
%HuBERT: A dictionary of 100 units is used, leading to a bitrate of 350bps. 
%VQVE: The VQ-VAE discrete code operates at a bitrate of 800bps.
% For both CPC and HuBERT, the k-means algorithm is applied to convert continuous frames to discrete codes, using the LibriSpeech clean-100h~\cite{Panayotov2015} dataset. 
For CPC and HuBERT, the k-means algorithm is trained on LibriSpeech clean-100h~\cite{Panayotov2015} dataset to convert continuous frames to discrete codes. We quantize both learned representations with $K=100$ centroids. Leading to a bitrate of 700bps for CPC and 350bps for HuBERT.

% VQ-VAE
Similarly to CPC models, we trained the VQ-VAE content encoder model on the ``clean'' 6K hours subset from the LibriLight dataset. We use an encoder operating on the raw signal to extract discrete units, similar to~\cite{jukebox}. In addition, ``random restarts'' were performed when the mean usage of a codebook vector fell below a predetermined threshold. Finally, we used HiFiGAN (architecture and objective) as the decoder instead of a simple convolutional decoder, as it improved the overall audio quality. This model encodes the raw audio into a sequence of discrete tokens from 256 possible tokens~\cite{garbacea2019low} with a hop size of 160 raw audio samples. The VQ-VAE discrete code operates at a bitrate of 800bps. We additionally experimented with 100 discrete units for VQ-VAE, however results were the best for 256. This finding is consistent with~\cite{garbacea2019low}.

% verification model
The speaker verification network uses the architecture proposed in~\cite{heigold2016end}. It was trained on the VoxCeleb2~\cite{voxceleb2} dataset, achieving a 7.4\% Equal Error Rate (EER) for speaker verification on the test split of the VoxCeleb1~\cite{Nagrani17} dataset.

% pitch
Only a single F0 representation is considered across all evaluated models, trained on the VCTK dataset.
% The F0 is extracted from the raw audio using YAAPT~\cite{yaapt} algorithm, using a window size of 20ms and a 5ms hop. 
The F0 is extracted from the raw audio using a window size of 20ms and a 5ms hop. 
As a result, the F0 sequence is sampled at 200Hz. 
% We apply the quantization described at Sec.~\ref{sec:method}, using a pitch codebook of $K'=20$ tokens and an encoder that downsamples the pitch by $\times16$. 
The quantization described at Sec.~\ref{sec:method}, is applied using an F0 codebook of $K'=20$ tokens and an encoder that downsamples the signal by $\times16$. Hence, the discrete F0 representation is sampled at 12.5Hz, leading to a bitrate of 65bps. The final bitrate of the evaluated codecs is the sum of the pitch code bitrate with the content code bitrate.

% \paragraph*{Evaluation Metrics}
% \smallskip
\noindent{\bf Evaluation Metrics\quad} 
We consider both subjective and objective evaluation metrics. For subjective tests, we report the Mean Opinion Scores (MOS). In which human evaluators rate the naturalness of audio samples on a scale of 1--5. Each experiment, included 50 randomly selected samples rated by 30 raters. For objective evaluation, we consider: (i) Equal Error Rate~(EER) as an automatic speaker verification metric obtained using a pre-trained speaker verification network. We report EER between test utterances and enrolled speakers; (ii) Voicing Decision Error (VDE)~\cite{nakatani2008method}, which measures the portion of frames with voicing decision error; (iii) F0 Frame Error (FFE)~\cite{chu2009reducing}, measures the percentage of frames that contain a deviation of more than 20\% in pitch value or have a voicing decision error; (iv) Word Error Rate (WER) and Phoneme Error Rate (PER), proxy metrics to the intelligibility of the generated audio. We used a pre-trained ASR network~\cite{baevski2020wav2vec} on both reconstructed and converted samples to calculate both metrics. %To generate target phonemes, the g2p-en~\cite{g2pE2019} Grapheme2Phoneme module was used.

% \vspace{-0.1cm}
% \smallskip
\noindent{\bf Reconstruction \& Conversion}
% \vspace{-0.1cm}
We start by reporting the reconstruction performance. Results are summarized in Table~\ref{tab:recon}. When considering the intelligibility of the reconstructed signal HuBERT reaches the lowest PER and WER scores across all models, where both CPC and HuBERT are superior to VQ-VAE. However, when considering F0 reconstruction VQ-VAE outperforms both HuBERT and CPC by a significant margin. This results are somewhat intuitive, bearing in mind VQ-VAE objective is to fully reconstruct the input signal. In terms of subjective evaluation, all models reach similar MOS scores, with one exception of CPC on LJ. 

%Notice, since the same F0 units are used for each method, this result implies the VQ-VAE units contain some information about the F0 of the signal, enabling better reconstruction. Regarding speaker information, the CPC gets the lowest EER. 

To better evaluate the disentanglement properties of each method with respect to speaker identity and F0, we conducted an additional set of experiments aiming at speaker conversion and F0 manipulation. For voice conversion, we converted each test utterance into five random target speakers. Next, we employed a speaker verification network, which extracts \emph{d-vector} representation to evaluate speaker-converted utterances' similarity to real speaker utterances (low error-rate indicates good conversion), providing measurement to the speaker identity's disentanglement from the evaluated coding method. The error-rate is reported between converted test utterances and enrolled speakers. For the LJ speech single speaker dataset, we converted samples from the VCTK dataset to the single speaker and enrolled all VCTK speakers together with the single speaker. Results are summarized in Table~\ref{tab:conv} (left). Unlike resynthesis results, on voice conversion CPC and HuBERT outperform VQ-VAE on both LJ and VCTK datasets, indicating VQ-VAE contains more information about the speaker in the encoded units, hence producing more artifacts. Notice, this also affects WER, PER, and the overall subjective quality (MOS). 

Next, to evaluate the presence of F0 in the discrete units, we flattened the F0 units before synthesizing the signal and calculated VDE and FFE with respect to the original F0 values. F0 flattening was done by setting the speakers' mean F0 value across all voiced frames. In this experiment, we expected units that contain F0 information to be better at F0 reconstruction over disentangled units. Results are summarized in Table~\ref{tab:conv} (right). Notice VQ-VAE can still reconstruct the F0 almost at the same level as when using the original F0 as conditioning (5.2 vs 7.03, and 5.59 vs 7.8), in contrast to CPC and HuBERT.

\begin{figure}[t!]
\centering
\includegraphics[width=0.65\columnwidth, trim={50 20 70 20}]{figures/codec_2.pdf}
% \caption{MUSHRA subjective listening test results as a function of bitrate per second for various methods. Purple dots denote the baseline methods, and green dots the proposed SSL based method.} 
\caption{MUSHRA subjective quality results as a function of bitrate per second. Purple dots denote the baseline methods, and green dots the proposed SSL based method.} 
\label{fig:codec}
\vspace{-0.5cm}
\end{figure}

% \vspace{-0.1cm}
% \smallskip
\noindent{\bf Speech Codec}
Our final experiment evaluates the obtained speech units as a low bitrate speech codec. 
% Therefore, we evaluate how the performance varies as a function of the number of discrete units. Changing the number of units is equivalent to varying the bitrate of the encoded signal. 
We use a subjective MUSHRA-type listening test~\cite{series2014method} to measure the perceived quality of the proposed speech codec with regard to its bitrate constraints. In MUSHRA evaluations, listeners are presented with a labeled uncompressed signal for reference, a set of test samples to rate, a copy of the uncompressed reference, and a low-quality anchor. Listeners are asked to rate each test utterance and the copy of the uncompressed reference with respect to the labeled reference in a scale of 1-100.

The experiment is performed on the VCTK dataset~\cite{vctk}. For evaluation, we used 20 utterances from 5 speakers. The set of speakers in the test data is disjoint with those in the training data. For this experiment, HuBERT models with 50, 100, and 200 units were trained as described in Sec.~\ref{sec:impl}. For comparison, we included other speech codecs in our evaluation: Opus~\cite{valin2012definition} wideband at 9 kbps VBR, Codec2~\cite{rowe2011codec} at 2.4 kbps and LPCNet~\cite{valin2019real} operating at 1.6 kbps. The LPCNet model was trained from scratch on the VCTK dataset following the experimental setup in~\cite{valin2019real}. The VQ-VAE model employs the HiFiGAN decoder trained on the LibriLight dataset to match the amount of data reported in~\cite{garbacea2019low}. We compressed the anchor sample with Speex~\cite{valin2016speex} at 4 kbps as a low anchor. Fig.~\ref{fig:codec} depicts the results. HuBERT with 50 units reaches the best MUSHRA score while its bitrate is only 365bps, which is significantly lower than the baseline methods.

%\section{Discussion}
%In this paper, 2D and 3D CNN models were used to generate pelvic sCTs from T1-weighted MR images. Our sCT generation methods were fully automated, requiring no deformable registration or manual segmentation of bone tissues. As shown in Figure~\ref{fig3}, the 2D and 3D CNN models generated high quality sCTs. MAE curves shown in Figure~\ref{fig4} indicated that both models could precisely estimate soft-tissue HU values but had difficulty in reproducing air and high-density bone tissues. 

The MAEs within the body contour across all patients were 40.5 $\pm$ 5.4 HU and 37.6 $\pm$ 5.1 HU for the 2D and 3D models, respectively. The time required for generating a pelvic sCT using our CNN models was about 5.5 s. Our MAE results are comparable to previous studies. Kim $et \ al.$\cite{RN41} presented a voxel-based weighted summation method that produced an MAE of 74.3 $\pm$ 3.9 HU. However, manual contouring of bone tissues required for this method can be tedious and time-consuming. An MAE of 40.5 $\pm$ 8.2 HU was achieved by Dowling $et \ al.$\cite{RN11} using an average MRI-CT atlas from 38 patients. Andreasen $et \ al.$\cite{RN42} reported an MAE of 54 $\pm$ 8 HU using an atlas-based method with pattern recognition, and its prediction time was about 20.8 min. Another random forest model proposed by Andreasen $et \ al.$\cite{RN43} generated sCTs with an MAE of 58 $pm$ 9 HU. A hybrid method suggested by Siversson $et \ al.$ \cite{RN45} obtained an MAE of 36.5 $\pm$ 4.1 HU when ignoring errors introduced by gas cavities. This hybrid method was implemented in the cloud-based commercial software MriPlanner (Spectronic Medical AB, Helsingborg, Sweden), which required 50 to 80 min to generate a sCT.\cite{RN45} The patch-based 3D context-aware generative adversarial network presented by Nie $et \ al.$\cite{RN26} achieved an MAE of 39.0 $\pm$ 4.6 HU. 

Our CNN models reproduced low-density bone as shown in Figure ~\ref{fig4}. The bone-region DSCs were 0.81 $\pm$ 0.04 and 0.82 $\pm$ 0.04 from the 2D and 3D models, respectively. These results are comparable to reported DSC results of 0.79 $\pm$ 0.12\cite{RN10} and 0.91$\pm$0.03{\cite{RN11}}, where the authors compared bone contours manually drawn on the sCT and CT.

It was feasible to train the proposed 3D model with 16 image volumes from scratch. Results of the Wilcoxon signed-rank tests shown in Table~\ref{tab1} demonstrated a statistically significant improvement in overall MAE, bone DSC, and bone precision of the 3D model compared to the 2D model. However, as shown in Figure~\ref{fig4}, the 2D model seemed to perform better in estimating the high-density bone HU values. It should be noted that smaller overall MAEs do not guarantee improved sCT dose calculation and patient positioning performance. While the models performed well, we will continue to acquire more patient data to potentially improve model accuracy and further test model differences.

As this was a retrospective study, the MR image voxel sizes were not matched, resulting in different voxel intensities between images. This may have affected the sCT generation accuracy although we applied intensity normalization. A potential study could examine how voxel size variations affects sCT estimation. 

The proposed 3D model can be implemented on a 12 GB GPU to process volumetric images with dimensions of 256 $\times$ 256 $\times$ 30. More GPU memory would be required to process higher resolution 3D images. Considering the limited access to multi-GPU systems, a 3D architecture with fewer convolutional layers could be considered to deal with higher resolutions. However, the performance could be affected by the reduced parameters and smaller receptive fields of the less complex model. Another approach would be to extract 30-slice sub-volumes from CT and MR images for training the 3D model. The sCT could then be generated by averaging 30-slice sCT sub-volumes produced by the model. 

A number of techniques could be investigated for improving model performance.  Nie $et \ al.$\cite{RN26} showed that introducing an additional adversarial discriminator improved overall sCT quality. The same approach could be adapted in our proposed 2D and 3D CNN models.  Non-rigid deformation\cite{RN44} could also be applied to both CT and MR images in the process of the on-the-fly data augmentation to produce more training pairs. Multiple MR images acquired with different sequences could be fed into models to provide more information for distinguishing different tissues. Multi-GPU systems with more memory would enable the exploration of larger batch sizes for training CNN models, which could reduce variances in gradient estimation and accelerate the training. 



\section{Conclusion and perspectives}
\label{sect:conclusion}

\begin{comment}
\begin{figure}
\includegraphics[width=\linewidth]{figs/beyond_tss_lesion.pdf}
\caption[]{End-to-End runtime lesion study of the entire MNIST dataset and the FMA featurized music dataset. Each of DROP's contributions provides a runtime improvement.}
\label{fig:beyond_lesion}
\end{figure}
\end{comment}



\section{Conclusion}
\label{sec:conclusion}

Advanced data analytics techniques must scale to rising data volumes. 
DR techniques offer a powerful toolkit when processing these datasets, with PCA frequently outperforming popular techniques in exchange for high computational cost. 
In response, we propose DROP, a new dimensionality reduction optimizer. 
DROP combines progressive sampling, progress estimation, and online aggregation to identify high quality low dimensional bases via PCA without processing the entire dataset by balancing the runtime of downstream tasks and achieved dimensionality. 
Thus, DROP provides a first step in bridging the gap between quality and efficiency in end-to-end DR for downstream \red{analytics}. 

%We revisit canonical operators for time series dimensionality reduction and the measurement study of~\cite{keogh-study}, and show that PCA is more effective than popular alternatives in the data mining literature often by a margin of over $2\times$ on average on gold-standard time series benchmark data sets with respect to output data dimension. More surprisingly, we empirically demonstrate that a small number of samples are sufficient to accurately characterize directions of maximum variance and obtain a high-quality low-dimensional transformation.




\bibliographystyle{unsrt}
\bibliography{bibliography}


\end{document}
