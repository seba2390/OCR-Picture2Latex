% Created May 2002 by Arjan Egges, Editorial Assistant
% of Computer Animation and Virtual Worlds

\documentclass[11pt,twocolumn]{scrartcl}

% Packages

\usepackage{casa_conf}	% The CASA style
\usepackage{graphicx}	% A package for graphics use (see figures)

% Added by us
\usepackage{amsmath}
\usepackage{tabularx, makecell}
\renewcommand\tabularxcolumn[1]{m{#1}}% for vertical centering text in X column
\usepackage{xcolor}
\usepackage{ulem}
\usepackage{wrapfig}
\usepackage[format=plain]{caption}
\usepackage{hyperref}

%\newcommand{\modify}[2]{{\color{blue} \sout{#1} #2}}
% without redacted content
%\newcommand{\modify}[2]{{\color{blue}#2}}
% without blue
\newcommand{\modify}[2]{#2}

\newcommand{\todo}[1]{{\color{red} TODO: #1}}
\newcommand{\leon}[1]{{\color{blue} #1}}
\newcommand{\alex}[1]{{\color{green} #1}}

\newcommand{\paragraphtitle}[1]{\vspace{5mm}\textbf{#1} \\}
\newcommand\etal{\textit{et al.~}}

\newenvironment{modified}
    {
    %\color{blue}
    }
    {}

% Define the title and complete author addresses here, please!

\title{Learning-based pose edition for efficient and interactive design}

% Remove author and institutional information for the review process
\author{Léon Victor$^{1,3}$, 
        Alexandre Meyer$^{1,2}$, 
        Saïda Bouakaz$^{1,2}$
        
        \\
        leon.victor@insa-lyon.fr, \{alexandre.meyer, saida.bouakaz\}@univ-lyon1.fr
       }
       
\date{%
    $^1$ Univ Lyon, LIRIS, UMR CNRS 5205\\%
    $^2$ Université Claude Bernard Lyon 1\\%
    $^3$ INSA Lyon\\%
    \vspace{5pt}
}

% Main document
\usepackage{capt-of, etoolbox}
\begin{document}
\makeatletter
\let\@oldmaketitle\@maketitle% Store \@maketitle
\renewcommand{\@maketitle}{\@oldmaketitle% Update \@maketitle to insert...
  \vspace{-10mm}
  \centering
  \includegraphics[width=0.8\linewidth]{teaser-casa.png}
  \captionof{figure}{Sample results of our method in various configuration. Our method is able to generate plausible poses given a starting pose (on the left) and some targets (in red), respecting skeleton constraints without having to explicitly specify them.}
    \bigskip}% ... an image
\makeatother
\vspace{-5pt}
\maketitle
\vspace{4pt}
\begin{abstract}
Authoring an appealing animation for a virtual character is a challenging task. In computer-aided keyframe animation artists define the key poses of a character by manipulating its underlying skeletons.
To look plausible, a character pose must respect many ill-defined constraints, and so the resulting realism greatly depends on the animator's skill and knowledge.
Animation software provide tools to help in this matter, relying on various algorithms to automatically enforce some of these constraints.
The increasing availability of motion capture data has raised interest in data-driven approaches to pose design, with the potential of shifting more of the task of assessing realism from the artist to the computer, and to provide easier access to non-experts. In this paper, we propose such a method, relying on neural networks to automatically learn the constraints from the data. We describe an efficient tool for pose design, allowing naïve users to intuitively manipulate a pose to create character animations.
\footnote{The code is available at \url{https://github.com/leonvictor/neural-pose-edition}}
\linebreak
\linebreak
\keywords{Character Pose Design, Machine Learning for Animation}
\end{abstract}


\section{Introduction}
Character animation is an essential part of computer-generated imagery industries such as feature films, cartoons or video games which make use of on-screen characters to tell stories, convey emotions and appeal to their audiences.
These characters are represented by 3-dimensional meshes whose motion is driven by an underlying skeleton. A common method to design and edit animations is keyframing: animator pose the character at desired time stamps (the key frames) and the computer interpolates between them to fill in the gaps.
Most animation software such as Blender or Maya provide interactive tools allowing users to pose a character by manipulating its underlying skeleton.
We propose an innovative solution that makes the pose editing process more affordable without compromising the quality of the results. The presented method leverages neural networks to implicitly learn the intricacies of a (human) skeleton and provide simple controls.
Our main goal is to create an intuitive real-time system that can produce appealing poses even for a novice user.

%Simple description of the method
Our framework relies on a few small networks requiring reasonable resources to train, with the added advantage of running quite fast at inference time. The core of our approach is an encoder-decoder trained on skeleton pose data, the task of which is to build a latent representation of the pose space, alleviating some of the limitations of the former. We then train a family of solver networks to work on this latent space in order to generate a pose satisfying user-defined target positions.
%%%%%%%%%%%%%%%%%%%%%%%%%%%%%%%%%%%%%%%%%%%%%%%%%%%%%%%%%%%%%%%%%%%%%%%%%%%%%%%%%%%%%%%%%%%%%%%%%%%%%%%%%%%%%%%%%%%%%%%%%%%%%%%%%%%%%%%%%%%%%%%%%%
\section{Related work}
\label{sect:related_work}
\section{Related Work}
%\mz{We lack a comparison to this paper: https://arxiv.org/abs/2305.14877}
%\anirudh{refine to be more on-topic?}
\iffalse
\paragraph{In-Context Learning} As language models have scaled, the ability to learn in-context, without any weight updates, has emerged. \cite{brown2020language}. While other families of large language models have emerged, in-context learning remains ubiquitous \cite{llama, bloom, gptneo, opt}. Although such as HELM \cite{helm} have arisen for systematic evaluation of \emph{models}, there is no systematic framework to our knowledge for evaluating \emph{prompting methods}, and validating prompt engineering heuristics. The test-suite we propose will ensure that progress in the field of prompt-engineering is structured and objectively evaluated. 

\paragraph{Prompt Engineering Methods} Researchers are interested in the automatic design of high performing instructions for downstream tasks. Some focus on simple heuristics, such as selecting instructions that have the lowest perplexity \cite{lowperplexityprompts}. Other methods try to use large language models to induce an instruction when provided with a few input-output pairs \cite{ape}. Researchers have also used RL objectives to create discrete token sequences that can serve as instructions \cite{rlprompt}. Since the datasets and models used in these works have very little intersection, it is impossible to compare these methods objectively and glean insights. In our work, we evaluate these three methods on a diverse set of tasks and models, and analyze their relative performance. Additionally, we recognize that there are many other interesting angles of prompting that are not covered by instruction engineering \cite{weichain, react, selfconsistency}, but we leave these to future work.

\paragraph{Analysis of Prompting Methods} While most prompt engineering methods focus on accuracy, there are many other interesting dimensions of performance as well. For instance, researchers have found that for most tasks, the selection of demonstrations plays a large role in few-shot accuracy \cite{whatmakesgoodicexamples, selectionmachinetranslation, knnprompting}. Additionally, many researchers have found that even permuting the ordering of a fixed set of demonstrations has a significant effect on downstream accuracy \cite{fantasticallyorderedprompts}. Prompts that are sensitive to the permutation of demonstrations have been shown to also have lower accuracies \cite{relationsensitivityaccuracy}. Especially in low-resource domains, which includes the large public usage of in-context learning, these large swings in accuracy make prompting less dependable. In our test-suite we include sensitivity metrics that go beyond accuracy and allow us to find methods that are not only performant but reliable.

\paragraph{Existing Benchmarks} We recognize that other holistic in-context learning benchmarks exist. BigBench is a large benchmark of 204 tasks that are beyond the capabilities of current LLMs. BigBench seeks to evaluate the few-shot abilities of state of the art large language models, focusing on performance metrics such as accuracy \cite{bigbench}. Similarly, HELM is another benchmark for language model in-context learning ability. Rather than only focusing on performance, HELM branches out and considers many other metrics such as robustness and bias \cite{helm}. Both BigBench and HELM focus on ranking different language model, while fix a generic instruction and prompt format. We instead choose to evaluate instruction induction / selection methods over a fixed set of models. We are the first ever evaluation script that compares different prompt-engineering methods head to head. 
\fi

\paragraph{In-Context Learning and Existing Benchmarks} As language models have scaled, in-context learning has emerged as a popular paradigm and remains ubiquitous among several autoregressive LLM families \cite{brown2020language, llama, bloom, gptneo, opt}. Benchmarks like BigBench \cite{bigbench} and HELM \cite{helm} have been created for the holistic evaluation of these models. BigBench focuses on few-shot abilities of state-of-the-art large language models, while HELM extends to consider metrics like robustness and bias. However, these benchmarks focus on evaluating and ranking \emph{language models}, and do not address the systematic evaluation of \emph{prompting methods}. Although contemporary work by \citet{yang2023improving} also aims to perform a similar systematic analysis of prompting methods, they focus on simple probability-based prompt selection while we evaluate a broader range of methods including trivial instruction baselines, curated manually selected instructions, and sophisticated automated instruction selection.

\paragraph{Automated Prompt Engineering Methods} There has been interest in performing automated prompt-engineering for target downstream tasks within ICL. This has led to the exploration of various prompting methods, ranging from simple heuristics such as selecting instructions with the lowest perplexity \cite{lowperplexityprompts}, inducing instructions from large language models using a few annotated input-output pairs \cite{ape}, to utilizing RL objectives to create discrete token sequences as prompts \cite{rlprompt}. However, these works restrict their evaluation to small sets of models and tasks with little intersection, hindering their objective comparison. %\mz{For paragraphs that only have one work in the last line, try to shorten the paragraph to squeeze in context.}

\paragraph{Understanding in-context learning} There has been much recent work attempting to understand the mechanisms that drive in-context learning. Studies have found that the selection of demonstrations included in prompts significantly impacts few-shot accuracy across most tasks \cite{whatmakesgoodicexamples, selectionmachinetranslation, knnprompting}. Works like \cite{fantasticallyorderedprompts} also show that altering the ordering of a fixed set of demonstrations can affect downstream accuracy. Prompts sensitive to demonstration permutation often exhibit lower accuracies \cite{relationsensitivityaccuracy}, making them less reliable, particularly in low-resource domains.

Our work aims to bridge these gaps by systematically evaluating the efficacy of popular instruction selection approaches over a diverse set of tasks and models, facilitating objective comparison. We evaluate these methods not only on accuracy metrics, but also on sensitivity metrics to glean additional insights. We recognize that other facets of prompting not covered by instruction engineering exist \cite{weichain, react, selfconsistency}, and defer these explorations to future work. 
    
\section{Proposed method}
\label{sect:proposedMethod}
\label{sect:proposed_method}
\subsection{Method overview}

\begin{figure*}[h]
    \centering
    \includegraphics[width=0.75\linewidth]{nn_ik.png}
    \caption{High level overview of the generation setup. The target joint's positions (yellow) are matched as closely as possible, while the other joints (green) should be as close as possible to the starting pose (blue).}
    \label{fig:generation_setup}
\end{figure*}

We propose a method to solve a high level pose design problem in which a pose is modified to reach desired target positions for some of its joints. We leverage the modelling power of neural networks to implicitly learn skeleton constraints from a pre-existing pose database.
Our method, illustrated in Fig. \ref{fig:generation_setup}, relies on two models: an auto-encoder to build an alternative latent pose space, and a solver model operating on this space to solve the pose design problem. 
We also describe an optional post-processing step to smooth out the remaining errors, and \modify{}{outline} a methodology using multiple instances of the solver model at once to work with a varying amount of targets.

\subsection{Data}
\label{sect:materiel}
    
We train the models using a dataset of human poses, obtained by processing multiple available motion-capture datasets from the literature: Emilya \cite{fourati_emilya_2014}, CMU \cite{CMU_BVH}, and the clips from Edinburgh university \cite{holden_deep_2016}. Each animation clip is retargeted to a standard skeleton following the scheme proposed by \cite{HoldenAE2015}. The global translation is removed, and each joint's position is calculated relative to the root joint, which is the projection of the pelvis on the floor. The unified skeleton is composed of 21 joints; using the joints' positions in space, a pose is described by $ 3 \times 21 = 63 $ float values concatenated in a single vector. The dataset is then formed by the individual poses in each clip. Before feeding them to the network we also normalize each pose by subtracting the mean and dividing by the standard deviation of each feature. \modify{}{With a few jittery clips manually removed, the final dataset used in the following experiments is composed of about 1,5 million poses.}

\subsection{Models description}
\subsubsection{Autoencoder}
Auto-encoders are made up of two neural networks tasked to learn efficient encodings of complex data. The encoder maps real data points to a learned, usually more compact, latent space; and the \modify{encoder}{decoder} maps them back to the original data space.
We build such an auto-encoder of poses in order to build a common operating space for the following solvers. Generating points in the latent space allows us to ensure that the output is always a plausible pose, as the decoder is trained to turn any and all latent point into them.

The encoder network is composed of two fully connected layers with \modify{195}{200} neurons and ReLU \cite{relu_2010} activations, followed by an output layer with no activation. The output layer's size is based on the number of
dimensions $d$ in which the latent representations are encoded. We empirically find that $d=64$ yields a good balance of representation accuracy and inference speed. The decoder is the exact reversed replica and uses the same set of weights. 

% alex: 63, un peu étrange de ne pas avoir 64, les infos aiment bien les puissance de 2
%Saida : d'accord avec Alex
The autoencoder's weights are optimized by minimizing the mean squared error (MSE) between the input pose $x$ and its reconstructed equivalent $\hat x$ (Eq. \ref{eq:loss-ae}). In the following sections we refer to the encoder as $E$, the decoder as $D$ and a latent encoding as $z$, i.e. $z = E(x)$ and $\hat x = D(z)$. 

\begin{equation}
    \label{eq:loss-ae}
    L_{ae} = MSE(x, \hat x) = \frac{1}{d} \sum_{i=1}^{d}(x_i - \hat x_i)^2
\end{equation}

The autoencoder is trained for 20 epoch with batches of 256 poses, using the Adam optimizer \cite{kingma_adam_2017} with a learning rate of $0.0001$.

\subsubsection{Pose solver}

An instance of the solver model $S_t$ is specialized to solve the IK problem for $n$ specific targets $t$ and is trained to generate a new pose from an input pose and the desired targets locations. 
As it operates on the latent space built by the autoencoder, it more precisely accepts and outputs a latent pose vector, i.e. with $p_t$ the \modify{}{concatenated} target positions, $\hat z = S_t(z, p_t)$.

The network is composed of three fully connected layers with 126 neurons and ReLu activations, and an output layer with $d$ neurons.

During training, we randomly sample an input pose $x$ from the dataset and feed it to the network. \modify{The targets are generated by taking the positions of the considered joints on a random pose in the same animation clip as the input pose.}{We also sample a second pose $x'$ from the same source clip to use as target.} We found that this association helped the network learning by not relying on random (and possibly unreachable) target positions.

%alex: the loss function in Equation 2 (le 'in')
\modify{Its}{The network's} weights are optimized to minimize the loss function in Eq.\ref{eq:loss-ik} designed to represent its high level objective: \modify{reaching the targets with the associated joints while staying as close to the starting pose as possible}{reaching the targets with the associated joints while retaining a realistic pose}. We guide the network toward this objective by using a modified mean squared error function \modify{Eq.\ref{eq:loss-ik}}{, separating the poses ($x$ in this example) in two sets of joints: $x_{target}$ the joints associated to the targets $t$, and $x_{rest}$ the others} \modify{The generated pose's target joints $\hat x_{target}$ should be close to the input targets $p_t$, while its other joints $\hat x_{rest}$ should minimize their motion}{}.
We introduce a constant $k$ to give more relative importance to the target term of the function\modify{}{, so that the non-targets joints of $x'$ are only used to nudge the final result toward a plausible pose}. In our experiments $k$ is set to $0.01$.\modify{A side effect of our loss function is that the target positions are not an absolute truth to be reached at all cost. The solver is rather encouraged to use them as guides, only reaching them precisely when the starting pose would not require too much of a change.}{}
%alex: k is set to 0.01 => avec un chiffre aussi petit ca a de l'effet quand même ?

\begin{equation}
    \label{eq:loss-ik}
    L_{s} = MSE(\hat x_{target}, x'_{target}) + k \cdot MSE(\hat x _{rest}, x'_{rest})
\end{equation}



An instance of the solver model is trained for \modify{15}{5} epochs with the Adam optimizer using a learning rate of 0.0001 and a batch size of 256.

\subsection{Post processing}

It is a common observation with neural networks working with joints position that the generated positions can be jittery, and the resulting poses can suffer from slight variations in bone lengths. 
Our model is no exception, and while the variation is not visually detectable most of the time, computing the total bone length difference between the input skeleton and the generated pose shows that it is present. These variations are naturally undesirable and can result in visual discomfort on the spectator's end. In order to alleviate the problem we apply an optional post-processing step to the resulting poses to ensure constant bone lengths. We use the backward step from FABRIK as it is very lightweight computation-wise. Our experiments show that following this process  lends better results at a small cost in computing time (see table \ref{table:results}).

\subsection{Solving other targets configurations}
\label{sect:multi-solver}
Even though our solvers are designed to generate a pose considering one to two targets at once, it is possible to use multiple instances side by side and to switch to the correct one with regard to the selected targets. In cases where the user desires to use an arbitrary number of targets
(to suggest a position for a fixed joint for example) we can combine the multiple instances by running them in sequence, i.e. $\hat z = (S_{t3} \circ S_{t2} \circ S_{t1})(z)$ for $t1, t2, t3$ various targets and $S_{ti}$ the solvers trained to reach them.
%%%%%%%%%%%%%%%%%%%%%%%%%%%%%%%%%%%%%%%%%%%%%%%%%%%%%%%%%%%%%%%%%%%%%%%%%%%%%%%%%%%%%%%%%%%%%%%%%%%%%%%%%%%%%%%%%%%%%%%%%%%%%%%%%%%%%%
\section{Results}
\label{sect:results}
%!TEX ROOT = ../../centralized_vs_distributed.tex

\section{{\titlecap{the centralized-distributed trade-off}}}\label{sec:numerical-results}

\revision{In the previous sections we formulated the optimal control problem for a given controller architecture
(\ie the number of links) parametrized by $ n $
and showed how to compute minimum-variance objective function and the corresponding constraints.
In this section, we present our main result:
%\red{for a ring topology with multiple options for the parameter $ n $},
we solve the optimal control problem for each $ n $ and compare the best achievable closed-loop performance with different control architectures.\footnote{
\revision{Recall that small (large) values of $ n $ mean sparse (dense) architectures.}}
For delays that increase linearly with $n$,
\ie $ f(n) \propto n $, 
we demonstrate that distributed controllers with} {few communication links outperform controllers with larger number of communication links.}

\textcolor{subsectioncolor}{Figure~\ref{fig:cont-time-single-int-opt-var}} shows the steady-state variances
obtained with single-integrator dynamics~\eqref{eq:cont-time-single-int-variance-minimization}
%where we compare the standard multi-parameter design 
%with a simplified version \tcb{that utilizes spatially-constant feedback gains
and the quadratic approximation~\eqref{eq:quadratic-approximation} for \revision{ring topology}
with $ N = 50 $ nodes. % and $ n\in\{1,\dots,10\} $.
%with $ N = 50 $, $ f(n) = n $ and $ \tau_{\textit{min}} = 0.1 $.
%\autoref{fig:cont-time-single-int-err} shows the relative error, defined as
%\begin{equation}\label{eq:relative-error}
%	e \doteq \dfrac{\optvarx-\optvar}{\optvar}
%\end{equation}
%where $ \optvar $ and $ \optvarx $ denote the the optimal and sub-optimal scalar variances, respectively.
%The performance gap is small
%and becomes negligible for large $ n $.
{The best performance is achieved for a sparse architecture with  $ n = 2 $ 
in which each agent communicates with the two closest pairs of neighboring nodes. 
This should be compared and contrasted to nearest-neighbor and all-to-all 
communication topologies which induce higher closed-loop variances. 
Thus, 
the advantage of introducing additional communication links diminishes 
beyond}
{a certain threshold because of communication delays.}

%For a linear increase in the delay,
\textcolor{subsectioncolor}{Figure~\ref{fig:cont-time-double-int-opt-var}} shows that the use of approximation~\eqref{eq:cont-time-double-int-min-var-simplified} with $ \tilde{\gvel}^* = 70 $
identifies nearest-neighbor information exchange as the {near-optimal} architecture for a double-integrator model
with ring topology. 
This can be explained by noting that the variance of the process noise $ n(t) $
in the reduced model~\eqref{eq:x-dynamics-1st-order-approximation}
is proportional to $ \nicefrac{1}{\gvel} $ and thereby to $ \taun $,
according to~\eqref{eq:substitutions-4-normalization},
making the variance scale with the delay.

%\mjmargin{i feel that we need to comment about different results that we obtained for CT and DT double-intergrator dynamics (monotonic deterioration of performance for the former and oscillations for the latter)}
\revision{\textcolor{subsectioncolor}{Figures~\ref{fig:disc-time-single-int-opt-var}--\ref{fig:disc-time-double-int-opt-var}}
show the results obtained by solving the optimal control problem for discrete-time dynamics.
%which exhibit similar trade-offs.
The oscillations about the minimum in~\autoref{fig:disc-time-double-int-opt-var}
are compatible with the investigated \tradeoff~\eqref{eq:trade-off}:
in general, 
the sum of two monotone functions does not have a unique local minimum.
Details about discrete-time systems are deferred to~\autoref{sec:disc-time}.
Interestingly,
double integrators with continuous- (\autoref{fig:cont-time-double-int-opt-var}) ad discrete-time (\autoref{fig:disc-time-double-int-opt-var}) dynamics
exhibits very different trade-off curves,
whereby performance monotonically deteriorates for the former and oscillates for the latter.
While a clear interpretation is difficult because there is no explicit expression of the variance as a function of $ n $,
one possible explanation might be the first-order approximation used to compute gains in the continuous-time case.
%which reinforce our thesis exposed in~\autoref{sec:contribution}.

%\begin{figure}
%	\centering
%	\includegraphics[width=.6\linewidth]{cont-time-double-int-opt-var-n}
%	\caption{Steady-state scalar variance for continuous-time double integrators with $ \taun = 0.1n $.
%		Here, the \tradeoff is optimized by nearest-neighbor interaction.
%	}
%	\label{fig:cont-time-double-int-opt-var-lin}
%\end{figure}
}

\begin{figure}
	\centering
	\begin{minipage}[l]{.5\linewidth}
		\centering
		\includegraphics[width=\linewidth]{random-graph}
	\end{minipage}%
	\begin{minipage}[r]{.5\linewidth}
		\centering
		\includegraphics[width=\linewidth]{disc-time-single-int-random-graph-opt-var}
	\end{minipage}
	\caption{Network topology and its optimal {closed-loop} variance.}
	\label{fig:general-graph}
\end{figure}

Finally,
\autoref{fig:general-graph} shows the optimization results for a random graph topology with discrete-time single integrator agents. % with a linear increase in the delay, $ \taun = n $.
Here, $ n $ denotes the number of communication hops in the ``original" network, shown in~\autoref{fig:general-graph}:
as $ n $ increases, each agent can first communicate with its nearest neighbors,
then with its neighbors' neighbors, and so on. For a control architecture that utilizes different feedback gains for each communication link
	(\ie we only require $ K = K^\top $) we demonstrate that, in this case, two communication hops provide optimal closed-loop performance. % of the system.}

Additional computational experiments performed with different rates $ f(\cdot) $ show that the optimal number of links increases for slower rates: 
for example, 
the optimal number of links is larger for $ f(n) = \sqrt{n} $ than for $ f(n) = n $. 
\revision{These results are not reported because of space limitations.}

%\section{Discussion}
%\mySection{Related Works and Discussion}{}
\label{chap3:sec:discussion}

In this section we briefly discuss the similarities and differences of the model presented in this chapter, comparing it with some related work presented earlier (Chapter \ref{chap1:artifact-centric-bpm}). We will mention a few related studies and discuss directly; a more formal comparative study using qualitative and quantitative metrics should be the subject of future work.

Hull et al. \citeyearpar{hull2009facilitating} provide an interoperation framework in which, data are hosted on central infrastructures named \textit{artifact-centric hubs}. As in the work presented in this chapter, they propose mechanisms (including user views) for controlling access to these data. Compared to choreography-like approach as the one presented in this chapter, their settings has the advantage of providing a conceptual rendezvous point to exchange status information. The same purpose can be replicated in this chapter's approach by introducing a new type of agent called "\textit{monitor}", which will serve as a rendezvous point; the behaviour of the agents will therefore have to be slightly adapted to take into account the monitor and to preserve as much as possible the autonomy of agents.

Lohmann and Wolf \citeyearpar{lohmann2010artifact} abandon the concept of having a single artifact hub \cite{hull2009facilitating} and they introduce the idea of having several agents which operate on artifacts. Some of those artifacts are mobile; thus, the authors provide a systematic approach for modelling artifact location and its impact on the accessibility of actions using a Petri net. Even though we also manipulate mobile artifacts, we do not model artifact location; rather, our agents are equipped with capabilities that allow them to manipulate the artifacts appropriately (taking into account their location). Moreover, our approach considers that artifacts can not be remotely accessed, this increases the autonomy of agents.

The process design approach presented in this chapter, has some conceptual similarities with the concept of \textit{proclets} proposed by Wil M. P. van der Aalst et al. \citeyearpar{van2001proclets, van2009workflow}: they both split the process when designing it. In the model presented in this chapter, the process is split into execution scenarios and its specification consists in the diagramming of each of them. Proclets \cite{van2001proclets, van2009workflow} uses the concept of \textit{proclet-class} to model different levels of granularity and cardinality of processes. Additionally, proclets act like agents and are autonomous enough to decide how to interact with each other.

The model presented in this chapter uses an attributed grammar as its mathematical foundation. This is also the case of the AWGAG model by Badouel et al. \citeyearpar{badouel14, badouel2015active}. However, their model puts stress on modelling process data and users as first class citizens and it is designed for Adaptive Case Management.

To summarise, the proposed approach in this chapter allows the modelling and decentralized execution of administrative processes using autonomous agents. In it, process management is very simply done in two steps. The designer only needs to focus on modelling the artifacts in the form of task trees and the rest is easily deduced. Moreover, we propose a simple but powerful mechanism for securing data based on the notion of accreditation; this mechanism is perfectly composed with that of artifacts. The main strengths of our model are therefore : 
\begin{itemize}
	\item The simplicity of its syntax (process specification language), which moreover (well helped by the accreditation model), is suitable for administrative processes;
	\item The simplicity of its execution model; the latter is very close to the blockchain's execution model \cite{hull2017blockchain, mendling2018blockchains}. On condition of a formal study, the latter could possess the same qualities (fault tolerance, distributivity, security, peer autonomy, etc.) that emanate from the blockchain;
	\item Its formal character, which makes it verifiable using appropriate mathematical tools;
	\item The conformity of its execution model with the agent paradigm and service technology.
\end{itemize}
In view of all these benefits, we can say that the objectives set for this thesis have indeed been achieved. However, the proposed model is perfectible. For example, it can be modified to permit agents to respond incrementally to incoming requests as soon as any prefix of the extension of a bud is produced. This makes it possible to avoid the situation observed on figure \ref{chap3:fig:execution-figure-4} where the associated editor is informed of the evolution of the subtree resulting from $C$ only when this one is closed. All the criticisms we can make of the proposed model in particular, and of this thesis in general, have been introduced in the general conclusion (page \pageref{chap5:general-conclusion}) of this manuscript.





\section{Conclusion and perspectives}
\label{sect:conclusion}
% \vspace{-0.5em}
\section{Conclusion}
% \vspace{-0.5em}
Recent advances in multimodal single-cell technology have enabled the simultaneous profiling of the transcriptome alongside other cellular modalities, leading to an increase in the availability of multimodal single-cell data. In this paper, we present \method{}, a multimodal transformer model for single-cell surface protein abundance from gene expression measurements. We combined the data with prior biological interaction knowledge from the STRING database into a richly connected heterogeneous graph and leveraged the transformer architectures to learn an accurate mapping between gene expression and surface protein abundance. Remarkably, \method{} achieves superior and more stable performance than other baselines on both 2021 and 2022 NeurIPS single-cell datasets.

\noindent\textbf{Future Work.}
% Our work is an extension of the model we implemented in the NeurIPS 2022 competition. 
Our framework of multimodal transformers with the cross-modality heterogeneous graph goes far beyond the specific downstream task of modality prediction, and there are lots of potentials to be further explored. Our graph contains three types of nodes. While the cell embeddings are used for predictions, the remaining protein embeddings and gene embeddings may be further interpreted for other tasks. The similarities between proteins may show data-specific protein-protein relationships, while the attention matrix of the gene transformer may help to identify marker genes of each cell type. Additionally, we may achieve gene interaction prediction using the attention mechanism.
% under adequate regulations. 
% We expect \method{} to be capable of much more than just modality prediction. Note that currently, we fuse information from different transformers with message-passing GNNs. 
To extend more on transformers, a potential next step is implementing cross-attention cross-modalities. Ideally, all three types of nodes, namely genes, proteins, and cells, would be jointly modeled using a large transformer that includes specific regulations for each modality. 

% insight of protein and gene embedding (diff task)

% all in one transformer

% \noindent\textbf{Limitations and future work}
% Despite the noticeable performance improvement by utilizing transformers with the cross-modality heterogeneous graph, there are still bottlenecks in the current settings. To begin with, we noticed that the performance variations of all methods are consistently higher in the ``CITE'' dataset compared to the ``GEX2ADT'' dataset. We hypothesized that the increased variability in ``CITE'' was due to both less number of training samples (43k vs. 66k cells) and a significantly more number of testing samples used (28k vs. 1k cells). One straightforward solution to alleviate the high variation issue is to include more training samples, which is not always possible given the training data availability. Nevertheless, publicly available single-cell datasets have been accumulated over the past decades and are still being collected on an ever-increasing scale. Taking advantage of these large-scale atlases is the key to a more stable and well-performing model, as some of the intra-cell variations could be common across different datasets. For example, reference-based methods are commonly used to identify the cell identity of a single cell, or cell-type compositions of a mixture of cells. (other examples for pretrained, e.g., scbert)


%\noindent\textbf{Future work.}
% Our work is an extension of the model we implemented in the NeurIPS 2022 competition. Now our framework of multimodal transformers with the cross-modality heterogeneous graph goes far beyond the specific downstream task of modality prediction, and there are lots of potentials to be further explored. Our graph contains three types of nodes. while the cell embeddings are used for predictions, the remaining protein embeddings and gene embeddings may be further interpreted for other tasks. The similarities between proteins may show data-specific protein-protein relationships, while the attention matrix of the gene transformer may help to identify marker genes of each cell type. Additionally, we may achieve gene interaction prediction using the attention mechanism under adequate regulations. We expect \method{} to be capable of much more than just modality prediction. Note that currently, we fuse information from different transformers with message-passing GNNs. To extend more on transformers, a potential next step is implementing cross-attention cross-modalities. Ideally, all three types of nodes, namely genes, proteins, and cells, would be jointly modeled using a large transformer that includes specific regulations for each modality. The self-attention within each modality would reconstruct the prior interaction network, while the cross-attention between modalities would be supervised by the data observations. Then, The attention matrix will provide insights into all the internal interactions and cross-relationships. With the linearized transformer, this idea would be both practical and versatile.

% \begin{acks}
% This research is supported by the National Science Foundation (NSF) and Johnson \& Johnson.
% \end{acks}

\bibliographystyle{unsrt}
\bibliography{bibliography}


\end{document}
