% Nouvelle partie (CASA)

In order to evaluate the results of our method we integrate our solver in an example posing software and compare its outcome with a comparable, non-neural method: FABRIK \cite{aristidou_fabrik:_2011}. We pick FABRIK for the traits that make it a popular IK solver: its simplicity and fast convergence times. We implement a full-body human skeleton solver as described in \cite{aristidou_extending_2016} but stay as close as possible to our method setup process by not manually implementing any joint orient constraints.

\subsection{Visual results}
\begin{figure}[h]
    \centering
    \includegraphics[width=0.45\textwidth]{results-casa.png}
    \caption{Starting from a pose and targets for two joints, an IK solver like FABRIK (middle) generates less realistic poses than our neural solver (bottom).}
    \label{fig:results-hands}
    \vspace{-15pt}
\end{figure}

Fig. \ref{fig:results-hands} showcases an example of how our method can be used to edit a pose by moving the targets around. in this case a single solver with the targets set to both skeleton's hands is used.

Our solver yields poses satisfying the constraints without breaking the implicit skeleton rules: the distance between limbs is constant, self-occlusion is avoided and the poses appear natural.
The side-by-side comparisons with FABRIK's results highlight the limits of working on kinematics chains with no prior on the human skeleton.

Example $(1)$ illustrates how the targets are used as guides rather than fixed, unbreakable rules. While FABRIK extends the full body, our solver generates a new pose where the torso is slightly twisted towards the right-hand target while the legs are spread to mimic maintaining balance.
Even though our method is aimed toward beginner animators, experienced ones could also find it useful. \modify{The results may lack precision but could a be sued}{It could for example be used} as a fast prototyping tool to flesh out the pose, while switching to more accurate and manipulation-heavy tools to focus on the details later on.

Examples of real-time usage of our method can also be found in the accompanying video.

\subsection{Combining solvers}

Figure \ref{fig:multi-solvers} demonstrates an example with the multi-solver setup described in \ref{sect:multi-solver}. In this example three solvers are used at once: for the two hands, the two ankles and the head. Compared to the FABRIK result, our method yields a plausible pose: the skeleton is bent down to meet the head target, but the general orientation of the pose is kept intact. The limbs also retain some sort of curvature rather than fully extending in an unnatural way. Here again some of the targets are not strictly reached, as the pose generated by earlier solvers in the chain are modified by the others further down, but the guidance provided by the targets is respected. This setup also incurs slightly slower runtimes (see Table \ref{table:results}) but is still faster than FABRIK.
%
\begin{figure}[h!]
    \centering
    \includegraphics[width=0.9\linewidth]{multi-solver-casa.png}
    \caption{Sample results solving multiple targets with a sequence of neural solvers. Targets are shown in red. 
    }
    \label{fig:multi-solvers}
\end{figure}
%
\subsection{Run times}


At run-time the complexity of the solver is fixed and regardless of the targets' positions, a single pass through the networks, which can be seen as just a few matrix multiplications, is enough to produce a result pose. This property coupled with the relatively small size of the networks allow for a fast solving process, as highlighted in table \ref{table:results}.

%%% Numerical results
\begin{table}
    \centering
    \begin{tabularx}{\linewidth} {
         >{\raggedright\arraybackslash}X
        | >{\centering\arraybackslash}X
        | >{\centering\arraybackslash}X
    }
        \hline
        Method & Memory footprint (kB) & Runtime (ms)**\\
        \hline
        FABRIK (2)  & -   & 6.56     \\
        Ours (2)     & 442 & 1.47 (3.03*) \\
        FABRIK (5)  & -   & 6.74 \\
        Ours (5)     & 826 & 3.36 (4.58*) \\
        \hline
    \end{tabularx}
    \\
    \vspace{2mm}
    *With post-processing \\
    **Average over 1000 random iterations
    \caption{Comparative numeric results of the neural and FABRIK solvers with two and five end-effectors (using the combined solver method). All experiments are run on a single CPU thread.}
    \label{table:results}
    \vspace{-15pt}
\end{table}

Compared to other data-driven pose methods, the computing-heavy part of our process is done once at training time. Even so, the training itself is kept short thanks to the modest size of the networks: around an hour for the auto-encoder and 15 minutes for the solvers, on a single GPU.

\subsection{Memory footprint}
An advantage of neural networks is the low memory footprint they hold. While other data-driven pose design methods require the pose database (or a compressed version of it) to be kept in memory, neural networks only require their trained weights. These can be quite heavy as well in the case of large models, but as ours are quite small, so are their weights. As a comparison point, \cite{wu_posing_2009} discloses a 30MB memory footprint while our full-body solver only takes up 826kB.

\begin{modified}
\subsection{Comparison with other pose edition approaches}

Huang et al. \cite{Huang_IK_MGDM_2017} proposed a general comparison chart for full-body IK methods, ranking common approaches by speed and subjective quality. Adding our solution to the chart (Fig. \ref{fig:comparisons}) highlights the useful spot it fills by striking a good balance between speed and accuracy. 
To the best of our knowledge, this work presents the first method leveraging neural networks for pose edition. It stands apart from previous learning-based approaches as the first one to combine real-time edition speed with fully learned skeleton constraints. In comparison, NAT-IK \cite{Huang_IK_MGDM_2017} uses soft learned constraints but still requires explicit, manual ones to be set. \cite{wu_posing_2009} does not, but the poses are not generated in real time. 

\begin{figure}[h]
    \centering
    \includegraphics[width=0.5\textwidth]{comparison.png}
    \caption{
    \begin{modified}
    General comparison of various full-body IK methods in terms of speed and quality.
    Style IK \cite{wu_posing_2009},
    NAT-IK \cite{Huang_IK_MGDM_2017},
    JDLS \cite{buss_selectively_2005},
    CCD \cite{wang_ccd_1991}
    FABRIK \cite{aristidou_fabrik:_2011}
    \end{modified}
    }
    \label{fig:comparisons}
\end{figure}
\end{modified}

