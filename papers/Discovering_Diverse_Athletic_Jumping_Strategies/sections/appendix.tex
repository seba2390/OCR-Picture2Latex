\appendix

\begin{table}[t]
\centering
\caption{Representative take-off state features for discovered high jumps.}
\begin{tabular}{|c|c|c|c|c|c|} 
\hline
Strategy & $v_z$ & $\omega_x$ & $\omega_z$ & $\alpha$ \\ 
\hline
Fosbury Flop & -2.40 & -3.00 & 1.00 & -0.05 \\
\hline
Western Roll (up) & -0.50 & 1.00 & -1.00 & 2.09 \\
\hline
Straddle & -2.21 & 1.00 & 0.88 & 1.65 \\
\hline
Front Kick & -0.52 & 1.00 & -0.26 & 0.45 \\
\hline
Side Dive & -1.83 & -2.78 & -0.32 & 1.18 \\
\hline
Side Jump & -1.99 & -1.44 & 0.44 & 0.70 \\
\hline
\end{tabular}
\label{tb:take-off-states}
\end{table}

%%%%%%%%%%%%%%%%%%%%%%%%%%%%%%%%%%%%%%%%%%%%%%%%

\begin{table}[t]
\centering
\caption{Representative take-off state features for discovered obstacle jumps.}
\begin{tabular}{|c|c|c|c|} 
\hline
Strategy & $\omega_x$ & $\omega_y$ & $\omega_z$ \\ 
\hline
Front Kick & 1.15 & -1.11 & 3.89 \\
\hline
Side Kick & 3.00 & 3.00 & -2.00 \\
\hline
Twist Jump (c) & -1.50 & 1.50 & -2.00 \\
\hline
Straddle & 0.00 & 0.00 & 1.00  \\
\hline
Twist Jump (cc) & -2.67 & 0.00 & -1.44 \\
\hline
Dive Turn & -0.74 & -2.15 & -0.41 \\
\hline
\end{tabular}
\label{tb:take-off-states-box}
\end{table}


\begin{figure}[t]
    \centering
    \begin{subfigure}[b]{\linewidth}
        \centering
        \scalebox{0.56}{\input{images/learning-curves/rwd.pgf}}
    \end{subfigure}
    \begin{subfigure}[b]{\linewidth}
        \centering
        \scalebox{0.56}{\begin{table}[t]
\centering
\caption{Curriculum parameters for learning jumping tasks. $z$ parameterizes the task difficulty, i.e., the crossbar height in high jumps and the obstacle width in obstacle jumps. $z_\text{min}$ and $z_\text{max}$ specify the range of $z$, and $\Delta z$ is the increment when moving to a higher difficulty level. $R_T$ is the accumulated reward threshold to move on to the next curriculum difficulty.}
\begin{tabular}{|c|c|c|c|c|c|} 
\hline
Task & $z_\text{min}$(cm) & $z_\text{max}$(cm) & $\Delta z$(cm) & $R_T$ \\ 
\hline
High jump     & 50   & 200 & 1 & 30     \\
\hline
Obstacle jump & 5  & 250 & 5 & 50   \\
\hline
\end{tabular}
\label{tb:curriculum}
\end{table}
}
    \end{subfigure}
    \caption{{Stage 1 DRL learning and curriculum scheduling curves for two high jump strategies. As DRL learning is stochastic, the curves shown are the average of five training runs. The shaded regions indicates the standard deviation.}}
    \label{fig:learning-curves}
\end{figure}

\section{Representative Take-off State Features}
\label{app:takeoffStates}
We list representative take-off state features discovered through BDS in Table~\ref{tb:take-off-states} for high jumps and Table~\ref{tb:take-off-states-box} for obstacle jumps. The approach angle $\alpha$ for high jumps is defined as the wall orientation in a facing-direction invariant frame. The orientation of the wall is given by the line $x\text{sin}\alpha - z\text{cos}\alpha = 0$. 

\section{Learning Curves}
\label{app:curves}
{We plot Stage 1 DRL learning and curriculum scheduling curves for two high jump strategies in Figure~\ref{fig:learning-curves}. An initial solution for the starting bar height $0.5m$ can be learned relatively quickly. After a certain bar height has been reached (around $1.4m$), the return starts to drop  because larger action offsets are needed to jump higher, which decreases the $r_{naturalness}$ in Equation~\ref{eq:r-pvae} and therefore the overall return in Equation~\ref{eq:stage1-reward}. Subjectively speaking, the learned motions remain as natural for high crossbars, as the lower return is due to the penalty on action offsets.}
