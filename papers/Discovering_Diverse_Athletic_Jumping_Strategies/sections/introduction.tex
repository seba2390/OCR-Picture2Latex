\section{Introduction}

Athletic endeavors are a function of strength, skill, and strategy. For the high-jump, the choice of strategy has been of particular historic importance. Innovations in techniques or strategies have repeatedly redefined world records over the past 150 years, culminating in the now well-established Fosbury flop (Brill bend) technique. In this paper, we demonstrate how to discover diverse strategies, as realized through physics-based controllers which are trained using reinforcement learning. We show that natural high-jump strategies can be learned without recourse to motion capture data, with the exception of a single generic run-up motion capture clip. We further demonstrate diverse solutions to a box-jumping task.

Several challenges stand in the way of being able to discover iconic athletic strategies such as those used for the high jump. The motions involve high-dimensional states and actions. The task is defined by a sparse reward, i.e., successfully making it over the bar or not. It is not obvious how to ensure that the resulting motions are natural in addition to being physically plausible. Lastly, the optimization landscape is multimodal in nature.

We take several steps to address these challenges. First, we identify the take-off state as a strong determinant of the resulting jump strategy, which is characterized by low-dimensional features such as the net angular velocity and approach angle in preparation for take-off. To efficiently explore the take-off states, we employ Bayesian diversity optimization. Given a desired take-off state, we first train a run-up controller that imitates a single generic run-up motion capture clip while also targeting the desired take-off state. The subsequent jump control policy is trained with the help of a curriculum, without any recourse to motion capture data. We make use of a pose variational autoencoder to define an action space that helps yield more natural poses and motions. We further enrich unique strategy variations by a second optimization stage which reuses the best discovered take-off states and encourages novel control policies.

\newpage
In summary, our contributions include:
\begin{itemize}
    \item A system which can discover common athletic high jump strategies,
        and execute them using learned controllers and physics-based simulation. The discovered strategies include the Fosbury flop (Brill bend), Western Roll, and a number of other styles. We further evaluate the system on box jumps and on a number of high-jump variations and ablations.
    \item The use of Bayesian diversity search for sample-efficient exploration of take-off states, which are strong determinants of resulting strategies.
    \item Pose variational autoencoders used in support of learning natural athletic motions.
\end{itemize}

