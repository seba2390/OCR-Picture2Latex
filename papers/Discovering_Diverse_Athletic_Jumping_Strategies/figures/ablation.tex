\begin{figure*}
    \centering
    \begin{subfigure}[b]{0.99\linewidth}
        \includegraphics[width=0.495\linewidth]{images/real/fosbury-novae.jpg}
        \includegraphics[width=0.495\linewidth]{images/real/straddle-novae.jpg}
        \caption{High jumps trained without P-VAE, given the initial state of Fosbury Flop and Straddle respectively. Please compare with Figure~\ref{fig:highjump-flop} and Figure~\ref{fig:highjump-straddle}.}
    \end{subfigure}
    \begin{subfigure}[b]{0.99\linewidth}
        \includegraphics[width=0.495\linewidth]{images/strategy-box/straddle-novae.jpg}
        \includegraphics[width=0.495\linewidth]{images/strategy-box/dive-novae.jpg}
        \caption{Obstacle jumps trained without P-VAE, given the initial state of Straddle and Twist Jump (cc) respectively. Please compare with Figure~\ref{fig:obstacle-straddle} and Figure~\ref{fig:obstacle-twistJumpCC}.}
    \end{subfigure}
    \caption{Jumping strategies learned without P-VAE. Although the character can still complete the tasks, the poses are less natural.}
    \label{fig:ablation}
\end{figure*}
