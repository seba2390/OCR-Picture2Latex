\begin{figure}
    \centering
    \begin{subfigure}[b]{0.99\linewidth}
        \includegraphics[width=0.99\linewidth]{images/real/force-limit.jpg}
        \caption{Fosbury Flop -- max height=$160cm$, performed by a character with a weaker take-off leg, whose take-off hip, knee and ankle torque limits are set to $60\%$ of the normal values.}
        \label{fig:variation-weakerLeg}
    \end{subfigure}
    \begin{subfigure}[b]{0.99\linewidth}
        \includegraphics[width=0.99\linewidth]{images/real/joint-limit.jpg}
        \caption{Fosbury Flop -- max height=$150cm$, performed by a character with an inflexible spine that does not permit arching backwards.}
        \label{fig:variation-inflexibleSpine}
    \end{subfigure}
    \begin{subfigure}[b]{0.99\linewidth}
        \includegraphics[width=0.99\linewidth]{images/real/fixed-knee.jpg}
        \caption{Scissor Kick -- max height=$130cm$, learned by a character with a cast on its take-off leg.}
        \label{fig:variation-fixedKnee}
    \end{subfigure}
    \begin{subfigure}[b]{0.99\linewidth}
    \includegraphics[width=0.99\linewidth]{images/real/foot-land.jpg}
    \caption{Front Kick -- max height=$120cm$, performed with an additional constraint requiring landing on feet.}
    \label{fig:variation-landonFoot}
    \end{subfigure}
    \caption{High jump variations. The first three policies are trained from the initial state of the Fosbury Flop discovered in Stage 1, and the last policy is trained from the initial state of the Front Kick discovered in Stage 1.}
    \label{fig:variations}
\end{figure}
