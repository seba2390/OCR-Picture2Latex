\begin{figure}
    \centering
    \begin{subfigure}[b]{0.99\linewidth}
        \includegraphics[width=0.99\linewidth]{images/strategy-box/direct-jump.jpg}
        \caption{Front Kick -- max width=$150cm$}
        \label{fig:obstacle-frontKick}
    \end{subfigure}
    \begin{subfigure}[b]{0.99\linewidth}
        \includegraphics[width=0.99\linewidth]{images/strategy-box/kick-left.jpg}
        \caption{Side Kick -- max width=$150cm$ }
        \label{fig:obstacle-sideKick}
    \end{subfigure}
    \begin{subfigure}[b]{0.99\linewidth}
        \includegraphics[width=0.99\linewidth]{images/strategy-box/self-rotate.jpg}
        \caption{Twist Jump (clockwise) -- max width=$150cm$}
        \label{fig:obstacle-twistJumpC}
    \end{subfigure}
    \begin{subfigure}[b]{0.99\linewidth}
        \includegraphics[width=0.99\linewidth]{images/strategy-box/straddle.jpg}
        \caption{ Straddle -- max width=$215cm$}
        \label{fig:obstacle-straddle}
    \end{subfigure}
    \begin{subfigure}[b]{0.99\linewidth}
        \includegraphics[width=0.99\linewidth]{images/strategy-box/dive-facing-up.jpg}
        \caption{Twist Jump (counterclockwise) -- max width=$250cm$}
        \label{fig:obstacle-twistJumpCC}
    \end{subfigure}
    \begin{subfigure}[b]{0.99\linewidth}
        \includegraphics[width=0.99\linewidth]{images/strategy-box/dive-facing-down.jpg}
        \caption{Dive Turn -- max width=$250cm$ }
        \label{fig:obstacle-diveTurn}
    \end{subfigure}
    \caption{Six obstacle jump strategies discovered by our learning framework in Stage 1, ordered by their maximum cleared obstacle width. For some of the strategies, the obstacle is split into two parts connected with dashed lines to enable better visualization of the poses over the obstacle.}
    \label{fig:obstacleJumps1}
\end{figure}
