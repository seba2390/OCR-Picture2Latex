\begin{figure}
    \centering
    \begin{subfigure}[b]{0.99\linewidth}
        \includegraphics[width=0.99\linewidth]{images/real/straddle.jpg}
        \caption{Straddle -- max height=$190cm$}
        \label{fig:highjump-straddle}
    \end{subfigure}
    \begin{subfigure}[b]{0.99\linewidth}
        \includegraphics[width=0.99\linewidth]{images/real/front-kick.jpg}
         \caption{Front Kick -- max height=$180cm$}
         \label{fig:highjump-frontkick}
    \end{subfigure}
    \begin{subfigure}[b]{0.99\linewidth}
        \includegraphics[width=0.99\linewidth]{images/real/western_roll_1.6.jpg}
         \caption{Western Roll (facing up) -- max height=$160cm$}
         \label{fig:highjump-rollup}
    \end{subfigure}
    \begin{subfigure}[b]{0.99\linewidth}
        \includegraphics[width=0.99\linewidth]{images/real/scissor.jpg}
         \caption{Scissor Kick -- max height=$150cm$}
         \label{fig:highjump-scissor}
    \end{subfigure}
    \begin{subfigure}[b]{0.99\linewidth}
        \includegraphics[width=0.99\linewidth]{images/real/side_dive.jpg}
         \caption{Side Dive -- max height=$130cm$}
         \label{fig:highjump-sidedive}
    \end{subfigure}
    \begin{subfigure}[b]{0.99\linewidth}
        \includegraphics[width=0.99\linewidth]{images/real/side-jump-2.jpg}
         \caption{Side Jump -- max height=$110cm$}
         \label{fig:highjump-sidejump}
    \end{subfigure}
    \caption{Six of the eight high jump strategies discovered by our learning framework, ordered by their maximum cleared height. Fosbury Flop and Western Roll (facing sideways) are shown in Figure~\ref{fig:teaser}. Note that Western Roll (facing up) and Scissor Kick differ in the choice of inner or outer leg as the take-off leg. {The Western Roll (facing sideways) and the Scissor Kick are learned in Stage 2. All other strategies are discovered in Stage 1.}}
    \label{fig:highJumps}
\end{figure}

