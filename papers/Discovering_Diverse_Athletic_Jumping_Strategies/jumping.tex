%\documentclass[acmtog,anonymous,review]{acmart}
\documentclass[acmtog]{acmart}
\acmSubmissionID{330}

\usepackage{booktabs} % For formal tables

% TOG prefers author-name bib system with square brackets
\citestyle{acmauthoryear}
%\setcitestyle{nosort,square} % nosort to allow for manual chronological ordering

\usepackage{multirow}
\usepackage{caption}
\usepackage{subcaption}
\usepackage{soul}
\usepackage{pgf,tikz,pgfplots}
\usepackage[ruled,linesnumbered]{algorithm2e} % For algorithms
\usepackage{comment}
\renewcommand{\algorithmcfname}{ALGORITHM}
\SetAlFnt{\small}
\SetAlCapFnt{\small}
\SetAlCapNameFnt{\small}
\SetAlCapHSkip{0pt}

% Metadata Information
\acmJournal{TOG}

\setcopyright{acmlicensed}\acmJournal{TOG}
\acmYear{2021}\acmVolume{40}\acmNumber{4}\acmArticle{91}\acmMonth{8} \acmDOI{10.1145/3450626.3459817}

% Document starts
\begin{document}
\title{Discovering Diverse Athletic Jumping Strategies}

\author{Zhiqi Yin}
\affiliation{
  \institution{Simon Fraser University}
     \city{Burnaby}
   \country{Canada}
}
\email{zhiqi_yin@sfu.ca}
\author{Zeshi Yang}
\affiliation{
  \institution{Simon Fraser University}
   \city{Burnaby}
   \country{Canada}
}
\email{zeshi_yang@sfu.ca}
\author{Michiel van de Panne}
\affiliation{
  \institution{University of British Columbia}
   \city{Vancouver}
   \country{Canada}
}
\email{van@cs.ubc.ca}
\author{KangKang Yin}
\affiliation{
  \institution{Simon Fraser University}
   \city{Burnaby}
   \country{Canada}
}
\email{kkyin@sfu.ca}


\newcommand*{\clip}{\mathop{\mathrm{Clip}}\nolimits}
\newcommand*{\diag}{\mathop{\mathrm{diag}}\nolimits}

\begin{teaserfigure}
 \centering
 \begin{subfigure}[b]{0.46\linewidth}
    \includegraphics[width=0.99\linewidth]{images/real/fosbury_arrow.jpg}
    \caption{Fosbury Flop -- max height=$200cm$}
    \label{fig:highjump-flop}
 \end{subfigure}
 \begin{subfigure}[b]{0.53\linewidth}
    \includegraphics[width=0.99\linewidth]{images/real/western_roll_1.9_arrow.jpg}
     \caption{Western Roll (facing sideways) -- max height=$195cm$}
    \label{fig:highjump-rollside}
 \end{subfigure}
 \caption{Two of the eight high jump strategies discovered by our two-stage learning framework, as achieved by physics-based control policies. The first stage is a sample-efficient Bayesian diversity search algorithm that explores the space of take-off states, as indicated by the black arrows. In the second stage we explicitly encourage novel policies given a fixed initial state discovered in the first stage.} 
\label{fig:teaser}
\end{teaserfigure}


\begin{abstract}
We present a framework that enables the discovery of diverse and natural-looking motion strategies for athletic skills such as the high jump. The strategies are realized as control policies for physics-based characters. Given a task objective and an initial character configuration, the combination of physics simulation and deep reinforcement learning (DRL) provides a suitable starting point for automatic control policy training. To facilitate the learning of realistic human motions, we propose a Pose Variational Autoencoder (P-VAE) to constrain the actions to a subspace of natural poses. In contrast to motion imitation methods, a rich variety of novel strategies can naturally emerge by exploring initial character states through a sample-efficient Bayesian diversity search (BDS) algorithm. A second stage of optimization that encourages novel policies can further enrich the unique strategies discovered. Our method allows for the discovery of diverse and novel strategies for athletic jumping motions such as high jumps and obstacle jumps with no motion examples and less reward engineering than prior work.
\end{abstract}

%
% The code below should be generated by the tool at
% http://dl.acm.org/ccs.cfm
% Please copy and paste the code instead of the example below.
%
\begin{CCSXML}
<ccs2012>
<concept>
<concept_id>10010147.10010371.10010352</concept_id>
<concept_desc>Computing methodologies~Animation</concept_desc>
<concept_significance>500</concept_significance>
</concept>
<concept>
<concept_id>10010147.10010371.10010352.10010379</concept_id>
<concept_desc>Computing methodologies~Physical simulation</concept_desc>
<concept_significance>300</concept_significance>
</concept>
<concept>
<concept_id>10003752.10010070.10010071.10010261</concept_id>
<concept_desc>Theory of computation~Reinforcement learning</concept_desc>
<concept_significance>300</concept_significance>
</concept>
<concept>
<concept_id>10002950.10003741.10003742.10003744</concept_id>
<concept_desc>Mathematics of computing~Algebraic topology</concept_desc>
<concept_significance>100</concept_significance>
</concept>
</ccs2012>
\end{CCSXML}

\ccsdesc[500]{Computing methodologies~Animation}
\ccsdesc[300]{Computing methodologies~Physical simulation}
\ccsdesc[300]{Theory of computation~Reinforcement learning}


%
% End generated code
%


\keywords{physics-based character animation, motion strategy, Bayesian optimization, control policy}


\maketitle

% \leavevmode
% \\
% \\
% \\
% \\
% \\
\section{Introduction}
\label{introduction}

AutoML is the process by which machine learning models are built automatically for a new dataset. Given a dataset, AutoML systems perform a search over valid data transformations and learners, along with hyper-parameter optimization for each learner~\cite{VolcanoML}. Choosing the transformations and learners over which to search is our focus.
A significant number of systems mine from prior runs of pipelines over a set of datasets to choose transformers and learners that are effective with different types of datasets (e.g. \cite{NEURIPS2018_b59a51a3}, \cite{10.14778/3415478.3415542}, \cite{autosklearn}). Thus, they build a database by actually running different pipelines with a diverse set of datasets to estimate the accuracy of potential pipelines. Hence, they can be used to effectively reduce the search space. A new dataset, based on a set of features (meta-features) is then matched to this database to find the most plausible candidates for both learner selection and hyper-parameter tuning. This process of choosing starting points in the search space is called meta-learning for the cold start problem.  

Other meta-learning approaches include mining existing data science code and their associated datasets to learn from human expertise. The AL~\cite{al} system mined existing Kaggle notebooks using dynamic analysis, i.e., actually running the scripts, and showed that such a system has promise.  However, this meta-learning approach does not scale because it is onerous to execute a large number of pipeline scripts on datasets, preprocessing datasets is never trivial, and older scripts cease to run at all as software evolves. It is not surprising that AL therefore performed dynamic analysis on just nine datasets.

Our system, {\sysname}, provides a scalable meta-learning approach to leverage human expertise, using static analysis to mine pipelines from large repositories of scripts. Static analysis has the advantage of scaling to thousands or millions of scripts \cite{graph4code} easily, but lacks the performance data gathered by dynamic analysis. The {\sysname} meta-learning approach guides the learning process by a scalable dataset similarity search, based on dataset embeddings, to find the most similar datasets and the semantics of ML pipelines applied on them.  Many existing systems, such as Auto-Sklearn \cite{autosklearn} and AL \cite{al}, compute a set of meta-features for each dataset. We developed a deep neural network model to generate embeddings at the granularity of a dataset, e.g., a table or CSV file, to capture similarity at the level of an entire dataset rather than relying on a set of meta-features.
 
Because we use static analysis to capture the semantics of the meta-learning process, we have no mechanism to choose the \textbf{best} pipeline from many seen pipelines, unlike the dynamic execution case where one can rely on runtime to choose the best performing pipeline.  Observing that pipelines are basically workflow graphs, we use graph generator neural models to succinctly capture the statically-observed pipelines for a single dataset. In {\sysname}, we formulate learner selection as a graph generation problem to predict optimized pipelines based on pipelines seen in actual notebooks.

%. This formulation enables {\sysname} for effective pruning of the AutoML search space to predict optimized pipelines based on pipelines seen in actual notebooks.}
%We note that increasingly, state-of-the-art performance in AutoML systems is being generated by more complex pipelines such as Directed Acyclic Graphs (DAGs) \cite{piper} rather than the linear pipelines used in earlier systems.  
 
{\sysname} does learner and transformation selection, and hence is a component of an AutoML systems. To evaluate this component, we integrated it into two existing AutoML systems, FLAML \cite{flaml} and Auto-Sklearn \cite{autosklearn}.  
% We evaluate each system with and without {\sysname}.  
We chose FLAML because it does not yet have any meta-learning component for the cold start problem and instead allows user selection of learners and transformers. The authors of FLAML explicitly pointed to the fact that FLAML might benefit from a meta-learning component and pointed to it as a possibility for future work. For FLAML, if mining historical pipelines provides an advantage, we should improve its performance. We also picked Auto-Sklearn as it does have a learner selection component based on meta-features, as described earlier~\cite{autosklearn2}. For Auto-Sklearn, we should at least match performance if our static mining of pipelines can match their extensive database. For context, we also compared {\sysname} with the recent VolcanoML~\cite{VolcanoML}, which provides an efficient decomposition and execution strategy for the AutoML search space. In contrast, {\sysname} prunes the search space using our meta-learning model to perform hyperparameter optimization only for the most promising candidates. 

The contributions of this paper are the following:
\begin{itemize}
    \item Section ~\ref{sec:mining} defines a scalable meta-learning approach based on representation learning of mined ML pipeline semantics and datasets for over 100 datasets and ~11K Python scripts.  
    \newline
    \item Sections~\ref{sec:kgpipGen} formulates AutoML pipeline generation as a graph generation problem. {\sysname} predicts efficiently an optimized ML pipeline for an unseen dataset based on our meta-learning model.  To the best of our knowledge, {\sysname} is the first approach to formulate  AutoML pipeline generation in such a way.
    \newline
    \item Section~\ref{sec:eval} presents a comprehensive evaluation using a large collection of 121 datasets from major AutoML benchmarks and Kaggle. Our experimental results show that {\sysname} outperforms all existing AutoML systems and achieves state-of-the-art results on the majority of these datasets. {\sysname} significantly improves the performance of both FLAML and Auto-Sklearn in classification and regression tasks. We also outperformed AL in 75 out of 77 datasets and VolcanoML in 75  out of 121 datasets, including 44 datasets used only by VolcanoML~\cite{VolcanoML}.  On average, {\sysname} achieves scores that are statistically better than the means of all other systems. 
\end{itemize}


%This approach does not need to apply cleaning or transformation methods to handle different variances among datasets. Moreover, we do not need to deal with complex analysis, such as dynamic code analysis. Thus, our approach proved to be scalable, as discussed in Sections~\ref{sec:mining}.
\section{RELATED WORK}

We build on prior work from several areas,
including character animation, diversity optimization, human pose modeling, and high-jump analysis from biomechanics and kinesiology.

\subsection{Character Animation}
Synthesizing natural human motion is a long-standing challenge in computer animation. We first briefly review kinematic methods, and then provide a more detailed review of physics-based methods. To the best of our knowledge, there are no previous attempts to synthesize athletic high jumps or obstacle jumps using either kinematic or physics-based approaches. Both tasks require precise coordination and exhibit multiple strategies.

\paragraph{Kinematic Methods} 
Data-driven kinematic methods have demonstrated their effectiveness for synthesizing high-quality human motions based on captured examples. Such kinematic models have evolved from graph structures~\cite{Kovar:2002:MotionGraphs,Safonova-Interpolated-Motion-Graphs}, to Gaussian Processes~\cite{levine2012continuous, ye2010synthesis-responsive}, and recently deep neural networks \cite{Holden:2017:PFNN, Zhang:2018:MANN, starke2019neural-state-machine, starke2020local-motion-phases, lee2018interactive-RNN, Ling20}. Non-parametric models that store all example frames have limited capability of generalizing to new motions due to their inherent nature of data interpolation \cite{Clavet16}. Compact parametric models learn an underlying low-dimensional motion manifold. Therefore they tend to generalize better as new motions not in the training dataset can be synthesized by sampling in the learned latent space~\cite{Holden16}. Completely novel motions and strategies, however, are still beyond their reach. Most fundamentally, kinematic models do not take into account physical realism, which is important for athletic motions. We thus cannot directly apply kinematic methods to our problem of discovering unseen strategies for highly dynamic motions. However, we do adopt a variational autoencoder (VAE) similar to the one in~\cite{Ling20} as a means to
improve the naturalness of our learned motion strategies.

\paragraph{Physics-based Methods}
Physics-based control and simulation methods generate motions with physical realism and environmental interactions. The key challenge is the design or learning of robust controllers. Conventional manually designed controllers have achieved significant success for locomotion, e.g.,~\cite{Yin07,wang2009optimizing,Coros10,wang2012optimizing,geijtenbeek2013flexible,lee2014locomotion,felis2016synthesis,jain2009optimization}. The seminal work from Hodgins \textit{et al.} demonstrated impressive controllers for athletic skills such as a handspring vault, a standing broad jump, a vertical leap, somersaults to different directions, and platform dives~\cite{Hodgins95,Wooten98}. Such handcrafted controllers are mostly designed with finite state machines (FSM) and heuristic feedback rules, which require deep human insight and domain knowledge, and tedious manual trial and error. Zhao and van de Panne~\shortcite{Zhao05} thus proposed an interface to ease such a design process, and demonstrated controllers for diving, skiing and snowboarding. Controls can also be designed using objectives and constraints adapted to each motion phase, e.g.,~\cite{jain2009optimization,deLasa2010},
or developed using a methodology that mimics human coaching~\cite{ha2014-human-coach}. In general, manually designed controllers remain hard to generalize to different strategies or tasks. 

With the wide availability of motion capture data, many research endeavors have been focused on tracking-based controllers, which are capable of reproducing high-quality motions by imitating motion examples. Controllers for a wide range of skills have been demonstrated through trajectory optimization \cite{sok2007simulating,da2008interactive,muico2009contact,lee2010data-driven-biped,ye2010optimal,lee2014locomotion}, sampling-based algorithms \cite{Liu:2010:Samcon,Liu:2015:Samcon2,Liu16}, and deep reinforcement learning ~\cite{Peng2017deepLoco,Peng:2018:DeepMimic,Peng:2018:SFV,ma2021spacetime,Liu18,Lee19}.
Tracking controllers have also been combined with kinematic motion generators to support interactive control of simulated characters~\cite{Bergamin:2019:DReCon,Park:2019:LPS,won2020scalable}. Even though tracking-based methods have demonstrated their effectiveness on achieving task-related goals~\cite{Peng:2018:DeepMimic}, the imitation objective inherently restricts them from generalizing to novel motion strategies fundamentally different from the reference. Most recently, style exploration has also been demonstrated within a physics-based DRL framework using spacetime bounds~\cite{ma2021spacetime}. However, these remain style variations rather than strategy variations. Moreover, high jumping motion capture examples are difficult to find. We obtained captures of three high jump strategies, which we use to compare our synthetic results to.

Our goal is to discover as many strategies as possible, so example-free methods are most suitable in our case. Various tracking-free methods have been proposed via trajectory optimization or deep reinforcement learning. Heess \textit{et al.} \shortcite{Heess:2017:Emergence} demonstrate a rich set of locomotion behaviors emerging from just complex environment interactions. However, the resulting motions show limited realism in the absence of effective motion quality regularization. Better motion quality is achievable with sophisticated reward functions and domain knowledge, such as sagittal symmetry, which do not directly generalize beyond locomotion \cite{Yu:2018:LSLL,coumans2016pybullet,Xie2020allsteps,mordatch2013-cio-locomotion,mordatch2015interactive}. Synthesizing diverse physics-based skills without example motions generally requires optimization with detailed cost functions that are engineered specifically for each skill~\cite{AlBorno13}, and often only works for simplified physical models~\cite{Mordatch12}.


\subsection{Diversity Optimization}
Diversity Optimization is a problem of great interest in artificial intelligence \cite{hebrard2005finding-diverse,ursem2002diversity,srivastava2007diverse-plan,coman2011generating-diverse,pugh2016quality-diversity,lehman2011novelty-search}. It is formulated as searching for a set of configurations such that the corresponding outcomes have a large diversity while satisfying a given objective. Diversity optimization has also been utilized in computer graphics applications~\cite{merrell2011interactive,Agrawal:2013:Diverse}. For example, a variety of 2D and simple 3D skills have been achieved through jointly optimizing task objectives and a diversity metric within 
a trajectory optimization framework~\cite{Agrawal:2013:Diverse}. Such methods are computationally prohibitive for our case as learning the athletic tasks involve expensive DRL training through non-differentiable simulations, e.g., a single strategy takes six hours to learn even on a high-end desktop. 
We propose a diversity optimization algorithm based on the successful Bayesian Optimization (BO) philosophy for sample efficient black-box function optimization.

In Bayesian Optimization, objective functions are optimized purely through function evaluations as no derivative information is available. A Bayesian statistical \textit{surrogate model}, usually a Gaussian Process (GP) \cite{rasmussen2003gaussian-process}, is maintained to estimate the value of the objective function along with the uncertainty of the estimation. An \textit{acquisition function} is then repeatedly maximized for fast decisions on where to sample next for the actual expensive function evaluation. The next sample needs to be promising in terms of maximizing the objective function predicted by the surrogate model, and also informative in terms of reducing the uncertainty in less explored regions of the surrogate model~\cite{jones1998-bo-expected-improvement, frazier2009-bo-knowledge-gradient, GP-bandit}. BO has been widely adopted in machine learning for parameter and hyperparameter optimizations \cite{snoek2015scalable,klein2017fast,kandasamy2018neural,kandasamy2020tuning,korovina2020chembo,snoek2012practical}. Recently BO has also seen applications in computer graphics~\cite{koyama2017sequential,koyama2020sequential}, such as parameter tuning for fluid animation systems \cite{brochu2007preference}. 

We propose a novel acquisition function to encourage discovery of diverse motion strategies. We also decouple the exploration from the maximization for more robust and efficient strategy discovery. We name this algorithm Bayesian Diversity Search (BDS). The BDS algorithm searches for diverse strategies by exploring a low-dimensional initial state space defined at the take-off moment. Initial states exploration has been applied to find appropriate initial conditions for desired landing controllers \cite{ha2012falling}. In the context of DRL learning, initial states are usually treated as hyperparameters rather than being explored.

Recently a variety of DRL-based learning methods have been proposed to discover diverse control policies in machine learning, e.g.,  \cite{Eysenbach19,zhang2019novel-policies,Sun2020novel-policies,achiam2018variational,sharma2019dynamics,haarnoja2018soft,conti2018improving,houthooft2016vime,hester2017intrinsically,schmidhuber1991curious}. These methods mainly encourage exploration of unseen states or actions by jointly optimizing the task and novelty objectives~\cite{zhang2019novel-policies}, or optimizing intrinsic rewards such as heuristically defined curiosity terms~\cite{Eysenbach19,sharma2019dynamics}. We adopt a similar idea for novelty seeking in Stage~2 of our framework after BDS, but with a novelty metric and reward structure more suitable for our goal. Coupled with the Stage~1 BDS, we are able to learn a rich set of strategies for challenging tasks such as athletic high jumping.

\subsection{Natural Pose Space}

In biomechanics and neuroscience, it is well known that muscle synergies, or muscle co-activations, serve as motor primitives for the central nervous system to simplify movement control of the underlying complex neuromusculoskeletal systems~\cite{Overduin15,Zhao19}. In character animation, human-like character models are much simplified, but are still parameterized by 30+ DoFs. Yet the natural human pose manifold learned from motion capture databases is of much lower dimension \cite{Holden16}. The movement of joints are highly correlated as typically they are strategically coordinated and co-activated. Such correlations have been modelled through traditional dimensionality reduction techniques such as PCA \cite{chai2005performance-lowd}, or more recently, Variational AutoEncoders (VAE) \cite{habibie2017,Ling20}.

We rely on a VAE learned from mocap databases to produce natural target poses for our DRL-based policy network. Searching behaviors in low dimensional spaces has been employed in physics-based character animation to both accelerate the nonlinear optimization and improve the motion quality \cite{Safonova:2004:OptPCA}. Throwing motions based on muscle synergies extracted from human experiments have been synthesized on a musculoskeletal model \cite{cruz2017synergy}. Recent DRL methods either directly imitate mocap examples \cite{Peng:2018:DeepMimic, won2020scalable}, which makes strategy discovery hard if possible; or adopt a \textit{de novo} approach with no example at all \cite{Heess2015learning}, which often results in extremely unnatural motions for human like characters. Close in spirit to our work is \cite{ranganath2019lowd-joint-coactivation}, where a low-dimensional PCA space learned from a single mocap trajectory is used as the action space of DeepMimic for tracking-based control. We aim to discover new strategies without tracking, and we use a large set of generic motions to deduce a task-and-strategy-independent natural pose space. We also add action offsets to the P-VAE output poses so that large joint activation can be achieved for powerful take-offs.

{Reduced or latent parameter spaces based on statistical analysis of poses have been used for grasping control \cite{Ciocarlie10,Andrews13,Osa18}. A Trajectory Generator (TG) can provide a compact parameterization that can enable learning of reactive policies for complex behaviors~\cite{Iscen18}.
Motion primitives can also be learned from mocap and then be composed to learn new behaviors~\cite{Peng19}.}


\subsection{History and Science of High Jump}

The high jump is one of the most technically complex, strategically nuanced, and physiologically demanding sports among all track and field events \cite{Donnelly14}. Over the past 100 years, high jump has evolved dramatically in the Olympics. Here we summarize the well-known variations \cite{SevenStyles,Donnelly14}, and we refer readers to our supplemental video for more visual illustrations.
\begin{itemize}
    \item {The Hurdle}: the jumper runs straight-on to the bar, raises one leg up to the bar, and quickly raises the other one over the bar once the first has cleared. The body clears the bar upright.
    \item {Scissor Kick}: the jumper approaches the bar diagonally, throws first the inside leg and then the other over the bar in a scissoring motion. The body clears the bar upright.
    \item {Eastern Cutoff}: the jumper takes off like the scissor kick, but extends his back and flattens out over the bar.  
    \item {Western Roll}:the jumper also approaches the bar diagonally, but the inner leg is used for the take-off, while the outer leg is thrust up to lead the body sideways over the bar.
    \item {The Straddle}: similar to Western Roll, but the jumper clears the bar face-down.
    \item {Fosbury Flop}: The jumper approaches the bar on a curved path and leans away from the bar at the take-off point to convert horizontal velocity into vertical velocity and angular momentum. In flight, the jumper progressively arches their shoulders, back, and legs in a rolling motion, and lands on their neck and back. The jumper's Center of Mass (CoM) can pass under the bar while the body arches and slide above the bar. It has been the favored high jump technique in Olympic competitions since used by Dick Fosbury in the 1968 Summer Olympics. It was concurrently developed by Debbie Brill.
\end{itemize}

In biomechanics, kinesiology, and physical education, high jumps have been analyzed to a limited extent. We adopt the force limits reported in \cite{Okuyama03} in our simulations. Dapena simulated a higher jump by making small changes to a recorded jump \cite{Dapena02}. Mathematical models of the Center of Mass (CoM) movement have been developed to offer recommendations to increase the effectiveness of high jumps~\cite{Adashevskiy13}.

\section{SYSTEM OVERVIEW}
\begin{figure}
\centering

\def\picScale{0.08}    % define variable for scaling all pictures evenly
\def\colWidth{0.5\linewidth}

\begin{tikzpicture}
\matrix [row sep=0.25cm, column sep=0cm, style={align=center}] (my matrix) at (0,0) %(2,1)
{
\node[style={anchor=center}] (FREEhand) {\includegraphics[width=0.85\linewidth]{figures/FREEhand.pdf}}; %\fill[blue] (0,0) circle (2pt);
\\
\node[style={anchor=center}] (rigid_v_soft) {\includegraphics[width=0.75\linewidth]{figures/FREE_vs_rigid-v8.pdf}}; %\fill[blue] (0,0) circle (2pt);
\\
};
\node[above] (FREEhand) at ($ (FREEhand.south west)  !0.05! (FREEhand.south east) + (0, 0.1)$) {(a)};
\node[below] (a) at ($ (rigid_v_soft.south west) !0.20! (rigid_v_soft.south east) $) {(b)};
\node[below] (b) at ($ (rigid_v_soft.south west) !0.75! (rigid_v_soft.south east) $) {(c)};
\end{tikzpicture}


% \begin{tikzpicture} %[every node/.style={draw=black}]
% % \draw[help lines] (0,0) grid (4,2);
% \matrix [row sep=0cm, column sep=0cm, style={align=center}] (my matrix) at (0,0) %(2,1)
% {
% \node[style={anchor=center}] {\includegraphics[width=\colWidth]{figures/photos/labFREEs3.jpg}}; %\fill[blue] (0,0) circle (2pt)
% &
% \node[style={anchor=center}] {\includegraphics[width=\colWidth, height=160pt]{figures/stewartRender.png}}; %\fill[blue] (0,0) circle (2pt);
% \\
% };

% %\node[style={anchor=center}] at (0,-5) (FREEstate) {\includegraphics[width=0.7\linewidth]{figures/FREEstate_noLabels2.pdf}};

% \end{tikzpicture}

\caption{\revcomment{2.3}{(a) A fiber-reinforced elastomerc enclosure (FREE) is a soft fluid-driven actuator composed of an elastomer tube with fibers wound around it to impose specific deformations under an increase in volume, such as extension and torsion. (b) A linear actuator and motor combined in \emph{series} has the ability to generate 2 dimensional forces at the end effector (shown in red), but is constrained to motions only in the directions of these forces. (b) Three FREEs combined in \emph{parallel} can generate the same 2 dimensional forces at the end effector (shown in red), without imposing kinematic constraints that prohibit motion in other directions (shown in blue).}}

% \caption{A fiber-reinforced elastomeric enclosure (FREE) (top) is a soft fluid-driven actuator composed of an elastomer tube with fibers wound around it to impose deformation in specific directions upon pressurization, such as extension and torsion. \revcomment{2.3}{In this paper we explore the potential of combining multiple FREEs in parallel to generate fully controllable multi-dimensional spacial forces}, such as in a parallel arrangement around a flexible spine element (bottom-left), or a Stewart Platform arrangement (bottom-right).}

\label{fig:overview}
\end{figure}


We now give an overview of our learning framework as illustrated in Figure~\ref{fig:overview}. Our framework splits athletic jumps into two phases: a run-up phase and a jump phase. The {\em take-off state} marks the transition between these two phases, and consists of a time instant midway through the last support phase before becoming airborne. The take-off state is key to our exploration strategy, as it is a strong determinant of the resulting jump strategy. We characterize the take-off state by a feature vector that captures key aspects of the state, such as the net angular velocity and body orientation. This defines a low-dimensional take-off feature space that we can sample in order to explore and evaluate a variety of motion strategies. While random sampling of take-off state features is straightforward, it is computationally impractical as evaluating one sample involves an expensive DRL learning process that takes hours even on modern machines. Therefore, we introduce a sample-efficient Bayesian Diversity Search (BDS) algorithm as a key part of our Stage~1 optimization process.

Given a specific sampled take-off state, we then need to produce an optimized run-up controller and a jump controller that result in the best possible corresponding jumps. This process has several steps. We first train a {\em }run-up controller, using deep reinforcement learning, that imitates a single generic run-up motion capture clip while also targeting the desired take-off state. For simplicity, the run-up controller and its training are not shown in Figure~\ref{fig:overview}. These are discussed in Section~\ref{sec:Experiments-Runup}. The main challenge lies with the synthesis of the actual jump controller which governs the remainder of the motion, and for which we wish to discover strategies without any recourse to known solutions.

The jump controller begins from the take-off state and needs to control the body during take-off, over the bar, and to prepare for landing. This poses a challenging learning problem because of the demanding nature of the task, the sparse fail/success rewards, and the difficulty of also achieving natural human-like movement. We apply two key insights to make this task learnable using deep reinforcement learning. First, we employ an action space defined by a subspace of natural human poses as modeled with a Pose Variational Autoencoder (P-VAE). Given an action parameterized as a target body pose, individual joint torques are then realized using PD-controllers. We additionally allow for regularized {\em offset} PD-targets that are added to the P-VAE targets to enable strong takeoff forces. Second, we employ a curriculum that progressively increases the task difficulty, i.e., the height of the bar, based on current performance.

A diverse set of strategies can already emerge after the Stage 1 BDS optimization. To achieve further strategy variations, we reuse the take-off states of the existing discovered strategies for another stage of optimization. The diversity is explicitly incentivized during this Stage 2 optimization via a novelty reward, which is focused specifically on features of the body pose at the peak height of the jump. As shown in Figure~\ref{fig:overview}, Stage~2 makes use of the same overall DRL learning procedure as in Stage~1, albeit with a slightly different reward structure.



\section{LEARNING NATURAL STRATEGIES}
\label{sec:discovery}
Given a character model, an environment, and a task objective, we aim to learn feasible natural-looking motion strategies using deep reinforcement learning. We first describe our DRL formulation in Section~\ref{sec:methods-DRL-formulation}. To improve the learned motion quality, we propose a Pose Variational Autoencoder (P-VAE) to constrain the policy actions in Section~\ref{sec:methods-PVAE}.

\subsection{DRL Formulation}
\label{sec:methods-DRL-formulation}
Our strategy learning task is formulated as a standard reinforcement learning problem, where the character interacts with the environment to learn a control policy which maximizes a long-term reward. The control policy $\pi_\theta(a|s)$ parameterized by $\theta$ models the conditional distribution over action $a \in \mathcal{A}$ given the character state $s \in \mathcal{S}$. At each time step $t$, the character interacts with the environment with action $a_t$ sampled from $\pi(a|s)$ based on the current state $s_t$. The environment then responds with a new state $s_{t+1}$ according to the transition dynamics $p(s_{t+1}|s_t,a_t)$, along with a reward signal $r_t$. The goal of reinforcement learning is to learn the optimal policy parameters $\theta^*$ which maximizes the expected return defined as
\begin{equation}
    J(\theta) = \mathbb{E}_{\tau\sim p_\theta(\tau)}\left[
        \sum_{t=0}^T{\gamma^t r_t}
    \right],
\end{equation}
where $T$ is the episode length, $\gamma \leq 1$ is a discount factor, and $p_\theta(\tau)$ is the probability of observing trajectory $\tau = \{s_0, a_0, s_1, ..., a_{T-1}, s_T\}$ given the current policy $\pi_\theta(a|s)$.

\paragraph{States}
The state $s$ describes the character configuration. We use a similar set of pose and velocity features as those proposed in DeepMimic \cite{Peng:2018:DeepMimic}, including relative positions of each link with respect to the root, their rotations parameterized in quaternions, along with their linear and angular velocities. Different from DeepMimic, our features are computed directly in the global frame without direction-invariant transformations for the studied jump tasks. The justification is that input features should distinguish states with different relative transformations between the character and the environment obstacle such as the crossbar. In principle, we could also use direction-invariant features as in DeepMimic, and include the relative transformation to the obstacle into the feature set. However, as proved in \cite{Ma19}, there are no direction-invariant features that are always singularity free. Direction-invariant features change wildly whenever the character's facing direction approaches the chosen motion direction, which is usually the global up-direction or the $Y$-axis. For high jump techniques such as the Fosbury flop, singularities are frequently encountered as the athlete clears the bar facing upward. Therefore, we opt to use global features for simplicity and robustness. Another difference from DeepMimic is that time-dependent phase variables are not included in our feature set. Actions are chosen purely based on the dynamic state of the character.

\paragraph{Initial States}
\label{sec:methods-initial-states}
The initial state $s_0$ is the state in which an agent begins each episode in DRL training. We explore a chosen low-dimensional feature space ($3\sim4D$) of the take-off states for learning diverse jumping strategies. As shown by previous work \cite{ma2021spacetime}, the take-off moment is a critical point of jumping motions, where the volume of the feasible region of the dynamic skill is the smallest. In another word, bad initial states will fail fast, which in a way help our exploration framework to find good ones quicker. Alternatively, we could place the agent in a fixed initial pose to start with, such as a static pose before the run-up. This is problematic for several reasons. First, different jumping strategies need different length for the run-up. The planar position and facing direction of the root is still a three dimensional space to be explored. Second, the run-up strategies and the jumping strategies do not correlate in a one-to-one fashion. Visually, the run-up strategies do not look as diverse as the jumping strategies. Lastly, starting the jumps from a static pose lengthens the learning horizon, and makes our learning framework based on DRL training even more costly. Therefore we choose to focus on just the jumping part of the jumps in this work, and leave the run-up control learning to DeepMimic, which is one of the state-of-the-art imitation-based DRL learning methods. More details are given in Section~\ref{sec:Experiments-Runup}.    

\paragraph{Actions}
The action $a$ is a target pose described by internal joint rotations. We parameterize 1D revolute joint rotations by scalar angles, and 3D spherical joint rotations by exponential maps \cite{Grassia:1998:ExpMap}. Given a target pose and the current character state, joint torques are computed through the Stable Proportional Derivative (SPD) controllers \cite{Tan:2011:SPD}. Our control frequency $f_{\text{control}}$ ranges from 10 $Hz$ to 30 $Hz$ depending on both the task and the curriculum. For challenging tasks like high jumps, it helps to quickly improve initial policies through stochastic evaluations at early training stages. A low-frequency policy enables faster learning by reducing the needed control steps, or in another word, the dimensionality and complexity of the actions $(a_0,...,a_T)$. This is in spirit similar to the 10 $Hz$ control fragments used in SAMCON-type controllers \cite{Liu16}. Successful low-frequency policies can then be gradually transferred to high-frequency ones according to a curriculum to achieve finer controls and thus smoother motions. We discuss the choice of control frequency in more detail in Section~\ref{sec:Experiments-Curriculum}.
\paragraph{Reward}
We use a reward function consisting of the product of two terms for all our strategy discovery tasks as follows:
\begin{equation}
\label{eq:stage1-reward}
    r = r_{\text{task}} \cdot r_{\text{naturalness}} 
\end{equation}    
where $r_{\text{task}}$ is the task objective and $r_{\text{naturalness}}$ is a naturalness reward term computed from the P-VAE to be described in Section \ref{sec:methods-PVAE}. For diverse strategy discovery, a simple $r_{\text{task}}$ which precisely captures the task objective is preferred. For example in high jumping, the agent receives a sparse reward signal at the end of the jump after it successfully clears the bar. In principle, we could transform the sparse reward into a dense reward to reduce the learning difficulty, such as to reward CoM positions higher than a parabolic trajectory estimated from the bar height. However in practice, such dense guidance reward can mislead the training to a bad local optimum, where the character learns to jump high in place rather than clears the bar in a coordinated fashion. Moreover, the CoM height and the bar height may not correlate in a simple way. For example, the CoM passes underneath the crossbar in Fosbury flops. As a result, a shaped dense reward function could harm the diversity of the learned strategies. We will discuss reward function settings for each task in more details in Section~\ref{sec:Experiments-Reward}.

\paragraph{Policy Representation}
We use a fully-connected neural network parameterized by weights $\theta$ to represent the control policy $\pi_\theta(a|s)$. Similar to the settings in \cite{Peng:2018:DeepMimic}, the network has two hidden layers with $1024$ and $512$ units respectively. ReLU activations are applied for all hidden units. Our policy maps a given state $s$ to a Gaussian distribution over actions $a=\mathcal{N}(\mu(s), \Sigma)$. The mean $\mu(s)$ is determined by the network output. The covariance matrix $\Sigma=\sigma I$ is diagonal, where $I$ is the identity matrix and $\sigma$ is a scalar variable measuring the action noise. We apply an annealing strategy to linearly decrease $\sigma$ from $0.5$ to $0.1$ in the first $1.0\times 10^7$ simulation steps, to encourage more exploration in early training and more exploitation in late training.

\paragraph{Training}
We train our policies with the Proximal Policy Optimization (PPO) method \cite{Schulman:2017:PPO}. PPO involves training both a policy network and a value function network. The value network architecture is similar to the policy network, except that there is only one single linear unit in the output layer. We train the value network with TD($\lambda$) multi-step returns. We estimate the advantage of the PPO policy gradient by the Generalized Advantage Estimator GAE($\lambda$) \cite{Generalized-Advantage-Estimation}.

\subsection{Pose Variational Autoencoder}\label{sec:methods-PVAE}
The dimension of natural human poses is usually much lower than the true degrees of freedom of the character model. We propose a generative model to produce natural PD target poses at each control step. More specifically, we train a Pose Variational Autoencoder (P-VAE) from captured natural human poses, and then sample its latent space to produce desired PD target poses for control. Here a pose only encodes internal joint rotations without the global root transformations. We use publicly available human motion capture databases to train our P-VAE. Note that none of these databases consist of high jumps or obstacle jumps specifically, but they already provide enough poses for us to learn the natural human pose manifold successfully.

\paragraph{P-VAE Architecture and Training}
Our P-VAE adopts the standard Beta Variational Autoencoder ($\beta$-VAE) architecture \cite{beta-VAE}. The encoder maps an input feature $x$ to a low-dimensional latent space, parameterized by a Gaussian distribution with a mean $\mu_x$ and a diagonal covariance $\Sigma_x$. The decoder maps a latent vector sampled from the Gaussian distribution back to the original feature space as $x'$. The training objective is to minimize the following loss function:
\begin{equation}
    \mathcal{L} = \mathcal{L}_\text{MSE}(x,x') + \beta \cdot \text{KL}(\mathcal{N}(\mu_x, \Sigma_x), \mathcal{N}(0, I)),
\end{equation}
where the first term is the MSE (Mean Squared Error) reconstruction loss, and the second term shapes the latent variable distribution to a standard Gaussian by measuring their Kulback-Leibler divergence. We set $\beta = 1.0\times10^{-5}$ in our experiments, so that the two terms in the loss function are within the same order of magnitude numerically.

We train the P-VAE on a dataset consisting of roughly $20,000$ poses obtained from the CMU and SFU motion capture databases. We include a large variety of motion skills, including walking, running, jumping, breakdancing, cartwheels, flips, kicks, martial arts, etc. The input features consist of all link and joint positions relative to the root in the local root frames, and all joint rotations with respect to their parents. We parameterize joint rotations by a $6$D representation for better continuity, as described in \cite{zhou2019continuity, Ling20}.

We model both the encoder and the decoder as fully connected neural networks with two hidden layers, each having 256 units with $tanh$ activation. We perform PCA (Principal Component Analysis) on the training data and choose $d_{\text{latent}} = 13$ to cover $85\%$ of the training data variance, where $d_{\text{latent}}$ is the dimension of the latent variable. We use the Adam optimizer to update network weights \cite{kingma2014adam}, with the learning rate set to $1.0\times10^{-4}$. Using a mini-batch size of $128$, the training takes $80$ epochs within $2$ minutes on an NVIDIA GeForce GTX 1080 GPU and an Intel i7-8700k CPU. We use this single pre-trained P-VAE for all our strategy discovery tasks to be described.

\paragraph{Composite PD Targets}
PD controllers provide actuation based on positional errors. So in order to reach the desired pose, the actual target pose needs to be offset by a certain amount. Such offsets are usually small to just counter-act the gravity for free limbs. However, for joints that interact with the environment, such as the lower body joints for weight support and ground takeoff, large offsets are needed to generate powerful ground reaction forces to propel the body forward or into the air. Such complementary offsets combined with the default P-VAE targets help realize natural poses during physics-based simulations. Our action space $\mathcal{A}$ is therefore $d_{\text{latent}} + d_{\text{offset}}$ dimensional, where $d_{\text{latent}}$ is the dimension of the P-VAE latent space, and $d_{\text{offset}}$ is the dimension of the DoFs that we wish to apply offsets for. We simply apply offsets to all internal joints in this work. Given $a=(a_{\text{latent}},a_{\text{offset}}) \in \mathcal{A}$ sampled from the policy $\pi_\theta(a|s)$, where $a_{\text{latent}}$ and $a_{\text{offset}}$ correspond to the latent and offset part of $a$ respectively, the final PD target is computed by $D_{\text{pose}}(a_{\text{latent}}) + a_{\text{offset}}$. Here $D_{\text{pose}}(\cdot)$ is a function that decodes the latent vector $a_{\text{latent}}$ to full-body joint rotations. We minimize the usage of rotation offsets by a penalty term as follows:
\begin{equation}
\label{eq:r-pvae}
    r_{\text{naturalness}} = 1 - \clip\left(\left(\frac{||a_{\text{offset}}||_1}{c_{\text{offset}}}\right)^2, 0, 1\right) ,
\end{equation}
where $c_{\text{offset}}$ is the maximum offset allowed. For tasks with only a sparse reward signal at the end, $||a_{\text{offset}}||_1$ in Equation~\ref{eq:r-pvae} is replaced by the average offset norm $\frac{1}{T}\sum_{t=0}^T{||a^{(t)}_{\text{offset}}||_1}$ across the entire episode. We use $L1$-norm rather than the commonly adopted $L2$-norm to encourage sparse solutions with fewer non-zero components \cite{tibshirani1996regression, chen2001atomic}, as our goal is to only apply offsets to essential joints to complete the task while staying close to the natural pose manifold prescribed by the P-VAE.
\section{LEARNING DIVERSE STRATEGIES}
\label{sec:diverse}
Given a virtual environment and a task objective, we would like to discover as many strategies as possible to complete the task at hand. Without human insights and demonstrations, this is a challenging task. To this end, we propose a two-stage framework to enable stochastic DRL to discover solution modes such as the Fosbury flop.

The first stage focuses on strategy discovery by exploring the space of initial states. For example in high jump, the Fosbury flop technique and the straddle technique require completely different initial states at take-off, in terms of the approaching angle with respect to the bar, the take-off velocities, and the choice of inner or outer leg as the take-off leg. A fixed initial state may lead to success of one particular strategy, but can miss other drastically different ones. We systematically explore the initial state space through a novel sample-efficient Bayesian Diversity Search (BDS) algorithm to be described in Section~\ref{sec:methods-bayesian-diversity-search}.

The output of Stage 1 is a set of diverse motion strategies and their corresponding initial states. Taken such a successful initial state as input, we then apply another pass of DRL learning to further explore more motion variations permitted by the same initial state. The intuition is to explore different local optima while maximizing the novelty of the current policy, compared to previously found ones. We describe our detailed settings for the Stage 2 novel policy seeking algorithm in Section~\ref{sec:methods-phase2}.

\subsection{Stage 1: Initial States Exploration with Bayesian Diversity Search}\label{sec:methods-bayesian-diversity-search}

In Stage 1, we perform diverse strategy discovery by exploring initial state variations, such as pose and velocity variations, at the take-off moment. We first extract a feature vector $f$ from a motion trajectory to characterize and differentiate between different strategies. A straightforward way is to compute the Euclidean distance between time-aligned motion trajectories, but we hand pick a low-dimensional visually-salient feature set as detailed in Section~\ref{sec:Experiments-Strategy-Features}. We also define a low-dimensional exploration space $\mathcal{X}$ for initial states, as exploring the full state space is computationally prohibitive. Our goal is to search for a set of representatives $X_n = \{x_1, x_2, ..., x_n | x_i \in \mathcal{X}\}$, such that the corresponding feature set $F_n= \{f_1, f_2, ..., f_n | f_i \in \mathcal{F}\}$ has a large diversity. Note that as DRL training and physics-based simulation are involved in producing the motion trajectories from an initial state, the computation of $f_i=g(x_i)$ is a stochastic and expensive black-box function. We therefore design a sample-efficient Bayesian Optimization (BO) algorithm to optimize for motion diversity in a guided fashion.

Our BDS (Bayesian Diversity Search) algorithm iteratively selects the next sample to evaluate from $\mathcal{X}$, given the current set of observations $X_t = \{x_1, x_2, ..., x_t\}$ and $F_t = \{f_1, f_2, ..., f_t\}$. More specifically, the next point $x_{t+1}$ is selected based on an acquisition function $a(x_{t+1})$ to maximize the diversity in $F_{t+1}=F_t \cup \{f_{t+1}\}$. We choose to maximize the minimum distance between $f_{t+1}$ and all $f_i \in F_t$:
\begin{equation}\label{eq:diversity-objective}
    a(x_{t+1}) = \min_{f_i \in F_t}{||f_{t+1} - f_i||} .
\end{equation}
Since evaluating $f_{t+1}$ through $g(\cdot)$ is expensive, we employ a surrogate model to quickly estimate $f_{t+1}$, so that the most promising sample to evaluate next can be efficiently found through Equation~\ref{eq:diversity-objective}.

We maintain the surrogate statistical model of $g(\cdot)$ using a Gaussian Process (GP) \cite{rasmussen2003gaussian-process}, similar to standard BO methods. A GP contains a prior mean $m(x)$ encoding the prior belief of the function value, and a kernel function $k(x,x')$ measuring the correlation between $g(x)$ and $g(x')$. More details of our specific $m(x)$ and $k(x,x')$ are given in Section~\ref{sec:Experiments-GP-Priors-Kernels}. Hereafter we assume a one-dimensional feature space $\mathcal{F}$. Generalization to a multi-dimensional feature space is straightforward as multi-output Gaussian Process implementations are readily available, such as \cite{GPflow2020multioutput-gp}. Given $m(\cdot)$, $k(\cdot, \cdot)$, and current observations $\{X_t, F_t\}$, posterior estimation of $g(x)$ for an arbitrary $x$ is given by a Gaussian distribution with mean $\mu_{t}$ and variance $\sigma^2_{t}$ computed in closed forms:
\begin{equation}\label{eq:gp-mu-sigma}
    \begin{gathered}
        \mu_{t}(x) = k(X_t, x)^T (K + \eta^{2} I)^{-1} \left(F_t - m(x)\right) + m(x), \\
        \sigma_{t}^{2}(x) = k(x, x) + \eta^{2} - k(X_t, x)^T(K + \eta^{2}I)^{-1} k(X_t,x) ,
    \end{gathered}
\end{equation}
where $X_t \in \mathbb{R}^{t \times \text{dim}(\mathcal{X})}$, $F_t \in \mathbb{R}^{t}$, $K \in \mathbb{R}^{t \times t}, K_{i,j} = k(x_{i}, x_{j})$, and $k(X_t, x) = [k(x,x_{1}),k(x,x_{2}),...k(x,x_{t})]^T$. $I$ is the identity matrix, and $\eta$ is the standard deviation of the observation noise. Equation~\ref{eq:diversity-objective} can then be approximated by
\begin{equation}\label{eq:diversity-approximation}
    \hat{a}(x_{t+1}) = \mathbb{E}_{\hat{f}_{t+1} \sim \mathcal{N}(\mu_t(x_{t+1}), \sigma^2_t(x_{t+1}))}\left[\min_{f_i \in F_t}{||\hat{f}_{t+1} - f_i||}\right] .
\end{equation}
Equation~\ref{eq:diversity-approximation} can be computed analytically for one-dimensional features, but gets more and more complicated to compute analytically as the feature dimension grows, or when the feature space is non-Euclidean as in our case with rotational features. Therefore, we compute Equation~\ref{eq:diversity-approximation} numerically with Monte-Carlo integration for simplicity.

The surrogate model is just an approximation to the true function, and has large uncertainty where observations are lacking. Rather than only maximizing the function value when picking the next sample, BO methods usually also take into consideration the estimated uncertainty to avoid being overly greedy. For example, GP-UCB (Gaussian Process Upper Confidence Bound), one of the most popular BO algorithms, adds a variance term into its acquisition function. Similarly, we could adopt a composite acquisition function as follows: 
\begin{equation}\label{eq:diversity-ucb}
    a'(x_{t+1}) = \hat{a}(x_{t+1}) + \beta\sigma_t(x_{t+1}),
\end{equation}
where $\sigma_t(x_{t+1})$ is the heuristic term favoring candidates with large uncertainty, and $\beta$ is a hyperparameter trading off exploration and exploitation (diversity optimization in our case). Theoretically well justified choice of $\beta$ exists for GP-UCB, which guarantees optimization convergence with high probability \cite{GP-bandit}. However in our context, no such guarantees hold as we are not optimizing $f$ but rather the diversity of $f$, the tuning of the hyperparameter $\beta$ is thus not trivial, especially when the strategy evaluation function $g(\cdot)$ is extremely costly. To mitigate this problem, we decouple the two terms and alternate between exploration and exploitation following a similar idea proposed in \cite{alternative-explore-exploit}. During exploration, our acquisition function becomes:
\begin{equation}\label{eq:explore-acquisition}
    a_\text{exp}(x_{t+1}) = \sigma_t(x_{t+1}).
\end{equation}
The sample with the largest posterior standard deviation is chosen as $x_{t+1}$ to be evaluated next:
\begin{equation}\label{eq:explore-formula-2}
    x_{t+1} = \mathop{\arg\max}\limits_{x}\sigma_t(x).
\end{equation}
Under the condition that $g(\cdot)$ is a sample from GP function distribution $\mathcal{GP}(m(\cdot), k(\cdot,\cdot))$, Equation~\ref{eq:explore-formula-2} can be shown to maximize the Information Gain $I$ on function $g(\cdot)$:
\begin{equation}\label{eq:explore-formula-1}
    x_{t+1} = \mathop{\arg\max}\limits_{x}I\left(X_t \cup \{x\}, F_t \cup \{g(x)\} ; g\right),
\end{equation}
where $I(A;B)=H(A)-H(A|B)$, and $H(\cdot)=\mathbb{E}\left[-\log p(\cdot)\right]$ is the Shannon entropy \cite{information-theory}. 

We summarize our BDS algorithm in Algorithm~\ref{algorithm::BDS}. The alternation of exploration and diversity optimization involves two extra hyperparameters $N_\text{exp}$ and $N_\text{opt}$, corresponding to the number of samples allocated for exploration and diversity optimization in each round. Compared to $\beta$ in Equation~\ref{eq:diversity-ucb}, $N_\text{exp}$ and $N_\text{opt}$ are much more intuitive to tune. We also found that empirically the algorithm performance is insensitive to the specific values of $N_\text{exp}$ and $N_\text{opt}$. The exploitation stage directly maximizes the diversity of motion strategies. We optimize $\hat{a}(\cdot)$ with a sampling-based method DIRECT (Dividing Rectangle) \cite{Jones2001-DIRECT}, since derivative information may not be accurate in the presence of function noise due to the Monte-Carlo integration. This optimization does not have to be perfectly accurate, since the surrogate model is an approximation in the first place. The exploration stage facilitates the discovery of diverse strategies by avoiding repeated queries on well-sampled regions. We optimize $a_\text{exp}(\cdot)$ using a simple gradient-based method L-BFGS \cite{LBFGS}.

\begin{algorithm}[t]
\caption{Bayesian Diversity Search}\label{algorithm::BDS}
\KwIn{Strategy evaluation function $g(\cdot)$, exploration count $N_\text{exp}$ and diversity optimization count $N_\text{opt}$, total sample count $n$.}
\KwOut{Initial states $X_n = \{x_1, x_2, ..., x_n\}$ for diverse strategies.}
$t=0$; $X_0 \leftarrow \varnothing$; $F_0 \leftarrow \varnothing$\;
Initialize GP surrogate model with random samples\;
\While{$t < n$}
{
\eIf{$t \% (N_\text{exp} + N_\text{opt}) < N_\text{exp}$}
{
$x_{t+1} \leftarrow \mathop{\arg\max}a_\text{exp}(\cdot)$ by L-BFGS; 
\tcp*[f]{Equation~\ref{eq:explore-acquisition}}
}
{
$x_{t+1} \leftarrow \mathop{\arg\max}\hat{a}(\cdot)$ by DIRECT; \tcp*[f]{Equation~\ref{eq:diversity-approximation}}
}
$f_{t+1} \leftarrow g(x_{t+1})$\; 
$X_{t+1} \leftarrow X_t \cup \{x_{t+1}\}$; $F_{t+1} \leftarrow F_t \cup \{f_{t+1}\}$\;
Update GP surrogate model with $X_{t+1}$, $F_{t+1}$;
\hspace{8pt}\tcp{Equation~\ref{eq:gp-mu-sigma}}
$t \leftarrow t+1$;
}
\Return $X_n$
\end{algorithm}

\subsection{Stage 2: Novel Policy Seeking}
\label{sec:methods-phase2}

In Stage 2 of our diverse strategy discovery framework, we explore potential strategy variations given a fixed initial state discovered in Stage 1. Formally, given an initial state $x$ and a set of discovered policies $\Pi = \{\pi_1, \pi_2, ..., \pi_n\}$, we aim to learn a new policy $\pi_{n+1}$ which is different from all existing $\pi_i \in \Pi$. This can be achieved with an additional policy novelty reward to be jointly optimized with the task reward during DRL training. We measure the novelty of policy $\pi_i$ with respect to $\pi_j$ by their corresponding motion feature distance $||f_i - f_j||$. The novelty reward function is then given by
\begin{equation}
    r_\text{novelty}(f) = \text{Clip}\left(\frac{\min_{\pi_i \in \Pi}{||f_i - f||}}{d_\text{threshold}}, 0.01, 1\right) ,
\end{equation}
which rewards simulation rollouts showing different strategies to the ones presented in the existing policy set. $d_\text{threshold}$ is a hyperparameter measuring the desired policy novelty to be learned next. Note that the feature representation $f$ here in Stage 2 can be the same as or different from the one used in Stage 1 for initial states exploration.

Our novel policy search is in principle similar to the idea of \cite{zhang2019novel-policies,Sun2020novel-policies}. However, there are two key differences. First, in machine learning, policy novelty metrics have been designed and validated only on low-dimensional control tasks. For example in \cite{zhang2019novel-policies}, the policy novelty is measured by the reconstruction error between states from the current rollout and previous rollouts encapsulated as a deep autoencoder. In our case of high-dimensional 3D character control tasks, however, novel state sequences do not necessarily correspond to novel motion strategies. We therefore opt to design discriminative strategy features whose distances are incorporated into the DRL training reward.

Second, we multiply the novelty reward with the task reward as the training reward, and adopt a standard gradient-based method PPO to train the policy. Additional optimization techniques are not required for learning novel strategies, such as the Task-Novelty Bisector method proposed in \cite{zhang2019novel-policies} that modifies the policy gradients to encourage novelty learning. Our novel policy seeking procedure always discovers novel policies since the character is forced to perform a different strategy. However, the novel policies may exhibit unnatural and awkward movements, when the given initial state is not capable of multiple natural strategies.
\section{Implementation: Ring Abstraction}
\label{sec:implement}
\subsection{Distributed \mbox{$G_t$} in QMC Solver}
\label{distributedG4}
Before introducing the communication phase of the ring abstraction layer,
it is important to understand how the authors distributed the large device array $G_t$ across MPI ranks.
%
Original $G_t$ was compared, and $G^d_t$ versions were distributed
(Figure~\ref{fig:compare_original_distributed_g4}). 


In the original $G_t$ implementation, the measurements---which were computed by matrix-matrix multiplication---are distributed statically and independently over the MPI ranks to avoid
inter-node communications. Each MPI rank keeps its partial copy of $G_{t,i}$ to accumulate 
measurements within a rank, where $i$ is the rank index. 
After all the measurements are finished, a reduction step is 
taken to accumulate $G_{t,i}$ across all MPI ranks into a final and complete
$G_t$ in the root MPI rank. The size of the $G_{t,i}$ in each rank is 
the same size as the final and complete $G_t$. 

With the distributed $G^d_t$ implementation, this large device array 
$G_t$ was evenly partitioned across all MPI ranks; each portion of it is local to each MPI rank.
%
Instead of keeping its partial copy of $G_t$, 
each rank now keeps an instance of $G^d_{t,i}$ to accumulate measurements 
of a portion or sub-slice of the final and complete $G_t$, where the notation
$d$ in $G^d_t$  refers to the distributed version, and $i$ means the $i$-th rank.
%
The $G^d_{t,i}$ size in each rank is 
reduced to $1/p$ of the size of the final and complete $G_t$, comparing the same configuration 
in original $G_t$ implementation, where $p$ is the number of MPI ranks used. 
%
For example, in Figure~\ref{fig:distributed_g4}, there are four ranks, and rank $i$
now only keeps $G^d_{t,i}$, which is one-fourth the size of the original $G_t$ array size.
% and contains values indexing from range of $[0, ..., N/4)$ in original $G_t$ array where $N$ is the 
% number of entries in  $G_t$  when viewed as a one-dimensional array.

To compute the final and complete $G^d_{t,i}$ for the distributed $G^d_t$ implementation, 
each rank must see every $G_{\sigma,i}$ from all ranks. 
%
In other words, each rank must broadcast the
locally generated $G_{\sigma,i}$ to the remainder of the other ranks at every measurement step. 
%
To efficiently perform this ``all-to-all'' broadcast, a ring abstraction layer was built (Section. \ref{section:ring_algorithm}), which circulates
all $G_{\sigma,i}$ across all ranks.

% over high-speed GPUs interconnect (GPUDirect RDMA) to facilitate the communication phase.

% \begin{figure}
% \centering
% \subfloat[Original $G_t$ implementation.]
%     {\includegraphics[width=\columnwidth]{original_g4.pdf}}\label{fig:original_g4}

% \subfloat[Distributed $G_t$ implementation.]
%     {\includegraphics[width=0.99\columnwidth]{distributed_g4.pdf} \label{fig:distributed_g4}}

% \caption{Comparison of the original $G_t$ vs. the distributed $G^d_t$ implementation. Each rank contains one GPU resource.}
% \label{fig:compare_original_distributed_g4} 
% \end{figure} 

\begin{figure}
\centering
     \begin{subfigure}[b]{\columnwidth}
         \centering
         \includegraphics[width=\textwidth]{images/original_g4.pdf}
         \caption{Original $G_t$ implementation.}
         \label{fig:original_g4}
     \end{subfigure}
     
    \begin{subfigure}[b]{\columnwidth}
         \centering
         \includegraphics[width=\textwidth]{images/distributed_g4.pdf}
         \caption{Distributed $G_t$ implementation.}
         \label{fig:distributed_g4}
     \end{subfigure}
     
\caption{Comparison of the original $G_t$ vs. the distributed $G^d_t$ implementation. Each rank contains one GPU resource.}
\label{fig:compare_original_distributed_g4}
\end{figure}

\subsection{Pipeline Ring Algorithm}
\label{section:ring_algorithm}
A pipeline ring algorithm was implemented that broadcasts the $G_{\sigma}$ 
array circularly during every measurement. 
%
The algorithm (Algorithm \ref{alg:ring_algorithm_code}) is 
visualized in Figure~\ref{fig:ring_algorithm_figure}.

\begin{algorithm}
\SetAlgoLined
    generateGSigma(gSigmaBuf)\; \label{lst:line:generateG2}
    updateG4(gSigmaBuf)\;       \label{lst:line:updateG4}
    %\texttt{\\}
    {$i\leftarrow 0$}\;         \label{lst:line:initStart}
    {$myRank \leftarrow worldRank$}\;          \label{lst:line:initRankId}
    {$ringSize \leftarrow mpiWorldSize$}\;      \label{lst:line:initRingSize}
    {$leftRank \leftarrow (myRank - 1 + ringSize) \: \% \: ringSize $}\;
    {$rightRank \leftarrow (myRank + 1 + ringSize) \: \% \: ringSize $}\;
    sendBuf.swap(gSigmaBuf)\;           \label{lst:line:initEnd}
    \While{$i< ringSize$}{
        MPI\_Irecv(recvBuf, source=leftRank, tag = recvTag, recvRequest)\; \label{lst:line:Irecv}
        MPI\_Isend(sendBuf, source=rightRank, tag = sendTag, sendRequest)\; \label{lst:line:Isend}
        MPI\_Wait(recvRequest)\;        \label{lst:line:recevBuffWait}
        
        updateG4(recvBuf)\;             \label{lst:line:updateG4_loop}
        
        MPI\_Wait(sendRequest)\;        \label{lst:line:sendBuffWait}
        
        sendBuf.swap(recvBuf)\;         \label{lst:line:swapBuff}
        i++\;
    }
\caption{Pipeline ring algorithm}
\label{alg:ring_algorithm_code}
\end{algorithm}

\begin{figure}
	\centering
	\includegraphics[width=\columnwidth, trim=0 5cm 0 0, clip]{images/ring_algorithm.pdf}
	\caption{Workflow of ring algorithm per iteration. }
	\label{fig:ring_algorithm_figure}
\end{figure}

At the start of every new measurement, a single-particle Green's function $G_{\sigma}$
 (Line~\ref{lst:line:generateG2}) is generated 
and then used to update $G^d_{t,i}$ (Line~\ref{lst:line:updateG4})
via the formula in Eq.~(\ref{eq:G4}).
%
% Different from original method that performs 
% full matrix-matrix multiplication (Equation~(\ref{eq:G4})), the current ring algorithm only performs partial
% matrix-matrix multiplication that contributes to $G^d_{t,i}$, a subslice of the final and complete $G_t$.
%
Between Lines \ref{lst:line:initStart} to \ref{lst:line:initEnd}, the algorithm 
initializes the indices
of left and right neighbors and prepares the sending message buffer from the
previously generated $G_{\sigma}$ buffer. 
%
The processes are organized as a ring so that the first and last rank are considered to be neighbors to each other. 
%
A \textit{swap} operation is used to avoid unnecessary memory copies for \textit{sendBuf} preparation.
%
A walker-accumulator thread allocates an additional \textit{recvBuf} buffer of the same size 
as \textit{gSigmaBuf} to hold incoming 
\textit{gSigmaBuf} buffer from \textit{leftRank}. 

The \textit{while} loop is the core part of the pipeline ring algorithm. 
%
For every iteration, each thread in a rank 
receives a $G_{\sigma}$ buffer from its left neighbor rank and sends a $G_{\sigma}$ buffer to its right neighbor rank. 
A synchronization step (Line~\ref{lst:line:recevBuffWait}) is performed
afterward to ensure that each rank receives a new buffer to update the 
local $G^d_{t,i}$ (Line~\ref{lst:line:updateG4_loop}). 
%
Another synchronization step
follows to ensure that all send requests are finalized 
(Line~\ref{lst:line:sendBuffWait}). Lastly, another \textit{swap} operation is used to exchange
content pointers between \textit{sendBuf} and \textit{recvBuf} to avoid unnecessary memory copy and prepare
for the next iteration of communication.
%
In the multi-threaded version (Section~\ref{subsec:multi-thread}), the thread of index, \textit{i}, only communicates with
	threads of index, \textit{i}, in neighbor ranks, and each thread allocates two buffers: \textit{sendBuff} and \textit{recvBuff}.

The \textit{while} loop will be terminated after $\mbox{\textit{ringSize}} - 1$ steps. By that time, 
each locally generated $G_{\sigma,i}$ will have traveled across all MPI ranks and
updated $G^d_{t,i}$ in all ranks. Eventually, each $G_{\sigma,i}$ reaches
to the left neighbor of its birth rank. For example, $G_{\sigma,0}$ generated from rank $0$ will end 
in last rank in the ring communicator.

Additionally, if the $G_t$ is too large to be stored in one node, 
it is optional to accumulate all $G^d_{t,i}$
at the end of all measurements. 
%
Instead, a parallel write into the file system could be taken.

\subsubsection{Sub-Ring Optimization.}

A sub-ring optimization strategy is further proposed to reduce message communication
times if the large device array $G_t$ can fit in fewer devices. 
%
The sub-ring algorithm is visualized in Figure~\ref{fig:subring_algorithm_figure}.

For the ring algorithm (Section~\ref{section:ring_algorithm}), the size of the ring communicator
(\textit{mpiWorldSize}) is set to the same size of the global \mbox{\texttt{MPI\_COMM\_WORLD}}, and thus the size of $G_t$ is equally 
distributed across all MPI ranks.

However, to complete the update to $G^d_{t,i}$ in one measurement, 
one $G_{\sigma,i}$
must travel \textit{mpiWorldSize} ranks. In total, 
there are \textit{mpiWorldSize} numbers of $G_{\sigma,i}$
being sent and received concurrently in one measurement 
in the global
\mbox{\texttt{MPI\_COMM\_WORLD}} 
communicator. If the size of $G^d_{t,i}$ is relatively small per rank, then this will cause high communication overhead.

If $G_t$ can be distributed and fitted in fewer devices, then a shorter travel distance is required 
for $G_{\sigma,i}$, thus reducing the communication overhead. One reduction
step was performed at the end of all measurements to accumulate $G^d_{t,s_i}$, 
where $s_i$ means $i$-th rank on the $s$-th sub-ring.

At the beginning of MPI initialization, the global \mbox{\texttt{MPI\_COMM\_WORLD}} was partitioned  into several new sub-ring communicators by using \mbox{\texttt{MPI\_Comm\_split}}. 
% where each new communicator represents conceptually a subring. 
The new
communicator information was passed to the DCA++ concurrency class by substituting the original global 
\mbox{\texttt{MPI\_COMM\_WORLD}} with this new communicator. Now, only a few minor modifications
are needed to transform the ring algorithm (Algorithm~\ref{alg:ring_algorithm_code})
to sub-ring Algorithm~\ref{alg:sub_ring_algorithm}. In Line~\ref{lst:line:initRankId}, \textit{myRank} is 
initialized to \textit{subRingRank} instead of \textit{worldRank}, where 
\textit{subRingRank} is the rank index in the local sub-ring communicator. 
%
In Line~\ref{lst:line:initRingSize}, \textit{ringSize} is initialized to \textit{subRingSize}
instead of \textit{mpiWorldSize}, where \textit{subRingSize} is the
size of the new communicator.
%
The general ring algorithm is a special case for the sub-ring algorithm because the
\textit{subRingSize} of the general ring algorithm is equal to \textit{mpiWorldSize}, and
there is only one sub-ring group throughout all MPI ranks.


\LinesNumberedHidden
\begin{algorithm}
    {$\mbox{\textit{myRank}} \leftarrow \mbox{\textit{subRingRank}}$}\;         
    {$\mbox{\textit{ringSize}} \leftarrow \mbox{\textit{subRingSize}}$}\;      
\caption{Modified ring algorithm to support sub-ring communication}
\label{alg:sub_ring_algorithm}
\end{algorithm}


\begin{figure}
	\centering
	\includegraphics[width=\columnwidth, trim=0 5cm 0 0, clip]{images/subring_alg.pdf}
	\caption{Workflow of sub-ring algorithm per iteration. Every consecutive $S$ rank forms a sub-ring communicator, 
	and no communication occurs between sub-ring communicators until all measurements are finished. Here, $S$ is the number of ranks in a sub-ring.}
	\label{fig:subring_algorithm_figure}
\end{figure}

\subsubsection{Multi-Threaded Ring Communication.}
\label{subsec:multi-thread}
To take advantage of the multi-threaded QMC model already in DCA++, 
multi-threaded ring communication support was further implemented in the ring algorithm.
%
Figure~\ref{fig:dca_overview} shows that in the original DCA++ method,
each walker-accumulator
thread in a rank is independent of each other, and all the threads in a 
rank synchronize only after all rank-local measurements are finished.
%
Moreover, during every measurement, each walker-accumulator thread
generates its own thread-private $G_{\sigma, i}$ to update $G_t$. 
%

The multi-threaded ring algorithm now allows concurrent message exchange so that threads of same rank-local thread index exchange their thread-private $G_{\sigma, i}$. 
%
Conceptually, there are $k$ parallel and independent rings, where $k$ 
is number of threads per rank, because threads of the same local thread ID
form a closed ring. 
%
For example, a thread of index $0$ in rank $0$ will send its $G_\sigma$ to 
the thread of index $0$ in rank $1$ and receive another $G_\sigma$ from thread index of $0$ 
from last rank in the ring algorithm.
%

The only changes in the ring algorithm are offsetting the tag values 
(\texttt{recvTag} and \texttt{sendTag}) by the thread index value. For example,
Lines~\ref{lst:line:Irecv} and ~\ref{lst:line:Isend} from 
Algorithm~\ref{alg:ring_algorithm_code} are modified to Algorithm~\ref{alg:multi_threaded_ring}.

\LinesNumberedHidden
\begin{algorithm}
        MPI\_Irecv(recvBuf, source=leftRank, tag = recvTag + threadId, recvRequest)\; 
        MPI\_Isend(sendBuf, source=rightRank, tag = sendTag + threadId, sendRequest)\;
\caption{Modified ring algorithm to support multi-threaded ring}
\label{alg:multi_threaded_ring}
\end{algorithm}

To efficiently send and receive $G_\sigma$, each thread
will allocate one additional \textit{recvBuff} to hold incoming 
\textit{gSigmaBuf} buffer from \textit{leftRank} and perform send/receive efficiently.
%
In the original DCA++ method, there are $k$ numbers of buffers of $G_\sigma$ 
size per rank, and in the multi-threaded ring method, there are $2k$
numbers of buffers of $G_\sigma$ size per rank, where $k$ is number of 
threads per rank.

%!TEX ROOT = ../../centralized_vs_distributed.tex

\section{{\titlecap{the centralized-distributed trade-off}}}\label{sec:numerical-results}

\revision{In the previous sections we formulated the optimal control problem for a given controller architecture
(\ie the number of links) parametrized by $ n $
and showed how to compute minimum-variance objective function and the corresponding constraints.
In this section, we present our main result:
%\red{for a ring topology with multiple options for the parameter $ n $},
we solve the optimal control problem for each $ n $ and compare the best achievable closed-loop performance with different control architectures.\footnote{
\revision{Recall that small (large) values of $ n $ mean sparse (dense) architectures.}}
For delays that increase linearly with $n$,
\ie $ f(n) \propto n $, 
we demonstrate that distributed controllers with} {few communication links outperform controllers with larger number of communication links.}

\textcolor{subsectioncolor}{Figure~\ref{fig:cont-time-single-int-opt-var}} shows the steady-state variances
obtained with single-integrator dynamics~\eqref{eq:cont-time-single-int-variance-minimization}
%where we compare the standard multi-parameter design 
%with a simplified version \tcb{that utilizes spatially-constant feedback gains
and the quadratic approximation~\eqref{eq:quadratic-approximation} for \revision{ring topology}
with $ N = 50 $ nodes. % and $ n\in\{1,\dots,10\} $.
%with $ N = 50 $, $ f(n) = n $ and $ \tau_{\textit{min}} = 0.1 $.
%\autoref{fig:cont-time-single-int-err} shows the relative error, defined as
%\begin{equation}\label{eq:relative-error}
%	e \doteq \dfrac{\optvarx-\optvar}{\optvar}
%\end{equation}
%where $ \optvar $ and $ \optvarx $ denote the the optimal and sub-optimal scalar variances, respectively.
%The performance gap is small
%and becomes negligible for large $ n $.
{The best performance is achieved for a sparse architecture with  $ n = 2 $ 
in which each agent communicates with the two closest pairs of neighboring nodes. 
This should be compared and contrasted to nearest-neighbor and all-to-all 
communication topologies which induce higher closed-loop variances. 
Thus, 
the advantage of introducing additional communication links diminishes 
beyond}
{a certain threshold because of communication delays.}

%For a linear increase in the delay,
\textcolor{subsectioncolor}{Figure~\ref{fig:cont-time-double-int-opt-var}} shows that the use of approximation~\eqref{eq:cont-time-double-int-min-var-simplified} with $ \tilde{\gvel}^* = 70 $
identifies nearest-neighbor information exchange as the {near-optimal} architecture for a double-integrator model
with ring topology. 
This can be explained by noting that the variance of the process noise $ n(t) $
in the reduced model~\eqref{eq:x-dynamics-1st-order-approximation}
is proportional to $ \nicefrac{1}{\gvel} $ and thereby to $ \taun $,
according to~\eqref{eq:substitutions-4-normalization},
making the variance scale with the delay.

%\mjmargin{i feel that we need to comment about different results that we obtained for CT and DT double-intergrator dynamics (monotonic deterioration of performance for the former and oscillations for the latter)}
\revision{\textcolor{subsectioncolor}{Figures~\ref{fig:disc-time-single-int-opt-var}--\ref{fig:disc-time-double-int-opt-var}}
show the results obtained by solving the optimal control problem for discrete-time dynamics.
%which exhibit similar trade-offs.
The oscillations about the minimum in~\autoref{fig:disc-time-double-int-opt-var}
are compatible with the investigated \tradeoff~\eqref{eq:trade-off}:
in general, 
the sum of two monotone functions does not have a unique local minimum.
Details about discrete-time systems are deferred to~\autoref{sec:disc-time}.
Interestingly,
double integrators with continuous- (\autoref{fig:cont-time-double-int-opt-var}) ad discrete-time (\autoref{fig:disc-time-double-int-opt-var}) dynamics
exhibits very different trade-off curves,
whereby performance monotonically deteriorates for the former and oscillates for the latter.
While a clear interpretation is difficult because there is no explicit expression of the variance as a function of $ n $,
one possible explanation might be the first-order approximation used to compute gains in the continuous-time case.
%which reinforce our thesis exposed in~\autoref{sec:contribution}.

%\begin{figure}
%	\centering
%	\includegraphics[width=.6\linewidth]{cont-time-double-int-opt-var-n}
%	\caption{Steady-state scalar variance for continuous-time double integrators with $ \taun = 0.1n $.
%		Here, the \tradeoff is optimized by nearest-neighbor interaction.
%	}
%	\label{fig:cont-time-double-int-opt-var-lin}
%\end{figure}
}

\begin{figure}
	\centering
	\begin{minipage}[l]{.5\linewidth}
		\centering
		\includegraphics[width=\linewidth]{random-graph}
	\end{minipage}%
	\begin{minipage}[r]{.5\linewidth}
		\centering
		\includegraphics[width=\linewidth]{disc-time-single-int-random-graph-opt-var}
	\end{minipage}
	\caption{Network topology and its optimal {closed-loop} variance.}
	\label{fig:general-graph}
\end{figure}

Finally,
\autoref{fig:general-graph} shows the optimization results for a random graph topology with discrete-time single integrator agents. % with a linear increase in the delay, $ \taun = n $.
Here, $ n $ denotes the number of communication hops in the ``original" network, shown in~\autoref{fig:general-graph}:
as $ n $ increases, each agent can first communicate with its nearest neighbors,
then with its neighbors' neighbors, and so on. For a control architecture that utilizes different feedback gains for each communication link
	(\ie we only require $ K = K^\top $) we demonstrate that, in this case, two communication hops provide optimal closed-loop performance. % of the system.}

Additional computational experiments performed with different rates $ f(\cdot) $ show that the optimal number of links increases for slower rates: 
for example, 
the optimal number of links is larger for $ f(n) = \sqrt{n} $ than for $ f(n) = n $. 
\revision{These results are not reported because of space limitations.}
% \vspace{-0.5em}
\section{Conclusion}
% \vspace{-0.5em}
Recent advances in multimodal single-cell technology have enabled the simultaneous profiling of the transcriptome alongside other cellular modalities, leading to an increase in the availability of multimodal single-cell data. In this paper, we present \method{}, a multimodal transformer model for single-cell surface protein abundance from gene expression measurements. We combined the data with prior biological interaction knowledge from the STRING database into a richly connected heterogeneous graph and leveraged the transformer architectures to learn an accurate mapping between gene expression and surface protein abundance. Remarkably, \method{} achieves superior and more stable performance than other baselines on both 2021 and 2022 NeurIPS single-cell datasets.

\noindent\textbf{Future Work.}
% Our work is an extension of the model we implemented in the NeurIPS 2022 competition. 
Our framework of multimodal transformers with the cross-modality heterogeneous graph goes far beyond the specific downstream task of modality prediction, and there are lots of potentials to be further explored. Our graph contains three types of nodes. While the cell embeddings are used for predictions, the remaining protein embeddings and gene embeddings may be further interpreted for other tasks. The similarities between proteins may show data-specific protein-protein relationships, while the attention matrix of the gene transformer may help to identify marker genes of each cell type. Additionally, we may achieve gene interaction prediction using the attention mechanism.
% under adequate regulations. 
% We expect \method{} to be capable of much more than just modality prediction. Note that currently, we fuse information from different transformers with message-passing GNNs. 
To extend more on transformers, a potential next step is implementing cross-attention cross-modalities. Ideally, all three types of nodes, namely genes, proteins, and cells, would be jointly modeled using a large transformer that includes specific regulations for each modality. 

% insight of protein and gene embedding (diff task)

% all in one transformer

% \noindent\textbf{Limitations and future work}
% Despite the noticeable performance improvement by utilizing transformers with the cross-modality heterogeneous graph, there are still bottlenecks in the current settings. To begin with, we noticed that the performance variations of all methods are consistently higher in the ``CITE'' dataset compared to the ``GEX2ADT'' dataset. We hypothesized that the increased variability in ``CITE'' was due to both less number of training samples (43k vs. 66k cells) and a significantly more number of testing samples used (28k vs. 1k cells). One straightforward solution to alleviate the high variation issue is to include more training samples, which is not always possible given the training data availability. Nevertheless, publicly available single-cell datasets have been accumulated over the past decades and are still being collected on an ever-increasing scale. Taking advantage of these large-scale atlases is the key to a more stable and well-performing model, as some of the intra-cell variations could be common across different datasets. For example, reference-based methods are commonly used to identify the cell identity of a single cell, or cell-type compositions of a mixture of cells. (other examples for pretrained, e.g., scbert)


%\noindent\textbf{Future work.}
% Our work is an extension of the model we implemented in the NeurIPS 2022 competition. Now our framework of multimodal transformers with the cross-modality heterogeneous graph goes far beyond the specific downstream task of modality prediction, and there are lots of potentials to be further explored. Our graph contains three types of nodes. while the cell embeddings are used for predictions, the remaining protein embeddings and gene embeddings may be further interpreted for other tasks. The similarities between proteins may show data-specific protein-protein relationships, while the attention matrix of the gene transformer may help to identify marker genes of each cell type. Additionally, we may achieve gene interaction prediction using the attention mechanism under adequate regulations. We expect \method{} to be capable of much more than just modality prediction. Note that currently, we fuse information from different transformers with message-passing GNNs. To extend more on transformers, a potential next step is implementing cross-attention cross-modalities. Ideally, all three types of nodes, namely genes, proteins, and cells, would be jointly modeled using a large transformer that includes specific regulations for each modality. The self-attention within each modality would reconstruct the prior interaction network, while the cross-attention between modalities would be supervised by the data observations. Then, The attention matrix will provide insights into all the internal interactions and cross-relationships. With the linearized transformer, this idea would be both practical and versatile.

% \begin{acks}
% This research is supported by the National Science Foundation (NSF) and Johnson \& Johnson.
% \end{acks}

\begin{acks}
We would like to thank the anonymous reviewers for their constructive feedback. We thank Beyond Capture for helping us with motion capture various high jumps. We also thank Toptal for providing us the jumping scene. Yin is partially supported by NSERC Discovery Grants Program RGPIN-06797 and RGPAS-522723. Van de Panne is partially supported by NSERC RGPIN-2020-05929.
\end{acks}

\bibliographystyle{ACM-Reference-Format}
\bibliography{jumping}

\chapter{Supplementary Material}
\label{appendix}

In this appendix, we present supplementary material for the techniques and
experiments presented in the main text.

\section{Baseline Results and Analysis for Informed Sampler}
\label{appendix:chap3}

Here, we give an in-depth
performance analysis of the various samplers and the effect of their
hyperparameters. We choose hyperparameters with the lowest PSRF value
after $10k$ iterations, for each sampler individually. If the
differences between PSRF are not significantly different among
multiple values, we choose the one that has the highest acceptance
rate.

\subsection{Experiment: Estimating Camera Extrinsics}
\label{appendix:chap3:room}

\subsubsection{Parameter Selection}
\paragraph{Metropolis Hastings (\MH)}

Figure~\ref{fig:exp1_MH} shows the median acceptance rates and PSRF
values corresponding to various proposal standard deviations of plain
\MH~sampling. Mixing gets better and the acceptance rate gets worse as
the standard deviation increases. The value $0.3$ is selected standard
deviation for this sampler.

\paragraph{Metropolis Hastings Within Gibbs (\MHWG)}

As mentioned in Section~\ref{sec:room}, the \MHWG~sampler with one-dimensional
updates did not converge for any value of proposal standard deviation.
This problem has high correlation of the camera parameters and is of
multi-modal nature, which this sampler has problems with.

\paragraph{Parallel Tempering (\PT)}

For \PT~sampling, we took the best performing \MH~sampler and used
different temperature chains to improve the mixing of the
sampler. Figure~\ref{fig:exp1_PT} shows the results corresponding to
different combination of temperature levels. The sampler with
temperature levels of $[1,3,27]$ performed best in terms of both
mixing and acceptance rate.

\paragraph{Effect of Mixture Coefficient in Informed Sampling (\MIXLMH)}

Figure~\ref{fig:exp1_alpha} shows the effect of mixture
coefficient ($\alpha$) on the informed sampling
\MIXLMH. Since there is no significant different in PSRF values for
$0 \le \alpha \le 0.7$, we chose $0.7$ due to its high acceptance
rate.


% \end{multicols}

\begin{figure}[h]
\centering
  \subfigure[MH]{%
    \includegraphics[width=.48\textwidth]{figures/supplementary/camPose_MH.pdf} \label{fig:exp1_MH}
  }
  \subfigure[PT]{%
    \includegraphics[width=.48\textwidth]{figures/supplementary/camPose_PT.pdf} \label{fig:exp1_PT}
  }
\\
  \subfigure[INF-MH]{%
    \includegraphics[width=.48\textwidth]{figures/supplementary/camPose_alpha.pdf} \label{fig:exp1_alpha}
  }
  \mycaption{Results of the `Estimating Camera Extrinsics' experiment}{PRSFs and Acceptance rates corresponding to (a) various standard deviations of \MH, (b) various temperature level combinations of \PT sampling and (c) various mixture coefficients of \MIXLMH sampling.}
\end{figure}



\begin{figure}[!t]
\centering
  \subfigure[\MH]{%
    \includegraphics[width=.48\textwidth]{figures/supplementary/occlusionExp_MH.pdf} \label{fig:exp2_MH}
  }
  \subfigure[\BMHWG]{%
    \includegraphics[width=.48\textwidth]{figures/supplementary/occlusionExp_BMHWG.pdf} \label{fig:exp2_BMHWG}
  }
\\
  \subfigure[\MHWG]{%
    \includegraphics[width=.48\textwidth]{figures/supplementary/occlusionExp_MHWG.pdf} \label{fig:exp2_MHWG}
  }
  \subfigure[\PT]{%
    \includegraphics[width=.48\textwidth]{figures/supplementary/occlusionExp_PT.pdf} \label{fig:exp2_PT}
  }
\\
  \subfigure[\INFBMHWG]{%
    \includegraphics[width=.5\textwidth]{figures/supplementary/occlusionExp_alpha.pdf} \label{fig:exp2_alpha}
  }
  \mycaption{Results of the `Occluding Tiles' experiment}{PRSF and
    Acceptance rates corresponding to various standard deviations of
    (a) \MH, (b) \BMHWG, (c) \MHWG, (d) various temperature level
    combinations of \PT~sampling and; (e) various mixture coefficients
    of our informed \INFBMHWG sampling.}
\end{figure}

%\onecolumn\newpage\twocolumn
\subsection{Experiment: Occluding Tiles}
\label{appendix:chap3:tiles}

\subsubsection{Parameter Selection}

\paragraph{Metropolis Hastings (\MH)}

Figure~\ref{fig:exp2_MH} shows the results of
\MH~sampling. Results show the poor convergence for all proposal
standard deviations and rapid decrease of AR with increasing standard
deviation. This is due to the high-dimensional nature of
the problem. We selected a standard deviation of $1.1$.

\paragraph{Blocked Metropolis Hastings Within Gibbs (\BMHWG)}

The results of \BMHWG are shown in Figure~\ref{fig:exp2_BMHWG}. In
this sampler we update only one block of tile variables (of dimension
four) in each sampling step. Results show much better performance
compared to plain \MH. The optimal proposal standard deviation for
this sampler is $0.7$.

\paragraph{Metropolis Hastings Within Gibbs (\MHWG)}

Figure~\ref{fig:exp2_MHWG} shows the result of \MHWG sampling. This
sampler is better than \BMHWG and converges much more quickly. Here
a standard deviation of $0.9$ is found to be best.

\paragraph{Parallel Tempering (\PT)}

Figure~\ref{fig:exp2_PT} shows the results of \PT sampling with various
temperature combinations. Results show no improvement in AR from plain
\MH sampling and again $[1,3,27]$ temperature levels are found to be optimal.

\paragraph{Effect of Mixture Coefficient in Informed Sampling (\INFBMHWG)}

Figure~\ref{fig:exp2_alpha} shows the effect of mixture
coefficient ($\alpha$) on the blocked informed sampling
\INFBMHWG. Since there is no significant different in PSRF values for
$0 \le \alpha \le 0.8$, we chose $0.8$ due to its high acceptance
rate.



\subsection{Experiment: Estimating Body Shape}
\label{appendix:chap3:body}

\subsubsection{Parameter Selection}
\paragraph{Metropolis Hastings (\MH)}

Figure~\ref{fig:exp3_MH} shows the result of \MH~sampling with various
proposal standard deviations. The value of $0.1$ is found to be
best.

\paragraph{Metropolis Hastings Within Gibbs (\MHWG)}

For \MHWG sampling we select $0.3$ proposal standard
deviation. Results are shown in Fig.~\ref{fig:exp3_MHWG}.


\paragraph{Parallel Tempering (\PT)}

As before, results in Fig.~\ref{fig:exp3_PT}, the temperature levels
were selected to be $[1,3,27]$ due its slightly higher AR.

\paragraph{Effect of Mixture Coefficient in Informed Sampling (\MIXLMH)}

Figure~\ref{fig:exp3_alpha} shows the effect of $\alpha$ on PSRF and
AR. Since there is no significant differences in PSRF values for $0 \le
\alpha \le 0.8$, we choose $0.8$.


\begin{figure}[t]
\centering
  \subfigure[\MH]{%
    \includegraphics[width=.48\textwidth]{figures/supplementary/bodyShape_MH.pdf} \label{fig:exp3_MH}
  }
  \subfigure[\MHWG]{%
    \includegraphics[width=.48\textwidth]{figures/supplementary/bodyShape_MHWG.pdf} \label{fig:exp3_MHWG}
  }
\\
  \subfigure[\PT]{%
    \includegraphics[width=.48\textwidth]{figures/supplementary/bodyShape_PT.pdf} \label{fig:exp3_PT}
  }
  \subfigure[\MIXLMH]{%
    \includegraphics[width=.48\textwidth]{figures/supplementary/bodyShape_alpha.pdf} \label{fig:exp3_alpha}
  }
\\
  \mycaption{Results of the `Body Shape Estimation' experiment}{PRSFs and
    Acceptance rates corresponding to various standard deviations of
    (a) \MH, (b) \MHWG; (c) various temperature level combinations
    of \PT sampling and; (d) various mixture coefficients of the
    informed \MIXLMH sampling.}
\end{figure}


\subsection{Results Overview}
Figure~\ref{fig:exp_summary} shows the summary results of the all the three
experimental studies related to informed sampler.
\begin{figure*}[h!]
\centering
  \subfigure[Results for: Estimating Camera Extrinsics]{%
    \includegraphics[width=0.9\textwidth]{figures/supplementary/camPose_ALL.pdf} \label{fig:exp1_all}
  }
  \subfigure[Results for: Occluding Tiles]{%
    \includegraphics[width=0.9\textwidth]{figures/supplementary/occlusionExp_ALL.pdf} \label{fig:exp2_all}
  }
  \subfigure[Results for: Estimating Body Shape]{%
    \includegraphics[width=0.9\textwidth]{figures/supplementary/bodyShape_ALL.pdf} \label{fig:exp3_all}
  }
  \label{fig:exp_summary}
  \mycaption{Summary of the statistics for the three experiments}{Shown are
    for several baseline methods and the informed samplers the
    acceptance rates (left), PSRFs (middle), and RMSE values
    (right). All results are median results over multiple test
    examples.}
\end{figure*}

\subsection{Additional Qualitative Results}

\subsubsection{Occluding Tiles}
In Figure~\ref{fig:exp2_visual_more} more qualitative results of the
occluding tiles experiment are shown. The informed sampling approach
(\INFBMHWG) is better than the best baseline (\MHWG). This still is a
very challenging problem since the parameters for occluded tiles are
flat over a large region. Some of the posterior variance of the
occluded tiles is already captured by the informed sampler.

\begin{figure*}[h!]
\begin{center}
\centerline{\includegraphics[width=0.95\textwidth]{figures/supplementary/occlusionExp_Visual.pdf}}
\mycaption{Additional qualitative results of the occluding tiles experiment}
  {From left to right: (a)
  Given image, (b) Ground truth tiles, (c) OpenCV heuristic and most probable estimates
  from 5000 samples obtained by (d) MHWG sampler (best baseline) and
  (e) our INF-BMHWG sampler. (f) Posterior expectation of the tiles
  boundaries obtained by INF-BMHWG sampling (First 2000 samples are
  discarded as burn-in).}
\label{fig:exp2_visual_more}
\end{center}
\end{figure*}

\subsubsection{Body Shape}
Figure~\ref{fig:exp3_bodyMeshes} shows some more results of 3D mesh
reconstruction using posterior samples obtained by our informed
sampling \MIXLMH.

\begin{figure*}[t]
\begin{center}
\centerline{\includegraphics[width=0.75\textwidth]{figures/supplementary/bodyMeshResults.pdf}}
\mycaption{Qualitative results for the body shape experiment}
  {Shown is the 3D mesh reconstruction results with first 1000 samples obtained
  using the \MIXLMH informed sampling method. (blue indicates small
  values and red indicates high values)}
\label{fig:exp3_bodyMeshes}
\end{center}
\end{figure*}

\clearpage



\section{Additional Results on the Face Problem with CMP}

Figure~\ref{fig:shading-qualitative-multiple-subjects-supp} shows inference results for reflectance maps, normal maps and lights for randomly chosen test images, and Fig.~\ref{fig:shading-qualitative-same-subject-supp} shows reflectance estimation results on multiple images of the same subject produced under different illumination conditions. CMP is able to produce estimates that are closer to the groundtruth across different subjects and illumination conditions.

\begin{figure*}[h]
  \begin{center}
  \centerline{\includegraphics[width=1.0\columnwidth]{figures/face_cmp_visual_results_supp.pdf}}
  \vspace{-1.2cm}
  \end{center}
	\mycaption{A visual comparison of inference results}{(a)~Observed images. (b)~Inferred reflectance maps. \textit{GT} is the photometric stereo groundtruth, \textit{BU} is the Biswas \etal (2009) reflectance estimate and \textit{Forest} is the consensus prediction. (c)~The variance of the inferred reflectance estimate produced by \MTD (normalized across rows).(d)~Visualization of inferred light directions. (e)~Inferred normal maps.}
	\label{fig:shading-qualitative-multiple-subjects-supp}
\end{figure*}


\begin{figure*}[h]
	\centering
	\setlength\fboxsep{0.2mm}
	\setlength\fboxrule{0pt}
	\begin{tikzpicture}

		\matrix at (0, 0) [matrix of nodes, nodes={anchor=east}, column sep=-0.05cm, row sep=-0.2cm]
		{
			\fbox{\includegraphics[width=1cm]{figures/sample_3_4_X.png}} &
			\fbox{\includegraphics[width=1cm]{figures/sample_3_4_GT.png}} &
			\fbox{\includegraphics[width=1cm]{figures/sample_3_4_BISWAS.png}}  &
			\fbox{\includegraphics[width=1cm]{figures/sample_3_4_VMP.png}}  &
			\fbox{\includegraphics[width=1cm]{figures/sample_3_4_FOREST.png}}  &
			\fbox{\includegraphics[width=1cm]{figures/sample_3_4_CMP.png}}  &
			\fbox{\includegraphics[width=1cm]{figures/sample_3_4_CMPVAR.png}}
			 \\

			\fbox{\includegraphics[width=1cm]{figures/sample_3_5_X.png}} &
			\fbox{\includegraphics[width=1cm]{figures/sample_3_5_GT.png}} &
			\fbox{\includegraphics[width=1cm]{figures/sample_3_5_BISWAS.png}}  &
			\fbox{\includegraphics[width=1cm]{figures/sample_3_5_VMP.png}}  &
			\fbox{\includegraphics[width=1cm]{figures/sample_3_5_FOREST.png}}  &
			\fbox{\includegraphics[width=1cm]{figures/sample_3_5_CMP.png}}  &
			\fbox{\includegraphics[width=1cm]{figures/sample_3_5_CMPVAR.png}}
			 \\

			\fbox{\includegraphics[width=1cm]{figures/sample_3_6_X.png}} &
			\fbox{\includegraphics[width=1cm]{figures/sample_3_6_GT.png}} &
			\fbox{\includegraphics[width=1cm]{figures/sample_3_6_BISWAS.png}}  &
			\fbox{\includegraphics[width=1cm]{figures/sample_3_6_VMP.png}}  &
			\fbox{\includegraphics[width=1cm]{figures/sample_3_6_FOREST.png}}  &
			\fbox{\includegraphics[width=1cm]{figures/sample_3_6_CMP.png}}  &
			\fbox{\includegraphics[width=1cm]{figures/sample_3_6_CMPVAR.png}}
			 \\
	     };

       \node at (-3.85, -2.0) {\small Observed};
       \node at (-2.55, -2.0) {\small `GT'};
       \node at (-1.27, -2.0) {\small BU};
       \node at (0.0, -2.0) {\small MP};
       \node at (1.27, -2.0) {\small Forest};
       \node at (2.55, -2.0) {\small \textbf{CMP}};
       \node at (3.85, -2.0) {\small Variance};

	\end{tikzpicture}
	\mycaption{Robustness to varying illumination}{Reflectance estimation on a subject images with varying illumination. Left to right: observed image, photometric stereo estimate (GT)
  which is used as a proxy for groundtruth, bottom-up estimate of \cite{Biswas2009}, VMP result, consensus forest estimate, CMP mean, and CMP variance.}
	\label{fig:shading-qualitative-same-subject-supp}
\end{figure*}

\clearpage

\section{Additional Material for Learning Sparse High Dimensional Filters}
\label{sec:appendix-bnn}

This part of supplementary material contains a more detailed overview of the permutohedral
lattice convolution in Section~\ref{sec:permconv}, more experiments in
Section~\ref{sec:addexps} and additional results with protocols for
the experiments presented in Chapter~\ref{chap:bnn} in Section~\ref{sec:addresults}.

\vspace{-0.2cm}
\subsection{General Permutohedral Convolutions}
\label{sec:permconv}

A core technical contribution of this work is the generalization of the Gaussian permutohedral lattice
convolution proposed in~\cite{adams2010fast} to the full non-separable case with the
ability to perform back-propagation. Although, conceptually, there are minor
differences between Gaussian and general parameterized filters, there are non-trivial practical
differences in terms of the algorithmic implementation. The Gauss filters belong to
the separable class and can thus be decomposed into multiple
sequential one dimensional convolutions. We are interested in the general filter
convolutions, which can not be decomposed. Thus, performing a general permutohedral
convolution at a lattice point requires the computation of the inner product with the
neighboring elements in all the directions in the high-dimensional space.

Here, we give more details of the implementation differences of separable
and non-separable filters. In the following, we will explain the scalar case first.
Recall, that the forward pass of general permutohedral convolution
involves 3 steps: \textit{splatting}, \textit{convolving} and \textit{slicing}.
We follow the same splatting and slicing strategies as in~\cite{adams2010fast}
since these operations do not depend on the filter kernel. The main difference
between our work and the existing implementation of~\cite{adams2010fast} is
the way that the convolution operation is executed. This proceeds by constructing
a \emph{blur neighbor} matrix $K$ that stores for every lattice point all
values of the lattice neighbors that are needed to compute the filter output.

\begin{figure}[t!]
  \centering
    \includegraphics[width=0.6\columnwidth]{figures/supplementary/lattice_construction}
  \mycaption{Illustration of 1D permutohedral lattice construction}
  {A $4\times 4$ $(x,y)$ grid lattice is projected onto the plane defined by the normal
  vector $(1,1)^{\top}$. This grid has $s+1=4$ and $d=2$ $(s+1)^{d}=4^2=16$ elements.
  In the projection, all points of the same color are projected onto the same points in the plane.
  The number of elements of the projected lattice is $t=(s+1)^d-s^d=4^2-3^2=7$, that is
  the $(4\times 4)$ grid minus the size of lattice that is $1$ smaller at each size, in this
  case a $(3\times 3)$ lattice (the upper right $(3\times 3)$ elements).
  }
\label{fig:latticeconstruction}
\end{figure}

The blur neighbor matrix is constructed by traversing through all the populated
lattice points and their neighboring elements.
% For efficiency, we do this matrix construction recursively with shared computations
% since $n^{th}$ neighbourhood elements are $1^{st}$ neighborhood elements of $n-1^{th}$ neighbourhood elements. \pg{do not understand}
This is done recursively to share computations. For any lattice point, the neighbors that are
$n$ hops away are the direct neighbors of the points that are $n-1$ hops away.
The size of a $d$ dimensional spatial filter with width $s+1$ is $(s+1)^{d}$ (\eg, a
$3\times 3$ filter, $s=2$ in $d=2$ has $3^2=9$ elements) and this size grows
exponentially in the number of dimensions $d$. The permutohedral lattice is constructed by
projecting a regular grid onto the plane spanned by the $d$ dimensional normal vector ${(1,\ldots,1)}^{\top}$. See
Fig.~\ref{fig:latticeconstruction} for an illustration of the 1D lattice construction.
Many corners of a grid filter are projected onto the same point, in total $t = {(s+1)}^{d} -
s^{d}$ elements remain in the permutohedral filter with $s$ neighborhood in $d-1$ dimensions.
If the lattice has $m$ populated elements, the
matrix $K$ has size $t\times m$. Note that, since the input signal is typically
sparse, only a few lattice corners are being populated in the \textit{slicing} step.
We use a hash-table to keep track of these points and traverse only through
the populated lattice points for this neighborhood matrix construction.

Once the blur neighbor matrix $K$ is constructed, we can perform the convolution
by the matrix vector multiplication
\begin{equation}
\ell' = BK,
\label{eq:conv}
\end{equation}
where $B$ is the $1 \times t$ filter kernel (whose values we will learn) and $\ell'\in\mathbb{R}^{1\times m}$
is the result of the filtering at the $m$ lattice points. In practice, we found that the
matrix $K$ is sometimes too large to fit into GPU memory and we divided the matrix $K$
into smaller pieces to compute Eq.~\ref{eq:conv} sequentially.

In the general multi-dimensional case, the signal $\ell$ is of $c$ dimensions. Then
the kernel $B$ is of size $c \times t$ and $K$ stores the $c$ dimensional vectors
accordingly. When the input and output points are different, we slice only the
input points and splat only at the output points.


\subsection{Additional Experiments}
\label{sec:addexps}
In this section, we discuss more use-cases for the learned bilateral filters, one
use-case of BNNs and two single filter applications for image and 3D mesh denoising.

\subsubsection{Recognition of subsampled MNIST}\label{sec:app_mnist}

One of the strengths of the proposed filter convolution is that it does not
require the input to lie on a regular grid. The only requirement is to define a distance
between features of the input signal.
We highlight this feature with the following experiment using the
classical MNIST ten class classification problem~\cite{lecun1998mnist}. We sample a
sparse set of $N$ points $(x,y)\in [0,1]\times [0,1]$
uniformly at random in the input image, use their interpolated values
as signal and the \emph{continuous} $(x,y)$ positions as features. This mimics
sub-sampling of a high-dimensional signal. To compare against a spatial convolution,
we interpolate the sparse set of values at the grid positions.

We take a reference implementation of LeNet~\cite{lecun1998gradient} that
is part of the Caffe project~\cite{jia2014caffe} and compare it
against the same architecture but replacing the first convolutional
layer with a bilateral convolution layer (BCL). The filter size
and numbers are adjusted to get a comparable number of parameters
($5\times 5$ for LeNet, $2$-neighborhood for BCL).

The results are shown in Table~\ref{tab:all-results}. We see that training
on the original MNIST data (column Original, LeNet vs. BNN) leads to a slight
decrease in performance of the BNN (99.03\%) compared to LeNet
(99.19\%). The BNN can be trained and evaluated on sparse
signals, and we resample the image as described above for $N=$ 100\%, 60\% and
20\% of the total number of pixels. The methods are also evaluated
on test images that are subsampled in the same way. Note that we can
train and test with different subsampling rates. We introduce an additional
bilinear interpolation layer for the LeNet architecture to train on the same
data. In essence, both models perform a spatial interpolation and thus we
expect them to yield a similar classification accuracy. Once the data is of
higher dimensions, the permutohedral convolution will be faster due to hashing
the sparse input points, as well as less memory demanding in comparison to
naive application of a spatial convolution with interpolated values.

\begin{table}[t]
  \begin{center}
    \footnotesize
    \centering
    \begin{tabular}[t]{lllll}
      \toprule
              &     & \multicolumn{3}{c}{Test Subsampling} \\
       Method  & Original & 100\% & 60\% & 20\%\\
      \midrule
       LeNet &  \textbf{0.9919} & 0.9660 & 0.9348 & \textbf{0.6434} \\
       BNN &  0.9903 & \textbf{0.9844} & \textbf{0.9534} & 0.5767 \\
      \hline
       LeNet 100\% & 0.9856 & 0.9809 & 0.9678 & \textbf{0.7386} \\
       BNN 100\% & \textbf{0.9900} & \textbf{0.9863} & \textbf{0.9699} & 0.6910 \\
      \hline
       LeNet 60\% & 0.9848 & 0.9821 & 0.9740 & 0.8151 \\
       BNN 60\% & \textbf{0.9885} & \textbf{0.9864} & \textbf{0.9771} & \textbf{0.8214}\\
      \hline
       LeNet 20\% & \textbf{0.9763} & \textbf{0.9754} & 0.9695 & 0.8928 \\
       BNN 20\% & 0.9728 & 0.9735 & \textbf{0.9701} & \textbf{0.9042}\\
      \bottomrule
    \end{tabular}
  \end{center}
\vspace{-.2cm}
\caption{Classification accuracy on MNIST. We compare the
    LeNet~\cite{lecun1998gradient} implementation that is part of
    Caffe~\cite{jia2014caffe} to the network with the first layer
    replaced by a bilateral convolution layer (BCL). Both are trained
    on the original image resolution (first two rows). Three more BNN
    and CNN models are trained with randomly subsampled images (100\%,
    60\% and 20\% of the pixels). An additional bilinear interpolation
    layer samples the input signal on a spatial grid for the CNN model.
  }
  \label{tab:all-results}
\vspace{-.5cm}
\end{table}

\subsubsection{Image Denoising}

The main application that inspired the development of the bilateral
filtering operation is image denoising~\cite{aurich1995non}, there
using a single Gaussian kernel. Our development allows to learn this
kernel function from data and we explore how to improve using a \emph{single}
but more general bilateral filter.

We use the Berkeley segmentation dataset
(BSDS500)~\cite{arbelaezi2011bsds500} as a test bed. The color
images in the dataset are converted to gray-scale,
and corrupted with Gaussian noise with a standard deviation of
$\frac {25} {255}$.

We compare the performance of four different filter models on a
denoising task.
The first baseline model (`Spatial' in Table \ref{tab:denoising}, $25$
weights) uses a single spatial filter with a kernel size of
$5$ and predicts the scalar gray-scale value at the center pixel. The next model
(`Gauss Bilateral') applies a bilateral \emph{Gaussian}
filter to the noisy input, using position and intensity features $\f=(x,y,v)^\top$.
The third setup (`Learned Bilateral', $65$ weights)
takes a Gauss kernel as initialization and
fits all filter weights on the train set to minimize the
mean squared error with respect to the clean images.
We run a combination
of spatial and permutohedral convolutions on spatial and bilateral
features (`Spatial + Bilateral (Learned)') to check for a complementary
performance of the two convolutions.

\label{sec:exp:denoising}
\begin{table}[!h]
\begin{center}
  \footnotesize
  \begin{tabular}[t]{lr}
    \toprule
    Method & PSNR \\
    \midrule
    Noisy Input & $20.17$ \\
    Spatial & $26.27$ \\
    Gauss Bilateral & $26.51$ \\
    Learned Bilateral & $26.58$ \\
    Spatial + Bilateral (Learned) & \textbf{$26.65$} \\
    \bottomrule
  \end{tabular}
\end{center}
\vspace{-0.5em}
\caption{PSNR results of a denoising task using the BSDS500
  dataset~\cite{arbelaezi2011bsds500}}
\vspace{-0.5em}
\label{tab:denoising}
\end{table}
\vspace{-0.2em}

The PSNR scores evaluated on full images of the test set are
shown in Table \ref{tab:denoising}. We find that an untrained bilateral
filter already performs better than a trained spatial convolution
($26.27$ to $26.51$). A learned convolution further improve the
performance slightly. We chose this simple one-kernel setup to
validate an advantage of the generalized bilateral filter. A competitive
denoising system would employ RGB color information and also
needs to be properly adjusted in network size. Multi-layer perceptrons
have obtained state-of-the-art denoising results~\cite{burger12cvpr}
and the permutohedral lattice layer can readily be used in such an
architecture, which is intended future work.

\subsection{Additional results}
\label{sec:addresults}

This section contains more qualitative results for the experiments presented in Chapter~\ref{chap:bnn}.

\begin{figure*}[th!]
  \centering
    \includegraphics[width=\columnwidth,trim={5cm 2.5cm 5cm 4.5cm},clip]{figures/supplementary/lattice_viz.pdf}
    \vspace{-0.7cm}
  \mycaption{Visualization of the Permutohedral Lattice}
  {Sample lattice visualizations for different feature spaces. All pixels falling in the same simplex cell are shown with
  the same color. $(x,y)$ features correspond to image pixel positions, and $(r,g,b) \in [0,255]$ correspond
  to the red, green and blue color values.}
\label{fig:latticeviz}
\end{figure*}

\subsubsection{Lattice Visualization}

Figure~\ref{fig:latticeviz} shows sample lattice visualizations for different feature spaces.

\newcolumntype{L}[1]{>{\raggedright\let\newline\\\arraybackslash\hspace{0pt}}b{#1}}
\newcolumntype{C}[1]{>{\centering\let\newline\\\arraybackslash\hspace{0pt}}b{#1}}
\newcolumntype{R}[1]{>{\raggedleft\let\newline\\\arraybackslash\hspace{0pt}}b{#1}}

\subsubsection{Color Upsampling}\label{sec:color_upsampling}
\label{sec:col_upsample_extra}

Some images of the upsampling for the Pascal
VOC12 dataset are shown in Fig.~\ref{fig:Colour_upsample_visuals}. It is
especially the low level image details that are better preserved with
a learned bilateral filter compared to the Gaussian case.

\begin{figure*}[t!]
  \centering
    \subfigure{%
   \raisebox{2.0em}{
    \includegraphics[width=.06\columnwidth]{figures/supplementary/2007_004969.jpg}
   }
  }
  \subfigure{%
    \includegraphics[width=.17\columnwidth]{figures/supplementary/2007_004969_gray.pdf}
  }
  \subfigure{%
    \includegraphics[width=.17\columnwidth]{figures/supplementary/2007_004969_gt.pdf}
  }
  \subfigure{%
    \includegraphics[width=.17\columnwidth]{figures/supplementary/2007_004969_bicubic.pdf}
  }
  \subfigure{%
    \includegraphics[width=.17\columnwidth]{figures/supplementary/2007_004969_gauss.pdf}
  }
  \subfigure{%
    \includegraphics[width=.17\columnwidth]{figures/supplementary/2007_004969_learnt.pdf}
  }\\
    \subfigure{%
   \raisebox{2.0em}{
    \includegraphics[width=.06\columnwidth]{figures/supplementary/2007_003106.jpg}
   }
  }
  \subfigure{%
    \includegraphics[width=.17\columnwidth]{figures/supplementary/2007_003106_gray.pdf}
  }
  \subfigure{%
    \includegraphics[width=.17\columnwidth]{figures/supplementary/2007_003106_gt.pdf}
  }
  \subfigure{%
    \includegraphics[width=.17\columnwidth]{figures/supplementary/2007_003106_bicubic.pdf}
  }
  \subfigure{%
    \includegraphics[width=.17\columnwidth]{figures/supplementary/2007_003106_gauss.pdf}
  }
  \subfigure{%
    \includegraphics[width=.17\columnwidth]{figures/supplementary/2007_003106_learnt.pdf}
  }\\
  \setcounter{subfigure}{0}
  \small{
  \subfigure[Inp.]{%
  \raisebox{2.0em}{
    \includegraphics[width=.06\columnwidth]{figures/supplementary/2007_006837.jpg}
   }
  }
  \subfigure[Guidance]{%
    \includegraphics[width=.17\columnwidth]{figures/supplementary/2007_006837_gray.pdf}
  }
   \subfigure[GT]{%
    \includegraphics[width=.17\columnwidth]{figures/supplementary/2007_006837_gt.pdf}
  }
  \subfigure[Bicubic]{%
    \includegraphics[width=.17\columnwidth]{figures/supplementary/2007_006837_bicubic.pdf}
  }
  \subfigure[Gauss-BF]{%
    \includegraphics[width=.17\columnwidth]{figures/supplementary/2007_006837_gauss.pdf}
  }
  \subfigure[Learned-BF]{%
    \includegraphics[width=.17\columnwidth]{figures/supplementary/2007_006837_learnt.pdf}
  }
  }
  \vspace{-0.5cm}
  \mycaption{Color Upsampling}{Color $8\times$ upsampling results
  using different methods, from left to right, (a)~Low-resolution input color image (Inp.),
  (b)~Gray scale guidance image, (c)~Ground-truth color image; Upsampled color images with
  (d)~Bicubic interpolation, (e) Gauss bilateral upsampling and, (f)~Learned bilateral
  updampgling (best viewed on screen).}

\label{fig:Colour_upsample_visuals}
\end{figure*}

\subsubsection{Depth Upsampling}
\label{sec:depth_upsample_extra}

Figure~\ref{fig:depth_upsample_visuals} presents some more qualitative results comparing bicubic interpolation, Gauss
bilateral and learned bilateral upsampling on NYU depth dataset image~\cite{silberman2012indoor}.

\subsubsection{Character Recognition}\label{sec:app_character}

 Figure~\ref{fig:nnrecognition} shows the schematic of different layers
 of the network architecture for LeNet-7~\cite{lecun1998mnist}
 and DeepCNet(5, 50)~\cite{ciresan2012multi,graham2014spatially}. For the BNN variants, the first layer filters are replaced
 with learned bilateral filters and are learned end-to-end.

\subsubsection{Semantic Segmentation}\label{sec:app_semantic_segmentation}
\label{sec:semantic_bnn_extra}

Some more visual results for semantic segmentation are shown in Figure~\ref{fig:semantic_visuals}.
These include the underlying DeepLab CNN\cite{chen2014semantic} result (DeepLab),
the 2 step mean-field result with Gaussian edge potentials (+2stepMF-GaussCRF)
and also corresponding results with learned edge potentials (+2stepMF-LearnedCRF).
In general, we observe that mean-field in learned CRF leads to slightly dilated
classification regions in comparison to using Gaussian CRF thereby filling-in the
false negative pixels and also correcting some mis-classified regions.

\begin{figure*}[t!]
  \centering
    \subfigure{%
   \raisebox{2.0em}{
    \includegraphics[width=.06\columnwidth]{figures/supplementary/2bicubic}
   }
  }
  \subfigure{%
    \includegraphics[width=.17\columnwidth]{figures/supplementary/2given_image}
  }
  \subfigure{%
    \includegraphics[width=.17\columnwidth]{figures/supplementary/2ground_truth}
  }
  \subfigure{%
    \includegraphics[width=.17\columnwidth]{figures/supplementary/2bicubic}
  }
  \subfigure{%
    \includegraphics[width=.17\columnwidth]{figures/supplementary/2gauss}
  }
  \subfigure{%
    \includegraphics[width=.17\columnwidth]{figures/supplementary/2learnt}
  }\\
    \subfigure{%
   \raisebox{2.0em}{
    \includegraphics[width=.06\columnwidth]{figures/supplementary/32bicubic}
   }
  }
  \subfigure{%
    \includegraphics[width=.17\columnwidth]{figures/supplementary/32given_image}
  }
  \subfigure{%
    \includegraphics[width=.17\columnwidth]{figures/supplementary/32ground_truth}
  }
  \subfigure{%
    \includegraphics[width=.17\columnwidth]{figures/supplementary/32bicubic}
  }
  \subfigure{%
    \includegraphics[width=.17\columnwidth]{figures/supplementary/32gauss}
  }
  \subfigure{%
    \includegraphics[width=.17\columnwidth]{figures/supplementary/32learnt}
  }\\
  \setcounter{subfigure}{0}
  \small{
  \subfigure[Inp.]{%
  \raisebox{2.0em}{
    \includegraphics[width=.06\columnwidth]{figures/supplementary/41bicubic}
   }
  }
  \subfigure[Guidance]{%
    \includegraphics[width=.17\columnwidth]{figures/supplementary/41given_image}
  }
   \subfigure[GT]{%
    \includegraphics[width=.17\columnwidth]{figures/supplementary/41ground_truth}
  }
  \subfigure[Bicubic]{%
    \includegraphics[width=.17\columnwidth]{figures/supplementary/41bicubic}
  }
  \subfigure[Gauss-BF]{%
    \includegraphics[width=.17\columnwidth]{figures/supplementary/41gauss}
  }
  \subfigure[Learned-BF]{%
    \includegraphics[width=.17\columnwidth]{figures/supplementary/41learnt}
  }
  }
  \mycaption{Depth Upsampling}{Depth $8\times$ upsampling results
  using different upsampling strategies, from left to right,
  (a)~Low-resolution input depth image (Inp.),
  (b)~High-resolution guidance image, (c)~Ground-truth depth; Upsampled depth images with
  (d)~Bicubic interpolation, (e) Gauss bilateral upsampling and, (f)~Learned bilateral
  updampgling (best viewed on screen).}

\label{fig:depth_upsample_visuals}
\end{figure*}

\subsubsection{Material Segmentation}\label{sec:app_material_segmentation}
\label{sec:material_bnn_extra}

In Fig.~\ref{fig:material_visuals-app2}, we present visual results comparing 2 step
mean-field inference with Gaussian and learned pairwise CRF potentials. In
general, we observe that the pixels belonging to dominant classes in the
training data are being more accurately classified with learned CRF. This leads to
a significant improvements in overall pixel accuracy. This also results
in a slight decrease of the accuracy from less frequent class pixels thereby
slightly reducing the average class accuracy with learning. We attribute this
to the type of annotation that is available for this dataset, which is not
for the entire image but for some segments in the image. We have very few
images of the infrequent classes to combat this behaviour during training.

\subsubsection{Experiment Protocols}
\label{sec:protocols}

Table~\ref{tbl:parameters} shows experiment protocols of different experiments.

 \begin{figure*}[t!]
  \centering
  \subfigure[LeNet-7]{
    \includegraphics[width=0.7\columnwidth]{figures/supplementary/lenet_cnn_network}
    }\\
    \subfigure[DeepCNet]{
    \includegraphics[width=\columnwidth]{figures/supplementary/deepcnet_cnn_network}
    }
  \mycaption{CNNs for Character Recognition}
  {Schematic of (top) LeNet-7~\cite{lecun1998mnist} and (bottom) DeepCNet(5,50)~\cite{ciresan2012multi,graham2014spatially} architectures used in Assamese
  character recognition experiments.}
\label{fig:nnrecognition}
\end{figure*}

\definecolor{voc_1}{RGB}{0, 0, 0}
\definecolor{voc_2}{RGB}{128, 0, 0}
\definecolor{voc_3}{RGB}{0, 128, 0}
\definecolor{voc_4}{RGB}{128, 128, 0}
\definecolor{voc_5}{RGB}{0, 0, 128}
\definecolor{voc_6}{RGB}{128, 0, 128}
\definecolor{voc_7}{RGB}{0, 128, 128}
\definecolor{voc_8}{RGB}{128, 128, 128}
\definecolor{voc_9}{RGB}{64, 0, 0}
\definecolor{voc_10}{RGB}{192, 0, 0}
\definecolor{voc_11}{RGB}{64, 128, 0}
\definecolor{voc_12}{RGB}{192, 128, 0}
\definecolor{voc_13}{RGB}{64, 0, 128}
\definecolor{voc_14}{RGB}{192, 0, 128}
\definecolor{voc_15}{RGB}{64, 128, 128}
\definecolor{voc_16}{RGB}{192, 128, 128}
\definecolor{voc_17}{RGB}{0, 64, 0}
\definecolor{voc_18}{RGB}{128, 64, 0}
\definecolor{voc_19}{RGB}{0, 192, 0}
\definecolor{voc_20}{RGB}{128, 192, 0}
\definecolor{voc_21}{RGB}{0, 64, 128}
\definecolor{voc_22}{RGB}{128, 64, 128}

\begin{figure*}[t]
  \centering
  \small{
  \fcolorbox{white}{voc_1}{\rule{0pt}{6pt}\rule{6pt}{0pt}} Background~~
  \fcolorbox{white}{voc_2}{\rule{0pt}{6pt}\rule{6pt}{0pt}} Aeroplane~~
  \fcolorbox{white}{voc_3}{\rule{0pt}{6pt}\rule{6pt}{0pt}} Bicycle~~
  \fcolorbox{white}{voc_4}{\rule{0pt}{6pt}\rule{6pt}{0pt}} Bird~~
  \fcolorbox{white}{voc_5}{\rule{0pt}{6pt}\rule{6pt}{0pt}} Boat~~
  \fcolorbox{white}{voc_6}{\rule{0pt}{6pt}\rule{6pt}{0pt}} Bottle~~
  \fcolorbox{white}{voc_7}{\rule{0pt}{6pt}\rule{6pt}{0pt}} Bus~~
  \fcolorbox{white}{voc_8}{\rule{0pt}{6pt}\rule{6pt}{0pt}} Car~~ \\
  \fcolorbox{white}{voc_9}{\rule{0pt}{6pt}\rule{6pt}{0pt}} Cat~~
  \fcolorbox{white}{voc_10}{\rule{0pt}{6pt}\rule{6pt}{0pt}} Chair~~
  \fcolorbox{white}{voc_11}{\rule{0pt}{6pt}\rule{6pt}{0pt}} Cow~~
  \fcolorbox{white}{voc_12}{\rule{0pt}{6pt}\rule{6pt}{0pt}} Dining Table~~
  \fcolorbox{white}{voc_13}{\rule{0pt}{6pt}\rule{6pt}{0pt}} Dog~~
  \fcolorbox{white}{voc_14}{\rule{0pt}{6pt}\rule{6pt}{0pt}} Horse~~
  \fcolorbox{white}{voc_15}{\rule{0pt}{6pt}\rule{6pt}{0pt}} Motorbike~~
  \fcolorbox{white}{voc_16}{\rule{0pt}{6pt}\rule{6pt}{0pt}} Person~~ \\
  \fcolorbox{white}{voc_17}{\rule{0pt}{6pt}\rule{6pt}{0pt}} Potted Plant~~
  \fcolorbox{white}{voc_18}{\rule{0pt}{6pt}\rule{6pt}{0pt}} Sheep~~
  \fcolorbox{white}{voc_19}{\rule{0pt}{6pt}\rule{6pt}{0pt}} Sofa~~
  \fcolorbox{white}{voc_20}{\rule{0pt}{6pt}\rule{6pt}{0pt}} Train~~
  \fcolorbox{white}{voc_21}{\rule{0pt}{6pt}\rule{6pt}{0pt}} TV monitor~~ \\
  }
  \subfigure{%
    \includegraphics[width=.18\columnwidth]{figures/supplementary/2007_001423_given.jpg}
  }
  \subfigure{%
    \includegraphics[width=.18\columnwidth]{figures/supplementary/2007_001423_gt.png}
  }
  \subfigure{%
    \includegraphics[width=.18\columnwidth]{figures/supplementary/2007_001423_cnn.png}
  }
  \subfigure{%
    \includegraphics[width=.18\columnwidth]{figures/supplementary/2007_001423_gauss.png}
  }
  \subfigure{%
    \includegraphics[width=.18\columnwidth]{figures/supplementary/2007_001423_learnt.png}
  }\\
  \subfigure{%
    \includegraphics[width=.18\columnwidth]{figures/supplementary/2007_001430_given.jpg}
  }
  \subfigure{%
    \includegraphics[width=.18\columnwidth]{figures/supplementary/2007_001430_gt.png}
  }
  \subfigure{%
    \includegraphics[width=.18\columnwidth]{figures/supplementary/2007_001430_cnn.png}
  }
  \subfigure{%
    \includegraphics[width=.18\columnwidth]{figures/supplementary/2007_001430_gauss.png}
  }
  \subfigure{%
    \includegraphics[width=.18\columnwidth]{figures/supplementary/2007_001430_learnt.png}
  }\\
    \subfigure{%
    \includegraphics[width=.18\columnwidth]{figures/supplementary/2007_007996_given.jpg}
  }
  \subfigure{%
    \includegraphics[width=.18\columnwidth]{figures/supplementary/2007_007996_gt.png}
  }
  \subfigure{%
    \includegraphics[width=.18\columnwidth]{figures/supplementary/2007_007996_cnn.png}
  }
  \subfigure{%
    \includegraphics[width=.18\columnwidth]{figures/supplementary/2007_007996_gauss.png}
  }
  \subfigure{%
    \includegraphics[width=.18\columnwidth]{figures/supplementary/2007_007996_learnt.png}
  }\\
   \subfigure{%
    \includegraphics[width=.18\columnwidth]{figures/supplementary/2010_002682_given.jpg}
  }
  \subfigure{%
    \includegraphics[width=.18\columnwidth]{figures/supplementary/2010_002682_gt.png}
  }
  \subfigure{%
    \includegraphics[width=.18\columnwidth]{figures/supplementary/2010_002682_cnn.png}
  }
  \subfigure{%
    \includegraphics[width=.18\columnwidth]{figures/supplementary/2010_002682_gauss.png}
  }
  \subfigure{%
    \includegraphics[width=.18\columnwidth]{figures/supplementary/2010_002682_learnt.png}
  }\\
     \subfigure{%
    \includegraphics[width=.18\columnwidth]{figures/supplementary/2010_004789_given.jpg}
  }
  \subfigure{%
    \includegraphics[width=.18\columnwidth]{figures/supplementary/2010_004789_gt.png}
  }
  \subfigure{%
    \includegraphics[width=.18\columnwidth]{figures/supplementary/2010_004789_cnn.png}
  }
  \subfigure{%
    \includegraphics[width=.18\columnwidth]{figures/supplementary/2010_004789_gauss.png}
  }
  \subfigure{%
    \includegraphics[width=.18\columnwidth]{figures/supplementary/2010_004789_learnt.png}
  }\\
       \subfigure{%
    \includegraphics[width=.18\columnwidth]{figures/supplementary/2007_001311_given.jpg}
  }
  \subfigure{%
    \includegraphics[width=.18\columnwidth]{figures/supplementary/2007_001311_gt.png}
  }
  \subfigure{%
    \includegraphics[width=.18\columnwidth]{figures/supplementary/2007_001311_cnn.png}
  }
  \subfigure{%
    \includegraphics[width=.18\columnwidth]{figures/supplementary/2007_001311_gauss.png}
  }
  \subfigure{%
    \includegraphics[width=.18\columnwidth]{figures/supplementary/2007_001311_learnt.png}
  }\\
  \setcounter{subfigure}{0}
  \subfigure[Input]{%
    \includegraphics[width=.18\columnwidth]{figures/supplementary/2010_003531_given.jpg}
  }
  \subfigure[Ground Truth]{%
    \includegraphics[width=.18\columnwidth]{figures/supplementary/2010_003531_gt.png}
  }
  \subfigure[DeepLab]{%
    \includegraphics[width=.18\columnwidth]{figures/supplementary/2010_003531_cnn.png}
  }
  \subfigure[+GaussCRF]{%
    \includegraphics[width=.18\columnwidth]{figures/supplementary/2010_003531_gauss.png}
  }
  \subfigure[+LearnedCRF]{%
    \includegraphics[width=.18\columnwidth]{figures/supplementary/2010_003531_learnt.png}
  }
  \vspace{-0.3cm}
  \mycaption{Semantic Segmentation}{Example results of semantic segmentation.
  (c)~depicts the unary results before application of MF, (d)~after two steps of MF with Gaussian edge CRF potentials, (e)~after
  two steps of MF with learned edge CRF potentials.}
    \label{fig:semantic_visuals}
\end{figure*}


\definecolor{minc_1}{HTML}{771111}
\definecolor{minc_2}{HTML}{CAC690}
\definecolor{minc_3}{HTML}{EEEEEE}
\definecolor{minc_4}{HTML}{7C8FA6}
\definecolor{minc_5}{HTML}{597D31}
\definecolor{minc_6}{HTML}{104410}
\definecolor{minc_7}{HTML}{BB819C}
\definecolor{minc_8}{HTML}{D0CE48}
\definecolor{minc_9}{HTML}{622745}
\definecolor{minc_10}{HTML}{666666}
\definecolor{minc_11}{HTML}{D54A31}
\definecolor{minc_12}{HTML}{101044}
\definecolor{minc_13}{HTML}{444126}
\definecolor{minc_14}{HTML}{75D646}
\definecolor{minc_15}{HTML}{DD4348}
\definecolor{minc_16}{HTML}{5C8577}
\definecolor{minc_17}{HTML}{C78472}
\definecolor{minc_18}{HTML}{75D6D0}
\definecolor{minc_19}{HTML}{5B4586}
\definecolor{minc_20}{HTML}{C04393}
\definecolor{minc_21}{HTML}{D69948}
\definecolor{minc_22}{HTML}{7370D8}
\definecolor{minc_23}{HTML}{7A3622}
\definecolor{minc_24}{HTML}{000000}

\begin{figure*}[t]
  \centering
  \small{
  \fcolorbox{white}{minc_1}{\rule{0pt}{6pt}\rule{6pt}{0pt}} Brick~~
  \fcolorbox{white}{minc_2}{\rule{0pt}{6pt}\rule{6pt}{0pt}} Carpet~~
  \fcolorbox{white}{minc_3}{\rule{0pt}{6pt}\rule{6pt}{0pt}} Ceramic~~
  \fcolorbox{white}{minc_4}{\rule{0pt}{6pt}\rule{6pt}{0pt}} Fabric~~
  \fcolorbox{white}{minc_5}{\rule{0pt}{6pt}\rule{6pt}{0pt}} Foliage~~
  \fcolorbox{white}{minc_6}{\rule{0pt}{6pt}\rule{6pt}{0pt}} Food~~
  \fcolorbox{white}{minc_7}{\rule{0pt}{6pt}\rule{6pt}{0pt}} Glass~~
  \fcolorbox{white}{minc_8}{\rule{0pt}{6pt}\rule{6pt}{0pt}} Hair~~ \\
  \fcolorbox{white}{minc_9}{\rule{0pt}{6pt}\rule{6pt}{0pt}} Leather~~
  \fcolorbox{white}{minc_10}{\rule{0pt}{6pt}\rule{6pt}{0pt}} Metal~~
  \fcolorbox{white}{minc_11}{\rule{0pt}{6pt}\rule{6pt}{0pt}} Mirror~~
  \fcolorbox{white}{minc_12}{\rule{0pt}{6pt}\rule{6pt}{0pt}} Other~~
  \fcolorbox{white}{minc_13}{\rule{0pt}{6pt}\rule{6pt}{0pt}} Painted~~
  \fcolorbox{white}{minc_14}{\rule{0pt}{6pt}\rule{6pt}{0pt}} Paper~~
  \fcolorbox{white}{minc_15}{\rule{0pt}{6pt}\rule{6pt}{0pt}} Plastic~~\\
  \fcolorbox{white}{minc_16}{\rule{0pt}{6pt}\rule{6pt}{0pt}} Polished Stone~~
  \fcolorbox{white}{minc_17}{\rule{0pt}{6pt}\rule{6pt}{0pt}} Skin~~
  \fcolorbox{white}{minc_18}{\rule{0pt}{6pt}\rule{6pt}{0pt}} Sky~~
  \fcolorbox{white}{minc_19}{\rule{0pt}{6pt}\rule{6pt}{0pt}} Stone~~
  \fcolorbox{white}{minc_20}{\rule{0pt}{6pt}\rule{6pt}{0pt}} Tile~~
  \fcolorbox{white}{minc_21}{\rule{0pt}{6pt}\rule{6pt}{0pt}} Wallpaper~~
  \fcolorbox{white}{minc_22}{\rule{0pt}{6pt}\rule{6pt}{0pt}} Water~~
  \fcolorbox{white}{minc_23}{\rule{0pt}{6pt}\rule{6pt}{0pt}} Wood~~ \\
  }
  \subfigure{%
    \includegraphics[width=.18\columnwidth]{figures/supplementary/000010868_given.jpg}
  }
  \subfigure{%
    \includegraphics[width=.18\columnwidth]{figures/supplementary/000010868_gt.png}
  }
  \subfigure{%
    \includegraphics[width=.18\columnwidth]{figures/supplementary/000010868_cnn.png}
  }
  \subfigure{%
    \includegraphics[width=.18\columnwidth]{figures/supplementary/000010868_gauss.png}
  }
  \subfigure{%
    \includegraphics[width=.18\columnwidth]{figures/supplementary/000010868_learnt.png}
  }\\[-2ex]
  \subfigure{%
    \includegraphics[width=.18\columnwidth]{figures/supplementary/000006011_given.jpg}
  }
  \subfigure{%
    \includegraphics[width=.18\columnwidth]{figures/supplementary/000006011_gt.png}
  }
  \subfigure{%
    \includegraphics[width=.18\columnwidth]{figures/supplementary/000006011_cnn.png}
  }
  \subfigure{%
    \includegraphics[width=.18\columnwidth]{figures/supplementary/000006011_gauss.png}
  }
  \subfigure{%
    \includegraphics[width=.18\columnwidth]{figures/supplementary/000006011_learnt.png}
  }\\[-2ex]
    \subfigure{%
    \includegraphics[width=.18\columnwidth]{figures/supplementary/000008553_given.jpg}
  }
  \subfigure{%
    \includegraphics[width=.18\columnwidth]{figures/supplementary/000008553_gt.png}
  }
  \subfigure{%
    \includegraphics[width=.18\columnwidth]{figures/supplementary/000008553_cnn.png}
  }
  \subfigure{%
    \includegraphics[width=.18\columnwidth]{figures/supplementary/000008553_gauss.png}
  }
  \subfigure{%
    \includegraphics[width=.18\columnwidth]{figures/supplementary/000008553_learnt.png}
  }\\[-2ex]
   \subfigure{%
    \includegraphics[width=.18\columnwidth]{figures/supplementary/000009188_given.jpg}
  }
  \subfigure{%
    \includegraphics[width=.18\columnwidth]{figures/supplementary/000009188_gt.png}
  }
  \subfigure{%
    \includegraphics[width=.18\columnwidth]{figures/supplementary/000009188_cnn.png}
  }
  \subfigure{%
    \includegraphics[width=.18\columnwidth]{figures/supplementary/000009188_gauss.png}
  }
  \subfigure{%
    \includegraphics[width=.18\columnwidth]{figures/supplementary/000009188_learnt.png}
  }\\[-2ex]
  \setcounter{subfigure}{0}
  \subfigure[Input]{%
    \includegraphics[width=.18\columnwidth]{figures/supplementary/000023570_given.jpg}
  }
  \subfigure[Ground Truth]{%
    \includegraphics[width=.18\columnwidth]{figures/supplementary/000023570_gt.png}
  }
  \subfigure[DeepLab]{%
    \includegraphics[width=.18\columnwidth]{figures/supplementary/000023570_cnn.png}
  }
  \subfigure[+GaussCRF]{%
    \includegraphics[width=.18\columnwidth]{figures/supplementary/000023570_gauss.png}
  }
  \subfigure[+LearnedCRF]{%
    \includegraphics[width=.18\columnwidth]{figures/supplementary/000023570_learnt.png}
  }
  \mycaption{Material Segmentation}{Example results of material segmentation.
  (c)~depicts the unary results before application of MF, (d)~after two steps of MF with Gaussian edge CRF potentials, (e)~after two steps of MF with learned edge CRF potentials.}
    \label{fig:material_visuals-app2}
\end{figure*}


\begin{table*}[h]
\tiny
  \centering
    \begin{tabular}{L{2.3cm} L{2.25cm} C{1.5cm} C{0.7cm} C{0.6cm} C{0.7cm} C{0.7cm} C{0.7cm} C{1.6cm} C{0.6cm} C{0.6cm} C{0.6cm}}
      \toprule
& & & & & \multicolumn{3}{c}{\textbf{Data Statistics}} & \multicolumn{4}{c}{\textbf{Training Protocol}} \\

\textbf{Experiment} & \textbf{Feature Types} & \textbf{Feature Scales} & \textbf{Filter Size} & \textbf{Filter Nbr.} & \textbf{Train}  & \textbf{Val.} & \textbf{Test} & \textbf{Loss Type} & \textbf{LR} & \textbf{Batch} & \textbf{Epochs} \\
      \midrule
      \multicolumn{2}{c}{\textbf{Single Bilateral Filter Applications}} & & & & & & & & & \\
      \textbf{2$\times$ Color Upsampling} & Position$_{1}$, Intensity (3D) & 0.13, 0.17 & 65 & 2 & 10581 & 1449 & 1456 & MSE & 1e-06 & 200 & 94.5\\
      \textbf{4$\times$ Color Upsampling} & Position$_{1}$, Intensity (3D) & 0.06, 0.17 & 65 & 2 & 10581 & 1449 & 1456 & MSE & 1e-06 & 200 & 94.5\\
      \textbf{8$\times$ Color Upsampling} & Position$_{1}$, Intensity (3D) & 0.03, 0.17 & 65 & 2 & 10581 & 1449 & 1456 & MSE & 1e-06 & 200 & 94.5\\
      \textbf{16$\times$ Color Upsampling} & Position$_{1}$, Intensity (3D) & 0.02, 0.17 & 65 & 2 & 10581 & 1449 & 1456 & MSE & 1e-06 & 200 & 94.5\\
      \textbf{Depth Upsampling} & Position$_{1}$, Color (5D) & 0.05, 0.02 & 665 & 2 & 795 & 100 & 654 & MSE & 1e-07 & 50 & 251.6\\
      \textbf{Mesh Denoising} & Isomap (4D) & 46.00 & 63 & 2 & 1000 & 200 & 500 & MSE & 100 & 10 & 100.0 \\
      \midrule
      \multicolumn{2}{c}{\textbf{DenseCRF Applications}} & & & & & & & & &\\
      \multicolumn{2}{l}{\textbf{Semantic Segmentation}} & & & & & & & & &\\
      \textbf{- 1step MF} & Position$_{1}$, Color (5D); Position$_{1}$ (2D) & 0.01, 0.34; 0.34  & 665; 19  & 2; 2 & 10581 & 1449 & 1456 & Logistic & 0.1 & 5 & 1.4 \\
      \textbf{- 2step MF} & Position$_{1}$, Color (5D); Position$_{1}$ (2D) & 0.01, 0.34; 0.34 & 665; 19 & 2; 2 & 10581 & 1449 & 1456 & Logistic & 0.1 & 5 & 1.4 \\
      \textbf{- \textit{loose} 2step MF} & Position$_{1}$, Color (5D); Position$_{1}$ (2D) & 0.01, 0.34; 0.34 & 665; 19 & 2; 2 &10581 & 1449 & 1456 & Logistic & 0.1 & 5 & +1.9  \\ \\
      \multicolumn{2}{l}{\textbf{Material Segmentation}} & & & & & & & & &\\
      \textbf{- 1step MF} & Position$_{2}$, Lab-Color (5D) & 5.00, 0.05, 0.30  & 665 & 2 & 928 & 150 & 1798 & Weighted Logistic & 1e-04 & 24 & 2.6 \\
      \textbf{- 2step MF} & Position$_{2}$, Lab-Color (5D) & 5.00, 0.05, 0.30 & 665 & 2 & 928 & 150 & 1798 & Weighted Logistic & 1e-04 & 12 & +0.7 \\
      \textbf{- \textit{loose} 2step MF} & Position$_{2}$, Lab-Color (5D) & 5.00, 0.05, 0.30 & 665 & 2 & 928 & 150 & 1798 & Weighted Logistic & 1e-04 & 12 & +0.2\\
      \midrule
      \multicolumn{2}{c}{\textbf{Neural Network Applications}} & & & & & & & & &\\
      \textbf{Tiles: CNN-9$\times$9} & - & - & 81 & 4 & 10000 & 1000 & 1000 & Logistic & 0.01 & 100 & 500.0 \\
      \textbf{Tiles: CNN-13$\times$13} & - & - & 169 & 6 & 10000 & 1000 & 1000 & Logistic & 0.01 & 100 & 500.0 \\
      \textbf{Tiles: CNN-17$\times$17} & - & - & 289 & 8 & 10000 & 1000 & 1000 & Logistic & 0.01 & 100 & 500.0 \\
      \textbf{Tiles: CNN-21$\times$21} & - & - & 441 & 10 & 10000 & 1000 & 1000 & Logistic & 0.01 & 100 & 500.0 \\
      \textbf{Tiles: BNN} & Position$_{1}$, Color (5D) & 0.05, 0.04 & 63 & 1 & 10000 & 1000 & 1000 & Logistic & 0.01 & 100 & 30.0 \\
      \textbf{LeNet} & - & - & 25 & 2 & 5490 & 1098 & 1647 & Logistic & 0.1 & 100 & 182.2 \\
      \textbf{Crop-LeNet} & - & - & 25 & 2 & 5490 & 1098 & 1647 & Logistic & 0.1 & 100 & 182.2 \\
      \textbf{BNN-LeNet} & Position$_{2}$ (2D) & 20.00 & 7 & 1 & 5490 & 1098 & 1647 & Logistic & 0.1 & 100 & 182.2 \\
      \textbf{DeepCNet} & - & - & 9 & 1 & 5490 & 1098 & 1647 & Logistic & 0.1 & 100 & 182.2 \\
      \textbf{Crop-DeepCNet} & - & - & 9 & 1 & 5490 & 1098 & 1647 & Logistic & 0.1 & 100 & 182.2 \\
      \textbf{BNN-DeepCNet} & Position$_{2}$ (2D) & 40.00  & 7 & 1 & 5490 & 1098 & 1647 & Logistic & 0.1 & 100 & 182.2 \\
      \bottomrule
      \\
    \end{tabular}
    \mycaption{Experiment Protocols} {Experiment protocols for the different experiments presented in this work. \textbf{Feature Types}:
    Feature spaces used for the bilateral convolutions. Position$_1$ corresponds to un-normalized pixel positions whereas Position$_2$ corresponds
    to pixel positions normalized to $[0,1]$ with respect to the given image. \textbf{Feature Scales}: Cross-validated scales for the features used.
     \textbf{Filter Size}: Number of elements in the filter that is being learned. \textbf{Filter Nbr.}: Half-width of the filter. \textbf{Train},
     \textbf{Val.} and \textbf{Test} corresponds to the number of train, validation and test images used in the experiment. \textbf{Loss Type}: Type
     of loss used for back-propagation. ``MSE'' corresponds to Euclidean mean squared error loss and ``Logistic'' corresponds to multinomial logistic
     loss. ``Weighted Logistic'' is the class-weighted multinomial logistic loss. We weighted the loss with inverse class probability for material
     segmentation task due to the small availability of training data with class imbalance. \textbf{LR}: Fixed learning rate used in stochastic gradient
     descent. \textbf{Batch}: Number of images used in one parameter update step. \textbf{Epochs}: Number of training epochs. In all the experiments,
     we used fixed momentum of 0.9 and weight decay of 0.0005 for stochastic gradient descent. ```Color Upsampling'' experiments in this Table corresponds
     to those performed on Pascal VOC12 dataset images. For all experiments using Pascal VOC12 images, we use extended
     training segmentation dataset available from~\cite{hariharan2011moredata}, and used standard validation and test splits
     from the main dataset~\cite{voc2012segmentation}.}
  \label{tbl:parameters}
\end{table*}

\clearpage

\section{Parameters and Additional Results for Video Propagation Networks}

In this Section, we present experiment protocols and additional qualitative results for experiments
on video object segmentation, semantic video segmentation and video color
propagation. Table~\ref{tbl:parameters_supp} shows the feature scales and other parameters used in different experiments.
Figures~\ref{fig:video_seg_pos_supp} show some qualitative results on video object segmentation
with some failure cases in Fig.~\ref{fig:video_seg_neg_supp}.
Figure~\ref{fig:semantic_visuals_supp} shows some qualitative results on semantic video segmentation and
Fig.~\ref{fig:color_visuals_supp} shows results on video color propagation.

\newcolumntype{L}[1]{>{\raggedright\let\newline\\\arraybackslash\hspace{0pt}}b{#1}}
\newcolumntype{C}[1]{>{\centering\let\newline\\\arraybackslash\hspace{0pt}}b{#1}}
\newcolumntype{R}[1]{>{\raggedleft\let\newline\\\arraybackslash\hspace{0pt}}b{#1}}

\begin{table*}[h]
\tiny
  \centering
    \begin{tabular}{L{3.0cm} L{2.4cm} L{2.8cm} L{2.8cm} C{0.5cm} C{1.0cm} L{1.2cm}}
      \toprule
\textbf{Experiment} & \textbf{Feature Type} & \textbf{Feature Scale-1, $\Lambda_a$} & \textbf{Feature Scale-2, $\Lambda_b$} & \textbf{$\alpha$} & \textbf{Input Frames} & \textbf{Loss Type} \\
      \midrule
      \textbf{Video Object Segmentation} & ($x,y,Y,Cb,Cr,t$) & (0.02,0.02,0.07,0.4,0.4,0.01) & (0.03,0.03,0.09,0.5,0.5,0.2) & 0.5 & 9 & Logistic\\
      \midrule
      \textbf{Semantic Video Segmentation} & & & & & \\
      \textbf{with CNN1~\cite{yu2015multi}-NoFlow} & ($x,y,R,G,B,t$) & (0.08,0.08,0.2,0.2,0.2,0.04) & (0.11,0.11,0.2,0.2,0.2,0.04) & 0.5 & 3 & Logistic \\
      \textbf{with CNN1~\cite{yu2015multi}-Flow} & ($x+u_x,y+u_y,R,G,B,t$) & (0.11,0.11,0.14,0.14,0.14,0.03) & (0.08,0.08,0.12,0.12,0.12,0.01) & 0.65 & 3 & Logistic\\
      \textbf{with CNN2~\cite{richter2016playing}-Flow} & ($x+u_x,y+u_y,R,G,B,t$) & (0.08,0.08,0.2,0.2,0.2,0.04) & (0.09,0.09,0.25,0.25,0.25,0.03) & 0.5 & 4 & Logistic\\
      \midrule
      \textbf{Video Color Propagation} & ($x,y,I,t$)  & (0.04,0.04,0.2,0.04) & No second kernel & 1 & 4 & MSE\\
      \bottomrule
      \\
    \end{tabular}
    \mycaption{Experiment Protocols} {Experiment protocols for the different experiments presented in this work. \textbf{Feature Types}:
    Feature spaces used for the bilateral convolutions, with position ($x,y$) and color
    ($R,G,B$ or $Y,Cb,Cr$) features $\in [0,255]$. $u_x$, $u_y$ denotes optical flow with respect
    to the present frame and $I$ denotes grayscale intensity.
    \textbf{Feature Scales ($\Lambda_a, \Lambda_b$)}: Cross-validated scales for the features used.
    \textbf{$\alpha$}: Exponential time decay for the input frames.
    \textbf{Input Frames}: Number of input frames for VPN.
    \textbf{Loss Type}: Type
     of loss used for back-propagation. ``MSE'' corresponds to Euclidean mean squared error loss and ``Logistic'' corresponds to multinomial logistic loss.}
  \label{tbl:parameters_supp}
\end{table*}

% \begin{figure}[th!]
% \begin{center}
%   \centerline{\includegraphics[width=\textwidth]{figures/video_seg_visuals_supp_small.pdf}}
%     \mycaption{Video Object Segmentation}
%     {Shown are the different frames in example videos with the corresponding
%     ground truth (GT) masks, predictions from BVS~\cite{marki2016bilateral},
%     OFL~\cite{tsaivideo}, VPN (VPN-Stage2) and VPN-DLab (VPN-DeepLab) models.}
%     \label{fig:video_seg_small_supp}
% \end{center}
% \vspace{-1.0cm}
% \end{figure}

\begin{figure}[th!]
\begin{center}
  \centerline{\includegraphics[width=0.7\textwidth]{figures/video_seg_visuals_supp_positive.pdf}}
    \mycaption{Video Object Segmentation}
    {Shown are the different frames in example videos with the corresponding
    ground truth (GT) masks, predictions from BVS~\cite{marki2016bilateral},
    OFL~\cite{tsaivideo}, VPN (VPN-Stage2) and VPN-DLab (VPN-DeepLab) models.}
    \label{fig:video_seg_pos_supp}
\end{center}
\vspace{-1.0cm}
\end{figure}

\begin{figure}[th!]
\begin{center}
  \centerline{\includegraphics[width=0.7\textwidth]{figures/video_seg_visuals_supp_negative.pdf}}
    \mycaption{Failure Cases for Video Object Segmentation}
    {Shown are the different frames in example videos with the corresponding
    ground truth (GT) masks, predictions from BVS~\cite{marki2016bilateral},
    OFL~\cite{tsaivideo}, VPN (VPN-Stage2) and VPN-DLab (VPN-DeepLab) models.}
    \label{fig:video_seg_neg_supp}
\end{center}
\vspace{-1.0cm}
\end{figure}

\begin{figure}[th!]
\begin{center}
  \centerline{\includegraphics[width=0.9\textwidth]{figures/supp_semantic_visual.pdf}}
    \mycaption{Semantic Video Segmentation}
    {Input video frames and the corresponding ground truth (GT)
    segmentation together with the predictions of CNN~\cite{yu2015multi} and with
    VPN-Flow.}
    \label{fig:semantic_visuals_supp}
\end{center}
\vspace{-0.7cm}
\end{figure}

\begin{figure}[th!]
\begin{center}
  \centerline{\includegraphics[width=\textwidth]{figures/colorization_visuals_supp.pdf}}
  \mycaption{Video Color Propagation}
  {Input grayscale video frames and corresponding ground-truth (GT) color images
  together with color predictions of Levin et al.~\cite{levin2004colorization} and VPN-Stage1 models.}
  \label{fig:color_visuals_supp}
\end{center}
\vspace{-0.7cm}
\end{figure}

\clearpage

\section{Additional Material for Bilateral Inception Networks}
\label{sec:binception-app}

In this section of the Appendix, we first discuss the use of approximate bilateral
filtering in BI modules (Sec.~\ref{sec:lattice}).
Later, we present some qualitative results using different models for the approach presented in
Chapter~\ref{chap:binception} (Sec.~\ref{sec:qualitative-app}).

\subsection{Approximate Bilateral Filtering}
\label{sec:lattice}

The bilateral inception module presented in Chapter~\ref{chap:binception} computes a matrix-vector
product between a Gaussian filter $K$ and a vector of activations $\bz_c$.
Bilateral filtering is an important operation and many algorithmic techniques have been
proposed to speed-up this operation~\cite{paris2006fast,adams2010fast,gastal2011domain}.
In the main paper we opted to implement what can be considered the
brute-force variant of explicitly constructing $K$ and then using BLAS to compute the
matrix-vector product. This resulted in a few millisecond operation.
The explicit way to compute is possible due to the
reduction to super-pixels, e.g., it would not work for DenseCRF variants
that operate on the full image resolution.

Here, we present experiments where we use the fast approximate bilateral filtering
algorithm of~\cite{adams2010fast}, which is also used in Chapter~\ref{chap:bnn}
for learning sparse high dimensional filters. This
choice allows for larger dimensions of matrix-vector multiplication. The reason for choosing
the explicit multiplication in Chapter~\ref{chap:binception} was that it was computationally faster.
For the small sizes of the involved matrices and vectors, the explicit computation is sufficient and we had no
GPU implementation of an approximate technique that matched this runtime. Also it
is conceptually easier and the gradient to the feature transformations ($\Lambda \mathbf{f}$) is
obtained using standard matrix calculus.

\subsubsection{Experiments}

We modified the existing segmentation architectures analogous to those in Chapter~\ref{chap:binception}.
The main difference is that, here, the inception modules use the lattice
approximation~\cite{adams2010fast} to compute the bilateral filtering.
Using the lattice approximation did not allow us to back-propagate through feature transformations ($\Lambda$)
and thus we used hand-specified feature scales as will be explained later.
Specifically, we take CNN architectures from the works
of~\cite{chen2014semantic,zheng2015conditional,bell2015minc} and insert the BI modules between
the spatial FC layers.
We use superpixels from~\cite{DollarICCV13edges}
for all the experiments with the lattice approximation. Experiments are
performed using Caffe neural network framework~\cite{jia2014caffe}.

\begin{table}
  \small
  \centering
  \begin{tabular}{p{5.5cm}>{\raggedright\arraybackslash}p{1.4cm}>{\centering\arraybackslash}p{2.2cm}}
    \toprule
		\textbf{Model} & \emph{IoU} & \emph{Runtime}(ms) \\
    \midrule

    %%%%%%%%%%%% Scores computed by us)%%%%%%%%%%%%
		\deeplablargefov & 68.9 & 145ms\\
    \midrule
    \bi{7}{2}-\bi{8}{10}& \textbf{73.8} & +600 \\
    \midrule
    \deeplablargefovcrf~\cite{chen2014semantic} & 72.7 & +830\\
    \deeplabmsclargefovcrf~\cite{chen2014semantic} & \textbf{73.6} & +880\\
    DeepLab-EdgeNet~\cite{chen2015semantic} & 71.7 & +30\\
    DeepLab-EdgeNet-CRF~\cite{chen2015semantic} & \textbf{73.6} & +860\\
  \bottomrule \\
  \end{tabular}
  \mycaption{Semantic Segmentation using the DeepLab model}
  {IoU scores on the Pascal VOC12 segmentation test dataset
  with different models and our modified inception model.
  Also shown are the corresponding runtimes in milliseconds. Runtimes
  also include superpixel computations (300 ms with Dollar superpixels~\cite{DollarICCV13edges})}
  \label{tab:largefovresults}
\end{table}

\paragraph{Semantic Segmentation}
The experiments in this section use the Pascal VOC12 segmentation dataset~\cite{voc2012segmentation} with 21 object classes and the images have a maximum resolution of 0.25 megapixels.
For all experiments on VOC12, we train using the extended training set of
10581 images collected by~\cite{hariharan2011moredata}.
We modified the \deeplab~network architecture of~\cite{chen2014semantic} and
the CRFasRNN architecture from~\cite{zheng2015conditional} which uses a CNN with
deconvolution layers followed by DenseCRF trained end-to-end.

\paragraph{DeepLab Model}\label{sec:deeplabmodel}
We experimented with the \bi{7}{2}-\bi{8}{10} inception model.
Results using the~\deeplab~model are summarized in Tab.~\ref{tab:largefovresults}.
Although we get similar improvements with inception modules as with the
explicit kernel computation, using lattice approximation is slower.

\begin{table}
  \small
  \centering
  \begin{tabular}{p{6.4cm}>{\raggedright\arraybackslash}p{1.8cm}>{\raggedright\arraybackslash}p{1.8cm}}
    \toprule
    \textbf{Model} & \emph{IoU (Val)} & \emph{IoU (Test)}\\
    \midrule
    %%%%%%%%%%%% Scores computed by us)%%%%%%%%%%%%
    CNN &  67.5 & - \\
    \deconv (CNN+Deconvolutions) & 69.8 & 72.0 \\
    \midrule
    \bi{3}{6}-\bi{4}{6}-\bi{7}{2}-\bi{8}{6}& 71.9 & - \\
    \bi{3}{6}-\bi{4}{6}-\bi{7}{2}-\bi{8}{6}-\gi{6}& 73.6 &  \href{http://host.robots.ox.ac.uk:8080/anonymous/VOTV5E.html}{\textbf{75.2}}\\
    \midrule
    \deconvcrf (CRF-RNN)~\cite{zheng2015conditional} & 73.0 & 74.7\\
    Context-CRF-RNN~\cite{yu2015multi} & ~~ - ~ & \textbf{75.3} \\
    \bottomrule \\
  \end{tabular}
  \mycaption{Semantic Segmentation using the CRFasRNN model}{IoU score corresponding to different models
  on Pascal VOC12 reduced validation / test segmentation dataset. The reduced validation set consists of 346 images
  as used in~\cite{zheng2015conditional} where we adapted the model from.}
  \label{tab:deconvresults-app}
\end{table}

\paragraph{CRFasRNN Model}\label{sec:deepinception}
We add BI modules after score-pool3, score-pool4, \fc{7} and \fc{8} $1\times1$ convolution layers
resulting in the \bi{3}{6}-\bi{4}{6}-\bi{7}{2}-\bi{8}{6}
model and also experimented with another variant where $BI_8$ is followed by another inception
module, G$(6)$, with 6 Gaussian kernels.
Note that here also we discarded both deconvolution and DenseCRF parts of the original model~\cite{zheng2015conditional}
and inserted the BI modules in the base CNN and found similar improvements compared to the inception modules with explicit
kernel computaion. See Tab.~\ref{tab:deconvresults-app} for results on the CRFasRNN model.

\paragraph{Material Segmentation}
Table~\ref{tab:mincresults-app} shows the results on the MINC dataset~\cite{bell2015minc}
obtained by modifying the AlexNet architecture with our inception modules. We observe
similar improvements as with explicit kernel construction.
For this model, we do not provide any learned setup due to very limited segment training
data. The weights to combine outputs in the bilateral inception layer are
found by validation on the validation set.

\begin{table}[t]
  \small
  \centering
  \begin{tabular}{p{3.5cm}>{\centering\arraybackslash}p{4.0cm}}
    \toprule
    \textbf{Model} & Class / Total accuracy\\
    \midrule

    %%%%%%%%%%%% Scores computed by us)%%%%%%%%%%%%
    AlexNet CNN & 55.3 / 58.9 \\
    \midrule
    \bi{7}{2}-\bi{8}{6}& 68.5 / 71.8 \\
    \bi{7}{2}-\bi{8}{6}-G$(6)$& 67.6 / 73.1 \\
    \midrule
    AlexNet-CRF & 65.5 / 71.0 \\
    \bottomrule \\
  \end{tabular}
  \mycaption{Material Segmentation using AlexNet}{Pixel accuracy of different models on
  the MINC material segmentation test dataset~\cite{bell2015minc}.}
  \label{tab:mincresults-app}
\end{table}

\paragraph{Scales of Bilateral Inception Modules}
\label{sec:scales}

Unlike the explicit kernel technique presented in the main text (Chapter~\ref{chap:binception}),
we didn't back-propagate through feature transformation ($\Lambda$)
using the approximate bilateral filter technique.
So, the feature scales are hand-specified and validated, which are as follows.
The optimal scale values for the \bi{7}{2}-\bi{8}{2} model are found by validation for the best performance which are
$\sigma_{xy}$ = (0.1, 0.1) for the spatial (XY) kernel and $\sigma_{rgbxy}$ = (0.1, 0.1, 0.1, 0.01, 0.01) for color and position (RGBXY)  kernel.
Next, as more kernels are added to \bi{8}{2}, we set scales to be $\alpha$*($\sigma_{xy}$, $\sigma_{rgbxy}$).
The value of $\alpha$ is chosen as  1, 0.5, 0.1, 0.05, 0.1, at uniform interval, for the \bi{8}{10} bilateral inception module.


\subsection{Qualitative Results}
\label{sec:qualitative-app}

In this section, we present more qualitative results obtained using the BI module with explicit
kernel computation technique presented in Chapter~\ref{chap:binception}. Results on the Pascal VOC12
dataset~\cite{voc2012segmentation} using the DeepLab-LargeFOV model are shown in Fig.~\ref{fig:semantic_visuals-app},
followed by the results on MINC dataset~\cite{bell2015minc}
in Fig.~\ref{fig:material_visuals-app} and on
Cityscapes dataset~\cite{Cordts2015Cvprw} in Fig.~\ref{fig:street_visuals-app}.


\definecolor{voc_1}{RGB}{0, 0, 0}
\definecolor{voc_2}{RGB}{128, 0, 0}
\definecolor{voc_3}{RGB}{0, 128, 0}
\definecolor{voc_4}{RGB}{128, 128, 0}
\definecolor{voc_5}{RGB}{0, 0, 128}
\definecolor{voc_6}{RGB}{128, 0, 128}
\definecolor{voc_7}{RGB}{0, 128, 128}
\definecolor{voc_8}{RGB}{128, 128, 128}
\definecolor{voc_9}{RGB}{64, 0, 0}
\definecolor{voc_10}{RGB}{192, 0, 0}
\definecolor{voc_11}{RGB}{64, 128, 0}
\definecolor{voc_12}{RGB}{192, 128, 0}
\definecolor{voc_13}{RGB}{64, 0, 128}
\definecolor{voc_14}{RGB}{192, 0, 128}
\definecolor{voc_15}{RGB}{64, 128, 128}
\definecolor{voc_16}{RGB}{192, 128, 128}
\definecolor{voc_17}{RGB}{0, 64, 0}
\definecolor{voc_18}{RGB}{128, 64, 0}
\definecolor{voc_19}{RGB}{0, 192, 0}
\definecolor{voc_20}{RGB}{128, 192, 0}
\definecolor{voc_21}{RGB}{0, 64, 128}
\definecolor{voc_22}{RGB}{128, 64, 128}

\begin{figure*}[!ht]
  \small
  \centering
  \fcolorbox{white}{voc_1}{\rule{0pt}{4pt}\rule{4pt}{0pt}} Background~~
  \fcolorbox{white}{voc_2}{\rule{0pt}{4pt}\rule{4pt}{0pt}} Aeroplane~~
  \fcolorbox{white}{voc_3}{\rule{0pt}{4pt}\rule{4pt}{0pt}} Bicycle~~
  \fcolorbox{white}{voc_4}{\rule{0pt}{4pt}\rule{4pt}{0pt}} Bird~~
  \fcolorbox{white}{voc_5}{\rule{0pt}{4pt}\rule{4pt}{0pt}} Boat~~
  \fcolorbox{white}{voc_6}{\rule{0pt}{4pt}\rule{4pt}{0pt}} Bottle~~
  \fcolorbox{white}{voc_7}{\rule{0pt}{4pt}\rule{4pt}{0pt}} Bus~~
  \fcolorbox{white}{voc_8}{\rule{0pt}{4pt}\rule{4pt}{0pt}} Car~~\\
  \fcolorbox{white}{voc_9}{\rule{0pt}{4pt}\rule{4pt}{0pt}} Cat~~
  \fcolorbox{white}{voc_10}{\rule{0pt}{4pt}\rule{4pt}{0pt}} Chair~~
  \fcolorbox{white}{voc_11}{\rule{0pt}{4pt}\rule{4pt}{0pt}} Cow~~
  \fcolorbox{white}{voc_12}{\rule{0pt}{4pt}\rule{4pt}{0pt}} Dining Table~~
  \fcolorbox{white}{voc_13}{\rule{0pt}{4pt}\rule{4pt}{0pt}} Dog~~
  \fcolorbox{white}{voc_14}{\rule{0pt}{4pt}\rule{4pt}{0pt}} Horse~~
  \fcolorbox{white}{voc_15}{\rule{0pt}{4pt}\rule{4pt}{0pt}} Motorbike~~
  \fcolorbox{white}{voc_16}{\rule{0pt}{4pt}\rule{4pt}{0pt}} Person~~\\
  \fcolorbox{white}{voc_17}{\rule{0pt}{4pt}\rule{4pt}{0pt}} Potted Plant~~
  \fcolorbox{white}{voc_18}{\rule{0pt}{4pt}\rule{4pt}{0pt}} Sheep~~
  \fcolorbox{white}{voc_19}{\rule{0pt}{4pt}\rule{4pt}{0pt}} Sofa~~
  \fcolorbox{white}{voc_20}{\rule{0pt}{4pt}\rule{4pt}{0pt}} Train~~
  \fcolorbox{white}{voc_21}{\rule{0pt}{4pt}\rule{4pt}{0pt}} TV monitor~~\\


  \subfigure{%
    \includegraphics[width=.15\columnwidth]{figures/supplementary/2008_001308_given.png}
  }
  \subfigure{%
    \includegraphics[width=.15\columnwidth]{figures/supplementary/2008_001308_sp.png}
  }
  \subfigure{%
    \includegraphics[width=.15\columnwidth]{figures/supplementary/2008_001308_gt.png}
  }
  \subfigure{%
    \includegraphics[width=.15\columnwidth]{figures/supplementary/2008_001308_cnn.png}
  }
  \subfigure{%
    \includegraphics[width=.15\columnwidth]{figures/supplementary/2008_001308_crf.png}
  }
  \subfigure{%
    \includegraphics[width=.15\columnwidth]{figures/supplementary/2008_001308_ours.png}
  }\\[-2ex]


  \subfigure{%
    \includegraphics[width=.15\columnwidth]{figures/supplementary/2008_001821_given.png}
  }
  \subfigure{%
    \includegraphics[width=.15\columnwidth]{figures/supplementary/2008_001821_sp.png}
  }
  \subfigure{%
    \includegraphics[width=.15\columnwidth]{figures/supplementary/2008_001821_gt.png}
  }
  \subfigure{%
    \includegraphics[width=.15\columnwidth]{figures/supplementary/2008_001821_cnn.png}
  }
  \subfigure{%
    \includegraphics[width=.15\columnwidth]{figures/supplementary/2008_001821_crf.png}
  }
  \subfigure{%
    \includegraphics[width=.15\columnwidth]{figures/supplementary/2008_001821_ours.png}
  }\\[-2ex]



  \subfigure{%
    \includegraphics[width=.15\columnwidth]{figures/supplementary/2008_004612_given.png}
  }
  \subfigure{%
    \includegraphics[width=.15\columnwidth]{figures/supplementary/2008_004612_sp.png}
  }
  \subfigure{%
    \includegraphics[width=.15\columnwidth]{figures/supplementary/2008_004612_gt.png}
  }
  \subfigure{%
    \includegraphics[width=.15\columnwidth]{figures/supplementary/2008_004612_cnn.png}
  }
  \subfigure{%
    \includegraphics[width=.15\columnwidth]{figures/supplementary/2008_004612_crf.png}
  }
  \subfigure{%
    \includegraphics[width=.15\columnwidth]{figures/supplementary/2008_004612_ours.png}
  }\\[-2ex]


  \subfigure{%
    \includegraphics[width=.15\columnwidth]{figures/supplementary/2009_001008_given.png}
  }
  \subfigure{%
    \includegraphics[width=.15\columnwidth]{figures/supplementary/2009_001008_sp.png}
  }
  \subfigure{%
    \includegraphics[width=.15\columnwidth]{figures/supplementary/2009_001008_gt.png}
  }
  \subfigure{%
    \includegraphics[width=.15\columnwidth]{figures/supplementary/2009_001008_cnn.png}
  }
  \subfigure{%
    \includegraphics[width=.15\columnwidth]{figures/supplementary/2009_001008_crf.png}
  }
  \subfigure{%
    \includegraphics[width=.15\columnwidth]{figures/supplementary/2009_001008_ours.png}
  }\\[-2ex]




  \subfigure{%
    \includegraphics[width=.15\columnwidth]{figures/supplementary/2009_004497_given.png}
  }
  \subfigure{%
    \includegraphics[width=.15\columnwidth]{figures/supplementary/2009_004497_sp.png}
  }
  \subfigure{%
    \includegraphics[width=.15\columnwidth]{figures/supplementary/2009_004497_gt.png}
  }
  \subfigure{%
    \includegraphics[width=.15\columnwidth]{figures/supplementary/2009_004497_cnn.png}
  }
  \subfigure{%
    \includegraphics[width=.15\columnwidth]{figures/supplementary/2009_004497_crf.png}
  }
  \subfigure{%
    \includegraphics[width=.15\columnwidth]{figures/supplementary/2009_004497_ours.png}
  }\\[-2ex]



  \setcounter{subfigure}{0}
  \subfigure[\scriptsize Input]{%
    \includegraphics[width=.15\columnwidth]{figures/supplementary/2010_001327_given.png}
  }
  \subfigure[\scriptsize Superpixels]{%
    \includegraphics[width=.15\columnwidth]{figures/supplementary/2010_001327_sp.png}
  }
  \subfigure[\scriptsize GT]{%
    \includegraphics[width=.15\columnwidth]{figures/supplementary/2010_001327_gt.png}
  }
  \subfigure[\scriptsize Deeplab]{%
    \includegraphics[width=.15\columnwidth]{figures/supplementary/2010_001327_cnn.png}
  }
  \subfigure[\scriptsize +DenseCRF]{%
    \includegraphics[width=.15\columnwidth]{figures/supplementary/2010_001327_crf.png}
  }
  \subfigure[\scriptsize Using BI]{%
    \includegraphics[width=.15\columnwidth]{figures/supplementary/2010_001327_ours.png}
  }
  \mycaption{Semantic Segmentation}{Example results of semantic segmentation
  on the Pascal VOC12 dataset.
  (d)~depicts the DeepLab CNN result, (e)~CNN + 10 steps of mean-field inference,
  (f~result obtained with bilateral inception (BI) modules (\bi{6}{2}+\bi{7}{6}) between \fc~layers.}
  \label{fig:semantic_visuals-app}
\end{figure*}


\definecolor{minc_1}{HTML}{771111}
\definecolor{minc_2}{HTML}{CAC690}
\definecolor{minc_3}{HTML}{EEEEEE}
\definecolor{minc_4}{HTML}{7C8FA6}
\definecolor{minc_5}{HTML}{597D31}
\definecolor{minc_6}{HTML}{104410}
\definecolor{minc_7}{HTML}{BB819C}
\definecolor{minc_8}{HTML}{D0CE48}
\definecolor{minc_9}{HTML}{622745}
\definecolor{minc_10}{HTML}{666666}
\definecolor{minc_11}{HTML}{D54A31}
\definecolor{minc_12}{HTML}{101044}
\definecolor{minc_13}{HTML}{444126}
\definecolor{minc_14}{HTML}{75D646}
\definecolor{minc_15}{HTML}{DD4348}
\definecolor{minc_16}{HTML}{5C8577}
\definecolor{minc_17}{HTML}{C78472}
\definecolor{minc_18}{HTML}{75D6D0}
\definecolor{minc_19}{HTML}{5B4586}
\definecolor{minc_20}{HTML}{C04393}
\definecolor{minc_21}{HTML}{D69948}
\definecolor{minc_22}{HTML}{7370D8}
\definecolor{minc_23}{HTML}{7A3622}
\definecolor{minc_24}{HTML}{000000}

\begin{figure*}[!ht]
  \small % scriptsize
  \centering
  \fcolorbox{white}{minc_1}{\rule{0pt}{4pt}\rule{4pt}{0pt}} Brick~~
  \fcolorbox{white}{minc_2}{\rule{0pt}{4pt}\rule{4pt}{0pt}} Carpet~~
  \fcolorbox{white}{minc_3}{\rule{0pt}{4pt}\rule{4pt}{0pt}} Ceramic~~
  \fcolorbox{white}{minc_4}{\rule{0pt}{4pt}\rule{4pt}{0pt}} Fabric~~
  \fcolorbox{white}{minc_5}{\rule{0pt}{4pt}\rule{4pt}{0pt}} Foliage~~
  \fcolorbox{white}{minc_6}{\rule{0pt}{4pt}\rule{4pt}{0pt}} Food~~
  \fcolorbox{white}{minc_7}{\rule{0pt}{4pt}\rule{4pt}{0pt}} Glass~~
  \fcolorbox{white}{minc_8}{\rule{0pt}{4pt}\rule{4pt}{0pt}} Hair~~\\
  \fcolorbox{white}{minc_9}{\rule{0pt}{4pt}\rule{4pt}{0pt}} Leather~~
  \fcolorbox{white}{minc_10}{\rule{0pt}{4pt}\rule{4pt}{0pt}} Metal~~
  \fcolorbox{white}{minc_11}{\rule{0pt}{4pt}\rule{4pt}{0pt}} Mirror~~
  \fcolorbox{white}{minc_12}{\rule{0pt}{4pt}\rule{4pt}{0pt}} Other~~
  \fcolorbox{white}{minc_13}{\rule{0pt}{4pt}\rule{4pt}{0pt}} Painted~~
  \fcolorbox{white}{minc_14}{\rule{0pt}{4pt}\rule{4pt}{0pt}} Paper~~
  \fcolorbox{white}{minc_15}{\rule{0pt}{4pt}\rule{4pt}{0pt}} Plastic~~\\
  \fcolorbox{white}{minc_16}{\rule{0pt}{4pt}\rule{4pt}{0pt}} Polished Stone~~
  \fcolorbox{white}{minc_17}{\rule{0pt}{4pt}\rule{4pt}{0pt}} Skin~~
  \fcolorbox{white}{minc_18}{\rule{0pt}{4pt}\rule{4pt}{0pt}} Sky~~
  \fcolorbox{white}{minc_19}{\rule{0pt}{4pt}\rule{4pt}{0pt}} Stone~~
  \fcolorbox{white}{minc_20}{\rule{0pt}{4pt}\rule{4pt}{0pt}} Tile~~
  \fcolorbox{white}{minc_21}{\rule{0pt}{4pt}\rule{4pt}{0pt}} Wallpaper~~
  \fcolorbox{white}{minc_22}{\rule{0pt}{4pt}\rule{4pt}{0pt}} Water~~
  \fcolorbox{white}{minc_23}{\rule{0pt}{4pt}\rule{4pt}{0pt}} Wood~~\\
  \subfigure{%
    \includegraphics[width=.15\columnwidth]{figures/supplementary/000008468_given.png}
  }
  \subfigure{%
    \includegraphics[width=.15\columnwidth]{figures/supplementary/000008468_sp.png}
  }
  \subfigure{%
    \includegraphics[width=.15\columnwidth]{figures/supplementary/000008468_gt.png}
  }
  \subfigure{%
    \includegraphics[width=.15\columnwidth]{figures/supplementary/000008468_cnn.png}
  }
  \subfigure{%
    \includegraphics[width=.15\columnwidth]{figures/supplementary/000008468_crf.png}
  }
  \subfigure{%
    \includegraphics[width=.15\columnwidth]{figures/supplementary/000008468_ours.png}
  }\\[-2ex]

  \subfigure{%
    \includegraphics[width=.15\columnwidth]{figures/supplementary/000009053_given.png}
  }
  \subfigure{%
    \includegraphics[width=.15\columnwidth]{figures/supplementary/000009053_sp.png}
  }
  \subfigure{%
    \includegraphics[width=.15\columnwidth]{figures/supplementary/000009053_gt.png}
  }
  \subfigure{%
    \includegraphics[width=.15\columnwidth]{figures/supplementary/000009053_cnn.png}
  }
  \subfigure{%
    \includegraphics[width=.15\columnwidth]{figures/supplementary/000009053_crf.png}
  }
  \subfigure{%
    \includegraphics[width=.15\columnwidth]{figures/supplementary/000009053_ours.png}
  }\\[-2ex]




  \subfigure{%
    \includegraphics[width=.15\columnwidth]{figures/supplementary/000014977_given.png}
  }
  \subfigure{%
    \includegraphics[width=.15\columnwidth]{figures/supplementary/000014977_sp.png}
  }
  \subfigure{%
    \includegraphics[width=.15\columnwidth]{figures/supplementary/000014977_gt.png}
  }
  \subfigure{%
    \includegraphics[width=.15\columnwidth]{figures/supplementary/000014977_cnn.png}
  }
  \subfigure{%
    \includegraphics[width=.15\columnwidth]{figures/supplementary/000014977_crf.png}
  }
  \subfigure{%
    \includegraphics[width=.15\columnwidth]{figures/supplementary/000014977_ours.png}
  }\\[-2ex]


  \subfigure{%
    \includegraphics[width=.15\columnwidth]{figures/supplementary/000022922_given.png}
  }
  \subfigure{%
    \includegraphics[width=.15\columnwidth]{figures/supplementary/000022922_sp.png}
  }
  \subfigure{%
    \includegraphics[width=.15\columnwidth]{figures/supplementary/000022922_gt.png}
  }
  \subfigure{%
    \includegraphics[width=.15\columnwidth]{figures/supplementary/000022922_cnn.png}
  }
  \subfigure{%
    \includegraphics[width=.15\columnwidth]{figures/supplementary/000022922_crf.png}
  }
  \subfigure{%
    \includegraphics[width=.15\columnwidth]{figures/supplementary/000022922_ours.png}
  }\\[-2ex]


  \subfigure{%
    \includegraphics[width=.15\columnwidth]{figures/supplementary/000025711_given.png}
  }
  \subfigure{%
    \includegraphics[width=.15\columnwidth]{figures/supplementary/000025711_sp.png}
  }
  \subfigure{%
    \includegraphics[width=.15\columnwidth]{figures/supplementary/000025711_gt.png}
  }
  \subfigure{%
    \includegraphics[width=.15\columnwidth]{figures/supplementary/000025711_cnn.png}
  }
  \subfigure{%
    \includegraphics[width=.15\columnwidth]{figures/supplementary/000025711_crf.png}
  }
  \subfigure{%
    \includegraphics[width=.15\columnwidth]{figures/supplementary/000025711_ours.png}
  }\\[-2ex]


  \subfigure{%
    \includegraphics[width=.15\columnwidth]{figures/supplementary/000034473_given.png}
  }
  \subfigure{%
    \includegraphics[width=.15\columnwidth]{figures/supplementary/000034473_sp.png}
  }
  \subfigure{%
    \includegraphics[width=.15\columnwidth]{figures/supplementary/000034473_gt.png}
  }
  \subfigure{%
    \includegraphics[width=.15\columnwidth]{figures/supplementary/000034473_cnn.png}
  }
  \subfigure{%
    \includegraphics[width=.15\columnwidth]{figures/supplementary/000034473_crf.png}
  }
  \subfigure{%
    \includegraphics[width=.15\columnwidth]{figures/supplementary/000034473_ours.png}
  }\\[-2ex]


  \subfigure{%
    \includegraphics[width=.15\columnwidth]{figures/supplementary/000035463_given.png}
  }
  \subfigure{%
    \includegraphics[width=.15\columnwidth]{figures/supplementary/000035463_sp.png}
  }
  \subfigure{%
    \includegraphics[width=.15\columnwidth]{figures/supplementary/000035463_gt.png}
  }
  \subfigure{%
    \includegraphics[width=.15\columnwidth]{figures/supplementary/000035463_cnn.png}
  }
  \subfigure{%
    \includegraphics[width=.15\columnwidth]{figures/supplementary/000035463_crf.png}
  }
  \subfigure{%
    \includegraphics[width=.15\columnwidth]{figures/supplementary/000035463_ours.png}
  }\\[-2ex]


  \setcounter{subfigure}{0}
  \subfigure[\scriptsize Input]{%
    \includegraphics[width=.15\columnwidth]{figures/supplementary/000035993_given.png}
  }
  \subfigure[\scriptsize Superpixels]{%
    \includegraphics[width=.15\columnwidth]{figures/supplementary/000035993_sp.png}
  }
  \subfigure[\scriptsize GT]{%
    \includegraphics[width=.15\columnwidth]{figures/supplementary/000035993_gt.png}
  }
  \subfigure[\scriptsize AlexNet]{%
    \includegraphics[width=.15\columnwidth]{figures/supplementary/000035993_cnn.png}
  }
  \subfigure[\scriptsize +DenseCRF]{%
    \includegraphics[width=.15\columnwidth]{figures/supplementary/000035993_crf.png}
  }
  \subfigure[\scriptsize Using BI]{%
    \includegraphics[width=.15\columnwidth]{figures/supplementary/000035993_ours.png}
  }
  \mycaption{Material Segmentation}{Example results of material segmentation.
  (d)~depicts the AlexNet CNN result, (e)~CNN + 10 steps of mean-field inference,
  (f)~result obtained with bilateral inception (BI) modules (\bi{7}{2}+\bi{8}{6}) between
  \fc~layers.}
\label{fig:material_visuals-app}
\end{figure*}


\definecolor{city_1}{RGB}{128, 64, 128}
\definecolor{city_2}{RGB}{244, 35, 232}
\definecolor{city_3}{RGB}{70, 70, 70}
\definecolor{city_4}{RGB}{102, 102, 156}
\definecolor{city_5}{RGB}{190, 153, 153}
\definecolor{city_6}{RGB}{153, 153, 153}
\definecolor{city_7}{RGB}{250, 170, 30}
\definecolor{city_8}{RGB}{220, 220, 0}
\definecolor{city_9}{RGB}{107, 142, 35}
\definecolor{city_10}{RGB}{152, 251, 152}
\definecolor{city_11}{RGB}{70, 130, 180}
\definecolor{city_12}{RGB}{220, 20, 60}
\definecolor{city_13}{RGB}{255, 0, 0}
\definecolor{city_14}{RGB}{0, 0, 142}
\definecolor{city_15}{RGB}{0, 0, 70}
\definecolor{city_16}{RGB}{0, 60, 100}
\definecolor{city_17}{RGB}{0, 80, 100}
\definecolor{city_18}{RGB}{0, 0, 230}
\definecolor{city_19}{RGB}{119, 11, 32}
\begin{figure*}[!ht]
  \small % scriptsize
  \centering


  \subfigure{%
    \includegraphics[width=.18\columnwidth]{figures/supplementary/frankfurt00000_016005_given.png}
  }
  \subfigure{%
    \includegraphics[width=.18\columnwidth]{figures/supplementary/frankfurt00000_016005_sp.png}
  }
  \subfigure{%
    \includegraphics[width=.18\columnwidth]{figures/supplementary/frankfurt00000_016005_gt.png}
  }
  \subfigure{%
    \includegraphics[width=.18\columnwidth]{figures/supplementary/frankfurt00000_016005_cnn.png}
  }
  \subfigure{%
    \includegraphics[width=.18\columnwidth]{figures/supplementary/frankfurt00000_016005_ours.png}
  }\\[-2ex]

  \subfigure{%
    \includegraphics[width=.18\columnwidth]{figures/supplementary/frankfurt00000_004617_given.png}
  }
  \subfigure{%
    \includegraphics[width=.18\columnwidth]{figures/supplementary/frankfurt00000_004617_sp.png}
  }
  \subfigure{%
    \includegraphics[width=.18\columnwidth]{figures/supplementary/frankfurt00000_004617_gt.png}
  }
  \subfigure{%
    \includegraphics[width=.18\columnwidth]{figures/supplementary/frankfurt00000_004617_cnn.png}
  }
  \subfigure{%
    \includegraphics[width=.18\columnwidth]{figures/supplementary/frankfurt00000_004617_ours.png}
  }\\[-2ex]

  \subfigure{%
    \includegraphics[width=.18\columnwidth]{figures/supplementary/frankfurt00000_020880_given.png}
  }
  \subfigure{%
    \includegraphics[width=.18\columnwidth]{figures/supplementary/frankfurt00000_020880_sp.png}
  }
  \subfigure{%
    \includegraphics[width=.18\columnwidth]{figures/supplementary/frankfurt00000_020880_gt.png}
  }
  \subfigure{%
    \includegraphics[width=.18\columnwidth]{figures/supplementary/frankfurt00000_020880_cnn.png}
  }
  \subfigure{%
    \includegraphics[width=.18\columnwidth]{figures/supplementary/frankfurt00000_020880_ours.png}
  }\\[-2ex]



  \subfigure{%
    \includegraphics[width=.18\columnwidth]{figures/supplementary/frankfurt00001_007285_given.png}
  }
  \subfigure{%
    \includegraphics[width=.18\columnwidth]{figures/supplementary/frankfurt00001_007285_sp.png}
  }
  \subfigure{%
    \includegraphics[width=.18\columnwidth]{figures/supplementary/frankfurt00001_007285_gt.png}
  }
  \subfigure{%
    \includegraphics[width=.18\columnwidth]{figures/supplementary/frankfurt00001_007285_cnn.png}
  }
  \subfigure{%
    \includegraphics[width=.18\columnwidth]{figures/supplementary/frankfurt00001_007285_ours.png}
  }\\[-2ex]


  \subfigure{%
    \includegraphics[width=.18\columnwidth]{figures/supplementary/frankfurt00001_059789_given.png}
  }
  \subfigure{%
    \includegraphics[width=.18\columnwidth]{figures/supplementary/frankfurt00001_059789_sp.png}
  }
  \subfigure{%
    \includegraphics[width=.18\columnwidth]{figures/supplementary/frankfurt00001_059789_gt.png}
  }
  \subfigure{%
    \includegraphics[width=.18\columnwidth]{figures/supplementary/frankfurt00001_059789_cnn.png}
  }
  \subfigure{%
    \includegraphics[width=.18\columnwidth]{figures/supplementary/frankfurt00001_059789_ours.png}
  }\\[-2ex]


  \subfigure{%
    \includegraphics[width=.18\columnwidth]{figures/supplementary/frankfurt00001_068208_given.png}
  }
  \subfigure{%
    \includegraphics[width=.18\columnwidth]{figures/supplementary/frankfurt00001_068208_sp.png}
  }
  \subfigure{%
    \includegraphics[width=.18\columnwidth]{figures/supplementary/frankfurt00001_068208_gt.png}
  }
  \subfigure{%
    \includegraphics[width=.18\columnwidth]{figures/supplementary/frankfurt00001_068208_cnn.png}
  }
  \subfigure{%
    \includegraphics[width=.18\columnwidth]{figures/supplementary/frankfurt00001_068208_ours.png}
  }\\[-2ex]

  \subfigure{%
    \includegraphics[width=.18\columnwidth]{figures/supplementary/frankfurt00001_082466_given.png}
  }
  \subfigure{%
    \includegraphics[width=.18\columnwidth]{figures/supplementary/frankfurt00001_082466_sp.png}
  }
  \subfigure{%
    \includegraphics[width=.18\columnwidth]{figures/supplementary/frankfurt00001_082466_gt.png}
  }
  \subfigure{%
    \includegraphics[width=.18\columnwidth]{figures/supplementary/frankfurt00001_082466_cnn.png}
  }
  \subfigure{%
    \includegraphics[width=.18\columnwidth]{figures/supplementary/frankfurt00001_082466_ours.png}
  }\\[-2ex]

  \subfigure{%
    \includegraphics[width=.18\columnwidth]{figures/supplementary/lindau00033_000019_given.png}
  }
  \subfigure{%
    \includegraphics[width=.18\columnwidth]{figures/supplementary/lindau00033_000019_sp.png}
  }
  \subfigure{%
    \includegraphics[width=.18\columnwidth]{figures/supplementary/lindau00033_000019_gt.png}
  }
  \subfigure{%
    \includegraphics[width=.18\columnwidth]{figures/supplementary/lindau00033_000019_cnn.png}
  }
  \subfigure{%
    \includegraphics[width=.18\columnwidth]{figures/supplementary/lindau00033_000019_ours.png}
  }\\[-2ex]

  \subfigure{%
    \includegraphics[width=.18\columnwidth]{figures/supplementary/lindau00052_000019_given.png}
  }
  \subfigure{%
    \includegraphics[width=.18\columnwidth]{figures/supplementary/lindau00052_000019_sp.png}
  }
  \subfigure{%
    \includegraphics[width=.18\columnwidth]{figures/supplementary/lindau00052_000019_gt.png}
  }
  \subfigure{%
    \includegraphics[width=.18\columnwidth]{figures/supplementary/lindau00052_000019_cnn.png}
  }
  \subfigure{%
    \includegraphics[width=.18\columnwidth]{figures/supplementary/lindau00052_000019_ours.png}
  }\\[-2ex]




  \subfigure{%
    \includegraphics[width=.18\columnwidth]{figures/supplementary/lindau00027_000019_given.png}
  }
  \subfigure{%
    \includegraphics[width=.18\columnwidth]{figures/supplementary/lindau00027_000019_sp.png}
  }
  \subfigure{%
    \includegraphics[width=.18\columnwidth]{figures/supplementary/lindau00027_000019_gt.png}
  }
  \subfigure{%
    \includegraphics[width=.18\columnwidth]{figures/supplementary/lindau00027_000019_cnn.png}
  }
  \subfigure{%
    \includegraphics[width=.18\columnwidth]{figures/supplementary/lindau00027_000019_ours.png}
  }\\[-2ex]



  \setcounter{subfigure}{0}
  \subfigure[\scriptsize Input]{%
    \includegraphics[width=.18\columnwidth]{figures/supplementary/lindau00029_000019_given.png}
  }
  \subfigure[\scriptsize Superpixels]{%
    \includegraphics[width=.18\columnwidth]{figures/supplementary/lindau00029_000019_sp.png}
  }
  \subfigure[\scriptsize GT]{%
    \includegraphics[width=.18\columnwidth]{figures/supplementary/lindau00029_000019_gt.png}
  }
  \subfigure[\scriptsize Deeplab]{%
    \includegraphics[width=.18\columnwidth]{figures/supplementary/lindau00029_000019_cnn.png}
  }
  \subfigure[\scriptsize Using BI]{%
    \includegraphics[width=.18\columnwidth]{figures/supplementary/lindau00029_000019_ours.png}
  }%\\[-2ex]

  \mycaption{Street Scene Segmentation}{Example results of street scene segmentation.
  (d)~depicts the DeepLab results, (e)~result obtained by adding bilateral inception (BI) modules (\bi{6}{2}+\bi{7}{6}) between \fc~layers.}
\label{fig:street_visuals-app}
\end{figure*}


\end{document}
