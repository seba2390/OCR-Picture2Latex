%%
%% This is file `sample-sigconf.tex',
%% generated with the docstrip utility.
%%
%% The original source files were:
%%
%% samples.dtx  (with options: `sigconf')
%% 
%% IMPORTANT NOTICE:
%% 
%% For the copyright see the source file.
%% 
%% Any modified versions of this file must be renamed
%% with new filenames distinct from sample-sigconf.tex.
%% 
%% For distribution of the original source see the terms
%% for copying and modification in the file samples.dtx.
%% 
%% This generated file may be distributed as long as the
%% original source files, as listed above, are part of the
%% same distribution. (The sources need not necessarily be
%% in the same archive or directory.)
%%
%% The first command in your LaTeX source must be the \documentclass command.
%%\documentclass[sigconf]{acmart}
%% NOTE that a single column version may be required for 
%% submission and peer review. This can be done by changing
%% the \doucmentclass[...]{acmart} in this template to 
\documentclass[manuscript,screen]{acmart}
%% 
%% To ensure 100% compatibility, please check the white list of
%% approved LaTeX packages to be used with the Master Article Template at
%% https://www.acm.org/publications/taps/whitelist-of-latex-packages 
%% before creating your document. The white list page provides 
%% information on how to submit additional LaTeX packages for 
%% review and adoption.
%% Fonts used in the template cannot be substituted; margin 
%% adjustments are not allowed.
%%
%%
%% \BibTeX command to typeset BibTeX logo in the docs
\AtBeginDocument{%
  \providecommand\BibTeX{{%
    \normalfont B\kern-0.5em{\scshape i\kern-0.25em b}\kern-0.8em\TeX}}}

%% Rights management information.  This information is sent to you
%% when you complete the rights form.  These commands have SAMPLE
%% values in them; it is your responsibility as an author to replace
%% the commands and values with those provided to you when you
%% complete the rights form.
\setcopyright{acmcopyright}
\copyrightyear{2021}
\acmYear{2021}
\acmDOI{10.1145/1122445.1122456}

%% These commands are for a PROCEEDINGS abstract or paper.
\acmConference[ACM CHI'21]{ACM CHI'21}{May 08--13, 2021}{Online, Earth}
\acmBooktitle{ACM CHI'21, June 03--05, 2021, Online, Earth}
\acmPrice{15.00}
\acmISBN{978-1-4503-XXXX-X/18/06}


%%
%% Submission ID.
%% Use this when submitting an article to a sponsored event. You'll
%% receive a unique submission ID from the organizers
%% of the event, and this ID should be used as the parameter to this command.
%%\acmSubmissionID{123-A56-BU3}

%%
%% The majority of ACM publications use numbered citations and
%% references.  The command \citestyle{authoryear} switches to the
%% "author year" style.
%%
%% If you are preparing content for an event
%% sponsored by ACM SIGGRAPH, you must use the "author year" style of
%% citations and references.
%% Uncommenting
%% the next command will enable that style.
%%\citestyle{acmauthoryear}

%%
%% end of the preamble, start of the body of the document source.
\begin{document}

%%
%% The "title" command has an optional parameter,
%% allowing the author to define a "short title" to be used in page headers.
\title{A Way to a Universal VR Accessibility Toolkit}

%%
%% The "author" command and its associated commands are used to define
%% the authors and their affiliations.
%% Of note is the shared affiliation of the first two authors, and the
%% "authornote" and "authornotemark" commands
%% used to denote shared contribution to the research.
\author{Felix J. Thiel}
\email{felix.thiel.18@ucl.ac.uk}
\orcid{0000-0002-7998-4270}
\author{Anthony Steed}
\email{A.Steed@ucl.ac.uk}
\orcid{0000-0001-9034-3020}
\affiliation{%
  \institution{Department of Computer Science, University College London}
  \streetaddress{66-72 Gower St}
  \city{London}
  \country{UK}
  \postcode{WC1E 6EA}
}


%%
%% By default, the full list of authors will be used in the page
%% headers. Often, this list is too long, and will overlap
%% other information printed in the page headers. This command allows
%% the author to define a more concise list
%% of authors' names for this purpose.
\renewcommand{\shortauthors}{Thiel and Steed, et al.}

%%
%% The abstract is a short summary of the work to be presented in the
%% article.
\begin{abstract}
    \textbf{This work was presented at the ACM CHI 2021 Workshop on Design and Creation of Inclusive User Interactions Through Immersive Media.} \\ \\
  Virtual Reality (VR) has become more and more popular with dropping prices for systems and a growing number of users. However, the issue of accessibility in VR has been hardly addressed so far and no uniform approach or standard exists at this time. In this position paper, we propose a customisable toolkit implemented at the system-level and discuss the potential benefits of this approach and challenges that will need to be overcome for a successful implementation.
\end{abstract}

%%
%% The code below is generated by the tool at http://dl.acm.org/ccs.cfm.
%% Please copy and paste the code instead of the example below.
%%

%\begin{CCSXML}
%<ccs2012>
%<concept>
%<concept_id>10010147.10010371.10010387.10010866</concept_id>
%<concept_desc>Computing methodologies~Virtual %reality</concept_desc>
%<concept_significance>500</concept_significance>
%</concept>
%<concept>
%<concept_id>10003120.10003121.10003124.10010866</concept_id>
%<concept_desc>Human-centered computing~Virtual reality</concept_desc>
%<concept_significance>500</concept_significance>
%</concept>
%</ccs2012>
%\end{CCSXML}

%\ccsdesc[500]{Computing methodologies~Virtual reality}
%\ccsdesc[500]{Human-centered computing~Virtual reality}

\begin{CCSXML}
<ccs2012>
<concept>
<concept_id>10003120.10011738.10011776</concept_id>
<concept_desc>Human-centered computing~Accessibility systems and tools</concept_desc>
<concept_significance>500</concept_significance>
</concept>
<concept>
<concept_id>10003120.10003121.10003124.10010866</concept_id>
<concept_desc>Human-centered computing~Virtual reality</concept_desc>
<concept_significance>500</concept_significance>
</concept>
</ccs2012>
\end{CCSXML}

\ccsdesc[500]{Human-centered computing~Accessibility systems and tools}
\ccsdesc[500]{Human-centered computing~Virtual reality}


%%
%% Keywords. The author(s) should pick words that accurately describe
%% the work being presented. Separate the keywords with commas.
\keywords{Virtual Reality, Accessibility Systems and Tools}

%% A "teaser" image appears between the author and affiliation
%% information and the body of the document, and typically spans the
%% page.

%\begin{teaserfigure}
%  \includegraphics[width=\textwidth]{sampleteaser}
%  \caption{Seattle Mariners at Spring Training, 2010.}
%  \Description{Enjoying the baseball game from the third-base
%  seats. Ichiro Suzuki preparing to bat.}
%  \label{fig:teaser}
%\end{teaserfigure}

%%
%% This command processes the author and affiliation and title
%% information and builds the first part of the formatted document.
\maketitle

\section{Introduction}
With the release of ever-cheaper virtual reality systems, the number of users of virtual reality (VR) is growing and its popularity is increasing. However, despite this trend, there is little work underway to address the unique accessibility challenges that come with VR. One of the main selling points of VR is that the user can interact with the virtual world in the same or at least a very similar way as they would in the real world. As a consequence of this, the user's body is primarily used as the input device for the applications which is a problem for physically impaired players. However, data gathered by other researchers indicates that this population is likely to be equally interested in the use of virtual reality as their able-bodied counterparts. In 2018, Beeston et al. \cite{beeston2018} tried to capture an image of the population of disabled players with a survey. They received 154 responses to their questionnaire and found that their participants do not differ much from able players in terms of play times and preferences. While this survey was focused solely on non-VR games, it still underpins the need for ways to make games, VR and non-VR alike, accessible to enable players of all abilities to share them. Based on this survey, it stands to reason that the rise in popularity of VR games will not be limited to the able-bodied population. Gerling et al. \cite{10.1145/3313831.3376265} also performed a survey on 25 wheelchair users about their views on virtual reality and found that there was general excitement about the potential of VR, though the participants also expressed concern about their ability to use it.

While this is a new problem for VR designers, accessibility issues have been widely explored in other media such as operating systems or televisions. Systems or devices in these classes are not designed to consider all kinds of impairments but they both provide tools to make them more accessible to users with specific impairments such as audio descriptions for deaf viewers or high contrast display modes for visually impaired computer users. However, for VR very little in terms of guidelines or tools exist that a seated player can use to make their experience more accessible to them. Some VR applications provide some accessibility options, but they are still in the minority. Examples of this are Half-Life: Alyx \cite{websiteHLAlyx} and Job Simulator \cite{websiteJobSimulator} that both offer subtitles and adjustments for seated players. However, these implementations are part of those games and can consequently not be used for other VR applications. Any other application remains inaccessible until their developer adds accessibility features.

We believe that this issue should not solely be left to individual developers but instead should be addressed on a system-level. It can not be expected that every developer adjusts their application to all potential impairments a player can have because they are manifold and it will most likely also not be an economically viable option. In the real world, the person's individual impairments are addressed by providing them with tools to overcome or at least alleviate the issue. These tools range from glasses to correct eye disorders, to powered wheelchairs and speech-generating devices. As the user's body is engaged with the input devices in VR, we find it fitting that a similar approach should be taken in VR that provides the user with a set of tools that they then can use and combine to fit their individual needs. They should also be able to take these tools, once configured, to other VR applications and use them in the same way one can use the same pair of glasses to read two different books.

\section{Proposal: A System-level Accessibility Toolkit} % Maybe different section title
We propose that instead of expecting the developers of each individual game to provide accessibility tools, this problem should be approached on the level of the VR system. Currently, the communication between VR games and the hardware goes through one or multiple VR frameworks, often built by the hardware manufacturers. Examples are Oculus's OVR framework and Valve's SteamVR. Accessibility tools located on this layer would be in a prime location to intercept and influence data coming from and going to the VR hardware. In one direction, the game output going to the headset can be changed and augmented. In the other direction, the user's input can be altered or new emulated input fed in. This is analogous to tools such as the Magnifier that sits in the Windows operating system and manipulates the output image and scales the mouse input before passing it along.

The scope of these tools could range from simpler tools that focus on a single type of data (e.g. move the player's body, enhance audio, amplify haptics, etc.) to complex cross-modal tools that transfer information from one channel into another (e.g. auralising visual information). They could also be used to add additional controls or in an attempt to loosen the link between the control scheme and environment to allow for easier integration of additional input devices. The latter concept has been previously described as ``vehicle pattern'' \cite{Steed2019Vehicle}.



We also suggest that these accessibility tools are made in a modular fashion that allows the user to mix-and-match the individual components and switch them on and off depending on their needs.
This again can be found in examples from other media such as operating systems that allow the user to mix individual components such as the Magnifier, high-contrast interfaces, and text-to-speech.

\section{Related Work} % Formerly "Background"
A very similar approach was undertaken by Zhao et al. \cite{zhao2019seeingvr} with their SeeingVR toolkit. It was designed to make VR applications more accessible to people with low vision (i.e. not blind, but also not correctable by glasses). The outcome was a set of 14 tools that provide visual and audio augmentation for people with low vision. Similarly to our proposal, the individual tools can be selected, adjusted, and combined by the user to fit their specific needs. To make their tools compatible with a large number of VR games, they are injecting code into existing Unity applications. As a result, nine of their 14 tools can be used with any Unity application without changes to it. The remaining five tools require the developers to include an SDK that they developed into their game.
Their overall approach of accessibility tools that can be added on top of existing games is very similar to our proposal. However, they do it at the level of the game engine, which excludes VR games that are made with other game engines such as the Unreal Engine or the Source Engine and the approach is at risk to changes in the game engine rendering behaviour. We propose to have the accessibility tools operate on the level of the VR-system that is located between the hardware and the game engine.

We are currently working towards our proposed toolkit with our work on co-piloting in VR \cite{ThielSteed}. With co-piloting, the controls of the game are shared across multiple input devices and players. This allows two players to act as one so that the second player can help the VR player with actions they can not perform on their own. This is particularly useful with seated VR players that lack the mobility and reach of a standing player. In the past year, we have run experiments with this interaction technique and are currently on the way to develop this into a toolkit on its own. Similar to SeeingVR, our Co-Piloting prototypes can be used with any application that fulfils a certain requirement. While SeeingVR required it to be a Unity application, our prototypes just require the application to be based on the SteamVR framework, which includes a larger number of popular VR games. As such, our prototypes are already operating on the system-level and demonstrate the feasibility of accessibility tools on the system-level.


\section{Potential Benefits of a System-Level Approach}

\subsection{Increase in Accessibility}
The biggest benefit of a system-level approach is that it would increase the accessibility of any VR game, past and future, without any changes by the original developer. This does not only benefit the developers who have less effort in making their games accessible but also the players because they have a wider selection of games that are accessible to them.

\subsection{Customisation and Ease of Use}
Another benefit is the potential of customisation. If the toolkit is based on modules that address different impairments, the players can use these building blocks to create a suite that is tailored to their needs. These combinations could also be saved as presets to remove the need to configure every game. It would also allow multiple users to share a system without the need for lengthy reconfiguration in between.

\subsection{Uniform Access}
A single toolkit that is usable across all VR applications also has the benefit that it will provide uniform access to the accessibility tools. When each game developer develops their own tools, their controls, capabilities, and availability  will likely differ between games and make using them more complex. A shared toolkit at the system-level could be used with all games and provide the same user experience and controls in all of them.

\section{Challenges of a System-Level Approach}

\subsection{System Integration and Support}
One big challenge is a potential lack of support by the system manufacturers. One issue we are struggling with during the development of our tools is that we need to use library hooks and undocumented interfaces to manipulate the player's virtual body reliably because the required matrices are not exposed to the outside. This also requires us to use an outdated version of SteamVR because later updates are not compatible with the library hooks any more. To prevent this kind of difficulties, a system-level accessibility toolkit will require support from the system manufacturers or upcoming standards like OpenXR. This support could be realised through an Accessibility API or an SDK that exposes the internals state of the VR systems.

\subsection{System Abuse}
To achieve their full potential, the tools will require extensive access to the input and output of the VR system. This comes with the danger that they get abused for cheating. A tool that moves the position of the virtual body in the VR scene can help a player with mobility issues play, but it can also give additional mobility and an unfair advantage in competitive games. This is not an easy challenge. The game could be informed about the player using any accessibility tools because these tools will use functionality and APIs offered by the VR system, but excluding those players or putting them on a separate ranking will once again separate players with and without impairments.

\subsection{Development and Maintenance}
Another challenge is the question of who will provide and support these tools. The accessibility tools of TV and operating systems are developed and maintained by the system manufacturers which could promise a good integration into the system and reliable maintenance. However, the manufacturers might just cover the most frequent impairments which limits the usefulness of the full system. A community-driven toolkit might offer a larger variety of tools that cover more impairments but brings the risk of creators and maintainers to later abandon their tools. This less structured approach also bears the risk of inconsistent control schemes between tools that interfere with each other.

\subsection{Control Conflicts}
Another challenge of this approach is that the controls of any of the system-level tools may conflict with the already existing controls of the games. This is not an issue for accessibility tools built into games, because the developers can incorporate them into their control scheme. One potential solution for this could be to use input channels that have not been used much as game input so far such as voice input or gaze input. Another possible solution for this could be a less rigid approach to control schemes as currently used with SteamVR. SteamVR provides the user with a system in which they can remap the game functions on their input device and share their mappings with the community. This would make the creation and distribution of mappings that are compatible with the controls of the accessibility tools much easier and more user-friendly.



%\section{Conclusion}
%While VR becomes more popular, there is only little effort at the moment to make it more accessible. In this position paper, we proposed an accessible toolkit that is based at the level of the VR system instead of being implemented in the games itself. The toolkit would consist of different tools for different impairments that the user can adjust and mix to fit their specific needs. We have also introduced the potential benefits of the proposed approach and what challenges need to be overcome for a successful implementation.
% Now what? Feels a bit inconclusive for a conclusion...

%\section{TODO}
%\begin{itemize}
    %\item Keywords, CCS, etc.
    %\item Grammar \& Spell check
%\end{itemize}



%%
%% The acknowledgments section is defined using the "acks" environment
%% (and NOT an unnumbered section). This ensures the proper
%% identification of the section in the article metadata, and the
%% consistent spelling of the heading.
\begin{acks}
%To XXXXXX, for testing the application. 
This project has received funding from the European Union’s Horizon 2020 research and innovation programme under grant agreement N° 856998.

\end{acks}

%%
%% The next two lines define the bibliography style to be used, and
%% the bibliography file.
\bibliographystyle{ACM-Reference-Format}
\bibliography{sample-base}

%%
%% If your work has an appendix, this is the place to put it.
%\appendix

%\section{Research Methods}


\end{document}
\endinput
%%
%% End of file `sample-sigconf.tex'.
