%!TEX root = ../JpsiVsChMult.tex



 \begin{figure}[htb]
  {\centering 
\resizebox*{.65\columnwidth}{!}{\includegraphics{RelativeYieldVsMult}} 
\par}
\caption{\label{fig:jpsiyield} Relative yield of inclusive \jpsi mesons, measured in three rapidity regions, as a function of relative charged-particle pseudorapidity, measured at mid-rapidity.
The error bars show the statistical uncertainties, and the boxes the systematic ones.
The dashed line is the first diagonal, plotted to guide the eye.
}
\end{figure}

The dependence of the relative \jpsi yield on the relative charged-particle pseudorapidity density for three \jpsi rapidity ranges is presented in Fig.~\ref{fig:jpsiyield}. An increase of the relative yield with charged-particle multiplicity is observed for all rapidity domains, with a similar behaviour at low multiplicities. At multiplicities beyond 1.5$-$2 times the event-average multiplicity, two different trends are observed. The relative yields at mid-rapidity and backward rapidity keep growing with the relative multiplicity in \ppb collisions similarly to the observation in \pp ~collisions at 7 TeV \cite{Abelev:2012aa}. At forward rapidity the trend is different. In this rapidity window a saturation of the relative yield sets in for high multiplicities. In lack of theoretical model calculations, it is unclear at the moment what is the cause of this observation.
We recall that the explored Bjorken $x$ ranges in the forward rapidity region are in the domain of shadowing/saturation, and that a variety of models \cite{Vogt:2013aa,Arleo:2013aa,Ferreiro:2014bia,Ma:2015sia,Ducloue:2015gfa} are fairly successful in describing the recent centrality-integrated and differential measurements of ALICE \cite{Adam:2015ac,Adam:2015jsa}, which correspond in terms of our relative multiplicities to $\dndeta/\average{\dndeta}\simeq2.5$ at most.

In Fig.~\ref{fig:jpsi2d} the \jpsi measurement at mid-rapidity is compared to that for prompt D mesons (average of D$^0$, D$^+$, and D$^{*+}$ species) for the \pt~ range 2$-$4 \gevc~ \cite{Adam:2016mkz}. Similar trends are seen for the \jpsi and D mesons, as observed earlier in pp collisions \cite{Adam:2015ota}. 

 \begin{figure}[htb]
  {\centering 
\resizebox*{.65\columnwidth}{!}{\includegraphics{Jpsi-Nch_comp_D.pdf}}
\par}
\caption{\label{fig:jpsi2d} Relative yield of inclusive \jpsi mesons as a function of relative charged-particle pseudorapidity density, measured at mid-rapidity, in comparison to D mesons (average of D$^0$, D$^+$, and D$^{*+}$ species), for the \pt ~interval 2-4 \gevc ~\cite{Adam:2016mkz}.
The error bars show the statistical uncertainties, and the boxes the systematic ones (additional systematic uncertainties due to the b feed-down contributions and the event normalisation are not shown for the D mesons).
}
\end{figure}


The nuclear modification factor for \jpsi production in \ppb collisions (\rpb) as a function of centrality was presented in \cite{Adam:2015jsa}. The relationship between geometry-related quantities, that quantify the centrality of the collision, and experimental observables in \ppb collisions may be subject to a selection bias \cite{Adam:2015aa} which needs care in interpretation. 
By performing the ratio of the nuclear modification factors at forward and backward rapidities as a function of multiplicity, the dependence on geometry-related quantities is eluded. The forward-to-backward nuclear modification factor ratio is defined as:

\begin{eqnarray} \label{rfb}
R_{\mathrm{FB}} &=& \frac{\rpb(2.03 < \ycms < 3.53)}{\rpb(-4.46 < \ycms < -2.96)} \\ \nonumber
 &=& \frac{Y^{J/\psi}_{\mathrm{pPb}} (2.03 < \ycms < 3.53)}{Y^{J/\psi}_{\mathrm{pPb}} (-4.46 < \ycms < -2.96)} \times
    \frac{\mathrm{d}\sigma^{J/\psi}_{\mathrm{pp}}/\mathrm{d}y (-4.46 < \ycms <
    -2.96)}{\mathrm{d}\sigma^{J/\psi}_{\mathrm{pp}}/\mathrm{d}y (2.03 < \ycms < 3.53) }
\end{eqnarray}

Since the average charged-particle multiplicities and their uncertainties are consistent with each other for the two sets of data, the values of $R_{\mathrm{FB}}$ are shown versus the average value of the two in each multiplicity bin. Note that, differently than for the case of the nuclear modification factor measurement in \cite{Abelev:2014aa}, for the present measurement the rapidity ranges are not symmetric with respect to \ycms ~= 0 to take advantage of all the signal yield, allowing the study up to high multiplicities. 
The values of the reference \pp ~cross section were obtained by means of an interpolation procedure using measurements at center-of-mass energies of 2.76 and 7 TeV \cite{ALICELHCbRefpp}. The resulting backward-to-forward ratio of \jpsi production cross sections in \pp ~collisions is $ 0.691 \pm 0.048$, leading to a global uncertainty on the $R_{\mathrm{FB}}$ measurement of 6.9\%.

For the $R_{\mathrm{FB}}$ ratio, the systematic uncertainties of the absolute yields in \ppb collisions (Tab.~\ref{tab:jpsisystComb}) are considered as uncorrelated between forward and backward rapidities, and therefore added in quadrature. The uncorrelated systematic uncertainties of the production cross sections in \pp ~collisions are the same as a function of multiplicity, so they are added in quadrature to the global uncertainty (quadratic sum of muon tracking, trigger and matching efficiency uncertainties) of the \ppb data, resulting in a total relative uncertainty of 11\%.

 \begin{figure}[htb]
  {\centering 
\resizebox*{.67\columnwidth}{!}{\includegraphics{RFBVsRelMult}} 
\par}
\caption{\label{fig:RFBvsMult}  $R_{\mathrm{FB}}$ of inclusive \jpsi in p-Pb collisions at $\sqrt{s_{\mathrm{NN}}} = $ 5.02 TeV as a function of relative charged-particle pseudorapidity density, measured at mid-rapidity. The red box around unity represents the global uncertainty.
The error bars show the statistical uncertainties, and the boxes the systematic ones.
}
\end{figure}

The $R_{\mathrm{FB}}$ ratio is shown as a function of the relative charged-particle pseudorapidity density in Fig.~\ref{fig:RFBvsMult}. In multiplicity-inclusive collisions 
for symmetric $y$ ranges at forward and backward rapidities \cite{Abelev:2014aa}, $R_{\mathrm{FB}}$ is smaller than unity and described by theoretical models. The present measurement shows that the suppression of \jpsi production at forward rapidity with respect to backward rapidity increases significantly with charged-particle multiplicity, since $R_{\mathrm{FB}}$ reaches values as low as 0.34 $\pm$ 0.06 (stat.) $\pm$ 0.05 (syst.). 
A forward-backward asymmetry can be noticed for inclusive charged-particle production studied in \cite{Adam:2015aa}. 
Even though the range of relative charged-particle multiplicities probed in that measurement is not as large as in the present measurement of \jpsi production, the apparent similarity of the trend seen in Fig.~\ref{fig:RFBvsMult} to soft particle production is intriguing.

In Fig.~\ref{fig:jpsimpt} the relative \mpt ~of \jpsi mesons at backward and forward rapidity is shown as a function of the relative charged-particle pseudorapidity density. The results are similar at forward and backward rapidities. An increase of the relative \mpt ~with multiplicity at low charged-particle multiplicity is observed, but for multiplicities beyond 1.5 times the average multiplicity it saturates.  For backward rapidity, the simultaneous increase of the yield and the saturation of the relative \mpt could be an indication of \jpsi production from an incoherent superposition of parton-parton interactions, as suggested by data on correlations of jet-like yields per trigger particle \cite{Abelev:2015aa}.

 \begin{figure}[htb]
  {\centering 
\resizebox*{.68\columnwidth}{!}{\includegraphics{RelativeMptWChargedVsMult}}
\par}
\caption{\label{fig:jpsimpt} Relative \mpt ~of \jpsi mesons for backward and forward rapidity as a function of the relative charged-particle pseudorapidity density, measured at mid-rapidity. The bars show the statistical uncertainties, and the boxes the systematic ones. The data for charged particles (h$^\pm$) \cite{Abelev:2013aa} are included for comparison. The latter are for $|\etacms|<0.3$ and with \pt ~in the range 0.15 to 10 \gevc ~and have an additional normalisation uncertainty of 3.4\%.
}
\end{figure}

The \pt ~broadening observed in the analysis of \jpsi production in \ppb collisions as a function of centrality \cite{Adam:2015jsa} is well described by initial and final-state multiple scattering of partons within the nuclear medium \cite{Kang:2013aa}. The comparison of data to model calculations, performed in \cite{Adam:2015jsa}, corresponds in terms of relative multiplicities to a range up to roughly $\dndeta/\average{\dndeta}=2.5$. It remains to be seen whether such models can explain the saturation observed in the relative \mpt ~of the \jpsi mesons for events with higher multiplicities.

It is interesting to contrast the observed saturation of \mpt~for \jpsi mesons with the monotonic increase of \mpt ~for charged hadrons (dominated by pion production) ~\cite{Abelev:2013bla} with the multiplicity measured at mid-rapidity also shown in Fig.~\ref{fig:jpsimpt}. Note that this measurement is for particles in $|\etacms|<0.3$ and with \pt ~in the range 0.15 to 10 \gevc, and it is relative to events with at least one particle in this kinematic range (for which $\average{\nch}=11.9 \pm 0.5$ and $\average{\pt}=0.696 \pm 0.024$ \gevc~\cite{Abelev:2013bla}).
Although the different kinematic regions may play a role and care is needed in the interpretation, it is apparent that the two observables, characterised by rather different production mechanisms (and momentum-transfer) exhibit different patterns in the multiplicity dependence of the average transverse momentum. 


