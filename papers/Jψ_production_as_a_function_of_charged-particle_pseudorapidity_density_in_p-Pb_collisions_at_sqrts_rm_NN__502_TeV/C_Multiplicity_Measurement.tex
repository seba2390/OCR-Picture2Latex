%!TEX root = ../JpsiVsChMult.tex

The charged-particle pseudorapidity density \dndeta is measured at midrapidity, $|\eta|<1$, and is based on the SPD information. Tracklets, i.e. track segments built from hit pairs in the two SPD layers, are used together with the interaction vertex position, which is also determined with the SPD information \cite{Abelev:2013ab}. Several quality criteria are applied to select only events with an accurate determination of the $z$ coordinate of the vertex, $z_{\mathrm{vtx}}$. To ensure full SPD acceptance for the tracklet multiplicity \ntr evaluation within $|\eta|<1$, the condition $|z_{\mathrm{vtx}}|< 10$ cm is applied for the selection of the events. 

During the data taking period about $8\%$ of the SPD channels were inactive, the exact fraction being time-dependent.
 The impact of the inactive channels of the SPD on the tracklet multiplicity measurement varies with $z_{\mathrm{vtx}}$. 
A $z_{\mathrm{vtx}}$-dependent correction factor is determined from data, as discussed in \cite{Abelev:2012aa}. This factor also takes into account the time-dependent variations of the fraction of inactive SPD channels.
The correction factor is randomised on an event-by-event basis using a Poisson distribution in order to emulate the dispersion between the true charged-particle multiplicity and the measured tracklet multiplicities.

The overall inefficiency, the production of secondary particles due to interactions in the detector material, particle decays and fake-tracklet reconstruction lead to a difference between the number of reconstructed tracklets and the true primary charged-particle multiplicity \nch ~(see details in \cite{Abelev:2013ab})\footnote{In this context, we regard as primary charged-particles all prompt charged particles including all decay products except products from weak decays of light flavour hadrons and of muons.}. 
Using  events simulated with the DPMJET event generator \cite{Roesler:2000he}, the correlation between the tracklet multiplicity (after the $z_{\mathrm{vtx}}$-correction), \ntrcorr, and the generated primary charged particles \nch ~is determined. 
The correction factor $\beta$ to obtain the average \dndeta value corresponding to a given \ntrcorr bin is computed from a linear fit of the \ntrcorr - \nch ~correlation. 
The charged-particle pseudorapidity density value in each multiplicity bin is given relative to the event-averaged value and is calculated as: 
$\dndetar = \dndeta / \average{\dndeta} = \beta \cdot \average{\ntrcorr} / \left(\Delta \eta \cdot \average{\dndeta}\right)$, where $\Delta\eta=2$ and \average{\dndeta} is the charged-particle pseudorapidity density for non-single diffractive (NSD) collisions, which was measured  to be $ \average{\dndeta} = 17.64 \pm 0.01 (\rm{stat.}) \pm 0.68 (\rm{syst.})$ %for \ppb collisions in $|\eta|<$ 1 
\cite{Abelev:2013ab}. The resulting values for the multiplicity bins are summarised in Tables~\ref{multiplicitypPb} and \ref{multiplicitypPbMidy} for forward and mid-rapidity, respectively.  
For the data at backward rapidity, the values are well within the uncertainties of those at forward rapidity.

 \begin{table}[h!]
\centering
\begin{tabular}{crrrr}
\toprule
\dndeta & \dndetar & $\sigma / \sigma_{MB}$\\
\midrule
4.8 $\pm$ 0.2 & 0.27 $\pm$ 0.01 & 26.4\% \\
10.9 $\pm$ 0.4 & 0.62 $\pm$ 0.03 & 14.3\% \\
14.6 $\pm$ 0.5 & 0.83 $\pm$ 0.03  & 7.9\% \\
17.8 $\pm$ 0.5 & 1.01 $\pm$ 0.04 & 9.6\% \\
21.4 $\pm$ 0.7 & 1.22 $\pm$ 0.05 & 8.5\% \\
25.0 $\pm$ 0.8 & 1.42 $\pm$ 0.06 & 7.2\% \\
28.6 $\pm$ 0.8 & 1.62 $\pm$ 0.06 & 6.0\% \\
32.7 $\pm$ 1.0 & 1.85 $\pm$ 0.07 & 6.7\% \\
37.8 $\pm$ 1.1 & 2.14 $\pm$ 0.08 & 4.6\% \\
44.2 $\pm$ 1.3 & 2.51 $\pm$ 0.10 & 4.2\% \\
54.3 $\pm$ 1.6 & 3.08 $\pm$ 0.12 & 2.4\% \\
71.4 $\pm$ 2.1 & 4.05 $\pm$ 0.16  & 0.3\% \\
\bottomrule
\end{tabular}
\caption{ \label{multiplicitypPb} Average charged-particle pseudorapidity density values (absolute and relative) in each multiplicity bin, obtained from \ntrcorr measured in the range $|\eta| <1$. The values correspond to the data sample used for the forward rapidity analysis. Only systematic uncertainties are shown since the statistical ones are negligible. The fraction of the MB cross section for each multiplicity bin is also indicated.}
\end{table}

\begin{table}[h!]
\centering
\begin{tabular}{crrrr}
\toprule
\dndeta & \dndetar & $\sigma / \sigma_{MB}$\\
\midrule
6.9 $\pm$ 0.2  & 0.39 $\pm$ 0.02 &  47.2\% \\
22.9 $\pm$ 0.6 & 1.30 $\pm$ 0.05 & 39.7\%  \\
42.3 $\pm$ 1.1 & 2.40 $\pm$ 0.10  & 10.9\%  \\
64.4  $\pm$ 1.6  &  3.65  $\pm$ 0.15  & 1.0\% \\
\bottomrule
\end{tabular}
\caption{ \label{multiplicitypPbMidy} 
As Table~\ref{multiplicitypPb}, but for the analysis of \jpsi production at mid-rapidity.
} 
\end{table}

The fraction of the MB cross section contained in each multiplicity bin ($\sigma / \sigma_{MB}$, derived from the respective event counts in the multiplicity bins and total number of MB events) is reported in Tables~\ref{multiplicitypPb} and \ref{multiplicitypPbMidy}. The softest MB events, which lead to absence of tracklets in $|\eta|<1$ are not accounted for in this analysis. They correspond to 
1.2\% of $\sigma_{MB}$ for the MB-triggered events and 1.9\% for the muon-triggered data; the difference is due to the different fraction of inactive channels in SPD and affects, albeit in a negligible way, only our first multiplicity bin.

The multiplicity selection in this analysis allows to sample the data in bins containing a small fraction of the MB cross section. Therefore, it gives the possibility to study the \jpsi production in rare high-multiplicity events which were not accessible in the centrality-based analysis \cite{Adam:2015jsa} (where the most-central event class corresponds to the range 2-10\% in $\sigma / \sigma_{MB}$).
