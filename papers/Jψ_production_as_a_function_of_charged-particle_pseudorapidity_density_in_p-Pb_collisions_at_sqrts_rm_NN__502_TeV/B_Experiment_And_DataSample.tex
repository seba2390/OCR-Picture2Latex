%!TEX root = ../JpsiVsChMult.tex

The ALICE central barrel detectors are located in a solenoidal magnetic field of 0.5 T. The main tracking devices in this region are the Inner Tracking System (ITS), which consists of six layers of silicon detectors around the beam pipe, and the Time Projection Chamber (TPC), a large cylindrical gaseous detector providing tracking and particle identification via specific energy loss. Tracks are reconstructed in the active volume of the TPC within the pseudorapidity range $|\eta|<0.9$ in the laboratory frame.  The first two layers of the ITS ($|\eta| < $ 2.0 and $|\eta| < $ 1.4), the Silicon Pixel Detector (SPD), are used for the collision vertex determination and the charged-particle multiplicity measurement. The minimum-bias (MB) events are triggered requiring the coincidence of the two V0 scintillator arrays covering 2.8 $< \eta <$ 5.1 and $-3.7 < \!\eta \!< \!-1.7$, respectively. The two neutron Zero Degree Calorimeters (ZDC), placed at 112.5 m on both sides of the interaction point, are used to reject electromagnetic interactions and beam-induced background. 
The muon spectrometer, covering $-4 < \eta < -2.5$, consists of a front absorber, a 3 $\rm{T} \cdot \rm{m}$ dipole magnet, ten tracking layers, and four trigger layers located behind an iron-wall filter. In addition to the MB trigger condition, the dimuon trigger requires the presence of two opposite-sign particles in the muon trigger chambers.  The trigger comprises a minimum transverse momentum requirement of \pt~$> 0.5$ GeV/$c$ at track level.  The single-muon trigger efficiency curve is not sharp; the efficiency reaches a plateau value of $\sim 96\%$ at $\pt \sim$~1.5 GeV/$c$. The ALICE detector is described in more detail in~\cite{Abelev:2008aa} and its performance is outlined in \cite{Abelev:2014ffa}. 

The results presented in this Letter are obtained with data recorded in 2013 in \ppb collisions at \snn ~= 5.02 TeV. MB events are used for the \jpsi reconstruction in the dielectron channel at mid-rapidity. The dimuon-triggered data have been taken with two beam configurations, allowing the coverage of both forward and backward rapidity ranges.
In the period when the dimuon-triggered data sample was collected, the MB interaction rate reached a maximum of 200 kHz, corresponding to a maximum pile-up probability of about 3\%. The MB-triggered events used for the dielectron channel analysis were collected in one of the beam configurations at a lower interaction rate (about 10 kHz) and consequently had a smaller pile-up probability of 0.2\%.

Due to the asymmetry of the beam energy per nucleon in \ppb collisions at the LHC, the nucleon-nucleon center-of-mass rapidity frame is shifted in rapidity by $\Delta y =0.465$ with respect to the laboratory frame in the direction of the proton beam. This leads to a rapidity coverage in the nucleon-nucleon center-of-mass system $-1.37 < y_{\rm{cms}} < 0.43$ for the MB events, while the coverage for the dimuon-triggered data for the two different beam configurations is $-4.46 < y_{\rm{cms}} <$ $-2.96$ (muon spectrometer located in the Pb-going direction) 
and 2.03 $< y_{\rm{cms}} <$ 3.53 (muon spectrometer located in the p-going direction). The integrated luminosities used in this analysis are 51.4$\pm$1.9 $\mu$b$^{-1}$ (mid-rapidity), 5.01$\pm$0.19 nb$^{-1}$ (forward $y$) and 5.81$\pm$0.20 nb$^{-1}$ (backward $y$).
