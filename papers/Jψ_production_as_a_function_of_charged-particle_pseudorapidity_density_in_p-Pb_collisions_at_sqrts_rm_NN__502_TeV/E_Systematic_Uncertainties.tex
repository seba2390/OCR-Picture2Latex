%!TEX root = ../JpsiVsChMult.tex

The systematic uncertainty of the overall average charged-particle pseudorapidity density was estimated to be 3.8\% \cite{Abelev:2013ab}. This includes effects related to the uncertainties in the simulations, detector acceptance and event selection efficiency, and it is dominated by the normalisation to the NSD event class. 
Possible correlation between the average multiplicity and that evaluated in a given bin would lead to a partial cancellation of certain sources of uncertainty when computing the relative multiplicity. As a conservative estimate, the uncertainty on the relative multiplicity is considered to be equal to the uncertainty on the overall charged-particle pseudorapidity density.

The influence of variations of the $\eta$ distribution in the calculation of the $\beta$ correction factors, is estimated from the difference between the average number of tracklets obtained in the data taken with the two different beam configurations. 
The corresponding uncertainty on the multiplicity determination amounts to 1\%. The uncertainties arising from the fit procedure of the \ntrcorr - \nch ~correlation in simulated events, used to obtain the correction factors, are also included. This uncertainty ranges between 0.2\% (at high multiplicity) and 2\% (at low multiplicity). The event selection related to the vertex quality has a 1\% effect on the average multiplicity in the lowest multiplicity bin and a negligible effect for the other bins. Due to the uncertainty on the determination of the multiplicity of the individual events, there could be a migration of events among the multiplicity bins (bin-flow). This bin-flow effect is determined by running the analysis several times with different seeds for the random factor of the multiplicity correction (bin-flow test). The bin-flow uncertainties are obtained from the dispersion of the average multiplicity values in the bin-flow tests for each multiplicity bin. Finally, the effect of pile-up is studied using a toy model that reproduces the main features of the multiplicity determination, and takes into account the mis-identification of multiple collisions in the same event. The contributions of bin-flow and pile-up to the measured multiplicities are found to be negligible for all the data sets (taken at different interaction rates). 
The bin-dependent uncertainty is added in quadrature to the 3.8\% uncertainty of  $\average{\dndeta}$, resulting in a systematic uncertainty of the relative charged-particle multiplicity of 4 $-$ 4.5\% depending on the multiplicity bin.


The yields reported here are provided relative to the event-average yield and the uncertainties are estimated for this ratio. The systematic uncertainties related to trigger, tracking and matching efficiency are correlated between the multiplicity-differential and the integrated determinations. They cancel out to a large extent. 

In the dimuon analysis, a combined systematic uncertainty which includes the \acef ~variations due to the uncertainty of the \jpsi \pt ~and rapidity input distributions used in the simulation and multiplicity bin-flow effects is derived. Due to the multiplicity bin-flow, and the fact that the invariant mass and $\average{p_{\rm{T}}^{\mu^{+}\mu^{-}}} (m_{\mu^{+}\mu^{-}})$ spectra are weighted by \acef, these uncertainties can not be computed separately. The combined uncertainty is obtained from the \rms ~~of the relative yield values obtained running the analysis several times with different seeds for the random factor of the multiplicity correction. In addition the systematic uncertainty for the signal extraction is estimated as the \rms ~~of the results obtained using different fitting assumptions for a given bin-flow test. 
The fit procedure is varied by adopting a pseudo-gaussian function for the signal, a polynomial times an exponential function for the background and by using two additional fitting ranges. The uncertainties due to the determination of the parameters of the signal tails is estimated by using several sets of parameters from different MC simulations. 
The uncertainty related to the computation method of the relative $F_{2\mu / MB}$ is estimated considering the difference between the two available methods to measure the factor in multiplicity bins \cite{Adam:2015jsa}. 
The effect of the vertex quality selection is estimated from the difference of the obtained yields with and without this selection. Finally, in order to determine the pile-up effect on the measured yield in each multiplicity bin, the pile-up toy model is extended by including the production of \jpsi using as input the measured yields as a function of multiplicity. The difference between the measured and toy MC yields is taken as systematic uncertainty. All these effects are uncorrelated within a given multiplicity bin, hence they are added quadratically to obtain the systematic uncertainty of the relative yield in a multiplicity bin. Also, these systematic uncertainties are considered as uncorrelated between the different rapidity intervals. A summary of the maximum and minimum relative yield systematic uncertainties is shown in Tab.~\ref{tab:jpsisystComb}. In addition, the 3.1\% uncertainty of the event-average yield normalisation to NSD, is reported separately. 

The systematic uncertainties are computed also for the absolute yields in multiplicity bins at forward and backward rapidities. The absolute yields are used to compute the ratio of the nuclear modification factors at forward and backward rapidities. The values of the uncertainties on the absolute yields are shown in parentheses in Tab.~\ref{tab:jpsisystComb}, when they are different from the ones obtained for the relative yield. In addition, for the absolute yield measurement, the muon tracking, trigger and matching efficiency uncertainties need to be taken into account \cite{Abelev:2014aa}. They amount to 4\% (6\%), 3\% (3.4\%) and 1\% (1\%) at forward (backward) rapidities.

\begin{table} [h!]
\centering
\begin{tabular}{cccc}
\toprule
Source & 2.03 $< y_{\mathrm{cms}} <$ 3.53 & $-$4.46 $< y_{\mathrm{cms}} < -$2.96 & $-$1.37 $< y_{\mathrm{cms}} <$ 0.43   \\
\midrule
Sig. Extr. & 0.6 $-$ 1.9\%  & 0.5 $-$ 1.8\% &  3.2 $-$ 8.4\% \\
$F_{2\mu / MB}$ method & 0.3 $-$ 3.9\% & 0.3 $-$ 3.9\% & not applicable \\
\acef/bin-flow & 1 $-$ 5.9\% & 1.4 $-$ 3.9\% &   3.1 $-$ 9.9\%  \\
Pile-up & 1 $-$ 4\% & 1 $-$ 1.5\% & negligible\\
Signal tails & 0.5\% (2\%) & 0.5\% (2\%) & not applicable\\
Vertex quality sel. & 0.3\%-0.6\%$^{*}$ (0.9\%$^{*}$) & 0.3\%-0.6\%$^{*}$ (0.9\%$^{*}$)& negligible \\
%\addlinespace[\spacebeforetotal]
\hline %\hline
Total & 2.1 $-$ 8.3\% (3.0 $-$ 8.6\%) & 2.1 $-$ 6.0\% (2.9 $-$ 6.4\%)  & 4.5 $-$ 13\% \\
\bottomrule
\end{tabular}
\caption{\label{tab:jpsisystComb} The relative systematic uncertainties  of the relative \jpsi ~yield measurement in the three rapidity ranges. The values in parentheses correspond to the absolute yield measurement when different from the relative ones. The ranges represent the minimum and maximum values of the uncertainties over the multiplicity bins. For the vertex quality selection, the uncertainties marked with $^{*}$ refer only to the lowest-multiplicity bin; for all other bins the value is 0.3\%. The trigger, tracking and matching efficiency uncertainties are not listed in the table.}
\end{table}

For the dielectron decay channel, the signal extraction uncertainty is derived based on the \rms  ~~value of the  different signal yield ratios  obtained for the variations of the background  and the signal integration window as in \cite{Adam:2015jsa}.  The uncertainty is largest for the highest multiplicity bins. Since the \pt ~distribution of \jpsi may depend on multiplicity, the unmeasured \pt ~spectrum leads to a multiplicity-dependent uncertainty, determined as in \cite{Adam:2015jsa}. As explained in section~\ref{sec:data}, the pile-up contamination is very low and the induced uncertainty is negligible for all the multiplicity intervals.
The uncertainty related to bin-flow is estimated with the same method as in the dimuon analysis.
The total systematic uncertainty varies as a function of multiplicity between 4.5\% and 13\%, see Tab.~\ref{tab:jpsisystComb}.


For the relative \jpsi \mpt, the effects of the uncertainty on the determination of \acef, the \mpt ~extraction procedure and bin-flow are computed together following the same procedure as for the relative yield. The \mpt ~extraction uncertainty is obtained from the dispersion of the results using different fit combinations, including variations of the invariant mass signal and background parameterisations, fitting range and the use of a second order polynomial times an exponential function for the \mpt of background dimuons. The effect of considering the \jpsi \mpt ~as independent of the invariant mass in the $\langle p_{\rm{T}}^{\mu^{+}\mu^{-}} \rangle$ fits is found to be negligible. The impact of fixing the signal and background parameters during the fitting procedure is observed to be negligible as well. The events removed by the vertex quality selection do not have reconstructed \jpsi and therefore the \mpt ~remains unmodified. Finally, using a pile-up toy model, it is shown that the pile-up has no effect on the \mpt ~measurement, except for the two bins corresponding to the largest multiplicities. All these effects are considered as uncorrelated in a given multiplicity bin and hence their respective uncertainties are added quadratically to obtain the relative \mpt ~systematic uncertainty in each multiplicity bin. These systematic uncertainties are considered as uncorrelated between the different rapidity intervals. The results of the uncertainties entering in the relative \mpt ~measurement are reported in Tab.~\ref{tab:jpsimptsystComb}. 

\begin{table} [h!]
\centering
\begin{tabular}{ccc}
\toprule
Source & 2.03 $< y_{\mathrm{cms}} <$ 3.53 & $-$4.46 $< y_{\mathrm{cms}} < -$2.96\\
\midrule
Sig. extr. & 0.2 $-$ 1.2\% & 0.2 $-$ 0.5\% \\
\acef/bin-flow & 0.5 $-$ 2.0\% & 0.7 $-$ 2.7\%\\
Pile-up & 0.2 $-$ 0.7\%$^{*}$ & 0.2 $-$ 0.8\%$^{*}$  \\
%\addlinespace[\spacebeforetotal]
\hline %\hline
Total & 0.6 $-$ 2.4\% &  0.7 $-$ 2.9\% \\
\bottomrule
\end{tabular}
\caption{\label{tab:jpsimptsystComb} The systematic uncertainties for the relative \mpt ~measurement at forward and backward rapidities. The values represent the minimum and maximum values of the uncertainties over the multiplicity bins. The uncertainties marked with $^{*}$ only refer to the two highest-multiplicity bins.}
\end{table}

