3D human pose estimation has ubiquitous applications in sport analysis, human-computer interaction, and fitness and dance teaching. While there has been remarkable progress in 3D pose estimation from a monocular image or video~\cite{hmrKanazawa17, Moon_2020_ECCV_I2L-MeshNet, kolotouros2019spin, kocabas2019vibe, xiang2019monocular}, inevitable challenges such as the depth ambiguity and the self-occlusion are still unsolved. 



\begin{figure}
     \centering
     \begin{subfigure}[h]{0.23\textwidth}
         \centering
         \includegraphics[width=\textwidth]{figures/cover/image-comp.jpg}
         \caption*{Input image}
     \end{subfigure}
     \begin{subfigure}[h]{0.23\textwidth}
         \centering
         \includegraphics[width=\textwidth]{figures/cover/smplify-comp.jpg}
         \caption*{SMPLify-X~\cite{SMPL-X:2019}}
     \end{subfigure}
     \vspace*{0.2cm}
     \begin{subfigure}[h]{0.45\textwidth}
         \centering
         \includegraphics[width=0.98\linewidth, trim=25 50 25 50]{figures/cover/scene_cover_green-comp.png}
         \caption*{3D visualization of our (left) and SMPLify-X (right) results}
     \end{subfigure}
     \vspace*{-0.2cm}
     \caption{While the state-of-the-art single-view 3D pose estimator~\cite{SMPL-X:2019} yields a small reprojection error, the recovered 3D poses may be erroneous due to the depth ambiguity. We make use of the mirror in the image to resolve the ambiguity and reconstruct more accurate human pose as well as the mirror geometry.}
     \vspace*{-0.5cm}
    \label{fig:demo1}
\end{figure}



In many scenes like dancing rooms and gyms, people are often in front of a mirror. In this case, we are able to see the person and his/her mirror image simultaneously. The mirror image actually provides an additional virtual view of the person, which can resolve the single-view depth ambiguity if the mirror is properly placed. Moreover, unseen part of the person can also be observed from the mirror image, so that the occlusion problem can be alleviated. 


In this paper, we investigate the feasibility of leveraging such mirror images to improve the accuracy of 3D human pose estimation. We develop an optimization-based framework with mirror symmetry constraints that are applicable without knowing the mirror geometry and camera parameters. We also provide a method to utilize the properties of vanishing points to recover the mirror normal along with the camera parameters, so that an additional mirror normal constraint can be imposed to further improve the human pose estimation accuracy. The effectiveness of our framework is validated on a new dataset for this new task with 3D pose ground-truth provided by a multi-view camera system. 


An important application of the proposed approach is to generate pseudo ground-truth annotations to train existing 3D pose estimators. To this end, we collect a large-scale set of Internet images that contain people and mirrors and generate 3D pose annotations with the proposed optimization method. The dataset is named Mirrored-Human.  
Compared with existing 3D human pose datasets~\cite{h36m_pami,mono-3dhp2017,vonMarcard2018} that are captured with very few subjects and background scenes, Mirrored-Human has a significantly larger diversity in human poses, appearances and backgrounds, as shown in Fig.~\ref{fig:dataset}. The experiments show that, by combining Mirrored-Human with existing datasets as training data, both accuracy and generalizability of existing 3D pose estimation methods can be significantly improved for both single-person and multi-person cases.   

In summary, we make the following contributions:
\begin{itemize}
    \item We introduce a new task of reconstructing  human pose from a single image in which we can see the person and the person's mirror image. 
    \item We develop a novel optimization-based framework with mirror symmetry constraints to solve this new task, as well as a method to recover mirror geometry from a single image.
    \item We collect a large-scale dataset named Mirrored-Human from the Internet, provide our reconstructed 3D poses as pseudo ground-truth, and show that training on this new dataset can improve the performance of existing 3D human pose estimators. 
\end{itemize}






