

Fig.~\ref{fig:method} presents the pipeline of our framework. Given an image that contains a person and a mirror, our goal is to recover the human mesh considering the mirror geometry. The key insight is that the person and his/her mirror image can be treated as two people, and we reconstruct them together with the mirror symmetry constraints. This section will be organized as follows. First, the formulation of single-person mesh recovery is introduced (Sec.~\ref{sec:spmr}). Then the mirror symmetry constraints that relate the two people will be elaborated (Sec.~\ref{sec:mi_geo}). Finally, the objective functions and the whole optimization are described (Sec.~\ref{sec:opt}).

\subsection{Human mesh recovery with SMPL model}
\label{sec:spmr}

We adopt the SMPL model~\cite{SMPL:2015} as our human representation. The SMPL model is a differentiable function $\bm M(\bm \theta, \bm \beta) \in \mathbb R^{3\times N_v}$ mapping the pose parameters $\bm \theta \in \mathbb R^{72}$ and the shape parameters $\bm{\beta} \in \mathbb R^{10}$ to a triangulated mesh with $N_v = 6890$ vertices. The 3D body joints $\bm J(\bm\theta, \bm\beta)$ of the model can be defined as a linear combination of the mesh vertices. Hence for $N_j$ joints, we defined the body joints $\bm J(\bm\theta, \bm\beta) \in \mathbb{R}^{3\times N_j} = \mathcal{J}(\bm M(\bm\theta, \bm\beta))$, where $\mathcal{J}$ is a pre-trained linear regressor. Let $\bm R \in SO(3)$ and $\bm T \in \mathbb R^3$ denote the global rotation and translation, respectively.

Given an image and the detected 2D bounding boxes, the 2D human keypoints $\bm W$ can be estimated with the cropped regions. The objective function for human mesh recovery generally consists of a reprojection term $L_{2d}$ and a prior term $L_p$ with respect to variables $\bm \theta$, $\bm \beta$, $\bm R$ and $\bm T$.

The reprojection term penalizes the weighted 2D distance between the estimated 2D keypoints $\bm{W}$ with the confidence $c$, and the corresponding projected SMPL joints:
\begin{equation}
    L_{2d} = \sum_i c_i\rho(\bm{W}_i - \Pi_K(\bm{R}\bm{J}(\bm\theta, \bm\beta)_i + \bm{T})),
\end{equation}
where $\Pi_K$ is the projection from 3D to 2D through the intrinsic parameter $K$. $\rho$ denotes the Geman-McClure robust error function for suppressing noisy detections. 

The human body priors are used to encourage realistic 3D human mesh results. Since the pose and shape parameters ($ \bm{\Tilde{\theta}}, \bm{\Tilde{\beta}}$) estimated by a neural network can be viewed as learned prior, the final results are supposed to be close to them:
\begin{equation}
    L_{p} = ||\bm{\theta} - \bm{\Tilde{\theta}}||_2^2  + \lambda_{\beta}|| \bm{\beta} - \bm{\Tilde{\beta}}||_2^2,
\end{equation}
where $\lambda_{\beta}$ is a weight.

\subsection{Mirror-induced constraints}
\label{sec:mi_geo}
If there is a mirror in the image, the relation between the person and the mirrored person can be used to enhance the reconstruction performance. This relation is a simple reflection transformation if the mirror geometry is known, which however is impracticable for an arbitrary image from the Internet. To tackle this problem and take advantage of the characteristic of the mirror, the following mirror-induced constraints are introduced, as illustrated in Fig.~\ref{fig:mirrorsym}. Note that all symbols with the superscript prime refer to variables related to the mirrored person unless specifically mentioned.

\paragraph{Mirror symmetry constraints:}
Since the adopted human representation disentangles the orientation $\bm R$, pose parameters $\bm \theta$ and shape parameters $\bm \beta$, $\bm \beta$ can be shared by the person and the mirrored person, and $\bm \theta$ is related to $\bm \theta'$ by a simple reflection operation as follows:
\begin{equation}
\label{eq:param}
    \bm{\beta}' = \bm{\beta}, ~\bm{\theta}' = \mathcal{S}(\bm \theta),
\end{equation}
where $\mathcal{S}(\cdot)$ denotes the reflection operation on axis angles. 
\begin{figure}[t]
\centering
\includegraphics[trim=3cm 18.5cm 10cm 3.5cm, width=0.8\linewidth,clip]{figures/mirrorloss.pdf}
\caption{\textbf{An illustration of mirror-induced constraints.} The line segment connecting the joint $\bm{J}_i$ and its mirrored joint $\bm{J}_i'$ has the direction $\bm n_i$ and the middle point $\bm p_i$. Theoretically, $\bm{n}_i // \bm{n}_j$, and $\bm{n}_i \perp \overline{\bm{p}_i\bm{p}_j}$. If the mirror normal $\bm{n}$ (red arrow) is known, $\bm{n} // \bm{n}_i$ and $\bm{n} // \bm{n}_j$ should be satisfied as well.}
\label{fig:mirrorsym}
\end{figure}

As Eq.~\ref{eq:param} does not take $\bm R$ and $\bm{T}$ into consideration, the constraint on 3D keypoints can be imposed to estimate the human orientation and position better. We abbreviate the global coordinates of the $i$-th joint $\bm{R}\bm{J}(\bm\theta, \bm\beta)_i + \bm{T}$ as $\bm{J}_i$. Given a pair of body joints $i, j$, we denote the direction of the line segment $\overline{\bm{J}_i\bm{J}_i'}$, $\overline{\bm{J}_j\bm{J}_j'}$ as $\bm{n}_i$, $\bm{n}_j$ and the middle point of them as $\bm p_i$, $\bm p_j$, respectively. Ideally, $\bm n_i$ should be parallel to $\bm n_j$ and $\bm p_i, \bm{p}_j$ are supposed to be on the mirror plane. Despite the fact that the mirror geometry is unknown, it needs to be satisfied that $\bm{n}_i$ is perpendicular to the line  $\overline{\bm{p}_i\bm{p}_j}$. So for 
any pair of joints, we minimize the sum of the L2 norm of the cross product between $\bm{n}_i$ and $\bm{n}_j$, and the inner product between $\bm{n}_i$ and $\bm{p}_j - \bm{p}_i$: 
\begin{equation}\label{eq:mirrorsym}
    L_{s} = \sum_{(i, j)}(||\bm{n}_i \times \bm{n}_j||_2 + || \bm{n}_i\cdot (\bm{p}_j - \bm{p}_i) ||_2).
\end{equation}

\paragraph{Mirror normal constraint:}
A mirror can be represented as a plane, parameterized as its normal and position. If its normal $\bm{n}$ is known, the geometric properties of the mirror can thus be utilized explicitly by constraining $\bm n_i$ and $\bm n$ to be parallel with the following loss function:
\begin{equation}\label{eq:mirrorgt}
    L_{n} =  \sum_i||\bm{n} \times \bm{n}_i||_2. 
\end{equation}
\vspace{-0.5cm}
\begin{figure}[t]
	\centering
	\includegraphics[width=1\linewidth,trim={8cm 8cm 8cm 7.5cm},clip]{figures/vp_small.pdf}
	\includegraphics[width=\linewidth, trim={2.5cm 2cm 1cm 3cm},clip]{figures/vpdemo.pdf}
	\vspace{-0.8cm}
	\caption{\textbf{Vanishing points in an image containing a person and a mirror.} In most cases at least two vanishing points can be found, where $\bm v_0$ comes from 2D human keypoints, and $\bm v_1$ comes from the annotated mirror edges. $O_c$ denotes the camera center. Note that $\overline{O_c v_0} // \bm n$ and $\overline{O_c v_1} \perp \bm n$, where $\bm n$ is the mirror normal.
	}
	\label{fig:vp}
\end{figure}
\paragraph{Mirror normal estimation:}
Though the mirror normal is not directly available, the vanishing points can be used to estimate it. The vanishing point of lines with direction $\bm n$ in 3D space is the intersection $\bm v$ of the image plane with a ray through the camera center with direction $\bm n$~\cite{hartley2003multiple}:
\begin{equation}
\bm v=K \bm n,
\label{eq:vanish}
\end{equation}
where the vanishing point $\bm v\in \mathbb{R}^3$ is in the form of homogeneous coordinates and $K$ is the camera intrinsic matrix. 

Eq.~\ref{eq:vanish} reveals that obtaining the mirror normal $\bm n$ requires both $K$ and $\bm v$. As the parallel lines connecting points on the real object and corresponding points on the mirrored object are perpendicular to the mirror, the vanishing point $\bm v$ with this direction can be estimated through their 2D positions.
To get such correspondences, some previous works require additional inputs such as masks \cite{Hu2005MultipleView3R}, which is infeasible for images from the Internet.
Fortunately, since 2D human keypoints provide robust semantic correspondences, \eg the left ankle of the real person and the right ankle of the mirrored person, this vanishing point can be acquired naturally and automatically in our setting ($\bm v_0$ in Fig.~\ref{fig:vp}).

Note that if the intrinsic matrix $K$ is provided, the mirror normal can thus be solved easily through $\bm n=K^{-1} \bm v_0$, otherwise $K$ should be calibrated from a single image if possible. From the projective geometry~\cite{hartley2003multiple}, we know that it is possible to calibrate the camera intrinsic parameters from a single image. Suppose the camera has zero skew and square pixels. The intrinsic matrix $K$ can be computed via three orthogonal vanishing points. Additionally, if the principal point is assumed to be in the image center (only the focal length is unknown), $K$ can be computed via only two orthogonal vanishing points. Please refer to the supplementary material for more details. 

As we have stated, one vanishing point $\bm v_0$ has been acquired based on reliable 2D human keypoints. Different from the general scene where finding orthogonal relations may be difficult, our setting contains richer information. Fig.~\ref{fig:vp} shows that if we annotate the mirror edges, at least one vanishing point $\bm v_1$ orthogonal to $\bm v_0$ can be obtained. With these vanishing points, the calibration can be performed. Note that images from the same video share the same intrinsic matrix $K$, thus the annotation process is not laborious.

The mirror normal constraint is optional, which depends on how easy it is to find mirror edges. In the experiment, we will show that our method can still achieve satisfactory performance without the mirror normal constraint.

\subsection{Objective function and optimization}
\label{sec:opt}
Combining all discussed above, the final objective function to optimize can be written as:
\begin{equation}
\label{eq:loss0}
\begin{split}
    \min_{\substack{\Theta, \Theta'}}~L_{2d}+L_{2d}' + \lambda_p (L_{p}&+L_{p}') + \lambda_s L_s + \lambda_n L_n \\
     s.t.~~ \bm{\beta}' = \bm{\beta}&, ~\bm{\theta}' = \mathcal{S}(\bm \theta),
\end{split}
\end{equation}
where $\Theta=\{\bm\theta, \bm\beta, \bm R, \bm T\}$ and $\Theta'=\{\bm\theta', \bm\beta', \bm R', \bm T'\}$. $L_{2d}'$ and $L_p'$ refer to the reprojection term and the prior term of the mirrored person, respectively. $\lambda_p$, $\lambda_s$ and $\lambda_n$ are weights. $\lambda_n$ is set to zero whenever the mirror normal is unavailable. If there are two or more people, the optimization can be done for each subject separately.

We optimize Eq.~\ref{eq:loss0} with respect to all parameters using L-BFGS and PyTorch. An off-the-shelf model~\cite{kolotouros2019spin} is adopted to generate the initial estimation. Given the 2D keypoints~\cite{sun2019deep, cao2017realtime}, $\bm R$ and $\bm T$ are further optimized by aligning the initial SMPL model to the 2D keypoints. To improve the robustness of the initialization, we select the person with smaller reprojection error and apply the selected pose parameter to the other person after a reflection operation. 
























































