\section{Evaluation}
\begin{figure}[t]
	\centering
	\includegraphics[width=1.0\linewidth,trim={0cm 0cm 0cm 0cm},clip]{figures/mv_mirror.jpg}
	\vspace{0.2mm}
	\caption{
	\textbf{The configuration of our evaluation set.} The poses of all subjects are captured with six cameras. The camera extrinsic paramters, camera intrinsic parameters, and mirror geometry are calibrated.
	}
	\label{fig:mv_mirror}
\end{figure}

\subsection{Dataset for evaluation}
Since no dataset exists for our task which contains both mirrors and labeled 3D human keypoints, we collect a new dataset with six calibrated HD cameras, as shown in Fig.~\ref{fig:mv_mirror}. All videos are recorded with a resolution of 1920$\times$1080 pixels at 30 fps. 
Each person is performing various actions in front of a large mirror. A calibration board is placed on the mirror, thus the mirror geometry can be easily determined, which can also be used to evaluate the estimated mirror geometry. Note that the ground-truth of 3D keypoints is generated from all views.
2D bounding boxes, 2D keypoints, and the correspondences between multiple people are annotated manually.

\subsection{Reconstruction evaluation and ablation study}
\label{sec:mv-mirror}

The 3D keypoints generated by this multi-view system are used as ground-truth to evaluate the reconstruction accuracy. Metrics include MPJPE, PA-MPJPE, and MRPE. MPJPE is the distance (mm) between predicted and ground-truth 3D keypoints after root alignment. PA-MPJPE is calculated with further Procrustes alignment. MRPE is defined as the distance (mm) between the predicted and the ground-truth root joint, which evaluates the absolute root position accuracy.
\begin{table}[t]
	\begin{center}
	\resizebox{\columnwidth}{!}{
	\begin{tabular}{lccc}
	\hline
 		Methods & MPJPE~$\downarrow$ & PA-MPJPE~$\downarrow$ & MRPE~$\downarrow$ \\ \hline
 		SMPLify-X \cite{SMPL-X:2019} & 143.73 & 90.57 & 368.00\\
 		SPIN \cite{kolotouros2019spin} & 109.79 & 67.42 & 167.62\\
 		Baseline (\cite{kolotouros2019spin}+\cite{SMPL-X:2019}) & 82.92& 61.47 & 147.44 \\ \hline
 		Ours (w/o $L_{s}$, $L_{n}$) & 53.81 & 34.48 & 108.43 \\
 		Ours (w/o $L_{n}$) & 39.52 & 33.24 & 101.91 \\
 		\rowcolor{gray!10}
 		\textbf{Ours (full)} & \textbf{38.77} & \textbf{32.96} & \textbf{93.22} \\
  	    \hline
	\end{tabular}
	}
	\end{center}
	\caption{\textbf{Quantitative analysis.} `Baseline' uses \cite{kolotouros2019spin} as initialization and \cite{SMPL-X:2019} as the optimization method to fit the body model to 2D keypoints for each person separately. 
	}
	\label{tab:ablation}
\end{table} 


\begin{figure}[t]
	\centering
	\includegraphics[width=0.8\linewidth, trim=10 0 50 50,clip]{figures/focal.pdf}
	\caption{\textbf{Sensitivity of pose error to focal length estimation.} The vertical black line indicates the ground-truth focal length.}
	\vspace{-1mm}
	\label{fig:focal}
\end{figure}
\paragraph{Pose Reconstruction:} 
Two previous methods are compared here. SMPLify-X~\cite{SMPL-X:2019} is the state-of-the-art optimization-based method and SPIN~\cite{kolotouros2019spin} is the state-of-the-art regression-based method. `Baseline' combines the two methods by using \cite{kolotouros2019spin} as initialization and \cite{SMPL-X:2019} as optimization. In Table~\ref{tab:ablation}, comparing `Ours (full)' with first three rows, the result shows that our approach outperforms previous methods by a large margin. 

The ablation study is performed to show the effect of our mirror-induced constraints. It can be observed in Table~\ref{tab:ablation} that without the mirror normal constraint (`Ours (w/o $L_{n}$)'), our method can still perform well, indicating our applicability in the case where vanishing points are hard to acquire. If $L_{s}$ is also discarded (`Ours (w/o $L_{s}$, $L_{n}$)'), MPJPE will degrade severely while PA-MPJPE has a relatively slight change, reflecting that the mirror symmetry constraint can adjust the human orientation effectively. `Ours (w/o $L_{s}$, $L_{n}$)' is better than the baseline since $\bm \theta$ and $\bm \beta$ are shared. For MRPE, more constraints will bring the better improvement and the position error of our full model is less than 10cm. 

We also measure the accuracy of mirror normal estimation. As we have stated, the ground-truth mirror normal comes from calibration. We use the average angle between the ground-truth and the estimated normal. The angle is 1.9$^\circ$ with the ground-truth focal length, and 4.1$^\circ$ with the estimated focal length, both of which are quite small.

\paragraph{Focal length estimation:} To evaluate the accuracy of focal length estimation, we simulate the case in which only the focal length is unknown and two orthogonal vanishing points are used. One vanishing point is computed by the lines parallel to the mirror and the other is from two ways: 1) if we label two lines that are perpendicular to the mirror from the scene, the error of focal length is 4.9\% of the true value, and 2) if we make use of 2D human keypoints, the error reduces to 3.6\%. It means that the 2D human keypoints provide reliable correspondences that are helpful to estimate the focal length.

To evaluate the influence of the predicted focal length on pose estimation, we modify the focal length to different values and calculate the reconstruction accuracy. As shown in Fig.~\ref{fig:focal}, when the focal length deviates from the ground-truth by less than 10\%, the change of the reconstruction error is less than 5\%, showing the stability of our algorithm.
\begin{figure}[t]
	\centering
	\includegraphics[width=1\linewidth]{figures/comb.jpg}
	\caption{\textbf{Samples in our Mirrored-Human dataset.} From left to right, each column shows the variety of actions, appearances, viewpoints, poses and multiple mirrors/people, respectively. }
	\label{fig:dataset}
\end{figure}
\section{Dataset}
\label{sec:dataset}
%\sarah{add statistics about distribution of merge patterns}
%\alexey{I added some numbers in the section 4 (around line 270). Detailed numbers are in Appendix. We can move it up here if needed...}
%To create a dataset for self-supervised pretraining, we clone all non-fork repositories with more than 20 stars in GitHub that have C, C++, C\#, Python, Java, JavaScript, TypeScript, PHP, Go, and Ruby as their top language. The resulting dataset comprises over 64 million source code files. 
%\chris{why do we list languages here that we don't ever evaluate on?  A reviewer will find this confusing and ask about it.  We found that language specific models work better than multi-lingual models, right?}

The finetuning dataset is mined from over 100,000 open source software repositories in multiple programming languages with merge conflicts. It contains commits from git histories with exactly two parents, which resulted in a merge conflict.  We replay \texttt{git merge} on the two parents to see if it generates any conflicts. Otherwise, we ignore the merge from our dataset. We use the approach introduced by~\citet{Dinella2021} to extract resolution regions---however, we do not restrict ourselves to conflicts with less than 30 lines only.  Lastly, we extract token-level conflicts and conflict resolution classification labels (introduced in Section \ref{formulation}) from line-level conflicts and resolutions. Tab.~\ref{tab:fintuning_dataset} provides a summary of the finetuning dataset.

\begin{table}[htb]
\centering
\caption{Number of merge conflicts in the dataset.}
\begin{tabular}{llllllllllll} \toprule
\textbf{Programming language} & \textbf{Development set}  & \textbf{Test set} \\ \midrule
C\# & 27874 & 6969 \\ 
JavaScript & 66573 & 16644\\ 
TypeScript & 22422 & 5606\\ 
Java & 103065 & 25767 \\ 
\bottomrule
\end{tabular}
\label{tab:fintuning_dataset}
\end{table}
The finetuning dataset is split into development and test sets in the proportion 80/20 at random at the file-level. The development set is further split into training and validation sets in 80/20 proportion at the merge conflict level.    


\subsection{Qualitative comparison}
Some representative results are visualized in Fig.~\ref{fig:results}, showing that our method can reconstruct accurate 3D human meshes as well as the mirror geometry. 
Compared with the monocular human mesh recovery algorithms~\cite{SMPL-X:2019, kolotouros2019spin}, our method produces much more consistent positions and orientations  (Fig.~\ref{fig:results}) and more accurate poses (Fig.~\ref{fig:comparisonpose}). More qualitative results can be found in the supplementary material.
