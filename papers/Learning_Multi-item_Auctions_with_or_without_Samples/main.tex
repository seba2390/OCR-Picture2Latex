%\documentclass[format=acmsmall, review=false]{acmart}

%\usepackage{acm-ec-17}
\documentclass[11pt]{article}

\usepackage{times}
\usepackage{amsmath,amssymb,amsthm,amsfonts,latexsym,bbm,xspace,graphicx,float,mathtools,dsfont}
\usepackage[backref,colorlinks,citecolor=blue,bookmarks=true]{hyperref}
%\usepackage[dvips, paper=letterpaper, top=1in, bottom=.75in, left=1in, right=1in, nohead, includefoot, footskip=.25in]{geometry}
\usepackage[dvips, paper=letterpaper, top=1in, bottom=.75in, left=0.9in, right=0.95in, nohead, includefoot, footskip=.25in]{geometry}
\usepackage{color}
\usepackage{comment}
\usepackage{algorithm}
\usepackage{algorithmic}
\newcommand{\ind}{\mathds{1}}
\usepackage{tweaklist}
%\usepackage{fullpage}
\usepackage{appendix}
\usepackage{cleveref}

\usepackage{accents}
\newcommand{\ubar}[1]{\underaccent{\bar}{#1}}
\newcommand{\cnote}[1]{\textcolor{red}{#1}}


\newtheorem{theorem}{Theorem}
%\newtheorem{informal}[theorem]{Informal Theorem}
\newtheorem{corollary}{Corollary}
\newtheorem{lemma}{Lemma}
\newtheorem{conjecture}{Conjecture}
\newtheorem{claim}{Claim}
\newtheorem{proposition}{Proposition}
\newtheorem{fact}{Fact}
\newtheorem{definition}{Definition}
\newtheorem{assumption}{Assumption}
\newtheorem{observation}{Observation}
\newtheorem{example}{Example}
\newtheorem{remark}{Remark}
\newtheorem{question}{Question}
\newtheorem{hypothesis}{Hypothesis}
\newtheorem{informaltheorem}{Informal Theorem}
\newcommand{\single}{\textsc{Single}}
\newcommand{\nf}{\textsc{Non-Favorite}}
\newcommand{\core}{\textsc{Core}}
\newcommand{\tail}{\textsc{Tail}}
\newcommand{\rev}{\textsc{Rev}}
\newcommand{\bvcg}{\textsc{BVCG}}
\newcommand{\copies}{\textsc{OPT}^{\textsc{Copies-UD}}}
\newcommand{\vcg}{\textsc{VCG}}
\newcommand{\srev}{\textsc{SRev}}
\newcommand{\brev}{\textsc{BRev}}
\newcommand{\aperev}{\textsc{APostEnRev}}
\newcommand{\prev}{\textsc{PostRev}}
\newcommand{\Med}{\textsc{Median}}
\newcommand{\Dim}{\text{Dim}}
\newcommand{\sold}{\textsc{SOLD}}
\newcommand{\bbeta}{\boldsymbol{\beta}}
%%%\newcommand{\qed}{\rule{7pt}{7pt}}
%%%\newcommand{\order}[1]{O(#1)}
\newenvironment{prevproof}[2]{\noindent {\em {Proof of {#1}~\ref{#2}:}}}{$\Box$\vskip \belowdisplayskip}

%\newenvironment{proof}{\ni {\em Proof:}}{\hfill $\Box$ \bigbreak}
%\newcommand{\qed}{\hfill $\Box$}
%%%\newenvironment{example}{{\bf Example:}}{\hfill $\Box$ \bigbreak}

\newcommand{\mychoose}[2]{\left( \begin{array}{c} #1 \\ #2 \end{array} \right)}
\newcommand{\Bs}{\bold{s_i}}
\newcommand{\Br}{\bold{r_i}}
\newcommand{\SI}{\bf{\sigma}}
\newcommand{\N}{\ensuremath{\mathbb{N}}} % The naturals
\newcommand{\Z}{\ensuremath{\mathbb{Z}}} % The integers
\newcommand{\R}{\ensuremath{\mathbb{R}}} % The reals
\newcommand{\Q}{\ensuremath{\mathbb{Q}}} % The rationals
\newcommand{\dx}{\text{dx}}
\newcommand{\lose}{\text{LOSE}}
\newcommand{\poly}{{\rm poly}}
\newcommand{\opt}{\text{OPT}}
\newcommand{\todo}[1]{{\bf \color{red} TODO: #1}}
\newcommand{\yangnote}[1]{{\color{blue}{#1}}}

\newcommand{\mingfeinote}[1]{{\color{black}{#1}}}
\newcommand{\notshow}[1]{{}}
\newcommand{\rt}{\rm{rt}}

\DeclareMathOperator{\Prob}{Pr}
\DeclareMathOperator{\E}{E}
%\def \ni {\noindent}
\def \bfm {\bf \boldmath}
\def \TT  {{\cal T}}
\def \DD  {{\mathcal{D}}}
%\def \DD  {{\mathcal{D}}}
\def \KK  {{\mathcal{K}}}
\def \AA  {{\mathcal{A}}}
\def \CC {{\mathcal{C}}}
\def \HH {{\mathcal{H}}}
\def \XX {{\mathcal{X}}}
\def \PP  {{\cal P}}
\def \GG  {{\cal G}}
\def \OO  {{\cal O}}
\def \NN  {{\cal N}}
\def \LL  {{\mathcal{L}}}
\def \BB  {{\cal B}}
\def \VV  {{\cal V}}
\def \EE  {{\mathcal{E}}}
\def \ee {\varepsilon}
\def \t  {\tilde}
\def \FF  {{\mathcal{F}}}
\def \SS  {{\mathcal{S}}}
\def \obj {{\cal O}}
\def \pc   {{\sf P}}                              %%% P complexity class
\def \npc  {{\sf NP}}                             %%% NP complexity class
\def \ncc  {{\sf NC}}                             %%% NC complexity class
\def \zppc  {{\sf ZPP}}                           %%% ZPP complexity class
\def \qpc  {{\sf DTIME}(n^{O({\sf polylog}(n))})} %%% QusiPoly complexity class
\def \snp  {{\sf Max SNP}}                        %%% Max SNP complexity class
\def \ptas {{\sf PTAS}~}
\def \gr   {{\sf Greedy-1}}
\def \grr   {{\sf Greedy}}
\def \E  {{\mathbb{E}}}
\def \Var{{\text{Var}}}
\def \polytope{{P(D)}}
\def \L{{\mathcal{L}}}

\DeclareMathOperator{\argmax}{argmax}
\DeclareMathOperator{\argmin}{argmin}

\def \ee {{\rm e}} % Euler number
\def\tp{{\tilde{\varphi}}}
\def\I{{\mathbb{I}}}
\definecolor{MyGray}{rgb}{0.8,0.8,0.8}

\newcommand{\graybox}[2]{\begin{figure}[t]
\colorbox{MyGray}{
\begin{minipage}{\textwidth}
{#2}
\end{minipage}
}\caption{{#1}}
\end{figure}
}
\setlength{\parskip}{0in}

\title{Learning Multi-item Auctions with (or without) Samples}

\author{Yang Cai\\
McGill University, Canada\\
cai@cs.mcgill.ca \and Constantinos Daskalakis\\ EECS and CSAIL, MIT, USA\\ costis@csail.mit.edu}

\begin{document}
\maketitle

\begin{abstract}
We provide algorithms that learn simple auctions whose revenue is approximately optimal in multi-item multi-bidder settings, for a wide range of bidder valuations including unit-demand, additive, constrained additive, XOS, and subadditive. We obtain our learning results in two settings. The first is the commonly studied setting where sample access to the bidders' distributions  over valuations is given, for both regular distributions and arbitrary distributions with bounded support. Here, our algorithms require polynomially many samples in the number of items and bidders. The second is a more general max-min learning setting that we introduce, where we are given ``approximate distributions,'' and we seek to compute a mechanism whose revenue is approximately optimal simultaneously for all ``true distributions'' that are close to the ones we were given. These results are more general in that they imply the sample-based results, and are also applicable in settings where we have no sample access to the underlying distributions but  have estimated them indirectly via market research or by observation of bidder behavior in previously run, potentially non-truthful auctions. 

All our results hold for valuation distributions satisfying the standard (and necessary) independence-across-items property. They also generalize and improve upon recent works of Goldner and Karlin~\cite{GoldnerK16} and Morgenstern and Roughgarden~\cite{MorgensternR16}, which have provided algorithms that learn approximately optimal multi-item mechanisms in more restricted settings with additive, subadditive and unit-demand valuations using sample access to distributions. We generalize these results to the complete unit-demand, additive, and XOS setting, to i.i.d. subadditive bidders, and to the max-min setting. 

Our results are enabled by new uniform convergence bounds for hypotheses classes under product measures. Our bounds result in exponential savings in sample complexity compared to bounds derived by bounding the VC dimension, and are of independent interest.
\end{abstract}
\thispagestyle{empty}
\addtocounter{page}{-1}
\newpage

% !TEX root = ../arxiv.tex

Unsupervised domain adaptation (UDA) is a variant of semi-supervised learning \cite{blum1998combining}, where the available unlabelled data comes from a different distribution than the annotated dataset \cite{Ben-DavidBCP06}.
A case in point is to exploit synthetic data, where annotation is more accessible compared to the costly labelling of real-world images \cite{RichterVRK16,RosSMVL16}.
Along with some success in addressing UDA for semantic segmentation \cite{TsaiHSS0C18,VuJBCP19,0001S20,ZouYKW18}, the developed methods are growing increasingly sophisticated and often combine style transfer networks, adversarial training or network ensembles \cite{KimB20a,LiYV19,TsaiSSC19,Yang_2020_ECCV}.
This increase in model complexity impedes reproducibility, potentially slowing further progress.

In this work, we propose a UDA framework reaching state-of-the-art segmentation accuracy (measured by the Intersection-over-Union, IoU) without incurring substantial training efforts.
Toward this goal, we adopt a simple semi-supervised approach, \emph{self-training} \cite{ChenWB11,lee2013pseudo,ZouYKW18}, used in recent works only in conjunction with adversarial training or network ensembles \cite{ChoiKK19,KimB20a,Mei_2020_ECCV,Wang_2020_ECCV,0001S20,Zheng_2020_IJCV,ZhengY20}.
By contrast, we use self-training \emph{standalone}.
Compared to previous self-training methods \cite{ChenLCCCZAS20,Li_2020_ECCV,subhani2020learning,ZouYKW18,ZouYLKW19}, our approach also sidesteps the inconvenience of multiple training rounds, as they often require expert intervention between consecutive rounds.
We train our model using co-evolving pseudo labels end-to-end without such need.

\begin{figure}[t]%
    \centering
    \def\svgwidth{\linewidth}
    \input{figures/preview/bars.pdf_tex}
    \caption{\textbf{Results preview.} Unlike much recent work that combines multiple training paradigms, such as adversarial training and style transfer, our approach retains the modest single-round training complexity of self-training, yet improves the state of the art for adapting semantic segmentation by a significant margin.}
    \label{fig:preview}
\end{figure}

Our method leverages the ubiquitous \emph{data augmentation} techniques from fully supervised learning \cite{deeplabv3plus2018,ZhaoSQWJ17}: photometric jitter, flipping and multi-scale cropping.
We enforce \emph{consistency} of the semantic maps produced by the model across these image perturbations.
The following assumption formalises the key premise:

\myparagraph{Assumption 1.}
Let $f: \mathcal{I} \rightarrow \mathcal{M}$ represent a pixelwise mapping from images $\mathcal{I}$ to semantic output $\mathcal{M}$.
Denote $\rho_{\bm{\epsilon}}: \mathcal{I} \rightarrow \mathcal{I}$ a photometric image transform and, similarly, $\tau_{\bm{\epsilon}'}: \mathcal{I} \rightarrow \mathcal{I}$ a spatial similarity transformation, where $\bm{\epsilon},\bm{\epsilon}'\sim p(\cdot)$ are control variables following some pre-defined density (\eg, $p \equiv \mathcal{N}(0, 1)$).
Then, for any image $I \in \mathcal{I}$, $f$ is \emph{invariant} under $\rho_{\bm{\epsilon}}$ and \emph{equivariant} under $\tau_{\bm{\epsilon}'}$, \ie~$f(\rho_{\bm{\epsilon}}(I)) = f(I)$ and $f(\tau_{\bm{\epsilon}'}(I)) = \tau_{\bm{\epsilon}'}(f(I))$.

\smallskip
\noindent Next, we introduce a training framework using a \emph{momentum network} -- a slowly advancing copy of the original model.
The momentum network provides stable, yet recent targets for model updates, as opposed to the fixed supervision in model distillation \cite{Chen0G18,Zheng_2020_IJCV,ZhengY20}.
We also re-visit the problem of long-tail recognition in the context of generating pseudo labels for self-supervision.
In particular, we maintain an \emph{exponentially moving class prior} used to discount the confidence thresholds for those classes with few samples and increase their relative contribution to the training loss.
Our framework is simple to train, adds moderate computational overhead compared to a fully supervised setup, yet sets a new state of the art on established benchmarks (\cf \cref{fig:preview}).

\section{Preliminaries}
Given a graph $G=(V,E)$, and vertex $u \in V$, let $\deg(u,G)$ be the degree of $u$ in $G$. 
Given a tree $T$ and $u, v \in T$, denote the $u$-$v$ path in $T$ by $\pi(u,v,T)$. When the tree $T$ is clear from the context, we may omit it and write $\pi(u,v)$. For a (possibly weighted) subgraph $G' \subseteq G$ and a vertex pair $s,t \in V$, let $\dist_{G'}(s,t)$ denote the length of the $s$-$t$ shortest path in $G'$. 

\paragraph{Fault-Tolerant Labeling Schemes.}
For a given graph $G$, let $\Pi: V\times V \times \mathcal{G} \to \mathbb{R}_{\geq 0}$ %\mtodo{a reviewer pointed out that the last element in the domain should probably be subgraphs of $G$ and not $G$, see which notation we want here.} \mertodo{I am not sure that it is needed as we consider $G$ in the fault-free setting and $G \setminus F$ in the FT setting. We can of course write $\Pi: V\times V \times \mathcal{G} \to \mathbb{R}_{\geq 0}$ where $\mathcal{G}$ is the family of all $G$-subgraphs, but I cannot see why we need it.} 
be a function defined on pairs of vertices and a subgraph $G' \subset G$, where $\mathcal{G}$ is the family of all subgraphs of $G$. For an integer parameter $f\geq 1$, an $f$-\emph{fault-tolerant labeling scheme} for a function $\Pi$ and a graph family $\mathcal{F}$ is a pair of functions $(L_{\Pi},D_{\Pi})$. The function $L_{\Pi}$ is called the \emph{labeling function}, and $D_{\Pi}$ is called the \emph{decoding function}. For every graph $G$ in the family $\mathcal{F}$, the labeling function $L_{\Pi}$ associates with each vertex $u \in V(G)$ and every edge $e \in E(G)$, a label $L_{\Pi}(u,G)$ (resp., $L_{\Pi}(e,G)$). It is then required that given the labels of any triplets $s,t, F \in V \times V \times E^f$, the decoding function $D_{\Pi}$ computes $\Pi(s,t, G \setminus F)$.  The primary complexity measure of a labeling scheme is the \emph{label length}, measured by the length (in bits) of the largest label it assigns to some vertices (or edges) in $G$ over all graphs $G \in \mathcal{F}$. An $f$-FT connectivity labeling scheme is required to output YES iff $s$ and $t$ are connected in $G \setminus F$.  In $f$-FT \emph{approximate distance labeling scheme} it is required to output an estimate for the $s$-$t$ distance in the graph $G \setminus F$. Formally, an $f$-FT labeling scheme is $q$\emph{-approximate} if the value $\delta(s,t,F)$ returned by the decoder algorithm satisfies that $\dist_{G \setminus F}(s,t)\leq \delta(s,t,F) \leq q \cdot \dist_{G \setminus F}(s,t)$.  Throughout the paper we provide randomized labeling schemes which provide a high probability guarantee of correctness for any fixed triplet $\langle s,t, F \rangle$. 


\paragraph{Fault-Tolerant Routing Schemes.} In the setting of FT routing scheme, one is given a pair of source $s$ and destination $t$ as well as $F$ edge faults, which are initially unknown to $s$. The routing scheme consists of \emph{preprocessing} and \emph{routing} algorithms. The preprocessing algorithm defines labels $L(u)$ to each of the vertices $u$, and a header $H(M)$ to the designated message $M$. In addition, it defines for every vertex $u$ a routing table $R(u)$. The labels and headers are usually required to be short, i.e., of poly-logarithmic bits. 
The routing procedure determines at each vertex $u$ the port-number on which $u$ should send the messages it receives. The computation of the next-hop is done by considering the header of the message $H(M)$, the label of the source and destination $L(s)$ and $L(t)$ and the routing table $R(u)$. The routing procedure at vertex $u$ might also edit the header of the message $H(M)$. The failing edges are not known in advance and can only be revealed by reaching (throughout the message routing) one of their endpoints. The \emph{space} of the scheme is determined based on maximal length of message headers, labels and the individual routing tables. The stretch of the scheme is measured by the ratio between the length of the path traversed until the message arrived its destination and the length of the shortest $s$-$t$ path in $G \setminus F$. In the more relaxed setting of \emph{forbidden-set routing schemes} the failing edges are given as input to the routing algorithm.

%\paragraph{Forbidden-Set Routing Schemes.} One of the key applications of labeling schemes is routing.
%In the setting of forbidden-set routing schemes, given the labels of $s$, $t$ and a set of forbidden edges $
%F$, it is required to determine the next-hop neighbor of $s$ on some short $s$-$t$ path in $G \setminus F$. The main two complexity measures are the stretch induced by the $s$-$t$ path encoded by the labels.  \mtodo{Maybe the definition should be more similar to the next one? (consider labels, tables, headers)}
%That is, given the labels of $u$, $v$ and the faults $F$, the decoder function returns the port-number of $u$'s neighbor lying on a $u$-$v$ path $P$ in $G \setminus F$ such that $P \subseteq G$ and $len(P)\leq s \cdot \dist(u,v, G \setminus F)$ for some approximation factor $s$. 

\section{Uniform Convergence under Product Measures}\label{sec:uniform convergence under product measure}

In this section, we develop machinery for obtaining uniform convergence bounds for hypotheses over product measures. Our goal is to save on the sample complexity implied by VC dimension bounds, as summarized in Table~\ref{tab:productVC}. Indeed, we obtain low sample complexity  bounds for indicators over \emph{single-intersecting} sets (see Definition~\ref{def:single-intersecting}), which play a key role in proving our results for learning approximately revenue-optimal auctions. Our main results of this section are Theorem~\ref{thm:uniform convergence for product measure PARTITION} for general functions, and Corollary~\ref{cor:VC for product measure} for sets.

We first define what type of uniform convergence bounds we seek to prove.

\begin{definition}[$(\epsilon,\delta)$-uniform convergence with respect to proxy measure]
	A hypothesis class $\HH$ of functions mapping domain set $\XX$ to $\mathbb{R}$ has {\em $(\epsilon,\delta)$-uniform convergence with sample complexity $s(\epsilon,\delta)$} iff, for all $\epsilon,\delta > 0$, there exists a {\em processing} ${\cal P}:{\cal X}^{s(\epsilon,\delta)} \rightarrow \Delta(\XX)$ such that for any distribution $\DD \in \Delta(\XX)$ when $k=s(\epsilon,\delta)$: 
	\begin{align}
	\Pr_{z_1,\cdots, z_k \sim \DD}\left[\sup_{g\in \HH}\left|\E_{z\sim {\cal P}(z_1,\cdots, z_k)}[g(z)] - \E_{z\sim \DD}[g(z)]\right|\leq \epsilon\right]\geq 1-\delta. \notag
	\end{align}
%$$\Pr\left[\sup_{g\in \HH}\left|\frac{1}{k}\cdot \sum_{j=1}^{k} g(z_j) - \E_{z\sim \DD}[g(z)]\right|\leq \epsilon\right]\geq 1-\delta.$$

When ${\cal X}$ is the Cartesian product of a collection of sets ${\cal X}_1,\ldots,{\cal X}_k$, i.e.~${\cal X} =\times_i {\cal X}_i$, we say that a hypothesis class $\HH$ as above has {\em $(\epsilon,\delta)$-p.m.~uniform convergence with sample complexity $s(\epsilon,\delta)$} if the above holds for all ${\cal D}$ that are product measures over ${\cal X}$.	
\end{definition}

Next we provide a simple lemma, which leads to a simple version of our main result stated as Theorem~\ref{thm:uniform convergence for product measure}. Our main result, stated as Theorem~\ref{thm:uniform convergence for product measure PARTITION}, follows.

%In particular, we show that if each of the projected hypothesis class uniformly converges with polynomial many samples, the original hypothesis class also uniformly converges with a polynomial number of samples. 
%
%In the end of this section, we apply our uniform convergence results to a particular hypothesis class called 

\begin{lemma} \label{lem:approx product dist}
	Let $\XX_1,\ldots, \XX_d$ be $d$ domain sets and $\HH$ be a hypothesis class with functions mapping from the product space $\times_{i=1}^{d} \XX_i$ to $\mathbb{R}$. For all $i\in[d]$, let $\HH_i$ be the projected hypothesis class of $\HH$ on $\XX_i$, that is, $\HH_i=\left\{g\ |\ \exists f\in \HH\ \exists\ a_{-i}\in \times_{j\neq i} \XX_j \ \forall\ x_i\in \XX_i, \ g(x_i) = f(x_i,a_{-i})\right\}$. For every $i\in[d]$, let $\DD_i$ and $\hat{\DD}_i$ be two distributions supported on $\XX_i$. Suppose for all $i\in[d]$, $$\sup_{g\in\HH_i}\left|\E_{x\sim \DD_i}[g(x)]-\E_{x\sim \hat{\DD}_i}[g(x)]\right|\leq \epsilon,$$ then $$\sup_{f\in\HH} \left|\E_{\boldsymbol{x}\sim \times_{i=1}^d \DD_i}\left[f(\boldsymbol{x})\right]-\E_{\boldsymbol{x}\sim \times_{i=1}^d \hat{\DD}_i}\left[f(\boldsymbol{x})\right]\right|\leq d\cdot \epsilon.$$
\end{lemma} 

\begin{proof}
	Let $\FF_i$ and $\hat{\FF}_i$ be the probability measure function for $\DD_i$ and $\hat{\DD}_i$ respectively. We will prove the statement using a hybrid argument. We create a sequence of product distributions $\{\DD^{(j)}\}_{j\leq d}$, where $\DD^{(j)}=\hat{\DD}_1\times\cdots\times\hat{\DD}_j\times \DD_{j+1}\times\cdots\times \DD_{d},$ and $\DD^{(0)}=\DD$, $\DD^{(d)}=\hat{\DD}$. To prove our claim, it suffices to show that for any integer $j\in[d]$, $$\left|\E_{\boldsymbol{x}\sim \DD^{(j-1)}}\left[f(\boldsymbol{x})\right]-\E_{\boldsymbol{x}\sim \DD^{(j)}}\left[f(\boldsymbol{x})\right]\right|\leq \epsilon.$$
	%Let us first fix some notations. Let $\EE_{-j}=\left\{x_{-j}\ |\ \exists x_j\in \mathbb{R},\ (x_j,x_{-j})\in \EE\right\}$, $\EE_j(a_{-j})= \left\{x_{j} \in \mathbb{R} \ |\ (x_j,a_{-j})\in \EE\right\}$ for all $j\in[d]$. As $\EE$ is single-intersecting, $\EE_j(x_{-j})$ is an interval for all $j$ and $x_{-j}$. Moreover, since $||\DD_j-\hat{\DD}_j||_K\leq \xi$, we have $|\Pr_{\DD_j}[x_j\in \EE_j(x_{-j})]-\Pr_{\hat{\DD}_j}[x_j\in \EE_j(x_{-j})]|\leq 2\xi$ for all $j$ and $x_{-j}$. Next, we bound the difference for the probability of event $\EE$ under $\DD^{(j-1)}$ and $\DD^{(j)}$.
	
	Next, we show how to derive this inequality.
	\begin{align*}
		&\left|\E_{\boldsymbol{x}\sim \DD^{(j-1)}}\left[f(\boldsymbol{x})\right]-\E_{\boldsymbol{x}\sim \DD^{(j)}}\left[f(\boldsymbol{x})\right]\right|\\
		=&\Bigg{|}\int_{\times_{i\neq j} \XX_i}\left(\int_{\XX_j}f(x_{j},x_{-j}) d \FF_j(x_j)\right) d\hat{\FF}_1(x_1)\cdots d\hat{\FF}_{j-1}(x_{j-1}) d\FF_{j+1}(x_{j+1})\cdots d\FF_{d}(x_{d})\\
		 & ~~~~~~~~~~~~~~~~~~~~ - \int_{\times_{i\neq j} \XX_i}\left(\int_{\XX_j}f(x_{j},x_{-j}) d \hat{\FF}_j(x_j)\right) d\hat{\FF}_1(x_1)\cdots d\hat{\FF}_{j-1}(x_{j-1}) d\FF_{j+1}(x_{j+1})\cdots d\FF_{d}(x_{d})\Bigg{|}\\
		 = & \left|\int_{\times_{i\neq j} \XX_i} \left(\E_{x_j\sim \DD_j}\left[f(x_j,x_{-j})\right]-\E_{x_j\sim \hat{\DD}_j}[f(x_j,x_{-j})]\right)
		  d\hat{\FF}_1(x_1)\cdots d\hat{\FF}_{j-1}(x_{j-1}) d\FF_{j+1}(x_{j+1})\cdots d\FF_{d}(x_{d})\right|\\
		  \leq & \epsilon \cdot\int_{\times_{i\neq j} \XX_i} d\hat{\FF}_1(x_1)\cdots d\hat{\FF}_{j-1}(x_{j-1}) d\FF_{j+1}(x_{j+1})\cdots d\FF_{d}(x_{d})\\
		   = & \epsilon
	\end{align*}
\end{proof}



\begin{theorem}\label{thm:uniform convergence for product measure}
Let $\XX_1,\ldots, \XX_d$ be $d$ domain sets and $\HH$  a hypothesis class of functions mapping from the product space $\times_{i=1}^{d} \XX_i$ to $\mathbb{R}$. For all $i\in[d]$, let $\HH_i$ be the projected hypothesis class of $\HH$ on $\XX_i$, that is $\HH_i=\left\{g\ |\ \exists f\in \HH\ \exists\ a_{-i}\in \times_{j\neq i} \XX_j \ \forall\ x_i\in \XX_i, \ g(x_i) = f(x_i,a_{-i})\right\}$. 

Suppose that, for all $i\in [d]$, $\HH_i$ has $(\epsilon,\delta)$-uniform convergence with sample complexity $s_i(\epsilon,\delta)$.  Then $\HH$ has $(\epsilon,\delta)$-p.m.~uniform convergence with sample complexity $s(\epsilon,\delta) = \max_{i\in[d]} s_i(\epsilon/d,\delta/d)$.

In particular, let $\boldsymbol{z}^{(1)},\ldots, \boldsymbol{z}^{(\ell)}$ be a sample of size $\ell = s(\epsilon,\delta)$ from a product measure $\times_{i\in[d]} \DD_i$. Define $\hat{\DD}_i = {\cal P}_i({z}^{(1)}_i,\ldots,z^{(\ell)}_i)$, for all $i\in[d]$, where $z^{(j)}_i$ is the $i$-th entry of sample $\boldsymbol{z}^{(j)}$ and ${\cal P}_i$ is the processing corresponding to $\HH_i$'s uniform convergence. Then 
$$\Pr_{\boldsymbol{z}^{(1)},\ldots, \boldsymbol{z}^{(\ell)}}\left[\sup_{f\in \HH}\left|\E_{\boldsymbol{z}\sim \times_{i\in[d]}\hat{\DD}_i}\left[f(\boldsymbol{z})\right] - \E_{\boldsymbol{z}\sim \times_{i\in[d]}{\DD}_i}\left[f(\boldsymbol{z})\right]\right|\leq \epsilon\right]\geq 1-\delta.$$
\end{theorem}

\begin{prevproof}{Theorem}{thm:uniform convergence for product measure}
	Since $\ell\geq s_i(\epsilon/d,\delta/d)$, $\Pr\left[\sup_{g\in \HH_i}\left|\E_{z\sim \hat{\DD}_i}[g(z)]- \E_{z\sim \DD_i}[g(z)]\right|\leq \epsilon/d\right]\geq 1-\delta/d$ for all $i\in[d]$. By the union bound, with probability at least $1- \delta$, $\sup_{g\in \HH_i}\left|\E_{z\sim \hat{\DD}_i}[g(z)]- \E_{z\sim \DD_i}[g(z)]\right|\leq \epsilon/d$ for all $i\in[d]$. According to Lemma~\ref{lem:approx product dist}, $\sup_{f\in \HH}\left|\E_{\boldsymbol{z}\sim \times_{i\in[d]}\hat{\DD}_i}\left[f(\boldsymbol{z})\right] - \E_{\boldsymbol{z}\sim \times_{i\in[d]}{\DD}_i}\left[f(\boldsymbol{z})\right]\right|\leq \epsilon$ with probability at least $1- \delta$.
\end{prevproof}

\begin{theorem}\label{thm:uniform convergence for product measure PARTITION}
Let $\XX_1,\ldots, \XX_d$ be $d$ domain sets and $\HH$ a hypothesis class of functions mapping from the product space $\times_{i=1}^{d} \XX_i$ to $\mathbb{R}$. For all $T\subseteq [d]$, let $\HH_T$ be the projected hypothesis class of $\HH$ on $\XX_T \equiv \times_{i \in T}\XX_i$, that is, $\HH_T=\left\{g\ |\ \exists f\in \HH\ \exists\ a_{-T}\in \times_{j\notin T} \XX_j \ \forall\ x_T\in \XX_T, \ g(x_T) = f(x_T,a_{-T})\right\}$. Suppose that, for all $T \subseteq [d]$, $\HH_T$ has $(\epsilon,\delta)$-p.m.~uniform convergence with sample complexity $s_T(\epsilon,\delta)$, and define 
\begin{align}s(\epsilon,\delta) = \min_{\begin{minipage}[h]{3.5cm}\centering $k$,~partitions\\$T_1 \sqcup T_2 \sqcup \ldots \sqcup T_k = [d]$\end{minipage}} \max_{i=1,\ldots,k} s_{T_i}(\epsilon/k,\delta/k). \label{eq:yang}
\end{align}
Then $\HH$ has $(\epsilon,\delta)$-p.m.~uniform convergence with sample complexity $s(\epsilon,\delta)$.

In particular, let $\boldsymbol{z}^{(1)},\ldots, \boldsymbol{z}^{(\ell)}$ be a sample of size $\ell = s(\epsilon,\delta)$ from a product measure $\times_{i\in[d]} \DD_i$. Suppose that the optimum of~\eqref{eq:yang} is attained at $k=\tilde{k}$ for partition $\tilde{T}_1 \sqcup \tilde{T}_2 \sqcup \ldots \sqcup \tilde{T}_{\tilde{k}} = [d]$. Define $\hat{\DD}_{\tilde{T}_i} = {\cal P}_{\tilde{T}_i}({\boldsymbol{z}}^{(1)}_{\tilde{T}_i},\ldots,{\boldsymbol{z}}^{(\ell)}_{\tilde{T}_i})$, for all $i\in[d]$,  where ${\boldsymbol{z}}^{(j)}_{\tilde{T}_i}$ contains the entries of sample $\boldsymbol{z}^{(j)}$ in coordinates $\tilde{T}_i$ and ${\cal P}_{\tilde{T}_i}$ is the processing corresponding to $\HH_{\tilde{T}_i}$'s uniform convergence. Then 
$$\Pr_{\boldsymbol{z}^{(1)},\ldots, \boldsymbol{z}^{(\ell)}}\left[\sup_{f\in \HH}\left|\E_{\boldsymbol{z}\sim \times_{i\in[\tilde{k}]}\hat{\DD}_{\tilde{T}_i}}\left[f(\boldsymbol{z})\right] - \E_{\boldsymbol{z}\sim \times_{i\in[d]}{\DD}_i}\left[f(\boldsymbol{z})\right]\right|\leq \epsilon\right]\geq 1-\delta.$$


%Then $\HH$ has $(\epsilon,\delta)$-uniform convergence with sample complexity $s(\epsilon,\delta)$.
\end{theorem}
\begin{prevproof}{Theorem}{thm:uniform convergence for product measure PARTITION}
For every possible partition use Theorem~\ref{thm:uniform convergence for product measure}.
\end{prevproof}


Nest, we specialize Theorem~\ref{thm:uniform convergence for product measure PARTITION} to indicator functions over sets.

%In the next Corollary, we show that if all functions in $\HH$ have $0$ or $1$ values and each $\HH_i$ has VC dimension $V_i$, 
\begin{corollary}\label{cor:VC for product measure}
	We use the same notation as in Theorem~\ref{thm:uniform convergence for product measure PARTITION}. Suppose that all functions in $\HH$ map $\times_{i=1}^{d} \XX_i$ to $\{0,1\}$, i.e.~they are indicators over sets. Suppose also that the VC dimension of $\HH_T$ (viewed as a collection of sets) is $V_T$. Define 
	\begin{align}V_{\max}=\min_{\begin{minipage}[h]{3.5cm}\centering $k$,~partitions\\$T_1 \sqcup T_2 \sqcup \ldots \sqcup T_k = [d]$\end{minipage}} \left\{ k^2 \cdot\max_{i=1,\ldots,k} V_{T_i}\right\}. \label{eq:costas}
	\end{align}
	Assume that the optimum of~\eqref{eq:costas} is attained at $k=\tilde{k}$ for partition $\tilde{T}_1 \sqcup \tilde{T}_2 \sqcup \ldots \sqcup \tilde{T}_{\tilde{k}} = [d]$. 
	
	Then
	%
	%Theorem~\ref{thm:uniform convergence for product measure PARTITION} implies that 
	$\ell = O\left(\frac{V_{\max}}{\epsilon^2}\cdot \ln \frac{\tilde{k}}{\epsilon}+\frac{\tilde{k}^2}{\epsilon^2}\cdot \ln \frac{\tilde{k}}{\delta} \right)$ samples from $\times_{i\in[d]}\DD_i$ suffice to obtain $(\epsilon,\delta)$-p.m. uniform convergence for $\HH$. 
	Formally,
	$$\Pr_{\boldsymbol{z}^{(1)},\ldots, \boldsymbol{z}^{(\ell)}}\left[\sup_{f\in \HH}\left|\E_{\boldsymbol{z}\sim \times_{i\in[\tilde{k}]}\hat{\DD}_{\tilde{T}_i}}\left[f(\boldsymbol{z})\right] - \E_{\boldsymbol{z}\sim \times_{i\in[d]}{\DD}_i}\left[f(\boldsymbol{z})\right]\right|\leq \epsilon\right]\geq 1-\delta,$$
	where for a given sample $\boldsymbol{z}^{(1)},\ldots, \boldsymbol{z}^{(\ell)}$ from a product distribution $\times_{i\in[d]} \DD_i$ the distributions $\hat{\DD}_{\tilde{T}_i}$ are defined to be uniform over ${\boldsymbol{z}}^{(1)}_{\tilde{T}_i},\ldots,{\boldsymbol{z}}^{(\ell)}_{\tilde{T}_i}$, where ${\boldsymbol{z}}^{(j)}_{\tilde{T}_i}$ contains the entries of sample $\boldsymbol{z}^{(j)}$ in coordinates $\tilde{T}_i$.
\end{corollary}

Table~\ref{tab:productVC} compares the sample complexity for uniform convergence implied by Theorem~\ref{thm:uniform convergence for product measure PARTITION} and Corollary~\ref{cor:VC for product measure} to that implied by VC theory, when the underlying measures are product. Suppose $\HH$ contains the indicator functions of all convex sets in $\mathbb{R}^d$. VC theory does not provide any finite sample  bound for  uniform convergence, as the VC dimension of $\HH$ is $\infty$. Do our results provide a finite bound? Notice that, for all $i$, $\HH_i$  simply contains all intervals in $\mathbb{R}$. Hence, $V_i=2$ and Corollary~\ref{cor:VC for product measure} implies that $\ell = O(\frac{d^2}{\epsilon^2}\cdot \left(\log \frac{d}{\delta}+\log \frac{d}{\epsilon}\right) )$ samples  suffice to obtain $(\epsilon,\delta)$-p.m.~uniform convergence for $\HH$ . In fact, our sample complexity bound can be improved to $O\left(\frac{d^2}{\epsilon^2}\cdot \log \frac{d}{\delta}\right)$, as $O\left(\frac{\log \frac{1}{\delta}}{\epsilon^2}\right)$ samples suffice to guarantee $(\epsilon,\delta)$-uniform convergence for all intervals in $\mathbb{R}$ due to the DKW inequality~\cite{DvoretzkyKW56}. 

%\todo{Expand the discussion here.}

In the next a few sections, we apply our uniform convergence results to learn a mechanism with approximately optimal revenue. A type of events called \emph{single-intersecting} (see Definition~\ref{def:single-intersecting})   plays a key role in our analysis. These events are defined based on the geometric shape of the corresponding sets. For example, balls, rectangles and all convex sets are single-intersecting, but this definition includes some non-convex sets as well, for example, ``cross-shaped'' sets. It turns out that being able to handle these non-convex sets is crucial for our results, as many events we care about are not convex but nonetheless are single-intersecting. 

\begin{definition}[Single-intersecting Events]\label{def:single-intersecting}
For any event $\EE$ in $\mathbb{R}^{\ell}$, $\EE$ is \textbf{single-intersecting} if the intersection of $\EE$ and any line that is parallel to one of the axes is an interval. More formally, for any $i\in[\ell]$ and any line $L_i=\left\{x \in \mathbb{R}^{\ell}\ |\ x_{-i}=a_{-i} \right \}$, where $a_{-i} \in \mathbb{R}^{\ell-1}$, the intersection of $L_i$ and $\EE$ is of the form $\left\{x \in \mathbb{R}^{\ell}\ |\ x_{-i}=a_{-i}, x_i\in[\ubar{a},\bar{a}] \right\}$ where $\ubar{a}\leq \bar{a}$. In particular, we allow $\ubar{a}$ to be $-\infty$ and $\bar{a}$ to be $+\infty$.
\end{definition}


\noindent We establish a uniform convergence bound for single-intersecting events by combing the DKW inequality and Theorem~\ref{thm:uniform convergence for product measure}.

\begin{lemma}\label{lem:uniform convergence for single-intersecting}
	For any integer $\ell$, let $\HH$ be the hypothesis class that contains all indicator functions for single-intersecting events in $\mathbb{R}^\ell$. Then $\HH$ has $(\epsilon,\delta)$-p.m. uniform convergence with sample complexity $O\left(\frac{\ell^2}{\epsilon^2}\cdot \log \frac{\ell}{\delta}\right)$.
\end{lemma}
\begin{proof}
	As the projected hypothesis class for the $i$-th coordinate simply contains all intervals in $\R$, the sample complexity for $(\epsilon,\delta)$-uniform convergence is $O({1 \over \epsilon^2}\cdot \log \frac{1}{\delta})$ due to the DKW inequality. The claim follows from Theorem~\ref{thm:uniform convergence for product measure}.
\end{proof}

Next, we show a slightly stronger statement, which is a type of uniform convergence bound when access to approximate distributions is given. More specifically, we argue that for any single-intersecting event, the difference in the probability of this event under two product distributions $\DD=\times_{i\in[\ell]}\DD_i$ and $\hat{\DD}=\times_{i\in[\ell]}\hat{\DD}_{i}$ is at most $2\xi\cdot \ell$, if $||\DD_i-\hat{\DD}_i||_K\leq \xi$ for all $i$. It is not hard to see that Lemma~\ref{lem:Kolmogorov stable for sc} and the DKW inequality imply Lemma~\ref{lem:uniform convergence for single-intersecting}.

\begin{lemma}\label{lem:Kolmogorov stable for sc}
For any integer $\ell$, let $\DD=\times_{i=1}^\ell \DD_i$ and $\hat{\DD}=\times_{i=1}^\ell \hat{\DD}_i$, where $\DD_i$ and $\hat{\DD}_i$ are both supported on $\mathbb{R}$ for any $i\in[\ell]$. If $||\DD_i-\hat{\DD}_i||_K\leq \xi$, $\left|\Pr_\DD[\EE]-\Pr_{\hat{\DD}}[\EE]\right|\leq 2\xi\cdot \ell$ for any single-intersecting event $\EE$.
\end{lemma}
\begin{proof}
	Let $\HH=\left\{\ind_{x\in \EE} :\EE\text{ is \emph{single-intersecting}}\right\}$. By the definition of single-intersecting events, $\HH_i$ is the set of the indicator functions of all intervals in $\mathbb{R}$ for any $i\in [\ell]$. Since $||\DD_i-\hat{\DD}_i||_K\leq \xi$, $$\sup_{g\in\HH_i}\left|\E_{x\sim \DD_i}[g(x)]-\E_{x\sim \hat{\DD}_i}[g(x)]\right|\leq 2\xi.$$ By Lemma~\ref{lem:approx product dist}, $$\sup_{f\in\HH} \left|\E_{\boldsymbol{x}\sim \DD}\left[f(\boldsymbol{x})\right]-\E_{\boldsymbol{x}\sim \hat{\DD}}\left[f(\boldsymbol{x})\right]\right|\leq 2\xi\cdot \ell.$$
\end{proof}

The following table (Table~\ref{tab:productVC}) summarizes some uniform convergence bounds implied by our results in this section.


\begin{table}[h]
	\centering
		\begin{tabular}{c || c | c}
			
			\hline\hline
			Hypotheses Class & \begin{minipage}[h]{4cm}\centering VC Bound\end{minipage} & \begin{minipage}[h]{7cm}\centering  Bounds from Theorem~\ref{thm:uniform convergence for product measure PARTITION} and Corollary~\ref{cor:VC for product measure}\end{minipage}\\
			\hline
			& &\\
			 axis-aligned rectangles in $\mathbb{R}^d$ &  $\tilde{O}(d /\epsilon^2)$ & $\tilde{O}(d /\epsilon^2)$	\\		
			polytopes with $k$ facets in $\mathbb{R}^d$ &  $\tilde{O}(d k /\epsilon^2)$ & $\tilde{O}(d \cdot \min\{d,k\} /\epsilon^2)$\\
			arbitrary convex sets in $\mathbb{R}^d$ & $\infty$ & $\tilde{O}(d^2 /\epsilon^2)$\\
			single-intersecting sets in $\mathbb{R}^d$ & $\infty$ & $\tilde{O}(d^2 /\epsilon^2)$
		\end{tabular}
	\caption{Number of samples required for $(\epsilon,\Theta(1))$-p.m. uniform convergence for different ${\cal H}$'s.}
	\label{tab:productVC}
\end{table}
\section{Constrained Additive Bidders: Uniform Convergence of the Revenue of Sequential Posted Price with Entry Fee Mechanisms}\label{sec:uniform convergence of SPEM}

We consider a specific class of mechanisms, namely Sequential Posted Price with Entry fee Mechanisms, a.k.a. \textbf{SPEM}s; see Algorithm~\ref{alg:spem-mech} for details. Cai and Zhao~\cite{CaiZ17} recently showed that if the bidders' valuations are XOS over independent items, the best SPEM achieves a constant fraction of the optimal revenue. \footnote{Cai and Zhao~\cite{CaiZ17} showed that the best ASPE or RSPM achieves a constant fraction of the optimal revenue. Clearly, any ASPE is also a SPEM, and any RSPM is simply a SPEM if we force the bidders to be unit-demand by only allowing each of them to purchase at most one item.} This section has two goals. The first is to show that, when bidders have constrained additive valuations over independent items, polynomially many samples suffice to guarantee uniform convergence for the revenue of all SPEMs, and hence our ability to select a near-optimal SPEM from polynomially many samples. This can be proven by applying our uniform convergence result for single-intersecting events (Lemma~\ref{lem:uniform convergence for single-intersecting}). The second (and stronger goal) is to show that we can learn a near-optimal SPEM under the max-min learning model (Theorem~\ref{thm:constrained additive Kolmogorov}). We show that the revenue of any SPEM changes no more than $O(\epsilon \cdot m^2\cdot n \cdot H)$ under the true and approximate valuation distributions (Theorem~\ref{thm:revenue stability under K-distance}), where $\epsilon$ is an upper bound of the Kolmogorov distance between the true and approximate distributions for every item marginal of every bidder. It is, of course, not hard to see that Theorem~\ref{thm:revenue stability under K-distance} and the DKW inequality imply uniform convergence of the revenue of all SPEMS. To establish Theorem~\ref{thm:revenue stability under K-distance}, we need to apply Lemma~\ref{lem:Kolmogorov stable for sc} instead of Lemma~\ref{lem:uniform convergence for single-intersecting}.


\begin{algorithm}[ht]
\begin{algorithmic}[1]
\REQUIRE A collection of prices $\{p_{ij}\}_{i\in[n], j\in[m]}$ and a collection of entry fee functions $\{\delta_i(\cdot)\}_{i\in[n]}$ where $\delta_i: 2^{[m]}\mapsto \mathbb{R}$ is bidder $i$'s entry fee function.
\STATE $S\gets [m]$
\FOR{$i \in [n]$}
	\STATE Show bidder $i$ {the} set of available items $S$ and set the entry fee for bidder $i$ to be ${\delta_i}(S)$.
    \IF{Bidder $i$ pays the entry fee ${\delta_i}(S)$}
        \STATE $i$ receives her favorite bundle $S_i^{*}$ and pays $\sum_{j\in S_i^{*}}p_{ij}$.
        \STATE $S\gets S\backslash S_i^{*}$.
    \ELSE
        \STATE $i$ gets nothing and pays $0$.
    \ENDIF
\ENDFOR
\end{algorithmic}
\caption{{\sf Sequential Posted Price with Entry Fee Mechanism (SPEM)}}
\label{alg:spem-mech}
\end{algorithm} 


 We first establish a technical lemma, which states that, for any set of items $S$, any set of prices $\{p_j\}_{j\in [m]}$ and entry fee $\delta$, the distribution over the {\em set of items} purchased by a constrained additive bidder whose valuation is drawn from  $\DD=\times_{j\in[m]} \DD_j$ and $\hat{\DD}=\times_{j\in[m]} \hat{\DD}_j$ has total variation distance at most $2m\xi$, if $||\DD_j-\hat{\DD}_j||_K\leq \xi$ for every item $j\in [m]$. This is quite surprising. Given that, for each set of items $S' \subseteq S$, the difference in the probability that the buyer will purchase this particular set $S'$ under $\DD$ and $\hat{\DD}$ could already be as large as {$\Theta(m\xi)$}, and the distribution has an exponentially large support size,  a trivial argument would give a bound of {$2^m\cdot \Theta(m\xi)$}. To overcome this analytical difficulty, we argue instead that for any collection of sets of items, the event that the buyer's favorite set lies in this collection is single-intersecting. Then our result follows from Lemma~\ref{lem:Kolmogorov stable for sc}. Notice that it is crucial that Lemma~\ref{lem:Kolmogorov stable for sc} holds for all events that are single-intersecting, as the event we consider here is clearly non-convex in general. 


\begin{lemma}\label{lem:stable demand set}
	For any set $S\subseteq [m]$, any prices $\{p_j\}_{j\in[m]}$ and entry fee $\delta(S)$, let $\LL$ and $\hat{\LL}$ be the distributions over the set of items purchased from $S$ by a constrained additive bidder under prices $\{p_j\}_{j\in[m]}$ and entry fee $\delta$ when her type is drawn from $\DD=\times_{j\in[m]} \DD_j$ and $\hat{\DD}=\times_{j\in[m]} \hat{\DD}_j$ respectively. If $||\DD_{j}-\hat{\DD}_{j}||_K\leq \xi$ for all item $j$, $||\LL-\hat{\LL}||_{TV}\leq 2m \xi.$
\end{lemma}
\begin{proof}
%Define function $f_{\KK,\boldsymbol{p}, S, \delta}:\mathbb{R}_{\geq 0}^m\mapsto \{0,1\}$ for any $\KK\subseteq 2^{[m]}$, $S\in[m]$, $\boldsymbol{p}\in \mathbb{R}_{\geq 0}^m$ and $\delta\in \mathbb{R}_{\geq 0}$. $f_{\KK,\boldsymbol{p}, S, \delta}(t)=1$ iff the constrained additive bidder with type $t$ will purchase a set of item $R\in \KK$ when the set of available items is $S$, the item prices are $\boldsymbol{p}=\{p_j\}_{j\in[m]}$ and the entry fee is $\delta$. Let $\HH$ be the hypothesis class that contains all $f_{\KK,\boldsymbol{p}, S, \delta}$'s. To prove the claim of this Lemma, it suffices to show that $$\sup_{f\in\HH} \left|\E_{\boldsymbol{x}\sim \DD}\left[f(\boldsymbol{x})\right]-\E_{\boldsymbol{x}\sim \hat{\DD}}\left[f(\boldsymbol{x})\right]\right|\leq 2m\cdot \epsilon.$$

For any set $R\subseteq S$, let $\EE_R$ be the event that the bidder purchases set $R$. Proving that the total variation distance between $\LL$ and $\hat{\LL}$ is no more than $2m\cdot \xi$ is the same as proving that for any $K\leq 2^{|S|}$, $\left|\ \Pr_{\DD}\left[t\in \bigcup_{\ell=1}^K \EE_{R_\ell}\right]-\Pr_{\hat{\DD}}\left[t\in \bigcup_{\ell=1}^K \EE_{R_\ell}\right]\right|\leq 2m\cdot \xi$ where $R_1,\cdots R_K$ are arbitrary distinct subsets of $S$. Since the dimension of the bidder's type space is $m$, if we can prove that $\bigcup_{\ell=1}^K \EE_{R_\ell}$ is always single-intersecting, our claim follows from Lemma~\ref{lem:Kolmogorov stable for sc}. 

%For every $j\in[m]$, let $\HH_j=\left\{g\ |\ \exists f\in \HH\ \exists\ a_{-j}\in \mathbb{R}^{m-1} \ \forall\ x\in\mathbb{R}, \ g(x) = f(x,a_{-j})\right\}$. Next, we will prove that all functions in $\HH_i$ are indicator functions of intervals in $\mathbb{R}$. From now on, we fix $\KK,\boldsymbol{p}, S$ and $\delta$. 
%For any set $R\subseteq S$, let $\EE_R$ be the event that the bidder purchases set $R$. %Proving that the total variation distance between $\LL$ and $\hat{\LL}$ is no more than $2m\cdot \xi$ is the same as proving that for any $K\leq 2^{|S|}$, $\left|\ \Pr_{D_i}\left[t_i\in \bigcup_{\ell=1}^K \EE_{R_\ell}\right]-\Pr_{\hat{D}_i}\left[t_i\in \bigcup_{\ell=1}^K \EE_{R_\ell}\right]\right|\leq 2m\cdot \xi$ where $R_1,\cdots R_K$ are arbitrary distinct subsets of $S$. Since the dimension of the bidder's type space is $m$, if we can prove that $\bigcup_{\ell=1}^K \EE_{R_\ell}$ is always single-intersecting, our claim follows from Lemma~\ref{lem:Kolmogorov stable for sc}. 
For any $j\in[m]$ and $a_{-j}\in \mathbb{R}_{\geq 0}^{{m}-1}$, let $L_j(a_{-j})=\left\{ (t_{j},a_{-j}) \ | t_{j}\in \mathbb{R}_{\geq 0} \right\}$. We claim that $L_j(a_{-j})$ intersects with at most two different $\EE_{U}$ and $\EE_{V}$ where $U$ and $V$ are subsets of $S$. 
	WLOG, we assume that  $(0,a_{-j})\in \EE_U$. 
	\begin{itemize}
		\item If $U=\emptyset$, that means the utility of the favorite set for type $(0,a_{-j})$ is smaller than the entry fee $\delta(S)$. If we increase the value of $t_{j}$, two cases could happen: (1) the utility of the favorite set is still lower than the entry fee; (2) the utility of the favorite set is higher than the entry fee. In case (1), $(t_{j},a_{-j})\in \EE_{\emptyset}$. In case (2), the bidder pays the entry fee and purchases her favorite set $V$. Then item $j$ must be in $V$, because otherwise the utility for set $V$ does not change from type $(0,a_{-j})$ to type $(t_{j},a_{-j})$. If we keep increasing $t_{j}$, bidder $i$'s favorite set remains to be $V$ and she keeps accepting the entry fee and purchasing $V$. Hence, $L_j(a_{-j})$ can intersect with at most one event $\EE_R$ where $R$ is non-empty.
		\item If $U\neq \emptyset$, that means $U$ is the favorite set of type $(0,a_{-j})$ and the utility for winning set $U$ is higher than the entry fee.  If we increase the value of $t_{j}$, two cases could happen: (1) $U$ remains the favorite set; (2) a different set $V$ becomes the new favorite set. In case (1), $(t_{j},a_{-j})\in \EE_U$. In case (2), item $j$ must lie in $V$ but not in $U$, otherwise how could $U$ be better than $V$ for type $(0,a_{-j})$ but worse for type $(t_{j},a_{-j})$.  If we keep increasing $t_{j}$, the bidder's favorite set remains to be $V$ and she keeps accepting the entry fee and purchasing $V$. Hence, $L_j(a_{-j})$ can intersect at most two different events.
	\end{itemize}
	
It is not hard to see that any event $\EE_R$ is an intersection of halfspaces, so the intersection of $L_j(a_{-j})$ with any event $\EE_R$ is an interval. {Also, notice that any type $t\in\mathbb{R}_{\geq 0}^{m}$ must lie in an event $\EE_R$ for some set $R\subseteq S$.} If $L_j(a_{-j})$ intersects with two different events $\EE_U$ and $\EE_V$, the two intersected intervals must lie back to back on $L_j(a_{-j})$. Otherwise, $L_j(a_{-j})$ intersects with at least three different events. Contradiction. Since $L_j(a_{-j})$ intersects with at most two different events, no matter which of these events are in $\{\EE_{R_\ell}\}_{\ell\in[K]}$, the intersection of $L_j(a_{-j})$ and $\bigcup_{\ell=1}^K \EE_{R_\ell}$ is either empty or an interval meaning $\bigcup_{\ell=1}^K \EE_{R_\ell}$ is single-intersecting. Now our claim simply follows from Lemma~\ref{lem:Kolmogorov stable for sc}.
\end{proof}

\begin{theorem}\label{thm:revenue stability under K-distance}
	Suppose all bidders' valuations are constrained additive over independent items. For any SPEM, let $\rev$ and $\widehat{\rev}$ be its expected revenue under $D$ and $\hat{D}$ respectively. If $D_{ij}$ and $\hat{D}_{ij}$ are both supported on $[0,H]$, and $||D_{ij}-\hat{D}_{ij}||_K\leq \xi$ for all $i\in[n]$ and $j\in[m]$, $$\left|\rev-\widehat{\rev}\right|\leq 2nm\xi\cdot \left(mH+\opt \right).$$	
	\end{theorem}
\begin{proof}
	We use a hybrid argument. Consider a sequence of distributions $\{D^{(i)}\}_{i\leq n}$, where $D^{(i)}=\hat{D}_1\times\cdots\times\hat{D}_i\times D_{i+1}\times\cdots\times D_{n},$ and $D^{(0)}=D$, $D^{(n)}=\hat{D}$.
	 We use $\rev^{(i)}$ to denote the expected revenue of the SPEM under $D^{(i)}$. To prove our claim, it suffices to argue that $\left|\rev^{(i-1)} -\rev^{(i)}\right|\leq 2\xi m\cdot \left(m\cdot H+\opt\right).$
	  We denote by $\SS_k$ and $\SS'_k$ the random set of items that remain available after visiting the first $k$ bidders under $D^{(i-1)}$ and $D^{(i)}$. Clearly, for $k\leq i-1$, $||\SS_k-\SS'_k||_{TV}=0$, so the expected revenue collected from the first $i-1$ bidders under $D^{(i-1)}$ and $D^{(i)}$ is the same. According to Lemma~\ref{lem:stable demand set}, $||\SS_i-\SS'_i||_{TV}\leq 2m\cdot \xi$. The total amount of money bidder $i$ spends can never be higher than her value for receiving all the items which is at most $m\cdot H$. So the difference in the expected revenue collected from bidder $i$ under  $D^{(i-1)}$ and $D^{(i)}$ is at most $2\xi\cdot m^2H$. Suppose $R$ is the set of remaining items after visiting the first $i$ bidders, then the expected revenue collected from the last $n-i$ bidders is the same under  $D^{(i-1)}$ and $D^{(i)}$, as these bidders have the same distributions. Moreover, this expected revenue is no more than $\opt$, since the optimal mechanism can simply just sell $R$ to the last $n-i$ bidders using the same prices and entry fee as in the SPEM we consider. Of course, for any fixed $R$, the probabilities that $\SS_i=R$ and $\SS'_i=R$ are different, but since for any $R$ the expected revenue from the last $n-i$ bidders is at most $\opt$, the difference in the expected revenue from the last $n-i$ bidders under  $D^{(i-1)}$ and $D^{(i)}$ is at most $||\SS_i-\SS'_i||_{TV} \cdot \opt \leq 2\xi\cdot m\opt$. Hence, the total difference between $\rev^{(i-1)}$ and  $\rev^{(i)}$ is at most $2\xi m\cdot \left(m H+\opt\right)$. 
	  %Furthermore, $\left|\rev-\widehat{\rev}(p,\delta)\right|\leq \sum_{i=1}^{n}\left|\rev^{(i-1)}(p,\delta) -\rev^{(i)}(p,\delta)\right|\leq 2 nm\xi \left(mH+\opt_{ASPE}\right).$
\end{proof}


\begin{theorem}\label{thm:constrained additive Kolmogorov}(Max-min Learning for Constrained Additive Bidders)
	When all bidders' valuations are constrained additive over independent items and for any bidder $i$ and any item $j$, $D_{ij}$ and $\hat{D}_{ij}$ are supported on $[0,H]$ and $||D_{ij}-\hat{D}_{ij}||_K\leq \epsilon$  for some $\epsilon=O(\frac{1}{nm})$, then with only access to $\hat{D}=\times_{i,j} \hat{D}_{ij}$, our algorithm can learn an RSPM or ASPE whose revenue is at least $\frac{\opt}{c}-{ \epsilon\cdot O(m^2n H)}$, where $\opt$ is the optimal revenue by any BIC mechanism under $D=\times_{i,j} D_{ij}$. $c>1$ is an absolute constant.
	\end{theorem}
	
Clearly, Theorem~\ref{thm:constrained additive Kolmogorov} also implies a polynomial sample complexity bound for learning an approximately revenue-optimal mechanism. A better sample complexity bound can be obtained directly, i.e.~without invoking the uniform convergence of the revenue of SPEMs, and  is stated as Theorem~\ref{thm:XOS sample} for the broader class of XOS valuations. Similarly, when bidders have simpler valuations, i.e., additive or unit-demand valuations, we can sharpen our results and achieve polynomial-time learnability of the approximately optimal mechanism using more specialized techniques. See Sections~\ref{sec:unit-demand} and~\ref{sec:additive} for details.
\subsection{Unit-demand Valuations: Polynomial-Time Learning} \label{sec:unit-demand}

In this section, we consider bidders with unit-demand valuations, sharpening our results to show how to learn approximately revenue-optimal mechanisms in polynomial time. It is shown in a sequence of works~\cite{ChawlaHMS10, KleinbergW12, CaiDW16} that there exists a sequential posted price mechanism (\textbf{SPM} see Algorithm~\ref{alg:seq-mech} for details) that achieves at least $\frac{1}{24}$ of the optimal revenue when bidders are unit-demand. We show that under all three distribution access models of Section~\ref{sec:prelim} there exists a polynomial-time algorithm that learns a sequential posted price mechanism whose revenue approximates the optimal revenue. We only sketch the proof here and postpone the details to Appendix~\ref{sec:unit-demand appx}.

\begin{theorem}\label{thm:unit-demand}
	When all bidders have unit-demand valuations and \begin{itemize}
		\item $D_{ij}$ is supported on $[0,H]$ for all bidder $i$ and item $j$, there exists a polynomial time algorithm that learns an SPM whose revenue is at least $\frac{\opt}{144}-\epsilon H$ with probability $1-\delta$ given  $O\left(\left(\frac{1}{\epsilon}\right)^2 \left(m^2 n\log \frac{n}{\epsilon} + \log \frac{1}{\delta}\right)\right)$ samples from $D$; or
		\item $D_{ij}$ is a regular distribution for all bidder $i$ and item $j$, there exists a polynomial time algorithm that learns a randomized SPM whose revenue is at least $\frac{\opt}{33}$ with probability $1-\delta$ given $O(\max\{m,n\}^2m^2 n^2\cdot \log \frac{nm}{\delta})$ samples from $D$; or
		\item we are only given access to $\hat{D}_{ij}$ where $||\hat{D}_{ij}-D_{ij}||_K\leq \epsilon$ for all bidder $i$ and item $j$, there is a polynomial time algorithm that constructs a randomized SPM whose revenue under $D$ is at least $\left(\frac{1}{4}-(n+m)\cdot \epsilon\right)\cdot\left(\frac{\opt}{8}-2\epsilon\cdot mnH\right)$\footnote{If we set $\epsilon$ to be $O(\frac{1}{m+n})$, this is the max-min guarantee we want to achieve.}.
	\end{itemize}
\end{theorem}

\noindent\textbf{Sample Access to Bounded Distributions:} the result is due to Morgenstern and Roughgarden~\cite{MorgensternR16}. 

\vspace{.05in}
\noindent\textbf{Direct Access to Approximate Distributions:} we first consider a convex program based on $D$ (see Figure~\ref{fig:CP unit demand}) which is usually referred to as the ex-ante relaxation of the revenue maximization problem~\cite{Alaei11}, and use its optimum as a proxy for $\opt$. Next, we consider a similar convex program based on $\hat{D}$ (see Figure~\ref{fig:CP unit demand approximate dist}) and show that the optima of the two convex programs are close to each other. Finally, we use techniques developed by Chawla et al.~\cite{ChawlaHMS10} to convert the optimal solution of the second convex program into a randomized SPM. We can show that the constructed randomized SPM achieves a revenue that approximates the optimum of the second convex program under $D$, which implies that the mechanism's revenue  also approximates the $\opt$. As we are given $\hat{D}$, we can solve the second convex program and convert its optimal solution into a randomized SPM in polynomial time. See Theorem~\ref{thm:UD Kolmogorov} in Appendix~\ref{sec:unit-demand Kolmogorov} for further details.

\vspace{.05in}
\noindent\textbf{Sample Access to Regular Distributions:} we use a similar convex program relaxation based approach as in the previous case. The main difference is that regular distributions could be unbounded and thus ruin the approximation guarantee. We show how to use the Extreme Value theorem in~\cite{CaiD11b} to truncate the distributions without hurting the revenue by much. See Theorem~\ref{thm:UD regular} in Appendix~\ref{sec:unit-demand regular} for further details.
\subsection{Additive Valuations: Polynomial-Time Learning}\label{sec:additive}
In this section, we consider bidders with additive valuations, again sharpening our results to show polynomial-time learnability. It is known that the better of the following two mechanisms achieves at least $\frac{1}{8}$ of the optimal revenue when all bidders have additive valuations~\cite{Yao15,CaiDW16}:

\vspace{.05in}	
\noindent\textbf{Selling Separately}: the mechanism sells each item separately using Myerson's optimal auction.

\vspace{.05in}	
\noindent \textbf{VCG with Entry Fee}: the mechanism solicits bids $\bold{b}=(b_1,\cdots, b_n)$ from the bidders, then offers each bidder $i$ the option to participate for an entry fee $e_i(b_{-i},D_i)$, which is the median of the random variable $\sum_{j\in[m]}(t_{ij}-\max_{k\neq i} b_{kj})^+$, where $t_i\sim D_i$\footnote{The entry fee function defined in~\cite{Yao15,CaiDW16} is slightly different. They showed that there exists an entry fee $X_i$, such that bidder $i$ accepts the entry fee with probability at least $1/2$. Then they argued that extracting $X_i/2$ as the revenue in the VCG with entry fee mechanism is enough to obtain a factor $8$ approximation. It is not hard to observe that our entry fee is accepted with probability exactly $1/2$, thus our entry fee is at least as large as $X_i$. So our mechanism also suffices to provide a factor $8$ approximation.}. This random variable is exactly bidder $i$'s utility when her type is $t_i$ and the other bidders' are $b_{-i}$. If bidder $i$ chooses to participate, she pays the entry fee and can take any item $j$ at price $\max_{k\neq i} b_{kj}$. Notice that the mechanism never over allocate any item, as only the highest bidder for an item can afford it. %Moreover, this mechanism is DSIC, because the entry fee and item prices for bidder $i$ only depend on the other bidders' bids and bidder $i$'s type distribution but not her bid.

Indeed, only counting the revenue from the entry fee in the second mechanism and the optimal revenue from selling the items separately already suffices to provide an $8$-approximation~\cite{Yao15, CaiDW16}. 

\begin{theorem}[\cite{CaiDW16}]\label{thm:UB additive}
	Let $\srev$ be the optimal revenue for selling the items separately and $\brev$ be the expected entry fee collected from the VCG with entry fee mechanism. Then $\opt\leq 6\cdot \srev+2\cdot\brev.$ 
\end{theorem}

Goldner and Karlin~\cite{GoldnerK16} showed that one sample suffices to learn a mechanism that achieves a constant fraction of the optimal revenue when $D_{ij}$ is regular for all $i\in[n]$ and $j\in[m]$. %In the rest of the section, we discuss
We show how to learn an approximately optimal mechanism in the other two models.
% (1) all $D_{ij}$ are supported on $[0,H]$, and (2) direct access to distributions $\hat{D}_{ij}$, where $||\hat{D}_{ij}-D_{ij}||_K\leq \epsilon$ for all $i\in[n]$ and $j\in[m]$.
\begin{theorem}\label{thm:additive}
	When the bidders have additive valuations and\begin{itemize}
		\item $D_{ij}$ is supported on $[0,H]$ for all bidder $i$ and item $j$, we can learn in polynomial time a mechanism whose expected revenue is at least $\frac{\opt}{32}-{\epsilon}\cdot H$ with probability $1-\delta$ given $O\left(\left(\frac{m}{\epsilon}\right)^2 \cdot\left(n\log n\log \frac{1}{\epsilon}+\log\frac{1}{\delta} \right)\right)$ samples from $D$; or
		\item  we are only given access to distributions $\hat{D}_{ij}$ where $||\hat{D}_{ij}-D_{ij}||_K\leq \epsilon$ for all bidder $i$ and item $j$, there is a polynomial time algorithm that constructs a mechanism whose expected revenue under $D$ is at least $\frac{\opt}{266}-96\epsilon\cdot mnH$ when $\epsilon\leq \frac{1}{16\max\{m,n\}}$.
	\end{itemize}
\end{theorem}

\noindent\textbf{Sample Access to Bounded Distributions:} Goldner and Karlin's proof~\cite{GoldnerK16} can be directly applied to the bounded distributions to show a single sample suffices to learn a mechanism whose expected revenue approximates the $\brev$. Then as $\srev$ is the revenue of $m$ separate single-item auctions, we can use the result in~\cite{MorgensternR16} to approximate it. See Theorem~\ref{thm:additive bounded} in Appendix~\ref{sec:additive bounded} for further details.

\vspace{.05in}
\noindent\textbf{Direct Access to Approximate Distributions:} for each single item, we apply Theorem~\ref{thm:unit-demand} to learn an individual auction, then run these learned auctions in parallel. Clearly, the combined auction's revenue approximates $\srev$. For $\brev$, we show that for every bidder $i$ and every bid profile $b_{-i}$ of the other bidders, the event that corresponds to bidder $i$ accepting any entry fee is \emph{single-intersecting} (see Definition~\ref{def:single-intersecting}). This implies that the probability for a bidder to accept an entry fee under $\hat{D}$ and $D$ is close (Lemma~\ref{lem:Kolmogorov stable for sc}). So we can essentially use the median of $\sum_{j\in[m]}(t_{ij}-\max_{k\neq i} b_{kj})^+$ with $t_i\sim\hat{D}_i$ as the entry fee. See Theorem~\ref{thm:additive Kolmogorov} in Appendix~\ref{sec:additive Kolmogorov} for further details.
\section{XOS Valuations} \label{sec:constrained additive}
In this section we go beyond constained additive valuations to show learnability of approximately revenue-optimal auctions from polynomially many samples. The better of the following two mechanisms is known to achieve a constant fraction of the optimal revenue, when bidders have valuations that are XOS over independent items~\cite{CaiZ17}.

%\begin{itemize}
\vspace{.05in}
\noindent\textbf{Rationed Sequential Posted Price Mechanism (RSPM)}: the mechanism is almost the same as SPM in Algorithm~\ref{alg:seq-mech}, except there is an extra constraint that every bidder can purchase at most one item.

\vspace{.05in}
\noindent\textbf{Anonymous Sequential Posted Price with Entry Fee Mechanism (ASPE)}: every buyer faces the same collection of item prices $\{p_j\}_{j\in[m]}$. The seller visits the bidders sequentially. For every bidder, the seller shows her all the available items (i.e. items that have not yet been purchased) and the associated price for each item, then asks her to pay a personalized entry fee which depends on her type distribution and the set of available items. If the bidder accepts the entry fee, she can proceed to purchase any available item at the given price; if she rejects the entry fee, she neither receives nor pays anything. See Algorithm~\ref{alg:aspe-mech} for details.
%\end{itemize}

\begin{theorem}\cite{CaiZ17}\label{thm:simple XOS}
	There exists a collection of prices $\{p^*_j\}_{j\in[m]}$, such that if we set the entry fee function $\delta^*_i(S)$ to be the median of bidder $i$'s utility for set $S$, either the ASPE$(p^*,\delta^*)$ or the best RSPM achieves at least a constant fraction of the optimal revenue when bidders' valuations are XOS over independent items. More formally, let $u^*_i(t_i, S)=\max_{S^*\subseteq S} v_i(t_i, S^*)-\sum_{j\in S^*} p^*_j$ be bidder $i$'s utility for the set of items $S$ when her type is $t_i$. We define $\delta^*_i(S)$ to be the median of the random variable $u^*_i(t_i,S)$ (with $t_i\sim D_i$) for any set $S\subseteq [m]$.  Moreover, the price $p^*_j$ for any item $j$ is no larger than $2G$, where $G = \max_{i,j} G_{ij}$ and $G_{ij}:= \sup_x \left\{ \Pr_{t_{ij}\sim D_{ij}}\left[V_i(t_{ij})\geq x\right]\geq \frac{1}{5\max\{m,n\}}\right\}$.
\end{theorem}

{%We provide an algorithm to learn both mechanisms when we are given direct access to an approximate distribution for constrained additive bidders in Section~\ref{sec:constrained additive kolomogorov}. 
Our goal next is to bound the sample complexity for learning a near-optimal RSPM and the ASPE described in Theorem~\ref{thm:simple XOS} under XOS valuations.} %See Appendix~\ref{sec:XOS bounded} for details.} 

We consider first the task of learning a near-optimal RSPM. In a RSPM, all bidders are restricted to be unit-demand, so the revenue of the best RSPM is upper bounded by the optimal revenue in the corresponding unit-demand setting. In Section~\ref{sec:unit-demand}, we have shown how to learn an approximately optimal mechanism for unit-demand bidders, and those algorithms can be used to approximate the best RSPM. 

So, for the rest of this section, it suffices to focus on learning an ASPE whose revenue approximates the revenue of the ASPE described in Theorem~\ref{thm:simple XOS}.  We will do this in Section~\ref{sec:XOS sample}. Before that,
%One might worry that the entry fee function could be complex and thus difficult to learn. It turns out if we choose the correct item prices $\{p_j\}_{j\in [m]}$, simply setting the entry fee to be the median of the bidder's utility for the available items suffices to obtain good revenue~\cite{CaiZ17}. More specifically, let $u_i(t_i, S)=\max_{S^*\subseteq S} v_i(t_i, S^*)-\sum_{j\in S^*} p_j$ be bidder $i$'s utility for the set of items $S$ and we define $\delta_i(S)$ to be the median of the random variable $u_i(t_i,S)$ while $t_i\sim D_i$ for any subset $S\subseteq [m]$.
we need a robust version of Theorem~\ref{thm:simple XOS}. In the next Lemma, we argue that if we use a collection of prices $\{p'_j\}_{j\in[m]}$ sufficiently close to $\{p^*_j\}_{j\in[m]}$ and entry fee $\delta'_i(S)$ sufficiently close to the median of the utility for every bidder $i$ and subset $S$, the better of the corresponding ASPE and the best RSPM still approximates the optimal revenue. We postpone the proof to Appendix~\ref{sec:appx XOS}.

\begin{lemma}\label{lem:approx ASPE}
	For any $\epsilon>0$ and $\mu\in[0,\frac{1}{4}]$, let $\{p'_j\}_{j\in[m]}$ be a collection of prices such that $|p'_j-p^*_j|\leq \epsilon$ for all $j\in[m]$, where $\{p^*_j\}_{j\in[m]}$ is the collection of prices in Theorem~\ref{thm:simple XOS}. Let $\delta'_i(S)$ be bidder $i$'s entry fee function such that $\Pr_{t_i\sim D_i}\left [u'_i(t_i,S)\geq \delta'_i(S)\right]\in [1/2-\mu,1/2+\mu]$ for any set $S\subseteq [m]$, where $u'_i(t_i,S) = \max_{S*\subseteq S} v_i(t_i,S^*)-\sum_{j\in S^*} p'_j$. Then, either the ASPE$(p',\delta')$ or the best RSPM achieves revenue at least $\frac{\opt}{\CC_1(\mu)}-\CC_2(\mu)\cdot (m+n)\cdot \epsilon$ when bidders' valuations are XOS over independent items. Both $\CC_1(\cdot)$ and $\CC_2(\cdot)$ are  monotonically increasing functions that only depend on $\mu$. 
\end{lemma}

\begin{definition}\label{def:eps mu ASPE}
	We say a collection of prices $\{p_j\}_{j\in[m]}$ is in the $B$-bounded $\epsilon$-net if $p_j$ is a multiple of $\epsilon$ and no larger than $B$ for any item $j$. For any collection of prices $\{p_j\}_{j\in[m]}$, we say the entry fee functions are $\mu$-balanced if for every bidder $i$ and every set $S\subseteq [m]$, her entry fee $\delta_i(S)$ satisfies  $\Pr_{t_i\sim D_i}[u_i(t_i,S)\geq \delta_i(S)]\in [1/2-\mu,1/2+\mu]$, where $u_i(t_i,S) = \max_{S*\subseteq S} v_i(t_i,S^*)-\sum_{j\in S^*} p_j$.
\end{definition}
%An easy corollary of Lemma~\ref{lem:approx ASPE} is that there exists an mechanism ASPE$(p,\delta)$ with $p$ lying in the $\epsilon$-net and that provides high revenue. More specifically,

\begin{corollary}\label{cor:discretization of prices}
	For  bidders with valuations that are XOS over independent items and any $\epsilon>0$, there exists a collection of prices $\{p_j\}_{j\in[m]}$ in the $2G$-bounded $\epsilon$-net such that for any $\mu$-balanced entry fee functions $\{\delta_i(\cdot)\}_{i\in[n]}$ with $\mu\in[0,\frac{1}{4}]$, either the ASPE$(p,\delta)$ or the best RSPM achieves revenue at least $\frac{\opt}{\CC_1(\mu)}-\CC_2(\mu)\cdot (m+n)\cdot \epsilon$. %where $\CC_1(\cdot)$ and $\CC_2(\cdot)$ are  monotonically increasing functions that only depend on $\mu$.
\end{corollary}

%In other words, if for every collection of prices $\{p_j\}_{j\in[m]}$ in the $\epsilon$-net we construct a set of $\mu$-balanced entry fee functions $\{\delta_i(\cdot)\}_{i\in[n]}$ for ${p}$, as guaranteed by Corollary~\ref{cor:discretization of prices}, the best mechanism among these ASPE$(p,\delta)$  has high revenue. 

\notshow{
\subsection{Constrained Additive Valuations: direct access to approximate distributions}\label{sec:constrained additive kolomogorov}

In this section, we consider the model where we only have access to an approximate distribution $\hat{D}$ and bidders have constrained additive valuations. %\cnote{TO TAKE OUT: As when the valuations are XOS over independent items, a bidders' type is not necessarily described as vectors in some Euclidean space. In particular, a bidder's type could be some abstract information specifying her preference\footnote{For example, the type distribution could be a categorical distribution about the color of the item.} about the items. For these distributions, our model does not apply, as it is not even clear how to define the Kolmogorov distance between any two such distributions.} 
%We restrict our attention to an important valuation class: constrained additive valuations. \yangnote{We will sketch the proof for the general XOS valuations in the next section.} %Since a constrained additive bidders' type is still a vector in $\mathbb{R}^m$, our model applies.
Our learning algorithm is a two-step procedure. In the first step, we argue that the approximate distribution $\hat{D}$ can be used to define a balanced entry fee function for each bidder under any prices. In particular, we propose to use the median of bidder $i$'s utility under $\hat{D}_i$ to compute the entry fees. The challenge is to show that, whenever $||D_{ij}-\hat{D}_{ij}||_{K}\leq \xi$ for every bidder $i$ and item $j$, the median of the utility of any bidder $i$ for any subset $S$ under any collection of prices $\{p_j\}_{j\in[m]}$ is not too different when the bidder's type is drawn from $D_i$ or $\hat{D}_i$; in particular, the entry fee functions thus defined from $\hat{D}$ are balanced. This is shown in Lemma~\ref{lem:Kolmogorov learn entry fee}. Given this lemma, we are able to construct an ASPE for every collection of prices $\{p_j\}_{j\in[m]}$ in the $\epsilon$-net. In the second step, we need to identify a mechanism  with high revenue among all ASPEs we constructed. If we have sample access to the  actual distribution, this is a simple task {as we can take a polynomial number of samples from $D$ and argue that with probability almost $1$, the empirical revenue of ASPE$(p,\delta)$ is close to its true expected revenue for every price vector $p$ in the $\epsilon$-net. So the ASPE with the highest empirical revenue also has high expected revenue.} %as we have shown in Lemma~\ref{lem:learn best ASPE}.
 But we only have access to an approximate distribution $\hat{D}$; can we argue that the expected revenue of any ASPE does not change much under $D$ or $\hat{D}$? Notice that this is certainly not true for arbitrary mechanisms, as it is easy to construct a DSIC single item auction with two bidders where the expected revenue under two distributions that are close in Kolmogorov distance is far away%\footnote{\todo{Write down this example later.}}
 . Surprisingly, we can argue that the expected revenue of any ASPE thus defined is within $\poly(n,m)\cdot H\cdot \xi$ when the bidders' types are drawn from $D$ or $\hat{D}$.

 To argue these results, we need to understand how the probability of events changes from distribution $D$ to $\hat{D}$. Clearly, if we consider arbitrary events, the difference in their probabilities under $D$ and $\hat{D}$ can be arbitrarily far. We need to identify structure in the events that are of interest to us, which allows us to argue that computations under $D$ and $\hat{D}$ are close. Since our distributions are supported on a subset of the Euclidean space, any event is simply a set in the Euclidean space. In Definition~\ref{def:single-intersecting}, we define a type of events called \emph{single-intersecting} events based on the geometric shape of the corresponding sets. For example, balls, rectangles and all convex sets are single-intersecting, but this definition includes some non-convex sets as well, for example, cross-shaped sets. It turns out that being able to include these non-convex sets is crucial for our result, as many events we care about are not convex but nonetheless are single-intersecting. Next we argue that for any single-intersecting event, the difference of the probability under $D$ and $\hat{D}$ only grows linearly in the dimension of the support space (Lemma~\ref{lem:Kolmogorov stable for sc}).
%This result is facilitated by combining an interesting property about Kolmogorov distance in product space and the geometry of bidders' behavior in ASPEs \footnote{\yangnote{I'm not sure what I was trying to say here... This is just a place holder to remind us write something about the proof.}}.
\notshow{\begin{definition}[Single-intersecting Events]\label{def:single-intersecting}
For any event $\EE$ in $\mathbb{R}^{\ell}$, $\EE$ is \textbf{single-intersecting} if the intersection of $\EE$ and any line that is parallel to one of the axes is an interval. More formally, for any $i\in[\ell]$ and any line $L_i=\left\{x \in \mathbb{R}^{\ell}\ |\ x_{-i}=a_{-i} \right \}$, where $a_{-i}$ is some $\ell-1$ dimensional vector in $\mathbb{R}^{\ell-1}$, the intersection of $L_i$ and $\EE$ is $\left\{x \in \mathbb{R}^{\ell}\ |\ x_{-i}=a_{-i}, x_i\in[\ubar{a},\bar{a}] \right\}$ for some real numbers $\ubar{a}$ and $\bar{a}$.
\end{definition}


\begin{lemma}\label{lem:Kolmogorov stable for sc}
For any integer $\ell$, let $\DD=\times_{i=1}^\ell \DD_i$ and $\hat{\DD}=\times_{i=1}^\ell \hat{\DD}_i$, where $\DD_i$ and $\hat{\DD}_i$ are both supported on $[0,H]$ for any $i\in[\ell]$. If $||\DD_i-\hat{\DD}_i||_K\leq \xi$, $\left|\Pr_\DD[\EE]-\Pr_{\hat{\DD}}[\EE]\right|\leq 2\xi\cdot \ell$ for any single-intersecting event $\EE$.
\end{lemma}
\begin{prevproof}{Lemma}{lem:Kolmogorov stable for sc} Let $\FF_i$ and $\hat{\FF}_i$ be the cdf for $\DD_i$ and $\hat{\DD}_i$ respectively. We will prove the statement using a hybrid argument. We create a sequence of product distributions $\{\DD^{(j)}\}_{j\leq \ell}$, where $\DD^{(j)}=\hat{\DD}_1\times\cdots\times\hat{\DD}_j\times \DD_{j+1}\times\cdots\times \DD_{\ell},$ and $\DD^{(0)}=\DD$, $\DD^{(\ell)}=\hat{\DD}$. To prove our claim, it suffices to show that for any integer $j\in[\ell]$, $\left|\Pr_{\DD^{(j-1)}}[\EE]-\Pr_{{\DD}^{(j)}}[\EE]\right|\leq 2\xi.$
	Let us first fix some notations. Let $\EE_{-j}=\left\{x_{-j}\ |\ \exists x_j\in \mathbb{R},\ (x_j,x_{-j})\in \EE\right\}$, $\EE_j(a_{-j})= \left\{x_{j} \in \mathbb{R} \ |\ (x_j,a_{-j})\in \EE\right\}$ for all $j\in[\ell]$. As $\EE$ is single-intersecting, $\EE_j(x_{-j})$ is an interval for all $j$ and $x_{-j}$. Moreover, since $||\DD_j-\hat{\DD}_j||_K\leq \xi$, we have $|\Pr_{\DD_j}[x_j\in \EE_j(x_{-j})]-\Pr_{\hat{\DD}_j}[x_j\in \EE_j(x_{-j})]|\leq 2\xi$ for all $j$ and $x_{-j}$. Next, we bound the difference for the probability of event $\EE$ under $\DD^{(j-1)}$ and $\DD^{(j)}$. 
	\begin{align*}
		&\left|\Pr_{\DD^{(j-1)}}[\EE]-\Pr_{{\DD}^{(j)}}[\EE]\right|\\
		=&\Big{|}\int_{\mathbb{R}^{\ell-1}}\ind{\left [x_{-j}\in \EE_{-j}\right]}\left(\int_{\mathbb{R}}\ind{\left[x_j\in \EE_j(x_{-j})\right]}d \FF_j(x_j)\right) d\hat{\FF}_1(x_1)\cdots d\hat{\FF}_{j-1}(x_{j-1}) d\FF_{j+1}(x_{j+1})\cdots d\FF_{\ell}(x_{\ell})\\
		 & - \int_{\mathbb{R}^{\ell-1}}\ind{\left [x_{-j}\in \EE_{-j}\right]}\left(\int_{\mathbb{R}}\ind{\left[x_j\in \EE_j(x_{-j})\right]}d \hat{\FF}_j(x_j)\right) d\hat{\FF}_1(x_1)\cdots d\hat{\FF}_{j-1}(x_{j-1}) d\FF_{j+1}(x_{j+1})\cdots d\FF_{\ell}(x_{\ell})\Big{|}\\
		 = & \left|\int_{\mathbb{R}^{\ell-1}}\ind{\left [x_{-j}\in \EE_{-j}\right]} \left(\Pr_{\DD_j}[x_j\in \EE_j(x_{-j})]-\Pr_{\hat{\DD}_j}[x_j\in \EE_j(x_{-j})]\right)
		  d\hat{\FF}_1(x_1)\cdots d\hat{\FF}_{j-1}(x_{j-1}) d\FF_{j+1}(x_{j+1})\cdots d\FF_{\ell}(x_{\ell})\right|\\
		  \leq & 2\xi\cdot\int_{\mathbb{R}^{\ell-1}}\ind{\left [x_{-j}\in \EE_{-j}\right]} d\hat{\FF}_1(x_1)\cdots d\hat{\FF}_{j-1}(x_{j-1}) d\FF_{j+1}(x_{j+1})\cdots d\FF_{\ell}(x_{\ell}) = 2\xi
	\end{align*}
	\end{prevproof}}
	%The second last inequality is because $\EE$ is single-intersecting so $\EE_j(x_{-j})$ is an interval for all $j$ and $x_{-j}$, also because $||\DD_j-\hat{\DD}_j||_K\leq \xi$, so $$
	It is quite easy to see that the single-intersecting condition can be relaxed to $k$-intersecting where the intersection of the event $\EE$ with any line parallel to an axis is the union of at most $k$ intervals. Indeed, using a proof almost identical to the proof of Lemma~\ref{lem:Kolmogorov stable for sc}, we can argue that the difference in the probability of an event $\EE$ under $\DD$ and $\hat{\DD}$ only grows linearly in the dimension of the support space and $k$. Next, we formally state the first step of our learning algorithm. The proof can be found in Appendix~\ref{sec:appx constrained additive}.
\begin{lemma}\label{lem:Kolmogorov learn entry fee}
Suppose $||D_{ij}-\hat{D}_{ij}||_K\leq \xi$ for any bidder $i$ and any item $j$. For any collection of prices $\{p_j\}_{j\in[m]}$, let $u^{(p)}_i(t_i,S)=\max_{S*\subseteq S} v_i(t_i,S^*)-\sum_{j\in S^*} p_j$. Define the entry fee $\delta_i^{(p)}(S)$
 of bidder $i$ for set $S$ under $\{p_j\}_{j\in[m]}$ to be the median of $u^{(p)}_i(t_i,S)$ when bidder $i$'s type $t_i$ is drawn from $\hat{D}_i$. Then $\left\{\delta_i^{(p)}(\cdot)\right\}_{i\in[n]}$ is a collection of $2m\xi$-balanced entry fee functions for any collection of prices $\{p_j\}_{j\in[m]}$.\end{lemma}

So far, for any collection of prices $\{p_j\}_{j\in[m]}$, we have defined (using $\hat{D}$) a collection of $2m\xi$-balanced entry fee functions $\{\delta_i^{(p)}(\cdot)\}_{i\in[n]}$. Next, we need to identify prices $\{p_j\}_{j\in[m]}$ such that the corresponding ASPE mechanism achieves high revenue under the actual distribution $D$. Our goal is to show that the expected revenue of simultaneously all ASPE mechanisms, defined for all prices using $\hat{D}$ as above, is not much different under $D$ and $\hat{D}$. We first establish a technical lemma, which states that, for any set of available items $S$ and entry fee $\delta_i(S)$, the distribution over the {\em set of items} purchased by bidder $i$ under $D_i$ and $\hat{D}_i$ has total variation distance at most $2m\xi$. This is quite surprising. Given that, for each set of items $S' \subseteq S$, the difference in the probability that the buyer will purchase this particular set $S'$ under $D_i$ and $\hat{D}_i$ could already be as large as {$\Theta(m\xi)$}, and the distribution has an exponentially large support size,  a trivial argument would give a bound of {$2^m\cdot \Theta(m\xi)$}. To overcome this analytical difficulty, we argue instead that for any collection of sets of items, the event that the buyer's favorite set lies in this collection is single-intersecting. Then our result follows from Lemma~\ref{lem:Kolmogorov stable for sc}. Notice that it is crucial that Lemma~\ref{lem:Kolmogorov stable for sc} holds for all events that are single-intersecting, as the event we consider here is clearly non-convex in general. 

\notshow{
\begin{lemma}\label{lem:stable favorite set}
	For any bidder $i$, any set $S\subseteq [m]$, any prices $\{p_j\}_{j\in[m]}$ and entry fee $\delta_i(S)$, let $\LL$ and $\hat{\LL}$ be the distributions over the set of items purchased from $S$ by bidder $i$ under prices $\{p_j\}_{j\in[m]}$ and entry fee $\delta_i(S)$ when her type is drawn from $D_i$ and $\hat{D}_i$ respectively. If $||D_{ij}-\hat{D}_{ij}||_K\leq \xi$ for all item $j$, $||\LL-\hat{\LL}||_{TV}\leq 2m \xi.$
\end{lemma}
\begin{proof}
	For any set $R\subseteq S$, let $\EE_R$ be the event that bidder $i$ purchases set $R$. Proving that the total variation distance between $\LL$ and $\hat{\LL}$ is no more than $2m\cdot \xi$ is the same as proving that for any $K\leq 2^{|S|}$, $\left|\ \Pr_{D_i}\left[t_i\in \bigcup_{\ell=1}^K \EE_{R_\ell}\right]-\Pr_{\hat{D}_i}\left[t_i\in \bigcup_{\ell=1}^K \EE_{R_\ell}\right]\right|\leq 2m\cdot \xi$ where $R_1,\cdots R_K$ are arbitrary distinct subsets of $S$. Since the dimension of the bidder's type space is $m$, if we can prove that $\bigcup_{\ell=1}^K \EE_{R_\ell}$ is always single-intersecting, our claim follows from Lemma~\ref{lem:Kolmogorov stable for sc}. 
	
	For any $j\in[m]$ and $a_{i,-j}\in [0,H]^{{m}-1}$, let $L_j(a_{i,-j})=\left\{ (t_{ij},a_{i,-j}) \ | t_{ij}\in [0,H] \right\}$. We claim that $L_j(a_{i,-j})$ intersects with at most two different $\EE_{U}$ and $\EE_{V}$ where $U$ and $V$ are subsets of $S$. 
	WLOG, we assume that  $(0,a_{i,-j})\in \EE_U$. 
	\begin{itemize}
		\item If $U=\emptyset$, that means the utility of the favorite set for type $(0,a_{i,-j})$ is smaller than the entry fee $\delta_i(S)$. If we increase the value of $t_{ij}$, two cases could happen: (1) the utility of the favorite set is still lower than the entry fee; (2) the utility of the favorite set is higher than the entry fee. In case (1), $(t_{ij},a_{i,-j})\in \EE_{\emptyset}$. In case (2), bidder $i$ pays the entry fee and purchases her favorite set $V$. Then item $j$ must be in $V$, because otherwise the utility for set $V$ does not change from type $(0,a_{i,-j})$ to type $(t_{ij},a_{i,-j})$. If we keep increasing $t_{ij}$, bidder $i$'s favorite set remains $V$ and she keeps paying the entry fee and purchasing $V$. Hence, $L_j(a_{i,-j})$ can intersect with at most one event $\EE_R$ where $R$ is non-empty.
		\item If $U\neq \emptyset$, that means $U$ is the favorite set of type $(0,a_{i,-j})$ and the utility for winning set $U$ is higher than the entry fee.  If we increase the value of $t_{ij}$, two cases could happen: (1) $U$ remains the favorite set; (2) a different set $V$ becomes the new favorite set. In case (1), $(t_{ij},a_{i,-j})\in \EE_U$. In case (2), item $j$ must be in $V$ and not in $U$, otherwise how could $U$ be better than $V$ for type $(0,a_{i,-j})$ but worse for type $(t_{ij},a_{i,-j})$.  If we keep increasing $t_{ij}$, bidder $i$'s favorite set remains to $V$ and she keeps paying the entry fee and purchasing $V$. Hence, $L_j(a_{i,-j})$ can intersect at most two different events.
	\end{itemize}
	
It is not hard to see that any event $\EE_R$ is an intersection of halfspaces, so the intersection of $L_j(a_{i,-j})$ with any event $\EE_R$ is an interval. {Also, notice that any type $t_i$ must lie in an event $\EE_R$ for some set $R\subseteq S$.} If $L_j(a_{i,-j})$ intersects with two different events $\EE_U$ and $\EE_V$, the two intersected intervals must lie back to back on $L_j(a_{i,-j})$. Otherwise, $L_j(a_{i,-j})$ intersects with at least three different events. Contradiction. Since $L_j(a_{i,-j})$ intersects with at most two different events, no matter which of these events are in $\{\EE_{R_\ell}\}_{\ell\in[K]}$, the intersection of $L_j(a_{i,-j})$ and $\bigcup_{\ell=1}^K \EE_{R_\ell}$ is either empty or an interval,%\footnote{\cnote{Should we add some extra explanation here, that if it intersects two events these must be back to back, so the intersection is not going to be the union of two disjoint intervals?}}
  which means $\bigcup_{\ell=1}^K \EE_{R_\ell}$ is single-intersecting. Now our claim simply follows from Lemma~\ref{lem:Kolmogorov stable for sc}.
\end{proof}
}

With Lemma~\ref{lem:stable favorite set}, we are ready to show that the revenue of any ASPE does not change much under $D$ and $\hat{D}$.\begin{lemma}\label{lem:difference in revenue Kolmogorov}
	For any ASPE$(p,\delta)$, let $\rev(p,\delta)$ and $\widehat{\rev}(p,\delta)$ be its expected revenue under $D$ and $\hat{D}$ respectively. If $||D_{ij}-\hat{D}_{ij}||_K\leq \xi$ for all $i\in[n]$ and $j\in[m]$, then $|\rev(p,\delta)-\widehat{\rev}(p,\delta)|\leq 2nm\xi\cdot \left(mH+\opt_{ASPE}\right)$,
	{where $\opt_{ASPE}$ is the optimal revenue obtainable by an ASPE mechanism.}
	\end{lemma}
\begin{prevproof}{Lemma}{lem:difference in revenue Kolmogorov}
	We use a hybrid argument. Consider a sequence of distributions $\{D^{(i)}\}_{i\leq n}$, where $D^{(i)}=\hat{D}_1\times\cdots\times\hat{D}_i\times D_{i+1}\times\cdots\times D_{n},$ and $D^{(0)}=D$, $D^{(n)}=\hat{D}$.
	 We use $\rev^{(i)}(p,\delta)$ to denote the expected revenue of ASPE$(p,\delta)$ under $D^{(i)}$. To prove our claim, it suffices to argue that $\left|\rev^{(i-1)}(p,\delta) -\rev^{(i)}(p,\delta)\right|\leq 2\xi m\cdot \left(m\cdot H+\opt\right).$
	  We denote by $\SS_k$ and $\SS'_k$ the random set of items that remain available after visiting the first $k$ bidders under $D^{(i-1)}$ and $D^{(i)}$. Clearly, for $k\leq i-1$, $||\SS_k-\SS'_k||_{TV}=0$, so the expected revenue collected from the first $i-1$ bidders under $D^{(i-1)}$ and $D^{(i)}$ is the same. According to Lemma~\ref{lem:stable favorite set}, $||\SS_i-\SS'_i||_{TV}\leq 2m\cdot \xi$. The total amount of money bidder $i$ spends can never be higher than her value for receiving all the items which is at most $m\cdot H$. So the difference in the expected revenue collected from bidder $i$ under  $D^{(i-1)}$ and $D^{(i)}$ is at most $2\xi\cdot m^2H$. Suppose $R$ is the set of remaining items after visiting the first $i$ bidders, then the expected revenue collected from the last $n-i$ bidders is the same under  $D^{(i-1)}$ and $D^{(i)}$, as these bidders have the same distributions. Moreover, this expected revenue is no more than $\opt_{ASPE}$, since the optimal ASPE can simply just sell $R$ to the last $n-i$ bidders using the same prices and entry fee as in ASPE$(p,\delta)$. Of course, for any fixed $R$, the probabilities that $\SS_i=R$ and $\SS'_i=R$ are different, but since for any $R$ the expected revenue from the last $n-i$ bidders is at most $\opt_{ASPE}$, the difference in the expected revenue from the last $n-i$ bidders under  $D^{(i-1)}$ and $D^{(i)}$ is at most $||\SS_i-\SS'_i||_{TV} \cdot \opt \leq 2\xi\cdot m\opt_{ASPE}$. Hence, the total difference between $\rev^{(i-1)}(p,\delta)$ and  $\rev^{(i)}(p,\delta)$ is at most $2\xi m\cdot \left(m H+\opt_{ASPE}\right)$. 
	  Furthermore, $\left|\rev(p,\delta)-\widehat{\rev}(p,\delta)\right|\leq \sum_{i=1}^{n}\left|\rev^{(i-1)}(p,\delta) -\rev^{(i)}(p,\delta)\right|\leq 2 nm\xi \left(mH+\opt_{ASPE}\right).$
\end{prevproof}

Using  Lemma~\ref{lem:difference in revenue Kolmogorov}, we can prove the main Theorem of this section. The proof is postponed to Appendix~\ref{sec:appx constrained additive}.

\begin{theorem}\label{thm:constrained additive Kolmogorov}
	When all bidders' valuations are drawn from distributions that are constrained additive, and for any bidder $i$ and any item $j$, $D_{ij}$ and $\hat{D}_{ij}$ are supported on $[0,H]$ and $||D_{ij}-\hat{D}_{ij}||_K\leq \xi$  for some $\xi=O(\frac{1}{nm})$, then with only access to $\hat{D}=\times_{i,j} \hat{D}_{ij}$, our algorithm can learn an RSPM and an ASPE such that the better of the two mechanisms has revenue at least $\frac{\opt}{c}-{\xi\cdot O(m^2n H)}$, where $\opt$ is the optimal revenue by any BIC mechanism under $D=\times_{i,j} D_{ij}$. $c>1$ is an absolute constant.%\footnote{\cnote{I'm confused why I don't see an $n^2$ in the error term (which would come from the RSPM error term), and also why there is no constant factor in front of the error term.} \yangnote{I added the constant factor in front of the error term, but I'm not sure why there will be an $n^2$ error term from RSPM is $\xi = O(1/nm)$.}} %\yangnote{With the same number of samples, we can learn in polynomial time with probability $1-\delta$ an SPM whose revenue is at least $\frac{\opt}{144}-\epsilon H$.}
\end{theorem}

}

\subsection{XOS Valuations: sample access to bounded and regular distributions}\label{sec:XOS sample}
In this section, we consider how to learn an ASPE with high revenue given sample access to $D$. Our learning algorithm is a two-step procedure. In the first step, we take a few samples from $D$ and use these samples to set the entry fee for every collection of prices $\{p_j\}_{j\in[m]}$ in the $\epsilon$-net. More specifically, to decide $\delta_i(S)$ we compute the utility of bidder $i$ for set $S$ under $\{p_j\}_{j\in[m]}$ over all the samples and take the empirical median among all these utilities to be $\delta_i(S)$. With a polynomial number of samples, we can guarantee that for any $\{p_j\}_{j\in[m]}$ in the $\epsilon$-net the computed entry fee functions $\{\delta_i(\cdot)\}_{i\in[n]}$ are $\mu$-balanced. Now, we have created an ASPE for every $\{p_j\}_{j\in[m]}$ in the $\epsilon$-net. In the second step, we take some fresh samples from $D$ and use them to estimate the revenue for each of the ASPEs we created in the first step,  then pick the one that has the highest empirical revenue. It is not hard to argue that with a polynomial number of samples the mechanism we pick has high revenue with probability almost $1$. Combining our algorithm with Theorem~\ref{thm:unit-demand}, we obtain the following theorem.
\begin{theorem}\label{thm:XOS sample}
	When all bidders' valuations are XOS over independent items and
	\begin{itemize}
		\item the random variable $V_i(t_{ij})$ is supported on $[0,H]$ for each bidder $i$ and item $j$, we can learn an RSPM and an ASPE such that with probability at least $1-\delta$ the better of the two mechanisms has revenue at least $\frac{\opt}{c_1}-\xi\cdot H$ for some absolute constant $c_1>1$ given $O\left((\frac{mn}{\xi})^2 \cdot (m\cdot\log \frac{m+n}{\xi} + \log \frac{1}{\delta})\right)$ samples from $D$; 
		\item  the random variable $V_i(t_{ij})$ is regular for each bidder $i$ and item $j$, we can learn an RSPM and an ASPE such that with probability at least $1-\delta$ the better of the two mechanisms has revenue at least $\frac{\opt}{c_2}$ for some absolute constant $c_2>1$ given $O\left(\max\{m,n\}^2m^2n^2 \left(m\log ({m+n}) + \log \frac{1}{\delta}\right)\right)$  samples from $D$.
	\end{itemize}
\end{theorem}

 The bounded case is proved as Theorem~\ref{thm:XOS bounded} in Appendix~\ref{sec:XOS bounded}.  The regular case is proved as Theorem~\ref{thm:XOS regular} in Appendix~\ref{sec:XOS bounded}.

\section{Subadditive Agents}\label{subadditive}
In this section we present a reduction from subadditive agents to XOS agents. More precisely, we show for every subadditive set function $f(.)$, there exists an XOS function $g(.)$, where $g$ is dominated by $f$ but the maxmin value of $g$ is within a logarithmic factor of the maxmin value of $f$. We begin by an observation. Suppose we are given a subadditive function $f$ on set $\domp(f)$, and we wish to approximate $f$ with an additive function $g$ which is dominated by $f$. In other words, we wish to find an additive function $g$ such that 
$$\forall S \subseteq \domp(f) \hspace{1cm} g(S) \leq f(S)$$
and $g(\domp(f))$ is maximized. One way to formulate $g$ is via a linear program. Suppose $\domp(f)=\{\ite_1,\ite_2,\ldots,\ite_m\}$ and let $g_1, g_2, \ldots, g_m$ be $m$ variables that describe $g$ in the following way:
$$\forall S \subseteq \domp(f) \hspace{1cm} g(S) = \sum_{\ite_i \in S} g_i.$$
Based on this formulation, we can find the optimal additive function $g$ by LP \ref{lp1}.
\begin{alignat}{3}\label{lp1}
\text{maximize: }& \hspace{0.5cm} &  \sum_{\ite_i \in \domp(f)}g_i & &\\
\text{subject to: }& & \sum_{\ite_i \in S}g_i \leq f(S) & \hspace{1cm}&\forall S \subseteq \domp(f)\nonumber\\
& & g_i \geq 0 & &\forall \ite_i \in \domp(f)\nonumber
\end{alignat}
We show the objective function of LP \ref{lp1} is lower bounded by $f(\domp(f)) / \log m$. The basic idea is to first write the dual program and then based on a probabilistic method, lower bound the optimal value of the dual program by $f(\domp(f))/ \log m$. 
\begin{lemma}\label{jj1}
	The optimal solution of LP \ref{lp1} is at least $f(\domp(f))/ \log m$.
\end{lemma}
\begin{proof}
To prove the lemma, we write the dual of LP \ref{lp1} as follows:
\begin{alignat}{3}\label{lp2}
\text{minimize: }& \hspace{0.5cm}& \sum_{S \subseteq \domp(f)} \alpha_S f(S)   & &\\
\text{subject to: }& & \sum_{S \ni \ite_i} \alpha_S \geq 1 & \hspace{1cm}&\forall \ite_i \in \domp(f)\nonumber\\
& & \alpha_S \geq 0 & &\forall S \subseteq \domp(f)\nonumber
\end{alignat}
By the strong duality theorem, the optimal solutions of LP \ref{lp1} and LP \ref{lp2} are equal~\cite{bachem1992linear}. Next, based on the optimal solution of LP \ref{lp2}, we define a randomized procedure to draw a set of elements: We start with an empty set $S^*$ and for every set $S \subseteq \domp(f)$ we add \textit{all} elements of $S$ to $S^*$ with probability $\alpha_S$. Since $f$ is subadditive, the marginal increase of $f(S^*)$ by adding elements of a set $S$ to $S^*$ is bounded by $f(S)$ and thus the expected value of $f(S^*)$ is bounded by the objective of LP \ref{lp2}. In other words:
\begin{equation}\label{jef0}
\mathbb{E}[f(S^*)] \leq \sum_{S \subseteq \domp(f)} \alpha_S f(S)
\end{equation}
Remark that we repeat this procedure for all subsets of $\domp(S)$ independently and thus for every $\ite_i \in \domp(f)$, $\sum_{S \ni \ite_i} \alpha_S \geq 1$ holds we have
\begin{equation}\label{jef1}
\mathsf{PR}[\ite_i \in S^*] \geq 1-1/e \simeq 0.632121 > 1/2
\end{equation}
for every element $\ite_i \in \domp(s)$. Now, with the same procedure, we draw $\lceil \log m \rceil + 2$ sets $S^*_1, S^*_2, \ldots, S^*_{\lceil \log m \rceil + 2}$ \textit{independently}. We define $\hat{S} = \bigcup S^*_i$. By Inequality \eqref{jef1} and the union bound we show
\begin{equation*}
\begin{split}
\mathsf{PR}[\hat{S} = \domp(f)] & \geq 1- \sum_{\ite_i \in \domp(i)} \mathsf{PR}[\ite_i \notin \hat{S}]\\
& = 1- \sum_{\ite_i \in \domp(i)} \mathsf{PR}[\ite_i \notin S^*_1 \text{ and } \ite_i \notin S^*_1 \text{ and } \ldots \text{ and } \ite_i \notin S^*_{\lceil \log m \rceil + 2}]\\
& = 1- \sum_{\ite_i \in \domp(i)} \prod_{j=1}^{\lceil \log m \rceil + 2} \mathsf{PR}[\ite_i \notin S^*_j]\\
& \geq 1- \sum_{\ite_i \in \domp(i)} \prod_{j=1}^{\lceil \log m \rceil + 2} 1/2\\
& = 1- \sum_{\ite_i \in \domp(i)} \prod_{j=1}^{\lceil \log m \rceil + 2} \mathsf{PR}[\ite_i \notin S^*_j]\\
& \geq 1- \sum_{\ite_i \in \domp(i)} 1/4m\\
& = 1- 1/4\\
& = 3/4\\
\end{split}
\end{equation*}
and thus $\mathbb{E}[f(\hat{S})] \geq 3/4 f(\domp(f))$. On the other hand, by the linearity of expectation and the fact that $f$ is subadditive we have:
\begin{equation*}
\begin{split}
\mathbb{E}[f(\hat{S})] &= \mathbb{E}[f(\bigcup S^*_i)]\\
& \leq \mathbb{E}[\sum f(S^*_i)]\\
& \leq (\lceil \log m \rceil + 2) (\sum_{S \subseteq \domp(f)} \alpha_S f(S))
\end{split}
\end{equation*}
Therefore $\sum_{S \subseteq \domp(f)} \alpha_S f(S) \geq 3/4 f(\domp(f)) / (\lceil \log m \rceil + 2)$, which means $$\sum_{S \subseteq \domp(f)} \alpha_S f(S) \geq f(\domp(f)) / (2\lceil \log m \rceil)$$ for big enough $m$. This shows the optimal solution of LP \ref{lp1} is lower bounded by $f(\domp(f)) / (2\lceil \log m \rceil)$ and the proof is complete.
\end{proof}

In what follows, based on Lemma \ref{jj1}, we provide a reduction from subadditive agents to XOS agents. An immediate corollary of Lemma \ref{jj1} is the following:
\begin{corollary}[of Lemma \ref{jj1}]\label{kk}
For any subadditive function $f$ and integer number $n$, there exists an XOS function $g$ such that
$$g(S) \leq f(S) \qquad \forall S \subseteq \domp(f)$$
and 
$$\MMS_g^n \geq \MMS_f^n/2\lceil \log n \rceil.$$
\end{corollary} 
\begin{proof}
	By definition, we can divide the items into $n$ disjoint sets such that the value of $f$ for every set is at least $\MMS_f^n$. Now, based on Lemma \ref{jj1}, we approximate $f$ for each set with an additive function $g_i$ wile losing a factor of at most $\lceil 2 |\log \domp(f)|\rceil$ and finally we set $g = \max g_i$. Based on Lemma \ref{jj1}, both conditions of this lemma are satisfied by $g$.
\end{proof}

Based on Theorem \ref{xosproof} and Lemma \ref{kk} one can show that a $1/10\lceil \log m \rceil$-$\MMS$ allocation is always possible for subadditive agents.
\begin{theorem}\label{subadditiveproof}
	The fair allocation problem with subadditive agents admits a $1/10\lceil \log m \rceil$-$\MMS$ allocation.
\end{theorem}
\appendix
\chapter{Supplementary Material}
\label{appendix}

In this appendix, we present supplementary material for the techniques and
experiments presented in the main text.

\section{Baseline Results and Analysis for Informed Sampler}
\label{appendix:chap3}

Here, we give an in-depth
performance analysis of the various samplers and the effect of their
hyperparameters. We choose hyperparameters with the lowest PSRF value
after $10k$ iterations, for each sampler individually. If the
differences between PSRF are not significantly different among
multiple values, we choose the one that has the highest acceptance
rate.

\subsection{Experiment: Estimating Camera Extrinsics}
\label{appendix:chap3:room}

\subsubsection{Parameter Selection}
\paragraph{Metropolis Hastings (\MH)}

Figure~\ref{fig:exp1_MH} shows the median acceptance rates and PSRF
values corresponding to various proposal standard deviations of plain
\MH~sampling. Mixing gets better and the acceptance rate gets worse as
the standard deviation increases. The value $0.3$ is selected standard
deviation for this sampler.

\paragraph{Metropolis Hastings Within Gibbs (\MHWG)}

As mentioned in Section~\ref{sec:room}, the \MHWG~sampler with one-dimensional
updates did not converge for any value of proposal standard deviation.
This problem has high correlation of the camera parameters and is of
multi-modal nature, which this sampler has problems with.

\paragraph{Parallel Tempering (\PT)}

For \PT~sampling, we took the best performing \MH~sampler and used
different temperature chains to improve the mixing of the
sampler. Figure~\ref{fig:exp1_PT} shows the results corresponding to
different combination of temperature levels. The sampler with
temperature levels of $[1,3,27]$ performed best in terms of both
mixing and acceptance rate.

\paragraph{Effect of Mixture Coefficient in Informed Sampling (\MIXLMH)}

Figure~\ref{fig:exp1_alpha} shows the effect of mixture
coefficient ($\alpha$) on the informed sampling
\MIXLMH. Since there is no significant different in PSRF values for
$0 \le \alpha \le 0.7$, we chose $0.7$ due to its high acceptance
rate.


% \end{multicols}

\begin{figure}[h]
\centering
  \subfigure[MH]{%
    \includegraphics[width=.48\textwidth]{figures/supplementary/camPose_MH.pdf} \label{fig:exp1_MH}
  }
  \subfigure[PT]{%
    \includegraphics[width=.48\textwidth]{figures/supplementary/camPose_PT.pdf} \label{fig:exp1_PT}
  }
\\
  \subfigure[INF-MH]{%
    \includegraphics[width=.48\textwidth]{figures/supplementary/camPose_alpha.pdf} \label{fig:exp1_alpha}
  }
  \mycaption{Results of the `Estimating Camera Extrinsics' experiment}{PRSFs and Acceptance rates corresponding to (a) various standard deviations of \MH, (b) various temperature level combinations of \PT sampling and (c) various mixture coefficients of \MIXLMH sampling.}
\end{figure}



\begin{figure}[!t]
\centering
  \subfigure[\MH]{%
    \includegraphics[width=.48\textwidth]{figures/supplementary/occlusionExp_MH.pdf} \label{fig:exp2_MH}
  }
  \subfigure[\BMHWG]{%
    \includegraphics[width=.48\textwidth]{figures/supplementary/occlusionExp_BMHWG.pdf} \label{fig:exp2_BMHWG}
  }
\\
  \subfigure[\MHWG]{%
    \includegraphics[width=.48\textwidth]{figures/supplementary/occlusionExp_MHWG.pdf} \label{fig:exp2_MHWG}
  }
  \subfigure[\PT]{%
    \includegraphics[width=.48\textwidth]{figures/supplementary/occlusionExp_PT.pdf} \label{fig:exp2_PT}
  }
\\
  \subfigure[\INFBMHWG]{%
    \includegraphics[width=.5\textwidth]{figures/supplementary/occlusionExp_alpha.pdf} \label{fig:exp2_alpha}
  }
  \mycaption{Results of the `Occluding Tiles' experiment}{PRSF and
    Acceptance rates corresponding to various standard deviations of
    (a) \MH, (b) \BMHWG, (c) \MHWG, (d) various temperature level
    combinations of \PT~sampling and; (e) various mixture coefficients
    of our informed \INFBMHWG sampling.}
\end{figure}

%\onecolumn\newpage\twocolumn
\subsection{Experiment: Occluding Tiles}
\label{appendix:chap3:tiles}

\subsubsection{Parameter Selection}

\paragraph{Metropolis Hastings (\MH)}

Figure~\ref{fig:exp2_MH} shows the results of
\MH~sampling. Results show the poor convergence for all proposal
standard deviations and rapid decrease of AR with increasing standard
deviation. This is due to the high-dimensional nature of
the problem. We selected a standard deviation of $1.1$.

\paragraph{Blocked Metropolis Hastings Within Gibbs (\BMHWG)}

The results of \BMHWG are shown in Figure~\ref{fig:exp2_BMHWG}. In
this sampler we update only one block of tile variables (of dimension
four) in each sampling step. Results show much better performance
compared to plain \MH. The optimal proposal standard deviation for
this sampler is $0.7$.

\paragraph{Metropolis Hastings Within Gibbs (\MHWG)}

Figure~\ref{fig:exp2_MHWG} shows the result of \MHWG sampling. This
sampler is better than \BMHWG and converges much more quickly. Here
a standard deviation of $0.9$ is found to be best.

\paragraph{Parallel Tempering (\PT)}

Figure~\ref{fig:exp2_PT} shows the results of \PT sampling with various
temperature combinations. Results show no improvement in AR from plain
\MH sampling and again $[1,3,27]$ temperature levels are found to be optimal.

\paragraph{Effect of Mixture Coefficient in Informed Sampling (\INFBMHWG)}

Figure~\ref{fig:exp2_alpha} shows the effect of mixture
coefficient ($\alpha$) on the blocked informed sampling
\INFBMHWG. Since there is no significant different in PSRF values for
$0 \le \alpha \le 0.8$, we chose $0.8$ due to its high acceptance
rate.



\subsection{Experiment: Estimating Body Shape}
\label{appendix:chap3:body}

\subsubsection{Parameter Selection}
\paragraph{Metropolis Hastings (\MH)}

Figure~\ref{fig:exp3_MH} shows the result of \MH~sampling with various
proposal standard deviations. The value of $0.1$ is found to be
best.

\paragraph{Metropolis Hastings Within Gibbs (\MHWG)}

For \MHWG sampling we select $0.3$ proposal standard
deviation. Results are shown in Fig.~\ref{fig:exp3_MHWG}.


\paragraph{Parallel Tempering (\PT)}

As before, results in Fig.~\ref{fig:exp3_PT}, the temperature levels
were selected to be $[1,3,27]$ due its slightly higher AR.

\paragraph{Effect of Mixture Coefficient in Informed Sampling (\MIXLMH)}

Figure~\ref{fig:exp3_alpha} shows the effect of $\alpha$ on PSRF and
AR. Since there is no significant differences in PSRF values for $0 \le
\alpha \le 0.8$, we choose $0.8$.


\begin{figure}[t]
\centering
  \subfigure[\MH]{%
    \includegraphics[width=.48\textwidth]{figures/supplementary/bodyShape_MH.pdf} \label{fig:exp3_MH}
  }
  \subfigure[\MHWG]{%
    \includegraphics[width=.48\textwidth]{figures/supplementary/bodyShape_MHWG.pdf} \label{fig:exp3_MHWG}
  }
\\
  \subfigure[\PT]{%
    \includegraphics[width=.48\textwidth]{figures/supplementary/bodyShape_PT.pdf} \label{fig:exp3_PT}
  }
  \subfigure[\MIXLMH]{%
    \includegraphics[width=.48\textwidth]{figures/supplementary/bodyShape_alpha.pdf} \label{fig:exp3_alpha}
  }
\\
  \mycaption{Results of the `Body Shape Estimation' experiment}{PRSFs and
    Acceptance rates corresponding to various standard deviations of
    (a) \MH, (b) \MHWG; (c) various temperature level combinations
    of \PT sampling and; (d) various mixture coefficients of the
    informed \MIXLMH sampling.}
\end{figure}


\subsection{Results Overview}
Figure~\ref{fig:exp_summary} shows the summary results of the all the three
experimental studies related to informed sampler.
\begin{figure*}[h!]
\centering
  \subfigure[Results for: Estimating Camera Extrinsics]{%
    \includegraphics[width=0.9\textwidth]{figures/supplementary/camPose_ALL.pdf} \label{fig:exp1_all}
  }
  \subfigure[Results for: Occluding Tiles]{%
    \includegraphics[width=0.9\textwidth]{figures/supplementary/occlusionExp_ALL.pdf} \label{fig:exp2_all}
  }
  \subfigure[Results for: Estimating Body Shape]{%
    \includegraphics[width=0.9\textwidth]{figures/supplementary/bodyShape_ALL.pdf} \label{fig:exp3_all}
  }
  \label{fig:exp_summary}
  \mycaption{Summary of the statistics for the three experiments}{Shown are
    for several baseline methods and the informed samplers the
    acceptance rates (left), PSRFs (middle), and RMSE values
    (right). All results are median results over multiple test
    examples.}
\end{figure*}

\subsection{Additional Qualitative Results}

\subsubsection{Occluding Tiles}
In Figure~\ref{fig:exp2_visual_more} more qualitative results of the
occluding tiles experiment are shown. The informed sampling approach
(\INFBMHWG) is better than the best baseline (\MHWG). This still is a
very challenging problem since the parameters for occluded tiles are
flat over a large region. Some of the posterior variance of the
occluded tiles is already captured by the informed sampler.

\begin{figure*}[h!]
\begin{center}
\centerline{\includegraphics[width=0.95\textwidth]{figures/supplementary/occlusionExp_Visual.pdf}}
\mycaption{Additional qualitative results of the occluding tiles experiment}
  {From left to right: (a)
  Given image, (b) Ground truth tiles, (c) OpenCV heuristic and most probable estimates
  from 5000 samples obtained by (d) MHWG sampler (best baseline) and
  (e) our INF-BMHWG sampler. (f) Posterior expectation of the tiles
  boundaries obtained by INF-BMHWG sampling (First 2000 samples are
  discarded as burn-in).}
\label{fig:exp2_visual_more}
\end{center}
\end{figure*}

\subsubsection{Body Shape}
Figure~\ref{fig:exp3_bodyMeshes} shows some more results of 3D mesh
reconstruction using posterior samples obtained by our informed
sampling \MIXLMH.

\begin{figure*}[t]
\begin{center}
\centerline{\includegraphics[width=0.75\textwidth]{figures/supplementary/bodyMeshResults.pdf}}
\mycaption{Qualitative results for the body shape experiment}
  {Shown is the 3D mesh reconstruction results with first 1000 samples obtained
  using the \MIXLMH informed sampling method. (blue indicates small
  values and red indicates high values)}
\label{fig:exp3_bodyMeshes}
\end{center}
\end{figure*}

\clearpage



\section{Additional Results on the Face Problem with CMP}

Figure~\ref{fig:shading-qualitative-multiple-subjects-supp} shows inference results for reflectance maps, normal maps and lights for randomly chosen test images, and Fig.~\ref{fig:shading-qualitative-same-subject-supp} shows reflectance estimation results on multiple images of the same subject produced under different illumination conditions. CMP is able to produce estimates that are closer to the groundtruth across different subjects and illumination conditions.

\begin{figure*}[h]
  \begin{center}
  \centerline{\includegraphics[width=1.0\columnwidth]{figures/face_cmp_visual_results_supp.pdf}}
  \vspace{-1.2cm}
  \end{center}
	\mycaption{A visual comparison of inference results}{(a)~Observed images. (b)~Inferred reflectance maps. \textit{GT} is the photometric stereo groundtruth, \textit{BU} is the Biswas \etal (2009) reflectance estimate and \textit{Forest} is the consensus prediction. (c)~The variance of the inferred reflectance estimate produced by \MTD (normalized across rows).(d)~Visualization of inferred light directions. (e)~Inferred normal maps.}
	\label{fig:shading-qualitative-multiple-subjects-supp}
\end{figure*}


\begin{figure*}[h]
	\centering
	\setlength\fboxsep{0.2mm}
	\setlength\fboxrule{0pt}
	\begin{tikzpicture}

		\matrix at (0, 0) [matrix of nodes, nodes={anchor=east}, column sep=-0.05cm, row sep=-0.2cm]
		{
			\fbox{\includegraphics[width=1cm]{figures/sample_3_4_X.png}} &
			\fbox{\includegraphics[width=1cm]{figures/sample_3_4_GT.png}} &
			\fbox{\includegraphics[width=1cm]{figures/sample_3_4_BISWAS.png}}  &
			\fbox{\includegraphics[width=1cm]{figures/sample_3_4_VMP.png}}  &
			\fbox{\includegraphics[width=1cm]{figures/sample_3_4_FOREST.png}}  &
			\fbox{\includegraphics[width=1cm]{figures/sample_3_4_CMP.png}}  &
			\fbox{\includegraphics[width=1cm]{figures/sample_3_4_CMPVAR.png}}
			 \\

			\fbox{\includegraphics[width=1cm]{figures/sample_3_5_X.png}} &
			\fbox{\includegraphics[width=1cm]{figures/sample_3_5_GT.png}} &
			\fbox{\includegraphics[width=1cm]{figures/sample_3_5_BISWAS.png}}  &
			\fbox{\includegraphics[width=1cm]{figures/sample_3_5_VMP.png}}  &
			\fbox{\includegraphics[width=1cm]{figures/sample_3_5_FOREST.png}}  &
			\fbox{\includegraphics[width=1cm]{figures/sample_3_5_CMP.png}}  &
			\fbox{\includegraphics[width=1cm]{figures/sample_3_5_CMPVAR.png}}
			 \\

			\fbox{\includegraphics[width=1cm]{figures/sample_3_6_X.png}} &
			\fbox{\includegraphics[width=1cm]{figures/sample_3_6_GT.png}} &
			\fbox{\includegraphics[width=1cm]{figures/sample_3_6_BISWAS.png}}  &
			\fbox{\includegraphics[width=1cm]{figures/sample_3_6_VMP.png}}  &
			\fbox{\includegraphics[width=1cm]{figures/sample_3_6_FOREST.png}}  &
			\fbox{\includegraphics[width=1cm]{figures/sample_3_6_CMP.png}}  &
			\fbox{\includegraphics[width=1cm]{figures/sample_3_6_CMPVAR.png}}
			 \\
	     };

       \node at (-3.85, -2.0) {\small Observed};
       \node at (-2.55, -2.0) {\small `GT'};
       \node at (-1.27, -2.0) {\small BU};
       \node at (0.0, -2.0) {\small MP};
       \node at (1.27, -2.0) {\small Forest};
       \node at (2.55, -2.0) {\small \textbf{CMP}};
       \node at (3.85, -2.0) {\small Variance};

	\end{tikzpicture}
	\mycaption{Robustness to varying illumination}{Reflectance estimation on a subject images with varying illumination. Left to right: observed image, photometric stereo estimate (GT)
  which is used as a proxy for groundtruth, bottom-up estimate of \cite{Biswas2009}, VMP result, consensus forest estimate, CMP mean, and CMP variance.}
	\label{fig:shading-qualitative-same-subject-supp}
\end{figure*}

\clearpage

\section{Additional Material for Learning Sparse High Dimensional Filters}
\label{sec:appendix-bnn}

This part of supplementary material contains a more detailed overview of the permutohedral
lattice convolution in Section~\ref{sec:permconv}, more experiments in
Section~\ref{sec:addexps} and additional results with protocols for
the experiments presented in Chapter~\ref{chap:bnn} in Section~\ref{sec:addresults}.

\vspace{-0.2cm}
\subsection{General Permutohedral Convolutions}
\label{sec:permconv}

A core technical contribution of this work is the generalization of the Gaussian permutohedral lattice
convolution proposed in~\cite{adams2010fast} to the full non-separable case with the
ability to perform back-propagation. Although, conceptually, there are minor
differences between Gaussian and general parameterized filters, there are non-trivial practical
differences in terms of the algorithmic implementation. The Gauss filters belong to
the separable class and can thus be decomposed into multiple
sequential one dimensional convolutions. We are interested in the general filter
convolutions, which can not be decomposed. Thus, performing a general permutohedral
convolution at a lattice point requires the computation of the inner product with the
neighboring elements in all the directions in the high-dimensional space.

Here, we give more details of the implementation differences of separable
and non-separable filters. In the following, we will explain the scalar case first.
Recall, that the forward pass of general permutohedral convolution
involves 3 steps: \textit{splatting}, \textit{convolving} and \textit{slicing}.
We follow the same splatting and slicing strategies as in~\cite{adams2010fast}
since these operations do not depend on the filter kernel. The main difference
between our work and the existing implementation of~\cite{adams2010fast} is
the way that the convolution operation is executed. This proceeds by constructing
a \emph{blur neighbor} matrix $K$ that stores for every lattice point all
values of the lattice neighbors that are needed to compute the filter output.

\begin{figure}[t!]
  \centering
    \includegraphics[width=0.6\columnwidth]{figures/supplementary/lattice_construction}
  \mycaption{Illustration of 1D permutohedral lattice construction}
  {A $4\times 4$ $(x,y)$ grid lattice is projected onto the plane defined by the normal
  vector $(1,1)^{\top}$. This grid has $s+1=4$ and $d=2$ $(s+1)^{d}=4^2=16$ elements.
  In the projection, all points of the same color are projected onto the same points in the plane.
  The number of elements of the projected lattice is $t=(s+1)^d-s^d=4^2-3^2=7$, that is
  the $(4\times 4)$ grid minus the size of lattice that is $1$ smaller at each size, in this
  case a $(3\times 3)$ lattice (the upper right $(3\times 3)$ elements).
  }
\label{fig:latticeconstruction}
\end{figure}

The blur neighbor matrix is constructed by traversing through all the populated
lattice points and their neighboring elements.
% For efficiency, we do this matrix construction recursively with shared computations
% since $n^{th}$ neighbourhood elements are $1^{st}$ neighborhood elements of $n-1^{th}$ neighbourhood elements. \pg{do not understand}
This is done recursively to share computations. For any lattice point, the neighbors that are
$n$ hops away are the direct neighbors of the points that are $n-1$ hops away.
The size of a $d$ dimensional spatial filter with width $s+1$ is $(s+1)^{d}$ (\eg, a
$3\times 3$ filter, $s=2$ in $d=2$ has $3^2=9$ elements) and this size grows
exponentially in the number of dimensions $d$. The permutohedral lattice is constructed by
projecting a regular grid onto the plane spanned by the $d$ dimensional normal vector ${(1,\ldots,1)}^{\top}$. See
Fig.~\ref{fig:latticeconstruction} for an illustration of the 1D lattice construction.
Many corners of a grid filter are projected onto the same point, in total $t = {(s+1)}^{d} -
s^{d}$ elements remain in the permutohedral filter with $s$ neighborhood in $d-1$ dimensions.
If the lattice has $m$ populated elements, the
matrix $K$ has size $t\times m$. Note that, since the input signal is typically
sparse, only a few lattice corners are being populated in the \textit{slicing} step.
We use a hash-table to keep track of these points and traverse only through
the populated lattice points for this neighborhood matrix construction.

Once the blur neighbor matrix $K$ is constructed, we can perform the convolution
by the matrix vector multiplication
\begin{equation}
\ell' = BK,
\label{eq:conv}
\end{equation}
where $B$ is the $1 \times t$ filter kernel (whose values we will learn) and $\ell'\in\mathbb{R}^{1\times m}$
is the result of the filtering at the $m$ lattice points. In practice, we found that the
matrix $K$ is sometimes too large to fit into GPU memory and we divided the matrix $K$
into smaller pieces to compute Eq.~\ref{eq:conv} sequentially.

In the general multi-dimensional case, the signal $\ell$ is of $c$ dimensions. Then
the kernel $B$ is of size $c \times t$ and $K$ stores the $c$ dimensional vectors
accordingly. When the input and output points are different, we slice only the
input points and splat only at the output points.


\subsection{Additional Experiments}
\label{sec:addexps}
In this section, we discuss more use-cases for the learned bilateral filters, one
use-case of BNNs and two single filter applications for image and 3D mesh denoising.

\subsubsection{Recognition of subsampled MNIST}\label{sec:app_mnist}

One of the strengths of the proposed filter convolution is that it does not
require the input to lie on a regular grid. The only requirement is to define a distance
between features of the input signal.
We highlight this feature with the following experiment using the
classical MNIST ten class classification problem~\cite{lecun1998mnist}. We sample a
sparse set of $N$ points $(x,y)\in [0,1]\times [0,1]$
uniformly at random in the input image, use their interpolated values
as signal and the \emph{continuous} $(x,y)$ positions as features. This mimics
sub-sampling of a high-dimensional signal. To compare against a spatial convolution,
we interpolate the sparse set of values at the grid positions.

We take a reference implementation of LeNet~\cite{lecun1998gradient} that
is part of the Caffe project~\cite{jia2014caffe} and compare it
against the same architecture but replacing the first convolutional
layer with a bilateral convolution layer (BCL). The filter size
and numbers are adjusted to get a comparable number of parameters
($5\times 5$ for LeNet, $2$-neighborhood for BCL).

The results are shown in Table~\ref{tab:all-results}. We see that training
on the original MNIST data (column Original, LeNet vs. BNN) leads to a slight
decrease in performance of the BNN (99.03\%) compared to LeNet
(99.19\%). The BNN can be trained and evaluated on sparse
signals, and we resample the image as described above for $N=$ 100\%, 60\% and
20\% of the total number of pixels. The methods are also evaluated
on test images that are subsampled in the same way. Note that we can
train and test with different subsampling rates. We introduce an additional
bilinear interpolation layer for the LeNet architecture to train on the same
data. In essence, both models perform a spatial interpolation and thus we
expect them to yield a similar classification accuracy. Once the data is of
higher dimensions, the permutohedral convolution will be faster due to hashing
the sparse input points, as well as less memory demanding in comparison to
naive application of a spatial convolution with interpolated values.

\begin{table}[t]
  \begin{center}
    \footnotesize
    \centering
    \begin{tabular}[t]{lllll}
      \toprule
              &     & \multicolumn{3}{c}{Test Subsampling} \\
       Method  & Original & 100\% & 60\% & 20\%\\
      \midrule
       LeNet &  \textbf{0.9919} & 0.9660 & 0.9348 & \textbf{0.6434} \\
       BNN &  0.9903 & \textbf{0.9844} & \textbf{0.9534} & 0.5767 \\
      \hline
       LeNet 100\% & 0.9856 & 0.9809 & 0.9678 & \textbf{0.7386} \\
       BNN 100\% & \textbf{0.9900} & \textbf{0.9863} & \textbf{0.9699} & 0.6910 \\
      \hline
       LeNet 60\% & 0.9848 & 0.9821 & 0.9740 & 0.8151 \\
       BNN 60\% & \textbf{0.9885} & \textbf{0.9864} & \textbf{0.9771} & \textbf{0.8214}\\
      \hline
       LeNet 20\% & \textbf{0.9763} & \textbf{0.9754} & 0.9695 & 0.8928 \\
       BNN 20\% & 0.9728 & 0.9735 & \textbf{0.9701} & \textbf{0.9042}\\
      \bottomrule
    \end{tabular}
  \end{center}
\vspace{-.2cm}
\caption{Classification accuracy on MNIST. We compare the
    LeNet~\cite{lecun1998gradient} implementation that is part of
    Caffe~\cite{jia2014caffe} to the network with the first layer
    replaced by a bilateral convolution layer (BCL). Both are trained
    on the original image resolution (first two rows). Three more BNN
    and CNN models are trained with randomly subsampled images (100\%,
    60\% and 20\% of the pixels). An additional bilinear interpolation
    layer samples the input signal on a spatial grid for the CNN model.
  }
  \label{tab:all-results}
\vspace{-.5cm}
\end{table}

\subsubsection{Image Denoising}

The main application that inspired the development of the bilateral
filtering operation is image denoising~\cite{aurich1995non}, there
using a single Gaussian kernel. Our development allows to learn this
kernel function from data and we explore how to improve using a \emph{single}
but more general bilateral filter.

We use the Berkeley segmentation dataset
(BSDS500)~\cite{arbelaezi2011bsds500} as a test bed. The color
images in the dataset are converted to gray-scale,
and corrupted with Gaussian noise with a standard deviation of
$\frac {25} {255}$.

We compare the performance of four different filter models on a
denoising task.
The first baseline model (`Spatial' in Table \ref{tab:denoising}, $25$
weights) uses a single spatial filter with a kernel size of
$5$ and predicts the scalar gray-scale value at the center pixel. The next model
(`Gauss Bilateral') applies a bilateral \emph{Gaussian}
filter to the noisy input, using position and intensity features $\f=(x,y,v)^\top$.
The third setup (`Learned Bilateral', $65$ weights)
takes a Gauss kernel as initialization and
fits all filter weights on the train set to minimize the
mean squared error with respect to the clean images.
We run a combination
of spatial and permutohedral convolutions on spatial and bilateral
features (`Spatial + Bilateral (Learned)') to check for a complementary
performance of the two convolutions.

\label{sec:exp:denoising}
\begin{table}[!h]
\begin{center}
  \footnotesize
  \begin{tabular}[t]{lr}
    \toprule
    Method & PSNR \\
    \midrule
    Noisy Input & $20.17$ \\
    Spatial & $26.27$ \\
    Gauss Bilateral & $26.51$ \\
    Learned Bilateral & $26.58$ \\
    Spatial + Bilateral (Learned) & \textbf{$26.65$} \\
    \bottomrule
  \end{tabular}
\end{center}
\vspace{-0.5em}
\caption{PSNR results of a denoising task using the BSDS500
  dataset~\cite{arbelaezi2011bsds500}}
\vspace{-0.5em}
\label{tab:denoising}
\end{table}
\vspace{-0.2em}

The PSNR scores evaluated on full images of the test set are
shown in Table \ref{tab:denoising}. We find that an untrained bilateral
filter already performs better than a trained spatial convolution
($26.27$ to $26.51$). A learned convolution further improve the
performance slightly. We chose this simple one-kernel setup to
validate an advantage of the generalized bilateral filter. A competitive
denoising system would employ RGB color information and also
needs to be properly adjusted in network size. Multi-layer perceptrons
have obtained state-of-the-art denoising results~\cite{burger12cvpr}
and the permutohedral lattice layer can readily be used in such an
architecture, which is intended future work.

\subsection{Additional results}
\label{sec:addresults}

This section contains more qualitative results for the experiments presented in Chapter~\ref{chap:bnn}.

\begin{figure*}[th!]
  \centering
    \includegraphics[width=\columnwidth,trim={5cm 2.5cm 5cm 4.5cm},clip]{figures/supplementary/lattice_viz.pdf}
    \vspace{-0.7cm}
  \mycaption{Visualization of the Permutohedral Lattice}
  {Sample lattice visualizations for different feature spaces. All pixels falling in the same simplex cell are shown with
  the same color. $(x,y)$ features correspond to image pixel positions, and $(r,g,b) \in [0,255]$ correspond
  to the red, green and blue color values.}
\label{fig:latticeviz}
\end{figure*}

\subsubsection{Lattice Visualization}

Figure~\ref{fig:latticeviz} shows sample lattice visualizations for different feature spaces.

\newcolumntype{L}[1]{>{\raggedright\let\newline\\\arraybackslash\hspace{0pt}}b{#1}}
\newcolumntype{C}[1]{>{\centering\let\newline\\\arraybackslash\hspace{0pt}}b{#1}}
\newcolumntype{R}[1]{>{\raggedleft\let\newline\\\arraybackslash\hspace{0pt}}b{#1}}

\subsubsection{Color Upsampling}\label{sec:color_upsampling}
\label{sec:col_upsample_extra}

Some images of the upsampling for the Pascal
VOC12 dataset are shown in Fig.~\ref{fig:Colour_upsample_visuals}. It is
especially the low level image details that are better preserved with
a learned bilateral filter compared to the Gaussian case.

\begin{figure*}[t!]
  \centering
    \subfigure{%
   \raisebox{2.0em}{
    \includegraphics[width=.06\columnwidth]{figures/supplementary/2007_004969.jpg}
   }
  }
  \subfigure{%
    \includegraphics[width=.17\columnwidth]{figures/supplementary/2007_004969_gray.pdf}
  }
  \subfigure{%
    \includegraphics[width=.17\columnwidth]{figures/supplementary/2007_004969_gt.pdf}
  }
  \subfigure{%
    \includegraphics[width=.17\columnwidth]{figures/supplementary/2007_004969_bicubic.pdf}
  }
  \subfigure{%
    \includegraphics[width=.17\columnwidth]{figures/supplementary/2007_004969_gauss.pdf}
  }
  \subfigure{%
    \includegraphics[width=.17\columnwidth]{figures/supplementary/2007_004969_learnt.pdf}
  }\\
    \subfigure{%
   \raisebox{2.0em}{
    \includegraphics[width=.06\columnwidth]{figures/supplementary/2007_003106.jpg}
   }
  }
  \subfigure{%
    \includegraphics[width=.17\columnwidth]{figures/supplementary/2007_003106_gray.pdf}
  }
  \subfigure{%
    \includegraphics[width=.17\columnwidth]{figures/supplementary/2007_003106_gt.pdf}
  }
  \subfigure{%
    \includegraphics[width=.17\columnwidth]{figures/supplementary/2007_003106_bicubic.pdf}
  }
  \subfigure{%
    \includegraphics[width=.17\columnwidth]{figures/supplementary/2007_003106_gauss.pdf}
  }
  \subfigure{%
    \includegraphics[width=.17\columnwidth]{figures/supplementary/2007_003106_learnt.pdf}
  }\\
  \setcounter{subfigure}{0}
  \small{
  \subfigure[Inp.]{%
  \raisebox{2.0em}{
    \includegraphics[width=.06\columnwidth]{figures/supplementary/2007_006837.jpg}
   }
  }
  \subfigure[Guidance]{%
    \includegraphics[width=.17\columnwidth]{figures/supplementary/2007_006837_gray.pdf}
  }
   \subfigure[GT]{%
    \includegraphics[width=.17\columnwidth]{figures/supplementary/2007_006837_gt.pdf}
  }
  \subfigure[Bicubic]{%
    \includegraphics[width=.17\columnwidth]{figures/supplementary/2007_006837_bicubic.pdf}
  }
  \subfigure[Gauss-BF]{%
    \includegraphics[width=.17\columnwidth]{figures/supplementary/2007_006837_gauss.pdf}
  }
  \subfigure[Learned-BF]{%
    \includegraphics[width=.17\columnwidth]{figures/supplementary/2007_006837_learnt.pdf}
  }
  }
  \vspace{-0.5cm}
  \mycaption{Color Upsampling}{Color $8\times$ upsampling results
  using different methods, from left to right, (a)~Low-resolution input color image (Inp.),
  (b)~Gray scale guidance image, (c)~Ground-truth color image; Upsampled color images with
  (d)~Bicubic interpolation, (e) Gauss bilateral upsampling and, (f)~Learned bilateral
  updampgling (best viewed on screen).}

\label{fig:Colour_upsample_visuals}
\end{figure*}

\subsubsection{Depth Upsampling}
\label{sec:depth_upsample_extra}

Figure~\ref{fig:depth_upsample_visuals} presents some more qualitative results comparing bicubic interpolation, Gauss
bilateral and learned bilateral upsampling on NYU depth dataset image~\cite{silberman2012indoor}.

\subsubsection{Character Recognition}\label{sec:app_character}

 Figure~\ref{fig:nnrecognition} shows the schematic of different layers
 of the network architecture for LeNet-7~\cite{lecun1998mnist}
 and DeepCNet(5, 50)~\cite{ciresan2012multi,graham2014spatially}. For the BNN variants, the first layer filters are replaced
 with learned bilateral filters and are learned end-to-end.

\subsubsection{Semantic Segmentation}\label{sec:app_semantic_segmentation}
\label{sec:semantic_bnn_extra}

Some more visual results for semantic segmentation are shown in Figure~\ref{fig:semantic_visuals}.
These include the underlying DeepLab CNN\cite{chen2014semantic} result (DeepLab),
the 2 step mean-field result with Gaussian edge potentials (+2stepMF-GaussCRF)
and also corresponding results with learned edge potentials (+2stepMF-LearnedCRF).
In general, we observe that mean-field in learned CRF leads to slightly dilated
classification regions in comparison to using Gaussian CRF thereby filling-in the
false negative pixels and also correcting some mis-classified regions.

\begin{figure*}[t!]
  \centering
    \subfigure{%
   \raisebox{2.0em}{
    \includegraphics[width=.06\columnwidth]{figures/supplementary/2bicubic}
   }
  }
  \subfigure{%
    \includegraphics[width=.17\columnwidth]{figures/supplementary/2given_image}
  }
  \subfigure{%
    \includegraphics[width=.17\columnwidth]{figures/supplementary/2ground_truth}
  }
  \subfigure{%
    \includegraphics[width=.17\columnwidth]{figures/supplementary/2bicubic}
  }
  \subfigure{%
    \includegraphics[width=.17\columnwidth]{figures/supplementary/2gauss}
  }
  \subfigure{%
    \includegraphics[width=.17\columnwidth]{figures/supplementary/2learnt}
  }\\
    \subfigure{%
   \raisebox{2.0em}{
    \includegraphics[width=.06\columnwidth]{figures/supplementary/32bicubic}
   }
  }
  \subfigure{%
    \includegraphics[width=.17\columnwidth]{figures/supplementary/32given_image}
  }
  \subfigure{%
    \includegraphics[width=.17\columnwidth]{figures/supplementary/32ground_truth}
  }
  \subfigure{%
    \includegraphics[width=.17\columnwidth]{figures/supplementary/32bicubic}
  }
  \subfigure{%
    \includegraphics[width=.17\columnwidth]{figures/supplementary/32gauss}
  }
  \subfigure{%
    \includegraphics[width=.17\columnwidth]{figures/supplementary/32learnt}
  }\\
  \setcounter{subfigure}{0}
  \small{
  \subfigure[Inp.]{%
  \raisebox{2.0em}{
    \includegraphics[width=.06\columnwidth]{figures/supplementary/41bicubic}
   }
  }
  \subfigure[Guidance]{%
    \includegraphics[width=.17\columnwidth]{figures/supplementary/41given_image}
  }
   \subfigure[GT]{%
    \includegraphics[width=.17\columnwidth]{figures/supplementary/41ground_truth}
  }
  \subfigure[Bicubic]{%
    \includegraphics[width=.17\columnwidth]{figures/supplementary/41bicubic}
  }
  \subfigure[Gauss-BF]{%
    \includegraphics[width=.17\columnwidth]{figures/supplementary/41gauss}
  }
  \subfigure[Learned-BF]{%
    \includegraphics[width=.17\columnwidth]{figures/supplementary/41learnt}
  }
  }
  \mycaption{Depth Upsampling}{Depth $8\times$ upsampling results
  using different upsampling strategies, from left to right,
  (a)~Low-resolution input depth image (Inp.),
  (b)~High-resolution guidance image, (c)~Ground-truth depth; Upsampled depth images with
  (d)~Bicubic interpolation, (e) Gauss bilateral upsampling and, (f)~Learned bilateral
  updampgling (best viewed on screen).}

\label{fig:depth_upsample_visuals}
\end{figure*}

\subsubsection{Material Segmentation}\label{sec:app_material_segmentation}
\label{sec:material_bnn_extra}

In Fig.~\ref{fig:material_visuals-app2}, we present visual results comparing 2 step
mean-field inference with Gaussian and learned pairwise CRF potentials. In
general, we observe that the pixels belonging to dominant classes in the
training data are being more accurately classified with learned CRF. This leads to
a significant improvements in overall pixel accuracy. This also results
in a slight decrease of the accuracy from less frequent class pixels thereby
slightly reducing the average class accuracy with learning. We attribute this
to the type of annotation that is available for this dataset, which is not
for the entire image but for some segments in the image. We have very few
images of the infrequent classes to combat this behaviour during training.

\subsubsection{Experiment Protocols}
\label{sec:protocols}

Table~\ref{tbl:parameters} shows experiment protocols of different experiments.

 \begin{figure*}[t!]
  \centering
  \subfigure[LeNet-7]{
    \includegraphics[width=0.7\columnwidth]{figures/supplementary/lenet_cnn_network}
    }\\
    \subfigure[DeepCNet]{
    \includegraphics[width=\columnwidth]{figures/supplementary/deepcnet_cnn_network}
    }
  \mycaption{CNNs for Character Recognition}
  {Schematic of (top) LeNet-7~\cite{lecun1998mnist} and (bottom) DeepCNet(5,50)~\cite{ciresan2012multi,graham2014spatially} architectures used in Assamese
  character recognition experiments.}
\label{fig:nnrecognition}
\end{figure*}

\definecolor{voc_1}{RGB}{0, 0, 0}
\definecolor{voc_2}{RGB}{128, 0, 0}
\definecolor{voc_3}{RGB}{0, 128, 0}
\definecolor{voc_4}{RGB}{128, 128, 0}
\definecolor{voc_5}{RGB}{0, 0, 128}
\definecolor{voc_6}{RGB}{128, 0, 128}
\definecolor{voc_7}{RGB}{0, 128, 128}
\definecolor{voc_8}{RGB}{128, 128, 128}
\definecolor{voc_9}{RGB}{64, 0, 0}
\definecolor{voc_10}{RGB}{192, 0, 0}
\definecolor{voc_11}{RGB}{64, 128, 0}
\definecolor{voc_12}{RGB}{192, 128, 0}
\definecolor{voc_13}{RGB}{64, 0, 128}
\definecolor{voc_14}{RGB}{192, 0, 128}
\definecolor{voc_15}{RGB}{64, 128, 128}
\definecolor{voc_16}{RGB}{192, 128, 128}
\definecolor{voc_17}{RGB}{0, 64, 0}
\definecolor{voc_18}{RGB}{128, 64, 0}
\definecolor{voc_19}{RGB}{0, 192, 0}
\definecolor{voc_20}{RGB}{128, 192, 0}
\definecolor{voc_21}{RGB}{0, 64, 128}
\definecolor{voc_22}{RGB}{128, 64, 128}

\begin{figure*}[t]
  \centering
  \small{
  \fcolorbox{white}{voc_1}{\rule{0pt}{6pt}\rule{6pt}{0pt}} Background~~
  \fcolorbox{white}{voc_2}{\rule{0pt}{6pt}\rule{6pt}{0pt}} Aeroplane~~
  \fcolorbox{white}{voc_3}{\rule{0pt}{6pt}\rule{6pt}{0pt}} Bicycle~~
  \fcolorbox{white}{voc_4}{\rule{0pt}{6pt}\rule{6pt}{0pt}} Bird~~
  \fcolorbox{white}{voc_5}{\rule{0pt}{6pt}\rule{6pt}{0pt}} Boat~~
  \fcolorbox{white}{voc_6}{\rule{0pt}{6pt}\rule{6pt}{0pt}} Bottle~~
  \fcolorbox{white}{voc_7}{\rule{0pt}{6pt}\rule{6pt}{0pt}} Bus~~
  \fcolorbox{white}{voc_8}{\rule{0pt}{6pt}\rule{6pt}{0pt}} Car~~ \\
  \fcolorbox{white}{voc_9}{\rule{0pt}{6pt}\rule{6pt}{0pt}} Cat~~
  \fcolorbox{white}{voc_10}{\rule{0pt}{6pt}\rule{6pt}{0pt}} Chair~~
  \fcolorbox{white}{voc_11}{\rule{0pt}{6pt}\rule{6pt}{0pt}} Cow~~
  \fcolorbox{white}{voc_12}{\rule{0pt}{6pt}\rule{6pt}{0pt}} Dining Table~~
  \fcolorbox{white}{voc_13}{\rule{0pt}{6pt}\rule{6pt}{0pt}} Dog~~
  \fcolorbox{white}{voc_14}{\rule{0pt}{6pt}\rule{6pt}{0pt}} Horse~~
  \fcolorbox{white}{voc_15}{\rule{0pt}{6pt}\rule{6pt}{0pt}} Motorbike~~
  \fcolorbox{white}{voc_16}{\rule{0pt}{6pt}\rule{6pt}{0pt}} Person~~ \\
  \fcolorbox{white}{voc_17}{\rule{0pt}{6pt}\rule{6pt}{0pt}} Potted Plant~~
  \fcolorbox{white}{voc_18}{\rule{0pt}{6pt}\rule{6pt}{0pt}} Sheep~~
  \fcolorbox{white}{voc_19}{\rule{0pt}{6pt}\rule{6pt}{0pt}} Sofa~~
  \fcolorbox{white}{voc_20}{\rule{0pt}{6pt}\rule{6pt}{0pt}} Train~~
  \fcolorbox{white}{voc_21}{\rule{0pt}{6pt}\rule{6pt}{0pt}} TV monitor~~ \\
  }
  \subfigure{%
    \includegraphics[width=.18\columnwidth]{figures/supplementary/2007_001423_given.jpg}
  }
  \subfigure{%
    \includegraphics[width=.18\columnwidth]{figures/supplementary/2007_001423_gt.png}
  }
  \subfigure{%
    \includegraphics[width=.18\columnwidth]{figures/supplementary/2007_001423_cnn.png}
  }
  \subfigure{%
    \includegraphics[width=.18\columnwidth]{figures/supplementary/2007_001423_gauss.png}
  }
  \subfigure{%
    \includegraphics[width=.18\columnwidth]{figures/supplementary/2007_001423_learnt.png}
  }\\
  \subfigure{%
    \includegraphics[width=.18\columnwidth]{figures/supplementary/2007_001430_given.jpg}
  }
  \subfigure{%
    \includegraphics[width=.18\columnwidth]{figures/supplementary/2007_001430_gt.png}
  }
  \subfigure{%
    \includegraphics[width=.18\columnwidth]{figures/supplementary/2007_001430_cnn.png}
  }
  \subfigure{%
    \includegraphics[width=.18\columnwidth]{figures/supplementary/2007_001430_gauss.png}
  }
  \subfigure{%
    \includegraphics[width=.18\columnwidth]{figures/supplementary/2007_001430_learnt.png}
  }\\
    \subfigure{%
    \includegraphics[width=.18\columnwidth]{figures/supplementary/2007_007996_given.jpg}
  }
  \subfigure{%
    \includegraphics[width=.18\columnwidth]{figures/supplementary/2007_007996_gt.png}
  }
  \subfigure{%
    \includegraphics[width=.18\columnwidth]{figures/supplementary/2007_007996_cnn.png}
  }
  \subfigure{%
    \includegraphics[width=.18\columnwidth]{figures/supplementary/2007_007996_gauss.png}
  }
  \subfigure{%
    \includegraphics[width=.18\columnwidth]{figures/supplementary/2007_007996_learnt.png}
  }\\
   \subfigure{%
    \includegraphics[width=.18\columnwidth]{figures/supplementary/2010_002682_given.jpg}
  }
  \subfigure{%
    \includegraphics[width=.18\columnwidth]{figures/supplementary/2010_002682_gt.png}
  }
  \subfigure{%
    \includegraphics[width=.18\columnwidth]{figures/supplementary/2010_002682_cnn.png}
  }
  \subfigure{%
    \includegraphics[width=.18\columnwidth]{figures/supplementary/2010_002682_gauss.png}
  }
  \subfigure{%
    \includegraphics[width=.18\columnwidth]{figures/supplementary/2010_002682_learnt.png}
  }\\
     \subfigure{%
    \includegraphics[width=.18\columnwidth]{figures/supplementary/2010_004789_given.jpg}
  }
  \subfigure{%
    \includegraphics[width=.18\columnwidth]{figures/supplementary/2010_004789_gt.png}
  }
  \subfigure{%
    \includegraphics[width=.18\columnwidth]{figures/supplementary/2010_004789_cnn.png}
  }
  \subfigure{%
    \includegraphics[width=.18\columnwidth]{figures/supplementary/2010_004789_gauss.png}
  }
  \subfigure{%
    \includegraphics[width=.18\columnwidth]{figures/supplementary/2010_004789_learnt.png}
  }\\
       \subfigure{%
    \includegraphics[width=.18\columnwidth]{figures/supplementary/2007_001311_given.jpg}
  }
  \subfigure{%
    \includegraphics[width=.18\columnwidth]{figures/supplementary/2007_001311_gt.png}
  }
  \subfigure{%
    \includegraphics[width=.18\columnwidth]{figures/supplementary/2007_001311_cnn.png}
  }
  \subfigure{%
    \includegraphics[width=.18\columnwidth]{figures/supplementary/2007_001311_gauss.png}
  }
  \subfigure{%
    \includegraphics[width=.18\columnwidth]{figures/supplementary/2007_001311_learnt.png}
  }\\
  \setcounter{subfigure}{0}
  \subfigure[Input]{%
    \includegraphics[width=.18\columnwidth]{figures/supplementary/2010_003531_given.jpg}
  }
  \subfigure[Ground Truth]{%
    \includegraphics[width=.18\columnwidth]{figures/supplementary/2010_003531_gt.png}
  }
  \subfigure[DeepLab]{%
    \includegraphics[width=.18\columnwidth]{figures/supplementary/2010_003531_cnn.png}
  }
  \subfigure[+GaussCRF]{%
    \includegraphics[width=.18\columnwidth]{figures/supplementary/2010_003531_gauss.png}
  }
  \subfigure[+LearnedCRF]{%
    \includegraphics[width=.18\columnwidth]{figures/supplementary/2010_003531_learnt.png}
  }
  \vspace{-0.3cm}
  \mycaption{Semantic Segmentation}{Example results of semantic segmentation.
  (c)~depicts the unary results before application of MF, (d)~after two steps of MF with Gaussian edge CRF potentials, (e)~after
  two steps of MF with learned edge CRF potentials.}
    \label{fig:semantic_visuals}
\end{figure*}


\definecolor{minc_1}{HTML}{771111}
\definecolor{minc_2}{HTML}{CAC690}
\definecolor{minc_3}{HTML}{EEEEEE}
\definecolor{minc_4}{HTML}{7C8FA6}
\definecolor{minc_5}{HTML}{597D31}
\definecolor{minc_6}{HTML}{104410}
\definecolor{minc_7}{HTML}{BB819C}
\definecolor{minc_8}{HTML}{D0CE48}
\definecolor{minc_9}{HTML}{622745}
\definecolor{minc_10}{HTML}{666666}
\definecolor{minc_11}{HTML}{D54A31}
\definecolor{minc_12}{HTML}{101044}
\definecolor{minc_13}{HTML}{444126}
\definecolor{minc_14}{HTML}{75D646}
\definecolor{minc_15}{HTML}{DD4348}
\definecolor{minc_16}{HTML}{5C8577}
\definecolor{minc_17}{HTML}{C78472}
\definecolor{minc_18}{HTML}{75D6D0}
\definecolor{minc_19}{HTML}{5B4586}
\definecolor{minc_20}{HTML}{C04393}
\definecolor{minc_21}{HTML}{D69948}
\definecolor{minc_22}{HTML}{7370D8}
\definecolor{minc_23}{HTML}{7A3622}
\definecolor{minc_24}{HTML}{000000}

\begin{figure*}[t]
  \centering
  \small{
  \fcolorbox{white}{minc_1}{\rule{0pt}{6pt}\rule{6pt}{0pt}} Brick~~
  \fcolorbox{white}{minc_2}{\rule{0pt}{6pt}\rule{6pt}{0pt}} Carpet~~
  \fcolorbox{white}{minc_3}{\rule{0pt}{6pt}\rule{6pt}{0pt}} Ceramic~~
  \fcolorbox{white}{minc_4}{\rule{0pt}{6pt}\rule{6pt}{0pt}} Fabric~~
  \fcolorbox{white}{minc_5}{\rule{0pt}{6pt}\rule{6pt}{0pt}} Foliage~~
  \fcolorbox{white}{minc_6}{\rule{0pt}{6pt}\rule{6pt}{0pt}} Food~~
  \fcolorbox{white}{minc_7}{\rule{0pt}{6pt}\rule{6pt}{0pt}} Glass~~
  \fcolorbox{white}{minc_8}{\rule{0pt}{6pt}\rule{6pt}{0pt}} Hair~~ \\
  \fcolorbox{white}{minc_9}{\rule{0pt}{6pt}\rule{6pt}{0pt}} Leather~~
  \fcolorbox{white}{minc_10}{\rule{0pt}{6pt}\rule{6pt}{0pt}} Metal~~
  \fcolorbox{white}{minc_11}{\rule{0pt}{6pt}\rule{6pt}{0pt}} Mirror~~
  \fcolorbox{white}{minc_12}{\rule{0pt}{6pt}\rule{6pt}{0pt}} Other~~
  \fcolorbox{white}{minc_13}{\rule{0pt}{6pt}\rule{6pt}{0pt}} Painted~~
  \fcolorbox{white}{minc_14}{\rule{0pt}{6pt}\rule{6pt}{0pt}} Paper~~
  \fcolorbox{white}{minc_15}{\rule{0pt}{6pt}\rule{6pt}{0pt}} Plastic~~\\
  \fcolorbox{white}{minc_16}{\rule{0pt}{6pt}\rule{6pt}{0pt}} Polished Stone~~
  \fcolorbox{white}{minc_17}{\rule{0pt}{6pt}\rule{6pt}{0pt}} Skin~~
  \fcolorbox{white}{minc_18}{\rule{0pt}{6pt}\rule{6pt}{0pt}} Sky~~
  \fcolorbox{white}{minc_19}{\rule{0pt}{6pt}\rule{6pt}{0pt}} Stone~~
  \fcolorbox{white}{minc_20}{\rule{0pt}{6pt}\rule{6pt}{0pt}} Tile~~
  \fcolorbox{white}{minc_21}{\rule{0pt}{6pt}\rule{6pt}{0pt}} Wallpaper~~
  \fcolorbox{white}{minc_22}{\rule{0pt}{6pt}\rule{6pt}{0pt}} Water~~
  \fcolorbox{white}{minc_23}{\rule{0pt}{6pt}\rule{6pt}{0pt}} Wood~~ \\
  }
  \subfigure{%
    \includegraphics[width=.18\columnwidth]{figures/supplementary/000010868_given.jpg}
  }
  \subfigure{%
    \includegraphics[width=.18\columnwidth]{figures/supplementary/000010868_gt.png}
  }
  \subfigure{%
    \includegraphics[width=.18\columnwidth]{figures/supplementary/000010868_cnn.png}
  }
  \subfigure{%
    \includegraphics[width=.18\columnwidth]{figures/supplementary/000010868_gauss.png}
  }
  \subfigure{%
    \includegraphics[width=.18\columnwidth]{figures/supplementary/000010868_learnt.png}
  }\\[-2ex]
  \subfigure{%
    \includegraphics[width=.18\columnwidth]{figures/supplementary/000006011_given.jpg}
  }
  \subfigure{%
    \includegraphics[width=.18\columnwidth]{figures/supplementary/000006011_gt.png}
  }
  \subfigure{%
    \includegraphics[width=.18\columnwidth]{figures/supplementary/000006011_cnn.png}
  }
  \subfigure{%
    \includegraphics[width=.18\columnwidth]{figures/supplementary/000006011_gauss.png}
  }
  \subfigure{%
    \includegraphics[width=.18\columnwidth]{figures/supplementary/000006011_learnt.png}
  }\\[-2ex]
    \subfigure{%
    \includegraphics[width=.18\columnwidth]{figures/supplementary/000008553_given.jpg}
  }
  \subfigure{%
    \includegraphics[width=.18\columnwidth]{figures/supplementary/000008553_gt.png}
  }
  \subfigure{%
    \includegraphics[width=.18\columnwidth]{figures/supplementary/000008553_cnn.png}
  }
  \subfigure{%
    \includegraphics[width=.18\columnwidth]{figures/supplementary/000008553_gauss.png}
  }
  \subfigure{%
    \includegraphics[width=.18\columnwidth]{figures/supplementary/000008553_learnt.png}
  }\\[-2ex]
   \subfigure{%
    \includegraphics[width=.18\columnwidth]{figures/supplementary/000009188_given.jpg}
  }
  \subfigure{%
    \includegraphics[width=.18\columnwidth]{figures/supplementary/000009188_gt.png}
  }
  \subfigure{%
    \includegraphics[width=.18\columnwidth]{figures/supplementary/000009188_cnn.png}
  }
  \subfigure{%
    \includegraphics[width=.18\columnwidth]{figures/supplementary/000009188_gauss.png}
  }
  \subfigure{%
    \includegraphics[width=.18\columnwidth]{figures/supplementary/000009188_learnt.png}
  }\\[-2ex]
  \setcounter{subfigure}{0}
  \subfigure[Input]{%
    \includegraphics[width=.18\columnwidth]{figures/supplementary/000023570_given.jpg}
  }
  \subfigure[Ground Truth]{%
    \includegraphics[width=.18\columnwidth]{figures/supplementary/000023570_gt.png}
  }
  \subfigure[DeepLab]{%
    \includegraphics[width=.18\columnwidth]{figures/supplementary/000023570_cnn.png}
  }
  \subfigure[+GaussCRF]{%
    \includegraphics[width=.18\columnwidth]{figures/supplementary/000023570_gauss.png}
  }
  \subfigure[+LearnedCRF]{%
    \includegraphics[width=.18\columnwidth]{figures/supplementary/000023570_learnt.png}
  }
  \mycaption{Material Segmentation}{Example results of material segmentation.
  (c)~depicts the unary results before application of MF, (d)~after two steps of MF with Gaussian edge CRF potentials, (e)~after two steps of MF with learned edge CRF potentials.}
    \label{fig:material_visuals-app2}
\end{figure*}


\begin{table*}[h]
\tiny
  \centering
    \begin{tabular}{L{2.3cm} L{2.25cm} C{1.5cm} C{0.7cm} C{0.6cm} C{0.7cm} C{0.7cm} C{0.7cm} C{1.6cm} C{0.6cm} C{0.6cm} C{0.6cm}}
      \toprule
& & & & & \multicolumn{3}{c}{\textbf{Data Statistics}} & \multicolumn{4}{c}{\textbf{Training Protocol}} \\

\textbf{Experiment} & \textbf{Feature Types} & \textbf{Feature Scales} & \textbf{Filter Size} & \textbf{Filter Nbr.} & \textbf{Train}  & \textbf{Val.} & \textbf{Test} & \textbf{Loss Type} & \textbf{LR} & \textbf{Batch} & \textbf{Epochs} \\
      \midrule
      \multicolumn{2}{c}{\textbf{Single Bilateral Filter Applications}} & & & & & & & & & \\
      \textbf{2$\times$ Color Upsampling} & Position$_{1}$, Intensity (3D) & 0.13, 0.17 & 65 & 2 & 10581 & 1449 & 1456 & MSE & 1e-06 & 200 & 94.5\\
      \textbf{4$\times$ Color Upsampling} & Position$_{1}$, Intensity (3D) & 0.06, 0.17 & 65 & 2 & 10581 & 1449 & 1456 & MSE & 1e-06 & 200 & 94.5\\
      \textbf{8$\times$ Color Upsampling} & Position$_{1}$, Intensity (3D) & 0.03, 0.17 & 65 & 2 & 10581 & 1449 & 1456 & MSE & 1e-06 & 200 & 94.5\\
      \textbf{16$\times$ Color Upsampling} & Position$_{1}$, Intensity (3D) & 0.02, 0.17 & 65 & 2 & 10581 & 1449 & 1456 & MSE & 1e-06 & 200 & 94.5\\
      \textbf{Depth Upsampling} & Position$_{1}$, Color (5D) & 0.05, 0.02 & 665 & 2 & 795 & 100 & 654 & MSE & 1e-07 & 50 & 251.6\\
      \textbf{Mesh Denoising} & Isomap (4D) & 46.00 & 63 & 2 & 1000 & 200 & 500 & MSE & 100 & 10 & 100.0 \\
      \midrule
      \multicolumn{2}{c}{\textbf{DenseCRF Applications}} & & & & & & & & &\\
      \multicolumn{2}{l}{\textbf{Semantic Segmentation}} & & & & & & & & &\\
      \textbf{- 1step MF} & Position$_{1}$, Color (5D); Position$_{1}$ (2D) & 0.01, 0.34; 0.34  & 665; 19  & 2; 2 & 10581 & 1449 & 1456 & Logistic & 0.1 & 5 & 1.4 \\
      \textbf{- 2step MF} & Position$_{1}$, Color (5D); Position$_{1}$ (2D) & 0.01, 0.34; 0.34 & 665; 19 & 2; 2 & 10581 & 1449 & 1456 & Logistic & 0.1 & 5 & 1.4 \\
      \textbf{- \textit{loose} 2step MF} & Position$_{1}$, Color (5D); Position$_{1}$ (2D) & 0.01, 0.34; 0.34 & 665; 19 & 2; 2 &10581 & 1449 & 1456 & Logistic & 0.1 & 5 & +1.9  \\ \\
      \multicolumn{2}{l}{\textbf{Material Segmentation}} & & & & & & & & &\\
      \textbf{- 1step MF} & Position$_{2}$, Lab-Color (5D) & 5.00, 0.05, 0.30  & 665 & 2 & 928 & 150 & 1798 & Weighted Logistic & 1e-04 & 24 & 2.6 \\
      \textbf{- 2step MF} & Position$_{2}$, Lab-Color (5D) & 5.00, 0.05, 0.30 & 665 & 2 & 928 & 150 & 1798 & Weighted Logistic & 1e-04 & 12 & +0.7 \\
      \textbf{- \textit{loose} 2step MF} & Position$_{2}$, Lab-Color (5D) & 5.00, 0.05, 0.30 & 665 & 2 & 928 & 150 & 1798 & Weighted Logistic & 1e-04 & 12 & +0.2\\
      \midrule
      \multicolumn{2}{c}{\textbf{Neural Network Applications}} & & & & & & & & &\\
      \textbf{Tiles: CNN-9$\times$9} & - & - & 81 & 4 & 10000 & 1000 & 1000 & Logistic & 0.01 & 100 & 500.0 \\
      \textbf{Tiles: CNN-13$\times$13} & - & - & 169 & 6 & 10000 & 1000 & 1000 & Logistic & 0.01 & 100 & 500.0 \\
      \textbf{Tiles: CNN-17$\times$17} & - & - & 289 & 8 & 10000 & 1000 & 1000 & Logistic & 0.01 & 100 & 500.0 \\
      \textbf{Tiles: CNN-21$\times$21} & - & - & 441 & 10 & 10000 & 1000 & 1000 & Logistic & 0.01 & 100 & 500.0 \\
      \textbf{Tiles: BNN} & Position$_{1}$, Color (5D) & 0.05, 0.04 & 63 & 1 & 10000 & 1000 & 1000 & Logistic & 0.01 & 100 & 30.0 \\
      \textbf{LeNet} & - & - & 25 & 2 & 5490 & 1098 & 1647 & Logistic & 0.1 & 100 & 182.2 \\
      \textbf{Crop-LeNet} & - & - & 25 & 2 & 5490 & 1098 & 1647 & Logistic & 0.1 & 100 & 182.2 \\
      \textbf{BNN-LeNet} & Position$_{2}$ (2D) & 20.00 & 7 & 1 & 5490 & 1098 & 1647 & Logistic & 0.1 & 100 & 182.2 \\
      \textbf{DeepCNet} & - & - & 9 & 1 & 5490 & 1098 & 1647 & Logistic & 0.1 & 100 & 182.2 \\
      \textbf{Crop-DeepCNet} & - & - & 9 & 1 & 5490 & 1098 & 1647 & Logistic & 0.1 & 100 & 182.2 \\
      \textbf{BNN-DeepCNet} & Position$_{2}$ (2D) & 40.00  & 7 & 1 & 5490 & 1098 & 1647 & Logistic & 0.1 & 100 & 182.2 \\
      \bottomrule
      \\
    \end{tabular}
    \mycaption{Experiment Protocols} {Experiment protocols for the different experiments presented in this work. \textbf{Feature Types}:
    Feature spaces used for the bilateral convolutions. Position$_1$ corresponds to un-normalized pixel positions whereas Position$_2$ corresponds
    to pixel positions normalized to $[0,1]$ with respect to the given image. \textbf{Feature Scales}: Cross-validated scales for the features used.
     \textbf{Filter Size}: Number of elements in the filter that is being learned. \textbf{Filter Nbr.}: Half-width of the filter. \textbf{Train},
     \textbf{Val.} and \textbf{Test} corresponds to the number of train, validation and test images used in the experiment. \textbf{Loss Type}: Type
     of loss used for back-propagation. ``MSE'' corresponds to Euclidean mean squared error loss and ``Logistic'' corresponds to multinomial logistic
     loss. ``Weighted Logistic'' is the class-weighted multinomial logistic loss. We weighted the loss with inverse class probability for material
     segmentation task due to the small availability of training data with class imbalance. \textbf{LR}: Fixed learning rate used in stochastic gradient
     descent. \textbf{Batch}: Number of images used in one parameter update step. \textbf{Epochs}: Number of training epochs. In all the experiments,
     we used fixed momentum of 0.9 and weight decay of 0.0005 for stochastic gradient descent. ```Color Upsampling'' experiments in this Table corresponds
     to those performed on Pascal VOC12 dataset images. For all experiments using Pascal VOC12 images, we use extended
     training segmentation dataset available from~\cite{hariharan2011moredata}, and used standard validation and test splits
     from the main dataset~\cite{voc2012segmentation}.}
  \label{tbl:parameters}
\end{table*}

\clearpage

\section{Parameters and Additional Results for Video Propagation Networks}

In this Section, we present experiment protocols and additional qualitative results for experiments
on video object segmentation, semantic video segmentation and video color
propagation. Table~\ref{tbl:parameters_supp} shows the feature scales and other parameters used in different experiments.
Figures~\ref{fig:video_seg_pos_supp} show some qualitative results on video object segmentation
with some failure cases in Fig.~\ref{fig:video_seg_neg_supp}.
Figure~\ref{fig:semantic_visuals_supp} shows some qualitative results on semantic video segmentation and
Fig.~\ref{fig:color_visuals_supp} shows results on video color propagation.

\newcolumntype{L}[1]{>{\raggedright\let\newline\\\arraybackslash\hspace{0pt}}b{#1}}
\newcolumntype{C}[1]{>{\centering\let\newline\\\arraybackslash\hspace{0pt}}b{#1}}
\newcolumntype{R}[1]{>{\raggedleft\let\newline\\\arraybackslash\hspace{0pt}}b{#1}}

\begin{table*}[h]
\tiny
  \centering
    \begin{tabular}{L{3.0cm} L{2.4cm} L{2.8cm} L{2.8cm} C{0.5cm} C{1.0cm} L{1.2cm}}
      \toprule
\textbf{Experiment} & \textbf{Feature Type} & \textbf{Feature Scale-1, $\Lambda_a$} & \textbf{Feature Scale-2, $\Lambda_b$} & \textbf{$\alpha$} & \textbf{Input Frames} & \textbf{Loss Type} \\
      \midrule
      \textbf{Video Object Segmentation} & ($x,y,Y,Cb,Cr,t$) & (0.02,0.02,0.07,0.4,0.4,0.01) & (0.03,0.03,0.09,0.5,0.5,0.2) & 0.5 & 9 & Logistic\\
      \midrule
      \textbf{Semantic Video Segmentation} & & & & & \\
      \textbf{with CNN1~\cite{yu2015multi}-NoFlow} & ($x,y,R,G,B,t$) & (0.08,0.08,0.2,0.2,0.2,0.04) & (0.11,0.11,0.2,0.2,0.2,0.04) & 0.5 & 3 & Logistic \\
      \textbf{with CNN1~\cite{yu2015multi}-Flow} & ($x+u_x,y+u_y,R,G,B,t$) & (0.11,0.11,0.14,0.14,0.14,0.03) & (0.08,0.08,0.12,0.12,0.12,0.01) & 0.65 & 3 & Logistic\\
      \textbf{with CNN2~\cite{richter2016playing}-Flow} & ($x+u_x,y+u_y,R,G,B,t$) & (0.08,0.08,0.2,0.2,0.2,0.04) & (0.09,0.09,0.25,0.25,0.25,0.03) & 0.5 & 4 & Logistic\\
      \midrule
      \textbf{Video Color Propagation} & ($x,y,I,t$)  & (0.04,0.04,0.2,0.04) & No second kernel & 1 & 4 & MSE\\
      \bottomrule
      \\
    \end{tabular}
    \mycaption{Experiment Protocols} {Experiment protocols for the different experiments presented in this work. \textbf{Feature Types}:
    Feature spaces used for the bilateral convolutions, with position ($x,y$) and color
    ($R,G,B$ or $Y,Cb,Cr$) features $\in [0,255]$. $u_x$, $u_y$ denotes optical flow with respect
    to the present frame and $I$ denotes grayscale intensity.
    \textbf{Feature Scales ($\Lambda_a, \Lambda_b$)}: Cross-validated scales for the features used.
    \textbf{$\alpha$}: Exponential time decay for the input frames.
    \textbf{Input Frames}: Number of input frames for VPN.
    \textbf{Loss Type}: Type
     of loss used for back-propagation. ``MSE'' corresponds to Euclidean mean squared error loss and ``Logistic'' corresponds to multinomial logistic loss.}
  \label{tbl:parameters_supp}
\end{table*}

% \begin{figure}[th!]
% \begin{center}
%   \centerline{\includegraphics[width=\textwidth]{figures/video_seg_visuals_supp_small.pdf}}
%     \mycaption{Video Object Segmentation}
%     {Shown are the different frames in example videos with the corresponding
%     ground truth (GT) masks, predictions from BVS~\cite{marki2016bilateral},
%     OFL~\cite{tsaivideo}, VPN (VPN-Stage2) and VPN-DLab (VPN-DeepLab) models.}
%     \label{fig:video_seg_small_supp}
% \end{center}
% \vspace{-1.0cm}
% \end{figure}

\begin{figure}[th!]
\begin{center}
  \centerline{\includegraphics[width=0.7\textwidth]{figures/video_seg_visuals_supp_positive.pdf}}
    \mycaption{Video Object Segmentation}
    {Shown are the different frames in example videos with the corresponding
    ground truth (GT) masks, predictions from BVS~\cite{marki2016bilateral},
    OFL~\cite{tsaivideo}, VPN (VPN-Stage2) and VPN-DLab (VPN-DeepLab) models.}
    \label{fig:video_seg_pos_supp}
\end{center}
\vspace{-1.0cm}
\end{figure}

\begin{figure}[th!]
\begin{center}
  \centerline{\includegraphics[width=0.7\textwidth]{figures/video_seg_visuals_supp_negative.pdf}}
    \mycaption{Failure Cases for Video Object Segmentation}
    {Shown are the different frames in example videos with the corresponding
    ground truth (GT) masks, predictions from BVS~\cite{marki2016bilateral},
    OFL~\cite{tsaivideo}, VPN (VPN-Stage2) and VPN-DLab (VPN-DeepLab) models.}
    \label{fig:video_seg_neg_supp}
\end{center}
\vspace{-1.0cm}
\end{figure}

\begin{figure}[th!]
\begin{center}
  \centerline{\includegraphics[width=0.9\textwidth]{figures/supp_semantic_visual.pdf}}
    \mycaption{Semantic Video Segmentation}
    {Input video frames and the corresponding ground truth (GT)
    segmentation together with the predictions of CNN~\cite{yu2015multi} and with
    VPN-Flow.}
    \label{fig:semantic_visuals_supp}
\end{center}
\vspace{-0.7cm}
\end{figure}

\begin{figure}[th!]
\begin{center}
  \centerline{\includegraphics[width=\textwidth]{figures/colorization_visuals_supp.pdf}}
  \mycaption{Video Color Propagation}
  {Input grayscale video frames and corresponding ground-truth (GT) color images
  together with color predictions of Levin et al.~\cite{levin2004colorization} and VPN-Stage1 models.}
  \label{fig:color_visuals_supp}
\end{center}
\vspace{-0.7cm}
\end{figure}

\clearpage

\section{Additional Material for Bilateral Inception Networks}
\label{sec:binception-app}

In this section of the Appendix, we first discuss the use of approximate bilateral
filtering in BI modules (Sec.~\ref{sec:lattice}).
Later, we present some qualitative results using different models for the approach presented in
Chapter~\ref{chap:binception} (Sec.~\ref{sec:qualitative-app}).

\subsection{Approximate Bilateral Filtering}
\label{sec:lattice}

The bilateral inception module presented in Chapter~\ref{chap:binception} computes a matrix-vector
product between a Gaussian filter $K$ and a vector of activations $\bz_c$.
Bilateral filtering is an important operation and many algorithmic techniques have been
proposed to speed-up this operation~\cite{paris2006fast,adams2010fast,gastal2011domain}.
In the main paper we opted to implement what can be considered the
brute-force variant of explicitly constructing $K$ and then using BLAS to compute the
matrix-vector product. This resulted in a few millisecond operation.
The explicit way to compute is possible due to the
reduction to super-pixels, e.g., it would not work for DenseCRF variants
that operate on the full image resolution.

Here, we present experiments where we use the fast approximate bilateral filtering
algorithm of~\cite{adams2010fast}, which is also used in Chapter~\ref{chap:bnn}
for learning sparse high dimensional filters. This
choice allows for larger dimensions of matrix-vector multiplication. The reason for choosing
the explicit multiplication in Chapter~\ref{chap:binception} was that it was computationally faster.
For the small sizes of the involved matrices and vectors, the explicit computation is sufficient and we had no
GPU implementation of an approximate technique that matched this runtime. Also it
is conceptually easier and the gradient to the feature transformations ($\Lambda \mathbf{f}$) is
obtained using standard matrix calculus.

\subsubsection{Experiments}

We modified the existing segmentation architectures analogous to those in Chapter~\ref{chap:binception}.
The main difference is that, here, the inception modules use the lattice
approximation~\cite{adams2010fast} to compute the bilateral filtering.
Using the lattice approximation did not allow us to back-propagate through feature transformations ($\Lambda$)
and thus we used hand-specified feature scales as will be explained later.
Specifically, we take CNN architectures from the works
of~\cite{chen2014semantic,zheng2015conditional,bell2015minc} and insert the BI modules between
the spatial FC layers.
We use superpixels from~\cite{DollarICCV13edges}
for all the experiments with the lattice approximation. Experiments are
performed using Caffe neural network framework~\cite{jia2014caffe}.

\begin{table}
  \small
  \centering
  \begin{tabular}{p{5.5cm}>{\raggedright\arraybackslash}p{1.4cm}>{\centering\arraybackslash}p{2.2cm}}
    \toprule
		\textbf{Model} & \emph{IoU} & \emph{Runtime}(ms) \\
    \midrule

    %%%%%%%%%%%% Scores computed by us)%%%%%%%%%%%%
		\deeplablargefov & 68.9 & 145ms\\
    \midrule
    \bi{7}{2}-\bi{8}{10}& \textbf{73.8} & +600 \\
    \midrule
    \deeplablargefovcrf~\cite{chen2014semantic} & 72.7 & +830\\
    \deeplabmsclargefovcrf~\cite{chen2014semantic} & \textbf{73.6} & +880\\
    DeepLab-EdgeNet~\cite{chen2015semantic} & 71.7 & +30\\
    DeepLab-EdgeNet-CRF~\cite{chen2015semantic} & \textbf{73.6} & +860\\
  \bottomrule \\
  \end{tabular}
  \mycaption{Semantic Segmentation using the DeepLab model}
  {IoU scores on the Pascal VOC12 segmentation test dataset
  with different models and our modified inception model.
  Also shown are the corresponding runtimes in milliseconds. Runtimes
  also include superpixel computations (300 ms with Dollar superpixels~\cite{DollarICCV13edges})}
  \label{tab:largefovresults}
\end{table}

\paragraph{Semantic Segmentation}
The experiments in this section use the Pascal VOC12 segmentation dataset~\cite{voc2012segmentation} with 21 object classes and the images have a maximum resolution of 0.25 megapixels.
For all experiments on VOC12, we train using the extended training set of
10581 images collected by~\cite{hariharan2011moredata}.
We modified the \deeplab~network architecture of~\cite{chen2014semantic} and
the CRFasRNN architecture from~\cite{zheng2015conditional} which uses a CNN with
deconvolution layers followed by DenseCRF trained end-to-end.

\paragraph{DeepLab Model}\label{sec:deeplabmodel}
We experimented with the \bi{7}{2}-\bi{8}{10} inception model.
Results using the~\deeplab~model are summarized in Tab.~\ref{tab:largefovresults}.
Although we get similar improvements with inception modules as with the
explicit kernel computation, using lattice approximation is slower.

\begin{table}
  \small
  \centering
  \begin{tabular}{p{6.4cm}>{\raggedright\arraybackslash}p{1.8cm}>{\raggedright\arraybackslash}p{1.8cm}}
    \toprule
    \textbf{Model} & \emph{IoU (Val)} & \emph{IoU (Test)}\\
    \midrule
    %%%%%%%%%%%% Scores computed by us)%%%%%%%%%%%%
    CNN &  67.5 & - \\
    \deconv (CNN+Deconvolutions) & 69.8 & 72.0 \\
    \midrule
    \bi{3}{6}-\bi{4}{6}-\bi{7}{2}-\bi{8}{6}& 71.9 & - \\
    \bi{3}{6}-\bi{4}{6}-\bi{7}{2}-\bi{8}{6}-\gi{6}& 73.6 &  \href{http://host.robots.ox.ac.uk:8080/anonymous/VOTV5E.html}{\textbf{75.2}}\\
    \midrule
    \deconvcrf (CRF-RNN)~\cite{zheng2015conditional} & 73.0 & 74.7\\
    Context-CRF-RNN~\cite{yu2015multi} & ~~ - ~ & \textbf{75.3} \\
    \bottomrule \\
  \end{tabular}
  \mycaption{Semantic Segmentation using the CRFasRNN model}{IoU score corresponding to different models
  on Pascal VOC12 reduced validation / test segmentation dataset. The reduced validation set consists of 346 images
  as used in~\cite{zheng2015conditional} where we adapted the model from.}
  \label{tab:deconvresults-app}
\end{table}

\paragraph{CRFasRNN Model}\label{sec:deepinception}
We add BI modules after score-pool3, score-pool4, \fc{7} and \fc{8} $1\times1$ convolution layers
resulting in the \bi{3}{6}-\bi{4}{6}-\bi{7}{2}-\bi{8}{6}
model and also experimented with another variant where $BI_8$ is followed by another inception
module, G$(6)$, with 6 Gaussian kernels.
Note that here also we discarded both deconvolution and DenseCRF parts of the original model~\cite{zheng2015conditional}
and inserted the BI modules in the base CNN and found similar improvements compared to the inception modules with explicit
kernel computaion. See Tab.~\ref{tab:deconvresults-app} for results on the CRFasRNN model.

\paragraph{Material Segmentation}
Table~\ref{tab:mincresults-app} shows the results on the MINC dataset~\cite{bell2015minc}
obtained by modifying the AlexNet architecture with our inception modules. We observe
similar improvements as with explicit kernel construction.
For this model, we do not provide any learned setup due to very limited segment training
data. The weights to combine outputs in the bilateral inception layer are
found by validation on the validation set.

\begin{table}[t]
  \small
  \centering
  \begin{tabular}{p{3.5cm}>{\centering\arraybackslash}p{4.0cm}}
    \toprule
    \textbf{Model} & Class / Total accuracy\\
    \midrule

    %%%%%%%%%%%% Scores computed by us)%%%%%%%%%%%%
    AlexNet CNN & 55.3 / 58.9 \\
    \midrule
    \bi{7}{2}-\bi{8}{6}& 68.5 / 71.8 \\
    \bi{7}{2}-\bi{8}{6}-G$(6)$& 67.6 / 73.1 \\
    \midrule
    AlexNet-CRF & 65.5 / 71.0 \\
    \bottomrule \\
  \end{tabular}
  \mycaption{Material Segmentation using AlexNet}{Pixel accuracy of different models on
  the MINC material segmentation test dataset~\cite{bell2015minc}.}
  \label{tab:mincresults-app}
\end{table}

\paragraph{Scales of Bilateral Inception Modules}
\label{sec:scales}

Unlike the explicit kernel technique presented in the main text (Chapter~\ref{chap:binception}),
we didn't back-propagate through feature transformation ($\Lambda$)
using the approximate bilateral filter technique.
So, the feature scales are hand-specified and validated, which are as follows.
The optimal scale values for the \bi{7}{2}-\bi{8}{2} model are found by validation for the best performance which are
$\sigma_{xy}$ = (0.1, 0.1) for the spatial (XY) kernel and $\sigma_{rgbxy}$ = (0.1, 0.1, 0.1, 0.01, 0.01) for color and position (RGBXY)  kernel.
Next, as more kernels are added to \bi{8}{2}, we set scales to be $\alpha$*($\sigma_{xy}$, $\sigma_{rgbxy}$).
The value of $\alpha$ is chosen as  1, 0.5, 0.1, 0.05, 0.1, at uniform interval, for the \bi{8}{10} bilateral inception module.


\subsection{Qualitative Results}
\label{sec:qualitative-app}

In this section, we present more qualitative results obtained using the BI module with explicit
kernel computation technique presented in Chapter~\ref{chap:binception}. Results on the Pascal VOC12
dataset~\cite{voc2012segmentation} using the DeepLab-LargeFOV model are shown in Fig.~\ref{fig:semantic_visuals-app},
followed by the results on MINC dataset~\cite{bell2015minc}
in Fig.~\ref{fig:material_visuals-app} and on
Cityscapes dataset~\cite{Cordts2015Cvprw} in Fig.~\ref{fig:street_visuals-app}.


\definecolor{voc_1}{RGB}{0, 0, 0}
\definecolor{voc_2}{RGB}{128, 0, 0}
\definecolor{voc_3}{RGB}{0, 128, 0}
\definecolor{voc_4}{RGB}{128, 128, 0}
\definecolor{voc_5}{RGB}{0, 0, 128}
\definecolor{voc_6}{RGB}{128, 0, 128}
\definecolor{voc_7}{RGB}{0, 128, 128}
\definecolor{voc_8}{RGB}{128, 128, 128}
\definecolor{voc_9}{RGB}{64, 0, 0}
\definecolor{voc_10}{RGB}{192, 0, 0}
\definecolor{voc_11}{RGB}{64, 128, 0}
\definecolor{voc_12}{RGB}{192, 128, 0}
\definecolor{voc_13}{RGB}{64, 0, 128}
\definecolor{voc_14}{RGB}{192, 0, 128}
\definecolor{voc_15}{RGB}{64, 128, 128}
\definecolor{voc_16}{RGB}{192, 128, 128}
\definecolor{voc_17}{RGB}{0, 64, 0}
\definecolor{voc_18}{RGB}{128, 64, 0}
\definecolor{voc_19}{RGB}{0, 192, 0}
\definecolor{voc_20}{RGB}{128, 192, 0}
\definecolor{voc_21}{RGB}{0, 64, 128}
\definecolor{voc_22}{RGB}{128, 64, 128}

\begin{figure*}[!ht]
  \small
  \centering
  \fcolorbox{white}{voc_1}{\rule{0pt}{4pt}\rule{4pt}{0pt}} Background~~
  \fcolorbox{white}{voc_2}{\rule{0pt}{4pt}\rule{4pt}{0pt}} Aeroplane~~
  \fcolorbox{white}{voc_3}{\rule{0pt}{4pt}\rule{4pt}{0pt}} Bicycle~~
  \fcolorbox{white}{voc_4}{\rule{0pt}{4pt}\rule{4pt}{0pt}} Bird~~
  \fcolorbox{white}{voc_5}{\rule{0pt}{4pt}\rule{4pt}{0pt}} Boat~~
  \fcolorbox{white}{voc_6}{\rule{0pt}{4pt}\rule{4pt}{0pt}} Bottle~~
  \fcolorbox{white}{voc_7}{\rule{0pt}{4pt}\rule{4pt}{0pt}} Bus~~
  \fcolorbox{white}{voc_8}{\rule{0pt}{4pt}\rule{4pt}{0pt}} Car~~\\
  \fcolorbox{white}{voc_9}{\rule{0pt}{4pt}\rule{4pt}{0pt}} Cat~~
  \fcolorbox{white}{voc_10}{\rule{0pt}{4pt}\rule{4pt}{0pt}} Chair~~
  \fcolorbox{white}{voc_11}{\rule{0pt}{4pt}\rule{4pt}{0pt}} Cow~~
  \fcolorbox{white}{voc_12}{\rule{0pt}{4pt}\rule{4pt}{0pt}} Dining Table~~
  \fcolorbox{white}{voc_13}{\rule{0pt}{4pt}\rule{4pt}{0pt}} Dog~~
  \fcolorbox{white}{voc_14}{\rule{0pt}{4pt}\rule{4pt}{0pt}} Horse~~
  \fcolorbox{white}{voc_15}{\rule{0pt}{4pt}\rule{4pt}{0pt}} Motorbike~~
  \fcolorbox{white}{voc_16}{\rule{0pt}{4pt}\rule{4pt}{0pt}} Person~~\\
  \fcolorbox{white}{voc_17}{\rule{0pt}{4pt}\rule{4pt}{0pt}} Potted Plant~~
  \fcolorbox{white}{voc_18}{\rule{0pt}{4pt}\rule{4pt}{0pt}} Sheep~~
  \fcolorbox{white}{voc_19}{\rule{0pt}{4pt}\rule{4pt}{0pt}} Sofa~~
  \fcolorbox{white}{voc_20}{\rule{0pt}{4pt}\rule{4pt}{0pt}} Train~~
  \fcolorbox{white}{voc_21}{\rule{0pt}{4pt}\rule{4pt}{0pt}} TV monitor~~\\


  \subfigure{%
    \includegraphics[width=.15\columnwidth]{figures/supplementary/2008_001308_given.png}
  }
  \subfigure{%
    \includegraphics[width=.15\columnwidth]{figures/supplementary/2008_001308_sp.png}
  }
  \subfigure{%
    \includegraphics[width=.15\columnwidth]{figures/supplementary/2008_001308_gt.png}
  }
  \subfigure{%
    \includegraphics[width=.15\columnwidth]{figures/supplementary/2008_001308_cnn.png}
  }
  \subfigure{%
    \includegraphics[width=.15\columnwidth]{figures/supplementary/2008_001308_crf.png}
  }
  \subfigure{%
    \includegraphics[width=.15\columnwidth]{figures/supplementary/2008_001308_ours.png}
  }\\[-2ex]


  \subfigure{%
    \includegraphics[width=.15\columnwidth]{figures/supplementary/2008_001821_given.png}
  }
  \subfigure{%
    \includegraphics[width=.15\columnwidth]{figures/supplementary/2008_001821_sp.png}
  }
  \subfigure{%
    \includegraphics[width=.15\columnwidth]{figures/supplementary/2008_001821_gt.png}
  }
  \subfigure{%
    \includegraphics[width=.15\columnwidth]{figures/supplementary/2008_001821_cnn.png}
  }
  \subfigure{%
    \includegraphics[width=.15\columnwidth]{figures/supplementary/2008_001821_crf.png}
  }
  \subfigure{%
    \includegraphics[width=.15\columnwidth]{figures/supplementary/2008_001821_ours.png}
  }\\[-2ex]



  \subfigure{%
    \includegraphics[width=.15\columnwidth]{figures/supplementary/2008_004612_given.png}
  }
  \subfigure{%
    \includegraphics[width=.15\columnwidth]{figures/supplementary/2008_004612_sp.png}
  }
  \subfigure{%
    \includegraphics[width=.15\columnwidth]{figures/supplementary/2008_004612_gt.png}
  }
  \subfigure{%
    \includegraphics[width=.15\columnwidth]{figures/supplementary/2008_004612_cnn.png}
  }
  \subfigure{%
    \includegraphics[width=.15\columnwidth]{figures/supplementary/2008_004612_crf.png}
  }
  \subfigure{%
    \includegraphics[width=.15\columnwidth]{figures/supplementary/2008_004612_ours.png}
  }\\[-2ex]


  \subfigure{%
    \includegraphics[width=.15\columnwidth]{figures/supplementary/2009_001008_given.png}
  }
  \subfigure{%
    \includegraphics[width=.15\columnwidth]{figures/supplementary/2009_001008_sp.png}
  }
  \subfigure{%
    \includegraphics[width=.15\columnwidth]{figures/supplementary/2009_001008_gt.png}
  }
  \subfigure{%
    \includegraphics[width=.15\columnwidth]{figures/supplementary/2009_001008_cnn.png}
  }
  \subfigure{%
    \includegraphics[width=.15\columnwidth]{figures/supplementary/2009_001008_crf.png}
  }
  \subfigure{%
    \includegraphics[width=.15\columnwidth]{figures/supplementary/2009_001008_ours.png}
  }\\[-2ex]




  \subfigure{%
    \includegraphics[width=.15\columnwidth]{figures/supplementary/2009_004497_given.png}
  }
  \subfigure{%
    \includegraphics[width=.15\columnwidth]{figures/supplementary/2009_004497_sp.png}
  }
  \subfigure{%
    \includegraphics[width=.15\columnwidth]{figures/supplementary/2009_004497_gt.png}
  }
  \subfigure{%
    \includegraphics[width=.15\columnwidth]{figures/supplementary/2009_004497_cnn.png}
  }
  \subfigure{%
    \includegraphics[width=.15\columnwidth]{figures/supplementary/2009_004497_crf.png}
  }
  \subfigure{%
    \includegraphics[width=.15\columnwidth]{figures/supplementary/2009_004497_ours.png}
  }\\[-2ex]



  \setcounter{subfigure}{0}
  \subfigure[\scriptsize Input]{%
    \includegraphics[width=.15\columnwidth]{figures/supplementary/2010_001327_given.png}
  }
  \subfigure[\scriptsize Superpixels]{%
    \includegraphics[width=.15\columnwidth]{figures/supplementary/2010_001327_sp.png}
  }
  \subfigure[\scriptsize GT]{%
    \includegraphics[width=.15\columnwidth]{figures/supplementary/2010_001327_gt.png}
  }
  \subfigure[\scriptsize Deeplab]{%
    \includegraphics[width=.15\columnwidth]{figures/supplementary/2010_001327_cnn.png}
  }
  \subfigure[\scriptsize +DenseCRF]{%
    \includegraphics[width=.15\columnwidth]{figures/supplementary/2010_001327_crf.png}
  }
  \subfigure[\scriptsize Using BI]{%
    \includegraphics[width=.15\columnwidth]{figures/supplementary/2010_001327_ours.png}
  }
  \mycaption{Semantic Segmentation}{Example results of semantic segmentation
  on the Pascal VOC12 dataset.
  (d)~depicts the DeepLab CNN result, (e)~CNN + 10 steps of mean-field inference,
  (f~result obtained with bilateral inception (BI) modules (\bi{6}{2}+\bi{7}{6}) between \fc~layers.}
  \label{fig:semantic_visuals-app}
\end{figure*}


\definecolor{minc_1}{HTML}{771111}
\definecolor{minc_2}{HTML}{CAC690}
\definecolor{minc_3}{HTML}{EEEEEE}
\definecolor{minc_4}{HTML}{7C8FA6}
\definecolor{minc_5}{HTML}{597D31}
\definecolor{minc_6}{HTML}{104410}
\definecolor{minc_7}{HTML}{BB819C}
\definecolor{minc_8}{HTML}{D0CE48}
\definecolor{minc_9}{HTML}{622745}
\definecolor{minc_10}{HTML}{666666}
\definecolor{minc_11}{HTML}{D54A31}
\definecolor{minc_12}{HTML}{101044}
\definecolor{minc_13}{HTML}{444126}
\definecolor{minc_14}{HTML}{75D646}
\definecolor{minc_15}{HTML}{DD4348}
\definecolor{minc_16}{HTML}{5C8577}
\definecolor{minc_17}{HTML}{C78472}
\definecolor{minc_18}{HTML}{75D6D0}
\definecolor{minc_19}{HTML}{5B4586}
\definecolor{minc_20}{HTML}{C04393}
\definecolor{minc_21}{HTML}{D69948}
\definecolor{minc_22}{HTML}{7370D8}
\definecolor{minc_23}{HTML}{7A3622}
\definecolor{minc_24}{HTML}{000000}

\begin{figure*}[!ht]
  \small % scriptsize
  \centering
  \fcolorbox{white}{minc_1}{\rule{0pt}{4pt}\rule{4pt}{0pt}} Brick~~
  \fcolorbox{white}{minc_2}{\rule{0pt}{4pt}\rule{4pt}{0pt}} Carpet~~
  \fcolorbox{white}{minc_3}{\rule{0pt}{4pt}\rule{4pt}{0pt}} Ceramic~~
  \fcolorbox{white}{minc_4}{\rule{0pt}{4pt}\rule{4pt}{0pt}} Fabric~~
  \fcolorbox{white}{minc_5}{\rule{0pt}{4pt}\rule{4pt}{0pt}} Foliage~~
  \fcolorbox{white}{minc_6}{\rule{0pt}{4pt}\rule{4pt}{0pt}} Food~~
  \fcolorbox{white}{minc_7}{\rule{0pt}{4pt}\rule{4pt}{0pt}} Glass~~
  \fcolorbox{white}{minc_8}{\rule{0pt}{4pt}\rule{4pt}{0pt}} Hair~~\\
  \fcolorbox{white}{minc_9}{\rule{0pt}{4pt}\rule{4pt}{0pt}} Leather~~
  \fcolorbox{white}{minc_10}{\rule{0pt}{4pt}\rule{4pt}{0pt}} Metal~~
  \fcolorbox{white}{minc_11}{\rule{0pt}{4pt}\rule{4pt}{0pt}} Mirror~~
  \fcolorbox{white}{minc_12}{\rule{0pt}{4pt}\rule{4pt}{0pt}} Other~~
  \fcolorbox{white}{minc_13}{\rule{0pt}{4pt}\rule{4pt}{0pt}} Painted~~
  \fcolorbox{white}{minc_14}{\rule{0pt}{4pt}\rule{4pt}{0pt}} Paper~~
  \fcolorbox{white}{minc_15}{\rule{0pt}{4pt}\rule{4pt}{0pt}} Plastic~~\\
  \fcolorbox{white}{minc_16}{\rule{0pt}{4pt}\rule{4pt}{0pt}} Polished Stone~~
  \fcolorbox{white}{minc_17}{\rule{0pt}{4pt}\rule{4pt}{0pt}} Skin~~
  \fcolorbox{white}{minc_18}{\rule{0pt}{4pt}\rule{4pt}{0pt}} Sky~~
  \fcolorbox{white}{minc_19}{\rule{0pt}{4pt}\rule{4pt}{0pt}} Stone~~
  \fcolorbox{white}{minc_20}{\rule{0pt}{4pt}\rule{4pt}{0pt}} Tile~~
  \fcolorbox{white}{minc_21}{\rule{0pt}{4pt}\rule{4pt}{0pt}} Wallpaper~~
  \fcolorbox{white}{minc_22}{\rule{0pt}{4pt}\rule{4pt}{0pt}} Water~~
  \fcolorbox{white}{minc_23}{\rule{0pt}{4pt}\rule{4pt}{0pt}} Wood~~\\
  \subfigure{%
    \includegraphics[width=.15\columnwidth]{figures/supplementary/000008468_given.png}
  }
  \subfigure{%
    \includegraphics[width=.15\columnwidth]{figures/supplementary/000008468_sp.png}
  }
  \subfigure{%
    \includegraphics[width=.15\columnwidth]{figures/supplementary/000008468_gt.png}
  }
  \subfigure{%
    \includegraphics[width=.15\columnwidth]{figures/supplementary/000008468_cnn.png}
  }
  \subfigure{%
    \includegraphics[width=.15\columnwidth]{figures/supplementary/000008468_crf.png}
  }
  \subfigure{%
    \includegraphics[width=.15\columnwidth]{figures/supplementary/000008468_ours.png}
  }\\[-2ex]

  \subfigure{%
    \includegraphics[width=.15\columnwidth]{figures/supplementary/000009053_given.png}
  }
  \subfigure{%
    \includegraphics[width=.15\columnwidth]{figures/supplementary/000009053_sp.png}
  }
  \subfigure{%
    \includegraphics[width=.15\columnwidth]{figures/supplementary/000009053_gt.png}
  }
  \subfigure{%
    \includegraphics[width=.15\columnwidth]{figures/supplementary/000009053_cnn.png}
  }
  \subfigure{%
    \includegraphics[width=.15\columnwidth]{figures/supplementary/000009053_crf.png}
  }
  \subfigure{%
    \includegraphics[width=.15\columnwidth]{figures/supplementary/000009053_ours.png}
  }\\[-2ex]




  \subfigure{%
    \includegraphics[width=.15\columnwidth]{figures/supplementary/000014977_given.png}
  }
  \subfigure{%
    \includegraphics[width=.15\columnwidth]{figures/supplementary/000014977_sp.png}
  }
  \subfigure{%
    \includegraphics[width=.15\columnwidth]{figures/supplementary/000014977_gt.png}
  }
  \subfigure{%
    \includegraphics[width=.15\columnwidth]{figures/supplementary/000014977_cnn.png}
  }
  \subfigure{%
    \includegraphics[width=.15\columnwidth]{figures/supplementary/000014977_crf.png}
  }
  \subfigure{%
    \includegraphics[width=.15\columnwidth]{figures/supplementary/000014977_ours.png}
  }\\[-2ex]


  \subfigure{%
    \includegraphics[width=.15\columnwidth]{figures/supplementary/000022922_given.png}
  }
  \subfigure{%
    \includegraphics[width=.15\columnwidth]{figures/supplementary/000022922_sp.png}
  }
  \subfigure{%
    \includegraphics[width=.15\columnwidth]{figures/supplementary/000022922_gt.png}
  }
  \subfigure{%
    \includegraphics[width=.15\columnwidth]{figures/supplementary/000022922_cnn.png}
  }
  \subfigure{%
    \includegraphics[width=.15\columnwidth]{figures/supplementary/000022922_crf.png}
  }
  \subfigure{%
    \includegraphics[width=.15\columnwidth]{figures/supplementary/000022922_ours.png}
  }\\[-2ex]


  \subfigure{%
    \includegraphics[width=.15\columnwidth]{figures/supplementary/000025711_given.png}
  }
  \subfigure{%
    \includegraphics[width=.15\columnwidth]{figures/supplementary/000025711_sp.png}
  }
  \subfigure{%
    \includegraphics[width=.15\columnwidth]{figures/supplementary/000025711_gt.png}
  }
  \subfigure{%
    \includegraphics[width=.15\columnwidth]{figures/supplementary/000025711_cnn.png}
  }
  \subfigure{%
    \includegraphics[width=.15\columnwidth]{figures/supplementary/000025711_crf.png}
  }
  \subfigure{%
    \includegraphics[width=.15\columnwidth]{figures/supplementary/000025711_ours.png}
  }\\[-2ex]


  \subfigure{%
    \includegraphics[width=.15\columnwidth]{figures/supplementary/000034473_given.png}
  }
  \subfigure{%
    \includegraphics[width=.15\columnwidth]{figures/supplementary/000034473_sp.png}
  }
  \subfigure{%
    \includegraphics[width=.15\columnwidth]{figures/supplementary/000034473_gt.png}
  }
  \subfigure{%
    \includegraphics[width=.15\columnwidth]{figures/supplementary/000034473_cnn.png}
  }
  \subfigure{%
    \includegraphics[width=.15\columnwidth]{figures/supplementary/000034473_crf.png}
  }
  \subfigure{%
    \includegraphics[width=.15\columnwidth]{figures/supplementary/000034473_ours.png}
  }\\[-2ex]


  \subfigure{%
    \includegraphics[width=.15\columnwidth]{figures/supplementary/000035463_given.png}
  }
  \subfigure{%
    \includegraphics[width=.15\columnwidth]{figures/supplementary/000035463_sp.png}
  }
  \subfigure{%
    \includegraphics[width=.15\columnwidth]{figures/supplementary/000035463_gt.png}
  }
  \subfigure{%
    \includegraphics[width=.15\columnwidth]{figures/supplementary/000035463_cnn.png}
  }
  \subfigure{%
    \includegraphics[width=.15\columnwidth]{figures/supplementary/000035463_crf.png}
  }
  \subfigure{%
    \includegraphics[width=.15\columnwidth]{figures/supplementary/000035463_ours.png}
  }\\[-2ex]


  \setcounter{subfigure}{0}
  \subfigure[\scriptsize Input]{%
    \includegraphics[width=.15\columnwidth]{figures/supplementary/000035993_given.png}
  }
  \subfigure[\scriptsize Superpixels]{%
    \includegraphics[width=.15\columnwidth]{figures/supplementary/000035993_sp.png}
  }
  \subfigure[\scriptsize GT]{%
    \includegraphics[width=.15\columnwidth]{figures/supplementary/000035993_gt.png}
  }
  \subfigure[\scriptsize AlexNet]{%
    \includegraphics[width=.15\columnwidth]{figures/supplementary/000035993_cnn.png}
  }
  \subfigure[\scriptsize +DenseCRF]{%
    \includegraphics[width=.15\columnwidth]{figures/supplementary/000035993_crf.png}
  }
  \subfigure[\scriptsize Using BI]{%
    \includegraphics[width=.15\columnwidth]{figures/supplementary/000035993_ours.png}
  }
  \mycaption{Material Segmentation}{Example results of material segmentation.
  (d)~depicts the AlexNet CNN result, (e)~CNN + 10 steps of mean-field inference,
  (f)~result obtained with bilateral inception (BI) modules (\bi{7}{2}+\bi{8}{6}) between
  \fc~layers.}
\label{fig:material_visuals-app}
\end{figure*}


\definecolor{city_1}{RGB}{128, 64, 128}
\definecolor{city_2}{RGB}{244, 35, 232}
\definecolor{city_3}{RGB}{70, 70, 70}
\definecolor{city_4}{RGB}{102, 102, 156}
\definecolor{city_5}{RGB}{190, 153, 153}
\definecolor{city_6}{RGB}{153, 153, 153}
\definecolor{city_7}{RGB}{250, 170, 30}
\definecolor{city_8}{RGB}{220, 220, 0}
\definecolor{city_9}{RGB}{107, 142, 35}
\definecolor{city_10}{RGB}{152, 251, 152}
\definecolor{city_11}{RGB}{70, 130, 180}
\definecolor{city_12}{RGB}{220, 20, 60}
\definecolor{city_13}{RGB}{255, 0, 0}
\definecolor{city_14}{RGB}{0, 0, 142}
\definecolor{city_15}{RGB}{0, 0, 70}
\definecolor{city_16}{RGB}{0, 60, 100}
\definecolor{city_17}{RGB}{0, 80, 100}
\definecolor{city_18}{RGB}{0, 0, 230}
\definecolor{city_19}{RGB}{119, 11, 32}
\begin{figure*}[!ht]
  \small % scriptsize
  \centering


  \subfigure{%
    \includegraphics[width=.18\columnwidth]{figures/supplementary/frankfurt00000_016005_given.png}
  }
  \subfigure{%
    \includegraphics[width=.18\columnwidth]{figures/supplementary/frankfurt00000_016005_sp.png}
  }
  \subfigure{%
    \includegraphics[width=.18\columnwidth]{figures/supplementary/frankfurt00000_016005_gt.png}
  }
  \subfigure{%
    \includegraphics[width=.18\columnwidth]{figures/supplementary/frankfurt00000_016005_cnn.png}
  }
  \subfigure{%
    \includegraphics[width=.18\columnwidth]{figures/supplementary/frankfurt00000_016005_ours.png}
  }\\[-2ex]

  \subfigure{%
    \includegraphics[width=.18\columnwidth]{figures/supplementary/frankfurt00000_004617_given.png}
  }
  \subfigure{%
    \includegraphics[width=.18\columnwidth]{figures/supplementary/frankfurt00000_004617_sp.png}
  }
  \subfigure{%
    \includegraphics[width=.18\columnwidth]{figures/supplementary/frankfurt00000_004617_gt.png}
  }
  \subfigure{%
    \includegraphics[width=.18\columnwidth]{figures/supplementary/frankfurt00000_004617_cnn.png}
  }
  \subfigure{%
    \includegraphics[width=.18\columnwidth]{figures/supplementary/frankfurt00000_004617_ours.png}
  }\\[-2ex]

  \subfigure{%
    \includegraphics[width=.18\columnwidth]{figures/supplementary/frankfurt00000_020880_given.png}
  }
  \subfigure{%
    \includegraphics[width=.18\columnwidth]{figures/supplementary/frankfurt00000_020880_sp.png}
  }
  \subfigure{%
    \includegraphics[width=.18\columnwidth]{figures/supplementary/frankfurt00000_020880_gt.png}
  }
  \subfigure{%
    \includegraphics[width=.18\columnwidth]{figures/supplementary/frankfurt00000_020880_cnn.png}
  }
  \subfigure{%
    \includegraphics[width=.18\columnwidth]{figures/supplementary/frankfurt00000_020880_ours.png}
  }\\[-2ex]



  \subfigure{%
    \includegraphics[width=.18\columnwidth]{figures/supplementary/frankfurt00001_007285_given.png}
  }
  \subfigure{%
    \includegraphics[width=.18\columnwidth]{figures/supplementary/frankfurt00001_007285_sp.png}
  }
  \subfigure{%
    \includegraphics[width=.18\columnwidth]{figures/supplementary/frankfurt00001_007285_gt.png}
  }
  \subfigure{%
    \includegraphics[width=.18\columnwidth]{figures/supplementary/frankfurt00001_007285_cnn.png}
  }
  \subfigure{%
    \includegraphics[width=.18\columnwidth]{figures/supplementary/frankfurt00001_007285_ours.png}
  }\\[-2ex]


  \subfigure{%
    \includegraphics[width=.18\columnwidth]{figures/supplementary/frankfurt00001_059789_given.png}
  }
  \subfigure{%
    \includegraphics[width=.18\columnwidth]{figures/supplementary/frankfurt00001_059789_sp.png}
  }
  \subfigure{%
    \includegraphics[width=.18\columnwidth]{figures/supplementary/frankfurt00001_059789_gt.png}
  }
  \subfigure{%
    \includegraphics[width=.18\columnwidth]{figures/supplementary/frankfurt00001_059789_cnn.png}
  }
  \subfigure{%
    \includegraphics[width=.18\columnwidth]{figures/supplementary/frankfurt00001_059789_ours.png}
  }\\[-2ex]


  \subfigure{%
    \includegraphics[width=.18\columnwidth]{figures/supplementary/frankfurt00001_068208_given.png}
  }
  \subfigure{%
    \includegraphics[width=.18\columnwidth]{figures/supplementary/frankfurt00001_068208_sp.png}
  }
  \subfigure{%
    \includegraphics[width=.18\columnwidth]{figures/supplementary/frankfurt00001_068208_gt.png}
  }
  \subfigure{%
    \includegraphics[width=.18\columnwidth]{figures/supplementary/frankfurt00001_068208_cnn.png}
  }
  \subfigure{%
    \includegraphics[width=.18\columnwidth]{figures/supplementary/frankfurt00001_068208_ours.png}
  }\\[-2ex]

  \subfigure{%
    \includegraphics[width=.18\columnwidth]{figures/supplementary/frankfurt00001_082466_given.png}
  }
  \subfigure{%
    \includegraphics[width=.18\columnwidth]{figures/supplementary/frankfurt00001_082466_sp.png}
  }
  \subfigure{%
    \includegraphics[width=.18\columnwidth]{figures/supplementary/frankfurt00001_082466_gt.png}
  }
  \subfigure{%
    \includegraphics[width=.18\columnwidth]{figures/supplementary/frankfurt00001_082466_cnn.png}
  }
  \subfigure{%
    \includegraphics[width=.18\columnwidth]{figures/supplementary/frankfurt00001_082466_ours.png}
  }\\[-2ex]

  \subfigure{%
    \includegraphics[width=.18\columnwidth]{figures/supplementary/lindau00033_000019_given.png}
  }
  \subfigure{%
    \includegraphics[width=.18\columnwidth]{figures/supplementary/lindau00033_000019_sp.png}
  }
  \subfigure{%
    \includegraphics[width=.18\columnwidth]{figures/supplementary/lindau00033_000019_gt.png}
  }
  \subfigure{%
    \includegraphics[width=.18\columnwidth]{figures/supplementary/lindau00033_000019_cnn.png}
  }
  \subfigure{%
    \includegraphics[width=.18\columnwidth]{figures/supplementary/lindau00033_000019_ours.png}
  }\\[-2ex]

  \subfigure{%
    \includegraphics[width=.18\columnwidth]{figures/supplementary/lindau00052_000019_given.png}
  }
  \subfigure{%
    \includegraphics[width=.18\columnwidth]{figures/supplementary/lindau00052_000019_sp.png}
  }
  \subfigure{%
    \includegraphics[width=.18\columnwidth]{figures/supplementary/lindau00052_000019_gt.png}
  }
  \subfigure{%
    \includegraphics[width=.18\columnwidth]{figures/supplementary/lindau00052_000019_cnn.png}
  }
  \subfigure{%
    \includegraphics[width=.18\columnwidth]{figures/supplementary/lindau00052_000019_ours.png}
  }\\[-2ex]




  \subfigure{%
    \includegraphics[width=.18\columnwidth]{figures/supplementary/lindau00027_000019_given.png}
  }
  \subfigure{%
    \includegraphics[width=.18\columnwidth]{figures/supplementary/lindau00027_000019_sp.png}
  }
  \subfigure{%
    \includegraphics[width=.18\columnwidth]{figures/supplementary/lindau00027_000019_gt.png}
  }
  \subfigure{%
    \includegraphics[width=.18\columnwidth]{figures/supplementary/lindau00027_000019_cnn.png}
  }
  \subfigure{%
    \includegraphics[width=.18\columnwidth]{figures/supplementary/lindau00027_000019_ours.png}
  }\\[-2ex]



  \setcounter{subfigure}{0}
  \subfigure[\scriptsize Input]{%
    \includegraphics[width=.18\columnwidth]{figures/supplementary/lindau00029_000019_given.png}
  }
  \subfigure[\scriptsize Superpixels]{%
    \includegraphics[width=.18\columnwidth]{figures/supplementary/lindau00029_000019_sp.png}
  }
  \subfigure[\scriptsize GT]{%
    \includegraphics[width=.18\columnwidth]{figures/supplementary/lindau00029_000019_gt.png}
  }
  \subfigure[\scriptsize Deeplab]{%
    \includegraphics[width=.18\columnwidth]{figures/supplementary/lindau00029_000019_cnn.png}
  }
  \subfigure[\scriptsize Using BI]{%
    \includegraphics[width=.18\columnwidth]{figures/supplementary/lindau00029_000019_ours.png}
  }%\\[-2ex]

  \mycaption{Street Scene Segmentation}{Example results of street scene segmentation.
  (d)~depicts the DeepLab results, (e)~result obtained by adding bilateral inception (BI) modules (\bi{6}{2}+\bi{7}{6}) between \fc~layers.}
\label{fig:street_visuals-app}
\end{figure*}

\newpage
%\bibliographystyle{ACM-Reference-Format}
\bibliographystyle{plain}	
\bibliography{Yang}
\end{document} 