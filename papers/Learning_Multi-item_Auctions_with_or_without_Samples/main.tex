%\documentclass[format=acmsmall, review=false]{acmart}

%\usepackage{acm-ec-17}
\documentclass[11pt]{article}

\usepackage{times}
\usepackage{amsmath,amssymb,amsthm,amsfonts,latexsym,bbm,xspace,graphicx,float,mathtools,dsfont}
\usepackage[backref,colorlinks,citecolor=blue,bookmarks=true]{hyperref}
%\usepackage[dvips, paper=letterpaper, top=1in, bottom=.75in, left=1in, right=1in, nohead, includefoot, footskip=.25in]{geometry}
\usepackage[dvips, paper=letterpaper, top=1in, bottom=.75in, left=0.9in, right=0.95in, nohead, includefoot, footskip=.25in]{geometry}
\usepackage{color}
\usepackage{comment}
\usepackage{algorithm}
\usepackage{algorithmic}
\newcommand{\ind}{\mathds{1}}
\usepackage{tweaklist}
%\usepackage{fullpage}
\usepackage{appendix}
\usepackage{cleveref}

\usepackage{accents}
\newcommand{\ubar}[1]{\underaccent{\bar}{#1}}
\newcommand{\cnote}[1]{\textcolor{red}{#1}}


\newtheorem{theorem}{Theorem}
%\newtheorem{informal}[theorem]{Informal Theorem}
\newtheorem{corollary}{Corollary}
\newtheorem{lemma}{Lemma}
\newtheorem{conjecture}{Conjecture}
\newtheorem{claim}{Claim}
\newtheorem{proposition}{Proposition}
\newtheorem{fact}{Fact}
\newtheorem{definition}{Definition}
\newtheorem{assumption}{Assumption}
\newtheorem{observation}{Observation}
\newtheorem{example}{Example}
\newtheorem{remark}{Remark}
\newtheorem{question}{Question}
\newtheorem{hypothesis}{Hypothesis}
\newtheorem{informaltheorem}{Informal Theorem}
\newcommand{\single}{\textsc{Single}}
\newcommand{\nf}{\textsc{Non-Favorite}}
\newcommand{\core}{\textsc{Core}}
\newcommand{\tail}{\textsc{Tail}}
\newcommand{\rev}{\textsc{Rev}}
\newcommand{\bvcg}{\textsc{BVCG}}
\newcommand{\copies}{\textsc{OPT}^{\textsc{Copies-UD}}}
\newcommand{\vcg}{\textsc{VCG}}
\newcommand{\srev}{\textsc{SRev}}
\newcommand{\brev}{\textsc{BRev}}
\newcommand{\aperev}{\textsc{APostEnRev}}
\newcommand{\prev}{\textsc{PostRev}}
\newcommand{\Med}{\textsc{Median}}
\newcommand{\Dim}{\text{Dim}}
\newcommand{\sold}{\textsc{SOLD}}
\newcommand{\bbeta}{\boldsymbol{\beta}}
%%%\newcommand{\qed}{\rule{7pt}{7pt}}
%%%\newcommand{\order}[1]{O(#1)}
\newenvironment{prevproof}[2]{\noindent {\em {Proof of {#1}~\ref{#2}:}}}{$\Box$\vskip \belowdisplayskip}

%\newenvironment{proof}{\ni {\em Proof:}}{\hfill $\Box$ \bigbreak}
%\newcommand{\qed}{\hfill $\Box$}
%%%\newenvironment{example}{{\bf Example:}}{\hfill $\Box$ \bigbreak}

\newcommand{\mychoose}[2]{\left( \begin{array}{c} #1 \\ #2 \end{array} \right)}
\newcommand{\Bs}{\bold{s_i}}
\newcommand{\Br}{\bold{r_i}}
\newcommand{\SI}{\bf{\sigma}}
\newcommand{\N}{\ensuremath{\mathbb{N}}} % The naturals
\newcommand{\Z}{\ensuremath{\mathbb{Z}}} % The integers
\newcommand{\R}{\ensuremath{\mathbb{R}}} % The reals
\newcommand{\Q}{\ensuremath{\mathbb{Q}}} % The rationals
\newcommand{\dx}{\text{dx}}
\newcommand{\lose}{\text{LOSE}}
\newcommand{\poly}{{\rm poly}}
\newcommand{\opt}{\text{OPT}}
\newcommand{\todo}[1]{{\bf \color{red} TODO: #1}}
\newcommand{\yangnote}[1]{{\color{blue}{#1}}}

\newcommand{\mingfeinote}[1]{{\color{black}{#1}}}
\newcommand{\notshow}[1]{{}}
\newcommand{\rt}{\rm{rt}}

\DeclareMathOperator{\Prob}{Pr}
\DeclareMathOperator{\E}{E}
%\def \ni {\noindent}
\def \bfm {\bf \boldmath}
\def \TT  {{\cal T}}
\def \DD  {{\mathcal{D}}}
%\def \DD  {{\mathcal{D}}}
\def \KK  {{\mathcal{K}}}
\def \AA  {{\mathcal{A}}}
\def \CC {{\mathcal{C}}}
\def \HH {{\mathcal{H}}}
\def \XX {{\mathcal{X}}}
\def \PP  {{\cal P}}
\def \GG  {{\cal G}}
\def \OO  {{\cal O}}
\def \NN  {{\cal N}}
\def \LL  {{\mathcal{L}}}
\def \BB  {{\cal B}}
\def \VV  {{\cal V}}
\def \EE  {{\mathcal{E}}}
\def \ee {\varepsilon}
\def \t  {\tilde}
\def \FF  {{\mathcal{F}}}
\def \SS  {{\mathcal{S}}}
\def \obj {{\cal O}}
\def \pc   {{\sf P}}                              %%% P complexity class
\def \npc  {{\sf NP}}                             %%% NP complexity class
\def \ncc  {{\sf NC}}                             %%% NC complexity class
\def \zppc  {{\sf ZPP}}                           %%% ZPP complexity class
\def \qpc  {{\sf DTIME}(n^{O({\sf polylog}(n))})} %%% QusiPoly complexity class
\def \snp  {{\sf Max SNP}}                        %%% Max SNP complexity class
\def \ptas {{\sf PTAS}~}
\def \gr   {{\sf Greedy-1}}
\def \grr   {{\sf Greedy}}
\def \E  {{\mathbb{E}}}
\def \Var{{\text{Var}}}
\def \polytope{{P(D)}}
\def \L{{\mathcal{L}}}

\DeclareMathOperator{\argmax}{argmax}
\DeclareMathOperator{\argmin}{argmin}

\def \ee {{\rm e}} % Euler number
\def\tp{{\tilde{\varphi}}}
\def\I{{\mathbb{I}}}
\definecolor{MyGray}{rgb}{0.8,0.8,0.8}

\newcommand{\graybox}[2]{\begin{figure}[t]
\colorbox{MyGray}{
\begin{minipage}{\textwidth}
{#2}
\end{minipage}
}\caption{{#1}}
\end{figure}
}
\setlength{\parskip}{0in}

\title{Learning Multi-item Auctions with (or without) Samples}

\author{Yang Cai\\
McGill University, Canada\\
cai@cs.mcgill.ca \and Constantinos Daskalakis\\ EECS and CSAIL, MIT, USA\\ costis@csail.mit.edu}

\begin{document}
\maketitle

\begin{abstract}
We provide algorithms that learn simple auctions whose revenue is approximately optimal in multi-item multi-bidder settings, for a wide range of bidder valuations including unit-demand, additive, constrained additive, XOS, and subadditive. We obtain our learning results in two settings. The first is the commonly studied setting where sample access to the bidders' distributions  over valuations is given, for both regular distributions and arbitrary distributions with bounded support. Here, our algorithms require polynomially many samples in the number of items and bidders. The second is a more general max-min learning setting that we introduce, where we are given ``approximate distributions,'' and we seek to compute a mechanism whose revenue is approximately optimal simultaneously for all ``true distributions'' that are close to the ones we were given. These results are more general in that they imply the sample-based results, and are also applicable in settings where we have no sample access to the underlying distributions but  have estimated them indirectly via market research or by observation of bidder behavior in previously run, potentially non-truthful auctions. 

All our results hold for valuation distributions satisfying the standard (and necessary) independence-across-items property. They also generalize and improve upon recent works of Goldner and Karlin~\cite{GoldnerK16} and Morgenstern and Roughgarden~\cite{MorgensternR16}, which have provided algorithms that learn approximately optimal multi-item mechanisms in more restricted settings with additive, subadditive and unit-demand valuations using sample access to distributions. We generalize these results to the complete unit-demand, additive, and XOS setting, to i.i.d. subadditive bidders, and to the max-min setting. 

Our results are enabled by new uniform convergence bounds for hypotheses classes under product measures. Our bounds result in exponential savings in sample complexity compared to bounds derived by bounding the VC dimension, and are of independent interest.
\end{abstract}
\thispagestyle{empty}
\addtocounter{page}{-1}
\newpage

Reinforcement learning has achieved great success in areas such as Game-playing \citep{silver2018general,vinyals2019grandmaster}, robotics \cite{kober2013reinforcement}, large language models \citep{ouyang2022training}, etc.
However, due to safety concerns or physical limitations, in some real-world reinforcement learning problems, we must consider additional constraints that may influence the optimal policy and the learning process \citep{garcia2015comprehensive}.
% For example, a robotic arm must not take actions that may cause harm to itself or the environments.
A standard framework to handle such cases is the constrained Markov Decision Process (CMDP) \citep{altman1999constrained}.
Within the CMDP framework, the agent has to maximize
the expected cumulative reward while
obeying a finite number of constraints, which are usually in the form of expected cumulative cost criteria.

However, we are sometimes concerned with the problem with a continuum of constraints.
For example,
the constraints we meet might be time-evolving or subject to uncertain parameters, which
cannot be formulated as an ordinary CMDP
(see Examples \ref{Example_Time_Evolving} and  \ref{Example_Uncertain}).
In this paper we would study a generalized CMDP  
to address the above problem.  Because the constraints are not only infinite-number but also lie
in a continuous set,
the generalization is not trivial. Fortunately, we find that we can borrow the idea behind semi-infinite programming (SIP) \citep{remez1934determination, hettich1993semi} to deal with the semi-infinite constraints.
Accordingly, we propose \emph{semi-infinitely constrained Markov decision processes} (SICMDPs)
as a novel complement to the ordinary CMDP framework.
%More specifically,  an SICMDP model %, we consider 
%contains a continuum of constraints whereas an ordinary CMDP contains a finite number of constraints. 

%This generalization is natural but not trivial. However, we can brows the idea  
%The idea is quite natural and can be backtracked
%to the practice of extending linear programming to linear semi-infinite programming (LSIP) %\cite{remez1934determination, GobernaLSIO1998}.
%In addition, 
%As a complementary approach to the ordinary CMDP framework, 
%SICMDP can be used to model these problems  which cannot be described by a finite number of constraints
%that are not covered by .
%For example,
%the restrictions we consider can be time-evolving or subject to uncertain parameters
%, thus
%cannot be described by a finite number of constraints but a continuum of constraints 
%(see Examples \ref{Example_Time_Evolving} and  \ref{Example_Uncertain}).

We also present two reinforcement learning algorithms to solve SICMDPs called SI-CRL and SI-CPO, respectively.
SI-CRL is a model-based reinforcement learning algorithm designed for tabular cases, and SI-CPO is a policy optimization algorithm for non-tabular cases.
% and analyze its performance both theoretically and empirically.
The main challenge is that we need to deal with a continuum of constraints, thus reinforcement learning algorithms for ordinary CMDPs do not work anymore.
In SI-CRL, we tackle this difficulty by first transforming the reinforcement learning problem to an equivalent LSIP problem, which can then be solved using methods in the LSIP literature like the dual exchange methods \citep{Hu1990,reemtsen1998numerical}.
In SI-CPO, we resort to the idea of cooperative stochastic approximation developed in \cite{lan2020algorithms, wei2020comirror}.
As far as we know, we are the first to introduce tools from semi-infinitely programming (SIP) into the reinforcement learning community for solving constrained reinforcement learning problems.

% To the best of our knowledge, we are the first to apply tools from semi-infinitely programming (SIP) to solve reinforcement learning problems.
Furthermore, we give theoretical analysis for both SI-CRL and SI-CPO.
We decompose the error of SI-CRL into two parts: the statistical error from approximating the true SICMDP with an offline dataset and the optimization error due to the fact that the solution of the LSIP problem obtained by the dual exchange method is inexact.
On the optimization side, we show that the iteration complexity of SI-CRL is $O\left(\left\{\mathrm{diam}(Y)L\sqrt{|\gS|^2|\gA|m}/\left[(1-\gamma)\epsilon\right]\right\}^m\right)$.
On the statistical side, we show that the sample complexity of SI-CRL is $\widetilde O\left(\frac{|S|^2|A|^2}{\epsilon^2(1-\gamma)^3}\right)$ if the offline dataset is generated by a generative model, and $\widetilde O\left(\frac{|S||A|}{\nu_{\min} \epsilon^2(1-\gamma)^3}\right)$ if the dataset is generated by a probability measure $\nu$ as considered in \cite{chen2019information}.
Here $\widetilde O$ means that all logarithm terms are discarded.
For SI-CPO, things become a little more complicated because other than the statistical error and the optimization error, we also need to consider the function approximation error, which comes from imperfect policy parametrizations.
It is shown if the function approximation error can be controlled to $O(\epsilon)$ order, the iteration complexity of SI-CPO is $\widetilde{O}\left(\frac{1}{\epsilon^2(1-\gamma)^6}\right)$ and the sample complexity of SI-CPO is $\widetilde{O}(\frac{1}{\epsilon^4(1-\gamma)^{10}})$.
Here our iteration complexity bound is equivalent to a typical $\widetilde O(1/\sqrt{T})$ global convergence rate.

We perform a set of numerical experiments to illustrate the SICMDP model and validate our proposed algorithms.
Specifically, we examine two numerical examples, namely the discharge of sewage and ship route planning.
Through the discharge of sewage example, we show the advantage of the SICMDP framework over the CMDP baseline obtained by naive discretization in modeling realistic sequential decision-making problems.
Moreover, we demonstrate the effectiveness of the SI-CRL and SI-CPO algorithms in such tabular environments. 
In the ship route planning example, we illustrate the benefits of the SICMDP framework and the ability of the SI-CPO algorithm to address complex continuous control tasks involving continuous state spaces with modern deep reinforcement learning techniques.

% In summary, our contributions are listed as follows.
% First, we present the SICMDP model, which can be viewed as a generalization of the ordinary CMDP model.
% Second, we propose an algorithm to perform reinforcement learning for SICMDPs, which is called SI-CRL, and we believe that we are the first to apply tools from SIP
% to solve reinforcement learning problems.
% Third, we give a theoretical analysis of SI-CRL and identify both its sample complexity and iteration complexity.
% In addition, we perform numerical experiments to illustrate the SICMDP model and validate the SI-CRL algorithm.
% \{This paragraph can be removed!!! \}






We study the problem of selling $m$ items to $n$ buyers. We denote a bundle of items as a quantity vector $\vec{q} \in \Z_{\geq 0}^m$. The number of units of item $i$ in the bundle is $q[i]$. The bundle consisting of only one copy of the $i^{th}$ item is denoted by the standard basis vector $\vec{e}_i$, where $e_i[i] = 1$ and $e_i[j] = 0$ for all $j \not= i$. Each buyer $j \in [n]$ has a valuation function $v_j$ over bundles of items. We denote an allocation as $Q = \left(\vec{q}_1, \dots, \vec{q}_n\right)$ where $\vec{q}_j$ is the bundle that buyer $j$ receives. The cost to produce $\vec{q}$ is $c\left(\vec{q}\right)$ and the cost to produce the allocation $Q$ is $c\left(Q\right)$.
Suppose there are $\kappa_i$ units available of item $i$. Let $K = \prod_{i = 1}^m \left(\kappa_i+1\right)$. We use $\vec{v}_j = \left(v_j\left(\vec{q}_1\right), \dots, v_j\left(\vec{q}_K\right)\right)$ to denote buyer $j$'s values for all of the $K$ bundles and we use $\vec{v} = \left(\vec{v}_1, \dots, \vec{v}_n\right)$ to denote a vector of buyer values. We use the notation $\cX$ to denote the set of all valuation vectors $\vec{v}$. Additive buyers have values $v_j\left(\vec{q}\right) = \sum_{i = 1}^m q[i] v_j\left(\vec{e}_i\right)$ and unit-demand buyers have values $v_j\left(\vec{q}\right) = \max_{i : q[i] \geq 1} v_j\left(\vec{e}_i\right)$. The mechanisms  we study are dominant strategy incentive compatible, so we assume that the bids equal the buyers' valuations.

There is an unknown distribution $\pazocal{D}$ over buyers' values. 
The notation $\profit_M\left(\vec{v}\right)$ denotes the profit of a mechanism $M$ on the valuation vector $\vec{v}$. We use the notation $\profit_{\dist}\left(M\right) = \E_{\vec{v} \sim \dist}\left[\profit_M\left(\vec{v}\right)\right]$ and for a set of samples $\sample$, we use the notation \[\profit_{\sample}\left(M\right) = \frac{1}{|\sample|}\sum_{\vec{v} \in \sample}\profit_M\left(\vec{v}\right).\]

We study real-valued functions parameterized by vectors $\vec{p}$ in $\R^d$, denoted as $f_{\vec{p}}:\domain \to \R.$ For a fixed $\vec{v} \in \domain$, we often consider $f_{\vec{p}}\left(\vec{v}\right)$ as a function of its parameters, which we denote as $f_{\vec{v}}\left(\vec{p}\right)$.
\section{Uniform Convergence under Product Measures}\label{sec:uniform convergence under product measure}

In this section, we develop machinery for obtaining uniform convergence bounds for hypotheses over product measures. Our goal is to save on the sample complexity implied by VC dimension bounds, as summarized in Table~\ref{tab:productVC}. Indeed, we obtain low sample complexity  bounds for indicators over \emph{single-intersecting} sets (see Definition~\ref{def:single-intersecting}), which play a key role in proving our results for learning approximately revenue-optimal auctions. Our main results of this section are Theorem~\ref{thm:uniform convergence for product measure PARTITION} for general functions, and Corollary~\ref{cor:VC for product measure} for sets.

We first define what type of uniform convergence bounds we seek to prove.

\begin{definition}[$(\epsilon,\delta)$-uniform convergence with respect to proxy measure]
	A hypothesis class $\HH$ of functions mapping domain set $\XX$ to $\mathbb{R}$ has {\em $(\epsilon,\delta)$-uniform convergence with sample complexity $s(\epsilon,\delta)$} iff, for all $\epsilon,\delta > 0$, there exists a {\em processing} ${\cal P}:{\cal X}^{s(\epsilon,\delta)} \rightarrow \Delta(\XX)$ such that for any distribution $\DD \in \Delta(\XX)$ when $k=s(\epsilon,\delta)$: 
	\begin{align}
	\Pr_{z_1,\cdots, z_k \sim \DD}\left[\sup_{g\in \HH}\left|\E_{z\sim {\cal P}(z_1,\cdots, z_k)}[g(z)] - \E_{z\sim \DD}[g(z)]\right|\leq \epsilon\right]\geq 1-\delta. \notag
	\end{align}
%$$\Pr\left[\sup_{g\in \HH}\left|\frac{1}{k}\cdot \sum_{j=1}^{k} g(z_j) - \E_{z\sim \DD}[g(z)]\right|\leq \epsilon\right]\geq 1-\delta.$$

When ${\cal X}$ is the Cartesian product of a collection of sets ${\cal X}_1,\ldots,{\cal X}_k$, i.e.~${\cal X} =\times_i {\cal X}_i$, we say that a hypothesis class $\HH$ as above has {\em $(\epsilon,\delta)$-p.m.~uniform convergence with sample complexity $s(\epsilon,\delta)$} if the above holds for all ${\cal D}$ that are product measures over ${\cal X}$.	
\end{definition}

Next we provide a simple lemma, which leads to a simple version of our main result stated as Theorem~\ref{thm:uniform convergence for product measure}. Our main result, stated as Theorem~\ref{thm:uniform convergence for product measure PARTITION}, follows.

%In particular, we show that if each of the projected hypothesis class uniformly converges with polynomial many samples, the original hypothesis class also uniformly converges with a polynomial number of samples. 
%
%In the end of this section, we apply our uniform convergence results to a particular hypothesis class called 

\begin{lemma} \label{lem:approx product dist}
	Let $\XX_1,\ldots, \XX_d$ be $d$ domain sets and $\HH$ be a hypothesis class with functions mapping from the product space $\times_{i=1}^{d} \XX_i$ to $\mathbb{R}$. For all $i\in[d]$, let $\HH_i$ be the projected hypothesis class of $\HH$ on $\XX_i$, that is, $\HH_i=\left\{g\ |\ \exists f\in \HH\ \exists\ a_{-i}\in \times_{j\neq i} \XX_j \ \forall\ x_i\in \XX_i, \ g(x_i) = f(x_i,a_{-i})\right\}$. For every $i\in[d]$, let $\DD_i$ and $\hat{\DD}_i$ be two distributions supported on $\XX_i$. Suppose for all $i\in[d]$, $$\sup_{g\in\HH_i}\left|\E_{x\sim \DD_i}[g(x)]-\E_{x\sim \hat{\DD}_i}[g(x)]\right|\leq \epsilon,$$ then $$\sup_{f\in\HH} \left|\E_{\boldsymbol{x}\sim \times_{i=1}^d \DD_i}\left[f(\boldsymbol{x})\right]-\E_{\boldsymbol{x}\sim \times_{i=1}^d \hat{\DD}_i}\left[f(\boldsymbol{x})\right]\right|\leq d\cdot \epsilon.$$
\end{lemma} 

\begin{proof}
	Let $\FF_i$ and $\hat{\FF}_i$ be the probability measure function for $\DD_i$ and $\hat{\DD}_i$ respectively. We will prove the statement using a hybrid argument. We create a sequence of product distributions $\{\DD^{(j)}\}_{j\leq d}$, where $\DD^{(j)}=\hat{\DD}_1\times\cdots\times\hat{\DD}_j\times \DD_{j+1}\times\cdots\times \DD_{d},$ and $\DD^{(0)}=\DD$, $\DD^{(d)}=\hat{\DD}$. To prove our claim, it suffices to show that for any integer $j\in[d]$, $$\left|\E_{\boldsymbol{x}\sim \DD^{(j-1)}}\left[f(\boldsymbol{x})\right]-\E_{\boldsymbol{x}\sim \DD^{(j)}}\left[f(\boldsymbol{x})\right]\right|\leq \epsilon.$$
	%Let us first fix some notations. Let $\EE_{-j}=\left\{x_{-j}\ |\ \exists x_j\in \mathbb{R},\ (x_j,x_{-j})\in \EE\right\}$, $\EE_j(a_{-j})= \left\{x_{j} \in \mathbb{R} \ |\ (x_j,a_{-j})\in \EE\right\}$ for all $j\in[d]$. As $\EE$ is single-intersecting, $\EE_j(x_{-j})$ is an interval for all $j$ and $x_{-j}$. Moreover, since $||\DD_j-\hat{\DD}_j||_K\leq \xi$, we have $|\Pr_{\DD_j}[x_j\in \EE_j(x_{-j})]-\Pr_{\hat{\DD}_j}[x_j\in \EE_j(x_{-j})]|\leq 2\xi$ for all $j$ and $x_{-j}$. Next, we bound the difference for the probability of event $\EE$ under $\DD^{(j-1)}$ and $\DD^{(j)}$.
	
	Next, we show how to derive this inequality.
	\begin{align*}
		&\left|\E_{\boldsymbol{x}\sim \DD^{(j-1)}}\left[f(\boldsymbol{x})\right]-\E_{\boldsymbol{x}\sim \DD^{(j)}}\left[f(\boldsymbol{x})\right]\right|\\
		=&\Bigg{|}\int_{\times_{i\neq j} \XX_i}\left(\int_{\XX_j}f(x_{j},x_{-j}) d \FF_j(x_j)\right) d\hat{\FF}_1(x_1)\cdots d\hat{\FF}_{j-1}(x_{j-1}) d\FF_{j+1}(x_{j+1})\cdots d\FF_{d}(x_{d})\\
		 & ~~~~~~~~~~~~~~~~~~~~ - \int_{\times_{i\neq j} \XX_i}\left(\int_{\XX_j}f(x_{j},x_{-j}) d \hat{\FF}_j(x_j)\right) d\hat{\FF}_1(x_1)\cdots d\hat{\FF}_{j-1}(x_{j-1}) d\FF_{j+1}(x_{j+1})\cdots d\FF_{d}(x_{d})\Bigg{|}\\
		 = & \left|\int_{\times_{i\neq j} \XX_i} \left(\E_{x_j\sim \DD_j}\left[f(x_j,x_{-j})\right]-\E_{x_j\sim \hat{\DD}_j}[f(x_j,x_{-j})]\right)
		  d\hat{\FF}_1(x_1)\cdots d\hat{\FF}_{j-1}(x_{j-1}) d\FF_{j+1}(x_{j+1})\cdots d\FF_{d}(x_{d})\right|\\
		  \leq & \epsilon \cdot\int_{\times_{i\neq j} \XX_i} d\hat{\FF}_1(x_1)\cdots d\hat{\FF}_{j-1}(x_{j-1}) d\FF_{j+1}(x_{j+1})\cdots d\FF_{d}(x_{d})\\
		   = & \epsilon
	\end{align*}
\end{proof}



\begin{theorem}\label{thm:uniform convergence for product measure}
Let $\XX_1,\ldots, \XX_d$ be $d$ domain sets and $\HH$  a hypothesis class of functions mapping from the product space $\times_{i=1}^{d} \XX_i$ to $\mathbb{R}$. For all $i\in[d]$, let $\HH_i$ be the projected hypothesis class of $\HH$ on $\XX_i$, that is $\HH_i=\left\{g\ |\ \exists f\in \HH\ \exists\ a_{-i}\in \times_{j\neq i} \XX_j \ \forall\ x_i\in \XX_i, \ g(x_i) = f(x_i,a_{-i})\right\}$. 

Suppose that, for all $i\in [d]$, $\HH_i$ has $(\epsilon,\delta)$-uniform convergence with sample complexity $s_i(\epsilon,\delta)$.  Then $\HH$ has $(\epsilon,\delta)$-p.m.~uniform convergence with sample complexity $s(\epsilon,\delta) = \max_{i\in[d]} s_i(\epsilon/d,\delta/d)$.

In particular, let $\boldsymbol{z}^{(1)},\ldots, \boldsymbol{z}^{(\ell)}$ be a sample of size $\ell = s(\epsilon,\delta)$ from a product measure $\times_{i\in[d]} \DD_i$. Define $\hat{\DD}_i = {\cal P}_i({z}^{(1)}_i,\ldots,z^{(\ell)}_i)$, for all $i\in[d]$, where $z^{(j)}_i$ is the $i$-th entry of sample $\boldsymbol{z}^{(j)}$ and ${\cal P}_i$ is the processing corresponding to $\HH_i$'s uniform convergence. Then 
$$\Pr_{\boldsymbol{z}^{(1)},\ldots, \boldsymbol{z}^{(\ell)}}\left[\sup_{f\in \HH}\left|\E_{\boldsymbol{z}\sim \times_{i\in[d]}\hat{\DD}_i}\left[f(\boldsymbol{z})\right] - \E_{\boldsymbol{z}\sim \times_{i\in[d]}{\DD}_i}\left[f(\boldsymbol{z})\right]\right|\leq \epsilon\right]\geq 1-\delta.$$
\end{theorem}

\begin{prevproof}{Theorem}{thm:uniform convergence for product measure}
	Since $\ell\geq s_i(\epsilon/d,\delta/d)$, $\Pr\left[\sup_{g\in \HH_i}\left|\E_{z\sim \hat{\DD}_i}[g(z)]- \E_{z\sim \DD_i}[g(z)]\right|\leq \epsilon/d\right]\geq 1-\delta/d$ for all $i\in[d]$. By the union bound, with probability at least $1- \delta$, $\sup_{g\in \HH_i}\left|\E_{z\sim \hat{\DD}_i}[g(z)]- \E_{z\sim \DD_i}[g(z)]\right|\leq \epsilon/d$ for all $i\in[d]$. According to Lemma~\ref{lem:approx product dist}, $\sup_{f\in \HH}\left|\E_{\boldsymbol{z}\sim \times_{i\in[d]}\hat{\DD}_i}\left[f(\boldsymbol{z})\right] - \E_{\boldsymbol{z}\sim \times_{i\in[d]}{\DD}_i}\left[f(\boldsymbol{z})\right]\right|\leq \epsilon$ with probability at least $1- \delta$.
\end{prevproof}

\begin{theorem}\label{thm:uniform convergence for product measure PARTITION}
Let $\XX_1,\ldots, \XX_d$ be $d$ domain sets and $\HH$ a hypothesis class of functions mapping from the product space $\times_{i=1}^{d} \XX_i$ to $\mathbb{R}$. For all $T\subseteq [d]$, let $\HH_T$ be the projected hypothesis class of $\HH$ on $\XX_T \equiv \times_{i \in T}\XX_i$, that is, $\HH_T=\left\{g\ |\ \exists f\in \HH\ \exists\ a_{-T}\in \times_{j\notin T} \XX_j \ \forall\ x_T\in \XX_T, \ g(x_T) = f(x_T,a_{-T})\right\}$. Suppose that, for all $T \subseteq [d]$, $\HH_T$ has $(\epsilon,\delta)$-p.m.~uniform convergence with sample complexity $s_T(\epsilon,\delta)$, and define 
\begin{align}s(\epsilon,\delta) = \min_{\begin{minipage}[h]{3.5cm}\centering $k$,~partitions\\$T_1 \sqcup T_2 \sqcup \ldots \sqcup T_k = [d]$\end{minipage}} \max_{i=1,\ldots,k} s_{T_i}(\epsilon/k,\delta/k). \label{eq:yang}
\end{align}
Then $\HH$ has $(\epsilon,\delta)$-p.m.~uniform convergence with sample complexity $s(\epsilon,\delta)$.

In particular, let $\boldsymbol{z}^{(1)},\ldots, \boldsymbol{z}^{(\ell)}$ be a sample of size $\ell = s(\epsilon,\delta)$ from a product measure $\times_{i\in[d]} \DD_i$. Suppose that the optimum of~\eqref{eq:yang} is attained at $k=\tilde{k}$ for partition $\tilde{T}_1 \sqcup \tilde{T}_2 \sqcup \ldots \sqcup \tilde{T}_{\tilde{k}} = [d]$. Define $\hat{\DD}_{\tilde{T}_i} = {\cal P}_{\tilde{T}_i}({\boldsymbol{z}}^{(1)}_{\tilde{T}_i},\ldots,{\boldsymbol{z}}^{(\ell)}_{\tilde{T}_i})$, for all $i\in[d]$,  where ${\boldsymbol{z}}^{(j)}_{\tilde{T}_i}$ contains the entries of sample $\boldsymbol{z}^{(j)}$ in coordinates $\tilde{T}_i$ and ${\cal P}_{\tilde{T}_i}$ is the processing corresponding to $\HH_{\tilde{T}_i}$'s uniform convergence. Then 
$$\Pr_{\boldsymbol{z}^{(1)},\ldots, \boldsymbol{z}^{(\ell)}}\left[\sup_{f\in \HH}\left|\E_{\boldsymbol{z}\sim \times_{i\in[\tilde{k}]}\hat{\DD}_{\tilde{T}_i}}\left[f(\boldsymbol{z})\right] - \E_{\boldsymbol{z}\sim \times_{i\in[d]}{\DD}_i}\left[f(\boldsymbol{z})\right]\right|\leq \epsilon\right]\geq 1-\delta.$$


%Then $\HH$ has $(\epsilon,\delta)$-uniform convergence with sample complexity $s(\epsilon,\delta)$.
\end{theorem}
\begin{prevproof}{Theorem}{thm:uniform convergence for product measure PARTITION}
For every possible partition use Theorem~\ref{thm:uniform convergence for product measure}.
\end{prevproof}


Nest, we specialize Theorem~\ref{thm:uniform convergence for product measure PARTITION} to indicator functions over sets.

%In the next Corollary, we show that if all functions in $\HH$ have $0$ or $1$ values and each $\HH_i$ has VC dimension $V_i$, 
\begin{corollary}\label{cor:VC for product measure}
	We use the same notation as in Theorem~\ref{thm:uniform convergence for product measure PARTITION}. Suppose that all functions in $\HH$ map $\times_{i=1}^{d} \XX_i$ to $\{0,1\}$, i.e.~they are indicators over sets. Suppose also that the VC dimension of $\HH_T$ (viewed as a collection of sets) is $V_T$. Define 
	\begin{align}V_{\max}=\min_{\begin{minipage}[h]{3.5cm}\centering $k$,~partitions\\$T_1 \sqcup T_2 \sqcup \ldots \sqcup T_k = [d]$\end{minipage}} \left\{ k^2 \cdot\max_{i=1,\ldots,k} V_{T_i}\right\}. \label{eq:costas}
	\end{align}
	Assume that the optimum of~\eqref{eq:costas} is attained at $k=\tilde{k}$ for partition $\tilde{T}_1 \sqcup \tilde{T}_2 \sqcup \ldots \sqcup \tilde{T}_{\tilde{k}} = [d]$. 
	
	Then
	%
	%Theorem~\ref{thm:uniform convergence for product measure PARTITION} implies that 
	$\ell = O\left(\frac{V_{\max}}{\epsilon^2}\cdot \ln \frac{\tilde{k}}{\epsilon}+\frac{\tilde{k}^2}{\epsilon^2}\cdot \ln \frac{\tilde{k}}{\delta} \right)$ samples from $\times_{i\in[d]}\DD_i$ suffice to obtain $(\epsilon,\delta)$-p.m. uniform convergence for $\HH$. 
	Formally,
	$$\Pr_{\boldsymbol{z}^{(1)},\ldots, \boldsymbol{z}^{(\ell)}}\left[\sup_{f\in \HH}\left|\E_{\boldsymbol{z}\sim \times_{i\in[\tilde{k}]}\hat{\DD}_{\tilde{T}_i}}\left[f(\boldsymbol{z})\right] - \E_{\boldsymbol{z}\sim \times_{i\in[d]}{\DD}_i}\left[f(\boldsymbol{z})\right]\right|\leq \epsilon\right]\geq 1-\delta,$$
	where for a given sample $\boldsymbol{z}^{(1)},\ldots, \boldsymbol{z}^{(\ell)}$ from a product distribution $\times_{i\in[d]} \DD_i$ the distributions $\hat{\DD}_{\tilde{T}_i}$ are defined to be uniform over ${\boldsymbol{z}}^{(1)}_{\tilde{T}_i},\ldots,{\boldsymbol{z}}^{(\ell)}_{\tilde{T}_i}$, where ${\boldsymbol{z}}^{(j)}_{\tilde{T}_i}$ contains the entries of sample $\boldsymbol{z}^{(j)}$ in coordinates $\tilde{T}_i$.
\end{corollary}

Table~\ref{tab:productVC} compares the sample complexity for uniform convergence implied by Theorem~\ref{thm:uniform convergence for product measure PARTITION} and Corollary~\ref{cor:VC for product measure} to that implied by VC theory, when the underlying measures are product. Suppose $\HH$ contains the indicator functions of all convex sets in $\mathbb{R}^d$. VC theory does not provide any finite sample  bound for  uniform convergence, as the VC dimension of $\HH$ is $\infty$. Do our results provide a finite bound? Notice that, for all $i$, $\HH_i$  simply contains all intervals in $\mathbb{R}$. Hence, $V_i=2$ and Corollary~\ref{cor:VC for product measure} implies that $\ell = O(\frac{d^2}{\epsilon^2}\cdot \left(\log \frac{d}{\delta}+\log \frac{d}{\epsilon}\right) )$ samples  suffice to obtain $(\epsilon,\delta)$-p.m.~uniform convergence for $\HH$ . In fact, our sample complexity bound can be improved to $O\left(\frac{d^2}{\epsilon^2}\cdot \log \frac{d}{\delta}\right)$, as $O\left(\frac{\log \frac{1}{\delta}}{\epsilon^2}\right)$ samples suffice to guarantee $(\epsilon,\delta)$-uniform convergence for all intervals in $\mathbb{R}$ due to the DKW inequality~\cite{DvoretzkyKW56}. 

%\todo{Expand the discussion here.}

In the next a few sections, we apply our uniform convergence results to learn a mechanism with approximately optimal revenue. A type of events called \emph{single-intersecting} (see Definition~\ref{def:single-intersecting})   plays a key role in our analysis. These events are defined based on the geometric shape of the corresponding sets. For example, balls, rectangles and all convex sets are single-intersecting, but this definition includes some non-convex sets as well, for example, ``cross-shaped'' sets. It turns out that being able to handle these non-convex sets is crucial for our results, as many events we care about are not convex but nonetheless are single-intersecting. 

\begin{definition}[Single-intersecting Events]\label{def:single-intersecting}
For any event $\EE$ in $\mathbb{R}^{\ell}$, $\EE$ is \textbf{single-intersecting} if the intersection of $\EE$ and any line that is parallel to one of the axes is an interval. More formally, for any $i\in[\ell]$ and any line $L_i=\left\{x \in \mathbb{R}^{\ell}\ |\ x_{-i}=a_{-i} \right \}$, where $a_{-i} \in \mathbb{R}^{\ell-1}$, the intersection of $L_i$ and $\EE$ is of the form $\left\{x \in \mathbb{R}^{\ell}\ |\ x_{-i}=a_{-i}, x_i\in[\ubar{a},\bar{a}] \right\}$ where $\ubar{a}\leq \bar{a}$. In particular, we allow $\ubar{a}$ to be $-\infty$ and $\bar{a}$ to be $+\infty$.
\end{definition}


\noindent We establish a uniform convergence bound for single-intersecting events by combing the DKW inequality and Theorem~\ref{thm:uniform convergence for product measure}.

\begin{lemma}\label{lem:uniform convergence for single-intersecting}
	For any integer $\ell$, let $\HH$ be the hypothesis class that contains all indicator functions for single-intersecting events in $\mathbb{R}^\ell$. Then $\HH$ has $(\epsilon,\delta)$-p.m. uniform convergence with sample complexity $O\left(\frac{\ell^2}{\epsilon^2}\cdot \log \frac{\ell}{\delta}\right)$.
\end{lemma}
\begin{proof}
	As the projected hypothesis class for the $i$-th coordinate simply contains all intervals in $\R$, the sample complexity for $(\epsilon,\delta)$-uniform convergence is $O({1 \over \epsilon^2}\cdot \log \frac{1}{\delta})$ due to the DKW inequality. The claim follows from Theorem~\ref{thm:uniform convergence for product measure}.
\end{proof}

Next, we show a slightly stronger statement, which is a type of uniform convergence bound when access to approximate distributions is given. More specifically, we argue that for any single-intersecting event, the difference in the probability of this event under two product distributions $\DD=\times_{i\in[\ell]}\DD_i$ and $\hat{\DD}=\times_{i\in[\ell]}\hat{\DD}_{i}$ is at most $2\xi\cdot \ell$, if $||\DD_i-\hat{\DD}_i||_K\leq \xi$ for all $i$. It is not hard to see that Lemma~\ref{lem:Kolmogorov stable for sc} and the DKW inequality imply Lemma~\ref{lem:uniform convergence for single-intersecting}.

\begin{lemma}\label{lem:Kolmogorov stable for sc}
For any integer $\ell$, let $\DD=\times_{i=1}^\ell \DD_i$ and $\hat{\DD}=\times_{i=1}^\ell \hat{\DD}_i$, where $\DD_i$ and $\hat{\DD}_i$ are both supported on $\mathbb{R}$ for any $i\in[\ell]$. If $||\DD_i-\hat{\DD}_i||_K\leq \xi$, $\left|\Pr_\DD[\EE]-\Pr_{\hat{\DD}}[\EE]\right|\leq 2\xi\cdot \ell$ for any single-intersecting event $\EE$.
\end{lemma}
\begin{proof}
	Let $\HH=\left\{\ind_{x\in \EE} :\EE\text{ is \emph{single-intersecting}}\right\}$. By the definition of single-intersecting events, $\HH_i$ is the set of the indicator functions of all intervals in $\mathbb{R}$ for any $i\in [\ell]$. Since $||\DD_i-\hat{\DD}_i||_K\leq \xi$, $$\sup_{g\in\HH_i}\left|\E_{x\sim \DD_i}[g(x)]-\E_{x\sim \hat{\DD}_i}[g(x)]\right|\leq 2\xi.$$ By Lemma~\ref{lem:approx product dist}, $$\sup_{f\in\HH} \left|\E_{\boldsymbol{x}\sim \DD}\left[f(\boldsymbol{x})\right]-\E_{\boldsymbol{x}\sim \hat{\DD}}\left[f(\boldsymbol{x})\right]\right|\leq 2\xi\cdot \ell.$$
\end{proof}

The following table (Table~\ref{tab:productVC}) summarizes some uniform convergence bounds implied by our results in this section.


\begin{table}[h]
	\centering
		\begin{tabular}{c || c | c}
			
			\hline\hline
			Hypotheses Class & \begin{minipage}[h]{4cm}\centering VC Bound\end{minipage} & \begin{minipage}[h]{7cm}\centering  Bounds from Theorem~\ref{thm:uniform convergence for product measure PARTITION} and Corollary~\ref{cor:VC for product measure}\end{minipage}\\
			\hline
			& &\\
			 axis-aligned rectangles in $\mathbb{R}^d$ &  $\tilde{O}(d /\epsilon^2)$ & $\tilde{O}(d /\epsilon^2)$	\\		
			polytopes with $k$ facets in $\mathbb{R}^d$ &  $\tilde{O}(d k /\epsilon^2)$ & $\tilde{O}(d \cdot \min\{d,k\} /\epsilon^2)$\\
			arbitrary convex sets in $\mathbb{R}^d$ & $\infty$ & $\tilde{O}(d^2 /\epsilon^2)$\\
			single-intersecting sets in $\mathbb{R}^d$ & $\infty$ & $\tilde{O}(d^2 /\epsilon^2)$
		\end{tabular}
	\caption{Number of samples required for $(\epsilon,\Theta(1))$-p.m. uniform convergence for different ${\cal H}$'s.}
	\label{tab:productVC}
\end{table}
\section{Constrained Additive Bidders: Uniform Convergence of the Revenue of Sequential Posted Price with Entry Fee Mechanisms}\label{sec:uniform convergence of SPEM}

We consider a specific class of mechanisms, namely Sequential Posted Price with Entry fee Mechanisms, a.k.a. \textbf{SPEM}s; see Algorithm~\ref{alg:spem-mech} for details. Cai and Zhao~\cite{CaiZ17} recently showed that if the bidders' valuations are XOS over independent items, the best SPEM achieves a constant fraction of the optimal revenue. \footnote{Cai and Zhao~\cite{CaiZ17} showed that the best ASPE or RSPM achieves a constant fraction of the optimal revenue. Clearly, any ASPE is also a SPEM, and any RSPM is simply a SPEM if we force the bidders to be unit-demand by only allowing each of them to purchase at most one item.} This section has two goals. The first is to show that, when bidders have constrained additive valuations over independent items, polynomially many samples suffice to guarantee uniform convergence for the revenue of all SPEMs, and hence our ability to select a near-optimal SPEM from polynomially many samples. This can be proven by applying our uniform convergence result for single-intersecting events (Lemma~\ref{lem:uniform convergence for single-intersecting}). The second (and stronger goal) is to show that we can learn a near-optimal SPEM under the max-min learning model (Theorem~\ref{thm:constrained additive Kolmogorov}). We show that the revenue of any SPEM changes no more than $O(\epsilon \cdot m^2\cdot n \cdot H)$ under the true and approximate valuation distributions (Theorem~\ref{thm:revenue stability under K-distance}), where $\epsilon$ is an upper bound of the Kolmogorov distance between the true and approximate distributions for every item marginal of every bidder. It is, of course, not hard to see that Theorem~\ref{thm:revenue stability under K-distance} and the DKW inequality imply uniform convergence of the revenue of all SPEMS. To establish Theorem~\ref{thm:revenue stability under K-distance}, we need to apply Lemma~\ref{lem:Kolmogorov stable for sc} instead of Lemma~\ref{lem:uniform convergence for single-intersecting}.


\begin{algorithm}[ht]
\begin{algorithmic}[1]
\REQUIRE A collection of prices $\{p_{ij}\}_{i\in[n], j\in[m]}$ and a collection of entry fee functions $\{\delta_i(\cdot)\}_{i\in[n]}$ where $\delta_i: 2^{[m]}\mapsto \mathbb{R}$ is bidder $i$'s entry fee function.
\STATE $S\gets [m]$
\FOR{$i \in [n]$}
	\STATE Show bidder $i$ {the} set of available items $S$ and set the entry fee for bidder $i$ to be ${\delta_i}(S)$.
    \IF{Bidder $i$ pays the entry fee ${\delta_i}(S)$}
        \STATE $i$ receives her favorite bundle $S_i^{*}$ and pays $\sum_{j\in S_i^{*}}p_{ij}$.
        \STATE $S\gets S\backslash S_i^{*}$.
    \ELSE
        \STATE $i$ gets nothing and pays $0$.
    \ENDIF
\ENDFOR
\end{algorithmic}
\caption{{\sf Sequential Posted Price with Entry Fee Mechanism (SPEM)}}
\label{alg:spem-mech}
\end{algorithm} 


 We first establish a technical lemma, which states that, for any set of items $S$, any set of prices $\{p_j\}_{j\in [m]}$ and entry fee $\delta$, the distribution over the {\em set of items} purchased by a constrained additive bidder whose valuation is drawn from  $\DD=\times_{j\in[m]} \DD_j$ and $\hat{\DD}=\times_{j\in[m]} \hat{\DD}_j$ has total variation distance at most $2m\xi$, if $||\DD_j-\hat{\DD}_j||_K\leq \xi$ for every item $j\in [m]$. This is quite surprising. Given that, for each set of items $S' \subseteq S$, the difference in the probability that the buyer will purchase this particular set $S'$ under $\DD$ and $\hat{\DD}$ could already be as large as {$\Theta(m\xi)$}, and the distribution has an exponentially large support size,  a trivial argument would give a bound of {$2^m\cdot \Theta(m\xi)$}. To overcome this analytical difficulty, we argue instead that for any collection of sets of items, the event that the buyer's favorite set lies in this collection is single-intersecting. Then our result follows from Lemma~\ref{lem:Kolmogorov stable for sc}. Notice that it is crucial that Lemma~\ref{lem:Kolmogorov stable for sc} holds for all events that are single-intersecting, as the event we consider here is clearly non-convex in general. 


\begin{lemma}\label{lem:stable demand set}
	For any set $S\subseteq [m]$, any prices $\{p_j\}_{j\in[m]}$ and entry fee $\delta(S)$, let $\LL$ and $\hat{\LL}$ be the distributions over the set of items purchased from $S$ by a constrained additive bidder under prices $\{p_j\}_{j\in[m]}$ and entry fee $\delta$ when her type is drawn from $\DD=\times_{j\in[m]} \DD_j$ and $\hat{\DD}=\times_{j\in[m]} \hat{\DD}_j$ respectively. If $||\DD_{j}-\hat{\DD}_{j}||_K\leq \xi$ for all item $j$, $||\LL-\hat{\LL}||_{TV}\leq 2m \xi.$
\end{lemma}
\begin{proof}
%Define function $f_{\KK,\boldsymbol{p}, S, \delta}:\mathbb{R}_{\geq 0}^m\mapsto \{0,1\}$ for any $\KK\subseteq 2^{[m]}$, $S\in[m]$, $\boldsymbol{p}\in \mathbb{R}_{\geq 0}^m$ and $\delta\in \mathbb{R}_{\geq 0}$. $f_{\KK,\boldsymbol{p}, S, \delta}(t)=1$ iff the constrained additive bidder with type $t$ will purchase a set of item $R\in \KK$ when the set of available items is $S$, the item prices are $\boldsymbol{p}=\{p_j\}_{j\in[m]}$ and the entry fee is $\delta$. Let $\HH$ be the hypothesis class that contains all $f_{\KK,\boldsymbol{p}, S, \delta}$'s. To prove the claim of this Lemma, it suffices to show that $$\sup_{f\in\HH} \left|\E_{\boldsymbol{x}\sim \DD}\left[f(\boldsymbol{x})\right]-\E_{\boldsymbol{x}\sim \hat{\DD}}\left[f(\boldsymbol{x})\right]\right|\leq 2m\cdot \epsilon.$$

For any set $R\subseteq S$, let $\EE_R$ be the event that the bidder purchases set $R$. Proving that the total variation distance between $\LL$ and $\hat{\LL}$ is no more than $2m\cdot \xi$ is the same as proving that for any $K\leq 2^{|S|}$, $\left|\ \Pr_{\DD}\left[t\in \bigcup_{\ell=1}^K \EE_{R_\ell}\right]-\Pr_{\hat{\DD}}\left[t\in \bigcup_{\ell=1}^K \EE_{R_\ell}\right]\right|\leq 2m\cdot \xi$ where $R_1,\cdots R_K$ are arbitrary distinct subsets of $S$. Since the dimension of the bidder's type space is $m$, if we can prove that $\bigcup_{\ell=1}^K \EE_{R_\ell}$ is always single-intersecting, our claim follows from Lemma~\ref{lem:Kolmogorov stable for sc}. 

%For every $j\in[m]$, let $\HH_j=\left\{g\ |\ \exists f\in \HH\ \exists\ a_{-j}\in \mathbb{R}^{m-1} \ \forall\ x\in\mathbb{R}, \ g(x) = f(x,a_{-j})\right\}$. Next, we will prove that all functions in $\HH_i$ are indicator functions of intervals in $\mathbb{R}$. From now on, we fix $\KK,\boldsymbol{p}, S$ and $\delta$. 
%For any set $R\subseteq S$, let $\EE_R$ be the event that the bidder purchases set $R$. %Proving that the total variation distance between $\LL$ and $\hat{\LL}$ is no more than $2m\cdot \xi$ is the same as proving that for any $K\leq 2^{|S|}$, $\left|\ \Pr_{D_i}\left[t_i\in \bigcup_{\ell=1}^K \EE_{R_\ell}\right]-\Pr_{\hat{D}_i}\left[t_i\in \bigcup_{\ell=1}^K \EE_{R_\ell}\right]\right|\leq 2m\cdot \xi$ where $R_1,\cdots R_K$ are arbitrary distinct subsets of $S$. Since the dimension of the bidder's type space is $m$, if we can prove that $\bigcup_{\ell=1}^K \EE_{R_\ell}$ is always single-intersecting, our claim follows from Lemma~\ref{lem:Kolmogorov stable for sc}. 
For any $j\in[m]$ and $a_{-j}\in \mathbb{R}_{\geq 0}^{{m}-1}$, let $L_j(a_{-j})=\left\{ (t_{j},a_{-j}) \ | t_{j}\in \mathbb{R}_{\geq 0} \right\}$. We claim that $L_j(a_{-j})$ intersects with at most two different $\EE_{U}$ and $\EE_{V}$ where $U$ and $V$ are subsets of $S$. 
	WLOG, we assume that  $(0,a_{-j})\in \EE_U$. 
	\begin{itemize}
		\item If $U=\emptyset$, that means the utility of the favorite set for type $(0,a_{-j})$ is smaller than the entry fee $\delta(S)$. If we increase the value of $t_{j}$, two cases could happen: (1) the utility of the favorite set is still lower than the entry fee; (2) the utility of the favorite set is higher than the entry fee. In case (1), $(t_{j},a_{-j})\in \EE_{\emptyset}$. In case (2), the bidder pays the entry fee and purchases her favorite set $V$. Then item $j$ must be in $V$, because otherwise the utility for set $V$ does not change from type $(0,a_{-j})$ to type $(t_{j},a_{-j})$. If we keep increasing $t_{j}$, bidder $i$'s favorite set remains to be $V$ and she keeps accepting the entry fee and purchasing $V$. Hence, $L_j(a_{-j})$ can intersect with at most one event $\EE_R$ where $R$ is non-empty.
		\item If $U\neq \emptyset$, that means $U$ is the favorite set of type $(0,a_{-j})$ and the utility for winning set $U$ is higher than the entry fee.  If we increase the value of $t_{j}$, two cases could happen: (1) $U$ remains the favorite set; (2) a different set $V$ becomes the new favorite set. In case (1), $(t_{j},a_{-j})\in \EE_U$. In case (2), item $j$ must lie in $V$ but not in $U$, otherwise how could $U$ be better than $V$ for type $(0,a_{-j})$ but worse for type $(t_{j},a_{-j})$.  If we keep increasing $t_{j}$, the bidder's favorite set remains to be $V$ and she keeps accepting the entry fee and purchasing $V$. Hence, $L_j(a_{-j})$ can intersect at most two different events.
	\end{itemize}
	
It is not hard to see that any event $\EE_R$ is an intersection of halfspaces, so the intersection of $L_j(a_{-j})$ with any event $\EE_R$ is an interval. {Also, notice that any type $t\in\mathbb{R}_{\geq 0}^{m}$ must lie in an event $\EE_R$ for some set $R\subseteq S$.} If $L_j(a_{-j})$ intersects with two different events $\EE_U$ and $\EE_V$, the two intersected intervals must lie back to back on $L_j(a_{-j})$. Otherwise, $L_j(a_{-j})$ intersects with at least three different events. Contradiction. Since $L_j(a_{-j})$ intersects with at most two different events, no matter which of these events are in $\{\EE_{R_\ell}\}_{\ell\in[K]}$, the intersection of $L_j(a_{-j})$ and $\bigcup_{\ell=1}^K \EE_{R_\ell}$ is either empty or an interval meaning $\bigcup_{\ell=1}^K \EE_{R_\ell}$ is single-intersecting. Now our claim simply follows from Lemma~\ref{lem:Kolmogorov stable for sc}.
\end{proof}

\begin{theorem}\label{thm:revenue stability under K-distance}
	Suppose all bidders' valuations are constrained additive over independent items. For any SPEM, let $\rev$ and $\widehat{\rev}$ be its expected revenue under $D$ and $\hat{D}$ respectively. If $D_{ij}$ and $\hat{D}_{ij}$ are both supported on $[0,H]$, and $||D_{ij}-\hat{D}_{ij}||_K\leq \xi$ for all $i\in[n]$ and $j\in[m]$, $$\left|\rev-\widehat{\rev}\right|\leq 2nm\xi\cdot \left(mH+\opt \right).$$	
	\end{theorem}
\begin{proof}
	We use a hybrid argument. Consider a sequence of distributions $\{D^{(i)}\}_{i\leq n}$, where $D^{(i)}=\hat{D}_1\times\cdots\times\hat{D}_i\times D_{i+1}\times\cdots\times D_{n},$ and $D^{(0)}=D$, $D^{(n)}=\hat{D}$.
	 We use $\rev^{(i)}$ to denote the expected revenue of the SPEM under $D^{(i)}$. To prove our claim, it suffices to argue that $\left|\rev^{(i-1)} -\rev^{(i)}\right|\leq 2\xi m\cdot \left(m\cdot H+\opt\right).$
	  We denote by $\SS_k$ and $\SS'_k$ the random set of items that remain available after visiting the first $k$ bidders under $D^{(i-1)}$ and $D^{(i)}$. Clearly, for $k\leq i-1$, $||\SS_k-\SS'_k||_{TV}=0$, so the expected revenue collected from the first $i-1$ bidders under $D^{(i-1)}$ and $D^{(i)}$ is the same. According to Lemma~\ref{lem:stable demand set}, $||\SS_i-\SS'_i||_{TV}\leq 2m\cdot \xi$. The total amount of money bidder $i$ spends can never be higher than her value for receiving all the items which is at most $m\cdot H$. So the difference in the expected revenue collected from bidder $i$ under  $D^{(i-1)}$ and $D^{(i)}$ is at most $2\xi\cdot m^2H$. Suppose $R$ is the set of remaining items after visiting the first $i$ bidders, then the expected revenue collected from the last $n-i$ bidders is the same under  $D^{(i-1)}$ and $D^{(i)}$, as these bidders have the same distributions. Moreover, this expected revenue is no more than $\opt$, since the optimal mechanism can simply just sell $R$ to the last $n-i$ bidders using the same prices and entry fee as in the SPEM we consider. Of course, for any fixed $R$, the probabilities that $\SS_i=R$ and $\SS'_i=R$ are different, but since for any $R$ the expected revenue from the last $n-i$ bidders is at most $\opt$, the difference in the expected revenue from the last $n-i$ bidders under  $D^{(i-1)}$ and $D^{(i)}$ is at most $||\SS_i-\SS'_i||_{TV} \cdot \opt \leq 2\xi\cdot m\opt$. Hence, the total difference between $\rev^{(i-1)}$ and  $\rev^{(i)}$ is at most $2\xi m\cdot \left(m H+\opt\right)$. 
	  %Furthermore, $\left|\rev-\widehat{\rev}(p,\delta)\right|\leq \sum_{i=1}^{n}\left|\rev^{(i-1)}(p,\delta) -\rev^{(i)}(p,\delta)\right|\leq 2 nm\xi \left(mH+\opt_{ASPE}\right).$
\end{proof}


\begin{theorem}\label{thm:constrained additive Kolmogorov}(Max-min Learning for Constrained Additive Bidders)
	When all bidders' valuations are constrained additive over independent items and for any bidder $i$ and any item $j$, $D_{ij}$ and $\hat{D}_{ij}$ are supported on $[0,H]$ and $||D_{ij}-\hat{D}_{ij}||_K\leq \epsilon$  for some $\epsilon=O(\frac{1}{nm})$, then with only access to $\hat{D}=\times_{i,j} \hat{D}_{ij}$, our algorithm can learn an RSPM or ASPE whose revenue is at least $\frac{\opt}{c}-{ \epsilon\cdot O(m^2n H)}$, where $\opt$ is the optimal revenue by any BIC mechanism under $D=\times_{i,j} D_{ij}$. $c>1$ is an absolute constant.
	\end{theorem}
	
Clearly, Theorem~\ref{thm:constrained additive Kolmogorov} also implies a polynomial sample complexity bound for learning an approximately revenue-optimal mechanism. A better sample complexity bound can be obtained directly, i.e.~without invoking the uniform convergence of the revenue of SPEMs, and  is stated as Theorem~\ref{thm:XOS sample} for the broader class of XOS valuations. Similarly, when bidders have simpler valuations, i.e., additive or unit-demand valuations, we can sharpen our results and achieve polynomial-time learnability of the approximately optimal mechanism using more specialized techniques. See Sections~\ref{sec:unit-demand} and~\ref{sec:additive} for details.
\subsection{Unit-demand Valuations: Polynomial-Time Learning} \label{sec:unit-demand}

In this section, we consider bidders with unit-demand valuations, sharpening our results to show how to learn approximately revenue-optimal mechanisms in polynomial time. It is shown in a sequence of works~\cite{ChawlaHMS10, KleinbergW12, CaiDW16} that there exists a sequential posted price mechanism (\textbf{SPM} see Algorithm~\ref{alg:seq-mech} for details) that achieves at least $\frac{1}{24}$ of the optimal revenue when bidders are unit-demand. We show that under all three distribution access models of Section~\ref{sec:prelim} there exists a polynomial-time algorithm that learns a sequential posted price mechanism whose revenue approximates the optimal revenue. We only sketch the proof here and postpone the details to Appendix~\ref{sec:unit-demand appx}.

\begin{theorem}\label{thm:unit-demand}
	When all bidders have unit-demand valuations and \begin{itemize}
		\item $D_{ij}$ is supported on $[0,H]$ for all bidder $i$ and item $j$, there exists a polynomial time algorithm that learns an SPM whose revenue is at least $\frac{\opt}{144}-\epsilon H$ with probability $1-\delta$ given  $O\left(\left(\frac{1}{\epsilon}\right)^2 \left(m^2 n\log \frac{n}{\epsilon} + \log \frac{1}{\delta}\right)\right)$ samples from $D$; or
		\item $D_{ij}$ is a regular distribution for all bidder $i$ and item $j$, there exists a polynomial time algorithm that learns a randomized SPM whose revenue is at least $\frac{\opt}{33}$ with probability $1-\delta$ given $O(\max\{m,n\}^2m^2 n^2\cdot \log \frac{nm}{\delta})$ samples from $D$; or
		\item we are only given access to $\hat{D}_{ij}$ where $||\hat{D}_{ij}-D_{ij}||_K\leq \epsilon$ for all bidder $i$ and item $j$, there is a polynomial time algorithm that constructs a randomized SPM whose revenue under $D$ is at least $\left(\frac{1}{4}-(n+m)\cdot \epsilon\right)\cdot\left(\frac{\opt}{8}-2\epsilon\cdot mnH\right)$\footnote{If we set $\epsilon$ to be $O(\frac{1}{m+n})$, this is the max-min guarantee we want to achieve.}.
	\end{itemize}
\end{theorem}

\noindent\textbf{Sample Access to Bounded Distributions:} the result is due to Morgenstern and Roughgarden~\cite{MorgensternR16}. 

\vspace{.05in}
\noindent\textbf{Direct Access to Approximate Distributions:} we first consider a convex program based on $D$ (see Figure~\ref{fig:CP unit demand}) which is usually referred to as the ex-ante relaxation of the revenue maximization problem~\cite{Alaei11}, and use its optimum as a proxy for $\opt$. Next, we consider a similar convex program based on $\hat{D}$ (see Figure~\ref{fig:CP unit demand approximate dist}) and show that the optima of the two convex programs are close to each other. Finally, we use techniques developed by Chawla et al.~\cite{ChawlaHMS10} to convert the optimal solution of the second convex program into a randomized SPM. We can show that the constructed randomized SPM achieves a revenue that approximates the optimum of the second convex program under $D$, which implies that the mechanism's revenue  also approximates the $\opt$. As we are given $\hat{D}$, we can solve the second convex program and convert its optimal solution into a randomized SPM in polynomial time. See Theorem~\ref{thm:UD Kolmogorov} in Appendix~\ref{sec:unit-demand Kolmogorov} for further details.

\vspace{.05in}
\noindent\textbf{Sample Access to Regular Distributions:} we use a similar convex program relaxation based approach as in the previous case. The main difference is that regular distributions could be unbounded and thus ruin the approximation guarantee. We show how to use the Extreme Value theorem in~\cite{CaiD11b} to truncate the distributions without hurting the revenue by much. See Theorem~\ref{thm:UD regular} in Appendix~\ref{sec:unit-demand regular} for further details.
\subsection{Additive Valuations: Polynomial-Time Learning}\label{sec:additive}
In this section, we consider bidders with additive valuations, again sharpening our results to show polynomial-time learnability. It is known that the better of the following two mechanisms achieves at least $\frac{1}{8}$ of the optimal revenue when all bidders have additive valuations~\cite{Yao15,CaiDW16}:

\vspace{.05in}	
\noindent\textbf{Selling Separately}: the mechanism sells each item separately using Myerson's optimal auction.

\vspace{.05in}	
\noindent \textbf{VCG with Entry Fee}: the mechanism solicits bids $\bold{b}=(b_1,\cdots, b_n)$ from the bidders, then offers each bidder $i$ the option to participate for an entry fee $e_i(b_{-i},D_i)$, which is the median of the random variable $\sum_{j\in[m]}(t_{ij}-\max_{k\neq i} b_{kj})^+$, where $t_i\sim D_i$\footnote{The entry fee function defined in~\cite{Yao15,CaiDW16} is slightly different. They showed that there exists an entry fee $X_i$, such that bidder $i$ accepts the entry fee with probability at least $1/2$. Then they argued that extracting $X_i/2$ as the revenue in the VCG with entry fee mechanism is enough to obtain a factor $8$ approximation. It is not hard to observe that our entry fee is accepted with probability exactly $1/2$, thus our entry fee is at least as large as $X_i$. So our mechanism also suffices to provide a factor $8$ approximation.}. This random variable is exactly bidder $i$'s utility when her type is $t_i$ and the other bidders' are $b_{-i}$. If bidder $i$ chooses to participate, she pays the entry fee and can take any item $j$ at price $\max_{k\neq i} b_{kj}$. Notice that the mechanism never over allocate any item, as only the highest bidder for an item can afford it. %Moreover, this mechanism is DSIC, because the entry fee and item prices for bidder $i$ only depend on the other bidders' bids and bidder $i$'s type distribution but not her bid.

Indeed, only counting the revenue from the entry fee in the second mechanism and the optimal revenue from selling the items separately already suffices to provide an $8$-approximation~\cite{Yao15, CaiDW16}. 

\begin{theorem}[\cite{CaiDW16}]\label{thm:UB additive}
	Let $\srev$ be the optimal revenue for selling the items separately and $\brev$ be the expected entry fee collected from the VCG with entry fee mechanism. Then $\opt\leq 6\cdot \srev+2\cdot\brev.$ 
\end{theorem}

Goldner and Karlin~\cite{GoldnerK16} showed that one sample suffices to learn a mechanism that achieves a constant fraction of the optimal revenue when $D_{ij}$ is regular for all $i\in[n]$ and $j\in[m]$. %In the rest of the section, we discuss
We show how to learn an approximately optimal mechanism in the other two models.
% (1) all $D_{ij}$ are supported on $[0,H]$, and (2) direct access to distributions $\hat{D}_{ij}$, where $||\hat{D}_{ij}-D_{ij}||_K\leq \epsilon$ for all $i\in[n]$ and $j\in[m]$.
\begin{theorem}\label{thm:additive}
	When the bidders have additive valuations and\begin{itemize}
		\item $D_{ij}$ is supported on $[0,H]$ for all bidder $i$ and item $j$, we can learn in polynomial time a mechanism whose expected revenue is at least $\frac{\opt}{32}-{\epsilon}\cdot H$ with probability $1-\delta$ given $O\left(\left(\frac{m}{\epsilon}\right)^2 \cdot\left(n\log n\log \frac{1}{\epsilon}+\log\frac{1}{\delta} \right)\right)$ samples from $D$; or
		\item  we are only given access to distributions $\hat{D}_{ij}$ where $||\hat{D}_{ij}-D_{ij}||_K\leq \epsilon$ for all bidder $i$ and item $j$, there is a polynomial time algorithm that constructs a mechanism whose expected revenue under $D$ is at least $\frac{\opt}{266}-96\epsilon\cdot mnH$ when $\epsilon\leq \frac{1}{16\max\{m,n\}}$.
	\end{itemize}
\end{theorem}

\noindent\textbf{Sample Access to Bounded Distributions:} Goldner and Karlin's proof~\cite{GoldnerK16} can be directly applied to the bounded distributions to show a single sample suffices to learn a mechanism whose expected revenue approximates the $\brev$. Then as $\srev$ is the revenue of $m$ separate single-item auctions, we can use the result in~\cite{MorgensternR16} to approximate it. See Theorem~\ref{thm:additive bounded} in Appendix~\ref{sec:additive bounded} for further details.

\vspace{.05in}
\noindent\textbf{Direct Access to Approximate Distributions:} for each single item, we apply Theorem~\ref{thm:unit-demand} to learn an individual auction, then run these learned auctions in parallel. Clearly, the combined auction's revenue approximates $\srev$. For $\brev$, we show that for every bidder $i$ and every bid profile $b_{-i}$ of the other bidders, the event that corresponds to bidder $i$ accepting any entry fee is \emph{single-intersecting} (see Definition~\ref{def:single-intersecting}). This implies that the probability for a bidder to accept an entry fee under $\hat{D}$ and $D$ is close (Lemma~\ref{lem:Kolmogorov stable for sc}). So we can essentially use the median of $\sum_{j\in[m]}(t_{ij}-\max_{k\neq i} b_{kj})^+$ with $t_i\sim\hat{D}_i$ as the entry fee. See Theorem~\ref{thm:additive Kolmogorov} in Appendix~\ref{sec:additive Kolmogorov} for further details.
\section{XOS Valuations} \label{sec:constrained additive}
In this section we go beyond constained additive valuations to show learnability of approximately revenue-optimal auctions from polynomially many samples. The better of the following two mechanisms is known to achieve a constant fraction of the optimal revenue, when bidders have valuations that are XOS over independent items~\cite{CaiZ17}.

%\begin{itemize}
\vspace{.05in}
\noindent\textbf{Rationed Sequential Posted Price Mechanism (RSPM)}: the mechanism is almost the same as SPM in Algorithm~\ref{alg:seq-mech}, except there is an extra constraint that every bidder can purchase at most one item.

\vspace{.05in}
\noindent\textbf{Anonymous Sequential Posted Price with Entry Fee Mechanism (ASPE)}: every buyer faces the same collection of item prices $\{p_j\}_{j\in[m]}$. The seller visits the bidders sequentially. For every bidder, the seller shows her all the available items (i.e. items that have not yet been purchased) and the associated price for each item, then asks her to pay a personalized entry fee which depends on her type distribution and the set of available items. If the bidder accepts the entry fee, she can proceed to purchase any available item at the given price; if she rejects the entry fee, she neither receives nor pays anything. See Algorithm~\ref{alg:aspe-mech} for details.
%\end{itemize}

\begin{theorem}\cite{CaiZ17}\label{thm:simple XOS}
	There exists a collection of prices $\{p^*_j\}_{j\in[m]}$, such that if we set the entry fee function $\delta^*_i(S)$ to be the median of bidder $i$'s utility for set $S$, either the ASPE$(p^*,\delta^*)$ or the best RSPM achieves at least a constant fraction of the optimal revenue when bidders' valuations are XOS over independent items. More formally, let $u^*_i(t_i, S)=\max_{S^*\subseteq S} v_i(t_i, S^*)-\sum_{j\in S^*} p^*_j$ be bidder $i$'s utility for the set of items $S$ when her type is $t_i$. We define $\delta^*_i(S)$ to be the median of the random variable $u^*_i(t_i,S)$ (with $t_i\sim D_i$) for any set $S\subseteq [m]$.  Moreover, the price $p^*_j$ for any item $j$ is no larger than $2G$, where $G = \max_{i,j} G_{ij}$ and $G_{ij}:= \sup_x \left\{ \Pr_{t_{ij}\sim D_{ij}}\left[V_i(t_{ij})\geq x\right]\geq \frac{1}{5\max\{m,n\}}\right\}$.
\end{theorem}

{%We provide an algorithm to learn both mechanisms when we are given direct access to an approximate distribution for constrained additive bidders in Section~\ref{sec:constrained additive kolomogorov}. 
Our goal next is to bound the sample complexity for learning a near-optimal RSPM and the ASPE described in Theorem~\ref{thm:simple XOS} under XOS valuations.} %See Appendix~\ref{sec:XOS bounded} for details.} 

We consider first the task of learning a near-optimal RSPM. In a RSPM, all bidders are restricted to be unit-demand, so the revenue of the best RSPM is upper bounded by the optimal revenue in the corresponding unit-demand setting. In Section~\ref{sec:unit-demand}, we have shown how to learn an approximately optimal mechanism for unit-demand bidders, and those algorithms can be used to approximate the best RSPM. 

So, for the rest of this section, it suffices to focus on learning an ASPE whose revenue approximates the revenue of the ASPE described in Theorem~\ref{thm:simple XOS}.  We will do this in Section~\ref{sec:XOS sample}. Before that,
%One might worry that the entry fee function could be complex and thus difficult to learn. It turns out if we choose the correct item prices $\{p_j\}_{j\in [m]}$, simply setting the entry fee to be the median of the bidder's utility for the available items suffices to obtain good revenue~\cite{CaiZ17}. More specifically, let $u_i(t_i, S)=\max_{S^*\subseteq S} v_i(t_i, S^*)-\sum_{j\in S^*} p_j$ be bidder $i$'s utility for the set of items $S$ and we define $\delta_i(S)$ to be the median of the random variable $u_i(t_i,S)$ while $t_i\sim D_i$ for any subset $S\subseteq [m]$.
we need a robust version of Theorem~\ref{thm:simple XOS}. In the next Lemma, we argue that if we use a collection of prices $\{p'_j\}_{j\in[m]}$ sufficiently close to $\{p^*_j\}_{j\in[m]}$ and entry fee $\delta'_i(S)$ sufficiently close to the median of the utility for every bidder $i$ and subset $S$, the better of the corresponding ASPE and the best RSPM still approximates the optimal revenue. We postpone the proof to Appendix~\ref{sec:appx XOS}.

\begin{lemma}\label{lem:approx ASPE}
	For any $\epsilon>0$ and $\mu\in[0,\frac{1}{4}]$, let $\{p'_j\}_{j\in[m]}$ be a collection of prices such that $|p'_j-p^*_j|\leq \epsilon$ for all $j\in[m]$, where $\{p^*_j\}_{j\in[m]}$ is the collection of prices in Theorem~\ref{thm:simple XOS}. Let $\delta'_i(S)$ be bidder $i$'s entry fee function such that $\Pr_{t_i\sim D_i}\left [u'_i(t_i,S)\geq \delta'_i(S)\right]\in [1/2-\mu,1/2+\mu]$ for any set $S\subseteq [m]$, where $u'_i(t_i,S) = \max_{S*\subseteq S} v_i(t_i,S^*)-\sum_{j\in S^*} p'_j$. Then, either the ASPE$(p',\delta')$ or the best RSPM achieves revenue at least $\frac{\opt}{\CC_1(\mu)}-\CC_2(\mu)\cdot (m+n)\cdot \epsilon$ when bidders' valuations are XOS over independent items. Both $\CC_1(\cdot)$ and $\CC_2(\cdot)$ are  monotonically increasing functions that only depend on $\mu$. 
\end{lemma}

\begin{definition}\label{def:eps mu ASPE}
	We say a collection of prices $\{p_j\}_{j\in[m]}$ is in the $B$-bounded $\epsilon$-net if $p_j$ is a multiple of $\epsilon$ and no larger than $B$ for any item $j$. For any collection of prices $\{p_j\}_{j\in[m]}$, we say the entry fee functions are $\mu$-balanced if for every bidder $i$ and every set $S\subseteq [m]$, her entry fee $\delta_i(S)$ satisfies  $\Pr_{t_i\sim D_i}[u_i(t_i,S)\geq \delta_i(S)]\in [1/2-\mu,1/2+\mu]$, where $u_i(t_i,S) = \max_{S*\subseteq S} v_i(t_i,S^*)-\sum_{j\in S^*} p_j$.
\end{definition}
%An easy corollary of Lemma~\ref{lem:approx ASPE} is that there exists an mechanism ASPE$(p,\delta)$ with $p$ lying in the $\epsilon$-net and that provides high revenue. More specifically,

\begin{corollary}\label{cor:discretization of prices}
	For  bidders with valuations that are XOS over independent items and any $\epsilon>0$, there exists a collection of prices $\{p_j\}_{j\in[m]}$ in the $2G$-bounded $\epsilon$-net such that for any $\mu$-balanced entry fee functions $\{\delta_i(\cdot)\}_{i\in[n]}$ with $\mu\in[0,\frac{1}{4}]$, either the ASPE$(p,\delta)$ or the best RSPM achieves revenue at least $\frac{\opt}{\CC_1(\mu)}-\CC_2(\mu)\cdot (m+n)\cdot \epsilon$. %where $\CC_1(\cdot)$ and $\CC_2(\cdot)$ are  monotonically increasing functions that only depend on $\mu$.
\end{corollary}

%In other words, if for every collection of prices $\{p_j\}_{j\in[m]}$ in the $\epsilon$-net we construct a set of $\mu$-balanced entry fee functions $\{\delta_i(\cdot)\}_{i\in[n]}$ for ${p}$, as guaranteed by Corollary~\ref{cor:discretization of prices}, the best mechanism among these ASPE$(p,\delta)$  has high revenue. 

\notshow{
\subsection{Constrained Additive Valuations: direct access to approximate distributions}\label{sec:constrained additive kolomogorov}

In this section, we consider the model where we only have access to an approximate distribution $\hat{D}$ and bidders have constrained additive valuations. %\cnote{TO TAKE OUT: As when the valuations are XOS over independent items, a bidders' type is not necessarily described as vectors in some Euclidean space. In particular, a bidder's type could be some abstract information specifying her preference\footnote{For example, the type distribution could be a categorical distribution about the color of the item.} about the items. For these distributions, our model does not apply, as it is not even clear how to define the Kolmogorov distance between any two such distributions.} 
%We restrict our attention to an important valuation class: constrained additive valuations. \yangnote{We will sketch the proof for the general XOS valuations in the next section.} %Since a constrained additive bidders' type is still a vector in $\mathbb{R}^m$, our model applies.
Our learning algorithm is a two-step procedure. In the first step, we argue that the approximate distribution $\hat{D}$ can be used to define a balanced entry fee function for each bidder under any prices. In particular, we propose to use the median of bidder $i$'s utility under $\hat{D}_i$ to compute the entry fees. The challenge is to show that, whenever $||D_{ij}-\hat{D}_{ij}||_{K}\leq \xi$ for every bidder $i$ and item $j$, the median of the utility of any bidder $i$ for any subset $S$ under any collection of prices $\{p_j\}_{j\in[m]}$ is not too different when the bidder's type is drawn from $D_i$ or $\hat{D}_i$; in particular, the entry fee functions thus defined from $\hat{D}$ are balanced. This is shown in Lemma~\ref{lem:Kolmogorov learn entry fee}. Given this lemma, we are able to construct an ASPE for every collection of prices $\{p_j\}_{j\in[m]}$ in the $\epsilon$-net. In the second step, we need to identify a mechanism  with high revenue among all ASPEs we constructed. If we have sample access to the  actual distribution, this is a simple task {as we can take a polynomial number of samples from $D$ and argue that with probability almost $1$, the empirical revenue of ASPE$(p,\delta)$ is close to its true expected revenue for every price vector $p$ in the $\epsilon$-net. So the ASPE with the highest empirical revenue also has high expected revenue.} %as we have shown in Lemma~\ref{lem:learn best ASPE}.
 But we only have access to an approximate distribution $\hat{D}$; can we argue that the expected revenue of any ASPE does not change much under $D$ or $\hat{D}$? Notice that this is certainly not true for arbitrary mechanisms, as it is easy to construct a DSIC single item auction with two bidders where the expected revenue under two distributions that are close in Kolmogorov distance is far away%\footnote{\todo{Write down this example later.}}
 . Surprisingly, we can argue that the expected revenue of any ASPE thus defined is within $\poly(n,m)\cdot H\cdot \xi$ when the bidders' types are drawn from $D$ or $\hat{D}$.

 To argue these results, we need to understand how the probability of events changes from distribution $D$ to $\hat{D}$. Clearly, if we consider arbitrary events, the difference in their probabilities under $D$ and $\hat{D}$ can be arbitrarily far. We need to identify structure in the events that are of interest to us, which allows us to argue that computations under $D$ and $\hat{D}$ are close. Since our distributions are supported on a subset of the Euclidean space, any event is simply a set in the Euclidean space. In Definition~\ref{def:single-intersecting}, we define a type of events called \emph{single-intersecting} events based on the geometric shape of the corresponding sets. For example, balls, rectangles and all convex sets are single-intersecting, but this definition includes some non-convex sets as well, for example, cross-shaped sets. It turns out that being able to include these non-convex sets is crucial for our result, as many events we care about are not convex but nonetheless are single-intersecting. Next we argue that for any single-intersecting event, the difference of the probability under $D$ and $\hat{D}$ only grows linearly in the dimension of the support space (Lemma~\ref{lem:Kolmogorov stable for sc}).
%This result is facilitated by combining an interesting property about Kolmogorov distance in product space and the geometry of bidders' behavior in ASPEs \footnote{\yangnote{I'm not sure what I was trying to say here... This is just a place holder to remind us write something about the proof.}}.
\notshow{\begin{definition}[Single-intersecting Events]\label{def:single-intersecting}
For any event $\EE$ in $\mathbb{R}^{\ell}$, $\EE$ is \textbf{single-intersecting} if the intersection of $\EE$ and any line that is parallel to one of the axes is an interval. More formally, for any $i\in[\ell]$ and any line $L_i=\left\{x \in \mathbb{R}^{\ell}\ |\ x_{-i}=a_{-i} \right \}$, where $a_{-i}$ is some $\ell-1$ dimensional vector in $\mathbb{R}^{\ell-1}$, the intersection of $L_i$ and $\EE$ is $\left\{x \in \mathbb{R}^{\ell}\ |\ x_{-i}=a_{-i}, x_i\in[\ubar{a},\bar{a}] \right\}$ for some real numbers $\ubar{a}$ and $\bar{a}$.
\end{definition}


\begin{lemma}\label{lem:Kolmogorov stable for sc}
For any integer $\ell$, let $\DD=\times_{i=1}^\ell \DD_i$ and $\hat{\DD}=\times_{i=1}^\ell \hat{\DD}_i$, where $\DD_i$ and $\hat{\DD}_i$ are both supported on $[0,H]$ for any $i\in[\ell]$. If $||\DD_i-\hat{\DD}_i||_K\leq \xi$, $\left|\Pr_\DD[\EE]-\Pr_{\hat{\DD}}[\EE]\right|\leq 2\xi\cdot \ell$ for any single-intersecting event $\EE$.
\end{lemma}
\begin{prevproof}{Lemma}{lem:Kolmogorov stable for sc} Let $\FF_i$ and $\hat{\FF}_i$ be the cdf for $\DD_i$ and $\hat{\DD}_i$ respectively. We will prove the statement using a hybrid argument. We create a sequence of product distributions $\{\DD^{(j)}\}_{j\leq \ell}$, where $\DD^{(j)}=\hat{\DD}_1\times\cdots\times\hat{\DD}_j\times \DD_{j+1}\times\cdots\times \DD_{\ell},$ and $\DD^{(0)}=\DD$, $\DD^{(\ell)}=\hat{\DD}$. To prove our claim, it suffices to show that for any integer $j\in[\ell]$, $\left|\Pr_{\DD^{(j-1)}}[\EE]-\Pr_{{\DD}^{(j)}}[\EE]\right|\leq 2\xi.$
	Let us first fix some notations. Let $\EE_{-j}=\left\{x_{-j}\ |\ \exists x_j\in \mathbb{R},\ (x_j,x_{-j})\in \EE\right\}$, $\EE_j(a_{-j})= \left\{x_{j} \in \mathbb{R} \ |\ (x_j,a_{-j})\in \EE\right\}$ for all $j\in[\ell]$. As $\EE$ is single-intersecting, $\EE_j(x_{-j})$ is an interval for all $j$ and $x_{-j}$. Moreover, since $||\DD_j-\hat{\DD}_j||_K\leq \xi$, we have $|\Pr_{\DD_j}[x_j\in \EE_j(x_{-j})]-\Pr_{\hat{\DD}_j}[x_j\in \EE_j(x_{-j})]|\leq 2\xi$ for all $j$ and $x_{-j}$. Next, we bound the difference for the probability of event $\EE$ under $\DD^{(j-1)}$ and $\DD^{(j)}$. 
	\begin{align*}
		&\left|\Pr_{\DD^{(j-1)}}[\EE]-\Pr_{{\DD}^{(j)}}[\EE]\right|\\
		=&\Big{|}\int_{\mathbb{R}^{\ell-1}}\ind{\left [x_{-j}\in \EE_{-j}\right]}\left(\int_{\mathbb{R}}\ind{\left[x_j\in \EE_j(x_{-j})\right]}d \FF_j(x_j)\right) d\hat{\FF}_1(x_1)\cdots d\hat{\FF}_{j-1}(x_{j-1}) d\FF_{j+1}(x_{j+1})\cdots d\FF_{\ell}(x_{\ell})\\
		 & - \int_{\mathbb{R}^{\ell-1}}\ind{\left [x_{-j}\in \EE_{-j}\right]}\left(\int_{\mathbb{R}}\ind{\left[x_j\in \EE_j(x_{-j})\right]}d \hat{\FF}_j(x_j)\right) d\hat{\FF}_1(x_1)\cdots d\hat{\FF}_{j-1}(x_{j-1}) d\FF_{j+1}(x_{j+1})\cdots d\FF_{\ell}(x_{\ell})\Big{|}\\
		 = & \left|\int_{\mathbb{R}^{\ell-1}}\ind{\left [x_{-j}\in \EE_{-j}\right]} \left(\Pr_{\DD_j}[x_j\in \EE_j(x_{-j})]-\Pr_{\hat{\DD}_j}[x_j\in \EE_j(x_{-j})]\right)
		  d\hat{\FF}_1(x_1)\cdots d\hat{\FF}_{j-1}(x_{j-1}) d\FF_{j+1}(x_{j+1})\cdots d\FF_{\ell}(x_{\ell})\right|\\
		  \leq & 2\xi\cdot\int_{\mathbb{R}^{\ell-1}}\ind{\left [x_{-j}\in \EE_{-j}\right]} d\hat{\FF}_1(x_1)\cdots d\hat{\FF}_{j-1}(x_{j-1}) d\FF_{j+1}(x_{j+1})\cdots d\FF_{\ell}(x_{\ell}) = 2\xi
	\end{align*}
	\end{prevproof}}
	%The second last inequality is because $\EE$ is single-intersecting so $\EE_j(x_{-j})$ is an interval for all $j$ and $x_{-j}$, also because $||\DD_j-\hat{\DD}_j||_K\leq \xi$, so $$
	It is quite easy to see that the single-intersecting condition can be relaxed to $k$-intersecting where the intersection of the event $\EE$ with any line parallel to an axis is the union of at most $k$ intervals. Indeed, using a proof almost identical to the proof of Lemma~\ref{lem:Kolmogorov stable for sc}, we can argue that the difference in the probability of an event $\EE$ under $\DD$ and $\hat{\DD}$ only grows linearly in the dimension of the support space and $k$. Next, we formally state the first step of our learning algorithm. The proof can be found in Appendix~\ref{sec:appx constrained additive}.
\begin{lemma}\label{lem:Kolmogorov learn entry fee}
Suppose $||D_{ij}-\hat{D}_{ij}||_K\leq \xi$ for any bidder $i$ and any item $j$. For any collection of prices $\{p_j\}_{j\in[m]}$, let $u^{(p)}_i(t_i,S)=\max_{S*\subseteq S} v_i(t_i,S^*)-\sum_{j\in S^*} p_j$. Define the entry fee $\delta_i^{(p)}(S)$
 of bidder $i$ for set $S$ under $\{p_j\}_{j\in[m]}$ to be the median of $u^{(p)}_i(t_i,S)$ when bidder $i$'s type $t_i$ is drawn from $\hat{D}_i$. Then $\left\{\delta_i^{(p)}(\cdot)\right\}_{i\in[n]}$ is a collection of $2m\xi$-balanced entry fee functions for any collection of prices $\{p_j\}_{j\in[m]}$.\end{lemma}

So far, for any collection of prices $\{p_j\}_{j\in[m]}$, we have defined (using $\hat{D}$) a collection of $2m\xi$-balanced entry fee functions $\{\delta_i^{(p)}(\cdot)\}_{i\in[n]}$. Next, we need to identify prices $\{p_j\}_{j\in[m]}$ such that the corresponding ASPE mechanism achieves high revenue under the actual distribution $D$. Our goal is to show that the expected revenue of simultaneously all ASPE mechanisms, defined for all prices using $\hat{D}$ as above, is not much different under $D$ and $\hat{D}$. We first establish a technical lemma, which states that, for any set of available items $S$ and entry fee $\delta_i(S)$, the distribution over the {\em set of items} purchased by bidder $i$ under $D_i$ and $\hat{D}_i$ has total variation distance at most $2m\xi$. This is quite surprising. Given that, for each set of items $S' \subseteq S$, the difference in the probability that the buyer will purchase this particular set $S'$ under $D_i$ and $\hat{D}_i$ could already be as large as {$\Theta(m\xi)$}, and the distribution has an exponentially large support size,  a trivial argument would give a bound of {$2^m\cdot \Theta(m\xi)$}. To overcome this analytical difficulty, we argue instead that for any collection of sets of items, the event that the buyer's favorite set lies in this collection is single-intersecting. Then our result follows from Lemma~\ref{lem:Kolmogorov stable for sc}. Notice that it is crucial that Lemma~\ref{lem:Kolmogorov stable for sc} holds for all events that are single-intersecting, as the event we consider here is clearly non-convex in general. 

\notshow{
\begin{lemma}\label{lem:stable favorite set}
	For any bidder $i$, any set $S\subseteq [m]$, any prices $\{p_j\}_{j\in[m]}$ and entry fee $\delta_i(S)$, let $\LL$ and $\hat{\LL}$ be the distributions over the set of items purchased from $S$ by bidder $i$ under prices $\{p_j\}_{j\in[m]}$ and entry fee $\delta_i(S)$ when her type is drawn from $D_i$ and $\hat{D}_i$ respectively. If $||D_{ij}-\hat{D}_{ij}||_K\leq \xi$ for all item $j$, $||\LL-\hat{\LL}||_{TV}\leq 2m \xi.$
\end{lemma}
\begin{proof}
	For any set $R\subseteq S$, let $\EE_R$ be the event that bidder $i$ purchases set $R$. Proving that the total variation distance between $\LL$ and $\hat{\LL}$ is no more than $2m\cdot \xi$ is the same as proving that for any $K\leq 2^{|S|}$, $\left|\ \Pr_{D_i}\left[t_i\in \bigcup_{\ell=1}^K \EE_{R_\ell}\right]-\Pr_{\hat{D}_i}\left[t_i\in \bigcup_{\ell=1}^K \EE_{R_\ell}\right]\right|\leq 2m\cdot \xi$ where $R_1,\cdots R_K$ are arbitrary distinct subsets of $S$. Since the dimension of the bidder's type space is $m$, if we can prove that $\bigcup_{\ell=1}^K \EE_{R_\ell}$ is always single-intersecting, our claim follows from Lemma~\ref{lem:Kolmogorov stable for sc}. 
	
	For any $j\in[m]$ and $a_{i,-j}\in [0,H]^{{m}-1}$, let $L_j(a_{i,-j})=\left\{ (t_{ij},a_{i,-j}) \ | t_{ij}\in [0,H] \right\}$. We claim that $L_j(a_{i,-j})$ intersects with at most two different $\EE_{U}$ and $\EE_{V}$ where $U$ and $V$ are subsets of $S$. 
	WLOG, we assume that  $(0,a_{i,-j})\in \EE_U$. 
	\begin{itemize}
		\item If $U=\emptyset$, that means the utility of the favorite set for type $(0,a_{i,-j})$ is smaller than the entry fee $\delta_i(S)$. If we increase the value of $t_{ij}$, two cases could happen: (1) the utility of the favorite set is still lower than the entry fee; (2) the utility of the favorite set is higher than the entry fee. In case (1), $(t_{ij},a_{i,-j})\in \EE_{\emptyset}$. In case (2), bidder $i$ pays the entry fee and purchases her favorite set $V$. Then item $j$ must be in $V$, because otherwise the utility for set $V$ does not change from type $(0,a_{i,-j})$ to type $(t_{ij},a_{i,-j})$. If we keep increasing $t_{ij}$, bidder $i$'s favorite set remains $V$ and she keeps paying the entry fee and purchasing $V$. Hence, $L_j(a_{i,-j})$ can intersect with at most one event $\EE_R$ where $R$ is non-empty.
		\item If $U\neq \emptyset$, that means $U$ is the favorite set of type $(0,a_{i,-j})$ and the utility for winning set $U$ is higher than the entry fee.  If we increase the value of $t_{ij}$, two cases could happen: (1) $U$ remains the favorite set; (2) a different set $V$ becomes the new favorite set. In case (1), $(t_{ij},a_{i,-j})\in \EE_U$. In case (2), item $j$ must be in $V$ and not in $U$, otherwise how could $U$ be better than $V$ for type $(0,a_{i,-j})$ but worse for type $(t_{ij},a_{i,-j})$.  If we keep increasing $t_{ij}$, bidder $i$'s favorite set remains to $V$ and she keeps paying the entry fee and purchasing $V$. Hence, $L_j(a_{i,-j})$ can intersect at most two different events.
	\end{itemize}
	
It is not hard to see that any event $\EE_R$ is an intersection of halfspaces, so the intersection of $L_j(a_{i,-j})$ with any event $\EE_R$ is an interval. {Also, notice that any type $t_i$ must lie in an event $\EE_R$ for some set $R\subseteq S$.} If $L_j(a_{i,-j})$ intersects with two different events $\EE_U$ and $\EE_V$, the two intersected intervals must lie back to back on $L_j(a_{i,-j})$. Otherwise, $L_j(a_{i,-j})$ intersects with at least three different events. Contradiction. Since $L_j(a_{i,-j})$ intersects with at most two different events, no matter which of these events are in $\{\EE_{R_\ell}\}_{\ell\in[K]}$, the intersection of $L_j(a_{i,-j})$ and $\bigcup_{\ell=1}^K \EE_{R_\ell}$ is either empty or an interval,%\footnote{\cnote{Should we add some extra explanation here, that if it intersects two events these must be back to back, so the intersection is not going to be the union of two disjoint intervals?}}
  which means $\bigcup_{\ell=1}^K \EE_{R_\ell}$ is single-intersecting. Now our claim simply follows from Lemma~\ref{lem:Kolmogorov stable for sc}.
\end{proof}
}

With Lemma~\ref{lem:stable favorite set}, we are ready to show that the revenue of any ASPE does not change much under $D$ and $\hat{D}$.\begin{lemma}\label{lem:difference in revenue Kolmogorov}
	For any ASPE$(p,\delta)$, let $\rev(p,\delta)$ and $\widehat{\rev}(p,\delta)$ be its expected revenue under $D$ and $\hat{D}$ respectively. If $||D_{ij}-\hat{D}_{ij}||_K\leq \xi$ for all $i\in[n]$ and $j\in[m]$, then $|\rev(p,\delta)-\widehat{\rev}(p,\delta)|\leq 2nm\xi\cdot \left(mH+\opt_{ASPE}\right)$,
	{where $\opt_{ASPE}$ is the optimal revenue obtainable by an ASPE mechanism.}
	\end{lemma}
\begin{prevproof}{Lemma}{lem:difference in revenue Kolmogorov}
	We use a hybrid argument. Consider a sequence of distributions $\{D^{(i)}\}_{i\leq n}$, where $D^{(i)}=\hat{D}_1\times\cdots\times\hat{D}_i\times D_{i+1}\times\cdots\times D_{n},$ and $D^{(0)}=D$, $D^{(n)}=\hat{D}$.
	 We use $\rev^{(i)}(p,\delta)$ to denote the expected revenue of ASPE$(p,\delta)$ under $D^{(i)}$. To prove our claim, it suffices to argue that $\left|\rev^{(i-1)}(p,\delta) -\rev^{(i)}(p,\delta)\right|\leq 2\xi m\cdot \left(m\cdot H+\opt\right).$
	  We denote by $\SS_k$ and $\SS'_k$ the random set of items that remain available after visiting the first $k$ bidders under $D^{(i-1)}$ and $D^{(i)}$. Clearly, for $k\leq i-1$, $||\SS_k-\SS'_k||_{TV}=0$, so the expected revenue collected from the first $i-1$ bidders under $D^{(i-1)}$ and $D^{(i)}$ is the same. According to Lemma~\ref{lem:stable favorite set}, $||\SS_i-\SS'_i||_{TV}\leq 2m\cdot \xi$. The total amount of money bidder $i$ spends can never be higher than her value for receiving all the items which is at most $m\cdot H$. So the difference in the expected revenue collected from bidder $i$ under  $D^{(i-1)}$ and $D^{(i)}$ is at most $2\xi\cdot m^2H$. Suppose $R$ is the set of remaining items after visiting the first $i$ bidders, then the expected revenue collected from the last $n-i$ bidders is the same under  $D^{(i-1)}$ and $D^{(i)}$, as these bidders have the same distributions. Moreover, this expected revenue is no more than $\opt_{ASPE}$, since the optimal ASPE can simply just sell $R$ to the last $n-i$ bidders using the same prices and entry fee as in ASPE$(p,\delta)$. Of course, for any fixed $R$, the probabilities that $\SS_i=R$ and $\SS'_i=R$ are different, but since for any $R$ the expected revenue from the last $n-i$ bidders is at most $\opt_{ASPE}$, the difference in the expected revenue from the last $n-i$ bidders under  $D^{(i-1)}$ and $D^{(i)}$ is at most $||\SS_i-\SS'_i||_{TV} \cdot \opt \leq 2\xi\cdot m\opt_{ASPE}$. Hence, the total difference between $\rev^{(i-1)}(p,\delta)$ and  $\rev^{(i)}(p,\delta)$ is at most $2\xi m\cdot \left(m H+\opt_{ASPE}\right)$. 
	  Furthermore, $\left|\rev(p,\delta)-\widehat{\rev}(p,\delta)\right|\leq \sum_{i=1}^{n}\left|\rev^{(i-1)}(p,\delta) -\rev^{(i)}(p,\delta)\right|\leq 2 nm\xi \left(mH+\opt_{ASPE}\right).$
\end{prevproof}

Using  Lemma~\ref{lem:difference in revenue Kolmogorov}, we can prove the main Theorem of this section. The proof is postponed to Appendix~\ref{sec:appx constrained additive}.

\begin{theorem}\label{thm:constrained additive Kolmogorov}
	When all bidders' valuations are drawn from distributions that are constrained additive, and for any bidder $i$ and any item $j$, $D_{ij}$ and $\hat{D}_{ij}$ are supported on $[0,H]$ and $||D_{ij}-\hat{D}_{ij}||_K\leq \xi$  for some $\xi=O(\frac{1}{nm})$, then with only access to $\hat{D}=\times_{i,j} \hat{D}_{ij}$, our algorithm can learn an RSPM and an ASPE such that the better of the two mechanisms has revenue at least $\frac{\opt}{c}-{\xi\cdot O(m^2n H)}$, where $\opt$ is the optimal revenue by any BIC mechanism under $D=\times_{i,j} D_{ij}$. $c>1$ is an absolute constant.%\footnote{\cnote{I'm confused why I don't see an $n^2$ in the error term (which would come from the RSPM error term), and also why there is no constant factor in front of the error term.} \yangnote{I added the constant factor in front of the error term, but I'm not sure why there will be an $n^2$ error term from RSPM is $\xi = O(1/nm)$.}} %\yangnote{With the same number of samples, we can learn in polynomial time with probability $1-\delta$ an SPM whose revenue is at least $\frac{\opt}{144}-\epsilon H$.}
\end{theorem}

}

\subsection{XOS Valuations: sample access to bounded and regular distributions}\label{sec:XOS sample}
In this section, we consider how to learn an ASPE with high revenue given sample access to $D$. Our learning algorithm is a two-step procedure. In the first step, we take a few samples from $D$ and use these samples to set the entry fee for every collection of prices $\{p_j\}_{j\in[m]}$ in the $\epsilon$-net. More specifically, to decide $\delta_i(S)$ we compute the utility of bidder $i$ for set $S$ under $\{p_j\}_{j\in[m]}$ over all the samples and take the empirical median among all these utilities to be $\delta_i(S)$. With a polynomial number of samples, we can guarantee that for any $\{p_j\}_{j\in[m]}$ in the $\epsilon$-net the computed entry fee functions $\{\delta_i(\cdot)\}_{i\in[n]}$ are $\mu$-balanced. Now, we have created an ASPE for every $\{p_j\}_{j\in[m]}$ in the $\epsilon$-net. In the second step, we take some fresh samples from $D$ and use them to estimate the revenue for each of the ASPEs we created in the first step,  then pick the one that has the highest empirical revenue. It is not hard to argue that with a polynomial number of samples the mechanism we pick has high revenue with probability almost $1$. Combining our algorithm with Theorem~\ref{thm:unit-demand}, we obtain the following theorem.
\begin{theorem}\label{thm:XOS sample}
	When all bidders' valuations are XOS over independent items and
	\begin{itemize}
		\item the random variable $V_i(t_{ij})$ is supported on $[0,H]$ for each bidder $i$ and item $j$, we can learn an RSPM and an ASPE such that with probability at least $1-\delta$ the better of the two mechanisms has revenue at least $\frac{\opt}{c_1}-\xi\cdot H$ for some absolute constant $c_1>1$ given $O\left((\frac{mn}{\xi})^2 \cdot (m\cdot\log \frac{m+n}{\xi} + \log \frac{1}{\delta})\right)$ samples from $D$; 
		\item  the random variable $V_i(t_{ij})$ is regular for each bidder $i$ and item $j$, we can learn an RSPM and an ASPE such that with probability at least $1-\delta$ the better of the two mechanisms has revenue at least $\frac{\opt}{c_2}$ for some absolute constant $c_2>1$ given $O\left(\max\{m,n\}^2m^2n^2 \left(m\log ({m+n}) + \log \frac{1}{\delta}\right)\right)$  samples from $D$.
	\end{itemize}
\end{theorem}

 The bounded case is proved as Theorem~\ref{thm:XOS bounded} in Appendix~\ref{sec:XOS bounded}.  The regular case is proved as Theorem~\ref{thm:XOS regular} in Appendix~\ref{sec:XOS bounded}.

\section{Subadditive Agents}\label{subadditive}
In this section we present a reduction from subadditive agents to XOS agents. More precisely, we show for every subadditive set function $f(.)$, there exists an XOS function $g(.)$, where $g$ is dominated by $f$ but the maxmin value of $g$ is within a logarithmic factor of the maxmin value of $f$. We begin by an observation. Suppose we are given a subadditive function $f$ on set $\domp(f)$, and we wish to approximate $f$ with an additive function $g$ which is dominated by $f$. In other words, we wish to find an additive function $g$ such that 
$$\forall S \subseteq \domp(f) \hspace{1cm} g(S) \leq f(S)$$
and $g(\domp(f))$ is maximized. One way to formulate $g$ is via a linear program. Suppose $\domp(f)=\{\ite_1,\ite_2,\ldots,\ite_m\}$ and let $g_1, g_2, \ldots, g_m$ be $m$ variables that describe $g$ in the following way:
$$\forall S \subseteq \domp(f) \hspace{1cm} g(S) = \sum_{\ite_i \in S} g_i.$$
Based on this formulation, we can find the optimal additive function $g$ by LP \ref{lp1}.
\begin{alignat}{3}\label{lp1}
\text{maximize: }& \hspace{0.5cm} &  \sum_{\ite_i \in \domp(f)}g_i & &\\
\text{subject to: }& & \sum_{\ite_i \in S}g_i \leq f(S) & \hspace{1cm}&\forall S \subseteq \domp(f)\nonumber\\
& & g_i \geq 0 & &\forall \ite_i \in \domp(f)\nonumber
\end{alignat}
We show the objective function of LP \ref{lp1} is lower bounded by $f(\domp(f)) / \log m$. The basic idea is to first write the dual program and then based on a probabilistic method, lower bound the optimal value of the dual program by $f(\domp(f))/ \log m$. 
\begin{lemma}\label{jj1}
	The optimal solution of LP \ref{lp1} is at least $f(\domp(f))/ \log m$.
\end{lemma}
\begin{proof}
To prove the lemma, we write the dual of LP \ref{lp1} as follows:
\begin{alignat}{3}\label{lp2}
\text{minimize: }& \hspace{0.5cm}& \sum_{S \subseteq \domp(f)} \alpha_S f(S)   & &\\
\text{subject to: }& & \sum_{S \ni \ite_i} \alpha_S \geq 1 & \hspace{1cm}&\forall \ite_i \in \domp(f)\nonumber\\
& & \alpha_S \geq 0 & &\forall S \subseteq \domp(f)\nonumber
\end{alignat}
By the strong duality theorem, the optimal solutions of LP \ref{lp1} and LP \ref{lp2} are equal~\cite{bachem1992linear}. Next, based on the optimal solution of LP \ref{lp2}, we define a randomized procedure to draw a set of elements: We start with an empty set $S^*$ and for every set $S \subseteq \domp(f)$ we add \textit{all} elements of $S$ to $S^*$ with probability $\alpha_S$. Since $f$ is subadditive, the marginal increase of $f(S^*)$ by adding elements of a set $S$ to $S^*$ is bounded by $f(S)$ and thus the expected value of $f(S^*)$ is bounded by the objective of LP \ref{lp2}. In other words:
\begin{equation}\label{jef0}
\mathbb{E}[f(S^*)] \leq \sum_{S \subseteq \domp(f)} \alpha_S f(S)
\end{equation}
Remark that we repeat this procedure for all subsets of $\domp(S)$ independently and thus for every $\ite_i \in \domp(f)$, $\sum_{S \ni \ite_i} \alpha_S \geq 1$ holds we have
\begin{equation}\label{jef1}
\mathsf{PR}[\ite_i \in S^*] \geq 1-1/e \simeq 0.632121 > 1/2
\end{equation}
for every element $\ite_i \in \domp(s)$. Now, with the same procedure, we draw $\lceil \log m \rceil + 2$ sets $S^*_1, S^*_2, \ldots, S^*_{\lceil \log m \rceil + 2}$ \textit{independently}. We define $\hat{S} = \bigcup S^*_i$. By Inequality \eqref{jef1} and the union bound we show
\begin{equation*}
\begin{split}
\mathsf{PR}[\hat{S} = \domp(f)] & \geq 1- \sum_{\ite_i \in \domp(i)} \mathsf{PR}[\ite_i \notin \hat{S}]\\
& = 1- \sum_{\ite_i \in \domp(i)} \mathsf{PR}[\ite_i \notin S^*_1 \text{ and } \ite_i \notin S^*_1 \text{ and } \ldots \text{ and } \ite_i \notin S^*_{\lceil \log m \rceil + 2}]\\
& = 1- \sum_{\ite_i \in \domp(i)} \prod_{j=1}^{\lceil \log m \rceil + 2} \mathsf{PR}[\ite_i \notin S^*_j]\\
& \geq 1- \sum_{\ite_i \in \domp(i)} \prod_{j=1}^{\lceil \log m \rceil + 2} 1/2\\
& = 1- \sum_{\ite_i \in \domp(i)} \prod_{j=1}^{\lceil \log m \rceil + 2} \mathsf{PR}[\ite_i \notin S^*_j]\\
& \geq 1- \sum_{\ite_i \in \domp(i)} 1/4m\\
& = 1- 1/4\\
& = 3/4\\
\end{split}
\end{equation*}
and thus $\mathbb{E}[f(\hat{S})] \geq 3/4 f(\domp(f))$. On the other hand, by the linearity of expectation and the fact that $f$ is subadditive we have:
\begin{equation*}
\begin{split}
\mathbb{E}[f(\hat{S})] &= \mathbb{E}[f(\bigcup S^*_i)]\\
& \leq \mathbb{E}[\sum f(S^*_i)]\\
& \leq (\lceil \log m \rceil + 2) (\sum_{S \subseteq \domp(f)} \alpha_S f(S))
\end{split}
\end{equation*}
Therefore $\sum_{S \subseteq \domp(f)} \alpha_S f(S) \geq 3/4 f(\domp(f)) / (\lceil \log m \rceil + 2)$, which means $$\sum_{S \subseteq \domp(f)} \alpha_S f(S) \geq f(\domp(f)) / (2\lceil \log m \rceil)$$ for big enough $m$. This shows the optimal solution of LP \ref{lp1} is lower bounded by $f(\domp(f)) / (2\lceil \log m \rceil)$ and the proof is complete.
\end{proof}

In what follows, based on Lemma \ref{jj1}, we provide a reduction from subadditive agents to XOS agents. An immediate corollary of Lemma \ref{jj1} is the following:
\begin{corollary}[of Lemma \ref{jj1}]\label{kk}
For any subadditive function $f$ and integer number $n$, there exists an XOS function $g$ such that
$$g(S) \leq f(S) \qquad \forall S \subseteq \domp(f)$$
and 
$$\MMS_g^n \geq \MMS_f^n/2\lceil \log n \rceil.$$
\end{corollary} 
\begin{proof}
	By definition, we can divide the items into $n$ disjoint sets such that the value of $f$ for every set is at least $\MMS_f^n$. Now, based on Lemma \ref{jj1}, we approximate $f$ for each set with an additive function $g_i$ wile losing a factor of at most $\lceil 2 |\log \domp(f)|\rceil$ and finally we set $g = \max g_i$. Based on Lemma \ref{jj1}, both conditions of this lemma are satisfied by $g$.
\end{proof}

Based on Theorem \ref{xosproof} and Lemma \ref{kk} one can show that a $1/10\lceil \log m \rceil$-$\MMS$ allocation is always possible for subadditive agents.
\begin{theorem}\label{subadditiveproof}
	The fair allocation problem with subadditive agents admits a $1/10\lceil \log m \rceil$-$\MMS$ allocation.
\end{theorem}
\appendix
\onecolumn


% \tableofcontents{}

% \newpage

\section*{Supplementary Material}
\addcontentsline{toc}{section}{Supplementary Material}


Throughout this discussion, 
we will make frequently use 
of the following standard results
concerning the exponential concentration 
of random variables:

\begin{lemma}[Hoeffding's inequality for independent RVs~\citep{hoeffding1994probability}] Let $Z_1, Z_2, \ldots, Z_n$ be independent bounded random variables with $Z_i \in [a,b]$ for all $i$, then 
    \begin{align*}
        \prob\left( \frac{1}{n} \sum_{i=1}^n (Z_i - \Expo{Z_i}) \ge t \right) \le \exp{\left( -\frac{2nt^2}{(b-a)^2} \right) }
    \end{align*} 
    and 
    \begin{align*}
        \prob\left( \frac{1}{n} \sum_{i=1}^n (Z_i - \Expo{Z_i}) \le -t \right) \le \exp{\left( -\frac{2nt^2}{(b-a)^2} \right) }
    \end{align*} 
    for all $t \ge 0$. 
\end{lemma}

\begin{lemma}[Hoeffding's inequality for sampling with replacement~\citep{hoeffding1994probability}] \label{lem:hoeffding_sampling} Let $\calZ = (Z_1, Z_2, \ldots, Z_N)$ be a finite population of $N$ points with $Z_i \in [a.b]$ for all $i$. Let $X_1, X_2, \ldots X_n$ be a random sample drawn without replacement from $\calZ$. Then for all $t \ge 0$, we have 
    \begin{align*}
        \prob\left( \frac{1}{n} \sum_{i=1}^n (X_i - \mu ) \ge t \right) \le \exp{\left( -\frac{2nt^2}{(b-a)^2} \right) }
    \end{align*} 
    and 
    \begin{align*}
        \prob\left( \frac{1}{n} \sum_{i=1}^n (X_i - \mu ) \le -t \right) \le \exp{\left( -\frac{2nt^2}{(b-a)^2} \right) } \,,
    \end{align*} 
    where $\mu = \frac{1}{N} \sum_{i=1}^{N} Z_i$. 
\end{lemma}

We now discuss one condition that generalizes the exponential concentration to dependent random variables.
\begin{condition}[Bounded difference inequality] \label{cond:BDC} Let $\calZ$ be some set and $\phi: \calZ^n \to \Real$. We say that $\phi$ satisfies the bounded difference assumption if 
there exists $c_1, c_2, \ldots c_n \ge 0$ s.t. for all $i$, we have 
\begin{align*}
    \sup_{Z_1,Z_2, \ldots,Z_n, Z_i^\prime \in \calZ^{n+1} } \abs{\phi (Z_1, \ldots, Z_i, \ldots, Z_n ) - \phi (Z_1, \ldots, Z_i^\prime, \ldots, Z_n ) } \le c_i \,.
\end{align*} 
\end{condition}

\begin{lemma}[McDiarmid’s inequality~\citep{mcdiarmid1989}] \label{lem:McDiarmid} Let $Z_1, Z_2, \ldots, Z_n$ be independent random variables on set $\calZ$ and $\phi : \calZ^n \to \Real$ satisfy bounded difference inequality (\codref{cond:BDC}). Then for all $t>0$, we have 
    \begin{align*}
        \prob\left( \phi(Z_1, Z_2, \ldots, Z_n) - \Expo{\phi(Z_1, Z_2, \ldots, Z_n)} \ge t \right) \le \exp{\left( -\frac{2t^2}{\sum_{i=1}^n c_i^2} \right) } 
    \end{align*} 
    and 
    \begin{align*}
        \prob\left( \phi(Z_1, Z_2, \ldots, Z_n) - \Expo{\phi(Z_1, Z_2, \ldots, Z_n)} \le -t \right) \le \exp{\left( -\frac{2t^2}{\sum_{i=1}^n c_i^2} \right) } \,.
    \end{align*} 
\end{lemma}


\section{Proofs from \secref{sec:ERM_training}}\label{app:proof_erm}

\textbf{Additional notation {} {}} Let $m_1$ be the number of mislabeled points ($\wt S_M$) and $m_2$ be the number of correctly labeled points ($\wt S_C$). Note $m_1 + m_2 = m$. 


\subsection{Proof of \thmref{thm:error_ERM}}


\begin{proof}[Proof of \lemref{lem:fit_mislabeled}] 
    The main idea of our proof is to regard 
    the clean portion of the data 
    ($S \cup \wt S_C$) as fixed.   
    Then, there exists an (unknown) classifier $f^*$ 
    that minimizes the expected risk
    calculated on the (fixed) clean data
    and (random draws of) the mislabeled data $\wt S_M$. 
    % 
    % 
    Formally, 
    \begin{align}
    f^* \defeq \argmin_{f \in \calF} \error_{\widecheck {\calD}} (f) \,, \label{eq:modified_ERM}
    \end{align}
    where $$\widecheck \calD = \frac{n}{m+n} \calS + \frac{m_2}{m+n} \wt \calS_C  + \frac{m_1}{m+n}\calDm \,.$$ 
    Note here that $\widecheck \calD$ is a combination 
    of the \emph{empirical distribution} 
    over correctly labeled data $S \cup \wt S_C$
    and the (population) distribution 
    over mislabeled data $\calDm$.
    Recall that 
    \begin{align}
    \wh f \defeq \argmin_{f \in \calF} \error_{\calS \cup \wt S} (f) \,. \label{eq:orig_ERM}
    \end{align}
    % 
    % 
    Since, $\widehat f$ minimizes 0-1 error 
    on $S \cup \wt S$, using ERM optimality on \eqref{eq:orig_ERM},  
    we have 
    \begin{align}
        \error_{\calS \cup \wt \calS}(\widehat f) \le \error_{
            \calS \cup \wt \calS}(f^*) \,.    \label{eq:step1}
    \end{align}
    Moreover, since $f^*$ is independent of $\wt S_M$, using Hoeffding's bound,
    % \footnote{For a fully rigorous argument,
    % refer to the complete proof in App.~\ref{app:proof_erm}.} 
    we have with probability at least $1-\delta$ that
    \begin{align}
      \error_{\wt \calS_M}(f^*) \le \error_{ \calDm}(f^*) +  \sqrt{\frac{\log(1/\delta)}{2 m_1}} \,. \label{eq:step2} 
    \end{align}
    %$ 
    %for some constant $c_1\le 1/2$. 
    Finally, since $f^*$ is the optimal classifier on $\widecheck \calD$, 
    we have 
    \begin{align}
        \error_{\widecheck \calD}(f^*) \le \error_{\widecheck \calD}(\widehat f) \,. \label{eq:step3}
    \end{align}
    Now to relate \eqref{eq:step1} and \eqref{eq:step3}, we multiply \eqref{eq:step2} by $\frac{m_1}{m+n}$ and add $\frac{n}{m+n} \error_{\calS} (f)  + \frac{m_2}{m+n} \error_{\wt \calS_C} (f)$ both the sides. Hence, 
    we can rewrite \eqref{eq:step2} as follows: 
    \begin{align}
        \error_{\calS \cup \wt\calS}(f^*) \le \error_{ \widecheck \calD}(f^*) +  \frac{m_1}{m+n}\sqrt{\frac{\log(1/\delta)}{2 m_1}} \,. \label{eq:step4} 
    \end{align}
    Now we combine equations \eqref{eq:step1}, \eqref{eq:step4}, and \eqref{eq:step3}, to get 
    \begin{align}
        \error_{\calS \cup \wt \calS}(\wh f) \le \error_{\widecheck \calD}(\wh f) +  \frac{m_1}{m+n}\sqrt{\frac{\log(1/\delta)}{2 m_1}} \,, 
    \end{align}
    which implies 
    \begin{align}
        \error_{ \wt \calS_M}(\wh f) \le \error_{\calDm}(\wh f) + \sqrt{\frac{\log(1/\delta)}{2 m_1}} \,. \label{eq:lemma1_final}
    \end{align}
    Since $\wt S$ is obtained by randomly labeling an unlabeled dataset, we assume $2m_1 \approx m$ \footnote{Formally, with probability at least $1-\delta$, we have  $(m - 2m_1)\le \sqrt{m\log(1/\delta)/2}$.}. Moreover, using $\error_{\calDm} = 1 - \error_{\calD}$ we obtain the desired result.   
    % Combining the above steps and using the fact 
    % that $\error_\calD = 1- \error_{\calDm} $, 
    % we obtain the desired result.
\end{proof}

\begin{proof}[Proof of \lemref{lem:mislabeled_error}]
    Recall $\error_{\wt S} (f) = \frac{m_1}{m} \error_{\wt S_M}(f) + \frac{m_2}{m} \error_{\wt S_C}(f)$. Hence, we have 
    \begin{align}
        2\error_{\wt S}(f) - \error_{\wt S_M}(f) - \error_{\wt S_C}(f) &= \left(\frac{2m_1}{m} \error_{\wt S_M}(f) - \error_{\wt S_M}(f)\right) + \left(\frac{2m_2}{m} \error_{\wt S_C}(f) - \error_{\wt S_C}(f)\right) \\ &= \left(\frac{2m_1}{m} - 1\right) \error_{\wt S_M}(f) + \left(\frac{2m_2}{m} - 1 \right)\error_{\wt S_C} (f) \,.
    \end{align} 
    Since the dataset is labeled uniformly at random, with probability at least $1-\delta$, we have  $\left(\frac{2m_1}{m} - 1\right) \le \sqrt{\frac{\log(1/\delta)}{2m}}$. Similarly, we have with probability at least $1-\delta$, $\left(\frac{2m_2}{m} - 1\right) \le \sqrt{\frac{\log(1/\delta)}{2m}}$. Using union bound, with probability at least $1-\delta$, we have
    % \begin{align}
    %     2\error_{\wt S} - \error_{\wt S_M}(f) - \error_{\wt S_C}(f) \le \sqrt{\frac{\log(2/\delta)}{2m}} \left(\error_{\wt S_M}(f) + \error_{\wt S_C}(f) \right) \le 2\sqrt{\frac{\log(2/\delta)}{2m}} \,. \label{eq:lemma2_final}
    % \end{align}
    \begin{align}
        2\error_{\wt S} - \error_{\wt S_M}(f) - \error_{\wt S_C}(f) \le \sqrt{\frac{\log(2/\delta)}{2m}} \left(\error_{\wt S_M}(f) + \error_{\wt S_C}(f) \right) \,. \label{eq:lemma2_prefinal}
    \end{align}
    With re-arranging $\error_{\wt S_M}(f) + \error_{\wt S_C}(f)$ and using the inequality $ 1- a\le \frac{1}{1+a} $, we have  
    \begin{align}
        2\error_{\wt S} - \error_{\wt S_M}(f) - \error_{\wt S_C}(f) \le 2\error_{\wt \calS} \sqrt{\frac{\log(2/\delta)}{2m}}  \,. \label{eq:lemma2_final}
    \end{align}

    % We obtain the desired result by using 
\end{proof}

\begin{proof}[Proof of \lemref{lem:clear_error}]
% Recall 0-1 error on each point  $(x,y) \in S \cup \wt S$ is given by $\I{ f(x)\ne y}$.
In the set of correctly labeled points $S \cup \wt S_C$, we have $S$ as a random subset of $S \cup \wt S_C$. Hence, using Hoeffding's inequality for sampling without replacement (\lemref{lem:hoeffding_sampling}), we have with probability at least $1-\delta$
\begin{align}
    \error_{\wt \calS_C} (\wh f)- \error_{\calS \cup \wt \calS_C}( \wh f) \le  \sqrt{\frac{\log(1/\delta)}{2m_2}} \,.
\end{align}
Re-writing $\error_{\calS \cup \wt \calS_C}( \wh f)$ as $\frac{m_2}{m_2 + n} \error_{\wt \calS_C }(\wh f) + \frac{n}{m_2 + n} \error_{\calS }(\wh f)$, we have with probability at least $1-\delta$
\begin{align}
   \left(\frac{n}{n+m_2}\right) \left(\error_{\wt \calS_C} (\wh f)- \error_{\calS}( \wh f) \right) \le  \sqrt{\frac{\log(1/\delta)}{2m_2}} \,.
\end{align}
As before, assuming $2m_2 \approx m$, we have with probability at least $1-\delta$ 
\begin{align}
    \error_{\wt \calS_C} (\wh f)- \error_{\calS}( \wh f) \le \left(1+\frac{m_2}{n}\right)  \sqrt{\frac{\log(1/\delta)}{m}} \le \left(1 + \frac{m}{2n}\right) \sqrt{\frac{\log(1/\delta)}{m}} \,. \label{eq:lemma3_final}
\end{align} 
\end{proof}

\begin{proof}[Proof of \thmref{thm:error_ERM}] 
    Having established these core intermediate results, we can now combine above three lemmas to prove the main result. 
    In particular, we bound the population error on clean data ($\error_\calD(\wh f)$) as follows:  
    \begin{enumerate}[(i)]
        \item First, use \eqref{eq:lemma1_final}, to obtain an upper bound on the population error on clean data, i.e., with probability at least $1-\delta/4$, we have
        \begin{align}
            \error_{ \calD} (\wh f) \le 1 - \error_{ \wt \calS_M}(\wh f) + \sqrt{\frac{\log(4/\delta)}{m}} \,. 
        \end{align}
        \item  Second, use \eqref{eq:lemma2_final}, to relate the error on the mislabeled fraction with error on clean portion of randomly labeled data and error on whole randomly labeled dataset, i.e., with probability at least $1-\delta/2$, we have 
        \begin{align}
            - \error_{\wt S_M}(f) \le \error_{\wt S_C}(f) - 2\error_{\wt S}  + 2\error_{\wt S} \sqrt{\frac{\log(4/\delta)}{2m}}  \,. 
        \end{align} 
        \item Finally, use \eqref{eq:lemma3_final} to relate the error on the clean portion of randomly labeled data and error on clean training data, i.e., with probability $1-\delta/4$, we have 
        \begin{align}
            \error_{\wt \calS_C} (\wh f)\le - \error_{\calS}( \wh f) + \left(1 + \frac{m}{2n} \right) \sqrt{\frac{\log(4/\delta)}{m}} \,. 
        \end{align} 
    \end{enumerate}

    Using union bound on the above three steps, we have with probability at least $1-\delta$: 
    \begin{align}
        \error_\calD (\wh f) \le \error_{\calS}(\wh f)   + 1 - 2\error_{\wt \calS}(\wh f)   + \left(\sqrt{2} \error_{\wt S} + 2 + \frac{m}{2n}\right)  \sqrt{\frac{\log(4/\delta)}{m}} \,.
    \end{align}
    % Note that $(1/\sqrt{2} + 2.5)$ is a loose constant. In experiments, we use the ratio $\frac{m}{n}$
    %  the exact error $\error_{\wt \calS}(\wh f)$ 
    % to evaluate R.H.S.    
\end{proof}

\subsection{Proof of \propref{prop:rademacher}}

\begin{proof}[Proof of \propref{prop:rademacher}]
    For a classifier $ f: \calX \to \{-1, 1\}$, we have $1 - 2\,\indict{ f(x) \ne y} = y \cdot f(x)$. Hence, by definition of $\error$, we have 
    \begin{align}
        1 -2\error_{\wt \calS}(f) = \frac{1}{m}\sum_{i=1}^m y_i \cdot f(x_i) \le \sup_{f \in \calF} \, \frac{1}{m} \sum_{i=1}^m y_i \cdot f(x_i)  \,. \label{eq:error_rademacher}
    \end{align}
    Note that for fixed inputs $(x_1, x_2, \ldots, x_m)$ in $\wt S$, $(y_1, y_2, \ldots y_m)$ are random labels. Define $\phi_1 (y_1, y_2, \ldots, y_m) \defeq \sup_{f \in \calF} \, \frac{1}{m} \sum_{i=1}^m y_i \cdot f(x_i)$. We have the following bounded difference condition on $\phi_1$. For all i, 
    \begin{align}
        \sup_{y_1, \ldots y_m, y_i^\prime \in \{-1, 1\}^{m+1} } \abs{ \phi_1 (y_1,\ldots, y_i, \ldots, y_m) - \phi_1 (y_1,\ldots, y_i^\prime, \ldots, y_m)  } \le 1/m \,. \label{cond1_rademacher}
    \end{align} 
    
    Similarly, we define $\phi_2 (x_1, x_2, \ldots, x_m) \defeq \Expt{ y_i \sim_U \{-1, 1\}  }{ \sup_{f \in \calF} \, \frac{1}{m}  \sum_{i=1}^m y_i \cdot f(x_i)}$. We have the following bounded difference condition on $\phi_2$. 
    For all i,
    \begin{align}
        \sup_{x_1, \ldots x_m, x_i^\prime \in \calX^{m+1} } \abs{ \phi_2 (x_1,\ldots, x_i, \ldots, x_m) - \phi_1 (x_1,\ldots, x_i^\prime, \ldots, x_m)  } \le 1/m \,. \label{cond2_rademacher}
    \end{align}
    Using McDiarmid’s inequality (\lemref{lem:McDiarmid}) twice 
    with Condition \eqref{cond1_rademacher} and \eqref{cond2_rademacher}, 
    with probability at least $1-\delta$, we have
    \begin{align}
        \sup_{f \in \calF} \, \frac{1}{m} \sum_{i=1}^m y_i \cdot f(x_i)  - \Expt{x,y}{\sup_{f \in \calF} \, \frac{1}{m} \sum_{i=1}^m y_i \cdot f(x_i) } \le \sqrt{\frac{2\log(2/\delta)}{m}} \,. \label{eq:final_rademacher}
    \end{align} 
    Combining \eqref{eq:error_rademacher} and \eqref{eq:final_rademacher}, we obtain the desired result. 
\end{proof}


\subsection{Proof of \thmref{thm:error_regularized_ERM}}

Proof of \thmref{thm:error_regularized_ERM} follows similar to the proof of \thmref{thm:error_ERM}. Note that the same results in \lemref{lem:fit_mislabeled}, \lemref{lem:mislabeled_error}, and \lemref{lem:clear_error} hold in the regularized ERM case. However, the arguments in the proof of \lemref{lem:fit_mislabeled} change slightly. Hence, we state the lemma for regularized ERM and prove it here for completeness. 

\begin{lemma} \label{lem:lemma1_reg}
    Assume the same setup as \thmref{thm:error_regularized_ERM}. 
    Then for any $\delta >0$, with probability at least  $1-\delta$ 
    over the random draws of mislabeled data $\wt S_M$, we have 
    \begin{align}
        \error_\calD(\widehat f)  \le 1 -\error_{\wt \calS_M}(\widehat f) + \sqrt{\frac{\log(1/\delta)}{m}}\,. 
    \end{align} 
\end{lemma}
\begin{proof}
    The main idea of the proof remains the same, i.e. regard 
    the clean portion of the data 
    ($S \cup \wt S_C$) as fixed.   
    Then, there exists a classifier $f^*$ 
    that is optimal over draws 
    of the mislabeled data $\wt S_M$. 

    
    Formally, 
    \begin{align}
    f^* \defeq \argmin_{f \in \calF} \error_{\widecheck {\calD}} (f)  + \lambda R(f) \,, \label{eq:modified_ERM_reg}
    \end{align}
    where $$\widecheck \calD = \frac{n}{m+n} \calS + \frac{m_1}{m+n} \wt \calS_C  + \frac{m_2}{m+n}\calDm \,.$$ That is, $\widecheck \calD$ a combination of 
    the \emph{empirical distribution} 
    over correctly labeled data $S \cup \wt S_C$
    % in $S\cup \wt S$ 
    and the (population) distribution 
    over mislabeled data $\calDm$.
    Recall that 
    \begin{align}
    \wh f \defeq \argmin_{f \in \calF} \error_{\calS \cup \wt S} (f) + \lambda R(f) \,. \label{eq:orig_ERM_reg}
    \end{align}
    % 
    % 
    Since, $\widehat f$ minimizes 0-1 error 
    on $S \cup \wt S$, using ERM optimality on \eqref{eq:orig_ERM},  
    we have 
    \begin{align}
        \error_{\calS \cup \wt \calS}(\widehat f) + \lambda R(\wh f) \le \error_{
            \calS \cup \wt \calS}(f^*) + \lambda R(f^*) \,.    \label{eq:step1_reg}
    \end{align}
    Moreover, since $f^*$ is independent of $\wt S_M$, using Hoeffding's bound,
    % \footnote{For a fully rigorous argument,
    % refer to the complete proof in App.~\ref{app:proof_erm}.} 
    we have with probability at least $1-\delta$ that
    \begin{align}
      \error_{\wt \calS_M}(f^*) \le \error_{ \calDm}(f^*) +  \sqrt{\frac{\log(1/\delta)}{2 m_1}} \,. \label{eq:step2_reg} 
    \end{align}
    %$ 
    %for some constant $c_1\le 1/2$. 
    Finally, since $f^*$ is the optimal classifier on $\widecheck \calD$, 
    we have 
    \begin{align}
        \error_{\widecheck \calD}(f^*) + \lambda R(f^*) \le \error_{\widecheck \calD}(\widehat f) + \lambda R(\wh f) \,. \label{eq:step3_reg}
    \end{align}
     Now to relate \eqref{eq:step1_reg} and \eqref{eq:step3_reg}, we can re-write the \eqref{eq:step2_reg} as follows: 
    \begin{align}
        \error_{\calS \cup \wt\calS}(f^*) \le \error_{ \widecheck \calD}(f^*) +  \frac{m_1}{m+n}\sqrt{\frac{\log(1/\delta)}{2 m_1}} \,. \label{eq:step4_reg} 
    \end{align}
    After adding $\lambda R(f^*)$ on both sides in \eqref{eq:step4_reg}, we combine equations \eqref{eq:step1_reg}, \eqref{eq:step4_reg}, and \eqref{eq:step3_reg}, to get 
    \begin{align}
        \error_{\calS \cup \wt \calS}(\wh f) \le \error_{\widecheck \calD}(\wh f) +  \frac{m_1}{m+n}\sqrt{\frac{\log(1/\delta)}{2 m_1}} \,, 
    \end{align}
    which implies 
    \begin{align}
        \error_{ \wt \calS_M}(\wh f) \le \error_{\calDm}(\wh f) + \sqrt{\frac{\log(1/\delta)}{2 m_1}} \,. \label{eq:lemma_reg_final}
    \end{align}
    Similar as before, since $\wt S$ is obtained by randomly labeling an unlabeled dataset, we assume 
    $2m_1 \approx m$. Moreover, using $\error_{\calDm} = 1 - \error_{\calD}$ we obtain the desired result. 
\end{proof}
% \begin{proof}[Proof of ]
    
% \end{proof}

\subsection{Proof of \thmref{thm:multiclass_ERM}}

To prove our results in the multiclass case,
we first state and prove lemmas
parallel to those
% We first state and prove lemmas 
% parallel 
% to the three lemmas 
used in the proof of balanced binary case. 
We then combine these results 
% in the three lemmas 
to obtain the result in \thmref{thm:multiclass_ERM}. 

Before stating the result, 
we define mislabeled distribution $\calDm$ for any $\calD$.
While $\calDm$ and $\calD$ share 
the same marginal distribution over inputs $\calX$,
the conditional distribution over labels $y$ 
given an input $x\sim \calD_\calX$ is changed as follows:
For any $x$, the Probability Mass Function (PMF) over $y$ is defined as:  
$p_{\calDm} (\cdot \vert x) \defeq \frac{1 - p_{\calD}(\cdot \vert x)}{k - 1}$, where $ p_{\calD}(\cdot \vert x)$ is the PMF over $y$ for the distribution $\calD$. 

\begin{lemma} \label{lem:fit_mislabeled_multi}
    Assume the same setup as \thmref{thm:multiclass_ERM}. 
    Then for any $\delta >0$, with probability at least  $1-\delta$ 
    over the random draws of mislabeled data $\wt S_M$, we have 
    \begin{align}
        \error_\calD(\widehat f)  \le (k-1)\left(1 -\error_{\wt \calS_M}(\widehat f)\right) + (k-1)\sqrt{\frac{\log(1/\delta)}{m}}\,. \label{eq:lemma1_multi}
    \end{align}   
\end{lemma} 

\begin{proof}
   
    The main idea of the proof remains the same.
    We begin by regarding the clean portion of the data 
    ($S \cup \wt S_C$) as fixed. 
    Then, there exists a classifier $f^*$ 
    that is optimal over draws 
    of the mislabeled data $\wt S_M$. 
    
    However, in the multiclass case,
    we cannot as easily relate the population error on mislabeled data 
    to the population accuracy on clean data.   
    While for binary classification, 
    % we could upper bound $\error_{\wt \calS_M}$ 
    % with $1-\error_\calD$ 
    we could lower bound the population accuracy $1-\error_\calD$
    with the empirical error on mislabeled data $\error_{\wt \calS_M}$ 
    (in the proof of \lemref{lem:fit_mislabeled}), 
    for multiclass classification, 
    error on the mislabeled data 
    and accuracy on the clean data 
    in the population 
    are not so directly related.  
    To establish \eqref{eq:lemma1_multi},
    we break the error on the 
    (unknown) mislabeled data 
    into two parts: one term corresponds 
    to predicting the true label on mislabeled data, 
    and the other corresponds to predicting 
    neither the true label 
    nor the assigned (mis-)label.  
    Finally, we relate these errors to their
    population counterparts to establish \eqref{eq:lemma1_multi}. 
    
    Formally, 
    \begin{align}
    f^* \defeq \argmin_{f \in \calF} \error_{\widecheck {\calD}} (f)  + \lambda R(f) \,, \label{eq:modified_ERM_reg2}
    \end{align}
    where $$\widecheck \calD = \frac{n}{m+n} \calS + \frac{m_1}{m+n} \wt \calS_C  + \frac{m_2}{m+n}\calDm \,.$$ 
    That is, $\widecheck \calD$ is a combination 
    of the \emph{empirical distribution} 
    over correctly labeled data $S \cup \wt S_C$
    % in $S\cup \wt S$ 
    and the (population) distribution 
    over mislabeled data $\calDm$.
    Recall that 
    \begin{align}
    \wh f \defeq \argmin_{f \in \calF} \error_{\calS \cup \wt S} (f) + \lambda R(f) \,. \label{eq:orig_ERM_reg2}
    \end{align}
    % 
    % 
    Following the exact steps from the proof of \lemref{lem:lemma1_reg}, 
    with probability at least $1-\delta$, we have  
    \begin{align}
        \error_{ \wt \calS_M}(\wh f) \le \error_{\calDm}(\wh f) + \sqrt{\frac{\log(1/\delta)}{2 m_1}} \,. \label{eq:lemma1_final_multi_prev}
    \end{align}
    Similar to before, since $\wt S$ is obtained 
    by randomly labeling an unlabeled dataset, 
    we assume 
    $\frac{k}{k-1} m_1 \approx m$. 
    
    Now we will relate $\error_{\calDm} (\wh f)$ with $\error_{\calD}(\wh f)$. 
    Let $y^T$ denote the (unknown) true label 
    for a mislabeled point $(x, y)$ 
    (i.e., label before replacing it with a mislabel). 
    \begin{align*}    
         \Expt{(x, y) \in \sim \calDm}{\indict{ \wh f(x) \ne y }}  &= \underbrace{\Expt{(x, y) \in \sim \calDm}{\indict{ \wh f(x) \ne y \land \wh f(x) \ne y^T}}}_{\RN{1}} \\ &\qquad \qquad + \underbrace{\Expt{(x, y) \in \sim \calDm}{\indict{ \wh f(x) \ne y \land \wh f(x) = y^T}}}_{\RN{2}} \,. \numberthis \label{eq:excess_term}
    \end{align*}
    Clearly, term 2 is one minus the accuracy 
    on the clean unseen data, i.e.,
    \begin{align}
        \RN{2} = 1 - \Expt{{x,y} \sim \calD}{ \indict{ \wh f(x) \ne y}} = 1- \error_{\calD}(\wh f) \,. \label{eq:term1}    
    \end{align}
    Next, we relate term 1 with the error on the unseen clean data. 
    We show that term 1 is equal to the error on the unseen clean data 
    scaled by $\frac{k-2}{k-1}$,
    where $k$ is the number of labels.
    Using the definition of mislabeled distribution $\calDm$,  
    we have 
    \begin{align}
        \RN{1} = \frac{1}{k-1} \left( \Expt{(x, y) \in \sim \calD}{ \sum_{i \in \calY \land i\ne y}  \indict{ \wh f(x) \ne i \land \wh f(x) \ne y}} \right) = \frac{k-2}{k-1} \error_{\calD}(\wh f) \,.\label{eq:term2}
    \end{align}    

    Combining the result in \eqref{eq:term1}, \eqref{eq:term2} and \eqref{eq:excess_term}, we have 
    \begin{align}
        \error_{\calDm}(\wh f) = 1- \frac{1}{k-1} \error_{\calD}(\wh f) \,.\label{eq:combine_terms}
    \end{align}
    Finally, combining the result in \eqref{eq:combine_terms} 
    with equation \eqref{eq:lemma1_final_multi_prev}, 
    we have with probability $1-\delta$, 
    \begin{align}
      \error_{\calD}(\wh f) \le  (k-1) \left( 1- \error_{ \wt \calS_M}(\wh f) \right)  + (k-1) \sqrt{\frac{k \log(1/\delta)}{ 2(k-1)m}} \,. \label{eq:lemma1_final_multi}
    \end{align}
\end{proof}

\begin{lemma} \label{lem:mislabeled_error_multi}
    Assume the same setup as \thmref{thm:multiclass_ERM}. 
    Then for any $\delta >0$, 
    with probability at least $1-\delta$ 
    over the random draws of $\wt S$, we have  
    % \begin{align}
        $$\abs{k\error_{\wt \calS}(\widehat f) - \error_{\wt \calS_C}(\widehat f) -  (k-1)\error_{\wt \calS_M}(\widehat f) } \le  2k\sqrt{\frac{\log(4/\delta)}{2m}}\,. $$ % \label{eq:lemma2}
    % \end{align}   
    %  for some constant $c_3 \le 1.0\,$.
\end{lemma} 


\begin{proof}
    Recall $\error_{\wt S} (f) = \frac{m_1}{m} \error_{\wt S_M}(f) + \frac{m_2}{m} \error_{\wt S_C}(f)$. Hence, we have 
    \begin{align*}
        k\error_{\wt S}(f) - (k-1)\error_{\wt S_M}(f) - \error_{\wt S_C}(f) &= (k-1)\left(\frac{k m_1}{(k-1) m} \error_{\wt S_M}(f) - \error_{\wt S_M}(f)\right) \\ & \qquad \qquad + \left(\frac{km_2}{m} \error_{\wt S_C}(f) - \error_{\wt S_C}(f)\right) \\ &= k \left[ \left(\frac{m_1}{m} - \frac{k-1}{k}\right) \error_{\wt S_M}(f) + \left(\frac{m_2}{m} - \frac{1}{k} \right) \error_{\wt S_C} (f) \right] \,.
    \end{align*} 
    Since the dataset is randomly labeled, 
    we have with probability at least $1-\delta$, 
    $\left(\frac{m_1}{m} - \frac{k-1}{k}\right) \le \sqrt{\frac{\log(1/\delta)}{2m}}$. 
    Similarly, we have with probability at least $1-\delta$, 
    $\left(\frac{m_2}{m} - \frac{1}{k}\right) \le \sqrt{\frac{\log(1/\delta)}{2m}}$. 
    Using union bound, we have with probability at least $1-\delta$
    % \begin{align}
    %     2\error_{\wt S} - \error_{\wt S_M}(f) - \error_{\wt S_C}(f) \le \sqrt{\frac{\log(2/\delta)}{2m}} \left(\error_{\wt S_M}(f) + \error_{\wt S_C}(f) \right) \le 2\sqrt{\frac{\log(2/\delta)}{2m}} \,. \label{eq:lemma2_final}
    % \end{align}
    \begin{align}
        k\error_{\wt S}(f) - (k-1)\error_{\wt S_M}(f) - \error_{\wt S_C}(f)  \le k \sqrt{\frac{\log(2/\delta)}{2m}} \left(\error_{\wt S_M}(f) + \error_{\wt S_C}(f) \right) \,. \label{eq:lemma2_final_multi}
    \end{align}

    % We obtain the desired result by using 
\end{proof}

\begin{lemma} \label{lem:clear_error_multi}
    Assume the same setup as \thmref{thm:multiclass_ERM}. 
    Then for any $\delta >0$, with probability at least $1-\delta$ 
    over the random draws of $\wt S_C$ and $S$, we have 
    % \begin{align}
        $$\abs{\error_{\wt \calS_C}(\widehat f) - \error_{\calS}(\widehat f) } \le 1.5 \sqrt{\frac{k\log(2/\delta)}{2m}}\,.$$ %\label{eq:lemma3}
    % \end{align}   
    % for some constant $c_2 \le 1.2\,$.
\end{lemma} 
\begin{proof}
    % Recall 0-1 error on each point  $(x,y) \in S \cup \wt S$ is given by $\I{ f(x)\ne y}$.
    In the set of correctly labeled points $S \cup \wt S_C$,
    we have $S$ as a random subset of $S \cup \wt S_C$. 
    Hence, using Hoeffding's inequality 
    for sampling without replacement 
    (\lemref{lem:hoeffding_sampling}), 
    we have with probability at least $1-\delta$
    \begin{align}
        \error_{\wt \calS_c} (\wh f)- \error_{\calS \cup \wt \calS_C}( \wh f) \le  \sqrt{\frac{\log(1/\delta)}{2m_2}} \,.
    \end{align}
    Re-writing $\error_{\calS \cup \wt \calS_C}( \wh f)$ 
    as $\frac{m_2}{m_2 + n} \error_{\wt \calS_C }(\wh f) + \frac{n}{m_2 + n} \error_{\calS }(\wh f)$, 
    we have with probability at least $1-\delta$
    \begin{align}
       \left(\frac{n}{n+m_2}\right) \left(\error_{\wt \calS_c} (\wh f)- \error_{\calS}( \wh f) \right) \le  \sqrt{\frac{\log(1/\delta)}{2m_2}} \,.
    \end{align}
    As before, assuming $km_2 \approx m$, 
    we have with probability at least $1-\delta$ 
    \begin{align}
        \error_{\wt \calS_c} (\wh f)- \error_{\calS}( \wh f) \le \left(1+\frac{m_2}{n}\right)  \sqrt{\frac{k\log(1/\delta)}{2m}} \le \left( 1 + \frac{1}{k}\right) \sqrt{\frac{k\log(1/\delta)}{2m}} \,. \label{eq:lemma3_final_multi}
    \end{align} 
\end{proof}

\begin{proof}[Proof of \thmref{thm:multiclass_ERM}] 
    Having established these core intermediate results, 
    we can now combine above three lemmas. 
    In particular, we bound the population error 
    on clean data ($\error_\calD(\wh f)$) as follows:  
    \begin{enumerate}[(i)]
        \item First, use \eqref{eq:lemma1_final_multi}, 
        to obtain an upper bound on the population error on clean data, 
        i.e., with probability at least $1-\delta/4$, we have
        \begin{align}
            \error_{ \calD} (\wh f) \le (k-1)\left(1 - \error_{ \wt \calS_M}(\wh f) \right) + (k-1) \sqrt{\frac{k\log(4/\delta)}{2(k-1)m}} \,. 
        \end{align}
        \item  Second, use \eqref{eq:lemma2_final_multi}
        to relate the error on the mislabeled fraction 
        with error on clean portion of randomly labeled data 
        and error on whole randomly labeled dataset, 
        i.e., with probability at least $1-\delta/2$, we have 
        \begin{align}
            - (k-1)\error_{\wt S_M}(f) \le \error_{\wt S_C}(f) - k\error_{\wt S}  + k\sqrt{\frac{\log(4/\delta)}{2m}}  \,. 
        \end{align} 
        \item Finally, use \eqref{eq:lemma3_final_multi} 
        to relate the error on the clean portion of randomly labeled data 
        and error on clean training data, 
        i.e., with probability $1-\delta/4$, we have 
        \begin{align}
            \error_{\wt \calS_C} (\wh f)\le - \error_{\calS}( \wh f) + \left(1 + \frac{m}{kn} \right) \sqrt{\frac{k\log(4/\delta)}{2m}} \,. 
        \end{align} 
    \end{enumerate}

    Using union bound on the above three steps, 
    we have with probability at least $1-\delta$: 
    \begin{align}
        \error_\calD (\wh f) \le \error_{\calS}(\wh f) + (k-1) - k\error_{\wt \calS}(\wh f)   + (\sqrt{k(k-1)} + k + \sqrt{k} + \frac{m}{n\sqrt{k}})  \sqrt{\frac{\log(4/\delta)}{2m}} \,.\label{eq:multiclass_ERM_final}
    \end{align}
    Simplifying the term in RHS of \eqref{eq:multiclass_ERM_final}, 
    we get the desired result. 
    % Note that since $\frac{m}{n\sqrt{k}}$ 
    % is much smaller than the sum of the other terms
    % the other terms in summation, 
    % we ignore $\frac{m}{n\sqrt{k}}$  
    % Z: ??? --- great
    % that 
    % them
    in the final bound. 
    % we ignore that in the final bound. 
    % Note that $(1/\sqrt{2} + 2.5)$ is a loose constant. In experiments, we use the ratio $\frac{m}{n}$
    %  the exact error $\error_{\wt \calS}(\wh f)$ 
    % to evaluate R.H.S.    
\end{proof}

\newpage
\section{Proofs from \secref{sec:linear_models}}\label{app:proof_gd}
We suppose that the parameters of the linear function 
are obtained via gradient descent on 
the following $L_2$ regularized problem: 
\begin{align}
    % n in denominator is avoided deliberately
    \calL_S(w; \lambda) \defeq \sum_{i=1}^n{(w^Tx_i - y_i)^2} + \lambda \norm{w}{2}^2 \,, \label{eq:l2_MSE_app}   
\end{align}
where $\lambda\ge0$ is a regularization parameter. 
We assume access to a clean dataset 
$S = \{(x_i, y_i)\}_{i=1}^n \sim \calD^n$ 
and randomly labeled dataset 
$\wt S = \{(x_i, y_i)\}_{i=n+1}^{n+m} \sim \wt \calD^m$. 
Let $\bX = [x_1, x_2, \cdots, x_{m+n}]$ 
and $\by = [y_1, y_2, \cdots, y_{m+n}]$. 
Fix a positive learning rate $\eta$ such that 
$\eta \le 1/\left(\norm{\bX^T\bX}{\text{op}} + \lambda^2\right)$ 
and an initialization $w_0 = 0$. 
% \todos{Assumption made for simplicty}. 
Consider the following gradient descent iterates 
to minimize objective \eqref{eq:l2_MSE_app} on $S \cup \wt S$:
\begin{align}
w_t = w_{t-1} - \eta \grad_w \calL_{S \cup \wt S} (w_{t-1}; \lambda) \quad \forall t=1,2,\ldots \label{eq:GD_iterates_app}
\end{align} 
Then we have $\{ w_t\}$ converge to the limiting solution 
$\wh w = \left( \bX^T\bX+\lambda \boldsymbol{I}\right)^{-1}\bX^T\by$. Define $\widehat f (x) \defeq f(x ; \wh w) $.  

% \subsection{\textcolor{red}{Errata}}

% We wish to correct the following error in the body:
% \codref{cond:error_stability} is not enough 
% to guarantee the result in \thmref{thm:linear}. 
% We now present a slightly stronger condition 
% called \emph{hypothesis stability} 
% under which we obtain a result 
% similar to \thmref{thm:linear}. 

% This error doesn't change the main arguments of the proof,
% where we show that the empirical train error 
% is less than or equal to the leave-one-out error.
% We need a stronger condition to relate leave-one-out error 
% with the population error of the original classifier. 
% Specifically, while \codref{cond:error_stability} 
% relates the average population error of leave-one-out classifiers 
% with the population error of the original classifier, 
% we need the new condition to show the concentration 
% of the empirical leave-one-out error 
% and average population error of leave-one-out classifiers. 
% main takeaway 

% Note that the new condition, 
% while being stronger than the previous one, 
% still doesn't imply generalization \citep{bousquet2002stability,elisseeff2003leave,abou2019exponential}. 
% Overall, the main results in \secref{sec:ERM_training} 
% and takeaways of the paper remain unaffected by the error.  

% We now present the new condition 
% and a corrected statement of \thmref{thm:linear}. 
% Recall, for a given training set $S \sim \calD^n $, 
% we use $S_{(i)}$ to denote the training set $S$ 
% with the $i^{\text{th}}$ point removed.

% \begin{condition}[Hypothesis Stability] 
%     \label{cond:hypothesis_stability}
%     We have $\beta$ hypothesis stability 
%     if our training algorithm $\calA$ satisfies the following: 
%     \begin{align*}
%     % ${\sum_{i=1}^n \frac{\error_{\calD}( f(\calA, S_{(i)}))}{n} - \error_\calD(f(\calA, S))} \le \beta\,$.
%     \forall i \in \{1,2,\ldots, n\}, \quad  \Expt{\calS, (x,y) \in \calD}{ \abs{\error\left( f(x) ,y  \right) - \error\left( f_{(i)}(x), y \right) }} \le \frac{\beta}{n} \,,
%     \end{align*}
%     where $f_{(i)} \defeq f(\calA, S_{(i)})$ and $ f \defeq f(\calA, S)$.
% \end{condition}

% \begin{theorem}[Correct statement of \thmref{thm:linear}] \label{thm:new_linear}
%     Assume that this gradient descent algorithm satisfies \codref{cond:hypothesis_stability}
%     with $\beta=\calO(1)$.  
%     Then for any $\delta >0$, with probability at least $1-\delta$ 
%     over the random draws of datasets $\wt S$ and $S$, we have:
%     \begin{align}
%         \error_\calD(\widehat f) \le \error_\calS(\widehat f) + 1 - 2 \error_{\wt\calS}(\widehat f) + \left(\frac{1}{\sqrt{2}} + 1.5 \right) \sqrt{\frac{\log(4/\delta)}{m}} + \sqrt{\frac{4}{\delta}\left(\frac{1}{m} +\frac{3\beta}{m+n} \right)}  \,. \label{eq:gd_error}
%     \end{align} 
%     % for some constant $c\le 3.2$.
% \end{theorem}

\subsection{Proof of \thmref{thm:linear}}
We use a standard result from linear algebra, 
namely the Shermann-Morrison formula 
\citep{sherman1950adjustment} for matrix inversion:  

\begin{lemma}[\citet{sherman1950adjustment}] \label{lem:sherman}
    Suppose $\bA \in \Real^{n \times n}$ 
    is an invertible square matrix 
    and $u,v \in \Real^n$ are column vectors. 
    Then $\bA + uv^T$ is invertible iff $1 + v^T \bA u \ne 0$ 
    and in particular
    \begin{align}
        (\bA + u v^T)^{-1} = \bA^{-1}  - \frac{\bA^{-1} uv^T \bA^{-1} }{ 1 + v^T \bA^{-1} u} \,.
    \end{align}   
\end{lemma}
\newcommand\byy[1]{\by_{\left(#1\right)}}
\newcommand\bXX[1]{\bX_{\left(#1\right)}}
\newcommand\ff[1]{\wh f_{\left(#1\right)}}

For a given training set $S \cup \wt S_C$, 
define leave-one-out error 
on mislabeled points in the training data 
as $$\error_{\text{LOO}(\wt S_M) } = \frac{\sum_{(x_i, y_i) \in \wt S_M} \error( f_{(i)}( x_i), y_i)}{ \abs{\wt S_M }} \,, $$
where $f_{(i)} \defeq f(\calA, (S \cup \wt S)_{(i)})$. 
To relate empirical leave-one-out error and population error 
with hypothesis stability condition, 
we use the following lemma:   

\begin{lemma}[\citet{bousquet2002stability}] \label{lem:stability_error}
    For the leave-one-out error, we have
    \begin{align}
        \Expo{ \left( \error_{\calDm}(\wh f) -\error_{\text{LOO}(\wt S_M) } \right)^2 } \le \frac{1}{2m_1}+  \frac{3\beta}{n + m}\,.
    \end{align}   
    % where $ f \defeq f(\calA, S \cup \wt S) $.
\end{lemma}

Proof of the above lemma is similar 
to the proof of Lemma 9 in \citet{bousquet2002stability} 
and can be found in \appref{app:proof_lem_error}. 
% 
% Before presenting the result, we introduce some notation. 
Before presenting the proof of \thmref{thm:linear}, 
we introduce some more notation. 
Let $\bX_{(i)}$ denote the matrix of covariates 
with the $i^{\text{th}}$ point removed. 
Similarly, let $\by_{(i)}$ be the array of responses 
with the $i^{\text{th}}$ point removed. 
Define the corresponding regularized GD solution 
as $\wh w_{(i)} = \left( \bXX{i}^T\bXX{i}+\lambda \boldsymbol{I}\right)^{-1}\bXX{i}^T\byy{i}$. 
Define $\ff{i}(x) \defeq f(x ; \wh w_{(i)}) $.

\begin{proof}[Proof of \thmref{thm:linear}]
    Because squared loss minimization does not imply 0-1 error minimization, 
    we cannot use arguments from \lemref{lem:fit_mislabeled}. 
    This is the main technical difficulty. 
    To compare the 0-1 error at a train point with an unseen point, 
    we use the closed-form expression for $\widehat{w}$ 
    and Shermann-Morrison formula 
    to upper bound training error 
    with leave-one-out cross validation error. 
    
    The proof is divided into three parts: 
    In part one, we show that 0-1 error 
    on mislabeled points in the training set 
    is lower than the error obtained 
    by leave-one-out error at those points. 
    In part two, we relate this leave-one-out error 
    with the population error on mislabeled distribution
    using \codref{cond:hypothesis_stability}.
    While the empirical leave-one-out error is an unbiased estimator 
    of the average population error of leave-one-out classifiers, 
    we need hypothesis stability 
    to control the variance 
    of empirical leave-one-out error. 
    Finally, in part three, we show 
    that the error on the mislabeled training points 
    can be estimated with just the randomly labeled 
    and clean training data (as in proof of \thmref{thm:error_ERM}).  

    \textbf{Part 1 {} {}} First we relate training error with leave-one-out error.        
    For any training point $(x_i, y_i)$ in $\wt S \cup S$, we have 
    \begin{align}
        \error(\wh f(x_i), y_i ) &= \indict{ y_i \cdot x_i^T \wh w < 0 } = \indict{ y_i \cdot x_i^T \left( \bX^T\bX+\lambda \boldsymbol{I}\right)^{-1}\bX^T\by < 0 } \\
        &= \indict{ y_i \cdot x_i^T \underbrace{\left( \bXX{i}^T\bXX{i} + x_i ^T x_i +\lambda \boldsymbol{I}\right)^{-1}}_{\RN{1}} (\bXX{i}^T\byy{i} + y_i \cdot x_i) < 0 } \,.
    \end{align}
    Letting $\bA = \left(\bXX{i}^T\bXX{i} +\lambda \boldsymbol{I}\right)$ 
    and using \lemref{lem:sherman} on term 1, we have 
    \begin{align}
        \error(\wh f(x_i), y_i ) &= \indict{ y_i \cdot x_i^T \left[\bA^{-1} -  \frac{\bA^{-1} x_i x_i^T \bA^{-1}}{ 1 + x_i ^T \bA^{-1} x_i } \right] (\bXX{i}^T\byy{i} + y_i \cdot x_i) < 0 } \\
        &= \indict{ y_i \cdot\left[ \frac{ x_i^T \bA^{-1} ( 1 + x_i ^T \bA^{-1} x_i ) -  x_i^T \bA^{-1} x_i x_i^T \bA^{-1}}{ 1 + x_i ^T \bA ^{-1}x_i } \right] (\bXX{i}^T\byy{i} + y_i \cdot x_i) < 0 } \\
        &= \indict{ y_i \cdot\left[ \frac{ x_i^T \bA^{-1}}{ 1 + x_i ^T \bA ^{-1}x_i } \right] (\bXX{i}^T\byy{i} + y_i \cdot x_i) < 0 } \,.
    \end{align}

    Since $1 + x_i^T \bA^{-1} x_i > 0$, we have 
    \begin{align}
        \error(\wh f(x_i), y_i ) &= \indict{ y_i \cdot x_i^T \bA^{-1} (\bXX{i}^T\byy{i} + y_i \cdot x_i) < 0 } \\
        &= \indict{ x_i^T \bA^{-1} x_i +  y_i \cdot x_i^T \bA^{-1} (\bXX{i}^T\byy{i}) < 0 } \\
        &\le \indict{ y_i \cdot x_i^T \bA^{-1} (\bXX{i}^T\byy{i}) < 0 } = \error(\ff{i}(x_i), y_i ) \,.\label{eq:LOO_error}
    \end{align}

    Using \eqref{eq:LOO_error}, we have 
    \begin{align}
        \error_{\wt \calS_M } (\wh f) \le \error_{\text{LOO} (\wt S_M)} \defeq \frac{\sum_{(x_i, y_i) \in \wt S_M} \error(\ff{i}(x_i), y_i ) }{\abs{\wt \calS_M}}\label{eq:LOO_error_final} \,.
    \end{align}
    \textbf{Part 2 {}{}} We now relate RHS in \eqref{eq:LOO_error_final} 
    with the population error on mislabeled distribution. 
    To do this, we leverage \codref{cond:hypothesis_stability} 
    and \lemref{lem:stability_error}. 
    In particular, we have 

    \begin{align}
        \Expt{\calS \cup \wt \calS_M }{ \left(\error_{\calDm}(\wh f) - \error_{\text{LOO} (\wt S_M)}\right)^2 } \le \frac{1}{2m_1} + \frac{3\beta}{m+n} \,.
    \end{align}

    Using Chebyshev's inequality, with probability at least $1-\delta$, we have 
    \begin{align}
        \error_{\text{LOO} (\wt S_M)} \le  \error_{\calDm}(\wh f)   + \sqrt{\frac{1}{\delta}\left(\frac{1}{2m_1} +\frac{3\beta}{m+n} \right)} \,. \label{eq:final_mislabeled_linear}
    \end{align}
    

    \textbf{Part 3 {}{}} Combining \eqref{eq:final_mislabeled_linear} and \eqref{eq:LOO_error_final}, we have 

    \begin{align}
        \error_{\wt \calS_M } (\wh f) \le \error_{\calDm}(\wh f)   + \sqrt{\frac{1}{\delta}\left(\frac{1}{2m_1} +\frac{3\beta}{m+n} \right)} \,. \label{eq:linear_parallel_lem1}
    \end{align}

    Compare \eqref{eq:linear_parallel_lem1} with \eqref{eq:lemma1_final} 
    in the proof of \lemref{lem:fit_mislabeled}. 
    We obtain a similar relationship 
    between $\error_{\wt \calS_M }$ and $\error_{\calDm}$ 
    but with a polynomial concentration 
    instead of exponential concentration. 
    In addition, since we just use concentration arguments 
    to relate mislabeled error to the errors
    on the clean and unlabeled portions 
    of the randomly labeled data, 
    we can directly use the results 
    in \lemref{lem:mislabeled_error} and \lemref{lem:clear_error}. 
    Therefore, combining results in \lemref{lem:mislabeled_error}, \lemref{lem:clear_error}, and \eqref{eq:linear_parallel_lem1} with union bound, 
    we have with probability at least $1-\delta$
    \begin{align}
        \error_\calD(\widehat f) \le \error_\calS(\widehat f) + 1 - 2 \error_{\wt\calS}(\widehat f) + \left(\sqrt{2}\error_{\wt\calS}(\widehat f) + 1 + \frac{m}{2n} \right) \sqrt{\frac{\log(4/\delta)}{m}} + \sqrt{\frac{4}{\delta}\left(\frac{1}{m} +\frac{3\beta}{m+n} \right)}  \,.
    \end{align}
    

       
\end{proof}

\subsection{Extension to multiclass classification} \label{app:multiclass_linear}
For multiclass problems with squared loss minimization, as standard practice, we consider one-hot encoding for the underlying label, i.e., a class label $c \in [k]$ is treated as $(0, \cdot, 0,1,0, \cdot, 0) \in \Real^k$ (with $c$-th coordinate being 1).  As before, we suppose that the parameters of the linear function 
are obtained via gradient descent on the following $L_2$ regularized problem: 
\begin{align}
    % n in denominator is avoided deliberately
    \calL_S(w; \lambda) \defeq \sum_{i=1}^n\norm{w^Tx_i - y_i}{2}^2 + \lambda \sum_{j=1}^k \norm{w_j}{2}^2 \,, \label{eq:l2_multiclass_MSE_app}   
\end{align}
where $\lambda\ge0$ is a regularization parameter. 
We assume access to a clean dataset 
$S = \{(x_i, y_i)\}_{i=1}^n \sim \calD^n$ 
and randomly labeled dataset 
$\wt S = \{(x_i, y_i)\}_{i=n+1}^{n+m} \sim \wt \calD^m$. 
Let $\bX = [x_1, x_2, \cdots, x_{m+n}]$ 
and $\by = [e_{y_1}, e_{y_2}, \cdots, e_{y_{m+n}}]$. 
Fix a positive learning rate $\eta$ such that 
$\eta \le 1/\left(\norm{\bX^T\bX}{\text{op}} + \lambda^2\right)$ 
and an initialization $w_0 = 0$. 
% \todos{Assumption made for simplicty}. 
Consider the following gradient descent iterates 
to minimize objective \eqref{eq:l2_MSE_app} on $S \cup \wt S$:
\begin{align}
{w_j}^t = {w_j}^{t-1} - \eta \grad_{w_j} \calL_{S \cup \wt S} (w^{t-1}; \lambda) \quad \forall t=1,2,\ldots \text{ and } j=1,2,\ldots,k  \,. \label{eq:GD_multi_iterates_app}
\end{align} 
Then we have $\{ {w_j}^t\}$ for all $j =1,2,\cdots, k$ converge to the limiting solution 
$\wh w_j = \left( \bX^T\bX+\lambda \boldsymbol{I}\right)^{-1}\bX^T\by_j$. Define $\widehat f (x) \defeq f(x ; \wh w) $.  

\begin{theorem}\label{thm:multi_linear}
    Assume that this gradient descent algorithm satisfies \codref{cond:hypothesis_stability}
    with $\beta=\calO(1)$.  
    Then for a multiclass classification problem wth $k$ classes, for any $\delta >0$, with probability at least $1-\delta$, we have:
    \begin{align*}
        \error_\calD(\widehat f) \le \error_\calS(\widehat f) &+ (k-1)\left(1 - \frac{k}{k-1} \error_{\wt\calS}(\widehat f) \right) \\ &+ \left(k + \sqrt{k} + \frac{m}{n\sqrt{k}} \right) \sqrt{\frac{\log(4/\delta)}{2m}} + \sqrt{k(k-1)} \sqrt{\frac{4}{\delta}\left(\frac{1}{m} +\frac{3\beta}{m+n} \right)}  \,. \numberthis \label{eq:gd_multi_error}
    \end{align*} 
    % for some constant $c\le 3.2$.
\end{theorem}
\begin{proof}
    The proof of this theorem is divided into two parts. In the first part, we relate the error on the mislabeled samples with the population error on the mislabeled data. Similar to the proof of \thmref{thm:linear}, we use Shermann-Morrison formula to upper bound training error with leave-one-out error on each $\wh w^j$. Second part of the proof follows entirely from the proof of \thmref{thm:multiclass_ERM}. In essence, the first part derives an equivalent of \eqref{eq:lemma1_final_multi_prev} for GD training with squared loss and then the second part follows from the proof  of \thmref{thm:multiclass_ERM}. 
    
    \textbf{Part-1:} Consider a training point $(x_i,y_i)$ in $\wt S \cup S $. For simplicity, we use $c_i$ to denote the class of $i$-th point and use $y_i$ as the corresponding one-hot embedding. Recall error in multiclass point is given by $\error(\wh f(x_i), y_i ) = \indict{ c_i \not \in \argmax x_i^T \wh w }$. Thus, there exists a $j \ne c_i \in [k]$, such that we have
     \begin{align}
        \error(\wh f(x_i), y_i ) &= \indict{ c_i \not \in \argmax x_i^T \wh w } = \indict{ x_i^T \wh w_{c_i} < x_i^T \wh w_{j}  } \\ &= \indict{ x_i^T \left( \bX^T\bX+\lambda \boldsymbol{I}\right)^{-1}\bX^T\by_{c_i} < x_i^T \left( \bX^T\bX+\lambda \boldsymbol{I}\right)^{-1}\bX^T\by_{j} } \\
        &= \indict{ x_i^T \underbrace{\left( \bXX{i}^T\bXX{i} + x_i ^T x_i +\lambda \boldsymbol{I}\right)^{-1}}_{\RN{1}} \left(\bXX{i}^T{\by_{c_i}}_{(i)} + x_i - \bXX{i}^T{\by_{j}}_{(i)}\right) < 0 } \,.
    \end{align}
    Letting $\bA = \left(\bXX{i}^T\bXX{i} +\lambda \boldsymbol{I}\right)$ 
    and using \lemref{lem:sherman} on term 1, we have 
    \begin{align}
        \error(\wh f(x_i), y_i ) &= \indict{ x_i^T \left[\bA^{-1} -  \frac{\bA^{-1} x_i x_i^T \bA^{-1}}{ 1 + x_i ^T \bA^{-1} x_i } \right]  \left(\bXX{i}^T{\by_{c_i}}_{(i)} + x_i - \bXX{i}^T{\by_{j}}_{(i)}\right) < 0 } \\
        &= \indict{ \left[ \frac{ x_i^T \bA^{-1} ( 1 + x_i ^T \bA^{-1} x_i ) -  x_i^T \bA^{-1} x_i x_i^T \bA^{-1}}{ 1 + x_i ^T \bA ^{-1}x_i } \right]  \left(\bXX{i}^T{\by_{c_i}}_{(i)} + x_i - \bXX{i}^T{\by_{j}}_{(i)}\right) < 0 } \\
        &= \indict{ \left[ \frac{ x_i^T \bA^{-1}}{ 1 + x_i ^T \bA ^{-1}x_i } \right]  \left(\bXX{i}^T{\by_{c_i}}_{(i)} + x_i - \bXX{i}^T{\by_{j}}_{(i)}\right) < 0} \,.
    \end{align}
    Since $1 + x_i^T \bA^{-1} x_i > 0$, we have 
    \begin{align}
        \error(\wh f(x_i), y_i ) &= \indict{ x_i^T \bA^{-1}  \left(\bXX{i}^T{\by_{c_i}}_{(i)} + x_i - \bXX{i}^T{\by_{j}}_{(i)}\right) < 0 } \\
        &= \indict{ x_i^T \bA^{-1} x_i +  x_i^T \bA^{-1}  \bXX{i}^T{\by_{c_i}}_{(i)}  - x_i^T\bA^{-1}  \bXX{i}^T{\by_{j}}_{(i)} < 0 } \\
        &\le \indict{  x_i^T \bA^{-1}  \bXX{i}^T{\by_{c_i}}_{(i)}  - x_i^T\bA^{-1}  \bXX{i}^T{\by_{j}}_{(i)} < 0  } = \error(\ff{i}(x_i), y_i ) \,.\label{eq:LOO_error_multi}
    \end{align}
    Using \eqref{eq:LOO_error_multi}, we have 
    \begin{align}
        \error_{\wt \calS_M } (\wh f) \le \error_{\text{LOO} (\wt S_M)} \defeq \frac{\sum_{(x_i, y_i) \in \wt S_M} \error(\ff{i}(x_i), y_i ) }{\abs{\wt \calS_M}}\label{eq:LOO_error_multi_final} \,.
    \end{align}
    
    We now relate RHS in \eqref{eq:LOO_error_final} 
    with the population error on mislabeled distribution. 
    Similar as before, to do this, we leverage \codref{cond:hypothesis_stability} 
    and \lemref{lem:stability_error}. Using  \eqref{eq:final_mislabeled_linear} and \eqref{eq:LOO_error_multi_final}, we have 
    \begin{align}
        \error_{\wt \calS_M } (\wh f) \le \error_{\calDm}(\wh f)   + \sqrt{\frac{1}{\delta}\left(\frac{1}{2m_1} +\frac{3\beta}{m+n} \right)} \,. \label{eq:linear_multi_parallel_lem1}
    \end{align}
    
    We have now derived a parallel to \eqref{eq:lemma1_final_multi_prev}. Using the same arguments in the proof of \lemref{lem:fit_mislabeled_multi}, we have 
    \begin{align}
      \error_{\calD}(\wh f) \le  (k-1) \left( 1- \error_{ \wt \calS_M}(\wh f) \right)  + (k-1)\sqrt{\frac{k}{\delta(k-1)}\left(\frac{1}{2m_1} +\frac{3\beta}{m+n} \right)}  \,. \label{eq:lemma1_linear_final_multi}
    \end{align}
    
    \textbf{Part-2:} We now combine the results in \lemref{lem:mislabeled_error_multi} and \lemref{lem:clear_error_multi} to obtain the final inequality in terms of quantities that can be computed from just the randomly labeled and clean data. Similar to the binary case, we obtained a polynomial concentration instead of exponential concentration. Combining \eqref{eq:lemma1_linear_final_multi} with \lemref{lem:mislabeled_error_multi} and \lemref{lem:clear_error_multi}, we have with probability at least $1-\delta$
    \begin{align*}
        \error_\calD(\widehat f) \le \error_\calS(\widehat f) &+ (k-1)\left(1 - \frac{k}{k-1} \error_{\wt\calS}(\widehat f) \right) \\ &+ \left(k + \sqrt{k} + \frac{m}{n\sqrt{k}} \right) \sqrt{\frac{\log(4/\delta)}{2m}} + \sqrt{k(k-1)} \sqrt{\frac{4}{\delta}\left(\frac{1}{m} +\frac{3\beta}{m+n} \right)}  \,. \numberthis \label{eq:gd_multi_error_proof}
    \end{align*} 
\end{proof}

\subsection{Discussion on \codref{cond:hypothesis_stability}} \label{app:discuss_cond1}
The quantity in LHS of \codref{cond:hypothesis_stability} 
measures how much the function learned by the algorithm 
(in terms of error on unseen point) will change 
when one point in the training set is removed. 
% Discussion on exponential concentration and stronger condition. 
% Notice that hypothesis stability implies error stability, i.e., \codref{cond:error_stability} \citep{bousquet2002stability}.  
% In summary, while error stability allowed us 
% to relate the average population error 
% of the leave-one-out classifiers 
% with the population error of the original classifier, 
We need hypothesis stability condition 
to control the variance of the empirical leave-one-out error to show concentration of average leave-one-error with the population error. 

Additionally, we note that while the dominating term in the RHS of \thmref{thm:linear} matches with the dominating term in ERM bound in \thmref{thm:error_ERM}, there is a polynomial concentration term 
(dependence on $1/\delta$ instead of $\log(\sqrt{1/\delta})$) 
in \thmref{thm:linear}. 
Since with hypothesis stability, 
we just bound the variance, 
the polynomial concentration is due 
to the use of Chebyshev's inequality 
instead of an exponential tail inequality
(as in \lemref{lem:fit_mislabeled}).
Recent works have highlighted that 
a slightly stronger condition than hypothesis stability 
can be used to obtain an exponential concentration 
for leave-one-out error \citep{abou2019exponential},
but we leave this for future work for now. 
% We leave 
% However, the constants 

% we also want to highlight  

\subsection{Formal statement and proof of \propref{prop:early_stop}} \label{app:formal_early_stop}

Before formally presenting the result, 
we will introduce some notation.  
By $\calL_{S}(w)$, we denote 
the objective in \eqref{eq:l2_MSE_app} with $\lambda=0$. 
Assume Singular Value Decomposition (SVD) of $\bX$
as $\sqrt{n} \bU \bS^{1/2} \bV^T$. 
Hence $\bX^T \bX = \bV \bS \bV^T$.
Consider the GD iterates defined in \eqref{eq:GD_iterates_app}. 
% 
We now derive closed form expression 
for the $t^\text{th}$ iterate of gradient descent:  
% 
\begin{align}
    w_t = w_{t-1} + \eta \cdot \bX^T (\by - \bX w_{t-1}) = (\bI - \eta \bV \bS \bV^T )w_{k-1} + \eta \bX^T \by \,.
\end{align}
Rotating by $\bV^T$, we get 
\begin{align}
    \wt w_t = (\bI - \eta\bS )\wt w_{k-1} + \eta \wt \by \label{eq:GD_recur},
\end{align}
where $\wt w_t = \bV^T w_t $ and $\wt \by = \bV^T \bX^T \by$. 
Assuming the initial point $w_0 = 0$ 
and applying the recursion in \eqref{eq:GD_recur}, we get
\begin{align}
    \wt w_t = \bS ^{-1} ( \bI - (\bI - \eta \bS)^k ) \wt \by \,, 
\end{align} 
Projecting solution back to the original space, we have 
\begin{align}
     w_t = \bV \bS ^{-1} ( \bI - (\bI - \eta \bS)^k ) \bV^T \bX^T \by \,. 
\end{align} 
% We will work with this GD solution at any iterate $t$ in the next proposition. 
Define $f_t(x) \defeq f(x;w_t)$ 
as the solution at the $t^{\text{th}}$ iterate. 
Let $\wt w_{\lambda} = \argmin_{w} \calL_\calS (w;\lambda) = (\bX^T \bX + \lambda \bI)^{-1} \bX^T \by = \bV (\bS + \lambda \bI )^{-1} \bV^T \bX^T \by $. 
% ) \,,$ for all $t=1,2,\ldots\,.$ 
and define $\wt f_\lambda(x) \defeq f(x;\wt w_\lambda)$ as the regularized solution. 
Assume $\kappa$ be the condition number 
of the population covariance matrix 
and let $s_\text{min}$ be the minimum positive 
singular value of the empirical covariance matrix. 
Our proof idea is inspired from recent work 
on relating gradient flow solution 
and regularized solution 
for regression problems \citep{ali2018continuous}. 
We will use the following lemma in the proof: 
\begin{lemma} \label{lem:ineq_soln}
    For all $x \in [0,1]$ and for all $ k \in \mathbb{N}$, 
    we have (a) $ \frac{kx}{1+kx} \le 1- (1-x)^k$ 
    and (b) $ 1- (1-x)^k \le 2 \cdot \frac{kx}{kx+1} $.
    %  where $g(c)$ is a constant dependent on $c$. For $c = 1$, $g(c) = 2.0$.   
\end{lemma}
\begin{proof}
    % [Proof of \lemref{lem:ineq_soln}]
    % Part (a) is easy. 
    Using $ (1-x)^k \le \frac{1}{1+kx}$, we have part (a). 
    For part (b), we numerically maximize 
    $\frac{ (1+kx ) (1 - (1-x)^k) }{kx}$ 
    for all $k\ge 1$ and for all $x \in [0, 1]$.  
\end{proof}

% 
% Next, 

\begin{prop}[Formal statement of \propref{prop:early_stop}] \label{prop:formal_early_stop}
Let $\lambda = \frac{1}{t\eta}$. 
For a training point $x$, we have 
\begin{align*}
    \Expt{x \sim \calS}{(f_t(x) - \wt f_\lambda(x))^2} &\le c(t,\eta) \cdot \Expt{x \sim \calS}{f_t(x)^2} \,, %\label{eq:early_stop}
\end{align*}
where $c(t, \eta) \defeq \min( 0.25, \frac{1}{s_\text{min}^2 t^2 \eta^2})$. 
Similarly for a test point, we have 
\begin{align*}
    \Expt{x \sim \calD_\calX}{(f_t(x) - \wt f_\lambda(x))^2} &\le \kappa \cdot c(t,\eta) \cdot \Expt{x \sim \calD_\calX}{f_t(x)^2} \,. %\label{eq:early_stop}
\end{align*}
\end{prop} 

\begin{proof}
    %%%%%%%%%%%%% 
    We want to analyze the expected squared difference output 
    of regularized linear regression 
    with regularization constant $\lambda = \frac{1}{\eta t}$ 
    and the gradient descent solution at the $t^\text{th}$ iterate. 
    We separately expand the algebraic expression 
    for squared difference at a training point and a test point. 
    % We start by considering the difference  
    Then the main step is to show that 
    $\left[ \bS ^{-1} ( \bI - (\bI - \eta \bS)^k )  - (\bS + \lambda \bI )^{-1}\right] \preceq c(\eta, t) \cdot \bS ^{-1} ( \bI - (\bI - \eta \bS)^k ) $.

    %%%%%%%%%%%%%
    
   \textbf{Part 1 {} {}} 
    First, we will analyze the squared difference 
    of the output at a training point 
    (for simplicity, we refer to $S \cup \wt S$ as $S$), i.e., 
    \begin{align}
        \Expt{ x \sim \calS }{\left(f_t(x) - \wt f_\lambda (x)\right)^2} &= \norm{\bX w_t - \bX \wt w_\lambda}{2}^2\\ &=   \norm{\bX \bV \bS ^{-1} ( \bI - (\bI - \eta \bS)^t ) \bV^T \bX^T \by - \bX \bV (\bS + \lambda \bI )^{-1} \bV^T \bX^T \by }{2}^2 \\
        &= \norm{\bX \bV \left(\bS ^{-1} ( \bI - (\bI - \eta \bS)^t ) - (\bS + \lambda \bI )^{-1} \right) \bV^T \bX^T \by  }{2} \\
        &=  \by^T \bV \bX \left( \underbrace{\bS ^{-1} ( \bI - (\bI - \eta \bS)^t ) - (\bS + \lambda \bI )^{-1}}_{\RN{1}} \right)^2 \bS \bV^T \bX^T \by \label{eq:train_GD_rel} \,.
        %  (\bX \bV \bS ^{-1} ( \bI - (\bI - \eta \bS)^k ) \bV^T \bX^T \by)^T \bX \bV \bS ^{-1} ( \bI - (\bI - \eta \bS)^k ) \bV^T \bX^T \by
    \end{align}
    We now separately consider term 1. 
    Substituting $\lambda = \frac{1}{t \eta}$, 
    we get
    \begin{align}
        \bS ^{-1} ( \bI - (\bI - \eta \bS)^t ) - (\bS + \lambda \bI )^{-1} &= \bS^{-1} \left( ( \bI - (\bI - \eta \bS)^t ) - (\bI + \bS^{-1} \lambda )^{-1}\right) \\
        &= \underbrace{\bS^{-1} \left( ( \bI - (\bI - \eta \bS)^t ) - (\bI + ( \bS t \eta)^{-1}  )^{-1}\right)}_{\bA} \,.
    \end{align}

    We now separately bound the diagonal entries in matrix $\bA$. 
    With $s_i$, we denote $i^{\text{th}}$ diagonal entry of $\bS$.
    Note that since $ \eta\le 1/\norm{S}{\text{op}}$, 
    for all $i$, $\eta s_i  \le 1$.  
    Consider $i^{\text{th}}$ diagonal term (which is non-zero) 
    of the diagonal matrix $\bA$, we have 
    \begin{align}
        \bA_{ii} = \frac{1}{s_i} \left(  1 - (1 - s_i \eta)^t - \frac{t \eta s_i}{1 + t \eta s_i } \right) &=  \frac{1 - (1 - s_i \eta)^t}{s_i} \left( \underbrace{ 1 - \frac{t \eta s_i}{(1 + t \eta s_i)(1 - (1 - s_i \eta)^t)}}_{\RN{2}} \right) \\ 
         &\le \frac{1}{2}\left[ \frac{1 - (1 - s_i \eta)^t}{ s_i} \right] \tag*{(Using \lemref{lem:ineq_soln} (b))} \,.
    \end{align} 
    Additionally, we can also show the following upper bound on term 2: 
    \begin{align}
         1 - \frac{t \eta s_i}{(1 + t \eta s_i)(1 - (1 - s_i \eta)^t)} &= \frac{(1 + t \eta s_i)(1 - (1 - s_i \eta)^t) - t \eta s_i }{(1 + t \eta s_i)(1 - (1 - s_i \eta)^t)} \\
         & \le  \frac{ 1 -  (1 - s_i \eta)^t - t \eta s_i (1 - s_i \eta)^t}{(1 + t \eta s_i)(1 - (1 - s_i \eta)^t)} \\
         & \le \frac{1}{t\eta s_i} \,. \tag{Using \lemref{lem:ineq_soln} (a)}
        %  &\le \frac{1}{2}\left[ \frac{1 - (1 - s_i \eta)^t}{ s_i} \right] \tag*{(Using \lemref{lem:ineq_soln})} \,.
    \end{align} 

    Combining both the upper bounds 
    on each diagonal entry $\bA_{ii}$, we have 
    \begin{align}
    \bA \preceq c_1(\eta, t) \cdot \bS^{-1} ( \bI - (\bI - \eta \bS)^t ) \,, \label{eq:upperbound_diagonal}
    \end{align}
    where $c_1(\eta, t ) = \min(0.5, \frac{1}{t s_i \eta })$. Plugging this into \eqref{eq:train_GD_rel}, we have 
    \begin{align}
        \Expt{ x \sim \calS }{\left(f_t(x) - \wt f_\lambda (x)\right)^2} &\le c(\eta, t) \cdot \by^T \bV \bX  \left( \bS^{-1} ( \bI - (\bI - \eta \bS)^t ) \right)^2 \bS \bV^T \bX^T \by \\
        &=   c(\eta, t) \cdot \by^T \bV \bX  \left( \bS^{-1} ( \bI - (\bI - \eta \bS)^t ) \right) \bS \left( \bS^{-1} ( \bI - (\bI - \eta \bS)^t ) \right) \bV^T \bX^T \by \\
        & =  c(\eta, t) \cdot \norm{\bX w_t}{2}^2 \\
        &= c(\eta, t) \cdot  \Expt{ x \sim \calS }{\left(f_t(x) \right)^2} \,,
    \end{align}
    where $c(\eta, t ) = \min(0.25, \frac{1}{t^2 s^2_i \eta^2 })$.

    \textbf{Part 2 {} {}} With $\bSigma$, 
    we denote the underlying true covariance matrix. 
    We now consider the squared difference of output at an unseen point: 
    \begin{align}
        \Expt{ x \sim \calD_{\calX} }{\left(f_t(x) - \wt f_\lambda (x)\right)^2} &= \Expt{x \sim \calD_{\calX}}{\norm{x^T w_t - x^T \wt w_\lambda}{2}} \\
        &=   \norm{x^T \bV \bS ^{-1} ( \bI - (\bI - \eta \bS)^t ) \bV^T \bX^T \by - x^T \bV (\bS + \lambda \bI )^{-1} \bV^T \bX^T \by }{2} \\
        &= \norm{x^T \bV \left(\bS ^{-1} ( \bI - (\bI - \eta \bS)^t ) - (\bS + \lambda \bI )^{-1} \right) \bV^T \bX^T \by  }{2} \\
        &= \by^T \bV \bX \left( \bS ^{-1} ( \bI - (\bI - \eta \bS)^t ) - (\bS + \lambda \bI )^{-1} \right) \bV^T \bSigma \bV \\ &\qquad \qquad \qquad \qquad \qquad \left( (\bI - (\bI - \eta \bS)^t ) - (\bS + \lambda \bI )^{-1} \right) \bV^T \bX^T \by \\
        &\le \sigma_{\text{max}} \cdot \by^T \bV \bX \left( \underbrace{\bS ^{-1} ( \bI - (\bI - \eta \bS)^t ) - (\bS + \lambda \bI )^{-1}}_{\RN{1}} \right)^2 \bV^T \bX^T \by \,, \label{eq:test_GD_rel}
        %  (\bX \bV \bS ^{-1} ( \bI - (\bI - \eta \bS)^k ) \bV^T \bX^T \by)^T \bX \bV \bS ^{-1} ( \bI - (\bI - \eta \bS)^k ) \bV^T \bX^T \by
    \end{align}
    where $\sigma_{\text{max}}$ is the maximum eigenvalue 
    of the underlying covariance matrix $\bSigma$. 
    Using the upper bound on term 1 in \eqref{eq:upperbound_diagonal}, 
    we have 
    \begin{align}
        \Expt{ x \sim \calD_{\calX} }{\left(f_t(x) - \wt f_\lambda (x)\right)^2} &\le \sigma_{\text{max}} \cdot c(\eta, t) \cdot \by^T \bV \bX  \left( \bS^{-1} ( \bI - (\bI - \eta \bS)^t ) \right)^2 \bV^T \bX^T \by \\
        &=   \kappa \cdot c(\eta, t) \cdot \sigma_{\text{min}}\cdot \norm{\bV \left( \bS^{-1} ( \bI - (\bI - \eta \bS)^t ) \right) \bV^T \bX^T \by}{2}^2 \\
        &\le \kappa \cdot c(\eta, t) \cdot \left[ \bV \left( \bS^{-1} ( \bI - (\bI - \eta \bS)^t ) \right) \bV^T \bX^T \right]^T \bSigma \\
        &\qquad \qquad \qquad \qquad \qquad \left[ \bV \left( \bS^{-1} ( \bI - (\bI - \eta \bS)^t ) \right) \bV^T \bX^T \right] \by \\
        & = \kappa \cdot c(\eta, t) \cdot \Expt{x \sim \calD_{\calX}}{\norm{x^T w_t}{2}} \,.
    \end{align}
% 
% 
    % Since $ \eta\le 1/\norm{S}{\text{op}}$, invoking \lemref{lem:ineq_soln} to upper bound term 1 with
\end{proof}

\subsection{Extension to deep learning} \label{appsubsec:ext_DL}
Under \asmpref{appsubsec:justifying_assumption1}, we present the formal result parallel to \thmref{thm:multiclass_ERM}. 
\begin{theorem} \label{thm:multiclass_ERM_algoA}
    Consider a multiclass classification problem 
    with $k$ classes. Under \asmpref{asmp:deep_models}, 
    for any $\delta >0$, with probability at least $1-\delta$,
    we have
    \vspace{-10pt}
    \begin{align*}
        \error_\calD(\widehat f)  \le \error_\calS(\widehat f) + (k-1) \left(1 - \tfrac{k}{k-1} \error_{\wt\calS}(\widehat f)\right) + c\sqrt{\frac{\log(\frac{4}{\delta})}{2m}} \,,\numberthis \label{eq:multiclass_ERM_deep}
    % \vspace{-20pt}
    \end{align*}
    for some constant $c \le ((c+1) k+\sqrt{k} + \frac{m}{n\sqrt{k}})$.
\end{theorem}

The proof follows exactly as in step (i) to (iii) in \thmref{thm:multiclass_ERM}.  

\subsection{Justifying~\asmpref{asmp:deep_models}} \label{appsubsec:justifying_assumption1}

Motivated by the analysis on linear models, we now discuss alternate (and weaker) conditions that imply \asmpref{asmp:deep_models}. 
We need hypothesis stability (\codref{cond:hypothesis_stability}) and the following assumption relating training error and leave-one-error: 

\begin{assumption} \label{asmp:loo_error}
Let $\wh f$ be a model obtained by training with algorithm $\calA$ on a mixture of clean $S$ and randomly labeled data $\wt S$. Then we assume we have 
\begin{align*}
    \error_{\wt \calS_M} (\wh f) \le  \error_{\text{LOO} (\wt S_M)} \,, 
\end{align*}
for all $(x_i, y_i) \in  \wt S_M$ where $\wh f_{(i)} \defeq f(\calA, S \cup {{}\wt S_M}_{(i)})$ and  $\error_{\text{LOO} (\wt S_M)} \defeq  \frac{\sum_{(x_i, y_i) \in \wt S_M} \error(\ff{i}(x_i), y_i ) }{\abs{\wt \calS_M}}$.  
\end{assumption}

% we assume this to extend our result (parallel to \thmref{thm:multi_linear}) for deep models. 
Intuitively, this assumption states that the error on a (mislabeled) datum $(x,y)$ included in the training set is less than the error on that datum $(x,y)$ obtained by a model trained on the training set $S - \{(x,y)\}$. We proved this for linear models trained with GD in the proof of \thmref{thm:multi_linear}. 
% 
\codref{cond:hypothesis_stability} with $\beta = \calO(1)$ and \asmpref{asmp:loo_error} together with \lemref{lem:stability_error} implies \asmpref{asmp:deep_models} with a polynomial residual term (instead of logarithmic in $1/\delta$): 
\begin{align}
     \error_{\calS_M} (\wh f) \le  \error_{\calDm}(\wh f)   + \sqrt{\frac{1}{\delta}\left(\frac{1}{m} +\frac{3\beta}{m+n} \right)} \,.
\end{align}
% Note that this  

\newpage 
\section{Additional experiments and details}\label{app:exp}
\newcommand\tab[1][1cm]{\hspace*{#1}}

\subsection{Datasets} \label{sec:app_dataset}

\textbf{Toy Dataset {} {}} Assume fixed constants $\mu$ and $\sigma$. For a given label $y$, we simulate features $x$ in our toy classification setup as follows: 
\begin{align*}
    x \defeq \texttt{concat} \left[ x_1, x_2\right] \quad \text{where} \quad  x_1 \sim  \calN( y \cdot \mu, \sigma^2 I_{d \times d}) \ \  \text{and} \ \  x_1 \sim  \calN( 0, \sigma^2 I_{d \times d}) \,.
\end{align*}  
% where $y$ is the true label and $x$ is the corresponding feature vector. 
In experiements throughout the paper, we fix dimention $d=100$, $\mu = 1.0 $, and $\sigma = \sqrt{d}$. Intuitively, $x_1$ carries the information about the underlying label and $x_2$ is additional noise independent of the underlying label. 

\textbf{CV datasets {} {}} We use MNIST~\citep{lecun1998mnist} and CIFAR10~\cite{krizhevsky2009learning}. 
% For binary tasks, 
We produce a binary variant from the multiclass classification problem by mapping classes $\{0,1,2,3,4\}$ to label $1$ and $\{ 5,6,7,8,9\}$ to label $-1$. For CIFAR dataset, we also use the standard data augementation of random crop and horizontal flip. PyTorch code is as follows: 

\texttt{(transforms.RandomCrop(32, padding=4),\\
\tab transforms.RandomHorizontalFlip())}

\textbf{NLP dataset {} {}} We use IMDb Sentiment analysis~\citep{maas2011learning} corpus.  

\subsection{Architecture Details} 

All experiments were run on NVIDIA GeForce RTX 2080 Ti GPUs. We used PyTorch~\citep{NEURIPS2019a9015} and Keras with Tensorflow~\citep{abadi2016tensorflow} backend for experiments. 
% , ELMo embeddings~\citep{Peters:2018}, and Hugging Face Transformers~\citep{wolf-etal-2020-transformers}. 

\textbf{Linear model {} {}} For the toy dataset, we simulate a linear model with scalar output and the same number of parameters as the number of dimensions.   

\textbf{Wide nets {} {}} To simulate the NTK regime, we experiment with $2-$layered wide nets. The PyTorch code for 2-layer wide MLP is as follows: 


\texttt{ nn.Sequential( \\
\tab     nn.Flatten(),\\
\tab    nn.Linear(input\_dims, 200000, bias=True),\\
\tab    nn.ReLU(),\\
\tab    nn.Linear(200000, 1, bias=True)\\
\tab     )}


We experiment both (i) with the second layer fixed at random initialization; (ii)  and updating both layers' weights.     

\textbf{Deep nets for CV tasks {} {}} We consider a 4-layered MLP. The PyTorch code for 4-layer MLP is as follows: 

\texttt{ nn.Sequential(nn.Flatten(), \\
\tab        nn.Linear(input\_dim, 5000, bias=True),\\
\tab        nn.ReLU(),\\
\tab        nn.Linear(5000, 5000, bias=True),\\
\tab        nn.ReLU(),\\
\tab        nn.Linear(5000, 5000, bias=True),\\
\tab        nn.ReLU(),\\
% \tab        nn.Linear(5000, 5000, bias=True),\\
% \tab        nn.ReLU(),\\
\tab        nn.Linear(1024, num\_label, bias=True)\\
\tab        )}

For MNIST, we use $1000$ nodes instead of $5000$ nodes in the hidden layer. 
% 
We also experiment with convolutional nets. In particular, we use ResNet18 \citep{he2016deep}. Implementation adapted from:  \url{https://github.com/kuangliu/pytorch-cifar.git}. 

\textbf{Deep nets for NLP {} {}} We use a simple LSTM model with embeddings intialized with ELMo embeddings~\citep{Peters:2018}. Code adapted from: \url{https://github.com/kamujun/elmo_experiments/blob/master/elmo_experiment/notebooks/elmo_text_classification_on_imdb.ipynb} 

We also evaluate our bounds with a BERT model. In particular, we fine-tune an off-the-shelf uncased BERT model~\citep{devlin2018bert}. Code adapted from Hugging Face Transformers~\citep{wolf-etal-2020-transformers}: \url{https://huggingface.co/transformers/v3.1.0/custom_datasets.html}. 


\subsection{Additonal experiments}

\textbf{Results with SGD on underparameterized linear models {} {}} 

\begin{figure*}[h]
    \centering 
    % \vspace{-15pt}
    % \includegraphics[width=0.9\linewidth]{example-image-a}
    \includegraphics[width=0.3\linewidth]{figures/lowdim-Gaussian-SGD.pdf}
    % \includegraphics[width=0.9\linewidth]{figures/{CIFAR10_rn=0.1_lr=0.2_wd=0.005}.png}
    \vspace{-5pt}
    \caption{ 
    % Predicted lower bound 
    % on different
    We plot the accuracy and corresponding bound 
    (RHS in \eqref{eq:erm}) at $\delta = 0.1$
    for toy binary classification task. 
    Results aggregated over $3$ seeds. 
    % i.e., $1-\error$ where $\error$ is the term in the RHS of \eqref{eq:erm}
    Accuracy vs fraction of unlabeled data (w.r.t clean data) 
    in the toy setup with a linear model trained with SGD. Results parallel to \figref{fig:error_binary}(a) with SGD.  }
    \label{fig:error_binary_linear}
    \vspace{-5pt}
\end{figure*}

\textbf{Results with wide nets on binary MNIST {} {}}

\begin{figure*}[h]
    \centering 
    % \vspace{-15pt}
    % \includegraphics[width=0.9\linewidth]{example-image-a}
    \subfigure[GD with MSE loss]{\includegraphics[width=0.3\linewidth]{figures/MNIST-GD_MSE.pdf}} \hfil
    \subfigure[SGD with CE loss]{\includegraphics[width=0.3\linewidth]{figures/MNIST-SGD_CE.pdf}}
    \subfigure[SGD with MSE loss]{\includegraphics[width=0.3\linewidth]{figures/MNIST-SGD_MSE-first-layer.pdf}}
    % \includegraphics[width=0.9\linewidth]{figures/{CIFAR10_rn=0.1_lr=0.2_wd=0.005}.png}
    \vspace{-5pt}
    \caption{ 
    % Predicted lower bound 
    % on different
    We plot the accuracy and corresponding bound 
    (RHS in \eqref{eq:erm}) at $\delta = 0.1$ 
    for binary MNIST classification. 
    Results aggregated over $3$ seeds. 
    % i.e., $1-\error$ where $\error$ is the term in the RHS of \eqref{eq:erm}
    Accuracy vs fraction of unlabeled data 
    for a 2-layer wide network on binary MNIST with both the layers training in (a,b) and only first layer training in (c). 
    Results parallel to \figref{fig:error_binary}(b) .  }
    \label{fig:error_binary_MNIST}
    \vspace{-5pt}
\end{figure*}

% \begin{figure*}[h]
%     \centering 
%     % \vspace{-15pt}
%     % \includegraphics[width=0.9\linewidth]{example-image-a}
%     \subfigure[GD with MSE loss]{\includegraphics[width=0.3\linewidth]{figures/MNIST.pdf}} \hfil
    
%     \subfigure[SGD with CE loss]{\includegraphics[width=0.3\linewidth]{figures/MNIST.pdf}}
%     % \includegraphics[width=0.9\linewidth]{figures/{CIFAR10_rn=0.1_lr=0.2_wd=0.005}.png}
%     \vspace{-5pt}
%     \caption{ 
%     % Predicted lower bound 
%     % on different
%     We plot the accuracy and corresponding bound 
%     (RHS in \eqref{eq:erm}) at $\delta = 0.1$
%     for binary MNIST classification. 
%     Results aggregated over $3$ seeds. 
%     % i.e., $1-\error$ where $\error$ is the term in the RHS of \eqref{eq:erm}
%     Accuracy vs fraction of unlabeled data 
%     for a 2-layer wide network on binary MNIST with just the first layer training. 
%     Results parallel to \figref{fig:error_binary}(b) with only the first layer training.  }
%     \label{fig:error_binary_MNIST}
%     \vspace{-5pt}
% \end{figure*}

\textbf{Results on CIFAR 10 and MNIST {} {}} 
% 
We plot epoch wise error curve for results in \tabref{table:multiclass}(\figref{fig:error_epoch_CIFAR10} and \figref{fig:error_epoch_MNIST}). We observe the same trend as in \figref{fig:error_CIFAR10}. Additionally, we plot an \emph{oracle bound} obtained by tracking the error on mislabeled data which nevertheless were predicted as true label. To obtain an exact emprical value of the oracle bound, we need underlying true labels for the randomly labeled data. 
% Note that our bound in \thmref{thm:multiclass_ERM}, lower bounds the accuracy as predicted by the oracle bound. 
While with just access to extra unlabeled data we cannot calculate oracle bound, we note that the oracle bound is very tight and never violated in practice underscoring an importamt aspect of generalization in multiclass problems. This highlight that even a stronger conjecture may hold in multiclass classification, i.e., error on mislabeled data (where nevertheless true label was predicted) lower bounds the population error on the distribution of mislabeled data and hence, the error on (a specific) mislabeled portion predicts the population accuracy on clean data. 
% 
On the other hand, the dominating term of in \thmref{thm:multiclass_ERM} is loose when compared with the oracle bound. The main reason, we believe is the pessimistic upper bound in \eqref{eq:lemma1_final_multi_prev} in the proof of \lemref{lem:fit_mislabeled_multi}. We leave an investigation on this gap for future. 
% of fit 

% However, oracle bound highlights two . One,  



\begin{figure}[h]
    \centering 
    % \vspace{-15pt}
    % \includegraphics[width=0.9\linewidth]{example-image-a}
    \subfigure[MLP]{\includegraphics[width=0.3\linewidth]{figures/CIFAR10-FNN.pdf}} \hfil
    \subfigure[ResNet]{\includegraphics[width=0.3\linewidth]{figures/CIFAR10-Resnet.pdf}}
    % \includegraphics[width=0.9\linewidth]{figures/{CIFAR10_rn=0.1_lr=0.2_wd=0.005}.png}
    % \vspace{-10pt}
    \caption{ Per epoch curves for CIFAR10 corresponding results in \tabref{table:multiclass}. As before, we just plot the dominating term in the RHS of \eqref{eq:multiclass_ERM} as predicted bound. Additionally, we also plot the predicted lower bound by the error on mislabeled data which nevertheless were predicted as true label. We refer to this as ``Oracle bound''. See text for more details. 
    % 
    % except for the stopping point. 
    % The bound predicted by RATT (RHS in \eqref{eq:multiclass_ERM}) is vacuous. 
    }\label{fig:error_epoch_CIFAR10}
    % \vspace{-15pt}
\end{figure}


\begin{figure}[h]
    \centering 
    % \vspace{-15pt}
    % \includegraphics[width=0.9\linewidth]{example-image-a}
    \subfigure[MLP]{\includegraphics[width=0.3\linewidth]{figures/MNIST-FNN.pdf}} \hfil
    \subfigure[ResNet]{\includegraphics[width=0.3\linewidth]{figures/MNIST-Resnet.pdf}}
    % \includegraphics[width=0.9\linewidth]{figures/{CIFAR10_rn=0.1_lr=0.2_wd=0.005}.png}
    % \vspace{-10pt}
    \caption{ Per epoch curves for MNIST corresponding results in \tabref{table:multiclass}. As before, we just plot the dominating term in the RHS of \eqref{eq:multiclass_ERM} as predicted bound. Additionally, we also plot the predicted lower bound by the error on mislabeled data which nevertheless were predicted as true label. We refer to this as ``Oracle bound''. See text for more details. 
    % 
    % except for the stopping point. 
    % The bound predicted by RATT (RHS in \eqref{eq:multiclass_ERM}) is vacuous. 
    }\label{fig:error_epoch_MNIST}
    % \vspace{-15pt}
\end{figure}

\textbf{Results on CIFAR 100 {} {}} 
% 
On CIFAR100, our bound in \eqref{eq:multiclass_ERM} yields vacous bounds. However, the oracle bound as explained above yields tight guarantees in the initial phase of the learning (i.e., when learning rate is less than $0.1$) (\figref{fig:error_CIFAR100}).  

\begin{figure}[h]
    \centering 
    % \vspace{-15pt}
    % \includegraphics[width=0.9\linewidth]{example-image-a}
    \includegraphics[width=0.3\linewidth]{figures/CIFAR100-Resnet.pdf}
    % \includegraphics[width=0.9\linewidth]{figures/{CIFAR10_rn=0.1_lr=0.2_wd=0.005}.png}
    % \vspace{-10pt}
    \caption{ Predicted lower bound by the error on mislabeled data which nevertheless were predicted as true label with ResNet18 on CIFAR100. We refer to this as ``Oracle bound''. See text for more details. 
    % 
    % except for the stopping point. 
    The bound predicted by RATT (RHS in \eqref{eq:multiclass_ERM}) is vacuous. 
    }\label{fig:error_CIFAR100}
    % \vspace{-15pt}
\end{figure}


% \paragraph{Experiments on CIFAR100} 


% \subsection{Model Selection using RATT}


\subsection{Hyperparameter Details}


\textbf{\figref{fig:error_CIFAR10} {} {}} We use clean training dataset of size $40,000$. We fix the amount of unlabeled data at $20\%$ of the clean size, i.e. we include additional $8,000$ points with randomly assigned labels. We use test set of $10,000$ points. For both MLP and ResNet, we use SGD with an initial learning rate of $0.1$ and momentum $0.9$. We fix the weight decay parameter at $5\times 10^{-4}$. After $100$ epochs, we decay the learning rate to $0.01$. We use SGD batch size of $100$. 

\textbf{\figref{fig:error_binary} (a) {} {}} We obtain a toy dataset according to the process described in \secref{sec:app_dataset}. We fix $d=100$ and create a dataset of $50,000$ points with balanced classes. Moreover, we sample additional covariates with the same procedure to create randomly labeled dataset. For both SGD and GD training, we use a fixed learning rate $0.1$.    

\textbf{\figref{fig:error_binary} (b) {} {}} Similar to binary CIFAR, we use clean training dataset of size $40,000$ and fix the amount of unlabeled data at $20\%$ of the clean dataset size. To train wide nets, we use a fixed learning of $0.001$ with GD and SGD. We decide the weight decay parameter and the early stopping point that maximizes our generalization bound (i.e. without peeking at unseen data ).  We use SGD batch size of $100$. 

\textbf{\figref{fig:error_binary} (c) {} {}} With IMDb dataset, we use a clean dataset of size $20,000$ and as before, fix the amount of unlabeled data at $20\%$ of the clean data. To train ELMo model, we use Adam optimizer with a fixed learning rate $0.01$ and weight decay $10^{-6}$ to minimize cross entropy loss. We train with batch size $32$ for 3 epochs. To fine-tune BERT model, we use Adam optimizer with learning rate $5\times 10^{-5}$ to minimize cross entropy loss. We train with a batch size of $16$ for 1 epoch.    

\textbf{\tabref{table:multiclass} {} {}} For multiclass datasets, we train both MLP and ResNet with the same hyperparameters as described before. We sample a clean training dataset of size $40,000$ and fix the amount of unlabeled data at $20\%$ of the clean size. We use SGD with an initial learning rate of $0.1$ and momentum $0.9$. We fix the weight decay parameter at $5\times 10^{-4}$. After $30$ epochs for ResNet and after $50$ epochs for MLP, we decay the learning rate to $0.01$.  We use SGD with batch size $100$. 
For \figref{fig:error_CIFAR100}, we use the same hyperparameters as 
CIFAR10 training, except we now decay learning rate after $100$ epochs. 


In all experiments, to identify the best possible accuracy on just the clean data, we use the exact same set of hyperparamters except the stopping point. We choose a stopping point that maximizes test performance. 

\subsection{Summary of experiments }

\begin{center}
    \begin{table}[H] 
        \centering
        \begin{tabular}{|c|c|c|c|} 
        \hline
        Classification type & Model category & Model & Dataset  \\ [0.5ex] 
        \hline
        \hline
        \multirow{10}{*}{Binary} & Low dimensional & Linear model & Toy Gaussain dataset  \\
                        \cline{2-4}
                         & Overparameterized 
                        %  & Linear model & Toy Gaussain dataset \\
                        %  \cline{3-4}
                        %  & & 2-layer wide net& Toy Gaussain dataset \\
                        %  \cline{3-4}
                         & \multirow{2}{*}{2-layer wide net} & \multirow{2}{*}{Binary MNIST} \\
                         & linear nets & &  
                         \\
                         \cline{2-4}                 
                         & \multirow{6}{*}{Deep nets} & \multirow{2}{*}{MLP} & Binary MNIST \\
                         \cline{4-4}
                         & &  & Binary CIFAR \\
                         \cline{3-4}
                         &  & \multirow{2}{*}{ResNet} & Binary MNIST \\
                         \cline{4-4}
                         & &  & Binary CIFAR \\
                         \cline{3-4}
                         &  & ELMo-LSTM model & IMDb Sentiment Analysis \\
                         \cline{3-4}
                         & & BERT pre-trained model & IMDb Sentiment Analysis \\
        \hline
        \multirow{5}{*}{Multiclass} & \multirow{5}{*}{Deep nets} & \multirow{2}{*}{MLP} & MNIST \\
                        \cline{4-4} 
                        & & & CIFAR10 \\                   
                        \cline{3-4}
                         &   & \multirow{3}{*}{ResNet} & MNIST \\
                         \cline{4-4}
                         &   & & CIFAR10 \\
                         \cline{4-4}
                         &   & & CIFAR100 \\
        \hline
        \end{tabular}
        % \caption{Summary of experiments performed} \label{table:experiments}
    \end{table}    
    % \footnotetext[6]{We use both MSE loss and cross-entropy loss.}
    % \footnotetext[6]{We try 2 variants: one with a fixed first layer and the other with both layers trainable.}
\end{center}

\newpage
\section{Proof of \lemref{lem:stability_error}} \label{app:proof_lem_error}

\begin{proof}[Proof of \lemref{lem:stability_error}]
    Recall, we have a training set $S \cup \wt S_C$. We defined leave-one-out error on mislabeled points as $$\error_{\text{LOO}(\wt S_M) } = \frac{\sum_{(x_i, y_i) \in \wt S_M} \error( f_{(i)}( x_i), y_i)}{ \abs{\wt S_M }} \,, $$
    where $f_{(i)} \defeq f(\calA, (S \cup \wt S)_{(i)})$. Define $S^\prime \defeq S \cup \wt S$. Assume $(x,y)$ and $(x^\prime,y^\prime)$ as i.i.d. samples from ${\calDm}$. 
    Using Lemma 25 in \citet{bousquet2002stability}, we have
    \begin{align*}
        \Expo{ \left( \error_{\calDm}(\wh f) -\error_{\text{LOO}(\wt S_M) } \right)^2 } \le & \Expt{ S^\prime, (x,y), (x^\prime,y^\prime) }{ \error(\wh f(x), y ) \error(\wh f(x^\prime), y^\prime )} - 2 \Expt{ S^\prime, (x,y) }{ \error(\wh f(x), y ) \error(f_{(i)}(x_i), y_i )} \\
        & + \frac{m_1-1}{m_1}\Expt{ S^\prime }{  \error(f_{(i)}(x_i), y_i )  \error(f_{(j)}(x_j), y_j )} + \frac{1}{m_1} \Expt{ S^\prime }{  \error(f_{(i)}(x_i), y_i ) } \,. \numberthis \label{eq:main_reln}
    \end{align*}
    We can rewrite the equation above as : 
    \begin{align*}
        \Expo{ \left( \error_{\calDm}(\wh f) -\error_{\text{LOO}(\wt S_M) } \right)^2 } \le &  \, \underbrace{\Expt{ S^\prime, (x,y), (x^\prime,y^\prime) }{ \error(\wh f(x), y ) \error(\wh f(x^\prime), y^\prime ) - \error(\wh f(x), y ) \error(f_{(i)}(x_i), y_i )}}_{\RN{1}} \\
        & + \underbrace{\Expt{ S^\prime }{  \error(f_{(i)}(x_i), y_i )  \error(f_{(j)}(x_j), y_j ) -  \error(\wh f(x), y ) \error(f_{(i)}(x_i), y_i )}}_{\RN{2}} \\ &+ \underbrace{\frac{1}{m_1} \Expt{ S^\prime }{  \error(f_{(i)}(x_i), y_i ) - \error(f_{(i)}(x_i), y_i )  \error(f_{(j)}(x_j), y_j ) }}_{\RN{3}} \,. \numberthis \label{eq:main_reln2}
    \end{align*}
    
    We will now bound term $\RN{3}$.  Using Cauchy-Schwarz's inequality, we have
    
    \begin{align}
        \Expt{ S^\prime }{  \error(f_{(i)}(x_i), y_i ) - \error(f_{(i)}(x_i), y_i )  \error(f_{(j)}(x_j), y_j ) }^2 &\le  \Expt{ S^\prime }{  \error(f_{(i)}(x_i), y_i ) }^2 \Expt{S^\prime}{1 -   \error(f_{(j)}(x_j), y_j ) }^2 \\
        &\le \frac{1}{4} \,.\label{eq:term1_lem12}
    \end{align}
    
    Note that since $(x_i,y_i)$, $(x_j ,y_j )$, $(x,y)$, and $(x^\prime, y^\prime)$ are all from same distribution $\calDm$, we directly incorporate the bounds on term $\RN{1}$ and $\RN{2}$ from the proof of Lemma 9 in \citet{bousquet2002stability}. Combining that with \eqref{eq:term1_lem12} and our definition of hypothesis stability in \codref{cond:hypothesis_stability}, we have the required claim. 
    
    
    % We now re-write term $\RN{1}$ as
    % \begin{align*}
    %         &\Expt{S^\prime, (x,y), (x^\prime,y^\prime) }{ \error(\wh f(x), y ) \error(\wh f(x^\prime), y^\prime ) - \error(\wh f(x), y ) \error(f_{(i)}(x_i), y_i )} \\ & \qquad = \Expt{ S^\prime, (x,y), (x^\prime,y^\prime) }{ \error(\wh f(x), y ) \error(\wh f  (x^\prime), y^\prime ) - \error(\wh f ^\prime(x), y ) \error(f_{(i)}(x^\prime), y^\prime )} \tag{Exchanging $(x_i, y_i)$ with $(x^\prime, y^\prime)$ in the second term} \\
    %         & \qquad = \Expt{ S^\prime, (x,y), (x^\prime,y^\prime) }{  \left(\error(\wh f(x), y )-  \error(f_{(i)}(x), y ) \right) \error(\wh f  (x^\prime), y^\prime )  } \\
    %         & \qquad  + \Expt{ S^\prime, (x,y), (x^\prime,y^\prime) }{  \left(\error(f_{(i)}(x), y ) -\error(\wh f ^\prime(x), y ) \right) \error(\wh f  (x^\prime), y^\prime )}  \\
    %         & \qquad +\Expt{ S^\prime, (x,y), (x^\prime,y^\prime) }{  \left( \error(\wh f  (x^\prime), y^\prime ) -  \error(f_{(i)}(x^\prime), y^\prime ) \right) \error(\wh f ^\prime(x), y ) }  \,, \numberthis \label{eq:term1_final}
    % \end{align*}
    % where $\wh f^\prime$ is the classifier obtained by training on $ S^\prime_{(i)} \cup \{ (x^\prime, y^\prime) \} $. Similarly we can re-write term $\RN{2}$ as 
    % \begin{align*}
    %     & \Expt{ S^\prime }{  \error(f_{(i)}(x_i), y_i )  \error(f_{(j)}(x_j), y_j ) -  \error(\wh f(x), y ) \error(f_{(i)}(x_i), y_i )} \\
    %     &\quad  = \Expt{ S^\prime, (x,y), (x^\prime,y^\prime)}{  \error(f^{\prime\prime}_{(i)}(x), y )  \error(f_{(j)}^{\prime}(x^\prime), y^\prime ) -  \error(\wh f(x), y ) \error(f_{(i)}(x_i), y_i )} \tag{Exchanging $(x_i, y_i)$ with $(x, y)$ and $(x_j, y_j)$ with $(x^\prime, y^\prime)$ in the first term}\\
    %     &\quad = \Expt{ S^\prime, (x,y), (x^\prime,y^\prime)}{  \error(f^{\prime\prime}_{(j)}(x), y )  \error(f_{(i)}^{\prime}(x^\prime), y^\prime ) -  \error(\wh f^\prime (x), y ) \error(f^\prime_{(j)}(x^\prime), y^\prime )} \tag{Exchanging $(x_i, y_i)$ and $(x_j, y_j)$ and then replacing $(x_j, y_j)$ with $(x^\prime, y^\prime)$ in the second term} \\
    %     & \quad = \Expt{ S^\prime, (x,y), (x^\prime,y^\prime) }{  \left( \error(f_{(i)}^{\prime}(x^\prime), y^\prime )   -  \error(\wh f^{\prime\prime}  (x^\prime), y^\prime ) \right)  \error(f^{\prime\prime}_{(j)}(x), y )   } \\
    %     & \quad  + \Expt{ S^\prime, (x,y), (x^\prime,y^\prime) }{  \left( \error(f^{\prime\prime}_{(j)}(x), y )  -\error(\wh f ^\prime(x), y ) \right) \error(\wh f^{\prime\prime}  (x^\prime), y^\prime )  }  \\
    %     & \quad+ \Expt{ S^\prime, (x,y), (x^\prime,y^\prime) }{  \left( \error(\wh f^{\prime\prime}  (x^\prime), y^\prime )  -  \error(f^\prime_{(j)}(x^\prime), y^\prime ) \right)  \error(\wh f^\prime (x), y ) }   \\
    %     & \quad = \Expt{ S^\prime, (x,y), (x^\prime,y^\prime) }{  \left( \error(f_{(i)}^{\prime}(x^\prime), y^\prime )   -  \error(\wh f (x^\prime), y^\prime ) \right)  \error(f_{(i)}(x_j), y_j )   } \\
    %     & \quad  + \Expt{ S^\prime, (x,y), (x^\prime,y^\prime) }{  \left( \error(f^{\prime\prime}_{(j)}(x), y )  -\error(\wh f (x), y ) \right) \error(\wh f^{\prime\prime}  (x_j), y_j )  }  \\
    %     & \quad+ \Expt{ S^\prime, (x,y), (x^\prime,y^\prime) }{  \left( \error(\wh f^{\prime\prime}  (x^\prime), y^\prime )  -  \error(f^\prime_{(j)}(x^\prime), y^\prime ) \right)  \error(\wh f^\prime (x^\prime), y^\prime ) }  \,, \numberthis \label{eq:term2_final}
    % \end{align*}
    % where $f^{\prime\prime}_{(j)}$ is trained on $S^\prime_{(j,i)} \cup {(x,y)}$, $f^{\prime}_{(i)}$ is trained on $S^\prime_{(j,i)} \cup {(x^\prime,y^\prime)}$, and $\wh f^{\prime\prime} $ is trained on $S^\prime_{(j)} \cup {(x,y)}$. Note in the last line we replaced $(x,y)$ by $(x_j, y_j)$ in the first term, replaced $(x^\prime,y^\prime)$ by $(x_j, y_j)$ in the second term and exchanged $(x_i,y_i)$ with $(x_j,y_j)$ and also $(x,y)$ and $(x^\prime, y^\prime)$
    
    
\end{proof}


% 
% 16th Century Version Control 
% 

% \onecolumn

% \section*{Supplementary Material}
% We will be using the following standard results
% on exponential concentration of random variables 
% all throughout the discussion:

% \begin{lemma}[Hoeffding's inequality for independent RVs~\citep{hoeffding1994probability}] Let $Z_1, Z_2, \ldots, Z_n$ be independent bounded random variables with $Z_i \in [a,b]$ for all $i$, then 
%     \begin{align*}
%         \prob\left( \frac{1}{n} \sum_{i=1}^n (Z_i - \Expo{Z_i}) \ge t \right) \le \exp{\left( -\frac{2nt^2}{(b-a)^2} \right) }
%     \end{align*} 
%     and 
%     \begin{align*}
%         \prob\left( \frac{1}{n} \sum_{i=1}^n (Z_i - \Expo{Z_i}) \le -t \right) \le \exp{\left( -\frac{2nt^2}{(b-a)^2} \right) }
%     \end{align*} 
%     for all $t \ge 0$. 
% \end{lemma}

% \begin{lemma}[Hoeffding's inequality for sampling with replacement~\citep{hoeffding1994probability}] \label{lem:hoeffding_sampling} Let $\calZ = (Z_1, Z_2, \ldots, Z_N)$ be a finite population of $N$ points with $Z_i \in [a.b]$ for all $i$. Let $X_1, X_2, \ldots X_n$ be a random sample drawn without replacement from $\calZ$. Then for all $t \ge 0$, we have 
%     \begin{align*}
%         \prob\left( \frac{1}{n} \sum_{i=1}^n (X_i - \mu ) \ge t \right) \le \exp{\left( -\frac{2nt^2}{(b-a)^2} \right) }
%     \end{align*} 
%     and 
%     \begin{align*}
%         \prob\left( \frac{1}{n} \sum_{i=1}^n (X_i - \mu ) \le -t \right) \le \exp{\left( -\frac{2nt^2}{(b-a)^2} \right) } \,,
%     \end{align*} 
%     where $\mu = \frac{1}{N} \sum_{i=1}^{N} Z_i$. 
% \end{lemma}

% We now discuss one condition that generalizes the exponential concentration to dependent random variables.
% \begin{condition}[Bounded difference inequality] \label{cond:BDC} Let $\calZ$ be some set and $\phi: \calZ^n \to \Real$. We say that $\phi$ satisfies the bounded difference assumption if 
% there exists $c_1, c_2, \ldots c_n \ge 0$ s.t. for all $i$, we have 
% \begin{align*}
%     \sup_{Z_1,Z_2, \ldots,Z_n, Z_i^\prime in \calZ^{n+1} } \abs{\phi (Z_1, \ldots, Z_i, \ldots, Z_n ) - \phi (Z_1, \ldots, Z_i^\prime, \ldots, Z_n ) } \le c_i \,.
% \end{align*} 
% \end{condition}

% \begin{lemma}[McDiarmid’s inequality~\citep{mcdiarmid1989}] \label{lem:McDiarmid} Let $Z_1, Z_2, \ldots, Z_n$ be independent random variables on set $\calZ$ and $\phi : \calZ^n \to \Real$ satisfy bounded difference assumption (\codref{cond:BDC}). Then for all $t>0$, we have 
%     \begin{align*}
%         \prob\left( \phi(Z_1, Z_2, \ldots, Z_n) - \Expo{\phi(Z_1, Z_2, \ldots, Z_n)} \ge t \right) \le \exp{\left( -\frac{2t^2}{\sum_{i=1}^n c_i^2} \right) } 
%     \end{align*} 
%     and 
%     \begin{align*}
%         \prob\left( \phi(Z_1, Z_2, \ldots, Z_n) - \Expo{\phi(Z_1, Z_2, \ldots, Z_n)} \le -t \right) \le \exp{\left( -\frac{2t^2}{\sum_{i=1}^n c_i^2} \right) } \,
%     \end{align*} 
% \end{lemma}


% \section{Proofs from \secref{sec:ERM_training}}\label{app:proof_erm}

% \textbf{Additional notation {} {}} Let $m_1$ be the number of mislabeled points ($\wt S_M$) and $m_2$ be the number of correctly labeled points ($\wt S_C$). Note $m_1 + m_2 = m$. 


% \subsection{Proof of \thmref{thm:error_ERM}}


% \begin{proof}[Proof of \lemref{lem:fit_mislabeled}] 
%     The main idea of our proof is to regard 
%     the clean portion of the data 
%     ($S \cup \wt S_C$) as fixed.   
%     Then, there exists a classifier $f^*$ 
%     that is optimal over draws 
%     of the mislabeled data $\wt S_M$. 
% % 
%     % 
%     Formally, 
%     \begin{align}
%     f^* \defeq \argmin_{f \in \calF} \error_{\widecheck {\calD}} (f) \,, \label{eq:modified_ERM}
%     \end{align}
%     where $$\widecheck \calD = \frac{n}{m+n} \calS + \frac{m_1}{m+n} \wt \calS_C  + \frac{m_2}{m+n}\calDm \,.$$ That is, $\widecheck \calD$ a combination of 
%     the \emph{empirical distribution} 
%     over correctly labeled data $S \cup \wt S_C$
%     % in $S\cup \wt S$ 
%     and the (population) distribution 
%     over mislabeled data $\calDm$.
%     Recall that 
%     \begin{align}
%     \wh f \defeq \argmin_{f \in \calF} \error_{\calS \cup \wt S} (f) \,. \label{eq:orig_ERM}
%     \end{align}
%     % 
%     % 
%     Since, $\widehat f$ minimizes 0-1 error 
%     on $S \cup \wt S$, using ERM optimality on \eqref{eq:orig_ERM},  
%     we have 
%     \begin{align}
%         \error_{\calS \cup \wt \calS}(\widehat f) \le \error_{
%             \calS \cup \wt \calS}(f^*) \,.    \label{eq:step1}
%     \end{align}
%     Moreover, since $f^*$ is independent of $\wt S_M$, using Hoeffding's bound,
%     % \footnote{For a fully rigorous argument,
%     % refer to the complete proof in App.~\ref{app:proof_erm}.} 
%     we have with probability at least $1-\delta$ that
%     \begin{align}
%       \error_{\wt \calS_M}(f^*) \le \error_{ \calDm}(f^*) +  \sqrt{\frac{\log(1/\delta)}{2 m_1}} \,. \label{eq:step2} 
%     \end{align}
%     %$ 
%     %for some constant $c_1\le 1/2$. 
%     Finally, since $f^*$ is the optimal classifier on $\widecheck \calD$, 
%     we have 
%     \begin{align}
%         \error_{\widecheck \calD}(f^*) \le \error_{\widecheck \calD}(\widehat f) \label{eq:step3}
%     \end{align}
%      Now to relate \eqref{eq:step1} and \eqref{eq:step3}, we can re-write the \eqref{eq:step2} as follows: 
%     \begin{align}
%         \error_{\calS \cup \wt\calS}(f^*) \le \error_{ \widecheck \calD}(f^*) +  \frac{m_1}{m+n}\sqrt{\frac{\log(1/\delta)}{2 m_1}} \,. \label{eq:step4} 
%     \end{align}
%     Now we combine equations \eqref{eq:step1}, \eqref{eq:step4}, and \eqref{eq:step3}, to get 
%     \begin{align}
%         \error_{\calS \cup \wt \calS}(\wh f) \le \error_{\widecheck \calD}(\wh f) +  \frac{m_1}{m+n}\sqrt{\frac{\log(1/\delta)}{2 m_1}} \,, 
%     \end{align}
%     which implies 
%     \begin{align}
%         \error_{ \wt \calS_M}(\wh f) \le \error_{\calDm}(\wh f) + \sqrt{\frac{\log(1/\delta)}{2 m_1}} \,. \label{eq:lemma1_final}
%     \end{align}
%     Since $\wt S$ is obtained by randomly labeling an unlabeled dataset, we assume $2m_1 \approx m$ \footnote{Formally, with probability at least $1-\delta$, we have  $(m - 2m_1)\le \sqrt{m\log(1/\delta)/2}$ }. Moreover, using $\error_{\calDm} = 1 - \error_{\calD}$ we obtain the desired result.   
%     % Combining the above steps and using the fact 
%     % that $\error_\calD = 1- \error_{\calDm} $, 
%     % we obtain the desired result.
% \end{proof}

% \begin{proof}[Proof of \lemref{lem:mislabeled_error}]
%     Recall $\error_{\wt S} (f) = \frac{m_1}{m} \error_{\wt S_M}(f) + \frac{m_2}{m} \error_{\wt S_C}(f)$. Hence, we have 
%     \begin{align}
%         2\error_{\wt S}(f) - \error_{\wt S_M}(f) - \error_{\wt S_C}(f) &= \left(\frac{2m_1}{m} \error_{\wt S_M}(f) - \error_{\wt S_M}(f)\right) + \left(\frac{2m_2}{m} \error_{\wt S_C}(f) - \error_{\wt S_C}(f)\right) \\ &= \left(\frac{2m_1}{m} - 1\right) \error_{\wt S_M}(f) + \left(\frac{2m_2}{m} - 1 \right)\error_{\wt S_C} (f) \,.
%     \end{align} 
%     Since the dataset is randomly labeled, with probability at least $1-\delta$, we have  $\left(\frac{2m_1}{m} - 1\right) \le \sqrt{\frac{\log(1/\delta)}{2m}}$. Similarly, we have with probability at least $1-\delta$, $\left(\frac{2m_2}{m} - 1\right) \le \sqrt{\frac{\log(1/\delta)}{2m}}$. Using union bound, we have with probability at least $1-\delta$
%     % \begin{align}
%     %     2\error_{\wt S} - \error_{\wt S_M}(f) - \error_{\wt S_C}(f) \le \sqrt{\frac{\log(2/\delta)}{2m}} \left(\error_{\wt S_M}(f) + \error_{\wt S_C}(f) \right) \le 2\sqrt{\frac{\log(2/\delta)}{2m}} \,. \label{eq:lemma2_final}
%     % \end{align}
%     \begin{align}
%         2\error_{\wt S} - \error_{\wt S_M}(f) - \error_{\wt S_C}(f) \le \sqrt{\frac{\log(2/\delta)}{2m}} \left(\error_{\wt S_M}(f) + \error_{\wt S_C}(f) \right) \,. \label{eq:lemma2_prefinal}
%     \end{align}
%     With re-arranging $\error_{\wt S_M}(f) + \error_{\wt S_C}(f)$ and using the inequality $ 1- a\le \frac{1}{1+a} $, we have  
%     \begin{align}
%         2\error_{\wt S} - \error_{\wt S_M}(f) - \error_{\wt S_C}(f) \le 2\error_{\wt \calS} \sqrt{\frac{\log(2/\delta)}{2m}}  \,. \label{eq:lemma2_final}
%     \end{align}

%     % We obtain the desired result by using 
% \end{proof}

% \begin{proof}[Proof of \lemref{lem:clear_error}]
% % Recall 0-1 error on each point  $(x,y) \in S \cup \wt S$ is given by $\I{ f(x)\ne y}$.
% In the set of correctly labeled points $S \cup \wt S_C$, we have $S$ as a random subset of $S \cup \wt S_C$. Hence, using Hoeffding's inequality for sampling without replacement (\lemref{lem:hoeffding_sampling}), we have with probability at least $1-\delta$
% \begin{align}
%     \error_{\wt \calS_c} (\wh f)- \error_{\calS \cup \wt \calS_C}( \wh f) \le  \sqrt{\frac{\log(1/\delta)}{2m_2}} \,.
% \end{align}
% Re-writing $\error_{\calS \cup \wt \calS_C}( \wh f)$ as $\frac{m_2}{m_2 + n} \error_{\wt \calS_C }(\wh f) + \frac{n}{m_2 + n} \error_{\calS }(\wh f)$, we have with probability at least $1-\delta$
% \begin{align}
%   \left(\frac{n}{n+m_2}\right) \left(\error_{\wt \calS_c} (\wh f)- \error_{\calS}( \wh f) \right) \le  \sqrt{\frac{\log(1/\delta)}{2m_2}} \,.
% \end{align}
% As before, assuming $2m_2 \approx m$, we have with probability at least $1-\delta$ 
% \begin{align}
%     \error_{\wt \calS_c} (\wh f)- \error_{\calS}( \wh f) \le \left(1+\frac{m_2}{n}\right)  \sqrt{\frac{\log(1/\delta)}{m}} \le 1.5 \sqrt{\frac{\log(1/\delta)}{m}} \,. \label{eq:lemma3_final}
% \end{align} 
% \end{proof}

% \begin{proof}[Proof of \thmref{thm:error_ERM}] 
%     Having established these core intermediate results, we can now combine above three lemmas to prove the main result. 
%     In particular, we bound the population error on clean data ($\error_\calD(\wh f)$) as follows:  
%     \begin{enumerate}[(i)]
%         \item First, use \eqref{eq:lemma1_final}, to obtain an upper bound on the population error on clean data, i.e., with probability at least $1-\delta/4$, we have
%         \begin{align}
%             \error_{ \calD} (\wh f) \le 1 - \error_{ \wt \calS_M}(\wh f) + \sqrt{\frac{\log(4/\delta)}{m}} \,. 
%         \end{align}
%         \item  Second, use \eqref{eq:lemma2_final}, to relate the error on the mislabeled fraction with error on clean portion of randomly labeled data and error on whole randomly labeled dataset, i.e., with probability at least $1-\delta/2$, we have 
%         \begin{align}
%             - \error_{\wt S_M}(f) \le \error_{\wt S_C}(f) - 2\error_{\wt S}  + \sqrt{\frac{\log(4/\delta)}{2m}}  \,. 
%         \end{align} 
%         \item Finally, use \eqref{eq:lemma3_final} to relate the error on the clean portion of randomly labeled data and error on clean training data, i.e., with probability $1-\delta/4$, we have 
%         \begin{align}
%             \error_{\wt \calS_C} (\wh f)\le - \error_{\calS}( \wh f) + \left(1 + \frac{m}{2n} \right) \sqrt{\frac{\log(4/\delta)}{m}} \,. 
%         \end{align} 
%     \end{enumerate}

%     Using union bound on the above three steps, we have with probability at least $1-\delta$: 
%     \begin{align}
%         \error_\calD (\wh f) \le \error_{\calS}(\wh f)   + 1 - 2\error_{\wt \calS}(\wh f)   + (1/\sqrt{2} + 2.5)  \sqrt{\frac{\log(4/\delta)}{m}} \,.
%     \end{align}
%     Note that $(1/\sqrt{2} + 2.5)$ is a loose constant. In experiments, we use the ratio $\frac{m}{n}$
%     %  the exact error $\error_{\wt \calS}(\wh f)$ 
%     to evaluate R.H.S.    
% \end{proof}

% \subsection{Proof of \propref{prop:rademacher}}

% \begin{proof}[Proof of \propref{prop:rademacher}]
%     For a classifier $ f: \calX \to \{-1, 1\}$, we have $1 - 2\,\indict{ f(x) \ne y} = y \cdot f(x)$. Hence, by definition of $\error$, we have 
%     \begin{align}
%         1 -2\error_{\wt \calS}(f) = \frac{1}{m}\sum_{i=1}^m y_i \cdot f(x_i) \le \sup_{f \in \calF} \, \frac{1}{m} \sum_{i=1}^m y_i \cdot f(x_i)  \,. \label{eq:error_rademacher}
%     \end{align}
%     Note that for fixed inputs $(x_1, x_2, \ldots, x_m)$ in $\wt S$, $(y_1, y_2, \ldots y_m)$ are random labels. Define $\phi_1 (y_1, y_2, \ldots, y_m) \defeq \sup_{f \in \calF} \, \frac{1}{m} \sum_{i=1}^m y_i \cdot f(x_i)$. We have the following bounded difference condition on $\phi_1$. For all i, 
%     \begin{align}
%         \sup_{y_1, \ldots y_m, y_i^\prime \in \{-1, 1\}^{m+1} } \abs{ \phi_1 (y_1,\ldots, y_i, \ldots, y_m) - \phi_1 (y_1,\ldots, y_i^\prime, \ldots, y_m)  } \le 1/m \,. \label{cond1_rademacher}
%     \end{align} 
    
%     Similarly define $\phi_2 (x_1, x_2, \ldots, x_m) \defeq \Expt{ y_i \sim_U \{-1, 1\}  }{ \sup_{f \in \calF} \, \frac{1}{m}  \sum_{i=1}^m y_i \cdot f(x_i)}$. We have the following bounded difference condition on $\phi_2$. For all i,
%     \begin{align}
%         \sup_{x_1, \ldots x_m, x_i^\prime \in \calX^{m+1} } \abs{ \phi_2 (x_1,\ldots, x_i, \ldots, x_m) - \phi_1 (x_1,\ldots, x_i^\prime, \ldots, x_m)  } \le 1/m \,. \label{cond2_rademacher}
%     \end{align}
%     Using McDiarmid’s inequality (\lemref{lem:McDiarmid}) twice with Condition \eqref{cond1_rademacher} and \eqref{cond2_rademacher}, with probability at least $1-\delta$, we have
%     \begin{align}
%         \sup_{f \in \calF} \, \frac{1}{m} \sum_{i=1}^m y_i \cdot f(x_i)  - \Expt{x,y}{\sup_{f \in \calF} \, \frac{1}{m} \sum_{i=1}^m y_i \cdot f(x_i) } \le \sqrt{\frac{2\log(2/\delta)}{m}} \label{eq:final_rademacher}
%     \end{align} 
%     Combining \eqref{eq:error_rademacher} and \eqref{eq:final_rademacher}, we obtain the desired result. 
% \end{proof}


% \subsection{Proof of \thmref{thm:error_regularized_ERM}}

% Proof of \thmref{thm:error_regularized_ERM} follows similar to the proof of \thmref{thm:error_ERM}. Note that the same results in \lemref{lem:fit_mislabeled}, \lemref{lem:mislabeled_error}, and \lemref{lem:clear_error} hold in the regularized ERM case. However, the arguments in the proof of \lemref{lem:fit_mislabeled} changes slightly. Hence, we state and prove a lemma parallel to \lemref{lem:fit_mislabeled} for completeness. 

% \begin{lemma} \label{lem:lemma1_reg}
%     Assume the same setup as \thmref{thm:error_regularized_ERM}. 
%     Then for any $\delta >0$, with probability at least  $1-\delta$ 
%     over the random draws of mislabeled data $\wt S_M$, we have 
%     \begin{align}
%         \error_\calD(\widehat f)  \le 1 -\error_{\wt \calS_M}(\widehat f) + \sqrt{\frac{\log(1/\delta)}{m}}\,. 
%     \end{align} 
% \end{lemma}
% \begin{proof}
%     The main idea of the proof remains the same, i.e. regard 
%     the clean portion of the data 
%     ($S \cup \wt S_C$) as fixed.   
%     Then, there exists a classifier $f^*$ 
%     that is optimal over draws 
%     of the mislabeled data $\wt S_M$. 

    
%     Formally, 
%     \begin{align}
%     f^* \defeq \argmin_{f \in \calF} \error_{\widecheck {\calD}} (f)  + \lambda R(f) \,, \label{eq:modified_ERM_reg}
%     \end{align}
%     where $$\widecheck \calD = \frac{n}{m+n} \calS + \frac{m_1}{m+n} \wt \calS_C  + \frac{m_2}{m+n}\calDm \,.$$ That is, $\widecheck \calD$ a combination of 
%     the \emph{empirical distribution} 
%     over correctly labeled data $S \cup \wt S_C$
%     % in $S\cup \wt S$ 
%     and the (population) distribution 
%     over mislabeled data $\calDm$.
%     Recall that 
%     \begin{align}
%     \wh f \defeq \argmin_{f \in \calF} \error_{\calS \cup \wt S} (f) + \lambda R(f) \,. \label{eq:orig_ERM_reg}
%     \end{align}
%     % 
%     % 
%     Since, $\widehat f$ minimizes 0-1 error 
%     on $S \cup \wt S$, using ERM optimality on \eqref{eq:orig_ERM},  
%     we have 
%     \begin{align}
%         \error_{\calS \cup \wt \calS}(\widehat f) + \lambda R(\wh f) \le \error_{
%             \calS \cup \wt \calS}(f^*) + \lambda R(f^*) \,.    \label{eq:step1_reg}
%     \end{align}
%     Moreover, since $f^*$ is independent of $\wt S_M$, using Hoeffding's bound,
%     % \footnote{For a fully rigorous argument,
%     % refer to the complete proof in App.~\ref{app:proof_erm}.} 
%     we have with probability at least $1-\delta$ that
%     \begin{align}
%       \error_{\wt \calS_M}(f^*) \le \error_{ \calDm}(f^*) +  \sqrt{\frac{\log(1/\delta)}{2 m_1}} \,. \label{eq:step2_reg} 
%     \end{align}
%     %$ 
%     %for some constant $c_1\le 1/2$. 
%     Finally, since $f^*$ is the optimal classifier on $\widecheck \calD$, 
%     we have 
%     \begin{align}
%         \error_{\widecheck \calD}(f^*) + \lambda R(f^*) \le \error_{\widecheck \calD}(\widehat f) + \lambda R(\wh f) \label{eq:step3_reg}
%     \end{align}
%      Now to relate \eqref{eq:step1_reg} and \eqref{eq:step3_reg}, we can re-write the \eqref{eq:step2_reg} as follows: 
%     \begin{align}
%         \error_{\calS \cup \wt\calS}(f^*) \le \error_{ \widecheck \calD}(f^*) +  \frac{m_1}{m+n}\sqrt{\frac{\log(1/\delta)}{2 m_1}} \,. \label{eq:step4_reg} 
%     \end{align}
%     After adding $\lambda R(f^*)$ on both sides in \eqref{eq:step4_reg}, we combine equations \eqref{eq:step1_reg}, \eqref{eq:step4_reg}, and \eqref{eq:step3_reg}, to get 
%     \begin{align}
%         \error_{\calS \cup \wt \calS}(\wh f) \le \error_{\widecheck \calD}(\wh f) +  \frac{m_1}{m+n}\sqrt{\frac{\log(1/\delta)}{2 m_1}} \,, 
%     \end{align}
%     which implies 
%     \begin{align}
%         \error_{ \wt \calS_M}(\wh f) \le \error_{\calDm}(\wh f) + \sqrt{\frac{\log(1/\delta)}{2 m_1}} \,. \label{eq:lemma_reg_final}
%     \end{align}
%     Similar as before, since $\wt S$ is obtained by randomly labeling an unlabeled dataset, we assume 
%     $2m_1 \approx m$. Moreover, using $\error_{\calDm} = 1 - \error_{\calD}$ we obtain the desired result. 
% \end{proof}
% % \begin{proof}[Proof of ]
    
% % \end{proof}

% \subsection{Proof of \thmref{thm:multiclass_ERM}}

% We first state and prove lemmas parallel to three lemmas used in the proof of balanced binary case. Then we combine the results in the three lemmas to obtain the result in \thmref{thm:multiclass_ERM}. 

% Before stating the result, we define mislabeled distribution $\calDm$ for any $\calD$. While $\calDm$ and $\calD$ share 
% the same marginal distribution over $\calX$, 
% the distribution over labels $y$ 
% given an input $x\sim \calD_\calX$ is changed.
% In particular, for any $x$, the pdf over $y$ is changed to:  
% $p_{\calDm} (\cdot \vert x) \defeq \frac{1 - p_{\calD}(\cdot \vert x)}{k - 1}$.

% \begin{lemma} \label{lem:fit_mislabeled_multi}
%     Assume the same setup as \thmref{thm:multiclass_ERM}. 
%     Then for any $\delta >0$, with probability at least  $1-\delta$ 
%     over the random draws of mislabeled data $\wt S_M$, we have 
%     \begin{align}
%         \error_\calD(\widehat f)  \le (k-1)\left(1 -\error_{\wt \calS_M}(\widehat f)\right) + (k-1)\sqrt{\frac{\log(1/\delta)}{m}}\,. \label{eq:lemma1_multi}
%     \end{align}   
% \end{lemma} 

% \begin{proof}
%     The main idea of the proof remains the same, i.e. regard 
%     the clean portion of the data 
%     ($S \cup \wt S_C$) as fixed. 
%     Then, there exists a classifier $f^*$ 
%     that is optimal over draws 
%     of the mislabeled data $\wt S_M$. 
    
%     However, we need to be careful while relating population error on mislabeled data with population accuracy on clean data.   
%     While for binary classification,  we could upper bound $\error_{\wt \calS_M}$ 
%     with $1-\error_\calD$  (in the proof of \lemref{lem:fit_mislabeled}), 
%     for multiclass classification, 
%     error on the mislabeled data 
%     and accuracy on the clean data 
%     in the population 
%     are not so directly related.  
%     To establish \eqref{eq:lemma1_multi},
%     we break the error on the 
%     (unknown) mislabeled data 
%     into two parts: one term corresponds 
%     to predicting the true label on mislabeled data, 
%     and the other corresponds to predicting 
%     neither the true label 
%     nor the assigned (mis-)label.  
%     Finally, we relate these errors to their
%     population counterparts to establish \eqref{eq:lemma1_multi}. 
    
%     Formally, 
%     \begin{align}
%     f^* \defeq \argmin_{f \in \calF} \error_{\widecheck {\calD}} (f)  + \lambda R(f) \,, \label{eq:modified_ERM_reg2}
%     \end{align}
%     where $$\widecheck \calD = \frac{n}{m+n} \calS + \frac{m_1}{m+n} \wt \calS_C  + \frac{m_2}{m+n}\calDm \,.$$ That is, $\widecheck \calD$ a combination of 
%     the \emph{empirical distribution} 
%     over correctly labeled data $S \cup \wt S_C$
%     % in $S\cup \wt S$ 
%     and the (population) distribution 
%     over mislabeled data $\calDm$.
%     Recall that 
%     \begin{align}
%     \wh f \defeq \argmin_{f \in \calF} \error_{\calS \cup \wt S} (f) + \lambda R(f) \,. \label{eq:orig_ERM_reg2}
%     \end{align}
%     % 
%     % 
%     Following the exact steps from the proof of \lemref{lem:lemma1_reg}, with probability at least $1-\delta$, we have  
%     \begin{align}
%         \error_{ \wt \calS_M}(\wh f) \le \error_{\calDm}(\wh f) + \sqrt{\frac{\log(1/\delta)}{2 m_1}} \,. \label{eq:lemma1_final_multi_prev}
%     \end{align}
%     Similar to before, since $\wt S$ is obtained by randomly labeling an unlabeled dataset, we assume 
%     $\frac{k}{k-1} m_1 \approx m$. 
    
%     Now we will relate $\error_\calDm (\wh f)$ with $\error_{\calD}(\wh f)$. Let $y^T$ denote the (unknown) true label for a mislabeled point $(x, y)$ (i.e., label before replacing it with a mislabel). 
%     \begin{align}    
%          \Expt{(x, y) \in \sim \calDm}{\indict{ \wh f(x) \ne y }}  &= \underbrace{\Expt{(x, y) \in \sim \calDm}{\indict{ \wh f(x) \ne y \land \wh f(x) \ne y^T}}}_{\RN{1}} + \underbrace{\Expt{(x, y) \in \sim \calDm}{\indict{ \wh f(x) \ne y \land \wh f(x) = y^T}}}_{\RN{2}} \,. \label{eq:excess_term}
%     \end{align}
%     Clearly, term 2 is one minus the accuracy on the clean unseen data, i.e. 
%     \begin{align}
%         \RN{2} = 1 - \Expt{{x,y} \sim \calD}{ \indict{ \wh f(x) \ne y}} = 1- \error_{\calD}(\wh f) \,. \label{eq:term1}    
%     \end{align}
%     Next, we  relate term 1 with the error on the unseen clean data. We show that term 1 is equal to the error on the unseen clean data scaled by $\frac{k-2}{k-1}$ where $k$ is the number of labels. Using the definition of mislabeled distribution $\calDm$,  we have 
%     \begin{align}
%         \RN{1} = \frac{1}{k-1} \left( \Expt{(x, y) \in \sim \calD}{ \sum_{i \in \calY \land i\ne y}  \indict{ \wh f(x) \ne i \land \wh f(x) \ne y}} \right) = \frac{k-2}{k-1} \error_{\calD}(\wh f) \,.\label{eq:term2}
%     \end{align}    

%     Combining the result in \eqref{eq:term1}, \eqref{eq:term2} and \eqref{eq:excess_term}, we have 
%     \begin{align}
%         \error_{\calDm}(\wh f) = 1- \frac{1}{k-1} \error_{\calD}(\wh f) \,.\label{eq:combine_terms}
%     \end{align}
%     Finally, combining the result in \eqref{eq:combine_terms} with equation \eqref{eq:lemma1_final_multi_prev}, we have with probability $1-\delta$, 
%     \begin{align}
%       \error_{\calD}(\wh f) \le  (k-1) \left( 1- \error_{ \wt \calS_M}(\wh f) \right)  + (k-1) \sqrt{\frac{k \log(1/\delta)}{ 2(k-1)m}} \,. \label{eq:lemma1_final_multi}
%     \end{align}
% \end{proof}

% \begin{lemma} \label{lem:mislabeled_error_multi}
%     Assume the same setup as \thmref{thm:multiclass_ERM}.  Then for any $\delta >0$, with probability at least $1-\delta$ over the random draws of $\wt S$, we have  
%     % \begin{align}
%         $$\abs{k\error_{\wt \calS}(\widehat f) - \error_{\wt \calS_C}(\widehat f) -  (k-1)\error_{\wt \calS_M}(\widehat f) } \le  2k\sqrt{\frac{\log(4/\delta)}{2m}}\,. $$ % \label{eq:lemma2}
%     % \end{align}   
%     %  for some constant $c_3 \le 1.0\,$.
% \end{lemma} 


% \begin{proof}
%     Recall $\error_{\wt S} (f) = \frac{m_1}{m} \error_{\wt S_M}(f) + \frac{m_2}{m} \error_{\wt S_C}(f)$. Hence, we have 
%     \begin{align}
%         k\error_{\wt S}(f) - (k-1)\error_{\wt S_M}(f) - \error_{\wt S_C}(f) &= (k-1)\left(\frac{k m_1}{(k-1) m} \error_{\wt S_M}(f) - \error_{\wt S_M}(f)\right) + \left(\frac{km_2}{m} \error_{\wt S_C}(f) - \error_{\wt S_C}(f)\right) \\ &= k \left[ \left(\frac{m_1}{m} - \frac{k-1}{k}\right) \error_{\wt S_M}(f) + \left(\frac{m_2}{m} - \frac{1}{k} \right) \error_{\wt S_C} (f) \right] \,.
%     \end{align} 
%     Since the dataset is randomly labeled, we have with probability at least $1-\delta$, $\left(\frac{m_1}{m} - \frac{k-1}{k}\right) \le \sqrt{\frac{\log(1/\delta)}{2m}}$. Similarly, we have with probability at least $1-\delta$, $\left(\frac{m_2}{m} - \frac{1}{k}\right) \le \sqrt{\frac{\log(1/\delta)}{2m}}$. Using union bound, we have with probability at least $1-\delta$
%     % \begin{align}
%     %     2\error_{\wt S} - \error_{\wt S_M}(f) - \error_{\wt S_C}(f) \le \sqrt{\frac{\log(2/\delta)}{2m}} \left(\error_{\wt S_M}(f) + \error_{\wt S_C}(f) \right) \le 2\sqrt{\frac{\log(2/\delta)}{2m}} \,. \label{eq:lemma2_final}
%     % \end{align}
%     \begin{align}
%         k\error_{\wt S}(f) - (k-1)\error_{\wt S_M}(f) - \error_{\wt S_C}(f)  \le k \sqrt{\frac{\log(2/\delta)}{2m}} \left(\error_{\wt S_M}(f) + \error_{\wt S_C}(f) \right) \,. \label{eq:lemma2_final_multi}
%     \end{align}

%     % We obtain the desired result by using 
% \end{proof}

% \begin{lemma} \label{lem:clear_error_multi}
%     Assume the same setup as \thmref{thm:multiclass_ERM}. 
%     Then for any $\delta >0$, with probability at least $1-\delta$ 
%     over the random draws of $\wt S_C$ and $S$, we have 
%     % \begin{align}
%         $$\abs{\error_{\wt \calS_C}(\widehat f) - \error_{\calS}(\widehat f) } \le 1.5 \sqrt{\frac{k\log(2/\delta)}{2m}}\,.$$ %\label{eq:lemma3}
%     % \end{align}   
%     % for some constant $c_2 \le 1.2\,$.
% \end{lemma} 
% \begin{proof}
%     % Recall 0-1 error on each point  $(x,y) \in S \cup \wt S$ is given by $\I{ f(x)\ne y}$.
%     In the set of correctly labeled points $S \cup \wt S_C$, we have $S$ as a random subset of $S \cup \wt S_C$. Hence, using Hoeffding's inequality for sampling without replacement (\lemref{lem:hoeffding_sampling}), we have with probability at least $1-\delta$
%     \begin{align}
%         \error_{\wt \calS_c} (\wh f)- \error_{\calS \cup \wt \calS_C}( \wh f) \le  \sqrt{\frac{\log(1/\delta)}{2m_2}} \,.
%     \end{align}
%     Re-writing $\error_{\calS \cup \wt \calS_C}( \wh f)$ as $\frac{m_2}{m_2 + n} \error_{\wt \calS_C }(\wh f) + \frac{n}{m_2 + n} \error_{\calS }(\wh f)$, we have with probability at least $1-\delta$
%     \begin{align}
%       \left(\frac{n}{n+m_2}\right) \left(\error_{\wt \calS_c} (\wh f)- \error_{\calS}( \wh f) \right) \le  \sqrt{\frac{\log(1/\delta)}{2m_2}} \,.
%     \end{align}
%     As before, assuming $km_2 \approx m$, we have with probability at least $1-\delta$ 
%     \begin{align}
%         \error_{\wt \calS_c} (\wh f)- \error_{\calS}( \wh f) \le \left(1+\frac{m_2}{n}\right)  \sqrt{\frac{k\log(1/\delta)}{2m}} \le \left( 1 + \frac{1}{k}\right) \sqrt{\frac{k\log(1/\delta)}{2m}} \,. \label{eq:lemma3_final_multi}
%     \end{align} 
% \end{proof}

% \begin{proof}[Proof of \thmref{thm:multiclass_ERM}] 
%     Having established these core intermediate results, we can now combine above three lemmas. 
%     In particular, we bound the population error on clean data ($\error_\calD(\wh f)$) as follows:  
%     \begin{enumerate}[(i)]
%         \item First, use \eqref{eq:lemma1_final_multi}, to obtain an upper bound on the population error on clean data, i.e., with probability at least $1-\delta/4$, we have
%         \begin{align}
%             \error_{ \calD} (\wh f) \le (k-1)\left(1 - \error_{ \wt \calS_M}(\wh f) \right) + (k-1) \sqrt{\frac{k\log(4/\delta)}{2(k-1)m}} \,. 
%         \end{align}
%         \item  Second, use \eqref{eq:lemma2_final_multi}, to relate the error on the mislabeled fraction with error on clean portion of randomly labeled data and error on whole randomly labeled dataset, i.e., with probability at least $1-\delta/2$, we have 
%         \begin{align}
%             - (k-1)\error_{\wt S_M}(f) \le \error_{\wt S_C}(f) - k\error_{\wt S}  + k\sqrt{\frac{\log(4/\delta)}{2m}}  \,. 
%         \end{align} 
%         \item Finally, use \eqref{eq:lemma3_final_multi} to relate the error on the clean portion of randomly labeled data and error on clean training data, i.e., with probability $1-\delta/4$, we have 
%         \begin{align}
%             \error_{\wt \calS_C} (\wh f)\le - \error_{\calS}( \wh f) + \left(1 + \frac{m}{kn} \right) \sqrt{\frac{k\log(4/\delta)}{2m}} \,. 
%         \end{align} 
%     \end{enumerate}

%     Using union bound on the above three steps, we have with probability at least $1-\delta$: 
%     \begin{align}
%         \error_\calD (\wh f) \le \error_{\calS}(\wh f) + (k-1) - k\error_{\wt \calS}(\wh f)   + (\sqrt{k(k-1)} + k + \sqrt{k} + \frac{m}{n\sqrt{k}})  \sqrt{\frac{\log(4/\delta)}{2m}} \,.
%     \end{align}
%     % Note that $\frac{m}{n\sqrt{k}}$ is much smaller than the other terms in addition. Hence, we ignore this in the final bound. 
%     % Note that $(1/\sqrt{2} + 2.5)$ is a loose constant. In experiments, we use the ratio $\frac{m}{n}$
%     %  the exact error $\error_{\wt \calS}(\wh f)$ 
%     % to evaluate R.H.S.    
% \end{proof}

% \newpage
% \section{Proofs from \secref{sec:linear_models}}\label{app:proof_gd}

% We suppose that the parameters of the linear function 
% are obtained via gradient descent on 
% the following $L_2$ regularized problem: 
% \begin{align}
%     % n in denominator is avoided deliberately
%     \calL_S(w; \lambda) \defeq \sum_{i=1}^n{(w^Tx_i - y_i)^2} + \lambda \norm{w}{2}^2 \,, \label{eq:l2_MSE_app}   
% \end{align}
% where $\lambda\ge0$ is a regularization parameter. 
% We assume access to a clean dataset 
% $S = \{(x_i, y_i)\}_{i=1}^n \sim \calD^n$ 
% and randomly labeled dataset 
% $\wt S = \{(x_i, y_i)\}_{i=n+1}^{n+m} \sim \wt \calD^m$. 
% Let $\bX = [x_1, x_2, \cdots, x_{m+n}]$ 
% and $\by = [y_1, y_2, \cdots, y_{m+n}]$. 
% Fix a positive learning rate $\eta$ such that 
% $\eta \le 1/\left(\norm{\bX^T\bX}{\text{op}} + \lambda^2\right)$ 
% and an initialization $w_0 = 0$. 
% % \todos{Assumption made for simplicty}. 
% Consider the following gradient descent iterates 
% to minimize objective \eqref{eq:l2_MSE_app} on $S \cup \wt S$:
% \begin{align}
% w_t = w_{t-1} - \eta \grad_w \calL_{S \cup \wt S} (w_{t-1}; \lambda) \quad \forall t=1,2,\ldots \label{eq:GD_iterates_app}
% \end{align} 
% Then we have $\{ w_t\}$ converge to the limiting solution 
% $\wh w = \left( \bX^T\bX+\lambda \boldsymbol{I}\right)^{-1}\bX^T\by$. Define $\widehat f (x) \defeq f(x ; \wh w) $.  

% \subsection{\textcolor{red}{Errata}}

% We wish to correct the following error in the body: \codref{cond:error_stability} is not enough to guarantee the result in \thmref{thm:linear}. We now present a slightly stronger condition called \emph{hypothesis stability} under which we obtain a result similar to \thmref{thm:linear}. 

% This error doesn't change the main arguments of the proof where we show that the empirical train error is less than or equal to the leave-one-out error. We need a stronger condition to relate leave-one-out error with the population error of the original classifier. Specifically, while \codref{cond:error_stability} relates the average population error of leave-one-out classifiers with the population error of the original classifier, we need the new condition to show the concentration of the empirical leave-one-out error and  average population error of leave-one-out classifiers. 
% % main takeaway 

% Note that the new condition, while being stronger than the previous one, still doesn't imply generalization~\cite{bousquet2002stability,elisseeff2003leave,abou2019exponential}. Overall, the main results in \secref{sec:ERM_training} and takeaways of the paper remain unaffected by the error.  

% We now present the new condition and a corrected statement of \thmref{thm:linear}. Recall, for a given training set $S \sim \calD^n $, 
% we use $S_{(i)}$ to denote the training set $S$ 
% with the $i^{\text{th}}$ point removed.

% \begin{condition}[Hypothesis Stability] 
%     \label{cond:hypothesis_stability}
%     We have $\beta$ hypothesis stability 
%     if our training algorithm $\calA$ satisfies the following: 
%     \begin{align*}
%     % ${\sum_{i=1}^n \frac{\error_{\calD}( f(\calA, S_{(i)}))}{n} - \error_\calD(f(\calA, S))} \le \beta\,$.
%     \forall i \in \{1,2,\ldots, n\}, \quad  \Expt{\calS, (x,y) \in \calD}{ \abs{\error\left( f(x) ,y  \right) - \error\left( f_{(i)}(x), y \right) }} \le \frac{\beta}{n} \,,
%     \end{align*}
%     where $f_{(i)} \defeq f(\calA, S_{(i)})$ and $ f \defeq f(\calA, S)$.
% \end{condition}

% \begin{theorem}[Correct statement of \thmref{thm:linear}] \label{thm:new_linear}
%     Assume that this gradient descent algorithm satisfies \codref{cond:hypothesis_stability}
%     with $\beta=\calO(1)$.  
%     Then for any $\delta >0$, with probability at least $1-\delta$ 
%     over the random draws of datasets $\wt S$ and $S$, we have:
%     \begin{align}
%         \error_\calD(\widehat f) \le \error_\calS(\widehat f) + 1 - 2 \error_{\wt\calS}(\widehat f) + \left(\frac{1}{\sqrt{2}} + 1.5 \right) \sqrt{\frac{\log(4/\delta)}{m}} + \sqrt{\frac{4}{\delta}\left(\frac{1}{m} +\frac{3\beta}{m+n} \right)}  \,. \label{eq:gd_error}
%     \end{align} 
%     % for some constant $c\le 3.2$.
% \end{theorem}

% \subsection{Proof of \thmref{thm:new_linear}}
% We use a standard result from linear algebra, namely Shermann-Morrison formula~\citep{sherman1950adjustment} for matrix inversion:  

% \begin{lemma}[\citet{sherman1950adjustment}] \label{lem:sherman}
%     Suppose $\bA \in \Real^{n \times n}$ is an invertible square matrix and $u,v \in \Real^n$ are column vectors. Then $\bA + uv^T$ is invertible iff $1 + v^T \bA u \ne 0$ and in particular
%     \begin{align}
%         (\bA + u v^T)^{-1} = \bA^{-1}  - \frac{\bA^{-1} uv^T \bA^{-1} }{ 1 + v^T \bA^{-1} u} \,.
%     \end{align}   
% \end{lemma}
% \newcommand\byy[1]{\by_{\left(#1\right)}}
% \newcommand\bXX[1]{\bX_{\left(#1\right)}}
% \newcommand\ff[1]{\wh f_{\left(#1\right)}}

% For a given training set $S \cup \wt S_C$, define leave-one-out error on mislabeled points in the training data as $$\error_{\text{LOO}(\wt S_M) } = \frac{\sum_{(x_i, y_i) \in \wt S_M} \error( f_{(i)}( x_i), y_i)}{ \abs{\wt S_M }} \,, $$
% where $f_{(i)} \defeq f(\calA, (S \cup \wt S)_{(i)})$. To relate empirical leave-one-out error and population error with hypothesis stability condition, we use the following lemma:   

% \begin{lemma}[\citet{bousquet2002stability}] \label{lem:stability_error}
%     For the leave-one-out error, we have
%     \begin{align}
%         \Expo{ \left( \error_{\calDm}(\wh f) -\error_{\text{LOO}(\wt S_M) } \right)^2 } \le \frac{1}{2m_1}+  \frac{3\beta}{n + m}\,.
%     \end{align}   
%     % where $ f \defeq f(\calA, S \cup \wt S) $.
% \end{lemma}

% Proof of the above lemma is similar to the proof of  Lemma 9 in \citet{bousquet2002stability} and can be found in \appref{app:proof_lem_error}. 
% % 
% % Before presenting the result, we introduce some notation. 
% Before presenting the proof of \thmref{thm:new_linear}, we introduce some more notation. Let $\bX_{(i)}$ denote the matrix of covariates with $i^{\text{th}}$ point removed. Similarly let $\by_{(i)}$ be the array of responses with $i^{\text{th}}$ point removed. Define the corresponding regularized GD solution as $\wh w_{(i)} = \left( \bXX{i}^T\bXX{i}+\lambda \boldsymbol{I}\right)^{-1}\bXX{i}^T\byy{i}$. Define $\ff{i}(x) \defeq f(x ; \wh w_{(i)}) $.

% \begin{proof}[Proof of \thmref{thm:new_linear}]
%     Because squared loss minimization does not imply 0-1 error minimization, we cannot use arguments from \lemref{lem:fit_mislabeled}. This is the main technical difficulty. To compare the 0-1 error at a train point with an unseen point, 
%     we use the closed-form expression for $\widehat{w}$ and Shermann-Morrison formula to upper bound training error with leave-one-out cross validation error. 
    
%     The proof is divided into three parts: In part one, we show that 0-1 error on mislabeled points in the training set is lower than the error obtained by leave-one-out error at those points. In part two, we relate this leave-one-out error with the population error on mislabeled distribution using \codref{cond:hypothesis_stability}. While the empirical leave-one-out error is unbiased estimator of the average population error of leave-one-out classifiers, we need hypothesis stability to control the variance of empirical leave-one-out error. Finally in part three, we show that the error on the mislabeled training points can be estimated with just the randomly labeled and  clean training data (as in proof of \thmref{thm:error_ERM}).  

%     \textbf{Part 1 {} {}} First we relate training error with leave-one-out error.        
%     For any 
%     training point $(x_i, y_i)$ in $\wt S \cup S$, we have 
%     \begin{align}
%         \error(\wh f(x_i), y_i ) &= \indict{ y_i \cdot x_i^T \wh w < 0 } = \indict{ y_i \cdot x_i^T \left( \bX^T\bX+\lambda \boldsymbol{I}\right)^{-1}\bX^T\by < 0 } \\
%         &= \indict{ y_i \cdot x_i^T \underbrace{\left( \bXX{i}^T\bXX{i} + x_i ^T x_i +\lambda \boldsymbol{I}\right)^{-1}}_{\RN{1}} (\bXX{i}^T\byy{i} + y \cdot x_i) < 0 }
%     \end{align}
%     Letting $\bA = \left(\bXX{i}^T\bXX{i} +\lambda \boldsymbol{I}\right)$ and using \lemref{lem:sherman} on term 1, we have 
%     \begin{align}
%         \error(\wh f(x_i), y_i ) &= \indict{ y_i \cdot x_i^T \left[\bA^{-1} -  \frac{\bA^{-1} x_i x_i^T \bA^{-1}}{ 1 + x_i ^T \bA^{-1} x_i } \right] (\bXX{i}^T\byy{i} + y \cdot x_i) < 0 } \\
%         &= \indict{ y_i \cdot\left[ \frac{ x_i^T \bA^{-1} ( 1 + x_i ^T \bA^{-1} x_i ) -  x_i^T \bA^{-1} x_i x_i^T \bA^{-1}}{ 1 + x_i ^T \bA ^{-1}x_i } \right] (\bXX{i}^T\byy{i} + y \cdot x_i) < 0 } \\
%         &= \indict{ y_i \cdot\left[ \frac{ x_i^T \bA^{-1}}{ 1 + x_i ^T \bA ^{-1}x_i } \right] (\bXX{i}^T\byy{i} + y \cdot x_i) < 0 } \,.
%     \end{align}

%     Since $1 + x_i^T \bA^{-1} x_i > 0$, we have 
%     \begin{align}
%         \error(\wh f(x_i), y_i ) &= \indict{ y_i \cdot x_i^T \bA^{-1} (\bXX{i}^T\byy{i} + y \cdot x_i) < 0 } \\
%         &= \indict{ x_i^T \bA^{-1} x_i +  y_i \cdot x_i^T \bA^{-1} (\bXX{i}^T\byy{i}) < 0 } \\
%         &\le \indict{ y_i \cdot x_i^T \bA^{-1} (\bXX{i}^T\byy{i}) < 0 } = \error(\ff{i}(x_i), y_i ) \,.\label{eq:LOO_error}
%     \end{align}

%     Using \eqref{eq:LOO_error}, we have 
%     \begin{align}
%         \error_{\wt \calS_M } (\wh f) \le \error_{\text{LOO} (S_M)} \defeq \frac{\sum_{(x_i, y_i) \in \wt S_M} \error(\ff{i}(x_i), y_i ) }{\abs{\wt \calS_M}}\label{eq:LOO_error_final}
%     \end{align}
%     \textbf{Part 2 {}{}} We now relate RHS in \eqref{eq:LOO_error_final} with the population error on mislabeled distribution. To do this, we leverage \codref{cond:hypothesis_stability} and \lemref{lem:stability_error}. In particular, we have 

%     \begin{align}
%         \Expt{\calS \cup \wt \calS_M }{ \left(\error_{\calDm}(\wh f) - \error_{\text{LOO} (S_M)}\right)^2 } \le \frac{1}{2m_1} + \frac{3\beta}{m+n} \,.
%     \end{align}

%     Using Chebyshev's inequality, with probability at least $1-\delta$, we have 
%     \begin{align}
%         \error_{\text{LOO} (S_M)} \le  \error_{\calDm}(\wh f)   + \sqrt{\frac{1}{\delta}\left(\frac{1}{2m_1} +\frac{3\beta}{m+n} \right)} \,. \label{eq:final_mislabeled_linear}
%     \end{align}
    

%     \textbf{Part 3 {}{}} Combining \eqref{eq:final_mislabeled_linear} and \eqref{eq:LOO_error_final}, we have 

%     \begin{align}
%         \error_{\wt \calS_M } (\wh f) \le \error_{\calDm}(\wh f)   + \sqrt{\frac{1}{\delta}\left(\frac{1}{2m_1} +\frac{3\beta}{m+n} \right)} \,. \label{eq:linear_parallel_lem1}
%     \end{align}

%     Compare \eqref{eq:linear_parallel_lem1}, with \eqref{eq:lemma1_final} in the proof of \lemref{lem:fit_mislabeled}. We obtain a similar relationship between $\error_{\wt \calS_M }$ and $\error_{\calDm}$ but with a polynomial concentration instead of exponential concentration. 
%     In addition, since we just use concentration arguments to relate mislabeled error with the error on clean portion and unlabeled portion, we can directly use the results in \lemref{lem:mislabeled_error} and \lemref{lem:clear_error}. Therefore, combining results in \lemref{lem:mislabeled_error}, \lemref{lem:clear_error}, and \eqref{eq:linear_parallel_lem1} with union bound, we have with probability at least $1-\delta$

%     \begin{align}
%         \error_\calD(\widehat f) \le \error_\calS(\widehat f) + 1 - 2 \error_{\wt\calS}(\widehat f) + \left(\frac{1}{\sqrt{2}} + 1.5 \right) \sqrt{\frac{\log(4/\delta)}{m}} + \sqrt{\frac{4}{\delta}\left(\frac{1}{m} +\frac{3\beta}{m+n} \right)}  \,.
%     \end{align}
    

       
% \end{proof}

% \subsection{Discussion on \codref{cond:hypothesis_stability}}

% The quantity in LHS of \codref{cond:hypothesis_stability} measures how much the function learned by the algorithm (in terms of error on unseen point) will change when one point in the training set is removed. 
% % Discussion on exponential concentration and stronger condition. 
% Notice that hypothesis stability implies error stability, i.e., \codref{cond:error_stability} ~\cite{bousquet2002stability}.  In summary, while error stability allowed us to relate the average population error of the leave-one-out classifiers with the population error of the original classifier, we need hypothesis stability condition to control the variance of the empirical leave-one-out error. 

% Additionally, we note that while the dominating term in the RHS of \thmref{thm:new_linear} matches with the dominating term in ERM bound in \thmref{thm:error_ERM}, there is a polynomial concentration term (dependence on $1/\delta$ instead of $\log(\sqrt{1/\delta})$) in  \thmref{thm:new_linear}. 
% Since with hypothesis stability, we just bound the variance,  the polynomial concentration is due to the use of Chebyshev's inequality instead of an exponential tail inequality (as in \lemref{lem:fit_mislabeled}).
% Recent works have highlighted that slightly stronger condition than hypothesis stability can be used to obtained an exponential concentration for leave-one-out error~\citep{abou2019exponential}, but we leave this for future work for now. 
% % We leave 
% % However, the constants 

% % we also want to highlight  

% \subsection{Formal statement and proof of  of \propref{prop:early_stop}}

% Before formally presenting the result, we will introduce some notation.  By $\calL_{S}(w)$, we denote 
% the objective in \eqref{eq:l2_MSE_app} with $\lambda=0$. 
% Assume Singular Value Decomposition (SVD) of $\bX$  as $\sqrt{n} \bU \bS^{1/2} \bV^T$. Hence $\bX^T \bX = \bV \bS \bV^T$.
% Consider the GD iterates defined in \eqref{eq:GD_iterates_app}. 
% % 
% We now derive closed form expression for the $t^\text{th}$ iterate of gradient descent:  
% % 
% \begin{align}
%     w_t = w_{t-1} + \eta \cdot \bX^T (\by - \bX w_{t-1}) = (\bI - \eta \bV \bS \bV^T )w_{k-1} + \eta \bX^T \by \,.
% \end{align}
% Rotating by $\bV^T$, we get 
% \begin{align}
%     \wt w_t = (\bI - \eta\bS )\wt w_{k-1} + \eta \wt \by \,, \label{eq:GD_recur}
% \end{align}
% where $\wt w_t = \bV^T w_t $ and $\wt \by = \bV^T \bX^T \by$. Assuming the initial point $w_0 = 0$ and applying the recursion in \eqref{eq:GD_recur}, we get
% \begin{align}
%     \wt w_t = \bS ^{-1} ( \bI - (\bI - \eta \bS)^k ) \wt \by \,, 
% \end{align} 
% Projecting solution back to the original space, we have 
% \begin{align}
%      w_t = \bV \bS ^{-1} ( \bI - (\bI - \eta \bS)^k ) \bV^T \bX^T \by \,, 
% \end{align} 
% % We will work with this GD solution at any iterate $t$ in the next proposition. 
% Define $f_t(x) \defeq f(x;w_t)$ as the solution at the $t^{\text{th}}$ iterate. 
% Let $\wt w_{\lambda} = \argmin_{w} \calL_\calS (w;\lambda) = (\bX^T \bX + \lambda \bI)^{-1} \bX^T \by = \bV (\bS + \lambda \bI )^{-1} \bV^T \bX^T \by $. 
% % ) \,,$ for all $t=1,2,\ldots\,.$ 
% and define $\wt f_\lambda(x) \defeq f(x;\wt w_\lambda)$ as the regularized solution. 
% Assume $\kappa$ be the condition number of the population covariance matrix 
% and 
% let $s_\text{min}$ be the minimum positive singular value of the empirical covariance matrix. Our proof idea is inspired from recent work on relating gradient flow solution and regularized solution for regression problems \citep{ali2018continuous}. We will use the following lemma in the proof: 
% \begin{lemma} \label{lem:ineq_soln}
%     For all $x \in [0,1]$ and for all $ k \in \mathbb{N}$, we have (a) $ \frac{kx}{1+kx} \le 1- (1-x)^k$ and (b) $ 1- (1-x)^k \le 2 \cdot \frac{kx}{kx+1} $.
%     %  where $g(c)$ is a constant dependent on $c$. For $c = 1$, $g(c) = 2.0$.   
% \end{lemma}
% \begin{proof}
%     % [Proof of \lemref{lem:ineq_soln}]
%     % Part (a) is easy. 
%     Using $ (1-x)^k \le \frac{1}{1+kx}$, we have part (a). For part (b), we numerically maximize $\frac{ (1+kx ) (1 - (1-x)^k) }{kx}$ for all $k\ge 1$ and for all $x \in [0, 1]$.  
% \end{proof}

% % 
% % Next, 

% \begin{prop}[Formal statement of \propref{prop:early_stop}] \label{prop:formal_early_stop}
% Let $\lambda = \frac{1}{t\eta}$. For a training point $x$, we have 
% \begin{align*}
%     \Expt{x \sim \calS}{(f_t(x) - \wt f_\lambda(x))^2} &\le c(t,\eta) \cdot \Expt{x \sim \calS}{f_t(x)^2} \,, %\label{eq:early_stop}
% \end{align*}
% where $c(t, \eta) \defeq \min( 0.25, \frac{1}{s_\text{min}^2 t^2 \eta^2})$. Similarly for a test point, we have 
% \begin{align*}
%     \Expt{x \sim \calD_\calX}{(f_t(x) - \wt f_\lambda(x))^2} &\le \kappa \cdot c(t,\eta) \cdot \Expt{x \sim \calD_\calX}{f_t(x)^2} \,. %\label{eq:early_stop}
% \end{align*}
% \end{prop} 

% \begin{proof}
%     %%%%%%%%%%%%% 
%     We want to analyze the expected squared difference output of regularized linear regression with regularization constant $\lambda = \frac{1}{\eta t}$ and gradient descent solution at $t^\text{th}$ iterate. We separately expand the algebraic expression for squared difference at a training point and a test point. 
%     % We start by considering the difference  
%     Then the main step is to show that  $\left[ \bS ^{-1} ( \bI - (\bI - \eta \bS)^k )  - (\bS + \lambda \bI )^{-1}\right] \preceq c(\eta, t) \cdot \bS ^{-1} ( \bI - (\bI - \eta \bS)^k ) $.

%     %%%%%%%%%%%%%
    
%   \textbf{Part 1 {} {}} 
%     First, we will analyze the squared difference of output at a training point (for simplicity, we refer to $S \cup \wt S$ as $S$), i.e. 
%     \begin{align}
%         \Expt{ x \sim \calS }{\left(f_t(x) - \wt f_\lambda (x)\right)^2} &= \norm{\bX w_t - \bX \wt w_\lambda}{2}^2 =   \norm{\bX \bV \bS ^{-1} ( \bI - (\bI - \eta \bS)^t ) \bV^T \bX^T \by - \bX \bV (\bS + \lambda \bI )^{-1} \bV^T \bX^T \by }{2}^2 \\
%         &= \norm{\bX \bV \left(\bS ^{-1} ( \bI - (\bI - \eta \bS)^t ) - (\bS + \lambda \bI )^{-1} \right) \bV^T \bX^T \by  }{2} \\
%         &=  \by^T \bV \bX \left( \underbrace{\bS ^{-1} ( \bI - (\bI - \eta \bS)^t ) - (\bS + \lambda \bI )^{-1}}_{\RN{1}} \right)^2 \bS \bV^T \bX^T \by \label{eq:train_GD_rel}
%         %  (\bX \bV \bS ^{-1} ( \bI - (\bI - \eta \bS)^k ) \bV^T \bX^T \by)^T \bX \bV \bS ^{-1} ( \bI - (\bI - \eta \bS)^k ) \bV^T \bX^T \by
%     \end{align}
%     We now separately consider term 1. Substituting $\lambda = \frac{1}{t \eta}$, we get
%     \begin{align}
%         \bS ^{-1} ( \bI - (\bI - \eta \bS)^t ) - (\bS + \lambda \bI )^{-1} &= \bS^{-1} \left( ( \bI - (\bI - \eta \bS)^t ) - (\bI + \bS^{-1} \lambda )^{-1}\right) \\
%         &= \underbrace{\bS^{-1} \left( ( \bI - (\bI - \eta \bS)^t ) - (\bI + ( \bS t \eta)^{-1}  )^{-1}\right)}_{\bA}
%     \end{align}

%     We now separately bound the diagonal entries in matrix $\bA$. 
%     With $s_i$, we denote $i^{\text{th}}$ diagonal entry of $\bS$. Note that since $ \eta\le 1/\norm{S}{\text{op}}$, for all $i$, $\eta s_i  \le 1$.  Consider $i^{\text{th}}$ diagonal term (which is non-zero) of the diagonal matrix $\bA$, we have 
%     \begin{align}
%         \bA_{ii} = \frac{1}{s_i} \left(  1 - (1 - s_i \eta)^t - \frac{t \eta s_i}{1 + t \eta s_i } \right) &=  \frac{1 - (1 - s_i \eta)^t}{s_i} \left( \underbrace{ 1 - \frac{t \eta s_i}{(1 + t \eta s_i)(1 - (1 - s_i \eta)^t)}}_{\RN{2}} \right) \\ 
%          &\le \frac{1}{2}\left[ \frac{1 - (1 - s_i \eta)^t}{ s_i} \right] \tag*{(Using \lemref{lem:ineq_soln} (b))} \,.
%     \end{align} 
%     Additionally, we can also show the following upper bound on term 2: 
%     \begin{align}
%          1 - \frac{t \eta s_i}{(1 + t \eta s_i)(1 - (1 - s_i \eta)^t)} &= \frac{(1 + t \eta s_i)(1 - (1 - s_i \eta)^t) - t \eta s_i }{(1 + t \eta s_i)(1 - (1 - s_i \eta)^t)} \\
%          & \le  \frac{ 1 -  (1 - s_i \eta)^t - t \eta s_i (1 - s_i \eta)^t}{(1 + t \eta s_i)(1 - (1 - s_i \eta)^t)} \\
%          & \le \frac{1}{t\eta s_i} \,. \tag{Using \lemref{lem:ineq_soln} (a)}
%         %  &\le \frac{1}{2}\left[ \frac{1 - (1 - s_i \eta)^t}{ s_i} \right] \tag*{(Using \lemref{lem:ineq_soln})} \,.
%     \end{align} 

%     Combining both the upper bounds on each diagonal entry $\bA_{ii}$, we have 
%     \begin{align}
%     \bA \preceq c_1(\eta, t) \cdot \bS^{-1} ( \bI - (\bI - \eta \bS)^t ) \,, \label{eq:upperbound_diagonal}
%     \end{align}
%     where $c_1(\eta, t ) = \min(0.5, \frac{1}{t s_i \eta })$. Plugging this into \eqref{eq:train_GD_rel}, we have 
%     \begin{align}
%         \Expt{ x \sim \calS }{\left(f_t(x) - \wt f_\lambda (x)\right)^2} &\le c(\eta, t) \cdot \by^T \bV \bX  \left( \bS^{-1} ( \bI - (\bI - \eta \bS)^t ) \right)^2 \bS \bV^T \bX^T \by \\
%         &=   c(\eta, t) \cdot \by^T \bV \bX  \left( \bS^{-1} ( \bI - (\bI - \eta \bS)^t ) \right) \bS \left( \bS^{-1} ( \bI - (\bI - \eta \bS)^t ) \right) \bV^T \bX^T \by \\
%         & =  c(\eta, t) \cdot \norm{\bX w_t}{2}^2 \\
%         &= c(\eta, t) \cdot  \Expt{ x \sim \calS }{\left(f_t(x) \right)^2} \,,
%     \end{align}
%     where $c(\eta, t ) = \min(0.25, \frac{1}{t^2 s^2_i \eta^2 })$.

%     \textbf{Part 2 {} {}} With $\bSigma$, we denote the underlying true covariance matrix. We now consider the squared difference of output at an unseen point: 
%     \begin{align}
%         \Expt{ x \sim \calD_{\calX} }{\left(f_t(x) - \wt f_\lambda (x)\right)^2} &= \Expt{x \sim \calD_{\calX}}{\norm{x^T w_t - x^T \wt w_\lambda}{2}} \\
%         &=   \norm{x^T \bV \bS ^{-1} ( \bI - (\bI - \eta \bS)^t ) \bV^T \bX^T \by - x^T \bV (\bS + \lambda \bI )^{-1} \bV^T \bX^T \by }{2} \\
%         &= \norm{x^T \bV \left(\bS ^{-1} ( \bI - (\bI - \eta \bS)^t ) - (\bS + \lambda \bI )^{-1} \right) \bV^T \bX^T \by  }{2} \\
%         &= \by^T \bV \bX \left( \bS ^{-1} ( \bI - (\bI - \eta \bS)^t ) - (\bS + \lambda \bI )^{-1} \right) \bV^T \bSigma \bV \\ &\qquad \qquad \qquad \qquad \qquad \left( (\bI - (\bI - \eta \bS)^t ) - (\bS + \lambda \bI )^{-1} \right) \bV^T \bX^T \by \\
%         &\le \sigma_{\text{max}} \cdot \by^T \bV \bX \left( \underbrace{\bS ^{-1} ( \bI - (\bI - \eta \bS)^t ) - (\bS + \lambda \bI )^{-1}}_{\RN{1}} \right)^2 \bV^T \bX^T \by \,, \label{eq:test_GD_rel}
%         %  (\bX \bV \bS ^{-1} ( \bI - (\bI - \eta \bS)^k ) \bV^T \bX^T \by)^T \bX \bV \bS ^{-1} ( \bI - (\bI - \eta \bS)^k ) \bV^T \bX^T \by
%     \end{align}
%     where $\sigma_{\text{max}}$ is the maximum eigenvalue of the underlying covariance matrix $\bSigma$. Using the upper bound on term 1 in \eqref{eq:upperbound_diagonal}, we have 
%     \begin{align}
%         \Expt{ x \sim \calD_{\calX} }{\left(f_t(x) - \wt f_\lambda (x)\right)^2} &\le \sigma_{\text{max}} \cdot c(\eta, t) \cdot \by^T \bV \bX  \left( \bS^{-1} ( \bI - (\bI - \eta \bS)^t ) \right)^2 \bV^T \bX^T \by \\
%         &=   \kappa \cdot c(\eta, t) \cdot \sigma_{\text{min}}\cdot \norm{\bV \left( \bS^{-1} ( \bI - (\bI - \eta \bS)^t ) \right) \bV^T \bX^T \by}{2}^2 \\
%         &\le \kappa \cdot c(\eta, t) \cdot \left[ \bV \left( \bS^{-1} ( \bI - (\bI - \eta \bS)^t ) \right) \bV^T \bX^T \right]^T \bSigma \\
%         &\qquad \qquad \qquad \qquad \qquad \left[ \bV \left( \bS^{-1} ( \bI - (\bI - \eta \bS)^t ) \right) \bV^T \bX^T \right] \by \\
%         & = \kappa \cdot c(\eta, t) \cdot \Expt{x \sim \calD_{\calX}}{\norm{x^T w_t}{2}} \,.
%     \end{align}
% % 
% % 
%     % Since $ \eta\le 1/\norm{S}{\text{op}}$, invoking \lemref{lem:ineq_soln} to upper bound term 1 with
% \end{proof}


% \newpage
% \section{Additional experiments and details}\label{app:exp}
% \newcommand\tab[1][1cm]{\hspace*{#1}}

% \subsection{Datasets} \label{sec:app_dataset}

% \textbf{Toy Dataset {} {}} Assume fixed constants $\mu$ and $\sigma$. For a given label $y$, we simulate features $x$ in our toy classification setup as follows: 
% \begin{align*}
%     x \defeq \texttt{concat} \left[ x_1, x_2\right] \quad \text{where} \quad  x_1 \sim  \calN( y \cdot \mu, \sigma^2 I_{d \times d}) \ \  \text{and} \ \  x_1 \sim  \calN( 0, \sigma^2 I_{d \times d}) \,.
% \end{align*}  
% % where $y$ is the true label and $x$ is the corresponding feature vector. 
% In experiements throughout the paper, we fix dimention $d=100$, $\mu = 1.0 $, and $\sigma = \sqrt{d}$. Intuitively, $x_1$ carries the information about the underlying label and $x_2$ is additional noise independent of the underlying label. 

% \textbf{CV datasets {} {}} We use MNIST~\citep{lecun1998mnist} and CIFAR10~\cite{krizhevsky2009learning}. 
% % For binary tasks, 
% We produce a binary variant from the multiclass classification problem by mapping classes $\{0,1,2,3,4\}$ to label $1$ and $\{ 5,6,7,8,9\}$ to label $-1$. For CIFAR dataset, we also use the standard data augementation of random crop and horizontal flip. PyTorch code is as follows: 

% \texttt{(transforms.RandomCrop(32, padding=4),\\
% \tab transforms.RandomHorizontalFlip())}

% \textbf{NLP dataset {} {}} We use IMDb Sentiment analysis~\citep{maas2011learning} corpus.  

% \subsection{Architecture Details} 

% All experiments were run on NVIDIA GeForce RTX 2080 Ti GPUs. We used PyTorch~\citep{NEURIPS2019a9015} and Keras with Tensorflow~\citep{abadi2016tensorflow} backend for experiments. 
% % , ELMo embeddings~\citep{Peters:2018}, and Hugging Face Transformers~\citep{wolf-etal-2020-transformers}. 

% \textbf{Linear model {} {}} For the toy dataset, we simulate a linear model with scalar output and the same number of parameters as the number of dimensions.   

% \textbf{Wide nets {} {}} To simulate the NTK regime, we experiment with $2-$layered wide nets. The PyTorch code for 2-layer wide MLP is as follows: 


% \texttt{ nn.Sequential( \\
% \tab     nn.Flatten(),\\
% \tab    nn.Linear(input\_dims, 200000, bias=True),\\
% \tab    nn.ReLU(),\\
% \tab    nn.Linear(200000, 1, bias=True)\\
% \tab     )}


% We experiment both (i) with the first layer fixed at random initialization; (ii)  and updating both layers' weights.     

% \textbf{Deep nets for CV tasks {} {}} We consider a 4-layered MLP. The PyTorch code for 4-layer MLP is as follows: 

% \texttt{ nn.Sequential(nn.Flatten(), \\
% \tab        nn.Linear(input\_dim, 5000, bias=True),\\
% \tab        nn.ReLU(),\\
% \tab        nn.Linear(5000, 5000, bias=True),\\
% \tab        nn.ReLU(),\\
% \tab        nn.Linear(5000, 5000, bias=True),\\
% \tab        nn.ReLU(),\\
% % \tab        nn.Linear(5000, 5000, bias=True),\\
% % \tab        nn.ReLU(),\\
% \tab        nn.Linear(1024, num\_label, bias=True)\\
% \tab        )}

% For MNIST, we use $1000$ nodes instead of $5000$ nodes in the hidden layer. 
% % 
% We also experiment with convolutional nets. In particular, we use ResNet18 \citep{he2016deep}. Implementation adapted from:  \url{https://github.com/kuangliu/pytorch-cifar.git}. 

% \textbf{Deep nets for NLP {} {}} We use a simple LSTM model with embeddings intialized with ELMo embeddings~\citep{Peters:2018}. Code adapted from: \url{https://github.com/kamujun/elmo_experiments/blob/master/elmo_experiment/notebooks/elmo_text_classification_on_imdb.ipynb} 

% We also evaluate our bounds with a BERT model. In particular, we fine-tune an off-the-shelf uncased BERT model~\citep{devlin2018bert}. Code adapted from Hugging Face Transformers~\citep{wolf-etal-2020-transformers}: \url{https://huggingface.co/transformers/v3.1.0/custom_datasets.html}. 


% \subsection{Additonal experiments}

% 1. SGD with linear models on cross entropy and MSE loss. 

% 2. CE loss and SGD. GD with MSE loss 

% 3. Binary MNIST with MLP. multiclass MNIST  

% \textbf{Results on CIFAR 10 {} {}} 
% % 
% We plot epoch wise error curve for results in \tabref{table:multiclass}. We observe the same trend as in \figref{fig:error_CIFAR10}. Additionally, we plot an \emph{oracle bound} obtained by tracking the error on mislabeled data which nevertheless were predicted as true label. To obtain an exact emprical value of the oracle bound, we need underlying true labels for the randomly labeled data. 
% % Note that our bound in \thmref{thm:multiclass_ERM}, lower bounds the accuracy as predicted by the oracle bound. 
% While with just access to extra unlabeled data we cannot calculate oracle bound, we note that the oracle bound is very tight and never violated in practice underscoring an importamt aspect of generalization in multiclass problems. This highlight that even a stronger conjecture may hold in multiclass classification, i.e., error on mislabeled data (where nevertheless true label was predicted) lower bounds the population error on the distribution of mislabeled data and hence, the error on (a specific) mislabeled portion predicts the population accuracy on clean data. 
% % 
% On the other hand, the dominating term of in \thmref{thm:multiclass_ERM} is loose when compared with the oracle bound. The main reason, we believe is the pessimistic upper bound in \eqref{eq:lemma1_final_multi_prev} in the proof of \lemref{lem:fit_mislabeled_multi}. We leave an investigation on this gap for future. 
% % of fit 

% % However, oracle bound highlights two . One,  



% \begin{figure}[h]
%     \centering 
%     % \vspace{-15pt}
%     % \includegraphics[width=0.9\linewidth]{example-image-a}
%     \includegraphics[width=0.4\linewidth]{figures/CIFAR10-FNN.pdf} \hfil
%     \includegraphics[width=0.4\linewidth]{figures/CIFAR10-Resnet.pdf}
%     % \includegraphics[width=0.9\linewidth]{figures/{CIFAR10_rn=0.1_lr=0.2_wd=0.005}.png}
%     % \vspace{-10pt}
%     \caption{ Per epoch curves for CIFAR10 corresponding results in \tabref{table:multiclass}. As before, we just plot the dominating term in the RHS of \eqref{eq:multiclass_ERM} as predicted bound. Additionally, we also plot the predicted lower bound by the error on mislabeled data which nevertheless were predicted as true label. We refer to this as ``Oracle bound''. See text for more details. 
%     % 
%     % except for the stopping point. 
%     % The bound predicted by RATT (RHS in \eqref{eq:multiclass_ERM}) is vacuous. 
%     }\label{fig:error_epoch_CIFAR10}
%     % \vspace{-15pt}
% \end{figure}


% \textbf{Results on CIFAR 100 {} {}} 
% % 
% On CIFAR100, our bound in \eqref{eq:multiclass_ERM} yields vacous bounds. However, the oracle bound as explained above yields tight guarantees in the initial phase of the learning (i.e., when learning rate is less than $0.1$). 

% \begin{figure}[h]
%     \centering 
%     % \vspace{-15pt}
%     % \includegraphics[width=0.9\linewidth]{example-image-a}
%     \includegraphics[width=0.4\linewidth]{figures/CIFAR100-Resnet.pdf}
%     % \includegraphics[width=0.9\linewidth]{figures/{CIFAR10_rn=0.1_lr=0.2_wd=0.005}.png}
%     % \vspace{-10pt}
%     \caption{ Predicted lower bound by the error on mislabeled data which nevertheless were predicted as true label with ResNet18 on CIFAR100. We refer to this as ``Oracle bound''. See text for more details. 
%     % 
%     % except for the stopping point. 
%     The bound predicted by RATT (RHS in \eqref{eq:multiclass_ERM}) is vacuous. 
%     }\label{fig:error_CIFAR100}
%     % \vspace{-15pt}
% \end{figure}


% % \paragraph{Experiments on CIFAR100} 



% \subsection{Hyperparameter Details}


% \textbf{\figref{fig:error_CIFAR10} {} {}} We use clean training dataset of size $40,000$. We fix the amount of unlabeled data at $20\%$ of the clean size, i.e. we include additional $8,000$ points with randomly assigned labels. We use test set of $10,000$ points. For both MLP and ResNet, we use SGD with an initial learning rate of $0.1$ and momentum $0.9$. We fix the weight decay parameter at $5\times 10^{-4}$. After $100$ epochs, we decay the learning rate to $0.01$. We use SGD batch size of $100$. 

% \textbf{\figref{fig:error_binary} (a) {} {}} We obtain a toy dataset according to the process described in \secref{sec:app_dataset}. We fix $d=100$ and create a dataset of $50,000$ points with balanced classes. Moreover, we sample additional covariates with the same procedure to create randomly labeled dataset. For both SGD and GD training, we use a fixed learning rate $0.1$.    

% \textbf{\figref{fig:error_binary} (b) {} {}} Similar to binary CIFAR, we use clean training dataset of size $40,000$ and fix the amount of unlabeled data at $20\%$ of the clean dataset size. To train wide nets, we use a fixed learning of $0.001$ with GD and SGD. We decide the weight decay parameter and the early stopping point that maximizes our generalization bound (i.e. without peeking at unseen data ).  We use SGD batch size of $100$. 

% \textbf{\figref{fig:error_binary} (c) {} {}} With IMDb dataset, we use a clean dataset of size $20,000$ and as before, fix the amount of unlabeled data at $20\%$ of the clean data. To train ELMo model, we use Adam optimizer with a fixed learning rate $0.01$ and weight decay $10^{-6}$ to minimize cross entropy loss. We train with batch size $32$ for 3 epochs. To fine-tune BERT model, we use Adam optimizer with learning rate $5\times 10^{-5}$ to minimize cross entropy loss. We train with a batch size of $16$ for 1 epoch.    

% \textbf{\tabref{table:multiclass} {} {}} For multiclass datasets, we train both MLP and ResNet with the same hyperparameters as described before. We sample a clean training dataset of size $40,000$ and fix the amount of unlabeled data at $20\%$ of the clean size. We use SGD with an initial learning rate of $0.1$ and momentum $0.9$. We fix the weight decay parameter at $5\times 10^{-4}$. After $30$ epochs for ResNet and after $50$ epochs for MLP, we decay the learning rate to $0.01$.  We use SGD with batch size $100$. 
% For \figref{fig:error_CIFAR100}, we use the same hyperparameters as 
% CIFAR10 training, except we now decay learning rate after $100$ epochs. 


% In all experiments, to identify the best possible accuracy on just the clean data, we use the exact same set of hyperparamters except the stopping point. We choose a stopping point that maximizes test performance. 

% \subsection{Summary of experiments }

% \begin{center}
%     \begin{table}[H] 
%         \centering
%         \begin{tabular}{|c|c|c|c|} 
%         \hline
%         Classification type & Model category & Model & Dataset  \\ [0.5ex] 
%         \hline
%         \hline
%         \multirow{9}{*}{Binary} & Low dimensional & Linear model & Toy Gaussain dataset  \\
%                         \cline{2-4}
%                          & \multirow{1}{*}{Overparameterized linear nets} 
%                         %  & Linear model & Toy Gaussain dataset \\
%                         %  \cline{3-4}
%                         %  & & 2-layer wide net& Toy Gaussain dataset \\
%                         %  \cline{3-4}
%                          & 2-layer wide net & Binary MNIST \\
%                          \cline{2-4}                 
%                          & \multirow{6}{*}{Deep nets} & \multirow{2}{*}{MLP} & Binary MNIST \\
%                          \cline{4-4}
%                          & &  & Binary CIFAR \\
%                          \cline{3-4}
%                          &  & \multirow{2}{*}{ResNet} & Binary MNIST \\
%                          \cline{4-4}
%                          & &  & Binary CIFAR \\
%                          \cline{3-4}
%                          &  & ELMo-LSTM model & IMDb Sentiment Analysis \\
%                          \cline{3-4}
%                          & & BERT pre-trained model & IMDb Sentiment Analysis \\
%         \hline
%         \multirow{5}{*}{Multiclass} & \multirow{5}{*}{Deep nets} & \multirow{2}{*}{MLP} & MNIST \\
%                         \cline{4-4} 
%                         & & & CIFAR10 \\                   
%                         \cline{3-4}
%                          &   & \multirow{3}{*}{ResNet} & MNIST \\
%                          \cline{4-4}
%                          &   & & CIFAR10 \\
%                          \cline{4-4}
%                          &   & & CIFAR100 \\
%         \hline
%         \end{tabular}
%         % \caption{Summary of experiments performed} \label{table:experiments}
%     \end{table}    
%     % \footnotetext[6]{We use both MSE loss and cross-entropy loss.}
%     % \footnotetext[6]{We try 2 variants: one with a fixed first layer and the other with both layers trainable.}
% \end{center}

% \newpage
% \section{Proof of \lemref{lem:stability_error}} \label{app:proof_lem_error}

% \begin{proof}[Proof of \lemref{lem:stability_error}]
%     Recall, we have a training set $S \cup \wt S_C$. We defined leave-one-out error on mislabeled points as $$\error_{\text{LOO}(\wt S_M) } = \frac{\sum_{(x_i, y_i) \in \wt S_M} \error( f_{(i)}( x_i), y_i)}{ \abs{\wt S_M }} \,, $$
%     where $f_{(i)} \defeq f(\calA, (S \cup \wt S)_{(i)})$. Define $S^\prime \defeq S \cup \wt S$. Assume $(x,y)$ and $(x^\prime,y^\prime)$ as i.i.d. samples from ${\calDm}$. 
%     Using Lemma 25 in \citet{bousquet2002stability}, we have
%     \begin{align*}
%         \Expo{ \left( \error_{\calDm}(\wh f) -\error_{\text{LOO}(\wt S_M) } \right)^2 } \le & \Expt{ S^\prime, (x,y), (x^\prime,y^\prime) }{ \error(\wh f(x), y ) \error(\wh f(x^\prime), y^\prime )} - 2 \Expt{ S^\prime, (x,y) }{ \error(\wh f(x), y ) \error(f_{(i)}(x_i), y_i )} \\
%         & + \frac{m_1-1}{m_1}\Expt{ S^\prime }{  \error(f_{(i)}(x_i), y_i )  \error(f_{(j)}(x_j), y_j )} + \frac{1}{m_1} \Expt{ S^\prime }{  \error(f_{(i)}(x_i), y_i ) } \,. \numberthis \label{eq:main_reln}
%     \end{align*}
%     We can rewrite the equation above as : 
%     \begin{align*}
%         \Expo{ \left( \error_{\calDm}(\wh f) -\error_{\text{LOO}(\wt S_M) } \right)^2 } \le &  \, \underbrace{\Expt{ S^\prime, (x,y), (x^\prime,y^\prime) }{ \error(\wh f(x), y ) \error(\wh f(x^\prime), y^\prime ) - \error(\wh f(x), y ) \error(f_{(i)}(x_i), y_i )}}_{\RN{1}} \\
%         & + \underbrace{\Expt{ S^\prime }{  \error(f_{(i)}(x_i), y_i )  \error(f_{(j)}(x_j), y_j ) -  \error(\wh f(x), y ) \error(f_{(i)}(x_i), y_i )}}_{\RN{2}} \\ &+ \underbrace{\frac{1}{m_1} \Expt{ S^\prime }{  \error(f_{(i)}(x_i), y_i ) - \error(f_{(i)}(x_i), y_i )  \error(f_{(j)}(x_j), y_j ) }}_{\RN{3}} \,. \numberthis \label{eq:main_reln2}
%     \end{align*}
    
%     We will now bound term $\RN{3}$.  Using Schwarz's inequality, we have
    
%     \begin{align}
%         \Expt{ S^\prime }{  \error(f_{(i)}(x_i), y_i ) - \error(f_{(i)}(x_i), y_i )  \error(f_{(j)}(x_j), y_j ) }^2 &\le  \Expt{ S^\prime }{  \error(f_{(i)}(x_i), y_i ) }^2 \Expt{S^\prime}{1 -   \error(f_{(j)}(x_j), y_j ) }^2 \\
%         &\le \frac{1}{4} \label{eq:term1_lem12}
%     \end{align}
    
%     Note that since $(x_i,y_i)$, $(x_j ,y_j )$, $(x,y)$, and $(x^\prime, y^\prime)$ are all from same distribution $\calDm$, we directly incorporate the bounds on term $\RN{1}$ and $\RN{2}$ from proof of Lemma 9 in \citet{bousquet2002stability}. Combining that with \eqref{eq:term1_lem12} and our definition of hypothesis stability in \codref{cond:hypothesis_stability}, we have the required claim. 
    
    
%     % We now re-write term $\RN{1}$ as
%     % \begin{align*}
%     %         &\Expt{S^\prime, (x,y), (x^\prime,y^\prime) }{ \error(\wh f(x), y ) \error(\wh f(x^\prime), y^\prime ) - \error(\wh f(x), y ) \error(f_{(i)}(x_i), y_i )} \\ & \qquad = \Expt{ S^\prime, (x,y), (x^\prime,y^\prime) }{ \error(\wh f(x), y ) \error(\wh f  (x^\prime), y^\prime ) - \error(\wh f ^\prime(x), y ) \error(f_{(i)}(x^\prime), y^\prime )} \tag{Exchanging $(x_i, y_i)$ with $(x^\prime, y^\prime)$ in the second term} \\
%     %         & \qquad = \Expt{ S^\prime, (x,y), (x^\prime,y^\prime) }{  \left(\error(\wh f(x), y )-  \error(f_{(i)}(x), y ) \right) \error(\wh f  (x^\prime), y^\prime )  } \\
%     %         & \qquad  + \Expt{ S^\prime, (x,y), (x^\prime,y^\prime) }{  \left(\error(f_{(i)}(x), y ) -\error(\wh f ^\prime(x), y ) \right) \error(\wh f  (x^\prime), y^\prime )}  \\
%     %         & \qquad +\Expt{ S^\prime, (x,y), (x^\prime,y^\prime) }{  \left( \error(\wh f  (x^\prime), y^\prime ) -  \error(f_{(i)}(x^\prime), y^\prime ) \right) \error(\wh f ^\prime(x), y ) }  \,, \numberthis \label{eq:term1_final}
%     % \end{align*}
%     % where $\wh f^\prime$ is the classifier obtained by training on $ S^\prime_{(i)} \cup \{ (x^\prime, y^\prime) \} $. Similarly we can re-write term $\RN{2}$ as 
%     % \begin{align*}
%     %     & \Expt{ S^\prime }{  \error(f_{(i)}(x_i), y_i )  \error(f_{(j)}(x_j), y_j ) -  \error(\wh f(x), y ) \error(f_{(i)}(x_i), y_i )} \\
%     %     &\quad  = \Expt{ S^\prime, (x,y), (x^\prime,y^\prime)}{  \error(f^{\prime\prime}_{(i)}(x), y )  \error(f_{(j)}^{\prime}(x^\prime), y^\prime ) -  \error(\wh f(x), y ) \error(f_{(i)}(x_i), y_i )} \tag{Exchanging $(x_i, y_i)$ with $(x, y)$ and $(x_j, y_j)$ with $(x^\prime, y^\prime)$ in the first term}\\
%     %     &\quad = \Expt{ S^\prime, (x,y), (x^\prime,y^\prime)}{  \error(f^{\prime\prime}_{(j)}(x), y )  \error(f_{(i)}^{\prime}(x^\prime), y^\prime ) -  \error(\wh f^\prime (x), y ) \error(f^\prime_{(j)}(x^\prime), y^\prime )} \tag{Exchanging $(x_i, y_i)$ and $(x_j, y_j)$ and then replacing $(x_j, y_j)$ with $(x^\prime, y^\prime)$ in the second term} \\
%     %     & \quad = \Expt{ S^\prime, (x,y), (x^\prime,y^\prime) }{  \left( \error(f_{(i)}^{\prime}(x^\prime), y^\prime )   -  \error(\wh f^{\prime\prime}  (x^\prime), y^\prime ) \right)  \error(f^{\prime\prime}_{(j)}(x), y )   } \\
%     %     & \quad  + \Expt{ S^\prime, (x,y), (x^\prime,y^\prime) }{  \left( \error(f^{\prime\prime}_{(j)}(x), y )  -\error(\wh f ^\prime(x), y ) \right) \error(\wh f^{\prime\prime}  (x^\prime), y^\prime )  }  \\
%     %     & \quad+ \Expt{ S^\prime, (x,y), (x^\prime,y^\prime) }{  \left( \error(\wh f^{\prime\prime}  (x^\prime), y^\prime )  -  \error(f^\prime_{(j)}(x^\prime), y^\prime ) \right)  \error(\wh f^\prime (x), y ) }   \\
%     %     & \quad = \Expt{ S^\prime, (x,y), (x^\prime,y^\prime) }{  \left( \error(f_{(i)}^{\prime}(x^\prime), y^\prime )   -  \error(\wh f (x^\prime), y^\prime ) \right)  \error(f_{(i)}(x_j), y_j )   } \\
%     %     & \quad  + \Expt{ S^\prime, (x,y), (x^\prime,y^\prime) }{  \left( \error(f^{\prime\prime}_{(j)}(x), y )  -\error(\wh f (x), y ) \right) \error(\wh f^{\prime\prime}  (x_j), y_j )  }  \\
%     %     & \quad+ \Expt{ S^\prime, (x,y), (x^\prime,y^\prime) }{  \left( \error(\wh f^{\prime\prime}  (x^\prime), y^\prime )  -  \error(f^\prime_{(j)}(x^\prime), y^\prime ) \right)  \error(\wh f^\prime (x^\prime), y^\prime ) }  \,, \numberthis \label{eq:term2_final}
%     % \end{align*}
%     % where $f^{\prime\prime}_{(j)}$ is trained on $S^\prime_{(j,i)} \cup {(x,y)}$, $f^{\prime}_{(i)}$ is trained on $S^\prime_{(j,i)} \cup {(x^\prime,y^\prime)}$, and $\wh f^{\prime\prime} $ is trained on $S^\prime_{(j)} \cup {(x,y)}$. Note in the last line we replaced $(x,y)$ by $(x_j, y_j)$ in the first term, replaced $(x^\prime,y^\prime)$ by $(x_j, y_j)$ in the second term and exchanged $(x_i,y_i)$ with $(x_j,y_j)$ and also $(x,y)$ and $(x^\prime, y^\prime)$
    
    
% \end{proof}
\newpage
%\bibliographystyle{ACM-Reference-Format}
\bibliographystyle{plain}	
\bibliography{Yang}
\end{document} 