\section*{\huge{Appendix}}
\section{Our Mechanisms}

Here are the detailed description of the two major mechanisms we use: Sequential Posted Price Mechanism (SPM) and Anonymous Sequential Posted Price with Entry Fee Mechanism (ASPE). We also use the Rationed Sequential Posted Price Mechanism (RSPM) when bidders are not unit-demand. RSPM is almost identical to SPM except that there is an extra constraint saying that no bidder can purchase more than one item.
\begin{algorithm}[ht]
\begin{algorithmic}[1]
\REQUIRE $P_{ij}$ is the price for bidder $i$ to purchase item $j$.
\STATE $S\gets [m]$
\FOR{$i \in [n]$}
	\STATE Show bidder $i$ {the} set of available items $S$.
	 \STATE $i$ purchases her favorite bundle $S_i^{*}\in \max_{S'\subseteq S} v_i(t_i, S') - \sum_{j\in S'} P_{ij}$ and pays $\sum_{j\in S_i^{*}}P_{ij}$.
        \STATE $S\gets S\backslash S_i^{*}$.
\ENDFOR
\end{algorithmic}
\caption{{\sf Sequential Posted Price Mechanism (SPM)}}
\label{alg:seq-mech}
\end{algorithm} 

\begin{algorithm}[ht]
\begin{algorithmic}[1]
\REQUIRE A collection of prices $\{p_{j}\}_{j\in[m]}$ and a collection of entry fee functions $\{\delta_i(\cdot)\}_{i\in[n]}$ where $\delta_i: 2^{[m]}\mapsto \mathbb{R}$ is bidder $i$'s entry fee function.
\STATE $S\gets [m]$
\FOR{$i \in [n]$}
	\STATE Show bidder $i$ {the} set of available items $S$ and set the entry fee for bidder $i$ to be ${\delta_i}(S)$.
    \IF{Bidder $i$ pays the entry fee ${\delta_i}(S)$}
        \STATE $i$ receives her favorite bundle $S_i^{*}$ and pays $\sum_{j\in S_i^{*}}p_{j}$.
        \STATE $S\gets S\backslash S_i^{*}$.
    \ELSE
        \STATE $i$ gets nothing and pays $0$.
    \ENDIF
\ENDFOR
\end{algorithmic}
\caption{{\sf Anonymous Sequential Posted Price with Entry Fee Mechanism (ASPE)}}
\label{alg:aspe-mech}
\end{algorithm} 




\section{Missing Details from Section~\ref{sec:unit-demand}}\label{sec:unit-demand appx}
\subsection{Unit-demand Valuations: direct access to approximate distributions}\label{sec:unit-demand Kolmogorov}
We first consider the model where we only have access to an approximate distribution $\hat{D}$. The following definition is crucial for proving our result. 

\begin{definition}
For any single dimensional distribution $\DD$ with cdf $F$, we define its revenue curve $R_{\DD}: [0,1]\mapsto \mathbb{R}_{\geq 0}$ as
	\begin{align*}
	{R}_{\DD} (q) = &\max  x\cdot \ubar{q}\cdot {F}^{-1}(1-\ubar{q}) +(1-x)\cdot \bar{q}\cdot {F}^{-1}(1-\bar{q})\\
	& \qquad\textbf{s.t. } x\cdot \ubar{q}+(1-x)\cdot \bar{q}=q\\
	&  \qquad\qquad x, \ubar{q}, \bar{q} \in [0,1]
\end{align*}
where $F^{-1}(1-p) = \sup\{x\in R: \Pr_{v\sim \DD}[v\geq x]\geq p\}$.
\end{definition}

% For any bidder $i$ and item $j$, define function $R_{\hat{D}_{ij}}: [0,1]\rightarrow \mathbb{R}_{\geq 0}$ as:
%We define $R_{\hat{D}_{ij}}$ similarly with respect to $\hat{F}_{ij}$.
%It is not hard to argue that $R_{\hat{D}_{ij}}(\cdot)$ is exactly the ironed revenue curve for distribution $\hat{D}_{ij}$.

\begin{lemma}[Folklore]
	Let ${\varphi}_{ij}(\cdot)$ and $\hat{\varphi}_{ij}(\cdot)$ be the ironed virtual value function for distribution $D_{ij}$ and $\hat{D}_{ij}$ respectively, then for any $q\in [0,1]$, ${R}_{D_{ij}}(q) = \int_{{F}^{-1}_{ij}(1-q)}^H {\varphi}(x) dF(x)$ and $R_{\hat{D}_{ij}}(q) = \int_{\hat{F}^{-1}_{ij}(1-q)}^H \hat{\varphi}(x) dF(x)$. Since the ironed virtual value function is monotonically non-decreasing, ${R}_{D_{ij}}(\cdot)$ and $R_{\hat{D}_{ij}}(\cdot)$ are concave functions. 
\end{lemma}
%\yangnote{Fill in the proof later.}

We provide an upper bound of the optimal revenue using $R_{D_{ij}}$ in the next Lemma. To do that, we first need the definition of the \emph{Single-Dimensional Copies Setting}.

\vspace{.1in}

\noindent\textbf{Single-Dimensional Copies Setting:} In the analysis for unit-demand bidders in~\cite{ChawlaHMS10, CaiDW16}, the optimal revenue is upper bounded by the optimal revenue in the single-dimensional copies setting defined in~\cite{ChawlaHMS10}. We use the same technique. We construct $nm$ agents, where agent $(i,j)$ has value $V_i(t_{ij})$ of being served with $t_{ij}\sim D_{ij}$, and we are only allow to use matchings, that is, for each $i$ at most one agent $(i,k)$ is served and for each $j$ at most one agent $(k,j)$ is served\footnote{This is exactly the copies setting used in~\cite{ChawlaHMS10}, if every bidder $i$ is unit-demand and has value $V_i(t_{ij})$ with type $t_i$. Notice that this unit-demand multi-dimensional setting is equivalent as adding an extra constraint, each buyer can purchase at most one item, to the original setting with subadditive bidders.}. %Under the new feasibility constraint, every buyer's valuation is unit-demand and bidder $i$ with type $t_i$ has value $V_i(t_{ij})$ for item $j$. Now we construct the copies setting as in~\cite{ChawlaHMS10}.
Notice that this is a single-dimensional setting, as each agent's type is specified by a single number. Let $\copies$ be the optimal BIC revenue in this copies setting.


\begin{lemma}\label{lem:UB for UD rev}
For unit-demand bidders, there exists a collection of non-negative numbers $\{q_{ij}\}_{i\in[n], j\in[m]}$ satisfying $\sum_i q_{ij}\leq 1$ for all $j\in [m]$ and $\sum_j q_{ij}\leq 1$ for all $i\in [n]$, such that the optimal revenue $$\opt\leq 4\cdot \sum_{i,j} R_{D_{ij}}(q_{ij}).$$
\end{lemma}
\begin{proof}
As shown in~\cite{CaiDW16}, $\opt\leq 4 \copies$. Let $q_{ij}$ be the ex-ante probability that agent $(i,j)$ is served in the optimal mechanism for the copies setting. Chawla et al.~\cite{ChawlaHMS10} showed that $\copies\leq \sum_{i,j} R_{D_{ij}}(q_{ij})$. Our statement follows from the two inequalities above.\end{proof}

Next, we consider a convex program (Figure~\ref{fig:CP unit demand}) and argue that the value of the optimal solution of this program is at least $\frac{1}{8}$ of the optimal revenue.
\begin{figure}[ht]
%\colorbox{MyGray}{
\begin{minipage}{\textwidth} 
\begin{align*}\label{prog:convex ud}
&\max \sum_{i,j} R_{D_{ij}}(q_{ij})\\
\textbf{s.t. }& \sum_i q_{ij}\leq \frac{1}{2}\qquad \text{ for all $j\in[m]$}\\
& \sum_j q_{ij}\leq \frac{1}{2}\qquad \text{ for all $i\in[n]$}\\
& q_{ij}\geq 0\qquad \text{ for all $i\in[n]$ and $j\in[m]$}
\end{align*}
\end{minipage}
\caption{A Convex Program for Unit-demand Bidders with Exact Distributions.}
\label{fig:CP unit demand}
\end{figure}

\begin{lemma}\label{lem:compare exact CP with opt}
	The optimal solution of convex program in Figure~\ref{fig:CP unit demand} is at least $\frac{\opt}{8}$.
\end{lemma}
\begin{proof}
	Let $\{q'_{ij}\}$ be the collection of nonnegative numbers in Lemma~\ref{lem:UB for UD rev}. Clearly, $\left\{\frac{q_{ij}'}{2}\right\}$ is a set of feasible solution for the convex program. Since $R_{D_{ij}}(\cdot)$ is concave, $R_{D_{ij}}\left(\frac{q_{ij}'}{2}\right)\geq \frac{R_{D_{ij}}(q'_{ij})}{2} + \frac{R_{D_{ij}}(0)}{2}=\frac{R_{D_{ij}}(q'_{ij})}{2}$. Therefore, $$\sum_{i,j} R_{D_{ij}}\left(\frac{q_{ij}'}{2}\right)\geq 
	\frac{1}{2}\cdot \sum_{i,j} R_{D_{ij}}(q'_{ij})\geq \frac{\opt}{8}.$$
\end{proof}

If we know all $F_{ij}$ exactly, we can solve the convex program (Figure~\ref{fig:CP unit demand}) and use the optimal solution to construct an SPM via an approach provided in~\cite{ChawlaHMS10,CaiDW16}. The constructed sequential posted mechanism has revenue at least $\frac{1}{4}$ of the optimal value of the convex program, which is at least $\frac{\opt}{32}$. Next, we show that with only access to $\hat{F}_{ij}$, we can essentially carry out the same approach. Consider a different convex program (Figure~\ref{fig:CP unit demand approximate dist}).
\begin{figure}[ht]
\begin{minipage}{\textwidth} 
\begin{align*}
&\max \sum_{i,j} R_{\hat{D}_{ij}}(q_{ij})\\
\textbf{s.t. }& \sum_i q_{ij}\leq \frac{1}{2} + n\cdot\epsilon\qquad \text{ for all $j\in[m]$}\\
& \sum_j q_{ij}\leq \frac{1}{2}+m\cdot\epsilon \qquad \text{ for all $i\in[n]$}\\
& q_{ij}\geq 0\qquad \text{ for all $i\in[n]$ and $j\in[m]$}
\end{align*}
\end{minipage}
\caption{A Convex Program for Unit-demand Bidders with Approximate Distributions.}
\label{fig:CP unit demand approximate dist}
\end{figure}

 Not that if the support size for all $\hat{D}_{ij}$ is upper bounded by some finite number $s$, the convex program above can be rewritten as a linear program with size $\poly(n,m,s)$.
In the following Lemma, we prove that the optimal values of the two convex programs above are close. %\todo{If time permits, prove that if the support size for all $\hat{D}_{ij}$ is at most some finite number $s$, the convex program above can be rewritten as a linear program with size $\poly(n,m,s)$.}
\begin{lemma}\label{lem:UD compare the two CP}
	Let $\{{q}^*_{ij}\}_{i\in[n],j\in[m]}$ and $\{\hat{q}_{ij}\}_{i\in[n],j\in[m]}$ be the optimal solution of the convex program in Figure~\ref{fig:CP unit demand} and~\ref{fig:CP unit demand approximate dist} respectively.	$$\sum_{i,j}R_{\hat{D}_{ij}}(\hat{q}_{ij})\geq \sum_{i,j}{R}_{D_{ij}}({q}^*_{ij})-\epsilon\cdot mn H.$$
	\end{lemma}

\begin{proof}
	We first fix some notations. For any bidder $i$ and item $j$, let $\ubar{q}^*_{ij}, \bar{q}^*_{ij}$ and $x_{ij}$ $\in[0,1]$ be the numbers satisfy that $x_{ij}\cdot \ubar{q}^*_{ij}\cdot {F_{ij}}^{-1}(1-\ubar{q}^*_{ij}) +(1-x_{ij})\cdot \bar{q}^*_{ij}\cdot {F_{ij}}^{-1}(1-\bar{q}^*_{ij})=R_{D_{ij}}(q^*_{ij})$ and $x_{ij}\cdot \ubar{q}^*_{ij} +(1-x_{ij})\cdot \bar{q}^*_{ij}=q^*_{ij}$. Let $\ubar{p}_{ij} = {F_{ij}}^{-1}(1-\ubar{q}^*_{ij})$, $\bar{p}_{ij} = {F_{ij}}^{-1}(1-\bar{q}^*_{ij})$, and  $q'_{ij} = x_{ij}\cdot \left(1-\hat{F}_{ij}(\ubar{p}_{ij})\right)+(1-x_{ij})\cdot \left(1-\hat{F}_{ij}(\bar{p}_{ij})\right)$. By the definition of $R_{\hat{D}_{ij}}(\cdot)$, \begin{equation}\label{eq:compare revenue curve}
		R_{\hat{D}_{ij}}(q_{ij}')\geq x_{ij}\cdot \left(1-\hat{F}_{ij}(\ubar{p}_{ij})\right)\cdot \ubar{p}_{ij}+(1-x_{ij})\cdot \left(1-\hat{F}_{ij}(\bar{p}_{ij})\right)\cdot \bar{p}_{ij}
	\end{equation} Since $||\hat{D}_{ij}-D_{ij}||_K\leq \epsilon$, $\hat{F}_{ij}(\ubar{p}_{ij})\in [1-\ubar{q}^*_{ij}-\epsilon, 1-\ubar{q}^*_{ij}+\epsilon]$ and $\hat{F}_{ij}(\bar{p}_{ij}) \in [1-\bar{q}^*_{ij}-\epsilon, 1-\bar{q}^*_{ij}+\epsilon]$. Hence, the RHS of inequality~(\ref{eq:compare revenue curve}) is greater than $R_{D_{ij}}(q_{ij}^*)-\epsilon\cdot H$. Therefore, $R_{\hat{D}_{ij}}(q_{ij}')\geq R_{D_{ij}}(q_{ij}^*)-\epsilon\cdot H$.
	
	Next, we argue that $\{q'_{ij}\}_{i\in[n],j\in[m]}$ is a feasible solution for the convex program in Figure~\ref{fig:CP unit demand approximate dist}. Since $1-\hat{F}_{ij}(\ubar{p}_{ij})\leq \ubar{q}^*_{ij}+\epsilon$ and $1-\hat{F}_{ij}(\bar{p}_{ij})\leq \bar{q}^*_{ij}+\epsilon$, $q'_{ij}\leq q^*_{ij}+\epsilon$. Thus, $\sum_i q'_{ij} \leq \sum_i q^*_{ij} + n\cdot \epsilon \leq \frac{1}{2}+n\cdot \epsilon$ for all $j\in[m]$. Similarly, we can prove $\sum_j q'_{ij}\leq \frac{1}{2}+m\cdot \epsilon$ for all $i\in[n]$. As $\{\hat{q}_{ij}\}_{i\in[n],j\in[m]}$ is the optimal solution for the second convex program, $\sum_{i,j}R_{\hat{D}_{ij}}(\hat{q}_{ij})\geq \sum_{i,j}R_{\hat{D}_{ij}}({q}'_{ij})\geq \sum_{i,j}{R}_{D_{ij}}({q}^*_{ij})-\epsilon\cdot mn H$.
 \end{proof}

Finally, we show how to use the optimal solution of the convex program in Figure~\ref{fig:CP unit demand approximate dist} to construct an SPM that approximates the optimal revenue well. We first provide a general transformation that turns any approximately feasible solution of convex program in Figure~\ref{fig:CP unit demand} to an SPM mechanism.

\begin{lemma}\label{lem:prices to SPM}
	 For any distribution $\DD=\times_{i\in[n], j\in[m]}\DD_{ij}$, given a collection of independent random variables $\{p_{ij}\}_{i\in[n],j\in[m]}$ such that $$\sum_{i\in[n]} \Pr_{p_{ij}, t_{ij}\sim \DD_{ij}}\left[t_{ij}\geq p_{ij}\right]\leq 1-\eta_1 \text{,\quad for all $j\in[m]$}$$ and $$\sum_{j\in[m]} \Pr_{p_{ij}, t_{ij}\sim \DD_{ij}}\left[t_{ij}\geq p_{ij}\right]\leq 1-\eta_2 \text{,\quad for all $i\in[n]$},$$ we can construct in polynomial time a randomized SPM such that the revenue under $\DD$ is at least $$\eta_1 \eta_2\cdot \sum_{i, j} \E_{p_{ij}}\left[p_{ij}\cdot \Pr_{t_{ij}\sim \DD_{ij}}\left[t_{ij}\geq p_{ij}\right]\right].$$
	 \end{lemma}

\begin{proof}
	Consider a randomized SPM that sells item $j$ to bidder $i$ at price $p_{ij}$. Notice that bidder $i$ purchases exactly item $j$ if all of the following three conditions hold: (i) for all bidders $\ell\neq i$, $t_{\ell j}$ is smaller than the corresponding price $p_{\ell j}$,  (ii) for all items $k\neq j$, $t_{ik}$ is smaller than the corresponding price $p_{ik}$, and (iii) $t_{ij}$ is greater than the corresponding price $p_{ij}$. These three conditions are independent from each other. The first condition holds with probability at least $1-\sum_{\ell\neq i} \Pr_{p_{\ell j}, t_{\ell j}\sim \DD_{\ell j}}\left[t_{\ell j}\geq p_{\ell j}\right]\geq \eta_1$. The second condition holds with probability at least $1-\sum_{k \neq j} \Pr_{p_{ik}, t_{ik}\sim \DD_{ik}}\left[t_{ik}\geq p_{ik}\right]\geq \eta_2$. When the first two conditions hold, bidder $i$ purchases item $j$ whenever she can afford it. Her expected payment is $\E_{p_{ij}}\left[p_{ij}\cdot \Pr_{t_{ij}\sim \DD_{ij}}\left[t_{ij}\geq p_{ij}\right]\right]$. Hence, the expected revenue for selling item $j$ to bidder $i$ is at least $\eta_1\eta_2\cdot \E_{p_{ij}}\left[p_{ij}\cdot \Pr_{t_{ij}\sim \DD_{ij}}\left[t_{ij}\geq p_{ij}\right]\right]$ and the total expected revenue is at least $\eta_1 \eta_2\cdot \sum_{i, j} \E_{p_{ij}}\left[p_{ij}\cdot \Pr_{t_{ij}\sim \DD_{ij}}\left[t_{ij}\geq p_{ij}\right]\right]$.
	
	\end{proof}

\begin{lemma}\label{lem:convert approx CP to SPM}
	Given any feasible solution $\{{q}_{ij}\}_{i\in[n],j\in[m]}$ of the convex program in Figure~\ref{fig:CP unit demand approximate dist}, we can construct a (randomized) SPM in polynomial time such that its revenue under $D$ is at least $\left(\frac{1}{4}-(n+m)\cdot \epsilon\right)\cdot \left( \sum_{i,j}R_{\hat{D}_{ij}}(q_{ij})-\epsilon\cdot nmH\right)$. 
\end{lemma}

\begin{proof}
	We first fix some notations. For any bidder $i$ and item $j$, let $\ubar{q}_{ij}, \bar{q}_{ij}$ and $x_{ij}$ $\in[0,1]$ be the numbers satisfying $x_{ij}\cdot \ubar{q}_{ij}\cdot \hat{F}_{ij}^{-1}(1-\ubar{q}_{ij}) +(1-x_{ij})\cdot \bar{q}_{ij}\cdot \hat{F}_{ij}^{-1}(1-\bar{q}_{ij})=R_{\hat{D}_{ij}}(q_{ij})$ and $x_{ij}\cdot \ubar{q}_{ij} +(1-x_{ij})\cdot \bar{q}_{ij}=q_{ij}$. We use $p_{ij}$ to denote a random variable that is $\ubar{p}_{ij} = \hat{F}_{ij}^{-1}(1-\ubar{q}_{ij})$ with probability $x_{ij}$ and $\bar{p}_{ij} = \hat{F}_{ij}^{-1}(1-\bar{q}_{ij})$ with probability $1-x_{ij}$.  
	
	Next, we construct a randomized SPM based on $\{p_{ij}\}_{i\in[n],j\in[m]}$ according to Lemma~\ref{lem:prices to SPM}. Note that $$\sum_{i\in[n]} \Pr_{p_{ij}, t_{ij}\sim D_{ij}}\left[t_{ij}\geq p_{ij}\right]\leq \sum_{i\in[n]} \left(\Pr_{p_{ij}, t_{ij}\sim \hat{D}_{ij}}\left[t_{ij}\geq p_{ij}\right]+\epsilon\right)=\sum_{i\in[n]} q_{ij}+n\epsilon\leq \frac{1}{2}+2n\epsilon$$ for all item $j$, and $$\sum_{j\in[m]} \Pr_{p_{ij}, t_{ij}\sim D_{ij}}\left[t_{ij}\geq p_{ij}\right]\leq \sum_{j\in[m]} \left(\Pr_{p_{ij}, t_{ij}\sim \hat{D}_{ij}}\left[t_{ij}\geq p_{ij}\right]+\epsilon\right)=\sum_{i\in[m]} q_{ij}+m\epsilon\leq \frac{1}{2}+2m\epsilon$$ for all bidder $i$. Hence, we can construct in polynomial time a randomized SPM with revenue at least \begin{align*} &\left(\frac{1}{2}-2n\epsilon\right)\left(\frac{1}{2}-2m\epsilon\right)\cdot \sum_{i, j} \E_{p_{ij}}\left[p_{ij}\cdot \Pr_{t_{ij}\sim D_{ij}}\left[t_{ij}\geq p_{ij}\right]\right]\\
 	\geq &  \left(\frac{1}{4}-(n+m)\epsilon\right)\sum_{i, j} \E_{p_{ij}}\left[p_{ij}\cdot \left(\Pr_{t_{ij}\sim \hat{D}_{ij}}\left[t_{ij}\geq p_{ij}\right]-\epsilon\right)\right]\\
 	\geq & \left(\frac{1}{4}-(n+m)\epsilon\right)\sum_{i, j} \left(R_{\hat{D}_{ij}}(q_{ij})-\epsilon\cdot nmH\right)
 \end{align*}
	The first inequality is because $\left|\left|D_{ij}-\hat{D}_{ij}\right|\right|_K\leq \epsilon$, and the second inequality is because $p_{ij}$ is upper bounded by $H$ and $\E_{p_{ij}}\left[p_{ij}\cdot \Pr_{t_{ij}\sim \hat{D}_{ij}}\left[t_{ij}\geq p_{ij}\right]\right]=R_{\hat{D}_{ij}}(q_{ij})$ by the definition of $p_{ij}$.
	
	
	 %For bidder $i$, her price for item $j$ is $\ubar{p}_{ij}$ with probability $x_{ij}$ and $\bar{p}_{ij}$ with probability $1-x_{ij}$. This choice is made independently from all the other prices. The probability that bidder $i$ can afford this price is $q'_{ij}=x_{ij}\cdot\left(1-F_{ij}(\ubar{p}_{ij})\right)+(1-x_{ij})\cdot \left(1-F_{ij}(\bar{p}_{ij})\right)$. Since $||\hat{D}_{ij}-D_{ij}||_K\leq \epsilon$, $q'_{ij}\in [q_{ij}-\epsilon,q_{ij}+\epsilon]$. Notice that when it is buyer $i$'s turn, she purchases exactly item $j$ if all of the following three conditions hold: (i) for all bidders $\ell\neq i$, $t_{\ell j}$ is smaller than the corresponding price,  (ii) for all items $k\neq j$, $t_{ik}$ is smaller than the corresponding price, and (iii) $t_{ij}$ is greater than the corresponding price. All three conditions are independent from each other. The first condition holds with probability at least $1-\sum_{\ell\neq i} q'_{\ell j}\geq 1- \sum_{\ell\neq i} (q_{\ell j}+\epsilon)\geq \frac{1}{2}-2n\cdot \epsilon$. The second condition holds with probability at least $1-\sum_{k\neq j} q'_{ik}\leq 1-\sum_{k\neq j} (q_{ik}+\epsilon)\geq \frac{1}{2}-2m\cdot \epsilon$. When the two conditions hold, bidder $i$ buys item $j$ if she can afford it. The expected payment is 
\notshow{	 \begin{align*}
	&x_{ij}\cdot \ubar{p}_{ij}\cdot \left(1-F_{ij}(\ubar{p}_{ij})\right)+(1-x_{ij})\cdot\bar{p}_{ij}\cdot \left(1-F_{ij}(\bar{p}_{ij})\right)\\
	\geq &x_{ij}\cdot \ubar{p}_{ij}\cdot (\ubar{q}_{ij}-\epsilon)+(1-x_{ij})\cdot\bar{p}_{ij}\cdot (\bar{q}_{ij}-\epsilon)\geq R_{\hat{D}_{ij}}(q_{ij})-\epsilon\cdot H.\end{align*} 
	Therefore, the expected revenue of the randomized SPM is at least $$\left(\frac{1}{2}-2n\cdot \epsilon\right)\cdot \left(\frac{1}{2}-2m\cdot \epsilon\right)\cdot \left( \sum_{i,j}R_{\hat{D}_{ij}}(q_{ij})-\epsilon\cdot nmH\right)\geq \left(\frac{1}{4}-(n+m)\cdot \epsilon\right)\cdot \left( \sum_{i,j}R_{\hat{D}_{ij}}(q_{ij})-\epsilon\cdot nmH\right).$$} 
	\end{proof}


\begin{theorem}\label{thm:UD Kolmogorov}
For unit-demand bidders, given distributions $\hat{D}_{ij}$ where $\left|\left|\hat{D}_{ij}-D_{ij}\right|\right|_K\leq \epsilon$ for all $i\in[n]$ and $j\in[m]$, there is a polynomial time algorithm that constructs a randomized SPM whose revenue under $D$ is at least $\left(\frac{1}{4}-(n+m)\cdot \epsilon\right)\cdot\left(\frac{\opt}{8}-2\epsilon\cdot mnH\right)$.
\end{theorem}

\begin{proof}
	Our algorithm first computes the optimal solution $\{\hat{q}_{ij}\}_{i\in[n],j\in[m]}$ for the convex program in Figure~\ref{fig:CP unit demand approximate dist}, then  constructs a randomized SPM based on $\{\hat{q}_{ij}\}_{i\in[n],j\in[m]}$ using Lemma~\ref{lem:convert approx CP to SPM}. It is not hard to see that our algorithm runs in polynomial time. By chaining the inequalities in Lemma~\ref{lem:compare exact CP with opt},~\ref{lem:UD compare the two CP} and~\ref{lem:convert approx CP to SPM}, we can argue that the revenue of our mechanism is at least $\left(\frac{1}{4}-(n+m)\cdot \epsilon\right)\cdot\left(\frac{\opt}{8}-2\epsilon\cdot mnH\right)$.
\end{proof}

\subsection{Unit-demand Valuations: sample access to bounded distributions} 

When the distributions $D_{ij}$ are all bounded, the following theorem provides the sample complexity.
\begin{theorem}\cite{MorgensternR16}\label{thm:UD bounded}
	When $D_{ij}$ is supported on $[0,H]$ for all bidder $i$ and item $j$, the sample complexity for $(\epsilon,\delta)$-uniformly learning the revenue of SPMs for unit-demand bidders is $O\left(\left(\frac{1}{\epsilon}\right)^2 \left(m^2 n\log n\log \frac{1}{\epsilon} + \log \frac{1}{\delta}\right)\right)$. That is, with probability $1-\delta$, the empirical revenue based on the samples for any SPM is within $\epsilon\cdot H$ of its true expected revenue. %samples suffices to learn with probability $1-\delta$ an SPM whose revenue is at most $\epsilon H$ less than the best SPM whose revenue is at least $\frac{\opt}{24}$. 
	{Moreover, with the same number of samples, there is a polynomial time algorithm that learns an SPM whose revenue is at least $\frac{\opt}{144}-\epsilon H$ with probability $1-\delta$. }\end{theorem}

\subsection{Unit-demand Valuations: sample access to regular distributions}\label{sec:unit-demand regular}
In this section, we show there exists a polynomial time algorithm that learns an SPM whose revenue is at least a constant fraction of the optimal revenue with polynomial in $n$ and $m$ samples. Note that unlike in the previous two models, the error of our learning algorithm is only multiplicative when the distributions are regular. First, we present a Lemma regarding the revenue curve function for regular distributions.


%                                                                                                                                                                                                                                                                                                                                                                                                                                                                                                                                                                                                                                                                                                                                                                                                                                                                                                                                                                                                                                                                                                                                                                                                                                                                                                                                                                                                                                                                                                                                                                                                                                                                                                                                                                                                                                                                               Let us first sketch the proof. First, we use the property that all distributions are regular to argue that restricting the posted prices to be less than some upper bound $H= O\left(\poly(n,m)\cdot \opt\right)$ will affect the best revenue achievable by an SPM by at most a small multiplicative factor. In the second step, we truncate the original distributions $D_{ij}$ at the upper bound $H$ and argue that the revenue for any SPM under $D$ and the truncated distribution are exactly the same. Next, we apply Theorem~\ref{thm:UD bounded} to the truncated distribution to obtain an SPM. The only issue here is that Theorem~\ref{thm:UD bounded} has an additive error that depends on $H$. Luckily, since $H$ is at most a polynomial times larger than $\opt$ we can simply choose $\epsilon$ to be inverse of some polynomial to convert the additive error of $\epsilon H$ into a multiplicative error.

\begin{lemma}\cite{CaiD11b}\label{lem:regular revenue curve concave}
	For any regular distribution $F$, let $R_F(\cdot)$ be the corresponding revenue curve. For any $0<q'\leq q\leq  p < 1$, $$(1-p)\cdot R_F(q')\leq  R_F(q).$$
	\end{lemma}
 Throughout this section, we use $Z$ to denote $\max\{m,n\}$ and $C$ to be a constant that will be specified later. Using Lemma~\ref{lem:regular revenue curve concave}, we show in the next Lemma that restricting $q_{ij}$ to be at least $\frac{1}{ CZ}$ does not affect the objective value of the convex program in Figure~\ref{fig:CP unit demand} by too much.

\begin{lemma}\label{lem:lowering high prices}
	Suppose $\{q_{ij}^*\}_{i\in[n], j\in[m]}$ is the optimal solution of the convex program in Figure~\ref{fig:CP unit demand}. Let $q'_{ij}=\max\{\frac{1}{CZ}, q^*_{ij}\}$, then $\sum_{i,j} R_{D_{ij}}(q'_{ij})\geq \left(1-\frac{1}{ CZ}\right)\cdot \sum_{i,j} R_{D_{ij}}(q^*_{ij})\geq \left(1-\frac{1}{ CZ}\right)\cdot \frac{\opt}{8}$.
\end{lemma}
\begin{proof}
	According to Lemma~\ref{lem:compare exact CP with opt}, $\sum_{i,j} R_{D_{ij}}(q^*_{ij})\geq \frac{\opt}{8}$. So to prove the statement, it suffices to argue that for any $i$ and $j$, $R_{D_{ij}}(q'_{ij})\geq \left(1-\frac{1}{CZ}\right)\cdot R_{D_{ij}}(q^*_{ij})$. If $q^*_{ij} = q'_{ij}$, this inequality clearly holds. If $q^*_{ij} \neq q'_{ij}$, $q^*_{ij}\leq q'_{ij}=\frac{1}{CZ}$. Since $F_{ij}$ is regular, we can apply Lemma~\ref{lem:regular revenue curve concave} to $q'_{ij}$ and $q^*_{ij}$ and obtain inequality $R_{D_{ij}}(q'_{ij})\geq \left(1-\frac{1}{C Z}\right)\cdot R_{D_{ij}}(q^*_{ij})$.
\end{proof}

Using Lemma~\ref{lem:lowering high prices}, we argue how to compute in polynomial time an approximately optimal SPM. Suppose $D'_{ij}$ is the distribution that we obtain after truncating $D_{ij}$ at a threshold $H_{ij}$\footnote{Let $t_{ij}\sim D_{ij}$, then $\min\{t_{ij},H_{ij}\}$ is the corresponding truncated random variable drawn from $D'_{ij}$.}, and we have direct access to a discrete distribution $\hat{D}'_{ij}$ such that $\left|\left|\hat{D}'_{ij}-D'_{ij}\right|\right|_K\leq \epsilon$ for all $i$ and $j$. We show in the following Lemma that the optimal solution of a convex program similar to the one in Figure~\ref{fig:CP unit demand approximate dist} but for $\{\hat{D}'_{ij}\}_{i\in[n],j\in[m]}$ can guide us to design an approximately optimal SPM under $D$ in polynomial time. As we have sample access to $D$, we will argue later that a polynomial number of samples suffices to generate $\{\hat{D}'_{ij}\}_{i\in[n],j\in[m]}$.%Then for any SPM that does not use price $p_{ij}> H_{ij}$, its revenue under $D'=\times_{i,j} D'_{ij}$ and $D$ is the same. Therefore, if we can learn an SPM with good revenue for $D'$, the same mechanism also achieves good revenue under $D$.
\notshow{\begin{lemma}\label{lem:UD low prices suffice}
	There exists a collection of deterministic prices $\{p'_{ij}\}_{i\in[n],j\in[m]}$ satisfying $F_{ij}(p'_{ij})\leq 1-\frac{1}{C\cdot Z}$ for all $i\in[n]$ and $j\in[m]$, such that the corresponding SPM achieves revenue of $\left(\frac{1}{2}-\frac{1}{C}\right)^2\cdot \left(1-\frac{1}{C\cdot Z}\right)\cdot \frac{\opt}{8}$.
\end{lemma} 
\begin{proof}
Let $\{q_{ij}^*\}_{i\in[n], j\in[m]}$ be the optimal solution of the convex program in Figure~\ref{fig:CP unit demand} and $q'_{ij}=\max\{\frac{1}{C\cdot Z}, q^*_{ij}\}$. For every $i$ and $j$, let $p'_{ij}$ to be $F^{-1}_{ij}(1-q'_{ij})$. Clearly, $F_{ij}(p'_{ij})\leq 1-\frac{1}{C\cdot Z}$. 

As $D_{ij}$ is regular, $R_{D_{ij}}(q'_{ij}) = p'_{ij} q'_{ij}$. Therefore, $\sum_{i,j} p'_{ij}q'_{ij}=\sum_{i,j} R_{D_{ij}}(q'_{ij})\geq \left(1-\frac{1}{CZ} \right)\cdot \frac{\opt}{8}$ according to Lemma~\ref{lem:UD low prices suffice}. Also, $\sum_i q'_{ij}\leq \frac{1}{2}+\frac{n}{CZ}\leq \frac{1}{2}+\frac{1}{C}$ for all item $j$, and $\sum_j q'_{ij}\leq \frac{1}{2}+\frac{m}{CZ}\leq \frac{1}{2}+\frac{1}{C}$ for all bidder $i$. According to Lemma~\ref{lem:prices to SPM}, the SPM with prices $\{p'_{ij}\}_{i\in[n],j\in[m]}$ has revenue at least $\left(\frac{1}{2}-\frac{1}{C}\right)^2\cdot \sum_{i,j} p'_{ij}q'_{ij}\geq \left(\frac{1}{2}-\frac{1}{C}\right)^2\cdot  \left(1-\frac{1}{CZ} \right)\cdot \frac{\opt}{8}$.

%Next, we argue that if we use $p_{ij}$ as the prices in an SPM, the expected revenue is at least $\left(\frac{1}{2}-\frac{1}{C}\right)^2\cdot \left(1-\frac{1}{C\cdot Z}\right)\cdot \frac{\opt}{8}$. For any $i$ and $j$, the probability that (i) all bidders $\ell\neq i$ cannot afford item $j$ and (ii) bidder $i$ cannot afford the price for any item $k\neq j$ is at least $\left(1-\sum_{\ell\neq i}q'_{\ell j}\right)\cdot \left(1-\sum_{k\neq j}q'_{ik}\right)$. Since both $\sum_\ell q^*_{\ell j}$ and $\sum_{k}q^*_{ik}$ are less than $\frac{1}{2}$, both $1-\sum_{\ell\neq i}q'_{\ell j}$ and $1-\sum_{k\neq j}q'_{ik}$ are at least $\frac{1}{2}-\frac{1}{C}$. Clearly, when (i) and (ii) are simultaneously satisfied, bidder $i$ purchases item $j$ whenever $t_{ij}$ is greater than $p_{ij}$. Therefore, the total revenue of this SPM is at least $\left(\frac{1}{2}-\frac{1}{C}\right)^2\cdot \sum_{i,j} p_{ij}\cdot q'_{ij}$. As $F_{ij}$ is regular, $p_{ij}\cdot q'_{ij}=R_{D_{ij}}(q'_{ij})$. Finally, we know that  $\sum_{i,j} R_{D_{ij}}(q'_{ij})\geq \left(1-\frac{1}{C\cdot Z}\right)\cdot \sum_{i,j} R_{D_{ij}}(q^*_{ij})\geq \left(1-\frac{1}{C\cdot Z}\right)\cdot \frac{\opt}{8}$ from Lemma~\ref{lem:lowering high prices} and~\ref{lem:compare exact CP with opt}. Combining this inequality with the lower bound of the revenue for the SPM, we can show that the revenue of our constructed SPM is at least $\left(\frac{1}{2}-\frac{1}{C}\right)^2\cdot \left(1-\frac{1}{C\cdot Z}\right)\cdot \frac{\opt}{8}$.
\end{proof}



\begin{lemma}\label{lem:UD additive bound}
	Let $\{H_{ij}\}_{i\in[n],j\in[m]}$ be a collection of positive numbers satisfying $F_{ij}(H_{ij})\in [1-\frac{1}{C\cdot Z}, 1-\frac{1}{3C\cdot Z}]$ for all $i\in[n]$ and $j\in[m]$. With $O\left(\left(\frac{1}{\epsilon}\right)^2 \left(m^2 n\log n\log \frac{1}{\epsilon} + \log \frac{1}{\delta}\right)\right)$ samples, we can learn an SPM with probability $1-\delta$ that achieves revenue at least $\left(\frac{1}{2}-\frac{1}{C}\right)^2\cdot \left(1-\frac{1}{C\cdot Z}\right)\cdot \frac{\opt}{8}-\epsilon\cdot\max_{i,j} H_{ij}$. \yangnote{With the same number of samples, we can learn in polynomial time an SPM with probability $1-\delta$ that achieves revenue at least $\left(\frac{1}{2}-\frac{1}{C}\right)^2\cdot \left(1-\frac{1}{C\cdot Z}\right)\cdot \frac{\opt}{48}-\epsilon\cdot\max_{i,j} H_{ij}$.}
\end{lemma}
\begin{proof}
	First, we truncate each $D_{ij}$ at $H_{ij}$ to create a bounded distribution $D'_{ij}$. This is straightforward, as we only need to cap the value for any sample from $D_{ij}$ at $H_{ij}$. According to Lemma~\ref{lem:UD low prices suffice}, there exists an SPM that achieves revenue at least $\left(\frac{1}{2}-\frac{1}{C}\right)^2\cdot \left(1-\frac{1}{C\cdot Z}\right)\cdot \frac{\opt}{8}$ under $D'=\times_{i,j} D'_{ij}$, as the price $p_{ij}$ used in the SPM is less than $H_{ij}$ for all $i,j$. Combining this observation with Theorem~\ref{thm:UD bounded}, we know that if we take $O\left(\left(\frac{1}{\epsilon}\right)^2 \left(m^2 n\log n\log \frac{1}{\epsilon} + \log \frac{1}{\delta}\right)\right)$ samples we can learn with probability $1-\delta$ an SPM whose revenue is at least $\left(\frac{1}{2}-\frac{1}{C}\right)^2\cdot \left(1-\frac{1}{C\cdot Z}\right)\cdot \frac{\opt}{8}-\epsilon\cdot\max_{i,j} H_{ij}$. Similarly, we can prove the computational friendly version of this claim. \end{proof}}
	
\begin{lemma}\label{lem:UD additive bound}
	Let $\{H_{ij}\}_{i\in[n],j\in[m]}$ be a collection of positive numbers satisfying $F_{ij}(H_{ij})\in [1-\frac{1}{C\cdot Z}, 1-\frac{1}{3C\cdot Z}]$ for all $i\in[n]$ and $j\in[m]$. Let $D'_{ij}$ be the distribution of the random variable $\min\{t_{ij}, H_{ij}\}$ where $t_{ij}\sim D_{ij}$, and $\hat{D}_{ij}'$ be a discrete distribution such that $\left|\left|\hat{D}'_{ij}-D'_{ij}\right|\right|_K\leq \epsilon$ for all $i\in[n]$ and $j\in[m]$. Suppose $s$ is an upper bound of the support size for any distribution $\hat{D}'_{ij}$, then given direct access to $\hat{D}_{ij}'$, we can compute in time polynomial in $n$, $m$ and $s$ a randomized SPM that achieves revenue at least $\left(\frac{1}{2}-\frac{1}{C}-2n\epsilon\right)\cdot\left(\frac{1}{2}-\frac{1}{C}-2m\epsilon\right)\cdot \left(\left(1-\frac{1}{ CZ}\right)\cdot \frac{\opt}{8}-2\epsilon\cdot nmH\right)$ under $D$, where $H= \max_{i,j} H_{ij}$. 
\end{lemma}
\begin{proof}
Consider the following convex program:

\begin{minipage}{\textwidth} 
\begin{align*}
&\max \sum_{i,j} R_{\hat{D}_{ij}'}(q_{ij})\\
\textbf{s.t. }& \sum_i q_{ij}\leq \frac{1}{2} + \frac{1}{C} +n\cdot\epsilon\qquad \text{ for all $j\in[m]$}\\
& \sum_j q_{ij}\leq \frac{1}{2}+\frac{1}{C} +m\cdot\epsilon \qquad \text{ for all $i\in[n]$}\\
& q_{ij}\geq 0\qquad \text{ for all $i\in[n]$ and $j\in[m]$}
\end{align*}
\end{minipage}
\vspace{.1in}

Let $\{q_{ij}^*\}_{i\in[n], j\in[m]}$ be the optimal solution of the convex program in Figure~\ref{fig:CP unit demand} and $q'_{ij}=\max\{\frac{1}{C\cdot Z}, q^*_{ij}\}$. For every $i$ and $j$, let $p'_{ij}=F^{-1}_{ij}(1-q'_{ij})$ and $\tilde{q}_{ij} = \Pr_{t_{ij}\sim \hat{D}'_{ij}}\left[t_{ij}\geq p'_{ij}\right]$. By the definition of $H_{ij}$, $p'_{ij}\leq H_{ij}$, so \begin{align}\label{ineq:LB inequality}
p'_{ij}\tilde{q}_{ij}\geq p'_{ij}\left(\Pr_{t_{ij}\sim {D}'_{ij}}\left[t_{ij}\geq p'_{ij}\right]-\epsilon\right)\geq p'_{ij} q'_{ij}-\epsilon\cdot H_{ij}=R_{D_{ij}}(q'_{ij})-\epsilon\cdot H_{ij}.
 \end{align}
$p'_{ij} q'_{ij}$ equals to $R_{D_{ij}}(q'_{ij})$ because $D_{ij}$ is a regular distribution.

Next, we argue that $\{\tilde{q}_{ij}\}_{i\in[n],j\in[m]}$ is a feasible solution of the convex program above. Observe that $$\sum_{i} \tilde{q}_{ij} \leq \sum_i q'_{ij}+n\epsilon \leq \sum_i \left(q^*_{ij}+\frac{1}{CZ}\right)+n\epsilon\leq \frac{1}{2} + \frac{1}{C} +n\epsilon$$ for all item $j\in[m]$ and $$\sum_{j} \tilde{q}_{ij} \leq \sum_j q'_{ij}+n\epsilon \leq \sum_j \left(q^*_{ij}+\frac{1}{CZ}\right)+m\epsilon\leq \frac{1}{2}+\frac{1}{C} +m\epsilon$$ for all bidder $i\in[n]$. 

 Let $\widehat{\opt}$ be the optimal solution of the convex program above. As $\{\tilde{q}_{ij}\}_{i\in[n],j\in[m]}$ is a feasible solution,
$$\widehat{\opt}\geq \sum_{i,j} R_{\hat{D}_{ij}'}(\tilde{q}_{ij})\geq \sum_{i,j} p'_{ij}\tilde{q}_{ij}\geq \sum_{i,j} R_{D_{ij}}(q'_{ij})-\epsilon\cdot nmH\geq \left(1-\frac{1}{ CZ}\right)\cdot \frac{\opt}{8}-\epsilon\cdot nmH.$$
The second last inequality is due to inequality~(\ref{ineq:LB inequality}) and the last inequality is due to Lemma~\ref{lem:lowering high prices}.

So far, we have argued that the optimal solution of our convex program has value close to the $\opt$. We will show in the second part of the proof that using the optimal solution of our convex program, we can construct an SPM whose revenue under $D$ is close to $\widehat{\opt}$. Let $\hat{q}_{ij}$ be the optimal solution of the convex program above and $\hat{p}_{ij}$ be the corresponding random price, that is, $\Pr_{\hat{p}_{ij}, t_{ij}\sim \hat{D}_{ij}'}\left[t_{ij}\geq \hat{p}_{ij}\right]=\hat{q}_{ij}$ and $R_{\hat{D}_{ij}'}(\hat{q}_{ij}) = \E_{\hat{p}_{ij}}\left[\hat{p}_{ij}\cdot \Pr_{t_{ij}\sim \hat{D}_{ij}'}\left[t_{ij}\geq \hat{p}_{ij}\right]\right]$. As $\hat{p}_{ij}\leq H_{ij}$, $$\Pr_{\hat{p}_{ij}, t_{ij}\sim {D}_{ij}}\left[t_{ij}\geq \hat{p}_{ij}\right]=\Pr_{\hat{p}_{ij}, t_{ij}\sim {D}'_{ij}}\left[t_{ij}\geq \hat{p}_{ij}\right]\in  [\hat{q}_{ij}-\epsilon,\hat{q}_{ij}+\epsilon].$$ Therefore, for all item $j$ $$\sum_{i} \Pr_{\hat{p}_{ij}, t_{ij}\sim {D}_{ij}}\left[t_{ij}\geq \hat{p}_{ij}\right]\leq \sum_{i} \Pr_{\hat{p}_{ij}, t_{ij}\sim \hat{D}_{ij}'}\left[t_{ij}\geq \hat{p}_{ij}\right] + n\epsilon = \sum_i \hat{q}_{ij} +n\epsilon \leq \frac{1}{2} + \frac{1}{C} +2n\epsilon$$ and for all bidder $i$	$$\sum_{j} \Pr_{\hat{p}_{ij}, t_{ij}\sim {D}_{ij}}\left[t_{ij}\geq \hat{p}_{ij}\right]\leq \sum_{j} \Pr_{\hat{p}_{ij}, t_{ij}\sim \hat{D}_{ij}'}\left[t_{ij}\geq \hat{p}_{ij}\right] + m\epsilon = \sum_j \hat{q}_{ij} +m\epsilon \leq \frac{1}{2} + \frac{1}{C} +2m\epsilon.$$ According to Lemma~\ref{lem:prices to SPM}, we can construct a randomized SPM with $\{\hat{p}_{ij}\}_{i\in[n],j\in[m]}$ whose revenue is at least $\left(\frac{1}{2}-\frac{1}{C}-2n\epsilon\right)\cdot\left(\frac{1}{2}-\frac{1}{C}-2m\epsilon\right)\cdot \sum_{i,j} \E_{\hat{p}_{ij}}\left[\hat{p}_{ij}\cdot \Pr_{t_{ij}\sim {D}_{ij}}\left[t_{ij}\geq \hat{p}_{ij}\right]\right]$ under $D$. 
Clearly, $$\E_{\hat{p}_{ij}}\left[\hat{p}_{ij}\cdot \Pr_{t_{ij}\sim {D}_{ij}}\left[t_{ij}\geq \hat{p}_{ij}\right]\right]\geq \E_{\hat{p}_{ij}}\left[\hat{p}_{ij}\cdot \left(\Pr_{t_{ij}\sim \hat{D}'_{ij}}\left[t_{ij}\geq \hat{p}_{ij}\right]-\epsilon\right)\right] \geq R_{\hat{D}'_{ij}}(\hat{q}_{ij})-\epsilon\cdot H_{ij}.$$
 Therefore, the revenue of the constructed randomized SPM under $D$ is at least \begin{align*} &\left(\frac{1}{2}-\frac{1}{C}-2n\epsilon\right)\cdot\left(\frac{1}{2}-\frac{1}{C}-2m\epsilon\right)\cdot \left(\widehat{\opt}-\epsilon\cdot nmH\right)\\
 	\geq & \left(\frac{1}{2}-\frac{1}{C}-2n\epsilon\right)\cdot\left(\frac{1}{2}-\frac{1}{C}-2m\epsilon\right)\cdot \left(\left(1-\frac{1}{ CZ}\right)\cdot \frac{\opt}{8}-2\epsilon\cdot nmH\right).
 \end{align*}
It is not hard to see that both $\{\hat{q}_{ij}\}_{i\in[n],j\in[m]}$ and $\{\hat{p}_{ij}\}_{i\in[n],j\in[m]}$ can be computed in time polynomial in $n$, $m$ and $s$.
	\end{proof}

	When $\epsilon$ is small enough, the additive error in Lemma~\ref{lem:UD additive bound} can be converted into a multiplicative error. Next, we argue that with a polynomial number of samples, we can learn $\{H_{ij}\}_{i\in[n],j\in[m]}$ and $\{\hat{D}'_{ij}\}_{i\in[n],j\in[m]}$ with enough accuracy.

\begin{theorem}\label{thm:UD regular}
If for all bidder $i$ and item $j$, $D_{ij}$ is a regular distribution, we can learn in polynomial time with probability $1-\delta$ a randomized SPM whose revenue is at least $\frac{\opt}{33}$ with $O\left(Z^2m^2 n^2\cdot \log \frac{nm}{\delta}\right)$ ($Z=\max\{m,n\}$) samples. %\yangnote{With the same number of samples, we can learn in polynomial time with probability $1-\delta$ an SPM whose revenue is at least $\frac{\opt}{198}$.}
\end{theorem}
\begin{proof}
	First, if we take $O\left(C^2\cdot Z^2\cdot \log \frac{nm}{\delta}\right)$ samples from each $D_{ij}$, we can find an $H_{ij}$ such that $F_{ij}(H_{ij})$ lies in$[1-\frac{1}{CZ}, 1-\frac{1}{3CZ}]$ with probability $1-\frac{\delta}{2nm}$. By the union bound, the probability that all $H_{ij}$ satisfy the requirement is at least $1-\frac{\delta}{2}$. From now on, we assume $F_{ij}(H_{ij})\in[1-\frac{1}{C\cdot Z}, 1-\frac{1}{3C\cdot Z}]$ for all $i$ and $j$. 
	Observe that $\opt \geq \max_{i,j} H_{ij}\cdot \frac{1}{3C\cdot Z}$, as the expected revenue for selling item $j$ to bidder $i$ at price $H_{ij}$ is at least $ \frac{H_{ij}}{3C\cdot Z}$.
	 Therefore, there exists sufficiently large constant $d$ and $C$, if $\epsilon=\frac{1}{d\cdot Z nm}$ the randomized SPM learned in Lemma~\ref{lem:UD additive bound} has revenue at least $\frac{\opt}{33}$.
	  According to the Dvoretzky-Kiefer-Wolfowitz (DKW) inequality~\cite{DvoretzkyKW56}, if we take $O\left(d^2 Z^2n^2m^2\cdot \log \frac{nm}{\delta} \right)$ samples from $D'_{ij}$ (we can take samples from $D_{ij}$ then cap the samples at $H_{ij}$) and let $\hat{D}'_{ij}$ be the uniform distribution over the samples, $\left|\left|D'_{ij}-\hat{D}'_{ij}\right|\right|_K\leq \frac{1}{d\cdot Z nm}$ with probability $1-\frac{\delta}{2nm}$. 
	  By the union bound,  $\left|\left|D'_{ij}-\hat{D}'_{ij}\right|\right|_K\leq \frac{1}{d\cdot Z nm}$ for all $i\in[n]$ and $j\in[m]$ with probability at least $1-\delta/2$.
	   Finally, by another union bound,  the $H_{ij}$ and $\hat{D}'_{ij}$ we learned from $O\left( Z^2n^2m^2\cdot \log \frac{nm}{\delta} \right)$ samples satisfy $F_{ij}(H_{ij})\in[1-\frac{1}{C\cdot Z}, 1-\frac{1}{3C\cdot Z}]$ and $\left|\left|D'_{ij}-\hat{D}'_{ij}\right|\right|_K\leq \frac{1}{d\cdot Z nm}$ for all $i$ and $j$ with probability at least $1-\delta$.
	   In other words, we can learn a randomized SPM whose revenue is at least $\frac{\opt}{33}$ with probability at least $1-\delta$ using $O\left( Z^2n^2m^2\cdot \log \frac{nm}{\delta} \right)$ samples.
	    Furthermore, the support size of any $\hat{D}'_{ij}$ is at most $O\left( Z^2n^2m^2\cdot \log \frac{nm}{\delta} \right)$ samples, so our learning algorithm runs in time polynomial in $n$ and $m$.
	
	%To turn the additive error in Lemma~\ref{lem:UD additive bound} into a multiplicative error, $\epsilon$ needs to be as small as  Hence, we can learn with probability $1-\frac{\delta}{2}$ an SPM whose revenue is at least $\left(\frac{1}{2}-\frac{1}{C}\right)^2\cdot \left(1-\frac{1}{C\cdot Z}\right)\cdot \frac{\opt}{8}-3C Z\cdot\opt\cdot  \epsilon$, if we are given $O\left(\left(\frac{1}{\epsilon}\right)^2 \left(m^2 n\log n\log \frac{1}{\epsilon} + \log \frac{1}{\delta}\right)\right)$ samples. We let $\epsilon$ to be $\frac{1}{C^2\cdot Z}$. If $C$ is sufficiently big, the SPM learnt has revenue at least $\frac{\opt}{33}$  with probability $1-\frac{\delta}{2}$ and the number of samples needed is $O\left(Z^2 \cdot\left(m^2 n\log n\log Z + \log \frac{1}{\delta}\right)\right)$. 
	
	%To sum up, with $$O\left(Z^2 \cdot\left(m^2 n\log n\log Z + \log \frac{1}{\delta}\right)\right)+O\left(Z^2\cdot \left(\log (nm)+\log \frac{1}{\delta}\right)\right)= O\left(Z^2 \cdot\left(m^2 n\log n\log Z + \log \frac{1}{\delta}\right)\right)$$ samples we can learn an SPM whose revenue is at least $\frac{\opt}{33}$ with probability $1-\delta$. Similarly, we can prove the computational friendly version of this claim.
\end{proof}

\section{Missing Details from Section~\ref{sec:additive}}
\subsection{Additive Valuations: sample access to bounded distributions}\label{sec:additive bounded}
As shown by Goldner and Karlin~\cite{GoldnerK16}, one sample suffices to design a mechanism that approximates $\brev$. The idea is to use the VCG with entry fee mechanism but replace the entry fee $e_i(b_{-i},D_i)$ for bidder $i$ with $e_i(b_{-i},s_i)=\sum_{j\in[m]}(s_{ij}-\max_{k\neq i} b_{kj})^+$, where $s_i$ is a sample drawn from $D_i$. It is easy to argue that for any $b_{-i}$, over the randomness of the sample $s_i$ and bidder $i$'s real type $t_i$, the event that $e_i(b_{-i},s_i)\geq e_i(b_{-i},D_i)$ and bidder $i$ accepts the entry fee $e_i(b_{-i},s_i)$ happens with probability at least $\frac{1}{8}$. As $\frac{1}{2}\cdot \sum_{i\in[n]} \E_{t}[e_i(t_{-i},D_i)]=\brev$, the expected revenue (over the randomness of the types and the samples) from their mechanism is at least $\frac{1}{8}\cdot \sum_{i\in[n]} \E_{t}[e_i(t_{-i},D_i)]=\frac{\brev}{4}$. Next, we show how to learn a mechanism that approximates $\srev$.

\begin{lemma}\label{lem:additive srev}
	When $D_{ij}$ is supported on $[0,H]$ for all bidder $i$ and item $j$, the sample complexity for $(\epsilon,\delta)$-uniformly learning the revenue of SPMs for additive bidders is $O\left(\left(\frac{1}{\epsilon}\right)^2 \left(m^2 n\log n\log \frac{1}{\epsilon} + \log \frac{1}{\delta}\right)\right)$. Moreover, we can learn in polynomial time an SPM whose revenue is at least $\frac{\srev}{4}- \frac{3\epsilon}{2}\cdot H$ with probability $1-\delta$ given the same number of samples. \end{lemma}
\begin{proof}
%Let $\opt_j$ be the optimal revenue for selling item $j$. By the Prophet inequality~\cite{Samuel-cahn84}, there exists an SPM for selling item $j$ that achieves at least $\opt_j/2$. 	Theorem~\ref{thm:UD bounded} implies that $O\left(\left(\frac{1}{\epsilon}\right)^2 \left(n\log n\log \frac{1}{\epsilon} + \log m\right)\right)$ samples suffices to $(\epsilon,\frac{1}{m^2})$-uniformly learn the revenue of single-item SPMs. In other words, the revenue of any SPM over the empirical distribution induced by the samples is within $\epsilon\cdot H$ of the true expected revenue with probability $\left(1-\frac{1}{m^2}\right)$. We use $ER_{opt}$ to denote the optimal empirical revenue obtained by any SPM. If we apply the Prophet inequality to the empirical distributions, we can construct an SPM whose empirical revenue $ER$ is at least $ER_{opt}/2$. Notice that $ER_{opt}$ is at most $\epsilon\cdot H$ less than the optimal revenue obtained by any SPM, which is at least $\opt_j/2$. Combining the two inequalities above, we have $ER\geq \opt_j/4-\epsilon/2\cdot H$ with probability $\left(1-\frac{1}{m^2}\right)$. Also, the true expected revenue of our SPM is at least $ER-\epsilon\cdot H$, so our SPM achieves expected revenue at least $\frac{\opt_j}{4}-\frac{3\epsilon}{2}\cdot H$ with probability  $\left(1-\frac{1}{m^2}\right)$. Since we have $m$ items, so the probability that our learned SPM has revenue at least $\frac{\opt_j}{4}-\frac{3\epsilon}{2}\cdot H$ for every item $j$ is at least $\left(1-\frac{1}{m}\right)$. Therefore, if we run our SPM separately on every item, our total expected revenue is at least $\frac{\srev}{4}-\frac{3\epsilon}{2}\cdot mH$ with probability $\left(1-\frac{1}{m}\right)$.

The first half of the Lemma was proved by Morgenstern and Roughgarden~\cite{MorgensternR16}.
 We show how to prove the second half of the claim.
Let $\opt_j$ be the optimal revenue for selling item $j$.
 By the prophet inequality~\cite{Samuel-cahn84}, there exists an SPM for selling item $j$ with a collection of prices $\{p_{ij}\}_{i\in[n]}$ that achieves revenue at least $\opt_j/2$.
  As the bidders are additive, if we run the SPMs for selling each item simultaneously, the expected revenue is exactly the sum of the revenue of the SPM mechanisms for auctioning a single item.
   Note that the simultaneous SPM is indeed a SPM for selling all items. %construct an SPM for selling all items and use price  $p_{ij}$ to sell item $j$ to bidder $i$, the expected revenue is exactly the sum of the revenue of the SPM mechanisms for auctioning a single item.
   Hence, there exists an SPM that achieves revenue at least $\opt/2$.
    Since the sample complexity for $(\epsilon,\delta)$-uniformly learning the revenue of SPMs is $O\left(\left(\frac{m}{\epsilon}\right)^2 \left(n\log n\log \frac{1}{\epsilon} + \log \frac{1}{\delta}\right)\right)$, the empirical revenue induced by the samples is within $\epsilon\cdot H$ of the true expected revenue with probability $1-\delta$ for any SPM. 

We use $ER_{opt}$ to denote the optimal empirical revenue obtained by any SPM.
 If we apply the prophet inequality to the empirical distribution, we can construct an SPM whose empirical revenue $ER$ is at least $ER_{opt}/2$. Notice that $ER_{opt}$ is at most $\epsilon\cdot H$ less than the optimal true expected revenue obtained by any SPM, which is at least $\opt/2$. Combining the two inequalities above, we have $ER\geq \opt/4-\epsilon/2\cdot H$ with probability $1-\delta$. Also, the true expected revenue of our SPM is at least $ER-\epsilon\cdot H$, so our SPM achieves expected revenue at least $\frac{\opt}{4}-\frac{3\epsilon}{2}\cdot H$ with probability  $1-\delta$. \end{proof}

Now we are ready to prove our Theorem for additive bidders when their valuations are bounded.
\begin{theorem}\label{thm:additive bounded}
	When the bidders have additive valuations and $D_{ij}$ is supported on $[0,H]$ for all bidder $i$ and item $j$, we can learn in polynomial time a mechanism whose expected revenue is at least $\frac{\opt}{32}-{\epsilon}\cdot H$ with probability $1-\delta$ given $$O\left(\left(\frac{m}{\epsilon}\right)^2 \cdot\left(n\log n\log \frac{1}{\epsilon}+\log\frac{1}{\delta} \right)\right)$$ samples from $D$. 
\end{theorem}
\begin{proof}
	According to Lemma~\ref{lem:additive srev}, we can learn a mechanism whose revenue is at least $\frac{\srev}{4}-\frac{\epsilon}{24}\cdot H$ with probability $1-\delta$ given $O\left(\left(\frac{m}{\epsilon}\right)^2\cdot \left(n\log n\log \frac{1}{\epsilon}+\log\frac{1}{\delta} \right)\right)$ samples. As we explained in the beginning of this section, with one sample from the distribution we can construct a randomized mechanism whose expected revenue is at least $\frac{\brev}{4}$. Therefore, the better of our two mechanisms has expected revenue at least $\frac{\opt}{32}-{\epsilon}\cdot H$ with probability $1-\delta$.
	\end{proof}
	
\subsection{Additive Valuations: direct access to approximate distributions}\label{sec:additive Kolmogorov}
	
In this section, we discuss how to learn an approximately optimal mechanism for additive bidders when we are given direct access to approximate value distributions. Again, we first show how to learn a mechanism whose revenue approximates $\srev$ then we provide another mechanism whose revenue approximates $\brev$.

\begin{lemma}\label{lem:SREV kolmogorov}
	For additive bidders, given distributions $\hat{D}_{ij}$ where $\left|\left|\hat{D}_{ij}-D_{ij}\right|\right|_K\leq \epsilon$ for all $i\in[n]$ and $j\in[m]$, there is a polynomial time algorithm that constructs a randomized SPM whose revenue under $D$ is at least $\left(\frac{1}{4}- \epsilon\cdot n \right)\cdot\left(\frac{\srev}{8}-2\epsilon\cdot mnH\right)$.
\end{lemma}
\begin{proof}
	Let $\opt_j$ be the optimal revenue for selling item $j$.
	 As the bidders are additive, if we can construct a randomized SPM $M_j$ for every item $j$ such that its expected revenue under $D$ is at least $\left(\frac{1}{4}- \epsilon\cdot n \right)\cdot\left(\frac{\opt_j}{8}-2\epsilon\cdot nH\right)$, running these $m$ randomized SPMs in parallel generates expected revenue at least $$\sum_{j\in[m]} \left(\frac{1}{4}- \epsilon\cdot n \right)\cdot\left(\frac{\opt_j}{8}-2\epsilon\cdot nH\right) = \left(\frac{1}{4}- \epsilon\cdot n \right)\cdot\left(\frac{\srev}{8}-2\epsilon\cdot mnH\right)$$ under $D$.
	  Due to Theorem~\ref{thm:UD Kolmogorov}, we can construct in polynomial time such a randomized SPM $M_j$ for each item $j$ based on $\times_{i\in[n]} \hat{D}_{ij}$.
\end{proof}

Next, we show how to choose the entry fee based on $\hat{D} = \times_{i, j} \hat{D}_{ij}$, so that the VCG with entry fee mechanism has revenue that approximates $\brev$ under the true distribution $D$. More specifically, we use the median of $i$'s utility under $\hat{D}_i = \times_{j\in[m]} \hat{D}_{ij}$ as bidder $i$'s entry fee. We prove the result in two steps. We first show that if we can use an entry fee function such that every bidder $i$ accepts her entry fee with probability between $[1/2-\eta,1/2]$ for any possible bid profiles $b_{-i}$ of the other bidders, the expected revenue is at least $(1/2-\eta)\cdot \brev$. Second, we show how to compute in polynomial time such entry fee functions with $\eta=O(m\epsilon)$ based on $\hat{D}$. 

\begin{lemma}\label{lem:approximate entry fee for BREV}
	Suppose for every bidder $i$, $d_i(\cdot): T_{-i}\mapsto R$ is a randomized entry fee function such that for any bid profile $b_{-i}\in T_{-i}$ of the other bidders $$\Pr_{t_i\sim D_i}\left[\sum_{j\in[m]} \left(t_{ij}-\max_{k\neq i} b_{kj}\right)^+\geq d_i(b_{-i})\right]\in\left[\frac{1}{2}-\eta,\frac{1}{2}\right]$$ with probability at least $1-\delta$. 
	Then if we use $d_i(\cdot)$ as the entry fee function in the VCG with entry fee mechanism, the expected revenue is at least %$\left(1-2\eta\right)\cdot\brev$.
	$\left(1-\delta-2\eta\right)\cdot\brev$.
\end{lemma}
\begin{proof}
	When $\Pr_{t_i\sim D_i}\left[\sum_{j\in[m]} \left(t_{ij}-\max_{k\neq i} b_{kj}\right)^+\geq d_i(b_{-i})\right]\in\left[\frac{1}{2}-\eta,\frac{1}{2}\right]$, $d_i(b_{-i})$ is no less than the original entry fee $e_i(b_{-i}, D_i)$ for any bid profile $b_{-i}$ of the other bidders. The expected revenue under the new entry fee functions is at least $(\left(1-\delta\right)\cdot\left(\frac{1}{2}-\eta\right)\cdot \sum_{i\in[n]} \E_{b_{-i}\sim D_{-i}}\left[e_i(b_{-i},D_i)\right]
 	\geq \left(1-\delta-2\eta\right)\cdot\brev$.
\end{proof}

\begin{lemma}\label{lem:learn approx entry fee from approx dist}
	For any bidder $i$ and any bid profile $b_{-i}$ from the other bidders, let $\FF_{i,b_{-i}}$ and $\hat{\FF}_{i,b_{-i}}$ be the distributions for the random variable $\sum_{j\in[m]} \left(t_{ij}-\max_{k\neq i} b_{kj}\right)^+$ when $t_i$ is drawn from $D_i$ and $\hat{D}_i$ respectively. If $\left|\left|D_{ij}-\hat{D}_{ij}\right|\right|_K\leq \epsilon$ for all bidder $i$ and item $j$, $\left|\left|\FF_{i,b_{-i}}-\hat{\FF}_{i,b_{-i}}\right|\right|_K\leq 2m\epsilon$ for all $i$ and $b_{-i}$. Moreover, when $m\epsilon\leq 1/16$, we can compute a randomized mechanism whose expected revenue is at least $\frac{\brev}{5}$.
\end{lemma}
\begin{proof}
	For any real number $x$, consider event $\EE_{i,b_{-i},x}=\left\{t_i\ \Big{|}\ \sum_{j\in[m]} \left(t_{ij}-\max_{k\neq i} b_{kj}\right)^+\geq x\right\}$. 
	It is easy to see that $\EE_{i,b_{-i},x}$ is single-intersecting for any any $i$, $b_{-i}$ and $x$.
	 According to Lemma~\ref{lem:Kolmogorov stable for sc}, $$\left|\Pr_{t_i\sim D_i}\left[\EE_{i,b_{-i},x}\right]-\Pr_{t_i\sim \hat{D}_i}\left[\EE_{i,b_{-i},x}\right]\right|\leq 2m\epsilon$$ for any $i$, $b_{-i}$ and $x$. 
	 Hence, $\left|\left|\FF_{i,b_{-i}}-\hat{\FF}_{i,b_{-i}}\right|\right|_K\leq 2m\epsilon$. 
	
	Next, we argue how to construct a randomized entry fee $d_i(b_{-i})$ in polynomial time with only sample access of $\hat{\FF}_{i,b_{-i}}$. 
	Suppose we take $k$ samples from $\hat{\FF}_{i,b_{-i}}$ and sort them in descending order $s_1\geq s_2\geq \cdots\geq s_k$.
	 Let the entry fee $d_i(b_{-i})$ to be $s_{\left\lceil\frac{5k}{16}\right\rceil}$.
	 By the Chernoff bound, with probability at least $1-\exp(-k/128)$ (over the randomness of the samples) $\Pr_{t_i\sim \hat{D_i}}\left[\sum_{j\in[m]} \left(t_{ij}-\max_{k\neq i} b_{kj}\right)^+\geq d_i(b_{-i}) \right]=\Pr_{t_i\sim \hat{D_i}}\left[\EE_{i,b_{-i},d_i(b_{-i})}\right]$ lies in $\left[\frac{1}{4},\frac{3}{8}\right]$.
	  Since $\Pr_{t_i\sim D_i}\left[\EE_{i,b_{-i},d_i(b_{-i})}\right]=\Pr_{t_i\sim \hat{D_i}}\left[\EE_{i,b_{-i},d_i(b_{-i})}\right]\pm 2m\epsilon$, $$\Pr_{t_i\sim D_i}\left[\EE_{i,b_{-i},d_i(b_{-i})}\right]\in [\frac{1}{8},\frac{1}{2}],$$ if $m\epsilon\leq 1/16$.
	   According to Lemma~\ref{lem:approximate entry fee for BREV}, the expected revenue under our entry fee $d_i(b_{-i})$ is at least $\left(\frac{1}{4}-\exp(-k/128)\right)\cdot\brev\geq \frac{\brev}{5}$ if we choose $k$ to be larger than some absolute constant. Clearly, the procedure above can be completed in polynomial time with access to $\hat{D}$.
\end{proof}

Combining Lemma~\ref{lem:SREV kolmogorov} and~\ref{lem:learn approx entry fee from approx dist}, we are ready to prove our main result of this section.
\begin{theorem}\label{thm:additive Kolmogorov}
If all bidders have additive valuations, given distributions $\hat{D}_{ij}$ where $\left|\left|\hat{D}_{ij}-D_{ij}\right|\right|_K\leq \epsilon$ for all $i\in[n]$ and $j\in[m]$, there is a polynomial time algorithm that constructs a mechanism whose expected revenue under $D$ is at least $\frac{\opt}{266}-96\epsilon\cdot mnH$ when $\epsilon\leq \frac{1}{16\max\{m,n\}}$.\end{theorem}
\begin{proof}
	Since $\epsilon\leq \frac{1}{16\max\{m,n\}}$, we can learn in polynomial time a randomized SPM whose revenue is at least $\frac{3}{16}\cdot\left(\frac{\srev}{8}-2\epsilon\cdot mnH\right)$ and a VCG with entry fee mechanism whose revenue is at least $\brev/5$. As $\opt\leq 6\cdot \srev+2\brev$ (Theorem~\ref{thm:UB additive}), the better of the two mechanisms we can learn in polynomial time has revenue at least $\frac{\opt}{266}-96\epsilon\cdot mnH$.
\end{proof}

\section{Missing Details from Section~\ref{sec:constrained additive}}\label{sec:appx XOS}

\begin{prevproof}{Lemma}{lem:approx ASPE}
We only sketch the proof here. Let $\prev$ denote the highest revenue obtainable by any RSPM. In~\cite{CaiZ17}, Cai and Zhao constructed an upper bound of the optimal revenue using duality and separated the upper bound into three components: $\single$, $\tail$ and $\core$. Both $\single$ and $\tail$ are within constant times the $\prev$, and the ASPE$(p^*,\delta^*)$ is used to bound the $\core$. It turns out one can use essentially the same proof as in~\cite{CaiZ17} to prove that the mechanism ASPE$(p',\delta')$ has revenue at least $a_1(\mu)\cdot \core-a_2(\mu)\cdot\prev-a_3(\mu)\cdot (n+m)\cdot \epsilon$ where  $a_1(\mu)$, $a_2(\mu)$ and $a_3(\mu)$ are functions that map $\mu$ to positive numbers. In other words, we can replace ASPE$(p^*,\delta^*)$ with ASPE$(p',\delta')$ and still obtain a constant factor approximation.
\end{prevproof}

\notshow{
\subsection{Missing Proofs from Section~\ref{sec:constrained additive kolomogorov}}\label{sec:appx constrained additive}
\begin{prevproof}{Lemma}{lem:Kolmogorov learn entry fee}
	To prove our claim, it suffices to prove that for any collection of prices $\{p_j\}_{j\in[m]}$, any bidder $i$ and any set of items $S$ that $\left|\Pr_{D_i}\left[u^{(p)}_i(t_i,S)\geq \delta_i^{(p)}(S)\right]-\Pr_{\hat{D}_i}\left[u^{(p)}_i(t_i,S)\geq \delta_i^{(p)}(S)\right]\right|\leq 2m\xi$. Let $\EE$ be the event that $u^{(p)}_i(t_i,S)\geq \delta_i^{(p)}(S)$. Since this is an event over bidder $i$'s type set, the dimension is $m$. If we can argue that $\EE$ is single-intersecting, our claim follows from Lemma~\ref{lem:Kolmogorov stable for sc}. 
	
	For any $j\in[m]$ and $a_{i,-j}\in [0,H]^{{m}-1}$, let $L_j(a_{i,-j})=\left\{ (t_{ij},a_{i,-j}) \ | t_{ij}\in [0,H] \right\}$, then $L_j(a_{i,-j})\cap \EE = \left\{ (t_{ij},a_{i,-j}) \ | t_{ij}\in [0,H] \text{ and }  u^{(p)}_i((t_{ij},a_{i,-j}),S)\geq \delta_i^{(p)} (S)\right\}$. If $(x_{ij}, a_{i,-j})\in L_j(a_{i,-j})\cap \EE$, clearly for any $x'_{ij}\geq x_{ij}$,  $u^{(p)}_i((x'_{ij},a_{i,-j}),S)\geq u^{(p)}_i((x_{ij},a_{i,-j}),S)$. Hence, $(x'_{ij}, a_{i,-j})\in L_j(a_{i,-j})\cap \EE$. Therefore, $\EE$ is single-intersecting.
\end{prevproof}

\notshow{
\begin{prevproof}{Lemma}{lem:difference in revenue Kolmogorov}
	We use a hybrid argument. Consider a sequence of distributions $\{D^{(i)}\}_{i\leq n}$, where $D^{(i)}=\hat{D}_1\times\cdots\times\hat{D}_i\times D_{i+1}\times\cdots\times D_{n},$ and $D^{(0)}=D$, $D^{(n)}=\hat{D}$.
	 We use $\rev^{(i)}(p,\delta)$ to denote the expected revenue of ASPE$(p,\delta)$ under $D^{(i)}$. To prove our claim, it suffices to argue that $\left|\rev^{(i-1)}(p,\delta) -\rev^{(i)}(p,\delta)\right|\leq 2\xi m\cdot \left(m\cdot H+\opt\right).$
	  We denote by $\SS_k$ and $\SS'_k$ the random set of items that remain available after visiting the first $k$ bidders under $D^{(i-1)}$ and $D^{(i)}$. Clearly, for $k\leq i-1$, $||\SS_k-\SS'_k||_{TV}=0$, so the expected revenue collected from the first $i-1$ bidders under $D^{(i-1)}$ and $D^{(i)}$ is the same. According to Lemma~\ref{lem:stable favorite set}, $||\SS_i-\SS'_i||_{TV}\leq 2m\cdot \xi$. The total amount of money bidder $i$ spends can never be higher than her value for receiving all the items which is at most $m\cdot H$. So the difference in the expected revenue collected from bidder $i$ under  $D^{(i-1)}$ and $D^{(i)}$ is at most $2\xi\cdot m^2H$. Suppose $R$ is the set of remaining items after visiting the first $i$ bidders, then the expected revenue collected from the last $n-i$ bidders is the same under  $D^{(i-1)}$ and $D^{(i)}$, as these bidders have the same distributions. Moreover, this expected revenue is no more than $\opt_{ASPE}$, since the optimal ASPE can simply just sell $R$ to the last $n-i$ bidders using the same prices and entry fee as in ASPE$(p,\delta)$. Of course, for any fixed $R$, the probabilities that $\SS_i=R$ and $\SS'_i=R$ are different, but since for any $R$ the expected revenue from the last $n-i$ bidders is at most $\opt_{ASPE}$, the difference in the expected revenue from the last $n-i$ bidders under  $D^{(i-1)}$ and $D^{(i)}$ is at most $||\SS_i-\SS'_i||_{TV} \cdot \opt \leq 2\xi\cdot m\opt_{ASPE}$. Hence, the total difference between $\rev^{(i-1)}(p,\delta)$ and  $\rev^{(i)}(p,\delta)$ is at most $2\xi m\cdot \left(m H+\opt_{ASPE}\right)$. 
	  Furthermore, $\left|\rev(p,\delta)-\widehat{\rev}(p,\delta)\right|\leq \sum_{i=1}^{n}\left|\rev^{(i-1)}(p,\delta) -\rev^{(i)}(p,\delta)\right|\leq 2 nm\xi \left(mH+\opt_{ASPE}\right).$
\end{prevproof}}

\begin{prevproof}{Theorem}{thm:constrained additive Kolmogorov}
 Let $\prev$ be the revenue of the optimal RSPM. According to Theorem~\ref{thm:UD Kolmogorov}, given $\hat{D}$, we can learn an RSPM that has revenue $\left(\frac{1}{4}-(n+m)\cdot \xi \right)\cdot \left(\frac{\prev}{8}-2\xi\cdot mnH\right)$. On the other hand, with access to $\hat{D}$, we can learn an ASPE such that either this ASPE or the best RSPM achieves a constant fraction of $\opt$ minus $\xi\cdot O(m^2nH)$. This can be proved by setting $\epsilon$ to be $O(\frac{\opt}{m+n})$ in Corollary~\ref{cor:discretization of prices}, then combine it with Lemma~\ref{lem:Kolmogorov learn entry fee} and~\ref{lem:difference in revenue Kolmogorov}. To sum up, we can learn an SPM and an ASPE with only access to $\hat{D}$ such that the better among the two achieves revenue at least $\frac{\opt}{c}-\xi\cdot O(m^2nH)$ under the actual distribution $D$ for some constant $c>1$.
\end{prevproof}
}

\subsection{Missing Proofs from Section~\ref{sec:XOS sample}}\label{sec:XOS bounded}
%As the algorithms in Section~\ref{sec:unit-demand} can be used to approximate the best RSPM. We on consider how to learn an ASPE with high revenue given sample access to bidders' valuation distributions $D$ in this section. %when the bidders' valuations are bounded. In particular, we assume that for any item $j$, any bidder $i$'s value $v_i(t_i,\{j\})$ for this item under any type $t_i\in T_i$ is no larger than $H$. As the $v_i(t_i,\cdot)$ is an XOS function, it also means that for any set $S\subseteq [m]$, $v_i(t_i,S)\leq H\cdot |S|$.
 %Our learning algorithm is a two-step procedure. In the first step, we take a few samples from $D$ and use these samples to set the entry fee for every collection of prices $\{p_j\}_{j\in[m]}$ in the $\epsilon$-net. More specifically, to decide $\delta_i(S)$ we compute the utility of bidder $i$ for set $S$ under $\{p_j\}_{j\in[m]}$ over all the samples and take the empirical median among all these utilities to be $\delta_i(S)$. With a polynomial number of samples, we can guarantee that for any $\{p_j\}_{j\in[m]}$ in the $\epsilon$-net the computed entry fee functions $\{\delta_i(\cdot)\}_{i\in[n]}$ are $\mu$-balanced. Now, we have created an ASPE for every $\{p_j\}_{j\in[m]}$ in the $\epsilon$-net. In the second step, we take some fresh samples from $D$ and use them to estimate the revenue for each of the ASPEs we created in the first step. Then pick the one that has the highest empirical revenue. It is not hard to argue that with a polynomial number of samples the mechanism we pick has high revenue with probability almost $1$. %Indeed, we show if we choose $\epsilon$ appropriately, the better of the ASPE mechanism we learned and the best RSPM mechanism achieves a constant fraction of the optimal revenue. 
 
 We formalize the first step of our algorithm in the following lemma.
\begin{lemma}\label{lem:learn median}
For any $B>0$, $\epsilon>0$, $\eta\in [0,1]$ and $\mu\in[0,\frac{1}{4}]$, suppose we take %$K=O\left(\frac{\log \frac{1}{\eta}+ \log n +(\log H +\log \frac{1}{\epsilon})\cdot m}{\mu^2}\right)$
$K=O\left(\frac{\log \frac{1}{\eta}+ \log n +m\log \frac{B}{\epsilon}}{\mu^2}\right)$ samples $t^{(1)},\cdots, t^{(K)}$ from $D$. For any collection of prices $\{p_j\}_{j\in[m]}$ in the $B$-bounded $\epsilon$-net, define the entry fee $\delta_i^{(p)}(S)$
 of bidder $i$ for set $S$ under $\{p_j\}_{j\in[m]}$ to be the median of $u_i(t^{(1)}_i,S),\cdots, u_i(t^{(K)}_i,S)$, where $u_i(t_i,S)=\max_{S*\subseteq S} v_i(t_i,S^*)-\sum_{j\in S^*} p_j$. Then with probability $1-\eta$, for any collection of prices $\{p_j\}_{j\in[m]}$ in the $B$-bounded $\epsilon$-net, $\left\{\delta_i^{(p)}(\cdot)\right\}_{i\in[n]}$ is a collection of $\mu$-balanced entry fee functions.\end{lemma}
 \begin{proof}
 	For any fixed $\{p_j\}_{j\in[m]}$, fixed bidder $i$ and fixed set $S$, it is easy to argue that the probability for $\Pr_{t_i\sim D_i}[u_i(t_i,S)\geq \delta_i^{(p)}(S)]$ to be larger than  $\frac{1}{2}+\mu$ or smaller than $\frac{1}{2}-\mu$ is at most 2$\exp(-2K\mu^2)$ due to the Chernoff bound. %Similarly, the probability for $\Pr_{t_i\sim D_i}[u_i(t_i,S)\geq \delta_i^{(p)}(S)]$ to be less than  $\frac{1}{2}-\mu$ is also at most $\exp(-2K\mu^2)$. 
 	Next, we take a union bound over all $\{p_j\}_{j\in[m]}$ in the $\epsilon$-net, all bidders and all possible subsets of $[m]$, so the probability that for any collection of prices $\{p_j\}_{j\in[m]}$ in the $\epsilon$-net $\{\delta_i^{(p)}(\cdot)\}_{i\in[n]}$ is a collection of $\mu$-balanced entry fee functions is at least $1- 2\exp(-2K\mu^2)\cdot \left(\frac{B}{\epsilon}\right)^m\cdot 2^m\cdot n$. If we take $K$ to be at least $\frac{\log \frac{1}{\eta}+ \log n +m\log \frac{B}{\epsilon}}{\mu^2}$, the success probability is at least $1-\eta$.
 \end{proof}

Next, we formalize the second step of our learning algorithm.
\begin{lemma}\label{lem:learn best ASPE}
	For any $B\geq 2G$, $\epsilon, \epsilon'>0$,  $\eta\in [0,1]$ and $\mu\in[0,\frac{1}{4}]$, suppose for every collection of prices $\{p_j\}_{j\in[m]}$ in the $B$-bounded $\epsilon$-net, $\{\delta^{(p)}_i(\cdot)\}_{i\in[n]}$ is a collection of $\mu$-balanced entry fee functions. We use $\mathcal{S}$ to denote the set that contains ASPE$(p,\delta^{(p)})$ for every $p$ in the $B$-bounded $\epsilon$-net. If we take %$K=O\left(\frac{\log \frac{1}{\eta}+(\log H +\log \frac{1}{\epsilon})\cdot m}{\epsilon'^2}\right)$
	$K=O\left(\frac{\log \frac{1}{\eta}+m\log \frac{B}{\epsilon}}{\epsilon'^2}\right)$ samples $t^{(1)},\cdots, t^{(K)}$ from $D$ and let ASPE$(p',\delta^{(p')})$ be the mechanism that has the highest revenue in $\mathcal{S}$. Then with probability at least $1-\eta$, the better of ASPE$(p',\delta^{(p')})$ and the best RSPM achieves revenue at least $\frac{\opt}{\CC_1(\mu)}-\CC_2(\mu)\cdot (m+n)\cdot \epsilon-2mnB\cdot \epsilon'$. 
\end{lemma}
\begin{proof}
	For any $\{p_j\}_{j\in[m]}$ in the $\epsilon$-net, define $\rev(p)$ to be the expected revenue of ASPE$(p,\delta^{(p)})$ and  $\widehat{\rev}(p)$ be the average revenue of ASPE$(p,\delta^{(p)})$ among the $K$ samples. First, we argue that $\widehat{\rev}(p)$ is a random variable that lies between $[0,mnB]$. The revenue from selling the items can be at most $mB$ as there are only $m$ items and $p_j\leq B$ for all $j\in[m]$. How about the entry fee? 
	For any bidder $i$, $$\Pr_{t_i\sim D_i}\left[v_i(t_i,[m])\geq mG\right]\leq \sum_{j\in[m]} \Pr_{t_{ij}\sim D_{ij}}\left[V_i(t_{ij})\geq G\right]\leq \frac{m}{5\max\{m,n\}}\leq\frac{1}{5}.$$ The first inequality is because $v_i(t_i,\cdot)$ is a subadditive function for every type $t_i\in T_i$, so for $v_i(t_i,[m])$ to be greater than $mG$, there must exist a item $j$ such that $V_i(t_{ij})\geq G$. The second inequality follows from the definition of $G$ in Theorem~\ref{thm:simple XOS}.
	
	 If there exists a set $S\subseteq[m]$ such that $\delta_i^{(p)}(S)> mG$, we have $$\Pr_{t_i\sim D_i}\left[v_i(t_i,[m])\geq mG\right]\geq \Pr_{t_i\sim D_i}\left[v_i(t_i,[m])\geq \delta_i^{(p)}(S)\right]\geq %\Pr_{t_i\sim D_i}\left[\max_{S*\subseteq S} v_i(t_i,S^*)-\sum_{j\in S^*} p_j\geq \delta_i^{(p)}\right]\geq
	  \frac{1}{2}-\mu\geq \frac{1}{4}.$$ Contradiction. Note that the second inequality is because $\delta_i^{(p)}(\cdot)$ is $\mu$-balanced. Hence, the entry fee is always upper bounded by $mG$ and $\widehat{\rev}(p)$ is at most $mnG+mB\leq mnB$. Also, notice that the expectation of $\widehat{\rev}(p)$ is exactly $\rev(p)$. 
	  By the Chernoff bound, $$\Pr\left[\left|\rev(p)-\widehat{\rev}(p)\right|\leq mnB\cdot \epsilon'\right]\geq 1-2\exp(-2K\cdot \epsilon'^2)$$ for any fixed $\{p_j\}_{j\in[m]}$. By the union bound, the probability that for all $\{p_j\}_{j\in[m]}$ in the $\epsilon$-net $$\left|\rev(p)-\widehat{\rev}(p)\right| \leq mnB\cdot \epsilon'$$ is at least $1-2\exp(-2K\cdot \epsilon'^2)\cdot \left(\frac{B}{\epsilon}\right)^m$, which is lower bounded by $1-\eta$ due to our choice of $K$. 
When this happens, the expected revenue of ASPE$(p',\delta^{(p')})$ is at most $2mnB\cdot \epsilon'$ less than the highest expected revenue achievable by any of these mechanisms, because 
$$\rev(p')\geq \widehat{\rev}(p')-mnB\cdot \epsilon' \geq \widehat{\rev}(p)-mnB\cdot \epsilon'\geq \rev(p)-2mnB\cdot \epsilon'$$ for any $p$ in the $\epsilon$-net. Combining this inequality with Corollary~\ref{cor:discretization of prices} completes our proof.
\end{proof}

Note that Lemma~\ref{lem:learn median} and \ref{lem:learn best ASPE} hold for all distributions $D$. The reason we require $D$ to be bounded or regular is because without these restrictions, we do not know how to approximate the best RSPM. In the following Theorem, we combine Lemma~\ref{lem:learn median}, ~\ref{lem:learn best ASPE} and Theorem~\ref{thm:UD bounded} to obtain the sample complexity of our learning algorithm for bounded distributions. 
\begin{theorem}\label{thm:XOS bounded}
	When all bidders' valuations are XOS over independent items and the random variable $V_i(t_{ij})$ is supported on $[0,H]$ for any bidder $i$ and any item $j$, with $O\left(\left(\frac{mn}{\xi}\right)^2 \cdot \left(m\cdot\log \frac{m+n}{\xi} + \log \frac{1}{\delta}\right)\right)$ samples from $D$, we can learn an RSPM and an ASPE such that with probability at least $1-\delta$ the better of the two mechanisms has revenue at least $\frac{\opt}{c}-\xi\cdot H$ for some absolute constant $c>1$. %and for any bidder $i$ and any item $j$ the random variable $V_i(t_{ij})$ is supported on $[0,H]$, then with  $O\left(\left(\frac{mn}{\xi}\right)^2 \cdot \left(m\cdot\log \frac{m+n}{\xi} + \log \frac{1}{\delta}\right)\right)$ samples our algorithm can learn an RSPM and an ASPE such that with probability at least $1-\delta$ the better of the two mechanisms has revenue at least $\frac{\opt}{c}-\xi\cdot H$. $c>1$ is an absolute constant. %\yangnote{With the same number of samples, we can learn in polynomial time with probability $1-\delta$ an SPM whose revenue is at least $\frac{\opt}{144}-\xi H$.}
\end{theorem}
\begin{proof}
With $O\left(\left(\frac{1}{\xi}\right)^2 \left(m^2 n\log n\log \frac{1}{\xi} + \log \frac{1}{\delta}\right)\right)$ samples, we can obtain an RSPM whose revenue is at least $\frac{1}{24}$ of the revenue of the best RSPM minus $\frac{\xi}{2}\cdot H$ with probability $1-\delta/2$ according to Theorem~\ref{thm:UD bounded}. 
Let $\mu$ be some fixed constant in $[0,\frac{1}{4}]$, $B=2H$, $\epsilon = \frac{\xi\cdot H}{6\CC_2(\mu) (m+n)}$ and $\epsilon' = \frac{\xi}{12mn}$. 
%We know that given $O\left(\log \frac{1}{\delta}+ \log n +m\log \frac{m+n}{\xi}\right)$ samples, we can construct $\mu$-balanced entry fee functions for all price vectors in the $B$-bounded $\epsilon$-net with probability $1-\delta/4$ due to Lemma~\ref{lem:learn median}.
According to Lemma~\ref{lem:learn median}, given $O\left(\log \frac{1}{\delta}+ \log n +m\log \frac{m+n}{\xi}\right)$ samples, we can construct an entry fee function for each price vector in the $B$-bounded $\epsilon$-net, such that all these entry fee functions are $\mu$-balanced with probability at least $1-\delta/4$.
 According to Lemma~\ref{lem:learn best ASPE}, we can learn an ASPE with $O\left(\left(\frac{mn}{\xi}\right)^2 \cdot \left(m\cdot\log \frac{m+n}{\xi} + \log \frac{1}{\delta}\right)\right)$ fresh samples from $D$, such that the better of the ASPE we learned and the best RSPM has revenue of at least $\frac{\opt}{\CC_1(\mu)}-\frac{\xi}{2}\cdot H$ with probability $1-\delta/4$. Combining the statements above, we can learn with probability $1-\delta$ a mechanism whose revenue is at least $\frac{\opt}{c}-\xi\cdot H$ with $O\left(\left(\frac{mn}{\xi}\right)^2 \cdot \left(m\cdot\log \frac{m+n}{\xi} + \log \frac{1}{\delta}\right)\right)$ samples.
\end{proof}

In the next Theorem, we combine Lemma~\ref{lem:learn median}, ~\ref{lem:learn best ASPE} and Theorem~\ref{thm:UD regular} to obtain the sample complexity of our learning algorithm for regular distributions.

\begin{theorem}\label{thm:XOS regular}
	When all bidders' valuations are XOS over independent items and the random variable $V_i(t_{ij})$ is regular for each item $j\in[m]$ and bidder $i\in[n]$, with $O\left(Z^2m^2n^2 \cdot \left(m\cdot\log ({m+n}) + \log \frac{1}{\delta}\right)\right)$ ($Z=\max\{m,n\}$) samples from $D$, we can learn an RSPM and an ASPE such that with probability at least $1-\delta$ the better of the two mechanisms has revenue at least $\frac{\opt}{c}$ for some absolute constant $c>1$. %and for any bidder $i$ and any item $j$ the random variable $V_i(t_{ij})$ is supported on $[0,H]$, then with  $O\left(\left(\frac{mn}{\xi}\right)^2 \cdot \left(m\cdot\log \frac{m+n}{\xi} + \log \frac{1}{\delta}\right)\right)$ samples our algorithm can learn an RSPM and an ASPE such that with probability at least $1-\delta$ the better of the two mechanisms has revenue at least $\frac{\opt}{c}-\xi\cdot H$. $c>1$ is an absolute constant. %\yangnote{With the same number of samples, we can learn in polynomial time with probability $1-\delta$ an SPM whose revenue is at least $\frac{\opt}{144}-\xi H$.}
\end{theorem}

\begin{proof}
According to Theorem~\ref{thm:UD regular}, we can learn with probability $1-\delta/2$ a randomized RSPM whose revenue is at least $\frac{1}{33}$ of the optimal RSPM with $O\left(Z^2m^2 n^2\cdot  \log \frac{nm}{\delta}\right)$ samples. Next, we learn an ASPE with high revenue.
	With $O\left( Z^2\cdot \log \frac{nm}{\delta} \right)$ samples from each $D_{ij}$, we can estimate $W_{ij}$ such that $$\Pr_{t_{ij}\sim D_{ij}}\left[V_i(t_{ij}) \geq W_{ij}\right]\in\left[\frac{1}{6Z}, \frac{1}{5Z}\right]$$ with probability $1-\frac{\delta}{4nm}$. By the union bound, the probability that all $W_{ij}$ satisfy the requirement is at least $1-\frac{\delta}{4}$. So with probability at least $1-\frac{\delta}{4}$, $W_{ij}\geq G_{ij}$ for all $i\in[n]$ and $j\in[m]$. 
	
	Let $B=2\cdot \max_{i,j} W_{ij}$, $\mu$ be some fixed constant in $[0,\frac{1}{4}]$, $\epsilon = \frac{\xi\cdot B}{\CC_2(\mu) Z(m+n)}$ and $\epsilon' = \frac{\xi}{2mnZ}$ for some small constant $\xi$, which will be specified later. We know that given $O\left(\log \frac{1}{\delta}+ \log n +m\log ({m+n})\right)$ samples, we can construct $\mu$-balanced entry fee functions for all price vectors in the $B$-bounded $\epsilon$-net with probability $1-\delta/8$ due to Lemma~\ref{lem:learn median}. According to Lemma~\ref{lem:learn best ASPE}, we can learn an ASPE with $$O\left(Z^2m^2n^2 \cdot \left(m\cdot\log (m+n) + \log \frac{1}{\delta}\right)\right)$$ fresh samples from $D$, such that the better of the ASPE we learned and the best RSPM has revenue of at least $\frac{\opt}{\CC_1(\mu)}-\frac{2\xi\cdot B}{Z}$ with probability $1-\delta/8$. Note that there exists a bidder $i$ and an item $j$ such that $W_{ij}=B/2$, so $\opt\geq \frac{B}{2}\cdot \frac{1}{6Z}$ and for sufficiently small $\xi$, $\frac{\opt}{\CC_1(\mu)}-\frac{2\xi\cdot B}{Z}\geq \frac{\opt}{2\CC_1(\mu)}$. Combining the statements above, we can learn with probability $1-\delta$ a mechanism whose revenue is at least $\frac{\opt}{c}$ for some absolute constant $c$ with $O\left(Z^2m^2n^2 \cdot \left(m\cdot\log (m+n) + \log \frac{1}{\delta}\right)\right)$ samples.

\end{proof}

\section{Learning Algorithms for Symmetric Bidders}\label{sec:symmetric appx}
\subsection{An Upper Bound of the Optimal Revenue for Symmetric Bidders}\label{sec:UB symmetric}
In this section, we introduce an upper bound to $\opt$ based on duality~\cite{CaiZ17}, which is crucial for us to prove the approximation ratios of our learning algorithms. 
We first fix some notation. Let $\prev$ be the highest revenue obtainable by any RSPM. As the bidders are symmetric, we drop the subscript $i$ when there is no confusion. In particular, we use $V(t_{ij})$ to denote bidder $i$'s value for winning item $j$ if her private information for item $j$ is $t_{ij}$, and $v(t_i,S)$ to denote bidder $i$'s value for set $S$ when her type is $t_i$. We use $D_j$ to denote the distribution of the private information about item $j$. Let $\sigma_{iS}(t)$ be the interim probability for bidder $i$ to receive exactly set $S\subseteq [m]$ when her type is $t$. % and $p_i(t)$ be the interim price for bidder $i$ when her type is $t$. %A mechanism $(\boldsymbol{\sigma}, \bold{p})$ is \emph{symmetric} if $\sigma_{iS}(t) = \sigma_{i'S}(t)$ and $p_i(t)=p_{i'}(t)$ for any type $t$, set $S\in[m]$ and any $i$ and $i'\in[n]$. It is not hard to observe that there always exists a revenue-optimal symmetric mechanism when the bidders are symmetric.

In~\cite{CaiZ17}, an upper bound of the optimal revenue is derived using duality theory. Their upper bound applies to asymmetric bidders with valuations that are subadditive over independent items. When the bidders are symmetric, we can simplify their upper bound. First, we need the definition of $b$-balanced thresholds.

\begin{definition}[$b$-balanced Thresholds]\label{def:b-balanced}
	For any constant $b\in (0,1)$, a collection of positive real numbers $\{\beta_j\}_{j\in[m]}$ is $b$-balanced if for all $i\in[n]$ and $j\in[m]$, $\Pr_{t_{ij}\sim D_j}\left [V(t_{ij})\geq \beta_j\right]\in[\frac{b}{n},\frac{b}{n-1}]$.
	%\begin{enumerate}
		%\item $\Pr_{t_{ij}\sim D_j}\left [V(t_{ij})\geq \beta_j\right]\leq \frac{b}{n-1}$.
		%\item $\Pr_{t_{ij}\sim D_{j}}\left[V(t_{ij})\geq \beta_j\right]\geq \frac{b}{n}$.
	%\end{enumerate}
\end{definition}

Note that when bidders are asymmetric, $b$-balanced thresholds are not guaranteed to exist, as there may not exist any $\beta_j$ that satisfies $\Pr_{t_{ij}\sim D_j}\left [V(t_{ij})\geq \beta_j\right]\in[\frac{b}{n},\frac{b}{n-1}]$ for all bidder $i$ simultaneously. Next, we define the $\core_\eta(\bbeta)$ which will be crucial for upper bounding the optimal revenue.\footnote{For readers that are familiar with the definition of the $\core$ in~\cite{CaiZ17}, $\core_\eta(\bbeta)$ is essentially the same term but adapted for symmetric bidders.}

\begin{definition}[\core]\label{def:core and tail}
%Given any $b$-balanced thresholds $\{\beta_j\}_{j\in[n]}$, let
%$$c(\boldsymbol{\beta}) :=\inf\left\{x\geq 0:\  \sum_j \Pr_{t_{j}\sim D_{j}}\left[V(t_{j})\geq \beta_{j}+x\right]\leq \frac{1}{2}\right\}.$$
Given any collection of %$b$-balanced
 thresholds $\{\beta_j\}_{j\in[n]}$ and a nonnegative constant $\eta\leq \frac{1}{4}$, \begin{itemize}
	\item if $\sum_{j\in[m]} \Pr_{t_{j}\sim D_{j}}\left[V(t_{j})\geq \beta_{j}\right]\leq \frac{1}{2}-\eta$, let $c_\eta(\boldsymbol{\beta})$ be $0$;
	\item otherwise, let $c_\eta(\boldsymbol{\beta})$ be a nonnegative number such that $\sum_{j\in[m]} \Pr_{t_{j}\sim D_{j}}\left[V(t_{j})\geq \beta_{j}+c_\eta(\boldsymbol{\beta})\right]\in  \left[\frac{1}{2}-\eta, \frac{1}{2}\right]$.
\end{itemize}
%For every type $t_i\in T_i$ of bidder $i$, let $\mathcal{C}_{\eta,i}(t_i)=\{j\ |\ V(t_{ij})< \beta_{j}+c_\eta(\bbeta)\}$.
For every type $t$, let $\mathcal{C}_{\eta}(t)=\{j\ |\ V(t_{ j})< \beta_{j}+c_\eta(\bbeta)\}$. Then, 
$$\core_\eta(\bbeta)= \max_{\sigma \in P(D)} \sum_{i\in[n]}\sum_{t_i\in T_i}f(t_i)\cdot \sum_{S\subseteq[m]}\sigma_{iS}(t_i)\cdot v\left(t_i,S\cap \mathcal{C}_{\eta}(t_i)\right),$$ where $P(D)$ is the set of all feasible interim allocation rules. That is, {$\core_\eta(\bbeta)$ is the maximum welfare a mechanism can extract out of the allocation of items whose individual value for the bidder they are allocated to is lower than the adjusted thresholds.}
\end{definition}

%\cnote{Where is the definition of the TAIL? Is it the remaining welfare?} \yangnote{We don't actually need the TAIL as it can be upper bounded by PostRev, and it's more complicated to define. We can directly state Theorem~\ref{thm:UB subadditive}, so we don't even need to mention SINGLE or TAIL.}

%In~\cite{CaiZ17}, every collection of thresholds induces an upper bound to the optimal revenue. In particular, for any choice of  $b$-balanced thresholds $\{\beta_j\}_{j\in[m]}$ and $\eta$
It was shown in~\cite{CaiZ17} that every collection of thresholds induces an upper bound to the optimal revenue. In particular, for any choice of thresholds $\{\beta_j\}_{j\in[m]}$ and $\eta$\footnote{In~\cite{CaiZ17}, the thresholds are allowed to depend on the identity of the bidder. More specifically, for any $i\in[n]$ and $j\in[m]$, there is an associated threshold $\beta_{ij}$. Their upper bound applies to asymmetric thresholds as well. Indeed, when the bidders are asymmetric, their upper bound is induced by a set of asymmetric thresholds. As we only discuss symmetric bidders in this section, we focus on symmetric thresholds for simplicity. Regarding $\eta$, Cai and Zhao only considered the case when $\eta=0$, but their analysis can be easily modified to accommodate any $\eta\leq 1/4$. See Theorem~\ref{thm:UB subadditive} for the modified upper bound.}, the revenue $\rev(M)$ of any BIC mechanism $M$ is upper bounded by  $$2\cdot \single(M,\bbeta)+4\cdot \tail_\eta(M,\bbeta)+4\cdot\core_\eta(M,\bbeta)\quad\text{(Adapted from Theorem 2 in~\cite{CaiZ17})}.$$ These terms depend on the choice of $\{\beta_j\}_{j\in[m]}$, $\eta$ as well as the mechanism $M$. We refer interested readers to~\cite{CaiZ17} for the definitions of these terms. To obtain a benchmark/upper bound of the optimal revenue, one can simply replace the above expression with $$2\cdot \max_M \single(M,\bbeta)+4\cdot \max_M \tail_\eta(M,\bbeta)+4\cdot\max_M \core_\eta(M,\bbeta).$$ {It is not hard to see that this benchmark may be impossible to approximate for certain choices of the thresholds. Just imagine the case when the thresholds are extremely high, then $\max_M \core_\eta(M,\bbeta)$ becomes the optimal social welfare which can be arbitrarily large comparing to the optimal revenue. What Cai and Zhao~\cite{CaiZ17} showed was that when the thresholds are $b$-balanced, this upper bound can indeed be approximated by the revenue of an RSPM and an ASPE. From now on, we only consider $b$-balanced thresholds.}

%\cnote{Didn't you just define CORE and TAIL? Maybe just say we refer the reader to [10] for the definition of SINGLE? Or is SINGLE very easy to state?} \yangnote{SINGLE and TAIL are not so easy to state and not really used in later sections, CORE is not exactly the same either.} 

 
Using results in~\cite{CaiZ17}, we can further simplify the benchmark. In particular, $\max_{M}\single(M,\bbeta)$ is less than $6\cdot\prev$ for all choices of $\{\beta_j\}_{j\in[n]}$ and $\max_{M}\tail_\eta(M,\bbeta)$ is less than $\frac{2}{1-b}\cdot\prev$ for any choice of $\eta$ and $b$-balanced thresholds $\{\beta_j\}_{j\in[n]}$. Moreover, $\max_{M}\core_\eta(M,\bbeta)\leq \core_\eta(\bbeta)$. Combining the inequalities above, we obtain the following Theorem.
\begin{theorem}[Adapted from~\cite{CaiZ17}]\label{thm:UB subadditive}
	When the bidders are symmetric and have valuations that are subadditive over independent items, for any constant $b\in (0,1)$, $\eta\leq \frac{1}{4}$ and a collection of $b$-balanced thresholds $\{\beta_j\}_{j\in[m]}$, $$\opt \leq \left(12+\frac{8}{1-b}\right)\cdot \prev + 4\cdot\core_\eta(\bbeta).$$
\end{theorem}


\subsection{Symmetric Bidders with XOS Valuations}\label{sec:symmetric XOS}
In this section, we show how to learn in polynomial time an approximately optimal mechanism for symmetric bidders with XOS valuations given sample access to the distributions. According to Theorem~\ref{thm:UB subadditive}, we only need to learn a mechanism that approximates $\prev$ and $\core_\eta(\bbeta)$. From Section~\ref{sec:unit-demand}, we know how to approximated $\prev$ in polynomial time, so we focus on learning a mechanism whose revenue  approximates $\core_\eta(\bbeta)$. 

First, we need a crucial property about XOS valuations.

\begin{lemma}[Supporting Prices~\cite{DobzinskiNS05}]\label{lem:supporting price}
If $v(t,\cdot)$ is an XOS function, for any subset $S\subseteq[m]$ there exists a collection of supporting prices $\left\{\theta^{S}_j (t)\right\}_{j\in S}$ for $v(t,S)$ such that\begin{enumerate}
	\item $v(t,S') \geq \sum_{j\in S'} \theta^{S}_j (t)$ for all $S'\subseteq S$ and
	\item $\sum_{j\in S}\theta^{S}_j (t) = {v(t,S)}$.
\end{enumerate}
\end{lemma}

Let $v'(t_i,S)=v\left(t_i,S\cap \CC_{\eta}(t_i)\right)$ and $\FF_i$ be the distribution of the valuation $v'(t_i,S)$. As the bidders are symmetric, $\FF_i=\FF_{i'}$ for any $i$ and $i'$. The $\core_\eta(\bbeta)$ is exactly the maximum expected social welfare if every bidder $i$'s valuation is drawn independently from $\FF_i$. Cai and Zhao~\cite{CaiZ17} showed how to use an ASPE to approximate this term. In the next Lemma, we construct the prices used in their ASPE and show its relation to $\core_\eta(\bbeta)$.


\begin{lemma}\label{lem:Q_j}(Adapted from~\cite{CaiZ17})
	Let every bidder $i$'s valuation be  $v'(t_i,S)=v\left(t_i,S\cap \CC_{\eta}(t_i)\right)$ when her type is $t_{i}$ and $\sigma^*$ be a symmetric allocation that achieves $\alpha$-fraction of the optimal social welfare with respect to $v'(\cdot,\cdot)$. 
	For every item $j\in[m]$, let $$Q_{\eta,j}=\frac{1}{2}\cdot \sum_{i\in[n]} \sum_{t_i\in T_i}f(t_i)\cdot\sum_{S:j\in S}\sigma_{iS}^{*}(t_i)\cdot \theta_j^{S\cap  \CC_{\eta}(t_i)}(t_i),$$ where $\left\{\theta_j^{S\cap  \CC_{\eta}(t_i)}(t_i)\right\}_{j\in S\cap  \CC_{\eta}(t_i)}$ is the supporting prices for $v\left(t_i,S\cap  \CC_{\eta}(t_i)\right)$.
	 Let $$u^*(t, S)=\max_{S^*\subseteq S} v(t, S^*)-\sum_{j\in S^*} Q_{\eta,j}$$ be a bidder's utility for the set of items $S$ when her type is $t$. 
	 We define $\delta^*(S)$ to be the median of the random variable $u^{*}(t,S)$ (with $t\sim \times_{j\in[m]}D_j$) for any set $S\subseteq [m]$.
	  The revenue of ASPE$\left(\{Q_{\eta,j}\}_{j\in[m]}, \delta^*\right)$ is at least $$\frac{\alpha\cdot \core_{\eta}(\bbeta)}{2}-\CC(b,\eta)\cdot \prev,$$ where $\CC(b,\eta)$ is a function that only depends on $b$ and $\eta$. %If we set the entry fee function $\delta^*_i(S)$ to be the median of bidder $i$'s utility for set $S$, either the ASPE$(Q_{\eta},\delta^*)$ or the best RSPM achieves at least a constant fraction of the optimal revenue when bidders' valuations are XOS over independent items. More formally, let $u^*_i(t_i, S)=\max_{S^*\subseteq S} v_i(t_i, S^*)-\sum_{j\in S^*} p^*_j$ be bidder $i$'s utility for the set of items $S$ when her type is $t_i$. We define $\delta^*_i(S)$ to be the median of the random variable $u_i(t_i,S)$ (with $t_i\sim D_i$) for any set $S\subseteq [m]$.%Then $$2\sum_{j\in[m]} Q_{\eta,j} = \core_\eta(\bbeta).$$
	\end{lemma}
\begin{proof}
We can essentially use the same proof in~\cite{CaiZ17} to prove that the expected revenue of the ASPE is at least $$\sum_{j\in[m]} Q_{\eta,j}-\CC(b,\eta)\cdot \prev.$$ For readers that are familiar with that proof, the only thing we need to make sure is that our choice of $\sigma^*$ and $\{\beta_j\}_{j\in[m]}$ satisfy Lemma 5 in~\cite{CaiZ17}. Since $\sigma^*$ is symmetric and $\{\beta_j\}_{j\in[m]}$ is $b$-balanced, for all bidder $i$ and item $j$ $$\sum_{k\neq i}\Pr_{t_{kj}\sim D_{j}}\left[V(t_{kj})\geq \beta_{j}\right]\leq \frac{b}{n-1}\cdot (n-1)=b,$$ and $$\Pr_{t_{ij}\sim D_j}\left[V(t_{ij}) \geq \beta_j\right]/b\geq 1/n\geq  \cdot \sum_{t_i\in T_i} f_i(t_i)\cdot \sum_{S: j\in S} \sigma^*_{iS}(t_i).$$

Next, we argue $\sum_{j\in[m]} Q_{\eta,j}\geq \frac{\alpha\cdot \core_{\eta}(\bbeta)}{2}$. Observe that $$\sum_{j\in[m]} Q_{\eta,j}=\frac{1}{2}\cdot \sum_{i\in[n]} \sum_{t_i\in T_i}f(t_i)\cdot\sum_{S}\sigma_{iS}^{*}(t_i)\cdot v'(t_i,S)\geq \alpha\cdot \core_{\eta}(\bbeta).$$ The last inequality is because $\core_{\eta}(\bbeta)$ is the maximum social welfare under $v'(\cdot,\cdot)$ and $\sigma^*$ achieves $\alpha$ fraction of that.
\end{proof}	
	
\begin{lemma}\label{lem:symmetric approx ASPE}
	For any $\epsilon>0$ and $\mu\in[0,\frac{1}{4}]$, let $\{Q_j\}_{j\in[m]}$ be a collection of prices such that $\left|Q_j-Q_{\eta,j}\right|\leq \epsilon$ for all $j\in[m]$. Let $\delta(S)$ be the entry fee function such that $\Pr_{t\sim \times_{j\in[m]} D_j}\left [u(t,S)\geq \delta(S)\right]\in [1/2-\mu,1/2+\mu]$ for any set $S\subseteq [m]$, where $u(t,S) = \max_{S*\subseteq S} v(t,S^*)-\sum_{j\in S^*} Q_j$. Then, the ASPE$(Q,\delta)$ achieves at least $\frac{\alpha\cdot\core_{\eta}(\bbeta)}{\BB_{1}(\mu)}-\BB_{2}(b,\eta,\mu)\cdot \prev-\BB_3(\mu)\cdot (m+n)\cdot \epsilon$ revenue when bidders' valuations are XOS over independent item. Both $\BB_1(\mu)$ and $\BB_{3}(\mu)$ are functions that only depend on $\mu$ and $\BB_{2}(b,\eta,\mu)$ is a function that only depends on $\mu$, $b$ and $\eta$. 
\end{lemma}
\begin{proof}
It turns out the proof in~\cite{CaiZ17} is robust enough to accommodate the error $\epsilon$ and $\mu$. We can prove the claim by following essentially the same analysis as in~\cite{CaiZ17}. We do not include the details here.\end{proof}
	
\subsubsection{Leaning the ASPE in Polynomial Time}

We first show how to learn a collection of $b$-balanced thresholds and the corresponding $c_\eta(\bbeta)$.
\begin{lemma}\label{lem:symmetric XOS learn beta and c}
	For any positive constant $b<1$ and $\eta\leq \frac{1}{4}$, there is a polynomial time algorithm that computes a collection of $b$-balanced thresholds $\{\beta_j\}_{j\in[m]}$ and $c_\eta(\bbeta)$ with probability $1-\delta$ using $O\left(m^2 n^4 \log \frac{m}{\delta}\right)$ samples from distribution $\times_{j\in[m]} D_j$.
\end{lemma}
 	\begin{proof}
 		Given $K=O\left(m^2n^4\left(\log m+\log \frac{1}{\delta}\right)\right)$ samples $t_j^{(1)},\ldots, t_j^{(K)}$ from distribution $D_j$, we construct $\FF_j$ as the uniform distribution over $V\left(t_j^{(1)}\right),\ldots,V\left(t_j^{(K)}\right)$. According to the DKW Theorem~\cite{DvoretzkyKW56}, with probability at least $1-\delta/m$, 
 		\begin{equation}\label{eq:symmetric XOS empirical}
 			\quad \left|\Pr_{t_{j}\sim D_j}\left[V(t_j)\geq x\right]-\Pr_{v_j\sim \FF_j}\left[v_j\geq x\right]\right|\leq \frac{1}{c\cdot mn^2}~~\text{for all $x$} 	
 		\end{equation}
 where $c$ is a constant that will be specified later. From now on, we assume that Inequality~(\ref{eq:symmetric XOS empirical}) holds for every $j$, which happens with probability $1-\delta$. 
 
As %for sufficiently large $c$, 
$1/K\leq \frac{b}{3n^2}\leq \frac{b}{n-1}-\frac{b}{3n^2}- \frac{b}{n}-\frac{b}{3n^2}$, there must exist a sample $t_j^{(\ell)}$ such that $\Pr_{v_j\sim \FF_j}\left[v_j\geq V\left(t_j^{(\ell)}\right)\right]\in \left[\frac{b}{n}+\frac{b}{3n^2}, \frac{b}{n-1}-\frac{b}{3n^2}\right]$. Let $\beta_j=V\left(t_j^{(\ell)}\right)$. 
Note that $$\Pr_{t_{j}\sim D_j}\left[V(t_j)\geq \beta_j\right]\in \left[\Pr_{v_j\sim \FF_j}\left[v_j\geq \beta_j \right]-\frac{1}{c\cdot mn^2}, \Pr_{v_j\sim \FF_j}\left[v_j\geq \beta_j\right]+\frac{1}{c\cdot mn^2}\right].$$ 
If $c$ is less than $\frac{b}{3}$, $\Pr_{t_{j}\sim D_j}\left[V(t_j)\geq \beta_j\right]\in\left[\frac{b}{n}, \frac{b}{n-1}\right]$. Thus, $\beta_j$ is $b$-balanced  for all item $j$.%To sum up, our choice of $\beta_j$ is $b$-balanced with probability $1-\delta/m$. By the union bound, the collection of thresholds $\{\beta_j\}_{j\in[m]}$ is $b$-balanced with probability $1-\delta$.
 		
 		Next, we argue how to learn $c_\eta(\bbeta)$. If $\sum_{j\in[m]} \Pr_{v_j\sim\FF_j}\left[v_j\geq \beta_j \right] \leq \frac{1}{2}-\frac{\eta}{2}$, let $c_{\eta}(\bbeta)=0$. 
		This is a valid choice, as $\sum_{j} \Pr_{t_{j}\sim D_j}\left[V(t_j)\geq \beta_j\right]$ is at most $\frac{1}{2}-\frac{\eta}{2}+\frac{1}{cn^2}\leq \frac{1}{2}$ %(for sufficiently large $c$)
		 according to inequality (\ref{eq:symmetric XOS empirical}). 
		 Suppose $\sum_{j\in[m]} \Pr_{v_j\sim\FF_j}\left[v_j\geq \beta_j \right] > \frac{1}{2}-\frac{\eta}{2}$, as $m/K< \frac{\eta}{4}$, there must exist some item $k\in[m]$ and a sample $V\left(t_k^{(\ell)}\right)\geq \beta_k$ such that $\sum_{j\in[m]} \Pr_{v_j\sim\FF_j}\left[v_j\geq \beta_j+V\left(t_k^{(\ell)}\right)- \beta_k \right]\in \left[\frac{1}{2}-\frac{\eta}{4},\frac{1}{2}-\frac{\eta}{2}\right]$.
		  Let $c_\eta(\bbeta)= V\left(t_k^{(\ell)}\right)- \beta_k$.
		   According to inequality~(\ref{eq:symmetric XOS empirical}),  $$\sum_{j\in[m]} \Pr_{t_{j}\sim D_j}\left[V(t_j)\geq \beta_j+c_\eta(\bbeta) \right]\in  \left[\frac{1}{2}-\frac{\eta}{4}-\frac{1}{cn^2},\frac{1}{2}-\frac{\eta}{2}+\frac{1}{cn^2}\right].$$
		    For sufficiently large $c$, $\sum_{j} \Pr_{t_{j}\sim D_j}\left[V(t_j)\geq \beta_j+c_\eta(\bbeta) \right]\in  \left[\frac{1}{2}-\eta,\frac{1}{2}\right]$. 
 		
 		Finding each $\beta_j$ takes $O(K\log K)$ time and finding the $c_\eta(\bbeta)$ takes $O(mK)$ time. So we can learn in polynomial time a collection of $b$-balanced thresholds $\{\beta_j\}_{j\in[m]}$ and $c_\eta(\bbeta)$ with probability $1-\delta$ using $O\left(m^2n^4\log \frac{m}{\delta}\right)$ samples.
 			\end{proof}
 			
 		Next, we show how to learn the prices of the ASPE. As showed by Feige~\cite{Feige09}, there exists a polynomial time algorithm that achieves $1-\frac{1}{e}$ fraction of the optimal social welfare when bidders have XOS valuations. We let $\sigma^*$ be the interim allocation rule induced by Feige's algorithm and estimate the prices by running Feige's algorithm on sampled valuation profiles.	To run Feige's algorithm, we need a demand oracle for bidder's valuations. In the following Lemma, we argue that $v'(t,\cdot)$ is an XOS function for any type $t$, and given a value (or demand, XOS) oracle for $v(t,\cdot)$, we can construct in polynomial time the corresponding oracle for $v'(t,\cdot)$. First, we define these oracles formally.
 
 \begin{definition}
 We consider the following three oracles for a bidder's valuation function $v(t,\cdot)$:
 \begin{itemize}
 	\item \textbf{Value oracle}: takes a set $S\subseteq [m]$ as the input and returns $v(t,S)$.
 	\item \textbf{Demand oracle}: takes a collection of prices $\{p_j\}_{j\in[m]}$ as an input and returns the favorite set under these prices, that is, $S^*\in \argmax_{S\in[m]} v(t,S)-\sum_{j\in S} p_j$.
 	\item \textbf{XOS oracle} (only when $v(t,\cdot)$ is XOS):  takes a set $S\subseteq [m]$ as the input and returns the supporting prices $\{\theta_j^{S}(t)\}_{j\in S}$ for $v(t,S)$.
 \end{itemize}
 \end{definition}

 \begin{lemma}\label{lem:truncated still XOS and oracle}
 	Given a collection of thresholds $\{\beta_j\}_{j\in[m]}$ and $c_{\eta}(\bbeta)$. %For any type $t$, let $\CC_{\eta}(t) =\{j\ |\ V(t_{j})< \beta_{j}+c_\eta(\bbeta)\}$.
 	For any set $S\subseteq[m]$, let $v'(t,S) = v(t,S\cap \CC_{\eta}(t))$. If $v(t,\cdot)$ is an XOS function, $v'(t,\cdot)$ is also an XOS function. Given a value (or demand, XOS) oracle for $v(t,\cdot)$, we can construct in polynomial time a value (or demand, XOS) oracle for $v'(t,\cdot)$.
 \end{lemma}			 
 	\begin{proof}
 		If $v(t,\cdot)$ is an XOS function, $v(t,\cdot)$ can be represented as the max of a collection of additive functions. Observe that if we change the values for items in $\CC_{\eta}(t)$ to $0$ in each of these additive functions, $v'(t,\cdot)$ equals to the max of this new collection of additive functions. Hence, $v'(t,\cdot)$ is also an XOS function.
 		
 		If we are given a value oracle for $v(t,\cdot)$, it is straightforward to construct a value oracle for $v'(t,\cdot)$.
 		 If we are given a demand oracle for $v(t,\cdot)$, here is how to construct a demand oracle for $v'(t,\cdot)$. 
 		 For every queried price vector $\{p_j\}_{j\in[m]}$, we change the price for each item outside $\CC_{\eta}(t)$ to $2v(t,[m])$ and keep the prices for the items in $\CC_{\eta}(t)$. 
 		 Let this new price vector be $p'$. We query the demand oracle of $v(t,\cdot)$ on $p'$. 
 		 The output set should also be the demand set for $v'(t,\cdot)$ under prices $p$, as the bidder can only afford items in $\CC_{\eta}(t)$ and $v'(t,S)=v(t,S)$ for any set $S\subseteq \CC_{\eta}(t)$. 
 		 Finally, we consider the XOS oracle. %We say an additive function supports a set if it is the additive function that has the highest value on that set. 
 		 For any set $S$, let $\left\{\theta^{S\cap \CC_\eta(t)}_j(t)\right\}_{j\in {S\cap \CC_\eta(t)}}$ be the supporting prices for %additive function supporting set ${S\cap \CC_\eta(t)}$
 		   $v(t,{S\cap \CC_\eta(t)})$. % That is $\sum_{j\in{S\cap \CC_\eta(t)}} \theta^{S\cap \CC_\eta(t)}_j=v(t,S\cap \CC_\eta(t))$. According to our construction of the additive functions for $v'(t,\cdot)$, the additive function that supports set $S$ for $v'(t,\cdot)$ has value
 		   Let $\gamma^{S}_j(t)=\theta^{S\cap \CC_\eta(t)}_j$ for all item $j$ in $\CC_\eta(t)\cap S$ and  $\gamma^{S}_j(t)=0$ for all item $j$ in $S-\CC_\eta(t)$. According to the definition of $v'(t,\cdot)$, $\{\gamma_j^S(t)\}_{j\in S}$ is the supporting price for $v'(t,S)$. So given an XOS oracle for $v(t,\cdot)$, we can compute the supporting price of any set $S$ for $v'(t,\cdot)$ in polynomial time.
 	\end{proof}	
 	
 	Lemma~\ref{lem:truncated still XOS and oracle} shows that $v'(t,\cdot)$ is also an XOS function for any type $t$ and with access to a demand oracle for $v(t,\cdot)$  we can construct a demand oracle for $v'(t,\cdot)$ in polynomial time.
 	 So we can indeed run Feige's algorithm on $v'$. In the next Lemma, we show how to learn a collection of prices $\{Q_j\}_{j\in[m]}$ and entry fee function $\delta(\cdot,\cdot)$ such that the corresponding ASPE has high revenue.
 	
\begin{lemma}\label{lem:symmetric XOS learning prices}
Given a collection of $b$-balanced thresholds $\{\beta_j\}_{j\in[m]}$ and $c_\eta(\bbeta)$, and access to value, demand and XOS oracles for valuation $v(t,\cdot)$ for every type $t$, there is a polynomial time algorithm that learns an ASPE$(\{Q_j\}_{j\in[m]},\delta)$ whose revenue is at least  $\frac{\core_{\eta}(\bbeta)}{\KK_1}-g(b,\eta)\cdot \prev-\KK_2\cdot \xi\cdot \opt$ with probability at least $1-\zeta$ using $O\left(n^3(m+n)^2\log \frac{m}{\zeta}\right)$ samples from $\times_{j\in[m]} D_j$, where $\KK_1$ and $\KK_2$ are positive absolute constants, and $g(b,\eta)$ is a function that only depends on $b$ and $\eta$.
\end{lemma}
\begin{proof}
	According to Lemma~\ref{lem:truncated still XOS and oracle}, we can construct value, demand and XOS oracles for valuation $v'(t,\cdot)$ given access to the corresponding oracles for $v(t,\cdot)$. We use $\{\gamma_j^S(t)\}_{j\in S}$ to denote the output of the XOS oracle for $v'(t,\cdot)$ on set $S$. In particular, $\gamma_j^S(t)=0$ for all $j\in S-\CC_\eta(t)$ and $\gamma_j^S(t)=\theta_j^{S\cap\CC_\eta(t)}(t)$ for all $j\in S\cap\CC_\eta(t)$, where $\{\theta_j^{S\cap\CC_\eta(t)}(t)\}_{j\in {S\cap\CC_\eta(t)}}$ is the supporting prices for $v(t,{S\cap\CC_\eta(t)})$. 
	Let $\AA(\boldsymbol{t})$ be the allocation computed by Feige's algorithm on the valuation profile $\left(v'(t_1,\cdot),\ldots,v'(t_n,\cdot)\right)$, where $\AA_i(\boldsymbol{t})$ denotes the set of items that bidder $i$ receives. 
	Let $\sigma^*$ be the interim allocation rule induced by $\AA(\cdot)$ when bidders types are all drawn from $\times_{j\in[m]} D_j$ independently. 
	That is, $\sigma^*_{iS}(t_i)=\Pr_{t_{-i}}\left[\AA_i(\boldsymbol{t})=S\right]$. 
	We use the same definition for $Q_{\eta,j}$ as in Lemma~\ref{lem:Q_j}. 
	In other words, $Q_{\eta,j}$ is the contribution of item $j$ to the social welfare under allocation rule $\sigma^*$, so we can rewrite it as $$\frac{1}{2}\cdot\E_{\boldsymbol{t}}\left[\sum_{i\in[n]} \ind\left[j\in \AA_i(\boldsymbol{t})\right]\cdot \gamma_j^{\AA_i(\boldsymbol{t})}(t_i)\right].$$ %We use $q_j$ to denote the random variable $\sum_{i\in[n]} \ind\left[j\in \AA_i(\boldsymbol{t})\right]\cdot \gamma_j^{\AA_i(\boldsymbol{t})}(t_i)$.
	
	Let $\boldsymbol{t^{(1)}},\ldots, \boldsymbol{t^{(K)}}$ be $K$ sampled type profiles, and $q^{(\ell)}=\frac{1}{2}\sum_{i\in[n]} \ind\left[j\in \AA_i(\boldsymbol{t^{(\ell)}})\right]\cdot \gamma_j^{\AA_i(\boldsymbol{t^{(\ell)}})}(t^{(\ell)}_i)$. 
	We set $Q_j$ to be $\frac{1}{K}\cdot\sum_{\ell\in[K]} q^{(\ell)}$. Since $\gamma^S_j(t)\leq \beta_j+c_\eta(\bbeta)$ for any $j$, $S$ and $t$, $Q_{j}\leq \beta_j+c_\eta(\bbeta)$. By the Chernoff bound, 
	
	$$\Pr\left[\left|Q_j-Q_{\eta,j}\right|\leq \epsilon\cdot \left(\beta_j+c_\eta(\bbeta)\right) \right]\geq 1-2\exp(-2K\cdot \epsilon^2).$$
	
	As $\{\beta_j\}_{j\in[m]}$ is a collection of $b$-balanced thresholds, we can obtain revenue $\beta_j\cdot \frac{b}{n}$ by only selling item $j$ to one bidder at price $\beta_j$. 
	Hence,  $\beta_j\leq \frac{n\cdot\opt}{b}$. 
	Now, consider a posted price mechanism that sells item $j$ at price $\beta_j+c_\eta(\bbeta)$. 
	A single bidder will purchase at least one item with probability at least $\sum_j \Pr_{t_{j}\sim D_{j}}\left[V(t_{j})\geq \beta_{j}+c_\eta(\boldsymbol{\beta})\right]$ which is no less than  $\frac{1}{2}-\eta$ if $c_\eta(\bbeta)>0$. 
	Hence, the revenue of this mechanism is at least $c_\eta(\bbeta)\cdot \left(\frac{1}{2}-\eta\right)$. 
	As $\eta\leq \frac{1}{4}$, $c_\eta(\bbeta)\leq 4\opt$. 
	If we let $\epsilon = \frac{\xi}{(m+n)\cdot\left(n/b+4\right)}$ for some small constant $\xi$ which will be specified later and $K =\frac{\log \frac{4m}{\zeta}}{2\epsilon^2}$, we have $\Pr\left[\left|Q_j-Q_{\eta,j}\right|\leq \frac{\xi}{m+n}\cdot\opt \right]\geq 1-\frac{\zeta}{2m}$. 
	In other words, with $O\left(n^3(m+n)^2\log \frac{m}{\zeta}\right)$ samples from $\times_{j\in[m]} D_j$ (as each $\boldsymbol{t}^{\ell}$ costs $n$ samples), we can learn in polynomial time a collection of prices $\{Q_j\}_{j\in [m]}$ such that $\left|Q_j-Q_{\eta,j}\right|\leq \frac{\xi}{m+n}\cdot\opt $ for all item $j$ with probability $1-\zeta/2$.
	
	Next, we consider the entry fee function. We use essentially the same argument as in Lemma~\ref{lem:learn median}. Suppose we take
$L$ samples $t^{(1)},\cdots, t^{(L)}$ from $\times_{j\in[m]} D_j$. Define the entry fee $\delta(S)$
 for set $S$ under $\{Q_j\}_{j\in[m]}$ to be the median of $u(t^{(1)},S),\cdots, u(t^{(L)},S)$, where $u(t,S)=\max_{S*\subseteq S} v(t,S^*)-\sum_{j\in S^*} p_j$. Given any constant $\mu\in[0,1/4]$, for any fixed set $S$, it is easy to argue that the probability for $\Pr_{t\sim \times_{j\in[m]} D_j}[u(t,S)\geq \delta(S)]$ to be larger than  $\frac{1}{2}+\mu$ or less than  $\frac{1}{2}-\mu$ is at most $2\exp(-2L\mu^2)$ due to the Chernoff bound. If we let $L$ to be $a\cdot \frac{m+\log 1/\zeta}{\mu^2}$ for a sufficiently large constant $a$, the probability that $\delta(\cdot)$ is a $\mu$-balanced entry fee function is at least $1-\zeta/2$ by the union bound.
 
 Hence, with $O\left(n^3(m+n)^2\log \frac{m}{\zeta}\right)$ samples from  $\times_{j\in[m]} D_j$, we can compute in polynomial time a collection of prices $\{Q_j\}_{j\in[m]}$ and a entry fee function $\delta(\cdot)$ such that the revenue of the ASPE$\left(\{Q_j\}_{j\in[m]},\delta(\cdot)\right)$ is at least $\frac{(1-1/e)\cdot\core_{\eta}(\bbeta)}{\BB_{1}(\mu)}-\BB_{2}(b,\eta,\mu)\cdot \prev-\xi\cdot\BB_3(\mu)\cdot \opt$ with probability $1-\zeta$ due to Lemma~\ref{lem:symmetric approx ASPE}. Our claim follows by fixing the value of $\mu$ to be some constant.
	\end{proof}

\begin{theorem}\label{thm:symmetric XOS}
	For symmetric bidders with valuations that are XOS over independent items, \begin{enumerate}
		\item when $V(t_j)$ is upper bounded by $H$ for any $j\in[m]$ and any $t_j$, with $$O\left(\left(n^5+m^2n^4\right)\cdot\log\frac{m}{\delta}+\left(\frac{1}{\epsilon}\right)^2 \left(m^2 n\log n\log \frac{1}{\epsilon} + \log \frac{1}{\delta}\right)\right)$$ samples from $\times_{j\in[m]} D_j$, we can learn in polynomial time with probability $1-\delta$ a mechanism whose revenue is at least $c_1\cdot \opt-\epsilon\cdot H$ for some absolute constant $c_1$;
		\item when the distribution of random variable $V(t_j)$ with $t_j\sim D_j$ is regular for all item $j\in[m]$, with %$$O\left(\left(n^5+m^2n^4\right)\cdot\log\frac{m}{\delta}+\max\{m,n\}^2m^2 n^2\cdot \left(\log n +\log m + \log \frac{1}{\delta}\right)\right)$$
		$$O\left(n^5\cdot\log\frac{m}{\delta}+\max\{m,n\}^2m^2 n^2\cdot  \log \frac{nm}{\delta}\right)$$ samples from $\times_{j\in[m]} D_j$, we can learn in polynomial time with probability $1-\delta$ a mechanism whose revenue is at least $c_2\cdot \opt$ for some absolute constant $c_2$.
	\end{enumerate}
\end{theorem}
\begin{prevproof}{Theorem}{thm:symmetric XOS}
Combining Lemma~\ref{lem:symmetric XOS learn beta and c}, Lemma~\ref{lem:symmetric XOS learning prices} and Theorem~\ref{thm:UB subadditive}, we know how to compute in polynomial time an ASPE whose revenue is at least $a_1\cdot \opt-a_2\cdot\prev$ with probability $1-\delta/2$ for some absolute constant $a_1$, $a_2$, and we only need $O\left(\left(n^5+m^2n^4\right)\cdot\log\frac{m}{\delta}\right)$ samples from $\times_{j\in[m]} D_j$. When the distributions are bounded, we can learn in polynomial time  an RSPM whose revenue is at least $\frac{\prev}{144}-\xi H$ with probability $1-\delta/2$ using $O\left(\left(\frac{1}{\xi}\right)^2 \left(m^2 n\log n\log \frac{1}{\epsilon} + \log \frac{1}{\delta}\right)\right)$ samples (Theorem~\ref{thm:UD bounded}). By choosing the ratio between $\xi$ and $\epsilon$ to be the right constant, we can show the first part of our claim. When $V(t_j)$ is a regular random variable for every item $j$, we can learn in polynomial time  an RSPM whose revenue is at least $\frac{\prev}{33}$ with probability $1-\delta/2$ using  $O\left(\max\{m,n\}^2m^2 n^2\cdot \log \frac{nm}{\delta}\right)$ samples (Theorem~\ref{thm:UD regular}). Therefore, we can learn a mechanism in polynomial time such that with probability $1-\delta$ whose revenue is at least a constant fraction of the $\opt$. This proves the second part of our claim.\end{prevproof}


\subsection{Symmetric Bidders with Subadditive Valuations}\label{sec:symmetric subadditive}
In this section, we argue that if the bidders are symmetric and $m=O(n)$, there exists a collection of $b$-balanced thresholds $\{\beta_j\}_{j\in[m]}$ for a fixed constant $b$, such that $\prev$ is within a constant fraction of the benchmark. Note that this argument only applies to symmetric bidders, as $b$-balanced thresholds may not even exist for asymmetric bidders.  %To overcome this difficulty, Cai and Zhao~\cite{CaiZ17} used a more sophisticated way to choose the thresholds in order to obtain a manageable benchmark for the optimal revenue, and then they used not only the $\prev$ but also the revenue of some ASPE to upper bound this benchmark.


We set $\eta=0$ for this section and drop the subscript $\eta$ when there is no confusion. %use $\core(\bbeta)$ to denote $\core_{0}(\bbeta)$.
 We show how to upper bound $\core(\bbeta)$ with $\prev$ by choosing a particular collection of $b$-balanced thresholds. Let $Z=\max\{m,n\}$ and $b = \frac{n}{3Z}$. It is not hard to see that $\frac{n}{3Z}$-balanced thresholds exist, as we can choose $\beta_j$ such that $\Pr_{t_j\sim D_j}[V(t_j)\geq \beta_j]=\frac{1}{3Z}$. %It is not hard to verify that our choice of $\{\beta_j\}_{j\in[m]}$ is indeed $\frac{n}{2Z}$-balanced. 

\begin{lemma}\label{lem:bounding core with beta}
	Let $\{\beta_j\}_{j\in[m]}$ be a collection of $\frac{n}{3Z}$-balanced thresholds, then $\core(\bbeta)\leq \sum_{j\in[m]} \beta_j$.
	\end{lemma}
\begin{proof}


As $\{\beta_j\}_{j\in[m]}$ are $\frac{n}{3Z}$-balanced, $\Pr_{t_{ij}\sim D_j}[V(t_{ij})\geq \beta_j]\leq \frac{n}{(n-1)\cdot 3Z}\leq \frac{1}{2Z}$. Therefore, $$\sum_{j\in[m]} \Pr_{t_{j}\sim D_{j}}\left[V(t_{j})\geq \beta_{j}\right]\leq \frac{1}{2},$$ so $c(\bbeta) = 0$. Next, we upper bound $\core(\bbeta)$ by $\sum_{j\in[m]} \beta_j$.

	\begin{align*}
		\core(\bbeta) = &\max_{\sigma \in P(D)} \sum_{i\in[n]}\sum_{t_i\in T_i}f(t_i)\cdot \sum_{S\subseteq[m]}\sigma_{iS}(t_i)\cdot v(t_i,S\cap \CC(t_i))\\
		\leq &\max_{\sigma \in P(D)} \sum_{i\in[n]}\sum_{t_i\in T_i}f(t_i)\cdot \left(\sum_{S\subseteq[m]}\sigma_{iS}(t_i)\cdot \sum_{j\in S} \beta_j\right)\\
		= &\max_{\sigma \in P(D)} \sum_{i\in[n]} \sum_{j\in[m]} \beta_j\cdot \left(\sum_{t_i\in T_i}f(t_i)\cdot \sum_{S: j\in S}\sigma_{iS}(t_i)\right)\\
		= &\max_{\sigma \in P(D)} \sum_j \beta_j\cdot  \left(\sum_{i\in[n]} \sum_{t_i\in T_i}f_i(t_i)\cdot \sum_{S: j\in S}\sigma_{iS}(t_i)\right)\\
		\leq & \sum_j \beta_j
	\end{align*}
	
	The first inequality is because $v(t_i,\cdot)$ is a subadditive function, so $$v(t_i,S\cap \mathcal{C}_i(t_i))\leq \sum_{j\in S\cap \mathcal{C}_i(t_i)} V(t_{ij})\leq \sum_{j\in S\cap \mathcal{C}_i(t_i)} \beta_j \leq \sum_{j\in S} \beta_j.$$ The last inequality is because $\sum_{i\in[n]}\sum_{t_i\in T_i}f_i(t_i)\cdot \sum_{S: j\in S}\sigma_{iS}(t_i)\leq 1$ is the ex-ante probability for bidder $i$ to receive item $j$, and for any feasible interim allocation $\sigma$, the sum of all bidders' ex-ante probabilities for receiving item $j$ should not exceed $1$.
\end{proof}

In the following Lemma, we demonstrate that $\sum_{j\in [m]} \beta_j$ is upper bounded by $\frac{9Z}{n}\cdot \prev$.

\begin{lemma}\label{lem:bounding beta with prev}
		Let $\{\beta_j\}_{j\in[m]}$ be a collection of $\frac{n}{3Z}$-balanced thresholds, $\prev\geq \frac{n}{9Z}\cdot \sum_{j\in [m]} \beta_j$.
\end{lemma}

\begin{proof}
	Let us consider an RSPM where the price for selling item $j$ to bidder $i$ is $\beta_j$. Bidder $i$ purchases item $j$ if that is the only item she can afford and no one else can afford item $j$. As $\{\beta_j\}_{j\in[m]}$ are $\frac{n}{3Z}$-balanced, the probability that no one else can afford item $j$ is at least $$\left(1-\sum_{k\neq i}\Pr_{t_{kj}\sim D_j}[V(t_{kj})\geq \beta_j]\right)\geq (1-\frac{n}{3Z})\geq \frac{2}{3}.$$ Also, the probability that $i$ cannot afford any item other than $j$ is at least $$\left(1-\sum_{\ell\neq j}\Pr_{t_{i\ell}\sim D_\ell}[V(t_{i\ell})\geq \beta_\ell]\right)\geq 1-\frac{n(m-1)}{3Z(n-1)}\geq\frac{1}{2}.$$ Therefore, bidder $i$ purchases item $j$ with probability at least $\frac{1}{3}\Pr_{t_{ij}\sim D_j}[V(t_{ij}\geq \beta_j)]\geq \frac{1}{9Z}$. Whenever this event happens, it contributes $\beta_j$ to the revenue. So the total revenue is at least $\sum_j\sum_i\frac{\beta_j}{9Z}= \frac{n}{9Z}\cdot \sum_j \beta_j$.
\end{proof}

Combining Theorem~\ref{thm:UB subadditive}, Lemma~\ref{lem:bounding core with beta} and~\ref{lem:bounding beta with prev}, we obtain the following Theorem.

\begin{theorem}\label{thm:subbadditive UB RSPM}
	For symmetric bidders with valuations that are subadditive over independent items, $$\opt\leq \left(24+\frac{36 \max\{n,m\}}{n}\right)\cdot \prev.$$ \end{theorem}
\begin{proof}
	Combining Lemma~\ref{lem:bounding core with beta} and~\ref{lem:bounding beta with prev}, we have $\prev\geq \frac{n}{9\max\{n,m\}}\cdot \core(\bbeta)$ if $\{\beta_j\}_{j\in[m]}$ is a collection of $\frac{n}{3\max\{n,m\}}$-balanced thresholds. By setting $b$ to be $\frac{n}{3\max\{n,m\}}$ and replacing $\core(\bbeta)$ with $\frac{9\max\{n,m\}}{n}\cdot \prev$ in Theorem~\ref{thm:UB subadditive}, we have $$\opt\leq \left(12+\frac{8}{1-\frac{n}{3\max\{n,m\}}}+\frac{36 \max\{n,m\}}{n}\right)\cdot \prev.$$ As $\frac{n}{3\max\{n,m\}}\leq 1/3$, $$\opt\leq \left(24+\frac{36 \max\{n,m\}}{n}\right)\cdot \prev.$$
\end{proof}

%Theorem~\ref{thm:subbadditive UB RSPM} shows that there exists an RSPM whose revenue is within a constant factor of the optimal revenue when $m=O(n)$. As the RSPM only sells 

\subsubsection{Learning an Approximately Optimal Mechanism for Symmetric Subadditive Bidders}

With Theorem~\ref{thm:subbadditive UB RSPM}, we only need to learn a mechanism that approximates the optimal revenue obtainable by any RSPM. The next Lemma connects RSPMs with SPMs in an induced unit-demand setting.
\begin{lemma}\label{lem: RSPM to UD}
	Consider $n$ symmetric bidders whose types are drawn independently from $\times_{j=1}^m D_j$. Let $\FF_j$ be the distribution for random variable $V(t_j)$ where $t_j\sim D_j$. We define an \emph{induced unit-demand} setting with  $n$ symmetric unit-demand bidders whose values for item $j$ are drawn independently from $\FF_j$. For any collection of prices $\{p_{ij}\}_{i\in[n],j\in[m]}$, the revenue of the RSPM with these prices in the original setting is exactly the same as the revenue of the SPM with these prices in the induced unit-demand setting.
\end{lemma}
\begin{proof}
As in an RSPM bidders can purchase at most one item, bidders behave exactly the same as in the induced unit-demand setting. Since the prices in the SPM and RSPM are the same, bidders purchase exactly the same items. Hence, the revenue is the same.\end{proof}

\begin{corollary}\label{cor:subadditive to UD}
For symmetric bidders with valuations that are subadditive over independent items, $$\opt^{UD}\geq \Omega\left(\frac{n}{Z}\right)\cdot\opt, $$ where $Z=\max\{m,n\}$ and $\opt^{UD}$ is the optimal revenue for the induced unit-demand setting.\end{corollary}
\begin{proof}
	Combine Theorem~\ref{thm:subbadditive UB RSPM} and Lemma~\ref{lem: RSPM to UD}.
\end{proof}

Lemma~\ref{lem: RSPM to UD} implies that learning an approximately optimal RSPM is equivalent as learning an approximately optimal SPM in the induced unit-demand setting. Next, we apply our results in Section~\ref{sec:unit-demand} to the induced unit-demand setting to learn an RSMP that approximates the optimal revenue in the original setting.

 %Given any setting with symmetric bidders with subadditive valuations over independent items, we construct a setting with symmetric unit-demand bidders. In particular, there are $n$ symmetric unit-demand bidders and $m$ items. Bidder $i$'s value for item $j$ is $V(t_{ij})$ with $t_{ij}\sim D_j$. It is not hard to see that any RSPM achieves exactly the same expected revenue in the original setting and in the corresponding unit-demand setting. So if we can learn an RSPM that approximates the optimal revenue in the unit-demand setting, that mechanism automatically approximates the optimal revenue in the original subadditive setting. Notice that an RSPM is the same as an SPM when bidders are unit-demand, so our results in Section~\ref{sec:unit-demand} directly apply here. %Let $\opt^{UD}$ be the optimal revenue in the corresponding unit-demand setting, then $\prev\leq \opt^{UD}$. 

In the next Theorem, we show that even though the bidders' valuations could be complex set functions, e.g., submodular, XOS and subadditive, as long as $m=O(n)$, the approximate distributions for the bidders' values for winning any \emph{single item} provides sufficient information to learn an approximately optimal mechanism. 

\begin{theorem} \label{thm:subadditive Kolmogorov}
	For symmetric bidders with valuations that are subadditive over independent items, let $\FF_j$ be the distribution of $V(t_j)$ where $t_j\sim D_j$. If $\FF_j$ is supported on $[0,H]$ for all $j\in[m]$, given distributions $\hat{\FF}_{j}$ where $\left|\left|\hat{\FF}_{j}-\FF_{j}\right|\right|_K\leq \epsilon$ for all $j\in[m]$, there is a polynomial time algorithm that constructs a randomized RSPM whose revenue under the true distribution $D$ is at least  $$\left(\frac{1}{4}-(n+m)\cdot \epsilon\right)\cdot\left(\Omega\left(\frac{n}{\max\{m,n\}}\right)\cdot \opt-2\epsilon\cdot mnH\right).$$
\end{theorem}
\begin{proof}
 Let $Z=\max\{m,n\}$. According to Corollary~\ref{cor:subadditive to UD}, $\opt^{UD}=\Omega\left(\frac{n}{Z}\right)\cdot \opt$. Since $||\hat{\FF}_{j}-\FF_{j}||_K\leq \epsilon$ for all $j\in[m]$, we can learn a randomized SPM in the induced unit-demand setting whose revenue under the true distribution is at least $\left(\frac{1}{4}-(n+m)\cdot \epsilon\right)\cdot\left(\frac{\opt^{UD}}{8}-2\epsilon\cdot mnH\right)$ based on Theorem~\ref{thm:UD Kolmogorov}.  By Lemma~\ref{lem: RSPM to UD},  we can construct an RSPM with the same collection of (randomized) prices and achieve revenue $$\left(\frac{1}{4}-(n+m)\cdot \epsilon\right)\cdot\left(\Omega\left(\frac{n}{Z}\right)\cdot \opt-2\epsilon\cdot mnH\right)$$ in the original setting.\end{proof}

If we are given sample access to bounded distributions, we show in the following Theorem that a polynomial number of samples suffices to learn an approximately optimal mechanism, when $m=O(n)$.

\begin{theorem}\label{thm:subadditive bounded}
	For symmetric bidders with valuations that are subadditive over independent items, let $\FF_j$ be the distribution of ~$V(t_j)$ where $t_j\sim D_j$. If $\FF_j$ is supported on $[0,H]$ for all $j\in[m]$, there is a polynomial time algorithm that learns  an RSPM whose revenue is $\Omega\left(\frac{n}{{\max\{m,n\}}}\right)\cdot \opt -\epsilon H$ with probability $1-\delta$ using $$O\left(\left(\frac{1}{\epsilon}\right)^2\cdot \left(m^2 n\log n\log \frac{1}{\epsilon} + \log \frac{1}{\delta}\right)\right)$$ samples.% suffices to learn in polynomial time with probability $1-\delta$ an RSPM whose revenue is $\Omega\left(\frac{n}{{\max\{m,n\}}}\right)\cdot \opt -\epsilon H$. %\yangnote{With the same number of samples, we can learn in polynomial time with probability $1-\delta$ an SPM whose revenue is at least $\frac{\opt}{144}-\epsilon H$.}
\end{theorem}
\begin{proof}
	According to Corollary~\ref{cor:subadditive to UD}, $\opt^{UD}=\Omega\left(\frac{n}{{\max\{m,n\}}}\right)\cdot \opt$. Due to Theorem~\ref{thm:UD bounded}, $$O\left(\left(\frac{1}{\epsilon}\right)^2\cdot \left(m^2 n\log n\log \frac{1}{\epsilon} + \log \frac{1}{\delta}\right)\right)$$ samples suffices to learn in polynomial time with probability $1-\delta$ an SPM with revenue at least $\Omega(\opt^{UD})-\epsilon\cdot H$ for the induced unit-demand setting. By Lemma~\ref{lem: RSPM to UD},  we can construct an RSPM with the same collection of prices and achieve revenue $\Omega\left(\frac{n}{{\max\{m,n\}}}\right)\cdot \opt -\epsilon H$ in the original setting. \end{proof}
	
Finally, if the distribution of the random variable $V(t_j)$ with $t_j\sim D_j$ is regular for all item $j\in[m]$, we prove in the next theorem that there exists a prior-independent mechanism that achieves a constant fraction of the optimal revenue if $m=O(n)$. Note that approximately optimal prior-independent mechanisms for symmetric unit-demand bidders are known due to the work by Devanur et al.~\cite{DevanurHKN11} and Roughgarden et al.~\cite{RoughgardenTY12}. Our result is obtained by combining Theorem~\ref{thm:subbadditive UB RSPM} and the afore-mentioned prior independent mechanisms.

\begin{theorem}\label{thm:subadditive regular}
	For symmetric bidders with valuations that are subadditive over independent items, let $\FF_j$ be the distribution of ~$V(t_j)$ where $t_j\sim D_j$. If $\FF_j$ is regular for all $j\in[m]$, there is a prior-independent mechanism with revenue at least $\Omega\left(\frac{n}{{\max\{m,n\}}}\right)\cdot \opt$. Moreover, the mechanism can be implemented efficiently.%\yangnote{With the same number of samples, we can learn in polynomial time with probability $1-\delta$ an SPM whose revenue is at least $\frac{\opt}{144}-\epsilon H$.}
\end{theorem}
\begin{proof}
	The mechanism in~\cite{DevanurHKN11} or~\cite{RoughgardenTY12} provides an approximately optimal prior-independent mechanism in the induced unit-demand setting. Let us use $M$ to denote this mechanism. Suppose we restrict every bidder to purchase at most one item in the original setting and then run mechanism $M$. The expected revenue is the same as $M$'s expected revenue in the induced setting. Since $M$'s expected revenue is  $\Omega(\opt^{UD})$ and $\opt^{UD}=\Omega\left(\frac{n}{{\max\{m,n\}}}\right)\cdot \opt$, the mechanism we constructed has revenue $\Omega\left(\frac{n}{{\max\{m,n\}}}\right)\cdot \opt$. Since $M$ can be implemented efficiently for unit-demand bidders, our mechanism can also be implemented efficiently.\end{proof}
