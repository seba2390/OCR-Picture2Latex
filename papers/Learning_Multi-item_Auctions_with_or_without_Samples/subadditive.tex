\section{Symmetric Bidders}\label{sec:symmetric bidders}
In this section, we consider symmetric bidders ($D_i=D_{i'}$ for all $i$ and $i'\in[n]$) with XOS and subadditive valuations. For XOS valuations, our goal is to improve our algorithms from Section~\ref{sec:constrained additive} to be computationally efficient under bidder symmetry. For subadditive valuations, our goal is to establish the learnability of approximately optimal mechanisms whose revenue improves as the number of bidders becomes comparable to the number of items. We only describe the results here and postpone the formal statements and proofs to Appendix~\ref{sec:symmetric appx}.
 \begin{itemize}
 	\item  \textbf{XOS valuations:} we can learn in polynomial time an approximately optimal mechanism with a polynomial number of samples when the valuations are XOS over independent items. {Our algorithm essentially estimates all the parameters needed to run the RSPM and ASPE used in~\cite{CaiZ17}. In general, it is not clear how to estimate these parameters efficiently. But when the bidders are symmetric, one only needs to consider ``symmetric parameters'' which greatly simplifies the search space and allows us to estimate all the parameters in polynomial time. }See Appendix~\ref{sec:symmetric XOS} for details.
 	\item \textbf{subadditive valuations:} when the valuations are subadditive over independent items,  the optimal revenue is at most $O\left(\frac{n}{\max\{m,n\}}\right)$ times larger than the highest revenue obtainable by an RSPM. In other words, if the number of items is within a constant times the number of bidders, an RSPM suffices to extract a constant fraction of the optimal revenue.  Applying our results for unit-demand bidders in Section~\ref{sec:unit-demand}, we can learn a nearly-optimal RSPM, which is also a good approximation to $\opt$. In fact, when the distribution for random variable $V_i(t_{ij})$ is regular for every bidder $i$ and item $j$, we can design a prior-independent mechanism that achieves a constant fraction of the optimal revenue. See Appendix~\ref{sec:symmetric subadditive} for details.
 \end{itemize}
 
