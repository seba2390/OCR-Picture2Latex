\subsection{Unit-demand Valuations: Polynomial-Time Learning} \label{sec:unit-demand}

In this section, we consider bidders with unit-demand valuations, sharpening our results to show how to learn approximately revenue-optimal mechanisms in polynomial time. It is shown in a sequence of works~\cite{ChawlaHMS10, KleinbergW12, CaiDW16} that there exists a sequential posted price mechanism (\textbf{SPM} see Algorithm~\ref{alg:seq-mech} for details) that achieves at least $\frac{1}{24}$ of the optimal revenue when bidders are unit-demand. We show that under all three distribution access models of Section~\ref{sec:prelim} there exists a polynomial-time algorithm that learns a sequential posted price mechanism whose revenue approximates the optimal revenue. We only sketch the proof here and postpone the details to Appendix~\ref{sec:unit-demand appx}.

\begin{theorem}\label{thm:unit-demand}
	When all bidders have unit-demand valuations and \begin{itemize}
		\item $D_{ij}$ is supported on $[0,H]$ for all bidder $i$ and item $j$, there exists a polynomial time algorithm that learns an SPM whose revenue is at least $\frac{\opt}{144}-\epsilon H$ with probability $1-\delta$ given  $O\left(\left(\frac{1}{\epsilon}\right)^2 \left(m^2 n\log \frac{n}{\epsilon} + \log \frac{1}{\delta}\right)\right)$ samples from $D$; or
		\item $D_{ij}$ is a regular distribution for all bidder $i$ and item $j$, there exists a polynomial time algorithm that learns a randomized SPM whose revenue is at least $\frac{\opt}{33}$ with probability $1-\delta$ given $O(\max\{m,n\}^2m^2 n^2\cdot \log \frac{nm}{\delta})$ samples from $D$; or
		\item we are only given access to $\hat{D}_{ij}$ where $||\hat{D}_{ij}-D_{ij}||_K\leq \epsilon$ for all bidder $i$ and item $j$, there is a polynomial time algorithm that constructs a randomized SPM whose revenue under $D$ is at least $\left(\frac{1}{4}-(n+m)\cdot \epsilon\right)\cdot\left(\frac{\opt}{8}-2\epsilon\cdot mnH\right)$\footnote{If we set $\epsilon$ to be $O(\frac{1}{m+n})$, this is the max-min guarantee we want to achieve.}.
	\end{itemize}
\end{theorem}

\noindent\textbf{Sample Access to Bounded Distributions:} the result is due to Morgenstern and Roughgarden~\cite{MorgensternR16}. 

\vspace{.05in}
\noindent\textbf{Direct Access to Approximate Distributions:} we first consider a convex program based on $D$ (see Figure~\ref{fig:CP unit demand}) which is usually referred to as the ex-ante relaxation of the revenue maximization problem~\cite{Alaei11}, and use its optimum as a proxy for $\opt$. Next, we consider a similar convex program based on $\hat{D}$ (see Figure~\ref{fig:CP unit demand approximate dist}) and show that the optima of the two convex programs are close to each other. Finally, we use techniques developed by Chawla et al.~\cite{ChawlaHMS10} to convert the optimal solution of the second convex program into a randomized SPM. We can show that the constructed randomized SPM achieves a revenue that approximates the optimum of the second convex program under $D$, which implies that the mechanism's revenue  also approximates the $\opt$. As we are given $\hat{D}$, we can solve the second convex program and convert its optimal solution into a randomized SPM in polynomial time. See Theorem~\ref{thm:UD Kolmogorov} in Appendix~\ref{sec:unit-demand Kolmogorov} for further details.

\vspace{.05in}
\noindent\textbf{Sample Access to Regular Distributions:} we use a similar convex program relaxation based approach as in the previous case. The main difference is that regular distributions could be unbounded and thus ruin the approximation guarantee. We show how to use the Extreme Value theorem in~\cite{CaiD11b} to truncate the distributions without hurting the revenue by much. See Theorem~\ref{thm:UD regular} in Appendix~\ref{sec:unit-demand regular} for further details.