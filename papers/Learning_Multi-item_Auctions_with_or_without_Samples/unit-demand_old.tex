\section{Unit-demand Valuations} \label{sec:unit-demand}

In this section, we show how to learn approximately revenue-optimal mechanisms for unit-demand valuations. It is known that there always exists a sequential posted price mechanism (SPM) that achieves at least $\frac{1}{24}$ of the optimal revenue~\cite{ChawlaHMS10, KleinbergW12, CaiDW16}. Our plan is to show that under all three models, we can learn a good sequential posted price mechanism that approximates the optimal revenue well.

\begin{algorithm}[ht]
\begin{algorithmic}[1]
\REQUIRE $P_{ij}$ is the price for bidder $i$ to purchase item $j$.
\STATE $S\gets [m]$
\FOR{$i \in [n]$}
	\STATE Show bidder $i$ {the} set of available items $S$.
	 \STATE $i$ purchases her favorite bundle $S_i^{*}\in \max_{S'\subseteq S} v_i(t_i, S') - \sum_{j\in S'} P_{ij}$ and pays $\sum_{j\in S_i^{*}}P_{ij}$.
        \STATE $S\gets S\backslash S_i^{*}$.
\ENDFOR
\end{algorithmic}
\caption{{\sf Sequential Posted Price Mechanism (SPM)}}
\label{alg:seq-mech}
\end{algorithm} 

\subsection{Unit-demand Valuation: direct access to approximate distributions}\label{sec:unit-demand Kolmogorov}
We first consider the model where we only have access to an approximate distribution $\hat{D}$. The following definition is crucial for proving our result. 

\begin{definition}
For any single dimensional distribution $\DD$ with cdf $F$, we define its revenue curve $R_{\DD}: [0,1]\mapsto \mathbb{R}_{\geq 0}$ as
	\begin{align*}
	{R}_{\DD} (q) = &\max  x\cdot \ubar{q}\cdot {F}^{-1}(1-\ubar{q}) +(1-x)\cdot \bar{q}\cdot {F}^{-1}(1-\bar{q})\\
	& \qquad\textbf{s.t. } x\cdot \ubar{q}+(1-x)\cdot \bar{q}=q\\
	&  \qquad\qquad x, \ubar{q}, \bar{q} \in [0,1]
\end{align*}
where $F^{-1}(1-p) = \sup\{x\in R: \Pr_{v\sim \DD}[v\geq x]\geq p\}$.
\end{definition}


% For any bidder $i$ and item $j$, define function $R_{\hat{D}_{ij}}: [0,1]\rightarrow \mathbb{R}_{\geq 0}$ as:
%We define $R_{\hat{D}_{ij}}$ similarly with respect to $\hat{F}_{ij}$.
%It is not hard to argue that $R_{\hat{D}_{ij}}(\cdot)$ is exactly the ironed revenue curve for distribution $\hat{D}_{ij}$.

\begin{lemma}
	Let ${\varphi}_{ij}(\cdot)$ and $\hat{\varphi}_{ij}(\cdot)$ be the ironed virtual value function for distribution $D_{ij}$ and $\hat{D}_{ij}$ respectively, then for any $q\in [0,1]$, ${R}_{D_{ij}}(q) = \int_{{F}^{-1}_{ij}(1-q)}^H {\varphi}(x) dF(x)$ and $R_{\hat{D}_{ij}}(q) = \int_{\hat{F}^{-1}_{ij}(1-q)}^H \hat{\varphi}(x) dF(x)$. Since the ironed virtual value function is monotonically non-decreasing, ${R}_{D_{ij}}(\cdot)$ and $R_{\hat{D}_{ij}}(\cdot)$ are concave functions. 
\end{lemma}
\yangnote{Fill in the proof later.}

Next, we provide an upper bound of the optimal revenue using $R_{D_{ij}}$.
\begin{lemma}\label{lem:UB for UD rev}
For unit-demand bidders, there exists a collection of non-negative numbers $\{q_{ij}\}_{i\in[n], j\in[m]}$ satisfying $\sum_i q_{ij}\leq 1$ for all $j\in [m]$ and $\sum_j q_{ij}\leq 1$ for all $i\in [n]$, such that the optimal revenue $$\opt\leq 4\cdot \sum_{i,j} R_{D_{ij}}(q_{ij}).$$
\end{lemma}
\begin{proof}
As shown in~\cite{CaiDW16}, $\opt\leq 4 \copies$. Let $q_{ij}$ be the ex-ante probability that agent $(i,j)$ is served in the optimal mechanism for the copies setting. Chawla et al.~\cite{ChawlaHMS10} showed that $\copies\leq \sum_{i,j} R_{D_{ij}}(q_{ij})$. Our statement follows from the two inequalities above.\end{proof}

Next, we consider a convex program (Figure~\ref{fig:CP unit demand}) and argue that the value of the optimal solution of this program is at least $\frac{1}{8}$ of the optimal revenue.
\begin{figure}[ht]
%\colorbox{MyGray}{
\begin{minipage}{\textwidth} 
\begin{align*}\label{prog:convex ud}
&\max \sum_{i,j} R_{D_{ij}}(q_{ij})\\
\textbf{s.t. }& \sum_i q_{ij}\leq \frac{1}{2}\qquad \text{ for all $j\in[m]$}\\
& \sum_j q_{ij}\leq \frac{1}{2}\qquad \text{ for all $i\in[n]$}\\
& q_{ij}\geq 0\qquad \text{ for all $i\in[n]$ and $j\in[m]$}
\end{align*}
\end{minipage}
\caption{A Convex Program for Unit-demand Bidders with Exact Distributions.}
\label{fig:CP unit demand}
\end{figure}

\begin{lemma}\label{lem:compare exact CP with opt}
	The optimal solution of convex program in Figure~\ref{fig:CP unit demand} is at least $\frac{\opt}{8}$.
\end{lemma}
\begin{proof}
	Let $\{q'_{ij}\}$ be the collection of nonnegative numbers in Lemma~\ref{lem:UB for UD rev}. Clearly, $\left\{\frac{q_{ij}'}{2}\right\}$ is a set of feasible solution for the convex program. Since $R_{D_{ij}}(\cdot)$ is concave, $R_{D_{ij}}\left(\frac{q_{ij}'}{2}\right)\geq \frac{R_{D_{ij}}(q'_{ij})}{2} + \frac{R_{D_{ij}}(0)}{2}=\frac{R_{D_{ij}}(q'_{ij})}{2}$. Therefore, $$\sum_{i,j} R_{D_{ij}}\left(\frac{q_{ij}'}{2}\right)\geq 
	\frac{1}{2}\cdot \sum_{i,j} R_{D_{ij}}(q'_{ij})\geq \frac{\opt}{8}.$$
\end{proof}

If we know all $F_{ij}$ exactly, we can solve the convex program (Figure~\ref{fig:CP unit demand}) and use the optimal solution to construct an SPM via an approach provided in~\cite{ChawlaHMS10,CaiDW16}. The constructed sequential posted mechanism has revenue at least $\frac{1}{4}$ of the optimal value of the convex program, which is at least $\frac{\opt}{32}$. Next, we show that with only access to $\hat{F}_{ij}$, we can essentially carry out the same approach. Consider a different convex program (Figure~\ref{fig:CP unit demand approximate dist}).
\begin{figure}[ht]
\begin{minipage}{\textwidth} 
\begin{align*}
&\max \sum_{i,j} R_{\hat{D}_{ij}}(q_{ij})\\
\textbf{s.t. }& \sum_i q_{ij}\leq \frac{1}{2} + n\cdot\epsilon\qquad \text{ for all $j\in[m]$}\\
& \sum_j q_{ij}\leq \frac{1}{2}+m\cdot\epsilon \qquad \text{ for all $i\in[n]$}\\
& q_{ij}\geq 0\qquad \text{ for all $i\in[n]$ and $j\in[m]$}
\end{align*}
\end{minipage}
\caption{A Convex Program for Unit-demand Bidders with Approximate Distributions.}
\label{fig:CP unit demand approximate dist}
\end{figure}

In the following Lemma, we prove that the optimal values of the two convex programs above are close. \todo{If time permits, prove that if the support size for all $\hat{D}_{ij}$ is at most some finite number $s$, the convex program above can be rewritten as a linear program with size $\poly(n,m,s)$.}
\begin{lemma}\label{lem:UD compare the two CP}
	Let $\{{q}^*_{ij}\}_{i\in[n],j\in[m]}$ and $\{\hat{q}_{ij}\}_{i\in[n],j\in[m]}$ be the optimal solution of the convex program in Figure~\ref{fig:CP unit demand} and~\ref{fig:CP unit demand approximate dist} respectively.	$$\sum_{i,j}R_{\hat{D}_{ij}}(\hat{q}_{ij})\geq \sum_{i,j}{R}_{D_{ij}}({q}^*_{ij})-\epsilon\cdot mn H.$$
	\end{lemma}

\begin{proof}
	We first fix some notations. For any bidder $i$ and item $j$, let $\ubar{q}^*_{ij}, \bar{q}^*_{ij}$ and $x_{ij}$ $\in[0,1]$ be the numbers satisfy that $x_{ij}\cdot \ubar{q}^*_{ij}\cdot {F_{ij}}^{-1}(1-\ubar{q}^*_{ij}) +(1-x_{ij})\cdot \bar{q}^*_{ij}\cdot {F_{ij}}^{-1}(1-\bar{q}^*_{ij})=R_{D_{ij}}(q^*_{ij})$ and $x_{ij}\cdot \ubar{q}^*_{ij} +(1-x_{ij})\cdot \bar{q}^*_{ij}=q^*_{ij}$. Let $\ubar{p}_{ij} = {F_{ij}}^{-1}(1-\ubar{q}^*_{ij})$, $\bar{p}_{ij} = {F_{ij}}^{-1}(1-\bar{q}^*_{ij})$, and  $q'_{ij} = x_{ij}\cdot \left(1-\hat{F}_{ij}(\ubar{p}_{ij})\right)+(1-x_{ij})\cdot \left(1-\hat{F}_{ij}(\bar{p}_{ij})\right)$. By the definition of $R_{\hat{D}_{ij}}(\cdot)$, \begin{equation}\label{eq:compare revenue curve}
		R_{\hat{D}_{ij}}(q_{ij}')\geq x_{ij}\cdot \left(1-\hat{F}_{ij}(\ubar{p}_{ij})\right)\cdot \ubar{p}_{ij}+(1-x_{ij})\cdot \left(1-\hat{F}_{ij}(\bar{p}_{ij})\right)\cdot \bar{p}_{ij}
	\end{equation} Since $||\hat{D}_{ij}-D_{ij}||_K\leq \epsilon$, $\hat{F}_{ij}(\ubar{p}_{ij})\in [1-\ubar{q}^*_{ij}-\epsilon, 1-\ubar{q}^*_{ij}+\epsilon]$ and $\hat{F}_{ij}(\bar{p}_{ij}) \in [1-\bar{q}^*_{ij}-\epsilon, 1-\bar{q}^*_{ij}+\epsilon]$. Hence, the RHS of inequality~(\ref{eq:compare revenue curve}) is greater than $R_{D_{ij}}(q_{ij}^*)-\epsilon\cdot H$. Therefore, $R_{\hat{D}_{ij}}(q_{ij}')\geq R_{D_{ij}}(q_{ij}^*)-\epsilon\cdot H$.
	
	Next, we argue that $\{q'_{ij}\}_{i\in[n],j\in[m]}$ is a feasible solution for the convex program in Figure~\ref{fig:CP unit demand approximate dist}. Since $1-\hat{F}_{ij}(\ubar{p}_{ij})\leq \ubar{q}^*_{ij}+\epsilon$ and $1-\hat{F}_{ij}(\bar{p}_{ij})\leq \bar{q}^*_{ij}+\epsilon$, $q'_{ij}\leq q^*_{ij}+\epsilon$. Thus, $\sum_i q'_{ij} \leq \sum_i q^*_{ij} + n\cdot \epsilon \leq \frac{1}{2}+n\cdot \epsilon$ for all $j\in[m]$. Similarly, we can prove $\sum_j q'_{ij}\leq \frac{1}{2}+m\cdot \epsilon$ for all $i\in[n]$. As $\{\hat{q}_{ij}\}_{i\in[n],j\in[m]}$ is the optimal solution for the second convex program, $\sum_{i,j}R_{\hat{D}_{ij}}(\hat{q}_{ij})\geq \sum_{i,j}R_{\hat{D}_{ij}}({q}'_{ij})\geq \sum_{i,j}{R}_{D_{ij}}({q}^*_{ij})-\epsilon\cdot mn H$.
 \end{proof}

Finally, we show how to use the optimal solution of the convex program in Figure~\ref{fig:CP unit demand approximate dist} to construct an SPM that approximates the optimal revenue well. We first provide a general transformation that turns any approximately feasible solution of convex program in Figure~\ref{fig:CP unit demand} to an SPM mechanism.

\begin{lemma}\label{lem:prices to SPM}
	 For any distribution $\DD=\times_{i\in[n], j\in[m]}\DD_{ij}$, given a collection of independent random variables $\{p_{ij}\}_{i\in[n],j\in[m]}$ such that $$\sum_{i\in[n]} \Pr_{p_{ij}, t_{ij}\sim \DD_{ij}}\left[t_{ij}\geq p_{ij}\right]\leq 1-\eta_1 \text{,\quad for all $j\in[m]$}$$ and $$\sum_{j\in[m]} \Pr_{p_{ij}, t_{ij}\sim \DD_{ij}}\left[t_{ij}\geq p_{ij}\right]\leq 1-\eta_2 \text{,\quad for all $i\in[n]$},$$ we can construct in polynomial time a randomized SPM such that the revenue under $\DD$ is at least $$\eta_1 \eta_2\cdot \sum_{i, j} \E_{p_{ij}}\left[p_{ij}\cdot \Pr_{t_{ij}\sim \DD_{ij}}\left[t_{ij}\geq p_{ij}\right]\right].$$
	 \end{lemma}

\begin{proof}
	Consider a randomized SPM that sells item $j$ to bidder $i$ at price $p_{ij}$. Notice that bidder $i$ purchases exactly item $j$ if all of the following three conditions hold: (i) for all bidders $\ell\neq i$, $t_{\ell j}$ is smaller than the corresponding price $p_{\ell j}$,  (ii) for all items $k\neq j$, $t_{ik}$ is smaller than the corresponding price $p_{ik}$, and (iii) $t_{ij}$ is greater than the corresponding price $p_{ij}$. These three conditions are independent from each other. The first condition holds with probability at least $1-\sum_{\ell\neq i} \Pr_{p_{\ell j}, t_{\ell j}\sim \DD_{\ell j}}\left[t_{\ell j}\geq p_{\ell j}\right]\geq \eta_1$. The second condition holds with probability at least $1-\sum_{k \neq j} \Pr_{p_{ik}, t_{ik}\sim \DD_{ik}}\left[t_{ik}\geq p_{ik}\right]\geq \eta_2$. When the first two conditions hold, bidder $i$ purchases item $j$ whenever she can afford it. Her expected payment is $\E_{p_{ij}}\left[p_{ij}\cdot \Pr_{t_{ij}\sim \DD_{ij}}\left[t_{ij}\geq p_{ij}\right]\right]$. Hence, the expected revenue for selling item $j$ to bidder $i$ is at least $\eta_1\eta_2\cdot \E_{p_{ij}}\left[p_{ij}\cdot \Pr_{t_{ij}\sim \DD_{ij}}\left[t_{ij}\geq p_{ij}\right]\right]$ and the total expected revenue is at least $\eta_1 \eta_2\cdot \sum_{i, j} \E_{p_{ij}}\left[p_{ij}\cdot \Pr_{t_{ij}\sim \DD_{ij}}\left[t_{ij}\geq p_{ij}\right]\right]$.
	
	\end{proof}

\begin{lemma}\label{lem:convert approx CP to SPM}
	Given any feasible solution $\{{q}_{ij}\}_{i\in[n],j\in[m]}$ of the convex program in Figure~\ref{fig:CP unit demand approximate dist}, we can construct a (randomized) SPM in polynomial time such that its revenue under $D$ is at least $\left(\frac{1}{4}-(n+m)\cdot \epsilon\right)\cdot \left( \sum_{i,j}R_{\hat{D}_{ij}}(q_{ij})-\epsilon\cdot nmH\right)$. 
\end{lemma}

\begin{proof}
	We first fix some notations. For any bidder $i$ and item $j$, let $\ubar{q}_{ij}, \bar{q}_{ij}$ and $x_{ij}$ $\in[0,1]$ be the numbers satisfying $x_{ij}\cdot \ubar{q}_{ij}\cdot \hat{F}_{ij}^{-1}(1-\ubar{q}_{ij}) +(1-x_{ij})\cdot \bar{q}_{ij}\cdot \hat{F}_{ij}^{-1}(1-\bar{q}_{ij})=R_{\hat{D}_{ij}}(q_{ij})$ and $x_{ij}\cdot \ubar{q}_{ij} +(1-x_{ij})\cdot \bar{q}_{ij}=q_{ij}$. We use $p_{ij}$ to denote a random variable that is $\ubar{p}_{ij} = \hat{F}_{ij}^{-1}(1-\ubar{q}_{ij})$ with probability $x_{ij}$ and $\bar{p}_{ij} = \hat{F}_{ij}^{-1}(1-\bar{q}_{ij})$ with probability $1-x_{ij}$.  
	
	Next, we construct a randomized SPM based on $\{p_{ij}\}_{i\in[n],j\in[m]}$ according to Lemma~\ref{lem:prices to SPM}. Note that $$\sum_{i\in[n]} \Pr_{p_{ij}, t_{ij}\sim D_{ij}}\left[t_{ij}\geq p_{ij}\right]\leq \sum_{i\in[n]} \left(\Pr_{p_{ij}, t_{ij}\sim \hat{D}_{ij}}\left[t_{ij}\geq p_{ij}\right]+\epsilon\right)=\sum_{i\in[n]} q_{ij}+n\epsilon\leq \frac{1}{2}+2n\epsilon$$ for all item $j$, and $$\sum_{j\in[m]} \Pr_{p_{ij}, t_{ij}\sim D_{ij}}\left[t_{ij}\geq p_{ij}\right]\leq \sum_{j\in[m]} \left(\Pr_{p_{ij}, t_{ij}\sim \hat{D}_{ij}}\left[t_{ij}\geq p_{ij}\right]+\epsilon\right)=\sum_{i\in[m]} q_{ij}+m\epsilon\leq \frac{1}{2}+2m\epsilon$$ for all bidder $i$. Hence, we can construct in polynomial time a randomized SPM with revenue at least \begin{align*} &\left(\frac{1}{2}-2n\epsilon\right)\left(\frac{1}{2}-2m\epsilon\right)\cdot \sum_{i, j} \E_{p_{ij}}\left[p_{ij}\cdot \Pr_{t_{ij}\sim D_{ij}}\left[t_{ij}\geq p_{ij}\right]\right]\\
 	\geq &  \left(\frac{1}{4}-(n+m)\epsilon\right)\sum_{i, j} \E_{p_{ij}}\left[p_{ij}\cdot \left(\Pr_{t_{ij}\sim \hat{D}_{ij}}\left[t_{ij}\geq p_{ij}\right]-\epsilon\right)\right]\\
 	\geq & \left(\frac{1}{4}-(n+m)\epsilon\right)\sum_{i, j} \left(R_{\hat{D}_{ij}}(q_{ij})-\epsilon\cdot nmH\right)
 \end{align*}
	The first inequality is because $\left|\left|D_{ij}-\hat{D}_{ij}\right|\right|_K\leq \epsilon$, and the second inequality is because $p_{ij}$ is upper bounded by $H$ and $\E_{p_{ij}}\left[p_{ij}\cdot \Pr_{t_{ij}\sim \hat{D}_{ij}}\left[t_{ij}\geq p_{ij}\right]\right]=R_{\hat{D}_{ij}}(q_{ij})$ by the definition of $p_{ij}$.
	
	
	 %For bidder $i$, her price for item $j$ is $\ubar{p}_{ij}$ with probability $x_{ij}$ and $\bar{p}_{ij}$ with probability $1-x_{ij}$. This choice is made independently from all the other prices. The probability that bidder $i$ can afford this price is $q'_{ij}=x_{ij}\cdot\left(1-F_{ij}(\ubar{p}_{ij})\right)+(1-x_{ij})\cdot \left(1-F_{ij}(\bar{p}_{ij})\right)$. Since $||\hat{D}_{ij}-D_{ij}||_K\leq \epsilon$, $q'_{ij}\in [q_{ij}-\epsilon,q_{ij}+\epsilon]$. Notice that when it is buyer $i$'s turn, she purchases exactly item $j$ if all of the following three conditions hold: (i) for all bidders $\ell\neq i$, $t_{\ell j}$ is smaller than the corresponding price,  (ii) for all items $k\neq j$, $t_{ik}$ is smaller than the corresponding price, and (iii) $t_{ij}$ is greater than the corresponding price. All three conditions are independent from each other. The first condition holds with probability at least $1-\sum_{\ell\neq i} q'_{\ell j}\geq 1- \sum_{\ell\neq i} (q_{\ell j}+\epsilon)\geq \frac{1}{2}-2n\cdot \epsilon$. The second condition holds with probability at least $1-\sum_{k\neq j} q'_{ik}\leq 1-\sum_{k\neq j} (q_{ik}+\epsilon)\geq \frac{1}{2}-2m\cdot \epsilon$. When the two conditions hold, bidder $i$ buys item $j$ if she can afford it. The expected payment is 
\notshow{	 \begin{align*}
	&x_{ij}\cdot \ubar{p}_{ij}\cdot \left(1-F_{ij}(\ubar{p}_{ij})\right)+(1-x_{ij})\cdot\bar{p}_{ij}\cdot \left(1-F_{ij}(\bar{p}_{ij})\right)\\
	\geq &x_{ij}\cdot \ubar{p}_{ij}\cdot (\ubar{q}_{ij}-\epsilon)+(1-x_{ij})\cdot\bar{p}_{ij}\cdot (\bar{q}_{ij}-\epsilon)\geq R_{\hat{D}_{ij}}(q_{ij})-\epsilon\cdot H.\end{align*} 
	Therefore, the expected revenue of the randomized SPM is at least $$\left(\frac{1}{2}-2n\cdot \epsilon\right)\cdot \left(\frac{1}{2}-2m\cdot \epsilon\right)\cdot \left( \sum_{i,j}R_{\hat{D}_{ij}}(q_{ij})-\epsilon\cdot nmH\right)\geq \left(\frac{1}{4}-(n+m)\cdot \epsilon\right)\cdot \left( \sum_{i,j}R_{\hat{D}_{ij}}(q_{ij})-\epsilon\cdot nmH\right).$$} 
	\end{proof}


\begin{theorem}\label{thm:UD Kolmogorov}
For unit-demand bidders, given distributions $\hat{D}_{ij}$ where $\left|\left|\hat{D}_{ij}-D_{ij}\right|\right|_K\leq \epsilon$ for all $i\in[n]$ and $j\in[m]$, there is a polynomial time algorithm that constructs a randomized SPM whose revenue under $D$ is at least $\left(\frac{1}{4}-(n+m)\cdot \epsilon\right)\cdot\left(\frac{\opt}{8}-2\epsilon\cdot mnH\right)$.
\end{theorem}

\begin{proof}
	Our algorithm first computes the optimal solution $\{\hat{q}_{ij}\}_{i\in[n],j\in[m]}$ for the convex program in Figure~\ref{fig:CP unit demand approximate dist}, then  constructs a randomized SPM based on $\{\hat{q}_{ij}\}_{i\in[n],j\in[m]}$ using Lemma~\ref{lem:convert approx CP to SPM}. It is not hard to see that our algorithm runs in polynomial time. By chaining the inequalities in Lemma~\ref{lem:compare exact CP with opt},~\ref{lem:UD compare the two CP} and~\ref{lem:convert approx CP to SPM}, we can argue that the revenue of our mechanism is at least $\left(\frac{1}{4}-(n+m)\cdot \epsilon\right)\cdot\left(\frac{\opt}{8}-2\epsilon\cdot mnH\right)$.
\end{proof}

\subsection{Unit-demand Valuation: sample access to bounded distributions} 

When the distributions $D_{ij}$ are all bounded, the following theorem provides the sample complexity.
\begin{theorem}\cite{MorgensternR16}\label{thm:UD bounded}
	When $D_{ij}$ is supported on $[0,H]$ for all bidder $i$ and item $j$, the sample complexity for $(\epsilon,\delta)$-uniformly learning the revenue of SPMs for unit-demand bidders is $O\left(\left(\frac{1}{\epsilon}\right)^2 \left(m^2 n\log n\log \frac{1}{\epsilon} + \log \frac{1}{\delta}\right)\right)$. That is, with probability $1-\delta$, the empirical revenue based on the samples for any SPM is within $\epsilon\cdot H$ of its true expected revenue. %samples suffices to learn with probability $1-\delta$ an SPM whose revenue is at most $\epsilon H$ less than the best SPM whose revenue is at least $\frac{\opt}{24}$. 
	\yangnote{Moreover, with the same number of samples, there is a polynomial time algorithm that learns an SPM whose revenue is at least $\frac{\opt}{144}-\epsilon H$ with probability $1-\delta$. } \cnote{Shouldn't there be a $(H/\epsilon)^2$ in the sample complexity??}\end{theorem}


\subsection{Unit-demand Valuation: sample access to regular distributions}
In this section, we show there exists a polynomial time algorithm that learns an SPM whose revenue is at least a constant fraction of the optimal revenue with polynomial in $n$ and $m$ samples. Note that unlike in the previous two models, the error of our learning algorithm is only multiplicative when the distributions are regular. First, we present a Lemma regarding the revenue curve function for regular distributions.


%                                                                                                                                                                                                                                                                                                                                                                                                                                                                                                                                                                                                                                                                                                                                                                                                                                                                                                                                                                                                                                                                                                                                                                                                                                                                                                                                                                                                                                                                                                                                                                                                                                                                                                                                                                                                                                                                               Let us first sketch the proof. First, we use the property that all distributions are regular to argue that restricting the posted prices to be less than some upper bound $H= O\left(\poly(n,m)\cdot \opt\right)$ will affect the best revenue achievable by an SPM by at most a small multiplicative factor. In the second step, we truncate the original distributions $D_{ij}$ at the upper bound $H$ and argue that the revenue for any SPM under $D$ and the truncated distribution are exactly the same. Next, we apply Theorem~\ref{thm:UD bounded} to the truncated distribution to obtain an SPM. The only issue here is that Theorem~\ref{thm:UD bounded} has an additive error that depends on $H$. Luckily, since $H$ is at most a polynomial times larger than $\opt$ we can simply choose $\epsilon$ to be inverse of some polynomial to convert the additive error of $\epsilon H$ into a multiplicative error.

\begin{lemma}\cite{CaiD11b}\label{lem:regular revenue curve concave}
	For any regular distribution $F$, let $R_F(\cdot)$ be the corresponding revenue curve. For any $0<q'\leq q\leq  p < 1$, $$(1-p)\cdot R_F(q')\leq  R_F(q).$$
	\end{lemma}
 Throughout this section, we use $Z$ to denote $\max\{m,n\}$ and $C$ to be a constant that will be specified later. Using Lemma~\ref{lem:regular revenue curve concave}, we show in the next Lemma that restricting $q_{ij}$ to be at least $\frac{1}{ CZ}$ does not affect the objective value of the convex program in Figure~\ref{fig:CP unit demand} by too much.

\begin{lemma}\label{lem:lowering high prices}
	Suppose $\{q_{ij}^*\}_{i\in[n], j\in[m]}$ is the optimal solution of the convex program in Figure~\ref{fig:CP unit demand}. Let $q'_{ij}=\max\{\frac{1}{CZ}, q^*_{ij}\}$, then $\sum_{i,j} R_{D_{ij}}(q'_{ij})\geq \left(1-\frac{1}{ CZ}\right)\cdot \sum_{i,j} R_{D_{ij}}(q^*_{ij})\geq \left(1-\frac{1}{ CZ}\right)\cdot \frac{\opt}{8}$.
\end{lemma}
\begin{proof}
	According to Lemma~\ref{lem:compare exact CP with opt}, $\sum_{i,j} R_{D_{ij}}(q^*_{ij})\geq \frac{\opt}{8}$. So to prove the statement, it suffices to argue that for any $i$ and $j$, $R_{D_{ij}}(q'_{ij})\geq \left(1-\frac{1}{CZ}\right)\cdot R_{D_{ij}}(q^*_{ij})$. If $q^*_{ij} = q'_{ij}$, this inequality clearly holds. If $q^*_{ij} \neq q'_{ij}$, $q^*_{ij}\leq q'_{ij}=\frac{1}{CZ}$. Since $F_{ij}$ is regular, we can apply Lemma~\ref{lem:regular revenue curve concave} to $q'_{ij}$ and $q^*_{ij}$ and obtain inequality $R_{D_{ij}}(q'_{ij})\geq \left(1-\frac{1}{C Z}\right)\cdot R_{D_{ij}}(q^*_{ij})$.
\end{proof}

Using Lemma~\ref{lem:lowering high prices}, we argue how to compute in polynomial time an approximately optimal SPM. Suppose $D'_{ij}$ is the distribution that we obtain after truncating $D_{ij}$ at a threshold $H_{ij}$\footnote{Let $t_{ij}\sim D_{ij}$, then $\min\{t_{ij},H_{ij}\}$ is the corresponding truncated random variable drawn from $D'_{ij}$.}, and we have direct access to a discrete distribution $\hat{D}'_{ij}$ such that $\left|\left|\hat{D}'_{ij}-D'_{ij}\right|\right|_K\leq \epsilon$ for all $i$ and $j$. We show in the following Lemma that the optimal solution of a convex program similar to the one in Figure~\ref{fig:CP unit demand approximate dist} but for $\{\hat{D}'_{ij}\}_{i\in[n],j\in[m]}$ can guide us to design an approximately optimal SPM under $D$ in polynomial time. As we have sample access to $D$, we will argue later that a polynomial number of samples suffices to generate $\{\hat{D}'_{ij}\}_{i\in[n],j\in[m]}$.%Then for any SPM that does not use price $p_{ij}> H_{ij}$, its revenue under $D'=\times_{i,j} D'_{ij}$ and $D$ is the same. Therefore, if we can learn an SPM with good revenue for $D'$, the same mechanism also achieves good revenue under $D$.
\notshow{\begin{lemma}\label{lem:UD low prices suffice}
	There exists a collection of deterministic prices $\{p'_{ij}\}_{i\in[n],j\in[m]}$ satisfying $F_{ij}(p'_{ij})\leq 1-\frac{1}{C\cdot Z}$ for all $i\in[n]$ and $j\in[m]$, such that the corresponding SPM achieves revenue of $\left(\frac{1}{2}-\frac{1}{C}\right)^2\cdot \left(1-\frac{1}{C\cdot Z}\right)\cdot \frac{\opt}{8}$.
\end{lemma} 
\begin{proof}
Let $\{q_{ij}^*\}_{i\in[n], j\in[m]}$ be the optimal solution of the convex program in Figure~\ref{fig:CP unit demand} and $q'_{ij}=\max\{\frac{1}{C\cdot Z}, q^*_{ij}\}$. For every $i$ and $j$, let $p'_{ij}$ to be $F^{-1}_{ij}(1-q'_{ij})$. Clearly, $F_{ij}(p'_{ij})\leq 1-\frac{1}{C\cdot Z}$. 

As $D_{ij}$ is regular, $R_{D_{ij}}(q'_{ij}) = p'_{ij} q'_{ij}$. Therefore, $\sum_{i,j} p'_{ij}q'_{ij}=\sum_{i,j} R_{D_{ij}}(q'_{ij})\geq \left(1-\frac{1}{CZ} \right)\cdot \frac{\opt}{8}$ according to Lemma~\ref{lem:UD low prices suffice}. Also, $\sum_i q'_{ij}\leq \frac{1}{2}+\frac{n}{CZ}\leq \frac{1}{2}+\frac{1}{C}$ for all item $j$, and $\sum_j q'_{ij}\leq \frac{1}{2}+\frac{m}{CZ}\leq \frac{1}{2}+\frac{1}{C}$ for all bidder $i$. According to Lemma~\ref{lem:prices to SPM}, the SPM with prices $\{p'_{ij}\}_{i\in[n],j\in[m]}$ has revenue at least $\left(\frac{1}{2}-\frac{1}{C}\right)^2\cdot \sum_{i,j} p'_{ij}q'_{ij}\geq \left(\frac{1}{2}-\frac{1}{C}\right)^2\cdot  \left(1-\frac{1}{CZ} \right)\cdot \frac{\opt}{8}$.

%Next, we argue that if we use $p_{ij}$ as the prices in an SPM, the expected revenue is at least $\left(\frac{1}{2}-\frac{1}{C}\right)^2\cdot \left(1-\frac{1}{C\cdot Z}\right)\cdot \frac{\opt}{8}$. For any $i$ and $j$, the probability that (i) all bidders $\ell\neq i$ cannot afford item $j$ and (ii) bidder $i$ cannot afford the price for any item $k\neq j$ is at least $\left(1-\sum_{\ell\neq i}q'_{\ell j}\right)\cdot \left(1-\sum_{k\neq j}q'_{ik}\right)$. Since both $\sum_\ell q^*_{\ell j}$ and $\sum_{k}q^*_{ik}$ are less than $\frac{1}{2}$, both $1-\sum_{\ell\neq i}q'_{\ell j}$ and $1-\sum_{k\neq j}q'_{ik}$ are at least $\frac{1}{2}-\frac{1}{C}$. Clearly, when (i) and (ii) are simultaneously satisfied, bidder $i$ purchases item $j$ whenever $t_{ij}$ is greater than $p_{ij}$. Therefore, the total revenue of this SPM is at least $\left(\frac{1}{2}-\frac{1}{C}\right)^2\cdot \sum_{i,j} p_{ij}\cdot q'_{ij}$. As $F_{ij}$ is regular, $p_{ij}\cdot q'_{ij}=R_{D_{ij}}(q'_{ij})$. Finally, we know that  $\sum_{i,j} R_{D_{ij}}(q'_{ij})\geq \left(1-\frac{1}{C\cdot Z}\right)\cdot \sum_{i,j} R_{D_{ij}}(q^*_{ij})\geq \left(1-\frac{1}{C\cdot Z}\right)\cdot \frac{\opt}{8}$ from Lemma~\ref{lem:lowering high prices} and~\ref{lem:compare exact CP with opt}. Combining this inequality with the lower bound of the revenue for the SPM, we can show that the revenue of our constructed SPM is at least $\left(\frac{1}{2}-\frac{1}{C}\right)^2\cdot \left(1-\frac{1}{C\cdot Z}\right)\cdot \frac{\opt}{8}$.
\end{proof}



\begin{lemma}\label{lem:UD additive bound}
	Let $\{H_{ij}\}_{i\in[n],j\in[m]}$ be a collection of positive numbers satisfying $F_{ij}(H_{ij})\in [1-\frac{1}{C\cdot Z}, 1-\frac{1}{3C\cdot Z}]$ for all $i\in[n]$ and $j\in[m]$. With $O\left(\left(\frac{1}{\epsilon}\right)^2 \left(m^2 n\log n\log \frac{1}{\epsilon} + \log \frac{1}{\delta}\right)\right)$ samples, we can learn an SPM with probability $1-\delta$ that achieves revenue at least $\left(\frac{1}{2}-\frac{1}{C}\right)^2\cdot \left(1-\frac{1}{C\cdot Z}\right)\cdot \frac{\opt}{8}-\epsilon\cdot\max_{i,j} H_{ij}$. \yangnote{With the same number of samples, we can learn in polynomial time an SPM with probability $1-\delta$ that achieves revenue at least $\left(\frac{1}{2}-\frac{1}{C}\right)^2\cdot \left(1-\frac{1}{C\cdot Z}\right)\cdot \frac{\opt}{48}-\epsilon\cdot\max_{i,j} H_{ij}$.}
\end{lemma}
\begin{proof}
	First, we truncate each $D_{ij}$ at $H_{ij}$ to create a bounded distribution $D'_{ij}$. This is straightforward, as we only need to cap the value for any sample from $D_{ij}$ at $H_{ij}$. According to Lemma~\ref{lem:UD low prices suffice}, there exists an SPM that achieves revenue at least $\left(\frac{1}{2}-\frac{1}{C}\right)^2\cdot \left(1-\frac{1}{C\cdot Z}\right)\cdot \frac{\opt}{8}$ under $D'=\times_{i,j} D'_{ij}$, as the price $p_{ij}$ used in the SPM is less than $H_{ij}$ for all $i,j$. Combining this observation with Theorem~\ref{thm:UD bounded}, we know that if we take $O\left(\left(\frac{1}{\epsilon}\right)^2 \left(m^2 n\log n\log \frac{1}{\epsilon} + \log \frac{1}{\delta}\right)\right)$ samples we can learn with probability $1-\delta$ an SPM whose revenue is at least $\left(\frac{1}{2}-\frac{1}{C}\right)^2\cdot \left(1-\frac{1}{C\cdot Z}\right)\cdot \frac{\opt}{8}-\epsilon\cdot\max_{i,j} H_{ij}$. Similarly, we can prove the computational friendly version of this claim. \end{proof}}
	
\begin{lemma}\label{lem:UD additive bound}
	Let $\{H_{ij}\}_{i\in[n],j\in[m]}$ be a collection of positive numbers satisfying $F_{ij}(H_{ij})\in [1-\frac{1}{C\cdot Z}, 1-\frac{1}{3C\cdot Z}]$ for all $i\in[n]$ and $j\in[m]$. Let $D'_{ij}$ be the distribution of the random variable $\min\{t_{ij}, H_{ij}\}$ where $t_{ij}\sim D_{ij}$, and $\hat{D}_{ij}'$ be a discrete distribution such that $\left|\left|\hat{D}'_{ij}-D'_{ij}\right|\right|_K\leq \epsilon$ for all $i\in[n]$ and $j\in[m]$. Suppose $s$ is an upper bound of the support size for any distribution $\hat{D}'_{ij}$, then given direct access to $\hat{D}_{ij}'$, we can compute in time polynomial in $n$, $m$ and $s$ a randomized SPM that achieves revenue at least $\left(\frac{1}{2}-\frac{1}{C}-2n\epsilon\right)\cdot\left(\frac{1}{2}-\frac{1}{C}-2m\epsilon\right)\cdot \left(\left(1-\frac{1}{ CZ}\right)\cdot \frac{\opt}{8}-2\epsilon\cdot nmH\right)$ under $D$, where $H= \max_{i,j} H_{ij}$. 
\end{lemma}
\begin{proof}
Consider the following convex program:

\begin{minipage}{\textwidth} 
\begin{align*}
&\max \sum_{i,j} R_{\hat{D}_{ij}'}(q_{ij})\\
\textbf{s.t. }& \sum_i q_{ij}\leq \frac{1}{2} + \frac{1}{C} +n\cdot\epsilon\qquad \text{ for all $j\in[m]$}\\
& \sum_j q_{ij}\leq \frac{1}{2}+\frac{1}{C} +m\cdot\epsilon \qquad \text{ for all $i\in[n]$}\\
& q_{ij}\geq 0\qquad \text{ for all $i\in[n]$ and $j\in[m]$}
\end{align*}
\end{minipage}
\vspace{.1in}

Let $\{q_{ij}^*\}_{i\in[n], j\in[m]}$ be the optimal solution of the convex program in Figure~\ref{fig:CP unit demand} and $q'_{ij}=\max\{\frac{1}{C\cdot Z}, q^*_{ij}\}$. For every $i$ and $j$, let $p'_{ij}=F^{-1}_{ij}(1-q'_{ij})$ and $\tilde{q}_{ij} = \Pr_{t_{ij}\sim \hat{D}'_{ij}}\left[t_{ij}\geq p'_{ij}\right]$. By the definition of $H_{ij}$, $p'_{ij}\leq H_{ij}$, so \begin{align}\label{ineq:LB inequality}
p'_{ij}\tilde{q}_{ij}\geq p'_{ij}\left(\Pr_{t_{ij}\sim {D}'_{ij}}\left[t_{ij}\geq p'_{ij}\right]-\epsilon\right)\geq p'_{ij} q'_{ij}-\epsilon\cdot H_{ij}=R_{D_{ij}}(q'_{ij})-\epsilon\cdot H_{ij}.
 \end{align}
$p'_{ij} q'_{ij}$ equals to $R_{D_{ij}}(q'_{ij})$ because $D_{ij}$ is a regular distribution.

Next, we argue that $\{\tilde{q}_{ij}\}_{i\in[n],j\in[m]}$ is a feasible solution of the convex program above. Observe that $$\sum_{i} \tilde{q}_{ij} \leq \sum_i q'_{ij}+n\epsilon \leq \sum_i \left(q^*_{ij}+\frac{1}{CZ}\right)+n\epsilon\leq \frac{1}{2} + \frac{1}{C} +n\epsilon$$ for all item $j\in[m]$ and $$\sum_{j} \tilde{q}_{ij} \leq \sum_j q'_{ij}+n\epsilon \leq \sum_j \left(q^*_{ij}+\frac{1}{CZ}\right)+m\epsilon\leq \frac{1}{2}+\frac{1}{C} +m\epsilon$$ for all bidder $i\in[n]$. 

 Let $\widehat{\opt}$ be the optimal solution of the convex program above. As $\{\tilde{q}_{ij}\}_{i\in[n],j\in[m]}$ is a feasible solution,
$$\widehat{\opt}\geq \sum_{i,j} R_{\hat{D}_{ij}'}(\tilde{q}_{ij})\geq \sum_{i,j} p'_{ij}\tilde{q}_{ij}\geq \sum_{i,j} R_{D_{ij}}(q'_{ij})-\epsilon\cdot nmH\geq \left(1-\frac{1}{ CZ}\right)\cdot \frac{\opt}{8}-\epsilon\cdot nmH.$$
The second last inequality is due to inequality~(\ref{ineq:LB inequality}) and the last inequality is due to Lemma~\ref{lem:lowering high prices}.

So far, we have argued that the optimal solution of our convex program has value close to the $\opt$. We will show in the second part of the proof that using the optimal solution of our convex program, we can construct an SPM whose revenue under $D$ is close to $\widehat{\opt}$. Let $\hat{q}_{ij}$ be the optimal solution of the convex program above and $\hat{p}_{ij}$ be the corresponding random price, that is, $\Pr_{\hat{p}_{ij}, t_{ij}\sim \hat{D}_{ij}'}\left[t_{ij}\geq \hat{p}_{ij}\right]=\hat{q}_{ij}$ and $R_{\hat{D}_{ij}'}(\hat{q}_{ij}) = \E_{\hat{p}_{ij}}\left[\hat{p}_{ij}\cdot \Pr_{t_{ij}\sim \hat{D}_{ij}'}\left[t_{ij}\geq \hat{p}_{ij}\right]\right]$. As $\hat{p}_{ij}\leq H_{ij}$, $$\Pr_{\hat{p}_{ij}, t_{ij}\sim {D}_{ij}}\left[t_{ij}\geq \hat{p}_{ij}\right]=\Pr_{\hat{p}_{ij}, t_{ij}\sim {D}'_{ij}}\left[t_{ij}\geq \hat{p}_{ij}\right]\in  [\hat{q}_{ij}-\epsilon,\hat{q}_{ij}+\epsilon].$$ Therefore, for all item $j$ $$\sum_{i} \Pr_{\hat{p}_{ij}, t_{ij}\sim {D}_{ij}}\left[t_{ij}\geq \hat{p}_{ij}\right]\leq \sum_{i} \Pr_{\hat{p}_{ij}, t_{ij}\sim \hat{D}_{ij}'}\left[t_{ij}\geq \hat{p}_{ij}\right] + n\epsilon = \sum_i \hat{q}_{ij} +n\epsilon \leq \frac{1}{2} + \frac{1}{C} +2n\epsilon$$ and for all bidder $i$	$$\sum_{j} \Pr_{\hat{p}_{ij}, t_{ij}\sim {D}_{ij}}\left[t_{ij}\geq \hat{p}_{ij}\right]\leq \sum_{j} \Pr_{\hat{p}_{ij}, t_{ij}\sim \hat{D}_{ij}'}\left[t_{ij}\geq \hat{p}_{ij}\right] + m\epsilon = \sum_j \hat{q}_{ij} +m\epsilon \leq \frac{1}{2} + \frac{1}{C} +2m\epsilon.$$ According to Lemma~\ref{lem:prices to SPM}, we can construct a randomized SPM with $\{\hat{p}_{ij}\}_{i\in[n],j\in[m]}$ whose revenue is at least $\left(\frac{1}{2}-\frac{1}{C}-2n\epsilon\right)\cdot\left(\frac{1}{2}-\frac{1}{C}-2m\epsilon\right)\cdot \sum_{i,j} \E_{\hat{p}_{ij}}\left[\hat{p}_{ij}\cdot \Pr_{t_{ij}\sim {D}_{ij}}\left[t_{ij}\geq \hat{p}_{ij}\right]\right]$ under $D$. 
Clearly, $$\E_{\hat{p}_{ij}}\left[\hat{p}_{ij}\cdot \Pr_{t_{ij}\sim {D}_{ij}}\left[t_{ij}\geq \hat{p}_{ij}\right]\right]\geq \E_{\hat{p}_{ij}}\left[\hat{p}_{ij}\cdot \left(\Pr_{t_{ij}\sim \hat{D}'_{ij}}\left[t_{ij}\geq \hat{p}_{ij}\right]-\epsilon\right)\right] = R_{\hat{D}'_{ij}}(\hat{q}_{ij})-\epsilon\cdot H_{ij}.$$ Therefore, the revenue of the constructed randomized SPM under $D$ is at least \begin{align*} &\left(\frac{1}{2}-\frac{1}{C}-2n\epsilon\right)\cdot\left(\frac{1}{2}-\frac{1}{C}-2m\epsilon\right)\cdot \left(\widehat{\opt}-\epsilon\cdot nmH\right)\\
 	\geq & \left(\frac{1}{2}-\frac{1}{C}-2n\epsilon\right)\cdot\left(\frac{1}{2}-\frac{1}{C}-2m\epsilon\right)\cdot \left(\left(1-\frac{1}{ CZ}\right)\cdot \frac{\opt}{8}-2\epsilon\cdot nmH\right).
 \end{align*}
It is not hard to see that both $\{\hat{q}_{ij}\}_{i\in[n],j\in[m]}$ and $\{\hat{p}_{ij}\}_{i\in[n],j\in[m]}$ can be computed in time polynomial in $n$, $m$ and $s$.
	\end{proof}

	When $\epsilon$ is small enough, we can turn the additive error in Lemma~\ref{lem:UD additive bound} into a multiplicative error. Next, we argue that with a polynomial number of samples, we can learn $\{H_{ij}\}_{i\in[n],j\in[m]}$ and $\{\hat{D}'_{ij}\}_{i\in[n],j\in[m]}$ with enough accuracy.

\begin{theorem}\label{thm:UD regular}
If for all bidder $i$ and item $j$, $D_{ij}$ is a regular distribution, we can learn in polynomial time with probability $1-\delta$ a randomized SPM whose revenue is at least $\frac{\opt}{33}$ with $O\left(Z^2m^2 n^2\cdot \left(\log n +\log m + \log \frac{1}{\delta}\right)\right)$ ($Z=\max\{m,n\}$) samples. %\yangnote{With the same number of samples, we can learn in polynomial time with probability $1-\delta$ an SPM whose revenue is at least $\frac{\opt}{198}$.}
\end{theorem}
\begin{proof}
	First, if we take $O\left(C^2\cdot Z^2\cdot \left(\log (nm)+\log \frac{1}{\delta}\right)\right)$ samples from each $D_{ij}$, we can estimate $H_{ij}$ such that $F_{ij}(H_{ij})\in[1-\frac{1}{C\cdot Z}, 1-\frac{1}{3C\cdot Z}]$ with probability $1-\frac{\delta}{2nm}$. By the union bound, the probability that all $H_{ij}$ satisfy the requirement is at least $1-\frac{\delta}{2}$. From now on, we only consider the case where $H_{ij}$ satisfies $F_{ij}(H_{ij})\in[1-\frac{1}{C\cdot Z}, 1-\frac{1}{3C\cdot Z}]$ for all $i$ and $j$. Next, we observe that $\opt \geq \max_{i,j} H_{ij}\cdot \frac{1}{3C\cdot Z}$, as the expected revenue for selling item $j$ to bidder $i$ with price $H_{ij}$ is at least $ \frac{H_{ij}}{3C\cdot Z}$. Therefore, there exists sufficiently large constant $d$ and $C$, if $\epsilon=\frac{1}{d\cdot Z nm}$ the randomized SPM learned in Lemma~\ref{lem:UD additive bound} has revenue at least $\frac{\opt}{33}$. According to the Dvoretzky-Kiefer-Wolfowitz (DKW) inequality~\cite{DvoretzkyKW56}, if we take $O\left(d^2 Z^2n^2m^2\cdot \left(\log \frac{1}{\delta}+\log n+\log m\right) \right)$ i.i.d. samples from $D'_{ij}$ (we can take samples from $D_{ij}$ then cap the samples at $H_{ij}$) and let $\hat{D}'_{ij}$ be the empirical distribution induced by the samples, $\left|\left|D'_{ij}-\hat{D}'_{ij}\right|\right|_K\leq \frac{1}{d\cdot Z nm}$ with probability $1-\frac{\delta}{2nm}$. By the union bound, $\left|\left|D'_{ij}-\hat{D}'_{ij}\right|\right|_K\leq \frac{1}{d\cdot Z nm}$ for all $i\in[n]$ and $j\in[m]$ with probability at least $1-\delta/2$. Finally, if we take another union bound over the choice of $\{H_{ij}\}_{i\in[n],j\in[m]}$ and $\{\hat{D}'_{ij}\}_{i\in[n],j\in[m]}$, we can learn a randomized SPM with probability at least $1-\delta$ using $O\left( Z^2n^2m^2\cdot \left(\log \frac{1}{\delta}+\log n+\log m\right) \right)$ samples. Furthermore, the support size of any $\hat{D}'_{ij}$ is at most $O\left( Z^2n^2m^2\cdot \left(\log \frac{1}{\delta}+\log n+\log m\right) \right)$, so our learning algorithm runs in time polynomial in $n$ and $m$.
	
	%To turn the additive error in Lemma~\ref{lem:UD additive bound} into a multiplicative error, $\epsilon$ needs to be as small as  Hence, we can learn with probability $1-\frac{\delta}{2}$ an SPM whose revenue is at least $\left(\frac{1}{2}-\frac{1}{C}\right)^2\cdot \left(1-\frac{1}{C\cdot Z}\right)\cdot \frac{\opt}{8}-3C Z\cdot\opt\cdot  \epsilon$, if we are given $O\left(\left(\frac{1}{\epsilon}\right)^2 \left(m^2 n\log n\log \frac{1}{\epsilon} + \log \frac{1}{\delta}\right)\right)$ samples. We let $\epsilon$ to be $\frac{1}{C^2\cdot Z}$. If $C$ is sufficiently big, the SPM learnt has revenue at least $\frac{\opt}{33}$  with probability $1-\frac{\delta}{2}$ and the number of samples needed is $O\left(Z^2 \cdot\left(m^2 n\log n\log Z + \log \frac{1}{\delta}\right)\right)$. 
	
	%To sum up, with $$O\left(Z^2 \cdot\left(m^2 n\log n\log Z + \log \frac{1}{\delta}\right)\right)+O\left(Z^2\cdot \left(\log (nm)+\log \frac{1}{\delta}\right)\right)= O\left(Z^2 \cdot\left(m^2 n\log n\log Z + \log \frac{1}{\delta}\right)\right)$$ samples we can learn an SPM whose revenue is at least $\frac{\opt}{33}$ with probability $1-\delta$. Similarly, we can prove the computational friendly version of this claim.
\end{proof}





