\section{Constrained Additive Bidders: Uniform Convergence of the Revenue of Sequential Posted Price with Entry Fee Mechanisms}\label{sec:uniform convergence of SPEM}

We consider a specific class of mechanisms, namely Sequential Posted Price with Entry fee Mechanisms, a.k.a. \textbf{SPEM}s; see Algorithm~\ref{alg:spem-mech} for details. Cai and Zhao~\cite{CaiZ17} recently showed that if the bidders' valuations are XOS over independent items, the best SPEM achieves a constant fraction of the optimal revenue. \footnote{Cai and Zhao~\cite{CaiZ17} showed that the best ASPE or RSPM achieves a constant fraction of the optimal revenue. Clearly, any ASPE is also a SPEM, and any RSPM is simply a SPEM if we force the bidders to be unit-demand by only allowing each of them to purchase at most one item.} This section has two goals. The first is to show that, when bidders have constrained additive valuations over independent items, polynomially many samples suffice to guarantee uniform convergence for the revenue of all SPEMs, and hence our ability to select a near-optimal SPEM from polynomially many samples. This can be proven by applying our uniform convergence result for single-intersecting events (Lemma~\ref{lem:uniform convergence for single-intersecting}). The second (and stronger goal) is to show that we can learn a near-optimal SPEM under the max-min learning model (Theorem~\ref{thm:constrained additive Kolmogorov}). We show that the revenue of any SPEM changes no more than $O(\epsilon \cdot m^2\cdot n \cdot H)$ under the true and approximate valuation distributions (Theorem~\ref{thm:revenue stability under K-distance}), where $\epsilon$ is an upper bound of the Kolmogorov distance between the true and approximate distributions for every item marginal of every bidder. It is, of course, not hard to see that Theorem~\ref{thm:revenue stability under K-distance} and the DKW inequality imply uniform convergence of the revenue of all SPEMS. To establish Theorem~\ref{thm:revenue stability under K-distance}, we need to apply Lemma~\ref{lem:Kolmogorov stable for sc} instead of Lemma~\ref{lem:uniform convergence for single-intersecting}.


\begin{algorithm}[ht]
\begin{algorithmic}[1]
\REQUIRE A collection of prices $\{p_{ij}\}_{i\in[n], j\in[m]}$ and a collection of entry fee functions $\{\delta_i(\cdot)\}_{i\in[n]}$ where $\delta_i: 2^{[m]}\mapsto \mathbb{R}$ is bidder $i$'s entry fee function.
\STATE $S\gets [m]$
\FOR{$i \in [n]$}
	\STATE Show bidder $i$ {the} set of available items $S$ and set the entry fee for bidder $i$ to be ${\delta_i}(S)$.
    \IF{Bidder $i$ pays the entry fee ${\delta_i}(S)$}
        \STATE $i$ receives her favorite bundle $S_i^{*}$ and pays $\sum_{j\in S_i^{*}}p_{ij}$.
        \STATE $S\gets S\backslash S_i^{*}$.
    \ELSE
        \STATE $i$ gets nothing and pays $0$.
    \ENDIF
\ENDFOR
\end{algorithmic}
\caption{{\sf Sequential Posted Price with Entry Fee Mechanism (SPEM)}}
\label{alg:spem-mech}
\end{algorithm} 


 We first establish a technical lemma, which states that, for any set of items $S$, any set of prices $\{p_j\}_{j\in [m]}$ and entry fee $\delta$, the distribution over the {\em set of items} purchased by a constrained additive bidder whose valuation is drawn from  $\DD=\times_{j\in[m]} \DD_j$ and $\hat{\DD}=\times_{j\in[m]} \hat{\DD}_j$ has total variation distance at most $2m\xi$, if $||\DD_j-\hat{\DD}_j||_K\leq \xi$ for every item $j\in [m]$. This is quite surprising. Given that, for each set of items $S' \subseteq S$, the difference in the probability that the buyer will purchase this particular set $S'$ under $\DD$ and $\hat{\DD}$ could already be as large as {$\Theta(m\xi)$}, and the distribution has an exponentially large support size,  a trivial argument would give a bound of {$2^m\cdot \Theta(m\xi)$}. To overcome this analytical difficulty, we argue instead that for any collection of sets of items, the event that the buyer's favorite set lies in this collection is single-intersecting. Then our result follows from Lemma~\ref{lem:Kolmogorov stable for sc}. Notice that it is crucial that Lemma~\ref{lem:Kolmogorov stable for sc} holds for all events that are single-intersecting, as the event we consider here is clearly non-convex in general. 


\begin{lemma}\label{lem:stable demand set}
	For any set $S\subseteq [m]$, any prices $\{p_j\}_{j\in[m]}$ and entry fee $\delta(S)$, let $\LL$ and $\hat{\LL}$ be the distributions over the set of items purchased from $S$ by a constrained additive bidder under prices $\{p_j\}_{j\in[m]}$ and entry fee $\delta$ when her type is drawn from $\DD=\times_{j\in[m]} \DD_j$ and $\hat{\DD}=\times_{j\in[m]} \hat{\DD}_j$ respectively. If $||\DD_{j}-\hat{\DD}_{j}||_K\leq \xi$ for all item $j$, $||\LL-\hat{\LL}||_{TV}\leq 2m \xi.$
\end{lemma}
\begin{proof}
%Define function $f_{\KK,\boldsymbol{p}, S, \delta}:\mathbb{R}_{\geq 0}^m\mapsto \{0,1\}$ for any $\KK\subseteq 2^{[m]}$, $S\in[m]$, $\boldsymbol{p}\in \mathbb{R}_{\geq 0}^m$ and $\delta\in \mathbb{R}_{\geq 0}$. $f_{\KK,\boldsymbol{p}, S, \delta}(t)=1$ iff the constrained additive bidder with type $t$ will purchase a set of item $R\in \KK$ when the set of available items is $S$, the item prices are $\boldsymbol{p}=\{p_j\}_{j\in[m]}$ and the entry fee is $\delta$. Let $\HH$ be the hypothesis class that contains all $f_{\KK,\boldsymbol{p}, S, \delta}$'s. To prove the claim of this Lemma, it suffices to show that $$\sup_{f\in\HH} \left|\E_{\boldsymbol{x}\sim \DD}\left[f(\boldsymbol{x})\right]-\E_{\boldsymbol{x}\sim \hat{\DD}}\left[f(\boldsymbol{x})\right]\right|\leq 2m\cdot \epsilon.$$

For any set $R\subseteq S$, let $\EE_R$ be the event that the bidder purchases set $R$. Proving that the total variation distance between $\LL$ and $\hat{\LL}$ is no more than $2m\cdot \xi$ is the same as proving that for any $K\leq 2^{|S|}$, $\left|\ \Pr_{\DD}\left[t\in \bigcup_{\ell=1}^K \EE_{R_\ell}\right]-\Pr_{\hat{\DD}}\left[t\in \bigcup_{\ell=1}^K \EE_{R_\ell}\right]\right|\leq 2m\cdot \xi$ where $R_1,\cdots R_K$ are arbitrary distinct subsets of $S$. Since the dimension of the bidder's type space is $m$, if we can prove that $\bigcup_{\ell=1}^K \EE_{R_\ell}$ is always single-intersecting, our claim follows from Lemma~\ref{lem:Kolmogorov stable for sc}. 

%For every $j\in[m]$, let $\HH_j=\left\{g\ |\ \exists f\in \HH\ \exists\ a_{-j}\in \mathbb{R}^{m-1} \ \forall\ x\in\mathbb{R}, \ g(x) = f(x,a_{-j})\right\}$. Next, we will prove that all functions in $\HH_i$ are indicator functions of intervals in $\mathbb{R}$. From now on, we fix $\KK,\boldsymbol{p}, S$ and $\delta$. 
%For any set $R\subseteq S$, let $\EE_R$ be the event that the bidder purchases set $R$. %Proving that the total variation distance between $\LL$ and $\hat{\LL}$ is no more than $2m\cdot \xi$ is the same as proving that for any $K\leq 2^{|S|}$, $\left|\ \Pr_{D_i}\left[t_i\in \bigcup_{\ell=1}^K \EE_{R_\ell}\right]-\Pr_{\hat{D}_i}\left[t_i\in \bigcup_{\ell=1}^K \EE_{R_\ell}\right]\right|\leq 2m\cdot \xi$ where $R_1,\cdots R_K$ are arbitrary distinct subsets of $S$. Since the dimension of the bidder's type space is $m$, if we can prove that $\bigcup_{\ell=1}^K \EE_{R_\ell}$ is always single-intersecting, our claim follows from Lemma~\ref{lem:Kolmogorov stable for sc}. 
For any $j\in[m]$ and $a_{-j}\in \mathbb{R}_{\geq 0}^{{m}-1}$, let $L_j(a_{-j})=\left\{ (t_{j},a_{-j}) \ | t_{j}\in \mathbb{R}_{\geq 0} \right\}$. We claim that $L_j(a_{-j})$ intersects with at most two different $\EE_{U}$ and $\EE_{V}$ where $U$ and $V$ are subsets of $S$. 
	WLOG, we assume that  $(0,a_{-j})\in \EE_U$. 
	\begin{itemize}
		\item If $U=\emptyset$, that means the utility of the favorite set for type $(0,a_{-j})$ is smaller than the entry fee $\delta(S)$. If we increase the value of $t_{j}$, two cases could happen: (1) the utility of the favorite set is still lower than the entry fee; (2) the utility of the favorite set is higher than the entry fee. In case (1), $(t_{j},a_{-j})\in \EE_{\emptyset}$. In case (2), the bidder pays the entry fee and purchases her favorite set $V$. Then item $j$ must be in $V$, because otherwise the utility for set $V$ does not change from type $(0,a_{-j})$ to type $(t_{j},a_{-j})$. If we keep increasing $t_{j}$, bidder $i$'s favorite set remains to be $V$ and she keeps accepting the entry fee and purchasing $V$. Hence, $L_j(a_{-j})$ can intersect with at most one event $\EE_R$ where $R$ is non-empty.
		\item If $U\neq \emptyset$, that means $U$ is the favorite set of type $(0,a_{-j})$ and the utility for winning set $U$ is higher than the entry fee.  If we increase the value of $t_{j}$, two cases could happen: (1) $U$ remains the favorite set; (2) a different set $V$ becomes the new favorite set. In case (1), $(t_{j},a_{-j})\in \EE_U$. In case (2), item $j$ must lie in $V$ but not in $U$, otherwise how could $U$ be better than $V$ for type $(0,a_{-j})$ but worse for type $(t_{j},a_{-j})$.  If we keep increasing $t_{j}$, the bidder's favorite set remains to be $V$ and she keeps accepting the entry fee and purchasing $V$. Hence, $L_j(a_{-j})$ can intersect at most two different events.
	\end{itemize}
	
It is not hard to see that any event $\EE_R$ is an intersection of halfspaces, so the intersection of $L_j(a_{-j})$ with any event $\EE_R$ is an interval. {Also, notice that any type $t\in\mathbb{R}_{\geq 0}^{m}$ must lie in an event $\EE_R$ for some set $R\subseteq S$.} If $L_j(a_{-j})$ intersects with two different events $\EE_U$ and $\EE_V$, the two intersected intervals must lie back to back on $L_j(a_{-j})$. Otherwise, $L_j(a_{-j})$ intersects with at least three different events. Contradiction. Since $L_j(a_{-j})$ intersects with at most two different events, no matter which of these events are in $\{\EE_{R_\ell}\}_{\ell\in[K]}$, the intersection of $L_j(a_{-j})$ and $\bigcup_{\ell=1}^K \EE_{R_\ell}$ is either empty or an interval meaning $\bigcup_{\ell=1}^K \EE_{R_\ell}$ is single-intersecting. Now our claim simply follows from Lemma~\ref{lem:Kolmogorov stable for sc}.
\end{proof}

\begin{theorem}\label{thm:revenue stability under K-distance}
	Suppose all bidders' valuations are constrained additive over independent items. For any SPEM, let $\rev$ and $\widehat{\rev}$ be its expected revenue under $D$ and $\hat{D}$ respectively. If $D_{ij}$ and $\hat{D}_{ij}$ are both supported on $[0,H]$, and $||D_{ij}-\hat{D}_{ij}||_K\leq \xi$ for all $i\in[n]$ and $j\in[m]$, $$\left|\rev-\widehat{\rev}\right|\leq 2nm\xi\cdot \left(mH+\opt \right).$$	
	\end{theorem}
\begin{proof}
	We use a hybrid argument. Consider a sequence of distributions $\{D^{(i)}\}_{i\leq n}$, where $D^{(i)}=\hat{D}_1\times\cdots\times\hat{D}_i\times D_{i+1}\times\cdots\times D_{n},$ and $D^{(0)}=D$, $D^{(n)}=\hat{D}$.
	 We use $\rev^{(i)}$ to denote the expected revenue of the SPEM under $D^{(i)}$. To prove our claim, it suffices to argue that $\left|\rev^{(i-1)} -\rev^{(i)}\right|\leq 2\xi m\cdot \left(m\cdot H+\opt\right).$
	  We denote by $\SS_k$ and $\SS'_k$ the random set of items that remain available after visiting the first $k$ bidders under $D^{(i-1)}$ and $D^{(i)}$. Clearly, for $k\leq i-1$, $||\SS_k-\SS'_k||_{TV}=0$, so the expected revenue collected from the first $i-1$ bidders under $D^{(i-1)}$ and $D^{(i)}$ is the same. According to Lemma~\ref{lem:stable demand set}, $||\SS_i-\SS'_i||_{TV}\leq 2m\cdot \xi$. The total amount of money bidder $i$ spends can never be higher than her value for receiving all the items which is at most $m\cdot H$. So the difference in the expected revenue collected from bidder $i$ under  $D^{(i-1)}$ and $D^{(i)}$ is at most $2\xi\cdot m^2H$. Suppose $R$ is the set of remaining items after visiting the first $i$ bidders, then the expected revenue collected from the last $n-i$ bidders is the same under  $D^{(i-1)}$ and $D^{(i)}$, as these bidders have the same distributions. Moreover, this expected revenue is no more than $\opt$, since the optimal mechanism can simply just sell $R$ to the last $n-i$ bidders using the same prices and entry fee as in the SPEM we consider. Of course, for any fixed $R$, the probabilities that $\SS_i=R$ and $\SS'_i=R$ are different, but since for any $R$ the expected revenue from the last $n-i$ bidders is at most $\opt$, the difference in the expected revenue from the last $n-i$ bidders under  $D^{(i-1)}$ and $D^{(i)}$ is at most $||\SS_i-\SS'_i||_{TV} \cdot \opt \leq 2\xi\cdot m\opt$. Hence, the total difference between $\rev^{(i-1)}$ and  $\rev^{(i)}$ is at most $2\xi m\cdot \left(m H+\opt\right)$. 
	  %Furthermore, $\left|\rev-\widehat{\rev}(p,\delta)\right|\leq \sum_{i=1}^{n}\left|\rev^{(i-1)}(p,\delta) -\rev^{(i)}(p,\delta)\right|\leq 2 nm\xi \left(mH+\opt_{ASPE}\right).$
\end{proof}


\begin{theorem}\label{thm:constrained additive Kolmogorov}(Max-min Learning for Constrained Additive Bidders)
	When all bidders' valuations are constrained additive over independent items and for any bidder $i$ and any item $j$, $D_{ij}$ and $\hat{D}_{ij}$ are supported on $[0,H]$ and $||D_{ij}-\hat{D}_{ij}||_K\leq \epsilon$  for some $\epsilon=O(\frac{1}{nm})$, then with only access to $\hat{D}=\times_{i,j} \hat{D}_{ij}$, our algorithm can learn an RSPM or ASPE whose revenue is at least $\frac{\opt}{c}-{ \epsilon\cdot O(m^2n H)}$, where $\opt$ is the optimal revenue by any BIC mechanism under $D=\times_{i,j} D_{ij}$. $c>1$ is an absolute constant.
	\end{theorem}
	
Clearly, Theorem~\ref{thm:constrained additive Kolmogorov} also implies a polynomial sample complexity bound for learning an approximately revenue-optimal mechanism. A better sample complexity bound can be obtained directly, i.e.~without invoking the uniform convergence of the revenue of SPEMs, and  is stated as Theorem~\ref{thm:XOS sample} for the broader class of XOS valuations. Similarly, when bidders have simpler valuations, i.e., additive or unit-demand valuations, we can sharpen our results and achieve polynomial-time learnability of the approximately optimal mechanism using more specialized techniques. See Sections~\ref{sec:unit-demand} and~\ref{sec:additive} for details.