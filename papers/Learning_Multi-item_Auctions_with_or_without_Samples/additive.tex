\subsection{Additive Valuations: Polynomial-Time Learning}\label{sec:additive}
In this section, we consider bidders with additive valuations, again sharpening our results to show polynomial-time learnability. It is known that the better of the following two mechanisms achieves at least $\frac{1}{8}$ of the optimal revenue when all bidders have additive valuations~\cite{Yao15,CaiDW16}:

\vspace{.05in}	
\noindent\textbf{Selling Separately}: the mechanism sells each item separately using Myerson's optimal auction.

\vspace{.05in}	
\noindent \textbf{VCG with Entry Fee}: the mechanism solicits bids $\bold{b}=(b_1,\cdots, b_n)$ from the bidders, then offers each bidder $i$ the option to participate for an entry fee $e_i(b_{-i},D_i)$, which is the median of the random variable $\sum_{j\in[m]}(t_{ij}-\max_{k\neq i} b_{kj})^+$, where $t_i\sim D_i$\footnote{The entry fee function defined in~\cite{Yao15,CaiDW16} is slightly different. They showed that there exists an entry fee $X_i$, such that bidder $i$ accepts the entry fee with probability at least $1/2$. Then they argued that extracting $X_i/2$ as the revenue in the VCG with entry fee mechanism is enough to obtain a factor $8$ approximation. It is not hard to observe that our entry fee is accepted with probability exactly $1/2$, thus our entry fee is at least as large as $X_i$. So our mechanism also suffices to provide a factor $8$ approximation.}. This random variable is exactly bidder $i$'s utility when her type is $t_i$ and the other bidders' are $b_{-i}$. If bidder $i$ chooses to participate, she pays the entry fee and can take any item $j$ at price $\max_{k\neq i} b_{kj}$. Notice that the mechanism never over allocate any item, as only the highest bidder for an item can afford it. %Moreover, this mechanism is DSIC, because the entry fee and item prices for bidder $i$ only depend on the other bidders' bids and bidder $i$'s type distribution but not her bid.

Indeed, only counting the revenue from the entry fee in the second mechanism and the optimal revenue from selling the items separately already suffices to provide an $8$-approximation~\cite{Yao15, CaiDW16}. 

\begin{theorem}[\cite{CaiDW16}]\label{thm:UB additive}
	Let $\srev$ be the optimal revenue for selling the items separately and $\brev$ be the expected entry fee collected from the VCG with entry fee mechanism. Then $\opt\leq 6\cdot \srev+2\cdot\brev.$ 
\end{theorem}

Goldner and Karlin~\cite{GoldnerK16} showed that one sample suffices to learn a mechanism that achieves a constant fraction of the optimal revenue when $D_{ij}$ is regular for all $i\in[n]$ and $j\in[m]$. %In the rest of the section, we discuss
We show how to learn an approximately optimal mechanism in the other two models.
% (1) all $D_{ij}$ are supported on $[0,H]$, and (2) direct access to distributions $\hat{D}_{ij}$, where $||\hat{D}_{ij}-D_{ij}||_K\leq \epsilon$ for all $i\in[n]$ and $j\in[m]$.
\begin{theorem}\label{thm:additive}
	When the bidders have additive valuations and\begin{itemize}
		\item $D_{ij}$ is supported on $[0,H]$ for all bidder $i$ and item $j$, we can learn in polynomial time a mechanism whose expected revenue is at least $\frac{\opt}{32}-{\epsilon}\cdot H$ with probability $1-\delta$ given $O\left(\left(\frac{m}{\epsilon}\right)^2 \cdot\left(n\log n\log \frac{1}{\epsilon}+\log\frac{1}{\delta} \right)\right)$ samples from $D$; or
		\item  we are only given access to distributions $\hat{D}_{ij}$ where $||\hat{D}_{ij}-D_{ij}||_K\leq \epsilon$ for all bidder $i$ and item $j$, there is a polynomial time algorithm that constructs a mechanism whose expected revenue under $D$ is at least $\frac{\opt}{266}-96\epsilon\cdot mnH$ when $\epsilon\leq \frac{1}{16\max\{m,n\}}$.
	\end{itemize}
\end{theorem}

\noindent\textbf{Sample Access to Bounded Distributions:} Goldner and Karlin's proof~\cite{GoldnerK16} can be directly applied to the bounded distributions to show a single sample suffices to learn a mechanism whose expected revenue approximates the $\brev$. Then as $\srev$ is the revenue of $m$ separate single-item auctions, we can use the result in~\cite{MorgensternR16} to approximate it. See Theorem~\ref{thm:additive bounded} in Appendix~\ref{sec:additive bounded} for further details.

\vspace{.05in}
\noindent\textbf{Direct Access to Approximate Distributions:} for each single item, we apply Theorem~\ref{thm:unit-demand} to learn an individual auction, then run these learned auctions in parallel. Clearly, the combined auction's revenue approximates $\srev$. For $\brev$, we show that for every bidder $i$ and every bid profile $b_{-i}$ of the other bidders, the event that corresponds to bidder $i$ accepting any entry fee is \emph{single-intersecting} (see Definition~\ref{def:single-intersecting}). This implies that the probability for a bidder to accept an entry fee under $\hat{D}$ and $D$ is close (Lemma~\ref{lem:Kolmogorov stable for sc}). So we can essentially use the median of $\sum_{j\in[m]}(t_{ij}-\max_{k\neq i} b_{kj})^+$ with $t_i\sim\hat{D}_i$ as the entry fee. See Theorem~\ref{thm:additive Kolmogorov} in Appendix~\ref{sec:additive Kolmogorov} for further details.