\section{Preliminaries}\label{sec:prelim}

We focus on revenue maximization in the combinatorial auction with $n$ independent bidders and $m$ heterogenous items. Each bidder has a valuation that is \textbf{subadditive over independent items} (see Definition~\ref{def:subadditive independent}). We denote bidder $i$'s type $t_i$ as $\langle t_{ij}\rangle_{j=1}^m$, where $t_{ij}$ is bidder $i$'s private information about item $j$. For each $i$, $j$, we assume $t_{ij}$ is drawn independently from the distribution $D_{ij}$. Let $D_i=\times_{j=1}^m D_{ij}$ be the distribution of bidder $i$'s type and $D=\times_{i=1}^n D_i$ be the distribution of the type profile. We use $T_{ij}$ (or $T_i, T$) and $f_{ij}$ (or $f_i, f$) to denote the support and density function of $D_{ij}$ (or $D_i, D$). For notational convenience, we let  $t_{-i}$ to be the types of all bidders except $i$. %and $t_{<i}$ (or $t_{\leq i})$ to be the types of the first $i-1$ (or $i$) bidders. 
Similarly, we define $D_{-i}$, $T_{-i}$ 
  and $f_{-i}$ for the corresponding  distributions, support sets and density functions. When bidder $i$'s type is $t_i$, her valuation for a set of items $S$ is denoted by $v_i(t_i,S)$. Throughout the paper we use $\opt$ to denote the optimal revenue obtainable by any randomized and Bayesian truthful mechanism.
  
    \begin{definition}~\cite{RubinsteinW15}\label{def:subadditive independent}
For every bidder $i$, whose type $t_i$ is drawn from a product distribution $F_i=\times_j F_{ij}$, her distribution, $\mathcal{V}_i$, over valuation functions $v_i(t_i,\cdot)$ is \textbf{subadditive over independent items} if:

\vspace{.05in}
\noindent \textbf{- $v_i(\cdot,\cdot)$ has no externalities}, i.e., for each $t_i\in T_i$ and $S\subseteq [m]$, $v_i(t_i,S)$ only depends on $\langle t_{ij}\rangle_{j\in S}$, formally, for any $t_i'\in T_i$ such that $t_{ij}'=t_{ij}$ for all $j\in S$, $v_i(t_i',S)=v_i(t_i,S)$.

\vspace{.05in}
\noindent \textbf{- $v_i(\cdot,\cdot)$ is monotone}, i.e., for all $t_i\in T_i$ and $U\subseteq V\subseteq [m]$, $v_i(t_i,U)\leq v_i(t_i,V)$.

\vspace{.05in}
\noindent \textbf{- $v_i(\cdot,\cdot)$ is subadditive}, i.e., for all $t_i\in T_i$ and $U$, $V\subseteq [m]$, $v_i(t_i,U\cup V)\leq v_i(t_i,U)+ v_i(t_i,V)$.

\vspace{.05in}
 We use $V_i(t_{ij})$ to denote $v_i(t_i,\{j\})$, as it only depends on $t_{ij}$. When $v_i(t_i,\cdot)$ is XOS (or constrained additive) for all $i$ and $t_i\in T_i$, we say $\mathcal{V}_i$ is XOS (or constrained additive) over independent items.
\end{definition}


  \begin{example}\label{eg:valuation}\cite{RubinsteinW15}
We may instantiate Definition~\ref{def:subadditive independent} to define restricted families of subadditive valuations as follows. In all cases, suppose $t=\{t_j\}_{j\in[m]}$ is drawn from $\times_j D_j$. To define a valuation function that is:
\vspace{.05in}

\noindent\textbf{- unit-demand}, we can take $t_j$ to be the value of item $j$, and set $v(t,S)=\max_{j\in S}t_j$.

\vspace{.05in}

\noindent \textbf{- additive}, we can take $t_j$ to be the value of item $j$, and set $v(t,S)=\sum_{j\in S}t_j$.

\vspace{.05in}

\noindent \textbf{- constrained additive}, we can take $t_j$ to be the value of item $j$, and set $v(t,S)=\max_{R\subseteq S, R\in \mathcal{I}}\sum_{j\in R} t_j$, for some downward closed set system ${\cal I} \subseteq 2^{[m]}$.

\vspace{.05in}

\noindent \textbf{- XOS (a.k.a. fractionally subadditive),} we can take $t_j=\{t_{j}^{(k)}\}_{k\in[K]}$ to encode all possible values associated with item $j$, and take $v(t,S)=\max_{k\in[K]}\sum_{j\in S}t_{j}^{(k)}$.

\vspace{.05in}
Note that constrained additive valuations contain additive and unit-demand valuations as special cases, and are contained in XOS valuations.
\end{example}

\noindent\textbf{Distribution Access Models}
\vspace{.05in}

\noindent We consider the following three different models to access the distributions.

\notshow{\vspace{.05in}

\noindent\textbf{Sample access to bounded distributions.} We assume that for any buyer $i$ and any type $t_i\in T_i$, her value $V_i(t_{ij})$ for any single item $j$ lies in $[0,H]$. 

\vspace{.05in}

	\noindent \textbf{Sample access to regular distributions.} We assume that for any buyer $i$ and any type $t_i\in T_i$, the distribution of her value $V_i(t_{ij})$ for any item $j$ is regular.
	
	\vspace{.05in}

	\noindent\textbf{Direct access to approximate distributions.} We assume that we have direct access to a distribution $\hat{D}=\times_{i\in[n], j\in[m]} \hat{D}_{ij}$, that is, we can query the pdf, cdf of $\hat{D}$ and take samples from $\hat{D}$. Moreover, for any buyer $i$ and any type $t_i\in T_i$, the distributions of the random variable $V_i(t_{ij})$ when $t_{ij}$ is sampled from $\hat{D}_{ij}$ or  $D_{ij}$ are within $\epsilon$ in Kolmogorov distance, and both distributions  are supported on $[0,H]$.}
{\begin{itemize}
	\item \textbf{Sample access to bounded distributions.} We assume that for any buyer $i$ and any type $t_i\in T_i$, her value $V_i(t_{ij})$ for any single item $j$ lies in $[0,H]$. 
	\item \textbf{Sample access to regular distributions.} We assume that for any buyer $i$ and any type $t_i\in T_i$, the distribution of her value $V_i(t_{ij})$ for any item $j$ is regular.
	\item \textbf{Direct access to approximate distributions.} We assume that we have direct access to a distribution $\hat{D}=\times_{i\in[n], j\in[m]} \hat{D}_{ij}$, for example we can query the pdf, cdf of $\hat{D}$ and take samples from $\hat{D}$. Moreover, for any buyer $i$ and any type $t_i\in T_i$, the distributions of the random variable $V_i(t_{ij})$ when $t_{ij}$ is sampled from $\hat{D}_{ij}$ or  $D_{ij}$ are within $\epsilon$ in Kolmogorov distance, and both distributions  are supported on $[0,H]$. %When the valuations are constrained additive, $t_{ij}= V_i(t_{ij})$ and we simply use $\hat{F}_{ij}$ and $\hat{f}_{ij}$ to denote the cdf and pdf for $\hat{D}_{ij}$. %\todo{Reword the model for only constrained additive.}%(\yangnote{This definition is sufficient for constrained additive, but for XOS we will need some stronger guarantee.})
\end{itemize}}


\begin{definition} \label{def:Kolm and TV}
The {\em Kolmogorov distance} between two distributions $P$ and $Q$ over $\mathbb{R}$, denoted $|| P-Q||_K$, is defined as $\sup_{x \in \mathbb{R}}|\Pr_{X \sim P}[X \le x]-\Pr_{X \sim Q}[X \le x]|$. The {\em total variation distance} between two probability measures $P$ and $Q$ on a sigma-algebra $\cal F$ of subsets of some sample space $\Omega$, denoted $|| P-Q||_{TV}$, is defined as $\sup_{E \in {\cal F}}|P({E})-Q({E})|$.
\end{definition}
%In the first two models, we aim to minimize the number of samples required to learn an approximately revenue-optimal mechanism. In the last model, our goal is to understand how small $\epsilon$ needs to be in order for us to learn an approximately revenue-optimal mechanism even though we only have access to $\hat{D}$. 

%We first consider the case where buyers have unit-demand and additive valuations, and show that under the three models above learning an approximately revenue-optimal mechanism either requires a polynomial number of samples (in the first two models) or $\epsilon$ only need to be as small as the inverse of some polynomial. Next, we extend our results to constrained additive and XOS valuations. Finally, we show that if the buyers are symmetric, we can learn a mechanism that achieves $\Omega(\frac{n}{\max\{m,n\}})$-fraction of the optimal revenue with either polynomially many samples or access to an approximate distribution with $\epsilon=O(\frac{1}{\poly(n,m)})$ even when the valuations are subadditive. When $\frac{m}{n}$ is a constant, this again provides a constant factor approximation. Moreover, in the case where the distribution for $V_i(t_{ij})$ is regular for every buyer $i$ and item $j$, our mechanism is prior-independent which means it does not require any access to the distribution.
\section{Summary of Our Results}

We summarize our results in the following two tables. Table~\ref{tab:sample-based results} contains all sample-based results and Table~\ref{tab:max-min learning results} contains all results under the max-min learning model. %In particular, when the distributions are regular, all our results only require $\poly(n,m)$ samples from the valuation distributions; when the distributions are arbitrary in $[0,H])$, all our results only require $\poly(n,m,1/\epsilon)$ samples from the valuation distribution, where $\epsilon$ is the additive error in the approximation.
\begin{table}[h]
	\centering
		\begin{tabular}{c || c | c| c| c}
			
			\hline\hline
			Valuations & \# bidders &   Distributions  & Approximation  & Sample Complexity \\
						\hline
												&&&&\\

									 additive~\cite{GoldnerK16} &  $n$  & regular & $\Omega(\opt)$ & $1$ \\	
			  additive &  $n$  & arbitrary $[0,H]$ & $\Omega(\opt)-\epsilon\cdot H$ & $\poly(n,m,1/\epsilon)$	\\					\hline
						&&&&\\

			unit-demand~\cite{MorgensternR16} &  $n$ & arbitrary $[0,H]$ & $\Omega(\opt)-\epsilon\cdot H$ & $\poly(n,m,1/\epsilon)$\\
			unit-demand &  $n$ & regular & $\Omega(\opt)$ & $\poly(n,m)$ \\
				\hline
										&&&&\\

			constrained additive &  $n$ &	 arbitrary $[0,H]$ & $\Omega(\opt)-\epsilon\cdot H$ & $\poly(n,m,1/\epsilon)$ \\

			constrained additive &  $n$ &	regular & $\Omega(\opt)$ & $\poly(n,m)$\\
			\hline
										&&&&\\
										XOS &  $n$ &	 arbitrary $[0,H]$ & $\Omega(\opt)-\epsilon\cdot H$ & $\poly(n,m,1/\epsilon)$\\

			XOS &  $n$ &	regular & $\Omega(\opt)$ & $\poly(n,m)$\\
			\hline
										&&&&\\

			subadditive~\cite{MorgensternR16} & $1$ & arbitrary $[0,H]$  &  $\Omega(\opt)-\epsilon\cdot H$ &$\poly(m,1/\epsilon)$\\
						subadditive & $n$ i.i.d. & arbitrary $[0,H]$  &  $\Omega\left(\frac{n}{\max\{n,m\}}\right)\cdot \opt-\epsilon\cdot H$ &$\poly(n,m,1/\epsilon)$\\
						subadditive & $n$ i.i.d. & regular & $\Omega\left(\frac{n}{\max\{n,m\}}\right)\cdot \opt$ & \emph{prior-independent}\\
\hline
		\end{tabular}
		%When the distributions are \textbf{regular}, all our results only require $\poly(n,m)$ samples from the valuation distributions. When the distributions are \textbf{arbitrary in $[0,H]$}, all our results only require $\poly(n,m,1/\epsilon)$ samples from the valuation distribution.
		
		%*For i.i.d. subadditive bidders, our mechanism is \emph{prior-independent} when the item marginals are regular.
	\caption{Summary of Our Sample-based Results.}
	\label{tab:sample-based results}
\end{table}

\begin{table}[h]
	\centering
		\begin{tabular}{c || c | c| c}
			
			\hline\hline
			Valuations & \# bidders &   Distributions  & Approximation \\
						\hline
												&&&\\
			  additive &  $n$  & arbitrary $[0,H]$ & $\Omega(\opt)-O(\epsilon\cdot n\cdot m\cdot H)$ 	\\					\hline
						&&&\\

			unit-demand&  $n$ & arbitrary $[0,H]$ & $\Omega(\opt)-O(\epsilon\cdot n\cdot m\cdot H)$  \\
				\hline
										&&&\\

			constrained additive &  $n$ &	 arbitrary $[0,H]$ & $\Omega(\opt)-O(\epsilon\cdot n\cdot m^2\cdot H)$\\

			\hline
										&&&\\


						subadditive & $n$ i.i.d. & arbitrary $[0,H]$  & $\Omega\left(\frac{n}{\max\{n,m\}}\right)\cdot \opt-O(\epsilon\cdot n \cdot m\cdot H)$\\
\hline
		\end{tabular}

	\caption{Summary of Our Max-min Learning Results.}
	\label{tab:max-min learning results}
\end{table}