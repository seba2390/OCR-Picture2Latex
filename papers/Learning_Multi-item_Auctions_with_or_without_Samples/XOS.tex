\section{XOS Valuations} \label{sec:constrained additive}
In this section we go beyond constained additive valuations to show learnability of approximately revenue-optimal auctions from polynomially many samples. The better of the following two mechanisms is known to achieve a constant fraction of the optimal revenue, when bidders have valuations that are XOS over independent items~\cite{CaiZ17}.

%\begin{itemize}
\vspace{.05in}
\noindent\textbf{Rationed Sequential Posted Price Mechanism (RSPM)}: the mechanism is almost the same as SPM in Algorithm~\ref{alg:seq-mech}, except there is an extra constraint that every bidder can purchase at most one item.

\vspace{.05in}
\noindent\textbf{Anonymous Sequential Posted Price with Entry Fee Mechanism (ASPE)}: every buyer faces the same collection of item prices $\{p_j\}_{j\in[m]}$. The seller visits the bidders sequentially. For every bidder, the seller shows her all the available items (i.e. items that have not yet been purchased) and the associated price for each item, then asks her to pay a personalized entry fee which depends on her type distribution and the set of available items. If the bidder accepts the entry fee, she can proceed to purchase any available item at the given price; if she rejects the entry fee, she neither receives nor pays anything. See Algorithm~\ref{alg:aspe-mech} for details.
%\end{itemize}

\begin{theorem}\cite{CaiZ17}\label{thm:simple XOS}
	There exists a collection of prices $\{p^*_j\}_{j\in[m]}$, such that if we set the entry fee function $\delta^*_i(S)$ to be the median of bidder $i$'s utility for set $S$, either the ASPE$(p^*,\delta^*)$ or the best RSPM achieves at least a constant fraction of the optimal revenue when bidders' valuations are XOS over independent items. More formally, let $u^*_i(t_i, S)=\max_{S^*\subseteq S} v_i(t_i, S^*)-\sum_{j\in S^*} p^*_j$ be bidder $i$'s utility for the set of items $S$ when her type is $t_i$. We define $\delta^*_i(S)$ to be the median of the random variable $u^*_i(t_i,S)$ (with $t_i\sim D_i$) for any set $S\subseteq [m]$.  Moreover, the price $p^*_j$ for any item $j$ is no larger than $2G$, where $G = \max_{i,j} G_{ij}$ and $G_{ij}:= \sup_x \left\{ \Pr_{t_{ij}\sim D_{ij}}\left[V_i(t_{ij})\geq x\right]\geq \frac{1}{5\max\{m,n\}}\right\}$.
\end{theorem}

{%We provide an algorithm to learn both mechanisms when we are given direct access to an approximate distribution for constrained additive bidders in Section~\ref{sec:constrained additive kolomogorov}. 
Our goal next is to bound the sample complexity for learning a near-optimal RSPM and the ASPE described in Theorem~\ref{thm:simple XOS} under XOS valuations.} %See Appendix~\ref{sec:XOS bounded} for details.} 

We consider first the task of learning a near-optimal RSPM. In a RSPM, all bidders are restricted to be unit-demand, so the revenue of the best RSPM is upper bounded by the optimal revenue in the corresponding unit-demand setting. In Section~\ref{sec:unit-demand}, we have shown how to learn an approximately optimal mechanism for unit-demand bidders, and those algorithms can be used to approximate the best RSPM. 

So, for the rest of this section, it suffices to focus on learning an ASPE whose revenue approximates the revenue of the ASPE described in Theorem~\ref{thm:simple XOS}.  We will do this in Section~\ref{sec:XOS sample}. Before that,
%One might worry that the entry fee function could be complex and thus difficult to learn. It turns out if we choose the correct item prices $\{p_j\}_{j\in [m]}$, simply setting the entry fee to be the median of the bidder's utility for the available items suffices to obtain good revenue~\cite{CaiZ17}. More specifically, let $u_i(t_i, S)=\max_{S^*\subseteq S} v_i(t_i, S^*)-\sum_{j\in S^*} p_j$ be bidder $i$'s utility for the set of items $S$ and we define $\delta_i(S)$ to be the median of the random variable $u_i(t_i,S)$ while $t_i\sim D_i$ for any subset $S\subseteq [m]$.
we need a robust version of Theorem~\ref{thm:simple XOS}. In the next Lemma, we argue that if we use a collection of prices $\{p'_j\}_{j\in[m]}$ sufficiently close to $\{p^*_j\}_{j\in[m]}$ and entry fee $\delta'_i(S)$ sufficiently close to the median of the utility for every bidder $i$ and subset $S$, the better of the corresponding ASPE and the best RSPM still approximates the optimal revenue. We postpone the proof to Appendix~\ref{sec:appx XOS}.

\begin{lemma}\label{lem:approx ASPE}
	For any $\epsilon>0$ and $\mu\in[0,\frac{1}{4}]$, let $\{p'_j\}_{j\in[m]}$ be a collection of prices such that $|p'_j-p^*_j|\leq \epsilon$ for all $j\in[m]$, where $\{p^*_j\}_{j\in[m]}$ is the collection of prices in Theorem~\ref{thm:simple XOS}. Let $\delta'_i(S)$ be bidder $i$'s entry fee function such that $\Pr_{t_i\sim D_i}\left [u'_i(t_i,S)\geq \delta'_i(S)\right]\in [1/2-\mu,1/2+\mu]$ for any set $S\subseteq [m]$, where $u'_i(t_i,S) = \max_{S*\subseteq S} v_i(t_i,S^*)-\sum_{j\in S^*} p'_j$. Then, either the ASPE$(p',\delta')$ or the best RSPM achieves revenue at least $\frac{\opt}{\CC_1(\mu)}-\CC_2(\mu)\cdot (m+n)\cdot \epsilon$ when bidders' valuations are XOS over independent items. Both $\CC_1(\cdot)$ and $\CC_2(\cdot)$ are  monotonically increasing functions that only depend on $\mu$. 
\end{lemma}

\begin{definition}\label{def:eps mu ASPE}
	We say a collection of prices $\{p_j\}_{j\in[m]}$ is in the $B$-bounded $\epsilon$-net if $p_j$ is a multiple of $\epsilon$ and no larger than $B$ for any item $j$. For any collection of prices $\{p_j\}_{j\in[m]}$, we say the entry fee functions are $\mu$-balanced if for every bidder $i$ and every set $S\subseteq [m]$, her entry fee $\delta_i(S)$ satisfies  $\Pr_{t_i\sim D_i}[u_i(t_i,S)\geq \delta_i(S)]\in [1/2-\mu,1/2+\mu]$, where $u_i(t_i,S) = \max_{S*\subseteq S} v_i(t_i,S^*)-\sum_{j\in S^*} p_j$.
\end{definition}
%An easy corollary of Lemma~\ref{lem:approx ASPE} is that there exists an mechanism ASPE$(p,\delta)$ with $p$ lying in the $\epsilon$-net and that provides high revenue. More specifically,

\begin{corollary}\label{cor:discretization of prices}
	For  bidders with valuations that are XOS over independent items and any $\epsilon>0$, there exists a collection of prices $\{p_j\}_{j\in[m]}$ in the $2G$-bounded $\epsilon$-net such that for any $\mu$-balanced entry fee functions $\{\delta_i(\cdot)\}_{i\in[n]}$ with $\mu\in[0,\frac{1}{4}]$, either the ASPE$(p,\delta)$ or the best RSPM achieves revenue at least $\frac{\opt}{\CC_1(\mu)}-\CC_2(\mu)\cdot (m+n)\cdot \epsilon$. %where $\CC_1(\cdot)$ and $\CC_2(\cdot)$ are  monotonically increasing functions that only depend on $\mu$.
\end{corollary}

%In other words, if for every collection of prices $\{p_j\}_{j\in[m]}$ in the $\epsilon$-net we construct a set of $\mu$-balanced entry fee functions $\{\delta_i(\cdot)\}_{i\in[n]}$ for ${p}$, as guaranteed by Corollary~\ref{cor:discretization of prices}, the best mechanism among these ASPE$(p,\delta)$  has high revenue. 

\notshow{
\subsection{Constrained Additive Valuations: direct access to approximate distributions}\label{sec:constrained additive kolomogorov}

In this section, we consider the model where we only have access to an approximate distribution $\hat{D}$ and bidders have constrained additive valuations. %\cnote{TO TAKE OUT: As when the valuations are XOS over independent items, a bidders' type is not necessarily described as vectors in some Euclidean space. In particular, a bidder's type could be some abstract information specifying her preference\footnote{For example, the type distribution could be a categorical distribution about the color of the item.} about the items. For these distributions, our model does not apply, as it is not even clear how to define the Kolmogorov distance between any two such distributions.} 
%We restrict our attention to an important valuation class: constrained additive valuations. \yangnote{We will sketch the proof for the general XOS valuations in the next section.} %Since a constrained additive bidders' type is still a vector in $\mathbb{R}^m$, our model applies.
Our learning algorithm is a two-step procedure. In the first step, we argue that the approximate distribution $\hat{D}$ can be used to define a balanced entry fee function for each bidder under any prices. In particular, we propose to use the median of bidder $i$'s utility under $\hat{D}_i$ to compute the entry fees. The challenge is to show that, whenever $||D_{ij}-\hat{D}_{ij}||_{K}\leq \xi$ for every bidder $i$ and item $j$, the median of the utility of any bidder $i$ for any subset $S$ under any collection of prices $\{p_j\}_{j\in[m]}$ is not too different when the bidder's type is drawn from $D_i$ or $\hat{D}_i$; in particular, the entry fee functions thus defined from $\hat{D}$ are balanced. This is shown in Lemma~\ref{lem:Kolmogorov learn entry fee}. Given this lemma, we are able to construct an ASPE for every collection of prices $\{p_j\}_{j\in[m]}$ in the $\epsilon$-net. In the second step, we need to identify a mechanism  with high revenue among all ASPEs we constructed. If we have sample access to the  actual distribution, this is a simple task {as we can take a polynomial number of samples from $D$ and argue that with probability almost $1$, the empirical revenue of ASPE$(p,\delta)$ is close to its true expected revenue for every price vector $p$ in the $\epsilon$-net. So the ASPE with the highest empirical revenue also has high expected revenue.} %as we have shown in Lemma~\ref{lem:learn best ASPE}.
 But we only have access to an approximate distribution $\hat{D}$; can we argue that the expected revenue of any ASPE does not change much under $D$ or $\hat{D}$? Notice that this is certainly not true for arbitrary mechanisms, as it is easy to construct a DSIC single item auction with two bidders where the expected revenue under two distributions that are close in Kolmogorov distance is far away%\footnote{\todo{Write down this example later.}}
 . Surprisingly, we can argue that the expected revenue of any ASPE thus defined is within $\poly(n,m)\cdot H\cdot \xi$ when the bidders' types are drawn from $D$ or $\hat{D}$.

 To argue these results, we need to understand how the probability of events changes from distribution $D$ to $\hat{D}$. Clearly, if we consider arbitrary events, the difference in their probabilities under $D$ and $\hat{D}$ can be arbitrarily far. We need to identify structure in the events that are of interest to us, which allows us to argue that computations under $D$ and $\hat{D}$ are close. Since our distributions are supported on a subset of the Euclidean space, any event is simply a set in the Euclidean space. In Definition~\ref{def:single-intersecting}, we define a type of events called \emph{single-intersecting} events based on the geometric shape of the corresponding sets. For example, balls, rectangles and all convex sets are single-intersecting, but this definition includes some non-convex sets as well, for example, cross-shaped sets. It turns out that being able to include these non-convex sets is crucial for our result, as many events we care about are not convex but nonetheless are single-intersecting. Next we argue that for any single-intersecting event, the difference of the probability under $D$ and $\hat{D}$ only grows linearly in the dimension of the support space (Lemma~\ref{lem:Kolmogorov stable for sc}).
%This result is facilitated by combining an interesting property about Kolmogorov distance in product space and the geometry of bidders' behavior in ASPEs \footnote{\yangnote{I'm not sure what I was trying to say here... This is just a place holder to remind us write something about the proof.}}.
\notshow{\begin{definition}[Single-intersecting Events]\label{def:single-intersecting}
For any event $\EE$ in $\mathbb{R}^{\ell}$, $\EE$ is \textbf{single-intersecting} if the intersection of $\EE$ and any line that is parallel to one of the axes is an interval. More formally, for any $i\in[\ell]$ and any line $L_i=\left\{x \in \mathbb{R}^{\ell}\ |\ x_{-i}=a_{-i} \right \}$, where $a_{-i}$ is some $\ell-1$ dimensional vector in $\mathbb{R}^{\ell-1}$, the intersection of $L_i$ and $\EE$ is $\left\{x \in \mathbb{R}^{\ell}\ |\ x_{-i}=a_{-i}, x_i\in[\ubar{a},\bar{a}] \right\}$ for some real numbers $\ubar{a}$ and $\bar{a}$.
\end{definition}


\begin{lemma}\label{lem:Kolmogorov stable for sc}
For any integer $\ell$, let $\DD=\times_{i=1}^\ell \DD_i$ and $\hat{\DD}=\times_{i=1}^\ell \hat{\DD}_i$, where $\DD_i$ and $\hat{\DD}_i$ are both supported on $[0,H]$ for any $i\in[\ell]$. If $||\DD_i-\hat{\DD}_i||_K\leq \xi$, $\left|\Pr_\DD[\EE]-\Pr_{\hat{\DD}}[\EE]\right|\leq 2\xi\cdot \ell$ for any single-intersecting event $\EE$.
\end{lemma}
\begin{prevproof}{Lemma}{lem:Kolmogorov stable for sc} Let $\FF_i$ and $\hat{\FF}_i$ be the cdf for $\DD_i$ and $\hat{\DD}_i$ respectively. We will prove the statement using a hybrid argument. We create a sequence of product distributions $\{\DD^{(j)}\}_{j\leq \ell}$, where $\DD^{(j)}=\hat{\DD}_1\times\cdots\times\hat{\DD}_j\times \DD_{j+1}\times\cdots\times \DD_{\ell},$ and $\DD^{(0)}=\DD$, $\DD^{(\ell)}=\hat{\DD}$. To prove our claim, it suffices to show that for any integer $j\in[\ell]$, $\left|\Pr_{\DD^{(j-1)}}[\EE]-\Pr_{{\DD}^{(j)}}[\EE]\right|\leq 2\xi.$
	Let us first fix some notations. Let $\EE_{-j}=\left\{x_{-j}\ |\ \exists x_j\in \mathbb{R},\ (x_j,x_{-j})\in \EE\right\}$, $\EE_j(a_{-j})= \left\{x_{j} \in \mathbb{R} \ |\ (x_j,a_{-j})\in \EE\right\}$ for all $j\in[\ell]$. As $\EE$ is single-intersecting, $\EE_j(x_{-j})$ is an interval for all $j$ and $x_{-j}$. Moreover, since $||\DD_j-\hat{\DD}_j||_K\leq \xi$, we have $|\Pr_{\DD_j}[x_j\in \EE_j(x_{-j})]-\Pr_{\hat{\DD}_j}[x_j\in \EE_j(x_{-j})]|\leq 2\xi$ for all $j$ and $x_{-j}$. Next, we bound the difference for the probability of event $\EE$ under $\DD^{(j-1)}$ and $\DD^{(j)}$. 
	\begin{align*}
		&\left|\Pr_{\DD^{(j-1)}}[\EE]-\Pr_{{\DD}^{(j)}}[\EE]\right|\\
		=&\Big{|}\int_{\mathbb{R}^{\ell-1}}\ind{\left [x_{-j}\in \EE_{-j}\right]}\left(\int_{\mathbb{R}}\ind{\left[x_j\in \EE_j(x_{-j})\right]}d \FF_j(x_j)\right) d\hat{\FF}_1(x_1)\cdots d\hat{\FF}_{j-1}(x_{j-1}) d\FF_{j+1}(x_{j+1})\cdots d\FF_{\ell}(x_{\ell})\\
		 & - \int_{\mathbb{R}^{\ell-1}}\ind{\left [x_{-j}\in \EE_{-j}\right]}\left(\int_{\mathbb{R}}\ind{\left[x_j\in \EE_j(x_{-j})\right]}d \hat{\FF}_j(x_j)\right) d\hat{\FF}_1(x_1)\cdots d\hat{\FF}_{j-1}(x_{j-1}) d\FF_{j+1}(x_{j+1})\cdots d\FF_{\ell}(x_{\ell})\Big{|}\\
		 = & \left|\int_{\mathbb{R}^{\ell-1}}\ind{\left [x_{-j}\in \EE_{-j}\right]} \left(\Pr_{\DD_j}[x_j\in \EE_j(x_{-j})]-\Pr_{\hat{\DD}_j}[x_j\in \EE_j(x_{-j})]\right)
		  d\hat{\FF}_1(x_1)\cdots d\hat{\FF}_{j-1}(x_{j-1}) d\FF_{j+1}(x_{j+1})\cdots d\FF_{\ell}(x_{\ell})\right|\\
		  \leq & 2\xi\cdot\int_{\mathbb{R}^{\ell-1}}\ind{\left [x_{-j}\in \EE_{-j}\right]} d\hat{\FF}_1(x_1)\cdots d\hat{\FF}_{j-1}(x_{j-1}) d\FF_{j+1}(x_{j+1})\cdots d\FF_{\ell}(x_{\ell}) = 2\xi
	\end{align*}
	\end{prevproof}}
	%The second last inequality is because $\EE$ is single-intersecting so $\EE_j(x_{-j})$ is an interval for all $j$ and $x_{-j}$, also because $||\DD_j-\hat{\DD}_j||_K\leq \xi$, so $$
	It is quite easy to see that the single-intersecting condition can be relaxed to $k$-intersecting where the intersection of the event $\EE$ with any line parallel to an axis is the union of at most $k$ intervals. Indeed, using a proof almost identical to the proof of Lemma~\ref{lem:Kolmogorov stable for sc}, we can argue that the difference in the probability of an event $\EE$ under $\DD$ and $\hat{\DD}$ only grows linearly in the dimension of the support space and $k$. Next, we formally state the first step of our learning algorithm. The proof can be found in Appendix~\ref{sec:appx constrained additive}.
\begin{lemma}\label{lem:Kolmogorov learn entry fee}
Suppose $||D_{ij}-\hat{D}_{ij}||_K\leq \xi$ for any bidder $i$ and any item $j$. For any collection of prices $\{p_j\}_{j\in[m]}$, let $u^{(p)}_i(t_i,S)=\max_{S*\subseteq S} v_i(t_i,S^*)-\sum_{j\in S^*} p_j$. Define the entry fee $\delta_i^{(p)}(S)$
 of bidder $i$ for set $S$ under $\{p_j\}_{j\in[m]}$ to be the median of $u^{(p)}_i(t_i,S)$ when bidder $i$'s type $t_i$ is drawn from $\hat{D}_i$. Then $\left\{\delta_i^{(p)}(\cdot)\right\}_{i\in[n]}$ is a collection of $2m\xi$-balanced entry fee functions for any collection of prices $\{p_j\}_{j\in[m]}$.\end{lemma}

So far, for any collection of prices $\{p_j\}_{j\in[m]}$, we have defined (using $\hat{D}$) a collection of $2m\xi$-balanced entry fee functions $\{\delta_i^{(p)}(\cdot)\}_{i\in[n]}$. Next, we need to identify prices $\{p_j\}_{j\in[m]}$ such that the corresponding ASPE mechanism achieves high revenue under the actual distribution $D$. Our goal is to show that the expected revenue of simultaneously all ASPE mechanisms, defined for all prices using $\hat{D}$ as above, is not much different under $D$ and $\hat{D}$. We first establish a technical lemma, which states that, for any set of available items $S$ and entry fee $\delta_i(S)$, the distribution over the {\em set of items} purchased by bidder $i$ under $D_i$ and $\hat{D}_i$ has total variation distance at most $2m\xi$. This is quite surprising. Given that, for each set of items $S' \subseteq S$, the difference in the probability that the buyer will purchase this particular set $S'$ under $D_i$ and $\hat{D}_i$ could already be as large as {$\Theta(m\xi)$}, and the distribution has an exponentially large support size,  a trivial argument would give a bound of {$2^m\cdot \Theta(m\xi)$}. To overcome this analytical difficulty, we argue instead that for any collection of sets of items, the event that the buyer's favorite set lies in this collection is single-intersecting. Then our result follows from Lemma~\ref{lem:Kolmogorov stable for sc}. Notice that it is crucial that Lemma~\ref{lem:Kolmogorov stable for sc} holds for all events that are single-intersecting, as the event we consider here is clearly non-convex in general. 

\notshow{
\begin{lemma}\label{lem:stable favorite set}
	For any bidder $i$, any set $S\subseteq [m]$, any prices $\{p_j\}_{j\in[m]}$ and entry fee $\delta_i(S)$, let $\LL$ and $\hat{\LL}$ be the distributions over the set of items purchased from $S$ by bidder $i$ under prices $\{p_j\}_{j\in[m]}$ and entry fee $\delta_i(S)$ when her type is drawn from $D_i$ and $\hat{D}_i$ respectively. If $||D_{ij}-\hat{D}_{ij}||_K\leq \xi$ for all item $j$, $||\LL-\hat{\LL}||_{TV}\leq 2m \xi.$
\end{lemma}
\begin{proof}
	For any set $R\subseteq S$, let $\EE_R$ be the event that bidder $i$ purchases set $R$. Proving that the total variation distance between $\LL$ and $\hat{\LL}$ is no more than $2m\cdot \xi$ is the same as proving that for any $K\leq 2^{|S|}$, $\left|\ \Pr_{D_i}\left[t_i\in \bigcup_{\ell=1}^K \EE_{R_\ell}\right]-\Pr_{\hat{D}_i}\left[t_i\in \bigcup_{\ell=1}^K \EE_{R_\ell}\right]\right|\leq 2m\cdot \xi$ where $R_1,\cdots R_K$ are arbitrary distinct subsets of $S$. Since the dimension of the bidder's type space is $m$, if we can prove that $\bigcup_{\ell=1}^K \EE_{R_\ell}$ is always single-intersecting, our claim follows from Lemma~\ref{lem:Kolmogorov stable for sc}. 
	
	For any $j\in[m]$ and $a_{i,-j}\in [0,H]^{{m}-1}$, let $L_j(a_{i,-j})=\left\{ (t_{ij},a_{i,-j}) \ | t_{ij}\in [0,H] \right\}$. We claim that $L_j(a_{i,-j})$ intersects with at most two different $\EE_{U}$ and $\EE_{V}$ where $U$ and $V$ are subsets of $S$. 
	WLOG, we assume that  $(0,a_{i,-j})\in \EE_U$. 
	\begin{itemize}
		\item If $U=\emptyset$, that means the utility of the favorite set for type $(0,a_{i,-j})$ is smaller than the entry fee $\delta_i(S)$. If we increase the value of $t_{ij}$, two cases could happen: (1) the utility of the favorite set is still lower than the entry fee; (2) the utility of the favorite set is higher than the entry fee. In case (1), $(t_{ij},a_{i,-j})\in \EE_{\emptyset}$. In case (2), bidder $i$ pays the entry fee and purchases her favorite set $V$. Then item $j$ must be in $V$, because otherwise the utility for set $V$ does not change from type $(0,a_{i,-j})$ to type $(t_{ij},a_{i,-j})$. If we keep increasing $t_{ij}$, bidder $i$'s favorite set remains $V$ and she keeps paying the entry fee and purchasing $V$. Hence, $L_j(a_{i,-j})$ can intersect with at most one event $\EE_R$ where $R$ is non-empty.
		\item If $U\neq \emptyset$, that means $U$ is the favorite set of type $(0,a_{i,-j})$ and the utility for winning set $U$ is higher than the entry fee.  If we increase the value of $t_{ij}$, two cases could happen: (1) $U$ remains the favorite set; (2) a different set $V$ becomes the new favorite set. In case (1), $(t_{ij},a_{i,-j})\in \EE_U$. In case (2), item $j$ must be in $V$ and not in $U$, otherwise how could $U$ be better than $V$ for type $(0,a_{i,-j})$ but worse for type $(t_{ij},a_{i,-j})$.  If we keep increasing $t_{ij}$, bidder $i$'s favorite set remains to $V$ and she keeps paying the entry fee and purchasing $V$. Hence, $L_j(a_{i,-j})$ can intersect at most two different events.
	\end{itemize}
	
It is not hard to see that any event $\EE_R$ is an intersection of halfspaces, so the intersection of $L_j(a_{i,-j})$ with any event $\EE_R$ is an interval. {Also, notice that any type $t_i$ must lie in an event $\EE_R$ for some set $R\subseteq S$.} If $L_j(a_{i,-j})$ intersects with two different events $\EE_U$ and $\EE_V$, the two intersected intervals must lie back to back on $L_j(a_{i,-j})$. Otherwise, $L_j(a_{i,-j})$ intersects with at least three different events. Contradiction. Since $L_j(a_{i,-j})$ intersects with at most two different events, no matter which of these events are in $\{\EE_{R_\ell}\}_{\ell\in[K]}$, the intersection of $L_j(a_{i,-j})$ and $\bigcup_{\ell=1}^K \EE_{R_\ell}$ is either empty or an interval,%\footnote{\cnote{Should we add some extra explanation here, that if it intersects two events these must be back to back, so the intersection is not going to be the union of two disjoint intervals?}}
  which means $\bigcup_{\ell=1}^K \EE_{R_\ell}$ is single-intersecting. Now our claim simply follows from Lemma~\ref{lem:Kolmogorov stable for sc}.
\end{proof}
}

With Lemma~\ref{lem:stable favorite set}, we are ready to show that the revenue of any ASPE does not change much under $D$ and $\hat{D}$.\begin{lemma}\label{lem:difference in revenue Kolmogorov}
	For any ASPE$(p,\delta)$, let $\rev(p,\delta)$ and $\widehat{\rev}(p,\delta)$ be its expected revenue under $D$ and $\hat{D}$ respectively. If $||D_{ij}-\hat{D}_{ij}||_K\leq \xi$ for all $i\in[n]$ and $j\in[m]$, then $|\rev(p,\delta)-\widehat{\rev}(p,\delta)|\leq 2nm\xi\cdot \left(mH+\opt_{ASPE}\right)$,
	{where $\opt_{ASPE}$ is the optimal revenue obtainable by an ASPE mechanism.}
	\end{lemma}
\begin{prevproof}{Lemma}{lem:difference in revenue Kolmogorov}
	We use a hybrid argument. Consider a sequence of distributions $\{D^{(i)}\}_{i\leq n}$, where $D^{(i)}=\hat{D}_1\times\cdots\times\hat{D}_i\times D_{i+1}\times\cdots\times D_{n},$ and $D^{(0)}=D$, $D^{(n)}=\hat{D}$.
	 We use $\rev^{(i)}(p,\delta)$ to denote the expected revenue of ASPE$(p,\delta)$ under $D^{(i)}$. To prove our claim, it suffices to argue that $\left|\rev^{(i-1)}(p,\delta) -\rev^{(i)}(p,\delta)\right|\leq 2\xi m\cdot \left(m\cdot H+\opt\right).$
	  We denote by $\SS_k$ and $\SS'_k$ the random set of items that remain available after visiting the first $k$ bidders under $D^{(i-1)}$ and $D^{(i)}$. Clearly, for $k\leq i-1$, $||\SS_k-\SS'_k||_{TV}=0$, so the expected revenue collected from the first $i-1$ bidders under $D^{(i-1)}$ and $D^{(i)}$ is the same. According to Lemma~\ref{lem:stable favorite set}, $||\SS_i-\SS'_i||_{TV}\leq 2m\cdot \xi$. The total amount of money bidder $i$ spends can never be higher than her value for receiving all the items which is at most $m\cdot H$. So the difference in the expected revenue collected from bidder $i$ under  $D^{(i-1)}$ and $D^{(i)}$ is at most $2\xi\cdot m^2H$. Suppose $R$ is the set of remaining items after visiting the first $i$ bidders, then the expected revenue collected from the last $n-i$ bidders is the same under  $D^{(i-1)}$ and $D^{(i)}$, as these bidders have the same distributions. Moreover, this expected revenue is no more than $\opt_{ASPE}$, since the optimal ASPE can simply just sell $R$ to the last $n-i$ bidders using the same prices and entry fee as in ASPE$(p,\delta)$. Of course, for any fixed $R$, the probabilities that $\SS_i=R$ and $\SS'_i=R$ are different, but since for any $R$ the expected revenue from the last $n-i$ bidders is at most $\opt_{ASPE}$, the difference in the expected revenue from the last $n-i$ bidders under  $D^{(i-1)}$ and $D^{(i)}$ is at most $||\SS_i-\SS'_i||_{TV} \cdot \opt \leq 2\xi\cdot m\opt_{ASPE}$. Hence, the total difference between $\rev^{(i-1)}(p,\delta)$ and  $\rev^{(i)}(p,\delta)$ is at most $2\xi m\cdot \left(m H+\opt_{ASPE}\right)$. 
	  Furthermore, $\left|\rev(p,\delta)-\widehat{\rev}(p,\delta)\right|\leq \sum_{i=1}^{n}\left|\rev^{(i-1)}(p,\delta) -\rev^{(i)}(p,\delta)\right|\leq 2 nm\xi \left(mH+\opt_{ASPE}\right).$
\end{prevproof}

Using  Lemma~\ref{lem:difference in revenue Kolmogorov}, we can prove the main Theorem of this section. The proof is postponed to Appendix~\ref{sec:appx constrained additive}.

\begin{theorem}\label{thm:constrained additive Kolmogorov}
	When all bidders' valuations are drawn from distributions that are constrained additive, and for any bidder $i$ and any item $j$, $D_{ij}$ and $\hat{D}_{ij}$ are supported on $[0,H]$ and $||D_{ij}-\hat{D}_{ij}||_K\leq \xi$  for some $\xi=O(\frac{1}{nm})$, then with only access to $\hat{D}=\times_{i,j} \hat{D}_{ij}$, our algorithm can learn an RSPM and an ASPE such that the better of the two mechanisms has revenue at least $\frac{\opt}{c}-{\xi\cdot O(m^2n H)}$, where $\opt$ is the optimal revenue by any BIC mechanism under $D=\times_{i,j} D_{ij}$. $c>1$ is an absolute constant.%\footnote{\cnote{I'm confused why I don't see an $n^2$ in the error term (which would come from the RSPM error term), and also why there is no constant factor in front of the error term.} \yangnote{I added the constant factor in front of the error term, but I'm not sure why there will be an $n^2$ error term from RSPM is $\xi = O(1/nm)$.}} %\yangnote{With the same number of samples, we can learn in polynomial time with probability $1-\delta$ an SPM whose revenue is at least $\frac{\opt}{144}-\epsilon H$.}
\end{theorem}

}

\subsection{XOS Valuations: sample access to bounded and regular distributions}\label{sec:XOS sample}
In this section, we consider how to learn an ASPE with high revenue given sample access to $D$. Our learning algorithm is a two-step procedure. In the first step, we take a few samples from $D$ and use these samples to set the entry fee for every collection of prices $\{p_j\}_{j\in[m]}$ in the $\epsilon$-net. More specifically, to decide $\delta_i(S)$ we compute the utility of bidder $i$ for set $S$ under $\{p_j\}_{j\in[m]}$ over all the samples and take the empirical median among all these utilities to be $\delta_i(S)$. With a polynomial number of samples, we can guarantee that for any $\{p_j\}_{j\in[m]}$ in the $\epsilon$-net the computed entry fee functions $\{\delta_i(\cdot)\}_{i\in[n]}$ are $\mu$-balanced. Now, we have created an ASPE for every $\{p_j\}_{j\in[m]}$ in the $\epsilon$-net. In the second step, we take some fresh samples from $D$ and use them to estimate the revenue for each of the ASPEs we created in the first step,  then pick the one that has the highest empirical revenue. It is not hard to argue that with a polynomial number of samples the mechanism we pick has high revenue with probability almost $1$. Combining our algorithm with Theorem~\ref{thm:unit-demand}, we obtain the following theorem.
\begin{theorem}\label{thm:XOS sample}
	When all bidders' valuations are XOS over independent items and
	\begin{itemize}
		\item the random variable $V_i(t_{ij})$ is supported on $[0,H]$ for each bidder $i$ and item $j$, we can learn an RSPM and an ASPE such that with probability at least $1-\delta$ the better of the two mechanisms has revenue at least $\frac{\opt}{c_1}-\xi\cdot H$ for some absolute constant $c_1>1$ given $O\left((\frac{mn}{\xi})^2 \cdot (m\cdot\log \frac{m+n}{\xi} + \log \frac{1}{\delta})\right)$ samples from $D$; 
		\item  the random variable $V_i(t_{ij})$ is regular for each bidder $i$ and item $j$, we can learn an RSPM and an ASPE such that with probability at least $1-\delta$ the better of the two mechanisms has revenue at least $\frac{\opt}{c_2}$ for some absolute constant $c_2>1$ given $O\left(\max\{m,n\}^2m^2n^2 \left(m\log ({m+n}) + \log \frac{1}{\delta}\right)\right)$  samples from $D$.
	\end{itemize}
\end{theorem}

 The bounded case is proved as Theorem~\ref{thm:XOS bounded} in Appendix~\ref{sec:XOS bounded}.  The regular case is proved as Theorem~\ref{thm:XOS regular} in Appendix~\ref{sec:XOS bounded}.
