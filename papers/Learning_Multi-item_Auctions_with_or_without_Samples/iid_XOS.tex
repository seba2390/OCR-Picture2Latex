\notshow{ 
\subsection{i.i.d. Buyers}
%For the SPM, when buyers are i.i.d. and regular, the supply limiting mechanism (by Tim, Inbal and Qiqi) achieves a constant fraction of its revenue and only requires $1$ sample. Need to figure out what to do when the distributions are bounded between $[0,1]$ or we only have $\hat{D}$.
In this section, we consider the case where bidders are symmetric, i.e., $D_{i}=D_{k}$ for any bidder $i$ and $k$. We simply use $D_j$ to denote the type distribution for item $j$, and we let $\DD =\times_{j} D_j$.


\begin{theorem}\cite{CaiZ17}
	For symmetric bidders, the better of an RSPM and the following ASPE achieves at least $\frac{1}{C}$ fraction of the optimal revenue for some absolute $C\geq 1$. Suppose $\{\beta_j\}_{j\in[m]}$ is a collection of real numbers such that $F_j(\beta_j)\in[1-\frac{1}{2n},1-\frac{1}{4n}]$ for all $j\in[m]$. If  $\sum_j F_j\left(\beta_j+c\right) \geq \frac{2}{3}$, let $c= 0$. Otherwise, let $c$ be a nonnegative real number such that $\sum_j F_j\left(\beta_j+c\right) \in [\frac{1}{2}, \frac{2}{3}]$.
\end{theorem}
For the anonymous sequential posted price with entry fee mechanism, we need to estimate the price $p_j$ for each item $j$, which requires the following statistics. Here let $t_i$ be the type of bidder $i$ and $v(t_i,S)$ be bidder $i$'s value for winning set $S$. In particular, let $V_{j}(t_i)$ be the value for winning just item $j$.
\begin{itemize}
\item $\beta_j := \Pr_{t\sim D}[V_j(t)\geq \beta_j]=1/2n$.
\item If $m>n$, $c:= \sum_j \Pr_{t\sim D}[V_j(t)\geq \beta_j+c]=1/2$, otherwise $c:=0$.
\item Let $C_i(t_i):=\{j\ |\ V_j(t_i)<\beta_j+c\}$.
\item Let $$p_j = \sum_i\sum_{t_i}f(t_i)\cdot \sum_{S:j\in S} \sigma_{iS}(t_i) \gamma_{j}^{S}(t_i),$$ where $\sigma_{iS}(t_i)$ is the interim probability for buyer $i$ to receive bundle $S$. Let $v_{j}^{(S)}(t_i)$ be the supporting price of item $j$ for set $S$ if the bidder's type is $t_i$ \footnote{$v_{j}^{(S)}(t_i)=t_{ij}$ when $t_i$ is additive subject to some constraints, but when $v_i$ is XOS the supporting prices depends on which bundle buyer $i$ is receiving.}, then $\gamma_{ij}^{(S)}(t_i)=v_{j}^{(S\cap C_i(t_i))}$ if $j\in S\cap C_i(t_i)$, and $\gamma_{ij}^{(S)}(t_i)=0$ otherwise.
\end{itemize}

What I can prove is that for any allocation rule $\sigma$, if we set the correct entry fee, then the anonymous posted price mechanism with entry fee has revenue at least $\frac{1}{2}\cdot \sum_j p_j$. Clearly, we need should use the $\sigma$ that maximizes $\sum_j p_j$.

Let $v'(t_i,S) = v(t_i, S\cap C_i(t_i))$ and $\sum_j p_j$ is the same as the social welfare for allocation $\sigma$ when $v'(\cdot,\cdot)$ is the valuation function. Clearly, $\sum_j p_j$ is maximized when we take $\sigma$ to be the welfare maximizing allocation for $v'(\cdot,\cdot)$.


I think as long as we can estimate all the $p_j$ within error $\delta$, we will lose revenue at most $O(m\delta)$. Of course, this will also require us to estimate $\beta_j$ and $c$. %For valuations that are additive subject to downwards closed constraints, it is easy to estimate $p_j$. As for any $S$,  $\bar{v}^{(S)}_{ij}=\bar{v}_{ij}=v_{ij}\cdot \ind[v_{ij}\leq \beta_j+c]$, so we only need to take $\sigma$ as the VCG allocation rule for $\bar{v}_{ij}$. But for XOS valuations, since the supporting prices depend on $S$, it is unclear what is the optimal $\sigma$.
}
