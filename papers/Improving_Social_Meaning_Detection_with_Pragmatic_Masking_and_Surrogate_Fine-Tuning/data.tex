% \section{Data}
\subsection{Pre-training Data}
\noindent \textbf{TweetEnglish Dataset.}
%%%%%%%%%%%%%%%%%%%%%%%%%%%%%%%%%%%%%%%%%
We extract $2.4$B English tweets\footnote{We select English tweets based on the Twitter language tag.} from a larger in-house dataset collected between $2014$ and $2020$. We lightly normalize tweets by removing usernames and hyperlinks and add white space between emojis to help our model identify individual emojis. We keep all the tweets, retweets, and replies but remove the `RT USER:' string in front of retweets. To ensure each tweet contains sufficient context for modeling, we filter out tweets shorter than $5$ English words (not counting the special tokens hashtag, emoji, USER, URL, and RT). We call this dataset \textbf{\texttt{TweetEng}}. Exploring the distribution of hashtags and emojis within TweetEng, we find that $18.5$\% of the tweets include at least one hashtag but no emoji, $11.5$\% have at least one emoji but no hashtag, and $2.2$\% have both at least one hashtag and at least one emoji. Investigating the hashtag and emoji location, we observe that $7.1$\% of the tweets use a hashtag as the last term, and that the last term of $6.7$\% of tweets is an emoji. We will use \texttt{TweetEng} as a general pool of data from which we derive for both our PMLM and SFT methods. 

%---------------------------------------
\textbf{PM Datasets.}\label{subsec:prag_masking:data}
%---------------------------------------
We extract five different subsets from \texttt{TweetEng} to explore the utility of our proposed PMLM method. Each of these five datasets comprises $150$M tweets as follows:
\texttt{\textbf{Naive}}. A randomly selected tweet set. Based on the distribution of hashtags and emojis in \texttt{TweetEng}, each sample in \texttt{Naive} still has some likelihood to include one or more hashtags and/or emojis. We are thus still able to perform our PM method on \texttt{Naive}. \texttt{\textbf{Naive-Remove}}. To isolate the utility of using pragmatic cues, we construct a dataset by removing all hashtags and emojis from \texttt{Naive}. \texttt{\textbf{Hashtag\_any}}. Tweets with at least one hashtag anywhere but no emojis. \texttt{\textbf{Emoji\_any}}. Tweets with at least one emoji anywhere but no hashtags. \texttt{\textbf{Hashtag\_end}}. Tweets with a hashtag as the last term but no emojis. \texttt{\textbf{Emoji\_end}}. Tweets with an emoji at the end of the tweet but no hashtags.\footnote{We perform an analysis based on two 10M random samples of tweets from Hashtag\_any and Emoji\_any, respectively. We find that on average there are 1.83 hashtags per tweet in Hashtag\_any and 1.88 emojis per tweet in Emoji\_any.}

%\subsection{Social Meaning Datasets}\label{subsec:sm_datasets}
\noindent\textbf{SFT Datasets.}\label{subsec:sft:data}
%%%%%%%%%%%%%%%%%%%%%%%%%%%%%%%%%%
We experiment with two SFT settings, one based on \textit{hashtags} (\textit{SFT-H}) and another based on \textit{emojis} (\textit{SFT-E}). For SFT-H, we utilize the \texttt{Hashtag\_end} dataset mentioned above. The dataset includes $5$M unique hashtags (all occurring at the end of tweets), but the majority of these are low frequency. We remove any hashtags occurring $< 200$ times, which gives us a set of $63K$ hashtags in $126M$ tweets. We split the tweets into Train ($80\%$), Dev ($10\%$), and Test ($10\%$). For each sample, we use the end hashtag as the sample label.\footnote{We use the last hashtag as the label if there are more than one hashtag in the end of a tweet. Different from PMLM, SFT is a multi-class single-label classification task. We plan to explore the multi-class multi-label SFT in the future.} We refer to this resulting dataset as \textbf{\texttt{Hashtag\_pred}}. For emoji SFT, we work with the \texttt{emoji\_end} dataset. Similar to SFT-H, we remove low-frequence emojis ($< 200$ times), extract the same number of tweets as \texttt{Hashtag\_pred}, and follow the same data splitting method. We acquire a total of $1,650$ unique emojis in final positions, which we assign as class labels and remove them from the original tweet body. We refer to this dataset as \textbf{\texttt{Emoji\_pred}}.

\subsection{Evaluation Benchmark}\label{sec:smpb} 
We collect $15$ datasets representing eight different social meaning tasks to evaluate our models, as follows:~\footnote{To facilitate reference, we give each dataset a name.}
 

% \begin{enumerate}[leftmargin=15pt]
\noindent\textbf{Crisis awareness.}~We use \texttt{Crisis\textsubscript{Oltea}} ~\cite{olteanu2014crisislex}, a corpus for identifying whether a tweet is related to a given disaster or not. %The dataset collects generated during six disasters, including hurricane, tornado, etc. 
    
 \noindent     \textbf{Emotion.} We utilize \texttt{Emo\textsubscript{Moham}}, introduced by~\citet{mohammad-2018-semeval}, for emotion recognition. We use the version adapted in \citet{barbieri-2020-tweeteval}.
 
\noindent\textbf{Hateful and offensive language.} We use \texttt{Hate\textsubscript{Waseem}}~\cite{waseem-2016-hateful}, \texttt{Hate\textsubscript{David}}~\cite{davidson-2017-hateoffensive}, and \texttt{Offense\textsubscript{Zamp}}~\cite{zampieri-2019-predicting}.

\noindent     \textbf{Humor.} We use the humor detection datasets \texttt{Humor\textsubscript{Potash}}~\cite{potash-2017-semeval} and \texttt{Humor\textsubscript{Meaney}}~\cite{meaney2021hahackathon}.

\noindent\textbf{Irony.} We utilize \texttt{Irony\textsubscript{Hee-A}} and  \texttt{Irony\textsubscript{Hee-B}} from \citet{van-hee2018semeval}. 
    % \begin{itemize}[leftmargin=10pt]
    %     \item \textbf{Irony\textsubscript{Hee-A}}~~ \citet{van-hee2018semeval} initially retrieve irony tweets using related hashtags (i.e.,  `\#irony', `\#sarcasm', and `not'). To enhance the quality of labeled data, they reassign the label based on human annotations. 
    %     \item \textbf{Irony\textsubscript{Hee-B}}~~ \citet{van-hee2018semeval} provide a more fine-grained annotation on the samples of Irony\textsubscript{Hee-A}. Each sample is annotated with a type of irony: (1) \textit{verbal irony by means of a polarity contrast}, (2) \textit{other types of verbal irony}, (3) \textit{situational irony}, or (4) \textit{non-ironic}.
    % \end{itemize}
  
\noindent     \textbf{Sarcasm.} %The goal in sarcasm detection is to predict whether a social media post is \textit{sarcastic} or \textit{not sarcastic}. 
We use four sarcasm datasets from \texttt{Sarc\textsubscript{Riloff}}~\cite{riloff2013sarcasm}, \texttt{Sarc\textsubscript{Ptacek}}~\cite{ptavcek2014sarcasm}, \texttt{Sarc\textsubscript{Rajad}}~\cite{rajadesingan2015sarcasm}, and \texttt{Sarc\textsubscript{Bam}}~\cite{bamman2015contextualized}. 
    % \begin{itemize}[leftmargin=10pt]
    %     \item \textbf{Sarcasm\textsubscript{Riloff}}~~ \citet{riloff2013sarcasm} create a manually annotated dataset for sarcasm detection. The sarcastic tweet are originally collected by sarcasm hashtags (`\#sarcasm' or `\#sarcastic'). 
    %     \item \textbf{Sarcasm\textsubscript{ptavcek}}~~ \citet{ptavcek2014sarcasm} introduce a distance supervised dataset for sarcasm detection. The `\#sarcasm' hashtag is used as the proxy to annotate a tweet is sarcastic or not. 
    %     \item \textbf{Sarcasm\textsubscript{Rajad}}~~ \citet{rajadesingan2015sarcasm} provide a English sarcasm detection corpus. They retrieve the sarcastic by using hashtag `\#sarcasm' and `\#not' as the distant indicator. The non-sarcastic tweets are random tweets without any seed hashtags.
    %     \item \textbf{Sarcasm\textsubscript{Bam}}~~ \citet{bamman2015contextualized} also utilize distant supervision to collect English sarcastic tweets. The sarcastic tweet uses the hashtag `\#sarcasm' or `\#sarcastic' as its last term. To investigate the sarcasm expression in the social context, they only extract tweets that are responses to another tweet.
    % \end{itemize}
 
\noindent\textbf{Sentiment.} We employ the three-way sentiment analysis dataset from \texttt{Senti\textsubscript{Rosen}}~\cite{rosenthal-2017-semeval}. 

\noindent\textbf{Stance.} We use \texttt{Stance\textsubscript{Moham}}, a stance detection dataset from \citet{mohammad-2016-semeval}. The task is to identify the position of a given tweet towards a target of interest. 

We use the Twitter API~\footnote{\url{https://developer.twitter.com/}} to crawl datasets which are available only in tweet ID form. We note that we could not download all tweets since some tweets get deleted by users or become inaccessible for some other reason. Since some datasets are old (dating back to 2013), we are only able to retrieve $73\%$ of the tweets on average (i.e., across the different datasets). %This inaccessibility of the data motivates us to paraphrase the datasets as we explain in Section~\ref{sec:para_model}. Before we paraphrase the data or use it in our various experiments, 
We normalize each tweet by replacing the user names and hyperlinks to the special tokens `USER' and `URL', respectively. %This ensures our paraphrased dataset will never have any actual usernames or hyperlinks, thereby protecting user identity. 
For datasets collected based on hashtags by original authors (i.e., \textit{distant supervision}), we also remove the seed hashtags from the original tweets. For datasets originally used in cross-validation, we acquire $80\%$ Train, $10\%$ Dev, and $10\%$ Test via random splits. For datasets that had training and test splits but not development data, we split off $10\%$ from training data into Dev. The data splits of each dataset are presented in Table~\ref{tab:gold_data}. %We now introduce our pragmatic masking method.
\section{The Semantic Urban Mesh Dataset}\label{sec:framework}
\subsection{Dataset Specification}

We have used Helsinki's 3D texture meshes as input and annotated them as a benchmark dataset of semantic urban meshes. 
The Helsinki's raw dataset covers about 12 $ km^2 $, and it was generated in 2017 from oblique aerial images that have about a 7.5 $cm$  ground sampling distance (GSD) using an off-the-shelf commercial software namely ContextCapture~\citep{contextcap}.
The source images have three colour channels (i.e., red, green, and blue) and are collected from an airplane with five cameras that have $80\%$ length coverage and $60\%$ side coverage.
To recover the 3D water bodies that do not fulfil the Lambertian hypothesis, 2D vector maps and ortho-photos are used when performing the surface reconstruction.
Furthermore, processing like aerial triangulation, dense image matching, and mesh surface reconstruction were all performed with ContextCapture.
It should be noticed that the entire region of Helsinki is split into tiles, and each of them covers about 250 $ m^2 $~\citep{kalasatamaReport}.
As shown in Figure \ref{fig:overview},  we have selected the central region of Helsinki as the study area, which includes 64 tiles and covers about 4 $km^2$ map area (8 $km^2$ surface area) in total.   

\subsection{Object Classes}
We define the semantic categories for urban meshes by the most common objects in the urban environment with unambiguous geometry and texture appearance.
Moreover, each triangle face is assigned to a label of one of the six semantic classes. 
Ambiguous regions (which account for about 2.6\% of the total mesh surface area), such as shadowed regions or distorted surfaces, are labelled as unclassified (see Figure \ref{fig:ambigious}).
The object classes we consider in the benchmark dataset are: 
\begin{itemize}
	\item \textbf{terrain}: roads, bridges, grass fields, and impervious surfaces;
	\item \textbf{building}: houses,high-rises, monuments, and security booths;
	\item \textbf{high vegetation}: trees, shrubs, and bushes;
	\item \textbf{water}: rivers, sea, and pools;
	\item \textbf{vehicle}: cars, buses, and lorries;  
	\item \textbf{boat}: boats, ships, freighters, and sailboats;
	\item \textbf{unclassified}: incomplete objects like buses and trains, distorted surfaces like tables, tents and facades, construction sites, underground walls.
\end{itemize}

\begin{figure}[!tb]
	\includegraphics[height=0.48\textwidth]{figures/overview_grids/yaxis.png}
	\begin{subfigure}[t]{0.48\textwidth}
		\includegraphics[width=\linewidth]{figures/overview_grids/texture_global_birdsview00.png}
		\includegraphics[width=\linewidth]{figures/overview_grids/xaxis.png}
		\label{fig:textop}
	\end{subfigure}
	\hspace*{\fill}
	\begin{subfigure}[t]{0.48\textwidth}		
		\includegraphics[width=\linewidth]{figures/overview_grids/semantic_global_birdsview00.png}
		\vspace*{-0.78cm}
		\begin{center}
		\includegraphics[width=0.8\linewidth]{figures/semantic_results/semantic_legend2.png}
		\end{center}
		\label{fig:semtop}
	\end{subfigure}
	\vspace*{-0.7cm}
	\caption{Overview of the semantic urban mesh benchmark.
	Left: the texture meshes covering about 4 $km^2$ map area. Right: the ground truth meshes.
	More views of the same scene (with different visualization styles) are shown in Figures \ref{fig:texside} and \ref{fig:semside}.}
	\label{fig:overview}
\end{figure}

\begin{figure}[!tb]
	\centering
	\begin{subfigure}[t]{0.48\textwidth}
		\includegraphics[width=\linewidth]{figures/ambigious/shadow_tex_zoom.png}
		\caption{}
	\end{subfigure}
	\hspace*{\fill}
	\begin{subfigure}[t]{0.48\textwidth}
		\includegraphics[width=\linewidth]{figures/ambigious/shadow_fc_zoom.png}
		\caption{}
	\end{subfigure}
	\begin{subfigure}[t]{0.48\textwidth}
		\includegraphics[width=\linewidth]{figures/ambigious/distort_tex_zoom.png}
		\caption{}
	\end{subfigure}
	\hspace*{\fill}
	\begin{subfigure}[t]{0.48\textwidth}
		\includegraphics[width=\linewidth]{figures/ambigious/distort_fc_zoom.png}
		\caption{}
	\end{subfigure}
	\caption{Ambiguous regions are labelled as unclassified (in black). 
		(a) Shadow region with texture.
		(b) Shadow region with semantic colour.
		(c) Distorted region with texture.
		(d) Distorted region with semantic colour.} 
	\label{fig:ambigious}
\end{figure}


\subsection{Semi-automatic Mesh Annotation}  \label{sec:mesh_annota}
Rather than manually labelling each triangle face of the raw meshes, we design a semi-automatic mesh labelling framework to accelerate the labelling process. Figure~\ref{fig:pipeline} shows the overall pipeline of our labelling workflow.

Given the fact that urban environments consist of a large number of planar regions in the data, we opt to label the data at the segment level instead of individual triangle faces. 
Specifically, we over-segment the input meshes into a set of planar segments. 
These segments can enrich local contextual information for feature extraction and serve as the basic annotation unit to improve annotation efficiency.

\begin{figure}[!tb]
	\centering
	\includegraphics[width=\textwidth]{figures/pipeline/pipeline_L1.png}
	\caption{The pipeline of the labelling workflow.}
	\label{fig:pipeline}
\end{figure}

Instead of randomly choosing a mesh tile as input for annotation and refinement, which is insufficient for manual annotation progress, we favour picking a mesh tile that is more difficult to classify.
Similar to active learning, we first compute the feature diversity (see Equation \ref{eq:fea_div}) to optimally select a mesh tile containing a variety of classes and objects at different scales and complexity.
The feature diversity $F_{m}$ of tile $m$ is computed as
\begin{equation}\label{eq:fea_div}
	F_{m}=\frac{\sum_{i=1}^{N_{f}}\left ( f_i - \bar{f} \right )^{2}}{N_{f}}
\end{equation}
where $f_i$ represents each handcrafted feature which describe in Section \ref{sec:initial_seg}, and $\bar{f}$ is mean value of a $N_{f}$ dimensional feature vector.
To acquire the first ground truth data, we manually annotate the mesh (with segments) that is selected with the highest feature diversity.
Then, we add the first labelled mesh into the training dataset for the supervised classification.
Specifically, we use the segment-based features as input for the classifier, and the output is a pre-labelled mesh dataset.
Next, we use the mesh annotation tool to manually refine the pre-labelled mesh according to the feature diversity.
Finally, the new refined mesh will be added to the training dataset to improve the automatic classification accuracy incrementally.


\subsubsection{Initial Segmentation}\label{sec:initial_seg}

To avoid redundant computations of numerous triangles, we first apply mesh over-segmentation (i.e., linear least-squares fitting of planes) based on region growing on the input data to group triangle faces into homogeneous regions~\citep{lafarge2012creating}.
Such grouped regions are beneficial for computing local contextual features.
We then extract both geometric and radiometric features from those mesh segments as follows: 
\begin{itemize}
	\item[$\bullet$] \textit{Eigen-based features} are computed from the covariance matrix of the triangle vertices with respect to the average centre within each segment, which is beneficial for identifying urban objects with various surface distributions.
	The linearity $= (\lambda_{1} - \lambda_{2}) / \lambda_{1}$, sphericity $= \lambda_{3}/ \lambda_{1}$ and change of curvature $= \lambda_{3} / (\lambda_{1} + \lambda_{2} + \lambda_{3})$ are computed based on the three eigenvalues $\lambda_{1} \geq \lambda_{2} \geq \lambda_{3}\geq 0$.
	The local eigenvectors $\mathbf{n}_{i} $ and the unit normal vector $\mathbf{n}_{z} $ along Z-axis are used to compute the verticality $=1-\left | \mathbf{n}_{i}\cdot \mathbf{n}_{z} \right | $~\citep{hackel2016fast}.
	Note that many eigen-based features have been studied in literature~\citep{hackel2016fast,west2004context,weinmann2013feature}, and some of them were designed for and tested on LiDAR point clouds. 
	\textcolor{ao}{
	These eigen-based features are mostly computed per point based on its spherical neighbourhood, which often contains noise and does not form a surface. 
	Our chosen eigen-based features are defined on a segment representing the surface of a mesh, and thus they can capture non-local geometric properties of an object.
	}
	Additionally, in this work, we have tested all eigen-based features from the literature~\citep{hackel2016fast}, and we only present the ones that are effective for texture meshes.
	\item[$\bullet$] \textit{Elevation} is divided into absolute elevation $z_{a}$, relative elevation $z_{r}$ and multiscale elevations $z_{m}$.
	Where $z_{a}$ is the average elevation of the segment;
	the relative elevation is computed as $z_{r} = z_{a}-z_{r_{min}}$;
	the multiscale elevation~\citep{Verdie2015,Rouhani2017} $z_{m} = \sqrt{\frac{z_{a} - z_{min}}{z_{max} - z_{min}}}$.
	And $z_{r_{min}}$ denotes the lowest elevation of the local largest ground segment computed within a cylindrical neighbourhood with 30 meters radius around the segment centre.
	$z_{min}$ and $z_{max}$ represent the local minimum and maximum elevation values of a cylindrical neighbourhood within the scale of 10 meters, 20 meters, and 40 meters.
	Such large cylindrical neighbourhoods allow to find the local ground considering the resilience to hilly environments, \textcolor{ao}{and the square root ensures that small relative height values (i.e., values smaller than 1 $ m $) get a larger elevation attribute to enlarge elevation differences between small objects and the local ground (e.g., cars against the ground, boats against the water surfaces).}
	More importantly, due to the influence of terrain fluctuations and various scales of urban objects, the elevation of these three categories can complement each other.
	\item[$\bullet$] \textit{Segment area} is computed as $area(S_k) = \sum_{i = 1}^{N} area(f_i) $, where $f_i$ denotes a triangle of the segment $S_k$, and $N$ denotes the total number of triangles in $S_k$.
	\item[$\bullet$] \textit{Triangle density} is defined as $density(S_k) = \frac{N}{area(S_k)} $,  which reveals the object complexity, especially for adaptive urban meshes.
	\item[$\bullet$] \textit{Interior radius of 3D medial axis transform (InMAT)}~\citep{ma20123d,peters2016robust} of a segment $S_k$ is formulated as $r_k = \frac{\sum_{i=1}^{M} r_i}{M}$, where $M$ denotes the total number of triangle vertices of $S_k$, and $r_i$ denotes the interior radius of the shrinking ball that touches the vertex $v_i$ within the segment $S_k$. 
	It is designed to distinguish objects with different scales. 
	\item[$\bullet$] \textit{HSV colour-based features} are derived from the RGB channel of the entire texture map.
	We use the HSV colour space since it can better differentiate different objects than RGB.
	We compute the average colour, the variance of the colour distribution of all pixels within each segment, and we further discretize it into a histogram that consists of 15 bins of the hue channel, five bins of the saturation channel, and five bins of the value channel.
	\item[$\bullet$] \textit{Greenness} $a_{g}$ is used to classify objects that are similar to green vegetation.
	Specifically, it is computed according to the averaged RGB colour of each segment via $a_{g}=G-0.39\cdot R-0.61\cdot B$~\citep{mckinnon2017comparing}. 
\end{itemize}
	All the above features are concatenated into a 44-dimensional feature vector used by our random forest (RF) classifier in the initial segmentation. 

\subsubsection{Annotation Tool for Refinement}

Because of the under-segmentation errors and the imperfect results of the semantic mesh segmentation process, we design a mesh annotation tool (see Figure \ref{fig:annotator}) to manually correct the labelling errors.
Our mesh annotation tool is developed based on the labelling tool of CGAL~\citep{cgal:eb-20b}.

\begin{figure}[!tb]
	\centering
	\includegraphics[width=\textwidth]{figures/annotator/annotator.png}
	\caption{The interface of our annotation tool for 3D texture meshes. }
	\label{fig:annotator}
\end{figure}

As shown in Table \ref{tab:annotation_operation}, it consists of three operation categories: view, selection, and annotation.
The	view operations provide essential functions for the user to manipulate the scene camera, such as translate, rotate, zoom, or set the new pivot for the scene.
In addition, to use textures as a reference for labelling, we map texture and face colour with a certain degree of transparency, and we visualize the segment border to differentiate each segment. 

\begin{table}[!tb]
	\centering
	\noindent\adjustbox{max width=0.8\textwidth}
	{
		\begin{threeparttable}
			\centering
			\begin{tabular}{ccc}
				\toprule
				Categories & Operations & Objects \\
				\midrule
				\multirow{4}[2]{*}{View} & Translate & Camera \\
				& Rotate & Camera \\
				& Zoom in / out & Camera \\
				& Set pivot & Camera \\
				\midrule
				\multirow{6}[2]{*}{Selection} & Multi-selection / Lasso & Triangles / Segments \\
				& Expand / Reduce & Triangles / Segments \\
				& Semantic selection & Segments \\
				& Split region & Segments \\
				& Planar region extraction & Triangles \\
				& Split mesh & Triangles \\
				\midrule
				\multirow{3}[2]{*}{Annotation} & Probability slider & Segments \\
				& Segment area slider & Segments \\
				& Progress bar & Triangles \\
				& Switch semantic view & Triangles \\ 
				& Labelling & Triangles / Segments \\
				\bottomrule
			\end{tabular}%
		\end{threeparttable}
	}
	\caption{Basic operations in our annotation tool.} 
	\label{tab:annotation_operation}%
\end{table}%


The	selection operations allow the user to select or deselect either triangle faces (see Figure \ref{fig:tri_sel}) or segments (see Figure \ref{fig:seg_sel}) freely via a brush or a lasso.
Specifically, the face selection operation is used to fix the under-segmentation errors and generate new segments, and the segment selection operation is to fix incorrect segment labels.

\begin{figure}[!tb]
	\centering
	\begin{subfigure}[t]{0.32\textwidth}
		\includegraphics[width=\linewidth]{figures/pipeline/tri_select_a.png}
		\caption{}
	\end{subfigure}
	\hspace*{\fill}
	\begin{subfigure}[t]{0.32\textwidth}
		\includegraphics[width=\linewidth]{figures/pipeline/tri_select_b.png}
		\caption{}
	\end{subfigure}
	\hspace*{\fill}
	\begin{subfigure}[t]{0.32\textwidth}
		\includegraphics[width=\linewidth]{figures/pipeline/tri_select_c.png}
		\caption{}
	\end{subfigure}
	\caption{An example of labelling by selecting triangles using the lasso tool (blue edges: segment boundaries). 
		(a) Before selection.
		(b) Lasso selection result (in red).
		(c) The correct label has been assigned to the selected region. 
		In this example, the label of the selected region has been changed from `ground' to `vehicle'.
	} 
	\label{fig:tri_sel}
\end{figure}


\begin{figure}[!tb]
	\centering
	\begin{subfigure}[t]{0.32\textwidth}
		\includegraphics[width=\linewidth]{figures/pipeline/seg_select_a.png}
		\caption{}
	\end{subfigure}
	\hspace*{\fill}
	\begin{subfigure}[t]{0.32\textwidth}
		\includegraphics[width=\linewidth]{figures/pipeline/seg_select_b.png}
		\caption{}
	\end{subfigure}
	\hspace*{\fill}
	\begin{subfigure}[t]{0.32\textwidth}
		\includegraphics[width=\linewidth]{figures/pipeline/seg_select_c.png}
		\caption{}
	\end{subfigure}
	\caption{An example of segment labelling. 
		(a) Part of a wall of the building was previously labelled as `high vegetation' (in green).
		(b) Segment selection result (in red).
		(c) The label of the selected segment has been corrected with the new label `building'.
	}
	\label{fig:seg_sel}
\end{figure}

We also allow the user to edit the selection of each individual segment with splitting functions (see Figure \ref{fig:pnp_func}) and automatic extraction of the most planar region (see Figure \ref{fig:seg_func}). 
As for splitting, we first detect the potential planar and non-planar segments marked by user strokes, and then the non-planar one is split according to the vertex-to-plane distance.
It allows generating candidate non-planar regions (with respect to the detected planar segment) for the user to edit, and
it is useful to split a segment that covers large non-planar regions or contains more than one dominant planar area.
To extract the most planar region, we apply the region growing algorithm~\citep{lafarge2012creating} within the selected segment to automatically generate the candidate triangle faces with user-defined thresholds (i.e., the maximum distance to the plane, the maximum accepted angle, and the minimum region size).
Such an operation allows the user to filter out some small bumpy regions of the selected segment.

\begin{figure}[!tb]
	\centering
	\begin{subfigure}[t]{0.48\textwidth}
		\includegraphics[width=\linewidth]{figures/annotator/pnp_pipeline1.png}
		\caption{}
	\end{subfigure}
	\hspace*{\fill}
	\begin{subfigure}[t]{0.48\textwidth}
		\includegraphics[width=\linewidth]{figures/annotator/pnp_pipeline2.png}
		\caption{}
	\end{subfigure}
	\caption{An example splitting planar and non-planar regions. 
		(a) The user draws a stroke (in red) across the border of the non-planar segment and the planar segment. 
		(b) The detected non-planar segment has been split into two parts (i.e., a non-planar region shown in red and a planar segment shown in green).
	} 
	\label{fig:pnp_func}
\end{figure}

\begin{figure}[!tb]
	\centering
	\begin{subfigure}[t]{0.48\textwidth}
		\includegraphics[width=\linewidth]{figures/annotator/planar_split_pipeline1.png}
		\caption{}
	\end{subfigure}
	\hspace*{\fill}
	\begin{subfigure}[t]{0.48\textwidth}
		\includegraphics[width=\linewidth]{figures/annotator/planar_split_pipeline3.png}
		\caption{}
	\end{subfigure}
	\caption{Editing an individual segment. 
		(a) A segment is selected (highlighted in green) for splitting. 
		(b) Automatic extraction of the most planar region (shown in red) within the selected segment according to user-defined thresholds.} 
	\label{fig:seg_func}
\end{figure}

Besides, probability and area-based sliders and a progress bar are provided in the annotation panel to improve annotation efficiency and experience, respectively. 
Specifically, the probability slider is introduced for the user to visually inspect the segments that are most likely misclassified.
Moreover, the user can further use it to inspect a specific class by switching the view to highlight a specific semantic class.
The segment area slider is used to identify isolated tiny segments, which commonly appear as errors.
The progress bar is used to indicate the estimated labelling progress during the annotation.
After performing the selection, the user can easily assign the corresponding label to the selected area.

%We describe the data crawling, preparation, and splits of evaluation datasets in Section~\ref{app:sec:crawling} in Appendix. 

To test our models under the \textbf{\textit{few-shot setting}}, we conduct few-shot experiments on varying percentages of the Train set of each task (i.e., $1\%$, $5\%$, $10\%$, $20\%$ \dots $90\%$). For each of these sizes, we randomly sample three times with replacement (\textit{as we report the average of three runs in our experiments}) and evaluate each model on the original Dev and Test sets. We also evaluate our models on the \textbf{\textit{zero-shot setting}} utilizing data from Arabic: \texttt{Emo\textsubscript{Mageed}}~\cite{mageed-2020-aranet}, \texttt{Irony\textsubscript{Ghan}}~\cite{idat2019}; Italian: \texttt{Emo\textsubscript{Bian}}~\cite{bianchi2021feel} and \texttt{Hate\textsubscript{Bosco}}~\cite{bosco2018overview}; and Spanish:    \texttt{Emo\textsubscript{Moham}}~\cite{mohammad-2018-semeval} and \texttt{Hate\textsubscript{Bas}}~\cite{basile-2019-semeval}. %We describe the data crawling, preparation, and splits of evaluation datasets in Section~\ref{app:sec:crawling} in Appendix. 



  
  %\footnote{Information about individual dataset sizes is in Appendix~\ref{append:datasets}.} This data attrition motivates our paraphrase work we describe in Section~\ref{subsec:paraphrasing}. We now describe our method to paraphrase the benchmark. 
 
% \textbf{Crisis awareness.} Crisis awareness task anlyze social media communication during the crises and mass emergencies. We utilize a dataset from \citet{olteanu2014crisislex}. 
% \begin{itemize}
%  \item 
% \textbf{Crisis\textsubscript{Oltea}}~~ \citet{olteanu2014crisislex}