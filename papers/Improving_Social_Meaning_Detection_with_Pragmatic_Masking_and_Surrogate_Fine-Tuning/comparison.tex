\vspace{-5pt}
\subsection{Model Comparisons}\label{subsec:comparisons}
The purpose of our work is to produce representations effective across all social meaning tasks, rather than a single given task. However, we still compare our best model (i.e., PragS2) on each dataset to the SOTA of that particular dataset and the published results on a Twitter evaluation benchmark~\cite{barbieri-2020-tweeteval}. \textit{All our reported results are an average of three runs}, and we report using the same respective metric adopted by original authors on each dataset. As Table~\ref{tab:compare} shows, our model achieves the best performance on eight out of $15$ datasets. On average, our models are $0.97$ points higher than the closest baseline, i.e., BERTweet. This shows the superiority of our methods, even when compared to models trained simply with MLM with $\sim3 \times$ more data ($850$M tweets for BERTweet vs. only $276$M for our best method). We also note that some SOTA models adopt task-specific approaches and/or require task-specific resources. For example,~\citet{bamman2015contextualized} utilize Stanford sentiment analyzer to identify the sentiment polarity of each word. In addition, task-specific methods can still be combined with our proposed approaches to improve performance on individual tasks.
% Please add the following required packages to your document preamble:
% \usepackage{booktabs}
\begin{table}[ht]
\centering
\scriptsize
\begin{tabular}{@{}llccrcH@{}}
\toprule
\multicolumn{1}{c}{\textbf{Task}}                & \multicolumn{1}{c}{\textbf{Metric}} & \multicolumn{1}{c}{\textbf{SOTA}} & \textbf{TwE}              & \multicolumn{1}{c}{\textbf{BTw}} & \multicolumn{1}{c}{\textbf{\begin{tabular}[c]{@{}c@{}}Ours \\ (PragS2)\end{tabular}}} & \multicolumn{1}{H}{\textbf{Setting}} \\ \midrule
Crisis\textsubscript{Oltea}  & M-$F_1$                           & 95.60\textsuperscript{$\star$}                             & -                         & \textbf{95.88}                                 & 95.68                                 & X1+SFT-E                          \\
Emo\textsubscript{Moham}     & M-$F_1$                           & \multicolumn{1}{c}{-}             & \multicolumn{1}{r}{78.50} & 80.14                                 & \textbf{80.50}                                 & PM-EE                               \\
Hate\textsubscript{Waseem}   & W-$F_1$                             & 73.62\textsuperscript{$\star\star$}                             & -                         & 88.00                                 & \textbf{88.36}                                 & SFT-H                                \\
Hate\textsubscript{David}    & W-$F_1$                             & 90.00\textsuperscript{$\dagger$}                             & -                         & \textbf{91.27}                                 & 91.01                                 & PM-N                                 \\
Humor\textsubscript{Potash}  & M-$F_1$                           & \multicolumn{1}{c}{-}             & -                         & 52.77                                 & \textbf{53.83}                                 & RM-EA                               \\
Humor\textsubscript{Meaney}  & M-$F_1$                           & \multicolumn{1}{c}{\textbf{98.54}$^=$}             & -                         & 94.46                                 &   94.49                                 & PM-HE                                \\
Irony\textsubscript{Hee-A}    & $F^{(i)}$          & 70.50\textsuperscript{$\dagger\dagger$}                             & \multicolumn{1}{r}{65.40} & 71.49                                 & \textbf{76.47}                                 & X2+SFT-H                          \\
Irony\textsubscript{Hee-B}    & M-$F_1$                           & 50.70\textsuperscript{$\dagger\dagger$}                             & -                         & 58.67                                 & \textbf{61.67}                                 & X2+SFT-H                          \\
Offense\textsubscript{-Zamp} & M-$F_1$                           & \textbf{82.90}\textsuperscript{$\ddagger$}                             & \multicolumn{1}{r}{80.50} & 78.49                                 & {79.50}                                 & X1+SFT-E                          \\
Sarc\textsubscript{Riloff}   & $F^{(s)}$          & 51.00\textsuperscript{$\ddagger\ddagger$}                             & -                         & 66.35                                 & \textbf{68.88}                                & SFT-H                                \\
Sarc\textsubscript{Ptacek}   & M-$F_1$                           & 92.37\textsuperscript{$\mathsection$}                             & -                         & \textbf{96.35}                                 & 96.24                                 & X2+SFT-H                          \\
Sarc\textsubscript{Rajad}    & Acc                                 & 92.94\textsuperscript{$\mathsection\mathsection$}                             & -                         & 95.29                                 & \textbf{95.66}                                 & X2+SFT-H                          \\
Sarc\textsubscript{Bam}      & Acc                                 & \textbf{85.10}\textsuperscript{$\|$}                             & -                         & 82.28                                 & 81.27                                & X1+SFT-E                          \\
Senti\textsubscript{Rosen}   & M-Rec                             & 68.50\textsuperscript{$\diamondsuit$}                             & \multicolumn{1}{r}{72.60} & \textbf{72.90}                                 & 71.76                                 & PM-N                                 \\
Stance\textsubscript{Moham}  &
Avg(a,f)             & 71.00\textsuperscript{$\circledcirc$}                             & \multicolumn{1}{r}{69.30} & 69.79                                 & \textbf{73.45}                                 & SFT-H                                \\\cdashline{1-7} 
\textbf{Average}                                          & \multicolumn{1}{c}{-}               & 77.02                             & \multicolumn{1}{r}{73.26} & 79.61                                 & \multicolumn{1}{c}{\textbf{80.58}}                                & \multicolumn{1}{H}{-}                \\ \bottomrule
\end{tabular} 
\caption{Model comparisons. \textbf{SOTA:} Best performance on each respective dataset. \textbf{TwE:} TweetEval~\cite{barbieri-2020-tweeteval} is a benchmark for tweet classification evaluation. \textbf{BTw:} BERTweet~\cite{nguyen-etal-2020-bertweet}. We compare using the same metrics employed on each dataset. \textbf{Metrics:} \textbf{M-$F_1$:} macro $F_1$,  \textbf{W-$F_1$:} weighted $F_1$, $F_1^{(i)}$: $F_1$ irony class, \textbf{$F_1^{(i)}$:} $F_1$ irony class, $F_1^{(s)}$: $F_1$ sarcasm class, \textbf{M-Rec:} macro recall,  \textbf{Avg(a,f)}: Average $F_1$ of the \textit{against} and \textit{in-favor} classes (three-way dataset). \textsuperscript{$\star$}~\citet{liu2020crisisbert}, \textsuperscript{$\star\star$}~\citet{waseem-2016-hateful}, \textsuperscript{$\dagger$}~\citet{davidson-2017-hateoffensive},
\textsuperscript{$^=$}~\citet{meaney2021hahackathon},
\textsuperscript{$\dagger\dagger$}~\citet{van-hee2018semeval}, \textsuperscript{$\ddagger$}~\citet{zampieri-2019-semeval}, \textsuperscript{$\ddagger\ddagger$}~\citet{riloff2013sarcasm}, \textsuperscript{$\mathsection$}~\citet{ptavcek2014sarcasm},  \textsuperscript{$\mathsection\mathsection$}~\citet{rajadesingan2015sarcasm}, \textsuperscript{$\|$}~\citet{bamman2015contextualized}, \textsuperscript{$\diamondsuit$}~\citet{rosenthal-2017-semeval}, \textsuperscript{$\circledcirc$}~\citet{mohammad-2016-semeval}. 
}\label{tab:compare}
\end{table}


% \begin{table}[ht]
% \centering
% \tiny
% \begin{tabular}{@{}lllcrrl@{}}
% \toprule
% \multicolumn{1}{c}{\textbf{Task}}                & \multicolumn{1}{c}{\textbf{Metric}} & \multicolumn{1}{c}{\textbf{SOTA}} & \textbf{TwE}              & \multicolumn{1}{c}{\textbf{BTw}} & \multicolumn{1}{c}{\textbf{Best}} & \multicolumn{1}{c}{\textbf{Setting}} \\ \midrule
% Crisis\textsubscript{Oltea}  & M-$F_1$                           & 95.60\textsuperscript{$\star$}                             & -                         & 95.88                                 & \textbf{96.02}                                 & X1+SFT-E                          \\
% Emo\textsubscript{Moham}     & M-$F_1$                           & \multicolumn{1}{c}{-}             & \multicolumn{1}{r}{78.50} & 80.14                                 & \textbf{82.18}                                 & PM-EE                               \\
% Hate\textsubscript{Waseem}   & W-$F_1$                             & 73.62\textsuperscript{$\star\star$}                             & -                         & 88.00                                 & \textbf{88.02}                                 & SFT-H                                \\
% Hate\textsubscript{David}    & W-$F_1$                             & 90.00\textsuperscript{$\dagger$}                             & -                         & 91.27                                 & \textbf{91.50}                                 & PM-N                                 \\
% Humor\textsubscript{Potash}  & M-$F_1$                           & \multicolumn{1}{c}{-}             & -                         & 52.77                                 & \textbf{57.14}                                 & RM-EA                               \\
% Humor\textsubscript{Meaney}  & M-$F_1$                           & \multicolumn{1}{c}{\textbf{98.54}$^=$}             & -                         & 94.46                                 & {94.52}                                 & PM-HE                                \\
% Irony\textsubscript{Hee-A}    & $F^{(i)}$          & 70.50\textsuperscript{$\dagger\dagger$}                             & \multicolumn{1}{r}{65.40} & 71.49                                 & \textbf{76.47}                                 & X2+SFT-H                          \\
% Irony\textsubscript{Hee-B}    & M-$F_1$                           & 50.70\textsuperscript{$\dagger\dagger$}                             & -                         & 58.67                                 & \textbf{61.67}                                 & X2+SFT-H                          \\
% Offense\textsubscript{-Zamp} & M-$F_1$                           & \textbf{82.90}\textsuperscript{$\ddagger$}                             & \multicolumn{1}{r}{80.50} & 78.49                                 & {81.34}                                 & X1+SFT-E                          \\
% Sarc\textsubscript{Riloff}   & $F^{(s)}$          & 51.00\textsuperscript{$\ddagger\ddagger$}                             & -                         & 66.35                                 & \textbf{70.02}                                & SFT-H                                \\
% Sarc\textsubscript{Ptacek}   & M-$F_1$                           & 92.37\textsuperscript{$\mathsection$}                             & -                         & \textbf{96.35}                                 & 96.24                                 & X2+SFT-H                          \\
% Sarc\textsubscript{Rajad}    & Acc                                 & 92.94\textsuperscript{$\mathsection\mathsection$}                             & -                         & 95.29                                 & \textbf{95.66}                                 & X2+SFT-H                          \\
% Sarc\textsubscript{Bam}      & Acc                                 & 85.10\textsuperscript{$\|$}                             & -                         & 82.28                                 & \textbf{82.55}                                & X1+SFT-E                          \\
% Senti\textsubscript{Rosen}   & M-Rec                             & 68.50\textsuperscript{$\diamondsuit$}                             & \multicolumn{1}{r}{72.60} & 72.90                                 & \textbf{73.63}                                 & PM-N                                 \\
% Stance\textsubscript{Moham}  &
% Avg(a,f)             & 71.00\textsuperscript{$\circledcirc$}                             & \multicolumn{1}{r}{69.30} & 69.79                                 & \textbf{72.70}                                 & SFT-H                                \\\cdashline{1-7} 
% Average                                          & \multicolumn{1}{c}{-}               & 77.02                             & \multicolumn{1}{r}{73.26} & 79.61                                 & \multicolumn{1}{r}{\textbf{81.31}}                                & \multicolumn{1}{c}{-}                \\ \bottomrule
% \end{tabular} 
% \caption{Model comparisons. \textbf{SOTA:} Best performance on each respective dataset. \textbf{TwE:} TweetEval~\cite{barbieri-2020-tweeteval} is a benchmark for tweet classification evaluation. \textbf{BTw:} BERTweet~\cite{nguyen-etal-2020-bertweet}, a SOTA Transformer-based pre-trained language model for English tweets. We compare using the same metrics employed on each dataset. \textbf{Metrics:} \textbf{M-$F_1$:} macro $F_1$,  \textbf{W-$F_1$:} weighted $F_1$, $F_1^{(i)}$: $F_1$ irony class, \textbf{$F_1^{(i)}$:} $F_1$ irony class, $F_1^{(s)}$: $F_1$ sarcasm class, \textbf{M-Rec:} macro recall,  \textbf{Avg(a,f)}: Average $F_1$ of the \textit{against} and \textit{in-favor} classes (three-way dataset). \textsuperscript{$\star$}~\citet{liu2020crisisbert}, \textsuperscript{$\star\star$}~\citet{waseem-2016-hateful}, \textsuperscript{$\dagger$}~\citet{davidson-2017-hateoffensive},
% \textsuperscript{$^=$}~\citet{meaney2021hahackathon},
% \textsuperscript{$\dagger\dagger$}~\citet{van-hee2018semeval}, \textsuperscript{$\ddagger$}~\citet{zampieri-2019-semeval}, \textsuperscript{$\ddagger\ddagger$}~\citet{riloff2013sarcasm}, \textsuperscript{$\mathsection$}~\citet{ptavcek2014sarcasm},  \textsuperscript{$\mathsection\mathsection$}~\citet{rajadesingan2015sarcasm}, \textsuperscript{$\|$}~\citet{bamman2015contextualized}, \textsuperscript{$\diamondsuit$}~\citet{rosenthal-2017-semeval}, \textsuperscript{$\circledcirc$}~\citet{mohammad-2016-semeval}. 
% }\label{tab:compare}
% \end{table}
%\footnote{Our best setting, X2+SFT-H, exploits two streams of data, for a total of $276$M tweets. This is still only $\sim 32$\% of the data in BERTweet.} %We discuss this further in Appendix xx. 
%As such, we set new SOTA on all tasks, therby further demonstrating the effectiveness of our proposed methods.



