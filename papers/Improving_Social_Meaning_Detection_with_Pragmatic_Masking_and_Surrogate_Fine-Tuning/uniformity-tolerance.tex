\section{Uniformity and Tolerance}\label{sec:uni-tole}
\citet{wang-2021-understanding} investigate representation quality measuring the uniformity of an embedding distribution and the tolerance to semantically similar samples. Given a dataset $D$ and an encoder $\Phi$, the uniformity metric is based on a gaussian potential kernel and is formulated as:
\begin{equation}
    Uniformity = log \mathop{\mathbb{E}}_{x_i, x_j\in D}[ e^{ t ||\Phi(x_i) - \Phi(x_j)||^2_2 } ], 
\end{equation}
where $t=2$.

The tolerance metric measures the mean of similarities of samples belonging to the same class, which defined as:
\begin{equation}
\small
    Tolerance = \mathop{\mathbb{E}}_{x_i, x_j\in D}[(\Phi(x_i)^T\Phi(x_j)) \cdot I_{l(x_i)=l(x_j)}],
\end{equation}

where $l(x_i)$ is the supervised label of sample $x_i$. $I_{l(x_i)=l(x_j)}$ is an indicator function, giving the value of $1$ for $l(x_i)=l(x_j)$ and the value of $0$ for $l(x_i)\neq l(x_j)$. In our experiments, we use gold development samples from $13$ our social meaning datasets. 
