\section{Experiments}\label{sec:experiment}
% \chiyu{Add a content guide}

\section{Data}
\label{sec:data}

We use data from the \textbf{Ultrasuite repository}\footnote{\label{fn:ultrasuite}\url{https://www.ultrax-speech.org/ultrasuite}} \citep{eshky2018ultrasuite}, consisting of synchronised ultrasound and audio data from child speech therapy sessions.
Ultrasuite currently contains three datasets of child speech.
Ultrax Typically Developing (UXTD) includes recordings of 58 typically developing children.
The remaining datasets include recordings from children with speech sound disorders collected over the course of assessment and therapy sessions: Ultrax Speech Sound Disorders  (UXSSD,  8  children)  and  Ultraphonix  (UPX,  20  children). 
Assessment sessions denote recordings at various stages of therapy: baseline (before therapy),  mid-therapy,  post-therapy  (immediately after therapy), and maintenance (several months after therapy).
For the child speech datasets, ultrasound was recorded with an Ultrasonix SonixRP machine using Articulate Assistant Advanced (AAA, \cite{articulate2010articulate}) software at $\sim$120fps with a 135\degree ~field of view.
A single B-Mode ultrasound frame has 412 echo returns for each of 63 scan lines, giving a $63\times412$ \enquote{raw} ultrasound frame capturing a mid-sagittal view of the tongue.
Samples from this data are illustrated in Figure \ref{fig:td_samples}.

To complement the Ultrasuite repository, we use the \textbf{Tongue and Lips corpus}\footnote{Available via the Ultrasuite Repository, see footnote \ref{fn:ultrasuite}.}(TaL, \cite{ribeiro2021tal}).
TaL is a corpus of synchronised ultrasound, audio, and lip videos from 82 adult native speakers of English.
Ultrasound in the TaL corpus was recorded using Articulate Instruments' Micro system \citep{articulate2010articulate} at $\sim$80fps with a 92\degree ~field of view.
TaL used a different transducer than the one in the Ultrasuite data collection.
Because of this, an ultrasound frame of the TaL corpus contains  842  echo  returns  for  each  of  64  scan  lines ($64\times842$ \enquote{raw} ultrasound frame).


\subsection{Implementation and Baselines}\label{subsec:baseline} 
%---------------------------------------
For both our experiments on PMLM (Section~\ref{subsec:pmlm_res}) and SFT (Section~\ref{subsec:sft_result}), we use the pre-trained English RoBERTa\textsubscript{Base}~\cite{liu2019roberta} model as the initial checkpoint model. We use this model, rather than a larger language model, since we run a large number of experiments and needed to be efficient with GPUs. We use the RoBERTa~\footnote{For short, we refer to the official released English RoBERTa\textsubscript{Base} as RoBERTa in the rest of the paper.} tokenizer to process each input sequence and pad or truncate the sequence to a maximal length of $64$ BPE tokens. We continue training RoBERTa with our proposed methods for five epochs with a batch size of $8,192$ and then fine-tune the further trained models on downstream datasets. We provide details about our hyper-parameters in Appendix\ref{subsec:models:hyperparameter}. Our \textbf{baseline (1)} fine-tunes original pre-trained RoBERTa on downstream datsets without any further training. %~\footnote{In this paper, further pre-training indicates that we take RoBERTa checkpoint and further train it on MLM objectives.}. 
Our \textbf{baseline (2)} fine-tunes a SOTA Transformer-based PLM for English tweets, i.e., BERTweet~\cite{nguyen-etal-2020-bertweet}, on downstream datasets. For PMLM experiments, we provide \textbf{baseline (3)}, which further pre-trains RoBERTa on \texttt{Naive-Remove} dataset with the random masking strategy and MLM objectives. We refer to this model as RM-NR. We now present our results.

%RM-NR surpasses RoBERTa\textsubscript{Base} on 10 out of 15 tasks and obtains slightly improvement on average macro F\textsubscript{1} (i.e., $0.02$).

