\vspace{-15pt}
%%%%%%%%%%%%%%%%%%%%%%%%%%%%%%%%%%%%%%%%%%%%%%%%%%
\subsection{SFT Experiments}\label{subsec:sft_result}
%%%%%%%%%%%%%%%%%%%%%%%%%%%%%%%%%%%%%%%%%%%%%%%%%%


% \textit{\textbf{Surrogate fine-tuning with Emojis.}}
%--------------------------
We conduct SFT using hashtags and emojis. We continue training the original RoBERTa on the \texttt{Hashtag\_pred} and \texttt{Emoji\_pred} dataset for $35$ epochs and refer to these trained models as \textbf{SFT-H} and \textbf{SFT-E}, respectively. 
%\footnote{As an aside, the surrogate fine-tuned model achieves $32.84$ accuracy and $26.21$ macro $F_1$ on the emoji prediction Test set.} 
To evaluate SFT-H and SFT-E, we further fine-tune the obtained models on $15$ task-specific datasets. 
As Table~\ref{tab:sft_res} shows, SFT-E outperforms the first baseline (i.e., RoBERTa) with $1.16$ $F_1$ scores. Comparing SFT-E and PMLM trained with the same dataset (PM-EE), we observe that the two models perform similarly ($76.94$ for SFT-E vs. $76.96$ for PM-EE). Our proposed SFT-H method is also highly effective. On average, SFT-H achieves $2.19$ and $0.87$ $F_1$ improvement over our baseline (1) and (2), respectively. SFT-H also yields sizeable improvements on datasets with smaller training samples, such Irony\textsubscript{Hee-B} (improvement of $7.84$ $F_1$) and Sarc\textsubscript{Riloff} (improvement of $6.65$ $F_1$). Comparing SFT-H to the PMLM model trained with the same dataset (i.e., PM-HE), we observe that SFT-H also outperforms PM-H with $1.38$ $F_1$. This result indicate that SFT can more effectively utilize the information from tweets with hashtags.
% achieve the best performance of $6$ tasks and an average $F_1$ of $77.43$.

% Please add the following required packages to your document preamble:
% \usepackage{booktabs}
\begin{table}[h]
\centering
\tiny
\begin{tabular}{lc:cc:cc|c}
\toprule
\multicolumn{1}{c}{\textbf{Task}}                     & \textbf{RB}    & \textbf{SFT-E} & \textbf{SFT-H} & \multicolumn{1}{c}{\textbf{PragS1}} & \multicolumn{1}{c|}{\textbf{PragS2}} &  \textbf{BTw}  \\ \midrule
Crisis\textsubscript{Oltea} & 95.95                           & \multicolumn{1}{c}{95.76}       & 95.87                           & \textbf{96.02}    & 95.68               &  95.88               \\
Emo\textsubscript{Moham}    & 77.99                           & 79.69                           & 78.69                           & \textbf{82.04}    & 80.50   & 80.14                            \\
Hate\textsubscript{Waseem}  & 57.34                           & 56.47                           & \textbf{63.97} & 60.92                              & 60.25           &  57.47                   \\
Hate\textsubscript{David}   & \textbf{77.71} & 76.45                           & 77.29                           & 77.00                              & 76.93           &   77.15                  \\
Humor\textsubscript{Potash} & 54.40                           & 54.75                           & \textbf{55.51} & 54.93                              & 53.83        & 52.77                      \\
Humor\textsubscript{Meaney} & 92.37                           & 93.82                           & 93.74                           & 93.68                              & \textbf{94.49}    &  94.46 \\
Irony\textsubscript{Hee-A}  & 73.93                           & 76.63                           & 76.22                           & 72.73                              & \textbf{79.89}    &   77.35 \\
Irony\textsubscript{Hee-B}  & 52.30                           & 57.59                           & 60.14                           & 56.11                              & \textbf{61.67}   &   58.67 \\
Offense\textsubscript{Zamp} & 80.13                           & 80.18                           & 79.82                           & \textbf{81.34}    & 79.50         & 78.49                     \\
Sarc\textsubscript{Riloff}  & 73.85                           & 78.34                           & \textbf{80.50} & 78.74                              & 80.49         &  78.81                     \\
Sarc\textsubscript{Ptacek}  & 95.09                           & 95.88                           & 96.01                           & 96.16                              & \textbf{96.24}    &  96.35  \\
Sarc\textsubscript{Rajad}   & 85.07                           & 86.80                           & 87.56                           & 87.48                              & \textbf{88.92}    &  87.58   \\
Sarc\textsubscript{Bam}     & 79.08                           & 81.48                           & 81.19                           & \textbf{82.53}    & 81.53          &  82.08                    \\
Senti\textsubscript{Rosen}  & 71.08                           & 71.27                           & 71.83                           & \textbf{72.07}    & 71.08         &   71.38                     \\
Stance\textsubscript{Moham} & 70.41                           & 69.06                           & \textbf{71.27} & 69.65                              & 70.77          & 67.41                    \\ \cdashline{1-7}
\textbf{Average}      & 75.78                           & 76.94                           & 77.97                           & 77.43                              & \textbf{78.12}    &   77.10\\ \bottomrule
\end{tabular}
\caption{Surrogate fine-tuning (SFT). \textbf{Baselines:} RB (RoBERTa) and BTw (BERTweet). \textbf{SFT-H:} SFT with hashtags. \textbf{SFT-E:} SFT with emojis. \textbf{PragS1:} PMLM with \texttt{Hashtag\_end} (best hashtag PM condition) followed by SFT-E. \textbf{PragS2:} PMLM with \texttt{Emoji\_any} (best emoji PM condition) followed by SFT-H.  }\label{tab:sft_res}  % X1= PM-HEnd; X2=PM-EAny \vspace{-8pt} 
\end{table}



% \begin{table}[h]
% \centering
% \footnotesize
% \begin{tabular}{@{}lrrH@{}}
% \toprule
% \multicolumn{1}{c}{\textbf{Task}}            & \multicolumn{1}{c}{\textbf{Baseline}} & \multicolumn{1}{c}{\textbf{SFT}} & \multicolumn{1}{H}{\textbf{TL-SFT}} \\ \midrule
% Crisis\textsubscript{Oltea} & 95.95                           & -0.08                               & -0.03                              \\
% Emo\textsubscript{Moham}    & 77.99                           & +0.70                               & -0.98                              \\
% Hate\textsubscript{Waseem}  & 57.34                           & +6.63                               & +0.06                              \\
% Hate\textsubscript{David}   & 77.71                           & -0.42                               & +0.03                              \\
% Humor\textsubscript{Potash} & 54.40                           & +1.11                               & -0.17                              \\
% Humor\textsubscript{Meaney} & 90.33                           & +1.69                               & -2.05                              \\
% Irony\textsubscript{Hee-A}  & 73.93                           & +2.29                               & -4.26                              \\
% Irony\textsubscript{Hee-B}  & 52.30                           & +7.84                               & -1.12                              \\
% Offense\textsubscript{Zamp} & 80.13                           & -0.31                               & -0.58                              \\
% Sarc\textsubscript{Riloff}  & 73.85                           & +6.65                               & +0.06                              \\
% Sarc\textsubscript{Ptacek}  & 95.09                           & +0.92                               & -0.08                              \\
% Sarc\textsubscript{Rajad}   & 85.07                           & +2.49                               & -0.54                              \\
% Sarc\textsubscript{Bam}     & 79.08                           & +2.11                               & -0.61                              \\
% Senti\textsubscript{Rosen}  & 71.08                           & +0.75                               & -0.70                              \\
% Stance\textsubscript{Moham} & 70.41                           & +0.86                               & -4.72                              \\\cdashline{1-4} 
% \textbf{Average}                             & 75.64                           & +2.22                               & -1.05                              \\ \bottomrule
% \end{tabular}
% \caption{Surrogate fine-tuning (SFT). \textbf{Baseline:} RoBERTa\textsubscript{BASE} without further pre-training. }\label{tab:sft_res} \vspace{-8pt}
% \end{table}

% \begin{table}[h]
% \centering
% \footnotesize
% \begin{tabular}{@{}lrrr@{}}
% \toprule
% \multicolumn{1}{c}{\textbf{Task}}            & \multicolumn{1}{c}{\textbf{RB}} & \multicolumn{1}{c}{\textbf{H-Pred}} & \multicolumn{1}{c}{\textbf{H-tok}} \\ \midrule
% Crisis\textsubscript{Oltea} & 95.95                           & 95.87                               & 95.92                              \\
% Emo\textsubscript{Moham}    & 77.99                           & 78.69                               & 77.01                              \\
% Hate\textsubscript{Waseem}  & 57.34                           & 63.97                               & 57.40                              \\
% Hate\textsubscript{David}   & 77.71                           & 77.29                               & 77.74                              \\
% Humor\textsubscript{Potash} & 54.40                            & 55.51                               & 54.23                              \\
% Humor\textsubscript{Meaney} & 90.33                           & 92.02                               & 88.28                              \\
% Irony\textsubscript{Hee-A}  & 73.93                           & 76.22                               & 69.67                              \\
% Irony\textsubscript{Hee-B}  & 52.30                            & 60.14                               & 51.18                              \\
% Offense\textsubscript{Zamp} & 80.13                           & 79.82                               & 79.55                              \\
% Sarc\textsubscript{Riloff}  & 73.85                           & 80.50                               & 73.91                              \\
% Sarc\textsubscript{Ptacek}  & 95.09                           & 96.01                               & 95.01                              \\
% Sarc\textsubscript{Rajad}   & 85.07                           & 87.56                               & 84.53                              \\
% Sarc\textsubscript{Bam}     & 79.08                           & 81.19                               & 78.47                              \\
% Senti\textsubscript{Rosen}  & 71.08                           & 71.83                               & 70.38                              \\
% Stance\textsubscript{Moham} & 70.41                           & 71.27                               & 65.69                              \\\cdashline{1-4} 
% \textbf{Average}                             & 75.64                           & 77.86                               & 74.60                              \\ \bottomrule
% \end{tabular}
% \caption{Surrogate Fine-Tuning. \textbf{RB}: RoBERTa\textsubscript{BASE}, \textbf{H-Pred}: hashtah prediction, \textbf{H-tok}: token-level hashtag identification.}
% \end{table}


% %but fails to surpass SFT-H and EA+SH. 
% We also fine-tune the best hashtag-based PLM, HE-PM, on \texttt{Emoji\_pred} dataset for $35$ epochs. %\footnote{The surrogate fine-tuned model achieves $33.19$ accuracy and $26.94$ macro $F_1$ on the Test set of \textt{Emoji\_pred}. \hl{Move these footnotes to appendix?}}
% This model is denoted HE+SE in Table~\ref{tab:mtl_res}. The HE+SE achieve the best performance of $6$ tasks and an average $F_1$ of $77.43$. 
 
% \textit{\textbf{Surrogate fine-tuning with hashtags.}}
%----------------------------------------
% We also perform SFT exploiting hashtags. we use the original RoBERTa model as our initial checkpoint and surrogate fine-tune on \texttt{Hashtag\_pred} for $35$ epochs.  %\footnote{The surrogate fine-tuned model achieves $61.24$ accuracy and $38.21$ macro $F_1$ on the Test set of \textt{Hashtag\_pred}.} 

% We then continue fine-tuning obtained models on each of the $15$ downstream datasets for evaluation. As Table~\ref{tab:sft_res} shows,  



%\footnote{The surrogate fine-tuned model achieves $61.25$ accuracy and $38.14$ macro $F_1$ on the Test set of \textt{Hashtag\_pred}.}  and continue fine-tuning EA+SH on the downstream tasks. As Table~\ref{tab:mtl_res} reveals, EA+SH achieves the best averaged macro $F_1$ (i.e., $78.12$). %and outperforms the SoTA pre-trained language model, BERTweet~\cite{nguyen-etal-2020-bertweet} with $1.00$ macro $F_1$.



%Same as downstream fine-tuning models, HP-FT pass the final hidden state of [CLS] token to a classification non-linear layer. We take model trained with 35 epochs~\footnote{The model is still training for 50 epochs. The model of 35th epoch is our latest and best model so far. } that achieves $61.24$ accuracy and $38.21$ macro $F_1$ on Test set of Hashtag\_pred. 


%that the HP-FT achieve significant improvement comparing with RoBERTa model. HP-FT outperforms RoBERTa with $2.22$ average macro $F_1$ over 15 tasks. Comparing with HE-P, HP-FT more effectively utilize the hashtag data with $1.48$ improvement. HP-FT 



%This suggests that fine-tuning on hashtag prediction as intermediate task is a more useful approach to adapt pre-trained LM to social media.

%\noindent\textbf{Hashtag Identification.}~~ To investigate the utility of token-level hashtag identification as intermediate task, we fine-tune RoBERTa on \texttt{Hashtag\_token} dataset for $30$ epochs and refer the fine-tuned model to \texttt{HT-FT}. HT-FT takes the final states of all tokens from Transformer layers and passes the hidden states to a non-linear layer of classification. The best HT-FT is fine-tuned with 30 epochs, which performs $98.14$ accuracy and $88.72$ macro $F_1$. HT-FT then is evaluated on $15$ downstream tasks. However, we observe that HT-FT obtain negative performance on $12$ tasks. The average macro $F_1$ is $75.59$.
\vspace{-5pt}
\subsection{Combining PM and SFT}\label{subsec:combine}
To further improve the PMLM with SFT, we take the best hashtag-based model (i.e., PM-HE in Table~\ref{tab:pmlm_res}) and fine-tune on \texttt{Emoji\_pred} (i.e., SFT-E) for $35$ epochs. We refer to this last setting as PM-HE+SFT-E but use the easier alias \textbf{PragS1} in Table~\ref{tab:sft_res}. We observe that PragS1 outperforms both, reaching an average $F_1$ of $77.43$ vs. $75.78$ for the baseline (1) and $76.94$ for SFT-E. Similarly, we also take the best emoji-based PMLM (i.e., PM-EA in Table~\ref{tab:pmlm_res}) and fine-tune on \texttt{Hashtag\_pred} SFT (i.e., SFT-H) for $35$ epochs. This last setting is referred to as PM-EA+SFT-H, but we again use the easier alias \textbf{PragS2}. Our best result is achieved with a combination of PM with emojis and SFT on hashtags (the PragS2 condition). This last model achieves an average $F_1$ of $78.12$ and is $2.34$ and $1.02$ average points higher than baselines of RoBERTa and BERTweet, respectively. 
 
