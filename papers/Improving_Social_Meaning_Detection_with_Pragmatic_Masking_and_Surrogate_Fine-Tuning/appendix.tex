% \clearpage
% \appendix
\appendixpage

\numberwithin{figure}{section}
\numberwithin{table}{section}
% %%%%%%%%%%%%%%%%%%%%%%%%%%%%%%%%%%%%%%%%%%%%%%%%% 
% \section{Data Crawling and Preparation}\label{app:sec:crawling}
% %%%%%%%%%%%%%%%%%%%%%%%%%%%%%%%%%%%%%%%%%%%%%%%%% 
% We use the Twitter API~\footnote{\url{https://developer.twitter.com/}} to crawl datasets which are available only in tweet ID form. We note that we could not download all tweets since some tweets get deleted by users or become inaccessible for some other reason. Since some datasets are old (dating back to 2013), we are only able to retrieve $73\%$ of the tweets on average (i.e., across the different datasets). %This inaccessibility of the data motivates us to paraphrase the datasets as we explain in Section~\ref{sec:para_model}. Before we paraphrase the data or use it in our various experiments, 
% We normalize each tweet by replacing the user names and hyperlinks to the special tokens `USER' and `URL', respectively. %This ensures our paraphrased dataset will never have any actual usernames or hyperlinks, thereby protecting user identity. 
% For datasets collected based on hashtags by original authors (i.e., \textit{distant supervision}), we also remove the seed hashtags from the original tweets. For datasets originally used in cross-validation, we acquire $80\%$ Train, $10\%$ Dev, and $10\%$ Test via random splits. For datasets that had training and test splits but not development data, we split off $10\%$ from training data into Dev. The data splits of each dataset are presented in Table~\ref{tab:gold_data}. %We now introduce our pragmatic masking method.
% \section{The Semantic Urban Mesh Dataset}\label{sec:framework}
\subsection{Dataset Specification}

We have used Helsinki's 3D texture meshes as input and annotated them as a benchmark dataset of semantic urban meshes. 
The Helsinki's raw dataset covers about 12 $ km^2 $, and it was generated in 2017 from oblique aerial images that have about a 7.5 $cm$  ground sampling distance (GSD) using an off-the-shelf commercial software namely ContextCapture~\citep{contextcap}.
The source images have three colour channels (i.e., red, green, and blue) and are collected from an airplane with five cameras that have $80\%$ length coverage and $60\%$ side coverage.
To recover the 3D water bodies that do not fulfil the Lambertian hypothesis, 2D vector maps and ortho-photos are used when performing the surface reconstruction.
Furthermore, processing like aerial triangulation, dense image matching, and mesh surface reconstruction were all performed with ContextCapture.
It should be noticed that the entire region of Helsinki is split into tiles, and each of them covers about 250 $ m^2 $~\citep{kalasatamaReport}.
As shown in Figure \ref{fig:overview},  we have selected the central region of Helsinki as the study area, which includes 64 tiles and covers about 4 $km^2$ map area (8 $km^2$ surface area) in total.   

\subsection{Object Classes}
We define the semantic categories for urban meshes by the most common objects in the urban environment with unambiguous geometry and texture appearance.
Moreover, each triangle face is assigned to a label of one of the six semantic classes. 
Ambiguous regions (which account for about 2.6\% of the total mesh surface area), such as shadowed regions or distorted surfaces, are labelled as unclassified (see Figure \ref{fig:ambigious}).
The object classes we consider in the benchmark dataset are: 
\begin{itemize}
	\item \textbf{terrain}: roads, bridges, grass fields, and impervious surfaces;
	\item \textbf{building}: houses,high-rises, monuments, and security booths;
	\item \textbf{high vegetation}: trees, shrubs, and bushes;
	\item \textbf{water}: rivers, sea, and pools;
	\item \textbf{vehicle}: cars, buses, and lorries;  
	\item \textbf{boat}: boats, ships, freighters, and sailboats;
	\item \textbf{unclassified}: incomplete objects like buses and trains, distorted surfaces like tables, tents and facades, construction sites, underground walls.
\end{itemize}

\begin{figure}[!tb]
	\includegraphics[height=0.48\textwidth]{figures/overview_grids/yaxis.png}
	\begin{subfigure}[t]{0.48\textwidth}
		\includegraphics[width=\linewidth]{figures/overview_grids/texture_global_birdsview00.png}
		\includegraphics[width=\linewidth]{figures/overview_grids/xaxis.png}
		\label{fig:textop}
	\end{subfigure}
	\hspace*{\fill}
	\begin{subfigure}[t]{0.48\textwidth}		
		\includegraphics[width=\linewidth]{figures/overview_grids/semantic_global_birdsview00.png}
		\vspace*{-0.78cm}
		\begin{center}
		\includegraphics[width=0.8\linewidth]{figures/semantic_results/semantic_legend2.png}
		\end{center}
		\label{fig:semtop}
	\end{subfigure}
	\vspace*{-0.7cm}
	\caption{Overview of the semantic urban mesh benchmark.
	Left: the texture meshes covering about 4 $km^2$ map area. Right: the ground truth meshes.
	More views of the same scene (with different visualization styles) are shown in Figures \ref{fig:texside} and \ref{fig:semside}.}
	\label{fig:overview}
\end{figure}

\begin{figure}[!tb]
	\centering
	\begin{subfigure}[t]{0.48\textwidth}
		\includegraphics[width=\linewidth]{figures/ambigious/shadow_tex_zoom.png}
		\caption{}
	\end{subfigure}
	\hspace*{\fill}
	\begin{subfigure}[t]{0.48\textwidth}
		\includegraphics[width=\linewidth]{figures/ambigious/shadow_fc_zoom.png}
		\caption{}
	\end{subfigure}
	\begin{subfigure}[t]{0.48\textwidth}
		\includegraphics[width=\linewidth]{figures/ambigious/distort_tex_zoom.png}
		\caption{}
	\end{subfigure}
	\hspace*{\fill}
	\begin{subfigure}[t]{0.48\textwidth}
		\includegraphics[width=\linewidth]{figures/ambigious/distort_fc_zoom.png}
		\caption{}
	\end{subfigure}
	\caption{Ambiguous regions are labelled as unclassified (in black). 
		(a) Shadow region with texture.
		(b) Shadow region with semantic colour.
		(c) Distorted region with texture.
		(d) Distorted region with semantic colour.} 
	\label{fig:ambigious}
\end{figure}


\subsection{Semi-automatic Mesh Annotation}  \label{sec:mesh_annota}
Rather than manually labelling each triangle face of the raw meshes, we design a semi-automatic mesh labelling framework to accelerate the labelling process. Figure~\ref{fig:pipeline} shows the overall pipeline of our labelling workflow.

Given the fact that urban environments consist of a large number of planar regions in the data, we opt to label the data at the segment level instead of individual triangle faces. 
Specifically, we over-segment the input meshes into a set of planar segments. 
These segments can enrich local contextual information for feature extraction and serve as the basic annotation unit to improve annotation efficiency.

\begin{figure}[!tb]
	\centering
	\includegraphics[width=\textwidth]{figures/pipeline/pipeline_L1.png}
	\caption{The pipeline of the labelling workflow.}
	\label{fig:pipeline}
\end{figure}

Instead of randomly choosing a mesh tile as input for annotation and refinement, which is insufficient for manual annotation progress, we favour picking a mesh tile that is more difficult to classify.
Similar to active learning, we first compute the feature diversity (see Equation \ref{eq:fea_div}) to optimally select a mesh tile containing a variety of classes and objects at different scales and complexity.
The feature diversity $F_{m}$ of tile $m$ is computed as
\begin{equation}\label{eq:fea_div}
	F_{m}=\frac{\sum_{i=1}^{N_{f}}\left ( f_i - \bar{f} \right )^{2}}{N_{f}}
\end{equation}
where $f_i$ represents each handcrafted feature which describe in Section \ref{sec:initial_seg}, and $\bar{f}$ is mean value of a $N_{f}$ dimensional feature vector.
To acquire the first ground truth data, we manually annotate the mesh (with segments) that is selected with the highest feature diversity.
Then, we add the first labelled mesh into the training dataset for the supervised classification.
Specifically, we use the segment-based features as input for the classifier, and the output is a pre-labelled mesh dataset.
Next, we use the mesh annotation tool to manually refine the pre-labelled mesh according to the feature diversity.
Finally, the new refined mesh will be added to the training dataset to improve the automatic classification accuracy incrementally.


\subsubsection{Initial Segmentation}\label{sec:initial_seg}

To avoid redundant computations of numerous triangles, we first apply mesh over-segmentation (i.e., linear least-squares fitting of planes) based on region growing on the input data to group triangle faces into homogeneous regions~\citep{lafarge2012creating}.
Such grouped regions are beneficial for computing local contextual features.
We then extract both geometric and radiometric features from those mesh segments as follows: 
\begin{itemize}
	\item[$\bullet$] \textit{Eigen-based features} are computed from the covariance matrix of the triangle vertices with respect to the average centre within each segment, which is beneficial for identifying urban objects with various surface distributions.
	The linearity $= (\lambda_{1} - \lambda_{2}) / \lambda_{1}$, sphericity $= \lambda_{3}/ \lambda_{1}$ and change of curvature $= \lambda_{3} / (\lambda_{1} + \lambda_{2} + \lambda_{3})$ are computed based on the three eigenvalues $\lambda_{1} \geq \lambda_{2} \geq \lambda_{3}\geq 0$.
	The local eigenvectors $\mathbf{n}_{i} $ and the unit normal vector $\mathbf{n}_{z} $ along Z-axis are used to compute the verticality $=1-\left | \mathbf{n}_{i}\cdot \mathbf{n}_{z} \right | $~\citep{hackel2016fast}.
	Note that many eigen-based features have been studied in literature~\citep{hackel2016fast,west2004context,weinmann2013feature}, and some of them were designed for and tested on LiDAR point clouds. 
	\textcolor{ao}{
	These eigen-based features are mostly computed per point based on its spherical neighbourhood, which often contains noise and does not form a surface. 
	Our chosen eigen-based features are defined on a segment representing the surface of a mesh, and thus they can capture non-local geometric properties of an object.
	}
	Additionally, in this work, we have tested all eigen-based features from the literature~\citep{hackel2016fast}, and we only present the ones that are effective for texture meshes.
	\item[$\bullet$] \textit{Elevation} is divided into absolute elevation $z_{a}$, relative elevation $z_{r}$ and multiscale elevations $z_{m}$.
	Where $z_{a}$ is the average elevation of the segment;
	the relative elevation is computed as $z_{r} = z_{a}-z_{r_{min}}$;
	the multiscale elevation~\citep{Verdie2015,Rouhani2017} $z_{m} = \sqrt{\frac{z_{a} - z_{min}}{z_{max} - z_{min}}}$.
	And $z_{r_{min}}$ denotes the lowest elevation of the local largest ground segment computed within a cylindrical neighbourhood with 30 meters radius around the segment centre.
	$z_{min}$ and $z_{max}$ represent the local minimum and maximum elevation values of a cylindrical neighbourhood within the scale of 10 meters, 20 meters, and 40 meters.
	Such large cylindrical neighbourhoods allow to find the local ground considering the resilience to hilly environments, \textcolor{ao}{and the square root ensures that small relative height values (i.e., values smaller than 1 $ m $) get a larger elevation attribute to enlarge elevation differences between small objects and the local ground (e.g., cars against the ground, boats against the water surfaces).}
	More importantly, due to the influence of terrain fluctuations and various scales of urban objects, the elevation of these three categories can complement each other.
	\item[$\bullet$] \textit{Segment area} is computed as $area(S_k) = \sum_{i = 1}^{N} area(f_i) $, where $f_i$ denotes a triangle of the segment $S_k$, and $N$ denotes the total number of triangles in $S_k$.
	\item[$\bullet$] \textit{Triangle density} is defined as $density(S_k) = \frac{N}{area(S_k)} $,  which reveals the object complexity, especially for adaptive urban meshes.
	\item[$\bullet$] \textit{Interior radius of 3D medial axis transform (InMAT)}~\citep{ma20123d,peters2016robust} of a segment $S_k$ is formulated as $r_k = \frac{\sum_{i=1}^{M} r_i}{M}$, where $M$ denotes the total number of triangle vertices of $S_k$, and $r_i$ denotes the interior radius of the shrinking ball that touches the vertex $v_i$ within the segment $S_k$. 
	It is designed to distinguish objects with different scales. 
	\item[$\bullet$] \textit{HSV colour-based features} are derived from the RGB channel of the entire texture map.
	We use the HSV colour space since it can better differentiate different objects than RGB.
	We compute the average colour, the variance of the colour distribution of all pixels within each segment, and we further discretize it into a histogram that consists of 15 bins of the hue channel, five bins of the saturation channel, and five bins of the value channel.
	\item[$\bullet$] \textit{Greenness} $a_{g}$ is used to classify objects that are similar to green vegetation.
	Specifically, it is computed according to the averaged RGB colour of each segment via $a_{g}=G-0.39\cdot R-0.61\cdot B$~\citep{mckinnon2017comparing}. 
\end{itemize}
	All the above features are concatenated into a 44-dimensional feature vector used by our random forest (RF) classifier in the initial segmentation. 

\subsubsection{Annotation Tool for Refinement}

Because of the under-segmentation errors and the imperfect results of the semantic mesh segmentation process, we design a mesh annotation tool (see Figure \ref{fig:annotator}) to manually correct the labelling errors.
Our mesh annotation tool is developed based on the labelling tool of CGAL~\citep{cgal:eb-20b}.

\begin{figure}[!tb]
	\centering
	\includegraphics[width=\textwidth]{figures/annotator/annotator.png}
	\caption{The interface of our annotation tool for 3D texture meshes. }
	\label{fig:annotator}
\end{figure}

As shown in Table \ref{tab:annotation_operation}, it consists of three operation categories: view, selection, and annotation.
The	view operations provide essential functions for the user to manipulate the scene camera, such as translate, rotate, zoom, or set the new pivot for the scene.
In addition, to use textures as a reference for labelling, we map texture and face colour with a certain degree of transparency, and we visualize the segment border to differentiate each segment. 

\begin{table}[!tb]
	\centering
	\noindent\adjustbox{max width=0.8\textwidth}
	{
		\begin{threeparttable}
			\centering
			\begin{tabular}{ccc}
				\toprule
				Categories & Operations & Objects \\
				\midrule
				\multirow{4}[2]{*}{View} & Translate & Camera \\
				& Rotate & Camera \\
				& Zoom in / out & Camera \\
				& Set pivot & Camera \\
				\midrule
				\multirow{6}[2]{*}{Selection} & Multi-selection / Lasso & Triangles / Segments \\
				& Expand / Reduce & Triangles / Segments \\
				& Semantic selection & Segments \\
				& Split region & Segments \\
				& Planar region extraction & Triangles \\
				& Split mesh & Triangles \\
				\midrule
				\multirow{3}[2]{*}{Annotation} & Probability slider & Segments \\
				& Segment area slider & Segments \\
				& Progress bar & Triangles \\
				& Switch semantic view & Triangles \\ 
				& Labelling & Triangles / Segments \\
				\bottomrule
			\end{tabular}%
		\end{threeparttable}
	}
	\caption{Basic operations in our annotation tool.} 
	\label{tab:annotation_operation}%
\end{table}%


The	selection operations allow the user to select or deselect either triangle faces (see Figure \ref{fig:tri_sel}) or segments (see Figure \ref{fig:seg_sel}) freely via a brush or a lasso.
Specifically, the face selection operation is used to fix the under-segmentation errors and generate new segments, and the segment selection operation is to fix incorrect segment labels.

\begin{figure}[!tb]
	\centering
	\begin{subfigure}[t]{0.32\textwidth}
		\includegraphics[width=\linewidth]{figures/pipeline/tri_select_a.png}
		\caption{}
	\end{subfigure}
	\hspace*{\fill}
	\begin{subfigure}[t]{0.32\textwidth}
		\includegraphics[width=\linewidth]{figures/pipeline/tri_select_b.png}
		\caption{}
	\end{subfigure}
	\hspace*{\fill}
	\begin{subfigure}[t]{0.32\textwidth}
		\includegraphics[width=\linewidth]{figures/pipeline/tri_select_c.png}
		\caption{}
	\end{subfigure}
	\caption{An example of labelling by selecting triangles using the lasso tool (blue edges: segment boundaries). 
		(a) Before selection.
		(b) Lasso selection result (in red).
		(c) The correct label has been assigned to the selected region. 
		In this example, the label of the selected region has been changed from `ground' to `vehicle'.
	} 
	\label{fig:tri_sel}
\end{figure}


\begin{figure}[!tb]
	\centering
	\begin{subfigure}[t]{0.32\textwidth}
		\includegraphics[width=\linewidth]{figures/pipeline/seg_select_a.png}
		\caption{}
	\end{subfigure}
	\hspace*{\fill}
	\begin{subfigure}[t]{0.32\textwidth}
		\includegraphics[width=\linewidth]{figures/pipeline/seg_select_b.png}
		\caption{}
	\end{subfigure}
	\hspace*{\fill}
	\begin{subfigure}[t]{0.32\textwidth}
		\includegraphics[width=\linewidth]{figures/pipeline/seg_select_c.png}
		\caption{}
	\end{subfigure}
	\caption{An example of segment labelling. 
		(a) Part of a wall of the building was previously labelled as `high vegetation' (in green).
		(b) Segment selection result (in red).
		(c) The label of the selected segment has been corrected with the new label `building'.
	}
	\label{fig:seg_sel}
\end{figure}

We also allow the user to edit the selection of each individual segment with splitting functions (see Figure \ref{fig:pnp_func}) and automatic extraction of the most planar region (see Figure \ref{fig:seg_func}). 
As for splitting, we first detect the potential planar and non-planar segments marked by user strokes, and then the non-planar one is split according to the vertex-to-plane distance.
It allows generating candidate non-planar regions (with respect to the detected planar segment) for the user to edit, and
it is useful to split a segment that covers large non-planar regions or contains more than one dominant planar area.
To extract the most planar region, we apply the region growing algorithm~\citep{lafarge2012creating} within the selected segment to automatically generate the candidate triangle faces with user-defined thresholds (i.e., the maximum distance to the plane, the maximum accepted angle, and the minimum region size).
Such an operation allows the user to filter out some small bumpy regions of the selected segment.

\begin{figure}[!tb]
	\centering
	\begin{subfigure}[t]{0.48\textwidth}
		\includegraphics[width=\linewidth]{figures/annotator/pnp_pipeline1.png}
		\caption{}
	\end{subfigure}
	\hspace*{\fill}
	\begin{subfigure}[t]{0.48\textwidth}
		\includegraphics[width=\linewidth]{figures/annotator/pnp_pipeline2.png}
		\caption{}
	\end{subfigure}
	\caption{An example splitting planar and non-planar regions. 
		(a) The user draws a stroke (in red) across the border of the non-planar segment and the planar segment. 
		(b) The detected non-planar segment has been split into two parts (i.e., a non-planar region shown in red and a planar segment shown in green).
	} 
	\label{fig:pnp_func}
\end{figure}

\begin{figure}[!tb]
	\centering
	\begin{subfigure}[t]{0.48\textwidth}
		\includegraphics[width=\linewidth]{figures/annotator/planar_split_pipeline1.png}
		\caption{}
	\end{subfigure}
	\hspace*{\fill}
	\begin{subfigure}[t]{0.48\textwidth}
		\includegraphics[width=\linewidth]{figures/annotator/planar_split_pipeline3.png}
		\caption{}
	\end{subfigure}
	\caption{Editing an individual segment. 
		(a) A segment is selected (highlighted in green) for splitting. 
		(b) Automatic extraction of the most planar region (shown in red) within the selected segment according to user-defined thresholds.} 
	\label{fig:seg_func}
\end{figure}

Besides, probability and area-based sliders and a progress bar are provided in the annotation panel to improve annotation efficiency and experience, respectively. 
Specifically, the probability slider is introduced for the user to visually inspect the segments that are most likely misclassified.
Moreover, the user can further use it to inspect a specific class by switching the view to highlight a specific semantic class.
The segment area slider is used to identify isolated tiny segments, which commonly appear as errors.
The progress bar is used to indicate the estimated labelling progress during the annotation.
After performing the selection, the user can easily assign the corresponding label to the selected area.



% \hl{Check T5 and hyper-parameter}
% \section{Low-Voltage Load Forecasting Datasets} 
\label{secdatasets}

A number of interesting features were discovered about the data in the reviewed 221 papers. Firstly, only 52 use at least one openly available datasets to illustrate the results, i.e. less than 24\% of the journals presented results that could be potentially replicated by the wider research community. Of these 52 papers using open data, 22 (or $42\%$)  of them used the Irish CER Smart Metering Project data~\cite{Commission2012csm}, four used data from UK Low Carbon London project~\cite{UK2014ulc}, four from Ausgrid\footnote{\url{https://www.ausgrid.com.au/Industry/Our-Research/Data-to-share/Solar-home-electricity-data}} and three used the UMass dataset. In other words, out of the papers using open data, $56\%$, presented results that used data from only four open data sets. 

The overuse of a particular dataset can result in biases (both conscious and unconscious) where methods are developed and tested but the features of the data may be well known or familiar from overuse. In these cases a scientifically rigorous experiment is impossible. Further, reliance on a single dataset (especially those which are no more than 2 years in length like the Irish CER dataset~\cite{Commission2012csm}) risks the development of models which may be based on spurious features and patterns and may not be representative of the wider energy system. This can be alleviated somewhat by including multiple open data sets, as it was done in some of the papers reviewed (\cite{abera2020mla, Laurinec2019due, Wang2018aef}). To support the LV forecast research community the authors are going to share a modifiable list of open data sets at the LV level. A list and some of the properties of the data are shown in Table \ref{tab:datasets}.  

With the rapid change in low carbon technologies being connected to the grid and home, new energy efficiency interventions, and adjustments in demand usage behaviours, demand data can quickly become irrelevant or unrepresentative. Further, such data sets are based on trials where participants are subject to incentives or other interventions. For example, different tariffs were considered for some households in the Irish CER dataset~\cite{Commission2012csm}. This means that their demand may not represent `normal', every-day behaviour.  

There are now some initiatives that are attempting to solve some of the issues of sparse and intermittent open data produced by limited innovation projects. An example of this is the Smart Energy Research Lab from UCL in the  UK\footnote{\url{https://www.ucl.ac.uk/bartlett/energy/research/energy-and-buildings/smart-energy-research-group-serg}} which is attempting to make smart meter data (as well as other useful data sets associated with the same homes) available on an ongoing basis for research from a wider range of participants as well as a large control group of households which will not participate in any initiatives or trials. 

\subsection{Data resolution}
Resolution is another important aspect of the data. Half-hourly data is the standard resolution of smart meter data although it can be as low as 10 or 15 minutes. The smallest stated resolution of the data within the papers had 48 with hourly data, 51 half-hourly, 23 as 15 minutes and 8 as 10 minutes. There was also a large number of papers where the resolution was not clear or no real data was presented (62 papers). Data with resolutions of between ten minutes and an hour are probably sufficient for demand control applications and are likely representative of what data is available in practice. However, this isn't sufficient for more high-resolution applications such as voltage control. In fact, only 11 papers considered data of resolution of 1 minute or less. This could pose a difficulty for validating the common voltage and Var control application in this review (see Section~\ref{sec:LVLF-applications}). A large number of papers where the resolution is not clear should also be a concern, as this prevents recreation of the results. 

\subsection{Forecast horizon}
Another crucial aspect of this review is the forecast horizon. Different horizons are useful for different applications. Short term (day to the week ahead) are typical for operational time scales, whereas long-term forecasts (over a year) are more useful for planning. The majority of the papers reviewed were at a short term time scale with 80 of the papers considering day ahead forecasts, another common horizon was an hour ahead. Very few papers went at shorter horizons than an hour (twelve). There were slightly more papers that forecast beyond a day (16 were between 2 days and a week ahead) and only 13 papers were at horizons of a month or more. Once again there was a large number of papers, 80 in total, where the horizon length was not identifiable. 

\subsection{Overview of LV datasets}
As we have already mentioned, the current choice of open datasets that can be used for a benchmark is very limited, relying mostly on the CER Irish smart meter data, which is now a decade old and has some selection bias limitations (most of the houses have 3-4 bedrooms etc.). In order to continually expand research in this area, a strategy is required to regularly open more diverse datasets, converging towards common formats and standards and clarifying licences and terms of usage. Clear license information is especially relevant for industry-based research. We have established a list of open datasets (see Table~\ref{tab:datasets} and \url{https://low-voltage-loadforecasting.github.io/}) with the hope that it will continue to grow, and that new methods for privacy and safety protection (anonymisation, aggregation, synthetic 'look alike' datasets etc.) will enable more availability of datasets in the future. Finally it is vital to provide proper and thorough documentation with the data sets. The quality of the datasets is not clear and in many cases any preprocessing or data-cleaning techniques are not provided with the data.

Most datasets are at the residential level collected from smart meters. More diverse datasets of non-residential customers and different grid levels (substations and transformers) are needed for better LV forecasting research. 

Some datasets that have been cited in the literature like PLAID~\cite{Medico2020ava} OCTES, BLUED~\cite{Anderson2012baf},  and DRED~\cite{UttamaNambi2015lle} were offline at the time of writing. The Pecanstreet Dataport~\cite{Pecan2018d} database was once publicly available for research but then closed, and now only a subset is still accessible. Other datasets are available for download but are hard to trace, as identifiers like a DOI or a paper to cite are not available. All of these obstruct the reproducibility of the research. Therefore, new datasets should be published on archiving platforms like IEEE data port~\footnote{\url{https://ieee-dataport.org/}}, Zenodo~\footnote{\url{https://zenodo.org/}}, Figshare~\footnote{\url{https://figshare.com/}}, or  arXiv~\footnote{\url{https://arxiv.org/help/submit\#datasets}}. A recent contribution to more reproducible time series research is the Monash Time Series Forecasting Archive~\cite{godahewa2021mts}. It contains the dataset of the UK Low Carbon London trial~\cite{UK2014ulc} and the UCI datasets~\cite{candanedo2017ddp, Hebrail2012ihe}.


\begin{sidewaystable*}
	\tiny
	\caption{Overview of Low-voltage Load Datasets (see online version for embedded hyperlink to the data set by clicking on the name).} \label{tab:datasets}
	\resizebox{!}{0.9\height}{
		\begin{tabular}{p{0.16\linewidth}p{0.05\linewidth}p{0.04\linewidth}p{0.05\linewidth}p{0.04\linewidth}p{0.04\linewidth}p{0.02\linewidth}p{0.02\linewidth}p{0.12\linewidth}p{0.14\linewidth}p{0.14\linewidth}}
			\\
			\toprule
			Name & Type  & No. Customers & Resolution & Duration & Intervention & Sub-metering & Weather avail. & Location & Other data provided & Access/Licence \\
			\midrule 
			\href{https://www.ea.tuwien.ac.at//projects/adres_concept/EN/}{ADRES}~\cite{einfalt2011kfa}
			& Households & 30    & 1 s   & 2 weeks & None  & No    & No    & Austria (Upper Austria) & Voltage & Free for Research (E-Mail) \\
			\href{https://www.ausgrid.com.au/Industry/Our-Research/Data-to-share/Solar-home-electricity-data}{Ausgrid Solar Home} &  Households &  300 & 30 min &   3 years & None &  No &  No &  Australia (NSW) &  &   No Licence \\
			\href{https://www.ausgrid.com.au/Industry/Our-Research/Data-to-share/Distribution-zone-substation-data}{Ausgrid substation data} &   Substation & 225 & 15 min &  20 years & None &  No &  No &  Australia (NSW) &  & No Licence \\
			\href{https://sourceforge.net/projects/greend/}{GREEND Electrical Energy Dataset (GREEND)}~\cite{monacchi2014gae} & Households & 8     & 1 s   & 3-6 months & None  & Yes   & No    & Austria, Italy & Occupancy, Building type & Free (Access Form) \\
			\href{https://archive.ics.uci.edu/ml/datasets/Appliances+energy+prediction}{UCI Appliances}~\cite{candanedo2017ddp} & Households & 1     & 10 min & 4.5 months & None  & No    & Yes   & Belgium (Mons) & Lights, Building information & Free (No Licence) \\
			\href{https://ieee-dataport.org/open-access/industrial-machines-dataset-electrical-load-disaggregation}{INDUSTRIAL MACHINES}\cite{Bandeira2018imd}
			& Industrial & 1 & 1 Hz  & 1 month & None  & Yes   & No    & Brasil (Minas Gerais) &       & CC BY \\
			\href{https://dataverse.harvard.edu/dataset.xhtml?persistentId=doi:10.7910/DVN/ZJW4LC}{Rainforest Automation Energy} \cite{Makonin2018rtr} & Households & 2     & 1 Hz  & 2 months & None  & Yes   & Yes   & Canada & Environmental, Heat Pump,  & CC BY \\
			\href{https://dataverse.harvard.edu/dataset.xhtml?persistentId=doi\%3A10.7910/DVN/FIE0S4\%20}{AMPds2} \cite{Makonin2016ata, Makonin2016ewa}  & Households & 1     & 1min  & 2 years & None  & Yes   & Yes   & Canada (Alberta) & Gas, Water, Building Type and Plan & CC BY \\
			\href{https://carleton.ca/sbes/publications/electric-demand-profiles-downloadable/}{Sustainable Building Energy Systems 2017} \cite{Johnson2017edf} & Households & 23    & 1 min & 1 year & None  & Yes   & No    & Canada (Ottawa) & Sociodemographic (Occupants, Age, Size) & Free (Attribution, E-Mail) \\
			\href{https://carleton.ca/sbes/publications/electric-demand-profiles-downloadable/}{Sustainable Building Energy Systems 2013} \cite{Saldanha2012mee} & Households & 12    & 1 min & 1 year & None  & Yes   & No    & Canada (Ottawa) & Sociodemographic (Occupants, Age, Size) & Free (Attribution, E-Mail) \\
			\href{https://data.lab.fiware.org/organization/9569f9bd-42bd-414f-b8d9-112553ea9dfb?tags=FINESCE}{FINESCE Horsens}
			& Households & 20    & 1 h   & several days & None  & Yes   & Yes   & Denmark (Horsens) & EV, PV, Heat Pump, Heating, Smart Home,  & CC BY-SA \\
			
			\href{https://archive.ics.uci.edu/ml/datasets/Individual+household+electric+power+consumption}{UCI Individual household electric power cons.}~\cite{Hebrail2012ihe} & Households & 1     & 1min  & 4 years & None  & Yes   & No    & France (Sceaux) & Reactive Power, Voltage & CC BY \\
			\href{https://mediatum.ub.tum.de/1375836}{BLOND-50}~\cite{Kriechbaumer2017bbo} & Commerical & 1     & 50 kHz & 213 days & None  & Yes   & No    & Germany &       & CC BY \\
			\href{https://mediatum.ub.tum.de/1375836}{BLOND-250}~\cite{Kriechbaumer2017bbo} & Commerical & 1     & 250 kHz & 50 days & None  & Yes   & No    & Germany &       & CC BY \\
			\href{https://zenodo.org/record/3855575\#.YKQgGKgzaUk}{Fresh Energy}~\cite{Beyertt2020fzb} & Households & 200   & 15 min & 1 year & Behaviorial & Yes   & No    & Germany & Agegroup, Gender of main customer & CC BY \\
			\href{https://data.lab.fiware.org/organization/9569f9bd-42bd-414f-b8d9-112553ea9dfb?tags=FINESCE}{FINESCE Factory} 
			& Industrial & 1     & 1 min & 2 days & None  & Yes   & No    & Germany (Aachen) & Machines & CC BY-SA \\
			\href{https://pvspeicher.htw-berlin.de/wp-content/uploads/MFH-Lastprofil_2014_17274_kWh.csv}{HTW Lichte Weiten}~\cite{htw2019ldb}
			& Households & 1 building & 15 minute & 1 year & None  & No    & No    & Germany (Berlin) &       & Free (No Licence) \\
			\href{https://pvspeicher.htw-berlin.de/veroeffentlichungen/daten/lastprofile/}{HTW Synthetic}~\cite{Tjaden2015rel} & Households & 74    & 1 s   & 1 year & None  & No    & No    & Germany (Representative) & Synthetic dataset merging & CC BY-NC \\
			\href{https://data.open-power-system-data.org/household_data/}{CoSSMic} \cite{Open2020dph} & Households, SME & 11    & 1min, 15min, 1H & 1-3 years & None  & Yes   & No    & Germany (South) & PV, EV, Type (Residential/SME) & CC BY \\
			\href{https://im.iism.kit.edu/sciber.php}{SciBER}~\cite{Staudt2018san} & Municipal & 107   & 15min & 3 years & None  & No    & No    & Germany (South) & Type (Office, Gym, ...) & CC BY \\
			\href{https://iawe.github.io/}{iAWE}~\cite{batra2013idi}
			& Households & 1     & 1 Hz  & 2 months & None  & Yes   & No    & India (New Delhi) & Water & Free (No Licence) \\
			\href{https://combed.github.io/}{COMBED}~\cite{Batra2014aco} & Commerical & 1     & 30 s  & 1 month & None  & Yes   & No    & India (New Delhi) &       & Free (No Licence) \\
			\href{http://www.ucd.ie/issda/data/commissionforenergyregulationcer/}{Irish CER Smart Metering Project data}~\cite{Commission2012csm} & Households, SME, Other & 3835  & 30min & 1.5 years & Tariff & No    & No    & Ireland & Type (Residential/SME/Other) & Free (Signed Access Form) \\
			
			\href{https://github.com/Nikasa1889/ShortTermLoadForecasting}{Hvaler Substation Level data}~\cite{DangHa2017lst} & Substation & 20    & 1 h   & 2 years & None  & No    & No    & Norway (Hvaler) &       & Free (No Licence) \\
			\href{http://web.lums.edu.pk/~eig/CXyzsMgyXGpW1sBo}{Energy Informatics Group Pakistan}~\cite{Pereira2014sap} & Households & 42    & 1 min & 1 year & None  & Yes   & No    & Pakistan & Sociodemographic (building properties, no of people, devices) & Free (No Licence) \\
			\href{https://archive.ics.uci.edu/ml/datasets/ElectricityLoadDiagrams20112014}{UCI Electricity Load Diagrams}~\cite{Godahewa2021ehd}
			& Different & 370   & 15 min & 2 years & None  & No    & No    & Portugal &       & Free (No Licence) \\
			
			\href{http://www.vs.inf.ethz.ch/res/show.html?what=eco-data}{Electricity Consumption and Occupancy (ECO)}~\cite{Christian2014ted, Wilhelm2015hom} & Households & 6     & 1 Hz  & 8 months & None  & Yes   & No    & Switzerland & Occupancy & CC BY \\
			\href{https://www.gov.uk/government/publications/household-electricity-survey--2}{Household Electricity Survey (HES)}~\cite{Zimmermann2012hes} & Households & ~{}250 & 2 min & 1 month (255) to 1 year (26) & None  & Yes   & No    & UK    & Consumer Archetype & Request \\
			\href{https://beta.ukdataservice.ac.uk/datacatalogue/studies/study?id=8634}{METER}~\cite{Grunewald2019muh} & Households & 529   & 1 min & 28 hours & None  & No    & No    & UK    & Activity data, Sociodemographic & Free for Research (Access Form) \\
			\href{https://datashare.ed.ac.uk/handle/10283/3647}{IDEAL Household Energy Dataset}~\cite{goddard2020ihe}
			& Households & 255   & 1 s     & 3 years & None  & Yes   & No    & UK    & Smart Home, Sociodemographic, energy awareness survey, room temperature and humidity, building characteristics & CC BY \\
			\href{http://www.networkrevolution.co.uk/resources/project-data/}{Customer-Led Network Revolution project data}~\cite{sidebotham2015cln} & Households, SMEs & ~{}12000 & 30 min & > 1 year & Time of Use & No    & No    & UK    & EV, PV, Heatpump, Tariff,  & CC BY-SA \\
			\href{https://jack-kelly.com/data/}{UK Domestic Appliance-Level Electricity (UK-DALE)}~\cite{Jack2015tud} & Households & 5     & 16 kHz, 1s & months, one house > 4 years & None  & Yes   & No    & UK (London area) &       & CC BY \\
			\href{https://data.london.gov.uk/dataset/smartmeter-energy-use-data-in-london-households}{UK Low Carbon London}~\cite{UK2014ulc,Godahewa2021lsm} & Households & 5567  & 30min & 2 years & Time of Use & No    &  No   & UK (London) & CACI Acorn group & Free (No Licence) \\
			\href{https://www.refitsmarthomes.org/datasets/}{REFIT}~\cite{Murray2016rel, Murray2017ael} & Households & 20    & 8 s   & 2 years & None  & Yes   & Yes   & UK (Loughborough) & PV, Gas, Water, Sociodemographic (Occupancy, Dwelling Age, Dwelling Type, No. Bedrooms) & CC BY \\
			\href{https://www.spenergynetworks.co.uk/pages/flexible_network_data_share.aspx}{Flexible Networks for a Low Carbon Future} & Substations & Several Secondary  & 30 min & 1 year & None  & No    & No    & UK (St Andrews, Whitchurch, Ruabon) &       & Free (Access Form) \\
			\href{https://ukerc.rl.ac.uk/DC/cgi-bin/edc_search.pl?GoButton=Detail\&WantComp=146\&\&RELATED=1}{NTVV Substations} & Substation & 316   & 5 s   & > 4 years & None  & No    & No    & UK (Thames Valley) &       & Open Access (Any purpose) \\
			\href{https://ukerc.rl.ac.uk/DC/cgi-bin/edc_search.pl?GoButton=Detail\&WantComp=147\&\&RELATED=1}{NTVV Smart Meter} & Buildings & 316   & 30 min & > 4 years & None  & No    & No    & UK (Thames Valley) &       & Open Access (Any purpose) \\
			
			\href{https://site.ieee.org/pes-iss/data-sets/}{IEEE PES Open Data Sets} 
			& Households, Commercial & 15    & 1 min, 5 min, 15 min & 2 weeks & None  & No    & No    & USA   & Connection limit & Free (No Licence) \\
			\href{http://redd.csail.mit.edu/}{Reference Energy Disaggregation Data Set (REDD)}~\cite{Kolter2011rap} & Households &  ~{}10 & 1 kHz & 3-19 days & None  & Yes   & No    & USA (Boston) & Voltage & Free (Attribution, E-Mail) \\
			\href{http://wzy.ece.iastate.edu/Testsystem.html}{Iowa Distribution Test Systems}~\cite{Bu2019atd} & Substation & 240 nodes & 1 H   & 1 year & None  & Yes   & No    & USA (Iowa) & Grid data & Free (Attribution) \\
			\href{https://www.pecanstreet.org/dataport/}{Pecanstreet Dataport (Academic)}~\cite{Pecan2018d} & Households & 30    & 1min, 15min, 1H & 2-3 years & None  & Yes   & Yes   & USA (mostly Austin and Boulder) & PV, EV, Water, Gas, Sociodemographic & Free for Research (Access Form) \\
			
			\href{https://neea.org/resources/rbsa-ii-combined-database}{Residential Building Stock Assessment}~\cite{Larson2014jua} & Households & 101   & 15 min & 27 months & None  & Yes   & No    & USA (North West Region) & Building Type (Single Family, Manufactured, Multifamily) & Free (Access Form) \\
			
			\href{http://lass.cs.umass.edu/projects/smart/}{SMART* Home 2017}~\cite{Barker2012sao} & Households & 7     & 1 Hz  & > 2 years & None  & Yes   & Yes   & USA (Western Massachussets) &       & Free (No Licence) \\
			\href{http://lass.cs.umass.edu/projects/smart/}{SMART* Apartment}~\cite{Barker2012sao} & Households & 114   & 1 min & 2 years & None  & No    & Yes   & USA (Western Massachussets) &       & Free (No Licence) \\
			\href{http://lass.cs.umass.edu/projects/smart/}{SMART* Occupancy}~\cite{Barker2012sao} & Households & 2     & 1 min & 3 weeks & None  & No    & No    & USA (Western Massachussets) & Occupancy & Free (No Licence) \\
			\href{http://lass.cs.umass.edu/projects/smart/}{SMART* Microgrid}~\cite{Barker2012sao} & Households & 443   & 1 min & 1 day & None  & No    & No    & USA (Western Massachussets) &       & Free (No Licence) \\
			\href{http://lass.cs.umass.edu/projects/smart/}{SMART* Home 2013}~\cite{Barker2012sao} & Households & 3     & 1 Hz  & 3 months & None  & Yes   & No    & USA (Western Massachussets) & Solar, Wind, Environmental, Smart Home, Voltage,  & Free (No Licence) \\
			\bottomrule
		\end{tabular}
	}
\end{sidewaystable*}

% \section{Dataset}
% Table~\ref{tab:gold_data} presents the distribution of 15 social meaning datasets. 
% \section{The Semantic Urban Mesh Dataset}\label{sec:framework}
\subsection{Dataset Specification}

We have used Helsinki's 3D texture meshes as input and annotated them as a benchmark dataset of semantic urban meshes. 
The Helsinki's raw dataset covers about 12 $ km^2 $, and it was generated in 2017 from oblique aerial images that have about a 7.5 $cm$  ground sampling distance (GSD) using an off-the-shelf commercial software namely ContextCapture~\citep{contextcap}.
The source images have three colour channels (i.e., red, green, and blue) and are collected from an airplane with five cameras that have $80\%$ length coverage and $60\%$ side coverage.
To recover the 3D water bodies that do not fulfil the Lambertian hypothesis, 2D vector maps and ortho-photos are used when performing the surface reconstruction.
Furthermore, processing like aerial triangulation, dense image matching, and mesh surface reconstruction were all performed with ContextCapture.
It should be noticed that the entire region of Helsinki is split into tiles, and each of them covers about 250 $ m^2 $~\citep{kalasatamaReport}.
As shown in Figure \ref{fig:overview},  we have selected the central region of Helsinki as the study area, which includes 64 tiles and covers about 4 $km^2$ map area (8 $km^2$ surface area) in total.   

\subsection{Object Classes}
We define the semantic categories for urban meshes by the most common objects in the urban environment with unambiguous geometry and texture appearance.
Moreover, each triangle face is assigned to a label of one of the six semantic classes. 
Ambiguous regions (which account for about 2.6\% of the total mesh surface area), such as shadowed regions or distorted surfaces, are labelled as unclassified (see Figure \ref{fig:ambigious}).
The object classes we consider in the benchmark dataset are: 
\begin{itemize}
	\item \textbf{terrain}: roads, bridges, grass fields, and impervious surfaces;
	\item \textbf{building}: houses,high-rises, monuments, and security booths;
	\item \textbf{high vegetation}: trees, shrubs, and bushes;
	\item \textbf{water}: rivers, sea, and pools;
	\item \textbf{vehicle}: cars, buses, and lorries;  
	\item \textbf{boat}: boats, ships, freighters, and sailboats;
	\item \textbf{unclassified}: incomplete objects like buses and trains, distorted surfaces like tables, tents and facades, construction sites, underground walls.
\end{itemize}

\begin{figure}[!tb]
	\includegraphics[height=0.48\textwidth]{figures/overview_grids/yaxis.png}
	\begin{subfigure}[t]{0.48\textwidth}
		\includegraphics[width=\linewidth]{figures/overview_grids/texture_global_birdsview00.png}
		\includegraphics[width=\linewidth]{figures/overview_grids/xaxis.png}
		\label{fig:textop}
	\end{subfigure}
	\hspace*{\fill}
	\begin{subfigure}[t]{0.48\textwidth}		
		\includegraphics[width=\linewidth]{figures/overview_grids/semantic_global_birdsview00.png}
		\vspace*{-0.78cm}
		\begin{center}
		\includegraphics[width=0.8\linewidth]{figures/semantic_results/semantic_legend2.png}
		\end{center}
		\label{fig:semtop}
	\end{subfigure}
	\vspace*{-0.7cm}
	\caption{Overview of the semantic urban mesh benchmark.
	Left: the texture meshes covering about 4 $km^2$ map area. Right: the ground truth meshes.
	More views of the same scene (with different visualization styles) are shown in Figures \ref{fig:texside} and \ref{fig:semside}.}
	\label{fig:overview}
\end{figure}

\begin{figure}[!tb]
	\centering
	\begin{subfigure}[t]{0.48\textwidth}
		\includegraphics[width=\linewidth]{figures/ambigious/shadow_tex_zoom.png}
		\caption{}
	\end{subfigure}
	\hspace*{\fill}
	\begin{subfigure}[t]{0.48\textwidth}
		\includegraphics[width=\linewidth]{figures/ambigious/shadow_fc_zoom.png}
		\caption{}
	\end{subfigure}
	\begin{subfigure}[t]{0.48\textwidth}
		\includegraphics[width=\linewidth]{figures/ambigious/distort_tex_zoom.png}
		\caption{}
	\end{subfigure}
	\hspace*{\fill}
	\begin{subfigure}[t]{0.48\textwidth}
		\includegraphics[width=\linewidth]{figures/ambigious/distort_fc_zoom.png}
		\caption{}
	\end{subfigure}
	\caption{Ambiguous regions are labelled as unclassified (in black). 
		(a) Shadow region with texture.
		(b) Shadow region with semantic colour.
		(c) Distorted region with texture.
		(d) Distorted region with semantic colour.} 
	\label{fig:ambigious}
\end{figure}


\subsection{Semi-automatic Mesh Annotation}  \label{sec:mesh_annota}
Rather than manually labelling each triangle face of the raw meshes, we design a semi-automatic mesh labelling framework to accelerate the labelling process. Figure~\ref{fig:pipeline} shows the overall pipeline of our labelling workflow.

Given the fact that urban environments consist of a large number of planar regions in the data, we opt to label the data at the segment level instead of individual triangle faces. 
Specifically, we over-segment the input meshes into a set of planar segments. 
These segments can enrich local contextual information for feature extraction and serve as the basic annotation unit to improve annotation efficiency.

\begin{figure}[!tb]
	\centering
	\includegraphics[width=\textwidth]{figures/pipeline/pipeline_L1.png}
	\caption{The pipeline of the labelling workflow.}
	\label{fig:pipeline}
\end{figure}

Instead of randomly choosing a mesh tile as input for annotation and refinement, which is insufficient for manual annotation progress, we favour picking a mesh tile that is more difficult to classify.
Similar to active learning, we first compute the feature diversity (see Equation \ref{eq:fea_div}) to optimally select a mesh tile containing a variety of classes and objects at different scales and complexity.
The feature diversity $F_{m}$ of tile $m$ is computed as
\begin{equation}\label{eq:fea_div}
	F_{m}=\frac{\sum_{i=1}^{N_{f}}\left ( f_i - \bar{f} \right )^{2}}{N_{f}}
\end{equation}
where $f_i$ represents each handcrafted feature which describe in Section \ref{sec:initial_seg}, and $\bar{f}$ is mean value of a $N_{f}$ dimensional feature vector.
To acquire the first ground truth data, we manually annotate the mesh (with segments) that is selected with the highest feature diversity.
Then, we add the first labelled mesh into the training dataset for the supervised classification.
Specifically, we use the segment-based features as input for the classifier, and the output is a pre-labelled mesh dataset.
Next, we use the mesh annotation tool to manually refine the pre-labelled mesh according to the feature diversity.
Finally, the new refined mesh will be added to the training dataset to improve the automatic classification accuracy incrementally.


\subsubsection{Initial Segmentation}\label{sec:initial_seg}

To avoid redundant computations of numerous triangles, we first apply mesh over-segmentation (i.e., linear least-squares fitting of planes) based on region growing on the input data to group triangle faces into homogeneous regions~\citep{lafarge2012creating}.
Such grouped regions are beneficial for computing local contextual features.
We then extract both geometric and radiometric features from those mesh segments as follows: 
\begin{itemize}
	\item[$\bullet$] \textit{Eigen-based features} are computed from the covariance matrix of the triangle vertices with respect to the average centre within each segment, which is beneficial for identifying urban objects with various surface distributions.
	The linearity $= (\lambda_{1} - \lambda_{2}) / \lambda_{1}$, sphericity $= \lambda_{3}/ \lambda_{1}$ and change of curvature $= \lambda_{3} / (\lambda_{1} + \lambda_{2} + \lambda_{3})$ are computed based on the three eigenvalues $\lambda_{1} \geq \lambda_{2} \geq \lambda_{3}\geq 0$.
	The local eigenvectors $\mathbf{n}_{i} $ and the unit normal vector $\mathbf{n}_{z} $ along Z-axis are used to compute the verticality $=1-\left | \mathbf{n}_{i}\cdot \mathbf{n}_{z} \right | $~\citep{hackel2016fast}.
	Note that many eigen-based features have been studied in literature~\citep{hackel2016fast,west2004context,weinmann2013feature}, and some of them were designed for and tested on LiDAR point clouds. 
	\textcolor{ao}{
	These eigen-based features are mostly computed per point based on its spherical neighbourhood, which often contains noise and does not form a surface. 
	Our chosen eigen-based features are defined on a segment representing the surface of a mesh, and thus they can capture non-local geometric properties of an object.
	}
	Additionally, in this work, we have tested all eigen-based features from the literature~\citep{hackel2016fast}, and we only present the ones that are effective for texture meshes.
	\item[$\bullet$] \textit{Elevation} is divided into absolute elevation $z_{a}$, relative elevation $z_{r}$ and multiscale elevations $z_{m}$.
	Where $z_{a}$ is the average elevation of the segment;
	the relative elevation is computed as $z_{r} = z_{a}-z_{r_{min}}$;
	the multiscale elevation~\citep{Verdie2015,Rouhani2017} $z_{m} = \sqrt{\frac{z_{a} - z_{min}}{z_{max} - z_{min}}}$.
	And $z_{r_{min}}$ denotes the lowest elevation of the local largest ground segment computed within a cylindrical neighbourhood with 30 meters radius around the segment centre.
	$z_{min}$ and $z_{max}$ represent the local minimum and maximum elevation values of a cylindrical neighbourhood within the scale of 10 meters, 20 meters, and 40 meters.
	Such large cylindrical neighbourhoods allow to find the local ground considering the resilience to hilly environments, \textcolor{ao}{and the square root ensures that small relative height values (i.e., values smaller than 1 $ m $) get a larger elevation attribute to enlarge elevation differences between small objects and the local ground (e.g., cars against the ground, boats against the water surfaces).}
	More importantly, due to the influence of terrain fluctuations and various scales of urban objects, the elevation of these three categories can complement each other.
	\item[$\bullet$] \textit{Segment area} is computed as $area(S_k) = \sum_{i = 1}^{N} area(f_i) $, where $f_i$ denotes a triangle of the segment $S_k$, and $N$ denotes the total number of triangles in $S_k$.
	\item[$\bullet$] \textit{Triangle density} is defined as $density(S_k) = \frac{N}{area(S_k)} $,  which reveals the object complexity, especially for adaptive urban meshes.
	\item[$\bullet$] \textit{Interior radius of 3D medial axis transform (InMAT)}~\citep{ma20123d,peters2016robust} of a segment $S_k$ is formulated as $r_k = \frac{\sum_{i=1}^{M} r_i}{M}$, where $M$ denotes the total number of triangle vertices of $S_k$, and $r_i$ denotes the interior radius of the shrinking ball that touches the vertex $v_i$ within the segment $S_k$. 
	It is designed to distinguish objects with different scales. 
	\item[$\bullet$] \textit{HSV colour-based features} are derived from the RGB channel of the entire texture map.
	We use the HSV colour space since it can better differentiate different objects than RGB.
	We compute the average colour, the variance of the colour distribution of all pixels within each segment, and we further discretize it into a histogram that consists of 15 bins of the hue channel, five bins of the saturation channel, and five bins of the value channel.
	\item[$\bullet$] \textit{Greenness} $a_{g}$ is used to classify objects that are similar to green vegetation.
	Specifically, it is computed according to the averaged RGB colour of each segment via $a_{g}=G-0.39\cdot R-0.61\cdot B$~\citep{mckinnon2017comparing}. 
\end{itemize}
	All the above features are concatenated into a 44-dimensional feature vector used by our random forest (RF) classifier in the initial segmentation. 

\subsubsection{Annotation Tool for Refinement}

Because of the under-segmentation errors and the imperfect results of the semantic mesh segmentation process, we design a mesh annotation tool (see Figure \ref{fig:annotator}) to manually correct the labelling errors.
Our mesh annotation tool is developed based on the labelling tool of CGAL~\citep{cgal:eb-20b}.

\begin{figure}[!tb]
	\centering
	\includegraphics[width=\textwidth]{figures/annotator/annotator.png}
	\caption{The interface of our annotation tool for 3D texture meshes. }
	\label{fig:annotator}
\end{figure}

As shown in Table \ref{tab:annotation_operation}, it consists of three operation categories: view, selection, and annotation.
The	view operations provide essential functions for the user to manipulate the scene camera, such as translate, rotate, zoom, or set the new pivot for the scene.
In addition, to use textures as a reference for labelling, we map texture and face colour with a certain degree of transparency, and we visualize the segment border to differentiate each segment. 

\begin{table}[!tb]
	\centering
	\noindent\adjustbox{max width=0.8\textwidth}
	{
		\begin{threeparttable}
			\centering
			\begin{tabular}{ccc}
				\toprule
				Categories & Operations & Objects \\
				\midrule
				\multirow{4}[2]{*}{View} & Translate & Camera \\
				& Rotate & Camera \\
				& Zoom in / out & Camera \\
				& Set pivot & Camera \\
				\midrule
				\multirow{6}[2]{*}{Selection} & Multi-selection / Lasso & Triangles / Segments \\
				& Expand / Reduce & Triangles / Segments \\
				& Semantic selection & Segments \\
				& Split region & Segments \\
				& Planar region extraction & Triangles \\
				& Split mesh & Triangles \\
				\midrule
				\multirow{3}[2]{*}{Annotation} & Probability slider & Segments \\
				& Segment area slider & Segments \\
				& Progress bar & Triangles \\
				& Switch semantic view & Triangles \\ 
				& Labelling & Triangles / Segments \\
				\bottomrule
			\end{tabular}%
		\end{threeparttable}
	}
	\caption{Basic operations in our annotation tool.} 
	\label{tab:annotation_operation}%
\end{table}%


The	selection operations allow the user to select or deselect either triangle faces (see Figure \ref{fig:tri_sel}) or segments (see Figure \ref{fig:seg_sel}) freely via a brush or a lasso.
Specifically, the face selection operation is used to fix the under-segmentation errors and generate new segments, and the segment selection operation is to fix incorrect segment labels.

\begin{figure}[!tb]
	\centering
	\begin{subfigure}[t]{0.32\textwidth}
		\includegraphics[width=\linewidth]{figures/pipeline/tri_select_a.png}
		\caption{}
	\end{subfigure}
	\hspace*{\fill}
	\begin{subfigure}[t]{0.32\textwidth}
		\includegraphics[width=\linewidth]{figures/pipeline/tri_select_b.png}
		\caption{}
	\end{subfigure}
	\hspace*{\fill}
	\begin{subfigure}[t]{0.32\textwidth}
		\includegraphics[width=\linewidth]{figures/pipeline/tri_select_c.png}
		\caption{}
	\end{subfigure}
	\caption{An example of labelling by selecting triangles using the lasso tool (blue edges: segment boundaries). 
		(a) Before selection.
		(b) Lasso selection result (in red).
		(c) The correct label has been assigned to the selected region. 
		In this example, the label of the selected region has been changed from `ground' to `vehicle'.
	} 
	\label{fig:tri_sel}
\end{figure}


\begin{figure}[!tb]
	\centering
	\begin{subfigure}[t]{0.32\textwidth}
		\includegraphics[width=\linewidth]{figures/pipeline/seg_select_a.png}
		\caption{}
	\end{subfigure}
	\hspace*{\fill}
	\begin{subfigure}[t]{0.32\textwidth}
		\includegraphics[width=\linewidth]{figures/pipeline/seg_select_b.png}
		\caption{}
	\end{subfigure}
	\hspace*{\fill}
	\begin{subfigure}[t]{0.32\textwidth}
		\includegraphics[width=\linewidth]{figures/pipeline/seg_select_c.png}
		\caption{}
	\end{subfigure}
	\caption{An example of segment labelling. 
		(a) Part of a wall of the building was previously labelled as `high vegetation' (in green).
		(b) Segment selection result (in red).
		(c) The label of the selected segment has been corrected with the new label `building'.
	}
	\label{fig:seg_sel}
\end{figure}

We also allow the user to edit the selection of each individual segment with splitting functions (see Figure \ref{fig:pnp_func}) and automatic extraction of the most planar region (see Figure \ref{fig:seg_func}). 
As for splitting, we first detect the potential planar and non-planar segments marked by user strokes, and then the non-planar one is split according to the vertex-to-plane distance.
It allows generating candidate non-planar regions (with respect to the detected planar segment) for the user to edit, and
it is useful to split a segment that covers large non-planar regions or contains more than one dominant planar area.
To extract the most planar region, we apply the region growing algorithm~\citep{lafarge2012creating} within the selected segment to automatically generate the candidate triangle faces with user-defined thresholds (i.e., the maximum distance to the plane, the maximum accepted angle, and the minimum region size).
Such an operation allows the user to filter out some small bumpy regions of the selected segment.

\begin{figure}[!tb]
	\centering
	\begin{subfigure}[t]{0.48\textwidth}
		\includegraphics[width=\linewidth]{figures/annotator/pnp_pipeline1.png}
		\caption{}
	\end{subfigure}
	\hspace*{\fill}
	\begin{subfigure}[t]{0.48\textwidth}
		\includegraphics[width=\linewidth]{figures/annotator/pnp_pipeline2.png}
		\caption{}
	\end{subfigure}
	\caption{An example splitting planar and non-planar regions. 
		(a) The user draws a stroke (in red) across the border of the non-planar segment and the planar segment. 
		(b) The detected non-planar segment has been split into two parts (i.e., a non-planar region shown in red and a planar segment shown in green).
	} 
	\label{fig:pnp_func}
\end{figure}

\begin{figure}[!tb]
	\centering
	\begin{subfigure}[t]{0.48\textwidth}
		\includegraphics[width=\linewidth]{figures/annotator/planar_split_pipeline1.png}
		\caption{}
	\end{subfigure}
	\hspace*{\fill}
	\begin{subfigure}[t]{0.48\textwidth}
		\includegraphics[width=\linewidth]{figures/annotator/planar_split_pipeline3.png}
		\caption{}
	\end{subfigure}
	\caption{Editing an individual segment. 
		(a) A segment is selected (highlighted in green) for splitting. 
		(b) Automatic extraction of the most planar region (shown in red) within the selected segment according to user-defined thresholds.} 
	\label{fig:seg_func}
\end{figure}

Besides, probability and area-based sliders and a progress bar are provided in the annotation panel to improve annotation efficiency and experience, respectively. 
Specifically, the probability slider is introduced for the user to visually inspect the segments that are most likely misclassified.
Moreover, the user can further use it to inspect a specific class by switching the view to highlight a specific semantic class.
The segment area slider is used to identify isolated tiny segments, which commonly appear as errors.
The progress bar is used to indicate the estimated labelling progress during the annotation.
After performing the selection, the user can easily assign the corresponding label to the selected area.

\section{Hyper-parameters and Procedure}\label{subsec:models:hyperparameter}

\textbf{Pragmatic Masking.} For pragmatic masking, we use the Adam optimizer with a weight decay of $0.01$~\cite{loshchilov2018decoupled} and a peak learning rate of $5e-5$. The number of the epochs is five. 

\noindent \textbf{Surrogate Fine-Tuning.} For surrogate fine-tuning, we fine-tune RoBERTa on surrogate classification tasks with the same Adam optimizer but use a peak learning rate of $2e-5$. %We fine-tune on \texttt{Hashtag\_pred} task for 50 epochs and \texttt{Hashtag\_token} task for 30 epochs, respectively. 

The pre-training and surrogate fine-tuning models are trained on eight Nvidia V$100$ GPUs ($32$G each). On average the running time is $24$ hours per epoch for PMLMs, $2.5$ hours per epoch for SFT models. All the models are implemented by Huggingface Transformers~\cite{wolf-2020-transformers}.%~\footnote{Our code is available here: \url{https://anonymous.4open.science/r/fd268167-947a-482c-aca7-7b54aa6b5ed9/}.}
%%%%%%%%%%%%%%%%%%%%

\noindent \textbf{Downstream Fine-Tuning.} We evaluate the further pre-trained models with pragmatic masking and surrogate fine-tuned models on the $15$ downstream tasks in Table~\ref{tab:gold_data}. We set maximal sequence length as $60$ for $13$ text classification tasks. For Crisis\textsubscript{Oltea} and Stance\textsubscript{Moham}, we append the topic term behind the post content, separate them by [SEP] token, and set maximal sequence length to $70$, especially. For all the tasks, we pass the hidden state of [CLS] token from the last Transformer-encoder layer through a non-linear layer to predict. Cross-Entropy calculates the training loss. We then use Adam with a weight decay of $0.01$ to optimize the model and fine-tune each task for $20$ epochs with early stop ($patience = 5$ epochs). We fine-tune the peak learning rate in a set of $\{1e-5, 5e-6\}$ and batch size in a set of $\{8, 32, 64\}$. We find the learning rate of $5e-6$ performs best across all the tasks. For the downstream tasks whose Train set is smaller than $15,000$ samples, the best mini-batch size is eight. The best batch size of other larger downstream tasks is $64$. For fine-tuning BERTweet, we use the hyperparameters identified in~\citet{nguyen-etal-2020-bertweet}, i.e., a fixed learning rate of $1e-5$ and a batch size of $32$. 

%\noindent\textbf{Multi-task fine-tuning.} For Multi-task fine-tuning, we use a batch size of $64$ for all the tasks and train with $30$ epochs without early stopping. We implement the multi-task learning using hard-sharing~\cite{caruana1997multitask} model. We send the final state of the [CLS] token to the non-linear layer of the corresponding task.

We use the same hyperparameters to run three times with random seeds for all downstream fine-tuning (unless otherwise indicated). All downstream task models are fine-tuned on four Nvidia V$100$ GPUs ($32$G each). At the end of each epoch, we evaluate the model on the Dev set and identify the model that achieved the highest performance on Dev as our best model. We then test the best model on the Test set. In order to compute the model's overall performance across $15$ tasks, we use same evaluation metric (i.e., macro $F_1$) for all tasks. We report the average Test macro $F_1$ of the best model over three runs. We also average the macro $F_1$ scores across $15$ tasks to present the model's overall performance. 
% \section{Summary of model performance}
% Table~\ref{tab:full_res} summarizes performance of our models across different settings. 
% % Please add the following required packages to your document preamble:
% \usepackage{booktabs}
% \usepackage{multirow}
\begin{sidewaystable}
\centering
% \begin{table}
\tiny
\centering
\begin{tabular}{@{}lccccccccccccccccccc@{}}
\toprule
\multicolumn{1}{c}{\multirow{2}{*}{\textbf{Task}}} & \multirow{2}{*}{\textbf{RoBERTa}} & \multirow{2}{*}{\textbf{RM-NR}} & \multirow{2}{*}{\textbf{BERTweet}} & \multicolumn{10}{c}{\textbf{PMLM}}                                                                                                                                     & \multicolumn{4}{c}{\textbf{SFT}}                                        & \multicolumn{2}{c}{\textbf{MT}}      \\ \cmidrule(lr){5-14} \cmidrule(lr){15-18} \cmidrule(lr){19-20} 
\multicolumn{1}{c}{}           &                    &                                   &                                    & \textbf{RM-N} & \textbf{PM-N}  & \textbf{RM-HA} & \textbf{PM-HA} & \textbf{RM-HE} & \textbf{PM-HE} & \textbf{RM-EA} & \textbf{PM-EA} & \textbf{RM-EE} & \textbf{PM-EE} & \textbf{SFT-E} & \textbf{X1+SFT-E} & \textbf{SFT-H} & \textbf{X2+SFT-H} & \textbf{RoBERTa} & \textbf{X2+SFT-H} \\ \midrule
Crisis\textsubscript{Oltea-14}    & 95.95   &       95.78                    & 95.88                              & 95.78         & 95.92          & 95.75          & 95.85          & 95.85          & 95.87          & 95.91          & 95.98          & 95.95          & 95.77          & 95.76          & \textbf{96.02}    & 95.87          & 95.68             & 95.88            & 95.70             \\
Emo\textsubscript{Moham-18}       & 77.99        &  79.15                     & 80.14                              & 79.43         & 80.73          & 80.31          & 78.56          & 79.51          & 80.15          & 80.03          & 81.09          & 81.28          & \textbf{82.18} & 79.69          & 82.04             & 78.69          & 80.50             & 75.79            & 75.85             \\
Hate\textsubscript{Waseem-16}     & 57.34          &  57.22                   & 57.47                              & 56.75         & 56.34          & 57.16          & 57.51          & 56.97          & 57.13          & 57.00          & 57.01          & 57.08          & 56.69          & 56.47          & 60.92             & \textbf{63.97} & 60.25             & 56.52            & 55.79             \\
Hate\textsubscript{David-17}      & 77.71        &     77.54                & 77.15                              & 77.47         & \textbf{78.28} & 76.87          & 77.46          & 77.55          & 77.22          & 78.13          & 78.26          & 78.16          & 77.93          & 76.45          & 77.00             & 77.29          & 76.93             & 77.07            & 77.13             \\
Humor\textsubscript{Potash-17}    & 54.40          &     54.80                   & 52.77                              & 55.45         & 55.26          & 55.32          & 52.49          & 50.06          & 54.60          & \textbf{57.14} & 55.10          & 55.25          & 55.57          & 54.75          & 54.93             & 55.51          & 53.83             & 53.33            & 50.07             \\
Humor\textsubscript{Meaney-21}    & 92.37     &      93.50                  & 94.46                              & 93.24         & 93.69          & 93.58          & 93.48          & 92.85          & \textbf{94.52} & 93.55          & 94.50          & 93.19          & 92.69          & 93.82          & 93.68             & 93.74          & 94.49             & 91.46            & 92.51             \\
Irony\textsubscript{Hee-18A}      & 73.93    &       74.46                   & 77.35                              & 74.52         & 74.97          & 74.50          & 75.16          & 73.97          & 76.24          & 75.34          & 77.93          & 74.40          & 75.62          & 76.63          & 72.73             & 76.22          & \textbf{79.89}    & 77.57            & 78.64             \\
Irony\textsubscript{Hee-18B}      & 52.30       &      50.70                    & 58.67                              & 52.91         & 53.79          & 51.43          & 49.29          & 50.41          & 54.76          & 54.94          & 56.09          & 54.73          & 52.47          & 57.59          & 56.11             & 60.14          & \textbf{61.67}    & 54.37            & 55.20             \\
Offense\textsubscript{-Zamp-19}   & 80.13     &     80.38                   & 78.49                              & 79.97         & 80.24          & 79.74          & 79.34          & 79.95          & 78.87          & 80.18          & 81.14          & 80.18          & 80.65          & 80.18          & \textbf{81.34}    & 79.82          & 79.50             & 79.65            & 79.26             \\
Sarc\textsubscript{Riloff-13}     & 73.85        &      73.90                  & 78.81                              & 72.02         & 75.24          & 71.42          & 74.72          & 74.16          & 75.88          & 76.52          & 77.93          & 76.30          & 80.10          & 78.34          & 78.74             & \textbf{80.50} & 80.49             & 77.41            & 78.09             \\
Sarc\textsubscript{Ptacek-14}     & 95.09          &      96.15             & \textbf{96.35}                     & 95.81         & 95.64          & 95.50          & 95.62          & 95.24          & 95.81          & 95.81          & 96.06          & 95.67          & 96.01          & 95.88          & 96.16             & 96.01          & 96.24             & 95.72            & 96.26             \\
Sarc\textsubscript{Rajad-15}      & 85.07    &    85.63                     & 87.58                              & 86.18         & 86.23          & 85.04          & 85.55          & 85.20          & 85.93          & 86.14          & 86.65          & 86.02          & 86.94          & 86.80          & 87.48             & 87.56          & \textbf{88.92}    & 85.99            & 88.10             \\
Sarc\textsubscript{Bam-15}        & 79.08   &    79.27                      & 82.08                              & 80.03         & 80.13          & 80.22          & 80.16          & 79.83          & 80.31          & 80.73          & 81.12          & 81.13          & 81.73          & 81.48          & 82.53             & 81.19          & 81.53             & 80.70            & \textbf{82.64}    \\
Senti\textsubscript{Rosen-17}     & 71.08           &      71.55            & 71.83                              & 72.03         & \textbf{72.65} & 72.10          & 71.99          & 71.84          & 71.82          & 72.24          & 71.98          & 72.27          & 71.56          & 71.27          & 72.07             & 71.83          & 71.08             & 70.66            & 69.48             \\
Stance\textsubscript{Moham-16}    & 70.41          &      67.00             & 67.41                              & 67.14         & 69.94          & 69.51          & 68.13          & 69.23          & 69.68          & 70.20          & 68.62          & 70.04          & 68.48          & 69.06          & 69.65             & \textbf{71.27} & 70.77             & 69.11            & 69.22             \\ \cdashline{1-20}
Average                                            & 75.78    &75.80                         & 77.10                              & 75.92         & 76.60          & 75.90          & 75.69          & 75.51          & 76.59          & 76.92          & 77.30          & 76.78          & 76.96          & 76.94          & 77.43             & 77.97          & \textbf{78.12}    & 76.08            & 76.26             \\ \bottomrule 


\end{tabular}
\caption{Summary performance of our models across different \textbf{(i)} PM and SFT settings, \textbf{(ii)} single and multi-task settings, and \textbf{(iii)} in comparison with the BERTweet~\cite{nguyen-etal-2020-bertweet}, a SOTA Twitter pre-trained language model. The metric is macro $F_1$.}\label{tab:full_res}
% \end{table}

\end{sidewaystable}
% \vspace{-5pt}
% \section{Paraphrasing Model and SMPB}


\section{Few-Shot Experiment}\label{sec:append_fewshot}
 Tables~\ref{tab:few_shot-roberta}, \ref{tab:few_shot-bertweet}, \ref{tab:few_shot-x1sfte}, and \ref{tab:few_shot-x2sfth} respectively, present the performance of RoBERTa, BERTweet, PragS1, and PragS2 on all our $15$ English downstream datasets and various few-shot settings.

\begin{table*}[ht]
\small
\centering
\begin{tabular}{@{}lccccccccccc@{}}
\toprule
\multicolumn{1}{c}{\textbf{Task}}                & \textbf{1} & \textbf{5} & \textbf{10} & \textbf{20} & \textbf{30} & \textbf{40} & \textbf{50} & \textbf{60} & \textbf{70} & \textbf{80} & \textbf{90} \\ \midrule
Crisis\textsubscript{Oltea-14}  & 94.67      & 95.36      & 95.55       & 95.74       & 95.90       & 95.81       & 95.89       & 95.84       & 95.99       & 96.03       & 96.11       \\
Emo\textsubscript{Moham-18}     & 14.10      & 30.36      & 71.76       & 73.62       & 76.26       & 77.02       & 77.59       & 77.19       & 77.38       & 77.84       & 78.86       \\
Hate\textsubscript{Waseem-16}   & 28.23      & 52.66      & 54.66       & 54.82       & 56.26       & 56.42       & 56.70       & 57.10       & 56.92       & 56.99       & 57.25       \\
Hate\textsubscript{David-17}    & 42.01      & 70.92      & 74.76       & 75.71       & 75.08       & 75.70       & 76.05       & 75.21       & 76.38       & 76.58       & 77.63       \\
Humor\textsubscript{Potash-17}  & 47.91      & 47.91      & 52.89       & 52.67       & 54.43       & 52.30       & 53.89       & 55.00       & 53.69       & 54.16       & 56.78       \\
Humor\textsubscript{Meaney-21}  & 53.44      & 89.50      & 89.47       & 90.12       & 91.95       & 91.65       & 92.33       & 91.96       & 92.65       & 91.78       & 92.27       \\
Irony\textsubscript{Hee-18A}    & 40.75      & 60.47      & 61.97       & 70.49       & 67.64       & 70.40       & 72.04       & 71.33       & 72.01       & 72.67       & 72.54       \\
Irony\textsubscript{Hee-18B}    & 19.41      & 26.27      & 43.61       & 46.47       & 44.78       & 48.41       & 50.40       & 51.65       & 51.80       & 53.15       & 53.17       \\
Offense\textsubscript{-Zamp-19} & 41.89      & 76.87      & 74.44       & 76.53       & 79.75       & 79.29       & 78.95       & 78.13       & 79.01       & 79.42       & 79.90       \\
Sarc\textsubscript{Riloff-13}   & 44.41      & 44.80      & 43.99       & 70.49       & 51.10       & 70.70       & 67.72       & 72.46       & 67.98       & 72.88       & 73.75       \\
Sarc\textsubscript{Ptacek-14}   & 81.57      & 85.92      & 87.18       & 88.78       & 89.84       & 91.33       & 91.76       & 92.38       & 93.58       & 94.29       & 94.98       \\
Sarc\textsubscript{Rajad-15}    & 68.52      & 77.80      & 78.47       & 81.59       & 82.60       & 82.58       & 83.61       & 83.77       & 84.44       & 84.76       & 84.43       \\
Sarc\textsubscript{Bam-15}      & 64.17      & 74.01      & 75.95       & 76.18       & 77.00       & 78.07       & 78.43       & 78.68       & 79.35       & 79.08       & 79.40       \\
Senti\textsubscript{Rosen-17}   & 64.84      & 68.00      & 69.95       & 70.10       & 70.51       & 70.04       & 71.70       & 70.07       & 70.12       & 70.30       & 71.17       \\
Stance\textsubscript{Moham-16}  & 25.20      & 44.73      & 62.03       & 62.67       & 65.11       & 65.44       & 64.97       & 65.74       & 68.59       & 68.54       & 69.21       \\ \cdashline{1-12}
Average                                          & 48.74      & 63.04      & 69.11       & 72.40       & 71.88       & 73.68       & 74.14       & 74.43       & 74.66       & 75.23       & 75.83       \\ \bottomrule
\end{tabular}
\caption{Full result of few-shot learning on Baseline (1), fine-tuning RoBERTa.}\label{tab:few_shot-roberta}
\end{table*}


\begin{table*}[ht]
\small
\centering
\begin{tabular}{@{}lccccccccccc@{}}
\toprule
\textbf{Task}                                    & \textbf{1} & \textbf{5} & \textbf{10} & \textbf{20} & \textbf{30} & \textbf{40} & \textbf{50} & \textbf{60} & \textbf{70} & \textbf{80} & \textbf{90} \\ \midrule
Crisis\textsubscript{Oltea-14}  & 94.71      & 94.95      & 95.38       & 95.32       & 95.60       & 95.53       & 95.78       & 95.72       & 95.65       & 95.71       & 95.68       \\
Emo\textsubscript{Moham-18}     & 21.68      & 17.29      & 66.13       & 75.03       & 76.50       & 77.72       & 76.20       & 79.16       & 79.22       & 79.37       & 80.58       \\
Hate\textsubscript{Waseem-16}   & 30.92      & 52.27      & 53.70       & 55.05       & 55.18       & 55.80       & 56.48       & 56.44       & 56.46       & 57.10       & 56.66       \\
Hate\textsubscript{David-17}    & 29.21      & 69.18      & 74.17       & 76.58       & 77.95       & 76.97       & 77.19       & 77.43       & 77.29       & 77.72       & 78.30       \\
Humor\textsubscript{Potash-17}  & 47.90      & 47.91      & 48.24       & 51.68       & 51.25       & 53.37       & 54.80       & 54.39       & 54.91       & 52.31       & 55.83       \\
Humor\textsubscript{Meaney-21}  & 52.07      & 90.67      & 92.43       & 92.68       & 93.50       & 93.32       & 92.88       & 93.52       & 94.31       & 94.18       & 94.55       \\
Irony\textsubscript{Hee-18A}    & 44.88      & 57.78      & 67.90       & 71.87       & 74.40       & 75.42       & 75.15       & 75.94       & 75.42       & 76.80       & 76.82       \\
Irony\textsubscript{Hee-18B}    & 17.16      & 20.69      & 27.30       & 39.72       & 46.40       & 49.26       & 50.29       & 51.41       & 54.08       & 54.08       & 55.49       \\
Offense\textsubscript{-Zamp-19} & 45.03      & 74.68      & 76.49       & 78.02       & 79.26       & 78.55       & 78.86       & 79.59       & 80.54       & 79.74       & 78.30       \\
Sarc\textsubscript{Riloff-13}   & 44.38      & 43.99      & 44.88       & 43.99       & 77.89       & 78.23       & 77.73       & 79.73       & 78.20       & 79.98       & 78.82       \\
Sarc\textsubscript{Ptacek-14}   & 85.36      & 88.06      & 89.18       & 90.58       & 91.44       & 92.60       & 93.44       & 93.64       & 94.40       & 95.30       & 95.77       \\
Sarc\textsubscript{Rajad-15}    & 47.01      & 81.87      & 83.24       & 84.22       & 85.31       & 85.38       & 85.73       & 85.86       & 86.11       & 86.77       & 86.76       \\
Sarc\textsubscript{Bam-15}      & 56.24      & 76.75      & 78.61       & 80.01       & 80.06       & 81.05       & 81.05       & 81.64       & 81.86       & 82.72       & 82.84       \\
Senti\textsubscript{Rosen-17}   & 65.42      & 67.96      & 69.85       & 70.38       & 71.24       & 71.49       & 71.76       & 71.29       & 71.49       & 72.29       & 71.63       \\
Stance\textsubscript{Moham-16}  & 25.69      & 25.36      & 24.27       & 59.25       & 61.58       & 63.45       & 62.31       & 65.08       & 66.64       & 66.54       & 67.63       \\ \cdashline{1-12}
Average                                          & 47.18      & 60.63      & 66.12       & 70.96       & 74.50       & 75.21       & 75.31       & 76.06       & 76.44       & 76.71       & 77.04       \\ \bottomrule
\end{tabular}
\caption{Full result of few-shot learning on BERTweet.}\label{tab:few_shot-bertweet}
\end{table*}



% \begin{table*}[]
% \begin{tabular}{@{}lccccccccccc@{}}
% \toprule
% \multicolumn{1}{c}{\textbf{Task}}                & \textbf{1}                & \textbf{5}                & \textbf{10}               & \textbf{20}               & \textbf{30}               & \textbf{40}               & \textbf{50}               & \textbf{60}               & \textbf{70}               & \textbf{80}               & \textbf{90}               \\ \midrule
% Crisis\textsubscript{Oltea-14}  & 93.88                     & 95.36                     & 95.40                     & 95.38                     & 95.75                     & 95.75                     & 95.70                     & 95.99                     & 95.83                     & 95.94                     & 96.00                     \\
% Emo\textsubscript{Moham-18}     & 14.16                     & 14.10                     & 69.35                     & 72.25                     & 74.55                     & 75.70                     & 77.17                     & 78.75                     & 77.51                     & 77.88                     & 78.73                     \\
% Hate\textsubscript{Waseem-16}   & 28.22                     & 49.99                     & 52.81                     & 54.73                     & 55.75                     & 56.02                     & 56.61                     & 56.89                     & 57.13                     & 57.45                     & 57.44                     \\
% Hate\textsubscript{David-17}    & 29.06                     & 69.43                     & 73.41                     & 75.96                     & 76.80                     & 76.63                     & 77.12                     & 75.97                     & 77.97                     & 76.79                     & 78.08                     \\
% Humor\textsubscript{Potash-17}  & 47.91                     & 47.91                     & 53.58                     & 49.19                     & 51.93                     & 54.92                     & 53.37                     & 52.28                     & 54.11                     & 55.32                     & 55.64                     \\
% Humor\textsubscript{Meaney-21}  & 66.56                     & 92.38                     & 92.05                     & 92.31                     & 91.91                     & 92.93                     & 93.05                     & 93.32                     & 93.58                     & 93.39                     & 93.43                     \\
% Irony\textsubscript{Hee-18A}    & 43.36                     & 44.04                     & 64.35                     & 68.79                     & 69.82                     & 71.54                     & 71.90                     & 74.03                     & 74.45                     & 73.35                     & 75.18                     \\
% Irony\textsubscript{Hee-18B}    & 21.41                     & 25.00                     & 37.08                     & 42.07                     & 43.65                     & 45.46                     & 47.26                     & 47.72                     & 49.33                     & 52.13                     & 52.17                     \\
% Offense\textsubscript{-Zamp-19} & 45.24                     & 72.46                     & 74.08                     & 76.92                     & 76.78                     & 78.42                     & 77.62                     & 78.34                     & 78.31                     & 80.07                     & 79.95                     \\
% Sarc\textsubscript{Riloff-13}   & 45.40                     & 43.19                     & 43.99                     & 51.89                     & 71.00                     & 71.84                     & 70.74                     & 71.03                     & 73.52                     & 75.48                     & 72.36                     \\
% Sarc\textsubscript{Ptacek-14}   & 82.61                     & 86.80                     & 88.11                     & 89.98                     & 90.93                     & 92.20                     & 92.85                     & 93.31                     & 94.32                     & 95.06                     & 95.34                     \\
% Sarc\textsubscript{Rajad-15}    & 54.99                     & 79.58                     & 80.19                     & 82.44                     & 83.53                     & 84.63                     & 84.46                     & 85.25                     & 85.63                     & 86.06                     & 85.80                     \\
% Sarc\textsubscript{Bam-15}      & 63.01                     & 74.47                     & 76.69                     & 77.02                     & 78.04                     & 79.08                     & 78.81                     & 79.86                     & 79.16                     & 79.79                     & \multicolumn{1}{l}{79.45} \\
% Senti\textsubscript{Rosen-17}   & 62.87                     & 67.68                     & 70.00                     & 69.97                     & 70.47                     & 71.14                     & 71.62                     & 71.24                     & 71.56                     & 71.02                     & \multicolumn{1}{l}{71.73} \\
% Stance\textsubscript{Moham-16}  & 29.13                     & 30.78                     & 47.80                     & 57.61                     & 63.45                     & 62.04                     & 62.60                     & 65.57                     & 66.18                     & 66.60                     & \multicolumn{1}{l}{67.98} \\ \cdashline{1-12}
% Average                                          & \multicolumn{1}{l}{48.52} & \multicolumn{1}{l}{59.54} & \multicolumn{1}{l}{67.93} & \multicolumn{1}{l}{70.44} & \multicolumn{1}{l}{72.96} & \multicolumn{1}{l}{73.89} & \multicolumn{1}{l}{74.06} & \multicolumn{1}{l}{74.64} & \multicolumn{1}{l}{75.24} & \multicolumn{1}{l}{75.76} & \multicolumn{1}{l}{75.95} \\ \bottomrule
% \end{tabular}
% \caption{Full result of few-shot learning on RM-NR.}\label{tab:few_shot-rmnr}
% \end{table*}

% \begin{table*}[]
% \begin{tabular}{@{}lccccccccccc@{}}
% \toprule
% \multicolumn{1}{c}{\textbf{Task}}                & \textbf{1} & \textbf{5} & \textbf{10} & \textbf{20} & \textbf{30} & \textbf{40} & \textbf{50} & \textbf{60} & \textbf{70} & \textbf{80} & \textbf{90} \\ \midrule
% Crisis\textsubscript{Oltea-14}  & 94.32      & 95.27      & 95.61       & 95.60       & 95.83       & 95.92       & 95.82       & 95.86       & 95.86       & 96.03       & 95.99       \\
% Emo\textsubscript{Moham-18}     & 14.13      & 14.10      & 72.82       & 74.77       & 76.15       & 78.23       & 77.79       & 80.11       & 79.39       & 79.13       & 79.57       \\
% Hate\textsubscript{Waseem-16}   & 28.23      & 50.77      & 53.09       & 54.44       & 55.82       & 56.23       & 56.86       & 57.02       & 56.35       & 56.42       & 56.77       \\
% Hate\textsubscript{David-17}    & 37.58      & 68.97      & 74.36       & 75.32       & 75.72       & 77.37       & 77.33       & 77.07       & 76.88       & 77.67       & 77.67       \\
% Humor\textsubscript{Potash-17}  & 47.90      & 47.91      & 50.07       & 52.27       & 52.72       & 54.38       & 56.33       & 53.73       & 53.11       & 55.77       & 54.55       \\
% Humor\textsubscript{Meaney-21}  & 65.63      & 92.16      & 91.99       & 92.38       & 92.54       & 93.47       & 93.45       & 93.46       & 93.93       & 93.41       & 93.63       \\
% Irony\textsubscript{Hee-18A}    & 42.90      & 41.97      & 68.79       & 72.28       & 72.41       & 72.42       & 72.60       & 74.40       & 75.14       & 74.03       & 75.11       \\
% Irony\textsubscript{Hee-18B}    & 18.80      & 34.51      & 39.29       & 44.93       & 45.48       & 47.92       & 49.19       & 50.55       & 50.75       & 50.77       & 53.73       \\
% Offense\textsubscript{-Zamp-19} & 48.50      & 74.39      & 75.84       & 74.96       & 78.10       & 78.45       & 80.04       & 79.06       & 80.40       & 80.29       & 79.83       \\
% Sarc\textsubscript{Riloff-13}   & 44.23      & 43.93      & 43.99       & 43.99       & 73.36       & 71.31       & 73.48       & 72.24       & 72.86       & 73.77       & 69.44       \\
% Sarc\textsubscript{Ptacek-14}   & 83.76      & 87.15      & 88.25       & 90.32       & 91.17       & 92.43       & 92.85       & 93.38       & 94.18       & 94.84       & 95.62       \\
% Sarc\textsubscript{Rajad-15}    & 55.39      & 79.67      & 81.08       & 83.30       & 84.28       & 84.76       & 85.18       & 85.87       & 85.51       & 85.93       & 85.68       \\
% Sarc\textsubscript{Bam-15}      & 66.66      & 74.60      & 77.65       & 76.98       & 78.35       & 78.43       & 79.04       & 79.23       & 78.74       & 79.09       & 79.90       \\
% Senti\textsubscript{Rosen-17}   & 62.10      & 68.09      & 69.39       & 70.68       & 71.26       & 71.13       & 72.13       & 72.44       & 71.46       & 72.03       & 72.52       \\
% Stance\textsubscript{Moham-16}  & 24.15      & 39.74      & 51.57       & 58.56       & 65.30       & 61.68       & 64.54       & 67.06       & 66.63       & 67.97       & 67.14       \\ \cdashline{1-12}
% Average                                          & 48.95      & 60.88      & 68.92       & 70.72       & 73.90       & 74.27       & 75.11       & 75.43       & 75.41       & 75.81       & 75.81       \\ \bottomrule
% \end{tabular}
% \caption{Full result of few-shot learning on RM-N.}\label{tab:few_shot-rmn}
% \end{table*}


\begin{table*}[ht]
\small
\centering
\begin{tabular}{@{}lccccccccccc@{}}
\toprule
\multicolumn{1}{c}{\textbf{Task}}                & \textbf{1} & \textbf{5} & \textbf{10} & \textbf{20} & \textbf{30} & \textbf{40} & \textbf{50} & \textbf{60} & \textbf{70} & \textbf{80} & \textbf{90} \\ \midrule
Crisis\textsubscript{Oltea-14}  & 94.35      & 95.34      & 95.37       & 95.74       & 95.85       & 95.83       & 95.92       & 95.92       & 95.91       & 95.98       & 95.86       \\
Emo\textsubscript{Moham-18}     & 36.95      & 64.31      & 74.68       & 77.94       & 79.79       & 80.19       & 80.23       & 80.19       & 80.30       & 80.78       & 81.27       \\
Hate\textsubscript{Waseem-16}   & 38.81      & 51.76      & 53.54       & 54.32       & 55.70       & 56.00       & 56.49       & 56.43       & 57.06       & 59.56       & 59.76       \\
Hate\textsubscript{David-17}    & 57.07      & 68.95      & 72.66       & 75.03       & 75.14       & 75.11       & 75.86       & 77.53       & 77.09       & 76.11       & 76.88       \\
Humor\textsubscript{Potash-17}  & 47.91      & 50.24      & 51.87       & 51.21       & 51.92       & 54.91       & 53.26       & 52.22       & 52.37       & 54.36       & 54.39       \\
Humor\textsubscript{Meaney-21}  & 87.10      & 91.79      & 92.16       & 92.42       & 92.80       & 93.01       & 93.05       & 93.53       & 93.64       & 93.86       & 93.70       \\
Irony\textsubscript{Hee-18A}    & 60.35      & 66.13      & 70.77       & 72.26       & 74.24       & 73.82       & 74.95       & 74.92       & 75.97       & 75.87       & 77.37       \\
Irony\textsubscript{Hee-18B}    & 29.82      & 36.42      & 41.72       & 46.50       & 50.14       & 53.57       & 52.63       & 55.80       & 54.23       & 55.92       & 56.62       \\
Offense\textsubscript{-Zamp-19} & 61.17      & 74.22      & 77.05       & 77.63       & 79.22       & 80.62       & 79.09       & 80.77       & 81.27       & 79.85       & 79.68       \\
Sarc\textsubscript{Riloff-13}   & 52.83      & 63.39      & 73.40       & 74.34       & 77.10       & 78.01       & 77.87       & 77.53       & 77.32       & 77.32       & 78.72       \\
Sarc\textsubscript{Ptacek-14}   & 85.64      & 87.81      & 88.87       & 89.90       & 91.17       & 92.18       & 92.82       & 93.64       & 94.00       & 95.08       & 95.68       \\
Sarc\textsubscript{Rajad-15}    & 82.80      & 84.95      & 85.84       & 85.79       & 86.62       & 86.39       & 86.84       & 86.96       & 86.81       & 87.14       & 87.02       \\
Sarc\textsubscript{Bam-15}      & 72.44      & 77.74      & 78.97       & 80.27       & 81.08       & 81.74       & 81.56       & 81.62       & 81.98       & 81.53       & 82.29       \\
Senti\textsubscript{Rosen-17}   & 59.48      & 65.39      & 69.06       & 69.29       & 70.18       & 70.32       & 71.51       & 71.42       & 71.28       & 71.87       & 72.13       \\
Stance\textsubscript{Moham-16}  & 31.80      & 49.63      & 56.29       & 60.94       & 64.59       & 64.58       & 65.44       & 67.27       & 68.23       & 67.95       & 68.13       \\ \cdashline{1-12}
Average                                          & 59.90      & 68.54      & 72.15       & 73.57       & 75.04       & 75.75       & 75.83       & 76.38       & 76.50       & 76.88       & 77.30       \\ \bottomrule
\end{tabular}
\caption{Full result of few-shot learning on PragS1.}\label{tab:few_shot-x1sfte}
\end{table*}


\begin{table*}[ht]
\small
\centering
\begin{tabular}{@{}lccccccccccc@{}}
\toprule
\multicolumn{1}{c}{\textbf{Task}}                & \textbf{1} & \textbf{5} & \textbf{10} & \textbf{20} & \textbf{30} & \textbf{40} & \textbf{50} & \textbf{60} & \textbf{70} & \textbf{80} & \textbf{90} \\ \midrule
Crisis\textsubscript{Oltea-14}  & 93.92      & 95.07      & 95.50       & 95.30       & 95.60       & 95.50       & 95.73       & 95.66       & 95.52       & 95.70       & 95.96       \\
Emo\textsubscript{Moham-18}     & 35.90      & 58.23      & 71.27       & 75.36       & 77.71       & 78.80       & 79.25       & 78.99       & 79.74       & 80.06       & 81.28       \\
Hate\textsubscript{Waseem-16}   & 43.42      & 53.24      & 59.36       & 54.85       & 55.51       & 56.32       & 56.57       & 56.52       & 56.91       & 61.08       & 63.86       \\
Hate\textsubscript{David-17}    & 57.30      & 71.09      & 73.10       & 75.37       & 77.25       & 74.36       & 75.91       & 77.72       & 75.76       & 77.30       & 76.59       \\
Humor\textsubscript{Potash-17}  & 49.75      & 51.72      & 51.59       & 52.39       & 54.80       & 53.39       & 52.82       & 52.31       & 53.41       & 53.82       & 54.26       \\
Humor\textsubscript{Meaney-21}  & 84.95      & 92.09      & 92.73       & 93.16       & 94.17       & 94.07       & 93.54       & 93.57       & 93.81       & 93.52       & 93.89       \\
Irony\textsubscript{Hee-18A}    & 57.95      & 68.51      & 71.96       & 73.41       & 75.17       & 75.66       & 75.60       & 77.34       & 76.72       & 77.49       & 77.79       \\
Irony\textsubscript{Hee-18B}    & 29.69      & 35.93      & 41.51       & 48.44       & 52.77       & 52.71       & 55.87       & 56.07       & 58.13       & 55.63       & 55.43       \\
Offense\textsubscript{-Zamp-19} & 52.61      & 70.40      & 74.09       & 76.45       & 78.80       & 78.02       & 76.90       & 79.53       & 79.35       & 79.73       & 79.42       \\
Sarc\textsubscript{Riloff-13}   & 49.57      & 64.07      & 75.80       & 75.46       & 78.28       & 78.93       & 78.89       & 78.31       & 79.71       & 78.86       & 79.04       \\
Sarc\textsubscript{Ptacek-14}   & 86.19      & 88.52      & 89.53       & 90.75       & 91.55       & 92.21       & 93.03       & 93.73       & 94.28       & 95.04       & 95.71       \\
Sarc\textsubscript{Rajad-15}    & 84.69      & 85.43      & 85.61       & 86.48       & 87.13       & 86.86       & 87.08       & 87.05       & 87.36       & 87.29       & 87.48       \\
Sarc\textsubscript{Bam-15}      & 73.40      & 77.28      & 77.88       & 79.84       & 79.40       & 80.29       & 80.31       & 80.32       & 80.60       & 80.95       & 80.39       \\
Senti\textsubscript{Rosen-17}   & 55.75      & 62.50      & 66.50       & 68.90       & 70.09       & 70.64       & 70.89       & 71.32       & 71.34       & 71.51       & 71.64       \\
Stance\textsubscript{Moham-16}  & 34.36      & 47.62      & 56.00       & 61.47       & 63.45       & 66.13       & 65.47       & 67.09       & 68.60       & 68.09       & 69.06       \\ \cdashline{1-12}
Average                                          & 59.30      & 68.11      & 72.16       & 73.84       & 75.44       & 75.59       & 75.86       & 76.37       & 76.75       & 77.07       & 77.45       \\ \bottomrule
\end{tabular}
\caption{Full result of few-shot learning on PragS2.}\label{tab:few_shot-x2sfth}
\end{table*}
\section{Uniformity and Tolerance}\label{sec:uni-tole}
\citet{wang-2021-understanding} investigate representation quality measuring the uniformity of an embedding distribution and the tolerance to semantically similar samples. Given a dataset $D$ and an encoder $\Phi$, the uniformity metric is based on a gaussian potential kernel and is formulated as:
\begin{equation}
    Uniformity = log \mathop{\mathbb{E}}_{x_i, x_j\in D}[ e^{ t ||\Phi(x_i) - \Phi(x_j)||^2_2 } ], 
\end{equation}
where $t=2$.

The tolerance metric measures the mean of similarities of samples belonging to the same class, which defined as:
\begin{equation}
\small
    Tolerance = \mathop{\mathbb{E}}_{x_i, x_j\in D}[(\Phi(x_i)^T\Phi(x_j)) \cdot I_{l(x_i)=l(x_j)}],
\end{equation}

where $l(x_i)$ is the supervised label of sample $x_i$. $I_{l(x_i)=l(x_j)}$ is an indicator function, giving the value of $1$ for $l(x_i)=l(x_j)$ and the value of $0$ for $l(x_i)\neq l(x_j)$. In our experiments, we use gold development samples from $13$ our social meaning datasets. 

% %%%%%%%%%%%%%%%%%%%%%%%%%%%%%%%%%%%%%%%%%
\subsection{Praphrase Model}\label{sec:para} 
%%%%%%%%%%%%%%%%%%%%%%%%%%%%%%%%%%%%%%%%%
In order to paraphrase our datasets, we fine-tune the T5\textsubscript{BASE} model~\cite{raffel2020exploring} on $4$ paraphrase datasets from PIT-2015 (tweet)~\cite{xu2015semeval},  {LanguageNet} (tweet)~\cite{lan2017continuously}, {Opusparcus} ~\cite{creutz2018open}  (video subtitle), and Quora Question Pairs (Q\&A website) \footnote{\url{https://www.quora.com/q/quoradata/First-Quora-Dataset-Release-Question-Pairs}}. For PIT-2015, LanguageNet and Opusparcus, we only keep sentence pairs with a semantic similarity score $\geq 70\%$. We then merge all datasets and shuffle them. Next, we split the data into Train, Dev, and Test ($80\%$, $10\%$, and $10\%$) and fine-tune T5 on the Train split for $20$ epochs with constant learning rate of $3e-4$. 
% \vspace{-5pt}
% \subsection{SMPB}
We fine-tune the paraphrase model on the Train split of each of our $15$ datasets, using top-$p$~\cite{holtzman2019curious} with $p=0.95$ to generate $10$ paraphrases for each gold sample. To remove any `paraphrases' that are just copies of the original tweet or near-duplicates, we use a simple tri-gram similarity method. That is, we remove any sequences whose similarity with the original tweet is $> 0.95$. We also remove paraphrases of the same tweet that are $> 50\%$ similar to one another. To ensure that paraphrases are not very different from the original tweets, we also remove any sequences whose similarity with original tweets $=0$. This process results in a paraphrasing dataset \texttt{Para-Clean}. Table~\ref{tab:append_para} shows the distribution of the resulting paraphrase datasets.  %Table~\ref{tab:para_exam_app} provides more paraphrase examples. 

\begin{table}[h]
\centering
\tiny
\resizebox{0.85\columnwidth}{!}{%
\begin{tabular}{@{}lrrrr@{}}
\toprule
\multicolumn{1}{c}{\textbf{Task}}                & \multicolumn{1}{c}{\textbf{Para1}} & \multicolumn{1}{c}{\textbf{Para2}} & \multicolumn{1}{c}{\textbf{Para4}} & \multicolumn{1}{c}{\textbf{Para5}} \\ \midrule
Crisis\textsubscript{Oltea}  & $48.1$K                             & $86.8$K                             & $120.7$K                           & $123.9$K                         \\
Emo\textsubscript{Moham}     & $3.3$K                              & $6.3$K                             & $10.9$K                           & $12.2$K                          \\
Hate\textsubscript{Waseem}   & $8.7$K                             & $16.6$K                           & $28.3$K                          & $31.7$K                          \\
Hate\textsubscript{David}    & $19.8$K                          & $38.2$K                          & $65.5$K                          & $73.4$K                           \\
Humor\textsubscript{Potash}  & $11.3$K                         & $21.8$K                            & $38.3$K                          & $44.0$K                             \\
Humor\textsubscript{Meaney}  & $8.0$K                            & $15.7$K                            & $28.7$K                        & $33.0$K                          \\
Irony\textsubscript{Hee-A}    & $3.5$K                          & $6.6$K                             & $11.4$K                           & $12.8$K                            \\
Irony\textsubscript{Hee-B}    & $3.5$K                            & $6.6$K                            & $11.5$K                           & $12.9$K                           \\
Offense\textsubscript{Zamp} & $11.9$K                            & $23.0$K                             & $39.6$K                           & $44.3$K                           \\
Sarc\textsubscript{Riloff}   & $1.4$K                            & $2.7$K                             & $4.6$K                            & $5.2$K                           \\
Sarc\textsubscript{Ptacek}   & $71.4$K                     & $138.9$K                            & $242$K                          & $272.4$K                          \\
Sarc\textsubscript{Rajad}    & $41.3$K                             & $78.3$K                           & $131.5$K                           & $146.6$K                          \\
Sarc\textsubscript{Bam}      & $11.9$K                          & $22.4$K                           & $37.5$K                           & $41.6$K                           \\
Senti\textsubscript{Rosen}   & $42.8$K                          & $84.3$K                           & $154.8$K                          & $178.1$K                        \\
Stance\textsubscript{Moham}  & $2.6$K                             & $4.7$K                             & $6.4$K                            & $6.6$K                              \\ \bottomrule
\end{tabular}%
} \vspace{-5pt}
\caption{Distribution of SMPB Train set.}\label{tab:append_para}
\end{table}
\vspace{-5pt}
We present an example paraphrase in Figure~\ref{fig:para}. We observe that the paraphrase model fails to generate emojis since emojis are out-of-vocabulary for the T5 model, but the model can preserve the overall semantics of the original tweet.
%~~~~~~~~~~~~~~~~~~~~~~~~
\begin{figure}[h]
\begin{centering}
 \includegraphics[scale=0.26]{emojis/example_para.PNG}\vspace{-7pt}
  \caption{Paraphrase example.} \vspace{-10pt}
  \label{fig:para}
\end{centering}
\end{figure}
%~~~~~~~~~~~~~~~~~~~~~~~~




% \begin{table}[h]
% \centering
% \footnotesize
% \begin{tabular}{@{}lrrrr@{}}
% \toprule
% \multicolumn{1}{c}{\textbf{Task}}                & \multicolumn{1}{c}{\textbf{Para1}} & \multicolumn{1}{c}{\textbf{Para2}} & \multicolumn{1}{c}{\textbf{Para4}} & \multicolumn{1}{c}{\textbf{Para5}} \\ \midrule
% Crisis\textsubscript{Oltea}  & 48,065                             & 86,765                             & 120,731                            & 123,879                            \\
% Emo\textsubscript{Moham}     & 3,257                              & 6,288                              & 10,886                             & 12,227                             \\
% Hate\textsubscript{Waseem}   & 8,683                              & 16,570                             & 28,323                             & 31,698                             \\
% Hate\textsubscript{David}    & 19,826                             & 38,177                             & 65,481                             & 73,367                             \\
% Humor\textsubscript{Potash}  & 11,325                             & 21,820                             & 38,318                             & 43,952                             \\
% Humor\textsubscript{Meaney}  & 7,200                              & 14,152                             & 25,741                             & 29,601                             \\
% Irony\textsubscript{Hee-A}    & 3,450                              & 6,646                              & 11,413                             & 12,777                             \\
% Irony\textsubscript{Hee-B}    & 3,450                              & 6,639                              & 11,469                             & 12,850                             \\
% Offense\textsubscript{Zamp} & 11,916                             & 22,951                             & 39,583                             & 44,348                             \\
% Sarc\textsubscript{Riloff}   & 1,413                              & 2,713                              & 4,628                              & 5,167                              \\
% Sarc\textsubscript{Ptacek}   & 71,433                             & 138,870                            & 242,065                            & 272,362                            \\
% Sarc\textsubscript{Rajad}    & 41,261                             & 78,328                             & 131,533                            & 146,605                            \\
% Sarc\textsubscript{Bam}      & 11,864                             & 22,434                             & 37,475                             & 41,570                             \\
% Senti\textsubscript{Rosen}   & 42,756                             & 84,299                             & 154,378                            & 178,147                            \\
% Stance\textsubscript{Moham}  & 2,622                              & 4,717                              & 6,425                              & 6,609                              \\ \bottomrule
% \end{tabular} 
% \caption{Distribution of SMPB Train set.}\label{tab:paraphrase}
% \end{table}
% 
% Please add the following required packages to your document preamble:
% \usepackage{multirow}
% \usepackage{graphicx}
% \usepackage[table,xcdraw]{xcolor}
% If you use beamer only pass "xcolor=table" option, i.e. \documentclass[xcolor=table]{beamer}
% \usepackage[normalem]{ulem}
% \useunder{\uline}{\ul}{}
\begin{table*}[h!]
\centering
\resizebox{\textwidth}{!}{%
\begin{tabular}{llc}
\toprule
\textbf{Original Tweet}                                                                                                                                                & \textbf{Paraphrase}                                                                                   & \textbf{Label}                                 \\ \toprule
                                                                                                                                                                       & You guys are horrible, avoid MMT                                                                       & { }                        \\
                                                                                                                                                                       & what I am doing is in my control, \#AvoidMMT, you guys are terrifying                                  & { }                        \\ 
                                                                                                                                                                       & You guys are \#terrorist. I have used everything I have to do.                                         & { }                        \\
\multirow{-4}{*}{\begin{tabular}[c]{@{}l@{}}USER but what I am doing is in my control, \\ \#AvoidMMT , you guys are \#terrible\end{tabular}}                           & You guys are awful, but I am going to stop doing it.                                                   & \multirow{-4}{*}{{ anger}} \\\hline
                                                                                                                                                                       & I hate when people say 'I need to talk to you or we need to talk to you'. I guess that's the problem.  &                                                \\
                                                                                                                                                                       & I hate when people tell me 'I need to talk to you or we need to speak' my anxiety immediately goes up. &                                                \\
\multirow{-3}{*}{\begin{tabular}[c]{@{}l@{}}I hate when people say 'I need to talk to you or we need to talk.' \\ My anxiety immediately goes up...\end{tabular}}      & Why am I afraid when people say 'I need to talk to you or we need to talk?'                            & \multirow{-3}{*}{anger}                        \\\hline
                                                                                                                                                                       & The 46th wedding I've ruined. When I hit 50 I can retire. It's nice to see yo                          &                                                \\
                                                                                                                                                                       & Here's the 47th wedding I've ruined. If I'm old enough to go on the 40s I can get married.             &                                                \\
\multirow{-3}{*}{This is the 47th wedding I've ruined. When I hit 50 I can retire.}                                                                                    & After a single wedding, I drew 47 weddings, and before I hit 50 I can retire"                          & \multirow{-3}{*}{humor}                        \\\hline
                                                                                                                                                                       & Sorry to disturb you. I have absolutely no idea what time I'll be on cam tomorrow.                     &                                                \\
\multirow{-2}{*}{\begin{tabular}[c]{@{}l@{}}Sorry guys I have absolutely no idea \\ what time i'll be on cam tomorrow but will keep you posted.\end{tabular}}          & I have absolutely no idea what time I'll be on camera tomorrow but I'll keep you posted                & \multirow{-2}{*}{sadness}                      \\\hline
                                                                                                                                                                       & "I'll buy you Dunkin' Donuts for \$5.                                                                  &                                                \\
                                                                                                                                                                       & Who wants to go with me for my tattoo tomorrow? I'll buy you a Dunkin' Donuts.                         &                                                \\
\multirow{-3}{*}{\begin{tabular}[c]{@{}l@{}}Who wants to go with me to get my tattoo tomorrow? \\ I'll buy you Dunkin doughnuts\end{tabular}}                          & Who wants to go with me to get my tattoo tomorrow?                                                     & \multirow{-3}{*}{neutral}                      \\\hline
                                                                                                                                                                       & The day before class please eat beans, onions and garlic. Also see the videos                          &                                                \\
                                                                                                                                                                       & "The Day Before Class. You should make that meal, (do you think).                                      &                                                \\
\multirow{-3}{*}{\begin{tabular}[c]{@{}l@{}}USER May I suggest, that you have a meal that is made with \\ beans, onions \& garlic, the day before class.\end{tabular}} & If you can eat just the day before class, make a wonderful meal with garlic, onions and beans.         & \multirow{-3}{*}{joy} \\                        
        \toprule               
\end{tabular}%
}
\caption{Examples of paraphrases in SMPB}
\label{tab:para_exam_app}
\end{table*}

% \subsection{Paraphrase-Based Methods}
% We present the average Test macro $F_1$ over three runs in Table~\ref{tab:para_res}. As Table~\ref{tab:para_res} shows, although none of our paraphrase-based models (Para-Models) exceed the gold RoBERTa baseline, our P4 model gets very close (i.e., only $0.35\%$ less). In addition, an ensemble of our Para-Models is slightly better than the gold RoBERTa baseline ($F_1=75.83$ vs. $75.78$). %We also explore intermediate fine-tuning where we try to further fine-tune the already fine-tuned paraphrase-based model on gold downstream data, but results we acquire are still below the performance of the model fine-tuned on gold without intermediate fine-tuning on paraphrase data (average $F_1$ $76.91$ vs. $78.12$).
% \begin{table*}[]
\scriptsize
\centering
\begin{tabular}{@{}lcccc:cc|cccc:cc@{}}
\toprule
\multicolumn{1}{c}{\multirow{2}{*}{\textbf{Task}}} & \multicolumn{6}{c}{\textbf{Baseline (RoBERTa\textsubscript{BASE})}}                                            & \multicolumn{6}{c}{\textbf{Our Best Method (X2+SFT-H)}}                                              \\ \cmidrule(lr){2-7} \cmidrule(lr){8-13} 
\multicolumn{1}{c}{}                               & \textbf{P1} & \textbf{P2} & \textbf{P4} & \textbf{P5} & \textbf{Best} & \textbf{Gold} & \textbf{P1} & \textbf{P2} & \textbf{P4} & \textbf{P5} & \textbf{Best} & \textbf{Gold} \\ \midrule
Crisis\textsubscript{Oltea}       & \textbf{95.54}       & 95.20       & 95.23       & 95.39       & 95.54         & 95.95         & 94.98       & 94.68       & 95.05       & \textbf{95.09}       & 95.09         & 95.68         \\
Emo\textsubscript{Moham}          & 77.06       & \textbf{77.43}       & 77.31       & 76.99       & 77.43         & 77.99         & 77.81       & \textbf{78.38}       & 77.61       & 77.84       & 78.38         & 80.50         \\
Hate\textsubscript{Waseem}        & 54.58       & \textbf{54.92}       & 54.83       & 54.63       & 54.92         & 57.34         & 53.48       & \textbf{54.39}       & 53.94       & 54.11       & 54.39         & 60.25         \\
Hate\textsubscript{David}         & 74.04       & 75.26       & \textbf{75.80}       & 74.97       & 75.80         & 77.71         & 73.12       & 74.48       & 74.08       & \textbf{74.66}       & 74.66         & 76.93         \\
Humor\textsubscript{Potash}       & 49.26       & 53.16       & \textbf{53.54}       & 53.34       & 53.54         & 54.40         & 49.61       & \textbf{54.39}       & 52.59       & 51.90       & 54.39         & 53.83         \\
Humor\textsubscript{Meaney}       & 91.71       & 91.66       & \textbf{92.28}       & 91.77       & 92.28         & 92.37         & 94.13       & \textbf{94.32}       & 93.95       & 94.16       & 94.32         & 94.49         \\
Irony\textsubscript{Hee-A}        & 71.00       & \textbf{71.53}       & 71.06       & 71.53       & 71.53         & 73.93         & 74.41       & 75.53       & 75.40       & \textbf{75.67}       & 75.67         & 79.89         \\
Irony\textsubscript{Hee-B}        & 43.24       & 45.69       & 48.02       & \textbf{48.33}       & 48.33         & 52.30         & 48.88       & \textbf{51.71}       & 50.56       & 49.69       & 51.71         & 61.67         \\
Offense\textsubscript{Zamp}       & 79.90       & 79.95       & \textbf{80.15}       & 79.57       & 80.15         & 80.13         & 75.96       & 78.22       & 78.23       & \textbf{78.80}       & 78.80         & 79.50         \\
Sarc\textsubscript{Riloff}        & 71.40       & \textbf{72.74}       & 72.45       & 72.26       & 72.74         & 73.85         & 76.91       & 77.59       & \textbf{79.41}       & 79.24       & 79.41         & 80.49         \\
Sarc\textsubscript{Ptacek}        & 90.54       & 92.58       & \textbf{93.70}       & 93.62       & 93.70        & 95.09         & 91.65       & 93.31       & \textbf{94.10}       & 93.98       & 94.10         & 96.24         \\
Sarc\textsubscript{Rajad}         & 81.00       & 81.94       & 82.53       & \textbf{82.64}       & 82.64         & 85.07         & 87.34       & 86.88       & \textbf{87.59}       & 87.38       & 87.59         & 88.92         \\
Sarc\textsubscript{Bam}           & 77.60       & 76.93       & \textbf{77.92}       & 77.76       & 77.92         & 79.08         & 80.86       & 80.06       & \textbf{81.53}       & 80.22       & 81.53         & 81.53         \\
Senti\textsubscript{Rosen}        & 70.50       & 70.95       & 71.51       & \textbf{71.52}       & 71.52         & 71.08         & 66.79       & 67.31       & \textbf{68.87}       & 68.29       & 68.87         & 71.08         \\
Stance\textsubscript{Moham}       & 67.75       & 69.07       & \textbf{69.30}       & 67.34       & 69.30         & 70.41         & 68.39       & \textbf{68.60}       & 68.58       & 67.03       & 68.60         & 70.77         \\ \cdashline{1-13} 
Average                                            & 73.01       & 73.93       & \textbf{74.38}       & 74.11       & \textbf{74.49}         & 75.78         & 74.29       & 75.32       & \textbf{75.43}       & 75.20       & \textbf{75.83}         & 78.12         \\ \bottomrule
\end{tabular} %\vspace{-5pt}
\caption{Results of paraphrasing-based model. \textbf{Gold} denotes a model fine-tuned with downstream, original Train data. \textbf{P$n$} indicates that the model is trained on Para$n$ training set. \textbf{Bold} denotes the best paraphrase result for each task within each of two settings (Baseline vs. X2+SFT-H). X2+SFT-H: Our model combining best settings of PM and SFT (see Table~\ref{tab:sft_res}). \textbf{Best}: Performance of the best paraphrasing setting. An ensemble of our best paraphrase-based models is slightly better than the gold-initialized baseline model. \vspace{-10pt}}~\label{tab:para_res} 
\end{table*}



% \begin{table*}[t]
% \scriptsize
% \centering
% \begin{tabular}{@{}lcccc:cccccc:cc@{}}
% \toprule
% \multicolumn{1}{c}{\multirow{2}{*}{\textbf{Task}}} & \multicolumn{6}{c}{\textbf{Baseline (RoBERTa\textsubscript{BASE})}}                                                                                        & \multicolumn{6}{c}{\textbf{Surrogate Fine-tuning (SFT)}}                                                                           \\ \cmidrule(l){2-13} 
% \multicolumn{1}{c}{}                               & \textbf{P1}                     & \textbf{P2}                     & \textbf{P4}                     & \textbf{P5}                     & \textbf{P-Best} & \multicolumn{1}{c|}{\textbf{Gold}} & \textbf{P1} & \textbf{P2}                     & \textbf{P4}                     & \textbf{P5}                     & \textbf{P-Best} & \textbf{Gold} \\ \midrule
% Crisis\textsubscript{Oltea}       & \textbf{95.54} & 95.20                           & 95.23                           & 95.39                           & 95.54           & \multicolumn{1}{c|}{95.95}         & 95.07       & 94.99                           & \textbf{95.12} & 95.06                           & 95.12           & 95.87         \\
% Emo\textsubscript{Moham}          & 77.06                           & \textbf{77.43} & 77.31                           & 76.99                           & 77.43           & \multicolumn{1}{c|}{77.99}         & 75.85       & \textbf{76.94} & 76.45                           & 76.63                           & 76.94           & 78.69         \\
% Hate\textsubscript{Waseem}        & 54.58                           & \textbf{54.92} & 54.83                           & 54.63                           & 54.92           & \multicolumn{1}{c|}{57.34}         & 54.01       & \textbf{54.89} & 54.65                           & 54.52                           & 54.89           & 63.97         \\
% Hate\textsubscript{David}         & 74.04                           & 75.26                           & \textbf{75.80} & 74.97                           & 75.80           & \multicolumn{1}{c|}{77.71}         & 72.54       & 74.06                           & 73.71                           & \textbf{75.27} & 75.27           & 77.29         \\
% Humor\textsubscript{Potash}       & 49.26                           & 53.16                           & \textbf{53.54} & 53.34                           & 53.54           & \multicolumn{1}{c|}{54.40}         & 50.38       & 51.99                           & 51.62                           & \textbf{53.22} & 53.22           & 55.51         \\
% Humor\textsubscript{Meaney}       & 88.67                           & 88.85                           & \textbf{89.11} & 88.16                           & 89.11           & \multicolumn{1}{c|}{90.33}         & 90.33       & 89.73                           & \textbf{90.46} & 89.21                           & 90.46           & 92.02         \\
% Irony\textsubscript{Hee-A}        & 71.00                           & 71.53                           & 71.06                           & \textbf{71.53} & 71.53           & \multicolumn{1}{c|}{73.93}         & 72.89       & 73.43                           & \textbf{75.28} & 74.29                           & 75.28           & 76.22         \\
% Irony\textsubscript{Hee-B}        & 43.24                           & 45.69                           & 48.02                           & \textbf{48.33} & 48.33           & \multicolumn{1}{c|}{52.30}         & 49.01       & 49.08                           & \textbf{51.82} & 51.14                           & 51.82           & 60.14         \\
% Offense\textsubscript{Zamp}       & 79.90                           & 79.95                           & \textbf{80.15} & 79.57                           & 80.15           & \multicolumn{1}{c|}{80.13}         & 77.98       & \textbf{79.41} & 78.96                           & 79.10                           & 79.41           & 79.82         \\
% Sarc\textsubscript{Riloff}        & 71.40                           & \textbf{72.74} & 72.45                           & 72.26                           & 72.74           & \multicolumn{1}{c|}{73.85}         & 74.27       & 76.86                           & 77.26                           & \textbf{78.36} & 78.36           & 80.50         \\
% Sarc\textsubscript{Ptacek}        & 90.54                           & 92.58                           & \textbf{93.70} & 93.62                           & 93.70           & \multicolumn{1}{c|}{95.09}         & 90.85       & 92.69                           & 93.57                           & \textbf{93.61} & 93.61           & 96.01         \\
% Sarc\textsubscript{Rajad}         & 81.00                           & 81.94                           & 82.53                           & \textbf{82.64} & 82.64           & \multicolumn{1}{c|}{85.07}         & 85.71       & 86.13                           & 86.19                           & \textbf{86.32} & 86.32           & 87.56         \\
% Sarc\textsubscript{Bam}           & 77.60                           & 76.93                           & \textbf{77.92} & 77.76                           & 77.92           & \multicolumn{1}{c|}{79.08}         & 80.20       & 80.05                           & \textbf{80.90} & 80.22                           & 80.90           & 81.19         \\
% Senti\textsubscript{Rosen}        & 70.50                           & 70.95                           & 71.51                           & \textbf{71.52} & 71.52           & \multicolumn{1}{c|}{71.08}         & 67.35       & 68.20                           & \textbf{69.18} & 68.59                           & 69.18           & 71.83         \\
% Stance\textsubscript{Moham}       & 67.75                           & 69.07                           & \textbf{69.30} & 67.34                           & 69.30           & \multicolumn{1}{c|}{70.41}         & 67.61       & \textbf{69.61} & 68.20                           & 69.47                           & 69.61           & 71.27         \\ \cdashline{1-13} 
% \textbf{Average}                  & 72.81                           & 73.75                           & 74.17                           & 73.87                           & 74.28           & \multicolumn{1}{c|}{75.64}         & 73.60       & 74.54                           & 74.89                           & 75.00                           & 75.36           & 77.86         \\ \bottomrule
% \end{tabular} \vspace{-5pt}
% \caption{Results of paraphrasing-based model. \textbf{Gold} denotes the model train with original Train dataset. \textbf{P$n$} indicates that the model is trained on Para$n$ training set. \textbf{Bold font} denotes the best paraphrase result for each task within each of two settings (Baseline vs. SFT). \textbf{P-Best} is the performance of the best paraphrasing setting. \vspace{-10pt}}~\label{tab:para_res} 
% \end{table*}




