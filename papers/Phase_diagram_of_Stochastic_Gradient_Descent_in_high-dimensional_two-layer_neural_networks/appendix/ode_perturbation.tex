\section{A lemma on ODE perturbation}

In this section, we prove a proposition that bounds the difference between an ODE solution and a perturbed version, for a bounded time $t$.

\begin{theorem}\label{th:ode_perturbation}
    Let $f, g: \mathbb R^n \to \mathbb R^n$ be two $L$-Lipschitz functions, and consider the following differential equations in $\mathbb R^n$:
    \begin{align*}
        %\frac{d x}{dt} &= f(x) + \epsilon g(x),\\
        %\frac{dy}{dt} &= f(y),
       \dv{\x}{t}  &= f( \x ) + \epsilon g( \x ),\\
        \dv{\bm{y}}{t}  &= f( \bm{y} ),
    \end{align*}
    where $\epsilon > 0$, and with the initial condition 
    %$x(0) = y(0)$
    $ \x (0) = \bm{y} (0) $ . Then, if $\T > 0$ is fixed, we have
    %\[ \lVert x(t) - y(t) \rVert_2 \leq c \epsilon e^{L \tau}  \]
     \[ \lVert \x (t) - \bm{y} (t) \rVert_2 \leq c \epsilon e^{L \tau}  \]
    for any $0 \leq t \leq \T$, with $c$ a constant independent from $\epsilon, \T$.
\end{theorem}

Before proving this proposition, we begin with a small lemma:
%\begin{lemma}\label{lem:norm_ode_solution}
%    Let $a, b \in \mathbb R$. The unique solution to the differential equation
%    \[ \frac{dz}{dt} = a z + b \sqrt{z} \]
%    with $z(0) = 0$ is given by
%    \[ z(t) = b^2 \frac{(e^{at/2} - 1)^2}{a^2} \]
%\end{lemma}
%\begin{proof}
%    The Cauchy-Lipschitz theorem ensures the existence and uniqueness of the solution, and the fact that $z(t)$ satisfies the equation is easily checked.
%\end{proof}

\begin{lemma}\label{lem:norm_ode_solution}
    Let $a, b > 0$, and $z: \mathbb R^+ \to \mathbb R^+$ a function satisfying
    \[ \dv{z}{t}  = a z + b \sqrt{z} \]
    with $z(0) = 0$. Then, for some constant $c > 0$, we have
    \[ z(t) \leq c \frac{b^2 e^{at}}{a^2} \quad \text{for all} \quad t\geq 0\]
\end{lemma}
\begin{proof}
    Upon considering the function $a^2 z(t / a) / b^2$ instead, we can assume that $a = b = 1$. Then, we have
    \[ \dv{z}{t}   \leq \max(z, 1) + \max(\sqrt{z}, 1), \]
    and the RHS is an increasing function. Hence, if $\tilde z$ is a solution of
    \[ \dv{\tilde z}{t}  = \max(z, 1) + \max(\sqrt{\tilde z}, 1), \]
    with $\tilde z(0) = 0$, then $z(t) \leq \tilde z(t)$ for all $t \geq 0$. Since the RHS of the above equation is Lipschitz everywhere, we can apply the Picard–Lindelöf theorem, and check that the unique solution to this equation is
    \[ \tilde z(t) = \begin{cases}
        2 t & \text{if } t\leq \frac{1}{2}\\
        (c_1 e^{t} -c_2)^2 & \text{otherwise}
    \end{cases},\]
    where $c_1$ and $c_2$ are ad hoc constants. The lemma then follows from adjusting the constant $c$ as needed.
\end{proof}

We are now in a position to show Theorem \ref{th:ode_perturbation}:
\begin{proof}
    Assume for simplicity that 
    %$x(0) = y(0) = 0$
    $\x(0) = \bm{y}(0) = \bm{0}$. We begin by bounding 
    %$x(t)$
    $\x (t) $; we have
    %\[ \frac{d \lVert x \rVert^2}{dt} = \langle x, \frac{dx}{dt} \rangle \leq \lVert x \rVert \lVert f(x) + \epsilon g(x) \rVert. \]
      \begin{equation}
        \nonumber
        \dv{\lVert \x \rVert^2}{t}  = 2  \x^\top  \dv{\x}{t}   \leq 2  \lVert \x \rVert  \; \lVert f( \x ) + \epsilon g( \x ) \rVert \;.
    \end{equation}
    By the Lipschitz condition,
    %\[ \lVert f(x) + \epsilon g(x) \rVert \leq \lVert f(0) + \epsilon g(0) \rVert + L \lVert x \rVert,   \]
     \begin{equation}
    \nonumber
        \lVert f( \x ) + \epsilon g( \x ) \rVert \leq \lVert f( \bm{0} ) + \epsilon g( \bm{0} ) \rVert + \frac{L}{2} \lVert \x \rVert \;,
    \end{equation}
    so that
    %\[ \frac{d \lVert x \rVert^2}{dt} \leq L \lVert x \rVert^2 + \lVert f(0) + \epsilon g(0) \rVert \lVert x \rVert \]
    \begin{equation}
    \nonumber
       \dv{\lVert \x \rVert^2}{t} \leq L \lVert \x \rVert^2 + 2 \lVert f ( \bm{0} ) + \epsilon g( \bm{0} ) \rVert \; \lVert \x \rVert   \;.
    \end{equation}
    %Applying Lemma \ref{lem:norm_ode_solution} and taking square roots on each side,
    Applying Lemma \ref{lem:norm_ode_solution} and taking square roots on each side,
    %\begin{equation}\label{eq:bound_norm_x}
    %    \lVert x(t) \rVert \leq \frac{\lVert f(0) + \epsilon g(0) \rVert }{L} e^{Lt/2} \leq \frac{\lVert f(0) + \epsilon g(0) \rVert }{L} e^{L\T/2} .
    %\end{equation}
       \begin{equation}
    \label{eq:bound_norm_x}
        \lVert \x (t) \rVert \leq c\frac{\lVert f( \bm{0} ) + \epsilon g( \bm{0} ) \rVert }{L} e^{Lt/2} \leq c\frac{\lVert f( \bm{0} ) + \epsilon g( \bm{0} ) \rVert }{L} e^{L\T/2} \;,
    \end{equation}
    for any $0 \leq t \leq \T$. Now, similarly,
    %\begin{align*}
    %\frac{d \lVert x-y \rVert^2}{dt} &\leq \lVert x-y \rVert \left\lVert \frac{d(x-y)}{dt} \right\rVert\\
    %&\leq \lVert x-y \rVert \lVert f(x) - f(y) + \epsilon g(x) \rVert \\
    %&\leq L \lVert x-y \rVert^2 + \epsilon \lVert g(x) \rVert \lVert x-y \rVert \\
    %&\leq L \lVert x-y \rVert^2 + \epsilon \left( \lVert g(0) \rVert + \lVert f(0) + \epsilon g(0) \rVert e^{L\T/2} \right) \lVert x-y \rVert,
    %\end{align*}
     \begin{align*}
     \dv{\lVert \x - \y \rVert^2}{t}  & \leq 2  \lVert \x - \y \rVert \left\lVert \dv{( \x - \y )}{t}  \right\rVert\\
    &\leq 2 \lVert \x - \y \rVert \; \lVert f( \x ) - f( \y ) + \epsilon g( \x ) \rVert \\
    &\leq L  \lVert \x - \y \rVert^2 + 2 \epsilon \lVert g( \x ) \rVert \; \lVert \x - \y \rVert \\
    &\leq L \lVert \x - \y \rVert^2 +  \epsilon \left( \lVert g( \bm{0} ) \rVert + c\lVert f( \bm{0} ) + \epsilon g( \bm{0} ) \rVert e^{L\T/2} \right) \lVert \x - \y \rVert \;,
    \end{align*}
    %having used \eqref{eq:bound_norm_x} on the last line.
    having used \eqref{eq:bound_norm_x} on the last line.
    This is again the setting of Lemma \ref{lem:norm_ode_solution}, which gives
    %\[ \lVert x-y \rVert \leq c_1 \epsilon e^{L\T/2} \frac{e^{Lt/2}}{L} \leq c_2 \epsilon e^{L\tau}. \]
      \begin{equation}
        \nonumber
        \lVert \x - \y \rVert \leq c_1 \epsilon e^{L\T/2} \frac{e^{Lt/2}}{L} \leq c_2 \epsilon e^{L\tau} \;.
    \end{equation}
\end{proof}

