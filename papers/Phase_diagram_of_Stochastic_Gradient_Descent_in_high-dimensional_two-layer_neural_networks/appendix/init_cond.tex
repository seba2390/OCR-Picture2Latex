\section{Initial conditions and symmetric teacher}\label{app:init_cond}

In this work we have constructed teacher matrices $ \W^* \in \R^{\hidt\times\inp} $ in order to have 
\begin{equation}
\label{eq:sym_teacher}
    \prs =  \frac{ \w^{*\top}_r \w^{*}_{s} }{d} = \delta_{rs} \;,
\end{equation}
where $ \w^{*}_r \equiv [\W^*]_r \in \R^\inp $ is the $r$-th row of the matrix $\W^*$. We have started by sampling $\hidt$ vectors of dimension $\inp$ uniformly on a ball of radius $\sqrt{\inp}$. Then we constructed an orthonormal basis using singular value decomposition. 

The initial student weights $\W^0 \in \R^{\hids\times\inp}$ were taken as 
\begin{equation}
\W^0 = \bm{A} \W^*     \;,
\end{equation}
with each row of $\bm{A} \in \R^{\hids \times \hidt}$ sampled uniformly on a ball of radius one. We acknowledge choosing initial student weights as linear combinations of the teacher can be artificial and shrinks the first plateau, but our focus on this work was the specialization phase. Nevertheless, this choice and Eq.~(\ref{eq:sym_teacher}) are particularly suitable to theoretical analysis. Once $\hidt$ and $\hids$ are fixed, the dimension $\inp$ can be varied without changing $ \Q^0$, $\M^0$ and $\P$, thereby removing any influence of different initial conditions for different $\inp$ and providing the reader better visualization on the learning curves. To clarify this point, consider the $j$-th row $ \w^{0}_j \equiv [\W^0]_j \in \R^\inp $ of $\W^0 $:
\begin{equation}
    \w^{0}_j   =  \sum_{r=1}^{\hidt} a_{jr} \w_{r}^*  \;,
\end{equation}
with $ a_{jr} \equiv [ \bm{A}  ]_{jr} $. Using Eq.~\eqref{eq:sym_teacher} one can write
\begin{equation}
\qjl^0 = \frac{\w^{0\top}_j \w^{0}_l }{d} = \sum_{r,r' =1}^{\hidt} a_{jr} a_{j r'}   \underbrace{\frac{\w^{*\top}_r \w^{*}_{r'} }{d}}_{=\delta_{ r r'} } = \sum_{r = 1}^{\hidt} a_{jr} a_{l r} \;.
\end{equation}
Similarly,
\begin{equation}
\mjr^0 =\frac{\w^{0\top}_j \w^{*}_r }{d}  =  a_{jr} \;.
\end{equation}
Thus once $\bm{A}$ is fixed, the input dimension $\inp$ can be varied without affecting the initial conditions. We chose to sample $ \bm{a}_j \equiv [ \bm{A}  ]_j \in \R^\hidt $ on a ball of radius one both to introduce some randomness on the initialization and to keep the initial parameters bounded by one.

We stress that we use these initial conditions to make the data comparable for varying dimension $d$ in the numerical illustrations. Our conclusions do not depend on this particular choice of initial conditions. If one simply takes random initialization $ \w_j \sim \gauss  (\w_j |\bm{0},\Id ) $ for each $j$, the full picture we have presented in this manuscript remains unchanged. In Figure \ref{fig:plot_d_init_not_inf} we present an example of curves within the blue region (see Section \ref{sec:main} for the characterization of this regime) with unconstrained Gaussian initialization. Dots represent simulations, while solid lines are obtained by integration of the ODEs given by Eqs.~(\ref{eq:qmode0}), with initial conditions adjusted to match simulations.

Although varying the initial population risk with $\inp$ slightly changes the exact position where the specialization transition starts, the particular initial conditions adopted in this work do not affect whether the specialization transition takes place or not, comparing to unconstrained Gaussian initialization.


\begin{figure}[H]
\begin{center}
\centerline{\includegraphics[width=0.45\textwidth]{figures/plot_sim_ode_blue_d1000_n008_k004_dscale_init_not_inform.pdf}}
\caption{Population risk dynamics for $\kap=\del=0$ (Saad \& Solla scaling) : $\hids_0 = 8$, $\hidt=4$, $\prs = \delta_{rs}$. Initialization: $ \w_j \sim \gauss  (\w_j |\bm{0},\Id ) $ for $j = 1, ..., \hids_0$. Activation function: $\act(x) = \erf(x/\sqrt{2})$. Data distribution: $\Prob (\x) =  \gauss(\x | \bm{0}, \Id )$. Dots represent simulations, while solid lines are obtained by integration of the ODEs given by Eqs.~(\ref{eq:qmode0}), with initial conditions adjusted to match simulations. Observe the difference on the initialization for different $\inp$.}
\label{fig:plot_d_init_not_inf}
\end{center}
\vskip -0.2in
\end{figure}


