
A deterministic semi-automaton is \emph{synchronizing} if it admits a reset word, i.e., a word which leads to a definite
state, regardless of the starting state. This notion has a wide range of applications, from software testing, circuit synthesis, communication engineering and the like, see~\cite{DBLP:journals/et/ChoJSP93,San2005,Vol2008}.  
The famous \v{C}ern\'y conjecture \cite{Cer64}
states that a minimal synchronizing word, for an $n$ state automaton, has length
at most $(n-1)^2$. %hier vielleicht ncoh mehr zu schreiben,  Shitov JALC-paper?
We refer to the mentioned survey articles for details~\cite{San2005,Vol2008}. 



Due to its importance, the notion of synchronization has undergone a range of generalizations and variations
for other automata models.
%There are different notions of synchronization
%for partial automata, or for non-deterministic automata. 
% It was noted in~\cite{Martyugin12} that in some  generalizations only certain paths, or input words, are allowed (namely those for which the input automaton is defined). In~\cite{Gusev:2012}
% the notion of constrained synchronization was %implicitly 
% introduced in connection with a reduction procedure
% for synchronizing automata, but it was not formulated as a computational problem in itself
% and only used as an auxiliary tool with a very specific constraint language.
The paper~\cite{DBLP:conf/mfcs/FernauGHHVW19} introduced the constrained synchronization problem (CSP\footnote{In computer science the
acronym CSP is usually used for the constraint satisfaction problem~\cite{Lecoutre09}. However, as here we are not concerned with
constrained satisfaction problems at all, no confusion should arise.}). 
In this problem, we search for a synchronizing word coming from a specific subset of allowed
input sequences. 
To sketch a few applications:

\section{Motivations for Empirical Study}
\label{sec:motivations}
The key question that we try to answer is when and why we should use standard
iteration space tiling over cache oblivious tiling.  The two approaches
perform similar partitioning of the iteration space, but the schedules given
to the partitions are different.  Theoretically, cache oblivious code seems to
have advantages over iteration space tiling.  However, many factors complicate
the actual performance, which made our initial experiments difficult to
interpret.  In this section, we describe the obstacles between the theory and
practice we have identified.

We use Single-Level Tiling (SLT) for iteration space tiling, and Cache
Oblivious Tiling (COT) for cache oblivious techniques in this
paper, which are further described in Section~\ref{sec:background}.

\paragraph{Recursion Overhead} This is a well-known overhead of
COT~\cite{yotov2007experimental}.  The recursion introduces overheads, such as
function call overhead, and increased register pressure.  Furthemore, the
functions force inter-procedural analysis/optimization, known to be more
difficult for compilers well.  Thus, the leaf tiles must be ``sufficiently
large'' to avoid excessive overhead due to the recursion.

 \paragraph{Recursive Split Constraints the Tile Sizes} In typical cache
 oblivious algorithms, the problem is recursively split into halves in each
 dimension. This is in fact a rather coarse-grained exploration of the
 hierarchical partitioning of the iteration space. For instance, if the
 current problem size is $B^3$, then the next sub-problem would be
 $(\frac{B}{2})^3$.  If the best problem size for utilizing a level of cache
 is $(B-x)^3$ where $x\ll \frac{B}{2}$ then the subproblems due to
 divide-and-conquer will not match the best.  This is another factor that
 necessitates fine tuning of leaf tile sizes even for COT, since the utilization
 rate of L1 cache has strong impact on performance.  

%\paragraph{COT Leads to Imbalanced Tiles} Current COT tools recursively split
%the problem into halves in each dimension.  If the original bounds are not
%powers of two, every power-of-two leaf will be paired with a non-power-of-two
%leaf.  Since leaf tile sizes are often carefully tuned, thismeans that half
%the leaves will be suboptimal.  Our code generator incorporates a simple
%optimization that ensures that such suboptimal leaf nodes only occur at the
%boundaries of the iteration space.

\paragraph{COT has more Conflict Misses} The divide-and-conquer execution
order may negatively affect cache interference, especially with high
dimensional data.  This happens when the memory is allocated such that the
accesses are contiguous along some direction in the iteration space (typically
along innermost canonical axis).  With lexicographic order of execution, this
contiguity is largely preserved in the tiled execution.  However,
divide-and-conquer executes neighboring tiles in all dimensions, and many of
those tiles access some distant location in memory.  In contrast to accessing
contiguous regions of memory, accessing various segments of the memory
increases the chances of conflicts.

\paragraph{Hardware Prefetching}  Modern architectures are equipped with
hardware prefetchers that can bring data to the L1 cache. When
having sufficient locality at L2 or LLC makes the program compute-bound, then
the latency to L2/LLC can be hidden by the prefetcher. For such programs, it is
unnecessary to tile for the fastest cache, and larger tiles targeting slower
caches improve performance by maximizing prefetcher
effectiveness~\cite{mehta2016turbotiling}. When the primary objective is speed,
the leaf tiles for COT should also be large, which negates the benefit of
divide-and-conquer, as the leafs are already targeting slower caches.
Prefetching have little impact on parallel executions, since prefetching is
bandwidth limited. When multiple cores try to prefetch at the same time,
the bandwidth limit is quickly reached, and the latency hiding effect is
lost. Furthermore, smaller tile sizes are better for parallel execution for
load balancing  reasons.


These factors limit the effectiveness of COT in various ways and are also
closely tied to the characteristics of the computation. Our empirical study
illustrate the impact of these factors on polyhedral computations.

% Local Variables: ***
% TeX-master: "TACO2017.tex" ***
% fill-column: 78 ***
% End: ***




%For further motivation and applications we refer to the aforementioned paper \cite{DBLP:conf/mfcs/FernauGHHVW19}.
In~\cite{DBLP:conf/mfcs/FernauGHHVW19}, a complete analysis of the complexity landscape when the constraint language is given by small partial automata was done. It is natural to extend this result to other language classes.%, or
%even to give a complete classification of all the complexity classes that could arise.
%For commutative regular constraint languages, a full classification of the realizable
%complexities was given in~\cite{DBLP:conf/cocoon/Hoffmann20}.
%In~\cite{DBLP:conf/ictcs/Hoffmann20}, it was shown that for polycyclic constraint languages, the problem is always in $\NP$.


In general there exist constraint languages yielding 
\PSPACE-complete constrained problems~\cite{DBLP:conf/mfcs/FernauGHHVW19}. 
A language is polycyclic~\cite{DBLP:conf/ictcs/Hoffmann20}, if it is recognizable by an automaton
such that every strongly connected component forms a single cycle,
and a language is sparse~\cite{DBLP:reference/hfl/Yu97} if only polynomially many words of a specific length are in the language.
As shown in~\cite{DBLP:conf/ictcs/Hoffmann20}
for polycyclic languages, which, as we show, equal the sparse regular
languages, the problem is always in \NP. This motivates investigating this class further. 
Also, as written in more detail in Remark~\ref{rem:motivation_strictly_bounded},
a subclass of these languages has a close relation to the commutative languages, and as for commutative constraint
languages a trichotomy result has been established~\cite{DBLP:conf/cocoon/Hoffmann20},
tackling the sparse languages seems to be the next logical step.
In fact, we show a dichotomy result for a subclass
that contains the class corresponding to the commutative languages.
Additionally, as has been noted in~\cite{DBLP:conf/mfcs/FernauGHHVW19},
the constraint language $ab^*a$ is the smallest language, in terms of a recognizing
automaton, giving an \NP-complete CSP. The class of languages
for which our dichotomy holds true contains this language.


%Here, we will look at the complexity landscape for strictly bounded regular constraint languages
%and bounded regular constraint languages induced by strongly self-synchronizing codes.

% Here, we will look at the complexity landscape for sparse regular constraint
% languages. We will show that they equal the bounded regular and the polycyclic languages.
% Then, we will investigate the problem for the strictly bounded constraint
% languages and constraint languages induced by strongly self-synchronizing codes.


Let us mention that restricting the solution space by a regular language
has also been applied in other areas, for example to topological sorting~\cite{DBLP:conf/icalp/AmarilliP18},
solving word equations~\cite{Diekert98TR,DBLP:journals/iandc/DiekertGH05}, constraint programming~\cite{DBLP:conf/cp/Pesant04}, or
shortest path problems~\cite{DBLP:journals/ipl/Romeuf88}.
The famous road coloring theorem~\cite{adler1970similarity,Trahtman09} states
that every finite strongly connected and directed aperiodic graph of uniform out-degree admits a labelling of its edges 
such that a synchronizing
automaton results. A related problem to our problem of constrained synchronization is to restrict the possible labelling(s), and
this problem was investigated in~\cite{DBLP:journals/jcss/VorelR19}.






\paragraph{Outline and Contribution.} 
%\subsection{Outlnie and Constribution}
Here, we look at the complexity landscape for sparse regular constraint
languages.
In Section~\ref{sec:sparse}
 we introduce the sparse languages and show that the regular sparse
 languages are characterized by
 polycyclic automata introduced in~\cite{DBLP:conf/ictcs/Hoffmann20}.
 A similar characterization in terms of non-deterministic
 automata was already given in~\cite[Lemma 2]{DBLP:journals/ijfcs/GawrychowskiKRS10}.
 In this sense, we extend this characterization to the deterministic
 case. As for polycyclic constraint automata the constrained
 problem is always in \NP, see~\cite[Theorem 2]{DBLP:conf/ictcs/Hoffmann20},
 we can deduce the same for sparse regular constraint languages, which
 equal the bounded regular languages~\cite{DBLP:journals/eik/LatteuxT84}.
 
 
In Section~\ref{sec:strictly_bounded_case}
we introduce the letter-bounded languages, a proper subset of the sparse languages,
and show that for letter-bounded  constraint languages, the constrained
synchronization problem is either in \PTIME\ or \NP-complete.
 
 
 The difficulty why we cannot handle the general case yet lies in the fact that
 in the reductions, %for arbitrary (sparse) constraint languages
 %used for the special case, 
 in the general case, we need auxiliary states and it is not clear
 how to handle them properly, i.e, 
 how to synchronize them properly while staying inside the constraint language.
 
 
 In Section~\ref{sec:strongly_self_sync}
 we introduce the class of strongly self-synchronizing
 codes. 
 The strongly self-synchronizing codes allow us to handle these auxiliary states mentioned before.
 We show that for homomorphisms given by such codes,
 the constrained problem for the homomorphic image of a language has the same computational
 complexity as for the original language.
 This result holds in general, and hence is of independent interest.
 Here we apply it to the special case
 of bounded, or sparse, regular languages given by such codes.
 
 
 Lastly, we present a bounded language giving an \NP-complete constrained
 problem that could not be handled by our methods so far.
 
 

