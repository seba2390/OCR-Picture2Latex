\section{Sparse and Bounded Regular Languages}
\label{sec:sparse}

Here, in Theorem~\ref{thm:sparse_in_NP}, we establish that for constraint languages from the class of sparse regular languages, which equals the class of the bounded regular languages~\cite{DBLP:journals/eik/LatteuxT84}, the constrained problem
is always in \NP.


A language $L \subseteq \Sigma^*$ is \emph{sparse},
%\cite{DBLP:journals/siamcomp/BermanH77,DBLP:journals/ijfcs/GawrychowskiKRS10,DBLP:journals/ipl/Hartmanis83,DBLP:series/wsscs/Hartmanis93b,DBLP:journals/iandc/HartmanisIS85,DBLP:conf/mfcs/HartmanisM80,DBLP:conf/focs/Mahaney80,DBLP:journals/jcss/Mahaney82,Pin2020,DBLP:conf/mfcs/SzilardYZS92,DBLP:reference/hfl/Yu97}
if there exists $c \ge 0$
such that, for every $n \ge 0$, we have
$L \cap \Sigma^n \in O(n^c)$.
Sparse languages were introduced into computational complexity
theory by Berman \& Hartmanis~\cite{DBLP:journals/siamcomp/BermanH77}.
Later, it was established by Mahaney that if there exists
a sparse \NP-complete set (under polynomial-time many-one reductions),
then $\PTIME = \NP$~\cite{DBLP:journals/jcss/Mahaney82}.
For a survey on the relevance of sparse sets in computational complexity theory, see~\cite{DBLP:conf/mfcs/HartmanisM80}.


A language $L \subseteq \Sigma^*$ is called \emph{bounded},
if there exist $w_1, \ldots, w_k \in \Sigma^*$
such that $L \subseteq w_1^* \ldots w_k^*$. %~\cite{GinsburgSpanier66,DBLP:journals/eik/LatteuxT84}.
Bounded languages were introduced by Ginsburg \& Spanier~\cite{GinsburgSpanier64}.

We will need the following representation of the bounded regular languages.

% noch begründen, warum bounded überhaupt erwähnt wird, und nicht direkt 
% nur mit dem begriff sparse gearbeitet wird todo -> wegen klasse strictly bounded

\begin{theorem}[\cite{GinsburgSpanier66}]
\label{thm:bounded_regular_form}
 A language $L \subseteq w_1^* \cdots w_k^*$ is regular if and only if
 it is a finite union of languages of the form $L_1 \cdots L_k$, where each $L_i \subseteq w_i^*$ is regular.
\end{theorem}

It is known that the class of sparse regular languages equals
the class of bounded regular languages~\cite{DBLP:journals/eik/LatteuxT84},
or see~\cite{Pin2020,DBLP:reference/hfl/Yu97}, where the bounded languages are not mentioned
but the equivalence is implied by their results and Theorem~\ref{thm:bounded_regular_form}.
The next results links this class to the polycylic PDFAs.

\begin{propositionrep}
\label{thm:bounded_characterization}
 %For regular $L \subseteq \Sigma^*$, the following are equivalent:
 %(1) $L$ is sparse, (2) $L$ is bounded, (3) $L$ is recognizable by a polycyclic PDFA.
 Let $L \subseteq \Sigma^*$ be regular. Then, $L$ is sparse
 if and only if it is recognizable by a polycyclic PDFA.
\end{propositionrep}
\begin{proof}
 In~\cite{DBLP:journals/eik/LatteuxT84} is was shown that the context-free sparse languages are precisely the context-free bounded languages, which
 gives our first two equivalences.
 A result from~\cite[Lemma 2]{DBLP:journals/ijfcs/GawrychowskiKRS10}
 readily implies that if a language is recognized by a polycyclic PDFA, then
 it must be sparse.
 Lastly, we show that every bounded regular language
 is recognizable by a polycyclic automaton, which finishes the proof.
 
 \medskip 
 
 \noindent\underline{Claim:} For $w \in \Sigma^*$.
  Then, any regular $L \subseteq w^*$
  is recognizable by a polycyclic PDFA.
 \begin{quote}
     \emph{Proof of the Claim.}
      Let $w \in \Sigma^*$ and $L \subseteq w^*$ be a regular language.
 If $w = \varepsilon$, then $L = \{\varepsilon\}$, which is obviously recognizable
 by a polycyclic automaton. So, suppose $|w| > 0$.
%  Suppose $\mathcal A = (\Sigma, Q, \delta, q_0, F)$
%  is an accepting PDFA for $L$ such that every state is accessible
%  and coacessible, which we can assume by Lemma~\ref{lem:accessible_coaccessible}.
%  Choose any $q \in Q$. Then, there exists $u_1, u_2 \in \Sigma^*$
%  such that $\delta(q_0, u_1) = q$
%  and $\delta(q, u_2) \in F$.
%  Hence, if $\delta(q, v) = q$,
%  then $u_1 v u_2 \in L$, so that $u_1 v u_2 \subseteq w^*$.
%  If $v = \varepsilon$, then $v\subseteq w^*$.
%  So, suppose $|v| > 0$.
%  Then $u_1 v u_2 = w^n$ for some $n > 0$.
%  But then, $v = v_1 w^k v_2$
%  for some maximal $k \ge 0$.
%
%  Every accepting automaton for a language $L \subseteq w^*$, for some $w \in \Sigma^*$, is obviously polycyclic.
%
% Todo, inverse hom, single final acceptable?
% Applying Theorem~\ref{thm:bounded_regular_form},
% we can write $L = L_1 \cup \ldots \cup L_n$
% such that, for any $i \in \{1,\ldots,n\}$, $L_i \subseteq w^*$ is regular.
 Let $a$ be an arbitrary symbol
 and define a homomorphism $\varphi : \{a\}^* \to \Sigma^*$
 by $\varphi(a^i) = w^i$, which is injective as $|w| > 0$ by assumption.
 Then, the unary language $\varphi^{-1}(L) = \{ a^i \mid w^i \in L\}$ is regular, as inverse
 homomorphisms preserve regularity.
 Hence, we can write it as a union of languages recognizable by automata
 with a single final state, which, by Lemma~\ref{lem::unary_single_final},
 have the form $\{ a^i \}$ for some $i \ge 0$
 or $\{ a^{i + jp} \mid j \ge 0 \}$ for some $i \ge 0, p > 0$.
 As the application of functions preserves union, and $L = \varphi(\varphi^{-1}(L))$ here,
 the language 
 $L$ is the union of the images of these languages.
 We have $\varphi(\{a^i \}) = \{ w^i \}$, and this singleton language
 is obviously recognizable by a polycyclic automaton,
 and we have $\varphi(\{ a^{i + jp} \mid j \ge 0 \}) = \{ w^{i+pj} \mid j \ge 0 \}$,
 and this language is also recognizable by an automaton that
 has an initial tail labelled by $w^i$ and a cycle labelled by $w^p$.
 So, as the polycyclic languages are closed under union~\cite[Proposition 6]{DBLP:conf/ictcs/Hoffmann20},
 we have shown that the language $L$ %languages $L_i, i \in \{1,\ldots,n\}$,
 is recognizable by some polycyclic automaton.     \emph{[End, Proof of the Claim]}
 \end{quote}
 Finally, as the languages recognizable by polycyclic automata
 are closed under concatenation and union~\cite[Proposition 5 and Proposition 6]{DBLP:conf/ictcs/Hoffmann20},
 by Theorem~\ref{thm:bounded_regular_form} every bounded regular language is recognizable by a polycyclic automaton.
\end{proof}

In~\cite[Theorem 2]{DBLP:conf/ictcs/Hoffmann20} it was shown that for polycyclic
constraint languages, the constrained problem is always in $\NP$.
So, we can deduce the next result.

\begin{theorem}
\label{thm:sparse_in_NP}
 If $L \subseteq \Sigma^*$ is sparse and regular, then $L\textsc{-Constr-Sync} \in \NP$.
\end{theorem}

We will need the following closure property stated in~\cite[Theorem 3.8]{DBLP:reference/hfl/Yu97}
of the sparse regular languages.

\begin{proposition}
 %The sparse regular languages are closed under morphisms.%homomorphic mappings.
 The class of sparse regular languages is closed under homomorphisms.
\end{proposition}

\begin{toappendix}
Note that sparse languages in general are not closed
under homomorphic mappings~\cite{Pin2020}.
As it is easy to see that the bounded languages are closed
under homomorphic mappings, this also implies that, in general,
the bounded languages do not equal the sparse languages.
\end{toappendix}


Note that the connection of the polycyclic languages to the sparse or bounded languages
was not noted in~\cite{DBLP:conf/ictcs/Hoffmann20}. However, a condition
characterizing the sparse regular languages
in terms of forbidden patterns was given in~\cite{Pin2020}, and
it was remarked that ``a minimal deterministic automaton recognises a sparse language if and only if it
does not contain two cycles reachable from one another''.
This is quite close to our characterization.
%and, probably, the author
%has had a similar intuition as spelled out with our explicit definition
%of a polycyclic automaton.
