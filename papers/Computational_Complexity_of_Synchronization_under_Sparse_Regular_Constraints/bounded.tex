
First, in Subsection~\ref{sec:constrained_sync_strictly_bounded},
we state a dichotomy
theorem for the computational complexity of $L\textsc{-Constr-Sync}$
when $L \subseteq \Sigma^*$ is  strictly bounded and regular.
%As the form in Theorem~\ref{thm:bounded_regular_form_unitary}
%and the stated criteria could be computed and checked,
In Subsection~\ref{sec:decision_procedure},
we derive a decision procedure for the computational complexity
in case of strictly bounded regular constraint languages,
if a partial automaton for the constraint language is given as input.
Finally, in Subsection~\ref{subsec:strongly_self_sync},
we will introduce strongly self-synchronizing codes, give an easy method
to construct such codes, and state a complete classification
for bounded regular languages induced by these codes.

\subsection{Synchronization with Strictly Bounded Regular Constraints}
\label{sec:constrained_sync_strictly_bounded}

By Theorem~\ref{thm:bounded_in_NP}, for strictly bounded languages, the constrained
synchronization problem is always in $\NP$. As a first step toward
our main theorem of this subsection, Theorem~\ref{thm:dichotomy},
let us formulate 
a criterion yielding $\NP$-hardness. Later, in Proposition~\ref{prop:stricly_bounded_P},
we state a sufficient condition for polynomial time solvability. 
Both criteria cover all cases, and together
with Theorem~\ref{thm:bounded_regular_form_unitary}, we derive
the final result Theorem~\ref{thm:dichotomy}.

\begin{proposition}
\label{prop:stricly_bounded_np_hard}
 Let $L \subseteq a_1^* \cdots a_k^*$ be a strictly bounded regular
 language. Then, the problem
 $L\textsc{-Constr-Sync}$ is $\NP$-hard
 if we can write 
 \[ 
  L = \bigcup_{i=1}^n A_1^{(i)} \cdots A_k^{(i)}
 \]
 with regular and non-empty $A_j^{(i)} \subseteq a_j^*$
 for $j \in \{1,\ldots, k\}$ and $i \in \{1,\ldots, n\}$,
 recognizable by automata with a single final state,
 such that there exist $i_0 \in \{1,\ldots,n\}$ and $1 \le j_1 < j_2 < j_3 \le k$ % todo i_0 ordentlich hinschreiben
 with $A_{j_2}^{(i_0)}$ infinite and $A_{j_1}^{(i_0)}$, $A_{j_3}^{(i_0)}$
 both do not equal $\{\varepsilon\}$. %Otherwise, it is in $\PTIME$.
 This even holds true when the input is restricted to semi-automata with a sink state.
\end{proposition}
\begin{proof}
 We construct a reduction from an instance
 of $\textsc{DisjointSetTransporter}$
 for unary input automata, which is $\NP$-complete in
 this case, by Theorem~\ref{prop:set_transporter_np_complete},
 to $L\textsc{-Constr-Sync}$ for $L$ as written in the statement\footnote{Note that the problem $\textsc{DisjointSetTransporter}$ is over a unary alphabet, but for $L\textsc{-Constr-Sync}$
 we have $|\Sigma| > 1$. Indeed, we need the additional letters in $\Sigma$ in our reduction.}
Let $L$ and $A_{j_1}^{(i_0)}, A_{j_2}^{(i_0)}, A_{j_3}^{(i_0)}$
as in the statement. 
Without loss of generality, we can assume $j_1 = 1$, $j_2 = 2$
and $j_3 = 3$; otherwise all arguments must be written with these general indices, which
only makes the notation more cumbersome, but the arguments essentially remain identical.
Also, as a shorthand,
we will omit the upper index, i.e.,
set $A_1 = A_1^{(i_0)}$, $A_2 = A_2^{(i_0)}$ and $A_3^{(i_0)} = A_3$
in the following arguments.
 % todo, ginsburg dahingehend umformulieren dass man diese unitären teile voraussetzen kann
As $A_1, A_2, A_3$ are unary regular languages recognizable by 
unary automata with a single final state, we can write, by Lemma~\ref{lem::unary_single_final},
\[
    A_1 = a_1^{r_1}(a_1^{p_1})^*, \quad 
    A_2 = a_2^{r_2}(a_2^{p_2})^*, \quad
    A_3 = a_3^{r_3}(a_3^{p_3})^*
\]
with, by the other assumptions, $p_2 > 0$ ($A_2$ infinite)
and $r_1 + p_1 > 0$, $r_3 + p_3 > 0$ ($A_1$, $A_3$ non-empty and do not equal $\{\varepsilon\}$).
% todo hinschreiben, wenn ncihts gesagt immer complete und deterministic
Let $\mathcal A = (\{a\}, Q, \delta)$ be a semi-automaton and $(\mathcal A, S, T)$
be an instance of \textsc{DisjointSetTransporter}.
We can suppose that $S$ and $T$ are non-empty, otherwise for $S = \emptyset$
it is solvable, and if $T = \emptyset$ we have no solution.
Construct a semi-automaton $\mathcal A' = (\Sigma, Q', \delta')$
by setting
\[
 Q' = S_{r_2} \cup \ldots \cup S_{1} \cup Q \cup Q_1 \cup \ldots \cup Q_{p_2-1} \cup \{ t \}. 
\]
where $t$ is a new state, $S_i = \{ s_i \mid s \in S \}$ for $i \in \{1,\ldots, r_2 \}$
are pairwise disjoint copies of $S$
and $Q_i = \{ q^i \mid q \in Q \}$ are\footnote{Observe
that by the indices a correspondence between these sets
is implied. I wrote the index
in $Q_i$ at the top to distinguish, for $s \in S$ and $i \in \{1,\ldots,\min\{r_2, p_2-1\}\}$, between
 $s_i \in S_i$ and $s^i \in Q_i$. Hence, for each $s \in S$ and $i \in \{1,\ldots, r_2\}$,
 the states $s$ and $s_i$ correspond to each other, and for $q \in Q$
 and $i \in \{1,\ldots, p_2-1\}$ the states $q$ and $q^i$.} 
 also pairwise disjoint 
copies of $Q$. Note that also $S_i \cap Q_j = \emptyset$
for $i \in \{1,\ldots, r_2 \}$ and $j \in \{1,\ldots, p_2-1\}$.
Furthermore, set $S_0 = S$ and $Q_0 = Q$ as a shorthand.
Choose any $\hat s \in S_{r_2}$, then, for $q \in Q$ and $x \in \Sigma$, the transition function is given by
\[
 \delta'(q, x) = \left\{
 \begin{array}{ll}
  s_{i-1} & \mbox{if } x = a_2 \mbox{ and } q = s_i \in S_i \mbox{ for some } i \in \{1,\ldots, r_2\}; \\ 
  \hat s & \mbox{if } x = a_1 \mbox{ and } q \in (Q \cup Q_1 \cup \ldots \cup Q_{p_2-1}) \setminus S; \\
  s_{r_2} & \mbox{if } x= a_1 \mbox{ and } q = s_i \in S_i \mbox{ for some } i \in \{0,\ldots,r_2\}; \\
  t       & \mbox{if } x = a_3 \mbox{ and } q \in T; \\
  q^{p_2-1} & \mbox{if } x = a_2 \mbox{ and } q \in Q; \\
  q^{i-1} & \mbox{if } x = a_2 \mbox{ and } q = q^i \in Q_i \mbox{ for some } i \in \{1,\ldots,p_2-1\}; \\
  q       & \mbox{otherwise}.
 \end{array}
 \right.
\]

% todo aufpassen dass r_1 und r_2 bzw p_1 und p_2 nicht verwechselt
% mit a auf S_{r_1}, dann die b's durchlesen, bis auf S, dann durchlaufen mit den Q_i's


\newcommand{\automatacloudother}[2][.44]{%
	\begin{scope}[#2]
		\node [rectangle,draw,thick,text width=8.1cm,minimum height=7.6cm,
		text centered,rounded corners, fill=white, name = re] {};
\end{scope}}    


\newcommand{\innerstateloop}{
\begin{scope}
  \node[state] (s1) at (0,0) {}; \node (s1label) at (0.5,-.1) {$\in Q$};
  \node[state] (s11) at (-0.5,0.6) {};  \node (s11label) at (0.32,0.6) {$\in Q_{p_2-1}$};
  %\node[state] (s14) at ( 0.5,0.6);
  \node (s12) at (-0.25,1.2) {};
  \node (s13) at ( 0.25,1.2) {};
  \path[->] (s1)  edge [bend left] node [left] {$b$} (s11);
%  \path[->] (s14) edge [bend left] node [right] {$b$} (s1);
   \path[->] (s11) edge [bend left] node [left] {$b$} (s12);
%   \path[->] (s13) edge [bend left] node [right] {$b$} (s14);
  \draw[dashed] (-0.25,1.2) -- (0.25,1.2);
\end{scope}
}

\begin{figure}[htb]
     \centering
    \scalebox{.65}{    
 \begin{tikzpicture}
 \tikzset{every state/.style={minimum size=1pt},>=stealth'}
 \node (cloud) at (0,0) {\tikz \automatacloudother{fill=gray!0,thick};};
 
  \node (reset1) at (0,3) {};
  \node (reset2) at (-10,2.5) {};
  \path[->] (reset1) edge [bend right] node [above] {$a$} (reset2);
  
  \node (reset3) at (0,-3) {};
  \node (reset4) at (-10,-2.5) {};
  \path[->] (reset3) edge [bend left] node [below] {$a$} (reset4);
  
  \node (reset5) at (-5.4,2.8) {};
  \node (reset6) at (-10,2.5) {};
  \path[->] (reset5) edge [bend right] node [above,pos=.3] {$a$} (reset6);
  
  \node (reset7) at (-5.4,-2.8) {};
  \node (reset8) at (-10,-2.5) {};
  \path[->] (reset7) edge [bend left] node [below,pos=.3] {$a$} (reset8);
  
 
  %\draw (-3,0) ellipse (1.1cm and 3.1cm);
  %\draw (-5.5,0) ellipse (1.1cm and 3.1cm);
  %\draw (-10,0) ellipse (1.1cm and 3.1cm);
  \draw[rounded corners] (-4.1,-3) rectangle (-1.7, 3) {};
  \draw[rounded corners] (-6.3,-3) rectangle (-4.8, 3) {};
  \draw[rounded corners] (-10.5,-3) rectangle (-9, 3) {};
    
  %\draw (3,0) ellipse (1cm and 3cm);
  \draw[rounded corners] (1.7,-3) rectangle (4.1, 3) {};
  
  \node[state] (t) at (7,0) {$t$};
  
  \node at (3,3.4) {{\LARGE $T$}};
  \node at (-3,3.4) {{\LARGE $S$}};
  \node at (-10,3.5) {{\LARGE $S_{r_2}$}};
  \node at (-5.5,3.5) {{\LARGE $S_{1}$}};
  \node at (0.1,4.1) {{\LARGE Original $\mathcal A$ (altered)}};
   % (altered for new alphabet)
  
  \node (s1) at (-.5,2) {\tikz \innerstateloop;};
  \node (s2) at (.5,-1.5) {\tikz \innerstateloop;};
  %\node (s2) at (-1.2,-2) {\tikz \innerstateloop;};
  
  \node (sT1) at (3,1.9) {\tikz \innerstateloop;}; \node (sT1copy) at (3,1.5) {};
  \node (sT2) at (2.8,0.4) {\tikz \innerstateloop;}; \node (sT2copy) at (2.8,0) {};
  \node (sT3) at (3.1,-1.5) {\tikz \innerstateloop;}; \node (sT3copy) at (3.15,-2) {};
  
  \node (sS1) at (-3,1.7) {\tikz \innerstateloop;}; \node (sS1copy) at (-2.95,1.15) {};
  \node (sS2) at (-2.7,0) {\tikz \innerstateloop;}; \node (sS2copy) at (-2.65,-.55) {};
  \node (sS3) at (-3,-1.8) {\tikz \innerstateloop;};\node (sS3copy) at (-3,-2.3) {};
   
  \node[state] (sS11) at (-5.5,1.7) {};
  \node[state] (sS12) at (-5.2,0) {};
  \node[state] (sS13) at (-5.7,-2) {};
  
  \node[state] (sSr1) at (-10,1.7) {};
  \node[state] (sSr2) at (-9.7,0) {};
  \node[state] (sSr3) at (-10.1,-2) {};
  
  \path[->] (t) edge [loop right] node {$\Sigma$} (t);
  
  % nichteinzeichnen, self loop implied
  \path[->] (sT1copy) edge [bend left=35] node [above] {$a$} (t)
            (sT2copy) edge [bend left=10] node [above] {$a$} (t)
            (sT3copy) edge node [above] {$a$} (t);
            
            
  \node (sSr1a) at (-8.5,1.7) {};
  \node (sSr2a) at (-8.2,0) {};
  \node (sSr3a) at (-8.7,-2) {};
  
  \node (sSr1b) at (-7.2,1.7) {};
  \node (sSr2b) at (-6.9,0) {};
  \node (sSr3b) at (-7.4,-2) {};
  
  \path[->] (sSr1) edge node [above,pos=.3] {$b$} (sSr1a);
  \path[->] (sSr2) edge node [above,pos=.27] {$b$} (sSr2a);
  \path[->] (sSr3) edge node [above] {$b$} (sSr3a);
 
  \path[->] (sSr1b) edge node [above,pos=.35] {$b$} (sS11);
  \path[->] (sSr2b) edge node [above] {$b$} (sS12);
  \path[->] (sSr3b) edge node [above] {$b$} (sS13);
  
  \path[->] (sS11) edge [bend right=10] node [above,pos=.38] {$b$} (sS1copy);
  \path[->] (sS12) edge [bend right=10] node [above,pos=.2] {$b$} (sS2copy);
  \path[->] (sS13) edge [bend right=10] node [above,pos=.4] {$b$} (sS3copy);
  
  \draw[dashed] (-8.5,1.7) -- (-7.2,1.7);
  \draw[dashed] (-8.2,0) -- (-6.9,0);
  \draw[dashed] (-8.7,-2) -- (-7.4,-2);
 \end{tikzpicture}}
  \caption{%Schematic illustration of the reduction from the proof of Proposition \ref{prop:stricly_bounded_np_hard}.
   The reduction from the proof sketch sketch of Lemma~\ref{lem:np_hardness}.
   The letter $a$ transfers everything surjectively onto $S_{r_2}$,
   indicated by four large arrows at the top and bottom and labelled 
   by $a$.
   The auxiliary states $Q_1, \ldots, Q_{p_2-1}$, which are meant
   to interpret a sequence $b^{p_2}$ like a single symbol in the original
   automaton, are also only indicated inside of $\mathcal A$, but not fully written out.}
  \label{fig:reduction_np_hard}
\end{figure}



Please see Figure~\ref{fig:reduction_np_hard} for a sketch
of the reduction.
Before showing that this gives a valid reduction, let us show the following property
of $\mathcal A'$.

\medskip

 % auch ohne einschränkung, nur reihenfolge wichtig... a_1^* a_2^* a_3^* ...
\noindent\underline{Claim:} %If $\mathcal A'$ has a synchronizing word
 For the constructed automaton $\mathcal A'$ we have:
\begin{align*} % a_1 kann weggelassen werden?
    \exists m \ge 0 : \delta(S, a^m) \subseteq T %& \Leftrightarrow \mathcal A'\mbox{ is synchronizing.} \\ 
    % % Nicht ganz, man könnte ja synchornisieren, indem man nach T mappt, dann auf sync
 % usw für jeden zustand einzeln.
                               & \Leftrightarrow \mathcal A'\mbox{ has a synchronizing word in $a_1a_2^{r_2}(a_2^{p_2})^*a_3$.} \\
                               & \Leftrightarrow \mathcal A'\mbox{ has a synchronizing word in $a_1a_2^*a_3$.} \\
                               & \Leftrightarrow \mathcal A'\mbox{ has a synchronizing word in $a_1^*a_2^*a_3^*(\Sigma \setminus \{a_1, a_2, a_3\})^*$.}
\end{align*} % erste gilt nicht unbedingt, da man ja in T "reinschaufeln" kann
%
% mit letzter, wenn in L, dann in letztem teil, also gibt es in A wie gesucht.
%
% wenn in A, zweiter teil, vorne und hinten was dranhängen wie gebraucht, um es in L zu realisieren.
\begin{quote}
%\begin{proof}[Proof of Claim]
 \emph{Proof of the claim.}
\begin{enumerate}
\item First, suppose $\delta(S, a^m) \subseteq T$.

 %Choose any $a_1^r \in A_1$ with $r > 0$
 %and $a_3^s \in A_3$ with $s > 0$. 
 By construction of $\mathcal A'$,
 for any $q, q' \in Q$,
 \begin{equation}\label{eqn:transition_Astar}
  \delta(q, a) = q'  \mbox{ in $\mathcal A$}  \Leftrightarrow \delta'(q, a_2^{p_2}) = q'  \mbox{ in $\mathcal A'$} .
 \end{equation}
 Also, $\delta'(a_1, Q'\setminus\{t\}) = S_{r_2}$
 and $\delta'(S_{r_2}, a_2^{r_2}) = S$.
 Combining these facts, we find
 \[
  \delta'(Q', a_1a_2^{r_2}a_2^{p_2m}) \subseteq T \cup \{t\}. 
 \]
 A final application of $a_3$ then maps
 all states in $T$ to the single sink %(and synchronizing) 
 state $t$.
 
\item Clearly, as $a_1a_2^{r_2}(a_2^{p_2})^*a_3 \subseteq a_1 a_2^* a_3$.

\item Clearly, as $a_1 a_2^* a_3 \subseteq a_1 a_2^* a_3(\Sigma\setminus\{a_1,a_2,a_3\})^*$.

\item Let $u = a_1^{p} a_2^q a_3^r v$ be a synchronizing word, $p,q,r \ge 0$, $v \in (\Sigma\setminus \{a_1,a_2,a_3\})^*$.

 \medskip

 Then, as $t$ is a sink state, $\delta'(Q', u) = \{t\}$.
 The only way to enter $t$ from the outside is to read $a_3$ at least once, and 
 as $t$ is a sink state, we have $\delta'(Q', a_1^p a_2^q a_3^r) = \{t\}$.
 Also, as for $q \notin T$, we have $\delta(q, a_3) = q$,
 we must have $\delta'(Q', a_1^p a_2^q) \subseteq T \cup \{t\}$,
 or more specifically, $\delta'(Q' \setminus \{t\}, a_1^p a_2^q) \subseteq T$.
 We distinguish two cases for~$p$.
 
 \begin{enumerate}
 \item If $p = 0$, then, in particular, $\delta'(S, a_2^q)  \subseteq T$.
     By construction of $\mathcal A'$, for any $q \in Q$,
 \[
  \delta'(q, a_2^n) \in Q
 \]
 if and only if $n \equiv 0\pmod{p_2}$.
 So, $q = p_2 m$ for some $m \ge 0$.
 Hence, by Equation~\eqref{eqn:transition_Astar} above from the first case, in $\mathcal A$,
 we find $\delta(S, a_2^m) \subseteq T$.
 
 \item  If $p > 0$, then $\delta'(Q' \setminus\{t\}, a_1^p) = S_{r_2}$.
 The only way to leave any state in $S_{r_2}$
 is to read $a_2$, which transfers $S_{r_2}$ to $S_{r_2-1}$.
 Reasoning similarly, we find that we have to read in $a_2$
 at least $r_2$ many times, which finally maps $S_{r_2}$
 onto $S_0 = S$. So, $q \ge r_2$. By construction of $\mathcal A'$, for any $q \in Q$,
 \[
  \delta'(q, a_2^n) \in Q
 \]
 if and only if $n \equiv 0\pmod{p_2}$.
 So, as $\delta'(S, a_2^{q - r_2}) \subseteq T$, $q - r_2 = p_2 m$ for some $m \ge 0$.
 Hence, by Equation~\eqref{eqn:transition_Astar} above, in $\mathcal A$,
 we find $\delta(S, a_2^m) \subseteq T$.
 \end{enumerate}
 
 

\end{enumerate}
This ends the proof of the claim. \emph{[End, proof of the claim.]}
%\end{proof}
\end{quote}

Finally, we show that we have some $m \ge 0$
such that $\delta(S, a^m) \subseteq T$
if and only if $\mathcal A'$ has a synchronizing word in $L$.

\begin{enumerate}
\item Suppose $\delta(s, a^m) \subseteq T$ for some $m \ge 0$.

 By the above claim, $\mathcal A'$ 
 has a synchronizing word $u$ in $a_1 a_2^* a_3$.
 As $A_1$ is non-empty
 and does not equal $\{\varepsilon\}$, we can choose $a_1^p \in A_1$ with $p > 0$.
 Similarly, choose $a_3^r \in A_3$ and any $v \in A_4 \cdots A_k$.
 Then
 \[
  a_1^{p-1} u a_3^{r-1} v \in A_1 A_2 A_3 A_4 \cdots A_k \subseteq L,
 \]
 and by Lemma~\ref{lem:append_sync}, this word also synchronizes $\mathcal A'$.

\item Suppose we have a synchronizing word $w \in L$
 for $\mathcal A'$.
 As $L \subseteq a_1^* a_2^* \cdots a_k^* \subseteq a_1^* a_2^* a_3^* (\Sigma \setminus\{a_1,a_2,a_3\})^*$,
 by the above claim,
 $\delta(S, a^m) \subseteq T$
 for some $m \ge 0$.
\end{enumerate}
So, our reduction is valid and this proves $\NP$-hardness
in this case.
\end{proof}

 
\begin{example}
 For the following languages $L \subseteq \{a,b,c,d\}^*$, the 
 constrained synchronization problem
 is $\NP$-complete.
 \begin{itemize}
 \item $L = ab^*c$,
 \item $L = aa(aaa)^*bbb^*d \cup a^*b \cup d^*$,
 \item $L = bbcc^*d^* \cup a$.
 \end{itemize}
 \end{example}
 
 Next, we exhibit a criterion that yields polynomial time solvable 
 problems.
 Together with Lemma~\ref{lem:union}, this will be
 used to derive our dichotomy result.
 
 
 % P, einzelteile der vereinigung 
\begin{propositionrep}
\label{prop:stricly_bounded_P}
 Suppose $\Sigma = \{a_1, \ldots, a_k\}$.
 Let $A_j \subseteq a_j^*$ be non-empty unary regular languages, recognized
 by automata with a single final state, for $j \in \{1,\ldots, k\}$.
 Set $L = A_1 \cdot\ldots\cdot A_k$.
 %
 % vorne |Q|, dann die möglcihkeiten in der mitte probieren, dann am ende unäre |Su| = 1, also hoch |Q|, dann in zyklus 
 % und gucken ob einer.
 %
 If for all $j \in \{1,\ldots, k\}$, $A_j$ infinite implies that either $A_i = \{\varepsilon\}$
 for all $i < j$ or $A_i = \{\varepsilon\}$ for all $i > j$ (or both), then $L\textsc{-Constr-Sync} \in \PTIME$. 
 Intuitively, if no infinite unary language lies strictly in the middle between 
 non-empty unary languages different from $\{\varepsilon\}$,
 then the problem is in $\PTIME$.
\end{propositionrep} 
\begin{proof}
 If all $A_j$, $j \in \{1,\ldots,k\}$ are finite, then $L$ is finite, and $L\textsc{-Constr-Sync}\in \PTIME$
 by Lemma~\ref{lem:finite}.
 Observe that the condition $A_i = \{\varepsilon\}$
 essentially means this letter is not allowed and could henceforth be ignored.
 So, if for example $A_i = \{\varepsilon\}$
 for all $i < j$ with $j \in \{1,\ldots,k\}$,
 then we have the
 smaller alphabet $\{ a_j, \ldots, a_k\}$.
 So, without loss of generality, we can assume
 that either $A_1$ is infinite, or $A_k$ is infinite,
 or both, to cover all cases. Note that the stated condition
 implies that at most one $A_j$, $j \in \{1,\ldots,k\}$
 is infinite.
 We handle each case separately.
 
 \begin{enumerate}
 \item[(i)] Only $A_1$ is infinite.
  
  By assumption, every $A_j \subseteq \{a_j\}$, $j \in \{1,\ldots,k\}$,
  is recognizable by a single state automaton. Hence, by Lemma~\ref{lem::unary_single_final}, we can write, as $A_1$ is infinite,
   $A_1 = a_1^i (a_1^p)^*$ with $i \ge 0$ and $p > 0$.
  Let $\mathcal A = (\Sigma, Q, \delta)$ be an input semi-automaton
  for $L\textsc{-Constr-Sync}$.
  As $\delta(Q, a_1) \subseteq Q$,
  we have, for any $n \ge 0$, $\delta(Q, a_1^{n+1}) \subseteq \delta(Q, a_1^n)$.
  So, as $Q$ is finite and the sequence of subsets
  cannot get arbitarily small, for some $0 \le n < |Q|$
  we have $|\delta(Q, a_1^{n+1})| = |\delta(Q, a_1^n)|$.
  But $|\delta(Q, a_1^{n+1})| = |\delta(Q, a_1^n)|$,
  as $\delta(Q, a_1^{n+1}) \subseteq \delta(Q, a_1^n)$,
  implies $\delta(Q, a_1^{n+1}) = \delta(Q, a_1^n)$.
  Then, the symbol $a_1$
  permutes the set $\delta(Q, a_1^n)$.
  Hence, $\delta(Q, a_1^{n+m}) = \delta(Q, a_1^n)$ for any $m \ge 0$.
  So, combining these observations,
  \begin{equation}\label{eqn:case_one_P}
   \{ \delta(Q, a_1^n) \mid n \ge 0 \} = \{ \delta(Q, a_1^n) \mid n \in \{0,\ldots, |Q|-1\} \}
  \end{equation}
  and $\delta(Q, a_1^{|Q| - 1 + m}) = \delta(Q, a_1^{|Q|-1})$
  for any $m \ge 0$. 
  Now, note that the language $A_2 \cdots A_k$
  is finite. So, to find out if we have any
  $a_1^{i + lp} u$ with $u \in A_2 \cdots A_k$
  that synchronizes the input semi-automaton,
  we only have to test if any of the words
  $
   a_i^{i + lp} u,
  $
  with $u \in A_2 \cdots A_k$
  and $l$ such that $i + lp \le \max\{|Q|-1 + p, i\}$,
  synchronizes $\mathcal A$.
  The number (and the length) of these words is linear bounded in $|Q|$ 
  and each could be checked in polynomial time by 
  feeding it into the input semi-automaton for each state and checking
  if a unique state results.
  Hence the problem is solvable in polynomial time.
   
 \item[(ii)] Only $A_k$ is infinite.
 
  Let $u \in A_1 \cdots A_{k-1}$. By assumption, there are only finitely many such
  words $u$. Set $S = \delta(Q, u)$ and $T = \delta(Q, a_k^{|Q|-1})$.
  As in case (i), $a_k$ permutes the states in $T$
  and as $S \subseteq Q$, we have  $\delta(S, a_k^{|Q|- 1}) \subseteq T$.
  So, as $a_k$ permutes $T$, it acts injective on the
  subset $\delta(S, a_k^{|Q|- 1})$.
  This gives $|\delta(S, a_k^{|Q|- 1 + n})| = |\delta(S, a_k^{|Q|- 1})|$
  for any $n \ge 0$. Together with $|\delta(S, a_k^{n + 1})| \le |\delta(S, a_k^{n})|$,
  we have
  \begin{equation}\label{eqn:case_two_P}
   \exists n \in \mathbb N_0 : |\delta(S, a_k^{n})| = 1 \Leftrightarrow |\delta(S, a_k^{|Q|- 1})| = 1.
  \end{equation}
%   If $|S| = 1$, then $u$ is already a synchronizing word and so any
%   word in $u \cdot A_k$ is also synchronizing by Lemma~\ref{lem:append_sync}.
%   So, suppose $|S| > 1$.
%   If $|\delta(S, a_k^n)| = 1$ for any $n$, then there exists
%   some $m \le |Q| - 1$ such that $|\delta(S, a_k^m)| = 1$
%   and $|\delta(S, a_k^{|Q| - 1})| = |\delta(S, a_k^{|Q| - 1 + n})|$
%   for any\footnote{But here, the sets might not be equal, for example, consider
%   some, but not all, states taken from a cycle for $a_k$} $n \ge 0$.
%   The last property is implied as $\delta(S, a_k^{|Q|- 1}) \subseteq T$,
%   and it implies the former, for if we have not mapped $S$ to a singleton
%   before reading $a_k$ at most $|Q|-1$ times, we will never do behind that point. 
%   Similarly, as in case (i), write $A_k = a_k^i(a_k^p)^*$.
  Choose any fixed $N \ge |Q| - 1$ with $a_k^N \in A_k$.
  Then, with the above considerations, we only have to test the finite
  number of words
  \[
   u\cdot a_k^{N}, \quad u \in A_1 \cdots A_{k-1}.
  \]
  The length of these words is linear bounded in $|Q|$ and 
  as each test, i.e., feeding the word into the input semi-automaton
  for each state and testing if a unique state results,
  could be performed in polynomial time, the problem is solvable in polynomial time.
  
 \item[(iii)] Both $A_1$ and $A_k$ are infinite.
  
  This is essentially a combinations of the arguments of case (i) and (ii).
  Let $\mathcal A = (\Sigma, Q, \delta)$ be an input semi-automaton
  and $\mathcal B = (\Sigma, P, \mu, p_0, F)$
  be a constraint automaton with $L = L(\mathcal B)$.
  First, consider only the language $A_1 \cdots A_{k-1}$.
  Then, as in case (ii), see Equation~\eqref{eqn:case_one_P},% todo argument geht auch für A_1 infinite, ohne dass unitär...
  \[
   \{ \delta(Q, a_1^n) \mid a_1^n \in A_1  \}
    = \{ \delta(Q, a_1^n) \mid 0 \le n < |Q| - 1 + |P|\mbox{ and } a_1^n \in A_1 \}.
  \]
  Note that we have written $0 \le n < |Q| - 1 + |P|$
  and not merely $\le |Q| - 1$ as an upper bound.
  The reason is that otherwise, if $a_1^{|Q|-1} \notin A_1$,
  we might miss the set $\delta(Q, a_1^{|Q|-1})$,
  but as $\delta(Q, a_1^{|Q|-1+m}) = \delta(Q, a_1^{|Q|-1})$
  for any $m \ge 0$ and $A_1$ is infinite, $\delta(Q, a_1^{|Q|-1}) \in \{ \delta(Q, a_1^n) \mid a_1^n \in A_1  \}$.
  However, if $a_1^n \in A_1$ for some $n \ge |Q| - 1 + |P|$,
  then, with $s = \mu(p_0, a_1^{|Q| - 1})$,
  by finiteness of $P$, among
  the states $s, \mu(s,a_1), \ldots, \mu(s, a_1^{n - |Q| + 1})$
  we find $0 \le m \le |P| - 1$ and $0 < r \le |P|$ with $m + r \le |P|$
  such that $\mu(s, a_1^{m+r}) = \mu(s, a_1^m)$.
  Then we have found a cycle and we can skip it, i.e.,
  \begin{align*}
   \mu(p_0, a_1^n) & = \mu(s, a_1^{n - |Q| + 1}) 
                     = \mu(\mu(s, a_1^{m+r}), a_1^{n - |Q| + 1 - (m+r)}) \\
                   & = \mu(\mu(s, a_1^m), a_1^{n - |Q| + 1 - (m+r)}) \\
                   & = \mu(s, a_1^{m + n - |Q| + 1 - m - r}) \\
                   & = \mu(p_0, a_1^{n-r}).
  \end{align*}
  But, as then $\mu(s, a_1^{n - r}) = \mu(s, a_1^n) \in F$
  we find $a_1^{n-r} \in A_1$. 
  Repeating this procedure, if $n - r \ge |Q| - 1 + |P|$,
  we ultimately find $|Q| - 1 \le m < |Q| - 1 + |P|$
  such that $a_1^m \in A_1$
  and $\delta(Q, a_1^{|Q| - 1}) = \delta(Q, a_1^m)$.
  Note that the language $A_2 \cdots A_{k-1}$
  is finite.
  Then, as in case (i), 
  we only have to consider the  words,
  whose length and number is linear bounded in $|Q|$,
  \[
   a_1^n \cdot u,\quad  0 \le n < |Q| - 1 + |P|, a_1^n \in A_1, u \in A_2 \cdots A_{k-1}
  \]
  and the corresponding sets
  \[
   S = \delta(Q, a_1^n \cdot u),
  \]
  and these are all possible sets in $\{ \delta(Q, a_1^n u) \mid a_1^n \in A_1, u \in A_2 \cdots A_{k-1} \}$.
  Fix any such subset $S$.
  Then, as in case (ii) and Equation~\eqref{eqn:case_two_P}, 
  choose any $N \ge |Q| - 1$ with $a_k^N \in A_k$ and
  we only have to compute $\delta(S, a_k^N)$
  and test if it is a singleton set.
  So, in total, we only have to test the words
  \[
   a_1^n \cdot u a_k^N, 0 \le n < |Q| - 1 + |P|, a_1^n \in A_1, u \in A_2 \cdots A_{k-1}.
  \]
  Their length and number is linear bounded in $|Q|$
  and computing the reachable state from each state of the input automaton,
  and testing if a unique state results, could be performed in polynomial
  time. Hence, the overall procedure could be performed in polynomial time.
 \end{enumerate}
 So, we have handled every case and the proof is complete.
 % oder zeigen A_1 oder A_k finite -> nur endlich viele wörter müssen getestet werden
 % mit (iii)
\end{proof}

\begin{example}
 For the following languages $L \subseteq \{a,b,c,d\}^*$, the 
 constrained synchronization problem
 is in $\PTIME$.
 \begin{itemize}
 \item $L = a^5bd \cup cd^4$,
 \item $L = a^5bd \cup cd^*$,
 \item $L = aa^*bbbbcd^* \cup bbbd(dd)^*$.
 \end{itemize}
 \end{example}
 
 
Now, we have everything together to state and prove our main theorem 
of this section.

\begin{theorem}[dichotomy for strictly bounded constraint languages]
\label{thm:dichotomy}
 Let $L \subseteq a_1^* \cdots a_k^*$ be strictly bounded and regular.
 Then, $L\textsc{-Constr-Sync}$ is either $\NP$-complete
 or in $\PTIME$.
\end{theorem}
\begin{proof}
 Let $L \subseteq a_1^* \cdots a_k^*$ be strictly bounded and regular.
 By Theorem~\ref{thm:bounded_regular_form_unitary}, 
 $L$ is a finite union of languages of the form $A_1 \cdots A_k$,
 where each $A_j \subseteq a_j^*$ is a non-empty unary regular language, recognizable
 by an automaton with a single final state.
 If any of these languages has the form
 stated in Proposition~\ref{prop:stricly_bounded_np_hard},
 i.e., a letter could appear infinitely often between two letters that
 are allowed to appear at least once, then the problem is $\NP$-hard.
 By Theorem~\ref{thm:bounded_in_NP} it is in $\NP$, hence $\NP$-complete.
 Otherwise, if a language $A_1 \cdots A_k$ does not fulfill
 the property stated in Proposition~\ref{prop:stricly_bounded_np_hard},
 then it fulfills the condition of Proposition~\ref{prop:stricly_bounded_P}.
 So, if none of these languages fulfill the condition
 stated in Proposition~\ref{prop:stricly_bounded_np_hard},
 they all give polynomial time solvable constrained problems
 by Proposition~\ref{prop:stricly_bounded_P}.
 By Lemma~\ref{lem:union}, the whole problem $L\textsc{-Constr-Sync}$
 is polynomial time solvable in this case.
\end{proof}

 % oder trichotomy ergebnis nutzen? kommutatives L für NP-vollständig a,b,c; b unendlich.
 % a \shuffle b^* \shuffle c

 % auch noch decision procedure, dafür vielleicht shallit nutzen,dass poly-growth entscheidbar,
 % oder selbst minimalautomatform bestimmen

 % 2-sate case Simga_{2,1} = empyset, also L = Sigma_11^* Sigma_12 Sigma_{21}^* nur
 % mit homs auf (a+b)^*c usw abbilden um pspace-härt zu erhalten.
  % und für anderes obiges nutzen.

 %
 % offensichtliches verfahren um zu entscheien ob strictly bounded
 %  mit allen permuatioenn (a_1* ... a_k^*)^C schneiden und gucken ob leer
 % für eins.
 %
 % oder min-aut characterisieren [a_{i-1} teil usw]
 % aber brauche ich ja nicht, annahme ist strictly bounded, dann einfach
 % die zerlegung bestimmten aus teilen von finalzustand.
 
\subsection{Deciding the Computational Complexity of the Constrained
Synchronization Problem}
\label{sec:decision_procedure}
 
 In this section, we will show that, given a PDFA recognizing a strictly bounded
 constraint language, we can decide, in polynomial time, if
 the constrained synchronization problem is in $\PTIME$
 or $\NP$-complete.
\begin{comment}
 In Proposition~\ref{prop:stricly_bounded_np_hard}
 and in Proposition~\ref{prop:stricly_bounded_P}, it was assumed
 that the unary languages of the form $A_j^{(i)} \subseteq \{a_j\}^*$
 for $j \in \{1,\ldots, k\}$ and $i \in \{1,\ldots,n\}$
 where recognizable by automata with a single final state.
 In the $\NP$-hardness reduction, this assumption was actually used,
 but Proposition~\ref{prop:stricly_bounded_P} could also  
 reformulated and proved without it. In the decision procedure,
 we do not bother to compute such a form, but only any form like
 in Theorem~\ref{thm:bounded_regular_form} for strictly bounded languages.
 This is justified by the next result.
 
 \begin{lemma} % todo, dann vielleicht komplett weglassen? und in props direkt zeigen
 \label{lem:form_unitary_or_not}
  Let $\Sigma = \{a_1, \ldots, a_k\}$ be an alphabet of size $k$
  and $L \subseteq a_1^* \cdots a_k^*$.
  If we can write 
  \[
   L = \bigcup_{i=1}^n A_1^{(i)} \cdots A_k^{(i)}
  \]
  with regular, non-empty $A_j^{(i)} \subseteq \{a_j\}^*$
  for $j \in \{1,\ldots,k\}$ and $i \in \{1,\ldots,n\}$
  such that there exists ... prop
  
  then, we can also write in as aunion with single state
  
  The same holds for the condition formulated in Proposition~\ref{prop:stricly_bounded_P}.
 \end{lemma}
\begin{proof}
 Notation as in the statement.
 By Lemma~\ref{lem::unary_single_final}, we can write each
 $A_j^{(i)}$ as a finite union of regular languages recognizable
 by an automaton with a single final state.
 So, if any $A_j^{(i)}$ is infinite, at least one part of this
 union must also be finite, and the same if $A_j^{(i)} \ne  \{\varepsilon\}$.
 Then, rewriting by using that concatenation
 distributes over union, we have a form fulfilling the condition
 of Proposition~\ref{prop:stricly_bounded_np_hard}.
 The same reasoning applies to the condition
 formulated in Proposition~\ref{prop:stricly_bounded_P}.
\end{proof}
 
 %
 % lemma aut acceptiert stricly bounded, { delta() = q}
 % hat die form, zeigen.
 
 Note a subtle issue here, and that is the reason why Lemma~\ref{lem:form_unitary_or_not} 
 actually comes up in this section.
 Given some automaton recognizing a strictly bounded language $L$,
 the form can be readily, and in polynomial time,
 $$
   L = \bigcup_{i=1}^n A_1^{(i)} \cdots A_k^{(i)}
 $$
 computed
\end{comment} 
 
 
\begin{theoremrep}
 Let $\Sigma = \{a_1, \ldots, a_k\}$
 be an alphabet of size $k$.
 For a given PDFA $\mathcal B = (\Sigma, P, \mu, p_0, F)$
 with $L(\mathcal B)\subseteq a_1^* \cdots a_k^*$,
 we can decide in polynomial time
 if $L(\mathcal B)\textsc{-Constr-Sync}$
 is $\NP$-complete or in $\PTIME$.
\end{theoremrep}
\begin{proofsketch}
 Let $\mathcal B = (\Sigma, P, \mu, p_0, F)$
 be a given PDFA with $L(\mathcal B) \subseteq a_1^* \cdots a_k^*$.
 For $1 \le j_1 < j_2 < j_3 \le k$
 set
 $
  L_{j_1, j_2, j_3} = \Sigma^* a_{j_1}^+ \Sigma^* a_{j_2}^{|P|}a_{j_2}^* \Sigma^* a_{j_3}^+ \Sigma^*.
 $
 Then, $L(\mathcal B)\textsc{-Constr-Sync}$
 is $\NP$-complete if and only if
 \[
  L(\mathcal B) \cap \left( \bigcup_{1 \le j_1 < j_2  < j_3 \le k} L_{j_1, j_2, j_3}\right)
  \ne \emptyset,
 \]
 which could be checked in polynomial time.
\end{proofsketch}
\begin{proof}
 Let $\mathcal B = (\Sigma, P, \mu, p_0, F)$
 be a given PDFA with $L(\mathcal B) \subseteq a_1^* \cdots a_k^*$.
 For $1 \le j_1 < j_2 < j_3 \le k$
 set\footnote{Note that, as we are only concerned with strictly bounded languages,
  $ % + notation einführen
  a_1^* \cdots a_{j_1-1}^* a_{j_1}^+ a_{j_1+1}^*\cdots a_{j_2-1}^* a_{j_2}^{|P|}a_{j_2}^* a_{j_2+1}^* \cdots a_{j_3-1}^*a_{j_3}^+ a_{j_3+1}^* \cdots a_k^*
 $
 would work as well for $L_{j_1, j_2, j_3}$ in the arguments to come.}
 $
  L_{j_1, j_2, j_3} = \Sigma^* a_{j_1}^+ \Sigma^* a_{j_2}^{|P|}a_{j_2}^* \Sigma^* a_{j_3}^+ \Sigma^*.
 $
 %Then $L(\mathcal B) \cap L_{j_1, j_2, j_3}$
 %if and only if $L(\mathcal B)$ contains 
\begin{comment}
 \medskip 
 
 \noindent\underline{Claim:} We have $L(\mathcal B) \cap L_{j_1, j_2, j_3} \ne \emptyset$
 if and only if 
 %$\{ w \in L(\mathcal B) \mid |w|_{a_{j_1}} > 0, |w|_{a_{j_3}} > 0 \mbox{ and }
 % \forall N \exists$
 $$
  \forall N \exists w \in L(\mathcal B) : |w|_{a_{j_1}} > 0, |w|_{a_{j_3}} > 0, |w|_{a_j2} \ge N.
 $$
 \begin{quote}
     \emph{Proof of the claim.}
      First, suppose $L(\mathcal B) \cap L_{j_1, j_2, j_3} \ne \emptyset$.
      Then, we find $w \in L(\mathcal B)$
      such that $|w|_{a_{j_1}} > 0$, $|w|_{a_{j_3}} > 0$
      and $|w|_{a_{j_2}} \ge |P|$.
      Write $w = ua_{j_2}^{|P|}v$ for some $u,v \in \Sigma^*$.
      By the pigenhole principle, as $w$ is read in $\mathcal B$,
      it has to traverse some state twice as it reads 
      the factor $a_{j_2}^{|P|}$, so we can pump
      some non-empty factor $a_{j_2}^p$ of it with $0 < p \le |P|$.
      Hence,
      \[
       ua_{j_2}^{|P| + ip}v \subseteq L(\mathcal B)
      \]
      for all $i \ge 0$. In particular, this yields that the stated formula is valid.
      
      Conversely, assume the formula is valid for $L(\mathcal B)$.
      As $L(\mathcal B) \subseteq a_1^* \cdots a_k^*$,
      if $w \in L(\mathcal B)$ we can write
      \[
       w = a_1^{|w|_{a_1}} \cdots a_k^{|w|_{a_k}}.
      \]
      Set $N = |P|$ and choose any $w \in L(\mathcal B)$, which
      existence is guaranteed by the formula.
      Then, by the above, $w \in L_{j_1, j_2, j_3}$.
     
     [\emph{End, proof of the claim.}]
 \end{quote}
\end{comment}
 By Theorem~\ref{thm:bounded_regular_form_unitary}, we know that we can write
 $
  L(\mathcal B) = \bigcup_{i=1}^n A_1^{(i)} \cdots A_k^{(i)}
 $
 with regular and non-empty $A_j^{(i)} \subseteq a_j^*$, $j \in \{1,\ldots,k\}$, recognizable by an automaton
 with a single final state.
 Note that we do not need to compute this form, but we will show that
 by intersecting $L(\mathcal B)$
 with $L_{j_1, j_2, j_3}$, we can deduce properties
 about the languages $A_1^{(i)} \cdots A_k^{(i)}$
 for $i \in \{1,\ldots,n\}$ that must hold for any union of the 
 this form.
 
 
%  Note that such a form is readily available, and hence computable, %in polynomialtime, 
%  by writing $L(\mathcal B)$ as a union of languages that lead $\mathcal B$
%  into a single final state, for each final state.
% \begin{comment}
%  But a subtle point remains, it is not clear that in this case,
%  the languages $A_j^{(i)}$ for $j \in \{1,\ldots, k\}$
%  and $i \in \{1,\ldots,n\}$ are acceptable by single state automata.
%  However, this is indeed the case.
%  As shown in~\cite{Eilenberg1974}, a language $L \subseteq \Sigma^*$
%  is acceptable by some PDFA with a single state if and only if,
%  for any $u,v \in L$, we have
%  $$
%   \{ w \in \Sigma^* \mid uw \in L \} = \{ w \in \Sigma^* \mid vw \in L \}.
%  $$
%  Now, consider the projections $\pi_{a_j} : \Sigma^* \to a_j^*$ given
%  by $\pi(a_j) = a_j$ and $\pi(a_l) = \varepsilon$ for each $l \ne j$.
%  We have $\pi_{a_j}(A_1^{(i)} \cdots A_k^{(i)}) = A_j^{(i)}$
%  for any $i \in \{1,\ldots, n\}$ and $j \in \{1,\ldots, k\}$.
%  Now, set $L = A_j^{(i)}$, a unary language over $\{a_j\}^*$,
%  and let $u, v \in L$.
%  Suppose $w \in \{a_j\}^*$ has the property $uw \in L$.
%  Hence, we have some $x_1 \in A_1^{(i)} \cdots A_k^{(i)}$
%  such that $\pi_{a_j}(x_1) = uw$.
%  As the letters in $x$ are ordered,
%  we know $x_1 = y_1uwz_1$ with $y_1 \in \{a_1,\ldots,a_{j-1}\}^*$
%  and $z_1 \in \{a_{j+1},\ldots,a_k\}^*$.
%  But by assumption, we also find $x_2 \in A_1^{(i)} \cdots A_k^{(i)}$
%  such that $\pi_{a_j}(x_2) = u$.
%  Now, we use that we consider a set of the form $A_1^{(i)} \cdots A_k^{(i)}$,
%  namely that we can combine the words in the parts $A_j^{(i)}$
%  independently. To be more specific, if
%  we write, using $u,v \in L$,
%  $$
%   x_2 = y_2 u z_2 \mbox{ and }  x_3 = y_3  v z_3
%  $$
%  with $y_2 u z_2, y_3 v z_3 \in A_1^{(i)} \cdots A_k^{(i)}$
%  and $y_2, y_3 \in \{a_1,\ldots, a_{j-1}\}^*$, $z_2, z_3 \in \{a_{j+1},\ldots, a_k\}^*$,
%  then
%  $$
%   y_2 u \in L \mbox{ and }
%   y_3 v \in L.
%  $$
% %  This property is preserved under homomorphic mappings, in particular
% %  under the projections $\pi_{a_j} : \Sigma^* \to a_j^*$ given
% %  by $\pi(a_j) = a_j$ and $\pi(a_l) = \varepsilon$ for each $l \ne j$,
% %  and $\pi_{a_j}(A_1^{(i)} \cdots A_k^{(i)}) = A_j^{(i)}$
% %  for each $i \in \{1,\ldots, n\}$ and $j \in \{1,\ldots, k\}$.
% \end{comment}
% \begin{comment}
%  But a subtle point remains, in general
%  the languages $A_j^{(i)}$ for $j \in \{1,\ldots, k\}$
%  and $i \in \{1,\ldots,n\}$ might not be acceptable by single state automata,
%  % Todo.
%  % Diese Annanme kann aber glaub ich auch weggelassen werden, Reduktion
%  % funktioniert denke ich trotzdem.
%  % Bzw kann ich in der Reduktion voraussetzen nur unendlich, und mir 
%  % dann den teil a_j^i (a_j^p)* rausnehmen.
%  % für das entscheidungsverfahren muss ich ja nur wissen dass es existiert,
%  % aber heisst ja nicht das die reduktion auch P-time konstruierbar sein muss.
%  as is assumed in the formulation of Proposition~\ref{prop:stricly_bounded_np_hard}.
%  However, by taking $\mathcal B$ with the altered singleton final state set
%  for some $A_j^{(i)}$, $j \in \{1,\ldots, k\}$ and $i \in \{1,\ldots,n\}$, 
%  replacing every transition
%  labelled by a letter that does not equal $a_j$
%  by $\varepsilon$, we get a non-deterministic $\varepsilon$-automaton~\cite{HopMotUll2001}
%  accepting $A_j^{(i)}$.
%  Then, we can determinize this automaton, and in the resulting DFA
%  write $A_j^{(i)}$ again as a union over languages that are acceptable
%  by automata with a single final state, and then $L(\mathcal B)$,
%  using that concatenation distributes over union,
%  as a union of concatenations of these languages.
%  But not that, by using non-deterministic automata and determinizing them,
%  in general this procedure does not run polynomial time.
% \end{comment}

%  Note a subtle point, in general
%  the languages $A_j^{(i)}$ for $j \in \{1,\ldots, k\}$
%  and $i \in \{1,\ldots,n\}$ might not be acceptable by single state automata.
%  But in Proposition~\ref{prop:stricly_bounded_np_hard} they are assumed to be acceptable
%  by such automata.
%  But this does not pose any problem for our decision procedure, 
%  for we can write these languages as union of such languages, which is always
%  possible by taking any accepting automaton and only consider
%  those words that lead into a single final state, and write
%  the original language as a union of these languages.
%  We know that, by a simple argument, if we can find 
%  $i_0 \in \{1,\ldots,n\}$ such that $A_{j_1}^{(i_0)}, A_{j_3}^{(i_0)}$ do not equal $\{\varepsilon\}$
%  and $A_{j_2}^{(i_0)}$ is infinite,
%  then we also find languages inside these that have these properties
%  and are acceptable by automata with a single final state.

 \medskip 
 
 \noindent\underline{Claim:} We have $L(\mathcal B) \cap L_{j_1, j_2, j_3} \ne \emptyset$
 if and only if we find $i_0 \in \{1,\ldots,n\}$ such
 that $A_{j_1}^{(i_0)}, A_{j_3}^{(i_0)}$ do not equal $\{\varepsilon\}$
 and $A_{j_2}^{(i_0)}$ is infinite.
 \begin{quote}
   \emph{Proof of the Claim.} First, suppose we have
   some $w \in L(\mathcal B) \cap L_{j_1, j_2, j_3}$.
   %Then, for some $i_0 \in \{1,\ldots, n\}$, we have
   %$w \in A_1^{(i_0)} \cdots A_k^{(i_0)} \cap L_{j_1, j_2, j_3}$.
   As $L(\mathcal B) \subseteq a_1^* \cdots a_k^*$, we have
   \[
    w = a_1^{|w|_{a_1}} \cdots a_k^{|w|_{a_k}}
   \]
   with $|w|_{a_{j_1}} > 0$, $|w|_{a_{j_3}} > 0$
   and $|w|_{a_{j_2}} \ge |P|$.
   By the pigenhole principle, as $w$ is read in $\mathcal B$,
      it has to traverse some state twice as it reads 
      the factor $a_{j_2}^{|w|_{a_{j_2}}}$. So, we can pump
      some non-empty factor $a_{j_2}^p$ of it with $0 < p \le |P|$.
      Hence, writing $w = ua_{j_2}^{|w|_{a_{j_2}}}v$ with $u,v \in \Sigma^*$,
      we have, for any $r \ge 0$,
      \[
             ua_{j_2}^{|w|_{a_{j_2}} + rp}v \subseteq L(\mathcal B).
      \] % set u ,v
   Again, using the pigeonhole principle, as we have a 
   finite union $L(\mathcal B) = \bigcup_{i=1}^n A_1^{(i)} \cdots A_k^{(i)}$, there
   exists $i_0 \in \{1,\ldots, n\}$ such that
   \[
    ua_{j_2}^{|w|_{a_{j_2}} + rp}v \in A_1^{(i_0)} \cdots A_k^{(i_0)}
   \]
   for infinitely many $r \ge 0$. This implies $a_{j_2}^{|w|_{a_{j_2}} + rp} \subseteq A_{j_2}^{(i_0)}$
   for infinitely many $r$. Hence $A_{j_2}^{(i_0)}$
   is infinite. Furthermore, we
   get $a_{j_1}^{|w|_{a_{j_1}}} \in A_{j_1}^{(i_0)}$
   and $a_{j_3}^{|w|_{a_{j_3}}} \in A_{j_3}^{(i_0)}$.
  
  
   Conversely, if we have $i_0 \in \{1,\ldots,n\}$
   such that $A_{j_1}^{(i_0)}$ and $A_{j_3}^{(i_0)}$
   do not equal $\{\varepsilon\}$ and $A_{j_2}^{(i_0)}$
   is infinite, then obviously
   \[
    A_1^{(i_0)} \cdots A_k^{(i_0)} \cap L_{j_1,j_2,j_3} \ne \emptyset 
   \]
   and as $A_1^{(i_0)} \cdots A_k^{(i_0)} \subseteq L(\mathcal B)$
   the claim follows.
   [\emph{End, proof of the Claim.}]
 \end{quote}
 
 
 Finally, given $L(\mathcal B)$, we compute
 \[
  L(\mathcal B) \cap \left( \bigcup_{1 \le j_1 < j_2  < j_3 \le k} L_{j_1, j_2, j_3}\right)
 \]
 which could be done in polynomial time. More precisely,
 we have to intersect $\binom{k}{3}$ languages with $L(\mathcal B)$,
 and each intersection could be done in time $O(|P|^2)$.
 By the above claim, if this intersection is non-empty, the condition of Proposition~\ref{prop:stricly_bounded_np_hard}
 is fulfilled  and, with Theorem~\ref{thm:bounded_in_NP}, $L(\mathcal B)\textsc{-Constr-Sync}$
 is $\NP$-complete.
 If the intersection is empty, then, for any $i \in \{1,\ldots,n\}$,
 the condition of Proposition~\ref{prop:stricly_bounded_P}
 if fulfilled, so by Lemma~\ref{lem:union}
 the problem $L(\mathcal B)\textsc{-Constr-Sync}$ is in $\PTIME$.
\end{proof}