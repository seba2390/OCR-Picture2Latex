

\section{Letter-Bounded Constraint Languages}
\label{sec:strictly_bounded_case}



% % begriff einführen, das man annehemn kann "nachbarn" verschieden
% % dann hom

% % L' = \{ b_1^n ... : a^n b^{n_\} ist das eindeutig?

% \begin{proposition}
%  hom verschieden, eine richtung klar
% \end{proposition}
% \begin{proof} % Gamma = {b1,..,bk}
% Let $\mathcal A = (\Gamma, Q, \delta)$ be an input automaton
% for which we want to know if it has a synchronizing word in $U$. %L' über Gamma
% Set $Q' = Q \times \{ 1, \ldots, k \}$
% and $\delta' : Q' \times \Sigma \to Q'$
% with 
% \[
%  \delta'((q, i), x) = \left\{
%  \begin{array}{ll}
%   (\delta(q, x), j)     & \mbox{if } x = \varphi(b_j); \\
%  % (\delta(q, x), i + 1) & \mbox{if } i < k \mbox{ and } x = \varphi(b_{i+1}); \\
%   (q, i)                & \mbox{otherwise.}
%  \end{array}\right.
% \]
% Then, $\mathcal A' = (\Sigma, Q', \delta')$
% has a synchronizing word in $L$
% if and only if $\mathcal A$ has a synchronizing word in $U$.
% % aber man muss jeweils mind einmal "rübergehen"




% \end{proof}



%  Let $L \subseteq w_1^* \cdots w_n^*$.
%  As $w^* w^* = w^*$, we can suppose that $w_i \ne w_{i+1}$
%  for $i \in \{1,\ldots,n-1\}$, which we will assume for the rest of this section.
%  In particular in the strictly bounded case we assume
%  that consecutive letter are distinct.\todo{wo brauche ich diese annahmen genau?}
 
 
 %In this section, we 
 Fix a constraint automaton $\mathcal B = (\Sigma, P, \mu, p_0, F)$.
 Let $a_1, \ldots, a_k \in \Sigma$ be a sequence of (not necessarily distinct)
 letters.
 In this section, we assume $L(\mathcal B) \subseteq a_1^* \cdots a_k^*$.
 A language which fulfills the above condition
 is called \emph{letter-bounded}.
 Note that the language $ab^*a$ given in the introduction as an example
 %, and in~\cite{DBLP:conf/mfcs/FernauGHHVW19} as the smallest, in terms of recognizing automata,
 %constraint language giving an \NP-complete problem,
 is letter-bounded. In fact, it is the language with the smallest
 recognizing automaton yielding an \NP-complete constrained problem~\cite{DBLP:conf/mfcs/FernauGHHVW19}.
 
 A language such that the $a_i$ are pairwise distinct, i.e., $a_i \ne a_j$
 for $i \ne j$, is called \emph{strictly bounded}.
 The class of strictly bounded languages has been extensively studied~\cite{DBLP:journals/mst/BlattnerC77,DBLP:journals/dam/DassowP99,Ginsburg66,GinsburgSpanier64,GinsburgSpanier66,HerrmannKMW17},
 where in~\cite{Ginsburg66,GinsburgSpanier64,GinsburgSpanier66} no name was introduced for them
 and in~\cite{HerrmannKMW17} they were called strongly bounded.
 %\footnote{The work~\cite{HerrmannKMW17}
 %seems to deviate from the standard terminology in other ways too by calling bounded languages
 %as introduced here word-bounded and refers to letter-bounded simply as bounded languages.}.
 The class of letter-bounded languages properly contains the strictly bounded languages.
 
 \begin{toappendix} 
 Note that the work~\cite{HerrmannKMW17}
 seems to deviate from the standard terminology, for example by calling bounded languages
 as introduced here word-bounded and refers to letter-bounded simply as bounded languages.
 \end{toappendix}
 

 
 \begin{remark}%[Motivation]%[A Motivation for Strictly Bounded Constraint Languages]
 \label{rem:motivation_strictly_bounded}
Let $\Sigma = \{b_1, \ldots, b_r\}$
be an alphabet of size $r$.
%and $\psi : \Sigma^* \to \mathbb N_0^m$
%be the \emph{Parikh morphism}
%given by $\psi(w) = (|w|_{b_1}, \ldots, |w|_{b_k})$
%for $w \in \Sigma^*$.
%Then, for $L \subseteq \Sigma$,
%set $\perm(L) = \{ w \in \Sigma^* \mid \exists u \in L \forall a \in \Sigma : |u|_a = |w|_a \}$,
%the \emph{commutative closure} of $L$.
Then, 
%between the commutative languages over $\Sigma$
%and the strictly bounded languages in $b_1^* \cdots b_k^*$, 
the mappings
\[
\Phi(L) = L \cap b_1^* \cdots b_r^* \mbox{ and }
\perm(L) = \{ w \in \Sigma^* \mid \exists u \in L\  \forall a \in \Sigma : |u|_a = |w|_a \}
\]
for $L \subseteq \Sigma^*$ are mutually inverse and inclusion preserving
between the languages in $b_1^* \cdots b_r^*$ and the commutative languages
in $\Sigma^*$, where a language $L \subseteq \Sigma^*$ is commutative 
if $\perm(L) = L$.
Furthermore, for strictly bounded languages of the form $B_1 \cdots B_r \subseteq b_1^* \cdots b_r^*$
with $B_j \subseteq \{b_j\}^*$, $j \in \{1,\ldots, r\}$, we have
$
 \perm(B_1 \cdots B_r) = B_1 \shuffle \cdots \shuffle B_r,
$
where
$U \shuffle V = \{ u_1 v_1 \cdots u_n v_n \mid u_i, v_i \in \Sigma^*, u_1 \cdots u_n \in U, v_1 \cdots v_n \in  V \}$ for $U, V \subseteq \Sigma^*$.
Hence, $\perm(L)$ is regularity-preserving
for strictly bounded languages.
More specifically, the above correspondence between
the two language classes is regularity-preserving in both directions.
For commutative constraint languages, a classification
of the complexity landscape has been achieved~\cite{DBLP:conf/cocoon/Hoffmann20}.
By the close relationship between commutative and certain strictly
bounded languages, it is natural to tackle
this language class next.
However, as shown in~\cite{DBLP:conf/cocoon/Hoffmann20},
for commutative constraint languages, we can realize $\PSPACE$-complete
problems, but, by Theorem~\ref{thm:sparse_in_NP},
for strictly bounded languages, the constrained problem is always in $\NP$.
However, by the above relations, Theorem~\ref{thm:bounded_regular_form} for languages in $b_1^* \cdots b_r^*$ is equivalent to~\cite[Theorem 5]{DBLP:conf/cocoon/Hoffmann20}, a representation result
for commutative regular languages.
\end{remark}
 
\begin{comment}

Next, we link the representation
of bounded languages given in Theorem~\ref{thm:bounded_regular_form}
to the languages $L_{j_1, j_2, j_3}$ defined in the beginning of
this section.

\begin{lemma}
\label{lem:L_j1j2j3_intersection}
 Let $L(\mathcal B) = \bigcup_{i=1}^n A_1^{(i)} \cdots A_k^{(i)}$
 with unary regular languages $A_j^{(i)} \subseteq \{a_j\}^*$.
 Then,
 $
  L(\mathcal B) \cap L_{j_1, j_2, j_3} \ne \emptyset 
 $
 if and only if we can find $i_0 \in \{1,\ldots,n\}$
 such that $A_{j_1}^{(i_0)}$ and $A_{j_3}^{(i_0)}$
 do not equal~$\{\varepsilon\}$
 and $A_{j_2}^{(i_0)}$ is infinite.
\end{lemma}
\begin{proof}
 First, suppose we have
   some $w \in L(\mathcal B) \cap L_{j_1, j_2, j_3}$.
   %Then, for some $i_0 \in \{1,\ldots, n\}$, we have
   %$w \in A_1^{(i_0)} \cdots A_k^{(i_0)} \cap L_{j_1, j_2, j_3}$.
   As $L(\mathcal B) \subseteq a_1^* \cdots a_k^*$, we have
   \[
    w = a_1^{|w|_{a_1}} \cdots a_k^{|w|_{a_k}}
   \]
   with $|w|_{a_{j_1}} > 0$, $|w|_{a_{j_3}} > 0$
   and $|w|_{a_{j_2}} \ge |P|$.
   By the pigeonhole principle, as $w$ is read in $\mathcal B$,
      it has to traverse some state twice as it reads 
      the factor $a_{j_2}^{|w|_{a_{j_2}}}$. So, we can pump
      some non-empty factor $a_{j_2}^p$ of it with $0 < p \le |P|$.
      Hence, writing $w = ua_{j_2}^{|w|_{a_{j_2}}}v$ with $u,v \in \Sigma^*$,
      we have, for any $r \ge 0$,
      \[
             ua_{j_2}^{|w|_{a_{j_2}} + rp}v \subseteq L(\mathcal B).
      \] % set u ,v
   Again, using the pigeonhole principle, as we have a 
   finite union \[ L(\mathcal B) = \bigcup_{i=1}^n A_1^{(i)} \cdots A_k^{(i)}, \]
   there
   exists $i_0 \in \{1,\ldots, n\}$ such that
   \[
    ua_{j_2}^{|w|_{a_{j_2}} + rp}v \in A_1^{(i_0)} \cdots A_k^{(i_0)}
   \]
   for infinitely many $r \ge 0$. 
   \todo{Ne, das geht nur bei strictly bounded.}
   This implies $a_{j_2}^{|w|_{a_{j_2}} + rp} \subseteq A_{j_2}^{(i_0)}$
   for infinitely many $r$. Hence $A_{j_2}^{(i_0)}$
   is infinite. Furthermore, we
   get $a_{j_1}^{|w|_{a_{j_1}}} \in A_{j_1}^{(i_0)}$
   and $a_{j_3}^{|w|_{a_{j_3}}} \in A_{j_3}^{(i_0)}$.
  
  
   Conversely, if we have $i_0 \in \{1,\ldots,n\}$
   such that $A_{j_1}^{(i_0)}$ and $A_{j_3}^{(i_0)}$
   do not equal $\{\varepsilon\}$ and $A_{j_2}^{(i_0)}$
   is infinite, then obviously
   \[
    A_1^{(i_0)} \cdots A_k^{(i_0)} \cap L_{j_1,j_2,j_3} \ne \emptyset 
   \]
   and as $A_1^{(i_0)} \cdots A_k^{(i_0)} \subseteq L(\mathcal B)$
   the claim follows.
   
   
   \medskip 
   
   % alternativ mit SCCs und Pfaden, zeigen dass die A_j^i durch Automaten mit weniger als
   % |P| Zuständen erkennbar.
   \qed
\end{proof}
\end{comment}

 Our first result says, intuitively,  
 that if in $A_1 \cdots A_k$ with $A_j$ unary and regular,
 if no infinite unary language $A_j$ over $\{a_j\}$ lies %strictly in the middle 
 between 
 non-empty unary languages
 over a distinct letter\footnote{Hence different from $\{\varepsilon\}$, as $\{\varepsilon\} \subseteq \{a\}^*$
 for $a \in \Sigma$.}  than $a_j$, %than the infinite unary language,
 then $(A_1 \cdots A_k)$\textsc{-Constr-Sync} is in~$\PTIME$.
 
%  Later, in Lemma~\ref{lem:np_hardness} and Theorem~\ref{thm:dichotomy}, we will use
%  the languages $L_{j_1, j_2, j_3}$. They
%  allow us to single out which letters appear infinitely
%  often between other letters in~$L(\mathcal B)$, i.e.,
%  express the opposite of the condition mentioned in Proposition~\ref{prop:stricly_bounded_P}.
 
\begin{propositionrep}
\label{prop:stricly_bounded_P}
 Let $A_j \subseteq \{a_j\}^*$ be unary regular languages
 %, recognized
 %by automata with a single final state, 
 for $j \in \{1,\ldots, k\}$.
 Set $L = A_1 \cdot\ldots\cdot A_k$.
 If for all $j \in \{1,\ldots, k\}$, $A_j$ infinite implies that $A_i \subseteq \{a_j\}^*$
 for all $i < j$ or $A_i \subseteq \{a_j\}^*$ for all $i > j$ (or both), then $L\textsc{-Constr-Sync} \in \PTIME$. 
\end{propositionrep} 
\begin{proof}
 Let $L = A_1 \cdots A_k$ with $A_j \subseteq \{a_j\}^*$ fulfill the assumption.
 If $A_j$ is infinite and for all $i < j$ we have $A_i \subseteq \{a_j\}^*$,
 then $A_1 \cdots A_j \subseteq \{a_j\}^*$, and similarly if
 for all $i > j$ we have $A_i \subseteq \{a_j\}^*$.
 So, by considering $(A_1 \cdots A_j) A_{j+1} \cdots A_k$
 or $A_1 \cdots A_{j-1} (A_j \cdots A_k)$, with $j$ maximal in the former case and minimal in the latter,
 without loss of generality, we can assume $j = 1$
 or $j = k$, i.e., we only have the cases $A_1$
 is infinite, $A_k$ is infinite or both are infinite or none is infinite,
 and, by maximality or minimality of $j$,
 in all these cases the languages $A_2, \ldots, A_{k-1}$ are all finite.
 
 
 
 
 Then, by Lemma~\ref{lem:union_single_final_state}, we can write $A_1$ and $A_k$
 as a finite union of unary languages recognizable by automata with a single final state.
 As concatenation distributes over union, if we do this for
 $A_1$ and $A_k$ and rewrite the language using the mentioned distributivity,
 we get a finite union of languages of the form
 \[
  A_1' A_2 \cdots A_{k_1} A_k'
 \]
 where $A_1'$ and $A_k'$ are recognizable by unary automata with a single final state
 and are either finite or infinite. Hence, by Lemma~\ref{lem:union},
 if we can show that the problem is in $\PTIME$ for each such language,
 the result follows. 
 So, without loss of generality, we assume
 from the start that $A_1$ or $A_k$ are recognizable by automata
 with a single final state.
 
 
 If all $A_j$, $j \in \{1,\ldots,k\}$ are finite, then $L$ is finite, and $L\textsc{-Constr-Sync}\in \PTIME$
 by Lemma~\ref{lem:finite}.
 We handle the remaining cases separately.
 
 \begin{enumerate}
 \item[(i)] Only $A_1$ is infinite.
  
  By assumption, every $A_j \subseteq \{a_j\}^*$, $j \in \{1,\ldots,k\}$,
  is recognizable by a single state automaton. Hence, by Lemma~\ref{lem::unary_single_final}, we can write, as $A_1$ is infinite,
   $A_1 = a_1^i (a_1^p)^*$ with $i \ge 0$ and $p > 0$.
  Let $\mathcal A = (\Sigma, Q, \delta)$ be an input semi-automaton
  for $L\textsc{-Constr-Sync}$.
  As $\delta(Q, a_1) \subseteq Q$,
  we have, for any $n \ge 0$, $\delta(Q, a_1^{n+1}) \subseteq \delta(Q, a_1^n)$.
  So, as $Q$ is finite and the sequence of subsets
  cannot get arbitarily small, for some $0 \le n < |Q|$
  we have $|\delta(Q, a_1^{n+1})| = |\delta(Q, a_1^n)|$.
  But $|\delta(Q, a_1^{n+1})| = |\delta(Q, a_1^n)|$,
  as $\delta(Q, a_1^{n+1}) \subseteq \delta(Q, a_1^n)$,
  implies $\delta(Q, a_1^{n+1}) = \delta(Q, a_1^n)$.
  Then, the symbol $a_1$
  permutes the set $\delta(Q, a_1^n)$.
  Hence, $\delta(Q, a_1^{n+m}) = \delta(Q, a_1^n)$ for any $m \ge 0$.
  So, combining these observations,
  \begin{equation}\label{eqn:case_one_P}
   \{ \delta(Q, a_1^n) \mid n \ge 0 \} = \{ \delta(Q, a_1^n) \mid n \in \{0,\ldots, |Q|-1\} \}
  \end{equation}
  and $\delta(Q, a_1^{|Q| - 1 + m}) = \delta(Q, a_1^{|Q|-1})$
  for any $m \ge 0$. 
  Now, note that the language $A_2 \cdots A_k$
  is finite. So, to find out if we have any
  $a_1^{i + lp} u$ with $u \in A_2 \cdots A_k$
  that synchronizes the input semi-automaton,
  we only have to test if any of the words
  $
   a_i^{i + lp} u,
  $
  with $u \in A_2 \cdots A_k$
  and $l$ such that $i + lp \le \max\{|Q|-1 + p, i\}$,
  synchronizes $\mathcal A$.
  The number (and the length) of these words is linear bounded in $|Q|$ 
  and each could be checked in polynomial time by 
  feeding it into the input semi-automaton for each state and checking
  if a unique state results.
  Hence the problem is solvable in polynomial time.
   
 \item[(ii)] Only $A_k$ is infinite.
 
  Let $u \in A_1 \cdots A_{k-1}$. By assumption, there are only finitely many such
  words $u$. Set $S = \delta(Q, u)$ and $T = \delta(Q, a_k^{|Q|-1})$.
  As in case (i), $a_k$ permutes the states in $T$
  and as $S \subseteq Q$, we have  $\delta(S, a_k^{|Q|- 1}) \subseteq T$.
  So, as $a_k$ permutes $T$, it acts injective on the
  subset $\delta(S, a_k^{|Q|- 1})$.
  This gives $|\delta(S, a_k^{|Q|- 1 + n})| = |\delta(S, a_k^{|Q|- 1})|$
  for any $n \ge 0$. Together with $|\delta(S, a_k^{n + 1})| \le |\delta(S, a_k^{n})|$,
  we have
  \begin{equation}\label{eqn:case_two_P}
   \exists n \in \mathbb N_0 : |\delta(S, a_k^{n})| = 1 \Leftrightarrow |\delta(S, a_k^{|Q|- 1})| = 1.
  \end{equation}
%   If $|S| = 1$, then $u$ is already a synchronizing word and so any
%   word in $u \cdot A_k$ is also synchronizing by Lemma~\ref{lem:append_sync}.
%   So, suppose $|S| > 1$.
%   If $|\delta(S, a_k^n)| = 1$ for any $n$, then there exists
%   some $m \le |Q| - 1$ such that $|\delta(S, a_k^m)| = 1$
%   and $|\delta(S, a_k^{|Q| - 1})| = |\delta(S, a_k^{|Q| - 1 + n})|$
%   for any\footnote{But here, the sets might not be equal, for example, consider
%   some, but not all, states taken from a cycle for $a_k$} $n \ge 0$.
%   The last property is implied as $\delta(S, a_k^{|Q|- 1}) \subseteq T$,
%   and it implies the former, for if we have not mapped $S$ to a singleton
%   before reading $a_k$ at most $|Q|-1$ times, we will never do behind that point. 
%   Similarly, as in case (i), write $A_k = a_k^i(a_k^p)^*$.
  Choose any fixed $N \ge |Q| - 1$ with $a_k^N \in A_k$.
  Then, with the above considerations, we only have to test the finite
  number of words
  \[
   u\cdot a_k^{N}, \quad u \in A_1 \cdots A_{k-1}.
  \]
  The length of these words is linear bounded in $|Q|$ and 
  as each test, i.e., feeding the word into the input semi-automaton
  for each state and testing if a unique state results,
  could be performed in polynomial time, the problem is solvable in polynomial time.
  
 \item[(iii)] Both $A_1$ and $A_k$ are infinite.
  
  This is essentially a combination of the arguments of case (i) and (ii).
  Let $\mathcal A = (\Sigma, Q, \delta)$ be an input semi-automaton
  and $\mathcal B = (\Sigma, P, \mu, p_0, F)$
  be a constraint automaton with $L = L(\mathcal B)$.
  First, consider only the language $A_1 \cdots A_{k-1}$.
  Then, as in case (i), see Equation~\eqref{eqn:case_one_P},% todo argument geht auch für A_1 infinite, ohne dass unitär...
  \[
   \{ \delta(Q, a_1^n) \mid a_1^n \in A_1  \}
    = \{ \delta(Q, a_1^n) \mid 0 \le n < |Q| - 1 + |P|\mbox{ and } a_1^n \in A_1 \}.
  \]
  Note that we have written $0 \le n < |Q| - 1 + |P|$
  and not merely $\le |Q| - 1$ as an upper bound.
  The reason is that otherwise, if $a_1^{|Q|-1} \notin A_1$,
  we might miss the set $\delta(Q, a_1^{|Q|-1})$,
  but as $\delta(Q, a_1^{|Q|-1+m}) = \delta(Q, a_1^{|Q|-1})$
  for any $m \ge 0$ and $A_1$ is infinite, $\delta(Q, a_1^{|Q|-1}) \in \{ \delta(Q, a_1^n) \mid a_1^n \in A_1  \}$.
  However, if $a_1^n \in A_1$ for some $n \ge |Q| - 1 + |P|$,
  then, with $s = \mu(p_0, a_1^{|Q| - 1})$,
  by finiteness of $P$, among
  the states $s, \mu(s,a_1), \ldots, \mu(s, a_1^{n - |Q| + 1})$
  we find $0 \le m \le |P| - 1$ and $0 < r \le |P|$ with $m + r \le |P|$
  such that $\mu(s, a_1^{m+r}) = \mu(s, a_1^m)$.
  Then we have found a cycle and we can skip it, i.e.,
  \begin{align*}
   \mu(p_0, a_1^n) & = \mu(s, a_1^{n - |Q| + 1}) 
                     = \mu(\mu(s, a_1^{m+r}), a_1^{n - |Q| + 1 - (m+r)}) \\
                   & = \mu(\mu(s, a_1^m), a_1^{n - |Q| + 1 - (m+r)}) \\
                   & = \mu(s, a_1^{m + n - |Q| + 1 - m - r}) \\
                   & = \mu(p_0, a_1^{n-r}).
  \end{align*}
  But, as then $\mu(s, a_1^{n - r}) = \mu(s, a_1^n) \in F$
  we find $a_1^{n-r} \in A_1$. 
  Repeating this procedure, if $n - r \ge |Q| - 1 + |P|$,
  we ultimately find $|Q| - 1 \le m < |Q| - 1 + |P|$
  such that $a_1^m \in A_1$
  and $\delta(Q, a_1^{|Q| - 1}) = \delta(Q, a_1^m)$.
  Note that the language $A_2 \cdots A_{k-1}$
  is finite.
  Then, as in case (i), 
  we only have to consider the  words,
  whose length and number is linear bounded in $|Q|$,
  \[
   a_1^n \cdot u,\quad  0 \le n < |Q| - 1 + |P|, a_1^n \in A_1, u \in A_2 \cdots A_{k-1}
  \]
  and the corresponding sets
  \[
   S = \delta(Q, a_1^n \cdot u),
  \]
  and these are all possible sets in $\{ \delta(Q, a_1^n u) \mid a_1^n \in A_1, u \in A_2 \cdots A_{k-1} \}$.
  Fix any such subset $S$.
  Then, as in case (ii) and Equation~\eqref{eqn:case_two_P}, 
  choose any $N \ge |Q| - 1$ with $a_k^N \in A_k$ and
  we only have to compute $\delta(S, a_k^N)$
  and test if it is a singleton set.
  So, in total, we only have to test the words
  \[
   a_1^n \cdot u a_k^N, 0 \le n < |Q| - 1 + |P|, a_1^n \in A_1, u \in A_2 \cdots A_{k-1}.
  \]
  Their length and number is linear bounded in $|Q|$
  and computing the reachable state from each state of the input automaton,
  and testing if a unique state results, could be performed in polynomial
  time. Hence, the overall procedure could be performed in polynomial time.
 \end{enumerate}
 So, we have handled every case and the proof is complete.\qed
 % oder zeigen A_1 oder A_k finite -> nur endlich viele wörter müssen getestet werden
 % mit (iii)
\end{proof}

Now, we state a sufficient condition for \NP-hardness over binary alphabets. 
This condition, together with Proposition~\ref{prop:hom_lower_bound_complexity},
allows us to handle the general case in Theorem~\ref{thm:dichotomy}.
Its application together with Proposition~\ref{prop:hom_lower_bound_complexity} shows, in some respect, that the language $ab^*a$
is the prototypical language giving \NP-hardness.
We give a proof sketch of Lemma~\ref{lem:np_hardness} at the end of this section.

\begin{toappendix}

In the proof of Lemma~\ref{lem:np_hardness}
we will need the following two lemmata.
For $n > 0$, set
\[
 L_n = (\Sigma^* a \Sigma^* b^{|P|} \Sigma^*)^n.
\]
Recall that $\mathcal B = (\Sigma, P, \mu, p_0, F)$.

\begin{lemma}
\label{lem:number_of_b_blocks}
 Let $\Sigma = \{a,b\}$ and $L(\mathcal B) \subseteq a_1^* \cdots a_k^*$
 with $a_i \in \Sigma $ and $n > 0$.
 Then, the following are equivalent:
 \begin{enumerate} 
 \item $L(\mathcal B) \cap L_n \ne \emptyset$,
 
 \item there exist $u_0, \ldots, u_n \in \Sigma^* a \Sigma^*$
  and $p_1, \ldots, p_n \ge |P|$
  such that \[
  u_0 b^{p_n} u_1 \cdots u_{n-1} b^{p_n} u_n \in L(\mathcal B),
  \]
 \item there exist $u_0, \ldots, u_n \in \Sigma^* a \Sigma^*$
  and $p_1, \ldots, p_n > 0$
  such that 
  \[ 
  u_0 (b^{p_1})^* u_1 \cdots u_{n-1} (b^{p_n})^* u_n \subseteq L(\mathcal B).
  \]
 \end{enumerate}
\end{lemma}
\begin{proof}
 That (1) implies (2) is obvious.
 As $p_i \ge |P|$, when reading these factors they have to induce a loop in $\mathcal B$,
 which implies (3).
 Lastly, if (3) holds true, as
 \[
  u_0 b^{|P|\cdot p_1} u_1 \cdots u_{n-1} b^{|P| \cdot p_n} u_n \in L(\mathcal B)
 \]
 and $u_i \in \Sigma^* a \Sigma^*$,
 we also find $u_0 b^{|P|\cdot p_1} u_1 \cdots u_{n-1} b^{|P| \cdot p_n} u_n \in L_n$
 and (1) follows.\qed
\end{proof}

\begin{lemma}
\label{lem:maximal_n}
 Let $\Sigma = \{a,b\}$ and $L(\mathcal B) \subseteq a_1^* \cdots a_k^*$
 with $a_i \in \Sigma $.
 Then, there exists a maximal $n$
 such that $L(\mathcal B) \cap L_n \ne \emptyset$
 and for this maximal $n$,
 we can assume that $u_i \notin \Sigma^* b^{|P|} \Sigma^*$
 for the $u_i$, $i \in \{0,\ldots,n\}$, as in the previous lemma
 and $n \le |P|$. 
\end{lemma}
\begin{proof}
 Recall $\mathcal B = (\Sigma, P, \mu, p_0, F)$
 Note that $\mathcal B$ must necessarily be polycylic (this is a slightly stronger
 claim than Theorem~\ref{thm:bounded_characterization}, as this theorem
 only asserts existence of some polycyclic automaton)
 after removing all states that are not coaccessible, i.e., states from which no final state is reachable,
 which could obviously be done without altering $L(\mathcal B)$.
 For if $\mathcal B$ is then not polycyclic, then some strongly
 connected component does not consists of a single cycle only
 and we find two distinct words $u, v$ and a state $p \in P$
 such that $\mu(p, u) = \mu(p, v) = p$ (see also the forbidden
 pattern in~\cite[Theorem 4.29]{Pin2020}).
 But then, if we choose $x,y \in \Sigma^*$
 such that $\mu(p_0, x) = p$
 and $\mu(p, y) \in F$, we find $x(u+v)^*y \subseteq L(\mathcal B)$.
 Set $m = \max\{|u|, |v|\}$
 Then, for $i > 0$,
 \[
  \{ w \in x(u+v)^*y : |w| \le |x| + i \cdot m\}
 \]
 contains $x(u+v)^i$, and $|x(u+v)^i| = 2^i$.
 So, $L(\mathcal B) \cap \{ w \in \Sigma^* : |w| \le n \}$
 contains at least $2^{\lfloor n - (|x| - |y|) / m \rfloor}$
 many words, i.e., it not sparse.
 Furthermore, as $L \cap \Sigma^n \in O(n^c)$
 as a function of $n$ if and only if $L \cap \{ w \in \Sigma^* : |w| \le n \} \in O(n^{c'})$
 as a function of $n$ for some $c,c' \ge 0$,
 the claim follows.
 
 So, we can assume $\mathcal B$ is polycyclic and every state is coaccessible.
 Now, note that this implies that every loop in $\mathcal B$ (or strongly connected component
 in this case) must be labelled by a single letter, for if we
 have $\mu(p, u) = p$ with $|u|_a > 0$ and $|u|_b > 0$
 and choose again $x,y$ such that $\mu(p_0, x) = p$
 and $\mu(p, y) \in F$,
 we find $xu^ky \in L(\mathcal B)$, which contradict $L(\mathcal B) \subseteq a_1^* \cdots a_k^*$.
 
 But then, note that if, for example, $aba \in L(\mathcal B)$,
 we must have $|P| \ge 2$, as $\mu(p_0, ab) \notin \{ p_0, \mu(p_0,a) \}$.
 Similarly, if we have a word that switches letters, every time a letter-switch
 occurs the state we end up in $\mathcal B$ must be a new state not visited before,
 for otherwise we would have a loop whose transition are not exclusively
 labelled by a single letter.
 
 So, this implies that 
 if we have a word as written in Lemma~\ref{lem:number_of_b_blocks}
 in $L(\mathcal B)$, then $n \le |P|$
 which implies that we can find a maximal $n$.
 That $u_i \notin \Sigma^* b^{|P|} \Sigma^*$
 is also implied by Lemma~\ref{lem:number_of_b_blocks}
 and the maximality of $n$. \qed
\end{proof}
\end{toappendix}


\begin{lemmarep}
\label{lem:np_hardness}
 Suppose $\Sigma = \{a,b\}$. %und a,b vertauscht unten mit homsatz.
 Let $L(\mathcal B) \subseteq \Sigma^*$ be letter-bounded.
 Then, $L(\mathcal B)$\textsc{-Constr-Sync}
 is $\NP$-hard %if and only %nebenbei zwischen die a^+ nochmal sigma packen bringt nichts.
 % da man annehmen kann a_{i+1} ungleich a_i
 if $L(\mathcal B) \cap \Sigma^* a  b^{|P|}b^*  a \Sigma^* \ne \emptyset$.
 % nur if teil, weil if and only wird impliziert im theorem!
\end{lemmarep}
\begin{toappendix}
\begin{figure}[htb]
     \centering
     \hspace*{-2.5cm}
\includegraphics[width=17cm]{reduction.png}
  \caption{%Schematic illustration of the reduction from the proof of Proposition \ref{prop:stricly_bounded_np_hard}.
   The reduction from the proof of Lemma~\ref{lem:np_hardness}
   in the special case $J = 3$ (see the proof for the definition of $J$)
   and two input automata $\mathcal A_1, \mathcal A_2$ over $\{b\}$. The automata
   $\mathcal A_{i,j}$ are inflated, according to Definition~\ref{def:inflate_aut},
   copies of $\mathcal A_i$
   for $i \in \{1,2\}$, $j \in \{1,2,3\}$. 
   The letter $a$ maps
   every state not associated with a path inside each $\mathcal A_{i,j}$ to the last innermost state that 
   is hit by an $a$ along the path leading into this automaton. This is only drawn for $\mathcal A_{1,1}$ but left
   out for the other automata, also, to give a more ``high-level'' drawing, the $b$-transitions
   are not drawn. On the right end is the sink state $t$. The paths stay inside the automata but leave
   as soon as an $a$ is read.}
  \label{fig:reduction}
\end{figure}

\begin{proof}[Proof of Lemma~\ref{lem:np_hardness}]
First, using Lemma~\ref{lem:maximal_n},
choose $J > 0$ maximal such that
\[
 L(\mathcal B) \cap (\Sigma^* a \Sigma^* b^{|P|} \Sigma^*)^J \ne \emptyset.
\]
As stated in the lemma, we have $J \le |P|$ (which implies the constrution to follow could
be carried out in polynomial time).
Then, by Lemma~\ref{lem:number_of_b_blocks}, there
exist $u_0, \ldots, u_J \in \Sigma^* a \Sigma^*$
and $p_1, \ldots, p_J > 0$ ($J > 0$) such that \todo{Anderer Bezeichner als $J$?}
\[
 u_0 (b^{p_1})^* u_1 \cdots u_{J-1} (b^{p_J})^* u_J \subseteq L(\mathcal B).
\]
 Let $N$ be $|P|$ times the least common multiple of the numbers $p_1, \ldots, p_J$.
 We give a reduction from the {\sc DFA-Intersection} for unary 
 input automata, which is \NP-complete in this case~\cite{stockmeyer1973word,fernau2017problems}.
 Let $\mathcal A_i = (\{b\}, Q_i, \delta_i, q_i, F_i)$
 for $i \in \{1,\ldots,k\}$ be unary input automata, and we want to know
 if they all accept a common word. The problem remains
 \NP-complete if we assume for no input automaton, a start state is also a final state.
 This is easily seen but could also be shown similar to~\cite[Proposition 1]{DBLP:conf/ictcs/Hoffmann20}.
 Also, we can assume $F_i \ne \emptyset$ for all $i \in \{1,\ldots,k\}$.

 We are going to construct a semi-automaton $\mathcal C = (\{a,b\}, Q, \delta)$. %intuition?

 Write $u_i = u_{i,1} \cdots u_{i,|u_i|}$ with $u_j \in \Sigma$.
 For each $i \in \{0,\ldots,J\}$, we construct a path labelled with $u_i$.
 Formally, let $P_i = \{ q_{i, 0}, \ldots, q_{i,|u_i|} \} \subseteq Q$
 be new states and set
 \[
  \delta(q_{i,j-1}, u_j) = q_{i,j}.
 \]
 Then, for each $\mathcal A_i$
 we construct $J$ (disjoint) 
 copies of $\mathcal A_i$ and inflate
 them according to Definition~\ref{def:inflate_aut} by $N$.
 Call the results $\mathcal A_{i, 1}, \mathcal A_{i,2}, \ldots, \mathcal A_{i,J}$
 with $\mathcal A_{i,j} = (\{b\}, Q_{i,j}, \delta_{i,j}, s_{i,j}, F_{i,j})$.
 Note these are unary automata over the letter $b$.
 Also, let $t \in Q$ be a new state, which will be a (global) sink state in $\mathcal C$, i.e.,
 we set $\delta(t, a) = \delta(t, b) = t$.
 Next, we describe how we interconnect these automata with the paths and with $t$.
 See also Figure~\ref{fig:reduction} for a sketch of the reduction in the special case $J = 3$
 and two input automata.
 
 
 
 \begin{enumerate}
 \item Let $j \in \{1,\ldots, J\}$. For each final state $q \in F_{i,j}$
  let $P_{i,q}$ be a disjoint copy of the path $P_i$ constructed above,
  except for one final state $q$ were we simply retain the path $P_i$, but also name it by $P_{i,q}$.
  By identifying states, we mean states that we have previously constructed are now merged
  to a single state in $Q$. We have to pay attention that this procedure does not introduces
  any non-determinism.
  We identify the state $q_{i,0}$ with $q$
  and continue to identify the states $q_{i,j}$ and $q' \in Q_{i,j}$
  if $q_{i,j-1}$ and $q'' \in Q_{i,j}$ were identified
  and $u_{i,j} = b$ and $q' = \delta_{i,j}(q'', b)$. As $u_i \in \Sigma^* a \Sigma^*$, this process has to come to a halt
  before we have identified $< J$ states.
  Note that the first state such that $q_{i,j-1}$ and $q''\in Q$
  were identified but not $q_{i,j}$ and $\delta_{i,j}(q'', b)$, i.e., were $u_{i,j} = a$,
  we have added an $a$-transition to $q_{i,j}$
  from $q'' = q_{i,j-1}$ in $\mathcal A_{i,j}$, i.e., this is the first
  $a$-transition we have added to $\mathcal A_{i,j}$ and it branches out of $\mathcal A_{i,j}$.
  
  Then, if $j \le J - 1$,
  identify the state $q_{i,J}$ with the start state $s_{i,j+1}$ of $\mathcal A_{i,j+1}$, i.e.,
  the path $P_{i,q}$ ends at this state.
  And if $j = J$, we identify the state $q_{i,J}$ with $t$.
  
 \item For the path $P_0$ identify its end state $q_{i,|u_0|}$
  with the start state $s_{i,1}$ of $\mathcal A_{i,1}$.
     
 \item Up to now, we still hav emissing transitions. In all the paths created, 
  every missing $b$-transition, i.e., were we have a state with an $a$-transition
  leading out but no $b$-transition, we add a self-loop labelled with $b$ to that state.
  For each path $P$ (including the copies constructed in the first step)
  let $p \in P$ be that state closest to the end state, but that does not equal
  the end state (by the identifications above, some end state might already have an $a$-transition
  that goes out of some automaton $\mathcal A_{i,j}$) and has an outgoing $a$-transition.
  Such a state exists as the $u_i \in \Sigma^* a \Sigma^*$.
  \todo{Hier uU statt auf den Zustand immer auf den letzten Zustand davor mit einer $a$-Transition mappen?}
  Then, for every state in $P$ that does not have an $a$-transition
  we add an $a$-transition going to $p$.
  Consider $\mathcal A_{i,j}$ and let $P$ some path (the specific choice does not matter)
  ending at the start state of $\mathcal A_{i,j}$.
  For each state $q \in Q_{i,j}$ that does not has an outgoing $a$-transition up to now,
  add an $a$-transition going to the state $p \in P$ described above in that path.
  This ensures later that, by reading an $a$, we end up in a well-defined situation.
 \end{enumerate}
 Then, put all the states created so far, i.e., those of the $\mathcal A_{i,j}$
 and those of the paths constructed, into $Q$ (note for each $i \in \{1,\ldots,k\}$
 we have constructed paths and automata, intuitively we have copied each $\mathcal A_i$, inflated
 the copies and interconnected them with the paths given by the $u_i$)
 and let $\delta$ be the transition as defined above or as given by $\mathcal A_{i,j}$
 on the state of these automata.
 
 
 We need the following property of $\mathcal C$. Suppose $i \in \{1,\ldots,k\}$, 
 $j \in \{1,\ldots,|J|\}$ and $w \in \{a,b\}^*$.
 
  \medskip 
 
\noindent\underline{Claim:}
  Let $q \in Q_{i,j} \setminus F_{i,j}$ with $\delta(q, w) = t$.
  Then, there exist 
  \[ 
  u_1, u_2 \cdots, u_{|J|-i+1} \subseteq \{b\}^*
  \]
  and $u,v\in \{a,b\}^*$
  such that $|u_i| \ge N$ and $|u_i|$ is divisible by $N$
  for all $i \in \{1,\ldots,|J|-i+1\}$
  and $v_1, \ldots, v_{|J|-i+1} \in \{a,b\}^*a\{a,b\}^*$
  so that 
  \[
   w = vu_1 v_1 u_2 v_2 \cdots u_{|J|-i+1} v_{|J|-i+1} u
  \]
  and $v \notin \Sigma^* b^N \Sigma^*$.
 \begin{quote} % genauer, muss jeweils teile dahinter von start auf final?, und wegen maximalitt |J| auch nicht mehr soclhe mit "echten" a's dazwischen.
     \emph{Proof of the Claim.} First, the state $q \in Q_{i,j}$ has to be mapped to a final
     state, which could only be done by a word containing at least $N$
     times the letter $b$,
     as in the inflated construction we can only go from non-auxiliary states
     to non-auxiliary states by reading at least that number of letters.
     However, before that we might read some word $v \in \{a,b\}^*$ that moves states around, does
     not has a consecutive sequence of more than $N$ $b$'s and hence, every $a$
     goes back to the start state. But at some point, this has to come to an end and we have
     to read a sequence of more than $N$ consecutive $b$'s.
     Additionally, by the construction of the inflation, the word
     that moves from a non-auxiliary state to another non-auxiliary state
     must have a number of $b$'s that is divisible by $N$.
     Also, observe that such a word must consists entirely of $b$, because
     for non-final states in $\mathcal A_{i,j}$ every $a$ maps back to the start state.
     Then, by construction (recall $u_i \in \Sigma^* a \Sigma^*$ for the labels
     of the paths constructed above) to move between the automata $\mathcal A_{i,j}$
     inside of $\mathcal C$
     we have to traverse a path
     where, on some part, we can only move forward by reading the letter $a$.
     After this, when we are at the start state of $\mathcal A_{i,j+1}$,
     as by assumption the start state is not final, we again have to read at least $N$
     times the letter $b$ and so on, until we have reached a final 
     state in $\mathcal A_{i,|J|}$.
     Then, we have to read at least one $a$ to map the final state to $t$, from which
     on, as $t$ is a sink state, we can read any word. 
     \emph{[End, Proof of the Claim]}
 \end{quote}
 
 
 The automaton $\mathcal C$ has a synchronizing word in $L$
 if and only if all the $\mathcal A_i$, $i \in \{1,\ldots,k\}$,
 accept a common word.
 
 \begin{enumerate}
 \item Assume we have a word $b^n$ accepted by all $\mathcal A_i$ for $i \in \{1,\ldots,k\}$. 
 Then, for
 \[
  w = u_0 b^{N\cdot n} u_1 \cdots u_{J-1} b^{N\cdot n} u_J
 \] 
 we have $w \in L$ and $w$ synchronizes $\mathcal A$.
 Note that, after reading $u_{j-1}$,
 the automaton $\mathcal A_{i,j}$
 is either in its start state, or the final $a$ in $u_{j-1}$
 has mapped some state in $\mathcal A_{i,j}$ to a state outside of $Q_{i,j}$.
 So, when reading $b^{N\cdot n}$, 
 as $\mathcal A_{i,j}$ equals  the inflation of $\mathcal A_i$ by $N$,
 we end up in a final state $F_{i,j}$.
 Then, we read $u_j$ to map those final states to the start state
 of the next automaton $\mathcal A_{i,j+1}$ or to $t$ if $j = J$.
 Note that all states in-between are either mapped
 to a start state of some $\mathcal A_{i,j}$, moved inside of some
 $\mathcal A_{i,j}$, or, when an $a$ is read and they are not mapped
 back to a state that ultimately ends in a start state of some $\mathcal A_{i,j}$
 are moved toward the state $t$.
 As we always read enough $a$ to always make a step towards the sink state $t$
 the result follows.\todo{genauer}
 
 
 
 \item  Assume $\mathcal A$ has a synchronizing word $w \in L$.
  Then, as $t$ is a sink state, the word $w$ must map every state to $t$.
  Consider the start state of some $\mathcal A_{i,1} = (\{b\}, Q_{i,1}, \delta_{i,1}, q_{i,1}, F_{i,1})$.
  By the above claim,  
  there exist $u_1, u_2 \cdots, u_{J} \subseteq \{b\}^*$
  such that $|u_i| \ge N$ and $|u_i|$ is divisible by $N$
  for all $i \in \{1,\ldots,|J|\}$
  and $v_1, \ldots, v_{J} \in \{a,b\}^*a\{a,b\}^*$ and $v, u \in \{a,b\}^*$
  so that 
  \[ 
    w = vu_1 v_1 u_2 v_2 \cdots u_{J} v_{J} u.
  \]
  By the above claim, Lemma~\ref{lem:maximal_n} and the maximal choice of $J$,
  we have 
  \[ 
  \{ v, v_1, \ldots, v_J \} \cap \Sigma^* b^{|P|} \Sigma^* = \emptyset,
  \] 
  i.e.,
  these words does not contains a sequence of more than $|P|$, and so in particular not more than $N$,
  consecutive $b$'s.
  
  
  Now, let $b^n$ be a maximal non-empty factor whose length $n$ is divisible by $N$ of $vu_1 v_1$
  and using only the letter $b$.
  Note that, by construction of $\mathcal A_{i,1}$,
  if we write $vu_1 v_1 = x b^n y$,
  we have $\delta_{i,1}(q_{i,1}, x) = q_{i,1}$.
  Then, we claim that $b^{n / N}$ is accepted
  by every automaton $\mathcal A_i$. 
  Fix an index $i \in \{1,\ldots,k\}$.
  By the construction of the inflation, this is equivalent
  with the condition that $b^n$ drives every automaton
  $\mathcal A_{i,j}$ for $j \in \{1,\ldots,|J|\}$
  from the start state to some final state.
  Suppose this is not the case. As the automata $\mathcal A_{i,j}$
  are isomorphic, i.e., they are copies of each other, we can assume this is not the case for $\mathcal A_{i,1}$, i.e., 
  we have $\delta_{i,1}(q_{i,1}, b^n) \notin F_{i,1}$.
  Then, consider the following suffix of $w$ (recall $xb^n y = v u_1 v_1$, and $y$ has to start with an $a$) 
  \[
   y u_2 v_2 \cdots u_{|J|} v_{|J|} u.
  \] 
  Note that if we have in $u$ a consecutive sequence of $b$'s
  of length more than $N$, the rest of $u$ also must consist of $b$'s only, i.e.,
  we cannot read an $a$ anymore.
  For suppose this is not the case and $u \in \Sigma^* b^N \Sigma^* a \Sigma^*$.
  We have $\delta(q_{i,1}, xb^n) = \delta(q_{i,1}, b^n) \in Q_{i,1} \setminus F_{i,1}$.
  By assumption, $\delta(q_{i,1}, w) = t$,
  and so we must have $\delta(q_{i,1}, y u_2 v_2 \cdots u_{|J|} v_{|J|} u) = t$.
  Applying the above claim again,
  yields that we can factorize $y u_2 v_2 \cdots u_{|J|} v_{|J|} u$
  such that we have at least $|J|$ blocks of consecutive 
  $b$'s broken up by at least one occurrence of the letter $a$
  between each such block.
  However, then 
  then
  \[
   w = x b^n y u_2 v_2 \cdots u_{|J|} v_{|J|} u,
  \]
  as $y$ starts with an $a$, we would get a factorization
  of $w$ with $|J| + 1$ blocks of consecutive $b$'s separated by words
  with at least one $a$, which is not possible by the maximal choice
  of $J$ and Lemma~\ref{lem:number_of_b_blocks}.
  
 \end{enumerate}
 So, this shows that this is a valid reduction.\qed
\end{proof}
\end{toappendix}




So, finally, we can state our main theorem of this section.
Recall that by Theorem~\ref{thm:sparse_in_NP},
and as the class of bounded regular languages equals
the class of sparse regular languages~\cite{DBLP:journals/eik/LatteuxT84}, for bounded regular constraint
languages, the constrained problem is, in our case, in \NP. 
 
 
\begin{theoremrep}[Dichotomy Theorem]
\label{thm:dichotomy}
 % a_i != a_{i+1} sonst nichts weiter
 % dann a_{j_1} auf a und so weiter hom bild
 % hom bild a^* b^* a^*
 % aber a_{j_2} verschieden von a_{j_1}, a_{j_3} über die vereinigne
 % wie in entscheidungsverfahren.
 %
 % aber aufpassen, P ist anders!!!! aber dann schlussfolgern für das P' auch unendlich
 % oder 2^{|P|} nehmen
 %
 % oder wie so eine "sigma"-menge, also für jedes n existiert eins - schnitt/vereinigung - schreiben. aber da nicht klar ob regulär.
 Let $a_1, \ldots, a_k \in \Sigma$ be a sequence of letters
 and $L \subseteq a_1^* \cdots a_k^*$ be regular.
 The problem $L\textsc{-Constr-Sync}$
 is \NP-complete if
 \[
  L \cap \left(\bigcup_{\substack{1 \le j_1 < j_2 < j_3 \le k \\ a_{j_2} \notin \{a_{j_1}, a_{j_3}\} }} L_{j_1,j_2,j_3} \right) \ne \emptyset
 \]
 with $L_{j_1, j_2, j_3} = \Sigma^* a_{j_1} \Sigma^* a_{j_2}^{|P|} \Sigma^* a_{j_3} \Sigma^*$
 for $1 \le j_1 < j_2 < j_3 \le k$ and solvable in polynomial time otherwise.
\end{theoremrep}
\begin{proof}
 Set $L = L(\mathcal B)$.
 %By Theorem~\ref{thm:bounded_regular_form}, we can
 %write % auf lemma/theorem verweisen, dass es immer so geht.
 %$L(\mathcal B) = \bigcup_{i=1}^n A_1^{(i)} \cdots A_k^{(i)}$ % erwähnen k gleich da man eps wählen kann, vielleicht als lemma diese form hinschreiben. todo
 %with unary regular languages $A_j^{(i)} \subseteq \{a_j\}^*$
 %for $j \in \{1,\ldots,k\}$.
 Let $j_1, j_2, j_3 \in \{1,\ldots,k\}$
 be such that $a_{j_2}\notin\{a_{j_1},a_{j_3}\}$, $j_1 < j_2 < j_3$
 and
 \[
  L(\mathcal B) \cap L_{j_1, j_2, j_3} \ne \emptyset.
 \]
 Then, there exists a word $u_1 a_{j_1} u_2 a_{j_2}^{|P|} u_3 a_{j_3} u_4 \in L(\mathcal B)$
 with $u_1, u_2, u_3, u_4 \in \Sigma^*$.
 By the pigeonhole principle, when reading the factor $b^{|P|}$,
 at least one state has to be traversed twice 
 and we find $p > 0$ such that $u_1 a u_2 b^{|P| + i\cdot p} u_3 a u_4$
 for any $i \ge 0$.
 
 
 
 Define a homomorphism $\varphi : \Sigma^* \to \{a,b\}^*$
 by $\varphi(a_{j_1}) = \varphi(a_{j_3}) = a$,
 $\varphi(a_{j_2}) = b$
 and, for the remaining letters, $\varphi(a) = \varepsilon$,
 if $a \in \Sigma \setminus \{a_{j_1}, a_{j_2}, a_{j_3}\}$.
 Then, $\varphi(L) \subseteq \varphi(a_1)^* \cdots \varphi(a_k)^*$
 is letter-bounded. % zeigen, dass unter hom abgeschlossen. aber klar
 Set $\Gamma = \{a,b\}$ and let $\mathcal B' = (\Gamma, P', \mu', p_0', F')$
 be a recognizing PDFA for $\varphi(L)$.
 %By Lemma~\ref{lem:L_j1j2j3_intersection},
 %there exists $i_0 \in \{1,\ldots,n\}$
 %such that $A_{j_1}^{(i_0)}$ and $A_{j_3}^{(i_0)}$
 %do not equal~$\{\varepsilon\}$
 %and $A_{j_2}^{(i_0)}$ is infinite.
 %As $\varphi$ is a homomorphism, we have 
 %$\varphi(A_1^{(i_0)} \cdots A_k^{(i_0)}) = \varphi(A_1^{(i_0)}) \cdots \varphi(A_k^{(i_0)})$. Furthermore,
 %$a^+ \cap \varphi(A_{j_1}^{(i_0)}) \ne \emptyset$,
 %$a^+ \cap \varphi(A_{j_3}^{(i_0)}) \ne \emptyset$
 %and $b^{|P'|}b^* \cap \varphi(A_{j_2}^{(i_0)}) \ne \emptyset$.
 %As the language $\varphi(A_1^{(i_0)}) \cdots \varphi(A_k^{(i_0)})$
 %is contained in $\varphi(L)$,
 %this yields
 We have
 \[
  \varphi(u_1) a \varphi(u_2) b^{|P| + i\cdot p} \varphi(u_3) a \varphi(u_4) \in \varphi(L) 
 \]
 for any $i \ge 0$. So,
 $
 \varphi(L) \cap \Gamma^* a^+ \Gamma^* b^{|P'|}b^* \Gamma^* a^+ \Gamma^* \ne \emptyset.
 $
%  However, note that, for $u \in \{a,b\}^*$,
%  \begin{multline*}
%       u \in \varphi(A_1^{(i_0)}) \cdots \varphi(A_k^{(i_0)}) \cap \Gamma^* a^+ \Gamma^* b^{|P'|}b^* \Gamma^* a^+ \Gamma^*
%   \\ \Leftrightarrow 
%   u \in \varphi(A_1^{(i_0)}) \cdots \varphi(A_k^{(i_0)}) \cap \Gamma^* a^+  b^{|P'|}b^* a^+ \Gamma^*.
%  \end{multline*}
% So,
% $
%  \varphi(L) \cap \Gamma^* a^+ b^{|P'|} b^* a^+ \Gamma^* \ne \emptyset. %todo genauer
% $
 By Lemma~\ref{lem:np_hardness}, $\varphi(L)$\textsc{-Constr-Sync}
 is \NP-hard and so, by Proposition~\ref{prop:hom_lower_bound_complexity},
 also $L\textsc{-Constr-Sync}$ is \NP-hard, and so, with Theorem~\ref{thm:sparse_in_NP},
 \NP-complete.
 
 
 Now, suppose
  $
 L(\mathcal B) \cap \left(\bigcup_{\substack{1 \le j_1 < j_2 < j_3 \le k \\ a_{j_2} \notin \{a_{j_1}, a_{j_3}\} }} L_{j_1,j_2,j_3} \right) = \emptyset.
 $
  By Theorem~\ref{thm:bounded_regular_form}, we can
 write % auf lemma/theorem verweisen, dass es immer so geht.
 $L(\mathcal B) = \bigcup_{i=1}^n A_1^{(i)} \cdots A_k^{(i)}$ % erwähnen k gleich da man eps wählen kann, vielleicht als lemma diese form hinschreiben. todo
 with unary regular languages $A_j^{(i)} \subseteq \{a_j\}^*$
 for $j \in \{1,\ldots,k\}$.
 Then, 
 \[ 
 ( A_1^{(i)} \cdots A_k^{(i)} ) \cap \left(\bigcup_{\substack{1 \le j_1 < j_2 < j_3 \le k \\ a_{j_2} \notin \{a_{j_1}, a_{j_3}\} }} L_{j_1,j_2,j_3} \right) = \emptyset
 \]
 for any $i \in \{1, \ldots, n\}$.
 However, this implies that for any $i \in \{1,\ldots,n\}$, if there exists $j \in \{1,\ldots, k\}$
 such that $A_j^{(i)}$ is infinite, 
 then for all $j' < j$, or for all $j' > j$, % besonders wenn es keine gibt, als remark nach der proposition?
 we have $A_{j'} \subseteq \{a_j\}^*$ (recall that if $A_{j'} = \{\varepsilon\}$, then
 this is also fulfilled).
 Hence, by Proposition~\ref{prop:stricly_bounded_P},
 we have $(A_1^{(i)} \cdots A_k^{(i)})\textsc{-Constr-Sync} \in \PTIME$
 and then, by Lemma~\ref{lem:union},
 $L(\mathcal B)\textsc{-Constr-Sync} \in \PTIME$.\qed
\end{proof}

As the languages $L_{j_1, j_2, j_3}$ are regular, we
can devise a polynomial-time algorithm which checks the condition
mentioned in Theorem~\ref{thm:dichotomy}. 
 
\begin{corollary} %\todo{nicht $sigma = {a-1, ...m, a_k}$ schreiben, weil es suggeriert die buchstaben wären alle verschieden.}
 Given a PDFA $\mathcal B$ and a sequence of letters $a_1, \ldots, a_k$
 as input such that $L(\mathcal B) \subseteq a_1^* \cdots a_k^*$,
 the complexity of $L(\mathcal B)$\textsc{-Constr-Sync}
 is decidable in polynomial-time.
\end{corollary}
\begin{proof}
 An automaton for each $L_{j_1, j_2, j_3}$
 has size linear in~$|P|$. So, by the product automaton construction~\cite{HopUll79}, non-emptiness of
 $L(\mathcal B)$ with each $L_{j_1, j_2, j_3}$
 could be checked in time $O(|P|^2)$.
 Doing this for every $L_{j_1, j_2, j_3}$
 gives a polynomial-time algorithm
 to check non-emptiness of the language written
 in Theorem~\ref{thm:dichotomy}.~\qed
\end{proof}

\begin{example}
 For the following constraint languages CSP is \NP-complete: $ab^*a$,
 $aa(aaa)^*bbb^*d \cup a^*b \cup d^*$, $bbcc^*d^* \cup a$.
 
 For the following constraint languages CSP is in \PTIME: $a^5bd \cup cd^4$,
 $a^5bd \cup cd^*$, $aa^*bbbbcd^* \cup bbbdd^*d$.
\end{example}

\begin{proof}[Proof Sketch for Lemma~\ref{lem:np_hardness}]
 We construct a reduction from an instance
 of $\textsc{DisjointSetTransporter}$\footnote{Note that the problem $\textsc{DisjointSetTransporter}$ is over a unary alphabet, but for $L\textsc{-Constr-Sync}$
 we have $|\Sigma| > 1$. Indeed, we need the additional letters in $\Sigma$.}
 for unary input automata.
 %, which is $\NP$-complete in
 %this case, by Theorem~\ref{prop:set_transporter_np_complete},
 %to $L\textsc{-Constr-Sync}$ for $L$ as written in the statement\footnote{Note that the problem $\textsc{DisjointSetTransporter}$ is over a unary alphabet, but for $L\textsc{-Constr-Sync}$
 %we have $|\Sigma| > 1$. Indeed, we need the additional letters in $\Sigma$ in our reduction.}.
 
 To demonstrate the basic idea, we only do the proof
 in the case $L \subseteq a^* b^* a^*$.
 %By Theorem~\ref{thm:bounded_regular_form},
 %we can write $L = \bigcup_{i=1}^n A_1^{(i)} A_2^{(i)} A_3^{(i)}$
 %with regular languages $A_1^{(i)}, A_3^{(i)} \subseteq \{a\}^*$
 %and $A_2^{(i)} \subseteq \{b\}^*$.
 By assumption we can deduce $a^{r_1} b^{r_2} a^{r_3} \in L(\mathcal B)$
 with $p_2 \ge |P|$ and $r_1, r_3 \ge 1$.
 By the pigeonhole principle, in $\mathcal B$, 
 when reading the factor $b^{r_2}$, at least one state has to be traversed twice.
 Hence, we find $0 < p_2 \le |P|$ such that $a^{r_1} b^{r_2 + i\cdot p_2} a^{r_3}
 \subseteq L(\mathcal B)$ for each $i \ge 0$.
 %Then, by Lemma~\ref{lem:L_j1j2j3_intersection},
 %we must find $i_0 \in \{1,\ldots,n\}$
 %such that $a^+ \cap A_1^{(i_0)} \ne \emptyset$, $a^+ \cap A_3^{(i_0)} \ne \emptyset$
 %and $A_2^{(i_0)}$ is infinite.
%  As a shorthand,
%  we set $A_1 = A_1^{(i_0)}$, $A_2 = A_2^{(i_0)}$ and $A_3^{(i_0)} = A_3$.
% %  By Lemma~\ref{todo}, the languages could be written
% %  as a finite union of languages recognizable
% %  by unary automata with a single final state.
% %  As concatenation distributes over union, 
% %  %we can then write $A_1^{(i_0)} A_2^{(i_0)} A_3^{(i_0)}$
% %  %as a finite union of languages
% %  without loss of generality, as by such a rearranging the above assumption
% %  still holds true for at least one part of the union, we can 
% %  suppose $A_1$,  $A_2$ and $A_3$
% %  are recognizable by unary automata with a single final state.
% Then, it is easy to see that we find numbers $r_i, p_i$ such that
% $
%     a^{r_1}(a^{p_1})^* \subseteq A_1,
%     b^{r_2}(b^{p_2})^* \subseteq A_2, 
%     a^{r_3}(a^{p_3})^* \subseteq A_3
% $
% with, by the other assumptions, $p_2 > 0$ ($A_2$ infinite)
% and $r_1 + p_1 > 0$, $r_3 + p_3 > 0$ ($A_1$, $A_3$ non-empty and do not equal $\{\varepsilon\}$).
% todo hinschreiben, wenn ncihts gesagt immer complete und deterministic


Let $\mathcal A = (\{c\}, Q, \delta)$ and $(\mathcal A, S, T)$
be an instance of \textsc{DisjointSetTransporter}.
We can assume $S$ and $T$ are non-empty, as for $S = \emptyset$
it is solvable, and if $T = \emptyset$ we have no solution.
Construct $\mathcal A' = (\Sigma, Q', \delta')$
by setting
$
 Q' = S_{r_2} \cup \ldots \cup S_{1} \cup Q \cup Q_1 \cup \ldots \cup Q_{p_2-1} \cup \{ t \},
$
where $t$ is a new state, $S_i = \{ s_i \mid s \in S \}$ for $i \in \{1,\ldots, r_2 \}$
are pairwise disjoint copies of $S$
and $Q_i = \{ q^i \mid q \in Q \}$ are\footnote{Observe
that by the indices a correspondence between the sets
is implied. The index
in $Q_i$ at the top to distinguish, for $s \in S$ and $i \in \{1,\ldots,\min\{r_2, p_2-1\}\}$, between
 $s_i \in S_i$ and $s^i \in Q_i$. Hence, for each $s \in S$ and $i \in \{1,\ldots, r_2\}$,
 the states $s$ and $s_i$ correspond to each other, and for $q \in Q$
 and $i \in \{1,\ldots, p_2-1\}$ the states $q$ and $q^i$.} 
 also pairwise disjoint 
copies of $Q$. Note that also $S_i \cap Q_j = \emptyset$
for $i \in \{1,\ldots, r_2 \}$ and $j \in \{1,\ldots, p_2-1\}$.
Set $S_0 = S$ %and $Q_0 = Q$ 
as a shorthand.
Choose any $\hat s \in S_{r_2}$, then, for $q \in Q$ and $x \in \Sigma$, the transition function is given by
\[
 \delta'(q, x) = \left\{
 \begin{array}{ll}
  s_{i-1} & \mbox{if } x = b \mbox{ and } q = s_i \in S_i \mbox{ for some } i \in \{1,\ldots, r_2\}; \\ 
  \hat s & \mbox{if } x = a \mbox{ and } q \in (Q \cup Q_1 \cup \ldots \cup Q_{p_2-1}) \setminus S; \\
  s_{r_2} & \mbox{if } x= a \mbox{ and } q = s_i \in S_i \mbox{ for some } i \in \{0,\ldots,r_2\}; \\
  t       & \mbox{if } x = a \mbox{ and } q \in T; \\
  q^{p_2-1} & \mbox{if } x = b \mbox{ and } q \in Q; \\
  q^{i-1} & \mbox{if } x = b \mbox{ and } q = q^i \in Q_i \mbox{ for some } i \in \{2,\ldots,p_2-1\}; \\
  \delta(q, c) & \mbox{if } x = b \mbox{ and } q = q^1 \in Q_1; \\
  q       & \mbox{otherwise}.
 \end{array}
 \right.
\]

\newcommand{\automatacloudother}[2][.44]{%
	\begin{scope}[#2]
		\node [rectangle,draw,thick,text width=8.1cm,minimum height=7.6cm,
		text centered,rounded corners, fill=white, name = re] {};
\end{scope}}    


\newcommand{\innerstateloop}{
\begin{scope}
  \node[state] (s1) at (0,0) {}; \node (s1label) at (0.5,-.1) {$\in Q$};
  \node[state] (s11) at (-0.5,0.6) {};  \node (s11label) at (0.32,0.6) {$\in Q_{p_2-1}$};
  %\node[state] (s14) at ( 0.5,0.6);
  \node (s12) at (-0.25,1.2) {};
  \node (s13) at ( 0.25,1.2) {};
  \path[->] (s1)  edge [bend left] node [left] {$b$} (s11);
%  \path[->] (s14) edge [bend left] node [right] {$b$} (s1);
   \path[->] (s11) edge [bend left] node [left] {$b$} (s12);
%   \path[->] (s13) edge [bend left] node [right] {$b$} (s14);
  \draw[dashed] (-0.25,1.2) -- (0.25,1.2);
\end{scope}
}

\begin{figure}[htb]
     \centering
    \scalebox{.65}{    
 \begin{tikzpicture}
 \tikzset{every state/.style={minimum size=1pt},>=stealth'}
 \node (cloud) at (0,0) {\tikz \automatacloudother{fill=gray!0,thick};};
 
  \node (reset1) at (0,3) {};
  \node (reset2) at (-10,2.5) {};
  \path[->] (reset1) edge [bend right] node [above] {$a$} (reset2);
  
  \node (reset3) at (0,-3) {};
  \node (reset4) at (-10,-2.5) {};
  \path[->] (reset3) edge [bend left] node [below] {$a$} (reset4);
  
  \node (reset5) at (-5.4,2.8) {};
  \node (reset6) at (-10,2.5) {};
  \path[->] (reset5) edge [bend right] node [above,pos=.3] {$a$} (reset6);
  
  \node (reset7) at (-5.4,-2.8) {};
  \node (reset8) at (-10,-2.5) {};
  \path[->] (reset7) edge [bend left] node [below,pos=.3] {$a$} (reset8);
  
 
  %\draw (-3,0) ellipse (1.1cm and 3.1cm);
  %\draw (-5.5,0) ellipse (1.1cm and 3.1cm);
  %\draw (-10,0) ellipse (1.1cm and 3.1cm);
  \draw[rounded corners] (-4.1,-3) rectangle (-1.7, 3) {};
  \draw[rounded corners] (-6.3,-3) rectangle (-4.8, 3) {};
  \draw[rounded corners] (-10.5,-3) rectangle (-9, 3) {};
    
  %\draw (3,0) ellipse (1cm and 3cm);
  \draw[rounded corners] (1.7,-3) rectangle (4.1, 3) {};
  
  \node[state] (t) at (7,0) {$t$};
  
  \node at (3,3.4) {{\LARGE $T$}};
  \node at (-3,3.4) {{\LARGE $S$}};
  \node at (-10,3.5) {{\LARGE $S_{r_2}$}};
  \node at (-5.5,3.5) {{\LARGE $S_{1}$}};
  \node at (0.1,4.1) {{\LARGE Original $\mathcal A$ (altered)}};
   % (altered for new alphabet)
  
  \node (s1) at (-.5,2) {\tikz \innerstateloop;};
  \node (s2) at (.5,-1.5) {\tikz \innerstateloop;};
  %\node (s2) at (-1.2,-2) {\tikz \innerstateloop;};
  
  \node (sT1) at (3,1.9) {\tikz \innerstateloop;}; \node (sT1copy) at (3,1.5) {};
  \node (sT2) at (2.8,0.4) {\tikz \innerstateloop;}; \node (sT2copy) at (2.8,0) {};
  \node (sT3) at (3.1,-1.5) {\tikz \innerstateloop;}; \node (sT3copy) at (3.15,-2) {};
  
  \node (sS1) at (-3,1.7) {\tikz \innerstateloop;}; \node (sS1copy) at (-2.95,1.15) {};
  \node (sS2) at (-2.7,0) {\tikz \innerstateloop;}; \node (sS2copy) at (-2.65,-.55) {};
  \node (sS3) at (-3,-1.8) {\tikz \innerstateloop;};\node (sS3copy) at (-3,-2.3) {};
   
  \node[state] (sS11) at (-5.5,1.7) {};
  \node[state] (sS12) at (-5.2,0) {};
  \node[state] (sS13) at (-5.7,-2) {};
  
  \node[state] (sSr1) at (-10,1.7) {};
  \node[state] (sSr2) at (-9.7,0) {};
  \node[state] (sSr3) at (-10.1,-2) {};
  
  \path[->] (t) edge [loop right] node {$\Sigma$} (t);
  
  % nichteinzeichnen, self loop implied
  \path[->] (sT1copy) edge [bend left=35] node [above] {$a$} (t)
            (sT2copy) edge [bend left=10] node [above] {$a$} (t)
            (sT3copy) edge node [above] {$a$} (t);
            
            
  \node (sSr1a) at (-8.5,1.7) {};
  \node (sSr2a) at (-8.2,0) {};
  \node (sSr3a) at (-8.7,-2) {};
  
  \node (sSr1b) at (-7.2,1.7) {};
  \node (sSr2b) at (-6.9,0) {};
  \node (sSr3b) at (-7.4,-2) {};
  
  \path[->] (sSr1) edge node [above,pos=.3] {$b$} (sSr1a);
  \path[->] (sSr2) edge node [above,pos=.27] {$b$} (sSr2a);
  \path[->] (sSr3) edge node [above] {$b$} (sSr3a);
 
  \path[->] (sSr1b) edge node [above,pos=.35] {$b$} (sS11);
  \path[->] (sSr2b) edge node [above] {$b$} (sS12);
  \path[->] (sSr3b) edge node [above] {$b$} (sS13);
  
  \path[->] (sS11) edge [bend right=10] node [above,pos=.38] {$b$} (sS1copy);
  \path[->] (sS12) edge [bend right=10] node [above,pos=.2] {$b$} (sS2copy);
  \path[->] (sS13) edge [bend right=10] node [above,pos=.4] {$b$} (sS3copy);
  
  \draw[dashed] (-8.5,1.7) -- (-7.2,1.7);
  \draw[dashed] (-8.2,0) -- (-6.9,0);
  \draw[dashed] (-8.7,-2) -- (-7.4,-2);
 \end{tikzpicture}}
  \caption{%Schematic illustration of the reduction from the proof of Proposition \ref{prop:stricly_bounded_np_hard}.
   The reduction from the proof sketch sketch of Lemma~\ref{lem:np_hardness}.
   The letter $a$ transfers everything surjectively onto $S_{r_2}$,
   indicated by four large arrows at the top and bottom and labelled 
   by $a$.
   The auxiliary states $Q_1, \ldots, Q_{p_2-1}$, which are meant
   to interpret a sequence $b^{p_2}$ like a single symbol in the original
   automaton, are also only indicated inside of $\mathcal A$, but not fully written out.}
  \label{fig:reduction_np_hard}
\end{figure}



Please see Figure~\ref{fig:reduction_np_hard} for a sketch
of the reduction.
For the constructed automaton $\mathcal A'$, the following could be shown:
$\exists m \ge 0 : \delta(S, c^m) \subseteq T$
if and only if $\mathcal A'$ has a synchronizing word in $ab^{r_2}(b^{p_2})^*a$
if and only if $\mathcal A'$ has a synchronizing word in $ab^*a$
if and only if $\mathcal A'$ has a synchronizing word in $a^*b^*a^*$.
% \begin{align*}
%     \exists m \ge 0 : \delta(S, c^m) \subseteq T 
%                               & \Leftrightarrow \mathcal A'\mbox{ has a synchronizing word in $ab^{r_2}(b^{p_2})^*a$.} \\
%                               & \Leftrightarrow \mathcal A'\mbox{ has a synchronizing word in $ab^*a$.} \\
%                               & \Leftrightarrow \mathcal A'\mbox{ has a synchronizing word in $a^*b^*a^*$}
% \end{align*} 

\begin{toappendix}

Next, we supply the proof of the claim made in the proof sketch of Lemma~\ref{lem:np_hardness}
from the main text.

\medskip

\noindent\underline{Claim:} 
 For the constructed automaton $\mathcal A'$ from
 the proof sketch of Lemma~\ref{lem:np_hardness} in the main text, we have:
\begin{align*}
    \exists m \ge 0 : \delta(S, c^m) \subseteq T 
                               & \Leftrightarrow \mathcal A'\mbox{ has a synchronizing word in $ab^{r_2}(b^{p_2})^*a$.} \\
                               & \Leftrightarrow \mathcal A'\mbox{ has a synchronizing word in $ab^*a$.} \\
                               & \Leftrightarrow \mathcal A'\mbox{ has a synchronizing word in $a^*b^*a^*$}
\end{align*} 
%\begin{quote}
%\begin{proof}[Proof of Claim]
 \emph{Proof of the Claim.}
 First, suppose $\delta(S, c^m) \subseteq T$.
 %Choose any $a^r \in A_1$ with $r > 0$
 %and $a^s \in A_3$ with $s > 0$. 
 By construction of $\mathcal A'$,
 for any $q, q' \in Q$,
 \begin{equation}\label{eqn:transition_Astar}
  \delta(q, c) = q'  \mbox{ in $\mathcal A$}  \Leftrightarrow \delta'(q, b^{p_2}) = q'  \mbox{ in $\mathcal A'$} .
 \end{equation}
 Also, $\delta'(Q'\setminus\{t\},a) = S_{r_2}$
 and $\delta'(S_{r_2}, b^{r_2}) = S$.
 Combining these facts, we find
 \[
  \delta'(Q', ab^{r_2}b^{p_2m}) \subseteq T \cup \{t\}. 
 \]
 A final application of $a$ then maps
 all states in $T$ to the single sink %(and synchronizing) 
 state~$t$.
 
 Clearly, as $ab^{r_2}(b^{p_2})^*a \subseteq a b^* a$
 and $a b^* a \subseteq a^* b^* a^*$, the next two implications are shown.
 Finally, to complete the argument, let $u = a^{p} b^q a^r$ be a synchronizing word, $p,q,r \ge 0$.
 Then, as $t$ is a sink state, $\delta'(Q', u) = \{t\}$.
 The only way to enter $t$ from the outside is to read $a$ at least once, and 
 as $t$ is a sink state, we have $\delta'(Q', a^p b^q a^r) = \{t\}$.
 Also, as for $q \notin T$, we have $\delta'(q, a) \notin T$,
 we must have $\delta'(Q', a^p b^q) \subseteq T \cup \{t\}$,
  or more specifically, $\delta'(Q' \setminus \{t\}, a^p b^q) \subseteq T$.
  We distinguish two cases for~$p$.
 
 \begin{enumerate}
 \item If $p = 0$, then, in particular, $\delta'(S, b^q)  \subseteq T$.
     By construction of $\mathcal A'$, for any $q \in Q$,
 \[
  \delta'(q, b^n) \in Q
 \]
 if and only if $n \equiv 0\pmod{p_2}$.
 So, $q = p_2 m$ for some $m \ge 0$.
 Hence, by Equation~\eqref{eqn:transition_Astar} above from the first case, in $\mathcal A$,
 we find $\delta(S, c^m) \subseteq T$.
 
 \item  If $p > 0$, then $\delta'(Q' \setminus\{t\}, a^p) = S_{r_2}$.
 The only way to leave any state in $S_{r_2}$
 is to read $b$, which transfers $S_{r_2}$ to $S_{r_2-1}$.
 Reasoning similarly, we find that we have to read in $b$
 at least $r_2$ many times, which finally maps $S_{r_2}$
 onto $S_0 = S$. So, $q \ge r_2$. By construction of $\mathcal A'$, for any $q \in Q$,
 \[
  \delta'(q, b^n) \in Q
 \]
 if and only if $n \equiv 0\pmod{p_2}$.
 So, as $\delta'(S, b^{q - r_2}) \subseteq T$, $q - r_2 = p_2 m$ for some $m \ge 0$.
 Hence, by Equation~\eqref{eqn:transition_Astar} above, in $\mathcal A$,
 we find $\delta(S, c^m) \subseteq T$.
 \end{enumerate}
This ends the proof of the claim. \emph{[End, proof of the Claim.]}
%\end{proof}
%\end{quote}
\end{toappendix}

%Finally, we show that we have some $m \ge 0$
%such that $\delta(S, c^m) \subseteq T$
%if and only if $\mathcal A'$ has a synchronizing word in $L$:
 Now, suppose $\delta(s, c^m) \subseteq T$ for some $m \ge 0$.
 By the above, $\mathcal A'$ 
 has a synchronizing word $u$ in $ab^{r_2}(b^{p_2})^*a$.
%  As $A_1$ is non-empty
%  and does not equal $\{\varepsilon\}$, we can choose $a^p \in A_1$ with $p > 0$.
%  Similarly, choose $a^r \in A_3$.
%  Then
%  $
%   a^{p-1} u a^{r-1} v \in A_1 A_2 A_3 \subseteq L,
%  $
%  and %by Lemma~\ref{lem:append_sync}, 
%  this word also synchronizes $\mathcal A'$.
 Then, $a^{r_1 - 1}u a^{r_3-1} \in L(\mathcal B)$ also synchronizes~$\mathcal A'$.
 
 
 Conversely, suppose we have a synchronizing word $w \in L$
 for $\mathcal A'$.
 As $L \subseteq a^* b^* a^*$
 by the above equivalences,
 $\delta(S, c^m) \subseteq T$
 for some $m \ge 0$. \qed
\end{proof}
