\documentclass[runningheads,envcountsame]{llncs}
%\documentclass{article}
%
%\usepackage{graphicx}
% Used for displaying a sample figure. If possible, figure files should
% be included in EPS format.
%
% If you use the hyperref package, please uncomment the following line
% to display URLs in blue roman font according to Springer's eBook style:
% \renewcommand\UrlFont{\color{blue}\rmfamily}

\usepackage{amssymb}
\usepackage{amsmath}
\usepackage{comment}
%\usepackage[color=gray!27]{todonotes}
\usepackage[disable]{todonotes}
\usepackage[utf8]{inputenc}
%\usepackage[english,russian]{babel}
\usepackage{inputenc}
\usepackage{shuffle}
\usepackage{multirow}
\usepackage{listings}
\usepackage{mathabx}
%\usepackage[sectionbib, square,sort,comma,numbers]{natbib}
%\usepackage{adjustbox}

%\usepackage{thmtools, thm-restate}
\usepackage{hyperref}

\usepackage{shuffle}

\usepackage{tikz}
\usetikzlibrary{positioning,shadows,arrows}
\usetikzlibrary{arrows,automata,positioning,calc}


% https://tex.stackexchange.com/questions/7262/diagonally-divided-table-cell?noredirect=1&lq=1
\usepackage{diagbox}
\usepackage{slashbox}
\usepackage{tikz}
\usetikzlibrary{matrix}


%todo extradatei für review-bemerkungen
\lstset{%
  language=[LaTeX]TeX,
  backgroundcolor=\color{gray!10},
  basicstyle=\ttfamily,
  breaklines=true,
  columns=fullflexible
}


%\usetikzlibrary{arrows,shapes,automata,positioning}

\newcommand{\perm}{\operatorname{perm}}
\newcommand{\stc}{\operatorname{sc}}
\newcommand{\lastsym}{\operatorname{last}}




\usepackage{blindtext,tikz}
\usetikzlibrary{calc}


% \usepackage{apxproof}
% \theoremstyle{plain}
% \newtheoremrep{theorem}{Theorem}%[section]
% \newtheoremrep{proposition}[theorem]{Proposition}
% \newtheoremrep{lemma}[theorem]{Lemma}
% \newtheoremrep{claim}[theorem]{Claim}
% \newtheoremrep{conjecture}[theorem]{Conjecture}
% \newtheoremrep{corollary}[theorem]{Corollary}
% \newtheoremrep{definition}[theorem]{Definition}
 
 
\usepackage{apxproof}
\theoremstyle{plain}
%\newcounter{theorem2}
% \newtheoremrep{theorem}{Theorem}[section] % wie mit einem zähler durchnummerieren?
% \newtheoremrep{proposition}[theorem]{Proposition}
% \newtheoremrep{lemma}[theorem]{Lemma}
% \newtheoremrep{claim}[theorem]{Claim}
% \newtheoremrep{conjecture}[theorem]{Conjecture}
% \newtheoremrep{corollary}[theorem]{Corollary}
% \theoremstyle{definition}
% \newtheoremrep{definition}[theorem]{Definition}
%\theoremstyle{remark}
%\newtheoremrep{example}[theorem]{Example}
%\newtheoremrep{remark}[theorem]{Remark}
\newtheoremrep{theorem}{Theorem}
\newtheoremrep{proposition}[theorem]{Proposition}
\newtheoremrep{lemma}[theorem]{Lemma}
\newtheoremrep{claim}[theorem]{Claim}
\newtheoremrep{conjecture}[theorem]{Conjecture}
\newtheoremrep{corollary}[theorem]{Corollary}
\theoremstyle{definition}
\newtheoremrep{definition}[theorem]{Definition}
 
% https://tex.stackexchange.com/questions/104098/create-a-claim-environment
\newenvironment{claiminproof}[1]{\medskip\par\noindent\underline{Claim:}\space#1}{}
\newenvironment{claimproof}[1]{\begin{quote}\par\noindent\emph{Proof of the Claim:}\space#1}{[\emph{End, Proof of the Claim}]\end{quote}}%\newline}
% {\leavevmode\unskip\penalty9999 \hbox{}\nobreak\hfill\quad\hbox{$\blacksquare$}}
 
% \usepackage[printwatermark]{xwatermark}
% \usepackage{xcolor}
% \usepackage{graphicx}
% \newwatermark[pagex={1},color=gray!20,angle=45,scale=1.5,xpos=-10,ypos=70]{This paper eligible for the best student paper award, \\ as I am a PhD student
% under the supervision of Prof. Dr. Henning Fernau.} %Submission for review. \\ Full proofs in Appendix.}



\usepackage{fancyhdr}

 
\DeclareMathOperator{\lcm}{lcm}

%\def\baselinestretch{0.99}
%\linespread{0.9}
  
 %\usepackage[small,compact]{titlesec}
%\usepackage[small]{titlesec}

%\usepackage[text={12cm,20cm}]{geometry}
% \usepackage[compact]{titlesec}
% \titlespacing*{\section}{0pt}{2ex}{1ex}
%\titlespacing*{\subsection}{0pt}{1.6ex}{0.7ex}

% http://www.terminally-incoherent.com/blog/2007/09/19/latex-squeezing-the-vertical-white-space/
% http://www-h.eng.cam.ac.uk/help/tpl/textprocessing/squeeze.html
% https://robjhyndman.com/hyndsight/squeezing-space-with-latex/
   
 
%\def\dotminus{\mathbin{\ooalign{\hss\raise1ex\hbox{.}\hss\cr
%  \mathsurround=0pt$-$}}} 
 
% https://tex.stackexchange.com/questions/103735/list-of-todos-todonotes-is-empty-with-llncs?noredirect=1
%\setcounter{tocdepth}{1}



% https://tex.stackexchange.com/questions/186677/big-shuffle-symbol
% large ops, copied from shuffle font package
\DeclareFontFamily{U}{bigshuffle}{}
\DeclareFontShape{U}{bigshuffle}{m}{n}{
  <5-8> s*[1.7] shuffle7
  <8->  s*[1.7] shuffle10
}{}
\DeclareSymbolFont{BigShuffle}{U}{bigshuffle}{m}{n}
\DeclareMathSymbol\bigshuffle{\mathop}{BigShuffle}{"001}
\DeclareMathSymbol\bigcshuffle{\mathop}{BigShuffle}{"002}

\newcommand{\suff}{\operatorname{Suff}}
\newcommand{\factor}{\operatorname{Fact}}

\newcommand{\pref}{\operatorname{Pref}}

\newcommand{\Orb}{\operatorname{Orb}}

\newcommand{\NC}{\textsf{NC}}
\newcommand{\NL}{\textsf{NL}}
\newcommand{\NP}{\textsf{NP}}
\newcommand{\PSPACE}{\textsf{PSPACE}}
\newcommand{\NPSPACE}{\textsf{NPSPACE}}
\newcommand{\PTIME}{\textsf{P}}
\newcommand{\XP}{\textsf{XP}}


% https://latex.org/forum/viewtopic.php?t=10877
% https://latex.org/forum/viewtopic.php?f=47&t=10862
% https://tex.stackexchange.com/questions/36423/random-unwanted-space-between-paragraphs
%\raggedbottom



% https://tex.stackexchange.com/questions/255673/problem-definition-environment
% https://www.overleaf.com/learn/latex/Environments
% https://de.overleaf.com/learn/latex/Counters
% https://en.wikibooks.org/wiki/LaTeX/Counters
% https://de.wikibooks.org/wiki/LaTeX-W%C3%B6rterbuch:_refstepcounter
\usepackage{tabularx,lipsum,environ,amsmath,amssymb}

%\usepackage{natbib}
%\usepackage{biblatex}
%\addbibresource{ms.bib}


\newcounter{problemcounter}
\makeatletter
\newcommand{\problemtitle}[1]{\gdef\@problemtitle{#1}}% Store problem title
\newcommand{\probleminput}[1]{\gdef\@probleminput{#1}}% Store problem input
\newcommand{\problemquestion}[1]{\gdef\@problemquestion{#1}}% Store problem question
\NewEnviron{decproblem}{
  \refstepcounter{problemcounter}
  \problemtitle{}\probleminput{}\problemquestion{}% Default input is empty
  \BODY% Parse input
  \par\addvspace{.5\baselineskip}
  \noindent
  \begin{tabularx}{\textwidth}{@{\hspace{\parindent}} l X c}
    \multicolumn{2}{@{\hspace{\parindent}}l}{\textbf{Decision Problem \theproblemcounter:} \@problemtitle} \\% Title
    \textbf{Input:} & \@probleminput \\% Input
    \textbf{Question:} & \@problemquestion% Question
  \end{tabularx}
  \par\addvspace{.5\baselineskip}
}
\makeatother
 
 
 
% % https://tex.stackexchange.com/questions/104098/create-a-claim-environment
 \newenvironment{myclaiminproof}[1]{\medskip\par\noindent\underline{Claim:}\space#1}{}
 \newenvironment{myclaimproof}[1]{\begin{quote}\par\noindent\emph{Proof of the Claim:}\space#1}{[\emph{End, Proof of the Claim}]\end{quote}}%\newline}
% % {\leavevmode\unskip\penalty9999 \hbox{}\nobreak\hfill\quad\hbox{$\blacksquare$}}




\renewcommand{\headrulewidth}{0pt}
\fancypagestyle{AllPages}{
\chead{2016 IEEE/ACM International Conference on Advances in Social Networks Analysis and Mining (ASONAM)}
}
\fancypagestyle{FirstPage}{
\chead{2016 IEEE/ACM International Conference on Advances in Social Networks Analysis and Mining (ASONAM)}
\lfoot{IEEE/ACM ASONAM 2016, August 18-21\\2016, San Francisco, CA,     USA\\
978-1-5090-2846-7/16/\$~\copyright~2016 IEEE}
}

\chead{2016 IEEE/ACM International Conference on Advances in Social Networks Analysis and Mining (ASONAM)}


\begin{document}

%
\title{Computational Complexity of Synchronization under Sparse Regular Constraints}
%\title{State Complexity of Projected Languages of Permutation Automata}
%\title{State Complexity of Projection on Permutation Automata}
\titlerunning{Synchronization under Sparse Regular Constraints}

%
%\titlerunning{Abbreviated paper title}
% If the paper title is too long for the running head, you can set
% an abbreviated paper title here
%
\author{Stefan Hoffmann\orcidID{0000-0002-7866-075X}}
%
\authorrunning{S. Hoffmann}
% First names are abbreviated in the running head.
% If there are more than two authors, 'et al.' is used.
%
\institute{Informatikwissenschaften, FB IV, 
  Universit\"at Trier, Germany, 
  \email{hoffmanns@informatik.uni-trier.de}}
%
\maketitle              % typeset the header of the contribution
%
\begin{abstract}
%  We investigate the constrained synchronization problem for weakly acyclic, or partially ordered,
%  input automata. We show that, for input automata of this type, the problem is always
%  in $\NP$.
%  Furthermore, we give a full classification of the realizable complexities for constraint
%  automata with at most two states and over a ternary alphabet.
%  For certain constraint languages, for which the general problem is $\PSPACE$-complete,
%  for weakly acyclic automata we get $\NP$-complete problems, whereas there are also
%  problems that are $\PSPACE$-complete in general, but for which
%  it is polynomial time solvable in our setting.
%  %when only weakly acyclic automata
%  %are considered as input. 
%  %In the course of our investigation, 
%  We also investigate
%  two problems related to subset synchronization, namely if there exists
%  a word mapping all states into a given target subset of states, and
%  if there exists a word mapping one subset into another. Both problems
%  are $\PSPACE$-complete in general, but in our setting the former is polynomial time solvable and the latter is $\NP$-complete.
%  %
%  % genauer bescchreiben
%  % sync wort in vorgegebener
%  % reg. sprache?
 The constrained synchronization problem (CSP) asks
 for a synchronizing word of a given input automaton
 contained in a regular set of constraints. It could be viewed
 as a special case of synchronization of a discrete event system
 under supervisory control.
 Here, we study the computational complexity of this %the constrained synchronization
 problem for the class of sparse regular constraint languages.
 We give a new characterization of sparse regular sets, which
 equal the bounded regular sets,
 and derive a full classification of the computational complexity
 of CSP for letter-bounded regular constraint
 languages, which properly
 contain the strictly bounded regular languages.
 %In addition, we derive a polynomial time decision procedure
 %for the complexity of the constrained synchronization problem, given
 %a constraint automaton recognizing a strictly bounded regular language
 %as input.
 Then, we introduce strongly self-synchronizing codes
 and investigate CSP for  bounded languages induced by these codes.
 With our previous result, we deduce a full classification
 for these languages as well.
 In both cases, depending on the constraint language, our problem
 becomes $\NP$-complete or polynomial time solvable.
 %Additionally, we state a new characterization of bounded languages.
 
 %
 % keywords überarbeiten
 % und hinweis für student best paper
\keywords{automata theory \and constrained synchronization \and computational complexity \and sparse languages \and bounded languages \and strongly self-synchronizing codes} 
\end{abstract}
%
%
%

%\thispagestyle{FirstPage}
% \thispagestyle{fancy}
% \chead{}%\hspace*{-4cm} 
% \lhead{}
% \definecolor{mygray}{gray}{0.6}
% \definecolor{mypink1}{rgb}{0.558, 0.188, 0.278}
% \lfoot{\footnotesize \textcolor{mypink1}{Paper eligible for best student \\ paper award. I am a PhD student \\
%  under the supervision of \\ Prof. Dr. Henning Fernau.}}

% https://tex.stackexchange.com/questions/7400/watermark-on-first-page-in-left-margin-like-arxiv

%\blinddocument 



\section{Introduction} %contribution
\label{sec:introduction}




\section{Introduction}  \label{sec:introduction}

\newcommand\inexpIntro[3]{#1?(#2,#3).}
\newcommand\rinexpIntro[3]{*#1?(#2,#3).}
\newcommand\outexpIntro[3]{#1!(#2,#3).}
\newcommand\outatomIntro[3]{#1!(#2,#3)}

We propose a fully automated method for proving termination of \(\pi\)-calculus processes.
Although there have been a lot of studies on termination analysis for the \(\pi\)-calculus
and related calculi~\cite{Deng06IC,Demangeon07,SangiorgiTermination,KobayashiHybrid,Yoshida04IC,DBLP:journals/jlp/DemangeonHS10,Venet98SAS}, most of them have been rather theoretical,
and there have been surprisingly little efforts in developing  fully automated termination
verification methods and tools based on them. To our knowledge,
Kobayashi's \typical{}~\cite{TyPiCal,KobayashiHybrid} is the only exception that
can prove termination of \(\pi\)-calculus processes (extended with natural numbers)
fully automatically, but its termination analysis is quite limited (see Section~\ref{sec:relatedwork}).

Our method is based on a reduction to termination analysis for sequential programs:
we translate a \(\pi\)-calculus process \(P\) to a sequential program \(S_P\), so that
if \(S_P\) is terminating, so is \(P\). The reduction allows us to use
powerful, mature methods and tools
for termination analysis of sequential programs~\cite{heizmann2016ultimate,freqterm,DBLP:conf/lics/PodelskiR04,Kuwahara2014Termination,DBLP:journals/cacm/CookPR11}.

The idea of the translation is to convert a chain of communications on replicated input
channels to a chain of recursive function calls of the target sequential program.
Let us consider the following Fibonacci process:
\begin{align*}
    & \rinexpIntro{\fib}{n}{r}
        \ifexp{n<2}{ \soutatom{r}{1} \\ &\quad}
                   { \nuexp{s_1} \nuexp{s_2} (\outatomIntro{\fib}{n-1}{s_1} \PAR \outatomIntro{\fib}{n-2}{s_2} \PAR \sinexp{s_1}{x}\sinexp{s_2}{y}\soutatom{r}{x+y}) \\}
    & \PAR \outatomIntro{\fib}{m}{r}
\end{align*}
Here, the process
$\rinexpIntro{\fib}{n}{r} \ldots$ is a function server that computes the \(n\)-th Fibonacci number
in parallel and returns the result to \(r\),
and $\outatom{\fib}{m}{r}$ sends a request for computing the \(m\)-th Fibonacci number;
those who are not familiar with the syntax of the \(\pi\)-calculus may wish to consult
Section~\ref{sec:targetlanguage} first.
To prove that the process above is terminating for any integer \(m\),
it suffices to show that there is no infinite chain of communications on $\fib$:
\[
    \fib(m,r) \to \fib(m_1,r_1) \to \fib(m_2,r_2) \to \cdots.
\]
We convert the process above to the following program:\footnote{The actual translation
  given later is a little more complex.}
\begin{verbatim}
 let rec fib(n) = if n<2 then () else (fib(n-1) [] fib(n-2)) in
 fib(m)
\end{verbatim}
Here, \texttt{[]} represents the non-deterministic choice.
Note that, although the calculation of Fibonacci numbers is not preserved,
for each chain of communications on \texttt{fib}, there is a corresponding
sequence of recursive calls:
\[
\mathtt{fib}(m) \to \mathtt{fib}(m_1) \to \mathtt{fib}(m_2) \to \cdots.
\]
Thus, the termination of the sequential program above implies the termination of
the original process.
As shown in the example above, (i) each communication on a replicated input channel
is converted to a function call, (ii) each communication on a non-replicated input
channel is just removed (or, in the actual translation, replaced by a call of
a trivial function defined by \(f(\seq{x})=(\,)\)), and (iii) parallel composition
is replaced by a non-deterministic choice.
We formalize the translation outlined above and prove its correctness.

The basic translation sketched above sometimes loses too much information.
For example, consider the following process:
\begin{align*}
    & \rinexpIntro{\pre}{n}{r} \soutatom{r}{n-1} \\
    & \PAR \rinexpIntro{f}{n}{r} \ifexp{n<0}{ \soutatom{r}{1} }
                                       { \nuexp{s} (\outatomIntro{\pre}{n}{s} \PAR \sinexp{s}{x}\outatomIntro{f}{x}{r}) } \\
    & \PAR \outatomIntro{f}{m}{r}
\end{align*}
The translation sketched above would yield:
\begin{verbatim}
  let pred(n) = n-1 in
  let rec f(n) = if n<0 then () else (pred(n) [] f(*)) in
  f(m)
\end{verbatim}
Here, \texttt{*} represents a non-deterministic integer: since we have removed
the input $\sinatom{s}{x}$, we do not have information about the value of \( x \).
As a result, the sequential program above is non-terminating, although the original
process is terminating.
To remedy this problem, we also refine the basic translation above by using a refinement
type system for the \(\pi\)-calculus. Using the refinement type system,
we can infer that the value of \(x\) in the original process is less than \(n\),
so that we can refine the definition of \texttt{f} to:
\begin{verbatim}
 let rec f(n) = ... else (pred(n) [] let x=* in assume(x<n);f(x))
\end{verbatim}
The target program is now terminating, from which
we can deduce that the original process is also terminating.
We have implemented an automated tool based on the refined translation above.

The contributions of this paper are summarized as follows.
\begin{itemize}
\item The formalization of the basic translation from the \(\pi\)-calculus
  (extended with integers) to sequential programs, and a proof of its correctness.
\item The formalization of a refined translation based on a refinement type system.
\item An implementation of the refined translation, including automated refinement type
  inference based on CHC solving, and experiments to evaluate the effectiveness of
  our method.
\end{itemize}

The rest of this paper is structured as follows.
Section~\ref{sec:targetlanguage} introduces the source and target languages
of our translation.
Section~\ref{sec:approach} 
formalizes the basic translation, and proves its correctness.
Section~\ref{sec:refinement} refines the basic translation by using a refinement type system.
Section~\ref{sec:implementation} reports an implementation and experiments.
Section~\ref{sec:relatedwork} discusses related work,
and Section~\ref{sec:conclusion} concludes the paper.


\section{Preliminaries and Definitions}
\label{sec:preliminaries}

\section{Preliminaries}\label{chpt:preliminiaries}
In this chapter we will introduce some of the mathematical background and notation needed for this thesis. In particular, we will shortly introduce the differential geometric description of spacetime in Section \ref{sec:spacetime_geometry} and give an introduction to the notion of global hyperbolicity and its connection to Green- and normally-hyperbolic operators in Section \ref{sec:global_hyperbolicity}. In a bit more detail, we will introduce the notion of differential forms and give explicit definitions, also in terms of an index based notation, in Section \ref{sec:differential_forms}. For completeness, in Section \ref{sec:cat-theory}, we present basic definitions of category theory. The reader familiar with these topics can safely skip this chapter and refer to it when interested in the chosen conventions.
%
%
%
%
%%%%%%
%%SPACTIME GEOMETRY
%%%%%
%
%
%
\subsection{Spacetime geometry}\label{sec:spacetime_geometry}
In GR, the universe is mathematically described as a four dimensional \emph{spacetime}, consisting of a smooth, four dimensional manifold \gls{M} (assumed to be Hausdorff, connected, oriented, time-oriented and para-compact) and a Lorentzian metric $g$. We will assume the signature of the Lorentzian metric $g$ to be $(-,+,+,+)$. The Levi-Civita connection on $(\M,g)$ is as usual denoted by \gls{nabla}.
Throughout this thesis, we treat spacetime as fixed, implementing a gravitational background determined classically by Einstein's field equations. Hence, we neglect any back-reaction of the fields on the metric, both in the quantum and the classical case. In that sense, we treat the fields as \emph{test fields}.\par
For the basic mathematical theory regarding Lorentzian manifolds, we refer to the literature: An introduction to the topic with an emphasis on the physical application in GR is for example given in \cite{wald_GR} and \cite{carroll_spacetime-and-gr}.
Here, we will shortly recap the notion of a tangent space and tangent bundle and generalize to the notion of a vector bundle which we will use in the general description of normally hyperbolic operators and differential forms.
In the following, we generalize the setting to an arbitrary smooth manifold $\N$ of dimension $N$ with either Lorentzian or Riemannian metric $k$.\par
%
%
A \emph{tangent vector} $v_x$ at point $x \in \N$ is a linear map $v_x : C^\infty(\N , \IR) \to \IR$ that obeys the Leibniz rule, that is, for $f,g \in C^\infty (\N,\IR)$ it holds $v_x(fg) = f(x)v_x(g) + v_x(f)g(x)$.
We define the \emph{tangent space} \gls{TxN} of $\N$ at $x$ as the real $N$-dimensional vector space of all tangent vectors at point $x$.
The disjoint union of all tangent spaces is called the \emph{tangent bundle} \gls{TN} of $\N$ and is itself a manifold of dimension $2N$. A \emph{vector field} is a map $v: \N \to T\N$ such that $v(x) \in T_x\N$.
The respective dual spaces, that is the space of all linear functionals, the \emph{co-tangent space} and the \emph{co-tangent bundle}, are denoted by \gls{TsxN} and \gls{TsN} respectively.\par
%
For Lorentzian manifolds, we call a tangent vector $v$ at $x \in \N$ \emph{timelike} if $k_{\mu \nu} v^\mu v^\nu < 0$, \emph{spacelike} if $k_{\mu \nu} v^\mu v^\nu > 0$ and \emph{null} (or lightlike) if $k_{\mu \nu} v^\mu v^\nu = 0$. At every point $x \in \N$, we define the set of all \emph{causal}, that is, either timelike or null, tangent vectors in the tangent space at $x$. This set is called the \emph{light cone} at $x$ and it is split up into two distinct parts, one that we call the future light cone, and one that we call the past light cone at $x$. Since we assume the manifold to be time orientable, there exists a smooth vector field $t$ that is timelike at every $x \in \N$. Given this time orientation, we identify the future (past) light cone with the set of tangent vectors $v \in T_x\N$ such that $k_{\mu\nu} v^\mu t^\nu < 0$ (respectively $> 0$). Therefore, a tangent vector $v$ at $x$ is called \emph{future directed} (past directed) if it lies in the future (past) light cone at $x$.\\
Accordingly, a curve $\gamma : I \to \N$ is called timelike (spacelike, null, causal, future or past directed) if its tangent vector $\dot{\gamma}$ is timelike (spacelike, null, causal, future or past directed) at every $x \in \N$.  For every point $x \in \N$ we define the \emph{causal future/past} \gls{causalfuturepast} of $x$ as the set of all points $q \in \N$ that can be reached by a future directed causal curve originating in $x$. For any subset $S \in \N$ we define $J^\pm (S) = \bigcup_{x \in S} J^\pm(x)$ and $J(S) = J^+(S) \cup J^- (S)$. Finally, the future/past domain of dependence $\gls{futurepastdomainofdependence}$ of a set $S \subset \N$ is the set of all points $x \in \N$ such that every inextendible causal curve through $x$ intersects $S$. The \emph{domain of dependence} \gls{domainofdependence} of $S$ is the union of the future and past domain of dependence of the set $S$.
For more details on the causal structure of spacetime we refer to for example \cite[Chapter 8]{wald_GR}.\par
%
%
%
The notion of tangent bundles can be generalized to the notion of a vector bundle. Instead of ``attaching'' the vector spaces $T_x \N$ to every point $x$ of the manifold, we allow for the occurrence of arbitrary vector spaces, called the fibres of the vector bundle. A vector bundle then consists of the base manifold, in our case $\N$, the total space and a map $\pi$ from the total space to the base manifold, that can be locally trivialized. At each point of the base manifold, the pre-image of $\pi$ is the fibre of the vector bundle. To be precise we define, following \cite{rudolph_schmidt}:
\begin{definition}[Vector bundle]
	A smooth \emph{vector bundle} over $\N$ is a tuple $\gls{vectorbundle} = (E,\N, \pi)$, where $E$ is a smooth manifold and $\pi : E \to \N$ is a smooth surjective map satisfying:
	\begin{enumerate}
		\item For every $x \in \N$, $\pi^{-1}(x)$ is a vector space, called the fibre of the bundle at point $x$.
		\item There exists a finite dimensional vector space $F$, an open covering $\left\{ U_\alpha\right\}_\alpha$ of $\N$ and a family of diffeomorphisms $\chi_\alpha : \pi^{-1}(U_\alpha) \to U_\alpha \times F$ such that for all $\alpha$ it holds $\chi_\alpha \comp \text{pr}_1 =  \restr{\pi}{\pi^{-1}(U_\alpha)}$ and for every $x \in \N$ the map $\text{pr}_2 \comp \restr{\chi_\alpha}{\pi^{-1}(x)} : \pi^{-1}(x) \to F$ is linear.
	\end{enumerate}
\end{definition}
Here, the maps $\text{pr}_1$ and $\text{pr}_2$ denote the projection onto the first respectively second component of an element in $U_\alpha \times F$. The properties graphically mean that \emph{locally}, the vector bundle ``looks like" the product of the base manifold with the fibre. The tuples $(U_\alpha, \chi_\alpha)$ are called \emph{local trivializations} of the vector bundle. Like for vector spaces, we can define the sum and product of vector bundles, by using the according vector space definitions on the fibres of the bundle.\par
Let $\mathfrak{X}, \mathfrak{Y}$ be vector bundles over $\N$ with fibres $X_x$ and $Y_x$ at $x \in \N$. We denote by \gls{whitneysum} the \emph{Whitney sum} of the two vector bundles - the vector bundle over $\N$ whose fibres are given by the direct sum $X_x \oplus Y_x$. Similarly, one obtains the local trivializations of the Whitney sum from the trivializations of $\mathfrak{X}, \mathfrak{Y}$ and direct sums.\par
Accordingly, let $\mathfrak{X}, \mathfrak{Y}$ be vector bundles over $\N$ and $\widetilde{\N}$, with fibres $X_x$ and $Y_{\tilde{x}}$ at $x \in \N$, $\tilde{x} \in \widetilde{\N}$ respectively. We denote by \gls{outerproductbundle} the \emph{outer product} of the two vector bundles - the vector bundle over $\N \times \widetilde{\N}$ whose fibres are given by the tensor products $X_x \otimes Y_x$. Similarly, one obtains the local trivializations of the outer product from the trivializations of $\mathfrak{X}, \mathfrak{Y}$ and tensor products. \par
%
Finally, we generalize the notion of vector fields:
\begin{definition}[Sections of vector bundles]
Let $\mathfrak{X}=(E,\N,\pi)$ be a vector bundle with fibres $X_x=\pi^{-1}(x)$ at $x \in \N$. A \emph{smooth section} of the vector bundle is a smooth map $\gamma : \N \to E$ such that $\gamma(x) \in X_x$ for all $x \in \N$. The \emph{vector space of smooth sections} of $\mathfrak{X}$ is denoted by \gls{gammax}, the one with compactly supported sections is as usual denoted by \gls{gammaxzero}.
\end{definition}
In this language, a vector field $v$ is just a smooth section of the tangent bundle of a manifold, $v \in \Gamma(T\N)$. One may therefore identify the physical notion of fields with smooth sections of vector bundles. This point of view will be used to define the notion of differential forms in Section \ref{sec:differential_forms}.\par
In this thesis, we usually are interested in complex valued functions (or sections in general). Therefore, we view all occurring vector bundles as complex, in the sense that we take two distinct copies of the vector bundle, one representing the real, one the imaginary part of the bundle. A section of that complex vector bundle is just a pair of two sections of the real vector bundle under consideration. From now, if not specified explicitly, we will view all vector bundles, including the tangent bundle $T\N$, as complex vector bundles. Accordingly, smooth sections of those bundles will in general be complex valued.
%
%
%
%
%
%
%
%
%%%%%%%
%%PARTIAL DIFFERENTIAL OPERATORS AND GLOBAL HYPERBOLICITY
%%%%%%%
%
%
%
\subsection{Partial differential operators and global hyperbolicity}\label{sec:global_hyperbolicity}
When dealing with field theories, whether classical or quantum, one is, of course, interested in the dynamics of the fields. These are usually described by some partial differential equation, often of second order. In the following, we give a short introduction to the theory of certain partial differential operators acting on smooth sections of a vector bundle over the spacetime $(\M,g)$.\par
%
As we have seen, these smooth sections are generalizations of the notion of a field.  In the following, let $\mathfrak{X}$ denote a vector bundle over the manifold $\M$ and let $P: \Gamma(\mathfrak{X}) \to \Gamma(\mathfrak{X})$ be a partial differential operator acting on smooth sections of the bundle. As in the case of flat spacetime, we are interested in basic questions regarding the differential equation $Pf = j$, for example: Can we formulate a (globally) well posed initial value problem? Does the differential equation possess (unique) solutions? To answer these questions, we will now restrict to the case where $P$ is linear and of second order, as it is often the case in physical applications. One can show that for a certain class of such operators, namely normally hyperbolic partial differential operators of second order, we can rigorously treat these questions.\par
Choosing local coordinates $x=(x_\mu)$ on $\M$ and a local trivialization of $\mathfrak{X}$, a linear partial differential operator of second order is called \emph{normally hyperbolic} if it takes the form
\begin{align}
	P = - \sum_{\mu,\nu} g^{\mu \nu} \partial_\mu \partial_\nu + \sum_{\alpha} A_\alpha (x) \partial_\alpha + B(x) \formspace,
\end{align}
where $A_\alpha$ and $B$ are matrix-valued coefficients depending smoothly on the coordinate $x$ (see. \cite[Chapter 1.5]{baer_ginoux_pfaeffle}). One can also formulate a coordinate independent definition in terms of the principal symbol, which we will not present here (see for example \cite[Section 1.5]{baer_ginoux_pfaeffle} ). \par
%
Normally hyperbolic operators possess unique fundamental solutions (see for example the fundamental solutions to the wave operator as noted in Lemma \ref{lem:fundamental_solution_wave_operator}). These fundamental solutions fulfill certain physically important properties, such as a finite propagation speed smaller than the speed of light. Furthermore, specifying the initial data on some space-like hypersurface $X \in  \M$ specifies a unique solution on the domain of dependence $D(X)$ of $X$. Due to these properties, one often calls normally hyperbolic operators just \emph{wave operators}. But to state a \emph{globally} well posed initial value problem for a wave equation, we need to restrict the class of spacetimes $\M$ under consideration to those that possess space-like hypersurfaces $X$ whose domain of dependence is all of the spacetime, $D(X) = \M$. This leads to the notion of \emph{globally hyperbolic} spacetimes:
\begin{definition}[Global Hyperbolicity]
	A spacetime $\M$ is called \emph{globally hyperbolic} if there exists a Cauchy surface $\gls{sigma}$ in $\M$.
\end{definition}
\noindent Here, a Cauchy surface is a space-like hypersurface $\Sigma \subset \M$ such that every inextendible causal curve $\gamma$ intersects $\Sigma$ exactly once. One can show that Cauchy surfaces fulfill the desired property mentioned above, that is,  $D(\Sigma) = \M$. Furthermore, one can show that any globally hyperbolic spacetime $\M$ is foliated by a one-parameter family $\left\{ \Sigma_t \right\}_t$ of Cauchy surfaces (see for example \cite[Theorem 8.3.14]{wald_GR}). \par
In physical applications, one often finds the dynamics of a theory to be described by wave operators. Most prominently, the Klein-Gordon operator $(\square + m^2)$ acting on scalar fields, or its generalization, the wave operator acting on differential forms introduced in Section \ref{sec:differential_forms}, is normally hyperbolic. But there are also important physical field theories that are not described by wave operators, such as the Proca field treated in this thesis. It turns out that the Proca operator (see Definition \ref{def:proca_operator}) is a so called \emph{Green-hyperbolic} operator. These are again partial differential operators $P$ of second order acting on smooth sections of some vector bundle, such that $P$ (and its dual $P'$) posses fundamental solutions. Obviously, normally hyperbolic operators are Green-hyperbolic, but the opposite is not true. One can generalize some results obtained by studying normally hyperbolic operators to Green-hyperbolic operators. An introduction to this topic is given in \cite{baer_green-hyperbolic}, where it is also shown that the Proca operator is Green-hyperbolic but not normally hyperbolic.\par
For our application, the notion of Green-hyperbolicity is not of vast importance, but it is worth mentioning that there exists a more detailed mathematical background on the treatment of such operators.
A very detailed description of normally hyperbolic operators on Lorentzian manifolds, including proofs of the above statements regarding the initial value problem and the existence of fundamental solutions, is given in \cite{baer_ginoux_pfaeffle}, also with an overview of quantization. A shorter introduction to the topic is for example treated in \cite{baer-ginoux_classical-and-quantum-fields}, also with a description of quantization.
%
%
%
%
%
%
%%%
%
%
%
%%
%%%%%%%%%
%%%DIFFERENTIAL FORMS
%%%%%%%%
%
%
%
\subsection{Differential forms}\label{sec:differential_forms}
%
%
Differential forms provide an elegant, coordinate independent description of calculus on smooth manifolds. In particular, they generalize the notion of line- and volume-integrals that are known from analysis. Differential forms play a remarkable role in physics, as one can argue that they indeed describe fundamental physical entities. As an example, instead of viewing a classical force as a vector, one can think of it, more closely related to experiments, as a differential one-form that assigns a scalar to a tangent vector of a curve. This scalar is the (infinitesimal) work associated with the force along the curve. Also, differential forms allow for an elegant geometric description of field theories, for example the Maxwell and Proca field theories that we encounter in this thesis. In Maxwell's classical theory of electromagnetism, instead of viewing the electric and magnetic field (which are conceptually just forces) as the fundamental physical entities, one introduces the \emph{vector potential}, a one-form, consisting of the scalar electric potential and the vector potential associated with the magnet field. Experiments like the Aharonov-Bohm experiment allow for an interpretation of the vector potential as the fundamental physical object, rather than the associated electromagnetic field. \\
Even more fundamentally, the two main theories of physics, General Relativity and the Standard Model of particle physics, are field theories. They are deeply connected to a geometric interpretation and can be elegantly described using differential forms. \par
%
%
Despite of all this, differential forms are usually not part of the standard curriculum of physicists. We shall therefore introduce the basic aspects and definitions regarding differential forms that are used in this thesis. For a more detailed introduction we refer to the literature: For example \cite[Chapter 2 and 4]{rudolph_schmidt} or \cite[Appendix B]{wald_GR} provide introductions to the topic.\par
%
%
In the following, let $\N$ denote a smooth $N$-dimensional manifold, assumed to be Hausdorff, connected, oriented and para-compact, with either Lorentzian or Riemannian metric $k$ and Levi-Civita connection $\nabla$. For a Lorentzian manifold we use the sign convention $(-,+,\dots,+)$ of the metric $k$. The number of negative eigenvalues of $k$ is denoted by $s$, so $s=0$ for a Riemannian manifold and, in our convention, $s=1$ for a Lorentzian manifold.
Later, we will specify to a four dimensional (globally hyperbolic) spacetime consisting of a four dimensional manifold $\M$ with Lorentzian metric $g$ and Cauchy surface $\Sigma$ with induced Riemannian metric $h$.
%
We define:
\begin{definition}[Differential form]
	Let $p\in \{0,1,\dots,N\}$. A \emph{differential form} $\omega$ of degree $p$, or $p$-form for short, on the manifold $\N$ is an anti-symmetric tensor field of rank $(0,p)$. That is, at every point $x \in \N$, $\omega_x$ is an anti-symmetric multi-linear map
	\begin{align}
	\omega_x : \underbrace{T_x \N \times T_x \N \times \cdots \times T_x \N}_{p\text{-times}} \to \IR \formspace.
	\end{align}
	We denote the vector space\footnote{Naturally, addition and scalar multiplication are defined point-wise.} of $p$-forms on $\N$ by $\gls{omegap}$, the space with compactly supported ones by \gls{omegapz}.
\end{definition}
As an example, a zero-form $f \in \Omega^0(\N)$ is just a $C^\infty$-function from $\N$ to $\IR$, hence we can identify $\Omega^0(\N) = C^\infty (\N, \IR)$. A one-form $A \in \Omega^1(\N)$ is nothing more than a co-vector field and in a physical context usually denoted in local coordinates by $A_\mu$. Note, that alternatively one can directly define a $p$-form as a smooth section of the $p$-th exterior product of the co-tangent bundle and hence identify $\Omega^p(\N) = \Gamma \big( \largewedge^k T^*\N\big)$. As mentioned in Section \ref{sec:spacetime_geometry}, we view the tangent bundle as a complex bundle. Therefore, the sections of that bundle will be complex valued functionals. In that fashion, we will usually view the spaces $\Omega^p(\N)$ as complex valued differential forms.\par
%
Next we define the basic operations, besides addition and scalar multiplication, that one can perform on differential forms.
%
\begin{definition}[Exterior product]
	Let $A \in \Omega^p(\N)$ be a $p$-form and  $B\in \Omega^q(\N)$ a $q$-form on $\N$. \\
	The \emph{exterior product} $\gls{wedge}:\Omega^p(\N) \times \Omega^q(\N) \to \Omega^{p+q} (\N)$ is defined by
	\begin{align}
	(A \wedge B)_{\mu_1\dots\mu_p \nu_1\dots\nu_q} = \frac{(p+q)!}{p!q!}\, A_{[\mu_1 \dots \mu_p} B_{\nu_1\dots\nu_q]} \formspace,
	\end{align}
	where the anti-symmetrization of a tensor $T$ is given through
	\begin{align}
	T_{[\mu_1\dots\mu_p]} = \frac{1}{p!} \sum\limits_{\sigma\in S_N }\textrm{sgn}(\sigma) T_{\sigma(\mu_1)\dots\sigma(\mu_p)} \formspace.
	\end{align}
\end{definition}
Here, $S_N$ denotes the symmetric group\footnote{Usually the symmetric group is defined as the set of permutations of $\{1,2,\dots,N\}$ but we chose the index to run over $\{0,1,\dots,N-1\}$, identifying the time component with zero rather then one.} of degree $N$, consisting of permutations of the set $\{0,1,\dots,N-1\}$.
With this notion of multiplication, point-wise addition and scalar multiplication, the space $\gls{omega} \coloneqq \bigoplus_{p = 0}^\infty \Omega^p(\N) = \bigoplus_{p = 0}^N \Omega^p(\N)$ becomes an algebra, usually called the Grassmann- or \emph{exterior algebra} of differential forms on $\N$. We have used that obviously $\Omega^k(\N) =0$ for $k >N$ due to the anti-symmetrization.\par
Furthermore, we find a notion of how to \emph{pullback} differential forms on manifolds to another manifold, for example the pullback of a differential form on the spacetime $\M$ to differential forms on its Cauchy surface $\Sigma$. Given a $C^\infty$-map $\psi: \widetilde{\N} \to \N$, where $\N, \widetilde{\N}$ are manifolds, we can naturally define the pullback of a function $f \in \Omega^0(\N)$ to a function $(\psi^* f) \in \Omega^0(\widetilde{\N})$ by composing $f$ with $\psi$:
\begin{align}
\psi^* f \coloneqq f \comp \psi \formspace.
\end{align}
\newpage
With the pullback of functions defined, we can define how to \emph{push forward}, or carry along, vector fields on $\widetilde{\N}$ to vector fields on $\N$: Let $f\in \Omega^0(\N)$ and $\tilde{v} \in \Gamma(T\widetilde{\N})$ and $\tilde{x} \in \widetilde{\N}$. Then
\begin{align}
(\psi_* \tilde{v})_{\psi(\tilde{x})} (f) \coloneqq \tilde{v}_{\tilde{x}}(\psi^* f)
\end{align}
defines the vector field $(\psi_* v) \in \Gamma(T\N)$. With these basic operations at hand, we can generalize to define the pullback of differential forms:
\begin{definition}[Pullback]\label{def:pullback}
	Let $\N, \widetilde{\N}$ be manifolds of dimension $N,\widetilde{N}$ respectively, and let $\psi: \widetilde{\N} \to \N$ be a smooth map. Then, $\psi$ defines an algebra homomorphism $\psi^* : \Omega(\N) \to  \Omega(\widetilde{\N})$,
	called the \emph{pullback} of differential forms. For $\omega \in \Omega^p(\N)$, $\tilde{x} \in \widetilde{\N}$ and $\tilde{v}_i \in T_x \widetilde{\N}$, $i=1,2,\dots,p$, it is defined by
	\begin{align}
	\left( \psi^* \omega \right)_{\tilde{x}}  (\tilde{v}_1,\tilde{v}_2,\dots,\tilde{v}_p) \coloneqq \omega_{\psi(\tilde{x})} (\psi_* \tilde{v}_1, \dots , \psi_* \tilde{v}_p) \formspace.
	\end{align}
\end{definition}
%
%
%
%
On the exterior algebra we find a duality, provided by the Hodge operator:
\begin{definition}[Hodge dual]
	The hodge star operator $\gls{hodge}: \Omega^p(\N) \to \Omega^{N-p}(\N)$ is defined through
	\begin{align}
	B \wedge *A = \frac{1}{p!} B^{\mu_1\dots\mu_p}A_{\mu_1\dots\mu_p} \dvolk \formspace,
	\end{align}
	which yields the coordinate representation
	\begin{align}
	(*A)_{\mu_{p+1}\dots\mu_N} = \frac{\detk}{p!} \, \epsilon_{\mu_1\dots\mu_N} A^{\mu_1\dots\mu_p} \formspace.
	\end{align}
\end{definition}
Here, \gls{levicivita} denotes the fully antisymmetric tensor of rank $N$ (Levi-Civita symbol) satisfying $\epsilon_{12,\dots,N} =1$ and the \emph{volume element} \gls{dvolk} is defined by
\begin{align}
\left( \gls{dvolk} \right)_{\alpha_1\dots\alpha_N} = \detk \, \epsilon_{\alpha_1\dots\alpha_N} \formspace.
\end{align}
In a sense, the volume element describes how the curvature of the manifold deforms a unit volume.
The duality follows from the important property of the Hodge operator as stated in the following lemma:
\begin{lemma}
	Let $*$ denote the Hodge star operator on the exterior algebra $\Omega(\N) $. It holds that
	\begin{align}
	** = (-1)^{s+p(N-p)} \, \mathbbm{1} \formspace,
	\end{align}
	which is trivially equivalent to $*^{-1} = (-1)^{s+p(N-p)} \, *$.
\end{lemma}
\begin{proof}
	Let $A \in \Omega^p(\N)$ be a $p$-form on $\N$. Then:
	\begin{align}
	(*{*A})_{\mu_1 \dots \mu_p}
	&= \frac{\detk \, \detk}{p! \, (N-p)!} \; \epsilon_{\alpha_{p+1}\dots\alpha_N \mu_1 \dots \mu_p}\;\epsilon^{\alpha_{1}\dots\alpha_N}\;A_{\alpha_1\dots\alpha_p} \notag\\
	&= (-1)^{p(N-p)} \frac{\detk \, \detk}{p! \, (N-p)!} \; \epsilon_{\alpha_{p+1}\dots\alpha_N \mu_1 \dots \mu_p}\;\epsilon^{\alpha_{p+1}\dots\alpha_{N}\alpha_1\dots\alpha_p}\;A_{\alpha_1\dots\alpha_p}  \notag\\
	&= (-1)^{s+p(N-p)} \delta\indices{^{[\alpha_{1}}_{\mu_{1}}}\, \dots \, \delta\indices{^{\alpha_p ] }_{\mu_p}} \;A_{\alpha_1\dots\alpha_p} \notag\\
	&=  (-1)^{s+p(N-p)}\;A_{\mu_1\dots\mu_p} \formspace
	\end{align}
	We have used Lemma \ref{lem:epsilon_contraction} and, in the last step, that the anti-symmetrization is absorbed by contraction because $A$ is antisymmetric.
\end{proof}
%
%
%
%
%
Furthermore, we can equip the exterior algebra with a differentiable structure, introducing the notion of the exterior derivative.
\begin{definition}[Exterior derivative]
	The \emph{exterior derivative} $\gls{d}:\Omega^p(\N) \to \Omega^{p+1} (\N)$ is defined by the following properties:
	\begin{enumerate}
		\item $d$ is linear
		\item $d$ obeys a graded Leibniz rule: Let $A \in \Omega^p(\N)$ and  $B\in \Omega^q(\N)$, then
		\begin{align}
		d(A \wedge B) = dA \wedge B + (-1)^p \, A \wedge dB
		\end{align}
		\item $d$ is nilpotent, that is,  $d^2 = 0$.
	\end{enumerate}
	In local coordinates, this is equivalent to the representation
	\begin{align}
	(dA)_{\mu \alpha_1\dots\alpha_p} = (p+1)\, \nabla_{[\mu}A_{\alpha_1\dots\alpha_p]} \formspace.
	\end{align}
\end{definition}
An important property of the exterior derivative is that it commutes (or rather intertwines its action) with pullbacks (see \cite[Proposition 4.1.7]{rudolph_schmidt}).
A $p$-form $\omega \in \Omega^p(\N)$ is called \emph{exact} if there is a $(p-1)$-form $\alpha \in \Omega^{p-1}(\N)$ such that $\omega = d\alpha$. We call $\omega$ \emph{closed} if $d \omega =0$. Accordingly, the space of closed $p$-forms is denoted by \gls{omegapd}, the space of exact ones by \gls{domegap}. As usual, the ones with compact support are denoted by a subscript zero. Note, that every exact form is closed, using that $d$ is by definition nilpotent, but the reverse is in general not true. It does hold, however, on certain manifolds with trivial topology, such as Minkowski spacetime. This is expressed in the so called Poincar\'e-Lemma (see for example \cite[Chapter 4]{bott_tu}) based on the study of de Rham cohomology.\par
%
Moreover, $N$-forms can naturally be integrated. Using local coordinates and a partition of unity, we define the integral of $N$-forms via the well known integration on $\IR^N$:
\begin{definition}[Integration on manifolds]
	Let $\left\{U_\alpha, \psi_\alpha\right\}_\alpha$ be an atlas of the manifold $\N$ and $\left\{\chi_\alpha\right\}_\alpha$ a partition of unity subordinate to the locally finite open cover $\left\{U_\alpha\right\}_\alpha$. Let $x^\mu_{(\alpha)}$ be a coordinate basis of $\psi$ on $U_\alpha$. For any $N$-form $\omega \in \Omega^N_0(\M)$ we define the integral
	\begin{align}
	\int\limits_{\N} \omega &\coloneqq \sum_{\alpha} \int\limits_{\psi_\alpha (U_\alpha)} w(x_{(\alpha)}^0,\dots,x_{(\alpha)}^1)\; dx_{(\alpha)}^0 \cdots dx_{(\alpha)}^{N-1} \formspace,
	\end{align}
	where $w$ are the components of $\omega$ in the coordinates $x_{(\alpha)}^\mu$, that is $\omega = w dx_{(\alpha)}^0 \wedge \cdots \wedge dx_{(\alpha)}^{N-1}$.
	This definition is independent of the choice of the atlas and the partition of unity (see \cite[Proposition 3.3]{bott_tu}).
\end{definition}
With integration at our disposal, we present an important theorem regarding the integration of exact differential forms:
\begin{theorem}[Stoke's Theorem]\label{thm:stokes}
	Let $\N$ be an oriented manifold of dimension $N$ and let its boundary $\partial \N$ be endowed with the induced orientation. Let $\gls{inclusionmap} : \partial \N \hookrightarrow \N$ be the inclusion operator.
	Let $\omega \in \Omega^{N-1}_0(\N)$ be a compactly supported $(N-1)$-form on $\N$. Then it holds
	\begin{align}
	\int\limits_\N d\omega = \int\limits_{\partial \N} i^*\omega \formspace.
	\end{align}
\end{theorem}
\begin{proof}
	A proof is given in most of the introductory literature on differential geometry (see for example \cite[Chapter 17, Theorem 2.1]{lang}).
	Note that one can equivalently formulate Stoke's theorem on a \emph{compact} manifold but for {arbitrary} (that is, in general not compactly supported) $(N-1)$-forms on the manifold (see for example \cite[Theorem 4.2.14]{rudolph_schmidt}). This will be of importance in later calculations.
\end{proof}
%
Furthermore, we can define a bilinear map on $\Omega^p(\N)$ using the integration of $N$-forms:
\begin{definition}
	Let $A,B \in \Omega^p(\N)$ such that their supports have a compact intersection. Define the bilinear map $\gls{innerprod} : \Omega^p(\N) \times \Omega^p(\N) \to \IC$ by
	\begin{align}
	\langle A, B \rangle_\N \coloneqq  \int_{\N } A \wedge * B = \int_{\N } A_{\mu_1 \dots \mu_p}B^{\mu_1 \dots \mu_p}\,\dvolk \formspace.
	\end{align}
\end{definition}
Since by definition $A \wedge * B$ is a compactly supported $N$-form, this is well defined. We may sometimes refer to $\langle \cdot , \cdot \rangle_\N$ as an inner product for simplicity, even though it is not positive definite.
%
%
%
%
%
Using the exterior derivative, we define the interior or co-derivative:
\begin{definition}[Interior derivative]
	The \emph{interior derivative} $\gls{delta} : \Omega^p(\N) \to \Omega^{p-1}(\N)$ is defined by
	\begin{align}
	\delta \coloneqq (-1)^{s+1+N(p-1)}\, {*{d*}} \formspace.
	\end{align}
	From the defining properties of $d$ and $*$ it follows $\delta^2 =0$.
\end{definition}
Here, $s$ again denotes the number of negative eigenvalues of the metric $k$ of $\N$. In accordance with our nomenclature, we call a $p$-form $\omega$ co-exact if there exists a $\alpha \in \Omega^{p+1}(\N)$ such that $\omega = \delta \alpha$ and co-closed if $\delta \omega = 0$. Accordingly, the spaces of co-closed and co-exact $p$-forms are denoted by \gls{omegapdelta} and \gls{deltaomegap} respectively.\par
Using the exterior and interior derivative we define the partial differential operator:
\begin{definition}[D'Alembert Operator]
	The d'Alembert (or Laplace - de Rham) operator $\gls{dalembert}: \Omega^p(\N) \to \Omega^{p}(\N)$ is defined by
	\begin{align}
	\square \coloneqq \delta d +d \delta \formspace.
	\end{align}
\end{definition}
By definition of the exterior and interior derivative, it is easy to show that $\square$ commutes with both $d$ and $\delta$:
\begin{align}
\square d &= (\delta d + d \delta )d \notag \\
&= d \delta d \notag \\
&= d (\delta d + d \delta) \formspace,
\end{align}
and analogously for $\delta$.
The d'Alembert operator, and its generalization to $(\square + m^2)$ for some constant $m > 0$, are important examples for a normally hyperbolic differential operators (see Section \ref{sec:global_hyperbolicity}) and we may therefore sometimes just refer to them as \emph{wave operators}.\par
The sign convention in the definition of the exterior derivative is chosen such that on any Lorentzian or Riemannian manifold the interior derivative is formally adjoint to the exterior derivative, that is,  for $A \in \Omega^{p}(\N)$ and $B \in \Omega^{p+1}(\N)$ it holds that
\begin{align}
\langle dA , B \rangle_{\N} = \langle A , \delta B \rangle_\N \formspace,
\end{align}
which leads to a representation in local coordinates of the Manifold given by:
\begin{align}
(\delta A)_{\mu_2\dots\mu_p} = - \nabla^{\mu_1}A_{\mu_1\dots\mu_p} \formspace.
\end{align}
To see that this is consistent, let $A \in \Omega^{p-1}(\N)$ and $B \in \Omega^{p}(\N)$ such that their supports have compact intersection.
We obtain, using Stoke's Theorem \ref{thm:stokes}:
\begin{align}
0 &= \int \limits_{\partial \N} i^* (A \wedge *B) \notag\\
&= \int \limits_{\N} d(A \wedge *B)  \notag\\
&= \int \limits_{\N} dA \wedge *B + (-1)^{p-1} A \wedge d{*B} \notag\\
&= \int \limits_{\N} dA \wedge *B + (-1)^{p-1} A \wedge *{*^{-1}}\underbrace{d{*B}}_{\textrm{is a } (N-p+1) \textrm{ form.}} \notag\\
&= \int \limits_{\N} dA \wedge *B + (-1)^{p-1}(-1)^{s+(N-p+1)(N-N+p-1)} A \wedge *{*d{*B}} \notag\\
&= \int \limits_{\N} dA \wedge *B + (-1)^{p+(1-p)(p-1)} A \wedge *\delta B \formspace.
\end{align}
It can easily be proven by induction that $\big(p+(1-p)(p-1)\big)$ is odd for any $p \in \IN$, which yields the result
\begin{align}
\langle dA , B \rangle_{\N} = \langle A , \delta B \rangle_\N \formspace.
\end{align}
The definitions stated above thus fulfill the requirement of formal adjointness of the exterior and interior derivate on an arbitrary Lorentzian or Riemannian manifold $\N$.
In local coordinates we use a partial integration to obtain
\begin{align}
\langle dA , B \rangle_\N &= \int \limits_{\N} dA \wedge * B \notag\\
%&= \int \limits_{\N} \frac{1}{p!} (dA)^{\alpha_1\dots\alpha_p}\,B_{\alpha_1 \dots \alpha_p} \, \dvolk \notag\\
&= \int \limits_{\N}  \frac{p}{p!} \nabla^{[\alpha_1}A^{\alpha_2\dots\alpha_p]}\,B_{\alpha_1 \dots \alpha_p} \, \dvolk \notag\\
&= \int \limits_{\N}  \frac{1}{(p-1)!} \nabla^{\alpha_1}A^{\alpha_2\dots\alpha_p}\,B_{\alpha_1 \dots \alpha_p} \, \dvolk \notag\\
&= - \int \limits_{\N}  \frac{1}{(p-1)!} A^{\alpha_2\dots\alpha_p}\, \nabla^{\alpha_1}B_{\alpha_1 \dots \alpha_p} \, \dvolk \notag\\
&= \langle A, \delta B \rangle_\N \formspace,
\end{align}
which yields
\begin{align}
-\nabla^{\alpha_1}B_{\alpha_1 \dots \alpha p} = (\delta B)_{\alpha_2 \dots \alpha_p}\formspace.
\end{align}
On the four dimensional spacetime $(\M,g)$ the definitions of the Hodge star operator and the interior derivative simplify, such that
\begin{align}
*_{(\M)}*_{(\M)} &= (-1)^{p+1} \mathbbm{1} \\
\delta_{(\M)} &= *_{(\M)}{d_{(\M)}*_{(\M)}} \formspace ,
\end{align}
holds on the spacetime $(\M,g)$ and
\begin{align}
*_{(\Sigma)}*_{(\Sigma)} &= \mathbbm{1} \\
\delta_{(\Sigma)} &= (-1)^p *_{(\Sigma)}{d_{(\Sigma)}*_{(\Sigma)}}
\end{align}
holds on  $(\Sigma,h)$. In the following we will drop the subscript ${(\M)}$, since we will perform all the calculations on a four dimensional spacetime, except when explicitly noted (for example with a subscript $(\Sigma)$).
%
%
%
%
%
%
%
%
%%%%%%
%%CATEGORY THEORY
%%%%%%
\subsection{Category theory}\label{sec:cat-theory}
The description of Quantum Field Theory on Curved Spacetimes (QFTCS) in the framework of \name{Brunetti}, \name{Fredenhagen} and \name{Verch} \cite{Brunetti_Fredenhagen_Verch} is based on category theory. In this thesis, we will not go into detail on those categorical aspects, however we will need some basic definitions to formulate the theory rigorously, that is namely the notion of a category and that of covariant functors, since, in the used framework, the generally covariant QFTCS is a functor.\par
Here, we present definitions given in \cite[Appendix A.1]{baer_ginoux_pfaeffle} and refer to the appropriate literature for details. We define:
\begin{definition}[Category]
	A \emph{category} $\mathsf{Cat}$ consists of the following:
	\begin{enumerate}
		\item a class $\mathsf{Obj}_\mathsf{Cat}$ whose members are called \emph{objects},
		\item a set $\mathsf{Mor}_\mathsf{Cat}(A,B)$, for any two objects $A,B \in \mathsf{Obj}_\mathsf{Cat}$, whose elements are called \emph{morphisms},
		\item for any three objects $A,B,C \in \mathsf{Obj}_\mathsf{Cat}$ there is a map
		\begin{align}
\mathsf{Mor}_\mathsf{Cat}(B,C) \times \mathsf{Mor}_\mathsf{Cat}(A,B) &\to \mathsf{Mor}_\mathsf{Cat}(A,C) \notag\\
(\psi,\phi) &\mapsto \psi \comp \phi
		\end{align}
		called the composition of morphisms subject to the relations:\vspace{4mm}
		\begin{enumerate}[label=(\arabic*)]
			\item for non equal pairs $(A,B)$, $(A',B')$ of objects, the sets $\mathsf{Mor}_\mathsf{Cat}(A,B)$ and $\mathsf{Mor}_\mathsf{Cat}(A',B')$ are disjoint,
			\item for every object $A$ there exists a morphism $\text{id}_A \in \mathsf{Mor}_\mathsf{Cat}(A,A)$ such that it holds for all objects $B$, morphisms $\psi \in \mathsf{Mor}_\mathsf{Cat}(B,A)$ and $\phi \in \mathsf{Mor}_\mathsf{Cat}(A,B)$
			\begin{align}
				\text{id}_A \comp \psi &= \psi \quad \text{and}\\
				\phi \comp \text{id}_A &= \phi \quad,
			\end{align}
			\item the composition law is associative, that is for an objects $A,B,C,D$ and any morphisms $\psi \in \mathsf{Mor}_\mathsf{Cat}(A,B)$, $\phi \in \mathsf{Mor}_\mathsf{Cat}(B,C)$ and $\chi \in \mathsf{Mor}_\mathsf{Cat}(C,D)$ it holds
			\begin{align}
				(\chi \comp \phi) \comp \psi = \chi \comp (\phi \comp \psi) \formspace.
			\end{align}
		\end{enumerate}
	\end{enumerate}
\end{definition}
%
%
%
\begin{definition}[Functor]
	Let $\mathsf{Cat1}$ and $\mathsf{Cat2}$ be categories. A \emph{covariant functor} $\mathscr{A}: \mathsf{Cat1} \to \mathsf{Cat2}$ consists of the map $\mathscr{A} : \mathsf{Obj}_\mathsf{Cat1} \to \mathsf{Obj}_\mathsf{Cat2}$ and maps $\mathscr{A}: \mathsf{Mor}_\mathsf{Cat1}(A,B) \to \mathsf{Mor}_\mathsf{Cat2}\big(\mathscr{A}(A),\mathscr{A}(B)\big)$ for any two objects $A,B \in \mathsf{Obj}_\mathsf{Cat1}$ such that
	\begin{enumerate}
		\item {the composition is preserved, that is for all objects $A,B,C \in \mathsf{Obj}_\mathsf{Cat1}$ and for any morphisms $\psi \in \mathsf{Mor}_\mathsf{Cat1}(A,B)$ and $\phi \in \mathsf{Mor}_\mathsf{Cat1}(B,C)$ it holds
		\begin{align}
			\mathscr{A}(\phi \comp \psi) = \mathscr{A}(\phi) \comp \mathscr{A}(\psi) \formspace,
		\end{align}}
		\item{
			$\mathscr{A}$ maps identities to identities, that is for any object $A \in \mathsf{Obj}_\mathsf{Cat1}$ it holds
			\begin{align}
				\mathscr{A}(\text{id}_\mathsf{A}) = \text{id}_{\mathscr{A}(A)} \formspace.
			\end{align}
			}
	\end{enumerate}
\end{definition}
%
%
%
%
%
%
%
%
%
%
%
%
%%%%%%
%%SIGN CONVENTIONS
%%%%%%
%
%
\subsection{Sign conventions}\label{sec:sign_conventions}
At certain points throughout this chapter we have had a freedom of choice regarding the signs of some entities, in particular the sign of the signature of the Lorentzian metric $g$ and that of the interior derivative $\delta$. Though at this stage the choice can be made arbitrarily, we want to make it in a way that in the end allows us to make certain physical interpretations on some parameters. More precisely, we want to interpret the parameter $m$ of the Klein-Gordon equation\footnote{or its generalization on $p$-forms} $(\square + m^2) f = 0$ for a zero-form $f \in \Omega^0(\M)$ as a mass in the physical sense. With the chosen sign convention for $\delta$ we find, using ${\delta}f = 0$:
\begin{align}
	\square f
	&= (\delta d + d \delta) f \notag\\
	&= \delta d f \notag\\
	&= - \nabla^\mu \nabla_\mu f \formspace.
\end{align}
In the following heuristic (local) argument we see
\begin{align}
	\square + m^2
	&= -\nabla^\mu \nabla_\mu + m^2 \notag\\
	&\sim \partial_t^2 + \sum_i \partial_i^2 + m^2\notag\\
	&\sim -E^2 + \abs{\vector{p}}^2 + m^2
\end{align}
which yields the correct relativistic relation of energy, momentum and mass according to $E^2 = \abs{\vector{p}}^2 + m^2$.
A similar calculation holds for the Klein-Gordon operator generalized to act on one-forms. If we had found a ``wrong'' relation between energy, momentum and mass, we would have had to adapt the chosen signs. Usually one chooses the sign of the metric and the interior derivative such that they are in some sense mathematically convenient (although one might disagree with another one's choice). We have made the choice of the metric, such that the Cauchy surfaces become Riemannian rather that ``anti-Riemannian'' (with an all minus signature), which seems more natural to some. Also, a lot of the used references on spacetime geometry (in particular the book by \name{Wald} \cite{wald_GR}) use this sign convention, which makes the application of certain formulas easier. As mentioned, the sign of the interior derivative was chosen such that it is formally adjoint to the exterior derivative (with respect the specified inner product) on all Lorentzian and Riemannian manifolds. It seemed convenient for the actual calculations to fix the sign regardless of the signature of the metric of the underlying manifold. One could equivalently have fixed the opposite sign, yielding the two derivatives to be skew-adjoint, which is also done in the literature. However, in the end, one has one freedom left to make the energy-momentum-mass relation work: that is the sign in front of the mass in the Klein-Gordon equation and all other wave equations accordingly. Hence, one regularly also finds the Klein-Gordon equation to be defined with a flipped sign of the mass term. But for our case, we want the mass $m$ in any wave equation to appear with a positive sign.
%
%



\section{Sparse and Bounded Regular Languages}
\label{sec:sparse}

Here, in Theorem~\ref{thm:sparse_in_NP}, we establish that for constraint languages from the class of sparse regular languages, which equals the class of the bounded regular languages~\cite{DBLP:journals/eik/LatteuxT84}, the constrained problem
is always in \NP.


A language $L \subseteq \Sigma^*$ is \emph{sparse},
%\cite{DBLP:journals/siamcomp/BermanH77,DBLP:journals/ijfcs/GawrychowskiKRS10,DBLP:journals/ipl/Hartmanis83,DBLP:series/wsscs/Hartmanis93b,DBLP:journals/iandc/HartmanisIS85,DBLP:conf/mfcs/HartmanisM80,DBLP:conf/focs/Mahaney80,DBLP:journals/jcss/Mahaney82,Pin2020,DBLP:conf/mfcs/SzilardYZS92,DBLP:reference/hfl/Yu97}
if there exists $c \ge 0$
such that, for every $n \ge 0$, we have
$L \cap \Sigma^n \in O(n^c)$.
Sparse languages were introduced into computational complexity
theory by Berman \& Hartmanis~\cite{DBLP:journals/siamcomp/BermanH77}.
Later, it was established by Mahaney that if there exists
a sparse \NP-complete set (under polynomial-time many-one reductions),
then $\PTIME = \NP$~\cite{DBLP:journals/jcss/Mahaney82}.
For a survey on the relevance of sparse sets in computational complexity theory, see~\cite{DBLP:conf/mfcs/HartmanisM80}.


A language $L \subseteq \Sigma^*$ is called \emph{bounded},
if there exist $w_1, \ldots, w_k \in \Sigma^*$
such that $L \subseteq w_1^* \ldots w_k^*$. %~\cite{GinsburgSpanier66,DBLP:journals/eik/LatteuxT84}.
Bounded languages were introduced by Ginsburg \& Spanier~\cite{GinsburgSpanier64}.

We will need the following representation of the bounded regular languages.

% noch begründen, warum bounded überhaupt erwähnt wird, und nicht direkt 
% nur mit dem begriff sparse gearbeitet wird todo -> wegen klasse strictly bounded

\begin{theorem}[\cite{GinsburgSpanier66}]
\label{thm:bounded_regular_form}
 A language $L \subseteq w_1^* \cdots w_k^*$ is regular if and only if
 it is a finite union of languages of the form $L_1 \cdots L_k$, where each $L_i \subseteq w_i^*$ is regular.
\end{theorem}

It is known that the class of sparse regular languages equals
the class of bounded regular languages~\cite{DBLP:journals/eik/LatteuxT84},
or see~\cite{Pin2020,DBLP:reference/hfl/Yu97}, where the bounded languages are not mentioned
but the equivalence is implied by their results and Theorem~\ref{thm:bounded_regular_form}.
The next results links this class to the polycylic PDFAs.

\begin{propositionrep}
\label{thm:bounded_characterization}
 %For regular $L \subseteq \Sigma^*$, the following are equivalent:
 %(1) $L$ is sparse, (2) $L$ is bounded, (3) $L$ is recognizable by a polycyclic PDFA.
 Let $L \subseteq \Sigma^*$ be regular. Then, $L$ is sparse
 if and only if it is recognizable by a polycyclic PDFA.
\end{propositionrep}
\begin{proof}
 In~\cite{DBLP:journals/eik/LatteuxT84} is was shown that the context-free sparse languages are precisely the context-free bounded languages, which
 gives our first two equivalences.
 A result from~\cite[Lemma 2]{DBLP:journals/ijfcs/GawrychowskiKRS10}
 readily implies that if a language is recognized by a polycyclic PDFA, then
 it must be sparse.
 Lastly, we show that every bounded regular language
 is recognizable by a polycyclic automaton, which finishes the proof.
 
 \medskip 
 
 \noindent\underline{Claim:} For $w \in \Sigma^*$.
  Then, any regular $L \subseteq w^*$
  is recognizable by a polycyclic PDFA.
 \begin{quote}
     \emph{Proof of the Claim.}
      Let $w \in \Sigma^*$ and $L \subseteq w^*$ be a regular language.
 If $w = \varepsilon$, then $L = \{\varepsilon\}$, which is obviously recognizable
 by a polycyclic automaton. So, suppose $|w| > 0$.
%  Suppose $\mathcal A = (\Sigma, Q, \delta, q_0, F)$
%  is an accepting PDFA for $L$ such that every state is accessible
%  and coacessible, which we can assume by Lemma~\ref{lem:accessible_coaccessible}.
%  Choose any $q \in Q$. Then, there exists $u_1, u_2 \in \Sigma^*$
%  such that $\delta(q_0, u_1) = q$
%  and $\delta(q, u_2) \in F$.
%  Hence, if $\delta(q, v) = q$,
%  then $u_1 v u_2 \in L$, so that $u_1 v u_2 \subseteq w^*$.
%  If $v = \varepsilon$, then $v\subseteq w^*$.
%  So, suppose $|v| > 0$.
%  Then $u_1 v u_2 = w^n$ for some $n > 0$.
%  But then, $v = v_1 w^k v_2$
%  for some maximal $k \ge 0$.
%
%  Every accepting automaton for a language $L \subseteq w^*$, for some $w \in \Sigma^*$, is obviously polycyclic.
%
% Todo, inverse hom, single final acceptable?
% Applying Theorem~\ref{thm:bounded_regular_form},
% we can write $L = L_1 \cup \ldots \cup L_n$
% such that, for any $i \in \{1,\ldots,n\}$, $L_i \subseteq w^*$ is regular.
 Let $a$ be an arbitrary symbol
 and define a homomorphism $\varphi : \{a\}^* \to \Sigma^*$
 by $\varphi(a^i) = w^i$, which is injective as $|w| > 0$ by assumption.
 Then, the unary language $\varphi^{-1}(L) = \{ a^i \mid w^i \in L\}$ is regular, as inverse
 homomorphisms preserve regularity.
 Hence, we can write it as a union of languages recognizable by automata
 with a single final state, which, by Lemma~\ref{lem::unary_single_final},
 have the form $\{ a^i \}$ for some $i \ge 0$
 or $\{ a^{i + jp} \mid j \ge 0 \}$ for some $i \ge 0, p > 0$.
 As the application of functions preserves union, and $L = \varphi(\varphi^{-1}(L))$ here,
 the language 
 $L$ is the union of the images of these languages.
 We have $\varphi(\{a^i \}) = \{ w^i \}$, and this singleton language
 is obviously recognizable by a polycyclic automaton,
 and we have $\varphi(\{ a^{i + jp} \mid j \ge 0 \}) = \{ w^{i+pj} \mid j \ge 0 \}$,
 and this language is also recognizable by an automaton that
 has an initial tail labelled by $w^i$ and a cycle labelled by $w^p$.
 So, as the polycyclic languages are closed under union~\cite[Proposition 6]{DBLP:conf/ictcs/Hoffmann20},
 we have shown that the language $L$ %languages $L_i, i \in \{1,\ldots,n\}$,
 is recognizable by some polycyclic automaton.     \emph{[End, Proof of the Claim]}
 \end{quote}
 Finally, as the languages recognizable by polycyclic automata
 are closed under concatenation and union~\cite[Proposition 5 and Proposition 6]{DBLP:conf/ictcs/Hoffmann20},
 by Theorem~\ref{thm:bounded_regular_form} every bounded regular language is recognizable by a polycyclic automaton.
\end{proof}

In~\cite[Theorem 2]{DBLP:conf/ictcs/Hoffmann20} it was shown that for polycyclic
constraint languages, the constrained problem is always in $\NP$.
So, we can deduce the next result.

\begin{theorem}
\label{thm:sparse_in_NP}
 If $L \subseteq \Sigma^*$ is sparse and regular, then $L\textsc{-Constr-Sync} \in \NP$.
\end{theorem}

We will need the following closure property stated in~\cite[Theorem 3.8]{DBLP:reference/hfl/Yu97}
of the sparse regular languages.

\begin{proposition}
 %The sparse regular languages are closed under morphisms.%homomorphic mappings.
 The class of sparse regular languages is closed under homomorphisms.
\end{proposition}

\begin{toappendix}
Note that sparse languages in general are not closed
under homomorphic mappings~\cite{Pin2020}.
As it is easy to see that the bounded languages are closed
under homomorphic mappings, this also implies that, in general,
the bounded languages do not equal the sparse languages.
\end{toappendix}


Note that the connection of the polycyclic languages to the sparse or bounded languages
was not noted in~\cite{DBLP:conf/ictcs/Hoffmann20}. However, a condition
characterizing the sparse regular languages
in terms of forbidden patterns was given in~\cite{Pin2020}, and
it was remarked that ``a minimal deterministic automaton recognises a sparse language if and only if it
does not contain two cycles reachable from one another''.
This is quite close to our characterization.
%and, probably, the author
%has had a similar intuition as spelled out with our explicit definition
%of a polycyclic automaton.




\section{Letter-Bounded Constraint Languages}
\label{sec:strictly_bounded_case}



% % begriff einführen, das man annehemn kann "nachbarn" verschieden
% % dann hom

% % L' = \{ b_1^n ... : a^n b^{n_\} ist das eindeutig?

% \begin{proposition}
%  hom verschieden, eine richtung klar
% \end{proposition}
% \begin{proof} % Gamma = {b1,..,bk}
% Let $\mathcal A = (\Gamma, Q, \delta)$ be an input automaton
% for which we want to know if it has a synchronizing word in $U$. %L' über Gamma
% Set $Q' = Q \times \{ 1, \ldots, k \}$
% and $\delta' : Q' \times \Sigma \to Q'$
% with 
% \[
%  \delta'((q, i), x) = \left\{
%  \begin{array}{ll}
%   (\delta(q, x), j)     & \mbox{if } x = \varphi(b_j); \\
%  % (\delta(q, x), i + 1) & \mbox{if } i < k \mbox{ and } x = \varphi(b_{i+1}); \\
%   (q, i)                & \mbox{otherwise.}
%  \end{array}\right.
% \]
% Then, $\mathcal A' = (\Sigma, Q', \delta')$
% has a synchronizing word in $L$
% if and only if $\mathcal A$ has a synchronizing word in $U$.
% % aber man muss jeweils mind einmal "rübergehen"




% \end{proof}



%  Let $L \subseteq w_1^* \cdots w_n^*$.
%  As $w^* w^* = w^*$, we can suppose that $w_i \ne w_{i+1}$
%  for $i \in \{1,\ldots,n-1\}$, which we will assume for the rest of this section.
%  In particular in the strictly bounded case we assume
%  that consecutive letter are distinct.\todo{wo brauche ich diese annahmen genau?}
 
 
 %In this section, we 
 Fix a constraint automaton $\mathcal B = (\Sigma, P, \mu, p_0, F)$.
 Let $a_1, \ldots, a_k \in \Sigma$ be a sequence of (not necessarily distinct)
 letters.
 In this section, we assume $L(\mathcal B) \subseteq a_1^* \cdots a_k^*$.
 A language which fulfills the above condition
 is called \emph{letter-bounded}.
 Note that the language $ab^*a$ given in the introduction as an example
 %, and in~\cite{DBLP:conf/mfcs/FernauGHHVW19} as the smallest, in terms of recognizing automata,
 %constraint language giving an \NP-complete problem,
 is letter-bounded. In fact, it is the language with the smallest
 recognizing automaton yielding an \NP-complete constrained problem~\cite{DBLP:conf/mfcs/FernauGHHVW19}.
 
 A language such that the $a_i$ are pairwise distinct, i.e., $a_i \ne a_j$
 for $i \ne j$, is called \emph{strictly bounded}.
 The class of strictly bounded languages has been extensively studied~\cite{DBLP:journals/mst/BlattnerC77,DBLP:journals/dam/DassowP99,Ginsburg66,GinsburgSpanier64,GinsburgSpanier66,HerrmannKMW17},
 where in~\cite{Ginsburg66,GinsburgSpanier64,GinsburgSpanier66} no name was introduced for them
 and in~\cite{HerrmannKMW17} they were called strongly bounded.
 %\footnote{The work~\cite{HerrmannKMW17}
 %seems to deviate from the standard terminology in other ways too by calling bounded languages
 %as introduced here word-bounded and refers to letter-bounded simply as bounded languages.}.
 The class of letter-bounded languages properly contains the strictly bounded languages.
 
 \begin{toappendix} 
 Note that the work~\cite{HerrmannKMW17}
 seems to deviate from the standard terminology, for example by calling bounded languages
 as introduced here word-bounded and refers to letter-bounded simply as bounded languages.
 \end{toappendix}
 

 
 \begin{remark}%[Motivation]%[A Motivation for Strictly Bounded Constraint Languages]
 \label{rem:motivation_strictly_bounded}
Let $\Sigma = \{b_1, \ldots, b_r\}$
be an alphabet of size $r$.
%and $\psi : \Sigma^* \to \mathbb N_0^m$
%be the \emph{Parikh morphism}
%given by $\psi(w) = (|w|_{b_1}, \ldots, |w|_{b_k})$
%for $w \in \Sigma^*$.
%Then, for $L \subseteq \Sigma$,
%set $\perm(L) = \{ w \in \Sigma^* \mid \exists u \in L \forall a \in \Sigma : |u|_a = |w|_a \}$,
%the \emph{commutative closure} of $L$.
Then, 
%between the commutative languages over $\Sigma$
%and the strictly bounded languages in $b_1^* \cdots b_k^*$, 
the mappings
\[
\Phi(L) = L \cap b_1^* \cdots b_r^* \mbox{ and }
\perm(L) = \{ w \in \Sigma^* \mid \exists u \in L\  \forall a \in \Sigma : |u|_a = |w|_a \}
\]
for $L \subseteq \Sigma^*$ are mutually inverse and inclusion preserving
between the languages in $b_1^* \cdots b_r^*$ and the commutative languages
in $\Sigma^*$, where a language $L \subseteq \Sigma^*$ is commutative 
if $\perm(L) = L$.
Furthermore, for strictly bounded languages of the form $B_1 \cdots B_r \subseteq b_1^* \cdots b_r^*$
with $B_j \subseteq \{b_j\}^*$, $j \in \{1,\ldots, r\}$, we have
$
 \perm(B_1 \cdots B_r) = B_1 \shuffle \cdots \shuffle B_r,
$
where
$U \shuffle V = \{ u_1 v_1 \cdots u_n v_n \mid u_i, v_i \in \Sigma^*, u_1 \cdots u_n \in U, v_1 \cdots v_n \in  V \}$ for $U, V \subseteq \Sigma^*$.
Hence, $\perm(L)$ is regularity-preserving
for strictly bounded languages.
More specifically, the above correspondence between
the two language classes is regularity-preserving in both directions.
For commutative constraint languages, a classification
of the complexity landscape has been achieved~\cite{DBLP:conf/cocoon/Hoffmann20}.
By the close relationship between commutative and certain strictly
bounded languages, it is natural to tackle
this language class next.
However, as shown in~\cite{DBLP:conf/cocoon/Hoffmann20},
for commutative constraint languages, we can realize $\PSPACE$-complete
problems, but, by Theorem~\ref{thm:sparse_in_NP},
for strictly bounded languages, the constrained problem is always in $\NP$.
However, by the above relations, Theorem~\ref{thm:bounded_regular_form} for languages in $b_1^* \cdots b_r^*$ is equivalent to~\cite[Theorem 5]{DBLP:conf/cocoon/Hoffmann20}, a representation result
for commutative regular languages.
\end{remark}
 
\begin{comment}

Next, we link the representation
of bounded languages given in Theorem~\ref{thm:bounded_regular_form}
to the languages $L_{j_1, j_2, j_3}$ defined in the beginning of
this section.

\begin{lemma}
\label{lem:L_j1j2j3_intersection}
 Let $L(\mathcal B) = \bigcup_{i=1}^n A_1^{(i)} \cdots A_k^{(i)}$
 with unary regular languages $A_j^{(i)} \subseteq \{a_j\}^*$.
 Then,
 $
  L(\mathcal B) \cap L_{j_1, j_2, j_3} \ne \emptyset 
 $
 if and only if we can find $i_0 \in \{1,\ldots,n\}$
 such that $A_{j_1}^{(i_0)}$ and $A_{j_3}^{(i_0)}$
 do not equal~$\{\varepsilon\}$
 and $A_{j_2}^{(i_0)}$ is infinite.
\end{lemma}
\begin{proof}
 First, suppose we have
   some $w \in L(\mathcal B) \cap L_{j_1, j_2, j_3}$.
   %Then, for some $i_0 \in \{1,\ldots, n\}$, we have
   %$w \in A_1^{(i_0)} \cdots A_k^{(i_0)} \cap L_{j_1, j_2, j_3}$.
   As $L(\mathcal B) \subseteq a_1^* \cdots a_k^*$, we have
   \[
    w = a_1^{|w|_{a_1}} \cdots a_k^{|w|_{a_k}}
   \]
   with $|w|_{a_{j_1}} > 0$, $|w|_{a_{j_3}} > 0$
   and $|w|_{a_{j_2}} \ge |P|$.
   By the pigeonhole principle, as $w$ is read in $\mathcal B$,
      it has to traverse some state twice as it reads 
      the factor $a_{j_2}^{|w|_{a_{j_2}}}$. So, we can pump
      some non-empty factor $a_{j_2}^p$ of it with $0 < p \le |P|$.
      Hence, writing $w = ua_{j_2}^{|w|_{a_{j_2}}}v$ with $u,v \in \Sigma^*$,
      we have, for any $r \ge 0$,
      \[
             ua_{j_2}^{|w|_{a_{j_2}} + rp}v \subseteq L(\mathcal B).
      \] % set u ,v
   Again, using the pigeonhole principle, as we have a 
   finite union \[ L(\mathcal B) = \bigcup_{i=1}^n A_1^{(i)} \cdots A_k^{(i)}, \]
   there
   exists $i_0 \in \{1,\ldots, n\}$ such that
   \[
    ua_{j_2}^{|w|_{a_{j_2}} + rp}v \in A_1^{(i_0)} \cdots A_k^{(i_0)}
   \]
   for infinitely many $r \ge 0$. 
   \todo{Ne, das geht nur bei strictly bounded.}
   This implies $a_{j_2}^{|w|_{a_{j_2}} + rp} \subseteq A_{j_2}^{(i_0)}$
   for infinitely many $r$. Hence $A_{j_2}^{(i_0)}$
   is infinite. Furthermore, we
   get $a_{j_1}^{|w|_{a_{j_1}}} \in A_{j_1}^{(i_0)}$
   and $a_{j_3}^{|w|_{a_{j_3}}} \in A_{j_3}^{(i_0)}$.
  
  
   Conversely, if we have $i_0 \in \{1,\ldots,n\}$
   such that $A_{j_1}^{(i_0)}$ and $A_{j_3}^{(i_0)}$
   do not equal $\{\varepsilon\}$ and $A_{j_2}^{(i_0)}$
   is infinite, then obviously
   \[
    A_1^{(i_0)} \cdots A_k^{(i_0)} \cap L_{j_1,j_2,j_3} \ne \emptyset 
   \]
   and as $A_1^{(i_0)} \cdots A_k^{(i_0)} \subseteq L(\mathcal B)$
   the claim follows.
   
   
   \medskip 
   
   % alternativ mit SCCs und Pfaden, zeigen dass die A_j^i durch Automaten mit weniger als
   % |P| Zuständen erkennbar.
   \qed
\end{proof}
\end{comment}

 Our first result says, intuitively,  
 that if in $A_1 \cdots A_k$ with $A_j$ unary and regular,
 if no infinite unary language $A_j$ over $\{a_j\}$ lies %strictly in the middle 
 between 
 non-empty unary languages
 over a distinct letter\footnote{Hence different from $\{\varepsilon\}$, as $\{\varepsilon\} \subseteq \{a\}^*$
 for $a \in \Sigma$.}  than $a_j$, %than the infinite unary language,
 then $(A_1 \cdots A_k)$\textsc{-Constr-Sync} is in~$\PTIME$.
 
%  Later, in Lemma~\ref{lem:np_hardness} and Theorem~\ref{thm:dichotomy}, we will use
%  the languages $L_{j_1, j_2, j_3}$. They
%  allow us to single out which letters appear infinitely
%  often between other letters in~$L(\mathcal B)$, i.e.,
%  express the opposite of the condition mentioned in Proposition~\ref{prop:stricly_bounded_P}.
 
\begin{propositionrep}
\label{prop:stricly_bounded_P}
 Let $A_j \subseteq \{a_j\}^*$ be unary regular languages
 %, recognized
 %by automata with a single final state, 
 for $j \in \{1,\ldots, k\}$.
 Set $L = A_1 \cdot\ldots\cdot A_k$.
 If for all $j \in \{1,\ldots, k\}$, $A_j$ infinite implies that $A_i \subseteq \{a_j\}^*$
 for all $i < j$ or $A_i \subseteq \{a_j\}^*$ for all $i > j$ (or both), then $L\textsc{-Constr-Sync} \in \PTIME$. 
\end{propositionrep} 
\begin{proof}
 Let $L = A_1 \cdots A_k$ with $A_j \subseteq \{a_j\}^*$ fulfill the assumption.
 If $A_j$ is infinite and for all $i < j$ we have $A_i \subseteq \{a_j\}^*$,
 then $A_1 \cdots A_j \subseteq \{a_j\}^*$, and similarly if
 for all $i > j$ we have $A_i \subseteq \{a_j\}^*$.
 So, by considering $(A_1 \cdots A_j) A_{j+1} \cdots A_k$
 or $A_1 \cdots A_{j-1} (A_j \cdots A_k)$, with $j$ maximal in the former case and minimal in the latter,
 without loss of generality, we can assume $j = 1$
 or $j = k$, i.e., we only have the cases $A_1$
 is infinite, $A_k$ is infinite or both are infinite or none is infinite,
 and, by maximality or minimality of $j$,
 in all these cases the languages $A_2, \ldots, A_{k-1}$ are all finite.
 
 
 
 
 Then, by Lemma~\ref{lem:union_single_final_state}, we can write $A_1$ and $A_k$
 as a finite union of unary languages recognizable by automata with a single final state.
 As concatenation distributes over union, if we do this for
 $A_1$ and $A_k$ and rewrite the language using the mentioned distributivity,
 we get a finite union of languages of the form
 \[
  A_1' A_2 \cdots A_{k_1} A_k'
 \]
 where $A_1'$ and $A_k'$ are recognizable by unary automata with a single final state
 and are either finite or infinite. Hence, by Lemma~\ref{lem:union},
 if we can show that the problem is in $\PTIME$ for each such language,
 the result follows. 
 So, without loss of generality, we assume
 from the start that $A_1$ or $A_k$ are recognizable by automata
 with a single final state.
 
 
 If all $A_j$, $j \in \{1,\ldots,k\}$ are finite, then $L$ is finite, and $L\textsc{-Constr-Sync}\in \PTIME$
 by Lemma~\ref{lem:finite}.
 We handle the remaining cases separately.
 
 \begin{enumerate}
 \item[(i)] Only $A_1$ is infinite.
  
  By assumption, every $A_j \subseteq \{a_j\}^*$, $j \in \{1,\ldots,k\}$,
  is recognizable by a single state automaton. Hence, by Lemma~\ref{lem::unary_single_final}, we can write, as $A_1$ is infinite,
   $A_1 = a_1^i (a_1^p)^*$ with $i \ge 0$ and $p > 0$.
  Let $\mathcal A = (\Sigma, Q, \delta)$ be an input semi-automaton
  for $L\textsc{-Constr-Sync}$.
  As $\delta(Q, a_1) \subseteq Q$,
  we have, for any $n \ge 0$, $\delta(Q, a_1^{n+1}) \subseteq \delta(Q, a_1^n)$.
  So, as $Q$ is finite and the sequence of subsets
  cannot get arbitarily small, for some $0 \le n < |Q|$
  we have $|\delta(Q, a_1^{n+1})| = |\delta(Q, a_1^n)|$.
  But $|\delta(Q, a_1^{n+1})| = |\delta(Q, a_1^n)|$,
  as $\delta(Q, a_1^{n+1}) \subseteq \delta(Q, a_1^n)$,
  implies $\delta(Q, a_1^{n+1}) = \delta(Q, a_1^n)$.
  Then, the symbol $a_1$
  permutes the set $\delta(Q, a_1^n)$.
  Hence, $\delta(Q, a_1^{n+m}) = \delta(Q, a_1^n)$ for any $m \ge 0$.
  So, combining these observations,
  \begin{equation}\label{eqn:case_one_P}
   \{ \delta(Q, a_1^n) \mid n \ge 0 \} = \{ \delta(Q, a_1^n) \mid n \in \{0,\ldots, |Q|-1\} \}
  \end{equation}
  and $\delta(Q, a_1^{|Q| - 1 + m}) = \delta(Q, a_1^{|Q|-1})$
  for any $m \ge 0$. 
  Now, note that the language $A_2 \cdots A_k$
  is finite. So, to find out if we have any
  $a_1^{i + lp} u$ with $u \in A_2 \cdots A_k$
  that synchronizes the input semi-automaton,
  we only have to test if any of the words
  $
   a_i^{i + lp} u,
  $
  with $u \in A_2 \cdots A_k$
  and $l$ such that $i + lp \le \max\{|Q|-1 + p, i\}$,
  synchronizes $\mathcal A$.
  The number (and the length) of these words is linear bounded in $|Q|$ 
  and each could be checked in polynomial time by 
  feeding it into the input semi-automaton for each state and checking
  if a unique state results.
  Hence the problem is solvable in polynomial time.
   
 \item[(ii)] Only $A_k$ is infinite.
 
  Let $u \in A_1 \cdots A_{k-1}$. By assumption, there are only finitely many such
  words $u$. Set $S = \delta(Q, u)$ and $T = \delta(Q, a_k^{|Q|-1})$.
  As in case (i), $a_k$ permutes the states in $T$
  and as $S \subseteq Q$, we have  $\delta(S, a_k^{|Q|- 1}) \subseteq T$.
  So, as $a_k$ permutes $T$, it acts injective on the
  subset $\delta(S, a_k^{|Q|- 1})$.
  This gives $|\delta(S, a_k^{|Q|- 1 + n})| = |\delta(S, a_k^{|Q|- 1})|$
  for any $n \ge 0$. Together with $|\delta(S, a_k^{n + 1})| \le |\delta(S, a_k^{n})|$,
  we have
  \begin{equation}\label{eqn:case_two_P}
   \exists n \in \mathbb N_0 : |\delta(S, a_k^{n})| = 1 \Leftrightarrow |\delta(S, a_k^{|Q|- 1})| = 1.
  \end{equation}
%   If $|S| = 1$, then $u$ is already a synchronizing word and so any
%   word in $u \cdot A_k$ is also synchronizing by Lemma~\ref{lem:append_sync}.
%   So, suppose $|S| > 1$.
%   If $|\delta(S, a_k^n)| = 1$ for any $n$, then there exists
%   some $m \le |Q| - 1$ such that $|\delta(S, a_k^m)| = 1$
%   and $|\delta(S, a_k^{|Q| - 1})| = |\delta(S, a_k^{|Q| - 1 + n})|$
%   for any\footnote{But here, the sets might not be equal, for example, consider
%   some, but not all, states taken from a cycle for $a_k$} $n \ge 0$.
%   The last property is implied as $\delta(S, a_k^{|Q|- 1}) \subseteq T$,
%   and it implies the former, for if we have not mapped $S$ to a singleton
%   before reading $a_k$ at most $|Q|-1$ times, we will never do behind that point. 
%   Similarly, as in case (i), write $A_k = a_k^i(a_k^p)^*$.
  Choose any fixed $N \ge |Q| - 1$ with $a_k^N \in A_k$.
  Then, with the above considerations, we only have to test the finite
  number of words
  \[
   u\cdot a_k^{N}, \quad u \in A_1 \cdots A_{k-1}.
  \]
  The length of these words is linear bounded in $|Q|$ and 
  as each test, i.e., feeding the word into the input semi-automaton
  for each state and testing if a unique state results,
  could be performed in polynomial time, the problem is solvable in polynomial time.
  
 \item[(iii)] Both $A_1$ and $A_k$ are infinite.
  
  This is essentially a combination of the arguments of case (i) and (ii).
  Let $\mathcal A = (\Sigma, Q, \delta)$ be an input semi-automaton
  and $\mathcal B = (\Sigma, P, \mu, p_0, F)$
  be a constraint automaton with $L = L(\mathcal B)$.
  First, consider only the language $A_1 \cdots A_{k-1}$.
  Then, as in case (i), see Equation~\eqref{eqn:case_one_P},% todo argument geht auch für A_1 infinite, ohne dass unitär...
  \[
   \{ \delta(Q, a_1^n) \mid a_1^n \in A_1  \}
    = \{ \delta(Q, a_1^n) \mid 0 \le n < |Q| - 1 + |P|\mbox{ and } a_1^n \in A_1 \}.
  \]
  Note that we have written $0 \le n < |Q| - 1 + |P|$
  and not merely $\le |Q| - 1$ as an upper bound.
  The reason is that otherwise, if $a_1^{|Q|-1} \notin A_1$,
  we might miss the set $\delta(Q, a_1^{|Q|-1})$,
  but as $\delta(Q, a_1^{|Q|-1+m}) = \delta(Q, a_1^{|Q|-1})$
  for any $m \ge 0$ and $A_1$ is infinite, $\delta(Q, a_1^{|Q|-1}) \in \{ \delta(Q, a_1^n) \mid a_1^n \in A_1  \}$.
  However, if $a_1^n \in A_1$ for some $n \ge |Q| - 1 + |P|$,
  then, with $s = \mu(p_0, a_1^{|Q| - 1})$,
  by finiteness of $P$, among
  the states $s, \mu(s,a_1), \ldots, \mu(s, a_1^{n - |Q| + 1})$
  we find $0 \le m \le |P| - 1$ and $0 < r \le |P|$ with $m + r \le |P|$
  such that $\mu(s, a_1^{m+r}) = \mu(s, a_1^m)$.
  Then we have found a cycle and we can skip it, i.e.,
  \begin{align*}
   \mu(p_0, a_1^n) & = \mu(s, a_1^{n - |Q| + 1}) 
                     = \mu(\mu(s, a_1^{m+r}), a_1^{n - |Q| + 1 - (m+r)}) \\
                   & = \mu(\mu(s, a_1^m), a_1^{n - |Q| + 1 - (m+r)}) \\
                   & = \mu(s, a_1^{m + n - |Q| + 1 - m - r}) \\
                   & = \mu(p_0, a_1^{n-r}).
  \end{align*}
  But, as then $\mu(s, a_1^{n - r}) = \mu(s, a_1^n) \in F$
  we find $a_1^{n-r} \in A_1$. 
  Repeating this procedure, if $n - r \ge |Q| - 1 + |P|$,
  we ultimately find $|Q| - 1 \le m < |Q| - 1 + |P|$
  such that $a_1^m \in A_1$
  and $\delta(Q, a_1^{|Q| - 1}) = \delta(Q, a_1^m)$.
  Note that the language $A_2 \cdots A_{k-1}$
  is finite.
  Then, as in case (i), 
  we only have to consider the  words,
  whose length and number is linear bounded in $|Q|$,
  \[
   a_1^n \cdot u,\quad  0 \le n < |Q| - 1 + |P|, a_1^n \in A_1, u \in A_2 \cdots A_{k-1}
  \]
  and the corresponding sets
  \[
   S = \delta(Q, a_1^n \cdot u),
  \]
  and these are all possible sets in $\{ \delta(Q, a_1^n u) \mid a_1^n \in A_1, u \in A_2 \cdots A_{k-1} \}$.
  Fix any such subset $S$.
  Then, as in case (ii) and Equation~\eqref{eqn:case_two_P}, 
  choose any $N \ge |Q| - 1$ with $a_k^N \in A_k$ and
  we only have to compute $\delta(S, a_k^N)$
  and test if it is a singleton set.
  So, in total, we only have to test the words
  \[
   a_1^n \cdot u a_k^N, 0 \le n < |Q| - 1 + |P|, a_1^n \in A_1, u \in A_2 \cdots A_{k-1}.
  \]
  Their length and number is linear bounded in $|Q|$
  and computing the reachable state from each state of the input automaton,
  and testing if a unique state results, could be performed in polynomial
  time. Hence, the overall procedure could be performed in polynomial time.
 \end{enumerate}
 So, we have handled every case and the proof is complete.\qed
 % oder zeigen A_1 oder A_k finite -> nur endlich viele wörter müssen getestet werden
 % mit (iii)
\end{proof}

Now, we state a sufficient condition for \NP-hardness over binary alphabets. 
This condition, together with Proposition~\ref{prop:hom_lower_bound_complexity},
allows us to handle the general case in Theorem~\ref{thm:dichotomy}.
Its application together with Proposition~\ref{prop:hom_lower_bound_complexity} shows, in some respect, that the language $ab^*a$
is the prototypical language giving \NP-hardness.
We give a proof sketch of Lemma~\ref{lem:np_hardness} at the end of this section.

\begin{toappendix}

In the proof of Lemma~\ref{lem:np_hardness}
we will need the following two lemmata.
For $n > 0$, set
\[
 L_n = (\Sigma^* a \Sigma^* b^{|P|} \Sigma^*)^n.
\]
Recall that $\mathcal B = (\Sigma, P, \mu, p_0, F)$.

\begin{lemma}
\label{lem:number_of_b_blocks}
 Let $\Sigma = \{a,b\}$ and $L(\mathcal B) \subseteq a_1^* \cdots a_k^*$
 with $a_i \in \Sigma $ and $n > 0$.
 Then, the following are equivalent:
 \begin{enumerate} 
 \item $L(\mathcal B) \cap L_n \ne \emptyset$,
 
 \item there exist $u_0, \ldots, u_n \in \Sigma^* a \Sigma^*$
  and $p_1, \ldots, p_n \ge |P|$
  such that \[
  u_0 b^{p_n} u_1 \cdots u_{n-1} b^{p_n} u_n \in L(\mathcal B),
  \]
 \item there exist $u_0, \ldots, u_n \in \Sigma^* a \Sigma^*$
  and $p_1, \ldots, p_n > 0$
  such that 
  \[ 
  u_0 (b^{p_1})^* u_1 \cdots u_{n-1} (b^{p_n})^* u_n \subseteq L(\mathcal B).
  \]
 \end{enumerate}
\end{lemma}
\begin{proof}
 That (1) implies (2) is obvious.
 As $p_i \ge |P|$, when reading these factors they have to induce a loop in $\mathcal B$,
 which implies (3).
 Lastly, if (3) holds true, as
 \[
  u_0 b^{|P|\cdot p_1} u_1 \cdots u_{n-1} b^{|P| \cdot p_n} u_n \in L(\mathcal B)
 \]
 and $u_i \in \Sigma^* a \Sigma^*$,
 we also find $u_0 b^{|P|\cdot p_1} u_1 \cdots u_{n-1} b^{|P| \cdot p_n} u_n \in L_n$
 and (1) follows.\qed
\end{proof}

\begin{lemma}
\label{lem:maximal_n}
 Let $\Sigma = \{a,b\}$ and $L(\mathcal B) \subseteq a_1^* \cdots a_k^*$
 with $a_i \in \Sigma $.
 Then, there exists a maximal $n$
 such that $L(\mathcal B) \cap L_n \ne \emptyset$
 and for this maximal $n$,
 we can assume that $u_i \notin \Sigma^* b^{|P|} \Sigma^*$
 for the $u_i$, $i \in \{0,\ldots,n\}$, as in the previous lemma
 and $n \le |P|$. 
\end{lemma}
\begin{proof}
 Recall $\mathcal B = (\Sigma, P, \mu, p_0, F)$
 Note that $\mathcal B$ must necessarily be polycylic (this is a slightly stronger
 claim than Theorem~\ref{thm:bounded_characterization}, as this theorem
 only asserts existence of some polycyclic automaton)
 after removing all states that are not coaccessible, i.e., states from which no final state is reachable,
 which could obviously be done without altering $L(\mathcal B)$.
 For if $\mathcal B$ is then not polycyclic, then some strongly
 connected component does not consists of a single cycle only
 and we find two distinct words $u, v$ and a state $p \in P$
 such that $\mu(p, u) = \mu(p, v) = p$ (see also the forbidden
 pattern in~\cite[Theorem 4.29]{Pin2020}).
 But then, if we choose $x,y \in \Sigma^*$
 such that $\mu(p_0, x) = p$
 and $\mu(p, y) \in F$, we find $x(u+v)^*y \subseteq L(\mathcal B)$.
 Set $m = \max\{|u|, |v|\}$
 Then, for $i > 0$,
 \[
  \{ w \in x(u+v)^*y : |w| \le |x| + i \cdot m\}
 \]
 contains $x(u+v)^i$, and $|x(u+v)^i| = 2^i$.
 So, $L(\mathcal B) \cap \{ w \in \Sigma^* : |w| \le n \}$
 contains at least $2^{\lfloor n - (|x| - |y|) / m \rfloor}$
 many words, i.e., it not sparse.
 Furthermore, as $L \cap \Sigma^n \in O(n^c)$
 as a function of $n$ if and only if $L \cap \{ w \in \Sigma^* : |w| \le n \} \in O(n^{c'})$
 as a function of $n$ for some $c,c' \ge 0$,
 the claim follows.
 
 So, we can assume $\mathcal B$ is polycyclic and every state is coaccessible.
 Now, note that this implies that every loop in $\mathcal B$ (or strongly connected component
 in this case) must be labelled by a single letter, for if we
 have $\mu(p, u) = p$ with $|u|_a > 0$ and $|u|_b > 0$
 and choose again $x,y$ such that $\mu(p_0, x) = p$
 and $\mu(p, y) \in F$,
 we find $xu^ky \in L(\mathcal B)$, which contradict $L(\mathcal B) \subseteq a_1^* \cdots a_k^*$.
 
 But then, note that if, for example, $aba \in L(\mathcal B)$,
 we must have $|P| \ge 2$, as $\mu(p_0, ab) \notin \{ p_0, \mu(p_0,a) \}$.
 Similarly, if we have a word that switches letters, every time a letter-switch
 occurs the state we end up in $\mathcal B$ must be a new state not visited before,
 for otherwise we would have a loop whose transition are not exclusively
 labelled by a single letter.
 
 So, this implies that 
 if we have a word as written in Lemma~\ref{lem:number_of_b_blocks}
 in $L(\mathcal B)$, then $n \le |P|$
 which implies that we can find a maximal $n$.
 That $u_i \notin \Sigma^* b^{|P|} \Sigma^*$
 is also implied by Lemma~\ref{lem:number_of_b_blocks}
 and the maximality of $n$. \qed
\end{proof}
\end{toappendix}


\begin{lemmarep}
\label{lem:np_hardness}
 Suppose $\Sigma = \{a,b\}$. %und a,b vertauscht unten mit homsatz.
 Let $L(\mathcal B) \subseteq \Sigma^*$ be letter-bounded.
 Then, $L(\mathcal B)$\textsc{-Constr-Sync}
 is $\NP$-hard %if and only %nebenbei zwischen die a^+ nochmal sigma packen bringt nichts.
 % da man annehmen kann a_{i+1} ungleich a_i
 if $L(\mathcal B) \cap \Sigma^* a  b^{|P|}b^*  a \Sigma^* \ne \emptyset$.
 % nur if teil, weil if and only wird impliziert im theorem!
\end{lemmarep}
\begin{toappendix}
\begin{figure}[htb]
     \centering
     \hspace*{-2.5cm}
\includegraphics[width=17cm]{reduction.png}
  \caption{%Schematic illustration of the reduction from the proof of Proposition \ref{prop:stricly_bounded_np_hard}.
   The reduction from the proof of Lemma~\ref{lem:np_hardness}
   in the special case $J = 3$ (see the proof for the definition of $J$)
   and two input automata $\mathcal A_1, \mathcal A_2$ over $\{b\}$. The automata
   $\mathcal A_{i,j}$ are inflated, according to Definition~\ref{def:inflate_aut},
   copies of $\mathcal A_i$
   for $i \in \{1,2\}$, $j \in \{1,2,3\}$. 
   The letter $a$ maps
   every state not associated with a path inside each $\mathcal A_{i,j}$ to the last innermost state that 
   is hit by an $a$ along the path leading into this automaton. This is only drawn for $\mathcal A_{1,1}$ but left
   out for the other automata, also, to give a more ``high-level'' drawing, the $b$-transitions
   are not drawn. On the right end is the sink state $t$. The paths stay inside the automata but leave
   as soon as an $a$ is read.}
  \label{fig:reduction}
\end{figure}

\begin{proof}[Proof of Lemma~\ref{lem:np_hardness}]
First, using Lemma~\ref{lem:maximal_n},
choose $J > 0$ maximal such that
\[
 L(\mathcal B) \cap (\Sigma^* a \Sigma^* b^{|P|} \Sigma^*)^J \ne \emptyset.
\]
As stated in the lemma, we have $J \le |P|$ (which implies the constrution to follow could
be carried out in polynomial time).
Then, by Lemma~\ref{lem:number_of_b_blocks}, there
exist $u_0, \ldots, u_J \in \Sigma^* a \Sigma^*$
and $p_1, \ldots, p_J > 0$ ($J > 0$) such that \todo{Anderer Bezeichner als $J$?}
\[
 u_0 (b^{p_1})^* u_1 \cdots u_{J-1} (b^{p_J})^* u_J \subseteq L(\mathcal B).
\]
 Let $N$ be $|P|$ times the least common multiple of the numbers $p_1, \ldots, p_J$.
 We give a reduction from the {\sc DFA-Intersection} for unary 
 input automata, which is \NP-complete in this case~\cite{stockmeyer1973word,fernau2017problems}.
 Let $\mathcal A_i = (\{b\}, Q_i, \delta_i, q_i, F_i)$
 for $i \in \{1,\ldots,k\}$ be unary input automata, and we want to know
 if they all accept a common word. The problem remains
 \NP-complete if we assume for no input automaton, a start state is also a final state.
 This is easily seen but could also be shown similar to~\cite[Proposition 1]{DBLP:conf/ictcs/Hoffmann20}.
 Also, we can assume $F_i \ne \emptyset$ for all $i \in \{1,\ldots,k\}$.

 We are going to construct a semi-automaton $\mathcal C = (\{a,b\}, Q, \delta)$. %intuition?

 Write $u_i = u_{i,1} \cdots u_{i,|u_i|}$ with $u_j \in \Sigma$.
 For each $i \in \{0,\ldots,J\}$, we construct a path labelled with $u_i$.
 Formally, let $P_i = \{ q_{i, 0}, \ldots, q_{i,|u_i|} \} \subseteq Q$
 be new states and set
 \[
  \delta(q_{i,j-1}, u_j) = q_{i,j}.
 \]
 Then, for each $\mathcal A_i$
 we construct $J$ (disjoint) 
 copies of $\mathcal A_i$ and inflate
 them according to Definition~\ref{def:inflate_aut} by $N$.
 Call the results $\mathcal A_{i, 1}, \mathcal A_{i,2}, \ldots, \mathcal A_{i,J}$
 with $\mathcal A_{i,j} = (\{b\}, Q_{i,j}, \delta_{i,j}, s_{i,j}, F_{i,j})$.
 Note these are unary automata over the letter $b$.
 Also, let $t \in Q$ be a new state, which will be a (global) sink state in $\mathcal C$, i.e.,
 we set $\delta(t, a) = \delta(t, b) = t$.
 Next, we describe how we interconnect these automata with the paths and with $t$.
 See also Figure~\ref{fig:reduction} for a sketch of the reduction in the special case $J = 3$
 and two input automata.
 
 
 
 \begin{enumerate}
 \item Let $j \in \{1,\ldots, J\}$. For each final state $q \in F_{i,j}$
  let $P_{i,q}$ be a disjoint copy of the path $P_i$ constructed above,
  except for one final state $q$ were we simply retain the path $P_i$, but also name it by $P_{i,q}$.
  By identifying states, we mean states that we have previously constructed are now merged
  to a single state in $Q$. We have to pay attention that this procedure does not introduces
  any non-determinism.
  We identify the state $q_{i,0}$ with $q$
  and continue to identify the states $q_{i,j}$ and $q' \in Q_{i,j}$
  if $q_{i,j-1}$ and $q'' \in Q_{i,j}$ were identified
  and $u_{i,j} = b$ and $q' = \delta_{i,j}(q'', b)$. As $u_i \in \Sigma^* a \Sigma^*$, this process has to come to a halt
  before we have identified $< J$ states.
  Note that the first state such that $q_{i,j-1}$ and $q''\in Q$
  were identified but not $q_{i,j}$ and $\delta_{i,j}(q'', b)$, i.e., were $u_{i,j} = a$,
  we have added an $a$-transition to $q_{i,j}$
  from $q'' = q_{i,j-1}$ in $\mathcal A_{i,j}$, i.e., this is the first
  $a$-transition we have added to $\mathcal A_{i,j}$ and it branches out of $\mathcal A_{i,j}$.
  
  Then, if $j \le J - 1$,
  identify the state $q_{i,J}$ with the start state $s_{i,j+1}$ of $\mathcal A_{i,j+1}$, i.e.,
  the path $P_{i,q}$ ends at this state.
  And if $j = J$, we identify the state $q_{i,J}$ with $t$.
  
 \item For the path $P_0$ identify its end state $q_{i,|u_0|}$
  with the start state $s_{i,1}$ of $\mathcal A_{i,1}$.
     
 \item Up to now, we still hav emissing transitions. In all the paths created, 
  every missing $b$-transition, i.e., were we have a state with an $a$-transition
  leading out but no $b$-transition, we add a self-loop labelled with $b$ to that state.
  For each path $P$ (including the copies constructed in the first step)
  let $p \in P$ be that state closest to the end state, but that does not equal
  the end state (by the identifications above, some end state might already have an $a$-transition
  that goes out of some automaton $\mathcal A_{i,j}$) and has an outgoing $a$-transition.
  Such a state exists as the $u_i \in \Sigma^* a \Sigma^*$.
  \todo{Hier uU statt auf den Zustand immer auf den letzten Zustand davor mit einer $a$-Transition mappen?}
  Then, for every state in $P$ that does not have an $a$-transition
  we add an $a$-transition going to $p$.
  Consider $\mathcal A_{i,j}$ and let $P$ some path (the specific choice does not matter)
  ending at the start state of $\mathcal A_{i,j}$.
  For each state $q \in Q_{i,j}$ that does not has an outgoing $a$-transition up to now,
  add an $a$-transition going to the state $p \in P$ described above in that path.
  This ensures later that, by reading an $a$, we end up in a well-defined situation.
 \end{enumerate}
 Then, put all the states created so far, i.e., those of the $\mathcal A_{i,j}$
 and those of the paths constructed, into $Q$ (note for each $i \in \{1,\ldots,k\}$
 we have constructed paths and automata, intuitively we have copied each $\mathcal A_i$, inflated
 the copies and interconnected them with the paths given by the $u_i$)
 and let $\delta$ be the transition as defined above or as given by $\mathcal A_{i,j}$
 on the state of these automata.
 
 
 We need the following property of $\mathcal C$. Suppose $i \in \{1,\ldots,k\}$, 
 $j \in \{1,\ldots,|J|\}$ and $w \in \{a,b\}^*$.
 
  \medskip 
 
\noindent\underline{Claim:}
  Let $q \in Q_{i,j} \setminus F_{i,j}$ with $\delta(q, w) = t$.
  Then, there exist 
  \[ 
  u_1, u_2 \cdots, u_{|J|-i+1} \subseteq \{b\}^*
  \]
  and $u,v\in \{a,b\}^*$
  such that $|u_i| \ge N$ and $|u_i|$ is divisible by $N$
  for all $i \in \{1,\ldots,|J|-i+1\}$
  and $v_1, \ldots, v_{|J|-i+1} \in \{a,b\}^*a\{a,b\}^*$
  so that 
  \[
   w = vu_1 v_1 u_2 v_2 \cdots u_{|J|-i+1} v_{|J|-i+1} u
  \]
  and $v \notin \Sigma^* b^N \Sigma^*$.
 \begin{quote} % genauer, muss jeweils teile dahinter von start auf final?, und wegen maximalitt |J| auch nicht mehr soclhe mit "echten" a's dazwischen.
     \emph{Proof of the Claim.} First, the state $q \in Q_{i,j}$ has to be mapped to a final
     state, which could only be done by a word containing at least $N$
     times the letter $b$,
     as in the inflated construction we can only go from non-auxiliary states
     to non-auxiliary states by reading at least that number of letters.
     However, before that we might read some word $v \in \{a,b\}^*$ that moves states around, does
     not has a consecutive sequence of more than $N$ $b$'s and hence, every $a$
     goes back to the start state. But at some point, this has to come to an end and we have
     to read a sequence of more than $N$ consecutive $b$'s.
     Additionally, by the construction of the inflation, the word
     that moves from a non-auxiliary state to another non-auxiliary state
     must have a number of $b$'s that is divisible by $N$.
     Also, observe that such a word must consists entirely of $b$, because
     for non-final states in $\mathcal A_{i,j}$ every $a$ maps back to the start state.
     Then, by construction (recall $u_i \in \Sigma^* a \Sigma^*$ for the labels
     of the paths constructed above) to move between the automata $\mathcal A_{i,j}$
     inside of $\mathcal C$
     we have to traverse a path
     where, on some part, we can only move forward by reading the letter $a$.
     After this, when we are at the start state of $\mathcal A_{i,j+1}$,
     as by assumption the start state is not final, we again have to read at least $N$
     times the letter $b$ and so on, until we have reached a final 
     state in $\mathcal A_{i,|J|}$.
     Then, we have to read at least one $a$ to map the final state to $t$, from which
     on, as $t$ is a sink state, we can read any word. 
     \emph{[End, Proof of the Claim]}
 \end{quote}
 
 
 The automaton $\mathcal C$ has a synchronizing word in $L$
 if and only if all the $\mathcal A_i$, $i \in \{1,\ldots,k\}$,
 accept a common word.
 
 \begin{enumerate}
 \item Assume we have a word $b^n$ accepted by all $\mathcal A_i$ for $i \in \{1,\ldots,k\}$. 
 Then, for
 \[
  w = u_0 b^{N\cdot n} u_1 \cdots u_{J-1} b^{N\cdot n} u_J
 \] 
 we have $w \in L$ and $w$ synchronizes $\mathcal A$.
 Note that, after reading $u_{j-1}$,
 the automaton $\mathcal A_{i,j}$
 is either in its start state, or the final $a$ in $u_{j-1}$
 has mapped some state in $\mathcal A_{i,j}$ to a state outside of $Q_{i,j}$.
 So, when reading $b^{N\cdot n}$, 
 as $\mathcal A_{i,j}$ equals  the inflation of $\mathcal A_i$ by $N$,
 we end up in a final state $F_{i,j}$.
 Then, we read $u_j$ to map those final states to the start state
 of the next automaton $\mathcal A_{i,j+1}$ or to $t$ if $j = J$.
 Note that all states in-between are either mapped
 to a start state of some $\mathcal A_{i,j}$, moved inside of some
 $\mathcal A_{i,j}$, or, when an $a$ is read and they are not mapped
 back to a state that ultimately ends in a start state of some $\mathcal A_{i,j}$
 are moved toward the state $t$.
 As we always read enough $a$ to always make a step towards the sink state $t$
 the result follows.\todo{genauer}
 
 
 
 \item  Assume $\mathcal A$ has a synchronizing word $w \in L$.
  Then, as $t$ is a sink state, the word $w$ must map every state to $t$.
  Consider the start state of some $\mathcal A_{i,1} = (\{b\}, Q_{i,1}, \delta_{i,1}, q_{i,1}, F_{i,1})$.
  By the above claim,  
  there exist $u_1, u_2 \cdots, u_{J} \subseteq \{b\}^*$
  such that $|u_i| \ge N$ and $|u_i|$ is divisible by $N$
  for all $i \in \{1,\ldots,|J|\}$
  and $v_1, \ldots, v_{J} \in \{a,b\}^*a\{a,b\}^*$ and $v, u \in \{a,b\}^*$
  so that 
  \[ 
    w = vu_1 v_1 u_2 v_2 \cdots u_{J} v_{J} u.
  \]
  By the above claim, Lemma~\ref{lem:maximal_n} and the maximal choice of $J$,
  we have 
  \[ 
  \{ v, v_1, \ldots, v_J \} \cap \Sigma^* b^{|P|} \Sigma^* = \emptyset,
  \] 
  i.e.,
  these words does not contains a sequence of more than $|P|$, and so in particular not more than $N$,
  consecutive $b$'s.
  
  
  Now, let $b^n$ be a maximal non-empty factor whose length $n$ is divisible by $N$ of $vu_1 v_1$
  and using only the letter $b$.
  Note that, by construction of $\mathcal A_{i,1}$,
  if we write $vu_1 v_1 = x b^n y$,
  we have $\delta_{i,1}(q_{i,1}, x) = q_{i,1}$.
  Then, we claim that $b^{n / N}$ is accepted
  by every automaton $\mathcal A_i$. 
  Fix an index $i \in \{1,\ldots,k\}$.
  By the construction of the inflation, this is equivalent
  with the condition that $b^n$ drives every automaton
  $\mathcal A_{i,j}$ for $j \in \{1,\ldots,|J|\}$
  from the start state to some final state.
  Suppose this is not the case. As the automata $\mathcal A_{i,j}$
  are isomorphic, i.e., they are copies of each other, we can assume this is not the case for $\mathcal A_{i,1}$, i.e., 
  we have $\delta_{i,1}(q_{i,1}, b^n) \notin F_{i,1}$.
  Then, consider the following suffix of $w$ (recall $xb^n y = v u_1 v_1$, and $y$ has to start with an $a$) 
  \[
   y u_2 v_2 \cdots u_{|J|} v_{|J|} u.
  \] 
  Note that if we have in $u$ a consecutive sequence of $b$'s
  of length more than $N$, the rest of $u$ also must consist of $b$'s only, i.e.,
  we cannot read an $a$ anymore.
  For suppose this is not the case and $u \in \Sigma^* b^N \Sigma^* a \Sigma^*$.
  We have $\delta(q_{i,1}, xb^n) = \delta(q_{i,1}, b^n) \in Q_{i,1} \setminus F_{i,1}$.
  By assumption, $\delta(q_{i,1}, w) = t$,
  and so we must have $\delta(q_{i,1}, y u_2 v_2 \cdots u_{|J|} v_{|J|} u) = t$.
  Applying the above claim again,
  yields that we can factorize $y u_2 v_2 \cdots u_{|J|} v_{|J|} u$
  such that we have at least $|J|$ blocks of consecutive 
  $b$'s broken up by at least one occurrence of the letter $a$
  between each such block.
  However, then 
  then
  \[
   w = x b^n y u_2 v_2 \cdots u_{|J|} v_{|J|} u,
  \]
  as $y$ starts with an $a$, we would get a factorization
  of $w$ with $|J| + 1$ blocks of consecutive $b$'s separated by words
  with at least one $a$, which is not possible by the maximal choice
  of $J$ and Lemma~\ref{lem:number_of_b_blocks}.
  
 \end{enumerate}
 So, this shows that this is a valid reduction.\qed
\end{proof}
\end{toappendix}




So, finally, we can state our main theorem of this section.
Recall that by Theorem~\ref{thm:sparse_in_NP},
and as the class of bounded regular languages equals
the class of sparse regular languages~\cite{DBLP:journals/eik/LatteuxT84}, for bounded regular constraint
languages, the constrained problem is, in our case, in \NP. 
 
 
\begin{theoremrep}[Dichotomy Theorem]
\label{thm:dichotomy}
 % a_i != a_{i+1} sonst nichts weiter
 % dann a_{j_1} auf a und so weiter hom bild
 % hom bild a^* b^* a^*
 % aber a_{j_2} verschieden von a_{j_1}, a_{j_3} über die vereinigne
 % wie in entscheidungsverfahren.
 %
 % aber aufpassen, P ist anders!!!! aber dann schlussfolgern für das P' auch unendlich
 % oder 2^{|P|} nehmen
 %
 % oder wie so eine "sigma"-menge, also für jedes n existiert eins - schnitt/vereinigung - schreiben. aber da nicht klar ob regulär.
 Let $a_1, \ldots, a_k \in \Sigma$ be a sequence of letters
 and $L \subseteq a_1^* \cdots a_k^*$ be regular.
 The problem $L\textsc{-Constr-Sync}$
 is \NP-complete if
 \[
  L \cap \left(\bigcup_{\substack{1 \le j_1 < j_2 < j_3 \le k \\ a_{j_2} \notin \{a_{j_1}, a_{j_3}\} }} L_{j_1,j_2,j_3} \right) \ne \emptyset
 \]
 with $L_{j_1, j_2, j_3} = \Sigma^* a_{j_1} \Sigma^* a_{j_2}^{|P|} \Sigma^* a_{j_3} \Sigma^*$
 for $1 \le j_1 < j_2 < j_3 \le k$ and solvable in polynomial time otherwise.
\end{theoremrep}
\begin{proof}
 Set $L = L(\mathcal B)$.
 %By Theorem~\ref{thm:bounded_regular_form}, we can
 %write % auf lemma/theorem verweisen, dass es immer so geht.
 %$L(\mathcal B) = \bigcup_{i=1}^n A_1^{(i)} \cdots A_k^{(i)}$ % erwähnen k gleich da man eps wählen kann, vielleicht als lemma diese form hinschreiben. todo
 %with unary regular languages $A_j^{(i)} \subseteq \{a_j\}^*$
 %for $j \in \{1,\ldots,k\}$.
 Let $j_1, j_2, j_3 \in \{1,\ldots,k\}$
 be such that $a_{j_2}\notin\{a_{j_1},a_{j_3}\}$, $j_1 < j_2 < j_3$
 and
 \[
  L(\mathcal B) \cap L_{j_1, j_2, j_3} \ne \emptyset.
 \]
 Then, there exists a word $u_1 a_{j_1} u_2 a_{j_2}^{|P|} u_3 a_{j_3} u_4 \in L(\mathcal B)$
 with $u_1, u_2, u_3, u_4 \in \Sigma^*$.
 By the pigeonhole principle, when reading the factor $b^{|P|}$,
 at least one state has to be traversed twice 
 and we find $p > 0$ such that $u_1 a u_2 b^{|P| + i\cdot p} u_3 a u_4$
 for any $i \ge 0$.
 
 
 
 Define a homomorphism $\varphi : \Sigma^* \to \{a,b\}^*$
 by $\varphi(a_{j_1}) = \varphi(a_{j_3}) = a$,
 $\varphi(a_{j_2}) = b$
 and, for the remaining letters, $\varphi(a) = \varepsilon$,
 if $a \in \Sigma \setminus \{a_{j_1}, a_{j_2}, a_{j_3}\}$.
 Then, $\varphi(L) \subseteq \varphi(a_1)^* \cdots \varphi(a_k)^*$
 is letter-bounded. % zeigen, dass unter hom abgeschlossen. aber klar
 Set $\Gamma = \{a,b\}$ and let $\mathcal B' = (\Gamma, P', \mu', p_0', F')$
 be a recognizing PDFA for $\varphi(L)$.
 %By Lemma~\ref{lem:L_j1j2j3_intersection},
 %there exists $i_0 \in \{1,\ldots,n\}$
 %such that $A_{j_1}^{(i_0)}$ and $A_{j_3}^{(i_0)}$
 %do not equal~$\{\varepsilon\}$
 %and $A_{j_2}^{(i_0)}$ is infinite.
 %As $\varphi$ is a homomorphism, we have 
 %$\varphi(A_1^{(i_0)} \cdots A_k^{(i_0)}) = \varphi(A_1^{(i_0)}) \cdots \varphi(A_k^{(i_0)})$. Furthermore,
 %$a^+ \cap \varphi(A_{j_1}^{(i_0)}) \ne \emptyset$,
 %$a^+ \cap \varphi(A_{j_3}^{(i_0)}) \ne \emptyset$
 %and $b^{|P'|}b^* \cap \varphi(A_{j_2}^{(i_0)}) \ne \emptyset$.
 %As the language $\varphi(A_1^{(i_0)}) \cdots \varphi(A_k^{(i_0)})$
 %is contained in $\varphi(L)$,
 %this yields
 We have
 \[
  \varphi(u_1) a \varphi(u_2) b^{|P| + i\cdot p} \varphi(u_3) a \varphi(u_4) \in \varphi(L) 
 \]
 for any $i \ge 0$. So,
 $
 \varphi(L) \cap \Gamma^* a^+ \Gamma^* b^{|P'|}b^* \Gamma^* a^+ \Gamma^* \ne \emptyset.
 $
%  However, note that, for $u \in \{a,b\}^*$,
%  \begin{multline*}
%       u \in \varphi(A_1^{(i_0)}) \cdots \varphi(A_k^{(i_0)}) \cap \Gamma^* a^+ \Gamma^* b^{|P'|}b^* \Gamma^* a^+ \Gamma^*
%   \\ \Leftrightarrow 
%   u \in \varphi(A_1^{(i_0)}) \cdots \varphi(A_k^{(i_0)}) \cap \Gamma^* a^+  b^{|P'|}b^* a^+ \Gamma^*.
%  \end{multline*}
% So,
% $
%  \varphi(L) \cap \Gamma^* a^+ b^{|P'|} b^* a^+ \Gamma^* \ne \emptyset. %todo genauer
% $
 By Lemma~\ref{lem:np_hardness}, $\varphi(L)$\textsc{-Constr-Sync}
 is \NP-hard and so, by Proposition~\ref{prop:hom_lower_bound_complexity},
 also $L\textsc{-Constr-Sync}$ is \NP-hard, and so, with Theorem~\ref{thm:sparse_in_NP},
 \NP-complete.
 
 
 Now, suppose
  $
 L(\mathcal B) \cap \left(\bigcup_{\substack{1 \le j_1 < j_2 < j_3 \le k \\ a_{j_2} \notin \{a_{j_1}, a_{j_3}\} }} L_{j_1,j_2,j_3} \right) = \emptyset.
 $
  By Theorem~\ref{thm:bounded_regular_form}, we can
 write % auf lemma/theorem verweisen, dass es immer so geht.
 $L(\mathcal B) = \bigcup_{i=1}^n A_1^{(i)} \cdots A_k^{(i)}$ % erwähnen k gleich da man eps wählen kann, vielleicht als lemma diese form hinschreiben. todo
 with unary regular languages $A_j^{(i)} \subseteq \{a_j\}^*$
 for $j \in \{1,\ldots,k\}$.
 Then, 
 \[ 
 ( A_1^{(i)} \cdots A_k^{(i)} ) \cap \left(\bigcup_{\substack{1 \le j_1 < j_2 < j_3 \le k \\ a_{j_2} \notin \{a_{j_1}, a_{j_3}\} }} L_{j_1,j_2,j_3} \right) = \emptyset
 \]
 for any $i \in \{1, \ldots, n\}$.
 However, this implies that for any $i \in \{1,\ldots,n\}$, if there exists $j \in \{1,\ldots, k\}$
 such that $A_j^{(i)}$ is infinite, 
 then for all $j' < j$, or for all $j' > j$, % besonders wenn es keine gibt, als remark nach der proposition?
 we have $A_{j'} \subseteq \{a_j\}^*$ (recall that if $A_{j'} = \{\varepsilon\}$, then
 this is also fulfilled).
 Hence, by Proposition~\ref{prop:stricly_bounded_P},
 we have $(A_1^{(i)} \cdots A_k^{(i)})\textsc{-Constr-Sync} \in \PTIME$
 and then, by Lemma~\ref{lem:union},
 $L(\mathcal B)\textsc{-Constr-Sync} \in \PTIME$.\qed
\end{proof}

As the languages $L_{j_1, j_2, j_3}$ are regular, we
can devise a polynomial-time algorithm which checks the condition
mentioned in Theorem~\ref{thm:dichotomy}. 
 
\begin{corollary} %\todo{nicht $sigma = {a-1, ...m, a_k}$ schreiben, weil es suggeriert die buchstaben wären alle verschieden.}
 Given a PDFA $\mathcal B$ and a sequence of letters $a_1, \ldots, a_k$
 as input such that $L(\mathcal B) \subseteq a_1^* \cdots a_k^*$,
 the complexity of $L(\mathcal B)$\textsc{-Constr-Sync}
 is decidable in polynomial-time.
\end{corollary}
\begin{proof}
 An automaton for each $L_{j_1, j_2, j_3}$
 has size linear in~$|P|$. So, by the product automaton construction~\cite{HopUll79}, non-emptiness of
 $L(\mathcal B)$ with each $L_{j_1, j_2, j_3}$
 could be checked in time $O(|P|^2)$.
 Doing this for every $L_{j_1, j_2, j_3}$
 gives a polynomial-time algorithm
 to check non-emptiness of the language written
 in Theorem~\ref{thm:dichotomy}.~\qed
\end{proof}

\begin{example}
 For the following constraint languages CSP is \NP-complete: $ab^*a$,
 $aa(aaa)^*bbb^*d \cup a^*b \cup d^*$, $bbcc^*d^* \cup a$.
 
 For the following constraint languages CSP is in \PTIME: $a^5bd \cup cd^4$,
 $a^5bd \cup cd^*$, $aa^*bbbbcd^* \cup bbbdd^*d$.
\end{example}

\begin{proof}[Proof Sketch for Lemma~\ref{lem:np_hardness}]
 We construct a reduction from an instance
 of $\textsc{DisjointSetTransporter}$\footnote{Note that the problem $\textsc{DisjointSetTransporter}$ is over a unary alphabet, but for $L\textsc{-Constr-Sync}$
 we have $|\Sigma| > 1$. Indeed, we need the additional letters in $\Sigma$.}
 for unary input automata.
 %, which is $\NP$-complete in
 %this case, by Theorem~\ref{prop:set_transporter_np_complete},
 %to $L\textsc{-Constr-Sync}$ for $L$ as written in the statement\footnote{Note that the problem $\textsc{DisjointSetTransporter}$ is over a unary alphabet, but for $L\textsc{-Constr-Sync}$
 %we have $|\Sigma| > 1$. Indeed, we need the additional letters in $\Sigma$ in our reduction.}.
 
 To demonstrate the basic idea, we only do the proof
 in the case $L \subseteq a^* b^* a^*$.
 %By Theorem~\ref{thm:bounded_regular_form},
 %we can write $L = \bigcup_{i=1}^n A_1^{(i)} A_2^{(i)} A_3^{(i)}$
 %with regular languages $A_1^{(i)}, A_3^{(i)} \subseteq \{a\}^*$
 %and $A_2^{(i)} \subseteq \{b\}^*$.
 By assumption we can deduce $a^{r_1} b^{r_2} a^{r_3} \in L(\mathcal B)$
 with $p_2 \ge |P|$ and $r_1, r_3 \ge 1$.
 By the pigeonhole principle, in $\mathcal B$, 
 when reading the factor $b^{r_2}$, at least one state has to be traversed twice.
 Hence, we find $0 < p_2 \le |P|$ such that $a^{r_1} b^{r_2 + i\cdot p_2} a^{r_3}
 \subseteq L(\mathcal B)$ for each $i \ge 0$.
 %Then, by Lemma~\ref{lem:L_j1j2j3_intersection},
 %we must find $i_0 \in \{1,\ldots,n\}$
 %such that $a^+ \cap A_1^{(i_0)} \ne \emptyset$, $a^+ \cap A_3^{(i_0)} \ne \emptyset$
 %and $A_2^{(i_0)}$ is infinite.
%  As a shorthand,
%  we set $A_1 = A_1^{(i_0)}$, $A_2 = A_2^{(i_0)}$ and $A_3^{(i_0)} = A_3$.
% %  By Lemma~\ref{todo}, the languages could be written
% %  as a finite union of languages recognizable
% %  by unary automata with a single final state.
% %  As concatenation distributes over union, 
% %  %we can then write $A_1^{(i_0)} A_2^{(i_0)} A_3^{(i_0)}$
% %  %as a finite union of languages
% %  without loss of generality, as by such a rearranging the above assumption
% %  still holds true for at least one part of the union, we can 
% %  suppose $A_1$,  $A_2$ and $A_3$
% %  are recognizable by unary automata with a single final state.
% Then, it is easy to see that we find numbers $r_i, p_i$ such that
% $
%     a^{r_1}(a^{p_1})^* \subseteq A_1,
%     b^{r_2}(b^{p_2})^* \subseteq A_2, 
%     a^{r_3}(a^{p_3})^* \subseteq A_3
% $
% with, by the other assumptions, $p_2 > 0$ ($A_2$ infinite)
% and $r_1 + p_1 > 0$, $r_3 + p_3 > 0$ ($A_1$, $A_3$ non-empty and do not equal $\{\varepsilon\}$).
% todo hinschreiben, wenn ncihts gesagt immer complete und deterministic


Let $\mathcal A = (\{c\}, Q, \delta)$ and $(\mathcal A, S, T)$
be an instance of \textsc{DisjointSetTransporter}.
We can assume $S$ and $T$ are non-empty, as for $S = \emptyset$
it is solvable, and if $T = \emptyset$ we have no solution.
Construct $\mathcal A' = (\Sigma, Q', \delta')$
by setting
$
 Q' = S_{r_2} \cup \ldots \cup S_{1} \cup Q \cup Q_1 \cup \ldots \cup Q_{p_2-1} \cup \{ t \},
$
where $t$ is a new state, $S_i = \{ s_i \mid s \in S \}$ for $i \in \{1,\ldots, r_2 \}$
are pairwise disjoint copies of $S$
and $Q_i = \{ q^i \mid q \in Q \}$ are\footnote{Observe
that by the indices a correspondence between the sets
is implied. The index
in $Q_i$ at the top to distinguish, for $s \in S$ and $i \in \{1,\ldots,\min\{r_2, p_2-1\}\}$, between
 $s_i \in S_i$ and $s^i \in Q_i$. Hence, for each $s \in S$ and $i \in \{1,\ldots, r_2\}$,
 the states $s$ and $s_i$ correspond to each other, and for $q \in Q$
 and $i \in \{1,\ldots, p_2-1\}$ the states $q$ and $q^i$.} 
 also pairwise disjoint 
copies of $Q$. Note that also $S_i \cap Q_j = \emptyset$
for $i \in \{1,\ldots, r_2 \}$ and $j \in \{1,\ldots, p_2-1\}$.
Set $S_0 = S$ %and $Q_0 = Q$ 
as a shorthand.
Choose any $\hat s \in S_{r_2}$, then, for $q \in Q$ and $x \in \Sigma$, the transition function is given by
\[
 \delta'(q, x) = \left\{
 \begin{array}{ll}
  s_{i-1} & \mbox{if } x = b \mbox{ and } q = s_i \in S_i \mbox{ for some } i \in \{1,\ldots, r_2\}; \\ 
  \hat s & \mbox{if } x = a \mbox{ and } q \in (Q \cup Q_1 \cup \ldots \cup Q_{p_2-1}) \setminus S; \\
  s_{r_2} & \mbox{if } x= a \mbox{ and } q = s_i \in S_i \mbox{ for some } i \in \{0,\ldots,r_2\}; \\
  t       & \mbox{if } x = a \mbox{ and } q \in T; \\
  q^{p_2-1} & \mbox{if } x = b \mbox{ and } q \in Q; \\
  q^{i-1} & \mbox{if } x = b \mbox{ and } q = q^i \in Q_i \mbox{ for some } i \in \{2,\ldots,p_2-1\}; \\
  \delta(q, c) & \mbox{if } x = b \mbox{ and } q = q^1 \in Q_1; \\
  q       & \mbox{otherwise}.
 \end{array}
 \right.
\]

\newcommand{\automatacloudother}[2][.44]{%
	\begin{scope}[#2]
		\node [rectangle,draw,thick,text width=8.1cm,minimum height=7.6cm,
		text centered,rounded corners, fill=white, name = re] {};
\end{scope}}    


\newcommand{\innerstateloop}{
\begin{scope}
  \node[state] (s1) at (0,0) {}; \node (s1label) at (0.5,-.1) {$\in Q$};
  \node[state] (s11) at (-0.5,0.6) {};  \node (s11label) at (0.32,0.6) {$\in Q_{p_2-1}$};
  %\node[state] (s14) at ( 0.5,0.6);
  \node (s12) at (-0.25,1.2) {};
  \node (s13) at ( 0.25,1.2) {};
  \path[->] (s1)  edge [bend left] node [left] {$b$} (s11);
%  \path[->] (s14) edge [bend left] node [right] {$b$} (s1);
   \path[->] (s11) edge [bend left] node [left] {$b$} (s12);
%   \path[->] (s13) edge [bend left] node [right] {$b$} (s14);
  \draw[dashed] (-0.25,1.2) -- (0.25,1.2);
\end{scope}
}

\begin{figure}[htb]
     \centering
    \scalebox{.65}{    
 \begin{tikzpicture}
 \tikzset{every state/.style={minimum size=1pt},>=stealth'}
 \node (cloud) at (0,0) {\tikz \automatacloudother{fill=gray!0,thick};};
 
  \node (reset1) at (0,3) {};
  \node (reset2) at (-10,2.5) {};
  \path[->] (reset1) edge [bend right] node [above] {$a$} (reset2);
  
  \node (reset3) at (0,-3) {};
  \node (reset4) at (-10,-2.5) {};
  \path[->] (reset3) edge [bend left] node [below] {$a$} (reset4);
  
  \node (reset5) at (-5.4,2.8) {};
  \node (reset6) at (-10,2.5) {};
  \path[->] (reset5) edge [bend right] node [above,pos=.3] {$a$} (reset6);
  
  \node (reset7) at (-5.4,-2.8) {};
  \node (reset8) at (-10,-2.5) {};
  \path[->] (reset7) edge [bend left] node [below,pos=.3] {$a$} (reset8);
  
 
  %\draw (-3,0) ellipse (1.1cm and 3.1cm);
  %\draw (-5.5,0) ellipse (1.1cm and 3.1cm);
  %\draw (-10,0) ellipse (1.1cm and 3.1cm);
  \draw[rounded corners] (-4.1,-3) rectangle (-1.7, 3) {};
  \draw[rounded corners] (-6.3,-3) rectangle (-4.8, 3) {};
  \draw[rounded corners] (-10.5,-3) rectangle (-9, 3) {};
    
  %\draw (3,0) ellipse (1cm and 3cm);
  \draw[rounded corners] (1.7,-3) rectangle (4.1, 3) {};
  
  \node[state] (t) at (7,0) {$t$};
  
  \node at (3,3.4) {{\LARGE $T$}};
  \node at (-3,3.4) {{\LARGE $S$}};
  \node at (-10,3.5) {{\LARGE $S_{r_2}$}};
  \node at (-5.5,3.5) {{\LARGE $S_{1}$}};
  \node at (0.1,4.1) {{\LARGE Original $\mathcal A$ (altered)}};
   % (altered for new alphabet)
  
  \node (s1) at (-.5,2) {\tikz \innerstateloop;};
  \node (s2) at (.5,-1.5) {\tikz \innerstateloop;};
  %\node (s2) at (-1.2,-2) {\tikz \innerstateloop;};
  
  \node (sT1) at (3,1.9) {\tikz \innerstateloop;}; \node (sT1copy) at (3,1.5) {};
  \node (sT2) at (2.8,0.4) {\tikz \innerstateloop;}; \node (sT2copy) at (2.8,0) {};
  \node (sT3) at (3.1,-1.5) {\tikz \innerstateloop;}; \node (sT3copy) at (3.15,-2) {};
  
  \node (sS1) at (-3,1.7) {\tikz \innerstateloop;}; \node (sS1copy) at (-2.95,1.15) {};
  \node (sS2) at (-2.7,0) {\tikz \innerstateloop;}; \node (sS2copy) at (-2.65,-.55) {};
  \node (sS3) at (-3,-1.8) {\tikz \innerstateloop;};\node (sS3copy) at (-3,-2.3) {};
   
  \node[state] (sS11) at (-5.5,1.7) {};
  \node[state] (sS12) at (-5.2,0) {};
  \node[state] (sS13) at (-5.7,-2) {};
  
  \node[state] (sSr1) at (-10,1.7) {};
  \node[state] (sSr2) at (-9.7,0) {};
  \node[state] (sSr3) at (-10.1,-2) {};
  
  \path[->] (t) edge [loop right] node {$\Sigma$} (t);
  
  % nichteinzeichnen, self loop implied
  \path[->] (sT1copy) edge [bend left=35] node [above] {$a$} (t)
            (sT2copy) edge [bend left=10] node [above] {$a$} (t)
            (sT3copy) edge node [above] {$a$} (t);
            
            
  \node (sSr1a) at (-8.5,1.7) {};
  \node (sSr2a) at (-8.2,0) {};
  \node (sSr3a) at (-8.7,-2) {};
  
  \node (sSr1b) at (-7.2,1.7) {};
  \node (sSr2b) at (-6.9,0) {};
  \node (sSr3b) at (-7.4,-2) {};
  
  \path[->] (sSr1) edge node [above,pos=.3] {$b$} (sSr1a);
  \path[->] (sSr2) edge node [above,pos=.27] {$b$} (sSr2a);
  \path[->] (sSr3) edge node [above] {$b$} (sSr3a);
 
  \path[->] (sSr1b) edge node [above,pos=.35] {$b$} (sS11);
  \path[->] (sSr2b) edge node [above] {$b$} (sS12);
  \path[->] (sSr3b) edge node [above] {$b$} (sS13);
  
  \path[->] (sS11) edge [bend right=10] node [above,pos=.38] {$b$} (sS1copy);
  \path[->] (sS12) edge [bend right=10] node [above,pos=.2] {$b$} (sS2copy);
  \path[->] (sS13) edge [bend right=10] node [above,pos=.4] {$b$} (sS3copy);
  
  \draw[dashed] (-8.5,1.7) -- (-7.2,1.7);
  \draw[dashed] (-8.2,0) -- (-6.9,0);
  \draw[dashed] (-8.7,-2) -- (-7.4,-2);
 \end{tikzpicture}}
  \caption{%Schematic illustration of the reduction from the proof of Proposition \ref{prop:stricly_bounded_np_hard}.
   The reduction from the proof sketch sketch of Lemma~\ref{lem:np_hardness}.
   The letter $a$ transfers everything surjectively onto $S_{r_2}$,
   indicated by four large arrows at the top and bottom and labelled 
   by $a$.
   The auxiliary states $Q_1, \ldots, Q_{p_2-1}$, which are meant
   to interpret a sequence $b^{p_2}$ like a single symbol in the original
   automaton, are also only indicated inside of $\mathcal A$, but not fully written out.}
  \label{fig:reduction_np_hard}
\end{figure}



Please see Figure~\ref{fig:reduction_np_hard} for a sketch
of the reduction.
For the constructed automaton $\mathcal A'$, the following could be shown:
$\exists m \ge 0 : \delta(S, c^m) \subseteq T$
if and only if $\mathcal A'$ has a synchronizing word in $ab^{r_2}(b^{p_2})^*a$
if and only if $\mathcal A'$ has a synchronizing word in $ab^*a$
if and only if $\mathcal A'$ has a synchronizing word in $a^*b^*a^*$.
% \begin{align*}
%     \exists m \ge 0 : \delta(S, c^m) \subseteq T 
%                               & \Leftrightarrow \mathcal A'\mbox{ has a synchronizing word in $ab^{r_2}(b^{p_2})^*a$.} \\
%                               & \Leftrightarrow \mathcal A'\mbox{ has a synchronizing word in $ab^*a$.} \\
%                               & \Leftrightarrow \mathcal A'\mbox{ has a synchronizing word in $a^*b^*a^*$}
% \end{align*} 

\begin{toappendix}

Next, we supply the proof of the claim made in the proof sketch of Lemma~\ref{lem:np_hardness}
from the main text.

\medskip

\noindent\underline{Claim:} 
 For the constructed automaton $\mathcal A'$ from
 the proof sketch of Lemma~\ref{lem:np_hardness} in the main text, we have:
\begin{align*}
    \exists m \ge 0 : \delta(S, c^m) \subseteq T 
                               & \Leftrightarrow \mathcal A'\mbox{ has a synchronizing word in $ab^{r_2}(b^{p_2})^*a$.} \\
                               & \Leftrightarrow \mathcal A'\mbox{ has a synchronizing word in $ab^*a$.} \\
                               & \Leftrightarrow \mathcal A'\mbox{ has a synchronizing word in $a^*b^*a^*$}
\end{align*} 
%\begin{quote}
%\begin{proof}[Proof of Claim]
 \emph{Proof of the Claim.}
 First, suppose $\delta(S, c^m) \subseteq T$.
 %Choose any $a^r \in A_1$ with $r > 0$
 %and $a^s \in A_3$ with $s > 0$. 
 By construction of $\mathcal A'$,
 for any $q, q' \in Q$,
 \begin{equation}\label{eqn:transition_Astar}
  \delta(q, c) = q'  \mbox{ in $\mathcal A$}  \Leftrightarrow \delta'(q, b^{p_2}) = q'  \mbox{ in $\mathcal A'$} .
 \end{equation}
 Also, $\delta'(Q'\setminus\{t\},a) = S_{r_2}$
 and $\delta'(S_{r_2}, b^{r_2}) = S$.
 Combining these facts, we find
 \[
  \delta'(Q', ab^{r_2}b^{p_2m}) \subseteq T \cup \{t\}. 
 \]
 A final application of $a$ then maps
 all states in $T$ to the single sink %(and synchronizing) 
 state~$t$.
 
 Clearly, as $ab^{r_2}(b^{p_2})^*a \subseteq a b^* a$
 and $a b^* a \subseteq a^* b^* a^*$, the next two implications are shown.
 Finally, to complete the argument, let $u = a^{p} b^q a^r$ be a synchronizing word, $p,q,r \ge 0$.
 Then, as $t$ is a sink state, $\delta'(Q', u) = \{t\}$.
 The only way to enter $t$ from the outside is to read $a$ at least once, and 
 as $t$ is a sink state, we have $\delta'(Q', a^p b^q a^r) = \{t\}$.
 Also, as for $q \notin T$, we have $\delta'(q, a) \notin T$,
 we must have $\delta'(Q', a^p b^q) \subseteq T \cup \{t\}$,
  or more specifically, $\delta'(Q' \setminus \{t\}, a^p b^q) \subseteq T$.
  We distinguish two cases for~$p$.
 
 \begin{enumerate}
 \item If $p = 0$, then, in particular, $\delta'(S, b^q)  \subseteq T$.
     By construction of $\mathcal A'$, for any $q \in Q$,
 \[
  \delta'(q, b^n) \in Q
 \]
 if and only if $n \equiv 0\pmod{p_2}$.
 So, $q = p_2 m$ for some $m \ge 0$.
 Hence, by Equation~\eqref{eqn:transition_Astar} above from the first case, in $\mathcal A$,
 we find $\delta(S, c^m) \subseteq T$.
 
 \item  If $p > 0$, then $\delta'(Q' \setminus\{t\}, a^p) = S_{r_2}$.
 The only way to leave any state in $S_{r_2}$
 is to read $b$, which transfers $S_{r_2}$ to $S_{r_2-1}$.
 Reasoning similarly, we find that we have to read in $b$
 at least $r_2$ many times, which finally maps $S_{r_2}$
 onto $S_0 = S$. So, $q \ge r_2$. By construction of $\mathcal A'$, for any $q \in Q$,
 \[
  \delta'(q, b^n) \in Q
 \]
 if and only if $n \equiv 0\pmod{p_2}$.
 So, as $\delta'(S, b^{q - r_2}) \subseteq T$, $q - r_2 = p_2 m$ for some $m \ge 0$.
 Hence, by Equation~\eqref{eqn:transition_Astar} above, in $\mathcal A$,
 we find $\delta(S, c^m) \subseteq T$.
 \end{enumerate}
This ends the proof of the claim. \emph{[End, proof of the Claim.]}
%\end{proof}
%\end{quote}
\end{toappendix}

%Finally, we show that we have some $m \ge 0$
%such that $\delta(S, c^m) \subseteq T$
%if and only if $\mathcal A'$ has a synchronizing word in $L$:
 Now, suppose $\delta(s, c^m) \subseteq T$ for some $m \ge 0$.
 By the above, $\mathcal A'$ 
 has a synchronizing word $u$ in $ab^{r_2}(b^{p_2})^*a$.
%  As $A_1$ is non-empty
%  and does not equal $\{\varepsilon\}$, we can choose $a^p \in A_1$ with $p > 0$.
%  Similarly, choose $a^r \in A_3$.
%  Then
%  $
%   a^{p-1} u a^{r-1} v \in A_1 A_2 A_3 \subseteq L,
%  $
%  and %by Lemma~\ref{lem:append_sync}, 
%  this word also synchronizes $\mathcal A'$.
 Then, $a^{r_1 - 1}u a^{r_3-1} \in L(\mathcal B)$ also synchronizes~$\mathcal A'$.
 
 
 Conversely, suppose we have a synchronizing word $w \in L$
 for $\mathcal A'$.
 As $L \subseteq a^* b^* a^*$
 by the above equivalences,
 $\delta(S, c^m) \subseteq T$
 for some $m \ge 0$. \qed
\end{proof}



\section{Constraints from Strongly Self-Synchronizing Codes}
\label{sec:strongly_self_sync}
%
% ergebnisse bounded, beispiele
% dann noch ab(ba)^*ab zeigen, und dann erstmal paper abschließen. vgl ictcs kriterium

Here, we introduce strongly self-synchronizing codes and investigate $L$\textsc{-Constr-Sync}
for bounded constraint languages $L \subseteq w_1^* \cdots w_k^*$
where $\{ w_1, \ldots, w_k \}$ is such a code. %a strongly self-synchronizing code.

Let $C \subseteq \Sigma^+$ be non-empty.
Then, $C$ is called a \emph{self-synchronizing code}~\cite{zbMATH03943051,DBLP:books/daglib/0025093,Hsieh1989SomeAP},
if $C^2 \cap \Sigma^+ C \Sigma^+ = \emptyset$. If, additionally, $C \subseteq \Sigma^n$
for some $n > 0$, then it is called\footnote{In~\cite{Hsieh1989SomeAP}
 this distinction is not made and self-synchronizing codes are also called comma-free codes.}
a \emph{comma-free code}~\cite{golomb_gordon_welch_1958}.
Every self-synchronizing code is an infix code, i.e., no proper factor of a word from $C$ is in $C$~\cite{Hsieh1989SomeAP}.
A \emph{strongly self-synchronizing code}
is a self-synchronizing code $C \subseteq \Sigma^+$ \todo{nicht bloss $\cap C\Sigma^+$ möglich? für beweis ausreichend?}
such that, additionally, $(\pref(C) \setminus C)C \cap \Sigma^*C \Sigma^+ = \emptyset$.
%, i.e, even when appending code words to proper prefixes
%If $\pref(C)C \cap \Sigma^+ C \Sigma^+ = \emptyset$,
%we call $C$ a \emph{strongly self-synchronizing code}.
\begin{comment}
A language $C \subseteq \Sigma^+$ is called a \emph{self-synchronizing code},
if for any $u,v\in C$, $w \in \Sigma^*$ and $x,y \in \Sigma^+$
where $u \ne v$ we have that $uv = xwy$ or $u = xw$ or $u = wx$ implies $w \notin C$.
% if for any $u,v \in C$ and $x,y \in \Sigma^*$,
% the condition $uv = xwy$ with $|x| > 0$ or $|y| > 0$
% implies $w \notin C$, i.e., no code word or concatenation
% of two codewords 
These codes, which are a generalization of comma-free codes, 
are well-studied~\cite{zbMATH03943051,DBLP:books/daglib/0025093,golomb_gordon_welch_1958}. % auch comma-free erwähnen, intuition, wie auf wikipedia?
As a new notion, we introduce \emph{strongly self-synchronizing codes} $C \subseteq \Sigma^+$
as languages which fulfill the condition that, for
any $u \in \pref(C)$ and $v \in C$, 
%with $u \notin \pref(v)$ 
%and $x,y \in \Sigma^+$,
%the conditions $uv = xwy$ or $v = xw$ or $v = wx$ imply $w \notin C$. % prefix und suffix erwähnen?
% zu schwach, kann ja in der mitte v als faktor auftreten
%we have $\{ u \} \subseteq \fact(uv) \cap C \subseteq \{ v, u \}$.
if we write $uv = x_1 \cdots x_n$
with $x_i \in \Sigma$ for $i \in \{1,\ldots, n\}$,
then, for any $j \in \{1,\ldots,n\}$ and $k \ge 0$ where $j + k \le n$,
we have that $x_j x_{j + 1} \cdots x_{j+k-1} \in C$
implies\footnote{If $k = 0$, we mean $x_j x_{j+1} \cdots x_{j + k - 1}$ to denote the empty
string. So, $k = |x_j x_{j+1} \cdots x_{j + k - 1}|$. However, note that by definition $\varepsilon\notin C$.} $j = |u| + 1$ and $k = |u| + |v|$
or $j = 1$ and $k = |u|$.
Intuitively, in $uv$ only the last $|v|$ symbols form a factor in $C$
and possibly the first $|u|$ symbols\footnote{Note that,
for example, stipulating that $\{ u \} \subseteq \factor(uv) \cap C \subseteq \{ v, u \}$
for $u \in \pref(C)$ and $v \in C$ is actually a weaker condition.
It allows, for instance, for the factors $u,v$ to occur in the middle of $uv$,
which is excluded.}.
%However, I have seen examples in the literature were such fine points were
%not paid attention too, probably because it is most often intuitively clear
%what is meant. For example, a precursor to self-synchronizing codes
%were comma-free codes, and the original definition~\cite{golomb 2x}
%does not mention that the concatenated words have to be distinct, but
%give examples as $\{ aa, bb \}$ for comma-free codes.
Recall that by definition $\varepsilon \notin C$,
and by choosing $u = \varepsilon$, we find that these codes are infix codes, i.e.,
no proper factor is a code word in $C$. Also, every strongly self-synchronizing code
is a self-synchronizing codes. In Remark~\ref{rem:code_construction},
we will give a method to construct strongly self-synchronizing codes.
\end{comment}

To give some intuition for the strongly self-synchronizing
codes, we also present an alternative characterization, a few examples and a way to construct such codes.

\begin{propositionrep}
A non-empty $C \subseteq \Sigma^+$
is a strongly self-synchronizing code
if and only if, for
all $u \in \pref(C)$ and $v \in C$, 
if we write $uv = x_1 \cdots x_n$
with $x_i \in \Sigma$ for $i \in \{1,\ldots, n\}$,
then, for all $j \in \{1,\ldots,n\}$ and $k \ge 1$ where $j + k - 1 \le n$,
we have that $x_j x_{j + 1} \cdots x_{j+k-1} \in C$
implies $j = |u| + 1$ and $k = |v|$
or $j = 1$ and $k = |u|$.
Intuitively, in $uv$ only the last $|v|$ symbols form a factor in $C$
and possibly the first $|u|$ symbols.
\end{propositionrep}
\begin{proof}
 Let $C \subseteq \Sigma^+$ be a strongly self-synchronizing code.
 Suppose $u \in \pref(C)$ and $v \in C$.
 If $u \notin C$, then we must have $uv \notin \Sigma^*C\Sigma^+$,
 so that, if $uv = x_1 \cdots x_n$ as in the statement,
 we have $x_j \cdots x_{j+k-1}$ if and only if $j = |u| + 1$
 and $k = |u| + |v|$.
 If $u \in C$, then, as $uv \notin \Sigma^+ C \Sigma^+$,
 we find that we have only the possibilities
 $j = 1$ and $k = |u|$ or $j = |u| + 1$ and $k = |u| + |v|$.
 
 Conversely, suppose $C \subseteq \Sigma^+$ is non-empty
 and fulfills the condition mentioned in the statement.
 If $u, v \in C$ and $uv \in \Sigma^+ C \Sigma^+$,
 then we can write $uv = x_1 \cdots x_n$ with $x_i \in \Sigma$ for $i \in \{1,\ldots,n\}$
 and find $2 \le i \le j \le n - 1$
 such that $x_i \cdots x_j \in C$, which contradicts
 the condition in the statement.
 Similarly, if $u \in \pref(C) \setminus C$
 and $v \in C$ with $uv \in \Sigma^*C\Sigma^+$,
 then we can write $uv = x_1 \cdots x_n$ with $x_i \in \Sigma$ for $i \in \{1,\ldots,n\}$
 and find $1 \le i \le j \le n - 1$
 such that $x_i \cdots x_j \in C$, which would contradict
 the condition too. So, we must
 have $C^2 \cap \Sigma^+ C \Sigma^+ = \emptyset$
 and $(\pref(C) \setminus C) C \cap \Sigma^* C \Sigma^+ = \emptyset$.\qed 
\end{proof}

%\begin{remark}
%  To give some intuition on strongly self-synchronizing
%  codes with respect to CSP:
%  These codes ensure that auxiliary states that have to be introduced
%  when passing from words to letters in the reductions
%  were synchronized too, i.e., in some sense the defining conditions ensure that input words
%  do not stay on certain paths between these auxiliary states.
%\end{remark}

 When passing from letters to words by applying a homomorphism, in the reductions,
 we have to introduce additional states. The definition of the strongly synchronizing
 codes was motivated by the demand that these states also have to be synchronized, which turns out to be difficult in general.

\begin{example}\label{ex:strongly_self_sync}
 The code $\{aacc,bbc,bac\}$
 is strongly self-synchronizing.
 The code $\{ aab, bccc, abc \}$ is self-synchronizing, but
 not strongly self-synchronizing as, for example, $(a)(abc)$ 
 contains $aab$ or $(aa)(bccc)$ contains $abc$. 
\end{example}

\begin{toappendix}
 To give a proof of the claim made in Example~\ref{ex:strongly_self_sync}.
\begin{proposition}
 The code $\{aacc,bbc,bac\}$ is strongly self-synchronizing.
\end{proposition}
\begin{proof}
 By checking all cases to combine prefixes with code words:
 \[ 
 \begin{array}{llll}
     \mbox{Non-empty prefixes of $aacc$:} & (a)aacc & (a)bbc & (a)bac \\ 
      & (aa)aacc  & (aa)bbc  & (aa)bac \\ 
      & (aac)aacc & (aac)bbc & (aac)bac \\ 
      & (aacc)aacc & (aacc)bbc & (aacc)bac \\ 
      \\
     \mbox{Non-empty prefixes of $bbc$:} & (b)aacc   & (b)bbc   & (b)bac \\    
     & (bb)aacc  & (bb)bbc  & (bb)bac \\ 
     & (bbc)aacc & (bbc)bbc & (bbc)bac \\ 
     \\
     \mbox{Non-empty prefixes of $bac$:} & (b)aacc   & (b)bbc   & (b)bac \\    
     & (ba)aacc  & (ba)bbc  & (ba)bac \\ 
     & (bac)aacc & (bac)bbc & (bac)bac. 
 \end{array}
 \]
 So, we see that the defining conditions are satisfied. Note that $C \cap \Sigma^*C\Sigma^+ = \emptyset$
 is always satisfied for self-synchronizing codes, as they are infix codes. \qed
\end{proof}
\end{toappendix}



% achso, ne quatsch, c^k+1varphi(L) ist ja nicht das bild varphi(c^{k+1}varphi)
\begin{remark}[Construction] %[Construction of Strongly Self-Synchronizing Codes]
\label{rem:code_construction}
% und damit auch weitere besipiele für self-sync strongly codes
% bzw gibt auch methode zur konstruktion solcher codes
% % darauf in einführung verweisen.
%  Theorem~\ref{thm:constr_sync_hom_strongly_self_sync}
%  is a true generalization of Theorem~\ref{thm:forward_hom}
%  because the code constructed in the assumption
%  is a strongly self-synchronizing code, as we will show next.
%  This also yields a method to construct infinitely many such codes.
% Take any präfix code $X \subseteq \Sigma^*$ and 
% symbol $c \in \Sigma$.
 %such that no word in $X$ begins with it, i.e.,
 %we have $c\Sigma^* \cap X = \emptyset$.
 Take any non-empty finite language $X \subseteq \Sigma^n$, $n > 0$,
 and a symbol $c \in \Sigma$ such that $\{c\}\Sigma^* \cap X = \emptyset$.
 Let $k=\max\{\,\ell\geq0\mid \exists u,v\in\Sigma^*:uc^\ell v\in X\,\}$.
 Then, $Y = c^{k+1}X$
 is a strongly self-synchronizing code.
 %For if $u \in \pref(Y)$ and $v \in Y$,
 %then  in $uv$ no proper factor, except the last $|v|$
 %symbols and possibly the first $u$ symbols, start with $c^{k+1}$. Hence,
 %no such factor is a code word from~$Y$.
\end{remark}

\begin{example}\label{ex:strongly_self_sync_construction}
Let $\Sigma = \{a,b,c\}$
and $C = \{ ab,ba, aa\}$.
Then, $\{ cab, cba, caa \}$ or $\{ bbab, bbaa \}$
are strongly self-synchronizing codes by Remark~\ref{rem:code_construction}.
%constructed according
%to the scheme described in Remark~\ref{rem:code_construction}.
\end{example}




Our next result, which holds in general, states conditions on a homomorphism
such that we not only have a reduction from the problem
for the homomorphic image to our original problem, as stated in Proposition~\ref{prop:hom_lower_bound_complexity},
but also a reduction in the other direction.

\begin{theoremrep}
\label{thm:constr_sync_hom_strongly_self_sync}
 Let $\varphi : \Sigma^* \to \Gamma^*$
 be a homomorphism such that $\varphi(\Sigma)$
 is a strongly self-synchronizing code and $|\varphi(\Sigma)| = |\Sigma|$.
 Then, for each regular $L \subseteq \Sigma^*$ we have
 $
  L\textsc{-Constr-Sync} \equiv_m^{\log} \varphi(L)\textsc{-Constr-Sync}.
 $
\end{theoremrep}
\begin{proof}
 By Proposition~\ref{prop:hom_lower_bound_complexity}, we have 
 $\varphi(L)\textsc{-Constr-Sync} \le_m^{\log} L\textsc{-Constr-Sync}$.
 
 
 Next, we give a reduction 
 from $L\textsc{-Constr-Sync}$ % problem, mit sink state. ne geht auch ohne, wenn in ausgang, dann eifnach noch ein beliebiges zeichen dranhängen vorne, andersrum werden die aus Q gesynct.
 to $\varphi(L)\textsc{-Constr-Sync}$.
 % aber kann man immer ein zeichen dranhängen??? d.h. L muss
 % erlauben, dass zu jedem wort w \in L es u \in \Sigma^+ gibt, so dass uw \in L.
 % bei bounded nicht unbedingt erfüllt...
 %
 % problem L-constr-Sync-sink definieren.
 %
 % für strongly connected äquivalent, wenn man neues zeichen einführt
 %
 %  von einem zustand mit neuen zeichen zu einem sink zustand, andere loopen
 % aber der zustand muss mit consraint-wort erreichbar sein.
 
 %
 % doch geht, wenn in q_x und y und q_{xy} nicht definiert, dann zu q_y, und so bewegt 
 % man sich "in gleihe Richtung" wie das erste ZEichen
 % geht nur vür |u|_i <= 2, also wenn nur ein hilfszustand, sonst größtes
 % suffix von xy so dass noch definiert?
 %
 % am ende soll gelten delta(q_x, \varphi(u)) = delta(q,\varphi(u))
 %  So, delta(q_x, y) = q_z mit z maximales suffix von xy so dass q_x definiert.
 %
 % x\varphi(u) nie präfix für x != eps, also ist varphi(U) das längste derartige suffix.
 %
 Write $\Sigma = \{a_1, \ldots, a_n\}$ with $n = |\Sigma|$
 and $u_i = \varphi(a_i)$ for $i \in \{1,\ldots,n\}$.
 Let $\mathcal A = (\Sigma, Q, \delta)$
 be an input semi-automaton for $L\textsc{-Constr-Sync}$.
 
  We construct a semi-automaton $\mathcal A' = (\Gamma, Q', \delta')$.
  The state set will be
  \[
   Q' = \{ q_x \mid q \in Q, x \in \pref(\{ u_1,\ldots,u_n \}) \setminus \{ u_1,\ldots,u_n \} \}. 
  \]
  By identifying $q_{\varepsilon}$ with the state $q \in Q$,
  we can assume $Q \subseteq Q'$.
  Then, for $q_x \in Q'$ and $y \in \Sigma$, 
  let $z$ be the longest suffix of $xy$
  such that $z \in \pref(\{ u_1,\ldots,u_n\})$ and set\footnote{Note
  the implicit correspondence between the states $q$
  and $q_z$ for $z \in \pref(\varphi(\Sigma)) \setminus \varphi(\Sigma)$.}
  \begin{equation}\label{eqn:def_delta_bar}
   \delta'(q_x, y) = \left\{
   \begin{array}{ll} % todo noch besser formatieren.
    q_{z}          & \mbox{if } z \in \pref(\{ u_1,\ldots,u_n\}) \setminus \{ u_1, \ldots, u_n \}; \\ 
    \delta(q, a_i)  & \mbox{if } \exists i \in \{ 1, \ldots, n \} : z = u_i.
   \end{array}
   \right.
  \end{equation}
  As $|\varphi(\Sigma)| = |\Sigma|$ and $\{ u_1, \ldots, u_n \}$ is a prefix code\footnote{A code
  is a prefix code, if no code word is the proper prefix of another code word.}, the transition function % todo das genauer? opben schreiben impliziert prefix-free, siehe mfcs19
  is well-defined.
%   Then, more generally, for $q_x, q_y \in Q'$
%   and $u \in \Sigma^*$, we
%   have %todo induktiv zeigen?
%   \begin{equation}
%       \delta'(q_x, u) = q_y % q delta
%       \Longleftrightarrow
%       \mbox{$y$ is the longest suffix of $xu$ in $f$
%   \end{equation}
  By construction, for any $u \in \Sigma^*$ and $q \in Q$,
  we have  
  \begin{equation}\label{eqn:comma_free_reduction}
      \delta(q, u) = \delta'(q, \varphi(u)).
  \end{equation}
  
  
  
  
\begin{comment}  
  But we need a more precise statement.
  
  \begin{myclaiminproof}
   Let $q_x, q'_z \in Q'$ and $y \in \Sigma^*$.
   Then,
   \[
    \delta'(q_x, y) = q'_z,
   \]
   where $xy = x_0 u_{i_1} x_1 u_{i_2}\cdots u_{i_m} x_m$
   for $i_1, \ldots, i_m \in \{1,\ldots,n\}$
   and $m$ is maximal with\footnote{Intuitively, these conditions
   express that $xy$ is ``parsed'' greedily for the factors
   from $\varphi(\Sigma)$.}:
   \begin{enumerate}
   \item $u_{i_j} \in \varphi(\Sigma)$ for $j \in \{1,\ldots,m\}$;
   \item for any $j \in \{0,\ldots, m-1\}$
    the word $x_j \in \Sigma^*$ is the shortest word 
    such that there exists $i_{j+1} \in \{1,\ldots,n\}$
    so that $u_{i_{j+1}} \in \varphi(\Sigma)$ and $x_0 u_{i_1} x_1 \cdots u_{i_{j}} x_j u_{i_{j+1}}$
    is a prefix of $xy$
    and $x_m \in \Sigma^*$ is the shortest word which contains
    no factor in $\varphi(\Sigma)$ and could be appended to give $xy$;
   \item $z$ is the longest suffix of $x_m$ in $\pref(\varphi(\Sigma))\setminus \varphi(\Sigma)$;
   \item $q' = \delta'(q, u_{i_1} \cdots u_{i_m}) = \delta(q, a_{i_1}\cdots a_{i_m})$,
    where $a_{i_j} \in \Sigma$ is the unique symbol with $u_{i_j} = \varphi(a_{i_j})$
    for $j \in \{1,\ldots,m\}$.
   \end{enumerate}
  \end{myclaiminproof}
  \begin{myclaimproof}
   We do induction on the length of $y$.
   If $y = \varepsilon$,
   as $q_x \in Q'$, and so $x \in \pref(\varphi(\Sigma)) \setminus \varphi(\Sigma)$,
   as $\varphi(\Sigma)$ is strongly self-synchronizing code,
   %\footnote{Actually, by a closer investigation
   %of the operational mode of the automaton, we see that from any state $q \in Q$
   %we never arrive at a state $q_x$ where $x$ contains a word from $\varphi(\Sigma)$
   %as a factor, even if $\varphi(\Sigma)$ is not strongly self-synchronizing, but only a prefix code.}, 
   the word cannot contain a word from $\varphi(\Sigma)$
   as a proper factor and so $m = 0$ in the above form and we find $z = x$
   and $q' = q$.
   So, now we evaluate $\delta'(q_x, ya)$ for $y \in \Sigma^*$ and $a \in \Sigma$.
   By induction hypothesis, we can write $\delta'(q_x, y) = q_z'$
   with a decomposition of $xy$ as written in the statement of the claim
   for some $m \ge 0$.
   Let $v$ be the longest suffix of $za$ which gives a word in $\pref(\varphi(\Sigma))$
   and write $za = wv$ with $w \in \Sigma^*$.
   If $v \notin \varphi(\Sigma)$,
   then
   \[ %todo x_ma = x'_ma setzen?
    xya = x_0 u_{i_1} x_1 u_{i_2}\cdots u_{i_m} (x_ma)
   \]
   is a new decomposition fulfilling the conditions of the claim
   but with $v$ in place of $z$ and $\delta'(q'_z, a) = q_{v}$.
   If $v\in \varphi(\Sigma)$, then let $i_{m+1} \in \{1,\ldots, n\}$
   be such that $v = u_{i_{m+1}}$ and we find
   \begin{equation}\label{eqn:claim}
    xya = x_0 u_{i_1} x_1 u_{i_2}\cdots u_{i_m} x_m' u_{i_{m+1}}
   \end{equation}
   for some $x_m' \in \Sigma^*$ with $x_m a = x_m' u_{i_{m+1}}$.
   By Equation~\eqref{eqn:comma_free_reduction}, $\delta'(q'_z, a) = \delta'(q', u_{i_{m+1}}) = \delta(q', a_{i_{m+1}})$
   and the claimed conditions are satisfied for the decomposition
   written in Equation~\eqref{eqn:claim}.
  \end{myclaimproof}
\end{comment}

 
\begin{comment}
 \begin{myclaiminproof}
  Let $i \in \{1,\ldots,n\}$ and $q_x \in Q'$.
  Then,
  \[
   \delta'(q_x, u_i) = \delta'(q, u_i) = \delta(q, a_i).
  \]
 \end{myclaiminproof}
 \begin{myclaimproof}
  If $x = \varepsilon$, this is a special case of Equation~\eqref{eqn:comma_free_reduction}.
  So, assume $x \ne \varepsilon$.
  As $\varphi(\Sigma)$
  is a variable-length comma-free code, so in particular a suffix code,
  the longest prefix of $u_i$ which, concatenated with $x$ in front,
  gives a word in $\pref(\{u_1, \ldots, u_n\})$
  is not $u_i$ itself.
  So, we can write $u_i = zay$ with $z,y \in \Sigma^*$ and $a \in \Sigma$ such that
  $z$ is the longest prefix for which
  $xz \in \pref(\{u_1, \ldots, u_n\})$
  holds true. Then,
  $xza \notin \pref(\{u_1, \ldots, u_n\})$
  and let $v$ be the longest suffix of $xza$
  in $\pref(\{u_1, \ldots, u_n\})$.
  By choice $za \in \pref(\{u_1, \ldots, u_n\})$.
  So, $|za| \le |v|$ and $za$ is a suffix of $v$.
  As $xza \notin\pref(\{u_1, \ldots, u_n\})$,
  we have $|v| < |xza|$.
  Write $v = x'za$ with $x' \in \Sigma^*$
  being a proper suffix of $x$.
  %As $q_x \in Q'$, which gives $x \in \pref(\{u_1, \ldots, u_n\}) \setminus \{u_1, \ldots,u_n\}$,
  Write $x = x'' x'$ with $x'' \in \Sigma^+$.
  Then, % todo vereinfachen? bilder zeichnen
  \[
   \delta'(q_{x''}, x') = q_x.
  \]
  %We have $xz \notin \{u_1, \ldots, u_n\}$,
  %as it is a proper factor of $xu_i$
  
  \begin{enumerate}
  \item $xz \notin \{u_1, \ldots, u_n\}$, $v \notin \{u_1, \ldots, u_n\}$
  
   Then, $\delta'(q_x, z) = q_{xz}$
   and $\delta'(q_{xz}, a) = q_v$.
      
  \end{enumerate}
  
  
  
  
  
  %
  % falsch, suffix, nicht präfix
  also $x' \in \pref(\{u_1, \ldots, u_n\}) \setminus \{u_1, \ldots,u_n\}$
  and we find a state $q_{x'} \in Q$.
  Then, as $vy = x'zay = x'u_i$, we can reason inductively, as $|x'| < |x|$,
  that
  \[
   \delta'(q_{x'}, u_i) = \delta'(q, u_i) = \delta(q, a_i).
  \]
  But if we write $x = x' x''$ with $x'' \in \Sigma^+$, by %todo hier equaino nummer referenzieren
  the definition of the transition function
  \[
   \delta'(q_{x'}, x'') = q_{x'x''} = q_x
  \]
  and so
  \[
   \delta(q_x, u_i) = \delta(\delta'(q_{x'}, x''), u_i)
  \]
  
  
  % wenn v \in \{u_1,...,u_n\} dann inducitively 
  
  As $\varphi(\Sigma)$
  is a comma-free code, we cannot have $v \in \{u_1, \ldots, u_n\}$,
  as $v$ is a proper factor of $xzay = xu_i$
  
  
  
  % wenn v nicht in {u1,...,un}
  By the definition of the transition function,
  \[
   \delta(q_x, xza) = q_v
  \]
 
  Then, 
  we must have
  $vy = u_i$.
  
  
 \end{myclaimproof}
\end{comment}

  Let $x \in \pref(\varphi(\Sigma)) \setminus \varphi(\Sigma)$
  and $u_i \in \varphi(\Sigma)$, $i \in \{1,\ldots,n\}$.
  Then, as $\varphi(\Sigma)$
  is a strongly self-synchronizing code, the word $xu_i$ does not contain
  a word from $\varphi(\Sigma)$ , except the suffix $u_i$,
  as a factor. Next, we will argue that, for the unique $a_i \in \Sigma$
  with $\varphi(a_i) = u_i$, the following equations 
  holds true:
  \begin{equation}\label{eqn:strongly_self_sync_transition}
   \delta'(q_x, u_i) = \delta'(q, u_i) = \delta(q, a_i).
  \end{equation}
  For if $v \in \pref(\{ u_i \}) \cap \Sigma$, then the longest suffix of $xv$
  in $\pref(\varphi(\Sigma))$ must be~$v$.
  First, it is a suffix
  from this set.
  Second, if there exists longer one, say $w$,
  then write $ww' \in \varphi(\Sigma)$ for some $w' \in \Sigma^*$.
  In that case, with $xv = x'w$ ($|x'| < |x|$), we have $x'ww' \in xu_i\Sigma^*$
  or $xu_i \in x'ww'\Sigma^+$.
  In the first case, $ww'$ contains the proper factor $u_i \in \varphi(\Sigma)$,
  which is not possible as $\varphi(\Sigma)$ is, in particular, an infix code.
  In the second case, $\{ x'u_i \} \cap \Sigma^* \varphi(\Sigma) \Sigma^+ \ne \emptyset$,
  which is excluded by the property of $\varphi(\Sigma)$ being strongly self-synchronizing.
  So, by the defining equation of $\delta'$, Equation~\eqref{eqn:def_delta_bar},
  if $v \notin \varphi(\Sigma)$, we have
  \[
   \delta'(q_x, v) = q_v,
  \]
  and if $v \in \varphi(\Sigma)$, then $v = u_i$, as $\varphi(\Sigma)$ is a prefix code,
  and
  \[
   \delta'(q_x, v) = \delta'(q_x, u_i) = \delta'(q, u_i) = \delta(q, a_i)
  \]
  with the unique $a_i \in \Sigma$ as above.
  So, in the latter case Equation~\eqref{eqn:strongly_self_sync_transition}
  was established. In the former case,
  if $u_i = vv'v''$, then $\delta'(q_v, v') = q_{vv'}$
  which is easily seen as we always read in a word giving a prefix from $\varphi(\Sigma)$, hence
  this word itself is the longest suffix from $\varphi(\Sigma)$.
  So, after reading the entire word $u_i$, by Equation~\eqref{eqn:def_delta_bar},
  Equation~\eqref{eqn:strongly_self_sync_transition}
  is implied.
  
  
  Lastly, we show that this gives a valid reduction.
  
  \begin{myclaiminproof}
   The automaton $\mathcal A = (\Sigma, Q, \delta)$
   has a synchronizing word in $L$
   if and only if $\mathcal A' = (\Gamma, Q', \delta')$
   has a synchronizing word in $\varphi(L)$.
  \end{myclaiminproof}
  \begin{myclaimproof}
   %\begin{enumerate}
   %\item 
   First, suppose there exists $u \in L$ such that $|\delta(Q, u)| = 1$.
    %By appending words if necessary, we can assume $|u| > 0$. %immernoch sync, zitieren AAHA, ncihtmehr in L unbedingt!!
    If $|Q| = 1$ every word is synchronizing and the statement is obviously true.
    So, we can assume $|Q| > 1$, which implies $|u| > 0$.
    Write $u = av$ with $a \in \Sigma$.
    % Let $q_x \in Q'$.
    % If $x = \varepsilon$, then, by Equation~\eqref{todo, todo in gleichung falsches alphabet!},
    % \[
    %  \delta'(q_x, \varphi(u)) = \delta(q, u).
    % \]
    % Now, suppose $x \ne \varepsilon$ and write $u = av$ with $a \in \Sigma$.
    By Equation~\eqref{eqn:strongly_self_sync_transition}, then, for any $x \in \pref(\varphi(\Sigma))\setminus\varphi(\Sigma)$,
    \[
     \delta'(q_x, \varphi(a)) = \delta(q, a).
    \]
    Hence, $\delta'(Q', \varphi(a)) = \delta(Q, a)$.
    As $\delta'(Q', \varphi(a)) \subseteq Q$, by Equation~\eqref{eqn:strongly_self_sync_transition},
    or its formulation for the special case of states in $Q$, Equation~\eqref{eqn:comma_free_reduction},
    we find
    \[
     \delta'(\delta(Q, a), \varphi(v))) = \delta(\delta(Q, a), v) = \delta(Q, u).
    \]
    The last set is, by assumption, a singleton set. Hence, the word $\varphi(u)$
    synchronizes~$\mathcal A'$. %sprechweise singelton set einführen? todo
   
    \medskip 
    
    Now, suppose there exists $u \in \varphi(L)$ such that $|\delta'(Q', u)| = 1$.
     Let $v \in \Sigma^*$ be such that $\varphi(v) = u$.
     By Equation~\eqref{eqn:strongly_self_sync_transition} (or Equation~\eqref{eqn:comma_free_reduction}),
     we have
     \[
      \delta(Q, v) = \delta'(Q, \varphi(v)).
     \]
     By assumption, the set on the right side is a singleton set. 
     Hence, $v$ synchronizes $\mathcal A$.
   %\end{enumerate}
  \end{myclaimproof}
  So, we find $L\textsc{-Constr-Sync} \le_m^{\log} \varphi(L)\textsc{-Constr-Sync}$
  and the proof is done.
\end{proof}

\begin{comment}
% Doch komplizierter, erstmal rauslassen.
% 
%We need the following fact, which is implied by the constructions
%in the previous proof. 
In general, regular languages are closed under homomorphic mappings,
but an exponential blow-ups might occur~\cite{projections paper}.
However, such a blow-up does not occur for mappings whose images
are strongly self-synchronizing codes.

\begin{proposition}
 Let $\varphi : \Sigma^* \to \Gamma^*$
 be a homomorphism such that $\varphi(\Sigma)$
 is a strongly self-synchronizing code and $\mathcal A = (\Sigma, Q, \delta, q_0, F)$
 be a PDFA.
 Then, we can construct in polynomial time
 a PDFA $\mathcal A' = (\Gamma, Q', \delta', q_0', F')$
 such that $\varphi(L) = L(\mathcal A')$.
\end{proposition}
\begin{proof}
%  Let $\mathcal A' = (\Gamma, Q', \delta', q_0', F')$
%  be the PDFA where the state set is given by Equation~\eqref{todo oben}
%  and the transition function given by Equation~\eqref{todo oben},
%  derived from the state set and transition function of $\mathcal A$.
%  As written in the proof of Theorem~\ref{thm:constr_sync_hom_strongly_self_sync},
%  we can assume $Q \subseteq Q'$.
%  Then, set $q_0' = q_0$ and $F' = F$.
%  By Equation~\eqref{eqn:strongly_self_sync_transition},
%  we have $\varphi(L) = L(\mathcal A')$.
%  % ne, nur varphi(L) \substeq L(\mathcal A'), wenn z.B. u \in L, dann uU x\varphi(u) 
 % mit der zurücklauf-Regel!!!! siehe den claim danach

 Todo, insbesondere präfixcode, teile dazwischen undefiniert lassen
 aber kann von teilen dazwische zu final laufen?
 dann aber suffix oder so, zeigen, dass hier nicht sein kann.
 
 
 Oder, ohne automaten? aber darstellunge mit j,p's muss auch berechnet werden...
 % unitär, d.h. single-state final bounded languages?
 
 
 Mit $C^*$ schneiden und nutzen, dass suffix code?
\end{proof}
\end{comment}

Finally, we apply Theorem~\ref{thm:constr_sync_hom_strongly_self_sync} to bounded languages
such that $\{ w_1, \ldots, w_k\}$ forms a strongly self-synchronizing code.

\begin{theoremrep}
 Let $L \subseteq w_1^* \cdots w_k^*$
 be regular such that $\{ w_1, \ldots, w_k \}$
 is a strongly self-synchronizing code.
 Then, $L\textsc{-Constr-Sync}$
 is either $\NP$-complete or in $\PTIME$. %solvable in polynomial time.
 %Moreover, the complexity itself, given a constraint PDFA as input,
 %could be decided in polynomial time.
\end{theoremrep}
\begin{proof}
 Let $\Gamma = \{ a_1, \ldots, a_n \}$
 be a new alphabet and let $\varphi : \Gamma^* \to \Sigma^*$
 be the homomorphism given by
 $\varphi(a_i) = w_i$ for $i \in \{1,\ldots, n\}$.
% As $\{ u_1, \ldots, u_n \}$ is a code, the homomorphism
% is injective % todo berstel/perrin zitieren?
% and 
 Let $U = \varphi^{-1}(L)$. 
 As every word in $L$
 is a concatentation of words from $\{ w_1, \ldots, w_n \}$,
 we have $L \subseteq \varphi(\Gamma^*)$.
 So, we find $\varphi(U) = L$.
%  By Theorem~\cite{todo}, we can write
%  \[
%   L = \bigcup_{r=1} 
%  \]
%  for numbers $j_i^{(r)}, p_i^{(r)}$
%  By Theorem~\cite{todo}, we can write $L$
%  as a finite union of languages of the form
%  \[
%   u_1^{j_1} (u_1^{p_1})^* \cdots u_n^{j_n} (u_n^{p_n})^*
%  \]
%  for numbers $j_i, p_i \ge 0$, $i \in \{1,\ldots,n\}$.
%  Now, observe that
%  \[
%   \varphi(  a_1^{j_1} (a_1^{p_1})^* \cdots a_n^{j_n} (a_n^{p_n})^* ) = 
%   u_1^{j_1} (u_1^{p_1})^* \cdots u_n^{j_n} (u_n^{p_n})^*.
%  \]
%  Also, as function application on sets preserves the union,
%  $\varphi(L)$
%  is a union of languages 
%  of the form $a_1^{j_1} (a_1^{p_1})^* \cdots a_n^{j_n} (a_n^{p_n})^*$
%  such that the numbers $j_i, p_i$, $i \in \{1,\ldots,n\}$,
%  are the same as those of the corresponding part in $L$.
%  So, we an apply Proposition~\ref{strictly bounded np-hard}
%  and Proposition~\ref{striclty bounded poly}
%  are preserved, i.e., we can apply those proposition

 By Theorem~\ref{thm:constr_sync_hom_strongly_self_sync},
 the languages $U$ and $L$
 have the same computational complexity.
 Also, as is easy to check, we have $U \subseteq a_1^* \cdots a_n^*$
 and $U$ is regular.
 So, by Theorem~\ref{thm:dichotomy}
 the constrained synchronization problem for $L$
 is either $\NP$-complete or in $\PTIME$.
\end{proof}

\begin{example}
 (1) $((aacc)(bbc)^*(bac))$\textsc{-Constr-Sync} is \NP-complete. \\
 (2) $((bbc)(aacc)(bac)^* \cup (bbc)^*)$\textsc{-Constr-Sync} is in \PTIME.
\end{example}


% \begin{theorem} \label{thm:forward_hom}
% 	Let $L\subseteq\Sigma^*$.
% 	%	Let $\mathcal B = (\Sigma, Q, \mu, q_0, F)$ and $\mathcal B' = (\Gamma, Q', \mu', q_0', F')$
% 	%	be two 	constraint automata.
% 	Let $\varphi : \Sigma \to \Gamma^*$ be a homomorphism %injective\todo{HF: Ich habe Inj. nicht benötigt!}
% 	such that $\varphi(\Sigma)$ is a prefix code.
% 	%and $\varphi(L(\mathcal B)) = L(\mathcal B')$.
% 	%	Let $c\notin\Gamma$.
% 	Let $c \in \Gamma$ with $\{c\}\Gamma^*\cap \varphi(\Sigma)=\emptyset$.
% 	Let $k:=\max\{\,\ell\geq0\mid \exists u,v\in\Gamma^*:uc^\ell v\in\varphi(\Sigma)\,\}$.
% 	% and let $k$ be the maximal number of consecutive appearances of $c$ in any code-word of $\varphi(\Sigma)$.
% 	Then
% 	$
% 	L%(\mathcal B)
% 	%	\textsc{-Constr-Sync} \leq^{\log}_m \{c\}\varphi(L)%(\mathcal B')
% 	\textsc{-Constr-Sync} \leq^{\log}_m \{c^{k+1}\}\varphi(L)%(\mathcal B')
% 	\textsc{-Constr-Sync}\,.
% 	$	
% 	%	with $L(\mathcal B'') = cL(\mathcal B')$.
% \end{theorem}



%
% Künstliches problem
%
%  Sigma mit Input-Automaten nur max {a,b} != Identität
%  mit ergebnissen 3-state case pspace-falls, np-fall alles realisierbar
%  
% aber ist praktische gleich zu alfphabet auf {a,b} einschränken...
%
% L \subseteq u^* v^* w^*
%
% ersetze durch (ccu)^* (ccv)^* (ccw)^*
%
% u_1^* ... u_n^* 
% kann man durch löschen eines buchstaben strongly self-sync code herstellen?
% falls ja, gelöschter buchstaben eingabeautomat identität, dann gleiche komplexität
%
% kann ich u^*v^*w^* entschieden, dann auch (ccu)^* (ccv)^* (ccw)^*, wobei
% c identität auf eingabeautomaten.
%
% andersrum definieren ccu, ccv, ccw als eingabe;
% so umdefinieren, dass in neuem autoamten ccu = u usw als abbildugnen
% wenn u,v,w einzelner buchstaben, dann geht es einfach. aber allgemein
% z.B. ccab, ccba, ccbb
% 
% kann man a,b so umdefinieren, dass ab = ccab usw?
% (btw entscheidungsproblem)
% 
% gegeben eine abbildung f und ein wort w über a,b
% Frage: Kann man a,b so als abbilungen belegen, dass f = w wenn w darüber ausgewertet?
% -> in dem fall klar, a^|w| = f, b = identität geht
% ein wenig wie gleichungen lösen

\section{Conclusion and Discussion}

We have looked at the constrained synchronization problem (Problem~\ref{def:problem_L-constr_Sync}) -- CSP for short -- for letter-bounded regular constraint languages and bounded languages induced by strongly self-synchronizing codes, thereby
continuing the investigation started in~\cite{DBLP:conf/mfcs/FernauGHHVW19}.
The complexity landscape in these cases is completely understood.
Only the complexity classes $\PTIME$ and $\NP$-complete arise.
%, and
%we have given conditions precisely when it is in $\PTIME$ and when it is $\NP$-complete.
In~\cite{DBLP:conf/ictcs/Hoffmann20} the question was raised if we can find sparse constraint languages
that give constrained problems complete for some candidate $\NP$-intermediate complexity class. At least for the
language classes investigated here
this is not the case. 
%Lastly, we have given a polynomial time procedure to decide the computational complexity of $L(\mathcal B)\textsc{-Constr-Sync}$, for a given automaton $\mathcal B$ accepting a strictly bounded regular language. 
For general sparse regular languages, it is still open if a corresponding
dichotomy theorem holds, or candidate $\NP$-intermediate problems arise. By the results obtained so far and the methods
of proofs, we conjecture that in fact a dichotomy
result holds true.


%
% Ne, kann sogar gemacht werden, da ja nur inverse bild automat gebaut werden muss, und das geht
% ohne blow-up.
%A decision procedure as exhibited in Subsection~\ref{sec:decision_procedure}
%could also be easily given for the constraint languages considered in Subsection~\ref{subsec:strongly_self_sync}
%by the methods of proof and a homomorphic mapping from a strictly bounded language. However, I do not know if this is also possible in polynomial time,  as 

%In a previous work~\cite{DBLP:conf/ictcs/Hoffmann20}, we have shown that for polycyclic languages,
%which equal the sparse regular languages by Theorem~\ref{thm:bounded_characterizations},
%CSP is always in \NP\  and also gave partial result for \NP-hardness and containment in \PTIME.

Let us relate our results to the previous work~\cite{DBLP:conf/ictcs/Hoffmann20}, where
partial results for \NP-hardness and containment in \PTIME\  were given.
Namely, by setting $\factor(L) = \{ v \in \Sigma^* \mid \exists u,w \in \Sigma^* : uvw \in L \}$
and $\mathcal B_{p,E} = (\Sigma, P, \mu, q, E)$
for $\mathcal B = (\Sigma, P, \mu, p_0, F)$ with $E \subseteq P$ and $q \in P$,
the following was stated.

\begin{proposition}[\cite{DBLP:conf/ictcs/Hoffmann20}]
\label{prop:NPc}
 Suppose we find $u, v \in \Sigma^*$ 
 such that we can write
$
 L = u v^* U
$
 for some non-empty language $U \subseteq \Sigma^*$
 with 
 $
  u \notin \factor(v^*), %\quad
  v \notin \factor(U) \mbox{ and } %\quad
  \pref(v^*) \cap U = \emptyset.
 $
 Then $L\textsc{-Constr-Sync}$ is $\NP$-hard.
\end{proposition}

\begin{proposition}[\cite{DBLP:conf/ictcs/Hoffmann20}]
\label{prop:NP_in_P}
  Let $\mathcal{B} = (\Sigma, P, \mu, p_0, F)$ be a polycyclic PDFA.
  If for every reachable $p \in P$ with $L(\mathcal B_{p, \{p\}}) \ne \{\varepsilon\}$ 
  we have $L(\mathcal B_{p_0, \{p\}}) \subseteq \suff(L(\mathcal B_{p, \{p\}}))$,
  then the problem $L(\mathcal B)\textsc{-Constr-Sync}$ is solvable
  in polynomial time.
\end{proposition}

Note that Proposition~\ref{prop:NPc} implies that $ab^*a$
gives an \NP-complete CSP. However, in the letter-bounded
case there exist constraint languages giving \NP-complete problems
for which this is not implied by Proposition~\ref{prop:NPc},
for example: $ab^*ba$, $ab^*ab$, $aa^*abb^*a$ or $ba^*b \cup a$.
Also, Proposition~\ref{prop:NP_in_P}
is weaker than our Proposition~\ref{prop:stricly_bounded_P}
in the case of letter-bounded constraints.
For example, it does not apply to $ab^*b$, every PDFA for this languages
has a loop exclusively labelled by the letter~$b$
and reachable after reading the letter $a$ from the start state, and
so words along this loop cannot have a word starting with $a$ as a suffix.






For general bounded languages, let us note the following implication of Propositions~\ref{prop:hom_lower_bound_complexity}
and~\ref{prop:stricly_bounded_P}.

\todo{Sind thin languages eigentlich die in $w^*$?}

\begin{propositionrep}
 Let $u,v \in \Sigma^*$. If $L \subseteq u^* v^*$ is regular, then $L$\textsc{-Constr-Sync} is solvable
 in polynomial time.
\end{propositionrep}
\begin{proof}
 Let $\Gamma = \{a,b\}$
 and $\varphi : \Gamma^* \to \Sigma^*$
 be the homomorphism given by $\varphi(a) = u$
 and $\varphi(b) = v$.
 Define $N = \{ (i,j) \mid u^i v^j \in L \}$
 and set $L' = \{ a^i b^j \mid (i,j) \in N \} \subseteq a^* b^*$.
 Then, $\varphi(L') = L$
 and by Proposition~\ref{prop:stricly_bounded_P}
 we have $L'\textsc{-Constr-Sync} \in \PTIME$.
 So, with Proposition~\ref{prop:hom_lower_bound_complexity}
 also $L\textsc{-Constr-Sync} \in \PTIME$.~\qed
\end{proof}

Next, in Proposition~\ref{prop:np_complete_case}, we give an example
of a bounded regular language yielding an $\NP$-complete synchronization problem,
 but for which this is
 not directly implied by the results we have so far.

% \begin{remark}\label{rem:np_complete_case}
%  The bounded language $L = (ab)(ba)^*(ab)$
%  gives an $\NP$-complete synchronization problem, but this is
%  not directly implied by our methods.
 
 \begin{propositionrep}\label{prop:np_complete_case}
  The problem $((ab)(ba)^*(ab))$\textsc{-Constr-Sync} is $\NP$-complete.
 \end{propositionrep}
 \begin{proof}
 We give a reduction from $\textsc{DisjointSetTransporter}$
 for unary input semi-automata, which is $\NP$-complete
 by Theorem~\ref{prop:set_transporter_np_complete}.
 Let $\mathcal A = (\{c\}, Q, \delta)$
 with $S, T \subseteq Q$ being disjoint.
 Construct the automaton $\mathcal A' = (\{a,b\}, Q', \delta')$
 with
 \[
  Q' = Q \cup \{ q_a \mid q \in Q \} \cup \{ q_b \mid q \in Q \} \cup \{ t \}. 
 \]
 Fix some $\hat s \in S$.
 Then, for $q \in Q$, set
 % loop q_a mit a, und q_b bei b
 % aber dass für reduktion von cd^*c - vielleicht auch bemerken.
%  \[
%   \delta'(q, x) = \left\{ 
%   \begin{array}{ll}
%   q_x  & \mbox{if } q \in Q; \\ 
%   q    & \mbox{if } q  
%   \end{array}
%   \right.
%  \]
\begin{align*}
    \delta'(t,   x) & = t \mbox{ for } x \in \Sigma \mbox{ and }
    \delta'(q,   x) = q_x \mbox{ for } x \in \Sigma; \\
    \delta'(q_b, b) & = t   \mbox{ for } q_b \in Q'  \mbox{ and } 
    \delta'(q_b, a) = \delta(q, c); \\
    \delta'(q_a, a) & = q_a \mbox{ for } q_a \in Q'; \\
    \delta'(q_a, b) & = \left\{ 
    \begin{array}{ll}
     \hat s & \mbox{if } q \in Q \setminus (T \cup S); \\
     q      & \mbox{if } q \in S; \\
     t      & \mbox{if } q \in T.
    \end{array}
    \right.
\end{align*}
 Then, there exists $n \ge 0$
 with $\delta(S, c^n) \subseteq T$
 if and only if $\mathcal A'$ has a synchronizing word in $L$.
 % S,T disjoint, deswegen ba mindestens einmal
 
 First, suppose there exists $n \ge 0$
 such that $\delta(S, c^n) \subseteq T$.
 By construction, $S \subseteq \delta'(Q', ab) \subseteq S \cup \{t\} \cup Q_b$,
 or more precisely $\delta'(Q', ab) = S \cup \{t\} \cup \{ q_b \mid q \in \delta(Q, c) \}$.
 Note that, as $S$ and $T$ are disjoint,
 we must have $n > 0$.
 As, for any $q \in Q$, $\delta'(q, ba) = \delta(q, c)$
 and $\delta(q_b, b) = t$,
 we find $\delta'(\delta'(Q', ab), (ba)^n) \subseteq T \cup \{t\}$,
 where we needed $n > 0$ to map those states in $\{ q_b \mid q \in \delta(Q, c) \}$
 to $T$.
 Finally, $\delta(T \cup \{t\}, ab) = \{t\}$
 and so $\delta'(Q', ab(ba)^nab) = \{t\}$.
 
 
 Conversely, suppose there exists $n \ge 0$
 such that $\delta'(Q', ab(ba)^nab)$
 is a singleton set. So, as $t$ is a sink state,
 $\delta'(Q', ab(ba)^nab) = \{ t \}$.
 By construction, a state in $Q'$ is mapped to $t$
 by $ab$
 if and only if it is contained in $T \cup \{t\}$.
 Hence, $\delta'(Q', ab(ba)^n) \subseteq T \cup \{t\}$.
 As before, $\delta'(Q', ab) = S \cup \{t\} \cup \{ q_b : q \in \delta(Q, c) \}$.
 In particular, we must have $\delta'(S, (ba)^n) \subseteq T \cup \{t\}$.
 As, for any $q \in Q$, $\delta'(q, ba) = \delta(q, c)$,
 this implies that $\delta'(S, (ba)^n) \subseteq T$
 and that for $u = c^n$ we have $\delta(S, c^n) \subseteq T$.
 
 
 By Theorem~\ref{thm:sparse_in_NP}, $L\textsc{-Constr-Sync}\in \NP$
 and by the above reduction the problem is $\NP$-complete.
 \end{proof}
%\end{remark}

 By Proposition~\ref{prop:np_complete_case},
 for the homomorphism $\varphi : \{a,b\}^* \to \{a,b\}^*$
 given by $\varphi(a) = ab$ and $\varphi(b) = ba$
 both problems $ab^*a$ and $\varphi(ab^*a) = ab(ba)^*ab$
 are \NP-complete. So, this is a homomorphisms
 which preserves, in this concrete instance, the computational complexity.
 But its image $\{ab,ba\}$ is not even a self-synchronizing code.\todo{Ich glaube in dem fall schon. die reduktion sollte immer gehen...}
 However, I do not know if this homomorphism always preserves the complexity.
 Similary, I do not know
 if the condition from Theorem~\ref{thm:constr_sync_hom_strongly_self_sync}
 characterizes those homomorphisms which preserve the complexity.
 %also shows that injective homomorphisms whose image forms a strongly
 %connected code do not characterize those homomorphisms



 In the reduction used in Lemma~\ref{lem:np_hardness}
 the resulting automaton has a sink state. However, in general, for questions
 of synchronizability it makes a difference if we have a sink state
 or not, at least with respect to the \v{C}ern\'y conjecture~\cite{Cer64},
 as for automata with a sink state this conjecture holds true,
 even with the better bound\footnote{In~\cite{DBLP:journals/tcs/Rystsov97}
 erroneously the bound $n(n+1)/2$ was reported as being sharp, but the overall argument
 in fact works to yield the sharp bound $n(n-1)/2$.}
 $\frac{n(n-1)}{2}$~\cite{DBLP:journals/tcs/Rystsov97,DBLP:journals/tcs/Volkov09}. However,
 even in~\cite{DBLP:conf/mfcs/FernauGHHVW19}
 certain reductions establishing \PSPACE-completeness
 use only automata with a sink state. Hence, for hardness
 these automata are sufficient at least in certain instances.
 So, it might be interesting to know
 if in terms of computational complexity of the CSP,
 %it makes no difference if we consider automata with or without a sink state.
 we can, without loss of generality, limit ourselves to input automata
 with a sink state. The methods of proof for the letter-bounded constraints
 show that in this case, we can actually do this, as these input automata
 are sufficient to establish all cases of intractability.
 

 Lastly, let us mention the following related problem\footnote{This was actually suggested
 by a reviewer of a previous version.} one could come up with.
 Fix a deterministic and complete semi-automaton~$\mathcal A$.
 Then, for input PDFAs~$\mathcal B$, what is the computational complexity to determine
 if $\mathcal A = (\Sigma, Q, \delta)$ has a synchronizing word in $L(\mathcal B)$?
 As the set of synchronizing words 
 $ \{ w \in \Sigma^* : |\delta(Q, w)| = 1 \} = \bigcup_{q \in Q} \bigcap_{q' \in Q} L((\Sigma, Q, \delta, q', \{q\})) $
 is a regular language, we have to test
 for non-emptiness of intersection of this fixed regular language 
 with $L(\mathcal B)$. This could be done in \NL, hence in \PTIME.
 

\smallskip \noindent {\footnotesize
\textbf{Acknowledgement.} I thank  anonymous reviewers
of a previous version for detailed feedback.
I also sincerely thank the reviewers of the current version (at least one reviewers saw both versions) for careful reading and giving valuable feedback to improve my scientific writing
and pointing to two instances were I overlooked, in retrospect, two simple conclusions.}





\bibliographystyle{splncs04}
\bibliography{ms} 
\end{document}