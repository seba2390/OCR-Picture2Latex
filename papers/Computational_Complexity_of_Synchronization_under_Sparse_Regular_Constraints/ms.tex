\documentclass[runningheads,envcountsame]{llncs}
%\documentclass{article}
%
%\usepackage{graphicx}
% Used for displaying a sample figure. If possible, figure files should
% be included in EPS format.
%
% If you use the hyperref package, please uncomment the following line
% to display URLs in blue roman font according to Springer's eBook style:
% \renewcommand\UrlFont{\color{blue}\rmfamily}

\usepackage{amssymb}
\usepackage{amsmath}
\usepackage{comment}
%\usepackage[color=gray!27]{todonotes}
\usepackage[disable]{todonotes}
\usepackage[utf8]{inputenc}
%\usepackage[english,russian]{babel}
\usepackage{inputenc}
\usepackage{shuffle}
\usepackage{multirow}
\usepackage{listings}
\usepackage{mathabx}
%\usepackage[sectionbib, square,sort,comma,numbers]{natbib}
%\usepackage{adjustbox}

%\usepackage{thmtools, thm-restate}
\usepackage{hyperref}

\usepackage{shuffle}

\usepackage{tikz}
\usetikzlibrary{positioning,shadows,arrows}
\usetikzlibrary{arrows,automata,positioning,calc}


% https://tex.stackexchange.com/questions/7262/diagonally-divided-table-cell?noredirect=1&lq=1
\usepackage{diagbox}
\usepackage{slashbox}
\usepackage{tikz}
\usetikzlibrary{matrix}


%todo extradatei für review-bemerkungen
\lstset{%
  language=[LaTeX]TeX,
  backgroundcolor=\color{gray!10},
  basicstyle=\ttfamily,
  breaklines=true,
  columns=fullflexible
}


%\usetikzlibrary{arrows,shapes,automata,positioning}

\newcommand{\perm}{\operatorname{perm}}
\newcommand{\stc}{\operatorname{sc}}
\newcommand{\lastsym}{\operatorname{last}}




\usepackage{blindtext,tikz}
\usetikzlibrary{calc}


% \usepackage{apxproof}
% \theoremstyle{plain}
% \newtheoremrep{theorem}{Theorem}%[section]
% \newtheoremrep{proposition}[theorem]{Proposition}
% \newtheoremrep{lemma}[theorem]{Lemma}
% \newtheoremrep{claim}[theorem]{Claim}
% \newtheoremrep{conjecture}[theorem]{Conjecture}
% \newtheoremrep{corollary}[theorem]{Corollary}
% \newtheoremrep{definition}[theorem]{Definition}
 
 
\usepackage{apxproof}
\theoremstyle{plain}
%\newcounter{theorem2}
% \newtheoremrep{theorem}{Theorem}[section] % wie mit einem zähler durchnummerieren?
% \newtheoremrep{proposition}[theorem]{Proposition}
% \newtheoremrep{lemma}[theorem]{Lemma}
% \newtheoremrep{claim}[theorem]{Claim}
% \newtheoremrep{conjecture}[theorem]{Conjecture}
% \newtheoremrep{corollary}[theorem]{Corollary}
% \theoremstyle{definition}
% \newtheoremrep{definition}[theorem]{Definition}
%\theoremstyle{remark}
%\newtheoremrep{example}[theorem]{Example}
%\newtheoremrep{remark}[theorem]{Remark}
\newtheoremrep{theorem}{Theorem}
\newtheoremrep{proposition}[theorem]{Proposition}
\newtheoremrep{lemma}[theorem]{Lemma}
\newtheoremrep{claim}[theorem]{Claim}
\newtheoremrep{conjecture}[theorem]{Conjecture}
\newtheoremrep{corollary}[theorem]{Corollary}
\theoremstyle{definition}
\newtheoremrep{definition}[theorem]{Definition}
 
% https://tex.stackexchange.com/questions/104098/create-a-claim-environment
\newenvironment{claiminproof}[1]{\medskip\par\noindent\underline{Claim:}\space#1}{}
\newenvironment{claimproof}[1]{\begin{quote}\par\noindent\emph{Proof of the Claim:}\space#1}{[\emph{End, Proof of the Claim}]\end{quote}}%\newline}
% {\leavevmode\unskip\penalty9999 \hbox{}\nobreak\hfill\quad\hbox{$\blacksquare$}}
 
% \usepackage[printwatermark]{xwatermark}
% \usepackage{xcolor}
% \usepackage{graphicx}
% \newwatermark[pagex={1},color=gray!20,angle=45,scale=1.5,xpos=-10,ypos=70]{This paper eligible for the best student paper award, \\ as I am a PhD student
% under the supervision of Prof. Dr. Henning Fernau.} %Submission for review. \\ Full proofs in Appendix.}



\usepackage{fancyhdr}

 
\DeclareMathOperator{\lcm}{lcm}

%\def\baselinestretch{0.99}
%\linespread{0.9}
  
 %\usepackage[small,compact]{titlesec}
%\usepackage[small]{titlesec}

%\usepackage[text={12cm,20cm}]{geometry}
% \usepackage[compact]{titlesec}
% \titlespacing*{\section}{0pt}{2ex}{1ex}
%\titlespacing*{\subsection}{0pt}{1.6ex}{0.7ex}

% http://www.terminally-incoherent.com/blog/2007/09/19/latex-squeezing-the-vertical-white-space/
% http://www-h.eng.cam.ac.uk/help/tpl/textprocessing/squeeze.html
% https://robjhyndman.com/hyndsight/squeezing-space-with-latex/
   
 
%\def\dotminus{\mathbin{\ooalign{\hss\raise1ex\hbox{.}\hss\cr
%  \mathsurround=0pt$-$}}} 
 
% https://tex.stackexchange.com/questions/103735/list-of-todos-todonotes-is-empty-with-llncs?noredirect=1
%\setcounter{tocdepth}{1}



% https://tex.stackexchange.com/questions/186677/big-shuffle-symbol
% large ops, copied from shuffle font package
\DeclareFontFamily{U}{bigshuffle}{}
\DeclareFontShape{U}{bigshuffle}{m}{n}{
  <5-8> s*[1.7] shuffle7
  <8->  s*[1.7] shuffle10
}{}
\DeclareSymbolFont{BigShuffle}{U}{bigshuffle}{m}{n}
\DeclareMathSymbol\bigshuffle{\mathop}{BigShuffle}{"001}
\DeclareMathSymbol\bigcshuffle{\mathop}{BigShuffle}{"002}

\newcommand{\suff}{\operatorname{Suff}}
\newcommand{\factor}{\operatorname{Fact}}

\newcommand{\pref}{\operatorname{Pref}}

\newcommand{\Orb}{\operatorname{Orb}}

\newcommand{\NC}{\textsf{NC}}
\newcommand{\NL}{\textsf{NL}}
\newcommand{\NP}{\textsf{NP}}
\newcommand{\PSPACE}{\textsf{PSPACE}}
\newcommand{\NPSPACE}{\textsf{NPSPACE}}
\newcommand{\PTIME}{\textsf{P}}
\newcommand{\XP}{\textsf{XP}}


% https://latex.org/forum/viewtopic.php?t=10877
% https://latex.org/forum/viewtopic.php?f=47&t=10862
% https://tex.stackexchange.com/questions/36423/random-unwanted-space-between-paragraphs
%\raggedbottom



% https://tex.stackexchange.com/questions/255673/problem-definition-environment
% https://www.overleaf.com/learn/latex/Environments
% https://de.overleaf.com/learn/latex/Counters
% https://en.wikibooks.org/wiki/LaTeX/Counters
% https://de.wikibooks.org/wiki/LaTeX-W%C3%B6rterbuch:_refstepcounter
\usepackage{tabularx,lipsum,environ,amsmath,amssymb}

%\usepackage{natbib}
%\usepackage{biblatex}
%\addbibresource{ms.bib}


\newcounter{problemcounter}
\makeatletter
\newcommand{\problemtitle}[1]{\gdef\@problemtitle{#1}}% Store problem title
\newcommand{\probleminput}[1]{\gdef\@probleminput{#1}}% Store problem input
\newcommand{\problemquestion}[1]{\gdef\@problemquestion{#1}}% Store problem question
\NewEnviron{decproblem}{
  \refstepcounter{problemcounter}
  \problemtitle{}\probleminput{}\problemquestion{}% Default input is empty
  \BODY% Parse input
  \par\addvspace{.5\baselineskip}
  \noindent
  \begin{tabularx}{\textwidth}{@{\hspace{\parindent}} l X c}
    \multicolumn{2}{@{\hspace{\parindent}}l}{\textbf{Decision Problem \theproblemcounter:} \@problemtitle} \\% Title
    \textbf{Input:} & \@probleminput \\% Input
    \textbf{Question:} & \@problemquestion% Question
  \end{tabularx}
  \par\addvspace{.5\baselineskip}
}
\makeatother
 
 
 
% % https://tex.stackexchange.com/questions/104098/create-a-claim-environment
 \newenvironment{myclaiminproof}[1]{\medskip\par\noindent\underline{Claim:}\space#1}{}
 \newenvironment{myclaimproof}[1]{\begin{quote}\par\noindent\emph{Proof of the Claim:}\space#1}{[\emph{End, Proof of the Claim}]\end{quote}}%\newline}
% % {\leavevmode\unskip\penalty9999 \hbox{}\nobreak\hfill\quad\hbox{$\blacksquare$}}




\renewcommand{\headrulewidth}{0pt}
\fancypagestyle{AllPages}{
\chead{2016 IEEE/ACM International Conference on Advances in Social Networks Analysis and Mining (ASONAM)}
}
\fancypagestyle{FirstPage}{
\chead{2016 IEEE/ACM International Conference on Advances in Social Networks Analysis and Mining (ASONAM)}
\lfoot{IEEE/ACM ASONAM 2016, August 18-21\\2016, San Francisco, CA,     USA\\
978-1-5090-2846-7/16/\$~\copyright~2016 IEEE}
}

\chead{2016 IEEE/ACM International Conference on Advances in Social Networks Analysis and Mining (ASONAM)}


\begin{document}

%
\title{Computational Complexity of Synchronization under Sparse Regular Constraints}
%\title{State Complexity of Projected Languages of Permutation Automata}
%\title{State Complexity of Projection on Permutation Automata}
\titlerunning{Synchronization under Sparse Regular Constraints}

%
%\titlerunning{Abbreviated paper title}
% If the paper title is too long for the running head, you can set
% an abbreviated paper title here
%
\author{Stefan Hoffmann\orcidID{0000-0002-7866-075X}}
%
\authorrunning{S. Hoffmann}
% First names are abbreviated in the running head.
% If there are more than two authors, 'et al.' is used.
%
\institute{Informatikwissenschaften, FB IV, 
  Universit\"at Trier, Germany, 
  \email{hoffmanns@informatik.uni-trier.de}}
%
\maketitle              % typeset the header of the contribution
%
\begin{abstract}
%  We investigate the constrained synchronization problem for weakly acyclic, or partially ordered,
%  input automata. We show that, for input automata of this type, the problem is always
%  in $\NP$.
%  Furthermore, we give a full classification of the realizable complexities for constraint
%  automata with at most two states and over a ternary alphabet.
%  For certain constraint languages, for which the general problem is $\PSPACE$-complete,
%  for weakly acyclic automata we get $\NP$-complete problems, whereas there are also
%  problems that are $\PSPACE$-complete in general, but for which
%  it is polynomial time solvable in our setting.
%  %when only weakly acyclic automata
%  %are considered as input. 
%  %In the course of our investigation, 
%  We also investigate
%  two problems related to subset synchronization, namely if there exists
%  a word mapping all states into a given target subset of states, and
%  if there exists a word mapping one subset into another. Both problems
%  are $\PSPACE$-complete in general, but in our setting the former is polynomial time solvable and the latter is $\NP$-complete.
%  %
%  % genauer bescchreiben
%  % sync wort in vorgegebener
%  % reg. sprache?
 The constrained synchronization problem (CSP) asks
 for a synchronizing word of a given input automaton
 contained in a regular set of constraints. It could be viewed
 as a special case of synchronization of a discrete event system
 under supervisory control.
 Here, we study the computational complexity of this %the constrained synchronization
 problem for the class of sparse regular constraint languages.
 We give a new characterization of sparse regular sets, which
 equal the bounded regular sets,
 and derive a full classification of the computational complexity
 of CSP for letter-bounded regular constraint
 languages, which properly
 contain the strictly bounded regular languages.
 %In addition, we derive a polynomial time decision procedure
 %for the complexity of the constrained synchronization problem, given
 %a constraint automaton recognizing a strictly bounded regular language
 %as input.
 Then, we introduce strongly self-synchronizing codes
 and investigate CSP for  bounded languages induced by these codes.
 With our previous result, we deduce a full classification
 for these languages as well.
 In both cases, depending on the constraint language, our problem
 becomes $\NP$-complete or polynomial time solvable.
 %Additionally, we state a new characterization of bounded languages.
 
 %
 % keywords überarbeiten
 % und hinweis für student best paper
\keywords{automata theory \and constrained synchronization \and computational complexity \and sparse languages \and bounded languages \and strongly self-synchronizing codes} 
\end{abstract}
%
%
%

%\thispagestyle{FirstPage}
% \thispagestyle{fancy}
% \chead{}%\hspace*{-4cm} 
% \lhead{}
% \definecolor{mygray}{gray}{0.6}
% \definecolor{mypink1}{rgb}{0.558, 0.188, 0.278}
% \lfoot{\footnotesize \textcolor{mypink1}{Paper eligible for best student \\ paper award. I am a PhD student \\
%  under the supervision of \\ Prof. Dr. Henning Fernau.}}

% https://tex.stackexchange.com/questions/7400/watermark-on-first-page-in-left-margin-like-arxiv

%\blinddocument 



\section{Introduction} %contribution
\label{sec:introduction}




% \leavevmode
% \\
% \\
% \\
% \\
% \\
\section{Introduction}
\label{introduction}

AutoML is the process by which machine learning models are built automatically for a new dataset. Given a dataset, AutoML systems perform a search over valid data transformations and learners, along with hyper-parameter optimization for each learner~\cite{VolcanoML}. Choosing the transformations and learners over which to search is our focus.
A significant number of systems mine from prior runs of pipelines over a set of datasets to choose transformers and learners that are effective with different types of datasets (e.g. \cite{NEURIPS2018_b59a51a3}, \cite{10.14778/3415478.3415542}, \cite{autosklearn}). Thus, they build a database by actually running different pipelines with a diverse set of datasets to estimate the accuracy of potential pipelines. Hence, they can be used to effectively reduce the search space. A new dataset, based on a set of features (meta-features) is then matched to this database to find the most plausible candidates for both learner selection and hyper-parameter tuning. This process of choosing starting points in the search space is called meta-learning for the cold start problem.  

Other meta-learning approaches include mining existing data science code and their associated datasets to learn from human expertise. The AL~\cite{al} system mined existing Kaggle notebooks using dynamic analysis, i.e., actually running the scripts, and showed that such a system has promise.  However, this meta-learning approach does not scale because it is onerous to execute a large number of pipeline scripts on datasets, preprocessing datasets is never trivial, and older scripts cease to run at all as software evolves. It is not surprising that AL therefore performed dynamic analysis on just nine datasets.

Our system, {\sysname}, provides a scalable meta-learning approach to leverage human expertise, using static analysis to mine pipelines from large repositories of scripts. Static analysis has the advantage of scaling to thousands or millions of scripts \cite{graph4code} easily, but lacks the performance data gathered by dynamic analysis. The {\sysname} meta-learning approach guides the learning process by a scalable dataset similarity search, based on dataset embeddings, to find the most similar datasets and the semantics of ML pipelines applied on them.  Many existing systems, such as Auto-Sklearn \cite{autosklearn} and AL \cite{al}, compute a set of meta-features for each dataset. We developed a deep neural network model to generate embeddings at the granularity of a dataset, e.g., a table or CSV file, to capture similarity at the level of an entire dataset rather than relying on a set of meta-features.
 
Because we use static analysis to capture the semantics of the meta-learning process, we have no mechanism to choose the \textbf{best} pipeline from many seen pipelines, unlike the dynamic execution case where one can rely on runtime to choose the best performing pipeline.  Observing that pipelines are basically workflow graphs, we use graph generator neural models to succinctly capture the statically-observed pipelines for a single dataset. In {\sysname}, we formulate learner selection as a graph generation problem to predict optimized pipelines based on pipelines seen in actual notebooks.

%. This formulation enables {\sysname} for effective pruning of the AutoML search space to predict optimized pipelines based on pipelines seen in actual notebooks.}
%We note that increasingly, state-of-the-art performance in AutoML systems is being generated by more complex pipelines such as Directed Acyclic Graphs (DAGs) \cite{piper} rather than the linear pipelines used in earlier systems.  
 
{\sysname} does learner and transformation selection, and hence is a component of an AutoML systems. To evaluate this component, we integrated it into two existing AutoML systems, FLAML \cite{flaml} and Auto-Sklearn \cite{autosklearn}.  
% We evaluate each system with and without {\sysname}.  
We chose FLAML because it does not yet have any meta-learning component for the cold start problem and instead allows user selection of learners and transformers. The authors of FLAML explicitly pointed to the fact that FLAML might benefit from a meta-learning component and pointed to it as a possibility for future work. For FLAML, if mining historical pipelines provides an advantage, we should improve its performance. We also picked Auto-Sklearn as it does have a learner selection component based on meta-features, as described earlier~\cite{autosklearn2}. For Auto-Sklearn, we should at least match performance if our static mining of pipelines can match their extensive database. For context, we also compared {\sysname} with the recent VolcanoML~\cite{VolcanoML}, which provides an efficient decomposition and execution strategy for the AutoML search space. In contrast, {\sysname} prunes the search space using our meta-learning model to perform hyperparameter optimization only for the most promising candidates. 

The contributions of this paper are the following:
\begin{itemize}
    \item Section ~\ref{sec:mining} defines a scalable meta-learning approach based on representation learning of mined ML pipeline semantics and datasets for over 100 datasets and ~11K Python scripts.  
    \newline
    \item Sections~\ref{sec:kgpipGen} formulates AutoML pipeline generation as a graph generation problem. {\sysname} predicts efficiently an optimized ML pipeline for an unseen dataset based on our meta-learning model.  To the best of our knowledge, {\sysname} is the first approach to formulate  AutoML pipeline generation in such a way.
    \newline
    \item Section~\ref{sec:eval} presents a comprehensive evaluation using a large collection of 121 datasets from major AutoML benchmarks and Kaggle. Our experimental results show that {\sysname} outperforms all existing AutoML systems and achieves state-of-the-art results on the majority of these datasets. {\sysname} significantly improves the performance of both FLAML and Auto-Sklearn in classification and regression tasks. We also outperformed AL in 75 out of 77 datasets and VolcanoML in 75  out of 121 datasets, including 44 datasets used only by VolcanoML~\cite{VolcanoML}.  On average, {\sysname} achieves scores that are statistically better than the means of all other systems. 
\end{itemize}


%This approach does not need to apply cleaning or transformation methods to handle different variances among datasets. Moreover, we do not need to deal with complex analysis, such as dynamic code analysis. Thus, our approach proved to be scalable, as discussed in Sections~\ref{sec:mining}.

\section{Preliminaries and Definitions}
\label{sec:preliminaries}

%!TEX root = hopfwright.tex
%

In this section we systematically recast the Hopf bifurcation problem in Fourier space. 
We introduce appropriate scalings, sequence spaces of Fourier coefficients and convenient operators on these spaces. 
To study Equation~\eqref{eq:FourierSequenceEquation} we consider Fourier sequences $ \{a_k\}$ and fix a Banach space in which these sequences reside. It is indispensable for our analysis that this space have an algebraic structure. 
The Wiener algebra of absolutely summable Fourier series is a natural candidate, which we use with minor modifications. 
In numerical applications, weighted sequence spaces with algebraic and geometric decay have been used to great effect to study periodic solutions which are $C^k$ and analytic, respectively~\cite{lessard2010recent,hungria2016rigorous}. 
Although it follows from Lemma~\ref{l:analytic} that the Fourier coefficients of any solution decay exponentially, we choose to work in a space of less regularity. 
The reason is that by working in a space with less regularity, we are better able to connect our results with the global estimates in \cite{neumaier2014global}, see Theorem~\ref{thm:UniqunessNbd2}.


%
%
%\begin{remark}
%	Although it follows from Lemma~\ref{l:analytic} that the Fourier coefficients of any solution decay exponentially, we choose to work in a space of less regularity, namely summable Fourier coefficients. This allows us to draw SOME MORE INTERESTING CONCLUSION LATER.
%	EXPLAIN WHY WE CHOOSE A NORM WITH ALMOST NO DECAY!
%	% of s Periodic solutions to Wright's equation are known to be real analytic and so their  Fourier coefficients must decay geometrically [Nussbaum].
%	% We do not use such a strong result;  any periodic solution must be continuously differentiable, by which it follows that $ \sum | c_k| < \infty$.
%\end{remark}


\begin{remark}\label{r:a0}
There is considerable redundancy in Equation~\eqref{eq:FourierSequenceEquation}. First, since we are considering real-valued solutions $y$, we assume $\c_{-k}$ is the complex conjugate of $\c_k$. This symmetry implies it suffices to consider Equation~\eqref{eq:FourierSequenceEquation} for $k \geq 0$.
Second, we may effectively ignore the zeroth Fourier coefficient of any periodic solution \cite{jones1962existence}, since it is necessarily equal to $0$. 
%In \cite{jones1962existence}, it is shown that if $y \not\equiv -1$ is a periodic solution of~\eqref{eq:Wright} with frequency $\omega$, then $ \int_0^{2\pi/\omega} y(t) dt =0$. 
		The self contained argument is as follows. 
		As mentioned in the introduction, any periodic solution to Wright's equation must satisfy $ y(t) > -1$ for all $t$. 
	By dividing Equation~\eqref{eq:Wright} by $(1+y(t))$, which never vanishes, we obtain
	\[
	\frac{d}{dt} \log (1 + y(t)) = - \alpha y(t-1).
	\]  
	Integrating over one period $L$ we derive the condition 
	$0=\int_0^L y(t) dt $.
	Hence $a_0=0$ for any periodic solution. 
	It will be shown in Theorem~\ref{thm:FourierEquivalence1} that a related argument implies that we do not need to consider Equation~\eqref{eq:FourierSequenceEquation} for $k=0$.
\end{remark}

%%%
%%%
%%%\begin{remark}\label{r:c0} 
%%%In \cite{jones1962existence}, it is shown that if $y \not\equiv -1$ is a periodic solution of~\eqref{eq:Wright} with frequency $\omega$, then $ \int_0^{2\pi/\omega} y(t) dt =0$. 
%%%PERHAPS TOO MUCH DETAIL HERE. The self contained argument is as follows.
%%%If $y \not\equiv -1$ then $y(t) \neq -1$ for all $t$, since if $y(t_0)=-1$ for some $t_0 \in \R$ then $y'(t_0)=0$ by~\eqref{eq:Wright} and in fact by differentiating~\eqref{eq:Wright} repeatedly one obtains that all derivatives of $y$ vanish at $t_0$. Hence $y \equiv -1$ by Lemma~\ref{l:analytic}, a contradiction. Now divide~\eqref{eq:Wright} by $(1+y(t))$, which never vanishes, to obtain
%%%\[
%%%  \frac{d}{dt} \log |1 + y(t)| = - \alpha y(t-1).
%%%\]  
%%%Integrating over one period we obtain $\int_0^L y(t) dt =0$.
%%%\end{remark}



%Furthermore, the condition that $y(t)$ is real forces $\c_{-k} = \overline{\c}_{k}$.  
%
We define the spaces of absolutely summable Fourier series
\begin{alignat*}{1}
	\ell^1 &:= \left\{ \{ \c_k \}_{k \geq 1} : 
    \sum_{k \geq 1} | \c_k| < \infty  \right\} , \\
	\ell^1_\bi &:= \left\{ \{ \c_k \}_{k \in \Z} : 
    \sum_{k \in \Z} | \c_k| < \infty  \right\} .
\end{alignat*} 
We identify any semi-infinite sequence $ \{ \c_k \}_{k \geq 1} \in \ell^1$ with the bi-infinite sequence $ \{ \c_k \}_{k \in \Z} \in \ell^1_\bi$ via the conventions (see Remark~\ref{r:a0})
\begin{equation}
  \c_0=0 \qquad\text{ and }\qquad \c_{-k} = \c_{k}^*. 
\end{equation}
In other word, we identify $\ell^1$ with the set
\begin{equation*}
   \ell^1_\sym := \left\{ \c \in \ell^1_\bi : 
	\c_0=0,~\c_{-k}=\c_k^* \right\} .
\end{equation*}
On $\ell^1$ we introduce the norm
\begin{equation}\label{e:lnorm}
  \| \c \| = \| \c \|_{\ell^1} := 2 \sum_{k = 1}^\infty |\c_k|.
\end{equation}
The factor $2$ in this norm is chosen to have a Banach algebra estimate.
Indeed, for $\c, \tilde{\c} \in \ell^1 \cong \ell^1_\sym$ we define
the discrete convolution 
\[
\left[ \c * \tilde{\c} \right]_k = \sum_{\substack{k_1,k_2\in\Z\\ k_1 + k_2 = k}} \c_{k_1} \tilde{\c}_{k_2} .
\]
Although $[\c*\tilde{\c}]_0$ does not necessarily vanish, we have $\{\c*\tilde{\c}\}_{k \geq 1} \in \ell^1 $ and 
\begin{equation*}
	\| \c*\tilde{\c} \| \leq \| \c \| \cdot  \| \tilde{\c} \| 
	\qquad\text{for all } \c , \tilde{\c} \in \ell^1, 
\end{equation*}
hence $\ell^1$ with norm~\eqref{e:lnorm} is a Banach algebra.

By Lemma~\ref{l:analytic} it is clear that any periodic solution of~\eqref{eq:Wright} has a well-defined Fourier series $\c \in \ell^1_\bi$. 
The next theorem shows that in order to study periodic orbits to Wright's equation we only need to study Equation~\eqref{eq:FourierSequenceEquation} 
for $k \geq 1$. For convenience we introduce the notation 
\[
G(\alpha,\omega,\c)_k=
( i \omega k + \alpha e^{ - i \omega k}) \c_k + \alpha \sum_{k_1 + k_2 = k} e^{- i \omega k_1} \c_{k_1} \c_{k_2} \qquad \text{for } k \in \N.
\]
We note that we may interpret the trivial solution $y(t)\equiv 0$ as a periodic solution of arbitrary period.
\begin{theorem}
\label{thm:FourierEquivalence1}
Let $\alpha>0$ and $\omega>0$.
If $\c \in \ell^1 \cong \ell^1_{\sym}$ solves
$G(\alpha,\omega,\c)_k =0$  for all $k \geq 1$,
then $y(t)$ given by~\eqref{eq:FourierEquation} is a periodic solution of~\eqref{eq:Wright} with period~$2\pi/\omega$.
Vice versa, if $y(t)$ is a periodic solution of~\eqref{eq:Wright} with period~$2\pi/\omega$ then its Fourier coefficients $\c \in \ell^1_\bi$ lie in $\ell^1_\sym \cong \ell^1$ and solve $G(\alpha,\omega,\c)_k =0$ for all $k \geq 1$.
\end{theorem}

\begin{proof}	
	If $y(t)$ is a periodic solution of~\eqref{eq:Wright} then it is real analytic by Lemma~\ref{l:analytic}, hence its Fourier series $\c$ is well-defined and $\c \in \ell^1_{\sym}$ by Remark~\ref{r:a0}.
	Plugging the Fourier series~\eqref{eq:FourierEquation} into~\eqref{eq:Wright} one easily derives that $\c$ solves~\eqref{eq:FourierSequenceEquation} for all $k \geq 1$.

To prove the reverse implication, assume that $\c \in \ell^1_\sym$ solves
Equation~\eqref{eq:FourierSequenceEquation} for all $k \geq 1$. Since $\c_{-k}
= \c_k^*$, Equation \eqref{eq:FourierSequenceEquation} is also satisfied for
all $k \leq -1$. It follows from the Banach algebra property and
\eqref{eq:FourierSequenceEquation} that $\{k \c_k\}_{k \in \Z} \in \ell^1_\bi$,
hence $y$, given by~\eqref{eq:FourierEquation}, is continuously differentiable.
% (and by bootstrapping one infers that $\{k^m c_k \} \in \ell^1_\bi$, 
% hence $y \in C^m$ for any $m \geq 1$).
	Since~\eqref{eq:FourierSequenceEquation} is satisfied for all $k \in \Z \setminus \{0\}$ (but not necessarily for $k=0$) one may perform the inverse Fourier transform on~\eqref{eq:FourierSequenceEquation} to conclude that
	$y$ satisfies the delay equation 
\begin{equation}\label{eq:delaywithK}
   	y'(t) = - \alpha y(t-1) [ 1 + y(t)] + C
\end{equation}
	for some constant $C \in \R$. 
   Finally, to prove that $C=0$ we argue by contradiction.
   Suppose $C \neq 0$. Then $y(t) \neq -1$ for all $t$.
   Namely, at any point where $y(t_0) =-1$ one would have $y'(t_0) = C$
   which has fixed sign,   hence it would follow that $y$ is not periodic
   ($y$ would not be able to cross $-1$ in the opposite direction, 
   preventing $y$  from being periodic).  
  We may thus divide~\eqref{eq:delaywithK} through by $1 + y(t)$ and obtain 
\begin{equation*}
	\frac{d}{dt} \log | 1 + y(t) | = - \alpha y(t-1) + \frac{C}{1+y(t)} .
\end{equation*}
	By integrating both sides of the equation over one period $L$ and by using that $\c_0=0$, we 
	obtain
	\[
	 C \int_0^L \frac{1}{1+y(t)} dt =0.
	\]
	Since the integrand is either strictly negative or strictly positive, this implies that $C=0$. Hence~\eqref{eq:delaywithK} reduces to~\eqref{eq:Wright},
	and $y$ satisfies Wright's equation. 
\end{proof}






To efficiently study Equation~\eqref{eq:FourierSequenceEquation}, we introduce the following linear operators on $ \ell^1$:
\begin{alignat*}{1}
   [K \c ]_k &:= k^{-1} \c_k  , \\ 
   [ U_\omega \c ]_k &:= e^{-i k \omega} \c_k  .
\end{alignat*}
The map $K$ is a compact operator, and it has a densely defined inverse $K^{-1}$. The domain of $K^{-1}$ is denoted by
\[
  \ell^K := \{ \c \in \ell^1 : K^{-1} \c \in \ell^1 \}.  
\]
The map $U_{\omega}$ is a unitary operator on $\ell^1$, but
it is discontinuous in $\omega$. 
With this notation, Theorem~\ref{thm:FourierEquivalence1} implies that our problem of finding a SOPS to~\eqref{eq:Wright} is equivalent to finding an $\c \in \ell^1$ such that
\begin{equation}
\label{e:defG}
  G(\alpha,\omega,\c) :=
  \left( i \omega K^{-1} + \alpha U_\omega \right) \c + \alpha \left[U_\omega \, \c \right] * \c  = 0.
\end{equation}


%In order for the solutions of Equation \ref{eq:FHat} to be isolated we need to impose a phase condition. 
%If there is a sequence $ \{ c_k \} $ which satisfies  Equation \ref{eq:FHat}, then $ y( t + \tau) = \sum_{k \in \Z} c_k e^{ i k \omega (t + \tau)}$ satisfies Wright's equation at parameter $\alpha$. 
%Fix $ \tau = - Arg[c_1] / \omega$ so that $ c_1  e^{ i \omega \tau} $ is a nonnegative real number. 
%By Proposition \ref{thm:FourierEquivalence1} it follows that $\{ c'_k \} =  \{c_k e^{ i \omega k \tau }   \}$ is a solution to Equation \ref{eq:FHat}, and furthermore that $ c'_1 = \epsilon$ for some $ \epsilon \geq 0$. 


Periodic solutions are invariant under time translation: if $y(t)$ solves Wright's equation, then so does $ y(t+\tau)$ for any $\tau \in \R$. 
We remove this degeneracy by adding a phase condition. 
Without loss of generality, if $\c \in \ell^1$ solves Equation~\eqref{e:defG}, we may assume that $\c_1 = \epsilon$ for some 
\emph{real non-negative}~$\epsilon$:
\[
  \ell^1_{\epsilon} := \{\c \in \ell^1 : \c_1 = \epsilon \} 
  \qquad \text{where } \epsilon \in \R,  \epsilon \geq 0.
\]
In the rest of our analysis, we will split elements $\c \in \ell^1$ into two parts: $\c_1$ and $\{\c_{k}\}_{k \geq 2}$.  
We define the basis elements $\e_j \in \ell^1$ for $j=1,2,\dots$ as
\[
  [\e_j]_k = \begin{cases}
  1 & \text{if } k=j, \\
  0 & \text{if } k \neq j.
  \end{cases}
\]
We note that $\| \e_j \|=2$. 
Then we can decompose
% We define
% \[
%   \tilde{\epsilon} := (\epsilon,0,0,0,\dots) \in \ell^1
% \]
% and
% For clarity when referring to sequences $\{c_{k}\}_{k \geq 2}$, we make the following definition:
% \[
% \ell^1_0  := \{ \tc \in \ell^1 : \tc_1 = 0 \}.
% \]
% With the
any $\c \in \ell^1_\epsilon$ uniquely as
\begin{equation}\label{e:aepsc}
  \c= \epsilon \e_1 + \tc \qquad \text{with}\quad 
  \tc \in \ell^1_0 := \{ \tc \in \ell^1 : \tc_1 = 0 \}.
\end{equation}
We follow the classical approach in studying Hopf bifurcations and consider 
$\c_1 = \epsilon$ to be a parameter, and then find periodic solutions with Fourier modes in $\ell^1_{\epsilon}$.
This approach rewrites the function $G: \R^2 \times \ell^K \to \ell^1$ as a function $\tilde{F}_\epsilon : \R^2 \times \ell^K_0 \to \ell^1$, where 
we denote 
\[
\ell^K_0 := \ell^1_0 \cap \ell^K.
\]
% I AM ACTUALLY NOT SURE IF YOU WANT TO DEFINE THIS WITH RANGE IN $\ell^1$
% OR WITH DOMAIN IN $\ell^1_0$ ?? IT SEEMS TO DEPEND ON WHICH GLOBAL STATEMENT YOU WANT/NEED TO MAKE!?
\begin{definition}
We define the $\epsilon$-parameterized family of  functions $\tilde{F}_\epsilon: \R^2 \times \ell^K_0  \to \ell^1$ 
by 
\begin{equation}
\label{eq:fourieroperators}
\tilde{F}_{\epsilon}(\alpha,\omega, \tc) := 
\epsilon [i \omega + \alpha e^{-i \omega}] \e_1 + 
( i \omega K^{-1} + \alpha U_{\omega}) \tc + 
\epsilon^2 \alpha e^{-i \omega}  \e_2  +
\alpha \epsilon L_\omega \tc + 
\alpha  [ U_{\omega} \tc] * \tc ,
\end{equation}
where
$L_\omega : \ell^1_0 \to \ell^1$ is given by
\[
   L_{\omega} := \sigma^+( e^{- i \omega} I + U_{\omega}) + \sigma^-(e^{i \omega} I + U_{\omega}),
\]
with $I$ the identity and  $\sigma^\pm$ the shift operators on $\ell^1$:
\begin{alignat*}{2}
\left[ \sigma^- a \right]_k &:=  a_{k+1}  , \\
\left[ \sigma^+ a \right]_k &:=  a_{k-1}  &\qquad&\text{with the convention } \c_0=0.
\end{alignat*}
The operator $ L_\omega$ is discontinuous in $\omega$ and $ \| L_\omega \| \leq 4$. 
\end{definition} 

%The maps $ \sigma^{+}$ and $ \sigma^-$ are shift up and shift down operators respectively. 
We reformulate Theorem~\ref{thm:FourierEquivalence1}  in terms of the map  $\tilde{F}$. 
We note that it follows from Lemma~\ref{l:analytic} and 
%\marginpar{Reformulate}
%one's choice of  
Equation~\eqref{eq:FourierSequenceEquation}  
%or Equation ~\eqref{eq:fourieroperators},
that the Fourier coefficients of any periodic solution of~\eqref{eq:Wright} lie in $\ell^K$.
These observations are summarized in the following theorem.
\begin{theorem}
\label{thm:FourierEquivalence2}
	Let $ \epsilon \geq 0$,  $\tc \in \ell^K_0$, $\alpha>0$ and $ \omega >0$. 
	Define $y: \R\to \R$ as 
\begin{equation}\label{e:ytc}
	y(t) = 
	\epsilon \left( e^{i \omega t }  + e^{- i \omega t }\right) 
	+  \sum_{k = 2}^\infty   \tc_k e^{i \omega k t }  + \tc_k^* e^{- i \omega k t } .
\end{equation}
%	and suppose that $ y(t) > -1$. 
	Then $y(t)$ solves~\eqref{eq:Wright} if and only if $\tilde{F}_{\epsilon}( \alpha , \omega , \tc) = 0$. 
	Furthermore, up to time translation, any periodic solution of~\eqref{eq:Wright} with period $2\pi/\omega$ is described by a Fourier series of the form~\eqref{e:ytc} with $\epsilon \geq 0$ and $\tc \in \ell^K_0$.
\end{theorem}


%We note that for $\epsilon>0$ such solutions are truly periodic, while for $\epsilon=0$ a zero of $\tilde{F}_\epsilon$ may either correspond to a periodic solution or to the trivial solution $y(t) \equiv 0$. 



% \begin{proof}
%  By Proposition \ref{thm:FourierEquivalence1}, it suffices to show that $\tilde{F}(\alpha,\omega,c) =0$ is equivalent to Equation \ref{eq:FourierSequenceEquation} being satisfied for $k \geq 1$.
%  Since Equation \ref{eq:FourierSequenceEquation} is equivalent to Equation \ref{eq:FHat}, we expand  Equation \ref{eq:FHat} by writing $ \hat{c} = \hat{\epsilon } + c$  where $ \hat{\epsilon} := (\epsilon,0,0,\dots) \in \ell^1$ as below:
%  \begin{equation}
%  0=  \left( i \omega K^{-1} + \alpha U_\omega \right) (\hat{\epsilon}+ c) + \alpha \left[U_\omega \, (\hat{\epsilon}+ c) \right] * (\hat{\epsilon}+ c) \label{eq:Intial}
%  \end{equation}
%  The RHS of Equation \ref{eq:Intial} is $ \tilde{F}(\alpha,\omega,c)$, so the theorem is proved.
% \end{proof}



Since we want to analyze a Hopf bifurcation, we will want to solve $\tilde{F}_\epsilon = 0$ for small values of~$\epsilon$. 
However, at the bifurcation point, $ D \tilde{F}_0(\pp  ,\pp , 0)$ is not invertible.
In order for our asymptotic analysis to be non-degenerate,
we work with a rescaled version of the problem. To this end, for any $\epsilon >0$, we rescale both $\tc$ and $\tilde{F}$ as follows. Let $\tc = \epsilon c$ and 
\begin{equation}\label{e:changeofvariables}
  \tilde{F}_\epsilon (\alpha,\omega,\epsilon c) = \epsilon F_\epsilon (\alpha,\omega,c).
\end{equation}
For $\epsilon>0$ the problem then reduces to finding zeros of 
\begin{equation}
\label{eq:FDefinition}
	F_\epsilon(\alpha,\omega, c) := 
	[i \omega + \alpha e^{-i \omega}] \e_1 + 
	( i \omega K^{-1} + \alpha U_{\omega}) c + 
	\epsilon \alpha e^{-i \omega} \e_2  +
	\alpha \epsilon L_\omega c + 
	\alpha \epsilon [ U_{\omega} c] * c.
\end{equation}
We denote the triple $(\alpha,\omega,c) \in \R^2 \times \ell^1_0$ by $x$.
To pinpoint the components of $x$ we use the projection operators
\[
   \pi_\alpha x = \alpha, \quad \pi_\omega x = \omega, \quad 
  \pi_c x = c \qquad\text{for any } x=(\alpha,\omega,c).
\]

After the change of variables~\eqref{e:changeofvariables} we now have an invertible Jacobian $D F_0(\pp  ,\pp , 0)$ at the bifurcation point.
On the other hand, for $\epsilon=0$ the zero finding problems for $\tilde{F}_\epsilon$ and $F_\epsilon$ are not equivalent. 
However, it follows from the following lemma that any nontrivial periodic solution having $ \epsilon=0$ must have a relatively large size when $ \alpha $ and $ \omega $ are close to the bifurcation point. 

\begin{lemma}\label{lem:Cone}
	Fix $ \epsilon \geq 0$ and $\alpha,\omega >0$. 
	Let
	\[
	b_* :=  \frac{\omega}{\alpha} - \frac{1}{2} - \epsilon  \left(\frac{2}{3}+ \frac{1}{2}\sqrt{2 + 2 |\omega-\pp| } \right).
	\]
Assume that $b_*> \sqrt{2} \epsilon$. 
Define
% \begin{equation*}%\label{e:zstar}
% 	z^{\pm}_* :=b_* \pm \sqrt{(b_*)^2- \epsilon^2 } .
% \end{equation*}
% \note[J]{Proposed change to match Lemma E.4}
\begin{equation}\label{e:zstar}
z^{\pm}_* :=b_* \pm \sqrt{(b_*)^2- 2 \epsilon^2 } .
\end{equation}
If there exists a $\tc \in \ell^1_0$ such that $\tilde{F}_\epsilon(\alpha, \omega,\tc) = 0$, then \\
\mbox{}\quad\textup{(a)} either $ \|\tc\| \leq  z_*^-$ or $ \|\tc\| \geq z_*^+  $.\\
\mbox{}\quad\textup{(b)} 
$ \| K^{-1} \tc \| \leq (2\epsilon^2+ \|\tc\|^2) / b_*$. 
\end{lemma}
\begin{proof}
	The proof follows from Lemmas~\ref{lem:gamma} and~\ref{lem:thecone} in Appendix~\ref{appendix:aprioribounds}, combined with the observation that
$\frac{\omega}{\alpha} - \gamma \geq b_*$,
% \[
%   \frac{\omega}{\alpha} - \gamma \geq b_*
%  \qquad\text{for all }
% | \alpha - \pp| \leq r_\alpha \text{ and } 
%   | \omega - \pp| \leq r_\omega.
% \]
with $\gamma$ as defined in Lemma~\ref{lem:gamma}.
\end{proof}

\begin{remark}\label{r:smalleps}
We note that for $\alpha < 2\omega$
\begin{alignat*}{1}
z^+_* &\geq   \frac{2 \omega - \alpha}{\alpha} 
- \epsilon \left(4/3+\sqrt{2 + 2 |\omega-\pp| } \, \right) + \cO(\epsilon^2)
\\[1mm]
z^-_* & \leq   \cO(\epsilon^2)
\end{alignat*}
for small $\epsilon$. 
Hence Lemma~\ref{lem:Cone} implies that for values of $(\alpha,\omega)$ near $(\pp,\pp)$ any solution has either $\|\tc\|$ of order 1 or $\|\tc\| =  \cO(\epsilon^2)$. 
The asymptotically small term bounding $z_*^-$ is explicitly calculated in Lemma~\ref{lem:ZminusBound}. 
A related consequence is that for $\epsilon=0$ there are no nontrivial solutions 
of $\tilde{F}_0(\alpha,\omega,\tc)=0$ with 
$\| \tc \| < \frac{2 \omega - \alpha}{\alpha} $. 
\end{remark}

\begin{remark}\label{r:rhobound}
In Section~\ref{s:contraction} we will work on subsets of $\ell^K_0$ of the form
\[
  \ell_\rho := \{ c \in \ell^K_0 : \|K^{-1} c\| \leq \rho \} .
\]
Part (b) of Lemma~\ref{lem:Cone} will be used in Section~\ref{s:global} to guarantee that we are not missing any solutions by considering $\ell_\rho$ (for some specific choice of $\rho$) rather than the full space $\ell^K_0$.
In particular, we infer from Remark~\ref{r:smalleps} that  small solutions (meaning roughly that $\|\tc\| \to 0$ as $\epsilon \to 0$)
satisfy $\| K^{-1} \tc \| = \cO(\epsilon^2)$.
\end{remark}

The following theorem guarantees that near the bifurcation point the problem of finding all periodic solutions is equivalent to considering the rescaled problem $F_\epsilon(\alpha,\omega,c)=0$.
\begin{theorem}
\label{thm:FourierEquivalence3}
\textup{(a)} Let $ \epsilon > 0$,  $c \in \ell^K_0$, $\alpha>0$ and $ \omega >0$. 
	Define $y: \R\to \R$ as 
\begin{equation}\label{e:yc}
	y(t) = 
	\epsilon \left( e^{i \omega t }  + e^{- i \omega t }\right) 
	+ \epsilon  \sum_{k = 2}^\infty   c_k e^{i \omega k t }  + c_k^* e^{- i \omega k t } .
\end{equation}
%	and suppose that $ y(t) > -1$. 
	Then $y(t)$ solves~\eqref{eq:Wright} if and only if $F_{\epsilon}( \alpha , \omega , c) = 0$.\\
\textup{(b)}
Let $y(t) \not\equiv 0$ be a periodic solution of~\eqref{eq:Wright} of period $2\pi/\omega$
 with Fourier coefficients $\c$.
Suppose $\alpha < 2\omega$ and $\| \c \| < \frac{2 \omega - \alpha}{\alpha} $.
Then, up to time translation, $y(t)$ is described by a Fourier series of the form~\eqref{e:yc} with $\epsilon > 0$ and $c \in \ell^K_0$.
\end{theorem}

\begin{proof}
Part (a) follows directly from Theorem~\ref{thm:FourierEquivalence2} and the  change of variables~\eqref{e:changeofvariables}.
To prove part (b) we need to exclude the possibility that there is a nontrivial solution with $\epsilon=0$. The asserted bound on the ratio of $\alpha$ and $\omega$ guarantees, by Lemma~\ref{lem:Cone} (see also Remark~\ref{r:smalleps}), that indeed $\epsilon>0$ for any nontrivial solution. 
\end{proof}

We note that in practice (see Section~\ref{s:global}) a bound on $\| \c \|$ is derived from a bound on $y$ or $y'$ using Parseval's identity.

\begin{remark}\label{r:cone}
It follows from Theorem~\ref{thm:FourierEquivalence3} and Remark~\ref{r:smalleps} that for values of $(\alpha,\omega)$ near $(\pp,\pp)$ any reasonably bounded solution satisfies $\| c\| =  O(\epsilon)$ as well as $\|K^{-1} c \| = O(\epsilon)$ asymptotically (as $\epsilon \to 0$).
These bounds will be made explicit (and non-asymptotic) for specific choices of the parameters in Section~\ref{s:global}.
\end{remark}

% We are able to rule out such large amplitude solutions using global estimates such as those in \cite{neumaier2014global}.
% Hence, near the bifurcation point, the problem of describing periodic solutions of~\eqref{eq:Wright} reduces to studying the family of zeros finding problems $F_\epsilon=0$.





%Specifically, if a solution having $ \epsilon = 0$ does in fact correspond to a nontrivial periodic solution and $\alpha  < 2\omega $, then $ \| \tilde{c} \| > 2 \omega \alpha^{-1} -1$. 
%%PERHAPS THIS NEEDS A FORMULATION AS A THEOREM AS WELL?
%%IN OTHER WORDS: ARE WE SURE WE HAVE FOUND ALL ZEROS OF $\tilde{F}_0$, I.E. ALL SOLUTIONS WITH $\epsilon=0$ NEAR THE BIFURCATION POINT? AFTER RESCALING THESE ARE INVISIBLE?
%%THERE SHOULD BE A STATEMENT ABOUT THIS SOMEWHERE! EITHER HERE OR SOME





We finish this section by defining a curve of approximate zeros $\bx_\epsilon$ of $F_\epsilon$ 
(see \cite{chow1977integral,hassard1981theory}). 
%(see \cite{chow1977integral,morris1976perturbative,hassard1981theory}). 


\begin{definition}\label{def:xepsilon}
Let
\begin{alignat*}{1}
	\balpha_\epsilon &:= \pp + \tfrac{\epsilon^2}{5} ( \tfrac{3\pi}{2} -1)  \\
	\bomega_\epsilon &:= \pp -  \tfrac{\epsilon^2}{5} \\
	\bc_\epsilon 	 &:= \left(\tfrac{2 - i}{5}\right) \epsilon \,  \e_2 \,.
\end{alignat*}
We define the approximate solution 
$ \bx_\epsilon := \left( \balpha_\epsilon , \bomega_\epsilon  , \bc_\epsilon \right)$
for all $\epsilon \geq 0$.
\end{definition}

We leave it to the reader to verify that both 
 $F_\epsilon(\pp,\pp,\bc_{\epsilon})=\cO(\epsilon^2)$ and $F_\epsilon(\bx_\epsilon)=\cO(\epsilon^2)$.
%%%	
%%%	
%%%	}{Better like this?}
%%%\annote[J]{ $F_\epsilon(\bx_0)=\cO(\epsilon^2)$ and $F_\epsilon(\bx_\epsilon)=\cO(\epsilon^2)$.}{I think we'd still need the $ \bar{c}_\epsilon$ term in $\bar{x}_0$ to be of order $ \epsilon$.}
%%%\remove[JB]{We show in Proposition A.1
%%%%\ref{prop:ApproximateSolutionWorks} 
%%% that any $ x \in \R^2 \times \ell^1_0$ which is $ \cO(\epsilon^2)$ close to $ \bar{x}_\epsilon $ will yield the estimate $F_\epsilon(x) = \cO(\epsilon^2)$.
%%%Hence choosing $\{ \pp , \pp, \bar{c}_\epsilon\}$ as our approximate solution would also have been a natural choice for performing an $\cO(\epsilon^2)$ analysis and would have simplified several of our calculations.
%%%However,} 
%%%
We choose to use the more accurate approximation 
for the $ \alpha$ and $ \omega $ components to improve our final quantitative results. 














%
% Values for $ (\alpha, \omega,c)$ which approximately solve $\tilde{F}(\alpha,\omega,c) = 0$  are computed in  \cite{chow1977integral,morris1976perturbative,hassard1981theory} and are as follows:
%  \begin{eqnarray}
%  \tilde{\alpha}( \epsilon) &:=& \pi /2 + \tfrac{\epsilon^2}{5} ( \tfrac{3\pi}{2} -1) \nonumber \\
%  \tilde{\omega}( \epsilon) &:=& \pi /2 -  \tfrac{\epsilon^2}{5} \label{eq:ScaleApprox} \\
%  \tc(\epsilon) 	  &:=& \{ \left(\tfrac{2 - i}{5}\right)  \epsilon^2 , 0,0, \dots \} \nonumber
%  \end{eqnarray}
% In Appendix \ref{sec:OperatorNorms} we illustrate an alternative method for deriving this approximation.
%
%
%
%
% We want to solve $ \tilde{F}(\alpha , \omega, \hat{c}) =0$ for small values of $ \epsilon$.
% However $ D \tilde{F}(\alpha , \omega , c)$ is not invertible at $ ( \pp , \pp , 0)$ when $ \epsilon = 0$.
% In order for our asymptotic analysis to be non-degenerate, we need to make the change of variables $ c \mapsto \epsilon c$.
% Under this change of variables, we define the function $ F$ below so that $ \tilde{F}(\alpha , \omega , \epsilon c) =\epsilon  F( \alpha , \omega , c)$.
%
%
%
% \begin{definition}
% Construct an $\epsilon$-parameterized family of densely defined functions  $F : \R^2 \oplus \ell^1 / \C \to \ell^1$ by:
% \begin{equation}
% \label{eq:FDefinition}
% 	F(\alpha,\omega, c) :=
% 	[i \omega + \alpha e^{-i \omega}]_1 +
% 	( i \omega K^{-1} + \alpha U_{\omega}) c +
% 	[\epsilon \alpha e^{-i \omega}]_2  +
% 	\alpha \epsilon L_\omega c +
% 	\alpha \epsilon [ U_{\omega} c] * c.
% \end{equation}
% \end{definition}

%%
%%
%%\begin{corollary}
%%	\label{thm:FourierEquivalence3}
%%	Fix $ \epsilon > 0$, and $ c \in \ell^1 / \C $, and $ \omega >0$. Define $y: \R\to \R$ as 
%%	\[
%%	y(t) = 
%%	\epsilon \left( e^{i \omega t }  + e^{- i \omega t }\right) 
%%	+  \epsilon  \left( \sum_{k = 2}^\infty   c_k e^{i \omega k t }  + \overline{c}_k e^{- i \omega k t } \right) 
%%	\]
%%	and suppose that $ y(t) > -1$. 
%%	Then $y(t)$ solves Wright's equation at parameter $ \alpha > 0 $ if and only if $ F( \alpha , \omega , c) = 0$ at parameter $ \epsilon$. 
%%	
%%	
%%	
%%\end{corollary}
%%
%%
%%\begin{proof}
%%	Since $ \tilde{F}(\alpha,\omega, \epsilon c) = \epsilon F( \alpha , \omega , c)$, the result follows from Theorem \ref{thm:FourierEquivalence2}.
%%\end{proof}

% If we can find $(\alpha , \omega, c)$ for which $ F( \alpha , \omega,c)=0$ at parameter $\epsilon$, then $ \tilde{F}(\alpha ,\omega, c)=0$.
% By Theorem \ref{thm:FourierEquivalence2} this amounts to finding a periodic solution to Wright's equation.
% Lastly, because we have performed the change of variables $ c \mapsto \epsilon c$, we need to  apply this change of variables to our approximate solution as well.
%
% \begin{definition}
% 	Define the approximate solution $ x( \epsilon) = \left\{ \alpha(\epsilon ) , \omega ( \epsilon ) , c(\epsilon) \right\}$ as below,  where $c(\epsilon) = \{ c_2( \epsilon) , 0 ,0 , \dots\} $.
% 	We may also write $ x_\epsilon = x(\epsilon) $.
% 	\begin{eqnarray}
% 	\alpha( \epsilon) &:=& \pi /2 + \tfrac{\epsilon^2}{5} ( \tfrac{3\pi}{2} -1) \nonumber \\
% 	\omega( \epsilon) &:=& \pi /2 -  \tfrac{\epsilon^2}{5} \label{eq:Approx} \\
% 	c_2(\epsilon) 	  &:=& \left(\tfrac{2 - i}{5}\right) \epsilon \nonumber
% 	\end{eqnarray}
%
% \end{definition}



\section{Sparse and Bounded Regular Languages}
\label{sec:sparse}

Here, in Theorem~\ref{thm:sparse_in_NP}, we establish that for constraint languages from the class of sparse regular languages, which equals the class of the bounded regular languages~\cite{DBLP:journals/eik/LatteuxT84}, the constrained problem
is always in \NP.


A language $L \subseteq \Sigma^*$ is \emph{sparse},
%\cite{DBLP:journals/siamcomp/BermanH77,DBLP:journals/ijfcs/GawrychowskiKRS10,DBLP:journals/ipl/Hartmanis83,DBLP:series/wsscs/Hartmanis93b,DBLP:journals/iandc/HartmanisIS85,DBLP:conf/mfcs/HartmanisM80,DBLP:conf/focs/Mahaney80,DBLP:journals/jcss/Mahaney82,Pin2020,DBLP:conf/mfcs/SzilardYZS92,DBLP:reference/hfl/Yu97}
if there exists $c \ge 0$
such that, for every $n \ge 0$, we have
$L \cap \Sigma^n \in O(n^c)$.
Sparse languages were introduced into computational complexity
theory by Berman \& Hartmanis~\cite{DBLP:journals/siamcomp/BermanH77}.
Later, it was established by Mahaney that if there exists
a sparse \NP-complete set (under polynomial-time many-one reductions),
then $\PTIME = \NP$~\cite{DBLP:journals/jcss/Mahaney82}.
For a survey on the relevance of sparse sets in computational complexity theory, see~\cite{DBLP:conf/mfcs/HartmanisM80}.


A language $L \subseteq \Sigma^*$ is called \emph{bounded},
if there exist $w_1, \ldots, w_k \in \Sigma^*$
such that $L \subseteq w_1^* \ldots w_k^*$. %~\cite{GinsburgSpanier66,DBLP:journals/eik/LatteuxT84}.
Bounded languages were introduced by Ginsburg \& Spanier~\cite{GinsburgSpanier64}.

We will need the following representation of the bounded regular languages.

% noch begründen, warum bounded überhaupt erwähnt wird, und nicht direkt 
% nur mit dem begriff sparse gearbeitet wird todo -> wegen klasse strictly bounded

\begin{theorem}[\cite{GinsburgSpanier66}]
\label{thm:bounded_regular_form}
 A language $L \subseteq w_1^* \cdots w_k^*$ is regular if and only if
 it is a finite union of languages of the form $L_1 \cdots L_k$, where each $L_i \subseteq w_i^*$ is regular.
\end{theorem}

It is known that the class of sparse regular languages equals
the class of bounded regular languages~\cite{DBLP:journals/eik/LatteuxT84},
or see~\cite{Pin2020,DBLP:reference/hfl/Yu97}, where the bounded languages are not mentioned
but the equivalence is implied by their results and Theorem~\ref{thm:bounded_regular_form}.
The next results links this class to the polycylic PDFAs.

\begin{propositionrep}
\label{thm:bounded_characterization}
 %For regular $L \subseteq \Sigma^*$, the following are equivalent:
 %(1) $L$ is sparse, (2) $L$ is bounded, (3) $L$ is recognizable by a polycyclic PDFA.
 Let $L \subseteq \Sigma^*$ be regular. Then, $L$ is sparse
 if and only if it is recognizable by a polycyclic PDFA.
\end{propositionrep}
\begin{proof}
 In~\cite{DBLP:journals/eik/LatteuxT84} is was shown that the context-free sparse languages are precisely the context-free bounded languages, which
 gives our first two equivalences.
 A result from~\cite[Lemma 2]{DBLP:journals/ijfcs/GawrychowskiKRS10}
 readily implies that if a language is recognized by a polycyclic PDFA, then
 it must be sparse.
 Lastly, we show that every bounded regular language
 is recognizable by a polycyclic automaton, which finishes the proof.
 
 \medskip 
 
 \noindent\underline{Claim:} For $w \in \Sigma^*$.
  Then, any regular $L \subseteq w^*$
  is recognizable by a polycyclic PDFA.
 \begin{quote}
     \emph{Proof of the Claim.}
      Let $w \in \Sigma^*$ and $L \subseteq w^*$ be a regular language.
 If $w = \varepsilon$, then $L = \{\varepsilon\}$, which is obviously recognizable
 by a polycyclic automaton. So, suppose $|w| > 0$.
%  Suppose $\mathcal A = (\Sigma, Q, \delta, q_0, F)$
%  is an accepting PDFA for $L$ such that every state is accessible
%  and coacessible, which we can assume by Lemma~\ref{lem:accessible_coaccessible}.
%  Choose any $q \in Q$. Then, there exists $u_1, u_2 \in \Sigma^*$
%  such that $\delta(q_0, u_1) = q$
%  and $\delta(q, u_2) \in F$.
%  Hence, if $\delta(q, v) = q$,
%  then $u_1 v u_2 \in L$, so that $u_1 v u_2 \subseteq w^*$.
%  If $v = \varepsilon$, then $v\subseteq w^*$.
%  So, suppose $|v| > 0$.
%  Then $u_1 v u_2 = w^n$ for some $n > 0$.
%  But then, $v = v_1 w^k v_2$
%  for some maximal $k \ge 0$.
%
%  Every accepting automaton for a language $L \subseteq w^*$, for some $w \in \Sigma^*$, is obviously polycyclic.
%
% Todo, inverse hom, single final acceptable?
% Applying Theorem~\ref{thm:bounded_regular_form},
% we can write $L = L_1 \cup \ldots \cup L_n$
% such that, for any $i \in \{1,\ldots,n\}$, $L_i \subseteq w^*$ is regular.
 Let $a$ be an arbitrary symbol
 and define a homomorphism $\varphi : \{a\}^* \to \Sigma^*$
 by $\varphi(a^i) = w^i$, which is injective as $|w| > 0$ by assumption.
 Then, the unary language $\varphi^{-1}(L) = \{ a^i \mid w^i \in L\}$ is regular, as inverse
 homomorphisms preserve regularity.
 Hence, we can write it as a union of languages recognizable by automata
 with a single final state, which, by Lemma~\ref{lem::unary_single_final},
 have the form $\{ a^i \}$ for some $i \ge 0$
 or $\{ a^{i + jp} \mid j \ge 0 \}$ for some $i \ge 0, p > 0$.
 As the application of functions preserves union, and $L = \varphi(\varphi^{-1}(L))$ here,
 the language 
 $L$ is the union of the images of these languages.
 We have $\varphi(\{a^i \}) = \{ w^i \}$, and this singleton language
 is obviously recognizable by a polycyclic automaton,
 and we have $\varphi(\{ a^{i + jp} \mid j \ge 0 \}) = \{ w^{i+pj} \mid j \ge 0 \}$,
 and this language is also recognizable by an automaton that
 has an initial tail labelled by $w^i$ and a cycle labelled by $w^p$.
 So, as the polycyclic languages are closed under union~\cite[Proposition 6]{DBLP:conf/ictcs/Hoffmann20},
 we have shown that the language $L$ %languages $L_i, i \in \{1,\ldots,n\}$,
 is recognizable by some polycyclic automaton.     \emph{[End, Proof of the Claim]}
 \end{quote}
 Finally, as the languages recognizable by polycyclic automata
 are closed under concatenation and union~\cite[Proposition 5 and Proposition 6]{DBLP:conf/ictcs/Hoffmann20},
 by Theorem~\ref{thm:bounded_regular_form} every bounded regular language is recognizable by a polycyclic automaton.
\end{proof}

In~\cite[Theorem 2]{DBLP:conf/ictcs/Hoffmann20} it was shown that for polycyclic
constraint languages, the constrained problem is always in $\NP$.
So, we can deduce the next result.

\begin{theorem}
\label{thm:sparse_in_NP}
 If $L \subseteq \Sigma^*$ is sparse and regular, then $L\textsc{-Constr-Sync} \in \NP$.
\end{theorem}

We will need the following closure property stated in~\cite[Theorem 3.8]{DBLP:reference/hfl/Yu97}
of the sparse regular languages.

\begin{proposition}
 %The sparse regular languages are closed under morphisms.%homomorphic mappings.
 The class of sparse regular languages is closed under homomorphisms.
\end{proposition}

\begin{toappendix}
Note that sparse languages in general are not closed
under homomorphic mappings~\cite{Pin2020}.
As it is easy to see that the bounded languages are closed
under homomorphic mappings, this also implies that, in general,
the bounded languages do not equal the sparse languages.
\end{toappendix}


Note that the connection of the polycyclic languages to the sparse or bounded languages
was not noted in~\cite{DBLP:conf/ictcs/Hoffmann20}. However, a condition
characterizing the sparse regular languages
in terms of forbidden patterns was given in~\cite{Pin2020}, and
it was remarked that ``a minimal deterministic automaton recognises a sparse language if and only if it
does not contain two cycles reachable from one another''.
This is quite close to our characterization.
%and, probably, the author
%has had a similar intuition as spelled out with our explicit definition
%of a polycyclic automaton.




\section{Letter-Bounded Constraint Languages}
\label{sec:strictly_bounded_case}



% % begriff einführen, das man annehemn kann "nachbarn" verschieden
% % dann hom

% % L' = \{ b_1^n ... : a^n b^{n_\} ist das eindeutig?

% \begin{proposition}
%  hom verschieden, eine richtung klar
% \end{proposition}
% \begin{proof} % Gamma = {b1,..,bk}
% Let $\mathcal A = (\Gamma, Q, \delta)$ be an input automaton
% for which we want to know if it has a synchronizing word in $U$. %L' über Gamma
% Set $Q' = Q \times \{ 1, \ldots, k \}$
% and $\delta' : Q' \times \Sigma \to Q'$
% with 
% \[
%  \delta'((q, i), x) = \left\{
%  \begin{array}{ll}
%   (\delta(q, x), j)     & \mbox{if } x = \varphi(b_j); \\
%  % (\delta(q, x), i + 1) & \mbox{if } i < k \mbox{ and } x = \varphi(b_{i+1}); \\
%   (q, i)                & \mbox{otherwise.}
%  \end{array}\right.
% \]
% Then, $\mathcal A' = (\Sigma, Q', \delta')$
% has a synchronizing word in $L$
% if and only if $\mathcal A$ has a synchronizing word in $U$.
% % aber man muss jeweils mind einmal "rübergehen"




% \end{proof}



%  Let $L \subseteq w_1^* \cdots w_n^*$.
%  As $w^* w^* = w^*$, we can suppose that $w_i \ne w_{i+1}$
%  for $i \in \{1,\ldots,n-1\}$, which we will assume for the rest of this section.
%  In particular in the strictly bounded case we assume
%  that consecutive letter are distinct.\todo{wo brauche ich diese annahmen genau?}
 
 
 %In this section, we 
 Fix a constraint automaton $\mathcal B = (\Sigma, P, \mu, p_0, F)$.
 Let $a_1, \ldots, a_k \in \Sigma$ be a sequence of (not necessarily distinct)
 letters.
 In this section, we assume $L(\mathcal B) \subseteq a_1^* \cdots a_k^*$.
 A language which fulfills the above condition
 is called \emph{letter-bounded}.
 Note that the language $ab^*a$ given in the introduction as an example
 %, and in~\cite{DBLP:conf/mfcs/FernauGHHVW19} as the smallest, in terms of recognizing automata,
 %constraint language giving an \NP-complete problem,
 is letter-bounded. In fact, it is the language with the smallest
 recognizing automaton yielding an \NP-complete constrained problem~\cite{DBLP:conf/mfcs/FernauGHHVW19}.
 
 A language such that the $a_i$ are pairwise distinct, i.e., $a_i \ne a_j$
 for $i \ne j$, is called \emph{strictly bounded}.
 The class of strictly bounded languages has been extensively studied~\cite{DBLP:journals/mst/BlattnerC77,DBLP:journals/dam/DassowP99,Ginsburg66,GinsburgSpanier64,GinsburgSpanier66,HerrmannKMW17},
 where in~\cite{Ginsburg66,GinsburgSpanier64,GinsburgSpanier66} no name was introduced for them
 and in~\cite{HerrmannKMW17} they were called strongly bounded.
 %\footnote{The work~\cite{HerrmannKMW17}
 %seems to deviate from the standard terminology in other ways too by calling bounded languages
 %as introduced here word-bounded and refers to letter-bounded simply as bounded languages.}.
 The class of letter-bounded languages properly contains the strictly bounded languages.
 
 \begin{toappendix} 
 Note that the work~\cite{HerrmannKMW17}
 seems to deviate from the standard terminology, for example by calling bounded languages
 as introduced here word-bounded and refers to letter-bounded simply as bounded languages.
 \end{toappendix}
 

 
 \begin{remark}%[Motivation]%[A Motivation for Strictly Bounded Constraint Languages]
 \label{rem:motivation_strictly_bounded}
Let $\Sigma = \{b_1, \ldots, b_r\}$
be an alphabet of size $r$.
%and $\psi : \Sigma^* \to \mathbb N_0^m$
%be the \emph{Parikh morphism}
%given by $\psi(w) = (|w|_{b_1}, \ldots, |w|_{b_k})$
%for $w \in \Sigma^*$.
%Then, for $L \subseteq \Sigma$,
%set $\perm(L) = \{ w \in \Sigma^* \mid \exists u \in L \forall a \in \Sigma : |u|_a = |w|_a \}$,
%the \emph{commutative closure} of $L$.
Then, 
%between the commutative languages over $\Sigma$
%and the strictly bounded languages in $b_1^* \cdots b_k^*$, 
the mappings
\[
\Phi(L) = L \cap b_1^* \cdots b_r^* \mbox{ and }
\perm(L) = \{ w \in \Sigma^* \mid \exists u \in L\  \forall a \in \Sigma : |u|_a = |w|_a \}
\]
for $L \subseteq \Sigma^*$ are mutually inverse and inclusion preserving
between the languages in $b_1^* \cdots b_r^*$ and the commutative languages
in $\Sigma^*$, where a language $L \subseteq \Sigma^*$ is commutative 
if $\perm(L) = L$.
Furthermore, for strictly bounded languages of the form $B_1 \cdots B_r \subseteq b_1^* \cdots b_r^*$
with $B_j \subseteq \{b_j\}^*$, $j \in \{1,\ldots, r\}$, we have
$
 \perm(B_1 \cdots B_r) = B_1 \shuffle \cdots \shuffle B_r,
$
where
$U \shuffle V = \{ u_1 v_1 \cdots u_n v_n \mid u_i, v_i \in \Sigma^*, u_1 \cdots u_n \in U, v_1 \cdots v_n \in  V \}$ for $U, V \subseteq \Sigma^*$.
Hence, $\perm(L)$ is regularity-preserving
for strictly bounded languages.
More specifically, the above correspondence between
the two language classes is regularity-preserving in both directions.
For commutative constraint languages, a classification
of the complexity landscape has been achieved~\cite{DBLP:conf/cocoon/Hoffmann20}.
By the close relationship between commutative and certain strictly
bounded languages, it is natural to tackle
this language class next.
However, as shown in~\cite{DBLP:conf/cocoon/Hoffmann20},
for commutative constraint languages, we can realize $\PSPACE$-complete
problems, but, by Theorem~\ref{thm:sparse_in_NP},
for strictly bounded languages, the constrained problem is always in $\NP$.
However, by the above relations, Theorem~\ref{thm:bounded_regular_form} for languages in $b_1^* \cdots b_r^*$ is equivalent to~\cite[Theorem 5]{DBLP:conf/cocoon/Hoffmann20}, a representation result
for commutative regular languages.
\end{remark}
 
\begin{comment}

Next, we link the representation
of bounded languages given in Theorem~\ref{thm:bounded_regular_form}
to the languages $L_{j_1, j_2, j_3}$ defined in the beginning of
this section.

\begin{lemma}
\label{lem:L_j1j2j3_intersection}
 Let $L(\mathcal B) = \bigcup_{i=1}^n A_1^{(i)} \cdots A_k^{(i)}$
 with unary regular languages $A_j^{(i)} \subseteq \{a_j\}^*$.
 Then,
 $
  L(\mathcal B) \cap L_{j_1, j_2, j_3} \ne \emptyset 
 $
 if and only if we can find $i_0 \in \{1,\ldots,n\}$
 such that $A_{j_1}^{(i_0)}$ and $A_{j_3}^{(i_0)}$
 do not equal~$\{\varepsilon\}$
 and $A_{j_2}^{(i_0)}$ is infinite.
\end{lemma}
\begin{proof}
 First, suppose we have
   some $w \in L(\mathcal B) \cap L_{j_1, j_2, j_3}$.
   %Then, for some $i_0 \in \{1,\ldots, n\}$, we have
   %$w \in A_1^{(i_0)} \cdots A_k^{(i_0)} \cap L_{j_1, j_2, j_3}$.
   As $L(\mathcal B) \subseteq a_1^* \cdots a_k^*$, we have
   \[
    w = a_1^{|w|_{a_1}} \cdots a_k^{|w|_{a_k}}
   \]
   with $|w|_{a_{j_1}} > 0$, $|w|_{a_{j_3}} > 0$
   and $|w|_{a_{j_2}} \ge |P|$.
   By the pigeonhole principle, as $w$ is read in $\mathcal B$,
      it has to traverse some state twice as it reads 
      the factor $a_{j_2}^{|w|_{a_{j_2}}}$. So, we can pump
      some non-empty factor $a_{j_2}^p$ of it with $0 < p \le |P|$.
      Hence, writing $w = ua_{j_2}^{|w|_{a_{j_2}}}v$ with $u,v \in \Sigma^*$,
      we have, for any $r \ge 0$,
      \[
             ua_{j_2}^{|w|_{a_{j_2}} + rp}v \subseteq L(\mathcal B).
      \] % set u ,v
   Again, using the pigeonhole principle, as we have a 
   finite union \[ L(\mathcal B) = \bigcup_{i=1}^n A_1^{(i)} \cdots A_k^{(i)}, \]
   there
   exists $i_0 \in \{1,\ldots, n\}$ such that
   \[
    ua_{j_2}^{|w|_{a_{j_2}} + rp}v \in A_1^{(i_0)} \cdots A_k^{(i_0)}
   \]
   for infinitely many $r \ge 0$. 
   \todo{Ne, das geht nur bei strictly bounded.}
   This implies $a_{j_2}^{|w|_{a_{j_2}} + rp} \subseteq A_{j_2}^{(i_0)}$
   for infinitely many $r$. Hence $A_{j_2}^{(i_0)}$
   is infinite. Furthermore, we
   get $a_{j_1}^{|w|_{a_{j_1}}} \in A_{j_1}^{(i_0)}$
   and $a_{j_3}^{|w|_{a_{j_3}}} \in A_{j_3}^{(i_0)}$.
  
  
   Conversely, if we have $i_0 \in \{1,\ldots,n\}$
   such that $A_{j_1}^{(i_0)}$ and $A_{j_3}^{(i_0)}$
   do not equal $\{\varepsilon\}$ and $A_{j_2}^{(i_0)}$
   is infinite, then obviously
   \[
    A_1^{(i_0)} \cdots A_k^{(i_0)} \cap L_{j_1,j_2,j_3} \ne \emptyset 
   \]
   and as $A_1^{(i_0)} \cdots A_k^{(i_0)} \subseteq L(\mathcal B)$
   the claim follows.
   
   
   \medskip 
   
   % alternativ mit SCCs und Pfaden, zeigen dass die A_j^i durch Automaten mit weniger als
   % |P| Zuständen erkennbar.
   \qed
\end{proof}
\end{comment}

 Our first result says, intuitively,  
 that if in $A_1 \cdots A_k$ with $A_j$ unary and regular,
 if no infinite unary language $A_j$ over $\{a_j\}$ lies %strictly in the middle 
 between 
 non-empty unary languages
 over a distinct letter\footnote{Hence different from $\{\varepsilon\}$, as $\{\varepsilon\} \subseteq \{a\}^*$
 for $a \in \Sigma$.}  than $a_j$, %than the infinite unary language,
 then $(A_1 \cdots A_k)$\textsc{-Constr-Sync} is in~$\PTIME$.
 
%  Later, in Lemma~\ref{lem:np_hardness} and Theorem~\ref{thm:dichotomy}, we will use
%  the languages $L_{j_1, j_2, j_3}$. They
%  allow us to single out which letters appear infinitely
%  often between other letters in~$L(\mathcal B)$, i.e.,
%  express the opposite of the condition mentioned in Proposition~\ref{prop:stricly_bounded_P}.
 
\begin{propositionrep}
\label{prop:stricly_bounded_P}
 Let $A_j \subseteq \{a_j\}^*$ be unary regular languages
 %, recognized
 %by automata with a single final state, 
 for $j \in \{1,\ldots, k\}$.
 Set $L = A_1 \cdot\ldots\cdot A_k$.
 If for all $j \in \{1,\ldots, k\}$, $A_j$ infinite implies that $A_i \subseteq \{a_j\}^*$
 for all $i < j$ or $A_i \subseteq \{a_j\}^*$ for all $i > j$ (or both), then $L\textsc{-Constr-Sync} \in \PTIME$. 
\end{propositionrep} 
\begin{proof}
 Let $L = A_1 \cdots A_k$ with $A_j \subseteq \{a_j\}^*$ fulfill the assumption.
 If $A_j$ is infinite and for all $i < j$ we have $A_i \subseteq \{a_j\}^*$,
 then $A_1 \cdots A_j \subseteq \{a_j\}^*$, and similarly if
 for all $i > j$ we have $A_i \subseteq \{a_j\}^*$.
 So, by considering $(A_1 \cdots A_j) A_{j+1} \cdots A_k$
 or $A_1 \cdots A_{j-1} (A_j \cdots A_k)$, with $j$ maximal in the former case and minimal in the latter,
 without loss of generality, we can assume $j = 1$
 or $j = k$, i.e., we only have the cases $A_1$
 is infinite, $A_k$ is infinite or both are infinite or none is infinite,
 and, by maximality or minimality of $j$,
 in all these cases the languages $A_2, \ldots, A_{k-1}$ are all finite.
 
 
 
 
 Then, by Lemma~\ref{lem:union_single_final_state}, we can write $A_1$ and $A_k$
 as a finite union of unary languages recognizable by automata with a single final state.
 As concatenation distributes over union, if we do this for
 $A_1$ and $A_k$ and rewrite the language using the mentioned distributivity,
 we get a finite union of languages of the form
 \[
  A_1' A_2 \cdots A_{k_1} A_k'
 \]
 where $A_1'$ and $A_k'$ are recognizable by unary automata with a single final state
 and are either finite or infinite. Hence, by Lemma~\ref{lem:union},
 if we can show that the problem is in $\PTIME$ for each such language,
 the result follows. 
 So, without loss of generality, we assume
 from the start that $A_1$ or $A_k$ are recognizable by automata
 with a single final state.
 
 
 If all $A_j$, $j \in \{1,\ldots,k\}$ are finite, then $L$ is finite, and $L\textsc{-Constr-Sync}\in \PTIME$
 by Lemma~\ref{lem:finite}.
 We handle the remaining cases separately.
 
 \begin{enumerate}
 \item[(i)] Only $A_1$ is infinite.
  
  By assumption, every $A_j \subseteq \{a_j\}^*$, $j \in \{1,\ldots,k\}$,
  is recognizable by a single state automaton. Hence, by Lemma~\ref{lem::unary_single_final}, we can write, as $A_1$ is infinite,
   $A_1 = a_1^i (a_1^p)^*$ with $i \ge 0$ and $p > 0$.
  Let $\mathcal A = (\Sigma, Q, \delta)$ be an input semi-automaton
  for $L\textsc{-Constr-Sync}$.
  As $\delta(Q, a_1) \subseteq Q$,
  we have, for any $n \ge 0$, $\delta(Q, a_1^{n+1}) \subseteq \delta(Q, a_1^n)$.
  So, as $Q$ is finite and the sequence of subsets
  cannot get arbitarily small, for some $0 \le n < |Q|$
  we have $|\delta(Q, a_1^{n+1})| = |\delta(Q, a_1^n)|$.
  But $|\delta(Q, a_1^{n+1})| = |\delta(Q, a_1^n)|$,
  as $\delta(Q, a_1^{n+1}) \subseteq \delta(Q, a_1^n)$,
  implies $\delta(Q, a_1^{n+1}) = \delta(Q, a_1^n)$.
  Then, the symbol $a_1$
  permutes the set $\delta(Q, a_1^n)$.
  Hence, $\delta(Q, a_1^{n+m}) = \delta(Q, a_1^n)$ for any $m \ge 0$.
  So, combining these observations,
  \begin{equation}\label{eqn:case_one_P}
   \{ \delta(Q, a_1^n) \mid n \ge 0 \} = \{ \delta(Q, a_1^n) \mid n \in \{0,\ldots, |Q|-1\} \}
  \end{equation}
  and $\delta(Q, a_1^{|Q| - 1 + m}) = \delta(Q, a_1^{|Q|-1})$
  for any $m \ge 0$. 
  Now, note that the language $A_2 \cdots A_k$
  is finite. So, to find out if we have any
  $a_1^{i + lp} u$ with $u \in A_2 \cdots A_k$
  that synchronizes the input semi-automaton,
  we only have to test if any of the words
  $
   a_i^{i + lp} u,
  $
  with $u \in A_2 \cdots A_k$
  and $l$ such that $i + lp \le \max\{|Q|-1 + p, i\}$,
  synchronizes $\mathcal A$.
  The number (and the length) of these words is linear bounded in $|Q|$ 
  and each could be checked in polynomial time by 
  feeding it into the input semi-automaton for each state and checking
  if a unique state results.
  Hence the problem is solvable in polynomial time.
   
 \item[(ii)] Only $A_k$ is infinite.
 
  Let $u \in A_1 \cdots A_{k-1}$. By assumption, there are only finitely many such
  words $u$. Set $S = \delta(Q, u)$ and $T = \delta(Q, a_k^{|Q|-1})$.
  As in case (i), $a_k$ permutes the states in $T$
  and as $S \subseteq Q$, we have  $\delta(S, a_k^{|Q|- 1}) \subseteq T$.
  So, as $a_k$ permutes $T$, it acts injective on the
  subset $\delta(S, a_k^{|Q|- 1})$.
  This gives $|\delta(S, a_k^{|Q|- 1 + n})| = |\delta(S, a_k^{|Q|- 1})|$
  for any $n \ge 0$. Together with $|\delta(S, a_k^{n + 1})| \le |\delta(S, a_k^{n})|$,
  we have
  \begin{equation}\label{eqn:case_two_P}
   \exists n \in \mathbb N_0 : |\delta(S, a_k^{n})| = 1 \Leftrightarrow |\delta(S, a_k^{|Q|- 1})| = 1.
  \end{equation}
%   If $|S| = 1$, then $u$ is already a synchronizing word and so any
%   word in $u \cdot A_k$ is also synchronizing by Lemma~\ref{lem:append_sync}.
%   So, suppose $|S| > 1$.
%   If $|\delta(S, a_k^n)| = 1$ for any $n$, then there exists
%   some $m \le |Q| - 1$ such that $|\delta(S, a_k^m)| = 1$
%   and $|\delta(S, a_k^{|Q| - 1})| = |\delta(S, a_k^{|Q| - 1 + n})|$
%   for any\footnote{But here, the sets might not be equal, for example, consider
%   some, but not all, states taken from a cycle for $a_k$} $n \ge 0$.
%   The last property is implied as $\delta(S, a_k^{|Q|- 1}) \subseteq T$,
%   and it implies the former, for if we have not mapped $S$ to a singleton
%   before reading $a_k$ at most $|Q|-1$ times, we will never do behind that point. 
%   Similarly, as in case (i), write $A_k = a_k^i(a_k^p)^*$.
  Choose any fixed $N \ge |Q| - 1$ with $a_k^N \in A_k$.
  Then, with the above considerations, we only have to test the finite
  number of words
  \[
   u\cdot a_k^{N}, \quad u \in A_1 \cdots A_{k-1}.
  \]
  The length of these words is linear bounded in $|Q|$ and 
  as each test, i.e., feeding the word into the input semi-automaton
  for each state and testing if a unique state results,
  could be performed in polynomial time, the problem is solvable in polynomial time.
  
 \item[(iii)] Both $A_1$ and $A_k$ are infinite.
  
  This is essentially a combination of the arguments of case (i) and (ii).
  Let $\mathcal A = (\Sigma, Q, \delta)$ be an input semi-automaton
  and $\mathcal B = (\Sigma, P, \mu, p_0, F)$
  be a constraint automaton with $L = L(\mathcal B)$.
  First, consider only the language $A_1 \cdots A_{k-1}$.
  Then, as in case (i), see Equation~\eqref{eqn:case_one_P},% todo argument geht auch für A_1 infinite, ohne dass unitär...
  \[
   \{ \delta(Q, a_1^n) \mid a_1^n \in A_1  \}
    = \{ \delta(Q, a_1^n) \mid 0 \le n < |Q| - 1 + |P|\mbox{ and } a_1^n \in A_1 \}.
  \]
  Note that we have written $0 \le n < |Q| - 1 + |P|$
  and not merely $\le |Q| - 1$ as an upper bound.
  The reason is that otherwise, if $a_1^{|Q|-1} \notin A_1$,
  we might miss the set $\delta(Q, a_1^{|Q|-1})$,
  but as $\delta(Q, a_1^{|Q|-1+m}) = \delta(Q, a_1^{|Q|-1})$
  for any $m \ge 0$ and $A_1$ is infinite, $\delta(Q, a_1^{|Q|-1}) \in \{ \delta(Q, a_1^n) \mid a_1^n \in A_1  \}$.
  However, if $a_1^n \in A_1$ for some $n \ge |Q| - 1 + |P|$,
  then, with $s = \mu(p_0, a_1^{|Q| - 1})$,
  by finiteness of $P$, among
  the states $s, \mu(s,a_1), \ldots, \mu(s, a_1^{n - |Q| + 1})$
  we find $0 \le m \le |P| - 1$ and $0 < r \le |P|$ with $m + r \le |P|$
  such that $\mu(s, a_1^{m+r}) = \mu(s, a_1^m)$.
  Then we have found a cycle and we can skip it, i.e.,
  \begin{align*}
   \mu(p_0, a_1^n) & = \mu(s, a_1^{n - |Q| + 1}) 
                     = \mu(\mu(s, a_1^{m+r}), a_1^{n - |Q| + 1 - (m+r)}) \\
                   & = \mu(\mu(s, a_1^m), a_1^{n - |Q| + 1 - (m+r)}) \\
                   & = \mu(s, a_1^{m + n - |Q| + 1 - m - r}) \\
                   & = \mu(p_0, a_1^{n-r}).
  \end{align*}
  But, as then $\mu(s, a_1^{n - r}) = \mu(s, a_1^n) \in F$
  we find $a_1^{n-r} \in A_1$. 
  Repeating this procedure, if $n - r \ge |Q| - 1 + |P|$,
  we ultimately find $|Q| - 1 \le m < |Q| - 1 + |P|$
  such that $a_1^m \in A_1$
  and $\delta(Q, a_1^{|Q| - 1}) = \delta(Q, a_1^m)$.
  Note that the language $A_2 \cdots A_{k-1}$
  is finite.
  Then, as in case (i), 
  we only have to consider the  words,
  whose length and number is linear bounded in $|Q|$,
  \[
   a_1^n \cdot u,\quad  0 \le n < |Q| - 1 + |P|, a_1^n \in A_1, u \in A_2 \cdots A_{k-1}
  \]
  and the corresponding sets
  \[
   S = \delta(Q, a_1^n \cdot u),
  \]
  and these are all possible sets in $\{ \delta(Q, a_1^n u) \mid a_1^n \in A_1, u \in A_2 \cdots A_{k-1} \}$.
  Fix any such subset $S$.
  Then, as in case (ii) and Equation~\eqref{eqn:case_two_P}, 
  choose any $N \ge |Q| - 1$ with $a_k^N \in A_k$ and
  we only have to compute $\delta(S, a_k^N)$
  and test if it is a singleton set.
  So, in total, we only have to test the words
  \[
   a_1^n \cdot u a_k^N, 0 \le n < |Q| - 1 + |P|, a_1^n \in A_1, u \in A_2 \cdots A_{k-1}.
  \]
  Their length and number is linear bounded in $|Q|$
  and computing the reachable state from each state of the input automaton,
  and testing if a unique state results, could be performed in polynomial
  time. Hence, the overall procedure could be performed in polynomial time.
 \end{enumerate}
 So, we have handled every case and the proof is complete.\qed
 % oder zeigen A_1 oder A_k finite -> nur endlich viele wörter müssen getestet werden
 % mit (iii)
\end{proof}

Now, we state a sufficient condition for \NP-hardness over binary alphabets. 
This condition, together with Proposition~\ref{prop:hom_lower_bound_complexity},
allows us to handle the general case in Theorem~\ref{thm:dichotomy}.
Its application together with Proposition~\ref{prop:hom_lower_bound_complexity} shows, in some respect, that the language $ab^*a$
is the prototypical language giving \NP-hardness.
We give a proof sketch of Lemma~\ref{lem:np_hardness} at the end of this section.

\begin{toappendix}

In the proof of Lemma~\ref{lem:np_hardness}
we will need the following two lemmata.
For $n > 0$, set
\[
 L_n = (\Sigma^* a \Sigma^* b^{|P|} \Sigma^*)^n.
\]
Recall that $\mathcal B = (\Sigma, P, \mu, p_0, F)$.

\begin{lemma}
\label{lem:number_of_b_blocks}
 Let $\Sigma = \{a,b\}$ and $L(\mathcal B) \subseteq a_1^* \cdots a_k^*$
 with $a_i \in \Sigma $ and $n > 0$.
 Then, the following are equivalent:
 \begin{enumerate} 
 \item $L(\mathcal B) \cap L_n \ne \emptyset$,
 
 \item there exist $u_0, \ldots, u_n \in \Sigma^* a \Sigma^*$
  and $p_1, \ldots, p_n \ge |P|$
  such that \[
  u_0 b^{p_n} u_1 \cdots u_{n-1} b^{p_n} u_n \in L(\mathcal B),
  \]
 \item there exist $u_0, \ldots, u_n \in \Sigma^* a \Sigma^*$
  and $p_1, \ldots, p_n > 0$
  such that 
  \[ 
  u_0 (b^{p_1})^* u_1 \cdots u_{n-1} (b^{p_n})^* u_n \subseteq L(\mathcal B).
  \]
 \end{enumerate}
\end{lemma}
\begin{proof}
 That (1) implies (2) is obvious.
 As $p_i \ge |P|$, when reading these factors they have to induce a loop in $\mathcal B$,
 which implies (3).
 Lastly, if (3) holds true, as
 \[
  u_0 b^{|P|\cdot p_1} u_1 \cdots u_{n-1} b^{|P| \cdot p_n} u_n \in L(\mathcal B)
 \]
 and $u_i \in \Sigma^* a \Sigma^*$,
 we also find $u_0 b^{|P|\cdot p_1} u_1 \cdots u_{n-1} b^{|P| \cdot p_n} u_n \in L_n$
 and (1) follows.\qed
\end{proof}

\begin{lemma}
\label{lem:maximal_n}
 Let $\Sigma = \{a,b\}$ and $L(\mathcal B) \subseteq a_1^* \cdots a_k^*$
 with $a_i \in \Sigma $.
 Then, there exists a maximal $n$
 such that $L(\mathcal B) \cap L_n \ne \emptyset$
 and for this maximal $n$,
 we can assume that $u_i \notin \Sigma^* b^{|P|} \Sigma^*$
 for the $u_i$, $i \in \{0,\ldots,n\}$, as in the previous lemma
 and $n \le |P|$. 
\end{lemma}
\begin{proof}
 Recall $\mathcal B = (\Sigma, P, \mu, p_0, F)$
 Note that $\mathcal B$ must necessarily be polycylic (this is a slightly stronger
 claim than Theorem~\ref{thm:bounded_characterization}, as this theorem
 only asserts existence of some polycyclic automaton)
 after removing all states that are not coaccessible, i.e., states from which no final state is reachable,
 which could obviously be done without altering $L(\mathcal B)$.
 For if $\mathcal B$ is then not polycyclic, then some strongly
 connected component does not consists of a single cycle only
 and we find two distinct words $u, v$ and a state $p \in P$
 such that $\mu(p, u) = \mu(p, v) = p$ (see also the forbidden
 pattern in~\cite[Theorem 4.29]{Pin2020}).
 But then, if we choose $x,y \in \Sigma^*$
 such that $\mu(p_0, x) = p$
 and $\mu(p, y) \in F$, we find $x(u+v)^*y \subseteq L(\mathcal B)$.
 Set $m = \max\{|u|, |v|\}$
 Then, for $i > 0$,
 \[
  \{ w \in x(u+v)^*y : |w| \le |x| + i \cdot m\}
 \]
 contains $x(u+v)^i$, and $|x(u+v)^i| = 2^i$.
 So, $L(\mathcal B) \cap \{ w \in \Sigma^* : |w| \le n \}$
 contains at least $2^{\lfloor n - (|x| - |y|) / m \rfloor}$
 many words, i.e., it not sparse.
 Furthermore, as $L \cap \Sigma^n \in O(n^c)$
 as a function of $n$ if and only if $L \cap \{ w \in \Sigma^* : |w| \le n \} \in O(n^{c'})$
 as a function of $n$ for some $c,c' \ge 0$,
 the claim follows.
 
 So, we can assume $\mathcal B$ is polycyclic and every state is coaccessible.
 Now, note that this implies that every loop in $\mathcal B$ (or strongly connected component
 in this case) must be labelled by a single letter, for if we
 have $\mu(p, u) = p$ with $|u|_a > 0$ and $|u|_b > 0$
 and choose again $x,y$ such that $\mu(p_0, x) = p$
 and $\mu(p, y) \in F$,
 we find $xu^ky \in L(\mathcal B)$, which contradict $L(\mathcal B) \subseteq a_1^* \cdots a_k^*$.
 
 But then, note that if, for example, $aba \in L(\mathcal B)$,
 we must have $|P| \ge 2$, as $\mu(p_0, ab) \notin \{ p_0, \mu(p_0,a) \}$.
 Similarly, if we have a word that switches letters, every time a letter-switch
 occurs the state we end up in $\mathcal B$ must be a new state not visited before,
 for otherwise we would have a loop whose transition are not exclusively
 labelled by a single letter.
 
 So, this implies that 
 if we have a word as written in Lemma~\ref{lem:number_of_b_blocks}
 in $L(\mathcal B)$, then $n \le |P|$
 which implies that we can find a maximal $n$.
 That $u_i \notin \Sigma^* b^{|P|} \Sigma^*$
 is also implied by Lemma~\ref{lem:number_of_b_blocks}
 and the maximality of $n$. \qed
\end{proof}
\end{toappendix}


\begin{lemmarep}
\label{lem:np_hardness}
 Suppose $\Sigma = \{a,b\}$. %und a,b vertauscht unten mit homsatz.
 Let $L(\mathcal B) \subseteq \Sigma^*$ be letter-bounded.
 Then, $L(\mathcal B)$\textsc{-Constr-Sync}
 is $\NP$-hard %if and only %nebenbei zwischen die a^+ nochmal sigma packen bringt nichts.
 % da man annehmen kann a_{i+1} ungleich a_i
 if $L(\mathcal B) \cap \Sigma^* a  b^{|P|}b^*  a \Sigma^* \ne \emptyset$.
 % nur if teil, weil if and only wird impliziert im theorem!
\end{lemmarep}
\begin{toappendix}
\begin{figure}[htb]
     \centering
     \hspace*{-2.5cm}
\includegraphics[width=17cm]{reduction.png}
  \caption{%Schematic illustration of the reduction from the proof of Proposition \ref{prop:stricly_bounded_np_hard}.
   The reduction from the proof of Lemma~\ref{lem:np_hardness}
   in the special case $J = 3$ (see the proof for the definition of $J$)
   and two input automata $\mathcal A_1, \mathcal A_2$ over $\{b\}$. The automata
   $\mathcal A_{i,j}$ are inflated, according to Definition~\ref{def:inflate_aut},
   copies of $\mathcal A_i$
   for $i \in \{1,2\}$, $j \in \{1,2,3\}$. 
   The letter $a$ maps
   every state not associated with a path inside each $\mathcal A_{i,j}$ to the last innermost state that 
   is hit by an $a$ along the path leading into this automaton. This is only drawn for $\mathcal A_{1,1}$ but left
   out for the other automata, also, to give a more ``high-level'' drawing, the $b$-transitions
   are not drawn. On the right end is the sink state $t$. The paths stay inside the automata but leave
   as soon as an $a$ is read.}
  \label{fig:reduction}
\end{figure}

\begin{proof}[Proof of Lemma~\ref{lem:np_hardness}]
First, using Lemma~\ref{lem:maximal_n},
choose $J > 0$ maximal such that
\[
 L(\mathcal B) \cap (\Sigma^* a \Sigma^* b^{|P|} \Sigma^*)^J \ne \emptyset.
\]
As stated in the lemma, we have $J \le |P|$ (which implies the constrution to follow could
be carried out in polynomial time).
Then, by Lemma~\ref{lem:number_of_b_blocks}, there
exist $u_0, \ldots, u_J \in \Sigma^* a \Sigma^*$
and $p_1, \ldots, p_J > 0$ ($J > 0$) such that \todo{Anderer Bezeichner als $J$?}
\[
 u_0 (b^{p_1})^* u_1 \cdots u_{J-1} (b^{p_J})^* u_J \subseteq L(\mathcal B).
\]
 Let $N$ be $|P|$ times the least common multiple of the numbers $p_1, \ldots, p_J$.
 We give a reduction from the {\sc DFA-Intersection} for unary 
 input automata, which is \NP-complete in this case~\cite{stockmeyer1973word,fernau2017problems}.
 Let $\mathcal A_i = (\{b\}, Q_i, \delta_i, q_i, F_i)$
 for $i \in \{1,\ldots,k\}$ be unary input automata, and we want to know
 if they all accept a common word. The problem remains
 \NP-complete if we assume for no input automaton, a start state is also a final state.
 This is easily seen but could also be shown similar to~\cite[Proposition 1]{DBLP:conf/ictcs/Hoffmann20}.
 Also, we can assume $F_i \ne \emptyset$ for all $i \in \{1,\ldots,k\}$.

 We are going to construct a semi-automaton $\mathcal C = (\{a,b\}, Q, \delta)$. %intuition?

 Write $u_i = u_{i,1} \cdots u_{i,|u_i|}$ with $u_j \in \Sigma$.
 For each $i \in \{0,\ldots,J\}$, we construct a path labelled with $u_i$.
 Formally, let $P_i = \{ q_{i, 0}, \ldots, q_{i,|u_i|} \} \subseteq Q$
 be new states and set
 \[
  \delta(q_{i,j-1}, u_j) = q_{i,j}.
 \]
 Then, for each $\mathcal A_i$
 we construct $J$ (disjoint) 
 copies of $\mathcal A_i$ and inflate
 them according to Definition~\ref{def:inflate_aut} by $N$.
 Call the results $\mathcal A_{i, 1}, \mathcal A_{i,2}, \ldots, \mathcal A_{i,J}$
 with $\mathcal A_{i,j} = (\{b\}, Q_{i,j}, \delta_{i,j}, s_{i,j}, F_{i,j})$.
 Note these are unary automata over the letter $b$.
 Also, let $t \in Q$ be a new state, which will be a (global) sink state in $\mathcal C$, i.e.,
 we set $\delta(t, a) = \delta(t, b) = t$.
 Next, we describe how we interconnect these automata with the paths and with $t$.
 See also Figure~\ref{fig:reduction} for a sketch of the reduction in the special case $J = 3$
 and two input automata.
 
 
 
 \begin{enumerate}
 \item Let $j \in \{1,\ldots, J\}$. For each final state $q \in F_{i,j}$
  let $P_{i,q}$ be a disjoint copy of the path $P_i$ constructed above,
  except for one final state $q$ were we simply retain the path $P_i$, but also name it by $P_{i,q}$.
  By identifying states, we mean states that we have previously constructed are now merged
  to a single state in $Q$. We have to pay attention that this procedure does not introduces
  any non-determinism.
  We identify the state $q_{i,0}$ with $q$
  and continue to identify the states $q_{i,j}$ and $q' \in Q_{i,j}$
  if $q_{i,j-1}$ and $q'' \in Q_{i,j}$ were identified
  and $u_{i,j} = b$ and $q' = \delta_{i,j}(q'', b)$. As $u_i \in \Sigma^* a \Sigma^*$, this process has to come to a halt
  before we have identified $< J$ states.
  Note that the first state such that $q_{i,j-1}$ and $q''\in Q$
  were identified but not $q_{i,j}$ and $\delta_{i,j}(q'', b)$, i.e., were $u_{i,j} = a$,
  we have added an $a$-transition to $q_{i,j}$
  from $q'' = q_{i,j-1}$ in $\mathcal A_{i,j}$, i.e., this is the first
  $a$-transition we have added to $\mathcal A_{i,j}$ and it branches out of $\mathcal A_{i,j}$.
  
  Then, if $j \le J - 1$,
  identify the state $q_{i,J}$ with the start state $s_{i,j+1}$ of $\mathcal A_{i,j+1}$, i.e.,
  the path $P_{i,q}$ ends at this state.
  And if $j = J$, we identify the state $q_{i,J}$ with $t$.
  
 \item For the path $P_0$ identify its end state $q_{i,|u_0|}$
  with the start state $s_{i,1}$ of $\mathcal A_{i,1}$.
     
 \item Up to now, we still hav emissing transitions. In all the paths created, 
  every missing $b$-transition, i.e., were we have a state with an $a$-transition
  leading out but no $b$-transition, we add a self-loop labelled with $b$ to that state.
  For each path $P$ (including the copies constructed in the first step)
  let $p \in P$ be that state closest to the end state, but that does not equal
  the end state (by the identifications above, some end state might already have an $a$-transition
  that goes out of some automaton $\mathcal A_{i,j}$) and has an outgoing $a$-transition.
  Such a state exists as the $u_i \in \Sigma^* a \Sigma^*$.
  \todo{Hier uU statt auf den Zustand immer auf den letzten Zustand davor mit einer $a$-Transition mappen?}
  Then, for every state in $P$ that does not have an $a$-transition
  we add an $a$-transition going to $p$.
  Consider $\mathcal A_{i,j}$ and let $P$ some path (the specific choice does not matter)
  ending at the start state of $\mathcal A_{i,j}$.
  For each state $q \in Q_{i,j}$ that does not has an outgoing $a$-transition up to now,
  add an $a$-transition going to the state $p \in P$ described above in that path.
  This ensures later that, by reading an $a$, we end up in a well-defined situation.
 \end{enumerate}
 Then, put all the states created so far, i.e., those of the $\mathcal A_{i,j}$
 and those of the paths constructed, into $Q$ (note for each $i \in \{1,\ldots,k\}$
 we have constructed paths and automata, intuitively we have copied each $\mathcal A_i$, inflated
 the copies and interconnected them with the paths given by the $u_i$)
 and let $\delta$ be the transition as defined above or as given by $\mathcal A_{i,j}$
 on the state of these automata.
 
 
 We need the following property of $\mathcal C$. Suppose $i \in \{1,\ldots,k\}$, 
 $j \in \{1,\ldots,|J|\}$ and $w \in \{a,b\}^*$.
 
  \medskip 
 
\noindent\underline{Claim:}
  Let $q \in Q_{i,j} \setminus F_{i,j}$ with $\delta(q, w) = t$.
  Then, there exist 
  \[ 
  u_1, u_2 \cdots, u_{|J|-i+1} \subseteq \{b\}^*
  \]
  and $u,v\in \{a,b\}^*$
  such that $|u_i| \ge N$ and $|u_i|$ is divisible by $N$
  for all $i \in \{1,\ldots,|J|-i+1\}$
  and $v_1, \ldots, v_{|J|-i+1} \in \{a,b\}^*a\{a,b\}^*$
  so that 
  \[
   w = vu_1 v_1 u_2 v_2 \cdots u_{|J|-i+1} v_{|J|-i+1} u
  \]
  and $v \notin \Sigma^* b^N \Sigma^*$.
 \begin{quote} % genauer, muss jeweils teile dahinter von start auf final?, und wegen maximalitt |J| auch nicht mehr soclhe mit "echten" a's dazwischen.
     \emph{Proof of the Claim.} First, the state $q \in Q_{i,j}$ has to be mapped to a final
     state, which could only be done by a word containing at least $N$
     times the letter $b$,
     as in the inflated construction we can only go from non-auxiliary states
     to non-auxiliary states by reading at least that number of letters.
     However, before that we might read some word $v \in \{a,b\}^*$ that moves states around, does
     not has a consecutive sequence of more than $N$ $b$'s and hence, every $a$
     goes back to the start state. But at some point, this has to come to an end and we have
     to read a sequence of more than $N$ consecutive $b$'s.
     Additionally, by the construction of the inflation, the word
     that moves from a non-auxiliary state to another non-auxiliary state
     must have a number of $b$'s that is divisible by $N$.
     Also, observe that such a word must consists entirely of $b$, because
     for non-final states in $\mathcal A_{i,j}$ every $a$ maps back to the start state.
     Then, by construction (recall $u_i \in \Sigma^* a \Sigma^*$ for the labels
     of the paths constructed above) to move between the automata $\mathcal A_{i,j}$
     inside of $\mathcal C$
     we have to traverse a path
     where, on some part, we can only move forward by reading the letter $a$.
     After this, when we are at the start state of $\mathcal A_{i,j+1}$,
     as by assumption the start state is not final, we again have to read at least $N$
     times the letter $b$ and so on, until we have reached a final 
     state in $\mathcal A_{i,|J|}$.
     Then, we have to read at least one $a$ to map the final state to $t$, from which
     on, as $t$ is a sink state, we can read any word. 
     \emph{[End, Proof of the Claim]}
 \end{quote}
 
 
 The automaton $\mathcal C$ has a synchronizing word in $L$
 if and only if all the $\mathcal A_i$, $i \in \{1,\ldots,k\}$,
 accept a common word.
 
 \begin{enumerate}
 \item Assume we have a word $b^n$ accepted by all $\mathcal A_i$ for $i \in \{1,\ldots,k\}$. 
 Then, for
 \[
  w = u_0 b^{N\cdot n} u_1 \cdots u_{J-1} b^{N\cdot n} u_J
 \] 
 we have $w \in L$ and $w$ synchronizes $\mathcal A$.
 Note that, after reading $u_{j-1}$,
 the automaton $\mathcal A_{i,j}$
 is either in its start state, or the final $a$ in $u_{j-1}$
 has mapped some state in $\mathcal A_{i,j}$ to a state outside of $Q_{i,j}$.
 So, when reading $b^{N\cdot n}$, 
 as $\mathcal A_{i,j}$ equals  the inflation of $\mathcal A_i$ by $N$,
 we end up in a final state $F_{i,j}$.
 Then, we read $u_j$ to map those final states to the start state
 of the next automaton $\mathcal A_{i,j+1}$ or to $t$ if $j = J$.
 Note that all states in-between are either mapped
 to a start state of some $\mathcal A_{i,j}$, moved inside of some
 $\mathcal A_{i,j}$, or, when an $a$ is read and they are not mapped
 back to a state that ultimately ends in a start state of some $\mathcal A_{i,j}$
 are moved toward the state $t$.
 As we always read enough $a$ to always make a step towards the sink state $t$
 the result follows.\todo{genauer}
 
 
 
 \item  Assume $\mathcal A$ has a synchronizing word $w \in L$.
  Then, as $t$ is a sink state, the word $w$ must map every state to $t$.
  Consider the start state of some $\mathcal A_{i,1} = (\{b\}, Q_{i,1}, \delta_{i,1}, q_{i,1}, F_{i,1})$.
  By the above claim,  
  there exist $u_1, u_2 \cdots, u_{J} \subseteq \{b\}^*$
  such that $|u_i| \ge N$ and $|u_i|$ is divisible by $N$
  for all $i \in \{1,\ldots,|J|\}$
  and $v_1, \ldots, v_{J} \in \{a,b\}^*a\{a,b\}^*$ and $v, u \in \{a,b\}^*$
  so that 
  \[ 
    w = vu_1 v_1 u_2 v_2 \cdots u_{J} v_{J} u.
  \]
  By the above claim, Lemma~\ref{lem:maximal_n} and the maximal choice of $J$,
  we have 
  \[ 
  \{ v, v_1, \ldots, v_J \} \cap \Sigma^* b^{|P|} \Sigma^* = \emptyset,
  \] 
  i.e.,
  these words does not contains a sequence of more than $|P|$, and so in particular not more than $N$,
  consecutive $b$'s.
  
  
  Now, let $b^n$ be a maximal non-empty factor whose length $n$ is divisible by $N$ of $vu_1 v_1$
  and using only the letter $b$.
  Note that, by construction of $\mathcal A_{i,1}$,
  if we write $vu_1 v_1 = x b^n y$,
  we have $\delta_{i,1}(q_{i,1}, x) = q_{i,1}$.
  Then, we claim that $b^{n / N}$ is accepted
  by every automaton $\mathcal A_i$. 
  Fix an index $i \in \{1,\ldots,k\}$.
  By the construction of the inflation, this is equivalent
  with the condition that $b^n$ drives every automaton
  $\mathcal A_{i,j}$ for $j \in \{1,\ldots,|J|\}$
  from the start state to some final state.
  Suppose this is not the case. As the automata $\mathcal A_{i,j}$
  are isomorphic, i.e., they are copies of each other, we can assume this is not the case for $\mathcal A_{i,1}$, i.e., 
  we have $\delta_{i,1}(q_{i,1}, b^n) \notin F_{i,1}$.
  Then, consider the following suffix of $w$ (recall $xb^n y = v u_1 v_1$, and $y$ has to start with an $a$) 
  \[
   y u_2 v_2 \cdots u_{|J|} v_{|J|} u.
  \] 
  Note that if we have in $u$ a consecutive sequence of $b$'s
  of length more than $N$, the rest of $u$ also must consist of $b$'s only, i.e.,
  we cannot read an $a$ anymore.
  For suppose this is not the case and $u \in \Sigma^* b^N \Sigma^* a \Sigma^*$.
  We have $\delta(q_{i,1}, xb^n) = \delta(q_{i,1}, b^n) \in Q_{i,1} \setminus F_{i,1}$.
  By assumption, $\delta(q_{i,1}, w) = t$,
  and so we must have $\delta(q_{i,1}, y u_2 v_2 \cdots u_{|J|} v_{|J|} u) = t$.
  Applying the above claim again,
  yields that we can factorize $y u_2 v_2 \cdots u_{|J|} v_{|J|} u$
  such that we have at least $|J|$ blocks of consecutive 
  $b$'s broken up by at least one occurrence of the letter $a$
  between each such block.
  However, then 
  then
  \[
   w = x b^n y u_2 v_2 \cdots u_{|J|} v_{|J|} u,
  \]
  as $y$ starts with an $a$, we would get a factorization
  of $w$ with $|J| + 1$ blocks of consecutive $b$'s separated by words
  with at least one $a$, which is not possible by the maximal choice
  of $J$ and Lemma~\ref{lem:number_of_b_blocks}.
  
 \end{enumerate}
 So, this shows that this is a valid reduction.\qed
\end{proof}
\end{toappendix}




So, finally, we can state our main theorem of this section.
Recall that by Theorem~\ref{thm:sparse_in_NP},
and as the class of bounded regular languages equals
the class of sparse regular languages~\cite{DBLP:journals/eik/LatteuxT84}, for bounded regular constraint
languages, the constrained problem is, in our case, in \NP. 
 
 
\begin{theoremrep}[Dichotomy Theorem]
\label{thm:dichotomy}
 % a_i != a_{i+1} sonst nichts weiter
 % dann a_{j_1} auf a und so weiter hom bild
 % hom bild a^* b^* a^*
 % aber a_{j_2} verschieden von a_{j_1}, a_{j_3} über die vereinigne
 % wie in entscheidungsverfahren.
 %
 % aber aufpassen, P ist anders!!!! aber dann schlussfolgern für das P' auch unendlich
 % oder 2^{|P|} nehmen
 %
 % oder wie so eine "sigma"-menge, also für jedes n existiert eins - schnitt/vereinigung - schreiben. aber da nicht klar ob regulär.
 Let $a_1, \ldots, a_k \in \Sigma$ be a sequence of letters
 and $L \subseteq a_1^* \cdots a_k^*$ be regular.
 The problem $L\textsc{-Constr-Sync}$
 is \NP-complete if
 \[
  L \cap \left(\bigcup_{\substack{1 \le j_1 < j_2 < j_3 \le k \\ a_{j_2} \notin \{a_{j_1}, a_{j_3}\} }} L_{j_1,j_2,j_3} \right) \ne \emptyset
 \]
 with $L_{j_1, j_2, j_3} = \Sigma^* a_{j_1} \Sigma^* a_{j_2}^{|P|} \Sigma^* a_{j_3} \Sigma^*$
 for $1 \le j_1 < j_2 < j_3 \le k$ and solvable in polynomial time otherwise.
\end{theoremrep}
\begin{proof}
 Set $L = L(\mathcal B)$.
 %By Theorem~\ref{thm:bounded_regular_form}, we can
 %write % auf lemma/theorem verweisen, dass es immer so geht.
 %$L(\mathcal B) = \bigcup_{i=1}^n A_1^{(i)} \cdots A_k^{(i)}$ % erwähnen k gleich da man eps wählen kann, vielleicht als lemma diese form hinschreiben. todo
 %with unary regular languages $A_j^{(i)} \subseteq \{a_j\}^*$
 %for $j \in \{1,\ldots,k\}$.
 Let $j_1, j_2, j_3 \in \{1,\ldots,k\}$
 be such that $a_{j_2}\notin\{a_{j_1},a_{j_3}\}$, $j_1 < j_2 < j_3$
 and
 \[
  L(\mathcal B) \cap L_{j_1, j_2, j_3} \ne \emptyset.
 \]
 Then, there exists a word $u_1 a_{j_1} u_2 a_{j_2}^{|P|} u_3 a_{j_3} u_4 \in L(\mathcal B)$
 with $u_1, u_2, u_3, u_4 \in \Sigma^*$.
 By the pigeonhole principle, when reading the factor $b^{|P|}$,
 at least one state has to be traversed twice 
 and we find $p > 0$ such that $u_1 a u_2 b^{|P| + i\cdot p} u_3 a u_4$
 for any $i \ge 0$.
 
 
 
 Define a homomorphism $\varphi : \Sigma^* \to \{a,b\}^*$
 by $\varphi(a_{j_1}) = \varphi(a_{j_3}) = a$,
 $\varphi(a_{j_2}) = b$
 and, for the remaining letters, $\varphi(a) = \varepsilon$,
 if $a \in \Sigma \setminus \{a_{j_1}, a_{j_2}, a_{j_3}\}$.
 Then, $\varphi(L) \subseteq \varphi(a_1)^* \cdots \varphi(a_k)^*$
 is letter-bounded. % zeigen, dass unter hom abgeschlossen. aber klar
 Set $\Gamma = \{a,b\}$ and let $\mathcal B' = (\Gamma, P', \mu', p_0', F')$
 be a recognizing PDFA for $\varphi(L)$.
 %By Lemma~\ref{lem:L_j1j2j3_intersection},
 %there exists $i_0 \in \{1,\ldots,n\}$
 %such that $A_{j_1}^{(i_0)}$ and $A_{j_3}^{(i_0)}$
 %do not equal~$\{\varepsilon\}$
 %and $A_{j_2}^{(i_0)}$ is infinite.
 %As $\varphi$ is a homomorphism, we have 
 %$\varphi(A_1^{(i_0)} \cdots A_k^{(i_0)}) = \varphi(A_1^{(i_0)}) \cdots \varphi(A_k^{(i_0)})$. Furthermore,
 %$a^+ \cap \varphi(A_{j_1}^{(i_0)}) \ne \emptyset$,
 %$a^+ \cap \varphi(A_{j_3}^{(i_0)}) \ne \emptyset$
 %and $b^{|P'|}b^* \cap \varphi(A_{j_2}^{(i_0)}) \ne \emptyset$.
 %As the language $\varphi(A_1^{(i_0)}) \cdots \varphi(A_k^{(i_0)})$
 %is contained in $\varphi(L)$,
 %this yields
 We have
 \[
  \varphi(u_1) a \varphi(u_2) b^{|P| + i\cdot p} \varphi(u_3) a \varphi(u_4) \in \varphi(L) 
 \]
 for any $i \ge 0$. So,
 $
 \varphi(L) \cap \Gamma^* a^+ \Gamma^* b^{|P'|}b^* \Gamma^* a^+ \Gamma^* \ne \emptyset.
 $
%  However, note that, for $u \in \{a,b\}^*$,
%  \begin{multline*}
%       u \in \varphi(A_1^{(i_0)}) \cdots \varphi(A_k^{(i_0)}) \cap \Gamma^* a^+ \Gamma^* b^{|P'|}b^* \Gamma^* a^+ \Gamma^*
%   \\ \Leftrightarrow 
%   u \in \varphi(A_1^{(i_0)}) \cdots \varphi(A_k^{(i_0)}) \cap \Gamma^* a^+  b^{|P'|}b^* a^+ \Gamma^*.
%  \end{multline*}
% So,
% $
%  \varphi(L) \cap \Gamma^* a^+ b^{|P'|} b^* a^+ \Gamma^* \ne \emptyset. %todo genauer
% $
 By Lemma~\ref{lem:np_hardness}, $\varphi(L)$\textsc{-Constr-Sync}
 is \NP-hard and so, by Proposition~\ref{prop:hom_lower_bound_complexity},
 also $L\textsc{-Constr-Sync}$ is \NP-hard, and so, with Theorem~\ref{thm:sparse_in_NP},
 \NP-complete.
 
 
 Now, suppose
  $
 L(\mathcal B) \cap \left(\bigcup_{\substack{1 \le j_1 < j_2 < j_3 \le k \\ a_{j_2} \notin \{a_{j_1}, a_{j_3}\} }} L_{j_1,j_2,j_3} \right) = \emptyset.
 $
  By Theorem~\ref{thm:bounded_regular_form}, we can
 write % auf lemma/theorem verweisen, dass es immer so geht.
 $L(\mathcal B) = \bigcup_{i=1}^n A_1^{(i)} \cdots A_k^{(i)}$ % erwähnen k gleich da man eps wählen kann, vielleicht als lemma diese form hinschreiben. todo
 with unary regular languages $A_j^{(i)} \subseteq \{a_j\}^*$
 for $j \in \{1,\ldots,k\}$.
 Then, 
 \[ 
 ( A_1^{(i)} \cdots A_k^{(i)} ) \cap \left(\bigcup_{\substack{1 \le j_1 < j_2 < j_3 \le k \\ a_{j_2} \notin \{a_{j_1}, a_{j_3}\} }} L_{j_1,j_2,j_3} \right) = \emptyset
 \]
 for any $i \in \{1, \ldots, n\}$.
 However, this implies that for any $i \in \{1,\ldots,n\}$, if there exists $j \in \{1,\ldots, k\}$
 such that $A_j^{(i)}$ is infinite, 
 then for all $j' < j$, or for all $j' > j$, % besonders wenn es keine gibt, als remark nach der proposition?
 we have $A_{j'} \subseteq \{a_j\}^*$ (recall that if $A_{j'} = \{\varepsilon\}$, then
 this is also fulfilled).
 Hence, by Proposition~\ref{prop:stricly_bounded_P},
 we have $(A_1^{(i)} \cdots A_k^{(i)})\textsc{-Constr-Sync} \in \PTIME$
 and then, by Lemma~\ref{lem:union},
 $L(\mathcal B)\textsc{-Constr-Sync} \in \PTIME$.\qed
\end{proof}

As the languages $L_{j_1, j_2, j_3}$ are regular, we
can devise a polynomial-time algorithm which checks the condition
mentioned in Theorem~\ref{thm:dichotomy}. 
 
\begin{corollary} %\todo{nicht $sigma = {a-1, ...m, a_k}$ schreiben, weil es suggeriert die buchstaben wären alle verschieden.}
 Given a PDFA $\mathcal B$ and a sequence of letters $a_1, \ldots, a_k$
 as input such that $L(\mathcal B) \subseteq a_1^* \cdots a_k^*$,
 the complexity of $L(\mathcal B)$\textsc{-Constr-Sync}
 is decidable in polynomial-time.
\end{corollary}
\begin{proof}
 An automaton for each $L_{j_1, j_2, j_3}$
 has size linear in~$|P|$. So, by the product automaton construction~\cite{HopUll79}, non-emptiness of
 $L(\mathcal B)$ with each $L_{j_1, j_2, j_3}$
 could be checked in time $O(|P|^2)$.
 Doing this for every $L_{j_1, j_2, j_3}$
 gives a polynomial-time algorithm
 to check non-emptiness of the language written
 in Theorem~\ref{thm:dichotomy}.~\qed
\end{proof}

\begin{example}
 For the following constraint languages CSP is \NP-complete: $ab^*a$,
 $aa(aaa)^*bbb^*d \cup a^*b \cup d^*$, $bbcc^*d^* \cup a$.
 
 For the following constraint languages CSP is in \PTIME: $a^5bd \cup cd^4$,
 $a^5bd \cup cd^*$, $aa^*bbbbcd^* \cup bbbdd^*d$.
\end{example}

\begin{proof}[Proof Sketch for Lemma~\ref{lem:np_hardness}]
 We construct a reduction from an instance
 of $\textsc{DisjointSetTransporter}$\footnote{Note that the problem $\textsc{DisjointSetTransporter}$ is over a unary alphabet, but for $L\textsc{-Constr-Sync}$
 we have $|\Sigma| > 1$. Indeed, we need the additional letters in $\Sigma$.}
 for unary input automata.
 %, which is $\NP$-complete in
 %this case, by Theorem~\ref{prop:set_transporter_np_complete},
 %to $L\textsc{-Constr-Sync}$ for $L$ as written in the statement\footnote{Note that the problem $\textsc{DisjointSetTransporter}$ is over a unary alphabet, but for $L\textsc{-Constr-Sync}$
 %we have $|\Sigma| > 1$. Indeed, we need the additional letters in $\Sigma$ in our reduction.}.
 
 To demonstrate the basic idea, we only do the proof
 in the case $L \subseteq a^* b^* a^*$.
 %By Theorem~\ref{thm:bounded_regular_form},
 %we can write $L = \bigcup_{i=1}^n A_1^{(i)} A_2^{(i)} A_3^{(i)}$
 %with regular languages $A_1^{(i)}, A_3^{(i)} \subseteq \{a\}^*$
 %and $A_2^{(i)} \subseteq \{b\}^*$.
 By assumption we can deduce $a^{r_1} b^{r_2} a^{r_3} \in L(\mathcal B)$
 with $p_2 \ge |P|$ and $r_1, r_3 \ge 1$.
 By the pigeonhole principle, in $\mathcal B$, 
 when reading the factor $b^{r_2}$, at least one state has to be traversed twice.
 Hence, we find $0 < p_2 \le |P|$ such that $a^{r_1} b^{r_2 + i\cdot p_2} a^{r_3}
 \subseteq L(\mathcal B)$ for each $i \ge 0$.
 %Then, by Lemma~\ref{lem:L_j1j2j3_intersection},
 %we must find $i_0 \in \{1,\ldots,n\}$
 %such that $a^+ \cap A_1^{(i_0)} \ne \emptyset$, $a^+ \cap A_3^{(i_0)} \ne \emptyset$
 %and $A_2^{(i_0)}$ is infinite.
%  As a shorthand,
%  we set $A_1 = A_1^{(i_0)}$, $A_2 = A_2^{(i_0)}$ and $A_3^{(i_0)} = A_3$.
% %  By Lemma~\ref{todo}, the languages could be written
% %  as a finite union of languages recognizable
% %  by unary automata with a single final state.
% %  As concatenation distributes over union, 
% %  %we can then write $A_1^{(i_0)} A_2^{(i_0)} A_3^{(i_0)}$
% %  %as a finite union of languages
% %  without loss of generality, as by such a rearranging the above assumption
% %  still holds true for at least one part of the union, we can 
% %  suppose $A_1$,  $A_2$ and $A_3$
% %  are recognizable by unary automata with a single final state.
% Then, it is easy to see that we find numbers $r_i, p_i$ such that
% $
%     a^{r_1}(a^{p_1})^* \subseteq A_1,
%     b^{r_2}(b^{p_2})^* \subseteq A_2, 
%     a^{r_3}(a^{p_3})^* \subseteq A_3
% $
% with, by the other assumptions, $p_2 > 0$ ($A_2$ infinite)
% and $r_1 + p_1 > 0$, $r_3 + p_3 > 0$ ($A_1$, $A_3$ non-empty and do not equal $\{\varepsilon\}$).
% todo hinschreiben, wenn ncihts gesagt immer complete und deterministic


Let $\mathcal A = (\{c\}, Q, \delta)$ and $(\mathcal A, S, T)$
be an instance of \textsc{DisjointSetTransporter}.
We can assume $S$ and $T$ are non-empty, as for $S = \emptyset$
it is solvable, and if $T = \emptyset$ we have no solution.
Construct $\mathcal A' = (\Sigma, Q', \delta')$
by setting
$
 Q' = S_{r_2} \cup \ldots \cup S_{1} \cup Q \cup Q_1 \cup \ldots \cup Q_{p_2-1} \cup \{ t \},
$
where $t$ is a new state, $S_i = \{ s_i \mid s \in S \}$ for $i \in \{1,\ldots, r_2 \}$
are pairwise disjoint copies of $S$
and $Q_i = \{ q^i \mid q \in Q \}$ are\footnote{Observe
that by the indices a correspondence between the sets
is implied. The index
in $Q_i$ at the top to distinguish, for $s \in S$ and $i \in \{1,\ldots,\min\{r_2, p_2-1\}\}$, between
 $s_i \in S_i$ and $s^i \in Q_i$. Hence, for each $s \in S$ and $i \in \{1,\ldots, r_2\}$,
 the states $s$ and $s_i$ correspond to each other, and for $q \in Q$
 and $i \in \{1,\ldots, p_2-1\}$ the states $q$ and $q^i$.} 
 also pairwise disjoint 
copies of $Q$. Note that also $S_i \cap Q_j = \emptyset$
for $i \in \{1,\ldots, r_2 \}$ and $j \in \{1,\ldots, p_2-1\}$.
Set $S_0 = S$ %and $Q_0 = Q$ 
as a shorthand.
Choose any $\hat s \in S_{r_2}$, then, for $q \in Q$ and $x \in \Sigma$, the transition function is given by
\[
 \delta'(q, x) = \left\{
 \begin{array}{ll}
  s_{i-1} & \mbox{if } x = b \mbox{ and } q = s_i \in S_i \mbox{ for some } i \in \{1,\ldots, r_2\}; \\ 
  \hat s & \mbox{if } x = a \mbox{ and } q \in (Q \cup Q_1 \cup \ldots \cup Q_{p_2-1}) \setminus S; \\
  s_{r_2} & \mbox{if } x= a \mbox{ and } q = s_i \in S_i \mbox{ for some } i \in \{0,\ldots,r_2\}; \\
  t       & \mbox{if } x = a \mbox{ and } q \in T; \\
  q^{p_2-1} & \mbox{if } x = b \mbox{ and } q \in Q; \\
  q^{i-1} & \mbox{if } x = b \mbox{ and } q = q^i \in Q_i \mbox{ for some } i \in \{2,\ldots,p_2-1\}; \\
  \delta(q, c) & \mbox{if } x = b \mbox{ and } q = q^1 \in Q_1; \\
  q       & \mbox{otherwise}.
 \end{array}
 \right.
\]

\newcommand{\automatacloudother}[2][.44]{%
	\begin{scope}[#2]
		\node [rectangle,draw,thick,text width=8.1cm,minimum height=7.6cm,
		text centered,rounded corners, fill=white, name = re] {};
\end{scope}}    


\newcommand{\innerstateloop}{
\begin{scope}
  \node[state] (s1) at (0,0) {}; \node (s1label) at (0.5,-.1) {$\in Q$};
  \node[state] (s11) at (-0.5,0.6) {};  \node (s11label) at (0.32,0.6) {$\in Q_{p_2-1}$};
  %\node[state] (s14) at ( 0.5,0.6);
  \node (s12) at (-0.25,1.2) {};
  \node (s13) at ( 0.25,1.2) {};
  \path[->] (s1)  edge [bend left] node [left] {$b$} (s11);
%  \path[->] (s14) edge [bend left] node [right] {$b$} (s1);
   \path[->] (s11) edge [bend left] node [left] {$b$} (s12);
%   \path[->] (s13) edge [bend left] node [right] {$b$} (s14);
  \draw[dashed] (-0.25,1.2) -- (0.25,1.2);
\end{scope}
}

\begin{figure}[htb]
     \centering
    \scalebox{.65}{    
 \begin{tikzpicture}
 \tikzset{every state/.style={minimum size=1pt},>=stealth'}
 \node (cloud) at (0,0) {\tikz \automatacloudother{fill=gray!0,thick};};
 
  \node (reset1) at (0,3) {};
  \node (reset2) at (-10,2.5) {};
  \path[->] (reset1) edge [bend right] node [above] {$a$} (reset2);
  
  \node (reset3) at (0,-3) {};
  \node (reset4) at (-10,-2.5) {};
  \path[->] (reset3) edge [bend left] node [below] {$a$} (reset4);
  
  \node (reset5) at (-5.4,2.8) {};
  \node (reset6) at (-10,2.5) {};
  \path[->] (reset5) edge [bend right] node [above,pos=.3] {$a$} (reset6);
  
  \node (reset7) at (-5.4,-2.8) {};
  \node (reset8) at (-10,-2.5) {};
  \path[->] (reset7) edge [bend left] node [below,pos=.3] {$a$} (reset8);
  
 
  %\draw (-3,0) ellipse (1.1cm and 3.1cm);
  %\draw (-5.5,0) ellipse (1.1cm and 3.1cm);
  %\draw (-10,0) ellipse (1.1cm and 3.1cm);
  \draw[rounded corners] (-4.1,-3) rectangle (-1.7, 3) {};
  \draw[rounded corners] (-6.3,-3) rectangle (-4.8, 3) {};
  \draw[rounded corners] (-10.5,-3) rectangle (-9, 3) {};
    
  %\draw (3,0) ellipse (1cm and 3cm);
  \draw[rounded corners] (1.7,-3) rectangle (4.1, 3) {};
  
  \node[state] (t) at (7,0) {$t$};
  
  \node at (3,3.4) {{\LARGE $T$}};
  \node at (-3,3.4) {{\LARGE $S$}};
  \node at (-10,3.5) {{\LARGE $S_{r_2}$}};
  \node at (-5.5,3.5) {{\LARGE $S_{1}$}};
  \node at (0.1,4.1) {{\LARGE Original $\mathcal A$ (altered)}};
   % (altered for new alphabet)
  
  \node (s1) at (-.5,2) {\tikz \innerstateloop;};
  \node (s2) at (.5,-1.5) {\tikz \innerstateloop;};
  %\node (s2) at (-1.2,-2) {\tikz \innerstateloop;};
  
  \node (sT1) at (3,1.9) {\tikz \innerstateloop;}; \node (sT1copy) at (3,1.5) {};
  \node (sT2) at (2.8,0.4) {\tikz \innerstateloop;}; \node (sT2copy) at (2.8,0) {};
  \node (sT3) at (3.1,-1.5) {\tikz \innerstateloop;}; \node (sT3copy) at (3.15,-2) {};
  
  \node (sS1) at (-3,1.7) {\tikz \innerstateloop;}; \node (sS1copy) at (-2.95,1.15) {};
  \node (sS2) at (-2.7,0) {\tikz \innerstateloop;}; \node (sS2copy) at (-2.65,-.55) {};
  \node (sS3) at (-3,-1.8) {\tikz \innerstateloop;};\node (sS3copy) at (-3,-2.3) {};
   
  \node[state] (sS11) at (-5.5,1.7) {};
  \node[state] (sS12) at (-5.2,0) {};
  \node[state] (sS13) at (-5.7,-2) {};
  
  \node[state] (sSr1) at (-10,1.7) {};
  \node[state] (sSr2) at (-9.7,0) {};
  \node[state] (sSr3) at (-10.1,-2) {};
  
  \path[->] (t) edge [loop right] node {$\Sigma$} (t);
  
  % nichteinzeichnen, self loop implied
  \path[->] (sT1copy) edge [bend left=35] node [above] {$a$} (t)
            (sT2copy) edge [bend left=10] node [above] {$a$} (t)
            (sT3copy) edge node [above] {$a$} (t);
            
            
  \node (sSr1a) at (-8.5,1.7) {};
  \node (sSr2a) at (-8.2,0) {};
  \node (sSr3a) at (-8.7,-2) {};
  
  \node (sSr1b) at (-7.2,1.7) {};
  \node (sSr2b) at (-6.9,0) {};
  \node (sSr3b) at (-7.4,-2) {};
  
  \path[->] (sSr1) edge node [above,pos=.3] {$b$} (sSr1a);
  \path[->] (sSr2) edge node [above,pos=.27] {$b$} (sSr2a);
  \path[->] (sSr3) edge node [above] {$b$} (sSr3a);
 
  \path[->] (sSr1b) edge node [above,pos=.35] {$b$} (sS11);
  \path[->] (sSr2b) edge node [above] {$b$} (sS12);
  \path[->] (sSr3b) edge node [above] {$b$} (sS13);
  
  \path[->] (sS11) edge [bend right=10] node [above,pos=.38] {$b$} (sS1copy);
  \path[->] (sS12) edge [bend right=10] node [above,pos=.2] {$b$} (sS2copy);
  \path[->] (sS13) edge [bend right=10] node [above,pos=.4] {$b$} (sS3copy);
  
  \draw[dashed] (-8.5,1.7) -- (-7.2,1.7);
  \draw[dashed] (-8.2,0) -- (-6.9,0);
  \draw[dashed] (-8.7,-2) -- (-7.4,-2);
 \end{tikzpicture}}
  \caption{%Schematic illustration of the reduction from the proof of Proposition \ref{prop:stricly_bounded_np_hard}.
   The reduction from the proof sketch sketch of Lemma~\ref{lem:np_hardness}.
   The letter $a$ transfers everything surjectively onto $S_{r_2}$,
   indicated by four large arrows at the top and bottom and labelled 
   by $a$.
   The auxiliary states $Q_1, \ldots, Q_{p_2-1}$, which are meant
   to interpret a sequence $b^{p_2}$ like a single symbol in the original
   automaton, are also only indicated inside of $\mathcal A$, but not fully written out.}
  \label{fig:reduction_np_hard}
\end{figure}



Please see Figure~\ref{fig:reduction_np_hard} for a sketch
of the reduction.
For the constructed automaton $\mathcal A'$, the following could be shown:
$\exists m \ge 0 : \delta(S, c^m) \subseteq T$
if and only if $\mathcal A'$ has a synchronizing word in $ab^{r_2}(b^{p_2})^*a$
if and only if $\mathcal A'$ has a synchronizing word in $ab^*a$
if and only if $\mathcal A'$ has a synchronizing word in $a^*b^*a^*$.
% \begin{align*}
%     \exists m \ge 0 : \delta(S, c^m) \subseteq T 
%                               & \Leftrightarrow \mathcal A'\mbox{ has a synchronizing word in $ab^{r_2}(b^{p_2})^*a$.} \\
%                               & \Leftrightarrow \mathcal A'\mbox{ has a synchronizing word in $ab^*a$.} \\
%                               & \Leftrightarrow \mathcal A'\mbox{ has a synchronizing word in $a^*b^*a^*$}
% \end{align*} 

\begin{toappendix}

Next, we supply the proof of the claim made in the proof sketch of Lemma~\ref{lem:np_hardness}
from the main text.

\medskip

\noindent\underline{Claim:} 
 For the constructed automaton $\mathcal A'$ from
 the proof sketch of Lemma~\ref{lem:np_hardness} in the main text, we have:
\begin{align*}
    \exists m \ge 0 : \delta(S, c^m) \subseteq T 
                               & \Leftrightarrow \mathcal A'\mbox{ has a synchronizing word in $ab^{r_2}(b^{p_2})^*a$.} \\
                               & \Leftrightarrow \mathcal A'\mbox{ has a synchronizing word in $ab^*a$.} \\
                               & \Leftrightarrow \mathcal A'\mbox{ has a synchronizing word in $a^*b^*a^*$}
\end{align*} 
%\begin{quote}
%\begin{proof}[Proof of Claim]
 \emph{Proof of the Claim.}
 First, suppose $\delta(S, c^m) \subseteq T$.
 %Choose any $a^r \in A_1$ with $r > 0$
 %and $a^s \in A_3$ with $s > 0$. 
 By construction of $\mathcal A'$,
 for any $q, q' \in Q$,
 \begin{equation}\label{eqn:transition_Astar}
  \delta(q, c) = q'  \mbox{ in $\mathcal A$}  \Leftrightarrow \delta'(q, b^{p_2}) = q'  \mbox{ in $\mathcal A'$} .
 \end{equation}
 Also, $\delta'(Q'\setminus\{t\},a) = S_{r_2}$
 and $\delta'(S_{r_2}, b^{r_2}) = S$.
 Combining these facts, we find
 \[
  \delta'(Q', ab^{r_2}b^{p_2m}) \subseteq T \cup \{t\}. 
 \]
 A final application of $a$ then maps
 all states in $T$ to the single sink %(and synchronizing) 
 state~$t$.
 
 Clearly, as $ab^{r_2}(b^{p_2})^*a \subseteq a b^* a$
 and $a b^* a \subseteq a^* b^* a^*$, the next two implications are shown.
 Finally, to complete the argument, let $u = a^{p} b^q a^r$ be a synchronizing word, $p,q,r \ge 0$.
 Then, as $t$ is a sink state, $\delta'(Q', u) = \{t\}$.
 The only way to enter $t$ from the outside is to read $a$ at least once, and 
 as $t$ is a sink state, we have $\delta'(Q', a^p b^q a^r) = \{t\}$.
 Also, as for $q \notin T$, we have $\delta'(q, a) \notin T$,
 we must have $\delta'(Q', a^p b^q) \subseteq T \cup \{t\}$,
  or more specifically, $\delta'(Q' \setminus \{t\}, a^p b^q) \subseteq T$.
  We distinguish two cases for~$p$.
 
 \begin{enumerate}
 \item If $p = 0$, then, in particular, $\delta'(S, b^q)  \subseteq T$.
     By construction of $\mathcal A'$, for any $q \in Q$,
 \[
  \delta'(q, b^n) \in Q
 \]
 if and only if $n \equiv 0\pmod{p_2}$.
 So, $q = p_2 m$ for some $m \ge 0$.
 Hence, by Equation~\eqref{eqn:transition_Astar} above from the first case, in $\mathcal A$,
 we find $\delta(S, c^m) \subseteq T$.
 
 \item  If $p > 0$, then $\delta'(Q' \setminus\{t\}, a^p) = S_{r_2}$.
 The only way to leave any state in $S_{r_2}$
 is to read $b$, which transfers $S_{r_2}$ to $S_{r_2-1}$.
 Reasoning similarly, we find that we have to read in $b$
 at least $r_2$ many times, which finally maps $S_{r_2}$
 onto $S_0 = S$. So, $q \ge r_2$. By construction of $\mathcal A'$, for any $q \in Q$,
 \[
  \delta'(q, b^n) \in Q
 \]
 if and only if $n \equiv 0\pmod{p_2}$.
 So, as $\delta'(S, b^{q - r_2}) \subseteq T$, $q - r_2 = p_2 m$ for some $m \ge 0$.
 Hence, by Equation~\eqref{eqn:transition_Astar} above, in $\mathcal A$,
 we find $\delta(S, c^m) \subseteq T$.
 \end{enumerate}
This ends the proof of the claim. \emph{[End, proof of the Claim.]}
%\end{proof}
%\end{quote}
\end{toappendix}

%Finally, we show that we have some $m \ge 0$
%such that $\delta(S, c^m) \subseteq T$
%if and only if $\mathcal A'$ has a synchronizing word in $L$:
 Now, suppose $\delta(s, c^m) \subseteq T$ for some $m \ge 0$.
 By the above, $\mathcal A'$ 
 has a synchronizing word $u$ in $ab^{r_2}(b^{p_2})^*a$.
%  As $A_1$ is non-empty
%  and does not equal $\{\varepsilon\}$, we can choose $a^p \in A_1$ with $p > 0$.
%  Similarly, choose $a^r \in A_3$.
%  Then
%  $
%   a^{p-1} u a^{r-1} v \in A_1 A_2 A_3 \subseteq L,
%  $
%  and %by Lemma~\ref{lem:append_sync}, 
%  this word also synchronizes $\mathcal A'$.
 Then, $a^{r_1 - 1}u a^{r_3-1} \in L(\mathcal B)$ also synchronizes~$\mathcal A'$.
 
 
 Conversely, suppose we have a synchronizing word $w \in L$
 for $\mathcal A'$.
 As $L \subseteq a^* b^* a^*$
 by the above equivalences,
 $\delta(S, c^m) \subseteq T$
 for some $m \ge 0$. \qed
\end{proof}



\section{Constraints from Strongly Self-Synchronizing Codes}
\label{sec:strongly_self_sync}
%
% ergebnisse bounded, beispiele
% dann noch ab(ba)^*ab zeigen, und dann erstmal paper abschließen. vgl ictcs kriterium

Here, we introduce strongly self-synchronizing codes and investigate $L$\textsc{-Constr-Sync}
for bounded constraint languages $L \subseteq w_1^* \cdots w_k^*$
where $\{ w_1, \ldots, w_k \}$ is such a code. %a strongly self-synchronizing code.

Let $C \subseteq \Sigma^+$ be non-empty.
Then, $C$ is called a \emph{self-synchronizing code}~\cite{zbMATH03943051,DBLP:books/daglib/0025093,Hsieh1989SomeAP},
if $C^2 \cap \Sigma^+ C \Sigma^+ = \emptyset$. If, additionally, $C \subseteq \Sigma^n$
for some $n > 0$, then it is called\footnote{In~\cite{Hsieh1989SomeAP}
 this distinction is not made and self-synchronizing codes are also called comma-free codes.}
a \emph{comma-free code}~\cite{golomb_gordon_welch_1958}.
Every self-synchronizing code is an infix code, i.e., no proper factor of a word from $C$ is in $C$~\cite{Hsieh1989SomeAP}.
A \emph{strongly self-synchronizing code}
is a self-synchronizing code $C \subseteq \Sigma^+$ \todo{nicht bloss $\cap C\Sigma^+$ möglich? für beweis ausreichend?}
such that, additionally, $(\pref(C) \setminus C)C \cap \Sigma^*C \Sigma^+ = \emptyset$.
%, i.e, even when appending code words to proper prefixes
%If $\pref(C)C \cap \Sigma^+ C \Sigma^+ = \emptyset$,
%we call $C$ a \emph{strongly self-synchronizing code}.
\begin{comment}
A language $C \subseteq \Sigma^+$ is called a \emph{self-synchronizing code},
if for any $u,v\in C$, $w \in \Sigma^*$ and $x,y \in \Sigma^+$
where $u \ne v$ we have that $uv = xwy$ or $u = xw$ or $u = wx$ implies $w \notin C$.
% if for any $u,v \in C$ and $x,y \in \Sigma^*$,
% the condition $uv = xwy$ with $|x| > 0$ or $|y| > 0$
% implies $w \notin C$, i.e., no code word or concatenation
% of two codewords 
These codes, which are a generalization of comma-free codes, 
are well-studied~\cite{zbMATH03943051,DBLP:books/daglib/0025093,golomb_gordon_welch_1958}. % auch comma-free erwähnen, intuition, wie auf wikipedia?
As a new notion, we introduce \emph{strongly self-synchronizing codes} $C \subseteq \Sigma^+$
as languages which fulfill the condition that, for
any $u \in \pref(C)$ and $v \in C$, 
%with $u \notin \pref(v)$ 
%and $x,y \in \Sigma^+$,
%the conditions $uv = xwy$ or $v = xw$ or $v = wx$ imply $w \notin C$. % prefix und suffix erwähnen?
% zu schwach, kann ja in der mitte v als faktor auftreten
%we have $\{ u \} \subseteq \fact(uv) \cap C \subseteq \{ v, u \}$.
if we write $uv = x_1 \cdots x_n$
with $x_i \in \Sigma$ for $i \in \{1,\ldots, n\}$,
then, for any $j \in \{1,\ldots,n\}$ and $k \ge 0$ where $j + k \le n$,
we have that $x_j x_{j + 1} \cdots x_{j+k-1} \in C$
implies\footnote{If $k = 0$, we mean $x_j x_{j+1} \cdots x_{j + k - 1}$ to denote the empty
string. So, $k = |x_j x_{j+1} \cdots x_{j + k - 1}|$. However, note that by definition $\varepsilon\notin C$.} $j = |u| + 1$ and $k = |u| + |v|$
or $j = 1$ and $k = |u|$.
Intuitively, in $uv$ only the last $|v|$ symbols form a factor in $C$
and possibly the first $|u|$ symbols\footnote{Note that,
for example, stipulating that $\{ u \} \subseteq \factor(uv) \cap C \subseteq \{ v, u \}$
for $u \in \pref(C)$ and $v \in C$ is actually a weaker condition.
It allows, for instance, for the factors $u,v$ to occur in the middle of $uv$,
which is excluded.}.
%However, I have seen examples in the literature were such fine points were
%not paid attention too, probably because it is most often intuitively clear
%what is meant. For example, a precursor to self-synchronizing codes
%were comma-free codes, and the original definition~\cite{golomb 2x}
%does not mention that the concatenated words have to be distinct, but
%give examples as $\{ aa, bb \}$ for comma-free codes.
Recall that by definition $\varepsilon \notin C$,
and by choosing $u = \varepsilon$, we find that these codes are infix codes, i.e.,
no proper factor is a code word in $C$. Also, every strongly self-synchronizing code
is a self-synchronizing codes. In Remark~\ref{rem:code_construction},
we will give a method to construct strongly self-synchronizing codes.
\end{comment}

To give some intuition for the strongly self-synchronizing
codes, we also present an alternative characterization, a few examples and a way to construct such codes.

\begin{propositionrep}
A non-empty $C \subseteq \Sigma^+$
is a strongly self-synchronizing code
if and only if, for
all $u \in \pref(C)$ and $v \in C$, 
if we write $uv = x_1 \cdots x_n$
with $x_i \in \Sigma$ for $i \in \{1,\ldots, n\}$,
then, for all $j \in \{1,\ldots,n\}$ and $k \ge 1$ where $j + k - 1 \le n$,
we have that $x_j x_{j + 1} \cdots x_{j+k-1} \in C$
implies $j = |u| + 1$ and $k = |v|$
or $j = 1$ and $k = |u|$.
Intuitively, in $uv$ only the last $|v|$ symbols form a factor in $C$
and possibly the first $|u|$ symbols.
\end{propositionrep}
\begin{proof}
 Let $C \subseteq \Sigma^+$ be a strongly self-synchronizing code.
 Suppose $u \in \pref(C)$ and $v \in C$.
 If $u \notin C$, then we must have $uv \notin \Sigma^*C\Sigma^+$,
 so that, if $uv = x_1 \cdots x_n$ as in the statement,
 we have $x_j \cdots x_{j+k-1}$ if and only if $j = |u| + 1$
 and $k = |u| + |v|$.
 If $u \in C$, then, as $uv \notin \Sigma^+ C \Sigma^+$,
 we find that we have only the possibilities
 $j = 1$ and $k = |u|$ or $j = |u| + 1$ and $k = |u| + |v|$.
 
 Conversely, suppose $C \subseteq \Sigma^+$ is non-empty
 and fulfills the condition mentioned in the statement.
 If $u, v \in C$ and $uv \in \Sigma^+ C \Sigma^+$,
 then we can write $uv = x_1 \cdots x_n$ with $x_i \in \Sigma$ for $i \in \{1,\ldots,n\}$
 and find $2 \le i \le j \le n - 1$
 such that $x_i \cdots x_j \in C$, which contradicts
 the condition in the statement.
 Similarly, if $u \in \pref(C) \setminus C$
 and $v \in C$ with $uv \in \Sigma^*C\Sigma^+$,
 then we can write $uv = x_1 \cdots x_n$ with $x_i \in \Sigma$ for $i \in \{1,\ldots,n\}$
 and find $1 \le i \le j \le n - 1$
 such that $x_i \cdots x_j \in C$, which would contradict
 the condition too. So, we must
 have $C^2 \cap \Sigma^+ C \Sigma^+ = \emptyset$
 and $(\pref(C) \setminus C) C \cap \Sigma^* C \Sigma^+ = \emptyset$.\qed 
\end{proof}

%\begin{remark}
%  To give some intuition on strongly self-synchronizing
%  codes with respect to CSP:
%  These codes ensure that auxiliary states that have to be introduced
%  when passing from words to letters in the reductions
%  were synchronized too, i.e., in some sense the defining conditions ensure that input words
%  do not stay on certain paths between these auxiliary states.
%\end{remark}

 When passing from letters to words by applying a homomorphism, in the reductions,
 we have to introduce additional states. The definition of the strongly synchronizing
 codes was motivated by the demand that these states also have to be synchronized, which turns out to be difficult in general.

\begin{example}\label{ex:strongly_self_sync}
 The code $\{aacc,bbc,bac\}$
 is strongly self-synchronizing.
 The code $\{ aab, bccc, abc \}$ is self-synchronizing, but
 not strongly self-synchronizing as, for example, $(a)(abc)$ 
 contains $aab$ or $(aa)(bccc)$ contains $abc$. 
\end{example}

\begin{toappendix}
 To give a proof of the claim made in Example~\ref{ex:strongly_self_sync}.
\begin{proposition}
 The code $\{aacc,bbc,bac\}$ is strongly self-synchronizing.
\end{proposition}
\begin{proof}
 By checking all cases to combine prefixes with code words:
 \[ 
 \begin{array}{llll}
     \mbox{Non-empty prefixes of $aacc$:} & (a)aacc & (a)bbc & (a)bac \\ 
      & (aa)aacc  & (aa)bbc  & (aa)bac \\ 
      & (aac)aacc & (aac)bbc & (aac)bac \\ 
      & (aacc)aacc & (aacc)bbc & (aacc)bac \\ 
      \\
     \mbox{Non-empty prefixes of $bbc$:} & (b)aacc   & (b)bbc   & (b)bac \\    
     & (bb)aacc  & (bb)bbc  & (bb)bac \\ 
     & (bbc)aacc & (bbc)bbc & (bbc)bac \\ 
     \\
     \mbox{Non-empty prefixes of $bac$:} & (b)aacc   & (b)bbc   & (b)bac \\    
     & (ba)aacc  & (ba)bbc  & (ba)bac \\ 
     & (bac)aacc & (bac)bbc & (bac)bac. 
 \end{array}
 \]
 So, we see that the defining conditions are satisfied. Note that $C \cap \Sigma^*C\Sigma^+ = \emptyset$
 is always satisfied for self-synchronizing codes, as they are infix codes. \qed
\end{proof}
\end{toappendix}



% achso, ne quatsch, c^k+1varphi(L) ist ja nicht das bild varphi(c^{k+1}varphi)
\begin{remark}[Construction] %[Construction of Strongly Self-Synchronizing Codes]
\label{rem:code_construction}
% und damit auch weitere besipiele für self-sync strongly codes
% bzw gibt auch methode zur konstruktion solcher codes
% % darauf in einführung verweisen.
%  Theorem~\ref{thm:constr_sync_hom_strongly_self_sync}
%  is a true generalization of Theorem~\ref{thm:forward_hom}
%  because the code constructed in the assumption
%  is a strongly self-synchronizing code, as we will show next.
%  This also yields a method to construct infinitely many such codes.
% Take any präfix code $X \subseteq \Sigma^*$ and 
% symbol $c \in \Sigma$.
 %such that no word in $X$ begins with it, i.e.,
 %we have $c\Sigma^* \cap X = \emptyset$.
 Take any non-empty finite language $X \subseteq \Sigma^n$, $n > 0$,
 and a symbol $c \in \Sigma$ such that $\{c\}\Sigma^* \cap X = \emptyset$.
 Let $k=\max\{\,\ell\geq0\mid \exists u,v\in\Sigma^*:uc^\ell v\in X\,\}$.
 Then, $Y = c^{k+1}X$
 is a strongly self-synchronizing code.
 %For if $u \in \pref(Y)$ and $v \in Y$,
 %then  in $uv$ no proper factor, except the last $|v|$
 %symbols and possibly the first $u$ symbols, start with $c^{k+1}$. Hence,
 %no such factor is a code word from~$Y$.
\end{remark}

\begin{example}\label{ex:strongly_self_sync_construction}
Let $\Sigma = \{a,b,c\}$
and $C = \{ ab,ba, aa\}$.
Then, $\{ cab, cba, caa \}$ or $\{ bbab, bbaa \}$
are strongly self-synchronizing codes by Remark~\ref{rem:code_construction}.
%constructed according
%to the scheme described in Remark~\ref{rem:code_construction}.
\end{example}




Our next result, which holds in general, states conditions on a homomorphism
such that we not only have a reduction from the problem
for the homomorphic image to our original problem, as stated in Proposition~\ref{prop:hom_lower_bound_complexity},
but also a reduction in the other direction.

\begin{theoremrep}
\label{thm:constr_sync_hom_strongly_self_sync}
 Let $\varphi : \Sigma^* \to \Gamma^*$
 be a homomorphism such that $\varphi(\Sigma)$
 is a strongly self-synchronizing code and $|\varphi(\Sigma)| = |\Sigma|$.
 Then, for each regular $L \subseteq \Sigma^*$ we have
 $
  L\textsc{-Constr-Sync} \equiv_m^{\log} \varphi(L)\textsc{-Constr-Sync}.
 $
\end{theoremrep}
\begin{proof}
 By Proposition~\ref{prop:hom_lower_bound_complexity}, we have 
 $\varphi(L)\textsc{-Constr-Sync} \le_m^{\log} L\textsc{-Constr-Sync}$.
 
 
 Next, we give a reduction 
 from $L\textsc{-Constr-Sync}$ % problem, mit sink state. ne geht auch ohne, wenn in ausgang, dann eifnach noch ein beliebiges zeichen dranhängen vorne, andersrum werden die aus Q gesynct.
 to $\varphi(L)\textsc{-Constr-Sync}$.
 % aber kann man immer ein zeichen dranhängen??? d.h. L muss
 % erlauben, dass zu jedem wort w \in L es u \in \Sigma^+ gibt, so dass uw \in L.
 % bei bounded nicht unbedingt erfüllt...
 %
 % problem L-constr-Sync-sink definieren.
 %
 % für strongly connected äquivalent, wenn man neues zeichen einführt
 %
 %  von einem zustand mit neuen zeichen zu einem sink zustand, andere loopen
 % aber der zustand muss mit consraint-wort erreichbar sein.
 
 %
 % doch geht, wenn in q_x und y und q_{xy} nicht definiert, dann zu q_y, und so bewegt 
 % man sich "in gleihe Richtung" wie das erste ZEichen
 % geht nur vür |u|_i <= 2, also wenn nur ein hilfszustand, sonst größtes
 % suffix von xy so dass noch definiert?
 %
 % am ende soll gelten delta(q_x, \varphi(u)) = delta(q,\varphi(u))
 %  So, delta(q_x, y) = q_z mit z maximales suffix von xy so dass q_x definiert.
 %
 % x\varphi(u) nie präfix für x != eps, also ist varphi(U) das längste derartige suffix.
 %
 Write $\Sigma = \{a_1, \ldots, a_n\}$ with $n = |\Sigma|$
 and $u_i = \varphi(a_i)$ for $i \in \{1,\ldots,n\}$.
 Let $\mathcal A = (\Sigma, Q, \delta)$
 be an input semi-automaton for $L\textsc{-Constr-Sync}$.
 
  We construct a semi-automaton $\mathcal A' = (\Gamma, Q', \delta')$.
  The state set will be
  \[
   Q' = \{ q_x \mid q \in Q, x \in \pref(\{ u_1,\ldots,u_n \}) \setminus \{ u_1,\ldots,u_n \} \}. 
  \]
  By identifying $q_{\varepsilon}$ with the state $q \in Q$,
  we can assume $Q \subseteq Q'$.
  Then, for $q_x \in Q'$ and $y \in \Sigma$, 
  let $z$ be the longest suffix of $xy$
  such that $z \in \pref(\{ u_1,\ldots,u_n\})$ and set\footnote{Note
  the implicit correspondence between the states $q$
  and $q_z$ for $z \in \pref(\varphi(\Sigma)) \setminus \varphi(\Sigma)$.}
  \begin{equation}\label{eqn:def_delta_bar}
   \delta'(q_x, y) = \left\{
   \begin{array}{ll} % todo noch besser formatieren.
    q_{z}          & \mbox{if } z \in \pref(\{ u_1,\ldots,u_n\}) \setminus \{ u_1, \ldots, u_n \}; \\ 
    \delta(q, a_i)  & \mbox{if } \exists i \in \{ 1, \ldots, n \} : z = u_i.
   \end{array}
   \right.
  \end{equation}
  As $|\varphi(\Sigma)| = |\Sigma|$ and $\{ u_1, \ldots, u_n \}$ is a prefix code\footnote{A code
  is a prefix code, if no code word is the proper prefix of another code word.}, the transition function % todo das genauer? opben schreiben impliziert prefix-free, siehe mfcs19
  is well-defined.
%   Then, more generally, for $q_x, q_y \in Q'$
%   and $u \in \Sigma^*$, we
%   have %todo induktiv zeigen?
%   \begin{equation}
%       \delta'(q_x, u) = q_y % q delta
%       \Longleftrightarrow
%       \mbox{$y$ is the longest suffix of $xu$ in $f$
%   \end{equation}
  By construction, for any $u \in \Sigma^*$ and $q \in Q$,
  we have  
  \begin{equation}\label{eqn:comma_free_reduction}
      \delta(q, u) = \delta'(q, \varphi(u)).
  \end{equation}
  
  
  
  
\begin{comment}  
  But we need a more precise statement.
  
  \begin{myclaiminproof}
   Let $q_x, q'_z \in Q'$ and $y \in \Sigma^*$.
   Then,
   \[
    \delta'(q_x, y) = q'_z,
   \]
   where $xy = x_0 u_{i_1} x_1 u_{i_2}\cdots u_{i_m} x_m$
   for $i_1, \ldots, i_m \in \{1,\ldots,n\}$
   and $m$ is maximal with\footnote{Intuitively, these conditions
   express that $xy$ is ``parsed'' greedily for the factors
   from $\varphi(\Sigma)$.}:
   \begin{enumerate}
   \item $u_{i_j} \in \varphi(\Sigma)$ for $j \in \{1,\ldots,m\}$;
   \item for any $j \in \{0,\ldots, m-1\}$
    the word $x_j \in \Sigma^*$ is the shortest word 
    such that there exists $i_{j+1} \in \{1,\ldots,n\}$
    so that $u_{i_{j+1}} \in \varphi(\Sigma)$ and $x_0 u_{i_1} x_1 \cdots u_{i_{j}} x_j u_{i_{j+1}}$
    is a prefix of $xy$
    and $x_m \in \Sigma^*$ is the shortest word which contains
    no factor in $\varphi(\Sigma)$ and could be appended to give $xy$;
   \item $z$ is the longest suffix of $x_m$ in $\pref(\varphi(\Sigma))\setminus \varphi(\Sigma)$;
   \item $q' = \delta'(q, u_{i_1} \cdots u_{i_m}) = \delta(q, a_{i_1}\cdots a_{i_m})$,
    where $a_{i_j} \in \Sigma$ is the unique symbol with $u_{i_j} = \varphi(a_{i_j})$
    for $j \in \{1,\ldots,m\}$.
   \end{enumerate}
  \end{myclaiminproof}
  \begin{myclaimproof}
   We do induction on the length of $y$.
   If $y = \varepsilon$,
   as $q_x \in Q'$, and so $x \in \pref(\varphi(\Sigma)) \setminus \varphi(\Sigma)$,
   as $\varphi(\Sigma)$ is strongly self-synchronizing code,
   %\footnote{Actually, by a closer investigation
   %of the operational mode of the automaton, we see that from any state $q \in Q$
   %we never arrive at a state $q_x$ where $x$ contains a word from $\varphi(\Sigma)$
   %as a factor, even if $\varphi(\Sigma)$ is not strongly self-synchronizing, but only a prefix code.}, 
   the word cannot contain a word from $\varphi(\Sigma)$
   as a proper factor and so $m = 0$ in the above form and we find $z = x$
   and $q' = q$.
   So, now we evaluate $\delta'(q_x, ya)$ for $y \in \Sigma^*$ and $a \in \Sigma$.
   By induction hypothesis, we can write $\delta'(q_x, y) = q_z'$
   with a decomposition of $xy$ as written in the statement of the claim
   for some $m \ge 0$.
   Let $v$ be the longest suffix of $za$ which gives a word in $\pref(\varphi(\Sigma))$
   and write $za = wv$ with $w \in \Sigma^*$.
   If $v \notin \varphi(\Sigma)$,
   then
   \[ %todo x_ma = x'_ma setzen?
    xya = x_0 u_{i_1} x_1 u_{i_2}\cdots u_{i_m} (x_ma)
   \]
   is a new decomposition fulfilling the conditions of the claim
   but with $v$ in place of $z$ and $\delta'(q'_z, a) = q_{v}$.
   If $v\in \varphi(\Sigma)$, then let $i_{m+1} \in \{1,\ldots, n\}$
   be such that $v = u_{i_{m+1}}$ and we find
   \begin{equation}\label{eqn:claim}
    xya = x_0 u_{i_1} x_1 u_{i_2}\cdots u_{i_m} x_m' u_{i_{m+1}}
   \end{equation}
   for some $x_m' \in \Sigma^*$ with $x_m a = x_m' u_{i_{m+1}}$.
   By Equation~\eqref{eqn:comma_free_reduction}, $\delta'(q'_z, a) = \delta'(q', u_{i_{m+1}}) = \delta(q', a_{i_{m+1}})$
   and the claimed conditions are satisfied for the decomposition
   written in Equation~\eqref{eqn:claim}.
  \end{myclaimproof}
\end{comment}

 
\begin{comment}
 \begin{myclaiminproof}
  Let $i \in \{1,\ldots,n\}$ and $q_x \in Q'$.
  Then,
  \[
   \delta'(q_x, u_i) = \delta'(q, u_i) = \delta(q, a_i).
  \]
 \end{myclaiminproof}
 \begin{myclaimproof}
  If $x = \varepsilon$, this is a special case of Equation~\eqref{eqn:comma_free_reduction}.
  So, assume $x \ne \varepsilon$.
  As $\varphi(\Sigma)$
  is a variable-length comma-free code, so in particular a suffix code,
  the longest prefix of $u_i$ which, concatenated with $x$ in front,
  gives a word in $\pref(\{u_1, \ldots, u_n\})$
  is not $u_i$ itself.
  So, we can write $u_i = zay$ with $z,y \in \Sigma^*$ and $a \in \Sigma$ such that
  $z$ is the longest prefix for which
  $xz \in \pref(\{u_1, \ldots, u_n\})$
  holds true. Then,
  $xza \notin \pref(\{u_1, \ldots, u_n\})$
  and let $v$ be the longest suffix of $xza$
  in $\pref(\{u_1, \ldots, u_n\})$.
  By choice $za \in \pref(\{u_1, \ldots, u_n\})$.
  So, $|za| \le |v|$ and $za$ is a suffix of $v$.
  As $xza \notin\pref(\{u_1, \ldots, u_n\})$,
  we have $|v| < |xza|$.
  Write $v = x'za$ with $x' \in \Sigma^*$
  being a proper suffix of $x$.
  %As $q_x \in Q'$, which gives $x \in \pref(\{u_1, \ldots, u_n\}) \setminus \{u_1, \ldots,u_n\}$,
  Write $x = x'' x'$ with $x'' \in \Sigma^+$.
  Then, % todo vereinfachen? bilder zeichnen
  \[
   \delta'(q_{x''}, x') = q_x.
  \]
  %We have $xz \notin \{u_1, \ldots, u_n\}$,
  %as it is a proper factor of $xu_i$
  
  \begin{enumerate}
  \item $xz \notin \{u_1, \ldots, u_n\}$, $v \notin \{u_1, \ldots, u_n\}$
  
   Then, $\delta'(q_x, z) = q_{xz}$
   and $\delta'(q_{xz}, a) = q_v$.
      
  \end{enumerate}
  
  
  
  
  
  %
  % falsch, suffix, nicht präfix
  also $x' \in \pref(\{u_1, \ldots, u_n\}) \setminus \{u_1, \ldots,u_n\}$
  and we find a state $q_{x'} \in Q$.
  Then, as $vy = x'zay = x'u_i$, we can reason inductively, as $|x'| < |x|$,
  that
  \[
   \delta'(q_{x'}, u_i) = \delta'(q, u_i) = \delta(q, a_i).
  \]
  But if we write $x = x' x''$ with $x'' \in \Sigma^+$, by %todo hier equaino nummer referenzieren
  the definition of the transition function
  \[
   \delta'(q_{x'}, x'') = q_{x'x''} = q_x
  \]
  and so
  \[
   \delta(q_x, u_i) = \delta(\delta'(q_{x'}, x''), u_i)
  \]
  
  
  % wenn v \in \{u_1,...,u_n\} dann inducitively 
  
  As $\varphi(\Sigma)$
  is a comma-free code, we cannot have $v \in \{u_1, \ldots, u_n\}$,
  as $v$ is a proper factor of $xzay = xu_i$
  
  
  
  % wenn v nicht in {u1,...,un}
  By the definition of the transition function,
  \[
   \delta(q_x, xza) = q_v
  \]
 
  Then, 
  we must have
  $vy = u_i$.
  
  
 \end{myclaimproof}
\end{comment}

  Let $x \in \pref(\varphi(\Sigma)) \setminus \varphi(\Sigma)$
  and $u_i \in \varphi(\Sigma)$, $i \in \{1,\ldots,n\}$.
  Then, as $\varphi(\Sigma)$
  is a strongly self-synchronizing code, the word $xu_i$ does not contain
  a word from $\varphi(\Sigma)$ , except the suffix $u_i$,
  as a factor. Next, we will argue that, for the unique $a_i \in \Sigma$
  with $\varphi(a_i) = u_i$, the following equations 
  holds true:
  \begin{equation}\label{eqn:strongly_self_sync_transition}
   \delta'(q_x, u_i) = \delta'(q, u_i) = \delta(q, a_i).
  \end{equation}
  For if $v \in \pref(\{ u_i \}) \cap \Sigma$, then the longest suffix of $xv$
  in $\pref(\varphi(\Sigma))$ must be~$v$.
  First, it is a suffix
  from this set.
  Second, if there exists longer one, say $w$,
  then write $ww' \in \varphi(\Sigma)$ for some $w' \in \Sigma^*$.
  In that case, with $xv = x'w$ ($|x'| < |x|$), we have $x'ww' \in xu_i\Sigma^*$
  or $xu_i \in x'ww'\Sigma^+$.
  In the first case, $ww'$ contains the proper factor $u_i \in \varphi(\Sigma)$,
  which is not possible as $\varphi(\Sigma)$ is, in particular, an infix code.
  In the second case, $\{ x'u_i \} \cap \Sigma^* \varphi(\Sigma) \Sigma^+ \ne \emptyset$,
  which is excluded by the property of $\varphi(\Sigma)$ being strongly self-synchronizing.
  So, by the defining equation of $\delta'$, Equation~\eqref{eqn:def_delta_bar},
  if $v \notin \varphi(\Sigma)$, we have
  \[
   \delta'(q_x, v) = q_v,
  \]
  and if $v \in \varphi(\Sigma)$, then $v = u_i$, as $\varphi(\Sigma)$ is a prefix code,
  and
  \[
   \delta'(q_x, v) = \delta'(q_x, u_i) = \delta'(q, u_i) = \delta(q, a_i)
  \]
  with the unique $a_i \in \Sigma$ as above.
  So, in the latter case Equation~\eqref{eqn:strongly_self_sync_transition}
  was established. In the former case,
  if $u_i = vv'v''$, then $\delta'(q_v, v') = q_{vv'}$
  which is easily seen as we always read in a word giving a prefix from $\varphi(\Sigma)$, hence
  this word itself is the longest suffix from $\varphi(\Sigma)$.
  So, after reading the entire word $u_i$, by Equation~\eqref{eqn:def_delta_bar},
  Equation~\eqref{eqn:strongly_self_sync_transition}
  is implied.
  
  
  Lastly, we show that this gives a valid reduction.
  
  \begin{myclaiminproof}
   The automaton $\mathcal A = (\Sigma, Q, \delta)$
   has a synchronizing word in $L$
   if and only if $\mathcal A' = (\Gamma, Q', \delta')$
   has a synchronizing word in $\varphi(L)$.
  \end{myclaiminproof}
  \begin{myclaimproof}
   %\begin{enumerate}
   %\item 
   First, suppose there exists $u \in L$ such that $|\delta(Q, u)| = 1$.
    %By appending words if necessary, we can assume $|u| > 0$. %immernoch sync, zitieren AAHA, ncihtmehr in L unbedingt!!
    If $|Q| = 1$ every word is synchronizing and the statement is obviously true.
    So, we can assume $|Q| > 1$, which implies $|u| > 0$.
    Write $u = av$ with $a \in \Sigma$.
    % Let $q_x \in Q'$.
    % If $x = \varepsilon$, then, by Equation~\eqref{todo, todo in gleichung falsches alphabet!},
    % \[
    %  \delta'(q_x, \varphi(u)) = \delta(q, u).
    % \]
    % Now, suppose $x \ne \varepsilon$ and write $u = av$ with $a \in \Sigma$.
    By Equation~\eqref{eqn:strongly_self_sync_transition}, then, for any $x \in \pref(\varphi(\Sigma))\setminus\varphi(\Sigma)$,
    \[
     \delta'(q_x, \varphi(a)) = \delta(q, a).
    \]
    Hence, $\delta'(Q', \varphi(a)) = \delta(Q, a)$.
    As $\delta'(Q', \varphi(a)) \subseteq Q$, by Equation~\eqref{eqn:strongly_self_sync_transition},
    or its formulation for the special case of states in $Q$, Equation~\eqref{eqn:comma_free_reduction},
    we find
    \[
     \delta'(\delta(Q, a), \varphi(v))) = \delta(\delta(Q, a), v) = \delta(Q, u).
    \]
    The last set is, by assumption, a singleton set. Hence, the word $\varphi(u)$
    synchronizes~$\mathcal A'$. %sprechweise singelton set einführen? todo
   
    \medskip 
    
    Now, suppose there exists $u \in \varphi(L)$ such that $|\delta'(Q', u)| = 1$.
     Let $v \in \Sigma^*$ be such that $\varphi(v) = u$.
     By Equation~\eqref{eqn:strongly_self_sync_transition} (or Equation~\eqref{eqn:comma_free_reduction}),
     we have
     \[
      \delta(Q, v) = \delta'(Q, \varphi(v)).
     \]
     By assumption, the set on the right side is a singleton set. 
     Hence, $v$ synchronizes $\mathcal A$.
   %\end{enumerate}
  \end{myclaimproof}
  So, we find $L\textsc{-Constr-Sync} \le_m^{\log} \varphi(L)\textsc{-Constr-Sync}$
  and the proof is done.
\end{proof}

\begin{comment}
% Doch komplizierter, erstmal rauslassen.
% 
%We need the following fact, which is implied by the constructions
%in the previous proof. 
In general, regular languages are closed under homomorphic mappings,
but an exponential blow-ups might occur~\cite{projections paper}.
However, such a blow-up does not occur for mappings whose images
are strongly self-synchronizing codes.

\begin{proposition}
 Let $\varphi : \Sigma^* \to \Gamma^*$
 be a homomorphism such that $\varphi(\Sigma)$
 is a strongly self-synchronizing code and $\mathcal A = (\Sigma, Q, \delta, q_0, F)$
 be a PDFA.
 Then, we can construct in polynomial time
 a PDFA $\mathcal A' = (\Gamma, Q', \delta', q_0', F')$
 such that $\varphi(L) = L(\mathcal A')$.
\end{proposition}
\begin{proof}
%  Let $\mathcal A' = (\Gamma, Q', \delta', q_0', F')$
%  be the PDFA where the state set is given by Equation~\eqref{todo oben}
%  and the transition function given by Equation~\eqref{todo oben},
%  derived from the state set and transition function of $\mathcal A$.
%  As written in the proof of Theorem~\ref{thm:constr_sync_hom_strongly_self_sync},
%  we can assume $Q \subseteq Q'$.
%  Then, set $q_0' = q_0$ and $F' = F$.
%  By Equation~\eqref{eqn:strongly_self_sync_transition},
%  we have $\varphi(L) = L(\mathcal A')$.
%  % ne, nur varphi(L) \substeq L(\mathcal A'), wenn z.B. u \in L, dann uU x\varphi(u) 
 % mit der zurücklauf-Regel!!!! siehe den claim danach

 Todo, insbesondere präfixcode, teile dazwischen undefiniert lassen
 aber kann von teilen dazwische zu final laufen?
 dann aber suffix oder so, zeigen, dass hier nicht sein kann.
 
 
 Oder, ohne automaten? aber darstellunge mit j,p's muss auch berechnet werden...
 % unitär, d.h. single-state final bounded languages?
 
 
 Mit $C^*$ schneiden und nutzen, dass suffix code?
\end{proof}
\end{comment}

Finally, we apply Theorem~\ref{thm:constr_sync_hom_strongly_self_sync} to bounded languages
such that $\{ w_1, \ldots, w_k\}$ forms a strongly self-synchronizing code.

\begin{theoremrep}
 Let $L \subseteq w_1^* \cdots w_k^*$
 be regular such that $\{ w_1, \ldots, w_k \}$
 is a strongly self-synchronizing code.
 Then, $L\textsc{-Constr-Sync}$
 is either $\NP$-complete or in $\PTIME$. %solvable in polynomial time.
 %Moreover, the complexity itself, given a constraint PDFA as input,
 %could be decided in polynomial time.
\end{theoremrep}
\begin{proof}
 Let $\Gamma = \{ a_1, \ldots, a_n \}$
 be a new alphabet and let $\varphi : \Gamma^* \to \Sigma^*$
 be the homomorphism given by
 $\varphi(a_i) = w_i$ for $i \in \{1,\ldots, n\}$.
% As $\{ u_1, \ldots, u_n \}$ is a code, the homomorphism
% is injective % todo berstel/perrin zitieren?
% and 
 Let $U = \varphi^{-1}(L)$. 
 As every word in $L$
 is a concatentation of words from $\{ w_1, \ldots, w_n \}$,
 we have $L \subseteq \varphi(\Gamma^*)$.
 So, we find $\varphi(U) = L$.
%  By Theorem~\cite{todo}, we can write
%  \[
%   L = \bigcup_{r=1} 
%  \]
%  for numbers $j_i^{(r)}, p_i^{(r)}$
%  By Theorem~\cite{todo}, we can write $L$
%  as a finite union of languages of the form
%  \[
%   u_1^{j_1} (u_1^{p_1})^* \cdots u_n^{j_n} (u_n^{p_n})^*
%  \]
%  for numbers $j_i, p_i \ge 0$, $i \in \{1,\ldots,n\}$.
%  Now, observe that
%  \[
%   \varphi(  a_1^{j_1} (a_1^{p_1})^* \cdots a_n^{j_n} (a_n^{p_n})^* ) = 
%   u_1^{j_1} (u_1^{p_1})^* \cdots u_n^{j_n} (u_n^{p_n})^*.
%  \]
%  Also, as function application on sets preserves the union,
%  $\varphi(L)$
%  is a union of languages 
%  of the form $a_1^{j_1} (a_1^{p_1})^* \cdots a_n^{j_n} (a_n^{p_n})^*$
%  such that the numbers $j_i, p_i$, $i \in \{1,\ldots,n\}$,
%  are the same as those of the corresponding part in $L$.
%  So, we an apply Proposition~\ref{strictly bounded np-hard}
%  and Proposition~\ref{striclty bounded poly}
%  are preserved, i.e., we can apply those proposition

 By Theorem~\ref{thm:constr_sync_hom_strongly_self_sync},
 the languages $U$ and $L$
 have the same computational complexity.
 Also, as is easy to check, we have $U \subseteq a_1^* \cdots a_n^*$
 and $U$ is regular.
 So, by Theorem~\ref{thm:dichotomy}
 the constrained synchronization problem for $L$
 is either $\NP$-complete or in $\PTIME$.
\end{proof}

\begin{example}
 (1) $((aacc)(bbc)^*(bac))$\textsc{-Constr-Sync} is \NP-complete. \\
 (2) $((bbc)(aacc)(bac)^* \cup (bbc)^*)$\textsc{-Constr-Sync} is in \PTIME.
\end{example}


% \begin{theorem} \label{thm:forward_hom}
% 	Let $L\subseteq\Sigma^*$.
% 	%	Let $\mathcal B = (\Sigma, Q, \mu, q_0, F)$ and $\mathcal B' = (\Gamma, Q', \mu', q_0', F')$
% 	%	be two 	constraint automata.
% 	Let $\varphi : \Sigma \to \Gamma^*$ be a homomorphism %injective\todo{HF: Ich habe Inj. nicht benötigt!}
% 	such that $\varphi(\Sigma)$ is a prefix code.
% 	%and $\varphi(L(\mathcal B)) = L(\mathcal B')$.
% 	%	Let $c\notin\Gamma$.
% 	Let $c \in \Gamma$ with $\{c\}\Gamma^*\cap \varphi(\Sigma)=\emptyset$.
% 	Let $k:=\max\{\,\ell\geq0\mid \exists u,v\in\Gamma^*:uc^\ell v\in\varphi(\Sigma)\,\}$.
% 	% and let $k$ be the maximal number of consecutive appearances of $c$ in any code-word of $\varphi(\Sigma)$.
% 	Then
% 	$
% 	L%(\mathcal B)
% 	%	\textsc{-Constr-Sync} \leq^{\log}_m \{c\}\varphi(L)%(\mathcal B')
% 	\textsc{-Constr-Sync} \leq^{\log}_m \{c^{k+1}\}\varphi(L)%(\mathcal B')
% 	\textsc{-Constr-Sync}\,.
% 	$	
% 	%	with $L(\mathcal B'') = cL(\mathcal B')$.
% \end{theorem}



%
% Künstliches problem
%
%  Sigma mit Input-Automaten nur max {a,b} != Identität
%  mit ergebnissen 3-state case pspace-falls, np-fall alles realisierbar
%  
% aber ist praktische gleich zu alfphabet auf {a,b} einschränken...
%
% L \subseteq u^* v^* w^*
%
% ersetze durch (ccu)^* (ccv)^* (ccw)^*
%
% u_1^* ... u_n^* 
% kann man durch löschen eines buchstaben strongly self-sync code herstellen?
% falls ja, gelöschter buchstaben eingabeautomat identität, dann gleiche komplexität
%
% kann ich u^*v^*w^* entschieden, dann auch (ccu)^* (ccv)^* (ccw)^*, wobei
% c identität auf eingabeautomaten.
%
% andersrum definieren ccu, ccv, ccw als eingabe;
% so umdefinieren, dass in neuem autoamten ccu = u usw als abbildugnen
% wenn u,v,w einzelner buchstaben, dann geht es einfach. aber allgemein
% z.B. ccab, ccba, ccbb
% 
% kann man a,b so umdefinieren, dass ab = ccab usw?
% (btw entscheidungsproblem)
% 
% gegeben eine abbildung f und ein wort w über a,b
% Frage: Kann man a,b so als abbilungen belegen, dass f = w wenn w darüber ausgewertet?
% -> in dem fall klar, a^|w| = f, b = identität geht
% ein wenig wie gleichungen lösen

\section{Conclusion and Discussion}

We have looked at the constrained synchronization problem (Problem~\ref{def:problem_L-constr_Sync}) -- CSP for short -- for letter-bounded regular constraint languages and bounded languages induced by strongly self-synchronizing codes, thereby
continuing the investigation started in~\cite{DBLP:conf/mfcs/FernauGHHVW19}.
The complexity landscape in these cases is completely understood.
Only the complexity classes $\PTIME$ and $\NP$-complete arise.
%, and
%we have given conditions precisely when it is in $\PTIME$ and when it is $\NP$-complete.
In~\cite{DBLP:conf/ictcs/Hoffmann20} the question was raised if we can find sparse constraint languages
that give constrained problems complete for some candidate $\NP$-intermediate complexity class. At least for the
language classes investigated here
this is not the case. 
%Lastly, we have given a polynomial time procedure to decide the computational complexity of $L(\mathcal B)\textsc{-Constr-Sync}$, for a given automaton $\mathcal B$ accepting a strictly bounded regular language. 
For general sparse regular languages, it is still open if a corresponding
dichotomy theorem holds, or candidate $\NP$-intermediate problems arise. By the results obtained so far and the methods
of proofs, we conjecture that in fact a dichotomy
result holds true.


%
% Ne, kann sogar gemacht werden, da ja nur inverse bild automat gebaut werden muss, und das geht
% ohne blow-up.
%A decision procedure as exhibited in Subsection~\ref{sec:decision_procedure}
%could also be easily given for the constraint languages considered in Subsection~\ref{subsec:strongly_self_sync}
%by the methods of proof and a homomorphic mapping from a strictly bounded language. However, I do not know if this is also possible in polynomial time,  as 

%In a previous work~\cite{DBLP:conf/ictcs/Hoffmann20}, we have shown that for polycyclic languages,
%which equal the sparse regular languages by Theorem~\ref{thm:bounded_characterizations},
%CSP is always in \NP\  and also gave partial result for \NP-hardness and containment in \PTIME.

Let us relate our results to the previous work~\cite{DBLP:conf/ictcs/Hoffmann20}, where
partial results for \NP-hardness and containment in \PTIME\  were given.
Namely, by setting $\factor(L) = \{ v \in \Sigma^* \mid \exists u,w \in \Sigma^* : uvw \in L \}$
and $\mathcal B_{p,E} = (\Sigma, P, \mu, q, E)$
for $\mathcal B = (\Sigma, P, \mu, p_0, F)$ with $E \subseteq P$ and $q \in P$,
the following was stated.

\begin{proposition}[\cite{DBLP:conf/ictcs/Hoffmann20}]
\label{prop:NPc}
 Suppose we find $u, v \in \Sigma^*$ 
 such that we can write
$
 L = u v^* U
$
 for some non-empty language $U \subseteq \Sigma^*$
 with 
 $
  u \notin \factor(v^*), %\quad
  v \notin \factor(U) \mbox{ and } %\quad
  \pref(v^*) \cap U = \emptyset.
 $
 Then $L\textsc{-Constr-Sync}$ is $\NP$-hard.
\end{proposition}

\begin{proposition}[\cite{DBLP:conf/ictcs/Hoffmann20}]
\label{prop:NP_in_P}
  Let $\mathcal{B} = (\Sigma, P, \mu, p_0, F)$ be a polycyclic PDFA.
  If for every reachable $p \in P$ with $L(\mathcal B_{p, \{p\}}) \ne \{\varepsilon\}$ 
  we have $L(\mathcal B_{p_0, \{p\}}) \subseteq \suff(L(\mathcal B_{p, \{p\}}))$,
  then the problem $L(\mathcal B)\textsc{-Constr-Sync}$ is solvable
  in polynomial time.
\end{proposition}

Note that Proposition~\ref{prop:NPc} implies that $ab^*a$
gives an \NP-complete CSP. However, in the letter-bounded
case there exist constraint languages giving \NP-complete problems
for which this is not implied by Proposition~\ref{prop:NPc},
for example: $ab^*ba$, $ab^*ab$, $aa^*abb^*a$ or $ba^*b \cup a$.
Also, Proposition~\ref{prop:NP_in_P}
is weaker than our Proposition~\ref{prop:stricly_bounded_P}
in the case of letter-bounded constraints.
For example, it does not apply to $ab^*b$, every PDFA for this languages
has a loop exclusively labelled by the letter~$b$
and reachable after reading the letter $a$ from the start state, and
so words along this loop cannot have a word starting with $a$ as a suffix.






For general bounded languages, let us note the following implication of Propositions~\ref{prop:hom_lower_bound_complexity}
and~\ref{prop:stricly_bounded_P}.

\todo{Sind thin languages eigentlich die in $w^*$?}

\begin{propositionrep}
 Let $u,v \in \Sigma^*$. If $L \subseteq u^* v^*$ is regular, then $L$\textsc{-Constr-Sync} is solvable
 in polynomial time.
\end{propositionrep}
\begin{proof}
 Let $\Gamma = \{a,b\}$
 and $\varphi : \Gamma^* \to \Sigma^*$
 be the homomorphism given by $\varphi(a) = u$
 and $\varphi(b) = v$.
 Define $N = \{ (i,j) \mid u^i v^j \in L \}$
 and set $L' = \{ a^i b^j \mid (i,j) \in N \} \subseteq a^* b^*$.
 Then, $\varphi(L') = L$
 and by Proposition~\ref{prop:stricly_bounded_P}
 we have $L'\textsc{-Constr-Sync} \in \PTIME$.
 So, with Proposition~\ref{prop:hom_lower_bound_complexity}
 also $L\textsc{-Constr-Sync} \in \PTIME$.~\qed
\end{proof}

Next, in Proposition~\ref{prop:np_complete_case}, we give an example
of a bounded regular language yielding an $\NP$-complete synchronization problem,
 but for which this is
 not directly implied by the results we have so far.

% \begin{remark}\label{rem:np_complete_case}
%  The bounded language $L = (ab)(ba)^*(ab)$
%  gives an $\NP$-complete synchronization problem, but this is
%  not directly implied by our methods.
 
 \begin{propositionrep}\label{prop:np_complete_case}
  The problem $((ab)(ba)^*(ab))$\textsc{-Constr-Sync} is $\NP$-complete.
 \end{propositionrep}
 \begin{proof}
 We give a reduction from $\textsc{DisjointSetTransporter}$
 for unary input semi-automata, which is $\NP$-complete
 by Theorem~\ref{prop:set_transporter_np_complete}.
 Let $\mathcal A = (\{c\}, Q, \delta)$
 with $S, T \subseteq Q$ being disjoint.
 Construct the automaton $\mathcal A' = (\{a,b\}, Q', \delta')$
 with
 \[
  Q' = Q \cup \{ q_a \mid q \in Q \} \cup \{ q_b \mid q \in Q \} \cup \{ t \}. 
 \]
 Fix some $\hat s \in S$.
 Then, for $q \in Q$, set
 % loop q_a mit a, und q_b bei b
 % aber dass für reduktion von cd^*c - vielleicht auch bemerken.
%  \[
%   \delta'(q, x) = \left\{ 
%   \begin{array}{ll}
%   q_x  & \mbox{if } q \in Q; \\ 
%   q    & \mbox{if } q  
%   \end{array}
%   \right.
%  \]
\begin{align*}
    \delta'(t,   x) & = t \mbox{ for } x \in \Sigma \mbox{ and }
    \delta'(q,   x) = q_x \mbox{ for } x \in \Sigma; \\
    \delta'(q_b, b) & = t   \mbox{ for } q_b \in Q'  \mbox{ and } 
    \delta'(q_b, a) = \delta(q, c); \\
    \delta'(q_a, a) & = q_a \mbox{ for } q_a \in Q'; \\
    \delta'(q_a, b) & = \left\{ 
    \begin{array}{ll}
     \hat s & \mbox{if } q \in Q \setminus (T \cup S); \\
     q      & \mbox{if } q \in S; \\
     t      & \mbox{if } q \in T.
    \end{array}
    \right.
\end{align*}
 Then, there exists $n \ge 0$
 with $\delta(S, c^n) \subseteq T$
 if and only if $\mathcal A'$ has a synchronizing word in $L$.
 % S,T disjoint, deswegen ba mindestens einmal
 
 First, suppose there exists $n \ge 0$
 such that $\delta(S, c^n) \subseteq T$.
 By construction, $S \subseteq \delta'(Q', ab) \subseteq S \cup \{t\} \cup Q_b$,
 or more precisely $\delta'(Q', ab) = S \cup \{t\} \cup \{ q_b \mid q \in \delta(Q, c) \}$.
 Note that, as $S$ and $T$ are disjoint,
 we must have $n > 0$.
 As, for any $q \in Q$, $\delta'(q, ba) = \delta(q, c)$
 and $\delta(q_b, b) = t$,
 we find $\delta'(\delta'(Q', ab), (ba)^n) \subseteq T \cup \{t\}$,
 where we needed $n > 0$ to map those states in $\{ q_b \mid q \in \delta(Q, c) \}$
 to $T$.
 Finally, $\delta(T \cup \{t\}, ab) = \{t\}$
 and so $\delta'(Q', ab(ba)^nab) = \{t\}$.
 
 
 Conversely, suppose there exists $n \ge 0$
 such that $\delta'(Q', ab(ba)^nab)$
 is a singleton set. So, as $t$ is a sink state,
 $\delta'(Q', ab(ba)^nab) = \{ t \}$.
 By construction, a state in $Q'$ is mapped to $t$
 by $ab$
 if and only if it is contained in $T \cup \{t\}$.
 Hence, $\delta'(Q', ab(ba)^n) \subseteq T \cup \{t\}$.
 As before, $\delta'(Q', ab) = S \cup \{t\} \cup \{ q_b : q \in \delta(Q, c) \}$.
 In particular, we must have $\delta'(S, (ba)^n) \subseteq T \cup \{t\}$.
 As, for any $q \in Q$, $\delta'(q, ba) = \delta(q, c)$,
 this implies that $\delta'(S, (ba)^n) \subseteq T$
 and that for $u = c^n$ we have $\delta(S, c^n) \subseteq T$.
 
 
 By Theorem~\ref{thm:sparse_in_NP}, $L\textsc{-Constr-Sync}\in \NP$
 and by the above reduction the problem is $\NP$-complete.
 \end{proof}
%\end{remark}

 By Proposition~\ref{prop:np_complete_case},
 for the homomorphism $\varphi : \{a,b\}^* \to \{a,b\}^*$
 given by $\varphi(a) = ab$ and $\varphi(b) = ba$
 both problems $ab^*a$ and $\varphi(ab^*a) = ab(ba)^*ab$
 are \NP-complete. So, this is a homomorphisms
 which preserves, in this concrete instance, the computational complexity.
 But its image $\{ab,ba\}$ is not even a self-synchronizing code.\todo{Ich glaube in dem fall schon. die reduktion sollte immer gehen...}
 However, I do not know if this homomorphism always preserves the complexity.
 Similary, I do not know
 if the condition from Theorem~\ref{thm:constr_sync_hom_strongly_self_sync}
 characterizes those homomorphisms which preserve the complexity.
 %also shows that injective homomorphisms whose image forms a strongly
 %connected code do not characterize those homomorphisms



 In the reduction used in Lemma~\ref{lem:np_hardness}
 the resulting automaton has a sink state. However, in general, for questions
 of synchronizability it makes a difference if we have a sink state
 or not, at least with respect to the \v{C}ern\'y conjecture~\cite{Cer64},
 as for automata with a sink state this conjecture holds true,
 even with the better bound\footnote{In~\cite{DBLP:journals/tcs/Rystsov97}
 erroneously the bound $n(n+1)/2$ was reported as being sharp, but the overall argument
 in fact works to yield the sharp bound $n(n-1)/2$.}
 $\frac{n(n-1)}{2}$~\cite{DBLP:journals/tcs/Rystsov97,DBLP:journals/tcs/Volkov09}. However,
 even in~\cite{DBLP:conf/mfcs/FernauGHHVW19}
 certain reductions establishing \PSPACE-completeness
 use only automata with a sink state. Hence, for hardness
 these automata are sufficient at least in certain instances.
 So, it might be interesting to know
 if in terms of computational complexity of the CSP,
 %it makes no difference if we consider automata with or without a sink state.
 we can, without loss of generality, limit ourselves to input automata
 with a sink state. The methods of proof for the letter-bounded constraints
 show that in this case, we can actually do this, as these input automata
 are sufficient to establish all cases of intractability.
 

 Lastly, let us mention the following related problem\footnote{This was actually suggested
 by a reviewer of a previous version.} one could come up with.
 Fix a deterministic and complete semi-automaton~$\mathcal A$.
 Then, for input PDFAs~$\mathcal B$, what is the computational complexity to determine
 if $\mathcal A = (\Sigma, Q, \delta)$ has a synchronizing word in $L(\mathcal B)$?
 As the set of synchronizing words 
 $ \{ w \in \Sigma^* : |\delta(Q, w)| = 1 \} = \bigcup_{q \in Q} \bigcap_{q' \in Q} L((\Sigma, Q, \delta, q', \{q\})) $
 is a regular language, we have to test
 for non-emptiness of intersection of this fixed regular language 
 with $L(\mathcal B)$. This could be done in \NL, hence in \PTIME.
 

\smallskip \noindent {\footnotesize
\textbf{Acknowledgement.} I thank  anonymous reviewers
of a previous version for detailed feedback.
I also sincerely thank the reviewers of the current version (at least one reviewers saw both versions) for careful reading and giving valuable feedback to improve my scientific writing
and pointing to two instances were I overlooked, in retrospect, two simple conclusions.}





\bibliographystyle{splncs04}
\bibliography{ms} 
\end{document}