%
% abstract: sparse erwähnen, auch anwendungen erwähnen.


% einführungstext, möglichkeiten vielfältig.

% noch weitere zu denen in mfcs19
\begin{description}

\item[Reset State.] %ursprungsbeispiel, aber auch homogene experimente sternberg paper

In~\cite{DBLP:conf/mfcs/FernauGHHVW19} one motivating example was the demand that a system, or automaton thereof, to synchronize has to first enter a ``directing'' mode, perform a sequence of
operations, and then has to leave this operating mode and enter the ``normal
operating mode'' again. In the most simple case, this constraint
could be modelled by $ab^*a$, which, as it turns out~\cite{DBLP:conf/mfcs/FernauGHHVW19},
yields an \NP-complete CSP.
Even more generally, it might be possible that a system -- a remotely controlled
rover on a distant planet, a satellite in orbit, or a lost autonomous vehicle
%  or an autonomous driving car in the service
%of a taxi company
-- is not allowed to execute all commands in every possible
order, but  certain commands are only allowed in certain order or after
other commands have been executed. All of this imposes constraints
on the possible reset sequences.


% bestimmte operationen nur wenn bestimmte andere, oder erst vertikal
\item[Part Orienters.] Suppose parts arrive at a manufacturing site and they
need to be sorted and oriented before assembly. Practical considerations
favor methods which require little or no sensing, employ simple devices,
and are as robust as possible. This could be achieved as follows.
We put parts to be oriented on a conveyor belt which takes them to the assembly
point and let the stream of the parts encounter a series of passive obstacles placed along
the belt. Much research on synchronizing automata was motivated
by this application~\cite{DBLP:journals/algorithmica/ChenI95,DBLP:journals/siamcomp/Eppstein90,DBLP:journals/trob/ErdmannM88,DBLP:journals/algorithmica/Goldberg93,DBLP:conf/focs/Natarajan86,DBLP:journals/ijrr/Natarajan89,Vol2008}
and I refer to~\cite{Vol2008} for
an illustrative example. 
Now, furthermore, assume the passive components could not be placed at random along
the belt, but have to obey some restrictions, or restrictions
in what order they are allowed to happen. These could be due
to the availability of components, requirements how to lay things out
or physical restrictions. %Here, these additional constraint  den satz am ende.



\item[Supervisory Control.] The CSP
 could also be viewed of as %a special case of 
 supervisory control
 of a discrete event system (DES) that is given by an automaton
 and whose event sequence is modelled by a formal language~\cite{DBLP:books/daglib/0034521,RamadgeWonham87,wonham2019}.
 %Without giving the details~\cite{lerhbuch}, 
 %Actually, our framework fits quite nicely with this approach
 In this framework, a DES has a set of controllable
 and uncontrollable events.
 %, and a supervisor is allowed
 %to take action for controllable events, but is unable to 
 %see the uncontrollable ones. 
 Dependent
 on the event sequence that occurred so far,
 the supervisor is able to restrict the set of events
 that are possible in the next step, where, however,
 he can only limit the use of controllable events.
%  Then, the supervisor is able to restrict the set
%  of controllable events, dependent
%  on the event sequence that occurred so far,
%  that are possible as the set of events to occur next in the DES.
 So, if we want to (globally) reset a finite state DES~\cite{Alves2020} under supervisory control, 
 this is equivalent to CSP.


% noch etwas umschreiben, sonst identsich mit Mischa-Text
\item[Biocomputing.] In~\cite{Benenson2003,Benenson2001} DNA molecules have been used as both
hardware and software for finite automata of nanoscaling size, see also~\cite{Vol2008}. 
For instance, Benenson et al~\cite{Benenson2003} produced ``a `soup of automata', that is, a solution
containing $3 \times 10^{12}$ identical automata per $\mu1$. All these molecular
automata can work in parallel on different inputs, thus ending up in different
and unpredictable states. In contrast to an electronic computer, one cannot
reset such a system by just pressing a button; instead, in order to synchronously bring
each automaton to its start state, one should spice the soup with (sufficiently
many copies of) a DNA molecule whose nucleotide sequences encodes a reset word''~\cite{Vol2008}.
Now, it might be possible that certain sequences, or subsequences,
are not possible as they might have unwanted biological side-effects, 
or might destroy the molecules at all.

\item[Reduction Procedure.] % orignal paper
 This example is more formal and comes from attempts to solve the \v{C}ern\'y conjecture~\cite{Vol2008}.
 In~\cite{Gusev2012} a special rank factorization~\cite{piziak99}
 %\footnote{This notion
 %comes from linear algebra~\cite{kk}, but actually, it is 
 %an instantiation of the more general observation that every map
 %factorizes in a surjective and an injective map~\cite{alluff}.}
 for automata was introduced from which
 smaller automata could be derived.
 Then, it was shown that the original automaton
 is synchronizing if and only if the reduced automaton
 admits a synchronizing word in a certain regular constraint language,
 and the reset threshold, i.e, the lengths of the shortest
 synchronizing word, of the original automaton
 could be bounded by that of the shortest one in the constraint
 language for the reduced automaton.
 %In my opinion, this approach is particularly interesting
 %as every synchronizing automaton has to admit a letter of rank
 %strictly smaller than the number of states of the automaton, i.e., 
 %such 
 
\end{description}


% kapitel bounded and sparse regular languages



%
% volkov beispiel mit rampe und danach "loch"
% für "low", weil low ja immer ansetzt wenn gleich hcoh
%
% 1. rample sichert dass welche die random liegen an einem "fusspunkt"
% ausgerichtet. Wenn dannach nicht der kleine teil unten, dann geht
% es über loch, sonst fällt es in das loch. hinter dem loch ist
% noch ein fließband. zum weiterbewegen  wenn es im loch feststeckt? "die teile
% von hinten "drücken", oer noch eine vorrichtung, die immer "leert" (oben was
% entlangfährt?)
% 