\section{Unifying The Two Worlds}
\label{sec:framework}

Feedback emerges as both a complex ecosystem, and a rich, but systematized, information source, with many attributes covered in educational work. We consolidate several of these features, introduced in Section \ref{sec:background_pedagogy}, to create a novel feedback framework, which we subsequently adapt for LLMs.


\subsection{The Unified Framework}


From the multi-disciplinary background presented in Section \ref{sec:background_pedagogy}, we derive a tripartite feedback ecosystem structure consisting of task, learner, and feedback components, depicted in Figure \ref{fig:base_framework}.

\begin{figure}[ht]
\centering
\includegraphics[width=1.0\columnwidth]{figures/base_framework_final.png}
\caption{The feedback ecosystem. The feedback's characteristics, the task, and the learner all influence how effectively the feedback is received.}
\label{fig:base_framework}
\end{figure}

\paragraph{Learning Sciences Grounding}
Each component has a set of attributes that can vary depending on a situation requiring feedback. 
As discussed in Section \ref{sec:background_pedagogy}, feedback is influenced by both \textit{task} and \textit{learner characteristics}. We break each of these two components down to more granular constituents. For the \textit{task} we introduce two attributes, its complexity (\ie its difficulty level), and its nature --- which we reduce to it being closed-answer (where there is a single correct answer or a finite set of them) or open-ended. The \textit{learner} is similarly divided into two sub-components: their particular feedback processing mechanism, and their prior knowledge (dependent on the task).
We also develop a more holistic view of the \textit{feedback} component itself, with three main features:
\begin{enumerate}[itemsep=0.05em,label=(\roman*)]
    \item \textit{timing}, whether feedback is provided immediately or after a given temporal delay,
    \item \textit{content}, capturing both the type and the format of the information provided in a feedback message, explored fully in section \ref{sec:taxonomy},
    and,
    \item \textit{source}, whether the feedback stems from a peer or an authority figure, a human or an AI model, possibly including their relevant proficiency levels. Different sources usually build different relationships with the learner, and will communicate in different language \textit{styles}, a dimension explored in Section \ref{sec:taxonomy}.
\end{enumerate}

\paragraph{Framework Interactions} As Figure \ref{fig:base_framework} depicts, various interactions occur between the task, the learner, and the feedback. For example, as stated in Section \ref{sec:background_pedagogy}, \citet{mason_providing_2001} argue that if a learner has little prior knowledge or the task has a low complexity, feedback should be immediate, but if the task complexity and student's prior knowledge are both high, then it should be delayed. \citet{narciss_feedback_2008} on the other hand, refers to \citet{Mathan2005FosteringTI}'s take on timing, which states that if the learner possesses metacognitive skills (for identifying and correcting errors), then feedback should be delayed, to first promote these skills.
The type of task also naturally conditions the feedback given. 
Finally, all aspects of feedback impact how the learner receives and processes the feedback.
Appendix \ref{app:felt_interactions} presents a more comprehensive overview of the interactions present in the framework.


\subsection{The LLM Adaption -- FELT}

The unifying framework is directly derived from learning sciences research, and as such, designed for the human learner. We are, however, interested in providing feedback to LLMs. Thus, we adapt this ecosystem to LLMs, where the LLM is viewed as the learner. In particular, we propose five modifications to the base framework: 
expanding the learner to reflect the model's size as well as the model training data and method (including its training objectives), the expansion of the task component to reflect the prompt instructions, and finally, the addition of an error component. The resulting framework, FELT (Feedback, Errors, Learner, Task) is displayed in 
Figure \ref{fig:felt_framework}. 

\begin{figure}[ht]
\centering
\includegraphics[width=1.0\columnwidth]{figures/felt_framework_final.png}
\caption{FELT framework. The feedback ecosystem is specifically adapted for LLMs.}
\label{fig:felt_framework}
\end{figure}

\paragraph{Expanding the Learner} We add three specific LLM extensions to the generic learner component: model size, and model training data and method. All three are important, and together capture the \textit{Prior Knowledge} of a model. Model size is directly linked to emergent abilities \citep{wei2022emergent} and the model's ability to effectively leverage feedback \citep{scheurer2022training, bai_constitutional_2022}. The data the model was trained on, as well as how it was trained, similarly encode the model's initial abilities to tackle any given task.

\paragraph{Expanding the Task} Another important factor to incorporate into the framework pertains to the instructions given in the prompt. Past research has shown the importance of stating the actions a model can take, such as outputting ``I don't know.'' \citep{zhou2023contextfaithful}. Similarly, how strongly the prompt encourages a model to incorporate feedback can favor overoptimization.

\paragraph{Introducing Errors} Finally, effective feedback may communicate information on where the learner is failing, requiring an understanding of the possible error modes for a given task, and which ones the learner is likely in. For example, guessing and committing systematic reasoning mistakes are reflections of differing understandings. Exploring the error space and identifying the mistakes made by a learner is an important extension to the base framework directly derived from pedagogical and psychology of education research.

\subsection{Feedback Integration}

The method used to transmit the feedback to the model influences how it is subsequently processed. \citet{fernandes_bridging_2023} identify three common feedback integration mechanisms: feedback-based imitation learning, joint-feedback modeling, and reinforcement learning. In addition to this, we also consider feedback use in in-context learning \citep{brown2020language}. The training objective will necessarily influence how the model is processing and incorporating feedback.
Typically, the training relies upon either scalar feedback (a single number encoding how much the model should be rewarded for its output) or a ranking (how well a given output did in relation to other candidate answers). However, this is simple information, and does not leverage the rich and complex information encoded in natural language feedback. 
Section \ref{sec:taxonomy} therefore comprehensively explores the different types of information that can be encoded in feedback.

