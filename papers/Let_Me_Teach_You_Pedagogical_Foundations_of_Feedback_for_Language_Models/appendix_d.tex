\section{Case Study Implementation Details}
\label{app:case_study}

\subsection{Dataset and Model}
The task presented in section \ref{sec:case_study} involved the summarizing of a news article describing a research development at EPFL in more approachable language and terminology. The summary had as an objective to be even more approachable to people outside the field. 

\paragraph{Dataset} The data for this task came from EPFL's Mediacom department, where they provided the authors with a set of 2370 entries of articles, summaries, and extra information (title, author, date, etc.). Out of these, 50 were chosen so that all articles relayed a newly published work. This was the only criteria for selection. All articles and summaries were pre-processed so as to remove HTML tags.

\paragraph{Model} The model used for this task was GPT-4 \citep{openai2023gpt4}, with its default hyperparameters, called through OpenAI's Chat Completions API. Thus, the model generated a single completion for each prompt, with a temperature of $1$, and with no limitation on the maximum number of tokens (beyond the model's own context length).

\subsection{Experiment Execution}
The experiment was run in two stages. In the initial phase, the model is asked to generate a first summary. It is then provided with feedback and asked to revise its original summary. In this experiment, two distinct types of feedback were provided: \textit{Task Elaboration} (TE) and \textit{Correct Answer} (CA).

\paragraph{Initial Generation}
Following OpenAI's Chat Completions API, the model prompting is done under a chat format. In this setting, the first \textit{message} is a system message stating \texttt{You are a helpful assistant.} This is then followed by an user message, with the following prompt: 

\texttt{Summarize the following article into a short but captivating snippet under around 100 tokens. It must describe both the problem and the approach used to solve it, as well as the venue where these findings were presented, whenever this information is available. \\
Article: [article body]}

The model's response message to this prompt is considered its original summary.

\paragraph{Revised Generation} The revised generation prompt is shares the chat prompting format. It contains the previous chat history, which includes not only the two messages outlined above but also the model's answer as an assistant message. To these three messages, a new user message is added, with the following content:

\texttt{Feedback: [feedback] \\
Please revise your original summary taking the feedback into consideration. If you feel the feedback is not appropriate or useful, you can disregard it.}

The \texttt{[feedback]} placeholder will have one of two different values, depending on the feedback type being provided:
\begin{itemize}
    \item \textbf{Correct Answer:} The feedback will be of the form \\
    \texttt{The correct answer is: [gold\_summary]} \\
    where \texttt{[gold\_summary]} is the summary provided by the Mediacom dataset,
    
    \item \textbf{Task Elaboration:} The feedback will be of the form \\
    \texttt{A good and captivating summary should first grab the reader's attention and make them curious to learn more. It is then important to factually and precisely state what the problem is, why it is important, the proposed solution, and, if published or being divulged, disclose where the reader can find it.}
\end{itemize}

Finally, as in the first stage, the model's response message to this prompt is considered its revised summary.

\subsection{Example Outputs}
In this subsection, we present a few examples outputs from the case study.

\subsubsection{Example 1}
\paragraph{\colorbox{YellowOrange}{Original Article}}
\texttt{The International Consortium of Investigative Journalists (ICIJ), which has over 200 members in 70 countries, has broken a number of important stories, particularly ones that expose medical fraud and tax evasion. One of its most famous investigations was the Panama Papers, a trove of millions of documents that revealed the existence of several hundred thousand shell companies whose owners included cultural figures, politicians, businesspeople and sports personalities. To complete an investigation of this size is only possible through international cooperation between journalists. When sharing such sensitive files, however, a leak can jeopardize not only the story’s publication, but also the safety of the journalists and sources involved. At the ICIJ’s behest, EPFL’s Security and Privacy Engineering (SPRING) Lab recently developed Datashare Network, a fully anonymous, decentralized system for searching and exchanging information. A paper about it will be presented during the Usenix Security Symposium, a worldwide reference for specialists, which will be held online from 12 to 14 August. \\
Anonymity at every stage \\
Anonymity is the backbone of the system. Users can search and exchange information without revealing their identity, or the content of their queries, either to colleagues or to the ICIJ. The Consortium ensures that the system is running properly but remains unaware of any information exchange. It issues virtual secure tokens that journalists can attach to their messages and documents to prove to others that they are Consortium members. A centralized file management system would be too conspicuous a target for hackers; since the ICIJ does not have servers in various jurisdictions, documents are typically stored on its members’ servers or computers. Users provide only the elements that enable others to link to their investigation. \\
Users searching for information enter keywords in the search engine. If the search produces hits, they can then contact colleagues – whose identity remains protected – who are in possession of potentially relevant documents. Search queries are sent encrypted to all users, if there is a macth the querier gets an alert and can decide whether they wish to enter in contact and share information. “Given the fact that users work in different time zones, some with only a few hours of internet access per day, it was critical that searches and responses could take place asynchronously,” notes Carmela Troncoso, who runs the SPRING Lab at the School of Computer and Communication Sciences (IC). Another messaging system, also secure and anonymous, is subsequently used for two-way exchanges. \\
Two completely new secure applications \\
“This system, which addresses real-world needs, has enabled SPRING to tackle some interesting challenges,” notes Troncoso. The research team drew on existing authentication mechanisms and anonymous communication primitives, which they then optimized. They also developed two completely new secure building blocks that are described in the paper being published today in connection with the Usenix Security Symposium – an asynchronous search engine and a messaging system. A new protocol, known as “multi-set private set intersection” (MS-PSI), ensures the security of the search engine, allowing users to easily search a large number of databases without increasing the risk of leaks. The messaging system relies on a large number of single-use virtual mailboxes and is based on the well-known “pigeonhole” system, which chooses one option at random, in this case one of the mailboxes. Currently, the system does not allow users to exchange documents. “At this stage in the process, journalists are using other secure messaging systems,” Troncoso says. \\
Working with the Consortium has allowed SPRING to frame new requirements that are rarely examined in the scientific literature. Datashare can be scaled to thousands of users and millions of documents while encrypting all communications. “The hurdles we encountered during the development process, however, have paved the way to a new area of research with significant potential for other fields,” Troncoso concludes.}


\paragraph{\colorbox{yellow}{Golden Summary}}
\texttt{An EPFL laboratory has developed Datashare Network, a decentralized search engine paired with a secure messaging system that allows investigative journalists to exchange information securely and anonymously. An scientific article on this subject will be presented during the Usenix Security Symposium which will be held online from August 12 to 14.}


\paragraph{\colorbox{pink}{GPT4 Original Summary}}
\texttt{EPFL's Security and Privacy Engineering Lab developed an anonymous, decentralized system called Datashare Network to aid the International Consortium of Investigative Journalists securely exchange information. Offering user anonymity, this system will be presented at the Usenix Security Symposium held online from 12-14 August.}

\paragraph{\colorbox{SkyBlue}{\textit{Correct Answer} Revision}}
\texttt{EPFL's Security and Privacy Engineering Lab created Datashare Network, a decentralized search engine with a secure messaging system that enables investigative journalists to exchange information anonymously and securely. A scientific paper on this solution will be showcased at the online Usenix Security Symposium from August 12 to 14.}

\paragraph{\colorbox{SpringGreen}{\textit{Task Elaboration} Revision}}
\texttt{Investigative journalism goes undercover with Datashare Network, a breakthrough anonymous, decentralized information-sharing system. Developed by EPFL's Security and Privacy Engineering Lab, the network protects journalists from exposure while tackling global issues like tax evasion and medical fraud. Discover how it secures data-sharing at the Usenix Security Symposium, held online from 12-14 August.}



\subsubsection{Example 2}
\paragraph{\colorbox{YellowOrange}{Original Article}}
\texttt{Thermoelectric materials in particular hold vast potential for use in energy applications because they generate electricity from waste heat, such as that generated by industrial processes, by car and truck engines, or simply by the sun. Reducing the thermal conductivity of these materials by a factor of three, for example, would completely revolutionize existing waste-heat recovery, and also all refrigeration and air-cooling technology. \\
A unique theory for all insulating materials \\
In the paper Unified theory of thermal transport in crystals and glasses, out in Nature Physics, Michele Simoncelli, a PhD student at EPFL’s Theory and Simulation of Materials (THEOS) Laboratory – together with Nicola Marzari, a professor at EPFL’s School of Engineering and head of THEOS and of the MARVEL NCCR, and Francesco Mauri, a professor at the University of Rome–Sapienza – present a novel theory that finally decodes the fundamental, atomistic origin of heat conduction. Up to now, different formulations needed to be used depending on the systems studied (e.g., ordered materials, like a silicon chip, or disordered, like in a glass), and there wasn’t a unified picture covering all possible cases. \\
This has now been made possible by deriving directly from the quantum mechanics of dissipative systems a transport equation that covers on equal footing diffusion, hopping, and tunneling of heat.
Waste heat recovery \\
This fundamental understanding will allow scientists and engineers to accurately predict the thermal conductivity of any insulating material (in metals, the heat is carried by the electrons, and that is well understood) – this is exceedingly important for thermoelectrics (i.e. materials that can convert heat into electricity), since these have both crystal- and glass-like properties, and are much needed for waste-heat recovery, or for refrigeration without greenhouse gases (and if you think refrigeration is boring, it is worth remembering that Albert Einstein spent many years trying to invent a new form of refrigerator). \\
In order to develop such next-generation technology, however, scientists first need to understand how and to what extent materials conduct heat. “Up to now, two different equations have been used for calculating thermal properties: one describes perfectly crystalline materials – that is, materials with highly ordered atomic structures – and the other one completely amorphous materials like glass, whose atoms do not follow an ordered pattern,” says Michele Simoncelli. These equations happened to work well in those special cases. “But between these two extremes lie a plethora of interesting cases, and neither equation worked – this is really where our contribution makes a profound difference” \\
* Illustration : Heat is the energy due to vibrations of atoms. The ridges represent the quantum vibrational energy levels of CsPbBr3, a promising thermoelectric material. Heat conduction originates from both diffusion of vibrational excitations (blurred spheres, following the energy level) and quantum tunnelling (jump between different energy levels, represented by blue waves).}

\paragraph{\colorbox{yellow}{Golden Summary}}
\texttt{EPFL and MARVEL researchers have developed a new theory for heat conduction that can finally describe and predict the thermal conductivity of any insulating material. This new formulation will let scientists make accurate predictions of thermal conductivity in a range of materials for critical engineering applications – from electronics to lasers to waste-heat recovery.}

\paragraph{\colorbox{pink}{GPT4 Original Summary}}
\texttt{Scientists at EPFL's Theory and Simulation of Materials Laboratory have developed a unified theory to predict the thermal conductivity of insulating materials, opening up possibilities for more efficient energy applications. The research breakthrough, published in Nature Physics, could lead to waste-heat recovery, revolutionising refrigeration and air-cooling technology without greenhouse gases. Understanding how materials conduct heat is key to developing next-generation thermoelectric technology.}

\paragraph{\colorbox{SkyBlue}{\textit{Correct Answer} Revision}}
\texttt{Researchers at EPFL and MARVEL have developed a groundbreaking theory for heat conduction, enabling the description and prediction of thermal conductivity in any insulating material. Published in Nature Physics, the new formulation offers potential for critical engineering applications such as waste-heat recovery, electronics, and lasers, enhancing energy efficiency in various industries.}

\paragraph{\colorbox{SpringGreen}{\textit{Task Elaboration} Revision}}
\texttt{Researchers have unlocked the secret to more efficient energy applications by developing a unified theory for predicting the thermal conductivity of insulating materials, transforming waste-heat recovery and revolutionizing refrigeration technology. With this innovative approach, published in Nature Physics, scientists can now accurately predict and optimize thermoelectric materials, paving the way for energy-efficient solutions without greenhouse gas emissions. Discover the full potential of this game-changing breakthrough at EPFL's Theory and Simulation of Materials Laboratory.}




\subsubsection{Example 3}
\paragraph{\colorbox{YellowOrange}{Original Article}}
\texttt{With its 500 km diameter, the asteroid Vesta is one of the largest known planet embryos. It came into existence at the same time as the Solar System. Spurring scientific interest, NASA sent the Dawn spacecraft on Vesta’s orbit for one year between July 2011 and July 2012. \\
Data gathered by Dawn were analyzed by a team of researchers from EPFL as well as the Universities of Bern (Switzerlanf), Brittany (France) and Arizona (USA). Conclusion : the asteroid's crust is almost three times thicker than expected. The study does not only have implications for the structure of this celestial object, located between Mars and Jupiter. Their results challenge a fundamental component in planet formation models, namely the composition of the original cloud of matter that aggregated together, heated, melted and then crystallized to form planets. \\
At EPFL’s Earth and Planetary Science Laboratory (EPSL), led by Philippe Gillet, Harold Clenet had a look at the composition of the rocks scattered across Vesta’s ground. "What is striking is the absence of a particular mineral, olivine, on the asteroid’s surface," said the researcher. Olivine is a main component of planetary mantles and should have been found in large quantities on the surface of Vesta, due to a double meteorite impact which, according to computer simulations, "dug" the celestial body’s southern pole to a depth of 80 km, catapulting large amounts of materials to the surface. \\
The two impacts were so powerful that more than 5 \% of Earth's meteorites come from Vesta. « But these cataclysms were not strong enough to pierce through the crust and reach the asteroid's mantle,» Harold Clenet continued. The meteorites originating from Vesta and found on Earth confirm this since they generally lack Olivine, or contain only minute amounts compared to the amount observed in planetary mantles. Also, the spacecraft Dawn did not find olivine in the vicinity of the two impact craters. « This means that the crust of the asteroid is not 30 km thick, as suggested by the models, but more than 80 km. » \\
Composition of planets \\
These discoveries challenge models that describe the formation of Vesta, and consequently the formation of rocky planets in the Solar System including planet Earth. Cooling theory and "re- melting" phenomena in the depths of previously solidified elements would also need to be reviewed. "The crust might have been thickened by the formation of "plutons", that is: igneous rock intrusions, hundreds of meters large, some of which emerged to the surface," explained the scientist. \\
If Vesta has less of a (olivine-rich) mantle and more of a (pyroxene-rich) crust, then the proportion of materials making up Vesta, and probably the Earth and other telluric planets (Mars, Venus, Mercury), is different from what was previously expected. \\
A more complex model of planet formation therefore has to be considered, one that takes into account not only the original composition of planets, but also their orbits, sizes and related cooling times. Vesta is the only known asteroid that has an earth-like structure - with a core, mantle and crust – making it an incredible laboratory for testing hypotheses and theories.}

\paragraph{\colorbox{yellow}{Golden Summary}}
\texttt{EPFL researchers have a better understanding of the asteroid Vesta and its internal structure, thanks to numerical simulations and data from the space mission Dawn. Their findings, published today in Nature, question contemporary models of rocky planet formation, including that of Earth.}

\paragraph{\colorbox{pink}{GPT4 Original Summary}}
\texttt{Data collected from asteroid Vesta poses challenges to planet formation models as it reveals the asteroid's crust is almost three times thicker than expected. The discovery, made by researchers from the Earth and Planetary Science Laboratory and partner universities, suggests that original assumptions about the composition of the cloud of matter that formed planets may be incorrect. The findings could lead to a more complex model of planet formation, taking into account composition, orbit, size, and cooling times.}

\paragraph{\colorbox{SkyBlue}{\textit{Correct Answer} Revision}}
\texttt{EPFL researchers have gained insights into asteroid Vesta's internal structure, utilizing numerical simulations and data from the Dawn space mission. The findings, published in Nature, indicate Vesta's crust is almost three times thicker than expected, questioning current models of rocky planet formation, including Earth's. This could lead to more complex planet formation models that consider composition, orbit, size, and cooling times.}

\paragraph{\colorbox{SpringGreen}{\textit{Task Elaboration} Revision}}
\texttt{Astonishing findings from asteroid Vesta are challenging our understanding of planet formation, as researchers discovered its crust is nearly triple the expected thickness. This breakthrough, made by EPFL's Earth and Planetary Science Laboratory and partner universities, questions key elements of planet formation models and paves the way for a more complex approach, considering composition, orbits, sizes, and cooling times. With Vesta being the only known asteroid with an Earth-like structure, this revelation creates a compelling lure for further explorations.}