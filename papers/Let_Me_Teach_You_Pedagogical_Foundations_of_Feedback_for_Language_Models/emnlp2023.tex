% This must be in the first 5 lines to tell arXiv to use pdfLaTeX, which is strongly recommended.
\pdfoutput=1
% In particular, the hyperref package requires pdfLaTeX in order to break URLs across lines.

\documentclass[11pt]{article}

% Remove the "review" option to generate the final version.
\usepackage{authblk}
\usepackage{EMNLP2023}

% Standard package includes
\usepackage{times}
\usepackage{latexsym}

% For proper rendering and hyphenation of words containing Latin characters (including in bib files)
\usepackage[T1]{fontenc}
% For Vietnamese characters
% \usepackage[T5]{fontenc}
% See https://www.latex-project.org/help/documentation/encguide.pdf for other character sets

% This assumes your files are encoded as UTF8
\usepackage[utf8]{inputenc}

% This is not strictly necessary and may be commented out.
% However, it will improve the layout of the manuscript,
% and will typically save some space.
\usepackage{microtype}

% This is also not strictly necessary and may be commented out.
% However, it will improve the aesthetics of text in
% the typewriter font.
\usepackage{inconsolata}

\usepackage{enumerate}
\usepackage{enumitem}
\usepackage{graphicx}
\usepackage{tabularx}
\usepackage{booktabs}

%TODO
\newcommand{\todo}[1]{{\color{red}{\bf [TODO]:~{#1}}}}

%THEOREMS
\newtheorem{theorem}{Theorem}
\newtheorem{corollary}{Corollary}
\newtheorem{lemma}{Lemma}
\newtheorem{proposition}{Proposition}
\newtheorem{problem}{Problem}
\newtheorem{definition}{Definition}
\newtheorem{remark}{Remark}
\newtheorem{example}{Example}
\newtheorem{assumption}{Assumption}

%HANS' CONVENIENCES
\newcommand{\define}[1]{\textit{#1}}
\newcommand{\join}{\vee}
\newcommand{\meet}{\wedge}
\newcommand{\bigjoin}{\bigvee}
\newcommand{\bigmeet}{\bigwedge}
\newcommand{\jointimes}{\boxplus}
\newcommand{\meettimes}{\boxplus'}
\newcommand{\bigjoinplus}{\bigjoin}
\newcommand{\bigmeetplus}{\bigmeet}
\newcommand{\joinplus}{\join}
\newcommand{\meetplus}{\meet}
\newcommand{\lattice}[1]{\mathbf{#1}}
\newcommand{\semimod}{\mathcal{S}}
\newcommand{\graph}{\mathcal{G}}
\newcommand{\nodes}{\mathcal{V}}
\newcommand{\agents}{\{1,2,\dots,N\}}
\newcommand{\edges}{\mathcal{E}}
\newcommand{\neighbors}{\mathcal{N}}
\newcommand{\Weights}{\mathcal{A}}
\renewcommand{\leq}{\leqslant}
\renewcommand{\geq}{\geqslant}
\renewcommand{\preceq}{\preccurlyeq}
\renewcommand{\succeq}{\succcurlyeq}
\newcommand{\Rmax}{\mathbb{R}_{\mathrm{max}}}
\newcommand{\Rmin}{\mathbb{R}_{\mathrm{min}}}
\newcommand{\Rext}{\overline{\mathbb{R}}}
\newcommand{\R}{\mathbb{R}}
\newcommand{\N}{\mathbb{N}}
\newcommand{\A}{\mathbf{A}}
\newcommand{\B}{\mathbf{B}}
\newcommand{\x}{\mathbf{x}}
\newcommand{\e}{\mathbf{e}}
\newcommand{\X}{\mathbf{X}}
\newcommand{\W}{\mathbf{W}}
\newcommand{\weights}{\mathcal{W}}
\newcommand{\alternatives}{\mathcal{X}}
\newcommand{\xsol}{\bar{\mathbf{x}}}
\newcommand{\y}{\mathbf{y}}
\newcommand{\Y}{\mathbf{Y}}
\newcommand{\z}{\mathbf{z}}
\newcommand{\Z}{\mathbf{Z}}
\renewcommand{\a}{\mathbf{a}}
\renewcommand{\b}{\mathbf{b}}
\newcommand{\I}{\mathbf{I}}
\DeclareMathOperator{\supp}{supp}
\newcommand{\Par}[2]{\mathcal{P}_{{#1} \to {#2}}}
\newcommand{\Laplacian}{\mathcal{L}}
\newcommand{\F}{\mathcal{F}}
\newcommand{\inv}[1]{{#1}^{\sharp}}
\newcommand{\energy}{Q}
\newcommand{\err}{\mathrm{err}}
\newcommand{\argmin}{\mathrm{argmin}}
\newcommand{\argmax}{\mathrm{argmax}}

\title{Let Me Teach You: \\ Pedagogical Foundations of Feedback for Language Models}

\author[1]{\textbf{Beatriz Borges}}
\author[2]{\textbf{Niket Tandon}}
\author[1]{\textbf{Tanja Käser}}
\author[1]{\textbf{Antoine Bosselut}}
\affil{EPFL \:\:\: $^2$Allen Institute for Artificial Intelligence}
\affil[ ]{\texttt{\{beatriz.borges, antoine.bosselut\}@epfl.ch}}


\begin{document}
\maketitle
\begin{abstract}
Natural Language Feedback (NLF) is an increasingly popular avenue to align Large Language Models (LLMs) to human preferences. Despite the richness and diversity of the information it can convey, NLF is often hand-designed and arbitrary. In a different world, research in pedagogy has long established several effective feedback models. 
In this opinion piece, we compile ideas from pedagogy to introduce \ours, a feedback framework for LLMs that outlines the various characteristics of the feedback space, and a feedback content taxonomy based on these variables. 
Our taxonomy offers both a general mapping of the feedback space, as well as pedagogy-established discrete categories, allowing us to empirically demonstrate the impact of different feedback types on revised generations. In addition to streamlining existing NLF designs, \ours also brings out new, unexplored directions for research in NLF. We make our taxonomy available to the community, providing guides and examples for mapping our categorizations to future resources.
\end{abstract}

3D human pose estimation has ubiquitous applications in sport analysis, human-computer interaction, and fitness and dance teaching. While there has been remarkable progress in 3D pose estimation from a monocular image or video~\cite{hmrKanazawa17, Moon_2020_ECCV_I2L-MeshNet, kolotouros2019spin, kocabas2019vibe, xiang2019monocular}, inevitable challenges such as the depth ambiguity and the self-occlusion are still unsolved. 



\begin{figure}
     \centering
     \begin{subfigure}[h]{0.23\textwidth}
         \centering
         \includegraphics[width=\textwidth]{figures/cover/image-comp.jpg}
         \caption*{Input image}
     \end{subfigure}
     \begin{subfigure}[h]{0.23\textwidth}
         \centering
         \includegraphics[width=\textwidth]{figures/cover/smplify-comp.jpg}
         \caption*{SMPLify-X~\cite{SMPL-X:2019}}
     \end{subfigure}
     \vspace*{0.2cm}
     \begin{subfigure}[h]{0.45\textwidth}
         \centering
         \includegraphics[width=0.98\linewidth, trim=25 50 25 50]{figures/cover/scene_cover_green-comp.png}
         \caption*{3D visualization of our (left) and SMPLify-X (right) results}
     \end{subfigure}
     \vspace*{-0.2cm}
     \caption{While the state-of-the-art single-view 3D pose estimator~\cite{SMPL-X:2019} yields a small reprojection error, the recovered 3D poses may be erroneous due to the depth ambiguity. We make use of the mirror in the image to resolve the ambiguity and reconstruct more accurate human pose as well as the mirror geometry.}
     \vspace*{-0.5cm}
    \label{fig:demo1}
\end{figure}



In many scenes like dancing rooms and gyms, people are often in front of a mirror. In this case, we are able to see the person and his/her mirror image simultaneously. The mirror image actually provides an additional virtual view of the person, which can resolve the single-view depth ambiguity if the mirror is properly placed. Moreover, unseen part of the person can also be observed from the mirror image, so that the occlusion problem can be alleviated. 


In this paper, we investigate the feasibility of leveraging such mirror images to improve the accuracy of 3D human pose estimation. We develop an optimization-based framework with mirror symmetry constraints that are applicable without knowing the mirror geometry and camera parameters. We also provide a method to utilize the properties of vanishing points to recover the mirror normal along with the camera parameters, so that an additional mirror normal constraint can be imposed to further improve the human pose estimation accuracy. The effectiveness of our framework is validated on a new dataset for this new task with 3D pose ground-truth provided by a multi-view camera system. 


An important application of the proposed approach is to generate pseudo ground-truth annotations to train existing 3D pose estimators. To this end, we collect a large-scale set of Internet images that contain people and mirrors and generate 3D pose annotations with the proposed optimization method. The dataset is named Mirrored-Human.  
Compared with existing 3D human pose datasets~\cite{h36m_pami,mono-3dhp2017,vonMarcard2018} that are captured with very few subjects and background scenes, Mirrored-Human has a significantly larger diversity in human poses, appearances and backgrounds, as shown in Fig.~\ref{fig:dataset}. The experiments show that, by combining Mirrored-Human with existing datasets as training data, both accuracy and generalizability of existing 3D pose estimation methods can be significantly improved for both single-person and multi-person cases.   

In summary, we make the following contributions:
\begin{itemize}
    \item We introduce a new task of reconstructing  human pose from a single image in which we can see the person and the person's mirror image. 
    \item We develop a novel optimization-based framework with mirror symmetry constraints to solve this new task, as well as a method to recover mirror geometry from a single image.
    \item We collect a large-scale dataset named Mirrored-Human from the Internet, provide our reconstructed 3D poses as pseudo ground-truth, and show that training on this new dataset can improve the performance of existing 3D human pose estimators. 
\end{itemize}







%\outline{DRAM Background}

{We provide the relevant background on DRAM organization, DRAM operation and commodity DRAM based PuM techniques. We refer the reader to prior works for more comprehensive background about DRAM organization and operation~\cite{salp,lee.hpca13,donghyuk-ddma,chang.sigmetrics17,ghose2018vampire,patel2017reaper,luo2020clr,ghose2019demystifying,kevinchang-thesis,yoongu-thesis,lee.thesis16,olgun2021quactrng}.}

\subsection{DRAM Background}
\label{sec:background-dram}
DRAM-based main memory is organized hierarchically. \new{Fig.~\ref{fig:dram-bank-timing-diagram} (top) depicts this organization.} A processor is connected to one or \newnew{more} {memory channels \new{(DDRx in the figure)~\boldone{}}}. Each channel has its own command, address, and data buses. Multiple {memory modules} can be plugged into a single channel. Each module contains several {DRAM chips}\new{~\boldtwo{}. Each chip} contains multiple {\new{DRAM} banks} \new{that can be accessed} independently\new{~\boldthree{}}.\revdel{A set of DDRx standards cluster multiple {banks} in {bank groups}~\cite{jedecDDR4,gddr5}.}
\newnew{D}ata transfers between DRAM memory modules and processors occur at {cache block} granularity. The cache block size is typically 64 bytes in \atb{current} systems.


\begin{figure}[h]
     \centering
     
     \begin{subfigure}[h]{.50\textwidth}
         \centering
         \includegraphics[width=\textwidth]{figures/dram-bank.pdf}
     \end{subfigure}
     \hfill
     \begin{subfigure}[h]{.45\textwidth}
         \centering
         \includegraphics[width=\textwidth]{figures/timing-diagram.pdf}
     \end{subfigure}
    
    \caption{DRAM organization (top). Timing diagram of \new{DRAM} commands (bottom).}
    
    \label{fig:dram-bank-timing-diagram}
\end{figure}

\new{Inside a DRAM bank, DRAM cells are laid out \newnew{as} a two dimensional array of wordlines \new{(i.e., DRAM rows)} and bitlines \newnew{(i.e., DRAM columns)~\boldfour{}. W}ordlines are depicted in blue and bitlines are depicted in red in Fig.~\ref{fig:dram-bank-timing-diagram}. Wordline drivers drive the wordlines and sense amplifers read the values on the bitlines.} \newnew{A DRAM cell is connected to a bitline via an access transistor\new{~\boldfive{}}}. When enabled, an access transistor allows charge to flow between a DRAM cell and the cell's bitline.

\noindent
\new{\textbf{DRAM Operation.}} When all DRAM rows in a {bank} are closed, DRAM bitlines are precharged to \newnew{a reference voltage level of} {$\frac{V_{DD}}{2}$}. The memory controller sends an activate ($ACT$) command \new{to the DRAM module} to drive a DRAM wordline (i.e., enable a DRAM row). Enabling a DRAM row starts the {charge sharing} process. \newnew{Each DRAM cell connected to the DRAM row starts sharing its charge with its bitline}. This \omi{causes} \newnew{the bitline voltage} to deviate from {$\frac{V_{DD}}{2}$} {(i.e., the charge in the cell perturbs the bitline voltage)}. The \new{sense amplifier} sense\newnew{s} the deviation in the bitline and amplif\newnew{ies} the voltage of the bitline either to {$V_{DD}$} or to $0$. \newnew{As such}, \newnew{an ACT command} copies one DRAM row to the \new{sense amplifiers} (i.e., row buffer). The memory controller can send READ\newnew{/}WRITE commands to transfer data from/to {the sense amplifier array}. {Once the memory controller needs to access another DRAM row, t}he memory controller can close the {enabled} DRAM row by sending a precharge (PRE) command on the command bus. The PRE command first disconnects DRAM cells from their bitlines by disabling the enabled wordline and then precharges the bitlines to {$\frac{V_{DD}}{2}$}.

%\outline{Timing Parameters}
\noindent
\new{\textbf{DRAM Timing Parameters.}} DRAM datasheets specify a set of timing parameters that define the minimum time window between valid combinations of DRAM commands~\cite{lee.hpca15, kevinchang-thesis, chang.sigmetrics16, kim2018solar}. The memory controller must wait for tRCD, tRAS, and tRP nanoseconds between successive ACT $\rightarrow$ RD, ACT $\rightarrow$ ACT, and PRE $\rightarrow$ ACT commands, respectively {(Figure~\ref{fig:dram-bank-timing-diagram}, bottom)}. Prior works show that these timing parameters can be violated (e.g., successive ACT $\rightarrow$ RD commands may be issued with a shorter time window than tRCD) to improve DRAM access latency~\cite{lee.hpca15, chang.sigmetrics16, kevinchang-thesis, lee.sigmetrics17, kim2018solar}, implement physical unclonable functions~\cite{kim.hpca18,talukder2019exploiting,orosa2021codic}, \newnew{generate true random numbers~\cite{olgun2021quactrng,olgun2021quactrngieee,kim.hpca19},} copy data~\cite{seshadri2013rowclone,gao2020computedram}, and perform bitwise AND/OR operations~\cite{seshadri.micro17,seshadri.thesis16,seshadri.arxiv16,seshadri.bookchapter17.arxiv,gao2020computedram} in commodity DRAM devices.

\noindent
\textbf{\new{{DRAM Internal Address Mapping.}}}
\Copy{R2/1C}{{DRAM manufacturers use DRAM-internal address mapping schemes~\cite{salp,cojocar2020susceptible,patel2022case} to translate from logical (e.g., row, bank, column) DRAM addresses \newnew{that are used by the memory controller} to physical DRAM addresses \newnew{that are internal to the DRAM chip} (e.g., the \newnew{physical} position of a DRAM row within the chip). These schemes allow (i) post-manufacturing row repair techniques to map erroneous DRAM rows to redundant DRAM rows and (ii) DRAM manufacturers to organize DRAM internals in a cost-efficient \newnew{and reliable} way~\cite{khan.dsn16,vandegoor2002address}. DRAM-internal address mapping schemes can be substantially different across different DRAM chips~\cite{barenghi2018software,cojocar2020susceptible,horiguchi1997redundancy,itoh2013vlsi,keeth2001dram,khan.dsn16,khan.micro17,kim-isca2014,lee.sigmetrics17,liu.isca13,orosaYaglikci2021deeper,saroiu2022price,patel2020bit,patel2022case}. Thus, \newnew{consecutive} logical DRAM row addresses might not point to physical DRAM rows in the same subarray.}}
%\outline{in-DRAM Computation: RowClone, D-RaNGe, AMBIT, ...}

%\todo{Juan: Good for the readers to get an idea on how PIM techniques work: AMBIT, RowClone... Recent effort on doing computation in DRAM.}

\subsection{PuM Techniques}

\label{sec:background_pudram}
Prior work proposes a variety of in-DRAM computation mechanisms (i.e., PuM techniques) that (i) have great potential to improve system performance and energy efficiency~\cite{chang.hpca16,seshadri.micro17, hajinazarsimdram,seshadri2013rowclone,seshadri2020indram,seshadri.bookchapter17.arxiv,seshadri.bookchapter17,seshadri.arxiv16,Seshadri:2015:ANDOR,angizi2019graphide, ferreira2021pluto} %and 
\juan{or} (ii) can provide low-cost security primitives~\cite{talukder2019prelatpuf,talukder2019exploiting,kim.hpca19,kim.hpca18,olgun2021quactrngieee,orosa2021codic}. A subset of these in-DRAM computation mechanisms are demonstrated on real DRAM chips~\cite{gao2020computedram,kim.hpca19, kim.hpca18, talukder2019exploiting,olgun2021quactrngieee,orosa2021codic}. {We describe the \newnew{major relevant} prior works \omi{briefly}}: 

%\textbf{RowClone~\cite{seshadri2013rowclone}} proposes minor modifications to DRAM arrays to enable bulk data copy \atb{and initialization} inside DRAM. \atb{RowClone activates two DRAM rows successively without precharging the bitlines. By the time the second row is activated, the bitlines are loaded with the data in the first row. Bulk-copy and initialization intensive workloads (e.g., fork, memcached~\cite{memcached}) can significantly benefit from the increase in copy and initialization throughput provided by RowClone.}

%\outline{Explain ComputeDRAM}

%\todo{This should come later (after we discuss AMBIT/RowClone). Or we can talk about it in the later sections.}
%\jgl{I think this section is good, after having introduced RowClone/Ambit. We do not need many details about these works here, since we can give more background in the specific use case.}

\noindent
\newnew{\textbf{RowClone~\cite{seshadri2013rowclone}} \omi{is} a low-cost DRAM architecture that can perform bulk data movement operations (e.g., copy, initialization) inside DRAM chips \omi{at high performance and low energy}.}

\noindent
\newnew{\textbf{Ambit~\cite{Seshadri:2015:ANDOR, seshadri.arxiv16, seshadri.micro17, seshadri.bookchapter17, seshadri2020indram}} \omi{is} a new DRAM substrate that can perform \omi{(i)} bitwise majority \omi{(and thus bitwise AND/OR)} \omi{operations} across three DRAM rows by simultaneously activating three DRAM rows \omi{and (ii) bitwise NOT} operation\omi{s} \omi{on a DRAM row} using 2-transistor 1-capacitor DRAM cells~\cite{kang2009one,lu2015improving}.}

\noindent
\textbf{ComputeDRAM~\cite{gao2020computedram}} demonstrates in-DRAM copy {(previously proposed by RowClone~\cite{seshadri2013rowclone})} and bitwise AND/OR operations {(previously proposed by Ambit~\cite{seshadri.micro17})} on real DDR3 chips. ComputeDRAM performs in-DRAM operations by issuing carefully-engineered, valid sequences of DRAM commands {with violated tRAS and tRP timing parameters (i.e., by not obeying manufacturer-recommended timing parameters defined in DRAM \newnew{chip} specifications~\cite{micron2016ddr4})}. By issuing command sequences {with violated timing parameters}, \omi{ComputeDRAM \omi{activates}} two or three DRAM rows in a DRAM bank \newnew{in quick succession} \omi{(i.e., performs two or three row activation\omi{s})}. \newnew{ComputeDRAM leverages (i) two row activation\omi{s} to transfer data between two DRAM rows and (ii) three row activation\omi{s} to perform the majority function in real unmodified DRAM chips.} 
%This allows multiple DRAM cells on a column to contribute to the charge sharing process. 
\revdel{ComputeDRAM leverages multiple row activation (i) to transfer data between two DRAM cells, and (ii) to perform {the} majority function across three rows {in real unmodified DRAM chips}.}

\noindent
%\outline{Explain D-RaNGe}
%\todo{This also should come later (but in background section).}
\textbf{D-RaNGe~\cite{kim.hpca19}} is a {state-of-the-art} high-throughput DRAM-based true random number generat{ion technique}. D-RaNGe leverages the randomness in DRAM activation (tRCD) failures as its entropy source. D-RaNGe extracts random bits from DRAM cells that fail with $50\%$ probability when accessed with a reduced \newnew{(i.e., violated)} tRCD. 
%The authors assess the quality of random bitstreams generated by D-RaNGe using various statistical tests and show that it generates high-quality true random numbers. 
D-RaNGe demonstrates {high-quality true random number generation} on a vast number of real DRAM chips across multiple generations.

\noindent
\newnew{\textbf{QUAC-TRNG~\cite{olgun2021quactrngieee}} demonstrates that four DRAM rows can be activated in a quick succession using an ACT-PRE-ACT command sequence \omi{(called QUAC)} with violated tRAS and tRP timing parameters in real DDR4 DRAM chips. QUAC-TRNG uses QUAC to generate true random numbers at high throughput and low latency.}

%\outline{RISC-V Rocket-Chip SoC generator}
%\subsection{RISC-V Rocket-Chip SoC Generator}
%Rocket-Chip is an open-source System-on-Chip (SoC) generator in Chisel3~\cite{chisel} hardware construction language, built by the RISC-V community~\cite{asanovic2016rocket}. Rocket-Chip is used to generate configurable RISC-V system designs.
%\jgl{This subsection seems to be coming from nowhere: "We use Rocket-Chip to implement a RISC-V system in our \X prototype."}

%\footnote{NIST Statistical Test Suite (STS) is a collection of statistical tests that are widely used in assessing the quality of random number generators. If a random number generator passes all NIST STS tests it is very likely that the random number generator outputs uncorrelated, random bitstreams.}
\section{Unifying The Two Worlds}
\label{sec:framework}

Feedback emerges as both a complex ecosystem, and a rich, but systematized, information source, with many attributes covered in educational work. We consolidate several of these features, introduced in Section \ref{sec:background_pedagogy}, to create a novel feedback framework, which we subsequently adapt for LLMs.


\subsection{The Unified Framework}


From the multi-disciplinary background presented in Section \ref{sec:background_pedagogy}, we derive a tripartite feedback ecosystem structure consisting of task, learner, and feedback components, depicted in Figure \ref{fig:base_framework}.

\begin{figure}[ht]
\centering
\includegraphics[width=1.0\columnwidth]{figures/base_framework_final.png}
\caption{The feedback ecosystem. The feedback's characteristics, the task, and the learner all influence how effectively the feedback is received.}
\label{fig:base_framework}
\end{figure}

\paragraph{Learning Sciences Grounding}
Each component has a set of attributes that can vary depending on a situation requiring feedback. 
As discussed in Section \ref{sec:background_pedagogy}, feedback is influenced by both \textit{task} and \textit{learner characteristics}. We break each of these two components down to more granular constituents. For the \textit{task} we introduce two attributes, its complexity (\ie its difficulty level), and its nature --- which we reduce to it being closed-answer (where there is a single correct answer or a finite set of them) or open-ended. The \textit{learner} is similarly divided into two sub-components: their particular feedback processing mechanism, and their prior knowledge (dependent on the task).
We also develop a more holistic view of the \textit{feedback} component itself, with three main features:
\begin{enumerate}[itemsep=0.05em,label=(\roman*)]
    \item \textit{timing}, whether feedback is provided immediately or after a given temporal delay,
    \item \textit{content}, capturing both the type and the format of the information provided in a feedback message, explored fully in section \ref{sec:taxonomy},
    and,
    \item \textit{source}, whether the feedback stems from a peer or an authority figure, a human or an AI model, possibly including their relevant proficiency levels. Different sources usually build different relationships with the learner, and will communicate in different language \textit{styles}, a dimension explored in Section \ref{sec:taxonomy}.
\end{enumerate}

\paragraph{Framework Interactions} As Figure \ref{fig:base_framework} depicts, various interactions occur between the task, the learner, and the feedback. For example, as stated in Section \ref{sec:background_pedagogy}, \citet{mason_providing_2001} argue that if a learner has little prior knowledge or the task has a low complexity, feedback should be immediate, but if the task complexity and student's prior knowledge are both high, then it should be delayed. \citet{narciss_feedback_2008} on the other hand, refers to \citet{Mathan2005FosteringTI}'s take on timing, which states that if the learner possesses metacognitive skills (for identifying and correcting errors), then feedback should be delayed, to first promote these skills.
The type of task also naturally conditions the feedback given. 
Finally, all aspects of feedback impact how the learner receives and processes the feedback.
Appendix \ref{app:felt_interactions} presents a more comprehensive overview of the interactions present in the framework.


\subsection{The LLM Adaption -- FELT}

The unifying framework is directly derived from learning sciences research, and as such, designed for the human learner. We are, however, interested in providing feedback to LLMs. Thus, we adapt this ecosystem to LLMs, where the LLM is viewed as the learner. In particular, we propose five modifications to the base framework: 
expanding the learner to reflect the model's size as well as the model training data and method (including its training objectives), the expansion of the task component to reflect the prompt instructions, and finally, the addition of an error component. The resulting framework, FELT (Feedback, Errors, Learner, Task) is displayed in 
Figure \ref{fig:felt_framework}. 

\begin{figure}[ht]
\centering
\includegraphics[width=1.0\columnwidth]{figures/felt_framework_final.png}
\caption{FELT framework. The feedback ecosystem is specifically adapted for LLMs.}
\label{fig:felt_framework}
\end{figure}

\paragraph{Expanding the Learner} We add three specific LLM extensions to the generic learner component: model size, and model training data and method. All three are important, and together capture the \textit{Prior Knowledge} of a model. Model size is directly linked to emergent abilities \citep{wei2022emergent} and the model's ability to effectively leverage feedback \citep{scheurer2022training, bai_constitutional_2022}. The data the model was trained on, as well as how it was trained, similarly encode the model's initial abilities to tackle any given task.

\paragraph{Expanding the Task} Another important factor to incorporate into the framework pertains to the instructions given in the prompt. Past research has shown the importance of stating the actions a model can take, such as outputting ``I don't know.'' \citep{zhou2023contextfaithful}. Similarly, how strongly the prompt encourages a model to incorporate feedback can favor overoptimization.

\paragraph{Introducing Errors} Finally, effective feedback may communicate information on where the learner is failing, requiring an understanding of the possible error modes for a given task, and which ones the learner is likely in. For example, guessing and committing systematic reasoning mistakes are reflections of differing understandings. Exploring the error space and identifying the mistakes made by a learner is an important extension to the base framework directly derived from pedagogical and psychology of education research.

\subsection{Feedback Integration}

The method used to transmit the feedback to the model influences how it is subsequently processed. \citet{fernandes_bridging_2023} identify three common feedback integration mechanisms: feedback-based imitation learning, joint-feedback modeling, and reinforcement learning. In addition to this, we also consider feedback use in in-context learning \citep{brown2020language}. The training objective will necessarily influence how the model is processing and incorporating feedback.
Typically, the training relies upon either scalar feedback (a single number encoding how much the model should be rewarded for its output) or a ranking (how well a given output did in relation to other candidate answers). However, this is simple information, and does not leverage the rich and complex information encoded in natural language feedback. 
Section \ref{sec:taxonomy} therefore comprehensively explores the different types of information that can be encoded in feedback.


\begin{figure*}[ht]
\center
\includegraphics[width=\textwidth]{figures/TaxonomyTable.png}
\caption{A mapping between the ten axes of our general taxonomy and the nine feedback content categories.}
\label{fig:taxonomy}
\end{figure*}

\section{Feedback Content Taxonomy}
\label{sec:taxonomy}

In Section \ref{sec:framework}, we presented an overview of the complex ecosystem of feedback, including an expansion specifically for LLMs (\ie \ours) that connects various background elements (\eg the learner, the task, the error types) to the actual feedback that must be given. In this section, we expand on our analysis of the \textit{content} dimension of feedback in \ours. 
Specifically, we present a taxonomy of feedback content under two different forms: a set of 10 broad axes along which feedback can vary, and a more concrete set of nine emergent categories for feedback topic. 
Figure \ref{fig:felt_framework} presents an overview of the two different presentations of this taxonomy, and the mapping between them.

We motivate this taxonomy to finely categorize current approaches to textual feedback that implicitly formulate feedback solely for \textit{utility} (\ie how useful is the feedback for guiding a model toward a suitable response). However, they do not categorize its content, leaving a conceptual gap about \textit{what} makes feedback useful. 
Our taxonomy stratifies the feedback space, allowing a deliberate and systematic study of feedback content.


\subsection{General Taxonomy}
\label{ssec:general_taxonomy}
We break down feedback content along ten dimensions that influence how feedback is formulated: 
\begin{enumerate}[itemsep=0.05em]
    \item \textit{length}, an indication of how much feedback feedback is given, possibly measured by counting its number of tokens,
    \item \textit{granularity}, a measure of the level of detail with which the feedback addresses the original answer --- it is not a measure of how much of the answer is being considered, but rather of the level of detail with which it is being considered,\footnote{For an open-answer example task, feedback might range from global learning meta-feedback, to global but task-specific, to paragraph-level, to sentence-level, to word-level, to token-level feedback.}
    \item \textit{applicability of instructions}, expressing both whether the feedback contains instructions, as well as how applicable those instructions are for the learner and their current understanding and approach to solving the task,
    \item \textit{answer coverage}, which registers how much of the learner's answer is considered to generate the given feedback. The feedback could be independent of the answer, or only relate to parts of the answer (\eg, focusing on a particular mistake), or the feedback might take the complete answer into consideration,
    \item \textit{criteria}, denoting which criteria the answer is being evaluated on: global evaluation, specific dimensions (e.g., fluency, engagement, etc.), or, alternatively, no  dimensions (the answer is not being evaluated),
    \item \textit{information novelty}, indicating the degree to which learner already had access to the information provided in the feedback, ranging from all information being previously known by the learner, to some information being unknown to the learner, to all information being novel for the learner,
    \item \textit{purpose}, measuring whether the feedback is being given to improve the learner's performance or to clarify the task,\footnote{In a pedagogical setting with human learners, other purposes are possible, such as regulating the student's emotions and motivation, but we do not consider these for the LLMs.}
    \item \textit{style}, capturing the level of the language used to transmit the feedback to the learner, which can range from simple, direct sentences to verbose and terminology-heavy language, 
    \item \textit{valence}, indicating whether the feedback is positive (signaling achievement) or negative (signaling need for improvement),
    \item \textit{mode}, denoting how the feedback is given to the learner, capturing two important aspects: whether the feedback is uni- or multi-modal, \footnote{Multi-modal feedback is naturally more suited for multi-modal tasks. For example, in an instance segmentation task, the correct (visual) answer could be provided alongside textual feedback on mistakes and how to correct them.} and whether it allows the user to submit multiple tries or not.
\end{enumerate}

\noindent Combined, these ten axes capture what we consider the most important qualities of a piece feedback to understand its impact on a given learner model. We hypothesize that each of these dimensions influences the model's revised response to varying degrees, but that all are worthy of individual study.

\subsection{Categorical Taxonomy}
\label{ssec:categorical_taxonomy}
Given these ten axes of feedback, we further define nine emergent, interpretable feedback categories that vary each of these axes (see Figure~\ref{fig:taxonomy}). These nine categories enable a more streamlined classification of prevalent forms of feedback, yielding a starting point to explore which components of feedback may be effective for a given task:

\begin{enumerate}[itemsep=0.05em]
    \item \textit{Global Verification}, a single, aggregate score for the task performance as a whole,
    \item \textit{Response Verification}, response-level classification of the answer (\eg \textit{right} or \textit{wrong)},
    \item \textit{Mistakes Verification}, error-level feedback, possibly with the number of mistakes,
    \item \textit{Correct Answer}, provides the correct answer,\footnote{While understanding the concept of the correct answer in a closed-answer task is trivial, the same cannot be said about an open-answer context. Providing an excellent answer for the task would be nothing more than a worked example and thus a \textit{Procedural Elaboration} type of feedback. Instead, in this context, the \textit{Correct Answer} becomes the rewritten version of the student's answer, improved to be excellent according to the evaluative standard, ideally with as few changes as possible.}
    \item \textit{Response Elaboration}, response-level, response-specific feedback addressing either the positive characteristics of the answer, its shortcomings, or both,
    \item \textit{Mistakes Elaboration}, elaboration feedback about types and sources of errors,
    \item \textit{Task Elaboration}, elaboration feedback about the task, such as its requirements, topic, the relevant concepts or terminology --- this is clarification feedback, and does not provide suggestions for the next steps,
    \item \textit{Procedural Elaboration}, task-specific elaboration feedback about how to solve or improve a solution for said task,
    \item \textit{Metacognition Elaboration}, elaboration feedback about general learning and problem-solving strategies
\end{enumerate}

\noindent Examples of each feedback type, as well as more detailed guidance on the characteristics of each type,  are presented in Appendix \ref{app:feedback_examples}. The nine categories, with the exception of the \textit{correct answer}, are grouped into two main types: verification and elaboration. Following the terminology established by educational works, verification feedback provides classification-only information, such as whether an answer is right or wrong, or whether a mistake was made or not. 
Elaboration feedback provides more meaningful, concrete, and detailed information.

In our formulation, the general taxonomy (\S\ref{ssec:general_taxonomy}) provides a set of 10 broad feedback dimensions that the categorical taxonomy composes into nine distinct feedback types.
These nine feedback types are the ones we believe to be most relevant for providing and classifying feedback, as they are interpretable, but remain consolidated from human learning research. 
%
Finally, one important characteristic of these categories is that they were designed to be as modular as the task allows them to be. 
This modularity and clear delineation of each feedback class enables the study of feedback types individually, but also of different combinations. 

\subsection{Mapping Previous NLP Research}

To demonstrate the applicability of our taxonomy, we classify feedback examples from prior works according to the nine categories outlined above. If a work employed more than one type of feedback in its examples, we mapped the work to appropriate feedback types.
To be best of our knowledge, we are the first to conduct a systematic exploration of textual feedback types to date.


\begin{table}[htb!]
  \begin{tabular}{p{0.14\textwidth}|p{0.28\textwidth}}
    \toprule
    \textbf{Feedback Type}  & \textbf{Works}  \\
    \midrule
    Global \newline Verification & \small \citet{weston_dialog-based_2016}; \citet{shi2022life}; \citet{scheurer2022training}; \citet{welleck2022generating}; \citet{wu2023finegrained}; \citet{shinn2023reflexion}; \citet{chen2023teaching}  \\
    \midrule
    Response \newline Verification & \small \citet{madaan2023selfrefine}; \citet{lightman2023lets} \\
    \midrule
    Mistakes \newline Verification & \small \textcolor{gray}{\citetColored{gray}{tandon-etal-2022-learning}}; \citet{saunders2022selfcritiquing}; \citet{welleck2022generating}; \citet{wu2023finegrained}; \citet{paul2023refiner}; \citet{chen2023teaching} \\
    \midrule
    Correct \newline Answer & \small \citet{weston_dialog-based_2016}; \citet{shi2022life}; \citet{saunders2022selfcritiquing} \\
    \midrule
    Response \newline Elaboration & \small \citet{shi2022life}; \citet{scheurer2022training}; \citet{madaan2023selfrefine} \\
    \midrule
    Mistakes \newline Elaboration & \small \citet{weston_dialog-based_2016}; \textcolor{gray}{\citetColored{gray}{tandon-etal-2022-learning}}; \citet{scheurer2022training}; \citet{saunders2022selfcritiquing}; \citet{shinn2023reflexion} \\
    \midrule
    Task \newline Elaboration & \small \citet{tandon-etal-2022-learning}; \citet{scheurer2022training}; \citet{madaan2023selfrefine}; \citet{chen2023teaching} \\
    \midrule
    Procedural \newline Elaboration & \small \citet{weston_dialog-based_2016};  \citet{tandon-etal-2022-learning};  \citet{shi2022life};  \citet{murty2022fixing};  \citet{saunders2022selfcritiquing};  \citet{welleck2022generating};  \citet{schick2022peer};  \citet{madaan2023selfrefine, shinn2023reflexion} \\
    \midrule
    Metacognition \newline Elaboration & None \\
    \bottomrule
  \end{tabular}
  \caption{The distribution of NLP textual feedback research as per the categorical taxonomy, highlighting that categorization remains an unexplored area of feedback research. \textcolor{gray}{\textit{n.b.}.: Works in grey collect the indicated types of textual feedback, but do not employ it.}}
  \label{tab:nlp_feedback_space}
\end{table}



Table \ref{tab:nlp_feedback_space} presents a cross-referencing of our categorical taxonomy with NLP works employing diverse forms of textual feedback. 
We observe that the examples of feedback in these works are spread across multiple categories, painting a chaotic picture of how feedback is formulated in current NLP research. While these works broadly consider the \textit{utility} of feedback in how it is \textit{applied} to LLMs, less focus has been given to how the \textit{content} of the feedback affects its utility (and no works ground this content to pedagogical foundations). 

Finally, we note that, while we pursue an interpretable categorization to classify these existing works on feedback in NLP, our general taxonomy showcases novel ways to compose feedback, providing an opportunity for further granularity when exploring the space of possible feedback types.




\section{Case Study}
\label{sec:case_study}

Having established an LLM-relevant feedback model at three levels of abstraction --- a feedback ecosystem (\S\ref{sec:framework}), feedback content dimensions (\S\ref{ssec:general_taxonomy}), and concrete feedback categories (\S\ref{ssec:categorical_taxonomy}) --- in this section we seek to validate its importance by conducting a simple case study.

Using a corpus of media articles written about research papers,\footnote{Details about the dataset, model hyperparameters, and prompts used are presented in Appendix \ref{app:case_study}.} we task GPT-4 \citep{openai2023gpt4} to summarize 50 articles on research papers in lay terms. Then, we provide 2 different types of feedback from our categorization: 
\textit{Correct Answer} (CA) and \textit{Task Elaboration} (TE), chosen due to their significantly different characteristics. We evaluate both the original and revised answer.\footnote{As this analysis was conducted on a total of 50 samples, producing 150 evaluations (one per feedback type plus the original summary for each data sample). The evaluation of each summary was conducted by one of the authors, according to an established set of criteria.} More details about the experiment setup and execution are available in Appendix \ref{app:case_study}.

\begin{figure}[t]
\centering
\includegraphics[width=1.0\columnwidth]{figures/case_study.png}
\caption{The impact of two different feedback types on a lay summary generation task. Both led to an improvement in average summary quality.}
\label{fig:case_study}
\end{figure}

\paragraph{Results} Figure \ref{fig:case_study} demonstrates that, on average, both types of feedback improve the generated summaries. Interesting, we find that \textit{task elaboration} feedback outperforms \textit{correct answer} feedback, which is remarkable, as feedback and evaluation in NLP are usually based on knowledge of the correct answer. However, in this case, feedback that reiterated or provided more information about desired summary qualities (\ie TE) emerged as the more \textit{useful} of the two feedback types.\footnote{This is not surprising from the learning sciences perspective --- as seen in Section \ref{sec:background_pedagogy}, CA does not possess any of the characteristics of effective feedback.}

Furthermore, the effect of these two feedback types emerges clearly when analyzed qualitatively. For example, TE feedback led to more promotional language --- frequently including words such ``groundbreaking,'' ``breakthrough,'' and ``revolutionize,'' which was encouraged by the task elaboration prompt, ``A good and captivating summary should
first grab the reader’s attention'' (see Appendix~\ref{app:case_study}). Meanwhile, CA feedback often led to more technical language. While this language was likely to be found in some quantity in the \textit{correct answer} (\ie the gold summary), it was amplified by the model's response, leading to lower scores. Output examples are presented in Appendix \ref{app:case_study}.

\paragraph{Feedback Cost} Both CA and TE are answer-independent, (\ie both are constructed independently from the content of the original answer). Yet, both led, on average, to a significant improvement over the initial summary, demonstrating clear advantages to providing models with feedback. 
TE, especially, is a simple type of feedback to generate, needing only to describe the task or relevant concepts with some level of detail. The positive impact of such straightforward additions to the generative pipeline reinforces the motivation for studying the space of feedback content more deeply. 

\section{Conclusion}
We have presented a novel tokenization scheme for Vision Transformers, replacing the standard uniform patch grid with a mixed-resolution sequence of tokens, where each token represents a patch of arbitrary size. We integrated the Quadtree algorithm with a novel feature-based saliency scorer to create mixed-resolution patch mosaics, making this work the first to use the Quadtree representations of RGB images as inputs for a neural network.

Through experiments in image classification, we have shown the capacity of standard Vision Transformer models to adapt to mixed-resolution tokenization via fine-tuning. Our Quadformer models achieve substantial accuracy gains compared to vanilla ViTs when controlling for the number of patches or GMACs. Although we do not use dedicated tools for accelerated inference, Quadformers also show gains when controlling for inference speed.

We believe that future work could successfully apply mixed-resolution ViTs to other computer vision tasks, especially those that involve large images with heterogeneous information densities, including tasks with dense outputs such as image generation and segmentation.


\newpage
\section*{Limitations}
\ours, our proposed framework to capture the full feedback ecosystem, is theorectically grounded and fairly unexplored. Some aspects, like the impact of timing, need to be reassessed for LLMs.

Our previous NLP research mapping (presented in Table \ref{tab:nlp_feedback_space}) was restricted to only the nine feedback content categories. Though they are cleanly delineated and encapsulate the main feedback types that emerged in pedagogical research, they do not encompass all the feedback space captured by the 10 general axes in the taxonomy.

Our case study was conducted exclusively in English. Models trained on other languages, especially lower-resource ones, might react to feedback differently, and it might have a weaker impact on revised generations. 

Additionally, we used GPT-4, a paid and closed-source LLM, whose details about architecture and training remain unknown.



\bibliography{anthology,custom}
\bibliographystyle{acl_natbib}

\appendix
\section{Lemma \ref{lemx} and its Proof}
\label{sec:app_a}
Here we state that we can condition on high-probability events.

\begin{lem}
	\label{lemx}
	Let $p \in (0,1)$, and $X \sim Bernoulli (p)$ be defined on a probability space $(\Omega, \mathcal{F}, P)$. Consider $B_1, B_2, \cdots $ be a sequence of events defined on the same probability space such that $P(B_n) \rightarrow 1$ as $n$ goes to infinity. Also, let $\textbf{Y}$ be a random vector (matrix) in the same probability space, then:
	\[I(X; \textbf{Y}) \rightarrow 0\ \ \text{iff}\ \  I(X; \textbf{Y} {|} B_n) \rightarrow 0. \]
\end{lem}

\begin{proof}
First, we prove that as n becomes large,
\begin{align}\label{eq:H1}
H(X {|} B_n)- H(X) \rightarrow 0.
\end{align}

Note that as $n$ goes to infinity,
\begin{align}
\no P\left(X=1\right) &=P\left(X=1 \bigg{|} B_n\right) P\left(B_n\right) + P\left(X=1 \bigg{|} \overline{B_n}\right) P\left(\overline{B_n}\right)\\
%\no   &=P\left(X=1 \bigg{|} B_n\right)\left(1-o(1)\right) + P\left(X=1 \bigg{|} \overline{B_n}\right)o(1)\\
%\no   & \rightarrow  P\left(X=1 \bigg{|} B_n\right)+ o(1)  \ \ (\textrm{as } n \rightarrow \infty) \\
\no &=P\left(X=1 \bigg{|} B_n\right),\ \
\end{align}
thus,
$\left(X \bigg{|} B_n\right) \xrightarrow{d} X$, and as $n$ goes to infinity,
 \[H\left(X {|} B_n\right)- H(X) \rightarrow 0.\]
Similarly, as $n$ becomes large,
\[ P\left(X=1 \bigg{|} \textbf{Y}=\textbf{y}\right) \rightarrow P\left(X=1 \bigg{|} \textbf{Y}=\textbf{y}, B_n\right),\ \
\]
and
\begin{align}\label{eq:H2}
H\left(X {|}  \textbf{Y}=\textbf{y},B_n\right)- H\left(X {|} \textbf{Y}=\textbf{y}\right) \rightarrow 0.
\end{align}
%Thus $(X {|}  Y_n=y,B_n) \xrightarrow{D} (X {|} Y_n=y)$, and so $H(X {|}  Y_n=y,B_n)- H(X {|} Y_n=y) \rightarrow 0 $ as $n \rightarrow \infty$.
Remembering that
\begin{align}\label{eq:H3}
I\left(X; \textbf{Y}\right)=H(X)-H(X {|} \textbf{Y}),
\end{align}
and using (\ref{eq:H1}), (\ref{eq:H2}), and (\ref{eq:H3}), we can conclude that as  $n$ goes to infinity,
\[I\left(X;\textbf{Y} {|} B_n\right) - I\left(X,\textbf{Y}\right) \rightarrow 0.\]
As a result, for large enough $n$,
\[I\left(X; \textbf{Y}\right) \rightarrow 0 \Longleftrightarrow I\left(X; \textbf{Y} {|} B_n\right) \rightarrow 0. \]
\end{proof}

\section{Proof of Lemma \ref{lem1}}
\label{sec:app_c}
Here, we provide a formal proof for Lemma \ref{lem1} which we restate as follows. The following lemma confirms that the number of elements in $J^{(n)}$ goes to infinity as $n$ becomes large.

	If $N^{(n)} \triangleq |J^{(n)}| $, then $N^{(n)} \rightarrow \infty$ with high probability as $n \rightarrow \infty$.  More specifically, there exists $\lambda>0$ such that
\[
	P\left(N^{(n)} > \frac{\lambda}{2}n^{\frac{\beta}{2}}\right) \rightarrow 1.
	\]

\begin{proof}
Define the events $A$, $B$ as
\[A \equiv  p_1\leq P_u\leq p_1+\epsilon_n\]
\[B \equiv p_1+\epsilon_n\leq Q_u\leq p_1+(1-2p_1)a_n.\]
Then, for $u \in \{1, 2, \dots, n\}$ and $0\leq p_1<\frac{1}{2}$:
	\begin{align}
	\no P\left(u\in J^{(n)}\right) &= P\left(A\ \cap \ B\right)\\
	\nonumber &= P\left(A\right) P\left(B \big{|}A \right). \ \
	\end{align}		
So, given $p_1 \in (0,1)$ and the assumption $0<\delta_1<f_p<\delta_2$, for $n$ large enough, we have
	\[
	P(A)  = \int_{p_1}^{ p_1+\epsilon_n}f_P(p) dp,
	\]
so, we can conclude that
	\[
	\epsilon_n\delta_1<P(A) <\epsilon_n\delta_2.
	\]
	We can find a $\delta$ such that $\delta_1<\delta<\delta_2$ and
\begin{equation}\label{eq:5}
	P( A) = \epsilon_n\delta.
\end{equation}
We know
	\[Q_u\bigg{|}P_u=p_u \sim Uniform \left[p_u,p_u+(1-2p_u)a_n\right],\]
	so, according to Figure \ref{fig:piqi_b}, for $p_1\leq p_u\leq p_1+\epsilon_n$,
	\begin{align}
	\no P\left(B | P_u=p_u\right ) &= \frac{p_1+(1-2p_1)a_n-p_1-\epsilon_n}{p_u+(1-2p_u)a_n-p_u} \\
	\nonumber &= \frac{(1-2p_1)a_n-\epsilon_n}{(1-2p_u)a_n}\\
	\nonumber &\geq \frac{(1-2p_1)a_n-\epsilon_n}{(1-2p_1)a_n} \\
	\nonumber &= 1- \frac{\epsilon_n}{(1-2p_1)a_n}, \ \
	\end{align}
which implies
\begin{align}
P\left(B | A\right ) \geq 1- \frac{\epsilon_n}{(1-2p_1)a_n}. \label{eq:6}
\end{align}
Using (\ref{eq:5}) and (\ref{eq:6}), we can conclude
	\[P\left(u\in J^{(n)}\right)\geq \epsilon_n\delta \left(1- \frac{\epsilon_n}{(1-2p_1)a_n}\right).\]
Then, we can say that $N^{(n)}$ has a binomial distribution with expected value of $N^{(n)}$ greater than $n\epsilon_n\delta \left(1- \frac{\epsilon_n}{(1-2p_1)a_n}\right)$, and by substituting $\epsilon_n$ and $a_n$, for any $c'>0$, we get
\[E\left[N^{(n)}\right] \geq \delta\left(n^{\frac{\beta}{2}}- \frac{1}{{c'(1-2p_1)}}\right)   \geq  \lambda n^{\frac{\beta}{2}}.\]

Now by using Chernoff bound, we have
\[P\left(N^{(n)} \leq (1- \theta) E\left[N^{(n)}\right]\right) \leq e^{-\frac{\theta^2}{2}E\left[N^{(n)}\right]},\]
so, if we assume $\theta=\frac{1}{2}$, we can conclude for large enough $n$,
\begin{align}
\nonumber P\left(N^{(n)} \leq \frac{\lambda}{2}n^{\frac{\beta}{2}}\right) &\leq P\left(N^{(n)} \leq \frac{E\left[N^{(n)}\right]}{2}\right)\\
\nonumber &\leq e^{-\frac{E[N^{(n)}]}{8}}\\
\nonumber &\leq e^{-\frac{\lambda n^{\frac{\beta}{2}}}{8}} \rightarrow 0.
\end{align}
As a result, $N^{(n)} \rightarrow \infty$ with high probability for large enough $n.$

\end{proof}
\section{Proof of Lemma \ref{lemOnePointFive}}
\label{sec:app_b}
Here we provide a formal proof for Lemma \ref{lemOnePointFive} which we restate as follows.

Let $N$ be a positive integer, and let $a_1, a_2, \cdots, a_N$ and $b_1, b_2, \cdots, b_N$ be real numbers such that $a_u \leq b_u$ for all $u$. Assume that $X_1, X_2, \cdots, X_N$ are $N$ independent random variables such that
\[X_u \sim Uniform[a_u,b_u]. \]
Let also $\gamma_1, \gamma_2, \cdots, \gamma_N$ be real numbers such that
\[ \gamma_j \in \bigcap_{u=1}^{N} [a_u, b_u] \ \ \textrm{for all }j \in \{1,2,\cdots,N\}. \]
Suppose that we know the event $E$ has occurred, meaning that the observed values of $X_u$'s is equal to the set of $\gamma_j$'s (but with unknown ordering), i.e.,
\[E \ \ \equiv \ \ \{X_1,X_2,\cdots,X_N\}= \{ \gamma_1, \gamma_2, \cdots, \gamma_N \}, \] then
\[P\left(X_1=\gamma_j |E\right)=\frac{1}{N}. \]

\begin{proof}
Define sets $\mathfrak{P}$ and $\mathfrak{P}_j$ as follows:
\[\mathfrak{P}= \textrm{The set of all permutations $\Pi$ on }\{1,2,\cdots,N\}. \]
\[\mathfrak{P}_j= \textrm{The set of all permutations $\Pi$ on }\{1,2,\cdots,N\} \textrm{ such that } \Pi(1)=j. \]
We have $|\mathfrak{P}|=N!$ and $|\mathfrak{P}|=(N-1)!$. Then
\begin{align*}
 P(X_1=\alpha_j |E)&=\frac{\sum_{\pi \in \mathfrak{P}_j} f_{X_1,X_2, \cdots,X_N} (\gamma_{\pi(1)}, \gamma_{\pi(2)}, \cdots, \gamma_{\pi(N)})} {\sum_{\pi \in \mathfrak{P}} f_{X_1,X_2, \cdots,X_N} (\gamma_{\pi(1)}, \gamma_{\pi(2)}, \cdots, \gamma_{\pi(N)})}\\
 &=\frac{(N-1)! \prod\limits_{u=1}^{N} \frac{1}{b_u-a_u}}{N! \prod\limits_{u=1}^{N} \frac{1}{b_u-a_u}}\\
 &=\frac{1}{N}.
\end{align*}
\end{proof}
\section{Case Study Implementation Details}
\label{app:case_study}

\subsection{Dataset and Model}
The task presented in section \ref{sec:case_study} involved the summarizing of a news article describing a research development at EPFL in more approachable language and terminology. The summary had as an objective to be even more approachable to people outside the field. 

\paragraph{Dataset} The data for this task came from EPFL's Mediacom department, where they provided the authors with a set of 2370 entries of articles, summaries, and extra information (title, author, date, etc.). Out of these, 50 were chosen so that all articles relayed a newly published work. This was the only criteria for selection. All articles and summaries were pre-processed so as to remove HTML tags.

\paragraph{Model} The model used for this task was GPT-4 \citep{openai2023gpt4}, with its default hyperparameters, called through OpenAI's Chat Completions API. Thus, the model generated a single completion for each prompt, with a temperature of $1$, and with no limitation on the maximum number of tokens (beyond the model's own context length).

\subsection{Experiment Execution}
The experiment was run in two stages. In the initial phase, the model is asked to generate a first summary. It is then provided with feedback and asked to revise its original summary. In this experiment, two distinct types of feedback were provided: \textit{Task Elaboration} (TE) and \textit{Correct Answer} (CA).

\paragraph{Initial Generation}
Following OpenAI's Chat Completions API, the model prompting is done under a chat format. In this setting, the first \textit{message} is a system message stating \texttt{You are a helpful assistant.} This is then followed by an user message, with the following prompt: 

\texttt{Summarize the following article into a short but captivating snippet under around 100 tokens. It must describe both the problem and the approach used to solve it, as well as the venue where these findings were presented, whenever this information is available. \\
Article: [article body]}

The model's response message to this prompt is considered its original summary.

\paragraph{Revised Generation} The revised generation prompt is shares the chat prompting format. It contains the previous chat history, which includes not only the two messages outlined above but also the model's answer as an assistant message. To these three messages, a new user message is added, with the following content:

\texttt{Feedback: [feedback] \\
Please revise your original summary taking the feedback into consideration. If you feel the feedback is not appropriate or useful, you can disregard it.}

The \texttt{[feedback]} placeholder will have one of two different values, depending on the feedback type being provided:
\begin{itemize}
    \item \textbf{Correct Answer:} The feedback will be of the form \\
    \texttt{The correct answer is: [gold\_summary]} \\
    where \texttt{[gold\_summary]} is the summary provided by the Mediacom dataset,
    
    \item \textbf{Task Elaboration:} The feedback will be of the form \\
    \texttt{A good and captivating summary should first grab the reader's attention and make them curious to learn more. It is then important to factually and precisely state what the problem is, why it is important, the proposed solution, and, if published or being divulged, disclose where the reader can find it.}
\end{itemize}

Finally, as in the first stage, the model's response message to this prompt is considered its revised summary.

\subsection{Example Outputs}
In this subsection, we present a few examples outputs from the case study.

\subsubsection{Example 1}
\paragraph{\colorbox{YellowOrange}{Original Article}}
\texttt{The International Consortium of Investigative Journalists (ICIJ), which has over 200 members in 70 countries, has broken a number of important stories, particularly ones that expose medical fraud and tax evasion. One of its most famous investigations was the Panama Papers, a trove of millions of documents that revealed the existence of several hundred thousand shell companies whose owners included cultural figures, politicians, businesspeople and sports personalities. To complete an investigation of this size is only possible through international cooperation between journalists. When sharing such sensitive files, however, a leak can jeopardize not only the story’s publication, but also the safety of the journalists and sources involved. At the ICIJ’s behest, EPFL’s Security and Privacy Engineering (SPRING) Lab recently developed Datashare Network, a fully anonymous, decentralized system for searching and exchanging information. A paper about it will be presented during the Usenix Security Symposium, a worldwide reference for specialists, which will be held online from 12 to 14 August. \\
Anonymity at every stage \\
Anonymity is the backbone of the system. Users can search and exchange information without revealing their identity, or the content of their queries, either to colleagues or to the ICIJ. The Consortium ensures that the system is running properly but remains unaware of any information exchange. It issues virtual secure tokens that journalists can attach to their messages and documents to prove to others that they are Consortium members. A centralized file management system would be too conspicuous a target for hackers; since the ICIJ does not have servers in various jurisdictions, documents are typically stored on its members’ servers or computers. Users provide only the elements that enable others to link to their investigation. \\
Users searching for information enter keywords in the search engine. If the search produces hits, they can then contact colleagues – whose identity remains protected – who are in possession of potentially relevant documents. Search queries are sent encrypted to all users, if there is a macth the querier gets an alert and can decide whether they wish to enter in contact and share information. “Given the fact that users work in different time zones, some with only a few hours of internet access per day, it was critical that searches and responses could take place asynchronously,” notes Carmela Troncoso, who runs the SPRING Lab at the School of Computer and Communication Sciences (IC). Another messaging system, also secure and anonymous, is subsequently used for two-way exchanges. \\
Two completely new secure applications \\
“This system, which addresses real-world needs, has enabled SPRING to tackle some interesting challenges,” notes Troncoso. The research team drew on existing authentication mechanisms and anonymous communication primitives, which they then optimized. They also developed two completely new secure building blocks that are described in the paper being published today in connection with the Usenix Security Symposium – an asynchronous search engine and a messaging system. A new protocol, known as “multi-set private set intersection” (MS-PSI), ensures the security of the search engine, allowing users to easily search a large number of databases without increasing the risk of leaks. The messaging system relies on a large number of single-use virtual mailboxes and is based on the well-known “pigeonhole” system, which chooses one option at random, in this case one of the mailboxes. Currently, the system does not allow users to exchange documents. “At this stage in the process, journalists are using other secure messaging systems,” Troncoso says. \\
Working with the Consortium has allowed SPRING to frame new requirements that are rarely examined in the scientific literature. Datashare can be scaled to thousands of users and millions of documents while encrypting all communications. “The hurdles we encountered during the development process, however, have paved the way to a new area of research with significant potential for other fields,” Troncoso concludes.}


\paragraph{\colorbox{yellow}{Golden Summary}}
\texttt{An EPFL laboratory has developed Datashare Network, a decentralized search engine paired with a secure messaging system that allows investigative journalists to exchange information securely and anonymously. An scientific article on this subject will be presented during the Usenix Security Symposium which will be held online from August 12 to 14.}


\paragraph{\colorbox{pink}{GPT4 Original Summary}}
\texttt{EPFL's Security and Privacy Engineering Lab developed an anonymous, decentralized system called Datashare Network to aid the International Consortium of Investigative Journalists securely exchange information. Offering user anonymity, this system will be presented at the Usenix Security Symposium held online from 12-14 August.}

\paragraph{\colorbox{SkyBlue}{\textit{Correct Answer} Revision}}
\texttt{EPFL's Security and Privacy Engineering Lab created Datashare Network, a decentralized search engine with a secure messaging system that enables investigative journalists to exchange information anonymously and securely. A scientific paper on this solution will be showcased at the online Usenix Security Symposium from August 12 to 14.}

\paragraph{\colorbox{SpringGreen}{\textit{Task Elaboration} Revision}}
\texttt{Investigative journalism goes undercover with Datashare Network, a breakthrough anonymous, decentralized information-sharing system. Developed by EPFL's Security and Privacy Engineering Lab, the network protects journalists from exposure while tackling global issues like tax evasion and medical fraud. Discover how it secures data-sharing at the Usenix Security Symposium, held online from 12-14 August.}



\subsubsection{Example 2}
\paragraph{\colorbox{YellowOrange}{Original Article}}
\texttt{Thermoelectric materials in particular hold vast potential for use in energy applications because they generate electricity from waste heat, such as that generated by industrial processes, by car and truck engines, or simply by the sun. Reducing the thermal conductivity of these materials by a factor of three, for example, would completely revolutionize existing waste-heat recovery, and also all refrigeration and air-cooling technology. \\
A unique theory for all insulating materials \\
In the paper Unified theory of thermal transport in crystals and glasses, out in Nature Physics, Michele Simoncelli, a PhD student at EPFL’s Theory and Simulation of Materials (THEOS) Laboratory – together with Nicola Marzari, a professor at EPFL’s School of Engineering and head of THEOS and of the MARVEL NCCR, and Francesco Mauri, a professor at the University of Rome–Sapienza – present a novel theory that finally decodes the fundamental, atomistic origin of heat conduction. Up to now, different formulations needed to be used depending on the systems studied (e.g., ordered materials, like a silicon chip, or disordered, like in a glass), and there wasn’t a unified picture covering all possible cases. \\
This has now been made possible by deriving directly from the quantum mechanics of dissipative systems a transport equation that covers on equal footing diffusion, hopping, and tunneling of heat.
Waste heat recovery \\
This fundamental understanding will allow scientists and engineers to accurately predict the thermal conductivity of any insulating material (in metals, the heat is carried by the electrons, and that is well understood) – this is exceedingly important for thermoelectrics (i.e. materials that can convert heat into electricity), since these have both crystal- and glass-like properties, and are much needed for waste-heat recovery, or for refrigeration without greenhouse gases (and if you think refrigeration is boring, it is worth remembering that Albert Einstein spent many years trying to invent a new form of refrigerator). \\
In order to develop such next-generation technology, however, scientists first need to understand how and to what extent materials conduct heat. “Up to now, two different equations have been used for calculating thermal properties: one describes perfectly crystalline materials – that is, materials with highly ordered atomic structures – and the other one completely amorphous materials like glass, whose atoms do not follow an ordered pattern,” says Michele Simoncelli. These equations happened to work well in those special cases. “But between these two extremes lie a plethora of interesting cases, and neither equation worked – this is really where our contribution makes a profound difference” \\
* Illustration : Heat is the energy due to vibrations of atoms. The ridges represent the quantum vibrational energy levels of CsPbBr3, a promising thermoelectric material. Heat conduction originates from both diffusion of vibrational excitations (blurred spheres, following the energy level) and quantum tunnelling (jump between different energy levels, represented by blue waves).}

\paragraph{\colorbox{yellow}{Golden Summary}}
\texttt{EPFL and MARVEL researchers have developed a new theory for heat conduction that can finally describe and predict the thermal conductivity of any insulating material. This new formulation will let scientists make accurate predictions of thermal conductivity in a range of materials for critical engineering applications – from electronics to lasers to waste-heat recovery.}

\paragraph{\colorbox{pink}{GPT4 Original Summary}}
\texttt{Scientists at EPFL's Theory and Simulation of Materials Laboratory have developed a unified theory to predict the thermal conductivity of insulating materials, opening up possibilities for more efficient energy applications. The research breakthrough, published in Nature Physics, could lead to waste-heat recovery, revolutionising refrigeration and air-cooling technology without greenhouse gases. Understanding how materials conduct heat is key to developing next-generation thermoelectric technology.}

\paragraph{\colorbox{SkyBlue}{\textit{Correct Answer} Revision}}
\texttt{Researchers at EPFL and MARVEL have developed a groundbreaking theory for heat conduction, enabling the description and prediction of thermal conductivity in any insulating material. Published in Nature Physics, the new formulation offers potential for critical engineering applications such as waste-heat recovery, electronics, and lasers, enhancing energy efficiency in various industries.}

\paragraph{\colorbox{SpringGreen}{\textit{Task Elaboration} Revision}}
\texttt{Researchers have unlocked the secret to more efficient energy applications by developing a unified theory for predicting the thermal conductivity of insulating materials, transforming waste-heat recovery and revolutionizing refrigeration technology. With this innovative approach, published in Nature Physics, scientists can now accurately predict and optimize thermoelectric materials, paving the way for energy-efficient solutions without greenhouse gas emissions. Discover the full potential of this game-changing breakthrough at EPFL's Theory and Simulation of Materials Laboratory.}




\subsubsection{Example 3}
\paragraph{\colorbox{YellowOrange}{Original Article}}
\texttt{With its 500 km diameter, the asteroid Vesta is one of the largest known planet embryos. It came into existence at the same time as the Solar System. Spurring scientific interest, NASA sent the Dawn spacecraft on Vesta’s orbit for one year between July 2011 and July 2012. \\
Data gathered by Dawn were analyzed by a team of researchers from EPFL as well as the Universities of Bern (Switzerlanf), Brittany (France) and Arizona (USA). Conclusion : the asteroid's crust is almost three times thicker than expected. The study does not only have implications for the structure of this celestial object, located between Mars and Jupiter. Their results challenge a fundamental component in planet formation models, namely the composition of the original cloud of matter that aggregated together, heated, melted and then crystallized to form planets. \\
At EPFL’s Earth and Planetary Science Laboratory (EPSL), led by Philippe Gillet, Harold Clenet had a look at the composition of the rocks scattered across Vesta’s ground. "What is striking is the absence of a particular mineral, olivine, on the asteroid’s surface," said the researcher. Olivine is a main component of planetary mantles and should have been found in large quantities on the surface of Vesta, due to a double meteorite impact which, according to computer simulations, "dug" the celestial body’s southern pole to a depth of 80 km, catapulting large amounts of materials to the surface. \\
The two impacts were so powerful that more than 5 \% of Earth's meteorites come from Vesta. « But these cataclysms were not strong enough to pierce through the crust and reach the asteroid's mantle,» Harold Clenet continued. The meteorites originating from Vesta and found on Earth confirm this since they generally lack Olivine, or contain only minute amounts compared to the amount observed in planetary mantles. Also, the spacecraft Dawn did not find olivine in the vicinity of the two impact craters. « This means that the crust of the asteroid is not 30 km thick, as suggested by the models, but more than 80 km. » \\
Composition of planets \\
These discoveries challenge models that describe the formation of Vesta, and consequently the formation of rocky planets in the Solar System including planet Earth. Cooling theory and "re- melting" phenomena in the depths of previously solidified elements would also need to be reviewed. "The crust might have been thickened by the formation of "plutons", that is: igneous rock intrusions, hundreds of meters large, some of which emerged to the surface," explained the scientist. \\
If Vesta has less of a (olivine-rich) mantle and more of a (pyroxene-rich) crust, then the proportion of materials making up Vesta, and probably the Earth and other telluric planets (Mars, Venus, Mercury), is different from what was previously expected. \\
A more complex model of planet formation therefore has to be considered, one that takes into account not only the original composition of planets, but also their orbits, sizes and related cooling times. Vesta is the only known asteroid that has an earth-like structure - with a core, mantle and crust – making it an incredible laboratory for testing hypotheses and theories.}

\paragraph{\colorbox{yellow}{Golden Summary}}
\texttt{EPFL researchers have a better understanding of the asteroid Vesta and its internal structure, thanks to numerical simulations and data from the space mission Dawn. Their findings, published today in Nature, question contemporary models of rocky planet formation, including that of Earth.}

\paragraph{\colorbox{pink}{GPT4 Original Summary}}
\texttt{Data collected from asteroid Vesta poses challenges to planet formation models as it reveals the asteroid's crust is almost three times thicker than expected. The discovery, made by researchers from the Earth and Planetary Science Laboratory and partner universities, suggests that original assumptions about the composition of the cloud of matter that formed planets may be incorrect. The findings could lead to a more complex model of planet formation, taking into account composition, orbit, size, and cooling times.}

\paragraph{\colorbox{SkyBlue}{\textit{Correct Answer} Revision}}
\texttt{EPFL researchers have gained insights into asteroid Vesta's internal structure, utilizing numerical simulations and data from the Dawn space mission. The findings, published in Nature, indicate Vesta's crust is almost three times thicker than expected, questioning current models of rocky planet formation, including Earth's. This could lead to more complex planet formation models that consider composition, orbit, size, and cooling times.}

\paragraph{\colorbox{SpringGreen}{\textit{Task Elaboration} Revision}}
\texttt{Astonishing findings from asteroid Vesta are challenging our understanding of planet formation, as researchers discovered its crust is nearly triple the expected thickness. This breakthrough, made by EPFL's Earth and Planetary Science Laboratory and partner universities, questions key elements of planet formation models and paves the way for a more complex approach, considering composition, orbits, sizes, and cooling times. With Vesta being the only known asteroid with an Earth-like structure, this revelation creates a compelling lure for further explorations.}

\end{document}
