\section{Pedagogical models of feedback}
\label{app:feedback_definitions}

\subsection{Defining feedback}

Table \ref{tab:feedback_defintions} presents an overview of the various definitions of feedback put forward by several pedagogical works.


\begin{table*}[h]
  \centering
  \begin{tabularx}{\textwidth}{p{3.95cm}|X}
    \toprule
    Work  & Feedback Definition  \\
    \midrule
    \citet{ramaprasad_definition_1983} & Information which changes the gap between "the actual level and the reference level of a system parameter." This is quite a strict definition -- if the information does not change the gap, it is not considered feedback, and information about the actual level, the reference level and their comparison is needed beforehand. \\
    \midrule
    \citet{kulhavy_feedback_1989} & Refer to a previous definition of feedback, whereby feedback is considered "any of the numerous procedures that are used to tell a learner if an instructional response is right or wrong" \citep{kulhavy_feedback_1977}.
    \\
    \midrule
    \citet{sadler_formative_1989} & "Information about how successfully something has been or is being done." \\
    \midrule
    \citet{butler_feedback_1995} & A way to update the learner's internal state and knowledge, and subsequently task execution (a more learner-centric model of feedback). \\
    \midrule
    \citet{kluger_effects_1996} & The information provided by an external agent on one or more aspects of task performance. Note this excludes the learner as a possible source of feedback.  \\
    \midrule
    \citet{mason_providing_2001} & Feedback "is any message generated in response to a learner's action." \\
    \midrule
    \citet{narciss_how_2004, narciss_feedback_2008} & "All post-response information which informs the learner on his/her actual state of learning or performance in order to regulate the further process of learning in the direction of the learning standards strived for." \\
    \midrule
    \citet{nicol_formative_2006} & Information relating the learner's current state to the goal state (both with regards to learning as well as performance). Importantly, they consider students generate internal feedback and that the better they are at self-regulation, the better they will be at either generating or leveraging feedback. \\
    \midrule
    \citet{hattie_power_2007} & Information generated by an agent about the learner's understanding or their performance. \\
    \midrule
    \citet{evans_making_2013} & Feedback "includes all feedback exchanges generated within assessment design, occurring within and beyond the immediate learning context, being overt or covert (actively and/or passively sought and/or received), and importantly, drawing from a range of sources." \\
    \midrule
    \citet{anastasiya_a_lipnevich_david_a_g_berg_jeffrey_k_smith_toward_2016} & Feedback is information transmitted to the learner with the intent of changing their understanding and execution, in order to improve learning. \\
    \midrule
    \citet{carless_development_2018} & Feedback as the process through which the student understands and integrates information from various sources in order to improve their learning or performance (a more learner-centric perspective). \\
    \midrule
    \citet{lipnevich_review_2021} & Feedback "is information that includes all or several components: students’ current state, information about where they are, where they are headed and how to get there, and can be presented by different agents (i.e., peer, teacher, self, task itself, computer). This information is expected to have a stronger effect on performance and learning if it encourages students to engage in active processing." \\
    \bottomrule
  \end{tabularx}
  \caption{Different pedagogical works' definitions of feedback.}
  \label{tab:feedback_defintions}
\end{table*}


\subsection{Categorizing feedback}

\citet{kulhavy_feedback_1989} model feedback as having two components: the verification component, $f_v$, which is a simple discrete classification of the answer as correct or incorrect, and the elaboration component, $f_e$, consists of three elements:
\begin{enumerate}[itemsep=0.05em,label=(\roman*)]
    \item \textit{type}, whether the feedback contains information derived from the current task (task-specific), not from the task but from the relevant lesson (instruction-based), or beyond the relevant lesson, such as new information, examples or analogies not previously introduced (extra-instructional), 
    \item \textit{form}, the difference in structure between the feedback and instruction or task specification messages, requiring increased processing the less similar it is\footnote{The \textit{form} element does not apply to \textit{extra-instructional type} feedback, as there is no structural comparison point possible}, and
    \item \textit{load}, the total amount of information in the feedback - from a single "correct/incorrect" bit to including the correct answer to even more informative feedback accompanying it with an explanation, for example.
\end{enumerate}

\citet{mason_providing_2001} propose 8 feedback categories, arguing different types of feedback are best suited for different learner characteristics, taking into account the students' proficiency and prior knowledge, as well as the task difficulty.
The eight categories are:
\begin{enumerate}[itemsep=0.05em,label=(\roman*)]
    \item \textit{no-feedback}, which presents a single grade, 
    \item \textit{knowledge-of-response}, which analogously to the aforementioned verification component, indicates whether the given answer is correct or incorrect, 
    \item \textit{answer-until-correct}, an iterative variant of knowledge-of-response feedback, not allowing the student to progress until they have provided the correct answer, 
    \item \textit{knowledge-of-correct-response}, which provides the correct answer, 
    \item \textit{topic-contingent}, which provides both knowledge-of-response feedback and, analogously to \citet{kulhavy_feedback_1989}'s instruction-based type of feedback, provides general information about the topic of the task, where the learner might locate the correct answer, 
    \item \textit{response-contingent}, which similarly provides knowledge-of-response feedback as well as an explanation of why the answer is wrong or right (mapping it to \citet{kulhavy_feedback_1989}'s extra-instructional type of feedback), 
    \item \textit{bug-related}, providing knowledge-of-response feedback and bug-related feedback, which relies on rule sets to identify procedural errors, and
    \item \textit{attribute-isolation}, which provides knowledge-of-response feedback as well as information on the essential attributes of the relevant concept, focusing the learner on its key components.
\end{enumerate}

\citet{narciss_how_2004, narciss_feedback_2008} present a detailed and comprehensive feedback model, taking into account many learner and task characteristics. They also present a content-related feedback classification scheme, with eight categories: 
\begin{enumerate}[itemsep=0.05em,label=(\roman*)]
    \item \textit{Knowledge of performance (KP)}, analogous to \citet{mason_providing_2001}'s no-feedback and \citet{kulhavy_feedback_1989}'s verification component for a multiple-question task, presents the learner with an aggregate score (e.g., percentage or number of correct answers out of the total number of questions), 
    \item \textit{Knowledge of result/response (KR)}, directly mirrors \citet{mason_providing_2001}'s knowledge-of-response and \citet{kulhavy_feedback_1989}'s verification component for each question or task, classifying it as either correct or incorrect,
    \item \textit{Knowledge of the correct results (KCR)}, equivalent to \citet{mason_providing_2001}'s knowledge-of-correct-response, indicating the correct answer to the learner,
    \item \textit{Knowledge about task constraints (KTC)}, somewhat similar to \citet{mason_providing_2001}'s topic-contingent feedback, is elaboration feedback about the task, containing hints, examples or explanations about the type of task, its rules, sub-tasks, requirements and other constraints, 
    \item \textit{Knowledge about concepts (KC)}, containing some resemblance to \citet{mason_providing_2001}'s attribute-isolation feedback, is elaboration feedback on the relevant concepts, providing hints, examples or explanations on technical terms, the concept or its context, attributes, or key components, 
    \item \textit{Knowledge about mistakes (KM)}, which parallels \citet{mason_providing_2001}'s bug-related feedback, provides elaboration feedback containing the number of mistakes, their location, and hints, examples or explanations on error types and sources, 
    \item \textit{Knowledge about how to proceed (KH)}, elaboration feedback on the general know-how of the task, containing hints, examples or explanations on error correction, task-specific solving strategies or processing steps, guiding questions and worked-out examples, and
    \item \textit{Knowledge about metacognition (KMC)}, elaboration feedback going beyond the context of the current task, and providing hints, examples, explanations, or guiding questions on metacognitive strategies.
\end{enumerate}

\noindent \citet{hattie_power_2007} present a small typology about the information being conveyed about the learner in the feedback message, presenting 3 questions feedback can answer: 
\begin{enumerate}[itemsep=0.05em,label=(\roman*)]
    \item where the learner is going (\textit{feed up}), 
    \item how they are going (\textit{feed back}), and 
    \item where to next (\textit{feed forward})
\end{enumerate} 
and argue feedback is effective if it answers all three. 