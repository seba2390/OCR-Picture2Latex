\section{Feedback in NLP}
\label{sec:background_nlp}


\subsection{How informative is feedback for LLMs?}

The value of feedback is derived from the implicit information it represents about human values and expectations, that would otherwise be extremely difficult to specify \citep{christiano2017deeprl}. While all forms of feedback are able to reflect this knowledge to some degree, not all of them can represent the same amount and granularity of information.

\paragraph{Feedback Representation} Feedback can assume different forms: numerical ratings, rankings, preferences, demonstrations, and fully textual information (which can either be based on a rigid template or unconstrained, free-form text --- structured and unstructured feedback, respectively). 

\paragraph{Learning From Feedback} The most popular RLHF methodologies usually collect either a numeric rating or ranking from human workers for classifying the \textit{quality} of feedback (typically focused on encouraging \textit{helpfulness} and \textit{honesty} while mitigating \textit{harmfulness}; \citealp{askell2021general}). 
RLF may also leverage demonstrations to finetune LLMs in a supervised fashion before the RLF stage takes place to reduce the subsequent search space \citep{ouyang_training_2022, bai_constitutional_2022}, with scalar or ranking data subsequently used to train the reward model. This approach attempts to address the intractable problem of designing an appropriate loss function to express the aforementioned goal of honest, helpful and harmless language models \citep{askell2021general}.

\paragraph{Feedback Alignment} However, the extent to which feedback (\ie, information that LLMs are finetuned on) transmits these goals remains unclear.  
For example, InstructGPT\footnote{And later OpenAI models such as \texttt{text-davinci-003}.} \citep{ouyang2022instructGPT} is finetuned on demonstration data, and subsequently trained using RLHF with a reward trained using comparison data (\ie, specifically, pairs of ranked generations). 
This feedback is limited in the amount of information it transmits. For a given prompt, marking demonstration A as better\footnote{We note such a format also obfuscates any bias and disagreement that occurred in reaching such a judgment} than demonstration B provides little information on the quality of A nor B, nor on whether A fully outclasses B, or whether B may surpass A in some some dimensions. In any case, such a format provides no information on how either demonstration can be improved. Taking both these limitations and human bias into account, RMs are likely to suffer from some degree of distortion and misalignment. Other approaches \citep{liu2023chain, gao2022simulating} to model training with human feedback also still rely on simple ranking or numerical feedback.
%
Constitutional AI  \citep{bai_constitutional_2022} employs a similar approach, but the feedback --- both the textual feedback used for initial supervised finetuning and the ranking feedback used to train the RM --- is generated by LLMs\footnote{Only the harmlessness feedback is generated by an LLM, human feedback is used for the helpfulness dimension.} rather than human workers (\ie, RLAIF). While using LLM-generated demonstrations makes the method more scalable for data collection, the same challenges of remain. 

\begin{figure}[t]
\centering
\includegraphics[width=1.0\columnwidth]{figures/final_rel_work.png}
\caption{Connecting feedback research in NLP to foundations of feedback in the Learning Sciences.}
\label{fig:base_framework}
\end{figure}

\paragraph{Informational Limitations} Recent works have started to acknowledge the limited information in the aforementioned feedback formulations, recognizing them as unsuited for capturing critically relevant information, such as different types of errors \citep{golovneva2023roscoe, wu2023finegrained}. 

\subsection{How can textual feedback improve LLMs?}

The most commonly used feedback formulations, scalar and ranking feedback, are thus limited in the information they can convey. An intuitive alternative is to instead leverage textual feedback. 

\paragraph{External Enhancement} Augmenting the model externally --- be it through data augmentation \citep{shi2022life}, external corrective feedback \citep{tandon-etal-2022-learning, madaan-etal-2022-memory, shinn2023reflexion} or natural language patches \citep{murty2022fixing} --- is one relatively straightforward approach to incorporating textual feedback into a LLM.


\paragraph{Revising Generations} Various works have instead introduced a secondary model, that either refines an original LLM's answer \citep{scheurer2022training, welleck2022generating, tandon-etal-2022-learning}, critiques it \citep{saunders2022selfcritiquing, paul2023refiner} or iteratively self-improves \citep{schick2022peer, chen2023teaching, madaan2023selfrefine}. Several of these approaches leverage the same LLM for both the original answer generation as well as its refinement, but all of them rely on textual feedback --- be it for eventual dataset augmentation and refinement, or as part of the input to the new answer revision. 
All these models, beyond leveraging some kind of natural language feedback, also target intermediate generations with their feedback, not the final outcome. This intermediate feedback is another mechanism to transmit more information to a model. Rather than increasing the feedback complexity, these approaches increase the number of feedback opportunities, through multiple iterations \citep{lightman2023lets}.



\paragraph{What is missing?}
A clear trend towards more informative feedback is underway, drifting away from the still dominant approach of reducing feedback to a single scalar or ranking. However, the textual feedback employed by different works are often completely different from one another. No work so far has taken up a true mapping of the feedback space, identified the different types of information that can be encoded in NLF, and allowed for an exploration of different feedback components and their effectiveness.


\subsection{What types of textual feedback have been explored in NLP?}

Given the poverty of feedback forms used to train LLMs, a variety of works have recently emerged to use natural language feedback to correct LLMs, but this area remains in its infancy. A recent survey on how feedback is receive and integrated with LLMs \citep{fernandes_bridging_2023} recognizes the limitation of current score-based approaches to feedback, and proposes that future work should leverage the much richer signal of NLF.  
%
\citet{shi2022life} distinguishes textual feedback depending on whether the feedback is being formally provided for the model's answer, or whether, remaining in the dialog setting, the user mentions they disliked the reply they received. 
%
\textsc{SELF-REFINE} \citep{madaan2023selfrefine} argues that the quality of the generated feedback is critical, though they only compare their ``actionable and specific'' LLM-generated feedback against ``generic feedback'' and the complete absence of feedback in an ablation study.
% 
\citet{wu2023finegrained} propose the introduction of finer-grained feedback at sub-sentence, sentence and full sentence levels --- and of three different error types: factual incorrectness, irrelevance, and information incompleteness. Despite the impressive performance of this approach, the feedback exploration is limited at only three specific types, and only preference rankings are used.
%
Finally, \citet{weston_dialog-based_2016} conducted the most thorough exploration to date, exploring 10 different dialogue-based supervision modes, which represent different interaction and feedback types. However, these modes often overlap information-wise, limiting the conclusions of the study.  



\section{Feedback in Education}
\label{sec:background_pedagogy}

In this paper, we study feedback in human learning to construct a comprehensive, theory-grounded feedback taxonomy that directly addresses the limited exploration of natural language feedback. 
%
We build off the work of \citet{lipnevich_review_2021}, who conducted a systematic review of work in the fields of education, psychology, information processing and assessment philosophy, to eventually select the 15 most relevant and influential works on feedback models research. In this section, we provide a brief overview of the key points of each of these works --- related to the definition, effectiveness, and characteristics of feedback --- and draw inspiration from them to subsequently propose a framework for feedback integration (\S\ref{sec:framework}) and a taxonomy for feedback content (\S\ref{sec:taxonomy}).

\subsection{What is feedback?}

Many prior works have proposed a definition of feedback, and all agree that feedback either is information or contains information provided to a learner.\footnote{However, this definition is not a sufficient condition for some of these works. \citet{carless_development_2018}, for example, reflects a learner-centric perspective, viewing feedback as the process through which the student understands and integrates information --- thus, without the student processing, there is no feedback even if the information is present.} Consensus starts to wane on the other properties feedback must possess, one of which we note as particularly interesting: roughly half of the feedback models  \citep{ramaprasad_definition_1983, butler_feedback_1995, narciss_how_2004, narciss_feedback_2008, nicol_formative_2006, hattie_power_2007, lipnevich_review_2021, panadero_review_2022} incorporate the idea of a \textit{gap}, stating that feedback should provide the learner with information about the difference between their actual performance and the target performance. In contrast, the remaining models do not explicitly bridge the necessity of a performance gap in their formulation of feedback.


\paragraph{Defining Feedback} As a result, while different studies may disagree on the breadth or specificity required for feedback, and the limitations on its content, purpose or effect to be considered feedback, a definition (which we adopt throughout this paper) nevertheless emerges from their points of consensus:\footnote{For an overview of all the different definitions of feedback discussed, please see Appendix \ref{app:feedback_definitions}.} \textit{any task-relevant information given to a learner, by any possible feedback-generating agent (including internal feedback)}. Note that we do not impose any constraints on the information that is given to the learner.

\subsection{What constitutes effective feedback?}
\label{sec:eff-feedback}

\citet{kluger_effects_1996} showed that in $38\%$ analyzed cases, feedback had a detrimental effect on a learner's performance, challenging the intuitive understanding of feedback as helpful information. This reality requires reflecting on the properties of feedback that effectively help learners improve. Three main conditions for ensuring helpful feedback have emerged from previous work: \textit{applicability}, \textit{learner regulation}, and \textit{personalization}.


\paragraph{Applicability} Feedback should be actionable, i.e., should support the learner in achieving the target performance. Applicable feedback develops naturally from the directives about clarifying the task's objective, providing quality information, and creating opportunities for the learner to improve. \citet{sadler_formative_1989} therefore suggest that feedback needs to identify a target performance, compare the learner's current performance to it, and engage in actions to reduce that difference. Similarly, \citet{hattie_power_2007} indicate that effective feedback needs to answer three questions: where the learner is going (the goal), how they can get there, and where to go next. Other works extend these definitions of effective feedback by including elements such as motivational and metacognitive aspects \cite{nicol_formative_2006} or aspects of teaching (\eg lesson design; \citealp{evans_making_2013}). 


\paragraph{Learner Regulation} Effective feedback produces a positive response in the learner. \citet{kluger_effects_1996} argue that, in response to feedback, a learner's attention will be directed to one of three levels: how to solve the task, the task as a whole, or meta-task processes (processes the learner is doing while performing the task). Others \cite{nicol_formative_2006,evans_making_2013} further note that effective feedback also enhances self-regulated learning behaviors. \citet{narciss_how_2004, narciss_feedback_2008} extend this definition by arguing that feedback can have three distinct types of impact: influence on the learner's cognitive abilities, their metacognitive skills, or their motivation and self-regulation. \citet{anastasiya_a_lipnevich_david_a_g_berg_jeffrey_k_smith_toward_2016} defend that when a student receives feedback, they produce cognitive and affective responses. Namely, the learner will judge how worthwhile the task is, how much they control the outcome, and how understandable the feedback is. In turn, this judgment produces a behavioral reaction, influencing their performance and learning. Similarly, \citet{panadero_review_2022} state that feedback impacts both the students' performance and learning as well as their affective processes and self-regulation. 



\paragraph{Personalization} Different types of feedback are best suited for different learner characteristics and should be adapted accordingly \citep{mason_providing_2001}. Furthermore, the learner's individual characteristics will directly impact how the process feedback \citep{anastasiya_a_lipnevich_david_a_g_berg_jeffrey_k_smith_toward_2016}. \\

\noindent We conclude that not all feedback is good feedback, though different models propose different rule sets for achieving effective feedback, suggesting a need for further exploration. Furthermore, feedback can be given with different purposes, from directly improving the learner's performance, to clarifying the task, to improving metacognitive skills, and to help them regulate their emotions, motivation, and inner-processes. We revisit this concept of feedback purpose in Section \ref{sec:taxonomy}.

\subsection{What are the characteristics of feedback?}
In Section \ref{sec:eff-feedback}, we summarized different properties of feedback, observing that not all types of feedback are effective for learning in every situation. A large body of research has therefore attempted to systematically categorize feedback based on its \textit{content}, how it is given (\textit{timing}, \textit{source}), and the variables influencing it (\textit{task}, \textit{learner}).

\subsubsection{What are the types of feedback?}
While there is a plethora of work on systematically categorizing feedback, previous work can be broadly divided into two groups: taxonomies of feedback focusing on the content of the feedback only and taxonomies taking into account the whole ecosystem of the feedback.\footnote{Appendix \ref{app:feedback_definitions} presents a more thorough definition of proposed feedback categories in the learning sciences.}

\paragraph{Narrow} 
Works in this category focus on the characteristics of the content only. \citet{kulhavy_feedback_1989}, for example, model feedback through a verification component, which is a simple discrete classification of the answer as correct or incorrect, and an elaboration component, which contains all other information. Other works \cite{hattie_power_2007, panadero_review_2022} suggest three categories for classifying feedback: (i) addressing the learner's performance goal, (ii) addressing the learner's current performance, and (iii) addressing the next steps the learner should undertake.

\paragraph{Broad} In contrast to the first group, works in the second group suggest a more comprehensive categorization of feedback including the whole feedback ecosystem. For example, several works propose different feedback categories that take into account characteristics of the learner (student proficiency, prior knowledge) and the task (difficulty) \citep{mason_providing_2001,narciss_how_2004,narciss_feedback_2008}. 


\subsubsection{How is feedback given?}
\label{sec:feedback_how}
Apart from its content and the ecosystem around it, feedback has also been characterized by the manner in which it is given. Two main components have emerged in the literature.

\paragraph{Source} Feedback can be given by different sources (\eg, teachers, peers, or even the learner themself). In their systematic review, \citet{lipnevich_review_2021} found that seven out of the 15 considered models view the source as an important characteristic of feedback. An additional three works distinguish feedback generated by an external source from feedback generated internally.

\paragraph{Timing} The timing of feedback is considered essential too, with previous differentiating between immediate feedback and delayed feedback. While early work \cite{bangert-drowns_instructional_1991} found delayed feedback to be more effective, more recent works \cite{mason_providing_2001} argue that the optimal timing of feedback depends on learner characteristics. For example, \citet{hattie_power_2007} state that the most beneficial timing depends mainly on the complexity of the task; complex tasks benefit from delayed feedback as they allow the learner to properly process the task. Other works \cite{narciss_feedback_2008} focus mainly on the learner characteristics, arguing that as long as the learner possesses the metacognitive skills required to spot and address mistakes, feedback should be delayed. 


We incorporate both feedback source and timing in our framework presented in Section 
\ref{sec:framework}. Other points of contention remain beyond these two dimensions, such as feedback valence, but consider valence as an element of the feedback's content, and as such discuss it only in Section \ref{sec:taxonomy}.

\subsubsection{What variables influence feedback?}
Finally, feedback cannot only be categorized according by its content and the way it is given, but is also characterized by the ecosystem surrounding it. In the works cited in the previous two sections, many authors mention the challenge of determining optimal feedback type in isolation. Instead, the characteristics of the \textit{task} and the \textit{learner} are important to take into account when giving feedback.

\paragraph{Task} The characteristics of a task have been shown to influence the optimal timing of feedback (see Section \ref{sec:feedback_how}) as well as the content of the feedback.  In particular, \citet{mason_providing_2001} takes into account the complexity of the task when choosing the most suitable feedback for a given setting and \citet{narciss_how_2004, narciss_feedback_2008} incorporate the task and the instructional content and goals into the instructional factors that affect feedback. Also the nature of the task (\eg closed versus open-answer) influences feedback content and processing \cite{lipnevich_review_2021}.

\paragraph{Learner} Previous work has also acknowledged the impact of learner characteristics on the effectiveness of feedback. 
\citet{mason_providing_2001}, for example, consider student achievement and prior knowledge as important factors directly impacting the best suited type of feedback. \citet{nicol_formative_2006} expands the learner's prior knowledge and proficiency into \textit{domain knowledge} and \textit{strategy knowledge} (along with \textit{motivational beliefs}), which are updated upon each attempt through both internal and external feedback.\footnote{Information for internal feedback generation is derived not only from the learner's initial state but also from their goals, tactics and strategies and their (internal) learning outcomes --- through self-regulatory processes.}
\citet{narciss_how_2004} and \citet{narciss_feedback_2008} flesh out the learner characteristics even further, including factors such as learning goals and motivation. Similarly, \citet{anastasiya_a_lipnevich_david_a_g_berg_jeffrey_k_smith_toward_2016} identify ``personality, general cognitive ability, receptivity to feedback, prior knowledge, and motivation'' as key learner characteristics that impact feedback processing.

Other works directly focus on feedback processing mechanisms of the learner. \citet{kluger_effects_1996} propose three processing levels: details about how to solve the task, the task itself as whole, and meta-task processes.
In contrast, \citet{hattie_power_2007} present four levels of feedback processing: (1) \textit{task level}, conveying how well the tasks are understood and performed, (2) \textit{process level}, the process needed to achieve the \textit{task level} understanding and performance, (3) \textit{self-regulation level}, related to self-monitoring and the direction and regulation of the learner's actions, and (4) \textit{self level}, which reflects personal evaluations about the learner as a whole.
