\section{The FELT framework's components}
\label{app:felt_interactions}

The FELT framework introduced in Section \ref{sec:framework} presents an important overview of all the factors that influence feedback and are in turn influenced by it. Figure \ref{fig:felt_framework} showcased a schematic overview of the FELT framework, integrating four distinct components: Feedback, Errors, Learner, and Task. In this appendix, we will outline more precisely each of the components of the FELT Framework, as well as the interactions between them.

\subsection{Task}
Typically, the task will be the first element to be defined. 

\paragraph{Nature of Task} In this paper, we have limited tha nature of the task to the answer type. Understanding itis fairly easy -- a task has a closed-answer if there is a finite set of correct answers, and an open-answer otherwise. Notably, tasks can contain both elements. For example the task "\textit{Write a quality 4-paragraph short-story}" has both open- and closed-answer elements. There is no finite set of answer of what a quality story is, but whether a story has 4 paragraphs, or not, is a binary closed-answer, as seen in Appendix \ref{app:feedback_examples}.

\paragraph{Complexity} The difficulty level of a task is harder to define as some measure of relativity is involved. We suggest anchoring this measurement to the average adult human capabilities. A simple arithmetic task will thus be considered very easy, whereas researching and writing a doctoral thesis would be seen as hard.

\paragraph{Prompt Instructions} The task instructions will be presented to the model at two distinct points in time: when first assigning the model this task, and when later providing feedback. With regards to the former, this element captures the degree to which the task is explained -- is the model explicitly aware of all criteria it should satisfy? With regards to the second pass, when feedback is provided, this dimension pertains instead with the degree of freedom it gives the LLM -- is the model forced to take the feedback into account, or can it consider only part of it, or even disregard it altogether if it deems it useless?

\subsection{Learner}
Either at the same time the task is defined or immediately after, the model to be tested will be chosen. The model choice influences two important features.

\paragraph{Prior Knowledge} The prior knowledge captures the LLM's abilities as a direct result of its size, training data, and training method. These, in turn, also reflect the model's purpose (e.g., was it designed to be helpful, harmless, entertaining, etc.). The prior knowledge thus captures the model's  representation of the learner, and in its architecture and parametric knowledge, it encodes the LLM's current abilities -- or its proficiency -- both in general and with regards to the specific task.

\paragraph{Feedback Processing Mechanism} Mainly defined by the experimental setup, the mechanism by which the model process feedback can vary significantly, and not all of them are able to leverage the same level of information. Imitation learning, for example, can only leverage information which was positively evaluated. As stated in Section \ref{sec:framework}, we identify 4 main processing mechanisms, 3 of which alter the model's parametric state -- feedback-based imitation learning, joint-feedback modeling, and reinforcement learning, as defined in \citet{fernandes_bridging_2023} -- and a fourth, non-parametric mode: in-context learning \citep{brown2020language}.

\subsection{Errors}
After both the task and learner are in place, the first pass of the experiment can be run, where the model will have its first attempt at solving the task. In this attempt, it is expected that the model will make some degree of mistakes -- which have two important characteristics.

\paragraph{Error Type} There are several possible types of errors, and their differences are significant. For example, an error made due to a guess only needs to provide the learner with the right information for it be be corrected, whereas a systematic error (for example, the mixing of British and American English spellings) will require a different, much more insistent, intervention. ROSCOE \citep{golovneva2023roscoe} proposes a taxonomy of step-by-step reasoning errors. While task dependent (i.e., there are grammar errors and arithmetic errors, rather than fully task independent failure modes), this taxonomy provides a good starting ground for the exploration of error types in NLP.

\paragraph{Error Severity} Besides the type of error, it is also important to take the severity of the error into account. Stating that Marie Curie was a German philosopher and stating that she won one Nobel Prize in her lifetime are both factually inaccurate -- but one is a severe, complete hallucination, while the other omitted she actually won the Nobel Prize twice. The more severe the error, the stronger, more insistent, and more corrective the feedback should be.

\subsection{Feedback}
Finally, after the model has finished its first attempt at the task, producing some number of errors, feedback can be provided on this attempt.

\paragraph{Timing} One easy to neglect aspect of feedback that pedagogy has shown to be impactful is timing -- whether the feedback is provided immediately after a task is attempted or whether there is a delay between the two actions. There are differing opinions amongst education researchers, but how to make feedback content more effective through timing merit research in LLMs. For example, in line with \citet{Mathan2005FosteringTI} and \citet{narciss_feedback_2008}'s take on timing -- delay feedback if the learner possesses metacognitive abilities that allow them to identify and possibly correct mistakes -- we posit feedback will be more effective if, content-wise, it is preceded by information on the answer's correctness and mistakes' existence and only after this metacognitive priming is the rest of the information presented.

\paragraph{Content} Section \ref{sec:taxonomy} explores feedback content in depth, presenting 10 impactful axes on which it can vary: length, granularity, applicability of instructions, answer coverage, criteria, information novelty, purpose, style, valence, and mode. It also presents a set of 9 emergent categories which, based on pedagogical research, we estimate to be the most promising one with regards to impact on revised model generations, and thus most deserving of further study.

\paragraph{Source} Finally, it is also important to consider the source of feedback, which might be an authority, such as an expert, an average human, another LLM, a rule-based system, among others. Different sources will reflect different authority and reliability levels.

\subsection{Interactions}
With a clear understanding of all the components and sub-components of the FELT framework, we can explore the influences that exist between them.

Both the task complexity and the learner's prior knowledge can impact the ideal feedback timing -- be it delayed when the learner has metacognitive skills \citep{narciss_feedback_2008} or enough task proficiency \citep{mason_providing_2001} they can identify where the mistake occurred, or, for example, immediate if they don't \citep{narciss_feedback_2008} or the task difficulty is low \citep{mason_providing_2001}.

With regards to the feedback content, the type of task \citep{butler_feedback_1995, kluger_effects_1996, mason_providing_2001, anastasiya_a_lipnevich_david_a_g_berg_jeffrey_k_smith_toward_2016} and both the error type and severity will have an impact \citep{narciss_how_2004, narciss_feedback_2008}. The nature of the task (open or closed answer) will directly condition the feedback that can be given in response to the model's answer, as well as how difficult it will be to produce it. For example, generating the correct answer for a multiple choice quiz or a story writing task will be two very different endeavors. Similarly, it is impossible to provide response elaboration feedback on a single multiple choice question.
The error type and severity will also influence the feedback content, as apart from directly dictating what mistakes verification and elaboration feedback can be given, they will also condition the ideal amount of detail and explanations to address the mistake at the most efficient level.

Finally, all aspects of feedback will influence the learner's feedback processing mechanism \citep{kulhavy_feedback_1989, sadler_formative_1989, bangert-drowns_instructional_1991, butler_feedback_1995, kluger_effects_1996, narciss_how_2004, nicol_formative_2006, narciss_feedback_2008, anastasiya_a_lipnevich_david_a_g_berg_jeffrey_k_smith_toward_2016, carless_development_2018}. All three dimensions of feedback have evident potential to directly influence how the model processes them. The instruction's permissiveness to consider or discard feedback will also impact the learner's feedback processing mechanism. This processing is, of course, dependent on the specific processing mechanism employed, and while some might be indifferent to some of these components -- like imitation learning, for example, which focuses exclusively on the feedback content -- others will be sensitive to all, including the task's prompt instructions -- such as in-context learning.
