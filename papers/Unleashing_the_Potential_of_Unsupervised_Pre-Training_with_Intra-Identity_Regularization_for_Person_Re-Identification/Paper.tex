% CVPR 2022 Paper Template
% based on the CVPR template provided by Ming-Ming Cheng (https://github.com/MCG-NKU/CVPR_Template)
% modified and extended by Stefan Roth (stefan.roth@NOSPAMtu-darmstadt.de)

\documentclass[10pt,twocolumn,letterpaper]{article}

%%%%%%%%% PAPER TYPE  - PLEASE UPDATE FOR FINAL VERSION
% \usepackage[review]{cvpr}      % To produce the REVIEW version
%\usepackage{cvpr}              % To produce the CAMERA-READY version
\usepackage[pagenumbers]{cvpr} % To force page numbers, e.g. for an arXiv version

% Include other packages here, before hyperref.
\usepackage{graphicx}
\usepackage{amsmath}
\usepackage{amssymb}
\usepackage{booktabs}

\usepackage{bm}
% It is strongly recommended to use hyperref, especially for the review version.
% hyperref with option pagebackref eases the reviewers' job.
% Please disable hyperref *only* if you encounter grave issues, e.g. with the
% file validation for the camera-ready version.
%
% If you comment hyperref and then uncomment it, you should delete
% ReviewTempalte.aux before re-running LaTeX.
% (Or just hit 'q' on the first LaTeX run, let it finish, and you
%  should be clear).
\usepackage[pagebackref,breaklinks,colorlinks]{hyperref}


% Support for easy cross-referencing
\usepackage[capitalize]{cleveref}
\crefname{section}{Sec.}{Secs.}
\Crefname{section}{Section}{Sections}
\Crefname{table}{Table}{Tables}
\crefname{table}{Tab.}{Tabs.}

%colorize:
\usepackage{xcolor}
\newcommand{\tcb}{\textcolor{blue}}
\newcommand{\tcr}{\textcolor{red}}

% Add by Xin:
%\newcommand{\etal}{\textit{et al}.~}
%\newcommand{\ie}{\textit{i}.\textit{e}.~}
\newcommand{\ieno}{\textit{i}.\textit{e}.}
%\newcommand{\eg}{\textit{e}.\textit{g}.~}
\newcommand{\egno}{\textit{e}.\textit{g}.} %there is no space
%\newcommand{\etc}{\textit{etc}.}
\newcommand{\etcno}{\textit{etc}} %there is no "."

%%%%%%%%% PAPER ID  - PLEASE UPDATE
\def\cvprPaperID{4122} % *** Enter the CVPR Paper ID here
\def\confName{CVPR}
\def\confYear{2022}

\newlength\savewidth\newcommand\shline{\noalign{\global\savewidth\arrayrulewidth
  \global\arrayrulewidth 1pt}\hline\noalign{\global\arrayrulewidth\savewidth}}

\newcommand{\tablestyle}[2]{\setlength{\tabcolsep}{#1}\renewcommand{\arraystretch}{#2}\centering\footnotesize}

\usepackage{array}
\newcolumntype{C}[1]{>{\centering\arraybackslash}m{#1}}
\newcolumntype{R}[1]{>{\raggedleft\arraybackslash}m{#1}}
\newcolumntype{P}[1]{>{\raggedright\arraybackslash}p{#1}}
\newcolumntype{M}[1]{>{\centering\arraybackslash}m{#1}}
\usepackage{multirow}


\begin{document}

%%%%%%%%% TITLE - PLEASE UPDATE
% \title{Unsupervised Pre-Training with Discriminative and Robust Representation Learning for Person Re-Identification}
\title{Unleashing the Potential of Unsupervised Pre-Training with Intra-Identity Regularization for Person Re-Identification}

\author{Zizheng Yang \qquad Xin Jin \qquad Kecheng Zheng \qquad Feng Zhao\thanks{Corresponding Author.}\\
University of Science and Technology of China\\
{\tt\small yzz6000@mail.ustc.edu.cn}
}


\maketitle

%%%%%%%%% ABSTRACT
\begin{abstract}
% Unleash potential of contrastive learning with intra-identity regularization for Person ReID Pre-training

% unleash potential of unsupervised pre-training with intra-identity regularization for person Re-Identification

% As a fine-grained classification task, person Re-identification (ReID) is more challenging than object classification on ImageNet.
% Existing ReID methods typically directly load the models' weights pre-trained on ImageNet for initialization, suffering from large domain gap between ImageNet and person ReID datasets.
% Although constrastive learning based pre-training methods have achieved great success, only very few works attempt to pre-train a specific model for person ReID initialization and their results are very limited.  There are two critical issues: \textbf{Consistency Ensurance} scheme, propose a \textbf{Intra-identity Contrastive Learning} algorithm 

% Only very few works attempt to pre-train a specific model for person ReID


Existing person re-identification (ReID) methods typically directly load the pre-trained ImageNet weights for initialization. However, as a fine-grained classification task, ReID is more challenging and exists a large domain gap between ImageNet classification. Inspired by the great success of self-supervised representation learning with contrastive objectives, in this paper, we design an Unsupervised Pre-training framework for ReID based on the contrastive learning (CL) pipeline, dubbed UP-ReID. During the pre-training, we attempt to address two critical issues for learning fine-grained ReID features: (1) the augmentations in CL pipeline may distort the discriminative clues in person images. (2) the fine-grained local features of person images are not fully-explored. Therefore, we introduce an \textbf{intra-identity} (I$^2$-)regularization in the UP-ReID, which is instantiated as two constraints coming from global image aspect and local patch aspect: a global consistency is enforced between augmented and original person images to increase robustness to augmentation, while an intrinsic contrastive constraint among local patches of each image is employed to fully explore the local discriminative clues. Extensive experiments on multiple popular Re-ID datasets, including PersonX, Market1501, CUHK03, and MSMT17, demonstrate that our UP-ReID pre-trained model can significantly benefit the downstream ReID fine-tuning and achieve state-of-the-art performance. Codes and models will be released to \url{https://github.com/Frost-Yang-99/UP-ReID}.



% Specifically, \tcr{a global consistency is first enforced between augmented and original person images, increasing robustness to augmentation. Second, we construct an intrinsic contrastive constraint among local patches of each image to fully explore the local discriminative clues for different instances.} For better local representation learning, a hard-mining strategy is designed based on the prior that human body is horizontally symmetric. 


% Extensive experiments on multiple widely used Re-ID datasets, including xxx, Market1501, xxx, and MSMT17, demonstrate that the pre-trained model by UP-ReID can significantly benefit the downstream ReID fine-tune and achieves a state-of-the-art performance.


% within each person to replace traditional instance-wise CL to fully explore local discriminative clues of the person images. 

% no matter they are supervised or unsupervised. Our model achieves state-of-the-art matching performance and superior convergence efficiency on multiple widely-used person ReID benchmarks. 
% The code will be released soon.
\end{abstract}




%%%%%%%%% BODY TEXT
\section{Introduction}
\label{sec:intro}


\begin{figure}[ht]
	\centering
	\begin{subfigure}{.45\textwidth}
		\centering
		\includegraphics[width=\textwidth]{Motivation1.pdf}
		\caption{Left: the two augmented images are still discriminative for general classification tasks. Right: the discriminative attributes of two person images are ruined by augmentation for person ReID.}
		\label{fig:augmentation}
	\end{subfigure}%
	% \hfill
	\vspace{2mm}
	\begin{subfigure}{.475\textwidth}
		\centering
		\includegraphics[width=\textwidth]{Motivation2.pdf}
		\caption{Top: a case that uses the global features for a failed ReID, where the positive samples share dissimilar appearance but the negative samples have a similar appearance instead. Bottom: a case that uses the fine-grained discriminative attributes, such as backpacks and bags, for a successful ReID where the person images are distinguishable and independent to clothing.}
		\label{fig:fine-grained}
	\end{subfigure}
	\vspace{-3mm}
	\caption{Two critical issues in the existing contrastive learning based pre-training methods, which should be well solved in the ReID-specific pre-training framework.}
	\vspace{-3mm}
\end{figure}

As a fine-grained classification problem, person re-identification (ReID) aims at identifying a specific person across non-overlapping camera views. Existing ReID methods have achieved a remarkable success in both supervised ~\cite{wang2018learning,suh2018part,shen2018person,zheng2021pose,zhang2019densely,jin2020uncertainty,jin2020semantics} and unsupervised~\cite{yang2019patch,lin2019bottom,ge2020mutual,ge2020self,jin2020style,dai2021cluster} domains. Most of these approaches directly leverage the weights pre-trained on ImageNet for model initialization, which is not optimal for ReID tasks, resulting in poor fine-tuning performance and slow convergence~\cite{wang2018learning,ge2020self}. The main reasons stem from two aspects: inapplicable pre-training method (ImageNet is more like a coarse-grained classification), and large domain gap between ImageNet and ReID datasets. Thus, how to efficiently pre-train a good ReID-specific initialization network is still under-explored.


% Nonetheless, there are few works focusing on pre-training a specific network for person ReID initialization and all of them have very limited performance.

% Inspired by 

Unsupervised pre-training has achieved a fast development with the great success of contrastive learning~\cite{he2020momentum,chen2020improved,chen2020simple,chen2021exploring,caron2020unsupervised}, which is taken as a pretext work, serving for different downstream supervised or unsupervised ReID fine-tuning algorithms. Going beyond the general pre-training task, this paper aims to propose a ReID-specific pre-training framework (\eg, pre-training a ResNet50~\cite{he2016deep} for learning discriminative ReID representations) on a large-scale unlabeled dataset. The pioneering work of~\cite{fu2021unsupervised} makes the first attempt on ReID pre-training and introduces a new large-scale unlabeled ReID dataset LUPerson. However, it directly transfers the general pre-training process based on contrastive learning that designed for ImageNet classification to ReID task, which ignores the fact that ReID is a fine-grained classification problem. This solution faces the following two critical issues:

% based on contrastive learning
% Through investigating the nature of the existing constrastive learning methods and those of the person ReID, we find two critical issues that need to be resolved.

The first issue comes from the augmentations used in the existing contrastive learning pipeline, which could possibly damage the discriminative attributes of person images. As shown in Figure~\ref{fig:augmentation}, different from the coarse-grained classification problem on ImageNet, the discriminative attributes of person images are prone to be destroyed by the augmentation operations. For example, in the ImageNet classification, although the augmentations applied to the pictures~(\eg, dogs and ships) may cause the lack of regional information, the remaining parts are still discriminative enough to support the model for distinguishing them. 
However, when applying the same augmentations to person images in ReID, it will cause a disaster---the most discriminative attributes (\ieno, trousers color) of person images are destructive, making them indistinguishable.


% (\eg, random erasing, random crop)
% support the model to identity them, which will cause the model to update in the opposite direction to what we expect.

% pre-train a model with high classification performance on ImageNet, which , which has been proved in numerous ReID methods

The second issue is that the fine-grained information of person images is not fully explored in previous pre-training methods. They typically only care about the learning of image-level global feature representations. Nevertheless, as a fine-grained classification task, ReID needs detailed local features in addition to global ones for the accurate identity matching~\cite{wang2018learning,yang2019patch,sun2018beyond}. As illustrated in Figure~\ref{fig:fine-grained}, the local fine-grained clues (\egno, backpacks, cross-body bags) are more helpful than global features w.r.t distinguishing different persons.

% The augmented-original consistency algorithm regards the similarity distribution of the original images as labels to rectify those of the augmented images.

To address the above issues, we introduce an \emph{intra-identity (I$^2$-)} regularization in our proposed ReID-specific pre-training framework UP-ReID. It consists of a \emph{global consistency} constraint between augmented and original person images, and an \emph{intrinsic contrastive} constraint among local patches of each image.
Specifically, we first enforce a global {consistency} to make the pre-training model be more invariant to augmentations. We feed the augmented images as well as the original images into the model and then narrow the similarity distance between them in distributions.
Second, we propose an \emph{intrinsic contrastive} constraint for the local information exploration. Instead of directly feeding the holistic augmented images, we partition them into multiple patches and then send these patches along with the holistic images to the network. After that, We compute an intrinsic contrastive loss among patches to encourage the model to learn both fine-grained and semantic-aware representations. Moreover, based on the prior knowledge that human body is horizontally symmetric, we establish a hard mining strategy for the calculation of this loss, which makes the training stable and thus improves the generalization ability of the pre-trained model.



% Based on the aforementioned analysis, we propose to add fine-grained local information exploration in the pre-training process for person ReID, and design a Patch-based Person ReID pre-training framework (PP-ReID). Our key idea aims to encourage the model to learn both patch-based local features and global features of pedestrian images during the pre-training phase. Based on the recent mainstream self-supervised learning framework (\egno, MoCo~\cite{he2020momentum}, SimCLR~\cite{chen2020simple}), we additionally introduce a patch-based contrastive loss to exploit local discriminative information of pedestrian images. Specifically, we first partition two augmented person images into multiple non-overlapping patches uniformly, and then feed them into the model together with the holistic image. These patches highlight the local information better, and we compute the patch-based contrastive loss across them to enforce the pre-trained model pay more attention on those fine-grained details. A global contrastive loss is also preserved for overall features aggregation. Moreover, based on the prior knowledge that human body is horizontally symmetric, we establish a hard-mining strategy to select the hard positive patch pairs and hard negative patch pairs for the calculation of the patch-based contrastive loss. This effectively avoids the training unstable issue and thus improves the generalization capability of our pre-trained model.

We summarize our main contributions as follows:

\begin{itemize}

\item To the best of our knowledge, this is the first attempt toward a ReID-specific pre-training framework dubbed UP-ReID by explicitly pinpointing the difference between the general pre-training and ReID pre-training.

% We propose a pre-training framework with intra-identity regularization for person ReID. 

\item Considering the particularity of ReID task, we introduce an intra-identity (I$^2$-)regularization in our ReID pre-training framework UP-ReID, which is instantiated from the global image level and local patch level.

% We propose a augmented-original consistency scheme to encourage the model to learn augmentation-invariant feature representations. The instability due to augmentations will be largely reduced during the pre-training process.

\item In the I$^2$-regularization, a global consistency is first enforced to increase the robustness of pre-training to data augmentations. An intrinsic contrastive constraint with prior-based hard mining strategy among local patches of person images is further introduced to fully explore the local discriminative clues.


% We design an identity-wise contrastive constraint to encourage the model to learn fine-grained features as well as global features simultaneously, which significantly improves the ReID performance of the pre-trained model.

% \item We further introduce a hard-mining strategy to local information exploration based on the prior knowledge that human body is horizontally symmetric.

\end{itemize}

Extensive experiments on multiple widely-used ReID benchmarks demonstrate the effectiveness of the proposed pre-training framework UP-ReID. It outperforms the state-of-the-art pre-training methods by prominent margins, and could benefit a series of downstream ReID-related tasks.


\begin{figure*}[t!]
\centering
\centering
\includegraphics[width=16.5cm, height=10.5cm]{Framework3_new.pdf}
\caption{Architecture of the proposed UP-ReID. Given an input image, we can get two different groups of augmented instances after two different augmentations and partition. Then, we feed them into the online encoder and EMA encoder respectively, together with the original images. A consistency loss is computed to narrow the gap between the similarity distribution of the augmented images and that of the original images. We also compute an intrinsic contrastive loss based on a delicately designed hard mining strategy. The EMA encoder features are used to update the queue bank. The online encoder is optimized by the gradient of the total loss, while the EMA encoder is updated by momentum-based moving average of the online encoder.}
\label{fig:Framework}
\vspace{-2mm}
\end{figure*}



\section{Related Work}

\subsection{Person ReID}

\noindent\textbf{Fully-supervised ReID approaches.} Fully-supervised ReID methods are based on supervised learning with labeled datasets and have achieved a great success~\cite{li2014deepreid,shen2018person,zhang2019densely,jin2020semantics}. These works can be divided into two mainstream branches. One focuses on designing effective optimization metrics (i.e., metric learning) for person ReID, such as hard triplet loss~\cite{hermans2017defense} and circle loss~\cite{sun2020circle}. On the other hand, learning fine-grained features is also a popular branch. PCB~\cite{sun2018beyond} and MGN~\cite{wang2018learning} both leverage local features of pedestrian images by manually splitting each holistic image into multiple sub-parts to achieve accurate person ReID. These methods are limited by the large-scale annotations and cannot be directly applied to unlabeled datasets.


\noindent\textbf{Unsupervised ReID approaches.} There are two typical categories of unsupervised person ReID: Unsupervised Domain Adaptation (UDA) based methods and Domain Generalization (DG) based methods. 1) UDA could handle the domain gap issue when the target domain data are accessible, which aims to learn a generic model from both labeled source data and unlabeled target data. The UDA-based methods can be further categorized into three main classes: style transfer based works~\cite{deng2018image,wei2018person,zheng2019joint}, attribute recognition based works~\cite{wang2018transferable,qi2019novel} and pseudo labeling based works~\cite{song2020unsupervised,ge2020mutual,ge2020self}. 2) DG is designed for a more challenging case where the target domain data are unavailable. Jin etal~\cite{jin2020style} designs a Style Normalization and Restitution(SNR) module to enhance the identity-relevant features and filter out the identity-irrelevant features for improving the model's generalization ability. In addition, meta-learning~\cite{zhao2021learning} is also employed as a popular way to achieve person ReID specific domain generalization. However, all these methods generally load the pre-trained ImageNet weights for initialization, and ignore the gap between the ImageNet classification and the fine-grained ReID task.

% Contrastive learning-based methods~\cite{chen2021ice,isobe2021towards} also achieves a great success in purely unsupervised ReID on small-scale datasets.

\subsection{Self-Supervised Representation Learning}

Based on the recently popular contrastive learning, the unsupervised pre-training has achieved a great success, and many representative works have achieved comparable or even slightly better performance than supervised works. Moco~\cite{he2020momentum} and Moco v2~\cite{chen2020improved} design a dynamic queue and introduce a momentum update mechanism to optimize a key encoder progressively. SimCLR\cite{chen2020simple} and SimCLR v2~\cite{chen2020big} also achieve great performance with a large batch size, rich data augmentations and a simple but effective projection head. BYOL~\cite{grill2020bootstrap} and SimSiam~\cite{chen2021exploring} further achieve great performance even without negative pairs. SwAV~\cite{caron2020unsupervised} replaces comparison between pairwise samples with comparison between cluster assignments of multiple views.

% \noindent\textbf{Pre-training for Person ReID.} 
The work of~\cite{fu2021unsupervised} proposes a new large-scale unlabeled dataset ``LUPerson'' which is large enough to support pre-training and makes the first attempt to pre-train specific models for person ReID initialization. However, since the work merely migrates the approach of pre-training models on ImageNet to ReID directly, it suffers from the instability issue (see Figure~\ref{fig:augmentation}) caused by augmentation and lacked of the exploration of fine-grained discriminative information of pedestrian images (see Figure~\ref{fig:fine-grained}). 
% In this work, we study how to design a ReID-specific pre-training framework to mitigate the above issues.
% In this work, we study how to design a pre-training framework that avoids data augmentation interference while fully using fine-grained local information for discriminative representation learning.


% Unsupervised  Pre-training  framework  for  ReID 

\section{Unsupervised Pre-training for ReID}


Person ReID training typically contains two procedures of pre-training and fine-tuning: (a) the model~(\eg, ResNet50) is first \emph{pre-trained} unsupervisedly on a large-scale dataset~(\eg, LUPerson~\cite{fu2021unsupervised}) with a pretext task, (b) and then the pre-trained model is utilized to initialize the backbone and \emph{fine-tuned} with small-scale labeled or unlabeled person ReID datasets~(\eg, Market1501~\cite{zheng2015scalable}). In this paper, we focus on the first phase, \ieno, how to pre-train a ReID-friendly model in an unsupervised manner.

%  on the downstream tasks

We first overview the whole pipeline of our UP-ReID in Section~\ref{section: overview}, and then introduce the proposed \emph{$I^2$}-regularization for pre-training, which comprises a global consistency constraint (Section~\ref{section: consistency constraint}) and an intrinsic contrastive constraint (Section~\ref{section: intrinsic contrastive contraint}). Last but not least, a prior-based hard mining strategy employed for local feature enhancement is discussed in Section~\ref{section: hard mining}.

\subsection{Overview}
\label{section: overview}


As illustrated in Figure~\ref{fig:Framework}, UP-ReID has two encoders: an online encoder $f_q$ and a momentum-based moving averaging (EMA) update encoder $f_k$. Both $f_q$ and $f_k$ are composed of a feature encoder and a projection head. The feature encoder is the model to be pre-trained~(\eg, ResNet50), and the projection head is a multi-layer perceptron. The online encoder $f_q$ will be updated by back-propagation and the EMA encoder $f_k$ will be slowly progressed through momentum-based moving average of the online encoder $f_q$, which is $\theta_k \gets m\theta_k + (1-m) \theta_q$. $\theta_k, \theta_q$ represent the parameters of $f_k, f_q$, and $m$ means the momentum coefficient.
% \begin{equation}
% \label{momentum update}
%     \theta_k \gets m\theta_k + (1-m) \theta_q,
% \end{equation}
% where $\theta_k, \theta_q$ represent the parameters of $f_k, f_q$, and $m$ means the momentum coefficient.


% which has been proved to be important in pre-training stage~\cite{chen2020improved,chen2021exploring}.


Given an input image $x$, we can get two different views of $x$, \ie, a query view ${x}_{q,0}$ and a key view ${x}_{k,0}$, after two different augmentations. 
Unlike previous contrastive learning methods that only take the augmented images ${x}_{q,0}$ and ${x}_{k,0}$ as the input, we also feed the original image $x$ into the network as shown in Figure~\ref{fig:Framework}. Then, we enforce a consistency loss $\mathcal{L}_{consist.}$ to narrow down the distance between the similarity distribution of the augmented images and that of the original images in a mini-batch, which is described in detail in Section~\ref{section: consistency constraint}.

Moreover, before feeding ${x}_{q,0}$ and ${x}_{k,0}$ into the network, we partition each of them into $M$ non-overlapping patches. Note that, all $2M$ patches are partitioned from the same person image $x$ actually. Then, we feed these patches along with the entire augmented images into the online encoder and EMA encoder. An intrinsic contrastive loss $\mathcal{L}_{inc}$ is computed over them to learn both fine-grained local representations and the semantic image-level representations, which is discussed in detail in Section~\ref{section: intrinsic contrastive contraint}. For a better fine-grained information exploration, a hard mining strategy is further introduced to the calculation of the intrinsic contrastive loss, which is presented in Section~\ref{section: hard mining}. Ultimately, the total optimization objective is defined as:
% $\mathcal{L}_{total} = \mathcal{L}_{consist.} + \mathcal{L}_{inc}$.
\begin{equation}
\label{total loss}
    \mathcal{L}_{total} = \mathcal{L}_{consist.} + \mathcal{L}_{inc}.
\end{equation}

Additionally, a dynamic queue bank is constructed to store the feature representations of previous mini-batches and provide sufficient negative samples for the current mini-batch training. In practice, we prepare a queue for the image-level features, \ie, ${Q}_{0}$, and a queue for each patch-level local features, \ie, ${Q}_{i}$, $i\in\{1,...,M\}$. All of these queues together constitute the queue bank and they are dynamically updated by the features extracted by the EMA encoder.

% It needs to be emphasized that the dynamic queue bank is necessary for our framework. Although contrastive learning methods without negative samples~\cite{grill2020bootstrap,chen2021exploring} has achieved great success, we find that the models pre-trained with both positive samples and negative samples have better performance than the models pre-trained with only positive samples in our experiments, which reveals that negative samples are essential to pre-training for person ReID.


\subsection{Consistency over Augmented-Original Images}
\label{section: consistency constraint}

Data augmentation plays a crucial role in contrastive learning. However, discriminative attributes of pedestrian images are very likely to be ruined by various augmentation operations (see Figure~\ref{fig:augmentation}). Due to the visual distortions caused by augmentation, a sample may be less similar to its positive instances but more similar to its negative samples instead, which inevitably imposes a negative effect on the pre-training process. 

% We believe that the discrepancy between two augmented instances from the same raw image must be within a reasonable range, \ie, both two augmented images contain similar visual patterns that are able to support the model to identify them; and two augmented instances from two different raw images should be discirminative enough to support the model to distinguish them.
% , due to the visual distortions caused by augmentation.

To alleviate this problem, we turn to the original images for help. Although the identity-related features are possibly destroyed in the augmented images, those discriminative clues still remain in the original person images, \ie, raw images before augmentation. Thus, we propose to use the similarity between the original images as ground truth to supervise the images that go through the data augmentation, \ie, maintain the consistency before and after data augmentations.

For a mini-batch of input person images $x_{r}$, we get two groups of images $x_q$ and $x_k$ after two different data augmentations. Then we feed them into the network and we get the online encoder features $q$ and EMA encoder features $k$ respectively, that is, $q=f_q(x_q)$ and $k=f_k(x_k)$. The similarity distribution is computed as:
\begin{equation}
\label{eqn: APD}
    A(q,k) = \bm{q}\cdot\bm{k^T},
\end{equation}
where $q$ and $k$ have been normalized by the normalization layer followed by the projection head. $A(\cdot)$ denotes the inter-instance similarity calculation function between two batches of images after two different kinds of augmentation.


Similarly, we perform the same technique to the original input images $x_{r}$, which is expressed as $q_{r}=f_q(x_{r})$ and $k_{r}=f_k(x_{r})$. Then, we calculate the inter-instance similarity  distribution over the original images:
\begin{equation}
\label{eqn: OPD}
    A(q_r, k_r) = \bm{{q}_{r}}\cdot\bm{{k}^T_{r}}.
\end{equation}

After that, we employ a Maximum Mean Discrepancy (MMD)~\cite{gretton2012kernel} metric to measure the difference between two distributions and construct a consistency loss based on it:
\begin{equation}
\label{eqn:MMD}
    \mathcal{L}_{consist.} = MMD\left(A(q,k), \hspace{1mm}A(q_r, k_r)\right).
\end{equation}
Note that the calculated similarity distribution over the original images $A(q_r, k_r)$ just serves as the ground truth to supervise those with augmentations and does not participate in the update. So, there is no gradient back-propagation for the features of the original images. The consistency loss $\mathcal{L}_{consist.}$ helps the model to deduce and restore the discriminative local regions that are distorted by data augmentations, and further encourages the model to learn discriminative feature representations between different instances.


\subsection{Intrinsic Contrastive Constraint}
\label{section: intrinsic contrastive contraint}


% 1. (explore)   intrinsic properties   (of person images through contrastive learning.)
% 2. 每个patch-level instances代表行人的一个部分。我们给每个都求了一个loss,可以fully explore

To explore the intrinsic properties of a person image, we also introduce an intrinsic contrastive constraint in our UP-ReID framework. Before feeding the augmented images of $x_{q,0}$ and $x_{k,0}$ into the network (here we use subscript `0' to denote the holistic person image), we partition each of them into $M$ non-overlapping patches uniformly,
\begin{equation}
    \mathcal\{{x}_{q,1}, ..., {x}_{q,M}\} = P(x_{q,0}),
    \label{AP for query}
\end{equation}
\vspace{-5mm}
\begin{equation}
    \mathcal\{{x}_{k,1}, ..., {x}_{k,M}\} = P(x_{k,0}),
    \label{AP for key}
\end{equation}
where $P$ represents the partition operation, $x_{q,i}$ denotes the $i$-th patch partitioned from $x_{q,0}$, and $x_{k,i}$ denotes the $i$-th patch partitioned from $x_{k,0}$. Then, we group them together and get two sets:
$\mathcal{X}_{q}=\{{x}_{q,i}\}^M_{i=0}$ and $\mathcal{X}_{k}=\{{x}_{k,i}\}^M_{i=0}$.

Taking the image set $\mathcal{X}_{q}$ as an example for illustration, $\mathcal{X}_{q}$ comprises an image-level holistic instance $x_{q,0}$ and $M$ patch-level local instances $x_{q,i}$ ($i \in \{1,...,M\}$). All of them come from the input image $x$ and belong to the same instance, \ie, the input $x$. In short, $x_{q,0}$ contains the image-level global information while $x_{q,i}$ ($i \in \{1,...,M\}$) highlights the local information. 

% By partitioning a person images into patches and group them together, we manually construct an identity with multiple instances, which turns the instance-wise contrastive learning to identity-wise contrastive learning. Different from those downstream cluster-based contrastive learning methods~\cite{isobe2021towards,dai2021cluster}, we do not use any cluster algorithm and thus there is no chance to introduce noisy samples to our groups.

As shown in Figure~\ref{fig:Framework}, we feed $\mathcal{X}_{q}$ and $\mathcal{X}_{k}$ into the online encoder $f_q$ and EMA encoder $f_k$, respectively, \ieno, $q_i=f_q(x_{q,i})$ and $k^+_i=f_k(x_{k,i})$, $i \in {0,1,...,M}$. To learn semantic-aware representations from the holistic images, we enforce a InfoNCE~\cite{oord2018representation} loss over the global features, which is formulated as:
\begin{equation}
\label{global contrastive loss}
    \mathcal{L}_{g} = - \text{log}\frac{\text{exp}(\bm{q_0}\cdot\bm{k^+_0} / \tau_1)}{\text{exp}(\bm{q_0}\cdot\bm{k^+_0} / \tau_1) + \sum_{j=0}^{N-1} \text{exp}(\bm{q_0}\cdot\bm{k^-_{0,j}/\tau_1})},
\end{equation}
where $\tau_1$ is the temperature hyper-parameter, $k^-_{0,j}$ is the negative sample in the image-level feature queue $Q_0$, and $N$ is the total number of negative samples in $Q_0$.

For the local fine-grained representation learning, we calculate a patch-wise contrastive loss over the patch-level instances. For the feature $q_i$, we denote its positive sample as $k^+_p$ and negative queue as $Q_n$. Formally, the patch-wise contrastive loss for the $i$-th patch $p_i$ is defined as:
\begin{equation}
\label{eqn:contrastive loss for patch i}
    \mathcal{L}_{p_i} = - \text{log}\frac{\text{exp}(\bm{q_i}\cdot\bm{k^+_p} / \tau_2)}{\text{exp}(\bm{q_i}\cdot\bm{k^+_p} / \tau_2) + \sum_{j=0}^{N-1} \text{exp}(\bm{q_i}\cdot\bm{k^-_{n,j}/\tau_2})},
\end{equation}
where $k^-_{n,j}$ is the negative sample in $Q_n$, and $\tau_2$ is the temperature hyper-parameter. The details about the selection of $k^+_p$ and $Q_n$ will be described in Section~\ref{section: hard mining}.

In order to fully explore the discriminative information contained in each body part of a pedestrian, we compute the aforementioned contrastive loss for each patch-level feature and take the weighted average sum of them as the final constraint. That is, the intrinsic contrastive loss is a weighted sum of $\mathcal{L}_{g}$ and multiple $\mathcal{L}_{p_i}$:
\begin{equation}
\label{identity-wise loss}
    \mathcal{L}_{inc} = \lambda_g * \mathcal{L}_{g} + \lambda_p * \frac{1}{M} \sum_{i=1}^{M} \mathcal{L}_{p_i},
\end{equation}
where $\lambda_g$ and $\lambda_p$ are the weighting parameters.




\subsection{Hard Mining for Local Feature Exploration}
\label{section: hard mining}

% ------figure patch selection----
\begin{figure}[t]
    \centering
    \includegraphics[width=8.2cm, height=7.2cm]{Patches.pdf}
    \setlength{\abovecaptionskip}{7pt}
    \vspace{-3mm}
    \caption{Illustration of our hard mining strategy. We choose two horizontally symmetric patches partitioned from the same instance as a positive pair, and two patches partitioned from different instances but at the same patch location as a negative pair.}
    \label{fig: hard mining}
    \vspace{-4mm}
\end{figure}
%-------------------------------

% % For better representation learning, we introduce a hard mining strategy to local information exploration in the intrinsic contrastive constraint.
% In general contrastive learning, 

For the patch-level feature $q_i$ in Eq.~\ref{eqn:contrastive loss for patch i}, $k^+_i$ and $Q_i$ should be the positive sample and negative queue, respectively, \ie, $k^+_p=k^+_i,Q_n=Q_i$, corresponding to same patch region as $q_i$. For better representation learning, based on the prior knowledge that human body is horizontally symmetric, we further develop an effective hard mining method to select the positive sample and negative queue for each patch-level feature, which is shown in Figure~\ref{fig: hard mining}. 
%  and can not optimize the entire framework consistently

\noindent\textbf{Hard Negative Queue Selection.}
The same body part of different persons could be discriminative, such as hair color and shoes color. Hence, for $q_i$ in Eq.~\ref{eqn:contrastive loss for patch i}, we choose the patches partitioned from different instances but at the same location as the negative samples (\ieno, $Q_n=Q_i$).

\noindent\textbf{Hard Positive Sample Selection.}
% For $q_i$, we choose a patch partitioned from the same person but at different location as the positive sample (\ieno, $k_p^+, p\neq i$). Despite this, we still need to ensure these two patches have similar color distribution to avoid an unstable training process.
Considering the prior knowledge that persons are horizontally symmetric, we choose two horizontally symmetric patches partitioned from the same instance as a positive pair. 
% \tcr{Specifically, for $q_i$ in Eq.~\ref{eqn:contrastive loss for patch i}, $k^+_p=k^+_{i\_hs}$, where $k^+_{i\_hs}$ is the EMA feature of the patch positioned horizontally symmetric to $i$.}
Specifically, in Eq.~\ref{eqn:contrastive loss for patch i}, we select the feature $k^+_{i\_hs}$ (\ieno, the horizontally symmetrical patch feature corresponding to position $i$) as the positive sample of $q_i$.

Intuitively, the human body structure and clothing are mostly horizontally symmetric, which indicates that two symmetric patches of the same person image contain very similar visual representative patterns~(\eg, color, texture). This is important for person ReID. Thus, choosing them as a positive pair to pre-train the model is reasonable.
% This is important for person ReID. Thus, choosing them as a positive pair to pre-train the model is reasonable.
% which is important because person ReID is very dependent on the visual representative patterns~(\eg, color, texture, etc). Thus choosing them as a positive pair to pre-train the model is reasonable.
On the other hand, due to the different capture environments caused by camera angles or human postures, a pedestrian image may not be completely symmetrical, which means two symmetric patches have similar primary visual information but there are still differences in details. Thus, choosing them as a positive pair can improve the model's ability to identify similar visual representation patterns under different situations, which further helps the model recognize the same pedestrian under different environments.

% which is important because person ReID is very dependent on the visual representative patterns~(\eg, color, texture, etc). Thus choosing them as a positive pair to pre-train the model is reasonable.
% But, for this case, the color information of these two patches are similar with a high probability, which is important because person ReID is very dependent on the color information. 
% The ReID models are prone to identify two pedestrian pictures with similar color distribution as the same person.
% We summarize the reasons for choosing two horizontally symmetric patches as a positive pair into two points. 1) two horizontally symmetric patches of the same person image more likely contain similar color information, and choosing them as a positive pair to pre-train the model is reasonable. 2) Influenced by various factors, the main color distribution of two symmetric patches is similar but there are differences in details. 
% there is still difference in the distribution of information between two horizontally symmetric patches. Choosing these two patches as a hard positive pair will improve the generalization ability of the pre-trained model. (\tcr{i cannot understand the second point, plz rewriting it.}) 
% a normal pedestrian image cannot be completely horizontally symmetric, which means that there is still difference in the distribution of information between two horizontally symmetric patches.
% Figure 3 shows our strategy for selecting patch-level positive and negative samples.

Given that there are still some extreme cases that are totally inconsistent with the prior knowledge of horizontal symmetry of the pedestrian pictures~(\eg, pedestrian pictures taken from the side), we also select the same position patch of the other view person image (\ie, $k^+_i$) as one of the positive samples of $q_i$. So, the patch-wise contrastive loss in Eq.~\ref{eqn:contrastive loss for patch i} is modified to:

\begin{equation}
\footnotesize
\label{eqn:contrastive loss for patch i with hard mining}
    \mathcal{L}_{p_i} = - \text{log}\frac{\sum_{k^+_p\in\mathcal{P}(i)}\text{exp}(\bm{q_i}\cdot\bm{k^+_p} / \tau_2)}{\sum_{k^+_p\in\mathcal{P}(i)}\text{exp}(\bm{q_i}\cdot\bm{k^+_p} / \tau_2) + \sum_{j=0}^{N-1} \text{exp}(\bm{q_i}\cdot\bm{k^-_{i,j}/\tau_2})},
\end{equation}
where $\mathcal{P}(i)=\{k^+_{i\_hs}, k^+_i\}$, $k^-_{i,j}\in Q_i$. 

%Given that there are still some extreme cases that are totally inconsistent with the prior knowledge of horizontal symmetry of the pedestrian pictures~(\eg, pedestrian pictures taken from the side), we also select the same position patch of the other view person image (\ie, $k^+_i$) as one of the positive samples of $q_i$. So, the patch-wise contrastive loss in Eq.~\ref{eqn:contrastive loss for patch i} is modified to:
% $k^+_{i\_hs}$ is the EMA feature of the patch positioned horizontally symmetric to $i$.
% \noindent\textbf{Hard-Mining Strategy.}
% Figure~\ref{fig: hard mining} shows the strategy for selecting patch-level positive and negative samples. 
% Different from the previous works~\cite{hermans2017defense}, we select positive and negative pairs based on the prior knowledge that persons are horizontally symmetric. We prove the effectiveness of our hard-mining strategy and compare it with the other online hard-minings in Section\tcr{...}.


% 1.防止extreme cases
% 2.增强网络泛化能力
% 3.提升patch-level contrastive loss对于augmentation中random horizontal flipping带来干扰的鲁棒性





% We choose two horizontally symmetric patches partitioned from the same person as a positive pair, and two patches partitioned from different persons but at the same patch location as a negative pair. 
% Intuitively, such hard-mining strategy improves the generalization ability of the pre-trained model by exploiting fine-grained features in partitioned patches.


% ----------------------------------------------------------------------------------------------------------------------------------------------
\section{Experiments}

\subsection{Implementation}

\noindent\textbf{Training details.} 
For fair comparison, we use ResNet50 as the pre-trained backbone model and SGD as the optimizer.
The input images are resized to 256$\times$128. The mini-batch size is set to 800, and the initial learning rate is $0.1$. In our experiments, $M$ is set to $8$, $N$ is set to 65536, $m$ is set to 0.9, $\tau_1$ and $\tau_2$ are both set to 0.1, $\lambda_g$ and $\lambda_p$ are set to $0.8$ and $0.2$.  The pre-training models are trained with 8$\times$2080Ti GPUs for 3 weeks under Pytorch framework.

\noindent\textbf{Augmentation and Patch Partition.} 
Data augmentation plays a crucial role in self-supervised contrastive learning. We utilize the same augmentation operations as ~\cite{fu2021unsupervised}.
As for partition, we adopt image-level partition strategy. Specifically, we first partition a holistic image into multiple horizontal stripes, and then divide each stripe vertically into two patches uniformly.
It is necessary to emphasize that we apply the global-level augmentation (\ieno, augmentation followed by partition) rather than patch-level augmentation (\ieno, partition followed by augmentation). Because the global-level augmentation is closer to the realistic data variation and will not break the inherent consistency among patches partitioned from the same person image. 

\noindent\textbf{Datasets.} 
We pre-train our model on ``LUPerson"~\cite{fu2021unsupervised} dataset. To demonstrate the superiority of our pre-trained model, we conduct extensive downstream experiments on four public ReID datasets, including CUHK03~\cite{li2014deepreid}, Market1501~\cite{zheng2015scalable}, PersonX~\cite{sun2019dissecting}, and MSMT17~\cite{wei2018person}. Note that we do not use DukeMTMC~\cite{zheng2017unlabeled} to avoid ethical issues.

\noindent\textbf{Evaluation Protocols.} 
Following the standard evaluation metrics, we use the cumulative matching characteristics at Rank1 and mean average precision (mAP) to evaluate the performance.


\begin{table*}[t]
\caption{Comparison of three representative supervised ReID methods using different pre-trained models in terms of mAP/Rank1 (\%). ``INSUP" refers to the supervised pre-trained model on ImageNet, ``Moco" and ``UP-ReID" refer to the Moco and our UP-ReID pre-trained models on LUPerson, respectively. More comparison results can be found in \textbf{Appendix}.}
\setlength{\tabcolsep}{3.3mm}
    \begin{subtable}[h]{0.5\textwidth}
        \centering
        \begin{tabular}{l|ccc}
        \shline
        Model & BDB~\cite{dai2019batch} & BOT~\cite{luo2019bag} & MGN~\cite{wang2018learning} \\
        \hline
        INSUP    & 76.7/79.4 & 62.0/63.9 & 70.5/71.2 \\ \hline
        Moco  & 78.9/81.5 & 66.7/66.3 & 74.7/75.4 \\ \hline
        UP-ReID & \textbf{79.6/82.6} & \textbf{68.7/69.1} & \textbf{85.3/87.6} \\ \shline
        \end{tabular}
        \caption{CUHK03}
        \label{tab:improve-cuhk}
    \end{subtable}
    \hfill
    \begin{subtable}[h]{0.5\textwidth}
    \centering
        \begin{tabular}{l|ccc}
        \shline
        Model  & BDB~\cite{dai2019batch} & BOT~\cite{luo2019bag} & MGN~\cite{wang2018learning} \\
        \hline
        INSUP    & 86.7/95.3 & 85.7/94.3 & 87.5/95.1 \\ \hline
        Moco  & 88.1/95.3 & 87.6/94.9 & 91.0/96.4 \\ \hline
        UP-ReID & \textbf{88.5/95.3} & \textbf{88.1/95.1} & \textbf{91.1/97.1} \\ \shline
        \end{tabular}
        \caption{Market1501}
        \label{tab:improve-market}
    \end{subtable}
    \hfill
    \begin{subtable}[h]{0.5\textwidth}
    \centering
        \begin{tabular}{l|ccc}
        \shline
        Model & BDB~\cite{dai2019batch} & BOT~\cite{luo2019bag} & MGN~\cite{wang2018learning} \\
        \hline
        INSUP    & 84.4/95.1 & 86.7/94.8 & 85.3/94.3 \\ \hline
        Moco  & 84.8/95.2 & 86.5/94.6 & 85.8/94.2 \\ \hline
        UP-ReID & \textbf{86.1/95.3} & \textbf{88.0/95.3} & \textbf{89.7/96.1} \\ \shline
        \end{tabular}
        \caption{PersonX}
        \label{tab:improve-PersonX}
    \end{subtable}
    \hfill
    \begin{subtable}[h]{0.5\textwidth}
    \centering
        \begin{tabular}{l|ccc}
        \shline
        Model & BDB~\cite{dai2019batch} & BOT~\cite{luo2019bag} & MGN~\cite{wang2018learning} \\
        \hline
        INSUP    & 49.2/77.4 & 53.4/76.8 & 61.5/84.0 \\ \hline
        Moco  & 51.2/78.1 & 53.2/75.4 & 62.9/83.9 \\ \hline
        UP-ReID & \textbf{52.4/78.7} & \textbf{56.2/78.1} & \textbf{63.3/84.3} \\ \shline
        \end{tabular}
        \caption{MSMT17}
        \label{tab:improve-msmt}
    \end{subtable}
    \vspace{-2mm}
    \label{table1: supervised reid comparison}
\end{table*}

%------------------------------------------------------------------------------------------------------------------------------------------------

\begin{figure*}[h!]
\centering
\begin{subfigure}{0.33\linewidth}
    \includegraphics[width=1.0\linewidth]{convergence_cuhk03.pdf}
    \caption{mAP learning curve on CUHK03}
    \label{subfig: convergence on cuhk03}
\end{subfigure}
\begin{subfigure}{0.33\linewidth}
    \includegraphics[width=1.0\linewidth]{convergence_market1501.pdf}
    \caption{mAP learning curve on Market1501}
    \label{subfig: convergence on market1501}
\end{subfigure}
\begin{subfigure}{0.33\linewidth}
    \includegraphics[width=1.0\linewidth]{convergence_personx.pdf}
    \caption{mAP learning curve on PersonX}
    \label{subfig: convergence on personx}
\end{subfigure}
\vspace{-2mm}
\caption{mAP learning curves with different pre-trained models in BDB on three datasets (CUHK03, Market1501, and PersonX) with the same training schedule. More comparison results can be found in \textbf{Appendix}.}
\label{fig: convergence rapidity}
\vspace{-2mm}
\end{figure*}


% \begin{figure*}[h!]
% \centering
% \centering
% \includegraphics[with=14.0cm, height=4.55cm]{convergence.pdf}
% \vspace{-4mm}
% \caption{mAP learning curves with different pre-trained models in BDB on three datasets (CUHK03, Market1501, and PersonX) with the same training schedule. \More comparison results can be found in \tcr{\textbf{Appendix}}.}
% % \caption{Convergence rapidity comparisons of using different pre-trained models in MGN on three datasets: CUHK03, Market1501 and DukeMTMC-reID. The fisrt row shows the $mAP$ and the second row shows the $cmc1$ increasing rate.}
% \label{fig: convergence rapidity}
% \vspace{-2mm}
% \end{figure*}
%------------------------------------------------------------

\subsection{Improvement on Supervised ReID}
In this section, we show the superiority of our UP-ReID by comparing with the model unsupervisedly pre-trained on LUPerson by Moco~\cite{fu2021unsupervised} and the commonly used supervised pre-trained model on ImageNet in three representative supervised ReID approaches: Batch DropBlock Network (BDB)~\cite{dai2019batch}, Strong Baseline (BOT)~\cite{luo2019bag} and Multiple Granularity Network (MGN)~\cite{wang2018learning}.
The BDB is re-implemented based on the open source code. As for BOT and MGN, we implement them in fast-reid~\cite{he2020fastreid}. 
% The last two methods are implemented in fast-reid~\cite{he2020fastreid}.

Table~\ref{table1: supervised reid comparison} shows the improvements in the three selected supervised ReID methods on four popular person ReID datasets.
It can be seen that, compared to initializing with Moco, the MGN with UP-ReID has achieved \textbf{12.2\%}, \textbf{0.7\%}, \textbf{1.9\%}, \textbf{0.4\%} improvements in terms of Rank1 on CUHK03, Market1501, PersonX, MSMT17, respectively; BOT also has achieved \textbf{2.8\%}, \textbf{0.2\%}, \textbf{0.7\%}, \textbf{2.7\%} improvements in terms of Rank1 on these four datasets. 

Figure~\ref{fig: convergence rapidity} describes the comparison of the convergence speed of applying different pre-trained models in method BDB at the early stage of fine-tuning. UP-ReID outperforms both Moco and INSUP with faster convergence on all three datasets. The performance enhancement is more noticeable on PersonX (see Figure~\ref{subfig: convergence on personx}). On the Market1501 where the advantage is not obvious, UP-ReID still holds the lead of Moco by 1.7\% mAP improvement (see Figure~\ref{subfig: convergence on market1501}).
% In most cases, the Moco pre-trained model on LUPerosn could outperform the model pre-trained on ImageNet, but there still exist some cases that they achieve similar results. Our UP-ReID model is much better than supervised ImageNet model in all cases. It demonstrates the effectiveness of our proposed UP-ReID method for person ReID.



\subsection{Improvement on Unsupervised ReID}

Our pre-trained model can also benefit unsupervised ReID methods. To demonstrate this, we test our pre-trained model on SpCL~\cite{ge2020self}. We evaluate the performance on Market1501 and PersonX.

In Table~\ref{tab:improvement-unsup}, M means purely unsupervised training on Market1501, and P $\rightarrow$ M means unsupervised domain adaptation whose source dataset is PersonX and target dataset is Market1501. As we can see, UP-ReID outperforms Moco by \textbf{2.9\%}, \textbf{6.3\%} in terms of mAP and \textbf{2.2\%}, \textbf{2.5\%} in terms of Rank1 on M and P $\rightarrow$ M, respectively. It further verifies that UP-ReID can achieve better superiority and generalization capability for person ReID.
Note that we implement SpCL by official OpenUnReid~\cite{ge2020self}.
% Note that, we use the official OpenUnReid codebase and the performance obtained is slightly lower than the original paper.

% -----------------------------------------------
\vspace{-3mm}
\begin{table}[h]
\caption{Performance (\%) comparisons of using different pre-trained models on unsupervised ReID method SpCL.}
\vspace{-3mm}
    \centering
    % \begin{tabular}{l|cc|cc}
    \begin{tabular}{P{1.5cm}|C{0.8cm}C{0.8cm}|C{0.8cm}C{0.8cm}}
    \shline
    \centering
    \multirow{2}{*}{Model} & \multicolumn{2}{c|}{M} & \multicolumn{2}{c}{P $\rightarrow$ M} \\
    \cline{2-5} & mAP & Rank1 & mAP & Rank1 \\
    \hline
    \centering
    INSUP    & 73.1 & 88.1 & 73.8 & 88.0 \\
    \centering
    Moco  & 72.2 & 87.8 & 72.4 & 88.4 \\
    \centering
    UP-ReID & \textbf{75.1} & \textbf{90.0} & \textbf{78.7} & \textbf{90.9} \\
    \shline
\end{tabular}
\vspace{-1mm}
\centering
\label{tab:improvement-unsup}
\end{table}
\vspace{-5mm}
% -----------------------------------------------



\subsection{Comparison with State-of-the-Art Methods}

In this section, we compare our results with state-of-the-art methods on CUHK03 and Market1501 datasets. Notice that we do not use any additional modules (\egno, IBN-Net) or post-processing methods (\egno, Re-Rank~\cite{zhong2017re}).We just simply apply UP-ReID pre-trained vanilla ResNet50 on MGN. As shown in Table~\ref{tab:sota}, MGN equipped with UP-ReID ResNet50 outperforms all compared methods on both datasets.


% In this section, we compare our results with state-of-the-art methods on CUHK03 and Market1501 datasets. Notice that we do not use any additional modules like IBN-Net or post-processing methods like Re-Rank~\cite{zhong2017re}. We just simply apply UP-ReID pre-trained vanilla ResNet50 on MGN. As shown in Table~\ref{tab:sota}, MGN equipped with UP-ReID ResNet50 outperforms all compared methods on both datasets.


% CUHK03 and Market1501. Regrettably, it is still slightly lagging behind TransReID on MSMT17 bacause TransReID adopts transformer architecture and utilizes camera information.

% \begin{table*}[t!]
% % \scriptsize
% % \footnotesize
% % \small
% \centering
% % \begin{tabular}{c|cc|cc|cc}
% \begin{tabular}{P{4cm}|C{1.4cm}C{1.4cm}|C{1.4cm}C{1.4cm}|C{1.4cm}C{1.4cm}}
%     \shline
%     \multirow{2}{*}{Methods} & \multicolumn{2}{c|}{CUHK03} & \multicolumn{2}{c|}{Market1501} & \multicolumn{2}{c}{MSMT17}\\
%     \cline{2-7}
%     \multicolumn{1}{c|}{} & mAP & cmc1 & mAP & cmc1 & mAP & cmc1 \\ 
%     \hline 
%     % PCB~\cite{sun2018beyond} (ECCV'18) & 57.5 & 63.7 & 81.6 & 93.8 & - & - \\
%     MGN~\cite{wang2018learning} (ACM MM'18) & 70.5 & 71.2 & 86.9 & 95.7 & - & - \\
%     ABDNet~\cite{chen2019abd} (ICCV'19) & - & - & 88.3 & 95.6 & \underline{60.8} & \underline{82.3} \\
%     % BDB~\cite{dai2019batch} (ICCV'19) & 76.7 & 79.4 & 86.7 & 95.3 & - & - \\
%     OSNet~\cite{zhou2019omni} (ICCV'19) & 67.8 & 72.3 & 84.9 & 94.8 & 52.9 & 78.7 \\
%     % P2Net~\cite{guo2019beyond} (ICCV'19) & 73.6 & 78.3 & 85.6 & 95.2 & - & - \\
%     % SCAL~\cite{chen2019self} (ICCV'19) & 72.3 & 74.8 & 89.3 & 95.8 & - & - \\
%     DSA~\cite{zhang2019densely} (CVPR'19) & 75.2 & 78.9 & 87.6 & 95.7 & - & - \\
%     DGNet~\cite{zheng2019joint} (CVPR'19) & - & - & 86.0 & 94.8 & 52.3 & 77.2 \\
%     GCP~\cite{park2020relation} (AAAI'20) & 75.6 & 77.9 & \underline{88.9} & 95.2 & - & - \\
%     SAN~\cite{jin2020semantics} (AAAI'20) & 76.4 & 80.1 & 88.0 & \underline{96.1} & 55.7 & 79.2  \\
%     ISP~\cite{zhu2020identity} (ECCV'20) & 74.1 & 76.5 & 88.6 & 95.3 & - & - \\
%     GASM~\cite{he2020guided} (ECCV'20) & - & - & 84.7 & 95.3 & 52.5 & 79.5 \\
%     RGA-SC~\cite{zhang2020relation} (CVPR'20) & \underline{77.4} & \underline{81.1} & 88.4 & \underline{96.1} & - & - \\
%     % HOReID~\cite{wang2020high} (CVPR'20) & - & - & 84.9 & 94.2 & - & - \\
%     AMD~\cite{chen2021explainable} (ICCV'21) & - & - & 87.1 & 94.8 & - & - \\
%     PGFL-KD~\cite{zheng2021pose} (ICCV'21) & - & - & 87.2 & 95.3 & - & - \\
%     % TransReID~\cite{he2021transreid} (ICCV'21) & - & - & \underline{89.5} & 95.2 & \textbf{67.4} & \textbf{85.3} \\
%     PAT~\cite{li2021diverse} (CVPR'21) & - & - & 88.0 & 95.4 & - & - \\
%     \hline
%     MGN+R50 (UP-ReID) & \textbf{85.3} & \textbf{87.6} & \textbf{91.1} & \textbf{97.1} & \textbf{63.3} & \textbf{84.3} \\
%     \shline
% \end{tabular}\\
% \caption{Performance (\%) comparisons with state-of-the-art approaches on three datasets. The best results are marked as bold and the second are masked by underline.}
% \label{tab:sota}
% \end{table*}

% \vspace{-2mm}
\begin{table}[h!]
% \scriptsize
% \footnotesize
\small
\centering
\caption{Performance (\%) comparisons with state-of-the-art approaches on CUHK03 and Market1501. The best results are marked as bold and the second ones are masked by underline. We show more comparison results in \textbf{Appendix}.}
\vspace{-2mm}
% \begin{tabular}{c|cc|cc}
\begin{tabular}{P{3.50cm}|C{0.67cm}C{0.67cm}|C{0.67cm}C{0.67cm}}
    \shline
    \multirow{2}{*}{Methods} & \multicolumn{2}{c|}{CUHK03} & \multicolumn{2}{c}{Market1501} \\
    \cline{2-5} & mAP & Rank1 & mAP & Rank1 \\ 
    \hline
    % \centering
    PCB~\cite{sun2018beyond} (ECCV'18) & 57.5 & 63.7 & 81.6 & 93.8 \\
    % MGN~\cite{wang2018learning} (ACMMM'18)& 70.5 & 71.2 & 86.9 & 95.7 \\
    % ABDNet~\cite{chen2019abd} (ICCV'19) & - & - & 88.3 & 95.6 \\
    % BDB~\cite{dai2019batch} & 76.7 & 79.4 & 86.7 & 95.3 \\
    OSNet~\cite{zhou2019omni} (ICCV'19) & 67.8 & 72.3 & 84.9 & 94.8 \\
    P2Net~\cite{guo2019beyond} (ICCV'19) & 73.6 & 78.3 & 85.6 & 95.2 \\
    SCAL~\cite{chen2019self} (ICCV'19) & 72.3 & 74.8 & 89.3 & 95.8 \\
    DSA~\cite{zhang2019densely} (CVPR'19) & 75.2 & 78.9 & 87.6 & 95.7 \\
    % DGNet~\cite{zheng2019joint} & - & - & 86.0 & 94.8 \\
    GCP~\cite{park2020relation} (AAAI'20) & 75.6 & 77.9 & 88.9 & 95.2 \\
    SAN~\cite{jin2020semantics} (AAAI'20) & 76.4 & 80.1 & 88.0 & \underline{96.1} \\
    ISP~\cite{zhu2020identity} (ECCV'20) & 74.1 & 76.5 & 88.6 & 95.3 \\
    GASM~\cite{he2020guided} (ECCV'20) & - & - & 84.7 & 95.3 \\
    RGA-SC~\cite{zhang2020relation} (CVPR'20) & \underline{77.4} & \underline{81.1} & 88.4 & \underline{96.1} \\
    HOReID~\cite{wang2020high} (CVPR'20) & - & - & 84.9 & 94.2 \\
    AMD~\cite{chen2021explainable} (ICCV'21) & - & - & 87.1 & 94.8 \\
    % PGFL~\cite{zheng2021pose} (ACMMM'21) & - & - & 87.2 & 95.3 \\
    TransReID~\cite{he2021transreid} (ICCV'21) & - & - & \underline{89.5} & 95.2  \\
    PAT~\cite{li2021diverse} (CVPR'21) & - & - & 88.0 & 95.4 \\
    \hline
    MGN+UP-ReID (Ours) & \textbf{85.3} & \textbf{87.6} & \textbf{91.1} & \textbf{97.1} \\
    \shline
\end{tabular}\\
\label{tab:sota}
\end{table}

% the pre-trained models (b) with consistency constraint and the (c) with intrinsic contrastive constraint
\subsection{Ablation Study}
In this section, we perform comprehensive ablation studies to demonstrate the effectiveness of our designs in the proposed UP-ReID. Here we fine-tune different pre-trained models with supervised ReID method MGN~\cite{wang2018learning} on CUHK03 to validate the effectiveness of each component. 

\noindent\textbf{Effectiveness of the Consistency Constraint and the Intrinsic Contrastive Constraint.}
Our UP-ReID consists of two key constraints: the consistency constraint (CC) and the intrinsic contrastive constraint (ICC). We evaluate the benefits of them in Table~\ref{tab: effectiveness of CC and ICC.}. Specifically, (b) Baseline with CC and (c) Baseline with ICC outperform the (a) Baseline by \textbf{4.4\%/4.8\%} and \textbf{6.7\%/8.2\%} in terms of mAP/Rank1 on CUHK03, respectively. With both two constraints, (d) UP-ReID achieves \textbf{85.3\%(+10.6\%)} mAP and \textbf{87.6\%(+12.2\%)} Rank1 on CUHK03, which demonstrates that CC and ICC are complementary and both vital to UP-ReID, jointly resulting in a superior performance.

We also evaluate the effectiveness of each components of our UP-ReID in terms of the convergence speed on CUHK03.
Figure~\ref{fig: ablation} plots the mAP learning curves of four different pre-trained models with MGN. As we can see, the (b) Baseline with CC and (c) Baseline with ICC achieve faster convergence than (a) Baseline. More importantly, (d) the UP-ReID with both constraints (\ieno, ICC and CC) achieves faster convergence than both (b) and (c) which only have one constraint.


% Figure~\ref{fig: ablation} plots the mAP learning curves of four different pre-trained models with MGN on CUHK03, where the models in Figure~\ref{fig: ablation} correspond to the models in Table~\ref{tab: effectiveness of CC and ICC.} one-to-one. As we can see, the Baseline with CC (b) and Baseline with ICC (c) achieve faster convergence than (a) Baseline. More importantly, (d) UP-ReID with both constraints achieves faster convergence than both (b) and (c) which only have one constraint.

The experimental results demonstrate that both the consistency constraint and the intrinsic contrastive constraint contribute to a better visual representation. The former is designed to counter the augmentation perturbations, and the latter is designed for detailed information exploration.

\vspace{-2mm}
\begin{table}[h!]
\caption{The ablation results of several variants of UP-ReID pre-trained models that are fine-tuned on CUHK03. The values in the brackets are the improvement compared to the Baseline.}
\vspace{-2mm}
\small
% \footnotesize
    \centering
    \begin{tabular}{c|cc|cc}
        \shline
        Model & CC & ICC & mAP & Rank1 \\
        \hline
        (a) Baseline & $\times$ & $\times$ & 74.7 & 75.4 \\
        (b) Baseline w CC & $\surd$ & $\times$ & 79.1(+4.4) & 80.2(+4.8) \\
        (c) Baseline w ICC & $\times$ & $\surd$ & 81.4(+6.7) & 83.6(+8.2) \\
        (d) UP-ReID & $\surd$ & $\surd$ & \textbf{85.3}(+10.6) & \textbf{87.6}(+12.2) \\
        \shline
    \end{tabular}
    \label{tab: effectiveness of CC and ICC.}
\end{table}
% \vspace{-3mm}

\vspace{-2mm}
\begin{figure}[h!]
    \centering
    \includegraphics[width=8.5cm, height=6.1cm]{Ablation.pdf}
    \setlength{\abovecaptionskip}{7pt}
    \vspace{-5mm}
    \caption{mAP learning curves of CUHK03 in MGN with four different pre-trained UP-ReID models. The models correspond to the models in Table~\ref{tab: effectiveness of CC and ICC.} one-to-one.}
    \label{fig: ablation}
\end{figure}


% The models corresponds to the models in Table~\ref{tab: effectiveness of CC and ICC.} one-to-one.

% \textbf{Effectiveness of Local Information Mining.} As shown in Table 3, ``GL+8PT'' has better performance than ``GL'', which demonstrates that the local information exploration indeed benefit to person ReID. Patch-based contrastive loss encourage the model to learn fine-grained local attributes of pedestrian images. Some id-related information ignored in the global features will be supplemented by the local features, which allows the model to learn a more discriminative feature representation from both holistic image and local patches. Due to the exploration of local information in pre-training stage, our pre-trained model could attend to both global information and local details of pedestrian images when it comes to fine-tune ReID tasks and get better results.


% ``GL+8PT'' has better performance than ``GL'', which demonstrates that the local information exploration indeed benefit to person ReID. Patch-based contrastive loss encourage the model to learn fine-grained local attributes of pedestrian images. Some id-related information ignored in the global features will be supplemented by the local features, which allows the model to learn a more discriminative feature representation from both holistic image and local patches. Due to the exploration of local information in pre-training stage, our pre-trained model could attend to both global information and local details of pedestrian images when it comes to fine-tune ReID tasks and get better results.


% We further investigate the influence of the number of the partitioned patches. As described in Table 3, $M$=8 outperforms $M$=4 by \textbf{5.2\%/0.8\%} in $mAP$ and \textbf{5.5\%/0.6\%} in $cmc1$ on CUHK03/Market1501, which also outperforms $M$=12 by \textbf{5.1\%/0.9\%} in $mAP$ and \textbf{4.7\%/0.7\%} in $cmc1$ on CUHK03/Market1501. Partitioning the global image into eight patches means each patch has a proper size, which is neither too large to ignore discriminative fine-grained attributes, nor too small to introduce too much noise.

% The hard mining of patch-based local features is proposed to improve the generalization capability of the pre-trained model. Its core idea is to encourage the model to exploit the responses between hard positive pairs and hard negative pairs. The former pairs come from the different positions of the same person image, and the latter pairs come from the same positions of the different person images.
% As shown in Table 3, ``GL+8PT+HM'' outperforms ``GL+8PT'' by \textbf{9.9\%/1.4\%} in $mAP$ and \textbf{11.4\%/0.6\%} in $cmc1$ on CUHK03/Market1501. This demonstrates that our proposed hard mining strategy for patch features learning improves the robustness and discriminative ability of the pre-trained model.


\noindent\textbf{Effectiveness of the Hard Mining Strategy.} For better representation learning, we introduce a hard mining (HM) strategy to the intrinsic contrastive constraint. As shown in Table~\ref{tab: effectiveness of Hard Mining.}, the UP-ReID without hard mining strategy (\ie, replace Eq.~\ref{eqn:contrastive loss for patch i with hard mining} with Eq.~\ref{eqn:contrastive loss for patch i}) has a 4.6\%/4.5\% drop in mAP/Rank1. Obviously, our hard mining strategy improves the discrimination and generalization capability of the pre-trained model.

Different from the previous works~\cite{hermans2017defense}, we select positive and negative pairs based on the prior knowledge that persons are horizontally symmetric instead of an online way. We further investigate the influence of different hard mining strategies and show more results in \textbf{Appendix}.

% We have discussed our hard mining selection in section 3.3, especially why we choose two horizontally symmetric patches as a positive pair. To verify the rationality of our strategy, we compare several schemes. \textbf{\emph{Random Positive}}: for patch $i$, we select a random patch partitioned from the same pedestrian but located differently as the positive sample. \textbf{\emph{Online Positive}}: instead of finding a hard positive patch sample for each query patch $i$, we only select the hardest positive pair among all $M\times M$ positive pairs. \textbf{\emph{Horizontally Symmetric Positive}}: our proposed selecting two horizontally symmetric patches as a positive pair. It needs to be clear that all three schemes have the same rule to select negative samples.
% \vspace{-2mm}
\begin{table}[h!]
\caption{Effectiveness of the hard mining strategy for ICC in our UP-ReID on CUHK03.}
% \vspace{-2mm}
    \centering
    \begin{tabular}{c|ccc}
        \shline
        Model & mAP & Rank1 & Rank5 \\
        \hline
        UP-ReID w/o HM & 80.7 & 83.1 & 93.1 \\
        UP-ReID w HM & \textbf{85.3} & \textbf{87.6} & \textbf{95.4} \\
        \shline
    \end{tabular}
    \vspace{-4mm}
    \label{tab: effectiveness of Hard Mining.}
\end{table}



\noindent\textbf{Influence of the Number of the Patch-Level Instances.}
Note that each patch-level instance is partitioned from the corresponding image-level instance. Different patch number ($M$) means different patch size. We investigate the influence of the patch-level instance number in the intrinsic contrastive constraint. 
As described in Table~\ref{tab: different number patches.}, $M$=8 outperforms $M$=4 by \textbf{4.0\%/4.5\%} in mAP/Rank1 on CUHK03, which also surpasses $M$=12 by \textbf{4.6\%/5.4\%} in mAP/Rank1. When $M$=8, each patch-level instance has a proper size, which is neither too large to ignore discriminative attributes, nor too small to introduce unnecessary noise.

\vspace{-1mm}
\begin{table}[h!]
\caption{Results of different number of patches in ICC.}
\vspace{-1mm}
    \centering
    \begin{tabular}{c|ccc}
        \shline
        Model & mAP & Rank1 & Rank5 \\
        \hline
        UP-ReID w $M=4$ & 81.3 & 83.1 & 92.6 \\
        UP-ReID w $M=12$ & 80.7 & 82.2 & 92.4 \\
        UP-ReID w $M=8$ & \textbf{85.3} & \textbf{87.6} & \textbf{95.4} \\
        \shline
    \end{tabular}
    \label{tab: different number patches.}
    \vspace{-5mm}
\end{table}


\section{Conclusion}
In this paper, we present two critical issues in applying contrastive learning to ReID pre-training task. Then, we propose a ReID-specific pre-training framework UP-ReID with an intra-identity regularization, which consists of a global consistency constraint and an intrinsic contrastive constraint. Moreover, we introduce a hard mining strategy to local information exploration for better representation learning. Extensive experiments demonstrate that UP-ReID could improve the downstream works performance with higher precision and much faster convergence. We hope more methods can be motivated such as unsupervised pre-training for ReID-specific transformers and apply UP-ReID to more downstream tasks~(\eg, occluded person ReID).


%%%%%%%%% REFERENCES
{\small
\bibliographystyle{ieee_fullname}
\bibliography{egbib}
}

\newpage
\appendix

\noindent\textbf{\Large Appendix}

\section{Datasets}
\subsection{Pre-training Dataset}
\noindent\textbf{LUPerson}~\cite{fu2021unsupervised} consists of 4,180,243 person images of over 200K identities extracted from 46,260 YouTube videos. YOLO-v5 trained on MS-COCO is utilized to extract each person instance in the sampled frame. It is worth noting that the LUPerson is large enough to support unsupervised person ReID feature learning.

\subsection{Fine-tuning Datasets}
\noindent\textbf{CUHK03}~\cite{li2014deepreid} contains 13,164 images of 1,360 pedestrians. Each identity is observed by 2 cameras. Note that CUHK03 offers both hand-labeled and DPM-detected bounding boxes, and the former is adopted in this paper.

\noindent\textbf{Market1501}~\cite{zheng2015scalable} contains 32,668 person images of 1,501 identities captured by 6 cameras. The training set consists of 12,936 images of 751 identities, the query set consists of 3,368 images, and the gallery set consists of 19,732 images of 750 identities.

\noindent\textbf{PersonX}~\cite{sun2019dissecting} is a large-scale data synthesis engine, which contains 1,266 manually designed identities and editable visual variables. Each identity is captured by 6 cameras.

\noindent\textbf{MSMT17}~\cite{wei2018person} contains of 126,441 images of 4,101 identities captured by 15 cameras. The training set consists of 30,248 person images of 1,041 identities, the query set consists of 11,659 images, and the gallery consists of 82,161 images of 3,060 identities.


\section{More Details about Data Augmentation}
Data augmentation plays a crucial role in self-supervised contrastive learning. We adopt popular augmentation operations including resizing, cropping, random grayscale, Gaussian blurring, horizontal flipping, and RandomErasing. Note that we abandon color jitter since person ReID is extremely dependent on color information~\cite{fu2021unsupervised}.


\section{Additional Results}

\subsection{More Results for Supervised ReID}
In Section 4.2 of the main body, we demonstrate that our UP-ReID can benefit the supervised ReID methods and show the results in Table 1. Here, we present the remaining results. Table~\ref{tab:comparison on PCB} shows the results of using different pre-trained models in the supervised fine-tuning ReID method PCB~\cite{sun2018beyond} on CUHK03, Market1501, and PersonX.


\begin{table}[h!]
	\centering
	\caption{Comparison of PCB method using different pre-trained models on three datasets in terms of mAP/Rank1 (\%).}
	\begin{tabular}{l|ccc}
		\shline
		Model & CUHK03 & Market1501 & PersonX \\
		\hline
		INSUP    & 59.5/69.9 & 78.0/92.6 & 80.9/92.7 \\ \hline
		Moco  & 58.3/72.8 & 79.3/92.9 & 80.7/92.9 \\ \hline
		UP-ReID & \textbf{60.1/74.1} & \textbf{80.0/93.1} & \textbf{81.7/93.2} \\ \shline
	\end{tabular}
	\label{tab:comparison on PCB}
\end{table}

We also show the comparison of the convergence speed of applying different pre-trained models in method MGN~\cite{wang2018learning} at the early stage of fine-tuning in Figure~\ref{fig: convergence rapidity in MGN}. As can be seen, UP-ReID achieves a faster convergence rapidity compared with Moco and INSUP on all the three datasets, which further demonstrates that the proposed UP-ReID can better benefit downstream ReID tasks.


\begin{figure*}[t!]
	\centering
	\begin{subfigure}{0.33\linewidth}
		\includegraphics[width=1.0\linewidth]{convergence_cuhk03_app.pdf}
		\caption{mAP learning curve on CUHK03}
		\label{subfig: convergence on cuhk03}
	\end{subfigure}
	\begin{subfigure}{0.33\linewidth}
		\includegraphics[width=1.0\linewidth]{convergence_market1501_app.pdf}
		\caption{mAP learning curve on Market1501}
		\label{subfig: convergence on market1501}
	\end{subfigure}
	\begin{subfigure}{0.33\linewidth}
		\includegraphics[width=1.0\linewidth]{convergence_personx_app.pdf}
		\caption{mAP learning curve on PersonX}
		\label{subfig: convergence on personx}
	\end{subfigure}
	\vspace{-2mm}
	\caption{mAP learning curves of different pre-trained models in MGN~\cite{wang2018learning} on three datasets (CUHK03, Market1501, and PersonX) with the same training schedule.}
	\label{fig: convergence rapidity in MGN}
\end{figure*}


\subsection{More Comparisons with State-of-the-Arts}
In Section 4.4 of the main body, we have shown some comparison results between our UP-ReID and state-of-the-art methods. Here, we extend the results in Table 3 and show the complete results of the comparison between UP-ReID and state-of-the-art methods in Table~\ref{tab:sota2} on three datasets, including CUHK03, Market1501, and MSMT17.
As we can see, MGN with our UP-ReID outperforms the other methods by at least \textbf{7.9\%/6.5\%} and \textbf{1.6\%/1.0\%} in terms of mAP/Rank1 on CUHK03 and Market1501, respectively. On the MSMT17 dataset, the TransReID~\cite{he2021transreid} achieves better performance. However, TransReID adopts transformer-based network and utilizes camera information additionally.


\begin{table*}[t!]
	% \scriptsize
	% \footnotesize
	% \small
	\centering
	% \begin{tabular}{c|cc|cc|cc}
	\caption{Complete performance (\%) comparisons with state-of-the-art approaches on CUHK03, Market1501, and MSMT17. The best results are marked as bold and the second ones are masked by underline.}
	\begin{tabular}{P{4.5cm}|C{1.4cm}C{1.4cm}|C{1.4cm}C{1.4cm}|C{1.4cm}C{1.4cm}}
		\shline
		\centering
		\multirow{2}{*}{Method} & \multicolumn{2}{c|}{CUHK03} & \multicolumn{2}{c|}{Market1501} & \multicolumn{2}{c}{MSMT17}\\
		\cline{2-7}
		\multicolumn{1}{c|}{} & mAP & cmc1 & mAP & cmc1 & mAP & cmc1 \\ 
		\hline 
		PCB~\cite{sun2018beyond} (ECCV'18) & 57.5 & 63.7 & 81.6 & 93.8 & - & - \\
		MGN~\cite{wang2018learning} (ACM MM'18) & 70.5 & 71.2 & 86.9 & 95.7 & - & - \\
		ABDNet~\cite{chen2019abd} (ICCV'19) & - & - & 88.3 & 95.6 & 60.8 & 82.3 \\
		BDB~\cite{dai2019batch} (ICCV'19) & 76.7 & 79.4 & 86.7 & 95.3 & - & - \\
		OSNet~\cite{zhou2019omni} (ICCV'19) & 67.8 & 72.3 & 84.9 & 94.8 & 52.9 & 78.7 \\
		P2Net~\cite{guo2019beyond} (ICCV'19) & 73.6 & 78.3 & 85.6 & 95.2 & - & - \\
		SCAL~\cite{chen2019self} (ICCV'19) & 72.3 & 74.8 & 89.3 & 95.8 & - & - \\
		DSA~\cite{zhang2019densely} (CVPR'19) & 75.2 & 78.9 & 87.6 & 95.7 & - & - \\
		DGNet~\cite{zheng2019joint} (CVPR'19) & - & - & 86.0 & 94.8 & 52.3 & 77.2 \\
		GCP~\cite{park2020relation} (AAAI'20) & 75.6 & 77.9 & 88.9 & 95.2 & - & - \\
		SAN~\cite{jin2020semantics} (AAAI'20) & 76.4 & 80.1 & 88.0 & \underline{96.1} & 55.7 & 79.2  \\
		ISP~\cite{zhu2020identity} (ECCV'20) & 74.1 & 76.5 & 88.6 & 95.3 & - & - \\
		GASM~\cite{he2020guided} (ECCV'20) & - & - & 84.7 & 95.3 & 52.5 & 79.5 \\
		RGA-SC~\cite{zhang2020relation} (CVPR'20) & \underline{77.4} & \underline{81.1} & 88.4 & \underline{96.1} & - & - \\
		HOReID~\cite{wang2020high} (CVPR'20) & - & - & 84.9 & 94.2 & - & - \\
		AMD~\cite{chen2021explainable} (ICCV'21) & - & - & 87.1 & 94.8 & - & - \\
		PGFL-KD~\cite{zheng2021pose} (ICCV'21) & - & - & 87.2 & 95.3 & - & - \\
		TransReID~\cite{he2021transreid} (ICCV'21) & - & - & \underline{89.5} & 95.2 & \textbf{67.4} & \textbf{85.3} \\
		PAT~\cite{li2021diverse} (CVPR'21) & - & - & 88.0 & 95.4 & - & - \\
		\hline
		MGN+R50 (UP-ReID) & \textbf{85.3} & \textbf{87.6} & \textbf{91.1} & \textbf{97.1} & \underline{63.3} & \underline{84.3} \\
		\shline
	\end{tabular}\\
	\label{tab:sota2}
\end{table*}


\section{Discussion about Hard Mining Strategy}
In Section 3.4 of the main body, we introduce our hard mining strategy in detail and experimentally prove its effectiveness in Section 4.5. Here we further discuss two points and give more insights about this design. The first one is that we choose hard positive samples and hard negative queues in a fixed way, which is an offline scheme instead of an online scheme. Would an online scheme be better? The second one comes from the positive samples selection. In Section 3.3 of the main body, we emphasize that all 2$M$ patch-level instances are partitioned from the input image $x$ actually. So, for each patch feature $q_i\in \mathcal{X}_{q}$, any of patch feature $k^+_p\in \mathcal{X}_{k}$ ($i,p \in \{1,...,M\}$) could be its positive sample. So, why do we have to choose patches at the same horizontal position instead of other patches as the positive samples?

%Note that we choose hard positive sample and hard negative queue in a fixed way, which is an offline scheme instead of an online scheme. In Section 3.3 of the main body, we emphasize that all 2$M$ patch-level instances are partitioned from the input image $x$ actually. So in theory, for each patch feature $q_i\in \mathcal{X}_{q}$, any of patch feature $k^+_p\in \mathcal{X}_{k}$ ($i,p \in \{1,...,M\}$) could be its positive sample. 

To answer the aforementioned questions and verify the reasonableness of our selection strategy, we compare it with several other schemes. \textbf{\emph{Random Positive Selection}}: for patch $i$, we randomly select a patch partitioned from the same pedestrian but located differently as the positive sample. \textbf{\emph{Online Positive Selection}}: instead of finding a hard positive patch sample for each query patch $i$, we only select the hardest positive pair among all the $M\times M$ positive pairs. \textbf{\emph{Horizontally Symmetric Positive Selection}}: the proposed selection strategy wherein two horizontally symmetric patches are selected as a positive pair. Note that all three schemes have the same rule to select negative samples. We show the curves of the patch-wise contrastive loss in the intrinsic contrastive constraint under these three selection strategies in Figure~\ref{fig: patch-wise contrastive loss}.
As we can see, the loss value in the scheme of ``Random-P'' is unstable and cannot reach a convergence. On the other hand, the loss value in the scheme of ``Online-P'' converges extremely slowly. 

We analyze that the scheme of ``Random Positive Selection'' and ``Online Positive Selection'' suffer from misalignment and can not guarantee that the selected positive pairs have similar visual information. Take the ``Random Positive Selection'' as an example, for $q_i\in \mathcal{X}_{q}$, we randomly select $k^+_p\in \mathcal{X}_{k}$ as the corresponding positive sample. However, without any constraint, the visual information contained in $q_i$ and $k^+_p$ may be very different~(\eg, $q_i$ represents the head of a person, while $k^+_p$ represents the shoes), which has a negative impact on the pre-training process.

Our hard mining strategy~(\ie, Horizontally Positive Selection) is based on the prior knowledge that persons are horizontally symmetric, which assures that the positive pairs are semantically matched. This avoids the negative impact caused by misalignment on the pre-training process.

\begin{figure}[h!]
	\centering
	\includegraphics[width=8.2cm, height=5.2cm]{Hard_Mining.pdf}
	\setlength{\abovecaptionskip}{7pt}
	\vspace{-3mm}
	\caption{The curves of the patch-wise contrastive loss in different selection strategies. ``Horizontally-P'', ``Random-P'', and ``Online-P'' mean Horizontally Symmetric Positive Selection, Random Positive Selection, and Online Positive Selection, respectively.}
	\label{fig: patch-wise contrastive loss}
\end{figure}


\section{Feature Visualization}
As discussed in the main body, model pre-trained by our UP-ReID has better discriminative feature learning ability than that pre-trained by Moco. We fine-tune these two models in BOT~\cite{luo2019bag} on Market1501 for a few epochs, respectively. Then, we visualize the feature responses of our UP-ReID and Moco in Figure~\ref{fig: Vis}. As we can see, in the downstream tasks, UP-ReID pre-trained model could capture identity-related attributes (\eg, trouser color) and fine-grained features (\eg, shoes color) better than Moco pre-trained model, which demonstrates the effectiveness of the proposed designs, like the intrinsic contrastive constraint, in our UP-ReID.

\begin{figure}[h!]
	\centering
	\includegraphics[width=8.2cm, height=6.2cm]{Vis.pdf}
	\setlength{\abovecaptionskip}{7pt}
	\vspace{-3mm}
	\caption{Visualization of the features corresponding to the Moco and our UP-ReID schemes.}
	\label{fig: Vis}
\end{figure}

\section{Broader Impacts}
As for positive impact, we demonstrate that a suitable pre-trained model can benefit downstream person ReID tasks with higher accuracy and faster convergence speed. This will improve efficiency and effectiveness of a series of ReID tasks and save human costs in these areas.

As for negative impact, many public ReID datasets are coming from unauthorized surveillance data, which may cause an invasion of privacy and other security issues. Thus, the collection process should be public and make sure that human subjects in the datasets are aware that they are being recorded. Strict regulation should also be established for ReID datasets to avoid ethical issues.



\end{document}
