\documentclass[superscriptaddress,twocolumn,showpacs,prb]{revtex4-1}
\usepackage[utf8]{inputenc}
\usepackage{amsmath}
%\usepackage{mathtools}
\usepackage{braket}
\usepackage{graphicx}
\usepackage{amsfonts}
\usepackage{pgfplots}
\usepackage{csquotes}
\usepackage{hhline}
\usepackage{amssymb}
\usepackage{listings}
\usepackage{color}

\definecolor{codegreen}{rgb}{0,0.6,0}
\definecolor{codegray}{rgb}{0.5,0.5,0.5}
\definecolor{codepurple}{rgb}{0.58,0,0.82}
\definecolor{backcolour}{rgb}{0.95,0.95,0.92}
 
\lstdefinestyle{mystyle}{
    backgroundcolor=\color{backcolour},   
    commentstyle=\color{codegreen},
    keywordstyle=\color{magenta},
    numberstyle=\tiny\color{codegray},
    stringstyle=\color{codepurple},
    basicstyle=\footnotesize,
    breakatwhitespace=false,         
    breaklines=true,                 
    captionpos=b,                    
    keepspaces=true,                 
    numbers=left,                    
    numbersep=5pt,                  
    showspaces=false,                
    showstringspaces=false,
    showtabs=false,                  
    tabsize=2
}
 
\lstset{style=mystyle}
\usepackage{subfigure}
%\usepackage{subcaption}
\newtheorem{teorema}{Theorem}
\newcommand*{\Comb}[2]{{}^{#1}C_{#2}}
\usepackage{dcolumn}
\usepackage{tabularx}
\setcounter{secnumdepth}{3}
\usepackage[colorlinks=true,linkcolor=blue,citecolor=blue,urlcolor=blue]{hyperref}
\usepackage{longtable}
\usepackage{braket}
%\newtheorem{teorema}{Theorem}
\usepackage{float}
\newcolumntype{C}{>{\centering\arraybackslash}X}
\begin{document}
\title{Supplementary Information: Demonstration of a general fault-tolerant quantum error detection code for $(2n+1)$-qubit entangled state on IBM 16-qubit quantum computer}

\author{Ranveer Kumar Singh}
\email{ranveersfl@gmail.com}
\affiliation{Department of Mathematics, \\Indian Institute of Science Education and Research Bhopal, Bhauri 462066, Madhya Pradesh, India}
\author{Bishvanwesha Panda}
\email{bishvanweshapanda@gmail.com}
\affiliation{Indian Institute of Science Education and Research Kolkata,\\ Mohanpur 741246, West Bengal, India}

\author{Bikash K. Behera}
\email{bkb18rs025@iiserkol.ac.in}
\author{Prasanta K. Panigrahi}
\email{pprasanta@iiserkol.ac.in}
\affiliation{Department of Physical Sciences,\\ Indian Institute of Science Education and Research Kolkata, Mohanpur 741246, West Bengal, India}

\maketitle

\section{Simulation of error detection protocol}
For simulating the error detection protocol, we used QISKit to take both simulation results. The QASM code for the same is as follows: 

\lstinputlisting[language=Python]

\subsection{Measurement data}
We performed all the simulations on QISKit and recorded the countings of each of the measurement result over the two ancillary error syndrome qubit in 8192 shots. From the countings, the probability of each error \textit{i.e.} bit-flip error, phase-flip error and arbitrary phase-change error was extracted. The data is shown in the table \ref{qed_table1} below.
\begin{table}[h!]
\begin{center}
 \begin{tabular}{c c c c c} 
 \hline
 \hline
 Error & $\{0 , +\}$ & $\{1 , +\}$&$\{0 , -\}$&$\{1 , -\}$ \\ [0.5ex] 
 \hline
 \hline
 $Y_{\pi/3}$ &0.747 &0 &0&0.253   \\ 
 \hline
 $X_{\pi/3}$ &0.75 &0.25&0&0  \\
 \hline
 $X_{\pi/3}Y_{\pi/3}$ & 0.56&0.185&0.066&0.188\\
 \hline
 $X_{\pi/3}Y_{2\pi/3}$ & 0.18&0.063&0.184&0.574   \\  
\hline
 $X_{2\pi/3}Y_{\pi/3}$ & 0.19&0.55&0.195&0.063   \\ 
 \hline
 $X_{2\pi/3}Y_{2\pi/3}$ & 0.06&0.19&0.056&0.185 \\
 \hline
  $R=X_{\pi/2}Y_{\pi/2}$&0.25&0.252&0.252&0.245\\
  \hline
  $H$&0&0.503&0.497&0\\[1ex] 
 \hline
 \hline
\end{tabular}
\caption{\textbf{Probability of each type of error.} Here $\{0,+\},\{1,+\},\{0,-\}$ and $\{1,-\}$ represent the two qubit states $\Ket{00},\Ket{10},\Ket{01}$ and $\Ket{11}$ respectively. $+$ is the shorthand for $\Ket{+}=\frac{1}{\sqrt{2}}\big(\Ket{0}+\Ket{1}\big)$ and $-$ is the shorthand for $\Ket{-}=\frac{1}{\sqrt{2}}\big(\Ket{0}-\Ket{1}\big)$. $\Ket{+},\Ket{-}$ are the states of the second ancillary syndrome qubit before the Hadamard operation in the circuit of Fig. \ref{qed_fig2} in the bit-flip and phase-flip cases respectively.}
\label{qed_table1}
\end{center}
\end{table}
\begin{table}[h!]
\begin{center}
 \begin{tabular}{c c c c c} 
 \hline
 \hline
 $\theta$ & $\{0 , +\}$ & $\{1 , +\}$&$\{0 , -\}$&$\{1 , -\}$ \\ [0.5ex] 
 \hline
 \hline
 $-\pi$ &0 &1 &0&0   \\ 
 \hline
 $-14\pi/15$ &0.012 &0.988&0&0  \\
 \hline
 $-13\pi/15$ & 0.045&0.955&0&0\\
 \hline
 $-12\pi/15$ & 0.092&0.908&0&0   \\  
\hline
 $-11\pi/15$ & 0.1644&0.8356&0&0.   \\ 
 \hline
 $-10\pi/15$ & 0.25&0.75&0&0 \\
 \hline
  $-9\pi/15$&0.35&0.65&0&0\\
  \hline
  $-8\pi/15$&0.45&0.55&0&0\\
  \hline
  $-7\pi/15$&0.55&0.45&0&0\\
  \hline
   $-6\pi/15$&0.65&0.35&0&0\\
  \hline
  $-5\pi/15$&0.752&0.248&0&0\\
  \hline
 $-4\pi/15$&0.843&0.157&0&0\\
  \hline
   $-3\pi/15$&0.905&0.095&0&0\\
  \hline
  $-2\pi/15$&0.952&0.048&0&0\\
  \hline
  $-\pi/15$&0.987&0.013&0&0\\
  \hline
  $0$&1&0&0&0\\
  \hline
  $\pi/15$&0.99&0.0091&0&0\\
  \hline
 $2\pi/15$&0.957&0.043&0&0\\
  [1ex] 
 \hline
 \hline
\end{tabular}
\caption{\textbf{Probability of each type of error for applied error $\varepsilon=X_{\theta}$ with varying $\theta$.}}
\label{qed_table2}
\end{center}
\end{table}
\begin{table}[h!]
\begin{center}
 \begin{tabular}{c c c c c} 
 \hline
 \hline
 $\theta$ & $\{0 , +\}$ & $\{1 , +\}$&$\{0 , -\}$&$\{1 , -\}$ \\ [0.5ex] 
 \hline
 \hline
 $3\pi/15$ &0.906 &0.094 &0&0   \\ 
 \hline
 $4\pi/15$ &0.831 &0.17&0&0  \\
 \hline
 $5\pi/15$ & 0.75&0.25&0&0\\
 \hline
 $6\pi/15$ & 0.658&0.342&0&0   \\  
\hline
 $7\pi/15$ & 0.554&0.446&0&0.   \\ 
 \hline
 $8\pi/15$ & 0.436&0.563&0&0 \\
 \hline
  $9\pi/15$&0.34&0.66&0&0\\
  \hline
  $10\pi/15$&0.25&0.75&0&0\\
  \hline
  $11\pi/15$&0.164&0.836&0&0\\
  \hline
   $12\pi/15$&0.094&0.91&0&0\\
  \hline
  $13\pi/15$&0.044&0.956&0&0\\
  \hline
  $14\pi/15$&0.012&0.988&0&0\\
  \hline
  $\pi$&0&1&0&0\\[1ex]
  \hline
 \hline
\end{tabular}
\caption{\textbf{Probability of each type of error for applied error $\varepsilon=X_{\theta}$ with varying $\theta$.} (continued\dots)}
\label{qed_table3}
\end{center}
\end{table}
\begin{table}[h!]
\begin{center}
 \begin{tabular}{c c c c c} 
 \hline
 \hline
 $\theta$ & $\{0 , +\}$ & $\{1 , +\}$&$\{0 , -\}$&$\{1 , -\}$ \\ [0.5ex] 
 \hline
 \hline
 $-\pi$ &0 &0 &0&1   \\ 
 \hline
 $-14\pi/15$ &0.011 &0&0&0.99  \\
 \hline
 $-13\pi/15$ & 0.044&0&0&0.956\\
 \hline
 $-12\pi/15$ & 0.098&0&0&0.902   \\  
\hline
 $-11\pi/15$ & 0.166&0&0&0.834   \\ 
 \hline
 $-10\pi/15$ & 0.251&0&0&0.75 \\
 \hline
  $-9\pi/15$&0.35&0&0&651\\
  \hline
  $-8\pi/15$&0.45&0&0&554\\
  \hline
  $-7\pi/15$&0.56&0&0&0.44\\
  \hline
  $-6\pi/15$&0.66&0&0&0.34\\
  \hline
  $-5\pi/15$&0.75&0&0&0.25\\
  \hline
 $-4\pi/15$&0.84&0&0&0.164\\
  \hline
  $-3\pi/15$&0.905&0&0&0.095\\
  \hline
  $-2\pi/15$&0.957&0&0&0.042\\
  \hline
  $-\pi/15$&0.988&0&0&0.012\\
  \hline
  $0$&1&0&0&0\\
  \hline
  $\pi/15$&0.989&0&0&0.011\\
  \hline
 $2\pi/15$&0.957&0&0&0.043\\
  [1ex] 
 \hline
 \hline
\end{tabular}
\caption{\textbf{Probability of each type of error for applied error $\varepsilon=Y_{\theta}$ with varying $\theta$.}}
\label{qed_table4}
\end{center}
\end{table}
\begin{table}[h!]
\begin{center}
 \begin{tabular}{c c c c c} 
 \hline
 \hline
 $\theta$ & $\{0 , +\}$ & $\{1 , +\}$&$\{0 , -\}$&$\{1 , -\}$ \\ [0.5ex] 
 \hline
 \hline
 $3\pi/15$ &0.905 &0 &0&0.095   \\ 
 \hline
 $4\pi/15$ &0.831 &0&0&0.17  \\
 \hline
 $5\pi/15$ & 0.751&0&0&0.25\\
 \hline
 $6\pi/15$ & 0.65&0&0&0.35   \\  
\hline
 $7\pi/15$ & 0.56&0&0&0.44   \\ 
 \hline
 $8\pi/15$ & 0.45&0&0&0.552 \\
 \hline
  $9\pi/15$&0.35&0&0&0.65\\
  \hline
  $10\pi/15$&0.25&0&0&0.75\\
  \hline
  $11\pi/15$&0.168&0&0&0.832\\
  \hline
  $12\pi/15$&0.092&0&0&0.908\\
  \hline
  $13\pi/15$&0.039&0&0&0.96\\
  \hline
 $14\pi/15$&0.012&0&0&0.99\\
  \hline
  $\pi$&0&0&0&1\\[1ex]
  \hline
 \hline
\end{tabular}
\caption{\textbf{Probability of each type of error for applied error $\varepsilon=Y_{\theta}$ with varying $\theta$.} (continued\dots) }
\label{qed_table5}
\end{center}
\end{table}
\begin{table}[h!]
\begin{center}
 \begin{tabular}{c c c c c} 
 \hline
 \hline
 $\theta$ & $\{0 , +\}$ & $\{1 , +\}$&$\{0 , -\}$&$\{1 , -\}$ \\ [0.5ex] 
 \hline
 \hline
 $-\pi$ &0 &0 &1&0   \\ 
 \hline
 $-14\pi/15$ &0.011 &0&0.988&0  \\
 \hline
 $-13\pi/15$ & 0.044&0&0.956&0\\
 \hline
 $-12\pi/15$ & 0.096&0&0.904&0   \\  
\hline
 $-11\pi/15$ & 0.163&0&0.837&0   \\ 
 \hline
 $-10\pi/15$ & 0.25&0&0.75&0 \\
 \hline
  $-9\pi/15$&0.35&0&0.65&0\\
  \hline
  $-8\pi/15$&0.45&0&0.55&0\\
  \hline
  $-7\pi/15$&0.55&0&0.45&0\\
  \hline
  $-6\pi/15$&0.65&0&0.35&0\\
  \hline
  $-5\pi/15$&0.75&0&0.25&0\\
  \hline
  $-4\pi/15$&0.83&0&0.17&0\\
  \hline
  $-3\pi/15$&0.91&0&0.09&0\\
  \hline
  $-2\pi/15$&0.96&0&0.04&0\\
  \hline
  $-\pi/15$&0.99&0&0.011&0\\
  \hline
  $0$&1&0&0&0\\
  \hline
  $\pi/15$&0.99&0&0.01&0\\
  \hline
 $2\pi/15$&0.96&0&0.043&0\\
  [1ex] 
 \hline
 \hline
\end{tabular}
\caption{\textbf{Probability of each type of error for applied error $\varepsilon=Z_{\theta}$ with varying $\theta$.}}
\label{qed_table6}
\end{center}
\end{table}
\begin{table}[h!]
\begin{center}
 \begin{tabular}{c c c c c} 
 \hline
 \hline
 $\theta$ & $\{0 , +\}$ & $\{1 , +\}$&$\{0 , -\}$&$\{1 , -\}$ \\ [0.5ex] 
 \hline
 \hline
 $3\pi/15$ &0.903 &0 &0.097&0   \\ 
 \hline
 $4\pi/15$ &0.831 &0&0.17&0  \\
 \hline
 $5\pi/15$ & 0.76&0&0.24&0\\
 \hline
 $6\pi/15$ & 0.65&0&0.34&0   \\  
\hline
 $7\pi/15$ & 0.55&0&0.45&0   \\ 
 \hline
 $8\pi/15$ & 0.44&0&0.56&0 \\
 \hline
  $9\pi/15$&0.35&0&0.65&0\\
  \hline
  $10\pi/15$&0.25&0&0.75&0\\
  \hline
  $11\pi/15$&0.17&0&0.83&0\\
  \hline
  $12\pi/15$&0.099&0&0.9&0\\
  \hline
  $13\pi/15$&0.04&0&0.96&0\\
  \hline
$14\pi/15$&0.011&0&0.989&0\\
  \hline
  $\pi$&0&0&1&0\\[1ex]
  \hline
 \hline
\end{tabular}
\caption{\textbf{Probability of each type of error for applied error $\varepsilon=Z_{\theta}$ with varying $\theta$.} (continued\dots)}
\label{qed_table7}
\end{center}
\end{table}
\end{document} 