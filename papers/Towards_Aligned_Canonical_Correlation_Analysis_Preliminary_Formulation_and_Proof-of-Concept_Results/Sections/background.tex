%\reminder{Describe the basic version of CCA we are looking into}

Canonical correlation analysis (CCA) is a powerful tool to learn the shared latent components of two datasets by projecting them to the same space and enforcing the similarity of the projected data. Given two centered and aligned datasets $\mathbf{X}\in \mathbb{R}^{D_x \times N}$ and $\mathbf{Y}\in \mathbb{R}^{Dy \times N}$ where $N$ is the number of samples, $D_x$ and $D_y$ represent the dimensions of $\mathbf{X}$ and $\mathbf{Y}$, respectively, one popular CCA formulation is seeking for the two projection matrices $\mathbf{U}\in\mathbb{R}^{d\times D_x}$ and $\mathbf{V}\in\mathbb{R}^{d\times D_y}$ with $d\ll \min (D_x,\,D_y)$, and shared representation/embedding $\mathbf{S}\in\mathbb{R}^{d \times N}$ by solving the following  problem
% \begin{align}
% 	\min_{\mathbf{U},\mathbf{V}, \mathbf{S}}  \quad&  \left \|\mathbf{U}{\mathbf{X}} -\mathbf{S}\right\|_F^2 + \left\|\mathbf{V}{\mathbf{Y}} -\mathbf{S}\right\|_F^2\label{eq:cca}
%  \end{align}
\begin{equation}
    {\textbf{min}}_{\mathbf{U,V,S}} || \mathbf{UX} - \mathbf{S} ||^{2}_{F} + || \mathbf{VY} - \mathbf{S} ||^{2}_{F}
 \label{eq:cca}
\end{equation}
\noindent under the constraint that $\mathbf{S}\mathbf{S}^\top=\mathbf{I}$ which avoids the trivial solution, i.e., $\mathbf{U}=\mathbf{0}$, $\mathbf{V}=\mathbf{0}$, and $\mathbf{S}=\mathbf{0}$, and ensures the $d$ latent components assembled in the rows of $\mathbf{S}$ are uncorrelated to each other. Here, the symbols $\top$ and $\|\cdot\|_F$ respectively stand for matrix transpose and Frobenius norm operators, and $\mathbf{I}$ is identity matrix with the suitable size. The minimization problem in Eq. \eqref{eq:cca} admits global optimal solution: the rows of $\mathbf{S}$ are the $d$ eigenvectors corresponding to the top-$d$ eigenvalues of $\mathbf{X}^\top(\mathbf{X}\mathbf{X}^\top)^{-1}\mathbf{X}+\mathbf{Y}^\top(\mathbf{Y}\mathbf{Y}^\top)^{-1}\mathbf{Y}$ with $(\cdot)^{-1}$ denoting the matrix inverse operator, $\mathbf{U} = \mathbf{S}\mathbf{X}^\top
(\mathbf{X}\mathbf{X}^\top)^{-1}$, and $\mathbf{V} =\mathbf{S}\mathbf{Y}^\top (\mathbf{Y}\mathbf{Y}^\top)^{-1}$, e.g., \cite{harold1936relations}.