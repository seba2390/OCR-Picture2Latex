%%
%% This is file `sample-acmtog.tex',
%% generated with the docstrip utility.
%%
%% The original source files were:
%%
%% samples.dtx  (with options: `acmtog')
%% 
%% IMPORTANT NOTICE:
%% 
%% For the copyright see the source file.
%% 
%% Any modified versions of this file must be renamed
%% with new filenames distinct from sample-acmtog.tex.
%% 
%% For distribution of the original source see the terms
%% for copying and modification in the file samples.dtx.
%% 
%% This generated file may be distributed as long as the
%% original source files, as listed above, are part of the
%% same distribution. (The sources need not necessarily be
%% in the same archive or directory.)
%%
%%
%% Commands for TeXCount
%TC:macro \cite [option:text,text]
%TC:macro \citep [option:text,text]
%TC:macro \citet [option:text,text]
%TC:envir table 0 1
%TC:envir table* 0 1
%TC:envir tabular [ignore] word
%TC:envir displaymath 0 word
%TC:envir math 0 word
%TC:envir comment 0 0
%%
%%
%% The first command in your LaTeX source must be the \documentclass command.


\documentclass[sigconf]{acmart}

%%
%% \BibTeX command to typeset BibTeX logo in the docs
\AtBeginDocument{%
  \providecommand\BibTeX{{%
    Bib\TeX}}}

%% Rights management information.  This information is sent to you
%% when you complete the rights form.  These commands have SAMPLE
%% values in them; it is your responsibility as an author to replace
%% the commands and values with those provided to you when you
%% complete the rights form.
\setcopyright{acmcopyright}
\copyrightyear{2023}
\acmYear{2023}
\acmDOI{XXXXXXX.XXXXXXX}
% \acmConference[MLG '23]{19th International Workshop on Mining and Learning with Graphs}{August XX,
%   2023}{Long Beach, CA}

%%
%% These commands are for a JOURNAL article.


%%
%% Submission ID.
%% Use this when submitting an article to a sponsored event. You'll
%% receive a unique submission ID from the organizers
%% of the event, and this ID should be used as the parameter to this command.
%%\acmSubmissionID{123-A56-BU3}

%%
%% For managing citations, it is recommended to use bibliography
%% files in BibTeX format.
%%
%% You can then either use BibTeX with the ACM-Reference-Format style,
%% or BibLaTeX with the acmnumeric or acmauthoryear sytles, that include
%% support for advanced citation of software artefact from the
%% biblatex-software package, also separately available on CTAN.
%%
%% Look at the sample-*-biblatex.tex files for templates showcasing
%% the biblatex styles.
%%

%%
%% The majority of ACM publications use numbered citations and
%% references.  The command \citestyle{authoryear} switches to the
%% "author year" style.
%%
%% If you are preparing content for an event
%% sponsored by ACM SIGGRAPH, you must use the "author year" style of
%% citations and references.
\citestyle{acmauthoryear}


\newcommand{\reminder}[1]{{\textsf{\textcolor{red}  {[#1]}}}}
% \newcommand{\fromOld}[1] {{\textsf{\textcolor{green}{[#1]}}}}

%\newcommand{\comment}[1]{ {\tiny *** #1 *** }  }
\newcommand{\hide}[1]{}

\usepackage{caption}
\usepackage{subfigure}
%\usepackage{subcaption}
\usepackage{multirow}
\usepackage{diagbox}
\usepackage{url}
\usepackage[ruled,linesnumbered]{algorithm2e}
\usepackage{algorithmic}

%%
%% end of the preamble, start of the body of the document source.
\begin{document}

%%
%% The "title" command has an optional parameter,
%% allowing the author to define a "short title" to be used in page headers.

%\title{Graph Regularizer Augmentation for Data Recovery from Multiple Aggregated Views}

\title{Towards Aligned Canonical Correlation Analysis: Preliminary Formulation and Proof-of-Concept Results}

%%
%% The "author" command and its associated commands are used to define
%% the authors and their affiliations.
%% Of note is the shared affiliation of the first two authors, and the
%% "authornote" and "authornotemark" commands
%% used to denote shared contribution to the research.

\author{Biqian Cheng}
\affiliation{%
  \institution{UC Riverside}
 \city{Riverside, CA}
  \country{USA}
  }
\email{bchen158@ucr.edu}

\author{Evangelos E. Papalexakis}
\affiliation{%
 \institution{UC Riverside}
 \city{Riverside, CA}
 \country{USA}
}
 \email{epapalex@cs.ucr.edu}
 
 \author{Jia Chen}
\affiliation{%
  \institution{UC Riverside}
  \city{Riverside, CA}
  \country{USA}
}
\email{jiac@ucr.edu}



%%
%% By default, the full list of authors will be used in the page
%% headers. Often, this list is too long, and will overlap
%% other information printed in the page headers. This command allows
%% the author to define a more concise list
%% of authors' names for this purpose.

%%
%% The abstract is a short summary of the work to be presented in the
%% article.
\begin{abstract}
Canonical Correlation Analysis (CCA) has been a tried and tested analytical model which seeks to jointly embed two or more views of the data in a maximally correlated latent space. In addition to a powerful data mining tool, CCA has many implications for cutting-edge self-supervised representation learning approaches, as one can cast a number of recent approaches as variants of that model. In this work we explore what happens when a fundamental assumption of that model breaks: what if the alignment across different views of the data is unknown? Typically, we would first attempt to align the different views, and subsequently apply CCA or any other model in order to embed the data views. Can we do better if we align and embed at the same time? In this work, we seek to jointly solve the alignment and embedding by formulating and solving both problems under the same umbrella of Aligned Canonical Correlation Analysis (ACCA). We present a preliminary formulation and alternating optimization algorithm and proof-of-concept results.

\end{abstract}


%%
%% Keywords. The author(s) should pick words that accurately describe
%% the work being presented. Separate the keywords with commas.
\keywords{Canonical Correlation Analysis, Alignment, Matching, Data Integration}

%%
%% This command processes the author and affiliation and title
%% information and builds the first part of the formatted document.
\maketitle

\section{Introduction}
\label{sec-intro}
\section{Introduction}  \label{sec:introduction}

\newcommand\inexpIntro[3]{#1?(#2,#3).}
\newcommand\rinexpIntro[3]{*#1?(#2,#3).}
\newcommand\outexpIntro[3]{#1!(#2,#3).}
\newcommand\outatomIntro[3]{#1!(#2,#3)}

We propose a fully automated method for proving termination of \(\pi\)-calculus processes.
Although there have been a lot of studies on termination analysis for the \(\pi\)-calculus
and related calculi~\cite{Deng06IC,Demangeon07,SangiorgiTermination,KobayashiHybrid,Yoshida04IC,DBLP:journals/jlp/DemangeonHS10,Venet98SAS}, most of them have been rather theoretical,
and there have been surprisingly little efforts in developing  fully automated termination
verification methods and tools based on them. To our knowledge,
Kobayashi's \typical{}~\cite{TyPiCal,KobayashiHybrid} is the only exception that
can prove termination of \(\pi\)-calculus processes (extended with natural numbers)
fully automatically, but its termination analysis is quite limited (see Section~\ref{sec:relatedwork}).

Our method is based on a reduction to termination analysis for sequential programs:
we translate a \(\pi\)-calculus process \(P\) to a sequential program \(S_P\), so that
if \(S_P\) is terminating, so is \(P\). The reduction allows us to use
powerful, mature methods and tools
for termination analysis of sequential programs~\cite{heizmann2016ultimate,freqterm,DBLP:conf/lics/PodelskiR04,Kuwahara2014Termination,DBLP:journals/cacm/CookPR11}.

The idea of the translation is to convert a chain of communications on replicated input
channels to a chain of recursive function calls of the target sequential program.
Let us consider the following Fibonacci process:
\begin{align*}
    & \rinexpIntro{\fib}{n}{r}
        \ifexp{n<2}{ \soutatom{r}{1} \\ &\quad}
                   { \nuexp{s_1} \nuexp{s_2} (\outatomIntro{\fib}{n-1}{s_1} \PAR \outatomIntro{\fib}{n-2}{s_2} \PAR \sinexp{s_1}{x}\sinexp{s_2}{y}\soutatom{r}{x+y}) \\}
    & \PAR \outatomIntro{\fib}{m}{r}
\end{align*}
Here, the process
$\rinexpIntro{\fib}{n}{r} \ldots$ is a function server that computes the \(n\)-th Fibonacci number
in parallel and returns the result to \(r\),
and $\outatom{\fib}{m}{r}$ sends a request for computing the \(m\)-th Fibonacci number;
those who are not familiar with the syntax of the \(\pi\)-calculus may wish to consult
Section~\ref{sec:targetlanguage} first.
To prove that the process above is terminating for any integer \(m\),
it suffices to show that there is no infinite chain of communications on $\fib$:
\[
    \fib(m,r) \to \fib(m_1,r_1) \to \fib(m_2,r_2) \to \cdots.
\]
We convert the process above to the following program:\footnote{The actual translation
  given later is a little more complex.}
\begin{verbatim}
 let rec fib(n) = if n<2 then () else (fib(n-1) [] fib(n-2)) in
 fib(m)
\end{verbatim}
Here, \texttt{[]} represents the non-deterministic choice.
Note that, although the calculation of Fibonacci numbers is not preserved,
for each chain of communications on \texttt{fib}, there is a corresponding
sequence of recursive calls:
\[
\mathtt{fib}(m) \to \mathtt{fib}(m_1) \to \mathtt{fib}(m_2) \to \cdots.
\]
Thus, the termination of the sequential program above implies the termination of
the original process.
As shown in the example above, (i) each communication on a replicated input channel
is converted to a function call, (ii) each communication on a non-replicated input
channel is just removed (or, in the actual translation, replaced by a call of
a trivial function defined by \(f(\seq{x})=(\,)\)), and (iii) parallel composition
is replaced by a non-deterministic choice.
We formalize the translation outlined above and prove its correctness.

The basic translation sketched above sometimes loses too much information.
For example, consider the following process:
\begin{align*}
    & \rinexpIntro{\pre}{n}{r} \soutatom{r}{n-1} \\
    & \PAR \rinexpIntro{f}{n}{r} \ifexp{n<0}{ \soutatom{r}{1} }
                                       { \nuexp{s} (\outatomIntro{\pre}{n}{s} \PAR \sinexp{s}{x}\outatomIntro{f}{x}{r}) } \\
    & \PAR \outatomIntro{f}{m}{r}
\end{align*}
The translation sketched above would yield:
\begin{verbatim}
  let pred(n) = n-1 in
  let rec f(n) = if n<0 then () else (pred(n) [] f(*)) in
  f(m)
\end{verbatim}
Here, \texttt{*} represents a non-deterministic integer: since we have removed
the input $\sinatom{s}{x}$, we do not have information about the value of \( x \).
As a result, the sequential program above is non-terminating, although the original
process is terminating.
To remedy this problem, we also refine the basic translation above by using a refinement
type system for the \(\pi\)-calculus. Using the refinement type system,
we can infer that the value of \(x\) in the original process is less than \(n\),
so that we can refine the definition of \texttt{f} to:
\begin{verbatim}
 let rec f(n) = ... else (pred(n) [] let x=* in assume(x<n);f(x))
\end{verbatim}
The target program is now terminating, from which
we can deduce that the original process is also terminating.
We have implemented an automated tool based on the refined translation above.

The contributions of this paper are summarized as follows.
\begin{itemize}
\item The formalization of the basic translation from the \(\pi\)-calculus
  (extended with integers) to sequential programs, and a proof of its correctness.
\item The formalization of a refined translation based on a refinement type system.
\item An implementation of the refined translation, including automated refinement type
  inference based on CHC solving, and experiments to evaluate the effectiveness of
  our method.
\end{itemize}

The rest of this paper is structured as follows.
Section~\ref{sec:targetlanguage} introduces the source and target languages
of our translation.
Section~\ref{sec:approach} 
formalizes the basic translation, and proves its correctness.
Section~\ref{sec:refinement} refines the basic translation by using a refinement type system.
Section~\ref{sec:implementation} reports an implementation and experiments.
Section~\ref{sec:relatedwork} discusses related work,
and Section~\ref{sec:conclusion} concludes the paper.


\section{Background}
\label{sec-backgound}

% Panoptic segmentation

% 3D segmentation

% Multi-object tracking

% Online 3D panoptic:

% PanopticFusion: (IROS 2019)
% https://arxiv.org/pdf/1903.01177.pdf
%
% - most similar to ours
% - PSPNet + M-RCNN + 2D fusion
% - volumetric mapping, 
% - greedy matching with IoU -> optimal only with 0.5 threshold
% - voxel & class weighting
% - CRF regularisation
%
% - good:
%
% - bad:
%  - CRF post-processing step
%  - greedy data-association
%    - can't be tuned for lower overlap ratios -> has to have high framerate, large changes in viewpoint could break this
%    - IoU: sensitive to 2D labels projecting over object borders (CRF and voxel weighting seem to alleviate this)

% Voxblox++: (Robotics & automation letters 2019)
% https://arxiv.org/pdf/1903.00268.pdf
% https://github.com/ethz-asl/voxblox-plusplus
%
% - M-RCNN + geometric segmentation + fusion 
% - data association of geometric segments with 3D overlap (no. points inside volume), fixed threshold for min number of points
% - instance label is assigned to a segment based on highest overlap
% - only one detected segment per reference label, as in PanopticFusion and Ours
% - TSDF Integration 
%
% good: 
% - because of geometric segmentation objects with no associated semantic class can also be segmented
% bad:
% - two different object segment types -> confusing, overly complicated ?
% - quite inaccurate (fixed below)

% Reconstructing Interactive 3D Scenes by Panoptic Mapping and CAD Model Alignments (ICRA 2021)
% https://arxiv.org/pdf/2103.16095.pdf
% https://github.com/hmz-15/Interactive-Scene-Reconstruction
%
% - based heavily on Voxblox++, much more accurate
% - Scene-graph ("contact graph") for mapping object relations
% - Search & replace voxels with CAD models, with geometrical and physical constraints
% - Object 6D pose
% - Format for robot interaction
%
% - Segmentation: bilateral fusion of geomatric and semantic segments -> reduce segmentation noise compared to Voxblox++
% - Fusion: triplet count improves consistency over Voxblox++ pairwise count strategy (take semantic label into account in addition to instance and geometry)
% - Fusion: instance labels are also combined if there is enough overlap with common geometric label for long enough time
%   - this means multiple detections can match the same reference unlike ours, voxblox++ and PanopticFusion ?
%

% Panoptic-MOPE: (ROBOTICS AND AUTOMATION LETTERS 2020)
% https://ieeexplore.ieee.org/stamp/stamp.jsp?tp=&arnumber=8977356
% https://github.com/hoangcuongbk80/Object-RPE/tree/panoptic-mope
%
% - novel RGB-D semantic segmentation model + M-RCNN
% - camera tracking based on "addaptively weighted optimization of geometric, appearance, and semantic cues"
% - surfel map: 
%   - how does it scale ? authors satate they tested on room-sized environments, but could be applied in larger scale as well ...
%     - could maybe be applied as VO in a SLAM algorithm ...
%   - demo only on a small pallet + surroundings, might not be applicable in large-scale SLAM

% US VS THEM:
%
% - based heavily on PanopticFusion, with modifications:
%   - instead of greedy data-association (which seems to be the case in others as well), we solve LAP (JPDA?)
%     - overlap threshold can be tuned, which renders the algorithm more flexible
%     - could be extended to dynamic tracking ?
%   - multiple options for association likelihood
%   - outlier rejection (either clustering or probabilistic)
%   - test different options for decreasing processing time
%   - no post-processing
%
% - model-agnostic:
%   - completely separated from segmentation
%   - does not care how point clouds are obtained -> applicable for LIDAR segmentation (e.g. EfficientLPS) as well
%
% - also agnostic to localisation method
%   - could, however, be utilised to find landmark locations / poses

% More compact version of this paragraph to introduction to save space?
%Panoptic segmentation -- proposed in \cite{panoptic_segmentation} -- aims to solve the unified task of semantic- and instance segmentation. Semantic classes are separated to \textit{stuff} -- amorphous, unquantifiable regions like sky, road or floor -- and \textit{things} -- quantifiable objects. The distinction between the two can vary depending on the application, but a semantic class can only belong to one or another. The article also proposes a unified panoptic evaluation metric, coined \textbf{Panoptic Quality} (PQ). Many 2D approaches to panoptic segmentation -- \textit{e.g.} \cite{panopticfpn,seamless,panoptic_deeplab,efficientps} -- have since been proposed. Deep neural networks for performing semantic- or instance segmentation directly on the 3D reconstruction -- \textit{e.g.} on \cite{scannet,s3dis,paris_lille_3d} -- have also been proposed, but since they require the reconstructed 3D scene, they are mostly offline approaches and therefore out of scope for this work. Some recent works also apply panoptic segmentation to point clouds -- \textit{e.g.} methods in the SemanticKITTI panoptic segmentation competition \cite{semantic_kitti} -- mostly aimed at segmenting LiDAR output. They are suitable for online processing, but similar to RGB-D images require a method for tracking object instances persistent in both time and space. In fact, our proposed method, as well as some others mentioned in this work, could use segmented LiDAR point clouds as an input similarly to RGB-D images.

PanopticFusion \cite{panopticfusion} is the first work to propose online integration of panoptic image segmentations to a 3D reconstruction. They integrate point clouds generated from segmented images to a TSDF voxel volume \cite{tsdf,voxblox} by greedily matching detected segments with the reconstruction and regulating each voxel's corresponding instance with a weighting function. Semantic labels are inferred in a bayesian manner based on confidence scores provided by the segmentation model. They also apply a Conditional Random Field (CRF) to regularise the reconstruction, improving results significantly. Voxblox++ \cite{voxblox++} -- introduced later the same year -- is a similar approach that also integrates segmented RGB-D images into a TSDF volume. It leverages geometric segmentation of depth images to improve instance segmentation accuracy. Both geometric and semantic segments are used to compute a pair-wise weight, which is used to greedily match them with segments in the reconstruction. Because of the geometric segmentation, the method allows segmentation of objects with no known semantic class in addition to objects recognised by the instance segmentation model. 

Recently, \cite{interactive_3d_scenes} built upon the idea of Voxblox++. They apply Voxblox++ for 3D instance integration, with two small but effective modifications: the pair-wise weight is replaced by a triplet weight that also takes semantic labels into account in the fusion, and -- in addition to geometric segments -- instance segments are fused if they overlap by a significant amount. The article introduces a method for searching and aligning CAD models to reconstructed objects based on geometry and semantic class, as well as geometrical and physical rules. With the CAD models, a contact graph and interactive virtual scene are reconstructed to allow a robot to simulate its interaction with the environment. SceneGraphFusion \cite{scenegraphfusion} is another approach that forms a scene graph online from a stream of RGB-D images, but unlike the above-mentioned approach, it generates the graph with a deep neural network, after which the panoptic labels for geometrically segmented portions of the 3D reconstruction are produced a side product.

Panoptic-MOPE \cite{panoptic_mope} is another recent approach, which integrates sequences of RGB-D images into a surfel reconstruction. Unlike other mentioned approaches -- which assume the camera pose either known or estimated elsewhere -- it also tracks camera movements based on geometric-, appearance- and semantic cues. The method also applies a novel RGB-D panoptic segmentation model. Although it is only tested on room-sized environments, the authors claim it could be scaled to larger environments as well.

%\section{Tensors and decompositions}
%\label{sec-tensorsdecomp}
%\input{Sections/TensorsandDecompositions}


\section{Proposed Method}
\label{sec-methods}
\section{Methods}\label{sec:methods}

Given a single facial image of an individual, our objective is to endow the pre-trained T2I model with the ability to efficiently re-contextualize this unique identity under various textual prompts. These prompts may include variations in clothing, accessories, styles, or backgrounds.


The overall framework is shown in Fig.\ref{fig:pipeline}, given a pre-trained T2I model, 
to achieve fast and identity-preserved image generation, we first directly encode the target identity into the word embedding  space (represented as the pseudo word $S*$) with the proposed $M^2$ ID encoder. Afterward,
$S*$ is integrated with the input template prompt for 
generating the text-guided image. To empower the editability for the target identity, a novel \emph{self-augmented editability learning} is further introduced to train the $M^2$ ID encoder with the editability objective.


In the following parts, we first briefly introduce the pre-trained diffusion-based text-to-image model used in our work, then describe our proposed  $M^2$ ID encoder and self-augmented editability learning in detail, respectively.

\subsection{Preliminary}
In this work, we adopt the open-sourced Stable Diffusion 2.1-base (SD) as our text-to-image model, which has been trained on billions of images and shows amazing image generation quality and prompt understanding. 

SD is a kind of Latent Diffusion Model (LDM) \cite{rombach2022high}. LDM firstly represents the input image $x$ in a lower resolution latent space $z$ via a Variational Auto-Encoder (VAE) \cite{kingma2013auto}. Then a text-conditioned diffusion model is trained to generate the latent code of the target image from text input $c$. The loss function of this diffusion model can be formulated as:
\begin{equation}
    \mathcal{L}_{diffusion} = \mathbb{E}_{\epsilon,z,c,t}[\lVert{\epsilon - \epsilon_{\theta}(z_t,c,t)}\rVert_2^2],
\end{equation}
where $\epsilon_{\theta}$ is the noise predicted by the model with learnable parameters $\theta$, $\epsilon$ is noise sampled from standard normal distribution, $t$ is the time step, and $z_t$ is noisy latent at the time step $t$.

During inference, the image is generated by two stages: latent code is first generated by the diffusion model, then the decoder is employed to map the latent code to image space. 

\subsection{$M^{2}$ ID Encoder}

To accurately represent the input face identity, we propose a novel Multi-word Multi-scale embedding ID encoder ($M^2$ ID encoder) for an accurate mapping, which is achieved by the multi-scale ID features extracted from a dedicated backbone then followed by multiple word embedding projection.




\myparagraph{Backbone.} We argue that an accurate representation of the face identity is crucial, while common image encoder CLIP (which is adopted by \emph{all} existing works) fails for that purpose since it can not capture the identity feature as accurately as the face ID encoder which has been trained for face identification tasks on the large-scale face dataset. As \cite{bhat2023face} shows, the current best CLIP VIT-L/14 model is still much worse than the face recognition model on top-1 face identification accuracy ($80.95\%$ vs $87.61\%$). Therefore, we employ a ViT backbone \cite{dosovitskiy2020image} pre-trained on a large-scale face recognition dataset to faithfully extract ID-aware features for input face.


\myparagraph{Multi-scale Feature.}  However, naively mapping the final layer's output identity vector $v_{final}$ could only bring sub-optimal identity preservation. The reason lies in that $v_{final}$ mainly contains the high-level semantics which is suitable for discriminative tasks (\eg, face identification) rather than generative tasks. For example, the same identity with different expressions should share similar representation under the face recognition training loss, while the generation requests more detailed information like facial expressions. Hence, only mapping the last layer representation could become an information bottleneck for the image generation task. To deal with the above problem, we propose to utilize multi-scale features from the face encoder to represent an identity more faithfully. Specifically, the identity vector is augmented by four CLS embeddings ($v_3$, $v_6$, $v_{12}$, $v_{12}$) from the 3rd, 6th, 9th, and 12th layer, respectively. Formally, the multi-scale feature from the ID encoder is depicted as follows:
\begin{equation}
% \setlength\abovedisplayskip{1pt}
% \setlength\belowdisplayskip{1pt}
V = [v_3, v_6, v_9, v_{12}, v_{final}].
\end{equation}

\myparagraph{Multi-word Embeddings.} The multi-scale feature is further projected into the word embedding domain. To maintain the original large-scale T2I model's generalization and editability, we leave all its parameters and structure unchanged. As a result, it raises the problem that a single word embedding is hard to faithfully represent the face's identity. Therefore, we further propose a multi-word projection mechanism to represent a face with multi-word embedding:
\begin{equation}
\begin{aligned}
s_{i} = MLP_i(V), \text{for } i = 1, ..., k,
\end{aligned}
\end{equation}
where $k$ is the number of embeddings . Experimentally, we set $k=2$ as depicted in Fig.\ref{fig:pipeline}.  Following \cite{gal2023designing}, $l_2$ regularization is further adapted to constrain the output embedding:

\begin{equation}
    \mathcal{L}_{reg} = \sum_{i=1}^k{\lVert{s_{i}}\rVert}.
\end{equation}

Benefiting from the above-dedicated ID feature, we can facilitate highly identity-preservation control in the embedding space only, without sacrificing pre-trained T2I model's editability caused by feature injection. 

\subsection{Self-Augmented Editability  
 Learning}
Current efficient methods are trained under the reconstruction paradigm, which is given an input face image $I$, the objective to learn a unique word $S*$ such that the $S*$ can reconstruct $I$. However, in real-world applications, we wish to generate a set of new images, such as "watercolor style of $S*$ face", "$S*$ as a police". As a result, there exists a huge inconsistency between training and testing. We hope we can rely on the inductive bias in the word embedding space to achieve editability, but in reality, as Fig.\ref{fig:exp_self_aug} shows, the generated image doesn't always follow the text prompt if we only train encoder under the reconstruction objective. 

To deal with the inconsistency between training and testing, in this paper, we propose a novel \emph{self-augmented editability learning} to take the editing task into the training phase. However, collecting such pair data for the editing task is difficult. Experimentally,  we notice that the current state-of-the-art general text-to-image models can generate celebrity (\eg, Boris Johnson, Emma Watson) in different contexts with good identity preservation and text-coherence. With this insight, The \emph{self-augmented editability learning} utilizes the pre-trained model itself to construct a self-augmented dataset by generating various celebrity faces along with the target edited celebrity images, which will be used to train the $M^2$ ID encoder with the editability objective. Formally, the construction of the dataset includes the following four steps:


\myparagraph {Step 1: Celebrity List Generation.} Firstly we collect a candidate celebrity list. The large language model (\ie, ChatGPT) is used to generate the most famous 400 names in four fields (\ie, sports players, singers, actors, and politicians). After filtering duplicate ones, we finally get 1015 celebrity names.

\myparagraph {Step 2: Celebrity Face Generation.} We propose to use generated face images rather than real images because the model has its own understanding of celebrity. Specifically, the celebrities who appeared less frequently in the Stable Diffusion training dataset are not very similar to the real person while these generated faces maintain a high level of identity resemblance. We use the prompt template "<celebrity-name> face, looking at the camera" to produce the source images, then followed by face crop and alignment operation to get face-only images. A face-only image is kept if its short size is larger than 128 pixels. 

\myparagraph {Step 3: Edit Prompts and Edited Images Generation.} We manually design a variety of prompts that contain images of celebrities in different jobs, styles, and accessories (\eg, "<celebrity-name> as a chef", "oil painting style, <celebrity-name> face"). Then these prompts are transformed to images by the T2I model as edited images, and the <celebrity-name> in prompts is replaced by the pseudo word $S^*$ as Editing Prompts.

\myparagraph {Step 4: Data Cleaning.} After the above procedures, we can now get the initial self-augmented dataset consisting of a set of triplets, <identity face, editing prompt, edited images>. Due to the instability of the current diffusion model, the edited images don't always follow the edit instructions. 
Therefore, we need to filter out the noise data in the self-augmented dataset. We employ ID Loss and CLIP score which reflect identity similarity and text-image consistency as the metrics to rank the edited images for every prompt, then the top $25\%$ triplets at kept as the final training set. 

Finally, we construct a high-quality self-augmented dataset from the pre-trained T2I model itself, which is then used for edit-oriented training.


\subsection{Training}
We combine the FFHQ \cite{karras2019style} and the self-augmented dataset to train our proposed $M^2$ ID encoder. The total loss consists of noise prediction loss of diffusion and the embedding regularization loss:  
\begin{equation}
\mathcal{L}_{total} = \mathcal{L}_{diffusion} + \lambda \mathcal{L}_{reg} ,
\end{equation} 
where $\lambda$ is the embedding regularization weight.




\section{Experimental Evaluation}
\label{sec-Experimentalevaluations}
To validate the effectiveness of our proposed model ACCA, we will generate synthetic data with groundtruth $\mathbf{P}$ and investigate the performance of estimated $\mathbf{P}$ in terms of the matching accuracy between the entities in $\mathbf{X}$ and $\mathbf{Y}$. In all numerical tests, we set the hyperparameters $\gamma_1$ and $\gamma_2$ to be $0.0001$. The initial $\mathbf{P}$ is obtained by solving the optimal matching directly using $\mathbf{X}$ and $\mathbf{Y}$ without considering the canonical correlation between the two datasets, i.e., solving the following minimization problem 
\begin{align}
\min_{\mathbf{P}}   & \left \|{\mathbf{X}} -\mathbf{YP}\right\|_F^2 +\gamma_1\|\mathbf{P}\mathbf{P}^\top -\mathbf{I}\|_F^2+\gamma_2\|\mathbf{P}^\top\mathbf{P} -\mathbf{I}\|_F^2\label{eq:initialp}
\end{align}
\noindent under the constraints specified in Eqs.\eqref{eq:con1}, \eqref{eq:con2}, and \eqref{eq:acca}. We use the \emph{scipy.optimize.minimize}  solver to find the optimal $\mathbf{P}$.

\subsection{Synthetic Data Generation}
\label{sec-Datasets}
We first generate the groundtruth latent representation of the two datasets, namely $\mathbf{Z}\in\mathbb{R}^{\bar{d}\times N}$, where the columns of $\mathbf{Z}$ are  $N$  i.i.d. samples drawn from  multivariate normal distribution with zero mean and identity covariance  of size $\bar{d}\times\bar{d} $. Next, two aligned datasets $\mathbf{X}$ and $\bar{\mathbf{Y}}\in\mathbb{R}^{D_y\times N}$ are generated from their shared latent representation $\mathbf{Z}$ through two independent random projections: $\mathbf{X}=\mathbf{W} \mathbf{Z} $ and $\mathbf{Y}=\mathbf{Q}\mathbf{Z} $ where $\mathbf{W}\in\mathbb{R}^{D_x \times \bar{d}}$ and $\mathbf{Q}\in\mathbb{R}^{D_y \times \bar{d}}$. For each experiment, the groundtruth $\bar{\mathbf{P}}$ is a random permutation matrix with only one entry in each row and column to be $1$ and the rest to be $0$s. Next, we have two \emph{unaligned} datasets: $\mathbf{X}$ and $\mathbf{Y}=\bar{\mathbf{Y}}\bar{\mathbf{P}}$. The involved parameters are set as follows: $N=20$, $\bar{d}=2$, $d=7$, $D_x=15$, and $D_y=10$.


\subsection{Experimental Results}
\label{sec-Results}
\begin{figure}[!htp]
  \includegraphics[width=0.8\linewidth]{Images/Overall_loss.pdf}
  \caption{Loss as a function of iterations}
  \label{fig:loss}
\end{figure}

After setting the entropy upper bound hyperparameter $\lambda$ to be $0.1$, we run $10$ times of Monte Carlo experiments and report the loss of Eq. \eqref{eq:acca_loss} for each iteration in Figure \ref{fig:loss}. The curve in Figure 1 represents the average loss per iteration and the width of the shade stands for the standard derivation of the loss. Clearly, our proposed ACCA converges to a stable point using the generated synthetic data.

\begin{figure}[!htp]
  \includegraphics[width=0.8\linewidth]{Images/top-K_Accuracy_entropy=0.1.pdf}
  \caption{Top-k Accuracy of ACCA and Random guess} 
  \label{fig:topk}
\end{figure}

In Figure \ref{fig:topk}, we report the top-$k$ matching accuracy with mean and standard deviation, defined as the percentage of rows in the estimated permutation $\mathbf{P}$ whose top $k$ entries' index set includes the nonzero entry index of the true permutation  $\bar{\mathbf{P}}$, with $k=1,2,3,4,$ and $5$, in comparison with such accuracy from random guess which is $k/N$.  According to our experimental records as shown in  figure \ref{fig:topk}, it's obvious that our ACCA framework has significantly better performance in predicting the potential alignment between two datasets, than that obtained from the random guess. 

\begin{figure*}[!htp]
    \subfigure[True $\mathbf{P}$]{\includegraphics[width = 0.19\textwidth]{Images/True_PI.pdf}}
    \subfigure[Entropy = 0.1; top-3 acc.: 0.519]{\includegraphics[width = 0.19\textwidth]{Images/Estimate_PI_entropy0.1.pdf}}
    \subfigure[Entropy = 0.5; top-3 acc.: 0.59]{\includegraphics[width = 0.19\textwidth]{Images/Estimate_PI_entropy0.5.pdf}}
    \subfigure[Entropy = 1; top-3 acc.: 0.575]{\includegraphics[width = 0.19\textwidth]{Images/Estimate_PI_entropy1.0.pdf}}
    \subfigure[Entropy = 2;
    top-3 acc.: 0.31]{\includegraphics[width = 0.19\textwidth]{Images/Estimate_PI_entropy2.0.pdf}}
    \caption{Estimated alignment matrix for different Entropy bounds.}
    \label{fig:alignment_matrices}
\end{figure*}

Next, we visualize the alignment performance with respect to different values of the hyperparameter $\lambda$ in Figure \ref{fig:alignment_matrices} where we plot the real permutation matrix $\bar{\mathbf{P}}$ and the estimated $\mathbf{P}$ as gray-scale images with darker grid blocks representing higher values of the corresponding entries of $\bar{\mathbf{P}}$ or $\mathbf{P}$. As uniform distribution leads to the highest entropy, $\lambda$ can not exceed $log(N)$ (=$N \times 1/N \times log(1/N)$). With  $\lambda$ increasing, more nonzero entries are showing up in $\mathbf{P}$ as expected. With proper setup of entropy bound hyperparameter, the performance of ACCA will be further improved, with the comparison of prediction accuracies related to different entropy cases in Figure \ref{fig:alignment_matrices}. 
%($=N\times 1/N \times log(1\N)$). 







\section{Conclusion \& Future Work}
\label{sec-Conclusion}

\begin{comment}
\begin{figure}
\includegraphics[width=\linewidth]{figs/beyond_tss_lesion.pdf}
\caption[]{End-to-End runtime lesion study of the entire MNIST dataset and the FMA featurized music dataset. Each of DROP's contributions provides a runtime improvement.}
\label{fig:beyond_lesion}
\end{figure}
\end{comment}



\section{Conclusion}
\label{sec:conclusion}

Advanced data analytics techniques must scale to rising data volumes. 
DR techniques offer a powerful toolkit when processing these datasets, with PCA frequently outperforming popular techniques in exchange for high computational cost. 
In response, we propose DROP, a new dimensionality reduction optimizer. 
DROP combines progressive sampling, progress estimation, and online aggregation to identify high quality low dimensional bases via PCA without processing the entire dataset by balancing the runtime of downstream tasks and achieved dimensionality. 
Thus, DROP provides a first step in bridging the gap between quality and efficiency in end-to-end DR for downstream \red{analytics}. 

%We revisit canonical operators for time series dimensionality reduction and the measurement study of~\cite{keogh-study}, and show that PCA is more effective than popular alternatives in the data mining literature often by a margin of over $2\times$ on average on gold-standard time series benchmark data sets with respect to output data dimension. More surprisingly, we empirically demonstrate that a small number of samples are sufficient to accurately characterize directions of maximum variance and obtain a high-quality low-dimensional transformation.




\begin{acks}
The authors would like to thank Yunshu Wu for initial discussions.
Research was supported by the National Science under CAREER grant no. IIS 2046086 and CREST Center for Multidisciplinary Research Excellence in Cyber-Physical Infrastructure Systems (MECIS) grant no. 2112650 and by the US Department of Transportation under University Transportation Center (UTC) on Railway Safety. Any opinions, findings, and conclusions or recommendations expressed in this material are those of the author(s) and do not necessarily reflect the views of the funding parties.

\end{acks}


\bibliographystyle{plainnat}
%{\footnotesize
\bibliography{main.bib}
%}
\end{document}
\endinput
