%% LaTeX Template for ISIT 2020
%%
%% by Stefan M. Moser, October 2017
%%
%% derived from bare_conf.tex, V1.4a, 2014/09/17, by Michael Shell
%% for use with IEEEtran.cls version 1.8b or later
%%
%% Support sites for IEEEtran.cls:
%%
%% http://www.michaelshell.org/tex/ieeetran/
%% http://moser-isi.ethz.ch/manuals.html#eqlatex
%% http://www.ctan.org/tex-archive/macros/latex/contrib/IEEEtran/
%%

%\documentclass[conference,letterpaper]{IEEEtran} % for ISIT submission
%\documentclass[journal,twocolumn,final]{IEEEtran} % for arxiv


%% depending on your installation, you may wish to adjust the top margin:
%\addtolength{\topmargin}{9mm}

%%%%%%
%% Packages:
%% Some useful packages (and compatibility issues with the IEEE format)
%% are pointed out at the very end of this template source file (they are
%% taken verbatim out of bare_conf.tex by Michael Shell).
%
% *** Do not adjust lengths that control margins, column widths, etc. ***
% *** Do not use packages that alter fonts (such as pslatex).         ***
%
\usepackage[utf8]{inputenc}
\usepackage[T1]{fontenc}
\usepackage[hidelinks]{hyperref}
\usepackage{url}
\usepackage{ifthen}
\usepackage{cite}
\usepackage[cmex10]{amsmath} % Use the [cmex10] option to ensure complicance
% with IEEE Xplore (see bare_conf.tex)
\usepackage{amsfonts,amssymb,amsmath,bbm}
\usepackage{graphicx}
\usepackage{color}
\usepackage{ifthen}
\usepackage{comment}

\usepackage{tikz,pgfplots}
\pgfplotsset{width=7cm,compat=1.3}

\usepackage[justification=centering]{caption}

\usepackage[ruled,vlined]{algorithm2e}

\usepackage[arxiv]{optional}  % arxiv/submission

%\usepackage{amsthm}

\usepackage{enumerate}

%% Please note that the amsthm package must not be loaded with
%% IEEEtran.cls because IEEEtran provides its own versions of
%% theorems. Also note that IEEEXplore does not accepts submissions
%% with hyperlinks, i.e., hyperref cannot be used.

\interdisplaylinepenalty=2500 % As explained in bare_conf.tex


% this allows align environments to break between pages!!!
\allowdisplaybreaks[1]

%%%%%%
% correct bad hyphenation here
\hyphenation{op-tical net-works semi-conduc-tor}

%%%%%%%%%%%%%%%%%%%%%%%%%%%
%%%%%%% DEFINITIONS %%%%%%%
%%%%%%%%%%%%%%%%%%%%%%%%%%%

\def\cA{\mathcal{A}}
\def\cB{\mathcal{B}}
\def\cC{\mathcal{C}}
\def\cE{\mathcal{E}}
\def\cF{\mathcal{F}}
\def\cG{\mathcal{G}}
\def\cP{\mathcal{P}}
\def\cR{\mathcal{R}}
\def\cS{\mathcal{S}}
\def\cT{\mathcal{T}}
\def\cU{\mathcal{U}}
\def\cV{\mathcal{V}}
\def\cW{\mathcal{W}}
\def\cX{\mathcal{X}}
\def\cY{\mathcal{Y}}
\def\cZ{\mathcal{Z}}
\def\cPi{\mathcal{\Pi}}

\def\tw{\tilde{w}}
\def\pih{\hat{\pi}}
\def\Pih{\hat{\Pi}}

\newtheorem{lemma}{Lemma}
\newtheorem{fact}{Fact}
\newtheorem{theorem}{Theorem}
\newtheorem{definition}{Definition}
%\newtheorem*{assumption*}{Assumption}

\newcommand{\ve}[1]{\underline{#1}}
\newcommand{\eqdef}{\stackrel{\scriptscriptstyle \triangle}{=}}
\newcommand{\mdef}{\stackrel{\text{\tiny def}}{=}}
\newcommand{\E}{\mathbb{E}}
\def\Real{\mathbb{R}}
\def\P{\mathbb{P}}
\DeclareMathOperator {\qd}{qd}
\DeclareMathOperator {\tr}{tr}

\def\bM{\mathbf{M}}


\usepackage{caption}
\usepackage{subcaption}


%%%%%%%%%%%%ketan
\newcommand{\kscomment}[1]{{\color{red} #1}}
\newcommand{\ksmargin}[1]{\marginpar{\color{red}\footnotesize [KS]:
		#1}}

\newcommand{\phcomment}[1]{{\color{blue}#1}}
\newcommand{\phmargin}[1]{\marginpar{\color{blue}\footnotesize [PH]:
		#1}}
%\newtheorem{proof}{Proof}
%\newtheorem*{proof*}{Proof}

\newcommand{\map}[3]{#1: #2 \rightarrow #3}
\newcommand{\signal}{\pi}
\newcommand{\RR}{\mathbb{R}}
\newcommand{\supp}{\mathrm{supp}}
\newcommand{\de}{\mathrm{d}}

%%%%%%%%%%%ketan

%\hoffset=0.0in
%\oddsidemargin=0in
%\evensidemargin=0in
%\textwidth=6.5in
%\marginparsep=0.0in
%\marginparwidth=0.0in

%\voffset=0.0in
%\topmargin=-0.5in
%\headheight=12pt
%\headsep=20pt
%\textheight=9.45in
%\footskip=36pt



%%%%%%%%%%%%%%%%%%%%%%%%%%%%%%%%%%%%%%%%%%%%%%%%%%%%%%%%%%%%%%%%%%%%%%%%%%%%%%%%%%%%%%%%%%%%%%%%%%%%%%%%%%%%

%%%\documentclass[letterpaper, 10 pt, conference]{ieeeconf}
%%%
%%%\IEEEoverridecommandlockouts
%%%\overrideIEEEmargins
%%%% The preceding line is only needed to identify funding in the first footnote. If that is unneeded, please comment it out.
%%%\usepackage{cite}
%%%\usepackage{amsmath,amssymb,amsfonts}
%%%\usepackage{algorithmic}
%%%\usepackage{graphicx}
%%%\usepackage{textcomp}
%%%\usepackage{bbm}
%%%\usepackage{color}
%%%
%%%\def\BibTeX{{\rm B\kern-.05em{\sc i\kern-.025em b}\kern-.08em
%%%		T\kern-.1667em\lower.7ex\hbox{E}\kern-.125emX}}
%%%
%%%\usepackage{enumitem}
\usepackage{mathtools}
%%%
%%%
\newcommand{\red}[1]{\textcolor{red}{#1}}

\newcommand{\cM}{\mathcal{M}}
\newcommand{\cN}{\mathcal{N}}
\newcommand{\cL}{\mathcal{L}}
\newcommand{\cI}{\mathcal{I}}
%\newcommand{\cT}{\mathcal{T}}
\newcommand{\cH}{\mathcal{H}}
%\newcommand{\cA}{\mathcal{A}}
\newcommand{\fm}{\mathfrak{m}}
\newcommand{\fu}{\mathfrak{u}}
\newcommand{\bB}{\mathbf{B}}
\newcommand{\bV}{\mathbf{V}}
\newcommand{\hv}{\hat{v}}
\newcommand{\chv}{\hat{V}}
\newcommand{\hhv}{\tilde{v}}
\newcommand{\vc}[1]{{#1}}
\newcommand{\mat}[1]{\mathbf{#1}}
\newcommand{\yy}{s} % available letters: d e o s

%\newcommand{\E}{\mathbb{E}}
\def\Real{\mathbb{R}}
\def\P{\mathbb{P}}
%%%
%%%\newtheorem{theorem}{{Theorem}}
%%%\newtheorem{lemma}[theorem]{{Lemma}}
%%%\newtheorem{proposition}[theorem]{{Proposition}}
%%%\newtheorem{corollary}[theorem]{{Corollary}}
%%%\newtheorem{claim}[theorem]{{Claim}}
%%%\newtheorem{definition}{{Definition}}
%%%\newtheorem{fact}{{Fact}}
%%%\newtheorem{IEEEproof}{{Proof}}
%%%
%%%%\usepackage[showframe=true]{geometry}
%%%%\usepackage{enumitem}
%%%
%%%%\usepackage{setspace}
%%%%\setlist[itemize]{leftmargin=*}
%%%%\setlist[enumerate]{leftmargin=*}
%%%
%%%% \setlength{\lineskiplimit}{1pt}
%%%%\setlength{\lineskip}{0pt}
%%%%\setlength{\abovedisplayskip}{2pt}
%%%%\setlength{\belowdisplayskip}{2pt}
%%%%\setlength{\abovedisplayshortskip}{1pt}
%%%%\setlength{\belowdisplayshortskip}{1pt}
%%%%\noindent{\baselineskip=0pt}
%%%
%%%% this allows align environments to break between pages!!!
%%%\allowdisplaybreaks[1]
%%%
%%%


%\setlength{\lineskiplimit}{1pt}
%\setlength{\lineskip}{1pt}
%\setlength{\abovedisplayskip}{2pt}
%\setlength{\belowdisplayskip}{2pt}
%\setlength{\abovedisplayshortskip}{1pt}
%\setlength{\belowdisplayshortskip}{1pt}    
