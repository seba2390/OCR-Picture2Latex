


\begin{table}[t]
    \centering
    \caption{Parameters of the \SI{140}{GHz} RFFE devices used in the analysis.}

    \setlength{\tabcolsep}{3pt}
    \begin{tabular}{|>{\raggedright}m{0.9in}|c|c|c|c|}
    \hline
    
    \textbf{Parameter} &
    \textbf{LNA$^{\boldsymbol{(1)}}$} &
    \textbf{LNA$^{\boldsymbol{(2)}}$} &
    \textbf{Mixer$^{\boldsymbol{(1)}}$} &
    \textbf{Mixer$^{\boldsymbol{(2)}}$}
    \tabularnewline \hline
    Design [\textmu m] & 4 &  2-4 & 1 & 1
    \tabularnewline
    Noise Figure [dB] & 7.50 & 7.48 & 21.53 & 20.47
    \tabularnewline
    Gain [dB] & 11.13 & 16.56 & -1.74 & -0.52
    \tabularnewline
    IIP3 [dBm] & -9.15 & -8.90 & -4.45 & -3.88
    \tabularnewline
    Power [mW] &  4.80 & 15.90 & 5.00 & 5.00
    \tabularnewline \hline
    \end{tabular}
    \label{tab:rffe140}
\end{table}


\section{Link-Layer Simulation} \label{sec:link}

% \textcolor{red}{
% Notis to do:
% \begin{itemize}
%     \item Create a block diagram for the RFFE.  You can use the one from the IEEE Access paper.
%     \item Describe the interference and signal model.
%     \item Describe the design options and why we are analyzing them. 
%     \item Describe the fitting procedure using an initial fit followed by non-linear least squares
%     \item Show plots of SNR out vs.\ signal and interfering SNR.
%     \item Describe when the model is a good fit.
%     \item Summarize fitted parameters in a table.
% \end{itemize}}

The model in Section~\ref{sec:model} is a simplified
abstraction of an actual RFFE.  In this section, 
we validate the model and extract the parameters 
for \eqref{eq:gamout} with realistic, circuit simulations
of potential RFFEs at \SI{28}{GHz} and \SI{140}{GHz}.


\subsection{Signal and interference model}
We consider a downlink scenario in a communication link 
between an NR basestation (gNB) and a mobile device (UE). 
For each slot, the gNB generates a physical downlink shared
channel (PDSCH) that includes both information and control 
signals. The receiver uses the demodulation reference 
signal (DM-RS) for practical channel estimation. 
To compensate the common phase error (CPE) the 3GPP 5G NR,
standard introduce the phase tracking reference signal (PT-RS). 
The receiver performs coherent CPE estimation using the algorithm 
described in~\cite{syrjala2019phase}. For time synchronization,
between the gNB and UE we utilize the primary (PSS) and the secondary synchronization signals (SSS).

To model the interference, we assume the presence of another gNB
that generates i.i.d. $\CN(0, 1)$ symbols in frequency-domain. The
symbols are modulated with OFDM to generate interference in an
adjacent band. Even though the signals from the two gNBs are
originally independent in the frequency domain, the presence of 
the non-linear processing introduces distortion to the main signal
from the adjacent band. As explained in Section~\ref{sec:model},
this increase the total received energy and causes the RFFE 
saturation point to happen much earlier.

%
% 140 GHz - Mixer (1) 
% Gain: -1.7360
%   IIP3: -4.4546
%     NF: 21.5290
%   PLO: -11
%  Power: 5
%
% 140 GHz - Mixer (2) 
%   Gain: -0.5185
%   IIP3: -3.8883
%     NF: 20.4715
%   PLO: -6
%  Power: 5
     
\begin{figure}[t]
    \centering
    \includegraphics[width = 0.99\linewidth]{fig/model_fit-eps-converted-to.pdf}
    \caption{Evaluation of the approximation model in \eqref{eq:gamout} for Design$^{(1)}$ at \SI{28}{GHz} for different input interference power.}
    \label{fig:model_fit}
\end{figure}



\subsection{Receiver configurations}
Similarly to~\cite{skrimponis2021towards} we consider a fully-digital superheterodyne receiver architecture as shown in Fig.~\ref{fig:arch}. The receiver has $N_\mathrm{rx}=16$ or $64$ antennas with independent RFFE processing. The received signal is amplified with a low-noise amplifier  (LNA), downconverted with an intermediate frequency (IF) mixer, amplified with an automated gain control (AGC) amplifier before the direct conversion mixer, and finally quantized with a pair of ADCs. The system use filters in the RF and IF domain to improve the image rejection and the dynamic range of the system. The actions of the RFFE devices for each receiver configuration is modeled with a different non-linear function $\Phi(\cdot)$.

In~\cite{skrimponis2021towards} the authors design and evaluate RFFE devices at \SI{140}{GHZ} based on a \SI{90}{nm} SiGE BiCMOS HBT technology. They focus on minimizing the power consumption of the RFFE devices at a certain performance in terms of gain, noise figure (NF), and the input third order intercept point (IIP3). The LNA designs vary in terms of number of stages, topology, and transistor size. While the proposed double-balanced active mixers are all based on the conventional Gilbert-cell design, they vary in transistor size. The mixer performance characteristics depend on the input power from the local oscillator (LO). To further optimize the power consumption of the receiver, the authors propose a novel LO distribution model. Specifically, the mixers in the same tile share a common LO driver using power dividers and amplifiers. They provide a method to determine the best configuration of LO drivers that achieve the same performance for a minimum power.

Using the components and the optimization framework described in~\cite{skrimponis2021towards} we select two of the optimized designs for the \SI{140}{GHz} systems in our analysis. These two designs achieve similar performance to the state-of-the-art design discussed in \cite{skrimponis2020power} while achieving a significant improvement in power consumption. Similarly, for the \SI{28}{GHz} devices we design two common base emitter base collector (CBEBC) LNAs, and two active mixers based on the common Gilbert cell design. We use the optimization framework in~\cite{skrimponis2021towards} to determine the number of LO drivers. For both frequencies, Design$^{(1)}$ use 4-bit ADCs while Design$^{(2)}$ use 5-bit ADC pairs. This assumption is based on prior works \cite{abbas2017millimeter,zhang2018low,abdelghany2018towards,dutta2019case} indicating that 4 bits are sufficient for the majority of cellular data and control operations. Based on a flash-based 4-bit ADC in \cite{Nasri2017}, we consider the ADC $\mathrm{FOM} = 65\;\mathrm{fJ/conv}$. We summarize the parameters for the devices at \SI{28}{GHz} and \SI{140}{GHz} in Table~\ref{tab:rffe28} and Table~\ref{tab:rffe140} respectively. 

% \textcolor{blue}{[SR:  Can someone provide
% a brief description of where the 28 GHz
% designs are coming from?
% Also, in Tables I and II (or somewhere else),
% can we put the total poewr consumption for
% the Designs 1 and 2 for the 28 and 140 GHz
% receivers.  ]}

\begin{table}[t]
    \centering
    \caption{Model and system parameters for the receiver designs at \SI{28}{GHz} and \SI{140}{GHz}.}
    \label{tab:model_param}
    \begin{tabular}{|>{\raggedright}m{0.5in}|c|c|c|c|}
        \hline
         \multirow{2}{*}{\textbf{Parameter}} & \multicolumn{2}{c|}{\textbf{\SI{28}{GHz}}} & \multicolumn{2}{c|}{\textbf{\SI{140}{GHz}}}
        \tabularnewline \cline{2-5}
        & $\text{Design}^{(1)}$ & $\text{Design}^{(2)}$ & $\text{Design}^{(1)}$ & $\text{Design}^{(2)}$
        
        \tabularnewline \hline
        \multicolumn{1}{|c|}{$\beta$} & 1.3865 & 1.2725 & 0.3099 & 0.1862
        \tabularnewline 
        
        \multicolumn{1}{|c|}{$\alpha_1$} & 0.0090 & 0.0024 & 0.0021 & 0.0004
        \tabularnewline 
        
        \multicolumn{1}{|c|}{$\alpha_2$} & 0.0058 & 0.0017 & 0.0014 & 0.0003
        \tabularnewline \hline
        
        \multicolumn{1}{|c|}{RX antennas} & 16 & 16 & 64 & 64
        \tabularnewline
        
        \multicolumn{1}{|c|}{NF [dB]} & 2.78 & 3.08 & 9.40 & 11.50
        \tabularnewline
        
        \multicolumn{1}{|c|}{Power [mW]} & 411 & 404 & 1682 & 1355
        \tabularnewline \hline
    \end{tabular}
\end{table}

\subsection{Model fitting}
For each receiver design we fit a model in form of
\eqref{eq:gamout}. Since this nonlinear function 
$\Phi(\cdot)$ depends on the parameters $\alpha_1,\alpha_2$, and
$\beta$ we can write the estimated output-SNR model as $\hat{\gamma}_\mathrm{out}(\gamma_\mathrm{sig}, \gamma_\mathrm{int}; \alpha_1, \alpha_2, \beta)$. For the \emph{initial 
heuristic} fit we set, 
\begin{align}
    \beta = \frac{1}{F}, \quad \alpha_1 = \alpha_2 = \frac{1}{\gamma_\mathrm{sat}F},
    \label{eq:fitinit}
\end{align}
where $F$ is the noise factor of the system, and $\gamma_\mathrm{sat}$
the saturation SNR in linear scale. % Note that these parameters 
%do not include any beamforming gain. %We use the downlink scenario  
%described in Section~\ref{sec:link} to obtain $\gamma_\mathrm{out}$ using \eqref{eq:inout_snr}.
We then  optimize the fit  using the non-linear least squares regression method and optimize a problem of the following form,
\begin{align}
   Q(\gamma_\mathrm{sig}, \gamma_\mathrm{int}) := \min_{\alpha_1, \alpha_2, \beta}\|& \gamma_\mathrm{out}(\gamma_\mathrm{sig}, \gamma_\mathrm{int}) \nonumber \\ &
    - \hat{\gamma}_\mathrm{out}(\gamma_\mathrm{sig}, \gamma_\mathrm{int}; \alpha_1, \alpha_2, \beta)) \|_2^2, \nonumber\\
\end{align}
where  $\gamma_\mathrm{out}$ are the measurements from the link-layer simulation using \eqref{eq:inout_snr}, and $\hat{\gamma}_\mathrm{out}$ is the estimate using the model in
\eqref{eq:gamout}. The optimized parameters for the \SI{140}{GHz} and \SI{28}{GHz} receiver
designs are summarized in Tab.~\ref{tab:model_param}. In Fig.~\ref{fig:model_fit} we show that the model in~\eqref{eq:gamout} provides a very good fit. In particular, we see the linear regime for low-input SNR and the saturation for high-input signal power. We show that as the input interference increases the model the saturation SNR is also changing.

