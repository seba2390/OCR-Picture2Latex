\begin{figure}[t]
    \centering
    \includegraphics[width = 0.99\linewidth]{fig/fig_analitic_model-eps-converted-to.pdf}
    % \begin{tikzpicture}
    %     \begin{axis}[restrict z to domain=0:inf,restrict x to domain=0:inf,restrict y to domain=0:inf]
    %     \addplot3[surf] {y/(1+y+0.5*x)};
    %     \end{axis}
    % \end{tikzpicture}
    \caption{Input-output SNR relation.}
    \label{fig:snr_inout}
\end{figure}

\begin{figure*}[!t]
    \centering
    % !TEX root = ../top.tex
% !TEX spellcheck = en-US

\begin{figure}[t]
\centering
\includegraphics[width=0.99\linewidth]{./fig/arch/arch.pdf}
\vspace{-3mm}
    \caption{{\bf Our single-stage approach.} We use an encoder-decoder architecture to progressively downsample the image and then to re-expand it. At each level of the decoder, we establish 3D-to-2D correspondences. Finally, we use a RANSAC-based PnP strategy~\cite{Lepetit09} 
    % \WJ{citation?} \YH{Fixed} 
     to infer a single reliable pose from these sets of correspondences. 
    % \WJ{Impossible to read text in printout, please try to have text in the figure with a similar font size as the surrounding article.}\YH{Fixed}
    }
\label{fig:arch}
\end{figure}

    \caption{High-level architecture of a fully-digital superheterodyne 
    receiver architecture. The architecture supports $N_\mathrm{rx}$ antennas and
    $N_\mathrm{str}$ digital streams. The light green boxes represent analog and the 
    dark-green boxes the digital components. In the RF front-end, 
    some component are not shown.}
    \label{fig:arch}
\end{figure*}

\section{Capacity Bound and Output SNR}
\label{sect:capacity}
Our goal is to characterize the performance of the discussed model in Sec.~\ref{sec:model} in terms of the spectral efficiency. To this end, similar to \cite{skrimponis2020efficient}, we make use of the concepts of the output SNR and input-output SNR relation described next. 

From Sec.~\ref{sec:model}, we have 
\begin{align}
    \label{eq:inoutre}
    \hat{\nbx} = \nbF\Phi(\nbF \herm \nbx, \nbD).
\end{align}

Assuming that the variables $\hat{x}_n$, $\nbd_n$, and $x_n$ have an underlying statistical model and are distributed as $\hat{X}$, $D$, and $X$, respectively, we can use the Bussgang-Rowe decomposition \cite{bussgang1952crosscorrelation,rowe1982memoryless} and model the non-linearity in the system (i.e., $\Phi(\cdot)$) as multiplying a scalar with its input and adding a noise which is uncorrelated with the input. More precisely, we can write
\begin{align}
\label{eq:lin_mod}
    \hat{X} = A X+ T,\quad \Exp |T|^2 &= \tau,
\end{align}
where
\begin{align}
\label{eq:alpha_tau_rx}
    A := \frac{\Exp\left[ \hat{X}^* X\right]}{\Exp\left|X\right|^2 }, \quad \tau :=  \Exp|\hat{X}- A X|^2,
    % \alpha_2 :=  \frac{1}{\overline{P}}\Exp|S-\alpha_{\rm rx}U|^2,
\end{align} 
with $X^*$ denoting the complex conjugate of $X$.

% In \cite{dutta2020capacity}, we have proved rigorously that in the system shown in Fig.~\ref{fig:abstract_model}, for many different functions $\Phi(\cdot)$, the elements of $\wbf$ can be viewed as i.i.d. $\CN(0, \tau)$.
In general, both $A$ and $\tau$ are functions of the 
input SNR $\gamma_\mathrm{in}$ 
so we may write $A = A(\gamma_{\mathrm{in}})$
and $\tau = \tau(\gamma_{\mathrm{in}})$. 
% Based on \cite[Theorem.~1]{dutta2020capacity} 
From \eqref{eq:lin_mod}, we can then define the 
{\it output SNR} off the desired signal $\nbs$ as,
\begin{equation} \label{eq:inout_snr}
    \gamma_{\mathrm{out}} = G(\gamma_{\mathrm{in}}) :=
    \frac{|A(\gamma_{\mathrm{in}})|^2}{\tau(\gamma_{\mathrm{in}})} 
    \gamma_{\rm sig}.
\end{equation}
This is the SNR that would be seen in attempting
to recover the input transmitted vector $\nbs$ from the
output vector $\hat{\nbs}$.  

Using the SNR enables us to bound the performance of the system in terms of the capacity. More precisely, using same steps as of \cite[Appendix~A]{skrimponis2020efficient}, we can show that the capacity of the system  can is lower bounded as,
\begin{equation}  \label{EQ:CAP_BAND_LIMITED1}
  C \geq  \frac{N_{\rm sig}}{N} f_s\log_2\left(1 +  \gamma_{\rm out}\right),
\end{equation}
where $f_s$ is the sample rate and $N_{\rm sig} = |I_{\rm sig}|$ represent the number of frequency bins for the signal. Moreover, assuming that the ADC performs oversampling with the ratio $\zeta$, using same steps as of \cite[Appendix~B]{skrimponis2020efficient}, we have

\begin{equation}  \label{EQ:CAP_BAND_LIMITED}
  C \geq  \frac{N_{\rm sig}\zeta}{N} f_s\log_2\left(1 +  \frac{\gamma_{\rm out}}{\zeta}\right),
\end{equation}
% where $f_s$ is the sample rate and $N_{\rm sig} = |I_{\rm sig}|$ represent the number of frequency bins for the signal. 


\begin{table}[t]
    \centering
    \caption{Parameters of the \SI{28}{GHz} RFFE devices used in the analysis.}

    \setlength{\tabcolsep}{3pt}
    \begin{tabular}{|>{\raggedright}m{0.9in}|c|c|c|c|}
    \hline
    
    \textbf{Parameter} &
    \textbf{LNA$^{\boldsymbol{(1)}}$} &
    \textbf{LNA$^{\boldsymbol{(2)}}$} &
    \textbf{Mixer$^{\boldsymbol{(1)}}$} &
    \textbf{Mixer$^{\boldsymbol{(2)}}$}
    \tabularnewline \hline
    Design [\textmu m] & 10 & 5 & 2 & 5
    \tabularnewline
    Noise Figure [dB] & 2.13 & 2.53 &  9.039 & 7.542
    \tabularnewline
    Gain [dB] & 14.26 & 12.85 &  0.16 &  3.558
    \tabularnewline
    IIP3 [dBm] & -1.456 & 0.603  & -3.1  & 2.1
    \tabularnewline
    Power [mW] &8.91 & 5.34 &   4.838 &   7.03
    \tabularnewline \hline
    \end{tabular}
    \label{tab:rffe28}
\end{table}

% New models
% 2 & 2
% 10.115 & 9.039
% & 
% 
% 4.8 & 4.8



In this paper, we will show through detailed simulations that the input-output SNR relation can be approximated in the form of
\begin{equation} \label{eq:gamout}
    \hat{\gamma}_{\rm out} = \frac{\beta \gamma_{\rm sig}}{
    1 + \alpha_1 \gamma_{\rm sig} + \alpha_2\gamma_{\rm int}},
\end{equation}
    where $\gamma_{\rm int} = \frac{1}{|I_{\rm int}|}\sum_{n \in I_{\rm int}}E_i[n]$.
for three parameters $\beta$ and $\alpha_1$ and $\alpha_2$
using which we can evaluate the receiver front-end performance. Intuitively, this formula suggest that due to the non-linearity in the system: (i) the signal energy is reduced; (ii) a ratio of the signal is distorted; (iii) a ratio of the adjacent band signal (i.e., interference) is leaked to the desired band.

From \eqref{eq:gamout}, we also observe that the output SNR saturates to the value of $\frac{\beta}{\alpha_1}$ as the input signal SNR increases. Furthermore, for higher values of the interference signal the saturation should accrue for lower values of the input SNR. One can also observe these from Fig.~\ref{fig:snr_inout} which illustrates \eqref{eq:gamout} for a fixed values of $\beta$, $\alpha_1$ and $\alpha_2$ and different values of desired and interference signal powers.
% signal-to-distortion-plus-interference-noise-and-ratio
% The trade offs between the received signals power,  




% To interpret the roles of the parameters
% Fig.~\ref{fig:snr_inout} plots
% $\gamma_{\mathrm{out}}$ vs.\ $\gamma_{\mathrm{sig}}$ and $\gamma_{\mathrm{int}}$ with both the values in dB-scale.  We see two distinct
% regimes.

% \paragraph*{Low input SNR regime}
% When $N_\mathrm{rx} \gamma_\mathrm{in}/F \ll \gamma_\mathrm{sat}$,
% the output SNR in \eqref{eq:inout_snr_nonlin} simplifies to
% \begin{equation} \label{eq:inout_snr_nl_low}
%     \gamma_{\mathrm{out}} = G(\gamma_{\mathrm{in}}) 
%     \approx\frac{N_\mathrm{rx}}{F}\gamma_{\mathrm{in}},
% \end{equation}
% which matches the linear case \eqref{eq:inout_snr_lin}.
% We will thus call $F$ in \eqref{eq:inout_snr_nonlin} the
% {\it effective noise figure}.  This regime
% appears on the left side of Fig.~\ref{fig:snr_inout}
% where $\gamma_\mathrm{out}$ (in dB) 
% increases linearly  with $\gamma_\mathrm{in}$ with an 
% offset given by beamforming gain $N_\mathrm{rx}$
% minus the effective noise figure $F$.

% We will see below that the low input SNR regime
% occurs when the RFFE components have sufficiently small
% input levels that they do not saturate and act linearly.
% In this case, the effective noise figure
% is determined by the standard noise figure analysis 
% with an additional term from the quantization noise at the ADC.

% \paragraph*{High input SNR regime}
% At very large input SNRs 
% ($N_\mathrm{rx} \gamma_\mathrm{in}/F \gg \gamma_\mathrm{sat}$),
% the output SNR in \eqref{eq:inout_snr_nonlin} simplifies to
% \begin{equation} \label{eq:inout_snr_nl_high}
%     \gamma_{\mathrm{out}} \approx G(\gamma_{\mathrm{in}}) 
%     \approx \gamma_\mathrm{sat}.
% \end{equation}
% Hence, the output SNR saturates as shown 
% in Fig.~\ref{fig:snr_inout}.  It is for this
% reason that $\gamma_\mathrm{sat}$ is called the saturation SNR.  
% In this case, we will see that the saturation SNR is determined
% by the nonlinearities in the devices and the quantization
% in the ADC.  

% \textcolor{red}{Abbas to do:
% \begin{itemize}
%     \item State the result on the Gaussianity 
%     \item State the capacity lower bound
%     \item Define output SNR
%     \item State the analytic model for the output SNR.
% \end{itemize}
% }
% We model the output SNR as:
% \begin{equation} \label{eq:gamout}
%     \gamma_{\rm out} = \frac{\beta \gamma_{\rm sig}}{
%     1 + \alpha_1 \gamma_{\rm sig} + \alpha_2\gamma_{\rm int}}.
% \end{equation}
