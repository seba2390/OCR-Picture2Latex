\section{Conclusions} \label{sec:conclusion}

Low-power designs for receivers above \SI{100}{GHz}
have traditionally relied on operating the
circuits with limited dynamic range either
via low-resolution ADCs or low-power mixers.
While there techniques have been successful
when considering in-band signal distortion,
the limited dynamic range can be an issue for
adjacent carrier interference.  
We have developed a simple 
mathematically model to describe this effect 
and fit the model parameters on realistic 
circuit designs at \SI{28}{GHz} and \SI{140}{GHz}.
The models were then be used in a simple
network simulation to estimate the effect of
adjacent carrier interfence in cellular systems
with two operators.  Our preliminary simulations
suggest that, at least under the parameters
considered, highly optimized power designs
are not significantly vulnerable to adjacent
carrier interference.  Future work can consider
other deployments where the adjacent carrier
interference could be higher.  For example,
short range local area signals operating adjacent
to cellular bands as well as mixed applications
such as terrestrial networks sharing spectrum 
with vehicular or UAV systems.  