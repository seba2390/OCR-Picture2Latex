\section{Introduction} \label{sec:intro}
A key challenge in millimeter wave (mmWave)
receiver front-ends is power consumption, particularly
for mobile and portable devices.
High power consumption is especially challenging
in emerging systems above \SI{100}{GHz} that need to support
a large number of array elements at high bandwidths
with relatively poor device efficiency \cite{mendez2015channel,rappaport2019wireless,skrimponis2020efficient,skrimponis2020power,yan2020dynamic,tan2020thz}.

Significant recent progress has been made with 
reduced power
architectures, most notably via low-resolution analog-to-digital converters (ADCs)
and phase shifters, as well low-power mixers \cite{naqvi2018review,abbas2017millimeter,zhang2018low,abdelghany2018towards,yan2019linearization,mo2017hybrid,jacobsson2017throughput}.  
A common theme in these designs
is to sacrifice dynamic range for lower power.
This design choice is often based on the fact that
communication systems in the mmWave bands typically operate
at relatively low signal-to-noise ratios (SNRs) per antenna
that can be recovered from beamforming.

However, most prior analyses of low-dynamic range
architectures have generally only considered the 
\emph{in-band} signal distortion.
Aggressive use of low dynamic range front-ends
introduces potential susceptibility of the receiver
to high power \emph{out-of-band} or 
adjacent carrier signals~\cite{marti2021hybrid,marti2021jammer,akhlaghpasand2020jamming}. 
These adjacent carrier signals can be particularly
large in uncoordinated 
cellular deployments with multiple carriers
as well as unlicensed use
\cite{rebato2017hybrid}.


 
\begin{figure}[t]
    \centering
    % \textcolor{red}{Abbas:  Add figure showing the model}
    \includegraphics[width = 0.99\linewidth]{fig/sys_3.pdf}
    \caption{Abstract mathematical model for a system
    with adjacent carrier interference and non-linear,
    noisy front-end. Sig. and Int. denote the desired signal and interference signal from the other transmitters, respectively.}
    % \textcolor{blue}{[SR:  Notis or Abbas,
    % Show "sig" and "int" going into
    % $\nbF^\herm$ (IFFT) which then outputs 
    % $\nbu$.  This way you see that the two
    % are in distinct bands.  You can remove "spectrum analyzer" .  Also shown $\widehat{\nbx}$ coming out of the top of $\nbF$.]} }
    % \caption{Abstract mathematical model for the system
    % with an adjacent carrier interference and non-linear,
    % noisy front-end. Vectors $\mathbf{s}$ and $\mathbf{v}$ are the equivalent frequency representations of the desired and interference signals, respectively.}
    \label{fig:abstract_model}
\end{figure}


For receiver design in conventional bands below \SI{6}{GHz},
it is well-known that interference rejection 
(also called \emph{blockers}) 
can be the dominant
driver of power consumption.
Some example low-power designs with good
out-of-band
blocker performance can be found in
\cite{sepidband2018cmos,trotskovsky20180,agrawal2018interferer}.
However, the design and analysis of 
low power designs under adjacent carrier
interference in the mmWave
range face unique challenges. 
Most importantly, standard
SAW and BAW (surface acoustic wave and bulk
acoustic wave) filters are not easily available in
these frequencies~\cite{mahon20175g}.
Recent work has attempted to use filters integrated into the package
\cite{watanabe2020review,siddiqui2021dual,gu2021antenna}, but these come
with added RF signal loss.  In addition, the nonlinearities inherent
in low-power mmWave designs can be difficult
to analyze and mitigate against.
For example, there has been some recent work
studying jamming signals on 
mmWave receivers with low-resolution ADCs
and other non-linear hardware impairments in
\cite{akhlaghpasand2020jamming,pirzadeh2019mitigation,marti2021hybrid,jacobsson2018massive}.
However, these works generally consider the case
where the jamming signal is in-band,
but spatially separated 
from the desired signal.  In this work,
we focus on adjacent carrier interference
where interfering signal is in a different
frequency band, as would occur in cellular,
licensed deployments.

For mmWave systems under adjacent
carrier interference, we thus wish to understand
the fundamental relation between power 
consumption and interference tolerance
and how this trade-off can be best optimized.
Towards this end, 
the contributions of the paper are three-fold:
First,  building on the work in
\cite{dutta2020capacity,skrimponis2020power,skrimponis2021towards,skrimponis2020efficient},
we provide a general methodology 
for mathematically analyzing the sensitivity of
receivers from adjacent carrier emissions.
Importantly, the framework can incorporate 
general power spectral densities of the interfering
signals, models
of the filters at various points in the receiver
chain, as well as non-linearity and quantization 
limits in the analog front-end.


Secondly, we apply the methodology to practical
designs
of \SI{28}{GHz} and \SI{140}{GHz} receivers.  
The analysis uses detailed circuit
and signal processing simulations to provide
realistic estimates on the power consumption
and elucidate the main bottlenecks in interference rejection
robustness.

Finally, we perform simple network simulations
at \SI{28}{GHz} and \SI{140}{GHz}
to estimate the frequency of high-power adjacent carrier
signals in likely cellular deployments with multiple carriers.
The simulations consider cases with non-co-located
cell sites,
which introduce the highest level of interference
\cite{rebato2017hybrid}.
