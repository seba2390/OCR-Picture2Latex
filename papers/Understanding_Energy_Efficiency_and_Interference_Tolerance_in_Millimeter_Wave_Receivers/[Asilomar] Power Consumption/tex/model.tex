% \begin{figure*}[!t]
%     \centering
%     % !TEX root = ../top.tex
% !TEX spellcheck = en-US

\begin{figure}[t]
\centering
\includegraphics[width=0.99\linewidth]{./fig/arch/arch.pdf}
\vspace{-3mm}
    \caption{{\bf Our single-stage approach.} We use an encoder-decoder architecture to progressively downsample the image and then to re-expand it. At each level of the decoder, we establish 3D-to-2D correspondences. Finally, we use a RANSAC-based PnP strategy~\cite{Lepetit09} 
    % \WJ{citation?} \YH{Fixed} 
     to infer a single reliable pose from these sets of correspondences. 
    % \WJ{Impossible to read text in printout, please try to have text in the figure with a similar font size as the surrounding article.}\YH{Fixed}
    }
\label{fig:arch}
\end{figure}

%     \caption{High-level architecture of a fully-digital superheterodyne 
%     receiver architecture. The architecture supports $N_\mathrm{rx}$ antennas and
%     $N_\mathrm{str}$ digital streams. The light green boxes represent analog and the 
%     dark-green boxes the digital components. In the RF front-end, 
%     some component are not shown}
%     \label{fig:arch}
% \end{figure*}

\section{System Model} \label{sec:model}

The model is an extension of the analysis framework in \cite{dutta2020capacity,skrimponis2020power,skrimponis2021towards,skrimponis2020efficient},
which studied a single transmitter and receiver. Here, we extend
the model to add interfering transmitters.
The basic set-up is shown in Fig.~\ref{fig:abstract_model}.

As in \cite{dutta2020capacity,skrimponis2020power,skrimponis2021towards,skrimponis2020efficient}, we model the transmissions in discrete-time 
frequency-domain and assume there are $N$
frequency bins.  We divide the bins into two groups:
\begin{itemize}
    \item $I_{\rm sig} \subseteq \{1,\ldots,N\}$ representing the frequency bins on which the desired signal is transmitted; and
    \item $I_{\rm int} = \{1,\ldots,N\} \backslash I_{\rm sig}$ representing the frequency bins on which the interference appears.
\end{itemize}
We assume that each transmitter performs digital beamforming and has 
only one output stream. We denote the frequency domain symbols for the
desired signal by $s_n$, $n\in I_{\rm sig}$
and the interfering signal samples by $v_n$,
$n \in I_{\rm int}$.  As a result, the input to the channel
is the signal,
\begin{equation}
    \nbu = \nbF \herm \nbx, \quad x_n = \begin{cases}
        s_n & \mbox{if } n \in I_{\rm sig} \\
        v_n & \mbox{if } n \in I_{\rm int}
        \end{cases},
\end{equation}
where $\nbF\herm \in \C^{n \times n}$ is the unitary 
IFFT matrix converting the frequency-domain vector
to time-domain. As for the power levels, we assume that the transmitted
symbols, $s_n$, 
are zero-mean complex Gaussian with 
energy per transmitted sample of $E_{\rm sig} = \E|s_n|^2$ and the interference symbols are also complex Gaussian with a frequency-dependent energy per sample, $E_{\rm int} = \E|v_n|^2$.
We will assume there is some reference
thermal noise level $N_0$ and let
\[
    \gamma_{\rm sig} := \frac{E_{\rm sig}}{N_0},
    \quad
    \gamma_{\rm int} := \frac{E_{\rm int}}{N_0},
\]
denote the signal and interference-to-noise ratios.

The samples $u_n$ are passed through a generic channel and receiver radio frequency front-end 
(RFFE) which are jointly modeled as a non-linear, memoryless function $\Phi(u_n, \nbd_n)$, where $\nbd_n$ are i.i.d.\ noise vectors with i.i.d. elements that include thermal noise
and noise from the non-linear components. Finally, the receiver performs digital beamforming using the vector $\nbw$
\begin{equation}
    r_n = \nbw\herm\Phi(u_n, \nbd_n).
\end{equation}