\documentclass[conference]{IEEEtran}
\usepackage[export]{adjustbox}
\usepackage{soul}
\usepackage{colortbl}
\usepackage{xcolor}
\usepackage{tabularx}
\usepackage{hhline}
\usepackage{pgfplots}
\usepackage{pgfplotstable}
\pgfplotsset{compat=1.7}
\usetikzlibrary{dsp,chains,arrows,calc,positioning}
\usetikzlibrary{shapes.multipart, decorations.pathreplacing}
\usepackage{subfigure}
% \usepackage{subfig}
\usepackage{mathtools}
\usepackage{soul} 
\usepackage{amsmath}
\usepackage{amssymb}
\usepackage{xspace}
\usepackage{xifthen}


% Units
\newcommand{\unit}[1]{\ensuremath{\mathrm{\,#1}}\xspace}
\newcommand{\Gyr}{\unit{Gyr}}
\newcommand{\eV}{\unit{eV}}
\newcommand{\keV}{\unit{keV}}
\newcommand{\MeV}{\unit{MeV}}
\newcommand{\GeV}{\unit{GeV}}
\newcommand{\TeV}{\unit{TeV}}
\newcommand{\MB}{\unit{MB}}
\newcommand{\GB}{\unit{GB}}
\newcommand{\TB}{\unit{TB}}
\newcommand{\degree}{\ensuremath{{}^{\circ}}\xspace}
\newcommand{\mas}{\unit{mas}}
\newcommand{\amin}{\unit{arcmin}}
\newcommand{\asec}{\unit{arcsec}}
\newcommand{\angstrom}{\unit{\AA}}
\newcommand{\um}{\unit{$\mu$m}}
\newcommand{\cm}{\unit{cm}}
\newcommand{\km}{\unit{km}}
\newcommand{\kms}{\km \second^{-1}}
\newcommand{\pc}{\unit{pc}}
\newcommand{\kpc}{\unit{kpc}}
\newcommand{\second}{\unit{s}}
\newcommand{\us}{\unit{$\mu$s}}
\newcommand{\photons}{\unit{ph}}
\newcommand{\photon}{\unit{ph}}
\newcommand{\sr}{\unit{sr}}
\newcommand{\Msolar}{\unit{M_\odot}}
\newcommand{\Msun}{\unit{M_\odot}}
\newcommand{\Mstar}{\unit{M_{*}}}
\newcommand{\Lsolar}{\unit{L_\odot}}
\newcommand{\Lsun}{\unit{L_\odot}}
\newcommand{\Lstar}{\unit{L_{*}}}
\newcommand{\Lum}{\ensuremath{ L }\xspace}
\newcommand{\Dsun}{\unit{D_\odot}}
\newcommand{\cmcubes}{\ensuremath{\cm^{3}\second^{-1}}\xspace}
\newcommand{\magn}{\unit{mag}}
%ADW: This is dangerous...
%\renewcommand{\mag}{\magn} 
\newcommand{\mmag}{\unit{mmag}}
\newcommand{\e}{\unit{e^{-}}}
\newcommand{\rms}{\unit{rms}}
\newcommand{\pix}{\unit{pix}}
\newcommand{\rmspix}{\unit{rms/pix}}
\newcommand{\ermspix}{\e \rmspix}
\newcommand{\feh}{{\rm [Fe/H]}}

\newcommand{\teff}{T_{\rm eff}}
%\newcommand{\mas}{\unit{mas}}
\newcommand{\yr}{\unit{yr}}
\newcommand{\masyr}{\unit{\mas \yr^{-1}}}
\usepackage{tikz}
\usepackage{circuitikz}
\usetikzlibrary{dsp,chains,arrows,calc,positioning}
\usetikzlibrary{shapes.multipart, decorations.pathreplacing}

\usetikzlibrary{arrows}
\usepackage{cite}
\usepackage{amsmath,amssymb,amsfonts}
\usepackage{algorithmic}
\usepackage{graphicx}
\usepackage{textcomp}
\usepackage{siunitx}
\usepackage{multirow}
\def\BibTeX{{\rm B\kern-.05em{\sc i\kern-.025em b}\kern-.08em
    T\kern-.1667em\lower.7ex\hbox{E}\kern-.125emX}}

\definecolor{lightGreen}{rgb}{0.5313, 0.7875, 0.7877}
\definecolor{mediumGreen}{rgb}{0.3296, 0.6354, 0.6358}
\definecolor{darkGreen}{rgb}{0.1154, 0.4519, 0.4522}

\definecolor{babyblueeyes}{rgb}{0.63, 0.79, 0.95}

\newcommand{\abbas}[1]{\textcolor{violet}{#1}}

\newcommand{\review}{\color{red}}

\newcommand{\hlc}[2][yellow]{ {\sethlcolor{#1} \hl{#2}} }
\newcommand\notis[1]{\hlc[yellow]{PS: #1}}
\usetikzlibrary{decorations.markings}
\input{lib/basestation}

\newcommand{\STAB}[1]{\begin{tabular}{@{}c@{}}#1\end{tabular}}

\begin{document}

\title{Understanding Energy Efficiency and 
Interference Tolerance 
in Millimeter Wave Receivers}



\author{
    \IEEEauthorblockN{Panagiotis 
    Skrimponis\IEEEauthorrefmark{1}, 
    Seongjoon Kang\IEEEauthorrefmark{1},
    Abbas Khalili\IEEEauthorrefmark{1},
    Wonho Lee\IEEEauthorrefmark{2},
    Navid Hosseinzadeh\IEEEauthorrefmark{2},
   }
    
    \IEEEauthorblockN{
    Marco Mezzavilla\IEEEauthorrefmark{1},
    Elza Erkip\IEEEauthorrefmark{1},
    Mark J. W. Rodwell\IEEEauthorrefmark{2}, 
    James F. Buckwalter\IEEEauthorrefmark{2},
    Sundeep Rangan\IEEEauthorrefmark{1}}
    
    \IEEEauthorblockN{
    \IEEEauthorrefmark{1}NYU Tandon School of Engineering,
    New York University, Brooklyn, NY}
    \IEEEauthorblockN{
    \IEEEauthorrefmark{2}Dept. ECE,  University of
    California, Santa Barbara}
    
    \thanks{P. Skrimponis, S. Kang, A. Khalili, M. Mezzavilla, E. Erkip and S. Rangan are with NYU Wireless, Tandon School of Engineering, New York University, Brooklyn, NY.
    They are supported under
    NSF grants 1952180, 1925079, 1564142, 
    1547332, the Semiconductor Research Corporation (SRC) and the industrial affiliates
    of NYU Wireless.
    }
    

}
\IEEEoverridecommandlockouts

\maketitle

\begin{abstract}
Power consumption is a key challenge
in millimeter wave (mmWave) receiver front-ends,
due to the need to support high dimensional antenna arrays
at wide bandwidths.  
Recently, there has been considerable work in developing 
low-power front-ends, often based on low-resolution ADCs
and low-power mixers.
A critical but less studied consequence of such designs is the
relatively low-dynamic range which in turn exposes the receiver 
to adjacent carrier interference and blockers.
This paper provides a general mathematical framework for 
analyzing the performance of mmWave front-ends
in the presence of out-of-band interference.
The goal is to elucidate the
 fundamental trade-off of power consumption, interference
tolerance and in-band performance.  
The analysis is combined with detailed network simulations
in cellular systems with multiple carriers, as well as detailed circuit
simulations of key components at \SI{140}{GHz}.  The analysis
reveals critical bottlenecks for low-power interference robustness
and suggests designs enhancements  for use in
practical systems.
\end{abstract}

Reinforcement learning has achieved great success in areas such as Game-playing \citep{silver2018general,vinyals2019grandmaster}, robotics \cite{kober2013reinforcement}, large language models \citep{ouyang2022training}, etc.
However, due to safety concerns or physical limitations, in some real-world reinforcement learning problems, we must consider additional constraints that may influence the optimal policy and the learning process \citep{garcia2015comprehensive}.
% For example, a robotic arm must not take actions that may cause harm to itself or the environments.
A standard framework to handle such cases is the constrained Markov Decision Process (CMDP) \citep{altman1999constrained}.
Within the CMDP framework, the agent has to maximize
the expected cumulative reward while
obeying a finite number of constraints, which are usually in the form of expected cumulative cost criteria.

However, we are sometimes concerned with the problem with a continuum of constraints.
For example,
the constraints we meet might be time-evolving or subject to uncertain parameters, which
cannot be formulated as an ordinary CMDP
(see Examples \ref{Example_Time_Evolving} and  \ref{Example_Uncertain}).
In this paper we would study a generalized CMDP  
to address the above problem.  Because the constraints are not only infinite-number but also lie
in a continuous set,
the generalization is not trivial. Fortunately, we find that we can borrow the idea behind semi-infinite programming (SIP) \citep{remez1934determination, hettich1993semi} to deal with the semi-infinite constraints.
Accordingly, we propose \emph{semi-infinitely constrained Markov decision processes} (SICMDPs)
as a novel complement to the ordinary CMDP framework.
%More specifically,  an SICMDP model %, we consider 
%contains a continuum of constraints whereas an ordinary CMDP contains a finite number of constraints. 

%This generalization is natural but not trivial. However, we can brows the idea  
%The idea is quite natural and can be backtracked
%to the practice of extending linear programming to linear semi-infinite programming (LSIP) %\cite{remez1934determination, GobernaLSIO1998}.
%In addition, 
%As a complementary approach to the ordinary CMDP framework, 
%SICMDP can be used to model these problems  which cannot be described by a finite number of constraints
%that are not covered by .
%For example,
%the restrictions we consider can be time-evolving or subject to uncertain parameters
%, thus
%cannot be described by a finite number of constraints but a continuum of constraints 
%(see Examples \ref{Example_Time_Evolving} and  \ref{Example_Uncertain}).

We also present two reinforcement learning algorithms to solve SICMDPs called SI-CRL and SI-CPO, respectively.
SI-CRL is a model-based reinforcement learning algorithm designed for tabular cases, and SI-CPO is a policy optimization algorithm for non-tabular cases.
% and analyze its performance both theoretically and empirically.
The main challenge is that we need to deal with a continuum of constraints, thus reinforcement learning algorithms for ordinary CMDPs do not work anymore.
In SI-CRL, we tackle this difficulty by first transforming the reinforcement learning problem to an equivalent LSIP problem, which can then be solved using methods in the LSIP literature like the dual exchange methods \citep{Hu1990,reemtsen1998numerical}.
In SI-CPO, we resort to the idea of cooperative stochastic approximation developed in \cite{lan2020algorithms, wei2020comirror}.
As far as we know, we are the first to introduce tools from semi-infinitely programming (SIP) into the reinforcement learning community for solving constrained reinforcement learning problems.

% To the best of our knowledge, we are the first to apply tools from semi-infinitely programming (SIP) to solve reinforcement learning problems.
Furthermore, we give theoretical analysis for both SI-CRL and SI-CPO.
We decompose the error of SI-CRL into two parts: the statistical error from approximating the true SICMDP with an offline dataset and the optimization error due to the fact that the solution of the LSIP problem obtained by the dual exchange method is inexact.
On the optimization side, we show that the iteration complexity of SI-CRL is $O\left(\left\{\mathrm{diam}(Y)L\sqrt{|\gS|^2|\gA|m}/\left[(1-\gamma)\epsilon\right]\right\}^m\right)$.
On the statistical side, we show that the sample complexity of SI-CRL is $\widetilde O\left(\frac{|S|^2|A|^2}{\epsilon^2(1-\gamma)^3}\right)$ if the offline dataset is generated by a generative model, and $\widetilde O\left(\frac{|S||A|}{\nu_{\min} \epsilon^2(1-\gamma)^3}\right)$ if the dataset is generated by a probability measure $\nu$ as considered in \cite{chen2019information}.
Here $\widetilde O$ means that all logarithm terms are discarded.
For SI-CPO, things become a little more complicated because other than the statistical error and the optimization error, we also need to consider the function approximation error, which comes from imperfect policy parametrizations.
It is shown if the function approximation error can be controlled to $O(\epsilon)$ order, the iteration complexity of SI-CPO is $\widetilde{O}\left(\frac{1}{\epsilon^2(1-\gamma)^6}\right)$ and the sample complexity of SI-CPO is $\widetilde{O}(\frac{1}{\epsilon^4(1-\gamma)^{10}})$.
Here our iteration complexity bound is equivalent to a typical $\widetilde O(1/\sqrt{T})$ global convergence rate.

We perform a set of numerical experiments to illustrate the SICMDP model and validate our proposed algorithms.
Specifically, we examine two numerical examples, namely the discharge of sewage and ship route planning.
Through the discharge of sewage example, we show the advantage of the SICMDP framework over the CMDP baseline obtained by naive discretization in modeling realistic sequential decision-making problems.
Moreover, we demonstrate the effectiveness of the SI-CRL and SI-CPO algorithms in such tabular environments. 
In the ship route planning example, we illustrate the benefits of the SICMDP framework and the ability of the SI-CPO algorithm to address complex continuous control tasks involving continuous state spaces with modern deep reinforcement learning techniques.

% In summary, our contributions are listed as follows.
% First, we present the SICMDP model, which can be viewed as a generalization of the ordinary CMDP model.
% Second, we propose an algorithm to perform reinforcement learning for SICMDPs, which is called SI-CRL, and we believe that we are the first to apply tools from SIP
% to solve reinforcement learning problems.
% Third, we give a theoretical analysis of SI-CRL and identify both its sample complexity and iteration complexity.
% In addition, we perform numerical experiments to illustrate the SICMDP model and validate the SI-CRL algorithm.
% \{This paragraph can be removed!!! \}





\section{The \MakeLowercase{i}W\MakeLowercase{inr}NFL model}
\label{sec:model}

In this section we are going to present the data we used to develop our in-game probability model as well as the design details of {\method}. 

{\bf Data: }In order to perform our analysis we utilize a dataset collected from NFL's Game Center for all the regular season games between the seasons 2009 and 2016. 
We access the data using the Python {\tt nflgame} API \cite{nflgame}. 
The dataset includes detailed play-by-play information for every game that took place during these seasons. 
This information is used to obtain the state of the game that will drive the design of {\method}. 
In total, we collected information for 2,048 regular season games and a total of 338,294 snaps/plays. 

{\bf Model: }
{\method} is based on a logistic regression model that calculates the probability of the home team winning given the current status of the game as: 

\begin{equation}
\Pr(H=1| \mathbf{x})= \frac{\exp(\mathbf{\weight}^T\cdot\mathbf{x})}{1+\exp(\mathbf{\weight}^T\cdot\mathbf{x})}
\label{eq:reg}
\end{equation}
where $H$ is the dependent random variable of our model representing whether the home team wins or not, $\mathbf{x}$ is the vector with the independent variables, while the coefficient vector $\mathbf{\weight}$ includes the weights for each independent variable and is estimated using the corresponding data.  
For a game of infinite duration a linear model could be a very good approximation.  
However, the boundary effects from the finite duration of a game create several non-linearities \cite{winston2012mathletics}.  
For this reason, we enhance our model - using the same set of features - with a Support Vector Machine classifier with radial kernel for the last three minutes of regulation.  
In order to obtain a probability output from the SVM classifier, we further use Platt's scaling \cite{platt1999probabilistic}: 

\begin{equation}
\Pr(H=1| \mathbf{x})= \frac{1}{1+\exp{(Af(x)+B)}}
\label{eq:platt}
\end{equation}
where $f(x)$ is the uncalibrated value produced by the SVM classifier: 

\begin{equation}
f(x) = \sum_{i} (\alpha_i y_i k(\mathbf{x}_i\cdot\mathbf{x}))+ b
\label{eq:svm}
\end{equation}
where $k(\mathbf{x},\mathbf{x}')$ is the kernel used for the SVM.   
Figure \ref{fig:iwinrNFL} depicts the simple flow chart of {\method}. 


\begin{figure}[t]
\begin{center}
\includegraphics[scale=0.35]{plots/iwinrNFL.pdf}%\vspacecap
 \caption{{\method} includes a linear and a non-linear component.}
 \label{fig:iwinrNFL}
\end{center}
\end{figure}

In order to describe the status of the game we use the following variables:

\begin{enumerate}
\item {\bf Ball Possession Team:} This binary feature captures whether the home or the visiting team has the ball possession
\item {\bf Score Differential:} This feature captures the current score differential (home - visiting)
\item {\bf Timeouts Remaining:} This feature is represented by two independent variables - one for the home and one for the away team - and they capture the number of timeouts remaining for each of the teams
%\item {\bf Quarter:} This feature captures the current quarter of the game
%\item {\bf Time Remaining:} This feature captures the time (in seconds) remaining for the current quarter to end
\item {\bf Time Elapsed: } This feature captures the time elapsed since the beginning of the game
\item {\bf Down:} This feature represents the down of the team in possession
\item {\bf Field Position:} This feature captures the distance covered by the team in possession from their own yard line
\item {\bf Yards-to-go:} This variables represents the number of yards needed for a first down
\item {\bf Ball Possession Time: } This variable captures the time that the offensive unit of the home team is on the field 
\item {\bf Ranking Differential: } This variable represents the difference of the win percentage for the two team (home - visiting)
\end{enumerate}

The last independent variable is representative of the power ranking difference between the two teams. 
Most of the existing models that include such a variable are using the Vegas line spread for each game.  
We choose not to do so for the following reason.  
The objective of the Vegas line is not to predict game outcomes but rather distribute money across the different bets.  
Exactly because of this objective the line is changing during the week before the game.  
While this line can change due to new information for the competing teams (e.g., injury updates), the line is mainly changing when a particular team has accumulated the majority of the bets. 
In this case it will also be hard to choose which line to use (e.g., the opening, the closing or some average of them).  
Therefore, we choose to use the win percentage differential of the two teams as an indicator of their strength (even though this has its own issues given the uneven schedule in NFL).  
However, note that if one would like to use the point spread as a variable this can be easily incorporated in the model. 
Table \ref{tab:iwinrnfl} presents the coefficients of the logistic regression model of {\method} with standardized independent variables for better comparisons. 


\begin{table}[ht]
\begin{center}
\def\sym#1{\ifmmode^{#1}\else\(^{#1}\)\fi}
\begin{tabular}{l*{1}{c}}
\toprule
                    &\multicolumn{1}{c}{(1)}\\
                    &\multicolumn{1}{c}{Winner}\\
\midrule
Possession Team (H)         &      0.41\sym{***}\\
                    &     (49.19)         \\
\addlinespace
Score Differential           &      3.59\sym{***}\\
                    &    (247.34)         \\
\addlinespace
Home Timeouts           &     0.12\sym{***}\\
                    &      (8.74)         \\
\addlinespace
Away Timeouts           &     -0.11\sym{***}\\
                    &    (-12.47)         \\
\addlinespace
Ball Possession Time  &     -0.05.\\
                    &    (-1.66)         \\
\addlinespace
Time Lapsed       &   -0.05.\\
                    &      (-1.66)         \\
\addlinespace
Down                &   -0.01         \\
                    &      (0.04)         \\
\addlinespace
Field Position            &   0.02\sym{**} \\
                    &      (2.71)         \\
\addlinespace
Yards-to-go                &  -0.01         \\
                    &      (0.23)         \\
\addlinespace
Rating differential         &       0.75\sym{***}\\
                    &     (80.47)         \\
\addlinespace
Intercept            &       0.57\sym{*}\\
                    &    (2.09)         \\
\midrule
Observations        &      338,294         \\
\bottomrule
\multicolumn{2}{l}{\footnotesize \textit{t} statistics in parentheses}\\
\multicolumn{2}{l}{\footnotesize \sym{$_.$} \(p<0.1\), \sym{*} \(p<0.05\), \sym{**} \(p<0.01\), \sym{***} \(p<0.001\)}\\
\end{tabular}
\end{center}
\caption{Standardized logisitic regression coefficients for {\method}.}
\label{tab:iwinrnfl}
\end{table}


As we can see, as one might have expected the current scoring differential exhibits the strongest correlation with the in-game win probability.  
The only factors that do not appear to be statistically significant predictors of the dependent variable are the down and the yards-to-go. 
Even though the corresponding coefficients are negative as one might have expected (e.g., being at an earlier down gives you more chances to advance the ball), they are not significant in estimating the win probability. 
On the contrary, all else being equal timeouts appear to be quiet important since they can help a team stop the clock, while teams with better win percentage appear to have an advantage as well, since this can be a sign of a better team. 
In the following section we provide a detailed evaluation of {\method}.
\begin{figure}[t]
    \centering
    \includegraphics[width = 0.99\linewidth]{fig/fig_analitic_model-eps-converted-to.pdf}
    % \begin{tikzpicture}
    %     \begin{axis}[restrict z to domain=0:inf,restrict x to domain=0:inf,restrict y to domain=0:inf]
    %     \addplot3[surf] {y/(1+y+0.5*x)};
    %     \end{axis}
    % \end{tikzpicture}
    \caption{Input-output SNR relation.}
    \label{fig:snr_inout}
\end{figure}

\begin{figure*}[!t]
    \centering
    % !TEX root = ../top.tex
% !TEX spellcheck = en-US

\begin{figure}[t]
\centering
\includegraphics[width=0.99\linewidth]{./fig/arch/arch.pdf}
\vspace{-3mm}
    \caption{{\bf Our single-stage approach.} We use an encoder-decoder architecture to progressively downsample the image and then to re-expand it. At each level of the decoder, we establish 3D-to-2D correspondences. Finally, we use a RANSAC-based PnP strategy~\cite{Lepetit09} 
    % \WJ{citation?} \YH{Fixed} 
     to infer a single reliable pose from these sets of correspondences. 
    % \WJ{Impossible to read text in printout, please try to have text in the figure with a similar font size as the surrounding article.}\YH{Fixed}
    }
\label{fig:arch}
\end{figure}

    \caption{High-level architecture of a fully-digital superheterodyne 
    receiver architecture. The architecture supports $N_\mathrm{rx}$ antennas and
    $N_\mathrm{str}$ digital streams. The light green boxes represent analog and the 
    dark-green boxes the digital components. In the RF front-end, 
    some component are not shown.}
    \label{fig:arch}
\end{figure*}

\section{Capacity Bound and Output SNR}
\label{sect:capacity}
Our goal is to characterize the performance of the discussed model in Sec.~\ref{sec:model} in terms of the spectral efficiency. To this end, similar to \cite{skrimponis2020efficient}, we make use of the concepts of the output SNR and input-output SNR relation described next. 

From Sec.~\ref{sec:model}, we have 
\begin{align}
    \label{eq:inoutre}
    \hat{\nbx} = \nbF\Phi(\nbF \herm \nbx, \nbD).
\end{align}

Assuming that the variables $\hat{x}_n$, $\nbd_n$, and $x_n$ have an underlying statistical model and are distributed as $\hat{X}$, $D$, and $X$, respectively, we can use the Bussgang-Rowe decomposition \cite{bussgang1952crosscorrelation,rowe1982memoryless} and model the non-linearity in the system (i.e., $\Phi(\cdot)$) as multiplying a scalar with its input and adding a noise which is uncorrelated with the input. More precisely, we can write
\begin{align}
\label{eq:lin_mod}
    \hat{X} = A X+ T,\quad \Exp |T|^2 &= \tau,
\end{align}
where
\begin{align}
\label{eq:alpha_tau_rx}
    A := \frac{\Exp\left[ \hat{X}^* X\right]}{\Exp\left|X\right|^2 }, \quad \tau :=  \Exp|\hat{X}- A X|^2,
    % \alpha_2 :=  \frac{1}{\overline{P}}\Exp|S-\alpha_{\rm rx}U|^2,
\end{align} 
with $X^*$ denoting the complex conjugate of $X$.

% In \cite{dutta2020capacity}, we have proved rigorously that in the system shown in Fig.~\ref{fig:abstract_model}, for many different functions $\Phi(\cdot)$, the elements of $\wbf$ can be viewed as i.i.d. $\CN(0, \tau)$.
In general, both $A$ and $\tau$ are functions of the 
input SNR $\gamma_\mathrm{in}$ 
so we may write $A = A(\gamma_{\mathrm{in}})$
and $\tau = \tau(\gamma_{\mathrm{in}})$. 
% Based on \cite[Theorem.~1]{dutta2020capacity} 
From \eqref{eq:lin_mod}, we can then define the 
{\it output SNR} off the desired signal $\nbs$ as,
\begin{equation} \label{eq:inout_snr}
    \gamma_{\mathrm{out}} = G(\gamma_{\mathrm{in}}) :=
    \frac{|A(\gamma_{\mathrm{in}})|^2}{\tau(\gamma_{\mathrm{in}})} 
    \gamma_{\rm sig}.
\end{equation}
This is the SNR that would be seen in attempting
to recover the input transmitted vector $\nbs$ from the
output vector $\hat{\nbs}$.  

Using the SNR enables us to bound the performance of the system in terms of the capacity. More precisely, using same steps as of \cite[Appendix~A]{skrimponis2020efficient}, we can show that the capacity of the system  can is lower bounded as,
\begin{equation}  \label{EQ:CAP_BAND_LIMITED1}
  C \geq  \frac{N_{\rm sig}}{N} f_s\log_2\left(1 +  \gamma_{\rm out}\right),
\end{equation}
where $f_s$ is the sample rate and $N_{\rm sig} = |I_{\rm sig}|$ represent the number of frequency bins for the signal. Moreover, assuming that the ADC performs oversampling with the ratio $\zeta$, using same steps as of \cite[Appendix~B]{skrimponis2020efficient}, we have

\begin{equation}  \label{EQ:CAP_BAND_LIMITED}
  C \geq  \frac{N_{\rm sig}\zeta}{N} f_s\log_2\left(1 +  \frac{\gamma_{\rm out}}{\zeta}\right),
\end{equation}
% where $f_s$ is the sample rate and $N_{\rm sig} = |I_{\rm sig}|$ represent the number of frequency bins for the signal. 


\begin{table}[t]
    \centering
    \caption{Parameters of the \SI{28}{GHz} RFFE devices used in the analysis.}

    \setlength{\tabcolsep}{3pt}
    \begin{tabular}{|>{\raggedright}m{0.9in}|c|c|c|c|}
    \hline
    
    \textbf{Parameter} &
    \textbf{LNA$^{\boldsymbol{(1)}}$} &
    \textbf{LNA$^{\boldsymbol{(2)}}$} &
    \textbf{Mixer$^{\boldsymbol{(1)}}$} &
    \textbf{Mixer$^{\boldsymbol{(2)}}$}
    \tabularnewline \hline
    Design [\textmu m] & 10 & 5 & 2 & 5
    \tabularnewline
    Noise Figure [dB] & 2.13 & 2.53 &  9.039 & 7.542
    \tabularnewline
    Gain [dB] & 14.26 & 12.85 &  0.16 &  3.558
    \tabularnewline
    IIP3 [dBm] & -1.456 & 0.603  & -3.1  & 2.1
    \tabularnewline
    Power [mW] &8.91 & 5.34 &   4.838 &   7.03
    \tabularnewline \hline
    \end{tabular}
    \label{tab:rffe28}
\end{table}

% New models
% 2 & 2
% 10.115 & 9.039
% & 
% 
% 4.8 & 4.8



In this paper, we will show through detailed simulations that the input-output SNR relation can be approximated in the form of
\begin{equation} \label{eq:gamout}
    \hat{\gamma}_{\rm out} = \frac{\beta \gamma_{\rm sig}}{
    1 + \alpha_1 \gamma_{\rm sig} + \alpha_2\gamma_{\rm int}},
\end{equation}
    where $\gamma_{\rm int} = \frac{1}{|I_{\rm int}|}\sum_{n \in I_{\rm int}}E_i[n]$.
for three parameters $\beta$ and $\alpha_1$ and $\alpha_2$
using which we can evaluate the receiver front-end performance. Intuitively, this formula suggest that due to the non-linearity in the system: (i) the signal energy is reduced; (ii) a ratio of the signal is distorted; (iii) a ratio of the adjacent band signal (i.e., interference) is leaked to the desired band.

From \eqref{eq:gamout}, we also observe that the output SNR saturates to the value of $\frac{\beta}{\alpha_1}$ as the input signal SNR increases. Furthermore, for higher values of the interference signal the saturation should accrue for lower values of the input SNR. One can also observe these from Fig.~\ref{fig:snr_inout} which illustrates \eqref{eq:gamout} for a fixed values of $\beta$, $\alpha_1$ and $\alpha_2$ and different values of desired and interference signal powers.
% signal-to-distortion-plus-interference-noise-and-ratio
% The trade offs between the received signals power,  




% To interpret the roles of the parameters
% Fig.~\ref{fig:snr_inout} plots
% $\gamma_{\mathrm{out}}$ vs.\ $\gamma_{\mathrm{sig}}$ and $\gamma_{\mathrm{int}}$ with both the values in dB-scale.  We see two distinct
% regimes.

% \paragraph*{Low input SNR regime}
% When $N_\mathrm{rx} \gamma_\mathrm{in}/F \ll \gamma_\mathrm{sat}$,
% the output SNR in \eqref{eq:inout_snr_nonlin} simplifies to
% \begin{equation} \label{eq:inout_snr_nl_low}
%     \gamma_{\mathrm{out}} = G(\gamma_{\mathrm{in}}) 
%     \approx\frac{N_\mathrm{rx}}{F}\gamma_{\mathrm{in}},
% \end{equation}
% which matches the linear case \eqref{eq:inout_snr_lin}.
% We will thus call $F$ in \eqref{eq:inout_snr_nonlin} the
% {\it effective noise figure}.  This regime
% appears on the left side of Fig.~\ref{fig:snr_inout}
% where $\gamma_\mathrm{out}$ (in dB) 
% increases linearly  with $\gamma_\mathrm{in}$ with an 
% offset given by beamforming gain $N_\mathrm{rx}$
% minus the effective noise figure $F$.

% We will see below that the low input SNR regime
% occurs when the RFFE components have sufficiently small
% input levels that they do not saturate and act linearly.
% In this case, the effective noise figure
% is determined by the standard noise figure analysis 
% with an additional term from the quantization noise at the ADC.

% \paragraph*{High input SNR regime}
% At very large input SNRs 
% ($N_\mathrm{rx} \gamma_\mathrm{in}/F \gg \gamma_\mathrm{sat}$),
% the output SNR in \eqref{eq:inout_snr_nonlin} simplifies to
% \begin{equation} \label{eq:inout_snr_nl_high}
%     \gamma_{\mathrm{out}} \approx G(\gamma_{\mathrm{in}}) 
%     \approx \gamma_\mathrm{sat}.
% \end{equation}
% Hence, the output SNR saturates as shown 
% in Fig.~\ref{fig:snr_inout}.  It is for this
% reason that $\gamma_\mathrm{sat}$ is called the saturation SNR.  
% In this case, we will see that the saturation SNR is determined
% by the nonlinearities in the devices and the quantization
% in the ADC.  

% \textcolor{red}{Abbas to do:
% \begin{itemize}
%     \item State the result on the Gaussianity 
%     \item State the capacity lower bound
%     \item Define output SNR
%     \item State the analytic model for the output SNR.
% \end{itemize}
% }
% We model the output SNR as:
% \begin{equation} \label{eq:gamout}
%     \gamma_{\rm out} = \frac{\beta \gamma_{\rm sig}}{
%     1 + \alpha_1 \gamma_{\rm sig} + \alpha_2\gamma_{\rm int}}.
% \end{equation}




\begin{table}[t]
    \centering
    \caption{Parameters of the \SI{140}{GHz} RFFE devices used in the analysis.}

    \setlength{\tabcolsep}{3pt}
    \begin{tabular}{|>{\raggedright}m{0.9in}|c|c|c|c|}
    \hline
    
    \textbf{Parameter} &
    \textbf{LNA$^{\boldsymbol{(1)}}$} &
    \textbf{LNA$^{\boldsymbol{(2)}}$} &
    \textbf{Mixer$^{\boldsymbol{(1)}}$} &
    \textbf{Mixer$^{\boldsymbol{(2)}}$}
    \tabularnewline \hline
    Design [\textmu m] & 4 &  2-4 & 1 & 1
    \tabularnewline
    Noise Figure [dB] & 7.50 & 7.48 & 21.53 & 20.47
    \tabularnewline
    Gain [dB] & 11.13 & 16.56 & -1.74 & -0.52
    \tabularnewline
    IIP3 [dBm] & -9.15 & -8.90 & -4.45 & -3.88
    \tabularnewline
    Power [mW] &  4.80 & 15.90 & 5.00 & 5.00
    \tabularnewline \hline
    \end{tabular}
    \label{tab:rffe140}
\end{table}


\section{Link-Layer Simulation} \label{sec:link}

% \textcolor{red}{
% Notis to do:
% \begin{itemize}
%     \item Create a block diagram for the RFFE.  You can use the one from the IEEE Access paper.
%     \item Describe the interference and signal model.
%     \item Describe the design options and why we are analyzing them. 
%     \item Describe the fitting procedure using an initial fit followed by non-linear least squares
%     \item Show plots of SNR out vs.\ signal and interfering SNR.
%     \item Describe when the model is a good fit.
%     \item Summarize fitted parameters in a table.
% \end{itemize}}

The model in Section~\ref{sec:model} is a simplified
abstraction of an actual RFFE.  In this section, 
we validate the model and extract the parameters 
for \eqref{eq:gamout} with realistic, circuit simulations
of potential RFFEs at \SI{28}{GHz} and \SI{140}{GHz}.


\subsection{Signal and interference model}
We consider a downlink scenario in a communication link 
between an NR basestation (gNB) and a mobile device (UE). 
For each slot, the gNB generates a physical downlink shared
channel (PDSCH) that includes both information and control 
signals. The receiver uses the demodulation reference 
signal (DM-RS) for practical channel estimation. 
To compensate the common phase error (CPE) the 3GPP 5G NR,
standard introduce the phase tracking reference signal (PT-RS). 
The receiver performs coherent CPE estimation using the algorithm 
described in~\cite{syrjala2019phase}. For time synchronization,
between the gNB and UE we utilize the primary (PSS) and the secondary synchronization signals (SSS).

To model the interference, we assume the presence of another gNB
that generates i.i.d. $\CN(0, 1)$ symbols in frequency-domain. The
symbols are modulated with OFDM to generate interference in an
adjacent band. Even though the signals from the two gNBs are
originally independent in the frequency domain, the presence of 
the non-linear processing introduces distortion to the main signal
from the adjacent band. As explained in Section~\ref{sec:model},
this increase the total received energy and causes the RFFE 
saturation point to happen much earlier.

%
% 140 GHz - Mixer (1) 
% Gain: -1.7360
%   IIP3: -4.4546
%     NF: 21.5290
%   PLO: -11
%  Power: 5
%
% 140 GHz - Mixer (2) 
%   Gain: -0.5185
%   IIP3: -3.8883
%     NF: 20.4715
%   PLO: -6
%  Power: 5
     
\begin{figure}[t]
    \centering
    \includegraphics[width = 0.99\linewidth]{fig/model_fit-eps-converted-to.pdf}
    \caption{Evaluation of the approximation model in \eqref{eq:gamout} for Design$^{(1)}$ at \SI{28}{GHz} for different input interference power.}
    \label{fig:model_fit}
\end{figure}



\subsection{Receiver configurations}
Similarly to~\cite{skrimponis2021towards} we consider a fully-digital superheterodyne receiver architecture as shown in Fig.~\ref{fig:arch}. The receiver has $N_\mathrm{rx}=16$ or $64$ antennas with independent RFFE processing. The received signal is amplified with a low-noise amplifier  (LNA), downconverted with an intermediate frequency (IF) mixer, amplified with an automated gain control (AGC) amplifier before the direct conversion mixer, and finally quantized with a pair of ADCs. The system use filters in the RF and IF domain to improve the image rejection and the dynamic range of the system. The actions of the RFFE devices for each receiver configuration is modeled with a different non-linear function $\Phi(\cdot)$.

In~\cite{skrimponis2021towards} the authors design and evaluate RFFE devices at \SI{140}{GHZ} based on a \SI{90}{nm} SiGE BiCMOS HBT technology. They focus on minimizing the power consumption of the RFFE devices at a certain performance in terms of gain, noise figure (NF), and the input third order intercept point (IIP3). The LNA designs vary in terms of number of stages, topology, and transistor size. While the proposed double-balanced active mixers are all based on the conventional Gilbert-cell design, they vary in transistor size. The mixer performance characteristics depend on the input power from the local oscillator (LO). To further optimize the power consumption of the receiver, the authors propose a novel LO distribution model. Specifically, the mixers in the same tile share a common LO driver using power dividers and amplifiers. They provide a method to determine the best configuration of LO drivers that achieve the same performance for a minimum power.

Using the components and the optimization framework described in~\cite{skrimponis2021towards} we select two of the optimized designs for the \SI{140}{GHz} systems in our analysis. These two designs achieve similar performance to the state-of-the-art design discussed in \cite{skrimponis2020power} while achieving a significant improvement in power consumption. Similarly, for the \SI{28}{GHz} devices we design two common base emitter base collector (CBEBC) LNAs, and two active mixers based on the common Gilbert cell design. We use the optimization framework in~\cite{skrimponis2021towards} to determine the number of LO drivers. For both frequencies, Design$^{(1)}$ use 4-bit ADCs while Design$^{(2)}$ use 5-bit ADC pairs. This assumption is based on prior works \cite{abbas2017millimeter,zhang2018low,abdelghany2018towards,dutta2019case} indicating that 4 bits are sufficient for the majority of cellular data and control operations. Based on a flash-based 4-bit ADC in \cite{Nasri2017}, we consider the ADC $\mathrm{FOM} = 65\;\mathrm{fJ/conv}$. We summarize the parameters for the devices at \SI{28}{GHz} and \SI{140}{GHz} in Table~\ref{tab:rffe28} and Table~\ref{tab:rffe140} respectively. 

% \textcolor{blue}{[SR:  Can someone provide
% a brief description of where the 28 GHz
% designs are coming from?
% Also, in Tables I and II (or somewhere else),
% can we put the total poewr consumption for
% the Designs 1 and 2 for the 28 and 140 GHz
% receivers.  ]}

\begin{table}[t]
    \centering
    \caption{Model and system parameters for the receiver designs at \SI{28}{GHz} and \SI{140}{GHz}.}
    \label{tab:model_param}
    \begin{tabular}{|>{\raggedright}m{0.5in}|c|c|c|c|}
        \hline
         \multirow{2}{*}{\textbf{Parameter}} & \multicolumn{2}{c|}{\textbf{\SI{28}{GHz}}} & \multicolumn{2}{c|}{\textbf{\SI{140}{GHz}}}
        \tabularnewline \cline{2-5}
        & $\text{Design}^{(1)}$ & $\text{Design}^{(2)}$ & $\text{Design}^{(1)}$ & $\text{Design}^{(2)}$
        
        \tabularnewline \hline
        \multicolumn{1}{|c|}{$\beta$} & 1.3865 & 1.2725 & 0.3099 & 0.1862
        \tabularnewline 
        
        \multicolumn{1}{|c|}{$\alpha_1$} & 0.0090 & 0.0024 & 0.0021 & 0.0004
        \tabularnewline 
        
        \multicolumn{1}{|c|}{$\alpha_2$} & 0.0058 & 0.0017 & 0.0014 & 0.0003
        \tabularnewline \hline
        
        \multicolumn{1}{|c|}{RX antennas} & 16 & 16 & 64 & 64
        \tabularnewline
        
        \multicolumn{1}{|c|}{NF [dB]} & 2.78 & 3.08 & 9.40 & 11.50
        \tabularnewline
        
        \multicolumn{1}{|c|}{Power [mW]} & 411 & 404 & 1682 & 1355
        \tabularnewline \hline
    \end{tabular}
\end{table}

\subsection{Model fitting}
For each receiver design we fit a model in form of
\eqref{eq:gamout}. Since this nonlinear function 
$\Phi(\cdot)$ depends on the parameters $\alpha_1,\alpha_2$, and
$\beta$ we can write the estimated output-SNR model as $\hat{\gamma}_\mathrm{out}(\gamma_\mathrm{sig}, \gamma_\mathrm{int}; \alpha_1, \alpha_2, \beta)$. For the \emph{initial 
heuristic} fit we set, 
\begin{align}
    \beta = \frac{1}{F}, \quad \alpha_1 = \alpha_2 = \frac{1}{\gamma_\mathrm{sat}F},
    \label{eq:fitinit}
\end{align}
where $F$ is the noise factor of the system, and $\gamma_\mathrm{sat}$
the saturation SNR in linear scale. % Note that these parameters 
%do not include any beamforming gain. %We use the downlink scenario  
%described in Section~\ref{sec:link} to obtain $\gamma_\mathrm{out}$ using \eqref{eq:inout_snr}.
We then  optimize the fit  using the non-linear least squares regression method and optimize a problem of the following form,
\begin{align}
   Q(\gamma_\mathrm{sig}, \gamma_\mathrm{int}) := \min_{\alpha_1, \alpha_2, \beta}\|& \gamma_\mathrm{out}(\gamma_\mathrm{sig}, \gamma_\mathrm{int}) \nonumber \\ &
    - \hat{\gamma}_\mathrm{out}(\gamma_\mathrm{sig}, \gamma_\mathrm{int}; \alpha_1, \alpha_2, \beta)) \|_2^2, \nonumber\\
\end{align}
where  $\gamma_\mathrm{out}$ are the measurements from the link-layer simulation using \eqref{eq:inout_snr}, and $\hat{\gamma}_\mathrm{out}$ is the estimate using the model in
\eqref{eq:gamout}. The optimized parameters for the \SI{140}{GHz} and \SI{28}{GHz} receiver
designs are summarized in Tab.~\ref{tab:model_param}. In Fig.~\ref{fig:model_fit} we show that the model in~\eqref{eq:gamout} provides a very good fit. In particular, we see the linear regime for low-input SNR and the saturation for high-input signal power. We show that as the input interference increases the model the saturation SNR is also changing.


\section{Network-Level Simulation} \label{sec:network}
%\textcolor{red}{
%SJ to do:
%\begin{itemize}
%    \item Describe the network topology with the two carriers
%    \item Create a table with all the parameters
%    \item Describe the channel model
%    \item Describe the arrays, how you performed cell association
%    \item Describe how you computed the SNR
%    \item Show the SNR and rate CDFs
%\end{itemize}}
The above analysis shows that the performance of the RFFE
degrades in the presence of strong out-of-band
interference when the front-end dynamic range is low.
This fact raises a basic question:  \emph{how often
is the adjacent carrier interference strong in practical 
systems?}  In this section, we perform a simple network simulation
to assess the effect of adjacent carrier interference in 
a downlink cellular system with two 
adjacent carriers, carriers A and B.

We take account of a wrap-around \SI{1}{km}~$\times$~\SI{1}{km} network area to conduct the network-level simulation. Subject to a inter-site distance (ISD), gNBs are deployed by homogeneous Poisson point process (HPPP) with density $\lambda = \frac{4}{\pi\times\text{ISD}^2}$, and accordingly, the same number of UEs are uniformly distributed. Furthermore, all gNBs and UEs are 
randomly assigned to carrier A or B.
We use the notation $\text{gNB}_{A}$ and $\text{gNB}_{B}$ for base stations, and  $\text{UE}_{A}$ and $\text{UE}_{B}$ for UEs.
Individual gNBs are multi-sectorized with $8\times8$ uniform rectangular arrays (URA) per sector which is tilted down by $-12^\circ$, while each UE is equiped with a URA. The full 
parameter settings are shown in Table~\ref{tab:sim_param}. 

\begin{figure}[t]
    \centering
    

\begin{tikzpicture}[every node/.append style={scale=1.0}]

    \definecolor{matlabplot0}{rgb}{0.1154, 0.4519, 0.4522}
    \definecolor{matlabplot1}{rgb}{0.5313, 0.7875, 0.7877}
    
    \tikzset{BS style/.style={
        draw,
        minimum width=1cm,
        minimum height=4cm,
        fill=gray,
        % fill2=green,
        inner sep=0,
        outer sep=0,
        scale=0.75
    }}
    \tikzset{UE Style/.style={
        draw,
        fill=gray,
        minimum width=1.2cm,
        minimum height=2cm,
        scale=0.5
    }}

    %% Draw UE, gNB
    \path (1.7, 0) coordinate (UE_A_pos);
    \path (4.3, 0) coordinate (UE_B_pos);
    \path (0, 3) coordinate (BS_A_pos);
    \path (6, 3) coordinate (BS_B_pos);
    \draw (BS_A_pos) node[BS style,BaseStationSimple] (BS_A) {};
    \draw (BS_B_pos) node[BS style,BaseStationSimple] (BS_B) {};
    \draw (UE_A_pos) node[UserTerminal,UE Style, button fill color=gray!40] (UE_A) {};
    \draw (UE_B_pos) node[UserTerminal,UE Style, button fill color=gray!40] (UE_B) {};
  
  
    %% Draw the beams for gNB_A
    \def\pos{-65.5}
    % \draw (BS_A.antenna center) node[Beam,minimum width=0.75cm,minimum height=2.5mm,rotate=\pos-55,fill=matlabplot0!30] () {};
    % \draw (BS_A.antenna center) node[Beam,minimum width=0.75cm,minimum height=2.5mm,rotate=\pos+55,fill=matlabplot0!30] () {};
    % \draw (BS_A.antenna center) node[Beam,minimum width=1cm,minimum height=3.25mm,rotate=\pos-25,fill=matlabplot0!60] () {};
    % \draw (BS_A.antenna center) node[Beam,minimum width=1cm,minimum height=3.25mm,rotate=\pos+25,fill=matlabplot0!60] () {};
    % \draw (BS_A.antenna center) node[Beam,minimum width=2cm,minimum height=4mm,rotate=\pos,fill=matlabplot0!80] () {};
    
    \draw (BS_A.antenna center) node[Beam,minimum width=0.5cm,minimum height=2mm,rotate=\pos-55,fill=matlabplot0!30] () {};
    \draw (BS_A.antenna center) node[Beam,minimum width=0.5cm,minimum height=2mm,rotate=\pos+55,fill=matlabplot0!30] () {};
    \draw (BS_A.antenna center) node[Beam,minimum width=0.75cm,minimum height=2.5mm,rotate=\pos-25,fill=matlabplot0!60] () {};
    \draw (BS_A.antenna center) node[Beam,minimum width=0.75cm,minimum height=2.5mm,rotate=\pos+25,fill=matlabplot0!60] () {};
    \draw (BS_A.antenna center) node[Beam,minimum width=1.5cm,minimum height=3mm,rotate=\pos,fill=matlabplot0!80] () {};
  
    %% Draw the beams for gNB_B
    \definecolor{gnbB_color}{rgb}{0.82, 0.1, 0.26}
    \def\pos{-114.5}
    % \draw (BS_B.antenna center) node[Beam,minimum width=0.75cm,minimum height=2.5mm,rotate=\pos-55,fill=gnbB_color!30] () {};
    % \draw (BS_B.antenna center) node[Beam,minimum width=0.75cm,minimum height=2.5mm,rotate=\pos+55,fill=gnbB_color!30] () {};
    % \draw (BS_B.antenna center) node[Beam,minimum width=1cm,minimum height=3.25mm,rotate=\pos-25,fill=gnbB_color!60] () {};
    % \draw (BS_B.antenna center) node[Beam,minimum width=1cm,minimum height=3.25mm,rotate=\pos+25,fill=gnbB_color!60] () {};
    % \draw (BS_B.antenna center) node[Beam,minimum width=2cm,minimum height=4mm,rotate=\pos,fill=gnbB_color!80] (beam_1) {};
    \draw (BS_B.antenna center) node[Beam,minimum width=0.5cm,minimum height=2mm,rotate=\pos-55,fill=gnbB_color!30] () {};
    \draw (BS_B.antenna center) node[Beam,minimum width=0.5cm,minimum height=2mm,rotate=\pos+55,fill=gnbB_color!30] () {};
    \draw (BS_B.antenna center) node[Beam,minimum width=0.75cm,minimum height=2.5mm,rotate=\pos-25,fill=gnbB_color!60] () {};
    \draw (BS_B.antenna center) node[Beam,minimum width=0.75cm,minimum height=2.5mm,rotate=\pos+25,fill=gnbB_color!60] () {};
    \draw (BS_B.antenna center) node[Beam,minimum width=1.5cm,minimum height=3mm,rotate=\pos,fill=gnbB_color!80] (beam_1) {};
    
    %% Add labels
    \draw node[label,below of=UE_A, xshift=0.1cm, yshift=0.1cm] () {${\rm UE}_{\rm A}$};
    \draw node[label,below of=UE_B, xshift=0.1cm, yshift=0.1cm] () {${\rm UE}_{\rm B}$};
    \draw node[label,below of=BS_A, xshift=0cm, yshift=-0.9cm] () {${\rm gNB}_{\rm A}$};
    \draw node[label,below of=BS_B, xshift=0cm, yshift=-0.9cm] () {${\rm gNB}_{\rm B}$};
    
    %% Add signal and interference lines
    \draw[densely dotted,decoration={markings,mark= at position 0.5 with {\arrow{latex}},}, postaction={decorate}] (BS_A.antenna center) -- (UE_B.north);
    \draw[densely dotted,decoration={markings,mark= at position 0.5 with {\arrow{latex}},},postaction={decorate}] (BS_B.antenna center) -- (UE_A.north);
    \draw[decoration={markings,mark= at position 0.75 with {\arrow{latex}},},postaction={decorate}] (BS_A.antenna center) -- (UE_A.north);
    \draw[decoration={markings,mark= at position 0.75 with {\arrow{latex}},},postaction={decorate}] (BS_B.antenna center) -- (UE_B.north);
    
    %% Add the extra legend
    \path (2.2, 4.3) coordinate (Legend);
    \draw (Legend) -- ++(-0.25,0);
    \draw (Legend) ++ (0.6,0) node[label] () {Signal};
    \draw[densely dotted] (Legend) ++(0,-0.4) -- ++(-0.25,0);
    \draw (Legend) ++ (1,-0.4) node[label] () {Interference};
\end{tikzpicture}
    \caption{Signal and interfering
    paths in a system with two carriers.}
    \label{fig:network}
\end{figure}

Fig.~\ref{fig:network} shows the scenario for analyzing the interference between adjacent carrier frequencies. A UE in carrier A 
will receive the desired signal from its serving BS and interference signal from non-serving
BSs in
carrier A in the same carrier and
from all BSs in carrier B.

\begin{figure*}[t]
\centering
    \subfigure[\SI{28}{GHz}]{\includegraphics[width=0.49\linewidth]{fig/snr_28-eps-converted-to.pdf}}
    \subfigure[\SI{140}{GHz}]{\includegraphics[width=0.49\linewidth]{fig/snr_140-eps-converted-to.pdf}}
    
 \caption{Estimated distribution of downlink SNRs with different carrier frequencies
    and different RFFE designs.  The plots
    show the SNR under the full model
    \eqref{eq:SNR} with distortion from
    adjacent carrier interference and 
    in-band signal; the SNR 
    with no adjacent carrier interference distortion ($\alpha_2=0)$; and
    the SNR with no in-band or
    adjacent carrier interference distortion ($\alpha_1=\alpha_2=0)$.}
  \label{fig:snr_distributions}
\end{figure*}

% \begin{figure*}[t]
%     \centering
%     \subfigure[\SI{28}{GHz}]{\includegraphics[width=0.45\linewidth]{fig/rate_28.eps}}
%     \subfigure[\SI{140}{GHz}]{\includegraphics[width=0.45\linewidth]{fig/rate_140.eps}}
%     \caption{Estimated distribution of the downlink rates for different carrier frequencies and RFFE designs.  }
%     \label{fig:rate_distributions}
% \end{figure*}

For every gNB-UE pair, we generate a multi-path channel for two different frequency cases, \SI{28}{GHz} and \SI{140}{GHz},  according to~\cite{3GPP38.901} and compute the SNR for downlink case. 
Specifically, we employ the path-loss model specified in~\cite{3GPP38.901} for the Urban Micro Street Canyon (Umi-Street-Canyon) environment. The 3GPP NR standard is very flexible and we assume the channel models described in
\cite{3GPP38.901} will hold for the \SI{140}{GHz} communication systems.
% Since the channel model of \SI{140}{GHz} carrier frequency is not known, we assume that the legacy 3GPP channel model in \cite{3GPP38.901} is still valid at \SI{140}{GHz}. 

Within each carrier, we then assume that each UE
is served by the strongest gNB.  As a simplification, 
the gNBs and UEs then beamform
along the strongest path with no regard to
interference nulling to other UEs.  
A sufficient number of UEs are dropped such
that we can obtain one UE served by each
sector in each gNB.  Hence, the simulation
drop represents one point in time where
each gNB is using its entire bandwidth 
on one UE.

With the channels and beamforming
direction, we can then estimate the effective
SINR at each UE.  
Following the model \eqref{eq:gamout},
we estimate the SINR as:
\begin{equation}
\label{eq:SNR}
    \gamma = \frac{\beta E_\mathrm{sig}^{a}}{E_\mathrm{kT} +
    \alpha_{1}E_\mathrm{tot}^{a} + \alpha_{2}E_\mathrm{tot}^{b}}.
\end{equation}
Here, $E_\mathrm{sig}^{a}$ 
and $E_\mathrm{int}^a$ 
is the energy 
per sample of the serving and interfering
signals including the beamforming and
element gains at the TX and RX,
and $E_\mathrm{kT}$ is the thermal noise.
The distortion from the non-linearities
is modeled by two terms:
$\alpha_{1}E_\mathrm{tot}^{a}$ captures
the distortion from the total power from all base stations in carrier A (serving and non-serving); $\alpha_{2}E_\mathrm{tot}^{b}$
captures the distortion from
the total power from all base stations
in the adjacent carrier, carrier B.
For these terms, we assume that the distortion
is spatially white so we do not 
add the RX beamforming gains on each path.
The terms however do include the TX element
and beamforming gains as well as the RX element
gain.

\begin{table}[t]
    \centering
    \caption{Network simulation parameters.}
    \setlength{\tabcolsep}{3pt}
    \label{tab:sim_param}
    \begin{tabular}{|>{\raggedright}m{1.6in}|c|c|}
        \hline
        \textbf{Parameter} & \multicolumn{2}{c|}{\textbf{Value}}
        \tabularnewline \hline
      
        Carrier frequency, [\SI{}{\GHz}] & $28$ & $140$
        \tabularnewline
        
        Total bandwidth, [\SI{}{\MHz}] & $190.80$ & $380.16$
        \tabularnewline 
        
        Sample rate, %$f_s$,
        [\SI{}{\MHz}] & $491.52$ & $1966.08$
        \tabularnewline 
        
        gNB antenna configuration &{$8\times8$} & {$16\times16$}
        \tabularnewline
        UE antenna configuration &{$4\times4$}& {$8\times8$}
        \tabularnewline \cline{2-3}
        
        Area [\SI{}{\meter}$^2$]  &  \multicolumn{2}{c|}{$1000 \times 1000 $}
        \tabularnewline
        
        UE and gNB min. distance [\SI{}{\meter}] &  \multicolumn{2}{c|}{$10$}
        \tabularnewline
        
        ISD [\SI{}{\meter}] & \multicolumn{2}{c|}{$200$}
        \tabularnewline
        
        gNB height [\SI{}{\meter}] &  \multicolumn{2}{c|}{$\ncalU(2,5)$}
        \tabularnewline
        
        UE height [\SI{}{\meter}] &  \multicolumn{2}{c|}{$1.6$}
        \tabularnewline
        
        gNB TX power [\SI{}{dBm}] & \multicolumn{2}{c|}{$30$}
        \tabularnewline
        
        gNB downtilt angle & \multicolumn{2}{c|}{$-12^\circ$}
        \tabularnewline
        
        gNB number of sectors & \multicolumn{2}{c|}{$3$}
        \tabularnewline
        
        Vertical half-power beamwidth %[$\theta_{\text{3dB}}$] 
        & \multicolumn{2}{c|}{$65^\circ$}
        \tabularnewline
        
        Horizontal half-power beamwidth % [$\phi_{\text{3dB}}$]
        & \multicolumn{2}{c|}{$65^\circ$}
        \tabularnewline \hline
    \end{tabular}
\end{table}


% Sundeep's notes:
% \begin{itemize}
%     \item $E_{\rm sig} = P_A T$ where $T=1/f_{s}$.
% \end{itemize}

Fig.~\ref{fig:snr_distributions} shows the SNR distributions  for the designs discussed in Section~\ref{sec:link} at \SI{28}{GHz} and \SI{140}{GHz}. As a performance benchmark,
we compare the SNR distribution under three models:
\begin{itemize}
    \item SNR with adjacent carrier interference and in-band distortion:  This is the model
    \eqref{eq:SNR} with the parameters
    for $\alpha_1$ and $\alpha_2$ in
    Table~\ref{tab:model_param} found from
    the circuit simulations.  The
    resulting SNR CDF is shown
    in the dashed line in Fig.~\ref{fig:snr_distributions}.
    
    \item SNR with no adjacent carrier interference and in-band distortion:  
    This is the model
    \eqref{eq:SNR} with the parameters
    for $\alpha_1$ in
    Table~\ref{tab:model_param} but
    $\alpha_2 = 0$.  The SNR CDF is shown in the
    dotted line in Fig.~\ref{fig:snr_distributions}.
    
    \item SNR with no adjacent carrier interference and  no in-band distortion:  
    This is the model
    \eqref{eq:SNR} with the parameters
    for $\alpha_1=\alpha_2=0$.  
    The resulting SNR CDF is shown in the
    solid line in Fig.~\ref{fig:snr_distributions}.

    
\end{itemize}
%we show an upper bound of the SNR considering a linear system that includes only the NF from the RFFE devices without any distortion (solid line).
%That is, the SNR in \eqref{eq:SNR} with
%$\alpha_1=\alpha_2=0$ so the effect of the
%distortion is removed.
% Similarly Fig.~\ref{fig:rate_distributions} shows the downlink rate distributions  for the designs discussed in Section~\ref{sec:link} at \SI{28}{GHz} and \SI{140}{GHz}.

Comparing the plots, we observe that the impact of distortion from adjacent carrier interference  is negligible.   This suggests that for these parameters,
there may be no need for extra filtering in the RF/IF or baseband to suppress the adjacent carrier interference.   Thus, we can conclude that by optimizing the RFFE devices and reducing the dynamic range of the system, we can improve the energy efficiency without being vulnerable from the adjacent carrier interference. Furthermore, as expected the designs at \SI{28}{GHz} have lower saturation points comparing to the \SI{140}{GHz} designs, %as they have lower number of antennas and so lower beamforming gain
due to the difference in the beamforming gain resulted from the difference in the number of antennas.

As explained in Section~\ref{sec:link}, at \SI{140}{GHz} we expect the designs to have different performance. In the low-SNR regime Design$^{(1)}$ performs better due to the lower NF, while Design$^{(2)}$ performs better in the high-SNR regime due to the larger number of ADC bits.

% \textcolor{blue}{[SR: I think the rate plots
% are wrong.  It looks like you are computing the
% rate via, $R=B \log_2(1 + \gamma)$.  This allows
% the rate to be very high since the system can
% benefit from very high SNRs.  But, 
% a more realistic model is:
% $R = B \max\{\rho_{\rm max},\alpha \log(1 + \gamma) \}$ where $\alpha=0.6$ and $\rho_{\rm max} = 4.5$ bps/Hz.  
% You should remove the plot, or correct it.]}

% mainly because of the beamforming gains

% We observe that the effect of adjacent carrier interference is negligible, and the SNR values are strongly affected by distortion instead. 

% In comparison with two carrier frequencies, \SI{28}{GHz} and \SI {140}{GHz}, the interference effects are largely imperceptible at \SI{140}{GHz} because of the shorter beam width at higher frequencies. In addition, we leverage the larger number of antenna arrays for \SI{140}{GHz} than \SI{28}{GHz}, which leads to larger saturation point of SNR values caused by the signal distortion than the case of \SI{28}{GHz}. \\
% \textcolor{red}{ 1) we need to explain the SNR gap in 140GHz for linear case and 2) the relationship between the signal distortion and frequency; why 140GHz shows little distortion effect compared to 28GHz case. 3) we might add the the saturation point for each frequency is reasonable considering access paper and antenna configurations, and more explanation. }





%% Notes for SNR plot 
% Desgin (i) = RFFE non linear distortion + interference
% Design (i) (w/o interference) = non-linear rffe w/o the presence of interference.
% Design (i) (linear) = linear RFFE no intereference 
% SJ: Let me know if you have any questions.

%In comparison with \SI{28}{GHz} case, the SNR values of \SI{140}{GHz} saturates at larger 

% \begin{figure*}[t]
% \centering
%     \subfigure[\SI{28}{GHz}]{\includegraphics[width=0.49\textwidth]{fig/snr_28.eps}}
%     \subfigure[\SI{140}{GHz}]{\includegraphics[width=\columnwidth]{fig/snr_140.eps}}
%   \caption{Rate distribution for two different frequency cases}
%   \label{fig:rate_distributions}
% \end{figure*}

%\textcolor{red}{
%Comments for SJ:
%\begin{itemize}
%    \item Explain beamforming in the simulation setup.
%    \item Elaborate on path loss
%    \item Explain figure 5 and 6
%\end{itemize}
%}

\begin{comment}
\begin{figure}
\includegraphics[width=\linewidth]{figs/beyond_tss_lesion.pdf}
\caption[]{End-to-End runtime lesion study of the entire MNIST dataset and the FMA featurized music dataset. Each of DROP's contributions provides a runtime improvement.}
\label{fig:beyond_lesion}
\end{figure}
\end{comment}



\section{Conclusion}
\label{sec:conclusion}

Advanced data analytics techniques must scale to rising data volumes. 
DR techniques offer a powerful toolkit when processing these datasets, with PCA frequently outperforming popular techniques in exchange for high computational cost. 
In response, we propose DROP, a new dimensionality reduction optimizer. 
DROP combines progressive sampling, progress estimation, and online aggregation to identify high quality low dimensional bases via PCA without processing the entire dataset by balancing the runtime of downstream tasks and achieved dimensionality. 
Thus, DROP provides a first step in bridging the gap between quality and efficiency in end-to-end DR for downstream \red{analytics}. 

%We revisit canonical operators for time series dimensionality reduction and the measurement study of~\cite{keogh-study}, and show that PCA is more effective than popular alternatives in the data mining literature often by a margin of over $2\times$ on average on gold-standard time series benchmark data sets with respect to output data dimension. More surprisingly, we empirically demonstrate that a small number of samples are sufficient to accurately characterize directions of maximum variance and obtain a high-quality low-dimensional transformation.



% %\documentclass[11pt, oneside]{article}   	% use "amsart" instead of "article" for AMSLaTeX format
%\documentclass[linenumbers,twocolumn]{aastex62}
\documentclass[twocolumn]{aastex62}
%\documentclass[reprint,nofootinbib]{revtex4-2}
%\usepackage{geometry}                		% See geometry.pdf to learn the layout options. There are lots.
%\geometry{letterpaper}                   		% ... or a4paper or a5paper or ... 
%\geometry{landscape}                		% Activate for rotated page geometry
%\usepackage[parfill]{parskip}    		% Activate to begin paragraphs with an empty line rather than an indent
%\usepackage{graphicx}				% Use pdf, png, jpg, or eps§ with pdflatex; use eps in DVI mode
%\usepackage{mathrsfs} 

\usepackage{amsmath}
\usepackage{amssymb}
\usepackage{empheq}
\usepackage{mathrsfs}
				
								% TeX will automatically convert eps --> pdf in pdflatex		
\usepackage{amssymb}

%SetFonts

%SetFonts

\newcommand{\G}{\mathcal{G}}
\newcommand{\M}{M_{\star}}
\newcommand{\Msun}{M_{\odot}}
\newcommand{\R}{\mathcal{R}}
\newcommand{\Ham}{\mathcal{H}}
\newcommand{\mn}{m_{\rm{N}}}
\newcommand{\an}{a_{\rm{N}}}
\newcommand{\en}{e_{\rm{N}}}
\newcommand{\Mn}{M_{\rm{N}}}
\newcommand{\order}{\mathcal{O}}




\begin{document}

%CTL throughout the article stands for CHECK THIS LATER

\title{The Stability Boundary of the Distant Scattered Disk}

\author{Konstantin Batygin}
\affiliation{Division of Geological and Planetary Sciences California Institute of Technology, Pasadena, CA 91125, USA}

\author{Rosemary A. Mardling}
\affiliation{School of Physics and Astronomy, Monash University, Victoria, 3800, Australia}

\author{David Nesvorn{\'y}}
\affiliation{Department of Space Studies, Southwest Research Institute, 1050 Walnut St., Suite 300, Boulder, CO 80302, USA}

\begin{abstract}
The distant scattered disk is a vast population of trans-Neptunian minor bodies that orbit the sun on highly elongated, long-period orbits. The orbital stability of scattered disk objects is primarily controlled by a single parameter -- their perihelion distance. While the existence of a perihelion boundary that separates chaotic and regular motion of long-period orbits is well established through numerical experiments, its theoretical basis as well as its semi-major axis dependence remain poorly understood. In this work, we outline an analytical model for the dynamics of distant trans-Neptunian objects and show that the orbital architecture of the scattered disk is shaped by an infinite chain of exterior $2:j$ resonances with Neptune. The widths of these resonances increase as the perihelion distance approaches Neptune's semi-major axis, and their overlap drives chaotic motion. Within the context of this theoretical picture, we derive an analytic criterion for instability of long-period orbits, and demonstrate that rapid dynamical chaos ensues when the perihelion drops below a critical value, given by $q_{\rm{crit}}=\an\,\big(\ln((24^2/5)\,(\mn/\Msun)\,(a/\an)^{5/2})\big)^{1/2}$. This expression constitutes an analytic boundary between the ``detached" and actively ``scattering" sub-populations of distant trans-Neptunian minor bodies. Additionally, we find that within the stochastic layer, the Lyapunov time of scattered disk objects approaches the orbital period, and show that the semi-major axis diffusion coefficient is approximated by $\mathcal{D}_a\sim(8/(5\,\pi))\,(\mn/\Msun)\,\sqrt{\G\,\Msun\,\an}\,\exp\big[-(q/\an)^2/2\big]$. We confirm our results with direct $N-$body simulations and highlight the connections between scattered disk dynamics and the Chirikov Standard Map. Implications of our results for the long-term evolution of minor bodies in the distant solar system are discussed.
\end{abstract}

\keywords{Orbital dynamics, Scattered disk objects, Perturbation theory}%Use showkeys class option if keyword
                              
%\maketitle

\section{Introduction} \label{sec:intro}

Among the various sub-populations of the icy debris that comprise the Kuiper belt, the most prominent -- both in terms of mass and radial extent -- is the scattered disk. A remnant of Neptune's early outward migration \citep{Nesvorny2018REV}, the scattered disk is largely made up of eccentric, low-inclination orbits that ``hug" the orbit of Neptune, maintaining a perihelion distance slightly above $q\gtrsim30\,$AU \citep{2020tnss.book...25M}. Interesting in its own right, the orbital architecture of the distant scattered disk is especially distinctive, as it provides an observational handle on the gravitational processes that have sculpted the outermost reaches of the solar system \citep{2004ApJ...617..645B, 2010ARA&A..48...47A, 2019PhR...805....1B, 2021arXiv210501065C}.

The dynamics of scattered disk objects (SDOs) have been studied in considerable detail over the past two and a half decades, and the general characteristics of their long-term evolution are relatively well understood (see \citealt{Saillenfest2020} and the references therein). Crudely speaking, objects with perihelion distance small enough to strongly interact with Neptune experience chaotic diffusion and eventually become Centaurs\footnote{Centaurs are broadly defined as objects with perihelion distance or semi-major axis that fall between the orbits of Jupiter and Neptune.}, or leave the solar system altogether. To be more precise, the survival probability of a chaotic SDO over the age of the sun is about $1\%$ \citep{Gomes2008}. Conversely, objects with large perihelia -- often referred to as the ``detached" population -- are immune to strong Neptune-induced perturbations, and simply orbit the sun on slowly precessing Keplerian orbits.

Today, orbital integration of scattered disk objects does not present a significant practical challenge. Well-tested symplectic integrators, predominantly based on the \citet{1991AJ....102.1528W} mapping, are widely available \citep{1998AJ....116.2067D, 1999MNRAS.304..793C, 2019MNRAS.485.5490R, 2019MNRAS.489.4632R}, bringing precise modeling of the distant solar system's long-term evolution within reach of virtually any modern Ghz-grade machine. Nevertheless, such numerical experiments can only solve for the emergent dynamics, not illuminate their theoretical basis. In other words, accurate realizations of orbital evolution can only expose \textit{what} the SDOs do, not \textit{why} they do it. Understanding the latter question requires a simplified analytic model. Here we develop such a model for the distant scattered disk with an eye towards quantifying its underlying dynamical structure and elucidating the processes that drive orbital diffusion, from analytic grounds. We begin by sketching out the statement of the problem. 

\paragraph{Statement of the Problem} The principal goal of the calculation we aim to carry out is easy to summarize: we wish to develop a simple theory for the long term behavior of highly eccentric, long-period minor bodies, subject to perturbations from Neptune. In other words, our goal is to solve the circular restricted three-body problem in a regime where the test particle possesses an orbital period much larger than that of the perturber, but still experiences material interactions with the planet, owing to the closeness of the perihelion distance to the planet's semi-major axis. The geometric setup of the problem is summarized in Figure \ref{Fig:SETUP}.

The circular restricted three-body problem is by no means a new problem, and the relevant literature spans centuries. Nevertheless, the vast majority of perturbation theory devoted to understanding the relevant dynamics is unsuitable for the problem at hand. Classical expansions of the planetary disturbing function (\citealt{1995CeMDA..62..193L, 2000Icar..147..129E} and the references therein) treat eccentricities and inclinations as small parameters, developing the governing Hamiltonian as a power-series in $e$ and $i$, while placing no constraints on the semi-major axis ratio with the exception of the formal requirement that the orbits do not cross. Conversely, the scattered disk is characterized by large (even near-unity) eccentricities, placing it outside of the domain of applicability of standard models.

As a means to circumvent the limitations of classical methods, various authors have reframed long-term evolution of the scattered disk as mapping problem. That is, rather than attempting to formulate a conventional perturbation theory, \citet{Malyshkin1999, 2004AJ....128.1418P, 2013Icar..222...20F, 2020PASP..132l4401K} envisioned the dynamics as a process wherein the test particle executes unperturbed Keplerian motion, with the exception of the perihelion, where it receives an energy kick of some magnitude that generally depends on the planetary mass and semi-major axis, as well as the particle's perihelion distance. Intuitive in its own right, the essence of this approach lies in the so-called \textit{Kepler Map} (see \citealt{2011NewA...16...94S} for a review). It is worth noting that this mapping was first derived by \citet{1986PhLA..117..328P}, and has since become an important tool for understanding a variety of physical phenomena, including those beyond the realm of dynamical astronomy \citep{Chirikov1989, 1987JPhB...20.5051G, 1988IJQE...24.1420C, 1988JPhB...21L.527J, 1994PhRvA..50..575S, 1998PhLA..241...53S}.

Despite the successes of the mapping approach in modeling chaotic motion at a vastly reduced computational cost, a full understanding of scattered disk dynamics remains incomplete. In particular, the crucial questions of which resonances underly the stochastic layer, and how the scattering process connects to a perturbative description of particle motion, remain to be elucidated. To address this issue, in this work we take an approach that is similar in spirit -- but not in detail -- to classical perturbation theories. More specifically, we adopt a description of the disturbing function as an infinite series developed in terms of a small parameter, which we take to be the semi-major axis ratio, $\alpha=\an/a$, rather than the eccentricity \citep{1962AJ.....67..300K,Laskar2010DISTFUNCT, 2013MNRAS.435.2187M}. As we show below, this Kaula-type expansion attractively lends itself to a simplified description of SDO dynamics and naturally illuminates the relationship between Neptune's exterior mean motion resonances and the scattering process.

The remainder of this paper is organized as follows. In section \ref{sec:anmodel}, we outline the basis of our analytical theory. The results are presented in section \ref{sec:results}. Particularly, in section \ref{sec:intmod}, we derive a simplified Hamiltonian model of particle motion. We formulate the stability boundary of the scattered disk in terms of a critical perihelion distance in section \ref{sec:chirikov}. We validate our analytic results with $N-$body simulations in section \ref{sec:numsim}. In section \ref{sec:standard}, we highlight the connection between our model and the Chirikov Standard Map, thus outlining the equivalence between our perturbative approach and scattering viewpoint. Finally, in section \ref{sec:analytic} we derive analytic estimates of the Lyapunov time and the semi-major axis diffusion coefficient within the scattered disk. We summarize and discuss our results in section \ref{sec:discuss}.

\begin{figure*}[t]
\centering
\includegraphics[width=0.8\textwidth]{SETUP.png}
\caption{Scattered disk dynamics modeled as a circular planar restricted three-body problem. A long-period scattered disk object (SDO) is envisioned to orbit the sun on a highly eccentric orbit with a perihelion distance that exceeds Neptune’s semi-major axis by a small margin. The SDO orbit is shown as a purple ellipse in the digram. Mutual inclination between Neptune and the SDO, as well as perturbations from other planets are neglected. The SDO is modeled as a test-particle. A quadrupole-level spherical harmonic expansion of Neptune’s gravitational potential illuminates that SDO evolution is facilitated by an infinite chain of Neptune's exterior $2:\chi$ resonances. Within the stochastic layer, dynamics of the test particle are primarily driven by the nearest $2:\chi$ resonance (where $\chi$ is an integer approximation to $2\,(a/a_{\rm{p}})^{3/2}$), while its chaotic evolution is facilitated by interactions with neighboring resonances. On the diagram, the nominal locations of $2:\chi\pm1$, $2:\chi\pm2$, $2:\chi\pm3$, and $2:\chi\pm4$ resonances adjacent to the SDO orbit are shown as green ellipses. As we discuss in the text, for the problem at hand, the resonance overlap criterion can be recast as a critical perihelion distance, $q_{\rm{crit}}$, below which chaotic evolution ensues.}
\label{Fig:SETUP}
\end{figure*}

\section{Perturbation Theory} \label{sec:anmodel}
As a starting point in our calculation, let us outline our basic assumptions. First and foremost, we treat the SDO as a massless test particle, and assume that its orbital period exceeds that of Neptune by a large margin (i.e., $\alpha\ll1$). Second, we assume that Neptune's eccentricity, $\en$, is sufficiently small to be negligible for our purposes. Third, we neglect all inclination-node dynamics, reducing the problem to a common plane (Figure \ref{Fig:SETUP}). 

\subsection{The Disturbing Function}
A general spherical harmonic expansion of the planetary disturbing function for the co-planar three-body problem is presented in \citet{2013MNRAS.435.2187M}. Employing the usual notation of celestial mechanics, the disturbing function is expressed as a quadruple infinite series:
\begin{align}
\R&=\frac{\G\,\mn}{a}\, \sum_{\ell=2}^{\infty} \, \sum_{m=m_{\rm{min}},\,2}^{\ell} \, \sum_{j' = -\infty}^{\infty} \, \sum_{j = -\infty}^{\infty} \zeta_m\, c_{\ell\,m}^2\,\mathcal{M}_\ell \nonumber \\
&\times \alpha^\ell\, X_{j'}^{\ell,m} (\en) \,X_{j}^{-(\ell+1),m} (e)\, \nonumber \\
&\times \cos\big(j'\,\Mn -j\,M+m(\omega-\omega_{\rm{N}})\big).
\label{R}
\end{align}
In the above expression, the unmarked variables ($a,M,e,\omega$) refer to the orbital elements of the SDO, while those with the subscript $\rm{N}$ correspond to Neptune\footnote{Technically, in equation (\ref{R}), Neptune's mass $\mn$ should be the reduced mass $\mu=\mn\,\Msun/(\mn+\Msun)\approx\mn$, but our choice to replace $\mu$ with $\mn$ is of no practical consequence.}. Note that for the planar problem, the distinction between the longitude and argument of perihelion vanishes, such that $\omega$ simply corresponds to the azimuthal orientation of the apsidal line (see Figure \ref{Fig:SETUP}). We will remark on the dimensionless constants $\zeta, c, \mathcal{M}$, as well as Hansen coefficients $X$ in greater detail below.

In equation (\ref{R}), the index $\ell$ informs the \textit{degree} of the spherical harmonic expansion. Because we are specifically interested in the $\alpha\ll1$ limit, our needs are sufficed by truncating the expansion at quadrupolar level, corresponding to $\ell_{\rm{max}}=2$. This removes the first sum of the series completely, as well as any dependence on the mass-factor $\mathcal{M}$ because $\mathcal{M}_2=1$ for all mass ratios \citep{2013MNRAS.435.2187M}. On a quantitative level, however, this assumption restricts the applicability of our model to long-period (e.g., $a\gtrsim400\,$AU) orbits.

Beyond the first sum, truncation of the series at $\ell=2$ sets manageable bounds of the \textit{order} of the expansion, $m$, such that the second sum runs from $m_{\rm{min}}=0$ to $m_{\rm{max}}=2$. Carrying on, the third sum can be eliminated fully, thanks to an important property of the Hansen coefficient $X_{j'}^{l,m}$. Specifically, it is possible to demonstrate that to leading order in eccentricity, $X_{j'}^{l,m}(\en)=\order(\en^{|m-j'|})$ \citep{1981CeMec..25..101H}, meaning that within the context of our adopted limit of $\en\rightarrow0$, all terms with $m\ne j'$ vanish. The dependence on the first Hansen coefficient in the series is further trivialized by the fact that $X_0^{2,0}(0) = X_1^{2,1}(0) = X_2^{2,2}(0) = 1$. In addition, because apsidal orientation is ill-defined at null eccentricity, we can set $\omega_{\rm{N}}=0$ without loss of generality.

A final simplification comes from the functional form of the constant $c_{\ell\,m}^2$. Written explicitly in terms of spherical harmonics $Y_{\ell,m}$, we have $c_{2\,m}^2=(8\,\pi/5)\,(Y_{2,m}(\pi/2,0))^2$. Crucially, $Y_{2,1}(\pi/2,0)=0$, meaning that all terms of order unity have zero amplitude, and the expansion only contains harmonics with $m=0$ and $m=2$. Writing out all remaining constants explicitly, we have $c_{2\,0}^2 = 1/2$, $c_{2\,2}^2 = 3/4$, $\zeta_0=1/2$, and $\zeta_2=1$. 

With the approximation scheme outlined above, the full quadrupole-level expression for the disturbing function takes the form:
\begin{align}
\R_{\rm{q}}&=\frac{\G\,\mn}{4\,a}\, \alpha^2  \sum_{j = -\infty}^{\infty} \bigg[\underbrace{X_{j}^{-3,0}\, \cos\big(j\,M\big)}_{m\,=\,0} \nonumber \\
&+ \underbrace{3\,X_{j}^{-3,2}\, \cos\big(j\,M- 2(\Mn-\omega)\big)}_{m\,=\,2} \bigg],
\label{Rq}
\end{align}
where the dependence of $X$ on $e$ is implied. Let us now classify these zeroth and second order (in $m$) terms, according to the dynamics they govern.

\subsection{$m=0$: short-periodic and secular terms}
Ignoring the pre-factor in equation (\ref{Rq}), let us begin by examining the first set of harmonics. Noting the general property of Hansen coefficients $X_{j}^{-(\ell+1),m}=X_{-j}^{-(\ell+1),-m}$, it is convenient to write the leading sum as:
\begin{align}
\sum_{j = -\infty}^{\infty} X_{j}^{-3,0}\, \cos\big(j\,M\big) &= (1-e^2)^{-3/2} \nonumber \\
&+2\,\sum_{j = 1}^{\infty} X_{j}^{-3,0} \cos\big(j\,M\big),
\label{sumexp}
\end{align}
where we have taken advantage of the fact that $X_{0}^{-3,0}$ can be evaluated in closed form \citep{1981CeMec..25..101H}.

The first member of the RHS of equation (\ref{sumexp}) is a pure secular term that governs the apsidal precession of an SDO due to the phase-averaged potential of Neptune. The remainder of the RHS is the epitome of short-periodic terms i.e., harmonics that average out on the orbital timescale and contribute virtually nothing to long-term orbital evolution. As is well known, these rapidly oscillating terms can be removed from the Hamiltonian all-together through a near-identity variable transformation, which essentially corresponds to a change from osculating to orbit-averaged (mean) orbital elements (see e.g., Ch. 2 of \citealt{Morbybook}). Thus, we are justified in dropping them from the expression.

Although marginally illuminating, our discussion of $m=0$ perturbations is neither interesting nor new. That is to say, at zeroth order in $m$, the quadrupole-level disturbing function contains no terms that can explain the underlying chaotic structure of the scattered disk. Therefore, scattering dynamics must arise at $m=2$ order, which we examine next. 

\subsection{$m=2$: resonant terms}
The functional form of the $\ell=2,m=2$ harmonic is easy to interpret: the critical argument
\begin{align}
\varphi=j\,M- 2(\Mn-\omega))=j(\lambda-\varpi)-2(\lambda_{\rm{N}}-\varpi)
\label{varphi}
\end{align}
governs the exterior $2:j$ mean-motion resonance with Neptune. Correspondingly, the functional form of equation (\ref{Rq}) indicates that the underlying dynamical structure of the distant scattered disk is nothing more than an infinite sequence of $2:j$ resonances. In fact, to the extent that the quadrupole-level expansion of $\R$ is an accurate representation of the planetary potential, $2:j$ resonances \textit{must} drive the scattering process, since no other harmonics exist in the expansion.

This notion immediately suggests that the stability boundary of the scattered disk (which separates the chaotic and regular dynamics) can be understood within the context of the \citet{Chirikov1979} resonance overlap criterion. We will examine this suspicion more closely below. For the time being, however, we will limit ourselves to simply writing down the model Hamiltonian for the SDO. Dropping short-periodic terms as described above, we have:
\begin{align}
&\Ham=-\frac{\G\,\Msun}{2\,a} - \frac{1}{4} \frac{\G\,\mn}{a}\, \alpha^2 \, (1-e^2)^{-3/2} +\mathcal{T} \nonumber \\
&- \frac{3}{4} \frac{\G\,\mn}{a}\, \alpha^2  \sum_{j = -\infty}^{\infty} X_{j}^{-3,2} \cos\big(j\,M- 2(n_{\rm{N}}\,t-\omega)\big),
\label{Hammy}
\end{align}
where in anticipation of canonical transformations that will follow, we have replaced $\Mn$ with $n_{\rm{N}}\,t$ and introduced a dummy action, $\mathcal{T}$, conjugate to time, in order to keep $\Ham$ formally autonomous.

\subsection{Computation of Hansen Coefficients $X_{j}^{-3,2}$}
The final piece that is needed to complete the specification of our framework is the evaluation of the integral-defined functions $X_{j}^{-3,2}$. Since their introduction by \citet{Hansen1885}, these coefficients have been earnestly studied in the literature (see e.g., \citealt{1970ceme.book.....H,1981CeMec..25..101H, 2008CeMDA.100..287S}), and the general consensus holds that closed-form expressions for coefficients with $j\neq0$ do not exist. Nevertheless, \citet{Sadov2006} has demonstrated that in the double limit of $e\rightarrow1^{-}$ \textit{and} $j\rightarrow\infty$ the specific coefficient $X_{j}^{-3,2}$ (which \citealt{Sadov2006} calls a Chernousko function with index $j-2$) approaches the asymptotic form:
\begin{align}
X_{j}^{-3,2}\,\xrightarrow[j\rightarrow\infty]{e\rightarrow1^{-}}\, -\frac{4}{9}(j-2).
\label{Sadov}
\end{align}

Even the most eccentric scattered disk objects within the current census of TNOs are insufficiently close to $e$ of unity for equation (\ref{Sadov}) to apply. Similarly, a series approximation of $X_{j}^{-3,2}$ in terms of $\sqrt{1-e^2}$ (see \citealt{2008CeMDA.100..287S}) does not converge rapidly enough to be quantitatively useful. Nevertheless, the quasi-linear scaling of $X_{j}^{-3,2}$ with $j$ is intriguing, and through numerical evaluation we have found that the relationship $X_{j}^{-3,2}\propto j$ holds with a surprising degree of accuracy along contours of constant $q=a\,(1-e)=(j/2)^{2/3}\,a_{\rm{N}} \,(1-e)$. Taking advantage of this, the slope of the linear relationship can be expressed as sole function of $q/\an$, and we have found that a simple Gaussian-like parameterization achieves satisfactory precision:
\begin{align}
X_{j}^{-3,2} \approx \frac{2\,j}{5}\exp\bigg[ -\bigg(\frac{q}{a_{\rm{N}}} \bigg)^2\, \bigg].
\label{param}
\end{align}
In fact, applied specifically to the observationally relevant $q\in(30,50)\,$AU range, equation (\ref{param}) agrees with direct evaluation of Hansen coefficients with $j>10$ down to a few percent. We will elaborate on the calculational advantage of evaluating $X_{j}^{-3,2}$ along a locus of constant perihelion further below.

It is interesting to compare the form of equation (\ref{param}) to expression (B5) of \citet{2013MNRAS.435.2187M}, which is based on an asymptotic expansion for the overlap integral representing the energy exchanged during one outer orbit (see also equations 3.55 and 3.73 of \citealt{Mardling2008Chaos}). In particular, the exponential decay of the low and high-eccentricity tails reflects the fact that an exponentially small amount of (specific) energy is exchanged between the orbits when the angular frequency of the test particle at perihelion is significantly different to the orbital frequency of Neptune. Conversely, significant energy is exchanged when these frequencies are similar. In fact, they are the same when $q/\an= (1+e)^{1/3}$ which for $e\sim 1$ corresponds to $q\approx38\,$AU. Thus, even before examining the onset of instability from the vantage point of the Chirikov criterion, we may intuitively expect the semi-major axis dependence of the perihelion stability boundary to be relatively shallow. %With our theoretical basis delineated, let us now proceed to write down the Chirikov criterion

\section{Results} \label{sec:results}
In order to evaluate the stability boundary of the scattered disk established by $2:j$ resonances, we must estimate the critical value of $q$ as a function of $a$, at which neighboring resonances overlap. Accordingly, we now project the separatrixes of the individual resonances onto the $q-a$ plane. As is usual for calculations of this type, the first step is to write down an integrable pendulum-like Hamiltonian for an isolated $2:\chi$ resonance, where $\chi$ is an integer nearest to $2\,(a/a_{\rm{p}})^{3/2}$. 

\subsection{An Integrable Model for an Isolated $2:\chi$ Resonance} \label{sec:intmod}
Because the resonance width is expected to be small compared to the SDO semi-major axis, a conventional approach to circumventing the inverse semi-major axis dependence of the Keplerian term in equation (\ref{Hammy}) is to Taylor-expand it around the nominal resonance location. Accordingly, in terms of conventional Delaunay variables $L=\sqrt{\G\,\Msun\,a}, l=M, G=L\,\sqrt{1-e^2}, g=\omega$ (see Ch. 2 of \citealt{MD99}), we have:
\begin{align}
&-\frac{1}{2}\bigg(\frac{\G\,\Msun}{L} \bigg)^2 \approx -\frac{1}{2}\bigg(\frac{\G\,\Msun}{[L]} \bigg)^2 + \bigg(\frac{\G\,\Msun}{[L]} \bigg)^2 \bigg( \frac{\delta L}{[L]} \bigg) \nonumber \\
& - \frac{3}{2} \bigg(\frac{\G\,\Msun}{[L]} \bigg)^2 \bigg( \frac{\delta L}{[L]} \bigg)^2 =[n]\,\bigg(\delta L - \frac{3}{2}\frac{\delta L^2}{[L]} \bigg)+\dots ,
\label{Hkeptaylor}
\end{align}
where $[L]=\sqrt{\G\,\M\,(\chi/2)^{2/3}\,\an}$ is the nominal action and $[n]=(2/\chi)\,n_{\rm{N}}$ is the mean motion at the center of the $2:\chi$ resonance. At this stage, it is convenient to adopt $\delta L=L-[L]$ as the action instead of $L$ itself, keeping in mind that translation of an action by a constant is always canonical. 

Let us now define a change of variables through a type-2 generating function:
\begin{align}
\mathcal{F}_2=(\underbrace{\chi\,l/2-(n_{\rm{N}}\,t-g)}_{\phi})\,\Phi+(\underbrace{l}_{\psi})\,\Psi+(\underbrace{t}_{\xi})\,\Xi.
\label{Hkeptaylor}
\end{align}
The actions conjugate to the new angles $\phi$, $\psi$, and $\xi$ are defined by the usual transformation equations:
\begin{align}
&\delta L=\frac{\partial\,\mathcal{F}_2}{\partial\,l}=\frac{\chi}{2}\,\Phi+\Psi \nonumber \\
&G=\frac{\partial\,\mathcal{F}_2}{\partial\,g}=\Phi \nonumber \\
&\mathcal{T}=\frac{\partial\,\mathcal{F}_2}{\partial\,t}=\Xi-n_{\rm{N}}\,\Phi.
\label{transform}
\end{align}

With the preliminaries (\ref{Hkeptaylor}) and (\ref{transform}) delineated, we are now in a position to write down an idealized Hamiltonian, $\Ham_{\chi}$, for each isolated resonance. Neglecting the unimportant $m=0$ secular term in equation (\ref{Hammy}) and retaining only the principal harmonic, we have:
\begin{align}
\Ham_{\chi}&=-\frac{3}{4}\frac{n_{\rm{N}}\,\chi}{[L]} \, \Phi^2-\frac{3\,n_{\rm{N}}}{[L]}\,\Psi\,\Phi \nonumber \\
&- \frac{3}{4} \frac{\G\,\mn}{a}\, \bigg(\frac{a_{\rm{N}}}{a} \bigg)^2\,X_{\chi}^{-3,2}\,\cos\big(2\,\phi \big) \nonumber \\
&+\underbrace{\Xi + \frac{2\,n_{\rm{N}}}{[\chi]}\,\Psi-\frac{3\,n_{\rm{N}}}{\chi\,[L]}\,\Psi^2 -\frac{1}{2}\bigg(\frac{\G\,\Msun}{[L]} \bigg)^2}_{\rm{const.}}
\label{Hchi}
\end{align}
Notice that upon switching to variables (\ref{transform}), the linear term in $\Phi$ arising from the $[n]\,\delta L$ term in equation (\ref{Hkeptaylor}) is exactly cancelled by the $-n_{\rm{N}}\,\Phi$ term that ensues from the dummy action $\mathcal{T}$, owing to the fact that $(\chi/2)\,[n]=n_{\rm{N}}$.

$\Ham_{\chi}$ is now independent of the angles $\psi$ and $\xi$, and the conjugate actions $\Psi$ and $\Xi$ are integrals of motion. Accordingly, all terms on the third line of equation (\ref{Hchi}) are constant and can simply be dropped from the Hamiltonian. Moreover, the linear action term (proportional to $\Phi\,\Psi$) can be absorbed into the leading term by adding $(3\,n_{\rm{N}}\Psi^2)/(\chi\,[L])$ to $\Ham_{\chi}$ and completing the square, such that the nonlinear action term becomes proportional to $\tilde{\Phi}^2=(\Phi-2\Psi/\chi)^2$. Then, adopting $\tilde{\Phi}$ as the new action conjugate to $\phi$ (again, by canonical translation) and substituting parameterization (\ref{param}) for the Hansen coefficient, we obtain the Hamiltonian of a mathematical pendulum:
\begin{align}
\Ham_{\chi}=&-\underbrace{\frac{3}{\an^2}\,\bigg( \frac{\chi}{2} \bigg)^{2/3}}_{\beta}\,\frac{\tilde{\Phi}^2}{2}\nonumber \\
&-\underbrace{\frac{6}{5}\frac{G\,\mn}{\an\,\chi}\exp\bigg[-\bigg(\frac{q}{\an}\bigg)^2\,\bigg]}_{\gamma}\,\cos(2\,\phi).
\label{Hchi2}
\end{align}

\begin{figure*}[t]
\centering
\includegraphics[width=\textwidth]{MEGNO.png}
\caption{Chaos map of the distant scattered disk, modeled within the framework of the circular planar restricted three-body problem. A heat map of the MEGNO chaos indicator, $Y$, is shown on the semi-major axis vs. perihelion plane. Blue regions of the diagram depict initial conditions that lead to regular motion, whereas yellow and red regions correspond to chaotic dynamics. Within the chaotic layer, the Lyapunov time of the SDO approaches the orbital period. The analytic threshold for chaotic motion ($q_{\rm{crit}}$, given by equation \ref{qcrit}) is shown with a thick black line. The nominal locations and widths of individual $2:\chi$ mean motion resonances are shown with thin green and white lines, respectively.}
\label{Fig:MEGNO}
\end{figure*}

It is worth noting that in the well-studied case of low-$e$ mean motion resonance dynamics \citep{MD99,Morbybook} the oscillation period of the resonant angle exceeds the orbital period by a large margin. This is not the case for the SDO scattering problem at hand: in the $q\sim \an$ regime, the ratio of the libration frequency to mean motion is given by $\sqrt{\beta\,\gamma}/n\sim (a/\an)^{5/4}\,\sqrt{\mn/\M}\sim1/4-1/2$. Thus, the orbital frequency exceeds the libration frequency only by a factor of a few.

The expression for the resonance width of a mathematical pendulum is well known: $\Delta\tilde{\Phi}=4\sqrt{\gamma/\beta}$ (Ch. 4 of \citealt{Morbybook}). It is important to understand, however, that this width -- expressed in terms of the canonical action $\tilde{\Phi}$ -- is ultimately related to the SDO's eccentricity. To relate this quantity to the resonance width in terms of the semi-major axis, we use the conservation of $\Psi$. In fact, it is straightforward to demonstrate that the conservation of $\Psi$ is nothing more than a re-statement of the conservation of the Tisserand parameter (Ch. 8 of \citealt{MD99,2013A&A...556A..28B}). Moreover, it is easy to show that for long-period and highly eccentric orbits, the conservation of the Tisserand parameter is equivalent to a conservation of the perihelion distance. A more detailed discussion of the physical meaning of the conservation of $\Psi$, and how it relates to other quasi-integrals of motion of the circular restricted three-body problem is presented in the Appendix.

\subsection{The Chirikov Criterion} \label{sec:chirikov}

From equation (\ref{transform}), it follows that $\Delta\delta L=\chi\,\Delta\tilde{\Phi}/2$. Direct substitution therefore yields: 
\begin{align}
\Delta a = 4\,a_{\rm{N}}\,\sqrt{\frac{2\,\chi\,\mn}{5\,\Msun}} \, \exp\bigg[-\bigg(\frac{q}{2\,\an}\bigg)^2\,\bigg].
\label{deltaa}
\end{align}
We note that while we arrived at this expression from the Hamiltonian formalism, an alternative approach would have been to start with the disturbing function (\ref{Rq}), write down Lagrange's equations of motion, and proceed to derive a pendulum-like equation of motion for the critical argument, $\phi$. Indeed, both approaches yield equivalent results (see e.g., Ch. 8.6. of \citealt{MD99}, section 2.3 of \citealt{2013MNRAS.435.2187M}; see also \citealt{1980AJ.....85.1122W, Mardling2008Chaos}). 

As stipulated by \citet{1959SPhD....4..390C,Chirikov1979}, the width of the resonance, $\Delta a$ should be compared with the distance between adjacent resonances, $\delta a = ([a]_{\chi+1}-[a]_{\chi-1})$. In the limit of large $\chi$, it is straightforward to show that $\delta a \approx (2\,\an/3)\,(2/\chi)^{1/3}$. The degree of resonance overlap is characterized by the ratio of $\Delta a$ and $\delta a/2$ (recall that neighboring resonances also widen in an equivalent way). Expressing this number in terms of SDO semi-major axis instead of $\chi$, we have:
\begin{align}
\frac{\Delta a}{\delta a} = \frac{24}{\sqrt{5}}\,\bigg(\frac{a}{\an}\bigg)^{5/4}\,\sqrt{\frac{\mn}{\Msun} }\,\exp\bigg[-\bigg(\frac{q}{2\,\an}\bigg)^2\,\bigg].
\label{K}
\end{align}
This result demonstrates an intriguing trend: along a locus of constant perihelion distance, the degree of overlap \textit{grows} with increasing particle semi-major axis. As importantly, we can set the overlap number equal to the critical value of unity ($\Delta a/\delta a\rightarrow1$), and invert this relation:
\begin{align}
q_{\rm{crit}}=\an\,\sqrt{\ln\bigg( \frac{24^2}{5}\,\frac{\mn}{\Msun }\,\bigg(\frac{a}{\an}\bigg)^{5/2}\bigg)}. 
\label{qcrit}
\end{align}
This expression yields a critical perihelion distance, $q_{\rm{crit}}$, as a function of semi-major axis, below which chaotic diffusion is expected to ensue. 

\subsection{Numerical Validation}  \label{sec:numsim}
The analytic calculations outlined above yield a compact result for the chaotic threshold of the scattered disk. This result is a key prediction of our theory that can be tested with numerical experiments in a straightforward manner. Here, we carry out this examination as a sequence of two sets of $N-$body simulations employing distinct levels of complexity. More specifically, our first task is to compare our expression with a chaos map generated within the context of an identical physical configuration -- the circular planar restricted three-body problem. This Sun-Neptune-SDO setup provides the closest point of comparison between our analytic theory and numerical calculations, and is equivalent to lifting the assumptions employed in reducing the complexity of the disturbing function (\ref{R}).

%\subsection{Circular Restricted Three-Body Problem}

A customary way to map out the boundaries between chaotic and regular motion is to compute the system's Lyapunov coefficient, $\Lambda$ (or its siblings), on a plane of initial conditions. Here, we follow this conventional approach, substituting the Lyapunov coefficient for the more-rapidly-convergent MEGNO chaos indicator, $Y$ \citep{MEGNO}. We carried out these simulations using the \texttt{REBOUND} gravitational dynamics software package \citep{2019MNRAS.485.5490R,2019MNRAS.489.4632R}, employing the \texttt{whfast} integration algorithm with an initial time-step of $\delta t=63$ code units\footnote{The code uses units where the gravitational constant $\G$ is set to unity, such that in a unit system that employs solar masses and astronomical units, this time-step corresponds to approximately 10 years, or equivalently, $6\%$ of Neptune's orbital period.}. We generated two such maps, with $a\sim500\,$AU and $a\sim1000\,$AU. Each integration spanned $\Delta t = 0.1\,$Myr for $a\sim500\,$AU runs but was increased to $\Delta t = 0.3\,$Myr for $a\sim1000\,$AU runs to accommodate the longer orbital period. The resolution of our grid of initial conditions in SDO perihelion distance and semi-major axis was set to $\delta q = \delta a = 0.1\,$AU. Neptune's eccentricity remained at $e_{\rm{N}}=0$ throughout the integrations. Additionally, all starting orbital angles were set to null values, with the exception of the SDO mean anomaly, which was initialized at $M=\pi$ (aphelion). 

The left and right panels of Figure \ref{Fig:MEGNO} show MEGNO maps centered around a SDO semi-major axes of $a=500$ and $1000\,$AU, respectively. On the same plane, we mark the locations of individual $2:j$ resonances with green lines and project their widths according to equation (\ref{deltaa}) with white curves. The critical perihelion distance, corresponding to marginal overlap given by equation (\ref{qcrit}), is shown with a thick black line. As the color-bar indicates, blue regions of the plot (where $Y\sim2$) correspond to regular motion while initial conditions depicted with red and yellow points (where $Y\sim\Lambda\,\Delta t/2$) indicate chaotic SDO dynamics. As a check on our simulations, we recomputed portions of the shown MEGNO maps with a different choice of integration algorithm (\texttt{IAS15}) and longer timespan ($3\times\Delta t$) and got equivalent results.

Overall, the analytic criterion (\ref{qcrit}) provides a satisfactory approximation for the boundary between regular particle motion and large-scale chaos. Nevertheless, we remark that this threshold is inexact, and fine structure, including that arising from higher-order resonances, causes equation (\ref{qcrit}) to underestimate the critical value of $q$ at some values of $a$ while overestimating it at others. To elaborate on this further, the fact that regular regions exist at $q<q_{\rm{crit}}$ may in part be attributed to the fact that regular islands exist within the chaotic sea even if there is substantial overlap. The existence of chaotic regions for $q>q_{\rm{crit}}$, however, likely illuminates the limitations of our analytic model. To this end, it is likely that a more detailed resonance overlap criterion that also accounts for octupole-level resonances could generate better agreement. Note further that the agreement between $N-$body simulations and our theory is somewhat better for $a=1000\,$AU than for $a=500\,$AU. This is not surprising, given that the assumptions of our model are better satisfied for increasingly long-period orbits.  

%\subsection{Circular Restricted Three-Body Problem}

\begin{figure}[t]
\centering
\includegraphics[width=\columnwidth]{NUMSIM.png}
\caption{Detailed models of the distant scattered disk. Orange points depict a model of an evolved scattered disk that is created exclusively by giant planet scattering, accounting for early migration of Neptune through the solar system. The purple points show a model scattered disk that is affected by giant planets as well as the galactic tide and passing stars. An inclination cut of $i<40\deg$ was applied to both models. The analytic threshold for chaos is shown with a thick black curve, as in Figure \ref{Fig:MEGNO}. While the resonance overlap criterion marks the boundary between regular and stochastic dynamics, it should not be interpreted as the boundary of the scattered disk itself. In an idealized scenario that only includes giant planet scattering, the near-conservation of the Tisserand parameter prevents SDOs from filling the entirety of the chaotic domain. In a more realistic model that also accounts for extrinsic effects, Galactic perturbations can raise and lower SDO perihelia across the chaotic threshold.}
\label{Fig:NUMSIM}
\end{figure}

With our analytic expression for the chaotic boundary verified through numerical experimentation, we now consider how this threshold for orbital stability compares with detailed models of the formation and evolution of the scattered disk. To this end, we reference the published simulation suite of \citet{Nesvorny2017}, where the genesis of the scattered disk was simulated accounting for the early outward migration of Neptune \citep{2005Natur.435..459T,2011ApJ...738...13B, 2016ApJ...825...94N}, and its long-term fate was self-consistently modeled subject to gravitational forcing from the giant planets as well as (optionally) the Galactic tide and passing stars.

The orbital structure of evolved ($t=4.5\,$Gyr) synthetic models of the distant scattered disk with $i<40\,\deg$ are contrasted against our analytic stability boundary in Figure \ref{Fig:NUMSIM}. More specifically, the results of a simulation where extrinsic effects were omitted are shown with orange points, while a scattered disk that is sculpted by Galactic tides and passing stars in concert with the planets is shown with purple dots. Upon examination, an important conclusion can immediately be drawn: the boundary of the scattered disk (meaning the parameter space occupied by the particles) does \textit{not} uniformly trace its chaotic threshold. That is, in absence of Galactic forcing, long-period particles retain relatively low perihelia with $q\lesssim36\,$AU and do not extend to the edge of the chaotic zone. Conversely, when the effects of the Galactic tide and passing stars are included, the resulting eccentricity modulation can lift the perihelia of SDOs well above the critical value for chaos, especially for $a\gtrsim1000\,$AU orbits.

These results can be understood within the framework of our model as follows. While the $q<q_{\rm{crit}}$ orbital domain is largely chaotic, the long-period SDO dynamics nevertheless approximately obey the conservation of the Tisserand parameter. As shown in the Appendix of this work, preservation of the Tisserand parameter (or analogously the resonant integral of motion $\Psi$ defined in equation \ref{transform}) is equivalent to evolution along a constant-perihelion contour for orbits with $a\gg\an$ and $e\sim1$. This near-conservation of the perihelion distance prevents SDOs from exploring the full range of parameter space spanned by the chaotic sea in simulations that only include planetary forcing. 

The opposite situation ensues in numerical experiments that include the Galactic tide. Under the action of the Galactic tide, all symmetry inherent to the circular restricted three-body problem is broken, allowing significant $q$ variation to take place. Accordingly, at sufficiently long orbital periods, SDOs can be carried to large perihelion distances with no regard for the chaotic boundary facilitated by Neptune. The transition between scattering-dominated dynamics and evolution primarily driven by the Galactic tide is relatively sharp, and occurs at a semi-major axis of $a\gtrsim1000\,$AU. Qualitatively, this shift corresponds to a point where the timescale associated with von Zeipel-Lidov-Kozai type perihelion oscillations facilitated by the Galactic tide becomes markedly shorter than the perihelion precession timescale forced by the giant planets.

The dynamical origins of $q\gtrsim36\,$AU $a\lesssim1000\,$AU objects are considerably more subtle. Notice that unlike their more distant counterparts, these objects follow the stability boundary of the scattered disk relatively well. Owing to comparatively rapid perihelion precession, the perihelia of these objects cannot be affected by the Galactic tide directly. Nevertheless, their lowered eccentricities indicate that they have been materially affected by the Galactic tide \textit{at some point}, implying that they must have attained $a\gtrsim1000\,$AU in the past. Correspondingly, these are objects that initially get scattered onto large heliocentric distances, and after significant Galactic perturbation diffuse back to smaller semi-major axes. As inward semi-major axis diffusion gets terminated at the chaos boundary, parameter space traced by $q_{\rm{crit}}$ gets filled in from the outside. Examination of individual time-series of particles in the simulations confirms this interpretation.


%This begs the question of how the chaotic threshold regulates the orbital 

%Cumulatively, these numerical experiments highlight two important aspects of our results: 

%the distant scattered disk is expected to stretch over an extended range of perihelion distances and 

%while our expression for the chaotic threshold, $q_{\rm{crit}}$, constitutes an analytic boundary between the ``detached" and actively ``scattering" sub-populations of distant trans-Neptunian minor bodies, SDOs with semi-major axes in excess of $\a\gtrsim$

\subsection{Linking the Scattered Disk with the Modulated Pendulum and the Standard Map} \label{sec:standard}
Against the backdrop of the perturbative treatment of the dynamics developed in the preceding sections, it is important to not forget that the more rudimentary -- but somewhat more physically intuitive -- picture of scattered disk dynamics is one wherein perturbations are envisioned as ``kicks" to the orbit that ensue when the SDO passes through perihelion and experiences a gravitational interaction with Neptune \citep{2004AJ....128.1418P, 2013Icar..222...20F}. Accordingly, it is useful to briefly examine the connection between our perturbative framework and this ``mapping" viewpoint.%, as well as other paradigmatic results of chaos theory.

To begin making the analogy, note that in the limit of large $\chi$, the Hansen coefficients $X_{\chi}^{-3,2}\approx X_{\chi \pm 1}^{-3,2}$. Thus, let us assume that the Hansen coefficients with neighboring indexes are not simply similar, but are in fact, equal to one-another. Under this approximation, we can factorize the Hansen coefficient in equation (\ref{Hammy}), to obtain a simple non-autonomous Hamiltonian that accounts for interactions between the primary $2:\chi$ resonance and its nearest neighbors:
\begin{align}
\Ham_{\chi\pm}&=\beta\,\frac{\tilde{\Phi}^2}{2}-\gamma\,\big( \cos(2\,\phi-l)+\cos(2\,\phi)+\cos(2\,\phi+l)\big)\nonumber \\
&=\beta\,\frac{\tilde{\Phi}^2}{2}-\gamma\,(1+2\cos(n\,t)) \,\cos(2\,\phi).
\label{modpend}
\end{align}
Note that here we have used a trigonometric identity and set $l=M=n\,t$ to arrive at the second line (recall further that $\beta$ and $\gamma$ are defined in equation \ref{Hchi2}). This expression corresponds to the Hamiltonian of a \textit{modulated pendulum}, where the modulation frequency is equal to the SDO's mean-motion (Ch. 4 of \citealt{Morbybook}). Recalling that the mean motion is faster than the libration frequency by a factor of a few, chaotic dynamics that arise within the context of our problem lie squarely outside of the ``adiabatic" domain.

Let us now push our luck, and extend the aforementioned approximation by assuming that \textit{all} Hansen coefficients in the infinite perturbation series are equal. Although seemingly crude, this approximation in fact holds relatively well in practice because the dynamics of any given resonance is most strongly affected by perturbations that are ``nearby" in action space (or equivalently, in frequency space). Indeed, the amplitudes of faraway resonances do not matter much, since the harmonics vary rapidly and the corresponding terms quickly average out (see e.g., \citealt{Wisdom1982} for a discussion). In this limit, we can imagine that the sum in equation (\ref{Hammy}) runs exclusively over the cosines. Thus, employing a Fourier representation of the periodic $\delta$-function, we can write:
\begin{align}
&\sum_{j=-\infty}^{\infty}\cos(j\,l+2\,\phi)=\cos(2\,\phi)\,\sum_{j=-\infty}^{\infty}\cos(j\,l) \nonumber\\
&= \frac{1}{2\,\pi} \cos(2\,\phi) \, \delta_{2\pi/n},
\label{deltacos}
\end{align}
where $\delta_{2\pi/n}$ represents an impulse comb that is applied with the orbital period of the SDO at $l=0$ (perihelion).

Substituting equation (\ref{deltacos}) back into the expression for $\Ham$, we see that when expanded in the vicinity of a $2:\chi$ resonance, Hamiltonian (\ref{Hammy}) takes on the familiar form of a periodically kicked pendulum: 
\begin{align}
\Ham=\beta\,\frac{\tilde{\Phi}^2}{2}- \frac{\gamma}{2\,\pi} \cos(2\,\phi)\, \delta_{2\pi/n}.
\label{kicked}
\end{align}
As is well known, Hamiltonian (\ref{kicked}) generates the \textit{Chirikov Standard Map} -- an emblematic model of chaotic dynamics (e.g., \citealt{LLbook,Chirikov1979}). In fact, the appearance of the Standard Map within the context of this problem acts as the bridge between our analytic framework and the scattering viewpoint. To this end, it is crucial to note that the Kepler Map discussed in section \ref{sec:intro}, is locally identical to the Standard Map, which is governed by the above Hamiltonian \citep{2011NewA...16...94S, 2009IJMPD..18.1903K}. The connection between the perturbative treatment of SDO evolution and a mapping approach to modeling the orbital motion is thus clear.

\subsection{Chaos in the Scattered Disk: Analytic Estimates} \label{sec:analytic}
An important motivation behind making the connections between our perturbative theory of scattered disk dynamics and archetypal models of chaotic motion described above, is that the latter naturally lend themselves to analytic estimates \citep{LLbook}. In this vein, previous work aimed at quantifying Lyapunov times and the action diffusion constants of main belt Asteroids \citep{1996AJ....112.1278H,1997AJ....114.1246M,1998CeMDA..71..243N} and Mercury \citep{Laskar2008,2011ApJ...739...31L, 2015ApJ...799..120B} played an important role in expanding our overall understanding of chaotic small body evolution within the inner solar system. Here we continue this program, and focus on quantifying the Lyapunov time and semi-major axis diffusion coefficient within the scattered disk, from analytic grounds.  

\paragraph{Lyapunov Time} Our estimate the SDO Lyapunov time, $\tau_{\rm{L}}$, follows directly from the analogy with a modulated pendulum equation (\ref{modpend}) made above. To outline the qualitative picture, recall that the resonance width of a mathematical pendulum scales as the square root of the factor that multiplies the harmonic term of the Hamiltonian. Because this factor is time-dependent in equation (\ref{modpend}), however, the separatrix in our problem is not steady, and instead pulsates at the modulation frequency. In the regime of strong resonance overlap -- which we can crudely assume for orbits with $q< q_{\rm{crit}}$ -- a large fraction of the resonant phase-space area is periodically swept by a homoclinic curve, that instills hyperbolicity upon the SDO trajectory with the same frequency (Ch. 9.4 of \citealt{Morbybook}). Therefore, to an order of magnitude, the SDO's Lyapunov time can be interpreted as the modulation period, which in the case of Hamiltonian (\ref{modpend}) is nothing other than the orbital period:
\begin{align}
\tau_{\rm{L}}\sim\Lambda^{-1}\sim\frac{2\,\pi}{n}=\sqrt{\frac{4\,\pi^2\,a^3}{\G\,\Msun}}.
\label{Lyapana}
\end{align}

The fact that the Lyapunov time in the scattered disk is comparable to the orbital period can be understood from intuitive grounds. While macroscopic divergence of neighboring trajectories may require multiple Lyapunov times to ensue (depending on the initial separation of nearby starting conditions), it is important to keep in mind that $\tau_{\rm{L}}$ itself is a measure of decoherence on a microscopic scale. Accordingly, two initially nearby trajectories within the scattered disk will experience perturbations from Neptune at slightly distinct phases, meaning that their separation in phase-space will be amplified on the orbital timescale.

To test this assertion, let us return to Figure \ref{Fig:MEGNO} and examine the values of the MEGNO chaos indicator that ensue within the stochastic layer. At $a=500\,$AU, where the SDO orbital period is approximately $11{,}000$ years, the chaotic domain is characterized by $Y\sim9$. Recalling that $Y\sim2\,\Delta t/\tau_{\rm{L}}$ with $\Delta t=0.1\,$Myr, we thus obtain $\tau_{\rm{L}}\sim2\times10^{4}\,$years -- a value comparable to the orbital period. We have further checked these results with a few traditional calculations of the Lyapunov times through direct integration of the variational equations (for SDOs randomly initialized with $31<q<36$ and $a=500\,$AU; \citealt{2016MNRAS.459.2275R}) and obtained estimates of $\tau_{\rm{L}}$ that were even closer to the orbital period. The MEGNO map at $a=1000\,$AU tells a similar story: with $\Delta t = 0.3\,$Myr and a characteristic $Y\sim20$, we obtain $\tau_{\rm{L}}\sim3\times10^{4}\,$years -- a value very close to the approximately $31{,}000$ year orbital period. 

\paragraph{Diffusion Coefficient} It is well established that within a stochastic system subject to vigorous mixing, the statistical properties of the actions obey the Fokker-Plank equation \citep{1945RvMP...17..323W}. Moreover, if the system is Hamiltonian, it can be shown that the Fokker-Plank equation reduces to the conventional diffusion equation, such that all of the relevant physics is encapsulated in the diffusion coefficient, $\mathcal{D}$. 

In the quasi-linear approximation, the value of $\mathcal{D}$ can be generally estimated as the product of the Lyapunov coefficient and the square of the resonant half-width \citep{Chirikov1979, LLbook}. The physical interpretation of this relation is that the resonant half-width represents a typical stochastic ``step-size" that a trajectory attains over a single decoherence (Lyapunov) time. For the problem at hand, the resonance half width, $\Delta a/2$, follows from equation (\ref{deltaa}), and we have already shown that $\tau_{\rm{L}}$ is well-approximated by the orbital period\footnote{Similar dynamics can arise in the case of first order resonances of a high degree, where a kick received during conjunction can produce changes in action that are comparable with the resonance width \citep{1994A&A...289..972S}}. The semi-major axis diffusion coefficient thus has the form:
\begin{align}
\mathcal{D}_a\sim\frac{\Delta a^2}{4\,\tau_{\rm{L}}}&=\frac{8}{5\,\pi}\frac{\mn\,\sqrt{\G\,\Msun\,\an}}{\Msun} \nonumber \\
&\times \exp\bigg[-\frac{1}{2}\bigg(\frac{q}{\an}\bigg)^2\,\bigg].
\label{Diffcoeff}
\end{align}
Note that this expression is independent of the particle's semi-major axis, and only depends on its perihelion distance. 

As a numerical check on our assumption that $\Delta a/2$ is truly a suitable approximation for a characteristic semi-major kick experienced by an SDO over a single orbital period, we ran 500 single-orbit Sun-Neptune-SDO simulations with $q=35\,$AU, randomized phases, and semi-major axis sampled uniformly in the $a=500\pm5\,$AU range. We then measured the aphelion-to-aphelion variation in particle semi-major axes, and found a mean value of $2.32\,$AU, in good agreement with the results of \citet{2013Icar..222...20F}. This quantity is close to the theoretically predicted value of $\Delta a/2 = 2.26\,$AU, leading us to conclude that equation (\ref{Diffcoeff}) provides an adequate approximation for the semi-major axis diffusion coefficient of long-period scattered disk objects.

%TO DO: This naturally leads to a connection with the standard map. Some ideas: the Lyapunov time is simply the resonance modulation time. Does this mean that within the chaotic layer, the Lyapunov timescale is just the frequency distance between resonances, meaning that it's just the orbital period? Could be... The diffusion coefficient is trivial to calculate in a similar fashion: just the resonance width squared, divided by the Lyapunov time....

\section{Discussion} \label{sec:discuss}

Owing to the unrelenting observational mapping of the trans-Neptunian solar system that has ensued over the last two decades, the orbital structure of the scattered disk continues to come into an ever-shaper focus. Several attempts have been made to describe the stochastic dynamics of this remarkable population of minor bodies. In this vein, $1:j$ resonances have been broadly discussed in the literature as an attractive theoretical explanation for the emergent behavior of actively scattering TNOs \citep{2004AJ....128.1418P, 2018AJ....155..260V, 2019CeMDA.131...39L}. Nevertheless, a complete understanding of the evolution of long-period orbits has remained incomplete.

In this work, we have approached the problem of scattered disk dynamics from a perturbative viewpoint. In particular, we have derived a simple Hamiltonian model for the orbital motion of long-period TNOs, based upon a quadrupole-level expansion of the planetary disturbing function \citep{1962AJ.....67..300K,Laskar2010DISTFUNCT,2013MNRAS.435.2187M}. Our analysis indicates that the scattered disk's dynamical machinery is comprised of a chain of $2:j$ resonances and that their overlap is responsible for driving chaotic motion. To be clear, $1:j$ harmonics are not entirely absent from the dynamical picture, but are smaller than $2:j$ resonances by a factor of $\en$ at quadrupole order, or a factor of $\alpha$ (i.e., appearing at octupole+ order) in the $\en\rightarrow0$ limit. We further demonstrate how our theoretical model can be reduced to the Chirikov Standard Map \citep{Chirikov1979}, illuminating the physical connection between resonant perturbations and the scattering process itself.

Interpreting the intersection point among nonlinear $2:j$ resonances as the dividing line between regular and stochastic motion, we have derived an analytic stability boundary of the distant scattered disk. In practice, this criterion is given by equation (\ref{qcrit}) and translates to a critical perihelion distance below which chaos ensues. For chaotic orbits that satisfy this criterion, we have obtained analytic estimates of Lyapunov time, $\tau_{\rm{L}}$ (equation \ref{Lyapana}), and the semi-major axis diffusion coefficient, $\mathcal{D}_a$ (equation \ref{Diffcoeff}). Importantly, these calculations indicate that within the strongly chaotic domain of the scattered disk, the Lyapunov time approaches the orbital period, while the semi-major axis diffusion coefficient is on the order of Neptune's angular momentum divided by the solar mass. Our analysis further shows that the semi-major axis diffusion rate (or equivalently, the rate of energy diffusion) is insensitive to the semi-major axis itself. Instead, $\mathcal{D}_a$ only depends on the perihelion distance -- a result that is consistent with previous findings \citep{2004AJ....128.1418P, 2013Icar..222...20F}.

Although compact and easy to implement, we caution that our results only strictly apply to long-period orbits, where quadrupole-level expansion of the planetary disturbing function provides an acceptable description of the long-term dynamics. We further remind the reader of the various approximations that we have employed in our formalism. Specifically, we have neglected Neptune's eccentricity along with perturbations arising from the other planets, and have limited our analysis to a common plane. Of course, the solar system is not a 2D restricted three-body problem, meaning that our analytic estimate of the stability boundary is, by construction, inexact. Still, a comparison of our results with direct $N$-body simulations indicates that our estimates are sufficiently close to their numerically computed counterparts to provide a useful blueprint for the dynamical architecture of the distant scattered disk.

We conclude this work by remarking that the stability boundary of the scattered disk does not correspond to a single value of the perihelion distance, as is often quoted in the literature. Instead, for long-period orbits, the critical perihelion distance slowly increases with semi-major axis. In other words, the gravitational ``reach" of Neptune's exterior resonances grows with $a$, such that chaos facilitated by Neptune covers a broader perihelion range at longer periods. Taken in isolation, however, scattered disk objects still obey the conservation of the Tisserand parameter, which is well approximated by the preservation of the perihelion distance for highly eccentric long-period orbits (see Appendix). This means that objects that stochastically diffuse outward through the scattered disk do so at approximately constant $q$, and Neptune scattering alone cannot readily populate the large-$a$ chaotic parameter space with $q\gtrsim36\,$AU. For this reason, the generation of chaotic high-perihelion TNOs must be interpreted as a dynamical signature of the interplay between Neptune's exterior $2:j$ resonances and extrinsic gravitational effects that sculpt the outermost regions of the solar system.

%Beyond illuminating the underlying physics, our result makes an important prediction regarding the perihelion gap of the distant scattered disk. At present, the census of distant (sometimes called ``extreme", although the boundary between extreme and run-of-the-mill objects is not a sharp one) TNOs is divided into two populations: those with $q<50\,$AU and those with $q>65\,$AU. The dearth of objects with perihelia in the $50-60\,$AU range is routinely referred to as the perihelion gap, and its dynamical origins have been a subject of considerable interest. Our work, however, indicates that the boundary of the scattered disk itself is not limited to a single value of $q$ and SDOs with semi-major axes in excess of $a\gtrsim1500\,$AU are expected to have $q\gtrsim65\,$AU. Indeed, it would appear that separating the Census of TNOs along the stability boundary itself is a more interesting exercise than sorting them according to $q$.



\begin{acknowledgments}
We are indebted to Alessandro Morbidelli, Matt Clement, and Mike Brown for illuminating discussions, as well as to Dan Tamayo for providing a thorough and insightful referee report. We are additionally grateful to Hanno Rein for sharing his expertise in numerical implementation of chaos indicators. K.B. is grateful to Caltech, and the David and Lucile Packard Foundation for their generous support. 
\end{acknowledgments}


\begin{appendix} \label{sec:app}

In the following text, we consider the relationship between the Jacobi constant, the Tisserand parameter, the perihelion distance, and the resonant integral of motion $\Psi$. To start this discussion, let us go back to the full Hamiltonian of the circular planar restricted three-body problem: 
\begin{align}
\Ham=\frac{1}{2}\bigg(P_r^2+\frac{P_{\theta}^2}{r^2} \bigg)-\frac{\G\,\Msun}{r}+V(r,\theta-n_{\rm{p}}\,t)+\mathcal{T},
\end{align}
where $P_r$ is the specific linear momentum conjugate to $r$, $P_{\theta}$ is the specific angular momentum conjugate to the azimuthal angle $\theta$, $\mathcal{T}$ is a dummy action conjugate to $t$, $V$ is the planetary potential, and $n_{\rm{p}}$ is the planetary mean motion. 

\paragraph{The Jacobi Constant} Arguably the most fundamental integral of the restricted three-body problem is the Jacobi constant, which follows directly from the Hamiltonian. Defining a contact transformation through the type-2 generating function $\mathcal{F}_2=(r)\,P_{r}'+(\theta-n_{\rm{p}}\,t)\,P_{\theta}'+(t)\,\Xi$, we have $P_r=P_r'$, $P_{\theta}=P_{\theta}'$, and $\mathcal{T}=\Xi-n_{\rm{p}}\,P_{\theta}$. This canonical change of variables corresponds to a transition into a reference frame that co-rotates with the planet at the orbital frequency $n_{\rm{p}}$, such that the new azimuthal angle is $\theta'=\theta-n_{\rm{p}}\,t$. 

Dropping the new constant dummy action $\Xi$, the Hamiltonian is now expressed as:
\begin{align}
\Ham=\frac{1}{2}\bigg(P_r'^2+\frac{P_{\theta}'^2}{r'^2} \bigg)-\frac{\G\,\Msun}{r'}+V(r',\theta')-n_{\rm{p}}\,P_{\theta}'.
\end{align}
Because $\Ham$ now has no explicit time dependence, it is conserved. This locum of energy in a rotating frame, $\Ham$, \textit{is} the Jacobi constant (technically, the conventional expression of the Jacobi constant differs from $\Ham$ by a factor of $-2$, but this is obviously irrelevant).

\paragraph{The Tisserand Parameter}  The above expression contains one term that is guaranteed to be much smaller than others: by virtue of being proportional to the planet-star mass ratio, $V(r',\theta')$ is assuredly negligible. Accordingly, employing the usual expression for the specific energy of a Keplerian orbit and noting that $P_{\theta}'=\sqrt{\G\,\Msun\,a\,(1-e^2)}$, we obtain:
\begin{align}
\Ham \approx - \frac{\G\,\Msun}{2a}-n_{\rm{p}}\,\sqrt{\G\,\Msun\,a\,(1-e^2)} +\order\bigg(\frac{m_{\rm{p}}}{\Msun} \bigg).
\end{align}
Scaling this expression by the inverse specific energy of the planet, $-2\,a_{\rm{p}}/(\G\,\Msun))$, we obtain the Tisserand parameter:
\begin{align}
T=\alpha+2\sqrt{\frac{1-e^2}{\alpha}}.
\end{align}

\paragraph{The Perihelion Distance} As discussed in the main text of the article, distant scattered disk orbits are characterized by small semi-major axis ratios $\alpha\ll1$ and near-unity eccentricities. Accordingly, expanding the above expression for $T$ to zeroth order in $\alpha$ around $0$ and first order in $e$ around $1$, we obtain
\begin{align}
T\approx2\,\sqrt{2}\,\sqrt{\frac{1-e}{\alpha}}+\order\big(\sqrt{\alpha},(1-e)^{3/2} \big).
\end{align}
Multiplying the square of this approximate expression for the Tisserand parameter by $a_{\rm{p}}/8$, we recover the perihelion distance:
\begin{align}
q=a\,(1-e)\approx\frac{a_{\rm{p}}}{8}\,T^2.
\end{align}

\paragraph{The Resonant Integral} A key consequence of the conservation of the action $\Psi$ (defined in equation \ref{transform}) is that changes in the Delaunay actions $L=\sqrt{\G\,\Msun\,a}$ and $G=\sqrt{\G\,\Msun\,a\,(1-e^2)}$, are related through $\Delta L=\chi\,\Delta G/2$. Let us examine this relationship in further detail. Returning to the ``exact" expression for the Tisserand parameter, let us express it in terms of Delaunay variables: 
\begin{align}
T=\sqrt{\frac{G^2}{\G\,\Msun\,a_{\rm{p}}}}+\frac{\G\,\Msun\,a_{\rm{p}}}{2\,L^2}
\end{align}

Taking the finite difference, we have:
\begin{align}
\sqrt{\frac{1}{\G\,\Msun\,a_{\rm{p}}}}\,\Delta G=\frac{\G\,\Msun\,a_{\rm{p}}}{[L]^3}\,\Delta L.
\end{align}
Rearranging the expression and noting that in the vicinity of a $2:\chi$ resonance $(a/a_{\rm{p}})^{3/2}\approx\chi/2$, we obtain
\begin{align}
\Delta L=\bigg(\frac{a}{a_{\rm{p}}} \bigg)^{3/2}\, \Delta G = \frac{\chi}{2}\,\Delta G.
\end{align}
This is result is identical to the one that ensues from the conservation of $\Psi$, implying that (to within an additive constant) $\Psi$ \textit{is} a near-resonant approximation to the Tisserand parameter.

The above formulae highlight the fact that the approximate maintenance of the perihelion distance by scattered disk objects, the preservation of the resonant action we employed in our analysis, as well as the near-constancy of the Tisserand parameter -- which is itself nothing other than an approximation to the Jacobi constant -- are all re-statements of the same conservation law.

\end{appendix}


\begin{thebibliography}{00}

%A

\bibitem[Adams(2010)]{2010ARA&A..48...47A} Adams, F.~C.\ 2010, \araa, 48, 47. doi:10.1146/annurev-astro-081309-130830


%B

\bibitem[Batygin et al.(2011)]{2011ApJ...738...13B} Batygin, K., Brown, M.~E., \& Fraser, W.~C.\ 2011, \apj, 738, 13. doi:10.1088/0004-637X/738/1/13

\bibitem[Batygin \& Morbidelli(2013)]{2013A&A...556A..28B} Batygin, K. \& Morbidelli, A.\ 2013, \aap, 556, A28. doi:10.1051/0004-6361/201220907

\bibitem[Batygin et al.(2015)]{2015ApJ...799..120B} Batygin, K., Morbidelli, A., \& Holman, M.~J.\ 2015, \apj, 799, 120. doi:10.1088/0004-637X/799/2/120

\bibitem[Batygin et al.(2019)]{2019PhR...805....1B} Batygin, K., Adams, F.~C., Brown, M.~E., et al.\ 2019, \physrep, 805, 1. doi:10.1016/j.physrep.2019.01.009

\bibitem[Brown et al.(2004)]{2004ApJ...617..645B} Brown, M.~E., Trujillo, C., \& Rabinowitz, D.\ 2004, \apj, 617, 645. doi:10.1086/422095

%C

\bibitem[Casati et al.(1988)]{1988IJQE...24.1420C} Casati, G., Guarneri, I., \& Shepelianskii, D.~L.\ 1988, IEEE Journal of Quantum Electronics, 24, 1420. doi:10.1109/3.982


\bibitem[Chambers(1999)]{1999MNRAS.304..793C} Chambers, J.~E.\ 1999, \mnras, 304, 793. doi:10.1046/j.1365-8711.1999.02379.x

\bibitem[Chirikov(1959)]{1959SPhD....4..390C} Chirikov, B.~V.\ 1959, Soviet Physics Doklady, 4, 390

\bibitem[Chirikov(1979)]{Chirikov1979} Chirikov, B.~V.\ 1979, \physrep, 52, 263. doi:10.1016/0370-1573(79)90023-1

\bibitem[Chirikov \& Vecheslavov(1989)]{Chirikov1989} Chirikov, R.~V. \& Vecheslavov, V.~V.\ 1989, \aap, 221, 146

\bibitem[Cincotta \& Sim{\'o}(2000)]{MEGNO} Cincotta, P.~M. \& Sim{\'o}, C.\ 2000, \aaps, 147, 205. doi:10.1051/aas:2000108

\bibitem[Clement \& Sheppard(2021)]{2021arXiv210501065C} Clement, M.~S. \& Sheppard, S.~S.\ 2021, arXiv:2105.01065


%D

\bibitem[Duncan et al.(1998)]{1998AJ....116.2067D} Duncan, M.~J., Levison, H.~F., \& Lee, M.~H.\ 1998, \aj, 116, 2067. doi:10.1086/300541


%E

\bibitem[Ellis \& Murray(2000)]{2000Icar..147..129E} Ellis, K.~M. \& Murray, C.~D.\ 2000, \icarus, 147, 129. doi:10.1006/icar.2000.6399


%F

\bibitem[Fouchard et al.(2013)]{2013Icar..222...20F} Fouchard, M., Rickman, H., Froeschl{\'e}, C., et al.\ 2013, \icarus, 222, 20. doi:10.1016/j.icarus.2012.10.027

%G

\bibitem[Gomes et al.(2008)]{Gomes2008} Gomes, R.~S., Fern{\'a}ndez, J.~A., Gallardo, T., et al.\ 2008, The Solar System Beyond Neptune, 259

\bibitem[Gontis \& Kaulakys(1987)]{1987JPhB...20.5051G} Gontis, V. \& Kaulakys, B.\ 1987, Journal of Physics B Atomic Molecular Physics, 20, 5051. doi:10.1088/0022-3700/20/19/016


%H

\bibitem[Hansen(1885)]{Hansen1885} Hansen, P.~A.,\ 1885, Abhandlungen der Mathematisch-Physischen Class der Königlich Sachsischen Gesselschaft der Wissenschaften, vol. 2, pp. 181–281. Leipzig

\bibitem[Hagihara(1970)]{1970ceme.book.....H} Hagihara, Y.\ 1970, Cambridge, Mass.: Massachusetts Institute of Technology (MIT), 1970

\bibitem[Holman \& Murray(1996)]{1996AJ....112.1278H} Holman, M.~J. \& Murray, N.~W.\ 1996, \aj, 112, 1278. doi:10.1086/118098

\bibitem[Hughes(1981)]{1981CeMec..25..101H} Hughes, S.\ 1981, Celestial Mechanics, 25, 101. doi:10.1007/BF01301812


%I

%J

\bibitem[Jensen et al.(1988)]{1988JPhB...21L.527J} Jensen, R.~V., Leopold, J.~G., \& Richards, D.\ 1988, Journal of Physics B Atomic Molecular Physics, 21, L527. doi:10.1088/0953-4075/21/17/001


%K

\bibitem[Kaula(1962)]{1962AJ.....67..300K} Kaula, W.~M.\ 1962, \aj, 67, 300. doi:10.1086/108729

\bibitem[Khain et al.(2020)]{2020PASP..132l4401K} Khain, T., Becker, J.~C., \& Adams, F.~C.\ 2020, \pasp, 132, 124401. doi:10.1088/1538-3873/abbd8a

\bibitem[Khriplovich \& Shepelyansky(2009)]{2009IJMPD..18.1903K} Khriplovich, I.~B. \& Shepelyansky, D.~L.\ 2009, International Journal of Modern Physics D, 18, 1903. doi:10.1142/S0218271809015758

%L

\bibitem[Lan \& Malhotra(2019)]{2019CeMDA.131...39L} Lan, L. \& Malhotra, R.\ 2019, Celestial Mechanics and Dynamical Astronomy, 131, 39. doi:10.1007/s10569-019-9917-1

\bibitem[Laskar \& Robutel(1995)]{1995CeMDA..62..193L} Laskar, J. \& Robutel, P.\ 1995, Celestial Mechanics and Dynamical Astronomy, 62, 193. doi:10.1007/BF00692088

\bibitem[Laskar(2008)]{Laskar2008} Laskar, J.\ 2008, \icarus, 196, 1. doi:10.1016/j.icarus.2008.02.017

\bibitem[Laskar \& Bou{\'e}(2010)]{Laskar2010DISTFUNCT} Laskar, J. \& Bou{\'e}, G.\ 2010, \aap, 522, A60. doi:10.1051/0004-6361/201014496

\bibitem[Lichtenberg \& Lieberman(1992)]{LLbook} Lichtenberg, A. \& Lieberman, M.\ 1992, Regular and Chaotic Dynamics, Second Edition, by A. Lichtenberg and M. Lieberman. Springer-Verlag, New York, ISBN 0-387-97745-7, 1992

\bibitem[Lithwick \& Wu(2011)]{2011ApJ...739...31L} Lithwick, Y. \& Wu, Y.\ 2011, \apj, 739, 31. doi:10.1088/0004-637X/739/1/31



%M

\bibitem[Malyshkin \& Tremaine(1999)]{Malyshkin1999} Malyshkin, L. \& Tremaine, S.\ 1999, \icarus, 141, 341. doi:10.1006/icar.1999.6174


\bibitem[Mardling(2008)]{Mardling2008Chaos} Mardling, R.~A.\ 2008, The Cambridge N-Body Lectures, 59. doi:10.1007/978-1-4020-8431-7\_3

\bibitem[Mardling(2013)]{2013MNRAS.435.2187M} Mardling, R.~A.\ 2013, \mnras, 435, 2187. doi:10.1093/mnras/stt1438

\bibitem[Morbidelli(2002)]{Morbybook} Morbidelli, A.\ 2002, Modern celestial mechanics : aspects of solar system dynamics. London: Taylor \& Francis, ISBN 0415279399

%\bibitem[Morbidelli \& Levison(2004)]{Morby2004} Morbidelli, A. \& Levison, H.~F.\ 2004, \aj, 128, 2564. doi:10.1086/424617

\bibitem[Morbidelli \& Nesvorn{\'y}(2020)]{2020tnss.book...25M} Morbidelli, A. \& Nesvorn{\'y}, D.\ 2020, The Trans-Neptunian Solar System, 25. doi:10.1016/B978-0-12-816490-7.00002-3

\bibitem[Murray \& Holman(1997)]{1997AJ....114.1246M} Murray, N. \& Holman, M.\ 1997, \aj, 114, 1246. doi:10.1086/118558

\bibitem[Murray \& Dermott(1999)]{MD99} Murray, C.~D. \& Dermott, S.~F.\ 1999, Solar system dynamics. Cambridge, UK: Cambridge University Press, ISBN 0-521-57295-9

%N

\bibitem[Nesvorn{\'y} \& Morbidelli(1998)]{1998CeMDA..71..243N} Nesvorn{\'y}, D. \& Morbidelli, A.\ 1998, Celestial Mechanics and Dynamical Astronomy, 71, 243. doi:10.1023/A:1008347020890

\bibitem[Nesvorn{\'y} \& Vokrouhlick{\'y}(2016)]{2016ApJ...825...94N} Nesvorn{\'y}, D. \& Vokrouhlick{\'y}, D.\ 2016, \apj, 825, 94. doi:10.3847/0004-637X/825/2/94

\bibitem[Nesvorn{\'y} et al.(2017)]{Nesvorny2017} Nesvorn{\'y}, D., Vokrouhlick{\'y}, D., Dones, L., et al.\ 2017, \apj, 845, 27. doi:10.3847/1538-4357/aa7cf6

\bibitem[Nesvorn{\'y}(2018)]{Nesvorny2018REV} Nesvorn{\'y}, D.\ 2018, \araa, 56, 137. doi:10.1146/annurev-astro-081817-052028

%O

%P

\bibitem[Pan \& Sari(2004)]{2004AJ....128.1418P} Pan, M. \& Sari, R.\ 2004, \aj, 128, 1418. doi:10.1086/423214

\bibitem[Petrosky(1986)]{1986PhLA..117..328P} Petrosky, T.~Y.\ 1986, Physics Letters A, 117, 328. doi:10.1016/0375-9601(86)90673-0


%Q

%R

\bibitem[Rein \& Tamayo(2016)]{2016MNRAS.459.2275R} Rein, H. \& Tamayo, D.\ 2016, \mnras, 459, 2275. doi:10.1093/mnras/stw644

\bibitem[Rein et al.(2019a)]{2019MNRAS.485.5490R} Rein, H., Hernandez, D.~M., Tamayo, D., et al.\ 2019, \mnras, 485, 5490. doi:10.1093/mnras/stz769

\bibitem[Rein et al.(2019b)]{2019MNRAS.489.4632R} Rein, H., Tamayo, D., \& Brown, G.\ 2019, \mnras, 489, 4632. doi:10.1093/mnras/stz2503


%S

\bibitem[Sadov(2006)]{Sadov2006} Sadov, S.~Y.\ 2006, Cosmic Research, 44, 160. https://doi.org/10.1134/S0010952506020080

\bibitem[Sadov(2008)]{2008CeMDA.100..287S} Sadov, S.~Y.\ 2008, Celestial Mechanics and Dynamical Astronomy, 100, 287. doi:10.1007/s10569-008-9123-z

\bibitem[Saillenfest(2020)]{Saillenfest2020} Saillenfest, M.\ 2020, Celestial Mechanics and Dynamical Astronomy, 132, 12. doi:10.1007/s10569-020-9954-9

\bibitem[Shepelyansky(1994)]{1994PhRvA..50..575S} Shepelyansky, D.~L.\ 1994, \pra, 50, 575. doi:10.1103/PhysRevA.50.575

\bibitem[Shevchenko(1998)]{1998PhLA..241...53S} Shevchenko, I.~I.\ 1998, Physics Letters A, 241, 53. doi:10.1016/S0375-9601(98)00093-0

\bibitem[Shevchenko(2011)]{2011NewA...16...94S} Shevchenko, I.~I.\ 2011, \na, 16, 94. doi:10.1016/j.newast.2010.06.008

\bibitem[Sidlichovsky \& Nesvorny(1994)]{1994A&A...289..972S} Sidlichovsky, M. \& Nesvorny, D.\ 1994, \aap, 289, 972


%T

\bibitem[Tsiganis et al.(2005)]{2005Natur.435..459T} Tsiganis, K., Gomes, R., Morbidelli, A., et al.\ 2005, \nat, 435, 459. doi:10.1038/nature03539


%U

%V

\bibitem[Volk et al.(2018)]{2018AJ....155..260V} Volk, K., Murray-Clay, R.~A., Gladman, B.~J., et al.\ 2018, \aj, 155, 260. doi:10.3847/1538-3881/aac268

%W

\bibitem[Wang \& Uhlenbeck(1945)]{1945RvMP...17..323W} Wang, M.~C. \& Uhlenbeck, G.~E.\ 1945, Reviews of Modern Physics, 17, 323. doi:10.1103/RevModPhys.17.323

\bibitem[Wisdom(1980)]{1980AJ.....85.1122W} Wisdom, J.\ 1980, \aj, 85, 1122. doi:10.1086/112778

\bibitem[Wisdom(1982)]{Wisdom1982} Wisdom, J.\ 1982, \aj, 87, 577. doi:10.1086/113132

\bibitem[Wisdom \& Holman(1991)]{1991AJ....102.1528W} Wisdom, J. \& Holman, M.\ 1991, \aj, 102, 1528. doi:10.1086/115978


%X

%Y

%Z


\end{thebibliography}

%\bibliography{apssamp}% Produces the bibliography via BibTeX.



\end{document}  
% \appendices
% \section{Proofs}
\subsection{Proof of Theorem \ref{th:inexactLS1}}
We start from a similar argument as in \cite[proof of Therorem~2]{Blumen}. 
%proof of \cite[Theorem~2]{Blumen}. 
Set $g := 2\nabla f(x^{k-1})=2A^T(Ax^{k-1}-y)$ and $\g:=2\nablaa^{\nug} f(x^{k-1})= g+2\eg^k$ for some vector $\eg^k$ which by definition~\eqref{eq:grad} is bounded $\norm{\eg^k}\leq \nug^k$. It follows that
\ifCLASSOPTIONtwocolumn
\begin{align*} 
&\norm{y-Ax^k}^2-\norm{y-Ax^{k-1}}^2	\\
&= \langle x^k-x^{k-1},g \rangle +\norm{A(x^k-x^{k-1})}^2 \\
&\leq \langle x^k-x^{k-1},g \rangle + \MM \norm{x^k-x^{k-1}}^2, 
\end{align*}
\else
\begin{align*} 
\norm{y-Ax^k}^2-\norm{y-Ax^{k-1}}^2	&= \langle x^k-x^{k-1},g \rangle +\norm{A(x^k-x^{k-1})}^2 \\
&\leq \langle x^k-x^{k-1},g \rangle + \MM \norm{x^k-x^{k-1}}^2, 
\end{align*}
\fi
where the last inequality follows from the ULE property in Definition \ref{def:Lip}. Assuming $\MM \leq 1/\mu$, we have
\ifCLASSOPTIONtwocolumn
\begin{align*}
&\langle x^k-x^{k-1},g \rangle + \MM \norm{x^k-x^{k-1}}^2 \\
& \leq \langle x^k-x^{k-1},g \rangle + \frac{1}{\mu} \norm{x^k-x^{k-1}}^2\\
&= \langle x^k-x^{k-1},\g \rangle + \frac{1}{\mu} \norm{x^k-x^{k-1}}^2 - \langle x^k-x^{k-1},2\eg^k \rangle\\
& = \frac{1}{\mu} \norm{x^k-x^{k-1}+\frac{\mu}{2} \g }^2 - \frac{\mu}{4} \norm{\g}^2 - \langle x^k-x^{k-1},2\eg^k \rangle.
\end{align*}
\else
\begin{align*}
\langle x^k-x^{k-1},g \rangle + \MM \norm{x^k-x^{k-1}}^2 
& \leq \langle x^k-x^{k-1},g \rangle + \frac{1}{\mu} \norm{x^k-x^{k-1}}^2\\
&= \langle x^k-x^{k-1},\g \rangle + \frac{1}{\mu} \norm{x^k-x^{k-1}}^2 - \langle x^k-x^{k-1},2\eg^k \rangle\\
& = \frac{1}{\mu} \norm{x^k-x^{k-1}+\frac{\mu}{2} \g }^2 - \frac{\mu}{4} \norm{\g}^2 - \langle x^k-x^{k-1},2\eg^k \rangle.
\end{align*}
\fi
Due to the update rule of Algorithm \eqref{eq:inIP} and the inexact (fixed-precision) projection step, we have
\ifCLASSOPTIONtwocolumn
\begin{align*}
	&\norm{x^k-x^{k-1}+\frac{\mu}{2} \g }^2 \\
	&\leq  \norm{\pp_{\Cc}(x^{k-1}-\frac{\mu}{2} \g)-x^{k-1}+\frac{\mu}{2} \g }^2 +(\nup^k)^2\\
	&\leq \norm{x^\gt-x^{k-1}+\frac{\mu}{2} \g }^2 +(\nup^k)^2.
\end{align*}
\else
\begin{align*}
\norm{x^k-x^{k-1}+\frac{\mu}{2} \g }^2 
&\leq  \norm{\pp_{\Cc}(x^{k-1}-\frac{\mu}{2} \g)-x^{k-1}+\frac{\mu}{2} \g }^2 +(\nup^k)^2\\
&\leq \norm{x^\gt-x^{k-1}+\frac{\mu}{2} \g }^2 +(\nup^k)^2.
\end{align*}
\fi
The last inequality holds for any member of $\Cc$ and thus here for $x^\gt$. Therefore we can write
\ifCLASSOPTIONtwocolumn
\begin{align}
&\norm{y-Ax^k}^2-\norm{y-Ax^{k-1}}^2 \nonumber	\\
%&=\langle x^{t+1}-x^t,g \rangle + \frac{1}{\mu} \norm{x^{t+1}-x^t}^2 \nonumber\\
&\leq \frac{1}{\mu} \norm{x^\gt-x^{k-1}+\frac{\mu}{2} \g }^2 - \frac{\mu}{4} \norm{\g}^2 \nonumber\\
&\qquad - \langle x^k-x^{k-1},2\eg^k \rangle +(\frac{\nup^k}{\sqrt\mu})^2 \nonumber \\
&= \langle x^\gt-x^{k-1},\g \rangle + \frac{1}{\mu} \norm{x^\gt-x^{k-1}}^2 \nonumber\\
&\qquad - \langle x^k-x^{k-1},2\eg^k \rangle 
 +(\frac{\nup^k}{\sqrt\mu})^2\nonumber\\
%&= \langle x^\gt-x^{k-1},g \rangle + \frac{1}{\mu} \norm{x^\gt-x^{k-1}}^2 - \langle x^k-x^*,2\eg^k \rangle +\frac{\nup^k}{\mu}\nonumber\\
&\leq \langle x^\gt-x^{k-1},g \rangle + \frac{1}{\mu} \norm{x^\gt-x^{k-1}}^2 \nonumber\\
&\qquad+2\nug^k\norm{x^k-x^*} +(\frac{\nup^k}{\sqrt\mu})^2. \label{eq:p1b2}
\end{align}
\else
\begin{align}
\norm{y-Ax^k}^2-\norm{y-Ax^{k-1}}^2 \nonumber	
%&=\langle x^{t+1}-x^t,g \rangle + \frac{1}{\mu} \norm{x^{t+1}-x^t}^2 \nonumber\\
&\leq \frac{1}{\mu} \norm{x^\gt-x^{k-1}+\frac{\mu}{2} \g }^2 - \frac{\mu}{4} \norm{\g}^2 
 - \langle x^k-x^{k-1},2\eg^k \rangle +(\frac{\nup^k}{\sqrt\mu})^2 \nonumber \\
&= \langle x^\gt-x^{k-1},\g \rangle + \frac{1}{\mu} \norm{x^\gt-x^{k-1}}^2 
 - \langle x^k-x^{k-1},2\eg^k \rangle 
+(\frac{\nup^k}{\sqrt\mu})^2\nonumber\\
%&= \langle x^\gt-x^{k-1},g \rangle + \frac{1}{\mu} \norm{x^\gt-x^{k-1}}^2 - \langle x^k-x^*,2\eg^k \rangle +\frac{\nup^k}{\mu}\nonumber\\
&\leq \langle x^\gt-x^{k-1},g \rangle + \frac{1}{\mu} \norm{x^\gt-x^{k-1}}^2 
+2\nug^k\norm{x^k-x^*} +(\frac{\nup^k}{\sqrt\mu})^2. \label{eq:p1b2}
\end{align}
\fi
The last line replaces $\g= g+2\eg^k$ and uses the Cauchy-Schwartz inequality. 


Similarly we use the LLE property in Definition \ref{def:Lip} to obtain an upper bound on $ \langle x^\gt-x^{k-1},g \rangle$:
\begin{align*} 
\langle x^\gt-x^{k-1},g \rangle 	&= w^2-\norm{y-Ax^{k-1}}^2 +\norm{A(x_0-x^{k-1})}^2 \\
&\leq w^2 -\norm{y-Ax^{k-1}}^2 +\mmx\norm{x^\gt-x^{k-1}}^2,
\end{align*}
where $w=\norm{ y-Ax^\gt}$. Replacing this bound in \eqref{eq:p1b2} and cancelling $-\norm{y-Ax^{k-1}}^2$ from both sides of the inequality yields
\ifCLASSOPTIONtwocolumn
\begin{align}
&\norm{y-Ax^k}^2- 2\nug^k\norm{x^k-x^\gt}\nonumber \\ 
&\leq \left(\frac{1}{\mu}-\mmx \right)\norm{x^{k-1}-x^\gt}^2 + (\frac{\nup^k}{\sqrt\mu})^2+w^2. \label{eq:p1b3}
\end{align}
\else
\begin{align}
\norm{y-Ax^k}^2- 2\nug^k\norm{x^k-x^\gt}
\leq \left(\frac{1}{\mu}-\mmx \right)\norm{x^{k-1}-x^\gt}^2 + (\frac{\nup^k}{\sqrt\mu})^2+w^2. \label{eq:p1b3}
\end{align}
\fi
We continue to lower bound the left-hand side of this inequality:
\ifCLASSOPTIONtwocolumn
\begin{align*}
&\norm{y-Ax^k}^2- 2\nug^k\norm{x^k-x^\gt}\\
&= \norm{A(x^k-x^\gt)}^2+w^2-2\langle y-Ax^\gt, A(x^k-x^\gt)\rangle\\
&- 2\nug^k\norm{x^k-x^\gt} \\
&\geq \norm{A(x^k-x^\gt)}^2+w^2-2w \norm{A(x^k-x^\gt)}\\
&- 2\nug^k\norm{x^k-x^\gt} \\
& \geq \mmx\norm{x^k-x^\gt}^2+w^2-2(w \sqrt{\MM}+\nug^k)\norm{x^k-x^\gt}\\
&= \left(\sqrt{\mmx}\norm{x^k-x^\gt}-\frac{\nug^k}{\sqrt{\mmx}}- \sqrt{\frac{\MM}{\mmx}}w\right)^2 \\
&- (\frac{\nug^k}{\sqrt{\mmx}})^2 -(\frac{\MM}{\mmx}-1)w^2.
\end{align*}
\else
\begin{align*}
\norm{y-Ax^k}^2- 2\nug^k\norm{x^k-x^\gt}
&= \norm{A(x^k-x^\gt)}^2+w^2-2\langle y-Ax^\gt, A(x^k-x^\gt)\rangle- 2\nug^k\norm{x^k-x^\gt} \\
&\geq \norm{A(x^k-x^\gt)}^2+w^2-2w \norm{A(x^k-x^\gt)}
- 2\nug^k\norm{x^k-x^\gt} \\
& \geq \mmx\norm{x^k-x^\gt}^2+w^2-2(w \sqrt{\MM}+\nug^k)\norm{x^k-x^\gt}\\
&= \left(\sqrt{\mmx}\norm{x^k-x^\gt}-\frac{\nug^k}{\sqrt{\mmx}}- \sqrt{\frac{\MM}{\mmx}}w\right)^2 - (\frac{\nug^k}{\sqrt{\mmx}})^2 -(\frac{\MM}{\mmx}-1)w^2.
\end{align*}
\fi
The first inequality uses the Cauchy-Schwartz's and the second inequality follows from the ULE and LLE properties. Using this bound together with \eqref{eq:p1b3} we get
\ifCLASSOPTIONtwocolumn
\begin{align*}
&\left(\sqrt{\mmx}\norm{x^k-x^\gt}-\frac{\nug^k}{\sqrt{\mmx}}- \sqrt{\frac{\MM}{\mmx}}w\right)^2\\
&\leq \left(\frac{1}{\mu}-\mmx \right)\norm{x^{k-1}-x^\gt}^2 + (\frac{\nug^k}{\sqrt{\mmx}})^2+ (\frac{\nup^k}{\sqrt\mu})^2+\frac{\MM}{\mmx}w^2 \\
&\leq \left(\sqrt{\frac{1}{\mu}-\mmx} \norm{x^{k-1}-x^\gt} + \frac{\nug^k}{\sqrt{\mmx}}+ \frac{\nup^k}{\sqrt\mu}+\sqrt{\frac{\MM}{\mmx}}w \right)^2.
\end{align*}
\else
\begin{align*}
\left(\sqrt{\mmx}\norm{x^k-x^\gt}-\frac{\nug^k}{\sqrt{\mmx}}- \sqrt{\frac{\MM}{\mmx}}w\right)^2
&\leq \left(\frac{1}{\mu}-\mmx \right)\norm{x^{k-1}-x^\gt}^2 + (\frac{\nug^k}{\sqrt{\mmx}})^2+ (\frac{\nup^k}{\sqrt\mu})^2+\frac{\MM}{\mmx}w^2 \\
&\leq \left(\sqrt{\frac{1}{\mu}-\mmx} \norm{x^{k-1}-x^\gt} + \frac{\nug^k}{\sqrt{\mmx}}+ \frac{\nup^k}{\sqrt\mu}+\sqrt{\frac{\MM}{\mmx}}w \right)^2.
\end{align*}
\fi
The last inequality assumes $\mu\leq \mmx^{-1}$ which holds since we previously assumed $\mu\leq \MM^{-1}$. As a result we deduce that
\begin{align}
\norm{x^k-x^\gt}\leq \rho \norm{x^{k-1}-x^\gt} + \nut^k + 2\frac{\sqrt{\MM}}{\mmx}w \label{eq:p1b4}
\end{align}
for $\rho$ and $\nut^k$ defined in Theorem \ref{th:inexactLS1}. Applying this bound recursively (and setting $x^0=0$) completes the proof:
\begin{align*}
\norm{x^k-x^\gt}\leq \rho^k \norm{x^\gt} + \sum_{i=1}^k \rho^{k-i} \nut^i + \frac{2\sqrt{\MM}}{\mmx(1-\rho)}w.
\end{align*} 
Note that for convergence we require $\rho<1$ and therefore, a lower bound on the step size which is $\mu> (2\mmx)^{-1}$. 

\subsection{Proof of Corollary~\ref{cor:decay}}
Following the error bound \eqref{eq:errbound} derived in   Theorem~\ref{th:inexactLS1} and by setting $\nut^k\leq C r^k$ we obtain:
		\eq{
			\norm{x^{k}-x^\gt}\leq  \rho^k \left(\norm{x^\gt}+C\sum_{i=1}^k (r/\rho)^{i}  \right)+ \frac{2\sqrt{\MM}}{\mmx(1-\rho)}w,			
		}
which for $r<\rho$ it implies 		
		\eq{
			\norm{x^{k}-x^\gt}\leq 
			 \rho^k \left(\norm{x^\gt}+\frac{C}{1-r/\rho}\right)+ \frac{2\sqrt{\MM}}{\mmx(1-\rho)}w,
		}
and for $r>\rho$ implies %and following \eqref{eq:errbound} we get		
\begin{align*}
\norm{x^{k}-x^\gt}&\leq  \rho^k \norm{x^\gt}+C r^k \sum_{i=1}^k (\rho/r)^{k-i}  + \frac{2\sqrt{\MM}}{\mmx(1-\rho)}w\\
&\leq r^k \left(\norm{x^\gt}+\frac{C}{1-\rho/r}\right)+ \frac{2\sqrt{\MM}}{\mmx(1-\rho)}w,	
\end{align*}
and for $r=\rho$ we immediately get
\eq{
\norm{x^{k}-x^\gt}\leq  \rho^k \norm{x^\gt}+C k \rho^k + \frac{2\sqrt{\MM}}{\mmx(1-\rho)}w.	
}
Note that there exists a constant $c$ such that for an arbitrary small $\xi>0$ it holds $k\rho^k\leq c(\rho+\xi)^k$. Therefore we also achieve a linear convergence for the case $r=\rho$.
\subsection{Proof of Theorem \ref{th:inexactLS2}}
As before set $g= 2A^T(Ax^{k-1}-y)$ and $\g= g+2\eg^k$ for some bounded gradient error vector $\eg^k$ i.e. $\norm{\eg^k}\leq \nug^k$. Note that 
here the update rule of Algorithm \eqref{eq:inIP2} uses the  $(1+\epsilon)$-approximate projection  which by definition \eqref{eq:eproj} implies
\ifCLASSOPTIONtwocolumn
\begin{align*}
&\norm{x^k-x^{k-1}+\frac{\mu}{2} \g }^2 =  \norm{\pp^{\epsilon}_{\Cc}(x^{k-1}-\frac{\mu}{2} \g)-x^{k-1}+\frac{\mu}{2} \g }^2\\
&\leq  (1+\epsilon)^2\norm{\pp_{\Cc}(x^{k-1}-\frac{\mu}{2} \g)-x^{k-1}+\frac{\mu}{2} \g }^2\\
&\leq \norm{x^\gt-x^{k-1}+\frac{\mu}{2} \g }^2 + \phi(\epsilon)^2\frac{\mu^2}{4}\norm{\g}^2
\end{align*}
\else
\begin{align*}
\norm{x^k-x^{k-1}+\frac{\mu}{2} \g }^2 &=  \norm{\pp^{\epsilon}_{\Cc}(x^{k-1}-\frac{\mu}{2} \g)-x^{k-1}+\frac{\mu}{2} \g }^2\\
&\leq  (1+\epsilon)^2\norm{\pp_{\Cc}(x^{k-1}-\frac{\mu}{2} \g)-x^{k-1}+\frac{\mu}{2} \g }^2\\
&\leq \norm{x^\gt-x^{k-1}+\frac{\mu}{2} \g }^2 + \phi(\epsilon)^2\frac{\mu^2}{4}\norm{\g}^2
\end{align*}
\fi
where $\phi(\epsilon):=\sqrt{2\epsilon+\epsilon^2}$. For the last inequality we replace $\pp_{\Cc}(x^{k-1}-\frac{\mu}{2} \g)$ with two feasible points $x^\gt,x^{k-1}\in \Cc$. 

As a result by only replacing $\nug^k$ with $\mu\phi(\epsilon)\norm{\g}/2$, we can follow identical steps as for the proof of Theorem \ref{th:inexactLS1} up to \eqref{eq:p1b4}, revise the definition of $\nut^k:={2\nug^k}/{\mmx} + {\sqrt{\mu}\phi(\epsilon)\norm{\g}}/(2\sqrt{{\mmx}})$ and write
\ifCLASSOPTIONtwocolumn
\begin{align*}
\norm{x^k-x^\gt}\leq& \sqrt{\frac{1}{\mu\mmx}-1} \norm{x^{k-1}-x^\gt} \\
&+ \frac{2\nug^k}{\mmx} +\frac{\phi(\epsilon)}{2} \sqrt{\frac{\mu}{\mmx}}\norm{\g} + 2\frac{\sqrt{\MM}}{\mmx}w. 
\end{align*}
\else
\begin{align*}
\norm{x^k-x^\gt}\leq \sqrt{\frac{1}{\mu\mmx}-1} \norm{x^{k-1}-x^\gt} 
+ \frac{2\nug^k}{\mmx} +\frac{\phi(\epsilon)}{2} \sqrt{\frac{\mu}{\mmx}}\norm{\g} + 2\frac{\sqrt{\MM}}{\mmx}w. 
\end{align*}
\fi
Note that so far we only assumed $\mu\leq \MM^{-1}$. 

On the other hand by triangle inequality we have
\begin{align*}
	\norm{\g}&\leq \norm{g}+2\nug^k\\
	&\leq 2\norm{A^TA(x^{k-1}-x^\gt)}+2\norm{A^T(y-Ax^\gt)}+2\nug^k \\
	&\leq 2\sqrt \MM\vertiii{A}\norm{(x^{k-1}-x^\gt)}+2\vertiii{A}w+2\nug^k\\
	&\leq 2\sqrt{ 1/\mu}\vertiii{A}\norm{(x^{k-1}-x^\gt)}+2\vertiii{A}w+2\nug^k.
\end{align*}
The third inequality uses the ULE property and the last one holds since $\mu\leq \MM^{-1}$.
Therefore, we get
\ifCLASSOPTIONtwocolumn
\begin{align*}
&\norm{x^k-x^\gt}\leq
\left(\sqrt{\frac{1}{\mu\mmx}-1}+ \phi(\epsilon)\frac{\vertiii{A}}{\sqrt{\mmx}}\right) \norm{x^{k-1}-x^\gt} \\
&+ \left( \frac{2}{\mmx} +\phi(\epsilon){\sqrt{\frac{\mu}{\mmx}}}\right) \nug^k 
+ \left( 2\frac{\sqrt{\MM}}{\mmx}+ \phi(\epsilon)\sqrt{\frac{\mu}{\mmx}} \vertiii{A} \right)w. 
\end{align*}
\else
\begin{align*}
\norm{x^k-x^\gt}\leq&
\left(\sqrt{\frac{1}{\mu\mmx}-1}+ \phi(\epsilon)\frac{\vertiii{A}}{\sqrt{\mmx}}\right) \norm{x^{k-1}-x^\gt} \\
&+ \left( \frac{2}{\mmx} +\phi(\epsilon){\sqrt{\frac{\mu}{\mmx}}}\right) \nug^k 
+ \left( 2\frac{\sqrt{\MM}}{\mmx}+ \phi(\epsilon)\sqrt{\frac{\mu}{\mmx}} \vertiii{A} \right)w. 
\end{align*}
\fi
Based on assumption $\phi(\epsilon)\frac{\vertiii{A}}{\sqrt{\mmx}}\leq \delta$ of the theorem  we can deduce
\ifCLASSOPTIONtwocolumn
\begin{align*}
\norm{x^k-x^\gt}\leq&
\rho \norm{x^{k-1}-x^\gt} + \left( \frac{2}{\mmx} +\frac{\sqrt \mu}{\vertiii{A}} \delta\right) \nug^k \\
&
+ \left( 2\frac{\sqrt{\MM}}{\mmx}+\sqrt{\mu}\delta \right)w 
\end{align*}
\else
\begin{align*}
\norm{x^k-x^\gt}\leq
\rho \norm{x^{k-1}-x^\gt} + \left( \frac{2}{\mmx} +\frac{\sqrt \mu}{\vertiii{A}} \delta\right) \nug^k 
+ \left( 2\frac{\sqrt{\MM}}{\mmx}+\sqrt{\mu}\delta \right)w 
\end{align*}
\fi
where $\rho=\sqrt{\frac{1}{\mu\mmx}-1}+\delta$.

Applying this bound recursively (and setting $x^0=0$) completes the proof:
\eq{
\norm{x^{k}-x^\gt}\leq  \rho^k \norm{x^\gt}+\kappa_g \sum_{i=1}^k \rho^{k-i} \nug^i+ \frac{\kappa_w}{1-\rho}w
}
for $\kappa_g, \kappa_w$ defined in Theorem \ref{th:inexactLS2}. The condition for convergence is $\rho<1$ which implies $\delta<1$ and a lower bound on the step size which is $\mu> (\mmx+(1-\delta)^2\mmx)^{-1}$. 
		


\bibliographystyle{IEEEtran}
\bibliography{bibl}
\end{document}