\documentclass[conference]{IEEEtran}
\usepackage[export]{adjustbox}
\usepackage{soul}
\usepackage{colortbl}
\usepackage{xcolor}
\usepackage{tabularx}
\usepackage{hhline}
\usepackage{pgfplots}
\usepackage{pgfplotstable}
\pgfplotsset{compat=1.7}
\usetikzlibrary{dsp,chains,arrows,calc,positioning}
\usetikzlibrary{shapes.multipart, decorations.pathreplacing}
\usepackage{subfigure}
% \usepackage{subfig}
\usepackage{mathtools}
\usepackage{booktabs} 
\usepackage{amsmath,url}
\let\Bbbk\relax
\usepackage{amssymb}
 
\usepackage{amsfonts}
% \usepackage{ctable}
\usepackage{multirow}
% \usepackage{algorithm}
% \usepackage{algpseudocode}
% \usepackage{pifont}
\usepackage{color}
% \usepackage{bbm}
\usepackage{enumitem}
\usepackage{dsfont}
\usepackage{graphicx}
\usepackage{subcaption}
% \let\comment\undefined
% \usepackage[commentmarkup=margin]{changes}
% NOTE: I have to undefined \comment since I want to use the \comment environment
% provided by the verbatim package




\newcommand{\todo}[1]{\textcolor{blue}{\bf #1}}
\newcommand{\fixme}[1]{\textcolor{red}{\bf #1}}

\newcommand{\mc}[3]{\multicolumn{#1}{#2}{#3}}
\newcommand{\mr}[2]{\multirow{#1}{0.10\textheight}{#2}}
% \newcommand{\he}[1]{{\textsf{\textcolor{red}{[From He: #1]}}}}

\newcommand{\mylistbegin}{
  \begin{list}{$\bullet$}
   {
     \setlength{\itemsep}{-2pt}
     \setlength{\leftmargin}{1em}
     \setlength{\labelwidth}{1em}
     \setlength{\labelsep}{0.5em} } }
\newcommand{\mylistend}{
   \end{list}  }

\newcommand{\eg}{\textit{e.g.}}
\newcommand{\xeg}{\textit{E.g.}}
\newcommand{\ie}{\textit{i.e.}}
\newcommand{\etc}{\textit{etc}}
\newcommand{\etal}{\textit{et al.}}
\newcommand{\wrt}{\textit{w.r.t.~}}
\newcommand{\header}[1]{{\vspace{+1mm}\flushleft \textbf{#1}}}
\newcommand{\sheader}[1]{{\flushleft \textit{#1}}}
\newcommand{\CGIR}{\textit{CGIR}}

\newcommand{\floor}[1]{\lfloor #1 \rfloor}
\newcommand{\ceil}[1]{\lceil #1 \rceil}

\newcommand{\bx}{\boldsymbol{x}}
\newcommand{\by}{\boldsymbol{y}}
\newcommand{\ba}{\boldsymbol{a}}
\newcommand{\bw}{\boldsymbol{w}}
\newcommand{\bW}{\boldsymbol{W}}
\newcommand{\bfn}{\boldsymbol{f}}
\newcommand{\blambda}{\boldsymbol{\lambda}}
\newcommand{\btheta}{\boldsymbol{\theta}}

\newcommand{\mcW}{\mathcal{W}}
\newcommand{\mcY}{\mathcal{Y}}
\newcommand{\mcS}{\mathcal{S}}
\newcommand{\mcA}{\mathcal{A}}
\newcommand{\mcV}{\mathcal{V}}
\newcommand{\mcE}{\mathcal{E}}
\newcommand{\mcG}{\mathcal{G}}

\DeclareMathOperator*{\argmax}{arg\,max}
\DeclareMathOperator*{\argmin}{arg\,min}

\let\comment\undefined
\usepackage{tikz}
\usepackage{circuitikz}
\usetikzlibrary{dsp,chains,arrows,calc,positioning}
\usetikzlibrary{shapes.multipart, decorations.pathreplacing}

\usetikzlibrary{arrows}
\usepackage{cite}
\usepackage{amsmath,amssymb,amsfonts}
\usepackage{algorithmic}
\usepackage{graphicx}
\usepackage{textcomp}
\usepackage{siunitx}
\usepackage{multirow}
\def\BibTeX{{\rm B\kern-.05em{\sc i\kern-.025em b}\kern-.08em
    T\kern-.1667em\lower.7ex\hbox{E}\kern-.125emX}}

\definecolor{lightGreen}{rgb}{0.5313, 0.7875, 0.7877}
\definecolor{mediumGreen}{rgb}{0.3296, 0.6354, 0.6358}
\definecolor{darkGreen}{rgb}{0.1154, 0.4519, 0.4522}

\definecolor{babyblueeyes}{rgb}{0.63, 0.79, 0.95}

\newcommand{\abbas}[1]{\textcolor{violet}{#1}}

\newcommand{\review}{\color{red}}

\newcommand{\hlc}[2][yellow]{ {\sethlcolor{#1} \hl{#2}} }
\newcommand\notis[1]{\hlc[yellow]{PS: #1}}
\usetikzlibrary{decorations.markings}
\input{lib/basestation}

\newcommand{\STAB}[1]{\begin{tabular}{@{}c@{}}#1\end{tabular}}

\begin{document}

\title{Understanding Energy Efficiency and 
Interference Tolerance 
in Millimeter Wave Receivers}



\author{
    \IEEEauthorblockN{Panagiotis 
    Skrimponis\IEEEauthorrefmark{1}, 
    Seongjoon Kang\IEEEauthorrefmark{1},
    Abbas Khalili\IEEEauthorrefmark{1},
    Wonho Lee\IEEEauthorrefmark{2},
    Navid Hosseinzadeh\IEEEauthorrefmark{2},
   }
    
    \IEEEauthorblockN{
    Marco Mezzavilla\IEEEauthorrefmark{1},
    Elza Erkip\IEEEauthorrefmark{1},
    Mark J. W. Rodwell\IEEEauthorrefmark{2}, 
    James F. Buckwalter\IEEEauthorrefmark{2},
    Sundeep Rangan\IEEEauthorrefmark{1}}
    
    \IEEEauthorblockN{
    \IEEEauthorrefmark{1}NYU Tandon School of Engineering,
    New York University, Brooklyn, NY}
    \IEEEauthorblockN{
    \IEEEauthorrefmark{2}Dept. ECE,  University of
    California, Santa Barbara}
    
    \thanks{P. Skrimponis, S. Kang, A. Khalili, M. Mezzavilla, E. Erkip and S. Rangan are with NYU Wireless, Tandon School of Engineering, New York University, Brooklyn, NY.
    They are supported under
    NSF grants 1952180, 1925079, 1564142, 
    1547332, the Semiconductor Research Corporation (SRC) and the industrial affiliates
    of NYU Wireless.
    }
    

}
\IEEEoverridecommandlockouts

\maketitle

\begin{abstract}
Power consumption is a key challenge
in millimeter wave (mmWave) receiver front-ends,
due to the need to support high dimensional antenna arrays
at wide bandwidths.  
Recently, there has been considerable work in developing 
low-power front-ends, often based on low-resolution ADCs
and low-power mixers.
A critical but less studied consequence of such designs is the
relatively low-dynamic range which in turn exposes the receiver 
to adjacent carrier interference and blockers.
This paper provides a general mathematical framework for 
analyzing the performance of mmWave front-ends
in the presence of out-of-band interference.
The goal is to elucidate the
 fundamental trade-off of power consumption, interference
tolerance and in-band performance.  
The analysis is combined with detailed network simulations
in cellular systems with multiple carriers, as well as detailed circuit
simulations of key components at \SI{140}{GHz}.  The analysis
reveals critical bottlenecks for low-power interference robustness
and suggests designs enhancements  for use in
practical systems.
\end{abstract}

% !TEX root = ../arxiv.tex

Unsupervised domain adaptation (UDA) is a variant of semi-supervised learning \cite{blum1998combining}, where the available unlabelled data comes from a different distribution than the annotated dataset \cite{Ben-DavidBCP06}.
A case in point is to exploit synthetic data, where annotation is more accessible compared to the costly labelling of real-world images \cite{RichterVRK16,RosSMVL16}.
Along with some success in addressing UDA for semantic segmentation \cite{TsaiHSS0C18,VuJBCP19,0001S20,ZouYKW18}, the developed methods are growing increasingly sophisticated and often combine style transfer networks, adversarial training or network ensembles \cite{KimB20a,LiYV19,TsaiSSC19,Yang_2020_ECCV}.
This increase in model complexity impedes reproducibility, potentially slowing further progress.

In this work, we propose a UDA framework reaching state-of-the-art segmentation accuracy (measured by the Intersection-over-Union, IoU) without incurring substantial training efforts.
Toward this goal, we adopt a simple semi-supervised approach, \emph{self-training} \cite{ChenWB11,lee2013pseudo,ZouYKW18}, used in recent works only in conjunction with adversarial training or network ensembles \cite{ChoiKK19,KimB20a,Mei_2020_ECCV,Wang_2020_ECCV,0001S20,Zheng_2020_IJCV,ZhengY20}.
By contrast, we use self-training \emph{standalone}.
Compared to previous self-training methods \cite{ChenLCCCZAS20,Li_2020_ECCV,subhani2020learning,ZouYKW18,ZouYLKW19}, our approach also sidesteps the inconvenience of multiple training rounds, as they often require expert intervention between consecutive rounds.
We train our model using co-evolving pseudo labels end-to-end without such need.

\begin{figure}[t]%
    \centering
    \def\svgwidth{\linewidth}
    \input{figures/preview/bars.pdf_tex}
    \caption{\textbf{Results preview.} Unlike much recent work that combines multiple training paradigms, such as adversarial training and style transfer, our approach retains the modest single-round training complexity of self-training, yet improves the state of the art for adapting semantic segmentation by a significant margin.}
    \label{fig:preview}
\end{figure}

Our method leverages the ubiquitous \emph{data augmentation} techniques from fully supervised learning \cite{deeplabv3plus2018,ZhaoSQWJ17}: photometric jitter, flipping and multi-scale cropping.
We enforce \emph{consistency} of the semantic maps produced by the model across these image perturbations.
The following assumption formalises the key premise:

\myparagraph{Assumption 1.}
Let $f: \mathcal{I} \rightarrow \mathcal{M}$ represent a pixelwise mapping from images $\mathcal{I}$ to semantic output $\mathcal{M}$.
Denote $\rho_{\bm{\epsilon}}: \mathcal{I} \rightarrow \mathcal{I}$ a photometric image transform and, similarly, $\tau_{\bm{\epsilon}'}: \mathcal{I} \rightarrow \mathcal{I}$ a spatial similarity transformation, where $\bm{\epsilon},\bm{\epsilon}'\sim p(\cdot)$ are control variables following some pre-defined density (\eg, $p \equiv \mathcal{N}(0, 1)$).
Then, for any image $I \in \mathcal{I}$, $f$ is \emph{invariant} under $\rho_{\bm{\epsilon}}$ and \emph{equivariant} under $\tau_{\bm{\epsilon}'}$, \ie~$f(\rho_{\bm{\epsilon}}(I)) = f(I)$ and $f(\tau_{\bm{\epsilon}'}(I)) = \tau_{\bm{\epsilon}'}(f(I))$.

\smallskip
\noindent Next, we introduce a training framework using a \emph{momentum network} -- a slowly advancing copy of the original model.
The momentum network provides stable, yet recent targets for model updates, as opposed to the fixed supervision in model distillation \cite{Chen0G18,Zheng_2020_IJCV,ZhengY20}.
We also re-visit the problem of long-tail recognition in the context of generating pseudo labels for self-supervision.
In particular, we maintain an \emph{exponentially moving class prior} used to discount the confidence thresholds for those classes with few samples and increase their relative contribution to the training loss.
Our framework is simple to train, adds moderate computational overhead compared to a fully supervised setup, yet sets a new state of the art on established benchmarks (\cf \cref{fig:preview}).

Online convex optimization with memory has emerged as an important and challenging area with a wide array of applications, see \citep{lin2012online,anava2015online,chen2018smoothed,goel2019beyond,agarwal2019online,bubeck2019competitively} and the references therein.  Many results in this area have focused on the case of online optimization with switching costs (movement costs), a form of one-step memory, e.g., \citep{chen2018smoothed,goel2019beyond,bubeck2019competitively}, though some papers have focused on more general forms of memory, e.g., \citep{anava2015online,agarwal2019online}. In this paper we, for the first time, study the impact of feedback delay and nonlinear switching cost in online optimization with switching costs. 

An instance consists of a convex action set $\mathcal{K}\subset\mathbb{R}^d$, an initial point $y_0\in\mathcal{K}$, a sequence of non-negative convex cost functions $f_1,\cdots,f_T:\mathbb{R}^d\to\mathbb{R}_{\ge0}$, and a switching cost $c:\mathbb{R}^{d\times(p+1)}\to\mathbb{R}_{\ge0}$. To incorporate feedback delay, we consider a situation where the online learner only knows the geometry of the hitting cost function at each round, i.e., $f_t$, but that the minimizer of $f_t$ is revealed only after a delay of $k$ steps, i.e., at time $t+k$.  This captures practical scenarios where the form of the loss function or tracking function is known by the online learner, but the target moves over time and measurement lag means that the position of the target is not known until some time after an action must be taken. 
To incorporate nonlinear (and potentially nonconvex) switching costs, we consider the addition of a known nonlinear function $\delta$ from $\mathbb{R}^{d\times p}$ to $\mathbb{R}^d$ to the structured memory model introduced previously.  Specifically, we have
\begin{align}
c(y_{t:t-p}) = \frac{1}{2}\|y_t-\delta(y_{t-1:t-p})\|^2,    \label{e.newswitching}
\end{align}
where we use $y_{i:j}$ to denote either $\{y_i, y_{i+1}, \cdots, y_j\}$ if $i\leq j$, or  $\{y_i, y_{i-1}, \cdots, y_j\}$ if $i > j$ throughout the paper. Additionally, we use $\|\cdot\|$ to denote the 2-norm of a vector or the spectral norm of a matrix.

In summary, we consider an online agent that interacts with the environment as follows:
% \begin{inparaenum}[(i)] 
\begin{enumerate}%[leftmargin=*]
    \item The adversary reveals a function $h_t$, which is the geometry of the $t^\mathrm{th}$ hitting cost, and a point $v_{t-k}$, which is the minimizer of the $(t-k)^\mathrm{th}$ hitting cost. Assume that $h_t$ is $m$-strongly convex and $l$-strongly smooth, and that $\arg\min_y h_t(y)=0$.
    \item The online learner picks $y_t$ as its decision point at time step $t$ after observing $h_t,$  $v_{t-k}$.
    \item The adversary picks the minimizer of the hitting cost at time step $t$: $v_t$. 
    \item The learner pays hitting cost $f_t(y_t)=h_t(y_t-v_t)$ and switching cost $c(y_{t:t-p})$ of the form \eqref{e.newswitching}.
\end{enumerate}

The goal of the online learner is to minimize the total cost incurred over $T$ time steps, $cost(ALG)=\sum_{t=1}^Tf_t(y_t)+c(y_{t:t-p})$, with the goal of (nearly) matching the performance of the offline optimal algorithm with the optimal cost $cost(OPT)$. The performance metric used to evaluate an algorithm is typically the \textit{competitive ratio} because the goal is to learn in an environment that is changing dynamically and is potentially adversarial. Formally, the competitive ratio (CR) of the online algorithm is defined as the worst-case ratio between the total cost incurred by the online learner and the offline optimal cost: $CR(ALG)=\sup_{f_{1:T}}\frac{cost(ALG)}{cost(OPT)}$.

It is important to emphasize that the online learner decides $y_t$ based on the knowledge of the previous decisions $y_1\cdots y_{t-1}$, the geometry of cost functions $h_1\cdots h_t$, and the delayed feedback on the minimizer $v_1\cdots v_{t-k}$. Thus, the learner has perfect knowledge of cost functions $f_1\cdots f_{t-k}$, but incomplete knowledge of $f_{t-k+1}\cdots f_t$ (recall that $f_t(y)=h_t(y-v_t)$).

Both feedback delay and nonlinear switching cost add considerable difficulty for the online learner compared to versions of online optimization studied previously. Delay hides crucial information from the online learner and so makes adaptation to changes in the environment more challenging. As the learner makes decisions it is unaware of the true cost it is experiencing, and thus it is difficult to track the optimal solution. This is magnified by the fact that nonlinear switching costs increase the dependency of the variables on each other. It further stresses the influence of the delay, because an inaccurate estimation on the unknown data, potentially magnifying the mistakes of the learner. 

The impact of feedback delay has been studied previously in online learning settings without switching costs, with a focus on regret, e.g., \citep{joulani2013online,shamir2017online}.  However, in settings with switching costs the impact of delay is magnified since delay may lead to not only more hitting cost in individual rounds, but significantly larger switching costs since the arrival of delayed information may trigger a very large chance in action.  To the best of our knowledge, we give the first competitive ratio for delayed feedback in online optimization with switching costs. 

We illustrate a concrete example application of our setting in the following.

\begin{example}[Drone tracking problem]
\label{example:drone} \emph{
Consider a drone with vertical speed $y_t\in\mathbb{R}$. The goal of the drone is to track a sequence of desired speeds $y^d_1,\cdots,y^d_T$ with the following tracking cost:}
\begin{equation}
    \sum_{t=1}^T \frac{1}{2}(y_t-y^d_t)^2 + \frac{1}{2}(y_t-y_{t-1}+g(y_{t-1}))^2,
\end{equation}
\emph{where $g(y_{t-1})$ accounts for the gravity and the aerodynamic drag. One example is $g(y)=C_1+C_2\cdot|y|\cdot y$ where $C_1,C_2>0$ are two constants~\cite{shi2019neural}. Note that the desired speed $y_t^d$ is typically sent from a remote computer/server. Due to the communication delay, at time step $t$ the drone only knows $y_1^d,\cdots,y_{t-k}^d$.}

\emph{This example is beyond the scope of existing results in online optimization, e.g.,~\cite{shi2020online,goel2019beyond,goel2019online}, because of (i) the $k$-step delay in the hitting cost $\frac{1}{2}(y_t-y_t^d)$ and (ii) the nonlinearity in the switching cost $\frac{1}{2}(y_t-y_{t-1}+g(y_{t-1}))^2$ with respective to $y_{t-1}$. However, in this paper, because we directly incorporate the effect of delay and nonlinearity in the algorithm design, our algorithms immediately provide constant-competitive policies for this setting.}
\end{example}


\subsection{Related Work}
This paper contributes to the growing literature on online convex optimization with memory.  
Initial results in this area focused on developing constant-competitive algorithms for the special case of 1-step memory, a.k.a., the Smoothed Online Convex Optimization (SOCO) problem, e.g., \citep{chen2018smoothed,goel2019beyond}. In that setting, \citep{chen2018smoothed} was the first to develop a constant, dimension-free competitive algorithm for high-dimensional problems.  The proposed algorithm, Online Balanced Descent (OBD), achieves a competitive ratio of $3+O(1/\beta)$ when cost functions are $\beta$-locally polyhedral.  This result was improved by \citep{goel2019beyond}, which proposed two new algorithms, Greedy OBD and Regularized OBD (ROBD), that both achieve $1+O(m^{-1/2})$ competitive ratios for $m$-strongly convex cost functions.  Recently, \citep{shi2020online} gave the first competitive analysis that holds beyond one step of memory.  It holds for a form of structured memory where the switching cost is linear:
$
    c(y_{t:t-p})=\frac{1}{2}\|y_t-\sum_{i=1}^pC_iy_{t-i}\|^2,
$
with known $C_i\in\mathbb{R}^{d\times d}$, $i=1,\cdots,p$. If the memory length $p = 1$ and $C_1$ is an identity matrix, this is equivalent to SOCO. In this setting, \citep{shi2020online} shows that ROBD has a competitive ratio of 
\begin{align}
    \frac{1}{2}\left( 1 + \frac{\alpha^2 - 1}{m} + \sqrt{\Big( 1 + \frac{\alpha^2 - 1}{m}\Big)^2 + \frac{4}{m}} \right),
\end{align}
when hitting costs are $m$-strongly convex and $\alpha=\sum_{i=1}^p\|C_i\|$. 


Prior to this paper, competitive algorithms for online optimization have nearly always assumed that the online learner acts \emph{after} observing the cost function in the current round, i.e., have zero delay.  The only exception is \citep{shi2020online}, which considered the case where the learner must act before observing the cost function, i.e., a one-step delay.  Even that small addition of delay requires a significant modification to the algorithm (from ROBD to Optimistic ROBD) and analysis compared to previous work. 

As the above highlights, there is no previous work that addresses either the setting of nonlinear switching costs nor the setting of multi-step delay. However, the prior work highlights that ROBD is a promising algorithmic framework and our work in this paper extends the ROBD framework in order to address the challenges of delay and non-linear switching costs. Given its importance to our work, we describe the workings of ROBD in detail in Algorithm~\ref{robd}. 

\begin{algorithm}[t!]
  \caption{ROBD \citep{goel2019beyond}}
  \label{robd}
\begin{algorithmic}[1]
  \STATE {\bfseries Parameter:} $\lambda_1\ge0,\lambda_2\ge0$
  \FOR{$t=1$ {\bfseries to} $T$}
  \STATE {\bfseries Input:} Hitting cost function $f_t$, previous decision points $y_{t-p:t-1}$
  \STATE $v_t\leftarrow\arg\min_yf_t(y)$
  \STATE $y_t\leftarrow\arg\min_yf_t(y)+\lambda_1c(y,y_{t-1:t-p})+\frac{\lambda_2}{2}\|y-v_t\|^2_2$
  \STATE {\bfseries Output:} $y_t$
  \ENDFOR
   
\end{algorithmic}
\end{algorithm}

Another line of literature that this paper contributes to is the growing understanding of the connection between online optimization and adaptive control. The reduction from adaptive control to online optimization with memory was first studied in \citep{agarwal2019online} to obtain a sublinear static regret guarantee against the best linear state-feedback controller, where the approach is to consider a disturbance-action policy class with some fixed horizon.  Many follow-up works adopt similar reduction techniques \citep{agarwal2019logarithmic, brukhim2020online, gradu2020adaptive}. A different reduction approach using control canonical form is proposed by \citep{li2019online} and further exploited by \citep{shi2020online}. Our work falls into this category.  The most general results so far focus on Input-Disturbed Squared Regulators, which can be reduced to online convex optimization with structured memory (without delay or nonlinear switching costs).  As we show in \Cref{Control}, the addition of delay and nonlinear switching costs leads to a significant extension of the generality of control models that can be reduced to online optimization. 
\begin{figure}[t]
    \centering
    \includegraphics[width = 0.99\linewidth]{fig/fig_analitic_model-eps-converted-to.pdf}
    % \begin{tikzpicture}
    %     \begin{axis}[restrict z to domain=0:inf,restrict x to domain=0:inf,restrict y to domain=0:inf]
    %     \addplot3[surf] {y/(1+y+0.5*x)};
    %     \end{axis}
    % \end{tikzpicture}
    \caption{Input-output SNR relation.}
    \label{fig:snr_inout}
\end{figure}

\begin{figure*}[!t]
    \centering
    % !TEX root = ../top.tex
% !TEX spellcheck = en-US

\begin{figure}[t]
\centering
\includegraphics[width=0.99\linewidth]{./fig/arch/arch.pdf}
\vspace{-3mm}
    \caption{{\bf Our single-stage approach.} We use an encoder-decoder architecture to progressively downsample the image and then to re-expand it. At each level of the decoder, we establish 3D-to-2D correspondences. Finally, we use a RANSAC-based PnP strategy~\cite{Lepetit09} 
    % \WJ{citation?} \YH{Fixed} 
     to infer a single reliable pose from these sets of correspondences. 
    % \WJ{Impossible to read text in printout, please try to have text in the figure with a similar font size as the surrounding article.}\YH{Fixed}
    }
\label{fig:arch}
\end{figure}

    \caption{High-level architecture of a fully-digital superheterodyne 
    receiver architecture. The architecture supports $N_\mathrm{rx}$ antennas and
    $N_\mathrm{str}$ digital streams. The light green boxes represent analog and the 
    dark-green boxes the digital components. In the RF front-end, 
    some component are not shown.}
    \label{fig:arch}
\end{figure*}

\section{Capacity Bound and Output SNR}
\label{sect:capacity}
Our goal is to characterize the performance of the discussed model in Sec.~\ref{sec:model} in terms of the spectral efficiency. To this end, similar to \cite{skrimponis2020efficient}, we make use of the concepts of the output SNR and input-output SNR relation described next. 

From Sec.~\ref{sec:model}, we have 
\begin{align}
    \label{eq:inoutre}
    \hat{\nbx} = \nbF\Phi(\nbF \herm \nbx, \nbD).
\end{align}

Assuming that the variables $\hat{x}_n$, $\nbd_n$, and $x_n$ have an underlying statistical model and are distributed as $\hat{X}$, $D$, and $X$, respectively, we can use the Bussgang-Rowe decomposition \cite{bussgang1952crosscorrelation,rowe1982memoryless} and model the non-linearity in the system (i.e., $\Phi(\cdot)$) as multiplying a scalar with its input and adding a noise which is uncorrelated with the input. More precisely, we can write
\begin{align}
\label{eq:lin_mod}
    \hat{X} = A X+ T,\quad \Exp |T|^2 &= \tau,
\end{align}
where
\begin{align}
\label{eq:alpha_tau_rx}
    A := \frac{\Exp\left[ \hat{X}^* X\right]}{\Exp\left|X\right|^2 }, \quad \tau :=  \Exp|\hat{X}- A X|^2,
    % \alpha_2 :=  \frac{1}{\overline{P}}\Exp|S-\alpha_{\rm rx}U|^2,
\end{align} 
with $X^*$ denoting the complex conjugate of $X$.

% In \cite{dutta2020capacity}, we have proved rigorously that in the system shown in Fig.~\ref{fig:abstract_model}, for many different functions $\Phi(\cdot)$, the elements of $\wbf$ can be viewed as i.i.d. $\CN(0, \tau)$.
In general, both $A$ and $\tau$ are functions of the 
input SNR $\gamma_\mathrm{in}$ 
so we may write $A = A(\gamma_{\mathrm{in}})$
and $\tau = \tau(\gamma_{\mathrm{in}})$. 
% Based on \cite[Theorem.~1]{dutta2020capacity} 
From \eqref{eq:lin_mod}, we can then define the 
{\it output SNR} off the desired signal $\nbs$ as,
\begin{equation} \label{eq:inout_snr}
    \gamma_{\mathrm{out}} = G(\gamma_{\mathrm{in}}) :=
    \frac{|A(\gamma_{\mathrm{in}})|^2}{\tau(\gamma_{\mathrm{in}})} 
    \gamma_{\rm sig}.
\end{equation}
This is the SNR that would be seen in attempting
to recover the input transmitted vector $\nbs$ from the
output vector $\hat{\nbs}$.  

Using the SNR enables us to bound the performance of the system in terms of the capacity. More precisely, using same steps as of \cite[Appendix~A]{skrimponis2020efficient}, we can show that the capacity of the system  can is lower bounded as,
\begin{equation}  \label{EQ:CAP_BAND_LIMITED1}
  C \geq  \frac{N_{\rm sig}}{N} f_s\log_2\left(1 +  \gamma_{\rm out}\right),
\end{equation}
where $f_s$ is the sample rate and $N_{\rm sig} = |I_{\rm sig}|$ represent the number of frequency bins for the signal. Moreover, assuming that the ADC performs oversampling with the ratio $\zeta$, using same steps as of \cite[Appendix~B]{skrimponis2020efficient}, we have

\begin{equation}  \label{EQ:CAP_BAND_LIMITED}
  C \geq  \frac{N_{\rm sig}\zeta}{N} f_s\log_2\left(1 +  \frac{\gamma_{\rm out}}{\zeta}\right),
\end{equation}
% where $f_s$ is the sample rate and $N_{\rm sig} = |I_{\rm sig}|$ represent the number of frequency bins for the signal. 


\begin{table}[t]
    \centering
    \caption{Parameters of the \SI{28}{GHz} RFFE devices used in the analysis.}

    \setlength{\tabcolsep}{3pt}
    \begin{tabular}{|>{\raggedright}m{0.9in}|c|c|c|c|}
    \hline
    
    \textbf{Parameter} &
    \textbf{LNA$^{\boldsymbol{(1)}}$} &
    \textbf{LNA$^{\boldsymbol{(2)}}$} &
    \textbf{Mixer$^{\boldsymbol{(1)}}$} &
    \textbf{Mixer$^{\boldsymbol{(2)}}$}
    \tabularnewline \hline
    Design [\textmu m] & 10 & 5 & 2 & 5
    \tabularnewline
    Noise Figure [dB] & 2.13 & 2.53 &  9.039 & 7.542
    \tabularnewline
    Gain [dB] & 14.26 & 12.85 &  0.16 &  3.558
    \tabularnewline
    IIP3 [dBm] & -1.456 & 0.603  & -3.1  & 2.1
    \tabularnewline
    Power [mW] &8.91 & 5.34 &   4.838 &   7.03
    \tabularnewline \hline
    \end{tabular}
    \label{tab:rffe28}
\end{table}

% New models
% 2 & 2
% 10.115 & 9.039
% & 
% 
% 4.8 & 4.8



In this paper, we will show through detailed simulations that the input-output SNR relation can be approximated in the form of
\begin{equation} \label{eq:gamout}
    \hat{\gamma}_{\rm out} = \frac{\beta \gamma_{\rm sig}}{
    1 + \alpha_1 \gamma_{\rm sig} + \alpha_2\gamma_{\rm int}},
\end{equation}
    where $\gamma_{\rm int} = \frac{1}{|I_{\rm int}|}\sum_{n \in I_{\rm int}}E_i[n]$.
for three parameters $\beta$ and $\alpha_1$ and $\alpha_2$
using which we can evaluate the receiver front-end performance. Intuitively, this formula suggest that due to the non-linearity in the system: (i) the signal energy is reduced; (ii) a ratio of the signal is distorted; (iii) a ratio of the adjacent band signal (i.e., interference) is leaked to the desired band.

From \eqref{eq:gamout}, we also observe that the output SNR saturates to the value of $\frac{\beta}{\alpha_1}$ as the input signal SNR increases. Furthermore, for higher values of the interference signal the saturation should accrue for lower values of the input SNR. One can also observe these from Fig.~\ref{fig:snr_inout} which illustrates \eqref{eq:gamout} for a fixed values of $\beta$, $\alpha_1$ and $\alpha_2$ and different values of desired and interference signal powers.
% signal-to-distortion-plus-interference-noise-and-ratio
% The trade offs between the received signals power,  




% To interpret the roles of the parameters
% Fig.~\ref{fig:snr_inout} plots
% $\gamma_{\mathrm{out}}$ vs.\ $\gamma_{\mathrm{sig}}$ and $\gamma_{\mathrm{int}}$ with both the values in dB-scale.  We see two distinct
% regimes.

% \paragraph*{Low input SNR regime}
% When $N_\mathrm{rx} \gamma_\mathrm{in}/F \ll \gamma_\mathrm{sat}$,
% the output SNR in \eqref{eq:inout_snr_nonlin} simplifies to
% \begin{equation} \label{eq:inout_snr_nl_low}
%     \gamma_{\mathrm{out}} = G(\gamma_{\mathrm{in}}) 
%     \approx\frac{N_\mathrm{rx}}{F}\gamma_{\mathrm{in}},
% \end{equation}
% which matches the linear case \eqref{eq:inout_snr_lin}.
% We will thus call $F$ in \eqref{eq:inout_snr_nonlin} the
% {\it effective noise figure}.  This regime
% appears on the left side of Fig.~\ref{fig:snr_inout}
% where $\gamma_\mathrm{out}$ (in dB) 
% increases linearly  with $\gamma_\mathrm{in}$ with an 
% offset given by beamforming gain $N_\mathrm{rx}$
% minus the effective noise figure $F$.

% We will see below that the low input SNR regime
% occurs when the RFFE components have sufficiently small
% input levels that they do not saturate and act linearly.
% In this case, the effective noise figure
% is determined by the standard noise figure analysis 
% with an additional term from the quantization noise at the ADC.

% \paragraph*{High input SNR regime}
% At very large input SNRs 
% ($N_\mathrm{rx} \gamma_\mathrm{in}/F \gg \gamma_\mathrm{sat}$),
% the output SNR in \eqref{eq:inout_snr_nonlin} simplifies to
% \begin{equation} \label{eq:inout_snr_nl_high}
%     \gamma_{\mathrm{out}} \approx G(\gamma_{\mathrm{in}}) 
%     \approx \gamma_\mathrm{sat}.
% \end{equation}
% Hence, the output SNR saturates as shown 
% in Fig.~\ref{fig:snr_inout}.  It is for this
% reason that $\gamma_\mathrm{sat}$ is called the saturation SNR.  
% In this case, we will see that the saturation SNR is determined
% by the nonlinearities in the devices and the quantization
% in the ADC.  

% \textcolor{red}{Abbas to do:
% \begin{itemize}
%     \item State the result on the Gaussianity 
%     \item State the capacity lower bound
%     \item Define output SNR
%     \item State the analytic model for the output SNR.
% \end{itemize}
% }
% We model the output SNR as:
% \begin{equation} \label{eq:gamout}
%     \gamma_{\rm out} = \frac{\beta \gamma_{\rm sig}}{
%     1 + \alpha_1 \gamma_{\rm sig} + \alpha_2\gamma_{\rm int}}.
% \end{equation}




\begin{table}[t]
    \centering
    \caption{Parameters of the \SI{140}{GHz} RFFE devices used in the analysis.}

    \setlength{\tabcolsep}{3pt}
    \begin{tabular}{|>{\raggedright}m{0.9in}|c|c|c|c|}
    \hline
    
    \textbf{Parameter} &
    \textbf{LNA$^{\boldsymbol{(1)}}$} &
    \textbf{LNA$^{\boldsymbol{(2)}}$} &
    \textbf{Mixer$^{\boldsymbol{(1)}}$} &
    \textbf{Mixer$^{\boldsymbol{(2)}}$}
    \tabularnewline \hline
    Design [\textmu m] & 4 &  2-4 & 1 & 1
    \tabularnewline
    Noise Figure [dB] & 7.50 & 7.48 & 21.53 & 20.47
    \tabularnewline
    Gain [dB] & 11.13 & 16.56 & -1.74 & -0.52
    \tabularnewline
    IIP3 [dBm] & -9.15 & -8.90 & -4.45 & -3.88
    \tabularnewline
    Power [mW] &  4.80 & 15.90 & 5.00 & 5.00
    \tabularnewline \hline
    \end{tabular}
    \label{tab:rffe140}
\end{table}


\section{Link-Layer Simulation} \label{sec:link}

% \textcolor{red}{
% Notis to do:
% \begin{itemize}
%     \item Create a block diagram for the RFFE.  You can use the one from the IEEE Access paper.
%     \item Describe the interference and signal model.
%     \item Describe the design options and why we are analyzing them. 
%     \item Describe the fitting procedure using an initial fit followed by non-linear least squares
%     \item Show plots of SNR out vs.\ signal and interfering SNR.
%     \item Describe when the model is a good fit.
%     \item Summarize fitted parameters in a table.
% \end{itemize}}

The model in Section~\ref{sec:model} is a simplified
abstraction of an actual RFFE.  In this section, 
we validate the model and extract the parameters 
for \eqref{eq:gamout} with realistic, circuit simulations
of potential RFFEs at \SI{28}{GHz} and \SI{140}{GHz}.


\subsection{Signal and interference model}
We consider a downlink scenario in a communication link 
between an NR basestation (gNB) and a mobile device (UE). 
For each slot, the gNB generates a physical downlink shared
channel (PDSCH) that includes both information and control 
signals. The receiver uses the demodulation reference 
signal (DM-RS) for practical channel estimation. 
To compensate the common phase error (CPE) the 3GPP 5G NR,
standard introduce the phase tracking reference signal (PT-RS). 
The receiver performs coherent CPE estimation using the algorithm 
described in~\cite{syrjala2019phase}. For time synchronization,
between the gNB and UE we utilize the primary (PSS) and the secondary synchronization signals (SSS).

To model the interference, we assume the presence of another gNB
that generates i.i.d. $\CN(0, 1)$ symbols in frequency-domain. The
symbols are modulated with OFDM to generate interference in an
adjacent band. Even though the signals from the two gNBs are
originally independent in the frequency domain, the presence of 
the non-linear processing introduces distortion to the main signal
from the adjacent band. As explained in Section~\ref{sec:model},
this increase the total received energy and causes the RFFE 
saturation point to happen much earlier.

%
% 140 GHz - Mixer (1) 
% Gain: -1.7360
%   IIP3: -4.4546
%     NF: 21.5290
%   PLO: -11
%  Power: 5
%
% 140 GHz - Mixer (2) 
%   Gain: -0.5185
%   IIP3: -3.8883
%     NF: 20.4715
%   PLO: -6
%  Power: 5
     
\begin{figure}[t]
    \centering
    \includegraphics[width = 0.99\linewidth]{fig/model_fit-eps-converted-to.pdf}
    \caption{Evaluation of the approximation model in \eqref{eq:gamout} for Design$^{(1)}$ at \SI{28}{GHz} for different input interference power.}
    \label{fig:model_fit}
\end{figure}



\subsection{Receiver configurations}
Similarly to~\cite{skrimponis2021towards} we consider a fully-digital superheterodyne receiver architecture as shown in Fig.~\ref{fig:arch}. The receiver has $N_\mathrm{rx}=16$ or $64$ antennas with independent RFFE processing. The received signal is amplified with a low-noise amplifier  (LNA), downconverted with an intermediate frequency (IF) mixer, amplified with an automated gain control (AGC) amplifier before the direct conversion mixer, and finally quantized with a pair of ADCs. The system use filters in the RF and IF domain to improve the image rejection and the dynamic range of the system. The actions of the RFFE devices for each receiver configuration is modeled with a different non-linear function $\Phi(\cdot)$.

In~\cite{skrimponis2021towards} the authors design and evaluate RFFE devices at \SI{140}{GHZ} based on a \SI{90}{nm} SiGE BiCMOS HBT technology. They focus on minimizing the power consumption of the RFFE devices at a certain performance in terms of gain, noise figure (NF), and the input third order intercept point (IIP3). The LNA designs vary in terms of number of stages, topology, and transistor size. While the proposed double-balanced active mixers are all based on the conventional Gilbert-cell design, they vary in transistor size. The mixer performance characteristics depend on the input power from the local oscillator (LO). To further optimize the power consumption of the receiver, the authors propose a novel LO distribution model. Specifically, the mixers in the same tile share a common LO driver using power dividers and amplifiers. They provide a method to determine the best configuration of LO drivers that achieve the same performance for a minimum power.

Using the components and the optimization framework described in~\cite{skrimponis2021towards} we select two of the optimized designs for the \SI{140}{GHz} systems in our analysis. These two designs achieve similar performance to the state-of-the-art design discussed in \cite{skrimponis2020power} while achieving a significant improvement in power consumption. Similarly, for the \SI{28}{GHz} devices we design two common base emitter base collector (CBEBC) LNAs, and two active mixers based on the common Gilbert cell design. We use the optimization framework in~\cite{skrimponis2021towards} to determine the number of LO drivers. For both frequencies, Design$^{(1)}$ use 4-bit ADCs while Design$^{(2)}$ use 5-bit ADC pairs. This assumption is based on prior works \cite{abbas2017millimeter,zhang2018low,abdelghany2018towards,dutta2019case} indicating that 4 bits are sufficient for the majority of cellular data and control operations. Based on a flash-based 4-bit ADC in \cite{Nasri2017}, we consider the ADC $\mathrm{FOM} = 65\;\mathrm{fJ/conv}$. We summarize the parameters for the devices at \SI{28}{GHz} and \SI{140}{GHz} in Table~\ref{tab:rffe28} and Table~\ref{tab:rffe140} respectively. 

% \textcolor{blue}{[SR:  Can someone provide
% a brief description of where the 28 GHz
% designs are coming from?
% Also, in Tables I and II (or somewhere else),
% can we put the total poewr consumption for
% the Designs 1 and 2 for the 28 and 140 GHz
% receivers.  ]}

\begin{table}[t]
    \centering
    \caption{Model and system parameters for the receiver designs at \SI{28}{GHz} and \SI{140}{GHz}.}
    \label{tab:model_param}
    \begin{tabular}{|>{\raggedright}m{0.5in}|c|c|c|c|}
        \hline
         \multirow{2}{*}{\textbf{Parameter}} & \multicolumn{2}{c|}{\textbf{\SI{28}{GHz}}} & \multicolumn{2}{c|}{\textbf{\SI{140}{GHz}}}
        \tabularnewline \cline{2-5}
        & $\text{Design}^{(1)}$ & $\text{Design}^{(2)}$ & $\text{Design}^{(1)}$ & $\text{Design}^{(2)}$
        
        \tabularnewline \hline
        \multicolumn{1}{|c|}{$\beta$} & 1.3865 & 1.2725 & 0.3099 & 0.1862
        \tabularnewline 
        
        \multicolumn{1}{|c|}{$\alpha_1$} & 0.0090 & 0.0024 & 0.0021 & 0.0004
        \tabularnewline 
        
        \multicolumn{1}{|c|}{$\alpha_2$} & 0.0058 & 0.0017 & 0.0014 & 0.0003
        \tabularnewline \hline
        
        \multicolumn{1}{|c|}{RX antennas} & 16 & 16 & 64 & 64
        \tabularnewline
        
        \multicolumn{1}{|c|}{NF [dB]} & 2.78 & 3.08 & 9.40 & 11.50
        \tabularnewline
        
        \multicolumn{1}{|c|}{Power [mW]} & 411 & 404 & 1682 & 1355
        \tabularnewline \hline
    \end{tabular}
\end{table}

\subsection{Model fitting}
For each receiver design we fit a model in form of
\eqref{eq:gamout}. Since this nonlinear function 
$\Phi(\cdot)$ depends on the parameters $\alpha_1,\alpha_2$, and
$\beta$ we can write the estimated output-SNR model as $\hat{\gamma}_\mathrm{out}(\gamma_\mathrm{sig}, \gamma_\mathrm{int}; \alpha_1, \alpha_2, \beta)$. For the \emph{initial 
heuristic} fit we set, 
\begin{align}
    \beta = \frac{1}{F}, \quad \alpha_1 = \alpha_2 = \frac{1}{\gamma_\mathrm{sat}F},
    \label{eq:fitinit}
\end{align}
where $F$ is the noise factor of the system, and $\gamma_\mathrm{sat}$
the saturation SNR in linear scale. % Note that these parameters 
%do not include any beamforming gain. %We use the downlink scenario  
%described in Section~\ref{sec:link} to obtain $\gamma_\mathrm{out}$ using \eqref{eq:inout_snr}.
We then  optimize the fit  using the non-linear least squares regression method and optimize a problem of the following form,
\begin{align}
   Q(\gamma_\mathrm{sig}, \gamma_\mathrm{int}) := \min_{\alpha_1, \alpha_2, \beta}\|& \gamma_\mathrm{out}(\gamma_\mathrm{sig}, \gamma_\mathrm{int}) \nonumber \\ &
    - \hat{\gamma}_\mathrm{out}(\gamma_\mathrm{sig}, \gamma_\mathrm{int}; \alpha_1, \alpha_2, \beta)) \|_2^2, \nonumber\\
\end{align}
where  $\gamma_\mathrm{out}$ are the measurements from the link-layer simulation using \eqref{eq:inout_snr}, and $\hat{\gamma}_\mathrm{out}$ is the estimate using the model in
\eqref{eq:gamout}. The optimized parameters for the \SI{140}{GHz} and \SI{28}{GHz} receiver
designs are summarized in Tab.~\ref{tab:model_param}. In Fig.~\ref{fig:model_fit} we show that the model in~\eqref{eq:gamout} provides a very good fit. In particular, we see the linear regime for low-input SNR and the saturation for high-input signal power. We show that as the input interference increases the model the saturation SNR is also changing.


\section{Network-Level Simulation} \label{sec:network}
%\textcolor{red}{
%SJ to do:
%\begin{itemize}
%    \item Describe the network topology with the two carriers
%    \item Create a table with all the parameters
%    \item Describe the channel model
%    \item Describe the arrays, how you performed cell association
%    \item Describe how you computed the SNR
%    \item Show the SNR and rate CDFs
%\end{itemize}}
The above analysis shows that the performance of the RFFE
degrades in the presence of strong out-of-band
interference when the front-end dynamic range is low.
This fact raises a basic question:  \emph{how often
is the adjacent carrier interference strong in practical 
systems?}  In this section, we perform a simple network simulation
to assess the effect of adjacent carrier interference in 
a downlink cellular system with two 
adjacent carriers, carriers A and B.

We take account of a wrap-around \SI{1}{km}~$\times$~\SI{1}{km} network area to conduct the network-level simulation. Subject to a inter-site distance (ISD), gNBs are deployed by homogeneous Poisson point process (HPPP) with density $\lambda = \frac{4}{\pi\times\text{ISD}^2}$, and accordingly, the same number of UEs are uniformly distributed. Furthermore, all gNBs and UEs are 
randomly assigned to carrier A or B.
We use the notation $\text{gNB}_{A}$ and $\text{gNB}_{B}$ for base stations, and  $\text{UE}_{A}$ and $\text{UE}_{B}$ for UEs.
Individual gNBs are multi-sectorized with $8\times8$ uniform rectangular arrays (URA) per sector which is tilted down by $-12^\circ$, while each UE is equiped with a URA. The full 
parameter settings are shown in Table~\ref{tab:sim_param}. 

\begin{figure}[t]
    \centering
    How many truly distinct AMMs of a given dimension are there?
Considered as a mathematical object,
much of an AMM's structure is captured by the link between valuations and their stable points.
We say that two AMMs are \emph{topologically equivalent}
if there is a stable-point preserving homeomorphism between their manifolds.
By itself, a homeomorphism between manifolds conveys little information,
but a homeomorphism that preserves stable points preserves the AMMs' common underlying structure.

In this section we show that \emph{all} AMMs over the same set of assets,
if they satisfy our axioms, are topologically equivalent.
More precisely,
for any two AMMs $A(x_1,\ldots,x_n)$ and $B(x_1,\ldots,x_n)$ over asset types
$X_1, \ldots, X_n$, satisfying our axioms,
there is a homeomorphism $\mu: A \to B$
such that $\bx$ and $\mu(\bx)$ are the stable states for the same valuation.
This proof relies on the uniqueness of stable points:
for example it would not hold if AMM functions were convex
instead of strictly convex.

Although topological equivalence implies a common mathematical structure,
two topologically equivalent AMMs may differ substantially
with respect to price slippage, fees,
or how expensive it is to move from one valuation's stable state to another's.

Recall from \lemmaref{stable-unique} and \lemmaref{stable-for-some}
that there is a unique function $\phi: \int(\Delta^n) \to A$
carrying each valuation to its unique stable point.

\begin{lemma}%L17
\lemmalabel{stable-state-biject}
\sloppy
For $A$, an AMM,
the stable point map
\begin{equation*}
    \phi:~\int(\Delta^n)~\to~A
\end{equation*}
is a continuous bijection.
\end{lemma}

\begin{proof}
The map $\phi$ is surjective by \lemmaref{stable-for-some},
and injective by \lemmaref{stable-unique}.
To show continuity,
consider the sequence
$\set{\bv_n}_{n=1}^{\infty} \subset \int(\Delta^n)$ where $\lim_{n \to \infty}\bv_n = \bv$.
Let $\tilde{\bx} = \lim_{n \to \infty} \phi(\bv_n)$ and $\bx^{*} = \phi(\bv)$.
  Suppose $\tilde{\bx} \neq \bx^{*}$.
  Note that $\bv \cdot \bx^{*} < \bv \cdot \tilde{\bx}$ by definition of stable point.
  Letting $\overline{\bx} = \frac{\bx^{*} + \tilde{\bx}}{2}$ by strict convexity we know $\overline{\bx} \in \int(\upper(A))$.
  We also have that $\bv \cdot \bx^{*} < \bv \cdot \overline{\bx} < \bv \cdot \tilde{\bx}$.
  Notice now that $\bv_n \cdot \overline{\bx} > \bv_n \cdot \phi(\bv_n)$ by definition so taking limits we get
  $\bv \cdot \overline{\bx} > \bv \cdot \tilde{\bx}$, a contradiction.
  Thus $\lim_{n \to \infty} \phi(\bv_n) = \phi(\bv)$.
\end{proof}

\begin{lemma}
\lemmalabel{stable-state-homeo}
For $A$, an AMM,
the stable point map
\begin{equation*}
    \phi:~\int(\Delta^n)~\to~A
\end{equation*}
is a homeomorphism.
\end{lemma}

\begin{proof}
From \lemmaref{stable-state-biject}, $\phi$ is both bijective and continuous,
so it is enough to show $\phi^{-1}$ is continuous.
For any $\bv$ with stable point $\bx$,
the first-order conditions imply that $\bv = \lambda \nabla A(\bx)$
for some non-zero (Lagrange multiplier) $\lambda \in \Reals$.
Since $A$ is strictly increasing, $\lambda > 0$.
Thus we can think of $\bv$ as a function of $\bx$,
written $\bv(\bx) = \lambda(\bx) \nabla A(\bx)$.
Because $A$ is continuously differentiable, $\nabla A(\bx)$ is continuous,
so it is enough to check $\lambda(\bx)$ is continuous.
Because $\bv(\bx)$ is a convex combination, and $\lambda(\bx)$ is unique,
\begin{equation*}
\lambda(\bx) = \frac{1}{\|\nabla A(\bx)\|_1},
\end{equation*}
which is continuous because each $\frac{\partial A(\bx)}{\partial x_i}$ is continuous.
It follows that $\phi^{-1}$ is continuous.
\end{proof}

\begin{theorem}
Let $A$ and $B$ be AMMs over the same set of assets.
There is a homeomorphism $\mu: A \to B$
that preserves stable points:
if $\bx$ is the stable point for valuation $\bv$ in $A$,
then $\mu(\bx)$ is the stable point for $\bv$ in $B$.
\end{theorem}

\begin{proof}
  By \lemmaref{stable-state-homeo},
  there exist homeomorphisms
  \begin{align*}
  \phi: \int(\Delta^n) &\rightarrow A,\\
  \phi': \int(\Delta^n) & \rightarrow B
  \end{align*}
  Their composition $\mu = \phi' \circ \phi^{-1}$ is also a homeomorphism.
  For $\bv \in \int(\Delta^n)$ with stable points $\bx \in A, \bx' \in B$,
  $\phi(\bv) = \bx$ and $\phi^{'}(\bv) = \bx^{'}$,
  so $\mu(\bx) = \phi^{'}(\phi^{-1}(\bx)) = \phi^{'}(\bv) = \bx^{'}$,
  implying that $\mu$ preserves stable points.
\end{proof}
Note that this result requires that $A$ have continuous first derivatives.
    \caption{Signal and interfering
    paths in a system with two carriers.}
    \label{fig:network}
\end{figure}

Fig.~\ref{fig:network} shows the scenario for analyzing the interference between adjacent carrier frequencies. A UE in carrier A 
will receive the desired signal from its serving BS and interference signal from non-serving
BSs in
carrier A in the same carrier and
from all BSs in carrier B.

\begin{figure*}[t]
\centering
    \subfigure[\SI{28}{GHz}]{\includegraphics[width=0.49\linewidth]{fig/snr_28-eps-converted-to.pdf}}
    \subfigure[\SI{140}{GHz}]{\includegraphics[width=0.49\linewidth]{fig/snr_140-eps-converted-to.pdf}}
    
 \caption{Estimated distribution of downlink SNRs with different carrier frequencies
    and different RFFE designs.  The plots
    show the SNR under the full model
    \eqref{eq:SNR} with distortion from
    adjacent carrier interference and 
    in-band signal; the SNR 
    with no adjacent carrier interference distortion ($\alpha_2=0)$; and
    the SNR with no in-band or
    adjacent carrier interference distortion ($\alpha_1=\alpha_2=0)$.}
  \label{fig:snr_distributions}
\end{figure*}

% \begin{figure*}[t]
%     \centering
%     \subfigure[\SI{28}{GHz}]{\includegraphics[width=0.45\linewidth]{fig/rate_28.eps}}
%     \subfigure[\SI{140}{GHz}]{\includegraphics[width=0.45\linewidth]{fig/rate_140.eps}}
%     \caption{Estimated distribution of the downlink rates for different carrier frequencies and RFFE designs.  }
%     \label{fig:rate_distributions}
% \end{figure*}

For every gNB-UE pair, we generate a multi-path channel for two different frequency cases, \SI{28}{GHz} and \SI{140}{GHz},  according to~\cite{3GPP38.901} and compute the SNR for downlink case. 
Specifically, we employ the path-loss model specified in~\cite{3GPP38.901} for the Urban Micro Street Canyon (Umi-Street-Canyon) environment. The 3GPP NR standard is very flexible and we assume the channel models described in
\cite{3GPP38.901} will hold for the \SI{140}{GHz} communication systems.
% Since the channel model of \SI{140}{GHz} carrier frequency is not known, we assume that the legacy 3GPP channel model in \cite{3GPP38.901} is still valid at \SI{140}{GHz}. 

Within each carrier, we then assume that each UE
is served by the strongest gNB.  As a simplification, 
the gNBs and UEs then beamform
along the strongest path with no regard to
interference nulling to other UEs.  
A sufficient number of UEs are dropped such
that we can obtain one UE served by each
sector in each gNB.  Hence, the simulation
drop represents one point in time where
each gNB is using its entire bandwidth 
on one UE.

With the channels and beamforming
direction, we can then estimate the effective
SINR at each UE.  
Following the model \eqref{eq:gamout},
we estimate the SINR as:
\begin{equation}
\label{eq:SNR}
    \gamma = \frac{\beta E_\mathrm{sig}^{a}}{E_\mathrm{kT} +
    \alpha_{1}E_\mathrm{tot}^{a} + \alpha_{2}E_\mathrm{tot}^{b}}.
\end{equation}
Here, $E_\mathrm{sig}^{a}$ 
and $E_\mathrm{int}^a$ 
is the energy 
per sample of the serving and interfering
signals including the beamforming and
element gains at the TX and RX,
and $E_\mathrm{kT}$ is the thermal noise.
The distortion from the non-linearities
is modeled by two terms:
$\alpha_{1}E_\mathrm{tot}^{a}$ captures
the distortion from the total power from all base stations in carrier A (serving and non-serving); $\alpha_{2}E_\mathrm{tot}^{b}$
captures the distortion from
the total power from all base stations
in the adjacent carrier, carrier B.
For these terms, we assume that the distortion
is spatially white so we do not 
add the RX beamforming gains on each path.
The terms however do include the TX element
and beamforming gains as well as the RX element
gain.

\begin{table}[t]
    \centering
    \caption{Network simulation parameters.}
    \setlength{\tabcolsep}{3pt}
    \label{tab:sim_param}
    \begin{tabular}{|>{\raggedright}m{1.6in}|c|c|}
        \hline
        \textbf{Parameter} & \multicolumn{2}{c|}{\textbf{Value}}
        \tabularnewline \hline
      
        Carrier frequency, [\SI{}{\GHz}] & $28$ & $140$
        \tabularnewline
        
        Total bandwidth, [\SI{}{\MHz}] & $190.80$ & $380.16$
        \tabularnewline 
        
        Sample rate, %$f_s$,
        [\SI{}{\MHz}] & $491.52$ & $1966.08$
        \tabularnewline 
        
        gNB antenna configuration &{$8\times8$} & {$16\times16$}
        \tabularnewline
        UE antenna configuration &{$4\times4$}& {$8\times8$}
        \tabularnewline \cline{2-3}
        
        Area [\SI{}{\meter}$^2$]  &  \multicolumn{2}{c|}{$1000 \times 1000 $}
        \tabularnewline
        
        UE and gNB min. distance [\SI{}{\meter}] &  \multicolumn{2}{c|}{$10$}
        \tabularnewline
        
        ISD [\SI{}{\meter}] & \multicolumn{2}{c|}{$200$}
        \tabularnewline
        
        gNB height [\SI{}{\meter}] &  \multicolumn{2}{c|}{$\ncalU(2,5)$}
        \tabularnewline
        
        UE height [\SI{}{\meter}] &  \multicolumn{2}{c|}{$1.6$}
        \tabularnewline
        
        gNB TX power [\SI{}{dBm}] & \multicolumn{2}{c|}{$30$}
        \tabularnewline
        
        gNB downtilt angle & \multicolumn{2}{c|}{$-12^\circ$}
        \tabularnewline
        
        gNB number of sectors & \multicolumn{2}{c|}{$3$}
        \tabularnewline
        
        Vertical half-power beamwidth %[$\theta_{\text{3dB}}$] 
        & \multicolumn{2}{c|}{$65^\circ$}
        \tabularnewline
        
        Horizontal half-power beamwidth % [$\phi_{\text{3dB}}$]
        & \multicolumn{2}{c|}{$65^\circ$}
        \tabularnewline \hline
    \end{tabular}
\end{table}


% Sundeep's notes:
% \begin{itemize}
%     \item $E_{\rm sig} = P_A T$ where $T=1/f_{s}$.
% \end{itemize}

Fig.~\ref{fig:snr_distributions} shows the SNR distributions  for the designs discussed in Section~\ref{sec:link} at \SI{28}{GHz} and \SI{140}{GHz}. As a performance benchmark,
we compare the SNR distribution under three models:
\begin{itemize}
    \item SNR with adjacent carrier interference and in-band distortion:  This is the model
    \eqref{eq:SNR} with the parameters
    for $\alpha_1$ and $\alpha_2$ in
    Table~\ref{tab:model_param} found from
    the circuit simulations.  The
    resulting SNR CDF is shown
    in the dashed line in Fig.~\ref{fig:snr_distributions}.
    
    \item SNR with no adjacent carrier interference and in-band distortion:  
    This is the model
    \eqref{eq:SNR} with the parameters
    for $\alpha_1$ in
    Table~\ref{tab:model_param} but
    $\alpha_2 = 0$.  The SNR CDF is shown in the
    dotted line in Fig.~\ref{fig:snr_distributions}.
    
    \item SNR with no adjacent carrier interference and  no in-band distortion:  
    This is the model
    \eqref{eq:SNR} with the parameters
    for $\alpha_1=\alpha_2=0$.  
    The resulting SNR CDF is shown in the
    solid line in Fig.~\ref{fig:snr_distributions}.

    
\end{itemize}
%we show an upper bound of the SNR considering a linear system that includes only the NF from the RFFE devices without any distortion (solid line).
%That is, the SNR in \eqref{eq:SNR} with
%$\alpha_1=\alpha_2=0$ so the effect of the
%distortion is removed.
% Similarly Fig.~\ref{fig:rate_distributions} shows the downlink rate distributions  for the designs discussed in Section~\ref{sec:link} at \SI{28}{GHz} and \SI{140}{GHz}.

Comparing the plots, we observe that the impact of distortion from adjacent carrier interference  is negligible.   This suggests that for these parameters,
there may be no need for extra filtering in the RF/IF or baseband to suppress the adjacent carrier interference.   Thus, we can conclude that by optimizing the RFFE devices and reducing the dynamic range of the system, we can improve the energy efficiency without being vulnerable from the adjacent carrier interference. Furthermore, as expected the designs at \SI{28}{GHz} have lower saturation points comparing to the \SI{140}{GHz} designs, %as they have lower number of antennas and so lower beamforming gain
due to the difference in the beamforming gain resulted from the difference in the number of antennas.

As explained in Section~\ref{sec:link}, at \SI{140}{GHz} we expect the designs to have different performance. In the low-SNR regime Design$^{(1)}$ performs better due to the lower NF, while Design$^{(2)}$ performs better in the high-SNR regime due to the larger number of ADC bits.

% \textcolor{blue}{[SR: I think the rate plots
% are wrong.  It looks like you are computing the
% rate via, $R=B \log_2(1 + \gamma)$.  This allows
% the rate to be very high since the system can
% benefit from very high SNRs.  But, 
% a more realistic model is:
% $R = B \max\{\rho_{\rm max},\alpha \log(1 + \gamma) \}$ where $\alpha=0.6$ and $\rho_{\rm max} = 4.5$ bps/Hz.  
% You should remove the plot, or correct it.]}

% mainly because of the beamforming gains

% We observe that the effect of adjacent carrier interference is negligible, and the SNR values are strongly affected by distortion instead. 

% In comparison with two carrier frequencies, \SI{28}{GHz} and \SI {140}{GHz}, the interference effects are largely imperceptible at \SI{140}{GHz} because of the shorter beam width at higher frequencies. In addition, we leverage the larger number of antenna arrays for \SI{140}{GHz} than \SI{28}{GHz}, which leads to larger saturation point of SNR values caused by the signal distortion than the case of \SI{28}{GHz}. \\
% \textcolor{red}{ 1) we need to explain the SNR gap in 140GHz for linear case and 2) the relationship between the signal distortion and frequency; why 140GHz shows little distortion effect compared to 28GHz case. 3) we might add the the saturation point for each frequency is reasonable considering access paper and antenna configurations, and more explanation. }





%% Notes for SNR plot 
% Desgin (i) = RFFE non linear distortion + interference
% Design (i) (w/o interference) = non-linear rffe w/o the presence of interference.
% Design (i) (linear) = linear RFFE no intereference 
% SJ: Let me know if you have any questions.

%In comparison with \SI{28}{GHz} case, the SNR values of \SI{140}{GHz} saturates at larger 

% \begin{figure*}[t]
% \centering
%     \subfigure[\SI{28}{GHz}]{\includegraphics[width=0.49\textwidth]{fig/snr_28.eps}}
%     \subfigure[\SI{140}{GHz}]{\includegraphics[width=\columnwidth]{fig/snr_140.eps}}
%   \caption{Rate distribution for two different frequency cases}
%   \label{fig:rate_distributions}
% \end{figure*}

%\textcolor{red}{
%Comments for SJ:
%\begin{itemize}
%    \item Explain beamforming in the simulation setup.
%    \item Elaborate on path loss
%    \item Explain figure 5 and 6
%\end{itemize}
%}
% \vspace{-0.5em}
\section{Conclusion}
% \vspace{-0.5em}
Recent advances in multimodal single-cell technology have enabled the simultaneous profiling of the transcriptome alongside other cellular modalities, leading to an increase in the availability of multimodal single-cell data. In this paper, we present \method{}, a multimodal transformer model for single-cell surface protein abundance from gene expression measurements. We combined the data with prior biological interaction knowledge from the STRING database into a richly connected heterogeneous graph and leveraged the transformer architectures to learn an accurate mapping between gene expression and surface protein abundance. Remarkably, \method{} achieves superior and more stable performance than other baselines on both 2021 and 2022 NeurIPS single-cell datasets.

\noindent\textbf{Future Work.}
% Our work is an extension of the model we implemented in the NeurIPS 2022 competition. 
Our framework of multimodal transformers with the cross-modality heterogeneous graph goes far beyond the specific downstream task of modality prediction, and there are lots of potentials to be further explored. Our graph contains three types of nodes. While the cell embeddings are used for predictions, the remaining protein embeddings and gene embeddings may be further interpreted for other tasks. The similarities between proteins may show data-specific protein-protein relationships, while the attention matrix of the gene transformer may help to identify marker genes of each cell type. Additionally, we may achieve gene interaction prediction using the attention mechanism.
% under adequate regulations. 
% We expect \method{} to be capable of much more than just modality prediction. Note that currently, we fuse information from different transformers with message-passing GNNs. 
To extend more on transformers, a potential next step is implementing cross-attention cross-modalities. Ideally, all three types of nodes, namely genes, proteins, and cells, would be jointly modeled using a large transformer that includes specific regulations for each modality. 

% insight of protein and gene embedding (diff task)

% all in one transformer

% \noindent\textbf{Limitations and future work}
% Despite the noticeable performance improvement by utilizing transformers with the cross-modality heterogeneous graph, there are still bottlenecks in the current settings. To begin with, we noticed that the performance variations of all methods are consistently higher in the ``CITE'' dataset compared to the ``GEX2ADT'' dataset. We hypothesized that the increased variability in ``CITE'' was due to both less number of training samples (43k vs. 66k cells) and a significantly more number of testing samples used (28k vs. 1k cells). One straightforward solution to alleviate the high variation issue is to include more training samples, which is not always possible given the training data availability. Nevertheless, publicly available single-cell datasets have been accumulated over the past decades and are still being collected on an ever-increasing scale. Taking advantage of these large-scale atlases is the key to a more stable and well-performing model, as some of the intra-cell variations could be common across different datasets. For example, reference-based methods are commonly used to identify the cell identity of a single cell, or cell-type compositions of a mixture of cells. (other examples for pretrained, e.g., scbert)


%\noindent\textbf{Future work.}
% Our work is an extension of the model we implemented in the NeurIPS 2022 competition. Now our framework of multimodal transformers with the cross-modality heterogeneous graph goes far beyond the specific downstream task of modality prediction, and there are lots of potentials to be further explored. Our graph contains three types of nodes. while the cell embeddings are used for predictions, the remaining protein embeddings and gene embeddings may be further interpreted for other tasks. The similarities between proteins may show data-specific protein-protein relationships, while the attention matrix of the gene transformer may help to identify marker genes of each cell type. Additionally, we may achieve gene interaction prediction using the attention mechanism under adequate regulations. We expect \method{} to be capable of much more than just modality prediction. Note that currently, we fuse information from different transformers with message-passing GNNs. To extend more on transformers, a potential next step is implementing cross-attention cross-modalities. Ideally, all three types of nodes, namely genes, proteins, and cells, would be jointly modeled using a large transformer that includes specific regulations for each modality. The self-attention within each modality would reconstruct the prior interaction network, while the cross-attention between modalities would be supervised by the data observations. Then, The attention matrix will provide insights into all the internal interactions and cross-relationships. With the linearized transformer, this idea would be both practical and versatile.

% \begin{acks}
% This research is supported by the National Science Foundation (NSF) and Johnson \& Johnson.
% \end{acks}
% \documentclass[11pt]{article}

%--------- Packages -----------
% \usepackage[margin=1in]{geometry}
\usepackage[margin=1in,footskip=0.25in]{geometry}
\usepackage{fullpage}
\usepackage{amssymb}
\usepackage{amsmath}
\usepackage{amsthm}
\usepackage{color}
\usepackage{enumitem}
\usepackage{url}
\usepackage{listings}
\usepackage{bbm}
\usepackage{matlab-prettifier}
\usepackage{mathtools}
\usepackage{caption}
\usepackage{algorithm}
\usepackage{algpseudocode}
\usepackage{subcaption}
\usepackage{float}
\usepackage{hyperref}

%-------------Notation--------------
%\usepackage{algorithm,algorithmic}
\usepackage{shorthands}
%----------Spacing-------------
\setlength{\topmargin}{-0.6 in}
\setlength{\headsep}{0.75 in}
\setlength{\parindent}{0 in}
\setlength{\parskip}{0.1 in}
\setlength{\hoffset}{0 in}



%----------Header--------------
\title{On a Relation Between the Rate-Distortion Function and \\Optimal Transport}
\author{Eric Lei}


%========= BEGIN DOCUMENT =========

\begin{document}
    \maketitle

    \section{Introduction}
    \subsection{Rate-Distortion}

    Let $X \sim P_X$ be the random variable we wish to compress, supported on $\mathcal{X}$. Let $\mathcal{Y}$ be the reproduction space, and $\dist: \mathcal{X} \times \mathcal{Y} \rightarrow \mathbb{R}_{\geq 0}$ be a distortion measure. The asymptotic limit on the minimum number of bits required to achieve average distortion $D$ is given by the rate-distortion function \cite{shannonRD, CoverThomas},  defined as
 %   \begin{definition} The rate-distortion function $R(D)$ of a source $P_X$ under distortion function $\dist$ is given by 
        \begin{equation}
    	    R(D) := \inf_{\substack{P_{Y|X}:  \EE_{P_{X,Y}}[\dist(X,Y)]  \leq D}} I(X;Y), 
    	    \label{eq:RD}
    	\end{equation} 
   % \end{definition} 
    where $P_{X,Y}=P_X P_{Y|X}$ is a joint distribution. Any rate-distortion pair $(R,D)$ satisfying $R > R(D)$ is achievable by some lossy source code, and no code can achieve a rate-distortion less than $R(D)$. It is important to note that $R(D)$ is achievable only under asymptotic blocklengths.
    

    Let $\DKL(\mu || \nu)$ be the Kullback-Leibler (KL) divergence, defined as $\EE_\mu\bracket{\log \frac{d\mu}{d\nu}}$ when the density $\frac{d\mu}{d\nu}$ exists and $+\infty$ otherwise. $R(D)$ has the following alternate form \cite[Ch.~10]{CoverThomas},
    \begin{equation}
    R(D) = \inf_{Q_Y} \inf_{\substack{P_{Y|X}:  \EE_{P_{X,Y}}[\dist(X,Y)]  \leq D}} \DKL(P_{X,Y}||P_X \otimes Q_Y).
    \label{eq:RD_alt}
    \end{equation}
    Due to the convex and strictly decreasing properties \cite{CoverThomas} of $R(D)$, it suffices to fix some $\lambda > 0$, and solve 
    \begin{equation}
        \inf_{Q_Y} \inf_{P_{Y|X}}\DKL(P_{X,Y}|| P_X \otimes Q_Y) + \lambda \mathop{\EE}_{P_{X,Y}}[\dist(X,Y)].
        \label{eq:RD_alt_regl}
    \end{equation}

    

    % \begin{lemma}[Double-Minimization Form, cf. {\cite[Ch.~10]{CoverThomas}}, {\cite{Yeung2002AFC}}]
    %     The minimizers $P^{(\beta)}_{Y|X}$, $Q_Y^{(\beta)}$ of
    %     \begin{equation}
	   %   \RD(\beta):=\inf_{Q_Y} \inf_{P_{Y|X}   } \DKL(P_{X,Y}|| P_X \otimes Q_Y) + \beta \mathop{\EE}_{P_{X,Y}}[\dist(X,Y)]
	   %   \label{eq:double}
	   %  \end{equation}
	   %  yield a unique point $R_\beta = \DKL(P_XP^{(\beta)}_{Y|X} || P_X \otimes Q_Y^{(\beta)})$ and $D_\beta = \EE_{P_X P^{(\beta)}_{Y|X}} [\dist(X,Y)]$ on the positive-rate regime of the rate-distortion curve, i.e. $R(D_\beta) = R_\beta$, such that $D_\beta < D_{\mathrm{max}}$ where $R(D_{\mathrm{max}}) = 0$.
    %     \begin{proof}
    %     See \cite[Lemma~10.8.1]{CoverThomas}.
    %     \end{proof}
    % \end{lemma}
	
    In discrete settings,  the optimizers $Q_Y, P_{Y|X}$ satisfy
    \begin{align}
        \frac{dP_{Y|X=x}}{dQ_Y}(x,y) &= \frac{e^{-\lambda \dist(x,y)}}{\int_{\mathcal{Y}} e^{-\lambda \dist(x,\tilde{y})}dQ_Y} , \label{eq:BA_1} \\
        Q_Y &= \int_{\mathcal{X}}  dP_{Y|X} dP_X. 
        \label{eq:BA_2}
    \end{align}

    This solution corresponds to a point on $R(D)$ corresponding to $\lambda$. The Blahut-Arimoto (BA) algorithm \cite{blahut, arimoto} solves \eqref{eq:RD_alt} by alternating steps on $P_{Y|X}$ and $Q_Y$ until convergence. Similarly, NERD \cite{NERD} provides a variant for continuous sources when only samples are available. Sweeping over $\lambda$ gives the entire rate-distortion curve.
    

    \subsection{Optimal Transport}
    Optimal transport theory has two formulations; the Monge problem, and the Kantorovich relaxation. Intuitively, suppose we have two sets of points $\mathfrak{X} = \{x_1,\dots,x_n\}$ and $\mathfrak{Y} = \{y_1,\dots,y_n\}$. The Monge problem seeks to find a matching (i.e., a bijection) between $\mathfrak{X}$ and $\mathfrak{Y}$ that minimizes the average pairwise cost. The Kantovorich problem relaxes the bijection requirement by allowing each $x_i$ to be soft-matched to all $y_j$'s, i.e., each $x_i$ can send some mass to each $y_j$ such that the total mass sent sums to 1. In the discrete case, they have the same optimal solution, which is actually given by a hard-matching, i.e., a bijection. In the case for general measures, they are only equivalent under certain conditions. We will only discuss the Kantorovich case since that is most relevant.

    The Kantorovich problem seeks to find the minimum distortion coupling between two probability measures $\mu$ and $\nu$,
    \begin{equation}
        W(\mu, \nu) := \inf_{\substack{\pi \in \Pi(\mu, \nu)}} \EE_{X,Y\sim \pi}[\dist(X,Y)],
    \end{equation}
    where $\pi$ is a joint distribution that marginalizes to $\mu$ and $\nu$. The Kantorovich problem can be regularized with an entropy term, 
    \begin{equation}
        S_\epsilon(\mu, \nu) := \inf_{\substack{\pi \in \Pi(\mu, \nu)}} \EE_{\pi}[\dist(X,Y)] + \epsilon \DKL(\pi||\mu \otimes \nu),
        \label{eq:entropic_OT}
    \end{equation}
    which is known as entropy-regularized optimal transport, with regularization $\epsilon > 0$. For discrete measures $\mu,\nu$, \eqref{eq:entropic_OT} can be solved efficiently using the Sinkhorn-Knopp algorithm \cite{sinkhorn, knopp_sinkhorn}.

    A brief introduction to optimal transport theory can be found in \cite{CourseNotesOT, ComputationalOT}. 

    \section{Optimal Transport and the Rate-Distortion Function}
    \subsection{Equivalence of Extremal Sinkhorn Divergence and Rate Distortion}
    We first show that entropic OT can be used to upper bound $R(D)$. First, observe that the inner minimization problem in \eqref{eq:RD_alt_regl} looks similar to the entropic OT problem. Let us define 
    \begin{align}
        S(D) :=& \inf_{Q_Y} \inf_{\substack{\pi_{XY} \in \Pi(P_X, Q_Y): \\ \EE_{\pi_{X,Y}}[\dist(X,Y)]  \leq D }} \DKL(\pi_{X,Y}||P_X \otimes Q_Y) \label{eq:SD_1}\\ 
        =&\inf_{Q_Y} \inf_{\substack{P_{Y|X}: \\ \EE_{P_{X,Y}}[\dist(X,Y)]  \leq D \\ Q_Y = \int_{\mathcal{X}} dP_{Y|X}P_X}} \DKL(P_{X,Y}||P_X \otimes Q_Y), \label{eq:SD_2}
    \end{align}
    which we call the \emph{Sinkhorn-distortion function}. Similar to $R(D)$, we can trace out $S(D)$ by sweeping over $\lambda > 0$, and solving the inner minimization \eqref{eq:entropic_OT} by setting  $\epsilon = \frac{1}{\lambda}$, and then optimizing over all $Q_Y$ for the outer minimization, which is convex in $Q_Y$ \cite{feydy2019interpolating}. We call this \texttt{SinkhornRD}. A variant of the Sinkhorn-distortion function is often used to solve generative modeling tasks with Sinkhorn divergences \cite{sinkhornGAN, salimans2018improving, shen2020sinkhorn}, where one wishes to find some $Q_Y \approx P_X$ by solving $\min_{Q_Y} S_\epsilon (P_x, Q_Y)$. The neural methods parallel NERD.

	    
	    \begin{proposition}[Sinkhorn Distortion Upper Bound]
	    \label{prop:SD_UB}
	    Given source $P_X$ on $\mathcal{X}$, reconstruction space $\mathcal{Y}$, and distortion measure $\dist$,  
	    \begin{equation}
	        R(D) \leq S(D).
	        \label{eq:RD-Sinkhorn}
	    \end{equation}
	    \end{proposition}
	   \begin{proof}
            The inner minimization problem of $R(D)$ in \eqref{eq:RD_alt} only has a marginal constraint on $P_X$, whereas the inner minimization of $S(D)$ in  \eqref{eq:SD_2} has an additional marginal constraint on $Q_Y$ as well.
    	\end{proof}
        
        

    	
    	% \begin{figure*}[t]
     %         \centering
     %         \begin{subfigure}[b]{0.475\textwidth}
     %             \centering
     %             \includegraphics[width=0.9\textwidth]{figures/RDcurve_discrete_bad.pdf}
     %             \caption{$R(D)$ (via Blahut-Arimoto) and entropic OT upper bound on a discrete source with 6 atoms.}
     %             \label{fig:simple}
     %         \end{subfigure}
     %         \hfill
     %         \begin{subfigure}[b]{0.475\textwidth}
     %             \centering
     %             \includegraphics[width=0.9\textwidth]{figures/RD_minmax_mnist_RD_sinkhorn.pdf}
     %             \caption{$R(D)$ (via NERD) and entropic OT upper bound on MNIST images.}  
     %            \label{fig:mnist_sinkhorn}
     %         \end{subfigure}
     %        \caption{Comparison of the RD-Sinkhorn objective with the true rate-distortion objective. }
     %     \end{figure*}
    	
    	% Under discrete settings, \eqref{eq:entropic_OT} can be solved efficiently via Sinkhorn iterations \cite{lightspeed}. We call the upper bound in \eqref{eq:RD-Sinkhorn} \textit{RD-Sinkhorn}. In order to illustrate the behavior of RD-Sinkhorn, we first examine the rate-distortion curve of a discretized Gaussian source over 6 atoms, shown in Fig.~\ref{fig:simple}. In such a setting, we can compute the true rate-distortion curve and RD-Sinkhorn easily and exactly, via Blahut-Arimoto and Sinkhorn iterations respectively.  Although RD-Sinkhorn traces a curve that is not convex, we see that for large portions of the rate-distortion curve, the gap is small and RD-Sinkhorn empirically provides a good estimate of the true rate-distortion function. 
     
    	% We then apply RD-Sinkhorn to an MNIST source by parametrizing $Q_Y$ with a generator neural network, similar to the work in \cite{sinkhornGAN}. To estimate $R(D)$ in this setting, we use the neural estimator NERD \cite{NERD}, and compare the curves in Fig.~\ref{fig:mnist_sinkhorn}.


        Next, we show that without further assumptions, $R(D)$ and $S(D)$ are equivalent.
        \begin{theorem}[Sinkhorn-Rate-Distortion Equivalence]
            For any source $P_X$ and distortion function $\dist: \mathcal{X} \times \mathcal{Y} \rightarrow \mathbb{R}_{\geq 0}$, it holds that
            \begin{equation}
                R(D) = S(D).
            \end{equation}
        \end{theorem}
        \begin{proof}
            We know that the optimizing distributions for $R(D)$ must satisfy \eqref{eq:BA_1} and \eqref{eq:BA_2} simultaneously. To show that $S(D)$ achieves the same objective as $R(D)$ on the same $P_X$ and distortion measure, it suffices to show that the $R(D)$-optimal $Q_Y^*$ and $P_{Y|X}^*$ are feasible for $S(D)$. From \cite[Ch.~4, Prop.~4.3]{ComputationalOT}, the optimal coupling in entropic OT is unique and has the form 
            \begin{equation}
            \frac{d\pi}{dP_X dQ_Y} (x,y) = u(x) e^{-\lambda \dist(x,y)} v(y)
            \end{equation}
            where $u(x), v(y)$ are dual variables that ensure valid couplings. The $R(D)$-optimal joint distribution $P_XP_{Y|X}^*$, which is guaranteed to be a coupling between $P_X$ and $Q_Y^*$ due to \eqref{eq:BA_2}, indeed has the form
            \begin{equation}
                \frac{dP_XP_{Y|X}^*}{dP_X dQ_Y^*}(x,y) = \frac{1}{\int_{\mathcal{Y}}e^{-\lambda \dist(x,y')}dQ_Y^*} \cdot e^{-\lambda \dist(x,y)} \cdot 1
            \end{equation}
           where the first term only depends on $x$ and the last term only depends on $y$.
            Since from Prop.~\ref{prop:SD_UB}, $R(D)$ is a lower bound of $S(D)$, we are done.
        \end{proof}
        \begin{remark}
            This result implies that (i) Sinkhorn generative models are equivalent to solving $R(D)$, and (ii) the proposed \emph{\texttt{SinkhornRD}} algorithm can also solve $R(D)$ and is an alternative to Blahut-Arimoto. \textcolor{blue}{Perhaps some implications of the rate-distortion optimal codebook from the lens of entropic OT?}
        \end{remark}

        \begin{figure}
            \centering
            \includegraphics[width=0.5\linewidth]{figures/5atoms_comparison.pdf}
            \caption{Equivalence of $S(D)$ and $R(D)$ on a discrete source with $\dist(x,y)=(x-y)^2$. }
            \label{fig:SD-RD}
        \end{figure}

        We numerically verify the equivalence in Fig.~\ref{fig:SD-RD} on a discrete source with 5 atoms under squared-error distortion. For $R(D)$, we use Blahut-Arimoto, and for $S(D)$, we solve the convex problem using sequential quadratic programming \cite{SLSQP, 2020SciPy} with $Q \mapsto S_{\epsilon}(P_X, Q)$ as the objective function, showing that the two different objectives result in the same function.

        \subsection{Entropic OT and the Rate Function}
        \begin{proposition}
            We have that 
            \begin{equation}
                R(D, Q_Y) = S(D, Q_Y)
            \end{equation}
            if and only if 
            \begin{equation}
                \int_{\mathcal{X}} \frac{e^{-\lambda \dist(x,y)}}{\int_{\mathcal{Y}} e^{-\lambda \dist(x,y')} dQ_Y} dP_X = 1, \quad \forall y \in \mathcal{Y}.
                \label{eq:rate_eq_cond}
            \end{equation}
        \end{proposition}
        \begin{proof}
            Note that for a given $Q_Y$, the solution to $R(D,Q_Y)$ is given by \eqref{eq:BA_1}. The optimizing $P_{Y|X}$, however, does not necessarily produce a valid coupling for the joint $P_X P_{Y|X}$ (we only know this holds for the $R(D)$-optimal $Q_Y$). In order for $P_XP_{Y|X}$ to be a valid coupling, the $Q_Y$ marginal constraint must be satisfied, i.e. $\forall y \in \mathcal{Y}$,
            \begin{align}
                \int_{\mathcal{X}} dP_{Y|X}(y) dP_X = dQ_Y(y) &\iff \int_{\mathcal{X}} \frac{e^{-\lambda \dist(x,y)}}{\int_{\mathcal{Y}} e^{-\lambda \dist(x,y')} dQ_Y} dQ_Y(y) dP_X(x) = dQ_Y(y)\\
                & \iff \int_{\mathcal{X}} \frac{e^{-\lambda \dist(x,y)}}{\int_{\mathcal{Y}} e^{-\lambda \dist(x,y')} dQ_Y} dP_X = 1.
            \end{align}
            When a coupling is satisfied, $P_XP_{Y|X}$ is feasible for $S(D, Q_Y)$, and since $R(D, Q_Y) \leq S(D, Q_Y)$, we are done.
        \end{proof}
        \begin{corollary}
            Equivalent when $P_X$ and $Q_Y$ are uniform discrete and the distortion matrix is symmetric (rows and columns are permutations of eachother). Note that \eqref{eq:BA_1} immediately provides the optimal Sinkhorn coupling in this case (matching two sets of points with symmetric cost). Also, note that this induces a weakly symmetric $R(D)$-optimal channel.
        \end{corollary}
        \begin{proof}
            When $P_X = \text{Unif}(\mathcal{X})$, $Q_Y = \text{Unif}(\mathcal{Y})$, \eqref{eq:rate_eq_cond} evaluates to 
            \begin{equation}
                \sum_{x \in \mathcal{X}} 
            \end{equation}
        \end{proof}
        

    \section{Optimal Transport and Channel Capacity}
    Given a channel $p(y|x)$, the channel capacity \cite{shannon48} describes its maximal rate of communication, given by 
    \begin{equation}
        C = \max_{r(x)} I(X;Y).
        \label{eq:capacity}
    \end{equation}
    We show that \eqref{eq:capacity} can also be interpreted as an extremal optimal transport problem. 
    \begin{theorem}
        For any channel $p(y|x)$, we have that 
        \begin{equation}
            C = 2\cdot  \max_{r(x)} S_{\frac{1}{2}}(r(x), r(y)),
        \end{equation}
        where $r(y) := \sum_{x \in \mathcal{X}} p(y|x)r(x)$, and the cost function that induces the Sinkhorn problem in \eqref{eq:entropic_OT} is given by $\dist(x,y) = - \log \frac{p(y|x)}{r(y)}$. 
        \begin{proof}
            Using \cite[Lemma.~10.8.1]{CoverThomas}, we have that 
            \begin{equation}
                C = \max_{r(x)} \min_{q(x|y)} \EE_{r(x)p(y|x)} \bracket*{-\log \frac{q(x|y)}{r(x)}},
                \label{eq:capacity_alt}
            \end{equation}
            with the optimal $q^*(x|y)$ satisfying $q^*(x|y)r(y) = r(x)p(y|x)$. Hence, the inner problem in \eqref{eq:capacity_alt} under $q^*(x|y)$ can be rewritten as 
            \begin{align}
                \EE_{r(x)p(y|x)}\bracket*{-\log \frac{q^*(x|y)}{r(x)}} &= 2\EE_{r(y)q^*(x|y)} \bracket*{-\log \frac{p(y|x)}{r(y)}} + \EE_{r(y)q^*(x|y)} \bracket*{\log \frac{q^*(x|y)}{r(x)}} \\
                &= 2\EE_{r(y)q^*(x|y)} \bracket*{\dist(x,y)} + \DKL(r(y)q^*(x|y)||r(y)r(x)). \label{eq:optimal_sinkhorn}
            \end{align}
            To show that \eqref{eq:optimal_sinkhorn} is equal to $2 \cdot S_{\frac{1}{2}}(r(x), r(y))$, we again use the fact from \cite[Ch.~4]{ComputationalOT} that the optimal Sinkhorn coupling is unique and has the form $\pi(x,y) = u(x) e^{-2 \dist()$
        \end{proof}
    \end{theorem}
    

    

    \section{Applications}
    Potentially:
    \begin{enumerate}
        \item Optimal transport lens of information theory
        \item Operational meaning of codes, coding for optimal transport/matching?
        \item Connecting generalization bounds in learning theory that use optimal transport and rate-distortion theory
    \end{enumerate}


\bibliographystyle{ieeetr} 
\bibliography{ref}



\end{document}



% \appendices
% \section{Proofs from \secref{sec:qkd}}
\label{app:proofs}

In \secref{sec:qkd} we show how to define the security of QKD in a
composable framework and relate this to the trace distance security
criterion introduced in \textcite{Ren05}. This composable treatment of
the security of QKD follows the literature \cite{BHLMO05,MR09}, and
the results presented in \secref{sec:qkd} may be found in
\textcite{BHLMO05,MR09} as well. The formulation of the statements
differs however from those works, since we use here the Abstract
Cryptography framework of \textcite{MR11}. So for completeness, we
provide here proofs of the main results from \secref{sec:qkd}.

\begin{proof}[Proof of \thmref{thm:qkd}]
  Recall that in \secref{sec:security.simulator} we fixed the
  simulator and show that to satisfy \eqnref{eq:qkd.security} it is
  sufficient for \eqnref{eq:qkd.security.2} to hold. Here, we will
  break \eqnref{eq:qkd.security.2} into security [\eqnref{eq:qkd.cor}]
  and correctness [\eqnref{eq:qkd.sec}], thus proving the theorem.

  Let us define $\gamma_{ABE}$ to be a state obtained from
  $\rho^{\top}_{ABE}$ [\eqnref{eq:qkd.security.tmp}] by throwing away
  the $B$ system and replacing it with a copy of $A$, i.e., \[
  \gamma_{ABE} = \frac{1}{1-p^\bot} \sum_{k_A,k_B \in \cK} p_{k_A,k_B}
  \proj{k_A,k_A} \otimes \rho^{k_A,k_B}_E.\] From the triangle
  inequality we get \begin{multline*} D(\rho^\top_{ABE},\tau_{AB} \otimes
  \rho^\top_{E}) \leq \\ D(\rho^\top_{ABE},\gamma_{ABE}) +
  D(\gamma_{ABE},\tau_{AB} \otimes \rho^\top_{E}) .\end{multline*}

Since in the states $\gamma_{ABE}$ and
$\tau_{AB} \otimes \rho^\top_{E}$ the $B$ system is a copy of the $A$
system, it does not modify the distance. Furthermore,
$\trace[B]{\gamma_{ABE}} =
\trace[B]{\rho^{\top}_{ABE}}$. Hence
\[D(\gamma_{ABE},\tau_{AB} \otimes \rho^\top_{E}) =
  D(\gamma_{AE},\tau_{A} \otimes \rho^\top_{E}) =
  D(\rho^\top_{AE},\tau_{A} \otimes \rho^\top_{E}).\]

For the other term note that
\begin{align*}
  & D(\rho^\top_{ABE},\gamma_{ABE}) \\
  & \qquad \leq \sum_{k_A,k_B} \frac{p_{k_A,k_B}}{1-p^{\bot}}
    D\left(\proj{k_A,k_B} \otimes \rho^{k_A,k_B}_E,\right. \\
  & \qquad \qquad \qquad \qquad \qquad \qquad \left.\proj{k_A,k_A} \otimes \rho^{k_A,k_B}_E \right)\\
  & \qquad = \sum_{k_A \neq k_B} \frac{p_{k_A,k_B}}{1-p^{\bot}} = \frac{1}{1-p^{\bot}}\Pr
  \left[ K_A \neq K_B \right].
\end{align*}
Putting the above together with \eqnref{eq:qkd.security.2}, we get
\begin{align*} & D(\rho_{ABE},\tilde{\rho}_{ABE}) \\
  & \qquad = (1-p^\bot)
  D(\rho^\top_{ABE},\tau_{AB} \otimes \rho^\bot_{E}) \\ & \qquad \leq \Pr
  \left[ K_A \neq K_B \right] + (1-p^\bot) D(\rho^\top_{AE},\tau_{A}
  \otimes \rho^\top_{E}). \qedhere \end{align*}
\end{proof}

\begin{proof}[Proof of \lemref{lem:robustness}]
  By construction, $\aK_\delta$ aborts with exactly the same
  probability as the real system. And because $\sigma^{\qkd}_E$
  simulates the real protocols, if we plug a converter $\pi_E$ in
  $\aK\sigma^{\qkd}_E$ which emulates the noisy channel $\aQ_q$ and
  blogs the output of the simulated authentic channel, then
  $\aK_\delta = \aK\sigma^{\qkd}_E\pi_E$. Also note that by
  construction we have
  $\aQ_q \| \aA' = \left(\aQ \| \aA\right) \pi_E$. Thus
  \begin{multline*} d\left( \pi_A^{\qkd}\pi_B^{\qkd}(\aQ_q \| \aA')
      ,\aK_\delta\right) \\ = d\left( \pi_A^{\qkd}\pi_B^{\qkd}\left(\aQ
        \| \aA\right) \pi_E , \aK\sigma^{\qkd}_E\pi_E\right). \end{multline*}

  Finally, because the converter $\pi_E$ on both the real and ideal
  systems can only decrease their distance (see
  \secref{sec:ac.systems}), the result follows.
\end{proof}


%%% Local Variables:
%%% TeX-master: "main.tex"
%%% End:



\bibliographystyle{IEEEtran}
\bibliography{bibl}
\end{document}