\documentclass[aps,pra,showpacs,twoside,twocolumn,10pt]{revtex4-1}
\usepackage[colorlinks=true, citecolor=red, urlcolor=blue ]{hyperref}
\usepackage{epsfig,newlfont,amssymb,amsfonts,amsmath,bm,subfigure,palatino,mathtools,amsthm,braket,soul,enumitem,color}
\usepackage[normalem]{ulem}
\newcommand{\mycite}[1]{\textcolor{red}{[#1]}}
\newcommand{\mycomment}[1]{\textcolor{blue}{[\textbf{#1}]}}
\newcommand{\stkout}[1]{\ifmmode\text{\sout{\ensuremath{#1}}}\else\sout{#1}\fi}

\begin{document}
%\title{A gedanken for the amplification of the \emph{grin} of a cat while decoupled from any other orthogonal properties (the \emph{snarl}) of the cat as well as the cat itself}
%\title{Amplification of \emph{grin} of a cat independent of the cat}
%\title{Amplification of \emph{grin} (\emph{eyeshight}) of a cat by dislocating the \emph{snarl} as well as the cat (by \emph{cataract} surgery)}
\title{%06April2022: 
Isolating noise and amplifying signal with quantum Cheshire cat}
\author{Ahana Ghoshal$^1$, Soham Sau$^{1,2,3}$, Debmalya Das$^{4,5}$, Ujjwal Sen$^1$}
\affiliation
{$^1$Harish-Chandra Research Institute,  A CI of Homi Bhabha National Institute, Chhatnag Road, Jhunsi, Prayagraj 211 019, India\\
%$^2$Harish-Chandra Research Institute, HBNI, Chhatnag Road, Jhunsi, Allahabad 211 019, India\\
$^2$Department of Physics, School of Physical Sciences, Central University of Rajasthan, Bandarsindri, Rajasthan, 305 817, India\\
$^3$RCQI, Institute of Physics, Slovak Academy of Sciences, Dúbravská Cesta 9, Bratislava 84511, Slovakia \\
$^4$Department of Physics and Astronomy, University of Rochester, Rochester, New York 14627, USA\\
$^5$Center for Coherence and Quantum Optics, University of Rochester, Rochester, New York 14627, USA}


\begin{abstract}
The so-called quantum Cheshire cat is a phenomenon in which an object, identified with a ``cat'', is dissociated from a  property of the object, identified with the ``grin'' of the cat.
%
%\textcolor{red}{[title and abstract paer dekhbo!]} The \emph{grin} (A property of an object) of a cat (the object) is shown to be decoupled from the cat itself and it is possible to detect the cat and its \emph{grin} in the different arms of a Mach-Zehnder interferometer by using weak value measurement in the gedanken of quantum Cheshire cats. 
We propose a thought experiment, similar to this phenomenon, 
%At first we take a \emph{clear} (noiseless) scenario where we propose a similar thought experiment 
with an interferometric setup,  where a property (a component of polarization) of an object (photon) can be separated from the object itself and can simultaneously be amplified when it is already decoupled from its object. We further show that this setup can be used to dissociate two complementary properties, e.g., two orthogonal components of polarization of a photon and identified with the grin and the snarl of a cat, from each other and one of them can be amplified while being detached from the other. Moreover, we extend the work to a noisy scenario, effected by a spin-orbit-coupling --like additional interaction term in the Hamiltonian for the measurement process, with the object in this scenario being identified with a so-called confused Cheshire cat. We devise a gedanken experiment in which such a ``confusion'' can be successfully dissociated from the system, and we find that the dissociation helps in the amplification of signals. %(\emph{confused} Cheshire cat) where the spin-orbit coupling may cause disturbances in some quantum technologies depending on the measurement of the required observable (grin of the cat) and devise a gedanken for the successful removal of the \emph{confusion} (noise) and simultaneously amplifying the \emph{grin} (weak value of the observable to be measured) by tickling (an interferometric setup).  
\end{abstract}

\maketitle

\section{Introduction}
In recent times, a technique known as weak-value amplification has been 
%widely 
used to amplify 
%\textcolor{red}{\sout{small or}} 
weak signals~\cite{U}. The method relies on extracting information from a system while minimally disturbing it~\cite{Barchielli1982, Busch1984, Caves1986}.  This weak measurement~\cite{AAV, Duck} of the system is performed using a weak coupling strength between the system and a meter.
%The whole process involves the system to prepare in a preselected state, after that by weak measurement of the system and then followed by a strong measreuement(postselection). 
 In the weak-value amplification method, a quantum system is initially prepared in a pure state, known as the pre-selected state, following which an observable is weakly measured. After the weak measurement of the observable, a strong measurement of a second observable is carried out on the principal system and a quantity called weak value~\cite{AAV,Duck,YA} is defined by post-selecting an outcome of the second measurement. The weak value of the observable is basically the average shift in the meter readings for the weak measurement corresponding to the post-selected state. Experimental observations of weak values have been reported in Refs.~\cite{Correa, Denkmayr, Sponar, Ashby, Pryde, Hosten, Lundeen, Cormann}.


%Although the current work discusses weak values in relation to signal amplification, it may be worthwhile to mention a few other applications of the idea.
%in passing. 
%\textcolor{cyan}{The factor which is responsible for the amplification and mostly exploited in this scheme is the weak value of an observable.} 
Although the present work discusses weak values in relation to signal amplification, it may be worthwhile to mention a few other applications of the idea.
Weak values can be used in the direct measurement of a photon wave function~\cite{Lundeen, L2} to measure the spin Hall effect~\cite{Hosten}, in quantum state tomography~\cite{tomo1, tomo2}, in the geometric description of quantum states~\cite{geometry}, and in state visualization~\cite{sv}. It also finds application in quantum thermometry~\cite{qt}, and measuring the expectation value of non-Hermitian operators~\cite{nonherm1, nonherm2}. 
%On the foundational side, w
Weak values have been shown to be acquire complex values~\cite{Jozsa} and weak values play important roles in the two-state vector formalism~\cite{twostate}, in the physical understanding of superoscillations~\cite{Super}, and in separating a quantum property from its system~\cite{Aharonov}. Weak measurements have been used to show a double violation of a Bell inequality by a single entangled pair~\cite{npj} and in quantum process tomography where sequential weak measurements are done on incompatible observables~\cite{m}.
A property of weak values that is of special interest to us is that it can lie outside the eigenvalue spectrum of the observable being weakly measured,  and can even be very large~\cite{AAV, Duck}. This aspect is exploited in weak-value amplification.

A significant concern in  experiments is the minimization of noise in the relevant signal and designing measurement techniques that achieve the same. Setting aside logistics and dependencies on other constraints imposed by the particular scenario, a measurement strategy with 
%\textcolor{red}{\sout{a greater signal to noise ratio}} 
a significantly reduced noise is typically  favored by experimentalists~\cite{SNR1, SNR2, SNR3, SNR4, SNR5, SNR6}. We consider a situation in which noise may be introduced during the process of weak-value amplification if the Hamiltonian coupling the system and the meter 
%\textcolor{red}{\sout{in Eq.~(\ref{eqn:ho})}} 
has extra undesired terms due to the effect of an environmental element. It therefore becomes  necessary to eliminate these terms or suppress them to obtain the amplification of the signal alone. 

An important ingredient of the strategy we discuss in this paper is a phenomenon known as the quantum Cheshire cat. In this gedanken experiment, based on a modified Mach-Zehnder interferometer, a photon can be detected in one arm while its circular polarization can be detected in the other arm, each being absent in the other arm, by measuring the respective weak values~\cite{Aharonov}. Thus, for a particular combination of pre-selected and post-selected states, the photon ``cat'' can be disembodied from its property ``grin'', leading to the name of the effect being chosen after a magical and enigmatic character in the celebrated literary work, \textit{Alice in Wonderland}~\cite{Alice}.
%[\textcolor{red}{[ei ref ta PRA soriye diyechhe]}]. 
This counter intuitive phenomenon has been experimentally observed using neutron interferometry in Ref.~\cite{Denkmayr} and using photon interferometry in Refs.~\cite{Correa, Sponar, Ashby, Kim2021}. The concept has been further expanded upon in Refs.~\cite{Bancal, At, Duprey, dynamic}, and different kinds of manipulations of the photon polarization, independent of the photon, have been achieved in Refs.~\cite{DasPati, Liu2020, DasSen, wavepar}.


%\textcolor{red}{[intro-r parer angshaTa pare dekhbo!]} 
In this paper, we present a gedanken experiment to amplify a property of a photon at a location, independent of the photon being present at the same location. We also show that the same gedanken setup can amplify one property of a photon independent of another property of the same, where the two properties are complementary and correspond to  non-commuting observables.
%orthogonal to each other.} 
We believe this is a technologically and fundamentally useful  amalgamation of the two areas, viz., the quantum Cheshire cat and weak-value amplification.
%, which could lead to potential applications in quantum information theory as well. 
As a demonstration, we consider a scenario where the polarization degree of the photon interacts with another degree of freedom and results in an additional term in the Hamiltonian that governs the measurement. We call the additional term 
the
spin-orbit coupling of the photon (cat), borrowing the nomenclature from the spin-orbit interaction in the relativistic treatment of an electron's dynamics. This additional term behaves as a noise component that changes the shift of the meter. The meter now gives a deflection proportional to the weak value of an effective observable that is different from the weak value of a polarization component, which was to be measured. To avoid the unwanted disturbances caused by the effect of noise in some amplification techniques, we then formulate a gedanken experiment to separate the required observable from the noisy part. We propose that, using this experimental setup, it is possible to reduce the noise effect on average (as a weak value primarily is) and amplify some quantum property by a weak-value measurement with a certain accuracy. 

%By taking the \emph{cat} analogy, the later part can be thought as \emph{an unusually big grin of the cat without any type of confusions, when the cat actually is in a confused (noisy) state}.
%'\emph{a cataract surgery of a cat, which results an extremely clear eyesight that the cat could have ever thought}'.%
\par
The rest of the paper is organized as follows. The ideas of weak measurement and quantum Cheshire cat are briefly recollected  in Sec.~\ref{section:2}. In Sec.~\ref{section:3}, we present the thought experiment, based on the concept of the quantum Cheshire cat, to amplify a property of a photon ($z$-component of polarization) independent of the photon itself. We extend the idea 
%with the same setup 
to amplify one property ($z$-component of polarization) independently of the other ($x$-component of polarization) for a noiseless ideal situation. 
%Following the previous literature, 
 In Sec.~\ref{section:4}, we evaluate the weak value of the effective observable, resulting from a noisy scenario, where the weak value is obtained from the shift of the meter, weakly coupled with the system, and where the noise is incorporated as a spin-orbit coupling. We also propose an experimental setup useful to amplify the weak value of the effective observable in that section. %\textcolor{red}{either the previous two or the following one sentence/s have to be there.} \textcolor{blue}{\sout{In Sec. \ref{section:4}, we identify an interferometric set-up of quantum Cheshire cat for amplifying the spin observable of a particle, separating the component of the spin to be measured from the spin-orbital coupling while the spin-orbital coupling is acting like a noise in the ideal system.}} 
 We summarize in Sec.~\ref{section: conclusion}. %conclude with some discussions on the implications of quantum Cheshire cat in Sec. \ref{section: conclusion}. \textcolor{red}{[last sentence not clear!]}



%\section{Preliminaries} 
%\label{section: Preli}
%In this section we discuss the theory of weak measurement, quantum Cheshire Cats and the technique of amplification of small signals with the help of weak value measurement procedure, %done in some previous papers \cite{AAV,Duck,U,Torres,Wei,nonU,WVAB,J}, 
%for the understanding of the goal of this paper. 
%\subsection{Weak Measurement}
%The weak value scheme~(\cite{AAV,Duck}) started with an experiment for measuring a spin component of a spin $\frac{1}{2}$ particle, obtaining a result which was far beyond the range of usual values. In this inceptionary work, the interaction Hamiltonian was taken as
%\begin{equation}
%\label{eqn:ho}
%    H_0=-g(t)\hat{q}\otimes \hat{X}
%\end{equation}
%where $\hat{q}$ is a canonical variable of the measuring device conjugate to momentum $\hat{p}$ and $g(t)$ is a time dependent coupling function with compact support near the time of the measurement (normalized such that its time integral is unity) and $\hat{X}$ is the initial observable to be measured. The initial state of the device is $\ket{\Phi_{in}}$ having an initial spread $\Delta P$. The $q$ and $p$-representation of $\ket{\Phi_{in}}$ are following, 
%\begin{eqnarray}
%\label{eq:meter}
 % \ket{\Phi_{in}}&=&
 % \begin{cases}
 % \int dq \; \phi_{in}(q)\ket{q} & (q\text{-representation})\\
%  \int dp \; \tilde{\phi}_{in}(p)\ket{p} & (p\text{-representation})
%  \end{cases}
%\end{eqnarray}
%where $\tilde{\phi}_{in}(p)$ is a Gaussian function with mean $0$ and width $\Delta p$ i.e.,
%\begin{eqnarray}
%\label{eq:meter1}
 % \phi_{in}(q)&\equiv&\langle q | \Phi_{in} \rangle= \exp \Big ( -\frac{q^2}{4\Delta^2}\Big), \nonumber\\
%  \tilde{\phi}_{in}(p)&\equiv&\langle p | \Phi_{in} \rangle= \exp \Big ( -\Delta^2 p^2 \Big)
%\end{eqnarray}
%Here $\Delta q \equiv \Delta$ and $\Delta p = 1/ 2 \Delta$ and $\hbar=1$. In the experiment, a quantum system is initially prepared in a pure state $\ket{\psi_{in}}$, also known as the pre-selected state and the weak measurement is executed by weakly entangling the system and the meter. After performing the weak measurement of the observable $\hat{X}$, a projective or strong measurement of a second observable $\hat{Y}$, which in general does not commute with $\hat{X}$, is performed and one of the outcomes, $\ket{\psi_f}$, is selected. This process is known as post-selection. The weak value of the observable $\hat{X}$ is basically the average shift in the meter readings for the weak measurement corresponding to the post-selected state. The weak value $X_w$ is interpreted as the value of an observable $\hat{X}$, between two strong measurements, one giving rise to the pre-selected state $\ket{\psi_{in}}$ and the other producing the post-selected state $\ket{\psi_f}$. The weak value of an observable $\hat{X}$ is defined as 
%\begin{equation}
 %   X_w = \frac{\Braket{\psi_f|X|\psi_{in}}}{{\Braket{\psi_f|\psi_{in}}}} 
%\label{eq:weak}
%\end{equation}
%The final state of the meter after the weak measurement becomes,
%\begin{eqnarray}
%\label{eq:shift}
%\phi_f &\approx& 
%\begin{cases}
%\Braket{\psi_f|\psi_{in}}\int dq \, e^{iqX_w}\exp(-\frac{q^2}{4\Delta^2})\ket{q} &\\ 
%\phantom{tara ei dinotatuku dekhe} (q-representation) &\\
%\Braket{\psi_f|\psi_{in}}\int dp \, \exp[-\Delta^2 (p-X_w)^2]\ket{q}&\\ 
%\phantom{tara ei dinotatuku dekhe} (p-representation) &
 %   \end{cases}
%\end{eqnarray}
%This is a broad Gaussian function centered at $X_w$. So, we get a deflection in the meter scale which is proportional to the weak value of the observable $\hat{X}$. If $\Delta p$ is much larger than the spread of eigenvalues of the observable, it corresponds to the regime of weak measurement. In this regime it cannot be measured with certainty whether the beam is displaced or not but by repeating the measurement for $N$ number of times and taking the average we can reduce the uncertainty of the beam displacement to $\Delta/N$ approximately. So the desired accuracy of the measurement can be obtained by increasing $N$ to a sufficiently large value. Here an obvious important point to be noted is that weak measurement is an average effect and not an immediate effect.   
%\subsection{Quantum Cheshire Cats}
%The basic exercise of quantum Chesire cat as demonstrated, in 2013~\cite{Aharonov}, is based on the weak measurement of the observables denoting the location of the photon and its circular polarization in the left ($\ket{L}$) and right ($\ket{R}$) arms of a Mach-Zehnder interferometer under suitable pre-selection and post-selection. The two basis states of circular polarisation of the photon is given by $\ket{+}$ and $\ket{-}$ which can be expressed as
%\begin{equation}
%\label{eq:6}
%        \ket{+}=\frac{1}{\sqrt{2}}(\ket{H}+i\ket{V}),\\
%        \ket{-}=\frac{1}{\sqrt{2}}(\ket{H}-i\ket{V})
%\end{equation}
%where $\ket{H}$ and $\ket{V}$ are the horizontal and vertical linear polarization states respectively. The pre-selected state is taken as, a photon is initially in a horizontal polarization state $\ket{H}$ and it is prepared in a state, which is in a equal superposition of two locations, i.e.,
%\begin{equation}
%    \ket{\Psi}=\frac{1}{\sqrt{2}}(i\ket{L}+\ket{R})\ket{H},
%    \label{eq:3}
%\end{equation}
%This state can be prepared by sending a horizontally polarized photon through a 50:50 beam-splitter \cite{Aharonov}. The post-selected state is chosen as,
%\begin{equation}
%\ket{\Phi}=\frac{1}{\sqrt{2}}(\ket{L}\ket{H}+\ket{R}\ket{V}).
%\label{eq:4}
%\end{equation} 
%The idea is such that, the measurement will be performed when the final state is in this post-selected state $\ket{\Phi}$. Otherwise if the state is orthogonal to $\ket{\Phi}$, the measurement will be discarded. The optical setup of this thought experiment is given in~\cite{Aharonov}. For tracing the location of the photon, the detectors, which are inserted in the left and right arms of the interferometer, measure the projectors $\Pi_L=\ket{L}\bra{L}$ and $\Pi_R=\ket{R}\bra{R}$ respectively and similarly the polarisation detectors measure the observables $\sigma_z^{(L)}=\Pi_L \otimes \sigma_z$ and $\sigma_z^{(R)}=\Pi_R \otimes \sigma_z$, where $\sigma_z=i(\ket{V}\bra{H}-\ket{H}\bra{V})$.\par %The operator $\sigma_z$ is given by
%\begin{equation}
%\sigma_z={\ket{+}\bra{+}-\ket{-}\bra{-}}
%\end{equation}
%is to perform measurements weakly and under suitable pre-selection and post-selection in the left ($\ket{L}$) and right ($\ket{R}$) arms of a Mach-Zehnder interferometer. The weak values of the operators $\Pi_L=\ket{L}\bra{L}$, $\Pi_R=\ket{R}\bra{R}$, $\sigma_z^L=\Pi_L\otimes \sigma_z$ and $\sigma_z^R=\Pi_L\otimes \sigma_z$ are sought for. 
     
    %These operators correspond to the measurements of the position of a photon and the position of the z component of polarization in the two arms. 
%and 
%where $\ket{V}$ denotes vertical polarization, then 
%The aforementioned weak values are measured to be \begin{eqnarray}
%     (\Pi_L)_w=1, && \quad (\Pi_R)_w=0 \nonumber\\
%     (\sigma_z^{(L)})_w=0, && \quad (\sigma_z^{(R)})_w=1
%\end{eqnarray}
%This indicates that the photon passed through the left arm but its $z$-component of polarisation passed through the right arm. Thus, the $z$-component of polarization is shown to exist without the presence of the photon in the right arm, which, analogically could be thought as a \emph{'grin without a cat'} from Alice in Wonderland~\cite{Alice}.
%\subsection{Amplification using weak values} 
%\label{subsection: amp}
%Weak value amplification has various applications in emergent quantum technologies like enhancement of small signals, phase estimation, precision measurement etc. In the year 1989 Aharonov, Albert and Vaidman (AAV) showed that the weak measurement procedure may have the values outside the eigenvalue spectrum~\cite{AAV}. In this sense one can amplify small signals using the weak values of an observable. They have applied their arguments on a component of spin of a spin- $\frac{1}{2}$ particle and examined that the weak value of the spin observable can turn out to be $100$ by using a Stern-Gerlach setup.\par
%The amplification of weak value depends on the fact that if the pre and post-selected states are almost orthogonal, so that the denominator of Eq. (\ref{eq:weak}) becomes very small, resulting the weak value of the observable to be arbitrarily high. But one should note very carefully that the amplification of weak values is not ever increasing and exploding, there is a limit on the probability of successful post-selection if the pre and post-selected states become more and more orthogonal. In a realistic set-up non-linear effects come into play and set a bound on the possibility of obtaining successful post-selection which is almost orthogonal to the pre-selected state. For further discussions on the limit of amplification of weak values see~\cite{Limits}.


%\textcolor{cyan}{
\section{Review of weak measurement and quantum Cheshire cat}
\label{section:2}
The weak-value scheme~\cite{AAV,Duck} started with a thought experiment for measuring a spin component of a quantum spin-$\frac{1}{2}$ particle, obtaining a result which was far beyond the range of usual values. In this work, the interaction Hamiltonian is usually taken as
%usually described by the Hamiltonian,
\begin{equation}
\label{eqn:ho}
    H_0=-g(t)\hat{A}\otimes \hat{q},
\end{equation}
where $\hat{q}$ is a canonical variable of the meter that is conjugate to momentum $\hat{p}$, $g(t)$ is a time-dependent coupling function with a compact support near the time of  measurement (normalized such that its time integral is unity), and $\hat{A}$ is the 
%\textcolor{red}{\sout{initial}}
observable to be measured. In~\cite{AAV,Duck}, $\hat{q}$ has a continuous spectrum. However, we will consider in this paper, instances where $\hat{q}$ can also have a discrete spectrum.
% After the weak measurement of the observable $\hat{A}$, a strong measurement of a suitable observable $B$ is carried out on the principal system and a quantity called weak value, $A_w$, is defined by post-selecting an outcome $\ket{\psi_f}$.
The weak value of $\hat{A}$ is defined as
%$A$~\cite{AAV,Duck,YA} 
\begin{equation}
    A_w = \frac{\Braket{\Psi_f|\hat{A}|\Psi_{in}}}{{\Braket{\Psi_f|\Psi_{in}}}}. 
\label{eq:weak}
\end{equation}

The concept of the quantum Cheshire cat~\cite{Aharonov} is based on weak measurement of observables. Typically in a quantum Cheshire cat setup, as seen in Fig.~\ref{fig1}(a), a photon having horizontal polarization $\ket{H}$ is fed into a $50:50$ beam-splitter $BS_1$ of a  Mach-Zehnder interferometer, creating the pre-selected state
\begin{equation}
    \ket{\Psi_{in}}=\frac{1}{\sqrt{2}}(i\ket{L}+\ket{R})\ket{H},
    \label{eq:psi_in}
\end{equation}
where $\ket{L}$ and $\ket{R}$ represent the left and right path degrees of freedom, respectively. We will consider the states of circular polarization of the photon, denoted by $\ket{+}$ and $\ket{-}$, and given by
\begin{equation}
\label{eq:6}
\ket{+}=\frac{1}{\sqrt{2}}(\ket{H}+i\ket{V}),\quad\ket{-}=\frac{1}{\sqrt{2}}(\ket{H}-i\ket{V}),
\end{equation}
as the computational basis. In particular, therefore, $\hat{\sigma}_z=\ket{+}\bra{+}-\ket{-}\bra{-}$ and $\hat{\sigma}_x=\ket{+}\bra{-}+\ket{-}\bra{+}$. Here $\ket{V}$ denotes the state of vertical polarization. This convention is in accordance with that adopted in~\cite{Aharonov}.
In the two arms of the interferometer, weak measurements of the location of the photon and that 
%the locations 
of the photon's circular polarization are carried out. The photon and its polarization interact weakly with appropriate meter states, resulting in deflections in the latter. This interaction is of the form defined in Eq.~(\ref{eqn:ho}).  Next, an arrangement of a half waveplate $HWP$, phase shifter $PS$, beam-splitter $BS_2$, polarization beam-splitter $PBS$ and detectors $D_1$, $D_2$ and $D_3$, elaborated in Fig.~\ref{fig1}(a), is used to post-select the state 
\begin{equation}
    \ket{\Psi_f}=\frac{1}{\sqrt{2}}(\ket{L}\ket{H}+\ket{R}\ket{V}).
    \label{eq:psi_f}
\end{equation}
%\textcolor{blue}{\sout{where $\ket{V}$ refers to vertical photon polarization.}} 
For this particular post-selected state, it can be seen that the distributions of the deflected meter states center around the weak values of the observables being measured in the arms. To trace the location of the photon, the meters, which are inserted in the left and right arms of the interferometer, measure the projectors $\hat{\Pi}_L=\ket{L}\bra{L}$ and $\hat{\Pi}_R=\ket{R}\bra{R}$, respectively, and similarly the polarization detectors measure the observables $\hat{\sigma}_z^{L}=\hat{\Pi}_L \otimes \hat{\sigma}_z$ and $\hat{\sigma}_z^{R}=\hat{\Pi}_R \otimes \hat{\sigma}_z$.
%, where $\hat{\sigma}_z=i(\ket{V}\bra{H}-\ket{H}\bra{V})$.
The corresponding weak values are
\begin{eqnarray}
     (\hat{\Pi}_L)_w=1, && \quad (\hat{\Pi}_R)_w=0, \nonumber\\
     (\hat{\sigma}_z^{L})_w=0, && \quad (\hat{\sigma}_z^{R})_w=1.
\end{eqnarray}
This indicates that the photon passed through the left arm but its $z$-component of polarization passed through the right arm.

%\vspace{1cm}


\section{Amplification of polarization of a photon without the photon}
\label{section:3}
\begin{figure*}
\centering
%\includegraphics[height=8cm,width=8cm]{new diagram 6- 12.pdf}%
\hspace{.25cm}%
%\includegraphics[height=8cm,width=8cm]{diagram.pdf}
%\caption{}
%\end{figure*}
%\begin{figure}
\includegraphics[height=8cm,width=8cm]{d1.pdf}
\includegraphics[height=8cm,width=8cm]{d2.pdf}
\caption{Quantum Cheshire cat without and with amplification. (a) Quantum Cheshire cat setup (without amplification). The areas shaded pink and blue carry out the pre-selection and post-selection, respectively. For the latter, only the clicks of detector $D_1$ are selected. Weak measurements of the photon position and the position of polarization are performed by interacting suitable meters weakly, in the two arms of the interferometer. (b) Setup for decoupling the $z$-component of polarization of a photon from the photon itself and amplifying it simultaneously. This configuration is also applicable for dissociating the $z$-component of polarization and the $x$-component of polarization of the photon, and amplifying the former simultaneously.}
\label{fig1}
\end{figure*}
%\begin{figure}

%\caption{Setup for decoupling the \emph{grin} ($z$-component of polarisation) of a cat (photon) from the cat (photon) itself and amplifing it simultaneously. $BS_1$ is a 50:50 beam-splitter, $P_1$ and $P_2$ are two black boxes consisting of polarising beam-splitters and phase shifters. $PBS_1$ is a polarising beam-splitter. The measuring devise contains a half wave plate ($HWP$), a phase shifter ($PS$), a beam-splitter ($BS_2$) and a polarising beam-splitter ($PBS_2$). $D_1$,$D_2$, $D_3$ and $D'$ are the detectors. For the functions of each elements of this configuration see the text and for the function of measuring devise see \cite{Aharonov}. The weak measurement is done between the second pre-selection and the post-selection.}
%\label{fig1}
%\end{figure}
Weak values can be used as a tool for amplifying small signals. Further, we have seen how a property can be separated from a quantum system using the technique of the quantum Cheshire cat. Now we aim to achieve the two phenomena simultaneously, namely, the separation of a property from the object and amplification of the separated property independently of the object. 
%This could be analogically described as a four line rhyme, which will be suitable with~\cite{Alice} and with our gedanken.
 %  \begin{quote}\emph{There was a cat who had a grin,\\ 
%   a physicist grabbed the cat, took her grin,\\ 
%   and made it way way bigger\\
 %  than it could ever have been.} 
 %  \end{quote}

%We take a second pre-selected state 
%\begin{equation}
     %\ket{\Psi_2}=i sin\frac{\theta}{2}\ket{R}\ket{H}+ \\cos\frac{\theta}{2}\frac{(\ket{L}+\ket{R})}{\sqrt{2}}\ket{V} \label{eq: preselect}
%\end{equation}

%This is carried out by placing two black boxes $P_1$ and $P_2$, which are some combinations of polarising beam-splitters and phase shifters, in left and right arm of the interferometer. $P_1$ causes a transformation $\ket{H} \rightarrow \cos\frac{\theta}{2}\ket{V}+ i sin\frac{\theta}{2}\ket{H}$ and $P_2$ causes $\ket{H} \rightarrow -i(\cos\frac{\theta}{2}\ket{V}+sin\frac{\theta}{2}\ket{H})$. Apart from this, a polarising beam-splitter (PBS) is being put in the left arm which has been chosen such that $\ket{H}$ is reflected and $\ket{V}$ is transmitted. $D_1$ will click if and only if the final state i.e., the state immediately before the $HWP$ is the post-selected state $\ket{\phi}$. \par  
%Apart from the three photon detectors $D$D_1$,$D_2$, $D_3$ and _1$,$D_2$, $D_3$, there is also a detector $D'$ which will detect the $\ket{H}$ (outcasted) part. \\

The interferometric setup of this gedanken experiment is presented in Fig.~\ref{fig1}(b). Let us begin by considering a photon, propagating along a path degree of freedom state denoted by $\ket{L^\prime}$, in a polarization state $\cos\frac{\theta}{2}\ket{H}+ \sin\frac{\theta}{2}\ket{V}$.
%, where $\ket{H}$ implies a horizontally polarized state and $\ket{V}$ implies a vertically polarized state. 
The photon is sent through a polarization beam-splitter $PBS_1$ that transmits horizontally polarized light and reflects the vertically polarized one, leading to the state
\begin{equation}
    \ket{\Psi_1}=\cos\frac{\theta}{2}\ket{L}\ket{H}+ \sin\frac{\theta}{2}\ket{R}\ket{V},
\end{equation}
where $\ket{L}$ and $\ket{R}$ are the two possible photon paths, viz., along the transmitted and reflected paths, respectively, forming the two arms of a Mach-Zehnder interferometer. Note that $\ket{L}$ and $\ket{L^\prime}$ are along the same path after and before the polarization beam-splitter $PBS_1$. In the right arm of the interferometer, we place a half waveplate $HWP_1$ that converts a vertical polarization into a horizontal one, and vice versa, followed by a $\pi$ phase shifter $P$, which introduces a phase $e^{i\pi}$ in the right path. We now have the state
\begin{equation}
    \ket{\Psi^\prime_1}=(\cos\frac{\theta}{2}\ket{L}-i\sin\frac{\theta}{2}\ket{R})\ket{H} . 
    \label{eq: preselect}
\end{equation} 
%ei obdi holo
In the parlance of weak values and the quantum Cheshire cat, this is the pre-selected state.  The post-selection involves a half waveplate $HWP$, a phase shifter $PS$, a beam-splitter $BS$, a second polarization beam-splitter $PBS$, and three detectors $D_1, D_2$ and $D_3$. The working principle is the same as discussed for the quantum Cheshire cat scenario without the amplification requirement. Thus the clicking of the detector $D_1$ can once again be solely selected to obtain the post-selected state
\begin{equation}
    \ket{\Psi_f}=\frac{1}{\sqrt{2}}(\ket{L}\ket{H}+\ket{R}\ket{V}).
    \label{eq:3}
\end{equation}
In the two arms of the interferometric setup, weak measurements of the position of the photon and its circular polarization are performed and the corresponding weak values are obtained. The weak values of the operators $\hat{\Pi}_L$ and $\hat{\Pi}_R$, denoting the positions of the photon in the left and right arms, and $\hat{\sigma}_z^{L}$ and $\hat{\sigma}_z^{R}$, denoting the positions of $z$-components of polarization in the two arms, are then measured to be  
\begin{eqnarray}
(\hat{\Pi}_L)_w&=&\frac{\Braket{\Psi_f|\hat{\Pi}_L|\Psi^\prime_1}}{\Braket{\Psi_f|\Psi^\prime_1}}=1,\nonumber\\
(\hat{\Pi}_R)_w&=&\frac{\Braket{\Psi_f|\hat{\Pi}_R|\Psi^\prime_1}}{\Braket{\Psi_f|\Psi^\prime_1}}=0,\nonumber\\
(\hat{\sigma}_z^{L})_w&=&\frac{\Braket{\Psi_f|\hat{\sigma}_z^{L}|\Psi^\prime_1}}{{\Braket{\Psi_f|\Psi^\prime_1}}}=0,\nonumber\\
(\hat{\sigma}_z^{R})_w&=&\frac{\Braket{\Psi_f|\hat{\sigma}_z^{R}|\Psi^\prime_1}}{\Braket{\Psi_f|\Psi^\prime_1}}=\tan\frac{\theta}{2}.
\label{eq:7}
\end{eqnarray}
Therefore, the photon is detected in the left arm and the $z$-component of polarization is detected in the right arm with a factor which could be amplified by varying the parameter $\theta$. Thus we have achieved the phenomenon of amplifying a property of an object independently of the object: The photon's  polarization component is being amplified in the right arm of the interferometer and the photon is not there.
%In the next section, we describe a process to separate and amplify two orthogonal components of spin and amplify any one of them. \\

 This thought experiment can be further extended by separating two orthogonal components of polarization and then amplifying one component independently of the other. Let us consider the two orthogonal components to be the $x$ and $z$-components of polarization, 
 viz. \(\hat{\sigma}_x\) and \(\hat{\sigma}_z\).
% \textcolor{red}{$\sigma_x=\ket{+}\bra{-}+\ket{-}\bra{+}$ and $\sigma_z=\ket{+}\bra{+}-\ket{-}\bra{-}$.} \textcolor{blue}{\sout{with $\hat{\sigma}_x=\ket{H}\bra{H}-\ket{V}\bra{V}$ and $\hat{\sigma}_z=i(\ket{V}\bra{H}-\ket{H}\bra{V})$.}} 
The operators $\hat{\sigma}_x^{L}=\hat{\Pi}_L\otimes \hat{\sigma}_x$ and $\hat{\sigma}_x^{R}=\hat{\Pi}_R\otimes \hat{\sigma}_x$ are measured to detect the $x$-component of polarization in the left and right arms of the interferometer, respectively. The corresponding weak values turn out to be
\begin{eqnarray}
(\hat{\sigma}_x^{L})_w&=&\frac{\Braket{\Psi_f|\hat{\sigma}_x^{L}|\Psi_1^{\prime}}}{\Braket{\Psi_f|\Psi_1^{\prime}}}=1, \nonumber\\
(\hat{\sigma}_x^{R})_w&=&\frac{\Braket{\Psi_f|\hat{\sigma}_x^{R}|\Psi_1^{\prime}}}{\Braket{\Psi_f|\Psi_1^{\prime}}}=0.
        %(\sigma_z^{(L)})_w&=&\frac{\Braket{\phi^{\prime}|\sigma_z^{(L)}|\psi_2^{\prime}}}{{\Braket{\phi^{\prime}|\psi_2^{\prime}}}}=0 \nonumber\\
        %(\sigma_z^{(R)})_w&=&\frac{\Braket{\phi^{\prime}|\sigma_z^{(R)}|\psi_2^{\prime}}}{{\Braket{\phi^{\prime}|\psi_2^{\prime}}}}=tan\frac{\theta}{2}
    \end{eqnarray}
    \par
    When coupled with the weak values of $\hat{\sigma}_z^{L}$ and $\hat{\sigma}_z^{R}$, these indicate that the $z$-component of polarization can be amplified independently of the $x$-component of polarization. 
    %It is seen that 
    The weak value of $\hat{\sigma}_z^{R}$ is seen to acquire a value of $\tan\frac{\theta}{2}$, which means that the weak value will result in outcomes beyond the eigenvalue spectrum in the regions $\theta \in (\pi/2,\pi)$
%     \textcolor{red}{\sout{for positive amplification,}} 
and $\theta \in (-\pi,-\pi/2)$. %\textcolor{red}{\sout{ for negative amplification.}}
    
    To realize of the weak-value measurement, 
    %we need to better understand the measurement process. As discussed earlier, 
    the principal system is made to weakly interact with a meter. Let us assume that the meter is initially in a state $\ket{\Phi_{in}}$. Suppose we intend to measure $\hat{\sigma}_z^R$. The weak interaction between the system and the meter 
    %results in
    can be effected by
     a joint unitary $\hat{U}_{\hat{\sigma}_z^R}=e^{-\int iH_{\hat{\sigma}_z^R}t dt}$ where
     \begin{equation}
         H_{\hat{\sigma}_z^R}=\frac{1}{\sqrt{2}}g(t)[(\hat{I}-\hat{\Pi}_R)\otimes\hat{\sigma}_z\otimes \hat{I} + \hat{\Pi}_R\otimes \hat{\sigma}_z\otimes \hat{q}].
\end{equation}
%\textcolor{blue}{\sout{Let us choose the meter to be a qubit system with the computational basis $\{\ket{0}_m,\ket{1}_m\}$. Suppose we intend to measure $\hat{\sigma}_z^R$. The weak interaction between the system and the meter 
    %results in
    %can be effected by
     %a joint unitary $\hat{U}_{\sigma_z^R}$ that couples the $z$-component of the photon polarization to the meter. A convenient unitary operator that can serve this purpose, also used in \cite{DasPati, DasSen}, is given by}}
%\textcolor{blue}{\begin{equation}
    %\stkout{\hat{U}_{\sigma_z^R}=\frac{1}{\sqrt{2}}[(\hat{I}-\hat{\Pi}_R)\otimes\hat{\sigma}_z\otimes \hat{I} + \hat{\Pi}_R\otimes \hat{\sigma}_z\otimes R^{-1}(\theta_g)ZR(\theta_g)]},
    %\label{eq:unitary_noiseless}
%\end{equation}
%\sout{where $Z=\ket{0}_m\bra{0}_m - \ket{1}_m\bra{1}_m$ and}
%\begin{eqnarray}
%\label{eqn:R}
    %\stkout{R(\theta_g)\ket{0}_m &=& \cos(2\theta_g)\ket{0}_m +  \sin(2\theta_g)\ket{1}_m, \nonumber\\}
    %\stkout{R(\theta_g)\ket{1}_m &=& \sin(2\theta_g)\ket{0}_m - \cos(2\theta_g)\ket{1}_m .} 
  %  \end{eqnarray}}
%    \textcolor{blue}{\sout{Set $g=4\theta_g$.}} 
For the post-selection given by Eq.~(\ref{eq:3}), the meter goes to the state 
    \begin{equation}
    \ket{\Phi_m}=\braket{\Psi_f|\Psi_{1}^{\prime}}[1-ig(\hat{\sigma}_z^R)_w \hat{q}\ket{\Phi_{in}}],
\end{equation}
 %   \textcolor{blue}{\begin{eqnarray}
        %\stkout{\ket{\Phi}_m = \ket{0}+g(\sigma_Z^R)_w \ket{1},}
%    \end{eqnarray}}
    with $(\hat{\sigma}_Z^R)_w=1$.
Hence, the meter shows a deflection that is directly proportional to the weak value of the measured observable. 
%\textcolor{red}{\sout{In case of using a discrete pointer, the construction of these joint weak unitary evolutions can be found in~\cite{Pryde,m}.
%See~\cite{npj,m, opticsguo} for the experimental realization of using discrete pointers.}} 
%\textcolor{blue}{\sout{Further details on the construction of these joint weak unitary evolutions can be found in~\cite{Pryde,m}.}}
 
    
\section{Amplification of the polarization of a photon in a noisy scenario}
\label{section:4}
In the preceding section, we laid out the procedure for amplifying a property of a quantum system in the absence of the system. We now consider a scenario in which the property we are looking to amplify is affected by noise, in a certain way.
We first present the situation where the weak-value amplification of the noise-affected observable is carried out. As expected, the amplified quantity will contain a contribution from the unwanted noise. We then get rid of the noise by separating it from the signal using a setup similar to that of the quantum Cheshire cat and amplify the signal. 

%Now we consider a more realistic (yet fictional) scenario. Let us consider that the cat is confused, i.e. she has many (or all) of its properties inside herself but in a blended manner. That potrays to us the noisy scenario. Now our desire to extract the grin from the \emph{confused} cheshire cat. In our everyday life clarity results in certainty and the state of confusion or puzzlement results in a foggy aka noisy situation. This is the motivation behind the analogy. As done earlier, we could sum up the situation in a small rhyme
%\begin{quote}
 %   \emph{The physicist met the cat, alone at the isle \\
 %   He asked the cat, "Want to smile?"\\
 %   "I'm confused, Meow!" came the reply\\
 %   The physicist felt pity and thought about,\\
 %   he came and tickled the cat\\
 %   and in that way her big grin came out!}\\
 %   \end{quote}
   
%The factor of noise plays a very significant part in practical scenarios involving weak values. %In this section, we present noise in a completely different perspective than previous.%
%The \emph{tickling} could be thought as the setup and gedanken proposed for the project. 
We can conveniently take the observable to be weakly measured, as the $z$-component of polarization $\hat{\sigma}_z$.
%, as defined earlier. 
We  recall that the measurement of the $z$-component of polarization ideally requires us to set up an interaction of the form
\begin{equation}
H_0=-g(t)\hat{\sigma}_z\otimes \hat{q},
\label{eqn:hoz}
\end{equation}
between the polarization degree of freedom and a convenient meter of our choosing, with meter variable $\hat{q}$. The variable $\hat{q}$ may be a discrete or a continuous meter variable. 
%depending on the system and the type of noise, if present in the 'ideal' setup. 
See~\cite{Pryde, npj,m, opticsguo} for experimental realizations of using discrete meter states. 
Let us now consider a  scenario in which there is noise in the interaction Hamiltonian that is analogous to a spin-orbit-coupling term $\hat{L}_x \otimes {\hat\sigma}_x$. The total Hamiltonian is now 
%\textcolor{red}{[eTa Thik achhe? Especially, pls check if the order of the systems in the tensor product is consistent with material before and after.]}\textcolor{blue}{[ha thik achhe.]}
%looks like
\begin{equation}
    H=-g\delta(t)\hat{I} \otimes \hat{\sigma}_z \otimes  \hat{q}+g^{\prime}  \hat{L}_x \otimes \hat{\sigma}_x \otimes \hat{I}.
\label{eq:Hnoisy}    
\end{equation}
%\begin{equation}
%\textcolor{red}{\stkout{    \hat{H}=-g\delta(t)\hat{q}\otimes\hat{\sigma}_z+g^{\prime} \hat{I}\otimes \hat{L}_x \otimes \hat{\sigma}_x.}}\nonumber
%\label{eq:Hnoisy}    
%\end{equation}
Here $g$ and $g^{\prime}$ are the coupling constants. The coupling between the system and the meter is an instantaneous coupling, while on the contrary, in the second term, there is no dependence of time on spin-orbit coupling. Our aim is to obtain the weak value of $\hat{\sigma}_z$. Due to the noise, the effective observable of the total system, resulting in the deflection in the meter, is different from that in the noiseless case obtained from Eq.~(\ref{eqn:hoz}) in~\cite{AAV,Duck}.
%\subsection{Calculation using a continuous meter}

Let us consider the initial state of the system as $\ket{\Psi_{in}}$ and the post-selected state as $\ket{\Psi_f}$. As mentioned before, the meter state could be chosen as a discrete or a continuous spectrum, depending on the type of the experimental setup. In our work, we consider both scenarios: One is with a meter variable $\hat{q}$ considered as discrete and the other is with a continuous state distribution of the same. As an example of a discrete meter state, we take a discrete Gaussian function with the standard deviation $\sqrt{2}\Delta$ as
\begin{equation}
\label{eq:dis}
\ket{\Phi_{in}}_{dis}= \sum_{k=-N}^{N}\exp \Big ( -\frac{q_k^2}{4\Delta^2}\Big)\ket{q_k},
\end{equation}
%where $q_1=-3\sqrt{2}\Delta$ and $q_m=3\sqrt{2}\Delta$. 
where we are assuming a discrete meter variable $q_k=k$ with $k$ having the values $0, \pm 1, \pm 2\ldots \pm N$. The corresponding $p$ representation of this meter state turns out to be
\begin{equation}
    \ket{\Phi_{in}}_{dis}=\sum_{l=-N}^{N} \exp \left ( -\Delta^2 p_l^2\right)\xi(p_l)\ket{p_l},
\end{equation}
where $\xi(p_l)=\sum_{k=-N}^{N} e^{-\frac{1}{4\Delta^2}\big(q_k+2i\Delta^2 p_l\big)^2}$ and $p_l=\frac{l}{(2N+1)}$ with $l$ having the values $0, \pm 1, \pm 2\ldots \pm N$.
%, and $\ket{p_l}=\ket{l}$.
%it is assumed that is independent of $p_i$.
In the limit of $N \rightarrow \infty$, $\xi(p_l)$ will be independent of $p_l$. %\textcolor{magenta}{This assumption comes from the consideration that discrete state space of the meter state is large and the discrete states of the meters are closely spaced, so that we can approximate to them as a continuous meter state.}
The parallel example in the continuous case may be taken as 
%\textcolor{red}{\sout{Let the initial state of the meter be}}
\begin{equation}
\ket{\Phi_{in}}_{con}= \int_{-\infty}^{+\infty} dq\exp \Big ( -\frac{q^2}{4\Delta^2}\Big)\ket{q},
\label{eq:meter} 
\end{equation}
with the $p$-representation,
\begin{equation}
\ket{\Phi_{in}}_{con}= \int_{-\infty}^{+\infty} dp\exp \Big ( -\Delta^2 p^2\Big)\ket{p}.
\label{eq:meterp} 
\end{equation}
To obtain the $p$ representations from the $q$ representation in both the discrete and  continuous cases, we neglect multiplicative constants.
%The state of the system here can be written as
%\begin{equation}
%    \ket{\psi_{in}}=\sum_{j} \alpha_j \ket{\alpha_j}_L \ket{\alpha_j}_S,
%\end{equation}
%We can write the meter state as
%\begin{eqnarray}
%    \ket{\phi_{in}}= \int \,dq \phi(q)\ket{q}=\int \,dp \tilde{\phi(p)}\ket{p}
%\end{eqnarray}
After post-selection, the final state of the meter is given by
\begin{equation}
  \ket{\Phi_f}=\bra{\Psi_f}\hat{U}(t)\ket{\Psi_{in}}\ket{\Phi_{in}},
  %=\bra{\Psi_f}e^{-\frac{i}{\hbar}\int_0^t H\,dt} \ket{\Psi_{in}}\ket{\Phi_{in}},
\end{equation}
where $t$ is the time of measurement. This is true for both the discrete and continuous meter states and hence we have omitted the subscripts. As the Hamiltonians at different times do not commute, the expansion of $\hat{U}(t)$ requires using the Dyson series expansion, leading to %\textcolor{red}{U(t)-er par pratham equality-Ta ki Thik?}\textcolor{blue}{[na na eta thik noy. Dyson series tai hbe. ager equality ta thakbe na.]}
\begin{eqnarray}
\label{eq:uni}
    \hat{U}(t)
    %=\textcolor{blue}{\stkout{e^{-\frac{i}{\hbar}\int_0^t H\,dt}}}
    &=& 1+ \sum_{n=1}^{\infty} \frac{1}{n!}\Big (\frac{-i}{\hbar}\Big)^n \int_{0}^{t}dt_1 \int_{0}^{t_1}dt_2 \cdots \int_{0}^{t_{n-1}}dt_{n-1}\nonumber\\ &&\phantom{ dure cholo jai}
    %\textcolor{blue}{
\times    \mathcal{J}[H(t_1)H(t_2) \cdots H(t_n)].
\end{eqnarray}
%\textcolor{red}{[oporer rhs-e, \(t\)-gulo-r opor ekTa ordering thakbe na? ebang sei orderingTa Eq. (\ref{ramakantakamar})-te ar thakbe na. tai na?]} \textcolor{blue}{[ha, R.H.S e ekta time ordering achhe. tahle $\mathcal{J}[H(t_1)H(t_2) \cdots H(t_n)]$ eivabe lekha valo tahle.]}
We re-define the coupling constants \(g\) and \(g'\) so that we can effectively set 
%Setting 
$\hbar=1$. Using time ordering %\textcolor{red}{[ekhane time ordering kothai kaj-e lagchhe?]} \textcolor{blue}{[Eq. (19) e second term ta jokhn $t_1>t_2$ ar 3rd term ta jokhn $t_2>t_1$.]} 
in the expansion,
%for the second term 
we have
\begin{eqnarray}
 1&+& (-i) \int_{0}^{t}dt_1 H(t_1)\nonumber\\
&+&\frac{1}{2!}(-i)^2 \int_{0}^{t}dt_1 \int_{0}^{t_1}dt_2 H(t_1)H(t_2)\nonumber\\
 &+&\frac{1}{2!}(-i)^2 \int_{0}^{t}dt_2 \int_{0}^{t_2}dt_1 H(t_2)H(t_1)+ \cdots.
\end{eqnarray}
Interchanging $t_1$ and $t_2$ in the last (displayed) term, we get
\begin{eqnarray}
 1&+& (-i) \int_{0}^{t}dt_1 H(t_1)\nonumber\\
&-& \int_{0}^{t}dt_1 \int_{0}^{t_1}dt_2 H(t_1)H(t_2)+ \cdots.
\label{ramakantakamar}
\end{eqnarray}
Substituting the Hamiltonian $H$ from Eq.~(\ref{eq:Hnoisy}) in this expression, the form of the unitary turns out to be 
%\textcolor{red}{[ordering of systems in tensor product has to be consistent with Eq.~(\ref{eq:Hnoisy}).]}
{\begin{eqnarray}
\label{eq:approx}
&&1-i({g\hat{I} \otimes \hat{\sigma}_z \otimes \hat{q}+g^\prime t \hat{L}_x\otimes \hat{\sigma}_x} \otimes \hat{I}) -[g^2(\hat{I} \otimes \hat{\sigma}_z \otimes \hat{q})^2 \nonumber\\ 
&+& g g^\prime t (\hat{L}_x \otimes \hat{\sigma}_x\hat{\sigma}_z \otimes \hat{q})+g^{\prime^2} \frac{t^2}{2}(\hat{L}_x \otimes \hat{\sigma}_x \otimes \hat{I})^2]+ \cdots. \quad \quad
\end{eqnarray}}
%Now let us choose the final postselected state to filter the results 
%\begin{equation}
 %   \ket{\psi_f}=\sum_{j} \beta_j \ket{\beta_j}_{L}\ket{\beta_j}_{\sigma}
 %   \end{equation}
The measurement time $t$ is chosen from the regime {$\frac{g}{g^{\prime}} \ll t \ll \frac{\sqrt{g}}{g^{\prime}}$}. Thus we can
now assume $g$ and $g^{\prime}$ to be sufficiently small to neglect the $g^2$ term in~(\ref{eq:approx}) and the higher-order terms that are not present in~(\ref{eq:approx}).
%all the terms that are second order in $g$ or $g^\prime$ and above. %\textcolor{red}{[ekhane ekTa gynaRakol achhe kina bujhchhi na! \(g<1\) hale kintu \(\sqrt{g}\), \(g\)-er theke boRo haye jabe!]}\textcolor{blue}{[eta ulto hye gechhe. $\frac{g}{g^{\prime}} \ll t \ll \frac{\sqrt{g}}{g^{\prime}}$ eta hbe.]} 
On the other hand, we retain the terms containing $gg^{\prime}t$ and $g^{\prime^2}t^2$. Using Eq. %\textcolor{red}{\sout{(\ref{eq:meter})}} 
(\ref{eq:dis}), the final state of the meter after weak measurement and post-selection reads  
\begin{eqnarray}
\label{eqn:zzz}
  \ket{\Phi_f}_{dis} &\approx&  \braket{\Psi_f|\Psi_{in}}\sum_{k=-N}^{N} e^{-\frac{q_k^2}{4\Delta^2}}\Big[1+i g q_k(\hat{\sigma}_z)_w\nonumber\\ 
  &-& i g^{\prime} t (\hat{L}_x\otimes \hat{\sigma}_x)_w-gg^\prime tq_k (\hat{L}_x\otimes \hat{\sigma}_x \hat{\sigma}_z)_w \nonumber\\
  &-& g^{\prime^2}\frac{t^2}{2}(\hat{L}_x\otimes \hat{\sigma}_x)_w^2\Big]\ket{q_k}.
  \end{eqnarray}
%Now with the help of Taylor expansion
 %   \begin{eqnarray}
  %      \braket{\psi_f|\hat{U}|\Psi_{in}}=\bra{\psi_f}\sum_{j} \alpha \beta \int \,dq \times e^{-\frac{q^2}{4\Delta^2}}( 1-g\hat{q}\otimes \hat{\sigma_z}\nonumber\\ 
   %    -g^\prime t \hat{L}\hat{\sigma_x}-\frac{1}{\hbar}g^2 \hat{q}^2 \hat{\sigma_z}^2+ gg^{\prime}t\hat{q}\otimes L \otimes \hat{S_x}\hat{S_z} + \cdots) \nonumber \\ &&\ket{\phi_{in}}
%    \end{eqnarray}
 %   neglecting terms containing $g^{2} g^{\prime}$ and $g\prime^2$ we get,
   % \begin{eqnarray}
   %  \braket{\psi_f|\psi_{in}} \int \,dq e^{-\frac{q^2}{4\Delta^2}}  (1-ig\hat{q}\frac{\braket{\psi_f|\hat{\sigma_z}|\psi_{in}}}\braket{{\psi_f|\psi_{in}}}\nonumber\\
 %- g^2\hat{q^2}\frac{\braket{\psi_f|\hat{\sigma_{z}^2}|\psi_{in}}}\braket{{\psi_f|\psi_{in}}} +ig^3\hat{q}^3 \nonumber\\ \frac{\braket{\psi_f|\hat{\sigma_{z}^3}|\psi_{in}}}\braket{{\psi_f|\psi_{in}}} -ig^\prime t \frac{\braket{\psi_f|\hat{L_x\otimes\sigma_x}|\psi_{in}}}{\psi_f|\psi_{in}} +gg^\prime t \hat{q} \frac{\braket{\psi_f|\hat{L_x\otimes(\sigma_x\sigma_z)}|\psi_{in}}}\braket{{\psi_f|\psi_{in}}} + \cdots 
  %  \end{eqnarray}
The weak values above are obtained using the definition in Eq. (\ref{eq:weak}). Now, in general, for an observable $\hat{A}$  and post-selection of $\ket{\Psi_f}$, with $A_w$ being the weak value of $\hat{A}$, the shifted meter state is given by
\begin{equation}
\label{eq:shift1}
 \ket{\Phi_f}_{dis} \approx \braket{\Psi_f|\Psi_{in}}\sum_{k=-N}^N e^{igq_kA_w}\exp\Big(-\frac{q_k^2}{4\Delta^2}\Big)\ket{q_k}.
\end{equation}
Comparing Eqs. (\ref{eqn:zzz}) and (\ref{eq:shift1}), we get
\begin{eqnarray}
e^{igq_kA_w}&=&1+igq_k(\hat{\sigma}_z)_w-ig^{\prime}t(\hat{L}_x\otimes\hat{\sigma}_z)_w \nonumber\\ 
&&-gg^\prime tq_k(\hat{L}_x\otimes\hat{\sigma}_x\hat{\sigma}_z)_w-g^{\prime^2}\frac{t^2}{2}(\hat{L}_x\otimes\hat{\sigma}_z)_w^2. \quad
\label{shift2}
\end{eqnarray}
 Let
\begin{eqnarray}
a_w&=&-ig^{\prime}t(\hat{L}_x\otimes\hat{\sigma}_z)_w-g^{\prime^2}\frac{t^2}{2}(\hat{L}_x\otimes\hat{\sigma}_z)_w^2\nonumber,\\
A^{\prime}_w&=&(\hat{\sigma}_z)_w+ig^\prime t (\hat{L}_x\otimes\hat{\sigma}_x\hat{\sigma}_z)_w.
\label{eq:weak_noise}
 \end{eqnarray}
%ei obdi holo
So, the final state of the meter can be rewritten as
\begin{equation}
 \ket{\Phi_f}_{dis} \approx \braket{\Psi_f|\Psi_{in}}\sum_{k=-N}^N\; e^{a_w} e^{igq_kA^{\prime}_w}\exp\Big(-\frac{q_k^2}{4\Delta^2}\Big)\ket{q_k},
\end{equation}
%Neglecting $g^2$ terms in between and those terms which would not contribute to the shift of the meter, we get
%\begin{eqnarray}
%\label{eqn:weak21}
 % A_w=-(\sigma_z)_w +g^\prime t((L_x\otimes\sigma_x\sigma_z)_w + \nonumber \\
 % (\sigma_z)_w (L_x\otimes\sigma_x)_w) 
%\end{eqnarray}
with the corresponding $p$-representation being
\begin{eqnarray}
 &&\ket{\Phi_f}_{dis}\approx \braket{\Psi_f|\Psi_{in}}\sum_{l=-N}^N e^{a_w} \exp\Big[-\Delta^2(p_l-gA^{\prime}_w)^2\Big]\nonumber\\
 &&\phantom{ami hridoyer bolite byakul sudha} \times \xi(p_l-gA_w^{\prime})\ket{p_l}\label{eq:30}\\
 &&\approx \braket{\Psi_f|\Psi_{in}}\sum_{l=-N}^N e^{a_w^{\prime}} \exp\Big[-\Delta^2\Big(p_l-gA^{\prime\prime}_w\Big)^2\Big]\ket{p_l}.\label{eq:31}
\end{eqnarray}
Here $a^{\prime}_w$ and $A^{\prime\prime}_w$ are implicitly defined via the expression~(\ref{eq:30}), which is equal to the  expression~(\ref{eq:31}) and $A^{\prime\prime}_w\rightarrow A^{\prime}_w$, $a^{\prime}_w\rightarrow a_w$ as $N\rightarrow \infty$. The factor $e^{a_w^{\prime}}$ does not contribute to the shift of the meter. So the deflection of the meter is proportional to the weak value $A_w^{\prime\prime}$. This $A_w^{\prime\prime}$ is difficult to be given in a closed (explicit) analytic form. In the continuous limit $N\rightarrow \infty$, when the meter state is taken as a continuous one, given in Eq.~(\ref{eq:meter}), the effective observable, measured in the weak measurement conjured by the noisy Hamiltonian in Eq. ({\ref{eq:Hnoisy}}), is given by
\begin{equation}
    A^{\prime}=\hat{\sigma}_z+ig^\prime t \hat{L}_x\otimes\hat{\sigma}_x\hat{\sigma}_z.
    \label{eq:A_w_noisy}
\end{equation}
%\textcolor{blue}{A discrete type pointer is required by some noise models in the experimental realization. We study this type of noisy situations in Appendix~\ref{discrete}.}
%Previously, we have discussed a noisy scenario which contained a two-body Hamiltonian. Noise can also act as in a less physical, three body scenario
The steps of the calculation for the continuous meter case are given in Appendix~\ref{discrete}. In the further discussion of our paper we will use the effective observable obtained in the continuous limit. The results for the discrete case will  however be close to those obtained using the effective observable \(A^{\prime}\) in Eq.~(\ref{eq:A_w_noisy}) for large \(N\). Moreover, there are instances below where the discrete and continuous cases match (in form).

If instead of $\hat{\sigma}_x$ the noise term in the Hamiltonian of Eq.~(\ref{eq:Hnoisy}) contains $\hat{\sigma}_z$, as in the cases, %$L_x \otimes \sigma_z$ and $L_z \otimes \sigma_z$, by calculating in similar fashion the resultant Hamiltonian would be obtained as
\begin{eqnarray}
    H_1=-g\delta(t)\hat{I} \otimes \hat{\sigma}_z \otimes  \hat{q}+g^{\prime}  \hat{L}_x \otimes \hat{\sigma}_z \otimes \hat{I},    \nonumber \\
    H_2=-g\delta(t)\hat{I} \otimes \hat{\sigma}_z \otimes  \hat{q}+g^{\prime}  \hat{L}_z \otimes \hat{\sigma}_z \otimes \hat{I},
    \label{eq:noise:parallel}
\end{eqnarray}
then by calculating the effective observable in a similar fashion, we get, respectively,
%instead of Eq. ({\ref{eq:Hnoisy}}) respectively and the effective observable would be 
\begin{eqnarray}
    &&A_1^{\prime}=\hat{\sigma}_z+ig^\prime t \hat{L}_x\otimes\hat{I},\nonumber \\
    &&A_2^{\prime}=\hat{\sigma}_z+ig^\prime t \hat{L}_z\otimes\hat{I}.
\end{eqnarray}

We can also consider a noisy interaction Hamiltonian of a different form. Precisely, we can 
%Let us 
take the noisy part of the interaction Hamiltonian to be a three-body term so that 
%\textcolor{blue}{\sout{the}} 
it is coupled to the meter with a coupling parameter $g(t)$ which has a compact support near the measurement time $t$: %\textcolor{red}{[Again, pls check if the change that i made is consistent in ordering with Eq.~(\ref{eq:Hnoisy}) and elsewhere.]} \textcolor{blue}{[ha, thik achhe.]}
    \begin{eqnarray}
        H^\prime=-g(t)\big(\hat{I}\otimes\hat{\sigma}_z \otimes \hat{q}-\hat{L}_x\otimes\hat{\sigma}_x \otimes \hat{q}\big).
        \label{eq:Hnoisysimple}
    \end{eqnarray}
Using the same method as before, we see that the effective observable resulting in the shift of the meter is 
\begin{equation}  
    A^\prime=\hat{\sigma}_z-\hat{L}_x\otimes\hat{\sigma}_x
    \label{eq:A_w_1_noisy}.
   \end{equation}


%\textcolor{red}{[ei Eq.~(\ref{eq:Hnoisysimple})-er HamiltonianTa-ke ar kothao use karbe na, tai to? If so, we have to somehow argue that this is not very relevant. Maybe because three-body interaction is unlikely? Or have to say that we will not consider it any further, without claiming it to be irrelevant, just to keep our claim on it, and come back to it maybe later in a separate paper? Or maybe we say that this is much simpler scenario, which can be easily handled in the method shown below?]} \textcolor{blue}{[ei Hamiltonian er jonyo ki result asbe seta ekhn eq. (34) r niche likhe diyechhi.]}
%In the succeeding subsections we discuss about the amplification of weak values in presence of noise by first isolating the noise. It may provide some advantages or disadvantages over the ideal case.
%\subsection{Amplification of weak value in noisy scenario}
%\label{section:4-a}


\begin{figure}
\includegraphics[height=4cm,width=8cm]{linear_final.pdf}
\caption{Setup for amplifying the weak value of the effective observable in the noisy scenario. The $L$-splitter adds an orbital angular momentum degree of freedom to the physical system. The $L^\prime$-splitter transmits the component of orbital angular momentum parallel to $\ket{v_a}$ and reflects any component orthogonal to it.}
\label{fig2}
\end{figure}

To demonstrate 
the working principle in either case [i.e., when the interaction Hamiltonian is given by either Eq.~(\ref{eq:Hnoisysimple}) or~(\ref{eq:Hnoisy})],
%how this \textcolor{red}{[kon "this"? ekhane, discussionTa Eq.~(\ref{eq:Hnoisysimple})-er janye kara hochchhe, na Eq.~(\ref{eq:Hnoisy})-er janye?]} \textcolor{blue}{[2 to case er jonyei. setup ta 2tor jonyei same. r weak value 2to likhe diyechhi.]} works, 
let us consider a set of pre-selected and post-selected states as follows:
\begin{eqnarray}
\ket{\chi_{in}}&=&\frac{1}{\sqrt{2}}(\ket{v_a}+i\ket{v_b})\otimes \ket{H},\nonumber\\
\ket{\chi_f}&=&\ket{v_a} \otimes(\cos\alpha\ket{H}+\sin\alpha\ket{V})
\label{pre_post_noise_1}.
    \end{eqnarray}
In these states, the first degree of freedom represents the angular momentum component \(\hat{L}_x\), while the second degree of freedom represents polarization. We assume that the orbital quantum number \(l\) is conserved, for the system under study, at 1. Correspondingly, the dimension of the space spanned by the eigenvectors of \(\hat{L}_x\) is three dimensional.
For simplicity,
%The lowest dimension of the orbital angular momentum $L_x$ is $3$, but for simplicity here 
we work in a scenario where one of the dimensions in this three-dimensional space is naturally or artificially forbidden. 
The remaining %discard one dimension and present the angular momentum operator in 
two-dimensional Hilbert space is spanned by orthonormal 
%basis 
vectors $\ket{v_a}$ and $\ket{v_b}$, with $\hat{L}_x$ represented as
\begin{equation}
    \hat{L}_x=-i(\ket{v_a}\bra{v_b}-\ket{v_b}\bra{v_a}),
\end{equation}
where 
    \begin{eqnarray}
        \ket{v_a}= \frac{1}{\sqrt{2}}\begin{pmatrix} 1 \\ 0 \\ 1
        \end{pmatrix}  , \quad
        \ket{v_b}= \begin{pmatrix} 0 \\- i \\ 0
        \end{pmatrix}, 
    \end{eqnarray}
expressed in the eigenbasis of \(\hat{L}_z\). To prepare the pre-selected state, we can send a photon with an initial polarization $\ket{H}$ through an arrangement, which we call the $L$-splitter  in Fig.~\ref{fig2}, where it acquires an angular momentum of $\frac{1}{\sqrt{2}}(\ket{v_a}+i\ket{v_b})$. We intend to measure the weak value of $\hat{\sigma}_z$ and so bring in a meter in the initial state given by Eq.~(\ref{eq:dis}) for the discrete pointer state and Eq.~(\ref{eq:meter}) for the continuous pointer state and set up an interaction of the form in Eq.~(\ref{eqn:hoz}). However, unknown to us, the form of interaction is actually as in Eq.~(\ref{eq:Hnoisy}). 
%\textcolor{red}{[naki Eq.~(\ref{eq:Hnoisy}) bolte chaichho?]} \textcolor{blue}{[Eq. (15) hbe eta. kore diyechhi.]}
To get the post-selected state, the photon passes through another arrangement, the $L^\prime$-splitter, that transmits orbital angular momentum $\ket{v_a}$ and reflects any orthogonal component towards the detector $D_3$. The transmitted photon is then passed through a polarization beam-splitter $PBS$, that is chosen so that it transmits light of polarization $\cos\alpha\ket{H}+\sin\alpha\ket{V}$ towards a detector $D_1$ and reflects any orthogonal polarization en route to detector $D_2$. Thus by selecting the clicks of $D_1$ alone, we can post-select the state $\ket{\chi_f}$. %\textcolor{red}{[naki ``post-select'' bolte chaichho?]} %$\ket{\chi_f}$.  
     
%The suffix $L$ and $S$ represent the Hilbert space of orbital and spin angular momentum respectively.
However, due to the inherent noise in the Hamiltonian of the form~(\ref{eq:Hnoisy}), as a result of the extra spin-orbital interaction, the weak value of the effective observable 
$A^{\prime}=\hat{\sigma}_z+ig^\prime t \hat{L}_x\otimes\hat{\sigma}_x\hat{\sigma}_z$ is measured (instead of \(\hat{\sigma}_z\)) to be %\textcolor{red}{[tan-er bhetar imaginary raye gyalo to!]} \textcolor{blue}{[asole ota multiply hoye thakbe. kore diyechhi.]}
%$A^{\prime}=\hat{\sigma}_z-\hat{L}_x\otimes\hat{\sigma}_x$ is measured to be
\begin{eqnarray}
 A^\prime_w={(g^\prime t+i)\tan\alpha}.
%    A^\prime_w=-i\tan\alpha(2g^\prime t+1)
\label{A_wk}
\end{eqnarray}
In contrast, if the unknown noise is a three-body interaction as in Eq.~(\ref{eq:Hnoisysimple}), the weak value of the effective observable given in Eq.~(\ref{eq:A_w_1_noisy}) turns out to be
\begin{equation}
   A_w^{\prime}=1+i \tan\alpha
    \label{A_wk1}
\end{equation} 
using the setup schematically demonstrated in Fig.~\ref{fig2}.




So, in both noisy situations [Eqs. (\ref{eq:Hnoisy}) and (\ref{eq:Hnoisysimple})] we can amplify the weak value of the respective effective observables [see Eqs. (\ref{eq:A_w_noisy}) and (\ref{eq:A_w_1_noisy})] by varying the parameter $\alpha$ using different $PBS$s. However, due to the presence of spin-orbit coupling in the system, we are unable to determine the weak value of $\hat{\sigma}_z$ as, %\sout{we do not achieve the amplification of $\hat{\sigma}_z$ here.} 
instead of $\hat{\sigma}_z$, the deflection of the meter is proportional to the weak value of an observable which contains an unwanted noise along with $\hat{\sigma}_z$. Hence, while attempting to amplify the weak value of the $z$-component of polarization,
%of a spin-$\frac{1}{2}$ system, 
we unintentionally amplify the weak value of a constituent noise. This, e.g., may cause disadvantages in  applications of quantum technologies, which depend on the weak-value enhancement of the polarization degree of freedom of the system. It is plausible that the presence of the noise effects considered is more probable than the ideal noiseless situation, and therefore it will be beneficial to design an experimental setup that can ``disembody'' the noise from the required observable.
%\textcolor{red}{[ager duTo sentence ekeybarei clear noi. kar weak value amplify kora gyalo, ar kar kora gyalo na? kora gyalo na, naki noise na thakle jataTa partum, tatoTa parlum na? kataTa partum, ar kataTa parlum?]}

\subsection{Disembodiment of noise from the ideal system using quantum Cheshire cats %\textcolor{cyan}{cats}
%\textcolor{red}{[cats or cat?]}
}
\label{section:5}



\begin{figure}
\includegraphics[height=8cm,width=8cm]{final_diagram.pdf}
\caption{
%\textcolor{red}{
%[change karbo!]}
Interferometric setup for separating the spin-orbit-coupling --like noise and simultaneously amplifying the signal corresponding to the chosen observable. See the text for details.
%from the required observable to be measured.
}
\label{fig3}
\end{figure}    

%Noise can create trouble in some applications of quantum mechanics depending on weak value measurements just like the cataracts making a cat's eyesight cloudy. 
%The situation can be presented by the parody,
%\begin{quote}\emph{Cataracts made the cat's eyes cloudy,\\ 
 %  A doctor made her eyesight so bright\\ 
  % that she can't imagine.} 
  % \end{quote}
  With the general working principle established earlier in Sec.~\ref{section:2}, we now proceed to get %\textcolor{blue}{\sout{rid}} 
  the amplified signal
  of the $z$-component of polarization by disassociating, with the help of the Cheshire cat mechanism. %\sout{off}
   
  The noise originates from the interaction with the unintended degree of freedom (in this case, $\hat{L}_x$) during the weak measurement process. 
  %\textcolor{red}{[ager sentenceTa bojha jachchhe na!]} 
  The intended pre-selected and post-selected states are $\ket{\Psi_1^\prime}$ and $\ket{\Psi_f}$, respectively [see Eqs.~(\ref{eq: preselect}) and~(\ref{eq:3})]. However, the photon passing through the $L$-splitter picks up a new degree of freedom, an angular momentum component given by $\frac{1}{\sqrt{2}}(\ket{v_a}+i\ket{v_b})$, as shown in Fig.~\ref{fig3}. So the effective pre-selected state is
    \begin{equation}
       \ket{\chi_{in}^{\prime}}=(\cos\frac{\theta}{2}\ket{L}- 
       i\sin\frac{\theta}{2}\ket{R}\otimes\frac{1}{\sqrt{2}}(\ket{v_a}+i\ket{v_b})\otimes \ket{H}.
       \label{eq:pree}
    \end{equation} 
To achieve the desired outcome, we are required to carry out post-selection in     
\begin{equation}
\ket{\chi_f^{\prime}}=(\cos\alpha\ket{L}\ket{H}+ \sin\alpha\ket{R}\ket{V})\otimes\ket{v_a}.
\label{eq:post}
\end{equation}
To obtain this state, the beam-splitter $BS_2$ needs to be of a transmission coefficient $\cos^2 \alpha$ and reflection coefficient $\sin^2 \alpha$ rather than being the $50:50$ type. Also, we assume that the state passes through a device, the $L^\prime$-splitter, that permits only the orbital angular momentum component along $\ket{v_a}$ towards the $PBS$ and reflects any orthogonal component towards the detector $D_3$, as shown in Fig.~\ref{fig3}.        
        The measured weak values are
        \begin{eqnarray}
        &&(\hat{\sigma}_z^L)_w = 0,\nonumber \\
        &&(\hat{\sigma}_z^R)_w = \tan\frac{\theta}{2}\tan\alpha, \nonumber \\
        &&(\hat{L}_x\otimes\hat{\sigma}_x)^L_w = 1, \nonumber \\
        &&(\hat{L}_x\otimes\hat{\sigma}_x)^R_w = 0. 
        \label{wk_final}
        \end{eqnarray}
Therefore, we can conclude that one can amplify the weak value of the observable $\hat{\sigma}_z$ separating it from the noise part $\hat{L}_x \otimes \hat{\sigma}_x$. Moreover, the enhancement of the weak value of $\hat{\sigma}_z^R$ can be more than that of the effective observables obtained in the two previous cases [described in Eqs.~(\ref{A_wk}) and~(\ref{A_wk1}), using a linear setup, as 
%demonstrated 
depicted
in Fig.~\ref{fig2}], because %\textcolor{blue}{
of the 
%\sout{in the weak value of $\hat{\sigma}_z^R$ there is an}} 
extra tuning parameter $\theta$. Therefore, even if we fix the parameter $\alpha$, we are still able to amplify the weak value of $\hat{\sigma}_z$ by changing the parameter $\theta$ with the use of different $PBS_1$s. So now we have successfully achieved our goal of amplifying the weak value of the required observable by splitting up the noise from the system. In addition, the enhancement of the weak value of the required observable is
%rather
greater than that of the effective observable detectable in the noisy situation [compare Eqs.~(\ref{A_wk}) and~(\ref{A_wk1}) with Eq.~(\ref{wk_final})]. %\textcolor{red}{[ager sentenceTa aro clearly likhte habe: darkar hale aro boRo kare lekho, tinte panchTa jata khushi sentence use koro! prathamata "previous" bolte kar kotha bola hochchhe, seTa clearly bolte habe: equation number, section, Hamiltonian, etc bole. dwitiyata, previous case-eo to ekTa tuning parameter chhilo nischai, jeTa use kare amra amplify korchhilum. ekhane tar thheke bhalo ki pelum, ebang kyano pelum? eiTai amader paper-er mool result. eTa-ke aro focussed karte habe, precise karte habe ki pelum!]} 
We need to choose 
%This can be achieved by choosing 
an initial state of suitable polarization in the preparation of the pre-selected state. Also, to measure the weak values in an experimental procedure, we have to construct a unitary parallel to the one mentioned in the noiseless situation. 
%\textcolor{red}{\sout{As our analysis is based on a noisy scenario that adapt continuous distribution of the pointer states (cf. Eq.~(\ref{eq:meter})), a continuous pointer %unlike the case presented in Eq. (\ref{eq:unitary_noiseless}), 
%has to be taken.}} 
The composite setup of the system and meter can be acted on by the Hamiltonian $H_{\hat{\sigma}_z^R}^{\prime}$ for measuring $\hat{\sigma}_z^R$, where
\begin{eqnarray}
H_{\hat{\sigma}_z^R}^{\prime} &=& g\delta(t-t^{\prime})[(\hat{I}-\hat{\Pi}_R) \otimes \hat{I} \otimes \hat{\sigma}_z \otimes \hat{I}\nonumber\\
&&\phantom{na go ei je dhula}+\hat{\Pi}_R \otimes \hat{I}\otimes\hat{\sigma}_z \otimes \hat{q}]. \label{eq:H_sigma}
\end{eqnarray} 
So the unitary generated by the Hamiltonian $H_{\hat{\sigma}_z^R}^{\prime}$ will be $U_{\hat{\sigma}_z^R}^{\prime}=e^{-\int iH_{\hat{\sigma}_z^R}^{\prime}t dt}$. After the post selection, the meter state turns out to be
\begin{equation}
\ket{\Phi_m}=\braket{\chi_f^{\prime}|\chi_{in}^{\prime}}[1-ig(\hat{\sigma}_z^R)_w \hat{q}\ket{\Phi_{in}}].
\end{equation}
Hence, the deflection of the meter state is proportional to the weak value of $\hat{\sigma}_z$ on the right arm, up to the first-order term in the expansion of the unitary.
%Let us consider the meter is a continuous system described by, $\ket{\Phi}_m=\int d\varepsilon \ket{\varepsilon}\braket{\varepsilon|\Phi}_m$, where $\{\ket{\varepsilon}\}$is a completer orthogonal basis having two basis elements $\ket{\varepsilon^{\prime}}$ and $\ket{\varepsilon^{''}}$.
%related by the completeness relation $\int d\varepsilon^{\prime} \braket{\varepsilon^{\prime}|\varepsilon^{''}}=\delta(\varepsilon^{\prime}-\varepsilon^{''})$. 
%The initial state of the meter is taken as $\ket{\varepsilon^{\prime}}$. The unitary in this case will be straightforward to construct similarly as in Eq.~(\ref{eq:unitary_noiseless}),
%\begin{eqnarray}
%   \hat{U}_{\sigma_z^R}=\frac{1}{\sqrt{2}}[(\hat{I}-\hat{\Pi}_R)\otimes \hat{I} \otimes\hat{\sigma}_z\otimes \hat{I} \nonumber \\+ \hat{\Pi}_R\otimes \hat{I} \otimes \hat{\sigma}_z\otimes R^{\prime^{-1}}(\theta_g)Z^{\prime}R^{\prime}(\theta(g)],
%\end{eqnarray}
%where $Z^{\prime}=\ket{\varepsilon^{\prime}}\bra{\varepsilon^{\prime}} - \ket{\varepsilon^{''}}\bra{\varepsilon^{''}}$ and 
%\begin{eqnarray}
%\label{eqn:R}
%R^{\prime}(\theta_g)\ket{\varepsilon^{\prime}} &=& \cos(2\theta_g)\ket{\varepsilon^{\prime}} +  \sin(2\theta_g)\ket{\varepsilon^{''}}, \nonumber\\
%R^{\prime}(\theta_g)\ket{\varepsilon^{''}} &=& \sin(2\theta_g)\ket{\varepsilon^{\prime}} - \cos(2\theta_g)\ket{\varepsilon^{''}}. 
%\label{eq:unitary_noise}
%    \end{eqnarray}}
%i.e. $e^{-\int H\,dt}$ (where $H$ is the respective Hamiltonian for the particular case. But, if in practice a situation arises that we could apply an unitary which is discrete, we could use the following method.} \par
%\textcolor{magenta}{The setup is for measuring $\hat{\sigma}_z^R$. Again, we set $g=4\theta_g$. For the post-selection given by Eq.~(\ref{eq:post}), the meter goes to the state 
%    \begin{eqnarray}
%        \ket{\Phi}_m =\int d\varepsilon \ket{\varepsilon}\bra{\varepsilon}\{(\sigma_Z^R)_w (\ket{\varepsilon^{\prime}}+g \ket{\varepsilon^{''}})\},
%    \end{eqnarray}
%So, the deflection of the meter is again proportional to the weak value $(\sigma_z^R)_w$. 
The constructed unitary is almost the same as in the noiseless scenario,
the only difference being that we have to incorporate an identity in the degree of freedom of orbital angular momentum. %\textcolor{blue}{\sout{,i.e., we have to use} 
%\textcolor{red}{[In this n the next equation, pls use the order of the systems as in Eq.~(\ref{eq:Hnoisy}). PLEASE DO IT CONSISTENTLY IN THE ENTIRE PAPER!!!]} \textcolor{blue}{[eta thik achhe.]}
%\begin{eqnarray}
   %\stkout{\hat{U}_{\sigma_z^R}=\frac{1}{\sqrt{2}}[(\hat{I}-\hat{\Pi}_R)\otimes \hat{I} \otimes\hat{\sigma}_z\otimes \hat{I}} \nonumber \\+ \stkout{\hat{\Pi}_R\otimes \hat{I} \otimes \hat{\sigma}_z\otimes R^{-1}(\theta_g)ZR(\theta(g)],}
%\end{eqnarray}
%\sout{and the same for measuring $(\hat{L}_x\otimes \hat{\sigma}_x)^L$ is}
 %   \begin{eqnarray}
        %\stkout{\hat{U}_{(L_x\otimes \hat{\sigma}_x)^L}=\frac{1}{\sqrt{2}}[(\hat{I}-\hat{\Pi}_L)\otimes (\hat{L}_x\otimes \hat{\sigma}_x) \otimes \hat{I}} \nonumber \\+ \stkout{ \hat{\Pi}_L \otimes (\hat{L}_x\otimes \hat{\sigma}_x) \otimes R^{-1}(\theta_g)ZR(\theta_g)].}
%    \end{eqnarray}
%\sout{$R(\theta_g)$ and $Z$ are the same as in the noiseless case.}} 
Similarly, to measure $(\hat{L}_x\otimes \hat{\sigma}_x)^L$, the unitary is $\hat{U}_{(\hat{L}_x\otimes \hat{\sigma}_x)^L}^{\prime}$, where
% \begin{eqnarray}
%        \hat{U}_{(L_x\otimes \hat{\sigma}_x)^L}=\frac{1}{\sqrt{2}}[(\hat{I}-\hat{\Pi}_L)\otimes (\hat{L}_x\otimes \hat{\sigma}_x) \otimes \hat{I} \nonumber \\+ \hat{\Pi}_L \otimes (\hat{L}_x\otimes \hat{\sigma}_x) \otimes R^{\prime^{-1}}(\theta_g)Z^{\prime}R^{\prime}(\theta_g)].
%        \label{eq:unitary_noise1}
%    \end{eqnarray}
\begin{eqnarray}
     H_{(\hat{L}_x\otimes \hat{\sigma}_x)^L}^{\prime} &=& g^{\prime}[(\hat{I}-\hat{\Pi}_L) \otimes \hat{L}_x \otimes \hat{\sigma}_x \otimes \hat{I}\nonumber\\
     &&\phantom{jodi tare}+ \hat{\Pi}_L \otimes \hat{L}_x\otimes\hat{\sigma}_x \otimes \hat{q}],
     \label{eq:H_last}
\end{eqnarray}
and here the final meter state ends up being
\begin{equation}
    \ket{\Phi_m}=\braket{\chi_f^{\prime}|\chi_{in}^{\prime}}[1-ig^{\prime}t(\hat{L}_x\otimes\hat{\sigma}_x)^R)_w \hat{q}\ket{\Phi_{in}}].
\end{equation}
%In case the noise is of discrete form a discrete pointer can be used as in Eq.~(\ref{eq:unitary_noiseless}).
%and the unitary for that purpose can be expressed exactly similar as the continuous case (Eqs.~(\ref{eq:unitary_noise}) and~(\ref{eq:unitary_noise1})) only the $R^{\prime}(\theta_g)$ and $Z^{\prime}$ are replaced by $R(\theta_g)$ and $Z$ respectively.
%See ~\cite{npj,m, opticsguo} for the experimental realization of using discrete pointers. %Also, the paper dealt the effect with two non-
%commuting observables which is highly contrasting to
%our situation as the primary signal (photon) always does
%commute with the noise signal.
Note that the subscript dis or con has been omitted here, as the same form of the equation is true in both cases.
%These experimental realizations of the noisy setup do not depend on the meter's initial state. The above-mentioned unitaries will remain unaltered whatever the meter's state, continuous or discrete.
\par
%\textcolor{red}{tolar part ta dekho !!! debmalya da} \\



In practice, a more complex scenario can arise when the noise couples with the measured observable, as in Eq.~(\ref{eq:noise:parallel}). In this case, one may not be able to find a suitable pre- and post-selection to decouple the noise from $\hat{\sigma}_z$. It could be possible to separate $\hat{\sigma}_z$ and $\hat{L}_x \otimes \hat{\sigma}_z$ or $\hat{L}_z \otimes \hat{\sigma}_z$, but in both arms of the interferometer, there still remains a contribution of $\hat{\sigma}_z$. Hence the complete dissociation of the $z$-component of polarization of the photon 
%completely 
from the noise part may not be achievable by this method.

In the noise model proposed here, the angular momentum degree of freedom of the system is acting as a source of noise and it couples to another degree of freedom of the system, which we want to measure. So it is like a subsystem interacting with the original system. In a realistic noisy scenario, to introduce a generic noise in the system, we take an auxiliary system from outside, operate a global unitary on the composite system-auxiliary setup, and then trace out the auxiliary. The same method can be followed with a different degree of freedom of the system instead of using the auxiliary system. The mathematical modeling will be the same in both instances. There are previous works where the noise is generated from a degree of freedom of the system itself. See, e.g., \cite{Zukowski1999,Zukowski2023}. Also, in \cite{Flores} it was shown that decoherence effects can be generated due to the coupling of mesoscopic variables of the system and internal degrees of freedom of the same. The center of mass of a system can have different degrees of freedom, which may interfere, and such interference can get effectively decohered due to the coupling of the center of mass of the system with the internal vibrational degrees of freedom, as was studied in~\cite{Brun}. See also~\cite{Hillery,Nikolic} in this regard. %We have also considered these types of noises where one of the system's internal degrees of freedom is acting as a noise source.


An additional point to be noted here is that  in practical scenarios, the noise source is usually unknown and we have to trace out the subsystem. In our case, for the measurement, we have to concentrate on the meter state, which is done by removing both degrees of freedom of the system from the total system including the meter. This tracing out is performed implicitly in the method of weak measurement. This implicit tracing out of the system has  also been performed in previous papers in this direction, e.g., in~\cite{AAV,Duck}. Note that for this implicit tracing out of the system, the noise source is also being traced out. So no additional explicit tracing out,  corresponding to the noise in the system, is required.
\section{Conclusion}
\label{section: conclusion}
%\textcolor{red}{[eTa pare dekhbo!!]} 
To summarize, we have proposed a thought experiment related to the so-called quantum Cheshire cat in which a component of polarization could be amplified independently of the photon, using interferometric arrangements. %This could easily be done using the other components of polarization ($x$ and $y$) also. 
Furthermore, we extended the setup to a scenario in which, of two complementary polarization components of a photon, one  can be amplified while being detached from the other.
Moreover, we considered a noisy scenario in which the noise is generated by a spin-orbit-coupling --like interaction term in the Hamiltonian governing the measurement process. 
We analyzed the amplification of a chosen observable in the presence of noise, both with and without the noise term being 
dissociated, on average, from the object by using a quantum Cheshire cat--inspired setup. 

%using two orthogonal properties of the photon, namely $x$ and $z$-component of polarization where we can amplify one of the polarization components independent of the other. We have extended our work for a noisy scenario where the spin-orbit coupling of a particle gives rise to a noise in the system and there is a possibility that the noise effect could create disturbances in precision measurement, phase estimation etc. using the weak value amplification method. To get rid of this unwanted trouble of measurement we have devised an interferometric setup by which we can decouple the noise from the ideal system on an average, so that we can measure the weak value of the required observable accurately on an average by increasing the number of measurements.\par


%The whole conception of quantum Cheshire cat, including what has been presented in this paper challenges the 'classical' fact that any of the property of a system is 'tied' up to the system at each time. We can see that some situations could be designed in such a way that a property gets detached from the object itself, which seems to be quite bizarre in the classical realm. We have gone further by amplifying a property independent of the system; which do not have any classical analog.\par 


It has been pointed out in the literature that the phenomena related to the quantum Cheshire cat are ``average'' effects. 
In particular, the weak values indicate average shifts of the meter,
%It is a point of dispute that the weak value is an average shift of the meter, 
conditioned on the pre-selected and the post-selected states, and the object and the property (or two properties of the same object) do not actually travel separately in each arm of the experimental setup; rather they do so only on an average. 
%In this case, We would like to remind the reader that the original quantum Cheshire Cat is, itself, by that logic, an average effect, with the photon and the polarization only decoupling on an average. 
Nonetheless, just like for the original quantum Cheshire cat, in our setups also, it has been ensured that the weak values observed are not classical averages, but quantum-mechanical mean shifts, obtained by considering coupling between the photon or polarization component and a meter. Let us also note here that the amplifications reported  do not imply that the amplified values could become arbitrarily large; nonlinear effects appear that limit the amplified values~\cite{Limits7}.\par %\textcolor{magenta}{
%A similar argument and further discussions can be found in ~\cite{DasSen}. 
%\sout{A definitive literature on the issue that the measurement process can be viewed as an unambiguous quantum physical scenario, although, the implementation there is done on sequential weak measurements~\cite{m} which is not the case in our situation, but the logic still holds. Also, the paper dealt the effect with two non-commuting observables which is highly contrasting to our situation as the primary signal (photon) always does commute with the noise signal.}} \textcolor{red}{[ei part ta bodh hy na lekhai valo. amader signal r noise commute kore na. eta Debmalya da ektu dekhe nio please. Paper ta introduction e refer kore diyechhi.]}
%Further discussions on this point can e.g. be found in~\cite{DasSen}.
%
%
%In~\cite{DasPati} it is shown that two quantum Cheshire cats (system) could exchange their grins (properties). When we combine it with this work, we can predict that it is possible to achieve the phenomena of separating a property, exchange them and amplify them independent of their respective systems.\par
%As it was mentioned in subsection \ref{subsection: amp}


\appendix
\section{Effective observable when meter state is continuously distributed}
\label{discrete}
For the initial meter state given in Eq.~(\ref{eq:meter}), the final meter state will take the form
\begin{eqnarray}
  \ket{\Phi_f}_{con}  &\approx&  \braket{\Psi_f|\Psi_{in}}\int dq \, e^{-\frac{q^2}{4\Delta^2}}\Big[1+i g q(\hat{\sigma}_z)_w\nonumber\\ 
  &-& i g^{\prime} t (\hat{L}_x\otimes \hat{\sigma}_x)_w-gg^\prime tq (\hat{L}_x\otimes \hat{\sigma}_x \hat{\sigma}_z)_w \nonumber\\
  &-& g^{\prime^2}\frac{t^2}{2}(\hat{L}_x\otimes \hat{\sigma}_x)_w^2\Big]\ket{q}. 
  \end{eqnarray}
  This can be written as
\begin{equation}
 \ket{\Phi_f}_{con} \approx \braket{\Psi_f|\Psi_{in}}\int \,dq\, e^{iqgA_w}\exp(-\frac{q^2}{4\Delta^2})\ket{q},
  \end{equation}
  with the same $a_w$ and $A^{\prime}_w$ as in Eq.~(\ref{eq:weak_noise}) and hence, the effective observable $A^{\prime}$ is the same as in Eq.~(\ref{eq:A_w_noisy}).
%ei obdi holo
So, the final state of the meter turns out to be
\begin{equation}
 \ket{\Phi_f}_{con} \approx \braket{\Psi_f|\Psi_{in}}\int \,dq\,e^{a_w} e^{iqgA^{\prime}_w}\exp(-\frac{q^2}{4\Delta^2})\ket{q},
\end{equation}
%Neglecting $g^2$ terms in between and those terms which would not contribute to the shift of the meter, we get
%\begin{eqnarray}
%\label{eqn:weak21}
 % A_w=-(\sigma_z)_w +g^\prime t((L_x\otimes\sigma_x\sigma_z)_w + \nonumber \\
 % (\sigma_z)_w (L_x\otimes\sigma_x)_w) 
%\end{eqnarray}
with the corresponding $p$-representation being
\begin{equation}
 \ket{\Phi_f}_{con} \approx \braket{\Psi_f|\Psi_{in}}\int \,dp\,e^{a_w} \exp[-\Delta^2(p-gA^{\prime}_w)^2]\ket{p}.
\end{equation}



\acknowledgements
A.G. and U.S. acknowledge partial support from the Department of Science and Technology, Government of India through QuEST grant (grant number DST/ICPS/QUST/Theme-3/2019/120). 



\begin{thebibliography}{100}
\bibitem{U} P. B. Dixon, D. J. Starling, A. N. Jordan, and J. C. Howell, \textit{Ultrasensitive Beam Deflection Measurement via Interferometric Weak Value Amplification}, Phys. Rev. Lett. \textbf{102}, 173601 (2009).

\bibitem{Busch1984}
 P. Busch and P. J. Lahti, \textit{On various joint measurements of position and momentum observables in quantum theory}, Phys. Rev. D \textbf{29}, 1634 (1984).
 
 \bibitem{Barchielli1982}
A. Barchielli, L. Lanz, and G.M. Prosperi,
\textit{A model for the macroscopic description and continual observations in quantum mechanics}, 
Il Nuovo Cimento B \textbf{72}, 79 (1982).

\bibitem{Caves1986}
  C. M. Caves, \textit{Quantum mechanics of measurements distributed in time. A path-integral formulation},  Phys. Rev. D \textbf{33}, 1643 (1986).
 
 \bibitem{AAV} Y. Aharonov, D. Z. Albert, and L. Vaidman, \textit{How the result of a measurement of a component of the spin of a spin-1/2 particle can turn out to be 100}, Phys. Rev. Lett. \textbf{60}, 1351 (1988).

\bibitem{Duck} I. M. Duck, P. M. Stevenson, and E. C. G. Sudarshan, \textit{The sense in which a ``weak measurement" of a spin-1/2 particle's spin component yields a value 100}, Phys. Rev. D \textbf{40}, 2112 (1989).

\bibitem{YA} Y. Aharonov, S. Popescu, and J. Tollaksen, \textit{A time-symmetric formulation of quantum mechanics}, Phys. Today \textbf{63}, 27 (2010).

\bibitem{Pryde} G. J. Pryde, J. L. O'Brien, A. G. White, T. C. Ralph, and H. M. Wiseman, \textit{Measurement of Quantum Weak Values of Photon Polarization}, Phys.Rev.Lett. \textbf{94}, 220405 (2005). 

\bibitem{Hosten} O. Hosten and P. Kwiat, \textit{Observation of the Spin Hall Effect of Light via Weak Measurements}, Science \textbf{319}, 787 (2008).

\bibitem{Lundeen} J. S. Lundeen, B. Sutherland, A. Patel, C. Stewart, and C. Bamber, \textit{Direct measurement of the quantum wave function}, Nature (London) \textbf{474}, 188 (2011).

\bibitem{Denkmayr} T. Denkmayr, H. Geppert, S. Sponar, H. Lemmel, A. Matzkin, J. Tollaksen, and Y. Hasegawa, \textit{Observation of a quantum Cheshire Cat in a matter-wave interferometer experiment}, Nat. Commun. \textbf{5}, 4492 (2014).

\bibitem{Correa} R. Corr\^{e}a , M. F. Santos , C. H. Monken, and P. Saldanha \textit{`Quantum Cheshire Cat' as simple quantum interference}, New J. Phys. \textbf{17} 053042 (2015).

\bibitem{Sponar} S. Sponar, T. Denkmayr, H. Geppert, and Y. Hasegawa, \textit{Fundamental Features of Quantum Dynamics Studied in Matter-Wave Interferometry—Spin Weak Values and the Quantum Cheshire-Cat}, Atoms \textbf{4}, 11 (2016).

\bibitem{Ashby} J. M. Ashby, P. D. Schwarz, and M. Schlosshauer, \textit{Observation of the quantum paradox of separation of a single photon from one of its properties}, Phys.Rev.A \textbf{94}, 012102 (2016).

\bibitem{Cormann} M. Cormann, M. Remy, B. Kolaric, and Y. Caudano, \textit{Revealing geometric phases in modular and weak values with a quantum eraser}, Phys. Rev. A \textbf{93}, 042124 (2016).

\bibitem{L2} J. S. Lundeen and C. Bamber, \textit{Procedure for Direct Measurement of General Quantum States Using Weak Measurement}, Phys. Rev. Lett. \textbf{108}, 070402 (2012).


\bibitem{tomo1} H. F. Hofmann, \textit{Complete characterization of post-selected quantum statistics using weak measurement tomography}, Phys. Rev. A \textbf{81}, 012103 (2010).

\bibitem{tomo2} S. Wu, \textit{State tomography via weak measurements}, Sci. Rep. \textbf{3}, 1193 (2013).







\bibitem{geometry} M. Cormann and Y. Caudano, \textit{Geometric description of modular and weak values in discrete quantum systems using the Majorana representation}, J. Phys. A: Math. Theor. \textbf{50}, 305302 (2017).

\bibitem{sv} H. Kobayashi, K. Nonaka, and Y. Shikano, \textit{Stereographical visualization of a polarization state using weak measurements with an optical-vortex beam}, Phys. Rev. A \textbf{89}, 053816 (2014).

\bibitem{qt} A. K. Pati, C. Mukhopadhyay, S. Chakraborty, and S. Ghosh, \textit{Quantum precision thermometry with weak measurements}, Phys. Rev. A 102, 012204 (2020).

\bibitem{nonherm1} A. K. Pati, U. Singh, and U. Sinha, \textit{Measuring non-Hermitian operators via weak values}, Phys. Rev. A \textbf{92}, 052120 (2015).

\bibitem{nonherm2} G. Nirala, S. N. Sahoo, A. K. Pati, and U. Sinha, \textit{Measuring average of non-Hermitian operator with weak value in a Mach-Zehnder interferometer}, Phys. Rev. A \textbf{99}, 022111 (2019).

\bibitem{Jozsa} R. Jozsa, \textit{Complex weak values in quantum measurement}, Phys. Rev. A \textbf{76}, 044103 (2007).

\bibitem{twostate} C. A. Chatzidimitriou-Dreismann, \textit{Weak Values and Two-State Vector Formalism in Elementary Scattering and Reflectivity—A New Effect}, Universe \textbf{5}, 58 (2019).

\bibitem{Super} M. Nairn, \textit{Superoscillations: Realisation of quantum weak values},	arXiv:2109.14404.








%\bibitem{YA} Y. Aharonov, S. Popescu, and J. Tollaksen, \textit{A time-symmetric formulation of quantum mechanics}, Phys. Today \textbf{63}, 27 (2010).





\bibitem{Aharonov} Y. Aharonov, S. Popescu, D. Rohrlich, and P. Skrzypczyk, \textit{Quantum Cheshire Cats}, New J. Phys. \textbf{15}, 113015 (2013).

\bibitem{npj} M.-J. Hu, Z.-Y. Zhou, X.-M. Hu, C.-F. Li, G.-C. Guo and Y.-S. Zhang, \textit{Observation of non-locality sharing among three observers
with one entangled pair via optimal weak measurement}, npj Quantum Inf. \textbf{4}, 63 (2018).

\bibitem{m} Y. Kim, Y.-S. Kim, S.-Y. Lee, S.-W. Han, S. Moon, Y.-H. Kim,  and Y.-W. Cho, \textit{Direct quantum process tomography via measuring sequential weak values of incompatible observables}. Nat Commun. \textbf{9}, 192 (2018).



\bibitem{SNR1} G. Scala, M. D'Angelo, A. Garuccio, S. Pascazio, and F. V. Pepe, \textit{Signal-to-noise properties of correlation plenoptic imaging with chaotic light}, Phys. Rev. A \textbf{99}, 053808 (2019).

\bibitem{SNR2} L. Li, X. Li, B. Zhang, and L. You, \textit{Enhancing test precision for local Lorentz-symmetry violation with entanglement}, Phys. Rev. A \textbf{99}, 042118 (2019).

\bibitem{SNR3} J.-H. Huang, F.-F. He, X.-Y. Duan, G.-J. Wang, and X.-Y. Hu, \textit{Improving the precision of weak-value-amplification with two cascaded Michelson interferometers based on Vernier-effect}, arXiv:2107.14395 %[physics.optics].

\bibitem{SNR4} Y. Jin, J. Yan, S. J. Rahman, X. Yu, and J. Zhang, \textit{Imaging the dipole scattering of an optically levitated dielectric nanoparticle}, Appl. Phys. Lett. \textbf{119}, 021106 (2021).

\bibitem{SNR5} Y.-Z. Ma, M. Jin, D.-L. Chen, Z.-Quan Zhou, C.-Feng Li, and G.-C. Guo, \textit{Elimination of noise in optically rephased photon echoes}, Nat. Commun. \textbf{12}, 4378 (2021).

\bibitem{SNR6} L. Bai, L. Zhang, Y. Yang, R. Chang, Y. Qin, J. He, X. Wen, and J. Wang, \textit{Enhancement of spin noise spectroscopy of rubidium atomic ensemble by using of the polarization squeezed light}, 	arXiv:2111.09572 %[quant-ph]. 

\bibitem{Alice} L. Carroll, \textit{Alice’s Adventures in Wonderland} (Macmillan, London, 1865).

\bibitem{Kim2021}
Y. Kim and D. G. Im, and Y. S. Kim, \textit{Observing the quantum Cheshire cat effect with noninvasive weak measurement}, npj Quantum Inf. \textbf{7}, 13 (2021).

\bibitem{Bancal} J.-D. Bancal, \textit{Quantum physics: Isolate the subject}, Nature Phys. \textbf{10}, 11 (2014).

\bibitem{At} D. P. Atherton, G. Ranjit, A. A .Geraci, and J. D. Weinstein, \textit{Observation of a classical Cheshire cat in an optical interferometer}, Opt. Lett. \textbf{40} 879 (2015).

\bibitem{Duprey} Q. Duprey, S. Kanjilal, U. Sinha, D. Home, and A. Matzkin, \textit{The Quantum Cheshire Cat effect: Theoretical basis and observational implications}, Ann. Phys. \textbf{391} 1 (2018).

\bibitem{dynamic} Y. Aharonov, E. Cohen, and S. Popescu, \textit{A dynamical quantum Cheshire Cat effect and implications for counterfactual communication}, Nat Commun. \textbf{12}, 4770 (2021).
%\bibitem{kim} Y. Kim, Y.-S. Kim, S.-Y. Lee, S.-W. Han, S. Moon, Y.-Ho. Kim and Y.-W. Cho, \textit{Direct quantum process tomography via measuring sequential weak values of incompatible observables}, Naure Communications, \textbf{9}, 192 (2018)

\bibitem{DasPati} D. Das and A. K. Pati, \textit{Can two quantum Cheshire cats exchange grins?}, New J. Phys., \textbf{22}, 063032 (2020).

\bibitem{DasSen} D. Das and U. Sen, \textit{Delayed choice of paths in the quantum paradox of separating different properties of a photon}, Phys. Rev. A \textbf{103}, 012228 (2021).

\bibitem{wavepar} P. Chowdhury, A. K. Pati, and J. L. Chen, \textit{Wave and particle properties can be spatially separated in a quantum entity} Photon. Res. \text{9}, 1379 (2021).

\bibitem{Liu2020} Z.-H. Liu, W.-W. Pan, X.-Y. Xu, M. Yang, J. Zhou, Z.-Y. Luo, K. Sun, J.-L. Chen, J.-S. Xu, C.-F. Li, and G.-C. Guo, \textit{Experimental exchange of grins between quantum Cheshire cats}, Nat. Commun. \textbf{11}, 3006 (2020).

\bibitem{opticsguo} J.-S. Chen, M.-J. Hu, X.-M. Hu, B.-H. Liu, Y.-F. Huang, C.-F. Li, C.-G. Guo, and Y.-S. Zhang, \textit{Experimental realization of sequential weak measurements of non-commuting Pauli observables}, Opt. Express \textbf{27}, 6089 (2019).

\bibitem{Zukowski1999} M. \.Zukowski, A. Zeilinger, M. A. Horne, and H. Weinfurter,   \textit{Independent Photons and Entanglement. A Short Overview.}, International Journal of Theoretical Physics \textbf{38}, 501 (1999). 

\bibitem{Zukowski2023} A. Sen(De), U. Sen, and Marek \.Zukowski, \textit{Output state in multiple entanglement swapping}, Phys. Rev. A \textbf{68}, 062301 (2003).

\bibitem{Flores} J. C. Flores, \textit{Decoherence from internal degrees of freedom for clusters of mesoparticles: a hierarchy of master equations}, J. Phys. A: Math. Gen. \textbf{31}, 8623 (1998).

%\bibitem{parv1} A.Patel and P. Kumar, \textit{Weak measurements, quantum-state collapse, and the Born rule}, Phys. Rev. A \textbf{96}, 022108 (2017).



%\bibitem{parv2} K. Snizhko, P. Kumar, N. Rao, and Y. Gefen, \textit{Quantum Zeno effect appears in stages},  Phys. Rev. Research \textbf{2} , 033512 (2020).

\bibitem{Brun} T. A. Brun and L. Mlodinow, \textit{Decoherence by coupling to internal vibrational modes}, Phys. Rev. A \textbf{94}, 052123 (2016).


\bibitem{Hillery} M. Hillery, L. Mlodinow, and V. Bu\v{z}ek, \textit{Quantum interference with molecules: The role of internal states}, Phys. Rev. A \textbf{71}, 062103 (2005).

\bibitem{Nikolic} H. Nikoli\'{c}, \textit{Internal environment: what is it like to be a Schr\"{o}dinger cat?}, Eur. J. Phys. \textbf{36}, 045003 (2015).


\bibitem{Limits7} T. Koike and S. Tanaka, \textit{Limits on amplification by Aharonov-Albert-Vaidman weak measurement}, Phys. Rev. A \textbf{84}, 062106 (2011).









 
 
 
 
 
 
 
 
 
 
 
 
 
 
 
 
 

%extra


%\bibitem{Popescu} Y. Guryanova, N. Brunner and S. Popescu, \textit{The Complete Quantum Cheshire Cat},  arXiv:1203.4215.
%\bibitem{Ib} I. Ibnouhsein and A. Grinbaum, \textit{Twin Quantum Cheshire Photons}, arXiv:1202.4894.
%\bibitem{Deco} M. Richter , B. Dziewit and J. Dajka , \textit{The Quantum Cheshire Cat Effect in the Presence of Decoherence}, Adv. Math. Phys. \textbf{2018}, 7060586 (2018).
%\bibitem{3box} A. Matzkin and A. K.Pan, \textit{Three-box paradox and 'Cheshire cat grin': the case of spin-1 atoms}, J. Phys.A:Math. Theor. \textbf{46}, 315307 (2013).
%\bibitem{DP2} D. Das D and A. K. Pati, \textit{Teleporting Grin of a Quantum Chesire Cat without cat}, arXiv:1903.01452.

%\bibitem{AKP} A. K. Pan, \textit{Disembodiment of arbitrary number of properties in quantum Cheshire cat experiment}, Eur. Phys. J. D, \textbf{74}, 151 (2020).

%\bibitem{noninvasive} Y. Kim, , DG. Im,, YS. Kim, et al.,\textit{ Observing the quantum Cheshire cat effect with noninvasive weak measurement}, npj Quantum Inf \textbf{7}, 13 (2021).
%\bibitem{uhome} S. N. Sahoo, D. Home, A. Matzkin, and U. Sinha, \textit{Comment on “Observing the "quantum Cheshire cat" effect with noninvasive weak measurement”}, arXiv:2006.00792. 
%\bibitem{monalisa} R. Ahmad, S. Nawaz, \textit{Quantum Mona Lisa Cat}, 	arXiv:2001.10184.














%\bibitem{qtc} Y. Pan et al. \textit{Weak measurement, projective measurement and quantum-to-classical transitions in electron-photon interactions}, arXiv:1910.11685.

%\bibitem{Zeno} Y. Aharonov, E. Cohen, S. Popescu, \textit{A Current of the Cheshire Cat's Smile: Dynamical Analysis of Weak Values}, arXiv:1510.03087.
%\bibitem{Torres} J. Torres, L. J. Salazar-Serrano, \textit{Weak value amplification: a view from quantum estimation theory that highlights what it is and what isn’t}, Sci Rep \textbf{6}, 19702 (2016).
%\bibitem{Wei} W. Wu, S .Chen, W. Xu, Z. Liu, R. Lou, L. Shen, H. Luo, S. Wen, and X. Yin, \textit{Weak-value amplification for the optical signature of topological phase transitions}, Photon. Res. \textbf{8}, B47 (2020).
%\bibitem{nonU} W. Liu, J. Martínez-Rincón, and J. C. Howell, \textit{Weak value amplification for nonunitary evolution}, Phys. Rev. A \textbf{100}, 012125 (2019).
%\bibitem{WVAB} J.  Ren, L. Qin, W. Feng, and X. Li, \textit{Weak-value-amplification analysis beyond the AAV limit of weak measurements}, Phys. Rev. A \textbf{102}, 042601 (2020).
%\bibitem{J} J. Sinclair, D. Spierings, A. Brodutch, A. M. Steinberg, \textit{Interpreting weak value amplification with a toy realist model}, Phys. Lett. A, \textbf{383},  (2019).

%\bibitem{shot} A. Nishizawa, K. Nakamura, and M.-K. Fujimoto, \textit{Weak-value amplification in a shot-noise-limited interferometer}, Phys. Rev. A \textbf{85}, 062108 (2012).
%\bibitem{noiseweak} J. E. Gray and A. D. Parks, \textit{Noise from the perspective of weak values}, Proc. SPIE \textbf{7702}, Quantum Information and Computation VIII, 77020P (2010).



\end{thebibliography}

\end{document}


\section{Extension of the gedanken to two orthogonal properties of the photon}
\label{section: 4}

%\begin{figure}
%\includegraphics[height=8cm,width=8cm]{diagram2.pdf}
%\caption{Configuration for the amplification of \emph{grin} whereas the \emph{snarl} is as usual. The setup is visually similar with the configuration of Fig. \ref{fig1} with an extra polarising beam-splitter $PBS_2^{\prime}$ and a detector $D^{''}$ in the right arm of the interferometer, although the functions of the apparatus are different. For the functions of the different elements of this setup see the text. The measuring device is also same as that of \cite{Aharonov}, the only difference is, the polarising beam-splitter $PBS_3$ is chosen suce that $\ket{V}$ is transmitted and $\ket{H}$ is reflected, whereas in the previous case (Fig. \ref{fig1}) and in \cite{Aharonov} the polarising beam-splitter with the opposite function is chosen. The weak measurement is done between the second pre-selection and the post-selection.}
%\label{fig2}
%\end{figure}

In the previous section, the position and the $z$-components of the spin have been shown to be separated and the spin component has been shown to be amplified by choosing a second pre-selected state Eq. (\ref{eq: preselect}). Now we will extend the gedanken by amplifying one component of spin by separating it from another (orthogonal) component of that spin, and the former will be amplified independent of the latter. \par
We take the two orthogonal properties to be $x$ and $z$-components of spin where they are given as $\sigma_x=\ket{H}\bra{H}+\ket{V}\bra{V}$ and $\sigma_z=i(\ket{V}\bra{H}-\ket{H}\bra{V})$ respectively. The weak values of the operators $\sigma_x^{(L)}=\Pi_L\otimes \sigma_x$, $\sigma_x^{(R)}=\Pi_R\otimes \sigma_x$, $\sigma_z^{(L)}=\Pi_L\otimes \sigma_z$ and $\sigma_z^{(R)}=\Pi_R\otimes \sigma_z$ are calculated. The $z$-component of spin is seen to be amplified independent of the $x$-component of spin, or analogically we can tell the \emph{grin} is bigger than usual whereas the \emph{snarl} is as usual.\par
To achieve this, we keep the first pre-selection as in the previous case Eq. (\ref{eq:3}) and prepare a state 
\begin{equation}
    \ket{\psi_2^{\prime}}=\cos\frac{\theta}{2}\ket{L}\ket{V}+i\sin\frac{\theta}{2}\ket{R}\ket{V}
\end{equation} 
as the second pre-selected state. The the post-selected state is taken as
\begin{equation}
    \ket{\phi^{\prime}}=\frac{1}{\sqrt{2}}(\ket{L}\ket{V}+\ket{R}\ket{H}) 
    \end{equation}
    Now, the weak values are calculated as 
    \begin{eqnarray}
        (\sigma_x^{(L)})_w&=&\frac{\Braket{\phi^{\prime}|\sigma_x^{(L)}|\psi_2^{\prime}}}{{\Braket{\phi^{\prime}|\psi_2^{\prime}}}}=1 \nonumber\\
        (\sigma_x^{(R)})_w&=&\frac{\Braket{\phi^{\prime}|\sigma_x^{(R)}|\psi_2^{\prime}}}{{\Braket{\phi^{\prime}|\psi_2^{\prime}}}}=0 \nonumber\\
        (\sigma_z^{(L)})_w&=&\frac{\Braket{\phi^{\prime}|\sigma_z^{(L)}|\psi_2^{\prime}}}{{\Braket{\phi^{\prime}|\psi_2^{\prime}}}}=0 \nonumber\\
        (\sigma_z^{(R)})_w&=&\frac{\Braket{\phi^{\prime}|\sigma_z^{(R)}|\psi_2^{\prime}}}{{\Braket{\phi^{\prime}|\psi_2^{\prime}}}}=tan\frac{\theta}{2}
    \end{eqnarray}
 In the setup, for the second pre-selection we put a phase shifter $P_2^{\prime}$ in the left arm which acts as $\ket{H} \rightarrow \cos\frac{\theta}{2}\ket{V}+ sin\frac{\theta}{2}\ket{H}$ and a phase shifter $P_1^{\prime}$ which acts as $\ket{H} \rightarrow \cos\frac{\theta}{2}+i sin\frac{\theta}{2}$ in the right arm. After that, two polarising beam-splitters ($PBS_1^{\prime}$ and $PBS_2^{\prime}$) are put one on each arm which reflects $\ket{H}$ and transmits $\ket{V}$. \par 
 %In this section, the post-selection apparatus, is practically the mirror image of the post-selection apparatus of the previous section's gedanken.\par
It is seen that the weak values of $\sigma_z^{(R)}$ in both the sections are seen to acquire a value $tan\frac{\theta}{2}$, which means the weak values will result in outcomes beyond the eigenvalue spectrum in the region $\theta \in (\pi/2,\pi)$ for positive amplification and $\theta \in (-\pi,-\pi/2)$ for negative amplification.

