In this section, our three-step methodology implemented in the R package \textsf{MultiVarSel} and available from the CRAN, 
is applied to a LC-MS (Liquid Chromatography-Mass Spectrometry) data set made of African copals samples. 
The samples correspond to  ethanolic extracts of copals produced by trees belonging to two genera \textit{Copaifera} (C) and \textit{Trachylobium} (T) with a second level of classification coming 
from the geographical provenance of the \textit{Copaifera} samples (West (W) or East (E) Africa). Since all the \textit{Trachylobium} samples come from East Africa, we have a single factor
having three levels: CE, CW and TE such that $n_{\textrm{CE}}=9$, $n_{\textrm{CW}}=8$ and $n_{\textrm{TE}}=13$.

In this section, we also compare the performance of our method with those of other techniques which are widely used in metabolomics. 


\subsection{Data pre-processing}

LC-MS chromatograms were aligned using the R package
XCMS proposed by \cite{Smith2006} with the following parameters: a signal to noise
ratio threshold of 10:1 for peak selection,  a step size of 0.2 min
and a minimum difference in \textit{m}/\textit{z} for peaks with
overlapping retention times of 0.05 amu. 
Sample filtering was also performed: To be considered as informative, as suggested by \cite{Kirwan2013}, a peak was required to be present in at least 80\% of the samples. 
Missing values imputation was realized using the KNN algorithm described in \cite{Hrydziuszko2012}.
Subsequently, the spectra were normalized to equalize signal
intensities to the median profile in order to reduce any variance
arising from differing dilutions of the biological extracts and
probabilistic quotient normalization (PQN) was used, see
\cite{Dieterle2006} for further details. In order to reduce the size of the data matrix,
selection of the adducts of interest [M+H]$^{+}$ was then performed 
using the CAMERA package of \cite{Kuhl2012}. A $n\times q$ matrix $\boldsymbol{Y}$ was then obtained and submitted to the statistical analyses.

\subsection{Application of our three-step approach}

The observations matrix $\boldsymbol{Y}$ is first centered and scaled in order to ensure that the
empirical mean in each column is 0 and that the empirical variance is 1.

\subsubsection{First step} A one-way ANOVA is fitted to each column of the observation matrix 
 $\boldsymbol{Y}$ in order to have access to an
  estimation $\widehat{\boldsymbol{E}}$ of the residual matrix
  $\boldsymbol{E}$. Then, the test proposed in Section \ref{sec:whitening_test} is applied. 
We found a $p$-value equal to zero
which indicates that the columns of $\widehat{\boldsymbol{E}}$ cannot
be considered as independent and hence that applying the whitening
strategy should improve the results.

\subsubsection{Second step} The different whitening strategies described in Section \ref{sec:estim_sigma_q} were applied and the highest 
$p$-value for the test described in Section \ref{sec:whitening_test} is obtained for the nonparametric whitening. 
More precisely, the $p$-values obtained for the AR(1) and the nonparametric dependence modeling
are equal to $1.5\times 10^{-4}$ and 0.5107, respectively. Hence, in the following we shall use the nonparametric modeling.


\subsubsection{Third step} The Lasso approach described in Section \ref{sec:lasso} was then applied to the
  whitened observations where  $\widehat{\boldsymbol{\Sigma}}_q$ is obtained by using the
  nonparametric modeling. The stability selection is then used with 5000 replications and a threshold equal to 1 in order to avoid false positive.

The Venn diagram of Figure \ref{fig:real_boulier} displays the repartition of the selected metabolites among the different classes CE, TE and CW.
We can see from this figure that at least one metabolite
is selected as a marker for each class (20 for TE, 22 for CW and 1 for
CE) for a total of 39 unique metabolites. More precisely, our methodology leads to a list of metabolites that mainly characterize a single class.



\begin{figure}[!h]
\begin{center}
\includegraphics[scale=0.15]{ven_our_classes.png}
\vspace{-10mm}
\caption{Venn diagram of the metabolites selected for each class by  \textsf{MultiVarSel}
 using a threshold equal to 1 in the stability selection stage.\label{fig:real_boulier}}
\end{center}
\end{figure}


\subsection{Comparison with existing methods}


The goal of this section is to compare the performance of our approach with those of methodologies classically used in metabolomics 
such as partial least square discriminant analysis (PLS-DA) 
and sparse partial least square discriminant analysis (sPLS-DA) devised by \cite{LeCao2011} and implemented in the R package \verb|MixOmics|. 

As recommended by \cite{LeCao2011}, we used two components for PLS-DA and sPLS-DA. Moreover, in order to make sPLS-DA comparable with our approach, 
20 variables are kept for each component in the sPLS-DA methodology. The corresponding results  are
displayed in Figure \ref{fig:2dplsda}. We can see from this figure that sPLS-DA exhibits better classification performance than the standard PLS-DA. 



\begin{figure}[!h]
\centering
\begin{tabular}{cc}
\includegraphics[scale=0.7]{plot2d_plsda.pdf} 
&\includegraphics[scale=0.55]{legende_plot2d_plsda.pdf} \\
\includegraphics[scale=0.7]{plot2d_splsda.pdf} & \\
\end{tabular}
\caption{2D scores plot of the PLS-DA and the sPLS-DA.\label{fig:2dplsda}}
\end{figure}


Since PLS-DA does not include a variable selection step we shall compare our approach only to sPLS-DA in the following. 
For comparing these methodologies Figure \ref{fig:pcaall} displays the PCA obtained when all the metabolites are kept on the one hand 
and when the metabolites are those selected by sPLS-DA or by our methodology on the other hand. We can see from this figure that, one the hand,
the approaches containing a variable selection step exhibit better classification performance and that, on the other hand, 
sPLS-DA and our method show similar performance from the classification point of view even if our approach is not designed for this purpose.



\begin{figure}[!h]
\centering
\begin{tabular}{cc}
\includegraphics[scale=0.7]{pca_all_2d_num.pdf} & 
\includegraphics[scale=0.55]{legende_plot2d_plsda.pdf}\\
\includegraphics[scale=0.7]{pca_splsda_2d_num.pdf}
& \\
\includegraphics[scale=0.7]{pca_our_2d_num.pdf}
& \\
\end{tabular}
\caption{PCA with all the metabolites and with the metabolites selected by sPLS-DA and our approach \textsf{MultiVarSel}.\label{fig:pcaall}}
\end{figure}


Figure \ref{fig:metabsel} displays the positions of the metabolites selected by our approach and sPLS-DA. We can see from this figure that out of the 39 selected 
metabolites, 6 metabolites are selected by both sPLS-DA and our methodology. The major difference between these two variable selection techniques is that
our method selects metabolites having a ratio $m/z$ smaller than 300 whereas the metabolites chosen by sPLS-DA lie within the range 300-400 $m/z$.


\begin{figure}[!h]
\begin{center}
\includegraphics[scale=0.75]{position_metab_selected.pdf}
\caption{Comparison of the metabolites selected by our approach \textsf{MultiVarSel} and by sPLS-DA.\label{fig:metabsel}}
\end{center}
\end{figure}



In order to further compare our methodology with sPLS-DA, we first propose to assess the stability of the selected variables (or metabolites). For this purpose, we 
performed 10 bootstrap resamplings of our original data and we compared the variables selected by both approaches. The results are displayed in
Figure \ref{fig:freq}  and in Table \ref{table:bootstrap_er_plsda}. Figure \ref{fig:freq} displays the frequencies at which each metabolite
has been selected by the two methods. We can see from this figure that the highest selection frequency of sPLS-DA is around 0.8 and that a lot of variables have a selection frequency smaller than 0.5. 
Moreover, we can see from Table \ref{table:bootstrap_er_plsda} which provides the number of metabolites which have been selected once (first row), 
twice (second row)..., that our approach selects 4 metabolites with a frequency equal to 1 which does not occur for sPLS-DA. Hence, from this point of view, our approach is
 more stable than sPLS-DA.





\begin{figure}[!h]
\begin{center}
\includegraphics[scale=0.8]{frequence_plots.pdf}
\caption{Frequencies of the metabolites selected by sPLS-DA and our approach \textsf{MultiVarSel}.\label{fig:freq}}
\end{center}
\end{figure}

\begin{table}[h!]
%\begin{minipage}[t]{.4\linewidth}
    \begin{tabular}{ccc}
    \hline
Nb of selection & Nb of selected metabolites & Nb of selected metabolites \\
  & by sPLS-DA &  by \textsf{MultiVarSel} \\
\hline
    1 & 117 &  143\\
  2 &  41& 54\\ 
  3 &  26 & 24\\ 
  4 &   4 & 20\\ 
  5 &   8 & 15\\ 
  6 &   5 & 6\\ 
  7 &   3 & 6\\ 
  8 &   2 & 3\\ 
  9 & 0  & 3\\ 
10 & 0 & 4\\
    \hline
    \end{tabular}
\caption{Number of times the different metabolites have been selected by sPLS-DA and our approach \textsf{MultiVarSel}.\label{table:bootstrap_er_plsda}}
\end{table}





Finally, we compare hereafter the set of variables provided by sPLS-DA
and our approach from the classification error point of view. Since our method is not designed for yielding a classification we give the selected variables to PLS-DA in order to obtain 
such a classification.  The estimation of the classification error rates are then obtained by using a 10-fold cross-validation.
The corresponding results are displayed in Tables \ref{table:CV_er_splsda} and \ref{table:CV_er_our_plsda}. We observe that the classification error rates of our approach
are on a par with those of sPLS-DA.


\begin{table}[ht]
\centering
\begin{tabular}{rrrr}
  \hline
 & TE & CE & CW \\ 
  \hline
TE & 0.92 & 0.33 & 0.00 \\ 
  CE & 0.08 & 0.67 & 0.00 \\ 
  CW & 0.00 & 0.00 & 1.00 \\ 
   \hline
\end{tabular}
\caption{Classification error rates for sPLS-DA.\label{table:CV_er_splsda}}
\end{table}

%our plsda 
\begin{table}[ht]
\centering
\begin{tabular}{rrrr}
  \hline
 & TE & CE & CW \\ 
  \hline
TE & 0.92 & 0.22 & 0.00 \\ 
  CE & 0.08 & 0.67 & 0.00 \\ 
  CW & 0.00 & 0.11 & 1.00 \\ 
   \hline
\end{tabular}
\caption{Classification error rates for our approach coupled with PLS-DA.\label{table:CV_er_our_plsda}}
\end{table}

We observe from these different investigations that our approach provides similar results as sPLS-DA in terms of classification even if our approach was not designed for this purpose and that 
it yields more stable variables (metabolites) than sPLS-DA for characterizing the different classes.

%%% Local Variables:
%%% mode: latex
%%% eval: (TeX-PDF-mode 1)
%%% TeX-master: "perrot_levy_chiquet_arxiv.tex"
%%% ispell-local-dictionary: "en_US"
%%% eval: (flyspell-mode 1)
%%% End: