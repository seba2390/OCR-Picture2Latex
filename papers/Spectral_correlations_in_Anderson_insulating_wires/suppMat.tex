\documentclass[twocolumn,showpacs,aps,prl]{revtex4}
%\usepackage{graphicx}
%\usepackage{amsmath}
%\usepackage{amssymb}
%\usepackage{color} 





\begin{document}





\title{Supplemental Material to:
``Spectral correlations in Anderson insulating wires" 
}

\author{M.~Marinho and  T.~Micklitz}

\affiliation{
Centro Brasileiro de Pesquisas F\'isicas, Rua Xavier Sigaud 150, 22290-180, Rio de Janeiro, Brazil 
}

\date{\today}

\pacs{72.15.Rn,73.20.Fz,03.75.-b,42.25.Dd}




\begin{abstract}

In this Supplemental Material, we summarize polar coordinates for $Q$-matrices 
and zero-modes of the transfermatrix Hamiltonian in the three Wigner-Dyson classes. 
We present details on the local generating function, the evaluation 
of the boundary integrals, and the calculation of the forward scattering peak. 


\end{abstract}


\maketitle




\section{Polar coordinates and zero modes}

For convenience of the reader we  
summarize polar coordinates and ground-state wave-functions (`zero-modes') 
for the three Wigner-Dyson classes. 

{\it Polar coordinates:---}The decomposition 
$Q=UQ_0U^{-1}$,
was introduced in the main text.
Here 
\begin{align}
&Q_0=\begin{pmatrix}
\cos\hat{\theta} & i\sin\hat{\theta} \\
-i\sin\hat{\theta}  & -\cos\hat{\theta} 
\end{pmatrix}_{\rm ra},
\quad
U=\begin{pmatrix}
u & 0 \\
0 & v
\end{pmatrix}_{\rm ra},
\end{align}
with 
$\hat{\theta}
={\rm diag}(i\hat{\theta}_{\rm bb}, \hat{\theta}_{\rm ff})_{\rm bf}$, 
and matrices $\hat{\theta}_{\rm bb,ff}$ in `${\rm tr}$'-sector 
depending on the fundamental symmetries,
\begin{align}
\hat{\theta}_{\rm bb}
&=
\begin{pmatrix}
\theta_{\rm b} & 0 \\
0 & \theta_{\rm b}
\end{pmatrix}_{\rm tr},
\quad
\hat{\theta}_{\rm ff}
=\begin{pmatrix}
\theta_{\rm f} & 0 \\
0 & \theta_{\rm f}
\end{pmatrix}_{\rm tr},
\quad ({\rm U})
\\
\hat{\theta}_{\rm bb}
&=
\begin{pmatrix}
\theta_{{\rm b},1} & \theta_{{\rm b},2} \\
\theta_{{\rm b},2} & \theta_{{\rm b},1}
\end{pmatrix}_{\rm tr},
\quad
\hat{\theta}_{\rm ff}
=\begin{pmatrix}
\theta_{\rm f} & 0 \\
0 & \theta_{\rm f}
\end{pmatrix}_{\rm tr},
\quad ({\rm O})
\\
\hat{\theta}_{\rm bb}
&=
\begin{pmatrix}
\theta_{\rm b} & 0 \\
0 & \theta_{\rm b}
\end{pmatrix}_{\rm tr},
\quad
\hat{\theta}_{\rm ff}
=\begin{pmatrix}
\theta_{{\rm f},1} & \theta_{{\rm f},2} \\
\theta_{{\rm f},2} & \theta_{{\rm f},1}
\end{pmatrix}_{\rm tr},
\quad ({\rm Sp}),
\end{align}
where compact and  
non-compact parameters $0<\theta_{\rm f} <\pi$ 
and
$\theta_{\rm b}>0$, respectively~\cite{SMEfetovBook}. 
In the main text we use radial variables 
$-1\leq \lambda_{\rm f}\equiv \cos\theta_{\rm f}\leq 1$, 
$1\leq \lambda_{\rm b}\equiv \cosh\theta_{\rm b}$, and suppress 
the graded index ${\rm b}$, ${\rm f}$ in favor of the `${\rm tr}$'-index, $1,2$.
It is convenient to parametrize 
$U=U_1U_2$,
with 
$U_i
={\rm diag}(u_i,  v_i)_{\rm ra}$ and 
$u_1$, $v_1$ containing all Grassmann variables and 
$u_2$, $v_2$ depending only on $c$-numbers. 
The latter are of no further 
relevance for us, and the former can be parametrized as  
$u_1(\hat{\eta})
= 1 - 2\hat{\eta}+ 2\hat{\eta}^2 - 4\hat{\eta}^3+ 6 \hat{\eta}^4$
and 
$v_1(\hat{\kappa})
= 1 - 2\hat{\kappa}+ 2\hat{\kappa}^2 - 4\hat{\kappa}^3+ 6 \hat{\kappa}^4$, 
where 
\begin{align}
\hat{\eta}&=
\begin{pmatrix}
0 & \eta^\dagger \\
-\eta & 0
\end{pmatrix}_{\rm bf},
\quad 
\hat{\kappa}
=
\begin{pmatrix}
0 & i\kappa^\dagger \\
-i\kappa & 0
\end{pmatrix}_{\rm bf},
\end{align}
with
\begin{align}
\eta
&=
\begin{pmatrix}
\eta_1 & 0\\
0 & -\eta_1^*
\end{pmatrix}_{\rm tr},
\quad
\kappa
=
\begin{pmatrix}
\kappa_1 & 0\\
0 & -\kappa_1^*
\end{pmatrix}_{\rm tr},
\quad ({\rm U})
\\
\eta
&=
\begin{pmatrix}
\eta_1 & \eta_2\\
-\eta_2^* & -\eta_1^*
\end{pmatrix}_{\rm tr},
\quad
\kappa
=
\begin{pmatrix}
\kappa_1 & \kappa_2 \\
-\kappa_2^* & -\kappa_1^*
\end{pmatrix}_{\rm tr},
\quad ({\rm O, Sp}).
\end{align}
We recall that
$(\eta_i^*)^*=-\eta_i$ and similar for $\kappa_i$, 
and notice that
$[u_1(\hat{\eta})]^{-1}=u_1(-\hat{\eta})$, 
and 
$[v_1(\hat{\kappa})]^{-1}=v_1(-\hat{\kappa})$~\cite{SMEfetovBook}.



{\it Zero-modes for Wigner-Dyson classes:---}The ground-state 
wave-functions  for the transfermatrix Hamiltonian 
of the Wigner-Dyson symmetry classes,  
recently derived in Ref.~\onlinecite{SMKhalaf}, 
read ($p=\sqrt{(\lambda+1)/2}$) 
\begin{widetext}
\begin{align}
&Y^{\rm U}_0(\lambda_{\rm b},\lambda_{\rm f})
=
4\sqrt{\eta}
\Big(
p_{\rm f} 
K_0(4\sqrt{\eta} p_{\rm b})I_1(4\sqrt{\eta}p_{\rm f})
 + 
p_{\rm b}
K_1(4\sqrt{\eta}p_{\rm b})I_0(4\sqrt{\eta}p_{\rm f})
\Big),
\\
&Y^{\rm O}_0(\lambda_1,\lambda_2,\lambda_{\rm f})
=
\sqrt{\eta}
\Big(
4p_1p_2
I_0(4\sqrt{\eta}p_{\rm f}) K_1(4\sqrt{\eta}p_1p_2)
+
{1\over p_{\rm f}}
( 1 + \lambda_1 +\lambda_2 +\lambda_{\rm f} )
I_1(4\sqrt{\eta}p_{\rm f}) K_0(4\sqrt{\eta}p_1p_2)
\Big),
\\
&Y^{\rm Sp}_0(\lambda_{\rm b},\lambda_1,\lambda_2)
=
\sqrt{\eta}
\Big(
4p_1p_2
I_1(4\sqrt{\eta}p_1p_2) K_0(4\sqrt{\eta}p_{\rm b})
+
{1\over p_{\rm b}}
( 1 + \lambda_1 +\lambda_2 +\lambda_{\rm b} )
I_0(4\sqrt{\eta}p_1p_2) K_1(4\sqrt{\eta}p_{\rm b})
\Big),
\\
&Y^{\rm Sp}_\pm(\lambda_{\rm b},\lambda_1,\lambda_2)
=
Y^{\rm Sp}_0(\lambda_{\rm b},\lambda_1,\lambda_2)
\pm 
Y^{\rm Sp}_0(\lambda_{\rm b},-\lambda_1,-\lambda_2).
\end{align} 
\end{widetext}




\section{Generating function}

Starting out from the generating function, 
introduced in the main text,
 \begin{align}
{\cal F}(\eta,\bold{x})
 &=
\left\langle 
{
\left[
{\rm str}\left(k\Lambda Q_\Lambda(\bold{x})\right)
\right]^2
\over
{\rm str}\left(\Lambda Q(\bold{x})\right)
}
\right\rangle_S,
\end{align}
it is verified that in polar coordinates
\begin{align}
{\rm str}\left(\Lambda Q\right)
&=
2\,{\rm str}( 
\cos\hat{\theta}), 
\\
{\rm str}\left(k\Lambda Q\right)
&=
{\rm str}\left(k \Lambda U_2^{-1} U^2_1U_2 Q_0\right),
\end{align}
where in the second line 
we employed cyclic invariance of the trace and anti-commutation of 
$\hat{\eta}$ and $k$.
 $k U_2^{-1}U_1^{2}U_2$ is a diagonal matrix in `${\rm ra}$'-sector,  
with `${\rm rr}$'-block 
\begin{align}
&
[ k U_2^{-1}U_1^{2}U_2]_{\rm rr}
\nonumber\\
&=
u_2^{-1}
k
\left[
1
-
8
\left(
\begin{smallmatrix}
\eta^\dagger\eta
-
4 (\eta^\dagger \eta)^2
& 0
\\ 0 
&
\eta \eta^\dagger 
-
4 (\eta  \eta^\dagger)^2 
\end{smallmatrix}\right)_{\rm bf}
\right]
u_2
+
... .
\end{align}
The omitted terms `$...$' summarize contributions  
off-diagonal in `${\rm bf}$'-sector, i.e. vanishing under the trace.
A similar expression involving $\kappa$ holds for the `${\rm aa}$'-block. 



In the unitary class 
 $\eta$ is diagonal in the 
 `${\rm tr}$'-sector, i.e. 
$\eta^\dagger\eta=-\eta \eta^\dagger
=\eta_1^\ast\eta_1\openone_{\rm tr}$,
and 
$(\eta^\dagger  \eta)^2=0$.
In the remaining classes $\eta$
has off-diagonal structure in the `{\rm tr}'-sector and 
$(\eta^\dagger  \eta)^2
=-(\eta  \eta^\dagger )^2
=2\eta_1^\ast \eta_2^\ast \eta_1\eta_2\openone_{\rm tr}$.
Recalling that   
in any correlation function we only need 
to account for contributions 
independent of Grassmann variables 
or being of maximal order of the latter 
(see e.g. Ref.~\onlinecite{SMZirnbauer}) 
we may simplify
\begin{align}
[ k U_2^{-1}U_1^{2}U_2]_{\rm rr}
&=
k
-
{\cal P}_{\cal G}^{\rm r}
+
...,
\end{align}
with 
${\cal P}_{\cal G}^{\rm r}\propto \eta_1^\ast\eta_1$ 
and
${\cal P}_{\cal G}^{\rm r}\propto \eta_1^\ast \eta_2^\ast \eta_1\eta_2$  
the (unnormalized) maximal polynomials 
of Grassmann variables in the retarded sector 
for the unitary, respectively, the orthogonal and symplectic
symmetry classes. 
A similar calculation for the `${\rm aa}$'-block  
then leads to 
 \begin{align}
{\rm str}\left(k\Lambda Q\right)
&=
{\rm str}( 
[2k+P_{\cal G}^{\rm r}
+
P_{\cal G}^{\rm a}]
\cos\hat{\theta}) 
+...,
\end{align}
where omitted contributions `$...$' give again  vanishing contributions to the correlation function.
That is, keeping only terms with non-vanishing contributions to the correlation function,
\begin{align}
\left[
{\rm str}\left(k\Lambda Q_\Lambda \right) 
\right]^2
&=
[{\rm str} 
(2k (\cos\hat{\theta}- \openone_4) )  
]^2
+
2 [{\rm str}( 
\cos\hat{\theta})  
]^2
P_{\cal G}, 
\end{align}
with 
$P_{\cal G}=
P_{\cal G}^{\rm r}
P_{\cal G}^{\rm a}$ 
the (unnormalized) maximal polynomial in Grassmann variables.
Similarly,
 \begin{align}
 \label{SMgf}
{\cal F}(\eta,\bold{x})
 &=
\left\langle 
{\rm str}( 
\cos\hat{\theta})
P_{\cal G} 
\right\rangle_S,
\end{align}
where we noticed that 
boundary terms 
(i.e. those independent of Grassmann variables) here vanish.
Finally, 
\begin{widetext}
\begin{align}
&\int (dx)\int (dy)\,
\langle
{\cal F}(\eta,\bold{x})
{\rm str}\left(\Lambda Q(\bold{y})\right)
\rangle_S
=
2
\int (dx)\int (dy)\,
\langle
{\rm str}(\cos\hat{\theta}(\bold{x}))\,
{\rm str}(\cos\hat{\theta}(\bold{y}))\,
P_{\cal G}(\bold{x})
\rangle_S,
\\
&\int (dx)\int (dy)\,
\langle
{\rm str}\left(k\Lambda Q_\Lambda(\bold{x})\right)
{\rm str}\left(k\Lambda Q_\Lambda(\bold{y})\right)
\rangle_S
=
2
\int (dx)\int (dy)\,
\langle
{\rm str}(\cos\hat{\theta}(\bold{x}))\,
{\rm str}(\cos\hat{\theta}(\bold{y}))\,
P_{\cal G}^{\rm a}(\bold{x})\,
P_{\cal G}^{\rm r}(\bold{y})
\rangle_S,
\end{align}
\end{widetext}
and both expression become identical upon 
 shifting 
$P_{\cal G}^{\rm r}(\bold{x})
\mapsto
P_{\cal G}^{\rm r}(\bold{x})
+
P_{\cal G}^{\rm r}(\bold{y})$, 
in the first term, 
and 
$P_{\cal G}^{\rm r}(\bold{y})
\mapsto
P_{\cal G}^{\rm r}(\bold{y})
+
P_{\cal G}^{\rm r}(\bold{x})$  in the second term, as discussed in the main text.
To fix all factors, we can then 
normalize the maximal polynomial of Grassmann
variables by considering the 
quantum-dot limit, where $Q(\bold{x})=Q_0$ is space-independent. That is,
for  
\begin{align}
P_{\cal G}
&={8\over\beta} P^0_{\cal G},
\end{align}
with $P^0_{\cal G}$ the normalized maximal polynomial of Grassmann variables 
(introduced in the main text) 
Eq.~\eqref{SMgf} recovers Wigner-Dyson statistics~\cite{SuppMatfn1}.







\section{Boundary terms}

In the main text we have derived the 
following 
representation for the generating function
for level-level correlations in one of 
the three Wigner-Dyson classes,
\begin{align}
\label{SMappbt}
{\cal F}(\eta)
&=
\int d{\cal R} \,
\partial_\lambda
 \sqrt{g} g^{\lambda\rho}
 \left( 
Y_0' 
\partial_\rho  Y_0
-
Y_0 
\partial_\rho  Y_0'
 \right),
\end{align}
where $Y_0'\equiv\partial_\eta Y_0$. 
We next discuss different ways to regularize Eq.~\eqref{SMappbt}, all  
leading to the same result. 

{\it Unitary class:---}Calculations are the simplest in the unitary class, 
where Eq.~\eqref{SMappbt} takes the form
\begin{widetext}
\begin{align}
{\cal F}^{\rm U}(\eta)
&=
\int_1^\infty d\lambda_{\rm b}
\int_{-1}^{1} d\lambda_{\rm f}\,
\left(
\partial_{\lambda_{\rm f}}  
{(1-\lambda_{\rm f}^2)G_{\rm f}(\lambda_{\rm b},\lambda_{\rm f})\over (\lambda_{\rm b}-\lambda_{\rm f})^2}
+
\partial_{\lambda_{\rm b}} 
{(\lambda_{\rm b}^2-1) G_{\rm b}(\lambda_{\rm b},\lambda_{\rm f}) \over (\lambda_{\rm b}-\lambda_{\rm f})^2}
\right),
\end{align}
\end{widetext}
and to simplify notation we introduced 
$G_{\rm f,b}
\equiv
Y_0' 
\partial_{\lambda_{\rm f,b}}   Y_0
-
Y_0 
\partial_{\lambda_{\rm f,b}}   Y_0'$. 
As discussed in the main text, the first (second) term vanishes 
at the boundary $\lambda_{\rm f}=1$ 
($\lambda_{\rm b}=1$). At the same time the Jacobian 
$\sqrt{g}=1/(\lambda_{\rm b}-\lambda_{\rm f})^2$
diverges at $\lambda_{\rm f}=\lambda_{\rm b}=1$.
To deal with this situation we regularize (some of) the radial variables shifting the 
limit of integration $1\mapsto 1^\pm\equiv 1\pm \epsilon$,
where the positive/negative sign applies for the bosonic/fermionic radial variable. 
Regularizing e.g. in the bosonic radial variable,
\begin{align}
{\cal F}^{\rm U}(\eta)
&=
2\lim_{\epsilon\to0}
\int_0^\infty 
dx_{\rm f}
{\epsilon G_{\rm f}(1, 1)
\over 
(x_{\rm f} +\epsilon)^2}
=
2G_{\rm f}(1, 1),
\end{align}
where $x_{\rm f}=1-\lambda_{\rm f}$
and we  used that in the limit $\epsilon\searrow 0$
we only 
need to keep leading contributions in $x_{\rm f}\ll1$. 
Similarly,  one finds upon
regularization in the fermionic radial variable
${\cal F}^{\rm U}(\eta)=-2G_{\rm b}(1, 1)$.
If, on the other hand, both variables are regularized,
\begin{align}
{\cal F}^{\rm U}(\eta)
&=
2\lim_{\epsilon\to0}
\left(
\int_\epsilon^\infty 
dx_{\rm f}
{\epsilon G_{\rm f}(1, 1)
\over 
(x_{\rm f} +\epsilon)^2}
-
\int_\epsilon^\infty 
dx_{\rm f}
{\epsilon G_{\rm f}(1, 1)
\over 
(x_{\rm f} +\epsilon)^2}
\right)
\nonumber\\
&
=
G_{\rm f}(1, 1)
-
G_{\rm b}(1, 1),
\end{align}
and the same is found 
in polar coordinates upon excluding $r=0$, 
i.e.
${\cal F}^{\rm U}(\eta)=
2\lim_{\epsilon\to0}
\int_0^{\pi/2} d\phi\,
\left(
{\cos^2\phi \, G_{\rm f}(1,1)\over(\cos\phi+\sin\phi)^2}
-
{\sin^2\phi \, G_{\rm b}(1,1)\over(\cos\phi+\sin\phi)^2}
\right)$.
Noting that
$G_{\rm f}(1,1)=-G_{\rm b}(1,1)$ 
it is verified that all of the above procedures give the same result. 



{\it Other classes:---}Calculations in the orthogonal and symplectic classes are more involved but follow 
the same line. Upon regularization, the boundary contribution $(\sqrt{g}g^{\lambda\rho})|_{\lambda=1^\pm}$
in Eq.~\eqref{SMappbt} reduces (up to a numerical factor) to a $\delta$-function in the remaining 
radial coordinates, fixing the latter to $\lambda=1$.
Zero-modes in all symmetry classes share the properties that   
$Y_0(\Lambda)=1$, $Y_0'(\Lambda)=0$,  and
$\partial_{\lambda}Y'_0|_{Q=\Lambda}=\pm \partial_{\rho}Y'_0|_{Q=\Lambda}$.
Here the positive sign applies if $\lambda$ and $\rho$ are both bosonic or fermionic
radial variables and the negative sign else, and we recall that $\Lambda$ corresponds  
to setting all radial variables $\lambda=1$.  These 
properties guarantee that the result is independent of which/how many integration limits are shifted.  

Changing e.g. in the orthogonal class 
the integration limit in the fermionic variable,
\begin{widetext}
\begin{align}
{\cal F}^{\rm O}(\eta)
&=
\lim_{\epsilon\to0}
\int_1^\infty d\lambda_1
\int_1^\infty d\lambda_2\,
\int_{-1}^{1-\epsilon} d\lambda_{\rm f}\,
\partial_{\lambda_{\rm f}} \sqrt{g} g^{\lambda_{\rm f} \lambda_{\rm f}} 
G_{\rm f}(\lambda_{\rm f},\lambda_1,\lambda_2)
\nonumber\\
&=
\lim_{\epsilon\to0}
\int_1^\infty d\lambda_1
\int_1^\infty d\lambda_2
\,
{4\epsilon^2 G_{\rm f}(1,\lambda_1,\lambda_2)
\over \left( 
\epsilon^2
+
\lambda_1^2+\lambda_2^2 
-
2\lambda_1\lambda_2
-
2\epsilon(1-\lambda_1\lambda_2)
\right)^2
}
=
2G_{\rm f}(1,1,1),
\end{align}
\end{widetext}
while a similar calculation with any of the two integration limits $\lambda_{1,2}$ shifted 
gives
${\cal F}^{\rm O}(\eta)
=-2G_{\rm b}(1,1,1)$, etc.
As stated in the main text, we may, therefore, equivalently present the final results
in a symmetrized form, e.g.
\begin{align}
{\cal F}^{\rm U}_0(\eta)
&=
\left(
\partial_{\lambda_{\rm b}} 
-
\partial_{\lambda_{\rm f}}
 \right)
Y_0'|_{\lambda_{\rm b}=\lambda_{\rm f}=1},
 \end{align}
in the unitary class, and similarly for the other symmetry classes.







\section{Forward peak}  

The time evolution of the forward scattering peak is, up to a normalization factor, 
given by the Fourier transform of the level-level correlation function. 
E.g. 
in the unitary class ($\eta=-i\omega/\Delta_\xi$)
\begin{align}
{\cal C}_{\rm fs}(t)
&
\propto
 \Delta_\xi^{-1} \int_{-\infty}^\infty d\omega\, e^{-i\omega t}
\partial_\eta K_0(4\sqrt{\eta})  I_0(4\sqrt{\eta})
\nonumber \\
&=
i\Delta_\xi t \theta(t)
\int_{0}^\infty dz
\, e^{-z t\Delta_\xi}
\Big(
K_0\left(4i\sqrt{z}\right)  
I_0\left(4i\sqrt{z}\right)
\nonumber\\
&\qquad \qquad 
-
K_0\left(-4i\sqrt{z}\right)  
I_0\left(-4i\sqrt{z}\right)
\Big)
\nonumber \\
&=
\Delta_\xi t  \theta(t)
\int_{0}^\infty dz
\, e^{-z t\Delta_\xi}
J_0\left(4\sqrt{z}\right)  
J_0\left(4\sqrt{z}\right)
\nonumber \\
&=
\theta(t)
 I_0\left( {8\over \Delta_\xi t} \right)
e^{-{8\over \Delta_\xi t}},
\end{align}
where in the third line we used that 
$K_n(-z)
=(-1)^nK_n(z)+\left( \log(z) - \log(-z) \right) I_n(z)$, 
$I_n(-z)
=(-1)^nI_n(z)$, and $I_n(iz)=i^nJ_n(z)$. 
A similar calculation for the remaining classes, 
leads to Eqs.~(19)-(21) stated in the main text.




%%%%%%%%%%%%%%%%%%%%%%%%%%%%%%%%%%%%%%%%%%%%%%%%%%%%%%%%%%%%%%%%%

\begin{thebibliography}{99}

\bibitem{SMEfetovBook}
K.~B.~Efetov, \textit{Supersymmetry in Disorder and Chaos} (Cambridge University Press, 1999). 

\bibitem{SMKhalaf}
E.~Khalaf, P.~M.~Ostrovsky, arXiv:1707.03369.

\bibitem{SMZirnbauer}
M.~Zirnbauer, Nucl. Phys. B {\bf 265}, 375 (1985).

\bibitem{SuppMatfn1}
Notice that we here used that integrals over 
variables parametrizing $U_2$ are normalized.

\end{thebibliography}





\end{document}

