%%%%%%%%%%%%%%%%%%%%%%%%%%%%%%%%%%%%%%%%%%%%%%%%%%%%%%%%%%%%%%%%%%%%%%%%%%%%%%%%
\section{Extended examples}\label{sec:examples}

%%%%%%%%%%%%%%%%%%%%%%%%%%%%%%%%%%%%%%%%%%%%%%%%%%%%%%%%%%%%%%%%%%%%%%%%%%%%%%%%
\subsection{Verifying a region of attraction}
For an initial application of our counterexample-guided Lyapunov function synthesis
procedure, we consider the linear dynamical system
$x(t+1)=Ax(t)$, where the $2\times 2$ matrix 
\[
	A = 
%	\begin{bmatrix}
%		0.70045415 & -0.26380309 \\
%		-0.22775951 & -0.46268814
%	\end{bmatrix}
	\begin{bmatrix}
		0.7005 & -0.2638 \\
		-0.2278 & -0.4627
	\end{bmatrix}
\]
is globally (Schur) stable, with spectral radius $\rho(A)=0.75$. The origin of
the dynamical system is globally exponentially stable.

We take $\mathcal{V}$ to be the class of max-of-affine Lyapunov functions. In
general this class (or the class of \acs{amn}-representable)
Lyapunov function candidates may not be large enough to prove global
exponential stability, even of linear dynamical systems. So instead, we verify
local stability in the box
$\mathcal{B}=\{x\in\reals^2 \mid \|x\|_\infty \leq 10\}$. In other
words, we verify the weaker claim that the set $\mathcal{B}$ is a region of
attraction for the equilibrium at the origin.  


Our algorithm terminates with the piecewise affine Lyapunov function
\begin{equation*}
\begin{aligned}
	V(x) &= \max\{
		x_2, 
		-0.1612x_1 -0.1020x_2,
		 0.4614x_1 +0.0155x_2,\\
		&-0.4212x_1 +0.1433x_2,
		-0.5156x_1 +0.0796x_2,
		 0.5036x_1 -0.1632x_2
	\}
\end{aligned}
\end{equation*}
using automatically generated counterexamples that were all constrained to
$\mathcal{B}$, see Figure~\ref{fig:lyap}. Although the function $V(x)$ is a
Lyapunov function by construction, we can illustrate that it decreases along
trajectories of the dynamical system $x(t+1)=Ax(t)$ by simulating the system at
several initial conditions within $\mathcal{B}$, and tracking the value of
$V(x(t))$, see Figure~\ref{fig:trajectories}.

\begin{figure}[htpb]
    \centering
    \begin{subfigure}[b]{0.49\linewidth}
        \centering
		% orig size: 0 -1 461 317 (left, lower, right, upper)
		\includegraphics[trim=75 0 50 0,clip,width=\textwidth]{code/fig/lyap_contours}
        \caption{Contours of $V(x)$}
		\label{fig:lyap_contours}
    \end{subfigure}
    \begin{subfigure}[b]{0.49\linewidth}
        \centering
		\includegraphics[trim=90 0 50 0,clip,width=\textwidth]{code/fig/lyap_3d}
        \caption{3D visualization of $V(x)$}
		\label{fig:lyap_3d}
    \end{subfigure}
	\caption{Contours (left, solid lines) and 3D visualization (right) of a
	candidate piecewise affine function, synthesized to decrease from the
	initial conditions (counterexamples) marked by a cross, $\times$, to their
	state-space locations marked by a plus, $+$, at the next time step. The
	counterexample conditions were generated automatically using our algorithm.
	The resulting function $V(x)$ certifies asymptotic stability of the origin
	for any initial condition in the box $\mathcal{B}=\{x\in\reals^2 \mid
	\|x\|_\infty \leq 10\}$.}
    \label{fig:lyap}
    %\vspace{-5ex}
\end{figure}

\begin{figure}[htpb]
    \centering
    \includegraphics[trim=10 10 20 20,clip,width=0.75\textwidth]{code/fig/trajectories}
	\caption{Lyapunov function value for trajectories starting at 200 random
	points in the box $\mathcal{B}=\{x\in\reals^2 \mid \|x\|_\infty \leq 10\}$.}
    \label{fig:trajectories}
\end{figure}

%%%%%%%%%%%%%%%%%%%%%%%%%%%%%%%%%%%%%%%%%%%%%%%%%%%%%%%%%%%%%%%%%%%%%%%%%%%%%%%%
\subsection{Verification of $\lambda$-contractive dynamical systems}
In this example, we verify the $\lambda$-contractiveness of a given closed loop
dynamical system, following the example in~\cite{Milani:1999}.
\begin{definition}[$\lambda$-contractive]
	Let $G$ be a matrix in $\reals^{m\times n}$ and $w$ a vector in $\reals^m$. 
	Define the polyhedron $S(G,w) = \left\lbrace x \mid Gx \preceq w \right\rbrace
	$. The set $S(G,w)$ is \emph{$\lambda$-contractive} with respect to the
	system~\eqref{eq:dtsys} 
	if there is a real $0<\lambda<1$ such that $x(t+1) \in
	S(G,w\epsilon\lambda)$ for all $0< \epsilon \leq 1$ and all $x(t)\in
	S(G,w\epsilon)$, \ie
	\begin{equation*}
		\Phi: \quad 
		\forall x .\, 
		\forall \epsilon .\,
		[(0<\epsilon\leq 1) \wedge (x\in S(G,w\epsilon))]
		\rightarrow (\amn(x)\in S(G,w\lambda \epsilon)).
	\end{equation*}
\end{definition}

Consider the closed loop system with state-feedback control
represented by the state equations
\begin{equation}
	\begin{aligned}
		x(t+1) &= Ax(t) + Bu(t), \\
		u(t) &= \text{sat}(Fx(t)),
	\end{aligned}
	\label{eq:SysEq} 
\end{equation}
where $x(t) \in \reals^n$, $u \in \reals$, $A \in \reals^{n\times n}$, $B \in
\reals^{n}$, and
\begin{align*}
\text{sat}(Fx) &= \begin{cases} -u^\text{min} &\text{if} ~ Fx < -u^\text{min}, 
\\Fx &\text{if} ~ -u^\text{min} \leq Fx \leq u^\text{max},
\\u^\text{max} &\text{if} ~ Fx > u^\text{max}.
\end{cases}
\end{align*}
In other words, the autonomous system has $\amn(x(t)) = Ax(t) + B\sat(Fx(t))$.

Given $G\in \reals^{r\times n}$ and $w \succ 0 \in \reals^r$, we can test the
$\lambda$-contractiveness of $S(G,w)$ by defining two AMN functions $V_1,V_2
: \reals^n \rightarrow \reals$, whose zero-sublevel sets represent 
the region inside the polyhedron $S(G,w)$.  We rearrange the inequality
$Gx\preceq w$, where $g_i$ is the $i$-th row of $G$, to give
\begin{align*}
	V_1(x) &= \max_i (g_ix - \epsilon w_i),\\
	V_2(\amn(x)) &= \max_i (g_i \amn(x) - \epsilon \lambda w_i).
\end{align*}

The negative of the condition $\Phi$ is
\begin{align*}
\label{eq:SatSpec} \neg \Phi &= \exists x,\epsilon.\, 
	x\in S(G,w\epsilon)\wedge 0<\epsilon\leq 1
	\wedge \neg(\amn(x) \in S(G,w\epsilon \lambda) ) \\
	&= \exists x,\epsilon.\,  (V_1(x)\leq 0) \wedge (V_2(x) > 0 ) \wedge (0< \epsilon \leq 1)
\end{align*}
If there exist $x$ and $\epsilon$ that satisfy $\neg\Phi$, then the
polyhedron $S(G,w)$ is not $\lambda$-contractive with respect to
system~\eqref{eq:SysEq}. However, if $\neg \Phi$ is \textsc{unsat}, then we can
say that the polyhedron $S(G,w)$ is $\lambda$-contractive within the region of
nonlinear behavior of system~\eqref{eq:SysEq}.

The example polyhedron $S(G,w)$ in~\cite{Milani:1999} can be verified as
$\lambda$-contractive:
\[
	G = \begin{bmatrix}
	0.2888 & -1.8350 \\
	0.9650 & -2.0576 \\
	1.0008 & 1.7891 \\
	1.5951 & -1.9866 \\
	2.0707 & -2.0590 \\
	-1.4970 & -1.5864 \\
	-0.2888 & 1.8350 \\
	-0.9650 & 2.0576 \\
	-1.0008 & -1.7891 \\
	- 1.5951 & 1.9866 \\
	1.4970 & 2.0590 \\
	- 2.0707 & 1.5864
	\end{bmatrix},
	\quad
	w = \begin{bmatrix}
	35.4375 \\
	48.2116 \\
	48.1152 \\
	62.5184 \\
	62.3934 \\
	76.2996 \\ 
	35.4375 \\
	48.2116 \\
	48.1152 \\
	62.5184 \\
	62.3934 \\
	76.2996 
	\end{bmatrix}.
\]
However, if we scale $w$ by $\delta = 1.01$, then we can find a counterexample,
\[
	x(t) = \begin{bmatrix} 
	38.9278177 \\ 2.27698913
	\end{bmatrix},
	\quad
	x(t+1) = \begin{bmatrix} 32.28074873 \\ -5.83874012	\end{bmatrix},
	\quad
	\epsilon = 0.99914198.
\]

The reference~\cite{Milani:1999} provides a set of independent linear programs,
a solution of which would synthesize the polyhedron $S(G,w)$, which is
$\lambda$-contractive with respect to closed loop system.  With AMNs, we have
an alternate, fully automatic method to validate piecewise affine Lyapunov
functions for stability analysis of LTI discrete time systems with saturated
closed loop control inputs.

%%%%%%%%%%%%%%%%%%%%%%%%%%%%%%%%%%%%%%%%%%%%%%%%%%%%%%%%%%%%%%%%%%%%%%%%%%%%%%%%%%%%%%%%%%%%%%%%%%%
\subsection{Nonconvex optimal control}
This example analyzes finite horizon optimal control problems with affine (both
convex and nonconvex) control constraints.  Consider the piecewise affine
discrete linear system
\[
	x(t+1) = Ax(t) + Bu(t),
\]
where
\[
	A = 
	\begin{bmatrix} 
			1 & \delta t \\ 
			0 & 1 
	\end{bmatrix},
	\quad
	B = 
	\begin{bmatrix}
		\frac{\delta t^2}{2m} \\
		\frac{\delta t}{m} 
	\end{bmatrix}, 
\]
and initial and goal states
\[
	\begin{bmatrix} 
		x_1(0)\\
		x_2(0)
	\end{bmatrix} = 
	\begin{bmatrix} 
		0\\
		0
	\end{bmatrix},
	\quad
	\begin{bmatrix} 
		x_1(N)\\
		x_2(N)
	\end{bmatrix} = 
	\begin{bmatrix} 
		1\\
		0
	\end{bmatrix}.
\]
The control magnitude is bounded above and below by
\[
	0.2/T^\text{tot} \leq \|u(t)\|_1 \leq 1/T^\text{tot}, \quad 
	T^\text{tot}=7.5\text{s}.
\]
Note that the upper bound is a convex constraint, while the lower bound is not.
The goal is to reach the goal state while minimizing the objective
\[
	J = \sum_{t=0}^{N-1} \|u(t)\|_1.
\]
We solve the (nonconvex) optimization problem
\[
	\begin{array}{ll}
		\mbox{minimize}   & \sum_{t=0}^{N-1} \|u(t)\|_1 \\
		\mbox{subject to} 
			& x(t+1) = A x(t) + Bu(t), \quad t=0,\ldots,N-1\\
			%& 0.2 \leq \|u(t)\| \leq 1, \quad t=0,\ldots,N\\
			& 0.2/T^\text{tot} \leq \|u(t)\| \leq 1/T^\text{tot}, \quad t=0,\ldots,N\\
			& x(0) = (0, 0), \quad x(N) = (1, 0).
	\end{array}
\]
with \amnet. Figure~\ref{fig:noncvx_control} summarizes the results. Note that
the lower bounds are enforced for the nonconvex optimization problem, leading
to a control schedule that obeys the lower bound on the control magnitude by
never coasting with $u(t)=0$; such a control schedule is not obvious from the
optimal schedule for the convex problem, but it is provably optimal.
%\begin{figure}[h!]
%	\centering
%	\begin{subfigure}{0.45\textwidth}
%		%\includegraphics[width = \textwidth]{fig/1D-CVX.svg}
%		\includegraphics[width=\textwidth]{fig/1D-CVX}
%		\caption{Convex (\textsc{cvxpy})}
%	\end{subfigure}~
%	\begin{subfigure}{0.45\textwidth}
%		%\includegraphics[width = \textwidth]{fig/1D-AMN.svg}
%		\includegraphics[width=\textwidth]{fig/1D-AMN}
%		\caption{Combined (\amnet)}
%	\end{subfigure}
%	\caption{Optimal control trajectories for the convex problem (upper bound
%	only) using \textsc{cvxpy}~\cite{cvxpy} (left) and  combined (upper and
%	lower bounds) using \amnet\ (right).} 
%	\label{fig:noncvx_control}
%\end{figure}

\begin{figure}[tbp]
	\centering
	\begin{subfigure}{0.95\textwidth}
		\includegraphics[width=\textwidth]{fig/1D-CVXshaded}
		\caption{Convex (\textsc{cvxpy})}
		\label{fig:noncvx_control_a}
	\end{subfigure}\\
	\begin{subfigure}{0.95\textwidth}
		\includegraphics[width=\textwidth]{fig/1D-AMNshaded}
		\caption{Combined (\amnet)}
		\label{fig:noncvx_control_b}
	\end{subfigure}
	\caption{Illustration of nonconvex optimization capabilities of \amnet.
	Optimal control trajectories for the convex problem (upper control bound
	only) using \textsc{cvxpy}~\cite{cvxpy} 
	and  combined (upper and lower control bounds) using \amnet.
	Shading indicates disallowed control inputs.} 
	\label{fig:noncvx_control}
\end{figure}

\clearpage

%\subsubsection{Minimum fuel planetary landing}
%We examine the planetary, soft landing problem presented in \cite{Harris:2014}. The equations of motion are:
%\begin{align}
%\ddot{x}(t) = u(t)
%\end{align}
%
%Where first three components $x_1$, $x_2$ and $x_3$ are the range, altitude and cross range respectively. The next three components are the their rates.
%
%\begin{align}
%x(0) &= \begin{bmatrix} 400 & 400 & 300 \end{bmatrix} ~ m,\\
%\dot{x}(0) &= -\begin{bmatrix} 10 & 10 & 75 \end{bmatrix} ~m/s,\\
%x(t_f) &= \begin{bmatrix} 0 & 0 & 0\end{bmatrix}~m,\\
%\dot{x}(t_f) &= \begin{bmatrix} 0 & 0 & 0 \end{bmatrix}~m/s. 
%\end{align}
%
%The discretized forms of these equations:
%\begin{align}
%A = \begin{bmatrix} I_{3\times 3}  & I_{3\times 3}* dt \\ 0_{3\times 3} & I_{3\times 3} \end{bmatrix} \quad & B = \begin{bmatrix} I_{3\times 3} * \frac{dt^2}{2} \\ I_{3\times 3} * dt \end{bmatrix}
%\end{align}
%
%A further constraint of a 45 degree approach is specified by:
%\begin{align}
%	x_1(t) - x_2(t) = 0.
%\end{align}
%
%The goal is to reach the goal state and minimize the fuel consumption:
%\begin{align}
%\min J = \int_{0}^{t_f} \| u(t) \|_1 dt.
%\end{align}
%
%The control magnitude is bounded above and below:
%\begin{align}
%2 \leq \| u(t) \|_1 \leq 10 m/s^2
%\end{align}
%
%
%\begin{figure}[h!]
%	\centering
%	\begin{subfigure}{0.45\textwidth}
%		%\includegraphics[width = \textwidth]{fig/3D-CVX.svg}
%		\includegraphics[width=\textwidth]{fig/3D-CVX}
%		\caption{Convex (cvxpy)}
%	\end{subfigure}~
%	\begin{subfigure}{0.45\textwidth}
%		%\includegraphics[width = \textwidth]{fig/3D-AMN.svg}
%		\includegraphics[width=\textwidth]{fig/3D-AMN}
%		\caption{Combined (AMN)}
%	\end{subfigure}
%	\caption{Optimal control trajectories for the convex problem (upper bound only) using cvxpy (left) and  combined (upper and lower bounds) using AMN (right).}
%\end{figure}


%%%%%%%%%%%%%%%%%%%%%%%%%%%%%%%%%%%%%%%%%%%%%%%%%%%%%%%%%%%%%%%%%%%%%%%%%%%%%%%%
\subsection{Characterizing classifier robustness}
To illustrate the use of \acsp{amn} in other neural network applications, we
used \textsc{TensorFlow} to train a simple and small neural network classifier
on the popular MNIST handwritten digit
dataset~\cite{TensorFlow:2015,LeCun:1998}.  To make conversion and verification
of the corresponding \acs{amn}
manageable, the dimension of the training data was first reduced from 784 
($28\times28$ handwritten digit images) to 40 using PCA whitening.
The resulting classifier was automatically converted to an \acs{amn}, and its
associated input-output relationship encoded as SMT constraints with our
accompanying \textsc{Amnet} modeling
toolbox.
The baseline, 20-unit hidden layer \acs{amn} 
$\amn^\mathsc{mnist}:\reals^{40}\to\reals^{10}$,
with ReLU nonlinearities between the
layers, achieved a classification rate of 
0.8736
on the test data set.

\begin{figure}[htbp]
    \centering
    \includegraphics{fig/mp/amnet-1}
    \caption{Using \textsc{Amnet} to convert a NN to an~\acs{amn}.}
    \label{fig:amnet1}
\end{figure}


The procedure to convert a NN with piecewise affine nonlinearities to an
\acs{amn} is entirely mechanical (\S{}\ref{sec:encoding}), and importantly,
indifferent to the specific optimization algorithm (\eg, gradient descent,
batching, regularization) used to train the original classifier. Therefore,
once the weights and biases of the NN are encoded in SMT, formal properties
of the original NN can be readily verified using our toolbox, see
Figure~\ref{fig:amnet1}.

A perturbation $\varepsilon = (\varepsilon_1, \ldots,
\varepsilon_5, 0, \ldots, 0) \in \reals^{40}$ was created to act on the 5 most significant
components of the dimensionality reduced input, $X_\mathrm{amn} =
X_\mathrm{pca} + \varepsilon$.
Each dimension of the perturbation was constrained to $-3 \leq \varepsilon_i
\leq 3$.  
The output layer of $\amn$ was constrained to produce a misclassification of
`5' by introducing the equality constraint
\[
\max(\amn^\mathsc{mnist}(X_\mathrm{amn})) =
\amn^\mathsc{mnist}(X_\mathrm{amn})_6.
\]
on the final classification layer.
The \textsc{z3} SMT solver was used to find a solution to these constraints,
shown below.  The resulting perturbed image was recovered by $X^T =
V^{T}X_\mathrm{amn}$.

\begin{figure}[htbp]
	\begin{subfigure}{.30\textwidth}
		\centering
		\includegraphics[width=\textwidth]{fig/original_reduced_7}
		\caption{Original image; classifies as `7'.}
		\label{fig:HOG_resolution}
	\end{subfigure}
	~
	\begin{subfigure}{.30\textwidth}
		\centering
		\includegraphics[width=\textwidth]{fig/diff}
		\caption{Perturbation visualization.}
		\label{fig:block_overlap}
	\end{subfigure}
	~
	\begin{subfigure}{.30\textwidth}
		\centering
		\includegraphics[width=\textwidth]{fig/smt_7_to_5}
		\caption{Perturbed image; classifies as `5'.}
		\label{fig:Peturbed_image}
	\end{subfigure}
	\caption{Perturbation on MNIST images.}
	\label{fig:MNIST_result}
\end{figure}

