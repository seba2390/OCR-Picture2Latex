\documentclass[aps,prl,notitlepage, superscriptaddress,longbibliography]{revtex4-1}
\pdfoutput=1

\usepackage[utf8]{inputenc}
\usepackage{amsmath,amssymb,amsmath,amsthm,bm,graphicx,xcolor}

\usepackage{empheq,etoolbox}
% Update display of subequation numbering (Xy) > (X.y)
\patchcmd{\subequations}% <cmd>
  {\theparentequation\alph{equation}}% <search>
  {\theparentequation.\alph{equation}}% <replace>
  {}{}% <success><failure>
  
\begin{document}

\author{Johannes Gedeon}
\affiliation{Hannover Centre for Optical Technologies, Institute for Transport and Automation Technology (Faculty of Mechanical Engineering), and Cluster of Excellence PhoenixD,
Leibniz University Hannover, 30167 Hannover, Germany}
\author{Emadeldeen Hassan}
\affiliation{Department of Electronics and Electrical Communications, Menoufia University, Menouf 32952, Egypt}
\affiliation{Department of Applied Physics and Electronics, Umeå University, SE-901 87 Umeå, Sweden}
\author{Antonio Cal{\`a} Lesina}
\affiliation{Hannover Centre for Optical Technologies, Institute for Transport and Automation Technology (Faculty of Mechanical Engineering), and Cluster of Excellence PhoenixD,
Leibniz University Hannover, 30167 Hannover, Germany}


\title{Supplementary information for \\
``Free-form inverse design of arbitrary dispersive materials in nanophotonics''}

\date{\today}
\maketitle

\section{Derivation of the adjoint equations} 
Here, we derive the adjoint equations for density-based topology optimization based on the time domain Maxwell equations and the Complex-conjugate Pole-residue material model with $\exp^{\,j\omega}$ time-dependency
\begin{equation}
\varepsilon_{\alpha \beta}(\omega)=\varepsilon_{\infty, \alpha \beta}+\frac{\sigma_{\alpha \beta}}{j \omega \varepsilon_0}+\sum_{p=1}^{P_{\alpha \beta}}\left(\frac{c_{p, \alpha \beta}}{j \omega-a_{p, \alpha \beta}}+\frac{c_{p, \alpha \beta}^*}{j \omega-a_{p, \alpha \beta}^*}\right),
\end{equation}
where $\varepsilon_{\infty, \alpha \beta}$ is the relative permittivity at infinite frequency, and $\sigma_{\alpha \beta}$ is the static conductivity. The indices $\alpha$ and $\beta$ denote the $x$, $y$ and $z$ component and $*$ represents the complex conjugation. In the following, we denote the design material with index $i=2$ and the background material with index $i=1$. For a given density value $\rho \in [0, 1]$ we apply a linear interpolation of the parameters and complex pole pairs of the following form:
\begin{subequations}\label{Eq:InterpolatedModel}
  \begin{empheq}[]{align}
\varepsilon_{\infty, \alpha \beta}(\rho) &:= (1-\rho)\,\varepsilon_{\infty, \alpha \beta}^{(1)} +  \rho\, \varepsilon_{\infty, \alpha \beta}^{(2)},\\
\sigma_{\alpha \beta}(\rho) &:= (1-\rho)\,\sigma_{\alpha \beta}^{(1)} +  \rho\, \sigma_{\alpha \beta}^{(2)} + \rho\,(1-\rho)\,\gamma,\\
{\textstyle \sum_{{\alpha \beta}}}(\omega, \rho) &:= \sum_{i=1}^{2}\kappa^{(i)}(\rho)\sum_{p=1}^{P_{\alpha \beta}^{(i)}}\left(\frac{c_{p, \alpha \beta}^{(i)}}{j \omega-a_{p, \alpha \beta}^{(i)}}+\frac{c_{p, \alpha \beta}^{(i)*}}{j \omega-a_{p, \alpha \beta}^{(i)*}}\right),
\end{empheq}
\end{subequations}
where $\kappa^{(1)}(\rho): = (1-\rho)$ and $\kappa^{(2)}(\rho): = \rho$. According to these equations, the interpolated relative permittivity can be written as
\begin{equation}
\varepsilon_{\alpha \beta}(\omega, \rho) = \varepsilon_{\infty, \alpha \beta}(\rho) + \frac{\sigma_{\alpha \beta}(\rho)}{j \omega \varepsilon_0} + {\textstyle \sum_{{\alpha \beta}}}(\omega, \rho).
\end{equation}
We assume that the \textit{figure of merit} (or \textit{objective}) we aim to maximize, is a functional of the electric field only, such that the optimization problem can be formulated as follows:
\begin{equation}
\begin{aligned}
& \max _{\rho} F[\mathbf{E}] \\
& \text { s.t. Maxwell's equations,}
\end{aligned}
\end{equation}
where boundary conditions and manufacturability constraints can be included. The functional derivate of the objective $F[\mathbf{E}]$ with respect to the density is
\begin{equation}\label{Eq:ObjectiveDerivative}
\frac{\delta F[\mathbf{E}]}{\delta \rho}=\frac{\delta F[\mathbf{E}]}{\delta \mathbf{E}} \cdot \frac{\mathrm{d} \mathbf{E}}{\mathrm{d} \rho},
\end{equation}
where the first multiplicator on the right hand side denotes the functional derivate of the objective function with respect to the electric field. We denote the derivative of any local function with respect to the density by ``$\mathrm{d}_{\rho}$'' in the following.\par 
We assume a diagonal permittivity tensor and non-magnetic materials in the following. 
We denote the spatial physical domain with $\Omega$, and the time interval as $I=[0,T]$, and consider an excitation of the \textit{forward} system by a pulse injected at $\partial \Omega$ at $t=0$, and vanishing fields for $t\in \partial I$. The full system of Maxwell equations for each spatial component $k\in\{1,2,3\}$ and $\forall (\mathbf{r},t) \in \Omega \times I$ of the forward system reads \cite{material}:
\begin{subequations}\label{Eq:Maxwells2}
  \begin{empheq}[]{align}
-(\nabla \times \mathbf{H})_{k} + \varepsilon_{0}\varepsilon_{\infty, k}\partial_{t}E_{k} + \sigma_{k} E_{k} + 2\sum_{i=1}^{2}\sum_{p=1}^{P_{k}^{(i)}}\kappa^{(i)}\Re\left\{\partial_{t}Q_{p, k}^{(i)}\right\} &= 0,\\[1pt]
\text{For } i =1,2 \text{ and } \forall p \in 1, \ldots, P_k^{(i)}: \partial_{t}Q_{p, k}^{(i)}-a_{p, k}^{(i)}Q_{p, k}^{(i)}- \varepsilon_{0}c_{p, k}^{(i)}E_{k} &= 0.\\[8pt]
\mu_0 \partial_t H_k+(\nabla \times \mathbf{E})_k &=0.
\end{empheq}
\end{subequations}
We emphasize our chosen material interpolation in Eq.~(\ref{Eq:InterpolatedModel}), the parameters $a_{p, k}^{(i)}$ and $c_{p, k}^{(i)}$ do \textit{not} depend on the density itself. In contrast, all fields depend implicitly on $\rho$, and the local functions interpolating the material such as $\varepsilon_{\infty, k}$, $\sigma_{k}$ and $\kappa^{(i)}$ depend directly on the spatial density distribution.\par
We define adjoint fields $\mathbf{\tilde{E}}$, $\mathbf{\tilde{H}}$ and $\tilde{Q}_{p, k}^{(i)}$, for $i =1,2$, $k\in\{1,2,3\}$ and $\forall p \in 1, \ldots, P_k^{(i)}$,
sharing the same properties as the forward fields, i.e. the electric and magnetic adjoint fields are real, and the adjoint auxiliary fields are allowed to be complex. 
We derivate the system of Eqs.~(\ref{Eq:Maxwells2}) with respect to $\rho$, and multiply Eq.~(\ref{Eq:Maxwells2}a) by $\tilde{E_k}$, Eq.~(\ref{Eq:Maxwells2}c) by $\tilde{H_k}$, and each of Eqs.~(\ref{Eq:Maxwells2}b) by a corresponding term $-\kappa^{(i)}\frac{\partial_{t}\tilde{Q}_{p, k}^{(i)}}{\varepsilon_{0} c_{p, k}^{(i)}}$, assuming a non-vanishing parameters $c_{p, k}^{(i)} \neq 0$. Furthermore, we sum over the spatial components and get
\begin{equation}\label{Eq:MaxwellsDerivE}
\begin{split}
-\sum_{k=1}^{3}\left\{(\nabla \times \mathrm{d}_{\rho}\mathbf{H})_{k}\tilde{E}_k 
+ \varepsilon_{0}(\mathrm{d}_{\rho}\varepsilon_{\infty, k})\tilde{E}_k \partial_{t}E_{k}
+ \varepsilon_{0}\varepsilon_{\infty, k}\tilde{E}_k \partial_{t}(\mathrm{d}_{\rho}E_{k})\right\} \;&\\[8pt]
+ \sum_{k=1}^{3}\left\{(\mathrm{d}_{\rho}\sigma_{k}) \tilde{E}_k E_{k}
+ \sigma_{k} \tilde{E}_k (\mathrm{d}_{\rho}E_{k})\right\}\;&\\[8pt]
+ \sum_{k=1}^{3}\sum_{i=1}^{2}\sum_{p=1}^{P_{k}^{(i)}}2(\mathrm{d}_{\rho}\kappa^{(i)})\tilde{E}_{k}\Re\left\{\partial_{t}Q_{p, k}^{(i)}\right\}  
+ \sum_{k=1}^{3}\sum_{i=1}^{2}\sum_{p=1}^{P_{k}^{(i)}}2\kappa^{(i)}\tilde{E}_{k}\Re\left\{\partial_{t}(\mathrm{d}_{\rho}Q_{p, k}^{(i)})\right\} &= 0,
\end{split}
\end{equation}
\begin{equation}\label{Eq:MaxwellsDerivH}
\begin{split}
\phantom{\;xxxxxxxxxxxxxxxxxxxxxxxxxxxxxxx}\sum_{k=1}^{3}\left\{\mu_0 \tilde{H_k}\partial_t \mathrm{d}_{\rho}H_k+ \tilde{H_k}(\nabla \times \mathrm{d}_{\rho}\mathbf{E})_k\right\} &=0.\\
\end{split}
\end{equation}
And for $i =1,2$ and $\forall p \in 1, \ldots, P_k^{(i)}$ we obtain:
\begin{equation}\label{Eq:MaxwellsDerivQ}
\begin{split}
\sum_{k=1}^{3}\left\{\frac{-\kappa^{(i)}}{\varepsilon_{0} c_{p, k}^{(i)}}\partial_{t}\tilde{Q}_{p, k}^{(i)}\partial_{t}(\mathrm{d}_{\rho}Q_{p, k}^{(i)})
+\frac{\kappa^{(i)} a_{p, k}^{(i)}}{\varepsilon_{0} c_{p, k}^{(i)}}\partial_{t}\tilde{Q}_{p, k}^{(i)} (\mathrm{d}_{\rho}Q_{p, k}^{(i)})
+\kappa^{(i)}\partial_{t}\tilde{Q}_{p, k}^{(i)}(\mathrm{d}_{\rho}E_{k})\right\} &= 0.
\end{split}
\end{equation}
By addition of the the complex conjugates (denoted as ``c.c.'') of Eqs.~(\ref{Eq:MaxwellsDerivQ}), and summing over the indices $i$ and $p$, we reduce the equations above to:
\begin{equation}\label{Eq:Qsum}
\begin{split}
\sum_{k=1}^{3}\sum_{i=1}^{2}\sum_{p=1}^{P_{k}^{(i)}}\left\{\frac{-\kappa^{(i)}}{\varepsilon_{0} c_{p, k}^{(i)}}\partial_{t}\tilde{Q}_{p, k}^{(i)}\partial_{t}(\mathrm{d}_{\rho}Q_{p, k}^{(i)})
+\frac{\kappa^{(i)} a_{p, k}^{(i)}}{\varepsilon_{0} c_{p, k}^{(i)}}\partial_{t}\tilde{Q}_{p, k}^{(i)} (\mathrm{d}_{\rho}Q_{p, k}^{(i)})+\mathrm{c.c.}\right\} \\
+\sum_{k=1}^{3}\sum_{i=1}^{2}\sum_{p=1}^{P_{k}^{(i)}}2\kappa^{(i)}\Re\left\{\partial_{t}\tilde{Q}_{p, k}^{(i)}\right\} (\mathrm{d}_{\rho}E_{k})  &= 0,
\end{split}
\end{equation}
where we used the identities $2\Re\left\{\partial_{t}\tilde{Q}_{p, k}^{(i)}\right\} = \partial_{t}\tilde{Q}_{p, k}^{(i)} + \partial_{t}\tilde{Q}_{p, k}^{*(i)}$, and $\mathrm{d}_{\rho}E^{*}_{k} = \mathrm{d}_{\rho}E_{k}$.\par 
For a better readability, we will waive the symbol ``$\mathrm{d}^3r\,\mathrm{d}t$'' denoting the differential of the variable $(\mathbf{r}, t)$ in all following integral expressions. Integrating over space and time $\Omega \times I$, considering $\rho$ not to be time-dependent, and applying integration by parts in Eqs.~(\ref{Eq:MaxwellsDerivE}) and (\ref{Eq:MaxwellsDerivH}), while taking the imposed boundary conditions into account, leads to
\begin{equation} \label{Eq:Integr1}
\begin{split}
\int_{\Omega}\rlap{$\overbrace{\phantom{\left.\sum_{k=1}^{3}\varepsilon_{0}(\mathrm{d}_{\rho}\varepsilon_{\infty, k})\tilde{E}_k E_{k}\right|_0 ^T }}^{=\;0}$}\left.\sum_{k=1}^{3}\varepsilon_{0}(\mathrm{d}_{\rho}\varepsilon_{\infty, k})\tilde{E}_k E_{k}\right|_0 ^T 
&- \int_{\Omega}\int_{I}\sum_{k=1}^{3}\varepsilon_{0}(\mathrm{d}_{\rho}\varepsilon_{\infty, k})\partial_{t}\tilde{E}_k E_{k}\\
+\int_{\Omega}\rlap{$\overbrace{\phantom{\left.\sum_{k=1}^{3}\varepsilon_{0}\varepsilon_{\infty, k}\tilde{E}_k (\mathrm{d}_{\rho}E_{k})\right|_0 ^T}}^{=\;0}$}\left.\sum_{k=1}^{3}\varepsilon_{0}\varepsilon_{\infty, k}\tilde{E}_k (\mathrm{d}_{\rho}E_{k})\right|_0 ^T 
&- \int_{\Omega}\int_{I}\sum_{k=1}^{3}\varepsilon_{0}\varepsilon_{\infty, k}\partial_{t}\tilde{E}_k (\mathrm{d}_{\rho}E_{k})\\
&+ \int_{\Omega}\int_{I} \sum_{k=1}^{3}(\mathrm{d}_{\rho}\sigma_{k}) \tilde{E}_k E_{k}
+ \int_{\Omega}\int_{I} \sum_{k=1}^{3}\sigma_{k} \tilde{E}_k (\mathrm{d}_{\rho}E_{k})\\
+\int_{\Omega}\rlap{$\overbrace{\phantom{\left.\sum_{k=1}^{3}\sum_{i=1}^{2}\sum_{p=1}^{P_{k}^{(i)}}2(\mathrm{d}_{\rho}\kappa^{(i)})\tilde{E}_{k}\Re\left\{Q_{p, k}^{(i)}\right\}\right|_0 ^T}}^{=\;0}$}\left.\sum_{k=1}^{3}\sum_{i=1}^{2}\sum_{p=1}^{P_{k}^{(i)}}2(\mathrm{d}_{\rho}\kappa^{(i)})\tilde{E}_{k}\Re\left\{Q_{p, k}^{(i)}\right\}\right|_0 ^T 
&- \int_{\Omega}\int_{I} \sum_{k=1}^{3}\sum_{i=1}^{2}\sum_{p=1}^{P_{k}^{(i)}}2(\mathrm{d}_{\rho}\kappa^{(i)})\partial_{t}\tilde{E}_{k}\Re\left\{Q_{p, k}^{(i)}\right\}\\
+\int_{\Omega}\rlap{$\overbrace{\phantom{\left.\sum_{k=1}^{3}\sum_{i=1}^{2}\sum_{p=1}^{P_{k}^{(i)}}2\kappa^{(i)}\tilde{E}_{k}\Re\left\{(\mathrm{d}_{\rho}Q_{p, k}^{(i)})\right\}\right|_0 ^T}}^{=\;0}$}\left.\sum_{k=1}^{3}\sum_{i=1}^{2}\sum_{p=1}^{P_{k}^{(i)}}2\kappa^{(i)}\tilde{E}_{k}\Re\left\{(\mathrm{d}_{\rho}Q_{p, k}^{(i)})\right\}\right|_0 ^T 
&- \int_{\Omega}\int_{I}\sum_{k=1}^{3}  \sum_{i=1}^{2}\sum_{p=1}^{P_{k}^{(i)}}2\kappa^{(i)}\partial_{t}\tilde{E}_{k}\Re\left\{(\mathrm{d}_{\rho}Q_{p, k}^{(i)})\right\}\\
-\int_{\Omega}\rlap{$\overbrace{\phantom{\left.\sum_{k=1}^{3}\partial_{k}(\mathrm{d}_{\rho}\mathbf{H} \times \tilde{\mathbf{E}} )_{k}\right|_0 ^T}}^{=\;0}$}\left.\sum_{k=1}^{3}\partial_{k}(\mathrm{d}_{\rho}\mathbf{H} \times \tilde{\mathbf{E}} )_{k}\right|_0 ^T 
&- \int_{\Omega}\int_{I}\sum_{k=1}^{3} (\nabla \times \mathbf{\tilde{E}})_{k} (\mathrm{d}_{\rho} H_{k})\\
+\int_{\Omega}\rlap{$\overbrace{\phantom{\left.\sum_{k=1}^{3}\mu_0 \tilde{H_k} \mathrm{d}_{\rho}H_k\right|_0 ^T}}^{=\;0}$}\left.\sum_{k=1}^{3}\mu_0 \tilde{H_k} \mathrm{d}_{\rho}H_k\right|_0 ^T 
&- \int_{\Omega}\int_{I} \sum_{k=1}^{3} \mu_0 \partial_{t}\tilde{H_k} (\mathrm{d}_{\rho}H_k) \\
+\int_{\Omega}\rlap{$\overbrace{\phantom{\left.\sum_{k=1}^{3}\partial_{k}(\mathbf{\tilde{H}} \times \mathrm{d}_{\rho}\mathbf{E})_k\right|_0 ^T}}^{=\;0}$}\left.\sum_{k=1}^{3}\partial_{k}(\mathbf{\tilde{H}} \times \mathrm{d}_{\rho}\mathbf{E})_k\right|_0 ^T 
&+ \int_{\Omega}\int_{I} \sum_{k=1}^{3} (\nabla \times \mathbf{\tilde{H}})_{k}(\mathrm{d}_{\rho}E_k)\\[8pt]
&=0.
\end{split}
\end{equation}
We do the same  for Eq.~(\ref{Eq:Qsum}) and obtain:
\begin{equation} \label{Eq:Integr2}
\begin{split}
&\int_{\Omega}\rlap{$\overbrace{\phantom{\left.\sum_{k=1}^{3}\sum_{i=1}^{2}\sum_{p=1}^{P_{k}^{(i)}}\left(\frac{-\kappa^{(i)}}{\varepsilon_{0} c_{p, k}^{(i)}}\partial_{t}\tilde{Q}_{p, k}^{(i)}(\mathrm{d}_{\rho}Q_{p, k}^{(i)}) + \mathrm{c.c}\right)\right|_0 ^T}}^{=\;0}$}\left.\sum_{k=1}^{3}\sum_{i=1}^{2}\sum_{p=1}^{P_{k}^{(i)}}\left\{\frac{-\kappa^{(i)}}{\varepsilon_{0} c_{p, k}^{(i)}}\partial_{t}\tilde{Q}_{p, k}^{(i)}(\mathrm{d}_{\rho}Q_{p, k}^{(i)})+ \mathrm{c.c}\right\}\right|_0 ^T \\
+ &\int_{\Omega}\int_{I} \sum_{k=1}^{3} \sum_{i=1}^{2}\sum_{p=1}^{P_{k}^{(i)}}\left\{\frac{\kappa^{(i)}}{\varepsilon_{0} c_{p, k}^{(i)}}\partial^{2}_{t}\tilde{Q}_{p, k}^{(i)}(\mathrm{d}_{\rho}Q_{p, k}^{(i)}) + \frac{\kappa^{(i)} a_{p, k}^{(i)}}{\varepsilon_{0} c_{p, k}^{(i)}}\partial_{t}\tilde{Q}_{p, k}^{(i)} (\mathrm{d}_{\rho}Q_{p, k}^{(i)}) + \mathrm{c.c.}\right\}\\
+&\int_{\Omega}\int_{I} \sum_{k=1}^{3}\sum_{i=1}^{2}\sum_{p=1}^{P_{k}^{(i)}}2\kappa^{(i)}\Re\left\{\partial_{t}\tilde{Q}_{p, k}^{(i)}\right\} (\mathrm{d}_{\rho}E_{k})\\[8pt]
&=0.
\end{split}
\end{equation}
By adding Eqs.~(\ref{Eq:Integr1}) and~(\ref{Eq:Integr2}), and the integral of the functional derivative in Eq.~(\ref{Eq:ObjectiveDerivative}) over $\Omega \times I$, we obtain
\begin{equation}\label{Eq:integralRearanged}
\begin{split}
&\int_{\Omega}\int_{I}\sum_{k=1}^{3} \left\{\left( (\nabla \times \mathbf{\tilde{H}})_{k} - \varepsilon_{0}\varepsilon_{\infty, k}\partial_{t}\tilde{E}_k  + \sigma_{k} \tilde{E}_k + 2\sum_{i=1}^{2}\sum_{p=1}^{P_{k}^{(i)}}\kappa^{(i)}\Re\left\{\partial_{t}\tilde{Q}_{p, k}^{(i)}\right\} - \left(\frac{\delta F[\mathbf{E}]}{\delta \mathbf{E}}\right)_{k}\right)           
(\mathrm{d}_{\rho}E_k)\right\}\\
 +&\int_{\Omega}\int_{I} \sum_{k=1}^{3} \sum_{i=1}^{2}\sum_{p=1}^{P_{k}^{(i)}}\left\{\left(\frac{\kappa^{(i)}}{\varepsilon_{0} c_{p, k}^{(i)}}\partial^{2}_{t}\tilde{Q}_{p, k}^{(i)} + \frac{\kappa^{(i)} a_{p, k}^{(i)}}{\varepsilon_{0} c_{p, k}^{(i)}}\partial_{t}\tilde{Q}_{p, k}^{(i)} - \kappa^{(i)}\partial_{t}\tilde{E}_{k}\right)(\mathrm{d}_{\rho}Q_{p, k}^{(i)}) + \mathrm{c.c.}\right\}\\
 +&\int_{\Omega}\int_{I} \sum_{k=1}^{3}\left\{\left( - \mu_0 \partial_{t}\tilde{H_k} - (\nabla \times \mathbf{\tilde{E}})_{k} \right)(\mathrm{d}_{\rho}H_k)\right\}\\[8pt]
 -&\int_{\Omega}\int_{I}\frac{\delta F[\mathbf{E}]}{\delta \rho} + \int_{\Omega}\int_{I}\sum_{k=1}^{3}\left\{ \varepsilon_{0}(\mathrm{d}_{\rho}\varepsilon_{\infty, k})\partial_{t}\tilde{E}_k E_{k} - (\mathrm{d}_{\rho}\sigma_{k}) \tilde{E}_k E_{k} + 2\;\sum_{i=1}^{2}\sum_{p=1}^{P_{k}^{(i)}}(\mathrm{d}_{\rho}\kappa^{(i)})\partial_{t}\tilde{E}_{k}\Re\left\{Q_{p, k}^{(i)}\right\}\right\}\\[8pt]
 &=0.
\end{split}
\end{equation}
We note that the Eq.~(\ref{Eq:integralRearanged}) is satisfied, if the following equations hold $\forall (\mathbf{r},t) \in \Omega \times I$ and each spatial component $k \in \{1,2,3\}$:
\begin{subequations}\label{Eq:AlmostAdjoint}
  \begin{empheq}[]{align}
(\nabla \times \mathbf{\tilde{H}})_{k} - \varepsilon_{0}\varepsilon_{\infty, k}\partial_{t}\tilde{E}_k  + \sigma_{k} \tilde{E}_k + 2\sum_{i=1}^{2}\sum_{p=1}^{P_{k}^{(i)}}\kappa^{(i)}\Re\left\{\partial_{t}\tilde{Q}_{p, k}^{(i)}\right\}&= \left(\frac{\delta F[\mathbf{E}]}{\delta \mathbf{E}}\right)_{k},\\[1pt]
\text{For } i =1,2 \text{ and } \forall p \in 1, \ldots, P_k^{(i)}: \partial_{t}\tilde{Q}_{p, k}^{(i)}+a_{p, k}^{(i)}\tilde{Q}_{p, k}^{(i)}- \varepsilon_{0}c_{p, k}^{(i)}\tilde{E}_{k} &= 0,\\[8pt]
\mu_0 \partial_{t}\tilde{H_k} +(\nabla \times \mathbf{\tilde{E}})_{k} &=0,
\end{empheq}
\end{subequations}
and if $\forall \mathbf{r}\in \Omega$ the gradient of the objective $\nabla_{\rho}F[\mathbf{E}]$ defined  as
\begin{equation}\label{Eq:IntegralKernel}
\begin{split}
\nabla_{\rho}F[\mathbf{E}]:=\int_{I}\frac{\delta F[\mathbf{E}]}{\delta \rho},
\end{split}
\end{equation}
satisfies the equation
\begin{equation}\label{Eq:Gradients2}
\begin{split}
\nabla_{\rho}F[\mathbf{E}]=&\phantom{+}\int_{I}\sum_{k=1}^{3}\varepsilon_{0}(\mathrm{d}_{\rho}\varepsilon_{\infty, k})\partial_{t}\tilde{E}_k E_{k} \\
&-\int_{I}\sum_{k=1}^{3}(\mathrm{d}_{\rho}\sigma_{k}) \tilde{E}_k E_{k} \\
 &+\int_{I}\sum_{k=1}^{3}\sum_{i=1}^{2}\sum_{p=1}^{P_{k}^{(i)}}2(\mathrm{d}_{\rho}\kappa^{(i)})\partial_{t}\tilde{E}_{k}\Re\left\{Q_{p, k}^{(i)}\right\}.
\end{split}
\end{equation}
Now, we perform transformations of the fields in Eqs.~(\ref{Eq:AlmostAdjoint}) to obtain a system of Maxwell equations for the adjoint system. First, we reverse the time and change the sign of the magnetic field $\mathbf{\tilde{H}}$  and the currents $\tilde{Q}_{p, k}^{(i)}$ accordingly, i.e.
\begin{subequations}\label{Eq:Tranformations}
  \begin{empheq}[]{align}
  \mathbf{E}(t)&\rightarrow \mathbf{E}(\tau)\\
\mathbf{\tilde{H}}(t)&\rightarrow-\mathbf{\tilde{H}}(\tau),\\
\tilde{Q}_{p, k}^{(i)}(t)&\rightarrow-\tilde{Q}_{p, k}^{(i)}(\tau), \quad \text{for } k\in\{1,2,3\},\;i =1,2 \text{ and } \forall p \in 1, \ldots, P_k^{(i)}.
\end{empheq}
\end{subequations}
where $\tau:= T - t$ denotes the time-reversed variable. Furthermore, we require vanishing fields for $\tau=0$. If we now apply the chain rule for the time derivatives of all time reversed functions, we finally obtain the adjoint system which holds $\forall (\mathbf{r}, \tau) \in \Omega \times [0, T]$,
\begin{subequations}\label{Eq:AdjointSystem}
  \begin{empheq}[]{align}
-(\nabla \times \mathbf{\tilde{H}})_{k} + \varepsilon_{0}\varepsilon_{\infty, k}\partial_{\tau}\tilde{E}_k  + \sigma_{k} \tilde{E}_k + 2\sum_{i=1}^{2}\sum_{p=1}^{P_{k}^{(i)}}\kappa^{(i)}\Re\left\{\partial_{\tau}\tilde{Q}_{p, k}^{(i)}\right\}&= \left(\overleftarrow{\frac{\delta F[\mathbf{E}]}{\delta \mathbf{E}}}\right)_{k},\\[1pt]
\text{For } i =1,2 \text{ and } \forall p \in 1, \ldots, P_k^{(i)}: \partial_{\tau}\tilde{Q}_{p, k}^{(i)}-a_{p, k}^{(i)}\tilde{Q}_{p, k}^{(i)}- \varepsilon_{0}c_{p, k}^{(i)}\tilde{E}_{k} &= 0,\\[8pt]
\mu_0 \partial_{\tau}\tilde{H_k} +(\nabla \times \mathbf{\tilde{E}})_{k} &=0.
\end{empheq}
\end{subequations}
Here, the symbol ``$\overleftarrow{}$'' over the adjoint source term denotes the time-reversal transformation. 
Applying these transformation on the gradients in Eq.~(\ref{Eq:Gradients2}), leads to
\begin{equation}
\begin{split}
\nabla_{\rho}F[\mathbf{E}]=&-\int_{I}\sum_{k=1}^{3}\varepsilon_{0}(\mathrm{d}_{\rho}\varepsilon_{\infty, k})\partial_{\tau}\tilde{E}_k \overleftarrow{E}_{k} \\
&-\int_{I}\sum_{k=1}^{3}(\mathrm{d}_{\rho}\sigma_{k}) \tilde{E}_k \overleftarrow{E}_{k} \\
 &-\int_{I}\sum_{k=1}^{3}\sum_{i=1}^{2}\sum_{p=1}^{P_{k}^{(i)}}2(\mathrm{d}_{\rho}\kappa^{(i)})\partial_{\tau}\tilde{E}_{k}\Re\left\{\overleftarrow{Q}_{p, k}^{(i)}\right\}.
\end{split}
\end{equation}
This equation is equivalent to
\begin{equation}
\begin{split}
\nabla_{\rho}F[\mathbf{E}]=&-\int_{I}\sum_{k=1}^{3}\varepsilon_{0}(\mathrm{d}_{\rho}\varepsilon_{\infty, k})\partial_{\tau}\tilde{E}_k \overleftarrow{E}_{k} \\
&-\int_{I}\sum_{k=1}^{3}(\mathrm{d}_{\rho}\sigma_{k}) \tilde{E}_k \overleftarrow{E}_{k} \\
 &+\int_{I}\sum_{k=1}^{3}\sum_{i=1}^{2}\sum_{p=1}^{P_{k}^{(i)}}2(\mathrm{d}_{\rho}\kappa^{(i)})\tilde{E}_{k}\Re\left\{\partial_{\tau}\overleftarrow{Q}_{p, k}^{(i)}\right\},
\end{split}
\end{equation}
if we again apply integration by parts on the last term and taking the imposed boundary conditions into account.

\bibliography{supplement.bib}


\end{document}