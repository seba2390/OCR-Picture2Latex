\section{Linear versus Tractable Contracts}\label{sec:hardness}

In this section, we show that linear contracts have the same worst-case performance as efficiently-computable ones.
%, \emph{i.e.}, those that can be computed in polynomial time.
%
In Section~\ref{sec:hardness_mult}, we prove that there is no tractable contract providing a multiplicative loss sublinear in the number of agent's types.
%
This shows that, even if linear contracts provide a bad multiplicative loss in the number of agent's types (Theorem~\ref{thm:linear_lower_bound}), this is also true for tractable contracts.
%
% In words, ``shit is shit''.
%
Then, in Section~\ref{sec:hardness_bi_apx} we show that the $\big( \rho, 2^{-\Omega \left( \rho \right)} \big)$-bi-approximation result for linear contacts (Theorem~\ref{thm:mult_add_bayes}) is the best one can possibly achieve in polynomial time.
%
Technically, it is \textsf{NP}-hard to compute a contract providing a $\big(\rho, 2^{-\omega( \rho)}\big)$-bi-approximation of an optimal one.
%, for any $\rho \geq 1$.



\subsection{The Limits of Tractable Contracts}\label{sec:hardness_mult}

In the following Theorem~\ref{thm:hard_independent}, we prove that it is \textsf{NP}-hard to design a contract that approximates the overall principal's expected utility in an optimal contract up to within any multiplicative factor that is sublinear in the number of agent's types $\ell$.
%
The theorem is based on a reduction from GAP-INDEPENDENT-SET, which is the promise problem of deciding, in a given graph, whether there exists an independent set involving at least some (large) fraction of nodes or all the independent sets encompass at most some (small) fraction of nodes~\citep{Zuckerman2007linear}.

\begin{theorem}\label{thm:hard_independent}
	In Bayesian principal-agent problems, for any $\epsilon>0$ it is \textnormal{\textsf{NP}}-hard to design a contract providing a multiplicative loss $O \left( \ell^{1-\epsilon} \right)$ of an optimal one, where $\ell$ is the number of agent's types.
\end{theorem}

\begin{proof}
	We reduce from GAP-INDEPENDENT-SET, which is a promise problem that formally reads as follows: given $\epsilon > 0$ and a graph $G= (V,E)$, with set of nodes $V$ and set of edges $E$, determine whether $G$ admits an independent set of size at least $|V|^{1-\epsilon}$ or all the independent sets of $G$ have size smaller than $|V|^{\epsilon}$.
	%
	The Bayesian principal-agent instances in the reduction have a number of agent's types $\ell = |V|$.
	%
	The main idea of the proof is to show that, provided $\ell$ is large enough, if $G$ admits an independent set of size at least $\ell^{1-\epsilon}$, then in the corresponding principal-agent setting there exists a contract in which the overall principal's expected utility is at least $\frac{1}{2} \, \ell^{1-\epsilon} \, 2^{-\ell-1}$; otherwise, the utility is at most $2 \, \ell^{\epsilon} \, 2^{-\ell-1}$ for any contract.
	%
	Since GAP-INDEPENDENT-SET is \textnormal{\textsf{NP}}-hard for every $\epsilon>0$~\citep{hastad1999clique,Zuckerman2007linear}, this is enough to prove the statement.
	
	\paragraph{Construction}
	%
	Given a graph $G=(V,E)$, we build a Bayesian principal-agent setting $(\Theta, A, \Omega)$ as follows.
	%
	For every node $v \in V$ in the graph $G$, there are two outcomes $\omega_v, \bar \omega_v \in \Omega$ such that $r_{\omega_v} = 1 $ and $r_{\bar \omega_v} = 0$.
	%
	Moreover, there is an additional auxiliary outcome $\bar \omega \in \Omega$ with $r_{\bar \omega} = 0$.
	%
	The agent type is uniformly selected from a set $\Theta = \{ \theta_v \}_{v \in V}$ of $\ell = |V|$ different types, each corresponding to a node in the graph.
	%
	Thus, the distribution $\mu \in \Delta_{\Theta}$ is such that $\mu_{\theta_v} = \frac{1}{\ell}$ for every $\theta_v \in \Theta$.
	%
	The agent has $m = \ell^2-\ell+1$ actions available.
	%
	There is an action $\bar a \in A$ that induces a distribution over outcomes $F_{\theta_v, \bar a}$ with $F_{\theta_v, \bar a, \omega_v} =F_{\theta_v, \bar a, \bar \omega_v} = \frac{1}{2}$ and has cost $c_{\theta_v, \bar a} = \frac{1}{2} - \ell 2^{-\ell-1}$, no matter the agent's type $\theta_v \in \Theta$.
	%
	Each of the remaining $\ell^2 - \ell$ actions, denoted as $a_{ui} \in A$, corresponds to a node $u \in V$ and an index $i \in [\ell-1]$.
	%
	They are characterized by outcome distributions defined as follows.
	%
	For every agent's type $\theta_v \in \Theta$ and action $a_{ui} \in A$ (with $v,u \in V$), the distribution $F_{\theta_v, a_{ui}}$ is such that:
	%
	\begin{itemize}
		\item If $(v,u) \in E$, then the reachable outcomes are $\omega_v$, $\bar \omega_v$, $\bar \omega_{u}$, and $\bar \omega$, which are reached with probabilities, respectively, $F_{\theta_v,a_{ui}, \omega_v}=2^{-i-1}$, $F_{\theta_v,a_{ui},\bar \omega_v}=F_{\theta_v,a_{ui},\bar \omega_{u}}=\frac{2}{3}2^{-i-1}$, and $F_{\theta_v,a_{ui},\bar \omega}= 1-\frac{7}{3}2^{-i-1}$;
		%
		\item If $(v,u) \notin E$, then the reachable outcomes are $\omega_v$ and $\bar \omega$, which are reached with probabilities $F_{\theta_v,a_{ui},\omega_v}=2^{-i-1}$ and $F_{\theta_v,a_{ui},\bar \omega}= 1-2^{-i-1}$, respectively.
	\end{itemize}
	%
	Moreover, the cost of each action $a_{ui} \in A$ is $c_{\theta_v,a_{ui}}=2^{-i-1}-(\ell-i) 2^{-\ell-1}$, for any type $\theta_v \in \Theta$.
	
	
	\paragraph{Overview}
	%
	%
	In the instances of the reduction, the principal's expected utility contribution due to an agent's type playing an action $a_{u i} \in A$ is small.
	%
	Thus, the principal's objective is to incentivize as many agent's types as possible to play $\bar a$, which is the action with the greatest difference between expected reward and cost.
	%
	In order to incentivize an agent of type $\theta_v \in \Theta$ to play $\bar a$ rather than an action $a_{u 1} \in A$, while still achieving a satisfactory expected utility from that, the principal must set some (large) payment on outcome $\bar \omega_v$, some (small) payment on outcome $\omega_v$, and no payment on $\bar \omega$.
	%
	Indeed, rewarding the last two outcomes prevents from reaching the desired principal's expected utility.
	%
	Moreover, the principal must \emph{not} set payments on outcomes $\bar \omega_u$ such that vertex $u$ is adjacent to $v$, otherwise an agent of type $\theta_v \in \Theta$ would be incentivized to play action $a_{u 1}$ rather $\bar a$.
	%
	This implies that the principal can extract a satisfactory utility only from agent's types whose corresponding vertices constitute an independent set of the graph $G$. 
	%	
	%	\textcolor{red}{The reduction use a construction such that the expected utility for the principal when the agent plays actions $a^v_i$ is small.
	%		Then, the principal wants to incentivize more agents as possible to play action $a^*$ (that is the action with greater difference between expected reward and cost).
	%		However, our reduction is build in such a way that is highly inefficient add payments on outcome $\omega^v$ (and $\bar \omega$, $\bar \omega^v$). Incentive an action $a^*$ over  $a_1^v$  in this way requires large payment.
	%		Then, to have a large expected utility from an agent of type $\theta_v$ the principal must add an payment on outcome $\omega_v$ (incentivizing $a^*$) and no payment on vertexes $v'$ adjacent to $v$ (that incentivize $a_1^{v'}$).
	%		This implies that the principal can extract large utility only from a set of types that are vertexes of an independent set.}
	
	
	\paragraph{Completeness}
	%
	Suppose that graph $G$ admits an independent set of size at least $|V|^{1-\epsilon}$.
	%
	Then, there exists a maximal independent set $V^\star \subseteq V$ of size $|V^\star| \geq |V|^{1-\epsilon}$ such that, for every node $v \notin V^\star$, there is a node $u \in V^\star$ with $(v,u) \in E$.
	%
	Let us define a contract with $p_{\bar \omega_v} = 1-\ell \, 2^{-\ell-1}$ for all $v \in V^\star$ and $p_{ \omega_v}=\frac{1}{3} \left( 1-\ell 2^{-\ell-1} \right)+\ell 2^{-\ell-1}$ for all $v  \notin V^\star$, while all the other payments are set to $0$.
	%
	First, we show that, given this contract, any agent of type $\theta_v \in \Theta$ with $v \in V^\star$ is incentivized to play action $\bar a$.
	%
	The expected utility of the agent by playing $\bar a$ is:
	\[
		P_{\theta_v,\bar a} - c_{\theta_v, \bar a} = \frac{1}{2} \left( 1-\ell \, 2^{-\ell-1} \right)- \left( \frac{1}{2}-\ell \, 2^{-\ell-1} \right)=\frac{\ell}{2}\, 2^{-\ell-1}.
	\]
	%
	As for the expected utility of playing an action $a_{u i} \in A$, two cases are possible.
	%
	If $(v,u) \notin E$, then $P_{\theta_v, a_{ui}} = 0$ (since there is no payment associated to $\omega_v$, being $v \in V^\star$); thus, the resulting agent's expected utility is negative.
	%
	Instead, if $(v,u) \in E$, then, by definition of independent set, it must be the case that $u \notin V^\star$, which implies that the only reachable outcome having non-zero payment is $\bar \omega_v$. Thus, in this case the agent's expected utility is:
	%
	\[
		P_{\theta_v,a_{ui}} - c_{\theta_v,  a_{ui}} \hspace{-1mm}=\frac{2}{3} \ 2^{-i-1}  \left( 1-\ell 2^{-\ell-1} \right) -\left[ 2^{-i-1}\hspace{-1mm}-(\ell-i)2^{-\ell-1}) \right]\hspace{-1mm}= 2^{-\ell-1} \left( \ell-i- \frac{2}{3} \ \ell2^{-i-1}- \frac{1}{3} \ 2^{-i+\ell} \right).
	\]
	%
	Then, for $i \geq \frac{\ell}{2}$, it immediately follows that:
	%
	\[
		P_{\theta_v,a_{ui}} - c_{\theta_v,  a_{ui}}  = 2^{-\ell-1} \left( \ell-i-\frac{2}{3}\ \ell2^{-i-1}-\frac{1}{3} \ 2^{-i+\ell} \right) \le \frac{\ell}{2}2^{-\ell-1} = P_{\theta_v,\bar a} - c_{\theta_v, \bar a},
	\]
	%
	while, for $i \leq \frac{\ell}{2}$, the same result follows from the fact that, provided $\ell$ is large enough, it holds:
	%
	\[
		2^{-\ell-1}\left( \ell-i-\frac{2}{3}\ \ell2^{-i-1}-\frac{1}{3} \ 2^{-i+\ell} \right) \le    2^{-\ell-1} \left( \ell - \frac{1}{3} \ 2^{\frac{\ell}{2}} \right)\le \frac{\ell}{2}2^{-\ell-1} ,
	\]
	%
	where the last inequality holds since $2^{\frac{\ell}{2}} \geq \frac{\ell}{2}$ for a sufficiently large $\ell$.
	%
	This shows that any agent of type $\theta_v$ with $v \in V^\star$ plays action $\bar a$.
	%
	Next, we prove that, whenever the agent's type $\theta_v \in \Theta$ is such that $v \notin V^\star$, then the agent plays an action $a_{u1}$ associated with a node $u \in V$ such that $(v,u) \in E$ and $u \in V^\star$.
	%
	Notice that one such node always exists since $V^\star$ is a maximal independent set.
	%
	Given that $p_{\bar \omega_v} = 0$, $p_{\bar \omega_{u}}= 1-\ell 2^{-\ell-1}$, and $p_{\omega_v}=\frac{1}{3} \left( 1-\ell 2^{-\ell-1} \right)+\ell 2^{-\ell-1}$, the expected utility of the agent by playing action $a_{u1}$ is:
	%
	\begin{align*}
		P_{\theta_v, a_{u1}} - c_{\theta_v, a_{u1}} & = \frac{2}{3}2^{-2} \left( 1-\ell 2^{-\ell-1} \right) + 2^{-2} \left[ \frac{1}{3} \left( 1-\ell 2^{-\ell-1} \right) + \ell 2^{-\ell-1} \right]- \left[ 2^{-2}-(\ell-1)2^{-\ell-1} \right]=\\
		& = (\ell-1)2^{-\ell-1}.
	\end{align*}
	%
	On the other hand, any action $a_{ui}$ provides the agent with an expected utility:
	%
	\begin{align*}
		P_{\theta_v, a_{ui}} - c_{\theta_v, a_{ui}} &= \frac{2}{3}2^{-i-1} \left( 1-\ell 2^{-\ell-1} \right) 	+ \\
		& \textcolor{white}{=\qquad} + 2^{-i-1} \left[ \frac{1}{3} \left( 1-\ell 2^{-\ell-1} \right) + \ell 2^{-\ell-1} \right]- \left[ 2^{-i-1}-(\ell-i)2^{-\ell-1} \right] = \\
		& = (\ell-i)2^{-\ell-1}\le (\ell-1)2^{-\ell-1} = P_{\theta_v, a_{u1}} - c_{\theta_v, a_{u1}}.
	\end{align*}
	%
	Moreover, it is easy to check that action $\bar a$ provides the agent with a negative utility, showing that any agent of type $\theta_v$ with $v \notin V^\star$ plays an action $a_{u1}$.
	%
	Finally, we can conclude that the overall principal's expected utility is:
	%
	\begin{align*}
		&\sum_{v \in V^\star } \mu_{\theta_v} \left( R_{\theta_v, \bar a} -P_{\theta_v, \bar a} \right) + \sum_{v \notin V^\star }  \mu_{\theta_v} \left( R_{\theta_v,  a_{u1}} -P_{\theta_v,  a_{u1}} \right)= \\
		&\qquad \,\, =\sum_{v \in V^\star } \frac{1}{\ell} \left[ \frac{1}{2}-\frac{1}{2} \left( 1-\ell 2^{-\ell-1} \right)\right] + \\
		&\qquad\,\, \textcolor{white}{=}\qquad + \sum_{v \notin V^\star } \frac{1}{\ell}  \left[  2^{-2} -\frac{2}{3}2^{-2} \left( 1-\ell 2^{-\ell-1} \right) - \frac{1}{3} 2^{-2}  \left( 1-\ell 2^{-\ell-1} \right) - 2^{-2} \ell 2^{-\ell-1}   \right] \geq \\
		&\qquad\,\, \geq \frac{1}{\ell} \ell^{1 -\epsilon}\left[ \frac{1}{2}-\frac{1}{2} \left( 1-\ell 2^{-\ell-1} \right)\right]  + \\ 
		&\qquad\,\, \textcolor{white}{=}\qquad +\frac{1}{\ell} \left( \ell - \ell^{1-\epsilon} \right) \left[  2^{-2} -\frac{2}{3}2^{-2} \left( 1-\ell 2^{-\ell-1} \right) - \frac{1}{3} 2^{-2}  \left( 1-\ell 2^{-\ell-1} \right) - 2^{-2} \ell 2^{-\ell-1}   \right]  = \\
		&\qquad\,\, = \frac{1}{\ell} \ell^{1 -\epsilon} \frac{\ell}{2}2^{-\ell-1} = \frac{1}{2}\ell^{1-\epsilon} 2^{-\ell-1},
	\end{align*}
	where we used the fact that $|V^\star| \geq |V|^{1-\epsilon} = \ell^{1-\epsilon}$ by assumption.
	
	
	\paragraph{Soundness}
	%
	We start showing that, if the principal deploys a contract that implements an action $a_{ui} \in A$, then she achieves an expected utility $R_{\theta_v, a_{ui}} - P_{\theta_v, a_{ui}} \leq 2^{-\ell}$, no matter the agent's type $\theta_v \in \Theta$.
	%
	First, notice that, by implementing actions $a_{u \, \ell-1} \in A$, the principal can obtain an expected utility at most of $R_{\theta_v, a_{u \, \ell-1}} - c_{\theta_v, a_{u \, \ell-1}} = 2^{-\ell}-2^{-\ell-1}=2^{-\ell-1}$ (due to IR constraints).
	%
	Next, suppose that the contract implements an action $a_{u i} \in A$ with $i \in [\ell-2]$ for an agent of type $\theta_v \in \Theta$.
	%
	Then, by IC constraints, action $a_{u i}$ must provide the agent with an expected utility greater than or equal to that achieved by playing $a_{u \, i+1}$, \emph{i.e.}, it must be $P_{\theta_v, a_{ui}} - c_{\theta_v, a_{ui}} \geq P_{\theta_v, a_{u \, i+1}} - c_{\theta_v, a_{u \, i+1}}$.
	%
	Two cases are possible.
	%
	In the first one, it holds $(v,u) \in E$, which implies that:
	%
	\begin{align*}
			&p_{\omega_v} 2^{-i-1} + p_{\bar \omega_v} \frac{2}{3} 2^{-i-1} + p_{\bar \omega_u} \frac{2}{3} 2^{-i-1}	+ p_{\bar \omega} \left(  1 - \frac{7}{3} 2^{-i-1} \right) - \left[  2^{-i-1} - \left( \ell-i \right) 2^{-\ell-1} \right] \geq \\
			& \,\, \geq p_{\omega_v} 2^{-i-2} + p_{\bar \omega_v} \frac{2}{3} 2^{-i-2} + p_{\bar \omega_u} \frac{2}{3} 2^{-i-2}	+ p_{\bar \omega} \left(  1 - \frac{7}{3} 2^{-i-2} \right) - \left[  2^{-i-2} - \left( \ell-i-1 \right) 2^{-\ell-1} \right].
	\end{align*}
	%
	Thus,
	%
	\[
		2^{-i-1} \left(   p_{\omega_v} + \frac{2}{3} p_{\bar \omega_v} + \frac{2}{3} p_{\bar \omega_{u}} - \frac{7}{3} p_{\bar \omega} \right) \geq 2^{-i-2} - 2^{-\ell-1}.
	\]
	%
	As a result, the expected utility of the principal when an agent of type $\theta_v$ plays an action $a_{u i}$ is:
	%
	\begin{align*}
		R_{\theta_v, a_{ui}} - P_{\theta_v, a_{ui}} & = 2^{-i-1} - \left[ p_{\omega_v} 2^{-i-1} + p_{\bar \omega_v} \frac{2}{3} 2^{-i-1} + p_{\bar \omega_u} \frac{2}{3} 2^{-i-1}	+ p_{\bar \omega} \left(  1 - \frac{7}{3} 2^{-i-1} \right)    \right] \leq \\
		& \leq 2^{-i-1} - \left( p_{\omega_v} 2^{-i-1} + p_{\bar \omega_v} \frac{2}{3} 2^{-i-1} + p_{\bar \omega_u} \frac{2}{3} 2^{-i-1}	-p_{\bar \omega} \frac{7}{3} 2^{-i-1}     \right) \leq \\
		& \leq 2^{-i-1} - 2 \left(  2^{-i-2} - 2^{-\ell-1} \right) = 2^{-\ell}.
	\end{align*}
	%
	A similar argument holds for the case in which $(v,u) \notin E$.
	%
	Now, given a contract, let $\Theta^\star \subseteq \Theta$ be the set of agent's types $\theta_v$ such that: \emph{(i)} the contract implements action $\bar a$ for an agent of type $\theta_v$; and \emph{(ii)} the principal achieves an expected utility strictly larger than $2^{-\ell}$ when an agent of type $\theta_v$ plays $\bar a$.
	%
	We prove that, for any contract, any pair of types $\theta_v, \theta_u \in \Theta^\star$ is such that $(v,u) \notin E$.
	%
	By contradiction, suppose that there exist $\theta_v, \theta_u \in \Theta^\star$ such that $(u,v) \in E$.
	%
	We distinguish two cases.
	%
	The first one is when $p_{\bar \omega_v} \le p_{\bar \omega_{u}}$.
	%
	Since action $\bar a $ must provide an agent of type $\theta_v$ with an expected utility greater than or equal to that obtained for action $a_{u1}$ (by IC constraints), we have that:
	%
	\[
		\frac{1}{2} p_{\omega_v} + \frac{1}{2} p_{\bar \omega_v} - \left( \frac{1}{2} - \ell 2^{-\ell-1}  \right)\geq 2^{-2} p_{\omega_v} + \frac{2}{3} 2^{-2}  \left(  p_{\bar \omega_v} + p_{\bar \omega_{u}} \right) + \left( 1 - \frac{7}{3} 2^{-2} \right) p_{\bar \omega} - \left[  2^{-2} - \left( \ell-1 \right) 2^{-\ell-1}\right].
	\]
	%
	By using the fact that $p_{\bar \omega_u} \geq p_{\bar \omega_{v}} \geq 0$ and $p_{\bar \omega} \geq 0$, and re-arranging the terms, we obtain that $2^{-2} \left(  p_{\omega_{v}} + p_{\bar \omega_{v}} \right) \geq 2^{-2} - 2^{-\ell-1}$, which implies that $P_{\theta_v, a_{\bar v}} = \frac{1}{2} p_{\omega_{v}} + \frac{1}{2} p_{\bar \omega_{v}} \geq \frac{1}{2} - 2^{-\ell}$.
	%
	Since $R_{\theta_v, \bar a} = \frac{1}{2}$, this results in a principal's expected utility at most of $2^{-\ell}$, which is a contradiction.
	%
	In the second case in which $p_{\bar \omega_v} \geq p_{\bar \omega_{u}}$, we reach a contradiction using an analogous argument for an agent of type $\theta_u$ (rather than $\theta_v$).
	%
	Thus, we can conclude that, for any contract, the set of nodes $v \in V$ such that $\theta_v \in \Theta^\star$ constitutes an independent set of the graph $G$.
	%
	Moreover, notice that the maximum expected utility that the principal can obtain when an agent of type $\theta_v \in \Theta$ plays action $\bar a$ is $R_{\theta_v, \bar a} - c_{\theta_v, \bar a} = \ell 2^{-\ell-1}$.
	%
	Since, by assumption, the largest independent set of $G$ has size at most $|V|^\epsilon = \ell^\epsilon$, we can conclude that in any contract the overall expected utility of the principal is:
	%
	\begin{align*}
		\sum_{\theta_v \in \Theta^\star} \mu_{\theta_v} \left( R_{\theta_v, \bar a} - P_{\theta_v, \bar a}  \right) + \sum_{\theta_v \in \Theta \setminus \Theta^\star} \mu_{\theta_v} \left( R_{\theta_v,  a^*(\theta_v)} - P_{\theta_v,  a^*(\theta_v)}  \right) & \leq  \frac{1}{\ell} \ell^\epsilon \ell 2^{-\ell-1} + \frac{1}{\ell} \left( \ell -\ell^\epsilon  \right) 2^{-\ell} \leq \\
		& \leq 2\,  \frac{1}{\ell} \, \ell^{1+\epsilon}  \, 2^{-\ell-1} = \\
		& = 2 \, \ell^\epsilon \, 2^{-\ell-1},
	\end{align*}
	%
	where the last inequality holds provided that $\ell$ is sufficiently large.
	%
\end{proof}




\subsection{The Limits of Bi-Approximations}\label{sec:hardness_bi_apx}

We show that, for any $\rho\ge 1$, it is \textnormal{\textsf{NP}}-hard to design a contract providing a $\big(\rho,2^{-\omega(\rho)} \big)$-bi-approximation of an optimal one.
%
To this end, we employ a reduction from a {promise problem} associated with \textsf{LABEL-COVER} instances, whose definition follows.

\begin{definition}[\textnormal{\textsf{LABEL-COVER}} instance]
	An instance of \textnormal{\textsf{LABEL-COVER}} is a tuple $( G, \Sigma, \Pi)$:
	%
	\begin{itemize}
		\item $G \coloneqq (U,V,E)$ is a \emph{bipartite graph} defined by two disjoint sets of nodes $U$ and $V$, connected by the edges in $E \subseteq U \times V$, which are such that all the nodes in $U$ have the same degree;
		%
		\item $\Sigma$ is a finite set of \emph{labels}; and
		%
		\item $\Pi \coloneqq \left\{ \Pi_e : \Sigma \to \Sigma \mid e \in E \right\}$ is a finite set of \emph{edge constraints}.
	\end{itemize}
	%
	Moreover, a \emph{labeling} of the graph $G$ is a mapping $\pi: U \cup V \to \Sigma$ that assigns a label to each vertex of $G$ such that all the edge constraints are satisfied.
	Formally, a labeling $\pi$ satisfies the constraint for an edge $e =(u,v) \in E$ if it holds that $ \pi(v) = \Pi_e(\pi(u))$.
	%
\end{definition}

The classical \textsf{LABEL-COVER} problem is the search problem of finding a valid labeling for a \textsf{LABEL-COVER} instance given as input.
%
In the following, we consider a different version of the problem, which is the {promise problem} associated with \textsf{LABEL-COVER} instances.
%, defined as follows.

\begin{definition}[\textnormal{\textsf{GAP-LABEL-COVER}$_{c,s}$}]
	For any pair of numbers $0 < s < c < 1$, we define \textnormal{\textsf{GAP-LABEL-COVER}}$_{c,s}$ as the following promise problem.
	%
	\begin{itemize}
		\item \textnormal{\texttt{Input:}} An instance $(G,\Sigma, \Pi)$ of \textnormal{\textsf{LABEL-COVER}} such that either one of the following is true:
		%
		\begin{itemize}
			\item there exists a labeling $\pi$ that satisfies at least a fraction $c$ of the edge constraints in $\Pi$;
			%
			\item any labeling $\pi$ satisfies less than a fraction $s$ of the edge constraints in $\Pi$.
		\end{itemize}
		%
		\item \textnormal{\texttt{Output:}} Determine which of the above two cases hold.
	\end{itemize}
	%
\end{definition}

In order to prove Theorem~\ref{thm:hard_label}, we use the following result due to~\citet{raz1998parallel}~and~\citet{arora1998proof}.

\begin{theorem}[\citet{raz1998parallel,arora1998proof}]\label{thm:hard_gap_label}
	For any $\epsilon > 0$, there exists a constant $k_\epsilon \in \mathbb{N}$ that depends on $\epsilon$ such that the promise problem \textnormal{\textsf{GAP-LABEL-COVER}}$_{1,\epsilon}$ restricted to inputs $(G, \Sigma, \Pi)$ with $|\Sigma| = k_\epsilon$ is \textnormal{\textsf{NP}}-hard.
\end{theorem}

Next, we show our main result.~\footnote{
%Then, we prove the optimality of the algorithm in Theorem \ref{thm:apx_lin}.
%		Formally, we prove that it is NP-Hard to  $\left(\rho, 2^{-O(\frac{1}{\rho})}\right)$-approximate the optimal contract for each $\rho >0$.
In order to prove Theorem~\ref{thm:hard_label}, we need that the difference between the overall principal's expected utility in the completeness part and that in the soundness part is at least $2^{-O(\rho)}$, otherwise a contract providing a $\big(\rho, 2^{-O(\rho)}\big)$-bi-approximation cannot distinguish between the two cases.
%
Thus, we cannot use the construction in Theorem~\ref{thm:hard_independent}, since it does \emph{not} enjoy this property.
%
Indeed, we would like that the principal's expected utility in the soundness case decreases at a rate of $2^{-O(\rho)}$ as $\rho$ increases, while in Theorem~\ref{thm:hard_independent} the principal's expected utility decreases with the number of agent's types, \emph{i.e.}, its maximum value is $2^{-\ell}$.
%
Moreover, in Theorem \ref{thm:hard_independent} we reduce from \textsf{GAP-INDEPENDENT-SET}, which has \emph{not} perfect completeness.
%
Thus, the principal can extract a satisfactory utility from at most a fraction $\ell^{-\epsilon}$ of the agent's types, which implies that the expected utility decreases with the number of agent's types.
%
In order to deal with these problems, we base our reduction on $\textsf{GAP-LABEL-COVER}_{c,s}$.
%
Using this problem, we have perfect completeness, though at the expense of the \textsf{NP}-hardness of approximating only to within any multiplicative constant factor.
%
This is sufficient for proving Theorem~\ref{thm:hard_label}, since it requires the \textsf{NP}-hardness of approximating up to within a multiplicative factor that is of the order of $\Theta(\rho)$.}
%\textcolor{red}{Formally, Theorem~\ref{thm:hard_label} shows that there it is \textnormal{\textsf{NP}}-hard to design a contract providing a $\left(  \rho ,2^{- d \rho + e} \right)$-bi-approximation of an optimal one, for two constants $d \in \mathbb{N}$ and $e \in \mathbb{Z}$ that do not depend on the model instance.}


\begin{theorem} \label{thm:hard_label}
	Given a Bayesian principal-agent setting, it is \textnormal{\textsf{NP}}-hard to design a contract providing a $\big(\rho,2^{-\omega(\rho)} \big)$-bi-approximation of an optimal one. Equivalently, for any $\rho \ge 1$, it is \textnormal{\textsf{NP}}-hard to design a contract providing a $\big(  \rho ,2^{- d \rho + e} \big)$-bi-approximation for two constants $d \in \mathbb{R}^+$, $e \in \mathbb{R}$.
\end{theorem}

\begin{proof}
	%
	Letting $\gamma \coloneqq \lceil {10\rho} \rceil$, we prove the result by means of a reduction from \textsf{GAP-LABEL-COVER}$_{1,\frac{1}{2\gamma}}$.
	%
	In particular, our construction is such that, if the \textsf{LABEL-COVER} instance admits a labeling that satisfies all the edge constraints (recall that $c=1$), then the corresponding Bayesian principal-agent setting admits a contract providing the principal with an overall expected utility at least of $(\gamma+2)2^{\gamma-4}$.
	%
	Instead, if at most a $\frac{1}{2\gamma}$ fraction of the edge constraints are satisfied by any labeling, then the principal expected utility is at most $2^{-\gamma-1}$ in any contract.
	%
	By Theorem~\ref{thm:hard_gap_label}, this implies that designing a contract giving a $\big( \rho,2^{-8 \rho-4} \big)$-bi-approximation (for any $\rho \ge 1$) is \textsf{NP}-hard.
	%
	%Indeed, the result is readily obtained by means of the following relation:
	Indeed, the following relation shows that a $\big( \rho,2^{-8 \rho-4} \big)$-bi-approximation algorithm can determine whether the \textsf{LABEL-COVER} instance admits a labeling that satisfies all the edge constraints or at most a $\frac{1}{2\gamma}$ fraction of the edge constraints are satisfied by any labeling:
	%
	\[
		\frac{1}{\rho} \left( \gamma+2 \right) 2^{\gamma-4}- 2^{-8\rho-4} \ge \frac{1}{\rho} \left( \gamma+2 \right) 2^{\gamma-4} -2^{-\gamma-3}> 10 \cdot \ 2^{\gamma-4}- 2^{-\gamma-3}\ge 2^{-\gamma-1}.
	\]
	%
	Next, we provide the formal definition of our reduction and prove its crucial properties.
	
	\paragraph{Construction}
	%
	Given an instance of \textsf{LABEL-COVER} $(G,\Sigma,\Pi)$ with a bipartite graph $G = (U,V,E)$, we build a Bayesian principal-agent setting $(\Theta,A,\Omega)$ as follows.
	%
	For every node $v \in U \cup V$ of $G$ and label $\sigma \in \Sigma$, there is an outcome $\omega_{v  \sigma} \in \Omega$ with reward $r_{\omega_{v  \sigma} } = 0$ to the principal.
	%
	Moreover, there are two additional outcomes $\omega_0, \omega_1 \in \Omega$ such that $r_{\omega_0} = 0$ and $r_{\omega_1} = 1$.
	%
	The agent can be of $\ell = |E|$ different types, each associated with an edge of $G$; formally, $\Theta = \{ \theta_e \}_{e \in E}$.
	%
	All the types have the same probability of occurring, being $\mu \in \Delta_\Theta$ such that $\mu_{\theta_e} = \frac{1}{\ell}$ for $e \in E$.
	%
	For the ease of presentation and w.l.o.g., we let each agent's type $\theta_e \in \Theta$ having a different action set $A_{\theta_e}$, so that, with an abuse of notation, $A = \{ A_{\theta_e} \}_{\theta_e \in \Theta}$.
	%
	Notice that, in order to recover a principal-agent setting in which each agent's type has the same set of actions, it is sufficient to add some dummy actions having zero cost for the agent and deterministically leading to outcome $\omega_0$ (with zero reward).
	%
	Each agent's type $\theta_e \in \Theta$ with $e = (u,v)$ has an action $a_{\sigma  \sigma'} \in A_{\theta_e}$ for every pair of labels such that $\sigma \in \Sigma$ and $\sigma' = \Pi_e(\sigma)$.
	%
	The action induces a probability distribution over outcomes $F_{\theta_e, a_{\sigma  \sigma'} }$ such that:
	%
	\begin{itemize}
		\item Outcome $\omega_1$ is reached half of the times, being $F_{\theta_e, a_{\sigma  \sigma'}, \omega_1 } = \frac{1}{2} $;
		%
		\item In the other half of the cases, outcomes $\omega_{u  \sigma}$ and $\omega_{v  \sigma'}$ are reached with equal probability, being $F_{\theta_e, a_{\sigma  \sigma'}, \omega_{u  \sigma} } = F_{\theta_e, a_{\sigma  \sigma'}, \omega_{v  \sigma'} } = \frac{1}{4} $.
	\end{itemize}
	%
	The cost of the action is $c_{\theta_e, a_{\sigma \sigma'}} = \frac{1}{2}-(\gamma+2) 2^{-\gamma-3}$, no matter the agent's type $\theta_e \in \Theta$.
	%
	Moreover, each agent's type $\theta_e \in \Theta$ with $e = (u,v)$ has an action $a_{i \sigma  \sigma'} \in A_{\theta_e}$ for every index $i \in [\gamma]$ and pair of labels $\sigma, \sigma' \in \Sigma$ such that $ \sigma' \neq  \Pi_e(\sigma) $.
	%
	The action probability distribution $F_{\theta_e, a_{i \sigma  \sigma'} }$ is such that:
	%
	\begin{itemize}
		\item Outcomes  $\omega_{u  \sigma}$ and $\omega_{v  \sigma'} $ are reached with the same (small) probability decreasing exponentially in the value of $i$, being $F_{\theta_e, a_{i \sigma  \sigma'}, \omega_{u  \sigma} } = F_{\theta_e, a_{i \sigma  \sigma'}, \omega_{v  \sigma'} } = 2^{-i-2}$;
		%
		\item Outcome $\omega_1$ is reached with a probability twice as large as that of the previous ones, as $F_{\theta_e, a_{i \sigma \sigma'} , \omega_1 } = 2^{-i-1}$; 
		%
		\item In all the other cases outcome $\omega_0$ is reached, since $F_{\theta_e, a_{i \sigma \sigma'} , \omega_0 } = 1 - 2^{-i}$.
	\end{itemize}
	%
	Finally, the cost of the action is $c_{\theta_e, a_{i \sigma \sigma'} } = 2^{-i-1}-(\gamma-i+2) 2^{-\gamma-3}$
	
	\paragraph{Overview}
	%
	The Bayesian principal-agent instances of the reduction have a structure similar to those in the proof of Theorem~\ref{thm:hard_independent}.
	%
	Here, the contribution to the overall principal's expected utility due to an agent's type playing an action $a_{i \sigma \sigma'} \in A$ is small.
	%
	Thus, the principal's objective is to incentivize as many agent's types as possible to play an action $a_{\sigma\sigma'}$.
	%
	We recall that, for each agent's type $\theta_{e} \in \Theta$ with $e = (u,v)$, there exists an action $a_{\sigma \sigma'}$ only if the labels $\sigma$ and $\sigma'$ satisfy the constraint for edge $e$, namely $\sigma' = \Pi_e(\sigma)$.
	%
	Moreover, in order for the principal to incentivize an agent's type to play $a_{\sigma \sigma'}$ and extract a satisfactory utility from that, the principal must commit to a contract that sets some payments on outcomes $\omega_{u \sigma}$ and $\omega_{v \sigma'}$.
	%
	More precisely, an agent of type $\theta_{e} \in \Theta$ with $e = (u,v)$ is incentivized to play $a_{\sigma \sigma'}$ if the payments on outcomes $\omega_{u \sigma}$ and $\omega_{v \sigma'}$ are equal and sufficiently large.
	%
	At the same time, there must \emph{not} be two labels $\sigma_u, \sigma_v \in \Sigma$ with $\sigma_u \neq \sigma$ and $\sigma_v \neq \sigma'$ such that a large payment is assigned to either $\omega_{u \sigma''}$ or $\omega_{v \sigma''}$, otherwise an agent of type $\theta_e$ would be incentivized to play action $a_{1 \sigma_u \sigma_v}$ rather than $a_{\sigma \sigma'}$.
	%
	Then, for every vertex $v \in U \cup V$ of the graph $G$, there exists a single label $\sigma \in \Sigma$ such that there is some payment on $\omega_{v \sigma}$ and these labels define a labeling that satisfies all the constraints of edges corresponding to agent's types that play action $a_{\sigma \sigma'}$ while resulting in a satisfactory principal's expected utility.
	%	
	%	\textcolor{red}{As in Theorem \ref{thm:hard_label}, we use a construction that forces the principal utility to be small when the agents plays actions $a^i_{\sigma,\sigma'}$. Then, the principal has large expected utility if he incentives a large number of agent's type to play an action $a^*_{\sigma,\sigma'}$. Recall that there exists an action $a^*{\sigma,\sigma'}$ only if $\sigma$ and $\sigma'$ satisfy the constraint. Moreover, to extract a large expected utility, the principal must incentivize the action adding payment to ${\sigma}$ and $\sigma'$.  Hence, an agent of type $\theta_{e}$, $e=(u,v)$ is incentivized to play $a^*_{\sigma,\sigma'}$ if the payments on $\sigma$ and $\sigma'$ are large (and equal). At the same time, the payments on outcomes $\omega_u,{\sigma''}$, $\sigma''\neq \sigma$ and on $\omega_v,{\sigma''}$, $\sigma'''\neq \sigma$ must be small, otherwise the action $a^1_{\sigma'',\sigma'''}$ has large payment.
	%	Then, for each $v\in U \cup V$, there is a single $\sigma$ with a payment and the set of labels with payments define an assigment that  satisfies the constraints of all the agent's types from which the principal extract a large utility.}

	\paragraph{Completeness}
	%
	Suppose the instance of \textsf{LABEL-COVER} $(G,\Sigma,\Pi)$ admits a labeling $\pi : U \cup V \to \Sigma$ that satisfies all the edge constraints in $\Pi$.
	%
	Let us define a contract such that $p_{\omega_{v \pi(v)}} = 1-(\gamma+2) 2^{-\gamma-3}$ for every node $v \in U \cup V$, while all the other payments are set to zero.
	%
	First, we show that, given this contract, an agent of type $\theta_{e} \in \Theta$ with $e= (u,v)$ is incentivized to play action $a_{\pi(u) \pi(v)}$.
	%
	Recall that, in our construction, an agent of type $\theta_{e}$ has action $a_{\pi(u) \pi(v)}$ available if and only if $\pi(v) = \Pi_e(\pi(u))$, which is always true since the labeling $\pi$ satisfies all the edge constraints by assumption.
	%
	Given the definition of the contract, it holds that $p_{\omega_{u \pi(v)}} = p_{\omega_{v \pi(v)}} = 1-(\gamma+2) 2^{-\gamma-3}$, while $p_{\omega_{u \sigma}} = 0$ for every $\sigma \in \Sigma \setminus \{ \pi(u) \}$ and $p_{\omega_{v \sigma}} = 0$ for every $\sigma \in \Sigma \setminus \{ \pi(v) \}$.
	%
	This implies that the expected utility of an agent of type $\theta_{e}$ by playing action $a_{\pi(u) \pi(v)}$ is:
	%
	\begin{align*}
		P_{\theta_{e}, a_{\pi(u) \pi(v)}} - c_{\theta_e, a_{\pi(u) \pi(v)}} & =\frac{1}{4} \left( p_{\omega_{u\sigma}} +p_{\omega_{v\sigma'}} \right) - \left[ \frac{1}{2}- (\gamma+2) 2^{-\gamma-3} \right]= \\
		& =-(\gamma+2) 2^{-\gamma-4} +(\gamma+2)2^{-\gamma-3}= \\
		& =(\gamma+2) 2^{-\gamma-4}.
	\end{align*}
	%
	Moreover, for any pair of labels $\sigma, \sigma' \in \Sigma$ such that $ \sigma' \neq  \Pi_e(\sigma) $, each action $a_{i \sigma \sigma'}$ for $i \in [\gamma]$ provides an expected utility of:
	%
	\begin{align*}
		P_{\theta_{e}, a_{i \sigma \sigma'}} - c_{\theta_e, a_{i \sigma\sigma'}} & =2^{-i-2} \left[ 1-(\gamma+2) 2^{-\gamma-3} \right] - \left[ 2^{-i-1}-(\gamma-i+2)2^{-\gamma-3} \right]= \\
		& = 2^{-\gamma-3} \left[ -2^{-i-2+\gamma+3}-(\gamma+2) 2^{-i-2}+ \gamma-i+2 \right] ,
	\end{align*}
	%
	which holds since it cannot be the case that both $p_{\omega_{u \sigma}}$ and $p_{\omega_{v \sigma'}}$ are different from zero, otherwise it would be $\pi(v) \in \Sigma \setminus \{ \pi(u) \}$, contradicting the fact that the labeling $\pi$ satisfies all the edge constraints.
	%
	We distinguish two cases.
	%
	In the first one, it holds $i\ge \frac{\gamma}{2}+1$.
	%
	Then,
	%
	\[
		P_{\theta_{e}, a_{i \sigma \sigma'}} - c_{\theta_e, a_{i \sigma\sigma'}}  \le 2^{-\gamma-3}  (\gamma-i+2) \le (\gamma+2) 2^{-\gamma-4} = P_{\theta_{e}, a_{\pi(u) \pi(v)}} - c_{\theta_e, a_{\pi(u) \pi(v)}}.
	\]
	%
	In the second case, it holds $i\le \frac{\gamma}{2}+1$, which implies that:
	%
	\[
		P_{\theta_{e}, a_{i \sigma \sigma'}} - c_{\theta_e, a_{i \sigma\sigma'}}  \leq 2^{-\gamma-3} (\gamma+2-2^{\frac{\gamma}{2}})\le \frac{\gamma}{2}2^{-\gamma-3} \leq P_{\theta_{e}, a_{\pi(u) \pi(v)}} - c_{\theta_e, a_{\pi(u) \pi(v)}},
	\]
	%
	where the second-last inequality holds since $\frac{\gamma}{2} + 2 \leq 2^{\frac{\gamma}{2}}$ for $\gamma \geq 4$.
	%
	Finally, it is easy to see that all the actions $a_{\sigma \sigma'}$ that are different from $a_{\pi(u) \pi(v)}$ provide an agent of type $\theta_{e}$ with an expected utility smaller than that achieved by playing $a_{\pi(u) \pi(v)}$.
	%
	This shows that the contract incentivizes each agent's type $\theta_{e} \in \Theta$ with $e = (u,v)$ to play action $a_{\pi(u) \pi(v)}$.
	%
	In conclusion, the overall expected utility of the principal is:
	%
	\begin{align*}
		\sum_{\theta_{e} \in \Theta} \mu_{\theta_{e}} \left( R_{\theta_{e}, a^*(\theta_{e})} - P_{\theta_{e}, a^*(\theta_{e})} \right) &= \frac{1}{\ell} \sum_{e = (u,v) \in E} R_{\theta_{e}, a_{\pi(u) \pi(v)} } - P_{\theta_{e}, a_{\pi(u) \pi(v)} } = \\
		& = \frac{1}{2} - \frac{1}{2} \left[  1-(\gamma+2) 2^{-\gamma-3} \right] =\\
		& = (\gamma+2) 2^{-\gamma-4}.
	\end{align*}
	
	\paragraph{Soundness}
	%
	We show that, if the \textsf{LABEL-COVER} instance is such that every labeling $\pi : U \cup V \to \Sigma$ satisfies at most a fraction $\frac{1}{2\gamma}$ of the edge constraints in $\Pi$, then, in the corresponding principal-agent setting, any contract provides the principal with an expected utility at most of $2^{-\gamma-1}$.
	%
	As a first step, we show that all the actions $a_{i \sigma \sigma'}$ provide the principal with an expected utility at most of $2^{-\gamma-2}$.
	%
	Assume that the agent has type $\theta_{e} \in \Theta$ with $e = (u,v)$ and that the contract deployed by the principal implements an action $a_{\gamma \sigma \sigma'}$ for an agent of type $\theta_{e}$, for some $\sigma, \sigma' \in \Sigma$ such that $\sigma' \neq \Pi_e(\sigma)$.
	%
	Then, the principal's expected reward is $R_{\theta_{e}, a_{\gamma \sigma \sigma'}} = 2^{-\gamma-1}$, while the agent's cost is $c_{\theta_{e}, a_{\gamma \sigma \sigma'}} = 2^{-\gamma-1}-2^{-\gamma-2}$, implying that the principal's expected utility is at most $2^{-\gamma-2}$.
	%
	Now, assume that the contract implements an action $a_{i \sigma \sigma'}$ with $i \in [\gamma] : i < \gamma$ for an agent of type $\theta_{e}$, for some $\sigma, \sigma' \in \Sigma$ such that $\sigma' \neq \Pi_e(\sigma)$.
	%
	Then, since the action $a_{i \sigma \sigma'}$ must be IC, it must provide the agent with an expected utility greater than or equal to that provided by action $a_{i+1 \, \sigma \sigma'}$.
	%
	Thus, it must be the case that $P_{\theta_e, a_{i \sigma \sigma'}} - c_{\theta_e, a_{i \sigma \sigma'}} \geq P_{\theta_e, a_{i+1 \, \sigma \sigma'}} - c_{\theta_e, a_{i+1 \,  \sigma \sigma'}} $, which implies that:
	%
	\begin{align*}
		&2^{-i-1} p_{\omega_1} + 2^{-i-2} p_{\omega_{u\sigma}} +2^{-i-2} p_{\omega_{v \sigma'}} + \left( 1-2^{-i} \right) p_{ \omega_0}    - \left[ 2^{-i-1}-(\gamma-i+2) 2^{-\gamma-3} \right]\ge\\
		& \geq 2^{-i-2} p_{\omega_1}+ 2^{-i-3} p_{\omega_{u \sigma}} +2^{-i-3} p_{\omega_{v \sigma'}} +  \left( 1-2^{-i-1} \right) p_{\omega_0}  - \left[ 2^{-i-2}-(\gamma-i+1) 2^{-\gamma-3} \right].
	\end{align*}
	%
	Thus, by re-arranging the terms and using the fact that $p_{\omega_0} \geq 0$, we get $2 p_{\omega_1}+ p_{\omega_{u \sigma}} +p_{\omega_{v \sigma'}} \ge 2- 2^{-\gamma+i}$, which implies that the principal's expected utility is:
	%
	\[
		R_{\theta_{e}, a_{i \sigma \sigma'} } - P_{\theta_e, a_{i \sigma \sigma'} } = 2^{-i-1} - 2^{-i-1} p_{\omega_1} - 2^{-i-2} p_{\omega_{u\sigma}} -2^{-i-2} p_{\omega_{v \sigma'}}- \left( 1-2^{-i} \right) p_{ \omega_0}    \leq 2^{-\gamma-2}.
	\]
	%
	This proves that any agent's action $a_{i \sigma \sigma'}$ provides the principal with an expected utility at most of $2^{-\gamma-2}$.
	%
	Next, we switch the attention to actions $a_{\sigma \sigma'}$.
	%
	Given a contract, let $\pi : U \cup V \to \Sigma$ be a labeling for the \textsf{LABEL-COVER} instance such that $\pi(v) \in \argmax_{\sigma \in \Sigma} p_{\omega_{v \sigma}}$ for every $v \in U \cup V$ (with ties broken arbitrarily).
	%
	We show that, for an agent of type $\theta_e \in \Theta$, the contract implements an action providing the principal with an expected utility greater than $2^{-\gamma-2}$ only if the labeling $\pi$ satisfies the constraint $\Pi_e$ associated with edge $e$.
	%
	By contradiction, suppose that $e = (u,v)$ and the constraint $\Pi_e$ is \emph{not} satisfied by $\pi$ since $\pi(v) \neq \Pi_e(\pi(u))$.
	%
	Then, there is an agent's action $a_{1 \sigma_u \sigma_v} \in A_{\theta_e}$ with $\sigma_u \in \argmax_{\sigma \in \Sigma} p_{\omega_{\omega_{u \sigma} }  }$ and $\sigma_v \in \argmax_{\sigma \in \Sigma} p_{\omega_{\omega_{v \sigma} }  }$ such that the agent's expected utility is:
	\[
		P_{\theta_e, a_{1 \sigma_u \sigma_v}} - c_{\theta_e, a_{1 \sigma_u \sigma_v}} = \frac{1}{4} p_{\omega_1} + \frac{1}{8} \left(  p_{\omega_{u \sigma_u} }  +  p_{\omega_{v \sigma_v} }  \right) + \frac{1}{2} p_{\omega_0} - \left[ \frac{1}{4} - (\gamma+1) 2^{-\gamma - 3} \right].
	\]
	%
	Moreover, all the actions $a_{\sigma \sigma'} \in A_{\theta_e}$ for $\sigma \in \Sigma$ and $\sigma' = \Pi_e(\sigma)$ provide the agent with a utility:
	%
	\[
		P_{\theta_e, a_{ \sigma \sigma'}} - c_{\theta_e, a_{ \sigma \sigma'}} =\frac{1}{2}p_{\omega_1}+\frac{1}{4} \left( p_{\omega_{u \sigma}}+p_{\omega_{v \sigma'}} \right)- \left[ \frac{1}{2}-(\gamma+2) 2^{-\gamma-3}\right],
	\]
	%
	where, by definition, it holds $p_{\omega_{u \sigma}}\le p_{\omega_{u\sigma_u}}$ and $p_{\omega_{v \sigma'}}\le p_{\omega_{v \sigma_v}}$.
	%
	Thus, since an action $a_{\sigma \sigma'}$ is IC only if it holds that $P_{\theta_e, a_{ \sigma \sigma'}} - c_{\theta_e, a_{ \sigma \sigma'}} \geq P_{\theta_e, a_{1 \sigma_u \sigma_v}} - c_{\theta_e, a_{1 \sigma_u \sigma_v}} $, we can conclude that $2p_{\omega_1} +p_{\omega_{u \sigma}}+p_{\omega_{v \sigma'}} \ge 2- 2^{-\gamma}$.
	%
	As a result, the expected utility of the principal is:
	%
	\[
		R_{\theta_e, a_{ \sigma \sigma'}  } - P_{\theta_e , a_{ \sigma \sigma'}  } = \frac{1}{2} - \frac{1}{2}p_{\omega_1} - \frac{1}{4} \left( p_{\omega_{u \sigma}}+p_{\omega_{v \sigma'}} \right) \leq 2^{-\gamma-2},
	\]
	%
	which is a contradiction.
	%
	Finally, the maximum expected utility the principal can achieve for an agent of any type $\theta_e \in \Theta$ is $\max_{a \in A_{\theta_e}} \left\{ R_{\theta_e, a} - c_{\theta_e, a} \right\} = (\gamma+2)2^{-\gamma-3}$.
	%
	By assumption, any labeling satisfies at most a fraction $\frac{1}{2\gamma} |E|$ of the edge constraints, thus, given any contract, at most a fraction $\frac{1}{2\gamma} \ell$ of agent's types play an action providing the principal with an expected utility greater than $2^{-\gamma-2}$ (and at most $(\gamma+2)2^{-\gamma-3}$).
	%
	Then, the overall principal's expected utility in any contract is:
	%
	\[
		\sum_{\theta_{e} \in \Theta} \mu_{\theta_{e}} \left( R_{\theta_{e}, a^*(\theta_{e})} - P_{\theta_{e}, a^*(\theta_{e})} \right) < \frac{1}{\ell} \cdot \frac{\ell}{2 \gamma} (\gamma+2)2^{-\gamma-3}+ \frac{1}{\ell} \left( \ell - \frac{\ell}{2 \gamma} \right)2^{-\gamma-2} \leq 2^{-\gamma+1},
	\]
	%
	which concludes the proof.
	%
\end{proof}



