\section{Introduction}\label{sec:introduction}


\emph{Principal-agent} problems are ubiquitous in real-world economies.
%
These problems model interactions between two parties, a \emph{principal} and an \emph{agent}, where the latter chooses an action that determines some externalities on the former.
%
We focus on \emph{hidden-action} models, where the principal cannot observe the action taken by the agent, but only a stochastic outcome that is probabilistically determined as a result of the agent's action.
%
Each action is associated with a corresponding cost for the agent, while the principal receives a reward for the resulting outcome.
%
As a result, the principal's objective is to incentive an agent's action that leads to favorable outcomes.
%
This is achieved by committing to an outcome-dependent payment scheme, usually called \emph{contract}.


Principal-agent problems are pervasive in classical economic scenarios.
%
A well-known textbook example of principal-agent problem is that of a salesperson (agent) working for a company (principal).
%
The former has to decide on the level of effort she wants to put in selling products for the company.
%
Naturally, the company cannot observe the chosen level of effort (action), but it is only aware of the number of products sold.
%
Assuming that this figure is correlated with the level of effort selected by the salesperson, the company can incentivize an high level of effort by paying a commission to the salesperson based on the actual number of sales.


Interactions involving a principal and an agent play a crucial role also in modern economies centered around digital means.
%
In spite of this, principal-agent problems received far less attention from the economics and computation community than auctions and, more in general, mechanism design problems (more details on related computational works appear later in this section).
%
Remarkably, principal-agent models may have potential applications in various real-world settings, such as, \emph{e.g.}, crowdsourcing platforms~\citep{ho2016adaptive}, blockchain-based smart contracts~\citep{cong2019blockchain}, and healthcare~\citep{bastani2016analysis}.


In this paper, we study a generalization of the classical hidden-action principal-agent problem.
%
In particular, we relax the assumption that the principal perfectly knows the agent by considering a \emph{Bayesian} setting in which the agent's type is unknown and randomly selected according to a given probability distribution, which is known to the principal.
%
Each agent's type is characterized by her own action costs and action-outcome distributions.
%
In the salesperson example, types may correspond to different skill profiles for the salesperson, \emph{e.g.}, a clever worker can achieve better sales results than a non-clever one by putting the same level of effort in her work.


In the literature on principal-agent problems, considerable attention has been devoted to \emph{linear contracts} (see, \emph{e.g.},~\cite{carroll2015robustness,carroll2019robustness,dutting2019simple}), which are pure-commission payment schemes that pay the agent a given fraction of the principal's reward associated with the obtained outcome.
%
These contracts enjoy some nice properties.
%
In particular, they are simple to understand---given their single-parameter nature---and, in non-Bayesian settings, they still provide good approximation guarantees with respect to a principal-optimal (possibly non-linear) contract~\citep{dutting2019simple}.
%
While in non-Bayesian principal-agent problems an optimal contract can be computed efficiently by using a linear program, this is no longer the case in our Bayesian setting.
%
This further motivates our focus on {linear contracts}, which can be optimized efficiently given their single-parameter nature.


\subsection{Original Contributions}

The main goal of our work is to analyze the properties of linear contracts in Bayesian principal-agent settings, in order to understand their approximation guarantees with respect to optimal contracts and \emph{tractable} ones, with the latter being defined as those that can be computed efficiently (\emph{i.e.}, in polynomial time).
%
In particular, we look at approximations of the principal's expected utility. 
%
Notice that, while optimal contracts are a natural benchmark in any principal-agent problem, the comparison with tractable contracts becomes relevant and fundamental in our Bayesian model, where an optimal contract cannot be computed efficiently, and, thus, the most natural benchmark is the family of all contracts that can be computed in polynomial time. 
%
% Our main finding is that linear contracts perform well among all the tractable ones.


After introducing all the required preliminary concepts in Section~\ref{sec:preliminaries}, we start our analysis by studying, in Section~\ref{sec:linear}, the approximation guarantees of linear contracts with respect to optimal ones in Bayesian principal-agent problems.
%
We show that, from a purely-multiplicative approximation perspective, linear contracts suffer from a loss with respect to an optimal contract that grows linearly in the number of agent's types.
%
This happens in degenerate instances in which the principal's rewards are exponentially small in the number of agent's types, thus suggesting that there is hope linear contracts could obtain a constant multiplicative approximation, at the expense of an exponentially-small additive loss.
%
This motivates the introduction of $\big( \rho, g(\rho) \big)$\emph{-bi-approximate} contracts, which are those providing the principal with an expected utility at least $\frac{OPT}{\rho} - g(\rho)$, where $OPT$ is the principal's expected utility in an optimal contract.
%
Our main result is that linear contracts give a $\big( \rho, 2^{-\Omega(\rho)}\big)$\emph{-bi-approximation} of an optimal contract, \emph{i.e.}, they guarantee a constant multiplicative approximation $\rho$ of the optimal principal's expected utility, at the expense of an exponentially-small additive loss $2^{-\Omega(\rho)}$.
%
We complement this result by showing that no linear contract can provide a $\big(  \rho ,2^{-\omega \left( \rho \right)} \big)$-approximation of an optimal one, even in non-Bayesian settings.
%
This implies that, using linear contracts, we can only obtain bi-approximations whose additive losses decrease at most exponentially in the multiplicative factor $\rho$.
%
Notice that our bi-approximation results also hold for the basic non-Bayesian case, complementing known approximation results of linear contracts in such setting~\citep{dutting2019simple} (see the related works for more details).


Then, in Section~\ref{sec:hardness}, we focus on the performances of linear contracts with respect to tractable ones in Bayesian settings, showing that, surprisingly, they perform well.
%
In particular, we show that there is no tractable contract providing a constant multiplicative loss with respect to an optimal one.
%
Formally, we prove that it is $\mathsf{NP}$-hard to design a contract with a multiplicative loss sublinear in the number of agent's types.
%
Then, we study the approximation guarantees of tractable contracts in terms of bi-approximations.
%
We prove that it is $\mathsf{NP}$-hard to design a contract that provides a $\big( \rho,2^{-\omega(\rho)} \big)$-bi-approximation of an optimal one, thus matching the lower bound of linear contracts.



We conclude with Section~\ref{sec:tractable}, where we show that there are some special cases of our Bayesian principal-agent problem in which an optimal contract can be computed in polynomial time.
%
In particular, this happens if we fix either the number of agent's types or the number of outcomes. 


\subsection{Related Works}

Hidden-action principal-agent problems have received considerable attention in the economic literature, where they usually fall under the umbrella of a broader subject called \emph{contract theory}, which is a fundamental pillar of microeconomic theory~\citep{shavell1979risk,grossman1983analysis,rogerson1985repeated,holmstrom1991multitask} (see the books by~\citet{mas1995microeconomic},~\citet{bolton2005contract},~and~\citet{laffont2009theory} for a detailed treatment of the subject).


The first computational studies on principal-agent problems appeared only recently.
%
Among them, it is worth discussing in detail that of~\citet{dutting2019simple}, which is perhaps the most related to ours.
%
\citet{dutting2019simple} study non-Bayesian principal-agent problems (\emph{i.e.}, a special case of our setting having only one agent's type), with a focus on linear contracts.
%
In particular, they show that linear contracts provide a constant multiplicative approximation of the principal's expected utility in an optimal contract, except in degenerate instances having the following three properties \emph{simultaneously}: there are many agent's actions, there is a big spread of rewards, and there is a big spread of costs.
%
Moreover, the results of~\citet{dutting2019simple} are tight.
%
In our work, we extend this comparison between linear and optimal contracts to our Bayesian settings.
%
However, apart from that, our work considerably departs from~\citep{dutting2019simple}, since our main focus is on understanding the performances of linear contracts with respect to tractable ones.
%
Notice that this is \emph{not} a concern for~\citet{dutting2019simple}, since, differently from the Bayesian setting, an optimal contract can be computed efficiently in classical (non-Bayesian) principal-agent problems.

There is a number of other computational works that study extensions of classical hidden-action principal-agent problems exhibiting some sort of combinatorial structure.
%
For instance, the work of~\citet{babaioff2006combinatorial} studies a model with multiple agents (see also its extended version~\citep{babaioff2012combinatorial} and its follow-ups~\citep{babaioff2009free,babaioff2010mixed}).
%
Its focus is on how complex combinations of agents' actions influence the resulting outcome in presence of inter-agent externalities, while in our model there is only one agent that can be of different types, and, thus, no externalities among agent's types are involved.
%
Moreover,~\citet{babaioff2006combinatorial} study settings in which each agent has only two actions, while in our model each agent's type can have an arbitrary number of actions.
%
Recently,~\citet{dutting2020complexity} study another principal-agent problem whose underlying structure is combinatorial, as a result of defining the outcome space implicitly through a suitably-defined succinct representation.

Other computational works on principal-agent problems worth citing are~\citep{babaioff2014contract}, which introduces a notion of contract complexity based on the number of different payments specified by the contract, and~\citep{ho2016adaptive}, which develops a dynamic model where, in each round, the principal determines a contract and an agent chooses an action, resulting in a reward for the principal.
%
These works considerably depart form ours, as they study rather different models.
%
The first one considers an $n$-player normal-form framework in which actions are \emph{not} hidden.
%
The second work uses multi-armed bandit techniques, and, thus, the goal is to minimize the principal’s regret over time.


In conclusion, we also point out that considerable attention (especially in the economic literature) has been devoted to the study of some \emph{robustness} properties of linear contracts in classical principal-agent problems~\citep{carroll2015robustness,carroll2019robustness}.
%
This perspective has also been taken by~\citet{dutting2019simple} using a more computationally-oriented point of view.
%

\paragraph{Note on Concurrent Work by~\citet{guruganesh2020contracts}}
%
The work by~\citet{guruganesh2020contracts}, which has been developed independently and concurrently with ours, studies the same Bayesian principal-agent problem that we address in this paper.
%
\citet{guruganesh2020contracts} characterize worst-case multiplicative approximation guarantees of linear contracts, comparing them with some benchmarks (including optimal contracts).
%, focusing on multiplicative approximations.
%
Among the results they provide, the closest to ours are discussed in the following.
%
First, they show a tight approximation guarantee for linear contracts, which is linear in the number of agent's actions and logarithmic in the number of agent's types when all agent's types share the same costs, while, if they may have different costs, it is linear in the number of types and actions.
%
This result is similar to our result in Section~\ref{sec:linear_vs_optimal}, where we only consider the dependency on the number of types.
%
Second, they show the hardness of computing a single contract or a menu of contracts approximating the optimal principal's expected utility ip to within a given constant multiplicative factor.
%
In Section~\ref{sec:hardness}, we show a stronger result.
%
In particular, we prove the hardness of computing a single contract with a multiplicative loss sublinear in the number of types.
%
Finally, they show that an optimal contract can be computed efficiently if we fix either the number of agent’s types or the number of outcomes.
%
This is equivalent to our results in Section~\ref{sec:tractable}.
%
In conclusion, even though~\citet{guruganesh2020contracts} study the same principal-agent problem, they focus on the approximation guarantees of linear contracts with respect to optimal ones and other possible benchmarks, while our main focus is their relation with efficiently computable contracts.



