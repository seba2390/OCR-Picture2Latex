\section{Discussion}\label{sec:conclusions}


Despite principal-agent problems are ubiquitous in real-world economic scenarios, computational works on these problems appeared only recently and they are limited to specific settings~\citep{babaioff2012combinatorial,dutting2019simple,dutting2020complexity}.
%
In this paper, we introduce and study a new Bayesian principal-agent model in which the principal is uncertain about the agent's type.
%
This makes a considerable step over classical (non-Bayesian) principal-agent settings, as there are many real-world problems in which it is unreasonable to assume that the principal has complete knowledge of the agent.
%
Moreover, our Bayesian model begets new computational challenges that make it worth studying on its own, since, differently from the non-Bayesian case, in our setting a principal-optimal contract cannot be computed efficiently.


Linear contracts are the \emph{de facto} standard usually employed in real-world principal-agent problems, given their relative implementation simplicity, due to them being based on a pure-commission principle.
%
As a result, the research on principal-agent problems (mainly in economics, but also in computer science~\citep{dutting2019simple}) strived to find mathematical justifications of why linear contracts are so popular in practice.
%
Recently-developed studies show that, in non-Bayesian principal-agent settings, linear contracts are approximately optimal except in some degenerate situations~\citep{dutting2019simple} and that they enjoy some robustness properties~\citep{carroll2015robustness,carroll2019robustness,dutting2019simple}.
%
Our results further justify the use of linear contracts, showing that, in more realistic settings as those captured by our Bayesian model, they are the best among all the contracts that can be designed with bounded computationally resources.