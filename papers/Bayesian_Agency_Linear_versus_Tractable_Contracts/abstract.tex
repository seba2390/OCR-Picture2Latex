% Abstract. Note that this must come before \maketitle.
\begin{abstract}
	We study \emph{principal-agent} problems in which a principal commits to an outcome-dependent payment scheme (a.k.a. \emph{contract}) so as to induce an agent to take a costly, unobservable action.
	%
	We relax the assumption that the principal perfectly knows the agent by considering a \emph{Bayesian} setting where the agent's type is unknown and randomly selected according to a given probability distribution, which is known to the principal.
	%
	Each agent's type is characterized by her own action costs and action-outcome distributions.
	%
	In the literature on non-Bayesian principal-agent problems, considerable attention has been devoted to \emph{linear contracts}, which are simple, pure-commission payment schemes that still provide nice approximation guarantees with respect to principal-optimal (possibly non-linear) contracts.
	%
	While in non-Bayesian settings an optimal contract can be computed efficiently, this is no longer the case for our Bayesian principal-agent problems.
	%
	This further motivates our focus on {linear contracts}, which can be optimized efficiently given their single-parameter nature.
	%
	Our goal is to analyze the properties of linear contracts in Bayesian settings, in terms of approximation guarantees with respect to optimal contracts and general \emph{tractable contracts} (\emph{i.e.}, efficiently-computable ones).
	
	First, we study the approximation guarantees of linear contracts with respect to optimal ones, showing that the former suffer from a multiplicative loss that grows linearly in the number of agent's types.
	%
	Nevertheless, we prove that linear contracts can still provide a constant multiplicative approximation $\rho$ of the optimal principal's expected utility, though at the expense of an exponentially-small additive loss $2^{-\Omega(\rho)}$.
	%
	Then, we switch to tractable contracts, showing that, surprisingly, linear contracts perform well among them.
	%
	In particular, we prove that it is \textsf{NP}-hard to design a contract providing a multiplicative loss sublinear in the number of agent's types, while the same holds for contracts that provide a constant multiplicative approximation $\rho$ at the expense of an additive loss $2^{-\omega(\rho)}$.
	%
	We conclude by showing that, in Bayesian principal-agent problems, an optimal contract can be computed efficiently if we fix either the number of agent's types or the number of outcomes.
\end{abstract}