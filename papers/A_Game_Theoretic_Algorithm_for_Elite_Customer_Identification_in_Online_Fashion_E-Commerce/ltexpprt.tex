%%%%%%%%%%%%%%%%%%%%%%%%%%  ltexpprt.tex  %%%%%%%%%%%%%%%%%%%%%%%%%%%%%%%%
%
% This is ltexpprt.tex, an example file for use with the SIAM LaTeX2E
% Preprint Series macros. It is designed to provide double-column output.
% Please take the time to read the following comments, as they document
% how to use these macros. This file can be composed and printed out for
% use as sample output.

% Any comments or questions regarding these macros should be directed to:
%
%                 Donna Witzleben
%                 SIAM
%                 3600 University City Science Center
%                 Philadelphia, PA 19104-2688
%                 USA
%                 Telephone: (215) 382-9800
%                 Fax: (215) 386-7999
%                 e-mail: witzleben@siam.org


% This file is to be used as an example for style only. It should not be read
% for content.

%%%%%%%%%%%%%%% PLEASE NOTE THE FOLLOWING STYLE RESTRICTIONS %%%%%%%%%%%%%%%

%%  1. There are no new tags.  Existing LaTeX tags have been formatted to match
%%     the Preprint series style.
%%
%%  2. Do not change the margins or page size!  Do not change from the default 
%%     text font!
%%
%%  3. You must use \cite in the text to mark your reference citations and
%%     \bibitem in the listing of references at the end of your chapter. See
%%     the examples in the following file. If you are using BibTeX, please
%%     supply the bst file with the manuscript file.
%%
%%  4. This macro is set up for two levels of headings (\section and
%%     \subsection). The macro will automatically number the headings for you.
%%
%%  5. No running heads are to be used for this volume.
%%
%%  6. Theorems, Lemmas, Definitions, Equations, etc. are to be double numbered,
%%     indicating the section and the occurrence of that element
%%     within that section. (For example, the first theorem in the second
%%     section would be numbered 2.1. The macro will
%%     automatically do the numbering for you.
%%
%%  7. Figures and Tables must be single-numbered.
%%     Use existing LaTeX tags for these elements.
%%     Numbering will be done automatically.
%%
%%  8. Page numbering is no longer included in this macro.
%%     Pagination will be set by the program committee.
%%
%%
%%%%%%%%%%%%%%%%%%%%%%%%%%%%%%%%%%%%%%%%%%%%%%%%%%%%%%%%%%%%%%%%%%%%%%%%%%%%%%%



\documentclass[twoside,leqno,twocolumn]{article}

% Comment out the line below if using A4 paper size
\usepackage[letterpaper]{geometry}
\usepackage{multirow}
\usepackage{longtable}
% \usepackage{float}
% \restylefloat{table}
\usepackage{ltexpprt}
\usepackage{amsmath,amssymb,mathtools}
\usepackage{algorithm}
\usepackage[noend]{algpseudocode}
\usepackage{multirow,array}

\providecommand{\keywords}[1]
{
\textbf{Keywords ---} #1
}

\begin{document}


%\setcounter{chapter}{2} % If you are doing your chapter as chapter one,
%\setcounter{section}{3} % comment these two lines out.

\title{\Large A Game Theoretic Algorithm for Elite Customer Identification in Online Fashion E-Commerce
% \thanks{Supported by GSF grants ABC123, DEF456, and GHI789.}
}
\author{
Chandramouli K\thanks{Myntra Designs Pvt. Ltd.}
\and Gopinath A\thanks{Myntra Designs Pvt. Ltd.}
\and Girish Satyanarayana \thanks{Myntra Designs Pvt. Ltd.}
\and Ravindra babu Tallamraju \thanks{Myntra Designs Pvt. Ltd.}
}

\date{}

\maketitle

% Copyright Statement
% When submitting your final paper to a SIAM proceedings, it is requested that you include 
% the appropriate copyright in the footer of the paper.  The copyright added should be 
% consistent with the copyright selected on the copyright form submitted with the paper.
% Please note that "20XX" should be changed to the year of the meeting.

% Default Copyright Statement
\fancyfoot[R]{\scriptsize{Copyright \textcopyright\ 2021 by SIAM\\
Unauthorized reproduction of this article is prohibited}}

% Depending on which copyright you agree to when you sign the copyright form, the copyright 
% can be changed to one of the following after commenting out the default copyright statement
% above.

%\fancyfoot[R]{\scriptsize{Copyright \textcopyright\ 2021\\
%Copyright for this paper is retained by authors}}

%\fancyfoot[R]{\scriptsize{Copyright \textcopyright\ 2021\\
%Copyright retained by principal author's organization}}

%\pagenumbering{arabic}
%\setcounter{page}{1}%Leave this line commented out.

\begin{abstract} \small\baselineskip=9pt 
Myntra is an online fashion e-commerce company based in India. At Myntra, a market leader in fashion e-commerce in India, customer experience is paramount and a significant portion of our resources are dedicated to it. Here we describe an algorithm that identifies eligible customers to enable preferential product return processing for them by Myntra. We declare the group of aforementioned eligible customers on the platform as elite customers. Our algorithm to identify eligible/elite customers is based on sound principles of game theory. It is simple, easy to implement and scalable.
\end{abstract}

\vspace{0.05cm}
\keywords{\begin{small} Game Theory, Nash Equilibrium, E-Commerce, Customer Experience, Product return\end{small}}



\IEEEraisesectionheading{\section{Introduction}}

\IEEEPARstart{V}{ision} system is studied in orthogonal disciplines spanning from neurophysiology and psychophysics to computer science all with uniform objective: understand the vision system and develop it into an integrated theory of vision. In general, vision or visual perception is the ability of information acquisition from environment, and it's interpretation. According to Gestalt theory, visual elements are perceived as patterns of wholes rather than the sum of constituent parts~\cite{koffka2013principles}. The Gestalt theory through \textit{emergence}, \textit{invariance}, \textit{multistability}, and \textit{reification} properties (aka Gestalt principles), describes how vision recognizes an object as a \textit{whole} from constituent parts. There is an increasing interested to model the cognitive aptitude of visual perception; however, the process is challenging. In the following, a challenge (as an example) per object and motion perception is discussed. 



\subsection{Why do things look as they do?}
In addition to Gestalt principles, an object is characterized with its spatial parameters and material properties. Despite of the novel approaches proposed for material recognition (e.g.,~\cite{sharan2013recognizing}), objects tend to get the attention. Leveraging on an object's spatial properties, material, illumination, and background; the mapping from real world 3D patterns (distal stimulus) to 2D patterns onto retina (proximal stimulus) is many-to-one non-uniquely-invertible mapping~\cite{dicarlo2007untangling,horn1986robot}. There have been novel biology-driven studies for constructing computational models to emulate anatomy and physiology of the brain for real world object recognition (e.g.,~\cite{lowe2004distinctive,serre2007robust,zhang2006svm}), and some studies lead to impressive accuracy. For instance, testing such computational models on gold standard controlled shape sets such as Caltech101 and Caltech256, some methods resulted $<$60\% true-positives~\cite{zhang2006svm,lazebnik2006beyond,mutch2006multiclass,wang2006using}. However, Pinto et al.~\cite{pinto2008real} raised a caution against the pervasiveness of such shape sets by highlighting the unsystematic variations in objects features such as spatial aspects, both between and within object categories. For instance, using a V1-like model (a neuroscientist's null model) with two categories of systematically variant objects, a rapid derogate of performance to 50\% (chance level) is observed~\cite{zhang2006svm}. This observation accentuates the challenges that the infinite number of 2D shapes casted on retina from 3D objects introduces to object recognition. 

Material recognition of an object requires in-depth features to be determined. A mineralogist may describe the luster (i.e., optical quality of the surface) with a vocabulary like greasy, pearly, vitreous, resinous or submetallic; he may describe rocks and minerals with their typical forms such as acicular, dendritic, porous, nodular, or oolitic. We perceive materials from early age even though many of us lack such a rich visual vocabulary as formalized as the mineralogists~\cite{adelson2001seeing}. However, methodizing material perception can be far from trivial. For instance, consider a chrome sphere with every pixel having a correspondence in the environment; hence, the material of the sphere is hidden and shall be inferred implicitly~\cite{shafer2000color,adelson2001seeing}. Therefore, considering object material, object recognition requires surface reflectance, various light sources, and observer's point-of-view to be taken into consideration.


\subsection{What went where?}
Motion is an important aspect in interpreting the interaction with subjects, making the visual perception of movement a critical cognitive ability that helps us with complex tasks such as discriminating moving objects from background, or depth perception by motion parallax. Cognitive susceptibility enables the inference of 2D/3D motion from a sequence of 2D shapes (e.g., movies~\cite{niyogi1994analyzing,little1998recognizing,hayfron2003automatic}), or from a single image frame (e.g., the pose of an athlete runner~\cite{wang2013learning,ramanan2006learning}). However, its challenging to model the susceptibility because of many-to-one relation between distal and proximal stimulus, which makes the local measurements of proximal stimulus inadequate to reason the proper global interpretation. One of the various challenges is called \textit{motion correspondence problem}~\cite{attneave1974apparent,ullman1979interpretation,ramachandran1986perception,dawson1991and}, which refers to recognition of any individual component of proximal stimulus in frame-1 and another component in frame-2 as constituting different glimpses of the same moving component. If one-to-one mapping is intended, $n!$ correspondence matches between $n$ components of two frames exist, which is increased to $2^n$  for one-to-any mappings. To address the challenge, Ullman~\cite{ullman1979interpretation} proposed a method based on nearest neighbor principle, and Dawson~\cite{dawson1991and} introduced an auto associative network model. Dawson's network model~\cite{dawson1991and} iteratively modifies the activation pattern of local measurements to achieve a stable global interpretation. In general, his model applies three constraints as it follows
\begin{inlinelist}
	\item \textit{nearest neighbor principle} (shorter motion correspondence matches are assigned lower costs)
	\item \textit{relative velocity principle} (differences between two motion correspondence matches)
	\item \textit{element integrity principle} (physical coherence of surfaces)
\end{inlinelist}.
According to experimental evaluations (e.g.,~\cite{ullman1979interpretation,ramachandran1986perception,cutting1982minimum}), these three constraints are the aspects of how human visual system solves the motion correspondence problem. Eom et al.~\cite{eom2012heuristic} tackled the motion correspondence problem by considering the relative velocity and the element integrity principles. They studied one-to-any mapping between elements of corresponding fuzzy clusters of two consecutive frames. They have obtained a ranked list of all possible mappings by performing a state-space search. 



\subsection{How a stimuli is recognized in the environment?}

Human subjects are often able to recognize a 3D object from its 2D projections in different orientations~\cite{bartoshuk1960mental}. A common hypothesis for this \textit{spatial ability} is that, an object is represented in memory in its canonical orientation, and a \textit{mental rotation} transformation is applied on the input image, and the transformed image is compared with the object in its canonical orientation~\cite{bartoshuk1960mental}. The time to determine whether two projections portray the same 3D object
\begin{inlinelist}
	\item increase linearly with respect to the angular disparity~\cite{bartoshuk1960mental,cooperau1973time,cooper1976demonstration}
	\item is independent from the complexity of the 3D object~\cite{cooper1973chronometric}
\end{inlinelist}.
Shepard and Metzler~\cite{shepard1971mental} interpreted this finding as it follows: \textit{human subjects mentally rotate one portray at a constant speed until it is aligned with the other portray.}



\subsection{State of the Art}

The linear mapping transformation determination between two objects is generalized as determining optimal linear transformation matrix for a set of observed vectors, which is first proposed by Grace Wahba in 1965~\cite{wahba1965least} as it follows. 
\textit{Given two sets of $n$ points $\{v_1, v_2, \dots v_n\}$, and $\{v_1^*, v_2^* \dots v_n^*\}$, where $n \geq 2$, find the rotation matrix $M$ (i.e., the orthogonal matrix with determinant +1) which brings the first set into the best least squares coincidence with the second. That is, find $M$ matrix which minimizes}
\begin{equation}
	\sum_{j=1}^{n} \vert v_j^* - Mv_j \vert^2
\end{equation}

Multiple solutions for the \textit{Wahba's problem} have been published, such as Paul Davenport's q-method. Some notable algorithms after Davenport's q-method were published; of that QUaternion ESTimator (QU\-EST)~\cite{shuster2012three}, Fast Optimal Attitude Matrix \-(FOAM)~\cite{markley1993attitude} and Slower Optimal Matrix Algorithm (SOMA)~\cite{markley1993attitude}, and singular value decomposition (SVD) based algorithms, such as Markley’s SVD-based method~\cite{markley1988attitude}. 

In statistical shape analysis, the linear mapping transformation determination challenge is studied as Procrustes problem. Procrustes analysis finds a transformation matrix that maps two input shapes closest possible on each other. Solutions for Procrustes problem are reviewed in~\cite{gower2004procrustes,viklands2006algorithms}. For orthogonal Procrustes problem, Wolfgang Kabsch proposed a SVD-based method~\cite{kabsch1976solution} by minimizing the root mean squared deviation of two input sets when the determinant of rotation matrix is $1$. In addition to Kabsch’s partial Procrustes superimposition (covers translation and rotation), other full Procrustes superimpositions (covers translation, uniform scaling, rotation/reflection) have been proposed~\cite{gower2004procrustes,viklands2006algorithms}. The determination of optimal linear mapping transformation matrix using different approaches of Procrustes analysis has wide range of applications, spanning from forging human hand mimics in anthropomorphic robotic hand~\cite{xu2012design}, to the assessment of two-dimensional perimeter spread models such as fire~\cite{duff2012procrustes}, and the analysis of MRI scans in brain morphology studies~\cite{martin2013correlation}.

\subsection{Our Contribution}

The present study methodizes the aforementioned mentioned cognitive susceptibilities into a cognitive-driven linear mapping transformation determination algorithm. The method leverages on mental rotation cognitive stages~\cite{johnson1990speed} which are defined as it follows
\begin{inlinelist}
	\item a mental image of the object is created
	\item object is mentally rotated until a comparison is made
	\item objects are assessed whether they are the same
	\item the decision is reported
\end{inlinelist}.
Accordingly, the proposed method creates hierarchical abstractions of shapes~\cite{greene2009briefest} with increasing level of details~\cite{konkle2010scene}. The abstractions are presented in a vector space. A graph of linear transformations is created by circular-shift permutations (i.e., rotation superimposition) of vectors. The graph is then hierarchically traversed for closest mapping linear transformation determination. 

Despite of numerous novel algorithms to calculate linear mapping transformation, such as those proposed for Procrustes analysis, the novelty of the presented method is being a cognitive-driven approach. This method augments promising discoveries on motion/object perception into a linear mapping transformation determination algorithm.



\section{Background and Related Work}~\label{sec:background}
%This section presents the background on MDE, ML, and MDE for systems with ML components. We further present the related work on the existing secondary and relevant studies.
\subsection{Model-driven Engineering}~\label{subsec:MDEBackground}
%The word \textit{model} originates from the Latin word \textit{modulus}, which means a measure, pattern, or example to follow~\cite{ludewig2003models}. 
%While modeling is relatively new to software engineering, it has been successfully applied for a long time in several traditional engineering domains~\cite{selic2012will,bucchiarone2020grand}. 
Model-driven Engineering (MDE) is a software development methodology that relies on models as the primary artifacts that drive the development process~\cite{ciccozzi2019execution, almonte2021recommender,hutchinson2011model}. This differs from traditional software development processes such as waterfall and agile, where the focus is on development phases like requirements engineering, design, and implementation, and models are only used as auxiliary artifacts to support these activities and serve as documentation~\cite{ciccozzi2019execution}. 
%In contrast to traditional software engineering using waterfall or agile methodology, where the focus is on the different phases of development, e.g., requirements engineering, design, implementation, and quality assurance, and the models are used to aid in requirements analysis or design, in MDE models are the primary artifact. 
The focus of MDE is on the continual refinement and transformation of models, beginning with computation-independent models (CIMs), to platform-independent models (PIMs) and then platform-specific models (PSMs)~\cite{brambilla2017model}. Finally, these models are transformed into code, documentation, configurations, and tests for the software system.

MDE relies on two key aspects: abstraction and automation~\cite{mohagheghi2009mde}. Models are abstractions of complex entities; they hide unwanted information so modelers can easily focus on areas of interest~\cite{schmidt2006model, brambilla2017model}. 
%Currently, MDE is the state-of-the-art in software abstraction~\cite{hutchinson2011model} by reducing complexity and offering a more intuitive and natural way to define software compared to programming languages~\cite{ciccozzi2019execution}. 
In MDE, models are automatically transformed into artifacts such as code, documentation, and other models to achieve various goals such as merging, translation, refinement, refactoring, or alignment~\cite{brambilla2017model}. These transformations help reduce developers' manual effort and production time by generating executable artifacts -- leading to improved software quality, reduced complexity, and decreased development time and effort~\cite{kelly2008domain}. There are two types of transformations in MDE: 1) Model-to-Text (M2T) transformations, for a given input model a M2T transformation produces a textual artifact such as code or documentation as output; and ) Model-to-model (M2M) transformations, for a given input model an M2M transformation produces a different kind of model, for example translating a model from one language to another~\cite{brambilla2017model}.

A model is created in a modeling language, conforming to a meta-model that defines the syntax and semantics of that language. There are two types of modeling languages: general-purpose languages (GPL) and domain-specific languages (DSL). GPLs are intended for modeling generic concepts applicable to multiple domains; some examples include the Unified Modeling Language (UML)~\cite{eriksson2003uml}, Petri-nets~\cite{peterson1977petri} and finite state machines~\cite{wagner2006modeling}. On the other hand, a DSL has modeling concepts tailored to a specific domain or context, like SysML for embedded systems, HTML for web page development, and SQL for database queries~\cite{brambilla2017model}.

While exploring the literature, one encounters terms similar to MDE: examples include model-driven architecture (MDA), model-driven development (MDD), and model-based engineering (MBE). MDA is an architectural standard~\cite{mda} developed by the Object Management Group (OMG) \cite{omg} for MDD. MDD refers to automatically generating artifacts from models, whereas MDE has a broader scope and includes analysis, validation~\cite{almonte2021recommender}, interoperability of artifacts and reverse engineering \cite{brambilla2017model}. MBE is a lighter version of MDE, where models are not necessarily the central focus of the engineering process; however, they provide critical support~\cite{brambilla2017model}. This SLR primarily focuses on MDE.

\subsection{Machine Learning}
Machine Learning (ML) is a branch of Artificial Intelligence (AI) that enables machines to learn patterns from data without being explicitly programmed~\cite{samuel1959machine}. ML algorithms are fed with existing data to \textit{train} them and produce an ML model. This trained ML model then has the capability to \textit{infer}, i.e., predict outcomes for new data inputs or also commonly known as \emph{ML model inference}~\cite{mueller2021machine}. For example, an ML model trained on stock prices for a company till September 2023 can predict stock prices in the following months. ML is preferable when solving problems that would require very complex and difficult-to-maintain traditional algorithms~\cite{geron2022hands}. Since ML algorithms can learn autonomously, they reduce complexity and facilitate easier maintenance~\cite{geron2022hands}. This ability of ML to minimise complexity, learn from changing data, and make future predictions is immensely valuable for businesses~\cite{lee2020machine}. According to a recent survey~\cite{rackspace2023report}, organizations report that applying ML increases employee efficiency by 20\%, innovation by 17\%, and lowers costs by 16\% -- leading to increased adoption of ML in practical settings~\cite{rackspace2023report}.

ML can further be divided into three broad categories: supervised learning, unsupervised learning, and reinforcement learning. The most suitable ML approach depends on the specific problem and data.
%
Supervised learning is when an ML algorithm is trained on a labeled dataset that has labels to define the meaning of data~\cite{mueller2021machine}. For example, a dataset with images labeled as ``cat'' or ``not cat'' images. Supervised learning algorithms learn to make classifications or predictions by learning patterns and relationships in labeled data~\cite{lee2020machine,mueller2021machine}. When the labels are discrete, this is known as \textit{classification} and when labels are continuous, this is known as \textit{regression}~\cite{mueller2021machine}. Once the algorithm is trained, the performance is evaluated on unseen or test data. Some popular supervised learning algorithms include linear regression, decision trees, naive Bayes classifier, support vector machines (SVM), random forest, and artificial neural networks (ANNs)~\cite{lee2020machine}. Supervised model applications include fraud detection and recommender systems~\cite{mueller2021machine}. 

Unsupervised learning is when an ML algorithm is trained on an unlabeled dataset with few or no labels to define the meaning of data~\cite{mueller2021machine,lee2020machine}. Unsupervised learning algorithms attempt to understand hidden patterns in data and group similar data together creating a classification of the data~\cite{mueller2021machine}. Unsupervised learning works without any guidance, hence it is most suitable for large volumes of data when classifications are unknown and data cannot be labeled~\cite{mueller2021machine}. Evaluating the performance of such algorithms can be challenging due to the lack of ground truth. Some popular unsupervised techniques include clustering, k-means, principal component analysis, and association rules~\cite{lee2020machine}. Applications of unsupervised models include customer segmentation and clustering user reviews~\cite{mueller2021machine}.

 Reinforcement learning is when an ML algorithm receives feedback on actions to guide the behavior toward an optimal outcome~\cite{mueller2021machine, lee2020machine}. Reinforcement learning algorithms are not trained with datasets; instead, they learn from trial and error in a simulated environment or a real-world environment~\cite{mueller2021machine}. Desired behaviors are rewarded and reinforcement learning algorithms attempt to maximize rewards through successful decisions~\cite{lee2020machine,mueller2021machine}. These algorithms are most suitable when sequential decision-making is required, interaction with an environment is possible and feedback is available. %However, reinforcement learning can be expensive since the algorithms require a large number of interactions with the environment to learn effectively. 
 Some popular reinforcement learning algorithms are Q-learning, temporal difference learning, hierarchal reinforcement learning, and policy gradient~\cite{lee2020machine}. Applications of reinforcement learning include robotics, self-driving cars, and game playing~\cite{lee2020machine}.
 
\subsection{Model-driven Engineering for Machine Learning (MDE4ML)}
%Models are a significant element of both MDE and ML. In MDE, models describe software systems in all phases of their life-cycle: requirements, design, implementation, testing and evolution~\cite{moin2022model}. ML models are mathematical models that learn patterns in data to make predictions~\cite{moin2022model}. 
Developing and managing systems with ML models and components is challenging. 
Some aspects of this complexity are immature requirements specification~\cite{kuwajima2020engineering, ahmad2023requirements}, constantly evolving data~\cite{baumann2022dynamic}, lack of ML domain knowledge~\cite{yohannis2022towards}, integration with traditional software \cite{atouani2021artifact}, responsible use of ML~\cite{yohannis2022towards}, and deployment and maintenance of ML models~\cite{kourouklidis2021model, langford2021modalas}. 

These complexities introduce several challenges. For example, Nils Baumann et al.~\cite{baumann2022dynamic} describe how challenging it is to handle changing datasets; ML engineers have to manually merge new and old datasets and re-train the entire ML model;  Benjamin Jahi et al. \cite{jahic2023semkis} point out how challenging it is to describe the dataset and neural network requirements to satisfy customer expectations;  Benjamin Benni et al. \cite{benni2019devops} state how the development of a correct ML pipeline is a highly demanding task, data scientists must have knowledge and experience to go through numerous data pre-processing and ML models to select the best one; and Kaan Koseler et al. \cite{koseler2019realization} mention the difficulties developers face when attempting to use ML techniques with big data, developers need to acquire knowledge of the problem space, domain and ML concepts. There is a need for solutions to efficiently and effectively address these challenges~\cite{raedler2023model}.
%All these challenges point towards the need for a technique that can efficiently and effectively address them.

A synergy between MDE and ML development exists, where software models are leveraged to drive the development and management of ML components~\cite{safdar2022modlf, yohannis2022towards, kourouklidis2021model}. This should not to be confused with AI or ML for MDE (AI4MDE), where intelligent agents and recommenders support users in modeling and related activities \cite{almonte2021recommender, gil2021artificial, boubekeur2020towards, saini2019teaching}. The application of MDE for ML-based systems (MDE4ML) offers many potential benefits to developers, such as reduced complexity~\cite{kourouklidis2021model, bucchiarone2020grand}, development effort, and time~\cite{yohannis2022towards,gatto2019modeling}. Domain experts, software engineers and ML novices can also take advantage of ML through the abstraction and automation of MDE \cite{shi2022feature,moin2022supporting, bucchiarone2020grand}. Additionally, MDE can also improve the quality of the ML-based system through easier maintainability, scalability~\cite{selic2003pragmatics}, reusability, and interoperability~\cite{brambilla2017model}.

\subsection{Key MDE4ML Related Work}
MDE4ML has received growing attention from researchers in recent years. We found six relevant secondary studies comprising SLRs, scoping reviews, and surveys. In their SLR \cite{raedler2023model}, the authors identify 15 primary studies on MDE for AI and analyze them with respect to MDE practices for the development of AI systems and the stages of AI development aligned with CRoss Industry Standard Process for Data Mining (CRISP-DM) \cite{wirth2000crisp} methodology. However, this study only considers a small subset of studies and performs a shallow analysis with no details about goals, end-users, types of models, implemented tools, and evaluation. A second SLR \cite{zafar2017systematic} reviews 24 papers on MDE for ML in the context of Big data analytics. This study has a narrower scope compared to ours and provides only a brief overview of the models, approaches, tools, and frameworks in the studies. In a third SLR \cite{li2022can}, 31 studies on no/low code platforms for ML applications are reviewed. This study is limited to no/low code approaches and therefore misses out on many other MDE for ML studies. A scoping review is presented in \cite{mardani2023model} on MDE for ML in IoT applications. The study examines 68 studies in depth; however, the review focuses more on MDE for IoT applications and only four of the selected studies apply ML techniques. A preliminary survey on DSLs for ML in Big data is presented in \cite{portugal2016preliminary}, with an extended version in \cite{portugal2016survey}. These surveys do not follow a systematic review process, include studies only for big data, and briefly highlight the DSLs and frameworks in the studies. From the analysis of existing literature, we found that the available secondary studies consist of limited subsets of papers on MDE for ML, lack analysis of key areas like goals, end-users, ML aspects, MDE approach details, evaluation methods, and limitations, and often do not follow a systematic and rigorous review process. Therefore, we aim to address these gaps in this SLR.


\section{Problem Statement}
At Myntra, a customer's product return request is processed in the following manner. Once a customer places a product return request, a doorstep pickup agent arrives and receives the product. After a quality check of the returned item at a designated location refund to the customer is initiated.

For a better customer experience we wish to identify elite customers and do away with this quality check process for these elite customers. To identify these elite customers, we model the interaction between Myntra and a generic customer as a game.

We have two players in our game i.e. $N=\{ \text{Myntra}, \text{customer}\}$. Myntra has two actions i.e. $A_{\text{Myntra}}=\{\text{immediate refund},\text{no immediate refund}\}$. Similarly customer has two actions i.e. $A_{\text{customer}}=\{\text{comply with return requirements}, \\ \text{don't comply with return requirements}\}$.
Here in the context of Myntra the strategy/action ``immediate refund" refers to the removal of quality check process.

We designed the utility matrix  given in Table \ref{utility_matrix} for this game.
Our rationality behind this particular choice for the utility matrix is as follows. For Myntra, with $u_{\text{Myntra}}\text{(No Immediate Refund, Don't Comply)}=0$ as baseline, $u_{\text{Myntra}}\text{(Immediate Refund, Comply)}=1$, 
$u_{\text{Myntra}}\text{(No Immediate Refund, Comply)}=2$ and $u_{\text{Myntra}}\text{(Immediate Refund, Don't Comply)}=-1$ reflect the satisfaction levels of Myntra in the customer-Myntra relationship. We assume the same for the customer and obtain the utility matrix shown in Table \ref{utility_matrix}.
  \begin{table}
    \setlength{\extrarowheight}{2pt}
    \begin{tabular}{*{4}{c|}}
      \multicolumn{2}{c}{} & \multicolumn{2}{c}{Customer}\\\cline{3-4}
      \multicolumn{1}{c}{} &  & Comply  & Don't Comply \\\cline{2-4}
      \multirow{2}*{Myntra}  & Immediate &    &  \\
      & Refund & $(1,1)$ & $(-1,2)$ \\\cline{2-4}
      & No Immediate &      &  \\
      & Refund & $(2,-1)$ & $(0,0)$ \\\cline{2-4}
    \end{tabular}
     \caption{Utility matrix of our game}
      \label{utility_matrix}
  \end{table}
 \vspace*{-\baselineskip}
 \section{Solution}
 Observe that (No Immediate Refund, Don't Comply) is a pure strategy Nash equilibrium for this game and No Immediate Refund is the corresponding strategy for Myntra. However we note that the relation between Myntra and the customer is an ongoing relation. Hence a repeated game that repeats with probability $\delta$ is a more appropriate model for Myntra-customer relationship.
 In this context the set of strategy vectors is given by $S= (A_{\text{Myntra}} \times A_{\text{customer}})^{\mathbb{N}}$, set of sequences with values in $A_{\text{Myntra}} \times A_{\text{customer}}.$ Here $\mathbb{N}$ denotes the set of Natural numbers. The utility function $v_{\text{Myntra}}:S\rightarrow \mathbb{R}$ is given by 
 $v_{\text{Myntra}}(s)=\sum_{k\geq0}\delta^{k} u_{\text{Myntra}}(s^{k+1})$ where $s=(s^{1},s^{2},\cdots)$ with $s^{k} \in A_{\text{Myntra}} \times A_{\text{customer}}, k \in \mathbb{N}$ and $u_{\text{Myntra}}$ is given by Table \ref{utility_matrix}. Similarly the utility function 
 $v_{\text{customer}}:S\rightarrow \mathbb{R}$ is given by 
 $v_{\text{customer}}(s)=\sum_{k\geq0}\delta^{k} u_{\text{customer}}(s^{k})$ where $s=(s^{1},s^{2},\cdots)$ with $s^{k} \in A_{\text{Myntra}} \times A_{\text{customer}}, k \in \mathbb{N}$ and $u_{\text{customer}}$ is given by Table \ref{utility_matrix}.
 
 Observe that in this setup the strategy vector $s^{*}=(s^{*i})_{i\in \mathbb{N}}$ with $s^{*i}=(\text{no immediate refund},\text{don't comply})$ is an equilibrium strategy vector with $(v_{\text{Myntra}},v_{\text{customer}})=(0,0)$ as any unilateral deviation in any repetition of the game by a player diminishes the utility of the player. Moreover the equilibrium strategy ``no immediate refund" is currently employed by Myntra.
 This strategy however is customer independent and treats all customers alike and does not enable identification of elite customers as well as preferential returns processing for elite customers.

 Consider the following strategy $s^*=(s^{*1},s^{*2},\cdots)$ with $s^{*1}$=(immediate refund, comply) and $s^{*k}=s^{*(k-1)}$ if $s^{*(k-1)}$=(immediate refund, comply) else (no immediate refund, don't comply), that is cooperate to start with and don't cooperate once non-cooperation is observed. We note that for this strategy the utility obtained is given by
 \begin{align*}
  &(v_{\text{Myntra}},v_{\text{customer}}) \\
 =&\left(\displaystyle\sum_{k\geq0}\delta^{k} u_{\text{Myntra}}(s^{*k}),\displaystyle\sum_{k\geq0}\delta^{k} u_{\text{customer}}(s^{*k})\right) \\
 =&\left(\displaystyle\sum_{k\geq0}\delta^{k}1^k,\displaystyle\sum_{k\geq0}\delta^{k}1^k\right)
 =\left(\frac{1}{1-\delta},\frac{1}{1-\delta}\right)
 \end{align*}
 Now for any unilateral deviation by a player, at any game repetition stage $k$, the utility $v_{i}=\sum_{0\leq l\leq k-1}\delta^{l}1^l+2\delta^{k}, i\in N$. For this $s^*$ to be an equilibrium, from the definition, we require 
 \begin{align*}
    \displaystyle\sum_{0\leq l\leq k-1}\delta^{l}1^l+2\delta^{k} &\leq \frac{1}{1-\delta} \\
    \implies \frac{1-\delta^{k}}{1-\delta}+2\delta^{k} &\leq \frac{1}{1-\delta}
    \implies  \delta \geq \frac{1}{2}
 \end{align*}
 Hence our strategy $s^*$ is a pure strategy Nash equilibrium if the probability of repetition of the game is at least $0.5$.
 
 On a separate note, a relevant equilibrium concept for the case of repeated games is subgame perfect Nash equilibrium \cite{YN}. We also note that $s^*$ can be shown to be a subgame perfect Nash equilibrium. 
 
 We note here an important observation. Our model of Myntra-customer relationship as a repeated game apart from explaining current Myntra strategy as an equilibrium strategy-there by validating the choice of utlity matrix- suggests an alternative equilibrium strategy vector given by $s^*$. 
 
 We also note here that there are other equilibrium strategies as well. For e.g.
 $s=(s^1,s^2,\cdots)$ with $s^1=$(immediate refund, comply) and $s^{k}=$(immediate refund, comply) if $s^{k-1}=$(immediate refund, comply) or (no immediate refund, does not comply) else $s^{k}=$(no immediate refund, don't comply) is an equilibrium strategy provided $\delta=1$ and each such equilibrium strategy gives an algorithm to identify elite customers. Here we chose a strategy that is most cautious from the point of view of Myntra.
 
 
 With this analysis in place we design the following algorithm that enables us to identify elite customers
%  Note that the set of strategy vectors in our one period game is $A=\{(I,C)(I,DC)(NI,C)(NI,DC)\}$. A strategy vector in $S$ is subgame perfect Nash equilibrium if it induces Nash equilibrium in every subgame.
 
\title{Multi-step  Rolling  Prediction  algorithm   }
\documentclass[12pt]{article}
\usepackage{fullpage}
\usepackage{times}
\usepackage{fancyhdr,graphicx,amsmath,amssymb}
\usepackage[ruled,vlined]{algorithm2e}
\include{pythonlisting}
\begin{document} 
\begin{algorithm}[H]
\SetAlgoLined
Input: train, test\;
Output: predictions\;
history $\leftarrow$ train\;
model1$\leftarrow$ARIMA-GARCH(history, order=(p,d,q)(P,Q))\;
 $N_t$ $\leftarrow$train - model.fittedvalues\;
model2$\leftarrow$SVR($N_t$)\;
for{\State each t  in  range(length(test))} do{
   hat1$\leftarrow$model1.forecast()\;
   hat2$\leftarrow$model2.forecast()\;
   prediction1.append(hat1)\;
   prediction2.append(hat2)\;
   history.append(hat1)\;
   $N_t$.append(hat2)\;
   model1$\leftarrow$ARIMA-GARCH(history, order=(p,d,q)(P,Q))\;
   model2$\leftarrow$SVR($N_t$)\;
 }
 Return prediction = prediction1+prediction2\;
 
 
 \caption{MRPA}
\end{algorithm}


\end{document}

\section{Experiments}\label{sec:expts}

\iffalse
\begin{center}
 \begin{tabular}{|| c | c | c ||} 
 \hline
 Task & RNN(Relu) & RNN(Tanh)  \\ [0.5ex] 
 \hline\hline
  Counter-2 & 0.15 & 0.1\\
  Counter-3 & 0.02 & 0.02\\
  Counter-4 & 0.06 & 0.02 \\
  Boolean-3 & 0.99 & 0.756 \\
  Boolean-5 & 0.0 & 0.766\\
  Shuffle-2 & 0.966 & 0.60\\
  Shuffle-4 & 0.599 & 0.202\\
  Shuffle-6 & 0.348 & 0.338\\ [1ex]
 \hline
\end{tabular}
\end{center}
\fi 


%\iffalse
\begin{figure}[!ht]
\centering
\iffalse
\includegraphics[width=.45\textwidth]{../ESN_RNN_normalizedseq/RNN_2.png}
\includegraphics[width=.45\textwidth]{../ESN_RNN_normalizedseq/RNN_4.png}
\includegraphics[width=.45\textwidth]{../ESN_RNN_normalizedseq/RNN_8.png}
\fi
%\iffalse
\begin{subfigure}
  \centering
  \includegraphics[width=0.5\linewidth]{ESN_RNN_normalizedseq/RNN_2.png}
%          \vspace{-2\baselineskip}
  \caption{Data dimension: 2}
\end{subfigure}
\begin{subfigure}
  \centering
  \includegraphics[width=0.5\linewidth]{ESN_RNN_normalizedseq/RNN_4.png}
%\vspace{-2\baselineskip}
  \caption{Data dimension: 4}
\end{subfigure}%\hfill
\begin{subfigure}
  \centering
  \includegraphics[width=0.5\linewidth]{ESN_RNN_normalizedseq/RNN_8.png}
%\vspace{-2\baselineskip}
  \caption{Data dimension: 8}
\end{subfigure}%
%\fi
\caption{Invertibitiliy of RNNs at random initialization: Checking behavior of inversion error with number of neurons and the sequence length at different data dimensions.}
\label{fig:RNN_inver}
\end{figure}
%\fi 
\textbf{RNN inversion at random initialization.} We consider a randomly initialized RNN, with the entries of the weights $\mathbf{W}$ and $\mathbf{A}$ randomly picked from the distribution $\mathcal{N}(0, 1)$. Sequences are generated i.i.d. from normal distribution i.e. for each sequence, $\bx^{(i)} \sim N(0, \mathbf{I})$ for each $i \in [L]$. We use SGD with batch size 128, momentum $0.9$ and learning rate $0.1$ to compute the linear matrix $\obW^{[L]}$ so that $\norm[0]{\obW^{[L]} \mathbf{h}^{(L)} - [\bx^{(1)}, \ldots, \bx^{(L)}]}^2$ is minimized. We compute the following two quantities on the test dataset, containing $1000$ sequences: average $L_2$ error given by $\mathbb{E}_{\bx} \frac{\norm[0]{\obW^{[L]} \mathbf{h}^{(L)} - [\bx^{(1)}, \ldots, \bx^{(L)}]}}{\norm[0]{[\bx^{(1)}, \ldots, \bx^{(L)}]}}$ and average $L_\infty$ error given by $\mathbb{E}_{\bx} \norm[0]{\obW^{[L]} \mathbf{h}^{(L)} - [\bx^{(1)}, \ldots, \bx^{(L)}]}_{\infty}$. We plot both the quantities for different settings of data dimension $d$, sequence length $L$ and the number of neurons $m$. $L$ takes values from the set $\{2, 4, 6\}$, $d$ takes from $\{2, 4, 8\}$ and $m$ takes from $\{500, 1000, 2000, 5000, 10000\}$ (Figure~\ref{fig:RNN_inver}). The trends support our bounds in Theorem~\ref{thm:Invertibility_ESN_outline}, i.e. the error increases with increasing $L$ and decreases with increasing $m$. Note that the data distribution is different from the one assumed in normalized sequence Def.~\ref{def:normalized_seq}. It was easier to conduct experiments in the current data setting and a similar statement as Thm.~\ref{thm:Invertibility_ESN_outline} can be given.



\textbf{Performance of RNNs on different regular languages. } We check the performance of RNNs on the formal language recognition task for a wide variety of regular languages. We follow the set-up in \cite{BhattamishraAG20} who conducted experiments on LSTMs etc. but not on RNNs.

%\section{Regular languages} \label{sec:regular}
We consider the regular languages as considered in \cite{BhattamishraAG20}.
Tomita grammars \cite{tomita:cogsci82} contain 7
regular languages representable by DFAs of small
sizes, a popular benchmark for evaluating recurrent models (see references in \cite{BhattamishraAG20}). 
We reproduce the definitions of the Tomita grammars from there verbatim:
Tomita Grammars are 7 regular langauges defined on the alphabet $\Sigma = \{0, 1\}$.
Tomita-1 has the regular expression $1^\ast$.
Tomita-2 is defined by the regular expression $(10)^\ast$.
Tomita-3 accepts the strings where odd number
of consecutive 1s are always followed by an even
number of $0$'s. Tomita-4 accepts the strings that
do not contain three consecutive $0$'s. In Tomita-5 only
the strings containing an even number of $0$'s and
even number of $1$'s are allowed. In Tomita-6 the
difference in the number of $1$'s and $0$'s should be
divisible by 3 and finally, Tomita-7 has the regular
expression $0^\ast 1^\ast 0^\ast 1^\ast$. 

We also check the performance of RNNs on $\mathrm{Parity}$, which contains all languages with strings of the form $(w_1, \ldots, w_L)$ s.t. $w_1 + \ldots + w_L = 1 \mod 2$. Languages $\mathcal{D}_n$ are recursively defined as the set of all strings of the form $(0w1)^{\ast}$, where $w \in \mathcal{D}_{n-1}$, with $\mathcal{D}_0$ containing only $\epsilon$, the empty word. Other languages considered are $(00)^{\ast}$, $(0101)^{\ast}$ and $(00)^{\ast}(11)^{\ast}$. Table~\ref{table:regular} shows the number of examples in train and test data, the range of the length of the strings in the language, and the test accuracy of the RNNs with activation functions $\relu$ and $\tanh$ on the regular languages mentioned above. 

\begin{center}
\begin{table}[!ht]
\centering
\begin{tabular}{|| c | c | c | c | c ||} 
 \hline
 Task & No. of Training/Test examples  & Range of length of strings & RNN(Relu) & RNN(Tanh)  \\ [0.5ex] 
 \hline\hline
 Tomita 1 & 50/100 & [2, 50] & 1.0 & 1.0 \\
 Tomita 2 & 25/50 & [2, 50] & 1.0 & 1.0 \\
 Tomita 3 & 10000/2000 & [2, 50] & 1.0 & 1.0 \\
 Tomita 4 & 10000/2000 & [2, 50] & 1.0 & 1.0 \\
 Tomita 5 & 10000/2000 & [2, 50] & 1.0 & 1.0\\
 Tomita 6 & 10000/2000 & [2, 50] & 1.0 & 1.0\\
 Tomita 7 & 10000/2000 & [2, 50] & 0.259 & 0.99\\
 Parity & 10000/2000 & [2, 50] & 1.0 & 1.0\\
 $\mathcal{D}_2$ & 10000/2000 & [2, 100] & 1.0 & 1.0 \\
 $\mathcal{D}_3$ & 10000/2000 & [2, 100] & 0.99 & 1.0\\
 $\mathcal{D}_4$ & 10000/2000 & [2, 100] & 1.0 & 0.99\\
 $(00)^{\ast}$ & 250/50 & [2, 500] & 1.0 & 1.0\\
 $(0101)^{\ast}$ & 125/25 & [4, 500] & 0.99 & 1.0 \\
 $(00)^{\ast}(11)^{\star}$ & 10000/2000 & [2, 200] & 0.99 & 1.0 
 %\\Dyck-1 & 0.99 & \textbf{0.91} 
 \\[1ex]
 \hline
\end{tabular}
\caption{Performance of RNNs on different regular languages.}
\label{table:regular}
\end{table}
\end{center}

%The test languages consist of Tomita grammars which constitute a popular benchmark, and a number of other languages including $\mathsf{PARITY}$ and cover a variety of capabilities needed to recognize regular languages.
%The details of the regular languages above can be found in \cite{BhattamishraAG20}; we only note that Tomita languages constitute a popular benchmark. 
We vary $m$, the dimension of the hidden state, in the range $[3, 32]$, used RMSProp optimizer~\cite{hinton2014coursera} with the smoothing constant $\alpha = 0.99$ and varied the learning rate in the range $[10^{-2}, 10^{-3}]$. For each language
we train models corresponding to each language
for $100$ epochs and a batch size of $32$. We experimented with two different activations $\relu$ and $\tanh$. 
%The best test accuracies achieved on different languages are given in table~\ref{table:regular} and these were all achieved for $m=32$.
In all but one case (Tomita 7 with ReLU) the test accuracies with near-perfect. This was the case across runs. Tomita 7 results could perhaps be improved by more extensive hyperparameter tuning. 
We train and test on strings of length up to 50, and in a few cases strings of larger lengths (when the number of strings in the language is small). 
%Details are in Appendix~\ref{sec:regular}.




\iffalse
\begin{figure*}

\centering
\includegraphics[width=.45\textwidth]{../ESN_RNN_normalizedseq/RNN_2.png}
\includegraphics[width=.45\textwidth]{../ESN_RNN_normalizedseq/RNN_4.png}
\includegraphics[width=.5\textwidth]{../ESN_RNN_normalizedseq/RNN_8.png}
\caption{Invertibility of RNNs at initialization}
%\label{fig:figure3}
\end{figure*}
%\FloatBarrier


\begin{center}
\begin{table}[!ht]
\centering
\begin{tabular}{|| c | c | c ||} 
 \hline
 Task & RNN(Relu) & RNN(Tanh)  \\ [0.5ex] 
 \hline\hline
 Tomita 1 & 1.0 & 1.0 \\
 Tomita 2 & 1.0 & 1.0 \\
 Tomita 3 & 1.0 & 1.0 \\
 Tomita 4 & 1.0 & 1.0 \\
 Tomita 5 & 1.0 & 1.0\\
 Tomita 6 & 1.0 & 1.0\\
 Tomita 7 & 0.259 & 0.99\\
 Parity & 1.0 & 1.0\\
 $\mathcal{D}_1$ & 1.0 & 1.0 \\
 $\mathcal{D}_2$ & 0.99 & 1.0\\
 $\mathcal{D}_4$ & 1.0 & 0.99\\
 $(aa)^{\ast}$ & 1.0 & 1.0\\
 $(abab)^{\ast}$ & 0.99 & 1.0 \\
 $(aa)^{\ast}(bb)^{\star}$ & 0.99 & 1.0 
 %\\Dyck-1 & 0.99 & \textbf{0.91} 
 \\[1ex]
 \hline
\end{tabular}
\caption{Performance of RNNs on different regular languages.}
\label{table:regular}
\end{table}
\end{center}
\fi


\section{Conclusion}
We have presented a neural performance rendering system to generate high-quality geometry and photo-realistic textures of human-object interaction activities in novel views using sparse RGB cameras only. 
%
Our layer-wise scene decoupling strategy enables explicit disentanglement of human and object for robust reconstruction and photo-realistic rendering under challenging occlusion caused by interactions. 
%
Specifically, the proposed implicit human-object capture scheme with occlusion-aware human implicit regression and human-aware object tracking enables consistent 4D human-object dynamic geometry reconstruction.
%
Additionally, our layer-wise human-object rendering scheme encodes the occlusion information and human motion priors to provide high-resolution and photo-realistic texture results of interaction activities in the novel views.
%
Extensive experimental results demonstrate the effectiveness of our approach for compelling performance capture and rendering in various challenging scenarios with human-object interactions under the sparse setting.
%
We believe that it is a critical step for dynamic reconstruction under human-object interactions and neural human performance analysis, with many potential applications in VR/AR, entertainment,  human behavior analysis and immersive telepresence.







\bibliographystyle{plain}
\bibliography{references}
\end{document}

% End of ltexpprt.tex