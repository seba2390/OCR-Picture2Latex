%%%%%%%%%%%%%%%%%%%%%%%%%%  ltexpprt.tex  %%%%%%%%%%%%%%%%%%%%%%%%%%%%%%%%
%
% This is ltexpprt.tex, an example file for use with the SIAM LaTeX2E
% Preprint Series macros. It is designed to provide double-column output.
% Please take the time to read the following comments, as they document
% how to use these macros. This file can be composed and printed out for
% use as sample output.

% Any comments or questions regarding these macros should be directed to:
%
%                 Donna Witzleben
%                 SIAM
%                 3600 University City Science Center
%                 Philadelphia, PA 19104-2688
%                 USA
%                 Telephone: (215) 382-9800
%                 Fax: (215) 386-7999
%                 e-mail: witzleben@siam.org


% This file is to be used as an example for style only. It should not be read
% for content.

%%%%%%%%%%%%%%% PLEASE NOTE THE FOLLOWING STYLE RESTRICTIONS %%%%%%%%%%%%%%%

%%  1. There are no new tags.  Existing LaTeX tags have been formatted to match
%%     the Preprint series style.
%%
%%  2. Do not change the margins or page size!  Do not change from the default 
%%     text font!
%%
%%  3. You must use \cite in the text to mark your reference citations and
%%     \bibitem in the listing of references at the end of your chapter. See
%%     the examples in the following file. If you are using BibTeX, please
%%     supply the bst file with the manuscript file.
%%
%%  4. This macro is set up for two levels of headings (\section and
%%     \subsection). The macro will automatically number the headings for you.
%%
%%  5. No running heads are to be used for this volume.
%%
%%  6. Theorems, Lemmas, Definitions, Equations, etc. are to be double numbered,
%%     indicating the section and the occurrence of that element
%%     within that section. (For example, the first theorem in the second
%%     section would be numbered 2.1. The macro will
%%     automatically do the numbering for you.
%%
%%  7. Figures and Tables must be single-numbered.
%%     Use existing LaTeX tags for these elements.
%%     Numbering will be done automatically.
%%
%%  8. Page numbering is no longer included in this macro.
%%     Pagination will be set by the program committee.
%%
%%
%%%%%%%%%%%%%%%%%%%%%%%%%%%%%%%%%%%%%%%%%%%%%%%%%%%%%%%%%%%%%%%%%%%%%%%%%%%%%%%



\documentclass[twoside,leqno,twocolumn]{article}

% Comment out the line below if using A4 paper size
\usepackage[letterpaper]{geometry}
\usepackage{multirow}
\usepackage{longtable}
% \usepackage{float}
% \restylefloat{table}
\usepackage{ltexpprt}
\usepackage{amsmath,amssymb,mathtools}
\usepackage{algorithm}
\usepackage[noend]{algpseudocode}
\usepackage{multirow,array}

\providecommand{\keywords}[1]
{
\textbf{Keywords ---} #1
}

\begin{document}


%\setcounter{chapter}{2} % If you are doing your chapter as chapter one,
%\setcounter{section}{3} % comment these two lines out.

\title{\Large A Game Theoretic Algorithm for Elite Customer Identification in Online Fashion E-Commerce
% \thanks{Supported by GSF grants ABC123, DEF456, and GHI789.}
}
\author{
Chandramouli K\thanks{Myntra Designs Pvt. Ltd.}
\and Gopinath A\thanks{Myntra Designs Pvt. Ltd.}
\and Girish Satyanarayana \thanks{Myntra Designs Pvt. Ltd.}
\and Ravindra babu Tallamraju \thanks{Myntra Designs Pvt. Ltd.}
}

\date{}

\maketitle

% Copyright Statement
% When submitting your final paper to a SIAM proceedings, it is requested that you include 
% the appropriate copyright in the footer of the paper.  The copyright added should be 
% consistent with the copyright selected on the copyright form submitted with the paper.
% Please note that "20XX" should be changed to the year of the meeting.

% Default Copyright Statement
\fancyfoot[R]{\scriptsize{Copyright \textcopyright\ 2021 by SIAM\\
Unauthorized reproduction of this article is prohibited}}

% Depending on which copyright you agree to when you sign the copyright form, the copyright 
% can be changed to one of the following after commenting out the default copyright statement
% above.

%\fancyfoot[R]{\scriptsize{Copyright \textcopyright\ 2021\\
%Copyright for this paper is retained by authors}}

%\fancyfoot[R]{\scriptsize{Copyright \textcopyright\ 2021\\
%Copyright retained by principal author's organization}}

%\pagenumbering{arabic}
%\setcounter{page}{1}%Leave this line commented out.

\begin{abstract} \small\baselineskip=9pt 
Myntra is an online fashion e-commerce company based in India. At Myntra, a market leader in fashion e-commerce in India, customer experience is paramount and a significant portion of our resources are dedicated to it. Here we describe an algorithm that identifies eligible customers to enable preferential product return processing for them by Myntra. We declare the group of aforementioned eligible customers on the platform as elite customers. Our algorithm to identify eligible/elite customers is based on sound principles of game theory. It is simple, easy to implement and scalable.
\end{abstract}

\vspace{0.05cm}
\keywords{\begin{small} Game Theory, Nash Equilibrium, E-Commerce, Customer Experience, Product return\end{small}}



\section{Introduction}
\label{sec:Introduction}


The goal in top-$\size$ recommendation is to recommend to each
consumer a small set of $\size$ items from a large collection of
items~\cite{cremonesi2010performance}.  For example, Netflix may want
to recommend $\size$ appealing movies to each consumer.  Collaborative
Filtering (CF)~\cite{herlocker2002empirical,lee2012comparative} is a
common top-$\size$ recommendation method.  CF infers user interests by
analyzing partially observed user-item interaction data, such as user
ratings on movies or historical purchase
logs~\cite{kanagal2012supercharging}. The main assumption in CF is that
users with similar interaction patterns have similar interests.


Standard CF methods for top-$\size$ recommendation focus on making  suggestions  that accurately reflect the user's preference history. However, as  observed in previous work,  CF recommendations are generally biased toward  popular items, leading to a rich get richer effect~\cite{vargas2014improving,steck2011item}.  The major reasons for this are \textit{popularity bias} and \textit{sparsity} of CF interaction data (detailed in Section~\ref{sec:related-work}). In a nutshell, to maintain  accuracy, recommendations are generated from the dense regions of the data,  where the popular items lie.  

However,  accurately suggesting popular items, may not be satisfactory for the consumers. For example, in Netflix, an accuracy-focused movie recommender may recommend ``Star Wars: The Force Awakens'' to users who have seen ``Star Wars: Rogue One''.  But, those users are probably already aware of ``The Force Awakens''. Considering additional factors, such as novelty of recommendations,  can lead to more effective suggestions~\cite{cremonesi2010performance,Castells2015,zhang2008avoiding,ziegler2005improving,zhang2012auralist}. 
%Second, accuracy-focused models typically achieve a   overall item-space coverage across their recommendations,  whereas high item-space coverage helps providers of the items increase revenue
%, users satisfaction since they are  likely already aware of or can find these items on their own.  

Focusing on popular items also adversely affects the satisfaction of  the providers of the items. This is because  accuracy-focused models typically achieve a  low overall item space coverage across their recommendations, whereas   high item space coverage helps providers of the items increase their revenue~\cite{vargas2014improving,Castells2015,adomavicius2011maximizing,anderson2006thelongtail, yin2012challenging,adomavicius2012improving}.
%accuracy-focused models typically achieve a

In contrast to the relatively small number of popular items, there are copious  {\it long-tail\/} items that have fewer observations (e.g., ratings) available. More precisely,  using the Pareto  principle (i.e.,~the $80/20$ rule),  long-tail items can be defined as items that generate the lower $20\%$ of observations~\cite{yin2012challenging}. Experimentally we found that these items correspond to almost $85\%$ of the items in several datasets (Sections~\ref{sec:Notation} and \ref{sec:Experiments}). %Table~\ref{tab:DatasetStatsticsSmall})


As previously shown, one way to improve the novelty of top-$\size$ sets is to recommend interesting long-tail items~\cite{cremonesi2010performance,ge2010beyond}.  The intuition  is that since they have fewer observations available,  they are more likely to be unseen~\cite{Kaminskas:2016:DSN:3028254.2926720}.  
 %For example, in online commerce,  newly added items are long-tail items that are yet to be discovered.  
Moreover, long-tail item promotion also results in higher overall coverage of the item space%, which increases profits for providers of the items
~\cite{vargas2014improving,Castells2015,zhang2008avoiding,zhang2012auralist,adomavicius2011maximizing,anderson2006thelongtail,yin2012challenging,jambor2010optimizing}. Because long-tail promotion reduces accuracy~\cite{steck2011item}, there are trade-offs to be explored.


%original submitted to ICDE
%This work studies three aspects of top-$\size$ recommendation: accuracy, novelty, and item-space coverage, and examines their trade-offs. In most previous work, predictions of a base recommendation system are re-ranked to handle their trade-offs~\cite{adomavicius2012improving,jambor2010optimizing,zhang2013personalize,wang2009portfolio}. Due to performance considerations, however, these techniques are not customized per user. For example,  parameters that balance the trade-off between novelty and accuracy are cross-validated at a global level.  This can be detrimental since users have varying preferences for  objectives such as long-tail novelty. We explore how to  automatically infer  user  preference for long-tail novelty, and how to leverage  it to correct  the popularity bias in standard recommender models. Our work does not rely on any additional contextual data, although such data, if available, can help promote newly-added long-tail items~\cite{agarwal2009regression,Saveski:2014:ICR:2645710.2645751}.

This work studies three aspects of top-$\size$ recommendation: accuracy, novelty, and item space coverage, and examines their trade-offs. In most previous work, predictions of a base recommendation algorithm are \textit{re-ranked} to handle these trade-offs~\cite{adomavicius2012improving,jambor2010optimizing,zhang2013personalize,wang2009portfolio}. The re-ranking models are computationally efficient but suffer from two drawbacks. First, due to performance considerations,  parameters that balance the trade-off between novelty and accuracy  are not customized per user. Instead they are cross-validated at a global level.  This can be detrimental since users have varying preferences for  objectives such as long-tail novelty. Second,  the re-ranking methods are often limited to a specific base recommender  that may be sensitive to dataset density. 
As a result, the datasets are pruned and the problem is studied in dense settings~\cite{adomavicius2012improving,ho2014likes}; but real world  scenarios are often sparse~\cite{kanagal2012supercharging,liu2017experimental}.   
% Because  dataset density can impact the performance of most base recommenders (like R-SVD), which in turn affects the performance of the re-ranking model, 

\iffalse
We address these limitations by directly inferring  user  preference for long-tail novelty  from interaction data.  This  allows us to customize the re-ranking  per user, and design a \textit{generic} framework, which resolves the second problem. In particular, since the long-tail novelty preferences are estimated independently of any base  recommender model, we can  plug-in an appropriate base recommender w.r.t. the dataset sparsity.% including ones that are more suitable for sparse settings.  

Modelling  user  preference for  long-tail novelty using only item popularity statistics, e.g., the average popularity of rated items as in~\cite{jugovac2017efficient}, disregards additional information like whether the user found the item interesting and the long-tail preferences of other users  of the items. \iffalse To incorporate them, we introduce the notion of  \emph{item long-tail importance}. Both  user long-tail preferences and item long-tail importance are dependent:  a user has high preference for discovering long-tail items if she is interested in important long-tail items, and an item that is associated with many of these kinds of users is likely to be more important.  We propose a joint optimization framework to directly learn,  from interaction data, both the users' long-tail preferences and the  items' long-tail importance. \fi
We propose an optimization approach that  incorporates  this information and  directly learns,  from interaction data, the users' long-tail novelty preferences.

Next, we use these learned preferences  to design a  top-$\size$ recommendation framework thats is generic, and provides customized balance between accuracy, novelty, and coverage. We refer to it as framework as GANC.  Using GANC, we design a novel algorithm, {\it Ordered Sampling-based Locally Greedy (OSLG)\/}, that relies on the learned long-tail novelty preferences  to scalably correct for popularity bias. Our work does not rely on any additional contextual data, although such data, if available, can help promote newly-added long-tail items~\cite{agarwal2009regression,Saveski:2014:ICR:2645710.2645751}. In summary:
\fi

We address the first limitation by directly inferring  user  preference for long-tail novelty  from interaction data.   Estimating these  preferences  using only item popularity statistics, e.g., the average popularity of rated items as in~\cite{jugovac2017efficient}, disregards additional information, like whether the user found the item interesting or the long-tail preferences of other users  of the items. We propose an approach that  incorporates  this information and  learns the users' long-tail novelty preferences from interaction data.

This approach allows us to customize the re-ranking  per user, and  design a \textit{generic} re-ranking framework, which resolves the second limitation of prior work. In particular, since the long-tail novelty preferences are estimated independently of any base recommender, we can  plug-in an appropriate one w.r.t. different factors, such as the dataset sparsity.

Our top-$\size$ recommendation framework, \textbf{GANC}, is \textbf{G}eneric, and provides customized balance between \textbf{A}ccuracy, \textbf{N}ovelty, and \textbf{C}overage. % Moreover, based on the learned long-tail novelty preferences, we also design a novel algorithm, {\it Ordered Sampling-based Locally Greedy (OSLG)\/}, that relies on the learned long-tail novelty preferences  to scalably correct for popularity bias. 
Our work does not rely on any additional contextual data, although such data, if available, can help promote newly-added long-tail items~\cite{agarwal2009regression,Saveski:2014:ICR:2645710.2645751}. In summary:

%Consider  the following toy example:
\vspace{-0.2cm}
\begin{table}[htb]
\centering
\scriptsize
%\small
\begin{tabular}{ccccccc} 
%\toprule
%&\multirow{2}{*}{}&\multicolumn{7}{c}{Ratings}\\
& & \cellcolor{blue!35}$w_1$ &\cellcolor{blue!18} $w_2$ & $\dots$ &\cellcolor{blue!8} $w_{89}$  &\cellcolor{blue!8} $w_{99}$   
\\
&   &$i_1$&$i_2$&$\dots$&$i_{89}$&$i_{90}$\\ 
\cmidrule(r){3-7} 	 
%\midrule
\cellcolor{red!35}$\theta_1$  &$u_1 $   &5 &   & $\dots$ &  &   \\
\cellcolor{red!28}$\theta_2$  &$u_2$     &5 &    & $\dots$ &  &  \\
 $\theta_3=?$  &$\bf u_3$  &5 &  &   $\dots$ &  &  \\
\cellcolor{red!10}$\theta_4$ & $u_4$  &  &5   & $\dots$ & &\\ 
\cellcolor{red!10}$\theta_5$ & $u_5$  &  & 5  & $\dots$ & &\\ 
$\theta_6=?$  & $\bf u_6$ & &5  &      $\dots$& &  \\ 
 & & $\hdots$  &$\hdots$   &$\hdots$   &$\hdots$   &$\hdots$  \\
%\midrule 
\cmidrule(r){3-7} 	 
\multicolumn{2}{c}{item pop.}  & 3  & 3  & $\dots$ &50&60\\  
%\bottomrule
%$ f_i$    &3  &3  &1  &3  &1  &2  \\  \hline
\end{tabular}
%#.
\caption{Simplified user-item interaction data. The user long-tail novelty preference ($\theta_u$), item long-tail importance weight ($w_i$) are highlighted. Darker colors indicate larger values. } \label{tab:example}
\end{table} 
\vspace{-0.2cm}
\begin{example}  
In Table~\ref{tab:example}, we are interested in estimating $\theta_3$ and $\theta_6$,  the long-tail preference of users $u_3$ and $u_6$ who have each rated a single movie. Additional ratings for other users  are not included here.  Considering only rating information, we observe $i_1$ and $i_2$ are  equally popular $|\mathcal{U}_{i_1}^{\trainset}| = |\mathcal{U}_{i_2}^{\trainset}|=3$, and $r_{31}=5$ and $r_{62}=5$. Using Eq.~\ref{eq:tfidf-risk}  we have $\theta_3 = \theta_6$. However, if we were given the long-tail preferences of the each item's user set, specifically that $u_1$ and $u_2$ have high long-tail preference (darker red), while $u_4$ and $u_5$ have lower long-tail preference (lighter red), we could conclude $i_1$ is a more important long-tail item compared to $i_2$ (indicated by a darker blue shade for $w_1$), and we expect  $\theta_3 \geq \theta_6$.

% On the other hand, if we knew that $u_4$ and $u_5$ have lower long-tail preference, we could conclude $i_2$ is a  less significant long-tail item. Therefore, However, if we  consider the long-tail preferences of other users, we may reason differently.    We need another variable $w_i$ which captures this information. 
%we would conclude that $u_3$ has higher long-tail preference compared to $u_6$, since the users $i_1$ is a more prominent long-tail item. 

% Relying only  on item popularity information, we would  conclude   $u_3$ and $u_6$ have equal long-tail preference, since $i_1$ and $i_2$ are  equally popular. However, considering  the second column,  long-tail preference of users,  long-tail importance for each item,  which captures the long-tail preference of its users. Since  that  both users of $i_1$ have high long-tail preference while  the users of $i_2$ have lower preference,  we may conclude $i_1$ is a more important long-tail item compared to $i_2$. Therefore, $u_3$'s long-tail preference should be at least as large as $u_6$'s preference. Specifically, consider two  items $i_1$ and $i_2$, with the following rating data: $i_1=\{u_1:5, u_2:5, u_3:5 \}$, $i_2=\{u_4:5, u_5:5, u_6:5\}$.  

%Table~\ref{tab:example} shows  simplified rating data. We want an estimate of the long-tail preference of $u_3$ and $u_6$, who have each  rated a single movie.  Relying only  on movie popularity information, we would  conclude   $u_3$ and $u_6$ have similar long-tail preference, since $m_1$ and $m_2$ are  equally popular. However, considering the long-tail preferences of other users of those movies, we may reason differently: since $u_1$ and $u_2$ have high long-tail preference, and $u_4$ and $u_5$ have low long-tail preference, $m_1$ is a more prominent long-tail item compared to $m_2$. Therefore, it is likely that $u_3$ has higher long-tail preference compared to $u_6$.considering the long-tail preferences of other users of those movies, we may reason differently.  For example, 
\label{ex:running}
\end{example}



%------------------------------

\iffalse
\begin{example}
Table~\ref{tab:example} shows rating data for a simplified system. %Note the user-item interaction matrix is sparse.
For this example, we define popular movies as those that have received  three or more ratings; $\{m_1, m_2, m_4\}$ are popular and  $\{m_3, m_5, m_6\}$ are niche movies. We observe $u_1$ and $u_3$  have rated relatively popular movies (risk-averse) while $u_2$ and $u_4$ have rated niche movies (risk-loving). 
\label{ex:running}
\end{example}

\begin{table}[htb]
\centering
\scriptsize
\begin{tabular}{ccccccc} 
\toprule
			&$m_1$ &$m_2$   &$m_3$    &$m_4$   &$m_5$ &$m_6$  \\ \hline 
$u_1 $ &5  &4  & - &-  &-  &-   \\
$u_2$  &-  &-  &-  &-  &5  &5   \\
$u_3$  &-  &4  &-  &5  &-  &-   \\
$u_4$  &-  &-  &3  &-  &-  &4   \\ 
$u_5$  &5  &-  &-  &3  &-  &-   \\ 
$u_6$  &4  &2  &-  &4  &-  &-   \\ 
\bottomrule
%$ f_i$    &3  &3  &1  &3  &1  &2  \\  \hline
\end{tabular}
\caption{User-Movie rating data} \label{tab:example}
\end{table}

It is essential to consider consumer characteristics in designing recommender systems so that they promote long-tail items to the right group of users and spread demand evenly between hit and niche items.  

\fi





%------------------------------
\iffalse
\begin{table}[htb]
\centering
\scriptsize
\begin{tabular}{ccccccc} 
\toprule
			&$m_1$ &$m_2$   &$m_3$    &$m_4$   &$m_5$ &$m_6$  \\ \hline 
$u_1 $ &\textbf{5}  & \textbf{4}  &\textcolor{gray}{ 1.2} &-  &-  &-   \\
$u_2$  &-  &-  &-  &-  & \textbf{5}  &\textbf{5}   \\
$u_3$  &-  &\textbf{4}  &-  &\textbf{5}  &-  &-   \\
$u_4$  &-  &-  &\textbf{3}  &-  &-  &\textbf{4}   \\ 
$u_5$  &\textbf{5}  &-  &-  &\textbf{3}  &-  &-   \\ 
$u_6$  &\textbf{4}  &\textbf{2}  &-  &\textbf{4}  &-  &-   \\ 
\bottomrule
%$ f_i$    &3  &3  &1  &3  &1  &2  \\  \hline
\end{tabular}
\caption{User-Movie rating data} \label{tab:example}
\end{table}
% $\mathcal{P}^1= \{ \mathcal{P}_1^1 \{i_1,i_2,i_3\}, \mathcal{P}_2^1:\{i_2,i_3,i_5\}  \}$
 %$\mathcal{P}^2= \{ \mathcal{P}_1^2: \{i_1,i_2,i_3\}, \mathcal{P}_2^2:\{i_2,i_5,i_6\}  \}$
 %$\mathcal{P}^3= \{ \mathcal{P}_1^3: \{i_7,i_8,i_9\}, \mathcal{P}_2^3:\{i_{10},i_{11},i_{12}\}  \}$
\begin{table}[htb]
\centering
\tiny
\begin{tabular}{ccc} 
\toprule
		&$u_1$&$u_2$  \\ \hline 
$\mathcal{P}^1 $ & $\{i_1,i_2,i_3\}$ & $\{i_2,i_3,i_5\} $ \\
$\mathcal{P}^2$ & $\{i_1,i_2,i_3\}$ & $\{i_2,i_5,i_6\} $ \\
$\mathcal{P}^3$ & $\{i_7,i_8,i_9\}$ & $\{i_{10},i_{11},i_{12} \}$ \\
\bottomrule
%$ f_i$    &3  &3  &1  &3  &1  &2  \\  \hline
\end{tabular}
\caption{Top-$\size$ allocations to users.} \label{tab:paretoExamples}
\end{table}
\fi


\iffalse
When considering long-tail items, it is important to consider consumers' willingness  to explore niche or unpopular items and their propensity towards similar items. In particular, they can be characterized by their  {\it risk degree\/} and {\it focusing degree\/}, respectively.  We compute these estimates  based on historical rating information. The following example further describes these notions in the context of movie rating data. 

\begin{example}  
Table~\ref{tab:example} shows rating data for a simplified system with $6$ users, $6$ movies, and $3$ genres. $m_i^{j}$ implies that movie $m_i$ belongs to genre $j$. Note the user-item interaction matrix is sparse. 
  For this setting, we define popular movies as those that have received  three or more ratings; $\{m_1, m_2, m_4\}$ are popular and  $\{m_3, m_5, m_6\}$ are niche movies. We now profile the users according to their risk and focusing degree. E.g., $u_1$ has rated relatively popular movies belonging to the same genre (risk-averse, high focusing degree); $u_2$ has rated niches movies in the same genre (risk-loving, high focusing degree); $u_3$ has rated popular movies in two different genres (risk-averse, low focusing degree), and $u_4$ has rated niches movies in two different genres (risk-loving, low focusing degree). 
\label{ex:running}
\end{example}
\begin{table}[htb]
\centering
\tiny
\begin{tabular}{ccccccc} 
\toprule
			&$m_1^{1}$ &$m_2^{1}$   &$m_3^{2}$    &$m_4^{3}$   &$m_5^{3}$ &$m_6^{3}$  \\ \hline 
$u_1 $ &5  &4  &-  &-  &-  &-   \\
$u_2$  &-  &-  &-  &-  &5  &5   \\
$u_3$  &-  &4  &-  &5  &-  &-   \\
$u_4$  &-  &-  &3  &-  &-  &4   \\ 
$u_5$  &5  &-  &-  &3  &-  &-   \\ 
$u_6$  &4  &2  &-  &4  &-  &-   \\ 
\bottomrule
%$ f_i$    &3  &3  &1  &3  &1  &2  \\  \hline
\end{tabular}
\caption{User-Movie rating data} \label{tab:example}
\end{table}
It is essential to consider these consumer characteristics in designing recommender systems so that they promote long-tail items to the right group of users and spread demand evenly between the hit and niche items.  
\fi
\iffalse
\begin{center}
\begin{figure*}[tp]
%\scalebox{0.5}{%
\resizebox{1\textwidth}{!}{%
%\small%\addtolength{\tabcolsep}{5pt}% below sums to 8
\begin{tabularx}{1.5\textwidth}{>{\hsize=2.5\hsize}X>{\hsize=2.5\hsize}X>{\hsize=0.5\hsize}X>{\hsize=0.5\hsize}X>{\hsize=0.5\hsize}X>{\hsize=0.5\hsize}X>{\hsize=0.5\hsize}X>{\hsize=0.5\hsize}X}
    \multirow{12}{*}{\includegraphics[scale=0.3]{codeForExample/popularity-movie.png}} & \multirow{12}{*}{\includegraphics[scale=0.3]{codeForExample/scatterplot.png}} & & & & & & \\
%   & &               &       &       &       &       &       \\
    & &\multicolumn{1}{l|}{}               &$m_1^{g1}$   	&$m_2^{g1}$    	&$m_3^{g2}$    &$m_4^{g2}$      &$m_5^{g3}$    \\ \cline{3-8}%\hline
    & &\multicolumn{1}{l|}{u1}          &5  &5  &-  &-   &-  \\
    & &\multicolumn{1}{l|}{u2}    		&-  &-  &4  &4  &5  \\
    & &\multicolumn{1}{l|}{u3}   			&1  &2  &1  &-  &-   \\
    & &\multicolumn{1}{l|}{u4}     		&1  &-  &-  &-  &-  \\
    & &               &       &       &       &       &       \\
    & &               &       &       &       &       &       \\
    & &               &       &       &       &       &       \\
    & &               &       &       &       &       &	\\
    \\
\end{tabularx}}
\caption{User-Movie interaction data a) Popularity-Movie histogram b)Movie genres/clusters c) User-Movie rating data} \label{fig:example}
\end{figure*}
\end{center}
\fi



%We propose a novel approach that allows us to  promote long-tail items in a targeted manner, thereby improving the novelty of top-$\size$ sets, the overall item-space coverage across recommendations, while maintaining reasonable levels of accuracy.

%Next, we integrate these learned preferences  in a generic  top-$\size$ recommendation framework to provide customized balance between accuracy and coverage.

%sequentially make recommendations, while adjusting its parameters with regard to the set of top-$\size$ recommendations made so far. However, since  sequential parameter updates  cause  scalability issues, we propose a sampling based algorithm. This variant of our framework, called {\it Ordered Sampling-based Locally Greedy (OSLG)\/},  allows us to  correct for the popularity bias in recommendations with regard to individual user long-tail preferences. 

%ICDE submission
%Our framework differs with  prior work in the following aspects:  unlike~\cite{adomavicius2011maximizing,adomavicius2012improving,zhang2013personalize,ho2014likes},  the long-tail preference personalization in our framework is learned rather than optimized using cross-validation or parameter tuning. In other words, our personalization method is independent of the underlying base  recommendation models.  Moreover, our framework is  generic. This enables us to  plug-in several base recommenders, and evaluate their  effectiveness without requiring  extensive tuning for the accuracy and coverage trade-off. 


%\vspace{-2.8pt}
\begin{itemize}

\item  We examine various measures for estimating user long-tail novelty preference in Section~\ref{sec:lt-pref} and formulate an optimization problem  to directly learn users' preferences for long-tail  items from interaction data in Section~\ref{sec:learning-lt-pref}. %In addition, we introduce several heuristics for measuring the user preference for less common items from historical rating data.% 

\item  We integrate the user preference estimates into GANC %, a generic re-ranking framework that provides customized balance between accuracy, novelty, and coverage 
(Section~\ref{sec:RiskbasedReranking}), and  introduce {\it Ordered Sampling-based Locally Greedy (OSLG)\/}, a scalable algorithm that relies  on user long-tail preferences to correct the popularity bias (Section~\ref{sec:optimizationAlgorithm}).
%We introduce OSLG, a scalable algorithm that relies  on user long-tail preferences to  maximize item space coverage \textcolor{red}{while maintaining acceptable levels of accuracy} (Section~\ref{sec:optimizationAlgorithm}).

\item   We conduct an extensive empirical study and evaluate performance from  accuracy, novelty, and coverage perspectives (Section~\ref{sec:Experiments}).  We use five  datasets with varying density and difficulty levels. %:  Netflix, MovieTweetings, and MovieLens (100K, 1M, 10M). 
  In contrast to most related work,  our evaluation considers realistic settings that include a large number of infrequent  items and users. %This enables us to study the impact of  data density on the performance trade-offs of several  state of the art top-$\size$ recommendation algorithms. %   %,  and use the all-items ranking protocol~\cite{steck2013evaluation,vargas2014improving}, where performance is measured using all items with train data. to evaluate the performance of several  state of the art top-$\size$ recommendation algorithms 
 
\item Our empirical results confirm that the performance of re-ranking models is impacted by the underlying   base recommender and the dataset density. Our generic approach enables us to easily incorporate a suitable base recommender to devise an effective solution for both dense and sparse settings. In dense settings, we use the same base recommender as existing re-ranking approaches, and we outperform them in accuracy and coverage metrics. For sparse settings, we plug-in a more suitable base recommender, and devise an effective solution that is competitive with existing top-$\size$ recommendation methods in accuracy and novelty. 

%Directly estimating the long-tail novelty preferences allows us to customize re-ranking per user, and  devise a generic framework.   
 
\end{itemize}

Section~\ref{sec:related-work} describes related work. Section~\ref{sec:conclusion} concludes.

\section{Background}
\label{sec:background}

%\SHAN{I think the background section is too long, maybe we can remove the SO3, SE3 parts, only keep SIM3, ellipsoid, SDF}

Rigid body orientation, pose, and similarity are represented using the $\text{SO}(3)$, $\text{SE}(3)$, and $\text{SIM}(3)$ Lie groups, respectively, defined as:
%
\begin{gather}
\label{eq:LieGroups}
\scaleMathLine{\begin{aligned}
\text{SO}(3) &\triangleq \crl{ \bfR \in \bbR^{3 \times 3} \mid \bfR^\top\bfR = \bfI, \det(\bfR) = 1},\\
\text{SE}(3) &\triangleq \crl{ \begin{bmatrix} \bfR & \bft\\\mathbf{0}^\top & 1 \end{bmatrix} \in \bbR^{4 \times 4} \,\bigg\vert\, \bfR \in SO(3), \bft \in \bbR^3}, \\
\text{SIM}(3) &\triangleq \crl{ \begin{bmatrix} s\bfR & \bft\\\mathbf{0}^\top & 1 \end{bmatrix} \in \bbR^{4 \times 4} \,\bigg\vert\, \bfR \in SO(3), \bft \in \bbR^3, s \in \bbR}.
\end{aligned}}
\raisetag{10ex}
\end{gather}
%
We overload $\bfxi_\times$ to denote a mapping from a vector in $\bbR^3$ or $\bbR^6$ or $\bbR^7$ to the Lie algebra $\mathfrak{so}(3)$, $\mathfrak{se}(3)$, or $\mathfrak{sim}(3)$, associated with the Lie groups in \eqref{eq:LieGroups}, defined as:
%
\begin{gather}
\label{eq:LieAlgebras}
\begin{aligned}
\mathfrak{so}(3) &\triangleq \crl{\bfxi_\times = \begin{bmatrix}0 & -\xi_3 & \xi_2\\\xi_3 & 0 & -\xi_1\\-\xi_2 & \xi_1 & 0 \end{bmatrix} \,\bigg\vert\, \bfxi \in \bbR^3},\\
\mathfrak{se}(3) &\triangleq \crl{\bfxi_\times = \begin{bmatrix} \bftheta_\times & \bfrho \\ \mathbf{0}^\top & 0 \end{bmatrix}\,\bigg\vert\, \bfxi = \begin{bmatrix} \bfrho \\ \bftheta\end{bmatrix} \in \bbR^6},\\
\mathfrak{sim}(3) &\triangleq \crl{\bfxi_\times = \begin{bmatrix} \sigma \bfI + \bftheta_\times & \bfrho \\ \mathbf{0}^\top & 0 \end{bmatrix} \,\bigg\vert\, \bfxi = \begin{bmatrix} \bfrho \\ \bftheta \\ \sigma \end{bmatrix} \in \bbR^7}.
\end{aligned}
\raisetag{12ex}
\end{gather}
%
We define an infinitesimal change of a Lie group element $\bfT$ via a left perturbation $\exp\prl{\bfxi_\times}\bfT$, using the exponential map $\exp\prl{\bfxi_\times}$ to retract a Lie algebra element $\bfxi_\times$ to the Lie group. Please refer to \cite[Ch.7]{BarfootBook} or \cite{Gao2017SLAM} for details. 

The coarse shape of a rigid body is represented using a \emph{quadric shape} \cite[Ch.3]{MVGBook}, $\crl{ \bfx \in \bbR^3 \mid \underline{\bfx}^\top \bfQ \underline{\bfx} \leq 0}$, where $\underline{\bfx} \triangleq [\bfx^\top, 1]^\top$ denotes the homogeneous coordinates of $\bfx$ and $\bfQ \in \bbR^{4 \times 4}$ is a symmetric matrix. An axis-aligned ellipsoid centered at the origin:
%
\begin{equation}
\label{eq:ellipsoid}
\mathcal{E}_{\bfu} \triangleq \crl{\bfx \in \bbR^3 \mid \bfx^\top \bfU^{-\top}\bfU^{-1}\bfx \leq 1},
\end{equation}
%
where $\bfU \triangleq \diag(\bfu)$ and the elements of the vector $\bfu \in \bbR^3$ specify the lengths of the semi-axes of $\mathcal{E}_{\bfu}$. An ellipsoid $\mathcal{E}_{\bfu}$ is a special case of a quadric shape with $\bfQ = \diag(\bfU^{-2},-1)$. 
% Instead of as the collection of points $\bfx$ contained in it, 
A quadric shape can also be defined in dual form, as the set of planes $\underline{\boldsymbol{\pi}} = \bfQ\underline{\bfx}$ that are tangent to the shape surface at each $\bfx$. This dual quadric surface definition is $\crl{ \bfpi \in \bbR^3 \mid \underline{\bfpi}^\top \bfQ^* \underline{\bfpi} = 0}$, where $\bfQ^* = \adj(\bfQ)$ is the adjugate of $\bfQ$.
%\footnote{If $\bfQ$ is invertible, $\bfQ^* = \adj(\bfQ) \triangleq \det(\bfQ)\bfQ^{-1}$ can be simplified to $\bfQ^* = \bfQ^{-1} = \diag(\bfU^2, -1)$ due to the scale-invariance of the dual quadric definition.}.
A dual quadric defined by $\bfQ^*$ can be scaled, rotated, or translated by a similarity transform $\bfT \in \text{SIM}(3)$ as $\bfT \bfQ^* \bfT^\top$. Similarity, a dual quadric can be projected to a lower-dimensional space by a projection matrix $\bfP = \begin{bmatrix} \bfI & \mathbf{0} \end{bmatrix}$ as $\bfP \bfQ^* \bfP^\top$.

The fine shape of a rigid body is represented as $\crl{\bfx \in \bbR^3 \mid f(\bfx) \leq 0}$ using the \emph{signed distance field} of a set $\calS \subset \bbR^3$:
%
\begin{equation}
f(\bfx) = \begin{cases}
  -d(\bfx,\partial\calS), & \bfx \in \calS,\\
  \phantom{-}d(\bfx,\partial\calS), & \bfx \notin \calS,
\end{cases}
\end{equation}
%
where $d(\bfx,\partial\calS)$ denotes the Euclidean distance from a point $\bfx \in \bbR^3$ to the boundary $\partial\calS$ of $\calS$.























%% \subsection{SE3}

%This section introduces the necessary mathmatical background. 
%The transformation in $\text{SE}(3)$ can be expressed as:
%\begin{equation}
%\bfT \triangleq \begin{bmatrix} \bfR & \bft\\\mathbf{0}^\top & 1 \end{bmatrix} \in \text{SE}(3)
%\end{equation}
%We overload $\bftheta_\times$ to denote the mapping from an axis-angle vector $\bftheta \in \mathbb{R}^3$ to a $3 \times 3$ skew-symmetric matrix $\bftheta_\times \in \mathfrak{so}(3)$ and the mapping from a position-rotation vector $\bfxi \in \mathbb{R}^6$ to a $4 \times 4$ twist matrix $\bfxi_\times \in \mathfrak{se}(3)$. We define an infinitesimal change of pose $\bfT \in SE(3)$ using a left perturbation $\exp\prl{\bfxi_\times}\bfT \in \text{SE}(3)$ (see~\cite[Ch.7]{BarfootBook}).

%% \subsection{SIM(3)}

%We use the space $\text{SIM}(3)$ of similarity transformations to capture scale $s$ in addition to pose: 
%\begin{equation}
%\bfT \triangleq \begin{bmatrix} s\bfR & \bft\\\mathbf{0}^\top & 1 \end{bmatrix} \in \text{SIM}(3).
%\end{equation}
%We also use $\bfxi$ to denote the corresponding Lie algebra $\mathfrak{sim}(3)$, as in \cite{Gao2017SLAM}:
%\begin{equation}
%\label{eq:sim3_to_SIM3}
%\scaleMathLine[0.91]{
%\begin{aligned}
%\mathfrak{sim}(3) \triangleq
%\left\{\bfxi_\times=
%\left[\begin{array}{cc}
%{\sigma \bfI+\bfphi_\times} & {\bfrho} \\
%\mathbf{0}^\top & {0}
%\end{array}\right] \in \mathbb{R}^{4 \times 4} \;\bigg\vert\; \bfxi = \left[\begin{array}{c}
%{\bfrho} \\
%{\bfphi} \\
%{\sigma}
%\end{array}\right] \in \mathbb{R}^{7}\right\}
%\end{aligned}
%}
%\end{equation}
%We define the operator:
%%\NA{what is the operator $\underline{\bfx}^\circledcirc$?}
%%\SHAN{I don't think we will use $\underline{\bfx}^\circledcirc$ anywhere}
%\begin{equation}
%\underline{\bfx}^\odot \triangleq \begin{bmatrix} \bfI_3 & -\bfx_\times & \bfx\\ \mathbf{0}^\top & \mathbf{0}^\top & 0 \end{bmatrix} \in \mathbb{R}^{4 \times 7}
%\end{equation}




%\begin{definition*}
%The \textit{signed distance field} of a set $\calS \subset \mathbb{R}^3$ is
%\begin{equation}
%f(\bfx) = \begin{cases}
%  -d(\bfx,\partial\calS) & \bfx \in \calS\\
%  d(\bfx,\partial\calS) & \bfx \notin \calS
%\end{cases}
%\end{equation}
%where $d(\bfx,\partial\calS)$ is the distance from a point $\bfx \in \mathbb{R}^3$ to the set boundary $\partial\calS$, and we use $d$ as a shorthand notation to denote this distance.
%\end{definition*}


%\begin{definition*}
%\textit{Huber error function} \cite{Huber1964Robust} with parameter $\delta > 0$ is:
%\begin{equation}
%\rho(r) \triangleq 
%\begin{cases}
%\frac{1}{2}r^2 & |r|\leq \delta,\\
%\delta(|r|-\frac{1}{2}\delta) & \text{else}.
%\end{cases}
%\end{equation}
%\end{definition*}
%whose gradient can be computed as 
%\[
%  \frac{\partial \rho(r)}{\partial r}
%  =\left\{\begin{array}{ll}
%    r & |r| \leq \delta \\
%    \text{sign}(r)\delta  & \text{ else}. 
%    \end{array}\right.
%\] 

%An axis-aligned ellipsoid centered at the origin can be described as:
%\begin{equation}
%\mathcal{E}_{\bfu} \triangleq \crl{\bfx \mid \bfx^\top \bfU^{-\top}\bfU^{-1}\bfx \leq 1},
%\end{equation}
%where $\bfU \triangleq \diag(\bfu)$ and the elements of the vector $\bfu = [a,b,c]^{\top}$ are the lengths of the semi-axes of $\mathcal{E}_{\bfu}$. In homogeneous coordinates, $\mathcal{E}_{\bfu}$ can be represented as a quadric surface~\cite[Ch.3]{MVGBook}, $\crl{ \bfx \mid \underline{\bfx}^\top \bfQ_{\bfu} \underline{\bfx} \leq 0}$, where $\bfQ_{\bfu} = \mathbf{diag}(\bfU^{-2},-1)$. This describes the ellipsoid as a collection of points lying on its surface. 

%Alternatively, a quadric can be defined by the set of planes $\underline{\boldsymbol{\pi}} = \bfQ_{\bfu}\underline{\bfx}$ tangent to its surface at $\underline{\bfx}$. This dual quadric surface is defined as $\crl{ \bfpi \mid \underline{\bfpi}^\top \bfQ_{\bfu}^* \underline{\bfpi} = 0}$, where $\bfQ_{\bfu}^* = \mathbf{adj}(\bfQ_{\bfu})$~\footnote{If $\bfQ_{\bfu}$ is invertible, $\bfQ_{\bfu}^* = \mathbf{adj}(\bfQ_{\bfu}) = \det(\bfQ_{\bfu})\bfQ_{\bfu}^{-1}$ can be simplified to $\bfQ_{\bfu}^* = \bfQ_{\bfu}^{-1} = \diag(\bfU^2, -1)$ due to the scale-invariance of the dual quadric definition.}. A dual quadric defined by $\bfQ_{\bfu}^*$ can be transformed by $\bfT \in SE(3)$ to another reference frame as $\bfQ^* = \bfT \bfQ_{\bfu}^* \bfT^\top$, which can be projected to a lower-dimensional space by $\bfP \in \mathbb{R}^{3\times 4}$ as $\bfP \bfQ^* \bfP^\top$. 
%$\bfQ^*$ can be parameterized as:
%\begin{equation}
%\label{eq:ellipsoid}
%\begin{aligned}
%\bfQ^*
%&=
%\mathbf{T} \bfQ_{\bfu}^* \mathbf{T}^{\top}=
%\left[\begin{array}{cc}
%\mathbf{R} & \bft \\
%\mathbf{0}^{\top} & 1
%\end{array}\right]\left[\begin{array}{cc}
%\mathbf{U} \mathbf{U}^{\top} & \mathbf{0} \\
%\mathbf{0} & -1
%\end{array}\right]\left[\begin{array}{ll}
%\mathbf{R}^{\top} & \mathbf{0} \\
%\bft^{\top} & 1
%\end{array}\right] \\ 
%&=
%\begin{pmatrix} 
%\mathbf{R} \mathbf{U} \mathbf{U}^\top \mathbf{R}^\top -  \bft \bft^\top & - \bft \\ -\bft^\top & -1
%\end{pmatrix}
%\end{aligned}
%\end{equation}




\section{Problem Statement}
At Myntra, a customer's product return request is processed in the following manner. Once a customer places a product return request, a doorstep pickup agent arrives and receives the product. After a quality check of the returned item at a designated location refund to the customer is initiated.

For a better customer experience we wish to identify elite customers and do away with this quality check process for these elite customers. To identify these elite customers, we model the interaction between Myntra and a generic customer as a game.

We have two players in our game i.e. $N=\{ \text{Myntra}, \text{customer}\}$. Myntra has two actions i.e. $A_{\text{Myntra}}=\{\text{immediate refund},\text{no immediate refund}\}$. Similarly customer has two actions i.e. $A_{\text{customer}}=\{\text{comply with return requirements}, \\ \text{don't comply with return requirements}\}$.
Here in the context of Myntra the strategy/action ``immediate refund" refers to the removal of quality check process.

We designed the utility matrix  given in Table \ref{utility_matrix} for this game.
Our rationality behind this particular choice for the utility matrix is as follows. For Myntra, with $u_{\text{Myntra}}\text{(No Immediate Refund, Don't Comply)}=0$ as baseline, $u_{\text{Myntra}}\text{(Immediate Refund, Comply)}=1$, 
$u_{\text{Myntra}}\text{(No Immediate Refund, Comply)}=2$ and $u_{\text{Myntra}}\text{(Immediate Refund, Don't Comply)}=-1$ reflect the satisfaction levels of Myntra in the customer-Myntra relationship. We assume the same for the customer and obtain the utility matrix shown in Table \ref{utility_matrix}.
  \begin{table}
    \setlength{\extrarowheight}{2pt}
    \begin{tabular}{*{4}{c|}}
      \multicolumn{2}{c}{} & \multicolumn{2}{c}{Customer}\\\cline{3-4}
      \multicolumn{1}{c}{} &  & Comply  & Don't Comply \\\cline{2-4}
      \multirow{2}*{Myntra}  & Immediate &    &  \\
      & Refund & $(1,1)$ & $(-1,2)$ \\\cline{2-4}
      & No Immediate &      &  \\
      & Refund & $(2,-1)$ & $(0,0)$ \\\cline{2-4}
    \end{tabular}
     \caption{Utility matrix of our game}
      \label{utility_matrix}
  \end{table}
 \vspace*{-\baselineskip}
 \section{Solution}
 Observe that (No Immediate Refund, Don't Comply) is a pure strategy Nash equilibrium for this game and No Immediate Refund is the corresponding strategy for Myntra. However we note that the relation between Myntra and the customer is an ongoing relation. Hence a repeated game that repeats with probability $\delta$ is a more appropriate model for Myntra-customer relationship.
 In this context the set of strategy vectors is given by $S= (A_{\text{Myntra}} \times A_{\text{customer}})^{\mathbb{N}}$, set of sequences with values in $A_{\text{Myntra}} \times A_{\text{customer}}.$ Here $\mathbb{N}$ denotes the set of Natural numbers. The utility function $v_{\text{Myntra}}:S\rightarrow \mathbb{R}$ is given by 
 $v_{\text{Myntra}}(s)=\sum_{k\geq0}\delta^{k} u_{\text{Myntra}}(s^{k+1})$ where $s=(s^{1},s^{2},\cdots)$ with $s^{k} \in A_{\text{Myntra}} \times A_{\text{customer}}, k \in \mathbb{N}$ and $u_{\text{Myntra}}$ is given by Table \ref{utility_matrix}. Similarly the utility function 
 $v_{\text{customer}}:S\rightarrow \mathbb{R}$ is given by 
 $v_{\text{customer}}(s)=\sum_{k\geq0}\delta^{k} u_{\text{customer}}(s^{k})$ where $s=(s^{1},s^{2},\cdots)$ with $s^{k} \in A_{\text{Myntra}} \times A_{\text{customer}}, k \in \mathbb{N}$ and $u_{\text{customer}}$ is given by Table \ref{utility_matrix}.
 
 Observe that in this setup the strategy vector $s^{*}=(s^{*i})_{i\in \mathbb{N}}$ with $s^{*i}=(\text{no immediate refund},\text{don't comply})$ is an equilibrium strategy vector with $(v_{\text{Myntra}},v_{\text{customer}})=(0,0)$ as any unilateral deviation in any repetition of the game by a player diminishes the utility of the player. Moreover the equilibrium strategy ``no immediate refund" is currently employed by Myntra.
 This strategy however is customer independent and treats all customers alike and does not enable identification of elite customers as well as preferential returns processing for elite customers.

 Consider the following strategy $s^*=(s^{*1},s^{*2},\cdots)$ with $s^{*1}$=(immediate refund, comply) and $s^{*k}=s^{*(k-1)}$ if $s^{*(k-1)}$=(immediate refund, comply) else (no immediate refund, don't comply), that is cooperate to start with and don't cooperate once non-cooperation is observed. We note that for this strategy the utility obtained is given by
 \begin{align*}
  &(v_{\text{Myntra}},v_{\text{customer}}) \\
 =&\left(\displaystyle\sum_{k\geq0}\delta^{k} u_{\text{Myntra}}(s^{*k}),\displaystyle\sum_{k\geq0}\delta^{k} u_{\text{customer}}(s^{*k})\right) \\
 =&\left(\displaystyle\sum_{k\geq0}\delta^{k}1^k,\displaystyle\sum_{k\geq0}\delta^{k}1^k\right)
 =\left(\frac{1}{1-\delta},\frac{1}{1-\delta}\right)
 \end{align*}
 Now for any unilateral deviation by a player, at any game repetition stage $k$, the utility $v_{i}=\sum_{0\leq l\leq k-1}\delta^{l}1^l+2\delta^{k}, i\in N$. For this $s^*$ to be an equilibrium, from the definition, we require 
 \begin{align*}
    \displaystyle\sum_{0\leq l\leq k-1}\delta^{l}1^l+2\delta^{k} &\leq \frac{1}{1-\delta} \\
    \implies \frac{1-\delta^{k}}{1-\delta}+2\delta^{k} &\leq \frac{1}{1-\delta}
    \implies  \delta \geq \frac{1}{2}
 \end{align*}
 Hence our strategy $s^*$ is a pure strategy Nash equilibrium if the probability of repetition of the game is at least $0.5$.
 
 On a separate note, a relevant equilibrium concept for the case of repeated games is subgame perfect Nash equilibrium \cite{YN}. We also note that $s^*$ can be shown to be a subgame perfect Nash equilibrium. 
 
 We note here an important observation. Our model of Myntra-customer relationship as a repeated game apart from explaining current Myntra strategy as an equilibrium strategy-there by validating the choice of utlity matrix- suggests an alternative equilibrium strategy vector given by $s^*$. 
 
 We also note here that there are other equilibrium strategies as well. For e.g.
 $s=(s^1,s^2,\cdots)$ with $s^1=$(immediate refund, comply) and $s^{k}=$(immediate refund, comply) if $s^{k-1}=$(immediate refund, comply) or (no immediate refund, does not comply) else $s^{k}=$(no immediate refund, don't comply) is an equilibrium strategy provided $\delta=1$ and each such equilibrium strategy gives an algorithm to identify elite customers. Here we chose a strategy that is most cautious from the point of view of Myntra.
 
 
 With this analysis in place we design the following algorithm that enables us to identify elite customers
%  Note that the set of strategy vectors in our one period game is $A=\{(I,C)(I,DC)(NI,C)(NI,DC)\}$. A strategy vector in $S$ is subgame perfect Nash equilibrium if it induces Nash equilibrium in every subgame.
 
\title{Multi-step  Rolling  Prediction  algorithm   }
\documentclass[12pt]{article}
\usepackage{fullpage}
\usepackage{times}
\usepackage{fancyhdr,graphicx,amsmath,amssymb}
\usepackage[ruled,vlined]{algorithm2e}
\include{pythonlisting}
\begin{document} 
\begin{algorithm}[H]
\SetAlgoLined
Input: train, test\;
Output: predictions\;
history $\leftarrow$ train\;
model1$\leftarrow$ARIMA-GARCH(history, order=(p,d,q)(P,Q))\;
 $N_t$ $\leftarrow$train - model.fittedvalues\;
model2$\leftarrow$SVR($N_t$)\;
for{\State each t  in  range(length(test))} do{
   hat1$\leftarrow$model1.forecast()\;
   hat2$\leftarrow$model2.forecast()\;
   prediction1.append(hat1)\;
   prediction2.append(hat2)\;
   history.append(hat1)\;
   $N_t$.append(hat2)\;
   model1$\leftarrow$ARIMA-GARCH(history, order=(p,d,q)(P,Q))\;
   model2$\leftarrow$SVR($N_t$)\;
 }
 Return prediction = prediction1+prediction2\;
 
 
 \caption{MRPA}
\end{algorithm}


\end{document}

\section{Experiments}\label{sec:expts}

\iffalse
\begin{center}
 \begin{tabular}{|| c | c | c ||} 
 \hline
 Task & RNN(Relu) & RNN(Tanh)  \\ [0.5ex] 
 \hline\hline
  Counter-2 & 0.15 & 0.1\\
  Counter-3 & 0.02 & 0.02\\
  Counter-4 & 0.06 & 0.02 \\
  Boolean-3 & 0.99 & 0.756 \\
  Boolean-5 & 0.0 & 0.766\\
  Shuffle-2 & 0.966 & 0.60\\
  Shuffle-4 & 0.599 & 0.202\\
  Shuffle-6 & 0.348 & 0.338\\ [1ex]
 \hline
\end{tabular}
\end{center}
\fi 


%\iffalse
\begin{figure}[!ht]
\centering
\iffalse
\includegraphics[width=.45\textwidth]{../ESN_RNN_normalizedseq/RNN_2.png}
\includegraphics[width=.45\textwidth]{../ESN_RNN_normalizedseq/RNN_4.png}
\includegraphics[width=.45\textwidth]{../ESN_RNN_normalizedseq/RNN_8.png}
\fi
%\iffalse
\begin{subfigure}
  \centering
  \includegraphics[width=0.5\linewidth]{ESN_RNN_normalizedseq/RNN_2.png}
%          \vspace{-2\baselineskip}
  \caption{Data dimension: 2}
\end{subfigure}
\begin{subfigure}
  \centering
  \includegraphics[width=0.5\linewidth]{ESN_RNN_normalizedseq/RNN_4.png}
%\vspace{-2\baselineskip}
  \caption{Data dimension: 4}
\end{subfigure}%\hfill
\begin{subfigure}
  \centering
  \includegraphics[width=0.5\linewidth]{ESN_RNN_normalizedseq/RNN_8.png}
%\vspace{-2\baselineskip}
  \caption{Data dimension: 8}
\end{subfigure}%
%\fi
\caption{Invertibitiliy of RNNs at random initialization: Checking behavior of inversion error with number of neurons and the sequence length at different data dimensions.}
\label{fig:RNN_inver}
\end{figure}
%\fi 
\textbf{RNN inversion at random initialization.} We consider a randomly initialized RNN, with the entries of the weights $\mathbf{W}$ and $\mathbf{A}$ randomly picked from the distribution $\mathcal{N}(0, 1)$. Sequences are generated i.i.d. from normal distribution i.e. for each sequence, $\bx^{(i)} \sim N(0, \mathbf{I})$ for each $i \in [L]$. We use SGD with batch size 128, momentum $0.9$ and learning rate $0.1$ to compute the linear matrix $\obW^{[L]}$ so that $\norm[0]{\obW^{[L]} \mathbf{h}^{(L)} - [\bx^{(1)}, \ldots, \bx^{(L)}]}^2$ is minimized. We compute the following two quantities on the test dataset, containing $1000$ sequences: average $L_2$ error given by $\mathbb{E}_{\bx} \frac{\norm[0]{\obW^{[L]} \mathbf{h}^{(L)} - [\bx^{(1)}, \ldots, \bx^{(L)}]}}{\norm[0]{[\bx^{(1)}, \ldots, \bx^{(L)}]}}$ and average $L_\infty$ error given by $\mathbb{E}_{\bx} \norm[0]{\obW^{[L]} \mathbf{h}^{(L)} - [\bx^{(1)}, \ldots, \bx^{(L)}]}_{\infty}$. We plot both the quantities for different settings of data dimension $d$, sequence length $L$ and the number of neurons $m$. $L$ takes values from the set $\{2, 4, 6\}$, $d$ takes from $\{2, 4, 8\}$ and $m$ takes from $\{500, 1000, 2000, 5000, 10000\}$ (Figure~\ref{fig:RNN_inver}). The trends support our bounds in Theorem~\ref{thm:Invertibility_ESN_outline}, i.e. the error increases with increasing $L$ and decreases with increasing $m$. Note that the data distribution is different from the one assumed in normalized sequence Def.~\ref{def:normalized_seq}. It was easier to conduct experiments in the current data setting and a similar statement as Thm.~\ref{thm:Invertibility_ESN_outline} can be given.



\textbf{Performance of RNNs on different regular languages. } We check the performance of RNNs on the formal language recognition task for a wide variety of regular languages. We follow the set-up in \cite{BhattamishraAG20} who conducted experiments on LSTMs etc. but not on RNNs.

%\section{Regular languages} \label{sec:regular}
We consider the regular languages as considered in \cite{BhattamishraAG20}.
Tomita grammars \cite{tomita:cogsci82} contain 7
regular languages representable by DFAs of small
sizes, a popular benchmark for evaluating recurrent models (see references in \cite{BhattamishraAG20}). 
We reproduce the definitions of the Tomita grammars from there verbatim:
Tomita Grammars are 7 regular langauges defined on the alphabet $\Sigma = \{0, 1\}$.
Tomita-1 has the regular expression $1^\ast$.
Tomita-2 is defined by the regular expression $(10)^\ast$.
Tomita-3 accepts the strings where odd number
of consecutive 1s are always followed by an even
number of $0$'s. Tomita-4 accepts the strings that
do not contain three consecutive $0$'s. In Tomita-5 only
the strings containing an even number of $0$'s and
even number of $1$'s are allowed. In Tomita-6 the
difference in the number of $1$'s and $0$'s should be
divisible by 3 and finally, Tomita-7 has the regular
expression $0^\ast 1^\ast 0^\ast 1^\ast$. 

We also check the performance of RNNs on $\mathrm{Parity}$, which contains all languages with strings of the form $(w_1, \ldots, w_L)$ s.t. $w_1 + \ldots + w_L = 1 \mod 2$. Languages $\mathcal{D}_n$ are recursively defined as the set of all strings of the form $(0w1)^{\ast}$, where $w \in \mathcal{D}_{n-1}$, with $\mathcal{D}_0$ containing only $\epsilon$, the empty word. Other languages considered are $(00)^{\ast}$, $(0101)^{\ast}$ and $(00)^{\ast}(11)^{\ast}$. Table~\ref{table:regular} shows the number of examples in train and test data, the range of the length of the strings in the language, and the test accuracy of the RNNs with activation functions $\relu$ and $\tanh$ on the regular languages mentioned above. 

\begin{center}
\begin{table}[!ht]
\centering
\begin{tabular}{|| c | c | c | c | c ||} 
 \hline
 Task & No. of Training/Test examples  & Range of length of strings & RNN(Relu) & RNN(Tanh)  \\ [0.5ex] 
 \hline\hline
 Tomita 1 & 50/100 & [2, 50] & 1.0 & 1.0 \\
 Tomita 2 & 25/50 & [2, 50] & 1.0 & 1.0 \\
 Tomita 3 & 10000/2000 & [2, 50] & 1.0 & 1.0 \\
 Tomita 4 & 10000/2000 & [2, 50] & 1.0 & 1.0 \\
 Tomita 5 & 10000/2000 & [2, 50] & 1.0 & 1.0\\
 Tomita 6 & 10000/2000 & [2, 50] & 1.0 & 1.0\\
 Tomita 7 & 10000/2000 & [2, 50] & 0.259 & 0.99\\
 Parity & 10000/2000 & [2, 50] & 1.0 & 1.0\\
 $\mathcal{D}_2$ & 10000/2000 & [2, 100] & 1.0 & 1.0 \\
 $\mathcal{D}_3$ & 10000/2000 & [2, 100] & 0.99 & 1.0\\
 $\mathcal{D}_4$ & 10000/2000 & [2, 100] & 1.0 & 0.99\\
 $(00)^{\ast}$ & 250/50 & [2, 500] & 1.0 & 1.0\\
 $(0101)^{\ast}$ & 125/25 & [4, 500] & 0.99 & 1.0 \\
 $(00)^{\ast}(11)^{\star}$ & 10000/2000 & [2, 200] & 0.99 & 1.0 
 %\\Dyck-1 & 0.99 & \textbf{0.91} 
 \\[1ex]
 \hline
\end{tabular}
\caption{Performance of RNNs on different regular languages.}
\label{table:regular}
\end{table}
\end{center}

%The test languages consist of Tomita grammars which constitute a popular benchmark, and a number of other languages including $\mathsf{PARITY}$ and cover a variety of capabilities needed to recognize regular languages.
%The details of the regular languages above can be found in \cite{BhattamishraAG20}; we only note that Tomita languages constitute a popular benchmark. 
We vary $m$, the dimension of the hidden state, in the range $[3, 32]$, used RMSProp optimizer~\cite{hinton2014coursera} with the smoothing constant $\alpha = 0.99$ and varied the learning rate in the range $[10^{-2}, 10^{-3}]$. For each language
we train models corresponding to each language
for $100$ epochs and a batch size of $32$. We experimented with two different activations $\relu$ and $\tanh$. 
%The best test accuracies achieved on different languages are given in table~\ref{table:regular} and these were all achieved for $m=32$.
In all but one case (Tomita 7 with ReLU) the test accuracies with near-perfect. This was the case across runs. Tomita 7 results could perhaps be improved by more extensive hyperparameter tuning. 
We train and test on strings of length up to 50, and in a few cases strings of larger lengths (when the number of strings in the language is small). 
%Details are in Appendix~\ref{sec:regular}.




\iffalse
\begin{figure*}

\centering
\includegraphics[width=.45\textwidth]{../ESN_RNN_normalizedseq/RNN_2.png}
\includegraphics[width=.45\textwidth]{../ESN_RNN_normalizedseq/RNN_4.png}
\includegraphics[width=.5\textwidth]{../ESN_RNN_normalizedseq/RNN_8.png}
\caption{Invertibility of RNNs at initialization}
%\label{fig:figure3}
\end{figure*}
%\FloatBarrier


\begin{center}
\begin{table}[!ht]
\centering
\begin{tabular}{|| c | c | c ||} 
 \hline
 Task & RNN(Relu) & RNN(Tanh)  \\ [0.5ex] 
 \hline\hline
 Tomita 1 & 1.0 & 1.0 \\
 Tomita 2 & 1.0 & 1.0 \\
 Tomita 3 & 1.0 & 1.0 \\
 Tomita 4 & 1.0 & 1.0 \\
 Tomita 5 & 1.0 & 1.0\\
 Tomita 6 & 1.0 & 1.0\\
 Tomita 7 & 0.259 & 0.99\\
 Parity & 1.0 & 1.0\\
 $\mathcal{D}_1$ & 1.0 & 1.0 \\
 $\mathcal{D}_2$ & 0.99 & 1.0\\
 $\mathcal{D}_4$ & 1.0 & 0.99\\
 $(aa)^{\ast}$ & 1.0 & 1.0\\
 $(abab)^{\ast}$ & 0.99 & 1.0 \\
 $(aa)^{\ast}(bb)^{\star}$ & 0.99 & 1.0 
 %\\Dyck-1 & 0.99 & \textbf{0.91} 
 \\[1ex]
 \hline
\end{tabular}
\caption{Performance of RNNs on different regular languages.}
\label{table:regular}
\end{table}
\end{center}
\fi


\section{Conclusion}
\label{sec:conclusion}
This paper presents a generic top-$\size$ recommendation framework for  trading-off accuracy, novelty, and coverage. To achieve this, we profile the users according to their preference for long-tail novelty. We examine various measures, and formulate an optimization problem to learn these user preferences from interaction data.  We then integrate the user preference estimates in our generic framework, GANC.  Extensive experiments on several datasets confirm that there are trade-offs between accuracy, coverage, and novelty. Almost all re-ranking models increase coverage and novelty at the cost of accuracy. However, existing re-ranking models typically rely on rating prediction models, and are hence more effective in dense settings. Using a generic approach, we can easily incorporate a suitable base accuracy recommender to devise an effective solution for both sparse and dense settings.  %Our results  also indicate there is no single method that outperforms other methods in all metrics. However our techniques obtain a significant improvement in coverage, while  . 
Although we integrated the  long-tail novelty preference estimates into a re-ranking framework, their use-case is not limited to these frameworks. In  the future, we intend to explore the temporal and topical dynamics of long-tail novelty preference, particularly in settings where contextual information is  available.  
%We achieve these objectives without using any additional contextual information.


\iffalse
While we focused on promoting long-tail items across users, we did not consider diversity of individual top-$\size$ recommendations, a factor that has been shown to positively affect consumer satisfaction. This is one direction for future work. Moreover, the sequential setting  in our work, creates a dependency between different batches, where,  the items recommended to a batch of users, depends on those recommended to previous batches. This dependency is created through the parameter $\mathbf{f}$, that is updated every time a top-$\size$ set  is allocated to a batch of users. A future direction for our work is to estimate a distribution over $\mathbf{f}$ that allows us to independently solve the problem for each user, leading to improvements across all performance metrics, including recommendation time. 

We design algorithms that take advantage of the structure in the value functions to obtain both efficient and scalable solutions. 
We design an algorithm that takes advantage of the structure in the value functions to obtain both efficient and scalable solutions. 

\textcolor{red}{Our  sequential  algorithms can be applied for batch recommendation contexts,~e.g., personalized email marketing, where based on prior interaction data between users and items,  a new round of recommendations must be sent to all users in the system.  However, the independent coverage algorithms lift the sequential setting restrictions and allow it be applied for re-ranking the output of base recommender in any setting. }A future direction for our work is to incorporate explicit diversity metrics in the framework. 
\fi


%We have a presented a submodular maximization framework to systematically trade-off relevance and diversity in recommendations to individual users and coverage across the item-space. This ensures both consumer and producer satisfaction. We model users according to their risk and focusing degrees and promote long-tail items to the right group of consumers. Consequently, we obtain a significant improvement in coverage while maintaining reasonable levels of user satisfaction. Furthermore, our methods are able to achieve a more balanced distribution across the set of recommended items. In the future, we plan to investigate the effect of using alternative base recommender systems. 

%Future Work
%However most of these methods assume that the ratings are missing at random (MAR). Since our method of generating recommendations is based on the completed matrix, assuming MAR might introduce additional bias, we will use methods which assume that the ratings at missing not at random (MNAR),explored in~\cite{steck2010training, icml2014c2_hernandez-lobatob14}. 	 
%Long Tail %Recently, authors in~\cite{cremonesi2010performance} conducted extensive experiments to evaluate the performances of various matrix factorization-based algorithms and neighborhood models on the task of recommending long tail items. Their experimental results show that long tail recommendation leads to a decrease in accuracy for all algorithms. They also showed that for this task, SVD outperforms other state-of-the-art algorithms. 




\bibliographystyle{plain}
\bibliography{references}
\end{document}

% End of ltexpprt.tex