\section{Introduction}
Game theory is a method of studying strategic interactions. It has it's origins in the works of John von Neumann
and Oskar Morgenstern \cite{OandJ}. Later works by prominent mathematicians and economists like John Nash, Reinhard Selten, Eric Maskin, Roger Myerson  etc., established the field of game theory with applications to a diverse range of areas like computer science, social science, biology, logic etc. 

Scientific study of customer interaction is an important problem in e-commerce. For e.g.  \cite{Ge} identifies factors that play a role in customer loyalty towards e-commerce. \cite{Ke} focuses on the value of e-commerce to a customer. \cite{HK} describes interface design for better customer interaction with the e-commerce platform. \cite{BG} empirically investigates the impact of social media on customer engagement in e-commerce.  

One way to mathematically model and investigate customer interaction with an e-commerce platform is via game theory. For e.g.
\cite{RDV} analyses customer behaviour as a function of reputation of the seller in e-commerce systems utilizing game theory methodologies.
\cite{GM} proposes tools based on game theory for better customer satisfaction in e-commerce. \cite{LLZ} models customer's after service interaction as a game and analyses it and suggests promotion strategies.
% \cite{LiLi} classifies customers in e-commerce using k-means algorithm

Here in our work in fashion e-commerce, we apply equilibrium concepts developed by Nash and Selten and design a novel classification algorithm for elite customer identification. In particular we model the customer-Myntra relationship as a repeated game and identify equilibrium strategy vectors. Based on this analysis, we then design a classification algorithm that identifies eligible/elite customers to enable preferential return processing for them. Similar type of work is present in economics literature. For e.g.\cite{CCB} models interaction between two anonymous economies as a repeated game and analyses equilibrium strategies. However our novelty is in the application of similar techniques in fashion e-commerce and in particular modeling customer-Myntra interaction. Our key contributions are summarized below. 

\subsection*{Our contributions:}
\begin{itemize}
    \item We model the problem of elite customer identification at Myntra as a repeated game.
    \item We solve the game for an equilibrium strategy.
    \item With the equilibrium strategy as a guide we design an algorithm that identifies elite customers.
    \item We compare and evaluate the algorithm on real world data available at Myntra.
\end{itemize}
