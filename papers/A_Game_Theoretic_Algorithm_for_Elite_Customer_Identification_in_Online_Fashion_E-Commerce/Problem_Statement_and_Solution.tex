\section{Problem Statement}
At Myntra, a customer's product return request is processed in the following manner. Once a customer places a product return request, a doorstep pickup agent arrives and receives the product. After a quality check of the returned item at a designated location refund to the customer is initiated.

For a better customer experience we wish to identify elite customers and do away with this quality check process for these elite customers. To identify these elite customers, we model the interaction between Myntra and a generic customer as a game.

We have two players in our game i.e. $N=\{ \text{Myntra}, \text{customer}\}$. Myntra has two actions i.e. $A_{\text{Myntra}}=\{\text{immediate refund},\text{no immediate refund}\}$. Similarly customer has two actions i.e. $A_{\text{customer}}=\{\text{comply with return requirements}, \\ \text{don't comply with return requirements}\}$.
Here in the context of Myntra the strategy/action ``immediate refund" refers to the removal of quality check process.

We designed the utility matrix  given in Table \ref{utility_matrix} for this game.
Our rationality behind this particular choice for the utility matrix is as follows. For Myntra, with $u_{\text{Myntra}}\text{(No Immediate Refund, Don't Comply)}=0$ as baseline, $u_{\text{Myntra}}\text{(Immediate Refund, Comply)}=1$, 
$u_{\text{Myntra}}\text{(No Immediate Refund, Comply)}=2$ and $u_{\text{Myntra}}\text{(Immediate Refund, Don't Comply)}=-1$ reflect the satisfaction levels of Myntra in the customer-Myntra relationship. We assume the same for the customer and obtain the utility matrix shown in Table \ref{utility_matrix}.
  \begin{table}
    \setlength{\extrarowheight}{2pt}
    \begin{tabular}{*{4}{c|}}
      \multicolumn{2}{c}{} & \multicolumn{2}{c}{Customer}\\\cline{3-4}
      \multicolumn{1}{c}{} &  & Comply  & Don't Comply \\\cline{2-4}
      \multirow{2}*{Myntra}  & Immediate &    &  \\
      & Refund & $(1,1)$ & $(-1,2)$ \\\cline{2-4}
      & No Immediate &      &  \\
      & Refund & $(2,-1)$ & $(0,0)$ \\\cline{2-4}
    \end{tabular}
     \caption{Utility matrix of our game}
      \label{utility_matrix}
  \end{table}
 \vspace*{-\baselineskip}
 \section{Solution}
 Observe that (No Immediate Refund, Don't Comply) is a pure strategy Nash equilibrium for this game and No Immediate Refund is the corresponding strategy for Myntra. However we note that the relation between Myntra and the customer is an ongoing relation. Hence a repeated game that repeats with probability $\delta$ is a more appropriate model for Myntra-customer relationship.
 In this context the set of strategy vectors is given by $S= (A_{\text{Myntra}} \times A_{\text{customer}})^{\mathbb{N}}$, set of sequences with values in $A_{\text{Myntra}} \times A_{\text{customer}}.$ Here $\mathbb{N}$ denotes the set of Natural numbers. The utility function $v_{\text{Myntra}}:S\rightarrow \mathbb{R}$ is given by 
 $v_{\text{Myntra}}(s)=\sum_{k\geq0}\delta^{k} u_{\text{Myntra}}(s^{k+1})$ where $s=(s^{1},s^{2},\cdots)$ with $s^{k} \in A_{\text{Myntra}} \times A_{\text{customer}}, k \in \mathbb{N}$ and $u_{\text{Myntra}}$ is given by Table \ref{utility_matrix}. Similarly the utility function 
 $v_{\text{customer}}:S\rightarrow \mathbb{R}$ is given by 
 $v_{\text{customer}}(s)=\sum_{k\geq0}\delta^{k} u_{\text{customer}}(s^{k})$ where $s=(s^{1},s^{2},\cdots)$ with $s^{k} \in A_{\text{Myntra}} \times A_{\text{customer}}, k \in \mathbb{N}$ and $u_{\text{customer}}$ is given by Table \ref{utility_matrix}.
 
 Observe that in this setup the strategy vector $s^{*}=(s^{*i})_{i\in \mathbb{N}}$ with $s^{*i}=(\text{no immediate refund},\text{don't comply})$ is an equilibrium strategy vector with $(v_{\text{Myntra}},v_{\text{customer}})=(0,0)$ as any unilateral deviation in any repetition of the game by a player diminishes the utility of the player. Moreover the equilibrium strategy ``no immediate refund" is currently employed by Myntra.
 This strategy however is customer independent and treats all customers alike and does not enable identification of elite customers as well as preferential returns processing for elite customers.

 Consider the following strategy $s^*=(s^{*1},s^{*2},\cdots)$ with $s^{*1}$=(immediate refund, comply) and $s^{*k}=s^{*(k-1)}$ if $s^{*(k-1)}$=(immediate refund, comply) else (no immediate refund, don't comply), that is cooperate to start with and don't cooperate once non-cooperation is observed. We note that for this strategy the utility obtained is given by
 \begin{align*}
  &(v_{\text{Myntra}},v_{\text{customer}}) \\
 =&\left(\displaystyle\sum_{k\geq0}\delta^{k} u_{\text{Myntra}}(s^{*k}),\displaystyle\sum_{k\geq0}\delta^{k} u_{\text{customer}}(s^{*k})\right) \\
 =&\left(\displaystyle\sum_{k\geq0}\delta^{k}1^k,\displaystyle\sum_{k\geq0}\delta^{k}1^k\right)
 =\left(\frac{1}{1-\delta},\frac{1}{1-\delta}\right)
 \end{align*}
 Now for any unilateral deviation by a player, at any game repetition stage $k$, the utility $v_{i}=\sum_{0\leq l\leq k-1}\delta^{l}1^l+2\delta^{k}, i\in N$. For this $s^*$ to be an equilibrium, from the definition, we require 
 \begin{align*}
    \displaystyle\sum_{0\leq l\leq k-1}\delta^{l}1^l+2\delta^{k} &\leq \frac{1}{1-\delta} \\
    \implies \frac{1-\delta^{k}}{1-\delta}+2\delta^{k} &\leq \frac{1}{1-\delta}
    \implies  \delta \geq \frac{1}{2}
 \end{align*}
 Hence our strategy $s^*$ is a pure strategy Nash equilibrium if the probability of repetition of the game is at least $0.5$.
 
 On a separate note, a relevant equilibrium concept for the case of repeated games is subgame perfect Nash equilibrium \cite{YN}. We also note that $s^*$ can be shown to be a subgame perfect Nash equilibrium. 
 
 We note here an important observation. Our model of Myntra-customer relationship as a repeated game apart from explaining current Myntra strategy as an equilibrium strategy-there by validating the choice of utlity matrix- suggests an alternative equilibrium strategy vector given by $s^*$. 
 
 We also note here that there are other equilibrium strategies as well. For e.g.
 $s=(s^1,s^2,\cdots)$ with $s^1=$(immediate refund, comply) and $s^{k}=$(immediate refund, comply) if $s^{k-1}=$(immediate refund, comply) or (no immediate refund, does not comply) else $s^{k}=$(no immediate refund, don't comply) is an equilibrium strategy provided $\delta=1$ and each such equilibrium strategy gives an algorithm to identify elite customers. Here we chose a strategy that is most cautious from the point of view of Myntra.
 
 
 With this analysis in place we design the following algorithm that enables us to identify elite customers
%  Note that the set of strategy vectors in our one period game is $A=\{(I,C)(I,DC)(NI,C)(NI,DC)\}$. A strategy vector in $S$ is subgame perfect Nash equilibrium if it induces Nash equilibrium in every subgame.
 