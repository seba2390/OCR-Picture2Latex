\documentclass[runningheads]{llncs}
\usepackage{lettrine}
\usepackage{cite}
%% Figure
\usepackage{graphicx}
\graphicspath{{./figures/}}
\usepackage[labelformat=parens, labelsep=space]{subcaption}
%% Math
\usepackage{lipsum} 
\usepackage[T1]{fontenc}
\usepackage[utf8]{inputenc}
\usepackage{mathtools} 
\DeclarePairedDelimiter\ceil{\lceil}{\rceil}
\DeclarePairedDelimiter\floor{\lfloor}{\rfloor}
\newcommand{\bigdelta}{\raisebox{-.35\baselineskip}{\huge\ensuremath{\Omega}}}
%% Algorithm
\usepackage[linesnumbered,algoruled,boxed,lined]{algorithm2e}
\SetKwInOut{Input}{input}
\SetKwInOut{Output}{output}
\SetKw{Continue}{continue}
\SetKw{Break}{break}
\newcommand*\mean[1]{\overline{#1}}
\newcommand{\norm}[1]{\left\lVert#1\right\rVert}
%% Tables
\usepackage{textcomp}
\usepackage[table, xcdraw]{xcolor} % The cell in the table can be highlighted
\usepackage{diagbox}
\usepackage{multirow,booktabs,color,soul,threeparttable}
\definecolor{hl}{RGB}{255,255,0}
\sethlcolor{hl}
\DeclareCaptionFont{tablecaptionfont}{\fontsize{9}{9.6}\sc}
\captionsetup[table]{font=tablecaptionfont}
%% Hyperref
\usepackage[colorlinks,citecolor=red,urlcolor=blue,bookmarks=false,hypertexnames=true]{hyperref} 

\begin{document}
\title{Niching Diversity Estimation for Multi-modal Multi-objective Optimization}
\author{Yiming Peng, Hisao Ishibuchi\thanks{Corresponding author: Hisao Ishibuchi, hisao@sustech.edu.cn.}}
\institute{Guangdong Provincial Key Laboratory of Brain-inspired Intelligent Computation, Department of Computer Science and Engineering, \\Southern University of Science and Technology, Shenzhen 518055, China.\\
    \email{11510035@mail.sustech.edu.cn, hisao@sustech.edu.cn}}
%
\maketitle              % typeset the header of the contribution
%
\begin{abstract}
    Niching is an important and widely used technique in evolutionary multi-objective optimization. Its applications mainly focus on maintaining diversity and avoiding early convergence to local optimum. Recently, a special class of multi-objective optimization problems, namely, multi-modal multi-objective optimization problems (MMOPs), started to receive increasing attention. In MMOPs, a solution in the objective space may have multiple inverse images in the decision space, which are termed as equivalent solutions. Since equivalent solutions are overlapping (i.e., occupying the same position) in the objective space, standard diversity estimators such as crowding distance are likely to select one of them and discard the others, which may cause diversity loss in the decision space. In this study, a general niching mechanism is proposed to make standard diversity estimators more efficient when handling MMOPs. In our experiments, we integrate our proposed niching diversity estimation method into SPEA2 and NSGA-II and evaluate their performance on several MMOPs. Experimental results show that the proposed niching mechanism notably enhances the performance of SPEA2 and NSGA-II on various MMOPs.

    \keywords{Niching \and Diversity Estimation \and Multi-modal Multi-objective Optimization.}
\end{abstract}
% Sections
\IEEEraisesectionheading{\section{Introduction}}

\IEEEPARstart{V}{ision} system is studied in orthogonal disciplines spanning from neurophysiology and psychophysics to computer science all with uniform objective: understand the vision system and develop it into an integrated theory of vision. In general, vision or visual perception is the ability of information acquisition from environment, and it's interpretation. According to Gestalt theory, visual elements are perceived as patterns of wholes rather than the sum of constituent parts~\cite{koffka2013principles}. The Gestalt theory through \textit{emergence}, \textit{invariance}, \textit{multistability}, and \textit{reification} properties (aka Gestalt principles), describes how vision recognizes an object as a \textit{whole} from constituent parts. There is an increasing interested to model the cognitive aptitude of visual perception; however, the process is challenging. In the following, a challenge (as an example) per object and motion perception is discussed. 



\subsection{Why do things look as they do?}
In addition to Gestalt principles, an object is characterized with its spatial parameters and material properties. Despite of the novel approaches proposed for material recognition (e.g.,~\cite{sharan2013recognizing}), objects tend to get the attention. Leveraging on an object's spatial properties, material, illumination, and background; the mapping from real world 3D patterns (distal stimulus) to 2D patterns onto retina (proximal stimulus) is many-to-one non-uniquely-invertible mapping~\cite{dicarlo2007untangling,horn1986robot}. There have been novel biology-driven studies for constructing computational models to emulate anatomy and physiology of the brain for real world object recognition (e.g.,~\cite{lowe2004distinctive,serre2007robust,zhang2006svm}), and some studies lead to impressive accuracy. For instance, testing such computational models on gold standard controlled shape sets such as Caltech101 and Caltech256, some methods resulted $<$60\% true-positives~\cite{zhang2006svm,lazebnik2006beyond,mutch2006multiclass,wang2006using}. However, Pinto et al.~\cite{pinto2008real} raised a caution against the pervasiveness of such shape sets by highlighting the unsystematic variations in objects features such as spatial aspects, both between and within object categories. For instance, using a V1-like model (a neuroscientist's null model) with two categories of systematically variant objects, a rapid derogate of performance to 50\% (chance level) is observed~\cite{zhang2006svm}. This observation accentuates the challenges that the infinite number of 2D shapes casted on retina from 3D objects introduces to object recognition. 

Material recognition of an object requires in-depth features to be determined. A mineralogist may describe the luster (i.e., optical quality of the surface) with a vocabulary like greasy, pearly, vitreous, resinous or submetallic; he may describe rocks and minerals with their typical forms such as acicular, dendritic, porous, nodular, or oolitic. We perceive materials from early age even though many of us lack such a rich visual vocabulary as formalized as the mineralogists~\cite{adelson2001seeing}. However, methodizing material perception can be far from trivial. For instance, consider a chrome sphere with every pixel having a correspondence in the environment; hence, the material of the sphere is hidden and shall be inferred implicitly~\cite{shafer2000color,adelson2001seeing}. Therefore, considering object material, object recognition requires surface reflectance, various light sources, and observer's point-of-view to be taken into consideration.


\subsection{What went where?}
Motion is an important aspect in interpreting the interaction with subjects, making the visual perception of movement a critical cognitive ability that helps us with complex tasks such as discriminating moving objects from background, or depth perception by motion parallax. Cognitive susceptibility enables the inference of 2D/3D motion from a sequence of 2D shapes (e.g., movies~\cite{niyogi1994analyzing,little1998recognizing,hayfron2003automatic}), or from a single image frame (e.g., the pose of an athlete runner~\cite{wang2013learning,ramanan2006learning}). However, its challenging to model the susceptibility because of many-to-one relation between distal and proximal stimulus, which makes the local measurements of proximal stimulus inadequate to reason the proper global interpretation. One of the various challenges is called \textit{motion correspondence problem}~\cite{attneave1974apparent,ullman1979interpretation,ramachandran1986perception,dawson1991and}, which refers to recognition of any individual component of proximal stimulus in frame-1 and another component in frame-2 as constituting different glimpses of the same moving component. If one-to-one mapping is intended, $n!$ correspondence matches between $n$ components of two frames exist, which is increased to $2^n$  for one-to-any mappings. To address the challenge, Ullman~\cite{ullman1979interpretation} proposed a method based on nearest neighbor principle, and Dawson~\cite{dawson1991and} introduced an auto associative network model. Dawson's network model~\cite{dawson1991and} iteratively modifies the activation pattern of local measurements to achieve a stable global interpretation. In general, his model applies three constraints as it follows
\begin{inlinelist}
	\item \textit{nearest neighbor principle} (shorter motion correspondence matches are assigned lower costs)
	\item \textit{relative velocity principle} (differences between two motion correspondence matches)
	\item \textit{element integrity principle} (physical coherence of surfaces)
\end{inlinelist}.
According to experimental evaluations (e.g.,~\cite{ullman1979interpretation,ramachandran1986perception,cutting1982minimum}), these three constraints are the aspects of how human visual system solves the motion correspondence problem. Eom et al.~\cite{eom2012heuristic} tackled the motion correspondence problem by considering the relative velocity and the element integrity principles. They studied one-to-any mapping between elements of corresponding fuzzy clusters of two consecutive frames. They have obtained a ranked list of all possible mappings by performing a state-space search. 



\subsection{How a stimuli is recognized in the environment?}

Human subjects are often able to recognize a 3D object from its 2D projections in different orientations~\cite{bartoshuk1960mental}. A common hypothesis for this \textit{spatial ability} is that, an object is represented in memory in its canonical orientation, and a \textit{mental rotation} transformation is applied on the input image, and the transformed image is compared with the object in its canonical orientation~\cite{bartoshuk1960mental}. The time to determine whether two projections portray the same 3D object
\begin{inlinelist}
	\item increase linearly with respect to the angular disparity~\cite{bartoshuk1960mental,cooperau1973time,cooper1976demonstration}
	\item is independent from the complexity of the 3D object~\cite{cooper1973chronometric}
\end{inlinelist}.
Shepard and Metzler~\cite{shepard1971mental} interpreted this finding as it follows: \textit{human subjects mentally rotate one portray at a constant speed until it is aligned with the other portray.}



\subsection{State of the Art}

The linear mapping transformation determination between two objects is generalized as determining optimal linear transformation matrix for a set of observed vectors, which is first proposed by Grace Wahba in 1965~\cite{wahba1965least} as it follows. 
\textit{Given two sets of $n$ points $\{v_1, v_2, \dots v_n\}$, and $\{v_1^*, v_2^* \dots v_n^*\}$, where $n \geq 2$, find the rotation matrix $M$ (i.e., the orthogonal matrix with determinant +1) which brings the first set into the best least squares coincidence with the second. That is, find $M$ matrix which minimizes}
\begin{equation}
	\sum_{j=1}^{n} \vert v_j^* - Mv_j \vert^2
\end{equation}

Multiple solutions for the \textit{Wahba's problem} have been published, such as Paul Davenport's q-method. Some notable algorithms after Davenport's q-method were published; of that QUaternion ESTimator (QU\-EST)~\cite{shuster2012three}, Fast Optimal Attitude Matrix \-(FOAM)~\cite{markley1993attitude} and Slower Optimal Matrix Algorithm (SOMA)~\cite{markley1993attitude}, and singular value decomposition (SVD) based algorithms, such as Markley’s SVD-based method~\cite{markley1988attitude}. 

In statistical shape analysis, the linear mapping transformation determination challenge is studied as Procrustes problem. Procrustes analysis finds a transformation matrix that maps two input shapes closest possible on each other. Solutions for Procrustes problem are reviewed in~\cite{gower2004procrustes,viklands2006algorithms}. For orthogonal Procrustes problem, Wolfgang Kabsch proposed a SVD-based method~\cite{kabsch1976solution} by minimizing the root mean squared deviation of two input sets when the determinant of rotation matrix is $1$. In addition to Kabsch’s partial Procrustes superimposition (covers translation and rotation), other full Procrustes superimpositions (covers translation, uniform scaling, rotation/reflection) have been proposed~\cite{gower2004procrustes,viklands2006algorithms}. The determination of optimal linear mapping transformation matrix using different approaches of Procrustes analysis has wide range of applications, spanning from forging human hand mimics in anthropomorphic robotic hand~\cite{xu2012design}, to the assessment of two-dimensional perimeter spread models such as fire~\cite{duff2012procrustes}, and the analysis of MRI scans in brain morphology studies~\cite{martin2013correlation}.

\subsection{Our Contribution}

The present study methodizes the aforementioned mentioned cognitive susceptibilities into a cognitive-driven linear mapping transformation determination algorithm. The method leverages on mental rotation cognitive stages~\cite{johnson1990speed} which are defined as it follows
\begin{inlinelist}
	\item a mental image of the object is created
	\item object is mentally rotated until a comparison is made
	\item objects are assessed whether they are the same
	\item the decision is reported
\end{inlinelist}.
Accordingly, the proposed method creates hierarchical abstractions of shapes~\cite{greene2009briefest} with increasing level of details~\cite{konkle2010scene}. The abstractions are presented in a vector space. A graph of linear transformations is created by circular-shift permutations (i.e., rotation superimposition) of vectors. The graph is then hierarchically traversed for closest mapping linear transformation determination. 

Despite of numerous novel algorithms to calculate linear mapping transformation, such as those proposed for Procrustes analysis, the novelty of the presented method is being a cognitive-driven approach. This method augments promising discoveries on motion/object perception into a linear mapping transformation determination algorithm.



\section{Related work}\label{sect:related}

\paragraph{{Recovery}} {The works most closely most closely related to ours are those based on the \emph{recovery} notion, that is, the type system of Gordon et al. \cite{GordonEtAl12} and the Pony language  \cite{ClebschEtAl15}.} Indeed, the capsule property has many variants in the literature, such as \emph{isolated} \cite{GordonEtAl12}, \emph{uniqueness} \cite{Boyland10} and \emph{external uniqueness}~\cite{ClarkeWrigstad03}, \emph{balloon} \cite{Almeida97,ServettoEtAl13a}, \emph{island} \cite{DietlEtAl07}. 
%The fact that aliasing can be controlled by using \emph{lent} (\emph{borrowed}) references is well-known~\cite{Boyland01,NadenEtAl12}.
However, before the work of Gordon et al. \cite{GordonEtAl12}, the capsule property was only ensured in simple situations, such as using a primitive deep clone operator, or composing subexpressions with the same property.

The important novelty of the type system of Gordon et al. \cite{GordonEtAl12} has been \emph{recovery}, that is, the ability to ensure properties (e.g., capsule or immutability) by keeping into account not only the expression itself but the way the surrounding context is used. {Notably,} an expression which does not use external mutable references is recognized to be a capsule. 
{In the Pony language \cite{ClebschEtAl15}  the ideas of Gordon et al. \cite{GordonEtAl12} are extended to a richer set of reference immutability permissions. In their terminology \texttt{value} is immutable, \texttt{ref} is mutable, \texttt{box} is similar to \emph{readonly} as often found in literature, different from our $\readable$ since it can be aliased. An ephemeral isolated reference \lstinline{iso^} is similar to a $\capsule$ reference in our calculus, whereas non ephemeral \texttt{iso} references offer destructive reads and are more
similar to isolated fields \cite{GordonEtAl12}. Finally, \texttt{tag} only allows object identity checks and \texttt{trn} (transition) is a subtype of \texttt{box} that can be converted to \texttt{value}, providing a way to create values without using isolated references. The last two qualifiers have no equivalent in our
work or in  \cite{GordonEtAl12}.}

Our {type system greatly enhances the recovery mechanism used in such previous work \cite{GordonEtAl12,ClebschEtAl15} by using lent references, and rules \rn{t-swap} and \rn{t-unrst}.} For instance, the examples in \refToFigure{TypingOne} and \refToFigure{TypingTwo} would be ill-typed in \cite{GordonEtAl12}. 

{A minor difference with the type systems of Gordon et al. \cite{GordonEtAl12} and Pony \cite{GordonEtAl12,ClebschEtAl15} is that we only allow fields to be $\mutable$ or $\imm$.
Allowing \emph{readonly} fields means holding a reference that is useful for observing but non making remote modifications. However, our type system supports the $\readable$ modifier rather than the \emph{readonly}, and the $\readable$ qualifier includes the $\lent$ restriction. Since something which is $\lent$ cannot be saved as part of a $\mutable$ object, $\lent$ fields are not compatible with the current design where objects are born $\mutable$. The motivation for supporting $\readable$ rather than \emph{readonly} is discussed in a specific point later.
Allowing $\capsule$ fields means that programs can store an externally unique object graph into the heap and decide later whether to unpack
 permanently or freeze the reachable objects.  This can be useful, but, as for $\readable$ versus \emph{readonly}, our opinion is that this power is hard to use for good, since
it requires destructive reads, as discussed in a specific point later. 
In most cases, the same expressive power can be achieved by having the
field as $\mutable$ and recovering the $\capsule$ property for the outer object.}

\paragraph{Capabilities}
 {In other proposals \cite{HallerOdersky10,CastegrenWrigstad16} types are compositions of one or more \emph{capabilities}. The modes of the capabilities in a type control how resources of that
type can be aliased. The compositional aspect of capabilities is an important difference
from type qualifiers, as accessing different parts of an object through different capabilities in the same type gives different properties. 
By using capabilities it is possible to obtain an expressivity which looks similar to our type system, even though with different sharing notions and syntactic constructs. For instance, the \emph{full encapsulation} notion in \cite{HallerOdersky10}\footnote{{This paper includes a very good survey of work in this area, notably explaining the difference between \emph{external uniqueness}~\cite{ClarkeWrigstad03} and \emph{full encapsulation}.}}, apart from the fact that sharing of immutable objects is not allowed, is equivalent to the guarantee of our $\capsule$ qualifier, while
our $\lent$ and their \Q|@transient| achieve similar results in different ways.}
Their model has a higher syntactic/logic overhead to explicitly  track regions.
As for all work before~\cite{GordonEtAl12}, objects need to be born \Q|@unique| and the type system 
permits to manipulate data preserving their uniqueness. With recovery~\cite{GordonEtAl12},
instead, we can forget about uniqueness, use normal code designed to work on conventional shared data, and then
recover the aliasing encapsulation property.

\paragraph{Destructive reads} Uniqueness can be enforced by destructive reads, i.e., assigning a copy of 
the unique reference to a variable an destroying the original reference, see
\cite{GordonEtAl12,Boyland10}. Traditionally, borrowing/fractional permissions~\cite{NadenEtAl12} are related to uniqueness  in the opposite way: a unique reference can be borrowed,
it is possible to track when all borrowed aliases are buried~\cite{Boyland01}, and then uniqueness can be recovered.
These techniques offers a sophisticate alternative to destructive reads. 
We also wish to avoid destructive reads. In our work, we ensure uniqueness by linearity, that is, by allowing at most
one use of a $\capsule$ reference.

In our opinion, programming with destructive reads is involved and hurts the correctness of the program, since it leads to the style of programming outlined below, where \Q@a.f@ is a unique/isolated field with destructive read.
\begin{lstlisting}
a.f=c.doStuff(a.f)//style suggested by destructive reads
\end{lstlisting}
The object referenced by \lstinline{a}{} has an \emph{unique/isolated} field \lstinline{f} containing an object \lstinline{b}.
This object \lstinline{b}{} is passed to a client \lstinline{c}{}, which can use (potentially modifying) it. A typical pattern is that the result of such computation is a reference to \lstinline{b}{}, which \lstinline{a}{} can then recover. This approach allows \emph{isolated} fields, as shown above, but has  a serious drawback:
an \emph{isolated} field can become unexpectedly not available (in the example, during execution of \lstinline{doStuff}{}), hence any object contract
involving such field can be broken.
{This can cause {subtle} bugs if \Q@a@ is in the reachable object graph of \Q@c@.}

In our approach, the  $\capsule$ qualifier cannot be applied to fields. Indeed, the ``only once'' use of capsule variables 
makes no sense on fields.
{However, we support the same level of control of the reachable object graph by passing mutable objects to clients as $\lent$, in order to control aliasing behaviour.
That is, the previous code can be rewritten} as follows:
\begin{lstlisting}
c.doStuff(a.f())//our suggested style
\end{lstlisting}
{where \Q@a.f()@ is a getter returning the field as $\lent$.
Note how, during the execution of \Q@doStuff@, \Q@a.f()@ is still available, and,} after the execution of \Q@doStuff@, the aliasing relation {for this field is the same as it was
before \Q@doStuff@ was called.}

\paragraph{Ownership} A {closely related} stream of research is that on \emph{ownership} (see an overview in~\cite{ClarkeEtAl13}) which, however, offers an {opposite} approach. In the ownership approach, it is provided a way to express and prove the ownership invariant\footnote{{Ownership invariant (owner-as-dominator):
Object $o_1$ is owned by object  $o_2$ if in the object graph $o_2$
is a dominator node for $o_1$;
that is, all paths from the roots of the graph (the stack variables)
to $o_1$ pass throw $o_2$.
Ownership invariant (owner-as-modifier):
Object $o_1$ is owned by object  $o_2$ if any field update over $o_1$
is initiated by $o_2$, that is, a call of a method of $o_2$ is present
in the stack trace.}}, which, however, is expected to be guaranteed by defensive cloning, as explained below. In our approach, instead, the capsule concept models an efficient \emph{ownership transfer}. In other words, when an object $\x$ is ``owned'' by another object $\y$, it remains always true that $\y$ can be only accessed only through $\x$, whereas the capsule notion is more dynamic: a capsule can be ``opened'', that is, assigned to a standard reference and modified, and then we can recover the original capsule guarantee. 

For example, assuming a graph with a list of nodes, and a constructor taking in input such list,
the following code establishes the ownership invariant using $\capsule$, and ensures that it cannot be violated using $\lent$.
\begin{lstlisting}
class Graph{
  private final NodeList nodes;
  private Graph(NodeList nodes){this.nodes=nodes; }

  static Graph factory(capsule NodeList nodes){
    return new Graph(nodes);
    }
  
  lent NodeList borrowNodes(mut){return nodes;}
}
\end{lstlisting}
Requiring the parameter of the \lstinline{factory}{} method to be a $\capsule$ guarantees that the list of nodes provided as argument is not referred from the external environment. 
The factory \emph{moves} an isolated portion of store as local store of the newly created object. 
Cloning, if needed, becomes responsibility of the client which provides the list of nodes to the graph. The getter tailors the exposure level of the private store. 

Without aliasing control ($\capsule$ qualifier),  in order to ensure ownership of its list of nodes, the {factory method} should clone the argument, since it comes from an external client environment.
This solution, called  \label{cloning} \emph{defensive cloning}~\cite{Bloch08}, is very popular in the Java community, but inefficient,
since it requires to duplicate the reachable object
graph of the parameter, until immutable nodes are
reached.\footnote{{In most languages, for owner-as-modifier defensive cloning is needed
only when new data is saved inside of an object, while for owner-as-dominator it is needed also when internal data are exposed.}}
Indeed, many programmers prefer to write {unsafe}
 code instead of using defensive cloning for efficiency reasons.
However, this unsafe approach is only possible when programmers have control of the client code, that is, they are not 
working in a library setting.
Indeed many important Java libraries (including the standard Java libraries) today
use defensive cloning to ensure ownership of their internal state.

As mentioned above, our approach is the opposite of the one offered by many ownership approaches, which provide a formal way to express  and prove the ownership invariant that, however, are expected to be guaranteed by defensive cloning. 
We, instead, model an efficient \emph{ownership transfer} through the capsule concept, then, 
duplication of memory, if needed, is performed on the client side\footnote{
Other work in literature supports ownership transfer, see for example~\cite{MullerRudich07, ClarkeWrigstad03}.
In literature it is however applied to uniquess/external uniqueness, thus not {the whole} reachable object graph is transfered.
}.

Moreover, depending on how we expose the owned data, we can closely model
both \emph{owners-as-dominators} (by providing no getter)
and \emph{owners-as-qualifiers} (by providing a \Q@read@ getter). In the example, the method \lstinline{borrowNodes}{} is an example of a $\lent$ getter, a third variant besides the two described on page \pageref{exposer}.  This variant is particularly useful in the case of a field which is owned, indeed, \Q@Graph@ instances can release the mutation control of their nodes without permanently {losing} the aliasing control.

In our approach all properties are deep. On the opposite side, most ownership approaches allows one to distinguish
subparts of the reachable object graph that are referred but not logically owned. This viewpoint has many advantages, for example the Rust language\footnote{\texttt{rust-lang.org}} leverages on ownership to control object deallocation without a garbage collector.
Rust employs a form of uniqueness that can be seen as a restricted ``owners-as-dominators" discipline.  
Rust lifetime parameters behave like additional ownership parameters~\cite{OstlundEtAl08}.

However, in most ownership based approaches 
it is not trivial to encode the concept of full encapsulation, while supporting (open) sub-typing and avoiding defensive cloning.
This depends on how any specific ownership approach entangles subtyping with 
 gaining extra ownership parameters
and extra references to global ownership domains.

\paragraph{Readable notion} Our $\readable$ qualifier is different from \emph{readonly} as used, e.g., in \cite{BirkaErnst04}.
 An object cannot be modified through a readable/readonly reference. However, 
$\readable$ also prevents aliasing.
As discussed in \cite{Boyland06}, readonly semantics can be easily misunderstood by
programmers. Indeed, some wrongly believe it means immutable, whereas the object denoted by a readonly reference can be modified through other references, while others do not realize that readonly data can still be saved in fields, and thus used as a secondary window to observe the change in the object state.
Our proposal addresses both problems, since we explicitly support the $\imm$ qualifier, hence it is more difficult for programmers to confuse the two concepts, and our $\readable$ (readonly + lent) data  cannot be saved in client's fields.

Javari~\cite{TschantzErnst05} also supports the \emph{readonly} type qualifier, and makes a huge design effort to support \emph{assignable} and \emph{mutable} fields, to have fine-grained readonly constraints.  The need of such flexibility is motivated by performance reasons.
In our design philosophy, we do not offer any way of breaking the invariants enforced by the type system. Since our invariants are very strong, we expect compilers to be able to perform optimization, thus recovering most of the efficiency lost to properly use immutable and readable objects.



\section{Proposed Method}
\label{sec: Proposed method}
\subsection{Niching diversity estimation}
In this section, we outline our proposed niching diversity estimation method for multi-modal multi-objective optimization. Here we first introduce a general representation for most diversity estimators used in MOEAs before diving into the details of the proposed method.

Generally, for a solution $\boldsymbol{x}_i$ in a solution set $\boldsymbol{S}$, its diversity regarding $\boldsymbol{S}$ can be expressed as follows:
\begin{equation}
	\textit{Diversity}(\boldsymbol{x}_i, \boldsymbol{S}) = \mathop{\bigdelta}_{\boldsymbol{x}_j\in\boldsymbol{S}, j \neq i}{C(\boldsymbol{x}_i, \boldsymbol{x}_j)},
	\label{eq: Standard diversity estimator}
\end{equation}
where $C$ is a function which calculates the diversity contribution from a pair of solutions $\boldsymbol{x}_i$ and $\boldsymbol{x}_j$ regarding their objective values, and $\Omega$ is an aggregation function (e.g., sum or mean) to combine the diversity contribution from each pair.
The idea of our proposed method is straightforward: to restrict the diversity estimation within a niche. We make the following simple modifications to Eq. (\ref{eq: Standard diversity estimator}):
\begin{equation}
	\textit{Niching-Diversity}(\boldsymbol{x}_i, \boldsymbol{S}) = \textit{Diversity}(\boldsymbol{x}_i, \boldsymbol{S}^\prime),
	\label{eq: Niching diversity estimator}
\end{equation}
where $\boldsymbol{S}^\prime$ contains all solutions in $\boldsymbol{S}$ which are in the same niche as $\boldsymbol{x}_i$. In our paper, the closest $k$ solutions in $\boldsymbol{S}$ to $\boldsymbol{x}_i$ in the \textbf{decision space} are considered as a niche.

From Eq. (\ref{eq: Niching diversity estimator}), we can see that the diversity estimation for each solution is limited to its neighbors in the decision space. With the niching strategy, solution distribution in the decision space is taken into consideration. Take Fig. \ref{fig: Difficulty when handling MMOPs} as an example, if $k = 2$ and crowding distance is used, the two nearest neighbors are selected for each solution (e.g., solution $A$) in the decision space, and the crowding distance is calculated using the selected neighbors in the objective space. Solution $B$ is unlikely to be chosen as a neighbor of solution $A$ (i.e., $B$ is not likely to be used for the crowding distance calculation of $A$). In this manner, the diversity of $A$ and $B$ can be estimated in a desirable manner for maintaining the decision space diversity. From this example, we can see that the proposed niching strategy can help diversity estimators in MOEAs to handle MMOPs properly with an appropriate value of $k$.

Compared to existing approaches we have discussed in Section \ref{sec: Existing multi-modal multi-objective optimization algorithms}, our proposed method does not rely on the actual implementations of diversity estimators. It is a general niching strategy that can be conveniently integrated into most diversity estimators in MOEAs.


\subsection{SPEA2 and NSGA-II with niching diversity estimation}
In this section, we select two classical MOEAs: SPEA2 and NSGA-II to demonstrate the procedure of our proposed niching diversity estimation method into MOEAs. The resulting algorithms are termed Niching-SPEA2 and Niching-NSGA-II, respectively.

In SPEA2 and NSGA-II, diversity estimation is only involved in the environmental selection procedure although the estimated diversity values may be used in other procedures. Therefore, we only describe the modified versions of environmental selection.

In the environmental selection procedure of Niching-SPEA2, the niching strategy is applied to both fitness calculation and archive truncation as outlined in lines 4 and 13 in Algorithm \ref{algo: Niching SPEA2}. In these two procedures, distance calculation is restricted by the niching strategy. For Niching-NSGA-II, in each generation, non-dominated sorting is employed to rank the whole population into several fronts. Afterward, the crowding distance is computed in each front with the proposed niching strategy.

\begin{algorithm}
	\caption{Environmental Selection Procedure of Niching-SPEA2.}
	\label{algo: Niching SPEA2}
	\Input{$\boldsymbol{P}$: input population;\\$N$: the number of survivors;}
	\Output{$\boldsymbol{Q}$: population for the next generation;}
	\tcc{Fitness assignment}
	$\boldsymbol{S} \gets$ the strength value for each solution in $\boldsymbol{P}$\;
	$\boldsymbol{R} \gets$ the raw fitness value for each solution in $\boldsymbol{P}$\;
	$\boldsymbol{D} \gets$ the \textbf{niching density value} for each solution in $\boldsymbol{P}$\;
	\For{$i=1,2,\ldots,|\boldsymbol{P}|$}{
		$\boldsymbol{F}(i) = \boldsymbol{R}(i) + \boldsymbol{D}(i)$; \tcp{fitness value}
	}
	\tcc{Archive truncation}
	$\boldsymbol{Q} \gets $ non-dominated solutions in $\boldsymbol{P}$\;
	\uIf{$|\boldsymbol{Q}| < N$}{
		$\boldsymbol{Q} \gets $ best $N$ solutions in $\boldsymbol{P}$ regarding their fitness values.
	}\Else{
		\While{$|\boldsymbol{Q}| > N$}{
			Repeatedly remove the solution with shortest distance to other solutions \textbf{in the same niche} from $\boldsymbol{Q}$.
		}
	}
\end{algorithm}





\section{Experiments}\label{sec:expts}

\iffalse
\begin{center}
 \begin{tabular}{|| c | c | c ||} 
 \hline
 Task & RNN(Relu) & RNN(Tanh)  \\ [0.5ex] 
 \hline\hline
  Counter-2 & 0.15 & 0.1\\
  Counter-3 & 0.02 & 0.02\\
  Counter-4 & 0.06 & 0.02 \\
  Boolean-3 & 0.99 & 0.756 \\
  Boolean-5 & 0.0 & 0.766\\
  Shuffle-2 & 0.966 & 0.60\\
  Shuffle-4 & 0.599 & 0.202\\
  Shuffle-6 & 0.348 & 0.338\\ [1ex]
 \hline
\end{tabular}
\end{center}
\fi 


%\iffalse
\begin{figure}[!ht]
\centering
\iffalse
\includegraphics[width=.45\textwidth]{../ESN_RNN_normalizedseq/RNN_2.png}
\includegraphics[width=.45\textwidth]{../ESN_RNN_normalizedseq/RNN_4.png}
\includegraphics[width=.45\textwidth]{../ESN_RNN_normalizedseq/RNN_8.png}
\fi
%\iffalse
\begin{subfigure}
  \centering
  \includegraphics[width=0.5\linewidth]{ESN_RNN_normalizedseq/RNN_2.png}
%          \vspace{-2\baselineskip}
  \caption{Data dimension: 2}
\end{subfigure}
\begin{subfigure}
  \centering
  \includegraphics[width=0.5\linewidth]{ESN_RNN_normalizedseq/RNN_4.png}
%\vspace{-2\baselineskip}
  \caption{Data dimension: 4}
\end{subfigure}%\hfill
\begin{subfigure}
  \centering
  \includegraphics[width=0.5\linewidth]{ESN_RNN_normalizedseq/RNN_8.png}
%\vspace{-2\baselineskip}
  \caption{Data dimension: 8}
\end{subfigure}%
%\fi
\caption{Invertibitiliy of RNNs at random initialization: Checking behavior of inversion error with number of neurons and the sequence length at different data dimensions.}
\label{fig:RNN_inver}
\end{figure}
%\fi 
\textbf{RNN inversion at random initialization.} We consider a randomly initialized RNN, with the entries of the weights $\mathbf{W}$ and $\mathbf{A}$ randomly picked from the distribution $\mathcal{N}(0, 1)$. Sequences are generated i.i.d. from normal distribution i.e. for each sequence, $\bx^{(i)} \sim N(0, \mathbf{I})$ for each $i \in [L]$. We use SGD with batch size 128, momentum $0.9$ and learning rate $0.1$ to compute the linear matrix $\obW^{[L]}$ so that $\norm[0]{\obW^{[L]} \mathbf{h}^{(L)} - [\bx^{(1)}, \ldots, \bx^{(L)}]}^2$ is minimized. We compute the following two quantities on the test dataset, containing $1000$ sequences: average $L_2$ error given by $\mathbb{E}_{\bx} \frac{\norm[0]{\obW^{[L]} \mathbf{h}^{(L)} - [\bx^{(1)}, \ldots, \bx^{(L)}]}}{\norm[0]{[\bx^{(1)}, \ldots, \bx^{(L)}]}}$ and average $L_\infty$ error given by $\mathbb{E}_{\bx} \norm[0]{\obW^{[L]} \mathbf{h}^{(L)} - [\bx^{(1)}, \ldots, \bx^{(L)}]}_{\infty}$. We plot both the quantities for different settings of data dimension $d$, sequence length $L$ and the number of neurons $m$. $L$ takes values from the set $\{2, 4, 6\}$, $d$ takes from $\{2, 4, 8\}$ and $m$ takes from $\{500, 1000, 2000, 5000, 10000\}$ (Figure~\ref{fig:RNN_inver}). The trends support our bounds in Theorem~\ref{thm:Invertibility_ESN_outline}, i.e. the error increases with increasing $L$ and decreases with increasing $m$. Note that the data distribution is different from the one assumed in normalized sequence Def.~\ref{def:normalized_seq}. It was easier to conduct experiments in the current data setting and a similar statement as Thm.~\ref{thm:Invertibility_ESN_outline} can be given.



\textbf{Performance of RNNs on different regular languages. } We check the performance of RNNs on the formal language recognition task for a wide variety of regular languages. We follow the set-up in \cite{BhattamishraAG20} who conducted experiments on LSTMs etc. but not on RNNs.

%\section{Regular languages} \label{sec:regular}
We consider the regular languages as considered in \cite{BhattamishraAG20}.
Tomita grammars \cite{tomita:cogsci82} contain 7
regular languages representable by DFAs of small
sizes, a popular benchmark for evaluating recurrent models (see references in \cite{BhattamishraAG20}). 
We reproduce the definitions of the Tomita grammars from there verbatim:
Tomita Grammars are 7 regular langauges defined on the alphabet $\Sigma = \{0, 1\}$.
Tomita-1 has the regular expression $1^\ast$.
Tomita-2 is defined by the regular expression $(10)^\ast$.
Tomita-3 accepts the strings where odd number
of consecutive 1s are always followed by an even
number of $0$'s. Tomita-4 accepts the strings that
do not contain three consecutive $0$'s. In Tomita-5 only
the strings containing an even number of $0$'s and
even number of $1$'s are allowed. In Tomita-6 the
difference in the number of $1$'s and $0$'s should be
divisible by 3 and finally, Tomita-7 has the regular
expression $0^\ast 1^\ast 0^\ast 1^\ast$. 

We also check the performance of RNNs on $\mathrm{Parity}$, which contains all languages with strings of the form $(w_1, \ldots, w_L)$ s.t. $w_1 + \ldots + w_L = 1 \mod 2$. Languages $\mathcal{D}_n$ are recursively defined as the set of all strings of the form $(0w1)^{\ast}$, where $w \in \mathcal{D}_{n-1}$, with $\mathcal{D}_0$ containing only $\epsilon$, the empty word. Other languages considered are $(00)^{\ast}$, $(0101)^{\ast}$ and $(00)^{\ast}(11)^{\ast}$. Table~\ref{table:regular} shows the number of examples in train and test data, the range of the length of the strings in the language, and the test accuracy of the RNNs with activation functions $\relu$ and $\tanh$ on the regular languages mentioned above. 

\begin{center}
\begin{table}[!ht]
\centering
\begin{tabular}{|| c | c | c | c | c ||} 
 \hline
 Task & No. of Training/Test examples  & Range of length of strings & RNN(Relu) & RNN(Tanh)  \\ [0.5ex] 
 \hline\hline
 Tomita 1 & 50/100 & [2, 50] & 1.0 & 1.0 \\
 Tomita 2 & 25/50 & [2, 50] & 1.0 & 1.0 \\
 Tomita 3 & 10000/2000 & [2, 50] & 1.0 & 1.0 \\
 Tomita 4 & 10000/2000 & [2, 50] & 1.0 & 1.0 \\
 Tomita 5 & 10000/2000 & [2, 50] & 1.0 & 1.0\\
 Tomita 6 & 10000/2000 & [2, 50] & 1.0 & 1.0\\
 Tomita 7 & 10000/2000 & [2, 50] & 0.259 & 0.99\\
 Parity & 10000/2000 & [2, 50] & 1.0 & 1.0\\
 $\mathcal{D}_2$ & 10000/2000 & [2, 100] & 1.0 & 1.0 \\
 $\mathcal{D}_3$ & 10000/2000 & [2, 100] & 0.99 & 1.0\\
 $\mathcal{D}_4$ & 10000/2000 & [2, 100] & 1.0 & 0.99\\
 $(00)^{\ast}$ & 250/50 & [2, 500] & 1.0 & 1.0\\
 $(0101)^{\ast}$ & 125/25 & [4, 500] & 0.99 & 1.0 \\
 $(00)^{\ast}(11)^{\star}$ & 10000/2000 & [2, 200] & 0.99 & 1.0 
 %\\Dyck-1 & 0.99 & \textbf{0.91} 
 \\[1ex]
 \hline
\end{tabular}
\caption{Performance of RNNs on different regular languages.}
\label{table:regular}
\end{table}
\end{center}

%The test languages consist of Tomita grammars which constitute a popular benchmark, and a number of other languages including $\mathsf{PARITY}$ and cover a variety of capabilities needed to recognize regular languages.
%The details of the regular languages above can be found in \cite{BhattamishraAG20}; we only note that Tomita languages constitute a popular benchmark. 
We vary $m$, the dimension of the hidden state, in the range $[3, 32]$, used RMSProp optimizer~\cite{hinton2014coursera} with the smoothing constant $\alpha = 0.99$ and varied the learning rate in the range $[10^{-2}, 10^{-3}]$. For each language
we train models corresponding to each language
for $100$ epochs and a batch size of $32$. We experimented with two different activations $\relu$ and $\tanh$. 
%The best test accuracies achieved on different languages are given in table~\ref{table:regular} and these were all achieved for $m=32$.
In all but one case (Tomita 7 with ReLU) the test accuracies with near-perfect. This was the case across runs. Tomita 7 results could perhaps be improved by more extensive hyperparameter tuning. 
We train and test on strings of length up to 50, and in a few cases strings of larger lengths (when the number of strings in the language is small). 
%Details are in Appendix~\ref{sec:regular}.




\iffalse
\begin{figure*}

\centering
\includegraphics[width=.45\textwidth]{../ESN_RNN_normalizedseq/RNN_2.png}
\includegraphics[width=.45\textwidth]{../ESN_RNN_normalizedseq/RNN_4.png}
\includegraphics[width=.5\textwidth]{../ESN_RNN_normalizedseq/RNN_8.png}
\caption{Invertibility of RNNs at initialization}
%\label{fig:figure3}
\end{figure*}
%\FloatBarrier


\begin{center}
\begin{table}[!ht]
\centering
\begin{tabular}{|| c | c | c ||} 
 \hline
 Task & RNN(Relu) & RNN(Tanh)  \\ [0.5ex] 
 \hline\hline
 Tomita 1 & 1.0 & 1.0 \\
 Tomita 2 & 1.0 & 1.0 \\
 Tomita 3 & 1.0 & 1.0 \\
 Tomita 4 & 1.0 & 1.0 \\
 Tomita 5 & 1.0 & 1.0\\
 Tomita 6 & 1.0 & 1.0\\
 Tomita 7 & 0.259 & 0.99\\
 Parity & 1.0 & 1.0\\
 $\mathcal{D}_1$ & 1.0 & 1.0 \\
 $\mathcal{D}_2$ & 0.99 & 1.0\\
 $\mathcal{D}_4$ & 1.0 & 0.99\\
 $(aa)^{\ast}$ & 1.0 & 1.0\\
 $(abab)^{\ast}$ & 0.99 & 1.0 \\
 $(aa)^{\ast}(bb)^{\star}$ & 0.99 & 1.0 
 %\\Dyck-1 & 0.99 & \textbf{0.91} 
 \\[1ex]
 \hline
\end{tabular}
\caption{Performance of RNNs on different regular languages.}
\label{table:regular}
\end{table}
\end{center}
\fi

\section{Conclusion}
We have presented a neural performance rendering system to generate high-quality geometry and photo-realistic textures of human-object interaction activities in novel views using sparse RGB cameras only. 
%
Our layer-wise scene decoupling strategy enables explicit disentanglement of human and object for robust reconstruction and photo-realistic rendering under challenging occlusion caused by interactions. 
%
Specifically, the proposed implicit human-object capture scheme with occlusion-aware human implicit regression and human-aware object tracking enables consistent 4D human-object dynamic geometry reconstruction.
%
Additionally, our layer-wise human-object rendering scheme encodes the occlusion information and human motion priors to provide high-resolution and photo-realistic texture results of interaction activities in the novel views.
%
Extensive experimental results demonstrate the effectiveness of our approach for compelling performance capture and rendering in various challenging scenarios with human-object interactions under the sparse setting.
%
We believe that it is a critical step for dynamic reconstruction under human-object interactions and neural human performance analysis, with many potential applications in VR/AR, entertainment,  human behavior analysis and immersive telepresence.





\section*{Acknowledgements}
This work was supported by National Natural Science Foundation of China (Grant No. 61876075), Guangdong Provincial Key Laboratory Grant(No. 2020B121201001), the Program for Guangdong Introducing Innovative and Enterpreneurial Teams (Grant No. 2017ZT07X386), Shenzhen Science and Technology Program (Grant No. KQTD2016112514355531), the Program for University Key Laboratory of Guangdong Province (Grant No. 2017KSYS008).
%
% ---- Bibliography ----
%
% BibTeX users should specify bibliography style 'splncs04'.
\bibliographystyle{splncs04}
\bibliography{reference.bib}
\end{document}
