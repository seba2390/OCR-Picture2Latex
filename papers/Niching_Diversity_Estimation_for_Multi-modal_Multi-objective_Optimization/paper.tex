\documentclass[runningheads]{llncs}
\usepackage{lettrine}
\usepackage{cite}
%% Figure
\usepackage{graphicx}
\graphicspath{{./figures/}}
\usepackage[labelformat=parens, labelsep=space]{subcaption}
%% Math
\usepackage{lipsum} 
\usepackage[T1]{fontenc}
\usepackage[utf8]{inputenc}
\usepackage{mathtools} 
\DeclarePairedDelimiter\ceil{\lceil}{\rceil}
\DeclarePairedDelimiter\floor{\lfloor}{\rfloor}
\newcommand{\bigdelta}{\raisebox{-.35\baselineskip}{\huge\ensuremath{\Omega}}}
%% Algorithm
\usepackage[linesnumbered,algoruled,boxed,lined]{algorithm2e}
\SetKwInOut{Input}{input}
\SetKwInOut{Output}{output}
\SetKw{Continue}{continue}
\SetKw{Break}{break}
\newcommand*\mean[1]{\overline{#1}}
\newcommand{\norm}[1]{\left\lVert#1\right\rVert}
%% Tables
\usepackage{textcomp}
\usepackage[table, xcdraw]{xcolor} % The cell in the table can be highlighted
\usepackage{diagbox}
\usepackage{multirow,booktabs,color,soul,threeparttable}
\definecolor{hl}{RGB}{255,255,0}
\sethlcolor{hl}
\DeclareCaptionFont{tablecaptionfont}{\fontsize{9}{9.6}\sc}
\captionsetup[table]{font=tablecaptionfont}
%% Hyperref
\usepackage[colorlinks,citecolor=red,urlcolor=blue,bookmarks=false,hypertexnames=true]{hyperref} 

\begin{document}
\title{Niching Diversity Estimation for Multi-modal Multi-objective Optimization}
\author{Yiming Peng, Hisao Ishibuchi\thanks{Corresponding author: Hisao Ishibuchi, hisao@sustech.edu.cn.}}
\institute{Guangdong Provincial Key Laboratory of Brain-inspired Intelligent Computation, Department of Computer Science and Engineering, \\Southern University of Science and Technology, Shenzhen 518055, China.\\
    \email{11510035@mail.sustech.edu.cn, hisao@sustech.edu.cn}}
%
\maketitle              % typeset the header of the contribution
%
\begin{abstract}
    Niching is an important and widely used technique in evolutionary multi-objective optimization. Its applications mainly focus on maintaining diversity and avoiding early convergence to local optimum. Recently, a special class of multi-objective optimization problems, namely, multi-modal multi-objective optimization problems (MMOPs), started to receive increasing attention. In MMOPs, a solution in the objective space may have multiple inverse images in the decision space, which are termed as equivalent solutions. Since equivalent solutions are overlapping (i.e., occupying the same position) in the objective space, standard diversity estimators such as crowding distance are likely to select one of them and discard the others, which may cause diversity loss in the decision space. In this study, a general niching mechanism is proposed to make standard diversity estimators more efficient when handling MMOPs. In our experiments, we integrate our proposed niching diversity estimation method into SPEA2 and NSGA-II and evaluate their performance on several MMOPs. Experimental results show that the proposed niching mechanism notably enhances the performance of SPEA2 and NSGA-II on various MMOPs.

    \keywords{Niching \and Diversity Estimation \and Multi-modal Multi-objective Optimization.}
\end{abstract}
% Sections
\section{Introduction}
\label{sec:Introduction}


The goal in top-$\size$ recommendation is to recommend to each
consumer a small set of $\size$ items from a large collection of
items~\cite{cremonesi2010performance}.  For example, Netflix may want
to recommend $\size$ appealing movies to each consumer.  Collaborative
Filtering (CF)~\cite{herlocker2002empirical,lee2012comparative} is a
common top-$\size$ recommendation method.  CF infers user interests by
analyzing partially observed user-item interaction data, such as user
ratings on movies or historical purchase
logs~\cite{kanagal2012supercharging}. The main assumption in CF is that
users with similar interaction patterns have similar interests.


Standard CF methods for top-$\size$ recommendation focus on making  suggestions  that accurately reflect the user's preference history. However, as  observed in previous work,  CF recommendations are generally biased toward  popular items, leading to a rich get richer effect~\cite{vargas2014improving,steck2011item}.  The major reasons for this are \textit{popularity bias} and \textit{sparsity} of CF interaction data (detailed in Section~\ref{sec:related-work}). In a nutshell, to maintain  accuracy, recommendations are generated from the dense regions of the data,  where the popular items lie.  

However,  accurately suggesting popular items, may not be satisfactory for the consumers. For example, in Netflix, an accuracy-focused movie recommender may recommend ``Star Wars: The Force Awakens'' to users who have seen ``Star Wars: Rogue One''.  But, those users are probably already aware of ``The Force Awakens''. Considering additional factors, such as novelty of recommendations,  can lead to more effective suggestions~\cite{cremonesi2010performance,Castells2015,zhang2008avoiding,ziegler2005improving,zhang2012auralist}. 
%Second, accuracy-focused models typically achieve a   overall item-space coverage across their recommendations,  whereas high item-space coverage helps providers of the items increase revenue
%, users satisfaction since they are  likely already aware of or can find these items on their own.  

Focusing on popular items also adversely affects the satisfaction of  the providers of the items. This is because  accuracy-focused models typically achieve a  low overall item space coverage across their recommendations, whereas   high item space coverage helps providers of the items increase their revenue~\cite{vargas2014improving,Castells2015,adomavicius2011maximizing,anderson2006thelongtail, yin2012challenging,adomavicius2012improving}.
%accuracy-focused models typically achieve a

In contrast to the relatively small number of popular items, there are copious  {\it long-tail\/} items that have fewer observations (e.g., ratings) available. More precisely,  using the Pareto  principle (i.e.,~the $80/20$ rule),  long-tail items can be defined as items that generate the lower $20\%$ of observations~\cite{yin2012challenging}. Experimentally we found that these items correspond to almost $85\%$ of the items in several datasets (Sections~\ref{sec:Notation} and \ref{sec:Experiments}). %Table~\ref{tab:DatasetStatsticsSmall})


As previously shown, one way to improve the novelty of top-$\size$ sets is to recommend interesting long-tail items~\cite{cremonesi2010performance,ge2010beyond}.  The intuition  is that since they have fewer observations available,  they are more likely to be unseen~\cite{Kaminskas:2016:DSN:3028254.2926720}.  
 %For example, in online commerce,  newly added items are long-tail items that are yet to be discovered.  
Moreover, long-tail item promotion also results in higher overall coverage of the item space%, which increases profits for providers of the items
~\cite{vargas2014improving,Castells2015,zhang2008avoiding,zhang2012auralist,adomavicius2011maximizing,anderson2006thelongtail,yin2012challenging,jambor2010optimizing}. Because long-tail promotion reduces accuracy~\cite{steck2011item}, there are trade-offs to be explored.


%original submitted to ICDE
%This work studies three aspects of top-$\size$ recommendation: accuracy, novelty, and item-space coverage, and examines their trade-offs. In most previous work, predictions of a base recommendation system are re-ranked to handle their trade-offs~\cite{adomavicius2012improving,jambor2010optimizing,zhang2013personalize,wang2009portfolio}. Due to performance considerations, however, these techniques are not customized per user. For example,  parameters that balance the trade-off between novelty and accuracy are cross-validated at a global level.  This can be detrimental since users have varying preferences for  objectives such as long-tail novelty. We explore how to  automatically infer  user  preference for long-tail novelty, and how to leverage  it to correct  the popularity bias in standard recommender models. Our work does not rely on any additional contextual data, although such data, if available, can help promote newly-added long-tail items~\cite{agarwal2009regression,Saveski:2014:ICR:2645710.2645751}.

This work studies three aspects of top-$\size$ recommendation: accuracy, novelty, and item space coverage, and examines their trade-offs. In most previous work, predictions of a base recommendation algorithm are \textit{re-ranked} to handle these trade-offs~\cite{adomavicius2012improving,jambor2010optimizing,zhang2013personalize,wang2009portfolio}. The re-ranking models are computationally efficient but suffer from two drawbacks. First, due to performance considerations,  parameters that balance the trade-off between novelty and accuracy  are not customized per user. Instead they are cross-validated at a global level.  This can be detrimental since users have varying preferences for  objectives such as long-tail novelty. Second,  the re-ranking methods are often limited to a specific base recommender  that may be sensitive to dataset density. 
As a result, the datasets are pruned and the problem is studied in dense settings~\cite{adomavicius2012improving,ho2014likes}; but real world  scenarios are often sparse~\cite{kanagal2012supercharging,liu2017experimental}.   
% Because  dataset density can impact the performance of most base recommenders (like R-SVD), which in turn affects the performance of the re-ranking model, 

\iffalse
We address these limitations by directly inferring  user  preference for long-tail novelty  from interaction data.  This  allows us to customize the re-ranking  per user, and design a \textit{generic} framework, which resolves the second problem. In particular, since the long-tail novelty preferences are estimated independently of any base  recommender model, we can  plug-in an appropriate base recommender w.r.t. the dataset sparsity.% including ones that are more suitable for sparse settings.  

Modelling  user  preference for  long-tail novelty using only item popularity statistics, e.g., the average popularity of rated items as in~\cite{jugovac2017efficient}, disregards additional information like whether the user found the item interesting and the long-tail preferences of other users  of the items. \iffalse To incorporate them, we introduce the notion of  \emph{item long-tail importance}. Both  user long-tail preferences and item long-tail importance are dependent:  a user has high preference for discovering long-tail items if she is interested in important long-tail items, and an item that is associated with many of these kinds of users is likely to be more important.  We propose a joint optimization framework to directly learn,  from interaction data, both the users' long-tail preferences and the  items' long-tail importance. \fi
We propose an optimization approach that  incorporates  this information and  directly learns,  from interaction data, the users' long-tail novelty preferences.

Next, we use these learned preferences  to design a  top-$\size$ recommendation framework thats is generic, and provides customized balance between accuracy, novelty, and coverage. We refer to it as framework as GANC.  Using GANC, we design a novel algorithm, {\it Ordered Sampling-based Locally Greedy (OSLG)\/}, that relies on the learned long-tail novelty preferences  to scalably correct for popularity bias. Our work does not rely on any additional contextual data, although such data, if available, can help promote newly-added long-tail items~\cite{agarwal2009regression,Saveski:2014:ICR:2645710.2645751}. In summary:
\fi

We address the first limitation by directly inferring  user  preference for long-tail novelty  from interaction data.   Estimating these  preferences  using only item popularity statistics, e.g., the average popularity of rated items as in~\cite{jugovac2017efficient}, disregards additional information, like whether the user found the item interesting or the long-tail preferences of other users  of the items. We propose an approach that  incorporates  this information and  learns the users' long-tail novelty preferences from interaction data.

This approach allows us to customize the re-ranking  per user, and  design a \textit{generic} re-ranking framework, which resolves the second limitation of prior work. In particular, since the long-tail novelty preferences are estimated independently of any base recommender, we can  plug-in an appropriate one w.r.t. different factors, such as the dataset sparsity.

Our top-$\size$ recommendation framework, \textbf{GANC}, is \textbf{G}eneric, and provides customized balance between \textbf{A}ccuracy, \textbf{N}ovelty, and \textbf{C}overage. % Moreover, based on the learned long-tail novelty preferences, we also design a novel algorithm, {\it Ordered Sampling-based Locally Greedy (OSLG)\/}, that relies on the learned long-tail novelty preferences  to scalably correct for popularity bias. 
Our work does not rely on any additional contextual data, although such data, if available, can help promote newly-added long-tail items~\cite{agarwal2009regression,Saveski:2014:ICR:2645710.2645751}. In summary:

%Consider  the following toy example:
\vspace{-0.2cm}
\begin{table}[htb]
\centering
\scriptsize
%\small
\begin{tabular}{ccccccc} 
%\toprule
%&\multirow{2}{*}{}&\multicolumn{7}{c}{Ratings}\\
& & \cellcolor{blue!35}$w_1$ &\cellcolor{blue!18} $w_2$ & $\dots$ &\cellcolor{blue!8} $w_{89}$  &\cellcolor{blue!8} $w_{99}$   
\\
&   &$i_1$&$i_2$&$\dots$&$i_{89}$&$i_{90}$\\ 
\cmidrule(r){3-7} 	 
%\midrule
\cellcolor{red!35}$\theta_1$  &$u_1 $   &5 &   & $\dots$ &  &   \\
\cellcolor{red!28}$\theta_2$  &$u_2$     &5 &    & $\dots$ &  &  \\
 $\theta_3=?$  &$\bf u_3$  &5 &  &   $\dots$ &  &  \\
\cellcolor{red!10}$\theta_4$ & $u_4$  &  &5   & $\dots$ & &\\ 
\cellcolor{red!10}$\theta_5$ & $u_5$  &  & 5  & $\dots$ & &\\ 
$\theta_6=?$  & $\bf u_6$ & &5  &      $\dots$& &  \\ 
 & & $\hdots$  &$\hdots$   &$\hdots$   &$\hdots$   &$\hdots$  \\
%\midrule 
\cmidrule(r){3-7} 	 
\multicolumn{2}{c}{item pop.}  & 3  & 3  & $\dots$ &50&60\\  
%\bottomrule
%$ f_i$    &3  &3  &1  &3  &1  &2  \\  \hline
\end{tabular}
%#.
\caption{Simplified user-item interaction data. The user long-tail novelty preference ($\theta_u$), item long-tail importance weight ($w_i$) are highlighted. Darker colors indicate larger values. } \label{tab:example}
\end{table} 
\vspace{-0.2cm}
\begin{example}  
In Table~\ref{tab:example}, we are interested in estimating $\theta_3$ and $\theta_6$,  the long-tail preference of users $u_3$ and $u_6$ who have each rated a single movie. Additional ratings for other users  are not included here.  Considering only rating information, we observe $i_1$ and $i_2$ are  equally popular $|\mathcal{U}_{i_1}^{\trainset}| = |\mathcal{U}_{i_2}^{\trainset}|=3$, and $r_{31}=5$ and $r_{62}=5$. Using Eq.~\ref{eq:tfidf-risk}  we have $\theta_3 = \theta_6$. However, if we were given the long-tail preferences of the each item's user set, specifically that $u_1$ and $u_2$ have high long-tail preference (darker red), while $u_4$ and $u_5$ have lower long-tail preference (lighter red), we could conclude $i_1$ is a more important long-tail item compared to $i_2$ (indicated by a darker blue shade for $w_1$), and we expect  $\theta_3 \geq \theta_6$.

% On the other hand, if we knew that $u_4$ and $u_5$ have lower long-tail preference, we could conclude $i_2$ is a  less significant long-tail item. Therefore, However, if we  consider the long-tail preferences of other users, we may reason differently.    We need another variable $w_i$ which captures this information. 
%we would conclude that $u_3$ has higher long-tail preference compared to $u_6$, since the users $i_1$ is a more prominent long-tail item. 

% Relying only  on item popularity information, we would  conclude   $u_3$ and $u_6$ have equal long-tail preference, since $i_1$ and $i_2$ are  equally popular. However, considering  the second column,  long-tail preference of users,  long-tail importance for each item,  which captures the long-tail preference of its users. Since  that  both users of $i_1$ have high long-tail preference while  the users of $i_2$ have lower preference,  we may conclude $i_1$ is a more important long-tail item compared to $i_2$. Therefore, $u_3$'s long-tail preference should be at least as large as $u_6$'s preference. Specifically, consider two  items $i_1$ and $i_2$, with the following rating data: $i_1=\{u_1:5, u_2:5, u_3:5 \}$, $i_2=\{u_4:5, u_5:5, u_6:5\}$.  

%Table~\ref{tab:example} shows  simplified rating data. We want an estimate of the long-tail preference of $u_3$ and $u_6$, who have each  rated a single movie.  Relying only  on movie popularity information, we would  conclude   $u_3$ and $u_6$ have similar long-tail preference, since $m_1$ and $m_2$ are  equally popular. However, considering the long-tail preferences of other users of those movies, we may reason differently: since $u_1$ and $u_2$ have high long-tail preference, and $u_4$ and $u_5$ have low long-tail preference, $m_1$ is a more prominent long-tail item compared to $m_2$. Therefore, it is likely that $u_3$ has higher long-tail preference compared to $u_6$.considering the long-tail preferences of other users of those movies, we may reason differently.  For example, 
\label{ex:running}
\end{example}



%------------------------------

\iffalse
\begin{example}
Table~\ref{tab:example} shows rating data for a simplified system. %Note the user-item interaction matrix is sparse.
For this example, we define popular movies as those that have received  three or more ratings; $\{m_1, m_2, m_4\}$ are popular and  $\{m_3, m_5, m_6\}$ are niche movies. We observe $u_1$ and $u_3$  have rated relatively popular movies (risk-averse) while $u_2$ and $u_4$ have rated niche movies (risk-loving). 
\label{ex:running}
\end{example}

\begin{table}[htb]
\centering
\scriptsize
\begin{tabular}{ccccccc} 
\toprule
			&$m_1$ &$m_2$   &$m_3$    &$m_4$   &$m_5$ &$m_6$  \\ \hline 
$u_1 $ &5  &4  & - &-  &-  &-   \\
$u_2$  &-  &-  &-  &-  &5  &5   \\
$u_3$  &-  &4  &-  &5  &-  &-   \\
$u_4$  &-  &-  &3  &-  &-  &4   \\ 
$u_5$  &5  &-  &-  &3  &-  &-   \\ 
$u_6$  &4  &2  &-  &4  &-  &-   \\ 
\bottomrule
%$ f_i$    &3  &3  &1  &3  &1  &2  \\  \hline
\end{tabular}
\caption{User-Movie rating data} \label{tab:example}
\end{table}

It is essential to consider consumer characteristics in designing recommender systems so that they promote long-tail items to the right group of users and spread demand evenly between hit and niche items.  

\fi





%------------------------------
\iffalse
\begin{table}[htb]
\centering
\scriptsize
\begin{tabular}{ccccccc} 
\toprule
			&$m_1$ &$m_2$   &$m_3$    &$m_4$   &$m_5$ &$m_6$  \\ \hline 
$u_1 $ &\textbf{5}  & \textbf{4}  &\textcolor{gray}{ 1.2} &-  &-  &-   \\
$u_2$  &-  &-  &-  &-  & \textbf{5}  &\textbf{5}   \\
$u_3$  &-  &\textbf{4}  &-  &\textbf{5}  &-  &-   \\
$u_4$  &-  &-  &\textbf{3}  &-  &-  &\textbf{4}   \\ 
$u_5$  &\textbf{5}  &-  &-  &\textbf{3}  &-  &-   \\ 
$u_6$  &\textbf{4}  &\textbf{2}  &-  &\textbf{4}  &-  &-   \\ 
\bottomrule
%$ f_i$    &3  &3  &1  &3  &1  &2  \\  \hline
\end{tabular}
\caption{User-Movie rating data} \label{tab:example}
\end{table}
% $\mathcal{P}^1= \{ \mathcal{P}_1^1 \{i_1,i_2,i_3\}, \mathcal{P}_2^1:\{i_2,i_3,i_5\}  \}$
 %$\mathcal{P}^2= \{ \mathcal{P}_1^2: \{i_1,i_2,i_3\}, \mathcal{P}_2^2:\{i_2,i_5,i_6\}  \}$
 %$\mathcal{P}^3= \{ \mathcal{P}_1^3: \{i_7,i_8,i_9\}, \mathcal{P}_2^3:\{i_{10},i_{11},i_{12}\}  \}$
\begin{table}[htb]
\centering
\tiny
\begin{tabular}{ccc} 
\toprule
		&$u_1$&$u_2$  \\ \hline 
$\mathcal{P}^1 $ & $\{i_1,i_2,i_3\}$ & $\{i_2,i_3,i_5\} $ \\
$\mathcal{P}^2$ & $\{i_1,i_2,i_3\}$ & $\{i_2,i_5,i_6\} $ \\
$\mathcal{P}^3$ & $\{i_7,i_8,i_9\}$ & $\{i_{10},i_{11},i_{12} \}$ \\
\bottomrule
%$ f_i$    &3  &3  &1  &3  &1  &2  \\  \hline
\end{tabular}
\caption{Top-$\size$ allocations to users.} \label{tab:paretoExamples}
\end{table}
\fi


\iffalse
When considering long-tail items, it is important to consider consumers' willingness  to explore niche or unpopular items and their propensity towards similar items. In particular, they can be characterized by their  {\it risk degree\/} and {\it focusing degree\/}, respectively.  We compute these estimates  based on historical rating information. The following example further describes these notions in the context of movie rating data. 

\begin{example}  
Table~\ref{tab:example} shows rating data for a simplified system with $6$ users, $6$ movies, and $3$ genres. $m_i^{j}$ implies that movie $m_i$ belongs to genre $j$. Note the user-item interaction matrix is sparse. 
  For this setting, we define popular movies as those that have received  three or more ratings; $\{m_1, m_2, m_4\}$ are popular and  $\{m_3, m_5, m_6\}$ are niche movies. We now profile the users according to their risk and focusing degree. E.g., $u_1$ has rated relatively popular movies belonging to the same genre (risk-averse, high focusing degree); $u_2$ has rated niches movies in the same genre (risk-loving, high focusing degree); $u_3$ has rated popular movies in two different genres (risk-averse, low focusing degree), and $u_4$ has rated niches movies in two different genres (risk-loving, low focusing degree). 
\label{ex:running}
\end{example}
\begin{table}[htb]
\centering
\tiny
\begin{tabular}{ccccccc} 
\toprule
			&$m_1^{1}$ &$m_2^{1}$   &$m_3^{2}$    &$m_4^{3}$   &$m_5^{3}$ &$m_6^{3}$  \\ \hline 
$u_1 $ &5  &4  &-  &-  &-  &-   \\
$u_2$  &-  &-  &-  &-  &5  &5   \\
$u_3$  &-  &4  &-  &5  &-  &-   \\
$u_4$  &-  &-  &3  &-  &-  &4   \\ 
$u_5$  &5  &-  &-  &3  &-  &-   \\ 
$u_6$  &4  &2  &-  &4  &-  &-   \\ 
\bottomrule
%$ f_i$    &3  &3  &1  &3  &1  &2  \\  \hline
\end{tabular}
\caption{User-Movie rating data} \label{tab:example}
\end{table}
It is essential to consider these consumer characteristics in designing recommender systems so that they promote long-tail items to the right group of users and spread demand evenly between the hit and niche items.  
\fi
\iffalse
\begin{center}
\begin{figure*}[tp]
%\scalebox{0.5}{%
\resizebox{1\textwidth}{!}{%
%\small%\addtolength{\tabcolsep}{5pt}% below sums to 8
\begin{tabularx}{1.5\textwidth}{>{\hsize=2.5\hsize}X>{\hsize=2.5\hsize}X>{\hsize=0.5\hsize}X>{\hsize=0.5\hsize}X>{\hsize=0.5\hsize}X>{\hsize=0.5\hsize}X>{\hsize=0.5\hsize}X>{\hsize=0.5\hsize}X}
    \multirow{12}{*}{\includegraphics[scale=0.3]{codeForExample/popularity-movie.png}} & \multirow{12}{*}{\includegraphics[scale=0.3]{codeForExample/scatterplot.png}} & & & & & & \\
%   & &               &       &       &       &       &       \\
    & &\multicolumn{1}{l|}{}               &$m_1^{g1}$   	&$m_2^{g1}$    	&$m_3^{g2}$    &$m_4^{g2}$      &$m_5^{g3}$    \\ \cline{3-8}%\hline
    & &\multicolumn{1}{l|}{u1}          &5  &5  &-  &-   &-  \\
    & &\multicolumn{1}{l|}{u2}    		&-  &-  &4  &4  &5  \\
    & &\multicolumn{1}{l|}{u3}   			&1  &2  &1  &-  &-   \\
    & &\multicolumn{1}{l|}{u4}     		&1  &-  &-  &-  &-  \\
    & &               &       &       &       &       &       \\
    & &               &       &       &       &       &       \\
    & &               &       &       &       &       &       \\
    & &               &       &       &       &       &	\\
    \\
\end{tabularx}}
\caption{User-Movie interaction data a) Popularity-Movie histogram b)Movie genres/clusters c) User-Movie rating data} \label{fig:example}
\end{figure*}
\end{center}
\fi



%We propose a novel approach that allows us to  promote long-tail items in a targeted manner, thereby improving the novelty of top-$\size$ sets, the overall item-space coverage across recommendations, while maintaining reasonable levels of accuracy.

%Next, we integrate these learned preferences  in a generic  top-$\size$ recommendation framework to provide customized balance between accuracy and coverage.

%sequentially make recommendations, while adjusting its parameters with regard to the set of top-$\size$ recommendations made so far. However, since  sequential parameter updates  cause  scalability issues, we propose a sampling based algorithm. This variant of our framework, called {\it Ordered Sampling-based Locally Greedy (OSLG)\/},  allows us to  correct for the popularity bias in recommendations with regard to individual user long-tail preferences. 

%ICDE submission
%Our framework differs with  prior work in the following aspects:  unlike~\cite{adomavicius2011maximizing,adomavicius2012improving,zhang2013personalize,ho2014likes},  the long-tail preference personalization in our framework is learned rather than optimized using cross-validation or parameter tuning. In other words, our personalization method is independent of the underlying base  recommendation models.  Moreover, our framework is  generic. This enables us to  plug-in several base recommenders, and evaluate their  effectiveness without requiring  extensive tuning for the accuracy and coverage trade-off. 


%\vspace{-2.8pt}
\begin{itemize}

\item  We examine various measures for estimating user long-tail novelty preference in Section~\ref{sec:lt-pref} and formulate an optimization problem  to directly learn users' preferences for long-tail  items from interaction data in Section~\ref{sec:learning-lt-pref}. %In addition, we introduce several heuristics for measuring the user preference for less common items from historical rating data.% 

\item  We integrate the user preference estimates into GANC %, a generic re-ranking framework that provides customized balance between accuracy, novelty, and coverage 
(Section~\ref{sec:RiskbasedReranking}), and  introduce {\it Ordered Sampling-based Locally Greedy (OSLG)\/}, a scalable algorithm that relies  on user long-tail preferences to correct the popularity bias (Section~\ref{sec:optimizationAlgorithm}).
%We introduce OSLG, a scalable algorithm that relies  on user long-tail preferences to  maximize item space coverage \textcolor{red}{while maintaining acceptable levels of accuracy} (Section~\ref{sec:optimizationAlgorithm}).

\item   We conduct an extensive empirical study and evaluate performance from  accuracy, novelty, and coverage perspectives (Section~\ref{sec:Experiments}).  We use five  datasets with varying density and difficulty levels. %:  Netflix, MovieTweetings, and MovieLens (100K, 1M, 10M). 
  In contrast to most related work,  our evaluation considers realistic settings that include a large number of infrequent  items and users. %This enables us to study the impact of  data density on the performance trade-offs of several  state of the art top-$\size$ recommendation algorithms. %   %,  and use the all-items ranking protocol~\cite{steck2013evaluation,vargas2014improving}, where performance is measured using all items with train data. to evaluate the performance of several  state of the art top-$\size$ recommendation algorithms 
 
\item Our empirical results confirm that the performance of re-ranking models is impacted by the underlying   base recommender and the dataset density. Our generic approach enables us to easily incorporate a suitable base recommender to devise an effective solution for both dense and sparse settings. In dense settings, we use the same base recommender as existing re-ranking approaches, and we outperform them in accuracy and coverage metrics. For sparse settings, we plug-in a more suitable base recommender, and devise an effective solution that is competitive with existing top-$\size$ recommendation methods in accuracy and novelty. 

%Directly estimating the long-tail novelty preferences allows us to customize re-ranking per user, and  devise a generic framework.   
 
\end{itemize}

Section~\ref{sec:related-work} describes related work. Section~\ref{sec:conclusion} concludes.

\section{Related Work}
\label{sec: Related Work}
In this section, we introduce two representative diversity estimators and discuss the difficulties they meet when handling MMOPs. Subsequently, some existing multi-modal multi-objective optimization algorithms are reviewed.
\subsection{Review of diversity estimators}
\subsubsection{Density in SPEA2}
In SPEA2 \cite{SPEA2}, each solution is assigned a density value which is used to calculate its fitness value. Eq. (\ref{eq: Density in SPEA2}) gives the density of a solution $\boldsymbol{x}$.
\begin{equation}
	\textit{Density} (\boldsymbol{x}) = \frac{1}{\sigma_k(\boldsymbol{x}) + 2},
	\label{eq: Density in SPEA2}
\end{equation}
where $\sigma_k(\boldsymbol{x})$ is the distance from $\boldsymbol{x}$ to its $k$-th nearest neighbor in the objective space. In SPEA2, $k$ is set to the square root of the total number of solutions in the current population as a general parameter setting.

Notice that in SPEA2, higher density means worse diversity in the objective space.

\subsubsection{Crowding distance}
Crowding distance is proposed along with the NSGA-II algorithm\cite{NSGAII} to preserve the diversity of the population in the objective space. The crowding distance of a solution $\boldsymbol{x}$ is given by the average side length of the hypercube constructed by its left and right neighbors in each objective. More precisely, for each objective, the left and right neighbors of $\boldsymbol{x}$ are the solutions at the left and right positions of $\boldsymbol{x}$ for that objective (i.e., in the list obtained by sorting the population in an increasing order of the objective values of that objective). The crowding distance of all boundary solutions (i.e., best solutions in any objectives) are set to $\infty$ to ensure that they are always selected. In NSGA-II, larger crowding distance values indicate better diversity. Formally, Eq. (\ref{eq: crowding distance}) calculates the crowding distance for a solution $\boldsymbol{x}$.
\begin{equation}
	\textit{Crowding-Distance} (\boldsymbol{x}) =
	\begin{cases}
		\infty                                                                     & ,\boldsymbol{x} \text{ is a boundary solution} \\
		\frac{1}{M}\sum_{m=1}^M[f_m(\boldsymbol{x}_{rm})-f_m(\boldsymbol{x}_{lm})] & ,\text{otherwise}
	\end{cases},
	\label{eq: crowding distance}
\end{equation}
where $M$ refers to the number of objectives, and $\boldsymbol{x}_{lm}$ and $\boldsymbol{x}_{rm}$ are the left and right neighbors of solution $\boldsymbol{x}$ regarding the $m$-th objective, respectively.
\subsection{Difficulties when handling MMOPs}
In most diversity estimators in MOEAs, the solution distribution in the decision space is out of consideration, which makes them inefficient on MMOPs. As we have discussed in Section \ref{sec: Introduction}, in MMOPs, equivalent solutions have the same or almost the same objective values. Consequently, they are usually not preferable in terms of diversity (in the objective space). For this reason, diversity estimators used in MOEAs are often responsible for the loss of equivalent solutions when tackling MMOPs. Fig. \ref{fig: Difficulty when handling MMOPs} gives an example when a diversity estimator such as crowding distance produces undesirable effects. In Fig. \ref{fig: Difficulty when handling MMOPs}, $A$ and $B$ are two Pareto optimal solutions on different (but equivalent) Pareto subsets (i.e., the upper and lower dash lines in (a)). Although $A$ and $B$ have similar objective values, the decision maker may want to keep both of them since they represent different implementations (i.e., they are different in the decision space). However, a diversity estimator tends to assign bad diversity values to them due to the small difference between their objective values. As a result, some of them are likely to be removed. From this example, we can see that solutions in different regions in the decision space should be considered separately when estimating solution diversity for MMOPs. Following this idea, we propose a niching diversity estimation method in Section \ref{sec: Proposed method}.

\begin{figure}
	\centering
	\includegraphics[width=.75\textwidth]{figures/RelatedWork/Difficulties}
	\caption{Explanation of the diversity loss in the decision space caused by diversity estimators when handling an MMOP. The dash lines in (a) and (b) denote the Pareto set and Pareto front, respectively.}
	\label{fig: Difficulty when handling MMOPs}
\end{figure}

\subsection{Multi-modal multi-objective optimization algorithms}
\label{sec: Existing multi-modal multi-objective optimization algorithms}
In most state-of-the-art multi-modal multi-objective evolutionary algorithms (MMEAs), the diversity in the decision space is maintained by niching strategies. Some MMEAs extend existing niching strategies in MOEAs to enable them to maintain the diversity in the objective space as well as in the decision space. For example, in \cite{OmniOptimizer}, Deb and Tiwari proposed one of the first MMEA called Omni-optimizer which modifies the crowding distance to measure the diversity in the decision space and the objective space simultaneously. Yue et. al. proposed a particle swarm optimizer named MO\_Ring\_PSO\_SCD \cite{MO_Ring_PSO_SCD} which adopts a similar modified crowding distance and a ring topology to create a niche structure. The DNEA algorithm \cite{DNEA} applies the fitness sharing \cite{Sharing} to both decision and objective spaces and combines them into a single sharing function. Some MMEAs are proposed with dedicated niching strategies in the decision space. Tanabe et. al. proposed a decomposition-based MMEA called MOEA/D-AD\cite{MOEAD_AD} where multiple solutions can be assigned to a weight vector, and a newly generated solution only competes with other solutions which are assigned to the same weight vector and neighboring to that solution in the decision space. In our previous study \cite{MOEAD_MM}, we proposed another decomposition-based MMEA which utilizes a clearing strategy in the decision space. Some MMEAs such as the algorithms proposed in \cite{DBSCAN_MMEA} and \cite{MMOEADC} use clustering approaches to maintain the niching structure in the decision space.
%%    TEMPLATE for articles submitted to the IWLS 2011
%%
%%
%%     Please do not remove lines commented out with %+
%%           these are for the editors' use.
%%

\documentclass[10pt]{article}
\usepackage{setspace}
\onehalfspacing
\usepackage{epsfig}
\usepackage{lscape}
\usepackage{adjustbox}
\usepackage{mathtools}
\usepackage{amsmath}
\usepackage{comment}
\usepackage{booktabs}
\usepackage{array}
\usepackage{xcolor}
\usepackage[margin=1 in]{geometry}
\usepackage[ruled,vlined]{algorithm2e}
\usepackage{amssymb}
\usepackage[%  
    colorlinks=true,
    pdfborder={0 0 0},
    linkcolor=blue
]{hyperref}


\usepackage[colorinlistoftodos]{todonotes}

\usepackage{textcomp}
\usepackage{caption}
\usepackage{multirow}
\usepackage{here}
\usepackage{amsmath}
\usepackage{dsfont}
\newcommand{\R}{\mathds{R}}
\newcommand{\E}{\mathds{E}}
\usepackage{graphicx}
\usepackage{subfig}
\allowdisplaybreaks

% For the mathematical formulation

\newcommand{\ti}{t} %For the time period index
\newcommand{\TI}{T}
\newcommand{\ka}{k} % For product index
\newcommand{\KA}{K}
\newcommand{\jey}{j} % For product index
\newcommand{\Bi}{B} %For the backlog
\newcommand{\Vi}{v} %For the order upto level
\newcommand{\Es}{S} %For the Substitution
\newcommand{\Zed}{z} %For the z variable
\newcommand{\x}{x} %For the x variable
\newcommand{\y}{y} %For the y variable

\newcommand{\InvPos}{inventory level after production }



%%%%%%%%%%%%%%%%%%%%%%%%%%%%%%%%%%%%%%%%%%%%%%%%%%%%%%%%%%%%%%%%%%%%%%%%%%%%
%%  Do not change these:
%\textwidth=6.0in  \textheight=8.25in

%%  Adjust these for your printer:
%\leftmargin=-0.3in   \topmargin=-0.20in


%%%%%%%%%%%%%%%%%%%%%%%%%%%%%%%%%%%%%%%%%%%%%%%%%%%%%%%%%%%%%%%%%%%%%%%%%%%%
%  personal abbreviations and macros
%    the following package contains macros used in this document:
%%%%%%%%%%%%%%%%%%%%%%%%%%%%%%%%%%%%%%%%%%%%%%%%%%%%%%%%%%%%%%%%%%%%%%%%%%%
%
%  To include an item in the INDEX of the conference volume,
%           flag it with    \index{<item name>}
%  The use of this macro is illustrated in the text.
%
%%%%%%%%%%%%%%%%%%%%%%%%%%%%%%%%%%%%%%%%%%%%%%%%%%%%%%%%%%%%%%%%%%%%%%%%%%%%%

\newcommand{\cred}{\color{red!65!black}}

\def\Title#1{\begin{center} {\Large {\bf #1} } \end{center}}

\begin{document}


\Title{Multi-stage stochastic lot sizing with substitution and joint $\alpha$ service level }

\bigskip\bigskip

%+\addtocontents{toc}{{\it First Author}}
%+\label{AuthorNameStart}

%%    TEMPLATE for articles submitted to the IWLS 2011
%%
%%
%%     Please do not remove lines commented out with %+
%%           these are for the editors' use.
%%












\section{Abstract}

We consider the lot sizing problem and stochastic demand with product substitution. Considering different production costs, the use of substitution can increase the revenue and customer satisfaction especially when the demand is uncertain. The goal is to minimize the total expected cost while satisfying a predetermined service level. In our model, we consider the $\alpha$ service level which limits the probability of stock outs, defined as a chance constraint. The stochasticity is represented as a scenario tree, and we investigate different solution policies for this chance-constrained multi-stage stochastic problem which are determined using mixed integer mathematical models. To solve the chance constraint models we use branch-and-cut based decomposition methods. Through extensive numerical experiments, we conclude with some managerial insights. \\

Note: This works is in progress under the supervision of professor Merve Bodur, and in collaboration with professor James Luedtke, as a visiting project at the University of Toronto supported by FRQNT funding.

\section{Introduction}


The basic lot sizing problem is a multi-period production planning problem which considers the trade-off between setup costs and inventory holding cost, and its goal is to define the optimal timing and quantity of production to minimize the total cost. The lot sizing problem has been extensively studied and applied in real-world situations. In the market, when customers cannot find a specific product, an option is to substitute it with another product which is initiated by the firm. Having a limited capacity and considering different production costs, using the substitution option can increase the revenue and customer satisfaction especially when dealing with demand uncertainty. Considering this option in production decisions may result in cost saving, and it is worthwhile to investigate. In this research, we consider the  lot sizing problem with stochastic demand and the possibility of product substitution. This problem has a practical relevance in electronics and steel industries where it is possible to substitute a lower-grade product with a higher-grade one. Semiconductors or microchips are good examples of these types of products~\cite{lang2010efficient}. 

The common approach to deal with the stochastic lot sizing problem is to minimize cost while satisfying a service level criterion. The $\alpha$ service level is an event-oriented service level which puts limits on the probability of stock outs. This service level is defined as a chance constraint, usually for each product separately. This service level can also be defined jointly over multiple products. In this research, the service level is defined based on the joint probability of the demand over all products. 

Assume that we have a stochastic inventory model with $|T|$ planning periods and $|K|$ products. Having $\overline{X}_{tk}$, as the random decision variable for production of product $\ka$ in period $\ti$  and $\overline{D}_{tk}$ as the random demand variable, the $\alpha$ service level is defined as constraint (\ref{eq:Sub_ST_Service}) and the joint $\alpha$ service level over all products is  defined as constraint (\ref{eq:Sub_ST_Service_joint}). In constraint (\ref{eq:Sub_ST_Service}), the probability ($\mathbb{P}$) of having positive inventory is greater than or equal to $\alpha$. This probability is defined jointly over all products in constraint (\ref{eq:Sub_ST_Service_joint}). \\
\begin{flalign}
&\mathbb{P}\left( \sum_{t'=1}^{t} \overline{X}_{ \ti' \ka} -\sum_{t'=1}^{t} \overline{D}_{t' k} \geq 0 \right) \geq \alpha &  {\forall \ka \in \KA}, \forall \ti  \in \TI  & \label{eq:Sub_ST_Service}\\
&\mathbb{P}\left(\sum_{t'=1}^{t} \overline{X}_{  \ti' \ka} -\sum_{t'=1}^{t} \overline{D}_{t' k} \geq 0~~~{\forall \ka \in \KA} \right)\geq \alpha &   \forall \ti  \in \TI  & \label{eq:Sub_ST_Service_joint}
\end{flalign}


%The basic assumption to approximate and solve  these models is that the demand for each product are independent from each other and there is no autocorrelation~\cite{tempelmeier2011column}. 
In this research, we model the stochastic lot sizing problem with substitution and a joint $\alpha$ service level.
%using a scenario tree.
%for which there is no need to have these assumptions and it is possible to use any demand scenarios. 
We follow the ``dynamic" strategy \cite{bookbinder1988strategies} for which the setups and production decisions are defined through the planning horizon and can be dynamically updated when the demands are realized. To solve this problem we consider a finite horizon problem and apply it in a rolling horizon environment. The stochastic demand is represented as a scenario tree for this problem.  
The challenge of these scenario-based stochastic models is that by increasing the number of scenarios the solution time will increase extensively and this makes it difficult to reach a good solution in a reasonable amount of time. In this research, we will use branch-and-cut based decomposition methods to deal with this challenge. 
The contribution of this research can be summarized as follows:

    
\begin{itemize}

\item Considering the multi-stage lot sizing problem with joint $\alpha$ service level constraints and firm-driven substitution, which is new to the literature.
  
\item Proposing a dynamic programming formulation for the finite horizon problem which can be applied in a rolling horizon fashion.

%\item Approximating the models as a two-stage stochastic programming using scenario tree in which the first stage decisions are setups and production amounts. The substitution decisions and amount of inventory and backlog are the recourse actions or second stage decisions.

\item Proposing efficient policies including deterministic and chance constraint policies to solve the resulting model and comparing these policies to solve this problem in a rolling horizon using simulation.

\item Providing managerial insight based on numerical experiments under different setting.% (as in ~\cite{jiang2017service} and  ~\cite{jiang2017production}) modified for the stochastic lot sizing problem with substitution.
    


\end{itemize}

\section{Literature review}
The related literature of this work can be categorized in two streams. The first part is dedicated to the inventory models and substitution and the second part is dedicated to the stochastic lot sizing problem and joint service level. To the best of our knowledge, no research has investigated stochastic lot sizing problem with substitution and joint service levels.
%and the current research addresses this gap in the literature.  

\subsection{Lot sizing problem and substitution}
\subsubsection{Deterministic models}
In the literature, there are two types of substitution, the demand-driven substitution and firm-driven substitution. In the demand-driven substitution the customer decides which product to substitute~\cite{zeppetella2017optimal}, while in the firm-driven case, it is the firm which makes the substitution decisions~\cite{rao2004multi}. Hsu et al.~\cite{hsu2005dynamic} studied two different versions of the dynamic uncapacitated lot sizing problem with one-way substitution, when there is a need for physical conversion before substitution, and when it does not require any conversion. The authors used dynamic programming and proposed a heuristic algorithm to solve the problem.  Lang and Domschke~\cite{lang2010efficient} considered the uncapacitated lot sizing problem with general substitution in which a specific class of demand can be satisfied by different products based on a substitution graph. They model the problem as an Mixed-integer linear program (MILP) and proposed a plant location reformulation and some valid inequalities for it.

\subsubsection{Stochastic models}
Many studies in the field of the stochastic inventory planning have considered the possibility of substitution. While the majority of them investigated the demand driven substitution, some research considered the firm driven substitutions. In this research we consider the firm driven substitutions. In the same vein, Bassok et al.~\cite{bassok1999single} investigated the single-period inventory problem with random demand and downward substitution in which a lower-grade item can be substituted with the ones with higher-grade. This model is an extension of the newsvendor problem and there is no setup cost in case of ordering. The sequence of decisions is as follows: first, the order quantity for each of the items is defined by the ordering decision, and in the next step when the demand is realized, the allocation decisions are defined. Rao et al.~\cite{rao2004multi} considered a single-period problem with stochastic demand and downward substitution, and model it as a two-stage stochastic program. In their model they consider the initial inventory and the ordering cost as well. 

Most of the research about the  product substitution with stochastic demand investigated the demand driven substitution. In these problems the customer may choose another product, if the original item cannot be found. This is known as ``stock out substitution". Akçay et al.~\cite{akccaycategory} investigated a single-period inventory planning problem with substitutable products. Considering the stochastic customer-driven and stock out based substitution, their optimization based method jointly defines the stocking of each product, while satisfying a service level. They adapt the Type II service level or ``fill rate" for each individual product and overall within a category of products.  

Another research stream considers the possibility of having multiple graded output items from a single input item, which is known as ``co-production"~\cite{ng2012robust}. In these problems, there is a hierarchy in the grade of output items and it is possible to substitute a lower grade item with the ones with higher grade~\cite{bitran1992ordering}. Hsu and Bassok~\cite{hsu1999random} considered the single-period production system with random demand  and random yields. Although they didn't mention ``co-production" in their research, their model defines the production amount of a single item and the allocation of its different output items to different demand classes~\cite{hsu1999random}.
Birtan and Dasu~\cite{bitran1992ordering} studied the multi-item, multi-period co-production problem with deterministic demand and random yield, and proposed two approximation algorithms to solve it. The first approximation is based on a rolling horizon implementation of the finite horizon stochastic model. For the second approximation, they considered the two-period, two-item problem and then applied a simple heuristic based on the optimal allocation policy for that, in a multi-period setting. Bitran and Leong~\cite{bitran1992deterministic} considered the same problem as Bitran and Dasu~\cite{bitran1992ordering} with the service level constraint. They proposed deterministic near-optimal approximations within a fixed planning horizon. To adapt their model to the revealed information, they applied the proposed model using simple heuristics in a rolling planning horizon~\cite{bitran1992deterministic}.  Bitran and Gilbert~\cite{bitran1994co} considered the co-production and random yield in a semiconductor industry and propose heuristic methods to solve it. 
 
%\subsection{Chance constraints optimization}

\subsection{Stochastic lot sizing problem and $\alpha$ service level constraints}
In the basic lot sizing problem, all the parameters are deterministic. Stochastic lot sizing problems address this restrictive assumption by considering uncertainty in different parameters such as demand and cost parameters. In lot sizing problems, we are dealing with the sequence of decisions over the planning horizon, and multi-stage stochastic programming is a method to incorporate uncertainty. In multi-stage stochastic problems, the uncertainty is typically represented as a scenario tree. 

 Haugen et al.~\cite{haugen2001progressive} considered the multi-stage uncapacitated lot sizing problem and proposed a progressive hedging algorithm to solve it. Guan and Miller~\cite{guan2008polynomial} proposed a dynamic programming algorithm for a similar version. Using the same algorithm, Guan~\cite{guan2011stochastic} studied the capacitated version of the problem with the possibility of backlogging. Lulli and Sen~\cite{lulli2004branch} proposed a branch and price algorithm for multi-stage stochastic integer programming and applied the method to the stochastic batch-sizing problem. In this problem, they consider that the demand, production, inventory and set up costs are uncertain. The difference between this problem and the lot sizing problem is that the production quantities are in batches and the production decision variables are discrete values which define the number of batches that will be produced. This problem is a more general case of lot sizing problem. In another research, Lulli and Sen~\cite{lulli2004branch}  proposed a scenario updating method for the stochastic batch-sizing problem. As backlogging is allowed in this problem, it is considered as a stochastic model with complete fixed recourse and all the scenarios are feasible.  

Another common approach to deal with stochastic demand is using service levels. In this context the planners put  a demand fulfillment criterion to mitigate the risk of demand uncertainty. Stochastic lot sizing problems with service level constraints have been studied extensively~\cite{tempelmeier2007stochastic}. One of the main service levels is the $\alpha$ service level which is an event-oriented service level, and imposes limits on the probability of a stock out. The $\alpha$ service level is presented as a chance constraint and is usually defined for each period and product separately. Bookbinder and Tan~\cite{bookbinder1988strategies} investigate stochastic lot sizing problems with an $\alpha$ service level and propose three different strategies based on the timing of the setup and production decisions, for this problems. These strategies are the \textit{static}, \textit{dynamic}, and \textit{static-dynamic} strategy. In the \textit{static} strategy, both the setup and production decisions are determined at the beginning of the planning horizon and they remain fixed when the demand is realized. In the \textit{dynamic} strategy, both the setup and production decisions are dynamically changed with the demand realization throughout the planning horizon. The \textit{static-dynamic} strategy is between these two strategies in which the setups are fixed at the beginning of the planning horizon and the production decisions are updated when the demands are realized. The $dynamic$ strategy can be modeled as a multi-stage stochastic lot sizing problem.  

There are some studies which define the $\alpha$ service level jointly over different planning periods. Liu and K{\"u}{\c{c}}{\"u}kyavuz~\cite{liu2018polyhedral} considered the uncapacitated lot sizing problem with a joint service level constraint and studied the polyhedral structure of the problem and proposed different valid inequalities and a reformulation for this problem. Jiang et al.~\cite{jiang2017production} considered the same problem with and without pricing decisions. Gicquel and Cheng~\cite{gicquel2018joint} investigated the capacitated version of the same problem, and following the same methodology as Jiang et al.~\cite{jiang2017production} they used a sample approximation method to solve it. This method is a variation of the sample average approximation method which is proposed by Luedtke and Ahmed~\cite{luedtke2008sample} to solve models with chance constraints using scenario sets. All the mentioned research consider single item models in which the joint service level is defined over all periods. In this research we consider substitution and the joint service level is defined over all products.


\section{Problem definition}
We consider a lot sizing problem with substitution in an infinite time horizon, in which we need to make decisions about the timing, production and substitution amounts, and accordingly define the potential inventories and backlogs. Figure~\ref{MultistageDynamics} illustrates the dynamics of decisions for this problem. There are $|\KA|$ different products with random demand and the notations are shorthand for the vectors which corresponding elements are defined for each product.
%and summarized in Table~\ref{tab:Sub_parameters}.
At each point of time, $t$, we know the initial state of the system, defined as the current on hand inventory, $\hat{\Vi}_{\ti}$, and the backlog, $\hat{\Bi}_{\ti}$, for different products. When the demand of period $\ti$ is realized, based on $\hat{\Vi}_{\ti}$ and $\hat{\Bi}_{\ti}$, two sets of decisions are made. The first set includes substitution, inventory, and backlog denoted by $\Es_{\ti}, I_{\ti}, \Bi_{\ti}$, respectively. 
%These variables are defined based on the \InvPos \todo{Note that ‘inventory position’ has a very specific meaning in the inventory literature: it is the sum of the on-hand and on-order inventory minus backorders and minus committed’. Is this specifically what you meant? }, $\hat{\Vi}$, and the amount of backlog $\hat{\Bi}$ coming from the previous period.
%These two parameters show the state of the system at beginning of the current period. 
The rest of the decisions in the current period are production, setup, and \InvPos at the end of current period which are denoted by $x_{\ti}, y_{\ti}, v_{\ti}$, respectively.  It should be noted that all these decisions are made simultaneously and the only difference is that the last three decisions are used to satisfy the future periods demand. In other words, the production in the current period can be used is the next periods and it is not available to satisfy the same period cumulative demand.
%$\Es_{\ti}, I_{\ti}, \Bi_{\ti}, x_{\ti}, y_{\ti}, v_{\ti} $ are shorthand for the vectors which corresponding elements 

The on hand inventory of a product can be used to satisfy its own demand or another product demand based on the substitution graph $E$. If $(k, j) \in E$ then product $\ka$  can fulfill demand of product $\jey$ but at a substitution cost of $stc_{tkj}$ per unit. All demand is either met (from inventory or substitution) or else is backlogged. In each period $\ti$, while insufficient inventory will lead to backlog denoted by $\Bi_{\ti \ka}$, unnecessary stocks will increase the holding cost. An inventory holding cost of $hc_{\ti \ka}$ per unit, is charged for the quantity being stored at the end of each period, denoted by $I_{\ti \ka}$. Furthermore, in each period in which production occurs, a setup has to be performed which incurs a fixed setup cost of $sc_{\ti \ka}$. We consider the trade-off between these costs and then define different decisions such that the random demand in the next period can be satisfied with high probability defined as the service level. 
%It should be noted that the production in the current period, is not available in the same period.
All different decisions are defined and updated sequentially  based on the state of the system and the realized demand which is in line with the  ``dynamic'' strategy that is defined for the stochastic lot sizing problem~\cite{bookbinder1988strategies}. %There are different operational cost  including setup cost, substitution cost, and inventory holding cost, which should be minimized.
  %in which different decisions are defined sequentially based on the state of the system and the realized demand. This is in line with the  ``dynamic'' strategy that is defined for the stochastic lot sizing problem in which both setup and production decisions are updated when the demands are realized ~\cite{bookbinder1988strategies}.


%\todo{We will need to discuss these dynamics, and they have to be discussed more clearly in the paper. Important.What is the difference between I and v?? in my understanding:
%$I_t$: the inventory after the demand realization and substitution decisions (but before the production decision in t is taken into account)
%$v_t$: inventory at the end of the period, so it would be $= I_t + x_t$}

%\todo{is it correct to say that demand in a period then happens before production. If so, state this explicitly. }.
%If we apply this model in a rolling horizon framework, after solving the model and fixing the decisions variables in the first period based on the realized demand, the values of $\Vi$, and $\Bi$ will become the initial state of the model in the next period. \todo{Distinguish between the problem definition and solution approach} 
 


%\begin{figure}[!h]
%\begin{center}
%\includegraphics[scale=0.5]{Diagram.png}
%\caption{Dynamics of the multi period model} 
%\label{Diagram}
%\end{center}
%\end{figure}

\begin{figure}[!h]
\begin{center}
\includegraphics[scale=0.6]{Diagram.png}
\caption{Dynamics of decisions at each stage} 
\label{MultistageDynamics}
\end{center}
\end{figure}



%which is adapted from  Lang and Domschke~\cite{lang2010efficient} for deterministic uncapacitated lot sizing problem with substitution. 
%This model considers a general substitution graph in which a demand class can be fulfilled with multiple products based on the substitution graph and at a substitution cost.


\begin{table}[H]
\centering
\caption{Parameters and decision variables for stochastic lot sizing model with substitution}
\begin{adjustbox}{width=1\textwidth,center=\textwidth}
\begin{tabular}{ll}
\toprule
{\bf Sets} & {\bf Definition} \\ \midrule
$\TI$  & Set of planning periods \\ 
$\KA$  & Set of products \\
%$N$  & Set of demand classes\\ 
%$V = K \cup N$  & Vertex set of substitution graph\\
$E \subseteq \KA \times \KA$  & Directed edges of substitution graph denoting feasible substitutions
%: $(k, j) \in E$ if product $\ka$  can fulfill demand of product $\jey$
\\
%$ G = (V,E)$  & Substitution graph \\
$ C_{k} = \{j \mid (k,j) \in E\}^*$  & Set of products whose demand can be fulfilled by product $\ka$  \\
$ P_{j} = \{k \mid (k,j) \in E\}^*$  & Set of products that can fulfill the demand of product $j$ \\
$ M(n) $  & Set of child nodes for node $n$. \\

{\bf Parameters} & {\bf Definition} \\ \midrule
$sc_{\ti \ka}$ & Setup cost for product {\it k} in period {\it t} \\ 
$hc_{\ti \ka}$  & Inventory holding cost for product {\it k} in period {\it t}  \\ 
$stc_{\ti \ka \jey }$  & Substitution cost if product $\ka$  is used to fulfill the demand of product $j$ in period {\it t}  \\ 
$pc_{\ti \ka}$  & Production cost for product {\it k} in period {\it t}  \\
$\alpha$  & Minimum required joint service level \\ 
$bc_{\ti \ka}$  & Backlog cost for product $\ka$  in period $\ti$ \\
$c_{\ti \ka}$  & A sufficiently large number \\ 
%$I_{k0}$ & The initial inventory for product $\ka$  \\ 
%${d}_{jt}$  & Demand for class {\it j} in period {\it t} (model input)\\ 
${D}_{\ti \ka}$ & Random demand variable for product {\ka } in period {\it t}  \\ 
$\hat{\Vi}_{p(n), \ka} $&  The amount of \InvPos for product $\ka$ at parent of node  $n$\\
$\hat{\Bi}_{p(n), \ka} $&  The amount of backlog for product $\ka$ at the parent of node $n$  \\
$q_{nm} $&  The probability of child node $m$ when we are at node $n$  \\
{\bf Decision variables} & {\bf Definition} \\ \midrule
$\y_{n \ka}$ & Binary variable which is equal to 1 if there is a setup for product {\it k} at node {\it n}, 0 otherwise \\ 
$\x_{n \ka}$ & Amount of production for product $\ka$  at node {\it n}  \\ 
$\Es_{n \ka \jey}$ & Amount of product $\ka$  used to fulfill the demand of product $\jey$  at  node {\it n}   \\
${I}_{n \ka}$ & Amount of physical inventory for product {\it k} at the end of period for node {\it n}  \\
${\Bi}_{n \ka}$ & Amount of Backlog for product {\it k} at the end of period for node {\it n}  \\
${\Vi}_{n \ka}$ & The \InvPos for product {\it \ka} at the end of period for node {\it n}  \\
 \bottomrule
 & $^*{ (C_k \text{ and } P_k  \text{ sets include } k) }$
\end{tabular}
\end{adjustbox}
 \label{tab:Sub_parameters}
\end{table}



 
 





%\begin{flalign}
%&pr((\sum_{t'=1}^{t} \sum_{k\in P_{j}} \overline{S}_{\ka \jey \ti'} -\sum_{t'=1}^{t} \overline{D}_{jt'}) \geq 0~~~~~\forall j \in N ) \geq \alpha_{c} &   \forall \ti  \in \TI,  & \label{eq:Sub_ST_Service}
%\end{flalign}


%Constraints (\ref{eq:Sub_ST_Service}) guarantee the minimum service level.
%These chance constraints ensure that the probability of a stock out is less than (1-$\alpha_{c}$). \\

The approach to deal with this infinite horizon problem is to consider a finite horizon problem, and then apply it is in a rolling horizon environment. This finite horizon problem, is a multi-stage stochastic lot sizing problem with the possibility of substitution, with $|\TI|$ planning periods, $|\KA|$ different products, and random demand. The future demand for the $|\TI|$ stages is represented as a scenario tree. In this  model, we need to decide about the setup timing, and the production and substitution amounts for each product through the planning periods. Being at a specific period, we should define the decisions such that the joint service level in the next period is satisfied.

\subsection{Dynamic programming model}


In this section, we present the dynamic programming formulation for the finite horizon stochastic lot sizing problem with a firm-driven substitution and joint service level constraint over all products. Different sets, parameters and decision variables are presented in Table~\ref{tab:Sub_parameters}. Considering the underlying scenario tree, the mathematical model (\ref{DynamicProgramming}) is proposed for the node {\it n} in the tree. $\ti_{n}$ is the time period corresponding to node $n$.  Each node in the tree has a parent node, $p(n)$, and set of child nodes, $M(n)$. 
%This model is based on the \InvPos and backlog at the beginning and end of the planning period corresponding to the current node {\it n}.
$\hat{\Vi}_{p(n)}$ and $\hat{\Bi}_{p(n)}$ indicate the vector for the state of the system at the current node for all products. Considering $D_{\ti \ka}$ as the random demand variable for product $\ka$ in period $\ti$, $\hat{D}_{n \ka}$ is its realization at node $n$. Considering the scenario tree (Figure~\ref{fig:Tree}), $D^{\ti \ka}$ is the of  random demand path from period 1 to period $\ti$ and $\hat{D}^{nk}$ is its realization until node $n$. We formulate the multi-stage stochastic lot sizing problem with substitution as the following dynamic program: 
\begin{figure}[!h]
\begin{center}
\includegraphics[scale=0.4]{TreeProp.png}
\caption{Underlying scenario tree} 
\label{fig:Tree}
\end{center}
\end{figure}

%Here, $\Vi_{n{\ka}} =$ current \InvPos of product $\ka$. \\
%Note: $I_{n{\ka}}$: the ending inventory of node $n$ for product $\ka$\\
\begin{subequations}
\label{DynamicProgramming} %\todo{Does it satisfy the optimality?}
\begin{flalign}
&F_{n}(\hat{\Vi}_{p(n)},\hat{\Bi}_{p(n)}) =  \min  \sum_{\ka \in \KA} ( sc_{\ti_n \ka}\y_{n \ka} + pc_{\ti_n \ka}\x_{n \ka} + hc_{\ti_n \ka} {I}_{n \ka}+  \sum_{j\in C_{\ka }}stc_{\ti_n \ka \jey} \Es_{n  \ka \jey})  + \sum_{m \in M(n)}  \underbrace{ q_{nm}F_{m}(\Vi_{n}, {\Bi_{n}}  ) }_{\text{future}}
%+ bc_{\ti_n \ka} {\Bi}_{n \ka} )
&  \label{eq:Dyn_g2_Sub_ST_Production_Flow}
%& + \sum_{m \in M(n)}  \underbrace{ q_{nm}F_{m}(\Vi_{n}, {\Bi_{n}}  ) }_{\text{future}} & \label{eq:Dyn_g2_Sub_ST_Production_Flow} 
\end{flalign} 
%\todo{ did you consider the unit production cost?}

\begin{flalign}
\text{s.t.} \ & \x_{\ti_n \ka} \leq c_{n \ka} \y_{n \ka} \quad  &\forall \ka  \in \KA  & \label{eq:Dyn_Sub_Setup}\\
%& \Zed_{m} = 0 \ \Rightarrow \ m \ \text{subproblem is feasible with } \Vi_{\cdot n} & \label{eq:Dyn_Sub_ST_Service1}\\
%& \sum_{m \in M(n)} q_{nm} \Zed_{m} \leq 1-\alpha, & &     \label{eq:Dyn_Sub_ST_Service}\\*[0.5cm]
%& v^+_{n \ka} - v^-_{n \ka} = I_{n  \ka} + \x_{n  \ka} - \Bi_{n  \ka}  \quad \forall \ka  \in \KA \\
%& \sum_{j\in C_{k}} {S}_{n \ka \jey} \leq v^+_{p(n), \ka} \quad \forall \ka  \in \KA \\
%&  I_{n \ka} - \Bi_{n \ka} = v^+_{p(n), \ka} - v^-_{p(n), \ka} - \sum_{j\in C_{k}} {S}_{n \ka \jey} + \sum_{j\in P_{k}} {S}_{n \jey \ka}  - D_{n \ka} \quad \forall \ka  \in \KA  &     \label{eq:Dyn_F_Sub_ST_Service}&\\
%& I_{n \ka} \leq v^+_{p(n), \ka} - \sum_{j\in C_{k}} {S}_{n \ka \jey} \quad \forall \ka  \in \KA \\*[0.5cm]
%%%% Like Rao's %%%%%
& \sum_{j\in P_{k}} {S}_{n \jey \ka} + \Bi_{n  \ka}  = \hat{D}_{n \ka} + \hat{\Bi}_{p(n), \ka} \quad &\forall \ka  \in \KA \label{eq:Bhat} \\
& \sum_{j\in C_{k}} {S}_{n \ka \jey} + I_{n \ka} = \hat{\Vi}_{p(n), \ka}  \quad &\forall \ka  \in \KA \label{eq:vhat}\\
& \Vi_{n \ka} = I_{n  \ka} + \x_{n  \ka}  \quad &\forall \ka  \in \KA  \label{eq:vdef}
%& v^-_{n \ka} = \Bi_{n  \ka}  \quad \forall \ka  \in \KA \\
\\
& \mathbb{P}_{D^{t_n+1}}\{ ({\Vi}_{n}, {\Bi}_{n} ) \in Q(D^{\ti_n+1} )| D^{\ti_n} = \hat{D}^{n} \} \geq \alpha& \label{eq:SL}
& \\*[0.5cm]
%%%%%%%%%%%%%%%%%%%%%%%
& {x}_{ n },  {v}_{ n },  {I}_{ n } , {\Bi}_{ n } \in \mathbb{R}_{+}^{|\KA|} , {S}_{n} \in \mathbb{R}_{+}^{|E|} ,{y}_{ n } \in \{0,1\}^{|\KA|} &  & \label{eq:Dyn_F_Sub_ST_bound1}
%\\
%&{S}_{n \ka \jey} \geq 0 & \forall(k,j) \in E  & \label{eq:Dyn_F_Sub_ST_bound3}&
%\\
%& (I_{\cdot n},\Bi_{\cdot n},\Es_{\cdot n}) \in %\mathcal{X}^{\text{rest}}_\ell &&\label{eq:Dyn_F_Sub_ST_bound4}&
\end{flalign}
\end{subequations}
%\todo{Are these vector $v_n, B_n$? they are only defined with index}

The objective function of the model at node $n$ shown as (\ref{eq:Dyn_g2_Sub_ST_Production_Flow}), in which $F_{n}(\hat{\Vi}_{p(n)},\hat{\Bi}_{p(n)}) $ represents the expected optimal cost of decisions from node $n$ in the scenario tree to the end of the horizon given the initial inventory level vector and backlog vector. The objective function is to minimize the current stage total cost plus the expected cost-to-go function. This cost includes the total setup cost, production cost, holding cost, and substitution cost.  $\Vi_n$ and $\Bi_n$ are shorthand for the vectors $\{ v_{nk}, k \in K\}$ and $\{ B_{nk}, k \in K\}$, respectively.
Constraints (\ref{eq:Dyn_Sub_Setup}) are the set up constraints which guarantee that when there is production, the setup variable is forced to take the value 1, and 0 otherwise. 

Constraints (\ref{eq:Bhat}) show that the demand of each product is satisfied by its own production and the substitution by other products or it will be backlogged to the next stage. In this constraint, $\Es_{\ka \ka n}$ is equal to the amount which is allocated to satisfy the demand of product $k$ from its own production. Constraints (\ref{eq:vhat}) show that the inventory of product $\ka$ at the beginning of the current period may be used to satisfy its own demand or other products demand through substitution or it will be stored. Constraints (\ref{eq:vdef}) define the \InvPos at the end of the current period which is equal to the amount of inventory (immediately after demand satisfaction) plus the amount of production for the future periods. 

Constraint~(\ref{eq:SL}) is the joint service level for period $\ti_n+1$. In this constraint $Q(D^\ti)$ is the set of \InvPos and backlog quantities in which customer demands given by $D^\ti$ can all be met and there is no stock out for any of the products. This constraint guarantees that the probability of having a feasible solution with respect to the service level is greater than or equal to $\alpha$. This probability is defined over the demand distribution until period $\ti_n+1$, having that part of the demand history until node $n$ is realized and known. Constraints~(\ref{eq:Dyn_F_Sub_ST_bound1}) define the domains of different variables in the model. 
%In addition to these constraints it is possible to add different types of constraint such as capacity constraints to the model.

%For example, if the company decide to satisfy a percentage of each product expected demand  by its own production, the following constraint can be added to the model.

 Note: In the stochastic case with service level, when we want to implement it in a rolling horizon environment, the feasibility of the next stage service level should be guaranteed.  In this model, it is not explicit that the recourse stage should be feasible. This can be satisfied for instance if we have at least one uncapacitated product option (whether by its own production or substitution) for each product. 
 %\todo{Is it only for the capacitated case that feasibility of the next stage is not guaranteed? It seems also to be the case for the uncapacitated case, since you have to decide on the production in period t before knowing the demand in period $t+1$.Do you need to explicitly model this recourse option in your model?}


%{\cred * Given stage t and the history, $\xi^t, \exists x^t(\xi^t) \in \chi^t(x^t) \rightarrow v(\xi^t)$\\
%$s.t.$ \\
%$P_{\xi^{t+1}|\xi^t}\{v^{t}(\xi^t)-\sum_{j\in c_k}\Es_{kj}(\xi^{t+1})+\sum_{j\in p_k}\Es_{jk}(\xi^{t+1}) \geq D^{t+1}(\xi^{t+1}) $ for some $\Es_{kj}(\xi^{t+1}) , \Es_{jk}(\xi^{t+1}) \in  \chi ^{t+1}(\xi^{t+1}) \} \geq \alpha$ \\




\section{Future work}

The future work of this research is listed as follows.

\subsection{Decision policies}
We will explain how we are going to define different decisions at each stage using different policies. These policies include the deterministic policies, and the chance constraint policy which considers the service level in the next stage. These policies are modeled as mixed integer programming models.

    \begin{itemize}
        \item Deterministic policies: In this set of policies the random demand is substituted by a deterministic value such as average or quantile of the demands. In the average policy all the random demands are substituted by their average, for each product and each period. In the quantile policy, being at a specific stage, the random demand for the next immediate stage is substituted by the $\alpha\%$ quantile value based on the child nodes demand, and for the rest of the periods by the average demand. In this set of policies, we do not impose any service level in the model, but the resulting service level will be used for their evaluation.
        \item Chance constraint policy: In this policy we approximate the demand scenario tree. For period $\ti _n +1$ we will consider the set of child nodes, but for the periods greater than $\ti _n +1$ we will use the average demand (Figure~\ref{fig:Appr}). 
        Considering a set of child nodes for node $n$, we will approximate the objective function of the model (\ref{DynamicProgramming}). This will be done by averaging over all the child nodes for period $\ti _n +1$. For the chance constraint in period $\ti _n +1$, we will add a binary variable to the model, which is defined for each child node. This binary variable is equal to 0, when we have no backlog for any of the products.   
        \begin{figure}[!h]
\begin{center}
\includegraphics[scale=0.4]{Approximationprop.png}
\caption{Approximating the scenario tree} 
\label{fig:Appr}
\end{center}
\end{figure}
Considering these approximations, all these policies are modeled as mixed integer programming models. For the proposed chance constrained policy in addition to the extensive form model, we will use the branch-and-cut method proposed by Luedtke~\cite{luedtke2014branch} with some justification for this problem.
 \end{itemize}
  \subsection{Policy evaluation and numerical experiments}
All the models with proposed policies will be implemented and evaluated in a rolling horizon. The stochastic demand is generated based on an autoregressive process. The average total cost and service level and their confidence intervals will be used to compare different policies. We will conduct numerical experiments to investigate the efficiency of the proposed solutions methodologies, and derive managerial insights in terms of policy comparison, and the effect of substitution under different parameter settings such as time between orders, holding cost, substitution cost, and service level.

    



\begin{thebibliography}{99}



%%

%\bibitem{RefJ}
% Format for Journal Reference
%Author, Article title, Journal, Volume, page numbers (year)
% Format for books
%\bibitem{RefB}
%Author, Book title, page numbers. Publisher, place (year)

%a,b

\bibitem{akccaycategory}
Akçay Yalçın. and Yunke Li and Harihara Prasad Natarajan, Category Inventory Planning With
Service Level Requirements and Dynamic Substitutions.
Production and Operations Management (2020), https://doi.org/doi:10.1111/poms.13240

\bibitem{bassok1999single}
Bassok, Yehuda and Anupindi, Ravi and Akella, Ram,
Single-period multiproduct inventory models with substitution, Operations Research, 47, 4, 632--642, (1999)
 

\bibitem{bookbinder1988strategies}
 Bookbinder, James H and Tan, Jin-Yan, Strategies for the probabilistic lot-sizing problem with service-level constraints, Management Science, 34, 9, 1096--1108,
 (1988)


\bibitem{bitran1992deterministic}
Bitran, Gabriel R and Leong, Thin-Yin, Deterministic approximations to co-production problems with service constraints and random yields, Management science, 38, 5, 724--742, (1992)
  
  
  \bibitem{bitran1992ordering}
Bitran, Gabriel R and Dasu, Sriram, Ordering policies in an environment of stochastic yields and substitutable demands,
Operations Research,
40, 5,
999--1017, (1992)

\bibitem{bitran1994co}
Bitran, Gabriel R and Gilbert, Stephen M,
  Co-production processes with random yields in the semiconductor industry, Operations Research, 42, 3, 476--491, (1994)
%c,d
\bibitem{chen2020dynamic}
Chen, Boxiao and Chao, Xiuli, Dynamic inventory control with stockout substitution and demand learning, Management Science, (2020) , https://doi.org/10.1287/mnsc.2019.3474
%e,f

%g,h

\bibitem{guan2011stochastic} Guan, Yongpei, Stochastic lot-sizing with backlogging: computational complexity analysis,  Journal of Global Optimization, 49, 4, 651--678, (2011)

\bibitem{guan2008polynomial}
  Guan, Yongpei and Miller, Andrew J, Polynomial-time algorithms for stochastic uncapacitated lot-sizing problems,  Operations Research, 56, 5, 1172--1183,
(2008)


\bibitem{gicquel2018joint}
Gicquel, C{\'e}line and Cheng, Jianqiang, A joint chance-constrained programming approach for the single-item capacitated lot-sizing problem with stochastic demand, Annals of Operations Research, 264, 1-2, 123--155, (2018)
\bibitem{helber2013dynamic}
 Helber, Stefan and Sahling, Florian and Schimmelpfeng, Katja, Dynamic capacitated lot sizing with random demand and dynamic safety stocks, OR Spectrum, 35, 1, 75--105, (2013)
 

\bibitem{hsu1999random}
Hsu, Arthur and Bassok, Yehuda, Random yield and random demand in a production system with downward substitution, Operations Research, 47, 2, 277--290,
  (1999)
  
\bibitem{hsu2005dynamic}
Hsu, Vernon Ning and Li, Chung-Lun and Xiao, Wen-Qiang, Dynamic lot size problems with one-way product substitution, IIE transactions, 37, 3, 201--215, (2005)

\bibitem{haugen2001progressive}Haugen, Kjetil K and L{\o}kketangen, Arne and Woodruff, David L, Progressive hedging as a meta-heuristic applied to stochastic lot-sizing, European Journal of Operational Research, 132, 1, 116--122, (2001)

%i,j

\bibitem{jiang2017service}
 Jiang, Yuchen and Shi, Cong and Shen, Siqian, Service Level Constrained Inventory Systems, Production and Operations Management, Wiley Online Library 28, 9, 2365–-2389,
 (2017)
 
 \bibitem{jiang2017production}
 Jiang, Yuchen and Xu, Juan and Shen, Siqian and Shi, Cong, Production planning problems with joint service-level guarantee: a computational study, International Journal of Production Research, Taylor \& Francis, 55, 1, 38--58,
 (2017)
 %l
 
\bibitem{lang2010efficient}
 Lang, Jan Christian and Domschke, Wolfgang, Efficient reformulations for dynamic lot-sizing problems with product substitution, OR spectrum,
32, 2, 263--291, (2010)

\bibitem{lulli2006heuristic} Lulli, Guglielmo and Sen, Suvrajeet, A heuristic procedure for stochastic integer programs with complete recourse, European Journal of Operational Research,
171, 3, 879--890, (2006)

\bibitem{lulli2004branch}Lulli, Guglielmo and Sen, Suvrajeet, A branch-and-price algorithm for multistage stochastic integer programming with application to stochastic batch-sizing problems, Management Science, 50, 6, 786--796, (2004)
\bibitem{liu2018polyhedral}
Liu, Xiao and K{\"u}{\c{c}}{\"u}kyavuz, Simge, A polyhedral study of the static probabilistic lot-sizing problem, Annals of Operations Research, 261, 1-2,
233--254, (2018)

\bibitem{liu2016decomposition}
Liu, Xiao and K{\"u}{\c{c}}{\"u}kyavuz, Simge and Luedtke, James, 
Decomposition algorithms for two-stage chance-constrained programs, Mathematical Programming, 157,
  1, 219--243, (2016)
 

\bibitem{luedtke2008sample}
Luedtke, James and Ahmed, Shabbir, A sample approximation approach for optimization with probabilistic constraints, 
SIAM Journal on Optimization, 19,
2, 674--699, (2008)


\bibitem{luedtke2014branch}
  Luedtke, James, A branch-and-cut decomposition algorithm for solving chance-constrained mathematical programs with finite support,
  Mathematical Programming, 146, 1, 219--244,
 (2014)
%m, n 

\bibitem{ng2012robust}
Ng, Tsan Sheng and Fowler, John and Mok, Ivy, Robust demand service achievement for the co-production newsvendor, IIE Transactions,
  44,
  5,
 327--341,
  (2012)
  
%o, p , q
%R
 \bibitem{rao2004multi}
 Rao, Uday S and Swaminathan, Jayashankar M and Zhang, Jun,
 Multi-product inventory planning with downward substitution, stochastic demand and setup costs,
 IIE Transactions, 36, 1, 59--71, (2004)
 %t 
 
 \bibitem{tempelmeier2007stochastic}
Tempelmeier, Horst, On the stochastic uncapacitated dynamic single-item lotsizing problem with service level constraints, European Journal of Operational Research, 181, 1, 184--194, (2007)
 

 \bibitem{tempelmeier2011column}
Tempelmeier, Horst, A column generation heuristic for dynamic capacitated lot sizing with random demand under a fill rate constraint, Omega, 39, 6, 627--633, (2011)
 % v
 

\bibitem{zeppetella2017optimal}
Zeppetella, Luca and Gebennini, Elisa and Grassi, Andrea and Rimini, Bianca, Optimal production scheduling with customer-driven demand substitution, International Journal of Production Research, 55, 6, 1692--1706, (2017)


 



\end{thebibliography}




\end{document}


\section{Experiments}\label{sec:expts}

\iffalse
\begin{center}
 \begin{tabular}{|| c | c | c ||} 
 \hline
 Task & RNN(Relu) & RNN(Tanh)  \\ [0.5ex] 
 \hline\hline
  Counter-2 & 0.15 & 0.1\\
  Counter-3 & 0.02 & 0.02\\
  Counter-4 & 0.06 & 0.02 \\
  Boolean-3 & 0.99 & 0.756 \\
  Boolean-5 & 0.0 & 0.766\\
  Shuffle-2 & 0.966 & 0.60\\
  Shuffle-4 & 0.599 & 0.202\\
  Shuffle-6 & 0.348 & 0.338\\ [1ex]
 \hline
\end{tabular}
\end{center}
\fi 


%\iffalse
\begin{figure}[!ht]
\centering
\iffalse
\includegraphics[width=.45\textwidth]{../ESN_RNN_normalizedseq/RNN_2.png}
\includegraphics[width=.45\textwidth]{../ESN_RNN_normalizedseq/RNN_4.png}
\includegraphics[width=.45\textwidth]{../ESN_RNN_normalizedseq/RNN_8.png}
\fi
%\iffalse
\begin{subfigure}
  \centering
  \includegraphics[width=0.5\linewidth]{ESN_RNN_normalizedseq/RNN_2.png}
%          \vspace{-2\baselineskip}
  \caption{Data dimension: 2}
\end{subfigure}
\begin{subfigure}
  \centering
  \includegraphics[width=0.5\linewidth]{ESN_RNN_normalizedseq/RNN_4.png}
%\vspace{-2\baselineskip}
  \caption{Data dimension: 4}
\end{subfigure}%\hfill
\begin{subfigure}
  \centering
  \includegraphics[width=0.5\linewidth]{ESN_RNN_normalizedseq/RNN_8.png}
%\vspace{-2\baselineskip}
  \caption{Data dimension: 8}
\end{subfigure}%
%\fi
\caption{Invertibitiliy of RNNs at random initialization: Checking behavior of inversion error with number of neurons and the sequence length at different data dimensions.}
\label{fig:RNN_inver}
\end{figure}
%\fi 
\textbf{RNN inversion at random initialization.} We consider a randomly initialized RNN, with the entries of the weights $\mathbf{W}$ and $\mathbf{A}$ randomly picked from the distribution $\mathcal{N}(0, 1)$. Sequences are generated i.i.d. from normal distribution i.e. for each sequence, $\bx^{(i)} \sim N(0, \mathbf{I})$ for each $i \in [L]$. We use SGD with batch size 128, momentum $0.9$ and learning rate $0.1$ to compute the linear matrix $\obW^{[L]}$ so that $\norm[0]{\obW^{[L]} \mathbf{h}^{(L)} - [\bx^{(1)}, \ldots, \bx^{(L)}]}^2$ is minimized. We compute the following two quantities on the test dataset, containing $1000$ sequences: average $L_2$ error given by $\mathbb{E}_{\bx} \frac{\norm[0]{\obW^{[L]} \mathbf{h}^{(L)} - [\bx^{(1)}, \ldots, \bx^{(L)}]}}{\norm[0]{[\bx^{(1)}, \ldots, \bx^{(L)}]}}$ and average $L_\infty$ error given by $\mathbb{E}_{\bx} \norm[0]{\obW^{[L]} \mathbf{h}^{(L)} - [\bx^{(1)}, \ldots, \bx^{(L)}]}_{\infty}$. We plot both the quantities for different settings of data dimension $d$, sequence length $L$ and the number of neurons $m$. $L$ takes values from the set $\{2, 4, 6\}$, $d$ takes from $\{2, 4, 8\}$ and $m$ takes from $\{500, 1000, 2000, 5000, 10000\}$ (Figure~\ref{fig:RNN_inver}). The trends support our bounds in Theorem~\ref{thm:Invertibility_ESN_outline}, i.e. the error increases with increasing $L$ and decreases with increasing $m$. Note that the data distribution is different from the one assumed in normalized sequence Def.~\ref{def:normalized_seq}. It was easier to conduct experiments in the current data setting and a similar statement as Thm.~\ref{thm:Invertibility_ESN_outline} can be given.



\textbf{Performance of RNNs on different regular languages. } We check the performance of RNNs on the formal language recognition task for a wide variety of regular languages. We follow the set-up in \cite{BhattamishraAG20} who conducted experiments on LSTMs etc. but not on RNNs.

%\section{Regular languages} \label{sec:regular}
We consider the regular languages as considered in \cite{BhattamishraAG20}.
Tomita grammars \cite{tomita:cogsci82} contain 7
regular languages representable by DFAs of small
sizes, a popular benchmark for evaluating recurrent models (see references in \cite{BhattamishraAG20}). 
We reproduce the definitions of the Tomita grammars from there verbatim:
Tomita Grammars are 7 regular langauges defined on the alphabet $\Sigma = \{0, 1\}$.
Tomita-1 has the regular expression $1^\ast$.
Tomita-2 is defined by the regular expression $(10)^\ast$.
Tomita-3 accepts the strings where odd number
of consecutive 1s are always followed by an even
number of $0$'s. Tomita-4 accepts the strings that
do not contain three consecutive $0$'s. In Tomita-5 only
the strings containing an even number of $0$'s and
even number of $1$'s are allowed. In Tomita-6 the
difference in the number of $1$'s and $0$'s should be
divisible by 3 and finally, Tomita-7 has the regular
expression $0^\ast 1^\ast 0^\ast 1^\ast$. 

We also check the performance of RNNs on $\mathrm{Parity}$, which contains all languages with strings of the form $(w_1, \ldots, w_L)$ s.t. $w_1 + \ldots + w_L = 1 \mod 2$. Languages $\mathcal{D}_n$ are recursively defined as the set of all strings of the form $(0w1)^{\ast}$, where $w \in \mathcal{D}_{n-1}$, with $\mathcal{D}_0$ containing only $\epsilon$, the empty word. Other languages considered are $(00)^{\ast}$, $(0101)^{\ast}$ and $(00)^{\ast}(11)^{\ast}$. Table~\ref{table:regular} shows the number of examples in train and test data, the range of the length of the strings in the language, and the test accuracy of the RNNs with activation functions $\relu$ and $\tanh$ on the regular languages mentioned above. 

\begin{center}
\begin{table}[!ht]
\centering
\begin{tabular}{|| c | c | c | c | c ||} 
 \hline
 Task & No. of Training/Test examples  & Range of length of strings & RNN(Relu) & RNN(Tanh)  \\ [0.5ex] 
 \hline\hline
 Tomita 1 & 50/100 & [2, 50] & 1.0 & 1.0 \\
 Tomita 2 & 25/50 & [2, 50] & 1.0 & 1.0 \\
 Tomita 3 & 10000/2000 & [2, 50] & 1.0 & 1.0 \\
 Tomita 4 & 10000/2000 & [2, 50] & 1.0 & 1.0 \\
 Tomita 5 & 10000/2000 & [2, 50] & 1.0 & 1.0\\
 Tomita 6 & 10000/2000 & [2, 50] & 1.0 & 1.0\\
 Tomita 7 & 10000/2000 & [2, 50] & 0.259 & 0.99\\
 Parity & 10000/2000 & [2, 50] & 1.0 & 1.0\\
 $\mathcal{D}_2$ & 10000/2000 & [2, 100] & 1.0 & 1.0 \\
 $\mathcal{D}_3$ & 10000/2000 & [2, 100] & 0.99 & 1.0\\
 $\mathcal{D}_4$ & 10000/2000 & [2, 100] & 1.0 & 0.99\\
 $(00)^{\ast}$ & 250/50 & [2, 500] & 1.0 & 1.0\\
 $(0101)^{\ast}$ & 125/25 & [4, 500] & 0.99 & 1.0 \\
 $(00)^{\ast}(11)^{\star}$ & 10000/2000 & [2, 200] & 0.99 & 1.0 
 %\\Dyck-1 & 0.99 & \textbf{0.91} 
 \\[1ex]
 \hline
\end{tabular}
\caption{Performance of RNNs on different regular languages.}
\label{table:regular}
\end{table}
\end{center}

%The test languages consist of Tomita grammars which constitute a popular benchmark, and a number of other languages including $\mathsf{PARITY}$ and cover a variety of capabilities needed to recognize regular languages.
%The details of the regular languages above can be found in \cite{BhattamishraAG20}; we only note that Tomita languages constitute a popular benchmark. 
We vary $m$, the dimension of the hidden state, in the range $[3, 32]$, used RMSProp optimizer~\cite{hinton2014coursera} with the smoothing constant $\alpha = 0.99$ and varied the learning rate in the range $[10^{-2}, 10^{-3}]$. For each language
we train models corresponding to each language
for $100$ epochs and a batch size of $32$. We experimented with two different activations $\relu$ and $\tanh$. 
%The best test accuracies achieved on different languages are given in table~\ref{table:regular} and these were all achieved for $m=32$.
In all but one case (Tomita 7 with ReLU) the test accuracies with near-perfect. This was the case across runs. Tomita 7 results could perhaps be improved by more extensive hyperparameter tuning. 
We train and test on strings of length up to 50, and in a few cases strings of larger lengths (when the number of strings in the language is small). 
%Details are in Appendix~\ref{sec:regular}.




\iffalse
\begin{figure*}

\centering
\includegraphics[width=.45\textwidth]{../ESN_RNN_normalizedseq/RNN_2.png}
\includegraphics[width=.45\textwidth]{../ESN_RNN_normalizedseq/RNN_4.png}
\includegraphics[width=.5\textwidth]{../ESN_RNN_normalizedseq/RNN_8.png}
\caption{Invertibility of RNNs at initialization}
%\label{fig:figure3}
\end{figure*}
%\FloatBarrier


\begin{center}
\begin{table}[!ht]
\centering
\begin{tabular}{|| c | c | c ||} 
 \hline
 Task & RNN(Relu) & RNN(Tanh)  \\ [0.5ex] 
 \hline\hline
 Tomita 1 & 1.0 & 1.0 \\
 Tomita 2 & 1.0 & 1.0 \\
 Tomita 3 & 1.0 & 1.0 \\
 Tomita 4 & 1.0 & 1.0 \\
 Tomita 5 & 1.0 & 1.0\\
 Tomita 6 & 1.0 & 1.0\\
 Tomita 7 & 0.259 & 0.99\\
 Parity & 1.0 & 1.0\\
 $\mathcal{D}_1$ & 1.0 & 1.0 \\
 $\mathcal{D}_2$ & 0.99 & 1.0\\
 $\mathcal{D}_4$ & 1.0 & 0.99\\
 $(aa)^{\ast}$ & 1.0 & 1.0\\
 $(abab)^{\ast}$ & 0.99 & 1.0 \\
 $(aa)^{\ast}(bb)^{\star}$ & 0.99 & 1.0 
 %\\Dyck-1 & 0.99 & \textbf{0.91} 
 \\[1ex]
 \hline
\end{tabular}
\caption{Performance of RNNs on different regular languages.}
\label{table:regular}
\end{table}
\end{center}
\fi

\section{Conclusion}
\label{sec:conclusion}
This paper presents a generic top-$\size$ recommendation framework for  trading-off accuracy, novelty, and coverage. To achieve this, we profile the users according to their preference for long-tail novelty. We examine various measures, and formulate an optimization problem to learn these user preferences from interaction data.  We then integrate the user preference estimates in our generic framework, GANC.  Extensive experiments on several datasets confirm that there are trade-offs between accuracy, coverage, and novelty. Almost all re-ranking models increase coverage and novelty at the cost of accuracy. However, existing re-ranking models typically rely on rating prediction models, and are hence more effective in dense settings. Using a generic approach, we can easily incorporate a suitable base accuracy recommender to devise an effective solution for both sparse and dense settings.  %Our results  also indicate there is no single method that outperforms other methods in all metrics. However our techniques obtain a significant improvement in coverage, while  . 
Although we integrated the  long-tail novelty preference estimates into a re-ranking framework, their use-case is not limited to these frameworks. In  the future, we intend to explore the temporal and topical dynamics of long-tail novelty preference, particularly in settings where contextual information is  available.  
%We achieve these objectives without using any additional contextual information.


\iffalse
While we focused on promoting long-tail items across users, we did not consider diversity of individual top-$\size$ recommendations, a factor that has been shown to positively affect consumer satisfaction. This is one direction for future work. Moreover, the sequential setting  in our work, creates a dependency between different batches, where,  the items recommended to a batch of users, depends on those recommended to previous batches. This dependency is created through the parameter $\mathbf{f}$, that is updated every time a top-$\size$ set  is allocated to a batch of users. A future direction for our work is to estimate a distribution over $\mathbf{f}$ that allows us to independently solve the problem for each user, leading to improvements across all performance metrics, including recommendation time. 

We design algorithms that take advantage of the structure in the value functions to obtain both efficient and scalable solutions. 
We design an algorithm that takes advantage of the structure in the value functions to obtain both efficient and scalable solutions. 

\textcolor{red}{Our  sequential  algorithms can be applied for batch recommendation contexts,~e.g., personalized email marketing, where based on prior interaction data between users and items,  a new round of recommendations must be sent to all users in the system.  However, the independent coverage algorithms lift the sequential setting restrictions and allow it be applied for re-ranking the output of base recommender in any setting. }A future direction for our work is to incorporate explicit diversity metrics in the framework. 
\fi


%We have a presented a submodular maximization framework to systematically trade-off relevance and diversity in recommendations to individual users and coverage across the item-space. This ensures both consumer and producer satisfaction. We model users according to their risk and focusing degrees and promote long-tail items to the right group of consumers. Consequently, we obtain a significant improvement in coverage while maintaining reasonable levels of user satisfaction. Furthermore, our methods are able to achieve a more balanced distribution across the set of recommended items. In the future, we plan to investigate the effect of using alternative base recommender systems. 

%Future Work
%However most of these methods assume that the ratings are missing at random (MAR). Since our method of generating recommendations is based on the completed matrix, assuming MAR might introduce additional bias, we will use methods which assume that the ratings at missing not at random (MNAR),explored in~\cite{steck2010training, icml2014c2_hernandez-lobatob14}. 	 
%Long Tail %Recently, authors in~\cite{cremonesi2010performance} conducted extensive experiments to evaluate the performances of various matrix factorization-based algorithms and neighborhood models on the task of recommending long tail items. Their experimental results show that long tail recommendation leads to a decrease in accuracy for all algorithms. They also showed that for this task, SVD outperforms other state-of-the-art algorithms. 


\section*{Acknowledgements}
This work was supported by National Natural Science Foundation of China (Grant No. 61876075), Guangdong Provincial Key Laboratory Grant(No. 2020B121201001), the Program for Guangdong Introducing Innovative and Enterpreneurial Teams (Grant No. 2017ZT07X386), Shenzhen Science and Technology Program (Grant No. KQTD2016112514355531), the Program for University Key Laboratory of Guangdong Province (Grant No. 2017KSYS008).
%
% ---- Bibliography ----
%
% BibTeX users should specify bibliography style 'splncs04'.
\bibliographystyle{splncs04}
\bibliography{reference.bib}
\end{document}
