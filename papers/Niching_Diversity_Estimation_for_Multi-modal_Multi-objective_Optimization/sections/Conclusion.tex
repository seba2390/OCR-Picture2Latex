\section{Concluding Remarks}
\label{sec: Conclusion}
In this paper, we proposed a niching diversity estimation method for multi-modal multi-objective optimization. First, we pointed out that standard diversity estimators in MOEAs meet some challenges when handling MMOPs. To address this issue, we proposed a general niching strategy which is applicable to existing MOEAs to enhance their performance on MMOPs. In our proposed niching strategy, only neighboring solutions in the decision space are involved in diversity estimation. In this manner, the proposed niching strategy is able to prevent the loss of equivalent Pareto optimal solutions. In our experimental studies, we incorporated the proposed niching strategy into two classical MOEAs: SPEA2 and NSGA-II. Experimental results on ten MMOPs clearly showed that the performance of the modified algorithms is notably improved compared to the original algorithms. Currently, we employed a simple niching strategy based on the $k$-th nearest neighbor. The value of $k$ was also simply specified by the square root of the sample size without considering the dimensionality of the decision space. The major contribution of this paper was to clearly illustrate that the incorporation of such a simple niching strategy significantly improved the performance of existing MOEAs on MMOPs. A future research direction can be examining the effect of the value of $k$ and to propose a more effective specification method. 
More experiments over various test problems using a wide variety of MOEAs can be conducted to examine the 
effects of our proposed niching mechanism on MOEAs. Moreover, another promising future research issue is developing more sophisticated and efficient niching strategies. In this research direction, the point may be how to handle the trade-off between the decision space performance and the objective space performance.