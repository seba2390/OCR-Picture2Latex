\section{Introduction}
\label{sec: Introduction}
\lettrine[findent=2pt]{\textbf{M}}{ }ulti-objective optimization problems (MOPs), which require an optimizer to optimize multiple conflicting objective functions simultaneously, have been actively studied in the past few decades. For consistency, in this paper, all objective functions are assumed to be converted to minimization problems. Due to the trade-off between conflicting objective functions, most MOPs have a set of Pareto optimal solutions (i.e., Pareto set), which cannot be dominated by any solutions. The projection of the Pareto set in the objective space is called the Pareto front. Generally, multi-objective optimization algorithms (MOEAs) try to find a solution set with good convergence (i.e., close to the Pareto front) and good diversity (i.e., well-distributed over the Pareto front). Therefore, diversity maintenance is a critical research topic across the field of evolutionary multi-objective optimization.

Most MOEAs are equipped with some diversity maintenance mechanisms. Naturally, diversity estimation, which is a procedure of assigning a numerical value to each solution reflecting its diversity, is a prerequisite for diversity maintenance. In the past few decades, various of diversity estimation methods have been developed along with the development of MOEAs. For instance, the well-known Pareto-based MOEA called NSGA-II \cite{NSGAII} selects solutions based on their Pareto ranks (as a primary criterion) obtained from the non-dominated sorting procedure and their crowding distance values (as a secondary criterion). In NSGA-II, crowding distance is a diversity estimator to estimate the diversity of each solution for environmental selection. Another Pareto-based algorithm called SPEA2 \cite{SPEA2} uses the Euclidean distance from each solution to its $k$-th nearest neighbor in the objective space to estimate the diversity of that solution. In the grid-based MOEA named PESA-II \cite{PESAII}, the diversity of each solution is given by the number of the solutions located in the same hyperbox in the objective space. As pointed out in \cite{SDE}, although the approaches used in different diversity estimators vary, the main idea is to estimate the diversity for a solution by measuring the similarity degree between that solution and other solutions in the population. In this paper, the term "diversity" always refers to the diversity in the objective space if not specified. Notice that larger diversity values are more preferable than smaller values.

Recently, multi-modal multi-objective optimization has become an active research topic. In MMOPs, the mapping from the decision space to the objective space is a many-to-one mapping instead of one-to-one mappings in standard MOPs. That is, multiple solutions with different decision values can have the same objective values. Such kind of solutions are termed as equivalent solutions. Fig. \ref{fig: MMOP example} gives an example of MMOP where solutions marked with the same number have the same objective values. Since it is unlikely for a real-world multi-objective optimization problem to have multiple solutions with exactly the same objective values, the definition of equivalent solutions can be relaxed as follows \cite{tanabe2020review}: 

\begin{figure}[htbp]
	\centering
	\includegraphics[width=.75\textwidth]{figures/Introduction/MMOP_example}
	\caption{Illustration of MMOP. The left and right figures show the decision and objective spaces, respectively, and the dash lines in (a) and (b) denote the Pareto set and Pareto front, respectively. Circles marked with the same number are solutions that have the same objective values.}
	\label{fig: MMOP example}
\end{figure}

\begin{definition}[Equivalent Solutions]
	Solutions $\boldsymbol{x}_1$ and $\boldsymbol{x}_2$ are equivalent solutions iff $d(\boldsymbol{F}(\boldsymbol{x}_1),\boldsymbol{F}(\boldsymbol{x}_2)) \leq \delta$,
	\label{def: equivalent solutions}
\end{definition}
where $\boldsymbol{F}$ is a vector containing all objective functions, $d$ denotes the Euclidean distance function, and $\delta$ is a positive threshold parameter specified by the end user.

Solving MMOPs is meaningful since they are common in many real-world applications, e.g., rocket engine design problems \cite{MMOP_rocket} and the space mission design problems \cite{space-mission-design} both can be formulated as MMOPs. In some engineering problems, obtained solutions can become infeasible or difficult to implement due to dynamically changing environments and constraints\cite{MMOEADC}. In this regard, equivalent solutions will be able to provide alternative implementations for the decision maker. For this reason, when solving MMOPs, it is a good strategy to try to search for as many equivalent solutions as possible if no user preference is given.

As pointed out in the literature \cite{MOEAD_AD}, standard MOEAs are usually unable to preserve multiple equivalent solutions. Since equivalent solutions are located in the same (or almost the same) position(s) in the objective space, diversity estimators tend to assign high density values to all of them. As a result, equivalent solutions are usually not preferable and likely to be removed in the environmental selection procedure, which leads to the failure of solving MMOPs. To tackle this problem, we propose the use of a simple niching strategy to make standard diversity estimators more efficient when handling MMOPs. Our approach is a simple, efficient, and parameterless mechanism which can be integrated into general diversity estimators in existing MOEAs.

The rest of the paper is organized as follows. Section \ref{sec: Related Work} revisits some representative diversity estimators in MOEAs and discusses their difficulties in the handling of MMOPs. In addition, Section \ref{sec: Related Work} also introduces some existing approaches for multi-modal multi-objective optimization. Next, Section \ref{sec: Proposed method} outlines our proposed niching diversity estimation method. Section \ref{sec: Experiments} reports experimental results. Lastly, concluding remarks and suggested future research directions are presented in Section \ref{sec: Conclusion}.