\section{Numerical Experiments}
\label{sec: Experiments}
\subsection{Experimental settings}
In our experiments, the performance of Niching-SPEA2 and Niching-NSGA-II as well as the corresponding original algorithms is evaluated on ten MMOPs. Specifically, we use the SYM-PART\cite{SYMPART}, the Omni-test\cite{OmniOptimizer}, and the MMF1--8\cite{MO_Ring_PSO_SCD} test problems for benchmarking. For the Omni-test problem, we set the number of decision variables to $3$. For the rest of test problems, the default parameter settings in the corresponding papers \cite{SYMPART, MO_Ring_PSO_SCD} are used. Each algorithm is evaluated on each test problem 31 times independently with population size 100 and 50,000 function evaluations. The niching parameter $k$ in Eq. (\ref{eq: Niching diversity estimator}) is set to $\floor{\sqrt{N}}$, where $N$ is the size of $\boldsymbol{S}$. This setting of $k$ is based on the suggestions in \cite{silverman1986density} for statistics and data analysis.

We choose two widely used indicators: $\operatorname{IGD}^+$\cite{IGDPlus} and IGDX\cite{IGDX} to evaluate the performance of an algorithm in the objective space and the decision space, respectively. Smaller $\operatorname{IGD}^+$ and IGDX values indicate better proximity of the obtained solution set to the Pareto front and the Pareto set, respectively.

\subsection{Experimental results}
To demonstrate the efficacy of our proposed niching strategy, first we visually examine the distribution of the solution sets found by Niching-NSGA-II and its original version on the SYM-PART test problem. Non-dominated solutions obtained from a single run of each algorithm are shown in Fig. \ref{fig: NSGA-II on SYM-PART} and Fig. \ref{fig: Niching-NSGA-II on SYM-PART}. In each figure, we select the run with the median $\operatorname{IGD}^+$ value among all 31 independent runs as a representative for visual examination.

\begin{figure}
	\centering
	\begin{subfigure}[b]{.49\textwidth}
		\includegraphics[width=\textwidth]{figures/Experiments/SYMPART_NSGA-II_X}
		\caption{Decision space.}
	\end{subfigure}
	\begin{subfigure}[b]{.49\textwidth}
		\includegraphics[width=\textwidth]{figures/Experiments/SYMPART_NSGA-II_F}
		\caption{Objective space.}
	\end{subfigure}
	\caption{Non-dominated solutions obtained by NSGA-II on SYM-PART. The black lines show Pareto optimal solutions, and red circles show the obtained solutions.}
	\label{fig: NSGA-II on SYM-PART}
\end{figure}

\begin{figure}
	\centering
	\begin{subfigure}[b]{.49\textwidth}
		\includegraphics[width=\textwidth]{figures/Experiments/SYMPART_Niching-NSGA-II_X}
		\caption{Decision space.}
	\end{subfigure}
	\begin{subfigure}[b]{.49\textwidth}
		\includegraphics[width=\textwidth]{figures/Experiments/SYMPART_Niching-NSGA-II_F}
		\caption{Objective space.}
	\end{subfigure}
	\caption{Obtained non-dominated solutions by Niching-NSGA-II on SYM-PART. The black lines show Pareto optimal solutions, and red circles show the obtained solutions.}
	\label{fig: Niching-NSGA-II on SYM-PART}
\end{figure}

Fig. \ref{fig: NSGA-II on SYM-PART} (a) clearly shows that NSGA-II is poorly performed on the SYM-PART test problem. Most obtained solutions are distributed in the upper three Pareto subsets, while almost no solution lies on the other six Pareto subsets. This is because NSGA-II is a standard MOEA without diversity maintenance mechanisms in the decision space. In comparison, Niching-NSGA-II clearly outperforms NSGA-II as shown in Fig. \ref{fig: Niching-NSGA-II on SYM-PART} (a), where all nine Pareto subsets are covered. This experimental result verifies that our proposed niching strategy can efficiently prevent the loss of equivalent solutions and preserve the diversity in the decision space. Regarding to the distribution in the objective space, in Figs. \ref{fig: NSGA-II on SYM-PART} (b) and \ref{fig: Niching-NSGA-II on SYM-PART} (b), Niching-NSGA-II slightly underperforms NSGA-II on the SYM-PART test problem. As reported in \cite{tanabe2020review}, this is because equivalent solutions have small (or even zero) contribution to the diversity in the objective space. These observations suggest that there is a clear trade-off between the diversity on the Pareto set in the decision space and the diversity on the Pareto front in the objective space when solving MMOPs.

\begin{table}
	\caption{Statistical comparison results regarding the IGDX indicator. Mean and standard deviation of IGDX values are shown. Better results are highlighted.}
	\label{table: IGDX comparison}
	\centering
	\centerline{
		\begin{tabular}{@{\hspace{.5em}}c@{\hspace{2em}}c@{\hspace{1em}}cc@{\hspace{2em}}c@{\hspace{1em}}c@{\hspace{.5em}}}
			\toprule
			                  & NSGA-II           & Niching-NSGA-II                               &  & SPEA2             & Niching-SPEA2                                       \\ \midrule
			Omni-test         & 1.2610 $\pm$ 0.35 & \cellcolor[HTML]{C0C0C0}0.3647 $\pm$ 0.15 $+$ &  & 1.3127 $\pm$ 0.36 & \cellcolor[HTML]{C0C0C0}1.2508 $\pm$ 0.23 $\approx$ \\
			SYM-PART          & 7.1278 $\pm$ 2.36 & \cellcolor[HTML]{C0C0C0}0.0647 $\pm$ 0.00 $+$ &  & 6.6313 $\pm$ 2.79 & \cellcolor[HTML]{C0C0C0}2.0840 $\pm$ 2.32 $+$       \\
			MMF1              & 0.1048 $\pm$ 0.03 & \cellcolor[HTML]{C0C0C0}0.0683 $\pm$ 0.00 $+$ &  & 0.1015 $\pm$ 0.02 & \cellcolor[HTML]{C0C0C0}0.0639 $\pm$ 0.00 $+$       \\
			MMF2              & 0.0565 $\pm$ 0.04 & \cellcolor[HTML]{C0C0C0}0.0205 $\pm$ 0.01 $+$ &  & 0.0734 $\pm$ 0.04 & \cellcolor[HTML]{C0C0C0}0.0422 $\pm$ 0.03 $+$       \\
			MMF3              & 0.0413 $\pm$ 0.02 & \cellcolor[HTML]{C0C0C0}0.0181 $\pm$ 0.01 $+$ &  & 0.0386 $\pm$ 0.02 & \cellcolor[HTML]{C0C0C0}0.0292 $\pm$ 0.03 $+$       \\
			MMF4              & 0.1659 $\pm$ 0.07 & \cellcolor[HTML]{C0C0C0}0.0432 $\pm$ 0.00 $+$ &  & 0.1312 $\pm$ 0.04 & \cellcolor[HTML]{C0C0C0}0.0426 $\pm$ 0.00 $+$       \\
			MMF5              & 0.2003 $\pm$ 0.04 & \cellcolor[HTML]{C0C0C0}0.1095 $\pm$ 0.00 $+$ &  & 0.1843 $\pm$ 0.03 & \cellcolor[HTML]{C0C0C0}0.1054 $\pm$ 0.00 $+$       \\
			MMF6              & 0.2468 $\pm$ 0.06 & \cellcolor[HTML]{C0C0C0}0.0972 $\pm$ 0.00 $+$ &  & 0.1995 $\pm$ 0.04 & \cellcolor[HTML]{C0C0C0}0.1516 $\pm$ 0.02 $+$       \\
			MMF7              & 0.0680 $\pm$ 0.03 & \cellcolor[HTML]{C0C0C0}0.0463 $\pm$ 0.00 $+$ &  & 0.0751 $\pm$ 0.03 & \cellcolor[HTML]{C0C0C0}0.0320 $\pm$ 0.00 $+$       \\
			MMF8              & 1.6918 $\pm$ 0.65 & \cellcolor[HTML]{C0C0C0}0.0999 $\pm$ 0.02 $+$ &  & 1.3505 $\pm$ 0.67 & \cellcolor[HTML]{C0C0C0}0.3508 $\pm$ 0.13 $+$       \\ \midrule
			$+$/$-$/$\approx$ & baseline          & 10/0/0                                        &  & baseline          & 9/0/1                                               \\ \bottomrule\end{tabular}
	}
\end{table}

\begin{table}
	\caption{Statistical comparison results regarding the $\operatorname{IGD}^+$ indicator. Mean and standard deviation of $\operatorname{IGD}^+$ values are shown. Better results are highlighted.}
	\label{table: IGD+ comparison}
	\centering
	\centerline{
		\begin{tabular}{@{\hspace{.5em}}c@{\hspace{2em}}c@{\hspace{1em}}cc@{\hspace{2em}}c@{\hspace{1em}}c@{\hspace{.5em}}}
			\toprule
			                  & NSGA-II                                   & Niching-NSGA-II       &  & SPEA2                                     & Niching-SPEA2         \\ \midrule
			Omni-test         & \cellcolor[HTML]{C0C0C0}0.0100 $\pm$ 0.00 & 0.0192 $\pm$ 0.00 $-$ &  & \cellcolor[HTML]{C0C0C0}0.0081 $\pm$ 0.00 & 0.0138 $\pm$ 0.00 $-$ \\
			SYM-PART          & \cellcolor[HTML]{C0C0C0}0.0081 $\pm$ 0.00 & 0.0124 $\pm$ 0.00 $-$ &  & \cellcolor[HTML]{C0C0C0}0.0074 $\pm$ 0.00 & 0.0138 $\pm$ 0.00 $-$ \\
			MMF1              & \cellcolor[HTML]{C0C0C0}0.0033 $\pm$ 0.00 & 0.0039 $\pm$ 0.00 $-$ &  & \cellcolor[HTML]{C0C0C0}0.0027 $\pm$ 0.00 & 0.0031 $\pm$ 0.00 $-$ \\
			MMF2              & \cellcolor[HTML]{C0C0C0}0.0034 $\pm$ 0.00 & 0.0041 $\pm$ 0.00 $-$ &  & \cellcolor[HTML]{C0C0C0}0.0031 $\pm$ 0.00 & 0.0051 $\pm$ 0.00 $-$ \\
			MMF3              & \cellcolor[HTML]{C0C0C0}0.0032 $\pm$ 0.00 & 0.0039 $\pm$ 0.00 $-$ &  & \cellcolor[HTML]{C0C0C0}0.0029 $\pm$ 0.00 & 0.0034 $\pm$ 0.00 $-$ \\
			MMF4              & \cellcolor[HTML]{C0C0C0}0.0031 $\pm$ 0.00 & 0.0045 $\pm$ 0.00 $-$ &  & \cellcolor[HTML]{C0C0C0}0.0026 $\pm$ 0.00 & 0.0038 $\pm$ 0.00 $-$ \\
			MMF5              & \cellcolor[HTML]{C0C0C0}0.0033 $\pm$ 0.00 & 0.0044 $\pm$ 0.00 $-$ &  & \cellcolor[HTML]{C0C0C0}0.0027 $\pm$ 0.00 & 0.0039 $\pm$ 0.00 $-$ \\
			MMF6              & \cellcolor[HTML]{C0C0C0}0.0033 $\pm$ 0.00 & 0.0043 $\pm$ 0.00 $-$ &  & \cellcolor[HTML]{C0C0C0}0.0027 $\pm$ 0.00 & 0.0039 $\pm$ 0.00 $-$ \\
			MMF7              & \cellcolor[HTML]{C0C0C0}0.0034 $\pm$ 0.00 & 0.0071 $\pm$ 0.00 $-$ &  & \cellcolor[HTML]{C0C0C0}0.0029 $\pm$ 0.00 & 0.0034 $\pm$ 0.00 $-$ \\
			MMF8              & \cellcolor[HTML]{C0C0C0}0.0026 $\pm$ 0.00 & 0.0034 $\pm$ 0.00 $-$ &  & \cellcolor[HTML]{C0C0C0}0.0023 $\pm$ 0.00 & 0.0034 $\pm$ 0.00 $-$ \\ \midrule
			$+$/$-$/$\approx$ & baseline                                  & 0/10/0                &  & baseline                                  & 0/10/0                \\ \bottomrule
		\end{tabular}}
\end{table}

Table \ref{table: IGDX comparison} and Table \ref{table: IGD+ comparison} present the statistical comparison results regarding the IGDX and $\operatorname{IGD}^+$ indicators, respectively. In each table, the Wilcoxon rank-sum test is performed with $p=0.05$ to compare the performance of Niching-SPEA2 and Niching-NSGA-II with their original algorithms. The symbols "$+$", "$-$", and "$\approx$" in each table indicated that the corresponding algorithm is outperform, underperform, and tied with the baseline in the statistical comparison.

Table \ref{table: IGDX comparison} clearly shows the superiority of Niching-SPEA2 and Niching-NSGA-II in comparison to the original versions regarding the IGDX indicator. That is, the two modified algorithms have significantly smaller IGDX values than the corresponding original algorithms for almost all test problems. The statistical comparison results further verify that the proposed niching approach can significantly improve the performance of SPEA2 and NSGA-II on various MMOPs. In Table \ref{table: IGD+ comparison}, we can see that the modified algorithms with the niching strategy have worse $\operatorname{IGD}^+$ values on all test problems than the original ones. This is consistent with the our previous observations on Fig. \ref{fig: NSGA-II on SYM-PART} and Fig. \ref{fig: Niching-NSGA-II on SYM-PART} (i.e., there exists a trade-off between the diversity in the decision and the objective spaces). However, from careful examinations of Table \ref{table: IGDX comparison} and Table \ref{table: IGD+ comparison}, we can see for many test problems that large improvement of the IGDX values in Table \ref{table: IGDX comparison} is obtained at the cost of small deterioration of the $\operatorname{IGD}^+$ values in Table \ref{table: IGD+ comparison}. The observation above demonstrates that the proposed niching strategy is a promising approach to multi-modal multi-objective optimization problem, whereas the handling of the trade-off remains an open question.