\section{Proposed Method}
\label{sec: Proposed method}
\subsection{Niching diversity estimation}
In this section, we outline our proposed niching diversity estimation method for multi-modal multi-objective optimization. Here we first introduce a general representation for most diversity estimators used in MOEAs before diving into the details of the proposed method.

Generally, for a solution $\boldsymbol{x}_i$ in a solution set $\boldsymbol{S}$, its diversity regarding $\boldsymbol{S}$ can be expressed as follows:
\begin{equation}
	\textit{Diversity}(\boldsymbol{x}_i, \boldsymbol{S}) = \mathop{\bigdelta}_{\boldsymbol{x}_j\in\boldsymbol{S}, j \neq i}{C(\boldsymbol{x}_i, \boldsymbol{x}_j)},
	\label{eq: Standard diversity estimator}
\end{equation}
where $C$ is a function which calculates the diversity contribution from a pair of solutions $\boldsymbol{x}_i$ and $\boldsymbol{x}_j$ regarding their objective values, and $\Omega$ is an aggregation function (e.g., sum or mean) to combine the diversity contribution from each pair.
The idea of our proposed method is straightforward: to restrict the diversity estimation within a niche. We make the following simple modifications to Eq. (\ref{eq: Standard diversity estimator}):
\begin{equation}
	\textit{Niching-Diversity}(\boldsymbol{x}_i, \boldsymbol{S}) = \textit{Diversity}(\boldsymbol{x}_i, \boldsymbol{S}^\prime),
	\label{eq: Niching diversity estimator}
\end{equation}
where $\boldsymbol{S}^\prime$ contains all solutions in $\boldsymbol{S}$ which are in the same niche as $\boldsymbol{x}_i$. In our paper, the closest $k$ solutions in $\boldsymbol{S}$ to $\boldsymbol{x}_i$ in the \textbf{decision space} are considered as a niche.

From Eq. (\ref{eq: Niching diversity estimator}), we can see that the diversity estimation for each solution is limited to its neighbors in the decision space. With the niching strategy, solution distribution in the decision space is taken into consideration. Take Fig. \ref{fig: Difficulty when handling MMOPs} as an example, if $k = 2$ and crowding distance is used, the two nearest neighbors are selected for each solution (e.g., solution $A$) in the decision space, and the crowding distance is calculated using the selected neighbors in the objective space. Solution $B$ is unlikely to be chosen as a neighbor of solution $A$ (i.e., $B$ is not likely to be used for the crowding distance calculation of $A$). In this manner, the diversity of $A$ and $B$ can be estimated in a desirable manner for maintaining the decision space diversity. From this example, we can see that the proposed niching strategy can help diversity estimators in MOEAs to handle MMOPs properly with an appropriate value of $k$.

Compared to existing approaches we have discussed in Section \ref{sec: Existing multi-modal multi-objective optimization algorithms}, our proposed method does not rely on the actual implementations of diversity estimators. It is a general niching strategy that can be conveniently integrated into most diversity estimators in MOEAs.


\subsection{SPEA2 and NSGA-II with niching diversity estimation}
In this section, we select two classical MOEAs: SPEA2 and NSGA-II to demonstrate the procedure of our proposed niching diversity estimation method into MOEAs. The resulting algorithms are termed Niching-SPEA2 and Niching-NSGA-II, respectively.

In SPEA2 and NSGA-II, diversity estimation is only involved in the environmental selection procedure although the estimated diversity values may be used in other procedures. Therefore, we only describe the modified versions of environmental selection.

In the environmental selection procedure of Niching-SPEA2, the niching strategy is applied to both fitness calculation and archive truncation as outlined in lines 4 and 13 in Algorithm \ref{algo: Niching SPEA2}. In these two procedures, distance calculation is restricted by the niching strategy. For Niching-NSGA-II, in each generation, non-dominated sorting is employed to rank the whole population into several fronts. Afterward, the crowding distance is computed in each front with the proposed niching strategy.

\begin{algorithm}
	\caption{Environmental Selection Procedure of Niching-SPEA2.}
	\label{algo: Niching SPEA2}
	\Input{$\boldsymbol{P}$: input population;\\$N$: the number of survivors;}
	\Output{$\boldsymbol{Q}$: population for the next generation;}
	\tcc{Fitness assignment}
	$\boldsymbol{S} \gets$ the strength value for each solution in $\boldsymbol{P}$\;
	$\boldsymbol{R} \gets$ the raw fitness value for each solution in $\boldsymbol{P}$\;
	$\boldsymbol{D} \gets$ the \textbf{niching density value} for each solution in $\boldsymbol{P}$\;
	\For{$i=1,2,\ldots,|\boldsymbol{P}|$}{
		$\boldsymbol{F}(i) = \boldsymbol{R}(i) + \boldsymbol{D}(i)$; \tcp{fitness value}
	}
	\tcc{Archive truncation}
	$\boldsymbol{Q} \gets $ non-dominated solutions in $\boldsymbol{P}$\;
	\uIf{$|\boldsymbol{Q}| < N$}{
		$\boldsymbol{Q} \gets $ best $N$ solutions in $\boldsymbol{P}$ regarding their fitness values.
	}\Else{
		\While{$|\boldsymbol{Q}| > N$}{
			Repeatedly remove the solution with shortest distance to other solutions \textbf{in the same niche} from $\boldsymbol{Q}$.
		}
	}
\end{algorithm}




