\documentclass[twoside, 11pt]{article}

\usepackage{graphicx, subfigure}
\graphicspath{{figures/}}
\usepackage[margin=1in, footnotesep=0.5in]{geometry}
%\usepackage[margin=0.5in, paperwidth=6in, paperheight=8in, footnotesep=0.5in]{geometry}
\usepackage[sort&compress, numbers]{natbib} \setlength{\bibsep}{0.0pt}
\usepackage{amsfonts, amsmath, amssymb, amsthm}
\usepackage{MnSymbol}
\usepackage{mathtools}
%\mathtoolsset{showonlyrefs}
\usepackage{enumitem}

%%%%%%%%%%%%%%%%%%%%%%%%%%%%%%%%%%
%%%%%%%%%%%%%%%%%%%%%%%%%%%%%%%%%%

% colors
\usepackage{color}
\definecolor{darkred}{RGB}{100,0,0}
\definecolor{darkgreen}{RGB}{0,100,0}
\definecolor{darkblue}{RGB}{0,0,150}

% links
\usepackage{hyperref}
\hypersetup{colorlinks=true, linkcolor=darkred, citecolor=darkgreen, urlcolor=darkblue}
\usepackage{url}

\input notation.tex

% anotation
\definecolor{purple}{rgb}{0.4,.1,.9}
\newcommand{\ery}[1]{\textcolor{purple}{[#1]}}
\newcommand{\enew}[1]{\textcolor{purple}{#1}}
\newcommand{\clement}[1]{{\color{blue}[#1]}}
\newcommand{\cnew}[1]{{\color{blue}#1}}
\newcommand{\daniel}[1]{\textcolor{red}{[#1]}}
\newcommand{\dnew}[1]{\textcolor{red}{#1}}

\newcommand\blfootnote[1]{%
  \begingroup
  \renewcommand\thefootnote{}\footnote{#1}%
  \addtocounter{footnote}{-1}%
  \endgroup
}

%%%%%%%%%%%%%%%%%%%%%%%%%%%%%%%%%%
%%%%%%%%%%%%%%%%%%%%%%%%%%%%%%%%%%

\pagestyle{myheadings}
\raggedbottom

\begin{document}
\thispagestyle{empty}

\title{Theoretical Foundations of Ordinal Multidimensional Scaling, Including Internal and External Unfolding}
%\footnotetext{
%We would like to thank XXX.
%This work was partially supported by the US National Science Foundation XXX.}
\author{
Ery Arias-Castro\,\footnote{Department of Mathematics and Halıcıoğlu Data Science Institute, University of California, San Diego} 
\and
Clément Berenfeld\,\footnote{Institute of Mathematics, University of Potsdam}
\and Daniel Kane\,\footnote{Department of Computer Science and Department of Mathematics, University of California, San Diego} 
}
\date{}
\maketitle

\begin{abstract}
We provide a comprehensive theory of multiple variants of ordinal multidimensional scaling, including external and internal unfolding. We do so in the continuous model of Shepard (1966). 
%\medskip\noindent
%{\em Keywords and phrases:}
\end{abstract}



\section{Introduction} 
\label{sec:introduction}

Ordinal or non-metric multidimensional scaling (MDS) consists in embedding a set of abstract items based on pairwise dissimilarity information. MDS has been an important part of data analysis within Psychometrics since at least the 1930s \cite{young1938discussion}. 
It is nowadays an integral part of multivariate analysis in Statistics and of unsupervised learning in Machine Learning, and is known under different names in various areas such as Mathematics and Computer Science (embedding of metric spaces) \cite{blumenthal1938distance}, including in Optimization (Euclidean distance matrix completion) \cite{laurent2001matrix}, and  Engineering (sensor network localization) \cite{priyantha2003anchor}. 
We provide references throughout the article.

In its ordinal form, the basic problem of MDS consists in finding a configuration of points in a given Euclidean space whose pairwise distances agree in ranking as much as possible with a partial ranking of the pairwise dissimilarities between the items.
This variant of the problem is particularly important in Psychometrics, where a human subject may be asked to compare objects in triads \cite{torgerson1952multidimensional} by answering questions such as ``Is item A closer to item B or item C?''.

Other variants of the problem that are also considered within Psychometrics are internal unfolding and external unfolding. Internal unfolding is the problem of positioning individuals and objects in space based on preference data \cite{coombs1950psychological, bennett1956determination, tucker1960intra}. In the ordinal variant of the problem, a ranking of the objects is available for each individual. This is a special case of MDS where some of the dissimilarities --- those between individuals and those between objects --- are simply missing. 
In external unfolding \cite{gower1968adding, carroll1972individual, carroll1967}, the positions of the objects are known, i.e., the objects are already embedded. 
%The problem is known as trilateration (in dimension~2) in the Engineering literature.

\subsection{Contribution and Content}
While a large body of work is dedicated to methods, there is much less available in terms of theory. In the present paper, we aim at giving a comprehensive theory of ordinal embedding in all these variants. To avoid technicalities, we adopt the continuous model pioneered by \citet{shepard1966metric} in his study of ordinal embedding. 
For comparison, in each variant of the problem that we consider, we provide an overview of relevant theoretical results in the metric setting. 

In \secref{external}, we consider ordinal external unfolding. 
In \secref{mds}, we consider ordinal multidimensional scaling in a setting where all triadic comparisons are available. We outline the reasoning of \citet{shepard1966metric} and contrast that with a different approach based on recent work by \citet{klein} and by ourselves \cite{arias2017some}. 
In \secref{internal}, we consider ordinal internal unfolding.
We close the paper with a brief discussion in \secref{discussion}.

The theory that we develop for external and internal unfolding is some of the only theory that we are aware of for these problems. We consider both the point model --- in which the individuals are to be embedded as points \cite{coombs1950psychological} --- and the vector model --- where the individuals are to be represented by vectors instead \cite{bennett1956determination, tucker1960intra}.
While internal unfolding is known to be difficult in practice as discussed, e.g., in \cite{busing2005avoiding} and in \cite[Ch 14, 15]{borg2005modern}, our theory shows that at least the problem is well-posed. However, theoretical study does not provide any insight on methodology.   

We do not consider the sort of item response models featured in \cite{RECKASE2006607,reckase2009multidimensional}, which have evolved from the ranking models that originated from early work of \citet{thurstone1927law}, \citet{bradley1952rank}, and others. 

\subsection{Notation}
For a positive integer $n$, $[n] := \{1,\dots,n\}$.
For a point $x \in \bbR^p$ and $r > 0$, $\ball(x,r)$ denotes the open ball centered at $x$ with radius $r$, and $\sphere(x,r)$ denotes the corresponding sphere.
The unit sphere will be denoted $\bbS^{p-1} := \{x \in \bbR^p : \|x\| = 1\}$.
For two distinct points $z,z' \in \bbR^p$, define $H(z,z') = \{x : \|x-z\| = \|x-z'\|\}$, which is the affine hyperplane passing through $\frac12(z+z')$ perpendicular to $z-z'$; we also define $H^+(z,z') = \{x : \|x-z\| < \|x-z'\|\}$, which is one of the two open half-spaces defined by that hyperplane. 


\section{External Unfolding} 
\label{sec:external}

{\em External unfolding} is the problem of locating an individual in space based on preferences for some objects that are already positioned in that space. 
The earlier reference we know of in the Psychometrics literature is from \citet{gower1968adding}, who presents the problem as an out-of-sample extension of classical scaling. (We are talking about the method of \citet{young1938discussion}, later refined by \citet{torgerson1952multidimensional}, although Gower refers to his own work \cite{gower1966some}.)
We also know of contemporary work by \citet{carroll1967}, although only indirectly by way of \cite{srinivasan1973linear}.
%
While the term `external unfolding' is favored in Psychometrics \cite[Sec 16.1]{borg2005modern}, where it is often attributed to \citet{carroll1972individual}, the problem is best known in Engineering as `trilateration', `multilateration', or simply `lateration'  \cite{chrzanowski1965theoretical, glossary1994, fang1986trilateration, yang2009indoor, savvides2001dynamic, aspnes2006theory}.  
%The problem has also been considered in Applied Mathematics broadly understood \cite{kearsley1998solution, laurent2001polynomial, bakonyi1995euclidian, grone1984positive}.   
In Engineering, the objects are often called `anchors' or `landmarks'.
Some background is provided for statisticians in \cite{navidi1998statistical}.

While most of that work (in particular, within Engineering) has been done on the metric variant of the problem, the non-metric or ordinal variant has also received some attention  \cite{srinivasan1973linear, massimino2021you, davenport2013lost, anderton2019scaling}.
All these works place themselves in the point model. 


\subsection{Point Model}
In the point model, the individual and the objects are represented as points in space \cite[Sec 14.1]{borg2005modern}. We do so in the non-metric or ordinal variant of the problem where an individual $x$ is to be located based on a ranking of the individual's preference for the objects. The objects are already embedded in space and preference is quantified in terms of the Euclidean distance.

\subsubsection{Discrete Setting}
\label{sec:external point discrete}
In the discrete (in fact, finite) setting --- which is the setting encountered in practice --- the problem can be described as follows: Given $y_1, \dots, y_n \in \bbR^p$ and a permutation $(r_1, \dots, r_n)$ of $(1, \dots, n)$,
\begin{equation}
\label{total_order_finite_point}
\text{Find $x \in \bbR^p$ such that $\|x-y_i\| < \|x-y_j\|$ whenever $r_i < r_j$.}
\end{equation}

The recent work by \citet{massimino2021you} provides the most comprehensive study to date. Even then, to simplify the analysis, the authors consider a variant of the problem where the design is random: objects are sampled iid from some isotropic normal distribution, yielding $y_1, \dots, y_n$ and $y'_1, \dots, y'_n$, and (in the noiseless setting) for an unknown point $x$ we have access to $\xi_i := \IND{\|x-y_i\| < \|x-y'_i\|}$ for all $i \in [n]$. That is, based on the pairs $(y_1, y'_1), \dots, (y_n, y'_n)$ and the comparisons $\xi_1, \dots, \xi_n$, the goal is to recover $x$. The assumption that the design is not only random but Gaussian is crucial to the analysis carried out in \cite{massimino2021you}.
In related work, \citet{pmlr-v97-canal19a} consider the problem of actively selecting the objects in order to maximize the accuracy in locating the individual. The authors provide an information lower for the problem and show that a Bayesian approach this propose matches that bound in order of magnitude.

\subsubsection{Continuum Setting}
\label{sec:external point continuum}
We are interested in the fundamental question of whether there is enough information to recover the unknown location of the individual. This is clearly not the case in the discrete setting of \secref{external point discrete} as the set of solutions is open. However, we contend that the solution set reduces to a singleton in the limit of an infinite number of objects. Instead of tackling this claim in a frontal manner, in order to avoid technicalities and get to the core of the question, we follow \citet{shepard1966metric} and consider a limit model in the continuum where the set of objects is continuously infinite. 
%Given $\cY \subset \bbR^p$ and a function $r: \cY \to \bbR$,
%\begin{equation}
%\text{Find $x \in \bbR^p$ such that $\|x-y\| < \|x-y'\|$ whenever $r(y) < r(y')$.}
%\end{equation}
This corresponds to the limit of the discrete model if we image the points representing the objects $y_1, \dots, y_n$ as filling a set, denoted $\cY$ below. 

We say that $x, x' \in \bbR^p$ are equivalent with respect to $\cY$ if  
\begin{equation}
\label{external_point_continuum}
\text{$\|x-y\| < \|x-y'\| \iff \|x'-y\| < \|x'-y'\|$, \quad for all $y, y' \in \cY$.}
\end{equation}
Any two such points are indistinguishable in terms of their preference for the objects.
The central question of whether an individual can be located based on its preference for the objects can be phrased as follows: Do equivalent points coincide? The answer is positive under some conditions on the set of objects.

\begin{theorem}
\label{thm:external_point}
If $\cY$ has non-empty interior, equivalent points must coincide. 
\end{theorem}

In the proof, we will use the fact that \eqref{external_point_continuum} implies
\begin{equation}
\label{external_point_continuum_equal}
\text{$\|x-y\| = \|x-y'\| \iff \|x'-y\| = \|x'-y'\|$, \quad for all $y, y' \in \cY$.}
\end{equation}

\begin{proof}
We prove the result in the more general case where $\cY$ contains three spheres whose centers are not collinear.
In that case, for any $x \in \bbR^p$ there are two spheres inside $\cY$, $\cS_1 := \sphere(z_1, t_1)$ and $\cS_2 := \sphere(z_2, t_2)$, such that $x, z_1, z_2$ are not collinear. 
For any $x'$ equivalent to $x$ with respect to $\cY$, we claim that $x' \in (xz_j)$ for $j = 1$ and $j = 2$. If true, we can immediately conclude since $x$ is the only point at the intersection of the two lines $\cL_1 := (xz_1)$ and $\cL_2 := (xz_2)$. 

We focus on proving that $x' \in \cL_1$. 
%For any pair of points $y \ne y'$, consider the hyperplane $H(y,y')$.
We use the fact that $\cL_1$ is the intersection of all the hyperplanes passing through $z_1$ with orthogonal direction perpendicular to $\cL_1$. Consider such a hyperplane $\cH$ with orthogonal direction $v$, so that $v \perp \cL_1$. The line passing through $z_1$ with direction $v$ intersects $\cS_1$ at two points, $y_-$ and $y_+$. 
Since $\cS_1 \subset \cY$, we have $y_-, y_+ \in \cY$, and the hyperplane they define is exactly $\cH$ by construction.
And, since $x\in \cL_1$ and $\cL_1 \subset \cH$, we must have $\|x-y_-\| = \|x-y_+\|$. 
By \eqref{external_point_continuum_equal}, we must also have that $\|x'-y_-\| = \|x'-y_+\|$, so that $x' \in \cH$ as well.
Since this is true for any hyperplane $\cH$ that contains $\cL_1$, we have proved that $x' \in \cL_1$.
\end{proof}



\subsection{Vector Model}
In the vector model, the individual and the objects are represented as vectors\footnote{In fact, the objects may be equivalently considered as points.} instead of points \cite[Sec 16.2]{borg2005modern}. We only consider the case where the dimension is $p \ge 2$, as otherwise the situation is trivial (and not useful).
In the non-metric or ordinal variant, the individual preference for the objects is quantified in terms of inner products instead of distances. 

\subsubsection{Discrete Setting}
\label{sec:external vector discrete}
The problem encountered in practice is as follows: Given $y_1, \dots, y_n \in \bbR^p$ and a permutation $(r_1, \dots, r_n)$ of $(1, \dots, n)$,
\begin{equation}
\label{total_order_finite_vector}
\text{Find $x \in \bbS^{p-1}$ such that $\<x, y_i\> < \<x, y_j\>$ whenever $r_i > r_j$.}
\end{equation}
(Note that change in the direction of the second inequality. We adopted this convention so that the ranks have the same meaning as they do in \eqref{total_order_finite_point}.)

We are not aware of any theory for this problem. 

\subsubsection{Continuous Setting}
\label{sec:external vector continuum}
In the continuous model, we assume that the objects fill a continuously infinite set denoted $\cY \subset \bbR^p$. 
We say that $x, x' \in \bbS^{p-1}$ are equivalent with respect to $\cY$ if  
\begin{equation}
\label{external_vector_continuum}
\text{$\<x, y\> < \<x, y'\> \iff \<x', y\> < \<x', y'\>$, \quad for all $y, y' \in \cY$.}
\end{equation}
Any two such vectors are indistinguishable based on their preference for the objects. The central question is the same: Do equivalent vectors coincide? The answer is positive under the same condition as in \thmref{external_point}.

\begin{theorem}
\label{thm:external_vector}
If $\cY$ has non-empty interior, equivalent vectors must coincide. 
\end{theorem}

\begin{proof}
We prove the result in the more general setting where $\cY$ contains a sphere. 
By translation and scaling, if needed, we may assume without loss of generality that $\cY$ contains $\bbS^{p-1}$. 

Take two equivalent vectors $x,x'$. 
Applying \eqref{external_vector_continuum} with $y' = x$, which we can do since $x \in \bbS^{p-1} \subset \cY$, we have that 
\begin{equation}
\text{$\<x, y\> < \<x, x\> \iff \<x', y\> < \<x', x\>$, \quad for all $y \in \bbS^{p-1}$.}
\end{equation}
Since $\<x, y\> < \<x, x\>$ is true for any $y \in \bbS^{p-1} \setminus \{x\}$, we have that $\<x', y\> < \<x', x\>$ for all $y \in \bbS^{p-1} \setminus \{x\}$, which is only possible if $x' = x$.
\end{proof}

It is also the case here that \eqref{external_vector_continuum} implies
\begin{equation}
\label{external_vector_continuum_equal}
\text{$\<x, y\> = \<x, y'\> \iff \<x', y\> = \<x', y'\>$, \quad for all $y, y' \in \cY$,}
\end{equation}
but we did not make use of this identity in the proof.
%It turns out that the result is almost true under the weaker identity \eqref{external_vector_continuum_equal}. Indeed, the reader is invited to check that, under the same condition on $\cY$, it implies that equivalent vectors must be collinear (identical or opposite) --- and this is clearly still true even when $\cY$ is the entire space.


%\begin{corollary}
%If $\cY$ has non-empty interior, equivalent points must coincide. 
%\end{corollary}


\section{Multidimensional Scaling} 
\label{sec:mds}
In {\em multidimensional scaling (MDS)}, we have $n$ items, and the goal is to locate all them in space based on some pairwise dissimilarity information \cite{borg2005modern, cox2000multidimensional, young1987multidimensional}.  
The problem is known under various names in various fields:  `embedding of metric spaces' in Mathematics and Computer Science \cite{blumenthal1938distance}; `Euclidean distance matrix completion' in Optimization \cite{laurent2001matrix}; `sensor network localization' in Engineering \cite{priyantha2003anchor}. 
MDS is an integral part of Multivariate Analysis \cite{MVA, seber2009multivariate}.

We focus on the non-metric or ordinal variant of the problem, which has been of great interest in Psychometrics and beyond.
The body of work on this problem is substantial. Methodological work was pioneered by \citet{MR0173342,MR0140376} and especially \citet{kruskal1964nonmetric,kruskal1964multidimensional}, and has continued to this day \cite{terada2014local, mair2022more, anderton2019scaling, liu2016towards, agarwal2007generalized, van2012stochastic, tamuz2011adaptively}. 
Not much theoretical work is available. \citet{shepard1966metric} pioneered some theory, which was elaborated much more recently by \citet{klein} and ourselves \cite{arias2017some}. 
\citet{jain2016finite} consider a situation where the available information is noisy or imprecise and approach the problem via maximum likelihood. 

In our study of the problem below, we focus on what \citet{torgerson1952multidimensional} calls the (complete) method of triads, and what is referred to as triadic comparisons in \cite{DELEEUW1982285} going back to early work of \citet{stumpf1880} in the late 1800s. We only treat the point model, as this seems to be the only model that is considered in the literature. For the sake of curiosity, we discuss a possible vector model in the \hyperref[appendix]{Appendix}.

\subsection{Discrete Setting}
\label{sec:mds discrete}
In the usual setting, ordinal MDS can be described as follows: 
Given a set of row ranks $(r_{ij} : i, j \in [n])$, with each $(r_{i1}, \dots, r_{in})$ being a permutation of $(1, \dots, n)$, and a dimension $p \ge 1$, 
\begin{equation}\label{mds_discrete}
\begin{gathered}
\text{Find $x_1, \dots, x_n \in \bbR^p$ such that $\|x_i-x_j\| < \|x_i-x_k\|$ whenever $r_{ij} < r_{ik}$.}
\end{gathered}
\end{equation}
We note that any similarity transformation of a solution is also a solution.

In regards to the same foundational question of uniqueness (up to a similarity transformation), except for some earlier work in dimension $p=1$ \cite{suppes1955axiomatization, aumann1958coefficients}, the first meaningful contribution appears to be that of \citet{shepard1966metric}. To simplify matters, Shepard considered a continuous limit model, as is routinely done in Physics, for example. We describe such a model in the following subsection, and elaborate on his reasoning. This is the model that inspired the continuous models of \secref{external point continuum} and \secref{external vector continuum}.
Further progress was made decades later by \citet{klein}, who considered the finite sample situation described above and derived conditions under which, in the asymptotic limit $n\to\infty$ where the points fill a subset of $\bbR^p$, the solutions are constrained to be similitudes of each other. 
Their results were later refined in some of our own work \cite{arias2017some}.

\subsection{Continuous Setting}
\label{sec:mds continuum}
Following \citet{shepard1966metric}, we consider a limit model in the continuum where the set of items forms an uncountably infinite subset $\cX \subset \bbR^p$.
%We note that the techniques used in \cite{klein, arias2017some} can be applied here to obtain a comparable result in the finite sample setting. 

In the discrete setting, two configurations in dimension $p$, $\{x_1, \dots, x_n\}$ and $\{x'_1, \dots, x'_n\}$, are indistinguishable if it holds that
\begin{equation}
%\label{mds_discrete}
\text{$\|x_i-x_j\| < \|x_i-x_k\| \iff \|x'_i-x'_j\| < \|x'_i-x'_k\|$, \quad for all $i,j,k \in [n]$.}
\end{equation}
In order to transition from the discrete setting to the continuous setting, we consider these configurations as being in correspondence via the function $x_i \mapsto x'_i$ defined on $\{x_1, \dots, x_n\}$. 
When considering equivalent configurations in the continuum, we are thus led to study injective functions $f : \cX \to \bbR^p$ satisfying 
\begin{equation}
\label{mds_continuum}
\text{$\|x-x'\| < \|x-x''\| \iff \|f(x)-f(x')\| < \|f(x)-f(x'')\|$, \quad for all $x, x', x'' \in \cX$.}
\end{equation}
As in \cite{klein}, we say that such a function is {\em weakly isotonic}.
We note that this property is equivalent to 
\begin{equation}
\label{mds_continuum_set}
f(\ball(x, \|x-x'\|) \cap \cX) = \ball(f(x), \|f(x)-f(x')\|)) \cap f(\cX), \quad \forall x,x' \in \cX. 
\end{equation}
We note that \eqref{mds_continuum} implies
\begin{equation}
\label{mds_continuum_equal}
\text{$\|x-x'\| = \|x-x''\| \iff \|f(x)-f(x')\| = \|f(x)-f(x'')\|$, \quad for all $x, x', x'' \in \cX$,}
\end{equation}
which is equivalent to
\begin{equation}\label{mds_continuum_equal_set}
f(\sphere(x, \|x-x'\|) \cap \cX) = \sphere(f(x), \|f(x)-f(x')\|)) \cap f(\cX), \quad \forall x,x' \in \cX.
\end{equation}

We first state the result in the same setting that \citet{shepard1966metric} considers, which crucially assumes that $f$ is a bijection of the entire space. We note that Shepard does that implicitly, even though being bijective is not an immediate consequence of being weakly isotonic.

\begin{proposition}
\label{prp:shepard}
Suppose that $\cX = \bbR^p$. Then any {\em bijective} function $f: \bbR^p \to \bbR^p$ satisfying \eqref{mds_continuum} must be a similarity transformation. 
\end{proposition}

When $f$ is surjective, \eqref{mds_continuum_equal_set} takes the form
\begin{equation}\label{f_shepard}
f(\sphere(x, \|x-x'\|)) = \sphere(f(x), \|f(x)-f(x')\|), \quad \forall x,x' \in \bbR^p.
\end{equation}
This implies that $f$ transforms any sphere into a sphere.
\citet{shepard1966metric} argues from there that ``every sphere-preserving transformation is either a similarity transformation or the product of an inversion (in a sphere) and an isometry", citing\footnote{We note that, strictly speaking, the result is stated for dimension three, and earlier in the same book, for dimension two. These are Theorems~6.71 and~7.71, at least in the 1969 edition of the book.} \cite[p 104]{coxeter1961introduction}.
He then goes on to say that ``The possibility of an inversive transformation can immediately be ruled out, however. It preserves neither the rank order of concentric spheres nor the equality of nonconcentric spheres, whereas both of these invariances are required by the given rank order of the interpoint distances." 

\medskip
We let Shepard's arguments stand on their own, and turn our attention to establishing a more general result using a different approach, following \cite{arias2017some} instead.

\begin{theorem}
\label{thm:mds}
Suppose that $\cX$ has non-empty interior. Then a weakly isotonic function on $\cX$ must coincide on $\cX$ with a similarity transformation.
\end{theorem}

\begin{proof}
%\cite[Prop~7]{klein}, and also \cite[Th~1]{arias2017some}, imply the result when $\cX$ is connected as well. 
First, suppose that $\cX$ is an open ball. Then the result follows from \cite[Prop~7]{klein} or \cite[Th~1]{arias2017some}.
We sketch the arguments following the latter reference. The line of reasoning is as follows: (1) a weakly isotonic function on such a set $\cX$ is continuous; (2) a weakly isotonic function on a convex set preserves midpoints, i.e., satisfies Jensen's functional equation; 
%\begin{equation}\label{jensen}
%g(\tfrac12(x+x')) = \tfrac12 g(x) + \tfrac12 g(x'), \quad \forall x,x'.
%\end{equation}
note that (1) and (2) combined yield that a weakly isotonic function on an open ball is affine (i.e., coincides on its domain with an affine function); and we conclude with (3) an affine function that satisfies \eqref{mds_continuum_equal} on a ball is a similitude (i.e., coincides on its domain with a similarity transformation). 
%We refer the reader to \cite{arias2017some}, and also \cite{klein}, for details.

Now, to prove the theorem as stated, let $\cB$ be any open ball contained in $\cX$ and let $f$ be weakly isotonic on $\cX$.
We know that $f$ coincides on $\cB$ with a similarity transformation. Without loss of generality, we may assume this similarity transformation to be the identity function, in which case $f(x) = x$ for all $x \in \cB$. 
In particular, $f(\cB) = \cB$, and via \eqref{mds_continuum}, we have, for any $x \in \cX$,
\begin{equation}
%\label{internal_point_continuum}
\text{$\|x-x'\| < \|x-x''\| \iff \|f(x)-x'\| < \|f(x)-x''\|$, \quad for all $x', x'' \in \cB$.}
\end{equation}
By \thmref{external_point}, this implies that $f(x) = x$. And this is true for any $x \in \cX$, so that $f$ coincides with the identity function on the entirety of $\cX$.
\end{proof}
%\begin{lemma}\label{lem:jensen}
%Let $\cB$ be an open ball of $\bbR^p$ and suppose $f: \cB \to \bbR^p$ is bounded and satisfies Jensen's functional equation:
%\begin{equation}\label{jensen}
%f(\tfrac12(x+x')) = \tfrac12 f(x) + \tfrac12 f(x'), \quad \forall x,x' \in \cB.
%\end{equation}
%Then $f$ coincides with an affine transformation on $\cB$.
%\end{lemma}
%\begin{proof}
%
%\end{proof}



\section{Internal Unfolding} 
\label{sec:internal}

In {\em internal unfolding}, we have $m$ individuals expressing preferences for $n$ objects, and the goal is to locate the $m$ individuals and the $n$ objects in space. Unlike in external unfolding (\secref{external}), the location of the objects is unknown.
The origins of internal unfolding in the Psychometrics literature date back to \citet{coombs1950psychological} for the point model and to \citet{bennett1956determination} and \citet{tucker1960intra} for the vector model --- although the latter coincides with some variants of factor analysis. 
The problem is also known as `multidimensional unfolding', or just `unfolding'  \cite{borg2005modern}.  

We consider non-metric or ordinal variant of the problem, with a focus on the {\em conditional} setting where each individual ranks the objects in order of preference without first agreeing with other individuals on some ordinal scale. This corresponds to the method of triads. We consider both the point model and the vector model.

It is well-known in the field that the problem is numerically difficult as the iterative method otherwise popular in the metric setting often return degenerate solutions. This is discussed at length in \cite{busing2005avoiding}. See also \cite[Ch 14, 15]{borg2005modern}.


\subsection{Point Model}
\label{sec:internal_point}

\subsubsection{Discrete Setting}
\label{sec:internal point discrete}
The practitioner adopting the point model is confronted with the following problem: 
Given a set of row ranks $(r_{ik} : i \in [m], k \in [n])$, with each $(r_{i1}, \dots, r_{in})$ being a permutation of $(1, \dots, n)$, and a dimension $p \ge 1$, 
\begin{equation}\label{internal point problem}
\begin{gathered}
\text{Find $x_1, \dots, x_m \in \bbR^p$ and $y_1, \dots, y_n \in \bbR^p$ such that $\|x_i-y_k\| < \|x_i-y_l\|$ whenever $r_{ik} < r_{il}$.}
\end{gathered}
\end{equation}
We note that any similarity transformation of a solution --- applied to both the individuals and the objects --- is also a solution.

\citet{bennett1960multidimensional} propose three methods for determining the smallest dimension where an exact embedding can be realized, and in the process offer some elementary observations on things like the number of isotonic regions (aka Voronoi cells).
A follow-up paper by the same authors \cite{hays1961multidimensional} considers extending the basic approach developed by \citet{coombs1950psychological} for the case of dimension $p=1$ to general $p > 1$ by studying the order of individuals when projected onto lines. 
This combinatorial and basic geometrical work was developed further by \citet{davidson1973geometrical, davidson1972geometrical}.
Beyond that, the only theoretical results we are aware of pertain to the study of degenerate solutions \cite{busing2005avoiding, davidson1973geometrical, de1983}.
%(Some theory is developed in \cite{chen2021unfolding} for a binary preference model that also falls under the umbrella name of multidimensional unfolding --- but the model is a very different.)


\subsubsection{Continuous Setting}
\label{sec:internal point continuum}
Again inspired by \citet{shepard1966metric}, we consider a limit model in the continuum where the number of individuals and the number of objects are both infinite, represented by subsets $\cX \subset \bbR^p$ and $\cY \subset \bbR^p$, respectively. 

Staying with the discrete model for a moment, two configurations in dimension $p$, one of them $\{x_1, \dots, x_m; y_1, \dots, y_n\}$ and the other $\{x'_1, \dots, x'_m; y'_1, \dots, y'_n\}$, are indistinguishable if it holds that
\begin{equation}
\label{internal_equivalent_point_discrete}
\text{$\|x_i-y_k\| < \|x_i-y_l\| \iff \|x'_i-y'_k\| < \|x'_i-y'_l\|$, \quad for all $(i,k,l) \in [m] \times [n] \times [n]$.}
\end{equation}
In preparation to pass to the continuum, we regard these configurations as being in correspondence via the pair of functions $x_i \mapsto x'_i$ and $y_k \mapsto y'_k$, defined on $\{x_1, \dots, x_m\}$ and $\{y_1, \dots, y_n\}$, respectively. 
When considering equivalent configurations in the continuum, we are thus led to study pairs of injective functions $f : \cX \to \bbR^p$ and $g: \cY \to \bbR^p$ satisfying 
\begin{equation}
\label{internal_point_continuum}
\text{$\|x-y\| < \|x-y'\| \iff \|f(x)-g(y)\| < \|f(x)-g(y')\|$, \quad for all $(x, y, y') \in \cX \times \cY \times \cY$.}
\end{equation}
This is equivalent to 
\begin{equation}
\label{internal_point_continuum_set}
g(\ball(x, \|x-y\|) \cap \cY) = \ball(f(x), \|f(x)-g(y)\|)) \cap g(\cY), \quad \text{for all }(x,y) \in \cX \times \cY. 
\end{equation}
We note that \eqref{internal_point_continuum} implies
\begin{equation}
\label{internal_point_continuum_equal}
\text{$\|x-y\| = \|x-y'\| \iff \|f(x)-g(y)\| = \|f(x)-g(y')\|$, \quad for all $(x, y, y') \in \cX \times \cY \times \cY$,}
\end{equation}
which is equivalent to
\begin{equation}\label{internal_point_continuum_equal_set}
g(\sphere(x, \|x-y\|) \cap \cY) = \sphere(f(x), \|f(x)-g(y)\|)) \cap g(\cY), \quad \forall (x,y) \in \cX \times \cY.
\end{equation}


We first establish the result in the setting that we believe \citet{shepard1966metric} would have considered.
%Note that \eqref{g_shepard} below is stronger than \eqref{internal_point_continuum_equal_set}, and while the latter follows from \eqref{internal_point_continuum}, it is not immediately clear that the former does. 

\begin{proposition}
\label{prp:shepard_internal}
In the situation where $\cX = \cY = \bbR^p$, consider any pair of injective functions $(f,g)$ satisfying \eqref{internal_point_continuum} such that $g(\bbR^p) = \bbR^p$. Then, it must be the case that $f = g = L$ for some similarity transformation $L$. 
\end{proposition}

With the additional assumption that $g(\bbR^p) = \bbR^p$, meaning that $g$ is not only injective but also surjective, \eqref{internal_point_continuum_equal_set} becomes
\begin{equation}\label{g_shepard}
g(\sphere(x, \|x-y\|)) = \sphere(f(x), \|f(x)-g(y)\|)), \quad \forall (x,y) \in \bbR^p \times \bbR^p.
\end{equation}

\begin{lemma} \label{lem:fffggg}
Consider any pair of injective functions $(f,g)$ satisfying \eqref{g_shepard}. If $x, x', x'' \in \cX$ and $y, y', y'' \in \cY$ are all collinear, then so are $f(x), f(x'), f(x''), g(y), g(y'), g(y'')$.
\end{lemma}

\begin{proof}
Assume without loss of generality that the points are all distinct. 

We first show that $f(x), f(x'), g(y)$ are collinear.
Indeed, since $x,x',y$ are collinear, and we just assumed that $x \ne x'$, it must be that $y$ is the only point at the intersection of $\cS := \sphere(x, \|x-y\|)$ and $\cS' := \sphere(x', \|x'-y\|)$.
Now, by \eqref{g_shepard}, 
\begin{align}
g(\cS) = \sphere(f(x), \|f(x)-g(y)\|), && g(\cS') = \sphere(f(x'), \|f(x')-g(y)\|),
\end{align}
and, by the fact that $g$ is injective, these two spheres only have one point in common, $g(y)$, and so they must be tangent as well. This then implies that their centers, $f(x)$ and $f(x')$, are collinear with their point of contact, $g(y)$. 

By the same token, $f(x), f(x'), g(y')$, $f(x), f(x'), g(y'')$, and also $f(x), f(x''), g(y)$, must be collinear. And from all this, we are able to conclude.
\end{proof}

\begin{proof}[Proof of \prpref{shepard_internal}]
It suffices to show that $f$ and $g$ coincide, as we can then deduce from \eqref{internal_point_continuum} that $f$ is weakly isotonic, and is therefore a similarity transformation via \thmref{mds}. 

Take any $z_0$. We want to show that $f(z_0) = g(z_0)$.
Consider $z_1,z'_1, z_2,z'_2$ on some sphere $\cS$ centered at $z_0$ such that $(z_1z'_1)$ and $ (z_2z'_2)$ intersect at $z_0$, so that $[z_1z'_1]$ and $[z_2z'_2]$ are diameters of the sphere. 
By \eqref{g_shepard}, $g(\cS)$ is a sphere centered at $f(z_0)$ and passing through $g(z_1),g(z'_1),g(z_2),g(z'_2)$. 
And, by \lemref{fffggg}, $f(z_0),g(z_1),g(z'_1)$ are collinear, and so are $f(z_0),g(z_2),g(z'_2)$, implying that $[g(z_1)g(z'_1)]$ and $[g(z_2)g(z'_2)]$ are diameters of $g(\cS)$. By the fact that $g$ is injective, we have that $g(z_1),g(z'_1),g(z_2),g(z'_2)$ are distinct, so that $f(z_0)$ is the only point at the intersection of the lines $(g(z_1)g(z'_1))$ and $(g(z_2)g(z'_2))$. However, \lemref{fffggg} also gives that $g(z_0),g(z_1),g(z'_1)$ are collinear, and that $g(z_0),g(z_2),g(z'_2)$ are collinear, implying in the same way that $g(z_0)$ is also at the intersection of these two lines, forcing $g(z_0) = f(z_0)$.
%We have thus established that $f = g$, everywhere. 
\end{proof}

We now state our result for the setting that we consider. Importantly, we do not assume that \eqref{g_shepard} holds, and the situation becomes substantially more complicated. 
We are able to establish the following result. 

\begin{theorem} \label{thm:internal_point}
Suppose that either $\cX = \bbR^p$ and $\cY$ has non-empty interior, or that $\cX$ has non-empty interior and $\cY = \bbR^p$, and consider any pair of injective functions $(f,g)$ satisfying \eqref{internal_point_continuum}. Then, there is a similarity $L$ such that $f = L$ on $\cX$ and $g = L$ on $\cY$. 
\end{theorem}

The proof occupies the rest of the section. Until the end of the proof, $(f,g)$ denotes a pair of injective functions satisfying \eqref{internal_point_continuum}.

\begin{proposition}
Suppose that on an open ball contained in $\cX \cap \cY$, $f = g = L$ for some similarity transformation $L$. Then $f = L$ on the entirety of $\cX$, and if either $\cX = \bbR^p$ or $\cY = \bbR^p$, then $g = L$ on the entirety of $\cY$.
\end{proposition}
 
\begin{proof}
Let $\cB$ denote such a ball and assume without loss of generality that $L = {\rm id}$ so that $f(x) = x$ for all $x \in \cB$ and $g(y) = y$ for all $y \in \cB$. 
We want to show that this extends to all $x\in\cX$ and all $y\in\cY$, respectively.

Take any $x\in\cX$. 
By \eqref{internal_point_continuum_equal} and the assumptions made here,  
\[\|x-y\| = \|x-y'\| \iff \|f(x)-y\| = \|f(x)-y'\|, \quad \forall y,y' \in \cB.\]
In words, $x$ and $f(x)$ are equivalent with respect to $\cB$. Since $\cB$ is open, we may apply \thmref{external_point} to obtain that $x$ and $f(x)$ coincide. Therefore, $f(x) = x$ for all $x \in \cX$.

To continue, assume without loss of generality that $\cB$ is the unit ball. 
Take any $y \in \cY$ not in $\cB$, for otherwise we already know that $g(y) = y$. 
First, assume that $\cX = \bbR^p$.
Let $u := y/\|y\|$ and define $y_\pm := \pm \frac12 u$, and then $x_\pm := \frac12(y+y_\pm)$. Because $y_\pm \in \cB$, we have $g(y_\pm) = y_\pm$, and by construction, $\|x_\pm - y\| = \|x_\pm - y_\pm\| =: r_\pm$, so that an application of  \eqref{internal_point_continuum_equal} gives
$\|x_\pm - g(y)\| = r_\pm$. This means that $g(y) \in \ball(x_-, r_-) \cap \ball(x_+, r_+)$. But, by construction, $\ball(x_-, r_-) \cap \ball(x_+, r_+) = \{y\}$, forcing $g(y) = y$.

Now, assume that $\cY = \bbR^p$. It suffices to show that $g(y) = y$ for all $y \in \cB_m := \ball(0, m)$ for all $m \ge 1$ integer, and we do that by induction. We can readily start the induction since this is true for $m = 1$ by assumption. Assume that this holds for some given $m \ge 1$ and take $y \in \cB_{m+1}\setminus \cB_m$.  
Let $u := y/\|y\|$, so that $y = (m+a) u$ for some $0 \le a < 1$.
Define $y_j := - (m-a/j) u$ for $j = 1,2$, and note that $y_j \in \cB_m$, so that $g(y_j) = y_j$. Let $x_j := \frac12(y+y_j)$, and note that $x_j \in \cB$ and also $\|x_j - y\| = \|x_j - y_j\| =: r_j$. The arguments are now the same. Indeed, applying \eqref{internal_point_continuum_equal} gives
$\|x_j - g(y)\| = r_j$, so that $g(y) \in \ball(x_1, r_1) \cap \ball(x_2, r_2)$. But, by construction, $\ball(x_1, r_1) \cap \ball(x_2, r_2) = \{y\}$, forcing $g(y) = y$.
\end{proof} 

In view of this proposition, it suffices to prove the theorem in a situation where $\cX = \cY$ is an open ball, which we denote $\cB$ henceforth. Unless specified otherwise, all the points belong to $\cB$.

For a subset $\cS \subset \bbR^p$ we define $\dim \cS$ to be the dimension of the affine space spanned by $\cS$, denoted $\Span \cS$. We denote by $\dir \cS$ the direction of $\Span\cS$ and by $\Vect \cS$ the vector space spanned by $\cS$. In particular, for any $s \in \cS$, we have $\dir \cS = \Vect(\cS-s)$ and $\Span\cS = s + \dir\cS$. We say that two subsets $\cS$ and $\cS'$ are parallel if $\dir \cS \subset \dir \cS'$ or $\dir \cS' \subset \dir \cS$.

We already saw that any pair of functions $(f,g)$ satisfying \eqref{internal_point_continuum} also satisfies \eqref{internal_point_continuum_set}, which here takes the form 
\begin{equation} \label{easy-iii}
g(\ball(x,\|y-x\|)\cap \cB) = \ball(f(x),\|g(y) - f(x)\|) \cap g(\cB), \quad \forall x,y \in \cB.
\end{equation}
It is also the case that 
\begin{equation} \label{easy-i}
f(H^+(y,y') \cap \cB) = H^+(g(y),g(y')) \cap f(\cB), \quad \forall y \ne y' \in \cB;
\end{equation}
\begin{equation} \label{easy-ii}
f(H(y,y') \cap \cB) = H(g(y),g(y')) \cap f(\cB), \quad \forall y \ne y' \in \cB.
\end{equation}
Recall that $H(y,y')$ is the hyperplane going through the midpoint of, and orthogonal to the line segment $[yy']$, while $H^+(y,y')$ is the half-space with boundary $H(y,y')$ and containing $y$.
%We also saw in \thmref{external_point} that, if $\cY$ has non-empty interior, $f$ is injective. 
%In addition, $g$ is injective on $(\dir \cX) \cap \cY$. To see this, assume $g(y) = g(y')$ for some $y, y' \in \cY$. Then according to \eqref{internal_point_continuum}, $\|y'-x\| = \|y-x\|$ for all $x \in \cX$, implying $
%\|y'\|^2-\|y\|^2 = 2\inner{x}{y'-y}$ for all $x \in \cX$, in turn implying that $\inner{z}{y'-y} = 0$ for all $z \in \dir \cX$.

\begin{lemma} \label{lem:fdimb} 
For any subset $\cS \subset \cB$, $\dim f(\cS) = \dim \cS$.
\end{lemma}


\begin{proof} 
We first prove that $\dim f(\cS) \geq \dim \cS$. 
Let $k=\dim \cS$ and let $x_1,\dots,x_{k+1}$ be affinely independent points of $\cS$. Assume that $\dim \{f(x_1),\dots,f(x_{k+1})\} < k$. Because the Vapnik--Chervonenkis dimension of affine hyperplanes in $\bbR^k$ is exactly $k+1$, the set $ \{f(x_1),\dots,f(x_{k+1})\}$ cannot be shattered, and we can find a subset $I \subset [k+1]$ such that $(f(x_i))_{i\in I}$ cannot be separated from $(f(x_i))_{i\in I^c}$. But because $x_1,\dots,x_{k+1}$ is affinely independent, it is shattered by affine hyperplanes and it exists $\cH$ that separates $(x_i)_{i\in I}$ from $(x_i)_{i \in I^c}$. Now we can find $y,y' \in \cB$ such that $\cH = H(y,y')$: indeed, it must be the case that both $I$ and $I^c$ are not empty, thus $\cH$ operates a non-trivial separation of $\{x_1,\dots,x_{k+1}\}$ and so intersects $\cB$; since $\cB$ is open, it is then easy to find two such points $y,y' \in \cB$. Now it is straight-forward to see that $H(g(y),g(y'))$ also separates $(f(x_i))_{i\in I}$ from $(f(x_i))_{i \in I^c}$ through \eqref{easy-i}, leading to a contradiction. Thus $\dim f(\cS) \geq \dim \{f(x_1),\dots,f(x_{k+1})\} \geq k = \dim \cS$.

We now prove that $\dim f(\cS) \leq \dim \cS$. We do so by descending induction on $\dim \cS$.
When $\dim\cS = p - 1$, then there exists distinct $y,y' \in \cB$ such that $\cS \subset H(y,y')$ and \eqref{easy-i} yields that $f(\cS) \subset H(g(y),g(y'))$ hence $\dim f(\cS) \leq p-1$ by injectivity of $g$. 
Now by induction, if $\dim \cS \leq p-2$, then at least $\dim f(\cS) \leq p - 1$. Now assume that $\dim f(\cS \cup \{x\}) = \dim f(\cS)$  for all $x \notin \Span \cS$. Then, according to the first part of the proof,
$$
\dim f(\cS) = \dim f(\cS \cup (\cB \setminus \Span \cS)) \geq \dim(\cB \setminus \Span \cS) = p,
$$
which is absurd. Therefore, there exists $x \notin \Span \cS$ such that $\dim f(\cS \cup \{x\}) = \dim f(\cS) + 1$. By induction, we get
$$
\dim f(\cS) = \dim f(\cS \cup \{x\}) - 1 \leq \dim (\cS \cup \{x\}) - 1 = \dim \cS.
$$
which ends the proof.
\end{proof}

\begin{lemma} \label{lem:rect} 
Let $\cR$ be any $2^p$-tuple of $\cB$ forming a hyperrectangle. Then $g(\cR)$ is also a $2^p$-tuple forming a hyperrectangle in the same configuration as $\cR$.
\end{lemma}

\begin{proof} 
It suffices to establish that, for any point set $y_1, y_2, y_3, y_4 \in \cB$ that forms a rectangle, the point set $g(y_1), g(y_2), g(y_3), g(y_4)$ also forms a rectangle. 

Assume without loss of generality that $y_1-y_2 = y_3 - y_4$.
In that case, we have $H(y_1,y_2) = H(y_3,y_4)$.
Then, by \eqref{easy-ii}, $H(g(y_1),g(y_2))$ and $H(g(y_3),g(y_4))$ both contain $f(H(y_1,y_2) \cap \cB)$, and because that subset has dimension $p-1$ by \lemref{fdimb}, it must be that $H(g(y_1),g(y_2)) = H(g(y_3),g(y_4)) =: \cH$.
In particular, $v_{12} := g(y_1)-g(y_2)$ is parallel to $v_{34} := g(y_3)-g(y_4)$. Similarly, since $y_1 - y_3 = y_2 - y_4$, we also have that $v_{13} :=  g(y_1)-g(y_3)$ is parallel to $v_{24} := g(y_2)-g(y_4)$. 
Notice that 
\[\tfrac12 (v_{13} + v_{24}) = \tfrac12 (g(y_1)+g(y_2)) - \tfrac12 (g(y_3)+g(y_4)) \in \cH - \cH \subset \dir \cH,\]
which by parallelism of $v_{13}$ and $v_{24}$ can only be true if both $v_{13}, v_{24} \in \dir \cH$. 
In particular, $v_{13}$ and $v_{24}$ are perpendicular to $v_{12}$ and $v_{34}$.
This proves that $g(y_1),g(y_2),g(y_3),g(y_4)$ forms a rectangle in the same configuration than $y_1, y_2, y_3, y_4$. As a result, $g(\cR)$ is a hyperrectangle in the same configuration as $\cR$.
\end{proof}

As an immediate corollary, we get the following.
\begin{corollary} \label{cor:gdimb} $g(\cB)$ has affine dimension $p$.
\end{corollary}

\begin{lemma} \label{lem:glineb} Let $\cL$ be a line intersecting $\cB$. Then $g(\cL \cap \cB)$ is contained in a line. Furthermore, if $\cL'$ is another line intersecting $\cB$ parallel to $\cL$, then $g(\cL'\cap \cB)$ is parallel to $g(\cL\cap\cB)$.
\end{lemma}

\begin{proof} Let $x,x'$ and $x''$ three points of $\cL \cap \cB$. Then we can construct two hyperrectangles $\cR$ and $\cR'$ of $\cB$ with a common facet and such that $[xx']$ and $[x'x'']$ are two edges of $\cR$ and $\cR'$ orthogonal to that common facet. Since $g(\cR)$ and $g(\cR')$ are two hyperrectangles in the same configuration as $\cR$ and $\cR'$, theyx also share a common facet, and $[g(x)g(x')]$ and $[g(x')g(x'')]$ must be orthogonal to that common hyperfacet. They are thus parallel, so that $g(x), g(x')$ and $g(x'')$ are colinear.

For the second part of the proof, we can build a third hyperrectangle $\cR''$ of $\cB$ which contains two edges that are supported on $\cL$ and $\cL'$. Since $g(\cR'')$ is again a hyperrectangle, the images of these edges are parallel, and so must be $g(\cL\cap\cB)$ and $g(\cL'\cap\cB)$.
\end{proof}


\begin{lemma} \label{lem:addb} 
Assume that $0 \in \cB$ and that $g(0) = 0$.
It holds that $g(y_0+y_1) = g(y_0)+g(y_1)$ for all $y_0,y_1\in \cB$ such that $0 \notin (y_0y_1)$ and $y_0+y_1 \in \cB$.
\end{lemma}
\begin{proof} 
Take $y_0 \in \cB$ and $\cL_0 = \Vect(y_0)$. Then, by \lemref{glineb}, $g(\cL_0 \cap \cB) \subset \Vect(g(y_0))$. Now let $y_1 \in \cB$ such that $0 \notin (y_0y_1)$ and such that $y_0+y \in \cB$, and denote $\cL_1 = \Vect(y_1)$. Since $g$ conserves parallelism also by \lemref{glineb}, $g(\{y_1+\cL_0\} \cap \cB)$ is contained in a line with direction $\Vect(g(y_0))$, so that $g(\{y_1+\cL_0\}\cap \cB) \subset g(y_1) + \Vect(g(y_0))$.
Similarly, $g(\{y_0+\cL_1\}\cap \cB) \subset g(y_0) + \Vect(g(y_1))$. 
Now, since $y_0+y_1$ is the intersection of $y_0+\cL_1$ and $y_1+\cL_0$, there holds
\begin{align*} 
g(y_0+y_1) &\in g\(\{y_1+\cL_0\} \cap \cB\) ~ \bigcap~ g\(\{y_0+\cL_1\}\cap\cB\) \\
&\subset \{g(y_1) + \Vect(g(y_0))\}\ \bigcap\ \{g(y_0) + \Vect(g(y_1))\} = \{g(y_0) + g(y_1)\}. \qedhere
\end{align*} 
\end{proof}

\begin{lemma} \label{lem:multb} 
Assume that $0 \in \cB$ and that $g(0) = 0$.
It holds that $g(-y) = -g(y)$ for all $y\in\cB$.
\end{lemma}
\begin{proof} 
Let $y_0 \in \cB \setminus \{0\}$ and let $y_1 \in \cB \setminus \Vect(y_0)$. Then there exists (a small) $z$ in $\cB$ that is not in $\Vect(y_0)$ or $\Vect(y_1)$ and such that $y_0 \pm z$ and $y_1 \pm z$ are in $\cB$. Then, thanks to \lemref{addb}, 
\begin{align}
g(y_0)+g(-y_0) 
&= g(z+y_0)+g(z-y_0)-2g(z) \\
&= g(2z)-2g(z) \\
&= g(z+y_1)+g(z-y_1)-2g(z) \\
&= g(y_1) + g(-y_1),
\end{align}
so that $g(y_0)+g(-y_0) \in  \Vect g(y_0) \cap \Vect g(y_1)$. This last intersection is $\{0\}$ by injectivity of $g$.
\end{proof}

\begin{lemma} \label{lem:fgb} 
Assume that $0 \in \cB$ and that $g(0) = 0$.
It holds that $f(0) = 0$. 
\end{lemma}
\begin{proof} 
Let $x \in \cB \setminus \{0\}$. 
By \eqref{easy-ii}, $f(H(x,-x) \cap \cB) \subset H(g(x),-g(x))$, and by \lemref{multb}, $H(g(x),-g(x)) = \Vect g(x)^\perp$. Thus, 
\[
f(0) \in \bigcap_{x\in \cB} \Vect g(x)^\perp = g(\cB)^\perp = \{0\},
\]
because $\dim g(\cB) = p$ thanks to Corollary \ref{cor:gdimb}. 
\end{proof}


\begin{proof}[Proof of \thmref{internal_point}]
Fix an arbitrary $x_0 \in \cB$ and define $g_0(x) := g(x+x_0)-g(x_0)$ and $f_0(x) := f(x+x_0)-g(x_0)$. Then $g_0$ and $f_0$ satisfy \eqref{internal_point_continuum}, $0 \in \cB - x_0$ and $g_0(0) = 0$. Therefore, thanks to \lemref{fgb}, $f_0(0) = 0$, and hence $f(x_0) = g(x_0)$. 
We have thus established that $f = g$ on $\cB$. Furthermore, combined with \eqref{internal_point_continuum}, we deduce that $g$ is weakly isotonic on $\cB$,  and must thus coincide with a similarity transformation on $\cB$ by way of \thmref{mds}. 
\end{proof}

We initially thought that we could work in \thmref{internal_point} under more general conditions on $\cX$ and $\cY$. The result might hold, for example, if $\cX$ and $\cY$ are open and have a non-empty intersection. We do not know whether this is the case or not. It is not even clear to us whether requiring that $\cX$ and $\cY$ intersect in that case is necessary or not. 
We note, however, that it is not sufficient that $\cX \cap \cY$ have non-empty interior --- a condition that would be in line with what was assumed in \thmref{external_point} and \thmref{mds}. Indeed, consider a situation where $\cX$ is the unit ball (open or closed) and $\cY = \cX \cup \cA$, $\cA := \{a u : a \ge 1\}$, with $u$ an arbitrary normed vector. In that case, $f = {\rm id}$ on $\cX$ and $g = {\rm id}$ on $\cX$ and increasing along $\cA$ with $g(\cA) \subset \cA$, satisfies \eqref{internal_point_continuum}.



\subsection{Vector Model}
\label{sec:internal_vector}

\subsubsection{Discrete Setting}
\label{sec:internal vector discrete}

In practical settings, the problem in the vector model variant of the problem is as follows: 
Given a set of row ranks $(r_{ik} : i \in [m], k \in [n])$, with each $(r_{i1}, \dots, r_{in})$ being a permutation of $(1, \dots, n)$, and a dimension $p \ge 1$, 
\begin{equation}
\begin{gathered}
\text{Find $x_1, \dots, x_m \in \bbS^{p-1}$ and $y_1, \dots, y_n \in \bbR^p$ such that $\<x_i, y_k\> < \<x_i, y_l\>$ whenever $r_{ik} > r_{il}$.}
\end{gathered}
\end{equation}
(Note that change in the direction of the second inequality. We adopted this convention so that the ranks have the same meaning as they do in \eqref{internal point problem}.)
It is also the case here that any similarity transformation of a solution is also a solution. In addition, there is one more degree of freedom that comes from arbitrarily scaling the objects.

In his pioneering study, \citet{bennett1956determination} provides some basic combinatorial insights into the problem, later enriched by \citet{davidson1973geometrical}.
Much closer to our contribution here, 
\citet{shepard1966metric} discusses the theoretical foundations of this problem in a continuous model, which we detail in the next subsection.


\subsubsection{Continuous Setting}
\label{sec:internal vector continuum}

In the continuous model inspired by \citet{shepard1966metric}, we consider a subset of individuals, $\cX \subset \bbS^{p-1}$, and a subset of objects, $\cY \subset \bbR^p$.  

In the discrete setting, two configurations in dimension $p$ (with $p \ge 2$, as before), one being $\{x_1, \dots, x_m; y_1, \dots, y_n\}$ and the other being $\{x'_1, \dots, x'_m; y'_1, \dots, y'_n\}$, are indistinguishable when
\begin{equation}
\label{internal_equivalent_vector_discrete}
\text{$\<x_i, y_k\> < \<x_i, y_l\> \iff \<x'_i,y'_k\> < \<x'_i,y'_l\>$, \quad for all $(i,k,l) \in [m] \times [n] \times [n]$.}
\end{equation}
As we did for the point model, we see these configurations as being in correspondence via $x_i \mapsto x'_i$ and $y_k \mapsto y'_k$. 
When passing to the continuum, we end up considering pairs of injective functions $f : \cX \to \bbS^{p-1}$ and $g: \cY \to \bbR^p$ satisfying 
\begin{equation}
\label{internal_equivalent_vector_continuum}
\text{$\<x,y\> < \<x,y'\> \iff \<f(x),g(y)\> < \<f(x),g(y')\>$, \quad for all $(x, y, y') \in \cX \times \cY \times \cY$.}
\end{equation}

\begin{theorem}
\label{thm:internal_vector}
Suppose that $\cX$ contains $p+1$ vectors in general position, and that $\cY$ is open and connected. Then any pair of injective functions $(f,g)$ satisfying \eqref{internal_equivalent_vector_continuum} must be of the form $f(x) = L^{-\top} x/\|L^{-\top} x\|$ and $g(y) = Ly + \tau$, for some invertible linear function $L$ and a vector $\tau \in \bbR^p$. 
\end{theorem}

Recall that a set of vectors is in general position if any $q$ of them are linearly independent for any $q \le p$. 
We note that \eqref{internal_equivalent_vector_continuum} implies
\begin{equation}
\label{internal_equivalent_vector_continuum_equal}
\text{$\<x,y\> = \<x,y'\> \iff \<f(x),g(y)\> = \<f(x),g(y')\>$, \quad for all $(x, y, y') \in \cX \times \cY \times \cY$.}
\end{equation}
We will use this identity in the proof below.
The following two lemmas operate under the conditions of \thmref{internal_vector}.

\begin{lemma}\label{lem:f}
If $x_1, \dots, x_q$ are linearly independent, so are $f(x_1), \dots, f(x_q)$. 
%Thus, $f(\cU)$ spans $\bbR^p$.
\end{lemma}

\begin{proof}
Take $x_1, \dots, x_q$ are linearly independent, and suppose for contradiction that $f(x_1), \dots, f(x_q)$ are not linearly independent, e.g., suppose $f(x_q)$ is a linear combination of $f(x_1), \dots, f(x_{q-1})$. Because $\cY$ has non-empty interior, there is $y,y' \in \cY$ such that $\<x_i, y-y'\> = 0$ for $i = 1, \dots, q-1$ and $\<x_q, y-y'\> \ne 0$. 
Since $\<x_i, y\> = \<x_i, y'\> $ for $i = 1, \dots, q-1$, we have via \eqref{internal_equivalent_vector_continuum_equal} that $\<f(x_i), g(y)\> = \<f(x_i), g(y')\>$ for $i = 1, \dots, q-1$. Since $f(x_q)$ is generated by the other $f(x_i)$, it must be the case that $\<f(x_q), g(y)\> = \<f(x_q), g(y')\>$.
However, $\<x_q, y\> \ne \<x_q, y'\>$, since $\<x_q, y-y'\> \ne 0$, and this implies again via \eqref{internal_equivalent_vector_continuum_equal} that $\<f(x_q), g(y)\> \ne \<f(x_q), g(y')\>$ --- a contradiction.
\end{proof}

%\begin{lemma}\label{lem:g_injective}
%Under the conditions of \thmref{internal_vector}, $g$ must be injective.
%\end{lemma}
%
%\begin{proof}
%Take $y \ne y'$ in $\cY$. Because ${\rm span}\, \cX = \bbR^p$, there is $x \in \cX$ such that $\<x,y\> \ne \<x,y'\>$, which then implies via \eqref{internal_equivalent_vector_continuum_equal} that $\<f(x),g(y)\> \ne \<f(x),g(y')\>$, which then forces $g(y) \ne g(y')$. 
%\end{proof}

\begin{lemma}
\label{lem:g_affine}
If $g$ locally affine, then the conclusions of \thmref{internal_vector} hold.
\end{lemma}

\begin{proof}
We first note that a function that is locally affine on an open connected set must be affine on the entire set --- see, e.g., \cite[Lem 8]{arias2017some}.
Therefore, we must have that $g(y) = L y + \tau$ on the whole of $\cY$, for some linear map $L$ and some translation vector $\tau$, and by the fact that $g$ is injective, $L$ must be invertible. 
We assume that $\tau = 0$ without loss of generality.

%Let $\cY_0$ be a connected component of $\cY$. Note that $\cY_0$ itself is open. 
Define $\tilde f(x) = f_L(x)/\|f_L(x)\|$ where $f_L(x) := L^{\top} f(x)$. Then, $\<f(x), g(y)\> = \<f_L(x), y\>$ and $\sign \<f_L(x), y\> = \sign \<\tilde f(x), y\>$ on $\cX \times \cB$. 
Therefore, for any $x\in \cX$, by \eqref{internal_equivalent_vector_continuum}, we have
\begin{equation}
\text{$\<x,y\> < \<x,y'\> \iff \<\tilde f(x), y\> < \<\tilde f(x), y'\>$, \quad for all $y, y' \in \cY$.}
\end{equation}
By \thmref{external_vector}, it must be the case that $\tilde f(x) = x$. This, and the fact that $f : \cX \to \bbS^{p-1}$, then implies that $f(x) = L^{-\top} x/\|L^{-\top} x\|$. And this is valid for all $x \in \cX$.
\end{proof}

\begin{proof}[Proof of \thmref{internal_vector}]
By \lemref{g_affine}, it suffices to show that $g$ is locally affine.

Let $x_0, \dots, x_p \in \cX$ be as in the statement. In particular, $x_1, \dots, x_p$ form a basis of the whole space $\bbR^p$, and $x_0$ does not belong to any hyperplane generated by any $p-1$ elements of that basis. 
By applying to $\cX$ and $\cY$ the orthogonal transformation that sends $x_i$ to $e_i$, the $i$-th canonical vector of $\bbR^p$, we may assume without loss of generality that $x_1, \dots, x_p$ form the canonical basis of $\bbR^p$. That being done, the condition on $x_0$ is that none of its coordinates are zero: We write $x_0 = (a_1, \dots, a_p)$ in the canonical basis. 

In view of \lemref{f}, it is also true that $f(x_0), \dots, f(x_p)$ are in general position. By applying to $f(\cX)$ and $g(\cY)$ the orthogonal transformation that sends $f(x_i)$ to $e_i$, we may also assume without loss of generality that $f(x_1), \dots, f(x_p)$ form the canonical basis of $\bbR^p$. 
Having done that, we have $f(x_i) = x_i$ (or $f(e_i) = e_i$) for $i = 1, \dots, p$.
And it is also true that $f(x_0)$ has none of its coordinates equal to zero: We write $f(x_0) = (b_1, \dots, b_p)$ in the canonical basis.  

%\begin{claim}\label{cla:g_univariate}
%We have $g(u_{1}, \dots, u_{p}) = (g_1(u_{1}), \dots, g_p(u_{p}))$ for increasing functions $g_1, \dots, g_p$.
%\end{claim}
Write $g(y) = (g_1(y), \dots, g_p(y))$.
Applying \eqref{internal_equivalent_vector_continuum_equal} with $x = e_m$, and writing $y = (u_{1}, \dots, u_{p})$ and $y' = (u'_{1}, \dots, u'_{p})$, we get
\begin{equation}
u_{m} = u'_{m} \iff g_m(y) = g_m(y'),
\end{equation}
implying that $g_m$ is only a function of the $m$-th coordinate. This allows us to rewrite $g(y) = (g_1(u_{1}), \dots, g_p(u_{p}))$. Then, applying \eqref{internal_equivalent_vector_continuum}, we get
\begin{equation}
u_{m} < u'_{m} \iff g_m(u_{m}) < g_m(u'_{m}),
\end{equation}
implying that $g_m$ is increasing.
%\qedclaim

%\begin{claim}\label{cla:g_local}
%$g$ is locally affine, with linear part a scalar multiple of $L :=\diag(a_1/b_1, \dots, a_p/b_p)$.
%\end{claim}
We want to show that $g$ is affine in some neighborhood of an arbitrary point $y_0 \in \cY$. By applying an appropriate translation and scaling to $\cY$, we may assume without loss of generality that $y_0$ is the origin ($y_0 = 0$) and that $\cY$ contains the hypercube $[-1,1]^p$. And by applying an appropriate translation to $g(\cY)$, we may also assume that $g(0) = 0$.
%
We now show that each $g_m$ is linear in a neighborhood of the origin. We do so for $m=1$.
%This is where make use of $x_0$ and $f(x_0)$.
Applying \eqref{internal_equivalent_vector_continuum_equal} with $x = x_0$, we have
\begin{equation}
\sum_{m=1}^p a_m u_m = \sum_{m=1}^p a_m u'_m
\iff \sum_{m=1}^p b_m g_m(u_m) = \sum_{m=1}^p b_m g_m(u'_m).
\end{equation}
Taking $u_3 = \cdots = u_p = 0$ and $u'_2 = \cdots = u'_p = 0$, we obtain the identity
\begin{equation}
a_1 u_1 + a_2 u_2 = a_1 u'_1
\iff b_1 g_1(u_1) + b_2 g_2(u_2) = b_1 g_1(u'_1),
\end{equation}
%With $u'_1 = 0$ (note that this choice is possible), and using the fact that $g(0) = 0$ so that $g_1(0) = 0$, we get
%\begin{equation}
%a_1 u_1 + a_2 u_2 = 0
%\iff b_1 g_1(u_1) + b_2 g_2(u_2) = 0.
%\end{equation}
which in turn implies
\begin{equation}
g_1(u_1 + \tfrac{a_2}{a_1} u_2)
= g_1(u_1) + \tfrac{b_2}{b_1} g_2(u_2).
\end{equation}
This is valid for $u_1, u_2 \in [-1,1]$ such that $|u_1 + \tfrac{a_2}{a_1} u_2| \le 1$.
Applying this with $u_1 = 0$, using the fact that $g_1(0) = 0$, we get
\begin{equation}
\label{g1g2}
g_1(\tfrac{a_2}{a_1} u_2)
= \tfrac{b_2}{b_1} g_2(u_2),
\end{equation}
whenever $|\tfrac{a_2}{a_1} u_2| \le 1$, and re-injecting that into the previous identity, and changing variables, we get
\begin{equation}
g_1(u_1 + v_1)
= g_1(u_1) + g_1(v_1),
\end{equation}
when $|u_1| \le 1$, $|v_1| \le \min(1, \tfrac{a_2}{a_1})$, and $|u_1+v_1| \le 1$.
Therefore, $g_1$ is additive in a neighborhood of the origin.
Although additive functions are not linear in general, it is a known fact that a monotone additive function on the real line is linear \cite[Ex 15 L-M]{bartle1964elements}. It is true that we have only shown that $g_1$ is additive in a neighborhood of the origin, not the entire real line, but the usual arguments apply to show that it is linear in the same neighborhood.\footnote{See, for example, \url{https://math.stackexchange.com/a/1143841}}  
\end{proof}

We note that the condition on $\cX$ in \thmref{internal_vector} cannot be weakened. To see this, consider the example situation where $\cX = \{e_1, \dots, e_p\}$, $\cY = \bbR^p$, $f(e_i) = e_i$ for all $i$, and $g(u_1, \dots, u_p) = (g_1(u_1), \dots, g_p(u_p))$ where each $g_m$ is increasing on $\bbR$.  
The assumption that $\cY$ is connected is also necessary. To see this, consider the example situation where $\cX = \{e_0, e_1, \dots, e_p\}$ with $e_0 \propto (1, \dots, 1)$, $\cY = \cY_0 \cup \cY_1$ with $\cY_0 := \ball(0, 1)$ and $\cY_1 := \ball(3e_0, 1)$, $f(e_i) = e_i$ for all $i$, and $g(y) = y + \tau_s$ on $\cY_s$, with $\tau_0 < \tau_1$ coordinate-wise.

%\begin{claim}\label{cla:g_collinear}
%$g$ preserves collinearity in any direction orthogonal to an element in the interior of $\cX$.
%\end{claim}
%Let $L$ be a line whose direction is orthogonal to some $x_0 \in \cX^\circ$, and take $y \ne y'$ in $L \cap \cY$. Let $M = (g(y) g(y'))$, which is indeed a line because $g(y) \ne g(y')$ by \claref{g_injective}.
%Let $\cU \subset \bbS^{p-1}$ be open such that $x_0 \in \cU \subset \cX$. Let $H$ denote the (linear, not just affine) hyperplane orthogonal to the direction of $L$. Note that, since $H \cap \cU $ contains $x_0$, it is not empty, and in fact it generates the entirety of $H$ (since $\cU$ is open). Now, because any $x \in H \cap \cU$ is orthogonal to $L$, we have $\<x, y-y'\> = 0$, implying $\<f(x), g(y) - g(y')\> = 0$ via \eqref{internal_equivalent_vector_continuum_equal}.  
%\qedclaim
%
%Let $x_0 \in \bbS^{p-1}$ and $s_0 > 0$ be such that $\cU := \ball(x_0, s_0) \cap \bbS^{p-1} \subset \cX$. Also, let $y_0 \in \bbR^p$ and $t_0 > 0$ be such that $\cV := \ball(y_0, t_0) \subset \cY$. We will first focus on these two sets.
%
%\begin{claim}\label{cla:f_linear}
%For $x \in \cX$ other than the canonical basis vectors, $f(x) = L^{-1} x$.
%\end{claim}


%\medskip
%\begin{remark}[\citet{shepard1966metric}'s arguments] 
%\end{remark}




\section{Discussion}
\label{sec:discussion}

The paper provides a comprehensive theory of ordinal embedding in some of its main variants in the continuous model of \citet{shepard1966metric}. 
We firmly believe that the sort of analysis done in \cite{klein, arias2017some} in the finite sample setting is possible in the context of external and internal unfolding, but carry this out would make the study substantially more technical. 

Although it might be possible to sharpen some of our results by dropping or weakening some assumptions, more importantly, we have avoided issues of missingness and errors in the data. 
While there is a good amount of some theory available in the metric setting on these two issues, for example, in \cite{arias2020perturbation, anderson2010formal, arias2022supervising, javanmard2013localization} (and of course the whole literature on Graph Rigidity Theory) there is comparatively little in the ordinal setting.
A situation with some missingness, where only local comparisons are available, is considered in the context of ordinal embedding in \cite{arias2017some}, and some theory is developed in a framework that allows for erroneous comparisons in \cite{jain2016finite}.
However, we do not see an elegant and useful way of extending the continuous model to allow for missingness and noise in the comparisons. 


\subsection*{Acknowledgements}
The work of EAC was partially supported by the US National Science Foundation (DMS 1916071).
The work of CB was supported by the Deutsche Foschungsgemeinschaft (German Research Foundation) on the French-German PRCI ANR ASCAI CA 1488/4-1 ``Aktive und Batch-Segmentierung, Clustering und Seriation: Grundlagen der KI''.
The work of DK was partially supported by the US National Science Foundation (NSF Medium Award CCF-2107547 and NSF Award CCF-1553288 CAREER).
%The authors have no competing interests to declare that are relevant to the content of this article.

{\small
\bibliographystyle{chicago}
\bibliography{ref}
}

\appendix
\section{Appendix} 
\label{appendix}

\subsection{Vector Model for Ordinal MDS}
We propose a possible vector model for ordinal MDS where the items are to be embedded as vectors in the unit sphere. We are not aware of publications considering this problem, at least not in the ordinal setting. 

\subsubsection{Discrete Setting}
In practice, when dealing with a finite set of items that need to be embedded in the unit sphere, the problem is as follows:
Given a set of row ranks $(r_{ij} : i, j \in [n])$, with each $(r_{i1}, \dots, r_{in})$ being a permutation of $(1, \dots, n)$, and a dimension $p \ge 1$, 
\begin{equation}
\begin{gathered}
\text{Find $x_1, \dots, x_n \in \bbS^{p-1}$ such that $\<x_i, x_j\> < \<x_i, x_k\>$ whenever $r_{ij} > r_{ik}$.}
\end{gathered}
\end{equation}
We note that any orthogonal transformation of a solution is also a solution.

In view of the fact that $\|x - x'\|^2 = 2 (1-\<x, x'\>)$ for all $x, x' \in \bbS^{p-1}$, the problem is in fact the same as \eqref{mds_discrete} but with the embedding space being $\bbS^{p-1}$ instead of $\bbR^p$.  

\subsection{Continuous Setting}
Reasoning as we did in throughout the paper, and in particular in \secref{mds continuum}, if we follow \citet{shepard1966metric} we arrive at a limit model in the continuum where the set of items forms an uncountably infinite subset $\cX \subset \bbS^{p-1}$, and we are led to study injective functions $f : \cX \to \bbS^{p-1}$ that are weakly isotonic, meaning that satisfy \eqref{mds_continuum}, or equivalently,
\begin{equation} \label{mds vector continuum}
\text{$\<x, x'\> < \<x,x''\> \iff \<f(x),f(x')\> < \<f(x),f(x'')\>$, \quad for all $x, x', x'' \in \cX$.}
\end{equation}
We note that \eqref{mds_continuum_equal} also applies here, or equivalently,
\begin{equation}
\label{mds vector continuum equal}
\text{$\<x, x'\> = \<x,x''\> \iff \<f(x),f(x')\> = \<f(x),f(x'')\>$, \quad for all $x, x', x'' \in \cX$.}
\end{equation}
We quickly note that this implies that 
\begin{equation}
\label{mds vector continuum opposites}
\text{$f(-x) = -f(x)$ for all $x \in \cX$ such that $-x \in \cX$.}
\end{equation}
%which is equivalent to
%\begin{equation}\label{mds_continuum_equal_set}
%f(\sphere(x, \|x-x'\|) \cap \cX) = \sphere(f(x), \|f(x)-f(x')\|)) \cap f(\cX), \quad \forall x,x' \in \cX.
%\end{equation}

We content ourselves with the following analog of \prpref{shepard}.

\begin{proposition}
\label{prp:shepard_sphere}
Suppose that $\cX = \bbS^{p-1}$. Then any {\em bijective} function $f: \bbS^{p-1} \to \bbS^{p-1}$ satisfying \eqref{mds vector continuum} must be an orthogonal transformation. 
\end{proposition}

The arguments are very similar to --- and simpler than --- those underlying \thmref{mds}. The setting of \prpref{shepard_sphere} is in place. 

\begin{lemma}
\label{lem:f_shepard_vector_continuous}
In the present context, $f$ is continuous.
\end{lemma}

\begin{proof}
%Let $\cS$ be shorthand for $\bbS^{p-1}$. 
Fix $x \ne x'$ in $\bbS^{p-1}$. Let $\cC$ denote the great circle passing through $x$ and $x'$, and let $x_0 = x, x_1, \dots, x_{m-1}, x_m$ be regularly placed long $\cC$ in sequence so that $\<x_{j-1}, x_j\> = \cos (2\pi/m)$ for all $j$. We choose $m$ largest so that $\cos (2\pi/m) \le \<x,x'\>$. By construction,
\begin{equation} \label{m lower bound}
m > \frac{2\pi}{\cos^{-1} \<x,x'\>} - 1. 
\end{equation}

Since 
\[\<x,x'\> = \<x_0, x'\>
 \ge \<x_0, x_1\> = \<x_1,x_2\> = \cdots = \<x_{m-1},x_m\>,\]
by \eqref{mds vector continuum} and \eqref{mds vector continuum equal}, we have 
\[\<f(x),f(x')\> \ge \<f(x_0), f(x_1)\> = \<f(x_1),f(x_2)\> = \cdots = \<f(x_{m-1}),f(x_m)\>.\]
Therefore, if we define $\theta := \cos^{-1} \<f(x_0), f(x_1)\>$, we have that $\theta \ge \cos^{-1} \<f(x),f(x')\>$ and that $\{f(x_0), f(x_1), \dots, f(x_{m-1})\}$ forms a $\theta$-packing of $\bbS^{p-1}$. It is well-known that such a packing must have cardinality $\le C_0 \theta^{-(p-1)}$ for some constant $C_0$ depending only on $p$. Hence,
\begin{equation} \label{m upper bound}
m \le C_0 \big[\cos^{-1} \<f(x),f(x')\>\big]^{-(p-1)}. 
\end{equation}

Combining \eqref{m lower bound} and \eqref{m upper bound}, we obtain
\[\cos^{-1} \<f(x),f(x')\> \le C_1 \big[\cos^{-1} \<x,x'\>\big]^{1/(p-1)},\]
for some other constant $C_1$ that depends only on $p$. 
This is equivalent to
\[\cos^{-1} \big(1 - \tfrac12 \|f(x)-f(x')\|^2\big) \le C_1 \big[\cos^{-1} \big(1 - \tfrac12 \|x-x'\|^2\big)\big]^{1/(p-1)}.\]
Since $\cos^{-1} : [-1,1] \to [0, \pi]$ is a homeomorphism with value $0$ at 1, this inequality implies that $f$ is continuous.
\end{proof}

\begin{lemma} 
\label{lem:f_shepard_vector_midpoints}
In the present context, $f$ preserves midpoints.
\end{lemma}

In the proof of this lemma, we will use the following analog of \eqref{f_shepard}, where the spheres are understood as being within $\bbS^{p-1}$:
\begin{equation}\label{f_shepard_vector}
f(\sphere(x, \|x-x'\|)) = \sphere(f(x), \|f(x)-f(x')\|), \quad \forall x,x' \in \bbS^{p-1}.
\end{equation}

\begin{proof}
Fix $x,x' \in \bbS^{p-1}$, distinct and not diametrically opposed. 
The midpoint of $x, x'$ is the unique equidistant point on the shortest arc joining them. Let $x_0$ denote that point. Staying within $\bbS^{p-1}$, let $\cS$ (resp.~$\cS'$) denote the sphere with diameter $[x x_0]$ (resp.~$[x' x_0]$), and let $x_1$ (resp.~$x'_1$) denote its center so that $\cS = \sphere(x_1, a)$ and $\cS' = \sphere(x'_1, a)$ with $a := \|x_1-x_0\| = \|x'_1-x_0\|$. By construction, $x_0$ is the only point at the intersection of these two spheres. 

By \eqref{f_shepard_vector}, $f(\cS) = \sphere(f(x_1), b)$ and $f(\cS') = \sphere(f(x'_1), b)$ with (using \eqref{mds_continuum_equal}) $b := \|f(x_1) - f(x_0)\| = \|f(x'_1) - f(x_0)\|$.
Moreover, by injectivity of $f$, $f(x_0)$ is the only point at the intersection of these two spheres of equal radius. Therefore, $f(x_0)$ must be the midpoint of $f(x)$ and $f(x')$.   
\end{proof}



\begin{proof}[Proof of \prpref{shepard_sphere}]
Let $e_1, \dots, e_p$ denote the canonical basis on $\bbR^p$. Without loss of generality, we may assume that $f(e_j) = e_j$ for all $j$. 
By \eqref{mds vector continuum opposites}, we then know that $f(-e_j) = -e_j$ for all $j$ as well.
Let $\cU_0 = \{\pm e_1, \dots, \pm e_p\}$ and define $\cU_1, \cU_2, \dots$ recursively by letting $\cU_t$ be made of the midpoints of any pair of points in $\cU_{t-1}$ that are not diametrically opposed. 
Then let $\cV = \cU_0 \cup \cU_1 \cup \dots$. 
By applying \lemref{f_shepard_vector_midpoints} recursively, we obtain that $f(x) = x$ for any point $x \in \cV$. 
But since $\cV$ is dense in $\bbS^{p-1}$, and $f$ is continuous (\lemref{f_shepard_vector_continuous}), we must have $f(x) = x$ for all $x \in \bbS^{p-1}$. 
\end{proof}
 

\end{document}


