\documentclass[9pt,reqno]{amsart}

%\documentclass[12pt]{report}
\usepackage{amsmath, amsfonts, amssymb, latexsym, amsthm}
\usepackage[pagewise]{lineno}
%\usepackage{geometry}
\usepackage{amsmath, amsfonts, amssymb, latexsym}
\usepackage[numbers,sort&compress]{natbib}
\usepackage{mathrsfs}
\usepackage{pdfcomment}
%\usepackage{showkeys}
\newcommand{\commentontext}[2]{\colorbox{yellow!60}{#1}\pdfcomment[color={0.234 0.867 0.211},hoffset=-6pt,voffset=10pt,opacity=0.5]{#2}}
\newcommand{\commentatside}[1]{\pdfcomment[color={0.045 0.278 0.643},icon=Note]{#1}}
\newcommand{\todo}[1]{\commentatside{#1}}
\newcommand{\TODO}[1]{\commentatside{#1}}


\usepackage{autobreak}
\allowdisplaybreaks[4]











%\usepackage{hyperref}  % 有了\usepackage{pdfcomment},这一句加不加都可以(吧)
\hypersetup{hidelinks,
	colorlinks=true,
	allcolors=black,
	pdfstartview=Fit,
	breaklinks=true
}

\everymath{\displaystyle}

\newtheorem{theorem}{Theorem}
\theoremstyle{plain}

\newtheorem{lemma}[theorem]{Lemma}
\newtheorem{definition}[theorem]{Definition}
\newtheorem{assumption}[theorem]{Assumption}
\newtheorem{proposition}[theorem]{Proposition}
\newtheorem{corollary}[theorem]{Corollary}
\newtheorem{remark}[theorem]{Remark}

\numberwithin{equation}{section}
\numberwithin{theorem}{section}

%\newcommand{\cV}{\mathcal{V}}
%\newcommand{\cY}{\mathcal{Y}}
%\newcommand{\cU}{\mathcal{U}}
%\newcommand{\cA}{\mathcal{A}}
%\newcommand{\cB}{\mathcal{B}}
%\newcommand{\cM}{\mathcal{M}}
%\newcommand{\cE}{\mathcal{E}}
%\newcommand{\cK}{\mathcal{K}}
%\newcommand{\cL}{\mathcal{L}}
%\newcommand{\cJ}{\mathcal{J}}
%\newcommand{\cP}{\mathcal{P}}
%\newcommand{\cF}{\mathcal{F}}
%\newcommand{\cH}{\mathcal{H}}
%\newcommand{\cT}{\mathcal{T}}
%\newcommand{\cW}{\mathcal{W}}
%\newcommand{\cS}{\mathcal{S}}
%\newcommand{\C}{\mathbb{C}}
%\newcommand{\Y}{\mathbb{Y}}
%\newcommand{\D}{\mathbb{D}}
%\newcommand{\E}{\mathbb{E}}
%\newcommand{\R}{\mathbb{R}}
%\newcommand{\F}{\mathbb{F}}
%\newcommand{\N}{\mathbb{N}}
%\newcommand{\Z}{\mathbb{Z}}
%\newcommand{\bP}{\mathbb{P}} % \P is used by the system


\def\ds{\displaystyle}
\def\ol{\overline}
\def\mb{\mathbb}
\def\om{\omega}
\def\Om{\Omega}
\def\ra{\rightarrow}
\def\d{\delta}
\def\e{\varepsilon}
\def\ol{\overline}
\def\wt{\widetilde}


\def\ds{\displaystyle}

\def\d{\delta}
\def\e{\varepsilon}
\def\ol{\overline}
\def\wt{\widetilde}

\def\Ito{It\^{o} }
\DeclareMathOperator{\esssup}{ess\,sup}
\DeclareMathOperator*{\arginf}{\mathrm{arginf}}
\DeclareMathOperator*{\di}{\mathrm{d} \!}
\DeclareMathOperator*{\supp}{\mathrm{supp}}
\DeclareMathOperator*{\Div}{\mathrm{div}}
\def\ol{\overline}
\def\mb{\mathbb}
\def\om{\omega}
\def\Om{\Omega}
\def\ra{\rightarrow}
\def\df{{\rm d}}
\def\mcH{\mathcal{H}}
\def\pt{\partial}




\makeatletter
\renewcommand{\theequation}{%
	\thesection.\arabic{equation}}
\@addtoreset{equation}{section}
\makeatother

\begin{document}
	
	\title{Null controllability of a kind of n-dimensional degenerate parabolic equation}
	%%%%%%%%%%%%%%%%%%%%%%%%%%%%%%%%%%%%%%%%%%%%%%%%%%%
	
	%%%%%%%%%%%%%%%%%%%%%%%%%%%%%%%%%%%%%%%%%%%%%%%%%%%
	
	

	
	\author{\sffamily Hongli Sun$^{1}$, Yuanhang Liu$^{2}$, Weijia Wu$^{2,*}$, Donghui Yang$^{2}$   \\
		{\sffamily\small $^1$ School of Mathematics, Physics and Big data, Chongqing University of Science and Technology,
			 Chongging 401331, China }\\
		{\sffamily\small $^2$ School of Mathematics and Statistics, Central South University, Changsha 410083, China\\ }
	}
	\footnotetext[2]{*Corresponding author: weijiawu@yeah.net }
	
	
	
%	\author{\sffamily Yuanhang Liu$^{1}$, Weijia Wu$^{1,*}$, Donghui Yang$^1$, Can Zhang$^2$   \\
%		{\sffamily\small $^1$ School of Mathematics and Statistics, Central South University, Changsha 410083, China. }\\
%		{\sffamily\small $^2$ School of Mathematics and Statistics, Wuhan University, Wuhan 430072, China. }
%	}
%	\footnotetext[1]{Corresponding author: weijiawu@yeah.net }
%	
	

	
	\email{honglisun@126.com}
	\email{liuyuanhang97@163.com}
	\email{weijiawu@yeah.net}
	\email{donghyang@outlook.com}
	
	\keywords{}
	\subjclass[2020]{93B05}
	
	\maketitle
	
	\begin{abstract}
		%We consider a class of two-dimensional degenerate parabolic equations with abstract coefficients. We provide results on improving the regularity and establish Carleman estimates for the corresponding equations by constructing specialized weight functions. As a result, we prove the null controllability of the associated equations. Furthermore, we present a specific example to illustrate the effectiveness of our methodology.
		In this paper, we investigate a class of $n$-dimensional degenerate parabolic equations with abstract coefficients. Our focus is on improving the regularity of solutions and establishing Carleman estimates for these equations through the construction of specialized weight functions. Using these results, we demonstrate the null controllability of the corresponding equations. Additionally, we provide a specific example to illustrate the efficacy of our methodology.
		
	\end{abstract}
	
	\pagestyle{myheadings}
	\thispagestyle{plain}
	\markboth{DEGENERATE PARABOLIC EQUATIONS WITH ABSTRACT COEFFICIENTS}{HONGLI SUN, WEIJIA WU AND DONGHUI YANG}
	
	
	
	\section{Introduction}
	Controllability is a fundamental concept in control theory that was first introduced by the renowned mathematician Kalman. It holds great importance in solving control problems within linear systems. The study of controllability for parabolic equations has a rich history spanning half a century, (see \cite{dolecki1977general,fattorini1971exact,fattorini1974uniform,russell1973unified,carleman1939probleme,hormander2013linear,hormander2009analysis,zuily1983uniqueness,lebeau1995controle,fursikov1996controllability,emanuilov1995controllability}), and can be categorized into two main branches: the controllability of non-degenerate parabolic equations and the controllability of degenerate parabolic equations. While there has been significant progress in analyzing the controllability of non-degenerate parabolic equations across various fields, research on the controllability of degenerate parabolic equations still remains relatively limited. 
	
	The degenerate parabolic equation, which is a common class of diffusion equations, can describe numerous physical phenomena. For example, the famous Crocco equation is a degenerate parabolic equation, which reflects the compatibility relationship between the change in total energy and entropy in steady flow and vorticity. Tornadoes follow the Crocco equation during the rotation process, and thus the study of controllability and optimal control problems of the Crocco equation is of great significance in meteorology (see \cite{martinez2003regional}). Similarly, the Black-Scholes equation (see \cite{sakthivel2008exact}), which is widely studied in finance, and the Kolmogorov equation (see \cite{calin2009heat,calin2010heat,anceschi2019survey}), are also degenerate parabolic equations that have important practical applications in our real life. Therefore, the study of the control problems of degenerate parabolic equations is of great significance.
	In \cite{cannarsa2004persistent}, the authors introduced the concepts of regional null controllability and regional persistent null controllability, and proved the regional controllability for a Crocco type linearized equation and for the nondegenerate heat equation in unbounded domains. Furthermore, in subsequent studies \cite{CA5,CA8,flores2010carleman,cannarsa2007null,cannarsa2008controllability}, the controllability of one-dimensional degenerate heat equations and other one-dimensional degenerate parabolic equations was demonstrated. After this, scholars have also investigated the problem of the controllability of high-dimensional degenerate equations, and the controllability of the Grushin-type operator has been extensively studied  (see \cite{anh2013null,cannarsa2013null,beauchard20152d,banerjee2022carleman}).
	In \cite{cannarsa2013null}, The authors obtained the Carleman estimate of the two-dimensional Grushin-type operator by using the Fourier decomposition,
	and further obtained the null controllability. Besides, there are many new results on the controllability of other high-dimensional degenerate equations.
	In  \cite{CA6}, they presented an effective approach to establish the controllability of two-dimensional degenerate equations and to address optimal control problems. Importantly, in the context of two-dimensional equations, the validity of the divergence theorem becomes crucial for meaningful boundary integrals along the degenerate boundary of the control region. Consequently, the inclusion of two-dimensional weighted Hardy inequalities and trace operators on weighted spaces becomes necessary to handle degenerate boundaries. 
	
	Inspired by the work of \cite{CA6}, this paper focuses on a more general class of $n$-dimensional degenerate parabolic equations. By employing the Carleman estimate technique, we establish the null controllability of these equations, where the degenerate coefficient is considered as an abstract function, and the control region is located near the boundary.
	
	It is worth noting that our model exhibits significant differences from that of \cite{CA6}. In \cite{CA6}, they need to use Hardy inequality and trace theory to obtain the well-posed results and Carleman estimates, but in this paper, due to different assumptions, different control regions and different weight functions, we do not need Hardy inequality and trace theory, but use a new method to obtain our Carleman's estimate by ``cutting off " the degenerate boundary. In other words, if we cut off the degenerate boundary, null controllability always holds. However, this is only one of the methods, and interior control is a further problem for us to consider.
	
	
	%In this paper, we consider
	%\begin{equation}\label{1.1}
	%\begin{cases}
	%\partial_{t}z - \sum_{i,j=1}^n\partial_{x_i}(A_{ij}\partial_{x_j}z) + bz=\chi_{\omega}g, & \mbox{in} \ Q,  \\
	%z=0 \ \mbox{or} \ A\nabla z \cdot \nu =0,  & \mbox{on} \ \Sigma,  \\
	%z(0)= z_{0},  & \mbox{in} \ \Omega,  
	%\end{cases}
	%\end{equation}
	%where $\Omega\subset \mathbb{R}^n$ with Lipschitz boundary, $\Gamma:= \partial\Omega, T>0$, $ Q:=\Omega\times (0,T), \Sigma:= \Gamma\times(0,T)$, let $\hat{\Om}\subset \Om$ be a nonempty open subset, $ \Omega_0 \subset\subset \hat{\Omega}$ is a nonempty open set with smooth boundary,  $\omega:=\Om\backslash\overline{\Omega_0}$ and $\chi_{\omega}$ is the corresponding characteristic function, 
	%% $b\in L^{\infty}(Q)$,
	%$g \in L^{2}(Q)$ is the control, $z_0 \in L^{2}(\Omega)$ is the initial data, $b\in L^{\infty}(Q)$ is a given function. We need the following assumption:
	%
	%\begin{assumption}\label{assume1}
	%the matrix-valued function $A=(A_{ij}(x))_{i,j=1}^n$ is positive definite for all $x \in \Om$, but may false at a subset of $\partial\Omega$, and $A_{ij}\in C^2(\overline{\hat{\Om}}), i,j=1,\cdots, n$, 
	%$$
	%\ \lambda|\xi|^2 \le \sum_{i,j=1}^n A_{ij}(x)\xi_i\xi_j \le \Lambda |\xi|^2, \ \forall \xi\in\mathbb{R}^n, \forall x \in \hat{\Om}, 
	%$$
	%here $0<\lambda \le \Lambda$ are two fixed constants.
	%\end{assumption}
	%From Assumption \ref{assume1}, it is known that the equation \eqref{1.1} is degenerate on part of the boundary, but non-degenerate in the interior.
	%
	%
	%%Obviously, $\sum_{i,j=1}^n\partial_{x_i}(A_{ij}\partial_{x_j}z)=\Div(A\nabla z)$.
	The remaining sections of this paper are structured as follows. In section 2, we present some preliminary results and demonstrate the well-posedness of problem \eqref{1.1}. In section 3, we provide Carleman estimates for the degenerate equation \eqref{1.1} and present the main results of this paper. In order to enhance the understanding of our model, section 4 provides a specific example and discusses how to improve the internal regularity and obtain Carleman estimates for the corresponding equation \eqref{4.1} to be formulated later.
	
	
	\section{main results}
	
	This section presents the main results of the paper, which are divided into two parts: Results in abstract form and Results in a specific example.
	\subsection{Results in abstract form}
	
	\hspace*{\fill}\\
	
	In this paper, we consider the following problem in a bounded domain $\Omega\subset \mathbb{R}^n$ with Lipschitz boundary:
	\begin{equation}\label{1.1}
		\begin{cases}
			\partial_{t}z - \sum_{i,j=1}^n\partial_{x_i}(A_{ij}\partial_{x_j}z) + bz=\chi_{\omega}g, & \mbox{in} \ Q,  \\
			z=0 \ \mbox{or} \ A\nabla z \cdot \nu =0,  & \mbox{on} \ \Sigma,  \\
			z(0)= z_{0},  & \mbox{in} \ \Omega.  
		\end{cases}
	\end{equation}
	Here, $Q:=\Omega\times (0,T)$ and $\Sigma:= \Gamma\times(0,T)$ denote the space-time domain and boundary, respectively. The set $\hat{\Omega}\subset \Omega$ is a nonempty open subset, and $\Omega_0 \subset\subset \hat{\Omega}$ is a nonempty open set with a smooth boundary. The set $\omega:=\Omega\backslash\overline{\Omega_0}$ is defined as the complement of the closure of $\Omega_0$, and $\chi_{\omega}$ is the corresponding characteristic function. The control function $g \in L^{2}(Q)$ and the initial data $z_0 \in L^{2}(\Omega)$ are given, while $b\in L^{\infty}(Q)$ is a known function.
	
	We impose the following assumption on the matrix-valued function $A=(A_{ij}(x))_{i,j=1}^n$:
	
	\begin{assumption}\label{assume1}
		The matrix-valued function $A=(A_{ij}(x))_{i,j=1}^n$ is positive definite for all $x \in \Om$, but may vanish at a subset of $\partial\Omega$, and $A_{ij}=A_{ji}\in C^2(\overline{\hat{\Om}})$, $i,j=1,\cdots, n$, 
		$$
		\ \rho|\xi|^2 \le \sum_{i,j=1}^n A_{ij}(x)\xi_i\xi_j \le \Lambda |\xi|^2, \ \forall \xi\in\mathbb{R}^n, \forall x \in \hat{\Om}, 
		$$
		here $0<\rho \le \Lambda$ are two fixed constants.
	\end{assumption}
	From Assumption \ref{assume1}, it is known that the equation \eqref{1.1} is degenerate on a part of the boundary, but non-degenerate in the interior.
	
	%\section{Well-posed results}
	%In this section we discuss the well-posedness of \eqref{1.1}. 
	We define the function spaces $\mathcal{H}^1(\Omega)$ and $\mathcal{H}^2(\Omega)$ as follows: 
	\begin{equation}\label{space}
		\begin{split}
			\mathcal{H}^1(\Om)
			&=\left\lbrace z\in L^2(\Om) \mid \nabla z A \nabla z \in L^1(\Om) \right\rbrace,\\
			\mathcal{H}^2(\Om)
			&=\left\lbrace z\in \mathcal{H}^1(\Om) \mid \Div(A\nabla z) \in L^2(\Om) \right\rbrace.
		\end{split}
	\end{equation}
	These spaces are equipped with the following scalar products:
	\begin{equation}\label{inner}
		\begin{split}
			(z,v)_{\mathcal{H}^1(\Om)}
			&:=\int_{\Omega} zv  dx  + \int_{\Omega} \nabla z A \nabla v dx,\\
			(z,v)_{\mathcal{H}^2(\Om)}
			&:=\int_{\Omega} zv  dx  + \int_{\Omega} \nabla z A \nabla v dx + \int_{\Omega} \Div(A\nabla z)\Div(A\nabla v) dx.
		\end{split}
	\end{equation}
	These spaces are endowed with the following norms:
	\begin{equation}\label{norm}
		\begin{split}
			\left\| z \right\|_{\mathcal{H}^1(\Om)} 
			&= \left\| z \right\|_{L^2(\Om)} + \left\| \nabla z A \nabla z\right\|_{L^1(\Om)},\\
			\left\| z \right\|_{\mathcal{H}^2(\Om)} 
			&= \left\| z \right\|_{\mathcal{H}^1(\Om)} + \left\| \Div(A\nabla z)\right\|_{L^2(\Om)}.
		\end{split}
	\end{equation}
	It can be easily verified that $(\mathcal{H}^1(\Omega), (\cdot,\cdot){\mathcal{H}^1(\Omega)})$ and $(\mathcal{H}^2(\Omega), (\cdot,\cdot){\mathcal{H}^2(\Omega)})$ are inner product spaces, and $(\mathcal{H}^1(\Omega), |\cdot|{\mathcal{H}^1(\Omega)})$ and $(\mathcal{H}^2(\Omega), |\cdot|{\mathcal{H}^2(\Omega)})$ are Banach spaces.
	
	Let $\mathcal{H}_0^1(\Omega)$ denote the closure of $C_0^\infty(\Omega)$ in the space $\mathcal{H}^1(\Omega)$. In other words,
	\begin{equation*}
		\mcH_0^1(\Omega)=\overline{C_0^\infty(\Omega)}^{\mcH^1(\Omega)}.
	\end{equation*} 
	We also introduce the operators $(\mathcal{A}_1,D(\mathcal{A}_1))$ and $(\mathcal{A}_2,D(\mathcal{A}_2))$ defined as follows:
	\begin{equation*}
		\begin{split}
			&\mathcal{A}_1 z= \sum_{i,j=1}^n\partial_{x_i}(A_{ij}\partial_{x_j}z),  \quad D(\mathcal{A}_1) = \mathcal{H}^2 (\Om)\cap \mcH_0^1(\Omega) ,\\
			&\mathcal{A}_2 z= \sum_{i,j=1}^n\partial_{x_i}(A_{ij}\partial_{x_j}z),  \quad D(\mathcal{A}_2) = \left\lbrace z\in \mathcal{H}^2 (\Om) \mid A\nabla z \cdot \nu = 0 \right\rbrace .
		\end{split}
	\end{equation*}
	Next, we will present some results related to the bilinear form $q(\cdot,\cdot)$ associated with the operators $\mathcal{A}_1$ and $\mathcal{A}_2$. But before that, we introduce an assumption.
	
	\begin{assumption}\label{assume2}
		When $A\nabla z \cdot \nu =0$, that is to say, in Newman's boundary condition, we assume that $A\nabla z \in (W^{1,1}(\Om))^n$.
	\end{assumption}
	\begin{remark}
		It should be noted that this assumption is used to prove Lemma \ref{div}, which will be presented later. However, Assumption \ref{assume2} is not strictly necessary. Following a similar approach to the case in \cite{BSV}, we can assume that $A\nabla z \in H^{-\frac{1}{2}}(\Gamma)$ and $\nu \in H^{\frac{1}{2}}(\Gamma)$.
	\end{remark}
	%Then we can establish the following Green's formula.
	%\begin{lemma}\label{div}
	%If $(u,v)\in D(\mathcal{A}_1)\times \mcH^1_0(\Omega)$ or $(u,v)\in D(\mathcal{A}_2)\times \mcH^1(\Omega)$, one has 
	%\begin{equation*}
	%\int_\Om \nabla u \cdot A \nabla v dx = -\int_\Om \Div (A \nabla u) vdx.
	%\end{equation*}
	%\end{lemma}
	%\begin{proof}
	%By Assumption \ref{assume2} and boundary conditions, this concludes the proof. 
	%\end{proof}
	%
	%\begin{lemma}\label{2.15}
	%The following results hold:
	%\begin{itemize}
	%    \item [(1)] The injection $i_1: D(\mathcal{A}_1)\to L^2(\Om)$ and $i_2: D(\mathcal{A}_2)\to L^2(\Om)$ is continuous with dense range;
	%    \item [(2)] The bilinear form $q_1(u,v):= \int_\Om \nabla u \cdot A \nabla v dx, \ (u,v)\in D(\mathcal{A}_1)\times \mcH^1_0(\Omega)$ and $q_2(u,v):= \int_\Om \nabla u \cdot A \nabla v dx, \ (u,v)\in D(\mathcal{A}_2)\times \mcH^1(\Omega),$ are continuous, positive, symmetric.
	%    \item [(3)]  The operator $(\mathcal{A}_1,D(\mathcal{A}_1)), (\mathcal{A}_2,D(\mathcal{A}_2))$ are self-adjoint and dissipative.
	%  \end{itemize}
	%%are the infinitesimal generators of the strongly continuous semigroups denoted by $e^{t\mathcal{A}_1}, e^{t\mathcal{A}_2}$ respectively.
	%\end{lemma}
	%\begin{proof}
	%Obviously, by definition of $D(\mathcal{A}_1)$ and $D(\mathcal{A}_2)$, it follows that $i_1,i_2$ are continuous. Thus, since $C_0^\infty (\Om)$ is dense in $D(\mathcal{A}_1)$ and $D(\mathcal{A}_2)$, it is also dense in $L^2(\Om)$. This proves (1). 
	%
	%Now, we prove (2). 
	%It is clear that $q_1,q_2$ is symmetrical and $q_1(u,u) \ge 0 , q_2(u,u) \ge 0$. Furthermore
	%$$
	%|q_1(u,v)|\le\left\| u\right\| _{\mcH_0^1(\Omega)} \left\| v\right\| _{\mcH_0^1(\Omega)},\ |q_2(u,v)|\le\left\| u\right\| _{\mcH^1(\Omega)} \left\| v\right\| _{\mcH^1(\Omega)}.
	%$$
	%Hence $q_1,q_2$ is continuous.
	%
	%Now suppose that $(u_1,v_1) \in  D(\mathcal{A}_1)\times \mcH^1_0(\Omega),(u_2,v_2)\in D(\mathcal{A}_2)\times \mcH^1(\Omega)$, by Lemma \ref{div} we have that
	%$$
	%q_1(u_1,v_1)=-\int_{\Om}\mathcal{A}_1 u_1 \cdot v_1 d x, \ q_2(u_2,v_2)=-\int_{\Om}\mathcal{A}_2 u_2 \cdot v_2 d x.
	%$$
	%Since
	%$$
	%\begin{aligned}
	%\left\| (\lambda I - \mathcal{A}_1)u_1\right\| _{L^2(\Om)}
	%&=\sup_{\|v_1\|_{L^2(\Om)} \le 1, v_1 \in \mcH_0^1(\Omega)} \int_{\Omega} \lambda v_1 u_1 - v_1 \mathcal{A}_1 u_1 dx
	%%=&\sup_{|v|_2 \le 1, v \in \mcH_0^1(\Omega)} \int_{\Omega} \lambda v u dx + q(u,v)\\
	%\ge  \lambda \left\| u_1\right\| _{L^2{\Om}}\\
	%\left\| (\lambda I - \mathcal{A}_2)u_2\right\| _{L^2(\Om)}
	%&=\sup_{\|v_2\|_{L^2(\Om)} \le 1, v_2 \in \mcH^1(\Omega)} \int_{\Omega} \lambda v_2 u_2 - v_2 \mathcal{A}_2 u_2 dx
	%%=&\sup_{|v|_2 \le 1, v \in \mcH^1(\Omega)} \int_{\Omega} \lambda v u dx + q(u,v)\\
	%\ge  \lambda \left\| u_2\right\| _{L^2{\Om}},
	%\end{aligned}
	%$$
	%and from the properties of $q_1,q_2$, we conclude that $\mathcal{A}_1$ and $\mathcal{A}_2$ are self-adjoint and dissipative.
	%% as a similar argument, we conclude that $\mathcal{A}_2$ is also self-adjoint and dissipative.
	%\end{proof}
	
	%
	%As a consequence, both $\mathcal{A}_1$ and $\mathcal{A}_2$ are the infinitesimal generators of the strongly continuous semigroups denoted by $e^{t\mathcal{A}_1}$, $e^{t\mathcal{A}_2}$ respectively. Moreover, the family of operators in $\mathcal{L}(L^2(\Om))$ given by
	%$$
	%B(t)u := b(t,\cdot)u, \quad t\in (0,T), \quad u\in L^2(\Om)
	%$$
	%can be seen as a family of bounded perturbation of $\mathcal{A}_1$ (resp. $\mathcal{A}_2$). Thus, using standard techniques(see \cite{bensoussan2007representation,cazenave1998introduction,showalter1995hilbert}), one can prove the following well-posedness results.
	
	\begin{theorem}\label{exist1}
		For any $g \in L^2(Q)$ and any $z_0 \in L^2(\Omega)$, there exists a unique solution $z \in C^0([0,T];L^2(\Omega)) \cap L^2(0,T;\mathcal{H}_0^1(\Omega))$ to equation \eqref{1.1}. Moreover, there exists a positive constant $C$ such that
		$$
		\sup_{t\in\left[0,T\right] } \|z(t)\|_{L^2(Q)}^2 + \int_{0}^{T} \left\| z(t)\right\|^2_{\mcH_0^1(\Omega)} d t \le C(\left\| z_0\right\|_{L^2(\Om)}^2 + \|g\|_{L^2(Q)}^2).
		$$
	\end{theorem}
	\begin{theorem}\label{exist2}
		For any $g \in L^2(Q)$ and any $z_0 \in L^2(\Omega)$, there exists a unique solution $z \in C^0([0,T];L^2(\Omega)) \cap L^2(0,T;\mathcal{H}^1(\Omega))$ to equation \eqref{1.1} with homogeneous Neumann boundary conditions. Furthermore, there exists a positive constant $C$ such that
		$$
		\sup_{t\in\left[0,T\right] } \|z(t)\|_{L^2(Q)}^2 + \int_{0}^{T} \left\| z(t)\right\|^2_{\mcH^1(\Omega)} d t \le C(\left\| z_0\right\|_{L^2(\Om)}^2 + \|g\|_{L^2(Q)}^2).
		$$
	\end{theorem}
	\begin{remark}
		It is worth noting that by adding a gradient term $c\nabla z$ to the left-hand side of equation \eqref{1.1} and imposing stronger conditions on the degenerate coefficient $A_{ij}(x)$, similar well-posedness results can be obtained.
	\end{remark}
	
	%\subsection{Internal regularity improvement}
	%Now, we have proved the existence of solutions to the equation \eqref{1.1}, and the solution $z \in C([0,T];L^{2}(\Omega))\cap L^{2}(0,T;\mathcal{H}_0^{1}(\Omega))$.
	%We are going to show that the regularity of the solutions can be improved in the interior of $\Omega$. By standard argument in \cite{EVANS}, one can prove the following result.
	%\begin{theorem}\label{Regularity}
	%For all the solution of equation \eqref{1.1}, one has $A\nabla u \in W^{1,1}(\Om_0)$ and $\nabla u A\nabla u \in W^{1,1}(\Om_0)$.
	%\end{theorem}
	%\begin{proof}
	%By standard argument in \cite{EVANS}, one can prove this result.
	%\end{proof}
	
	%\begin{remark}
	%It is worth noting that, on the boundary, we cannot improve the regularity due to the specificity of the equation we are considering. Based on this fact, on the boundary, we do not have the fact $(A \nabla u \cdot \nabla \sigma) A \nabla u, (A\nabla u\cdot\nabla u)A\nabla\sigma\in (W^{1,1}(\Omega))^2$ (see the estimate of $I_4$ about the boundary term in the following section), in other words, the functions $(A \nabla u \cdot \nabla \sigma) A \nabla u, (A\nabla u\cdot\nabla u)A\nabla\sigma$ do not have trace on $\partial\Omega$, which makes our research approach very different from  \cite{CA5,CA6}, and this is also the reason that we choose the control domain is $\omega:= \Om\backslash \overline{\Om_0}$ in the following section.
	%\end{remark}
	
	%\section{Carleman esitimate and Null Controllability results}
	
	%\subsection{main results}
	Let us consider the adjoint equation corresponding to \eqref{1.1}:
	%Obviously, the adjoint equation of \eqref{1.1} is
	\begin{equation}\label{3.1}
		\begin{cases}
			\partial_{t}w + \Div(A\nabla w) -bw=f, & \mbox{in} \ Q,  \\
			w=0 \ \mbox{or} \ A\nabla w \cdot \nu =0,  & \mbox{on} \ \Sigma,  \\
			w(x,  y,  T)= w_T,  & \mbox{in} \ \Omega.  
		\end{cases}
	\end{equation}
	The main results of this paper are the Carleman inequality and observability inequality stated below.
	\begin{theorem}\label{TH2}
		There exist positive constants $C, s_0, \lambda_0$ such that for any $\lambda \ge \lambda_0$, $s \ge s_0$, and any solution $u$ to \eqref{3.1}, the following inequality holds:
		\begin{equation}\label{3.2}
			\begin{split}
				&s^{-1}\iint_{Q} \xi^{-1} (|u_t |^2 + \left|\Div(A\nabla u)\right|^2) dx dt +C s^2 \lambda^2 \int_0^T \int_{\Om\backslash \omega} \xi^{2}|u|^2dx  d t\\
				&+C\iint_Q s^3 \lambda^4 \xi^3\left|A \nabla \eta \cdot \nabla \eta \right|^2|u|^2 dx dt  + C s  \lambda^2 \iint_Q\xi|\nabla u \cdot A \nabla \eta|^2 dx  d t\\
				\leq& C\left\|e^{-s \sigma} f\right\|^2
				+Cs^3 \lambda^3 \int_0^T \int_{\omega} \xi^3|u|^2dx  d t.
			\end{split}
		\end{equation}
	\end{theorem}
	By a standard argument, we can conclude the following theorem (see \cite{FE,ZZ2006}).
	\begin{theorem}\label{TH4}
		For a fixed $T>0$ and an open set $\omega \subset \Omega$ as defined previously, assuming that \eqref{3.2} holds, there exists a positive constant $C>0$ such that for any $w_T \in L^2(\Omega)$, the solution to \eqref{3.1} satisfies
		\begin{equation}\label{3.4}
			\int_{\Omega}|w(x, 0)|^2 d x \leq C \iint_{\omega \times(0, T)}|w|^2 d x d t.
		\end{equation}
	\end{theorem}
	Using the duality between controllability and observability, proving controllability is equivalent to establishing an observability property for the adjoint system \eqref{3.1} (see \cite{rockafellar1967duality,lions1992remarks}). From this, we can deduce the null controllability of \eqref{1.1}.
	\begin{theorem}\label{TH3}
		For a fixed $T>0$ and an open set $\omega \subset \Omega$ as defined previously, assuming that \eqref{3.2} holds, there exists a control $g \in L^2(Q)$ such that the solution $z$ of \eqref{1.1} satisfies
		$$
		z(\cdot, T)=0, \text { in } \Omega.
		$$
		Moreover, there exists a constant $C=C(T, \omega)>0$ such that
		$$
		\|g\|_{L^2(\Om)} \leq C\left|z_0\right| .
		$$
	\end{theorem}
	\subsection{Results in a specific example}
	
	\hspace*{\fill}\\
	
	To illustrate the results of this paper, we consider a specific example of a two-dimensional degenerate parabolic equation:
	\begin{equation}\label{4.1}
		\begin{cases}
			\partial_{t}z - (y^{\alpha_y}\partial_{xx}z +x^{\alpha_x}\partial_{yy}z )=\chi_{\omega}g, & \mbox{in} \ Q,  \\
			z(x,y,t)=0 \ \mbox{or} \ A\nabla z \cdot \nu =0, & \mbox{on} \ \Sigma,  \\
			z(x,  y,  0)= z_{0}(x,  y),  & \mbox{in} \ \Omega,  
		\end{cases}
	\end{equation}
	where $\alpha_x$, $\alpha_y \in (0,2)$, $\Omega=(0,1)\times(0,1)$, $\Gamma:= \partial\Omega$, $T>0$, $Q:=\Omega\times (0,T)$, $\Sigma:= \Gamma\times(0,T)$, $\omega \subset \Omega$ is a nonempty open set, and $\chi_{\omega}$ is the corresponding characteristic function. The control $g \in L^{2}(Q)$ and $z_0 \in L^{2}(\Omega)$ is the initial data.
	
	The function space, inner product, and norm are defined as in equations \eqref{space}-\eqref{norm}, and the matrix-valued function $A:\overline{\Omega} \to M_{2\times 2}(\mathbb{R})$ is given by
	\begin{equation*}
		\begin{pmatrix} y^{\alpha_y} & 0 \\ 0 &x^{\alpha_x} \end{pmatrix}.
	\end{equation*}
	%Define 
	%\begin{equation*}
	%\mathcal{H}^1(\Om)=\left\lbrace z\in L^2(\Om) \mid \nabla z A \nabla z \in L^1(\Om) \right\rbrace ,
	%\end{equation*}
	%\begin{equation*}
	%\mathcal{H}^2(\Om)=\left\lbrace z\in \mathcal{H}^1(\Om) \mid \Div(A\nabla z) \in L^2(\Om) \right\rbrace ,
	%\end{equation*}
	%with scalar product
	%\begin{equation*}
	%(z,v)_{\mathcal{H}^1(\Om)}:=\int_{\Omega} zv  dx dy + \int_{\Omega} \nabla z A \nabla v dx,
	%\end{equation*}
	%\begin{equation*}
	%(z,v)_{\mathcal{H}^2(\Om)}:=\int_{\Omega} zv  dx dy + \int_{\Omega} \nabla z A \nabla v dx + \int_{\Omega} \Div(A\nabla z)\Div(A\nabla v) dx,
	%\end{equation*}
	%endowed with the norm
	%\begin{equation*}
	%\left\| z \right\|_{\mathcal{H}^1(\Om)} = \left\| z \right\|_{L^2(\Om)} + \left\| \nabla z A \nabla z\right\|_{L^1(\Om)}  .
	%\end{equation*}
	%\begin{equation*}
	%\left\| z \right\|_{\mathcal{H}^2(\Om)} = \left\| z \right\|_{\mathcal{H}^1(\Om)} + \left\| \Div(A\nabla z)\right\|_{L^2(\Om)}  .
	%\end{equation*}
	Let $\mathcal{B}_1$ and $\mathcal{B}2$ be operators defined as follows:
	\begin{equation*}
		\begin{split}
			&\mathcal{B}_1 z= y^{\alpha_y}\partial_{xx}z +x^{\alpha_x}\partial_{yy}z,  \quad D(\mathcal{B}_1) = \mathcal{H}^2 (\Om)\cap \mcH_0^1(\Omega) ,\\
			&\mathcal{B}_2 z= y^{\alpha_y}\partial_{xx}z +x^{\alpha_x}\partial_{yy}z,  \quad D(\mathcal{B}_2) = \left\lbrace z\in \mathcal{H}^2 (\Om) \mid B\nabla z \cdot \nu = 0 \right\rbrace .
		\end{split}
	\end{equation*}
	It can be easily verified that $(\mathcal{B}_1,D(\mathcal{B}_1))$ and $(\mathcal{B}_2,D(\mathcal{B}_2))$ are self-adjoint and dissipative operators with dense domain. Consequently, $(\mathcal{B}_1,D(\mathcal{B}_1))$ and $(\mathcal{B}_2,D(\mathcal{B}_2))$ serve as the infinitesimal generators of the strongly continuous semigroups $e^{t\mathcal{B}_1}$ and $e^{t\mathcal{B}_2}$, respectively. As a result, similar well-posedness results can be established as in Theorem \ref{exist1} and Theorem \ref{exist2}.
	
	It is worth noting that in this example, we can also apply the Galerkin method described in \cite{EVANS} to establish the existence of solutions in weakly degenerate cases. This is possible due to the compact embedding result derived in \cite{sun2023extremal}.
	
	Once the existence of solutions is established, we can proceed to investigate the controllability of the equation \eqref{4.1}. The adjoint equation corresponding to \eqref{4.1} is given by
	\begin{equation}\label{adjoint}
		\begin{cases}
			\partial_{t}w + \Div(A\nabla w)=f, & \mbox{in} \ Q,  \\
			w(x,y,t)=0, \ \mbox{or} \ A\nabla w \cdot \nu =0, & \mbox{on} \ \Sigma,  \\
			w(x,  y,  T)= w_T,  & \mbox{in} \ \Omega.  
		\end{cases}
	\end{equation}
	By employing Carleman estimates, we can derive similar results on null controlability as presented in Theorems \ref{TH2}, \ref{TH4}, and \ref{TH3}.
	
	
	\section{Well-posed results}
	In this section, we investigate the well-posedness of \eqref{1.1}.
	
	By utilizing Assumption \ref{assume2}, we establish the following Green's formula.
	\begin{lemma}\label{div}
		For $(u,v)\in D(\mathcal{A}1)\times \mcH^1_0(\Omega)$ or $(u,v)\in D(\mathcal{A}2)\times \mcH^1(\Omega)$, the following equality holds:
		\begin{equation*}
			\int_\Om \nabla u \cdot A \nabla v dx = -\int_\Om \Div (A \nabla u) vdx.
		\end{equation*}
	\end{lemma}
	\begin{proof}
		The proof follows directly from Assumption \ref{assume2} and the boundary conditions.
	\end{proof}
	
	\begin{lemma}\label{2.15}
		The following results hold:
		\begin{itemize}
			\item [(1)] The injection $i_1: D(\mathcal{A}_1)\to L^2(\Om)$ and $i_2: D(\mathcal{A}2)\to L^2(\Om)$ are continuous with dense range.
			\item [(2)] The bilinear form $q_1(u,v):= \int_\Om \nabla u \cdot A \nabla v dx, \ (u,v)\in D(\mathcal{A}_1)\times \mcH^1_0(\Omega)$ and $q_2(u,v):= \int_\Om \nabla u \cdot A \nabla v dx, \ (u,v)\in D(\mathcal{A}_2)\times \mcH^1(\Omega),$ are continuous, positive, symmetric.
			\item [(3)]  The operators $(\mathcal{A}_1,D(\mathcal{A}_1))$ and $(\mathcal{A}_2,D(\mathcal{A}_2))$ are self-adjoint and dissipative.
		\end{itemize}
		%are the infinitesimal generators of the strongly continuous semigroups denoted by $e^{t\mathcal{A}_1}, e^{t\mathcal{A}_2}$ respectively.
	\end{lemma}
	\begin{proof}
		Clearly, the continuity of $i_1$ and $i_2$ follows directly from the definition of $D(\mathcal{A}_1)$ and $D(\mathcal{A}_2)$. Consequently, since $C_0^\infty (\Om)$ is dense in $D(\mathcal{A}_1)$, it is also dense in $L^2(\Om)$. Moreover, since $C_0^\infty (\Om)\subset \mathcal{H}^2(\Om)\subset L^2(\Om)$, we have $\mathcal{H}^2(\Om)$ is dense in $L^2(\Om)$. This establishes (1).
		
		Next, we prove (2). It is evident that $q_1$ and $q_2$ are symmetric, and $q_1(u,u) \ge 0$ and $q_2(u,u) \ge 0$ for all $u$. Furthermore,
		$$
		|q_1(u,v)|\le\left\| u\right\| _{\mcH_0^1(\Omega)} \left\| v\right\| _{\mcH_0^1(\Omega)},\ |q_2(u,v)|\le\left\| u\right\| _{\mcH^1(\Omega)} \left\| v\right\| _{\mcH^1(\Omega)}.
		$$
		Hence, $q_1$ and $q_2$ are continuous.
		
		Now, let $(u_1,v_1) \in D(\mathcal{A}_1)\times \mcH^1_0(\Omega)$ and $(u_2,v_2)\in D(\mathcal{A}_2)\times \mcH^1(\Omega)$. By applying Lemma \ref{div}, we have
		$$
		q_1(u_1,v_1)=-\int_{\Om}\mathcal{A}_1 u_1 \cdot v_1 d x, \ q_2(u_2,v_2)=-\int_{\Om}\mathcal{A}_2 u_2 \cdot v_2 d x.
		$$
		Furthermore, since
		$$
		\begin{aligned}
			\left\| (\gamma I - \mathcal{A}_1)u_1\right\| _{L^2(\Om)}
			&=\sup_{\|v_1\|_{L^2(\Om)} \le 1, v_1 \in \mcH_0^1(\Omega)} \int_{\Omega} \gamma v_1 u_1 - v_1 \mathcal{A}_1 u_1 dx
			%=&\sup_{|v|_2 \le 1, v \in \mcH_0^1(\Omega)} \int_{\Omega} \gamma v u dx + q(u,v)\\
			\ge  \gamma \left\| u_1\right\| _{L^2(\Om)},\\
			\left\| (\gamma I - \mathcal{A}_2)u_2\right\| _{L^2(\Om)}
			&=\sup_{\|v_2\|_{L^2(\Om)} \le 1, v_2 \in \mcH^1(\Omega)} \int_{\Omega} \gamma v_2 u_2 - v_2 \mathcal{A}_2 u_2 dx
			%=&\sup_{|v|_2 \le 1, v \in \mcH^1(\Omega)} \int_{\Omega} \gamma v u dx + q(u,v)\\
			\ge  \gamma \left\| u_2\right\| _{L^2(\Om)},
		\end{aligned}
		$$
		and a similar inequality holds for $(\gamma I - \mathcal{A}_2)u_2$, we conclude that $\mathcal{A}_1$ and $\mathcal{A}_2$ are self-adjoint and dissipative, as desired.
	\end{proof}
	
	
	Consequently, both $\mathcal{A}_1$ and $\mathcal{A}_2$ serve as the infinitesimal generators of the strongly continuous semigroups denoted by $e^{t\mathcal{A}_1}$ and $e^{t\mathcal{A}_2}$, respectively. Additionally, the family of operators in $\mathcal{L}(L^2(\Om))$ given by
	$$
	B(t)u := b(t,\cdot)u, \quad t\in (0,T), \quad u\in L^2(\Om)
	$$
	can be regarded as a family of bounded perturbations of $\mathcal{A}_1$ (resp. $\mathcal{A}_2$). Consequently, utilizing standard techniques (see \cite{bensoussan2007representation,cazenave1998introduction,showalter1995hilbert}), one can establish the validity of Theorem \ref{exist1} and Theorem \ref{exist2}.
	We have now demonstrated the existence of solutions to equation \eqref{1.1}, with the solution $z \in C([0,T];L^{2}(\Omega))\cap L^{2}(0,T;\mathcal{H}_0^{1}(\Omega))$. Our aim is to establish that the regularity of the solutions can be enhanced within the interior of $\Omega$. By employing standard arguments found in \cite{EVANS}, we can establish the following result.
	\begin{theorem}\label{Regularity}
		For all solutions of equation \eqref{1.1}, it holds that $A\nabla u \in W^{1,1}(\Om_0)$ and $\nabla u A\nabla u \in W^{1,1}(\Om_0)$.
	\end{theorem}
	
	\section{Carleman esitimates}
	Let us now derive a Carleman estimate. Consider $\eta \in C_0^3(\Omega)$ satisfying
	\begin{equation*}
		\eta(x) :=
		\begin{cases}
			=0, & x \in \Omega\backslash\hat{\Om},\\
			>0, & x \in \hat{\Om},
		\end{cases}
	\end{equation*}
	and
	\begin{equation*}
		\quad \left\| \nabla \eta \right\|_{L^2(\Om)} \ge C >0 \ \mbox{in} \ \Om_0, \quad  \nabla \eta=0 \ \mbox{on} \ \partial \hat{\Om}.
	\end{equation*}
	%By the classic arguments in \cite{CA6}, we can move $(\frac{1+\delta}{2},\frac{1+\delta}{2})$ to $\omega$, then we have
	%\begin{equation*}
	%\left| \nabla \eta\right| \ge C > 0, \ \mbox{in} \ \overline{\Omega \backslash \omega},
	%\end{equation*}
	Define
	\begin{equation*}
		\begin{split}
			& \theta(t):=[t(T-t)]^{-4}, \quad \xi(x, t):=\theta(t) e^{ \lambda(8|\eta|_\infty+\eta (x))}, \quad \sigma(x, t):=\theta(t) e^{10 \lambda|\eta|_\infty}-\xi(x, t).
		\end{split}
	\end{equation*}
	In what follows, $C>0$ represents a generic constant, and $w$ denotes a solution of equation \eqref{3.1}. We can assume, using standard arguments, that $w$ possesses sufficient regularity. Specifically, we consider $w \in H^1(0,T; \mathcal{H}_0^1(\Omega))$ with homogeneous Dirichlet boundary conditions and $w \in H^1(0,T; \mathcal{H}^1(\Omega))$ with homogeneous Neumann boundary conditions.
	
	
	Consider $s>s_0>0$ and introduce
	\begin{equation*}
		u=e^{-s\sigma} w.  
	\end{equation*}
	Then, the following properties hold for $u$:
	\begin{itemize}
		\item [($i$)] $u=\frac{\partial u}{\partial x_i}=0$ at $t=0$ and $t=T$;
		\item [($ii$)] $u=0$ or $A\nabla u \cdot \nu =0 $ on $\Sigma$;
		\item [($iii$)] If $P_1 u:=u_t+s \Div(u A \nabla \sigma)+s \nabla \sigma A \nabla u$ and $P_2 u:=\Div(A \nabla u)+s^2 u \nabla \sigma A \nabla \sigma+s \sigma_t u$, then $P_1 u+P_2 u=e^{-s \sigma} f$.
	\end{itemize}
	From item ($iii$), it follows that
	\begin{equation}\label{PP}
		\left\|P_1 u\right\|^2+\left\|P_2 u\right\|^2+2\left(P_1 u, P_2 u\right)=\left\|e^{-s \sigma} f\right\|^2.
	\end{equation}
	
	Let us define $\left(P_1 u, P_2 u\right)=I_1+\cdots+I_4$, where
	\begin{equation*}
		\begin{split}
			& I_1:=\left(\Div(A \nabla u)+s^2 u \nabla \sigma \cdot A \nabla \sigma+s \sigma_t u, u_t\right), \\
			& I_2:=s^2\left(\sigma_t u, \Div(u A \nabla \sigma)+\nabla \sigma A \nabla u\right), \\
			&I_3:=s^3\left( u \nabla \sigma \cdot A \nabla \sigma, \Div(u A \nabla \sigma)+\nabla \sigma \cdot A \nabla u\right),\\
			& I_4:=s(\Div(A \nabla u), \Div(u A \nabla \sigma)+\nabla \sigma \cdot A \nabla u).
		\end{split}
	\end{equation*}
	By differentiating with respect to the time variable $t$, we can improve the regularity of $u$, leading to $u_t \in \mathcal{H}_0^1(\Omega)$. From Theorem \ref{div} and item (i), we have
	\begin{equation}\label{I11}
		\begin{split}
			I_1=&\iint_Q u_t \Div (A\nabla u) +s^2 u \nabla \sigma \cdot A\nabla \sigma u_t + s\sigma_t u u_t dx  dt\\
			=&\int_{\Om} s^2  \nabla \sigma A\nabla \sigma\cdot \frac{1}{2}u^2 dx  \bigg|_0^T + \int_{\Om} s\sigma_t\cdot \frac{1}{2}u^2 dx  \bigg|_0^T+ \iint_\Sigma u_t A\nabla u \cdot \nu ds dt\\
			&-\iint_{Q}  A\nabla u \cdot \nabla u_t dx  dt -\frac{1}{2}\iint_{Q}(s\sigma_t + s^2 \nabla \sigma \cdot A\nabla \sigma)_t u^2 dx  dt\\
			=& - \frac{1}{2}\int_{\Om}  A\nabla u \cdot \nabla u dx  \bigg|_0^T -\frac{1}{2}\iint_{Q}(s\sigma_t + s^2 \nabla \sigma \cdot A\nabla \sigma)_t u^2 dx  dt\\
			=& -\frac{1}{2}\iint_{Q}(s\sigma_t + s^2 \nabla \sigma \cdot A\nabla \sigma)_t u^2 dx  dt.
		\end{split}
	\end{equation}
	Indeed, we have $\iint_\Sigma u_t A\nabla u \cdot \nu ds dt=0$ for the Dirichlet boundary condition since $u_t \in \mathcal{H}_0^1(\Omega)$, and $\iint_\Sigma u_t A\nabla u \cdot \nu ds dt=0$ for the Neumann boundary condition since $A\nabla u\cdot \nu=0$. Furthermore, we obtain
	\begin{equation}\label{I22}
		\begin{split}
			I_2  =& s^2  \iint_Q \sigma_t u (\Div(u A \nabla \sigma) + A\nabla u \cdot \nabla \sigma) dx  dt
			= s^2  \iint_Q \sigma_t u (\Div( A \nabla \sigma) + 2 A\nabla u \cdot \nabla \sigma) dx  dt\\
			=& s^2  \iint_Q \sigma_t u \Div( A \nabla \sigma) + \Div( A \nabla \sigma) \sigma_t u^2  dx  dt\\
			=& s^2 \iint_\Sigma \sigma_t u^2 A \nabla \sigma  \cdot \nu dsdt - s^2  \iint_Q \Div(\sigma_t A \nabla \sigma)u^2 dx dt + s^2  \iint_Q \Div( A \nabla \sigma) \sigma_t u^2 dx  dt\\
			=& - s^2  \iint_Q \Div( A \nabla \sigma) \sigma_t u^2 dx  dt -s^2\iint_Q  A \nabla \sigma \cdot\nabla \sigma_t u^2    dx  dt + s^2\iint_Q \Div( A \nabla \sigma) \sigma_t u^2 dx dt\\
			=& -s^2\iint_Q  A \nabla \sigma \cdot\nabla \sigma_t u^2    dx  dt.
		\end{split}
	\end{equation}
	In the fourth equality, we can observe that $s^2 \iint_\Sigma \sigma_t u^2 A \nabla \sigma \cdot \nu dsdt=0$ for the Dirichlet boundary condition, as $u \in \mathcal{H}_0^1(\Omega)$, and $s^2 \iint_\Sigma \sigma_t u^2 A \nabla \sigma \cdot \nu dsdt=0$ for the Neumann boundary condition, as $\nabla \eta=0$ on $\partial \Omega$.
	
	Similarly, in $I_3$ below, $s^3 \iint_\Sigma u^2 (A \nabla \sigma \cdot \nabla \sigma) A \nabla \sigma \cdot \nu ds dt=0$ holds for the Dirichlet boundary condition due to $u \in \mathcal{H}_0^1(\Omega)$, and $s^3 \iint_\Sigma u^2 (A \nabla \sigma \cdot \nabla \sigma) A \nabla \sigma \cdot \nu ds dt=0$ holds for the Neumann boundary condition as $\nabla \eta=0$ on $\partial \Omega$.
	In Equation (\ref{I33}), we have
	\begin{equation}\label{I33}
		\begin{split}
			I_3  =&s^3 \iint_Q u A \nabla \sigma \cdot \nabla \sigma (\Div (u A \nabla \sigma) + A\nabla u \cdot \nabla \sigma ) dx  dt\\
			=&s^3 \iint_\Sigma u^2 (A \nabla \sigma \cdot \nabla \sigma) A \nabla \sigma \cdot \nu  ds dt -s^3 \iint_Q u A \nabla \sigma \cdot \nabla (u A \nabla \sigma \cdot \nabla \sigma  ) dx  dt\\
			&+ s^3 \iint_Q  (A \nabla \sigma \cdot \nabla u) ( A \nabla \sigma \cdot \nabla \sigma  ) u dx  dt\\
			=&-s^3\iint_Q  A\nabla \sigma \cdot \nabla ( A\nabla \sigma \cdot\nabla \sigma) u^2 dx  dt.
		\end{split}
	\end{equation}
	Similarly, for $I_4$, we have
	\begin{equation*}
		\begin{split}
			I_4= & s \iint_Q \Div(A \nabla u)\left(  \Div(u A \nabla \sigma)+ A \nabla u \cdot \nabla \sigma\right)  dx  dt \\%1
			= & s \iint_Q \Div(A \nabla u)\left( A \nabla u \cdot \nabla \sigma+ u \Div(A \nabla \sigma)+ A \nabla u \cdot \nabla \sigma\right)  dx  dt \\%2
			= & s \iint_Q \Div(A \nabla u) u \Div(A \nabla \sigma)dx  dt + 2s \iint_Q \Div(A \nabla u) A \nabla u \cdot \nabla \sigma   dx  dt \\%3
			= & s \iint_\Sigma u \Div(A \nabla \sigma )A \nabla u \cdot \nu dsdt + 2s \iint_\Sigma  (A \nabla u \cdot \nabla \sigma) A \nabla u \cdot \nu ds dt \\%4
			&-s \iint_Q A \nabla u \cdot  \nabla\left( u \Div(A \nabla \sigma)\right) dx  dt -2s \iint_{Q} A \nabla u\cdot \nabla \left( A \nabla u \cdot \nabla \sigma \right)  dx  dt \\%5
			= &
			-s \iint_Q A \nabla u \cdot  \nabla u \Div(A \nabla \sigma)dx  dt-s \iint_Q u A \nabla u\cdot \nabla (\Div(A \nabla \sigma)) dx  dt\\%6
			&-2s \iint_{Q} A \nabla u\cdot \nabla(A \nabla u \cdot \nabla \sigma)   dx  dt, 
		\end{split}
	\end{equation*}
	here, in the forth equality, $s \iint_\Sigma u \Div(A \nabla \sigma )A \nabla u \cdot \nu dsdt=0$ can be found in the Dirichlet boundary condition since $u \in \mathcal{H}_0^1(\Om)$ and $s \iint_\Sigma u \Div(A \nabla \sigma )A \nabla u \cdot \nu dsdt=0$ in the Neumann boundary condition since $\nabla \eta=0$ on $\partial \Om$.
	Since
	\begin{equation*}
		\begin{split}
			&-2s \iint_{Q} A \nabla u \cdot \nabla (A\nabla u \cdot \nabla \sigma) dx dt\\
			=&-2s \sum_{i=1}^{2} \iint_{Q} (A\nabla \sigma)_i A \nabla u \cdot \frac{\partial}{\partial x_i}(\nabla u) + (\nabla u)_i A \nabla u \cdot \nabla (A \nabla \sigma)_i dx dt,\\
			%=&-2s \iint_{Q} A \nabla u \left[ y^{\alpha_y} \frac{\partial \sigma}{\partial x}\nabla \frac{\partial u}{\partial x}  + x^{\alpha_x} \frac{\partial \sigma}{\partial y}\nabla \frac{\partial u}{\partial y} + \frac{\partial u}{\partial x} \nabla (y^{\alpha_y} \frac{\partial \sigma}{\partial x}) + \frac{\partial u}{\partial y} \nabla (x^{\alpha_x} \frac{\partial \sigma}{\partial y}) \right] dx dt,
		\end{split}
	\end{equation*}
	where
	\begin{equation*}
		\begin{split}
			&-2s \sum_{i=1}^{2} \iint_{Q} (A\nabla \sigma)_i A \nabla u \cdot \frac{\partial}{\partial x_i}(\nabla u) dx dt\\%1
			%=& -2s \iint_Q y^{\alpha_y} \frac{\partial \sigma}{\partial x}  A \nabla u \cdot \frac{\partial}{\partial x}(\nabla u) dx dt - 2s \iint_Q x^{\alpha_x} \frac{\partial \sigma}{\partial y}  A \nabla u \cdot \frac{\partial}{\partial y}(\nabla u) dx dt\\%2
			=& -s \sum_{i=1}^{2}\iint_{Q} (A\nabla \sigma)_i \left[ \frac{\partial}{\partial x_i} (A \nabla u \cdot \nabla u) - \frac{\partial A}{\partial x_i} \nabla u \cdot \nabla u \right] dx dt\\%2
			=&-s \iint_\Sigma (A \nabla u \cdot \nabla u)(A\nabla \sigma) \cdot \nu dsdt + s\iint_{Q} (A \nabla u \cdot \nabla u ) \Div(A\nabla \sigma) dx dt\\
			&+s \sum_{i=1}^{2} \iint_{Q} (A\nabla \sigma)_i \frac{\partial A}{\partial x_i} \nabla u \cdot \nabla u dx dt\\%4
			=&  s \iint_{Q} A\nabla u \cdot \nabla u \Div(A\nabla \sigma) dx dt
			+ s  \sum_{i=1}^{2}\iint_{Q} (A\nabla \sigma)_i \frac{\partial A}{\partial x_i} \nabla u \cdot \nabla u dx dt, 
		\end{split}
	\end{equation*}
	here $-s \iint_\Sigma (A \nabla u \cdot \nabla u)(A\nabla \sigma) \cdot \nu dsdt=0$ on $\partial \Om$ since $\nabla \eta=0$ on $\partial \Om$.
	
	
	Hence, we have
	\begin{equation}\label{I44}
		\begin{split}
			I_4 = & -s \iint_Q u A \nabla u\cdot \nabla (\Div(A \nabla \sigma))dx  dt
			-2s  \sum_{i=1}^{2}\iint_{Q} (\nabla u)_i A \nabla u \cdot \nabla (A \nabla \sigma)_i dx dt  \\
			&
			+ s  \sum_{i=1}^{2}\iint_{Q} (A\nabla \sigma)_i \frac{\partial A}{\partial x_i} \nabla u \cdot \nabla u dx dt.
		\end{split}
	\end{equation}
	By combining \eqref{I11} to \eqref{I44}, we can conclude that
	\begin{equation}\label{PPu}
		\begin{aligned}
			&\left(P_1 u, P_2 u\right)\\
			= & -s^3 \iint_Q A \nabla \sigma\cdot \nabla(\nabla \sigma A \nabla \sigma)|u|^2 dx  dt
			-2s  \sum_{i=1}^{2}\iint_{Q} (\nabla u)_i A \nabla u \cdot \nabla (A \nabla \sigma)_i dx dt \\
			& -2s^2 \iint_Q \nabla \sigma A \nabla \sigma_t|u|^2 dx  dt+ s \sum_{i=1}^{2} \iint_{Q} (A\nabla \sigma)_i \frac{\partial A}{\partial x_i} \nabla u \cdot \nabla u dx dt\\
			&
			-s \iint_Q u A \nabla u \nabla(\Div(A \nabla \sigma)) dx  dt-\frac{s}{2} \iint_Q \sigma_{t t}|u|^2 dx  dt.
			%+2 s\iint_{\Sigma}(\nabla u A \nabla \sigma) A \nabla u \cdot \nu d s d t -s\iint_{\Sigma}(\nabla u A \nabla u) A \nabla \sigma \cdot \nu d s d t .
		\end{aligned}
	\end{equation}
	Let us denote the seven integrals on the right-hand side of \eqref{PPu} by $T_1, \cdots, T_6$. Now, we will estimate each of them.
	% For the integral on the boundary we will use the following result.
	%\begin{equation*}%需验证
	%2 s\iint_{\Sigma}(\nabla u A \nabla \sigma) A \nabla u \cdot \nu d s d t -s\iint_{\Sigma}(\nabla u A \nabla u) A \nabla \sigma \cdot \nu d s d t  \geq 0 .
	%\end{equation*}
	Using the definitions of $\sigma$ and $\xi$ and the properties of $\eta$, we have $\nabla \sigma = - \lambda \xi \nabla \eta$, $\nabla \xi = \lambda \xi \nabla \eta$, $A \nabla \sigma \cdot \nabla \sigma = \lambda^2 \xi^2 A\nabla \eta \cdot \nabla \eta$, and $\nabla( A \nabla \sigma \cdot \nabla \sigma) = \lambda^2 \xi^2 \nabla (A\nabla \eta \cdot \nabla \eta) + 2\lambda^2 \xi^2 \nabla \eta \left| A \nabla \eta \cdot \nabla \eta \right|$. Therefore, we can rewrite $T_1$ as follows:
	\begin{equation}
		\begin{aligned}
			T_1
			&=-s^3 \iint_Q A \nabla \sigma \nabla(\nabla \sigma A \nabla \sigma)|u|^2 dx  dt\\
			&=-s^3 \iint_Q  \left( -\lambda \xi A \nabla \eta \right) \left( \lambda^2\xi^2 \nabla(\nabla \eta A \nabla \eta) + 2\lambda^3\xi^2 \nabla \eta \left| A \nabla \eta \cdot \nabla \eta \right| ^2\right) |u|^2 dx  dt\\
			&=2s^3 \lambda^4 \iint_Q \xi^3\left|A \nabla \eta \cdot \nabla \eta \right|^2|u|^2 dx  dt
			+ s^3 \lambda^3 \iint_Q \xi^3 A\nabla \eta \cdot \nabla(A\nabla \eta \cdot \nabla\eta)|u|^2 dx  dt.
		\end{aligned}
	\end{equation}
	Since $A\nabla \eta \cdot\nabla(A\nabla \eta \cdot \nabla\eta)$ is bounded in $\overline{\Omega}$, we have
	$$
	s^3 \lambda^3 \int_{0}^{T}\int_{\omega} \xi^3 A\nabla \eta \cdot \nabla(A\nabla \eta \cdot \nabla\eta)|u|^2 dx  dt \ge -C s^3 \lambda^3 \int_{0}^{T}\int_{\omega} \xi^3 |u|^2 dx  dt,
	$$
	and $|A \nabla \eta \cdot \nabla \eta|\ge C>0$ in $\overline{\Omega} \setminus \omega$. Thus, we have
	$$
	\begin{aligned}
		&s^3 \lambda^3 \int_{0}^{T}\int_{\Om\backslash \omega} \xi^3 A\nabla \eta \cdot \nabla(A\nabla \eta \cdot \nabla\eta)|u|^2 dx  dt\\
		\ge& -C s^3 \lambda^3 \int_{0}^{T}\int_{\Om\backslash \omega} \xi^3 |u|^2 dx  dt\\
		\ge&
		-C s^3 \lambda^3 \int_{0}^{T}\int_{\Om\backslash \omega} \xi^3 \left|A \nabla \eta \cdot \nabla \eta \right|^2 |u|^2 dx  dt\\
		\ge&
		-C s^3 \lambda^3 \iint_{Q} \xi^3 \left|A \nabla \eta \cdot \nabla \eta \right|^2 |u|^2 dx  dt.
	\end{aligned}
	$$
	Hence, we can rewrite $T_1$ as:
	\begin{equation}\label{T1}
		\begin{aligned}
			T_1
			\geq C s^3 \lambda^4 \iint_Q \xi^3\left|A \nabla \eta \cdot \nabla \eta \right|^2|u|^2 dx  dt-C s^3 \lambda^3\int_0^T \int_{\omega} \xi^3|u|^2 dx  dt.
		\end{aligned}
	\end{equation}
	Similarly, for $T_2$, we have:
	\begin{equation}\label{T2}
		\begin{aligned}
			T_2 =&-2s \sum_{i=1}^{2} \iint_{Q} (\nabla u)_i A \nabla u \cdot \nabla (A \nabla \sigma)_i dx dt\\
			=&-2s \iint_{Q}( D(A\nabla \sigma) A\nabla u) \cdot \nabla u dx  dt\\
			=&2s \iint_{Q}( \lambda^2 \xi (A\nabla \eta \cdot \nabla u) A\nabla \eta + \lambda \xi D(A\nabla \eta) A\nabla u ) \cdot \nabla u dx  dt\\
			=&2s \lambda^2\iint_{Q}  \xi \left|  A\nabla \eta \cdot \nabla u \right| ^2 dx  dt  + 2s\lambda \iint_{Q}\xi D(A\nabla \eta) A\nabla u  \cdot \nabla u dx  dt\\
			\ge& Cs \lambda^2\iint_{Q}  \xi \left|  A\nabla \eta \cdot \nabla u \right| ^2 dx  dt  - C s\lambda \iint_{Q}\xi A\nabla u  \cdot \nabla u dx  dt.
		\end{aligned}
	\end{equation}
	Furthermore, for $T_3$, as can be seen from the definition of $\xi$, we have $\xi \xi_t \le \xi^3$, thus
	\begin{equation*}
		\begin{aligned}
			T_3 &=-2 s^2 \iint_{Q} A\nabla \sigma \cdot \nabla \sigma_t u^2 dx dt
			=-2 s^2 \lambda^2 \iint_{Q} \xi \xi_t \left|A \nabla \eta \cdot \nabla \eta \right|  u^2 dx dt\\
			&\ge-2 s^2 \lambda^2 \iint_{Q} \xi^3 \left|A \nabla \eta \cdot \nabla \eta \right|^2  u^2 dx dt.
		\end{aligned}
	\end{equation*}
	Similarly, we can express $T_3$ as:
	\begin{equation}\label{T3}
		\begin{aligned}
			T_3
			\ge &-C s^2 \lambda^2 \iint_{Q} \xi^3 \left|A \nabla \eta \cdot \nabla \eta \right|^2  u^2 dx dt -C s^2 \lambda^2 \int_{0}^{T}\int_{\omega} \xi^3   u^2 dx dt,
		\end{aligned}
	\end{equation}
	and $T_4$ as:
	\begin{equation}\label{T4}
		\begin{aligned}
			T_4 =&s \iint_{Q} (A\nabla \sigma)_i \frac{\partial A}{\partial x_i} \nabla u \cdot \nabla u dx dt
			\ge  -C s \lambda \iint_Q  \xi A \nabla u \cdot \nabla u dx  dt.
		\end{aligned}
	\end{equation}
	Furthermore, by utilizing the definitions of $\sigma$ and $\xi$, we obtain:
	\begin{equation}\label{T5}
		\begin{split}
			T_5  =&-s \iint_Q u A \nabla u \cdot \nabla(\Div(A \nabla \sigma)) dx  dt\\
			=&s \lambda^3 \iint_Q \xi u A \nabla u \cdot \nabla \eta \left( A \nabla \eta \cdot \nabla \eta\right)  dx  dt
			+ s \lambda^2 \iint_Q \xi u A \nabla u \cdot \nabla\left(A \nabla \eta \cdot \nabla \eta\right) dx  dt \\
			& +s \lambda^2 \iint_Q \xi u A \nabla u \cdot \nabla \eta \Div(A \nabla \eta)   dx  dt+s \lambda \iint_Q \xi u A \nabla u \cdot \nabla(\Div(A \nabla \eta)) dx  dt.
		\end{split}
	\end{equation}
	Let us denote $T_{51}, \cdots, T_{54}$ as the seven integrals on the right-hand side of \eqref{T5}. Then we have:
	\begin{equation}\label{T51}
		\begin{split}
			T_{51}=&s \lambda^3 \iint_Q \xi u A \nabla u \cdot \nabla \eta \left( A \nabla \eta \cdot \nabla \eta\right)  dx  dt\\
			\geq&
			-Cs^2 \lambda^4 \iint_Q \xi\left|A \nabla \eta \cdot \nabla \eta\right|^2|u|^2 dx  dt
			-C\lambda^2 \iint_Q \xi| A\nabla u \cdot\nabla \eta|^2 dx  dt,
		\end{split}
	\end{equation}
	and
	\begin{equation}\label{T52}
		\begin{split}
			T_{52}=&s \lambda^2 \iint_Q \xi u A \nabla u \cdot \nabla\left(A \nabla \eta \cdot \nabla \eta\right) dx  dt\\
			\ge&-Cs^2 \lambda^3\iint_{Q} \xi \left| A \nabla \eta \cdot \nabla \eta \right| u^2 dx  dt -Cs^2 \lambda^3\int_0^T \int_{\omega} \xi \left| A \nabla u \cdot \nabla u \right|  u^2 dx  dt\\
			&- C\lambda \iint_Q  \xi A \nabla u \cdot \nabla u dx  dt.
		\end{split}
	\end{equation}
	Then we have
	\begin{equation}\label{T53}
		\begin{split}
			T_{53}=&s \lambda^2 \iint_Q \xi u A \nabla u \cdot \nabla \eta  \Div(A \nabla \eta) dx  dt\\
			\ge& -Cs^2 \lambda^3\iint_Q \xi \left| A \nabla \eta \cdot \nabla \eta \right|^2  u^2  dx  dt -Cs^2 \lambda^3\int_0^T \int_{\omega} \xi u^2 dx  dt\\
			&- C\lambda \iint_Q \xi A \nabla u \cdot \nabla u dx  dt,
		\end{split}
	\end{equation}
	and
	\begin{equation}\label{T54}
		\begin{split}
			T_{54}=&s \lambda \iint_Q \xi u A \nabla u \nabla(\Div(A \nabla \eta))dx  d t\\
			\ge&
			-C s^2 \lambda \iint_Q \xi \left|A \nabla \eta \cdot \nabla \eta\right|^2 u^2dx  d t-C s^2 \lambda \int_0^T \int_{\omega}\xi  u^2dx  d t\\
			&-C \lambda \iint_Q \xi A \nabla u \cdot \nabla udx  d t.
		\end{split}
	\end{equation}
	Combining \eqref{T5} to \eqref{T54}, we obtain:
	\begin{equation}\label{T55}
		\begin{split}
			T_5 \geq&-Cs^2 \lambda^4 \iint_Q \xi\left|A \nabla \eta \cdot \nabla \eta\right|^2 |u|^2dx  d t -C\lambda^2\iint_{Q} \xi \left| A\nabla u\cdot \nabla \eta \right| ^2 dx  dt\\
			&-C s^2 \lambda^3 \int_0^T \int_{\omega} \xi|u|^2dx  d t- C\lambda \iint_Q  \xi A \nabla u \cdot \nabla u dx  d t.
		\end{split}
	\end{equation}
	Finally, as we can observe from the definitions of $\xi$ and $\sigma$, we have $\sigma_{tt} \leq \xi^{\frac{3}{2}}$. Hence,
	\begin{equation}\label{T6}
		T_6 =-\frac{s}{2} \iint_Q \sigma_{t t}|u|^2dx  d t \geq-C s \iint_Q \xi^{3 / 2}|u|^2dx  d t.
	\end{equation}
	From equations \eqref{T1}-\eqref{T4}, \eqref{T55}, and \eqref{T6}, we deduce the following inequality:
	\begin{equation}\label{PPU}
		\begin{split}
			\left(P_1 u, P_2 u\right)
			\geq& C\iint_Q s^3 \lambda^4 \xi^3 |\nabla \eta \cdot A \nabla \eta|^2 |u|^2 dx dt  + C s  \lambda^2 \iint_Q\xi|\nabla u \cdot A \nabla \eta|^2 dx  d t\\
			&-Cs^3 \lambda^3 \int_0^T \int_{\omega} \xi^3|u|^2dx  d t
			- Cs \lambda \iint_Q \xi A \nabla u \cdot \nabla udx  d t\\
			&- Cs \iint_Q \xi^{3 / 2}|u|^2dx  d t.
		\end{split}
	\end{equation}
	Clearly, $s \iint_Q \xi^{3 / 2}|u|^2dx dt$ can be absorbed by other terms. Combining \eqref{PP} and \eqref{PPU}, we can conclude that
	\begin{equation}\label{LLU2}
		\begin{split}
			&\left\|P_1 u\right\|^2+\left\|P_2 u\right\|^2+C\iint_Q s^3 \lambda^4 \xi^3\left|A \nabla \eta \cdot \nabla \eta \right|^2|u|^2 dx dt  + C s  \lambda^2 \iint_Q\xi|\nabla u \cdot A \nabla \eta|^2 dx  d t\\
			\leq& C\left\|e^{-s \sigma} f\right\|^2
			+Cs \lambda \iint_Q \xi A \nabla u \cdot \nabla udx  d t+Cs^3 \lambda^3 \int_0^T \int_{\omega} \xi^3|u|^2dx  d t.
		\end{split}
	\end{equation}
	Furthermore, we can deduce that
	\begin{equation*}
		s^2 \lambda^2 \iint_Q \xi^{2}|u|^2dx  d t=s^2 \lambda^2 \int_0^T \int_{\Om\backslash \omega} \xi^{2}|u|^2dx  d t + s^2 \lambda^2 \int_0^T \int_{\omega} \xi^{2}|u|^2dx  d t.
	\end{equation*}
	Clearly, the second term on the right in \eqref{LLU2} can be absorbed by the other terms. Let us now consider the first term on the right, where we have
	\begin{equation*}
		\begin{split}
			s^2 \lambda^2 \int_0^T \int_{\Om\backslash \omega} \xi^{2}|u|^2dx  d t
			&\le Cs^2 \lambda^2 \int_0^T \int_{\Om\backslash \omega} \xi^{2}\left|A \nabla \eta \cdot \nabla \eta \right|^2|u|^2dx  d t\\
			&\le Cs^2 \lambda^2 \iint_Q \xi^{2}\left|A \nabla \eta \cdot \nabla \eta \right|^2|u|^2dx  d t.
		\end{split}
	\end{equation*}
	Therefore, we have
	\begin{equation}\label{LLU3}
		\begin{split}
			&\left\|P_1 u\right\|^2+\left\|P_2 u\right\|^2+C s^2 \lambda^2 \int_0^T \int_{\Om\backslash \omega} \xi^{2}|u|^2dx  d t\\
			&+  C\iint_Q s^3 \lambda^4 \xi^3\left|A \nabla \eta \cdot \nabla \eta \right|^2|u|^2 dx dt  + C s  \lambda^2 \iint_Q\xi|\nabla u \cdot A \nabla \eta|^2 dx  d t\\
			\leq& C\left\|e^{-s \sigma} f\right\|^2
			+Cs \lambda \iint_Q \xi A \nabla u \cdot \nabla udx  d t+Cs^3 \lambda^3 \int_0^T \int_{\omega} \xi^3|u|^2dx  d t.
		\end{split}
	\end{equation}
	Using the definitions of $P_1 u$ and $P_2 u$, we can observe that
	\begin{equation*}
		\begin{split}
			&s^{-1} \iint_Q \xi^{-1}\left|u_t\right|^2dx  d t\\
			=&s^{-1} \iint_Q \xi^{-1}(P_1 u -s u \Div(A\nabla \sigma)- 2s \nabla u \cdot A\nabla \sigma)^2dx  d t\\
			\le& Cs^{-1}\left\|P_1 u\right\|^2+s   \iint_Q \xi^{-1}|u|^2|\Div(A \nabla \sigma)|^2dx  d t +C s \iint_Q \xi^{-1}|\nabla u A \nabla \sigma|^2dx  d t \\
			\leq& Cs^{-1}\left\|P_1 u\right\|^2+C s \lambda^4 \iint_Q \xi\left| A \nabla \eta \cdot \nabla \eta \right|^2 |u|^2dx  d t \\
			&+C s \lambda^2  \iint_Q \xi|u|^2dx  d t+C s \lambda^2 \iint_Q \xi|\nabla u \cdot A \nabla \eta|^2dx  d t,
		\end{split}
	\end{equation*}
	and
	\begin{equation*}
		\begin{split}
			&s^{-1}   \iint_Q \xi^{-1}\left|\Div(A\nabla u)\right|^2dx  d t\\
			=& s^{-1}   \iint_Q \xi^{-1}(P_2 u -s^2 u \nabla \sigma \cdot A \nabla \sigma - s \sigma_t u)^2dx  d t\\
			\leq& C s^{-1} \left\|P_2 u\right\|^2+s^3   \iint_Q \xi^{-1}|u|^2\left|\nabla \sigma \cdot A \nabla \sigma\right|^2dx  d t +C s   \iint_Q \xi^{-1}\left|\sigma_t\right|^2|u|^2dx  d t \\
			\leq& C s^{-1} \left\|P_2 u\right\|^2+C s^3 \lambda^4   \iint_Q \xi^3|\nabla \eta \cdot A \nabla \eta|^2|u|^2dx  d t +C s   \iint_Q \xi^{2}|u|^2dx  d t.
		\end{split}
	\end{equation*}
	From equation \eqref{LLU3}, we obtain the inequality
	\begin{equation}\label{LLU4}
		\begin{split}
			&s^{-1}\iint_{Q} \xi^{-1} (|u_t |^2 + \left|\Div(A\nabla u)\right|^2) dx dt +C s^2 \lambda^2 \int_0^T \int_{\Om\backslash \omega} \xi^{2}|u|^2dx  d t\\
			&+C\iint_Q s^3 \lambda^4 \xi^3\left|A \nabla \eta \cdot \nabla \eta \right|^2|u|^2 dx dt  + C s  \lambda^2 \iint_Q\xi|\nabla u \cdot A \nabla \eta|^2 dx  d t\\
			\leq& C\left\|e^{-s \sigma} f\right\|^2
			+Cs \lambda \iint_Q \xi A \nabla u \cdot \nabla udx  d t+Cs^3 \lambda^3 \int_0^T \int_{\omega} \xi^3|u|^2dx  d t.
		\end{split}
	\end{equation}
	Considering the term $s \lambda \iint_Q \xi A \nabla u \cdot \nabla udx d t$, we have
	\begin{equation*}
		\begin{split}
			s \lambda \iint_Q \xi A \nabla u \cdot \nabla udx  d t=& s \lambda \iint_\Sigma \xi u A \nabla u \cdot \nu d s d t - s \lambda \iint_Q \Div(\xi A\nabla u) udx  d t\\
			=&- s \lambda \iint_Q \Div(\xi A\nabla u) udx  d t.
		\end{split}
	\end{equation*}
	Then, we can further simplify the expression as follows:
	\begin{equation*}
		\begin{split}
			&- s \lambda \iint_Q \Div(\xi A\nabla u) udx  d t\\
			=&- s \lambda^2 \iint_Q \xi u\nabla \eta \cdot A\nabla u dx  d t - s \lambda \iint_Q \xi\Div( A\nabla u) udx  d t\\
			\le& C \lambda^2 \iint_Q \xi \left|  \nabla \eta \cdot A\nabla u \right| ^2dx  d t + Cs^2 \lambda^2 \iint_Q \xi u^2dx  d t\\
			&+C s^{-1} \lambda^{-1} \iint_Q \xi^{-1} \left| \Div( A\nabla u)\right| ^2dx  d t + Cs^3 \lambda^3 \iint_Q \xi^3 u^2dx  d t.
		\end{split}
	\end{equation*}
	Thus, the term $s \lambda \iint_Q \xi A \nabla u \cdot \nabla udx d t$ can be absorbed by the other terms. This implies that:
	\begin{equation}\label{LLU5}
		\begin{split}
			&s^{-1}\iint_{Q} \xi^{-1} (|u_t |^2 + \left|\Div(A\nabla u)\right|^2) dx dt +C s^2 \lambda^2 \int_0^T \int_{\Om\backslash \omega} \xi^{2}|u|^2dx  d t\\
			&+C\iint_Q s^3 \lambda^4 \xi^3\left|A \nabla \eta \cdot \nabla \eta \right|^2|u|^2 dx dt  + C s  \lambda^2 \iint_Q\xi|\nabla u \cdot A \nabla \eta|^2 dx  d t\\
			\leq& C\left\|e^{-s \sigma} f\right\|^2
			+Cs^3 \lambda^3 \int_0^T \int_{\omega} \xi^3|u|^2dx  d t.
		\end{split}
	\end{equation}
	By employing classical arguments, we can then revert back to the original variable $w$ and conclude the result.
	
	\section{A specific example}
	
	
	As stated earlier in this paper, the well-posedness of this example can be easily verified. In the following, we provide some results regarding the improvement of interior regularity and Carleman estimates.
	\subsection{Internal regularity improvement}
	
	\hspace*{\fill}\\
	
	Having established the existence of solutions to equation \eqref{4.1}, we now focus on demonstrating that the regularity of the solutions to \eqref{adjoint} can be enhanced within the interior of $\Omega$. Specifically, we can achieve improved regularity by differentiating with respect to the time variable $t$, resulting in $u_t \in \mathcal{H}_0^1(\Omega)$.
	\begin{theorem}\label{regularity}
		For all the solution of equation \eqref{adjoint}, one has $y^{\alpha_y}|\partial_{xy} w|^2, y^{\alpha_y}|\partial_{xx} w|^2,\\ x^{\alpha_x}|\partial_{xy} w|^2$, and $ x^{\alpha_x}|\partial_{yy} w|^2 \in L^1_{\rm loc}(\Om)$.
	\end{theorem}
	\begin{proof}
		Let us introduce a function $\psi = \psi(x,y)$ satisfying the following properties:
		\begin{equation*}
			\psi \in C_0^\infty (\Om), \ \psi =1 \ \mbox{in} \ \Om^{2\epsilon}, \ \psi =0 \ \mbox{in} \ \Om_{\epsilon}, \ \mbox{and} \ 0\le \psi \le 1,
		\end{equation*}
		where $\Omega^{2\epsilon} := (2\epsilon, 1-2\epsilon) \times (2\epsilon, 1-2\epsilon)$ and $\Omega_{\epsilon} := \Omega \setminus \Omega^{\epsilon}$.
		Let us define the operators as follows:
		\begin{equation*}
			D_x^h (w):= \frac{w(x+h,y)-w(x,y)}{h}, \ \mbox{and} \ v:= D_x^{-h}(\psi^2 D_x^h (w)).
		\end{equation*}
		Multiplying the equation in \eqref{adjoint} by $v$ and integrating over $\Omega$, we obtain
		\begin{equation*}
			\int_{\Om^{\epsilon}} v w_t dxdy + \int_{\Om^{\epsilon}} v \Div(A\nabla w) dxdy =\int_{\Om^{\epsilon}} v f dxdy.
		\end{equation*}
		Next, let us consider the second term on the left-hand side:
		\begin{equation*}
			\begin{split}
				&\int_{\Om^{\epsilon}} v \Div(A\nabla w) dxdy=\int_{\Om^{\epsilon}} D_x^{-h}(\psi^2 D_x^h (w)) \Div(A\nabla w) dxdy\\
				=& -\int_{\Om^{\epsilon}} \nabla (D_x^{-h}(\psi^2 D_x^h (w))) \cdot A\nabla w dxdy
				= -\int_{\Om^{\epsilon}} D_x^{-h} (\nabla(\psi^2 D_x^h (w))) \cdot A\nabla w dxdy\\
				=&\int_{\Om^{\epsilon}}  (\nabla(\psi^2 D_x^h (w)))  D_x^{h}(A\nabla w) dxdy
				=\int_{\Om^{\epsilon}}  (2\psi \nabla \psi D_x^h (w)+ \psi^2 \nabla (D_x^h (w)) D_x^{h}(A\nabla w) dxdy\\
				=&\int_{\Om^{\epsilon}}  2\psi \nabla \psi D_x^h (w) D_x^{h}(A\nabla w) dxdy + \int_{\Om^{\epsilon}}
				\psi^2 \nabla (D_x^h (w)) D_x^{h}(A\nabla w) dxdy\\
				=&\int_{\Om^{\epsilon}}  2\psi \nabla \psi D_x^h (w) D_x^{h}(A\nabla w) dxdy+\int_{\Om^{\epsilon}}\psi^2 \nabla (D_x^h (w)) D_x^{h}(A) \nabla w dxdy\\
				&+ \int_{\Om^{\epsilon}}\psi^2 \nabla (D_x^h (w)) A^h \nabla D_x^h (w)  dxdy,
			\end{split}
		\end{equation*}
		where
		\begin{equation}
			A^h := \begin{pmatrix} y^{\alpha_y} & 0 \\ 0 &(x+h)^{\alpha_x} \end{pmatrix},
		\end{equation}
		and
		\begin{equation}\label{re1}
			\begin{split}
				&\int_{\Om^{\epsilon}}\psi^2 \nabla (D_x^h (w)) A^h \nabla D_x^h (w)  dxdy\\
				=&\int_{\Om^{\epsilon}}\psi^2 \left( y^{\alpha_y} |\partial_{x} (D_x^h (w))|^2 + (x+h)^{\alpha_x} |\partial_{y} (D_x^h (w))|^2\right)  dxdy\\
				=&\int_{\Om^{\epsilon}} D_x^{-h}(\psi^2 D_x^h (w)) f dxdy - \int_{\Om^{\epsilon}} D_x^{-h}(\psi^2 D_x^h (w)) w_t dxdy\\
				&-\int_{\Om^{\epsilon}}\psi^2 \nabla (D_x^h (w)) D_x^{h}(A) \nabla w dxdy-\int_{\Om^{\epsilon}}  2\psi \nabla \psi D_x^h (w) D_x^{h}(A\nabla w) dxdy.
			\end{split}
		\end{equation}
		We can now proceed to estimate the terms on the right-hand side. Since we can assume $f \in H^1(\Omega)$, we have
		\begin{equation*}
			\begin{split}
				\int_{\Om^{\epsilon}} D_x^{-h}(\psi^2 D_x^h (w)) f dxdy &= -\int_{\Om^{\epsilon}} \psi^2 D_x^h (w) D_x^h (f) dxdy\\
				&\le \int_{\Om^{\epsilon}} \psi^2 D_x^h (w) D_x^h (f) dxdy\le C \left\| f\right\|_{H^1(\Om^{\epsilon})} + C \left\| w\right\|_{\mathcal{H}_0^1(\Om^{\epsilon})}.
			\end{split}
		\end{equation*}
		Similarly, we obtain 
		\begin{equation*}
			\begin{split}
				\int_{\Om^{\epsilon}} D_x^{-h}(\psi^2 D_x^h (w)) w_t dxdy \le C \left\| w\right\|_{\mathcal{H}_0^1(\Om^{\epsilon})} + C \left\| w_t\right\|_{\mathcal{H}_0^1(\Om^{\epsilon})},
			\end{split}
		\end{equation*}
		and
		\begin{equation*}
			\begin{split}
				-\int_{\Om^{\epsilon}}\psi^2 \nabla (D_x^h (w)) D_x^{h}(A) \nabla w dxdy \le& 
				C\int_{\Om^{\epsilon}}\psi^2 \partial_{y} (D_x^h (w))  \partial_{y} w dxdy\\
				\le& \gamma \int_{\Om^{\epsilon}} \psi^2 x^{\alpha_x} |\partial_{y} (D_x^h (w))|^2 dxdy + \frac{C}{\gamma} \left\| w\right\|_{\mathcal{H}_0^1(\Om^{\epsilon})}.
			\end{split}
		\end{equation*}
		Finally, we estimate the last term as follows:
		\begin{equation*}
			\begin{split}
				&-\int_{\Om^{\epsilon}}  2\psi \nabla \psi D_x^h (w) D_x^{h}(A\nabla w) dxdy\\
				=&-\int_{\Om^{\epsilon}}  2\psi \nabla \psi D_x^h (w) D_x^{h}(A)\nabla w dxdy
				-\int_{\Om^{\epsilon}}  2\psi \nabla \psi D_x^h (w) D_x^{h}(\nabla w)A^h dxdy\\
				\le&  \gamma \int_{\Om^{\epsilon}} \psi^2 x^{\alpha_x} |\partial_{y} (D_x^h (w))|^2 dxdy 
				+\gamma \int_{\Om^{\epsilon}} \psi^2 \nabla (D_x^h (w)) A^h \nabla D_x^h (w)  dxdy+ \frac{C}{\gamma} \left\| w\right\|_{\mathcal{H}_0^1(\Om^{\epsilon})}.
			\end{split}
		\end{equation*}
		Obviously, the first and second terms on the right-hand side can be absorbed by equation \eqref{re1}. Thus, we have 
		\begin{equation*}
			\begin{split}
				\int_{\Om^{\epsilon}} \psi^2 \nabla (D_x^h (w)) A\nabla D_x^h (w)  dxdy\le C\int_{\Om^{\epsilon}} \psi^2 \nabla (D_x^h (w)) A^h \nabla D_x^h (w)  dxdy\le C.
			\end{split}
		\end{equation*}
		Similarly, we obtain
		\begin{equation*}
			\begin{split}
				\int_{\Om^{\epsilon}} \psi^2 \nabla (D_y^h (w)) A\nabla D_y^h (w)  dxdy \le C.
			\end{split}
		\end{equation*}
		This concludes the proof.
	\end{proof}
	\begin{remark}
		It is worth noting that, on the boundary, we cannot improve the regularity due to the specific nature of the equation we are considering. Consequently, on the boundary, we do not have the property $(A \nabla u \cdot \nabla \sigma) A \nabla u, (A\nabla u\cdot\nabla u)A\nabla\sigma\in (W^{1,1}(\Omega))^2$ (see the estimate of $I_4$ for the boundary term in the following section). In other words, the functions $(A \nabla u \cdot \nabla \sigma) A \nabla u$ and $(A\nabla u\cdot\nabla u)A\nabla\sigma$ do not have a trace on $\partial\Omega$. This distinction shapes our research approach significantly differently from \cite{CA5,CA6}. It is also the reason why we choose the control domain as $\omega_0$ in the subsequent subsection.
	\end{remark}
	
	
	\subsection{Carleman esitimates}
	
	\hspace*{\fill}\\
	
	The control system \eqref{4.1} we are studying is a specific case of the previously mentioned \eqref{1.1}. However, the weight function we have chosen in the following analysis is a special weight function that allows for more convenient calculations. Therefore, we need to provide a Carleman estimate that differs slightly from the previous one.
	
	Let us now establish a Carleman estimate. We know that the adjoint equation of \eqref{4.1} is given by
	\begin{equation}\label{4.3}
		\begin{cases}
			\partial_{t}w + \Div(A\nabla w)=f, & \mbox{in} \ Q,  \\
			w(x,y,t)=0, \ \mbox{or} \ A\nabla w \cdot \nu =0, & \mbox{on} \ \Sigma,  \\
			w(x,  y,  T)= w_T,  & \mbox{in} \ \Omega.  
		\end{cases}
	\end{equation} 
	Fix $\delta>0$. Denote $(\delta,1)\times(\delta,1)$ by $\Omega^\delta$, $\Omega\backslash \Omega^\delta$ by $\Omega_\delta$, and $\Sigma_\delta:= \partial \Omega^\delta \times(0,T)$. Let
	\begin{equation*}
		\eta(x,y) :=
		\begin{cases}
			\frac{(x-\delta)^2 (y-\delta)^2 (x-1)^2(y-1)^2}{2}, & x \in \Omega^\delta,\\
			0, & x \in \Omega_\delta.
		\end{cases}
	\end{equation*}
	It is evident that $\eta \in C^2(\overline{\Omega})$, $\eta > 0$ in $\Omega^\delta$, and $\eta = 0$ on $\partial\Omega^\delta$. By utilizing the classical arguments in \cite{CA6}, we can transform $(\frac{1+\delta}{2},\frac{1+\delta}{2})$ to $\omega_0$. As a result, we obtain
	\begin{equation*}
		\left| \nabla \eta\right| \ge C > 0, \ \mbox{in} \ \overline{\Omega \backslash \omega_0},
	\end{equation*}
	where $\omega_0$ is a nonempty open set satisfying $\Omega_\delta \subset \omega_0 \subset \Omega$, and $\Gamma \subset \partial\omega_0$. We define
	\begin{equation*}
		\begin{split}
			& \theta(t):=[t(T-t)]^{-4}, \quad \xi(x, t):=\theta(t) e^{ \lambda(8|\eta|_\infty+\eta (x))}, \quad \sigma(x, t):=\theta(t) e^{10 \lambda|\eta|_\infty}-\xi(x, t).
		\end{split}
	\end{equation*}
	
	In the following discussion, $C>0$ represents a generic constant that depends solely on $T$ and $\alpha_x, \alpha_y$. We assume that $w$ is a sufficiently regular solution of \eqref{4.3}. Moreover, we consider $w \in H^1(0,T; \mathcal{H}_0^1(\Omega))$.
	
	For $s>s_0>0$, we introduce
	\begin{equation*}
		u=e^{-s\sigma} w.  
	\end{equation*}
	Then, the following properties hold:
	\begin{itemize}
		\item [($i$)] $u=\frac{\partial u}{\partial x_i}=0$ at $t=0$ and $t=T$;
		\item [($ii$)] $u=0$ or $A\nabla u \cdot \nu =0 $ on $\Sigma$;
		\item [($iii$)] If $P_1 u:=u_t+s \mathrm{div}(u A \nabla \sigma)+s \nabla \sigma A \nabla u$ and $P_2 u:=\mathrm{div}(A \nabla u)+s^2 u \nabla \sigma A \nabla \sigma+s \sigma_t u$, then $P_1 u+P_2 u=e^{-s \sigma} f$.
	\end{itemize}
	From item ($iii$), it follows that
	\begin{equation}\label{P}
		\left\|P_1 u\right\|^2+\left\|P_2 u\right\|^2+2\left(P_1 u, P_2 u\right)=\left\|e^{-s \sigma} f\right\|^2.
	\end{equation}
	
	We decompose $\left(P_1 u, P_2 u\right)$ into four parts: $I_1$, $I_2$, $I_3$, and $I_4$, defined as follows:
	\begin{equation*}
		\begin{split}
			& I_1:=\left(\Div(A \nabla u)+s^2 u \nabla \sigma \cdot A \nabla \sigma+s \sigma_t u, u_t\right), \\
			& I_2:=s^2\left(\sigma_t u, \Div(u A \nabla \sigma)+\nabla \sigma A \nabla u\right), \\
			&I_3:=s^3\left( u \nabla \sigma \cdot A \nabla \sigma, \Div(u A \nabla \sigma)+\nabla \sigma \cdot A \nabla u\right),\\
			& I_4:=s(\Div(A \nabla u), \Div(u A \nabla \sigma)+\nabla \sigma \cdot A \nabla u).
		\end{split}
	\end{equation*}
	Before proceeding with the calculations, we introduce the following theorem, which will be useful.
	\begin{theorem}\label{DIV}
		For all $v\in \mathcal{H}_0^1(\Omega)$, it holds that
		\begin{equation*}
			\int_\Om v \Div(A\nabla u) dx dy= -\int_\Om \nabla v \cdot A\nabla u dx dy. 
		\end{equation*}
		\begin{proof}
			Let $v_n \in C_0^\infty(\Omega)$ be a sequence converging to $v$ in $H^1(\Omega)$, which is possible due to the density of $C_0^\infty(\Omega)$ in $\mathcal{H}_0^1(\Omega)$. Then, we have
			\begin{align*}
				\begin{split}
					\int_\Om v \Div(A\nabla u) dx dy
					&= \lim\limits_{n\to \infty} \int_\Om v_n \Div(A\nabla u) dx dy=\lim\limits_{n\to \infty}\left(  v_n,\Div(A\nabla u)\right)\\  &=-\lim\limits_{n\to \infty}\left( \nabla v_n, A\nabla u\right).
				\end{split}
			\end{align*}
			Using integration by parts, we expand the above expression as follows:
			\begin{equation*}
				\begin{split}
					-\lim\limits_{n\to \infty}\left( \nabla v_n, A\nabla u\right)=&\lim\limits_{n\to \infty}-\left( \frac{\partial v_n}{\partial x}, y^{\alpha_y}\partial_x u\right)-\left( \frac{\partial v_n}{\partial y}, x^{\alpha_x}\partial_y u\right)\\
					=&\lim\limits_{n\to \infty}-\left(y^{\frac{\alpha_y}{2}} \frac{\partial v_n}{\partial x}, y^{\frac{\alpha_y}{2}}\partial_x u\right)-\left(x^{\frac{\alpha_x}{2}} \frac{\partial v_n}{\partial y}, x^{\frac{\alpha_x}{2}}\partial_y u\right)\\
					=&\lim\limits_{n\to \infty}-\int_{\Om} y^{\alpha_y}\frac{\partial v_n}{\partial x}\frac{\partial u}{\partial x} dx dy - \int_{\Om} x^{\alpha_x}\frac{\partial v_n}{\partial y}\frac{\partial u}{\partial y} dx dy\\
					=&\lim\limits_{n\to \infty}-\int_{\Om} \nabla v_n A \nabla u dx dy =-\int_{\Om} \nabla v A \nabla u dx dy.
				\end{split}
			\end{equation*}
			Thus, the result is established.
		\end{proof}
	\end{theorem}
	From Theorem \ref{DIV} and item (i), we obtain the following expressions:
	\begin{equation}\label{I1}
		\begin{split}
			I_1=&\iint_Q u_t \Div (A\nabla u) +s^2 u \nabla \sigma \cdot A\nabla \sigma u_t + s\sigma_t u u_t dx dy dt\\
			=&\int_{\Om} s^2  \nabla \sigma A\nabla \sigma\cdot \frac{1}{2}u^2  \bigg|_0^T dx dy  + \int_{\Om} s\sigma_t\cdot \frac{1}{2}u^2 \bigg|_0^T dx dy + \iint_\Sigma u_t A\nabla u \cdot \nu ds dt\\
			&-\iint_{Q} A\nabla u \cdot \nabla u_t dx dy dt -\frac{1}{2}\iint_{Q}(s\sigma_t + s^2 \nabla \sigma \cdot A\nabla \sigma)_t u^2 dx dy dt\\
			=& - \frac{1}{2}\int_{\Om}  A\nabla u \cdot \nabla u \bigg|_0^T dx dy  -\frac{1}{2}\iint_{Q}(s\sigma_t + s^2 \nabla \sigma \cdot A\nabla \sigma)_t u^2 dx dy dt\\
			=& -\frac{1}{2}\iint_{Q}(s\sigma_t + s^2 \nabla \sigma \cdot A\nabla \sigma)_t u^2 dx dy dt,
		\end{split}
	\end{equation}
	and
	\begin{equation}\label{I2}
		\begin{split}
			I_2  =& s^2  \iint_Q \sigma_t u (\Div(u A \nabla \sigma) + A\nabla u \cdot \nabla \sigma) dx dy dt
			= s^2  \iint_Q \sigma_t u (\Div( A \nabla \sigma) + 2 A\nabla u \cdot \nabla \sigma) dx dy dt\\
			=& s^2  \iint_Q \sigma_t u \Div( A \nabla \sigma) + \Div( A \nabla \sigma) \sigma_t u^2  dx dy dt\\
			=& s^2 \iint_\Sigma \sigma_t u^2 A \nabla \sigma  \cdot \nu dsdt - s^2  \iint_Q \Div(\sigma_t A \nabla \sigma)u^2 dx dydt\\
			&+ s^2  \iint_Q \Div( A \nabla \sigma) \sigma_t u^2 dx dy dt\\
			=& - s^2  \iint_Q \Div( A \nabla \sigma) \sigma_t u^2 dx dy dt -\iint_Q  A \nabla \sigma \cdot\nabla \sigma_t u^2    dx dy dt+ s^2\iint_Q \Div( A \nabla \sigma) \sigma_t u^2 dx dydt\\
			=& -s^2\iint_Q  A \nabla \sigma \cdot\nabla \sigma_t u^2    dx dy dt.
		\end{split}
	\end{equation}
	Similarly, we can deduce the following expressions:
	\begin{equation}\label{I3}
		\begin{split}
			I_3  =&s^3 \iint_Q u A \nabla \sigma \cdot \nabla \sigma (\Div (u A \nabla \sigma) + A\nabla u \cdot \nabla \sigma ) dx dy dt\\
			=&s^3 \iint_\Sigma u^2 (A \nabla \sigma \cdot \nabla \sigma) A \nabla \sigma \cdot \nu  ds dt -s^3 \iint_Q u A \nabla \sigma \cdot \nabla (u A \nabla \sigma \cdot \nabla \sigma  ) dx dy dt\\
			&+ s^3 \iint_Q  (A \nabla \sigma \cdot \nabla u) ( A \nabla \sigma \cdot \nabla \sigma  ) u dx dy dt\\
			=&-s^3\iint_Q  A\nabla \sigma \cdot \nabla ( A\nabla \sigma \cdot\nabla \sigma) u^2 dx dy dt,
		\end{split}
	\end{equation}
	and
	\begin{align*}
		\begin{split}
			I_4= & s \iint_Q \Div(A \nabla u)\left(  \Div(u A \nabla \sigma)+ A \nabla u \cdot \nabla \sigma\right)  dx dy dt \\%1
			= & s \iint_Q \Div(A \nabla u)\left( A \nabla u \cdot \nabla \sigma+ u \Div(A \nabla \sigma)+ A \nabla u \cdot \nabla \sigma\right)  dx dy dt \\%2
			= & s \iint_Q \Div(A \nabla u) u \Div(A \nabla \sigma)dx dy dt + 2s \iint_Q \Div(A \nabla u) A \nabla u \cdot \nabla \sigma   dx dy dt \\%3
			= & s \iint_\Sigma u \Div(A \nabla \sigma )A \nabla u \cdot \nu dsdt + 2s \iint_{\Sigma_\delta}  (A \nabla u \cdot \nabla \sigma) A \nabla u \cdot \nu_\delta ds dt \\%4
			&-s \iint_Q A \nabla u \cdot  \nabla\left( u \Div(A \nabla \sigma)\right) dx dy dt -2s \iint_{\Omega^\delta} A \nabla u\cdot \nabla \left( A \nabla u \cdot \nabla \sigma \right)  dx dy dt \\%5
			= &
			-s \iint_Q A \nabla u \cdot  \nabla u \Div(A \nabla \sigma)dx dy dt-s \iint_Q u A \nabla u\cdot \nabla (\Div(A \nabla \sigma)) dx dy dt\\%6
			&-2s \iint_{Q} A \nabla u\cdot \nabla(A \nabla u \cdot \nabla \sigma)   dx dy dt. 
		\end{split}
	\end{align*}
	We observe that $2s \iint_{\Sigma_\delta} (A \nabla u \cdot \nabla \sigma) A \nabla u \cdot \nu_\delta  ds  dt = 0$ since $\nabla \sigma = 0$ on $\Sigma_\delta$. Moreover,
	\begin{align*}
		\begin{split}
			&-2s \iint_{Q} A \nabla u \cdot \nabla (A\nabla u \cdot \nabla \sigma) dx dydt\\
			=&-2s \sum_{i=1}^{2} \iint_{Q} (A\nabla \sigma)_i A \nabla u \cdot \frac{\partial}{\partial x_i}(\nabla u) + (\nabla u)_i A \nabla u \cdot \nabla (A \nabla \sigma)_i dx dydt\\
			=&-2s \iint_{Q} A \nabla u \left[ y^{\alpha_y} \frac{\partial \sigma}{\partial x}\nabla \frac{\partial u}{\partial x}  + x^{\alpha_x} \frac{\partial \sigma}{\partial y}\nabla \frac{\partial u}{\partial y} + \frac{\partial u}{\partial x} \nabla (y^{\alpha_y} \frac{\partial \sigma}{\partial x}) + \frac{\partial u}{\partial y} \nabla (x^{\alpha_x} \frac{\partial \sigma}{\partial y}) \right] dx dydt.
		\end{split}
	\end{align*}
	Here, we have
	\begin{equation*}
		\begin{split}
			&-2s \iint_{Q} A \nabla u \left[ y^{\alpha_y} \frac{\partial \sigma}{\partial x}\nabla \frac{\partial u}{\partial x}  + x^{\alpha_x} \frac{\partial \sigma}{\partial y}\nabla \frac{\partial u}{\partial y}\right] dx dydt\\%1
			=& -2s \iint_Q y^{\alpha_y} \frac{\partial \sigma}{\partial x}  A \nabla u \cdot \frac{\partial}{\partial x}(\nabla u) dx dydt - 2s \iint_Q x^{\alpha_x} \frac{\partial \sigma}{\partial y}  A \nabla u \cdot \frac{\partial}{\partial y}(\nabla u) dx dydt\\%2
			=& -s \iint_{Q} y^{\alpha_y} \frac{\partial \sigma}{\partial x} \left[ \frac{\partial}{\partial x} (A \nabla u \cdot \nabla u) - \frac{\partial A}{\partial x} \nabla u \cdot \nabla u \right] dx dydt\\
			&-s \iint_{Q} x^{\alpha_x} \frac{\partial \sigma}{\partial y} \left[ \frac{\partial}{\partial y} (A \nabla u \cdot \nabla u) - \frac{\partial A}{\partial y} \nabla u \cdot \nabla u \right] dx dydt\\%3
			=&-s \iint_{\Sigma_\delta} (A \nabla u \cdot \nabla u)(y^{\alpha_y} \frac{\partial \sigma}{\partial x}) \cdot \nu_x dsdt + s\iint_{Q} A \nabla u \cdot \nabla u \frac{\partial}{\partial x} (y^{\alpha_y} \frac{\partial \sigma}{\partial x}) dx dydt\\
			& +s \iint_{Q} y^{\alpha_y} \frac{\partial \sigma}{\partial x} \frac{\partial A}{\partial x} \nabla u \cdot \nabla u dx dydt
			-s \iint_{\Sigma_\delta} (A \nabla u \cdot \nabla u)(x^{\alpha_x} \frac{\partial \sigma}{\partial y}) \cdot \nu_y dsdt\\
			& + s\iint_{Q} A \nabla u \cdot \nabla u \frac{\partial}{\partial y} (x^{\alpha_x} \frac{\partial \sigma}{\partial y}) dx dydt +s \iint_{Q} x^{\alpha_x} \frac{\partial \sigma}{\partial y} \frac{\partial A}{\partial y} \nabla u \cdot \nabla u dx dydt\\%5
			=&  s \iint_{Q} A\nabla u \cdot \nabla u \Div(A\nabla \sigma) dx dydt
			+ s\iint_{Q} y^{\alpha_y} \frac{\partial \sigma}{\partial x} (\alpha_x x^{\alpha_x-1}\left| \frac{\partial u}{\partial y}\right| ^2) dx dydt\\
			&+ s\iint_{Q} x^{\alpha_x} \frac{\partial \sigma}{\partial y} (\alpha_y y^{\alpha_y-1}\left| \frac{\partial u}{\partial x}\right| ^2) dx dydt.
		\end{split}
	\end{equation*}
	Hence, we obtain the following expression:
	\begin{equation}\label{I4}
		\begin{split}
			I_4 = & -s \iint_Q u A \nabla u\cdot \nabla (\Div(A \nabla \sigma))dx dy dt
			\\
			&-2s \iint_{Q} A\nabla u \cdot \left[ \frac{\partial u}{\partial x} \nabla (y^{\alpha_y} \frac{\partial \sigma}{\partial x}) + \frac{\partial u}{\partial y} \nabla (x^{\alpha_x} \frac{\partial \sigma}{\partial y}) \right]  dx dy dt\\
			&+ s\iint_{Q} y^{\alpha_y} \frac{\partial \sigma}{\partial x} (\alpha_x x^{\alpha_x-1}\left| \frac{\partial u}{\partial y}\right| ^2) dx dydt
			+ s\iint_{Q} x^{\alpha_x} \frac{\partial \sigma}{\partial y} (\alpha_y y^{\alpha_y-1}\left| \frac{\partial u}{\partial x}\right| ^2) dx dydt. 
		\end{split}
	\end{equation}
	From equations \eqref{I1} to \eqref{I4}, we conclude that
	\begin{equation}\label{Pu}
		\begin{aligned}
			&\left(P_1 u, P_2 u\right)\\
			= & -s^3 \iint_Q A \nabla \sigma\cdot \nabla(\nabla \sigma A \nabla \sigma)|u|^2 dx dy dt\\
			&-2s \iint_{Q} A\nabla u \cdot \left[ \frac{\partial u}{\partial x} \nabla (y^{\alpha_y} \frac{\partial \sigma}{\partial x}) + \frac{\partial u}{\partial y} \nabla (x^{\alpha_x} \frac{\partial \sigma}{\partial y}) \right]  dx dy dt \\
			& -2s^2 \iint_Q \nabla \sigma A \nabla \sigma_t|u|^2 dx dy dt+ s\iint_{Q} y^{\alpha_y} \frac{\partial \sigma}{\partial x} \left( \alpha_x x^{\alpha_x-1}\left| \frac{\partial u}{\partial y}\right| ^2\right)  dx dydt\\
			&+ s\iint_{Q} x^{\alpha_x} \frac{\partial \sigma}{\partial y} \left( \alpha_y y^{\alpha_y-1}\left| \frac{\partial u}{\partial x}\right| ^2\right)  dx dydt
			-s \iint_Q u A \nabla u \nabla(\Div(A \nabla \sigma)) dx dy dt\\
			&-\frac{s}{2} \iint_Q \sigma_{t t}|u|^2 dx dy dt.
			%+2 s\iint_{\Sigma}(\nabla u A \nabla \sigma) A \nabla u \cdot \nu d s d t -s\iint_{\Sigma}(\nabla u A \nabla u) A \nabla \sigma \cdot \nu d s d t .
		\end{aligned}
	\end{equation}
	Let us denote the seven integrals on the right-hand side of equation \eqref{Pu} as $J_1, \cdots, J_6$. Our goal now is to estimate each of these integrals. By utilizing the definitions of $\sigma$ and $\xi$, as well as the properties of $\eta$, we can derive the following relationships: $\nabla \sigma = - \lambda \xi \nabla \eta$, $\nabla \xi = \lambda \xi \nabla \eta$, $A \nabla \sigma \cdot \nabla \sigma = \lambda^2 \xi^2 A\nabla \eta \cdot \nabla \eta$, and $\nabla( A \nabla \sigma \cdot \nabla \sigma) = \lambda^2 \xi^2 \nabla (A\nabla \eta \cdot \nabla \eta) + 2\lambda^3 \xi^2 \nabla \eta \left| A \nabla \eta \cdot \nabla \eta \right|$. Thus, we have
	\begin{equation}
		\begin{aligned}
			J_1
			&=-s^3 \iint_Q A \nabla \sigma \nabla(\nabla \sigma A \nabla \sigma)|u|^2 dx dy dt\\
			&=-s^3 \iint_Q  \left( -\lambda \xi A \nabla \eta \right) \left( \lambda^2\xi^2 \nabla(\nabla \eta A \nabla \eta) + 2\lambda^3\xi^2 \nabla \eta \left| A \nabla \eta \cdot \nabla \eta \right| ^2\right) |u|^2 dx dy dt\\
			&=2s^3 \lambda^4 \iint_Q \xi^3\left|A \nabla \eta \cdot \nabla \eta \right|^2|u|^2 dx dy dt
			+ s^3 \lambda^3 \iint_Q \xi^3 A\nabla \eta \cdot \nabla(A\nabla \eta \cdot \nabla\eta)|u|^2 dx dy dt.
		\end{aligned}
	\end{equation}
	Since $A\nabla \eta \cdot\nabla(A\nabla \eta \cdot \nabla\eta)$ is bounded in $\overline{\Omega}$, we can deduce that
	$$
	s^3 \lambda^3 \int_{0}^{T}\int_{\omega_0} \xi^3 A\nabla \eta \cdot \nabla(A\nabla \eta \cdot \nabla\eta)|u|^2 dx dy dt \ge -C s^3 \lambda^3 \int_{0}^{T}\int_{\omega_0} \xi^3 |u|^2 dx dy dt.
	$$
	Considering the inequality $|A \nabla \eta \cdot \nabla \eta|\ge C>0$ in $\overline{\Omega} \backslash \omega_0$, we can deduce the following:
	$$
	\begin{aligned}
		&s^3 \lambda^3 \int_{0}^{T}\int_{\Om\backslash \omega_0} \xi^3 A\nabla \eta \cdot \nabla(A\nabla \eta \cdot \nabla\eta)|u|^2 dx dy dt\\
		\ge& -C s^3 \lambda^3 \int_{0}^{T}\int_{\Om\backslash \omega_0} \xi^3 |u|^2 dx dy dt\\
		\ge&
		-C s^3 \lambda^3 \int_{0}^{T}\int_{\Om\backslash \omega_0} \xi^3 \left|A \nabla \eta \cdot \nabla \eta \right|^2 |u|^2 dx dy dt\\
		\ge&
		-C s^3 \lambda^3 \iint_{Q} \xi^3 \left|A \nabla \eta \cdot \nabla \eta \right|^2 |u|^2 dx dy dt.
	\end{aligned}
	$$
	Thus, we can conclude that
	\begin{equation}\label{J1}
		\begin{aligned}
			J_1
			\geq C s^3 \lambda^4 \iint_Q \xi^3\left|A \nabla \eta \cdot \nabla \eta \right|^2|u|^2 dx dy dt-C s^3 \lambda^3  \int_0^T \int_{\omega_0} \xi^3|u|^2 dx dy dt,
		\end{aligned}
	\end{equation}
	and
	\begin{equation}\label{J2}
		\begin{aligned}
			J_2 =&-2s  \iint_{Q} A\nabla u \cdot \left[ \frac{\partial u}{\partial x} \nabla (y^{\alpha_y} \frac{\partial \sigma}{\partial x}) + \frac{\partial u}{\partial y} \nabla (x^{\alpha_x} \frac{\partial \sigma}{\partial y}) \right]  dx dy dt\\
			=&-2s \iint_{Q}( D(A\nabla \sigma) A\nabla u) \cdot \nabla u dx dy dt\\
			=&2s \iint_{Q}( \lambda^2 \xi (A\nabla \eta \cdot \nabla u) A\nabla \eta + \lambda \xi D(A\nabla \eta) A\nabla u ) \cdot \nabla u dx dy dt\\
			=&2s \lambda^2\iint_{Q}  \xi \left|  A\nabla \eta \cdot \nabla u \right| ^2 dx dy dt  + 2s\lambda \iint_{Q}\xi D(A\nabla \eta) A\nabla u  \cdot \nabla u dx dy dt\\
			\ge& Cs \lambda^2\iint_{Q}  \xi \left|  A\nabla \eta \cdot \nabla u \right| ^2 dx dy dt  - C s\lambda \iint_{Q}\xi A\nabla u  \cdot \nabla u dx dy dt.
		\end{aligned}
	\end{equation}
	From the definition of $\xi$, it is evident that $\xi \xi_t \le \xi^3$. Therefore, we can write
	\begin{equation*}
		\begin{aligned}
			J_3 &=-2 s^2 \iint_{Q} A\nabla \sigma \cdot \nabla \sigma_t u^2 dx dydt
			=-2 s^2 \lambda^2 \iint_{Q} \xi \xi_t \left|A \nabla \eta \cdot \nabla \eta \right|  u^2 dx dydt\\
			&\ge-2 s^2 \lambda^2 \iint_{Q} \xi^3 \left|A \nabla \eta \cdot \nabla \eta \right|  u^2 dx dydt.
		\end{aligned}
	\end{equation*}
	Likewise, we obtain
	\begin{equation}\label{J3}
		\begin{aligned}
			J_3
			\ge &-C s^2 \lambda^2 \iint_{Q} \xi^3 \left|A \nabla \eta \cdot \nabla \eta \right|^2  u^2 dx dydt -C s^2 \lambda^2 \int_{0}^{T}\int_{\omega_0} \xi^3 \left|A \nabla \eta \cdot \nabla \eta \right|^2  u^2 dx dydt,
		\end{aligned}
	\end{equation}
	and
	\begin{equation}\label{J4}
		\begin{aligned}
			J_4 =&s\iint_{Q} y^{\alpha_y} \frac{\partial \sigma}{\partial x} \left( \alpha_x x^{\alpha_x-1}\left| \frac{\partial u}{\partial y}\right| ^2\right)  dx dydt
			+ s\iint_{Q} x^{\alpha_x} \frac{\partial \sigma}{\partial y} \left( \alpha_y y^{\alpha_y-1}\left| \frac{\partial u}{\partial x}\right| ^2\right)  dx dydt\\
			\ge  & -Cs \lambda \iint_Q \alpha_x \xi y^{\alpha_y+2}  x^{\alpha_x } \left| \frac{\partial u}{\partial y}\right|^2  dx dy dt 
			-Cs \lambda \iint_Q \alpha_y \xi x^{\alpha_x +2}  y^{\alpha_y } \left| \frac{\partial u}{\partial x}\right|^2  dx dy dt \\
			\ge & -C s \lambda \iint_Q  \xi A \nabla u \cdot \nabla u dx dy dt.
		\end{aligned}
	\end{equation}
	By using the definitions of $\sigma$ and $\xi$, we can derive the following expression:
	\begin{equation}\label{J5}
		\begin{split}
			J_5  =&-s \iint_Q u A \nabla u \cdot \nabla(\Div(A \nabla \sigma)) dx dy dt\\
			=&s \lambda^3 \iint_Q \xi u A \nabla u \cdot \nabla \eta \left( A \nabla \eta \cdot \nabla \eta\right)  dx dy dt
			+ s \lambda^2 \iint_Q \xi u A \nabla u \cdot \nabla\left(A \nabla \eta \cdot \nabla \eta\right) dx dy dt \\
			& +s \lambda^2 \iint_Q \xi u A \nabla u \cdot \nabla \eta \Div(A \nabla \eta)   dx dy dt+s \lambda \iint_Q \xi u A \nabla u \cdot \nabla(\Div(A \nabla \eta)) dx dy dt.
		\end{split}
	\end{equation}
	Let us denote the seven integrals on the right-hand side of \eqref{J5} as $J_{51}, \cdots, J_{54}$. Then we have
	\begin{equation}\label{J51}
		\begin{split}
			J_{51}=&s \lambda^3 \iint_Q \xi u A \nabla u \cdot \nabla \eta \left( A \nabla \eta \cdot \nabla \eta\right)  dx dy dt\\
			\geq&
			-Cs^2 \lambda^4 \iint_Q \xi\left|A \nabla \eta \cdot \nabla \eta\right|^2|u|^2 dx dy dt
			-C\lambda^2 \iint_Q \xi| A\nabla u \cdot\nabla \eta|^2 dx dy dt,
		\end{split}
	\end{equation}
	and
	\begin{equation}\label{J52}
		\begin{split}
			J_{52}=&s \lambda^2 \iint_Q \xi u A \nabla u \cdot \nabla\left(A \nabla \eta \cdot \nabla \eta\right) dx dy dt\\
			\ge&-Cs^2 \lambda^3\iint_{Q} \xi \left| A \nabla \eta \cdot \nabla \eta \right| u^2 dx dy dt -Cs^2 \lambda^3\int_0^T \int_{\omega_0} \xi \left| A \nabla u \cdot \nabla u \right|  u^2 dx dy dt\\
			&- C\lambda \iint_Q  \xi A \nabla u \cdot \nabla u dx dy dt.
		\end{split}
	\end{equation}
	Then we have
	\begin{equation}\label{J53}
		\begin{split}
			J_{53}=&s \lambda^2 \iint_Q \xi u A \nabla u \cdot \nabla \eta \Div(A \nabla \eta) dx dy dt\\
			\ge& -Cs^2 \lambda^3\iint_Q \xi \left| A \nabla \eta \cdot \nabla \eta \right|^2  u^2  dx dy dt -Cs^2 \lambda^3\int_0^T \int_{\omega_0} \xi u^2 dx dy dt\\
			&- C\lambda \iint_Q \xi A \nabla u \cdot \nabla u dx dy dt,
		\end{split}
	\end{equation}
	and
	\begin{equation}\label{J54}
		\begin{split}
			J_{54}=&s \lambda \iint_Q \xi u A \nabla u \nabla(\Div(A \nabla \eta))dx dy d t\\
			\ge&
			-C s^2 \lambda \iint_Q \xi \left|A \nabla \eta \cdot \nabla \eta\right|^2 u^2dx dy d t-C s^2 \lambda \int_0^T \int_{\omega_0}\xi  u^2dx dy d t\\
			&-C \lambda \iint_Q \xi A \nabla u \cdot \nabla udx dy d t.
		\end{split}
	\end{equation}
	Combining \eqref{J5} to \eqref{J54}, we obtain
	\begin{equation}\label{J55}
		\begin{split}
			J_5 \geq&-Cs^2 \lambda^4 \iint_Q \xi\left|A \nabla \eta \cdot \nabla \eta\right|^2 |u|^2dx dy d t -C\lambda^2\iint_{Q} \xi \left| A\nabla u\cdot \nabla \eta \right| ^2 dx dy dt\\
			&-C s^2 \lambda^3 \int_0^T \int_{\omega_0} \xi|u|^2dx dy d t- C\lambda \iint_Q  \xi A \nabla u \cdot \nabla u dx dy d t.
		\end{split}
	\end{equation}
	Finally, as can be observed from the definitions of $\xi$ and $\sigma$, we have $\sigma_{tt} \le \xi^{3/2}$. Hence,
	\begin{equation}\label{J6}
		J_6 =-\frac{s}{2} \iint_Q \sigma_{t t}|u|^2dx dy d t \geq-C s \iint_Q \xi^{3 / 2}|u|^2dx dy d t.
	\end{equation}
	From \eqref{J1}-\eqref{J4} and \eqref{J55}-\eqref{J6}, we deduce that
	\begin{equation}\label{PU}
		\begin{split}
			\left(P_1 u, P_2 u\right)
			\geq& C\iint_Q s^3 \lambda^4 \xi^3 |\nabla \eta \cdot A \nabla \eta|^2 |u|^2 dx dydt  + C s  \lambda^2 \iint_Q\xi|\nabla u \cdot A \nabla \eta|^2 dx dy d t\\
			&-Cs^3 \lambda^3 \int_0^T \int_{\omega_0} \xi^3|u|^2dx dy d t
			- Cs \lambda \iint_Q \xi A \nabla u \cdot \nabla udx dy d t\\
			&- Cs \iint_Q \xi^{3 / 2}|u|^2dx dy d t.
		\end{split}
	\end{equation}
	It is evident that the term $s \iint_Q \xi^{3 / 2}|u|^2dx dy dt$ can be absorbed by other terms. By combining \eqref{P} and \eqref{PU}, we conclude that
	\begin{equation}\label{LU2}
		\begin{split}
			&\left\|P_1 u\right\|^2+\left\|P_2 u\right\|^2+C\iint_Q s^3 \lambda^4 \xi^3\left|A \nabla \eta \cdot \nabla \eta \right|^2|u|^2 dx dydt  + C s  \lambda^2 \iint_Q\xi|\nabla u \cdot A \nabla \eta|^2 dx dy d t\\
			\leq& C\left\|e^{-s \sigma} f\right\|^2
			+Cs \lambda \iint_Q \xi A \nabla u \cdot \nabla udx dy d t+Cs^3 \lambda^3 \int_0^T \int_{\omega_0} \xi^3|u|^2dx dy d t.
		\end{split}
	\end{equation}
	Moreover, we can deduce that
	\begin{equation*}
		s^2 \lambda^2 \iint_Q \xi^{2}|u|^2dx dy d t=s^2 \lambda^2 \int_0^T \int_{\Om\backslash \omega_0} \xi^{2}|u|^2dx dy d t + s^2 \lambda^2 \int_0^T \int_{\omega_0} \xi^{2}|u|^2dx dy d t.
	\end{equation*}
	Clearly, the second term on the right can be absorbed by the other terms in \eqref{LU2}. Let us consider the first term on the right, we have
	\begin{equation*}
		\begin{split}
			s^2 \lambda^2 \int_0^T \int_{\Om\backslash \omega_0} \xi^{2}|u|^2dx dy d t
			&\le Cs^2 \lambda^2 \int_0^T \int_{\Om\backslash \omega_0} \xi^{2}\left|A \nabla \eta \cdot \nabla \eta \right|^2|u|^2dx dy d t\\
			&\le Cs^2 \lambda^2 \iint_Q \xi^{2}\left|A \nabla \eta \cdot \nabla \eta \right|^2|u|^2dx dy d t.
		\end{split}
	\end{equation*}
	Then we obtain
	\begin{equation}\label{LU3}
		\begin{split}
			&\left\|P_1 u\right\|^2+\left\|P_2 u\right\|^2+C s^2 \lambda^2 \int_0^T \int_{\Om\backslash \omega_0} \xi^{2}|u|^2dx dy d t\\
			&+  C\iint_Q s^3 \lambda^4 \xi^3\left|A \nabla \eta \cdot \nabla \eta \right|^2|u|^2 dx dydt  + C s  \lambda^2 \iint_Q\xi|\nabla u \cdot A \nabla \eta|^2 dx dy d t\\
			\leq& C\left\|e^{-s \sigma} f\right\|^2
			+Cs \lambda \iint_Q \xi A \nabla u \cdot \nabla udx dy d t+Cs^3 \lambda^3 \int_0^T \int_{\omega_0} \xi^3|u|^2dx dy d t.
		\end{split}
	\end{equation}
	Now, using the definitions of $P_1 u$ and $P_2 u$, we observe that
	\begin{equation*}
		\begin{split}
			&s^{-1} \iint_Q \xi^{-1}\left|u_t\right|^2dx dy d t\\
			=&s^{-1} \iint_Q \xi^{-1}(P_1 u -s u \Div(A\nabla \sigma)- 2s \nabla u \cdot A\nabla \sigma)^2dx dy d t\\
			\le& Cs^{-1}\left\|P_1 u\right\|^2+s   \iint_Q \xi^{-1}|u|^2|\Div(A \nabla \sigma)|^2dx dy d t +C s \iint_Q \xi^{-1}|\nabla u A \nabla \sigma|^2dx dy d t \\
			\leq& Cs^{-1}\left\|P_1 u\right\|^2+C s \lambda^4 \iint_Q \xi\left| A \nabla \eta \cdot \nabla \eta \right|^2 |u|^2dx dy d t \\
			&+C s \lambda^2  \iint_Q \xi|u|^2dx dy d t+C s \lambda^2 \iint_Q \xi|\nabla u \cdot A \nabla \eta|^2dx dy d t,
		\end{split}
	\end{equation*}
	and
	\begin{equation*}
		\begin{split}
			&s^{-1}   \iint_Q \xi^{-1}\left|\Div(A\nabla u)\right|^2dx dy d t\\
			=& s^{-1}   \iint_Q \xi^{-1}(P_2 u -s^2 u \nabla \sigma \cdot A \nabla \sigma - s \sigma_t u)^2dx dy d t\\
			\leq& C s^{-1} \left\|P_2 u\right\|^2+s^3   \iint_Q \xi^{-1}|u|^2\left|\nabla \sigma \cdot A \nabla \sigma\right|^2dx dy d t +C s   \iint_Q \xi^{-1}\left|\sigma_t\right|^2|u|^2dx dy d t \\
			\leq& C s^{-1} \left\|P_2 u\right\|^2+C s^3 \lambda^4   \iint_Q \xi^3|\nabla \eta \cdot A \nabla \eta|^2|u|^2dx dy d t +C s   \iint_Q \xi^{2}|u|^2dx dy d t.
		\end{split}
	\end{equation*}
	From \eqref{LU3}, we obtain
	\begin{equation}\label{LU4}
		\begin{split}
			&s^{-1}\iint_{Q} \xi^{-1} (|u_t |^2 + \left|\Div(A\nabla u)\right|^2) dx dydt +C s^2 \lambda^2 \int_0^T \int_{\Om\backslash \omega_0} \xi^{2}|u|^2dx dy d t\\
			&+C\iint_Q s^3 \lambda^4 \xi^3\left|A \nabla \eta \cdot \nabla \eta \right|^2|u|^2 dx dydt  + C s  \lambda^2 \iint_Q\xi|\nabla u \cdot A \nabla \eta|^2 dx dy d t\\
			\leq& C\left\|e^{-s \sigma} f\right\|^2
			+Cs \lambda \iint_Q \xi A \nabla u \cdot \nabla udx dy d t+Cs^3 \lambda^3 \int_0^T \int_{\omega_0} \xi^3|u|^2dx dy d t.
		\end{split}
	\end{equation}
	Since
	\begin{equation*}
		\begin{split}
			s \lambda \iint_Q \xi A \nabla u \cdot \nabla udx dy d t=& s \lambda \iint_\Sigma \xi u A \nabla u \cdot \nu d s d t - s \lambda \iint_Q \Div(\xi A\nabla u) udx dy d t\\
			=&- s \lambda \iint_Q \Div(\xi A\nabla u) udx dy d t,
		\end{split}
	\end{equation*}
	and then,
	\begin{equation*}
		\begin{split}
			&- s \lambda \iint_Q \Div(\xi A\nabla u) udx dy d t\\
			=&- s \lambda^2 \iint_Q \xi u\nabla \eta \cdot A\nabla u dx dy d t - s \lambda \iint_Q \xi\Div( A\nabla u) udx dy d t\\
			\le& C \lambda^2 \iint_Q \xi \left|  \nabla \eta \cdot A\nabla u \right| ^2dx dy d t + Cs^2 \lambda^2 \iint_Q \xi u^2dx dy d t\\
			&+C s^{-1} \lambda^{-1} \iint_Q \xi^{-1} \left| \Div( A\nabla u)\right| ^2dx dy d t + Cs^3 \lambda^3 \iint_Q \xi^3 u^2dx dy d t,
		\end{split}
	\end{equation*}
	thus, $s \lambda \iint_Q \xi A \nabla u \cdot \nabla udx dy dt$ can be absorbed by other terms. Therefore, we have
	\begin{equation}\label{LU5}
		\begin{split}
			&s^{-1}\iint_{Q} \xi^{-1} (|u_t |^2 + \left|\Div(A\nabla u)\right|^2) dx dydt +C s^2 \lambda^2 \int_0^T \int_{\Om\backslash \omega_0} \xi^{2}|u|^2dx dy d t\\
			&+C\iint_Q s^3 \lambda^4 \xi^3\left|A \nabla \eta \cdot \nabla \eta \right|^2|u|^2 dx dydt  + C s  \lambda^2 \iint_Q\xi|\nabla u \cdot A \nabla \eta|^2 dx dy d t\\
			\leq& C\left\|e^{-s \sigma} f\right\|^2
			+Cs^3 \lambda^3 \int_0^T \int_{\omega_0} \xi^3|u|^2dx dy d t.
		\end{split}
	\end{equation}
	Using classical arguments, we can transform back to the original variable $w$ and conclude the result.
	
	
	
	\vspace{3mm}
	
	\noindent{\bf Acknowledgement}
	
	\vspace{2mm}
	
	This work is supported by the National Natural Science Foundation of China, the Science-Technology Foundation of Hunan Province.
	
	
	
	
	
	
	\bibliographystyle{abbrvnat}
	%    Insert the bibliography data here.
	\bibliography{ref20230507.bib}
	%\begin{thebibliography}{10}
	%	
	%	\bibitem{Evans} L.C. Evans, Partial Differential Equations, American Mathematical Society Providence, Rholde Island, 2010. 
	%	
	%	
	%\end{thebibliography}
\end{document}













\documentclass[9pt,reqno]{amsart}

%\documentclass[12pt]{report}
\usepackage{amsmath, amsfonts, amssymb, latexsym, amsthm}
\usepackage[pagewise]{lineno}
%\usepackage{geometry}
\usepackage{amsmath, amsfonts, amssymb, latexsym}
\usepackage[numbers,sort&compress]{natbib}
\usepackage{mathrsfs}
\usepackage{pdfcomment}
%\usepackage{showkeys}
\newcommand{\commentontext}[2]{\colorbox{yellow!60}{#1}\pdfcomment[color={0.234 0.867 0.211},hoffset=-6pt,voffset=10pt,opacity=0.5]{#2}}
\newcommand{\commentatside}[1]{\pdfcomment[color={0.045 0.278 0.643},icon=Note]{#1}}
\newcommand{\todo}[1]{\commentatside{#1}}
\newcommand{\TODO}[1]{\commentatside{#1}}

%\usepackage{hyperref}  % 有了\usepackage{pdfcomment},这一句加不加都可以(吧)
\hypersetup{hidelinks,
	colorlinks=true,
	allcolors=black,
	pdfstartview=Fit,
	breaklinks=true
}

\everymath{\displaystyle}

\newtheorem{theorem}{Theorem}
\theoremstyle{plain}

\newtheorem{lemma}[theorem]{Lemma}
\newtheorem{definition}[theorem]{Definition}
\newtheorem{assumption}[theorem]{Assumption}
\newtheorem{proposition}[theorem]{Proposition}
\newtheorem{corollary}[theorem]{Corollary}
\newtheorem{remark}[theorem]{Remark}

\numberwithin{equation}{section}
\numberwithin{theorem}{section}

\newcommand{\cV}{\mathcal{V}}
\newcommand{\cY}{\mathcal{Y}}
\newcommand{\cU}{\mathcal{U}}
\newcommand{\cA}{\mathcal{A}}
\newcommand{\cB}{\mathcal{B}}
\newcommand{\cM}{\mathcal{M}}
\newcommand{\cE}{\mathcal{E}}
\newcommand{\cK}{\mathcal{K}}
\newcommand{\cL}{\mathcal{L}}
\newcommand{\cJ}{\mathcal{J}}
\newcommand{\cP}{\mathcal{P}}
\newcommand{\cF}{\mathcal{F}}
\newcommand{\cH}{\mathcal{H}}
\newcommand{\cT}{\mathcal{T}}
\newcommand{\cW}{\mathcal{W}}
\newcommand{\cS}{\mathcal{S}}
\newcommand{\C}{\mathbb{C}}
\newcommand{\Y}{\mathbb{Y}}
\newcommand{\D}{\mathbb{D}}
\newcommand{\E}{\mathbb{E}}
\newcommand{\R}{\mathbb{R}}
\newcommand{\F}{\mathbb{F}}
\newcommand{\N}{\mathbb{N}}
\newcommand{\Z}{\mathbb{Z}}
\newcommand{\bP}{\mathbb{P}} % \P is used by the system


\def\ds{\displaystyle}
\def\ol{\overline}
\def\mb{\mathbb}
\def\om{\omega}
\def\Om{\Omega}
\def\ra{\rightarrow}
\def\d{\delta}
\def\e{\varepsilon}
\def\ol{\overline}
\def\wt{\widetilde}


\def\ds{\displaystyle}

\def\d{\delta}
\def\e{\varepsilon}
\def\ol{\overline}
\def\wt{\widetilde}

\def\Ito{It\^{o} }
\DeclareMathOperator{\esssup}{ess\,sup}
\DeclareMathOperator*{\arginf}{\mathrm{arginf}}
\DeclareMathOperator*{\di}{\mathrm{d} \!}
\DeclareMathOperator*{\supp}{\mathrm{supp}}
\DeclareMathOperator*{\Div}{\mathrm{div}}
\def\ol{\overline}
\def\mb{\mathbb}
\def\om{\omega}
\def\Om{\Omega}
\def\ra{\rightarrow}
\def\df{{\rm d}}
\def\mcH{\mathcal{H}}
\def\pt{\partial}








\makeatletter
\renewcommand{\theequation}{%
	\thesection.\arabic{equation}}
\@addtoreset{equation}{section}
\makeatother

\begin{document}
	\title{Null controllability of a kind of n-dimensional degenerate parabolic equation}
	%%%%%%%%%%%%%%%%%%%%%%%%%%%%%%%%%%%%%%%%%%%%%%%%%%%
	
	%%%%%%%%%%%%%%%%%%%%%%%%%%%%%%%%%%%%%%%%%%%%%%%%%%%
	
	
	
	
	%
	%\author{\sffamily Hongli Sun$^{1}$, Yuanhang Liu$^{2}$, Weijia Wu$^{2,*}$, Donghui Yang$^{2}$   \\
	%	{\sffamily\small $^1$ School of Mathematics, Physics and Big data, Chongqing University of Science and Technology,\\
	%		\sffamily\small Chongging 401331, China }\\
	%	{\sffamily\small $^2$ School of Mathematics and Statistics, Central South University, Changsha 410083, China\\ }
	%}
	%\footnotetext[2]{Corresponding author: weijiawu@yeah.net }
	
	
	
	\author{\sffamily Yuanhang Liu$^{1}$, Weijia Wu$^{1,*}$, Donghui Yang$^1$, Can Zhang$^2$   \\
		{\sffamily\small $^1$ School of Mathematics and Statistics, Central South University, Changsha 410083, China. }\\
		{\sffamily\small $^2$ School of Mathematics and Statistics, Wuhan University, Wuhan 430072, China. }
	}
	\footnotetext[1]{Corresponding author: weijiawu@yeah.net }
	
	
	
	
	
	
	
	
	
	
	
	
	
	
	
	
	
	
	
	
	
	\email{honglisun@126.com}
	\email{liuyuanhang97@163.com}
	\email{weijiawu@yeah.net}
	\email{donghyang@outlook.com}
	
	\keywords{}
	\subjclass[2020]{93B05}
	
	\maketitle
	
	\begin{abstract}
		%We consider a class of two-dimensional degenerate parabolic equations with abstract coefficients. We provide results on improving the regularity and establish Carleman estimates for the corresponding equations by constructing specialized weight functions. As a result, we prove the null controllability of the associated equations. Furthermore, we present a specific example to illustrate the effectiveness of our methodology.
		In this paper, we investigate a class of $n$-dimensional degenerate parabolic equations with abstract coefficients. Our focus is on improving the regularity of solutions and establishing Carleman estimates for these equations through the construction of specialized weight functions. Using these results, we demonstrate the null controllability of the corresponding equations. Additionally, we provide a specific example to illustrate the efficacy of our methodology.
		
	\end{abstract}
	
	\pagestyle{myheadings}
	\thispagestyle{plain}
	\markboth{DEGENERATE PARABOLIC EQUATIONS WITH ABSTRACT COEFFICIENTS}{HONGLI SUN, WEIJIA WU AND DONGHUI YANG}
	
	
	
	\section{Introduction}
	Controllability is a fundamental concept in control theory that was first introduced by the renowned mathematician Kalman. It holds great importance in solving control problems within linear systems. The study of controllability for parabolic equations has a rich history spanning half a century, (see \cite{dolecki1977general,fattorini1971exact,fattorini1974uniform,russell1973unified,carleman1939probleme,hormander2013linear,hormander2009analysis,zuily1983uniqueness,lebeau1995controle,fursikov1996controllability,emanuilov1995controllability}), and can be categorized into two main branches: the controllability of non-degenerate parabolic equations and the controllability of degenerate parabolic equations. While there has been significant progress in analyzing the controllability of non-degenerate parabolic equations across various fields, research on the controllability of degenerate parabolic equations still remains relatively limited. 
	
	The degenerate parabolic equation, which is a common class of diffusion equations, can describe numerous physical phenomena. For example, the famous Crocco equation is a degenerate parabolic equation, which reflects the compatibility relationship between the change in total energy and entropy in steady flow and vorticity. Tornadoes follow the Crocco equation during the rotation process, and thus the study of controllability and optimal control problems of the Crocco equation is of great significance in meteorology (see \cite{martinez2003regional}). Similarly, the Black-Scholes equation (see \cite{sakthivel2008exact}), which is widely studied in finance, and the Kolmogorov equation (see \cite{calin2009heat,calin2010heat,anceschi2019survey}), are also degenerate parabolic equations that have important practical applications in our real life. Therefore, the study of the control problems of degenerate parabolic equations is of great significance.
	In \cite{cannarsa2004persistent}, the authors introduced the concepts of regional null controllability and regional persistent null controllability, and proved the regional controllability for a Crocco type linearized equation and for the nondegenerate heat equation in unbounded domains. Furthermore, in subsequent studies \cite{CA5,CA8,flores2010carleman,cannarsa2007null,cannarsa2008controllability}, the controllability of one-dimensional degenerate heat equations and other one-dimensional degenerate parabolic equations was demonstrated. After this, scholars have also investigated the problem of the controllability of high-dimensional degenerate equations, and the controllability of the Grushin-type operator has been extensively studied  (see \cite{anh2013null,cannarsa2013null,beauchard20152d,banerjee2022carleman}).
	In \cite{cannarsa2013null}, The authors obtained the Carleman estimate of the two-dimensional Grushin-type operator by using the Fourier decomposition,
	and further obtained the null controllability. Besides, there are many new results on the controllability of other high-dimensional degenerate equations.
	In  \cite{CA6}, they presented an effective approach to establish the controllability of two-dimensional degenerate equations and to address optimal control problems. Importantly, in the context of two-dimensional equations, the validity of the divergence theorem becomes crucial for meaningful boundary integrals along the degenerate boundary of the control region. Consequently, the inclusion of two-dimensional weighted Hardy inequalities and trace operators on weighted spaces becomes necessary to handle degenerate boundaries. 
	
	Inspired by the work of \cite{CA6}, this paper focuses on a more general class of $n$-dimensional degenerate parabolic equations. By employing the Carleman estimate technique, we establish the null controllability of these equations, where the degenerate coefficient is considered as an abstract function, and the control region is located near the boundary.
	
	It is worth noting that our model exhibits significant differences from that of \cite{CA6}. In \cite{CA6}, they need to use Hardy inequality and trace theory to obtain the well-posed results and Carleman estimates, but in this paper, due to different assumptions, different control regions and different weight functions, we do not need Hardy inequality and trace theory, but use a new method to obtain our Carleman's estimate by ``cutting off " the degenerate boundary. In other words, if we cut off the degenerate boundary, null controllability always holds. However, this is only one of the methods, and interior control is a further problem for us to consider.
	
	
	%In this paper, we consider
	%\begin{equation}\label{1.1}
	%\begin{cases}
	%\partial_{t}z - \sum_{i,j=1}^n\partial_{x_i}(A_{ij}\partial_{x_j}z) + bz=\chi_{\omega}g, & \mbox{in} \ Q,  \\
	%z=0 \ \mbox{or} \ A\nabla z \cdot \nu =0,  & \mbox{on} \ \Sigma,  \\
	%z(0)= z_{0},  & \mbox{in} \ \Omega,  
	%\end{cases}
	%\end{equation}
	%where $\Omega\subset \mathbb{R}^n$ with Lipschitz boundary, $\Gamma:= \partial\Omega, T>0$, $ Q:=\Omega\times (0,T), \Sigma:= \Gamma\times(0,T)$, let $\hat{\Om}\subset \Om$ be a nonempty open subset, $ \Omega_0 \subset\subset \hat{\Omega}$ is a nonempty open set with smooth boundary,  $\omega:=\Om\backslash\overline{\Omega_0}$ and $\chi_{\omega}$ is the corresponding characteristic function, 
	%% $b\in L^{\infty}(Q)$,
	%$g \in L^{2}(Q)$ is the control, $z_0 \in L^{2}(\Omega)$ is the initial data, $b\in L^{\infty}(Q)$ is a given function. We need the following assumption:
	%
	%\begin{assumption}\label{assume1}
	%the matrix-valued function $A=(A_{ij}(x))_{i,j=1}^n$ is positive definite for all $x \in \Om$, but may false at a subset of $\partial\Omega$, and $A_{ij}\in C^2(\overline{\hat{\Om}}), i,j=1,\cdots, n$, 
	%$$
	%\ \lambda|\xi|^2 \le \sum_{i,j=1}^n A_{ij}(x)\xi_i\xi_j \le \Lambda |\xi|^2, \ \forall \xi\in\mathbb{R}^n, \forall x \in \hat{\Om}, 
	%$$
	%here $0<\lambda \le \Lambda$ are two fixed constants.
	%\end{assumption}
	%From Assumption \ref{assume1}, it is known that the equation \eqref{1.1} is degenerate on part of the boundary, but non-degenerate in the interior.
	%
	%
	%%Obviously, $\sum_{i,j=1}^n\partial_{x_i}(A_{ij}\partial_{x_j}z)=\Div(A\nabla z)$.
	The remaining sections of this paper are structured as follows. In section 2, we present some preliminary results and demonstrate the well-posedness of problem \eqref{1.1}. In section 3, we provide Carleman estimates for the degenerate equation \eqref{1.1} and present the main results of this paper. In order to enhance the understanding of our model, section 4 provides a specific example and discusses how to improve the internal regularity and obtain Carleman estimates for the corresponding equation \eqref{4.1} to be formulated later.
	
	
	\section{main results}
	This section presents the main results of the paper, which are divided into two parts: Results in abstract form and Results in a specific example.
	\subsection{Results in abstract form}
	
	\hspace*{\fill}
	
	In this paper, we consider the following problem in a bounded domain $\Omega\subset \mathbb{R}^n$ with Lipschitz boundary:
	\begin{equation}\label{1.1}
		\begin{cases}
			\partial_{t}z - \sum_{i,j=1}^n\partial_{x_i}(A_{ij}\partial_{x_j}z) + bz=\chi_{\omega}g, & \mbox{in} \ Q,  \\
			z=0 \ \mbox{or} \ A\nabla z \cdot \nu =0,  & \mbox{on} \ \Sigma,  \\
			z(0)= z_{0},  & \mbox{in} \ \Omega.  
		\end{cases}
	\end{equation}
	Here, $Q:=\Omega\times (0,T)$ and $\Sigma:= \Gamma\times(0,T)$ denote the space-time domain and boundary, respectively. The set $\hat{\Omega}\subset \Omega$ is a nonempty open subset, and $\Omega_0 \subset\subset \hat{\Omega}$ is a nonempty open set with a smooth boundary. The set $\omega:=\Omega\backslash\overline{\Omega_0}$ is defined as the complement of the closure of $\Omega_0$, and $\chi_{\omega}$ is the corresponding characteristic function. The control function $g \in L^{2}(Q)$ and the initial data $z_0 \in L^{2}(\Omega)$ are given, while $b\in L^{\infty}(Q)$ is a known function.
	
	We impose the following assumption on the matrix-valued function $A=(A_{ij}(x))_{i,j=1}^n$:
	
	\begin{assumption}\label{assume1}
		The matrix-valued function $A=(A_{ij}(x))_{i,j=1}^n$ is positive definite for all $x \in \Om$, but may vanish at a subset of $\partial\Omega$, and $A_{ij}=A_{ji}\in C^2(\overline{\hat{\Om}})$, $i,j=1,\cdots, n$, 
		$$
		\ \rho|\xi|^2 \le \sum_{i,j=1}^n A_{ij}(x)\xi_i\xi_j \le \Lambda |\xi|^2, \ \forall \xi\in\mathbb{R}^n, \forall x \in \hat{\Om}, 
		$$
		here $0<\rho \le \Lambda$ are two fixed constants.
	\end{assumption}
	From Assumption \ref{assume1}, it is known that the equation \eqref{1.1} is degenerate on a part of the boundary, but non-degenerate in the interior.
	
	%\section{Well-posed results}
	%In this section we discuss the well-posedness of \eqref{1.1}. 
	We define the function spaces $\mathcal{H}^1(\Omega)$ and $\mathcal{H}^2(\Omega)$ as follows: 
	\begin{equation}\label{space}
		\begin{split}
			\mathcal{H}^1(\Om)
			&=\left\lbrace z\in L^2(\Om) \mid \nabla z A \nabla z \in L^1(\Om) \right\rbrace,\\
			\mathcal{H}^2(\Om)
			&=\left\lbrace z\in \mathcal{H}^1(\Om) \mid \Div(A\nabla z) \in L^2(\Om) \right\rbrace.
		\end{split}
	\end{equation}
	These spaces are equipped with the following scalar products:
	\begin{equation}\label{inner}
		\begin{split}
			(z,v)_{\mathcal{H}^1(\Om)}
			&:=\int_{\Omega} zv  dx  + \int_{\Omega} \nabla z A \nabla v dx,\\
			(z,v)_{\mathcal{H}^2(\Om)}
			&:=\int_{\Omega} zv  dx  + \int_{\Omega} \nabla z A \nabla v dx + \int_{\Omega} \Div(A\nabla z)\Div(A\nabla v) dx.
		\end{split}
	\end{equation}
	These spaces are endowed with the following norms:
	\begin{equation}\label{norm}
		\begin{split}
			\left\| z \right\|_{\mathcal{H}^1(\Om)} 
			&= \left\| z \right\|_{L^2(\Om)} + \left\| \nabla z A \nabla z\right\|_{L^1(\Om)},\\
			\left\| z \right\|_{\mathcal{H}^2(\Om)} 
			&= \left\| z \right\|_{\mathcal{H}^1(\Om)} + \left\| \Div(A\nabla z)\right\|_{L^2(\Om)}.
		\end{split}
	\end{equation}
	It can be easily verified that $(\mathcal{H}^1(\Omega), (\cdot,\cdot){\mathcal{H}^1(\Omega)})$ and $(\mathcal{H}^2(\Omega), (\cdot,\cdot){\mathcal{H}^2(\Omega)})$ are inner product spaces, and $(\mathcal{H}^1(\Omega), |\cdot|{\mathcal{H}^1(\Omega)})$ and $(\mathcal{H}^2(\Omega), |\cdot|{\mathcal{H}^2(\Omega)})$ are Banach spaces.
	
	Let $\mathcal{H}_0^1(\Omega)$ denote the closure of $C_0^\infty(\Omega)$ in the space $\mathcal{H}^1(\Omega)$. In other words,
	\begin{equation*}
		\mcH_0^1(\Omega)=\overline{C_0^\infty(\Omega)}^{\mcH^1(\Omega)}.
	\end{equation*} 
	We also introduce the operators $(\mathcal{A}_1,D(\mathcal{A}_1))$ and $(\mathcal{A}_2,D(\mathcal{A}_2))$ defined as follows:
	\begin{equation*}
		\begin{split}
			&\mathcal{A}_1 z= \sum_{i,j=1}^n\partial_{x_i}(A_{ij}\partial_{x_j}z),  \quad D(\mathcal{A}_1) = \mathcal{H}^2 (\Om)\cap \mcH_0^1(\Omega) ,\\
			&\mathcal{A}_2 z= \sum_{i,j=1}^n\partial_{x_i}(A_{ij}\partial_{x_j}z),  \quad D(\mathcal{A}_2) = \left\lbrace z\in \mathcal{H}^2 (\Om) \mid A\nabla z \cdot \nu = 0 \right\rbrace .
		\end{split}
	\end{equation*}
	Next, we will present some results related to the bilinear form $q(\cdot,\cdot)$ associated with the operators $\mathcal{A}_1$ and $\mathcal{A}_2$. But before that, we introduce an assumption.
	
	\begin{assumption}\label{assume2}
		When $A\nabla z \cdot \nu =0$, that is to say, in Newman's boundary condition, we assume that $A\nabla z \in (W^{1,1}(\Om))^n$.
	\end{assumption}
	\begin{remark}
		It should be noted that this assumption is used to prove Lemma \ref{div}, which will be presented later. However, Assumption \ref{assume2} is not strictly necessary. Following a similar approach to the case in \cite{BSV}, we can assume that $A\nabla z \in H^{-\frac{1}{2}}(\Gamma)$ and $\nu \in H^{\frac{1}{2}}(\Gamma)$.
	\end{remark}
	%Then we can establish the following Green's formula.
	%\begin{lemma}\label{div}
	%If $(u,v)\in D(\mathcal{A}_1)\times \mcH^1_0(\Omega)$ or $(u,v)\in D(\mathcal{A}_2)\times \mcH^1(\Omega)$, one has 
	%\begin{equation*}
	%\int_\Om \nabla u \cdot A \nabla v dx = -\int_\Om \Div (A \nabla u) vdx.
	%\end{equation*}
	%\end{lemma}
	%\begin{proof}
	%By Assumption \ref{assume2} and boundary conditions, this concludes the proof. 
	%\end{proof}
	%
	%\begin{lemma}\label{2.15}
	%The following results hold:
	%\begin{itemize}
	%    \item [(1)] The injection $i_1: D(\mathcal{A}_1)\to L^2(\Om)$ and $i_2: D(\mathcal{A}_2)\to L^2(\Om)$ is continuous with dense range;
	%    \item [(2)] The bilinear form $q_1(u,v):= \int_\Om \nabla u \cdot A \nabla v dx, \ (u,v)\in D(\mathcal{A}_1)\times \mcH^1_0(\Omega)$ and $q_2(u,v):= \int_\Om \nabla u \cdot A \nabla v dx, \ (u,v)\in D(\mathcal{A}_2)\times \mcH^1(\Omega),$ are continuous, positive, symmetric.
	%    \item [(3)]  The operator $(\mathcal{A}_1,D(\mathcal{A}_1)), (\mathcal{A}_2,D(\mathcal{A}_2))$ are self-adjoint and dissipative.
	%  \end{itemize}
	%%are the infinitesimal generators of the strongly continuous semigroups denoted by $e^{t\mathcal{A}_1}, e^{t\mathcal{A}_2}$ respectively.
	%\end{lemma}
	%\begin{proof}
	%Obviously, by definition of $D(\mathcal{A}_1)$ and $D(\mathcal{A}_2)$, it follows that $i_1,i_2$ are continuous. Thus, since $C_0^\infty (\Om)$ is dense in $D(\mathcal{A}_1)$ and $D(\mathcal{A}_2)$, it is also dense in $L^2(\Om)$. This proves (1). 
	%
	%Now, we prove (2). 
	%It is clear that $q_1,q_2$ is symmetrical and $q_1(u,u) \ge 0 , q_2(u,u) \ge 0$. Furthermore
	%$$
	%|q_1(u,v)|\le\left\| u\right\| _{\mcH_0^1(\Omega)} \left\| v\right\| _{\mcH_0^1(\Omega)},\ |q_2(u,v)|\le\left\| u\right\| _{\mcH^1(\Omega)} \left\| v\right\| _{\mcH^1(\Omega)}.
	%$$
	%Hence $q_1,q_2$ is continuous.
	%
	%Now suppose that $(u_1,v_1) \in  D(\mathcal{A}_1)\times \mcH^1_0(\Omega),(u_2,v_2)\in D(\mathcal{A}_2)\times \mcH^1(\Omega)$, by Lemma \ref{div} we have that
	%$$
	%q_1(u_1,v_1)=-\int_{\Om}\mathcal{A}_1 u_1 \cdot v_1 d x, \ q_2(u_2,v_2)=-\int_{\Om}\mathcal{A}_2 u_2 \cdot v_2 d x.
	%$$
	%Since
	%$$
	%\begin{aligned}
	%\left\| (\lambda I - \mathcal{A}_1)u_1\right\| _{L^2(\Om)}
	%&=\sup_{\|v_1\|_{L^2(\Om)} \le 1, v_1 \in \mcH_0^1(\Omega)} \int_{\Omega} \lambda v_1 u_1 - v_1 \mathcal{A}_1 u_1 dx
	%%=&\sup_{|v|_2 \le 1, v \in \mcH_0^1(\Omega)} \int_{\Omega} \lambda v u dx + q(u,v)\\
	%\ge  \lambda \left\| u_1\right\| _{L^2{\Om}}\\
	%\left\| (\lambda I - \mathcal{A}_2)u_2\right\| _{L^2(\Om)}
	%&=\sup_{\|v_2\|_{L^2(\Om)} \le 1, v_2 \in \mcH^1(\Omega)} \int_{\Omega} \lambda v_2 u_2 - v_2 \mathcal{A}_2 u_2 dx
	%%=&\sup_{|v|_2 \le 1, v \in \mcH^1(\Omega)} \int_{\Omega} \lambda v u dx + q(u,v)\\
	%\ge  \lambda \left\| u_2\right\| _{L^2{\Om}},
	%\end{aligned}
	%$$
	%and from the properties of $q_1,q_2$, we conclude that $\mathcal{A}_1$ and $\mathcal{A}_2$ are self-adjoint and dissipative.
	%% as a similar argument, we conclude that $\mathcal{A}_2$ is also self-adjoint and dissipative.
	%\end{proof}
	
	%
	%As a consequence, both $\mathcal{A}_1$ and $\mathcal{A}_2$ are the infinitesimal generators of the strongly continuous semigroups denoted by $e^{t\mathcal{A}_1}$, $e^{t\mathcal{A}_2}$ respectively. Moreover, the family of operators in $\mathcal{L}(L^2(\Om))$ given by
	%$$
	%B(t)u := b(t,\cdot)u, \quad t\in (0,T), \quad u\in L^2(\Om)
	%$$
	%can be seen as a family of bounded perturbation of $\mathcal{A}_1$ (resp. $\mathcal{A}_2$). Thus, using standard techniques(see \cite{bensoussan2007representation,cazenave1998introduction,showalter1995hilbert}), one can prove the following well-posedness results.
	
	\begin{theorem}\label{exist1}
		For any $g \in L^2(Q)$ and any $z_0 \in L^2(\Omega)$, there exists a unique solution $z \in C^0([0,T];L^2(\Omega)) \cap L^2(0,T;\mathcal{H}_0^1(\Omega))$ to equation \eqref{1.1}. Moreover, there exists a positive constant $C$ such that
		$$
		\sup_{t\in\left[0,T\right] } \|z(t)\|_{L^2(Q)}^2 + \int_{0}^{T} \left\| z(t)\right\|^2_{\mcH_0^1(\Omega)} d t \le C(\left\| z_0\right\|_{L^2(\Om)}^2 + \|g\|_{L^2(Q)}^2).
		$$
	\end{theorem}
	\begin{theorem}\label{exist2}
		For any $g \in L^2(Q)$ and any $z_0 \in L^2(\Omega)$, there exists a unique solution $z \in C^0([0,T];L^2(\Omega)) \cap L^2(0,T;\mathcal{H}^1(\Omega))$ to equation \eqref{1.1} with homogeneous Neumann boundary conditions. Furthermore, there exists a positive constant $C$ such that
		$$
		\sup_{t\in\left[0,T\right] } \|z(t)\|_{L^2(Q)}^2 + \int_{0}^{T} \left\| z(t)\right\|^2_{\mcH^1(\Omega)} d t \le C(\left\| z_0\right\|_{L^2(\Om)}^2 + \|g\|_{L^2(Q)}^2).
		$$
	\end{theorem}
	\begin{remark}
		It is worth noting that by adding a gradient term $c\nabla z$ to the left-hand side of equation \eqref{1.1} and imposing stronger conditions on the degenerate coefficient $A_{ij}(x)$, similar well-posedness results can be obtained.
	\end{remark}
	
	%\subsection{Internal regularity improvement}
	%Now, we have proved the existence of solutions to the equation \eqref{1.1}, and the solution $z \in C([0,T];L^{2}(\Omega))\cap L^{2}(0,T;\mathcal{H}_0^{1}(\Omega))$.
	%We are going to show that the regularity of the solutions can be improved in the interior of $\Omega$. By standard argument in \cite{EVANS}, one can prove the following result.
	%\begin{theorem}\label{Regularity}
	%For all the solution of equation \eqref{1.1}, one has $A\nabla u \in W^{1,1}(\Om_0)$ and $\nabla u A\nabla u \in W^{1,1}(\Om_0)$.
	%\end{theorem}
	%\begin{proof}
	%By standard argument in \cite{EVANS}, one can prove this result.
	%\end{proof}
	
	%\begin{remark}
	%It is worth noting that, on the boundary, we cannot improve the regularity due to the specificity of the equation we are considering. Based on this fact, on the boundary, we do not have the fact $(A \nabla u \cdot \nabla \sigma) A \nabla u, (A\nabla u\cdot\nabla u)A\nabla\sigma\in (W^{1,1}(\Omega))^2$ (see the estimate of $I_4$ about the boundary term in the following section), in other words, the functions $(A \nabla u \cdot \nabla \sigma) A \nabla u, (A\nabla u\cdot\nabla u)A\nabla\sigma$ do not have trace on $\partial\Omega$, which makes our research approach very different from  \cite{CA5,CA6}, and this is also the reason that we choose the control domain is $\omega:= \Om\backslash \overline{\Om_0}$ in the following section.
	%\end{remark}
	
	%\section{Carleman esitimate and Null Controllability results}
	
	%\subsection{main results}
	Let us consider the adjoint equation corresponding to \eqref{1.1}:
	%Obviously, the adjoint equation of \eqref{1.1} is
	\begin{equation}\label{3.1}
		\begin{cases}
			\partial_{t}w + \Div(A\nabla w) -bw=f, & \mbox{in} \ Q,  \\
			w=0 \ \mbox{or} \ A\nabla w \cdot \nu =0,  & \mbox{on} \ \Sigma,  \\
			w(x,  y,  T)= w_T,  & \mbox{in} \ \Omega.  
		\end{cases}
	\end{equation}
	The main results of this paper are the Carleman inequality and observability inequality stated below.
	\begin{theorem}\label{TH2}
		There exist positive constants $C, s_0, \lambda_0$ such that for any $\lambda \ge \lambda_0$, $s \ge s_0$, and any solution $u$ to \eqref{3.1}, the following inequality holds:
		\begin{equation}\label{3.2}
			\begin{split}
				&s^{-1}\iint_{Q} \xi^{-1} (|u_t |^2 + \left|\Div(A\nabla u)\right|^2) dx dt +C s^2 \lambda^2 \int_0^T \int_{\Om\backslash \omega} \xi^{2}|u|^2dx  d t\\
				&+C\iint_Q s^3 \lambda^4 \xi^3\left|A \nabla \eta \cdot \nabla \eta \right|^2|u|^2 dx dt  + C s  \lambda^2 \iint_Q\xi|\nabla u \cdot A \nabla \eta|^2 dx  d t\\
				\leq& C\left\|e^{-s \sigma} f\right\|^2
				+Cs^3 \lambda^3 \int_0^T \int_{\omega} \xi^3|u|^2dx  d t.
			\end{split}
		\end{equation}
	\end{theorem}
	By a standard argument, we can conclude the following theorem (see \cite{FE,ZZ2006}).
	\begin{theorem}\label{TH4}
		For a fixed $T>0$ and an open set $\omega \subset \Omega$ as defined previously, assuming that \eqref{3.2} holds, there exists a positive constant $C>0$ such that for any $w_T \in L^2(\Omega)$, the solution to \eqref{3.1} satisfies
		\begin{equation}\label{3.4}
			\int_{\Omega}|w(x, 0)|^2 d x \leq C \iint_{\omega \times(0, T)}|w|^2 d x d t.
		\end{equation}
	\end{theorem}
	Using the duality between controllability and observability, proving controllability is equivalent to establishing an observability property for the adjoint system \eqref{3.1} (see \cite{rockafellar1967duality,lions1992remarks}). From this, we can deduce the null controllability of \eqref{1.1}.
	\begin{theorem}\label{TH3}
		For a fixed $T>0$ and an open set $\omega \subset \Omega$ as defined previously, assuming that \eqref{3.2} holds, there exists a control $g \in L^2(Q)$ such that the solution $z$ of \eqref{1.1} satisfies
		$$
		z(\cdot, T)=0, \text { in } \Omega.
		$$
		Moreover, there exists a constant $C=C(T, \omega)>0$ such that
		$$
		\|g\|_{L^2(\Om)} \leq C\left|z_0\right| .
		$$
	\end{theorem}
	\subsection{Results in a specific example}
	
	\hspace*{\fill}
	
	To illustrate the results of this paper, we consider a specific example of a two-dimensional degenerate parabolic equation:
	\begin{equation}\label{4.1}
		\begin{cases}
			\partial_{t}z - (y^{\alpha_y}\partial_{xx}z +x^{\alpha_x}\partial_{yy}z )=\chi_{\omega}g, & \mbox{in} \ Q,  \\
			z(x,y,t)=0 \ \mbox{or} \ A\nabla z \cdot \nu =0, & \mbox{on} \ \Sigma,  \\
			z(x,  y,  0)= z_{0}(x,  y),  & \mbox{in} \ \Omega,  
		\end{cases}
	\end{equation}
	where $\alpha_x$, $\alpha_y \in (0,2)$, $\Omega=(0,1)\times(0,1)$, $\Gamma:= \partial\Omega$, $T>0$, $Q:=\Omega\times (0,T)$, $\Sigma:= \Gamma\times(0,T)$, $\omega \subset \Omega$ is a nonempty open set, and $\chi_{\omega}$ is the corresponding characteristic function. The control $g \in L^{2}(Q)$ and $z_0 \in L^{2}(\Omega)$ is the initial data.
	
	The function space, inner product, and norm are defined as in equations \eqref{space}-\eqref{norm}, and the matrix-valued function $A:\overline{\Omega} \to M_{2\times 2}(\mathbb{R})$ is given by
	\begin{equation*}
		\begin{pmatrix} y^{\alpha_y} & 0 \\ 0 &x^{\alpha_x} \end{pmatrix}.
	\end{equation*}
	%Define 
	%\begin{equation*}
	%\mathcal{H}^1(\Om)=\left\lbrace z\in L^2(\Om) \mid \nabla z A \nabla z \in L^1(\Om) \right\rbrace ,
	%\end{equation*}
	%\begin{equation*}
	%\mathcal{H}^2(\Om)=\left\lbrace z\in \mathcal{H}^1(\Om) \mid \Div(A\nabla z) \in L^2(\Om) \right\rbrace ,
	%\end{equation*}
	%with scalar product
	%\begin{equation*}
	%(z,v)_{\mathcal{H}^1(\Om)}:=\int_{\Omega} zv  dx dy + \int_{\Omega} \nabla z A \nabla v dx,
	%\end{equation*}
	%\begin{equation*}
	%(z,v)_{\mathcal{H}^2(\Om)}:=\int_{\Omega} zv  dx dy + \int_{\Omega} \nabla z A \nabla v dx + \int_{\Omega} \Div(A\nabla z)\Div(A\nabla v) dx,
	%\end{equation*}
	%endowed with the norm
	%\begin{equation*}
	%\left\| z \right\|_{\mathcal{H}^1(\Om)} = \left\| z \right\|_{L^2(\Om)} + \left\| \nabla z A \nabla z\right\|_{L^1(\Om)}  .
	%\end{equation*}
	%\begin{equation*}
	%\left\| z \right\|_{\mathcal{H}^2(\Om)} = \left\| z \right\|_{\mathcal{H}^1(\Om)} + \left\| \Div(A\nabla z)\right\|_{L^2(\Om)}  .
	%\end{equation*}
	Let $\mathcal{B}_1$ and $\mathcal{B}2$ be operators defined as follows:
	\begin{equation*}
		\begin{split}
			&\mathcal{B}_1 z= y^{\alpha_y}\partial_{xx}z +x^{\alpha_x}\partial_{yy}z,  \quad D(\mathcal{B}_1) = \mathcal{H}^2 (\Om)\cap \mcH_0^1(\Omega) ,\\
			&\mathcal{B}_2 z= y^{\alpha_y}\partial_{xx}z +x^{\alpha_x}\partial_{yy}z,  \quad D(\mathcal{B}_2) = \left\lbrace z\in \mathcal{H}^2 (\Om) \mid B\nabla z \cdot \nu = 0 \right\rbrace .
		\end{split}
	\end{equation*}
	It can be easily verified that $(\mathcal{B}_1,D(\mathcal{B}_1))$ and $(\mathcal{B}_2,D(\mathcal{B}_2))$ are self-adjoint and dissipative operators with dense domain. Consequently, $(\mathcal{B}_1,D(\mathcal{B}_1))$ and $(\mathcal{B}_2,D(\mathcal{B}_2))$ serve as the infinitesimal generators of the strongly continuous semigroups $e^{t\mathcal{B}_1}$ and $e^{t\mathcal{B}_2}$, respectively. As a result, similar well-posedness results can be established as in Theorem \ref{exist1} and Theorem \ref{exist2}.
	
	It is worth noting that in this example, we can also apply the Galerkin method described in \cite{EVANS} to establish the existence of solutions in weakly degenerate cases. This is possible due to the compact embedding result derived in \cite{sun2023extremal}.
	
	Once the existence of solutions is established, we can proceed to investigate the controllability of the equation \eqref{4.1}. The adjoint equation corresponding to \eqref{4.1} is given by
	\begin{equation}\label{adjoint}
		\begin{cases}
			\partial_{t}w + \Div(A\nabla w)=f, & \mbox{in} \ Q,  \\
			w(x,y,t)=0, \ \mbox{or} \ A\nabla w \cdot \nu =0, & \mbox{on} \ \Sigma,  \\
			w(x,  y,  T)= w_T,  & \mbox{in} \ \Omega.  
		\end{cases}
	\end{equation}
	By employing Carleman estimates, we can derive similar results on null controlability as presented in Theorems \ref{TH2}, \ref{TH4}, and \ref{TH3}.
	
	
	\section{Well-posed results}
	In this section, we investigate the well-posedness of \eqref{1.1}.
	
	By utilizing Assumption \ref{assume2}, we establish the following Green's formula.
	\begin{lemma}\label{div}
		For $(u,v)\in D(\mathcal{A}1)\times \mcH^1_0(\Omega)$ or $(u,v)\in D(\mathcal{A}2)\times \mcH^1(\Omega)$, the following equality holds:
		\begin{equation*}
			\int_\Om \nabla u \cdot A \nabla v dx = -\int_\Om \Div (A \nabla u) vdx.
		\end{equation*}
	\end{lemma}
	\begin{proof}
		The proof follows directly from Assumption \ref{assume2} and the boundary conditions.
	\end{proof}
	
	\begin{lemma}\label{2.15}
		The following results hold:
		\begin{itemize}
			\item [(1)] The injection $i_1: D(\mathcal{A}_1)\to L^2(\Om)$ and $i_2: D(\mathcal{A}2)\to L^2(\Om)$ are continuous with dense range.
			\item [(2)] The bilinear form $q_1(u,v):= \int_\Om \nabla u \cdot A \nabla v dx, \ (u,v)\in D(\mathcal{A}_1)\times \mcH^1_0(\Omega)$ and $q_2(u,v):= \int_\Om \nabla u \cdot A \nabla v dx, \ (u,v)\in D(\mathcal{A}_2)\times \mcH^1(\Omega),$ are continuous, positive, symmetric.
			\item [(3)]  The operators $(\mathcal{A}_1,D(\mathcal{A}_1))$ and $(\mathcal{A}_2,D(\mathcal{A}_2))$ are self-adjoint and dissipative.
		\end{itemize}
		%are the infinitesimal generators of the strongly continuous semigroups denoted by $e^{t\mathcal{A}_1}, e^{t\mathcal{A}_2}$ respectively.
	\end{lemma}
	\begin{proof}
		Clearly, the continuity of $i_1$ and $i_2$ follows directly from the definition of $D(\mathcal{A}_1)$ and $D(\mathcal{A}_2)$. Consequently, since $C_0^\infty (\Om)$ is dense in $D(\mathcal{A}_1)$, it is also dense in $L^2(\Om)$. Moreover, since $C_0^\infty (\Om)\subset \mathcal{H}^2(\Om)\subset L^2(\Om)$, we have $\mathcal{H}^2(\Om)$ is dense in $L^2(\Om)$. This establishes (1).
		
		Next, we prove (2). It is evident that $q_1$ and $q_2$ are symmetric, and $q_1(u,u) \ge 0$ and $q_2(u,u) \ge 0$ for all $u$. Furthermore,
		$$
		|q_1(u,v)|\le\left\| u\right\| _{\mcH_0^1(\Omega)} \left\| v\right\| _{\mcH_0^1(\Omega)},\ |q_2(u,v)|\le\left\| u\right\| _{\mcH^1(\Omega)} \left\| v\right\| _{\mcH^1(\Omega)}.
		$$
		Hence, $q_1$ and $q_2$ are continuous.
		
		Now, let $(u_1,v_1) \in D(\mathcal{A}_1)\times \mcH^1_0(\Omega)$ and $(u_2,v_2)\in D(\mathcal{A}_2)\times \mcH^1(\Omega)$. By applying Lemma \ref{div}, we have
		$$
		q_1(u_1,v_1)=-\int_{\Om}\mathcal{A}_1 u_1 \cdot v_1 d x, \ q_2(u_2,v_2)=-\int_{\Om}\mathcal{A}_2 u_2 \cdot v_2 d x.
		$$
		Furthermore, since
		$$
		\begin{aligned}
			\left\| (\gamma I - \mathcal{A}_1)u_1\right\| _{L^2(\Om)}
			&=\sup_{\|v_1\|_{L^2(\Om)} \le 1, v_1 \in \mcH_0^1(\Omega)} \int_{\Omega} \gamma v_1 u_1 - v_1 \mathcal{A}_1 u_1 dx
			%=&\sup_{|v|_2 \le 1, v \in \mcH_0^1(\Omega)} \int_{\Omega} \gamma v u dx + q(u,v)\\
			\ge  \gamma \left\| u_1\right\| _{L^2(\Om)},\\
			\left\| (\gamma I - \mathcal{A}_2)u_2\right\| _{L^2(\Om)}
			&=\sup_{\|v_2\|_{L^2(\Om)} \le 1, v_2 \in \mcH^1(\Omega)} \int_{\Omega} \gamma v_2 u_2 - v_2 \mathcal{A}_2 u_2 dx
			%=&\sup_{|v|_2 \le 1, v \in \mcH^1(\Omega)} \int_{\Omega} \gamma v u dx + q(u,v)\\
			\ge  \gamma \left\| u_2\right\| _{L^2(\Om)},
		\end{aligned}
		$$
		and a similar inequality holds for $(\gamma I - \mathcal{A}_2)u_2$, we conclude that $\mathcal{A}_1$ and $\mathcal{A}_2$ are self-adjoint and dissipative, as desired.
	\end{proof}
	
	
	Consequently, both $\mathcal{A}_1$ and $\mathcal{A}_2$ serve as the infinitesimal generators of the strongly continuous semigroups denoted by $e^{t\mathcal{A}_1}$ and $e^{t\mathcal{A}_2}$, respectively. Additionally, the family of operators in $\mathcal{L}(L^2(\Om))$ given by
	$$
	B(t)u := b(t,\cdot)u, \quad t\in (0,T), \quad u\in L^2(\Om)
	$$
	can be regarded as a family of bounded perturbations of $\mathcal{A}_1$ (resp. $\mathcal{A}_2$). Consequently, utilizing standard techniques (see \cite{bensoussan2007representation,cazenave1998introduction,showalter1995hilbert}), one can establish the validity of Theorem \ref{exist1} and Theorem \ref{exist2}.
	We have now demonstrated the existence of solutions to equation \eqref{1.1}, with the solution $z \in C([0,T];L^{2}(\Omega))\cap L^{2}(0,T;\mathcal{H}_0^{1}(\Omega))$. Our aim is to establish that the regularity of the solutions can be enhanced within the interior of $\Omega$. By employing standard arguments found in \cite{EVANS}, we can establish the following result.
	\begin{theorem}\label{Regularity}
		For all solutions of equation \eqref{1.1}, it holds that $A\nabla u \in W^{1,1}(\Om_0)$ and $\nabla u A\nabla u \in W^{1,1}(\Om_0)$.
	\end{theorem}
	
	\section{Carleman esitimates}
	Let us now derive a Carleman estimate. Consider $\eta \in C_0^3(\Omega)$ satisfying
	\begin{equation*}
		\eta(x) :=
		\begin{cases}
			=0, & x \in \Omega\backslash\hat{\Om},\\
			>0, & x \in \hat{\Om},
		\end{cases}
	\end{equation*}
	and
	\begin{equation*}
		\quad \left\| \nabla \eta \right\|_{L^2(\Om)} \ge C >0 \ \mbox{in} \ \Om_0, \quad  \nabla \eta=0 \ \mbox{on} \ \partial \hat{\Om}.
	\end{equation*}
	%By the classic arguments in \cite{CA6}, we can move $(\frac{1+\delta}{2},\frac{1+\delta}{2})$ to $\omega$, then we have
	%\begin{equation*}
	%\left| \nabla \eta\right| \ge C > 0, \ \mbox{in} \ \overline{\Omega \backslash \omega},
	%\end{equation*}
	Define
	\begin{equation*}
		\begin{split}
			& \theta(t):=[t(T-t)]^{-4}, \quad \xi(x, t):=\theta(t) e^{ \lambda(8|\eta|_\infty+\eta (x))}, \quad \sigma(x, t):=\theta(t) e^{10 \lambda|\eta|_\infty}-\xi(x, t).
		\end{split}
	\end{equation*}
	In what follows, $C>0$ represents a generic constant, and $w$ denotes a solution of equation \eqref{3.1}. We can assume, using standard arguments, that $w$ possesses sufficient regularity. Specifically, we consider $w \in H^1(0,T; \mathcal{H}_0^1(\Omega))$ with homogeneous Dirichlet boundary conditions and $w \in H^1(0,T; \mathcal{H}^1(\Omega))$ with homogeneous Neumann boundary conditions.
	
	
	Consider $s>s_0>0$ and introduce
	\begin{equation*}
		u=e^{-s\sigma} w.  
	\end{equation*}
	Then, the following properties hold for $u$:
	\begin{itemize}
		\item [($i$)] $u=\frac{\partial u}{\partial x_i}=0$ at $t=0$ and $t=T$;
		\item [($ii$)] $u=0$ or $A\nabla u \cdot \nu =0 $ on $\Sigma$;
		\item [($iii$)] If $P_1 u:=u_t+s \Div(u A \nabla \sigma)+s \nabla \sigma A \nabla u$ and $P_2 u:=\Div(A \nabla u)+s^2 u \nabla \sigma A \nabla \sigma+s \sigma_t u$, then $P_1 u+P_2 u=e^{-s \sigma} f$.
	\end{itemize}
	From item ($iii$), it follows that
	\begin{equation}\label{PP}
		\left\|P_1 u\right\|^2+\left\|P_2 u\right\|^2+2\left(P_1 u, P_2 u\right)=\left\|e^{-s \sigma} f\right\|^2.
	\end{equation}
	
	Let us define $\left(P_1 u, P_2 u\right)=I_1+\cdots+I_4$, where
	\begin{equation*}
		\begin{split}
			& I_1:=\left(\Div(A \nabla u)+s^2 u \nabla \sigma \cdot A \nabla \sigma+s \sigma_t u, u_t\right), \\
			& I_2:=s^2\left(\sigma_t u, \Div(u A \nabla \sigma)+\nabla \sigma A \nabla u\right), \\
			&I_3:=s^3\left( u \nabla \sigma \cdot A \nabla \sigma, \Div(u A \nabla \sigma)+\nabla \sigma \cdot A \nabla u\right),\\
			& I_4:=s(\Div(A \nabla u), \Div(u A \nabla \sigma)+\nabla \sigma \cdot A \nabla u).
		\end{split}
	\end{equation*}
	By differentiating with respect to the time variable $t$, we can improve the regularity of $u$, leading to $u_t \in \mathcal{H}_0^1(\Omega)$. From Theorem \ref{div} and item (i), we have
	\begin{equation}\label{I11}
		\begin{split}
			I_1=&\iint_Q u_t \Div (A\nabla u) +s^2 u \nabla \sigma \cdot A\nabla \sigma u_t + s\sigma_t u u_t dx  dt\\
			=&\int_{\Om} s^2  \nabla \sigma A\nabla \sigma\cdot \frac{1}{2}u^2 dx  \bigg|_0^T + \int_{\Om} s\sigma_t\cdot \frac{1}{2}u^2 dx  \bigg|_0^T+ \iint_\Sigma u_t A\nabla u \cdot \nu ds dt\\
			&-\iint_{Q}  A\nabla u \cdot \nabla u_t dx  dt -\frac{1}{2}\iint_{Q}(s\sigma_t + s^2 \nabla \sigma \cdot A\nabla \sigma)_t u^2 dx  dt\\
			=& - \frac{1}{2}\int_{\Om}  A\nabla u \cdot \nabla u dx  \bigg|_0^T -\frac{1}{2}\iint_{Q}(s\sigma_t + s^2 \nabla \sigma \cdot A\nabla \sigma)_t u^2 dx  dt\\
			=& -\frac{1}{2}\iint_{Q}(s\sigma_t + s^2 \nabla \sigma \cdot A\nabla \sigma)_t u^2 dx  dt.
		\end{split}
	\end{equation}
	Indeed, we have $\iint_\Sigma u_t A\nabla u \cdot \nu ds dt=0$ for the Dirichlet boundary condition since $u_t \in \mathcal{H}_0^1(\Omega)$, and $\iint_\Sigma u_t A\nabla u \cdot \nu ds dt=0$ for the Neumann boundary condition since $A\nabla u\cdot \nu=0$. Furthermore, we obtain
	\begin{equation}\label{I22}
		\begin{split}
			I_2  =& s^2  \iint_Q \sigma_t u (\Div(u A \nabla \sigma) + A\nabla u \cdot \nabla \sigma) dx  dt
			= s^2  \iint_Q \sigma_t u (\Div( A \nabla \sigma) + 2 A\nabla u \cdot \nabla \sigma) dx  dt\\
			=& s^2  \iint_Q \sigma_t u \Div( A \nabla \sigma) + \Div( A \nabla \sigma) \sigma_t u^2  dx  dt\\
			=& s^2 \iint_\Sigma \sigma_t u^2 A \nabla \sigma  \cdot \nu dsdt - s^2  \iint_Q \Div(\sigma_t A \nabla \sigma)u^2 dx dt + s^2  \iint_Q \Div( A \nabla \sigma) \sigma_t u^2 dx  dt\\
			=& - s^2  \iint_Q \Div( A \nabla \sigma) \sigma_t u^2 dx  dt -s^2\iint_Q  A \nabla \sigma \cdot\nabla \sigma_t u^2    dx  dt + s^2\iint_Q \Div( A \nabla \sigma) \sigma_t u^2 dx dt\\
			=& -s^2\iint_Q  A \nabla \sigma \cdot\nabla \sigma_t u^2    dx  dt.
		\end{split}
	\end{equation}
	In the fourth equality, we can observe that $s^2 \iint_\Sigma \sigma_t u^2 A \nabla \sigma \cdot \nu dsdt=0$ for the Dirichlet boundary condition, as $u \in \mathcal{H}_0^1(\Omega)$, and $s^2 \iint_\Sigma \sigma_t u^2 A \nabla \sigma \cdot \nu dsdt=0$ for the Neumann boundary condition, as $\nabla \eta=0$ on $\partial \Omega$.
	
	Similarly, in $I_3$ below, $s^3 \iint_\Sigma u^2 (A \nabla \sigma \cdot \nabla \sigma) A \nabla \sigma \cdot \nu ds dt=0$ holds for the Dirichlet boundary condition due to $u \in \mathcal{H}_0^1(\Omega)$, and $s^3 \iint_\Sigma u^2 (A \nabla \sigma \cdot \nabla \sigma) A \nabla \sigma \cdot \nu ds dt=0$ holds for the Neumann boundary condition as $\nabla \eta=0$ on $\partial \Omega$.
	In Equation (\ref{I33}), we have
	\begin{equation}\label{I33}
		\begin{split}
			I_3  =&s^3 \iint_Q u A \nabla \sigma \cdot \nabla \sigma (\Div (u A \nabla \sigma) + A\nabla u \cdot \nabla \sigma ) dx  dt\\
			=&s^3 \iint_\Sigma u^2 (A \nabla \sigma \cdot \nabla \sigma) A \nabla \sigma \cdot \nu  ds dt -s^3 \iint_Q u A \nabla \sigma \cdot \nabla (u A \nabla \sigma \cdot \nabla \sigma  ) dx  dt\\
			&+ s^3 \iint_Q  (A \nabla \sigma \cdot \nabla u) ( A \nabla \sigma \cdot \nabla \sigma  ) u dx  dt\\
			=&-s^3\iint_Q  A\nabla \sigma \cdot \nabla ( A\nabla \sigma \cdot\nabla \sigma) u^2 dx  dt.
		\end{split}
	\end{equation}
	Similarly, for $I_4$, we have
	\begin{equation*}
		\begin{split}
			I_4= & s \iint_Q \Div(A \nabla u)\left(  \Div(u A \nabla \sigma)+ A \nabla u \cdot \nabla \sigma\right)  dx  dt \\%1
			= & s \iint_Q \Div(A \nabla u)\left( A \nabla u \cdot \nabla \sigma+ u \Div(A \nabla \sigma)+ A \nabla u \cdot \nabla \sigma\right)  dx  dt \\%2
			= & s \iint_Q \Div(A \nabla u) u \Div(A \nabla \sigma)dx  dt + 2s \iint_Q \Div(A \nabla u) A \nabla u \cdot \nabla \sigma   dx  dt \\%3
			= & s \iint_\Sigma u \Div(A \nabla \sigma )A \nabla u \cdot \nu dsdt + 2s \iint_\Sigma  (A \nabla u \cdot \nabla \sigma) A \nabla u \cdot \nu ds dt \\%4
			&-s \iint_Q A \nabla u \cdot  \nabla\left( u \Div(A \nabla \sigma)\right) dx  dt -2s \iint_{Q} A \nabla u\cdot \nabla \left( A \nabla u \cdot \nabla \sigma \right)  dx  dt \\%5
			= &
			-s \iint_Q A \nabla u \cdot  \nabla u \Div(A \nabla \sigma)dx  dt-s \iint_Q u A \nabla u\cdot \nabla (\Div(A \nabla \sigma)) dx  dt\\%6
			&-2s \iint_{Q} A \nabla u\cdot \nabla(A \nabla u \cdot \nabla \sigma)   dx  dt, 
		\end{split}
	\end{equation*}
	here, in the forth equality, $s \iint_\Sigma u \Div(A \nabla \sigma )A \nabla u \cdot \nu dsdt=0$ can be found in the Dirichlet boundary condition since $u \in \mathcal{H}_0^1(\Om)$ and $s \iint_\Sigma u \Div(A \nabla \sigma )A \nabla u \cdot \nu dsdt=0$ in the Neumann boundary condition since $\nabla \eta=0$ on $\partial \Om$.
	Since
	\begin{equation*}
		\begin{split}
			&-2s \iint_{Q} A \nabla u \cdot \nabla (A\nabla u \cdot \nabla \sigma) dx dt\\
			=&-2s \sum_{i=1}^{2} \iint_{Q} (A\nabla \sigma)_i A \nabla u \cdot \frac{\partial}{\partial x_i}(\nabla u) + (\nabla u)_i A \nabla u \cdot \nabla (A \nabla \sigma)_i dx dt,\\
			%=&-2s \iint_{Q} A \nabla u \left[ y^{\alpha_y} \frac{\partial \sigma}{\partial x}\nabla \frac{\partial u}{\partial x}  + x^{\alpha_x} \frac{\partial \sigma}{\partial y}\nabla \frac{\partial u}{\partial y} + \frac{\partial u}{\partial x} \nabla (y^{\alpha_y} \frac{\partial \sigma}{\partial x}) + \frac{\partial u}{\partial y} \nabla (x^{\alpha_x} \frac{\partial \sigma}{\partial y}) \right] dx dt,
		\end{split}
	\end{equation*}
	where
	\begin{equation*}
		\begin{split}
			&-2s \sum_{i=1}^{2} \iint_{Q} (A\nabla \sigma)_i A \nabla u \cdot \frac{\partial}{\partial x_i}(\nabla u) dx dt\\%1
			%=& -2s \iint_Q y^{\alpha_y} \frac{\partial \sigma}{\partial x}  A \nabla u \cdot \frac{\partial}{\partial x}(\nabla u) dx dt - 2s \iint_Q x^{\alpha_x} \frac{\partial \sigma}{\partial y}  A \nabla u \cdot \frac{\partial}{\partial y}(\nabla u) dx dt\\%2
			=& -s \sum_{i=1}^{2}\iint_{Q} (A\nabla \sigma)_i \left[ \frac{\partial}{\partial x_i} (A \nabla u \cdot \nabla u) - \frac{\partial A}{\partial x_i} \nabla u \cdot \nabla u \right] dx dt\\%2
			=&-s \iint_\Sigma (A \nabla u \cdot \nabla u)(A\nabla \sigma) \cdot \nu dsdt + s\iint_{Q} (A \nabla u \cdot \nabla u ) \Div(A\nabla \sigma) dx dt\\
			&+s \sum_{i=1}^{2} \iint_{Q} (A\nabla \sigma)_i \frac{\partial A}{\partial x_i} \nabla u \cdot \nabla u dx dt\\%4
			=&  s \iint_{Q} A\nabla u \cdot \nabla u \Div(A\nabla \sigma) dx dt
			+ s  \sum_{i=1}^{2}\iint_{Q} (A\nabla \sigma)_i \frac{\partial A}{\partial x_i} \nabla u \cdot \nabla u dx dt, 
		\end{split}
	\end{equation*}
	here $-s \iint_\Sigma (A \nabla u \cdot \nabla u)(A\nabla \sigma) \cdot \nu dsdt=0$ on $\partial \Om$ since $\nabla \eta=0$ on $\partial \Om$.
	
	
	Hence, we have
	\begin{equation}\label{I44}
		\begin{split}
			I_4 = & -s \iint_Q u A \nabla u\cdot \nabla (\Div(A \nabla \sigma))dx  dt
			-2s  \sum_{i=1}^{2}\iint_{Q} (\nabla u)_i A \nabla u \cdot \nabla (A \nabla \sigma)_i dx dt  \\
			&
			+ s  \sum_{i=1}^{2}\iint_{Q} (A\nabla \sigma)_i \frac{\partial A}{\partial x_i} \nabla u \cdot \nabla u dx dt.
		\end{split}
	\end{equation}
	By combining \eqref{I11} to \eqref{I44}, we can conclude that
	\begin{equation}\label{PPu}
		\begin{aligned}
			&\left(P_1 u, P_2 u\right)\\
			= & -s^3 \iint_Q A \nabla \sigma\cdot \nabla(\nabla \sigma A \nabla \sigma)|u|^2 dx  dt
			-2s  \sum_{i=1}^{2}\iint_{Q} (\nabla u)_i A \nabla u \cdot \nabla (A \nabla \sigma)_i dx dt \\
			& -2s^2 \iint_Q \nabla \sigma A \nabla \sigma_t|u|^2 dx  dt+ s \sum_{i=1}^{2} \iint_{Q} (A\nabla \sigma)_i \frac{\partial A}{\partial x_i} \nabla u \cdot \nabla u dx dt\\
			&
			-s \iint_Q u A \nabla u \nabla(\Div(A \nabla \sigma)) dx  dt-\frac{s}{2} \iint_Q \sigma_{t t}|u|^2 dx  dt.
			%+2 s\iint_{\Sigma}(\nabla u A \nabla \sigma) A \nabla u \cdot \nu d s d t -s\iint_{\Sigma}(\nabla u A \nabla u) A \nabla \sigma \cdot \nu d s d t .
		\end{aligned}
	\end{equation}
	Let us denote the seven integrals on the right-hand side of \eqref{PPu} by $T_1, \cdots, T_6$. Now, we will estimate each of them.
	% For the integral on the boundary we will use the following result.
	%\begin{equation*}%需验证
	%2 s\iint_{\Sigma}(\nabla u A \nabla \sigma) A \nabla u \cdot \nu d s d t -s\iint_{\Sigma}(\nabla u A \nabla u) A \nabla \sigma \cdot \nu d s d t  \geq 0 .
	%\end{equation*}
	Using the definitions of $\sigma$ and $\xi$ and the properties of $\eta$, we have $\nabla \sigma = - \lambda \xi \nabla \eta$, $\nabla \xi = \lambda \xi \nabla \eta$, $A \nabla \sigma \cdot \nabla \sigma = \lambda^2 \xi^2 A\nabla \eta \cdot \nabla \eta$, and $\nabla( A \nabla \sigma \cdot \nabla \sigma) = \lambda^2 \xi^2 \nabla (A\nabla \eta \cdot \nabla \eta) + 2\lambda^2 \xi^2 \nabla \eta \left| A \nabla \eta \cdot \nabla \eta \right|$. Therefore, we can rewrite $T_1$ as follows:
	\begin{equation}
		\begin{aligned}
			T_1
			&=-s^3 \iint_Q A \nabla \sigma \nabla(\nabla \sigma A \nabla \sigma)|u|^2 dx  dt\\
			&=-s^3 \iint_Q  \left( -\lambda \xi A \nabla \eta \right) \left( \lambda^2\xi^2 \nabla(\nabla \eta A \nabla \eta) + 2\lambda^3\xi^2 \nabla \eta \left| A \nabla \eta \cdot \nabla \eta \right| ^2\right) |u|^2 dx  dt\\
			&=2s^3 \lambda^4 \iint_Q \xi^3\left|A \nabla \eta \cdot \nabla \eta \right|^2|u|^2 dx  dt
			+ s^3 \lambda^3 \iint_Q \xi^3 A\nabla \eta \cdot \nabla(A\nabla \eta \cdot \nabla\eta)|u|^2 dx  dt.
		\end{aligned}
	\end{equation}
	Since $A\nabla \eta \cdot\nabla(A\nabla \eta \cdot \nabla\eta)$ is bounded in $\overline{\Omega}$, we have
	$$
	s^3 \lambda^3 \int_{0}^{T}\int_{\omega} \xi^3 A\nabla \eta \cdot \nabla(A\nabla \eta \cdot \nabla\eta)|u|^2 dx  dt \ge -C s^3 \lambda^3 \int_{0}^{T}\int_{\omega} \xi^3 |u|^2 dx  dt,
	$$
	and $|A \nabla \eta \cdot \nabla \eta|\ge C>0$ in $\overline{\Omega} \setminus \omega$. Thus, we have
	$$
	\begin{aligned}
		&s^3 \lambda^3 \int_{0}^{T}\int_{\Om\backslash \omega} \xi^3 A\nabla \eta \cdot \nabla(A\nabla \eta \cdot \nabla\eta)|u|^2 dx  dt\\
		\ge& -C s^3 \lambda^3 \int_{0}^{T}\int_{\Om\backslash \omega} \xi^3 |u|^2 dx  dt\\
		\ge&
		-C s^3 \lambda^3 \int_{0}^{T}\int_{\Om\backslash \omega} \xi^3 \left|A \nabla \eta \cdot \nabla \eta \right|^2 |u|^2 dx  dt\\
		\ge&
		-C s^3 \lambda^3 \iint_{Q} \xi^3 \left|A \nabla \eta \cdot \nabla \eta \right|^2 |u|^2 dx  dt.
	\end{aligned}
	$$
	Hence, we can rewrite $T_1$ as:
	\begin{equation}\label{T1}
		\begin{aligned}
			T_1
			\geq C s^3 \lambda^4 \iint_Q \xi^3\left|A \nabla \eta \cdot \nabla \eta \right|^2|u|^2 dx  dt-C s^3 \lambda^3\int_0^T \int_{\omega} \xi^3|u|^2 dx  dt.
		\end{aligned}
	\end{equation}
	Similarly, for $T_2$, we have:
	\begin{equation}\label{T2}
		\begin{aligned}
			T_2 =&-2s \sum_{i=1}^{2} \iint_{Q} (\nabla u)_i A \nabla u \cdot \nabla (A \nabla \sigma)_i dx dt\\
			=&-2s \iint_{Q}( D(A\nabla \sigma) A\nabla u) \cdot \nabla u dx  dt\\
			=&2s \iint_{Q}( \lambda^2 \xi (A\nabla \eta \cdot \nabla u) A\nabla \eta + \lambda \xi D(A\nabla \eta) A\nabla u ) \cdot \nabla u dx  dt\\
			=&2s \lambda^2\iint_{Q}  \xi \left|  A\nabla \eta \cdot \nabla u \right| ^2 dx  dt  + 2s\lambda \iint_{Q}\xi D(A\nabla \eta) A\nabla u  \cdot \nabla u dx  dt\\
			\ge& Cs \lambda^2\iint_{Q}  \xi \left|  A\nabla \eta \cdot \nabla u \right| ^2 dx  dt  - C s\lambda \iint_{Q}\xi A\nabla u  \cdot \nabla u dx  dt.
		\end{aligned}
	\end{equation}
	Furthermore, for $T_3$, as can be seen from the definition of $\xi$, we have $\xi \xi_t \le \xi^3$, thus
	\begin{equation*}
		\begin{aligned}
			T_3 &=-2 s^2 \iint_{Q} A\nabla \sigma \cdot \nabla \sigma_t u^2 dx dt
			=-2 s^2 \lambda^2 \iint_{Q} \xi \xi_t \left|A \nabla \eta \cdot \nabla \eta \right|  u^2 dx dt\\
			&\ge-2 s^2 \lambda^2 \iint_{Q} \xi^3 \left|A \nabla \eta \cdot \nabla \eta \right|^2  u^2 dx dt.
		\end{aligned}
	\end{equation*}
	Similarly, we can express $T_3$ as:
	\begin{equation}\label{T3}
		\begin{aligned}
			T_3
			\ge &-C s^2 \lambda^2 \iint_{Q} \xi^3 \left|A \nabla \eta \cdot \nabla \eta \right|^2  u^2 dx dt -C s^2 \lambda^2 \int_{0}^{T}\int_{\omega} \xi^3   u^2 dx dt,
		\end{aligned}
	\end{equation}
	and $T_4$ as:
	\begin{equation}\label{T4}
		\begin{aligned}
			T_4 =&s \iint_{Q} (A\nabla \sigma)_i \frac{\partial A}{\partial x_i} \nabla u \cdot \nabla u dx dt
			\ge  -C s \lambda \iint_Q  \xi A \nabla u \cdot \nabla u dx  dt.
		\end{aligned}
	\end{equation}
	Furthermore, by utilizing the definitions of $\sigma$ and $\xi$, we obtain:
	\begin{equation}\label{T5}
		\begin{split}
			T_5  =&-s \iint_Q u A \nabla u \cdot \nabla(\Div(A \nabla \sigma)) dx  dt\\
			=&s \lambda^3 \iint_Q \xi u A \nabla u \cdot \nabla \eta \left( A \nabla \eta \cdot \nabla \eta\right)  dx  dt
			+ s \lambda^2 \iint_Q \xi u A \nabla u \cdot \nabla\left(A \nabla \eta \cdot \nabla \eta\right) dx  dt \\
			& +s \lambda^2 \iint_Q \xi u A \nabla u \cdot \nabla \eta \Div(A \nabla \eta)   dx  dt+s \lambda \iint_Q \xi u A \nabla u \cdot \nabla(\Div(A \nabla \eta)) dx  dt.
		\end{split}
	\end{equation}
	Let us denote $T_{51}, \cdots, T_{54}$ as the seven integrals on the right-hand side of \eqref{T5}. Then we have:
	\begin{equation}\label{T51}
		\begin{split}
			T_{51}=&s \lambda^3 \iint_Q \xi u A \nabla u \cdot \nabla \eta \left( A \nabla \eta \cdot \nabla \eta\right)  dx  dt\\
			\geq&
			-Cs^2 \lambda^4 \iint_Q \xi\left|A \nabla \eta \cdot \nabla \eta\right|^2|u|^2 dx  dt
			-C\lambda^2 \iint_Q \xi| A\nabla u \cdot\nabla \eta|^2 dx  dt,
		\end{split}
	\end{equation}
	and
	\begin{equation}\label{T52}
		\begin{split}
			T_{52}=&s \lambda^2 \iint_Q \xi u A \nabla u \cdot \nabla\left(A \nabla \eta \cdot \nabla \eta\right) dx  dt\\
			\ge&-Cs^2 \lambda^3\iint_{Q} \xi \left| A \nabla \eta \cdot \nabla \eta \right| u^2 dx  dt -Cs^2 \lambda^3\int_0^T \int_{\omega} \xi \left| A \nabla u \cdot \nabla u \right|  u^2 dx  dt\\
			&- C\lambda \iint_Q  \xi A \nabla u \cdot \nabla u dx  dt.
		\end{split}
	\end{equation}
	Then we have
	\begin{equation}\label{T53}
		\begin{split}
			T_{53}=&s \lambda^2 \iint_Q \xi u A \nabla u \cdot \nabla \eta  \Div(A \nabla \eta) dx  dt\\
			\ge& -Cs^2 \lambda^3\iint_Q \xi \left| A \nabla \eta \cdot \nabla \eta \right|^2  u^2  dx  dt -Cs^2 \lambda^3\int_0^T \int_{\omega} \xi u^2 dx  dt\\
			&- C\lambda \iint_Q \xi A \nabla u \cdot \nabla u dx  dt,
		\end{split}
	\end{equation}
	and
	\begin{equation}\label{T54}
		\begin{split}
			T_{54}=&s \lambda \iint_Q \xi u A \nabla u \nabla(\Div(A \nabla \eta))dx  d t\\
			\ge&
			-C s^2 \lambda \iint_Q \xi \left|A \nabla \eta \cdot \nabla \eta\right|^2 u^2dx  d t-C s^2 \lambda \int_0^T \int_{\omega}\xi  u^2dx  d t\\
			&-C \lambda \iint_Q \xi A \nabla u \cdot \nabla udx  d t.
		\end{split}
	\end{equation}
	Combining \eqref{T5} to \eqref{T54}, we obtain:
	\begin{equation}\label{T55}
		\begin{split}
			T_5 \geq&-Cs^2 \lambda^4 \iint_Q \xi\left|A \nabla \eta \cdot \nabla \eta\right|^2 |u|^2dx  d t -C\lambda^2\iint_{Q} \xi \left| A\nabla u\cdot \nabla \eta \right| ^2 dx  dt\\
			&-C s^2 \lambda^3 \int_0^T \int_{\omega} \xi|u|^2dx  d t- C\lambda \iint_Q  \xi A \nabla u \cdot \nabla u dx  d t.
		\end{split}
	\end{equation}
	Finally, as we can observe from the definitions of $\xi$ and $\sigma$, we have $\sigma_{tt} \leq \xi^{\frac{3}{2}}$. Hence,
	\begin{equation}\label{T6}
		T_6 =-\frac{s}{2} \iint_Q \sigma_{t t}|u|^2dx  d t \geq-C s \iint_Q \xi^{3 / 2}|u|^2dx  d t.
	\end{equation}
	From equations \eqref{T1}-\eqref{T4}, \eqref{T55}, and \eqref{T6}, we deduce the following inequality:
	\begin{equation}\label{PPU}
		\begin{split}
			\left(P_1 u, P_2 u\right)
			\geq& C\iint_Q s^3 \lambda^4 \xi^3 |\nabla \eta \cdot A \nabla \eta|^2 |u|^2 dx dt  + C s  \lambda^2 \iint_Q\xi|\nabla u \cdot A \nabla \eta|^2 dx  d t\\
			&-Cs^3 \lambda^3 \int_0^T \int_{\omega} \xi^3|u|^2dx  d t
			- Cs \lambda \iint_Q \xi A \nabla u \cdot \nabla udx  d t\\
			&- Cs \iint_Q \xi^{3 / 2}|u|^2dx  d t.
		\end{split}
	\end{equation}
	Clearly, $s \iint_Q \xi^{3 / 2}|u|^2dx dt$ can be absorbed by other terms. Combining \eqref{PP} and \eqref{PPU}, we can conclude that
	\begin{equation}\label{LLU2}
		\begin{split}
			&\left\|P_1 u\right\|^2+\left\|P_2 u\right\|^2+C\iint_Q s^3 \lambda^4 \xi^3\left|A \nabla \eta \cdot \nabla \eta \right|^2|u|^2 dx dt  + C s  \lambda^2 \iint_Q\xi|\nabla u \cdot A \nabla \eta|^2 dx  d t\\
			\leq& C\left\|e^{-s \sigma} f\right\|^2
			+Cs \lambda \iint_Q \xi A \nabla u \cdot \nabla udx  d t+Cs^3 \lambda^3 \int_0^T \int_{\omega} \xi^3|u|^2dx  d t.
		\end{split}
	\end{equation}
	Furthermore, we can deduce that
	\begin{equation*}
		s^2 \lambda^2 \iint_Q \xi^{2}|u|^2dx  d t=s^2 \lambda^2 \int_0^T \int_{\Om\backslash \omega} \xi^{2}|u|^2dx  d t + s^2 \lambda^2 \int_0^T \int_{\omega} \xi^{2}|u|^2dx  d t.
	\end{equation*}
	Clearly, the second term on the right in \eqref{LLU2} can be absorbed by the other terms. Let us now consider the first term on the right, where we have
	\begin{equation*}
		\begin{split}
			s^2 \lambda^2 \int_0^T \int_{\Om\backslash \omega} \xi^{2}|u|^2dx  d t
			&\le Cs^2 \lambda^2 \int_0^T \int_{\Om\backslash \omega} \xi^{2}\left|A \nabla \eta \cdot \nabla \eta \right|^2|u|^2dx  d t\\
			&\le Cs^2 \lambda^2 \iint_Q \xi^{2}\left|A \nabla \eta \cdot \nabla \eta \right|^2|u|^2dx  d t.
		\end{split}
	\end{equation*}
	Therefore, we have
	\begin{equation}\label{LLU3}
		\begin{split}
			&\left\|P_1 u\right\|^2+\left\|P_2 u\right\|^2+C s^2 \lambda^2 \int_0^T \int_{\Om\backslash \omega} \xi^{2}|u|^2dx  d t\\
			&+  C\iint_Q s^3 \lambda^4 \xi^3\left|A \nabla \eta \cdot \nabla \eta \right|^2|u|^2 dx dt  + C s  \lambda^2 \iint_Q\xi|\nabla u \cdot A \nabla \eta|^2 dx  d t\\
			\leq& C\left\|e^{-s \sigma} f\right\|^2
			+Cs \lambda \iint_Q \xi A \nabla u \cdot \nabla udx  d t+Cs^3 \lambda^3 \int_0^T \int_{\omega} \xi^3|u|^2dx  d t.
		\end{split}
	\end{equation}
	Using the definitions of $P_1 u$ and $P_2 u$, we can observe that
	\begin{equation*}
		\begin{split}
			&s^{-1} \iint_Q \xi^{-1}\left|u_t\right|^2dx  d t\\
			=&s^{-1} \iint_Q \xi^{-1}(P_1 u -s u \Div(A\nabla \sigma)- 2s \nabla u \cdot A\nabla \sigma)^2dx  d t\\
			\le& Cs^{-1}\left\|P_1 u\right\|^2+s   \iint_Q \xi^{-1}|u|^2|\Div(A \nabla \sigma)|^2dx  d t +C s \iint_Q \xi^{-1}|\nabla u A \nabla \sigma|^2dx  d t \\
			\leq& Cs^{-1}\left\|P_1 u\right\|^2+C s \lambda^4 \iint_Q \xi\left| A \nabla \eta \cdot \nabla \eta \right|^2 |u|^2dx  d t \\
			&+C s \lambda^2  \iint_Q \xi|u|^2dx  d t+C s \lambda^2 \iint_Q \xi|\nabla u \cdot A \nabla \eta|^2dx  d t,
		\end{split}
	\end{equation*}
	and
	\begin{equation*}
		\begin{split}
			&s^{-1}   \iint_Q \xi^{-1}\left|\Div(A\nabla u)\right|^2dx  d t\\
			=& s^{-1}   \iint_Q \xi^{-1}(P_2 u -s^2 u \nabla \sigma \cdot A \nabla \sigma - s \sigma_t u)^2dx  d t\\
			\leq& C s^{-1} \left\|P_2 u\right\|^2+s^3   \iint_Q \xi^{-1}|u|^2\left|\nabla \sigma \cdot A \nabla \sigma\right|^2dx  d t +C s   \iint_Q \xi^{-1}\left|\sigma_t\right|^2|u|^2dx  d t \\
			\leq& C s^{-1} \left\|P_2 u\right\|^2+C s^3 \lambda^4   \iint_Q \xi^3|\nabla \eta \cdot A \nabla \eta|^2|u|^2dx  d t +C s   \iint_Q \xi^{2}|u|^2dx  d t.
		\end{split}
	\end{equation*}
	From equation \eqref{LLU3}, we obtain the inequality
	\begin{equation}\label{LLU4}
		\begin{split}
			&s^{-1}\iint_{Q} \xi^{-1} (|u_t |^2 + \left|\Div(A\nabla u)\right|^2) dx dt +C s^2 \lambda^2 \int_0^T \int_{\Om\backslash \omega} \xi^{2}|u|^2dx  d t\\
			&+C\iint_Q s^3 \lambda^4 \xi^3\left|A \nabla \eta \cdot \nabla \eta \right|^2|u|^2 dx dt  + C s  \lambda^2 \iint_Q\xi|\nabla u \cdot A \nabla \eta|^2 dx  d t\\
			\leq& C\left\|e^{-s \sigma} f\right\|^2
			+Cs \lambda \iint_Q \xi A \nabla u \cdot \nabla udx  d t+Cs^3 \lambda^3 \int_0^T \int_{\omega} \xi^3|u|^2dx  d t.
		\end{split}
	\end{equation}
	Considering the term $s \lambda \iint_Q \xi A \nabla u \cdot \nabla udx d t$, we have
	\begin{equation*}
		\begin{split}
			s \lambda \iint_Q \xi A \nabla u \cdot \nabla udx  d t=& s \lambda \iint_\Sigma \xi u A \nabla u \cdot \nu d s d t - s \lambda \iint_Q \Div(\xi A\nabla u) udx  d t\\
			=&- s \lambda \iint_Q \Div(\xi A\nabla u) udx  d t.
		\end{split}
	\end{equation*}
	Then, we can further simplify the expression as follows:
	\begin{equation*}
		\begin{split}
			&- s \lambda \iint_Q \Div(\xi A\nabla u) udx  d t\\
			=&- s \lambda^2 \iint_Q \xi u\nabla \eta \cdot A\nabla u dx  d t - s \lambda \iint_Q \xi\Div( A\nabla u) udx  d t\\
			\le& C \lambda^2 \iint_Q \xi \left|  \nabla \eta \cdot A\nabla u \right| ^2dx  d t + Cs^2 \lambda^2 \iint_Q \xi u^2dx  d t\\
			&+C s^{-1} \lambda^{-1} \iint_Q \xi^{-1} \left| \Div( A\nabla u)\right| ^2dx  d t + Cs^3 \lambda^3 \iint_Q \xi^3 u^2dx  d t.
		\end{split}
	\end{equation*}
	Thus, the term $s \lambda \iint_Q \xi A \nabla u \cdot \nabla udx d t$ can be absorbed by the other terms. This implies that:
	\begin{equation}\label{LLU5}
		\begin{split}
			&s^{-1}\iint_{Q} \xi^{-1} (|u_t |^2 + \left|\Div(A\nabla u)\right|^2) dx dt +C s^2 \lambda^2 \int_0^T \int_{\Om\backslash \omega} \xi^{2}|u|^2dx  d t\\
			&+C\iint_Q s^3 \lambda^4 \xi^3\left|A \nabla \eta \cdot \nabla \eta \right|^2|u|^2 dx dt  + C s  \lambda^2 \iint_Q\xi|\nabla u \cdot A \nabla \eta|^2 dx  d t\\
			\leq& C\left\|e^{-s \sigma} f\right\|^2
			+Cs^3 \lambda^3 \int_0^T \int_{\omega} \xi^3|u|^2dx  d t.
		\end{split}
	\end{equation}
	By employing classical arguments, we can then revert back to the original variable $w$ and conclude the result.
	
	\section{A specific example}
	
	
	As stated earlier in this paper, the well-posedness of this example can be easily verified. In the following, we provide some results regarding the improvement of interior regularity and Carleman estimates.
	\subsection{Internal regularity improvement}
	
	\hspace*{\fill}
	
	Having established the existence of solutions to equation \eqref{4.1}, we now focus on demonstrating that the regularity of the solutions to \eqref{adjoint} can be enhanced within the interior of $\Omega$. Specifically, we can achieve improved regularity by differentiating with respect to the time variable $t$, resulting in $u_t \in \mathcal{H}_0^1(\Omega)$.
	\begin{theorem}\label{regularity}
		For all the solution of equation \eqref{adjoint}, one has $y^{\alpha_y}|\partial_{xy} w|^2, y^{\alpha_y}|\partial_{xx} w|^2,\\ x^{\alpha_x}|\partial_{xy} w|^2$, and $ x^{\alpha_x}|\partial_{yy} w|^2 \in L^1_{\rm loc}(\Om)$.
	\end{theorem}
	\begin{proof}
		Let us introduce a function $\psi = \psi(x,y)$ satisfying the following properties:
		\begin{equation*}
			\psi \in C_0^\infty (\Om), \ \psi =1 \ \mbox{in} \ \Om^{2\epsilon}, \ \psi =0 \ \mbox{in} \ \Om_{\epsilon}, \ \mbox{and} \ 0\le \psi \le 1,
		\end{equation*}
		where $\Omega^{2\epsilon} := (2\epsilon, 1-2\epsilon) \times (2\epsilon, 1-2\epsilon)$ and $\Omega_{\epsilon} := \Omega \setminus \Omega^{\epsilon}$.
		Let us define the operators as follows:
		\begin{equation*}
			D_x^h (w):= \frac{w(x+h,y)-w(x,y)}{h}, \ \mbox{and} \ v:= D_x^{-h}(\psi^2 D_x^h (w)).
		\end{equation*}
		Multiplying the equation in \eqref{adjoint} by $v$ and integrating over $\Omega$, we obtain
		\begin{equation*}
			\int_{\Om^{\epsilon}} v w_t dxdy + \int_{\Om^{\epsilon}} v \Div(A\nabla w) dxdy =\int_{\Om^{\epsilon}} v f dxdy.
		\end{equation*}
		Next, let us consider the second term on the left-hand side:
		\begin{equation*}
			\begin{split}
				&\int_{\Om^{\epsilon}} v \Div(A\nabla w) dxdy=\int_{\Om^{\epsilon}} D_x^{-h}(\psi^2 D_x^h (w)) \Div(A\nabla w) dxdy\\
				=& -\int_{\Om^{\epsilon}} \nabla (D_x^{-h}(\psi^2 D_x^h (w))) \cdot A\nabla w dxdy
				= -\int_{\Om^{\epsilon}} D_x^{-h} (\nabla(\psi^2 D_x^h (w))) \cdot A\nabla w dxdy\\
				=&\int_{\Om^{\epsilon}}  (\nabla(\psi^2 D_x^h (w)))  D_x^{h}(A\nabla w) dxdy
				=\int_{\Om^{\epsilon}}  (2\psi \nabla \psi D_x^h (w)+ \psi^2 \nabla (D_x^h (w)) D_x^{h}(A\nabla w) dxdy\\
				=&\int_{\Om^{\epsilon}}  2\psi \nabla \psi D_x^h (w) D_x^{h}(A\nabla w) dxdy + \int_{\Om^{\epsilon}}
				\psi^2 \nabla (D_x^h (w)) D_x^{h}(A\nabla w) dxdy\\
				=&\int_{\Om^{\epsilon}}  2\psi \nabla \psi D_x^h (w) D_x^{h}(A\nabla w) dxdy+\int_{\Om^{\epsilon}}\psi^2 \nabla (D_x^h (w)) D_x^{h}(A) \nabla w dxdy\\
				&+ \int_{\Om^{\epsilon}}\psi^2 \nabla (D_x^h (w)) A^h \nabla D_x^h (w)  dxdy,
			\end{split}
		\end{equation*}
		where
		\begin{equation}
			A^h := \begin{pmatrix} y^{\alpha_y} & 0 \\ 0 &(x+h)^{\alpha_x} \end{pmatrix},
		\end{equation}
		and
		\begin{equation}\label{re1}
			\begin{split}
				&\int_{\Om^{\epsilon}}\psi^2 \nabla (D_x^h (w)) A^h \nabla D_x^h (w)  dxdy\\
				=&\int_{\Om^{\epsilon}}\psi^2 \left( y^{\alpha_y} |\partial_{x} (D_x^h (w))|^2 + (x+h)^{\alpha_x} |\partial_{y} (D_x^h (w))|^2\right)  dxdy\\
				=&\int_{\Om^{\epsilon}} D_x^{-h}(\psi^2 D_x^h (w)) f dxdy - \int_{\Om^{\epsilon}} D_x^{-h}(\psi^2 D_x^h (w)) w_t dxdy\\
				&-\int_{\Om^{\epsilon}}\psi^2 \nabla (D_x^h (w)) D_x^{h}(A) \nabla w dxdy-\int_{\Om^{\epsilon}}  2\psi \nabla \psi D_x^h (w) D_x^{h}(A\nabla w) dxdy.
			\end{split}
		\end{equation}
		We can now proceed to estimate the terms on the right-hand side. Since we can assume $f \in H^1(\Omega)$, we have
		\begin{equation*}
			\begin{split}
				\int_{\Om^{\epsilon}} D_x^{-h}(\psi^2 D_x^h (w)) f dxdy &= -\int_{\Om^{\epsilon}} \psi^2 D_x^h (w) D_x^h (f) dxdy\\
				&\le \int_{\Om^{\epsilon}} \psi^2 D_x^h (w) D_x^h (f) dxdy\le C \left\| f\right\|_{H^1(\Om^{\epsilon})} + C \left\| w\right\|_{\mathcal{H}_0^1(\Om^{\epsilon})}.
			\end{split}
		\end{equation*}
		Similarly, we obtain 
		\begin{equation*}
			\begin{split}
				\int_{\Om^{\epsilon}} D_x^{-h}(\psi^2 D_x^h (w)) w_t dxdy \le C \left\| w\right\|_{\mathcal{H}_0^1(\Om^{\epsilon})} + C \left\| w_t\right\|_{\mathcal{H}_0^1(\Om^{\epsilon})},
			\end{split}
		\end{equation*}
		and
		\begin{equation*}
			\begin{split}
				-\int_{\Om^{\epsilon}}\psi^2 \nabla (D_x^h (w)) D_x^{h}(A) \nabla w dxdy \le& 
				C\int_{\Om^{\epsilon}}\psi^2 \partial_{y} (D_x^h (w))  \partial_{y} w dxdy\\
				\le& \gamma \int_{\Om^{\epsilon}} \psi^2 x^{\alpha_x} |\partial_{y} (D_x^h (w))|^2 dxdy + \frac{C}{\gamma} \left\| w\right\|_{\mathcal{H}_0^1(\Om^{\epsilon})}.
			\end{split}
		\end{equation*}
		Finally, we estimate the last term as follows:
		\begin{equation*}
			\begin{split}
				&-\int_{\Om^{\epsilon}}  2\psi \nabla \psi D_x^h (w) D_x^{h}(A\nabla w) dxdy\\
				=&-\int_{\Om^{\epsilon}}  2\psi \nabla \psi D_x^h (w) D_x^{h}(A)\nabla w dxdy
				-\int_{\Om^{\epsilon}}  2\psi \nabla \psi D_x^h (w) D_x^{h}(\nabla w)A^h dxdy\\
				\le&  \gamma \int_{\Om^{\epsilon}} \psi^2 x^{\alpha_x} |\partial_{y} (D_x^h (w))|^2 dxdy 
				+\gamma \int_{\Om^{\epsilon}} \psi^2 \nabla (D_x^h (w)) A^h \nabla D_x^h (w)  dxdy+ \frac{C}{\gamma} \left\| w\right\|_{\mathcal{H}_0^1(\Om^{\epsilon})}.
			\end{split}
		\end{equation*}
		Obviously, the first and second terms on the right-hand side can be absorbed by equation \eqref{re1}. Thus, we have 
		\begin{equation*}
			\begin{split}
				\int_{\Om^{\epsilon}} \psi^2 \nabla (D_x^h (w)) A\nabla D_x^h (w)  dxdy\le C\int_{\Om^{\epsilon}} \psi^2 \nabla (D_x^h (w)) A^h \nabla D_x^h (w)  dxdy\le C.
			\end{split}
		\end{equation*}
		Similarly, we obtain
		\begin{equation*}
			\begin{split}
				\int_{\Om^{\epsilon}} \psi^2 \nabla (D_y^h (w)) A\nabla D_y^h (w)  dxdy \le C.
			\end{split}
		\end{equation*}
		This concludes the proof.
	\end{proof}
	\begin{remark}
		It is worth noting that, on the boundary, we cannot improve the regularity due to the specific nature of the equation we are considering. Consequently, on the boundary, we do not have the property $(A \nabla u \cdot \nabla \sigma) A \nabla u, (A\nabla u\cdot\nabla u)A\nabla\sigma\in (W^{1,1}(\Omega))^2$ (see the estimate of $I_4$ for the boundary term in the following section). In other words, the functions $(A \nabla u \cdot \nabla \sigma) A \nabla u$ and $(A\nabla u\cdot\nabla u)A\nabla\sigma$ do not have a trace on $\partial\Omega$. This distinction shapes our research approach significantly differently from \cite{CA5,CA6}. It is also the reason why we choose the control domain as $\omega_0$ in the subsequent subsection.
	\end{remark}
	
	
	\subsection{Carleman esitimates}
	
	\hspace*{\fill}
	
	The control system \eqref{4.1} we are studying is a specific case of the previously mentioned \eqref{1.1}. However, the weight function we have chosen in the following analysis is a special weight function that allows for more convenient calculations. Therefore, we need to provide a Carleman estimate that differs slightly from the previous one.
	
	Let us now establish a Carleman estimate. We know that the adjoint equation of \eqref{4.1} is given by
	\begin{equation}\label{4.3}
		\begin{cases}
			\partial_{t}w + \Div(A\nabla w)=f, & \mbox{in} \ Q,  \\
			w(x,y,t)=0, \ \mbox{or} \ A\nabla w \cdot \nu =0, & \mbox{on} \ \Sigma,  \\
			w(x,  y,  T)= w_T,  & \mbox{in} \ \Omega.  
		\end{cases}
	\end{equation} 
	Fix $\delta>0$. Denote $(\delta,1)\times(\delta,1)$ by $\Omega^\delta$, $\Omega\backslash \Omega^\delta$ by $\Omega_\delta$, and $\Sigma_\delta:= \partial \Omega^\delta \times(0,T)$. Let
	\begin{equation*}
		\eta(x,y) :=
		\begin{cases}
			\frac{(x-\delta)^2 (y-\delta)^2 (x-1)^2(y-1)^2}{2}, & x \in \Omega^\delta,\\
			0, & x \in \Omega_\delta.
		\end{cases}
	\end{equation*}
	It is evident that $\eta \in C^2(\overline{\Omega})$, $\eta > 0$ in $\Omega^\delta$, and $\eta = 0$ on $\partial\Omega^\delta$. By utilizing the classical arguments in \cite{CA6}, we can transform $(\frac{1+\delta}{2},\frac{1+\delta}{2})$ to $\omega_0$. As a result, we obtain
	\begin{equation*}
		\left| \nabla \eta\right| \ge C > 0, \ \mbox{in} \ \overline{\Omega \backslash \omega_0},
	\end{equation*}
	where $\omega_0$ is a nonempty open set satisfying $\Omega_\delta \subset \omega_0 \subset \Omega$, and $\Gamma \subset \partial\omega_0$. We define
	\begin{equation*}
		\begin{split}
			& \theta(t):=[t(T-t)]^{-4}, \quad \xi(x, t):=\theta(t) e^{ \lambda(8|\eta|_\infty+\eta (x))}, \quad \sigma(x, t):=\theta(t) e^{10 \lambda|\eta|_\infty}-\xi(x, t).
		\end{split}
	\end{equation*}
	
	In the following discussion, $C>0$ represents a generic constant that depends solely on $T$ and $\alpha_x, \alpha_y$. We assume that $w$ is a sufficiently regular solution of \eqref{4.3}. Moreover, we consider $w \in H^1(0,T; \mathcal{H}_0^1(\Omega))$.
	
	For $s>s_0>0$, we introduce
	\begin{equation*}
		u=e^{-s\sigma} w.  
	\end{equation*}
	Then, the following properties hold:
	\begin{itemize}
		\item [($i$)] $u=\frac{\partial u}{\partial x_i}=0$ at $t=0$ and $t=T$;
		\item [($ii$)] $u=0$ or $A\nabla u \cdot \nu =0 $ on $\Sigma$;
		\item [($iii$)] If $P_1 u:=u_t+s \mathrm{div}(u A \nabla \sigma)+s \nabla \sigma A \nabla u$ and $P_2 u:=\mathrm{div}(A \nabla u)+s^2 u \nabla \sigma A \nabla \sigma+s \sigma_t u$, then $P_1 u+P_2 u=e^{-s \sigma} f$.
	\end{itemize}
	From item ($iii$), it follows that
	\begin{equation}\label{P}
		\left\|P_1 u\right\|^2+\left\|P_2 u\right\|^2+2\left(P_1 u, P_2 u\right)=\left\|e^{-s \sigma} f\right\|^2.
	\end{equation}
	
	We decompose $\left(P_1 u, P_2 u\right)$ into four parts: $I_1$, $I_2$, $I_3$, and $I_4$, defined as follows:
	\begin{equation*}
		\begin{split}
			& I_1:=\left(\Div(A \nabla u)+s^2 u \nabla \sigma \cdot A \nabla \sigma+s \sigma_t u, u_t\right), \\
			& I_2:=s^2\left(\sigma_t u, \Div(u A \nabla \sigma)+\nabla \sigma A \nabla u\right), \\
			&I_3:=s^3\left( u \nabla \sigma \cdot A \nabla \sigma, \Div(u A \nabla \sigma)+\nabla \sigma \cdot A \nabla u\right),\\
			& I_4:=s(\Div(A \nabla u), \Div(u A \nabla \sigma)+\nabla \sigma \cdot A \nabla u).
		\end{split}
	\end{equation*}
	Before proceeding with the calculations, we introduce the following theorem, which will be useful.
	\begin{theorem}\label{DIV}
		For all $v\in \mathcal{H}_0^1(\Omega)$, it holds that
		\begin{equation*}
			\int_\Om v \Div(A\nabla u) dx dy= -\int_\Om \nabla v \cdot A\nabla u dx dy. 
		\end{equation*}
		\begin{proof}
			Let $v_n \in C_0^\infty(\Omega)$ be a sequence converging to $v$ in $H^1(\Omega)$, which is possible due to the density of $C_0^\infty(\Omega)$ in $\mathcal{H}_0^1(\Omega)$. Then, we have
			\begin{equation*}
				\begin{split}
					\int_\Om v \Div(A\nabla u) dx dy
					&= \lim\limits_{n\to \infty} \int_\Om v_n \Div(A\nabla u) dx dy=\lim\limits_{n\to \infty}\left(  v_n,\Div(A\nabla u)\right)\\  &=-\lim\limits_{n\to \infty}\left( \nabla v_n, A\nabla u\right).
				\end{split}
			\end{equation*}
			Using integration by parts, we expand the above expression as follows:
			\begin{equation*}
				\begin{split}
					-\lim\limits_{n\to \infty}\left( \nabla v_n, A\nabla u\right)=&\lim\limits_{n\to \infty}-\left( \frac{\partial v_n}{\partial x}, y^{\alpha_y}\partial_x u\right)-\left( \frac{\partial v_n}{\partial y}, x^{\alpha_x}\partial_y u\right)\\
					=&\lim\limits_{n\to \infty}-\left(y^{\frac{\alpha_y}{2}} \frac{\partial v_n}{\partial x}, y^{\frac{\alpha_y}{2}}\partial_x u\right)-\left(x^{\frac{\alpha_x}{2}} \frac{\partial v_n}{\partial y}, x^{\frac{\alpha_x}{2}}\partial_y u\right)\\
					=&\lim\limits_{n\to \infty}-\int_{\Om} y^{\alpha_y}\frac{\partial v_n}{\partial x}\frac{\partial u}{\partial x} dx dy - \int_{\Om} x^{\alpha_x}\frac{\partial v_n}{\partial y}\frac{\partial u}{\partial y} dx dy\\
					=&\lim\limits_{n\to \infty}-\int_{\Om} \nabla v_n A \nabla u dx dy =-\int_{\Om} \nabla v A \nabla u dx dy.
				\end{split}
			\end{equation*}
			Thus, the result is established.
		\end{proof}
	\end{theorem}
	From Theorem \ref{DIV} and item (i), we obtain the following expressions:
	\begin{equation}\label{I1}
		\begin{split}
			I_1=&\iint_Q u_t \Div (A\nabla u) +s^2 u \nabla \sigma \cdot A\nabla \sigma u_t + s\sigma_t u u_t dx dy dt\\
			=&\int_{\Om} s^2  \nabla \sigma A\nabla \sigma\cdot \frac{1}{2}u^2  \bigg|_0^T dx dy  + \int_{\Om} s\sigma_t\cdot \frac{1}{2}u^2 \bigg|_0^T dx dy + \iint_\Sigma u_t A\nabla u \cdot \nu ds dt\\
			&-\iint_{Q} A\nabla u \cdot \nabla u_t dx dy dt -\frac{1}{2}\iint_{Q}(s\sigma_t + s^2 \nabla \sigma \cdot A\nabla \sigma)_t u^2 dx dy dt\\
			=& - \frac{1}{2}\int_{\Om}  A\nabla u \cdot \nabla u \bigg|_0^T dx dy  -\frac{1}{2}\iint_{Q}(s\sigma_t + s^2 \nabla \sigma \cdot A\nabla \sigma)_t u^2 dx dy dt\\
			=& -\frac{1}{2}\iint_{Q}(s\sigma_t + s^2 \nabla \sigma \cdot A\nabla \sigma)_t u^2 dx dy dt,
		\end{split}
	\end{equation}
	and
	\begin{equation}\label{I2}
		\begin{split}
			I_2  =& s^2  \iint_Q \sigma_t u (\Div(u A \nabla \sigma) + A\nabla u \cdot \nabla \sigma) dx dy dt
			= s^2  \iint_Q \sigma_t u (\Div( A \nabla \sigma) + 2 A\nabla u \cdot \nabla \sigma) dx dy dt\\
			=& s^2  \iint_Q \sigma_t u \Div( A \nabla \sigma) + \Div( A \nabla \sigma) \sigma_t u^2  dx dy dt\\
			=& s^2 \iint_\Sigma \sigma_t u^2 A \nabla \sigma  \cdot \nu dsdt - s^2  \iint_Q \Div(\sigma_t A \nabla \sigma)u^2 dx dydt\\
			&+ s^2  \iint_Q \Div( A \nabla \sigma) \sigma_t u^2 dx dy dt\\
			=& - s^2  \iint_Q \Div( A \nabla \sigma) \sigma_t u^2 dx dy dt -\iint_Q  A \nabla \sigma \cdot\nabla \sigma_t u^2    dx dy dt+ s^2\iint_Q \Div( A \nabla \sigma) \sigma_t u^2 dx dydt\\
			=& -s^2\iint_Q  A \nabla \sigma \cdot\nabla \sigma_t u^2    dx dy dt.
		\end{split}
	\end{equation}
	Similarly, we can deduce the following expressions:
	\begin{equation}\label{I3}
		\begin{split}
			I_3  =&s^3 \iint_Q u A \nabla \sigma \cdot \nabla \sigma (\Div (u A \nabla \sigma) + A\nabla u \cdot \nabla \sigma ) dx dy dt\\
			=&s^3 \iint_\Sigma u^2 (A \nabla \sigma \cdot \nabla \sigma) A \nabla \sigma \cdot \nu  ds dt -s^3 \iint_Q u A \nabla \sigma \cdot \nabla (u A \nabla \sigma \cdot \nabla \sigma  ) dx dy dt\\
			&+ s^3 \iint_Q  (A \nabla \sigma \cdot \nabla u) ( A \nabla \sigma \cdot \nabla \sigma  ) u dx dy dt\\
			=&-s^3\iint_Q  A\nabla \sigma \cdot \nabla ( A\nabla \sigma \cdot\nabla \sigma) u^2 dx dy dt,
		\end{split}
	\end{equation}
	and
	\begin{equation*}
		\begin{split}
			I_4= & s \iint_Q \Div(A \nabla u)\left(  \Div(u A \nabla \sigma)+ A \nabla u \cdot \nabla \sigma\right)  dx dy dt \\%1
			= & s \iint_Q \Div(A \nabla u)\left( A \nabla u \cdot \nabla \sigma+ u \Div(A \nabla \sigma)+ A \nabla u \cdot \nabla \sigma\right)  dx dy dt \\%2
			= & s \iint_Q \Div(A \nabla u) u \Div(A \nabla \sigma)dx dy dt + 2s \iint_Q \Div(A \nabla u) A \nabla u \cdot \nabla \sigma   dx dy dt \\%3
			= & s \iint_\Sigma u \Div(A \nabla \sigma )A \nabla u \cdot \nu dsdt + 2s \iint_{\Sigma_\delta}  (A \nabla u \cdot \nabla \sigma) A \nabla u \cdot \nu_\delta ds dt \\%4
			&-s \iint_Q A \nabla u \cdot  \nabla\left( u \Div(A \nabla \sigma)\right) dx dy dt -2s \iint_{\Omega^\delta} A \nabla u\cdot \nabla \left( A \nabla u \cdot \nabla \sigma \right)  dx dy dt \\%5
			= &
			-s \iint_Q A \nabla u \cdot  \nabla u \Div(A \nabla \sigma)dx dy dt-s \iint_Q u A \nabla u\cdot \nabla (\Div(A \nabla \sigma)) dx dy dt\\%6
			&-2s \iint_{Q} A \nabla u\cdot \nabla(A \nabla u \cdot \nabla \sigma)   dx dy dt. 
		\end{split}
	\end{equation*}
	We observe that $2s \iint_{\Sigma_\delta} (A \nabla u \cdot \nabla \sigma) A \nabla u \cdot \nu_\delta  ds  dt = 0$ since $\nabla \sigma = 0$ on $\Sigma_\delta$. Moreover,
	\begin{equation*}
		\begin{split}
			&-2s \iint_{Q} A \nabla u \cdot \nabla (A\nabla u \cdot \nabla \sigma) dx dydt\\
			=&-2s \sum_{i=1}^{2} \iint_{Q} (A\nabla \sigma)_i A \nabla u \cdot \frac{\partial}{\partial x_i}(\nabla u) + (\nabla u)_i A \nabla u \cdot \nabla (A \nabla \sigma)_i dx dydt\\
			=&-2s \iint_{Q} A \nabla u \left[ y^{\alpha_y} \frac{\partial \sigma}{\partial x}\nabla \frac{\partial u}{\partial x}  + x^{\alpha_x} \frac{\partial \sigma}{\partial y}\nabla \frac{\partial u}{\partial y} + \frac{\partial u}{\partial x} \nabla (y^{\alpha_y} \frac{\partial \sigma}{\partial x}) + \frac{\partial u}{\partial y} \nabla (x^{\alpha_x} \frac{\partial \sigma}{\partial y}) \right] dx dydt.
		\end{split}
	\end{equation*}
	Here, we have
	\begin{equation*}
		\begin{split}
			&-2s \iint_{Q} A \nabla u \left[ y^{\alpha_y} \frac{\partial \sigma}{\partial x}\nabla \frac{\partial u}{\partial x}  + x^{\alpha_x} \frac{\partial \sigma}{\partial y}\nabla \frac{\partial u}{\partial y}\right] dx dydt\\%1
			=& -2s \iint_Q y^{\alpha_y} \frac{\partial \sigma}{\partial x}  A \nabla u \cdot \frac{\partial}{\partial x}(\nabla u) dx dydt - 2s \iint_Q x^{\alpha_x} \frac{\partial \sigma}{\partial y}  A \nabla u \cdot \frac{\partial}{\partial y}(\nabla u) dx dydt\\%2
			=& -s \iint_{Q} y^{\alpha_y} \frac{\partial \sigma}{\partial x} \left[ \frac{\partial}{\partial x} (A \nabla u \cdot \nabla u) - \frac{\partial A}{\partial x} \nabla u \cdot \nabla u \right] dx dydt\\
			&-s \iint_{Q} x^{\alpha_x} \frac{\partial \sigma}{\partial y} \left[ \frac{\partial}{\partial y} (A \nabla u \cdot \nabla u) - \frac{\partial A}{\partial y} \nabla u \cdot \nabla u \right] dx dydt\\%3
			=&-s \iint_{\Sigma_\delta} (A \nabla u \cdot \nabla u)(y^{\alpha_y} \frac{\partial \sigma}{\partial x}) \cdot \nu_x dsdt + s\iint_{Q} A \nabla u \cdot \nabla u \frac{\partial}{\partial x} (y^{\alpha_y} \frac{\partial \sigma}{\partial x}) dx dydt\\
			& +s \iint_{Q} y^{\alpha_y} \frac{\partial \sigma}{\partial x} \frac{\partial A}{\partial x} \nabla u \cdot \nabla u dx dydt
			-s \iint_{\Sigma_\delta} (A \nabla u \cdot \nabla u)(x^{\alpha_x} \frac{\partial \sigma}{\partial y}) \cdot \nu_y dsdt\\
			& + s\iint_{Q} A \nabla u \cdot \nabla u \frac{\partial}{\partial y} (x^{\alpha_x} \frac{\partial \sigma}{\partial y}) dx dydt +s \iint_{Q} x^{\alpha_x} \frac{\partial \sigma}{\partial y} \frac{\partial A}{\partial y} \nabla u \cdot \nabla u dx dydt\\%5
			=&  s \iint_{Q} A\nabla u \cdot \nabla u \Div(A\nabla \sigma) dx dydt
			+ s\iint_{Q} y^{\alpha_y} \frac{\partial \sigma}{\partial x} (\alpha_x x^{\alpha_x-1}\left| \frac{\partial u}{\partial y}\right| ^2) dx dydt\\
			&+ s\iint_{Q} x^{\alpha_x} \frac{\partial \sigma}{\partial y} (\alpha_y y^{\alpha_y-1}\left| \frac{\partial u}{\partial x}\right| ^2) dx dydt.
		\end{split}
	\end{equation*}
	Hence, we obtain the following expression:
	\begin{equation}\label{I4}
		\begin{split}
			I_4 = & -s \iint_Q u A \nabla u\cdot \nabla (\Div(A \nabla \sigma))dx dy dt
			\\
			&-2s \iint_{Q} A\nabla u \cdot \left[ \frac{\partial u}{\partial x} \nabla (y^{\alpha_y} \frac{\partial \sigma}{\partial x}) + \frac{\partial u}{\partial y} \nabla (x^{\alpha_x} \frac{\partial \sigma}{\partial y}) \right]  dx dy dt\\
			&+ s\iint_{Q} y^{\alpha_y} \frac{\partial \sigma}{\partial x} (\alpha_x x^{\alpha_x-1}\left| \frac{\partial u}{\partial y}\right| ^2) dx dydt
			+ s\iint_{Q} x^{\alpha_x} \frac{\partial \sigma}{\partial y} (\alpha_y y^{\alpha_y-1}\left| \frac{\partial u}{\partial x}\right| ^2) dx dydt. 
		\end{split}
	\end{equation}
	From equations \eqref{I1} to \eqref{I4}, we conclude that
	\begin{equation}\label{Pu}
		\begin{aligned}
			&\left(P_1 u, P_2 u\right)\\
			= & -s^3 \iint_Q A \nabla \sigma\cdot \nabla(\nabla \sigma A \nabla \sigma)|u|^2 dx dy dt\\
			&-2s \iint_{Q} A\nabla u \cdot \left[ \frac{\partial u}{\partial x} \nabla (y^{\alpha_y} \frac{\partial \sigma}{\partial x}) + \frac{\partial u}{\partial y} \nabla (x^{\alpha_x} \frac{\partial \sigma}{\partial y}) \right]  dx dy dt \\
			& -2s^2 \iint_Q \nabla \sigma A \nabla \sigma_t|u|^2 dx dy dt+ s\iint_{Q} y^{\alpha_y} \frac{\partial \sigma}{\partial x} \left( \alpha_x x^{\alpha_x-1}\left| \frac{\partial u}{\partial y}\right| ^2\right)  dx dydt\\
			&+ s\iint_{Q} x^{\alpha_x} \frac{\partial \sigma}{\partial y} \left( \alpha_y y^{\alpha_y-1}\left| \frac{\partial u}{\partial x}\right| ^2\right)  dx dydt
			-s \iint_Q u A \nabla u \nabla(\Div(A \nabla \sigma)) dx dy dt\\
			&-\frac{s}{2} \iint_Q \sigma_{t t}|u|^2 dx dy dt.
			%+2 s\iint_{\Sigma}(\nabla u A \nabla \sigma) A \nabla u \cdot \nu d s d t -s\iint_{\Sigma}(\nabla u A \nabla u) A \nabla \sigma \cdot \nu d s d t .
		\end{aligned}
	\end{equation}
	Let us denote the seven integrals on the right-hand side of equation \eqref{Pu} as $J_1, \cdots, J_6$. Our goal now is to estimate each of these integrals. By utilizing the definitions of $\sigma$ and $\xi$, as well as the properties of $\eta$, we can derive the following relationships: $\nabla \sigma = - \lambda \xi \nabla \eta$, $\nabla \xi = \lambda \xi \nabla \eta$, $A \nabla \sigma \cdot \nabla \sigma = \lambda^2 \xi^2 A\nabla \eta \cdot \nabla \eta$, and $\nabla( A \nabla \sigma \cdot \nabla \sigma) = \lambda^2 \xi^2 \nabla (A\nabla \eta \cdot \nabla \eta) + 2\lambda^3 \xi^2 \nabla \eta \left| A \nabla \eta \cdot \nabla \eta \right|$. Thus, we have
	\begin{equation}
		\begin{aligned}
			J_1
			&=-s^3 \iint_Q A \nabla \sigma \nabla(\nabla \sigma A \nabla \sigma)|u|^2 dx dy dt\\
			&=-s^3 \iint_Q  \left( -\lambda \xi A \nabla \eta \right) \left( \lambda^2\xi^2 \nabla(\nabla \eta A \nabla \eta) + 2\lambda^3\xi^2 \nabla \eta \left| A \nabla \eta \cdot \nabla \eta \right| ^2\right) |u|^2 dx dy dt\\
			&=2s^3 \lambda^4 \iint_Q \xi^3\left|A \nabla \eta \cdot \nabla \eta \right|^2|u|^2 dx dy dt
			+ s^3 \lambda^3 \iint_Q \xi^3 A\nabla \eta \cdot \nabla(A\nabla \eta \cdot \nabla\eta)|u|^2 dx dy dt.
		\end{aligned}
	\end{equation}
	Since $A\nabla \eta \cdot\nabla(A\nabla \eta \cdot \nabla\eta)$ is bounded in $\overline{\Omega}$, we can deduce that
	$$
	s^3 \lambda^3 \int_{0}^{T}\int_{\omega_0} \xi^3 A\nabla \eta \cdot \nabla(A\nabla \eta \cdot \nabla\eta)|u|^2 dx dy dt \ge -C s^3 \lambda^3 \int_{0}^{T}\int_{\omega_0} \xi^3 |u|^2 dx dy dt.
	$$
	Considering the inequality $|A \nabla \eta \cdot \nabla \eta|\ge C>0$ in $\overline{\Omega} \backslash \omega_0$, we can deduce the following:
	$$
	\begin{aligned}
		&s^3 \lambda^3 \int_{0}^{T}\int_{\Om\backslash \omega_0} \xi^3 A\nabla \eta \cdot \nabla(A\nabla \eta \cdot \nabla\eta)|u|^2 dx dy dt\\
		\ge& -C s^3 \lambda^3 \int_{0}^{T}\int_{\Om\backslash \omega_0} \xi^3 |u|^2 dx dy dt\\
		\ge&
		-C s^3 \lambda^3 \int_{0}^{T}\int_{\Om\backslash \omega_0} \xi^3 \left|A \nabla \eta \cdot \nabla \eta \right|^2 |u|^2 dx dy dt\\
		\ge&
		-C s^3 \lambda^3 \iint_{Q} \xi^3 \left|A \nabla \eta \cdot \nabla \eta \right|^2 |u|^2 dx dy dt.
	\end{aligned}
	$$
	Thus, we can conclude that
	\begin{equation}\label{J1}
		\begin{aligned}
			J_1
			\geq C s^3 \lambda^4 \iint_Q \xi^3\left|A \nabla \eta \cdot \nabla \eta \right|^2|u|^2 dx dy dt-C s^3 \lambda^3  \int_0^T \int_{\omega_0} \xi^3|u|^2 dx dy dt,
		\end{aligned}
	\end{equation}
	and
	\begin{equation}\label{J2}
		\begin{aligned}
			J_2 =&-2s  \iint_{Q} A\nabla u \cdot \left[ \frac{\partial u}{\partial x} \nabla (y^{\alpha_y} \frac{\partial \sigma}{\partial x}) + \frac{\partial u}{\partial y} \nabla (x^{\alpha_x} \frac{\partial \sigma}{\partial y}) \right]  dx dy dt\\
			=&-2s \iint_{Q}( D(A\nabla \sigma) A\nabla u) \cdot \nabla u dx dy dt\\
			=&2s \iint_{Q}( \lambda^2 \xi (A\nabla \eta \cdot \nabla u) A\nabla \eta + \lambda \xi D(A\nabla \eta) A\nabla u ) \cdot \nabla u dx dy dt\\
			=&2s \lambda^2\iint_{Q}  \xi \left|  A\nabla \eta \cdot \nabla u \right| ^2 dx dy dt  + 2s\lambda \iint_{Q}\xi D(A\nabla \eta) A\nabla u  \cdot \nabla u dx dy dt\\
			\ge& Cs \lambda^2\iint_{Q}  \xi \left|  A\nabla \eta \cdot \nabla u \right| ^2 dx dy dt  - C s\lambda \iint_{Q}\xi A\nabla u  \cdot \nabla u dx dy dt.
		\end{aligned}
	\end{equation}
	From the definition of $\xi$, it is evident that $\xi \xi_t \le \xi^3$. Therefore, we can write
	\begin{equation*}
		\begin{aligned}
			J_3 &=-2 s^2 \iint_{Q} A\nabla \sigma \cdot \nabla \sigma_t u^2 dx dydt
			=-2 s^2 \lambda^2 \iint_{Q} \xi \xi_t \left|A \nabla \eta \cdot \nabla \eta \right|  u^2 dx dydt\\
			&\ge-2 s^2 \lambda^2 \iint_{Q} \xi^3 \left|A \nabla \eta \cdot \nabla \eta \right|  u^2 dx dydt.
		\end{aligned}
	\end{equation*}
	Likewise, we obtain
	\begin{equation}\label{J3}
		\begin{aligned}
			J_3
			\ge &-C s^2 \lambda^2 \iint_{Q} \xi^3 \left|A \nabla \eta \cdot \nabla \eta \right|^2  u^2 dx dydt -C s^2 \lambda^2 \int_{0}^{T}\int_{\omega_0} \xi^3 \left|A \nabla \eta \cdot \nabla \eta \right|^2  u^2 dx dydt,
		\end{aligned}
	\end{equation}
	and
	\begin{equation}\label{J4}
		\begin{aligned}
			J_4 =&s\iint_{Q} y^{\alpha_y} \frac{\partial \sigma}{\partial x} \left( \alpha_x x^{\alpha_x-1}\left| \frac{\partial u}{\partial y}\right| ^2\right)  dx dydt
			+ s\iint_{Q} x^{\alpha_x} \frac{\partial \sigma}{\partial y} \left( \alpha_y y^{\alpha_y-1}\left| \frac{\partial u}{\partial x}\right| ^2\right)  dx dydt\\
			\ge  & -Cs \lambda \iint_Q \alpha_x \xi y^{\alpha_y+2}  x^{\alpha_x } \left| \frac{\partial u}{\partial y}\right|^2  dx dy dt 
			-Cs \lambda \iint_Q \alpha_y \xi x^{\alpha_x +2}  y^{\alpha_y } \left| \frac{\partial u}{\partial x}\right|^2  dx dy dt \\
			\ge & -C s \lambda \iint_Q  \xi A \nabla u \cdot \nabla u dx dy dt.
		\end{aligned}
	\end{equation}
	By using the definitions of $\sigma$ and $\xi$, we can derive the following expression:
	\begin{equation}\label{J5}
		\begin{split}
			J_5  =&-s \iint_Q u A \nabla u \cdot \nabla(\Div(A \nabla \sigma)) dx dy dt\\
			=&s \lambda^3 \iint_Q \xi u A \nabla u \cdot \nabla \eta \left( A \nabla \eta \cdot \nabla \eta\right)  dx dy dt
			+ s \lambda^2 \iint_Q \xi u A \nabla u \cdot \nabla\left(A \nabla \eta \cdot \nabla \eta\right) dx dy dt \\
			& +s \lambda^2 \iint_Q \xi u A \nabla u \cdot \nabla \eta \Div(A \nabla \eta)   dx dy dt+s \lambda \iint_Q \xi u A \nabla u \cdot \nabla(\Div(A \nabla \eta)) dx dy dt.
		\end{split}
	\end{equation}
	Let us denote the seven integrals on the right-hand side of \eqref{J5} as $J_{51}, \cdots, J_{54}$. Then we have
	\begin{equation}\label{J51}
		\begin{split}
			J_{51}=&s \lambda^3 \iint_Q \xi u A \nabla u \cdot \nabla \eta \left( A \nabla \eta \cdot \nabla \eta\right)  dx dy dt\\
			\geq&
			-Cs^2 \lambda^4 \iint_Q \xi\left|A \nabla \eta \cdot \nabla \eta\right|^2|u|^2 dx dy dt
			-C\lambda^2 \iint_Q \xi| A\nabla u \cdot\nabla \eta|^2 dx dy dt,
		\end{split}
	\end{equation}
	and
	\begin{equation}\label{J52}
		\begin{split}
			J_{52}=&s \lambda^2 \iint_Q \xi u A \nabla u \cdot \nabla\left(A \nabla \eta \cdot \nabla \eta\right) dx dy dt\\
			\ge&-Cs^2 \lambda^3\iint_{Q} \xi \left| A \nabla \eta \cdot \nabla \eta \right| u^2 dx dy dt -Cs^2 \lambda^3\int_0^T \int_{\omega_0} \xi \left| A \nabla u \cdot \nabla u \right|  u^2 dx dy dt\\
			&- C\lambda \iint_Q  \xi A \nabla u \cdot \nabla u dx dy dt.
		\end{split}
	\end{equation}
	Then we have
	\begin{equation}\label{J53}
		\begin{split}
			J_{53}=&s \lambda^2 \iint_Q \xi u A \nabla u \cdot \nabla \eta \Div(A \nabla \eta) dx dy dt\\
			\ge& -Cs^2 \lambda^3\iint_Q \xi \left| A \nabla \eta \cdot \nabla \eta \right|^2  u^2  dx dy dt -Cs^2 \lambda^3\int_0^T \int_{\omega_0} \xi u^2 dx dy dt\\
			&- C\lambda \iint_Q \xi A \nabla u \cdot \nabla u dx dy dt,
		\end{split}
	\end{equation}
	and
	\begin{equation}\label{J54}
		\begin{split}
			J_{54}=&s \lambda \iint_Q \xi u A \nabla u \nabla(\Div(A \nabla \eta))dx dy d t\\
			\ge&
			-C s^2 \lambda \iint_Q \xi \left|A \nabla \eta \cdot \nabla \eta\right|^2 u^2dx dy d t-C s^2 \lambda \int_0^T \int_{\omega_0}\xi  u^2dx dy d t\\
			&-C \lambda \iint_Q \xi A \nabla u \cdot \nabla udx dy d t.
		\end{split}
	\end{equation}
	Combining \eqref{J5} to \eqref{J54}, we obtain
	\begin{equation}\label{J55}
		\begin{split}
			J_5 \geq&-Cs^2 \lambda^4 \iint_Q \xi\left|A \nabla \eta \cdot \nabla \eta\right|^2 |u|^2dx dy d t -C\lambda^2\iint_{Q} \xi \left| A\nabla u\cdot \nabla \eta \right| ^2 dx dy dt\\
			&-C s^2 \lambda^3 \int_0^T \int_{\omega_0} \xi|u|^2dx dy d t- C\lambda \iint_Q  \xi A \nabla u \cdot \nabla u dx dy d t.
		\end{split}
	\end{equation}
	Finally, as can be observed from the definitions of $\xi$ and $\sigma$, we have $\sigma_{tt} \le \xi^{3/2}$. Hence,
	\begin{equation}\label{J6}
		J_6 =-\frac{s}{2} \iint_Q \sigma_{t t}|u|^2dx dy d t \geq-C s \iint_Q \xi^{3 / 2}|u|^2dx dy d t.
	\end{equation}
	From \eqref{J1}-\eqref{J4} and \eqref{J55}-\eqref{J6}, we deduce that
	\begin{equation}\label{PU}
		\begin{split}
			\left(P_1 u, P_2 u\right)
			\geq& C\iint_Q s^3 \lambda^4 \xi^3 |\nabla \eta \cdot A \nabla \eta|^2 |u|^2 dx dydt  + C s  \lambda^2 \iint_Q\xi|\nabla u \cdot A \nabla \eta|^2 dx dy d t\\
			&-Cs^3 \lambda^3 \int_0^T \int_{\omega_0} \xi^3|u|^2dx dy d t
			- Cs \lambda \iint_Q \xi A \nabla u \cdot \nabla udx dy d t\\
			&- Cs \iint_Q \xi^{3 / 2}|u|^2dx dy d t.
		\end{split}
	\end{equation}
	It is evident that the term $s \iint_Q \xi^{3 / 2}|u|^2dx dy dt$ can be absorbed by other terms. By combining \eqref{P} and \eqref{PU}, we conclude that
	\begin{equation}\label{LU2}
		\begin{split}
			&\left\|P_1 u\right\|^2+\left\|P_2 u\right\|^2+C\iint_Q s^3 \lambda^4 \xi^3\left|A \nabla \eta \cdot \nabla \eta \right|^2|u|^2 dx dydt  + C s  \lambda^2 \iint_Q\xi|\nabla u \cdot A \nabla \eta|^2 dx dy d t\\
			\leq& C\left\|e^{-s \sigma} f\right\|^2
			+Cs \lambda \iint_Q \xi A \nabla u \cdot \nabla udx dy d t+Cs^3 \lambda^3 \int_0^T \int_{\omega_0} \xi^3|u|^2dx dy d t.
		\end{split}
	\end{equation}
	Moreover, we can deduce that
	\begin{equation*}
		s^2 \lambda^2 \iint_Q \xi^{2}|u|^2dx dy d t=s^2 \lambda^2 \int_0^T \int_{\Om\backslash \omega_0} \xi^{2}|u|^2dx dy d t + s^2 \lambda^2 \int_0^T \int_{\omega_0} \xi^{2}|u|^2dx dy d t.
	\end{equation*}
	Clearly, the second term on the right can be absorbed by the other terms in \eqref{LU2}. Let us consider the first term on the right, we have
	\begin{equation*}
		\begin{split}
			s^2 \lambda^2 \int_0^T \int_{\Om\backslash \omega_0} \xi^{2}|u|^2dx dy d t
			&\le Cs^2 \lambda^2 \int_0^T \int_{\Om\backslash \omega_0} \xi^{2}\left|A \nabla \eta \cdot \nabla \eta \right|^2|u|^2dx dy d t\\
			&\le Cs^2 \lambda^2 \iint_Q \xi^{2}\left|A \nabla \eta \cdot \nabla \eta \right|^2|u|^2dx dy d t.
		\end{split}
	\end{equation*}
	Then we obtain
	\begin{equation}\label{LU3}
		\begin{split}
			&\left\|P_1 u\right\|^2+\left\|P_2 u\right\|^2+C s^2 \lambda^2 \int_0^T \int_{\Om\backslash \omega_0} \xi^{2}|u|^2dx dy d t\\
			&+  C\iint_Q s^3 \lambda^4 \xi^3\left|A \nabla \eta \cdot \nabla \eta \right|^2|u|^2 dx dydt  + C s  \lambda^2 \iint_Q\xi|\nabla u \cdot A \nabla \eta|^2 dx dy d t\\
			\leq& C\left\|e^{-s \sigma} f\right\|^2
			+Cs \lambda \iint_Q \xi A \nabla u \cdot \nabla udx dy d t+Cs^3 \lambda^3 \int_0^T \int_{\omega_0} \xi^3|u|^2dx dy d t.
		\end{split}
	\end{equation}
	Now, using the definitions of $P_1 u$ and $P_2 u$, we observe that
	\begin{equation*}
		\begin{split}
			&s^{-1} \iint_Q \xi^{-1}\left|u_t\right|^2dx dy d t\\
			=&s^{-1} \iint_Q \xi^{-1}(P_1 u -s u \Div(A\nabla \sigma)- 2s \nabla u \cdot A\nabla \sigma)^2dx dy d t\\
			\le& Cs^{-1}\left\|P_1 u\right\|^2+s   \iint_Q \xi^{-1}|u|^2|\Div(A \nabla \sigma)|^2dx dy d t +C s \iint_Q \xi^{-1}|\nabla u A \nabla \sigma|^2dx dy d t \\
			\leq& Cs^{-1}\left\|P_1 u\right\|^2+C s \lambda^4 \iint_Q \xi\left| A \nabla \eta \cdot \nabla \eta \right|^2 |u|^2dx dy d t \\
			&+C s \lambda^2  \iint_Q \xi|u|^2dx dy d t+C s \lambda^2 \iint_Q \xi|\nabla u \cdot A \nabla \eta|^2dx dy d t,
		\end{split}
	\end{equation*}
	and
	\begin{equation*}
		\begin{split}
			&s^{-1}   \iint_Q \xi^{-1}\left|\Div(A\nabla u)\right|^2dx dy d t\\
			=& s^{-1}   \iint_Q \xi^{-1}(P_2 u -s^2 u \nabla \sigma \cdot A \nabla \sigma - s \sigma_t u)^2dx dy d t\\
			\leq& C s^{-1} \left\|P_2 u\right\|^2+s^3   \iint_Q \xi^{-1}|u|^2\left|\nabla \sigma \cdot A \nabla \sigma\right|^2dx dy d t +C s   \iint_Q \xi^{-1}\left|\sigma_t\right|^2|u|^2dx dy d t \\
			\leq& C s^{-1} \left\|P_2 u\right\|^2+C s^3 \lambda^4   \iint_Q \xi^3|\nabla \eta \cdot A \nabla \eta|^2|u|^2dx dy d t +C s   \iint_Q \xi^{2}|u|^2dx dy d t.
		\end{split}
	\end{equation*}
	From \eqref{LU3}, we obtain
	\begin{equation}\label{LU4}
		\begin{split}
			&s^{-1}\iint_{Q} \xi^{-1} (|u_t |^2 + \left|\Div(A\nabla u)\right|^2) dx dydt +C s^2 \lambda^2 \int_0^T \int_{\Om\backslash \omega_0} \xi^{2}|u|^2dx dy d t\\
			&+C\iint_Q s^3 \lambda^4 \xi^3\left|A \nabla \eta \cdot \nabla \eta \right|^2|u|^2 dx dydt  + C s  \lambda^2 \iint_Q\xi|\nabla u \cdot A \nabla \eta|^2 dx dy d t\\
			\leq& C\left\|e^{-s \sigma} f\right\|^2
			+Cs \lambda \iint_Q \xi A \nabla u \cdot \nabla udx dy d t+Cs^3 \lambda^3 \int_0^T \int_{\omega_0} \xi^3|u|^2dx dy d t.
		\end{split}
	\end{equation}
	Since
	\begin{equation*}
		\begin{split}
			s \lambda \iint_Q \xi A \nabla u \cdot \nabla udx dy d t=& s \lambda \iint_\Sigma \xi u A \nabla u \cdot \nu d s d t - s \lambda \iint_Q \Div(\xi A\nabla u) udx dy d t\\
			=&- s \lambda \iint_Q \Div(\xi A\nabla u) udx dy d t,
		\end{split}
	\end{equation*}
	and then,
	\begin{equation*}
		\begin{split}
			&- s \lambda \iint_Q \Div(\xi A\nabla u) udx dy d t\\
			=&- s \lambda^2 \iint_Q \xi u\nabla \eta \cdot A\nabla u dx dy d t - s \lambda \iint_Q \xi\Div( A\nabla u) udx dy d t\\
			\le& C \lambda^2 \iint_Q \xi \left|  \nabla \eta \cdot A\nabla u \right| ^2dx dy d t + Cs^2 \lambda^2 \iint_Q \xi u^2dx dy d t\\
			&+C s^{-1} \lambda^{-1} \iint_Q \xi^{-1} \left| \Div( A\nabla u)\right| ^2dx dy d t + Cs^3 \lambda^3 \iint_Q \xi^3 u^2dx dy d t,
		\end{split}
	\end{equation*}
	thus, $s \lambda \iint_Q \xi A \nabla u \cdot \nabla udx dy dt$ can be absorbed by other terms. Therefore, we have
	\begin{equation}\label{LU5}
		\begin{split}
			&s^{-1}\iint_{Q} \xi^{-1} (|u_t |^2 + \left|\Div(A\nabla u)\right|^2) dx dydt +C s^2 \lambda^2 \int_0^T \int_{\Om\backslash \omega_0} \xi^{2}|u|^2dx dy d t\\
			&+C\iint_Q s^3 \lambda^4 \xi^3\left|A \nabla \eta \cdot \nabla \eta \right|^2|u|^2 dx dydt  + C s  \lambda^2 \iint_Q\xi|\nabla u \cdot A \nabla \eta|^2 dx dy d t\\
			\leq& C\left\|e^{-s \sigma} f\right\|^2
			+Cs^3 \lambda^3 \int_0^T \int_{\omega_0} \xi^3|u|^2dx dy d t.
		\end{split}
	\end{equation}
	Using classical arguments, we can transform back to the original variable $w$ and conclude the result.
	
	
	
	\vspace{3mm}
	
	\noindent{\bf Acknowledgement}
	
	\vspace{2mm}
	
	This work is supported by the National Natural Science Foundation of China, the Science-Technology Foundation of Hunan Province.
	
	
	
	
	
	
	\bibliographystyle{abbrvnat}
	%    Insert the bibliography data here.
	\bibliography{ref20230507.bib}
	%\begin{thebibliography}{10}
	%	
	%	\bibitem{Evans} L.C. Evans, Partial Differential Equations, American Mathematical Society Providence, Rholde Island, 2010. 
	%	
	%	
	%\end{thebibliography}
\end{document}














