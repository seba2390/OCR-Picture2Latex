We present pricing mechanisms for several online resource allocation
problems which obtain tight or nearly tight approximations to social
welfare. In our settings, buyers arrive online and purchase bundles of
items; buyers' values for the bundles are drawn from known
distributions. This problem is closely related to the so-called
prophet-inequality of \citet{KS-78} and its extensions in recent
literature. Motivated by applications to cloud economics, we consider
two kinds of buyer preferences. In the first, items correspond to
different units of time at which a resource is available; the items
are arranged in a total order and buyers desire {\em intervals} of
items. The second corresponds to bandwidth allocation over a tree
network; the items are edges in the network and buyers desire {\em
paths}.




% We study an online resource allocation problem where a seller has many
% items with multiplicities to offer, and buyers are interested in buying
% bundles of items with values drawn from a known distribution. Our
% goal is to design a truthful online allocation mechanism that
% maximizes social welfare. This problem is closely related to the
% so-called prophet-inequality of Krengel et al. and its extensions in
% recent literature. Motivated by applications to cloud economics, we
% consider two kinds of buyer preferences. In the first, items are
% arranged in a total order, and buyers are interested in buying {\em
% intervals} of items. In the second, the items are the edges in a
% network, and buyers are interested in buying {\em paths}; we focus on
% the special case of tree networks.

Because buyers' preferences have complementarities in the settings we
consider, recent constant-factor approximations via item prices do not
apply, and indeed strong negative results are known. We develop
{\em static, anonymous bundle pricing} mechanisms.
% In the settings we consider buyers' preferences have
% complementarities, so that recently-developed constant-factor

%---the seller partitions his supply into bundles of
%items and determines a price for each bundle. The buyers arrive in
%arbitrary order and purchase their favorite bundles while supply
%lasts.

For the interval preferences setting, we show that static, anonymous
bundle pricings achieve a sublogarithmic competitive ratio, which is
optimal (within constant factors) over the class of all online
allocation algorithms, truthful or not. For the path preferences
setting, we obtain a nearly-tight logarithmic competitive ratio. Both
of these results exhibit an exponential improvement over item pricings
for these settings. Our results extend to settings where the seller
has multiple copies of each item, with the competitive ratio
decreasing linearly with supply. Such a gradual tradeoff between
supply and the competitive ratio for welfare was previously known only
for the single item prophet inequality.





% % \bmnote{idea: this problem is super important for applications and because it's
% % a classic problem in off- and online algorithm design}
% % Allocating limited resources among competing requests is a fundamental
% % challenge. We study Our benchmark is
% % the expected offline optimal welfare.

% We study an online resource allocation problem where a seller has many items
% with multiplicities to offer and the items are arranged in a total order.
% Buyers are interested in buying {\em intervals} of items, and have different
% values for different intervals, drawn from a known distribution. The seller's
% goal is to design an online allocation mechanism that maximizes social welfare.
% % One feature of this setting that distinguishes it from
% % previous work on online mechanisms is that buyers' preferences have
% % complementarities.
% Importantly, buyers' preferences have complementarities, so that
% recently-developed constant-factor approximations via item prices do
% not apply, and indeed strong negative results are known.

% We consider a static, anonymous bundle pricing mechanism---the seller
% partitions his supply into bundles of items and determines a price
% for each bundle.
% % based on his knowledge of the buyers' value distributions.
%  The buyers arrive % one by one 
% in arbitrary order and
% purchase their favorite bundles while supply lasts. We show that when
% buyers value intervals up to length $L$, bundle pricing achieves an
% %
% % $O\left(\frac{\log L}{B(\log\log L-\log B)}\right)$ approximation to the optimal
% % social welfare, where $B\ge 1$ denotes the minimum multiplicity of any
% % item. Complementing this result, we show that {\em any} online
% % algorithm for this problem, even one that does not respect incentive
% % constraints, must have an approximation ratio of $\Omega\left(\frac{\log
% % L}{B\log\log L}\right)$. Bundle pricing is therefore an optimal mechanism
% % for this problem, and incentive constraints impose at most an
% % additional constant-factor loss in approximation.
% %
% $O\left(\frac{\log L}{\log\log L}\right)$ approximation to the optimal social
% welfare. This result is tight by a known lower bound on {\em any} online
% algorithm for this problem, even one that does not respect incentive
% constraints. Bundle pricing is therefore an optimal mechanism, and incentive
% constraints impose at most an additional constant-factor loss in approximation.

% We also extend our result to settings in which the seller has multiple copies
% of each item. When $B \ge 1$ denotes the minimum multiplicity of any item, we
% show that the approximation factor decays like $\frac1B$, and this is tight.
% Finally, we show that our bounds continue to hold (within constant factors)
% when the seller faces convex production costs. 

% % We also extend our main result to a ``large-market'' setting, in which the
% % seller has at least $B$ copies of each item; here, we show that $\log L$ copies
% % of each item are both necessary and sufficient to guarantee a constant
% % approximation. Finally, we generalize our results to a setting in which the
% % seller faces convex production costs.
