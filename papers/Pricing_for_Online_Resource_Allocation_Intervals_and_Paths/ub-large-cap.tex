\subsection{The large markets setting}
\label{sec:large-cap-ub}

In this section we consider the setting where every item is available in large
supply. Specifically, let $B:= \min_t B_t$. We show that as $B$ increases, the
approximation ratio achieved by bundle pricing gradually decreases. 

\begin{numberedtheorem}{\ref{cor:large-cap-ub}}
For the interval preferences setting, if every item has at least $B>0$
copies available, then a static, anonymous bundle pricing achieves a
competitive ratio of
\[
    O\left(\frac1B\frac{\log L}{\log\log L - \log B}\right)
\] 
when $B<\log L$, and $O(1)$ otherwise.
% In the fixed-capacity setting, with $B = \min_t\capt < \log L$, there exists
% a static, anonymous bundle pricing $(\Pi,\prices)$ such that
% \[
%     \opt \le O\left(\frac1B\frac{\log L}{\log\log L - \log B}\right)
%         \sw(\Pi,\prices).
% \]
% When $B \ge \log L$, there exists a bundle pricing which gives an
% $O(1)$-approximation.
\end{numberedtheorem}

% \begin{theorem}
%     \label{thm:large-capacity}
%     For the fixed capacity setting with $B= \min_t B_t< \log L$, there exists a
%     static, anonymous bundle pricing that achieves an approximation ratio of
%     \[
%         \frac 1B \cdot \frac{\log L}{\log \log L - \log B}
%     \]
%     for social welfare. When $B\ge\log L$, an $O(1)$ approximation ratio is
%     achieved. Given access to an optimal feasible fractional allocation, the
%     pricing can be constructed in polynomial time.
% \end{theorem}

\begin{proof}
  Let $k=\frac 12\min\{B,\log L\}$ and $\alpha=L^{1/k}$. We will
  partition both the item supply and the jobs into multiple instances,
  such that on the one hand, the fractional solution confined to jobs
  within an instance will be feasible for the item supply in that
  instance; on the other hand, within each instance job lengths will
  differ by a factor of at most $\alpha$. Then, applying
  \Cref{thm:arbit-capacity} will give us a fractional unit allocation
  for every instance with a factor of
  $\Omega\big(\frac{\log\log \alpha}{\log \alpha}\big)$ loss in social
  welfare. Applying \Cref{lem:FGL} to the union of these unit
  allocations then implies the theorem.

  Let $x^*$ be the optimal fractional allocation. Divide all jobs into
  $k$ groups $U_1,\cdots,U_k$ according to length: $U_i$ contains all
  jobs of length between $\alpha^{i-1}$ and $\alpha^i$.  Let $x_i$ denote the
  allocation $x^*$ confined to the set $U_i$ scaled by $1/2$, that is,
  $x_i = \half x^*_{U_i}$.  We now specify the supply for instance $i$. Let
  \[ B_{it} = \left\lceil\half\sum_{j\in U_i:t\in
      I_j}x^*_{j}\right\rceil\]
Note that no item is over-provisioned (although some supply may be wasted) :
    \begin{equation*}
        \sum_{i=1}^{k} B_{it} \leq \sum_{i=1}^{k}
                \left(1+\half\sum_{j\in U_i:t\in I_j}x^*_{j}\right) 
        \leq k+\frac{1}{2}B_t \leq B_t. 
    \end{equation*}
    Note also that $x_i$ is feasible for the supply $\{B_{it}\}$.

    Applying Theorem~\ref{thm:arbit-capacity} and \Cref{lem:FGL} to
    the instances defined above then implies an approximation factor
    of $O\Big(\frac{\log \alpha}{\log\log \alpha}\Big)$, which is $O\Big(\frac
    1B\frac{\log L}{\log\log L-\log B}\Big)$ for $B<\log L$, and
    $O(1)$ otherwise.
\end{proof}

 % Divide the entire supply into $k$
 %  parts, each part assigned to one group of jobs. Denote by $B_{it}$
 %  the number of copies of item $t$ assigned to group $U_i$; that is,
 %  let
 %  $\displaystyle B_{it} = \left\lceil\half\sum_{j\in U_i:t\in
 %      I_j}q_jx^*_{j}\right\rceil$.  Then this is a feasible division
 %  (some supply may be wasted) since

 %    Consider the following fractional allocation rule: each group of jobs can
 %    only be allocated to the corresponding supply, while the allocation is
 %    changed to $x'=\frac{1}{2}x^*$.  This is indeed a feasible fractional
 %    allocation, while the social welfare achieved is half of $\fopt$.  By
 %    Theorem~\ref{thm:arbit-capacity} there is a fractional unit allocation for
 %    each group of jobs which in total achieves at least an $\Omega\big(\frac{\log\log
 %    a}{\log a}\big)$ fraction of $\fopt$. Thus
 %    \begin{equation*}
 %        \fopt\le O\Big(\frac{\log a}{\log\log a}\Big)\uafopt =
 %        O\Big(\frac{\log L}{B\log\log L-\log B}\Big)\uafopt.
 %    \end{equation*}

