\section{Introduction}

Consider an online resource allocation setting in which a seller offers
multiple items for sale and buyers with preferences over bundles of items
arrive over time. We desire an incentive-compatible mechanism for allocating
items to buyers that maximizes social welfare---the total value obtained by
all the buyers from the allocation. In the offline setting, this is easily
achieved by the VCG mechanism. In the online context, however, we would like
the mechanism to allocate items and charge payments as buyers arrive, without
waiting for future arrivals.  What would such a mechanism look like, and can it
obtain close to the optimal social welfare?

% In this paper,
We consider two settings for online resource allocation motivated by
applications in cloud economics. In the {\em interval preferences} setting, a
cloud provider has multiple copies of a single resource available to allocate
over time. Customers have jobs that require renting the resource for some
amount of time. If we think of each time unit as a different item,
% in this setting
customers desire {\em intervals} of items. Different intervals,
corresponding to scheduling a job at different times or renting the resource
for different lengths of time, may bring different values to the customer. This
model closely follows the framework described in \cite{babaioff2017era}. Our
second setting, the {\em path preferences setting}, models bandwidth allocation
over a communication network. Each customer is a source-sink pair in the
network that wishes to communicate, and assigns values to paths in the network
between the source and the sink. The items are the edges in the network. We
focus on the special case where the network is a tree.\footnote{Observe that
  the interval preferences setting is a special case of the path preferences
  setting.} See \cite{Jalaparti2016} for further motivation behind this
setting.

It is easy to observe that because of the online nature of the problem
no algorithm for online resource allocation, truthful or otherwise,
can obtain optimal social welfare: the competitive ratio is at least
$2$ even when there is only a single item for sale and the seller
knows both the order of arrival of buyers as well as their value
distributions.\footnote{Suppose there is one item and two buyers. The
  buyer that arrives first has value $1$ for the item; the second
  buyer has value $1/\eps$ with probability $\eps$, and $0$ otherwise,
  for some small $\eps>0$. The optimal allocation is to give the item
  to the first buyer if the second buyer has zero value for it, and
  otherwise give it to the second buyer. This achieves social welfare
  of $2-\eps$ in expectation over the buyers' values. However, any
  online algorithm produces an allocation with expected welfare at
  most~$1$.}  In a remarkable result, \citet{FGL15} showed that this
gap of 2 is the worst possible over a large class of buyer
preferences: a particularly simple and natural incentive-compatible
mechanism, namely {\bf posted item pricing}, achieves a
$2$-approximation to the optimal social welfare in those settings.

Posted pricing is perhaps the most ubiquitous real-world mechanism for
allocating goods to consumers. Supermarkets are a familiar example:
the store determines prices for items, which may be sold individually
or packaged into bundles. Customers arrive in arbitrary order and
purchase the items they most desire at the advertised prices, unless
they're sold out. Many other domains have a similar sequential posted
pricing format, from airfares to online retail to concert tickets.
\citeauthor{FGL15}'s results apply to settings where buyers' values
over bundles of items are fractionally subadditive, a.k.a. XOS. In
these settings, the seller determines a price for each item based on
his knowledge of the buyers' value distributions. These prices
are anonymous and static, meaning that the same prices are offered to
each customer and remain unchanged until supply runs out.

However, the settings we consider exhibit complementarity in buyers' values:
buyers may require certain minimal bundles of items to satisfy their
requirements; anything less brings them zero value. For these settings,
\citet{FGL15} show that anonymous item pricings cannot achieve a competitive
ratio better than linear in the degree of complementarity (that is, the size of
the minimal desired bundle of items). See Footnote~\ref{note:item-pricing-L}
for an example. Is a better competitive ratio possible? Can it be achieved
through a truthful mechanism as simple as static, anonymous item pricings?

\vspace{0.15in}

\parbox[c]{0.95\textwidth}{{\bf We show that near-optimal
    competitive ratios can be achieved for the interval and path
    preferences settings via a static, anonymous bundle pricing
    mechanism. }}

\vspace{0.15in}

Our mechanism is a {\bf posted bundle pricing}: the seller partitions
items into bundles, and prices each bundle based on his knowledge of
the distribution of buyers' preferences. Customers arrive in arbitrary
order and purchase the bundles they most desire at the advertised
prices, unless they're sold out. As in \citeauthor{FGL15}'s work, our
bundle pricings are static and anonymous.

We now elaborate on our results and techniques.

\subsection{Our results}

Recall that in the interval preferences setting, items are arranged in
a total order and buyers desire intervals of items. We assume that
each buyer's value function is drawn independently from some arbitrary
but known distribution over possible value vectors. The seller's
computational problem is essentially a stochastic online interval
packing. Let $L$ denote the length of the longest interval that may be
desired by some buyer. \citet{FGL15} show that no item pricing can
achieve a $o(L)$ competitive ratio in this
setting. \citet{im2011secretary} previously showed that in fact no
online algorithm can achieve a competitive ratio of
$o\left(\frac{\log L}{\log\log L}\right)$. Our first main result matches this
lower bound to within constant factors.

\begin{theorem}
\label{thm:unit-cap-ub}
For the interval preferences setting, there exists a static, anonymous bundle
pricing with competitive ratio $O\left(\frac{\log L}{\log\log
    L}\right)$ for social welfare.
\end{theorem}

The interval preferences setting was studied recently by
\citet{CDH+17}, who also designed a static {\em item} pricing
mechanism for the problem. \citeauthor{CDH+17} showed that when the
item supply is large, specifically, when the seller has
$\tilde\Omega(L^6/\eps^3)$ copies of each item for some $\eps>0$,
static anonymous item pricings achieve a $1-\eps$ approximation to
social welfare. \citeauthor{CDH+17}'s result suggests that there is a
tradeoff between item supply and the performance of posted pricings in
this setting. Our second main result maps out this tradeoff exactly
(to within constant factors) for {\em bundle} pricing. We show that
the approximation factor decreases inversely with item supply, and is
a constant when supply is $\Omega(\log L)$.  In other words, to
achieve a constant approximation via bundle pricing, we require an
exponentially smaller bound on the item supply relative to that
required by \citeauthor{CDH+17} for a $1-\eps$
approximation. Furthermore, this tradeoff is tight to within constant
factors. 

\begin{theorem}
\label{cor:large-cap-ub}
For the interval preferences setting, if every item has at least $B>0$
copies available, then a static, anonymous bundle pricing achieves a
competitive ratio of
\[
    O\left(\frac1B\frac{\log L}{\log\log L - \log B}\right)
\] 
when $B<\log L$, and $O(1)$ otherwise.
\end{theorem}

\begin{theorem}
\label{thm:lb}
For the interval preferences setting, if every item has at least $B>0$
copies available, no online algorithm can obtain a competitive ratio
of $o\left(\frac1B\frac{\log L}{\log\log L}\right)$ for social
welfare.
\end{theorem}

We then turn to the path preferences setting, which appears to be
considerably harder than the special case of interval
preferences. % Recall that here buyers desire paths in a tree
% network. Observe that
In this setting buyers are single-minded in that they desire a
particular path, although their value for this path
is unknown to the seller. As before let $L$ denote the length of the
longest bundle/path that any buyer desires. We first observe that no
online algorithm can achieve a subpolynomial in $L$ competitive ratio
for this setting (see Theorem~\ref{thm:treelowerbound}). We therefore
explore competitive ratios as functions of $H$, the ratio of the
maximum possible value to the minimum possible non-zero value. In
terms of $H$, the construction of \citet{im2011secretary} provides a
lower bound of $\Omega(\log H/\log\log H)$ on the competitive ratio of
any online algorithm. We nearly match this lower bound:

\begin{theorem}
\label{thm:trees-ub}
For the path preferences setting, there exists a static, anonymous bundle
pricing with competitive ratio $O\left(\log H\right)$ for social welfare.
\end{theorem}

%\scnote{Write about improvement with B.}

As for the interval preferences setting, we obtain a linear tradeoff
between item supply and the performance of bundle pricing for the path
preferences setting.
\begin{theorem}
\label{cor:tree-large-cap-ub}
For the path preferences setting, if every edge has at least $B>0$
copies available, then a static, anonymous bundle pricing achieves a
competitive ratio of
\(
    O\left(\frac1B\log H\right)
\)
for welfare.
\end{theorem}
Theorems~\ref{thm:unit-cap-ub} and \ref{cor:large-cap-ub} for the
interval preferences setting are proved in
Section~\ref{sec:ub-interval}. Section~\ref{sec:lowerbound} presents
the lower bounds---Theorems~\ref{thm:lb} and
\ref{thm:treelowerbound}. Our main results for the path preferences
setting, Theorem~\ref{thm:trees-ub} and Theorem~\ref{cor:tree-large-cap-ub}, are proved in
Section~\ref{sec:trees-ub}. Some of our results extend with the same
competitive ratios to settings with non-decreasing marginal production
costs instead of a fixed supply; this setting is discussed in{}
Section~\ref{sec:costs}. Any proofs skipped in the main body of the
paper can be found in Section~\ref{sec:deferred}.

All of our theorems are constructive but require an explicit
description of the buyers' value distributions. The pricings
guaranteed above can be constructed in time polynomial in the sum over
the buyers of the sizes of the supports of their value distributions.

\subsection{Related work and our techniques}

Competitive ratios for welfare in online settings are known as prophet
inequalities following the work of \citet{KS-78} and \citet{Cahn84} on
the special case of allocating a single item. Most arguments for
prophet inequalities follow a standard approach, introduced by
\citet{KW12} and further developed by \citet{FGL15}, of setting
``balanced'' prices or thresholds for each buyer: informally, prices
should be low enough that buyers can afford their optimal allocations,
and at the same time high enough so that allocating items
non-optimally recovers as revenue a good fraction of the optimal
welfare that is ``blocked'' by that allocation. Given such balanced
prices, we can then account for the social welfare of the online
allocation in two components---the seller's share of the welfare,
namely his revenue, and the buyers' share of the welfare, namely their
utility. On the one hand, the prices of any items/bundles sold by the
mechanism contribute to the seller's revenue. On the other hand, any
item/bundle that goes unsold in the online allocation contributes to
the buyers' utility: the buyer who receives it in the optimal
allocation forgoes it in the online allocation for another one with
even higher utility. The argument asserts that in either case good
social welfare is achieved.

When buyers have values with complements and an item pricing is used
this argument breaks down. In particular, it may be the case that a
bundle allocated by the optimal solution goes unsold because only one
of the items in the bundle is sold out.  The loss of the buyer's
utility in this case may not be adequately covered by the revenue
generated by the sold subset of items.\footnote{\label{note:item-pricing-L} Consider, for
  concreteness, the following example due to \citeauthor{FGL15}
  Suppose there are two buyers and $L$ items; the first has value 1
  for any non-empty allocation (i.e., is unit-demand with value 1 for
  every item), and the second has value $L - \eps$ for the set of all
  items and value zero for every subinterval. Observe that for any
  item prices, either the first buyer will be willing to purchase some
  item, thereby blocking the second buyer from purchasing anything, or
  else the second buyer will be unwilling to purchase the full set of
  items. As the optimal welfare is $L-\eps$, no item prices lead to
  better than an $O(L)$-approximation.} We therefore consider pricing
bundles of items.

Recently \citet{DFKL-17} developed a framework for obtaining balanced
prices via a sort of extension theorem. They showed that if one can
define prices achieving a somewhat stronger balance condition in the
full-information setting, where the seller knows the buyers' value
functions exactly, then a good approximation through posted prices can
be obtained in the Bayesian setting as well. A significant benefit of
using this approach is that it suffices to focus on the
full-information setting, and the designer need no longer worry about
the value distribution.

In the full-information setting, even with arbitrary values over
bundles of items, it is easy to construct a static, anonymous bundle
pricing that achieves a $2$-approximation for welfare, as
demonstrated, for example, by \citet{Cohen-Addad-16}.
%\footnote{See Theorem 5.1 in that work.} 
Unfortunately, these prices do not satisfy the strong balance
condition of \citeauthor{DFKL-17} In fact we do not know of any
distribution-independent way of defining bundle prices in the full
information setting that satisfy the balance condition of
\citeauthor{DFKL-17} with approximation factors matching the ones we
achieve. Furthermore, while a main goal of our work is to establish a
tradeoff between supply and competitive ratio for welfare, the
framework of \citeauthor{DFKL-17} does not seem to lend itself to such
a tradeoff.

The crux of our argument for both of our settings lies in constructing a
distribution-dependent partition of items into bundles, and pricing (subsets
of) these bundles. For the interval preferences setting we construct a
partition for which there exists an allocation of bundles to buyers such that
each buyer receives at most one bundle and a good fraction of the social
welfare is achieved. Given such a bundling, we are essentially left with a
``unit-demand'' setting\footnote{A buyer is said to have unit-demand
  preferences if he desires buying at most one item. In our context, buyers may
  desire buying multiple bundles, but the ``optimum'' we compare against
  allocates at most one to each buyer. This allows for the same charging
  arguments that work in the unit-demand case.} for which a prophet inequality
can be constructed using the techniques described above.  We call such an
allocation a {\em unit allocation}. The main technical depth of this result
lies in constructing such a bundling.

In fact, it is straightforward to construct a bundling that leads to an $O(\log
L)$ competitive ratio:\footnote{A detailed discussion of this solution can be
  found in \cite{im2011secretary}.} pick a random power of $2$ between $1$ and
$L$; partition items into intervals of that length starting with a random
offset; and construct an optimal allocation that allocates entire bundles in
the constructed partition. However, our improved $O(\log L/\log\log L)$
approximation requires much more care in partitioning items into bundles of
many different sizes, and requires the partitioning to be done in a
distribution-dependent manner.

% In order to achieve a tight bound on the
% competitive ratio, we need to treat separately items whose welfare
% contributions come from bundles of many different sizes and those
% whose welfare contributions are dominated by bundles of a certain size.

As mentioned earlier, the path preferences setting generalizes
interval preferences, but appears to be much harder. In particular, we
don't know how to obtain a constant competitive ratio even when all
desired paths are of equal length. We show, however, that if all
buyers have equal values and all edges have equal capacity, it becomes
possible to identify up to two most contentious edges on every path,
and the problem behaves like one where every buyer desires only two
items. For this special case, it becomes possible to construct a
pricing using techniques from \cite{FGL15}. Unfortunately, this
argument falls apart when different items have different multiplicity.
In order to deal with multiplicities, we present a different kind of
partitioning of items into bundles or layers and a constrained
allocation that we call a {\em layered allocation}, such that each
layer behaves like a unit-capacity setting. These ideas altogether
lead to an $O(\log H)$ competitive ratio.

As mentioned previously, our approach lends itself to achieving
tradeoffs between item supply and competitive ratio. On the one hand,
when different items are available to different extents, we need to be
careful in constructing a partition into bundles and in some places
this complicates our arguments. On the other hand, large supply allows
us some flexibility in partitioning the instance into multiple smaller
instances, leading to improved competitive ratios. The key to enabling
this partitioning is a composability property of our analysis: suppose
we have multiple disjoint instances of items and buyers, for each of
which in isolation our argument provides a good welfare guarantee;
then running these instances together and allowing buyers
to purchase bundles from any instance provides at least half the sum
of the individual welfares. In the path preferences setting, obtaining
this composability crucially requires buyers to be single-minded.

\paragraph{Discussion and open questions.} Two implications of our
results seem worthy of further study. First, for the settings we
consider as well as those studied previously, posted pricings perform
nearly as well as arbitrary online algorithms, truthful or not. Can
this be formalized into a meta-theorem that holds for broader
contexts? Second, a modest increase in supply brings about significant
improvements in competitive ratio for the settings we study. However,
once the competitive ratio hits a certain constant factor, our
techniques do not provide any further improvement. Can this tradeoff
be extended all the way to a $1+\epsilon$ competitive ratio?

Another natural open problem is to extend our guarantees for the path
preferences setting to general graphs. This appears challenging. Our
arguments rely on a fractional relaxation of the optimal
allocation. General graphs exhibit an integrality gap that is
polynomial in the size of the graph even when all buyers are
single-minded, have the same values (0 or 1 with some probability),
and have identical path lengths; this integrality gap is driven by the
combinatorial structure exhibited by paths in graphs. Indeed this
setting appears to be as hard as the most general setting with no
constraints on buyers' values. It may nevertheless be possible to obtain a
non-trivial competitive ratio relative to a different relaxation of
the offline optimum.

% Our interval preferences model is related to many classic scheduling
% problems generalized by the Job Interval Selection Problem (JISP) (see
% \cite{Spieksma99,COR06} and references therein).  These works focus on
% finding optimal schedules in the absence of incentive constraints, and
% seek to maximize throughput rather than weighted value. Optimal
% solutions are computationally hard to find, but e.g.  \cite{COR06}
% give a polynomial-time 1.582-approximation. Because our buyers arrive
% online and we seek to maximize weighted value, we require different
% techniques and cannot hope for a constant approximation.

% SPMs were first studied for the problem of revenue maximization, where
% computing the optimal mechanism is a computationally hard problem and no simple
% characterizations of optimal mechanisms are known.
\paragraph{Other related work.}
Sequential pricing mechanisms (SPMs) were first studied for the problem of
revenue maximization, where computing the optimal mechanism is a
computationally hard problem and no simple characterizations of optimal
mechanisms are known. A series of works (e.g., \cite{CHK-07, BH-08, CHMS-10,
CMS-10, bilw-focs14, RW-15, CM-16, CZ-17}) showed that in settings where buyers
have subadditive values, SPMs achieve constant-factor approximations to
revenue. In most interesting settings, good approximations to revenue
necessarily require non-anonymous pricings. As a result, techniques in this
literature are quite different from those for welfare.  \cite{GHKSV14} gives
(non-truthful) online algorithms which obtain constant-factor approximations in
our settings when supply is unit-capacity, buyers are single-minded, and all
values are identical.  There is also a long line of work on welfare-maximizing
mechanisms with buyers arriving online in the worst case setting where the
seller has no prior information about buyers' values \citep{lavi2015online,
hajiaghayi2005online, cole2008prompt, azar2011prompt, azar2015truthful,
CDH+17}. The worst case setting generally exhibits very different techniques
and results relative to the Bayesian setting we study.



% \subsection{Old stuff follows}

% % Can we
% % design a different online mechanism that achieves a good
% % approximation?

% % \citet{FGL15} also illustrated the limitations of posted item pricing: when
% % buyers' values exhibit complementarity, anonymous item pricings no longer
% % guarantee a $2$-approximation---in fact, the approximation factor worsens
% % linearly with the degree of complementarity. (See Section~\ref{sec:techniques}
% % for an example.) Can we design a different online mechanism that achieves a
% % good approximation?

% % In this paper, we consider a special case of the online social welfare
% % maximization problem with complementarities, namely buyers with {\bf
% %   interval preferences}.  We assume that items are totally ordered and
% % buyers desire intervals over items.  Buyers' values for different
% % intervals may be arbitrarily correlated.\footnote{The only restriction
% %   we place on values is that they are monotone: if an interval
% %   contains another, then it is valued at no less than the latter's
% %   value.} As an example of such a setting, consider the classic job
% % scheduling problem in which jobs with various sizes must be assigned
% % to a machine to maximize the total value (or weight) of assigned
% % jobs. The items can then be thought of as units of time when the
% % machine is available. Scheduling a job corresponds to allocating an
% % interval of time to the job.

% Because interval preferences exhibit complementarity,\footnote{For
%   example, for a buyer with a job of length 10 in the job scheduling
%   setting, allocating any fewer than 10 units brings 0 value, but
%   allocating 10 units or more may bring a large value.} as
% \citeauthor{FGL15} show, item pricings no longer achieve a constant-factor
% approximation to social welfare. In fact, when buyers desire intervals up to
% length $L$, no online algorithm, even one ignoring incentive constraints, can
% achieve an approximation factor of $o(\log L/\log \log L)$ (see
% \cite{im2011secretary} and \Cref{sec:lowerbound}). 

% Our first main result is that {\em there exists a simple,
%   incentive-compatible mechanism for the interval preferences setting
%   that achieves a competitive ratio matching the above lower bound} --
% namely $O(\log L/\log \log L)$. Our mechanism is a {\bf posted bundle
%   pricing}: the seller partitions items into bundles, and prices each
% bundle based on his knowledge of the distribution of buyers'
% preferences. Customers arrive in arbitrary order and purchase the
% bundles they most desire at the advertised prices, unless they're sold
% out. As in \citeauthor{FGL15}'s work, our bundle pricings are static
% and anonymous.

% {\em Large markets.} The interval preferences setting was studied
% recently by \citet{CDH+17}, who also designed a static {\em item} pricing
% mechanism for the problem. \citeauthor{CDH+17} showed that when the item supply
% is large, specifically, when the seller has $\tilde\Omega(L^6/\eps^3)$ copies
% of each item for some $\eps>0$, static anonymous item pricings achieve a
% $1-\eps$ approximation to social welfare. \citeauthor{CDH+17}'s result suggests
% that there is a tradeoff between item supply and the performance of posted
% pricings in this setting. Our second main result maps out this tradeoff exactly
% (to within constant factors) for {\em bundle} pricing. We show that the
% approximation factor decreases inversely with item supply, and is a constant
% when supply is $\Omega(\log L)$.  In other words, to achieve a constant
% approximation via bundle pricing, we require an exponentially smaller bound on
% the item supply relative to that required by \citeauthor{CDH+17} for a $1-\eps$
% approximation.  See formal statement below.

% % Let $B$ denote the minimum number of units of any item available. We
% % show, on the one hand, that no online algorithm, even one ignoring
% % incentive constraints, can achieve an approximation factor of
% % $o(\frac {\log L}{B\log \log L})$. On the other hand, we show that static
% % anonymous bundle pricings achieve a competitive ratio of $O(\frac{\log
% % L}{B(\log \log L-\log B)})$ in this setting. Our upper and lower
% % bounds are tight within constant factors for any value of $B$ that is
% % $O(\log L/\log \log L)$, as well as any value that is $\Omega(\log
% % L)$.  

% {\em Production costs.} Our results also extend to settings where the
% seller does not have a fixed supply of items, but can produce any number of
% copies of each item at a non-decreasing marginal cost.
% % (i.e. the total cost of allocating $n$ copies is a convex function of $n$).
% \citet{Sekar17} previously extended \citeauthor{FGL15}'s result for XOS
% buyers to this setting, showing that static and anonymous item prices obtain a
% 2-approximation to the optimal welfare. 
% % Our model for production costs is identical to that of \citet{Sekar17}, who
% % extends \citeauthor{FGL15}'s result for XOS buyers to the convex costs setting.
% For the interval preferences setting with costs, we show that
% anonymous bundle pricings achieve the same asymptotic approximation
% ratio as in the fixed supply setting. Unlike \citeauthor{Sekar17}'s
% work, however, our pricing is not static---bundle prices may increase
% with cost as more and more items are sold. We leave open the problem
% of designing a static pricing for this case.

% % However, whereas \citeauthor{Sekar17} is able to extend the item-pricing result
% % with static prices, his techniques do not apply to our setting for two reasons.
% % First, our bundles may overlap (i.e. the same item may be offered as part of
% % two or more bundles), so the cost of a given bundle depends also on sales of
% % other bundles.  And second, the cost of allocating a bundle depends on the
% % subset of items within the bundle that are actually consumed. Thus, a given
% % bundle does not have a well-defined cost curve. Nevertheless, we show that the
% % approximation achieved by static prices in the fixed-capacity setting can be
% % obtained via adaptive prices.

% Altogether, our results show that posted pricings are an optimal
% (within constant factors) online mechanism for maximizing social
% welfare for a large class of buyer preferences exhibiting
% complementarities.% \\ 

% % Consider an online resource allocation setting in which a seller offers
% % multiple items for sale and buyers with preferences over bundles of items
% % arrive over time. We desire an incentive-compatible mechanism for allocating
% % items to buyers that maximizes social welfare---the total value obtained by
% % all the buyers from the allocation. In the offline setting, this is easily
% % achieved by the VCG mechanism. In the online context, however, we would like
% % the mechanism to allocate items and charge payments as buyers arrive, without
% % waiting for future arrivals.  What would such a mechanism look like, and can it
% % obtain close to the optimal social welfare?



% % % One candidate mechanism is posted pricing.  Perhaps no economic system for
% % % allocating goods to consumers is as ubiquitous as posted prices.
% % Posted prices are perhaps the most ubiquitous economic system for allocating
% % goods to consumers. Supermarkets are a familiar example: the store
% % determines prices for items, which may be sold individually or packaged into
% % bundles. Customers arrive in arbitrary order and purchase the items they most
% % desire at the advertised prices, unless they're sold out. Many other domains
% % have a similar {\bf sequential posted pricing} format, from airfares to online
% % retail to concert tickets. But does posted pricing result in allocations that
% % are socially desirable?

% % % Observe that no online algorithm is able to obtain the optimal social welfare,
% % % even when inputs are stochastic and in the absence of incentive constraints.
% % % Suppose, for example, that there is one item to sell and two buyers. The buyer
% % % that arrives first has a value of $1$ for the item. The second buyer has a
% % % value of $1/\eps$ with probability $\eps$, and $0$ otherwise. The optimal
% % % allocation achieves a social welfare of $2-\eps$ in expectation over the
% % % buyers' values; however, any online mechanism or algorithm will produce an
% % % allocation with expected welfare at most $1$, a gap of $2$ from the optimum. In
% % % the online setting, we therefore cannot hope to achieve an approximation to
% % % social welfare better than $2$. 

% % In a remarkable result, \citet{FGL15} showed that posted pricing obtains a
% % $2$-approximation to social welfare in a large class of online stochastic
% % resource allocation settings. This precisely matches the lower
% % bound\footnote{Suppose there is one item and two buyers. The buyer that arrives
% % first has value $1$ for the item; the second buyer has value $1/\eps$ with
% % probability $\eps$, and $0$ otherwise. The optimal allocation achieves social
% % welfare of $2-\eps$ in expectation over the buyers' values; however, any online
% % algorithm produces an allocation with expected welfare at most $1$.} on {\em
% % any} online algorithm, demonstrating that incentive constraints impose no
% % additional loss in approximation. Their result applies when buyers have
% % fractionally subadditive preferences over bundles of items. The mechanism is a
% % simple, anonymous item pricing---the seller uses his knowledge of the buyers'
% % demand distribution to determine prices for individual items.  These prices are
% % then announced, and buyers arrive over time to purchase their favorite bundles
% % of items at these prices while supplies last. Unfortunately, \citeauthor{FGL15}
% % show that when buyers' preferences have complemetarities, item pricings cannot
% % obtain a good approximation to social welfare (see \Cref{sec:techniques}).

% % In this paper, we consider a special case of the problem with
% % complementarities, namely buyers with {\bf interval preferences}.  We assume
% % that items are totally ordered and buyers desire intervals over items.  Buyers'
% % values for different intervals may be arbitrarily correlated.\footnote{The only
% % restriction we place on values is that they are monotone: if an interval
% % contains another, then it is valued at no less than the latter's value.} As an
% % example of such a setting, consider the classic job scheduling problem in which
% % jobs with various sizes must be assigned to a set of machines to maximize the
% % total value (or weight) of assigned jobs. The items can then be thought of as
% % units of time when a machine is available. Because interval preferences exhibit
% % complementarity,\footnote{For example, for a buyer with a job of length 10 in
% % the job scheduling setting, allocating any fewer than 10 units brings 0 value,
% % but allocating 10 units or more may bring a large value.} they do not fit the
% % purview of settings considered by \citeauthor{FGL15}. 

% % For this important special case,
% % % we present results similar to those of \citeauthor{FGL15}:
% % we show that anonymous posted pricings achieve an approximation to social
% % welfare that matches (within constant factors) the best competitive ratio
% % achievable even in the absence of incentive constraints.  However, even without
% % incentive constraints, the online setting precludes the possibility of a
% % constant factor approximation. Both our upper and lower bounds depend
% % logarithmically on the length of the longest interval desired by a buyer.
% % Furthermore, due to the complementarity inherent in buyers' preferences, item
% % pricings perform poorly. It is therefore essential to price bundles of items.
% % Our mechanism partitions the items offline into bundles and assigns a static
% % anonymous price to each bundle. 

% % % applications of this special case: job scheduling, tou pricing?
% % \paragraph{Application: job scheduling.}
% % The problem of pricing cloud resources represents an immediate application.
% % Following previous work~\citep{azar2015truthful, ERA16, CDH+17}, we consider a
% % model in which a cloud provider has multiple copies of a single resource to
% % allocate over time.  Each buyer has a job that requires the use of the resource
% % for some number of time steps.  Buyers' values for obtaining the resource may
% % depend on when the resource is allocated to them.  For example, some buyers may
% % have a hard deadline by which their job must be completed, while others' values
% % may degrade steadily over time.  \citet{CDH+17} proposed a simple pricing-based
% % market mechanism for this problem. The seller announces in advance a price per
% % unit of resource for each time unit, which may vary over time. Each buyer then
% % considers available blocks of time that meet their requirements, and purchases
% % the one that maximizes their value minus the price.

% % The recent push towards ``time-of-use'' (TOU) pricing in electricity markets
% % provides another important and timely application.\footnote{For example, the
% % California Public Utilities Commision has proposed transitioning all
% % residential electricity customers to time-of-use rates by
% % 2019~\citep{CPUC-AB327, TOU-factsheet}.} Traditionally, electricity has been
% % sold at a flat rate based on usage.  TOU pricing offers a means to better
% % manage demand fluctuations and achieve more efficient utilization: prices are
% % set higher or lower than normal during periods when very high or very low
% % demand is expected, thereby incentivizing customers to move temporally flexible
% % demand from peak to non-peak periods. To what extent can TOU pricing counter
% % stochasticity in demand and achieve efficient allocation?  As power grids
% % increasingly utilize renewable sources of energy, the varying costs associated
% % with different sources become increasingly significant. Do costs and supply
% % fluctuations hurt the effectiveness of TOU pricing?

% % \noindent
% % We now describe our model and results formally.  

% \subsection{Model and results}
% \label{sec:model}

% Formally, we consider a setting with a totally ordered set $T$ of items and $n$
% buyers.  Each buyer $i \in [n]$ has a unit-demand valuation defined over
% intervals: for any set of items $\alloc$, $\vali(\alloc) = \max_{I \subseteq
% \alloc} \, \vali(I)$ where $I$ ranges over all contiguous intervals contained
% in $X$.  Valuations are monotone, and we assume $\vali(\emptyset)=0$.  Our goal
% is to maximize the buyers' total welfare---that is, allocate a set of
% non-intersecting intervals to maximize the sum of the corresponding values.

% Our setting is Bayesian: buyer $i$'s value function is drawn from
% independent distribution $\disti[i]$, and the seller knows the prior
% $\dists = \disti[1]\times\cdots\times\disti[n]$. We emphasize that
% values may be correlated across intervals, but not across buyers.

% Our main result concerns a seller with just one copy of each item; we refer to
% this as the {\em unit-capacity} setting. We extend this result to settings in
% which the seller has multiple identical copies of each good, up to a capacity
% $B_t$ for item $t$; we refer to this as the {\em fixed-capacity} setting.
% Our most general results apply to a setting in which the seller can acquire
% additional copies of each item at non-decreasing cost. In particular, we assume
% the seller incurs cost $\cti$ when allocating\footnote{Note that the seller can
% acquire items online; supply need not be provisioned before demand is
% realized.} the $i$th copy of item $t$, and $\cti \ge \cti[ti']$ for $i > i'$.
% Note that this generalizes the fixed-capacity setting by taking $\cti = 0$ for
% $i \le B_t$ and $\infty$ otherwise. 

% The mechanisms we study are static and anonymous sequential bundle
% pricings. The seller partitions the multiset of items into bundles,
% $\Pi=(S_1, \cdots, S_K)$, where each $S_k$ is an interval. For each
% bundle $S_k$, the seller announces a price $\pricek$.
% %Let $\menu = \{S_k\}_{[K]}$ and $\prices = (\pricek)_{[K]}$.
% Buyers arrive in adversarial order. When buyer $i$ arrives, she selects a
% subset of remaining unsold bundles to maximize her value for the items she
% receives less the total payment. Let $(\Pi,\prices)$ denote the
% partition into items and corresponding pricing.

% % \subsection{Results}

% %\paragraph{Upper bounds on item pricings.}
% Let $\opt$ denote the optimal expected social welfare, and
% $\sw(\Pi,\prices)$ denote the expected social welfare obtained by the
% static, anonymous bundle pricing $(\Pi,\prices)$. We obtain the
% following results. Our bounds are stated in terms of $L$, the maximum
% length\footnote{Recall that $\vali$ is monotone; $L$ is thus the size
%   of a maximal interval $I$ such that $\vali(I)$ is strictly greater
%   than $\vali(I')$ for every subinterval $I'$.} of any interval
% desired by any buyer.\footnote{In fact, our analysis can be easily
%   modified to depend instead on $\Delta$, the maximum {\em ratio}
%   between the lengths of any two intervals; the results hold with the
%   same dependence on $\Delta$.}  
% %(Recall that $\vali$ is monotone; $L$
% %is thus the size of a maximal interval $I$ such that $\vali(I)$ is
% %strictly greater than $\vali(I')$ for every subinterval $I'$.)

% % Our main result is
% %a tight bound on the welfare of a bundle pricing.
% \begin{theorem}
% \label{thm:unit-cap-ub}
% In the fixed-capacity setting, there exists a static, anonymous bundle
% pricing $(\Pi,\prices)$ such that
% \[
%     \opt \le O\left(\frac{\log L}{\log\log L}\right)\sw(\Pi,\prices).
% \]
% \end{theorem}
% % This matches a lower bound by \citet{im2011secretary} which applies to {\em
% % any} online algorithm (i.e., without incentive constraints) for the
% % unit-capacity setting. See
% % \Cref{sec:lowerbound} for a discussion on lower bounds.

% % For the fixed-capacity setting (i.e., items with arbitrary multiplicity $\ge
% % 1$) we give a reduction to the unit-capacity setting which loses at most an
% % additional constant factor.
% % \begin{corollary}
% % \label{cor:arbitrary-cap-ub}
% % In the {\em fixed-capacity setting}, there exists a static, anonymous bundle
% % pricing $(\menu,\prices)$ such that
% % \[
% %     \opt \le O\left(\frac{\log L}{\log\log L}\right)\sw(\menu,\prices).
% % \]
% % \end{corollary}

% \noindent
% In the setting where capacities $B_t$ are large, we obtain an improved bound.
% %In fact, given a lower bound on the multiplicity of every item, we prove a
% %bound which is potentially much stronger.
% \begin{theorem}
% \label{cor:large-cap-ub}
% In the fixed-capacity setting, with $B = \min_t\capt < \log L$, there exists
% a static, anonymous bundle pricing $(\Pi,\prices)$ such that
% \[
%     \opt \le O\left(\frac1B\frac{\log L}{\log\log L - \log B}\right)
%         \sw(\Pi,\prices).
% \]
% When $B \ge \log L$, there exists a bundle pricing which gives an
% $O(1)$-approximation.
% \end{theorem}

% In \Cref{sec:lowerbound}, we prove that these bounds are tight within
% constant factors for most values of $B$. Our lower bound is an
% extension of a lower bound by \citet{im2011secretary} which applies to
% the unit-capacity setting.
% \begin{theorem}
% \label{thm:lb}
%   In the fixed-capacity setting, with $B = \min_t\capt$, no
%   online algorithm can obtain an approximation factor of
%   $o\left(\frac1B\frac{\log L}{\log\log L}\right)$ for social welfare.
% \end{theorem}

% Finally, we extend the above results to settings with non-decreasing
% marginal production costs. Because the costs setting contains the
% fixed-capacity setting as a special case, this result is also tight.
% \begin{theorem}
% \label{thm:costs-ub}
% In the costs setting, there exists an anonymous bundle pricing
% $(\Pi,\prices)$ such that
% \[
%     \opt \le O\left(\frac{\log L}{\log\log L}\right)\sw(\Pi,\prices).
% \]
% \end{theorem}

% All of our theorems are constructive but require an explicit
% description of the buyers' value distributions. The pricings
% guaranteed above can be constructed in time polynomial in the sum over
% the buyers of the sizes of the supports of their value distributions.


% \subsection{Challenges and techniques}
% \label{sec:techniques}

% We begin by reviewing \citet{FGL15}'s analysis of the social welfare
% achieved by items pricings. Suppose for simplicity that buyers are
% unit demand, that is, every buyer wants to buy a single item. The
% social welfare of an allocation can be accounted for in two
% components---the seller's share of the welfare, namely his revenue,
% and the buyers' share of the welfare, namely their utility. Consider
% any particular item $t$ and suppose that for a particular draw of
% buyer valuations and arrivals, the item gets sold to buyer $j$ in the
% optimal allocation. \citeauthor{FGL15} argue that if this item gets
% sold, then it generates good revenue for the seller. If the item
% doesn't get sold, then since it was still available when buyer $j$
% arrived and $j$ bought a different item, $j$ must have obtained good
% utility. In either case, good social welfare is achieved. 

% % set prices of items in such a manner that
% % the social welfare of the optimal allocation gets split evenly between
% % those two components. Then, if an item gets sold, it generates the
% % same revenue for the seller as in the optimal allocation. On the other
% % hand, if an item goes unsold, we can argue for XOS (fractionally
% % subadditive) buyers that the buyer assigned this item in the optimal
% % allocation obtains good utility from having bought a different set of
% % items. In either case, good social welfare is guaranteed. The beauty
% % of this argument is that it depends only on whether or not an item has
% % sold at the end of the SPM, and not on {\em how} the process plays
% % out. It is therefore agnostic to the order of arrival of the agents.

% When buyers have values with complements---that is, the value for a bundle is
% greater than the value for the items separately---this argument breaks down. In
% particular, it may be the case that a bundle allocated by the optimal solution
% goes unsold because one of the items in the bundle is sold out.  The loss of
% the buyer's utility in this case may not be adequately covered by the revenue
% generated by the sold subset of items. Consider, for concreteness, the
% following example due to \citeauthor{FGL15} Suppose there are two buyers and
% $L$ items; the first has value 1 for any non-empty allocation (i.e., is
% unit-demand with value 1 for every item), and the second has value $L - \eps$
% for the set of all items and value zero for every subinterval. Observe that for
% any item prices, either the first buyer will be willing to purchase some item,
% thereby blocking the second buyer from purchasing anything, or else the second
% buyer will be unwilling to purchase the full set of items. As the optimal
% welfare is $L-\eps$, no item prices lead to better than an
% $O(L)$-approximation.

% \paragraph{Meta-items.} A natural avenue for extending
% \citeauthor{FGL15}'s argument from item pricings to bundle pricings is
% to think of each bundle as a meta-item. The trouble is that meta-items
% share supply. A buyer arriving in the market may be unable to purchase
% a meta-item (i.e., bundle) $S$ because prior sales of other
% overlapping meta-items have depleted the supply of some item in
% $S$. In this case, once again, the buyer's utility and the seller's
% revenue may both be low.

% % The loss of the buyer's utility in this event may not be
% % adequately covered by the revenue generated by the other overlapping
% % sales.

% To deal with this issue, we ask whether there is a partition of the
% multiset of all items into meta-items that satisfies the following
% properties: (1) no two meta-items share supply; (2) the partitioning
% is independent of the instantiation of buyers' values; and (3) for
% each instantiation of buyers' values, it is possible to generate an
% assignment of meta-items to buyers so that each buyer gets at most one
% meta-item, and not much social welfare is lost in expectation. If such
% a partitioning exists, it immediately implies a bundle pricing with a
% good approximation: we simply price each bundle corresponding to a
% meta-item just as Feldman et al.'s argument prices items. Condition
% (3) in particular allows us to charge any lost revenue from not
% selling a meta-item to the utility obtained by the buyer allocated
% that meta-item in the optimal allocation.\footnote{We note that for
%   the \citeauthor{FGL15} style argument to work, it does not matter
%   how many meta-items a buyer desires or eventually buys in the
%   mechanism. All that matters is that the near-optimal solution we are
%   comparing against allocates at most one meta-item to each buyer.} We
% call such a partitioning along with the special allocation guaranteed
% by (3) a {\em unit allocation}.

% \paragraph{Finding unit allocations.} Optimal allocations are not necessarily
% unit allocations, and there is necessarily a loss in welfare from
% imposing the conditions described above. However, as long as we can
% find a unit allocation that obtains a good fraction of the optimal
% welfare, it guarantees us a good bundle pricing. The technical core of
% our paper is then in constructing such unit allocations. In the
% interval preferences setting, if the bundles desired by all buyers are
% of roughly the same length (within constant factors, say), it is
% straightforward to construct a unit allocation that achieves a
% constant fraction of the optimal welfare. This immediately leads to an
% $O(\log L)$ approximation. Improving this bound to
% $O(\log L/\log\log L)$ requires much more care in partitioning items
% into bundles of many different sizes.

% \paragraph{Extensions to more general valuations.} Since our analysis
% extends \citeauthor{FGL15}'s arguments from items to meta-items, it is
% easy to generalize to settings where buyers' valuations are additive
% or XOS over interval-valued functions. Our techniques imply the same
% bounds via bundle pricings for this more general class of
% valuations. We omit the details for these extensions.

% % \paragraph{Bundle pricings and unit allocations.}
% % In light of the limitations on item-pricing demonstrated above, we explore the
% % power of bundle pricings. Our analysis is based on a fractional relaxation of
% % the problem. Given a fractional solution which can be partitioned into
% % solutions over disjoint bundles of items, we treat each bundle as a
% % ``meta-item'' and set prices in a manner very similar to \cite{FGL15}. For each
% % bundle, we set a price which balances the revenue obtained when the bundle is
% % sold and the utility that buyers could have obtained by purchasing the
% % bundle.\footnote{This style of argument is originally due to \citet{KW12}, but
% %   that work did not describe it using the terminology of revenue and utility.}
% % The bundle pricing thus obtains a 2-approximation to the welfare of our
% % fractional solution, analogous to the 2-approximation of \citeauthor{FGL15}
% % However, in order for this analysis to go through, we require that the
% % fractional solution not allocate items across bundle boundaries and that each
% % bundle be allocated at most once.\footnote{For example,
% %   the bundle $\{1,2\}$ could be allocated to the three intervals $\{1\},
% %   \{2\}$, and $\{1,2\}$, with each interval receiving an allocation of $1/2$.
% %   This is a feasible fractional allocation, but the bundle as a whole is
% %   allocated $3/2$ times.}
% % We call an allocation which satisfies these two properties a {\em fractional
% % unit allocation}. 

% % \paragraph{Constructing good unit allocations.}
% % Our main result is the construction of a fractional unit allocation as described above
% % which does not lose too much welfare. The straightforward approach is to bucket
% % interval allocations by length, obtaining $\log L$ buckets, and choose the
% % bucket with the largest fractional value. It is not difficult to find a
% % fractional unit allocation in this manner which obtains an $O(\log L)$
% % approximation. Obtaining a tight $O(\frac{\log L}{\log\log L})$-approximation,
% % however, requires significantly refining this approach. Crucially, our choice
% % of bundles depends on the fractional solution itself. Starting with a
% % fractional allocation, we bucket allocations as above, and then further
% % classify allocations in each bucket as either heavy-weight or light-weight.
% % Within each bucket, heavy-weight allocations can be grouped into a single
% % bundle that accounts for the fractional welfare obtained from the corresponding
% % items. Light-weight allocations, however, must be bundled together with
% % light-weight allocations from other buckets. This classification into heavy-
% % and light-weight jobs requires further refining buckets by value, in addition
% % to length, and carefully charging the fractional welfare in each bucket against
% % the total fractional welfare attributable to the items in the corresponding
% % bundle.

% % \bmnote{talk about extensions}


% For welfare maximization, following the work of \citet{FGL15}, \citet{DFKL-17}
% developed a general framework for obtaining approximations to social welfare
% through SPMs. They show that if prices can be set in the full-information
% setting to meet certain balance conditions, then this leads to an approximation
% in the Bayesian setting. However, this approach necessarily leads to SPMs that
% are non-anonymous and adaptive---that is, the pricing offered to a buyer depends
% both on the identity of the buyer and previous sales. As such their approach
% does not apply to the problems we study.

% There is a long line of work on mechanisms for the job scheduling setting
% \citep{lavi2015online, hajiaghayi2005online, cole2008prompt, azar2011prompt,
% azar2015truthful, CDH+17}, but with the exception of \citep{CDH+17}, this work
% considers the worst-case setting where the seller has no knowledge about the
% buyers' preferences.
