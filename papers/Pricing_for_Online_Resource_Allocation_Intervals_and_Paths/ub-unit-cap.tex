\subsection{The unit capacity setting}
\label{sec:unit-cap-ub}

% We begin with some definitions. (Move some of these to section 2?) Let $x$ denote a fractional allocation. For a subset $A$ of jobs, we use $x_A$ to denote the fractional allocation confined to set $A$ and zeroed out everywhere else. That is, $(x_A)_j = x_j$ for $j\in A$ and $=0$ for $j\not\in A$.
% $\fval(x)$ is the total fractional value in $x$ without worrying about feasiblity: $\fval(x) = \sum_j v_j x_j$ and $\fval(x_A) = \sum_{j\in A} v_j x_j$. 
% $\fwt(x)$ is the total fractional weight of $x$: $\fwt(x) = \sum_j x_j$ and $\fwt(x_A) = \sum_{j\in A} x_j$. 
% 
% We say that a fractional allocation $x$ is a {\em fractional unit allocation} if there exists a partition of the items into intervals $\{T_1, T_2, T_3, \cdots\}$ such that:
% \begin{itemize}
% \item For all $j\in U$ with $x_j>0$, there exists an index $i$ with $I_j\subseteq T_i$ and $I_j\cap T_{i'}=\emptyset$ for all $i'\ne i$.
% \item Denoting $A_i = \{j: I_j\subseteq T_i\}$, we have $\fwt(x_{A_i})\le 1$.
% \end{itemize}

The main claim of this section is that for any feasible fractional allocation
$x$ there exists a fractional unit allocation $x'$ such that $\fval(x)\le
O(\beta)\fval(x')$ where $\beta$ is defined such that $\beta\log\beta = \log
L$. Then, if $x$ is the optimal fractional allocation for the given instance,
we can apply Lemma~\ref{lem:FGL} to the allocation $x'$ to obtain a pricing
that gives an $O(\beta)$ approximation. Observe that $\beta=O\big(\frac{\log
L}{\log\log L}\big)$, so we get the desired approximation factor. A formal
statement of this claim is given at the end of this subsection.

Before we prove the claim, let us discuss the intuition behind our analysis.
One approach for producing a fractional unit allocation is to find an
appropriate partition of items into disjoint intervals $\{T_1, T_2, \cdots\}$,
and remove from $x$ all jobs that do not fit neatly into one of the intervals.
We can then rescale the fractional allocations of jobs within each interval
$T_i$, so that their total fractional weight is $1$. The challenge in carrying
out this approach is that if the intervals $T_i$ are too short, then they leave
out too much value in the form of long jobs. On the other hand, if they are too
long, then we may require a large renormalizing factor for the weight, again
reducing the value significantly. 

To account for these losses in a principled manner we consider a suite
of nested partitions, one at each length scale, of which there are
$\log L$ in number.\footnote{All of our arguments extend trivially to
  settings where $L$ denotes the ratio of lengths of the largest to the smallest
  interval of interest, because the number of relevant length scales
  is logarithmic in this ratio.} We then place a job of length $\ell$ in the interval that contains it (if one exists) at length scale $\sim 2\ell$. The intervals over all of the length scales together capture much of the fractional value of the allocation $x$. Furthermore, the fractional weight within each interval can be bounded by a constant. At this point, any single partition gives us a fractional unit allocation. However, since the number of length scales is $\log L$, picking a single one of these unit allocations only gives an $O(\log L)$ approximation in the worst case. In order to do better, we argue that there are two possibilities: either (1) it is the case that many intervals have a much larger than average contribution to total weight, in which case grouping these together provides a good unit allocation; or, (2) it is the case that most intervals have low total weight, so that we can put together multiple length scales to obtain a unit allocation without incurring a large renormalization factor for weight.

We now proceed to prove the claim. The proof consists of modifying $x$ to obtain $x'$ through a series of refining steps. Steps 1 and 2 define the suite of partitions and placement of jobs into intervals as discussed above. Step 3 provides a criterion for distinguishing between the cases (1) and (2) above. Step 4 then provides an analysis of case (1), and Step 5 an analysis of case (2).

\subsubsection*{Step 1: Filtering low value jobs}

We first filter out jobs that do not contribute enough to the solution $x$ depending on when they are scheduled. Accordingly, we define:
\begin{itemize}
    \item For each job $j$, the {\em value density} of $j$ is $\densj = v_j/|I_j|$.
    \item For each item $t$, the {\em fractional value at $t$} is $\fvt(x) =
      \sum_{j : I_j \ni t} \densj x_j$.  Note $\fval(x) = \sum_t\fvt(x)$. We drop the argument $x$ when it is clear from the context.
    \item For any set of items $I$ (that may or may not be an interval), define $\fvI = \sum_{t \in I}\fvt$.
\end{itemize}

In this step, we remove from consideration jobs with fractional value less than half the total fractional value of their interval. In particular, let $U_1 = \{j : \valj\ge\half \fvI[\intj] \}$. The following lemma shows that we do not lose too much value in doing so.
\begin{lemma}
    \label{lem:small-v-bound}
%    Let $J = \{j : \valj \ge \half\fvI[\intj](x)\}$. Then $\fval(x_J) \ge \half \fval(x)$.
    For any fractional allocation $x$ and the set of jobs $U_1$ as
    defined above, we have \[\fval(x_{U_1})\ge \half\fval(x).\]
\end{lemma}
\noindent
The proof of Lemma~\ref{lem:small-v-bound} is a simple counting
argument and we defer it to \Cref{sec:deferred}.


\subsubsection*{Step 2: Bucketing}
\label{sec:bucketing}

In the rest of the argument, we will group jobs according to both their value
and the length of their interval. Let $\ell\in\{1, \cdots, \lceil\log
L\rceil+1\}$ denote a length scale, and let $a\in\{1, \cdots, \lceil \log \vmax
\rceil\}$ denote a value scale. We will further partition jobs of similar
length across non-overlapping intervals. Accordingly, let $\offset$ be an
offset picked u.a.r. from $[2^{\lmax}]$ where $\lmax = \lceil\log L\rceil+1$.
The partition corresponding to length scale $\ell$ then consists of
length-$2^{\ell}$ intervals $\{\Int_{\ell,k}\}_{k\in\Z}$ where
\[
    \Int_{\ell,k}:=[\offset+k 2^{\ell}+1, \offset+(k+1)2^{\ell}].
\]
See \Cref{fig:bucketing}.

We are now ready to define job groups formally. Group $G_{\ell, k,a}$ consists of every job $j$ of length between $2^{\ell-2}$ and $2^{\ell-1}$ and value between $2^{a-1}$ and $2^a$ whose interval lies within the $k$th interval at length scale $\ell$: $I_j\subseteq \Int_{\ell,k}$. We say that the jobs in $G_{\ell, k,a}$ are assigned to interval $\Int_{\ell,k}$.
\[
    G_{\ell, k,a} := \{j: \lceil \log |I_j| \rceil = \ell-1 \quad\Land\quad
            \lceil \log v_j \rceil = a \quad\Land\quad
            I_j\subseteq \Int_{\ell,k} \}
\]

\noindent
$G_{\ell, k}$ denotes all the jobs ``assigned'' to $\Int_{\ell,k}$: $G_{\ell,k}
= \Union_a G_{\ell,k,a}$. 

Observe that our choice of the offset $\offset$ may cause us to drop some jobs,
in particular those that do not fit neatly into one of the intervals in the
partition corresponding to the relevant length scale. Let $U_2 =
\Union_{\ell,k,a} G_{\ell,k,a}$. The next lemma bounds the loss in value at
this step.

\begin{lemma}
    \label{lem:hierarchy-alignment}
    For any fractional allocation $x$ and with $U_1$ and $U_2$ defined as above, $\fval(x_{U_2})\ge \half \fval(x_{U_1})$.
\end{lemma}
\begin{proof}
  Recall that $\offset$ is chosen u.a.r. from $2^\lmax$. Moreover, the length of any job $j$ is at most half the length of the intervals at the scale at which $j$ is considered for bucketing. Therefore, $j$ survives with probability at least $1/2$.
\end{proof}

\begin{figure}
    \centering
    \newcommand{\cellwidth}{0.5 cm}
    \newcommand{\cellheight}{0.8 cm}

    \begin{tikzpicture}
        [x=\cellwidth,y=-\cellheight,node distance=0 cm,outer sep=0 pt,line
        width=0.66pt]

        \tikzstyle{cell}=[rectangle,draw,
            minimum height=\cellheight,
            anchor=north west,
            text centered]
        \tikzstyle{w16}=[cell,minimum width=16*\cellwidth]
        \tikzstyle{w8}=[cell,minimum width=8*\cellwidth]
        \tikzstyle{w4}=[cell,minimum width=4*\cellwidth]
        \tikzstyle{w2}=[cell,minimum width=2*\cellwidth]
        \tikzstyle{w1}=[cell,minimum width=1*\cellwidth]
        \tikzstyle{job}=[rectangle,draw,fill=gray!30,anchor=north west]

        % top rows
        \node[anchor=south] at (0, 0) {$\offset + 1$};
        \node[anchor=south] at (16,0) {$\offset + \lmax + 1$};
        \node[w16] at (0, 0) {$\Int_{\lmax,0}$};
        \node[w8]  at (0, 1) {$\Int_{\lmax-1,0}$};
        \node[w8]  at (8, 1) {$\Int_{\lmax-1,1}$};
        \foreach \k in {0,1,2,3} {
            \node[w4] at (4*\k, 2) {$\Int_{\lmax-2,\k}$};
        }

        % horizontal continuation...
        \foreach \y in {0,1,2,3, 4.5,5.5} {
            \draw[dotted] (0, \y) -- (-1,\y);
            \draw[dotted] (16,\y) -- (17,\y);
        }

        % gap
        \foreach \x in {0,4,...,16} {
            \draw[dotted] (\x,3) -- (\x,3.5);
            \draw[dotted] (\x,4.5) -- (\x,4);
        }

        % bottom row
        \foreach \x in {0,...,15} {
            \node[w1] at (\x, 4.5) {};
        }

        % jobs
        \node[job,minimum width=8*\cellwidth] at (   0,6) {$j_1$};
        \node[job,minimum width=6*\cellwidth] at ( 3.5,7) {$j_2$};
        \node[job,minimum width=2*\cellwidth] at (11.5,6) {$j_3$};
    \end{tikzpicture}
    \caption{\small{Bucketing jobs by length as in \hyperref[sec:bucketing]{Step 2}.
    Jobs $j_1$ and $j_2$ have length scale $2^\lmax$ (i.e., lengths between
    $2^{\lmax-2}$, exclusive, and $2^{\lmax-1}$, inclusive) and will therefore
    be assigned to bucket $\Int_{\lmax,0}$. Job $j_3$, however, will be
    dropped; it has length scale $\lmax-2$, but does not fit entirely within
    any bucket at that scale.}}
    \label{fig:bucketing}
\end{figure}

\subsubsection*{Step 3: Classification into heavy-weight and light-weight sets of jobs}

We first discuss the intuition behind Steps 3 and 4. Consider a single
group $G_{\ell,k,a}$. Let us assume briefly, for simplicity, that all
jobs in this group have the same value $v$. Recall that all of the
jobs $j\in G_{\ell,k,a}$ by virtue of being in $U_1$ satisfy the
property that $\fvI[\intj]\le 2\valj=2v$. Furthermore, since every job
$j\in G_{\ell,k,a}$ is of size at least $2^{\ell-2}$ whereas the
interval covered by $G_{\ell,k,a}$, namely $\Int_{\ell,k}$, is of size
$2^\ell$, the total fractional value of $\Int_{\ell,k}$ is\footnote{To
  be precise, the total value of $\Union_{j\in G_{\ell,k,a}} \intj$ in
  this case is at most $8v$.} comparable to $v$.  Therefore, as long
as the jobs in the group $G_{\ell,k,a}$ have sufficient total
fractional weight, this group of jobs alone would recover (a constant
fraction of) the fractional value of the interval $\Int_{\ell,k}$. In
other words, we could then immediately throw away jobs in other groups
(corresponding to other length or value scales) that overlap with this
interval. Step 3 filters out such ``heavy-weight'' groups of jobs and
in Step 4 we argue that these groups form a good unit
allocation. Intervals that do not have any length scales with
heavy-weight groups of jobs are relegated to Step~5.

We now present the details of this argument. First, for every group
$G_{\ell,k,a}$, we break the group up into contiguous
components. Observe, in particular, that the set of items covered by
jobs in $G_{\ell,k,a}$, namely $\Union_{j\in G_{\ell,k,a}} \intj$, may
not be an interval itself but is composed of at most three disjoint
intervals because each contiguous component has length at least a
quarter of $|\Int_{\ell,k}|$. Consider each of the at most three
corresponding sets of overlapping jobs. We will use
$G_{\ell,k,a}^{(i)}$ for $i\in \{1,2,3\}$ to denote these sets. For
any such set $G$ of overlapping jobs, let
$\Int_G = \Union_{j\in G} \intj$ denote the interval of items it
covers.

Next, we classify these sets of jobs into heavy-weight or light-weight. In the following definition, we hide the arguments $\ell,k,a,i$ to simplify notation.
\[\heavy := \{G : \fwt(x_G) \ge 1/6\beta\} \quad \quad \light := \{G : \fwt(x_G) < 1/6\beta\} \]

% For any group $G$ (with the arguments being implicit), if we have $\fwt(x_G) \ge 1/6\beta$, call the set $G$ heavy-weight; let $\heavy$ be the collection of all such sets. Otherwise, call the set light-weight, and let $\light$ be the collection of all such sets. 
\noindent
The following fact is immediate:

\begin{fact}
  \label{fact:heavy-light-classification}
%  $\fvI[\cup_{G\in\heavy}\Int_G] + \sum_{G\in\light} \fval(x_G) \ge $
  $\sum_{G\in\heavy}\fval(x_G) + \sum_{G\in\light} \fval(x_G) \ge \fval(x_{U_2})$. 
%  $\fvI[\cup_{G\in\heavy}\Int_G] + \fvI[\cup_{G\in\light}\Int_G]  \ge \fval(x_{U_2})$. 
\end{fact}

We will now proceed to construct two fractional unit allocations, one of which
extracts the value of the heavy-weight sets (see Step 4), and the other
extracts the value of the light-weight sets (see Step~5).

\subsubsection*{Step 4: Extracting the value of heavy-weight intervals}

Consider any heavy-weight set $G\in\heavy$ (with the arguments $\ell$, $a$,
etc. being implicit). The following lemma is implicit in the discussion above
and follows from the observations that (1) all jobs in $G$ are high-value jobs
(that is, they belong to $U_1$); (2) they have roughly the same value (within a
factor of $2$); (3) the interval $\Int_G$ can be covered by at most $6$ such
jobs; and (4) the total weight of $G$, $\fwt(x_G)$, is at most $1/6\beta$ by
the definition of heavy-weight sets. 
\begin{lemma}
    \label{lem:heavy-Gs}
    For all $j\in G$, $\valj \ge \frac 1{12} \fvI[\Int_G]$, and therefore
    $\fval(x_G)\ge \frac 1{72\beta} \fvI[\Int_G]$.
\end{lemma}

There are two remaining issues in going from the allocation $x_{\Union_\heavy
G}$ to a unit-allocation. First, the intervals $\{\Int_G\}_{G\in\heavy}$
overlap, and second, the fractional weight of $G$ can be larger than $1$. The
second issue can be dealt with by rescaling $x$ by an appropriate factor. To
deal with the first, we state the following simple lemma without proof.
\begin{lemma}
    \label{lem:interval-cover}
    For any given collection $\mathcal C$ of intervals, one can efficiently
    construct two sets $S_1, S_2\subseteq \mathcal C$ such that 
    \begin{itemize}
        \item $S_1$ (and likewise $S_2$) is composed of disjoint intervals;
            that is, $I\cap I' = \emptyset$ for all $I\ne I'\in S_1$.
        \item Together they cover the entire collection $\mathcal C$; that is,
            $\Union_{I\in S_1\union S_2} I = \Union_{I\in\mathcal C} I$.
    \end{itemize}
\end{lemma}

We can now put these lemmas together to construct a unit allocation that covers the fractional value of heavy-weight intervals.

\begin{lemma}
  \label{lem:heavy-bound}
  There exists a fractional unit allocation $\tilde{x}_\heavy$ such that 
\[\sum_{G\in\heavy}\fval(x_G)\le O(\beta)\fval(\tilde{x}_\heavy).\]
\end{lemma}
\begin{proof}
%\scnote{Defer}
    Apply Lemma~\ref{lem:interval-cover} to the collection
    $\{\Int_G\}_{G\in\heavy}$ to obtain sets $S_1$ and $S_2$. We think of $S_1$
    and $S_2$ as sets of groups in $\heavy$. Assume without loss of generality
    that $S_1$ has larger fractional value than $S_2$. Let $\tilde{x}_\heavy$
    be the fractional allocation $x_{\Union_{G\in S_1} G}$ scaled down by a
    factor of $4$. Then we have:
%, it is immediate from Lemmas~\ref{lem:heavy-Gs} and
%    \ref{lem:interval-cover} that 
\begin{align*}
    \fval(\tilde{x}_\heavy) = \frac 14 \fval(x, \union_{G\in S_1} G) & \ge
            \frac 1{O(\beta)} \sum_{G\in S_1} \fvI[\Int_G] \\
    & \ge \frac 1{O(\beta)} \fvI[\union_{G\in\heavy}\Int_G]\ge \frac 1{O(\beta)} \sum_{G\in\heavy}\fval(x_G).
\end{align*}
Here the first inequality follows from applying \Cref{lem:heavy-Gs} to
every $G\in S_1$, and the fact that intervals in $S_1$ are
disjoint. The second inequality follows by recalling that the
intervals in $S_1$ and $S_2$ together cover the
collection$\{\Int_G\}_{G\in\heavy}$, and so,
$\fvI[\union_{G\in S_1}\Int_G]\ge
\half\fvI[\union_{G\in\heavy}\Int_G]$.

It remains to show that $\tilde{x}_\heavy$ is a unit allocation. To
see this, consider the partition of jobs into groups $G$ in $S_1$.
The corresponding collection of intervals
$\{\Int_G\}_{G\in S_1}$ forms a partition of the items by virtue of
the fact that the intervals corresponding to groups in $S_1$ are
disjoint. It remains to argue that $\fwt(\tilde{x}_\heavy, G)\le 1$,
that is, $\fwt(x_G)\le 4$ for all $G\in\heavy$. 

To prove this claim, observe that since each job in $G$ has length
at least a quarter of the length of $\Int_G$, we can find up to four
items in $\Int_G$ such that each job in $G$ contains at least one of
the four items in its interval. Since $x$ is a feasible fractional
allocation, the total weight of all jobs containing any one of those
items is at most $1$, and therefore the total weight of jobs in
$G$ altogether is at most~$4$.
\end{proof}

\subsubsection*{Step 5: Extracting the value of light-weight intervals}

We now consider the light-weight groups $G_{\ell,k,a}^{(i)}$. As discussed at the beginning of this section, in order to obtain a good approximation from these sets, we must construct a unit allocation out of jobs at multiple length scales.

Define $\Gtilde_{\ell,k,a} = \Union \{G_{\ell,k,a}^{(i)}\in\light\}$
to be the set of all jobs in $\Int_{\ell,k}$ with value scale $a$ that
belong to light-weight groups. Let $\Gtilde_{\ell,k}=\Union_a \Gtilde_{\ell,k,a}$
denote all light-weight jobs assigned to $\Int_{\ell,k}$.
Since each individual group
$G_{\ell,k,a}^{(i)}$ in $\light$ has total weight at most $1/6\beta$,
we have that $\fwt(\Gtilde_{\ell,k,a})\le 1/2\beta$. In order to
obtain a partition of jobs and items, we would now like to associate a
single set of jobs with each partition $\Int_{\ell,k}$ that is
simultaneously high value and low weight. Unfortunately, the set
$\Gtilde_{\ell,k}$ may have very large total weight since
it combines together many low weight sets. We use the fact that the
values of jobs in these low weight sets increase geometrically to
argue that it is possible to extract a subset of jobs from
$\Gtilde_{\ell,k}$ that is both light-weight (i.e. has
total weight at most $1/\beta$) and captures a large fraction of the
total value in $\Gtilde_{\ell,k}$.

\begin{lemma}
    \label{lem:a-cell-bound}
    For each interval $\Int_{\ell,k}$, there exists a set of jobs
    $S_{\ell,k} \subseteq \Gtilde_{\ell,k}$ such that
    \begin{enumerate}[label=\roman*., leftmargin=2\parindent]
        \item $\fwt(x, S_{\ell,k}) \le \frac1\beta$ and
        \item $\fval(x, S_{\ell,k}) \ge \frac16 \fval(x, \Gtilde_{\ell,k})$.
    \end{enumerate}
\end{lemma}

% \begin{proof}
%     Since all jobs in $\Gtilde_{\ell,k,a}$ have value between $2^{a-1}$ and $2^a$, we have
%     \[
%         2^{a-1}\fwt(x, \Gtilde_{\ell,k,a}) \leq\fval(x,\Gtilde_{\ell,k,a}) \leq
%             2^a\fwt(x, \Gtilde_{\ell,k,a}).
%     \]
%     For every $u\in\{1,\cdots,\lceil\log \vmax\rceil\}$, define
%     $S_u=\sum_{a\leq u}2^a\fwt(x,\Gtilde_{\ell,k,a})$.  Then $S_{\lceil\log
%     \vmax\rceil}$ is an upperbound of $\fval(x, {\Union_a
%     \Gtilde_{\ell,k,a}})$.  Let $m$ be such that $S_m\leq
%     \frac{1}{3}S_{\lceil\log \vmax\rceil}$ and
%     $S_{m+1}>\frac{1}{3}S_{\lceil\log \vmax\rceil}$.  Consider the following
%     two cases.

%     \smallskip
%     {\em Case 1.} If $S_{m+1}-S_m = 2^{m+1}\fwt(x,\Gtilde_{\ell,k,m+1}) >
%     \frac{1}{3}S_{\lceil\log \vmax\rceil}$, then
%     \begin{align*}
%         \fval(x,\Gtilde_{\ell,k,m+1}) &\geq 2^{m}\fwt(x,\Gtilde_{\ell,k,m+1}) \\
%             &> \frac{1}{6}S_{\lceil\log \vmax\rceil} \\
%             &\geq\frac16 \fval(x,\union_a \Gtilde_{\ell,k,a}).
%     \end{align*}
%     Since $\fwt(x, \Gtilde_{\ell,k,m+1})\leq \frac{1}{2\beta}<\frac{1}{\beta}$,
%     in this case setting $\Gtilde_{\ell,k}=\Gtilde_{\ell,k,m+1}$ satisfies both
%     conditions of the lemma.

%     \smallskip
%     {\em Case 2.} If $S_{m+1}-S_m\leq \frac{1}{3}S_{\lceil\log \vmax\rceil}$,
%     then $S_{m+1}\leq\frac{2}{3}S_{\lceil\log \vmax\rceil}$, and
%     \begin{align*}
%         \sum_{a>m+1}\fval(x,\Gtilde_{\ell,k,a}) &\geq
%                 \frac{1}{2}(S_{\lceil\log \vmax\rceil}-S_{m+1}) \\ 
%         &\geq \frac{1}{6}S_{\lceil\log\vmax\rceil} \\
%         &\geq \frac16\fval(x,\union_a \Gtilde_{\ell,k,a}).
%     \end{align*}
%     Meanwhile, note that
%     $\sum_{a>m+1}2^a\fwt(x,\Gtilde_{\ell,k,a}) < \frac{2}{3}S_{\lceil\log
%     \vmax\rceil} < 2S_{m+1}.$ Then
%     \begin{align*}
%         \sum_{a>m+1}\fwt(x,\Gtilde_{\ell,k,a}) &\leq
%                 \frac{1}{2^{m+2}}\sum_{a>m+1} 2^a\fwt(x,\Gtilde_{\ell,k,a}) \\
%         &< \frac{1}{2^{m+2}}\cdot2S_{m+1} \\
%         &= \frac{1}{2^{u+1}}\sum_{a\leq m+1}2^a\fwt(x,\Gtilde_{\ell,k,a}) \\
%         &\leq \frac{1}{2^{u+1}}\sum_{a\leq m+1}2^a\frac{1}{2\beta} < \frac{1}{\beta}.
%     \end{align*}
%     Thus setting $\Gtilde_{\ell,k}=\Union_{a>m+1}\Gtilde_{\ell,k,a}$ satisfies
%     both conditions of the lemma.
% \end{proof}

We defer the proof of Lemma~\ref{lem:a-cell-bound} to
\Cref{sec:deferred}. The remainder of our analysis then hinges on the fact that if we consider the partition into intervals at some length scale $\ell$, namely $\{\Int_{\ell,k}\}_{k\in\Z}$, and consider for every interval in this partition the set of all jobs in this interval at the $\log \beta$ length scales below $\ell$, the total fractional weight of these jobs is at most $1$. We therefore obtain a unit allocation while capturing the fractional value in $\log\beta$ consecutive scales.

\begin{lemma}
    \label{lem:light-bound}
    There exists a fractional unit allocation $\tilde{x}_\light$ such that 
    \[
        \sum_{G\in\light} \fval(x_G) \le
        O\left(\frac{\log L}{\log\beta}\right)\fval(\tilde{x}_\light).
    \]
\end{lemma}

We defer the proof of Lemma~\ref{lem:light-bound} to \Cref{sec:deferred}.
% \begin{proof}
%   \scnote{TBA, defer} \ytnote{Notation too messy, need clean up}
%   Notice that for any interval $\Int_{\ell,k}$ of length scale $\ell$, there are $2^{\log\beta}-1<\beta$ intervals of length scale between 
%   $\ell-\log\beta+1$ and $\ell$ that is contained $\Int_{\ell,k}$. Consider the following set of jobs
%   \begin{equation*}
%   H_{\ell,k}=\Union_{\ell',k':\ell-\log\beta+1\leq \ell'\leq \ell, \Int_{\ell',k'}\subseteq \Int_{\ell,k}}\Gtilde_{\ell',k'},
%   \end{equation*}
%   here $\Gtilde_{\ell',k'}$ are defined in Lemma~\ref{lem:a-cell-bound}. More intuitively, $H_{\ell,k}$ is the set of light-weighted jobs 
%   selected by Lemma~\ref{lem:a-cell-bound} from intervals within $\log \beta$ scales from it, while the total fractional weight of the jobs is less than 1.
%    Then $\fwt[H_{\ell,k}]\leq 1$, while 
%   \begin{equation*}
%   \fval[x_{H_{\ell,k}}]\geq\frac{1}{6}\sum_{\ell',k':\ell-\log\beta+1\leq \ell'\leq \ell, \Int_{\ell',k'}\subseteq \Int_{\ell,k}}\fval[x_{\light\cap G_{\ell',k'}}],
%   \end{equation*}
%   i.e. jobs in $H_{\ell,k}$ provide $\beta$-approximation of fraction value of light-weight jobs from intervals within $\log \beta$ scales from
%   $\Int_{\ell,k}$. Let $H_{\ell}=\Union_k H_{\ell,k}$. Then $x_{H_{\ell}}$ is a fractional unit allocation (with respect to intervals $\Int_{\ell,k}$ for every $k$), with fractional value being a good approximation of all light-weight intervals within length scale $(\ell-\log\beta,\ell]$:
%   \begin{equation*}
%   \fval[x_{H_{\ell}}]\geq\frac{1}{6}\sum_{\ell',k':\ell-\log\beta+1\leq \ell'\leq \ell}\fval[x_{\light\cap G_{\ell',k'}}].
%   \end{equation*}
%   Since there are $O(\log L)$ length scales in total, there must exist an $\ell$ such that the fractional value of all light-weight intervals within length scale $(\ell-\log\beta,\ell]$ is $\Omega\left(\frac{\log \beta}{\log L}\right)$ of $\sum_{G\in\light}\fval(x_G)$. Let
%   $\tilde{x}_\light=x_{H_{\ell}}$, then $\tilde{x}_\light$ is a fractional unit allocation, while \\$\sum_{G\in\light}\fval(x_G)\le O\left(\frac{\log L}{\log\beta}\right)\fval(\tilde{x}_\light)$.
% \end{proof}


\subsubsection*{Putting everything together}

Combining Lemmas~\ref{lem:small-v-bound} and \ref{lem:hierarchy-alignment},
Fact~\ref{fact:heavy-light-classification}, and Lemmas~\ref{lem:heavy-bound}
and \ref{lem:light-bound}, we get that the better of the two unit
allocations $\tilde{x}_\light$ and $\tilde{x}_\heavy$ provides an $O(\beta)$
approximation to the total fractional value of $x$, where we used the fact that
$\log L/\log \beta = \beta$.

\begin{theorem}
\label{thm:unit-capacity}
    For every fractional allocation $x$ for the unit-capacity setting, there
    exists a fractional unit allocation $x'$ such that
    \[
        \fval(x)\le  O(\log L/\log\log L)\fval(x').
    \]
\end{theorem}

