\section{Lower Bounds}
\label{sec:lowerbound}

We now turn to lower bounds. In this section, we ignore incentive
constraints and bound the social welfare that can be achieved by {\em
  any} online algorithm for the online resource allocation problem.
We consider the case of interval preferences in
Section~\ref{sec:lb-int} and path preferences in
Section~\ref{sec:pathlowerbound}.

\subsection{Lower bound for interval preferences}
\label{sec:lb-int}

We focus on the special case where every buyer is single minded. That
is, for all $i$ and $\vali\sim\disti$, there exists an interval $I$
and a scalar $v$ such that $\vali(I')=v$ for all $I'\supseteq I$ and
$0$ otherwise. This setting is also called {\em online interval
  scheduling}. A lower bound for the unit-capacity version of this
problem was previously developed by \citet{im2011secretary}. Although
\citeauthor{im2011secretary}'s focus was on the secretary-problem
version (i.e. with random arrival order and adversarial values rather
than worst-case arrival order and stochastic values), their lower
bound construction uses jobs drawn independently from a fixed
distribution, and it thus provides a lower bound for the Bayesian
adversarial order setting as well. We restate
\citeauthor{im2011secretary}'s result; our proof of
Theorem~\ref{thm:lb} builds upon this result.

    \begin{lemma}\label{ISSPlemma}
      (Restated from Theorem 1.3 in \cite{im2011secretary}) For the
      unit-capacity setting of the online interval scheduling problem,
      every randomized online algorithm has a competitive ratio of
      $\Omega\left(\frac{\log L}{\log\log L}\right)$.
    \end{lemma}

\begin{numberedtheorem}{\ref{thm:lb}}
  For the online interval scheduling problem, every randomized online
  algorithm has approximation ratio
  $\Omega\left(\frac{\log L}{B\log\log L}\right)$, where
  $B=\min_{t}B_t$.
\end{numberedtheorem}


    % For the case $B=1$, the algorithmic problem we study is exacty the
    % prophet-inequality version of the interval-scheduling problem studied by
    % \citet{im2011secretary}.  Although they studied the secretary-problem
    % version (i.e. random arrival order rather than worst-case), their lower
    % bound construction uses jobs drawn independently from a fixed distribution,
    % and it thus provides a lower bound for the prophet-inequality setting as
    % well.


%They proposed the following construction. Assume that there are $h$ level of nodes
%in an $h^2$-ary tree, each node in level $\ell$ represents 
%an interval of size $h^{2(h-\ell)}$: the set of intervals in level $\ell$ is
%$\{\left((k-1)h^{2(h-\ell)},kh^{2(h-\ell)}\right]:k\in[h^{2(\ell-1)}]\}$. At each
%time, a single-minded job arrives aiming at one of the $H=\sum_{\ell\in[h]}h^{2(\ell-1)}$
%intervals in the tree, with value the same as the length of the interval. The authors
%proved that when $\frac{H}{h}$ jobs come in total, optimal offline allocation 
%can get welfare $\Theta(h^{2(h-1)})$, while any randomized online algorithm can only
%achieve welfare $O(h^{2h-3})$. Thus the approximation ratio is 
%$\Omega(h)=\Omega\left(\frac{\log L}{\log\log L}\right)$, where $L=h^{2(h-1)}$ is the 
%length of longest interval in the tree.

\begin{proof}
  We argue that if there exists an online algorithm with competitive 
  ratio $\alpha=o\left(\frac{\log L}{B\log\log L}\right)$ for the setting with minimum
  capacity $B$, then we can construct an online algorithm for the
  unit-capacity setting with competitive ratio $O(B\alpha) = o\left(\frac{\log
    L}{\log\log L}\right)$, contradicting Lemma~\ref{ISSPlemma} above.

  Our main tool is the following lemma which shows that, for
    any set of intervals, if the intervals arrive in arbitrary order, there
    exists an online algorithm that colors them (i.e. such that no two
    overlapping intervals have the same color) using not many more colors than
    optimal.

    \begin{lemma}\label{coloringlemma}
        (Restated from Theorem 5 in \cite{kierstead1981extremal}) For any set
        of intervals $U$ such that every point belongs to at most $B$ intervals
        in $U$, there exists an online coloring algorithm $\coloralg$ utilizing
        at most $3B-2$ colors. 
    \end{lemma}

    Suppose that for some $B>1$ there exists an online algorithm $\alg$ for
    markets of size $B$ with competitive ratio $\alpha=o\big(\frac{\log L}{B\log\log
    L}\big)$. Consider the following algorithm $\alg'$ for the unit-capacity
    setting. Let $c$ be an integer chosen u.a.r. from $[3B-2]$. As each buyer $i$
    arrives, if $\alg$ accepts $i$, then input $I_i$ to $\coloralg$; if
    $\coloralg$ assigns color $c$ to $I_i$, then accept $i$.

    % \begin{algorithm}[t]
    %     \SetAlgoNoLine
    %     \KwIn{Set of jobs $U$}
    %     Let $c$ be an random integer chosen uniformly at random from 1 to $3B-2$\;
    %     Run $\alg$ on $U$ where the number of copies of each item is set to be
    %     $B_t=B$\;
    %     \For{each job $j$ that arrives
    %     }{
    %         \If{$j$ is accepted by $\alg$ and allocated interval $I_j$}{
    %             input $I_j$ to online interval coloring algorithm $\coloralg$\;
    %             \If{$j$ is assigned to color $c$}{
    %                 $\alg'$ accepts $j$ and allocate $I_j$ to it
    %             }
    %         }
    %         Otherwise reject $j$\;
    %     }
    %     \caption{$\alg'$ for $B_t=1$}
    %     \label{euclid}
    % \end{algorithm}

    % Now we prove that $\alg'$ has approximation ratio $o\big(\frac{\log
    % L}{B\log\log L}\big)$.
    % For any instantiation of jobs, by Lemma
    % \ref{coloringlemma} it is successfully colored with one of the $3B-2$
    % colors.
    By Lemma~\ref{coloringlemma}, the expected value of intervals with color
    $c$ is a $\frac{1}{3B-2}$-fraction of the social welfare obtained by
    $\alg$. Thus the approximation ratio of $\alg'$ for the unit-capacity
    setting is $(3B-2)\alpha = o\big(\frac{\log L}{\log\log L}\big)$, which contradicts
    Lemma~\ref{ISSPlemma}.
    % the inapproximability result of $B=1$.
\end{proof}

\subsection{Lower bound for path preferences}
\label{sec:pathlowerbound}
In this section, we show that for the ``online path scheduling''
problem on trees no online algorithm can achieve subpolynomial
approximation with respect to $L$, the length of the longest possible
path in the instance. 

\begin{theorem}
    \label{thm:treelowerbound}
    There exists an instance of the path preferences setting, where
    every buyer desires a path of length between $1$ and $L$, for
    which every randomized online algorithm has competitive ratio
    $\Omega\left(\sqrt{\frac{L}{\log L}}\right)$ for social welfare.
    % For the path interval scheduling problem on trees, every randomized
    % online algorithm has approximation ratio
    % $\Omega\left(\sqrt{\frac{L}{\log L}}\right)$.
\end{theorem}

\begin{proof}
  Consider a complete binary tree with height $L$ and capacity $1$ on
  every edge in the tree. We will define an instance with a unique
  buyer for every (node, leaf) pair in the tree where the leaf belongs
  to the subtree rooted at the node. Formally, define the level of
  edges bottom-up: for example, the leaf edges (i.e. edges adjacent to
  a leaf node) have level 0, and the edges adjacent to root have level
  $L-1$. There are $n_\ell=2^{L-\ell}$ edges at level $\ell$, and each
  of these edges has $2^{\ell}$ leaves in its subtree. For every level
  $\ell$, every edge $e$ at level $\ell$, and every leaf edge $e'$ in
  the subtree rooted at $e$, our instance contains a unique buyer that
  arrives independently with probability $\frac{1}{2^\ell}$ and has
  value $v_\ell=2^\ell$. We say that the buyer is at level
  $\ell$. Note that there are exactly $2^L$ distinct buyers at each
  level $\ell$, and therefore $L2^L$ buyers in all.

% Say a
%   path ``begins with'' an edge $e$, if $e$ is the only peak edge of
%   the path. For each inner edge $e$ with level $\ell$ in the tree,
%   there are $2^\ell$ distinct paths that goes down to a leaf node
%   beginning with $e$. For every such path, there is a buyer with value
%   $v_\ell=2^\ell$ that demands only this path and arrives
%   independently with probability $\frac{1}{2^\ell}$. Say a buyer is at
%   level $\ell$, if it begins with an edge at level $\ell$.

  Let us now compute the offline optimal welfare $\opt$ for this
  instance. We claim $\opt=\Omega(L2^L)$. Consider a greedy allocation
  that admits buyers in order from the highest to the lowest level:
  every buyer whose path is still available when considered gets
  allocated with probability $1/2$. Note that when a level $\ell$ path
  beginning with edge $e$ is considered, it can be allocated if and
  only if $e$ is not allocated to previous buyers. We now make two
  observations. First, the expected number of buyers at level
  $\ell'>\ell$ whose paths contain $e$ and that arrive is at most
  $\sum_{\ell'>\ell} 2^{\ell}/2^{\ell'} < 1$. So, the probability that
  an edge $e$ is allocated prior to considering buyers at its level is
  at most $1/2$. Second, if prior to considering level $\ell$ buyers,
  edge $e$ is available, then the probability that it gets allocated
  to a level $\ell$ buyer is at least
  $1-(1-\frac{1}{2^{\ell+1}})^{2^\ell}\geq 1-\frac{1}{\sqrt{\textrm{e}}}$. Therefore, the total
  contribution of level $\ell$ buyers to the social welfare is at
  least $\frac{1}{2}\left(1-\frac{1}{\sqrt{\textrm{e}}}\right)n_\ell v_\ell=\Omega(2^L)$. This proves our claim.

% Before buyers at level $\ell$ arrive, at most half
%   of the edges are allocated. For each unallocated edge $e$ at level
%   $\ell$, since there are $2^\ell$ buyers beginning with it arriving
%   and each with probability $\frac{1}{2^\ell}$, then with probability
%   $1-(1-\frac{1}{2^\ell})^{2^\ell}\geq 1-\frac{1}{e}$ one buyer
%   beginning with $e$ will get allocated. Thus the total expected
%   welfare contributed from buyers at level $\ell$ is at least
%   $\frac{1}{2}\left(1-\frac{1}{e}\right)n_\ell
%   v_\ell=\Omega(2^L)$. Therefore the total welfare achieved by this
%   greedy algorithm is $\Omega(L2^L)$, thus $\opt=\Omega(L2^L)$.

  Next we prove via a charging argument that the welfare of any online
  algorithm is bounded by $O(2^L\sqrt{L\log L})$. Assume that 
  buyers arrive in order from lowest level to highest level.
  % , and consider any online
  % algorithm. Say a job ``covers'' a leaf edge $e'$, if its beginning
  % edge $e$ is on the path from $e'$ to the root. Note that if a job
  % covering a leaf edge $e'$ is admitted, then no job at a higher level
  % whose path contains $e$ can be admitted. Each job at level $\ell$
  % covers exactly $2^{\ell}=v_\ell$ leaf edges. When a buyer beginning
  % with edge $e$ at level $\ell$ arrives, it can be accepted with
  % probability at most $\frac{m(e)}{2^\ell}$, where $m(e)$ is the
  % number of leaf edges below $e$ left uncovered by previous allocated
  % jobs. 
  Construct a grid with height $L$ and width $2^L$ as follows. Each
  edge at level $\ell$ corresponds to a size $1\times 2^\ell$
  rectangle in the grid at the same level; see Figure
  \ref{fig:lb-example}. With each leaf edge, we also associate the
  entire column above its cell in the grid with the edge. So each cell
  in the grid is indexed by a level and a column corresponding to a
  leaf edge. As buyers arrive and the online algorithm makes
  allocation decisions, we mark cells in the grid to indicate
  availability. Initially all cells are unmarked. Whenever the online
  algorithm allocates a path to a buyer at level $\ell$ with first
  edge $e$, we mark all cells in the $1\times 2^\ell$ rectangle
  corresponding to edge $e$, as well as all cells above this
  rectangle. Now suppose that a buyer at level $\ell$ arrives whose
  path begins with edge $e$ and ends at leaf edge $e'$. If the cell
  corresponding to column $e'$ and level $\ell$ in the grid is already
  marked, this means the buyer cannot be legally allocated because
  a buyer starting at some other edge on the path from $e$ to $e'$ was
  previously allocated. Let $m(e)$ be the number of unmarked cells 
  in the rectangle corresponding to $e$ when buyers beginning with $e$
  start to arrive.
  If the online algorithm allocates to a buyer
  with starting edge $e$, the total number of cells that get marked at
  this iteration is exactly $(L-\ell)m(e)$.

    \begin{figure}[htbp]
        \centering
        \subfigure{
        \begin{minipage}[b]{0.8\textwidth}
            \includegraphics[width=0.48\textwidth]{lb_tree1}
            \includegraphics[width=0.48\textwidth]{lb_grid1}
            \center{(a)}
 %           \caption{\label{fig:lb-example}\small{An example of a binary tree and its
 %             corresponding grid.}}
        \end{minipage}
        }\\
        \subfigure{
        \begin{minipage}[b]{0.8\textwidth}
            \includegraphics[width=0.48\textwidth]{lb_tree2}
            \includegraphics[width=0.48\textwidth]{lb_grid2}
            \center{(b)}
%            \caption{\label{fig:lb-example-2}\small{Jobs arrive bottom-up and get allocated. The
%              grid shows the cells marked by each job.}}
        \end{minipage}
        }
        \caption{\label{fig:lb-example} \small{(a) An example of a binary tree and its
              corresponding grid; (b) Jobs arrive bottom-up and get allocated. The
              grid shows the cells marked by each job.}}
    \end{figure}

    Say an edge $e$ at level $\ell$ is ``good'' if when the buyers beginning with $e$
    start to arrive, we have $(L-\ell)m(e)\geq 2^\ell\sqrt{L/\log
      L}$. For any buyer beginning with a good edge that is allocated
    by the algorithm, it marks a rectangle in the grid with area
    $(L-\ell)m(e)\geq \sqrt{L/\log L}\cdot v_\ell$, i.e
    $\sqrt{L/\log L}$ times the value of the buyer. Since the total
    area of the grid is $L2^L$, the total value of buyers with
    good edges allocated in the algorithm is at most
    $\frac{1}{\sqrt{L/\log L}}L2^L=2^L\sqrt{L\log L}$.

    Now consider the welfare contributed by buyers at bad edges. At
    any such bad edge $e$ at level $\ell$, at most $m(e)$ buyers
    beginning with $e$ can get allocated; the expected number of such
    buyers that arrive is at most
    $\frac{m(e)}{2^\ell}<\frac{\sqrt{L/\log L}}{L-\ell}$. Thus their
    total welfare contribution is at most
    \begin{equation*}
      \sum_{\ell=0}^{L-1}n_\ell v_\ell\cdot\frac{\sqrt{L/\log L}}{L-\ell}=2^L\sqrt{L/\log L}\sum_{\ell=0}^{L-1}\frac{1}{L-\ell}=O(2^L\sqrt{L\log L}).
    \end{equation*}

    Summing up the total welfare contribution from buyers at good and
    bad edges gives us a bound of $O(2^L\sqrt{L\log L})$ on the total
    welfare obtained by any online algorithm.
\end{proof}
