\section{Model and definitions}
\label{sec:prelim}

We consider a setting with $n$ buyers and a set of items $\items$. We
index buyers by $i$ and items by $t$ (or $e$ for edges). Buyer $i$'s
valuation function is denoted $\vali: 2^{\items} \rightarrow \Re^+$,
with $\vali(\emptyset)=0$. Our setting is Bayesian: $\vali$ is drawn
from a distribution $\disti[i]$ that is independent of other buyers
and known to the seller. We emphasize that values may be correlated
across bundles, but not across buyers. Let $B_t$ denote the number of
copies available, a.k.a.  supply or capacity, for item $t$. Let $B =
\min_t B_t$. The {\em unit-capacity} setting is a special case where
$B=1$. An allocation is an assignment of bundles of items to buyers
such that no item is allocated more than its number of copies
available. Our goal is to maximize the buyers' total welfare---that
is, the sum over values each buyer derives from his allocated bundle.

\paragraph{Jobs.} We now describe assumptions and notational
shortcuts for buyers' value functions that hold without loss of
generality and simplify exposition. First, we assume that values are
monotone: for all $i$ and $\vali$, and all bundles
$S\subset S'\subseteq \items$, $\vali(S)\le\vali(S')$. Second, we assume
that each buyer's value distribution is a discrete
distribution\footnote{For constructive versions of our results, we
  require the supports of the distributions to be explicitly given,
  however, our arguments about the existence of a good pricing work
  also for continuous distributions.} with 
  finite support. In other words, each buyer $i$ can only have
  finitely many possible valuation functions: with probability $\qvi$,
  buyer $i$'s values are given by the known function $\vali$.
  %a support of size $2$, with
  %one of the two value functions in the support being the all-zeros
  %function.\footnote{This can be enforced by ``splitting'' each buyer
  %into multiple buyers, one for each value function in the support of
  %the original buyer. The arrivals of these new buyers are negatively
  %correlated. It is easy to observe that revenue and welfare
  %guarantees for posted pricings continue to hold under negatively
  %correlated arrivals.} In other words, with some probability $\qvi$,
  %buyer $i$'s values are given by the fixed and known function $\vali$,
  %and with probability $1-\qvi$, his value is $0$ for every bundle of
  %items. When the non-zero value function is instantiated, we say that
  %the buyer has arrived.

Third, we think of a buyer $i$ with value function $\vali$
as being a collection of ``jobs'' represented by tuples $(i,\vali,S)$,
one for each bundle $S$ of items desired by the buyer. Informally,
each job is a potential (minimal) allocation to a buyer with a given
value.\footnote{In particular we remove ``duplicates'', or bundles $S$
  such that for some $S'\subsetneq S$, $\vali(S')=\vali(S)$.} Let
$\Jvi = \{(i,\vali, S)\}$ denote the set of all such jobs for a buyer
$i$ with value function $\vali$; let $U$ denote the union of these
sets over all possible buyers and value functions. When a buyer $i$
with value $\vali$ arrives, we interpret this event as the
simultaneous arrival of all of the jobs in $\Jvi$. If the buyer is
allocated the bundle $S$, we say that job $(i,\vali,S)$ is
allocated. In what follows, it will be convenient to consider jobs as
fundamental entities. We therefore identify a job $(i,\vali,S)$ by the
single index $j$.  Then $\intj$ is the corresponding bundle of items,
$\vj$ is $\vali(I_j)$, and $\qj$ is $\qvi$.

\paragraph{Interval and path preferences.} In the interval preferences
setting, the set $\items$ of items is totally ordered, and buyers
assign values to intervals of items. In particular, for any bundle $S$
of items,
$\vali(S) = \max_{\text{intervals } I \subseteq S} \, \vali(I)$ where
$I$ ranges over all contiguous intervals contained in
$\items$. Accordingly, jobs as defined above also correspond to
intervals. In the path preferences setting, $\items$ corresponds to
the set of edges in a given tree. Each buyer $i$ has a fixed and known
path, denoted $P_i$, and a scalar (random) value $\vali$, with
$\vali(S) = \vali$ for $S\supset P_i$ and $0$ otherwise. Accordingly,
buyers are {\em single-minded} and each instantiated value function of
a buyer is associated with a single job.

\paragraph{Bundle pricings.} The mechanisms we study are static and
anonymous bundle pricings. Let $\prices$ denote such a pricing
function. For the interval preferences setting, we partition the
multiset of items into disjoint intervals and price each interval in
the partition. For the path preferences setting, we partition items
into disjoint ``layers'' and construct a different pricing function for
each layer, which assigns a price to every path contained in that
layer.\footnote{The pricing is essentially a layer-specific
  item pricing, with bundle totals subject to a layer-specific reserve
  price.} Observe that different copies of an item end up in different
bundles/layers, and may therefore be priced differently.  Buyers
arrive in adversarial order. When buyer $i$ arrives, he selects a
subset of remaining unsold bundles to maximize his value for the items
contained in the subset minus the total payment as specified by the
pricing $\prices$.

\vspace{0.1in}
\noindent
Let $\opt$ denote the hindsight/offline optimal expected social welfare, and
$\sw(\prices)$ denote the expected social welfare obtained by the
static, anonymous bundle pricing $\prices$.

% \subsection{Fractional allocations}

% Our analysis makes use of {\em fractional allocations}. A fractional allocation
% $x$ assigns an $\xjki$-fraction of each item in $\intji$ to buyer $j$ when
% her value is $\valjk$.\footnote{Note that fractional allocations are used only
% in determining prices and in bounding the expected optimal (integer) solution,
% not in the mechanism itself.}

% An allocation $x$ is a feasible fractional allocation if it satisfies the
% following constraints.
% \begin{align*}
%     %\max \quad & \sum_j\sum_{k=1}^{\suppj}\qjk\sum_{i=1}^{\intsj}
%             %\valjki\xjki \\
%     \sum_j \sum_{k=1}^\suppj \qjk \sum_{i:\intji\ni t}\xjki & \le B_t
%             \quad \forall t \;\textrm{(supply)} \\
%          \|\xjk\|_1 & \le 1 \quad \forall j, k \;\textrm{(demand)} \\
%          x & \ge 0
% \end{align*}

\paragraph{A fractional relaxation.}
A {\em fractional allocation} $x$ assigns an $\xj$-fraction of each item in
$\intj$ to job $j$.\footnote{Note that fractional allocations are used only in
determining prices and in bounding the expected optimal (integer) solution, not
in the mechanism itself.} A fractional allocation is feasible if it satisfies
the supply and demand constraints of the integral problem: no item may be
allocated more than $\capt$ times and no buyer obtains additional value for more
than one bundle. Let $\feas$ denote the polytope defined by constraints 
\eqref{eq:supply}--\eqref{eq:nonneg} below. 
%Then $x$ is a feasible fractional 
%allocation if and only if $x \in \feas$. 
\begin{align}
    \hfill \sum_{j : \intj \ni t} \xj &\le \capt \quad \forall t
            & & \textrm{(supply)} \label{eq:supply} \\
    \sum_{j \in \Jvi} \xj & \le \qvi \quad \forall \vali
            & & \textrm{(demand)} \label{eq:demand} \\
    x & \ge 0 & & \label{eq:nonneg}
\end{align}
For any $x$, $\fval(x)$ is the total fractional value in $x$ (without
regard for feasibility), and $\fwt(x)$ is the total fractional weight
of $x$:
\[\fval(x) = \sum_j v_j x_j \quad \text{and} \quad \fwt(x) = \sum_j
  x_j.\] 
\noindent
For a subset $A$ of jobs, we use $x_A$ (and sometimes $(x,A)$) to
denote the fractional allocation confined to set $A$ and zeroed out
everywhere else. That is, $(x_A)_j = x_j$ for $j\in A$ and $(x_A)_j=0$
for $j\not\in A$.
\[\fval(x_A) = \sum_{j\in A} v_j x_j\quad \text{and} \quad \fwt(x_A) = \sum_{j\in A} x_j.\]  
\noindent
Fractional allocations provide an upper bound on the optimal welfare;
see \Cref{sec:deferred} for the proof of a more general statement
(Lemma~\ref{lem:fopt-upperbound-costs}). Here
$\fopt := \max_{x \in \feas} \fval(x)$.
% The following lemma is a special case of
% Lemma~\ref{lem:fopt-upperbound-costs};
\begin{lemma}
    \label{lem:fopt-upperbound}
    $\fopt \ge \opt$.
\end{lemma}

% \paragraph{Unit allocations.} We now define a class of fractional
% allocations with properties that guarantee the existence of a
% bundle-pricing mechanism which obtains almost the same welfare. The
% intent is to decompose the fractional allocation across disjoint
% bundles such that the fractional value assigned to each bundle can be
% recovered by pricing that bundle individually.

% % value over bundles, such
% % that the welfare generated by each bundle can be charged to the fractional
% % allocation.

% % Our analysis here follows very closely the techniques of \citeauthor{FGL15}.
% % Our main contribution, in \Cref{sec:upperbound}, is a proof that we can find
% % such fractional allocations achieving asymptotically optimal welfare.

% \begin{definition} A fractional allocation $x$ is a {\em fractional
%     unit allocation} if there exists a partition of the multiset of
%   items (where item $t$ has multiplicity $\capt$) into bundles
%   $\{T_1, T_2, T_3, \cdots\}$, and a corresponding partition of jobs
%   $j\in U$ with $x_j>0$ into sets $\{A_1, A_2, A_3, \cdots\}$, such
%   that:
%     \begin{itemize}
%         \item For all $j\in U$ with $x_j>0$, there is exactly one
%           index $k$ with $j\in A_k$. 
%         \item For all $k$ and $j\in A_k$, $I_j\subseteq T_k$.
%         \item For all $k$, we have $\fwt(x_{A_k}) \le 1$.
% % there exists an index $k_j$ with
% %             $\intj\subseteq T_{k_j}$ and $\intj\cap T_{k'}=\emptyset$ for all
% %             $k'\ne k_j$.
% %         \item Denoting $A_k = \{j: k_j = k\}$, we have $\fwt(x_{A_k}) \le 1$.
%     \end{itemize}
% \end{definition}

% %This definition enables us to charge the outcome of a bundle pricing (where the
% %bundles are the sets $T_k$) to the value of the fractional allocation.
% A note on the terminology: we call fractional allocations satisfying
% the above definition {\em unit} allocations because once the partition
% of items is specified, each job can be assigned at most one bundle in
% the partition and each bundle can be fractionally assigned to at most
% one job. % We emphasize that each item $t$ may appear in at most $\capt$ different sets $T_k$, and
% % each job with nonzero fractional allocation is associated with exactly one
% % set $T_k$.

% We note that given any fractional unit allocation $x$, for any
% instantiation of values, it is possible to define a pricing function
% over the bundles $T_k$ that is $(1,1)$-balanced with respect to $x$
% with the framework of \cite{DFKL-17}. This is because fractional unit
% allocations behave essentially like feasible allocations for
% unit-demand buyers. For completeness we present a simpler
% first-principles argument based on the techniques of \citet{FGL15}
% showing that such a pricing obtains at least half the value of the
% fractional unit allocation. The proof is deferred to
% Section~\ref{sec:deferred}.

% % We apply the techniques of \citet{FGL15} to show that there exists a bundle
% % pricing (where the bundles are the sets $T_k$) which obtains at least half the
% % value of the fractional unit allocation.  Crucially, the argument accounts for
% % welfare {\em per bundle}, not per buyer: either a bundle is sold, in which case
% % the revenue accounts for the fractional value, or else the utility of buyers
% % who {\em could} have purchased the bundle accounts for the value. Thus, we obtain a
% % bundle pricing with static\footnote{Note that in the partition of
% %   items associated with the fractional unit allocation, multiple
% %   bundles may correspond to identical sets of items. Our argument
% %   determines a price for each bundle individually, and may therefore
% %   pick different prices for copies of the same bundle of items. This
% %   is easy to fix. We can, for example, average out the fractional
% %   allocation associated with each copy of the bundle, without
% %   affecting the quality of the fractional allocation. Our argument
% %   would then assign the same price to each copy.} and anonymous prices.

% \begin{lemma}
%     \label{lem:FGL}
%     For any feasible fractional unit allocation $x$, there exists a static,
%     anonymous bundle pricing $\prices$ such that
%     \[
%         \sw(\prices) \ge \half \fval(x).
%     \]
% \end{lemma}

