\subsection{Extension to arbitrary capacities}
\label{sec:arbit-cap-ub}

We will now give a reduction from the setting with arbitrary item
multiplicities to the unit-capacity case discussed in the previous section.
Once again we start with an arbitrary fractional allocation $x$ and construct a
unit allocation $x'$ with large fractional value. The main idea behind the
reduction is to first partition the fractional allocation $x$ into ``layers''.
Layer $r$ corresponds to the $r$th copy of each item (if it exists). Each job
is assigned to one layer, with its fractional allocation appropriately scaled
down, in such a manner that the layers together capture much of the fractional
value of $x$. We then apply Theorem~\ref{thm:unit-capacity} to each layer
separately, obtaining unit allocations for each layer. The union of these unit
allocations immediately gives us a unit allocation overall.

\begin{theorem}
    \label{thm:arbit-capacity}
    For every feasible fractional allocation $x$ in the interval
    preferences setting with arbitrary capacities, there exists a
    fractional unit allocation $x'$, such that
    \[
        \fval(x)\le  O(\log L/\log\log L)\fval(x').
    \]
\end{theorem}

\begin{proof}
  We begin by defining layers. Recall that $B_t$ denotes the number of
  copies of item $t$ that are available. We assume without loss of
  generality that $B_t = \lceil \sum_{j: t\in \intj} x_j\rceil$ for
  all $t$ (and otherwise redefine $B_t$ to be the latter
  quantity). Let $\Bmax = \max_t B_t$. Then we define the $r$th layer
  for $r\in [\Bmax]$ as $\layer:=\{t\in\items:B_t\geq r\}$. 

  We will now construct a fractional allocation $\xr$ for each layer
  $\layer$ with the properties that {\em (i)} for each $r$, $\xr$ is
  feasible with respect to the set of items in $\layer$, and {\em
    (ii)} $\sum_r \fval(\xr) \ge \frac 14 \fval(x)$.

    We proceed via induction over $\Bmax$. For the base case, suppose that
    $\Bmax\le 4$. Then we define $\xr[1] = \frac 14 x$ and $\xr=0$ for all
    $r>1$. Observe that both the properties {\em (i)} and {\em (ii)} are
    satisfied by this definition. For the inductive step, we pick a set $S$ of
    jobs as given by the following lemma (proved in \Cref{sec:deferred}). 

    \begin{lemma}
        \label{lem:greedy-layer}
        For any feasible fractional allocation $x$ in the interval
        preferences setting with arbitrary capacities, one can
        efficiently construct a set $S$ of jobs such that the total
        fractional weight of $x_S$ at any item $t$ is at least
        $\min\{1,B_t\}$ and at most $4$.  Formally, for all items $t$,
        $\min\{1,B_t\}\le \sum_{j\in S: t\in \intj} x_j < 4$.
    \end{lemma}

    Having constructed such a set $S$, we set $\xr[1] = \frac 14 x_S$, and
    recursively construct allocations for the remaining layers using $x-x_S$.
    Observe that $\frac 14 x_S$ is feasible for $\layer[1]$ by the definition
    of $S$. Furthermore, by removing jobs in $S$ from $x$, we reduce $\Bmax$ by
    at least $1$, and therefore, we can apply the inductive hypothesis to
    construct allocations for the remaining layers. This provides us with {\em
    (i)} and {\em (ii)} as desired above.

    Finally, for each layer $\layer$, we apply Theorem~\ref{thm:unit-capacity}
    to the allocation $\xr$ to obtain a fractional unit allocation $\xpr$ for
    that layer. Then, the allocation $x' = \sum_r \xpr$ is a fractional unit
    allocation with $\fval(x)\le O(\beta) \fval(x)$.
\end{proof}

Combining \Cref{thm:arbit-capacity} with
Lemmas~\ref{lem:fopt-upperbound} and \ref{lem:FGL} immediately implies
Theorem~\ref{thm:unit-cap-ub}, which we state here for completeness.

\begin{numberedtheorem}{\ref{thm:unit-cap-ub}}
For the interval preferences setting, there exists a static, anonymous bundle
pricing with competitive ratio $O\left(\frac{\log L}{\log\log
    L}\right)$ for social welfare.
\end{numberedtheorem}

% In the fixed-capacity setting, there exists a static, anonymous bundle
% pricing $(\Pi,\prices)$ such that
% \[
%     \opt \le O\left(\frac{\log L}{\log\log L}\right)\sw(\Pi,\prices).
% \]

