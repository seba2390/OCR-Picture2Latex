\subsection{The large capacity setting}
\label{sec:trees-large-cap-ub}

In this section we consider the setting where every edge is available in large
supply. Recall that we define $B:= \min_t B_t$. 
We show that as $B$ increases, the
approximation ratio achieved by bundle pricing gradually decreases. 

\begin{numberedtheorem}{\ref{cor:tree-large-cap-ub}}
For the path preferences setting, if every edge has at least $B>0$
copies available, then a static, anonymous bundle pricing achieves a
competitive ratio of
\(
    O\left(\frac1B\log H\right)
\)
for social welfare.
% In the fixed-capacity setting, with $B = \min_t\capt < \log L$, there exists
% a static, anonymous bundle pricing $(\Pi,\prices)$ such that
% \[
%     \opt \le O\left(\frac1B\frac{\log L}{\log\log L - \log B}\right)
%         \sw(\Pi,\prices).
% \]
% When $B \ge \log L$, there exists a bundle pricing which gives an
% $O(1)$-approximation.
\end{numberedtheorem}

% \begin{theorem}
%     \label{thm:large-capacity}
%     For the fixed capacity setting with $B= \min_t B_t< \log L$, there exists a
%     static, anonymous bundle pricing that achieves an approximation ratio of
%     \[
%         \frac 1B \cdot \frac{\log L}{\log \log L - \log B}
%     \]
%     for social welfare. When $B\ge\log L$, an $O(1)$ approximation ratio is
%     achieved. Given access to an optimal feasible fractional allocation, the
%     pricing can be constructed in polynomial time.
% \end{theorem}

\begin{proof}
  Let $k=\frac 12 B$ and $\alpha=H^{1/k}$.  The proof technique is
  basically identical to that of \Cref{cor:large-cap-ub}, where we
  partition both the item supply and the jobs into $k$ instances, such
  that on the one hand, the fractional solution confined to jobs
  within an instance will be feasible for the item supply in that
  instance; on the other hand, within each instance job values will
  differ by a factor of at most $\alpha$.  Then, applying
  \Cref{thm:trees-ub} will give us a static, anonymous pricing for
  every instance individually with a factor of
  $\Omega\big(\frac{1}{\log \alpha}\big)$ loss in social welfare. Now
  consider running these instances in parallel, with each buyer
  allowed to purchase bundles from any of the instances. The
  utility-revenue analysis in \Cref{lem:single-value-class} continues
  to apply, providing the same guarantees on welfare. However, we may
  double count some part of the welfare: suppose a buyer $j$ is
  ``assigned'' by the layered allocation to one instance, but
  purchases a bundle in another instance; then we may account for both
  his contribution to utility for the first instance as well as his
  contribution to revenue for the second instance. This double
  counting only costs us a factor of $2$ in the overall welfare, and
  we obtain an $O(\log\alpha)=O\left(\frac{1}{B}\log H\right)$
  approximation.
\end{proof}