\subsection{From layered allocations to bundle pricings}
\label{sec:trees-unit-cap}

In this section we will demonstrate the existence of a good bundle
pricing given any fractional layered allocation:

\begin{lemma}
\label{lem:single-value-class}
    Given a feasible fractional layered allocation $y$, there exists a bundle
    pricing $\prices$ such that
    \[
        \fval(y) \le O\left(\frac{\vmax(y)}{\vmin(y)}\right) \sw(\prices).
    \]
\end{lemma}

To develop intuition for our approach consider the special case where
every edge in the tree has capacity $1$ (that is, the layered
allocation has a single layer), and every job has equal value. We
claim that essentially the most contentious edges in a job's path are
its peak edges -- those closest to the root. In particular, if a job
gets ``blocked'' by another admitted job on one or more of its peak
edges, then we should try to recover its lost value in the form of
revenue from the blocking edge. What if a job gets blocked on a
non-peak edge? We can then charge the lost value of this job to the
value of the blocking job---it is easy to observe that any such
blocking job is charged at most once in this manner. In effect, every
job has up to two important edges, namely its peak edges, that
determine whether and to what extent the job will contribute to the
solution's revenue or its utility. Upon focusing on these two edges,
then, the instance behaves like one where every desired bundle has
size $2$ and we can construct a bundle pricing that achieves a
competitive ratio of $O(1)$. In order to convert this argument into a
proof we need to deal with several complications: jobs have different
values; different arms of a job may be allocated at different layers
in the fractional layered allocation; etc.

\begin{proof}
We start by introducing some notation. Given the fractional layered
allocation $y$, let $\{T_1, T_2, \cdots\}$ be the partition of
items/edges into layers, and $\{A_1, A_2, \cdots\}$ be the
corresponding partition of arms. Henceforth we will think of the
different copies of an edge as distinct items, and when we refer to
some edge/item $e$ it will be understood that this edge corresponds to
some particular set $T_k$. For a $j\in U$ with $y_j>0$, and for the
index $k$ such that $\armj\in A_k$, we will redefine $P_{\armj}$ (and
likewise, $P_j$) as the subset of edges of $T_k$ that correspond to
this arm.\footnote{That is, $P_{\armj}$ corresponds to the specific
  copies of edges in $T_k$ and not any copies that form the path
  corresponding to this arm.} Accordingly, we also redefine the peak
edges, $\peakj$, of a job $j$ with $y_j>0$ as the copies of its peak
edges that belong to $T_k$ corresponding to $\armj\in A_k$. Note that
we may assume without loss of generality (via rescaling, e.g.) that
$\vmin(y)=1$.

We are now ready to define our pricing:
\begin{itemize}
\item For any layer $k$, edge $e\in T_k$, and job $j$ such that
  $\armj\in A_k$, define the contribution of $j$ to $e$'s price as
  $\contrib_{ej} = \frac 14 y_j v_j \ind[e \in \peakj]$. For all other
  pairs $(j, e)$, set $\contrib_{ej} = 0$.
\item For any $k$ and edge $e\in T_k$, define the price of $e$
  as $\phat_e = \sum_j \contrib_{ej}$.
\item For any $k$ and a bundle $B\subset T_k$ that corresponds to a
  ``monotone'' path,\footnote{By a monotone path we mean one that has
    a single peak edge.} define the price of the bundle as
  $p_B = \max(1, \sum_{e\in B} \phat_e)$. All other bundles are priced at
  infinity.
\end{itemize}

\noindent
Our mechanism offers the static, anonymous bundle prices $\prices$ as
defined above. Now consider any instantiation of job arrivals. We
consider three events related to job $j$:
\begin{itemize}
\item $\mathcal E^1_j$ occurs if, at the time of $j$'s arrival, all edges in
  $\peakj$ are unsold;
\item $\mathcal E^2_j$ occurs if, after all jobs have arrived, at least one of
  the edges in $\peakj$ is sold;
\item $G_j$ occurs when either of the above holds: $G_j = {\mathcal
    E}^1_j \union {\mathcal E}^2_j$.
\end{itemize}

We now make the following two claims. The first bounds the
contribution to the welfare of the pricing from ``good'' jobs for
which event $G_j$ occurs. The second bounds the loss from the ``bad''
jobs. Here $\util(\prices)$ denotes the total buyer utility generated
by the pricing $\prices$ and $\rev(\prices)$ denotes the total revenue
generated.

\begin{claim}
\label{claim:good-bound}
\begin{equation}
  \label{eq:good-bound}
     \util(\prices) + 4\;\rev(\prices) + \sw(\prices) \ge \sum_jy_j\valj\prob{G_j} - \half\fval(y). 
\end{equation}
\end{claim}

\begin{claim}
\label{claim:bad-bound}
\begin{equation}
  \label{eq:bad-bound}
  2\vmax(y)\sw(\prices) \ge \sum_j y_j\valj \prob{\overbar{G_j}}. 
\end{equation}
\end{claim}

\noindent
Given the two claims, adding Equations~\eqref{eq:good-bound} and
\eqref{eq:bad-bound}, and using the fact that
$\sw(\prices) = \util(\prices) + \rev(\prices) $, we have
    \begin{align*}
        (2\vmax(y) + 5)\;\sw(\prices) &\ge \sum_j y_j\valj - \half\fval(y)
        = \half\fval(y)
    \end{align*}
The theorem follows. It remains to prove the claims.

\begin{proofof}{Claim~\ref{claim:good-bound}}
  For a job $j$ with $y_j>0$, let $p(j) = \sum_{e\in P_j} \phat_e$
  denote the price of the ``intended'' bundle for $j$. We first write
  the utility of the pricing $\prices$:
  \begin{align}
    \util(\prices) &\ge \sum_j y_j \prob{\mathcal E^1_j}
                     \big(\valj - \max\{1, p(j)\}\big) \nonumber \\
                   &= \sum_j y_j \prob{\mathcal E^1_j} \big(\valj - p(j)\big) -
                     \sum_j y_j \prob{\mathcal E^1_j}\big(1 - p(j)\big)^+\nonumber\\
                  &= \sum_j \valj y_j \prob{\mathcal E^1_j} - \sum_j
                    \big(y_j \prob{\mathcal E^1_j} \sum_{e\in P_j} \phat_e\big) -
                     \sum_j y_j \prob{\mathcal E^1_j}\big(1 - p(j)\big)^+
                     \label{eq:pricing-util-lb}
  \end{align}
  where the inequality follows because each job is a profit-maximizer
  and, when not blocked, will pay the sum of the prices (or 1) if this
  is less than its value.

  We bound the three terms in \eqref{eq:pricing-util-lb} separately,
  beginning with the third. Note that job $j$ always purchases some
  bundle containing its path when it arrives, if event $\mathcal E^1_j$ occurs. Therefore,
  since $(1 - p(j))^+ \le 1$ and the single-minded buyer's value for
  any bundle containing his path is least 1,
    \begin{equation} 
       \label{eq:price-floor-bound}
        \sum_j y_j\prob{A_j}\big(1 - p(j)\big)^+ \le \sw(\prices).
    \end{equation}

\noindent
    For the second term, we have:
    \begin{align}
      \sum_j \left(y_j \prob{\mathcal E^1_j} \sum_{e\in P_j} \phat_e\right) 
       \le \sum_e \left(\phat_e \sum_{j:e\in P_j} y_j \right) & \le \sum_e \phat_e \nonumber\\
      & = \sum_{e,j} \contrib_{e,j} = \sum_j \sum_{e\in\peakj} \frac
        14 y_jv_j 
      \le \frac 12 \fval(y)
       \label{eq:sec-term-bound}
    \end{align}
    \noindent
    where the second inequality follows from the definition of
    fractional layered allocations.

    We now bound the first term in \eqref{eq:pricing-util-lb}. For
    edge $e$, let $\mathcal E^2_e$ denote that, after all jobs have
    arrived, edge $e$ is sold. Recall that
    $G_j = \mathcal E^1_j \cup \mathcal E^2_j = \mathcal E^1_j \cup
    (\cup_{e\in\peakj} \mathcal E^2_e) $. Therefore,
    \begin{align}
      v_jy_j\ind[G_j] & \le v_jy_j\ind[\mathcal E^1_j] +
      \sum_{e\in\peakj} v_jy_j\ind[\mathcal E^2_e]\nonumber\\
      & = v_jy_j\ind[\mathcal E^1_j] +
      \sum_{e} 4\contrib_{ej}\ind[\mathcal E^2_e]\nonumber\\
\intertext{Summing over all $j$,}
      \sum_j v_jy_j\ind[G_j] & \le \sum_j  v_jy_j\ind[\mathcal E^1_j]
        + 4\sum_e \ind[\mathcal E^2_e]\sum_j \contrib_{ej}\nonumber\\
      & = \sum_j  v_jy_j\ind[\mathcal E^1_j] + 4 \sum_e 
      \ind[\mathcal E^2_e]\phat_e\nonumber\\
      & \le \sum_j  v_jy_j\ind[\mathcal E^1_j] + 4 \rev(\prices)
\label{eq:first-term-bound}
    \end{align}
    The claim now follows by putting together
    Equations~\eqref{eq:pricing-util-lb},
    \eqref{eq:price-floor-bound}, \eqref{eq:sec-term-bound}, and
    \eqref{eq:first-term-bound}.
\end{proofof}

\begin{proofof}{Claim~\ref{claim:bad-bound}}
  Fix a particular instantiation of jobs and consider a job $j$ for
  which $G_j$ does not occur. Then $j$ is blocked at the time of its
  arrival on some non-peak edge in $P_j$. Let $e$ be the edge closest
  to the root among all such blocking edges. Then, $e$ is a peak edge
  for the job $j'$ that is allocated this edge. We charge the quantity
  $v_jq_j$ to $j'$.

  Now observe that all of the jobs $j$ that are charged to some job
  $j'$ along edge $e\in T_k$ contain the parent edge of $e$, call it $e'$, in their
  arm $\armj\in A_k$. Then we can use the fact that
  $\fwt(y, \{j: \armj\in A_k \text{ and } P_{\armj}\ni e'\}) \le 1$ from the
  definition of fractional layered allocations to assert that the
  total weight of such jobs is at most $1$. In other words, the total
  charge on $j'$ is at most 
  \[\sum_{j: j\text{ charges }j'} v_j q_j \le \vmax(y) \sum_{j: j\text{ charges }j'} q_j \le 2\vmax(y) \le 2\vmax(y) v_{j'}\]
  where the factor of $2$ arises from the fact that $j'$ could be
  charged along both of its peak edges, and the last inequality follows
  by recalling that $v_{j'}\ge \vmin(y)=1$ for any job $j'$ allocated
  by the pricing.

  The claim now follows by summing the above inequality over jobs $j'$
  allocated by the pricing, and taking expectations of both sides over
  possible instantiations.
\end{proofof}
This completes the proof of the lemma.
  \end{proof}
  % The proof relies on the fact that, in any
  %   feasible layered solution, for every job $j$ there exists at most two jobs
  %   $j'_1,j'_2$ that are in the solution and blocks $j$ from above, since for
  %   each arm of $j$ there can be only one job blocking that arm from above.
  %   Thus, the total fractional weight of such jobs is at most 2.
  %   \begin{align*}
  %       \sw(\prices) &= \sum_j \prob{j \in \alg} \\
  %       &\ge \sum_j \Big(\frac{1}{2}\!\sum_{j'\text{ intersects an arm of }j\text{ from above}}q_{j'}\Big)
  %       \prob{j\in \alg} \\
  %       &\ge \frac{1}{2}\sum_j q_j \prob{\overbar{G_j}}.
  %   \end{align*}


% \begin{proof}


%     Conceptually, we start by defining item prices such that item $t$ in
%     layer $k$ is priced at $p_{tk}$. Jobs arrive in arbitrary order and
%     purchase bundles of items at the sum of the constituent item prices. We
%     offer only bundles such that all of the items in each arm come from the
%     same layer, but jobs may purchase arms belonging to any layer, as supplies
%     last.

%     To account for the welfare generated by our pricing, we define the
%     following events. For each job $j$, assume that the layered fractional
%     allocation assigns $\larmj$ to $\layeri[\klj]$ and $\rarmj$ to
%     $\layeri[\krj]$. Then we say
%     \begin{itemize}
%         \item $A_j$ occurs when, at the time of $j$'s arrival, all edges
%             in $\intj[\larmj]$ are unsold in layer $\klj$ and all edges in
%             $\intj[\rarmj]$ are unsold in layer $\krj$;
%         \item $B_j$ occurs when, after all jobs have arrived, the peak
%             edge of $\larmj$ (or $\rarmj$) is sold in layer $\klj$ (or
%             $\krj$); and
%         \item $G_j$ occurs when either of the above holds: $G_j = A_j \union B_j$.
%     \end{itemize}
%     Intuitively, $G_j$ represents a ``good'' event for the welfare of job $j$:
%     either $j$ is not blocked, or at least one of its peak edges has been
%     purchased. We set prices specifically so that revenue generated by the sale
%     of either peak edge compensates for the blocked value. Finally, we show
%     that the total value of jobs which experience the ``bad'' event
%     $\overbar{G_j}$ can be charged to the value of the jobs that cause the bad
%     events.

%     This last charging argument relies on the fact that, on a tree, any job can
%     cause the bad event for at most one job in the optimal solution. However,
%     we cannot guarantee that the value of the blocking job is more than
%     $\vmin(y)$, while the value of the blocked job may be as large as
%     $\vmax(y)$. Thus the ratio $\frac{\vmax(y)}{\vmin(y)}$ in our bound.

%     In fact, there may be jobs present in the system with value less than
%     $\vmin(y)$. In order for the bound to hold when these jobs\footnote{Note that
%     jobs with value greater than $\vmax(y)$ do not pose any such problem.} are
%     present, we must guarantee that such jobs will not block any job $j$ with
%     $y_j > 0$. We therefore set the price of every bundle $S$ to
%     $\max\{\vmin(y), \sum_{(t,k)\in S} p_{t,k}\}$.

%     We now define our prices formally.    For each edge $t$ and layer $k$, let
%     $F_{kt} = \{\armj\in \layeri[k] : t \in \peakj[\armj]\}$ denote all arms in
%     layer $k$ which have $t$ as a peak edge.  Define $\alpha_{t\armj k} = \half
%     y_j v_{\armj}\ind[\armj \in F_{kt}]$ to be half of the fractional welfare
%     contribution of arm $\armj$ to edge $t$ in layer $k$.  Intuitively, we
%     associate all of the welfare of each job to its peak edge(s). (Recall that
%     we define $\valj[\armj] = \half \valj$ if $\larmj$ and $\rarmj$ are both
%     non-empty.) For each edge $t$ in layer $k$, set its price to $p_{kt} =
%     \half\sum_{\armj \in F_{kt}}y_{\armj}\valj[\armj] = \sum_{\armj \ni
%     t}\alpha_{t\armj k}$. Note that $\sum_{k,t} p_{kt} = \half\fval(y)$ by
%     definition. 

%     For the remainder of the proof, assume $\vmin(y) = 1$ (we can always
%     rescale by $\vmin(y)^{-1}$). For notational convenience, let $p(j) =
%     \sum_{a\in\{1,2\}}\sum_{t\in \intj[\armj]}p_{\kaj t}$ be the sum of prices
%     associated with the items job $j$ was assigned by the layered fractional
%     allocation. Recalling that we set a minimum price of 1, the utility is at
%     least
%     \begin{align}
%         \util(\prices) &\ge \sum_j y_j \prob{A_j}
%                 \big(\valj - \max\{1, p(j)\}\big) \nonumber \\
%         &= \sum_j y_j \prob{A_j} \big(\valj - p(j)\big) -
%                 \sum_j y_j \prob{A_j}\big(1 - p(j)\big)^+
%                 \label{eq:pricing-util-lb}
%     \end{align}
%     where the inequality follows because each job is a profit-maximizer
%     and, when not blocked, will pay the sum of the prices (or 1) if this is
%     less than its value.

%     We bound the two terms in \eqref{eq:pricing-util-lb} separately, beginning
%     with the second. Note that job $j$ always purchases some bundle when it
%     arrives and event $A_j$ occurs, so the probability that $j$ buys some
%     bundle is at least $y_j\prob{A_j}$.  Therefore, since $(1 - p(j))^+ \le 1$
%     and every bundle has price at least 1,
%     \begin{equation}
%         \label{eq:price-floor-bound}
%         \sum_j y_j\prob{A_j}\big(1 - p(j)\big)^+ \le \rev(\prices).
%     \end{equation}

%     We now bound the first term in \eqref{eq:pricing-util-lb}. Let $t_1$ be the
%     peak edge of $\larmj$ and $t_2$ be the peak edge of $\rarmj$. Then
%     \begin{equation*}
%         y_j v_j \ind[A_j] +
%         2(\alpha_{t_1\larmj \klj}+\alpha_{t_2\rarmj \krj})\ind[B_j]
%         \ge y_j v_j \ind[G_j].
%     \end{equation*}
%     Let $\soldtkj$ be the event that item $t$ in layer $k$ has sold when job
%     $j$ arrives, and let $\soldtk$ be the event that item $t$ in layer $k$ is
%     sold after all jobs have arrived.  Notice that
%     \[
%         \ind[\soldtk] \geq \ind[\soldtkj]
%     \]
%     and
%     \[
%         \ind[\soldtkj[t_1][\klj]] + \ind[\soldtkj[t_2][\krj]] \ge \ind[B_j],
%     \]
%     so that
%     \begin{equation*}
%         y_j v_j \ind[A_j] + 4\sum_{a\in\{1,2\}}\sum_{t\in \armj}
%         \alpha_{t\armj\kaj}\ind[\soldtk[t][\kaj]] \ge y_j \valj \ind[G_j].
%     \end{equation*}
%     Taking the expectation over $v_{-j}$, we have
%     \begin{equation}
%         \label{eq:help-i-need-a-label}
%         y_j v_j \prob{A_j} + 4\sum_{a\in\{1,2\}}\sum_{t\in
%         \intj[\armj]}\alpha_{t\armj\kaj}\prob{\soldtk[t][\kaj]} \ge y_j \valj \prob{G_j}.
%     \end{equation}

%     Returning to the term to be bounded,
%     \begin{align*}
%         \sum_j y_j \prob{A_j} \big(\valj - p(j)\big)
%         &\ge \sum_j y_j \valj\prob{G_j} - 4\sum_{a\in\{1,2\}}\sum_{t\in
%                 \intj[\armj]}\alpha_{t\armj\kaj}\prob{\soldtk[i][\kaj]} \\
%                 &\qquad- \sum_j\sum_{a\in\{1,2\}}\sum_{t\in \intj[\armj]}p_{\kaj t} \prob{A_j} \\
%         &\ge \sum_j y_j\valj\prob{G_j} - 4\sum_k\sum_{t} p_{kt}\prob{\soldtk} -
%                 \sum_k\sum_{t} p_{kt}.
%     \end{align*}
%     The first inequality follows from \eqref{eq:help-i-need-a-label}, and the
%     second from $\prob{A_j}\leq 1$. Combining with\eqref{eq:price-floor-bound},
%     observing that the second term is equal to 4 times the revenue and then
%     rearranging, we have
%     \begin{equation}
%         \label{eq:good-bound}
%         \util(\prices) + 4\;\rev(\prices) \ge \sum_jy_j\valj\prob{G_j} - \half\fval(y)
%     \end{equation}

% \end{proof}

Armed with this lemma, we can now complete the proof of
Theorem~\ref{thm:trees-ub}. Start with the optimal fractional
allocation $y$, and partition all jobs into $\log H$ value classes,
where each class contains jobs with values within a factor of $2$ of
each other. One of these classes, call it $C$, contributes more than a
$\log H$ fraction to the fractional value of $y$. Then, applying
Lemma~\ref{lem:single-value-class} to the allocation $y_C$ gives
us the following theorem.

\begin{numberedtheorem}{\ref{thm:trees-ub}}
For the path preferences setting, there exists a static, anonymous bundle
pricing with competitive ratio $O\left(\log H\right)$ for social welfare.
\end{numberedtheorem}
