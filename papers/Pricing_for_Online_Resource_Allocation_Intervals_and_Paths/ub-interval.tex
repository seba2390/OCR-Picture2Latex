\section{Bundle Pricing for Interval Preferences}
\label{sec:ub-interval}

In this section we present our main results,
Theorems~\ref{thm:unit-cap-ub} and \ref{cor:large-cap-ub}, for the
interval preferences setting. We begin by defining a special kind of
fractional allocations that we call {\em fractional unit
  allocations}. These have properties that guarantee the existence of
a bundle-pricing mechanism which obtains a good fraction of the
welfare of the fractional allocation. The intent is to decompose the
fractional allocation across disjoint bundles such that the fractional
value assigned to each bundle can be recovered by pricing that bundle
individually. Section~\ref{sec:frac-unit-alloc} presents a definition of
unit allocations and their connection to pricing.

% Recall that Lemma~\ref{lem:FGL} shows
% that in order to obtain a good approximation ratio via bundle
% pricings, it suffices to design a good fractional unit allocation. 

The remaining technical content of this section then focuses on
designing fractional unit allocations. We begin with the special case
of unit-capacities in Section~\ref{sec:unit-cap-ub}, where we show the
existence of an $O(\log L/\log\log L)$-approximate fractional unit
allocation. In Section~\ref{sec:arbit-cap-ub} we extend our analysis
to the general case of arbitrary multiplicities, proving
Theorem~\ref{thm:unit-cap-ub}. Finally, in
Section~\ref{sec:large-cap-ub} we show that if the capacity for every
item is large enough, specifically at least $B$, then the
approximation ratio decreases by a factor of $\Theta(B)$
(Theorem~\ref{cor:large-cap-ub}).

\subsection{Fractional unit allocations}
\label{sec:frac-unit-alloc}

\begin{definition} A fractional allocation $x$ is a {\em fractional
    unit allocation} if there exists a partition of the multiset of
  items (where item $t$ has multiplicity $\capt$) into bundles
  $\{T_1, T_2, T_3, \cdots\}$, and a corresponding partition of jobs
  $j\in U$ with $x_j>0$ into sets $\{A_1, A_2, A_3, \cdots\}$, such
  that:
    \begin{itemize}
        \item For all $j\in U$ with $x_j>0$, there is exactly one
          index $k$ with $j\in A_k$. 
        \item For all $k$ and $j\in A_k$, $I_j\subseteq T_k$.
        \item For all $k$, we have $\fwt(x_{A_k}) \le 1$.
   \end{itemize}
\end{definition}

A note on the terminology: we call fractional allocations satisfying
the above definition {\em unit} allocations because once the partition
of items is specified, each job can be assigned at most one bundle in
the partition and each bundle can be fractionally assigned to at most
one job. % We emphasize that each item $t$ may appear in at most $\capt$ different sets $T_k$, and
% each job with nonzero fractional allocation is associated with exactly one
% set $T_k$.

We note that given any fractional unit allocation $x$, for any
instantiation of values, it is possible to define a pricing function
over the bundles $T_k$ that is $(1,1)$-balanced with respect to $x$
within the framework of \cite{DFKL-17}. This is because fractional unit
allocations behave essentially like feasible allocations for
unit-demand buyers. For completeness we present a simpler
first-principles argument based on the techniques of \citet{FGL15}
showing that such a pricing obtains at least half the value of the
fractional unit allocation. The proof is deferred to
Section~\ref{sec:deferred}.

\begin{lemma}
    \label{lem:FGL}
    For any feasible fractional unit allocation $x$, there exists a static,
    anonymous bundle pricing $\prices$ such that
    \[
        \sw(\prices) \ge \half \fval(x).
    \]
\end{lemma}

