\section{Deferred proofs}
\label{sec:deferred}

\subsection{Proof of Lemma~\ref{lem:FGL}}

\begin{numberedlemma}{\ref{lem:FGL}}
   For any feasible fractional unit allocation $x$, there exists a static,
    anonymous bundle pricing $\prices$ such that
    \[
        \sw(\prices) \ge \half \fval(x).
    \]
\end{numberedlemma}
\begin{proof}
    Observe that we can write the social welfare of any $\prices$ as
    the sum of the expected revenue of the seller and the total
    expected utility obtained by the buyers:
    \[
        \sw(\price) = \rev(\price) + \util(\price).
    \]
    Let $\{T_k\}$ and $\{A_k\}$ be the partition of the items and jobs
    respectively corresponding to the unit allocation $x$. For each
    bundle $T_k$, let $W_k = \fwt(x_{A_k})$, and set price
    $\pricek = \frac1{2W_k}\fval(x_{A_k})$. For this setting of
    prices, we will bound the revenue and utility terms separately,
    and show that their sum is at least
    $\half\fval(x) = \half\sum_{k}\fval(x_{A_k})$.

    For any realization of the buyers' values and arrival order, let $Z_k$
    indicate whether bundle $k$ is purchased by {\em any} buyer at these
    prices.  The revenue term is then
    \begin{align}
        \rev(\price)  &= \sum_{k}\Pr[Z_k=1]\pricek \nonumber \\
            &\ge \half\sum_{k}\Pr[Z_k=1]\fval(x_{A_k}). \label{eq:rev}
    \end{align}
    The inequality follows from the fact that $W_k \le 1$ by the
    definition of unit allocations.
    
    For the utility, note that buyer $i$ with value $\vali$ will purchase an
    available bundle maximizing her utility. Define $k_j$ to be the index $k$
    such that $j \in A_k$. The buyer's utility is at
    least\footnote{Here we use the notation $y^+$ to denote $\max\{0,y\}$.}
    \begin{align}
        u_i(\vali) &\ge \expect[Z]{\max_{j\in\Jvi} \indicate\{Z_{k_j}=0\}
                \big(\vali(\intj) - \price[k_j]\big)^+} \nonumber \\
            &\ge \frac1{\qvi}\sum_{j\in\Jvi} \Pr[Z_{k_j}=0]\xj(\vj -
                \price[k_j])^+. \label{eq:vali-util}
    \end{align}
    The second inequality follows from \eqref{eq:demand}.

    Summing over all buyers and all valuations, we have
    \begin{align}
        \util(\price)  &\ge \sum_{i,\vali}\qvi u_i(\vali) \nonumber \\
              &\ge \sum_{i,\vali}\sum_{j\in\Jvi}\Pr[Z_{k_j}=0]\xj(\vj -
                  \price[k_j])^+ \nonumber \\
              &\ge \sum_k \Pr[Z_k=0]\sum_{j\in A_k} \xj(\vj - \pricek) 
                  \nonumber \\
              &\ge \sum_k \Pr[Z_k=0]
                \Big(\fval(x_{A_k})-W_k\pricek  \Big) 
                  \nonumber \\
              &\ge \sum_k \Pr[Z_k=0]
                \Big(\fval(x_{A_k})-\frac 12 \fval(x_{A_k})\Big) 
                  \nonumber \\
              &= \half \sum_k \Pr[Z_k=0] \fval(x_{A_k}) \label{eq:util}
    \end{align}

    The result now follows by summing \eqref{eq:rev} and \eqref{eq:util}.
\end{proof}

\subsection{Proof of Lemma~\ref{lem:small-v-bound}}

\begin{numberedlemma}{\ref{lem:small-v-bound}}
For any fractional allocation $x$, there exists a set of jobs $U_1$ as defined
in \Cref{sec:unit-cap-ub} such that $\fval(x_{U_1})\ge \half\fval(x)$.
\end{numberedlemma}
\begin{proof}
We bound the total fractional value of jobs left out of $U_1$:
    \begin{align*}
        \sum_{j\not\in U_1} v_jx_j &< \sum_{j\not\in U_1} 
                \left(\half\sum_{t\in I_j}\fvt\right) x_j \\
            &= \half \sum_{j\not\in U_1} x_j \left(\sum_{t\in I_j}
                \sum_{j' : I_{j'} \ni t} \densj[j'] x_{j'} \right) \\
            &= \half \sum_t\left(\sum_{\substack{j\not\in U_1\\I_j\ni t}}x_j\right)
                \sum_{j' : I_{j'} \ni t} \densj[j'] x_{j'} \\
            &\le \half \sum_t\sum_{j : I_j \ni t} \densj x_j = \half \fval(x).
    \end{align*}
\end{proof}

\subsection{Proof of Lemma~\ref{lem:a-cell-bound}}

\begin{numberedlemma}{\ref{lem:a-cell-bound}}
    For each interval $\Int_{\ell,k}$, there exists a set of jobs
    $S_{\ell,k} \subseteq \Gtilde_{\ell,k}$ such that
    \begin{enumerate}[label=\roman*., leftmargin=2\parindent]
        \item $\fwt(x, S_{\ell,k}) \le \frac1\beta$ and
        \item $\fval(x, S_{\ell,k}) \ge \frac16 \fval(x, \Gtilde_{\ell,k})$.
    \end{enumerate}

\end{numberedlemma}

\begin{proof}
    Since all jobs in $\Gtilde_{\ell,k,a}$ have value between $2^{a-1}$ and $2^a$, we have
    \[
        2^{a-1}\fwt(x, \Gtilde_{\ell,k,a}) \leq\fval(x,\Gtilde_{\ell,k,a}) \leq
            2^a\fwt(x, \Gtilde_{\ell,k,a}).
    \]
    For every $u\in\{1,\cdots,\lceil\log \vmax\rceil\}$, define
    $S_u=\sum_{a\leq u}2^a\fwt(x,\Gtilde_{\ell,k,a})$.  Then $S_{\lceil\log
    \vmax\rceil}$ is an upperbound of $\fval(x, \Gtilde_{\ell,k})$.  Let $m$ be such that $S_m\leq
    \frac{1}{3}S_{\lceil\log \vmax\rceil}$ and
    $S_{m+1}>\frac{1}{3}S_{\lceil\log \vmax\rceil}$.  Consider the following
    two cases.

    \medskip
    {\em Case 1.} If $S_{m+1}-S_m = 2^{m+1}\fwt(x,\Gtilde_{\ell,k,m+1}) >
    \frac{1}{3}S_{\lceil\log \vmax\rceil}$, then
    \begin{align*}
        \fval(x,\Gtilde_{\ell,k,m+1}) &\geq 2^{m}\fwt(x,\Gtilde_{\ell,k,m+1}) \\
            &> \frac{1}{6}S_{\lceil\log \vmax\rceil} \\
            &\geq\frac16 \fval(x,\Gtilde_{\ell,k}).
    \end{align*}
    Since $\fwt(x, \Gtilde_{\ell,k,m+1})\leq \frac{1}{2\beta}<\frac{1}{\beta}$,
    setting $S_{\ell,k}=\Gtilde_{\ell,k,m+1}$ satisfies both conditions
    of the lemma.

    \medskip
    {\em Case 2.} If $S_{m+1}-S_m\leq \frac{1}{3}S_{\lceil\log \vmax\rceil}$,
    then $S_{m+1}\leq\frac{2}{3}S_{\lceil\log \vmax\rceil}$, and
    \begin{align*}
        \sum_{a>m+1}\fval(x,\Gtilde_{\ell,k,a}) &\geq
                \frac{1}{2}(S_{\lceil\log \vmax\rceil}-S_{m+1}) \\ 
        &\geq \frac{1}{6}S_{\lceil\log\vmax\rceil} \\
        &\geq \frac16\fval(x,\Gtilde_{\ell,k}).
    \end{align*}
    Meanwhile, note that
    $\sum_{a>m+1}2^a\fwt(x,\Gtilde_{\ell,k,a}) < \frac{2}{3}S_{\lceil\log
    \vmax\rceil} < 2S_{m+1}.$ Then
    \begin{align*}
        \sum_{a>m+1}\fwt(x,\Gtilde_{\ell,k,a}) &\leq
                \frac{1}{2^{m+2}}\sum_{a>m+1} 2^a\fwt(x,\Gtilde_{\ell,k,a}) \\
        &< \frac{1}{2^{m+2}}\cdot2S_{m+1} \\
        &= \frac{1}{2^{u+1}}\sum_{a\leq m+1}2^a\fwt(x,\Gtilde_{\ell,k,a}) \\
        &\leq \frac{1}{2^{u+1}}\sum_{a\leq m+1}2^a\frac{1}{2\beta} < \frac{1}{\beta}.
    \end{align*}
    Thus setting $S_{\ell,k}=\Union_{a>m+1}\Gtilde_{\ell,k,a}$ satisfies
    both conditions of the lemma.
\end{proof}

\subsection{Proof of Lemma~\ref{lem:light-bound}}

\begin{numberedlemma}{\ref{lem:light-bound}}
    There exists a fractional unit allocation $\tilde{x}_\light$ such that 
    \[
        \sum_{G\in\light} \fval(x_G) \le
        O\left(\frac{\log L}{\log\beta}\right)\fval(\tilde{x}_\light).
    \]
\end{numberedlemma}

\begin{proof}
  Fix any length scale $\ell$ and consider the partition of items into
  intervals $\Int_{\ell,k}$. We will construct a fractional unit
  allocation corresponding to this partition. We first associated with
  the interval $\Int_{\ell,k}$ the set of all jobs contained inside
  this interval that have length scale between $\ell-\log\beta+1$ and
  $\ell$. Formally, we define
    \begin{equation*}
        H_{\ell,k}=\Union_{\substack{ \ell',k':
            \Int_{\ell',k'}\subseteq \Int_{\ell,k} \\ \ell'\in
            (\ell-\log\beta, \ell]}} S_{\ell',k'},
    \end{equation*}
   where $S_{\ell',k'}$ is defined in
   Lemma~\ref{lem:a-cell-bound}. See Figure~\ref{fig:intervalblock} for 
   a graphical illustration of $H_{\ell,k}$.
   Let $H_\ell = \cup_k H_{\ell,k}$
   denote the set of all jobs that belong to the partition given by
   $\{H_{\ell,k}\}$. 

   \begin{figure}
       \centering
    \newcommand{\cellwidth}{0.5 cm}
    \newcommand{\cellheight}{0.8 cm}

    \begin{tikzpicture}
        [x=\cellwidth,y=-\cellheight,node distance=0 cm,outer sep=0 pt,line
        width=0.66pt]

        \tikzstyle{cell}=[rectangle,draw,
            minimum height=\cellheight,
            anchor=north west,
            text centered]
        \tikzstyle{w16}=[cell,minimum width=16*\cellwidth]
        \tikzstyle{w8}=[cell,minimum width=8*\cellwidth]
        \tikzstyle{w4}=[cell,minimum width=4*\cellwidth]
        \tikzstyle{w2}=[cell,minimum width=2*\cellwidth]
        \tikzstyle{w1}=[cell,minimum width=1*\cellwidth]
        \tikzstyle{job}=[rectangle,draw,fill=gray!30,anchor=north west]

        % top rows
        \node[anchor=north] at (-2, 0.2) {level $\ell$};
        \node[anchor=north] at (-2, 1.2) {level $\ell-1$};
        \node[anchor=north] at (-2, 2.2) {level $\ell-2$};
        \node[anchor=north] at (-3, 4.7) {level $\ell-\log \beta+1$};
        \node[w16] at (0, 0) {$\Int_{\ell,k}$};
        \node[w8]  at (0, 1) {$\Int_{\ell-1,2k}$};
        \node[w8]  at (8, 1) {$\Int_{\ell-1,2k+1}$};
        \node[w4]  at (0, 2) {$\Int_{\ell-2, 4k}$};
        \foreach \k in {1,2,3} {
            \node[w4] at (4*\k, 2) {$\Int_{\ell-2, 4k+\k}$};
        }

        % horizontal continuation...
        \foreach \y in {0,1,2,3, 4.5,5.5} {
            \draw[dotted] (0, \y) -- (-1,\y);
            \draw[dotted] (16,\y) -- (17,\y);
        }

        % gap
        \foreach \x in {0,4,...,16} {
            \draw[dotted] (\x,3) -- (\x,3.5);
            \draw[dotted] (\x,4.5) -- (\x,4);
        }

        % bottom row
        \foreach \x in {0,...,15} {
            \node[w1] at (\x, 4.5) {};
        }

    \end{tikzpicture}
    \caption{Intervals associated with $H_{\ell,k}$ in
       Lemma~\ref{lem:light-bound}. There are $\beta-1$ intervals in the block,
       and from each interval $\Int_{\ell',k'}\subseteq \Int_{\ell,k}$ we pick
       a set of jobs $S_{\ell',k'}$ with fractional weight at most $1/\beta$.}
    \label{fig:intervalblock}
\end{figure}

   We now claim that $x_{H_\ell}$ is a fractional unit allocation. To
   see this, observe that for each $(\ell,k)$ there are exactly
   $2^{\log\beta}-1<\beta$ pairs $(\ell',k')$ with
   $\Int_{\ell',k'}\subseteq \Int_{\ell,k}$ and
   $\ell'\in (\ell-\log\beta, \ell]$. $H_{\ell,k}$ is therefore a
   union of at most $\beta$ groups
   $S_{\ell',k'}$. Lemma~\ref{lem:a-cell-bound} now implies that
   $\fwt(x, H_{\ell,k})\le \beta \frac 1\beta = 1$.

   Lemma~\ref{lem:a-cell-bound} also implies that the total fractional
   value captured by $x_{H_\ell}$ is a constant fraction of the total
   value of all light weight groups at length scales in
   $(\ell-\log\beta, \ell]$: 
   \begin{align*}
     \fval(x,H_\ell) = \sum_k \fval(x,H_{\ell,k}) & = \sum_k 
       \sum_{\ell'\in (\ell-\log\beta, \ell]} \fval(x, S_{\ell',k})\\
     & \ge \frac 16 \sum_k 
       \sum_{\ell'\in (\ell-\log\beta, \ell]}\fval(x, \Gtilde_{\ell',k})\\
     & = \frac 16 \sum_{\substack{G\in \light;\\ G \text{ at length scale } \ell'\in (\ell-\log\beta, \ell]}}\fval(x_G)
   \end{align*}

   Now consider the $\lceil\lmax/\log\beta\rceil$ length scales in $\{1, \cdots,
   \lmax\}$ that are multiples of $\log \beta$. By our argument above,
   the fractional allocations $x_{H_\ell}$ corresponding to such
   length scales together capture a sixth of all of the
   fractional value in $\light$. Therefore, there exists some $\ell$
   such that 
   \[\fval(x,H_\ell)\ge \Omega\left(\frac
     {\log\beta}{\lmax}\right)\sum_{G\in\light} \fval(x_G).\]
   The corresponding unit allocation $x_{H_\ell}$ satisfies the
   requirements of the lemma.
\end{proof}
% Let $H_{\ell,k}$ denote
%   the set of all jobs belonging to light weight groups $G\in \light$
%   that belong to a length scale 

%     \ytnote{Notation too messy, need clean up}
%     Notice that for any interval $\Int_{\ell,k}$ of length scale $\ell$, there
%     are $2^{\log\beta}-1<\beta$ intervals of length scale between
%     $\ell-\log\beta+1$ and $\ell$ which are subintervals of $\Int_{\ell,k}$.
%     Consider the following set of jobs:
%     \begin{equation*}
%         H_{\ell,k}=\Union_{\ell',k':\ell-\log\beta+1 \leq \ell'\leq \ell,
%         \Int_{\ell',k'}\subseteq \Int_{\ell,k}}\Gtilde_{\ell',k'},
%     \end{equation*}
%     where $\Gtilde_{\ell',k'}$ is defined in Lemma~\ref{lem:a-cell-bound}. More
%     intuitively, $H_{\ell,k}$ is the set of light-weighted jobs selected by
%     Lemma~\ref{lem:a-cell-bound} from intervals within $\log \beta$ scales from
%     it, while the total fractional weight of the jobs is less than 1.  Then
%     $\fwt(x, H_{\ell,k})\leq 1$, while 
%     \begin{equation*}
%         \fval(x,H_{\ell,k}) \geq \frac16\sum_{\ell',k':\ell-\log\beta+1\leq
%         \ell'\leq \ell, \Int_{\ell',k'}\subseteq
%         \Int_{\ell,k}}\fval(x,\light\cap G_{\ell',k'}).
%     \end{equation*}
%     That is, jobs in $H_{\ell,k}$ provide a $\beta$-approximation to the
%     fractional value of light-weight jobs from intervals within $\log \beta$
%     scales from $\Int_{\ell,k}$. Let $H_{\ell}=\Union_k H_{\ell,k}$. Then
%     $x_{H_{\ell}}$ is a fractional unit allocation (with respect to intervals
%     $\Int_{\ell,k}$ for every $k$), with fractional value being a good
%     approximation of all light-weight intervals within length scale
%     $(\ell-\log\beta,\ell]$:
%     \begin{equation*}
%         \fval(x,H_{\ell}) \geq
%         \frac16\sum_{\ell',k':\ell-\log\beta+1\leq \ell'\leq \ell}\fval(x,\light\cap G_{\ell',k'}).
%     \end{equation*}
%     Since there are $O(\log L)$ length scales in total, there must exist an
%     $\ell$ such that the fractional value of all light-weight intervals within
%     length scale $(\ell-\log\beta,\ell]$ is $\Omega\left(\frac{\log \beta}{\log
%     L}\right)$ of $\sum_{G\in\light}\fval(x_G)$. Let
%     $\tilde{x}_\light=x_{H_{\ell}}$, then $\tilde{x}_\light$ is a fractional
%     unit allocation, while
%     \[
%         \sum_{G\in\light}\fval(x_G) \le
%                 O\left(\frac{\log L}{\log\beta}\right)\fval(\tilde{x}_\light).
%     \]
% \end{proof}

\subsection{Proof of Lemma~\ref{lem:greedy-layer}}

\begin{numberedlemma}{\ref{lem:greedy-layer}}
    For any feasible fractional allocation $x$ in the interval preferences
    setting with arbitrary capacities, one can efficiently
    construct a set $S$ of jobs such that the total fractional weight of
    $x_S$ at any item $t$ is at least $\min\{1,B_t\}$ and at most $4$.
    Formally, for all items $t$, $\min\{1,B_t\}\le \sum_{j\in S: t\in
    \intj} x_j < 4$.
\end{numberedlemma}

\begin{proof}
    We build up the set $S$ greedily as follows. On each iteration,
    starting from the earliest item $t$ which is not fully covered, we add
    the job whose interval contains $t$ and ends the latest. More formally,
    let $S_0 = \emptyset$. For any set $S$, let $(x,S)|_t := \sum_{j \in S
    : \intj \ni t} x_j$. On iteration $i$, find $t_i = \min \{t:
    (x,S_{i-1})|_t < \min(1, \capt)\}$. Find $j_i = \argmax\{t \in \intj :
    j \not\in S_{i-1}, t_i \in \intj\}$, and set $S_i = S_{i-1} \union
    \{j_i\}$.  Let $S = S_{i^*}$, where $i^*$ is the last iteration.

    By construction, $\min\{1,\capt\}\le\sum_{j\in S: t\in\intj}x_j$ for all
    $t$.  Suppose, by way of contradiction, that there exists $t'$ such that
    $\sum_{j\in S: t'\in\intj}x_j \ge 4$. Let $\{j_1,j_2,\cdots\} \subseteq S$
    be all jobs in $S$ containing $t'$, indexed in the order in which they were
    added to $S$. Let $a$ be such that $1 \le \sum_{i=1}^ax_{j_i} < 2$, and let
    $U_< = \{j_1,\cdots,j_a\}$. Similarly, sort the jobs such that they are
    indexed in order by decreasing end time; let $b$  be such that $1 \le
    \sum_{i=1}^bx_{j_i} < 2$, and let $U_> = \{j_1,\cdots,j_b\}$. Note that
    $\sum_{j\in U_< \union U_>} x_j < 4$, so there exists a job $j'$ which is
    not in either $U_<$ or $U_>$. This job was added to $S$ after all jobs in
    $U_<$ were added, and it ends before all jobs in $U_>$. Let $i'$ be the
    iteration on which $j'$ was added to $S$; we will show there is no item
    $t_{i'}$ for which the algorithm would have added $j'$.
    
    Observe that, because the algorithm covers items starting with the earliest
    uncovered item first, at every iteration $i$ the set of covered items
    (i.e., the set $\{t : (x,S_{i-1})|_t \ge \min(1,\capt)\}$) is a prefix of
    $T$, the set of all items. Therefore, by the time the algorithm has added
    $U_<$ to $S$, all items up to and including $t'$ have been covered. So
    $t_{i'}$ cannot come before or be equal to $t'$.  But all jobs in $U_>$
    contain item $t'$, and therefore must have been added to $S$ before job
    $j'$ because they end later. So item $t_{i'}$ cannot come after $t'$.
    Therefore job $j'$ cannot have been added to $S$.
\end{proof}

\subsection{Proof of Lemma~\ref{lem:peeling}}

\begin{numberedlemma}{\ref{lem:peeling}}
    Given a set $U$ of jobs and a non-zero fractional allocation $y$,
    there exist sets of arms $\tlayer \ne \emptyset$ and $D$ such that 
    \begin{enumerate}[label=\roman*.]
        \item $\tlayer \intersect D = \emptyset$;
        \item $\tlayer \intersect D' = \emptyset$, where $D' = \{\armj : j^{a'}
            \in D, a\ne a'\}$;
        %\item $y'_{\tlayer \union D} = 0$,
        %\item $y'$ is feasible for the multiset of items $T \setminus T_l$,
        \item for all $t \in E_\tlayer$, $\fwe(y, \tlayer \union D)\ge
          \min\{1, \fwe(y)\}$; \label{item:lowerbound}
          % $w_t(\tlayer \union D, y) \ge \min\{1, w_t(U, y)\}$
        \item for all $t \in E_\tlayer$, $\fwe(y, \tlayer) \le 7$; and
          % $w_t(\tlayer, y) \le 7$; and
            \label{item:upperbound}
        \item $\fwt(y_\tlayer) \ge 2 \fwt(y_D)$.
    \end{enumerate}
    % where $T_\tlayer = \{t : w_t(\tlayer, y) > 0\}$. 
    Furthermore, $\tlayer$ and $D$ can be found efficiently.
   % Given a set of jobs $U$ and a non-zero feasible fractional allocation $y$,
   %  there exist sets of arms $\tlayer \ne \emptyset$ and $D$ such that 
   %  \begin{enumerate}[label=\roman*.]
   %      \item $\tlayer \intersect D = \emptyset$,
   %      %\item $y'_{\tlayer \union D} = 0$,
   %      %\item $y'$ is feasible for the multiset of items $T \setminus T_l$,
   %      \item for all $t \in T_\tlayer$, $w_t(\tlayer \union D, y) \ge \min\{1, w_t(U, y)\}$,
   %      \item for all $t \in T_\tlayer$, $w_t(\tlayer, y) \le 7$, and
   %      \item $\fwt(y_\tlayer) \ge 2 \fwt(y_D)$,
   %  \end{enumerate}
   %  where $T_\tlayer = \{t : w_t(\tlayer, y) > 0\}$. Furthermore, $\tlayer$ and $D$
   %  can be found efficiently.
\end{numberedlemma}

\begin{proof}
    % If $w_t(U, y) \le 6$ for all $t \in T$, then set $\tlayer = U$, $D =
    % \emptyset$, and $y' = 0$. This satisfies the conditions. So assume that
    % there exists $t$ such that $w_t(U, y) > 6$. Pick such a $t$ with maximal
    % depth $d(t)$. Thus, for all $t'$ in the subtree below $t$, $w_{t'}(U, y)
    % \le 6$.

    % Suppose there is a set of arms $S$ which begin at or above $t$ and
    % terminate at $t$ such that $w_t(S,y) > 3$. Then let $\armj[j_1], \armj[j_2],
    % \cdots, \armj[j_k], \cdots, \armj[j_\ell]$ be a subset of $S$ sorted in
    % increasing order by length such that
    % \[
    %     \sum_{i=1}^{\ell-1}y_{\armj[j_i]} < 3 \le \sum_{i=1}^\ell y_{\armj[j_i]}
    % \]
    % and
    % \[
    %     \sum_{i=k+1}^{\ell} y_{\armj[j_i]} < 1 \le \sum_{i=k}^\ell y_{\armj[j_i]}.
    % \]
    % Then let $\tlayer = \{\armj[j_1], \cdots, \armj[j_k]\}$ and $D =
    % \{\armj[j_{k+1}], \cdots, \armj[j_\ell]\}$.  Because we are considering
    % arms which terminate at $t$, $\intj[{\armj[j_i]}] \subseteq
    % \intj[{\armj[j_{i+1}]}]$ for all $i \in [\ell-1]$. In particular, $w_e(\tlayer
    % \union D)= \sum_{i=1}^{k}y_{\armj[j_i]}\ge 1$ for all $e \in T_\tlayer$.
    % Observe that $\fwt(\tlayer, y) \ge 2 \fwt(D, y)$, so let $y'$ be $y$ with
    % all allocations to arms in $\tlayer$ and $D$ zeroed out. Formally, set $A =
    % \tlayer\union D$, and set $y' = y_{U\setminus A}$. This satisfies the
    % conditions of the lemma.

    % Suppose there is no set $S$ of arms terminating at $t$ such that $w_t(S, y)
    % > 3$. Then there must exist a set of jobs including $t$ but terminating
    % below $t$ with total weight at least $3$. Let $t_1, \cdots, t_d$ be the $d$
    % edges below $t$.  Let $S_i$ be the set of arms that contain both $t$ and
    % $t_i$, and $T_i$ be the set of arms containing only edges within the subtree
    % planted at $t_i$. Because $w_{t_i}(S_i,y) \le 6$ for all $i$ by our choice
    % of $t$, we can find $m$ such that $3 \le \sum_{i=1}^m w_t(S_i,y) < 9$.
    % \[
    %     \sum_{i=1}^{\ell-1}y_{\armj[j_i]} < 6 \le \sum_{i=1}^\ell y_{\armj[j_i]}
    % \]
    % and
    % \[
    %     \sum_{i=k+1}^{\ell} y_{\armj[j_i]} < 2 \le \sum_{i=k}^\ell y_{\armj[j_i]}.
    % \]

    If $\fwe(y) \le 6$ for all $t \in T$, then set $\tlayer = U$ and $D =
    \emptyset$. This satisfies the conditions. So assume that
    there exists $t$ such that $\fwe(y) > 6$. Pick such a $t$ with maximal
    depth $d(t)$. Thus, for all $t'$ in the subtree below $t$, $\fwe[t'](y)
    \le 6$.

    Suppose there is a set of arms $S$ which begin at or above $t$ and
    terminate at $t$ such that $\fwe(y, S) > 3$. Define the {\em depth} of an arm
    to be the depth of the highest edge in its desired interval, so that arms
    with low depth desire intervals beginning higher in the tree. Then let
    $\armj[j_1], \armj[j_2], \cdots, \armj[j_k], \cdots, \armj[j_\ell]$ be a
    subset of $S$ sorted in decreasing order by depth such that
    \[
        \sum_{i=1}^{\ell-1}y_{\armj[j_i]} < 3 \le \sum_{i=1}^\ell y_{\armj[j_i]}
    \]
    and
    \[
        \sum_{i=k+1}^{\ell} y_{\armj[j_i]} < 1 \le \sum_{i=k}^\ell y_{\armj[j_i]}.
    \]
    Then let $\tlayer = \{\armj[j_1], \cdots, \armj[j_k]\}$ and $D =
    \{\armj[j_{k+1}], \cdots, \armj[j_\ell]\}$.  Because we are considering
    arms which terminate at $t$, $\intj[{\armj[j_i]}] \subseteq
    \intj[{\armj[j_{i+1}]}]$ for all $i \in [\ell-1]$. In particular,
    $\fwe[e](y, \tlayer
    \union D)\geq \sum_{i=k}^{\ell}y_{\armj[j_i]}\ge 1$ for all $e \in T_\tlayer$.
    Observe that $\fwt(y_\tlayer) \ge 2 \fwt(y_D)$. %so let $y'$ be $y$ with
    %all allocations to arms in $\tlayer$ and $D$ zeroed out. Formally, set $A =
    %\tlayer\union D$, and set $y' = y_{U\setminus A}$. 
    This satisfies the conditions of the lemma.

    Suppose there is no set $S$ of arms terminating at $t$ such that
    $\fwe(y, S)
    > 3$. Then there must exist a set of arms including $t$ but terminating
    below $t$ with total weight at least $3$. Let $t_1, \cdots, t_d$ be the $d$
    edges below $t$.  Let $S_i$ be the set of arms that contain both $t$ and
    $t_i$, and $T_i$ be the set of arms containing only edges within the
    subtree planted at $t_i$. Because $\fwe[t_i](y, S_i) \le 6$ for all $i$ by our
    choice of $t$, we can find $m$ such that $3 \le \sum_{i=1}^m
    \fwe(y, S_i) <
    9$. Let $S = \Union_{i=1}^m S_i$. Let $c = \fwe(y, S)$, so $3 \le c < 9$.
    Let $\armj[j_1], \cdots, \armj[j_k], \cdots, \armj[j_\ell]$ be the arms in
    $S$ sorted in decreasing order by depth, as before, where 
    \[
        2\sum_{i=k+1}^{\ell} y_{\armj[j_i]} \le \sum_{i=1}^k y_{\armj[j_i]}\leq 7
    \]
    and
    \[
        \sum_{i=k}^{\ell} y_{\armj[j_i]} \geq 1.
    \]

    \begin{figure}[tbp]
    \centering
        \subfigure{
        \begin{minipage}[b]{1.1\textwidth}
            \includegraphics[width=0.33\textwidth]{tree_example1}
            \includegraphics[width=0.33\textwidth]{tree_example2}
            \includegraphics[width=0.33\textwidth]{tree_example3}
 %           \caption{\label{fig:lb-example}\small{An example of a binary tree and its
 %             corresponding grid.}}
        \end{minipage}
        }
        \caption{\small{An example for one step of layering
            algorithm. (Left figure:) At the beginning of the layering
            step, there are 101 jobs: $(s-t-e_1)$, $(t-e_2)$,
            $(t-e_3)$, $\cdots$, $(t-e_{100})$, $(u)$. Each job has
            value 1 and allocation 0.9. The capacity of edge t is 90,
            while the capacity of other edges are 1. (Middle figure:)
            In the first step of the layering process, job $(s-t-e_1)$
            is in set $D$, jobs $(t-e_2)$, $(t-e_3)$, $(t-e_3)$ and
            $(u)$ are in set $\tlayer$. The job in set $D$ is deleted,
            while the jobs in set $\tlayer$ have their weight scaled
            down and allocated to one layer. (Right figure:) The
            capacity of edges after the first step of layering
            procedure. The capacity of edges that have positive weight
            in set $\tlayer$ are decreased by 1.}}
    \end{figure}
 
    Then set $\tlayer = \{\armj[j_i] : i = 1, \cdots, k\} \union T_1 \union
    \cdots \union T_m$ and $D = \{\armj[j_i] : i = k+1, \cdots, \ell\}$. For
    each edge $e\in T_\tlayer$, $\fwe(y, \tlayer)\leq 7$. The same as in
    previous case for each edge $e$ that is not below $t$, $\fwe[e](y,
    \tlayer \union
    D)\geq \sum_{i=k}^{\ell}y_{\armj[j_i]}\ge 1$; for each edge $e$ below $t$
    in $T_\tlayer$, $\fwe[e](y, \tlayer \union D)=\fwe(y)$. $\fwt(y_\tlayer) \ge
    \sum_{i=1}^k y_{\armj[j_i]}\ge 2\sum_{i=k+1}^{\ell}
    y_{\armj[j_i]}=\fwt(y_D)$.  Thus every condition of the lemma is satisfied.
\end{proof}

   %  \begin{figure}[htbp]
   %  \centering
   % \includegraphics[width=0.7\textwidth]{tree_example1}
   %    \caption{\small{An example for one step of layering algorithm. At the beginning of the layering step, there are 101 jobs: $(s-t-e_1)$, $(t-e_2)$, $(t-e_3)$, $\cdots$, $(t-e_{100})$, $(u)$. Each job has value 1 and allocation 0.9. The capacity of edge t is 90, while the capacity of other edges are 1.}}
   %  \end{figure}
   %  \begin{figure}[htbp]
   %  \centering
   %  \includegraphics[width=0.7\textwidth]{tree_example2}
   %    \caption{\small{In the first step of the layering process, job $(s-t-e_1)$ is in set $D$, jobs $(t-e_2)$, $(t-e_3)$, $(t-e_3)$ and $(u)$
   %    are in set $\tlayer$. The job in set $D$ is deleted, while the jobs in set $\tlayer$ have their weight scaled down and allocated to
   %    one layer.}}
   %  \end{figure}
   %  \begin{figure}[htbp]
   %  \centering
   %  \includegraphics[width=0.7\textwidth]{tree_example3}
   %    \caption{\small{The capacity of edges after the first step of layering procedure. The capacity of edges that have positive weight in set 
   %    $\tlayer$ are decreased by 1.}}
   %  \end{figure}



  %  \begin{figure}[htbp]
  %  \centering
  %  \subfigure{
  %  \begin{minipage}[b]{0.4\textwidth}
  %  \centering
  %    \includegraphics[width=\textwidth]{tree_example1}
  %    \caption{At the beginning of a layering step, there are 101 jobs: $(s-t-e_1)$, $(t-e_2)$, $(t-e_3)$, $\cdots$, $(t-e_{100})$, $(u)$. 
  %    Each job has value 1 and allocation 0.9. The capacity of edge t is 90, while the capacity of other edges are 1.}
  %  \end{minipage}
  %  }\\
  %  \subfigure{
  %  \begin{minipage}[b]{0.4\textwidth}
  %  \centering
  %    \includegraphics[width=\textwidth]{tree_example2}
  %    \caption{In the first step of the layering process, job $(s-t-e_1)$ is in set $D$, jobs $(t-e_2)$, $(t-e_3)$, $(t-e_3)$ and $(u)$
  %    are in set $\tlayer$. The job in set $D$ is deleted, while the jobs in set $\tlayer$ have their weight scaled down and allocated to
  %    one layer.}
  %  \end{minipage}
  %  }\\
  %  \subfigure{
  %  \begin{minipage}[b]{0.4\textwidth}
  %  \centering
  %    \includegraphics[width=\textwidth]{tree_example3}
  %    \caption{The capacity of edges after the first step of layering procedure. The capacity of edges that have positive weight in set 
  %    $\tlayer$ are decreased by 1.}
  %  \end{minipage}
  %  }
  %  \caption{An example of a step of layering algorithm.}
  %\end{figure}

%\newpage

\subsection{Proof of Lemma~\ref{lem:fopt-upperbound-costs}}

\begin{numberedlemma}{\ref{lem:fopt-upperbound-costs}}
    $\fopt_{\costs} \ge \opt_{\costs}$.
\end{numberedlemma}
\begin{proof}
    Let $x^*_j$ be the probability that job $j$ is allocated in optimal offline allocation, where
    the randomness is over all possible realizations of the values. Then $x^*$ is a fractional
    allocation of jobs.
    Let $d_{tr}$ be the probability that the $r$th copy of item $t$ is allocated in 
    offline optimal allocation. Then $0\leq d_{tr}\leq 1$, $b_t(x^*)=\sum_{r}d_{tr}$,
    \begin{equation*}
        \opt_{\costs}=\sum_{j}v_jx^*_j-\sum_{t,r}d_{tr}c_{tr}.
    \end{equation*}
    To prove the lemma, it suffices to show that $\fval(x^*,\costs)\geq\opt_{\costs}$, which is equivalent 
    to proving $\sum_{t,r}b_{tr}c_{tr}\leq\sum_{t,r}d_{tr}c_{tr}$. This is true since for every $t$ and $k$,
    $\sum_{r\geq k}d_{tr}\geq\sum_{r\geq k}b_{tr}$, and 
    \begin{eqnarray*}
        \sum_{r\geq 1}d_{tr}c_{tr}&=&c_{t1}\sum_{r\geq 1}d_{tr}+(c_{t2}-c_{t1})\sum_{r\geq 2}d_{tr}+(c_{t3}-c_{t2})\sum_{r\geq 3}d_{tr}+\cdots\\
        &\geq&c_{t1}\sum_{r\geq 1}b_{tr}+(c_{t2}-c_{t1})\sum_{r\geq 2}b_{tr}+(c_{t3}-c_{t2})\sum_{r\geq 3}b_{tr}+\cdots=\sum_{r\geq 1}b_{tr}c_{tr}.
    \end{eqnarray*} 
\end{proof}

\subsection{Proof of Lemma~\ref{lem:FGL-costs}}

\begin{numberedlemma}{\ref{lem:FGL-costs}}
    For any cost vector $\costs$ and a fractional unit allocation with
    costs, $(x,\tau)$, there exists an anonymous bundle pricing
    $\prices$ such that
    \[
        \sw(\prices) \ge \half \fval(x,\tau,\costs).
    \]
\end{numberedlemma}
\begin{proof}
    We will give a reduction to the fixed-capacity setting without costs and
    apply Lemma~\ref{lem:FGL}.  Recall that in a fractional unit allocation
    with costs, each job $j$ with $x_j > 0$ is explicitly associated with a
    particular bundle, and each bundle is explicitly associated with a
    particular copy of each item contained in the bundle.  Thus, the allocation
    defines a unique charging of the total cost to individual jobs. We
    construct a pricing instance in the fixed-capacity setting by subtracting
    from each job's value the cost of serving it. That is, for every job $j$ in
    the instance with costs, we construct a job $j'$ in the instance without
    costs with value 
    \begin{equation*}
        v_{j'}=v_j-\sum_{t\in I, (t,r)\in \tau_k}c_{tr},
    \end{equation*}
    where $k$ is such that $j \in A_k$ (i.e., the fractional unit allocation
    with costs assigns job $j$ to bundle $k$).  Thus the two instances have
    identical fractional values, i.e.  $\fval'(x)=\fval(x,\tau,\costs)$. By
    Lemma~\ref{lem:FGL}, there exists a static, anonymous bundle pricing
    $\prices'$ such that
    \begin{equation*}
        \sw'(\prices') \ge \half \fval'(x),
    \end{equation*}
    and each set $\tau_k$ is assigned a price $\price'_k$. We use this bundle
    pricing to associate the following prices $\prices$ to the original
    instance with costs. For each bundle $\tau_k$, for each interval $I
    \subseteq T_k$, set price
    \[
        \price_{\tau_k,I} = \price'_k + \sum_{t\in I, (t,r)\in \tau_k} c_{tr}.
    \]
    We refer to the price $\price'_k$ as the base price for bundle $\tau_k$,
    and the sum over costs as the surcharge for interval $I$.
    % For every interval $I$, its price is
    % $\min_{k:T_k\supseteq I}\prices'(\tau_k)+\sum_{t\in I, (t,r)\in
    % \tau_k}c_{tr}$ in the beginning.

    Our menu lists a price for every subinterval of every bundle, covering
    every copy of every item.\footnote{In practice, of course, for each
    interval the menu can list just the minimum price over all unsold bundles.}
    However, the mechanism sells at most one interval per bundle: when a buyer
    selects her preferred interval, the mechanism removes all intervals
    corresponding to the same bundle from the menu. Thus, our menu is adaptive
    (prices of unsold intervals change in response to demand for other
    intervals), though it remains anonymous.
    % Our adaptive bundle pricing runs as follows. Say a set $\tau_k$ is
    % purchased by job $(j,v_j,I_j)$ if when $j$ comes, it is allocated and
    % charged price $\prices'(\tau_k)+\sum_{t\in I_j, (t,r)\in \tau_k}c_{tr}$.
    % Once a set is purchased, the price of each interval $I$ is updated to 
    % \begin{equation*}
    %     \prices(I) = \min_{k:T_k\supseteq I, \tau_k\textrm{ unsold}} \prices'(\tau_k)+\sum_{t\in I, (t,r)\in \tau_k}c_{tr}.
    % \end{equation*}

    We now claim that the social welfare generated by pricing $\prices$ in the
    instance with costs is precisely equal to the social welfare generated by
    pricing $\prices'$ for the new fixed-capacity instance. We prove this
    pointwise by coupling the buyers' values and arrival order in the two
    settings, and then by induction over the number of buyers that have already
    arrived. Our inductive hypothesis is that at any point of time, the set of
    bundles available in the new instance is exactly the same as the set of
    unsold bundles in the original instance.  By the way we
    construct buyers' values in the new instance, buyers' preferences over
    bundles in the two instances are exactly coupled. Thus, the buyers' total
    utility is the same in each instance. Furthermore, the revenue generated by
    the base price of each bundle is equal to the revenue generated in the
    fixed-capacity instance, and the surcharge covers the production costs.

    This concludes the proof of the lemma.
\end{proof}

\subsection{Proof of Lemma~\ref{lem:arbit-cost}}

\begin{numberedlemma}{\ref{lem:arbit-cost}}
    For all non-decreasing costs $\costs$ and every fractional allocation $x$,
    there exists a fractional unit allocation $(x',\tau)$ such that
    \[
        \fval(x,\costs)\le  O(\log L/\log\log L)\fval(x',\tau,\costs).
    \]
\end{numberedlemma}

\begin{proof}
  We partition the multiset of items as well as the fractional
  allocation $x$ into layers using the procedure described in the
  proof of Theorem~\ref{thm:arbit-capacity} in
  Section~\ref{sec:arbit-cap-ub}. Let $Y_r$ denote the $r$th layer and
  $x^{(r)}$ denote the unscaled fractional allocation associated with
  layer $r$, so that $\sum_r x^{(r)} = x$. Let $S^{(r)}$ denote the
  set of jobs associated with $x^{(r)}$, so that
  $x^{(r)} = x_{S^{(r)}}$. Observe that by the manner in which we
  create layers, the sets $S^{(r)}$ partition the set of all jobs.

  Let
  $b'_{tr} = \sum_{j\in S^{(r)}: \intj \ni t}x_j = \sum_{j: \intj \ni
    t} x^{(r)}_j$ be the total fractional weight of jobs containing
  item $t$ in level $r$. Let $\hat{r}_t$ be the maximum layer index
  $r$ such that $b'_{tr}>0$. By the nature of the layering process,
  every layer below $\hat{r}_t$ is ``filled'': $1 \le b'_{tr} < 4$ for
  every $r < \hat{r}_t$. Then
  $\sum_{r\geq 1}b'_{tr}=\sum_{r\geq 1}b_{tr}=b_t$, and
  $\sum_{r\geq k}b'_{tr}\le \sum_{r\geq k}b_{tr}$ for every $k>1$,
  where the latter inequality follows by recalling that $b_{tr}\le 1$
  and $b'_{tr}\ge 1$ for all $r$. If we associate cost $c_{tr}$ with
  item $t$ in all jobs assigned to layer $r$, then the total cost of
  item $t$ is
\begin{eqnarray*}
\sum_{r\geq 1}b'_{tr}c_{tr}&=&c_{t1}\sum_{r\geq 1}b'_{tr}+(c_{t2}-c_{t1})\sum_{r\geq 2}b'_{tr}+(c_{t3}-c_{t2})\sum_{r\geq 3}b'_{tr}+\cdots\\
&\leq&c_{t1}\sum_{r\geq 1}b_{tr}+(c_{t2}-c_{t1})\sum_{r\geq 2}b_{tr}+(c_{t3}-c_{t2})\sum_{r\geq 3}b_{tr}+\cdots=\sum_{r\geq 1}b_{tr}c_{tr},
\end{eqnarray*} 
    with the last term being exactly the cost of item $t$ in
    fractional allocation $x$. 

    Now consider the fractional allocation $\xr$ for each layer
    $\layer$, defined as $\xr=x^{(r)}/4$. This is a supply feasible
    allocation of items in layer $\layer$. Furthermore, summing over
    all layers, the sum of jobs' valuations under this suite of
    allocations is exactly a quarter of that under $x$, whereas the
    cost is exactly a quarter of the cost
    $\sum_{r\geq 1}b'_{tr}c_{tr}$ defined above for the unscaled
    allocations $x^{(r)}$. We therefore conclude that
    \[\sum_r \fval(\xr) = \frac 14\sum_r \left(\sum_{j\in S^{(r)}} v_jx_j - b'_{tr}c_{tr}\right) \ge \frac{1}{4} \fval(x).\]

    We will now convert each $\xr$ into a unit allocation
    corresponding to layer $\layer$ by reducing it to an appropriate
    unit-capacity setting without costs. Observe that we can write
    \[\sum_r \fval(\xr) = \frac 14\sum_r \left(\sum_{j\in S^{(r)}} v_jx_j - b'_{tr}c_{tr}\right) = \frac{1}{4}\sum_{j\in S^{(r)}} \Big(v_j-\sum_{t\in I_j}c_{tr}\Big)x_j.\]

    Define a unit-capacity setting over the set of items $\layer$,
    where the value of a job $j$ is given by
    $v'_j=v_j-\sum_{t\in I_j}c_{tr}$. The allocation $\xr$ is
    demand-feasible for this setting. We can therefore apply
    Theorem~\ref{thm:unit-capacity} to obtain a unit allocation $\xpr$
    for this setting with the property that
    $\fval(\xr)\le O(\log L/\log\log L) \fval(\xpr)$. Let
    $\{T_{1r}, T_{2r}, \cdots\}$ denote the partition of items in
    $\layer$ corresponding to this unit allocation. Set
    $\tau_{kr} = \{(t,r): t\in T_{kr}\}$. Then the pair
    $(\sum_r \xpr,\{\tau_{k,r}\})$ forms a unit allocation for our
    original setting that obtains the claimed welfare bound.
\end{proof}

