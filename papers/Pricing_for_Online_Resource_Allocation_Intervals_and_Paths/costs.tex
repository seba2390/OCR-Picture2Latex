\section{Convex Production Costs with Interval Preferences}
\label{sec:costs}

In this section we show that our results for interval preferences, in
particular Theorems~\ref{thm:unit-cap-ub} and \ref{cor:large-cap-ub},
generalize to the setting where the seller can obtain additional
copies of each item at increasing marginal costs. In particular, we
assume the seller incurs cost $\cti$ when allocating\footnote{Note
  that the seller can acquire items online; supply need not be
  provisioned before demand is realized.} the $i$th copy of item $t$,
and $\cti \ge \cti[ti']$ for $i > i'$.  Note that this generalizes the
fixed-capacity setting by taking $\cti = 0$ for $i \le B_t$ and
$\infty$ otherwise.  We begin by revisiting our definition of
fractional allocations and associated notation, which we defined for
fixed-capacity settings in \Cref{sec:prelim}.

% Our most general results apply to a setting in which the seller can acquire
% additional copies of each item at non-decreasing cost. 

In this setting, a fractional allocation $x$ no longer need satisfy an explicit
supply constraint, but the fractional value achieved by $x$ is different, since
the costs of supply must be taken into account.  We begin by introducing
notation for the cost of a given fractional allocation. For any fractional
allocation $x$, denote by $b_t(x)=\sum_{j:\intj \ni t}x_j$ the (fractional)
number of copies of item $t$ allocated in $x$. Let $b_{tr}(x)$ be the fraction
of copy $r$ of item $t$ sold:
\[
    b_{tr}(x) = \begin{cases}
        1 & r\leq b_t(x) \\
        b_t(x)-\lfloor b_t(x)\rfloor & r = \lfloor b_t(x)\rfloor+1 \\
        0 & r>\lfloor b_t(x)\rfloor+1.
    \end{cases}
\]
For simplicity we use $b_t$ to denote $b_t(x)$ and $b_{tr}$ to denote
$b_{tr}(x)$ in later analysis. Observe that the total cost for all copies of
item $t$ in $x$ is $\sum_{r}b_{tr}c_{tr}$.  Then the fractional value achieved
by $x$ is
\[
    \fval(x,\costs) = \sum_j v_jx_j - \sum_{t,r}b_{tr}c_{tr}.
\]
As in the fixed-capacity setting, we bound the optimal welfare by the optimal
fractional welfare; define
\[
    \fopt := \max_{x} \fval(x,\costs),
\]
where $x$ ranges over $x \ge 0$ satisfying the demand constraint
(Equation~\ref{eq:demand}).\footnote{Although not stated here in this
  manner, $\fopt$ can be expressed as the optimum of a linear program.}  We defer our proof of
Lemma~\ref{lem:fopt-upperbound-costs} to \Cref{sec:deferred}.
\begin{lemma}
    \label{lem:fopt-upperbound-costs}
    $\fopt_{\costs} \ge \opt_{\costs}$.
\end{lemma}

We now show that our analysis (theorems in
\Cref{sec:ub-interval} along with Lemma~\ref{lem:FGL}) extends to the
setting with convex production costs as well. We begin by updating our
definition of fractional unit allocations. Recall that unit
allocations implicitly define a partition of items into bundles and
associate each job with at most one bundle. When items have production
costs, the value of a fractional unit allocation depends on which copy
of each item belongs to any particular bundle in the partition. This
association of copies to bundles is not necessarily unique. So we
extend the definition of unit allocations to make the partitioning of
items into bundles explicit. 

\begin{definition} A {\em fractional unit allocation with costs} is a
  pair $(x,\tau)$, where $x$ is a demand-feasible fractional allocation and
  $\tau= \{\tau_1, \tau_2, \tau_3, \cdots\}$ is a partition of the multiset of
  items $\{(t,r)\}_{t\in T, r\in\Z^+}$, if there exists a partition of
  jobs $j\in U$ with $x_j>0$ into sets $\{A_1, A_2, A_3, \cdots\}$,
  such that:
    \begin{itemize}
    \item For all $k$, the set of items contained in $\tau_k$, $T_k=\{t:
      (t,r)\in \tau_k\}$, forms an interval.
        \item For all $j\in U$ with $x_j>0$, there is exactly one
          index $k$ with $j\in A_k$. 
        \item For all $k$ and $j\in A_k$, $I_j\subseteq T_k$.
        \item For all $k$, we have $\fwt(x_{A_k}) \le 1$.
% there exists an index $k_j$ with
%             $\intj\subseteq T_{k_j}$ and $\intj\cap T_{k'}=\emptyset$ for all
%             $k'\ne k_j$.
%         \item Denoting $A_k = \{j: k_j = k\}$, we have $\fwt(x_{A_k}) \le 1$.
    \end{itemize}
    The fractional value of $(x,\tau)$ is defined as:
    \[\fval(x,\tau,\costs) = \sum_j v_jx_j - \sum_k \sum_{(t,r)\in
        \tau_k}  \sum_{j\in A_k: t\in I_j} c_{tr}x_j.\]
\end{definition}

\noindent
We can now state a counterpart to Lemma~\ref{lem:FGL}, with the proof
deferred to \Cref{sec:deferred}.
\begin{lemma}
    \label{lem:FGL-costs}
    For any cost vector $\costs$ and a fractional unit allocation with
    costs, $(x,\tau)$, there exists an anonymous bundle pricing
    $\prices$ such that
    \[
        \sw(\prices) \ge \half \fval(x,\tau,\costs).
    \]
\end{lemma}

We emphasize that the pricing returned by the above lemma is not
necessarily static. For each bundle $\tau_k$ in the unit allocation
$(x,\tau)$, our pricing associates a price with each interval
$I\subseteq T_k$. Prices for different such subsets $I$ can be
different and depend on the costs of the respective items. When one of
the intervals in $\tau_k$ is bought, the seller removes from the menu
all other intervals corresponding to $\tau_k$. Moreover, different
copies of the same bundle can have different prices, because they cost
different amounts to the seller. We leave open the question of
obtaining a static bundle pricing with a good approximation factor for
this setting.

Next we state and prove a counterpart of
Theorem~\ref{thm:arbit-capacity}: given any fractional allocation $x$
we can construct a fractional unit allocation that captures an
$O(\log L/\log\log L)$ fraction of the value of $x$. The proof is
essentially a reduction to the fixed-capacity setting. We first
partition the allocation $x$ into ``layers'' as in the proof of
Theorem~\ref{thm:arbit-capacity}. The fractional weight for every item
in the allocation for any single layer is at most 1. In this case, it
becomes possible to apply the unit-capacity theorem
(Theorem~\ref{thm:unit-capacity}) to construct a
partition of items in that layer and a corresponding unit
allocation. Putting these layer-by-layer unit-allocations together
gives us an overall fractional unit allocation.


% Next, we show that Theorem~\ref{thm:arbit-capacity}, which asserts the
% existence of a good unit allocation, generalizes to the setting with costs.
% The proof is essentially the same as in the fixed-capacity setting: we first
% prove that if none of the items has fractional weight greater than 1, then it
% is possible to reduce this case to the setting of unit capacity without cost.
% Otherwise, we partition the fractional allocation into layers, and each job is
% assigned to one layer, while the total cost does not increase. 

% \begin{lemma}
%     \label{lem:unit-cost}
%     For all non-decreasing costs $\costs$ and every fractional allocation $x$
%     such that $\max_t b_t(x)\leq 1$, there exists a fractional unit allocation
%     $(x',\tau)$, such that
%     \[
%         \fval(x,\costs)\le  O(\log L/\log\log L)\fval(x',\tau,\costs).
%     \]
% \end{lemma}

% \begin{proof}
%     Notice that if $\max_t b_t(x)\leq 1$, then
%     \[
%         \fval(x,\costs) = \sum_j v_jx_j - \sum_{r}b_{tr}c_{tr} =
%                 \sum_j \Big(v_j-\sum_{t\in I_j}c_{t1}\Big)x_j.
%     \]
%     Thus the problem reduces to the unit-capacity setting, where the value of
%     each job $j$ is replaced by $v'_j=v_j-\sum_{t\in I_j}c_{t1}$. We
%     can then apply Theorem~\ref{thm:unit-capacity} to obtain a
%     fractional unit allocation. 
%     % The
%     % correctness of the lemma now follows immediately from
%     % Theorem~\ref{thm:unit-capacity}.
% \end{proof}


\begin{lemma}
    \label{lem:arbit-cost}
    For all non-decreasing costs $\costs$ and every fractional allocation $x$,
    there exists a fractional unit allocation $(x',\tau)$ such that
    \[
        \fval(x,\costs)\le  O(\log L/\log\log L)\fval(x',\tau,\costs).
    \]
\end{lemma}

% \begin{proof}
%   We partition the multiset of items as well as the fractional
%   allocation $x$ into layers using the procedure described in the
%   proof of Theorem~\ref{thm:arbit-capacity} in
%   Section~\ref{sec:arbit-cap-ub}. Let $Y_r$ denote the $r$th layer and
%   $x^{(r)}$ denote the unscaled fractional allocation associated with
%   layer $r$, so that $\sum_r x^{(r)} = x$. Let $S^{(r)}$ denote the
%   set of jobs associated with $x^{(r)}$, so that
%   $x^{(r)} = x_{S^{(r)}}$. Observe that by the manner in which we
%   create layers, the sets $S^{(r)}$ partition the set of all jobs.

%   Let
%   $b'_{tr} = \sum_{j\in S^{(r)}: \intj \ni t}x_j = \sum_{j: \intj \ni
%     t} x^{(r)}_j$ be the total fractional weight of jobs containing
%   item $t$ in level $r$. Let $\hat{r}_t$ be the maximum layer index
%   $r$ such that $b'_{tr}>0$. By the nature of the layering process,
%   every layer below $\hat{r}_t$ is ``filled'': $1 \le b'_{tr} < 4$ for
%   every $r < \hat{r}_t$. Then
%   $\sum_{r\geq 1}b'_{tr}=\sum_{r\geq 1}b_{tr}=b_t$, and
%   $\sum_{r\geq k}b'_{tr}\le \sum_{r\geq k}b_{tr}$ for every $k>1$,
%   where the latter inequality follows by recalling that $b_{tr}\le 1$
%   and $b'_{tr}\ge 1$ for all $r$. If we associate cost $c_{tr}$ with
%   item $t$ in all jobs assigned to layer $r$, then the total cost of
%   item $t$ is
% \begin{eqnarray*}
% \sum_{r\geq 1}b'_{tr}c_{tr}&=&c_{t1}\sum_{r\geq 1}b'_{tr}+(c_{t2}-c_{t1})\sum_{r\geq 2}b'_{tr}+(c_{t3}-c_{t2})\sum_{r\geq 3}b'_{tr}+\cdots\\
% &\leq&c_{t1}\sum_{r\geq 1}b_{tr}+(c_{t2}-c_{t1})\sum_{r\geq 2}b_{tr}+(c_{t3}-c_{t2})\sum_{r\geq 3}b_{tr}+\cdots=\sum_{r\geq 1}b_{tr}c_{tr},
% \end{eqnarray*} 
%     with the last term being exactly the cost of item $t$ in
%     fractional allocation $x$. 

%     Now consider the fractional allocation $\xr$ for each layer
%     $\layer$, defined as $\xr=x^{(r)}/4$. This is a supply feasible
%     allocation of items in layer $\layer$. Furthermore, summing over
%     all layers, the sum of jobs' valuations under this suite of
%     allocations is exactly a quarter of that under $x$, whereas the
%     cost is exactly a quarter of the cost
%     $\sum_{r\geq 1}b'_{tr}c_{tr}$ defined above for the unscaled
%     allocations $x^{(r)}$. We therefore conclude that
%     \[\sum_r \fval(\xr) = \frac 14\sum_r \left(\sum_{j\in S^{(r)}} v_jx_j - b'_{tr}c_{tr}\right) \ge \frac{1}{4} \fval(x).\]

%     We will now convert each $\xr$ into a unit allocation
%     corresponding to layer $\layer$ by reducing it to an appropriate
%     unit-capacity setting without costs. Observe that we can write
%     \[\sum_r \fval(\xr) = \frac 14\sum_r \left(\sum_{j\in S^{(r)}} v_jx_j - b'_{tr}c_{tr}\right) = \frac{1}{4}\sum_{j\in S^{(r)}} \Big(v_j-\sum_{t\in I_j}c_{tr}\Big)x_j.\]

%     Define a unit-capacity setting over the set of items $\layer$,
%     where the value of a job $j$ is given by
%     $v'_j=v_j-\sum_{t\in I_j}c_{tr}$. The allocation $\xr$ is
%     demand-feasible for this setting. We can therefore apply
%     Theorem~\ref{thm:unit-capacity} to obtain a unit allocation $\xpr$
%     for this setting with the property that
%     $\fval(\xr)\le O(\log L/\log\log L) \fval(\xpr)$. Let
%     $\{T_{1r}, T_{2r}, \cdots\}$ denote the partition of items in
%     $\layer$ corresponding to this unit allocation. Set
%     $\tau_{kr} = \{(t,r): t\in T_{kr}\}$. Then the pair
%     $(\sum_r \xpr,\{\tau_{k,r}\})$ forms a unit allocation for our
%     original setting that obtains the claimed welfare bound.
% \end{proof}

\noindent
Lemmas~\ref{lem:fopt-upperbound-costs}, \ref{lem:FGL-costs} and
\ref{lem:arbit-cost} together imply the following upper bound for the
costs setting.

%\begin{numberedtheorem}{\ref{thm:costs-ub}}
\begin{theorem}
\label{thm:costs-ub}
For the interval preferences setting with increasing marginal costs on
items, there exists an anonymous bundle pricing $\prices$ such that
\[
    \opt \le O\left(\frac{\log L}{\log\log L}\right)\sw(\prices).
\]
\end{theorem}
%\end{numberedtheorem}

The lower bound of $\Omega\left(\frac{\log L}{\log\log L}\right)$ also extends
to the setting with costs, as the unit-capacity setting is a special case.
