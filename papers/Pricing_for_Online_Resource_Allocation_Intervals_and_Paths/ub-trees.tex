\section{Bundle Pricings for Path Preferences}
\label{sec:trees-ub}

We now turn to the path preferences setting. Recall that here the
items correspond to edges in a fixed tree. Each buyer $i$ desires a
path $P_i\subset T$. Our techniques for this setting are similar to
those in Section~\ref{sec:ub-interval} --- we construct a partition of
the items into layers, construct a corresponding fractional allocation
that respects this layer structure in that every buyer is allocated
items from at most two layers, and no layer has too much fractional
weight. We show in Section~\ref{sec:layered-alloc} that such a good
{\em fractional layered allocation} always exists. In
Section~\ref{sec:trees-unit-cap} we show that a layered allocation
guarantees a good bundle pricing. In
Section~\ref{sec:trees-large-cap-ub} we present the corresponding
``large markets'' result, namely that the competitive ratio decreases
linearly with capacity.

\subsection{Fractional layered allocations}
\label{sec:ub-tree-layered}

% Here, however, we must deal with branching in
% the graph, which introduces additional challenges in finding a layered
% fractional solution and complicates charging the welfare of blocked jobs.
% \bmnote{this doesn't make sense if we're using DFKL for intervals: rewrite}

As in the case of fractional unit allocations that we defined for the interval
preferences setting, we would like to construct a partition of items into
bundles and a corresponding allocation of jobs such that each job fits cleanly
into one bundle, no bundle contains too much fractional weight, and no item
belongs to more bundles than its multiplicity. Unfortunately, it turns out that
due to the rich combinatorial structure among paths in trees, such an
allocation cannot be constructed without losing too much value in the worst
case.
% Consider, in particular, the following example.
% \begin{example}
%     \label{ex:star-layering}
%     Let $T$ be a star graph with $n+1$ edges. Edge $e_0$ has capacity $n$, and
%     all other edges have capacity 1. There are $n$ jobs with weight 0.9 each,
%     and job $j \in [n]$ desires the path $(e_0, e_j)$. There are $n-1$
%     additional jobs with weight 0.5 each such that, for each $j \in [n-1]$, one
%     of these jobs desires the path $(e_j,e_{j+1})$. There is no way to
%     decompose this into bundles such that each job belongs to one bundle, each
%     bundle has constant fractional weight, and each item is associated with at
%     most one bundle.
% \end{example}

To overcome this issue, we simplify the combinatorial structure by
orienting the underlying tree with an arbitrary root and breaking each
path into two ``monotone'' components. Formally, for an arbitrary
rooting of the underlying tree, define the depth of an edge $t\in T$,
denoted $d(t)$, to be the number of edges on the shortest path from
the root to this edge, including $t$ itself.\footnote{So, the edges
  incident on the root have depth $1$.} For a job $j$ with
corresponding path $P_j$, define the {\em peak} of $j$'s desired path
to be the edge(s) closest to the root:
$\peakj = \argmin_{t \in \intj} \{d(t)\}$. Let $\larmj$ and $\rarmj$
denote the two subpaths of $\intj$ that begin at one of the peak edges
in $\peakj$ and consist of all subsequent edges in $\intj$ with
increasing depth; we call these the {\em arms} of the job. Let
$P_{\larmj}$ and $P_{\larmj}$ denote the corresponding subpaths. Note
that one of these arms may be empty; in that case, we take $\rarmj$ to
be the empty arm.

\begin{definition} A fractional allocation $x$ is a {\em fractional
    layered allocation} if there exists a partition of the multiset of
  items (where item $t$ has multiplicity $\capt$) into bundles
  $\{T_1, T_2, T_3, \cdots\}$, and a corresponding partition of the
  arms with non-zero weight under $x$, $\cup_{j\in U: x_j>0} \{\larmj,\rarmj\}$,
  into sets $\{A_1, A_2, A_3, \cdots\}$, such that:
    \begin{itemize}
        \item For all $j\in U$ with $x_j>0$, there is exactly one
          index $k$ with $\larmj\subset A_k$ and, if $\rarmj$ is not
          empty, exactly one index $k'$ with $\rarmj\in A_{k'}$. 
          \item For all $k$ and $\armj\in A_k$, $P_{\armj}\subseteq T_k$. 
        \item For all $k$ and $t\in T_k$, we have $\fwt(x,
          \{j: \armj\in A_k \text{ and } P_{\armj}\ni t\}) \le 1$.
   \end{itemize}
\end{definition}

Observe that there are two main differences between the layered
allocations defined above and fractional unit allocations as defined
in Section~\ref{sec:frac-unit-alloc}. First, layered allocations partition arms into layers
rather than entire jobs, so a job can end up in two different
layers. Second, and more importantly, a layer can have very large
total weight. All we guarantee is that the weight of any single edge
in a layer will be bounded by $1$. As such, the pricing we construct
is allowed to allocate multiple subpaths of a single layer to
different buyers in
sequence.

% We can now define a layered fractional allocation for the tree setting.
% \begin{definition}
%     Given a tree $T$ and jobs $U$, a fractional allocation $y$ is a {\em
%     layered fractional allocation} if there exist layers
%     $\{\layeri[1],\layeri[2],\cdots\}$ which satisfy the following conditions:
%     \begin{itemize}
%         \item for all $j$ with $y_{\larmj} > 0$, $\larmj \in \layeri$ (and
%             $\rarmj \in \layeri[i']$) for exactly one $i$ (and exactly one
%             $i'$, if $\rarmj$ is not empty)
%         \item for all $i$ and all $t$, $w_t(\layeri) \le 1$
%         \item for all $t$, $|\{i : w_t(\layeri) > 0\}| \le B_t$
%     \end{itemize}
% \end{definition}

% Given a fractional allocation $x$ over jobs, we extend our definitions of
% $\fwt$ and $\fval$ to arms. We use the notation $y$ to indicate a fractional
% allocation over arms. Given a feasible allocation $x$ over jobs, the allocation
% $y_{\larmj} = y_{\rarmj} = x_j$ is a feasible fractional allocation. For a job
% $j$, if both arms are defined, we take the value associated with each arm to be
% half the value of the corresponding job: thus, the total value of the two arms
% is equal to the value of the job. That is, $\valj[\larmj] = \valj[\rarmj] =
% \half \valj$. If $\rarmj$ is empty, we take $\valj[\rarmj] = y_{\rarmj} = 0$.
% For a set of arms $S$ and an edge $t$, define $w_t(S) = \sum_{\armj \in S}
% y_{\armj}$.  Finally, define
% \[
%     \fval(y) = \sum_j \big(\valj[\larmj] y_{\larmj} + \valj[\rarmj]
%     y_{\rarmj}\big).
% \]

\subsection{Layering a tree}
\label{sec:layered-alloc}

We now show that good fractional layered allocations always exist. For
a fractional allocation $x$, let $\vmax(x) = \max_{j: \xj>0} v_j$ and
$\vmin(x) = \min_{j: \xj>0} v_j$. 

\begin{lemma}
    \label{thm:layered-tree}
    For every feasible fractional allocation $x$, one can efficiently
    construct a feasible layered fractional allocation $\tilde{y}$
    such that
    \[
        \fval(x) \le O\left(\frac{\vmax(x)}{\vmin(x)}\right) \fval(\tilde{y}).
    \]
\end{lemma}

Before we proceed we need more notation. For a subset of jobs (or
arms) $J$, let $E_J$ denote the set of edges collectively desired by
those jobs: $E_J = \cup_{j\in J} P_j$. For an edge $t$, let
$\fwe(x) := \fwt(x,\{j:P_j\ni t\})$ denote the total fractional weight
of jobs (or arms) whose paths contain $t$. Likewise, let $\fwe(x_A)$
or $\fwe(x,A)$ denote the fractional weight on $t$ from jobs (or
arms) in set $A$.

We construct the layering recursively. Informally, given a fractional
allocation $x$, we find a set of jobs $\tlayer$ such that the fractional demand
on any item desired by a job in $\tlayer$ is a constant. That is, for
all $t\in E_\tlayer$, $\fwe(x,\tlayer)\le c$. 
% let $T_\tlayer
% = \Union_{j \in \tlayer} \intj$; then, for all $t \in T_\tlayer$, $w_t(\tlayer)
% \le c$. 
We scale the
fractional allocation of jobs in $\tlayer$ by $1/c$, and take this scaled
allocation along with the set $E_\tlayer$ to be the first layer. We thus obtain a feasible
layer, and lose a factor of at most $c$ in fractional value.

However, in order to apply this step recursively and arrive at a
feasible layered allocation, we need the remaining fractional solution
to be feasible for the items remaining after reducing the capacity of
every item in $E_\tlayer$ by one. Therefore, we need the fractional
weight on every edge in $E_\tlayer$ to decrease by at least 1 (or else
to zero) after removing $\tlayer$.  It may not be possible to find a
set $\tlayer$ which satisfies both this and the constant-demand
condition described above. Instead, we find a set $D$ such that jobs
in $\tlayer$ and $D$ together account for one unit of demand on every
edge in $E_\tlayer$, and then we drop $D$.  Because we drop jobs in
$D$ rather than assigning them to a layer, we do not need to ensure
that the fractional weight on edges in $E_D$ decreases by any
particular amount. We pick a set $D$ with small fractional weight
relative to that of $\tlayer$, so that we can charge its lost
fractional value to $\tlayer$.
%, and this step loses at most a factor of
%$O\left(\frac{\vmax(x)}{\vmin(x)}\right)$.

Finally, although we have informally described the argument in terms
of {\em jobs}, in our formal argument we consider sets of {\em
  arms}. This introduces an extra complication: when we drop the set
$D$ of arms, we must also drop their sibling arms. We show that we can
do this without losing more than another constant factor in fractional
value.

We formalize the recursive step in the following lemma; the proof
is deferred to Section~\ref{sec:deferred}.
\begin{lemma}
    \label{lem:peeling}
    Given a set $U$ of jobs and a non-zero fractional allocation $y$,
    there exist sets of arms $\tlayer \ne \emptyset$ and $D$ such that 
    \begin{enumerate}[label=\roman*.]
        \item $\tlayer \intersect D = \emptyset$;
        \item $\tlayer \intersect D' = \emptyset$, where $D' = \{\armj : j^{a'}
            \in D, a\ne a'\}$;
        %\item $y'_{\tlayer \union D} = 0$,
        %\item $y'$ is feasible for the multiset of items $T \setminus T_l$,
        \item for all $t \in E_\tlayer$, $\fwe(y, \tlayer \union D)\ge
          \min\{1, \fwe(y)\}$; \label{item:lowerbound}
          % $w_t(\tlayer \union D, y) \ge \min\{1, w_t(U, y)\}$
        \item for all $t \in E_\tlayer$, $\fwe(y, \tlayer) \le 7$; and
          % $w_t(\tlayer, y) \le 7$; and
            \label{item:upperbound}
        \item $\fwt(y_\tlayer) \ge 2 \fwt(y_D)$. \label{item:wtbound}
    \end{enumerate}
    % where $T_\tlayer = \{t : w_t(\tlayer, y) > 0\}$. 
    Furthermore, $\tlayer$ and $D$ can be found efficiently.
\end{lemma}

\begin{proofof}{\Cref{thm:layered-tree}}
  Let $E$ denote the {\em multiset} of edges, with the given
  multiplicities, in the given tree. Given a set of arms $U$ and a
  feasible fractional allocation $x$, we apply \Cref{lem:peeling} to
  generate the sets of arms $\tlayer$ and $D$.  Let $D'$ be the set of
  ``sibling'' arms of arms in $D$. Recall that $E_\tlayer$ is the set
  of edges contained in the paths desired by $j\in \tlayer$. Let
  $E'=E\setminus E_\tlayer$, and $x' = x-x_\tlayer-x_D-x_{D'}$.

  First, we observe that $x'$ is feasible for the multiset $E'$ of
  edges. This is because on the one hand, the multiplicity of each
  edge $t$ in $E_\tlayer$ decreases by 1. On the other hand,
  $\fwe(x)-\fwe(x')\ge \fwe(x,\tlayer\cup D)\ge \min\{1, \fwe(x)\}$ by
  property {\it \ref{item:lowerbound}} in \Cref{lem:peeling}. So,
  either the fractional weight on $t$ decreases by $1$, or it is equal
  to $0$.

  We set $T_1 = E_\tlayer$, $\tilde{A}_1 = \tlayer$, $D_1=D$, and
  $D'_1=D'$, and construct the remaining partition of edges and arms
  by applying \Cref{lem:peeling} recursively to allocation
  $x'$. Observe that our rescursive applications may discard in the
  form of the sets $D'_k$ some arms that have previously been included
  in the sets $\tilde{A}_{k'}$ for $k'<k$. Define
  $A_k = \tilde{A}_k\setminus (\cup_{k'} D'_{k'})$. We then define
  $\tilde{y}_j = x_j/7$ for jobs $j$ with arms in $\cup_k A_k$, and
  $0$ otherwise. It is now easy to see that $\tilde{y}$ is a
  fractional layered allocation, where the third property follows from
  property {\it \ref{item:upperbound}} in \Cref{lem:peeling}.

  It remains to prove that the fractional value of $\tilde{y}$ is
  large enough. Property {\it \ref{item:wtbound}} in
  \Cref{lem:peeling} tells us that
  $\fwt(x, \tilde{A}_k)\ge \fwt(x, D_k\cup D'_k)$. Summing over $k$
  and removing the contribution of $D'_k$ from each side, we get
  \[\fwt(x,\cup_k A_k) \ge \fwt(x,\cup_k D_k) \ge \half\fwt(x,\cup_k
  (D_k\cup D'_k)),\] 
  which implies $\fwt(x,\cup_k A_k)\ge \frac 13 \fwt(x)$, or
  $\fwt(\tilde{y})\ge \frac 1{21} \fwt(x)$.

  Finally, since the values of all jobs with positive weight under $x$
  lie in the range $[\vmin(x),\vmax(x)]$, we get that
  $\fval(\tilde{y})\ge \Omega\left(\frac{\vmin(x)}{\vmax(x)}\right)\fval(x)$.
\end{proofof}

% Let $\tlayer_1, \tlayer_2, $ etc. be the $\tlayer$
%   sets constructed in successive applications of
%   \Cref{lem:peeling}. Let $D_1, D_2,$ etc. and $D'_1, D'_2,$ etc. be
%   the corresponding ``dropped'' sets of arms constructed. Property XX
%   tells us that $\fwt(x,\cup_k (D_k\cup D'_k) )$



%     Then note that the arms in $D'$ receive the same allocation in $y$ as arms
%     in $D$, so $\fwt(y_{D})+\fwt(y_{D'})\leq 2\fwt(y_D)$.  Define
%     $\alpha=\frac{\vmax(x)}{\vmin(x)}$.  Since $\fwt(y_\tlayer)\geq 2\fwt(y_D)$
%     and each arm's value is in the range $[\vmin(x)/2 ,\alpha \vmin(x)]$, we
%     have
%     \[
%         \fval(y_\tlayer)\geq \frac{1}{2\alpha}\fval(y_{D}+y_{D'})=\frac{1}{O(\alpha)}\fval(y_\tlayer+y_{D}+y_{D'}).
%     \]
%     This means that the jobs in $\tlayer$ have total value that is
%     a $\frac{1}{O(\alpha)}$-fraction of the value of all jobs removed in this step.  By
%     property {\it \ref{item:upperbound}} in \Cref{lem:peeling}, $\frac{1}{7}y_\tlayer$ is a
%     feasible allocation in $T_\tlayer$. Let $\tilde{y}_1=\frac{1}{7}y_\tlayer$,
%     then $\tilde{y}_1$ is a feasible allocation in $T_\tlayer$ with
%     $\fval(\tilde{y}_1)\geq \frac{1}{O(\alpha)}\fval(y_\tlayer+y_{D}+y_{D'})$.

%     Now we set $A_1 = J$ and
%     recurse on the arms $U' = U \setminus (\tlayer \union D \union D')$ with
%     allocation $y-y_\tlayer-y_D-y_{D'}$. Then
%     $\sum_{r}\fval(\tilde{y}_r)\geq\frac{1}{O(\alpha)}\fval(x)$, while
%     $\tilde{y}_r$ is a feasible allocation in $T_r$ for each $r$. Thus
%     $\sum_{r}\tilde{y}_r$ is a layered allocation that satisfies
%     \Cref{thm:layered-tree}.


