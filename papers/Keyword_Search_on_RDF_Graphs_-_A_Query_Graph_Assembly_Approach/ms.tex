\documentclass[sigconf]{acmart}

\usepackage{booktabs} % For formal tables
%\usepackage{etex}
%\usepackage{epstopdf}
%\usepackage{amsmath}
%%\usepackage{txfonts}
%\usepackage{amssymb}
%\usepackage{times}
\usepackage{graphicx}
%\usepackage{epsfig}
%%\usepackage{hyperref}
\usepackage[linesnumbered,ruled,noend]{algorithm2e}
%\usepackage[noend]{algorithmic}
\usepackage{multirow}
%\usepackage{listings}
\usepackage{threeparttable}
\usepackage{tikz}
%\usepackage[T1]{fontenc}
\usepackage{pgfplots}
\usepackage{pgfplotstable}
%\usepackage{colortbl}
%\usepackage{array}
%\usepackage{eurosym}
\usepackage{caption}
\usepackage{subcaption}
%\usepackage{tikz}
%\usetikzlibrary{patterns}

%\usepackage{url}
%\usepackage{amsfonts}
%\usepackage{breakurl}
%\usepackage{tabularx}
%\usepackage{makecell}
%%\usepackage{floatrow}
%\usepackage{balance}  % for  \balance command ON LAST PAGE  (only there!)
%\usepackage{soul}
%\usepackage{times}
%\usepackage{indentfirst}
%\usepackage{verbatim}

\newcommand{\nop}[1]{}%
\newcommand{\Paragraph}[1]{~\vspace*{-0.9\baselineskip}\\{\bf #1}}
% Copyright
%\setcopyright{none}
%\setcopyright{acmcopyright}
%\setcopyright{acmlicensed}
%\setcopyright{rightsretained}
%\setcopyright{usgov}
%\setcopyright{usgovmixed}
%\setcopyright{cagov}
%\setcopyright{cagovmixed}


\copyrightyear{2017}
\acmYear{2017}
\setcopyright{acmcopyright}
\acmConference{CIKM'17 }{November 6--10, 2017}{Singapore,
	Singapore}\acmPrice{15.00}\acmDOI{10.1145/3132847.3132957}
\acmISBN{978-1-4503-4918-5/17/11}

\fancyhead{}
\settopmatter{printacmref=false, printfolios=false}

\begin{document}
\title{Keyword Search on RDF Graphs --- A Query Graph Assembly Approach}

 \author{
	% author names are typeset in 11pt, which is the default size in the author block
	{Shuo Han{${^1}$}, Lei Zou{${^1}$}, Jeffery Xu Yu{${^2}$}, Dongyan Zhao{${^1}$}} \\
	% add some space between author names and affils
%	\vspace{1.6mm}\\
	\fontsize{10}{\baselineskip}\selectfont\itshape $~^{1}$Peking University, China;\\
	\fontsize{10}{\baselineskip}\selectfont\itshape $~^{2}$ The Chinese University of Hong Kong, China; \\
	\fontsize{9}{\baselineskip}\selectfont\ttfamily\upshape $\{$hanshuo,zoulei,zhaody$\}$@pku.edu.cn, yu@se.cuhk.edu.hk\\
}



\begin{abstract}
Keyword search provides ordinary users an easy-to-use interface for querying RDF data. Given the input keywords, in this paper, we study how to assemble a query graph that is to represent user's query intention accurately and efficiently. Based on the input keywords, we first obtain the elementary query graph building blocks, such as entity/class vertices and predicate edges. Then, we formally define the \emph{query graph assembly (QGA)} problem. Unfortunately, we prove theoretically that QGA is a NP-complete problem. In order to solve that, we design some heuristic lower bounds and propose a bipartite graph matching-based best-first search algorithm. The algorithm's time complexity is $O(k^{2l} \cdot l^{3l})$, where $l$ is the number of the keywords and $k$ is a tunable parameter, i.e., the maximum number of candidate entity/class vertices and predicate edges allowed to match each keyword. Although QGA is intractable, both $l$ and $k$ are small in practice. Furthermore, the algorithm's time complexity does not depend on the RDF graph size, which guarantees the good scalability of our system in large RDF graphs. Experiments on DBpedia and Freebase confirm the superiority of our system on both effectiveness and efficiency. 
\end{abstract}

%
% The code below should be generated by the tool at
% http://dl.acm.org/ccs.cfm
% Please copy and paste the code instead of the example below. 
%
%\begin{CCSXML}
%	<ccs2012>
%	<concept>
%	<concept_id>10002951.10003317</concept_id>
%	<concept_desc>Information systems~Information retrieval</concept_desc>
%	<concept_significance>500</concept_significance>
%	</concept>
%	<concept>
%	<concept_id>10002951.10003317.10003325.10003327</concept_id>
%	<concept_desc>Information systems~Query intent</concept_desc>
%	<concept_significance>500</concept_significance>
%	</concept>
%	<concept>
%	<concept_id>10002951.10003317.10003325.10003330</concept_id>
%	<concept_desc>Information systems~Query reformulation</concept_desc>
%	<concept_significance>500</concept_significance>
%	</concept>
%	</ccs2012>
%\end{CCSXML}
%
%\ccsdesc[500]{Information systems~Information retrieval}
%\ccsdesc[500]{Information systems~Query intent}
%\ccsdesc[500]{Information systems~Query reformulation}



\keywords{keyword search; RDF; graph data management}

\maketitle

% \leavevmode
% \\
% \\
% \\
% \\
% \\
\section{Introduction}
\label{introduction}

AutoML is the process by which machine learning models are built automatically for a new dataset. Given a dataset, AutoML systems perform a search over valid data transformations and learners, along with hyper-parameter optimization for each learner~\cite{VolcanoML}. Choosing the transformations and learners over which to search is our focus.
A significant number of systems mine from prior runs of pipelines over a set of datasets to choose transformers and learners that are effective with different types of datasets (e.g. \cite{NEURIPS2018_b59a51a3}, \cite{10.14778/3415478.3415542}, \cite{autosklearn}). Thus, they build a database by actually running different pipelines with a diverse set of datasets to estimate the accuracy of potential pipelines. Hence, they can be used to effectively reduce the search space. A new dataset, based on a set of features (meta-features) is then matched to this database to find the most plausible candidates for both learner selection and hyper-parameter tuning. This process of choosing starting points in the search space is called meta-learning for the cold start problem.  

Other meta-learning approaches include mining existing data science code and their associated datasets to learn from human expertise. The AL~\cite{al} system mined existing Kaggle notebooks using dynamic analysis, i.e., actually running the scripts, and showed that such a system has promise.  However, this meta-learning approach does not scale because it is onerous to execute a large number of pipeline scripts on datasets, preprocessing datasets is never trivial, and older scripts cease to run at all as software evolves. It is not surprising that AL therefore performed dynamic analysis on just nine datasets.

Our system, {\sysname}, provides a scalable meta-learning approach to leverage human expertise, using static analysis to mine pipelines from large repositories of scripts. Static analysis has the advantage of scaling to thousands or millions of scripts \cite{graph4code} easily, but lacks the performance data gathered by dynamic analysis. The {\sysname} meta-learning approach guides the learning process by a scalable dataset similarity search, based on dataset embeddings, to find the most similar datasets and the semantics of ML pipelines applied on them.  Many existing systems, such as Auto-Sklearn \cite{autosklearn} and AL \cite{al}, compute a set of meta-features for each dataset. We developed a deep neural network model to generate embeddings at the granularity of a dataset, e.g., a table or CSV file, to capture similarity at the level of an entire dataset rather than relying on a set of meta-features.
 
Because we use static analysis to capture the semantics of the meta-learning process, we have no mechanism to choose the \textbf{best} pipeline from many seen pipelines, unlike the dynamic execution case where one can rely on runtime to choose the best performing pipeline.  Observing that pipelines are basically workflow graphs, we use graph generator neural models to succinctly capture the statically-observed pipelines for a single dataset. In {\sysname}, we formulate learner selection as a graph generation problem to predict optimized pipelines based on pipelines seen in actual notebooks.

%. This formulation enables {\sysname} for effective pruning of the AutoML search space to predict optimized pipelines based on pipelines seen in actual notebooks.}
%We note that increasingly, state-of-the-art performance in AutoML systems is being generated by more complex pipelines such as Directed Acyclic Graphs (DAGs) \cite{piper} rather than the linear pipelines used in earlier systems.  
 
{\sysname} does learner and transformation selection, and hence is a component of an AutoML systems. To evaluate this component, we integrated it into two existing AutoML systems, FLAML \cite{flaml} and Auto-Sklearn \cite{autosklearn}.  
% We evaluate each system with and without {\sysname}.  
We chose FLAML because it does not yet have any meta-learning component for the cold start problem and instead allows user selection of learners and transformers. The authors of FLAML explicitly pointed to the fact that FLAML might benefit from a meta-learning component and pointed to it as a possibility for future work. For FLAML, if mining historical pipelines provides an advantage, we should improve its performance. We also picked Auto-Sklearn as it does have a learner selection component based on meta-features, as described earlier~\cite{autosklearn2}. For Auto-Sklearn, we should at least match performance if our static mining of pipelines can match their extensive database. For context, we also compared {\sysname} with the recent VolcanoML~\cite{VolcanoML}, which provides an efficient decomposition and execution strategy for the AutoML search space. In contrast, {\sysname} prunes the search space using our meta-learning model to perform hyperparameter optimization only for the most promising candidates. 

The contributions of this paper are the following:
\begin{itemize}
    \item Section ~\ref{sec:mining} defines a scalable meta-learning approach based on representation learning of mined ML pipeline semantics and datasets for over 100 datasets and ~11K Python scripts.  
    \newline
    \item Sections~\ref{sec:kgpipGen} formulates AutoML pipeline generation as a graph generation problem. {\sysname} predicts efficiently an optimized ML pipeline for an unseen dataset based on our meta-learning model.  To the best of our knowledge, {\sysname} is the first approach to formulate  AutoML pipeline generation in such a way.
    \newline
    \item Section~\ref{sec:eval} presents a comprehensive evaluation using a large collection of 121 datasets from major AutoML benchmarks and Kaggle. Our experimental results show that {\sysname} outperforms all existing AutoML systems and achieves state-of-the-art results on the majority of these datasets. {\sysname} significantly improves the performance of both FLAML and Auto-Sklearn in classification and regression tasks. We also outperformed AL in 75 out of 77 datasets and VolcanoML in 75  out of 121 datasets, including 44 datasets used only by VolcanoML~\cite{VolcanoML}.  On average, {\sysname} achieves scores that are statistically better than the means of all other systems. 
\end{itemize}


%This approach does not need to apply cleaning or transformation methods to handle different variances among datasets. Moreover, we do not need to deal with complex analysis, such as dynamic code analysis. Thus, our approach proved to be scalable, as discussed in Sections~\ref{sec:mining}.
\section{Related Work}
%\mz{We lack a comparison to this paper: https://arxiv.org/abs/2305.14877}
%\anirudh{refine to be more on-topic?}
\iffalse
\paragraph{In-Context Learning} As language models have scaled, the ability to learn in-context, without any weight updates, has emerged. \cite{brown2020language}. While other families of large language models have emerged, in-context learning remains ubiquitous \cite{llama, bloom, gptneo, opt}. Although such as HELM \cite{helm} have arisen for systematic evaluation of \emph{models}, there is no systematic framework to our knowledge for evaluating \emph{prompting methods}, and validating prompt engineering heuristics. The test-suite we propose will ensure that progress in the field of prompt-engineering is structured and objectively evaluated. 

\paragraph{Prompt Engineering Methods} Researchers are interested in the automatic design of high performing instructions for downstream tasks. Some focus on simple heuristics, such as selecting instructions that have the lowest perplexity \cite{lowperplexityprompts}. Other methods try to use large language models to induce an instruction when provided with a few input-output pairs \cite{ape}. Researchers have also used RL objectives to create discrete token sequences that can serve as instructions \cite{rlprompt}. Since the datasets and models used in these works have very little intersection, it is impossible to compare these methods objectively and glean insights. In our work, we evaluate these three methods on a diverse set of tasks and models, and analyze their relative performance. Additionally, we recognize that there are many other interesting angles of prompting that are not covered by instruction engineering \cite{weichain, react, selfconsistency}, but we leave these to future work.

\paragraph{Analysis of Prompting Methods} While most prompt engineering methods focus on accuracy, there are many other interesting dimensions of performance as well. For instance, researchers have found that for most tasks, the selection of demonstrations plays a large role in few-shot accuracy \cite{whatmakesgoodicexamples, selectionmachinetranslation, knnprompting}. Additionally, many researchers have found that even permuting the ordering of a fixed set of demonstrations has a significant effect on downstream accuracy \cite{fantasticallyorderedprompts}. Prompts that are sensitive to the permutation of demonstrations have been shown to also have lower accuracies \cite{relationsensitivityaccuracy}. Especially in low-resource domains, which includes the large public usage of in-context learning, these large swings in accuracy make prompting less dependable. In our test-suite we include sensitivity metrics that go beyond accuracy and allow us to find methods that are not only performant but reliable.

\paragraph{Existing Benchmarks} We recognize that other holistic in-context learning benchmarks exist. BigBench is a large benchmark of 204 tasks that are beyond the capabilities of current LLMs. BigBench seeks to evaluate the few-shot abilities of state of the art large language models, focusing on performance metrics such as accuracy \cite{bigbench}. Similarly, HELM is another benchmark for language model in-context learning ability. Rather than only focusing on performance, HELM branches out and considers many other metrics such as robustness and bias \cite{helm}. Both BigBench and HELM focus on ranking different language model, while fix a generic instruction and prompt format. We instead choose to evaluate instruction induction / selection methods over a fixed set of models. We are the first ever evaluation script that compares different prompt-engineering methods head to head. 
\fi

\paragraph{In-Context Learning and Existing Benchmarks} As language models have scaled, in-context learning has emerged as a popular paradigm and remains ubiquitous among several autoregressive LLM families \cite{brown2020language, llama, bloom, gptneo, opt}. Benchmarks like BigBench \cite{bigbench} and HELM \cite{helm} have been created for the holistic evaluation of these models. BigBench focuses on few-shot abilities of state-of-the-art large language models, while HELM extends to consider metrics like robustness and bias. However, these benchmarks focus on evaluating and ranking \emph{language models}, and do not address the systematic evaluation of \emph{prompting methods}. Although contemporary work by \citet{yang2023improving} also aims to perform a similar systematic analysis of prompting methods, they focus on simple probability-based prompt selection while we evaluate a broader range of methods including trivial instruction baselines, curated manually selected instructions, and sophisticated automated instruction selection.

\paragraph{Automated Prompt Engineering Methods} There has been interest in performing automated prompt-engineering for target downstream tasks within ICL. This has led to the exploration of various prompting methods, ranging from simple heuristics such as selecting instructions with the lowest perplexity \cite{lowperplexityprompts}, inducing instructions from large language models using a few annotated input-output pairs \cite{ape}, to utilizing RL objectives to create discrete token sequences as prompts \cite{rlprompt}. However, these works restrict their evaluation to small sets of models and tasks with little intersection, hindering their objective comparison. %\mz{For paragraphs that only have one work in the last line, try to shorten the paragraph to squeeze in context.}

\paragraph{Understanding in-context learning} There has been much recent work attempting to understand the mechanisms that drive in-context learning. Studies have found that the selection of demonstrations included in prompts significantly impacts few-shot accuracy across most tasks \cite{whatmakesgoodicexamples, selectionmachinetranslation, knnprompting}. Works like \cite{fantasticallyorderedprompts} also show that altering the ordering of a fixed set of demonstrations can affect downstream accuracy. Prompts sensitive to demonstration permutation often exhibit lower accuracies \cite{relationsensitivityaccuracy}, making them less reliable, particularly in low-resource domains.

Our work aims to bridge these gaps by systematically evaluating the efficacy of popular instruction selection approaches over a diverse set of tasks and models, facilitating objective comparison. We evaluate these methods not only on accuracy metrics, but also on sensitivity metrics to glean additional insights. We recognize that other facets of prompting not covered by instruction engineering exist \cite{weichain, react, selfconsistency}, and defer these explorations to future work. 
\section{Problem Definition}\label{sec:problemdef}

\begin{figure} [t]
	\centering
	\scalebox{1.05}[1.15]
	{
		\resizebox{\linewidth}{!}
		{
			\includegraphics[scale=1.0]{visio_pics/rdf_graph_new.pdf}
		}
	}
	\vspace{-0.1in}
	\caption{A Sample of DBpedia RDF Graph.}
	\label{fig:rdfgraph}
	\vspace{-0.2in}
\end{figure}

In this section, we define our problem and review the terminologies used throughout this paper. As a de facto standard model of knowledge base, RDF represents the assertions by $\langle$subject, predicate, object$\rangle$ triples. An RDF dataset can be represented as a graph naturally, where subjects and objects are vertices and predicates denote directed edges between them. A running example of RDF graph is illustrated in Figure \ref{fig:rdfgraph}. Formally, we have the definition about RDF graph as follows.


%\vspace{-0.1in}
\begin{definition} \label{def:rdfgraph}
	\textbf{ (RDF Graph) }
	An \textit{RDF graph} is denoted as $G(V, E)$, where
	$V$ is the set of entity and class vertices corresponding to subjects and objects of RDF triples,
	and $E$ is the set of directed relation edges with their labels corresponding to predicates of RDF triples.

\end{definition}

Note that RDF triple's object may be literal value, for example, $\langle$res:Alan\_Turing, dbo:deathDate, ``1954-06-07'' $\textasciicircum{}\textasciicircum{}$xms:date$\rangle$. We treat all literal values as entity vertices in RDF graph, and literal types (e.g. xms:date as for ``1954-06-07'') as class vertices. 

SPARQL is the standard structural query language of RDF, which can also be represented as a \emph{query graph} defined as follows.
%Answering SPARQLs is equivalent to evaluate the query by subgraph matching using homomorphism \cite{zou2014gstore}. 

\begin{definition} \textbf{ (Query Graph) }
	A \textit{query graph} is denoted as $Q(V_Q, E_Q)$, where
	$V_Q$ consists of entity vertices, class vertices, and vertex \emph{variables},
	and $E_Q$ consists of relation edges as well as edge \emph{variables}.
\end{definition}

%Although SPARQL provides a systematic approach to accessing RDF graph, the complexity of the SPARQL syntax makes it hard to use. 
In this paper, we study the keyword search problem over RDF graph. Given a keyword token sequence $RQ = \{k_1, k_2, ..., k_m\}$, our problem is to interpret $RQ$ as a \emph{query graph} $Q$. 

\begin{figure}
\centering

\def\picScale{0.08}    % define variable for scaling all pictures evenly
\def\colWidth{0.5\linewidth}

\begin{tikzpicture}
\matrix [row sep=0.25cm, column sep=0cm, style={align=center}] (my matrix) at (0,0) %(2,1)
{
\node[style={anchor=center}] (FREEhand) {\includegraphics[width=0.85\linewidth]{figures/FREEhand.pdf}}; %\fill[blue] (0,0) circle (2pt);
\\
\node[style={anchor=center}] (rigid_v_soft) {\includegraphics[width=0.75\linewidth]{figures/FREE_vs_rigid-v8.pdf}}; %\fill[blue] (0,0) circle (2pt);
\\
};
\node[above] (FREEhand) at ($ (FREEhand.south west)  !0.05! (FREEhand.south east) + (0, 0.1)$) {(a)};
\node[below] (a) at ($ (rigid_v_soft.south west) !0.20! (rigid_v_soft.south east) $) {(b)};
\node[below] (b) at ($ (rigid_v_soft.south west) !0.75! (rigid_v_soft.south east) $) {(c)};
\end{tikzpicture}


% \begin{tikzpicture} %[every node/.style={draw=black}]
% % \draw[help lines] (0,0) grid (4,2);
% \matrix [row sep=0cm, column sep=0cm, style={align=center}] (my matrix) at (0,0) %(2,1)
% {
% \node[style={anchor=center}] {\includegraphics[width=\colWidth]{figures/photos/labFREEs3.jpg}}; %\fill[blue] (0,0) circle (2pt)
% &
% \node[style={anchor=center}] {\includegraphics[width=\colWidth, height=160pt]{figures/stewartRender.png}}; %\fill[blue] (0,0) circle (2pt);
% \\
% };

% %\node[style={anchor=center}] at (0,-5) (FREEstate) {\includegraphics[width=0.7\linewidth]{figures/FREEstate_noLabels2.pdf}};

% \end{tikzpicture}

\caption{\revcomment{2.3}{(a) A fiber-reinforced elastomerc enclosure (FREE) is a soft fluid-driven actuator composed of an elastomer tube with fibers wound around it to impose specific deformations under an increase in volume, such as extension and torsion. (b) A linear actuator and motor combined in \emph{series} has the ability to generate 2 dimensional forces at the end effector (shown in red), but is constrained to motions only in the directions of these forces. (b) Three FREEs combined in \emph{parallel} can generate the same 2 dimensional forces at the end effector (shown in red), without imposing kinematic constraints that prohibit motion in other directions (shown in blue).}}

% \caption{A fiber-reinforced elastomeric enclosure (FREE) (top) is a soft fluid-driven actuator composed of an elastomer tube with fibers wound around it to impose deformation in specific directions upon pressurization, such as extension and torsion. \revcomment{2.3}{In this paper we explore the potential of combining multiple FREEs in parallel to generate fully controllable multi-dimensional spacial forces}, such as in a parallel arrangement around a flexible spine element (bottom-left), or a Stewart Platform arrangement (bottom-right).}

\label{fig:overview}
\end{figure}

\section{QGA: Hardness and Algorithm}
\label{sec:querygraphassembling}
Unfortunately, QGA is proved to be NP-complete in Section \ref{sec:hardness}. To solve that, we transform QGA into a constrained bipartite graph matching problem and design a practical efficient algorithm to find the optimal $Q$.  

\subsection{Hardness Analysis}\label{sec:hardness}

\begin{theorem}
	The query graph assembly problem is NP-complete.
\end{theorem}

\vspace{-0.1in}
\begin{proof}
	The decision version of QGA is defined as follows: 
	Given $n$ vertex sets $\Upsilon=\{V_1, ..., V_n\}$ and $m$ edge sets $\Gamma=\{E_1, ..., E_m\}$ and a threshold $\theta$, QGA is to decide if there exists an assembly query graph $Q$, where it satisfies the two constraints in Definition \ref{def:querygraphassembling} and the total assembly cost $cost(Q) \leq \theta$. Obviously, an instance of QGA can be verified in polynomial time. Thus, QGA belongs to NP class. 

	We construct a polynomial time reduction from 3-SAT (a classical NP-complete problem) to QGA. More specifically, given any instance $I$ of 3-SAT, we can generate an instance $I^{\prime}$ of QGA within polynomial time, where the decision value (TRUE/FALSE) of $I$ is \emph{equivalent} to $I^{\prime}$.
	
	\textbf{\emph{Any instance $I$ of 3-SAT:}}
	Without loss of generality, we define an instance $I$ of 3-SAT as follows: Given a set of $p$ boolean variables $U=\{u_1,u_2,...,u_p\}$ and a set of $q$ 3-clauses $C=\{c_1,c_2,...,c_q\}$ on $U$, The problem is to decide whether there exists a truth assignment for each variable in $U$ that satisfies all clauses in $C$.
	
	\textbf{\emph{Corresponding instance $I^{\prime}$ of QGA:}}
	Given a variable set $U$ and a 3-clause set $C$ (of instance $I$), we build a group of vertex sets $\Upsilon$ and a group of edge sets $\Gamma$ for the instance $I^{\prime}$.  $\Upsilon$ consists of the two parts $\Upsilon_{U}$ and $\Upsilon_{C}$, which are defined as follows:
	
	\begin{enumerate}
	\item For each variable $u_i \in U$, we introduce a vertex set $\{u_i,\overline{u_i}\}$ into $\Upsilon_{U}$. We call $u_i$ and $\overline{u_i}$ as \emph{variable vertices} of $I^{\prime}$. 
	\item For each 3-clause $c_j \in C$, we introduce a vertex set having a single vertex $\{c_j\}$ into $\Upsilon_{C}$. We call $c_j$ as a \emph{clause vertex} of $I^{\prime}$.  
	\end{enumerate}
	where $\Upsilon=\Upsilon_{U} \cup \Upsilon_{C}$.

	We also introduce $q$ disengaged edges into $\Gamma$. Each edge is a singleton set $\{e_j\}$. These $q$ edges can be used to connect any two vertices in $\Upsilon$. We set edge weight $w(e_j)=0$ if and only if $e_j$ connects the clause vertex $c_j$ and a member variable vertex $u_i$ (or $\overline{u_i}$) in 3-clause $c_j$. Otherwise, the edge weights are 1.
	
	The corresponding instance $I^{\prime}$ is defined as follows: Given two groups $\Upsilon$ and $\Gamma$ (explained above), and a threshold $\theta=0$, the problem is to decide whether there exists a graph $Q$, satisfying the two constraints in Definition \ref{def:querygraphassembling}, and $cost(Q)\leq 0$. Since edge weights are no less than 0, hence our goal is to construct $Q$ with $cost(Q)=0$.
	
	\textbf{\emph{Equivalence:}}
	Next, we will show the equivalence between the instance $I$ (of 3-SAT) and the instance $I^{\prime}$ (of QGA), i.e., $I \Leftrightarrow I^{\prime}$.

	Assume that the answer to $I$ is TRUE, which means that we have a truth assignment for each variable $u_i$ so that all 3-clauses $c_j$ are satisfied. According to the truth assignment in $I$, we can construct a graph $Q$ of $I^{\prime}$ as follows: for each clause vertex $c_j$, we connect $c_j$ with variable vertex $u_i$ (or $\overline{u_i}$) if $u_i=1$ (or $\overline{u_i}=1$, i.e, $u_i=0$) and $u_i$ is included in the 3-clause $c_j$. There may exist multiple variable vertices $u_i$  (or $\overline{u_i}$) satisfying the above condition. We connect $c_j$ with arbitrary one of them to form edge $(c_j,u_i)$ (or $(c_j,\overline{u_i})$).
	These edge weights are 0. 
	%	These edge weights are 0, as shown in Figure \ref{fig:3satproof}.
	Thus, we construct a graph $Q$ satisfying the two constraints in Definition \ref{def:querygraphassembling} and $cost(Q)=0$.	It means the answer to $I^{\prime}$ is TRUE, i.e., $I \Rightarrow I^{\prime}$.
	
	Assume the answer to $I^{\prime}$ is TRUE. It means that we can construct a graph $Q$ having the following characters:
	\begin{enumerate}
		\item For $i=1,...,p$, one of variable vertex $\{u_i, \overline{u_i}\}$ is selected; and for $j=1,...,q$, clause vertex $c_j$ is selected; \emph{/*one vertex of each vertex set in $\Upsilon$ is selected*/}
		\item For $j=1,...,q$, each edge $e_j$ is selected; and $e_j$ connects the clause vertex $c_j$ and a variable vertex $u_i$(or $\overline{u_i}$) that corresponds to one of the three member variables of 3-clause $c_j$.  \emph{/*one edge of each edge set in $\Gamma$ is selected, and $cost(Q)=0$*/}
	\end{enumerate} 
	If a variable vertex $u_i$ (or $\overline{u_i}$) is selected, we set $u_i=1$ (or $\overline{u_i}=1$, i.e., $u_i=0$).  
	Since each clause vertex $c_j$ is connected to one selected variable vertex $u_i$ (or $\overline{u_i}$) that is in the 3-clause $c_j$, the corresponding variable $u_i=1$ (or $\overline{u_i}=1$). It means that 3-clause $c_j=1$ as $u_i$ (or $\overline{u_i}$) is included in $c_j$.  Hence, the answer to $I$ is also TRUE, i.e., $I \Leftarrow I^{\prime}$.
	
	In summary, we have reduced 3-SAT to QGA, where the former is a NP-complete problem. Therefore, we have proved that QGA is NP-complete. 

%	\vspace{-0.1in}
%	\textbf{\emph{Reduction Process:}}
%	We construct a polynomial time reduction from 3-SAT (a classical NP-complete problem) to QGA. Let us recall the 3-SAT problem. Given a set of $p$ boolean variables $U=\{u_1,u_2,...,u_p\}$ and a set of $q$ 3-clauses $C=\{c_1,c_2,...,c_q\}$ on $U$, 3-SAT is to decide whether there exists a truth assignment for each variable in $U$ that satisfies all clauses in $C$.
%%	For example, given $U=\{u_1,u_2,u_3,u_4\}$ and $C=\{c_1=u_1  \vee \overline {u_2 }  \vee u_4, c_2= u_2  \vee \overline {u_3 }  \vee \overline {u_4 } \}$, there exists an assignment $\{u_1=1;u_2=1;u_3=0;u_4=1\}$ that satisfies all 3-clauses $c_1$ and $c_2$ in $C$. 
%	
%	 Given a variable set $U$ and a 3-clause set $C$, we build a group of vertex sets $\Upsilon$ and a group of edge sets $\Gamma$ for the QGA problem.  $\Upsilon$ consists of the two parts $\Upsilon_{U}$ and $\Upsilon_{C}$, which are defined as follows:
%	
%	\begin{enumerate}
%		\item For each variable $u_i \in U$, we introduce a vertex set $\{u_i,\overline{u_i}\}$ into $\Upsilon_{U}$. In QGA, we call $u_i$ and $\overline{u_i}$ as \emph{variable vertices}. 
%		\item For each 3-clause $c_j \in C$, we introduce a vertex set having a single vertex $\{c_j\}$ into $\Upsilon_{C}$. In QGA, we call $c_j$ as a \emph{clause vertex}.  
%	\end{enumerate}
%	where $\Upsilon=\Upsilon_{U} \cup \Upsilon_{C}$.
%	
%	Assume that there are $q$ 3-clauses $c_j$ ($j=1,...,q$) in the 3-SAT problem. We introduce $q$ edges into $\Gamma$. Each edge is a singleton set $\{e_j\}$. These $q$ edges can be used to connect any two vertices in $\Upsilon$. We set edge weight $w(e_j)=0$ if and only if $e_j$ connects the clause vertex $c_j$ and a member variable vertex $u_i$ (or $\overline{u_i}$) in 3-clause $c_j$. Otherwise, the edge weights are 1.
%%	For example, there are two 3-clauses $c_1=u_1  \vee \overline {u_2}  \vee u_4 $ and $c_2= u_2  \vee \overline {u_3}  \vee \overline {u_4}$. Only when the edge $e_1$ connects $c_1$ with $u_1$, $\overline {u_2}$ and  $u_4$, and the edge $e_2$ connects $c_2$ with $u_2$, $\overline{u_3}$ and $\overline{u_4}$, the edge weights are 0. Other connections' edge weights are 1. Figure \ref{fig:3satproof} illustrates that. 
%	
%	The reduced QGA instance is described as follows: Given two groups $\Upsilon$ and $\Gamma$ (defined above), and a threshold $\theta=0$, the problem is to decide whether there exists a graph $Q$, which satisfies the two constraints in Definition \ref{def:querygraphassembling}, and $cost(Q)\leq 0$. Since edge weights are no less than 0, thus, our goal is to construct $Q$ with $cost(Q)=0$.
%	
%	Let us prove that the decision of QGA is equivalent to that over 3-SAT. Assume that the answer to QGA is TRUE. It means that we can construct a graph $Q$ having the following characters:
%	\begin{enumerate}
%		\item For $i=1,..,p$, one of variable vertex $\{u_i, \overline{u_i}\}$ is selected;  and for $j=1,...,q$, clause vertex $c_j$ is selected; \emph{/*one vertex of each vertex set in $\Upsilon$ is selected*/}
%		\item For $j=1,...,q$, each edge $e_j$ is selected; and $e_j$ connects the clause vertex $c_j$ and a variable vertex $u_i$(or $\overline{u_i}$) that corresponds to one of the three member variables of 3-clause $c_j$.  \emph{/*one edge of each edge set in $\Gamma$ is selected, and $cost(Q)=0$*/}
%	\end{enumerate} 
%	If a variable vertex $u_i$ (or $\overline{u_i}$) is selected, we set $u_i=1$ (or $\overline{u_i}=1$, i.e., $u_i=0$).  
%	Since each clause vertex $c_j$ is connected to one selected variable vertex $u_i$ (or $\overline{u_i}$) that is in the 3-clause $c_j$, the corresponding variable $u_i=1$ (or $\overline{u_i}=1$). It means that 3-clause $c_j=1$ as $u_i$ (or $\overline{u_i}$) is included in $c_j$.  Therefore, we prove that if the answer to QGA is TRUE then the answer to 3-SAT is also TRUE. 
%	
%%	For example, given two 3-clauses $c_1=u_1  \vee \overline {u_2}  \vee u_4 $ and $c_2= u_2  \vee \overline {u_3}  \vee \overline {u_4}$, we build six vertex sets in Figure \ref{fig:3satproof}. The upper four sets correspond to the variable vertices and the lower two are the clause vertices. Since there are two 3-clauses, we introduce two singleton edge sets in the assembly query graph. Only the solid edges' weight are 0. Assume that we find a QGA solution whose $cost(Q)=0$ (i.e, the answer to QGA is TRUE), as depicted by the red bold solid lines, i.e., the edge connecting $c_1$ and ${u_1}$ and the one connecting $c_2$ and $\overline{u_3}$. Thus, we set variable $u_1=1$ and $\overline{u_3}=1$ (i.e., $u_3=0$). This is a truth assignment satisfying both $c_1$ and $c_2$ no matter what assignment of other variables. It means that it is also true to the 3-SAT instance if we find a solution to the QGA instance. 
%	
%	Let us prove that the answer to 3-SAT is FALSE if the answer to the QGA is FLASE. For the ease of the proof, we prove the corresponding converse negative proposition, i.e, the answer to QGA is TRUE if the answer to 3-SAT is TRUE.  
%	
%	Assume that the answer to 3-SAT is TRUE, which means that we have a truth assignment for each variable $u_i$ so that all 3-clauses $c_j$ are satisfied. According to the truth assignment in 3-SAT, we can construct a graph $Q$ as follows: for each clause vertex $c_j$ in QGA, we connect $c_j$ with variable vertex $u_i$ (or $\overline{u_i}$) if $u_i=1$ (or $\overline{u_i}=1$, i.e, $u_i=0$) and $u_i$ is included in the 3-clause $c_j$. There may exist multiple variable vertices $u_i$  (or $\overline{u_i}$) satisfying the above condition. We connect $c_j$ with one of these variable vertices $u_i$  (or $\overline{u_i}$) to form edge $(c_j,u_i)$ (or $(c_j,\overline{u_i})$).
%	These edge weights are 0. 
%%	These edge weights are 0, as shown in Figure \ref{fig:3satproof}.
%	Thus, we construct a graph $Q$ satisfying the two constraints in Definition \ref{def:querygraphassembling} and $cost(Q)=0$.	
%	
%%	\begin{figure} [h]
%%		\centering
%%		\scalebox{0.80} [0.80]
%%		{
%%			\resizebox{\linewidth}{!}
%%			{
%%				\includegraphics[scale=1.0]{visio_pics/qga_3sat_reduce.pdf}
%%			}
%%		}
%%		\caption{Visualizing the Reduction Process.}
%%		\label{fig:3satproof}
%%		%\vspace{-0.1in}
%%	\end{figure}
%
%	In summary, we have reduced 3-SAT to QGA, while the former is a classical NP-complete problem. Therefore, we have proved that QGA is NP-complete. 
\end{proof}


%\subsection{Algorithm} \label{sec:algorithm}
%Since QGA problem is NP-complete, we aim to design a practical efficient algorithm that can return the exact result. A brute force idea is to enumerate all the possible assembly combinations of the candidate vertices $\Gamma$ and edges $\Upsilon$, check whether each of the combinations is a valid query graph $Q$, and calculate its $cost(Q)$. Then we find the valid $Q$ with minimum $cost(Q)$, as the final result. Suppose $AQ$ consists of $n$ entity/class terms and $m$ relation terms, and each term is matched to at most $k$ candidate vertices (or predicate edges). Then the overall enumeration space is $\left( \begin{array}{l}  n \\  2 \\  \end{array} \right)^m \cdot k^{m+n}$. It is hard to design pruning conditions for such a brute force strategy to reduce its search space. But fortunately, we can transform the QGA problem into a constrained bipartite graph model, then propose a best-first search algorithm with some powerful pruning techniques. We elaborate the model and the algorithm in the following subsections. 


\subsection{Assembly Bipartite Graph Model} \label{sec:general_case}
Since QGA is NP-complete, we transform it to an equivalent bipartite graph model with some constraints. Based on the bipartite graph model, we can design a best-first search algorithm with powerful pruning strategies.

Let us recall the definition of QGA (Definition \ref{def:querygraphassembling}), each of the $n$ entity/class terms is matched to a candidate vertex set $V_i$, and each of the $m$ relation terms is matched to a candidate edge set $E_j$. For example, ``USA'' is matched to $\{$res:USA$\_$Today, res:United$\_$States$\}$, and ``graduate from'' is matched to $\{$dbo:almaMater, dbo:education$\}$. The multiple choices in a candidate vertex/edge set indicate the ambiguity of keywords. If we adopt a pipeline style mechanism to address the keyword disambiguation and the query graph generation separately, we need to select exactly one element from each candidate vertex/edge set in the first phase. In our example, res:United$\_$States is the correct interpretation of ``USA'', and dbo:almaMater should be selected for ``graduate from''. If we simply adopt the string-based matching score \cite{li2011faerie} for keyword disambiguation, the matching score of res:USA$\_$Today may be higher than res:United$\_$States. In this case, if we only select the one with highest matching score, the correct answer will be missed due to the entity linking error. However, a robust solution should be error-tolerant with the ability to construct a correct query graph that is of interest to users even in the presence of noises and errors in the first phase. In our QGA solution, we allow the ambiguity of keywords (i.e., allowing one term matching several candidates) in the first phase, and push down the disambiguation to the query graph assembly step. For example, although the matching score of ``USA'' to res:USA$\_$Today is higher than that to res:United$\_$States, the former's assembly cost is much larger than the latter. Thus, we can still obtain the correct query graph $Q$. We propose the \emph{assembly bipartite graph matching} model to handle the ambiguity of keywords and the ambiguity of query graph structures uniformly. 

\begin{definition}\textbf{ (Assembly Bipartite Graph) }. \label{def:assemblygraph}
	Each entity/class term $t_i^{v}$ corresponds to a set $V_i$ of vertices ($1\leq i \leq n$) and each relation term $t_j^{e}$ corresponds to a set $E_j$ of predicate edges ($1\leq j \leq m$).  
	
	An assembly bipartite graph $\mathbb{B}(V_{L},V_{R},E_{\mathbb{B}})$ is defined as follows:
	
	\begin{enumerate}
		\item Vertex pair set $V_{i_1} \times V_{i_2}=\{(v_{i_1},v_{i_2}) | 1\leq i_1 < i_2 \leq n \wedge v_{i_1} \in V_{i_1}  \wedge v_{i_2} \in V_{i_2}\}$.
		\item $V_{L}=\bigcup\nolimits_{1 \le i_1  < i_2  \le n } {(V_{i_1} \times V_{i_2})} $.
		%\item Edge set $Y_{j}=\{p_j | \alpha+1 \leq j \leq n \wedge p_j \in E_{j}  \}$.
		\item  $V_{R}=\bigcup\nolimits_{1 \le j \le m} {E_j }  $.
		\item there is a crossing edge $e$ between any node $(v_{i_1}, v_{i_2})$ in $V_{L}$ and any node $p_j$ in $V_{R}$ ($1 \le i_1  < i_2  \le n$, $1 \le j \le m$), which is denoted as $e(\langle v_{i_1},v_{i_2}\rangle, p_j)$. Edge weight $w(e)=w(\langle v_{i_1},v_{i_2}\rangle, p_j)$, where $w(\langle v_{i_1},v_{i_2}\rangle, p_j)$ denotes the triple assembly cost.  
	\end{enumerate} 
\end{definition}


\begin{figure} [t]
	\begin{center}
		\vspace{-0.15in}
		\includegraphics[scale=0.55]{visio_pics/assembly_bipartite_graph.pdf}
		%\vspace{-0.1in}
		\caption{An Example of Assembly Bipartite Graph.}
		
		\label{fig:asm_bigraph}
		\vspace{-0.15in}
	\end{center}
\end{figure}

\begin{example} 
Let us recall the running example. Each term's candidate matchings are given in Figure \ref{fig:graph_elements_exp2}, i.e., $V_1=\{$dbo:Scientist$\}$, $V_2=\{$dbo:University$\}$, and $V_3=\{$res:United\_Today, res:United\_States$\}$. Thus, there are three vertex pair sets $V_1 \times V_2$, $V_2 \times V_3$ and $V_1 \times V_3$. In Figure \ref{fig:asm_bigraph}, each vertex pair set is highlighted by a dash circle. There are two predicate edge sets $E_1=\{$dbo:almaMater, dbo:education$\}$ and $E_2=\{$dbo:country, dbo:location$\}$, which are also illustrated in Figure \ref{fig:asm_bigraph} using dash circles. 
\end{example}

It is worth noting that there are some \emph{conflict} relations among the crossing edges in $\mathbb{B}$.  For example, crossing edge $e(\langle$dbo:Scientist, res:USA\_Today$\rangle$, dbo:country) conflicts with $e^{\prime}(\langle$dbo:University, res:United\_States$\rangle$, dbo:almaMater) (in Figure \ref{fig:asm_bigraph}), since the semantic term ``USA'' corresponds to two different entity vertices res:USA\_\\Today and res:United\_States in $e$ and $e^{\prime}$, respectively. In this case, the constraints of QGA (in Definition \ref{def:querygraphassembling}) will be violated if $e$ and $e^{\prime}$ occur in the same matching.
Considering the above example, we formulate the \emph{conflict relation} among crossing edges in $\mathbb{B}$. 

\begin{definition} \textbf{ (Conflict Relation) }.\label{def:conflict}	
	For any two crossing edges $e(\langle v_{i_1},v_{i_2}\rangle, p_j)$ and $e^{\prime}(\langle v_{i_1}^{\prime},v_{i_2}^{\prime}\rangle, p_j^{\prime})$ in an assembly bipartite graph $\mathbb{B}$, we say that $e$ is \emph{conflict} with $e^{\prime}$ if at least one of the following conditions holds:
	
	\begin{enumerate}
		\item $v_{i_1}$ and $v_{i_1}^{\prime}$ (or $v_{i_2}$ and $v_{i_2}^{\prime}$) come from the same vertex set $V_{i_1}$ (or $V_{i_2}$) and  $v_{i_1} \neq v_{i_1}^{\prime}$ (or $v_{i_2} \neq v_{i_2}^{\prime}$);		
		\ /*Two different vertices come from the same vertex set; an entity/class term $t_i^{v}$ cannot map to multiple vertices*/
		\item $p_j$ and $p_j^{\prime}$ come from the same edge set $E_j$ and $p_j \neq p_j^{\prime}$.		
		\ /*Two different predicates come from the same edge set; an relation term $t_j^{e}$ cannot map to multiple predicate edges*/
		\item $(v_{i_1}=v_{i_1}^{\prime} \wedge v_{i_2}=v_{i_2}^{\prime}) \vee (p_j = p_j^{\prime})$ 		
		%/*Two vertex pairs are the same with each other*/
		\ /*$e$ and $e^{\prime}$ share a common endpoint in $V_{L}$ or $V_{R}$; a vertex pair cannot be assigned two different predicate edges and a predicate edge cannot connect two different vertex pairs */
	\end{enumerate}
	
	
\end{definition}

In order to be consistent with the constraints of QGA, we also redefine \emph{matching} as follows.
\begin{definition}\textbf{ (Matching) }\label{def:matchingnew}. Given an assembly bipartite graph $\mathbb{B}(V_{L},V_{R},E_{\mathbb{B}})$, a \emph{matching} of $\mathbb{B}$ is a subset $M$ ($M \subseteq E_{\mathbb{B}}$) of its crossing edges, no two of which are \emph{conflict} with each other.
\end{definition}

\begin{definition}\textbf{ (Matching Cost) }. The \emph{matching cost} of $M$ is $w(M) = \sum\nolimits_{e \in M} {w(e)}$, where $w(e)$ is defined in Definition \ref{def:assemblygraph}(4). 
\end{definition}
It is easy to know that solving QGA problem is equivalent to finding a size-$m$ \emph{matching} over $\mathbb{B}$ with the minimum \emph{matching cost}. A matching edge in $\mathbb{B}$ stands for an assembled edge in $Q$.
%From the assembly bipartite graph model, we design some lower bounds for computing the matching cost, based on which, we propose a best-first search algorithm to find the optimal solution in the following subsection. 

%\vspace{-0.1in}
\subsection{Condensed Bipartite Graph}

\begin{figure} [t]
	\begin{center}
		%\vspace{-0.15in}
		\includegraphics[scale=0.65]{visio_pics/condensed_bipartite_graph.pdf}
		%\vspace{-0.1in}
		\caption{An Example of Condensed Bipartite Graph.}
		\label{fig:con_bigraph}
		   \vspace{-0.2in}
	\end{center}
\end{figure}


According to the condition (2) and (3) in Definition \ref{def:conflict}, each predicate edge set $E_j$ has only one $p_j$ occurring in the \emph{matching} $M$. Thus, we can condense each predicate edge set into one node $E_j$, leading to a \emph{condensed bipartite graph} $\mathbb{B}^{*}(V_{L}^*,V_{R}^*,E_{\mathbb{B}}^*)$, as shown in Figure \ref{fig:con_bigraph}. There is a crossing edge between any node $(v_{i_1},v_{i_2})$ in $V_{L}^*$ and any node $E_j$ in $V_{R}^*$. The edge weight is defined as $w(e) = w(\langle v_{i_1} ,v_{i_2}\rangle ,E_j ) = MIN_{p_j  \in E_j } \{ w(\langle v_{i_1} ,v_{i_2}\rangle,p_j)\}$, where $w(\langle v_{i_1 } ,v_{i_2}\rangle,p_j)$ is defined in Definition \ref{def:assemblygraph}(4). In order to find the size-$m$ matching with the minimum cost in $\mathbb{B}^{*}$, we propose QGA Algorithm (i.e., Algorithm \ref{alg:bfmatch}). 


\subsection{QGA Algorithm}

A search state is denoted as $\{M,Z, cost(M), LB(M)\}$, where $M$ records the current partial matching, i.e., a set of currently selected matching edges, $Z$ records a set of unmatched edges that are not \emph{conflict} with edges in $M$. Obviously, each edge $e \in Z$ can be appended to $M$ to enlarge the current matching. Initially, $M=\phi$, and $Z$ records all crossing edges in the condensed bipartite graph $\mathbb{B}^{*}$ (Line 3 in Algorithm \ref{alg:bfmatch}). $cost(M)$ denotes the current partial matching cost, i.e., $cost(M)= \sum\nolimits_{e \in M} {w(e)}$. $LB(M)$ denotes the lower bound of the current partial matching $M$. We will discuss how to compute the lower bound $LB(M)$ later. All search states are stored in a priority queue $H$ in the non-decreasing order of the lower bound $LB(M)$ (Line 2). Furthermore, we maintain a threshold $\theta$ to be the current minimum matching cost. Initially, $\theta = \infty$. In each iteration, we pop a head state $\{M,Z, cost(M), LB(M)\}$.
We enumerate all unmatched edges $e \in Z$ in non-decreasing order of $w(e)$ to generate subsequent search states. Each $e \in Z$ is moved from $Z$ to $M$ to obtain a new matching $M^{\prime}$ (Line 9). We remove $e$ and all edges in $Z$ that are conflict with $e$ to obtain $Z^{\prime}$ (Line 10). We also update $cost(M^{\prime})$ and $LB(M^{\prime})$ (Lines 11-12). Then we check whether $M^{\prime}$ is a size-$m$ matching over $\mathbb{B}^{*}$(i.e. end state). If so, we update the threshold $\theta$ if $cost(M^{\prime}) < \theta$ and record $M^{\prime}$ as the current optimal matching $M_{opt}$  (Lines 13-16). Otherwise, we push the new state $\{M^{\prime},Z^{\prime}, cost(M^{\prime}), LB(M^{\prime})\}$ into $H$ (Line 18). The algorithm keeps iterating until that the threshold $\theta$ is less than the lower bound of the current head state in $H$ (Line 6) or $H$ is empty. 
\begin{algorithm} 
	\caption{QGA Algorithm} \label{alg:bfmatch}
	\KwIn{Condensed bipartite graph $B^*(V_{L}^*, V_{R}^*, E_B^*)$,
		and conflict relations among $E_B^*$.}
	\KwOut{The optimal assembly query graph $Q$.}
	$M_{opt} := \phi$, $\theta := \infty$ \;
	$H := \phi$ ;\tcp{min-heap, sort by lower bound}
	$H \leftarrow \{M:=\emptyset, Z := E_B^*, cost(M) := 0, LB(M) := 0\}$ \;
	\While{$H$ is not empty}
	{
		$\{M, Z, cost(M), LB(M)\} \leftarrow H.pop$ \;
		\If{$LB(M) >= \theta$}{\textbf{break} \;}
		
		\For{each crossing edge $e \in Z$}
		{
			$M^{\prime} := M \cup \{e\}$; $Z := Z \setminus \{e\}$ \;
			$Z^{\prime} = Z \setminus \{e^\prime | e^\prime \in Z \wedge e^\prime\ conflict\ with\ e\}$ \;
			$cost(M^{\prime}) := cost(M) + w(e)$ \;
			Compute $LB(M^{\prime})$\;
			
			\If{$|M^{\prime}| = m$}
			{
				\If{$cost(M^{\prime}) < \theta$}
				{
					$\theta := cost(M^{\prime})$ \;
					$M_{opt} := M'$ \;
				}       
			}
			\Else
			{
				$H \leftarrow \{M', Z', cost(M^{\prime}), LB(M^{\prime})\}$ \;
			}       
		}
	}
	Build query graph $Q$ according to $M_{opt}$ \;
	\Return $Q$
\end{algorithm}

\vspace{-0.1in}
\subsection{Computing Lower Bound} \label{sec:lower_bound}
Considering a search state $\{M, Z, cost(M), LB(M)\}$, we discuss how to compute $LB(M)$. Let $|M|$ denote the number of edges in the current matching. Since a maximum matching in $B^{*}$ should contain $m$ matching edges (i.e., covering all $E_j$), we need to select the other $(m-|M|)$ no-conflict edges from $Z$, to form a size-$m$ matching. We denote  $\mathcal{M}(Z)$ as the minimum weighted size-$(m-|M|)$ matching of $Z$. Thus, the best result that can be reached from $M$ is $M \cup \mathcal{M}(Z)$. To ensure the correctness of Algorithm \ref{alg:bfmatch}, $LB(M) \le cost(M \cup \mathcal{M}(Z))$ must always be satisfied. A good lower bound should have the following two characters: (1)  $LB(M)$ is as close to $cost(M \cup \mathcal{M}(Z))$ as possible; (2) the computation cost of $LB(M)$ is small. From this intuition, we propose three different lower bounds as follows.

\Paragraph{Naive-LB:}
The naive method to compute $LB(M)$ is to select the top-$(m-|M|)$ unmatched edges $\{e_1,...,e_{m-|M|}\}$ in $Z$ with the minimum weights and compute $Navie$-$LB(M) = cost(M) + \sum\nolimits_{i = 1}^{i = m  - |M|} {w(e_i)}$. In the implementation of Algorithm \ref{alg:bfmatch}, $Z$ is stored by a linked list, and always kept in the non-decreasing order of $w(e)$. Therefore, the complexity of computing Naive-LB is $O(m-|M|)$. Since the top-$(m-|M|)$ unmatched edges in $Z$ may be conflict with each other, Naive-LB is not tight, and has the weakest pruning power comparing with the following two lower bounds.

\Paragraph{KM-LB:}
We ignore the conflict relations among unmatched edges in $Z$, and find the minimum weighted size-$(m-|M|)$ matching $\mathcal{M}_{KM}(Z)$ by KM Algorithm \cite{kuhn1955hungarian} in $O(|Z|^3)$ time. Then we compute $KM$-$LB(M)= cost(M) + cost(\mathcal{M}_{KM}(Z))$. KM-LB is much tighter than Naive-LB, but the computation cost is expensive.

\Paragraph{Greedy-LB:}
Inspired by the matching-based KM-LB, we propose a greedy strategy (Algorithm \ref{alg:greedy_lb}) to find an approximate matching of $Z$ in $O(|Z|)$ time. It has been proved in \cite{preis1999linear} that the result return by Algorithm \ref{alg:greedy_lb} (denote as $\mathcal{M}_{greedy}(Z)$) is a $1/2$-approximation of the optimal matching $\mathcal{M}_{KM}(Z)$. i.e. $cost(\mathcal{M}_{KM}(Z)) \ge \frac{cost(\mathcal{M}_{greedy}(Z))}{2}$. Thus we compute $Greedy$-$LB(M)= cost(M) + \frac{cost(\mathcal{M}_{greedy}(Z))}{2}$. Greedy-LB is considered as the trade-off between Naive-LB and KM-LB, because of its medium tightness and computation cost. Experiment in Section \ref{sec:evaluate_qga_bounds} confirms that Greedy-LB gains the best performance among them.

\begin{algorithm} [t]
	\caption{Greedy-LB Algorithm} \label{alg:greedy_lb}
	\KwIn{Unmatched edge set $Z$.}
	\KwOut{An $1/2$-approximate matching $\mathcal{M}_{greedy}$.}
	$\mathcal{M}_{greedy} := \phi$ \;
	\For{each $e \in Z$ in non-decreasing order of $w(e)$}
	{
		\If{$e$ do not share common vertices with $\mathcal{M}_{greedy}$}
		{
			$\mathcal{M}_{greedy} := \mathcal{M}_{greedy} \cup \{e\}$
		}
		\Return $\mathcal{M}_{greedy}$ \;
	}
\end{algorithm}

\vspace{-0.1in}
\subsection{Time Complexity Analysis} 
In a condensed bipartite graph $\mathbb{B}^{*}(V_{L}^*,V_{R}^*,E_{\mathbb{B}}^*)$, $|V_L^*| \le k^2 \cdot \left( \begin{array}{l}  n \\  2 \\  \end{array} \right)$, because there are $\left( \begin{array}{l}  n \\  2 \\  \end{array} \right)$ vertex pair set, each contains at most $k^2$ vertex pairs, and $V_R^*$ consists of $m$ condensed predicate nodes, i.e., $|V_R^*|=m$. Therefore, the total number of crossing edges is $|E_{\mathbb{B}}^*|=|V_R^*|\times|V_L^*| \le mk^2\left( \begin{array}{l}  n \\  2 \\  \end{array} \right) \le mn^2k^2$. Since a maximum matching in $\mathbb{B}^{*}$ should contain $m$ matching edges, we can find at most $\left( \begin{array}{l}  mn^2k^2 \\  \ \ \ m \\  \end{array} \right) \le (mn^2k^2)^m$ different maximum matchings. Since $n+m \le l$, by replacing $n$ and $m$ by $l$, we get the overall search space $O(k^{2l} \cdot l^{3l})$.


\subsection{Implicit Relation Prediction}
As we know, keywords are more flexible than NL question sentences. In some cases, users may omit some relation terms. For example, users may input ``scientist graduate from university USA'', where the keyword ``locate'' is omitted. In this case, for humans, it is trivial to infer that the user means ``an university located in USA''. Let us recall our QGA approach. If we omit ``locate'' in the running example, there is only one relation term. So, the query graph $Q$ cannot be connected if we use only one predicate edge to connect three vertices. We can patch our solution to connect $Q$ as follows.
%Suppose that the query graph $Q$ generated by QGA Algorithm has $r$ connected components $\mathcal{P}_1$, $\mathcal{P}_2$,...,$\mathcal{P}_r$. In order to connect them, we work as follows:

\begin{definition} \textbf{ (Relation Prediction Graph) }
	Suppose that $Q$ consists of $r$ connected components: $Q=\{\mathcal{P}_1, \mathcal{P}_2, ..., \mathcal{P}_r\}$. A relation prediction graph $P(V_P,E_P)$ is a complete graph and defined as follows:
	\begin{itemize}
		\item $V_P$ consists of $r$ vertices, where each vertex $v_P^i \in V_P$ corresponds to a connected component $\mathcal{P}_i$.
		\item $E_P$ consists of $\frac{r(r-1)}{2}$ labeled weighted edges. For any $(\mathcal{P}_i, \mathcal{P}_j)$, we find the minimum assembly cost triple $e(\langle v_i, v_j \rangle, p)$, where $v_i \in V(\mathcal{P}_i)$, $v_j \in V(\mathcal{P}_j)$, and $p \in \mathcal{U}$. Note that $V(\mathcal{P}_i)$ denotes all vertices in $P_i$, and $\mathcal{U}$ denotes all predicates in RDF graph $G$. Then we add a corresponding edge $e_P(v_P^i,v_P^j)$ into $E_P$, where the weight $w(e_P)=w(\langle v_i, v_j \rangle, p)$, and the label $l(e_P)=(\langle v_i, v_j \rangle, p)$.
	\end{itemize}
\end{definition}
Thus the relation prediction task is modeled as finding a minimum spanning tree $T$ on $P$, which can be solved in linear time. Through the label $(\langle v_i, v_j \rangle, p)$ of the tree edge in $T$, we can know that $v_i, v_j \in V_Q$ should be connected by the predicate edge $p$.



\section{Evaluation}
\label{sec:evaluation}
\begin{table*}[!t]
\begin{center}
%\small
\caption {Benchmarks and applications for the study of the application-level resilience}
\vspace{-5pt}
\label{tab:benchmark}
\tiny
\begin{tabular}{|p{1.7cm}|p{7.5cm}|p{4cm}|p{2.5cm}|}
\hline
\textbf{Name} 	& \textbf{Benchmark description} 		& \textbf{Execution phase for evaluation}  			& \textbf{Target data objects}             \\ \hline \hline
CG (NPB)             & Conjugate Gradient, irregular memory access (input class S)   & The routine conj\_grad in the main computation loop  & The arrays $r$ and $colidx$     \\\hline
MG (NPB)    	       & Multi-Grid on a sequence of meshes (input class S)             & The routine mg3P in the main computation loop & The arrays $u$ and $r$ 	\\ \hline
FT (NPB)             & Discrete 3D fast Fourier Transform (input class S)            & The routine fftXYZ in the main computation loop  & The arrays $plane$ and $exp1$    \\ \hline
BT (NPB)             & Block Tri-diagonal solver (input class S)         		& The routine x\_solve in the main computation loop & The arrays $grid\_points$ and $u$	\\ \hline
SP (NPB)             & Scalar Penta-diagonal solver (input class S)         		& The routine x\_solve in the main computation loop & The arrays $rhoi$ and $grid\_points$  \\ \hline
LU (NPB)            & Lower-Upper Gauss-Seidel solver (input class S)        	& The routine ssor 	& The arrays $u$ and $rsd$ \\ \hline \hline
LULESH~\cite{IPDPS13:LULESH} & Unstructured Lagrangian explicit shock hydrodynamics (input 5x5x5) & 
The routine CalcMonotonicQRegionForElems 
& The arrays $m\_elemBC$ and $m\_delv\_zeta$ \\ \hline
AMG2013~\cite{anm02:amg} & An algebraic multigrid solver for linear systems arising from problems on unstructured grids (we use  GMRES(10) with AMG preconditioner). We use a compact version from LLNL with input matrix $aniso$. & The routine hypre\_GMRESSolve & The arrays $ipiv$ and $A$   \\ \hline
%$hierarchy.levels[0].R.V$ \\ \hline
\end{tabular}
\end{center}
\vspace{-5pt}
\end{table*}

%We evaluate the effectiveness of ARAT, and 
%We use ARAT to study the application-level resilience.
%The goal is to demonstrate 
%that aDVF can be a very useful metric to quantify the resilience of data objects
%at the application level. 
We study 12 data objects from six benchmarks of the NAS parallel benchmark (NPB) suite (we use SNU\_NPB-1.0.3) and 4 data objects from two scientific applications. 
%which is a c version of NPB 3.3, but ARAT can work for Fortran.
Those data objects are chosen to be representative: they have various data access patterns and participate in various execution phases.  
%For the benchmarks, we use CLASS S as the input problems and use the default compiler options of NPB.
For those benchmarks and applications, we use their default compiler options, and use gcc 4.7.3 and LLVM 3.4.2 for trace generation.
To count the algorithm-level fault masking, we use the default convergence thresholds (or the fault tolerance levels) for those benchmarks.
Table~\ref{tab:benchmark} gives 
%for->on by anzheng
detailed information on the benchmarks and applications.
The maximum fault propagation path for aDVF analysis is set to 10 by default.
%the value shadowing threshold is set as 0.01 (except for BT, we use $1 \times 10^{-6}$).
%These value shadowing thresholds are chosen such that any error corruption
%that results in the operand's value variance less than 1\% (for the threshold 0.01) or 0.0001\% (for the threshold $1 \times 10^{-6}$) during the 
%trace analysis does not impact the outcome correctness of six benchmarks.
%LU: check the newton-iteration residuals against the tolerance levels
%SP: check the newton-iteration residuals against the tolerance levels
%BT: check the newton-iteration residuals against the tolerance levels

\subsection{Resilience Modeling Results}
%We use ARAT to calculate aDVF values of 16 data objects. 
Figure~\ref{fig:aDVF_3tiers_profiling}
shows the aDVF results and breaks them down into the three levels 
(i.e., the operation-level, fault propagation level, and algorithm-level).
Figure~\ref{fig:aDVF_3classes_profiling} shows the 
%for->of by anzheng
results for the analyses at the levels of the operation and fault propagation,
and further breaks down the results into 
the three classes (i.e., the value overwriting, logical and comparison operations,
and value shadowing). %based on the reasons of the fault masking.
We have multiple interesting findings from the results.

\begin{figure*}
	\centering
        \includegraphics[width=0.8\textwidth]{three_tiers_gray.pdf}
% * <azguolu@gmail.com> 2017-03-23T03:20:28.808Z:
%
% ^.
        \vspace{-5pt}
        \caption{The breakdown of aDVF results based on the three level analysis. The $x$ axis is the data object name.}
        \vspace{-8pt}
        \label{fig:aDVF_3tiers_profiling}
\end{figure*}


\begin{figure*}
	\centering
	\includegraphics[width=0.8\textwidth]{three_types_gray.pdf}
	\vspace{-5pt}
	\caption{The breakdown of aDVF results based on the three classes of fault masking. The $x$ axis is the data object name. \textit{zeta} and \textit{elemBC} in LULESH are \textit{m\_delv\_zeta} and \textit{m\_elemBC} respectively.} % Anzheng
	\vspace{-5pt}
	\label{fig:aDVF_3classes_profiling}
    %\vspace{-5pt}
\end{figure*}

(1) Fault masking is common across benchmarks and applications.
Several data objects (e.g., $r$ in CG, and $exp1$ and $plane$ in FT)
have aDVF values close to 1 in Figure~\ref{fig:aDVF_3tiers_profiling}, 
which indicates that most of operations working on these data objects
have fault masking.
However, a couple of data objects have much less intensive fault masking.
For example, the aDVF value of $colidx$ in CG is 0.28 (Figure~\ref{fig:aDVF_3tiers_profiling}). 
Further study reveals that $colidx$ is an array to store column indexes of sparse matrices, and there is few operation-level or fault propagation-level fault masking  (Figure~\ref{fig:aDVF_3classes_profiling}).
The corruption of it can easily cause segmentation fault caught by the
algorithm-level analysis. 
$grid\_points$ in SP and BT also have a relatively small aDVF value (0.14 and 0.38 for SP and BT respectively in Figure~\ref{fig:aDVF_3tiers_profiling}).
Further study reveals that $grid\_points$ defines input problems for SP and BT. 
A small corruption of $grid\_points$ 
%change->changes by anzheng
can easily cause major changes in computation
caught by the fault propagation analysis. 

The data object $u$ in BT also has a relatively small aDVF value (0.82 in Figure~\ref{fig:aDVF_3tiers_profiling}).
Further study reveals that $u$ is read-only in our target code region
for matrix factorization and Jacobian, neither of which is friendly
for fault masking.
Furthermore, the major fault masking for $u$ comes from value shadowing,
and value shadowing only happens in a couple of the least significant bits 
of the operands that reference $u$, which further reduces the value of aDVF.
%also reduces fault masking.

(2) The data type is strongly correlated with fault masking.
Figure~\ref{fig:aDVF_3tiers_profiling} reveals that the integer data objects ($colidx$ in CG, $grid\_points$ in BT and SP, $m\_elemBC$ in LULESH) appear to be 
more sensitive to faults than the floating point data objects 
($u$ and $r$ in MG, $exp1$ and $plane$ in FT, $u$ and $rsd$ in LU, $m\_delv\_zeta$ in LULESH, and $rhoi$ in SP).
In HPC applications, the integer data objects are commonly employed to
define input problems and bound computation boundaries (e.g., $colidx$ in CG and $grid\_points$ in BT), 
or track computation status (e.g., $m\_elemBC$ in LULESH). Their corruption 
%these integer data objects
is very detrimental to the application correctness. 

(3) Operation-level fault masking is very common.
For many data objects, the operation-level fault masking contributes 
more than 70\% of the aDVF values. For $r$ in CG, $exp1$ in FT, and $rhoi$ in SP,
the contribution of the operation-level fault masking is close to 99\% (Figure~\ref{fig:aDVF_3tiers_profiling}).

Furthermore, the value shadowing is a very common operation level fault masking,
especially for floating point data objects (e.g., $u$ and $r$ in BT, $m\_delv\_zeta$ in LULESH, and $rhoi$ in SP in Figure~\ref{fig:aDVF_3classes_profiling}).
This finding has a very important indication for studying the application resilience.
In particular, the values of a data object can be different across different input problems. If the values of the data object are different, 
then the number of fault masking events due to the value shadowing will be different. 
Hence, we deduce that the application resilience
can be correlated with the input problems,
because of the correlation between the value shadowing and input problems. 
We must consider the input problems when studying the application resilience.
This conclusion is consistent with a very recent work~\cite{sc16:guo}.

(4) The contribution of the algorithm-level fault masking to the application resilience can be nontrivial.
For example, the algorithm-level fault masking contributes 19\% of the aDVF value for $u$ in MG and 27\% for $plane$ in FT (Figure~\ref{fig:aDVF_3tiers_profiling}).
The large contribution of algorithm-level fault masking in MG is consistent with
the results of existing work~\cite{mg_ics12}. 
For FT (particularly 3D FFT), the large contribution of algorithm-level fault masking in $plane$ (Figure~\ref{fig:aDVF_3tiers_profiling})
comes from frequent transpose and 1D FFT computations that average out 
or overwrite the data corruption.
CG, as an iterative solver, is known to have the algorithm-level fault masking
because of the iterative nature~\cite{2-shantharam2011characterizing}.
Interestingly, the algorithm-level fault masking in CG contributes most to the resilience of $colidx$ which is a vulnerable integer data object (Figure~\ref{fig:aDVF_3tiers_profiling}).

%Our study reveals the algorithm-level fault masking of CG from
%two perspectives. First, $a$ in CG, which is an array for intermediate results,
%has few algorithm-level fault masking (0.008\%);
%Second, $x$ in CG, which is a result vector, has 5.4\% of the aDVF value coming from the algorithm-level fault masking.
%This result indicates that the effects of the algorithm-level fault masking
%are not uniform across data objects. 

(5) Fault masking at the fault propagation level is small.
For all data objects, the contribution of the fault masking at the level of fault propagation is less than 5\% (Figure~\ref{fig:aDVF_3tiers_profiling}).
For 6 data objects ($r$ and $colidx$ in CG, $grid\_points$ and $u$ in BT, and 
$grid\_points$ and $rhoi$ in SP),  there is no fault masking at the level of fault propagation.
In combination with the finding 4, we conclude that once the fault
is propagated, it is difficult to mask it because of the contamination of
more data objects after fault propagation, and only the algorithm semantics can tolerate  propagated faults well. 
%This finding is consistent with our sensitivity analysis. 

(6) Fault masking by logical and comparison operations is small,
%For all data objects, the fault masking contributions due to logical and comparison operations are very small, 
comparing with the contributions of value shadowing and overwriting (Figure~\ref{fig:aDVF_3classes_profiling}). 
Among all data objects, 
the logical and comparison operations in $grid\_points$ in BT contribute the most (25\% contribution in Figure~\ref{fig:aDVF_fine_profiling}), 
because of intensive ICmp operations (integer comparison). %logical OR and SHL (left shifting).


(7) The resilience varies across data objects. %within the same application.
This fact is especially pronounced in two data objects $colidx$ and $r$ in CG (Figure~\ref{fig:aDVF_3tiers_profiling}).
 $colidx$ has aDVF much smaller than $r$, which means $colidx$ is much less resilient than $r$ (see finding 1 for a detailed analysis on $colidx$). 
Furthermore, $colidx$ and $r$ have different algorithm-level
fault masking (see finding 4 for a detailed analysis).

\begin{comment}
\textbf{Finding 7: The resilience of the same data objects varies across different applications.}
This fact is especially pronounced in BT and SP.
BT and SP address the same numerical problem but with different algorithms.
BT and SP have the same data objects, $qs$ and $rhoi$, but
$qs$ manifests different resilience in BT and SP.
This result is interesting, because it indicates that by using
different algorithms, we have opportunities to
improve the resilience of data objects.
\end{comment}

To further investigate the reasons for fault masking, 
we break down the aDVF results at the granularity of LLVM instructions,
based on the analyses at the levels of operation and fault propagation.
The results are shown in Figure~\ref{fig:aDVF_fine_profiling}.
%Because of the space limitation, 
%we only show one data object per benchmark, but each selected data object has the most diverse fault masking events within the corresponding benchmark.
%Based on Figure~\ref{fig:aDVF_fine_profiling}, we have another interesting finding.

(8) Arithmetic operations make a lot of contributions to fault masking.
%For $r$ in CG, $r$ in MG, $exp1$ in FT, $u$ in BT, $qs$ in SP, and $u$ in LU,
%the arithmetic operations, FMul (100\%), Add (16\%), FMul (85\%), 
%FMul (94\%), FMul (28\%), and FAdd (50\%)
For $r$ in CG, $u$ in BT, $plane$ and $exp1$ in FT, $m\_elemBC$ in LULESH, 
arithmetic operations (addition, multiplication, and division) contribute to almost 100\% of the fault masking (Figure~\ref{fig:aDVF_fine_profiling}).  
%(at the operation level and the fault propagation level).
%For $qs$ in SP and $u$ in LU, the store operation also makes
%important contributions as the arithmetic operations because of value overwriting.

\begin{figure*}
	\centering
	\includegraphics[width=0.77\textheight, height=0.23\textheight]{pie_chart.pdf}
	\vspace{-10pt}
	\caption{Breakdown of the aDVF results based on the analyses at the levels of operation and fault propagation}
    \vspace{-10pt}
	\label{fig:aDVF_fine_profiling}
\end{figure*}


\subsection{Sensitivity Study}
\label{sec:eval_sen}
%\textbf{change the fault propagation threshold and study the sensitivity of analysis to the threshold}
ARAT uses 10 as the default fault propagation analysis threshold. 
The fault propagation analysis will not go beyond 10 operations. Instead,
we will use deterministic fault injection after 10 operations. 
In this section, we study the impact of this threshold on the modeling accuracy. We use a range of threshold values and examine how the aDVF value varies and whether
the identification of fault masking varies. 
Figure~\ref{fig:sensitivity_error_propagation} shows the results for 
%add , after BT by anzheng
multiple data objects in CG, BT, and SP.
We perform the sensitivity study for all 16 data objects.
%in six benchmarks and two applications.
Due to the page space limitation, we only show the results for three data objects,
but we summarize the sensitivity study results for all data objects in this section.
%but other data objects in all benchmarks have the same trend.

Our results reveal that the identification of fault masking by tracking fault propagation is not significantly 
affected by the fault propagation analysis threshold. Even if we use a rather large threshold (50), 
the variation of aDVF values is 4.48\% on average among all data objects,
and the variation at each of the three levels of analysis (the operation level, fault propagation level,  and algorithm level) is less than 5.2\% on average. 
In fact, using a threshold value of 5 is sufficiently accurate in most of the cases (14 out of 16 data objects).
This result is consistent with our finding 5 (i.e., fault masking at the fault propagation level is small). %in most benchmarks).
However, we do find a data object ($m\_elementBC$ in LULESH) %and $exp1$ in FT) 
showing relatively high-sensitive (up to 15\% variation) to the threshold. For this uncommon data object, using 50 as the fault propagation path is sufficient. 

%In other words, even though using a larger threshold value can identify more error masking by tracking error 
%propagation, the implicit error masking induced by the error propagation is very limited.

\begin{figure}
		\begin{center}
		\includegraphics[width=0.48\textwidth,height=0.11\textheight]{sensi_study_gray.pdf}
		\vspace{-15pt}
		\caption{Sensitivity study for fault propagation threshold}
		\label{fig:sensitivity_error_propagation}
		\end{center}
\vspace{-15pt}
\end{figure}


\begin{comment}
\subsection{Comparison with the Traditional Random Fault Injection}
%\textbf{compare with the traditional fault injection to verify accuracy}
To show the effectiveness of our resilience modeling, we compare traditional random fault injection
and our analytical modeling. Figure~\ref{fig:comparison_fi} and Table~\ref{tab:comparison} show the results.
The figure shows the success rate of all random fault injection. The ``success'' means the application
outcome is verified successfully by the benchmarks and the execution does not have any segfault. The success rate is used as a metric
to evaluate the application resilience.

We use a data-oriented approach to perform random fault injection.
In particular, given a data object, for each fault injection test we trigger a bit flip
in an operand of a random instruction, and this operand must be a reference to the
target data object. We develop a tool based on PIN~\cite{pintool} to implement the above fault injection functionality.
For each data object, we conduct five sets of random fault injection tests, 
and each set has 200 tests (in total 1000 tests per data object). 
We show the results for CG and FT in this section, but we find that
the conclusions we draw from CG and FT are also valid for the other four benchmarks.


%\begin{table*}
%\label{tab:success_rate}
%\begin{centering}
%\renewcommand\arraystretch{1.1}
%\begin{tabular}{|c|c|c|c|c|c|c|}
%\hline 
%Success Rate (Difference) & Test set 1 & Test set 2 & Test set 3 & Test set 4 & Test set 5 & Average\tabularnewline
%\hline 
%\hline 
%CG-a & 66.1\% (11.7\%) & 68.5\% (15.7\%) & 56.7\% (4.21\%) & 61.3\% (3.57\%) & 43.3\% (26.8\%) & 59.2\%\tabularnewline
%\hline 
%CG-x & 99.2\% (2.2\%) & 98.6\% (1.5\%) & 96.5\% (0.63\%) & 97.8\% (0.64\%) & 93.6\% (3.7\%) & 97.1\%\tabularnewline
%\hline 
%CG-colidx & 36.8\% (12.7\%) & 49.6\% (17.8\%) & 40.2\% (4.6\%) & 52.6\% (24.9\%) & 31.4\% (25.4\%) & 42.1\%\tabularnewline
%\hline 
%FT-exp1 & 52.7\% (1.4\%) & 22.6\% (56.5\%) & 78.5\% (51.0\%) & 60.7\% (16.7\%) & 45.4\% (12.7\%) & 51.9\%\tabularnewline
%\hline 
%FT-plane & 82.1\% (2.5\%) & 79.3\% (5.6\%) & 99.5\% (18.2\%) & 93.2\% (10.7\%) & 66.8\% (20.6\%) & 84.2\%\tabularnewline
%\hline 
%\end{tabular}
%\par\end{centering}
%\caption{XXXXX}
%\end{table*}


\begin{table*}
\begin{centering}
\caption{\small The results for random fault injection. The numbers in parentheses for each set of tests (200 tests per set) are the success rate difference from the average success rate of 1000 fault injection tests.}
\label{tab:comparison}
\renewcommand\arraystretch{1.1}
\begin{tabular}{|c|p{2.2cm}|p{2.2cm}|p{2.2cm}|p{2.2cm}|p{2.2cm}|p{1.8cm}|}
\hline 
       %& Test set 1 & Test set 2 & Test set 3 & Test set 4 & Test set 5 & Average\tabularnewline
       & \hspace{13pt} Test set 1 \hspace{1pt}/  & \hspace{13pt} Test set 2 \hspace{1pt}/ & \hspace{13pt} Test set 3 \hspace{1pt}/ & \hspace{13pt} Test set 4 \hspace{1pt}/ & \hspace{13pt} Test set 5 \hspace{1pt}/ & Ave. of all test / \\
       & success rate (diff.) & success rate (diff.) & success rate (diff.) & success rate (diff.) & success rate (diff.) & \hspace{5pt} success rate \\
\hline 
\hline 
CG-a & 66.1\% (6.9\%) & 68.5\% (9.3\%) & 56.7\% (-2.5\%) & 61.3\% (2.1\%) & 43.3\% (-15.9\%) & 59.2\%\tabularnewline
\hline 
CG-x & 99.2\% (2.1\%) & 98.6\% (1.5\%) & 96.5\% (-0.6\%) & 97.8\% (0.7\%) & 93.6\% (-3.5\%) & 97.1\%\tabularnewline
\hline 
CG-colidx & 36.8\% (-5.3\%) & 49.6\% (7.5\%) & 40.2\% (-2.0\%) & 52.6\% (10.5\%) & 31.4\% (-10.7\%) & 42.1\%\tabularnewline
\hline 
FT-exp1 & 52.7\% (0.8\%) & 22.6\% (-29.3\%) & 78.5\% (26.6\%) & 60.7\% (8.8\%) & 45.4\% (-6.5\%) & 51.9\%\tabularnewline
\hline 
FT-plane & 82.1\% (-2.1\%) & 79.3\% (-4.9\%) & 99.5\% (15.3\%) & 93.2\% (9.0\%) & 66.8\% (-17.4\%) & 84.2\%\tabularnewline
\hline 
\end{tabular}
\par\end{centering}
\vspace{-0.4cm}
\end{table*}

\begin{figure}
	\begin{center}
		\includegraphics[width=0.48\textwidth,keepaspectratio]{verifi-study.png}
		\caption{The traditional random fault injection vs. ARAT}
		\label{fig:comparison_fi}
	\end{center}
\vspace{-0.7cm}
\end{figure}


We first notice from Table~\ref{tab:comparison} that 
%across 5 sets of random fault injection tests, there are big variances (up to 55.9\% in $exp1$ of FT) in terms of the success rate. 
the results of 5 test sets can be quite different from each other and from 1000 random fault inject tests (up to 29.3\%).
1000 fault injection tests provide better statistical significance than 200 fault injection tests.
We expect 1000 fault injection tests potentially provide higher accuracy to quantify the application resilience.
The above result difference is clearly an indication to the randomness of fault injection, and there
is no guarantee on the random fault injection accuracy.

%In Figure~\ref{fig:comparison_fi}, 
We compare the success rate of 1000 fault inject tests with the aDVF value (Fig.~\ref{fig:comparison_fi}). 
We find that the order of the success rate of the three data objects in CG (i.e., $colidx < a < x$) and the two data objects in FT 
(i.e., $exp1 < plane$) is the same as the order of the aDVF values of these data objects. 
%In fact, 1000 fault injection tests
%account for \textcolor{blue}{\textbf{xxx\%}} of total memory references to the data object,
%and provide better resilience quantification than 200 fault injection tests.
The same order (or the same resilience trend)
%between our approach and the random fault injection based on a large number of tests 
is a demonstration of the effectiveness of our approach.
Note that the values of the aDVF and success rate %for a data object
cannot be exactly the same (even if we have sufficiently large numbers of random fault injection), 
because aDVF and random fault injection quantify
the resilience based on different metrics.
Also, the random fault injection can miss some fault masking events that can be captured by our approach.

\end{comment}
% \vspace{-0.5em}
\section{Conclusion}
% \vspace{-0.5em}
Recent advances in multimodal single-cell technology have enabled the simultaneous profiling of the transcriptome alongside other cellular modalities, leading to an increase in the availability of multimodal single-cell data. In this paper, we present \method{}, a multimodal transformer model for single-cell surface protein abundance from gene expression measurements. We combined the data with prior biological interaction knowledge from the STRING database into a richly connected heterogeneous graph and leveraged the transformer architectures to learn an accurate mapping between gene expression and surface protein abundance. Remarkably, \method{} achieves superior and more stable performance than other baselines on both 2021 and 2022 NeurIPS single-cell datasets.

\noindent\textbf{Future Work.}
% Our work is an extension of the model we implemented in the NeurIPS 2022 competition. 
Our framework of multimodal transformers with the cross-modality heterogeneous graph goes far beyond the specific downstream task of modality prediction, and there are lots of potentials to be further explored. Our graph contains three types of nodes. While the cell embeddings are used for predictions, the remaining protein embeddings and gene embeddings may be further interpreted for other tasks. The similarities between proteins may show data-specific protein-protein relationships, while the attention matrix of the gene transformer may help to identify marker genes of each cell type. Additionally, we may achieve gene interaction prediction using the attention mechanism.
% under adequate regulations. 
% We expect \method{} to be capable of much more than just modality prediction. Note that currently, we fuse information from different transformers with message-passing GNNs. 
To extend more on transformers, a potential next step is implementing cross-attention cross-modalities. Ideally, all three types of nodes, namely genes, proteins, and cells, would be jointly modeled using a large transformer that includes specific regulations for each modality. 

% insight of protein and gene embedding (diff task)

% all in one transformer

% \noindent\textbf{Limitations and future work}
% Despite the noticeable performance improvement by utilizing transformers with the cross-modality heterogeneous graph, there are still bottlenecks in the current settings. To begin with, we noticed that the performance variations of all methods are consistently higher in the ``CITE'' dataset compared to the ``GEX2ADT'' dataset. We hypothesized that the increased variability in ``CITE'' was due to both less number of training samples (43k vs. 66k cells) and a significantly more number of testing samples used (28k vs. 1k cells). One straightforward solution to alleviate the high variation issue is to include more training samples, which is not always possible given the training data availability. Nevertheless, publicly available single-cell datasets have been accumulated over the past decades and are still being collected on an ever-increasing scale. Taking advantage of these large-scale atlases is the key to a more stable and well-performing model, as some of the intra-cell variations could be common across different datasets. For example, reference-based methods are commonly used to identify the cell identity of a single cell, or cell-type compositions of a mixture of cells. (other examples for pretrained, e.g., scbert)


%\noindent\textbf{Future work.}
% Our work is an extension of the model we implemented in the NeurIPS 2022 competition. Now our framework of multimodal transformers with the cross-modality heterogeneous graph goes far beyond the specific downstream task of modality prediction, and there are lots of potentials to be further explored. Our graph contains three types of nodes. while the cell embeddings are used for predictions, the remaining protein embeddings and gene embeddings may be further interpreted for other tasks. The similarities between proteins may show data-specific protein-protein relationships, while the attention matrix of the gene transformer may help to identify marker genes of each cell type. Additionally, we may achieve gene interaction prediction using the attention mechanism under adequate regulations. We expect \method{} to be capable of much more than just modality prediction. Note that currently, we fuse information from different transformers with message-passing GNNs. To extend more on transformers, a potential next step is implementing cross-attention cross-modalities. Ideally, all three types of nodes, namely genes, proteins, and cells, would be jointly modeled using a large transformer that includes specific regulations for each modality. The self-attention within each modality would reconstruct the prior interaction network, while the cross-attention between modalities would be supervised by the data observations. Then, The attention matrix will provide insights into all the internal interactions and cross-relationships. With the linearized transformer, this idea would be both practical and versatile.

% \begin{acks}
% This research is supported by the National Science Foundation (NSF) and Johnson \& Johnson.
% \end{acks}

\vspace{-0.05in}
\begin{acks}
Lei Zou was supported by the National Key Research and Development Program of China (2016YFB1000603)  and NSFC (No. 61622201 and 61532010).
Jeffery Xu Yu was supported by the Research Grant Council of Hong Kong SAR, China (No. 14221716).
Lei Zou is the corresponding author of this work.
\end{acks}
\vspace{-0.05in}
\bibliographystyle{ACM-Reference-Format}
\bibliography{sigproc} 

\end{document}
