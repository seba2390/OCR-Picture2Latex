\documentclass[sigconf]{acmart}

\usepackage{booktabs} % For formal tables
%\usepackage{etex}
%\usepackage{epstopdf}
%\usepackage{amsmath}
%%\usepackage{txfonts}
%\usepackage{amssymb}
%\usepackage{times}
\usepackage{graphicx}
%\usepackage{epsfig}
%%\usepackage{hyperref}
\usepackage[linesnumbered,ruled,noend]{algorithm2e}
%\usepackage[noend]{algorithmic}
\usepackage{multirow}
%\usepackage{listings}
\usepackage{threeparttable}
\usepackage{tikz}
%\usepackage[T1]{fontenc}
\usepackage{pgfplots}
\usepackage{pgfplotstable}
%\usepackage{colortbl}
%\usepackage{array}
%\usepackage{eurosym}
\usepackage{caption}
\usepackage{subcaption}
%\usepackage{tikz}
%\usetikzlibrary{patterns}

%\usepackage{url}
%\usepackage{amsfonts}
%\usepackage{breakurl}
%\usepackage{tabularx}
%\usepackage{makecell}
%%\usepackage{floatrow}
%\usepackage{balance}  % for  \balance command ON LAST PAGE  (only there!)
%\usepackage{soul}
%\usepackage{times}
%\usepackage{indentfirst}
%\usepackage{verbatim}

\newcommand{\nop}[1]{}%
\newcommand{\Paragraph}[1]{~\vspace*{-0.9\baselineskip}\\{\bf #1}}
% Copyright
%\setcopyright{none}
%\setcopyright{acmcopyright}
%\setcopyright{acmlicensed}
%\setcopyright{rightsretained}
%\setcopyright{usgov}
%\setcopyright{usgovmixed}
%\setcopyright{cagov}
%\setcopyright{cagovmixed}


\copyrightyear{2017}
\acmYear{2017}
\setcopyright{acmcopyright}
\acmConference{CIKM'17 }{November 6--10, 2017}{Singapore,
	Singapore}\acmPrice{15.00}\acmDOI{10.1145/3132847.3132957}
\acmISBN{978-1-4503-4918-5/17/11}

\fancyhead{}
\settopmatter{printacmref=false, printfolios=false}

\begin{document}
\title{Keyword Search on RDF Graphs --- A Query Graph Assembly Approach}

 \author{
	% author names are typeset in 11pt, which is the default size in the author block
	{Shuo Han{${^1}$}, Lei Zou{${^1}$}, Jeffery Xu Yu{${^2}$}, Dongyan Zhao{${^1}$}} \\
	% add some space between author names and affils
%	\vspace{1.6mm}\\
	\fontsize{10}{\baselineskip}\selectfont\itshape $~^{1}$Peking University, China;\\
	\fontsize{10}{\baselineskip}\selectfont\itshape $~^{2}$ The Chinese University of Hong Kong, China; \\
	\fontsize{9}{\baselineskip}\selectfont\ttfamily\upshape $\{$hanshuo,zoulei,zhaody$\}$@pku.edu.cn, yu@se.cuhk.edu.hk\\
}



\begin{abstract}
Keyword search provides ordinary users an easy-to-use interface for querying RDF data. Given the input keywords, in this paper, we study how to assemble a query graph that is to represent user's query intention accurately and efficiently. Based on the input keywords, we first obtain the elementary query graph building blocks, such as entity/class vertices and predicate edges. Then, we formally define the \emph{query graph assembly (QGA)} problem. Unfortunately, we prove theoretically that QGA is a NP-complete problem. In order to solve that, we design some heuristic lower bounds and propose a bipartite graph matching-based best-first search algorithm. The algorithm's time complexity is $O(k^{2l} \cdot l^{3l})$, where $l$ is the number of the keywords and $k$ is a tunable parameter, i.e., the maximum number of candidate entity/class vertices and predicate edges allowed to match each keyword. Although QGA is intractable, both $l$ and $k$ are small in practice. Furthermore, the algorithm's time complexity does not depend on the RDF graph size, which guarantees the good scalability of our system in large RDF graphs. Experiments on DBpedia and Freebase confirm the superiority of our system on both effectiveness and efficiency. 
\end{abstract}

%
% The code below should be generated by the tool at
% http://dl.acm.org/ccs.cfm
% Please copy and paste the code instead of the example below. 
%
%\begin{CCSXML}
%	<ccs2012>
%	<concept>
%	<concept_id>10002951.10003317</concept_id>
%	<concept_desc>Information systems~Information retrieval</concept_desc>
%	<concept_significance>500</concept_significance>
%	</concept>
%	<concept>
%	<concept_id>10002951.10003317.10003325.10003327</concept_id>
%	<concept_desc>Information systems~Query intent</concept_desc>
%	<concept_significance>500</concept_significance>
%	</concept>
%	<concept>
%	<concept_id>10002951.10003317.10003325.10003330</concept_id>
%	<concept_desc>Information systems~Query reformulation</concept_desc>
%	<concept_significance>500</concept_significance>
%	</concept>
%	</ccs2012>
%\end{CCSXML}
%
%\ccsdesc[500]{Information systems~Information retrieval}
%\ccsdesc[500]{Information systems~Query intent}
%\ccsdesc[500]{Information systems~Query reformulation}



\keywords{keyword search; RDF; graph data management}

\maketitle

\section{Introduction}  \label{sec:introduction}

\newcommand\inexpIntro[3]{#1?(#2,#3).}
\newcommand\rinexpIntro[3]{*#1?(#2,#3).}
\newcommand\outexpIntro[3]{#1!(#2,#3).}
\newcommand\outatomIntro[3]{#1!(#2,#3)}

We propose a fully automated method for proving termination of \(\pi\)-calculus processes.
Although there have been a lot of studies on termination analysis for the \(\pi\)-calculus
and related calculi~\cite{Deng06IC,Demangeon07,SangiorgiTermination,KobayashiHybrid,Yoshida04IC,DBLP:journals/jlp/DemangeonHS10,Venet98SAS}, most of them have been rather theoretical,
and there have been surprisingly little efforts in developing  fully automated termination
verification methods and tools based on them. To our knowledge,
Kobayashi's \typical{}~\cite{TyPiCal,KobayashiHybrid} is the only exception that
can prove termination of \(\pi\)-calculus processes (extended with natural numbers)
fully automatically, but its termination analysis is quite limited (see Section~\ref{sec:relatedwork}).

Our method is based on a reduction to termination analysis for sequential programs:
we translate a \(\pi\)-calculus process \(P\) to a sequential program \(S_P\), so that
if \(S_P\) is terminating, so is \(P\). The reduction allows us to use
powerful, mature methods and tools
for termination analysis of sequential programs~\cite{heizmann2016ultimate,freqterm,DBLP:conf/lics/PodelskiR04,Kuwahara2014Termination,DBLP:journals/cacm/CookPR11}.

The idea of the translation is to convert a chain of communications on replicated input
channels to a chain of recursive function calls of the target sequential program.
Let us consider the following Fibonacci process:
\begin{align*}
    & \rinexpIntro{\fib}{n}{r}
        \ifexp{n<2}{ \soutatom{r}{1} \\ &\quad}
                   { \nuexp{s_1} \nuexp{s_2} (\outatomIntro{\fib}{n-1}{s_1} \PAR \outatomIntro{\fib}{n-2}{s_2} \PAR \sinexp{s_1}{x}\sinexp{s_2}{y}\soutatom{r}{x+y}) \\}
    & \PAR \outatomIntro{\fib}{m}{r}
\end{align*}
Here, the process
$\rinexpIntro{\fib}{n}{r} \ldots$ is a function server that computes the \(n\)-th Fibonacci number
in parallel and returns the result to \(r\),
and $\outatom{\fib}{m}{r}$ sends a request for computing the \(m\)-th Fibonacci number;
those who are not familiar with the syntax of the \(\pi\)-calculus may wish to consult
Section~\ref{sec:targetlanguage} first.
To prove that the process above is terminating for any integer \(m\),
it suffices to show that there is no infinite chain of communications on $\fib$:
\[
    \fib(m,r) \to \fib(m_1,r_1) \to \fib(m_2,r_2) \to \cdots.
\]
We convert the process above to the following program:\footnote{The actual translation
  given later is a little more complex.}
\begin{verbatim}
 let rec fib(n) = if n<2 then () else (fib(n-1) [] fib(n-2)) in
 fib(m)
\end{verbatim}
Here, \texttt{[]} represents the non-deterministic choice.
Note that, although the calculation of Fibonacci numbers is not preserved,
for each chain of communications on \texttt{fib}, there is a corresponding
sequence of recursive calls:
\[
\mathtt{fib}(m) \to \mathtt{fib}(m_1) \to \mathtt{fib}(m_2) \to \cdots.
\]
Thus, the termination of the sequential program above implies the termination of
the original process.
As shown in the example above, (i) each communication on a replicated input channel
is converted to a function call, (ii) each communication on a non-replicated input
channel is just removed (or, in the actual translation, replaced by a call of
a trivial function defined by \(f(\seq{x})=(\,)\)), and (iii) parallel composition
is replaced by a non-deterministic choice.
We formalize the translation outlined above and prove its correctness.

The basic translation sketched above sometimes loses too much information.
For example, consider the following process:
\begin{align*}
    & \rinexpIntro{\pre}{n}{r} \soutatom{r}{n-1} \\
    & \PAR \rinexpIntro{f}{n}{r} \ifexp{n<0}{ \soutatom{r}{1} }
                                       { \nuexp{s} (\outatomIntro{\pre}{n}{s} \PAR \sinexp{s}{x}\outatomIntro{f}{x}{r}) } \\
    & \PAR \outatomIntro{f}{m}{r}
\end{align*}
The translation sketched above would yield:
\begin{verbatim}
  let pred(n) = n-1 in
  let rec f(n) = if n<0 then () else (pred(n) [] f(*)) in
  f(m)
\end{verbatim}
Here, \texttt{*} represents a non-deterministic integer: since we have removed
the input $\sinatom{s}{x}$, we do not have information about the value of \( x \).
As a result, the sequential program above is non-terminating, although the original
process is terminating.
To remedy this problem, we also refine the basic translation above by using a refinement
type system for the \(\pi\)-calculus. Using the refinement type system,
we can infer that the value of \(x\) in the original process is less than \(n\),
so that we can refine the definition of \texttt{f} to:
\begin{verbatim}
 let rec f(n) = ... else (pred(n) [] let x=* in assume(x<n);f(x))
\end{verbatim}
The target program is now terminating, from which
we can deduce that the original process is also terminating.
We have implemented an automated tool based on the refined translation above.

The contributions of this paper are summarized as follows.
\begin{itemize}
\item The formalization of the basic translation from the \(\pi\)-calculus
  (extended with integers) to sequential programs, and a proof of its correctness.
\item The formalization of a refined translation based on a refinement type system.
\item An implementation of the refined translation, including automated refinement type
  inference based on CHC solving, and experiments to evaluate the effectiveness of
  our method.
\end{itemize}

The rest of this paper is structured as follows.
Section~\ref{sec:targetlanguage} introduces the source and target languages
of our translation.
Section~\ref{sec:approach} 
formalizes the basic translation, and proves its correctness.
Section~\ref{sec:refinement} refines the basic translation by using a refinement type system.
Section~\ref{sec:implementation} reports an implementation and experiments.
Section~\ref{sec:relatedwork} discusses related work,
and Section~\ref{sec:conclusion} concludes the paper.

The industry standard for pose edition is to create rigs, a collection of pieces of software designed to manipulate a character's skeleton. The rig describes the skeleton's bones, how they relate to each other, are constrained in their possible motion and are deformed. These rules are loosely specified and creating a good rig requires a detailed understanding of physics and anatomy, as well as technical and artistic skills. Rigging is thus a time consuming task even for experienced animators, and even more so in large scale productions which often require a different in-depth rig for each character in the cast.
Previous work has helped alleviate this difficulty by providing efficient tools to speed up/and or ease the rigging process, relying on inverse kinematics or data-driven methods.
\subsection{Character pose design}
\subsubsection{Inverse Kinematics (IK)}
IK solvers are a family of methods commonly used in robotics, engineering and computer graphics, in which the parameterization of a kinematic chain is determined from the position of its end effector.
They are a staple tool in pose design software, ensuring the respect of elementary constraints during pose edition. Their de-facto role is to guarantee the length of the limbs, and in some cases to enforce the orientation angle range of a joint.
Many IK solutions have been studied over the years \cite{aristidou_inverse_2018}; usually revolving around approximated linearizations or heuristics. 

Numerical methods require a set of iterations to achieve a satisfactory solution formulated by a cost function to be minimized.
IK solutions can generally be divided into three sub-categories: Jacobian \cite{Siciliano_Handbook_Robot_2007}, Newtonians \cite{cohen_ik_1996} and Heuristics. Most software implement heuristic methods such as Cyclic Coordinate Descent (CCD) \cite{wang_ccd_1991} or 
Forward-Backward Reaching IK (FABRIK) \cite{aristidou_fabrik:_2011} due to their simplicity and extensibility. 

The main drawback of 
these solvers is that they manipulate kinematic chains without taking into account many morphological aspects that make a pose more or less plausible. They offer a first level of help to users but are not sufficient to guarantee a realistic pose. Many joints constraints are dependent on each other and require subjective, human-made approximations.

\subsubsection{Data-driven pose edition}
Data-driven methods offer promising opportunities to solve these approximations. Using real-life data can help in modelling the complex inter-dependencies of skeletons and providing users with smarter edition tools.
While it is still an early field of research, some solutions have been studied. Wu \etal \cite{wu_posing_2009} propose a method for natural character posing from a large motion database. It employs adaptive KD-clustering to select a representative frame from a database and sparse approximations to accelerate training and posing. 
Huang \etal in \cite{Huang_IK_MGDM_2017} present a method based on the formulation of multi-variate Gaussian distribution models (MGDMs), which learn the joint constraints of a kinematic skeleton from motion capture data. 

Some work has also been dedicated to finding new editing interfaces. \modify{}{Instead of the usual setup manipulating joints directly, Guay \etal \cite{guay_line_2013} articulate a framework based on the conceptual "line of action" which describes the overall pose dynamics. They provide a mathematical definition of the line of action, and a interface in which the software modifies the pose to follow a user-provided line. In the same line of though} Garcia \etal \cite{garcia_sketching_2019} propose \modify{a method transforming doodle of trajectories (position and orientation over time) }{a virtual reality-based interface where the user's hands motion (position and orientation over time) are transformed} into sequences of actions and then into detailed character animations using a dataset of parametrized motion clips automatically fitted to the trajectory. 

% ==> DL et Latent Space. 
\subsection{Neural modelling of human motion}
Neural networks have received a great amount of attention over the last decade and shown impressive result in modelling complex data. Human motion has not been spared and deep learning methods have proven their capability of generating realistic motion in a number of difficult cases. 

The literature in neural-based animation include example in user-controlled character navigation \cite{Holden2017} and interactions with the environment \cite{starke_neural_2019}. 
Holden \etal \cite{Holden2020} also show that neural networks can be used to replace parts of existing data-driven methods, improving their scalability potential.
More recently, some work has also focused on improving smaller parts of the animation pipeline rather than replacing it completely. Berson et al. \cite{berson_intuitive_2020} leverage neural networks to provide an interactive system to edit facial animation. 

% Wrap up
Data-driven IK and pose editing can relieve animators from time-consuming, back-and-forth pose adjustments by applying constraints extracted from real-world data. Recently, neural-network-based approaches have demonstrated their ability to model the intricacies of human motion while scaling to large amount of data and retaining a fast inference time. In this paper we seek to take advantage of these properties to create an efficient posing tool, intuitively usable even by a inexperienced user.
\section{Problem Definition}\label{sec:problemdef}

\begin{figure} [t]
	\centering
	\scalebox{1.05}[1.15]
	{
		\resizebox{\linewidth}{!}
		{
			\includegraphics[scale=1.0]{visio_pics/rdf_graph_new.pdf}
		}
	}
	\vspace{-0.1in}
	\caption{A Sample of DBpedia RDF Graph.}
	\label{fig:rdfgraph}
	\vspace{-0.2in}
\end{figure}

In this section, we define our problem and review the terminologies used throughout this paper. As a de facto standard model of knowledge base, RDF represents the assertions by $\langle$subject, predicate, object$\rangle$ triples. An RDF dataset can be represented as a graph naturally, where subjects and objects are vertices and predicates denote directed edges between them. A running example of RDF graph is illustrated in Figure \ref{fig:rdfgraph}. Formally, we have the definition about RDF graph as follows.


%\vspace{-0.1in}
\begin{definition} \label{def:rdfgraph}
	\textbf{ (RDF Graph) }
	An \textit{RDF graph} is denoted as $G(V, E)$, where
	$V$ is the set of entity and class vertices corresponding to subjects and objects of RDF triples,
	and $E$ is the set of directed relation edges with their labels corresponding to predicates of RDF triples.

\end{definition}

Note that RDF triple's object may be literal value, for example, $\langle$res:Alan\_Turing, dbo:deathDate, ``1954-06-07'' $\textasciicircum{}\textasciicircum{}$xms:date$\rangle$. We treat all literal values as entity vertices in RDF graph, and literal types (e.g. xms:date as for ``1954-06-07'') as class vertices. 

SPARQL is the standard structural query language of RDF, which can also be represented as a \emph{query graph} defined as follows.
%Answering SPARQLs is equivalent to evaluate the query by subgraph matching using homomorphism \cite{zou2014gstore}. 

\begin{definition} \textbf{ (Query Graph) }
	A \textit{query graph} is denoted as $Q(V_Q, E_Q)$, where
	$V_Q$ consists of entity vertices, class vertices, and vertex \emph{variables},
	and $E_Q$ consists of relation edges as well as edge \emph{variables}.
\end{definition}

%Although SPARQL provides a systematic approach to accessing RDF graph, the complexity of the SPARQL syntax makes it hard to use. 
In this paper, we study the keyword search problem over RDF graph. Given a keyword token sequence $RQ = \{k_1, k_2, ..., k_m\}$, our problem is to interpret $RQ$ as a \emph{query graph} $Q$. 

\section{System Overview} \label{sec:overview}

In this section, we give an overview of our keyword search system. Our approach can be summarized as a two-phase framework, as illustrated in Figure \ref{fig:overview}. 

\begin{figure} [h]
	\centering
	\scalebox{1.0}[1.0]
	{
		\resizebox{\linewidth}{!}
		{
			\includegraphics[scale=0.5]{visio_pics/approach_overview_simp.pdf}
		}
	}
	\vspace{-0.3in}
	\caption{An Overview of Our Approach.}
	\label{fig:overview}
	\vspace{-0.2in}
\end{figure}

\subsection{Phase-I: Segmentation and Annotation} \label{sec:phase1}
The first phase is to segment the keyword token sequence $RQ = \{k_1, k_2, ..., k_m\}$ into several \emph{terms} and each term is annotated with one of the three characters $\{$entity, class, relation$\}$. The converted query is called \emph{annotated query}. Formally, we denote an \textit{annotated query} as $AQ= \{t_1:c_1, t_2:c_2, ..., t_l:c_l\}$, where each $t_i$ is a term and  $c_i \in \{$entity, class, relation$\}$. Note that the first phase (i.e., segmentation and annotation) is not the focus of this paper, as it has been studied extensively \cite{hua2015short,han2011generative, li2011faerie, nakashole2012patty,cai2013large}. We briefly describe the implementation of the first phase as follows. 

For each continuous subsequence $s$ in $RQ$, we check whether it could be matched to an entity, a class, or a relation of the RDF dataset, by employing the existing techniques of entity linking \cite{han2011generative,li2011faerie, ratinov2011local} and relation paraphrasing \cite{nakashole2012patty,cai2013large}. If $s$ is matched, we regard $s$ as a \emph{candidate term} $t_i$, and annotate it with the corresponding character (entity, class, or relation).
We may find that two candidate terms $t_i$ and $t_j$ \emph{overlap} with each other. We say $t_i$ overlaps with $t_j$ if and only if they have at least one common token. Obviously, if two terms overlap, they cannot occur at the same segmentation result. For example, ``university'' and ``university locate USA'' cannot occur in the same segmentation result. We build a \emph{candidate term graph} to describe the mutually exclusive relations: (1) each candidate term $t_i$ is represented a vertex; (2) there is an edge between $t_i$ and $t_j$ if and only of there is \emph{no} overlapping tokens between $t_i$ and $t_j$. Thus each maximal clique in the candidate term graph stands for a possible segmentation result. To obtain top-$N$ best $AQ$, we employ the maximal clique algorithm \cite{bron1973algorithm}, and adopt the pairwise metrics in \cite{hua2015short} to rank the segmentation result.
In out example, we get the top-2 $AQ$ as shown in Figure \ref{fig:overview}.

In the first phase, we have converted keyword token sequence $RQ$ into top-$N$ $AQ$ by some off the shelf techniques. Furthermore, these terms in $AQ$ have been matched to some elementary query graph building blocks (i.e., entity/class vertices and predicate edges). Specifically, if a term $t_i$ is annotated with ``entity'' or ``class'', it will be matched to candidate entity/class vertices in RDF graph.
%We do not distinguish entity vertex or class vertex until generating SPARQL, since we have the same operation on both of them.
If a term $t_i$ is annotated with ``relation'', it will be matched to a set of candidate predicates.

\vspace{-0.05in}
\begin{example}\label{example:aq}
	Given a keyword token sequence $RQ = \{$scientist, graduate, from, university, locate, USA$\}$, we obtain the annotated query $AQ=\{$``scientist'':class, ``graduate from'':relation, ``university'':class, ``locate'':relation, ``USA'':entity $\}$, where ``scientist'' is matched to $\{$dbo:Scientist$\}$, ``university'' is matched to $\{$dbo:University$\}$, ``USA'' is matched to two possible entities $\{$res:USA$\_$Today, res:United$\_$\\States$\}$ due to the ambiguity. Also, the relations ``graduate from'' and ``locate'' also match to two candidate predicates $\{$dbo:almaMater, dbo:education$\}$ and $\{$dbo:country, dbo:location$\}$
\end{example}

\vspace{-0.1in}
\subsection{Phase-II: Query Graph Assembly}
In the second phase, we concentrate on how to assemble a query graph $Q$ based on these elementary building blocks. Formally, the query graph assembly problem is defined as follows:

\vspace{-0.05in}
\begin{definition} \textbf{(Query Graph Assembly Problem)}\label{def:querygraphassembling}
	Given $n$ terms $t^{v}_i$ ($i=1,...,n$) annotated with ``entity'' or ``class'', and $m$ terms $t_j^{e}$ ($j=1,...,m$) annotated with ``relation'', each term $t^{v}_i$ is matched to a set $V_i$ of candidate entity/class vertices and each $t^{e}_j$ is matched to a set $E_j$ of candidate predicate edges. Let $\Upsilon=\{V_1, ..., V_n\}$ and $\Gamma=\{E_1, ..., E_m\}$. A valid assembly query graph is denoted as $Q (V_Q,E_Q)$, which satisfies the following constraints:
	\begin{enumerate}
		\item $|V_Q|=n$, and $\forall V_i \in \Upsilon, V_Q \cap V_i \not= \phi$; 
		\emph{/*each entity or class vertex set $V_i$ has exactly one vertex in  $Q$*/}
		\item $|E_Q|=m$, and $\forall E_j \in \Gamma, E_Q \cap E_j \not= \phi$. 
		\emph{/*each predicate edge set $E_j$ has exactly one edge in $Q$*/}
	\end{enumerate}
	Each edge $e(\langle v_1, v_2\rangle,p) \in E_Q$ connects a pair of vertices $\langle v_1, v_2\rangle \in V_Q$ by a predicate $p$.
	The assembly cost of $Q$ is
	\begin{equation} \label{equ:cost}
	cost(Q)=\sum_{e(\langle v_1, v_2\rangle,p) \in E_Q}{w(\langle v_1, v_2\rangle,p)}
	\end{equation}
	where  $w(\langle v_1, v_2\rangle,p)$ denotes the triple assembly cost. 
	
	
	The \emph{query graph assembly} (QGA for short) problem is to construct a valid graph $Q$ with the minimum assembly cost. 
\end{definition} 

%\vspace{-0.06in}

There are two aspects that should be explained for QGA.

\textbf{\emph{1. Constraints:}}
The two constraints in Definition \ref{def:querygraphassembling} mean that each term $t^{v}_i$ ($1\leq i \leq n$) only corresponds to a single entity/class vertex in $Q$. For example, although ``USA'' may match two candidate entities dbo:USA$\_$Today and dbo:United$\_$States, in the final query graph $Q$, ``USA'' only matches a single entity (dbo:United$\_$States). It is analogue for the relation term $t_j^{e}$ ($1\leq j \leq m$). 

\textbf{\emph{2. Disengaged Edges:}}
A \emph{predicate edge} $e(\langle \cdot,\cdot \rangle,p)$ (in $E_j$) does not have two fixed endpoints but its edge label is fixed to predicate $p$. Thus, a predicate edge can be also called a \emph{disengaged} edge. The triple assembly cost $w(\langle v_1, v_2\rangle,p)$ measures the goodness of assembling $\langle v_1, v_2\rangle$ and $p$ into an edge in $Q$. Then the goal of the QGA problem is to determine the endpoints of $e(\langle \cdot,\cdot \rangle,p)$ to minimize the overall $cost(Q)$.

After finding the optimal $Q$ with minimum $cost(Q)$, we can translate it to SPARQL statements naturally, as illustrated in Figure \ref{fig:overview}.

\subsection{Graph Embedding Cost Model}
Note that the triple assembly cost $w(\langle v_1, v_2\rangle,p)$ can be any positive cost function, which does not affect the hardness of QGA. In other words, the QGA problem is a general computing framework to interpret the input keywords as SPARQL, which does not depend on any specific triple assembly cost function.

\begin{figure} [b]
	\centering
	\scalebox{0.55} [0.50]
	{
		\resizebox{\linewidth}{!}
		{
			\includegraphics[scale=1.0]{visio_pics/transe_visualizing.pdf}
		}
	}
	\caption{Visualizing the Intuition of Graph Embedding.}
	\label{fig:transe_visualizing}
	\vspace{-0.2in}
\end{figure}

The only thing affected by the selection of triple assembly cost function is the system's accuracy. A good cost function can guide to assemble correct query graph $Q$ that implies users' query intention. The process of assembling $\langle v_1, v_2\rangle$ and $p$ into a triple is analogue to ``link prediction'' problem in the RDF knowledge graph \cite{miller2009nonparametric}. Given two entity/class vertices $v_1$ and $v_2$, the link prediction is to ``predict'' the predicate $p$ between $v_1$ and $v_2$, and $w(\langle v_1,v_2\rangle, p)$ is a \emph{measure} of the prediction. Recent research show that the graph embedding technique is superior to other traditional approaches, such as \cite{miller2009nonparametric,nickel2011three,jenatton2012latent}. In the graph embedding model, all subjects (s), objects (o) and predicates (p) are encoded as multi-dimensional vectors $\overrightarrow{s}$, $\overrightarrow{p}$ and $\overrightarrow{o}$ such that $\overrightarrow s  + \overrightarrow p  \approx \overrightarrow o $ if $\langle s,p,o \rangle \in G$ (i.e., $\langle s,p,o\rangle$ is a triple in RDF graph); while $\overrightarrow s  + \overrightarrow p$ should be far away from $\overrightarrow o$ otherwise. Figure \ref{fig:transe_visualizing} visualizes the intuition. From the intuition, the structural information among entities, classes and relations in RDF graph is embedded into vectors. Therefore, we define the triple assembly cost based on graph embedding vectors as follows.


\begin{definition}\textbf{ (Triple Assembly Cost) }. \label{def:tripleassemblycost}
	Given two entity/class vertices $v_1$ and $v_2$ and a predicate edge $p$, the \emph{cost} of assembly triple $(v_1,p,v_2)$ is denoted as follows:
	\begin{equation}\label{equ:tripleassembly}
	w(\langle v_1, v_2\rangle, p) = MIN(| \overrightarrow{v_1} + \overrightarrow p  - \overrightarrow {v_2 } |, | \overrightarrow{v_2} + \overrightarrow p  - \overrightarrow {v_1 } |)
	\end{equation}
	where $\overrightarrow{v_1}$, $\overrightarrow {v_2}$ and $\overrightarrow {p}$ are the encoded multi-dimensional vectors  of $v_1$, $v_2$ and $p$, respectively. 
	%where $|x|_+$ denotes the positive part of $x$. 
\end{definition}  

\begin{figure} [h]
	\centering
	\scalebox{1.0}
	{
		\resizebox{\linewidth}{!}
		{
			\includegraphics[scale=1.0]{visio_pics/query_graph_elements_example2.pdf}
		}
	}
%	\caption{Candidate Entity/Class Vertices and Predicate Edges.}
	\caption{Elementary Query Graph Building Blocks.}
	\label{fig:graph_elements_exp2}
	\vspace{-0.3in}
\end{figure}

\begin{figure} [h]
	\newcommand{\mywidth}{0.23\textwidth}
	\centering
	\begin{subfigure}[t]{\mywidth}
		\centering
		\resizebox{\linewidth}{!}
		{
			\includegraphics{visio_pics/query_assembly_graph_q1.pdf}
			
		}
		\caption{$cost(Q_1)=1.76$}
		\label{fig:assembly_query_graph_q1}
	\end{subfigure}
	\begin{subfigure}[t]{\mywidth}
		\centering
		\resizebox{1.0\linewidth}{!}
		{
			\includegraphics{visio_pics/query_assembly_graph_q2.pdf}
		}
		\caption[font=\small]{$cost(Q_2)=2.46$}
		%        \vspace{0.1in}
		\label{fig:assembly_query_graph_q2}
	\end{subfigure}
	\caption{Possible Assembly Query Graphs.}
	%    \vspace{-0.1in}
	\label{fig:assembly_query_graph}
	\vspace{-0.1in}
\end{figure}

\begin{example} In our example, there are three entity/class terms ``scientist'', ``university'', ``USA'' and two relation terms ``graduate from''  and ``locate''. Their corresponding entity/class vertices and predicate edges are shown in Figure \ref{fig:graph_elements_exp2}. There are two different assembly query graph $Q_1$ and $Q_2$ in Figure \ref{fig:assembly_query_graph}, among which $cost(Q_1)<cost(Q_2)$. Thus, the QGA problem result is $Q_1$ (Figure \ref{fig:assembly_query_graph_q1}).
%It means that we interpret the keywords as a query graph $Q$ and evaluate $Q$ using SPARQL query engine to return answers to users. 
\end{example}
\section{QGA: Hardness and Algorithm}
\label{sec:querygraphassembling}
Unfortunately, QGA is proved to be NP-complete in Section \ref{sec:hardness}. To solve that, we transform QGA into a constrained bipartite graph matching problem and design a practical efficient algorithm to find the optimal $Q$.  

\subsection{Hardness Analysis}\label{sec:hardness}

\begin{theorem}
	The query graph assembly problem is NP-complete.
\end{theorem}

\vspace{-0.1in}
\begin{proof}
	The decision version of QGA is defined as follows: 
	Given $n$ vertex sets $\Upsilon=\{V_1, ..., V_n\}$ and $m$ edge sets $\Gamma=\{E_1, ..., E_m\}$ and a threshold $\theta$, QGA is to decide if there exists an assembly query graph $Q$, where it satisfies the two constraints in Definition \ref{def:querygraphassembling} and the total assembly cost $cost(Q) \leq \theta$. Obviously, an instance of QGA can be verified in polynomial time. Thus, QGA belongs to NP class. 

	We construct a polynomial time reduction from 3-SAT (a classical NP-complete problem) to QGA. More specifically, given any instance $I$ of 3-SAT, we can generate an instance $I^{\prime}$ of QGA within polynomial time, where the decision value (TRUE/FALSE) of $I$ is \emph{equivalent} to $I^{\prime}$.
	
	\textbf{\emph{Any instance $I$ of 3-SAT:}}
	Without loss of generality, we define an instance $I$ of 3-SAT as follows: Given a set of $p$ boolean variables $U=\{u_1,u_2,...,u_p\}$ and a set of $q$ 3-clauses $C=\{c_1,c_2,...,c_q\}$ on $U$, The problem is to decide whether there exists a truth assignment for each variable in $U$ that satisfies all clauses in $C$.
	
	\textbf{\emph{Corresponding instance $I^{\prime}$ of QGA:}}
	Given a variable set $U$ and a 3-clause set $C$ (of instance $I$), we build a group of vertex sets $\Upsilon$ and a group of edge sets $\Gamma$ for the instance $I^{\prime}$.  $\Upsilon$ consists of the two parts $\Upsilon_{U}$ and $\Upsilon_{C}$, which are defined as follows:
	
	\begin{enumerate}
	\item For each variable $u_i \in U$, we introduce a vertex set $\{u_i,\overline{u_i}\}$ into $\Upsilon_{U}$. We call $u_i$ and $\overline{u_i}$ as \emph{variable vertices} of $I^{\prime}$. 
	\item For each 3-clause $c_j \in C$, we introduce a vertex set having a single vertex $\{c_j\}$ into $\Upsilon_{C}$. We call $c_j$ as a \emph{clause vertex} of $I^{\prime}$.  
	\end{enumerate}
	where $\Upsilon=\Upsilon_{U} \cup \Upsilon_{C}$.

	We also introduce $q$ disengaged edges into $\Gamma$. Each edge is a singleton set $\{e_j\}$. These $q$ edges can be used to connect any two vertices in $\Upsilon$. We set edge weight $w(e_j)=0$ if and only if $e_j$ connects the clause vertex $c_j$ and a member variable vertex $u_i$ (or $\overline{u_i}$) in 3-clause $c_j$. Otherwise, the edge weights are 1.
	
	The corresponding instance $I^{\prime}$ is defined as follows: Given two groups $\Upsilon$ and $\Gamma$ (explained above), and a threshold $\theta=0$, the problem is to decide whether there exists a graph $Q$, satisfying the two constraints in Definition \ref{def:querygraphassembling}, and $cost(Q)\leq 0$. Since edge weights are no less than 0, hence our goal is to construct $Q$ with $cost(Q)=0$.
	
	\textbf{\emph{Equivalence:}}
	Next, we will show the equivalence between the instance $I$ (of 3-SAT) and the instance $I^{\prime}$ (of QGA), i.e., $I \Leftrightarrow I^{\prime}$.

	Assume that the answer to $I$ is TRUE, which means that we have a truth assignment for each variable $u_i$ so that all 3-clauses $c_j$ are satisfied. According to the truth assignment in $I$, we can construct a graph $Q$ of $I^{\prime}$ as follows: for each clause vertex $c_j$, we connect $c_j$ with variable vertex $u_i$ (or $\overline{u_i}$) if $u_i=1$ (or $\overline{u_i}=1$, i.e, $u_i=0$) and $u_i$ is included in the 3-clause $c_j$. There may exist multiple variable vertices $u_i$  (or $\overline{u_i}$) satisfying the above condition. We connect $c_j$ with arbitrary one of them to form edge $(c_j,u_i)$ (or $(c_j,\overline{u_i})$).
	These edge weights are 0. 
	%	These edge weights are 0, as shown in Figure \ref{fig:3satproof}.
	Thus, we construct a graph $Q$ satisfying the two constraints in Definition \ref{def:querygraphassembling} and $cost(Q)=0$.	It means the answer to $I^{\prime}$ is TRUE, i.e., $I \Rightarrow I^{\prime}$.
	
	Assume the answer to $I^{\prime}$ is TRUE. It means that we can construct a graph $Q$ having the following characters:
	\begin{enumerate}
		\item For $i=1,...,p$, one of variable vertex $\{u_i, \overline{u_i}\}$ is selected; and for $j=1,...,q$, clause vertex $c_j$ is selected; \emph{/*one vertex of each vertex set in $\Upsilon$ is selected*/}
		\item For $j=1,...,q$, each edge $e_j$ is selected; and $e_j$ connects the clause vertex $c_j$ and a variable vertex $u_i$(or $\overline{u_i}$) that corresponds to one of the three member variables of 3-clause $c_j$.  \emph{/*one edge of each edge set in $\Gamma$ is selected, and $cost(Q)=0$*/}
	\end{enumerate} 
	If a variable vertex $u_i$ (or $\overline{u_i}$) is selected, we set $u_i=1$ (or $\overline{u_i}=1$, i.e., $u_i=0$).  
	Since each clause vertex $c_j$ is connected to one selected variable vertex $u_i$ (or $\overline{u_i}$) that is in the 3-clause $c_j$, the corresponding variable $u_i=1$ (or $\overline{u_i}=1$). It means that 3-clause $c_j=1$ as $u_i$ (or $\overline{u_i}$) is included in $c_j$.  Hence, the answer to $I$ is also TRUE, i.e., $I \Leftarrow I^{\prime}$.
	
	In summary, we have reduced 3-SAT to QGA, where the former is a NP-complete problem. Therefore, we have proved that QGA is NP-complete. 

%	\vspace{-0.1in}
%	\textbf{\emph{Reduction Process:}}
%	We construct a polynomial time reduction from 3-SAT (a classical NP-complete problem) to QGA. Let us recall the 3-SAT problem. Given a set of $p$ boolean variables $U=\{u_1,u_2,...,u_p\}$ and a set of $q$ 3-clauses $C=\{c_1,c_2,...,c_q\}$ on $U$, 3-SAT is to decide whether there exists a truth assignment for each variable in $U$ that satisfies all clauses in $C$.
%%	For example, given $U=\{u_1,u_2,u_3,u_4\}$ and $C=\{c_1=u_1  \vee \overline {u_2 }  \vee u_4, c_2= u_2  \vee \overline {u_3 }  \vee \overline {u_4 } \}$, there exists an assignment $\{u_1=1;u_2=1;u_3=0;u_4=1\}$ that satisfies all 3-clauses $c_1$ and $c_2$ in $C$. 
%	
%	 Given a variable set $U$ and a 3-clause set $C$, we build a group of vertex sets $\Upsilon$ and a group of edge sets $\Gamma$ for the QGA problem.  $\Upsilon$ consists of the two parts $\Upsilon_{U}$ and $\Upsilon_{C}$, which are defined as follows:
%	
%	\begin{enumerate}
%		\item For each variable $u_i \in U$, we introduce a vertex set $\{u_i,\overline{u_i}\}$ into $\Upsilon_{U}$. In QGA, we call $u_i$ and $\overline{u_i}$ as \emph{variable vertices}. 
%		\item For each 3-clause $c_j \in C$, we introduce a vertex set having a single vertex $\{c_j\}$ into $\Upsilon_{C}$. In QGA, we call $c_j$ as a \emph{clause vertex}.  
%	\end{enumerate}
%	where $\Upsilon=\Upsilon_{U} \cup \Upsilon_{C}$.
%	
%	Assume that there are $q$ 3-clauses $c_j$ ($j=1,...,q$) in the 3-SAT problem. We introduce $q$ edges into $\Gamma$. Each edge is a singleton set $\{e_j\}$. These $q$ edges can be used to connect any two vertices in $\Upsilon$. We set edge weight $w(e_j)=0$ if and only if $e_j$ connects the clause vertex $c_j$ and a member variable vertex $u_i$ (or $\overline{u_i}$) in 3-clause $c_j$. Otherwise, the edge weights are 1.
%%	For example, there are two 3-clauses $c_1=u_1  \vee \overline {u_2}  \vee u_4 $ and $c_2= u_2  \vee \overline {u_3}  \vee \overline {u_4}$. Only when the edge $e_1$ connects $c_1$ with $u_1$, $\overline {u_2}$ and  $u_4$, and the edge $e_2$ connects $c_2$ with $u_2$, $\overline{u_3}$ and $\overline{u_4}$, the edge weights are 0. Other connections' edge weights are 1. Figure \ref{fig:3satproof} illustrates that. 
%	
%	The reduced QGA instance is described as follows: Given two groups $\Upsilon$ and $\Gamma$ (defined above), and a threshold $\theta=0$, the problem is to decide whether there exists a graph $Q$, which satisfies the two constraints in Definition \ref{def:querygraphassembling}, and $cost(Q)\leq 0$. Since edge weights are no less than 0, thus, our goal is to construct $Q$ with $cost(Q)=0$.
%	
%	Let us prove that the decision of QGA is equivalent to that over 3-SAT. Assume that the answer to QGA is TRUE. It means that we can construct a graph $Q$ having the following characters:
%	\begin{enumerate}
%		\item For $i=1,..,p$, one of variable vertex $\{u_i, \overline{u_i}\}$ is selected;  and for $j=1,...,q$, clause vertex $c_j$ is selected; \emph{/*one vertex of each vertex set in $\Upsilon$ is selected*/}
%		\item For $j=1,...,q$, each edge $e_j$ is selected; and $e_j$ connects the clause vertex $c_j$ and a variable vertex $u_i$(or $\overline{u_i}$) that corresponds to one of the three member variables of 3-clause $c_j$.  \emph{/*one edge of each edge set in $\Gamma$ is selected, and $cost(Q)=0$*/}
%	\end{enumerate} 
%	If a variable vertex $u_i$ (or $\overline{u_i}$) is selected, we set $u_i=1$ (or $\overline{u_i}=1$, i.e., $u_i=0$).  
%	Since each clause vertex $c_j$ is connected to one selected variable vertex $u_i$ (or $\overline{u_i}$) that is in the 3-clause $c_j$, the corresponding variable $u_i=1$ (or $\overline{u_i}=1$). It means that 3-clause $c_j=1$ as $u_i$ (or $\overline{u_i}$) is included in $c_j$.  Therefore, we prove that if the answer to QGA is TRUE then the answer to 3-SAT is also TRUE. 
%	
%%	For example, given two 3-clauses $c_1=u_1  \vee \overline {u_2}  \vee u_4 $ and $c_2= u_2  \vee \overline {u_3}  \vee \overline {u_4}$, we build six vertex sets in Figure \ref{fig:3satproof}. The upper four sets correspond to the variable vertices and the lower two are the clause vertices. Since there are two 3-clauses, we introduce two singleton edge sets in the assembly query graph. Only the solid edges' weight are 0. Assume that we find a QGA solution whose $cost(Q)=0$ (i.e, the answer to QGA is TRUE), as depicted by the red bold solid lines, i.e., the edge connecting $c_1$ and ${u_1}$ and the one connecting $c_2$ and $\overline{u_3}$. Thus, we set variable $u_1=1$ and $\overline{u_3}=1$ (i.e., $u_3=0$). This is a truth assignment satisfying both $c_1$ and $c_2$ no matter what assignment of other variables. It means that it is also true to the 3-SAT instance if we find a solution to the QGA instance. 
%	
%	Let us prove that the answer to 3-SAT is FALSE if the answer to the QGA is FLASE. For the ease of the proof, we prove the corresponding converse negative proposition, i.e, the answer to QGA is TRUE if the answer to 3-SAT is TRUE.  
%	
%	Assume that the answer to 3-SAT is TRUE, which means that we have a truth assignment for each variable $u_i$ so that all 3-clauses $c_j$ are satisfied. According to the truth assignment in 3-SAT, we can construct a graph $Q$ as follows: for each clause vertex $c_j$ in QGA, we connect $c_j$ with variable vertex $u_i$ (or $\overline{u_i}$) if $u_i=1$ (or $\overline{u_i}=1$, i.e, $u_i=0$) and $u_i$ is included in the 3-clause $c_j$. There may exist multiple variable vertices $u_i$  (or $\overline{u_i}$) satisfying the above condition. We connect $c_j$ with one of these variable vertices $u_i$  (or $\overline{u_i}$) to form edge $(c_j,u_i)$ (or $(c_j,\overline{u_i})$).
%	These edge weights are 0. 
%%	These edge weights are 0, as shown in Figure \ref{fig:3satproof}.
%	Thus, we construct a graph $Q$ satisfying the two constraints in Definition \ref{def:querygraphassembling} and $cost(Q)=0$.	
%	
%%	\begin{figure} [h]
%%		\centering
%%		\scalebox{0.80} [0.80]
%%		{
%%			\resizebox{\linewidth}{!}
%%			{
%%				\includegraphics[scale=1.0]{visio_pics/qga_3sat_reduce.pdf}
%%			}
%%		}
%%		\caption{Visualizing the Reduction Process.}
%%		\label{fig:3satproof}
%%		%\vspace{-0.1in}
%%	\end{figure}
%
%	In summary, we have reduced 3-SAT to QGA, while the former is a classical NP-complete problem. Therefore, we have proved that QGA is NP-complete. 
\end{proof}


%\subsection{Algorithm} \label{sec:algorithm}
%Since QGA problem is NP-complete, we aim to design a practical efficient algorithm that can return the exact result. A brute force idea is to enumerate all the possible assembly combinations of the candidate vertices $\Gamma$ and edges $\Upsilon$, check whether each of the combinations is a valid query graph $Q$, and calculate its $cost(Q)$. Then we find the valid $Q$ with minimum $cost(Q)$, as the final result. Suppose $AQ$ consists of $n$ entity/class terms and $m$ relation terms, and each term is matched to at most $k$ candidate vertices (or predicate edges). Then the overall enumeration space is $\left( \begin{array}{l}  n \\  2 \\  \end{array} \right)^m \cdot k^{m+n}$. It is hard to design pruning conditions for such a brute force strategy to reduce its search space. But fortunately, we can transform the QGA problem into a constrained bipartite graph model, then propose a best-first search algorithm with some powerful pruning techniques. We elaborate the model and the algorithm in the following subsections. 


\subsection{Assembly Bipartite Graph Model} \label{sec:general_case}
Since QGA is NP-complete, we transform it to an equivalent bipartite graph model with some constraints. Based on the bipartite graph model, we can design a best-first search algorithm with powerful pruning strategies.

Let us recall the definition of QGA (Definition \ref{def:querygraphassembling}), each of the $n$ entity/class terms is matched to a candidate vertex set $V_i$, and each of the $m$ relation terms is matched to a candidate edge set $E_j$. For example, ``USA'' is matched to $\{$res:USA$\_$Today, res:United$\_$States$\}$, and ``graduate from'' is matched to $\{$dbo:almaMater, dbo:education$\}$. The multiple choices in a candidate vertex/edge set indicate the ambiguity of keywords. If we adopt a pipeline style mechanism to address the keyword disambiguation and the query graph generation separately, we need to select exactly one element from each candidate vertex/edge set in the first phase. In our example, res:United$\_$States is the correct interpretation of ``USA'', and dbo:almaMater should be selected for ``graduate from''. If we simply adopt the string-based matching score \cite{li2011faerie} for keyword disambiguation, the matching score of res:USA$\_$Today may be higher than res:United$\_$States. In this case, if we only select the one with highest matching score, the correct answer will be missed due to the entity linking error. However, a robust solution should be error-tolerant with the ability to construct a correct query graph that is of interest to users even in the presence of noises and errors in the first phase. In our QGA solution, we allow the ambiguity of keywords (i.e., allowing one term matching several candidates) in the first phase, and push down the disambiguation to the query graph assembly step. For example, although the matching score of ``USA'' to res:USA$\_$Today is higher than that to res:United$\_$States, the former's assembly cost is much larger than the latter. Thus, we can still obtain the correct query graph $Q$. We propose the \emph{assembly bipartite graph matching} model to handle the ambiguity of keywords and the ambiguity of query graph structures uniformly. 

\begin{definition}\textbf{ (Assembly Bipartite Graph) }. \label{def:assemblygraph}
	Each entity/class term $t_i^{v}$ corresponds to a set $V_i$ of vertices ($1\leq i \leq n$) and each relation term $t_j^{e}$ corresponds to a set $E_j$ of predicate edges ($1\leq j \leq m$).  
	
	An assembly bipartite graph $\mathbb{B}(V_{L},V_{R},E_{\mathbb{B}})$ is defined as follows:
	
	\begin{enumerate}
		\item Vertex pair set $V_{i_1} \times V_{i_2}=\{(v_{i_1},v_{i_2}) | 1\leq i_1 < i_2 \leq n \wedge v_{i_1} \in V_{i_1}  \wedge v_{i_2} \in V_{i_2}\}$.
		\item $V_{L}=\bigcup\nolimits_{1 \le i_1  < i_2  \le n } {(V_{i_1} \times V_{i_2})} $.
		%\item Edge set $Y_{j}=\{p_j | \alpha+1 \leq j \leq n \wedge p_j \in E_{j}  \}$.
		\item  $V_{R}=\bigcup\nolimits_{1 \le j \le m} {E_j }  $.
		\item there is a crossing edge $e$ between any node $(v_{i_1}, v_{i_2})$ in $V_{L}$ and any node $p_j$ in $V_{R}$ ($1 \le i_1  < i_2  \le n$, $1 \le j \le m$), which is denoted as $e(\langle v_{i_1},v_{i_2}\rangle, p_j)$. Edge weight $w(e)=w(\langle v_{i_1},v_{i_2}\rangle, p_j)$, where $w(\langle v_{i_1},v_{i_2}\rangle, p_j)$ denotes the triple assembly cost.  
	\end{enumerate} 
\end{definition}


\begin{figure} [t]
	\begin{center}
		\vspace{-0.15in}
		\includegraphics[scale=0.55]{visio_pics/assembly_bipartite_graph.pdf}
		%\vspace{-0.1in}
		\caption{An Example of Assembly Bipartite Graph.}
		
		\label{fig:asm_bigraph}
		\vspace{-0.15in}
	\end{center}
\end{figure}

\begin{example} 
Let us recall the running example. Each term's candidate matchings are given in Figure \ref{fig:graph_elements_exp2}, i.e., $V_1=\{$dbo:Scientist$\}$, $V_2=\{$dbo:University$\}$, and $V_3=\{$res:United\_Today, res:United\_States$\}$. Thus, there are three vertex pair sets $V_1 \times V_2$, $V_2 \times V_3$ and $V_1 \times V_3$. In Figure \ref{fig:asm_bigraph}, each vertex pair set is highlighted by a dash circle. There are two predicate edge sets $E_1=\{$dbo:almaMater, dbo:education$\}$ and $E_2=\{$dbo:country, dbo:location$\}$, which are also illustrated in Figure \ref{fig:asm_bigraph} using dash circles. 
\end{example}

It is worth noting that there are some \emph{conflict} relations among the crossing edges in $\mathbb{B}$.  For example, crossing edge $e(\langle$dbo:Scientist, res:USA\_Today$\rangle$, dbo:country) conflicts with $e^{\prime}(\langle$dbo:University, res:United\_States$\rangle$, dbo:almaMater) (in Figure \ref{fig:asm_bigraph}), since the semantic term ``USA'' corresponds to two different entity vertices res:USA\_\\Today and res:United\_States in $e$ and $e^{\prime}$, respectively. In this case, the constraints of QGA (in Definition \ref{def:querygraphassembling}) will be violated if $e$ and $e^{\prime}$ occur in the same matching.
Considering the above example, we formulate the \emph{conflict relation} among crossing edges in $\mathbb{B}$. 

\begin{definition} \textbf{ (Conflict Relation) }.\label{def:conflict}	
	For any two crossing edges $e(\langle v_{i_1},v_{i_2}\rangle, p_j)$ and $e^{\prime}(\langle v_{i_1}^{\prime},v_{i_2}^{\prime}\rangle, p_j^{\prime})$ in an assembly bipartite graph $\mathbb{B}$, we say that $e$ is \emph{conflict} with $e^{\prime}$ if at least one of the following conditions holds:
	
	\begin{enumerate}
		\item $v_{i_1}$ and $v_{i_1}^{\prime}$ (or $v_{i_2}$ and $v_{i_2}^{\prime}$) come from the same vertex set $V_{i_1}$ (or $V_{i_2}$) and  $v_{i_1} \neq v_{i_1}^{\prime}$ (or $v_{i_2} \neq v_{i_2}^{\prime}$);		
		\ /*Two different vertices come from the same vertex set; an entity/class term $t_i^{v}$ cannot map to multiple vertices*/
		\item $p_j$ and $p_j^{\prime}$ come from the same edge set $E_j$ and $p_j \neq p_j^{\prime}$.		
		\ /*Two different predicates come from the same edge set; an relation term $t_j^{e}$ cannot map to multiple predicate edges*/
		\item $(v_{i_1}=v_{i_1}^{\prime} \wedge v_{i_2}=v_{i_2}^{\prime}) \vee (p_j = p_j^{\prime})$ 		
		%/*Two vertex pairs are the same with each other*/
		\ /*$e$ and $e^{\prime}$ share a common endpoint in $V_{L}$ or $V_{R}$; a vertex pair cannot be assigned two different predicate edges and a predicate edge cannot connect two different vertex pairs */
	\end{enumerate}
	
	
\end{definition}

In order to be consistent with the constraints of QGA, we also redefine \emph{matching} as follows.
\begin{definition}\textbf{ (Matching) }\label{def:matchingnew}. Given an assembly bipartite graph $\mathbb{B}(V_{L},V_{R},E_{\mathbb{B}})$, a \emph{matching} of $\mathbb{B}$ is a subset $M$ ($M \subseteq E_{\mathbb{B}}$) of its crossing edges, no two of which are \emph{conflict} with each other.
\end{definition}

\begin{definition}\textbf{ (Matching Cost) }. The \emph{matching cost} of $M$ is $w(M) = \sum\nolimits_{e \in M} {w(e)}$, where $w(e)$ is defined in Definition \ref{def:assemblygraph}(4). 
\end{definition}
It is easy to know that solving QGA problem is equivalent to finding a size-$m$ \emph{matching} over $\mathbb{B}$ with the minimum \emph{matching cost}. A matching edge in $\mathbb{B}$ stands for an assembled edge in $Q$.
%From the assembly bipartite graph model, we design some lower bounds for computing the matching cost, based on which, we propose a best-first search algorithm to find the optimal solution in the following subsection. 

%\vspace{-0.1in}
\subsection{Condensed Bipartite Graph}

\begin{figure} [t]
	\begin{center}
		%\vspace{-0.15in}
		\includegraphics[scale=0.65]{visio_pics/condensed_bipartite_graph.pdf}
		%\vspace{-0.1in}
		\caption{An Example of Condensed Bipartite Graph.}
		\label{fig:con_bigraph}
		   \vspace{-0.2in}
	\end{center}
\end{figure}


According to the condition (2) and (3) in Definition \ref{def:conflict}, each predicate edge set $E_j$ has only one $p_j$ occurring in the \emph{matching} $M$. Thus, we can condense each predicate edge set into one node $E_j$, leading to a \emph{condensed bipartite graph} $\mathbb{B}^{*}(V_{L}^*,V_{R}^*,E_{\mathbb{B}}^*)$, as shown in Figure \ref{fig:con_bigraph}. There is a crossing edge between any node $(v_{i_1},v_{i_2})$ in $V_{L}^*$ and any node $E_j$ in $V_{R}^*$. The edge weight is defined as $w(e) = w(\langle v_{i_1} ,v_{i_2}\rangle ,E_j ) = MIN_{p_j  \in E_j } \{ w(\langle v_{i_1} ,v_{i_2}\rangle,p_j)\}$, where $w(\langle v_{i_1 } ,v_{i_2}\rangle,p_j)$ is defined in Definition \ref{def:assemblygraph}(4). In order to find the size-$m$ matching with the minimum cost in $\mathbb{B}^{*}$, we propose QGA Algorithm (i.e., Algorithm \ref{alg:bfmatch}). 


\subsection{QGA Algorithm}

A search state is denoted as $\{M,Z, cost(M), LB(M)\}$, where $M$ records the current partial matching, i.e., a set of currently selected matching edges, $Z$ records a set of unmatched edges that are not \emph{conflict} with edges in $M$. Obviously, each edge $e \in Z$ can be appended to $M$ to enlarge the current matching. Initially, $M=\phi$, and $Z$ records all crossing edges in the condensed bipartite graph $\mathbb{B}^{*}$ (Line 3 in Algorithm \ref{alg:bfmatch}). $cost(M)$ denotes the current partial matching cost, i.e., $cost(M)= \sum\nolimits_{e \in M} {w(e)}$. $LB(M)$ denotes the lower bound of the current partial matching $M$. We will discuss how to compute the lower bound $LB(M)$ later. All search states are stored in a priority queue $H$ in the non-decreasing order of the lower bound $LB(M)$ (Line 2). Furthermore, we maintain a threshold $\theta$ to be the current minimum matching cost. Initially, $\theta = \infty$. In each iteration, we pop a head state $\{M,Z, cost(M), LB(M)\}$.
We enumerate all unmatched edges $e \in Z$ in non-decreasing order of $w(e)$ to generate subsequent search states. Each $e \in Z$ is moved from $Z$ to $M$ to obtain a new matching $M^{\prime}$ (Line 9). We remove $e$ and all edges in $Z$ that are conflict with $e$ to obtain $Z^{\prime}$ (Line 10). We also update $cost(M^{\prime})$ and $LB(M^{\prime})$ (Lines 11-12). Then we check whether $M^{\prime}$ is a size-$m$ matching over $\mathbb{B}^{*}$(i.e. end state). If so, we update the threshold $\theta$ if $cost(M^{\prime}) < \theta$ and record $M^{\prime}$ as the current optimal matching $M_{opt}$  (Lines 13-16). Otherwise, we push the new state $\{M^{\prime},Z^{\prime}, cost(M^{\prime}), LB(M^{\prime})\}$ into $H$ (Line 18). The algorithm keeps iterating until that the threshold $\theta$ is less than the lower bound of the current head state in $H$ (Line 6) or $H$ is empty. 
\begin{algorithm} 
	\caption{QGA Algorithm} \label{alg:bfmatch}
	\KwIn{Condensed bipartite graph $B^*(V_{L}^*, V_{R}^*, E_B^*)$,
		and conflict relations among $E_B^*$.}
	\KwOut{The optimal assembly query graph $Q$.}
	$M_{opt} := \phi$, $\theta := \infty$ \;
	$H := \phi$ ;\tcp{min-heap, sort by lower bound}
	$H \leftarrow \{M:=\emptyset, Z := E_B^*, cost(M) := 0, LB(M) := 0\}$ \;
	\While{$H$ is not empty}
	{
		$\{M, Z, cost(M), LB(M)\} \leftarrow H.pop$ \;
		\If{$LB(M) >= \theta$}{\textbf{break} \;}
		
		\For{each crossing edge $e \in Z$}
		{
			$M^{\prime} := M \cup \{e\}$; $Z := Z \setminus \{e\}$ \;
			$Z^{\prime} = Z \setminus \{e^\prime | e^\prime \in Z \wedge e^\prime\ conflict\ with\ e\}$ \;
			$cost(M^{\prime}) := cost(M) + w(e)$ \;
			Compute $LB(M^{\prime})$\;
			
			\If{$|M^{\prime}| = m$}
			{
				\If{$cost(M^{\prime}) < \theta$}
				{
					$\theta := cost(M^{\prime})$ \;
					$M_{opt} := M'$ \;
				}       
			}
			\Else
			{
				$H \leftarrow \{M', Z', cost(M^{\prime}), LB(M^{\prime})\}$ \;
			}       
		}
	}
	Build query graph $Q$ according to $M_{opt}$ \;
	\Return $Q$
\end{algorithm}

\vspace{-0.1in}
\subsection{Computing Lower Bound} \label{sec:lower_bound}
Considering a search state $\{M, Z, cost(M), LB(M)\}$, we discuss how to compute $LB(M)$. Let $|M|$ denote the number of edges in the current matching. Since a maximum matching in $B^{*}$ should contain $m$ matching edges (i.e., covering all $E_j$), we need to select the other $(m-|M|)$ no-conflict edges from $Z$, to form a size-$m$ matching. We denote  $\mathcal{M}(Z)$ as the minimum weighted size-$(m-|M|)$ matching of $Z$. Thus, the best result that can be reached from $M$ is $M \cup \mathcal{M}(Z)$. To ensure the correctness of Algorithm \ref{alg:bfmatch}, $LB(M) \le cost(M \cup \mathcal{M}(Z))$ must always be satisfied. A good lower bound should have the following two characters: (1)  $LB(M)$ is as close to $cost(M \cup \mathcal{M}(Z))$ as possible; (2) the computation cost of $LB(M)$ is small. From this intuition, we propose three different lower bounds as follows.

\Paragraph{Naive-LB:}
The naive method to compute $LB(M)$ is to select the top-$(m-|M|)$ unmatched edges $\{e_1,...,e_{m-|M|}\}$ in $Z$ with the minimum weights and compute $Navie$-$LB(M) = cost(M) + \sum\nolimits_{i = 1}^{i = m  - |M|} {w(e_i)}$. In the implementation of Algorithm \ref{alg:bfmatch}, $Z$ is stored by a linked list, and always kept in the non-decreasing order of $w(e)$. Therefore, the complexity of computing Naive-LB is $O(m-|M|)$. Since the top-$(m-|M|)$ unmatched edges in $Z$ may be conflict with each other, Naive-LB is not tight, and has the weakest pruning power comparing with the following two lower bounds.

\Paragraph{KM-LB:}
We ignore the conflict relations among unmatched edges in $Z$, and find the minimum weighted size-$(m-|M|)$ matching $\mathcal{M}_{KM}(Z)$ by KM Algorithm \cite{kuhn1955hungarian} in $O(|Z|^3)$ time. Then we compute $KM$-$LB(M)= cost(M) + cost(\mathcal{M}_{KM}(Z))$. KM-LB is much tighter than Naive-LB, but the computation cost is expensive.

\Paragraph{Greedy-LB:}
Inspired by the matching-based KM-LB, we propose a greedy strategy (Algorithm \ref{alg:greedy_lb}) to find an approximate matching of $Z$ in $O(|Z|)$ time. It has been proved in \cite{preis1999linear} that the result return by Algorithm \ref{alg:greedy_lb} (denote as $\mathcal{M}_{greedy}(Z)$) is a $1/2$-approximation of the optimal matching $\mathcal{M}_{KM}(Z)$. i.e. $cost(\mathcal{M}_{KM}(Z)) \ge \frac{cost(\mathcal{M}_{greedy}(Z))}{2}$. Thus we compute $Greedy$-$LB(M)= cost(M) + \frac{cost(\mathcal{M}_{greedy}(Z))}{2}$. Greedy-LB is considered as the trade-off between Naive-LB and KM-LB, because of its medium tightness and computation cost. Experiment in Section \ref{sec:evaluate_qga_bounds} confirms that Greedy-LB gains the best performance among them.

\begin{algorithm} [t]
	\caption{Greedy-LB Algorithm} \label{alg:greedy_lb}
	\KwIn{Unmatched edge set $Z$.}
	\KwOut{An $1/2$-approximate matching $\mathcal{M}_{greedy}$.}
	$\mathcal{M}_{greedy} := \phi$ \;
	\For{each $e \in Z$ in non-decreasing order of $w(e)$}
	{
		\If{$e$ do not share common vertices with $\mathcal{M}_{greedy}$}
		{
			$\mathcal{M}_{greedy} := \mathcal{M}_{greedy} \cup \{e\}$
		}
		\Return $\mathcal{M}_{greedy}$ \;
	}
\end{algorithm}

\vspace{-0.1in}
\subsection{Time Complexity Analysis} 
In a condensed bipartite graph $\mathbb{B}^{*}(V_{L}^*,V_{R}^*,E_{\mathbb{B}}^*)$, $|V_L^*| \le k^2 \cdot \left( \begin{array}{l}  n \\  2 \\  \end{array} \right)$, because there are $\left( \begin{array}{l}  n \\  2 \\  \end{array} \right)$ vertex pair set, each contains at most $k^2$ vertex pairs, and $V_R^*$ consists of $m$ condensed predicate nodes, i.e., $|V_R^*|=m$. Therefore, the total number of crossing edges is $|E_{\mathbb{B}}^*|=|V_R^*|\times|V_L^*| \le mk^2\left( \begin{array}{l}  n \\  2 \\  \end{array} \right) \le mn^2k^2$. Since a maximum matching in $\mathbb{B}^{*}$ should contain $m$ matching edges, we can find at most $\left( \begin{array}{l}  mn^2k^2 \\  \ \ \ m \\  \end{array} \right) \le (mn^2k^2)^m$ different maximum matchings. Since $n+m \le l$, by replacing $n$ and $m$ by $l$, we get the overall search space $O(k^{2l} \cdot l^{3l})$.


\subsection{Implicit Relation Prediction}
As we know, keywords are more flexible than NL question sentences. In some cases, users may omit some relation terms. For example, users may input ``scientist graduate from university USA'', where the keyword ``locate'' is omitted. In this case, for humans, it is trivial to infer that the user means ``an university located in USA''. Let us recall our QGA approach. If we omit ``locate'' in the running example, there is only one relation term. So, the query graph $Q$ cannot be connected if we use only one predicate edge to connect three vertices. We can patch our solution to connect $Q$ as follows.
%Suppose that the query graph $Q$ generated by QGA Algorithm has $r$ connected components $\mathcal{P}_1$, $\mathcal{P}_2$,...,$\mathcal{P}_r$. In order to connect them, we work as follows:

\begin{definition} \textbf{ (Relation Prediction Graph) }
	Suppose that $Q$ consists of $r$ connected components: $Q=\{\mathcal{P}_1, \mathcal{P}_2, ..., \mathcal{P}_r\}$. A relation prediction graph $P(V_P,E_P)$ is a complete graph and defined as follows:
	\begin{itemize}
		\item $V_P$ consists of $r$ vertices, where each vertex $v_P^i \in V_P$ corresponds to a connected component $\mathcal{P}_i$.
		\item $E_P$ consists of $\frac{r(r-1)}{2}$ labeled weighted edges. For any $(\mathcal{P}_i, \mathcal{P}_j)$, we find the minimum assembly cost triple $e(\langle v_i, v_j \rangle, p)$, where $v_i \in V(\mathcal{P}_i)$, $v_j \in V(\mathcal{P}_j)$, and $p \in \mathcal{U}$. Note that $V(\mathcal{P}_i)$ denotes all vertices in $P_i$, and $\mathcal{U}$ denotes all predicates in RDF graph $G$. Then we add a corresponding edge $e_P(v_P^i,v_P^j)$ into $E_P$, where the weight $w(e_P)=w(\langle v_i, v_j \rangle, p)$, and the label $l(e_P)=(\langle v_i, v_j \rangle, p)$.
	\end{itemize}
\end{definition}
Thus the relation prediction task is modeled as finding a minimum spanning tree $T$ on $P$, which can be solved in linear time. Through the label $(\langle v_i, v_j \rangle, p)$ of the tree edge in $T$, we can know that $v_i, v_j \in V_Q$ should be connected by the predicate edge $p$.



%!TEX root = main.tex
\section{Evaluation}
\label{sec:eval}

In this section, we evaluate the performance of our unsupervised Ordered Word Mover's Distance metric and supervised Multi-scale Sentence Matching model with factorized sentences as input. We apply our algorithms to semantic textual similarity estimation tasks and sentence pair paraphrase identification tasks, based on four datasets: STSbenchmark, SICK, MSRP and MSRvid. 

\subsection{Experimental Setup}
\label{subsec:setup}


\begin{table}[tb]
  \caption{Description of evaluation datasets.}
  \label{tab:datasets}
  \begin{tabular}{lllll}
    \toprule
    Dataset & Task & Train & Dev & Test\\
    \midrule
    STSbenchmark & Similarity scoring & $5748$ & $1500$ & $1378$ \\
    SICK & Similarity scoring & $4500$ & $500$ & $4927$ \\
    MSRP & Paraphrase identification & $4076$ & - & $1725$ \\
    MSRvid & Similarity scoring & $750$ & - & $750$ \\
    \bottomrule
  \end{tabular}
  \vspace{-2mm}
\end{table}

We will start with a brief description for each dataset:
\begin{itemize}
\item \textbf{STSbenchmark}\cite{cer2017semeval}: it is a dataset for semantic textual similarity (STS) estimation. The task is to assign a similarity score to each sentence pair on a scale of 0.0 to 5.0, with 5.0 being the most similar.

\item \textbf{SICK}\cite{marelli2014sick}: it is another STS dataset from the SemEval 2014 task 1. It has the same scoring mechanism as STSbenchmark, where 0.0 represents the least amount of relatedness and 5.0 represents the most.

\item \textbf{MSRvid}: the Microsoft Research Video Description Corpus contains 1500 sentences that are concise summaries on the content of a short video. Each pair of sentences is also assigned a semantic similarity score between 0.0 and 5.0. 

\item \textbf{MSRP}\cite{quirk2004monolingual}: the Microsoft Research Paraphrase Corpus is a set of 5800 sentence pairs collected from news articles on the Internet. Each sentence pair is labeled 0 or 1, with 1 indicating that the two sentences are paraphrases of each other.
\end{itemize}

Table \ref{tab:datasets} shows a detailed breakdown of the datasets used in evaluation.
For STSbenchmark dataset we use the provided train/dev/test split.
The SICK dataset does not provide development set out of the box, so we extracted 500 instances from the training set as the development set.
For MSRP and MSRvid, since their sizes are relatively small to begin with, we did not create any development set for them.

One metric we used to evaluate the performance of our proposed models on the task of semantic textual similarity estimation is the Pearson Correlation coefficient, commonly denoted by $r$. Pearson Correlation is defined as:
\begin{equation}
\label{eq:pearson}
 r = cov(X,Y) /( \sigma_X \sigma_Y),
\end{equation}
where $cov(X,Y)$ is the co-variance between distributions X and Y, and $\sigma_X$, $\sigma_Y$ are the standard deviations of X and Y.
The Pearson Correlation coefficient can be thought as a measure of how well two distributions fit on a straight line. Its value has range [-1, 1], where a value of 1 indicates that data points from two distribution lie on the same line with a positive slope.
% Due to this unique property, we believe the Pearson Correlation coefficient is a strong indicator of the performance of our metric. 

Another metric we utilized is the Spearman's Rank Correlation coefficient. Commonly denoted by $r_s$, the Spearman's Rank Correlation coefficient shares a similar mathematical expression with the Pearson Correlation coefficient, but it is applied to ranked variables.
Formally it is defined as \cite{wiki:spearman}:
\begin{equation}
\label{eq:spearman}
 \rho = cov(rg_X, rg_Y) / (\sigma_{rg_X} \sigma_{rg_Y}),
\end{equation}
where $rg_X$, $rg_Y$ denotes the ranked variables derived from $X$ and $Y$. $cov(rg_X,rg_Y)$, $\sigma_{rg_X}$, $\sigma_{rg_Y}$ corresponds to the co-variance and standard deviations of the rank variables. The term ranked simply means that each instance in X is ranked higher or lower against every other instances in X and the same for Y. We then compare the rank values of X and Y with \ref{eq:spearman}. Like the Pearson Correlation coefficient, the Spearman's Rank Correlation coefficient has an output range of [-1, 1], and it measures the monotonic relationship between X and Y. A Spearman's Rank Correlation value of 1 implies that as X increases, Y is guaranteed to increase as well.
The Spearman's Rank Correlation is also less sensitive to noise created by outliers compared to the Pearson Correlation.

For the task of paraphrase identification, the classification accuracy of label $1$ and the F1 score are used as metrics. 

In the supervised learning portion, we conduct the experiments on the aforementioned four datasets. We use training sets to train the models, development set to tune the hyper-parameters and each test set is only used once in the final evaluation. For datasets without any development set, we will use cross-validation in the training process to prevent overfitting, that is, use $10\%$ of the training data for validation and the rest is used in training. For each model, we carry out training for 10 epochs. We then choose the model with the best validation performance to be evaluated on the test set.  


\subsection{Unsupervised Matching with OWMD}
\label{subsec:eval-owmd}

To evaluate the effectiveness of our Ordered Word Mover's Distance metric, we first take an unsupervised approach towards the similarity estimation task on the STSbenchmark, SICK and MSRvid datasets. Using the distance metrics listed in Table \ref{tab:compare-pearson} and \ref{tab:compare-spearman}, we first computed the distance between two sentences, then calculated the Pearson Correlation coefficients and the Spearman's Rank Correlation coefficients between all pair's distances and their labeled scores. We did not use the MSRP dataset since it is a binary classification problem.


In our proposed Ordered Word Mover's Distance metric, distance between two sentences is calculated using the order preserving Word Mover's Distance algorithm. For all three datasets, we performed hyper-parameter tuning using the training set and calculated the Pearson Correlation coefficients on the test and development set. We found that for the STSbenchmark dataset, setting $\lambda_1=10$, $\lambda_2=0.03$ produces the most optimal result. For the SICK dataset, a combination of $\lambda_1=3.5$, $\lambda_2=0.015$ works best. And for the MSRvid dataset, the highest Pearson Correlation is attained when $\lambda_1=0.01$, $\lambda_2=0.02$.
We maintain a max iteration of 20 since in our experiments we found that it is sufficient for the correlation result to converge.
During hyper-parameter tuning we discovered that using the Euclidean metric along with $\sigma=10$ produces better results, so all OWMD results summarized in Table \ref{tab:compare-pearson} and \ref{tab:compare-spearman} are acquired under these parameter settings. Finally, it is worth mentioning that our OWMD metric calculates the distances using factorized versions of sentences, while all other metrics use the original sentences. Sentence factorization is a necessary preprocessing step for the OWMD metric.


We compared the performance of Ordered Word Mover's Distance metric with the following methods:

\begin{itemize}
\item \textbf{Bag-of-Words (BoW)}: in the Bag-of-Words metric, distance between two sentences is computed as the cosine similarity between the word counts of the sentences.

\item \textbf{LexVec}~\cite{salle2016enhancing}: calculate the cosine similarity between the  averaged 300-dimensional LexVec word embedding of the two sentences. 

\item \textbf{GloVe}~\cite{pennington2014glove}: calculate the cosine similarity between the averaged 300-dimensional GloVe 6B word embedding of the two sentences. 

\item \textbf{Fastext}~\cite{joulin2016bag}: calculate the cosine similarity between the averaged 300-dimensional Fastext word embedding of the two sentences. 

\item \textbf{Word2vec}~\cite{mikolov2013efficient}: calculate the cosine similarity between the averaged 300-dimensional Word2vec word embedding of the two sentences.

\item \textbf{Word Mover's Distance (WMD)}~\cite{kusner2015word}: estimating the semantic distance between two sentences by WMD introduced in Sec.~\ref{sec:owmd}.
\end{itemize} 


\begin{table}[tb]
  \caption{Pearson Correlation results on different distance metrics.}
  \label{tab:compare-pearson}
  \begin{tabular}{c|cc|cc|c}
    \toprule
    \multirow{2}{*}{Algorithm} & \multicolumn{2}{c}{STSbenchmark} & \multicolumn{2}{c}{SICK} & MSRvid\\ 
     & Test & Dev & Test & Dev & Test\\
    \midrule
    BoW & $0.5705$ & $0.6561$ & $0.6114$ & $0.6087$ & $0.5044$ \\
    LexVec & $0.5759$ & $0.6852$ & $0.6948$ & $\mathbf{0.6811}$ & $0.7318$\\
    GloVe & $0.4064$ & $0.5207$ & $0.6297$ & $0.5892$  & $0.5481$ \\
    Fastext & $0.5079$ & $0.6247$ & $0.6517$ & $0.6421$  & $0.5517$  \\
    Word2vec & $0.5550$ & $0.6911$ & $\mathbf{0.7021}$ & $0.6730$  & $0.7209$  \\
    WMD & $0.4241$ & $0.5679$ & $0.5962$ & $0.5953$  & $0.3430$  \\
    OWMD & $\mathbf{0.6144}$ & $\mathbf{0.7240}$ & $0.6797$ & $0.6772$  & $\mathbf{0.7519}$  \\
    \bottomrule
  \end{tabular}
  \vspace{-4mm}
\end{table}

\begin{table}[tb]
  \caption{Spearman's Rank Correlation results on different distance metrics.}
  \label{tab:compare-spearman}
  \begin{tabular}{c|cc|cc|c}
    \toprule
    \multirow{2}{*}{Algorithm} & \multicolumn{2}{c}{STSbenchmark} & \multicolumn{2}{c}{SICK} & MSRvid\\ 
     & Test & Dev & Test & Dev & Test\\
    \midrule
    BoW & $0.5592$ & $0.6572$ & $0.5727$ & $0.5894$ & $0.5233$ \\
    LexVec & $0.5472$ & $0.7032$ & $0.5872$ & $0.5879$ & $0.7311$\\
    GloVe & $0.4268$ & $0.5862$ & $0.5505$ & $0.5490$  & $0.5828$ \\
    Fastext & $0.4874$ & $0.6424$ & $0.5739$ & $0.5941$  & $0.5634$  \\
    Word2vec & $0.5184$ & $0.7021$ & $0.6082$ & $0.6056$  & $0.7175$  \\
    WMD & $0.4270$ & $0.5781$ & $0.5488$ & $0.5612$  & $0.3699$  \\
    OWMD & $\mathbf{0.5855}$ & $\mathbf{0.7253}$ & $\mathbf{0.6133}$ & $\mathbf{0.6188}$  & $\mathbf{0.7543}$  \\
    \bottomrule
  \end{tabular}
  \vspace{-2mm}
\end{table}


Table \ref{tab:compare-pearson} and Table \ref{tab:compare-spearman} compare the performance of different metrics in terms of the Pearson Correlation coefficients and the Spearman's Rank Correlation coefficients.
We can see that the result of our OWMD metric achieves the best performance on all the datasets in terms of the Spearman's Rank Correlation coefficients.
It also produced the best Pearson Correlation results on the STSbenchmark and the MSRvid dataset, while the performance on SICK datasets are close to the best.
This can be attributed to the two characteristics of OWMD. First, the input sentence is re-organized into a predicate-argument structure using the sentence factorization tree. Therefore, corresponding semantic units in the two sentences will be aligned roughly in order. Second, our OWMD metric takes word positions into consideration and penalizes disordered matches. Therefore, it will produce less mismatches compared with the WMD metric.

% On the SICK dataset, although the result of our metric falls slightly behind Word2vec, LexVec on the test set and Word2vec on the development set, we still believe that it is a superior metric because it produced competitive results across multiple datasets. 

% Table \ref{tab:compare-spearman} presents the Spearman's Rank Correlation coefficients acquired with the same distance metrics. We can observe that our OWMD metric achieves the highest correlation scores on all three datasets. Which proves once again that OWMD is a better distance metric for the task of semantic similarity detection.

\subsection{Supervised Multi-scale Semantic Matching}
\label{subsec:eval-multilayer}

\begin{table*}[tb]
  \caption{A comparison among different supervised learning models in terms of accuracy, F1 score, Pearson's $r$ and Spearman's $\rho$ on various test sets.}
  \label{tab:sts}
  \begin{tabular}{c|cc|cc|cc|cc}
    \toprule
    \multirow{2}{*}{Model} & \multicolumn{2}{c}{MSRP} & \multicolumn{2}{c}{SICK} & \multicolumn{2}{c}{MSRvid} & \multicolumn{2}{c}{STSbenchmark}\\ 
     & Acc.(\%) & F1(\%) & $r$ & $\rho$ & $r$ & $\rho$ & $r$ & $\rho$ \\
    \midrule
    MaLSTM & $66.95$ & $73.95$ & $0.7824$ & $0.71843$ & $0.7325$ & $0.7193$ & $0.5739$ & $0.5558$\\
    Multi-scale MaLSTM & $\mathbf{74.09}$ & $\mathbf{82.18}$ & $\mathbf{0.8168}$ & $\mathbf{0.74226}$ & $\mathbf{0.8236}$ & $\mathbf{0.8188}$ & $\mathbf{0.6839}$ & $\mathbf{0.6575}$\\
    \midrule
    HCTI & $73.80$ & $80.85$ & $0.8408$ & $0.7698$ & $\mathbf{0.8848}$ & $\mathbf{0.8763}$  & $\mathbf{0.7697}$ & $\mathbf{0.7549}$ \\
    Multi-scale HCTI & $\mathbf{74.03}$ & $\mathbf{81.76}$ & $\mathbf{0.8437}$ & $\mathbf{0.7729}$ & $0.8763$ & $0.8686$  & $0.7269$ & $0.7033$  \\
    \bottomrule
  \end{tabular}
  \vspace{-2mm}
\end{table*}

The use of sentence factorization can improve both existing unsupervised metrics and existing supervised models. 
% We extend the normal Siamese model to Fig. \ref{fig:network} to take advantage of different level of information in the factorized sentence. 
To evaluate how the performance of existing Siamese neural networks can be improved by our sentence factorization technique and the multi-scale Siamese architecture, we implemented two types of Siamese sentence matching models, HCTI \cite{mueller2016siamese} and MaLSTM \cite{shao2017hcti}. HCTI is a Convolutional Neural Network (CNN) based Siamese model, which achieves the best Pearson Correlation coefficient on STSbenchmark dataset in SemEval2017 competition (compared with all the other neural network models). MaLSTM is a Siamese adaptation of the Long Short-Term Memory (LSTM) network for learning sentence similarity. As the source code of HCTI is not released in public, we implemented it according to \cite{shao2017hcti} by Keras \cite{chollet2015keras}. With the same parameter settings listed in paper \cite{shao2017hcti} and tried our best to optimize the model, we got a Pearson correlation of 0.7697 (0.7833 in paper \cite{shao2017hcti}) in STSbencmark test dataset.

We extended HCTI and MaLSTM to our proposed Siamese architecture in Fig. \ref{fig:network}, namely the Multi-scale MaLSTM and the Multi-scale HCTI. To evaluate the performance of our models, the experiment is conducted on two tasks: 1) semantic textual similarity estimation based on the STSbenchmark, MSRvid, and SICK2014 datasets; 2) paraphrase identification based on the MSRP dataset.

Table \ref{tab:sts} shows the results of HCTI, MaLSTM and our multi-scale models on different datasets. Compared with the original models, our models with multi-scale semantic units of the input sentences as network inputs significantly improved the performance on most datasets. 
Furthermore, the improvements on different tasks and datasets also proved the general applicability of our proposed architecture.

Compared with MaLSTM, our multi-scaled Siamese models with factorized sentences as input perform much better on each dataset. For MSRvid and STSbenmark dataset, both Pearson's $r$ and Spearman's $\rho$ increase about $10\%$ with Multi-scale MaLSTM. Moreover, the Multi-scale MaLSTM achieves the highest accuracy and F1 score on the MSRP dataset compared with other models listed in Table \ref{tab:sts}.

There are two reasons why our Multi-scale MaLSTM significantly outperforms MaLSTM model. First, for an input sentence pair, 
we explicitly model their semantic units with the factorization algorithm.
%we explicitly model the different scales of semantics of them with the semantic units produced by our sentence factorization algorithm. 
Second, our multi-scaled network architecture is 
specifically designed
%specially adapted to 
for multi-scaled sentences representations. Therefore, it is able to explicitly match a pair of sentences at different granularities.

We also report the results of HCTI and Multi-scale HCTI in Table \ref{tab:sts}. For the paraphrase identification task, our model shows better accuracy and F1 score on MSRP dataset. For the semantic textual similarity estimation task, the performance varies across datasets. On the SICK dataset, the performance of Multi-scale HCTI is close to HCTI with slightly better Pearson' $r$ and Spearman's $\rho$. However, the Multi-scale HCTI is not able to outperform HCTI on MSRvid and STSbenchmark. HCTI is still the best neural network model on the STSbenchmark dataset, and the MSRvid dataset is a subset of STSbenchmark.
Although HCTI has strong performance on these two datasets, it performs worse than our model on other datasets.
% Overall, the experimental results demonstrated the superior applicability and generalizability of our proposed models.
Overall, the experimental results demonstrated the general applicability of our proposed model architecture, which performs well on various semantic matching tasks.

% \begin{table}[tb]
%   \caption{Results of Accuracy and F1 score on MSRP test dataset.}
%   \label{tab:MSRP result}
%   \begin{tabular}{lllll}
%     \toprule
%     Model & Acc.(\%) & F1(\%)  \\
%     \midrule
%     MaLSTM & $66.95$ & $73.95$ \\
%     Factorized MaLSTM & $\mathbf{74.09}$ & $\mathbf{82.18}$ \\
%     HCTI & $73.80$ & $80.85$ \\
%     Factorized HCTI & $\mathbf{74.03}$ & $\mathbf{81.76}$ \\
%     \bottomrule
%   \end{tabular}
%   \vspace{0mm}
% \end{table}


% \begin{table}[tb]
%   \caption{Results of Pearson's $r$ and Spearman's $\rho$ on SICK test dataset.}
%   \label{tab:SICK result}
%   \begin{tabular}{lllll}
%     \toprule
%     Model & r & $\rho$ \\
%     \midrule
%     MaLSTM & $0.7824$ & $0.71843$ \\
%     Factorized MaLSTM & $\mathbf{0.8168}$ & $\mathbf{0.74226}$ \\
%     HCTI & $0.8408$ & $\mathbf{0.7698}$ \\
%     Factorized HCTI & $\mathbf{0.8429}$ & $0.7676$ \\
%     \bottomrule
%   \end{tabular}
%   \vspace{0mm}
% \end{table}


% \begin{table}[tb]
%   \caption{Results of Pearson's $r$ and Spearman's $\rho$ on MSRvid test dataset.}
%   \label{tab:MSRvid result}
%   \begin{tabular}{lll}
%     \toprule
%     Model & r & $\rho$  \\
%     \midrule
%     MaLSTM & $0.7325$ & $0.7193$ \\
%     Factorized MaLSTM & $\mathbf{0.8236}$ & $\mathbf{0.8188}$ \\
%     HCTI & $\mathbf{0.8848}$ & $\mathbf{0.8763}$ \\
%     Factorized HCTI & $0.8763$ & $0.8686$ \\
%     \bottomrule
%   \end{tabular}
%   \vspace{0mm}
% \end{table}



% \begin{table}[tb]
%   \caption{Results of Pearson's $r$ and Spearman's $\rho$ on STSbenchmark test dataset.}
%   \label{tab:STSbenchmark result}
%   \begin{tabular}{lllll}
%     \toprule
%     Model & r & $\rho$ \\
%     \midrule
%     MaLSTM & $0.5739$ & $0.5558$ \\
%     Factorized MaLSTM & $\mathbf{0.6839}$ & $\mathbf{0.6575}$ \\
%     HCTI & $\mathbf{0.7697}$ & $\mathbf{0.7549}$ \\
%     Factorized HCTI & $0.7269$ & $0.7033$ \\
%     \bottomrule
%   \end{tabular}
%   \vspace{0mm}
% \end{table}





\begin{comment}
\begin{figure}
\includegraphics[width=\linewidth]{figs/beyond_tss_lesion.pdf}
\caption[]{End-to-End runtime lesion study of the entire MNIST dataset and the FMA featurized music dataset. Each of DROP's contributions provides a runtime improvement.}
\label{fig:beyond_lesion}
\end{figure}
\end{comment}



\section{Conclusion}
\label{sec:conclusion}

Advanced data analytics techniques must scale to rising data volumes. 
DR techniques offer a powerful toolkit when processing these datasets, with PCA frequently outperforming popular techniques in exchange for high computational cost. 
In response, we propose DROP, a new dimensionality reduction optimizer. 
DROP combines progressive sampling, progress estimation, and online aggregation to identify high quality low dimensional bases via PCA without processing the entire dataset by balancing the runtime of downstream tasks and achieved dimensionality. 
Thus, DROP provides a first step in bridging the gap between quality and efficiency in end-to-end DR for downstream \red{analytics}. 

%We revisit canonical operators for time series dimensionality reduction and the measurement study of~\cite{keogh-study}, and show that PCA is more effective than popular alternatives in the data mining literature often by a margin of over $2\times$ on average on gold-standard time series benchmark data sets with respect to output data dimension. More surprisingly, we empirically demonstrate that a small number of samples are sufficient to accurately characterize directions of maximum variance and obtain a high-quality low-dimensional transformation.




\vspace{-0.05in}
\begin{acks}
Lei Zou was supported by the National Key Research and Development Program of China (2016YFB1000603)  and NSFC (No. 61622201 and 61532010).
Jeffery Xu Yu was supported by the Research Grant Council of Hong Kong SAR, China (No. 14221716).
Lei Zou is the corresponding author of this work.
\end{acks}
\vspace{-0.05in}
\bibliographystyle{ACM-Reference-Format}
\bibliography{sigproc} 

\end{document}
