\section{Problem Definition}\label{sec:problemdef}

\begin{figure} [t]
	\centering
	\scalebox{1.05}[1.15]
	{
		\resizebox{\linewidth}{!}
		{
			\includegraphics[scale=1.0]{visio_pics/rdf_graph_new.pdf}
		}
	}
	\vspace{-0.1in}
	\caption{A Sample of DBpedia RDF Graph.}
	\label{fig:rdfgraph}
	\vspace{-0.2in}
\end{figure}

In this section, we define our problem and review the terminologies used throughout this paper. As a de facto standard model of knowledge base, RDF represents the assertions by $\langle$subject, predicate, object$\rangle$ triples. An RDF dataset can be represented as a graph naturally, where subjects and objects are vertices and predicates denote directed edges between them. A running example of RDF graph is illustrated in Figure \ref{fig:rdfgraph}. Formally, we have the definition about RDF graph as follows.


%\vspace{-0.1in}
\begin{definition} \label{def:rdfgraph}
	\textbf{ (RDF Graph) }
	An \textit{RDF graph} is denoted as $G(V, E)$, where
	$V$ is the set of entity and class vertices corresponding to subjects and objects of RDF triples,
	and $E$ is the set of directed relation edges with their labels corresponding to predicates of RDF triples.

\end{definition}

Note that RDF triple's object may be literal value, for example, $\langle$res:Alan\_Turing, dbo:deathDate, ``1954-06-07'' $\textasciicircum{}\textasciicircum{}$xms:date$\rangle$. We treat all literal values as entity vertices in RDF graph, and literal types (e.g. xms:date as for ``1954-06-07'') as class vertices. 

SPARQL is the standard structural query language of RDF, which can also be represented as a \emph{query graph} defined as follows.
%Answering SPARQLs is equivalent to evaluate the query by subgraph matching using homomorphism \cite{zou2014gstore}. 

\begin{definition} \textbf{ (Query Graph) }
	A \textit{query graph} is denoted as $Q(V_Q, E_Q)$, where
	$V_Q$ consists of entity vertices, class vertices, and vertex \emph{variables},
	and $E_Q$ consists of relation edges as well as edge \emph{variables}.
\end{definition}

%Although SPARQL provides a systematic approach to accessing RDF graph, the complexity of the SPARQL syntax makes it hard to use. 
In this paper, we study the keyword search problem over RDF graph. Given a keyword token sequence $RQ = \{k_1, k_2, ..., k_m\}$, our problem is to interpret $RQ$ as a \emph{query graph} $Q$. 
