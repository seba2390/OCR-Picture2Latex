\begin{figure}[!htbp]
    \centering
    \small 
    \begin{subfigure}[t]{0.3\linewidth}
        \includegraphics[width=\linewidth]{CrossGatys/AddCrossGatys_content122_style_style48.png}
        \includegraphics[width=\linewidth]{MulCG/MulCrossGatys_content122_style48.png}
    \end{subfigure}
    \begin{subfigure}[t]{0.3\linewidth}
        \includegraphics[width=\linewidth]{CrossGatys/AddCrossGatys_content122_style_style6.png}
        \includegraphics[width=\linewidth]{MulCG/MulCrossGatys_content122_style6.png}
    \end{subfigure}
    \begin{subfigure}[t]{0.3\linewidth}
        \includegraphics[width=\linewidth]{CrossGatys/AddCrossGatys_content122_style_style117.png} 
        \includegraphics[width=\linewidth]{MulCG/MulCrossGatys_content122_style117.png}
    \end{subfigure}
    
    \caption{\em 
     Multiplicative loss produces good style transfer results.
    {\bf Top row:} style transfers using cross layer gram matrices and additive loss, with a good choice of $\alpha$. {\bf
      Bottom row:} style transfers using cross layer gram matrices and multiplicative loss, where no choice of $\alpha$ is required. Notice the emphasis of  content outline in the multiplicative loss images. 
    }
      \label{fig:MulCG}
    \end{figure}

\begin{figure}[!htbp]
    \centering
    \small 
    \begin{subfigure}[t]{0.3\linewidth}
        \includegraphics[width=\linewidth]{details/MulCrossGatyscontent122style86.jpg}
        \includegraphics[width=\linewidth]{details/MulALLCrossGatys600content122style86.jpg}
    \end{subfigure}
    \begin{subfigure}[t]{0.3\linewidth}
        \includegraphics[width=\linewidth]{details/MulCrossGatyscontent122style6.jpg}
        \includegraphics[width=\linewidth]{details/MulALLCrossGatys600content122style6.jpg}
    \end{subfigure}
    \begin{subfigure}[t]{0.3\linewidth}
        \includegraphics[width=\linewidth]{details/MulCrossGatyscontent122style117.jpg}
        \includegraphics[width=\linewidth]{details/MulALLCrossGatys600content122style117.jpg}
    \end{subfigure}
    \caption{\em All pairs distinct cross-layer style transfer yields somewhat better results than descending pairs.  {\bf
        Top row:} cross-layer style transfer using descending pairs (i.e. (R51, R41); (R41, R31); (R31, R21); (R21, R11)).
      {\bf Bottom row:}  cross-layer style transfer using all pairs distinct (i.e  all distinct pairs from R51...R11).
    There are fewer bubbles; color localization and value is improved; and line breaks are fewer.
      \label{fig:CGALL}}
    \end{figure}

\begin{figure}[!htbp]
    \centering
    \begin{subfigure}[b]{0.3\linewidth}
      \includegraphics[width=\linewidth, height=\linewidth]{./CrossGatys/style_84.png}
      %\caption{Style.}
    \end{subfigure}
    \begin{subfigure}[b]{0.3\linewidth}
      \includegraphics[width=\linewidth]{Gatys/GatysBased_content122_style84.png}
      %\caption{ours.}
    \end{subfigure}
    \begin{subfigure}[b]{0.3\linewidth}
      \includegraphics[width=\linewidth]{CrossGatys/AddCrossGatys_content122_style_style84.png}
      %\caption{WCT.}
    \end{subfigure}
  
  
    \begin{subfigure}[b]{0.3\linewidth}
      \includegraphics[width=\linewidth, height=\linewidth]{CrossGatys/style_86.png}
    \end{subfigure}
    \begin{subfigure}[b]{0.3\linewidth}
      \includegraphics[width=\linewidth]{Gatys/GatysBased_content122_style86.png}
    \end{subfigure}
    \begin{subfigure}[b]{0.3\linewidth}
      \includegraphics[width=\linewidth]{CrossGatys/AddCrossGatys_content122_style_style86.png}
    \end{subfigure}
    
    
    \begin{subfigure}[b]{0.3\linewidth}
      \includegraphics[width=\linewidth, height=\linewidth]{oldtable/style79.png}
    \end{subfigure}
    \begin{subfigure}[b]{0.3\linewidth}
      \includegraphics[width=\linewidth]{oldtable/Gatys_style79.png}
    \end{subfigure}
    \begin{subfigure}[b]{0.3\linewidth}
      \includegraphics[width=\linewidth]{oldtable/AddCross_style79.png}
    \end{subfigure}  
    
    \begin{subfigure}[b]{0.3\linewidth}
      \includegraphics[width=\linewidth, height=\linewidth]{oldtable/style8.png}
  \caption{Style}
    \end{subfigure}
    \begin{subfigure}[b]{0.3\linewidth}
      \includegraphics[width=\linewidth]{oldtable/Gatys_style8.png}
  \caption{Within-Layer}
    \end{subfigure}
    \begin{subfigure}[b]{0.3\linewidth}
      \includegraphics[width=\linewidth]{oldtable/AddCross_style8.png}
  \caption{Cross-Layer}
    \end{subfigure} 
      \caption{\em {\bf Left:} styles to transfer; {\bf center:} results using within-layer
      loss; {\bf right:} results using cross-layer loss.  There are
      visible advantages to using the cross-layer loss. Note how cross-layer preserves the shape of the abstract color blocks (top row); 
  avoids smearing large paint strokes (second row); preserves the overall structure of the curves as much as possible
  (third row); and produces color blocks with thin boundaries (fourth row).
    }\label{fig:cf2}
  \end{figure}
  
  \begin{figure}[h]
  \centering
  \includegraphics[clip, trim=3.5cm 9.4cm 8cm 4.5cm, width=\linewidth]{PlotsNCharts/Layer-wise_compare_withinGramNcrossGram.pdf}
  
  
  \caption{\em This figure shows what happens when one controls only one (or one pair) of layers with the style loss.
  {\bf Left:} controlling a single layer, with a within-layer gram matrix.  {\bf Center:} controlling two
  layers in sequence, but each with a within-layer gram matrix.  {\bf Right:} controlling a two layers
  in sequence, but using only a cross-layer gram matrix.  Notice that, as one would expect, controlling
  cross-layer gram matrices results in more pronounced effects and a wider range of spatial scales of effect.
  Furthermore, in comparison to controlling  a pair of within-layer gram matrices, one is controlling fewer
  parameters.}
  \label{fig:layer_wise}
  \end{figure}
  
  
  \begin{figure}[!htbp]
    \centering
    \begin{subfigure}[b]{0.3\linewidth}
      \includegraphics[width=\linewidth, height=\linewidth]{./CrossGatys/style_75.png}
      %\caption{Style.}
    \end{subfigure}
    \begin{subfigure}[b]{0.3\linewidth}
      \includegraphics[width=\linewidth]{Gatys/GatysBased_content122_style75.png}
      %\caption{ours.}
    \end{subfigure}
    \begin{subfigure}[b]{0.3\linewidth}
      \includegraphics[width=\linewidth]{CrossGatys/AddCrossGatys_content122_style_style75.png}
      %\caption{WCT.}
    \end{subfigure}
  
    \begin{subfigure}[b]{0.3\linewidth}
      \includegraphics[width=\linewidth, height=\linewidth]{CrossGatys/style_48.png}
      %\caption{Style.}
    \end{subfigure}
    \begin{subfigure}[b]{0.3\linewidth}
      \includegraphics[width=\linewidth]{Gatys/GatysBased_content122_style48.png}
      %\caption{ours.}
    \end{subfigure}
    \begin{subfigure}[b]{0.3\linewidth}
      \includegraphics[width=\linewidth]{CrossGatys/AddCrossGatys_content122_style_style48.png}
      %\caption{WCT.}
    \end{subfigure}
  
    \begin{subfigure}[b]{0.3\linewidth}
      \includegraphics[width=\linewidth, height=\linewidth]{CrossGatys/style_6.png}
    \end{subfigure}
    \begin{subfigure}[b]{0.3\linewidth}
      \includegraphics[width=\linewidth]{Gatys/GatysBased_content122_style6.png}
    \end{subfigure}
    \begin{subfigure}[b]{0.3\linewidth}
      \includegraphics[width=\linewidth]{CrossGatys/AddCrossGatys_content122_style_style6.png}
    \end{subfigure}
    \begin{subfigure}[b]{0.3\linewidth}
      \includegraphics[width=\linewidth,
      height=\linewidth]{CrossGatys/style_117.png}
  \caption{Style}
    \end{subfigure}
    \begin{subfigure}[b]{0.3\linewidth}
      \includegraphics[width=\linewidth]{Gatys/GatysBased_content122_style117.png}
  \caption{Within-Layer}
    \end{subfigure}
    \begin{subfigure}[b]{0.3\linewidth}
      \includegraphics[width=\linewidth]{CrossGatys/AddCrossGatys_content122_style_style117.png}
  \caption{Cross-Layer}
    \end{subfigure}  
    \caption{\em {\bf Left:} styles to transfer; {\bf center:} results using within-layer
      loss; {\bf right} results using cross-layer loss.  There are
      visible advantages to using the cross-layer loss. Note how cross-layer preserves large black areas (top row); 
  creates an improved appearance of relief for the acrylic strokes (second row); preserves the overall structure of the
  rods (third row); and ensures each string has a dot on each end (fourth row).
    }\label{fig:cf1}
  \end{figure}
  \begin{figure*}[!htbp]
  \centering
  \small 
  \begin{subfigure}[t]{0.15\textwidth}
      
      \includegraphics[width=\linewidth]{Texture/Page-8-Image-74.png}
  
      \includegraphics[width=\linewidth]{Texture/Page-8-Image-80.png}
  
      \includegraphics[width=\linewidth]{Texture/Page-8-Image-90.png}
      \includegraphics[width=\linewidth]{Texture/Page-8-Image-92.png}
      \includegraphics[width=\linewidth]{Texture/Page-8-Image-66.png}
  
      \caption{Styles}
  \end{subfigure}
  \begin{subfigure}[t]{0.15\textwidth}
      
      \includegraphics[width=\linewidth]{Texture_Gatys/TextureGatysBased_content171_style166.png}
      \includegraphics[width=\linewidth]{Texture_Gatys/TextureGatysBased_content171_style169.png}
      \includegraphics[width=\linewidth]{Texture_Gatys/TextureGatysBased_content171_style174.png}
      \includegraphics[width=\linewidth]{Texture_Gatys/TextureGatysBased_content171_style175.png}  
      \includegraphics[width=\linewidth]{Texture_Gatys/TextureGatysBased_content171_style162.png}
  
      \caption{Within-layers}
  \end{subfigure}
  
  \begin{subfigure}[t]{0.15\textwidth}
      
      \includegraphics[width=\linewidth]{Texture_ours/ShiftTextureAddCrossGatys_content171_style166_iteration600.png}
      \includegraphics[width=\linewidth]{Texture_ours/ShiftTextureAddCrossGatys_content171_style169_iteration600.png}
      \includegraphics[width=\linewidth]{Texture_ours/ShiftTextureAddCrossGatys_content171_style174_iteration600.png}
      \includegraphics[width=\linewidth]{Texture_ours/ShiftTextureAddCrossGatys_content171_style175_iteration600.png}
      \includegraphics[width=\linewidth]{Texture_ours/ShiftTextureAddCrossGatys_content171_style162_iteration600.png}
      \caption{CG}
  \end{subfigure}
  \begin{subfigure}[t]{0.15\textwidth}
  
      \includegraphics[width=\linewidth]{Texture_ours/ShiftTextureMulALLCrossGatys600_content171_style166.png}
      \includegraphics[width=\linewidth]{Texture_ours/ShiftTextureMulALLCrossGatys600_content171_style169.png}
      \includegraphics[width=\linewidth]{Texture_ours/ShiftTextureMulALLCrossGatys600_content171_style174.png}
      \includegraphics[width=\linewidth]{Texture_ours/ShiftTextureMulALLCrossGatys600_content171_style175.png}
      \includegraphics[width=\linewidth]{Texture_ours/ShiftTextureMulALLCrossGatys600_content171_style162.png}
  
      \caption{more CGs}
  \end{subfigure}
  
  \begin{subfigure}[t]{0.153\textwidth}
      
      \includegraphics[width=\linewidth]{Texture_WCT_WCT/styles166_Texturelayer1.jpg}
  
      \includegraphics[width=\linewidth]{Texture_WCT_WCT/styles169_Texturelayer1.jpg}
      \includegraphics[width=\linewidth]{Texture_WCT_WCT/styles174_Texturelayer1.jpg}
      \includegraphics[width=\linewidth]{Texture_WCT_WCT/styles175_Texturelayer1.jpg}
      \includegraphics[width=\linewidth]{Texture_WCT_WCT/styles162_Texturelayer1.jpg}
      \caption{WCT}
  \end{subfigure}
  \begin{subfigure}[t]{0.153\textwidth}
      
      \includegraphics[width=\linewidth]{Texture_WCT_ours/styles166_Texturelayer1.jpg}
      \includegraphics[width=\linewidth]{Texture_WCT_ours/styles169_Texturelayer1.jpg}
      \includegraphics[width=\linewidth]{Texture_WCT_ours/styles174_Texturelayer1.jpg}
      \includegraphics[width=\linewidth]{Texture_WCT_ours/styles175_Texturelayer1.jpg}
      \includegraphics[width=\linewidth]{Texture_WCT_ours/styles162_Texturelayer1.jpg}
      \caption{FCT}
  \end{subfigure}
  \caption{Texture synthesis comparison: Except the first column as
    style, the rest of columns from left to right are respectively generated by
    within-layer gram matrix, CG (cross-layer gram matrices), more CG (all cross-layer gram matrices between R51,R4,R31,R21,R11 are considered), WCT, and FCT. We can see that
    either in Gatys vs ours or WCT vs FCT, the cross-layer gram
    matrix indeed shows the improvement on texture patterns. }
  \label{fig:texture}
  \end{figure*}

  \begin{figure}[h]
    \centering
    \begin{subfigure}[b]{\linewidth}
      \includegraphics[width=\linewidth]{WCT/style_catAl_cat12_WCT_in2_styles53.png}
       \caption{\em Our method shows better color grouping in the stylized image.}
    \end{subfigure}
    \begin{subfigure}[b]{\linewidth}
      \includegraphics[width=\linewidth]{WCT/style_catAl_cat12_WCT_in2_styles83.png}
      \caption{\em  Many black spots in original WCT, which is not observed in our method.}
    \end{subfigure}
    \begin{subfigure}[b]{\linewidth}
      \includegraphics[width=\linewidth]{WCT/style_catAl_cat12_WCT_in2_styles95.png}
      \caption{\em  Ours improved the color contrast, because cross-layer gram matrices preserve longer scale color pattern.}
    \end{subfigure}
    \begin{subfigure}[b]{\linewidth}
      \includegraphics[width=\linewidth]{WCT/style_catAl_cat12_WCT_in2_styles109.png}
      \caption{\em Note our method does not have the blue color shift present in WCT.}
    \end{subfigure}
    \begin{subfigure}[b]{\linewidth}
      \includegraphics[width=\linewidth]{WCT/style_catAl_cat12_WCT_in2_styles118.png}
      \caption{\em WCT has many artificial pattern which is not seen in original style image, and ours largely reduce it.}
    \end{subfigure}
    \begin{subfigure}[b]{\linewidth}
      \includegraphics[width=\linewidth]{WCT/style_catAl_cat12_WCT_in2_styles7.png}
      \caption{\em Color blocks are better organized in ours.}
    \end{subfigure}
    \caption{\em In each row, {\bf first:} the style image; {\bf second:} transfer using FCT with descending sequences
      (i.e. (R51, R41, R31, R21, R11); (R41, R31, R21, R11); (R31, R21, R11); etc); {\bf third:} transfer using FCT with pairwise descending sequences (i.e. (R51, R41); (R41, R31); (R31, R21); and (R21, R11)); {\bf fourth} transfer using
      WCT \protect \cite{UST}
    \label{fig:WCT1}}
  \end{figure}
  
  
  
  % {\bf Scales:}  A crop of the style image will effectively result in transferring larger style elements.  We expect that,
  % when style elements are large compared to the content, cross-layer methods will have a strong advantage because they
  % will be better able to preserve structural relations that make up style elements.   Qualitative evidence supports this
  % view (Figure~\ref{fig:scale2} and Figure~\ref{fig:scale3}).
  
  \begin{figure}[h!]
    \centering %Hard_to_easy/styles_-_101.jpg
      \begin{subfigure}[b]{0.157\linewidth}
      \includegraphics[width=\linewidth]{Hard_to_easy/styles_-_101.jpg}
      \caption{Style}
    \end{subfigure}
    \begin{subfigure}[b]{0.27\linewidth}
      \includegraphics[width=\linewidth, height=\linewidth]{Hard_to_easy/MulCrossGatys_content122_style632_imagesize768.png}
      \includegraphics[width=\linewidth]{Hard_to_easy/MulGatys_content122_style632_imagesize768.png}
      \caption{Style size 768}
    \end{subfigure}
    \begin{subfigure}[b]{0.27\linewidth}
      \includegraphics[width=\linewidth]{Hard_to_easy/MulCrossGatys_content122_style632_imagesize512.png}
      \includegraphics[width=\linewidth, height=\linewidth]{Hard_to_easy/MulGatys_content122_style632_imagesize512.png}
      \caption{Style size 512}
    \end{subfigure} 
    \begin{subfigure}[b]{0.27\linewidth}
      \includegraphics[width=\linewidth]{Hard_to_easy/MulCrossGatys_content122_style632_imagesize256.png}
      \includegraphics[width=\linewidth]{Hard_to_easy/MulGatys_content122_style632_imagesize256.png}
      \caption{Style size 256}
    \end{subfigure}
  
    \caption{\em  Each row of stylized images shows a  transfer with the same style, but where the style image has been cropped to
      different sizes (style elements are {\em large} (=edge length 768), {\em medium} (=edge length 512) and {\em small}
      (=edge length 256), reading left to right).  The first row shows cross layer loss, the second row within layer loss.
      Note that, when style elements are large, the cross-layer loss is better at preserving their structure (e.g.,
    the large circles have fewer wiggles, etc.). Loss is multiplicative, notice the emphasis on outlines from multiplicative loss.
    }
    \label{fig:scale2}
  \end{figure}
  
  
  \begin{figure}[h!]
    \centering
  
    \begin{subfigure}[b]{0.3\linewidth}
      \includegraphics[width=\linewidth]{Hard_to_easy/MulCrossGatys_content122_style601_imagesize768.png}
    \end{subfigure}
    \begin{subfigure}[b]{0.3\linewidth}
      \includegraphics[width=\linewidth]{Hard_to_easy/MulCrossGatys_content122_style601_imagesize512.png}
    \end{subfigure}
    \begin{subfigure}[b]{0.3\linewidth}
      \includegraphics[width=\linewidth]{Hard_to_easy/MulCrossGatys_content122_style601_imagesize256.png}
    \end{subfigure}
  
    \begin{subfigure}[b]{0.3\linewidth}
      \includegraphics[width=\linewidth]{Hard_to_easy/MulGatys_content122_style601_imagesize768.png}
      \caption{Style size 768}
    \end{subfigure}
    \begin{subfigure}[b]{0.3\linewidth}
      \includegraphics[width=\linewidth]{Hard_to_easy/MulGatys_content122_style601_imagesize512.png}
      \caption{Style size 512}
    \end{subfigure}
    \begin{subfigure}[b]{0.3\linewidth}
      \includegraphics[width=\linewidth]{Hard_to_easy/MulGatys_content122_style601_imagesize256.png}
      \caption{Style size 256}
    \end{subfigure}
  \caption{\em  Each row shows a style transfer with the same style, but where the style image has been cropped to
      different sizes (style elements are {\em large} (=edge length 768), {\em medium} (=edge length 512) and {\em small}
      (=edge length 256), reading left to right).  The first row shows cross layer loss, the second row within layer loss.
      Note that, when style elements are large, the cross-layer loss is better at preserving their structure (e.g., the long scale color coherence is preserved, and
    the large paint strokes have more detail and more relief etc.)  Loss is multiplicative, notice the emphasis on outlines from multiplicative loss.
   }
    \label{fig:scale3}
  \end{figure}
  
  
  
  
  
  \begin{figure*}[!htbp]
  \centering
  \small 
  
      \includegraphics[width=0.24\linewidth]{good/MulCrossGatys_content31_style10_weight0_000000_loss0_iteration400.png}
      \includegraphics[width=0.24\linewidth]{good/MulCrossGatys_content31_style151_weight0_000000_loss0_iteration400.png}
      \includegraphics[width=0.24\linewidth]{good/MulCrossGatys_content31_style161_weight0_000000_loss0_iteration400.png}
      \includegraphics[width=0.24\linewidth]{good/MulCrossGatys_content31_style97_weight0_000000_loss0_iteration400.png}
      
      \includegraphics[width=0.24\linewidth]{good/MulCrossGatys_content31_style40_weight0_000000_loss0_iteration400.png}
      \includegraphics[width=0.24\linewidth]{good/MulCrossGatys_content31_style48_weight0_000000_loss0_iteration400.png}
      \includegraphics[width=0.24\linewidth]{good/MulCrossGatys_content31_style67_weight0_000000_loss0_iteration400.png}
      \includegraphics[width=0.24\linewidth]{good/MulCrossGatys_content31_style95_weight0_000000_loss0_iteration400.png}
  \caption{good horse.  }
  \label{fig:good1}
  \end{figure*}
  
  \begin{figure*}[!htbp]
  \centering
  \small 
  
      \includegraphics[width=0.24\linewidth]{good/MulCrossGatys_content33_style10_weight0_000000_loss1_iteration400.png}
      \includegraphics[width=0.24\linewidth]{good/MulCrossGatys_content33_style151_weight0_000000_loss0_iteration400.png}
      \includegraphics[width=0.24\linewidth]{good/MulCrossGatys_content33_style161_weight0_000000_loss0_iteration400.png}
      \includegraphics[width=0.24\linewidth]{good/MulCrossGatys_content33_style40_weight0_000000_loss0_iteration400.png}
      
      \includegraphics[width=0.24\linewidth]{good/MulCrossGatys_content33_style48_weight0_000000_loss0_iteration400.png}
      \includegraphics[width=0.24\linewidth]{good/MulCrossGatys_content33_style62_weight0_000000_loss0_iteration400.png}
      \includegraphics[width=0.24\linewidth]{good/MulCrossGatys_content33_style86_weight0_000000_loss0_iteration400.png}
      \includegraphics[width=0.24\linewidth]{good/MulCrossGatys_content33_style67_weight0_000000_loss0_iteration400.png}
      
      \includegraphics[width=0.24\linewidth]{good/MulCrossGatys_content33_style6_weight0_000000_loss1_iteration400.png}
      \includegraphics[width=0.24\linewidth]{good/MulCrossGatys_content33_style89_weight0_000000_loss0_iteration400.png}
      \includegraphics[width=0.24\linewidth]{good/MulCrossGatys_content33_style95_weight0_000000_loss0_iteration400.png}
      \includegraphics[width=0.24\linewidth]{good/MulCrossGatysLargerWeightHigherlayer_content33_style85_weight0_000000_loss29926_iteration400.png}
      
          \includegraphics[width=0.24\linewidth]{good/MulCrossGatysLargerWeightHigherlayer_content33_style100_weight0_000000_loss19126_iteration400.png}
      \includegraphics[width=0.24\linewidth]{good/MulCrossGatysLargerWeightHigherlayer_content33_style128_weight0_000000_loss33741_iteration400.png}
      \includegraphics[width=0.24\linewidth]{good/MulCrossGatysLargerWeightHigherlayer_content33_style59_weight0_000000_loss36060_iteration400.png}
      \includegraphics[width=0.24\linewidth]{good/MulCrossGatysLargerWeightHigherlayer_content33_style88_weight0_000000_loss181419_iteration400.png}
  \caption{good sea food.  }
  \label{fig:good1}
  \end{figure*}
  
  \begin{figure*}[!htbp]
  \centering
  \small 
  
      \includegraphics[width=0.24\linewidth]{good/MulCrossGatysLargerWeightHigherlayer_content36_style10_weight0_000000_loss34809_iteration400.png}
      \includegraphics[width=0.24\linewidth]{good/MulCrossGatysLargerWeightHigherlayer_content36_style151_weight0_000000_loss2128_iteration400.png}
      \includegraphics[width=0.24\linewidth]{good/MulCrossGatysLargerWeightHigherlayer_content36_style161_weight0_000000_loss2255_iteration400.png}
      \includegraphics[width=0.24\linewidth]{good/MulCrossGatysLargerWeightHigherlayer_content36_style40_weight0_000000_loss9278_iteration400.png}
      
      \includegraphics[width=0.24\linewidth]{good/MulCrossGatysLargerWeightHigherlayer_content36_style48_weight0_000000_loss15369_iteration400.png}
      \includegraphics[width=0.24\linewidth]{good/MulCrossGatysLargerWeightHigherlayer_content36_style86_weight0_000000_loss14895_iteration400.png}
      \includegraphics[width=0.24\linewidth]{good/MulCrossGatysLargerWeightHigherlayer_content36_style97_weight0_000000_loss21764_iteration400.png}
      \includegraphics[width=0.24\linewidth]{good/MulCrossGatysLargerWeightHigherlayer_content36_style95_weight0_000000_loss9644_iteration400.png}
      
      \includegraphics[width=0.24\linewidth]{good/MulCrossGatysLargerWeightHigherlayer_content36_style6_weight0_000000_loss39099_iteration400.png}
      \includegraphics[width=0.24\linewidth]{good/MulCrossGatys_content36_style122_weight0_000000_loss0_iteration400.png}
      \includegraphics[width=0.24\linewidth]{good/MulCrossGatysLargerWeightHigherlayer_content36_style62_weight0_000000_loss11151_iteration400.png}
      \includegraphics[width=0.24\linewidth]{good/MulCrossGatys_content36_style67_weight0_000000_loss0_iteration400.png}
  \caption{good mountain.  }
  \label{fig:good1}
  \end{figure*}
  
  \begin{figure*}[!htbp]
  \centering
  \small 
  \includegraphics[width=0.24\linewidth]{good/MulCrossGatys_content196_style100_weight0_000000_loss0_iteration400.png}
      \includegraphics[width=0.24\linewidth]{good/MulCrossGatys_content196_style10_weight0_000000_loss0_iteration400.png}
      \includegraphics[width=0.24\linewidth]{good/MulCrossGatys_content196_style122_weight0_000000_loss0_iteration400.png}
      \includegraphics[width=0.24\linewidth]{good/MulCrossGatys_content196_style40_weight0_000000_loss0_iteration400.png}
      
      \includegraphics[width=0.24\linewidth]{good/MulCrossGatys_content196_style59_weight0_000000_loss0_iteration400.png}
      \includegraphics[width=0.24\linewidth]{good/MulCrossGatys_content196_style48_weight0_000000_loss0_iteration400.png}
      \includegraphics[width=0.24\linewidth]{good/MulCrossGatys_content196_style62_weight0_000000_loss0_iteration400.png}
      \includegraphics[width=0.24\linewidth]{good/MulCrossGatys_content196_style67_weight0_000000_loss0_iteration400.png}
      
      \includegraphics[width=0.24\linewidth]{good/MulCrossGatys_content196_style85_weight0_000000_loss0_iteration400.png}
      \includegraphics[width=0.24\linewidth]{good/MulCrossGatys_content196_style6_weight0_000000_loss0_iteration400.png}
      \includegraphics[width=0.24\linewidth]{good/MulCrossGatys_content196_style86_weight0_000000_loss0_iteration400.png}
      \includegraphics[width=0.24\linewidth]{good/MulCrossGatys_content196_style89_weight0_000000_loss0_iteration400.png}
      
          \includegraphics[width=0.24\linewidth]{good/MulCrossGatys_content196_style88_weight0_000000_loss0_iteration400.png}
      \includegraphics[width=0.24\linewidth]{good/MulCrossGatys_content196_style94_weight0_000000_loss0_iteration400.png}
      \includegraphics[width=0.24\linewidth]{good/MulCrossGatys_content196_style97_weight0_000000_loss0_iteration400.png}
      \includegraphics[width=0.24\linewidth]{good/MulCrossGatys_content196_style95_weight0_000000_loss0_iteration400.png}
      
      \includegraphics[width=0.24\linewidth]{good/MulCrossGatys_content196_style129_weight0_000000_loss5_iteration400.png}
      \includegraphics[width=0.24\linewidth]{good/MulCrossGatys_content196_style151_weight0_000000_loss0_iteration400.png}
      \includegraphics[width=0.24\linewidth]{good/MulCrossGatys_content196_style161_weight0_000000_loss0_iteration400.png}
      \includegraphics[width=0.24\linewidth]{good/MulCrossGatys_content196_style311_weight0_000000_loss0_iteration400.png}
  \caption{good mountain.  }
  \label{fig:good1}
  \end{figure*}
  