This difference in symmetry groups is important.  Risser argues that
the symmetries of gram matrices in Gatys' method could lead to
unstable reconstructions; they control this effect using feature
histograms.  What causes the effect is that the symmetry rescales features while shifting the mean.  
For the cross-layer loss, the symmetry cannot rescale, and cannot shift the mean.  In turn, the instability
identified in that paper does not apply to the cross-layer gram matrix and our results could not be improved by adopting
a histogram loss.

Write $\vect{x}_{i}$, (resp $\vect{y}_i$ for the feature vector at the $i$'th location (of $N$ in total)
in the first (resp second) layer.  Write $\matx{X}^T=\left[\vect{x}_{1}, \ldots,
\vect{x}_{N}\right]$, etc.   

{\bf Symmetries of the first layer:} Now assume that the first layer has been normalized to zero mean and
unit covariance.  There is no loss of generality, because the whitening transform
can be written into the expression for the group. Write ${\cal
  G}(\matx{W})=(1/N)\matx{W}^T\matx{W}$ for the operator that forms
the within layer gram matrix. We have ${\cal G}(\matx{X})=\matx{I}$.   
Now consider an affine action on layer 1, mapping $\matx{X}_1$ to $\matx{X}_1^*=\matx{X}_1 \matx{A}+\vect{1}\vect{b}^T$; then for this to be a symmetry, we must have
$G(\matx{X}_1^*)= \matx{A}\matx{A}^T+\vect{b}\vect{b}^T=\matx{I}$.  In
turn, the symmetry group can be constructed by: choose $\vect{b}$
which does not have unit length; factor
$N(\matx{I}-\vect{b}\vect{b}^T)$ to obtain $\matx{A}(\vect{b})$ (for
example, by using a cholesky transformation); then any element of the
group is a pair $\left(\vect{b}, \matx{A}(\vect{b})\matx{U}\right)$
where $\matx{U}$ is orthonormal.  Note that factoring will fail for
$\vect{b}$ a unit vector, whence the restriction. 

{\bf The second layer:}   We will assume that the map between layers
of features is linear.  This assumption is not true in practice, but major
differences between symmetries observed under these conditions likely
result in differences when the map is linear.  We can analyze for two
cases: first, all units in the map observe only one input feature
vector (i.e. 1x1 convolutions; the {\em point sample} case); second, spatial homogeneity in the
layers.

{\bf The point sample case:} Assume that every unit in the map
observes only one input feature from the previous layer (1x1
convolutions).   We have $\matx{Y}=\matx{X}\matx{M}+\vect{1}\vect{n}^T$, because the map 
between layers is linear. Now consider the effect on the second layer.
We have ${\cal G}(\matx{Y})=\matx{M}\matx{M}^T+\vect{n}\vect{n}^T$.
Choose some symmetry group element for the first layer,
$\left(\vect{b}, \matx{A}\right)$.  The gram matrix for the second
layer becomes ${\cal G}(\matx{Y}^*)$, where
$\matx{Y}^*=(\matx{X} \matx{A}+\vect{1}\vect{b}^T)\matx{M}^T+\vect{1}\vect{n}^T.$
Recalling that $\matx{A}\matx{A}^T+\vect{b}\vect{b}^T=\matx{I}$ and
$\matx{X}^T\vect{1}=0$, we have 
\[
{\cal
  G}(\matx{Y}^*)=\matx{M}\matx{M}^T+\vect{n}\vect{n}^T+\vect{n}\vect{b}^T\matx{M}^T+\matx{M}\vect{b}\vect{n}^T
\]
so that ${\cal G}(\matx{X}_2^*)={\cal G}(\matx{X}_2)$ if
$\matx{M}\vect{b}=0$.  This is relatively easy to achieve with
$\vect{b}\neq 0$.

{\bf Spatial homogeneity:} Now assume the map between layers has
convolutions with maximum support $r \times r$.  Write $u$ for an
index that runs over the whole feature map, and $\psi(\vect{x}_u)$ for
a stacking operator that scans the convolutional support in fixed
order and stacks the resulting features. For example, given a 3x3
convolution and indexing in 2D, we might have
\[
\psi(\vect{x}_{22})=\left(\begin{array}{c}\vect{x}_{11}\\
\vect{x}_{12}\\
\ldots\\
\vect{x}_{33}
\end{array}\right)
\]

In this case, there is some $\matx{M}$, $\vect{n}$ so that 
$\vect{y}_u=\matx{M}\psi(\vect{x}_u)+\vect{n}$.  We ignore the
effects of edges to simplify notation (though this argument may go
through if edges are taken into account).  Then there is some
$\matx{M}$, $\vect{n}$ so we can write 
\[
{\cal G}(\matx{Y})=(1/N) \sum_u
\matx{M}\psi(\vect{x}_u)\psi(\vect{x}_u)^T\matx{M}^T+\vect{n}\vect{n}^T
\]
Now assume further that
layer 1 has the following (quite restrictive) spatial homogeneity
property: for pairs of feature vectors within the layer $\vect{x}_{i,j}$, $\vect{x}_{i+\delta,
  j+\delta}$ with $\mid \! \delta\!\mid \leq r$ (ie within a convolution window of one
another), we have $\expect{\vect{x}_{i,j}\vect{x}_{i+\delta,
    j+\delta}}=\matx{I}$.  This assumption is consistent with image
autocorrelation functions (which fall off fairly slowly), but is still
strong. Write $\phi$ for an operator
that stacks $r \times r$ copies of its argument as appropriate, so
\[\phi(\matx{I})=\left(\begin{array}{ccc}
\matx{I}&\ldots&\matx{I}\\
\ldots &\ldots \ldots \\
\matx{I}&\ldots&\matx{I}
\end{array}\right).
\]
Then
$G(\matx{Y})=\matx{M}\phi(\matx{I})\matx{M}^T+\vect{n}\vect{n}^T$.
If there is some affine action on layer 1, we have
$G(\matx{Y}^*)=\matx{M}\left(\psi(\matx{A})\phi(\matx{I})\psi(\matx{A}^T)+\psi(\vect{b})\psi(\vect{b}^T)\right)\matx{M}^T+\vect{n}\vect{n}^T$,
where we have overloaded $\psi$ in the natural way.  Now if
$\matx{M}\psi(\vect{b})=0$ and $\matx{A}\matx{A}^T+\vect{b}\vect{b}^T=\matx{I}$, ${\cal
  G}(\matx{Y}^*)={\cal G}(\matx{Y})$. 



{\bf The cross-layer gram matrix:}  Symmetries of the cross-layer gram matrix are very different.  Write
${\cal G}(\matx{X}, \matx{Y})=(1/N) \matx{X}^T\matx{Y}$ for
the cross layer gram matrix.  

{\bf Cross-layer, point sample case:} Here (recalling $\matx{X}^T\vect{1}=0$)we have ${\cal G}(\matx{X},
\matx{Y})=\matx{M}^T$.    Now choose some symmetry group element for the first layer,
$\left(\matx{A}, \vect{b}\right)$.  The cross-layer gram matrix
becomes 
\begin{eqnarray*}
{\cal G}(\matx{X}^*, \matx{Y}^*)&=&(1/N) (\matx{A}
\matx{X}^T+\vect{b}\vect{1}^T)
\left[(\matx{X}
  \matx{A}^T+\vect{1}\vect{b}^T)\matx{M}^T+\vect{1}\vect{n}^T\right]\\
&=&\matx{M}^T+\vect{b}\vect{n}^T
\end{eqnarray*}
(recalling that $\matx{A}\matx{A}^T+\vect{b}\vect{b}^T=\matx{I}$ and
$\matx{X}^T\vect{1}=0$).  But this means that the symmetry requires
$\vect{b}=\vect{0}$; in turn, we must have $\matx{A}\matx{A}^T=\matx{I}$.


{\bf Cross-layer, homogeneous case:} We have
\[
{\cal G}(\matx{X}, \matx{Y})=(1/N)\sum_u \vect{x}_u\left[
  \psi(\vect{x}_u)^T\matx{M}^T+\vect{n}^T\right]=\matx{M}^T.\]
Now choose some symmetry group element for the first layer,
$\left(\matx{A}, \vect{b}\right)$.  The cross-layer gram matrix
becomes 
\begin{eqnarray*}
{\cal G}(\matx{X}^*, \matx{Y}^*)&=&(1/N)\sum_u \left(\matx{A} \vect{x}_u+\vect{b}\right)\left[\left(\psi(\vect{x}_u)^T\psi(\matx{A}^T)+\psi(\vect{b})\right)\matx{M}^T+\vect{n}^T\right]\\
&=&\matx{M}^T+\vect{b}\vect{n}^T
\end{eqnarray*}
(recalling the spatial homogeneity assumption, that $\matx{A}\matx{A}^T+\vect{b}\vect{b}^T=\matx{I}$ and
$\matx{X}_1^T\vect{1}=0$).  But this means that the symmetry requires
$\vect{b}=\vect{0}$; in turn, we must have
$\matx{A}\matx{A}^T=\matx{I}$.
