\appendix{Appendix}
\renewcommand{\theequation}{A.\arabic{equation}}
\setcounter{equation}{0}
\section{Affine connection, spin connection, vielbein, curvature tensors with torsion}
In this appendix, we show some equations about affine connection, spin connection, vielbein, 
curvature tensors with torsion. 
The covariant derivatives of vielbein vanish. 
\begin{align}
\tilde{\nabla}_\mu e^a_\nu 
& =\partial_\mu e^a_\nu-\tilde{\omega}^a_{\ b\mu}e^b_\nu-\tilde{\Gamma}_{\ \nu\mu}^\lambda e^a_\lambda=0, \label{nablae1}\\
\tilde{\nabla}_\mu e_a^\nu & =\partial_\mu e_a^\nu-\tilde{\omega}_{a\ \mu}^{\ b}e_b^\nu+\tilde{\Gamma}^\nu_{\ \lambda\mu}e_a^\lambda
=0. \label{nablae2}
\end{align}
From (\ref{nablae1}), we obtain the relation between the affine connection and the spin connection expressed by
\begin{equation}
\tilde{\Gamma}_{\ \nu\mu}^\lambda = e^\lambda_a(\partial_\mu e^a_\nu-\tilde{\omega}^a_{\ b\mu}e^b_\nu). \label{Gamma}
\end{equation}
We also provide some else equations about  
the affine connection, the spin connection and the vielbein as follows:
\begin{align}
% \tilde{\Gamma}^\lambda_{\rho\nu} & = e_a^\lambda(\partial_\nu e_\rho^a-\tilde{\omega}^a_{\ b\nu}e_\rho^b) \nonumber\\
\partial_\mu\tilde{\Gamma}^\lambda_{\rho\nu} & =
(\partial_\mu e_a^\lambda)(\partial_\nu e_\rho^a-\tilde{\omega}^a_{\ b\nu}e_\rho^b)
+e_a^\lambda\left((\partial_\mu\partial_\nu e_\rho^a)-(\partial_\mu\tilde{\omega}^{ab}_{\ \ \nu})e_{b\rho}
-\tilde{\omega}^{ab}_{\ \ \nu}(\partial_\mu e_{b \rho})\right), \\
\partial_\nu\tilde{\Gamma}^\lambda_{\rho\mu} & =
(\partial_\nu e_a^\lambda)(\partial_\mu e_\rho^a-\tilde{\omega}^a_{\ b\mu}e_\rho^b)
+e_a^\lambda\left((\partial_\nu\partial_\mu e_\rho^a)-(\partial_\nu\tilde{\omega}^{ab}_{\ \ \mu})e_{b\rho}
-\tilde{\omega}^{ab}_{\ \ \mu}(\partial_\nu e_{b \rho})\right), \\
\tilde{\Gamma}^\lambda_{\sigma\mu}\tilde{\Gamma}^\sigma_{\rho\nu} & =
e_a^\lambda(\partial_\mu e_\sigma^a-\tilde{\omega}^{ab}_{\ \ \mu}e_{b\sigma})
e^\sigma_c(\partial_\nu e_\rho^c-\tilde{\omega}^{cd}_{\ \ \nu}e_{d\rho}), \nonumber \\
& = e_a^\lambda(\partial_\mu e_\sigma^a)e_c^\sigma\left((\partial_\nu e_\rho^c)-\tilde{\omega}^{cd}_{\ \ \nu}e_{d\rho}\right)
-e_a^\lambda \tilde{\omega}^{ab}_{\ \ \mu}\eta_{bc}\left((\partial_\nu e_\rho^c)-\tilde{\omega}^{cd}_{\ \ \nu}e_{d\rho}\right), \\
\tilde{\Gamma}^\lambda_{\sigma\nu}\tilde{\Gamma}^\sigma_{\rho\mu} 
& = e_a^\lambda(\partial_\nu e_\sigma^a)e_c^\sigma\left((\partial_\mu e_\rho^c)-\tilde{\omega}^{cd}_{\ \ \mu}e_{d\rho}\right)
-e_a^\lambda \tilde{\omega}^{ab}_{\ \ \nu}\eta_{bc}\left((\partial_\mu e_\rho^c)-\tilde{\omega}^{cd}_{\ \ \mu}e_{d\rho}\right). 
\label{GammaGamma}
\end{align}
Here, since 
\begin{equation}
e_c^\sigma(\partial_\nu e_\rho^c) = (\partial_\nu e_c^\sigma e_\rho^c)-(\partial_\nu e_c^\sigma)e_\rho^c =
(\partial_\nu \delta_\rho^\sigma)-(\partial_\nu e_c^\sigma)e_\rho^c=-(\partial_\nu e_c^\sigma)e_\rho^c, 
\end{equation}
we obtain one more equation as follows:
\begin{align}
e_a^\lambda(\partial_\mu e_\sigma^a)e_c^\sigma\left((\partial_\nu e_\rho^c)
-\tilde{\omega}^{cd}_{\ \ \nu}e_{d\rho}\right)
-e_a^\lambda(\partial_\nu e_\sigma^a)e_c^\sigma\left((\partial_\mu e_\rho^c)
-\tilde{\omega}^{cd}_{\ \ \mu}e_{d\rho}\right) & \nonumber \\
= -(\partial_\mu e_a^\lambda)
\left(
(\partial_\nu e_\rho^a)-\tilde{\omega}^{ad}_{\ \ \nu}e_{d\rho}
\right)
 +(\partial_\nu e_a^\lambda)\left((\partial_\mu e_\rho^a)
 -\tilde{\omega}^{ad}_{\ \ \mu}e_{d\rho}\right). & \label{epe}
\end{align}
Using Eq.(\ref{Gamma})-Eq.(\ref{epe}), we obtain the expression of the Riemann curvature tensor $\tilde{R}^\lambda_{\ \rho\mu\nu}$ 
with torsion as follows:
\begin{align}
\tilde{R}^\lambda_{\ \rho\mu\nu} & =
\partial_\mu\tilde{\Gamma}^\lambda_{\rho\nu}-\partial_\nu\tilde{\Gamma}^\lambda_{\rho\mu}
+\tilde{\Gamma}^\lambda_{\sigma\mu}\tilde{\Gamma}^\sigma_{\rho\nu}
-\tilde{\Gamma}^\lambda_{\sigma\nu}\tilde{\Gamma}^\sigma_{\rho\mu} \nonumber \\
= & (\partial_\mu e_a^\lambda)(\partial_\nu e_\rho^a)-(\partial_\nu e_a^\lambda)(\partial_\mu e_\rho^a)
-(\partial_\mu e_a^\lambda)\tilde{\omega}^{ab}_{\ \ \nu}e_{b\rho}
+(\partial_\nu e_a^\lambda)\tilde{\omega}^{ab}_{\ \ \mu}e_{b\rho} \nonumber \\
& - e_a^\lambda\left((\partial_\mu\tilde{\omega}^{ab}_{\ \ \nu})
-(\partial_\nu\tilde{\omega}^{ab}_{\ \ \mu})\right)e_{b\rho}
- e_a^\lambda\left(\tilde{\omega}^{ab}_{\ \ \nu}(\partial_\mu e_{b\rho})
-\tilde{\omega}^{ab}_{\ \ \mu}(\partial_\nu e_{b\rho})\right)\nonumber\\
& +e_a^\lambda(\partial_\mu e_\sigma^a)(e_c^\rho(\partial_\nu e_\rho^c)-\tilde{\omega}^{cd}_{\ \ \nu}e_{d\rho})
-e_a^\lambda \tilde{\omega}^{ab}_{\ \ \mu}\eta_{bc}\left((\partial_\nu e_\rho^c)-\tilde{\omega}^{cd}_{\ \ \nu}e_{d\rho}\right)
\nonumber \\
& -e_a^\lambda(\partial_\nu e_\sigma^a)e_c^\sigma\left((\partial_\mu e_\rho^c)-\tilde{\omega}^{cd}_{\ \ \mu}e_{d\rho}\right) 
+e_a^\lambda \tilde{\omega}^{ab}_{\ \ \nu}\eta_{bc}\left((\partial_\mu e_\rho^c)-\tilde{\omega}^{cd}_{\ \ \mu}e_{d \rho}\right) 
\nonumber \\
= & -e_a^\lambda e_{b \rho} \left(
(\partial_\mu\tilde{\omega}^{ab}_{\ \ \nu})-(\partial_\nu\tilde{\omega}^{ab}_{\ \ \mu})
-(\tilde{\omega}^{ac}_{\ \ \mu}\tilde{\omega}_{c\ \nu}^{\ b}
-\tilde{\omega}^{ac}_{\ \ \nu}\tilde{\omega}_{c\ \mu}^{\ b})
\right). \label{tildeR}
\end{align}
We also note some equations on traces of gamma matrices and their product with curvature and torsion tensors. 
\begin{align}
Tr(\gamma^{\mu\nu}) = 0, \ \ & Tr(\gamma^{\mu\nu}\gamma^{\rho\sigma}) = 4(g^{\mu\sigma}g^{\nu\rho}-g^{\mu\rho}g^{\nu\sigma}), \\
Tr\frac{1}{8}(\gamma^{\mu\nu}\gamma^{\rho\sigma})\tilde{R}_{\rho\sigma\mu\nu} 
= & \frac{4}{8}(g^{\mu\sigma}g^{\nu\rho}-g^{\mu\rho}g^{\nu\sigma})\tilde{R}_{\rho\sigma\mu\nu}
=-g^{\nu\sigma}\tilde{R}^\rho_{\ \sigma\rho\nu}= -\tilde{R}, \label{gamma2R}\\
Tr(-\frac{1}{16}\gamma^{\alpha\beta}\gamma^{\rho\sigma}T_{\mu\alpha\beta}T^\mu_{\ \rho\sigma}) = &
-\frac{4}{16}(g^{\alpha\sigma}g^{\beta\rho}-g^{\alpha\rho}g^{\beta\sigma})T_{\mu\alpha\beta}T^\mu_{\ \rho\sigma}
=\frac{1}{2}T^{\mu\rho\sigma}T_{\mu\rho\sigma}.
\end{align}
The Riemann curvature tensor $\tilde{R}^\lambda_{\ \rho\mu\nu}$, the Ricci tensor $\tilde{R}_{\rho\nu}$ and the 
curvature $\tilde{R}$ with torsion are related to those without torsion by
 \begin{align}
\tilde{R}^\lambda_{\ \rho\mu\nu} & = 
R^\lambda_{\ \rho\mu\nu}+\nabla_\mu Y^\lambda_{\ \rho\nu}-\nabla_\nu Y^\lambda_{\ \rho\mu}
+Y^\lambda_{\ \sigma\mu}Y^\sigma_{\ \rho\nu}-Y^\lambda_{\ \sigma\nu}Y^\sigma_{\ \rho\mu}, \label{tcurvet}\\
\tilde{R}_{\rho\nu} & =
\tilde{R}^\lambda_{\ \rho\lambda\nu} = R_{\rho\nu}+\nabla_\lambda Y^\lambda_{\ \rho\nu}-\nabla_\nu Y^\lambda_{\ \rho\lambda}
+Y^\lambda_{\ \sigma\lambda}Y^\sigma_{\ \rho\nu}-Y^\lambda_{\ \sigma\nu}Y^\sigma_{\ \rho\lambda}, \label{tRicci}\\
\tilde{R} & =R-2\nabla_\mu T^{\alpha\mu}_{\ \ \alpha}-T^{\alpha\mu}_{\ \ \ \alpha}T^\beta_{\ \mu\beta}
+\frac{1}{4}T^{\mu\nu\lambda}T_{\mu\nu\lambda}+\frac{1}{2}T^{\mu\nu\lambda}T_{\lambda\nu\mu} \nonumber \\
& = R + \nabla_\mu(\ovl{\psi}^\mu\gamma_\nu\psi^\nu)
-\frac{1}{4}\ovl{\psi}^\mu\gamma^\alpha\psi_\alpha\ovl{\psi}_\mu\gamma^\beta\psi_\beta
+\frac{1}{8}\ovl{\psi}^\nu\gamma^\mu\psi^\lambda\ovl{\psi}_\nu\gamma_\lambda\psi_\mu 
\nonumber \\
& + \frac{1}{16}\ovl{\psi}^\nu\gamma^\mu\psi^\lambda\ovl{\psi}_\nu\gamma_\mu\psi_\lambda. \label{tcurve}
\end{align} 
\renewcommand{\theequation}{B.\arabic{equation}}
\setcounter{equation}{0}
\section{Supertrace of $\mathbb{Z}$, $\mathbb{Z}\mathbb{Z}$, $\Omega_{\mu\nu}\Omega^{\mu\nu}$}
Using Eq.(\ref{Zphi}), (\ref{Zpsi}), the supertrace of the matrix $\mathbb{Z}$ of (\ref{matrixZ}) and 
$\mathbb{Z}\mathbb{Z}$ are given by 
\begin{align}
{\rm Str} \mathbb{Z} = & 4Z^{(\varphi)}-Tr Z^{(\psi)} = (1+4\lambda)\tilde{R} - \frac{1}{2}T^{\mu\rho\sigma}T_{\mu\rho\sigma}, \\
% \end{equation}
% \begin{align}
{\rm Str} \mathbb{Z}\mathbb{Z} = & 4 Z^{(\varphi)}Z^{(\varphi)}-Tr Z^{(\psi)}Z^{(\psi)} \nonumber \\
= & (4\lambda^2-\frac{1}{4})\tilde{R}^2-\tilde{R}_{\nu\rho}(-\tilde{R}^{\nu\rho}+\tilde{R}^{\rho\nu}) 
-\frac{1}{4}\tilde{R}_{\mu\nu\lambda\rho}(
 \tilde{R}^{\mu\nu\lambda\rho}+\tilde{R}^{\mu\lambda\rho\nu}+\tilde{R}^{\mu\rho\nu\lambda}
 ) \nonumber \\
& -\frac{1}{4}\tilde{R}_{\mu\nu\lambda\rho}(
 \tilde{R}^{\lambda\rho\mu\nu}+\tilde{R}^{\lambda\nu\rho\mu}+\tilde{R}^{\nu\rho\lambda\mu}
) 
-\frac{1}{2}(\tilde{\nabla}_\mu T^\mu_{\ \alpha\beta})(\tilde{\nabla}_\nu T^{\nu \beta\alpha}) \nonumber \\
& - \frac{1}{8}(
T^{\xi\mu\nu}T_{\xi\mu\nu}T_{\zeta\alpha\beta}T^{\zeta\alpha\beta}
+2T^{\xi\mu\nu}T_{\xi\alpha\beta}T_{\zeta\mu\nu}T^{\zeta\alpha\beta}
+4T^{\xi\mu\nu}T_{\xi\alpha\beta}T_{\zeta\mu}^{\ \ \alpha}T^{\zeta\beta}_{\ \ \nu}) \nonumber \\
& -(1+4\lambda)\tilde{R}\tilde{\nabla}_\alpha Q^\alpha
+2\tilde{R}_{\rho\nu} \tilde{\nabla}_\sigma T^{\sigma \nu\rho}
+(\frac{1}{2}+2\lambda)\tilde{R}Q_\mu Q^\mu 
+\frac{1}{4}\tilde{R}T_{\kappa\rho\sigma}T^{\kappa\rho\sigma} \nonumber \\
& -\tilde{R}_{\lambda\rho\mu\nu}T_\sigma^{\ \lambda\mu}T^{\sigma\rho\nu} 
 +\frac{1}{2}\tilde{R}_{\lambda\rho\mu\nu}T_\sigma^{\mu\nu}T^{\sigma\lambda\rho} 
 -\frac{1}{2}(\tilde{\nabla}_\mu Q^\mu)T_{\nu\rho\sigma}T^{\nu\rho\sigma} \nonumber \\
& -\frac{1}{4}Q^\alpha Q_\alpha T_{\kappa\mu\nu}T^{\kappa\mu\nu}. \label{StrZZ}
\end{align}
In the same way, using (\ref{Omegaphi}), (\ref{Omegapsi}), we obtain the supertrace of $\Omega_{\mu\nu}\Omega^{\mu\nu}$ as 
follows:
\begin{align}
\lefteqn{{\rm Str}\Omega_{\mu\nu}\Omega^{\mu\nu}} \nonumber \\
= &
4\Omega^{(\varphi)}_{\mu\nu}\Omega_{(\varphi)}^{\mu\nu}-Tr \Omega_{\mu\nu}^{(\psi)}\Omega^{\mu\nu}_{(\psi)}
\nonumber \\
= & \frac{1}{2}\tilde{R}_{\mu\nu\alpha\beta}\tilde{R}^{\mu\nu\alpha\beta}
+g^{\mu\nu}g^{\rho\sigma}\left(
(\tilde{\nabla}_\mu T_\rho^{\ \alpha\beta})(\tilde{\nabla}_\nu T_{\sigma\alpha\beta})
-(\tilde{\nabla}_\mu T_{\rho\alpha\beta})(\tilde{\nabla}_\sigma T_\nu^{\ \alpha\beta})
\right)\nonumber \\
& +\frac{1}{4}(T_\alpha^{\ \mu\nu}T^{\alpha\lambda}_{\ \ \ \nu}T_{\beta\mu}^{\ \ \ \sigma}T^\beta_{\ \lambda\sigma}
-T_\alpha^{\ \mu\nu}T^{\alpha\lambda}_{\ \ \ \sigma}T_{\beta\mu}^{\ \ \ \sigma}T^\beta_{\ \lambda\nu}
-2T_\alpha^{\ \mu\nu}T^{\alpha\lambda}_{\ \ \ \sigma}T^\beta_{\ \mu\nu}T_{\beta\lambda}^{\ \ \ \sigma}
\nonumber \\
& -32T_\alpha^{\ \mu\nu}T^\alpha_{\ \beta\rho}T_\mu^{\ \lambda\beta}T_{\nu\lambda}^{\ \ \ \rho}
) 
% \nonumber\\ & 
+4\tilde{R}^{\lambda\rho\mu\nu}\tilde{\nabla}_\mu T_{\nu\lambda\rho}
\nonumber \\
& -8\tilde{R}_{\alpha\beta\mu\nu}T^{\mu\sigma\alpha}T_{\ \sigma}^{\nu\ \beta}
-\tilde{R}_{\alpha\beta\mu\nu}T^{\sigma\mu\nu}T_\sigma^{\ \alpha\beta}
-16(\tilde{\nabla}_\mu T_\nu^{\ \alpha\beta})T^{\mu\sigma}_{\ \ \alpha}T^\nu_{\ \ \sigma\beta}\nonumber \\
& -2(\tilde{\nabla}_\mu T_\nu^{\ \alpha\beta})T^{\lambda\mu\nu}T_{\lambda\alpha\beta}. \label{StrOmegaOmega}
\end{align}
When we develop terms of the second power of the Ricci and Riemann curvature tensors in the equation (\ref{StrZZ}), 
we can use equations as follows:
% If we want to express equations (\ref{StrZZ}) and (\ref{StrOmegaOmega}) in terms of 
% curvature and Ricci tensors and curvature without tildes, we can use equations (\ref{tcurvet}), (\ref{tRicci}), 
% (\ref{tcurve}) and equations as follows:
\begin{align}
\lefteqn{-\frac{1}{2}\tilde{R}_{\nu\rho}(-\tilde{R}^{\nu\rho}+\tilde{R}^{\rho\nu})+\tilde{R}_{\mu\nu}\tilde{\nabla}_\lambda
T^{\lambda\nu\mu}-\frac{1}{4}(\tilde{\nabla}_\mu T^\mu_{\ \alpha\beta})(\tilde{\nabla}_\nu T^{\nu\beta\alpha}) }\nonumber \\
& =  \frac{1}{4}(\tilde{\nabla}_\mu Q_\nu - \tilde{\nabla}_\nu Q_\mu-Q_\alpha T^\alpha_{\ \mu\nu})
(\tilde{\nabla}^\mu Q^\nu - \tilde{\nabla}^\nu Q^\mu-Q_\beta T^{\beta\mu\nu}) \nonumber \\
& = \frac{1}{2}\left((\tilde{\nabla}_\mu Q_\nu)(\tilde{\nabla}^\mu Q^\nu)-(\tilde{\nabla}_\mu Q_\nu)(\tilde{\nabla}^\nu Q^\mu) -(\tilde{\nabla}_\mu Q_\nu)Q_\alpha T^{\alpha\mu\nu}\right) 
+Q_\alpha T^{\alpha\mu\nu}Q_\beta T^\beta_{\ \mu\nu}.
\end{align}
\begin{align}
\lefteqn{-\frac{1}{8}\tilde{R}_{\mu\nu\lambda\rho}(
\tilde{R}^{\mu\nu\lambda\rho}+\tilde{R}^{\mu\lambda\rho\nu}+\tilde{R}^{\mu\rho\nu\lambda}
 +\tilde{R}^{\lambda\rho\mu\nu}+\tilde{R}^{\lambda\nu\rho\mu}+\tilde{R}^{\nu\rho\lambda\mu}
)}\nonumber \\
= & -\frac{1}{16}(\tilde{\nabla}_\mu T_{\rho\alpha\beta})(\tilde{\nabla}^\mu T^{\rho\alpha\beta})
-\frac{1}{8}(\tilde{\nabla}_\mu T_{\rho\alpha\beta})(\tilde{\nabla}^\alpha T^{\rho\beta\mu})
+\frac{1}{16}(\tilde{\nabla}_\mu T_{\rho\alpha\beta})(\tilde{\nabla}^\rho T^{\mu\alpha\beta}) \nonumber \\
& -\frac{1}{8}(\tilde{\nabla}_\mu T_{\rho\alpha\beta})(\tilde{\nabla}^\alpha T^{\beta\mu\rho})
-\frac{1}{8}(\tilde{\nabla}_\mu T_{\rho\alpha\beta})(\tilde{\nabla}^\mu T^{\beta\rho\alpha})
-\frac{1}{8}(\tilde{\nabla}_\mu T_{\rho\alpha\beta})(\tilde{\nabla}^\rho T^{\beta\alpha\mu})
\nonumber \\
& +\frac{1}{8}(\tilde{\nabla}_\mu T_{\rho\alpha\beta})(\tilde{\nabla}^\alpha T^{\mu\beta\rho}) 
% \nonumber \\
-\frac{1}{8}(\tilde{\nabla}_\mu T_\nu^{\ \alpha\beta})(
  2T^{\sigma\mu}_{\ \ \ \alpha}T_{\beta\sigma}^{\ \ \ \nu}+ 2T^{\sigma\nu}_{\ \ \ \alpha}T_{\beta\ \sigma}^{\ \mu}
  +2T^{\sigma\nu\mu}T_{\alpha\beta\sigma}
\nonumber \\
& -T^{\mu\nu\sigma}T_{\sigma\alpha\beta}
  -2T^{\sigma\nu}_{\ \ \ \alpha}T^\mu_{\ \beta\sigma}+2T^{\sigma\mu}_{\ \ \ \alpha}T^\nu_{\ \beta\sigma})
% \nonumber \\ & 
+\frac{3}{8}(\tilde{\nabla}_\mu T_\nu^{\ \alpha\beta})T_{\sigma\alpha\beta}T^{\nu\mu\sigma} \nonumber \\
& +\frac{1}{8}
T^{\xi\mu\nu}T_{\alpha\xi\beta}(T^{\zeta\ \beta}_{\ \nu}T_{\mu\zeta}^{\ \ \ \alpha}
+T^{\zeta\alpha\beta}T_{\mu\nu\zeta}
+T^{\zeta\alpha}_{\ \ \ \nu}T_{\mu\zeta}^{\ \ \ \beta} 
% \nonumber \\ & 
-T^{\zeta\ \alpha}_{\ \mu}T_{\ \zeta\nu}^{\beta}
-\frac{1}{2}T^\zeta_{\ \nu\mu}T_{\beta\zeta\alpha} \nonumber \\
& -T^{\zeta\ \beta}_{\ \mu}T_{\ \zeta\nu}^{\alpha}
-\frac{1}{2}T^\zeta_{\ \nu\mu}T^{\alpha\ \beta}_{\ \zeta} 
-T_{\zeta\mu\nu}T^{\alpha\zeta\beta}-2T^{\zeta\beta\nu}T^\alpha_{\ \mu\zeta}).
% \lefteqn{\tilde{R}_{\nu\rho}(-\tilde{R}^{\nu\rho}+\tilde{R}^{\rho\nu})
% =-\frac{1}{2}(\tilde{R}^{\nu\rho}-\tilde{R}^{\rho\nu})(\tilde{R}_{\nu\rho}-\tilde{R}_{\rho\nu})}\nonumber \\
% = & -\frac{1}{2}\left(\nabla_\lambda T^{\lambda\rho\nu}+\nabla^\rho T^{\lambda\nu}_{\ \ \lambda}
% -\nabla^\nu T^{\lambda\rho}_{\ \ \lambda}+T^\lambda_{\ \ \ \sigma\lambda}T^{\sigma\rho\nu}
% +\frac{1}{2}(T_\sigma^{\ \nu\lambda}T^{\rho\sigma}_{\ \ \lambda}+T^{\nu\lambda\sigma}T_{\sigma\ \lambda}^{\ \rho} )
% \right)
% \nonumber\\
% & \ \ \ \ \ \ \ \left(\nabla_\alpha T^\alpha_{\ \rho\nu}+\nabla_\rho T^\alpha_{\ \nu\alpha}
% -\nabla_\nu T^\alpha_{\ \rho\alpha}+T^\alpha_{\ \beta\alpha}T^\beta_{\ \rho\nu}
% +\frac{1}{2}(T^{\beta\ \alpha}_{\ \nu}T_{\rho\beta\alpha}+T_\nu^{\ \alpha\beta}T_{\beta\rho\alpha})
% \right) \\
% \lefteqn{
%\tilde{R}_{\mu\nu\lambda\rho}(\tilde{R}^{\mu\nu\lambda\rho}+\tilde{R}^{\mu\lambda\rho\nu}+\tilde{R}^{\mu\rho\nu\lambda})
% } \nonumber \\
% = & (\nabla_\lambda T_{\mu\nu\rho})(\nabla^\lambda T^{\mu\nu\rho})
% +2(\nabla_\rho T_{\mu\lambda\nu})(\nabla^\lambda T^{\mu\nu\rho}) \nonumber \\
% & +2(T_{\mu\sigma\rho}+T_{\sigma\rho\mu}+T_{\rho\sigma\mu})T^\sigma_{\ \lambda\nu}\nabla^\lambda T^{\mu\nu\rho}
% +(\nabla^\lambda T^{\mu\nu\rho})T^\sigma_{\ \nu\rho}(T_{\mu\sigma\lambda}+T_{\sigma\lambda\mu}+T_{\lambda\sigma\mu})\nonumber \\
% & +\frac{1}{4}(T_{\mu\sigma\lambda}+T_{\lambda\sigma\mu})T^\sigma_{\ \nu\rho}
% (T_{\ \alpha}^{\mu\ \lambda}+T_{\ \alpha}^{\lambda\ \mu})T^{\alpha\nu\rho}
% +\frac{1}{4}T_{\sigma\lambda\mu}T^\sigma_{\ \nu\rho}T_{\alpha}^{\ \lambda\mu}T^{\alpha\nu\rho} \nonumber \\
% & +\frac{1}{2}(T_{\mu\sigma\nu}+T_{\sigma\nu\mu}+T_{\nu\sigma\mu})T^\sigma_{\ \rho\lambda}
% (T_{\ \alpha}^{\mu\ \lambda}+T_{\alpha}^{\ \lambda\mu}+T_{\ \alpha}^{\lambda\ \mu})T^{\alpha\nu\rho}
\end{align}