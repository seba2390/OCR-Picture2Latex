

%ALTERNATIVE FIRST SENTENTCE: Homophily and different group sizes can disproportionally effect the visibility of nodes in social networks.

%Homophily can cause disproportionate visibility of unequally-sized groups in social networks. 
Homophily can put minority groups at a disadvantage by restricting their ability to establish links with people from a majority group. This can limit the overall visibility of minorities in the network. Building on a Barab\'{a}si-Albert model variation with groups and homophily, we show how the visibility of minority groups in social networks is a function of (i) their relative group size and (ii) the presence or absence of homophilic behavior. We provide an analytical solution for this problem and demonstrate the existence of asymmetric behavior. Finally, we study the visibility of minority groups in examples of real-world social networks: sexual contacts, scientific collaboration, and scientific citation. Our work presents a foundation for assessing the visibility of minority groups in social networks in which homophilic or heterophilic behaviour is present. %<M> Not sure about the last sentence


%As a consequence, studies of minorities in social networks can  


%It is not surprising that the position of individuals in their social network can influence their visibility and power in the network. In social networks, individuals may differ according to various static and dynamic attributes such as ethnicity, gender or economic wealth. These attributes are often unequally distributed in a social network which leads to the formation of sub-populations of different sizes and potentially also segregation. 
%Many social networks exhibit intrinsic characteristic such as preferential attachment (the tendency to connect with popular others) and homophily (the tendency to connect with similar others). However, when homophily guides the formation of new edges in imbalanced sub-populations (e.g. unequal number of men/women in scientific collaboration), the visibility and power of individuals depends on the size of the sub-populations and the degree to which homophily is presented in the network.

%In this paper, we explore the relationship between homophily, sub-population size and the visibility of individuals which is often determined by information retrieval algorithms such as pagerank. We show analytically and numerically which conditions may alter the visibility of a minority in the presence of homophily.




%This problem can be more severe in the presence of an unequal population. For instance, in scientific disciplines where number of men and women or people in color are not equally distributed, such inequality combined with homophilic interactions can impact receiving novel ideas or information. Previous research were mainly focused on efficiency of navigational strategies in networks and did not consider to what extend the intrinsic property of the networks namely homophily, can create searchability advantages to a certain population.  