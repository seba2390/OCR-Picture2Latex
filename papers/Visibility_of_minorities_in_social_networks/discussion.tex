\section{Discussion}



We demonstrate analytically and numerically that the visibility of nodes (measured by degree) is influenced by relative group size difference and homophily and the visibility has asymmetric and non-linear properties. As the size of a minority group decreases, minority can benefit more from heterophilic interactions and suffer from homophilic interactions. However, minorities can recover their visibility by full in-group support. Although our model makes simple assumptions such as all members of the same group behave similarly and are equally active, it lays a theoretical foundation for studying how the inherent properties of networks would lead to biases in visibility or ranking of groups, in particular minorities. %Neglecting the relative group size differences and homophily would lead to sever consequences for minorities particularly in search engines or ranking algorithms.

Our work can be extended in multiple ways. First, new ranking algorithms can be devised to harness relative group size differences and homophily to ensure the representativeness of minorities and correct for potential biases. Second, the model can be extended to account for directionality and multiple attributes in networks and multiplex networks. Third, this model can be used to study community detections in annotated networks \cite{newman2016structure}, sampling hard-to-reach populations \cite{shaghaghi2011approaches} or evaluating the performance of classifiers in machine learning tasks \cite{hardt2016equality,dwork2012fairness}. We anticipate that this work will inspire more empirical and theoretical exploration on the impact of network structure on the visibility and ranking of minorities to help establish more equality and fairness in society. 








