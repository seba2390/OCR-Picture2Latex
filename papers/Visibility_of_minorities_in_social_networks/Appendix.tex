%\section{Appendix}

\section{Degree distribution - continuum approximation}


%In the previous models of network growth with fitness the main assumption is that the fitness of the nodes is fixed and is independent on other nodes. \phsi{what is fitness? have we introduced that?} There are only few papers that consider the interactive terms and fitness e.g. when nodes are positioned in different geometric coordinates.

%In our model, the probability of nodes to receive links from the newly arrived nodes depend on the relative social distance that they have to the arrival node. This probability can vary depend on the arrival nodes. 

Here we use continuum theory similar to \cite{bianconi2001competition} to compute the degree growth for each group. The probability that a newly entered node $j$ choose node $i$ depends on the relative ``fitness'' of the node $i$ towards the node $j$. We approximate the relative fitness by averaging fitness depends on the probability of visiting each group of nodes. Let us denote the average fitness of a node in group $a$ as $\bar{h_{a}}$.

The rate of degree growth for a node in each time step depends on number of arrival links $m$, the relative fitness of the node to other nodes and its degree:

\begin{equation}
\label{eq:k_growth}
    \frac{\partial k_i}{\partial t} = m \frac{ \bar{h_{i}} k_i}{\sum_{l} \bar{h_{l}} k_l}
\end{equation}
 
At each time step, node $i$ has $m$ chances to be selected. The sum in the denominator goes over all links that occur from $t=0$ to $t = t$.


Assuming that the node $i$ joins the network at time $t_0$, the average degree of the node at time $t$ is
\begin{equation}
\label{eq:degree_growth}
    k_i(t,t_{0}) = m  (\frac{t}{t_0})^{\beta(\bar{h_{i}})}
\end{equation}

where we define
\begin{equation}
\label{eq:beta}
    \beta(\bar{h_{i}}) = \frac{ \bar{h_{i}} }{\sum_{l} \bar{h_{l}} k_l} 
\end{equation}

%in which $C_j$ is the sum of probabilities to attach to group $a$ or $b$.

we can rewrite Eq.~\ref{eq:k_growth} as follows:
\begin{equation}
\label{eq:k_growth2}
\frac{\partial k_i}{\partial t} = m k_i \beta(\bar{h_{i}})
\end{equation}

The exponent $\beta(\bar{h_{i}})$ is bounded, $0< \beta(\bar{h_{i}})< 1$, because number of links always increase for a node and number of links cannot increase faster than $t$. 

The main difference of this model with the classical preferential attachment model is the parameter $\bar{h_{i}}$ which regulates the homophily between node $i$ to all other nodes. If we assume that node $i$ belong to group $a$, then the average fitness of nodes in group $a$, $\bar{h_{a}}$, depends on probability of visiting nodes from the same group multiply their relative homophily and probability of visiting nodes from different group and their relative homophily:

\begin{equation}
\label{eq:fitness_group}
    \bar{h_{a}} = f_{a} \delta(j-a) (h_{ja}) + \delta(j-b)  f_{b} (h_{jb})
\end{equation}


In which $f_{a}$ and $f_{b}$ represent the fraction of nodes in group $a$ and $b$ and $h_{ja}$ relative homophily between node from group $j$ to group $a$.

%Note that in this simple model the attractiveness for the nodes in group B is complementary.  


Let's now consider the mean of the sum in the denominator of the exponent in equation \ref{eq:beta}. In the continuum approximation, the sum can be written as integral over nodes that are born in different time ($t_{0}$):

\begin{equation}
   \ev{\sum_{l} h_{l} k_l} = \int \bar{h_{l}} d\bar{h} \rho(\bar{h}) \int_{1}^{t} dt_{0} k_i(t,t_{0}) 
\end{equation}

In the case of homophilic graph with two groups, $\rho(\bar{h})$ is equivalent to the size of each group.


Inserting $k_i(t,t_{0})$  into the equation, and neglecting $t^{\beta}$ when $t$ is large, $t \to \infty$, we get,
\begin{equation}
   \ev{\sum_{l} h_{l} k_l} = C m t
\end{equation}
where 
\begin{equation}
\label{eq:c}
   C = \int \rho(\bar{h}) d\bar{h} \frac{h_{ij}}{1-\beta(\bar{h})}
\end{equation}

$C$ is the denominator of eq.~\ref{eq:beta} and it shows the growth of connectivity probability. Inserting that into Eq.~\ref{eq:k_growth2}, we get
\begin{equation}
\label{eq:beta_2}
    \beta(\bar{h}) = \frac{\bar{h}}{C}
\end{equation}

Note that in the presence of two groups, $C$ consists of two parts, $C_{a}$ and $C_{b}$:
\begin{equation}
\label{eq:c_all}
   C = C_{a} + C_{b}
\end{equation}

Using eq.~\ref{eq:c} we can compute the probability growth for each group. For group $a$ we have:

\begin{align*}
\label{eq:c2}
  C_{a} &= \int_{\bar{h}} \rho(\bar{h}) d\bar{h} \frac{1}{1 - \beta(\bar{h}) } \\
    &= \frac{f_{a} h_{aa}}{1 - \beta(\bar{h_a})   } + \frac{f_{b} h_{ab}}{1 - \beta(\bar{h_b}) } 
\end{align*}

and
\begin{equation}
\label{eq:cc}
  C_{b} = \frac{f_{b} h_{bb}}{1 - \beta(\bar{h_b})   } + \frac{f_{a} h_{ba}}{1 - \beta(\bar{h_a}) } 
\end{equation}



Since in each time step only one node is arriving, the sum of all degrees should be equivalent to all the incoming nodes and links 

\begin{equation}
   \ev{\sum_{i}  k_i} =  \int_{1}^{t} dt_{0} k_i(t,t_{0}) = mt \sum_s f_s / (1 - \bar{q_s}) = 2mt
\end{equation}

therefore get an additional identity
\begin{equation}
   \sum_s f_s / (1 - \bar{{h}_s}) = 2
\end{equation}



where $s$ represents number of groups which in our case group $a$ and $b$. We then can solve the self-consistent equation and get the value of $C$ and therefore determine the exponent $\beta$ for each group. In \cite{ferretti2012features} the authors derived the generalized form of this model using rate equation approach \cite{krapivsky2000connectivity} for nodes that are distributed in geometrical space.

The cumulative probability that a node with fitness $\bar{h}$ has a degree larger than $k$ is

\begin{align*}
    P(k(t)> k) &= P(t_{0} < t(m/k)^{1/\beta}) = t (m/k) ^{C/\bar{h}}
\end{align*}

Thus the probability of a node to have $k$ links, is given by

\begin{equation}
\label{eq:degree_exponents}
    p(k) = \int_{\bar{h}} \rho(\bar{h}) d\bar{h} \frac{\partial P(k)}{\partial k} =  \int_{\bar{h}} \rho(\bar{h}) d\bar{h} \frac{C}{\bar{h}} (m/k)^{\frac{1}{\beta(\bar{h})} + 1}
\end{equation}

\begin{equation}
\label{eq:degree_exponent}
    \gamma(h) =\frac{1}{\beta(\bar{h})} + 1
\end{equation}

The slope of the distribution is determined by the exponent $\frac{C}{\bar{h}} + 1$. In the case of original Barab\'{a}si-Albert model with $C=2$ and $q = 1$, we get $p(k) \propto k^{-3}$. The same is true in our model with equal homophily for minorities and majorities. If homophily is equal to $0.5$, the attractiveness for both population is the same and the sum of probabilities over time $C(t)$ converges to $1$. Therefore the slope of the distributing will be 3.





%$C$ has a quadratic form with two solutions. Since in this model $q/C < 1$ for every value of $q$, only one solution of $C$ is acceptable. From the fraction of nodes in each group and homophily parameter (social distance), we can determine the analytical value for $C$. In addition $C$ can be calculated empirically from the simulation. In figure \ref{fig:degree_dynamic} inset, we show that the analytical prediction of $C$ is in good agreement with the empirical value. 


%\section{Group size and top degree ranks}

%\begin{figure*}[]
%\centering
%\includegraphics[width=0.40\linewidth]{figs/top_ranks/compare_degrank_01_corrected.pdf}
%\includegraphics[width=0.40\linewidth]{figs/top_ranks/compare_degrank_02_corrected.pdf}
%\includegraphics[width=0.40\linewidth]{figs/top_ranks/compare_degrank_03_corrected.pdf}
%\includegraphics[width=0.40\linewidth]{figs/top_ranks/compare_degrank_04_corrected.pdf}
%\includegraphics[width=0.40\linewidth]{figs/top_ranks/compare_degrank_05_corrected.pdf}

%\caption{\textbf{Fraction of minority nodes that are found in the top k\% of nodes with highest degree.} Each subplot represents a different minority fraction.   }
%\label{fig:exponents_asym}
%\end{figure*}









%The fitness model has been used to model the network structure of the World Wide Web ; Experience versus Talent Shapes the Structure of the Web , Kong et al pnas 2008

%Ref: Stat. Mech. of Complex Net. http://www3.nd.edu/~networks/Publication%20Categories/03%20Journal%20Articles/Physics/StatisticalMechanics_Rev%20of%20Modern%20Physics%2074,%2047%20%282002%29.pdf

%\subsection{Derivation of degree}
%- plot degree dist vs. similitude
%- derive prob. of the degree dist.



%\begin{figure*}[]
%\centering
%\includegraphics[width=0.8\linewidth]{figs/degree_dynamic/deg_dist_all.pdf}
%\caption{\textbf{Degree distribution for minorities (red) and majorities (blue) in 5 different regimes;} a) $h = 0.0$, b) $h = 0.2$, c)$h = 0.5$ d)$h = 0.8$ e)$h = 1$. In all cases, minority fraction is set to $0.2$. Number of nodes in the network is set to $5 \times 10^{3}$ and results are averaged over 20 runs. The Analytical prediction is shown by dashed line.}
%\label{fig:degree_dist_simulation}
%\end{figure*}



%\begin{figure*}[]
%\centering
%\includegraphics[width=0.40\linewidth]{figs/assortativity/homophily_assortativity_f_01.pdf}
%\includegraphics[width=0.40\linewidth]{figs/assortativity/homophily_assortativity_f_02.pdf}
%\includegraphics[width=0.40\linewidth]{figs/assortativity/homophily_assortativity_f_03.pdf}
%\includegraphics[width=0.40\linewidth]{figs/assortativity/homophily_assortativity_f_04.pdf}
%\includegraphics[width=0.40\linewidth]{figs/assortativity/homophily_assortativity_f_05.pdf}

%\caption{\textbf{The relation between assortativity and homophily.} }
%\label{fig:exponents_asym}
%\end{figure*}

%\begin{figure*}[]
%\centering
%\includegraphics[width=0.8\linewidth]{figs/degree_dynamic/deg_growth_all.pdf}
%\caption{\textbf{Degree connectivity versus time for majority (blue) and minority (red) nodes in 5 different regimes} a) $h = 0.0$, b) $h = 0.2$, c)$h = 0.5$ d)$h = 0.8$ e)$h = 1$. In all cases, the minority fraction is set to $0.2$. One can see that degree of the minority group grows faster in heterophilic network with $h = 0.2$ than the majority degree in homophilic networks with $h = 0.8$. Number of nodes in the network is $5 \times 10^{3}$ and results are averaged over 20 runs. \kaf{Under construction!}}
%\label{fig:degree_dynamic_simulation}
%\end{figure*}