\section{Results}


\textbf{Model} 

\noindent We use the well-known model of preferential attachment proposed by Barab\'{a}si and Albert \cite{Barabasi99}, and we incorporate  homophily as an additional parameter to the model \cite{de2013scale,avin2015homophily}. Thus, the mechanism of tie formation in our model is influenced by the interplay between preferential attachment, via the degree of nodes, and homophily, via node attributes. A more general version of this model, known as the fitness model, was first proposed by Bianconi and Barab\'{a}si  \cite{bianconi2001competition}. In this model, the probability of a connection is the product of the degree and fitness of the node. However, the fitness of a node is assumed to be constant regardless of the presence of other nodes. In our model, the fitness of a node also depends on the attributes of other nodes.



We model social networks with two groups of nodes, in which all the nodes from the same group behave similarly. Let us call the two groups $a$ and $b$. We define a tunable homophily parameter $h$ that regulates the tendency of individuals to connect with other individuals who belong to the same group. The homophily parameter ranges between 0 to 1, $h \in [0,1]$, where $0$ means that the nodes from one group are attracted only by nodes from the other group (heterophily), $1$ means that the nodes connect only with similar nodes (homophily), and $0.5$ indicates a homogeneous mixing with respect to group affiliation. The model consists of $N$ nodes and two attributes that initially are assigned to two groups with given sizes. We call $f_a$ the fraction of nodes that belong to group $a$, and $f_b = 1 - f_a$ the fraction of nodes that belong to group $b$. We shall refer to group $a$ as the minority and group $b$ as the majority, so that $f_b \ge f_a$. At each time step, a newly arriving node $j$ randomly attaches to $m$ pre-existing nodes by evaluating their degree and group membership. Multiple linkage between two nodes is not allowed. The probability of node $j$ to connect to node $i$ is given by:

\begin{equation}
\Pi_{i} = \frac{h_{ij} k_{i}}{\sum_{l} h_{lj} k_{l}}
\label{eq:homophilic_BA}
\end{equation}
where $k_{i}$ is the degree of node $i$ and $h_{ij}$ is the homophily between the two nodes.


In general, the homophily parameter defines the probability of within and across group connections. For example, in the case of two groups, we have two homophily parameters: $h_{aa}$ (probability of connection between members of group $a$), $h_{bb}$ (probability of connection between members of group $b$), and the probability between groups ($h_{ab}$ and $h_{ba}$) are complementary probabilities ($h_{ab} = 1- h_{aa}$, $h_{ba} = 1- h_{bb}$) . As a simplification, one can assume homophily is regulated by only one parameter $h$, considering that homophily is symmetric and complementary: $h_{aa} = h_{bb} = h$ and  $h_{ab} = h_{ba} = 1 - h$. In this paper, we first provide the results for the simple case of symmetric homophily and then discuss asymmetric homophily. 

\begin{figure*}[]
\centering
\includegraphics[width=0.9\linewidth]{degree_exponent.pdf}
\caption{\textbf{The analytical and numerical degree exponent of minority (A) and majority (B) versus homophily for various minority sizes.} The degree distribution follows a power-law function, $p(k) \propto k^{\gamma}$ in which the exponent of the distribution ($\gamma$) is shown by tuning homophily ($h$) and group sizes (shown by different colors). The dashed lines are the expected degree exponents derived from our analytical derivation (see Methods) and the dots represent the fitted value from the simulations of over 5,000 nodes. The analytical results are in excellent agreement with simulation. The size of the minority fraction is shown for the ranges between $0.1$ and $0.5$. For minority nodes (A), in the heterophily regime ($h < 0.5$), the degree exponent ranges from $-2$ to $-3$, which represents the advantage of these nodes to grow their degree to large values. In the homophilic regime ($h > 0.5$), the exponent shows a non-linear behaviour; as the degree exponent decreases, the advantage of the minorities to grow their degree becomes limited. However, this effect can be compensated in high homophilic regime by in-group support, which explains why the exponent increases for $h > 0.8$. For majorities (B), the heterophilic situation limits their advantage of growing their degree, in particular for small minority fractions. In homophilic regime, the exponent of the majority degree always remains close to $-3$ since the majority nodes do not gain extra advantage due to large group sizes. }
\label{fig:degree_dist_analytical}
\end{figure*}



\noindent \textbf{Degree growth}

\noindent Figure~\ref{fig:fig1} illustrates the dynamics of the degree growth by tuning homophily. The minority fraction is fixed to $0.2$. Our model is generalized and incorporates two types of network interactions. For the parameter range of $ 0 \leq h \leq 0.5$ the network is heterophilic, and for the range of $ 0.5 \leq h \leq 1$, the network is homophilic. In the heterophilic regime, the degree of the minority group grows faster than the degree of the majorities (see Fig.~\ref{fig:fig1} second row). The complete heterophilic case is equivalent to the formation of bipartite networks ($h=0$).  The difference in the degree growth reduces gradually as heterophily decreases, until we reach the homogeneous mixing case ($h=0.5$), in which groups do not matter anymore and we recover the original Barab\'{a}si-Albert growth model for both groups.

In the homophilic regime ($ 0.5 \leq h \leq 1$), the degree of the majority grows faster than the degree of the minority until a certain point $h = 0.8$. After that, the difference in growth decreases until we reach the fully homophilic case ($h = 1$) in which the network is split between the two groups, each having the same degree growth. The extreme homophilic case resembles societies in which women and men are completely segregated at schools or some universities, e.g., in Iran or Saudi Arabia \cite{mehran2003paradox}.

\begin{figure*}[]
\centering
\includegraphics[width=0.9\linewidth]{degree_rank.pdf}
\caption{\textbf{Visibility of minorities in relation to relative group size and homophily.} 
A) Average cumulative degree of the minority as a function of homophily, for different minority proportion (10\% - 50\%).
In a balanced population (0.5, pink line), both groups share half of the links, independently of the level of homophily. As the size of the minority decreases, the inequality in the share of degree increases.
In a homogeneous-mixing case ($h = 0.5$), the rank corresponds to the expected population size shown by the gray dashed lines. 
In heterophilic regimes ($ 0 \leq h < 0.5$), the minority takes advantage of the population size effect. In homophilic regimes ($ 0.5 < h \leq 1$), we observe that the degree of minorities is below the expectation and it is recovered only in the extreme homophilic case ($h = 1$) by full in-group support. 
B) Fraction of minority nodes that are found in the top d\% of nodes with the highest degree. The fraction of nodes belonging to the minority ($f_a$) is set to 0.2. If the group membership does not impact the attractiveness of nodes, we expect that the presence of the minority in the top d\% is proportional to its relative size (dashed line). However, the results are sensitive to the homophily parameter. In the heterophilic case ($0 \leq h < 0.5$), minorities are over-represented in the top d\%. In the homophilic case ($0.5<h \leq 1$), minorities are under-represented in the top d\%. In the case of homogeneous mixing ($h=0.5$) or complete homophily ($h = 1.0$), minorities are presented in the top d\% as expected from their relative size. 
}

\label{fig:degree_rank}
\end{figure*}




%There are other variations of this model to incorporate similarity and popularity that use geometric model of fitness \cite{papadopoulos2012popularity}.%<M> What's the point of this last sentence?

\noindent \textbf{Impact of homophily and group size on degree distribution and visibility}

\noindent Figure~\ref{fig:degree_dist_analytical} shows the exponent of the degree distribution for the minority (Fig.~\ref{fig:degree_dist_analytical}A) and majority (Fig.~\ref{fig:degree_dist_analytical}B) by tuning homophily and group sizes. We determine analytically the exact exponent of the degree growth and the degree distribution as a function of homophily ($h$) and minority size ($f_a$) (see Methods). The degree exponent illustrates the ability of nodes to stretch their degrees to high values and thus receive more visibility. Let us denote the degree distribution $p(k) \sim k^{\gamma}$, where $\gamma$ is the exponent of the degree distribution. When both groups are of equal size ($f_a = 0.5$), the model recovers the exponent $\gamma = -3$ for the degree distributions of both groups, as predicted from the classical  Barab\'{a}si-Albert model. In the heterophilic regime ($h<0.5$), as the size of the minority decreases, the exponent of the degree distribution of the minority increases, which indicates that the distribution stretches to larger values. The opposite situation occurs for majorities; as the size of the minority decreases, the exponent of the degree decreases which indicates that majorities are limited in stretching their degree to large values. 


The homophilic regime ($h>0.5$) exhibits interesting behaviour. While the exponent of the degree distribution for the majority does not change much when we tune group size or homophily, there is a non-linear effect for the minority. As homophily increases, the exponent decreases until we reach a certain homophily value ($h \simeq 0.8$), and increases afterwards (see Fig.~\ref{fig:degree_dist_analytical}A). In the extreme homophilic case ($h = 1.0$) the degree growth of both groups is similar to the homogeneous mixing case ($h = 0.5$) and so are the exponents of the degree distributions. 

This non-linear behaviour can be explained by the interplay between homophily and relative group size differences. Both determine the amount of competition faced by the nodes of different groups. For the majority, heterophilic conditions are not beneficial, since nodes are mostly attracted by the minority, which as a consequence becomes extremely popular. Therefore, majority nodes have difficulties competing for the attention of the newly arriving nodes. In the homophilic regime, the majority is relatively indifferent because they compete for attention mostly among themselves. 

For the minority, heterophilic situations are most beneficial. They receive the most attention from the majority, and the competition for attention among minority nodes is relatively low since they are a small group. In homophilic situations, it is much more difficult for minority nodes to attract newly arriving nodes due to the competition with the majority, which is not only larger in size but also contains more popular nodes. However, in the case of extreme homophily, no competition exists between the nodes of different groups, and thus both groups compete only among themselves. The degrees of nodes in both groups grow similarly and their degree distributions are the same as in the homogeneously mixed case with the only difference that the network is split between the groups.





\noindent \textbf{Visibility of minorities in top ranks}


\noindent So far we have observed that homophily and the differences in group sizes have an effect on the degree growth and the degree distribution of groups. Although these findings may be rather trivial, the outcome of such interactions on the visibility of groups is striking.


Figure \ref{fig:degree_rank}A depicts the average total degree share of the minority as a function of homophily. Colors represent different minority sizes. The results for the majority group are complementary. In the extreme heterophilic case ($h = 0$), a minority group that represents 20\% of the total population (light blue line) receives more than 40\% of all degrees. This result resembles the idea of majority illusion in which the majority of nodes perceives the opinion of the minority as the majority opinion because they are exposed mainly to minority nodes \cite{lerman2015majority}. As the homophily between groups increases up to 0.5, the average total degree decreases to what we would expect from the size of the minorities (dashed gray lines). In the homophilic case ($0.5 \leq h \leq 1$), the degree drops below what we would expect from the population size, and thus, the minority group as a whole is penalized for the homophilic behaviour. In the extreme homophilic situation ($h = 1$), the minority group can take advantage of full in-group support and as a result the degree returns to the expectation that is proportional to the group size. 




If we wish to examine only the top-ranked nodes, which is a realistic scenario for users who want to explore a ranked list of items (as in search user interfaces), the results are even more striking. Figure \ref{fig:degree_rank}B illustrates the probability of finding minorities in the top $d$\% of nodes ranked by degree. For example $d = 0.2$ means the fraction of nodes in the top 20\% of the nodes ranked by degree. 
In the heterophilic case (brown shades), nodes from the minority are overrepresented in the top ranked nodes. In the homophilic case (green shades), nodes from the minority are underrepresented, an effect especially important for small top $d$\,\%. Given the fact that nodes with high degree are very stable in terms of their rank \cite{ghoshal2011ranking}, these results suggest that in homophilic networks, the majority stabilizes its position at high ranks and leaves little opportunity for minorities to appear in the top ranks. In heterophilic cases, the roles are reversed: minority nodes stabilize their position at high ranks.    Given the fact that many social networks are homophilic with respect to attributes such as gender or ethnicity, our results suggest that in homophilic networks majorities occupy the high ranks and minorities tend to appear towards the lower ranks compared to what we would expect from the minority size.








