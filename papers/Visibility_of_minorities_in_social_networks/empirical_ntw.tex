\begin{figure*}[]
\centering
%\includegraphics[width=0.33\linewidth]{figs/empirical_ntw/deg_dist_brazil.eps}
%\includegraphics[width=0.33\linewidth]{figs/empirical_ntw/deg_dist_DBLP.eps}
%\includegraphics[width=0.33\linewidth]{figs/empirical_ntw/deg_dist_APS.eps}

\includegraphics[width=1\linewidth]{empirical_deg_dist_combine.pdf}

\caption{\textbf{Degree distribution of majority and minority groups in three empirical networks.} A) Sexual contact network with sex-workers (blue) and sex-buyers (orange). B) Collaboration network with men (blue) and women (orange). C) American Physical Society (APS) citation network among two category of topics: Classical Statistical Mechanics (CSM, orange) and Quantum Statistical Mechanics (QSM, blue). The fraction of minorities are shown in the plots ($f_{min}$). The dashed line is the fitted line using the maximum likelihood estimate. The exponent of the fit (Fit) is compared with the analytical exponent derived from our model (Model). Our model is able to produce a realistic degree exponent for empirical networks with various types of homophily and group sizes.}
\label{fig:deg_dist_empirical}
\end{figure*}



\begin{figure*}[]
\centering
\includegraphics[width=1\linewidth]{empirical_analytical_topk_std.pdf}
\caption{\textbf{Visibility of minority groups in the top d\% degree rank for three empirical networks}. A) Sexual contact network (minority = sex buyers). B) Scientific collaboration network (minority = women). C) Scientific citation network (minority = Classical Statistical mechanics (CSM)). The solid line is measured from the empirical network and the dashed orange lines are the predicted values from the synthetic networks with a similar homophily parameter for $5,000$ nodes and averaged over $100$ simulations. The dashed gray line is the relative size of the minority. In the heterophilic case of the network of sexual contacts, the minority is overrepresented with respect to its size. In the collaboration network where homophily is moderate, the minority is underrepresented or it is  close to its relative size. In the case of the citation network that is extremely homophilic, the minority is highly underrepresented. These results provide empirical evidence for a visibility bias in empirical networks.}
\label{fig:topk_empirical}
\end{figure*}




\noindent \textbf{Visibility bias in empirical social networks}




\noindent We provide evidence for the emergence of degree inequality and rank differences in real social networks via three collected empirical social networks that exhibit various ranges of group size and homophily: sexual contacts, scientific collaboration and scientific citation.



We first determine the value of the homophily parameter in empirical networks. Established methods to quantify homophily include assortativity mixing \cite{newman2003mixing} and dyadicity \cite{park2007distribution}. These measures are to quantify the significance level of outgroup links compared to random expectation. However, real social networks do not necessarily exhibit symmetric homophilic behaviours. Observing only the edges between groups does not capture this potential asymmetric behaviour between groups. For example, if homophily among minorities (fraction size = $0.2$), is $0.1$ and for majorities is $0.7$, the assortativity by definition will be close to zero and similar to the case in which homophily is equal to $0.5$ for both groups. However, in this case we would expect that the number of edges that exist within the majority group is far greater than the number of edges within the minority group, after correcting for the group size. Therefore, to fully grasp asymmetric homophilic behavior, we need to consider the fraction of links that run between groups and within groups. Given the number of links that run between each group and the relation between group sizes, homophily, and degree exponent, we can analytically determine the homophily parameter for each group (see Methods and Fig.~\ref{fig:exponents_asym}).



%Results are shown in figure~\ref{fig:analytical_edges}. For simplicity, we fix the value of the homophily parameter in one group and show the relation between the tunable homophily and the fraction of edges for the other group. The dashed lines corresponds to the results of the analytical derivation, given a number of edges for each group. The value of the homophily parameter extracted from the simulations is shown by the dots.

%In the case of homophily fixed in one group to $0.5$ and same group size (panel left), we observe as expected a sigmoid function for both groups. For large value of homophily ($h_{aa}, (h_{bb}) = 1$), the fraction of edges between nodes of the same group converges to the size of the group. As the size of the minority decreases, the gap between the fraction of edges for minority and majority widen. Given the same parameters in the top panel, the majority gains an advantage in receiving links within itself partly because of the increase in their degree exponent and large group size differences. The bottom panel shows the results when the homophily is fixed to $0.2$. Results follow similar curves as for $h = 0.5$ and it shown minor differences in the fraction values for the intermediate range of homophily. Note that here we assume all member of the same group behave similarly. It would be interesting for future work to study heterogenous behavior inside a group. \gem{No top and bottom panel in the figure}

 

%First, we assess the homophily in real networks. We do this by evaluating the relation between homophily in our model and widely used measure of homophily known as assortativity mixing \cite{newman2003mixing}. Our evaluation confirms that there is linear relation between the homophily parameter ($h$) and assortativity ($r$), $r = 2 h -1$. This linear relation is well-consistent across population size of minorities and only show a slight deviation for small minority groups. This suggests that we can accurately assess the homophily parameter of empirical networks from one to one projection to its assortativity.





The analytical derivation enables us to accurately estimate the value of the homophily parameter in empirical networks by using only the number of edges within each group given our model parameters. We then focus on three examples of networks that exhibit high heterophily (sexual contacts), moderate homophily (scientific collaboration) and high homophily (scientific citation). We assume that all networks are undirected and we focus on one node attribute (e.g. gender or scientific field).

%Here we present three types of network: a strong heterophilic network of sexual contacts, a moderate homophilic network of scientific collaboration and a strong homophilic of scientific citation.

The first network captures sexual contacts between sex-workers and sex-buyers  \cite{rocha2011simulated}. The network consists of 16,730 nodes and 39,044 edges. There are 10,106 sex-workers and  6,624 sex-buyers (minority size $f_a = 0.4$). In this network, no edges among members of the same type exist and consequently the homophily parameter is equal to 0 for both groups, $h_{aa} = h_{bb} = 0$. 

%DBLP : min_min = 43655  maj_maj = 484445 , maj_min = 222501

%h_{aa} = 0.56
%h_{bb} = 0.57

The second network, which exhibits moderate homophily and relative group size difference, depicts scientific collaborations in computer science extracted from DBLP \cite{DBLP}. We used a new method that combines names and images to infer the gender of the scientists with high accuracy \cite{karimi2016inferring}. We focus on a 4-years snapshot of the network. After removing ambiguous names, the resulting network consists of 280,200 scientists and 750,601 edges (paper co-authorships) with 63,356 female scientists and 216,844 male scientists ($f_a = 0.23$).
We measure the homophily among women ($h_{aa} = 0.56$) and among men ($h_{bb} = 0.57$) and find a slight tendency for men to connect more among themselves than women.

%\kaf{I am now checking github dataset as another benchmark.}

The last network captures scientific citations in the American Physical Society (APS) corpus that exhibits strong homophily. Citation networks reveal how much attention communities around different topics attribute to each other. We use PACS identifier to select papers on the same topics. In this case we chose statistical physics, thermodynamics and nonlinear dynamical system sub-fields (PACS = 05). Within a specific sub-field there are many sub-topics that form communities of various sizes. To make the data comparable with our model, we choose two sub-topics that are relevant, namely classical statistical mechanics (CSM - 05.20.-y) and quantum statistical mechanics (QSM - 05.30.-d). 
%\wac{these 2 sub-fields belong to the three fields you mention before??? Or what happens with the three fields now? Do we need to mention them?}
The resulting network consist of 1,853 scientific papers and 3,627 citation links. The minority group in these two sub-topics is CSM ($f_a = 0.38$).
We find weaker homophily for the CSM papers ($h_{aa} = 0.8$) than for QSM papers ($h_{bb} = 1$), which indicates asymmetric, homophilic behavior in citation networks. 
%In other words, papers in QSM almost never cite papers from CSM, but CSM papers cite papers from QSM. %fariba: we can't say that since the data is not following our model perfectly


%To evaluate the model against the data, we first estimate the exponent of the degree distribution analytically by using the values of the homophily parameter estimated from the empirical observed edges within groups.

To evaluate our model against the data, we compare the exponent of the empirical degree distribution with the exponent generated from our model given the same empirical homophily and group size values. To estimate the exponent of the empirical degree distribution we use the maximum-likelihood fitting method \cite{clauset2009power,alstott2014powerlaw}. The exponent of the degree distribution generated from the model is calculated analytically (see Methods).

%\wac{I do not understand the previous sentence. Maybe better: To evaluate the model against the data, we compare the exponent of the empirical degree distribution with the exponent that our model produces. We first fit our model to the data by using the empirical observed group size differences and per-group homophily as input for the network generation model. Next we use the model to generate many networks and estimate the exponent of the degree distributions and its variation.} 

Figure \ref{fig:deg_dist_empirical} displays degree distribution of minorities and majorities in the three empirical networks. The exponent of the degree distributions of networks generated with our model agrees well with the empirical degree exponents using maximum-likelihood fit.

Similar to the previous section, we also examine the top nodes ranked by degree. Figure \ref{fig:topk_empirical} illustrates the probability of finding minorities in the top $d$\,\% of nodes ranked by degree. In the heterophilic case of the network of sexual contacts Fig.~\ref{fig:deg_dist_empirical}A, the minority is overrepresented with respect to its size.  In the
scientific collaboration Fig.~\ref{fig:topk_empirical}B in which homophily is moderate, the minority rank is close to its relative size. In the case of the scientific citation Fig.~\ref{fig:topk_empirical}C which is extremely homophilic, the representation of the minority is highly underestimated. We provide the results of the ranks in synthetic networks with similar homophilic parameters (dashed orange lines). Despite the simplicity of the model compare to the empirical data, the majority of ranks fall well within the standard deviation of the model. These results provide empirical evidence for a visibility bias in empirical networks and the usefulness of the model to capture biases. 

%\wac{within 1 or 2 standard deviations of the model? Or within  95\% confidence interval of the model? Maybe thats better. error range is a bit sloppy and sounds negative}. 
