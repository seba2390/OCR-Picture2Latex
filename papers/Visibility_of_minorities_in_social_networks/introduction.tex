\noindent 


\noindent Social networks are comprised of individuals with a variety of attributes, such as race, age, educational background, and gender. Commonly, these attributes are distributed unequally in the population. For example, in many schools across the United States and Europe, Asian or Blacks belong to a minority group \cite{moody2001race}, or women in science and engineering fields are a minority \cite{beede2011women}. In addition, homophily, the tendency to associate with similar others, is observed in many social networks, ranging from friendship to marriage to business partnerships \cite{mcpherson2001birds,moody2001race,baerveldt2004students,mislove2010you,fiore2005homophily}. One study has shown that in school friendships, Asians and Blacks are biased towards interacting with their own race at a rate >7 times higher than Whites and that homophily has a nonlinear relation with respect to relative group sizes \cite{currarini2010identifying}. %Given this empirical evidence, 
However, the extent to which homophilic behaviour combined with group size differences can put minority groups at a disadvantage by limiting their visibility in social networks is not well understood.

Understanding the factors that impact the visibility of minorities has gained importance in recent years since algorithms have been widely used for ranking individuals in various application domains, including search engines \cite{brin1998anatomy,kleinberg1999hubs,horowitz2010anatomy}, recommender systems \cite{zhou2012state,king2010introduction}, or hiring processes \cite{boyd2014networked,miller2015can,chalfin2016productivity}. These rankings are critical, since they can influence the visibility of individuals and the opportunities afforded to them. Rankings are commonly based on the topological structure of networks, and hence, the position of individuals in their social network significantly influences their visibility. In particular, in networks in which one group of individuals is smaller in size (minority), visibility can have a crucial impact on the representation of the whole group. 
This raises fundamental questions about the effects of group sizes and the different mechanisms of tie formation on the visibility of minorities in social networks. 
%Although many models in networks have utilized tie formation mechanisms, little is known about the effect of group sizes and homophily on the topology of the social networks and its impact on the visibility of minorities. 

%\kaf{the word topology is necessary in the above paragraph}



In this study, we utilize two main mechanisms for tie formation, homophily \cite{mcpherson2001birds}, and preferential attachment \cite{Barabasi99}, to systematically study how relative size differences between groups in social networks, with various levels of homophily, impact the visibility of nodes.  %We build on a variation of the Barab\'{a}si-Albert preferential attachment model which accounts for group sizes and different extents of homophily, ranging from extreme heterophilic to extreme homophilic behaviour. 
In recent years, models have been proposed that consider homophily \cite{bramoulle2012homophily,currarini2010identifying}, or a combination of homophily and preferential attachment in the tie formation process \cite{papadopoulos2012popularity,de2013scale,avin2015homophily}. We build on these models by systematically exploring the parameter range for homophily and group size differences to explain the emergent properties of networks and their impact on the visibility of minority and majority groups. We define \emph{visibility} as the  importance of the node in the network, which is commonly measured by the degree of connectivity. Our results (cf. Figure ~\ref{fig:fig1} top row for an illustration) show that the visibility of nodes in such settings is \emph{disproportionate}---i.e. visibility is not proportional to the size of the group and varies by homophily.
Since the formation of links in such networks is driven by preferential attachment and homophily, we find that majority nodes are more visible in homophilic networks than expected, whereas minority nodes are more visible in heterophilic networks. Surprisingly, visibility has an \emph{asymmetrical} and non-linear effect in both homophilic and heterophilic regimes. We provide an analytical solution that predicts the exponent of the degree distribution and demonstrate the presence of this asymmetric effect. We show evidence of a disproportionate visibility in three empirical networks (sexual contacts, scientific collaboration, and scientific citation) with different ranges of homophily and group size.


In the following sections, we show the analytical and numerical results of the effect of homophily and group sizes on the degree and visibility of nodes in social networks. We then discuss the impact of the parameters on the ranking of nodes that belong to different groups. Finally, we show that our model captures network properties, such as the degree distributions and ranks of the majority and minority in empirical social networks with different group sizes and different degrees of homophily. 




\begin{figure*}[]
\centering
\includegraphics[width=.99\linewidth]{fig1_final.pdf}
\caption{\textbf{Disproportionate visibility and asymmetric effects of homophily on Barab\'{a}si-Albert networks with minority and majority groups.} The minority group (orange nodes) represents 20\% of the population. 
Homophily is regulated by parameter $h$.
Panel A represents a maximally heterophilic network ($h=0$). As homophily increases, nodes prefer to connect with nodes of the same color. Panel E
represents a maximally homophilic network ($h=1.0$).  The top row is a schematic of the network topology generated from the model (Eq.~\ref{eq:homophilic_BA}) for a small network with 100 nodes. The second row represents the resulting degree growth over simulation time steps and the third row represents the degree distribution generated from the model for two types of nodes. The inset in the third row depicts the share of total degree for minority and majority groups and the dashed lines show the fraction size of the group. In the heterophilic regime ($ 0 \leq h <0.5$), the degree of the minority group grows faster than majority. In the homophilic regime ($0.5<h\leq 1$) the growth of the degree slows down for minorities.  
The network size for the second and third row are generated for $N = 5000$ nodes. The results are averaged over 20 simulations.  }

\label{fig:fig1}
\end{figure*}

