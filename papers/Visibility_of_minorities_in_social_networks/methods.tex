\section{Methods}

Here, we provide the analytical derivation of degree growth and the exponent of the degree distribution of the model. We do this using two approaches; exact derivation and continuum approximation (see Appendix).

\subsection{Exact degree dynamics}
Let $K_a(t)$ and $K_b(t)$ be the sum of the degrees of nodes from group $a$ and $b$ respectively. Since the overall growth of the network follows a Barab\'asi-Albert process, the evolution of these quantities verify:
\begin{equation}\label{eq:kakb}
    K_a(t) + K_b(t) = K(t) = 2mt
\end{equation}
where $m$ is the number of new links in the network at each time step $t$. Let us denote the relative fraction of group size for each group as $f_a$ and $f_b$. The evolution of $K_a$ and $K_b$ is given in discrete time by:

\small
\begin{widetext}
\begin{equation}
 \left\{
  \begin{array}{l}
    K_a(t+\Delta t) = K_a(t) + m\left(f_a\left(1 + \dfrac{h_{aa}K_a(t)}{h_{aa}K_a(t) + h_{ab}K_b(t)}\right) + f_b\dfrac{h_{ba}K_a(t)}{h_{bb}K_b(t) + h_{ba}K_a(t)}\right)\Delta t\\
    \\
    K_b(t+\Delta t) = K_b(t) + m\left(f_b\left(1 + \dfrac{h_{bb}K_b(t)}{h_{bb}K_b(t) + h_{ba}K_a(t)}\right) + f_a\dfrac{h_{ab}K_b(t)}{h_{aa}K_a(t) + h_{ab}K_b(t)}\right)\Delta t\\
  \end{array}
 \right.
\end{equation}
\normalsize
which in the limit $\Delta t \rightarrow 0$ gives:
\small
\begin{equation}\label{eq:evol_ka}
\left\{
  \begin{array}{l}
    \dfrac{dK_a}{dt} = m\left(f_a\left(1 + \dfrac{h_{aa}K_a(t)}{h_{aa}K_a(t) + h_{ab}K_b(t)}\right) + f_b\dfrac{h_{ba}K_a(t)}{h_{bb}K_b(t) + h_{ba}K_a(t)}\right)\\
    \\
    \dfrac{dK_b}{dt} = m\left(f_b\left(1 + \dfrac{h_{bb}K_b(t)}{h_{bb}K_b(t) + h_{ba}K_a(t)}\right) + f_a\dfrac{h_{ab}K_b(t)}{h_{aa}K_a(t) + h_{ab}K_b(t)}\right)\\
  \end{array}
\right.
\end{equation}
\end{widetext}

\normalsize

\begin{figure}[]
\centering
\includegraphics[width=0.8\linewidth]{exp.pdf}
\caption{\textbf{Evolution of the exponents for the degree growth, symmetrical homophily.} The exponents $\beta_a$ (minority) and $\beta_b$ (majority) are defined in eqs.~(\ref{eq:ca}) and (\ref{eq:cb}). $h = h_{aa} = h_{bb}$ is the homophily parameter and the numbers indicate the fraction of nodes belonging to the minority group (parameter $f_a$).}
\label{fig:exponents}
\end{figure}

These equations verify that for $h_{aa} = h_{bb} = 0$ and $h_{ab} = h_{ba} = 1$ (perfectly heterophilic network) we get:
\small
\begin{equation}
\left\{
  \begin{array}{l}
    \dfrac{dK_a}{dt} = m\\
    \\
    \dfrac{dK_b}{dt} = m\\
  \end{array}
\right.
\end{equation}
\normalsize
and thus for the evolution of the degree of a single node:
\small
\begin{equation}
\left\{
  \begin{array}{l}
    \dfrac{dk_a}{dt} = mf_b\dfrac{k_a}{\sum_i q_ik_i} = mf_b\dfrac{k_a}{K_b(t)} = f_b\dfrac{k_a}{t}\\
    \\
    \dfrac{dk_b}{dt} = mf_a\dfrac{k_b}{\sum_i q_ik_i} = mf_a\dfrac{k_b}{K_a(t)} = f_a\dfrac{k_b}{t}\\
  \end{array}
\right.
\end{equation}
\normalsize
which gives:
\begin{equation}
\left\{
  \begin{array}{l}
    k_a \propto t^{f_b}\\
    k_b \propto t^{f_a}\\
  \end{array}
\right.
\end{equation}
Similarly, for $h_{aa} = h_{bb} = 1$ and $h_{ab} = h_{ba} = 0$ (perfectly homophilic network) we get:
\small
\begin{equation}
\left\{
  \begin{array}{l}
    \dfrac{dK_a}{dt} = 2mf_a\\
    \\
    \dfrac{dK_b}{dt} = 2mf_b\\
  \end{array}
\right.
\end{equation}
\normalsize
and thus for the evolution of the degree of a single node:
\small
\begin{equation}
\left\{
  \begin{array}{l}
    \dfrac{dk_a}{dt} = mf_a\dfrac{k_a}{\sum_i q_ik_i} = mf_a\dfrac{k_a}{K_a(t)} = \dfrac{k_a}{2t}\\
    \\
    \dfrac{dk_b}{dt} = mf_b\dfrac{k_b}{\sum_i q_ik_i} = mf_b\dfrac{k_b}{K_b(t)} = \dfrac{k_b}{2t}\\
  \end{array}
\right.
\end{equation}
\normalsize
which gives:
\begin{equation}
\left\{
  \begin{array}{l}
    k_a \propto t^{1/2}\\
    k_b \propto t^{1/2}\\
  \end{array}
\right.
\end{equation}
\normalsize

Let's make the hypothesis that $K_a(t)$ and $K_b(t)$ are linear functions of time, so that $K_a(t) = Cmt$ and $K_b(t) = (2-C)mt$ given Eq.~(\ref{eq:kakb}). Using Eq.~(\ref{eq:evol_ka}), we thus have:
\small
\begin{widetext}
\begin{equation}
    \dfrac{dK_a}{dt} = Cm = m\left(f\left(1 + \dfrac{h_{aa}Cmt}{h_{aa}Cmt + h_{ab}(2mt - Cmt)}\right) + (1-f)\dfrac{h_{ba}Cmt}{h_{bb}(2mt - Cmt) + h_{ba}Cmt}\right)
\end{equation}
\normalsize
which can be rewritten as:
\begin{equation}\label{eq:C}
\begin{split}
  (h_{aa} - h_{ab})(h_{ba} - h_{bb})C^3 \\
  + ((2h_{bb} - (1-f)h_{ba})(h_{aa} - h_{ab}) + (2h_{ab} - f(2h_{aa} - h_{ab}))(h_{ba} - h_{bb}))C^2 \\
  + (2h_{bb}(2h_{ab} - f(2h_{aa} - h_{ab})) - 2fh_{ab}(h_{ba} - h_{bb}) - 2(1-f)h_{ba}h_{ab})C \\
  - 4fh_{ab}h_{bb} = 0
\end{split}
\end{equation}
\end{widetext}



\begin{figure*}[]
\centering
\includegraphics[width=0.23\linewidth]{exp_0_1_b.pdf}
\includegraphics[width=0.23\linewidth]{exp_0_2_b.pdf}
\includegraphics[width=0.23\linewidth]{exp_0_3_b.pdf}
\includegraphics[width=0.23\linewidth]{exp_0_4_b.pdf}

\includegraphics[width=0.23\linewidth]{exp_0_1_a.pdf}
\includegraphics[width=0.23\linewidth]{exp_0_2_a.pdf}
\includegraphics[width=0.23\linewidth]{exp_0_3_a.pdf}
\includegraphics[width=0.23\linewidth]{exp_0_4_a.pdf}
\caption{\textbf{Evolution of the exponents for the degree growth, asymmetrical homophily.} The exponents $\beta_a$ and $\beta_b$ are defined in eqs.~(\ref{eq:ca}) and (\ref{eq:cb}). $h_{aa}$ and $h_{bb}$ are the homophily parameters. Bottom row shows the behaviour of $\beta_a$ and top row the behaviour of $\beta_b$. Columns are ordered according to the fraction of nodes belonging to the majority group (parameter $f_b$), respectively $f_a = 0.1$, 0.2, 0.3 and 0.4 from left to right. The dashed red lines indicate the symmetrical case plotted in Fig.~\ref{fig:exponents}.}
\label{fig:exponents_asym}
\end{figure*}


This equation for $C$ can be numerically solved. Within the ranges of values of the parameters, it has three real solutions, but only one in the interval $[0,2]$ and thus valid in this case. We can then derive the evolution of the degree of a single node for both groups in the general case. Let's define:
\small
\begin{equation}
\begin{split}
  Y_a(t) &= h_{aa}K_a(t) + h_{ab}K_b(t)\\
  &= h_{aa}Cmt + h_{ab}(2-C)mt\\
  &= mt(h_{aa}C + h_{ab}(2-C))\\
\end{split}
\end{equation}
and
\begin{equation}
\begin{split}
  Y_b(t) &= h_{ba}K_a(t) + h_{bb}K_b(t)\\
  &= h_{ba}Cmt + h_{bb}(2-C)mt\\
  &= mt(h_{ba}C + h_{bb}(2-C))\\
\end{split}
\end{equation}
\normalsize
For group $a$, we have:

\begin{equation}\label{eq:ca}
\begin{split}
  \dfrac{dk_a}{dt} &= mf_a\dfrac{h_{aa}k_a}{Y_a} + mf_b\dfrac{h_{ba}k_a}{Y_b}\\
  &= \dfrac{k_a}{t}\left(\dfrac{f_ah_{aa}}{h_{aa}C + h_{ab}(2-C)} + \dfrac{f_bh_{ba}}{h_{ba}C + h_{bb}(2-C)}\right)\\
  &= \dfrac{k_a}{t}\beta_a\\
\end{split}
\end{equation}
and thus:
\begin{equation}
  k_a(t) \propto t^{\beta_a}\\
\end{equation}
Similarly, for group $b$ we have:
\begin{equation}\label{eq:cb}
\begin{split}
  \dfrac{dk_b}{dt} &= mf_b\dfrac{h_{bb}k_b}{Y_b} + mf_a\dfrac{h_{ab}k_b}{Y_a}\\
  &= \dfrac{k_b}{t}\left(\dfrac{f_bh_{bb}}{h_{ba}C + h_{bb}(2-C)} + \dfrac{f_ah_{ab}}{h_{aa}C + h_{ab}(2-C)}\right)\\
  &= \dfrac{k_b}{t}\beta_b\\
\end{split}
\end{equation}
and thus:
\begin{equation}
  k_b(t) \propto t^{\beta_b}
\end{equation}

We plot the evolution of these exponents $\beta_a$ and $\beta_b$ in the special case where $h_{aa} = h_{bb} = h$ and $h_{ab} = h_{ba} = 1-h$ (Fig.~\ref{fig:exponents}). The general case where homophily is not symmetrical is shown in the contour plot in figure \ref{fig:exponents_asym}. The dashed red lines indicate the previous case of symmetric homophily.  





%\mst{I would suggest to add grey dashed lines (horiz/vert) where haa = hbb = 0.5, this helps reading the plot}

Finally, as has been shown before, there is an inverse relation between the exponent of the degree growth and the exponent of the degree distribution ($p(k) \propto k^\gamma$), as follow \cite{Barabasi99,bianconi2001competition} :

\begin{equation}
\label{eq:degree_exponent_SI}
    \gamma =\frac{1}{\beta} + 1
\end{equation}

In the case where homophily is equal to 0.5 for both groups, we have  $\beta_{a} = \beta_{b} = 0.5$, in which the model converges to classic BA model with degree exponent $p(k) \propto k^{-3}$.  

\subsection{Estimating asymmetric homophily parameters}

\begin{figure*}[]
\centering
\includegraphics[width=0.255\linewidth]{edges_f_5_h_5_compare.pdf}
\includegraphics[width=0.24\linewidth]{edges_f_4_h_5_compare.pdf}
\includegraphics[width=0.24\linewidth]{edges_f_3_h_5_compare.pdf}
\includegraphics[width=0.24\linewidth]{edges_f_2_h_5_compare.pdf}

\caption{\textbf{Analytical and numerical estimation of the fraction of edges that run within each group of nodes versus homophily.} Fractions of edges within each group are denoted by $m_{aa}$ and $m_{bb}$. The homophily parameter is tuned for one group and fixed for another group.  Panels from left to right are generated for various minority sizes. The numerical results are shown by points in the plot. The analytical results are shown by dashed lines. As the size of the minority decreases, the gap between the fraction of edges for minority (orange lines when majority homophily is fixed ($h_{bb} = 0.5$)) and majority (blue lines when minority homophily is fixed ($h_{aa} = 0.5$)) widen.
The analytical results are derived by estimating expected homophily from number of edges and they are in excellent agreement with the numerical results. }
\label{fig:analytical_edges}
\end{figure*}


The analytical derivations in the previous section enable us to estimate the homophily parameter given the fraction of edges that exist within each group in empirical networks. %This has an important application since estimating asymmetric homophily in empirical networks is challenging. 

%Here, we show analytically and numerically that given a number of edges within each group, it is possible to precisely estimate the value of the homophily parameter for each group, within the framework of our model. 

In a network with $M$ number of edges, let's assume $M_{aa}$ is the number of edges linking two nodes of the group $a$  (ingroup links) and similarly $M_{bb}$ is the number of edges linking nodes of the group $b$. The probability to have an ingroup link in group $a$ can then be defined as $m_{aa} = \frac{M_{aa}}{M}$, which depends on the group size $f_a$, the homophily parameter $h_{aa}$ and the relative degree growth exponents $\beta_{a}$ and $\beta_{b}$:
\begin{equation}
m_{aa} = \frac{f_{a}^2 h_{aa}\alpha_{a}}{f_{a}^2 h_{aa}\alpha_{a} + f_{a} f_{b} h_{ab}\alpha_{b}}
\end{equation}

A similar formula can be written for the group $b$:
\begin{equation}
m_{bb} = \frac{f_{b}^2 h_{bb}\alpha_{b}}{f_{b}^2 h_{bb}\alpha_{b} + f_{b} f_{a} h_{ba}\alpha_{a}}
\end{equation}

where $\alpha_{a} = \frac{\beta_{a}}{\beta_{a} + \beta_{b}}$ and $\alpha_{b} = \frac{\beta_{b}}{\beta_{a} + \beta_{b}}$ are relative degree exponents for each group. 

Note that in the general case homophily can be asymmetric, $h_{ab} \neq h_{ba}$. From our previous analytical calculations, we know the relation between the exponent $\beta$, the group size and the homophily parameter by numerically solving equation \ref{eq:C} given $m_{aa}$ and $m_{bb}$. We can then solve these nonlinear dynamical equations and determine the expected homophily $h_{aa}$ and $h_{bb}$ for group $a$ and $b$. 

Results are shown in figure~\ref{fig:analytical_edges}. For simplicity, we fix the value of the homophily parameter in one group and show the relation between tunable homophily and the fraction of edges for the other group. The dashed lines corresponds to the results of the analytical derivation, given a number of edges for each group. The value of the homophily parameter extracted from the simulations is shown by the dots.
In the case of homophily fixed for one group at $0.5$ and same group size (panel left), we observe as expected a sigmoid function for both groups. For large value of homophily ($h_{aa}, h_{bb} = 1$), the fraction of edges between nodes of the same group converges to the size of the group. As the size of the minority decreases, the gap between the fraction of edges for the minority (orange lines when the majority homophily is fixed ($h_{bb} = 0.5$)) and the majority (blue lines when the minority homophily is fixed ($h_{aa} = 0.5$)) widen. By tuning the group size and fixing the homophily parameter for minorities, the majority gains an advantage by receiving links within itself partly because of the increase in their degree exponent and large group size differences (blue lines). 

%Note that here we assume all member of the same group behave similarly. It would be interesting for future work to study heterogenous behavior inside a group. 



