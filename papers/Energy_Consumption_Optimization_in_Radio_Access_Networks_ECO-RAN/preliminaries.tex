\section{Preliminaries}

\subsection{Mobile Networks}


% \subsubsection{Terminology}
% \begin{itemize}
% \item \textbf{GSM:} standard that describe 2G protocols.
% \item \textbf{UMTS:} 3G network standard, based on GSM.
% \item \textbf{LTE:} 4G network standard, based on GSM and UMTS.
% \item \textbf{Site:} Fixed physical location where base station is installed.
% \item \textbf{Base (transceiver) station (BS):} Fixed physical station serving as link between user device and carrier's network (3G terminology).
% \item \textbf{eNodeB:} Evolved Node B. LTE version of base station.
% \item \textbf{Sector:} Collection of cells covering the same direction (given as an angle).
% \item \textbf{Cell:} cellular antenna.
% \item \textbf{Capacity:} Amount of traffic that the network can handle at any given time.
% \item \textbf{Coverage:} Physical area covered by network. Inside area, phones will be able to complete a call.
% \item \textbf{KPI:} Key Performance Indicator.
% \item \textbf{UE:} User equipment.
% \item \textbf{PRB:} Physical resource block. Smallest time/frequency unit. 
% \item \textbf{OFDMA:} Orthogonal frequency-division multiple access.
% \item \textbf{SC-FDMA:} Single carrier frequency division multiple access.
% \item \textbf{OTDOA:} Observed time difference of arrival.
% \end{itemize}
In this work we consider some geographical area where there is a
number of \emph{base stations}. Base stations have number of
\emph{cells} and every \emph{cell} operate in some \emph{frequencies}.
The geographical location is discredited using \emph{pixels}. Cells
provide coverage to pixels and each pixel has a traffic demand.
Figure~\ref{fig:networkmap} shows a map with the building of the
Department of Computer Science at Aalborg University. Base stations
are in pink, every base station contain some cells, and pixels
correspond to the grid elements.


\subsubsection{Frequency Layers}
Each base station usually consists of a number of cells broadcasting
on different frequencies. Lower frequencies are for coverage while
higher frequencies are for capacity. For 4G, the 800 MHz frequency
layer is considered the coverage layer and must not be turned off in
the current setup. There is room for optimization at the higher
frequency layers as the needed capacity fluctuates a lot during a
typical day.  The 4G (LTE) frequencies are:
%\begin{itemize}
%\item
  E: 800 MHz,
%\item
  V: 900 MHz,
% \item
  T: 1800 MHz,
%\item
  A: 2100 MHz,
%\item
  L: 2600 MHz. 
%\end{itemize}

  \begin{figure}[t]
    \centering
    \includegraphics[width=\textwidth]{networkMap.png}
    \caption{Base station with three sectors. Sector 1 consists of one 800 cell and one 1800 cell.}
    \label{fig:networkmap}
  \end{figure}

  
\subsubsection{Power Saving}

The goal is to shut down capacity layers then they are not
needed. e.g.\ during the night. A current contraint of the system is
to maintain coverage to all pixels. In the current system, to ensure this constraint the 800
MHz layer can not be shut down.

  % \begin{figure}[b]
  %   \label{basestation}
  %   \centering
  %   \includegraphics[scale=1]{base_station.png}
  %   \caption{Base station with three sectors. Sector 1 consists of one 800 cell and one 1800 cell.}
  % \end{figure}


\subsubsection{Historical Data}

The company 2Solve has relevant historical data, e.g.\ the traffic
demands for every base station. There is also information about the
signal strengh for every pixel and for every frequency layer. Our
simulations will be based on existing historical data.


%%% Local Variables:
%%% mode: latex
%%% TeX-master: "main"
%%% End:
