\section{Introduction}

In accordance with the enormous expansion of mobile networks in
Denmark and the rest of the world, the number of mobile masts
providing coverage has exploded, and with the upcoming expansion of
5G, there will be even more mobile radio devices that require power.

In recent years, it has been in the interest of the mobile operators
to bring the power consumption, and the first steps have already been
taken.  These measures are based on semi-automatic procedures and with
strong assumptions e.g.\ everyone follows the same patterns.
%
A more fully automated approach to the problem, based on artificial
intelligence, is desirable and expected to be able to further reduce
power consumption.
%
Furthermore, in connection with the sales activities, both in and
outside Europe, it has been made clear that the mobile operators are
increasingly concerned about mobile network power consumption, now and
especially in the future.  The background for this is that electricity
consumption in the mobile network will increase significantly with the
introduction of 5G and several 4G frequency layers.  There is
therefore both a considerable financial gain by minimizing power
consumption, and also a growing interest in contributing to the Danish
climate action, where the goal is for Denmark to be energy-renewable
in 2050.
%

There are thus already good market leadership advantages for companies
that can demonstrate that they are actively making an effort to
achieve this goal.  Together with one of our partners 2Operate we
carried out feasibility studies that show that some Nokia-specific
functions in Nokia's operating system can force the radio units to
switch off at certain times -- e.g.\ at 01:00-06:00. That is it is
possible to synthesize and implement schedulers which turn on and off
power cells.

Conservative calculations show that the savings potential will be €100
– €300 annually per mobile mast per company. In Denmark, they have
fire mobile operators together approx. 10,000 locations with their
equipment.  This gives a total savings potential of DKK 1 - 3
million. euros annually or up to 50 million KWh. Worldwide, this
potential will be many times higher. 


%%% Local Variables:
%%% mode: latex
%%% TeX-master: "main"
%%% End:
