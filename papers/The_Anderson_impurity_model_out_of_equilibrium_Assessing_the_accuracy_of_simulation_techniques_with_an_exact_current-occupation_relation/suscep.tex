% !TEX TS-program = pdflatex
%\documentclass[preprint,showpacs,singlecolumn,superscriptaddress]{revtex4-1}
%\documentclass[preprint,groupedaddress,showpacs,superscriptaddress,amssymb,amsmath]{revtex4-1}
%\documentclass[preprint,showpacs,sup
%erscriptaddress, prb,preprint,amssymb,amsmath]{revtex4-1}
%\documentclass[aps,preprint,groupedaddress,showpacs,superscriptaddress,prb,amssymb,amsmath]{revtex4-1}
\documentclass[aps,pra,twocolumn,groupedaddress,showpacs,superscriptaddress,amssymb,amsmath]{revtex4-1}
\usepackage{graphicx}
%\usepackage{caption}
\usepackage{tabularx}
\usepackage{color}
\usepackage{amsmath}
\usepackage{comment}
%\usepackage[export]{adjustbox}
%\usepackage{placeins}
\newcommand{\be}{\begin{equation}}
\newcommand{\ee}{\end{equation}}
\newcommand{\bea}{\begin{eqnarray}}
\newcommand{\eea}{\end{eqnarray}}
\usepackage{dcolumn}
\usepackage{hyperref}
\usepackage{bm}
\usepackage{epsf}
\usepackage{subfigure}
\usepackage{epstopdf}%
%\usepackage[colorlinks,citecolor=blue,linkcolor=red]{hyperref}
\setcounter{MaxMatrixCols}{30}%
\usepackage{amsfonts}
\newcommand{\mb}[1]{\mbox{\boldmath $#1$}}
\newcommand{\mathbs}[1]{\mbox{\boldmath$#1$}}
\bibliographystyle{apsrev}
\newcommand{\epl}{\texttt{epl}}
\def\vrulefill{\leaders\vrule\vfill}

\DeclareMathOperator{\trace}{Tr}

\def\tp{{t^\prime}}
\def\ap{{a^\prime}}
\def\bp{{b^\prime}}
\def\ip{{i^\prime}}
\def\jp{{j^\prime}}
\def\kp{{k^\prime}}
\def\lp{{l^\prime}}
%\def\mp{{m^\prime}}
\def\np{{n^\prime}}
\def\sp{{s^\prime}}
%\def\ll{{\langle\langle}}


\begin{document}
%HARE KRISHNA
%\title{Charge susceptibility and dynamical conductance for nonequilibrium systems}
%\title{Assessment of the accuracy of numerical schemes based on exact relations}
%Assessment of the consistency of approximate }
%Assessing the  accuracy of numerical methods for simulating transport in the Anderson model with

% Nonequilibrium
\title
{The Anderson impurity model out-of-equilibrium: 
Assessing the accuracy of simulation techniques with an exact current-occupation relation}



\author{Bijay Kumar Agarwalla}
\affiliation{Chemical Physics Theory Group, Department of Chemistry, and
Centre for Quantum Information and Quantum Control,
University of Toronto, 80 Saint George St., Toronto, Ontario, Canada M5S 3H6}
\author{Dvira Segal}
\affiliation{Chemical Physics Theory Group, Department of Chemistry, and
Centre for Quantum Information and Quantum Control,
University of Toronto, 80 Saint George St., Toronto, Ontario, Canada M5S 3H6}

\date{\today}

\begin{abstract}
%
We study the interacting, symmetrically coupled single impurity Anderson model.
By employing the nonequilibrium Green's function formalism,
we establish an exact relationship between the 
steady-state charge current flowing through the impurity (dot) and its occupation.
%
We argue that the steady state current-occupation relation can be used to assess the consistency 
of simulation techniques, and identify spurious transport phenomena.
We test this relation in two different model variants:
First, we study the Anderson-Holstein model in the strong electron-vibration coupling limit 
using the polaronic quantum master equation method. 
We find that the current-occupation relation
is violated numerically in standard calculations, with simulations bringing up incorrect
transport effects. Using a numerical procedure, we resolve the problem efficiently.
%
Second, we simulate the Anderson model with electron-electron interaction on the dot
using a deterministic numerically-exact time-evolution scheme. 
Here, we observe that the current-occupation relation is satisfied in the steady-state limit---even before results
converge to the exact limit.
%
%This work demonstrates that
%Validating the steady state current-occupation relation is thus beneficial for
%establishing the consistency and accuracy of simulation techniques, 
%and for identifying spurious transport effects. 
%
\end{abstract}
\maketitle

%=====================================================
\section{Introduction}
\label{Sintro}

% history + define model
The single impurity Anderson model \cite{Anderson} plays a central role 
in many-body condensed-phases physics. While it was originally introduced for describing the
thermodynamics of magnetic impurities in non-magnetic metals \cite{Kondo}, it has
received significant attention more recently in the context of transport at the nanoscale \cite{Galperin-review}.
In the standard nonequilibrium setup, a single electronic level (impurity, dot) with two interacting electrons
is coupled to two voltage-biased electronic reservoirs (metals).
This simple scenario can be made more involved by including the interaction of electrons with 
local vibrations (``Anderson-Holstein model"), and by considering multiple electronic states \cite{Galperin-review}.


% concrete applications
The Anderson impurity model had greatly contributed to the 
understanding of charge and energy transport mechanisms        
in nanoscale systems, quantum dots and molecular junctions, by
revealing rich dynamical and steady state behavior, see e.g.  Ref. \cite{GalperinS}.
%In particular, %referring to recent experiments,
%the SIAM model was proven to be useful in predicting numerous
%aspects of charge and energy transport characteristics in correlated nanoscale systems, e.g., quantum dots or molecular junctions.
Moreover, the model serves as a playground for developing and benchmarking 
approximation schemes and simulation techniques; %for descibing transport in many-body impurity systems.
a very partial list includes 
renormalization group techniques \cite{Wilson, RG}, Green's function approaches \cite{MW93,Galperin-review},
the multilayer multiconfiguration time-dependent Hartree method \cite{Wang1,Wang2,Wang3,Rabani14}, 
Hierarchical equations of motion \cite{HEOM1,HEOM2,HEOM3},
Quantum Monte Carlo \cite{Lothar,Coheninch,CohenHT} and influence-functional path integral tools \cite{egger1,egger2,IF1,IF2}.

% universal relations 
Beyond the study of nonlinear transport characteristics, the search for universal relations between 
different nonequilibrium observables has been central to nonequilibrium physics. 
For example, linear irreversible thermodynamics is summarized in terms of the universal Green-Kubo relations, 
with the conductance expressed in terms of an equilibrium density correlation function. 
Onsager's reciprocity theorem relates cross-transport coefficients. 
Moreover, in the past two decades various universal relations have been discovered for out-of-equilibrium systems,
now collectively referred to as ``fluctuation symmetries" \cite{fluc1,fluc2}. 
%This relations quantify the second law of thermodynamics for small scale systems.  

%paper by NESS and Dash
%For such charge conducting setups, 
%recently, an attempt was made to understand the relationship between 
%charge susceptibility and nonlinear dynamical conductance \cite{Ness} at finite bias.
%Performing numerical simulations, the authors claim that these two quantities are directly
%related to each other but in a different way than in the equilibrium case.  

In this paper, we discuss an exact relation between the steady state charge current and the average 
charge occupation for the interacting, symmetrically coupled single impurity model. % coupled symmetrically to fermionic leads. 
We argue that this relationship can be used to assess the consistency and accuracy of approximation techniques,
and in particular identify spurious transport effects.
We test this current-occupation relationship in two models: 
(i) the single-electron Anderson-Holstein model, by using a polaronic quantum master equation (QME), 
and (ii) the electron-electron interacting Anderson model, by employing a numerically exact time evolution scheme.
As we show below, in both cases the violation/verification of the 
current-occupation relationship reveals important information 
on the performance (consistency, accuracy) of simulation techniques.

% Here
The paper is organized as follows. 
In Sec.~\ref{SModel}, we present the model Hamiltonian and derive the exact relation between 
the charge current and the average occupation on the dot. 
We test this relation in Sec. \ref{SAH}, 
for the Anderson-Holstein model, and in Sec. \ref{SINFPI}, for the case with electron interaction on the dot.
We conclude in Sec. \ref{Ssum}.

%=======================================

\section{Exact relation between current and occupation}
\label{SModel}

We are interested in the steady state behavior of the Anderson impurity model. The total Hamiltonian
$H=H_0+H_n$ includes the noninteracting (quadratic) term $H_0$ and the many-body 
(nonlinear) contribution $H_n$. Here,
%
\be
H_0= H_{d0} + H_L + H_R + H_{dL} + H_{dR},
\ee 
%
where  $H_{d0}=\epsilon_d d^{\dagger}d$ comprises a single electronic site (dot) with
creation and annihilation operators $d^{\dagger}$, $d$, respectively.
The dot is placed in contact with two non-interacting fermionic environments (metal leads), $\alpha=L,R$, 
$H_{\alpha}= \sum_{k} \epsilon_{k \alpha} c^{\dagger}_{k \alpha} c_{k \alpha}$, 
where $k$ is the index for the electron momentum
and $c_{k \alpha}^{\dagger}$ ($c_{k \alpha})$ is the creation (annihilation) operator in the lead $\alpha$. 
Electron hopping between the dot and the leads is described by the standard tunnelling Hamiltonian
$H_{d \alpha}= \sum_{k} v_{k \alpha} c_{k\alpha}^{\dagger} d + h.c.$, 
with $v_{k\alpha}$ as the metal-dot coupling energy.
%
$H_n$ includes many-body interactions---assumed to affect or
couple to the $d$, $d^{\dagger}$ operators. 
For example, $H_n$ may comprise electron-electron (e-e) repulsion interaction on the dot (generalizing then
$H_0$ to include two spin species). As well, $H_n$ may
collect additional degrees of freedom, phonons and photons,
and their interaction with electrons on the dot.

The two metal leads are maintained out of equilibrium by introducing different chemical potentials on the leads, 
$\mu_L\neq \mu_R$, and different temperatures, $T_L\neq T_R$ (in our simulations below we assume equal temperatures).
The resulting steady state charge current, say from the $L$ lead towards the dot, 
is given by the celebrated Meir-Wingreen formula  \cite{meir-wingreen,Haug-book} (we assume $\hbar=1$),
%
\be
\langle I_L \rangle = e \int_{-\infty}^{\infty} \frac{d\epsilon}{2 \pi} \, 
\big[G_d^{>}(\epsilon) \,\Sigma_L^{<}(\epsilon) - G_d^{<}(\epsilon) \, \Sigma_L^{>}(\epsilon) \big].
\label{eq:MW}
\ee 
%
This formal expression holds for an arbitrarily-interacting impurity. 
Here, the principal object is the one-electron interacting Green's function
$G_d(\tau, \tau')=-i\langle {\rm T_c}d(\tau)d^{\dagger}(\tau') \rangle$, defined
on the Keldysh contour  with $T_c$ as the contour-ordering operator
\cite{Haug-book, Rammer_review, Rammer_book}.
%
Projecting to real time, then to the energy  domain,
$G_{d}^{</>}(\epsilon)$ are the lesser and greater components of the one-electron Green's function.
$\Sigma_{L,R}^{</>}(\epsilon)$ are the corresponding self-energy terms, 
responsible for electron transfer in and out of the central system to the $L$ and $R$ metal leads.
These terms can be analytically obtained \cite{Haug-book, Rammer_review, Rammer_book},
%
\bea
\Sigma_{\alpha}^{<}(\epsilon) = i f_{\alpha}(\epsilon) \Gamma_{\alpha}(\epsilon),  \,\,\,\,
\Sigma_{\alpha}^{>}(\epsilon) = -i \big(1- f_{\alpha}(\epsilon)\big) \Gamma_{\alpha}(\epsilon),
\nonumber\\
\eea
%
with $\Gamma_{\alpha}(\epsilon)= 2 \pi \sum_{k} |v_{k\alpha }|^2 \delta (\epsilon-\epsilon_{k\alpha})$ as the impurity-lead
hybridization energy.
$f_{\alpha}(\epsilon)=1/[e^{\beta_{\alpha} (\epsilon-\mu_{\alpha})}+1]$ is the Fermi-Dirac 
distribution function for the $\alpha$ lead, $\beta_{\alpha}=1/T_{\alpha}$ is the inverse temperature, $k_B\equiv 1$.

From here on we assume that the impurity is symmetrically coupled to the two leads, 
$\Gamma(\epsilon)\equiv \Gamma_{\alpha}(\epsilon)$. 
Eq.~(\ref{eq:MW}) then reduces to the simple form
%
\be
\langle I_L \rangle  = e \int_{-\infty}^{\infty} \frac{d\epsilon}{4 \pi } \, \big[f_L(\epsilon)-f_R(\epsilon)\big] \,\Gamma(\epsilon) \,A(\epsilon; \{\mu_{\alpha},T_{\alpha}\}),
\label{eq:Lan-cur}
\ee
%
with
%
\bea
A(\epsilon; \{\mu_{\alpha},T_{\alpha}\}) = i \left[G_d^r(\epsilon)-G_d^a(\epsilon)\right],
\eea
%
as the so-called (nonequilibrium) spectral density function of the dot. 
Here, $G^{r/a}_d(\epsilon)$ are the retarded ($r$) and advanced ($a$) Green's functions. 
In general, for an interacting system, the spectral function depends on the electronic-bath parameters,
namely, the chemical potential $\mu_{\alpha}$ and the temperature $T_{\alpha}$, in a non-trivial way. 
%
The causality condition for the retarded and the advanced  Green's functions ensures the 
sum-rule for the spectral function, 
$\int _{-\infty}^{\infty} \frac{d\epsilon}{2\pi}  A(\epsilon; \{\mu_{\alpha},T_{\alpha}\})=1$.

A formal expression for $A(\epsilon; \{\mu_{\alpha},T_{\alpha}\})$ can be organized from the Dyson's equation 
\cite{bijay-wang-review, Haug-book, Rammer_review, Rammer_book} 
for the  interacting Green's function,
%
\be
G_d^{r/a}(\epsilon) = G_{0,d}^{r/a}(\epsilon) + G_{0,d}^{r/a}(\epsilon)\, \Sigma_n^{r/a}(\epsilon)\,G_d^{r/a}(\epsilon).
\label{dyson-eq}
\ee
%
Here, $G_{0,d}^{r/a}(\epsilon)$ is the Green's function for the dot, evaluated with the 
quadratic part of the Hamiltonian, $H_0$.
$\Sigma_n^{r/a}(\epsilon)$ is the nonlinear self-energy component, 
arising from many-body interactions on the dot.
Alternatively, we can write Eq.~(\ref{dyson-eq}) as 
%
\be
\Big[G_d^{r/a}(\epsilon)\Big]^{-1} = \Big[G_{0,d}^{r/a}(\epsilon)\Big]^{-1} - \Sigma_n^{r/a}(\epsilon).
\ee
%
This form allows us to write down the following expression for the spectral function,
%
\be
 A(\epsilon; \{\mu_{\alpha},T_{\alpha}\}) = 2 \, G_d^r(\epsilon) 
\left[ \Gamma(\epsilon) + \Gamma_n(\epsilon) /2 \right] \, G_d^a(\epsilon),
\label{eq:spectral}
\ee
%
with the many-body self energy
%
\bea
\Gamma_n(\epsilon) \equiv i \left[\Sigma_n^r(\epsilon) -\Sigma_n^a(\epsilon)\right]= 
i \left[\Sigma_n^>(\epsilon) -\Sigma_n^<(\epsilon)\right].
\eea
%
This function is responsible for broadening the spectral function of electrons---on top of the 
broadening from the metal-impurity hybridization.
Note that, in general, temperature and bias may affect all components  of the interacting Green's function.

Next, we write down an expression for the averaged electronic population on the dot. 
It can be related to the lesser component of the interacting Green's function, i.e.,
%
\bea
\langle n_d \rangle &\equiv& \langle d^{\dagger} d \rangle = -i \int_{-\infty}^{\infty} \frac{d\epsilon}{2 \pi} G_d^<(\epsilon) \nonumber \\
&=&-i \int_{-\infty}^{\infty}  \,\frac{d\epsilon}{2 \pi} \, G_d^r(\epsilon) \Sigma_{tot}^{<}(\epsilon) G_d^a(\epsilon).
\label{eq:navg}
\eea
%
The total self energy
$\Sigma_{tot}^{<}(\epsilon)= \Sigma_L^{<}(\epsilon)+ \Sigma_R^{<}(\epsilon)+\Sigma_n^{<}(\epsilon)$ 
includes contributions from the leads and from many-body interactions on the dot.
In the second line of the above equation we employ the Keldysh equation 
\cite{Kadanoff, Keldysh, book1,bijay-wang-review} for the lesser component $G_d^{<}(\epsilon)$.

% DDD
Taking the metal-molecule hybridization $\Gamma$ to be a constant, independent of energy,
simple algebraic manipulations of Eqs.~(\ref{eq:Lan-cur}) and (\ref{eq:navg}) result in
the following relation between the charge current and the average dot occupation,
%
\begin{widetext}
\be
\langle I_L \rangle =
e \Gamma \Big[\langle n_d\rangle
-\int_{-\infty}^{\infty} \frac{d\epsilon}{2\pi } 
\big[ G_d^r(\epsilon) \, \Big(-i \Sigma_n^{<}(\epsilon) - \frac{f_L(\epsilon) + f_R(\epsilon)}{2} \Gamma_n(\epsilon) \Big)\, G_d^a(\epsilon) \big] 
- \int_{-\infty}^{\infty} \frac{d\epsilon}{2\pi }   A(\epsilon; \{\mu_{\alpha},T_{\alpha}\})  f_R(\epsilon) \Big].
\label{eq:central-eq}
\ee
\end{widetext}
%
This exact result holds only for symmetric junctions with constant hybridization $\Gamma$---yet 
irrespective of the nature and strength of many-body interactions on the dot. % DDDD

The discussion so far is rather standard  \cite{Haug-book},
yet it is illuminating to examine the different terms in Eq. (\ref{eq:central-eq}).
From this relation, it appears as if the departure of the
charge current from the average site population depends on two factors: (i) The deviation from the equilibrium 
fluctuation-dissipation theorem for the nonlinear self-energy component, 
as reflected by the second term in the R.H.S of Eq.~(\ref{eq:central-eq}). (ii) 
The integral over the spectral function of the interacting system, weighted by the right-lead distribution.
However, one can show that the first contribution vanishes identically 
under both equilibrium and nonequilibrium conditions, based on two fundamental principles.  

First, at equilibrium,  $\mu_L\!=\!\mu_R\!=\!\mu$ and $T_L\!=\!T_R\!=\!T$,  $f(\epsilon)=f_{\alpha}(\epsilon)$,
the detailed balance condition between the lesser and greater components of the Green's functions 
$G_d^{>}(\epsilon) = -e^{\beta(\epsilon-\mu)} \, G_d^{<}(\epsilon)$ ensures the fluctuation-dissipation relation for the nonlinear self-energy, 
%
\be
\Sigma_n^{<}(\epsilon) = -f(\epsilon) \left[\Sigma_n^r(\epsilon) - \Sigma_n^a(\epsilon)\right]
%
=i f(\epsilon) \Gamma_n(\epsilon).
\ee
%
As a result, the second term in the R.H.S of Eq.~(\ref{eq:central-eq}) vanishes, 
and we recover the standard expression for the average 
site population, in equilibrium, $\langle n_d \rangle^{eq} = \int \frac{d\epsilon}{2\pi } A(\epsilon, \{\mu, T\})f(\epsilon)$. 
%For the differential conductance one also recovers the Eq.~(\ref{eq-condu}) in the equilibrium limit. 

What happens under general nonequilibrium conditions ($\mu_L \!\neq\! \mu_R$ and $T_L \neq T_R$)? 
In this case, current conservation (in steady state) imposes a constraint on the nonlinear self-energy, 
given as \cite{Haug-book}
%
\be
\!\!\int_{-\infty}^{\infty} \frac{d\epsilon}{2\pi} \, \Big[G_d^{>}(\epsilon) \Sigma_{n}^{<}(\epsilon) - G_d^{<}(\epsilon) \Sigma_{n}^{>}(\epsilon)\Big]=0.
\ee
%
This expression can be reorganized as 
%
\bea
\!\!\int_{-\infty}^{\infty} \!\!\frac{d\epsilon}{2\pi}\Big[\Big(G_d^{>}(\epsilon)\!&-&\!G_d^{<}(\epsilon)\Big) \Sigma_{n}^{<}(\epsilon) \nonumber \\
\!&-&\! G_d^{<}(\epsilon) \Big(\Sigma_{n}^{>}(\epsilon)\!-\!\Sigma_{n}^{<}(\epsilon)\Big)\Big]=0,
\nonumber\\
\eea
%
which implies that
%
\bea
\int_{-\infty}^{\infty} \frac{d\epsilon}{2\pi} \, 
\Big[A(\epsilon) \Sigma_{n}^{<}(\epsilon)-G_d^{<}(\epsilon) \Gamma_n(\epsilon)\Big]=0.
\eea
%
Plugging in the spectral function, Eq.~(\ref{eq:spectral}), we
receive the following condition,
%
\bea
&&\int_{-\infty}^{\infty} \frac{d\epsilon}{2\pi } \left[ G_d^r(\epsilon) \, \Big(-i \Sigma_n^{<}(\epsilon) - \frac{f_L(\epsilon) + f_R(\epsilon)}{2} \Gamma_n(\epsilon) \Big)\, G_d^a(\epsilon) \right]
\nonumber\\
&&=0.
\label{eq:cond}
\eea
%
This combination identically annihilates the second term in the R.H.S. of Eq.~(\ref{eq:central-eq}). 
Therefore, the relationship between the steady state charge current and the average electronic population, 
Eq.~(\ref{eq:central-eq}), in fact simplifies to 
%
\be
\langle I_L \rangle =e \Gamma \Big[\langle n_d\rangle-\int_{-\infty}^{\infty} \frac{d\epsilon}{2\pi }   A(\epsilon; \{\mu_{\alpha},T_{\alpha}\})  f_R(\epsilon) \Big].
\label{eq:cur-pop}
\ee
%
This expression can be also derived by imposing the current conservation condition at the starting point, in Eq.~(\ref{eq:MW}) \cite{Ness,NessC}.

Herewith, we will consider the bias dependence to enter only through the left-lead Fermi function. 
Few simple conclusions then immediately follow. % from Eq.~(\ref{eq:cur-pop}). 
First, if the spectral function is independent of bias 
(for example, if $H_n$ is ignored), we receive the following relation 
between the differential conductance $G(V)\equiv d\langle I_L \rangle/{dV}$ 
and the charge susceptibility $\chi_d(V) \equiv d\langle n_d \rangle/{dV}$, 
%
\be
G(V) = e \, \Gamma \,\chi_d(V).
\ee 
%
Moreover, in the limit $\mu_R \to -\infty$ the second term in the R.H.S of Eq.~(\ref{eq:cur-pop}) vanishes, and we arrive 
at the following equality, valid for an {\it arbitrarily  interacting} system, 
%
\be
\langle I_L \rangle =e \Gamma \langle n_d\rangle.
\label{eq:high-bias}
\ee
% 
This seemingly trivial, steady state current-occupation
relation is the focus of our discussion throughout the rest of the paper.
Assumptions involved are that the junction is symmetric, $\Gamma(\epsilon)$ is constant,
and that the current is uni-directional, left to right. 
%
%We argue that
%one can use Eq. (\ref{eq:high-bias}) to assess, in a nontrivial way, 
%the consistency and accuracy of simulation techniques.
%In particular, in the next sections
%we test this relation on the Anderson-Holstein and the electron-electron (e-e) interacting Anderson impurity models.
%
Why is Eq. (\ref{eq:high-bias}) interesting?
Typically, to examine convergence one {\it separately} studies the approach of the current and occupation
to fixed values. However, this relation offers an additional test for evaluating simulation tools.
In the next sections,
we test this equality on the Anderson-Holstein model and the electron-interacting Anderson impurity model.
We demonstrate below that by considering Eq. (\ref{eq:high-bias}), we identify inconsistencies in simulation techniques,
resulting in e.g. spurious transport phenomena.
Another reason for simplifying Eq. (\ref{eq:cur-pop}), and using the specific limit (\ref{eq:high-bias}) is that in many techniques, as used below, 
one does not have a direct access to the impurity spectral function  $A(\epsilon; \{\mu_{\alpha},T_{\alpha}\})$, rather, only the current and the dot occupation 
can be computed.
% D2

Before turning to particular models, we recall that phase-loss and inelastic effects can be introduced into a
noninteracting transport behavior 
(under the Hamiltonian $H_0$) by attaching the dot to  ``Buttiker probes"  \cite{Buttiker}.  For recent applications
in the context of quantum dot and molecular transport junctions see  \cite{Salil1,Kilgour1,Kilgour2}.
Working with e.g. the dephasing probe technique, one can readily prove that in the case of a single impurity,
the method exactly satisfies Eq. (\ref{eq:cur-pop}). 
Similarly, a cheap approach for introducing inelastic effects,
by adding a broadening into the noninteracting transmission function,
satisfies Eq. (\ref{eq:high-bias}). For a recent theory-experiment study of 
coherent and incoherent effects in single-dot junctions
see \cite{Nijhuis}. % DDD


%===============================================================
% Figure 1
%\begin{wide}
\begin{center}
\begin{figure*}[t]
\includegraphics[width=20cm]{1.eps}%strong_new1.eps} % 14
\caption{
Steady state charge current $\langle I_L\rangle$ (solid) 
and average population $\Gamma \langle n_d \rangle$ ($\circ$) as a function of 
left-lead chemical potential $\mu_L$ for strong electron-phonon coupling strength $\gamma_0/\omega_0=2$. 
Matrix elements $M_{q,q'}$ are calculated (a) using the Franck Condon formula, Eq. (\ref{eq:FC}),
(b) using  numerical diagonalization at finite $N$.
The number of levels included for the truncated harmonic oscillator are $N=20$ (black), $N=50$ (red), $N=100$ (blue),
and $N=200$ (magenta). The last case is demonstrated only with Method II, as we could not evaluate
high order FC factors with the analytical formula (\ref{eq:FC}).
Parameters are $\epsilon_d-\alpha_0^2\omega_0$=0, $\omega_0$ =1, $\gamma_0$=2,  % DDD
$\mu_R=-100$,  $T_L=T_R=0.05$.
}
\label{Fig1}
\end{figure*}
\end{center}
%\end{wide}

% Figure 2
\begin{center}
\begin{figure*}
\includegraphics[width=19.5cm]{2d.eps} %  %medium_new1.eps}
\caption{
Steady state charge current and average population as a function of left-lead chemical potential $\mu_L$
for intermediate electron-phonon coupling, $\gamma_0/\omega_0=0.5$. 
Matrix elements $M_{q,q'}$ are calculated (a) using the Franck Condon formula, Eq. (\ref{eq:FC}),
(b) using numerical diagonalization at finite $N$.
$N=20$ (black), $N=50$ (red), $N=100$ (blue), $N=200$ (magenta), $N=300$ (green).
The last two values are only demonstrated using Method II.
The insets display the differential conductance $G(V)$ for $N=100$. 
Other parameters are the same as in Fig.~(\ref{Fig1}).}
\label{Fig2}\end{figure*}
\end{center}
%============================================================

%===========================================================
\section{Anderson-Holstein Model: polaronic QME}
\label{SAH}

%\subsection{Analytical expressions}
%\subsection{Polaronic quantum master equation}
%approach with Frank Condocn factors}

% spinless or single electrons?

As our first example, we consider the spinless Anderson-Holstein model \cite{AHcomment}
in which electrons on  the dot are coupled 
to a single bosonic (photonic or phononic) mode, creation and annihilation operators $b^{\dagger}$, $b$.
For simplicity, in this example we do not include e-e repulsion, but it is trivial to consider that. 
The Hamiltonian of the dot, along with the bosonic component is
%
\be
H_{d0}+H_n= \epsilon_d \, n_d + \omega_0 \, b^{\dagger} b+ \gamma_0\, n_d\, (b+b^{\dagger}).
\label{eq:AH}
\ee
%
Recall that $\epsilon_d$ is the electronic site energy, $n_d =d^{\dagger} d$ is the number operator for the dot electrons.
Other parameters are $\omega_0$ as the frequency of the bosonic mode and $\gamma_0$ as the electron-boson coupling energy. 
This model is often examined in the context of phononic-vibrational effects in molecular electronic
junctions \cite{Galperin-review}. It can be realized as well in circuit-QED experiments \cite{Kontos-cQED, LeHur-cQED}.
In what follows, we refer to the boson mode as a ``phonon".

In the limit of weak electron-phonon coupling (single boson excitation/de-excitation), 
we only include the lowest non-zero order (the second order) in the nonlinear electronic self-energy 
 $\Sigma_{n}^{<,>,r,a}(\omega)$.
It can be obtained by following the standard Hartree and Fock-like diagrams \cite{Galperin-review}. 
The Hartree term trivially satisfies Eq.~(\ref{eq:cond}): Since it is local in contour time, 
$\Sigma_{H}^{</>}(\omega)=0$. 
The Fock diagram, in contrast, has non-zero lesser and greater components,  but it satisfies 
the condition (\ref{eq:cond}). Therefore, we conclude that Eq.~(\ref{eq:cur-pop}) [and obviously Eq.~(\ref{eq:high-bias})]
hold under a perturbative treatment at the Hartree-Fock level. 
 
Next, we consider the strong electron-phonon coupling limit, and test Eq. (\ref{eq:high-bias}). We use 
the polaronic quantum master equation (QME) approach, as employed e.g. in
Refs.~\cite{Mitra,KochSela,Koch1,Koch2,Wege,Thoss11a, Thoss11b,Simine,Gernot1,Gernot2}. 
%
We briefly review the principles of this approach.
To examine the strong coupling limit, one first performs the small-polaron transformation so as to eliminate
the electron-phonon interaction term from within $H_n$. As a result of the transformation,
the dot energy is renormalized, $\epsilon_d \to \epsilon_d - \gamma_0^2/\omega_0$, 
while the dot-metal tunneling term is dressed by the translational operator 
$D(\alpha_0)= e^{-\alpha_0 (b^{\dagger}-b)}$, where $\alpha_0=\frac{\gamma_0}{\omega_0}$ is a dimensionless
electron-phonon interaction parameter, $c^{\dagger}_{k\alpha}d \rightarrow D(\alpha_0)c^{\dagger}_{k\alpha}d $


A kinetic quantum master equation can be derived by following standard approximations,  namely, 
the secular Born-Markov approximation, assuming (i) weak metal-dot coupling $\Gamma<T,\Delta \mu$,
(ii) fast decay of the baths' (metals) correlation functions vs.
slow tunneling dynamics, (iii) fast decay of vibrational coherences.
While this equation captures only lowest-order processes in the tunnelling energy,
it treats the electron-phonon interaction exactly---in that order of the tunnelling coupling.

We define a reduced density matrix (RDM) $\rho_s$ for the interacting impurity, which includes the dot and the local phonon.
The diagonal elements of the RDM,
$p^{n}_{q}(t) \equiv \langle n,q| \rho_s(t)|n, q\rangle$, satisfy kinetic-like equations of motion
%
%then perform the perturbative expansion to the second order of the system-bath coupling strength.
%${|n,q\rangle}$, $n$ refers to empty and charged state for the site and $q$ denotes the phonon quanta.
\be
\dot{p}^{n}_{q} =  \sum_{n', q'} \Big( p_{q'}^{n'} k_{q' \to q}^{n' \to n} - p_{q}^{n} \, k_{q \to q'}^{n \to n'}\Big).
\label{eq:pop}
\ee
%
Here, $n=0,1$ represents an empty or occupied electronic site and $q$ identifies the 
state of the phonon mode $(q = 0, 1, 2, \cdots)$.  
%
Once steady state is achieved, $\dot{p}_q^{n,ss}=0$, 
the averaged charge current and population
can be computed from the following expressions ($\alpha=L,R$),
%
\bea
\langle I_{\alpha} \rangle &=& \sum_{q, q'}  \Big( k^{0 \to 1}_{q \to q', \alpha} \, p_q^{0,ss} - k^{1 \to 0}_{q \to q', \alpha} \, p_q^{1,ss} \Big), 
\label{eq:AHcur}
\\
%\nonumber\\
\langle n_d \rangle &=& \sum_{q} p_{q}^{1,ss}.
\label{eq:AHn}
\eea
%
For convenience, we set $e=1$.
In Eq.~(\ref{eq:pop}), $k_{q \to q'}^{n \to n'}=\sum_{\alpha} k_{q \to q',\alpha }^{n \to n'}$ 
is the total rate constant for the transition $|n, q\rangle \to |n', q'\rangle$. 
It is easy to verify that this scheme conserves current, $\langle I_L\rangle + \langle I_R\rangle =0$.
Note that according to our sign convention, the current is positive when flowing towards the dot.
The rate constants satisfy
% 
\bea
k_{q \to q', \alpha }^{ 0 \to 1} &=& \Gamma f_{\alpha} \big(\epsilon_d - \frac{\gamma_0^2}{\omega_0} + \omega_0 (q'-q)\big) \big|M_{q,q'}\big|^2 \nonumber \\ 
k_{q \to q', \alpha }^{1 \to 0} &=& \Gamma \Big[1-f_{\alpha} \big(\epsilon_d - \frac{\gamma_0^2}{\omega_0} + \omega_0 (q'-q)\big)\big] \big|M_{q,q'}\big|^2,
\nonumber\\
\label{eq:rates}
\eea
%
%where $\epsilon_{0,q}= q \omega_0$ and $\epsilon_{1,q}= \epsilon_d - \frac{g^2}{\omega_0} + q \omega_0$.  
where in these expressions, the Fermi function is calculated at the renormalized energy, with a certain number of quanta 
$\omega_0$, and
%$\epsilon_d - \frac{\gamma_0^2}{\omega_0}$. 
%
\bea
M_{q,q'} = \langle q| e^{-\frac{\gamma_0}{\omega_0} (b^{\dagger}-b)} | q'\rangle,
\eea
% 
are the matrix elements of the shift operator. For a harmonic (boson) mode, 
these are the familiar Franck-Condon (FC) factors.
In terms of the dimensionless parameter $\alpha_0$, the FC factors are given by,
%
\bea
M_{q,q'} &\equiv&\langle q |  e^{-\alpha_0(b^{\dagger}-b)}|q'\rangle, \,\,\,\,\, q,q'=0,1,2... 
\nonumber\\
&=& sign(q'-q)^{q-q'}\alpha_0^{q_M-q_m}e^{-\alpha_0^2/2}\sqrt{\frac{q_m!}{q_M!}} L_{q_m}^{q_M-q_m}(\alpha_0^2),
\nonumber\\
\label{eq:FC}
\eea
%
with $q_m=\min\{q,q'\}$,  $q_M=\max\{q,q'\}$, and $L_{a}^{b}(x)$ as the generalized Laguerre polynomials.

It is straightforward to prove that the polaronic quantum master equation as detailed here fulfills 
the exact relation (\ref{eq:high-bias}). 
This fact is not obvious, given that several approximations are involved in the derivation, potentially 
defecting exact relations.
%
We begin with Eq.~(\ref{eq:AHcur}). In the limit
$\mu_R \to -\infty$, right-lead induced rates simplify as follows: $k_{q \to q', R }^{ 0 \to 1}=0$,
 and $k_{q \to q', R }^{1 \to 0} \approx \Gamma\, |M_{q,q'}|^2$.
The current at the right contact becomes
%
\bea
\langle I_R \rangle = -\Gamma\, \sum_{q} p_{q}^{1,ss}\, \sum_{q'} |M_{q,q'}|^2.
\label{eq:proof}
\eea
%
Next, we use the completeness relation for the bosonic manifold, $\sum_{q=0}^{\infty} |q \rangle \langle q| = I$,
with $I$ as the identity operator, the unitarity of the polaron transformation, $MM^{\dagger}=M^{\dagger}M=I$, 
and current conservation,
and get $\langle I_L \rangle = \Gamma\ \sum_q p_{q}^{1,ss} $, proving Eq. (\ref{eq:high-bias}).


In Fig. \ref{Fig1}, we present numerical simulations of the average current and the dot's population, 
based on Eqs. (\ref{eq:AHcur})-(\ref{eq:AHn}).
%with the FC factors evaluaed from (\ref{eq:FC}).
Obviously, in practice we truncate the harmonic spectrum, to include
a finite number $N$ of basis states for the mode. The particular choice of $N$
depends on model parameters: the applied
bias voltage, temperature, mode frequency, and the many-body interaction strength. 
%
In panel (a), the matrix elements in Eq. (\ref{eq:rates}) are evaluated 
with the analytical form for the FC factors, Eq. (\ref{eq:FC}). We refer to this approach as ``Method I".
%Using $N=20,50,100$,
Surprisingly, we observe a clear violation of the {\it exact} relation (\ref{eq:high-bias}).
This is unsettling since we had just {\it proved}
that formally, the polaronic QME technique should satisfy this relation. What is wrong then?
The answer is subtle---yet with significant ramifications. 
It is inconsistent to calculate the matrix elements $M_{q,q'}$ with the exact-analytical FC formula---yet work with 
a finite (even if large) basis for the mode $|q\rangle$. 
%In other words, for consistency, since we perfom a finite summation in Eq. (\ref{eq:AHcur}),
%we must calculate $M_{q,q'}$ for that finite basis.

How can we rectify this, to satisfy Eq. (\ref{eq:high-bias}) in numerical simulations? 
Since we perform a summation over a finite basis in Eq. (\ref{eq:AHcur}),
we must calculate $M_{q,q'}$ in that finite basis,
without relying on the analytical (infinite-$N$) expression for the FC factors. This procedure,  ``Method II",
is explained in the Appendix, see also Ref. \cite{Simine}.
In fact, Method II, which is fully numerical, is very efficient:
While in Method I 
large factorials and high-order Laguerre polynomials should be calculated for high quantum states.
the diagonalization scheme (Method II) described in the Appendix provides all matrix elements $M_{q,q'}$
in a single operation. 
%
Using this method, we demonstrate in Fig. \ref{Fig1}(b) that the  current-occupation relation is exactly satisfied,
even before convergence to the exact ($N\rightarrow \infty$) limit is reached.

In Fig. \ref{Fig2} we illustrate another critical deficiency of Method I,
using an intermediate electron-phonon coupling strength. 
In panel (a) we observe that Method I not only violates  the exact 
relation (\ref{eq:high-bias}), it further leads to a fundamentally incorrect function, namely, 
negative differential conductance (NDC). This spurious effect, highlighted in the inset of panel (a),
survives even when the basis set is large, $N=100$. %---when the current seemingly converges to the infinite-level limit. 
%
Beyond that, we are not able to compute higher order FC factors based on Eq. (\ref{eq:FC}).
In contrast, in panel (b) we show that Method II 
does not manifest the NDC phenomenon for any choice of $N$, even when few levels are employed.
Furthermore, Method II can be used to calculate the behavior of the junction
under very high voltages, when many vibrational states should be taken into account.

%The derivation of our polaronic QME assumes intermediate-strong coupling, resulting in
%the fast decay of vibrational coherences (non-secular terms). Therefore, one should be careful and use this technique only for 
%$\alpha_0 >1$. However, Figure \ref{Fig2} is still quite instructive 
%as it demonstrates that one can receive physically correct, convergent results with Method II, contrasting
%flawed answers of Method I.

To conclude this discussion: One could argue that for large enough $N$, Eq. (\ref{eq:high-bias}) 
should be satisfied numerically---even when the FC factors
are evaluated analytically based on the harmonic spectrum.
However, our simulations demonstrate that Method I is prone to 
fundamental errors, and the fully-numerical Method II is superior is several ways:
(i) It identically satisfies the exact relation (\ref{eq:high-bias}).
(ii) It is highly efficient as one obtains all matrix elements $M_{q,q'}$ in a single operation.
(iii)  Method I may yield  incorrect features, even when many (seemingly sufficient) states are included,
as we had demonstrated in Fig. \ref{Fig2}.


%====================

% Figure 3
\begin{figure*}
%\includegraphics[width=\columnwidth] 
\includegraphics[width=17cm]{3b.eps}         %{1.pdf}
\caption{
Anderson model with an onsite electron-electron interaction.
Transient dynamics for the charge current (dashed) and dot population (full) for
(a) $U$=0.1 and (b) $U$=0.3.
Other parameters are $\delta t=1$, $D=\pm 1$, $\mu_L=0.2$, $\mu_R=-1$, $1/T_{L,R}=200$, $\epsilon_d+U/2=0.3$,
$\Gamma=0.05$, $L=200$ states per spin per bath.
Different lines correspond to different memory size, $N_s=2-5$, see the legend.
}
\label{Fig3}
\end{figure*}


%==========================================
\section{Interacting impurity model: Path integral simulations}
\label{SINFPI}

In this section, we test the  exact relation (\ref{eq:high-bias}) for 
the Anderson impurity model, including the interaction of spin-up and spin-down electrons on the dot.
We simulate the dynamics from a certain initial condition by employing a time-evolution scheme, 
an iterative, numerically-exact influence functional path integral approach.
The principles of this method, and its application to the interacting Anderson dot model, are detailed in Refs. \cite{IF1,IF2}.
Why is it nontrivial to verify Eq. (\ref{eq:high-bias}) with this tool?
In this numerically exact method (as well we in other numerical techniques), 
different observables are independently calculated directly from their definitions.
This should be contrasted e.g. to Green's function-based perturbative methods, where different transport observables are expressed
in terms of the spectral function, see Sec. \ref{SModel}. 
Given the centrality of the interacting Anderson impurity model 
for solving extended models through the dynamical mean field theory,
assessing accuracy of simulation tools is crucial.

We introduce the model. The dot Hamiltonian and the onsite interaction are 
%
\bea
H_{d0}+H_n= \sum_{\sigma} \epsilon_d n_{d,\sigma} + Un_{d,\uparrow}n_{d, \downarrow},
\eea
%
with $\sigma=\uparrow,\downarrow$, $n_{d,\sigma}=d^{\dagger}_{\sigma}d_{\sigma}$. $U$ denotes the e-e interaction strength.
The total Hamiltonian reads
%
\bea
H &=&  %H_{d0}+H_n
\sum_{\sigma} \epsilon_d n_{d,\sigma} + Un_{d,\uparrow}n_{d, \downarrow}
\nonumber\\
&+& \sum_{\alpha,k,\sigma}\epsilon_{k\alpha\sigma} c_{k\alpha\sigma}^{\dagger} c_{k\alpha\sigma}
+\sum_{\alpha,k,\sigma} v_{k\alpha\sigma} c_{k\alpha\sigma}^{\dagger}d_{\sigma} + h.c. 
\nonumber\\
\eea
% 
We assume spin degeneracy, and set a symmetric junction 
with the metal-dot hybridization energy
$\Gamma= 2\pi \sum_{k}|v_{k\alpha\sigma}|^2\delta(\epsilon-\epsilon_{k\alpha\sigma})$.
The two metals are described by flat bands with sharp cutoffs at $D=\pm 1$.
At the initial time, the metals are prepared in a nonequilibrium condition, with $\mu_L=0.2$ and $\mu_R=-1$.
The dot is empty at the beginning of our simulation.

We follow, separately, the real-time dynamics of
the dot occupation and the current with
a deterministic numerically exact approach, 
an influence functional path integral (INFPI) technique, described in details in Refs.  \cite{IF1,IF2}.
This method relies on the observation that in out-of-equilibrium (and/or finite
temperature) situations, bath correlations have a finite range,
allowing for their truncation beyond a memory time dictated
by the voltage bias and the temperature \cite{Makri1}. 
%Taking advantage
%of this fact, an iterative-deterministic time-evolution scheme
%was developed in \cite{IF1,IF2}, where 
%Convergence with respect to
%the memory length can (in principle) be reached. 
As convergence is facilitated at large bias, the method is perfectly suited for testing Eq. (\ref{eq:high-bias}).

%The principles of the INFPI approach have been detailed in
%Refs. \cite{IF1,IF2}, where it has been adopted for investigating
%dissipation effects in the nonequilibrium spin-fermion model,
%and the population and the current dynamics in correlated
%quantum dots, by investigating the single-impurity Anderson
%model \cite{IF1} and the two-level spinless Anderson dot \cite{IF2}.

As elaborated in Refs.  \cite{IF1,IF2}, INFPI suffers from several sources of numerical errors. 
First, the metals are discretized to 
include finite number of electronic states. 
We find here that fora band extending $D=\pm 1$, $L=200$ states
are sufficient to describe a seemingly irreversible dynamics up to $\Gamma t\sim 5$.
%we reach convergence in
%the time interval of interest. 
Other sources of error are the Trotter error,
given the approximate factorization of the 
short time evolution operator into quadratic and
many-body terms, and the memory error, resulting from the
truncation of the influence functional. Convergence is verified by demonstrating
that results are insensitive to the time step $\delta t$ and the memory
size $\tau_c=N_s\delta t$, with $N_s$ an integer.
%once the proper memory time is accounted for.

%As mentioned above, different observables (occupation, current, coherence,...) are indeopedently propagated within INFPI,
%thus one may question whether Eq. (\ref{eq:high-bias}) would be identically satified, or only in the asymptotic exact limit.

In Fig.~(\ref{Fig3}) we display the transient dynamics for a certain spin species, $\langle I_{\sigma}\rangle$  and 
$\Gamma \langle n_{d,\sigma} \rangle$, for two different values of the interaction strength $U$. 
For simplicity, we set $e=1$.
The current displayed is the average at the two contacts, $\langle I_{\sigma}\rangle = \langle I_{L\sigma} - I_{R\sigma}\rangle/2 $.
%We confirmed (not shown) that results converge with respect to the number of metallic states $L$ and the time step $\delta t$.
It is remarkable to observe that an excellent agreement between these expectation values 
is obtained as we approach the long time limit, 
even before convergence to the exact limit ($N_s\rightarrow \infty$, $\delta t\rightarrow 0$) is achieved, see panel (b).
%
We can rationalize this agreement as follows. 
The INFPI method is deterministic, and it conserves the trace of the density matrix. 
For a given choice $N_s$, there are $2^{2N_s}$ paths that should be summed over, covering all the configurations
of the fictitious spin (received from the Hubbard-Stratonovich transformation). 
By simulating the population and current while
keeping all paths for a given $N_s$ and $\delta t$, we essentially maintain the dimension of 
the ``Hilbert space" of the problem. As a result, we
satisfy the exact relation (\ref{eq:high-bias})---even when simulation results are away from the exact answer.
%Note that in Fig. ~(\ref{Fig3}), deviations between the current and occupation result from 
%It is also interesting to note that this relation allows us to further identify the (quasi) steady state limit.
Small deviations observed in Fig. \ref{Fig3} arise from (i) not arriving yet at the (quasi) steady state limit,  (ii)
representing the leads with finite number of states, (iii) using parameters deviating from the
assumptions behind Eq. (\ref{eq:high-bias}): $D=\pm1$ and $\mu_R=-1$. 


We argue that it is important to validate numerically exact methods
by verifying (\ref{eq:high-bias}) in steady state. The fact that INFPI preserves this relation even before convergence 
allows us to regard simulation results for $\langle n_d \rangle$ and $\langle I\rangle$ 
as physically-meaningful values for the true population and current.
%in the sense that they do not violate basic relations.
Techniques based on sampling and filtering paths are promising avenues for accelerating path integral simulations 
\cite{Makri2,Makri3}.
Nevertheless, these methods break the exact current-population relation, and one would need
to affirm convergence by verifying Eq. \ref{eq:high-bias}.

%==========================================

\section{Summary}
\label{Ssum}

By employing the nonequilibrium Green's function approach, we 
had established an exact relationship between the steady state electronic current 
and the average electronic occupation for a symmetric, interacting single impurity junction. 
We had tested this relation in two cases, for the Anderson Holstein model, and for the e-e interacting Anderson dot model,
using perturbative and numerically exact approaches, respectively. 
 
For the Anderson Holstein model, testing the relation had assisted us in uncovering a subtle inconsistency in the
popular polaron QME numerical implementation---with important ramifications for clearing out spurious transport effects.
In the case of the electron-interacting Anderson dot model we found that a deterministic (trace conserving)
influence functional path integral approach obeys the relation
even before convergence to the exact-asymptotic limit is reached. 

Simulating the Anderson dot model is a nontrivial task. 
Testing Eq. (\ref{eq:high-bias}) is suggested here as a viable mean for carefully scrutinizing
the consistency and accuracy of numerical schemes. 
Other consistency checks involve the study of transport symmetries, e.g. the  Onsager theorem under
perturbative methods \cite{Wacker}.
Finally, if full-counting statistics information is available,
one could probe the validity of approximate tools 
by comparing high order cumulants \cite{bijay-recon}, 
or more fundamentally, by confirming that the nonequilibrium fluctuation symmetry is satisfied \cite{bijay-vibration}.

%This work concerns the nonequilibrium Anderson impurity model, examined with different tools:
%nonequilibrium Green's function approach, master equation techniques, and path integral simulations.

%\acknowledgments
We acknowledge Herve Ness for helpful discussions.
The work of DS and BKA was supported by the Natural Sciences and Engineering Research Council of Canada, 
the Canada Research Chair Program, 
and the Centre for Quantum Information and Quantum Control (CQIQC) at the University of Toronto.

%=========================

\renewcommand{\theequation}{A\arabic{equation}}
\setcounter{equation}{0}  % reset counter

\section*{Appendix: Anderson dot model coupled to an $N$-level system}

We describe here ``Method II", a fully numerical scheme for calculating current and population in the 
Anderson-Holstein model at the level of the polaronic rate equation.
%
In this model, electrons on the dot interact with a local bosonic mode, (\ref{eq:AH}).
Here we consider a more general construction, with the dot coupled to an $N$-level system. The interacting (mode 
displacement) operator is similarly represented by a finite matrix.
%
For completeness, we recount the different terms in the total Hamiltonian
%
\bea
H= H_{d0}+ H_L+H_R + H_{dL}+H_{dR}+ H_n.
\eea
%
The dot interacts with two metals,
$H_{\alpha} =\sum_{k}\epsilon_{k\alpha} c_{k\alpha}^{\dagger}c_{k\alpha}$
with the tunneling Hamiltonian
%
\bea
H_{dL}+H_{dR}=\sum_{\alpha,k}\left(v_{k\alpha}c_{k\alpha}^{\dagger}d 
+v_{k\alpha}^*d^{\dagger}c_{k\alpha}\right).
\label{eq:Apphyb}
\eea
%
The Hamiltonian of the dot and the $N$-level entity is
%
\bea
H_{d0}+H_n&=& \epsilon_d n_d 
\nonumber\\
&+&
\sum_{q=0}^{N-1}\epsilon_q |q\rangle \langle q| + \alpha_0 n_d \sum_{q,q'}F_{q,q'}|q\rangle \langle q'|.
\eea
%
As before, $n_d=d^{\dagger}d$ denotes the electron occupation number operator for the dot.
The Hamiltonian of the $N$-level system is written in its energy representation
with states $|q\rangle$, $q,q'=0,1,...,N-1$.
This finite system is coupled to the electron number operator via the operator $F$,
$\alpha_0$ is a dimensionless parameter.
%
It is useful to define the $N$-level Hamiltonian separately, 
%
\bea
H_{N}=\sum_q\epsilon_q|q\rangle \langle q|
+\alpha_0\sum_{q,q'}F_{q,q'}|q\rangle\langle q'|.
\label{eq:Himp}
\eea
%
This operator is hermitian, and it can be diagonalized with a unitary transformation
%
\bea
\bar H_{N}= e^AH_{N} e^{-A},
\label{eq:U}
\eea
%
where $A^{\dagger}=-A$ is an anti-hermitian operator.
We now introduce a related unitary operator, $e^{A n_d}$. Note that
$e^{A n_d}de^{-A n_d}=de^{-A}$ and that $e^{A n_d}d^{\dagger}e^{-A n_d}=d^{\dagger}e^{A}$.
Operating on the original, total Hamiltonian, $\bar H=e^{A n_d}He^{-A n_d}$, we get
%
\bea
\bar H&=&\sum_{\alpha=L,R,k}\epsilon_{k\alpha}c_{k\alpha}^{\dagger}c_{k\alpha} + \epsilon_d  n_d
\nonumber\\
&+&
\sum_{\alpha,k}\left(v_{k\alpha}c_{k\alpha}^{\dagger}de^{-A} +v_{k\alpha}^*d^{\dagger}c_{k\alpha}e^{A}\right)
\nonumber\\
&+& (1- n_d)\sum_q\epsilon_{q}|q\rangle\langle q| + n_d \bar H_{N}.
%e^{A} H_{imp}e^{-A}.
\label{eq:barH}
\eea
%
%This form is  convenient for developing  a polaronic quantum Master equation approach, 
%as described in the main text, perturbative in the metal-molecule
%tunneling element, but exact, to that order, in the impurity-electron coupling 
%\cite{Mitra, Koch1,Koch2,Simine}.

To simulate the Anderson-Holstein model with a local harmonic  mode linearly coupled to the dot
we assume a certain-large $N$ (such that $N\omega_0 \gg  \Delta\mu$, $T$). We
set $\epsilon_q=q\omega_0$ and
discretize the bosonic displacement operator $b^{\dagger}+b$,
%
\bea
F_{q,q'}=\omega_0\sum_{q,q'}\sqrt q|q\rangle \langle q'|\delta_{q'=q-1} +h.c.
\label{eq:Fppp}
\eea
%
Here $\delta_{q,q'}$ is the Kronecker delta. 
In the limit $N\rightarrow \infty$, 
$A=\alpha_0(b^{\dagger}-b)$ and $\bar H_{N}= \omega_0b^{\dagger}b-\alpha_0^2\omega_0$. %\cite{Simine}.
Therefore, in this limit we can write down the polaronic Hamiltonian as 
%
\bea
\bar{H} %&=&  e^{A n_d}He^{-A n_d}
%\nonumber\\ 
&=&
\sum_{\alpha,k}\epsilon_{k\alpha} c_{k\alpha}^{\dagger}c_{k\alpha} +\epsilon_d  n_d
\nonumber\\
&+& \sum_{\alpha,k}\left[v_{k\alpha}c^{\dagger}_{k\alpha}de^{-A} +v_{k\alpha}^*d^{\dagger}c_{k\alpha}e^{A} \right]
\nonumber\\
&+&  \sum_{q}q\omega_0|q\rangle \langle q| - \alpha_0^2 \omega_0 n_d.
\label{eq:barHAH}
\eea
%
This Hamiltonian is the starting point for our ``Method II" QME calculations. 
We use {\it finite} number of levels $N$,
with $e^A$ obtained from the diagonalization of $H_N$ in Eq. (\ref{eq:Himp}).
Isn't there an inconsistency in this approach? We calculate $e^A$ based on a finite representation for the local mode, yet
 still assume that its spectrum after diagonalization follows the harmonic-like behavior.
%$\bar H_{N}=\sum_q \omega_0q |q\rangle \langle q| -\alpha_0^2\omega_0$.
%
Indeed, if we back-transform,
$e^{-A} \left[\sum_q \omega_0q |q\rangle \langle q| -\alpha_0^2\omega_0\right]e^{A}$, 
for a finite $N$, we will not receive Eq. (\ref{eq:Himp}) identically with (\ref{eq:Fppp}), rather, we will get
a mode slightly deviating from the harmonic spectrum.
%slightly shifted values for the original $N$-level system.
%In other words, for a finite $N$ Eq. (\ref{eq:barHAH}) corresponds to a mode (in the original representation) 
%slightly deviating from the harmonic spectrum.
%with the deviation disappaing as we go to $N\rightarrow \infty$.
What is crucial though for satisfying Eq. (\ref{eq:high-bias}) is that
matrix elements $M_{q,q'}$ are calculated with a complete-finite basis.
%matrix elements which determine the rates in the polaronic master equation,
%\bea
%M_{q,q'}=\langle q | e^{-A}|q'\rangle,
%\eea
%
%so as the proof detailed around Eq. (\ref{eq:proof}) holds.

Summing up, in method II we put together an $N\times N$-sized matrix $H_N$ (\ref{eq:Himp}) with the discretized 
displacement operator (\ref{eq:Fppp}), and diagonalize it.
The elements of the transformation matrix $M_{q,q'}=\langle q|e^A|q'\rangle$
correspond to the Franck-Condon factors of Eq. (\ref{eq:FC}). 
%Assuming Eq. \ref{eq:}
They are used to calculate the rate constants (\ref{eq:rates}), then current and population
following Eq. (\ref{eq:AHcur})-(\ref{eq:AHn}).

It is important to note that both Methods, I and II, are defective:
In Method I, we use a truncated harmonic spectrum, but with FC factors evaluated based on an infinite harmonic manifold.
In Method II,  the matrix elements corresponding to the FC factors
are received from the numerical diagonalization of a finite matrix, 
but the energies of the $N$-level mode are taken equi-distant from a perfect harmonic spectrum.
In both techniques, the asymptotic limit $N\rightarrow \infty$ provides the exact result. 
While both approaches are inexact, %in the way we treat the $N$-level system and the polaronic translation element, 
Method II is superior since it preserves the charge-occupation symmetry.

%=========================
\begin{thebibliography}{0}

%\bibitem{commentG}

\bibitem{Anderson}
P. W. Anderson, Phys. Rev. {\bf 124}, 41 (1961).

\bibitem{Kondo}
A. C. Hewson, {\it The Kondo Problem to Heavy Fermions}, (Cambridge
University Press, Cambridge, England, 1993).

\bibitem{Galperin-review}
M. Galperin, M. A. Ratner, and A. Nitzan, J. Phys.: Condens. Matter {\bf 19}, 103201 (2007).
%Molecular transport junctions: vibrational effects

\bibitem{GalperinS}
M. Galperin, M. A. Ratner, A. Nitzan, and A. Troisi,
%Nuclear Coupling and Polarization in Molecular Transport Junctions: Beyond Tunneling to Function
Science {\bf 319}, 1056 (2008).

\bibitem{Wilson}
K. Wilson, Rev. Mod. Phys. {\bf 47}, 773 (1975).

\bibitem{RG}
R. Bulla, T. A. Costi, and T. Pruschke, Rev. Mod. Phys. {\bf 80}, 395 (2008).

\bibitem{MW93}
Y. Meir, N. S. Wingreen, and P. A. Lee, Phys. Rev. Lett. {\bf 70}, 2601 (1993).
% Low-temperature transport through a quantum dot: The Anderson model out of equilibrium

\bibitem{Wang1}
H. Wang and M. Thoss
J. Chem. Phys. {\bf 131}, 024114 (2009).
% Numerically exact quantum dynamics for indistinguishable particles: The multilayer multiconfiguration time-dependent Hartree theory in second quantization representation

\bibitem{Wang2}
%Numerically exact, time-dependent treatment of vibrationally coupled electron transport in single-molecule junctions
H. Wang, I. Pshenichnyuk, R. H\"artle, and M. Thoss,
J. Chem. Phys.  {\bf 135}, 244506  (2011).

\bibitem{Wang3}
%Numerically exact, time-dependent study of correlated electron transport in model molecular junctions
H. Wang and M. Thoss,
J. Chem. Phys. {\bf 138} (13), 134704 (2013).
%

\bibitem{Rabani14}
E. Y. Wilner, H. Wang, M. Thoss, and E. Rabani,
%"Nonequilibrium Quantum Systems with Electron-Phonon Interactions:  Transient Dynamics and Approach to Steady State", 
Phys. Rev. B {\bf 89}, 205129 (2014).


\bibitem{HEOM1}
%Hierarchical Liouville-Space Approach for Accurate and Universal Characterization of Quantum Impurity Systems
Z.-H. Li, N.-H. Tong, X. Zheng, D. Hou, J.-H. Wei, J. Hu, and Y.-J. Yan,
Phys. Rev. Lett. {\bf 109}, 266403  (2012).
% http://journals.aps.org/prl/abstract/10.1103/PhysRevLett.109.266403#fulltext


\bibitem{HEOM2}
R. H\"artle, G. Cohen, D. R. Reichman, and A. J. Millis,
%Transport through an Anderson impurity:  Current ringing, non-linear magnetization and a direct comparison of continuous-time quantum Monte Carlo and hierarchical quantum master equations
Phys. Rev. B {\bf 92}, 085430 (2015).

\bibitem{HEOM3}
C. Schinabeck, A. Erpenbeck, R. H\"artle, and M. Thoss,
%Hierarchical Quantum Master Equation Approach to Electronic-Vibrational Coupling in Nonequilibrium Transport through Nanosystems
Phys. Rev. B {\bf 94}, 201407R (2016).
% http://link.aps.org/doi/10.1103/PhysRevB.94.201407

\bibitem{Lothar}
%dIagrammatic Monte Carlo simulation of nonequilibrium systems 
P. Werner, T. Oka, and A. J. Millis, Phys. Rev. B {\bf 79}, 035320 (2009);
L. Muhlbacher and E. Rabani, Phys. Rev. Lett. {\bf 100} 176403 (2008);
% Real-Time Path Integral Approach to Nonequilibrium Many-Body Quantum Systems
M. Schiro and M. Fabrizio, Phys. Rev. B {\bf 79}, 153302 (2009).
% Real-time diagrammatic Monte Carlo for nonequilibrium quantum transport  


\bibitem{Coheninch}
G. Cohen, E. Gull, D. R. Reichman, and A. J. Millis, 
%"Taming the Dynamical Sign Problem in Real-Time Evolution of Quantum Many-Body Problems", 
Phys. Rev. Lett. {\bf 115}, 266802 (2015).


\bibitem{CohenHT}
H. T. Chen, G. Cohen, A. J. Millis, and D. R. Reichman, 
%"Anderson-Holstein model in two flavors of the noncrossing approximation", 
Phys. Rev. B {\bf 93}, 174309 (2016)


\bibitem{egger1}
S. Weiss, J. Eckel, M. Thorwart, and R. Egger,
Phys. Rev. B {\bf 77}, 195316 (2008).
% Iterative real-time path integral approach to nonequilibrium quantum transport 

\bibitem{egger2}
R. H\"utzen, S. Weiss, M. Thorwart, and R. Egger,
%Iterative summation of path integrals for nonequilibrium molecular quantum transport
Phys. Rev. B {\bf 85}, 121408 (R) (2012).

\bibitem{IF1}
D. Segal, A. J. Millis, and D. R. Reichman,
%Numerically exact path integral simulation of nonequilibrium quantum transport and dissipation
Phys. Rev. B {\bf 82}, 205323 (2010).

\bibitem{IF2}
D. Segal, A. J. Millis, and D. R. Reichman,
%Nonequilibrium transport in quantum impurity models: Exact path integral simulations
Phys. Chem. Chem. Phys. {\bf 13}, 14378 (2011).


%fluctuation-theorems 
\bibitem{fluc1} M. Esposito, U. Harbola, and S. Mukamel, 
Rev. Mod. Phys. {\bf 81}, 1665 (2009).

\bibitem{fluc2} 
M. Campisi, P. H\"anggi, and P. Talkner, 
Rev. Mod. Phys. {\bf 83}, 771 (2011).


\bibitem {meir-wingreen} Y. Meir and N. Wingreen,
Phys. Rev. Lett. {\bf 68}, 2512 (1992).
%Landauer formula for the current through an interacting electron region

\bibitem{Haug-book}
H. Haug and A.-P. Jauho,
{\it Quantum Kinetics in Transport and Optics of Semiconductors}, 2nd ed. (Springer, New York, 2008).


\bibitem{Rammer_review}
J. Rammer and H. Smith, Rev. Mod. Phys. {\bf 58}, 323 (1986).
%Quantum field-theoretical methods in transport theory of metals, Rev. Mod. Phys., 1986, 58(2): 323

\bibitem{Rammer_book} J. Rammer, {\it Quantum Field Theory of Non-Equilibrium
States, Cambridge}, (Cambridge University Press, 2007).

\bibitem{bijay-wang-review}
J.-S. Wang, B. K. Agarwalla, H. Li,
and J. Thingna, Front. Physics {\bf 9}, 673 (2014).


%\bibitem{Nijhuis} 
%A. R. Garrigues, L. Yuan,
%L. Wang, E. R. Mucciolo, D. Thompson, E. del Barco, and C. A. Nijhuis,
%%A Single-Level Tunnel Model to Account for Electrical Transport through Single Molecule- and Self Assembled Monolayer-based Junctions, 
%Sci. Rep. {\bf 6}, 26517 (2016). % DOI: 10.1038/srep26517


%\bibitem{Groot-book} 
%S. R. De Groot and P. Mazur, 
%{\it Non-Equilibrium Thermodynamics} (North-Holland, Amsterdam, 1984).

\bibitem{Kadanoff} L. P. Kadanoff and G. Baym, {\it Quantum Statistical Mechanics}, Benjamin/Cummings, 1962

\bibitem{Keldysh} L. V. Keldysh, Diagram technique for nonequilibrium processes,
Sov. Phys. JETP,  {\bf 20}, 1018 (1965).

\bibitem{book1} P. Danielewicz 
{\it Quantum theory of nonequilibrium processes (I)}, Ann. Phys.,  {\bf 152}, 239 (1984).

\bibitem{NessC}
It is interesting to note that based on Eq. (\ref{eq:Lan-cur}), one can immediately organize Eq. (\ref{eq:cur-pop}) in a similar form,
%
\be
\langle I_L \rangle =-e \Gamma \Big[\langle n_d\rangle - \int_{-\infty}^{\infty} \frac{d\epsilon}{2\pi }   A(\epsilon; \{\mu_{\alpha},T_{\alpha}\})  f_L(\epsilon) \Big].
\label{eq:cur-pop2}
\ee
% D2
This expression was the starting point in previous studies \cite{Ness} over the relation between the charge current and the dot occupation.


%\bibitem{nessC}
%Recently, an attempt was made to understand the relationship between 
%charge susceptibility and nonlinear dynamical conductance at finite bias \cite{Ness}.
%Performing numerical simulations, the authors claim that these two quantities are directly
%related to each other---but in a different way than in the equilibrium case.  
%Our work here clarifies this confusion. %DDD


\bibitem{Ness}
H. Ness and L. K. Dash, Phys. Rev. Lett. {\bf  108}, 126401 (2012).
%nonequilibrium charge susceptibility and dynamical conductance : Identification of scattering processes in quantum transport
% http://journals.aps.org/prl/abstract/10.1103/PhysRevLett.108.126401

\bibitem{Buttiker} 
M. B\"uttiker, Phys. Rev. B {\bf 32} , 1846 (1985); ibid  {\bf 33}, 3020 (1986).

\bibitem{Salil1}
S. Bedkihal , M. Bandyopadhyay, and D. Segal,
%The probe technique far-from-equilibrium: Magnetic field symmetries of nonlinear transport
Euro. Phys. J. B {\bf 86} 506 (2013).


\bibitem{Kilgour1}
M. Kilgour and D. Segal,
%Charge transport in molecular junctions: From tunneling to hopping with the probe technique
J. Chem. Phys. {\bf 143} 024111 (2015).

\bibitem{Kilgour2}
M. Kilgour and D. Segal,
%Inelastic effects in molecular transport junctions: The probe technique at high bias Pic
J. Chem. Phys. {\bf 144}, 124107 (2016). 

\bibitem{Nijhuis} 
A. R. Garrigues, L. Yuan,
L. Wang, E. R. Mucciolo, D. Thompson, E. del Barco, and C. A. Nijhuis,
%%A Single-Level Tunnel Model to Account for Electrical Transport through Single Molecule- and Self Assembled Monolayer-based Junctions, 
Sci. Rep. {\bf 6}, 26517 (2016). % DOI: 10.1038/srep26517



\bibitem{AHcomment}
One can account for the spin degree of freedom
by revising the rate constants (\ref{eq:rates}) to include prefactors, 
$s(n=0\to n=1)=2$ and $s(n=1\to n=0)=1$.
This modification does not affect any of our observations.

\bibitem{Kontos-cQED}
M. R. Delbecq, V. Schmitt, F. D. Parmentier, N. Roch, J. J.
Viennot, G. Feve, B. Huard, C. Mora, A. Cottet, and T. Kontos,
Phys. Rev. Lett. {\bf 107}, 256804 (2011).
%Coupling a Quantum Dot, Fermionic Leads, and a Microwave Cavity on a Chip

\bibitem{LeHur-cQED}
M. Schir\'o and K. Le Hur, Phys. Rev. B {\bf 89}, 195127 (2014).

\bibitem{Mitra}
A. Mitra, I. Aleiner, and A. J. Millis,
Phys. Rev. B {\bf 69}, 245302 (2004).  

\bibitem{KochSela}
J.  Koch,  F.  von  Oppen,  Y.  Oreg,  and  E.  Sela,
Phys.  Rev.  B {\bf 70},  195107 (2004).

\bibitem{Koch1} J. Koch and F. von Oppen, 
Phys. Rev. Lett. {\bf 94}, 206804 (2005).
%Franck-Condon Blockade and Giant Fano Factors in Transport through Single Molecules

\bibitem{Koch2}  
J. Koch, F. von Oppen, and A. V. Andreev, 
Phys. Rev. B {\bf 74}, 205438 (2006).
%Theory of the Franck-Condon blockade regime

% CHECK
\bibitem{Wege}
%Kinetic equations for transport through single-molecule transistors
M. Leijnse and M. R. Wegewijs,
Phys. Rev. B {\bf 78}, 235424 (2008).

\bibitem{Thoss11a}
R. H\"artle and M. Thoss, 
Phys. Rev. B {\bf 83},125419 (2011).
% Vibrational instabilities in resonant electron transport through single-molecule junctio

\bibitem{Thoss11b}
R. H\"artle and M. Thoss, Phys. Rev. B {\bf 83},115414 (2011).
% Resonant electron transport in single-molecule junctions: Vibrational excitation, rectification,
%negative differential resistance, and local cooling



\bibitem{Simine}
L. Simine and D. Segal,
J. Chem. Phys. {\bf 141}, 014704 (2014).

\bibitem{Gernot1}
G. Schaller, T. Krause, T. Brandes, and M. Esposito,
%Single electron transistor strongly coupled to vibrations: Counting Statistics and Fluctuation Theorem,
New J. Phys. {\bf 15}, 033032 (2013).

\bibitem{Gernot2}
T. Krause, T. Brandes, M. Esposito, and G. Schaller,
%Thermodynamics of the polaron master equation at finite bias,
J. Chem. Phys. {\bf 142}, 134106 (2015),

%\bibitem{bijay-photon} 
%B. K. Agarwalla, M. Kulkarni, S. Mukamel, and D. Segal, Phys. Rev. B XXX

%\bibitem{Agarwalla_Beil} B. K. Agarwalla, J.-H. Jiang, and 
%D. Segal, Beilstein J. Nanotechnol. {\bf 6}, 2129 (2015).

\bibitem{Makri1}
N. Makri, 
%“Numerical path integral techniques for long-time quantum dynamics of dissipative systems”, 
J. Math. Phys.  {\bf 36}, 2430 (1995). 
%

\bibitem{Makri2}
V. Jadhao and N. Makri, 
%"Iterative Monte Carlo for quantum dynamics", 
J. Chem. Phys. (Communication) {\bf 129}, 161102 (2008).

\bibitem{Makri3}
R.  Lambert  and  N.  Makri,  
%“Memory  path  propagator  matrix  for  long-time  dissipative charge transport dynamics” (invited article for D. R. Herschbach Festschrift), 
Mol.  Phys. {\bf 110}, 1967 (2012). 


\bibitem{Wacker}
%Violation of Onsager's theorem in approximate master equation approaches
K. M. Seja, G. Kiršanskas, C. Timm, and A. Wacker,
Phys. Rev. B {\bf 94}, 165435 (2016).
% 



\bibitem{bijay-recon}
B. K. Agarwalla and D. Segal,
%Reconciling perturbative approaches in phonon assisted transport junctions Pic arXiv:1512.01740
J. Chem.  Phys. {\bf 144}, 074102 (2016).


\bibitem{bijay-vibration} 
B. K. Agarwalla, J.-H. Jiang, D. Segal, Phys. Rev. B {\bf 92}, 245418 (2015).
%Full counting statistics of vibrationally assisted electronic conduction: Transport and fluctuations of thermoelectric efficiency 


%\bibitem{Thoss1} A. Erpenbeck, R. H\"artle, and M. Thoss, 
%Phys. Rev. B {\bf 91}, 195418 (2015).
%Effect of nonadiabatic electronic-vibrational interactions on the transport properties of single-molecule junctions
%http://journals.aps.org/prb/abstract/10.1103/PhysRevB.91.195418


%\bibitem{Ora} 
%O. Entin-Wohlman, Y. Imry, and A. Aharony, 
%Phys. Rev. B {\bf 80}, 035417 (2009).
%Voltage-induced singularities in transport through molecular junctions


\end{thebibliography}
 
\end{document}


