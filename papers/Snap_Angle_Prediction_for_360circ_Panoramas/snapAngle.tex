% last updated in April 2002 by Antje Endemann
% Based on CVPR 07 and LNCS, with modifications by DAF, AZ and elle, 2008 and AA, 2010, and CC, 2011; TT, 2014; AAS, 2016

\documentclass[runningheads]{llncs}
\usepackage{graphicx}
\usepackage{amsmath,amssymb} % define this before the line numbering.
%\usepackage{ruler}
\usepackage{color}
\usepackage{times}
\usepackage{epsfig}
\usepackage{gensymb}
\usepackage{amsfonts}
\usepackage{booktabs}


\usepackage[width=122mm,left=12mm,paperwidth=146mm,height=193mm,top=12mm,paperheight=217mm]{geometry}

%\newcommand{\KGthree}{\textcolor{magenta}}
%\newcommand{\KG}{\textcolor{black}}
%\newcommand{\KGtwo}{\textcolor{black}}
%\newcommand{\cc}{\textcolor{green}}
%\newcommand{\BX}{\textcolor{black}}
%\newcommand{\bx}{\textcolor{blue}}
%\newcommand{\bo}{\textcolor{red}}

\renewcommand{\thefootnote}{\fnsymbol{footnote}}
\newcommand{\KGthree}{\textcolor{black}}
\newcommand{\KG}{\textcolor{black}}
\newcommand{\KGtwo}{\textcolor{black}}
\newcommand{\cc}{\textcolor{black}}
\newcommand{\BX}{\textcolor{black}}
\newcommand{\bx}{\textcolor{black}}
\newcommand{\bo}{\textcolor{black}}
\newcommand{\boc}{\textcolor{black}}
\DeclareMathOperator*{\argmin}{argmin}





\begin{document}
% \renewcommand\thelinenumber{\color[rgb]{0.2,0.5,0.8}\normalfont\sffamily\scriptsize\arabic{linenumber}\color[rgb]{0,0,0}}
% \renewcommand\makeLineNumber {\hss\thelinenumber\ \hspace{6mm} \rlap{\hskip\textwidth\ \hspace{6.5mm}\thelinenumber}}
% \linenumbers
%\pagestyle{headings}
%\mainmatter
%\def\ECCV18SubNumber{88}  % Insert your submission number here

\title{Snap Angle Prediction for 360$^{\circ}$ Panoramas} % Replace with your title

\titlerunning{Snap Angle Prediction for 360$^{\circ}$ Panorama}

\authorrunning{B. Xiong and K. Grauman}

\author{Bo Xiong\inst{1} \and Kristen Grauman\inst{2}}
\institute{University of Texas at Austin \\
\and  Facebook AI Research \\
\email{bxiong@cs.utexas.edu,grauman@fb.com}\footnote[1]{\emph{On leave from University of Texas at Austin (grauman@cs.utexas.edu).}}}




\maketitle

\begin{abstract}
360$^{\circ}$ panoramas are a rich medium, yet notoriously difficult to visualize in the 2D image plane.  We explore how intelligent rotations of a spherical image may enable content-aware projection with fewer perceptible distortions.  Whereas existing approaches assume the viewpoint is fixed, intuitively some viewing angles within the sphere preserve high-level objects better than others.  To discover the relationship between these optimal \emph{snap angles} and the spherical panorama's content, we develop a reinforcement learning approach for the cubemap projection model.  Implemented as a deep recurrent neural network, our method selects a sequence of rotation actions and receives reward for avoiding cube boundaries that overlap with important foreground objects. \boc{ We show our approach creates more visually pleasing panoramas while using 5x less computation than the baseline.}
% Our results demonstrate the impact both qualitatively and quantitatively. 
 

\keywords{\KGthree{360$^{\circ}$ panoramas, content-aware projection, foreground objects}}
\end{abstract}





% \leavevmode
% \\
% \\
% \\
% \\
% \\
\section{Introduction}
\label{introduction}

AutoML is the process by which machine learning models are built automatically for a new dataset. Given a dataset, AutoML systems perform a search over valid data transformations and learners, along with hyper-parameter optimization for each learner~\cite{VolcanoML}. Choosing the transformations and learners over which to search is our focus.
A significant number of systems mine from prior runs of pipelines over a set of datasets to choose transformers and learners that are effective with different types of datasets (e.g. \cite{NEURIPS2018_b59a51a3}, \cite{10.14778/3415478.3415542}, \cite{autosklearn}). Thus, they build a database by actually running different pipelines with a diverse set of datasets to estimate the accuracy of potential pipelines. Hence, they can be used to effectively reduce the search space. A new dataset, based on a set of features (meta-features) is then matched to this database to find the most plausible candidates for both learner selection and hyper-parameter tuning. This process of choosing starting points in the search space is called meta-learning for the cold start problem.  

Other meta-learning approaches include mining existing data science code and their associated datasets to learn from human expertise. The AL~\cite{al} system mined existing Kaggle notebooks using dynamic analysis, i.e., actually running the scripts, and showed that such a system has promise.  However, this meta-learning approach does not scale because it is onerous to execute a large number of pipeline scripts on datasets, preprocessing datasets is never trivial, and older scripts cease to run at all as software evolves. It is not surprising that AL therefore performed dynamic analysis on just nine datasets.

Our system, {\sysname}, provides a scalable meta-learning approach to leverage human expertise, using static analysis to mine pipelines from large repositories of scripts. Static analysis has the advantage of scaling to thousands or millions of scripts \cite{graph4code} easily, but lacks the performance data gathered by dynamic analysis. The {\sysname} meta-learning approach guides the learning process by a scalable dataset similarity search, based on dataset embeddings, to find the most similar datasets and the semantics of ML pipelines applied on them.  Many existing systems, such as Auto-Sklearn \cite{autosklearn} and AL \cite{al}, compute a set of meta-features for each dataset. We developed a deep neural network model to generate embeddings at the granularity of a dataset, e.g., a table or CSV file, to capture similarity at the level of an entire dataset rather than relying on a set of meta-features.
 
Because we use static analysis to capture the semantics of the meta-learning process, we have no mechanism to choose the \textbf{best} pipeline from many seen pipelines, unlike the dynamic execution case where one can rely on runtime to choose the best performing pipeline.  Observing that pipelines are basically workflow graphs, we use graph generator neural models to succinctly capture the statically-observed pipelines for a single dataset. In {\sysname}, we formulate learner selection as a graph generation problem to predict optimized pipelines based on pipelines seen in actual notebooks.

%. This formulation enables {\sysname} for effective pruning of the AutoML search space to predict optimized pipelines based on pipelines seen in actual notebooks.}
%We note that increasingly, state-of-the-art performance in AutoML systems is being generated by more complex pipelines such as Directed Acyclic Graphs (DAGs) \cite{piper} rather than the linear pipelines used in earlier systems.  
 
{\sysname} does learner and transformation selection, and hence is a component of an AutoML systems. To evaluate this component, we integrated it into two existing AutoML systems, FLAML \cite{flaml} and Auto-Sklearn \cite{autosklearn}.  
% We evaluate each system with and without {\sysname}.  
We chose FLAML because it does not yet have any meta-learning component for the cold start problem and instead allows user selection of learners and transformers. The authors of FLAML explicitly pointed to the fact that FLAML might benefit from a meta-learning component and pointed to it as a possibility for future work. For FLAML, if mining historical pipelines provides an advantage, we should improve its performance. We also picked Auto-Sklearn as it does have a learner selection component based on meta-features, as described earlier~\cite{autosklearn2}. For Auto-Sklearn, we should at least match performance if our static mining of pipelines can match their extensive database. For context, we also compared {\sysname} with the recent VolcanoML~\cite{VolcanoML}, which provides an efficient decomposition and execution strategy for the AutoML search space. In contrast, {\sysname} prunes the search space using our meta-learning model to perform hyperparameter optimization only for the most promising candidates. 

The contributions of this paper are the following:
\begin{itemize}
    \item Section ~\ref{sec:mining} defines a scalable meta-learning approach based on representation learning of mined ML pipeline semantics and datasets for over 100 datasets and ~11K Python scripts.  
    \newline
    \item Sections~\ref{sec:kgpipGen} formulates AutoML pipeline generation as a graph generation problem. {\sysname} predicts efficiently an optimized ML pipeline for an unseen dataset based on our meta-learning model.  To the best of our knowledge, {\sysname} is the first approach to formulate  AutoML pipeline generation in such a way.
    \newline
    \item Section~\ref{sec:eval} presents a comprehensive evaluation using a large collection of 121 datasets from major AutoML benchmarks and Kaggle. Our experimental results show that {\sysname} outperforms all existing AutoML systems and achieves state-of-the-art results on the majority of these datasets. {\sysname} significantly improves the performance of both FLAML and Auto-Sklearn in classification and regression tasks. We also outperformed AL in 75 out of 77 datasets and VolcanoML in 75  out of 121 datasets, including 44 datasets used only by VolcanoML~\cite{VolcanoML}.  On average, {\sysname} achieves scores that are statistically better than the means of all other systems. 
\end{itemize}


%This approach does not need to apply cleaning or transformation methods to handle different variances among datasets. Moreover, we do not need to deal with complex analysis, such as dynamic code analysis. Thus, our approach proved to be scalable, as discussed in Sections~\ref{sec:mining}.
\section{Related Work}
%\mz{We lack a comparison to this paper: https://arxiv.org/abs/2305.14877}
%\anirudh{refine to be more on-topic?}
\iffalse
\paragraph{In-Context Learning} As language models have scaled, the ability to learn in-context, without any weight updates, has emerged. \cite{brown2020language}. While other families of large language models have emerged, in-context learning remains ubiquitous \cite{llama, bloom, gptneo, opt}. Although such as HELM \cite{helm} have arisen for systematic evaluation of \emph{models}, there is no systematic framework to our knowledge for evaluating \emph{prompting methods}, and validating prompt engineering heuristics. The test-suite we propose will ensure that progress in the field of prompt-engineering is structured and objectively evaluated. 

\paragraph{Prompt Engineering Methods} Researchers are interested in the automatic design of high performing instructions for downstream tasks. Some focus on simple heuristics, such as selecting instructions that have the lowest perplexity \cite{lowperplexityprompts}. Other methods try to use large language models to induce an instruction when provided with a few input-output pairs \cite{ape}. Researchers have also used RL objectives to create discrete token sequences that can serve as instructions \cite{rlprompt}. Since the datasets and models used in these works have very little intersection, it is impossible to compare these methods objectively and glean insights. In our work, we evaluate these three methods on a diverse set of tasks and models, and analyze their relative performance. Additionally, we recognize that there are many other interesting angles of prompting that are not covered by instruction engineering \cite{weichain, react, selfconsistency}, but we leave these to future work.

\paragraph{Analysis of Prompting Methods} While most prompt engineering methods focus on accuracy, there are many other interesting dimensions of performance as well. For instance, researchers have found that for most tasks, the selection of demonstrations plays a large role in few-shot accuracy \cite{whatmakesgoodicexamples, selectionmachinetranslation, knnprompting}. Additionally, many researchers have found that even permuting the ordering of a fixed set of demonstrations has a significant effect on downstream accuracy \cite{fantasticallyorderedprompts}. Prompts that are sensitive to the permutation of demonstrations have been shown to also have lower accuracies \cite{relationsensitivityaccuracy}. Especially in low-resource domains, which includes the large public usage of in-context learning, these large swings in accuracy make prompting less dependable. In our test-suite we include sensitivity metrics that go beyond accuracy and allow us to find methods that are not only performant but reliable.

\paragraph{Existing Benchmarks} We recognize that other holistic in-context learning benchmarks exist. BigBench is a large benchmark of 204 tasks that are beyond the capabilities of current LLMs. BigBench seeks to evaluate the few-shot abilities of state of the art large language models, focusing on performance metrics such as accuracy \cite{bigbench}. Similarly, HELM is another benchmark for language model in-context learning ability. Rather than only focusing on performance, HELM branches out and considers many other metrics such as robustness and bias \cite{helm}. Both BigBench and HELM focus on ranking different language model, while fix a generic instruction and prompt format. We instead choose to evaluate instruction induction / selection methods over a fixed set of models. We are the first ever evaluation script that compares different prompt-engineering methods head to head. 
\fi

\paragraph{In-Context Learning and Existing Benchmarks} As language models have scaled, in-context learning has emerged as a popular paradigm and remains ubiquitous among several autoregressive LLM families \cite{brown2020language, llama, bloom, gptneo, opt}. Benchmarks like BigBench \cite{bigbench} and HELM \cite{helm} have been created for the holistic evaluation of these models. BigBench focuses on few-shot abilities of state-of-the-art large language models, while HELM extends to consider metrics like robustness and bias. However, these benchmarks focus on evaluating and ranking \emph{language models}, and do not address the systematic evaluation of \emph{prompting methods}. Although contemporary work by \citet{yang2023improving} also aims to perform a similar systematic analysis of prompting methods, they focus on simple probability-based prompt selection while we evaluate a broader range of methods including trivial instruction baselines, curated manually selected instructions, and sophisticated automated instruction selection.

\paragraph{Automated Prompt Engineering Methods} There has been interest in performing automated prompt-engineering for target downstream tasks within ICL. This has led to the exploration of various prompting methods, ranging from simple heuristics such as selecting instructions with the lowest perplexity \cite{lowperplexityprompts}, inducing instructions from large language models using a few annotated input-output pairs \cite{ape}, to utilizing RL objectives to create discrete token sequences as prompts \cite{rlprompt}. However, these works restrict their evaluation to small sets of models and tasks with little intersection, hindering their objective comparison. %\mz{For paragraphs that only have one work in the last line, try to shorten the paragraph to squeeze in context.}

\paragraph{Understanding in-context learning} There has been much recent work attempting to understand the mechanisms that drive in-context learning. Studies have found that the selection of demonstrations included in prompts significantly impacts few-shot accuracy across most tasks \cite{whatmakesgoodicexamples, selectionmachinetranslation, knnprompting}. Works like \cite{fantasticallyorderedprompts} also show that altering the ordering of a fixed set of demonstrations can affect downstream accuracy. Prompts sensitive to demonstration permutation often exhibit lower accuracies \cite{relationsensitivityaccuracy}, making them less reliable, particularly in low-resource domains.

Our work aims to bridge these gaps by systematically evaluating the efficacy of popular instruction selection approaches over a diverse set of tasks and models, facilitating objective comparison. We evaluate these methods not only on accuracy metrics, but also on sensitivity metrics to glean additional insights. We recognize that other facets of prompting not covered by instruction engineering exist \cite{weichain, react, selfconsistency}, and defer these explorations to future work. 
\section{Our Approach}
We formulate the problem as an anisotropic diffusion process and the diffusion tensor is learned through a deep CNN directly from the given image, which guides the refinement of the output.

\begin{figure}[t]
\includegraphics[width=1.0\textwidth]{fig/CSPN_SPN2.pdf}
\caption{Comparison between the propagation process in SPN~\cite{liu2017learning} and CPSN in this work.}
\label{fig:compare}
\end{figure}

\subsection{Convolutional Spatial Propagation Network}
% demonstrate the thereom is hold when turns to be convolution.
Given a depth map $D_o \in \spa{R}^{m\times n}$ that is output from a network, and image $\ve{X} \in \spa{R}^{m\times n}$, our task is to update the depth map to a new depth map $D_n$ within $N$ iteration steps, which first reveals more details of the image, and second improves the per-pixel depth estimation results. 

\figref{fig:compare}(b) illustrates our updating operation. Formally, without loss of generality, we can embed the $D_o$ to some hidden space $\ve{H} \in \spa{R}^{m \times n \times c}$. The convolutional transformation functional with a kernel size of $k$ for each time step $t$ could be written as,
\begin{align}
    \ve{H}_{i, j, t + 1} &= \sum\nolimits_{a,b = -(k-1)/2}^{(k-1)/2} \kappa_{i,j}(a, b) \odot \ve{H}_{i-a, j-b, t} \nonumber \\
\mbox{where,~~~~}
    \kappa_{i,j}(a, b) &= \frac{\hat{\kappa}_{i,j}(a, b)}{\sum_{a,b, a, b \neq 0} |\hat{\kappa}_{i,j}(a, b)|}, \nonumber\\
    \kappa_{i,j}(0, 0) &= 1 - \sum\nolimits_{a,b, a, b \neq 0}\kappa_{i,j}(a, b)
\label{eqn:cspn}
\end{align}
where the transformation kernel $\hat{\kappa}_{i,j} \in \spa{R}^{k\times k \times c}$ is the output from an affinity network, which is spatially dependent on the input image. The kernel size $k$ is usually set as an odd number so that the computational context surrounding pixel $(i, j)$ is symmetric.
$\odot$ is element-wise product. Following~\cite{liu2017learning}, we normalize kernel weights between range of $(-1, 1)$ so that the model can be stabilized and converged by satisfying the condition $\sum_{a,b, a,b \neq 0} |\kappa_{i,j}(a, b)| \leq 1$. Finally, we perform this iteration $N$ steps to reach a stationary distribution.

% theorem, it follows diffusion with PDE 
%\addlinespace
\noindent\textbf{Correspondence to diffusion process with a partial differential equation (PDE).} \\
Similar with~\cite{liu2017learning}, here we show that our CSPN holds all the desired properties of SPN.
Formally, we can rewrite the propagation in \equref{eqn:cspn} as a process of diffusion evolution by first doing column-first vectorization of feature map $\ve{H}$ to $\ve{H}_v \in \spa{R}^{\by{mn}{c}}$.
\begin{align}
     \ve{H}_v^{t+1} = 
     \begin{bmatrix}
    1-\lambda_{0, 0}  & \kappa_{0,0}(1,0) & \cdots & 0 \\
    \kappa_{1,0}(-1,0)   & 1-\lambda_{1, 0} & \cdots & 0 \\
    \vdots & \vdots & \ddots & \vdots \\
    \vdots & \cdots & \cdots & 1-\lambda_{m,n} \\
\end{bmatrix} = \ve{G}\ve{H}_v^{t}
\label{eqn:vector}
\end{align}
where $\lambda_{i, j} = \sum_{a,b}\kappa_{i,j}(a,b)$ and $\ve{G}$ is a $\by{mn}{mn}$ transformation matrix. The diffusion process expressed with a partial differential equation (PDE) is derived as follows, 
\begin{align}
     \ve{H}_v^{t+1} &= \ve{G}\ve{H}_v^{t} = (\ve{I} - \ve{D} + \ve{A})\ve{H}_v^{t} \nonumber\\
     \ve{H}_v^{t+1} - \ve{H}_v^{t} &= - (\ve{D} - \ve{A}) \ve{H}_v^{t} \nonumber\\
     \partial_t \ve{H}_v^{t+1} &= -\ve{L}\ve{H}_v^{t}
\label{eqn:proof}
\end{align}
where $\ve{L}$ is the Laplacian matrix, $\ve{D}$ is the diagonal matrix containing all the $\lambda_{i, j}$, and $\ve{A}$ is the affinity matrix which is the off diagonal part of $\ve{G}$.

In our formulation, different from~\cite{liu2017learning} which scans the whole image in four directions~(\figref{fig:compare}(a)) sequentially, CSPN propagates a local area towards all directions at each step~(\figref{fig:compare}(b)) simultaneously, \ie with~\by{k}{k} local context, while larger context is observed when recurrent processing is performed, and the context acquiring rate is in an order of $O(kN)$.

In practical, we choose to use convolutional operation due to that it can be efficiently implemented through image vectorization, yielding real-time performance in depth refinement tasks.

Principally, CSPN could also be derived from loopy belief propagation with sum-product algorithm~\cite{kschischang2001factor}. However, since our approach adopts linear propagation, which is efficient while just a special case of pairwise potential with L2 reconstruction loss in graphical models. Therefore, to make it more accurate, we call our strategy convolutional spatial propagation in the field of diffusion process.

\begin{figure}[t]
\centering
\includegraphics[width=0.9\textwidth]{fig/hist.pdf}
\caption {(a) Histogram of RMSE with depth maps from~\cite{Ma2017SparseToDense} at given sparse depth points.  (b) Comparison of gradient error between depth maps with sparse depth replacement (blue bars) and with ours CSPN (green bars), where ours is much smaller. Check~\figref{fig:gradient} for an example. Vertical axis shows the count of pixels.}
\label{fig:hist}
\end{figure}

\subsection{Spatial Propagation with Sparse Depth Samples}
In this application, we have an additional sparse depth map $D_s$ (\figref{fig:gradient}(b)) to help estimate a depth depth map from a RGB image. Specifically, a sparse set of pixels are set with real depth values from some depth sensors, which can be used to guide our propagation process. 

Similarly, we also embed the sparse depth map $D_s = \{d_{i,j}^s\}$ to a hidden representation $\ve{H}^s$,  and we can write the updating equation of $\ve{H}$ by simply adding a replacement step after performing \equref{eqn:cspn}, 
\begin{align}
    \ve{H}_{i, j, t+1} = (1 - m_{i, j}) \ve{H}_{i, j, t+1}  +  m_{i, j} \ve{H}_{i, j}^s 
\label{eqn:cspn_sp}
\end{align}
where $m_{i, j} = \spa{I}(d_{i, j}^s > 0)$ is an indicator for the availability of sparse depth at $(i, j)$. 

In this way, we guarantee that our refined depths have the exact same value at those valid pixels in sparse depth map. Additionally, we propagate the information from those sparse depth to its surrounding pixels such that the smoothness between the sparse depths and their neighbors are maintained. 
Thirdly, thanks to the diffusion process, the final depth map is well aligned with image structures. 
This fully satisfies the desired three properties for this task which is discussed in our introduction (\ref{sec:intro}). 

% it performs a non-symmetric propagation where the information can only be diffused from the given sparse depth to others, while not the other way around.

% still follows PDE
In addition, this process is still following the diffusion process with PDE, where the transformation matrix can be built by simply replacing the rows satisfying $m_{i, j} = 1$ in $\ve{G}$ (\equref{eqn:vector}), which are corresponding to sparse depth samples, by $\ve{e}_{i + j*m}^T$. Here $\ve{e}_{i + j*m}$ is an unit vector with the value at $i + j*m$ as 1.
Therefore, the summation of each row is still $1$, and obviously the stabilization still holds in this case.

\begin{figure}[t]
\centering
\includegraphics[width=0.95\textwidth]{fig/fig2.pdf}
\caption{Comparison of depth map~\cite{Ma2017SparseToDense} with sparse depth replacement and with our CSPN \wrt smoothness of depth gradient at sparse depth points. (a) Input image. (b) Sparse depth points. (c) Depth map with sparse depth replacement. (d) Depth map with our CSPN with sparse depth points. We highlight the differences in the red box.}
\label{fig:gradient}
\end{figure}

Our strategy has several advantages over the previous state-of-the-art sparse-to-dense methods~\cite{Ma2017SparseToDense,LiaoHWKYL16}.
In \figref{fig:hist}(a), we plot a histogram of depth displacement from ground truth at given sparse depth pixels from the output of Ma \etal~\cite{Ma2017SparseToDense}. It shows the accuracy of sparse depth points cannot preserved, and some pixels could have very large displacement (0.2m), indicating that directly training a CNN for depth prediction does not preserve the value of real sparse depths provided. To acquire such property, 
one may simply replace the depths from the outputs with provided sparse depths at those pixels, however, it yields non-smooth depth gradient \wrt surrounding pixels. 
In~\figref{fig:gradient}(c), we plot such an example, at right of the figure, we compute Sobel gradient~\cite{sobel2014history} of the depth map along x direction, where we can clearly see that the gradients surrounding pixels with replaced depth values are non-smooth.
We statistically verify this in \figref{fig:hist}(b) using 500 sparse samples, the blue bars are the histogram of gradient error  at sparse pixels by comparing the gradient of the depth map with sparse depth replacement and of ground truth depth map. We can see the difference is significant, 2/3 of the sparse pixels has large gradient error.
Our method, on the other hand, as shown with the green bars in \figref{fig:hist}(b), the average gradient error is much smaller, and most pixels have zero error. In\figref{fig:gradient}(d), we show the depth gradients surrounding sparse pixels are smooth and close to ground truth, demonstrating the effectiveness of our propagation scheme. 

% Finally, in our experiments~\ref{sec:exp}, we validate the number of iterations $N$ and kernel size $k$ used for doing the CSPN.


\subsection{Complexity Analysis}
\label{subsec:time}

As formulated in~\equref{eqn:cspn}, our CSPN takes the operation of convolution, where the complexity of using CUDA with GPU for one step CSPN is $O(\log_2(k^2))$, where $k$ is the kernel size. This is because CUDA uses parallel sum reduction, which has logarithmic complexity. In addition,  convolution operation can be performed parallel for all pixels and channels, which has a constant complexity of $O(1)$. Therefore, performing $N$-step propagation, the overall complexity for CSPN is $O(\log_2(k^2)N)$, which is irrelevant to image size $(m, n)$.

SPN~\cite{liu2017learning} adopts scanning row/column-wise propagation in four directions. Using $k$-way connection and running in parallel, the complexity for one step is $O(\log_2(k))$. The propagation needs to scan full image from one side to another, thus the complexity for SPN is $O(\log_2(k)(m + n))$. Though this is already more efficient than the densely connected CRF proposed by~\cite{philipp2012dense}, whose implementation complexity with permutohedral lattice is $O(mnN)$, ours $O(\log_2(k^2)N)$ is more efficient since the number of iterations $N$ is always much smaller than the size of image $m, n$. We show in our experiments (\secref{sec:exp}), with $k=3$ and $N=12$, CSPN already outperforms SPN with a large margin (relative $30\%$), demonstrating both efficiency and effectiveness of the proposed approach.


\subsection{End-to-End Architecture}
\label{subsec:unet}
\begin{figure}[t]
\centering
\includegraphics[width=0.95\textwidth,height=0.45\textwidth]{fig/framework2.pdf}
\caption{Architecture of our networks with mirror connections for  depth estimation via transformation kernel prediction with CSPN (best view in color). Sparse depth is an optional input, which can be embedded into the CSPN to guide the depth refinement.}
\label{fig:arch}
\end{figure}

We now explain our end-to-end network architecture to predict both the transformation kernel and the depth value, which are the inputs to CSPN for depth refinement.
 As shown in \figref{fig:arch}, our network has some similarity with that from Ma \etal~\cite{Ma2017SparseToDense}, with the final CSPN layer that outputs a dense depth map.  
 
For predicting the transformation kernel $\kappa$ in \equref{eqn:cspn}, 
rather than building a new deep network for learning affinity same as Liu \etal~\cite{liu2017learning}, we branch an additional output from the given network, which shares the same feature extractor with the depth network. This helps us to save memory and time cost for joint learning of both depth estimation and transformation kernels prediction. 

Learning of affinity is dependent on fine grained spatial details of the input image. However, spatial information is weaken or lost with the down sampling operation during the forward process of the ResNet in~\cite{laina2016deeper}. Thus, we add mirror connections similar with the U-shape network~\cite{ronneberger2015u} by directed concatenating the feature from encoder to up-projection layers as illustrated by ``UpProj$\_$Cat'' layer in~\figref{fig:arch}. Notice that it is important to carefully select the end-point of mirror connections. Through experimenting three possible positions to append the connection, \ie after \textit{conv}, after \textit{bn} and after \textit{relu} as shown by the ``UpProj'' layer in~\figref{fig:arch} , we found the last position provides the best results by validating with the NYU v2 dataset (\secref{subsec:ablation}). 
In doing so, we found not only the depth output from the network is better recovered, and the results after the CSPN is additionally refined, which we will show the experiment section~(\secref{sec:exp}).
Finally we adopt the same training loss as~\cite{Ma2017SparseToDense}, yielding an end-to-end learning system.


%!TEX ROOT = ../../centralized_vs_distributed.tex

\section{{\titlecap{the centralized-distributed trade-off}}}\label{sec:numerical-results}

\revision{In the previous sections we formulated the optimal control problem for a given controller architecture
(\ie the number of links) parametrized by $ n $
and showed how to compute minimum-variance objective function and the corresponding constraints.
In this section, we present our main result:
%\red{for a ring topology with multiple options for the parameter $ n $},
we solve the optimal control problem for each $ n $ and compare the best achievable closed-loop performance with different control architectures.\footnote{
\revision{Recall that small (large) values of $ n $ mean sparse (dense) architectures.}}
For delays that increase linearly with $n$,
\ie $ f(n) \propto n $, 
we demonstrate that distributed controllers with} {few communication links outperform controllers with larger number of communication links.}

\textcolor{subsectioncolor}{Figure~\ref{fig:cont-time-single-int-opt-var}} shows the steady-state variances
obtained with single-integrator dynamics~\eqref{eq:cont-time-single-int-variance-minimization}
%where we compare the standard multi-parameter design 
%with a simplified version \tcb{that utilizes spatially-constant feedback gains
and the quadratic approximation~\eqref{eq:quadratic-approximation} for \revision{ring topology}
with $ N = 50 $ nodes. % and $ n\in\{1,\dots,10\} $.
%with $ N = 50 $, $ f(n) = n $ and $ \tau_{\textit{min}} = 0.1 $.
%\autoref{fig:cont-time-single-int-err} shows the relative error, defined as
%\begin{equation}\label{eq:relative-error}
%	e \doteq \dfrac{\optvarx-\optvar}{\optvar}
%\end{equation}
%where $ \optvar $ and $ \optvarx $ denote the the optimal and sub-optimal scalar variances, respectively.
%The performance gap is small
%and becomes negligible for large $ n $.
{The best performance is achieved for a sparse architecture with  $ n = 2 $ 
in which each agent communicates with the two closest pairs of neighboring nodes. 
This should be compared and contrasted to nearest-neighbor and all-to-all 
communication topologies which induce higher closed-loop variances. 
Thus, 
the advantage of introducing additional communication links diminishes 
beyond}
{a certain threshold because of communication delays.}

%For a linear increase in the delay,
\textcolor{subsectioncolor}{Figure~\ref{fig:cont-time-double-int-opt-var}} shows that the use of approximation~\eqref{eq:cont-time-double-int-min-var-simplified} with $ \tilde{\gvel}^* = 70 $
identifies nearest-neighbor information exchange as the {near-optimal} architecture for a double-integrator model
with ring topology. 
This can be explained by noting that the variance of the process noise $ n(t) $
in the reduced model~\eqref{eq:x-dynamics-1st-order-approximation}
is proportional to $ \nicefrac{1}{\gvel} $ and thereby to $ \taun $,
according to~\eqref{eq:substitutions-4-normalization},
making the variance scale with the delay.

%\mjmargin{i feel that we need to comment about different results that we obtained for CT and DT double-intergrator dynamics (monotonic deterioration of performance for the former and oscillations for the latter)}
\revision{\textcolor{subsectioncolor}{Figures~\ref{fig:disc-time-single-int-opt-var}--\ref{fig:disc-time-double-int-opt-var}}
show the results obtained by solving the optimal control problem for discrete-time dynamics.
%which exhibit similar trade-offs.
The oscillations about the minimum in~\autoref{fig:disc-time-double-int-opt-var}
are compatible with the investigated \tradeoff~\eqref{eq:trade-off}:
in general, 
the sum of two monotone functions does not have a unique local minimum.
Details about discrete-time systems are deferred to~\autoref{sec:disc-time}.
Interestingly,
double integrators with continuous- (\autoref{fig:cont-time-double-int-opt-var}) ad discrete-time (\autoref{fig:disc-time-double-int-opt-var}) dynamics
exhibits very different trade-off curves,
whereby performance monotonically deteriorates for the former and oscillates for the latter.
While a clear interpretation is difficult because there is no explicit expression of the variance as a function of $ n $,
one possible explanation might be the first-order approximation used to compute gains in the continuous-time case.
%which reinforce our thesis exposed in~\autoref{sec:contribution}.

%\begin{figure}
%	\centering
%	\includegraphics[width=.6\linewidth]{cont-time-double-int-opt-var-n}
%	\caption{Steady-state scalar variance for continuous-time double integrators with $ \taun = 0.1n $.
%		Here, the \tradeoff is optimized by nearest-neighbor interaction.
%	}
%	\label{fig:cont-time-double-int-opt-var-lin}
%\end{figure}
}

\begin{figure}
	\centering
	\begin{minipage}[l]{.5\linewidth}
		\centering
		\includegraphics[width=\linewidth]{random-graph}
	\end{minipage}%
	\begin{minipage}[r]{.5\linewidth}
		\centering
		\includegraphics[width=\linewidth]{disc-time-single-int-random-graph-opt-var}
	\end{minipage}
	\caption{Network topology and its optimal {closed-loop} variance.}
	\label{fig:general-graph}
\end{figure}

Finally,
\autoref{fig:general-graph} shows the optimization results for a random graph topology with discrete-time single integrator agents. % with a linear increase in the delay, $ \taun = n $.
Here, $ n $ denotes the number of communication hops in the ``original" network, shown in~\autoref{fig:general-graph}:
as $ n $ increases, each agent can first communicate with its nearest neighbors,
then with its neighbors' neighbors, and so on. For a control architecture that utilizes different feedback gains for each communication link
	(\ie we only require $ K = K^\top $) we demonstrate that, in this case, two communication hops provide optimal closed-loop performance. % of the system.}

Additional computational experiments performed with different rates $ f(\cdot) $ show that the optimal number of links increases for slower rates: 
for example, 
the optimal number of links is larger for $ f(n) = \sqrt{n} $ than for $ f(n) = n $. 
\revision{These results are not reported because of space limitations.}
\section{Conclusions}
\label{sec:conclusions}

In this paper, we apply shared-workload techniques at the \sql level for
improving the throughput of \qaasl systems without incurring in additional
query execution costs. Our approach is based on query rewriting for grouping
multiple queries together into a single query to be executed in one go. This
results in a significant reduction of the aggregated data access done by the
shared execution compared to executing queries independently.

%execution times and costs of the shared scan operator when
%varying query selectivity and predicate evaluation. We observed that for
%\athena, although the cost only depends on the amount of data read, it is
%conditioned to its ability to use its statistics about the data. In some cases
%a wrong query execution plan leads to higher query execution costs, which the
%end-user has to pay. 

%\bigquery's minimum query execution cost is determined by
%the input size of a query.  However, the query cost can increase depending not
%just in the amount of computation it requires, but in the mix of resources the
%query requires.  

We presented a cost and runtime evaluation of the shared operator driving data access costs. 
Our experimental study using the TPC-H benchmark confirmed the benefits of our
query rewrite approach. Using a shared execution approach reduces significantly
the execution costs. For \athena, we are able to make it 107x cheaper and for
\bigquery, 16x cheaper taking into account Query 10 which we cannot execute,
but 128x if it is not taken into account. Moreover, when having queries that do
not share their entire execution plan, i.e., using a single global plan, we
demonstrated that it is possible to improve throughput and obtain a 10x cost
reduction in \bigquery.

%We followed the TPC systems pricing guideline for
%computing how expensive is to have a TPC-H workload working on the evaluated
%\qaasl systems. The result is that even though we are able to reduce overall
%costs a TPC-H workload in 15x for \bigquery (128x excluding query 10 which we
%could not optimize) and in 107x for \athena, the overall price is at least 10x
%more expensive than the cheapest system price published by the TPC.

There are multiple ways to extend our work. The first is
to implement a full \sql-to-\sql translation layer to encapsulate the proposed
per-operator rewrites.  Another one is to incorporate the initial work on
building a cost-based optimizer for shared execution
\cite{Giannikis:2014:SWO:2732279.2732280} as an external component for \qaasl
systems.  Moreover, incorporating different lines of work (e.g., adding
provenance computation \cite{GA09} capabilities) also based on query
rewriting is part of our future work to enhance our system.


\vspace{-5pt}
\paragraph{Acknowledgements} \boc{This research is supported in part by NSF IIS-1514118 and a Google Faculty Research Award. We also gratefully acknowledge a GPU donation from Facebook.}

\clearpage





\bibliographystyle{splncs}
\bibliography{snapAngle}





\section{Supplementary Material}







\subsection{Justification for predicting snap angle in azimuth only}

This section accompanies Sec. 3.1 in the main paper.

As discussed in the paper, views from a horizontal camera position (elevation 0$^{\circ}$) are typically more plausible than other elevations due to the human recording bias.  The human photographer typically holds the camera with the ``top" to the sky/ceiling.  

Figure~\ref{fig:supp_elevation} shows examples of rotations in elevation for several panoramas.
We see that the recording bias makes the 0 (canonical) elevation fairly good as it is.  In contrast, rotation in elevation often makes the cubemaps appear tilted (e.g., the building in the second row). Without loss of generality, we focus on snap angles in azimuth only, and jointly
optimize the front/left/right/back faces of the cube.






\begin{figure*}[t!]
\centering
\renewcommand{\tabcolsep}{0pt}
\includegraphics[width=1\columnwidth]{images/eccv_supp.pdf}%change to qual_pdf for a larger one

\caption{Example of cubemaps when rotating in elevation. Views from a horizontal camera position (elevation 0$^{\circ}$) are more informative than others due to the natural human recording bias. In addition, rotation in elevation often makes cubemap faces appear tilted (e.g., building in the second row).  Therefore, we optimize for the azimuth snap angle only.}
\label{fig:supp_elevation}
\end{figure*}


%\clearpage





\subsection{Details on network architecture}

This section accompanies Sec. 3.2 in the main paper.

\begin{figure*}[t]
\centering
\renewcommand{\tabcolsep}{0pt}
\includegraphics[width=0.95\columnwidth]{images/supp_net.pdf}%change to qual_pdf for a larger one
\caption{A detailed diagram showing our network architecture. In the top, the small schematic shows the connection between each network module. Then we present the details of each module in the bottom. Our network proceeds from left to right. The \textit{feature extractor} consists of a sequence of convolutions (with kernel size and convolution stride written under the diagram) followed by a fully connected layer. In the bottom, ``FC'' denotes a fully connected layer and ``ReLU'' denotes a rectified linear unit. The \textit{aggregator} is a recurrent neural network. The ``Delay'' layer stores its current internal hidden state and outputs them at the next time step. In the end, the \textit{predictor} samples an action stochastically based on the multinomial pdf from the Softmax layer.}
\label{fig:supp_net}
\end{figure*}
    





Recall that our framework consists of four modules: a \textit{rotator}, a \textit{feature extractor}, an \textit{aggregator}, and a \textit{snap angle predictor}. At each time step, it processes the data and produces a cubemap (\textit{rotator}), extracts learned features (\textit{feature extractor}), integrates information over time (\textit{aggregator}), and predicts the next snap angle (\textit{snap angle predictor}). We show the details of each module in Figure~\ref{fig:supp_net}.



\subsection{Training details for \textsc{Pano2Vid(P2V)~\cite{su2016pano2vid}-adapted}}\label{sec:pano}

This section accompanies Sec 4.1 in the main paper.

For each of the activity categories in our 360$^{\circ}$ dataset (Disney, Parade, etc.), 
we query Google Image Search engine and then manually filter out irrelevant images. We obtain about $300$ images for each category.   
We use the Web images as positive training samples and randomly sampled panorama subviews as negative training sampling.

\clearpage

\subsection{Interface for collecting important objects/persons}
 
This section accompanies Sec. 4.2 in the main paper.

We present the interface for Amazon Mechanical Turk that is used to collect important objects and persons. The interface is developed based on~\cite{profx}. We present crowdworkers the panorama and instruct them to label any important objects with a bounding box---as many as they wish. A total of $368$ important objects/persons are labeled. The maximum number of labeled important objects/persons for a single image is $7$.


\begin{figure*}[t]
\centering
\renewcommand{\tabcolsep}{0pt}
\includegraphics[width=0.95\columnwidth]{images/supp_label_object.pdf}%change to qual_pdf for a larger one
\caption{Interface for Amazon Mechanical Turk when collecting important objects/persons.
We present crowdworkers the panorama and instruct them to label any important objects with a bounding box---as many as they wish.}
\label{fig:supp_important}
\end{figure*}




\subsection{More results when ground-truth objectness maps are used as input}
\bo{This section accompanies Sec. 4.2 in the main paper.}



  
\bo{We first got manual labels for the pixel-wise foreground for 50 randomly selected panoramas ($200$ faces). The IoU score between the pixel objectness prediction and ground truth is 0.66 with a recall of 0.82, whereas alternate foreground methods we tried \cite{5432215,jiang2013saliency} obtain only 0.35-0.40 IoU and 0.67-0.74 recall  on the same data.}
 
 \bo{We then compare results when either ground-truth foreground or pixel objectness is used as input (without re-training our model) and report the foreground disruption with respect to the ground-truth. ``Best possible" is the oracle result. Pixel objectness serves as a good proxy for ground-truth foreground maps. Error decreases with each rotation. Table~\ref{tab:gt} pinpoints to what extent having even better foreground inputs would also improve snap angles.}
 
 
 
 
 
 
 
 
 \begin{table}[t]
\centering
 \begin{tabular}{ |c|c|c|c|c|c|c|c|}
\hline

Budget (T)  & 2 & 4 & 6 & 8  & 10 & Best Possible & \textsc{Canonical}    \\ \hline
Ground truth input & 0.274 & 0.259 & 0.248 & 0.235 & 0.234 & 0.231 & 0.352\\ \hline
Pixel objectness input & 0.305 & 0.283 & 0.281 & 0.280  & 0.277 & 0.231 & 0.352\\ \hline



\end{tabular}

\vspace{4pt}
\caption{Decoupling pixel objectness and snap angle performance: 
 error (lower is better) when ground truth or pixel objectness is used as input.}
\label{tab:gt}
\end{table}

 
 
 


\subsection{Interface for user study on perceived quality}

This section accompanies Sec. 4.3 in the main paper.

We present the interface for the user study on perceived quality in Figure~\ref{fig:supp_human}. Workers were required to rate each image into one of five categories: (a) The first set is significantly better, (b) The first set is somewhat better, (c) Both sets look similar, (d) The second set is somewhat better and (e) The second set is significantly better. We also instruct them to avoid to choose option (c) unless it is really necessary. Since the task can be ambiguous and subjective, we issued each task to 5 distinct workers. Every time a comparison between the two sets receives a rating of category (a), (b), (c), (d) or (e) from any of the 5 workers, it receives 2, 1, 0, -1, -2 points, respectively. We add up scores from all five workers to collectively decide which set is better.


\begin{figure*}[t]
\centering
\renewcommand{\tabcolsep}{0pt}
\includegraphics[width=0.95\columnwidth]{images/supp_human.pdf}%change to qual_pdf for a larger one
%\vspace{-10pt}
\caption{Interface for user study on perceived quality. Workers were required to rate each image into one of five categories. We issue each sample to 5 distinct workers.}
\label{fig:supp_human}
\end{figure*}











\subsection{The objectness map input for failure cases}
\bo{This section accompanies Sec. 4.3 in the main paper.}

\bo{Our method fails to preserve object integrity if pixel objectness fails to recognize foreground objects. Please see Fig.~\ref{fig:fail} for examples. The round stage is not found in the foreground, and so ends up distorted by our snap angle prediction method. In addition, the current solution cannot effectively handle the case when a foreground object is too large to fit in a single cube face. 
}



\begin{figure}

\centering
\renewcommand{\tabcolsep}{0pt}
\includegraphics[width=1\columnwidth]{images/snap_angle_fail_pixelmap.pdf}%change to rl.eps for a single column version

\caption{\bo{Pixel objectness map (right) for failure cases of snap angle prediction. In the top row, the round stage is not found in the foreground, and so ends up distorted by our snap angle.}}
\label{fig:fail}
\end{figure}

\clearpage








\subsection{Additional cubemap output examples}

This section accompanies Sec. 4.3 in the main paper.

Figure~\ref{fig:supp_qual1} presents additional cubemap examples.  
Our method produces cubemaps that place important objects/persons in the same cube face to preserve the foreground integrity. For example, in the top right of Figure~\ref{fig:supp_qual1}, our method places the person in a single cube face whereas the default cubemap splits the person onto two different cube faces.

Figure~\ref{fig:supp_qual3} shows failure cases.
A common reason for failure is that pixel objectness~\cite{jain2017pixel} does not recognize some important objects as foreground. For example, in the top right of Figure~\ref{fig:supp_qual3}, our method creates a distorted train by
splitting the train onto three different cube faces because pixel objectness does not recognize the train as foreground.



\begin{figure*}[t]
\centering
\renewcommand{\tabcolsep}{0pt}
\includegraphics[width=1\columnwidth]{images/supp_s1.pdf}%change to qual_pdf for a larger one
\vspace{-10pt}
\caption{Qualitative examples of default \textsc{Canonical} cubemaps and our snap angle cubemaps.  Our method produces cubemaps that place important objects/persons in the same cube face to preserve the foreground integrity.}
\label{fig:supp_qual1}
\end{figure*}








\begin{figure*}[t]
\centering
\renewcommand{\tabcolsep}{0pt}
\includegraphics[width=1\columnwidth]{images/supp_failure.pdf}%change to qual_pdf for a larger one
\caption{Qualitative examples of default \textsc{Canonical} cubemaps and our snap angle cubemaps. We show failure cases here. In the top left, pixel objectness~\cite{jain2017pixel} does not recognize the stage as foreground, and therefore our method splits the stage onto two different cube faces, creating a distorted stage. In the top right, our method creates a distorted train by
splitting the train onto three different cube faces because pixel objectness does not recognize the train as foreground.}
\label{fig:supp_qual3}
\end{figure*}


\subsection{Setup details for recognition experiment}

This section accompanies Sec. 4.4 in the main paper.

Recall our goal for the recognition experiment is to distinguish the four activity categories in our 360 dataset (Disney, Parade, etc.). We build a training dataset by querying Google Image Search engine for each activity category and then manually filtering out irrelevant images.  These are the same as the positive images in Sec.~\ref{sec:pano} above.

We use Resnet-101 architecture~\cite{he2016deep} as our classifier. The network is first pre-trained on Imagenet~\cite{russakovsky2015imagenet} and then we finetune
 it on our dataset with SGD and a mini-batch size of 32. The learning rate starts from 0.01 with a weight decay of 0.0001 and a momentum of 0.9. We train the network 
until convergence and select the best model based on a validation set.








\end{document}
