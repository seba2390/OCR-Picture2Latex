\section{Introduction}

Crowdfunding provides individuals and organizations with the opportunity to raise funds for 
innovative projects, charitable causes, or public services. By raising the visibility of these fund-raising campaigns to Internet scale, crowdfunding sites like Kickstarter, Indiegogo, and Donorschoose dramatically increase the pool of potential backers and chances of successful fund-raising. Despite this potential, most crowdfunding campaigns still fail to reach their funding goals. On Kickstarter, the largest crowdfunding platform, only 36\% of the campaigns, were successfully funded in 2016~\cite{kickstarterstats}.

In %Kickstarter's 
the prevalent
\textit{all-or-nothing} funding model, campaigns that fail to meet their funding goals receive none of the contributions: even if a campaign attracts 99\% of its funding goal, it will not be funded. Currently, 1445 Kickstarter campaigns are unsuccessful, despite having raised 81-99\% of their goal~\cite{kickstarterstats}. In contrast, campaigns that exceed their goals keep all contributions; `Exploding Kittens' and `Pebble' raised 87,825\% and 4,067\% of their funding goals respectively. This raises the question: \textit{can we redistribute contributions to successfully fund more campaigns?}

Clearly, a redistribution where excess funds from over-funded campaigns are given to under-funded ones, can lead to an overall increase in successful campaigns. A na\"ive redistribution, however, can be detrimental to both the overall quality of campaigns and to the degree of funding. For example, if campaign organizers believe that extra funds will eventually be redistributed in their favor, they may be less motivated to produce higher quality campaigns; if funders think their donations could end up backing campaigns that they do not like, they may not be willing to contribute.

Therefore, redistribution must be done carefully. Consider, a donor, Sandy, who backs three simultaneous campaigns. Assuming that Sandy wishes all three campaigns to succeed regardless of how other donors give, she may be willing to accept a redistribution of her funds across the three campaigns if it allows them to meet their funding goal. She is less likely to accept a redistribution that allocates her funds to other campaigns that she does not support. Redistribution schemes may either honor Sandy's giving preferences, by only redistributing her funds among the campaigns she has contributed to (\textit{\cpr schemes}) or ignore Sandy's preferences and redistribute funds to any active campaign (\textit{\car schemes}).


In this paper we explore several possible crowdfunding donation redistribution schemes. We begin by defining several archetypes that represent intuitive redistribution policies (\sref{sec:model}). We then analyze the potential improvements to \textit{efficiency} --- measured in terms of campaigns successfully funded\footnote{Viewing crowdfunding platforms as online markets where the goal is to match donor contributions to successful campaigns, the percentage of successful campaigns within a platform is a good approximation of its efficiency~\cite{wash-returnrule}.} --- achievable by these redistribution schemes using real crowdfunding data from \lag. We find that an aggressive \car can boost campaign success rates from 37\% to 79\%, but such a policy comes at the cost of completely ignoring donor preferences. We build a case for \cpr policies that strike a balance between increasing successful campaigns from 37\% to 48\% and respecting giving preferences (\sref{sec:results}). Finally, we discuss the potential implications of various redistribution policies when implemented on crowdfunding platforms (\sref{sec:discussion}). We contextualize our work within the existing research on crowdfunding platforms that aim to improve success outcomes for campaigns and platforms (\sref{sec:background}). 