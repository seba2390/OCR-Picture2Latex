\subsection{Redistribution Schemes}
%\subsection{The Optimization Problem}
\label{sec:model}

We analyze four redistribution schemes broadly classified into \textit{\car} or \textit{\cpr} schemes. \Car schemes allow a donor's contributions to be redistributed to any campaign within the platform. These include \textit{na\"ive redistribution} and \textit{repurposing} schemes. \Cpr schemes only shuffle a donor's contributions within the set of campaigns the donor contributed to. These include both \textit{unordered-} and \textit{ordered-} \cpr schemes. With ordering, \cpr ensures that if a donor contributes more to one campaign over another then even after a redistribution, he/she would still contribute more to that campaign.

We formally define each of these schemes within the framework of an optimization problem where the goal is to maximize the number of campaigns that meet their goals.

%We formulate the optimization problem of maximizing the number of campaigns that meets their goals with the help of a linear program.

Let $n$ be the number of distinct donors that donated on the \lag platform and $m$ be the number of distinct campaigns. 

We represent an actual contribution a donor $i \in {1,..., n}$ made to a campaign $j \in {1,..., m}$ with $A_{i,j}$. A campaign, $j$, has a fund-raising goal of $G_j$. We represent whether a campaign has met its goal with a success indicator variable, $I_j$.
\[I_j =
\begin{cases}
    1, & \text{if} \ \displaystyle \sum_{i=1}^{n} A_{i, j} \geq G_j\\
    0, & \text{otherwise}
\end{cases}\]

Every campaign has a start date $s_j$ and an end date $e_j$. Each contribution $A_{i,j}$ is made on a specific date\footnote{If a donor, $i$, makes multiple contributions to the same campaign, $j$, we encode only one contribution $A_{i,j}$ equal to the sum of all such contributions and set $d_{i,j}$ equal to the date of the first contribution.} $d_{i,j}$: $s_j \leq d_{i, j} \leq e_j$.

After a redistribution of funds, we denote the donation a donor $i$ makes to a campaign $j$ with $R_{i, j}$. We now represent whether a campaign meets its goal with another success indicator variable, $I'_j$.
\[I'_j \leq \frac{\sum_{i = 1}^{n} R_{i, j}}{G_j}
\] 

Our redistribution goal is to maximize the number of successful projects,
\[\max \sum_{j = 1}^{m} I'_j \]

\textit{All four schemes must satisfy the following three constraints.} 

\begin{enumerate}[label={[\bf\sc{All\arabic*}]},wide =1em]
\item \emph{Once a winner, always a winner:} If a campaign met its goal without redistribution, a redistribution of funds should not cause this campaign to fail. 
\[\forall j, \;\ I'_j \geq I_j \]

\item \emph{Fixed Budget:} Each donor cannot give more than the sum of his/her original contributions across all campaigns and a donor cannot make a negative contribution.
\[ \forall i, \;\ \sum_{j = 1}^{m} R_{i,j} \leq \sum_{j = 1}^{m} A_{i,j}\]
\[\forall i,j, \;\ R_{i, j} \geq 0\]

\item \emph{Redistribute across live, overlapping campaigns:}
For each contribution $A_{i, j}$ made on $d_{i,j}$, we define overlapping campaigns as follows:
\[\mathcal{O}_{i, j} = \{x : \ e_x \geq d_{i, j} \wedge s_x \leq e_j \}\]

%\[\mathcal{O}_{i, j} = \{u, c: \ e_c \geq d_{i, j} \wedge d_{u, c} \leq e_j\}\]

%\[\mathcal{O}_{i, j} = \{x: \ e_x \geq d_{i, j} \wedge d_{i, x} \leq e_j\}\]

%Figure \ref{fig:overlap} visualizes the overlap region of a particular contribution when restricted to a single donor $x$. 

%The following constraint ensures that we do not redistribute funds from a contribution to campaigns that have already ended or campaigns that the user contributed to after the current campaign ended:

The following constraint ensures that we redistribute funds only within overlapping campaigns.

\[\forall i,j, \;\ \sum_{i, y \in \mathcal{O}_{i, j}} R_{i, y} \leq \sum_{i, x \in \mathcal{O}_{i, j}} A_{i, x}\]

%\[\forall R_{i, j}, \;\ R_{i, j} \leq \sum_{x \in \mathcal{O}_{i, j}} A_{i, x}\]

\end{enumerate}

We also try to eliminate the following unfavorable redistributions with the following optional constraints. We drop these \textit{nice} constraints if the optimization problem becomes infeasible:
\begin{enumerate}[label={[\bf\sc{Nice\arabic*}]},wide =1em]
    \item \textit{Avoid allocating more funds to a previously overfunded campaign:}
    \[\forall j, \;\ I_j = 1 \rightarrow \sum_{i=1}^n R_{i, j} \leq \sum_{i=1}^n A_{i, j}\]
    
    \item \textit{Avoid allocating more funds to a failed campaign:} If no redistribution of funds can save a campaign, then we should not allocate additional funds to that campaign. 
    \[\forall j, \;\ I'_j = 0 \rightarrow \sum_{i=1}^n R_{i, j} \leq \sum_{i=1}^n A_{i, j}\]
    
    \item \textit{Avoid surpassing the goal of a previously unsuccessful campaign:} 
    \[\forall j, \;\ I_j = 0 \rightarrow  \sum_{i=1}^n R_{i, j} \leq G_j \]
\end{enumerate}


\subsubsection{\Car}

\paragraph{Na\"ive redistribution} 

This scheme does not require any additional constraints to the base constraints listed above.


\paragraph{Repurposing} 

Repurposing also requires that we only redistribute funds from failed campaigns. We encode this with the following constraint:

\begin{enumerate}[label={[\bf\sc{Rep}]},wide =1em]
    \item \textit{Winners keep all}: 
    \[\forall i,j, \;\ I_j = 1 \rightarrow R_{i, j} = A_{i, j} \]
\end{enumerate}

No other redistribution scheme imposes the ``winners keep all" constraint.

\subsubsection{\Cpr}

Our two \cpr schemes must also ensure the following:

\begin{enumerate}[label={[\bf\sc{CP\arabic*}]},wide =1em]

\item \emph{No new campaigns:} For a given donor, we should only redistribute his/her funds among the campaigns he/she contributed to.
\[\forall i,j, \;\ A_{i, j} = 0 \rightarrow R_{i, j} = 0 \]

\item \emph{Redistribute across live, overlapping campaign contributions:} This constraint tightens the overlap constraints of na\"ive and repurposing redistribution schemes to overlapping \textit{contributions} by the same donor. We redefine overlapping campaign contributions as follows:

\[\mathcal{O}_{i, j} = \{x: \ e_x \geq d_{i, j} \wedge d_{i, x} \leq e_j\}\]

Figure \ref{fig:overlap} visualizes the overlap region of a particular contribution when restricted to a single donor $u$. 

The following constraint ensures that we do not redistribute funds from a contribution to campaigns that have already ended or campaigns that the donor contributed to after the current campaign ended:

\[\forall i,j, \;\ R_{i, j} \leq \sum_{x \in \mathcal{O}_{i, j}} A_{i, x}\]

\begin{figure}[htb]
    \centering
    \includegraphics[width=1\linewidth]{figures/overlap.pdf}
    \caption{The contribution made by donor $u$ on $d_{u,2}$ for campaign $2$ overlaps with contributions made on $d_{u,1}$ and $d_{u,4}$. 
    For donor $u$, a redistribution of funds to campaign 2, can only be allowed from contributions made to campaign 1 and 4.}
    \label{fig:overlap}
\end{figure}

\end{enumerate}

\paragraph{Ordered \cpr}

Finally, the ordered \cpr scheme requires the following constraint:

\begin{enumerate}[label={[\bf\sc{Order}]},wide =1em]
\item \emph{Preserve Preference Ordering:} If a donor contributes more funds to one campaign compared to another, both absolutely and relatively to their respective goals, then a redistribution of funds should preserve this preference ordering.
\[\forall i,x,y, \;\  (A_{i, x} > A_{i, y}) \land (A_{i, x}/G_x > A_{i, y}/G_y) \\
\rightarrow  R_{i, x} \geq R_{i, y} \]

We relax preference ordering to eliminate contributions to \textit{failed} campaigns (i.e. campaigns that do not succeed for any feasible redistribution) with the following constraint instead:
\begin{multline*}
% \forall x,y, \;\  A_{i, x} > A_{i, y} \land I'_x=1 \land I'_y=1 \rightarrow I'_x \times R_{i,x} \geq (I'_x + 2(I'_y - 1))\times R_{i,y}
\forall i,x,y, \;\  (A_{i, x} > A_{i, y}) \land (A_{i, x}/G_x > A_{i, y}/G_y) \\
\land (I'_x = 1) \land (I'_y = 1) 
\rightarrow R_{i, x} \geq R_{i, y}
\end{multline*}
\end{enumerate}


