\section{Background}
\label{sec:background}

The goal of all crowdfunding platforms is generally the same: to match a campaign's needs with funding from donors. Campaigns are only successful if a sufficient number of donors \textit{coordinate} to contribute to a project. 

\subsection{Crowdfunding Mechanisms}

Crowdfunding campaigns are often used to fund discrete public goods that require a certain amount of money to be raised to be useful~\cite{belleflamme, greenberg2012crowdfunding, wash-returnrule}. Kickstarter employs an \textit{all-or-nothing} model or \textit{return rule} where a project collects money only if the funding goal is met. The contrasting model supported by IndieGoGo is a \textit{keep-it-all} or \textit{direct donation} model where donations are retained even if the funding goal is not met. These crowdfunding mechanisms have been shown to have a significant impact on donor perceptions of campaigns, donor willingness to contribute, and the eventual success of campaigns~\cite{cumming2014crowdfunding, wash-returnrule}.

Specifically, Wash and Solomon showed through a series of controlled experiments that the return rule increases donors' willingness to donate to riskier projects and thus more accurately reflects individual preferences rather than the funding of projects that are more likely to be funded~\cite{wash-returnrule}. However, while more projects are successfully funded through the return rule, this benefit comes with a reduction in efficiency 
when individuals donate to their own preferences rather than those of the crowd.

\textit{These results motivate our work on understanding whether mechanisms like redistribution can simultaneously increase the number of successfully funded projects while respecting individual preferences.}

\subsection{Social Proof and Coordination}

Crowdfunding campaigns rely heavily on social proof. Displaying information like total funds raised and the number of donors can signal to the individual the collective valuation of the crowd. Participation by other donors signals the quality and credibility of the campaign, removes apprehensions, provides evidence of reciprocity, and establishes norms for how much to donate. These signals provide critical evidence that helps donors coordinate around supporting high value campaigns~\cite{mollick2014dynamics, cotterill2011impacts}.

Unfortunately, crowdfunding platforms do not always give donors accurate information about the crowd's beliefs~\cite{wash-time}: the crowd's valuation of a project induced by donation amounts on crowdfunding sites may be (i) delayed, 
(ii) misrepresented, 
or (iii) overshadowed by other projects~\cite{wash-time, andreoni1988free, wash-skew}.\footnote{Solomon et al. show that highly successful star projects can actually hinder other projects~\cite{wash-skew}.} As a result, campaigns can fail to reach their funding goal and the overall distribution of donations may not resemble the crowd's actual valuations. Codo~\cite{codo} is one crowdfunding system that can potentially mitigate this issue by allowing individual donors to make independent valuations that are stipulated on the crowd's valuations. 

In relation to these works, \textit{donation redistribution can be construed as a coordination of donors' funds}. 


\subsection{Improving Outcomes}

Recent research investigated several different factors influencing crowdfunding outcomes. Gerber et al. describe the motivations of both campaign organizers and donors for using crowdfunding as a fundraising tool~\cite{Gerber:2013:CMD:2562181.2530540}. Hui et al. explore the role that a crowdfunding project's community can play in its success~\cite{Hui:2014:URC:2531602.2531715}. It has also been demonstrated that the use of persuasive language~\cite{Mitra:2014:LGP:2531602.2531656} and status updates~\cite{xu2014show} can influence the chances of crowdfunding success. 

Social networks can also be leveraged to increase donations. A donation request or tagging from family or others who have donated can increase individual contributions~\cite{cotterill2011impacts}. Social networks can be used by organizers to improve fundraising outcomes if the existing networks can be activated and expanded~\cite{Hui:2014:ULS:2598510.2598539}. 
Finally, the timing of donations is a factor that affects donations~\cite{wash-time}, and donors are more willing to donate more to projects that are nearing completion~\cite{complete-effect}.

Real crowdfunding sites use a variety of incentives for donors to donate. Direct inducements include gifts, stretch goals, early access, and even equity.

\textit{Our work explores several possible automated reallocation schemes that can improve outcomes for both campaign organizers and donors.}


\subsection{Redistribution in the Wild}

Giving societies or donor clubs are organizations that explicitly gather funds from donors and allocates them toward specific projects. These giving societies are usually unified under high-level causes (e.g., education, environmental conservation, etc.). In this manner, the donation process is re-framed from being a highly personal choice to a collective action. The donor slightly relaxes their personal autonomy to defer to the valuations and expertise of the collective. Giving clubs fit within our \cpr schemes as donors implicitly agree to a redistribution of their funds across the subset of campaigns managed by the giving club.

 Donorschoose is a crowdfunding site focused on education that specifically targets high-poverty public schools to increase classroom engagement. Success rates are relatively high on DonorsChoose ($\approx 74\%$). Projects are vetted by the website staff and funds to an unsuccessful project are not refunded, but repurposed. Donors can choose alternate projects to fund instead or allow the platform operators or the teacher whose campaign was unsuccessful to repurpose the funds as she sees fit. \textit{We compare repurposing to other redistribution effects.}
 