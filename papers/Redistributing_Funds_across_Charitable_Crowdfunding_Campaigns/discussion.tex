\section{Discussion}
\label{sec:discussion}

Even though our empirical analysis shows that redistribution leads to clear efficiency gains, successfully implementing redistribution within a crowdfunding platform is not without complications.  A na\"ive redistribution scheme can be both objectionable and detrimental. Taking funds from `rich' campaigns to give to `poor' ones may be seen as unfair and a violation of donor intentions.  Other secondary effects could also arise. The overall quality of funded campaigns may drop due to weaker projects being funded. If funders believe their funds may back projects that they do not deem worthy, they may also be less likely to contribute. In this section we discuss these and other challenges. We consider the incentives from the perspective of donors, campaign organizers, and platform designers and describe the implications of redistribution. 

\subsection{For Donors}


Donors' reasons and motivations for giving are diverse~\cite{Gerber:2013:CMD:2562181.2530540}. 
Donors are often emotionally invested in the campaigns they donate to: 
taking funds from one campaign and giving it to another may cause donors to feel slighted. 
Imagine donating \$50 to produce a film you are excited to see only to find that your funds were redistributed to make a potato salad that you cannot eat. Crowdfunding platforms could mitigate this reaction through the following approaches: (i) by asking backers up front whether they are willing to accept a redistribution, (ii) by respecting their giving preferences or (iii) by orienting a donor's mindset toward accepting total utilitarianism. Both offline giving clubs and DonorsChoose implement variants of our redistribution schemes and through a combination of the above approaches.

Within our categorization, offline giving clubs are similar in spirit to \cpr schemes. Giving clubs manage several campaigns that share a unified high-level cause such as education, environmental conservation, etc.. By re-framing the donation process as a collective action rather than an individualistic one, donors indicate their higher-level preference of the giving club's overall cause and either elected members or the majority dictates the distribution of the collected funds. 
On an online crowdfunding platform, similar campaigns can be grouped into giving themes with redistribution schemes automatically allocating funds across the grouped campaigns.

DonorsChoose implements a repurposing scheme, but is able to successfully do so through the combination of (i) having a cohesive theme of education, (ii) integrating redistribution into the charitable ethos of their platform, and (iii) vetting all projects. 

\subsection{For Campaign Organizers}

The question of reallocation on the part of organizers is similarly nuanced. The policy that additional funds given to a campaign may be reallocated to other (similar) campaigns may cause organizers to feel cheated of their raised funds. 

However, in the case of discrete public goods, the requested amount is determined by a specific need rather than of arbitrary size. In other words, campaign organizers should only be requesting the amount that they actually need; if they wanted to raise more, then they should ask for more initially since the amount requested for discrete public goods is based off of a cost estimate.

There are also very practical reasons why reallocation of excess funds can be \textit{good} for organizers. The most obvious benefit is if a campaign is unable to raise sufficient funds (over 50\% of Kickstarter's projects~\cite{mollick2014dynamics}), redistribution provides such a campaign with a better chance of success. Less obviously, Mollick et al. found evidence that projects with substantial (200\%) excess funds often deliver \textit{worse} results than those that just meet their goals~\cite{mollick2014dynamics}. Kickstarter's blog posted a series on how over-funded projects deal with the extra influx of funds~\cite{excess-blog}; organizers who find themselves in these situations are generally unprepared and must come up with strategies on the fly to cope with excess funds.

On the flip side, the existence of reallocation may also entice project organizers to try and game the system. The risk of not meeting goals and losing all funds is the same as before redistribution, but organizers now have an additional chance to meet their fundraising goal. Thus, organizers have an increased incentive to raise the fundraising goal hoping that reallocation provides an additional chance in their favor. These and other second-order effects and analysis of possible attacks against the system are beyond the scope of this paper. 

As with the donor's interface, the redistribution policy should be presented up front to the organizers so they can decide whether this is suited to their needs. DonorsChoose adopts this approach and further allows the organizers themselves to decide how to redistribute some of the funds they raised if they fail. This can give the organizers a sense of control, even if ultimately the moneys are given to other projects. This model appears to work well when all projects are vetted. 



\subsection{For the Crowdfunding Designer}

As with any site-level changes, the crowdfunding platform designer must consider a wide-array of issues. These questions arise from donor and organizer perspectives, but also in relation to implementation on the site and meta-level implications for the platform's brand.
Modern crowdfunding campaigns align multiple incentives to motivate donors to contribute, including: progress updates, donor incentives (e.g. gifts and prizes), entertaining previews, and a sense of community. 
How then does redistribution interact with such incentives? One possibility is by completely automating the redistribution process the platform may lose some of the engagement that entices donors to return and give more. It is likely that there is no ``correct'' choice, but instead that donors fall along a spectrum of desired engagement levels.

Another question for crowdfunding designers is: when should the redistribution be conducted? Immediately after a campaign ends would require the consideration of other campaigns ending at similar times or across certain time windows. Alternatively, having large amounts of idle donations is wasteful and potentially problematic from the designer's standpoint.
There are also second-order effects to giving patterns that may emerge. For example, most donations occur at the start of a campaign and then towards the end of the campaign~\cite{wash-time} and it is satisfying for donors to see to a project's completion. Will redistribution dampen these effects or could we redistribute in a manner than enhances them?


Looking at the numbers from DonorsChoose, it is clear that they have successfully implemented the repurposing redistribution scheme by bringing together several elements that work well together. Once a crowdfunding platform successfully implements redistribution, the benefits could also increase as these policies become the norm. The effects of incentives, different ways of incorporating redistribution, and second-order effects are interesting questions worth examining in future work.