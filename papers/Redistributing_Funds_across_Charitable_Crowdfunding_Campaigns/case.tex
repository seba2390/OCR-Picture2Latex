
\section{Empirical Case for Fund Redistribution}

In this section, we describe how we implement and analyze two classes of redistribution schemes: \car and \cpr. We study the efficiency models with real data from the \lag crowdfunding platform.

\subsection{The \lag Dataset}
\label{sec:data}

\textit{\lag} is a niche crowdfunding platform focused on the Muslim community. \lag went live in October 2013. Since its inception, \lag has raised more than 3 million dollars for more than 300 projects. Among its notable campaigns are a campaign to rebuild African-American churches destroyed by arson in 2015 and a campaign to fund Adnan Syed's legal team, both raising more than 100,000 dollars.


We analyze \lag's entire donation trace since its inception. We eliminate active campaigns from our analysis (i.e. campaigns whose end date is after our snapshot date of January, 2016) and campaigns that remained live for more than seven days after their end date\footnote{It is difficult to determine whether campaigns overlap, if they have no set end date.}. \lag keeps track of offline contributions made by anonymous donors directly to the campaign organizers. We cannot redistribute these funds as they are not raised through the platform.

Figure \ref{fig:over-time} illustrates the number of active campaigns per month. Figure \ref{fig:goals} charts the distribution of campaign goals.

\begin{figure}[htbp]
    \centering
    \includegraphics[width=1\linewidth]{figures/active_projects.pdf}
    \caption{Number of active campaigns per month.}
    \label{fig:over-time}
\end{figure}

\begin{figure}[htbp]
    \centering
    \includegraphics[width=1\linewidth]{figures/project_goals.pdf}
    \caption{A dot-plot of campaign goals. The mean campaign goal marked with the vertical black line is 19,517 USD.}
    \label{fig:goals}
\end{figure}


\begin{table}[htpb]
    {\cf \small
    \centering
    \begin{tabular}{p{0.79\linewidth}c}
        \toprule
        \textbf{Property} &  \\
        \midrule
        Total number of campaigns & 228 \\
        Distinct donors & 7935 \\
        Repeat donors & 1342 \\
        Percentage of repeat donors &  16.9\% \\
        Average campaigns backed per repeat donor & 3.7  \\
        Maximum campaigns backed per repeat donor & 142 \\
        Average time interval between consecutive donations & 96 days \\
        Average life span of a campaign & 44 days \\
         \bottomrule
    \end{tabular}
    }
    \caption{Summary Statistics for \lag's Campaigns \& Donors.}
    \label{tab:donorstats}
\end{table}



\begin{figure*}
    \centering
    \includegraphics[width=1\linewidth]{figures/both_spans.pdf}
    \caption{Each {\color{tangerine} tangerine} dot represents the time period between two consecutive donations by the same donor to different campaigns. Each {\color{ocean} ocean} dot represents the life span of a campaign.}
    \label{fig:span}
\end{figure*}


\subsection{Redistribution Schemes}
%\subsection{The Optimization Problem}
\label{sec:model}

We analyze four redistribution schemes broadly classified into \textit{\car} or \textit{\cpr} schemes. \Car schemes allow a donor's contributions to be redistributed to any campaign within the platform. These include \textit{na\"ive redistribution} and \textit{repurposing} schemes. \Cpr schemes only shuffle a donor's contributions within the set of campaigns the donor contributed to. These include both \textit{unordered-} and \textit{ordered-} \cpr schemes. With ordering, \cpr ensures that if a donor contributes more to one campaign over another then even after a redistribution, he/she would still contribute more to that campaign.

We formally define each of these schemes within the framework of an optimization problem where the goal is to maximize the number of campaigns that meet their goals.

%We formulate the optimization problem of maximizing the number of campaigns that meets their goals with the help of a linear program.

Let $n$ be the number of distinct donors that donated on the \lag platform and $m$ be the number of distinct campaigns. 

We represent an actual contribution a donor $i \in {1,..., n}$ made to a campaign $j \in {1,..., m}$ with $A_{i,j}$. A campaign, $j$, has a fund-raising goal of $G_j$. We represent whether a campaign has met its goal with a success indicator variable, $I_j$.
\[I_j =
\begin{cases}
    1, & \text{if} \ \displaystyle \sum_{i=1}^{n} A_{i, j} \geq G_j\\
    0, & \text{otherwise}
\end{cases}\]

Every campaign has a start date $s_j$ and an end date $e_j$. Each contribution $A_{i,j}$ is made on a specific date\footnote{If a donor, $i$, makes multiple contributions to the same campaign, $j$, we encode only one contribution $A_{i,j}$ equal to the sum of all such contributions and set $d_{i,j}$ equal to the date of the first contribution.} $d_{i,j}$: $s_j \leq d_{i, j} \leq e_j$.

After a redistribution of funds, we denote the donation a donor $i$ makes to a campaign $j$ with $R_{i, j}$. We now represent whether a campaign meets its goal with another success indicator variable, $I'_j$.
\[I'_j \leq \frac{\sum_{i = 1}^{n} R_{i, j}}{G_j}
\] 

Our redistribution goal is to maximize the number of successful projects,
\[\max \sum_{j = 1}^{m} I'_j \]

\textit{All four schemes must satisfy the following three constraints.} 

\begin{enumerate}[label={[\bf\sc{All\arabic*}]},wide =1em]
\item \emph{Once a winner, always a winner:} If a campaign met its goal without redistribution, a redistribution of funds should not cause this campaign to fail. 
\[\forall j, \;\ I'_j \geq I_j \]

\item \emph{Fixed Budget:} Each donor cannot give more than the sum of his/her original contributions across all campaigns and a donor cannot make a negative contribution.
\[ \forall i, \;\ \sum_{j = 1}^{m} R_{i,j} \leq \sum_{j = 1}^{m} A_{i,j}\]
\[\forall i,j, \;\ R_{i, j} \geq 0\]

\item \emph{Redistribute across live, overlapping campaigns:}
For each contribution $A_{i, j}$ made on $d_{i,j}$, we define overlapping campaigns as follows:
\[\mathcal{O}_{i, j} = \{x : \ e_x \geq d_{i, j} \wedge s_x \leq e_j \}\]

%\[\mathcal{O}_{i, j} = \{u, c: \ e_c \geq d_{i, j} \wedge d_{u, c} \leq e_j\}\]

%\[\mathcal{O}_{i, j} = \{x: \ e_x \geq d_{i, j} \wedge d_{i, x} \leq e_j\}\]

%Figure \ref{fig:overlap} visualizes the overlap region of a particular contribution when restricted to a single donor $x$. 

%The following constraint ensures that we do not redistribute funds from a contribution to campaigns that have already ended or campaigns that the user contributed to after the current campaign ended:

The following constraint ensures that we redistribute funds only within overlapping campaigns.

\[\forall i,j, \;\ \sum_{i, y \in \mathcal{O}_{i, j}} R_{i, y} \leq \sum_{i, x \in \mathcal{O}_{i, j}} A_{i, x}\]

%\[\forall R_{i, j}, \;\ R_{i, j} \leq \sum_{x \in \mathcal{O}_{i, j}} A_{i, x}\]

\end{enumerate}

We also try to eliminate the following unfavorable redistributions with the following optional constraints. We drop these \textit{nice} constraints if the optimization problem becomes infeasible:
\begin{enumerate}[label={[\bf\sc{Nice\arabic*}]},wide =1em]
    \item \textit{Avoid allocating more funds to a previously overfunded campaign:}
    \[\forall j, \;\ I_j = 1 \rightarrow \sum_{i=1}^n R_{i, j} \leq \sum_{i=1}^n A_{i, j}\]
    
    \item \textit{Avoid allocating more funds to a failed campaign:} If no redistribution of funds can save a campaign, then we should not allocate additional funds to that campaign. 
    \[\forall j, \;\ I'_j = 0 \rightarrow \sum_{i=1}^n R_{i, j} \leq \sum_{i=1}^n A_{i, j}\]
    
    \item \textit{Avoid surpassing the goal of a previously unsuccessful campaign:} 
    \[\forall j, \;\ I_j = 0 \rightarrow  \sum_{i=1}^n R_{i, j} \leq G_j \]
\end{enumerate}


\subsubsection{\Car}

\paragraph{Na\"ive redistribution} 

This scheme does not require any additional constraints to the base constraints listed above.


\paragraph{Repurposing} 

Repurposing also requires that we only redistribute funds from failed campaigns. We encode this with the following constraint:

\begin{enumerate}[label={[\bf\sc{Rep}]},wide =1em]
    \item \textit{Winners keep all}: 
    \[\forall i,j, \;\ I_j = 1 \rightarrow R_{i, j} = A_{i, j} \]
\end{enumerate}

No other redistribution scheme imposes the ``winners keep all" constraint.

\subsubsection{\Cpr}

Our two \cpr schemes must also ensure the following:

\begin{enumerate}[label={[\bf\sc{CP\arabic*}]},wide =1em]

\item \emph{No new campaigns:} For a given donor, we should only redistribute his/her funds among the campaigns he/she contributed to.
\[\forall i,j, \;\ A_{i, j} = 0 \rightarrow R_{i, j} = 0 \]

\item \emph{Redistribute across live, overlapping campaign contributions:} This constraint tightens the overlap constraints of na\"ive and repurposing redistribution schemes to overlapping \textit{contributions} by the same donor. We redefine overlapping campaign contributions as follows:

\[\mathcal{O}_{i, j} = \{x: \ e_x \geq d_{i, j} \wedge d_{i, x} \leq e_j\}\]

Figure \ref{fig:overlap} visualizes the overlap region of a particular contribution when restricted to a single donor $u$. 

The following constraint ensures that we do not redistribute funds from a contribution to campaigns that have already ended or campaigns that the donor contributed to after the current campaign ended:

\[\forall i,j, \;\ R_{i, j} \leq \sum_{x \in \mathcal{O}_{i, j}} A_{i, x}\]

\begin{figure}[htb]
    \centering
    \includegraphics[width=1\linewidth]{figures/overlap.pdf}
    \caption{The contribution made by donor $u$ on $d_{u,2}$ for campaign $2$ overlaps with contributions made on $d_{u,1}$ and $d_{u,4}$. 
    For donor $u$, a redistribution of funds to campaign 2, can only be allowed from contributions made to campaign 1 and 4.}
    \label{fig:overlap}
\end{figure}

\end{enumerate}

\paragraph{Ordered \cpr}

Finally, the ordered \cpr scheme requires the following constraint:

\begin{enumerate}[label={[\bf\sc{Order}]},wide =1em]
\item \emph{Preserve Preference Ordering:} If a donor contributes more funds to one campaign compared to another, both absolutely and relatively to their respective goals, then a redistribution of funds should preserve this preference ordering.
\[\forall i,x,y, \;\  (A_{i, x} > A_{i, y}) \land (A_{i, x}/G_x > A_{i, y}/G_y) \\
\rightarrow  R_{i, x} \geq R_{i, y} \]

We relax preference ordering to eliminate contributions to \textit{failed} campaigns (i.e. campaigns that do not succeed for any feasible redistribution) with the following constraint instead:
\begin{multline*}
% \forall x,y, \;\  A_{i, x} > A_{i, y} \land I'_x=1 \land I'_y=1 \rightarrow I'_x \times R_{i,x} \geq (I'_x + 2(I'_y - 1))\times R_{i,y}
\forall i,x,y, \;\  (A_{i, x} > A_{i, y}) \land (A_{i, x}/G_x > A_{i, y}/G_y) \\
\land (I'_x = 1) \land (I'_y = 1) 
\rightarrow R_{i, x} \geq R_{i, y}
\end{multline*}
\end{enumerate}





\subsection{Results}
\label{sec:results}

Table \ref{tab:overall} shows the increase in campaigns funded after each redistribution scheme. Except for a single campaign that was successfully funded by \cpr and failed by \car, the set of campaigns funded by ordered \cpr is a subset of unordered \cpr, which is a subset of repurposing, which in turn is a subset of na\"ive redistribution.



\begin{table}[htpb]
    {\cf \small
    \centering
    \begin{tabular}{p{0.74\linewidth}cc}
        \toprule
        \textbf{Projects funded with ... } &  \\
        \midrule
        Original contributions (no redistribution) & 85 & 37\% \\
        \midrule
        \textit{\Car} & \\
        \ \ \ \ Na\"ive & 180 & 79\% \\  % (+42\%) \\
        \ \ \ \ Repurposing & 175 & 77\% \\
        \midrule
        \textit{\Cpr} & \\
        \ \ \ \ Unordered & 109 & 48\% \\
        \ \ \ \ Ordered & 99 & 43\% \\
        \bottomrule
    \end{tabular}
    }
    \caption{Successful campaigns for each redistribution scheme.}
    \label{tab:overall}
\end{table}

The increase in successful campaigns with \cpr schemes is low when compared to \car schemes. Yet, if we consider (i) the small proportion of repeat donors, roughly 17\% of the donor base (Table \ref{tab:donorstats}), (ii) the mean number of campaigns a repeat donor contributes to, less than 4 campaigns (Table \ref{tab:donorstats}), and (iii) the mean gap between consecutive donations, 96 days, in relation to the average life span of campaigns, 44 days (Figure \ref{fig:span}), the increases in campaign success brought about by \cpr are actually fairly surprising.


\subsubsection{Trade off: Efficiency vs. Choice}

\begin{figure*}[htbp]
    \centering
    \includegraphics[width=1\linewidth]{figures/naive.pdf}
    \includegraphics[width=1\linewidth]{figures/repurpose.pdf}
    \caption{The behaviour of \car schemes with \lag's contributions. The {\color{tangerine} tangerine} colored bars indicate funds deducted 
    from a campaign. The {\color{ocean} ocean} colored bars indicate funds allocated to a campaign after redistribution.}
    \label{fig:inter}
\end{figure*}

Naturally, \car leads to more efficiency in terms of campaigns meeting their goal %for the total amount of money donated 
than \cpr.

Figure \ref{fig:inter} illustrates the behavior of \car schemes. With na\"ive redistribution, almost all over-funded campaigns lose most of their excess funds. We observe that repurposing is slightly less efficient than na\"ive redistribution as it allows successful campaigns to keep their excess funds. This contrasts with the behavior of \cpr schemes (Figure \ref{fig:intra}), where only a few over-funded campaigns lose some of their excess funds, but donor choices (Fig. \ref{fig:intra}-top) or ordered preferences (Fig. \ref{fig:intra}-bottom) are preserved.

In general, our definition of \car favors campaigns with small goals and disfavors campaigns with large goals regardless of how much of their goals they initially raised. The short tail section, the noticeable number of post-redistribution successful campaigns that initially raised less than 10\% of their goal and the noticeable number of failed campaigns that initially raised more than 50\% of their goal in Figure \ref{fig:inter} illustrate this behavior. Table \ref{tab:diff} also shows that on average \car supports campaigns that are further from their goal.\footnote{As a redistribution scheme funds more campaigns the average difference from the goal naturally increases and becomes closer to the distribution of differences of failed campaigns with original contributions. Therefore, this measure should be interpreted carefully.}

\begin{table}[htpb]
    {\cf \small
    \centering
    \begin{tabular}{p{0.5\linewidth}P{0.125\linewidth}P{0.2\linewidth}}
        \toprule
        \textbf{Across ... } & \multicolumn{2}{c}{\textbf{Average}} \\
        & goal & diff. from goal \\
        \midrule

All campaigns 	&	19517 & 11548 (40\%) \\
Only successful campaigns &	11221 & -1090 (-12\%)  \\

Only failed campaigns & 24448 & 19059 (70\%)\\
\midrule
\multicolumn{3}{l}{\textit{Only successful campaigns after ... redistribution}} \\

        Na\"ive 	 &	5749 & 	3580 (66\%) \\
        Repurposing &	5337 &	3312 (66\%) \\
        Unordered choice-preserving &	6429&		1357 (34\%) \\
Ordered choice-preserving &	6411 &	1455 (31\%)\\
        \bottomrule
    \end{tabular}
    }
    \caption{Summary statistics on differences from goal across successful campaigns under different redistribution schemes.}
    \label{tab:diff}
\end{table}


This behavior is most clearly manifested in one redistribution outcome where a campaign that raised 97\% of its goal of 26,000 USD, with 800 USD dollars remaining. Both na\"ive redistribution and repurposing fail this campaign to support others. Both \cpr schemes, however, help this campaign meet its goal.

This may seem like a shortcoming of the optimization objective, which staunchly maximizes the number of campaigns funded. This is an artifact of our optimization objective. If different resolution properties are desired, platform designers could change the optimization problem to also minimize the sum of (absolute/percentage) differences reallocated. Designers must also weigh maximizing success against minimizing differences, where each weight assignment leads to different outcomes.


\begin{figure*}[htbp]
    \centering
    \includegraphics[width=1\linewidth]{figures/u-intra.pdf}
    \includegraphics[width=1\linewidth]{figures/o-intra.pdf}
    \caption{The behaviour of \cpr schemes with \lag's contributions. The {\color{tangerine} tangerine} colored bars indicate funds deducted 
    from a campaign. The {\color{ocean} ocean} colored bars indicate funds allocated to a campaign after redistribution.}
    \label{fig:intra}
\end{figure*}

\Cpr is biased toward helping campaigns with multiple donors as they are more likely to benefit from a redistribution of funds from their supporting donor base. The average percentage of repeat donors (i.e. donors who also donated to other campaigns) per campaign for the successful campaigns post \cpr is 61-63\%. This is much higher than the overall average percentage of repeat donors per campaign 48\% (Table \ref{tab:repeat}). The one campaign that only \cpr schemes successfully funded had 120 donors, 64 (53\%) of them were repeat donors.

\begin{table}[htpb]
    {\cf \small
    \centering
    \begin{tabular}{p{0.5\linewidth}P{0.125\linewidth}P{0.2\linewidth}}
        \toprule
        
        \textbf{Across ... } & \multicolumn{2}{c}{\textbf{Average}} \\
        
        & num. of donors & num. of repeat donors \\
        \midrule
        All campaigns	& 51 & 22 (48\%)	 \\
        Only successful campaigns & 	87 	& 35 (47\%)	\\
        Only failed campaigns	& 29 & 14 (49\%)	\\
        \midrule
        \multicolumn{3}{l}{\textit{Only successful campaigns after ... redistribution}} \\
        Na\"ive & 22 & 11 (50\%)\\
        Repurposing &	21 & 11 (50\%) \\
        Unordered choice-preserving &	47 & 29 (61\%) \\
        Ordered choice-preserving & 55 & 38 (63\%) \\      
        \bottomrule
    \end{tabular}
    }
    \caption{Summary repeat donor statistics across successful campaigns under different redistribution schemes.}
    \label{tab:repeat}
\end{table}

\subsubsection{Effect of Donor Acceptance}
\label{sec:acceptance}

\begin{figure*}[htbp]
    \centering
    \begin{subfigure}[t]{1\columnwidth}
        \includegraphics[width=1\columnwidth]{figures/accept_multiple.pdf}
        \caption{The mean increase in funded campaigns as the percentage of donors accept a redistribution scheme. 
        }
        \label{fig:accept}
    \end{subfigure}
    \hspace*{\fill}%
    \begin{subfigure}[t]{1\columnwidth}
        \includegraphics[width=1\columnwidth]{figures/project_accept_multiple.pdf}
        \caption{The mean increase in funded campaigns as the percentage of campaigns accept a redistribution scheme.
        }
        \label{fig:org-accept}
    \end{subfigure}
    \caption{The mean increase in funded campaigns as the percentage of donors (a) or campaigns (b) accept a redistribution scheme.
    For each acceptance percentage, we construct random samples of donors (a) or campaigns (b) who accept the redistribution (both in terms of outflow or inflow of funds) 
    to estimate the variance: the error bars mark one standard deviation from the plotted mean.}
\end{figure*}

We simulated the behavior of each scheme under different donor acceptance ratios as follows: for each acceptance percentage $p\%$, we randomly picked $p\%$ of the donor population to agree to a redistribution of their funds. The remaining $(100 - p)\%$ rejects any redistribution. We generated several samples for each $p\%$ to compute the mean number of successful campaigns and the variance for a given $p\%$. 

Figure \ref{fig:accept} illustrates the effect increasing the number of donors who accept a redistribution on efficiency. 

With \car, an acceptance ratio of only 10\% of the donors can lead to almost half of the improvement achievable when all donors accept the redistribution. In contrast, \cpr leads to a more gradual improvement as the acceptance ratio increases.

\subsubsection{Effect of Organizer Acceptance}
As with donor acceptance rates, we also simulated a percentage of campaigns organizers accepting a redistribution of funds to their campaigns, both in terms of outflow and inflow of funds.

Figure \ref{fig:org-accept} illustrates the effect increasing the number of organizers who accept a redistribution on efficiency. In this case, both \car and \cpr show the same linear behavior in response to increases of accepting organizers.
