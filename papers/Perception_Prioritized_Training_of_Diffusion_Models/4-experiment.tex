\section{Experiment}
\label{sec:experiment}
We start by exhibiting the effectiveness of our new training objective over the baseline objective in~\cref{sec:experiment_baseline}. Then, we compare with prior literature of various types of generative models in~\cref{sec:experiment_literature}. Finally, we conduct analysis studies to further support our method in~\cref{sec:experiment_analysis}. Samples generated with our models are shown in~\cref{fig:teaser}.


\begin{table}[]
\centering
\begin{tabular}{l|r|cc|cc}
\hline
                        & \multicolumn{1}{l|}{}     & \multicolumn{2}{c|}{FID-50k$\downarrow$}                             & \multicolumn{2}{c}{KID-50k$\downarrow$}                             \\ \hline
Dataset                 & \multicolumn{1}{l|}{Step} & \multicolumn{1}{c}{Base} & \multicolumn{1}{c|}{Ours} & \multicolumn{1}{c}{Base} & \multicolumn{1}{c}{Ours} \\ \hline
\multirow{2}{*}{FFHQ}   & 1000                      &             7.86         &         \textbf{6.92}      &       3.85          &      \textbf{3.46}        \\
                        & 500                       &            8.41          &         \textbf{6.97}     &       4.48        &         \textbf{3.56}         \\  \hline
\multirow{2}{*}{CUB}    & 1000                      &            9.60   &         \textbf{6.95}         &      3.49              &         \textbf{2.38}         \\
                        & 250                       &              10.26     &         \textbf{6.32}  &        4.06        &     \textbf{1.93}         \\  \hline
\multirow{2}{*}{AFHQ-D} & 1000                      &       12.47       &         \textbf{11.55}     &         4.79     &  \textbf{4.10}         \\
                        & 250                       &         12.95      &         \textbf{11.66}         &         5.25         &         \textbf{4.20}         \\  \hline
Flowers                 & 250                       &             20.01     &         \textbf{17.29}   &       16.8       &     \textbf{14.8}         \\
MetFaces                & 250                       &              44.34       &         \textbf{36.80}      &          22.1       &      \textbf{17.6}         \\ \hline
\end{tabular}
\caption{\textbf{Quantitative comparison.} Diffusion models trained with our weighting scheme achieve consistent improvement over the baseline at various datasets and sampling steps, in terms of both FID and KID ($\times 10^3$).}
\label{table:fid_various}
\end{table}

% $\text{6.97}_{~\textcolor{cyan}{(-1.44)}}$

\subsection{Comparison to the Baseline}
\label{sec:experiment_baseline}
\textbf{Quantitative comparison.  } 
We trained diffusion models by optimizing training objectives with both baseline and our weighting scheme on FFHQ~\cite{stylegan}, AFHQ-dog~\cite{choi2020stargan}, MetFaces~\cite{karras2020training}, and CUB~\cite{wah2011caltech} datasets. These datasets contain approximately 70k, 50k, 1k, and 12k images respectively.  We resized and center-cropped data to 256$\times$256 pixels, following the pre-processing performed by ADM~\cite{dhariwal2021diffusion}.

\cref{table:fid_various} shows the results. Our method consistently exhibits superior performance to the baseline in terms of FID and KID. The results suggest that our weighting scheme imposes a good inductive bias for training diffusion models, regardless of the dataset. Our method outperforms the baseline by a large margin especially on MetFaces, which contains only 1k images. Hence, we assume that wasting the model capacity on learning imperceptible details is very harmful when training with limited data. 


\textbf{Qualitative comparison.  }
We observe that diffusion models trained with the baseline objective are likely to suffer color shift artifacts, as shown in~\cref{fig:quality_compare}. We assume that the baseline training objective unnecessarily focuses on the imperceptible details; therefore, it fails to learn global color schemes properly. In contrast, our objective encourages the model to learn global and holistic concepts in a dataset.

\subsection{Comparison to the Prior Literature}
\label{sec:experiment_literature}
We compare diffusion models trained with our method to existing models on FFHQ~\cite{stylegan}, Oxford flowers~\cite{nilsback2008automated}, and CelebA-HQ~\cite{karras2017progressive} datasets, as shown in~\cref{table:fid_ffhq2}. We use 256$\times$256 resolutions for all datasets. We achieve state-of-the-art FIDs on the Oxford Flowers and CelebA-HQ datasets. While our models are trained with $T=1000$, we already achieve state-of-the-art with reduced sampling steps; 250 and 500 steps respectively. On FFHQ, we achieve a superior result to most models except StyleGAN2~\cite{karras2020analyzing}, whose architecture was carefully designed for FFHQ. We note that our method brought the diffusion model closer to the state-of-the-art, and scaling model architectures and sampling steps will further improve the performance.


\begin{figure*}[t!]
  \centering
  \includegraphics[width=1.0\linewidth]{figures/qualitative_compare.png}
  \caption{\textbf{Qualitative comparison.} Uncurated samples generated during training. The number of images seen for training displayed on the top. We observed that the baseline suffer color shift problem at early iterations of the training (FFHQ) or even at the convergence (MetFaces). The baseline weighting scheme fails to focus on global coherence and waste model capacity on the imperceptible details.}
  \label{fig:quality_compare}
\end{figure*}


\begin{table}[]
\centering
\begin{tabular}{lllr}
\hline
Dataset                                                                   & Method   & Type & FID$\downarrow$ \\ \hline
\multirow{10}{*}{FFHQ}                                                     & $\text{BigGAN}_{\text{~ICLR'19}}$~\cite{brock2018large}  & GAN  &   12.4  \\
                                                                          & $\text{UNet GAN}_{\text{~CVPR'20}}$~\cite{schonfeld2020u}  &  GAN  &  10.9   \\
                                                                          & $\text{StyleGAN2}_{\text{~CVPR'20}}$~\cite{karras2020analyzing}   & GAN  & \textbf{3.73}  \\
                                                                          & $\text{NVAE}_{\text{~NeurIPS'20}}$~\cite{vahdat2020nvae} &  VAE   &   26.02  \\
                                                                          & $\text{VDVAE}_{\text{~ICLR'21}}$~\cite{child2020very}  &  VAE  &   33.5  \\
                                                                          & $\text{VQGAN}_{\text{~CVPR'21}}$\cite{esser2021taming}  &  GAN+AR &   9.6  \\
                                                                          & $\text{D2C}_{\text{~NeurIPS'21}}$~\cite{sinha2021d2c} &  Diff  &   13.04  \\
                                                                          & Baseline (500 step)   &  Diff  &  8.41   \\
                                                                          & P2 (500 step)     &  Diff &  6.97   \\
                                                                          & P2 (1000 step)     &  Diff &  \underline{6.92}  \\ \hline
\multirow{5}{*}{\begin{tabular}[c]{@{}l@{}}Oxford\\ Flower\end{tabular}} & $\text{PGGAN}_{\text{~ICLR'18}}$~\cite{karras2017progressive} &  GAN  &  64.40   \\
                                                                          & $\text{StyleGAN1}_{\text{~CVPR'19}}$~\cite{stylegan}      &  GAN   &  64.70   \\
                                                                          & $\text{MSG-GAN}_{\text{~CVPR'20}}$~\cite{karnewar2020msg}     & GAN   & \underline{19.60}  \\
                                                                          & Baseline (250step)      &   Diff  &  20.01   \\ 
                                                                          & P2 (250step)      &   Diff  &  \textbf{17.29}   \\ \hline
\multirow{8}{*}{\begin{tabular}[c]{@{}l@{}}CelebA\\ -HQ\end{tabular}}     & $\text{PGGAN}_{\text{~ICLR'18}}$~\cite{karras2017progressive}   & GAN   &  8.03   \\
                                                                          & $\text{GLOW}_{\text{~NeurIPS'18}}$~\cite{kingma2018glow}        &   Flow &   68.93  \\
                                                                          & $\text{ALAE}_{\text{~CVPR'20}}$~\cite{pidhorskyi2020adversarial}   &  GAN  &   19.21  \\
                                                                          & $\text{NVAE}_{\text{~NeurIPS'20}}$~\cite{vahdat2020nvae}   &  VAE &   29.76  \\
                                                                          & $\text{VAEBM}_{\text{~ICLR'21}}$~\cite{xiao2020vaebm}   &  VAE+EM  &  20.38   \\
                                                                          & $\text{VQGAN}_{\text{~CVPR'21}}$~\cite{esser2021taming}  &    GAN+AR  &   10.70  \\
                                                                          & $\text{LSGM}_{\text{~NeurIPS'21}}$~\cite{vahdat2021score}    &  VAE+Diff &  \underline{7.22} \\
                                                                          & P2 (500step)   &   Diff   &  \textbf{6.91} \\ \hline
\end{tabular}
\caption{\textbf{Comparison to prior literature.} FFHQ results reproduced from~\cite{parmar2021cleanfid,esser2021taming,sinha2021d2c}, Oxford Flower from~\cite{karnewar2020msg}, and CelebA-HQ from from~\cite{vahdat2021score}. Except for GANs, we achieve superior results.}
\vspace{-1em}
\label{table:fid_ffhq2}
\end{table}

\subsection{Analysis}
\label{sec:experiment_analysis}
In this section, we analyze whether our weighting scheme is robust to the model configurations, number of sampling steps, and sampling schedules.

\textbf{Model configuration matters?  }
Previous experiments are conducted using our default model for fair comparisons. Here, we show that P2 weighting is effective regardless of the model configurations. \cref{table:architecture} shows that our method achieves consistently superior performance to the baseline for various configurations. We investigated for following variations: replacing the BigGAN~\cite{brock2018large} residual block with the residual block of Ho et al.~\cite{ho2020denoising}, removing self-attention at 16$\times$16, using two BigGAN~\cite{brock2018large} residual blocks, and training our default model with a learning rate of $2.5\times10^{-5}$. Our default model contains a single BigGAN~\cite{brock2018large} residual block and is trained with a learning rate $2\times10^{-5}$. Our weighting scheme consistently improves FID and KID by a large margin, across various model configurations. Our method is especially effective when the self-attention is removed ((c)), indicating that P2 encourages learning global dependency.
\vspace{0.5em}

\textbf{Sampling step matters?  }
We trained our models on 1000 diffusion steps following the convention of previous studies. However, it requires more than 10 min to generate a high-resolution image with a modern GPU. Nichol et al.~\cite{nichol2021improved} have shown that their sampling strategy maintains performance even when reducing the sampling steps. They also observed that using the DDIM~\cite{song2020denoising} sampler was effective when using 50 or fewer sampling steps.

\cref{fig:speed} shows the FID scores of various sampling steps with models trained on the FFHQ. A model trained with our weighting scheme consistently outperforms the baseline by considerable margins. It should be noted that our weighting scheme consistently achieves better performance with half the number of sampling steps required by the baseline.

\textbf{Why not schedule sampling steps?  } 
In addition to the consistent improvement across various sampling steps, we sweep over the sampling steps in~\cref{table:sweep}. Sweeping sampling schedules slightly improves FID and KID but does not reach our improvement. Our method is more effective compared to scheduling sampling steps, as we improve the \textit{model training}, which benefits predictions at all steps.



\section{Architecture}
\label{sec:arch}
Four goals guided our design:

\begin{enumerate}
    \item \emph{Discoverability}. Users should be able to use their pre-existing 
        knowledge of Kubernetes to discover what their application is doing, how 
        to modify it, and how Streams works in general. Streams applications 
        should also be discoverable by other workloads on the same Kubernetes 
        cluster, through standard Kubernetes mechanisms.
    \item \emph{Composability}. By using Kubernetes first-class service
        endpoints, cloud native Streams should interoperate with other applications
        and middleware without further configuration. 
    \item \emph{Application state preservation}. Stateless services are easy to
        manage in Kubernetes, as simply restarting them is always an option.
        But most Streams applications have state. We have an implicit contract
        with users that once they deploy an application, they will not lose any
        accumulated state---barring application failure and users taking
        explicit action to restart it.
    \item \emph{Backwards compatibility}. The cloud native version of Streams 
        should run legacy Streams applications unchanged. This goal means that 
        we cannot change any public APIs, and our task is to find the most 
        Kubernetes-like way to express functionality originally designed 
        for an on-premises cluster.
\end{enumerate}

We also have one anti-goal: we do not maintain API compatibility with the legacy
platform. Trying to do so would force cloud native Streams to take on
responsibilities that should belong to Kubernetes, which would end up
conflicting with our stated goals. We want user's \emph{applications} to remain
unchanged, but we are assuming that they are adopting cloud native Streams as
part of an overall effort to consolidate and simplify management and
administration.

\subsection{Overview}

All aspects of a Streams application exposed to users are represented as CRDs or
existing Kubernetes resources. We apply the patterns described in
\S~\ref{sec:patterns}: each resource is managed by a controller; when we
need to monitor the status of multiple kinds of resources in order to take an
action we use a conductor; and when multiple actors need to change the state for
a particular resource, we use a coordinator.

The CRD is the foundational unit in our design. CRDs are exposed to users in the
same way as any other Kubernetes resource, which means that representing Streams
concepts as CRDs gains not just native integration into the Kubernetes system,
but also the user interfaces. Any state that we must maintain goes into a CRD;
all state not in CRDs is ephemeral and can be lost without consequence.
Kubernetes delivers reliable event notifications when CRDs and other resources
are created, deleted and modified. Reacting to these notifications in
controllers and conductors is the primary communication mechanism between all of
the actors in our system.

The CRDs in Figure~\ref{fig:actors} define the following resources:

\begin{itemize}
    \item \code{Job}: A single Streams job. The job 
        controller initiates the job submission and tracks unique job identifiers. 
        The job conductor manages the job submission process and update 
        its status when completed.
    \item \code{ProcessingElement}: A PE in a job. The 
        PE controller tracks launch count and restores voluntarily deleted PEs.
    \item \code{ParallelRegion}: A single parallel region in a job. It 
        exposes a \code{width} attribute that can be directly altered by users using 
        \code{kubectl edit} or the Kubernetes client API. The parallel region
        controller handles width changes applied to parallel regions.
   \item \code{HostPool}: A host pool in a job.
   \item \code{Import}: An imported stream in a job. The import
       controller monitors the addition and modification of these resources and 
       matches them with export resources.
   \item \code{Export}: An exported stream in a job. The export
       controller monitors the addition and modification of these resources and 
       matches them with import resources.
    \item \code{ConsistentRegion}: A consistent region in a job. The 
        consistent region controller coordinates application checkpoints and 
        restarts for a single region.
    \item \code{ConsistentRegionOperator}: tracks all consistent regions in a 
        job. Created on-demand during a job submission with a consistent region.
        Its controllers monitor the deployments used to create the operators.
\end{itemize}


We also leverage the following Kubernetes resources:

\begin{itemize}
    \item \code{ConfigMap}: Shares job specific configuration between controllers and pods, 
        such as the graph metadata used by PEs to inform them of the operators they contain 
        and how to connect to other PEs. 
    \item \code{Deployment}: Manages the instance operator and the consistent region operator.
    \item \code{Pod}: Executes PEs. We use a pod controller to monitor and 
        manage the life cycle of pods within jobs. The pod conductor waits until all required
        resources are available before starting a pod for a PE.
    \item \code{Service}: Exports PE entrypoints as well as user-defined 
        services within PEs.
\end{itemize}

Figure~\ref{fig:actors} shows how our actors interact with each other. There
are four kinds of actions they can take:

\begin{enumerate}
    \item \emph{observes}: the actor either receives events from Kubernetes about that 
        resource, or passively views its store.
    \item \emph{creates}: the actor creates new instances of the resource through 
        commands to Kubernetes.
    \item \emph{deletes}: the actor deletes particular instances of a resource 
        through commands to Kubernetes.
    \item \emph{modifies}: the actor makes changes to an already created resource 
        through that resource's coordinator.
\end{enumerate}

None of our actors communicate directly with each other; all communication happens 
by creating, modifying or deleting Kubernetes resources.

Figure~\ref{fig:streams_v4_v6} depicts the deployed artifacts of an instance of cloud
native Streams.  The Streams \emph{instance operator} contains all the
controllers, conductors and coordinators for a Streams instance. Each Kubernetes
namespace can have one Streams instance operator. The instance operator maps to
the legacy concept of an instance. The legacy concept of a Streams
domain---management of Streams instances---is no longer needed as the Kubernetes
cluster serves that role.

In Streams, the PE is the vehicle for executing user code. PEs contain an
arbitrary number of user operators and the application runtime.  In our design,
we always assign one PE to a pod. This design decision is fundamental. It
allows us to: tie a runtime PE's life cycle to that of the pod that contains
it; fully offload PE scheduling to Kubernetes; rely on the Kubernetes DNS
service for establishing direct TCP connections between our PEs.  Any other
design would have required bespoke implementations of life cycle management,
scheduling and network name resolution.

\begin{figure}[t!]
  \centering
  \includegraphics[width=1\linewidth]{figures/streams_v4_v6}
  \caption{Legacy and cloud native Streams deployments}
  \label{fig:streams_v4_v6}
\end{figure}

While the PE runtime is unchanged, we had to implement a new translation layer
between the PE and Kubernetes. It implements the platform abstraction for the PE
runtime, as well as instantiates and initializes the PE. At runtime, it
also: collects metrics from the PE using a pre-existing interface and exposes
them to Prometheus~\cite{prometheus}; monitors and reports the status of all
PE-to-PE connections; and monitors liveness and reports it to Kubernetes.

\subsection{Loose Coupling}
\label{sec:coupling}

The legacy Streams platform was tightly coupled, which lead to operational
difficulty and implementation complexity. Cloud native Streams applies the
concept of loose coupling.

\paragraph{Name resolution:} PEs communicate with each other over TCP
connections. It is the platform's responsibility to define PE ports, give them
names, and allow PEs to find each other's ports by those names.

In legacy Streams, each PE port is assigned a \code{(peId, portId)} tuple called
a \emph{port label} that uniquely identifies that port in the instance.
At initialization, PEs must establish their remote connections to other PEs
using their port labels. To that end, each PE first creates the socket receiver
for each of its receiver ports, determines its local TCP port, and publishes to 
ZooKeeper its mapping of port label to \code{(hostname, tcpPort)}. PEs already
know the port label that each sending port needs to connect to through graph
metadata that the platform provides at PE startup. After publishing its own
receiver port labels, each PE then looks up the translation of the remote
receiver port label in ZooKeeper for each of its own sender ports and
establishes those connections.  Even with some caching (used to reestablish lost
connections), the thundering herd aspect of this initialization process and the
strain it applies to the ZooKeeper ensemble delays initial deployment times.

Cloud native Streams relies on the Kubernetes name resolution system to resolve
inter-PE connections. There are similarities with the legacy system name
resolution system, as it relies on etcd to store its Service configurations, and
it also has some currently unresolved latency issues~\cite{codacydns,machudns}.
But from an application perspective, it is easier to use as name resolution is
done using standard BSD functions such as \code{gethostbyname()}. Lastly, from
an administration perspective, it is simpler to manage as it binds directly to
the container's \code{/etc/resolv.conf} subsystem and can be easily superseded.

\paragraph{Message bus:} The message bus in legacy Streams between the platform
and PEs uses full-duplex, synchronous communication channels implemented with
JMX~\cite{jmx}. All initiated communications must succeed. Failed
communications are retried with increasing backoff delays before being
escalated as more general system failures where it may restart a PE. As job
count and PE size increases, communications tend to time out more frequently,
leading to failure escalations reducing responsiveness. We have witnessed tens
of minutes to list all the PEs in an overloaded instance.

Cloud native Streams decouples the instance operator from the PEs by relying on
the states stored in Kubernetes to achieve operational availability. The
controller pattern (\S~\ref{sec:controller}) is used by all agents
interested in keeping track of those states. Agents that need to notify the
instance operator of internal state changes do so through Kubernetes events.  In
turn, the instance operator synchronously applies those changes to the custom
resources, preventing potential race conditions (\S~\ref{sec:coordinator}).

The SPL runtime, including the PEs, are implemented in C++. As of this writing,
no library capable of implementing our controller pattern is available in C/C++.
As an alternative, we temporarily resort to a set of REST services hosted by the
instance operator. State changes within agents other than the instance operator
periodically send REST operations to those services to notify the operator of
internal changes. In turn, the operator applies these changes to the related
custom resources. Implementing the proper library and removing the REST layer is
part of our future work.

\subsection{Fault Tolerance and Rolling Upgrades}

Fault tolerance and general high-availability is a primary goal in the design of
Streams since streaming applications are expected to run for months without
interruption. To that end, the legacy Streams platform was designed such that:

\begin{enumerate}
    \item All platform related state is persisted in ZooKeeper. Upon failure, 
        platform services restart and retrieve their state from ZooKeeper.
    \item Streaming applications continue to run during platform service
        failures or upgrades.
    \item Applications seamlessly resume operations after the loss of a
        PE or a host.
\end{enumerate}

In cloud native Streams, we use Kubernetes to preserve
these attributes.

\paragraph{Persistent states:} Kubernetes exposes state persistence to users
through CRDs. Cloud native Streams makes heavy use of CRDs 
to maintain states critical for recovery. However, where the legacy
platform implementation favored storing the state of the system as-is,
cloud native Streams stores only what is necessary and sufficient
to reach the current state of the system through recomputation. The reasons
behind that radical shift in the computation versus space trade-off of
our system are:

\begin{enumerate}
    \item We discovered through empirical measurements that the
        amount of time required to perform state recomputation is negligible
        compared to other operations in the system and appear instantaneous to
        human users.
    \item Minimizing the amount of data persisted drastically reduces the
        pressure on the persistent ensemble.
    \item Re-computing intermediate state simplifies the design of our system.
\end{enumerate}

\paragraph{Instance operator:} The Streams instance operator is designed to be
resilient to its pod restarting.  All of the actors in the instance operator
will receive the full history of Kubernetes events that they are subscribed to,
allowing them to catch-up to the current state of the system.  The applications
themselves do not need the instance operator for normal operation, so they can
continue unharmed. Because of this resiliency, the instance operator can easily
recover from failure. Upgrades are also trivial: change the image for the
instance operator and restart the pod. The combination of how we defined our
CRDs, the patterns we use to manage them and Kubernetes' reliable event delivery
enable these capabilities.

\paragraph{Applications:} We consider two types of application failures:
voluntary failures, when a user deletes a resource; and involuntary failures,
when a PE crashes or a node becomes unavailable. The voluntary deletion of job
resources are caught by the \code{onDeletion()} callback in that resource's
controller. In this situation, the deleted resource is recreated
by the controller if the owning job exists and is in the \code{Submitted} state.

Special care needs to be taken in the event of pod failure or PE deletion. To
maintain Streams' application consistency logic
(\S~\ref{sec:consistentregion}), restarting a pod needs to be coordinated with
both the PE and pod controllers through a causal chain (\S~\ref{sec:causal}). 


\begin{figure}[t!]
  \centering
  \includegraphics[width=0.9\linewidth]{figures/speed.png}
  \caption{\textbf{Reducing sampling steps.} FID as a function of sampling steps. Our method is superior to the baseline regardless of the sampling steps. Samples generated following~\cite{nichol2021improved} and DDIM~\cite{song2020denoising}. Models trained on FFHQ dataset.}
  \label{fig:speed}
\end{figure}

\begin{table}[]
\centering
\begin{tabular}{cccc}
\hline
Method                & Schedule    & FID-10k$\downarrow$   & KID-10k$\downarrow$ \\ \hline
\multirow{4}{*}{Base} & 250 uniform & 10.88 & 5.91 \\
                      & 130-60-60   & 10.62  &  5.14    \\
                      & 60-130-60   & 12.23  &  7.54  \\
                      & 500 uniform & 9.74  & 4.48 \\ \hline
Ours                  & 250 uniform & \textbf{8.92}  & \textbf{4.24}  \\ \hline
\end{tabular}
\caption{\textbf{Sweeping sampling schedule.} Schedules expressed as a sequence of integers which are numbers of steps assigned to one-third of the diffusion process. 130-60-60 implies consuming more steps near $t=0$. Modifying the sampling schedule can slightly improve performance, but does not exceed our improvement. %Even doubling sampling steps of the baseline is inferior to our result.
}

\label{table:sweep}
\end{table}


