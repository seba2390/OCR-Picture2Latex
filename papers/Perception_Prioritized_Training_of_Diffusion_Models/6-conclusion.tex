\section{Conclusion}
\label{sec:conclusion}
We proposed perception prioritized weighting, a new weighting scheme for the training objective of the diffusion models. We investigated how the model learns visual concepts at each noise level during training, and divided diffusion steps into three groups. %We found that the model learns rich contexts by learning to recover signals from heavily corrupted images. %, while the model learns imperceptible details when recovering signals from slightly corrupted images. 
We showed that even the simplest choice (P2) improves diffusion models across datasets, model configurations, and sampling steps. 
Designing a more sophisticated weighting scheme may further improve the performance, which we leave as future work. We believe that our method will open new opportunities to boost the performance of diffusion models. % used across various domains~\cite{kong2020diffwave,chen2020wavegrad} and tasks~\cite{choi2021ilvr,meng2021sdedit,saharia2021image}.

\vspace{1em}
\textbf{Acknowledgements:} This work was supported by Institute of Information \& communications Technology Planning \& Evaluation (IITP) grant funded by the Korea government (MSIT) [NO.2021-0-01343, Artificial Intelligence Graduate School Program (Seoul National University)], LG AI Research, Samsung SDS, AIRS Company in Hyundai Motor and Kia through HMC/KIA-SNU AI Consortium Fund, and the BK21 FOUR program of the Education and Research Program for Future ICT Pioneers, Seoul National University in 2022.