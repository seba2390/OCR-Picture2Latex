\documentclass[aps,twocolumn,amsmath,amssymb, superscriptaddress]{revtex4}
%\documentclass[aps,twocolumn,floatfix, showpacs, superscriptaddress]{revtex4-1}
\usepackage[pdftex]{graphicx}% Include figure files
 \usepackage{amsmath}
\usepackage[T1]{fontenc}
\usepackage{amssymb}
\usepackage{amsfonts}
\usepackage{bm}
 \usepackage{amsmath} 
%\usepackage[breaklinks=true,colorlinks=true,linkcolor=blue,urlcolor=blue,citecolor=blue]{hyperref}
\usepackage{lipsum}
%\usepackage{widetext}
\usepackage{amsfonts} 
\usepackage{amssymb, mathrsfs}
\usepackage{braket}
\usepackage{graphicx} 
\usepackage{subfigure}
\usepackage{bbm}
%\usepackage[colorlinks=true 
%,urlcolor=blue
%,anchorcolor=blue
%,citecolor=red
%,filecolor=blue
%,linkcolor=blue
%,menucolor=blue
%,pagecolor=blue
%,linktocpage=true
%,pdfproducer=medialab
%]{hyperref}
%\usepackage{ragged2e}
% amellal@yandex.ru
\def\beq{\begin{equation}}
\def\eeq{\end{equation}}
\def\bsp{\begin{split}}
\def\esp{\end{split}}
\def\bea{\begin{eqnarray}}
\def\eea{\end{eqnarray}}
\def\ba{\begin{array}}
\def\ea{\end{array}}
\def\nn{\nonumber \\}
\def\vsp#1{\vspace{#1}}
\def\ha{\hat  h (\hat  X, \hat  P)}
\def\ev{\exp {-i{\ha\over\hbar}}}
\def\undertext#1{$\underline{\hbox{#1}}$}
\def\haf#1{{{#1}\over 2}}
\def\dg{\dagger}
\def\sg{\sigma}
\def\up{\uparrow}
\def\dw{\downarrow}
\def\lb{\left(}
\def\rb{\right)}
\def\lr{\left|}
\def\rr{\right|}
\def\l.{\left.}
\def\r.{\right.}
\def\lam{t^r}
\def\ra{\rangle}
\def\la{\langle}
\def\pr{\prime}
\def\bo{\bold{k}}
\def\inty{\int_{-\infty}^\infty}


\def\x{\times}
\def\ox{\otimes}
\def\dag{\dagger}

%\bibliographystyle{prsty}
%\numberwithin{equation}{section}
%\renewcommand\theequation{\arabic{section}.\arabic{equation}}
\begin{document}

\date{\today}
\title{ Magnonic triply-degenerate nodal points\\
Supplemental Material}
\author{S. A. Owerre}
\affiliation{Perimeter Institute for Theoretical Physics, 31 Caroline St. N., Waterloo, Ontario N2L 2Y5, Canada.}


\maketitle


\section{  Spin transformation}
Generally, we consider  noncoplanar  chiral spin configurations, which can be induced by the in-plane DMI or an external  magnetic field. In this case there is a spin-canting behaviour out-of-plane leading to a nonzero scalar spin chirality.  In this paper, we consider noncoplanar spin structure induced by an applied external magnetic field along the stacking $z$ direction, $\mathcal H_Z=- H\sum_{i,\ell} S_{i,\ell}^z$ in units of $g\mu_B$. Next,  we perform a rotation about the $z$-axis  by spin orientated angles $\theta_i$ and then about $y$-axis by the magnetic-field-induced spin-canting angle $\eta$. The total rotation matrix is given by
\begin{align}
\mathcal{R}_z(\theta_{i,\ell})\cdot\mathcal{R}_y(\eta)
=\begin{pmatrix}
\cos\theta_{i,\ell}\cos\eta & -\sin\theta_{i,\ell} & \cos\theta_i\sin\eta\\
\sin\theta_{i,\ell}\cos\eta & \cos\theta_{i,\ell} &\sin\theta_{i,\ell}\sin\eta\\
-\sin\eta & 0 &\cos\eta
\end{pmatrix}.
\end{align}
 Consequently, the spins transform as $ \bold{S}_i=\mathcal{R}_z(\theta_{i,\ell})\cdot\mathcal{R}_y(\eta)\cdot\bold S_i^\prime,$
where prime denotes spins in the rotated frame. The classical ground state energy  is given by \begin{align}
E_{\text{cl}}&= 6NS^2\Big[-\frac{J_T}{2}\lb 1 - {3}\cos^2\eta\rb-\frac{J_D}{2}\lb 1 - 2\cos^2\eta\rb\\&\nonumber-\frac{\sqrt{3}}{2}D_z\sin^2\eta -\frac{J_\perp}{2}(1-2\cos^2\eta)-H\cos\eta\Big],
\end{align}
where $N$ is the number of sites per unit cell, and the magnetic field is rescaled in unit of $S$. Minimizing this energy yields the canting angle $\cos\eta = H/H_s$, where $H_s=3J_T+2J_D+\sqrt{3}D_z+ 2J_\perp$ is the saturation field. The spin interactions  contributing to the free magnon model (in terms of the  Holstein-Primakoff  bosons) are  given by



  \begin{align}
  \mathcal H_J&= \sum_{ \la ij\ra,\ell}J_{ij}\big[\cos\theta_{ij,\ell} \bold{ S}_{i,\ell}^\prime\cdot \bold{ S}_{j,\ell}^\prime+ \sin\theta_{ij,\ell}\cos\eta \hat{\bold z}\cdot\lb \bold{ S}_{i,\ell}^\prime\times\bold{ S}_{j,\ell}^\prime\rb \\&\nonumber+2\sin^2\lb\frac{\theta_{ij,\ell}}{2}\rb\lb\sin^2\eta  S_{i,\ell}^{\prime x}S_{j,\ell}^{\prime x} +\cos^2\eta S_{i,\ell}^{\prime z} S_{j,\ell}^{\prime z}\rb\big],
  \end{align}
  
 \begin{align}
\mathcal   H_{D_z}&= -D_z\sum_{\la ij\ra,\ell}\Big[\cos\theta_{ij,\ell}\cos\eta {\bf \hat z}\cdot \lb \bold S_{i,\ell}^\prime\times\bold S_{j,\ell}^\prime\rb  \\&\nonumber - \sin\theta_{ij,\ell}\lb \cos^2\eta S_{i,\ell}^{\prime x}S_{j,\ell}^{\prime x} + S_{i,\ell}^{\prime y}S_{j,\ell}^{ \prime y}+\sin^2\eta S_{i,\ell}^{\prime z}S_{j,\ell}^{\prime z}\rb\Big], 
     \end{align}
 
    \begin{align}
  \mathcal H_{J_\perp}&= J_\perp\sum_{ i,\la \ell\ell^\prime\ra}\big[\cos\theta_{\ell\ell^\prime} \bold{ S}_{i,\ell}^\prime\cdot \bold{ S}_{i,\ell^\prime}^\prime  \\&\nonumber  +2\sin^2\lb\frac{\theta_{\ell\ell^\prime}}{2}\rb\lb\sin^2\eta  S_{i,\ell}^{\prime x}S_{i,\ell^\prime}^{\prime x} +\cos^2\eta S_{i,\ell}^{\prime z} S_{i,\ell^\prime}^{\prime z}\rb\big],
  \label{rotp}
  \end{align}
  \begin{align}
 \mathcal H_Z= -H\cos\eta\sum_{i,\ell} S_{i,\ell}^{\prime z},
  \end{align}
where $\theta_{\alpha\beta}=\theta_{\alpha}-\theta_{\beta}$. For antiferromagnetic interlayer coupling $J_\perp>0$ we have $\theta_{\ell\ell^\prime}=\pi$, whereas  $\theta_{\ell\ell^\prime}=0$ for ferromagnetic interlayer $J_\perp<0$. In this latter case  only the first term in  Eq.~\eqref{rotp} is nonzero. In addition, $\theta_{ij}=\pi$ on the dimer bonds $J_D$, whereas $\theta_{ij}=2\pi/3$ on the triangular  bonds $J_T$.  The noncoplanar  (umbrella) spin configuration has a nonzero scalar spin chirality given by $\chi_{ijk;l}=  {\bf S}_{i,\ell}^\prime\cdot({\bf S}_{j,\ell}^\prime\times{\bf S}_{k,\ell}^\prime)$, and it is induced only within the decorated honeycomb lattice  planes. 
 \begin{figure}
\centering
\includegraphics[width=1\linewidth]{TP}
\caption{Color online. Magnonic TP phase below (a), at (b), and above (c) TP$_2$  for fixed  $k_z^{\text{TP}_2}=1.6520\times 2\pi/c$. The parameters are the same as  in Fig.~\ref{TDNLM}(a).}
\label{TP}
\end{figure} 
\section{Holstein-Primakoff  bosons}

Next, we introduce the  Holstein-Primakoff  bosons: $
S_{i,\ell}^{z}= S-a_{i,\ell}^\dagger a_{i,\ell},~  S_{i,\ell}^{+} \approx  \sqrt{2S}a_{i,\ell}=\lb S_{i,\ell}^{-}\rb^\dg$, where $ S_{i,\ell}^{\pm}=S_{i,\ell}^{x}\pm i S_{i,\ell}^{y}$ and $a_{i,\ell}^\dagger(a_{i,\ell})$ are the bosonic creation (annihilation) operators. In the following we consider the case $J_\perp>0$. The case $J_\perp<0$ can be derived in a similar way. The magnon hopping Hamiltonians are given by
\begin{align}
\mathcal H_{J_T-D_z}&= S\sum_{\la ij\ra,\ell}\big[ t^z(a_{i,\ell}^\dg a_{i,\ell} +a_{j,\ell}^\dg a_{j,\ell}) \\&\nonumber + t^r(e^{-i\phi_{ij,\ell}}a_{i,\ell}^\dg a_{j,\ell} + h.c.)+ t^o(a_{i,\ell}^\dg a_{j,\ell}^\dg + h.c.)\big],
\end{align}
\begin{align}
\mathcal H_{J_D}&= S\sum_{\la ij\ra,\ell}\big[ t^{z}_D(a_{i,\ell}^\dg a_{i,\ell} +a_{j,\ell}^\dg a_{j,\ell}) \\&\nonumber + t^{r}_D(a_{i,\ell}^\dg a_{j,\ell} + h.c.)+ t^{o}_D(a_{i,\ell}^\dg a_{j,\ell}^\dg + h.c.)\big],
\end{align}
 \begin{align}
\mathcal H_{J_\perp}&= S\sum_{i,\ell} t_\perp^za_{i,\ell}^\dg a_{i,\ell}  \\&\nonumber + S\sum_{i,\la \ell\ell^\prime\ra}\big[ t_\perp^r(a_{i,\ell}^\dg a_{i,\ell^\prime} + h.c.) + t_\perp^o(a_{i,\ell}^\dg a_{i,\ell^\prime}^\dg + h.c.)\big],
\end{align}
\begin{align}
\mathcal H_{Z}&=H\cos\eta\sum_{i,\ell}a_{i,\ell}^\dg a_{i,\ell}.
\end{align}
   The  solid angle subtended by three non-coplanar spins  is given by $\phi_{ij}=\pm\phi$, where $\phi=\tan^{-1}[t_{2}^r/t_{1}^r]$. The parameters of the tight binding model are given by 

\begin{align}
&t^z= -J_T\lb -\frac{1}{2}+\frac{3}{2}\cos^2\eta\rb+\frac{ \sqrt{3}D_z}{2}\sin^2\eta,\\
&t^r=\sqrt{(t_1^r)^2+(t_2^r)^2},\\
&t_1^r= J_T\lb-\frac{1}{2} +\frac{3\sin^2\eta}{4}\rb-\frac{\sqrt{3}D_z}{2}\lb 1-\frac{\sin^2\eta}{2}\rb,\\
&t_2^r= \cos\eta\lb -\frac{\sqrt{3}J_T}{2}+\frac{D_z}{2}\rb,\\
&t^{o}=\frac{\sin^2\eta}{2}\lb \frac{3J_T}{2}+\frac{\sqrt{3}D_z}{2}\rb,\\
&t^{z}_D= \frac{J_D}{2}\lb -1+2\cos^2\eta\rb,\\
&t^r_D= J_D\lb-1 +\sin^2\eta\rb,\\
& t^{o}_D=J_D\sin^2\eta,\\
& t_\perp^z= -\frac{J_{\perp}}{2}\cos 2\eta,~
t_\perp^r=-  J_{\perp}\cos 2\eta,~
t_\perp^o=J_{\perp} \sin^2\eta.
\end{align}

 
\section{Magnon Hamiltonian}

Next, we Fourier transform into momentum space. The resulting  Hamiltonian 
using the  basis vector $\Psi^\dg_\bo=(\psi^\dg_\bo,\psi^\dg_{-\bo})$, where $\psi^\dg_\bo=(a_{\bo 1}^{\dg},\thinspace a_{\bo 2}^{\dg},\thinspace a_{\bo 3}^{\dg}, \thinspace a_{\bo 4}^{\dg},\thinspace a_{\bo 5}^{\dg},\thinspace a_{\bo 6}^{\dg} )$, is given by 
\begin{align}
\boldsymbol{\mathcal{H}}(k_z,k_\parallel)= 
\begin{pmatrix}
\boldsymbol{\mathcal{A}}(k_z,k_\parallel, \phi) & \boldsymbol{\mathcal{B}}(k_z,k_\parallel)\\
\boldsymbol{\mathcal{B}}^*(-k_z,-k_\parallel)&\boldsymbol{\mathcal{A}}^*(-k_z,-k_\parallel,\phi)
\end{pmatrix},
\label{ham}
\end{align}
where 
\begin{align}
\boldsymbol{\mathcal{A}}(k_z,k_\parallel)= &
\begin{pmatrix}
\bold{A}_1(k_z,\phi) & \bold{A}_2(k_\parallel)\\
\bold{A}_2(-k_\parallel)&\bold{A}_1(k_z,\phi)
\end{pmatrix},\\
\boldsymbol{\mathcal{B}}(k_z,k_\parallel)= &
\begin{pmatrix}
\bold{B}_1(k_z) & \bold{B}_2(k_\parallel)\\
\bold{B}_2(-k_\parallel)& \bold{B}_1(k_z)
\end{pmatrix},
\end{align}
 
\begin{align}
\bold{A}_1(k_z,\phi)=& 
\begin{pmatrix}
t_0(k_z)&t^r e^{-i\phi}&t^r e^{i\phi}\\
t^r e^{i\phi}&t_0(k_z)&t^r e^{-i\phi}\\
t^r e^{-i\phi}&t^r e^{i\phi}&t_0(k_z)
\end{pmatrix},\\
\bold{A}_2(k_\parallel)=&t^{r}_D
\begin{pmatrix}
 e^{ik_{2\parallel}}&0&0\\
0& e^{ik_{1\parallel}}&0\\
0&0&1
\end{pmatrix},
\end{align}
\begin{align}
\bold{B}_1(k_z)= &
\begin{pmatrix}
t_\perp^o\cos k_z c&t^o &t^o \\
t^o &t_\perp^o\cos k_z c&t^o \\
t^o &t^o &t_\perp^o\cos k_z c
\end{pmatrix},\\
\bold{B}_2(k_\parallel)=&t^o_D
\begin{pmatrix}
e^{ik_{2\parallel}}&0&0\\
0& e^{ik_{1\parallel}}&0\\
0&0&1\\
\end{pmatrix},
\end{align}
where $t_0(k_z)=H\cos\eta+2(t^z + t^{z}_D+t_\perp^{z})=J_T+J_D +\sqrt{3}D_z+J_\perp+t_\perp^r\cos k_z c$ and $k_{i\parallel}=k_\parallel\cdot \bold{e}_{i}$, with  $k_\parallel=(k_x,k_y)$ and  $\bo=(k_\parallel,k_z)$. The lattice basis vectors are chosen as $\bold e_1=2a\hat{x}$ and  $\bold e_2=a(\hat{x} + \sqrt{3}\hat{ y})$.  The magnon band structure below, at, and above TP$_2$ are shown in Fig.~\ref{TP}. 

\end{document}


