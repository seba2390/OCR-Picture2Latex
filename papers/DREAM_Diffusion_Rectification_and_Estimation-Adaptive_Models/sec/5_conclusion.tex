\section{Conclusion}
This paper introduces DREAM, a novel training framework designed to address the training-sampling discrepancy in conditional diffusion models with minimal code modifications. DREAM comprises two key components: diffusion rectification and estimation adaptation. Diffusion rectification extends the existing training framework for diffusion models by aligning training more closely with sampling through self-estimation. Estimation adaptation optimizes the balance between accuracy and fidelity by adaptively incorporating ground-truth information. When applied to SISR tasks, DREAM effectively bridges the gap between training and sampling. Extensive experiments demonstrate that DREAM enhances distortion and perception metrics across various diffusion-based SR baselines. It also speeds up training, improves sampling efficiency, and achieves robust OOD performance across diverse datasets and scales. 

While DREAM is mainly utilized for SR in this work, its capabilities are applicable to a range of dense visual prediction tasks. Future research may investigate its use in both low-level vision tasks, such as inpainting and deblurring, and high-level vision tasks like semantic segmentation and depth estimation. Additionally, exploring DREAM's application in both unconditional and conditional image generation presents an intriguing direction for future work.

% \newpage

