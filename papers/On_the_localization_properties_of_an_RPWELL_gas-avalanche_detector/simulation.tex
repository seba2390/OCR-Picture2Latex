%%%%%%%%%%%%%%%%%%%%%%%%%%%%%%%%%%%%%%%%%%%%%%%%%%%%%%%%%%%%%%%%%%%%%%%%%%%%%%
\section{Monte Carlo simulations}
\label{sec: Simulation}
%%%%%%%%%%%%%%%%%%%%%%%%%%%%%%%%%%%%%%%%%%%%%%%%%%%%%%%%%%%%%%%%%%%%%%%%%%%%%%

The RPWELL detector operation was modeled by a Monte Carlo (MC) simulation. A simplified THGEM-electrode geometry was chosen, with the same parameters as indicated in section~\ref{sec: RPWELL}, but with a "homogeneous" array of holes - without the central 1.3~mm spacing (figure~\ref{fig: RPWELL detector}). Considering the number of holes involved in the measurement, this difference in geometry should not affect the results significantly. The simulation was performed on an event-by-event basis, and it included the physics processes detailed below. The detector performance was reproduced for a set of 20,000 events generated randomly across the detector plane. To emulate the strip readout, all the estimated quantities were integrated along the y-axis and presented, when needed, as a function of the x-axis only.

\subsection{Event generator}
Using the Garfield simulation framework~\cite{veenhof2015garfield}, together with the neBEM solver~\cite{muhkopadhyay2006computation}, Heed~\cite{smirnov2005modeling} and Magboltz software~\cite{biagi2016magboltz}, we implemented a single-event generator. To make this method CPU-effective, we didn't simulate all the physics processes from first principles, but instead we used measured data as inputs whenever this was possible, in particular:
\begin{itemize}
\item Single-electron gain spectrum, measured with UV-photons in similar experimental conditions.
\item Cluster-charge spectrum, from the test-beam with muons (figure~\ref{fig: spectrum - residuals histo}-b).
\item Typical signal shapes from the readout strips, digitized by the SRS/APV25 electronics (figure~\ref{fig: induced signal}).
\end{itemize}

For each event we considered the following physics processes:
\begin{enumerate}
\item \underline{Primary-electron clusters generation} by a 150~GeV muon traversing the detector, distributed along a track within the drift gap. This process was simulated as described in appendix~\ref{sec: PE library}.
\item \underline{Primary-electron drift and diffusion} under electric field towards the THGEM and their focusing into the holes. 
\item \underline{Primary-electron multiplication} in the high-field region of the multiplier holes by avalanche process. 
\item \underline{Current-signal induction on the anode strips} by the drift of avalanche-generated electrons and ions towards the anode and the THGEM top respectively. 
The long ($\sim$2~$\upmu$s) current pulse was shaped and digitized by the readout electronics, with a shaping time of $\sim$100~ns. While in these simulations we used as input the measured signal (see details in appendix~\ref{sec: appendix}), investigating the signal formation process in detail will be part of another dedicated study.
\item \underline{Event position reconstruction} by the charge-weighted signals. Residuals are defined as the difference between the original event position and the reconstructed one.
\end{enumerate}

More details on the single-event generator are given in the appendix. 

\subsection{Simulation results}
\label{sec: simulation results}

The simulations were performed for the detector's operation conditions yielding the best experimental position resolution, namely with \dvrpwell= 975~V.

Figure~\ref{fig: simulated spectrum} shows the Landau fit of the simulated charge spectrum. It is in good agreement with the measured one (figure~\ref{fig: spectrum - residuals histo}-b). Figure~\ref{fig: simulation residuals - profile}-a shows the simulated residuals histogram; the position resolution of 0.22~mm RMS is superior to the experimental one of 0.28~mm shown in figure~\ref{fig: spectrum - residuals histo}-a. The small difference could be due to detector noise and gain non-uniformity.
The simulated profile of the reconstructed track-position (top panel of figure~\ref{fig: simulation residuals - profile}-b) is in good agreement with the experimental distribution shown in the middle panel of figure~\ref{fig: profile - local residuals}. In both cases, the peaks are correlated with the THGEM holes, due to charge focusing into holes. The characteristic pattern of the residuals as a function of the event position, due to primary-charge focusing into the holes (bottom panel of figure~\ref{fig: profile - local residuals}), is clearly visible in the simulated data (bottom panel of figure~\ref{fig: simulation residuals - profile}-b). In figure~\ref{fig: local residuals zoom} we compare the simulated local residuals with the measured ones in a small detector region. The patterns are very similar, with the residuals reaching zero values in correspondence to THGEM holes and the central region between holes; the latter due to charge sharing (see figure~\ref{fig: charge sharing} in the appendix).


\begin{figure}[h]
\centering
\includegraphics[scale=0.3]{figures/spectrum.pdf}
\caption{Simulated RPWELL cluster charge-spectra for muons at normal incidence fitted to a Landau function with the indicated most probable value (MPV) and sigma parameter. \nech gas; effective gain $\sim$3$\times$10$^3$; 1~mm strips pitch.}\label{fig: simulated spectrum}
\end{figure}

We could also reproduce the position resolution dependence on the incidence angle and on the drift field, as can be seen in figure~\ref{fig: HV scan - drift scan}-b and \ref{fig: angle scan}-b. The simulation confirms the measured degradation of the position resolution at increasing angles and the negligible effect of the drift field up to 1.5~kV/cm.
To demonstrate that the detector localization properties strongly depend on the THGEM hole pitch, we simulated the position resolution for different pitch values. Varying the pitch from 0.8~mm to 1.5~mm resulted in an almost two-fold degradation in the position resolution (figure~\ref{fig: simulated pitch scan}-a). In the same pitch range, the detector gain was not affected, but we could see some primary electron losses at the largest hole-pitch value (figure~\ref{fig: simulated pitch scan}-b).
Aiming at improving the RPWELL localization properties, we simulated the performance of the detector operating in \arch. For this simulation we kept the same parameters as for \nech, except for the primary-electron library (see the appendix), that was recalculated for \arch with \dvrpwell= 1800~V, giving an effective gain of $\sim$3$\times$10$^3$ ( similar to the one in \nech at \dvrpwell= 975~V).
Because of the larger number of primary electrons produced by a MIP in argon with respect to neon, the fluctuations of the charge sharing among THGEM holes in this configuration were smaller, resulting in an improved position resolution - compared to \nech - of 0.19~mm RMS at normal incidence. The calculated RMS at incidence angles up to 40$^\circ$ is shown in figure~\ref{fig: angle scan}-b. Compared to neon, the degradation at increasing incidence angles is better, 3-times instead of 4-times. This is because in argon the primary electron distribution along the track is more uniform, preserving a better correlation between the particle position and the reconstructed cluster.

Good agreement is found between the measured detector performances and the simulated one. This indicates that the most relevant physics processes contributing to the position resolution were taken into account correctly.  

\begin{figure}[h]
\begin{subfigure}[t]{0.45\textwidth}\caption{}
\includegraphics[scale=0.3]{figures/residuals_histo.pdf}
\end{subfigure}
\begin{subfigure}[t]{0.6\textwidth}\caption{}
\includegraphics[scale=0.45]{figures/beam_profile_res_local.pdf}
\end{subfigure}
\caption{Simulation of the RPWELL operated in \nech mixture detecting 150~GeV muons at normal incidence: a) Residuals histogram, resulting in 0.22~mm RMS position resolution value. b) Simulated detector response. Reconstructed clusters position histogram along the x-axis (top), and local residuals pattern (bottom). Effective gain $\sim$3$\times$10$^3$; 1~mm strips pitch.}\label{fig: simulation residuals - profile}
\end{figure}

\begin{figure}[h]
\begin{subfigure}[t]{0.5\textwidth}\caption{}
\includegraphics[scale=0.3]{figures/run0312_residuals_local_ZSF1_W3_x6-19_y0-170_zoom.pdf}
\end{subfigure}
\begin{subfigure}[t]{0.5\textwidth}\caption{}
\includegraphics[scale=0.3]{figures/local_residuals_zoom.pdf}
\end{subfigure}
\caption{Local residual patterns for muons at normal incidence in the RPWELL operated in \nech mixture: measured (a) and simulated (b). Effective gain $\sim$3$\times$10$^3$; 1~mm strips pitch.}\label{fig: local residuals zoom}
\end{figure}

\begin{figure}[h]
\begin{subfigure}[t]{0.5\textwidth}\caption{}
\includegraphics[scale=0.3]{figures/pitch_scan.pdf}
\end{subfigure}
\begin{subfigure}[t]{0.5\textwidth}\caption{}
\includegraphics[scale=0.3]{figures/electron_fucusing_efficiency.pdf}
\end{subfigure}
\caption{Simulated RPWELL position resolution for muons at normal incidence (a) and electron focusing efficiency into the THGEM holes (b) as a function of the THGEM holes pitch in \nech mixture. Effective gain $\sim$3$\times$10$^3$; 1~mm strips pitch.}\label{fig: simulated pitch scan}
\end{figure}



