%%%%%%%%%%%%%%%%%%%%%%%%%%%%%%%%%%%%%%%%%%%%%%%%%%%%%%%%%%%%%%%%%%%%%%%%%%%%%%
\appendix

\section{Monte Carlo single event generator}
\label{sec: appendix}
%%%%%%%%%%%%%%%%%%%%%%%%%%%%%%%%%%%%%%%%%%%%%%%%%%%%%%%%%%%%%%%%%%%%%%%%%%%%%%
The Monte Carlo simulation single-event generator is described. To make it CPU- effective, we didn't simulate all physics processes involved from first principles, but we rather used measured data as inputs whenever this was possible - in particular the following ones:
\begin{itemize}
\item Primary-electrons library (see section~\ref{sec: PE library} below).
\item Single-electron gain spectrum measured with UV photons in similar experimental conditions. This was fitted to an exponential.
\item Cluster-charge spectrum from the test-beam muons (figure~\ref{fig: spectrum - residuals histo}-b).
\item Typical signal shapes from the readout strips digitized by the SRS/APV25 electronics (figure~\ref{fig: induced signal}). We fitted it to a double Gaussian to consider the tails due to the charge spread on the resistive-plate surface. This effect is explained in~\cite{lin2014signal} for a resistive Micromegas detector and will be investigated in a future work. In figure~\ref{fig: simulated signal} we show an example of a simulated signal. 
\end{itemize}

We modeled the full detector geometry using the Garfield simulation framework~\cite{veenhof2015garfield}, together with the neBEM solver~\cite{muhkopadhyay2006computation}, Heed~\cite{smirnov2005modeling} and Magboltz software~\cite{biagi2016magboltz}. 
Below we describe the generation of the primary-electrons library and the single-event generator which contains all the steps of the Monte Carlo simulation.

\begin{figure}[h]
\centering
\includegraphics[scale=0.3]{figures/signal_sim.pdf}
\caption{Simulated digitized signal from the SRS/APV25 readout from strips.}\label{fig: simulated signal}
\end{figure}

\subsection{Primary electrons library}
\label{sec: PE library}

As an input for the Monte Carlo simulation we created a library, which contains the information about the primary charges production and their drift into the THGEM holes for simulated muon track events. 
The steps that we followed for each event are:

\begin{enumerate}
\item \underline{Primary electrons production}: Heed~\cite{smirnov2005modeling} was used to generate clusters of ionization primary electrons (PE) along the 150~GeV muon track. 
\item \underline{Charge sharing and time-cut}: the Microscopic Drift Routine of Garfield~\cite{veenhof2015garfield} was used to drift the primary electrons along the field lines into the RPWELL holes. Considering the finite shaping time ($\sim$100~ns) of the APV25 readout chip~\cite{french2001design} and the arrival-time-spread of the primary electrons into the THGEM holes ($\sim$250~ns from-first-to-last~\cite{peisert1984drift}),  we decided to apply an event-dependent cut on the arrival time of the electrons before counting them. 
We let the electrons reaching the high field region to initiate avalanches. All avalanche electrons and ions were drifted to the RP top and THGEM top respectively. We used the Shockley Ramo theorem~\cite{ramo1939currents,shockley1938currents} to calculate the induced current on a uniform readout anode. This current pulse was then convoluted with the readout electronics transfer function~\cite{french2001design}. The resulting signal peaking time was considered as the time-cut. Only the primary electrons reaching a hole before then were considered. Figure~\ref{fig: charge sharing} shows, for perpendicular tracks at different distances from the hole-center, the fraction of primary charge reaching the closest hole and the first neighbor before and after applying the time cut. As can be seen, when the track is passing in the middle between two holes, the primary charge is shared equally.
\end{enumerate}

A set of 2000 events was produced for different track distances on the x-axis from a reference hole center (0 to 0.96~mm in steps of 0.05~mm). For each event we stored the number of primary electrons reaching the reference THGEM hole and also the next 7 holes in the direction of the track inclination. We neglected the holes in the other side since very rarely electrons were drifted there. Electrons reaching different holes along the y-axis are summed up.

A similar set was produced for different track incidence angles (0$^\circ$ to 40$^\circ$ in steps of 10$^\circ$). 

\begin{figure}[h]
\begin{subfigure}[t]{0.3\textwidth}\caption{}
\includegraphics[scale=0.2]{figures/charge_sharing_nocut_2D_000mm.pdf}
\end{subfigure}
\begin{subfigure}[t]{0.3\textwidth}\caption{}
\includegraphics[scale=0.2]{figures/charge_sharing_nocut_2D_024mm.pdf}
\end{subfigure}
\begin{subfigure}[t]{0.3\textwidth}\caption{}
\includegraphics[scale=0.2]{figures/charge_sharing_nocut_2D_048mm.pdf}
\end{subfigure}\\
\begin{subfigure}[t]{0.3\textwidth}\caption{}
\includegraphics[scale=0.2]{figures/charge_sharing_2D_024mm.pdf}
\end{subfigure}
\begin{subfigure}[t]{0.3\textwidth}\caption{}
\includegraphics[scale=0.2]{figures/charge_sharing_2D_024mm.pdf}
\end{subfigure}
\begin{subfigure}[t]{0.3\textwidth}\caption{}
\includegraphics[scale=0.2]{figures/charge_sharing_2D_048mm.pdf}
\end{subfigure}
\caption{The fraction of primary charge reaching the closest hole and the first neighbor for perpendicular tracks at different distances from the hole center, before (a) and after (b) applying the time cut (see text).}\label{fig: charge sharing}
\end{figure}

\subsection{Single-event Monte Carlo generator}
\label{sec: single event generator}

For each event the steps were: 
\begin{enumerate}
\item Generate randomly an x-position, x$\mathrm{_{tr}}$.
\item Draw randomly from the relevant primary-electrons library the number of PEs reaching the reference hole and the neighboring holes.
\item Calculate the total event charge and the fraction produced in each hole: 
\begin{enumerate}
\item Draw a random number g$\mathrm{_{ij}}$ from the single electron exponential gain distribution. This number represents the single primary electron avalanche gain, where \textit{i} corresponds to the hole where the primary electron ended and \textit{j} identifies the specific primary electron.  
\item Draw a random number Q distributed accordingly to the Landau fit to the cluster-charge spectrum (figure~\ref{fig: spectrum - residuals histo}-b).  This number represents the total charge produced in the event.
\item The fraction of the total charge produced in the hole \textit{i}, was then calculated as Q$\mathrm{_i}$= S$\mathrm{_i}\cdot$Q, where S$\mathrm{_i}$ = $\mathrm{\sum_j}$g$\mathrm{_{ij}}$/ $\mathrm{\sum_{ij}}$g$\mathrm{_{ij}}$.
\end{enumerate}
\item Produce the digitized signal (figure~\ref{fig: simulated signal}):
\begin{enumerate}
\item Create typical signal's double-Gaussian function f$\mathrm{_i}$ for each relevant hole. The mean of the distribution was set to the hole center with a small correction towards the track position (this effect is described in~\cite{rubin2013optical}).
\item Create histogram representing the readout. The bin width corresponds to the strips pitch, and the bin center corresponds to the strip position.
\item For each hole \textit{i}, fill the histogram with a number of events equal to Q$\mathrm{_i}$ distributed according to f$\mathrm{_i}$.
\end{enumerate}
\item Calculate the reconstructed event position: 
The reconstructed position of the simulated event was obtained by the charge-weighted centroid of the readout histogram x$\mathrm{_{rec}}$= $\mathrm{\sum x_iq_i}$/Q, where x$\mathrm{_i}$ is the bin center, and q$\mathrm{_i}$ is the bin content. The simulated residual for each event was calculated as RES= x$\mathrm{_{tr}}$-x$\mathrm{_{rec}}$. To effectively reproduce the effect of the ZSF threshold we considered only bins with >20 entries.
\end{enumerate}
