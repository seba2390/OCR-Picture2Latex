%%%%%%%%%%%%%%%%%%%%%%%%%%%%%%%%%%%%%%%%%%%%%%%%%%%%%%%%%%%%%%%%%%%%%%%%%%%%%%
\section{Summary and discussion}
\label{sec: Summary and discussion}
%%%%%%%%%%%%%%%%%%%%%%%%%%%%%%%%%%%%%%%%%%%%%%%%%%%%%%%%%%%%%%%%%%%%%%%%%%%%%%

Robust THGEM-based particle detectors were conceived for applications requiring particle tracking over large area, at moderate sub-millimeter localization resolutions. The localization properties of an RPWELL detector were investigated in this work for the first time. The detector comprises of a single-stage THGEM electrode, with 0.5~mm diameter holes, with 1~mm pitch; it is coupled to a readout plane with 1-D strips, via a glass resistive plate. Measurements with 150~GeV muons showed that the position resolution improves with increasing detector operation voltage (effective gain); this is attributed to the better signal-to-noise ratio causing an enhanced effective charge sharing among multiplier holes. The drift field in the conversion gap did not affect the resolution, indicating that in this configuration in \nech the electron transverse diffusion doesn't play a major role. The best position resolution measured in the present configuration, at normal incidence, is 0.28~mm RMS, at an effective gain $\sim$3$\times$10$^3$; it is about 4-fold smaller than the holes pitch. 
The position resolution deteriorated under an angular particle incidence. The measured resolution degraded from 0.28 to 0.8~mm RMS for respective incidence angles of 0$^\circ$ to 40$^\circ$. The main factor determining this degradation is that the primary electrons are produced in several clusters randomly distributed along a track. At non-0 incidence angles, the fraction of primary charges reaching each hole is not correlated uniquely with the track position. A possible way to contrast this effect would be using higher gas densities, obtaining a more uniform primary electron distribution along a track, and having a smaller drift gap, resulting in a smaller lateral primary charge distribution.

Monte Carlo simulations, incorporating the physics phenomena contributing to the position resolution, were carried out for predicting the expected detector performance. The simulation results reproduced rather well the experimental ones and confirm the prediction that the RPWELL electrode geometry plays a dominant factor in determining the position resolution.
This is very different from GEM-like structures where the hole size and pitch are significantly smaller, with primary charges being always distributed among several holes; this permits calculating their center-of-gravity - yielding position resolutions of a few tens of microns~\cite{lener2016mu,carnegie2005resolution}. 
The position resolution of the RPWELL can be improved if the charge sharing between the holes is improved. This can be achieved using different gaseous mixtures or optimizing the THGEM geometry. 