%%%%%%%%%%%%%%%%%%%%%%%%%%%%%%%%%%%%%%%%%%%%%%%%%%%%%%%%%%%%%%%%%%%%%%%%%%%%%%
\section{Test-beam results}
\label{sec: Results}
%%%%%%%%%%%%%%%%%%%%%%%%%%%%%%%%%%%%%%%%%%%%%%%%%%%%%%%%%%%%%%%%%%%%%%%%%%%%%%

\begin{figure}[h]
\begin{subfigure}[t]{0.5\textwidth}\caption{}
\includegraphics[scale=0.3]{figures/run0312_residuals_histo_ZSF1_W3_x6-19_y0-170.pdf}
\end{subfigure}
\begin{subfigure}[t]{0.5\textwidth}\caption{}
\includegraphics[scale=0.3]{figures/run0312_spectrum_calibrated_ZSF1_W3_x6-19_y0-170.pdf}
\end{subfigure}
\caption{Residuals histogram (a) and calibrated cluster-charge spectrum (b) recorded with the detector shown in figure~\ref{fig: RPWELL detector}. The cluster-charge spectrum is fitted to a Landau function. The residuals histogram is fitted to a Gaussian, which RMS defines the detector position resolution (here 0.28mm RMS). \nech gas; \dvrpwell= 975~V (effective gain $\sim$3$\times$10$^3$); 1~mm strips pitch; 50~Hz 150~GeV/c muons at normal incidence. }\label{fig: spectrum - residuals histo}
\end{figure}

The best position resolution of the RPWELL detector was obtained at the maximum achievable voltage \dvrpwell = 975~V\footnote{Note that as in~\cite{bressler2016first} (in similar conditions but with a different RP material), \dvrpwell could not be raised above 975~V, due to the appearance of small discharges, at the nA level.}, corresponding to an effective gain of $\sim$3$\times$10$^3$. In figure~\ref{fig: spectrum - residuals histo}-a we show the residuals histogram with its Gaussian fit; the corresponding RMS-value of the distribution is 0.28~mm. In figure~\ref{fig: spectrum - residuals histo}-b the cluster-charge spectrum for the same measurement is fitted to a Landau function.  At lower voltage values the localization properties of the detector degraded, as shown in figure~\ref{fig: HV scan - drift scan}-a were the residuals histogram RMS is plotted as a function of \dvrpwell. This can be explained as follows: position resolution better than the hole spacing is obtained only if the charge is shared between several holes. A better signal-to-noise ratio due to higher operation voltages gives a better sensitivity also to avalanches starting from small primary charge, resulting in effectively higher charge sharing between holes, and therefore improved position resolution. The relationship between position resolution and holes multiplicity is explained in more detail below and in section~\ref{sec: Simulation}. 

\begin{figure}[h]
\begin{subfigure}[t]{0.5\textwidth}\caption{}
\includegraphics[scale=0.3]{figures/HV_scan_160815_2_posres_ZSF1_W3_x8-20_y0-170.pdf}
\end{subfigure}
\begin{subfigure}[t]{0.5\textwidth}\caption{}
\includegraphics[scale=0.3]{figures/drift_scan_meassim.pdf}
\end{subfigure}
\caption{a) The measured RPWELL position resolution (residuals RMS) as a function of \dvrpwell.  b) The measured and simulated (see section~\ref{sec: Simulation}) position resolution at the maximum reachable value of \dvrpwell~ (effective gain $\sim$3$\times$10$^3$) for different values of the drift field. \nech gas; 1~mm strips pitch; 50~Hz 150~GeV/c muons at normal incidence. }\label{fig: HV scan - drift scan}
\end{figure}

\begin{figure}[h]
\centering
\includegraphics[scale=0.7]{figures/run0312_beam_profile_tracker_local-res_ZSF1_W3_x6-19_y0-170.pdf}
\caption{The reconstructed muon-beam profile along the x-axis measured by the tracker (top) is compared to the reconstructed beam profile recorded by the RPWELL detector (middle). (bottom) Local-residuals pattern for the same data-set. The peaks in RPWELL-detector distribution correspond to the holes locations. \nech gas; \dvrpwell= 975~V(effective gain $\sim$3$\times$10$^3$); 1~mm strips pitch; 50~Hz 150~GeV/c muons at normal incidence.}\label{fig: profile - local residuals}
\end{figure}

\begin{figure}[h]
\begin{subfigure}[t]{0.5\textwidth}\caption{}
\includegraphics[scale=0.3]{figures/run0610_residuals_local_ZSF1_W5_x38-47_y20-190.pdf}
\end{subfigure}
\begin{subfigure}[t]{0.5\textwidth}\caption{}
\includegraphics[scale=0.3]{figures/run0610_residuals_local_ZSF1_W5_x38-47_y20-190_corr.pdf}
\end{subfigure}
\caption{The local residual value vs particle track location at an incidence angle of $\uptheta$= 40$^\circ$, before (a) and after (b) linear correction. \nech gas; \dvrpwell= 975~V(effective gain $\sim$3$\times$10$^3$); 1.5~mm strips pitch; 50~Hz 150~GeV/c muons.}\label{fig: angle correction}
\end{figure}

\begin{figure}[h]
\begin{subfigure}[t]{0.5\textwidth}\caption{}
\includegraphics[scale=0.3]{figures/Angle_scan_posres_histo_ZSF1_W5_160823.pdf}
\end{subfigure}
\begin{subfigure}[t]{0.5\textwidth}\caption{}
\includegraphics[scale=0.3]{figures/angle_scan_meassim.pdf}
\end{subfigure}
\caption{Distributions of the residuals (a); measured and simulated (see section~\ref{sec: Simulation}) RMS position resolution (b) for different particle-incident angles. \nech gas; \dvrpwell= 975~V (effective gain $\sim$3$\times$10$^3$); 1.5~mm strips pitch; 50~Hz 150~GeV/c muons.}\label{fig: angle scan}
\end{figure}

The effect of the drift field on the position resolution was found negligible as shown in figure~\ref{fig: HV scan - drift scan}-b together with the results of the simulation explained in section~\ref{sec: Simulation}. This suggests that the transverse electron diffusion in \nech does not contribute significantly to the detector performance.  
Looking at the local detector response suggests that the THGEM-electrode geometry plays a significant role in determining the detector's position resolution. Figure~\ref{fig: profile - local residuals}-top shows the reconstructed muon beam distribution along the x-axis as measured by the tracker for the same data set as in figure~\ref{fig: spectrum - residuals histo}. The equivalent measurement by the RPWELL detector is depicted in figure~\ref{fig: profile - local residuals}-middle. The measured RPWELL-detector distribution clearly reproduces the THGEM-holes pattern shown in figure~\ref{fig: RPWELL detector}-b. For the same measurement, figure~\ref{fig: profile - local residuals}-bottom shows a 2-D representation of the measured residuals as a function of the track position. The local residuals pattern can be explained as follows: for particles impinging at normal incidence (90$^\circ$ to the detector plane) in the position of a hole center, most of the ionization electrons along a track are focused into the same hole (except for occasional delta electrons and other electrons getting large transverse diffusion), where an avalanche develops inducing a signal. The reconstructed event position in this case is close to the  center of the hole, resulting in a residual value close to 0. The residual grows linearly with the track distance from the hole-center, due to an increasing number of electrons directed to a neighboring hole. At the limit, for tracks hitting the central region in-between two holes, the primary charge is shared - on average - equally among them, and the residual is again close to 0. In section~\ref{sec: Simulation} we study this effect in more detail with a dedicated model simulation.
We also studied the case of an angular particle incidence, in which the ionization electrons within the drift gap (here 5~mm) are, in most cases, collected by more than a single hole. The position resolution was evaluated at different track incidence angles, rotating the chamber around the y-axis from 0$^\circ$ to 40$^\circ$. The rotation had to be corrected for when determining the track position on the x-y plane: a particle traversing the detector perpendicularly to its plane ($\uptheta$= 0$^\circ$), at a point x, will traverse it at a point x' when the detector is rotated by an angle $\uptheta$, with x'= x/cos$\uptheta$. This appears clearly in the local-residuals plot, as shown in figure~\ref{fig: angle correction}-a for $\uptheta$= 40$^\circ$. To correct for this effect and correctly measure the position resolution we fitted the local-residuals plot to a line that was then subtracted from each residual. The result of this correction is shown in figure~\ref{fig: angle correction}-b. The typical local-residuals pattern shown in figure~\ref{fig: profile - local residuals}-bottom for $\uptheta$= 0$^\circ$ vanishes at large angles, since the fraction of primary charges reaching each hole is not anymore correlated uniquely with the track position. This effect results also in a degradation of the position resolution, as reflected in the residuals histograms plotted in figure~\ref{fig: angle scan}-a for different incidence angles. In figure~\ref{fig: angle scan}-b we show the measured position resolution as a function of the incidence angle together with the results of the simulation explained in section~\ref{sec: Simulation}. At 40$^\circ$ the position resolution is 0.82~mm RMS - about a 4-fold degradation compared to the perpendicular incidence.


