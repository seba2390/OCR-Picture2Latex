%%%%%%%%%%%%%%%%%%%%%%%%%%%%%%%%%%%%%%%%%%%%%%%%%%%%%%%%%%%%%%%%%%%%%%%%%%%%%%
\begin{figure}[h]
\begin{subfigure}[t]{0.6\textwidth}\caption{}
\includegraphics[scale=0.15]{figures/DSC_0125.JPG}
\end{subfigure}
\begin{subfigure}[t]{0.4\textwidth}\caption{}
\includegraphics[scale=1]{figures/THGEM_square.png}
\end{subfigure}
\caption{A photograph of the RPWELL detector's THGEM-electrodes plane (cathode removed), made of 9 tiles (a) with a 0.96~mm pitch square holes-pattern (b). Note the larger 1.3~mm pitch between the central rows.}\label{fig: RPWELL detector}
\end{figure}


\section{Experimental setup and methodology}
\label{sec: setup and methodology}
%%%%%%%%%%%%%%%%%%%%%%%%%%%%%%%%%%%%%%%%%%%%%%%%%%%%%%%%%%%%%%%%%%%%%%%%%%%%%%

\subsection{The RPWELL detector}\label{sec: RPWELL}
We assembled an RPWELL detector with an area of 9$\times$9~cm$^2$; the multiplier is segmented into nine 3$\times$3~cm$^2$ THGEM-electrode tiles as shown in figure~\ref{fig: RPWELL detector}-a. Each 0.8~mm thick THGEM tile had a 2$\times$2~cm$^2$ squared hole pattern (figure~\ref{fig: RPWELL detector}-b): 0.5~mm hole-diameter, with 0.1~mm etched rims and a pitch of 0.96~mm (except for a central cross with a 1.3~mm pitch). This peculiar geometry permitted comparing present results with that of~\cite{bressler2016first}. The detector scheme and operation principle is shown in figure~\ref{fig: RPWELL scheme}.  Particle-induced ionization electrons deposited within the 5~mm long drift gap (defined by the cathode and the top THGEM electrode) are focused into the THGEM holes, where they initiate an avalanche multiplication. A resistive plate (RP) is directly coupled to the bare THGEM-bottom face; the avalanche induces a charge through the RP, onto the strips-patterned anode.  The RP in this work is made of 0.6~mm thick doped silicate glass with a bulk resistivity of $\sim$10$^{10}~\Omega\cdot$~cm~\cite{wang2010development}. The avalanche electrons travel across the RP and are evacuated to ground through the RP bottom surface, coated with a thin layer of graphite (similarly to the RWELL~\cite{arazi2014laboratory}); its surface resistivity is $\sim$3~M$\Omega/\Box$ and it is connected to ground through a side copper strip. The RP bottom side is coated with a 1~mm thick polymer epoxy layer, on top of the resistive film, to provide mechanical support to the glass and to protect the resistive layer when handling the detector.  In this configuration the RP is electrically decoupled from the readout anode (placed at 1.6~mm from the THGEM bottom, 1~mm from the resistive film), which had copper strips, divided into three groups of 1~mm, 1.5~mm and 2~mm pitch, separated from each other by 50~$\upmu$m. Each group of strips recorded signals induced from three THGEM tiles. Signals from the strips were processed by an APV25/SRS system~\cite{martoiu2013development,french2001design} and the data were analyzed with dedicated software, as described below.
The detector was operated in \nech at ambient conditions, at a flow of 30~cc/min. The electrodes were polarized with individual HV power-supply CAEN A1833P and A1821N boards, remotely controlled with a CAEN SY2527 unit. The voltage and current of each channel were monitored and stored. All inputs were connected through low-pass filters.  Unless stated otherwise, the voltage across the drift gap was kept at \dvdrift = 250~V, resulting in a drift field of 0.5~kV/cm. The voltage across the RPWELL, \dvrpwell,  varied between 900-975~V.

\begin{figure}[h]
\centering
\includegraphics[scale=1]{figures/Glass_RPWELL_configuration.pdf}
\caption{A schematic (not to scale) view of the RPWELL detector assembly and operation. From top to bottom the elements are: cathode, single-sided THGEM, resistive plate and readout strips.}\label{fig: RPWELL scheme}
\end{figure}


\subsection{Tracking, readout system and analysis framework}
\label{sec: tracking and readout}

The detector was installed at the CERN-SPS H2 test-beam area and investigated with a flux of $\sim$50~Hz/cm$^2$ 150~GeV muons. The CERN-RD51 Micromegas telescope provided the trigger and precise tracking of the muons. A detailed description of the telescope is given in~\cite{karakostas2010micromegas,karakostas2012telescope}. The RPWELL chamber was placed along the beam line in between the layers of the telescope. The data acquisition system, common to the RPWELL and the tracker, was based on the SRS/APV25 readout electronics~\cite{martoiu2013development,french2001design}). The data processing framework is described in detail in~\cite{bressler2016first}. For each readout channel a threshold relative to the pedestal noise was set, using a common Zero-order Suppression Factor (ZSF).  

The tracker and the RPWELL were aligned using the data measured in dedicated runs. We define the THGEM surface as an x-y plane, with the x-axis being perpendicular to the readout strips; this 1-D detector-readout did not provide any information in the y-axis.  For each strip-pitch region, only tracks hitting the detector in the central part of a tile, 13~mm along the x-axis, were considered; this assured that the induced signals are confined within the area covered by the strips.
Among all the muons measured in the RPWELL, we selected for analysis only those that yielded a reconstructed track perpendicular to the x-y plane (or at a fixed angle in dedicated measurements). 
In figure~\ref{fig: induced signal}, we show the typical induced-charge distribution on the 1~mm readout strips, measured by the SRS/APV25 electronics. The $\sim$2.4~mm RMS value of a double-Gaussian fit is independent of the total charge. The lateral spread of the induced-charge distribution depends, instead, strongly on the maximum distance between the readout plane and the fast moving charges produced in the avalanche, i.e. from the resistive-plate top face (as shown also in~\cite{cortesi2007investigations}). This distance was chosen large enough to distribute the induced signal among several strips, allowing defining the center-of-gravity of the induced-charge distribution with a resolution superior to the strip pitch. Considering charge clusters of neighboring strips with signal above threshold, we define a cluster position: x$\mathrm{_{cl}}$ = $\mathrm{\sum_i(x_i\cdot q_i)/\sum_iq_i}$, where x$\mathrm{_i}$  is the position of the center of strip \textit{i} and q$\mathrm{_i}$ is the charge induced on it. The cluster charge is defined as Q$\mathrm{_{cl}}$ = $\mathrm{\sum_iq_i}$. The strip multiplicity is defined as the number of strips in a cluster.  For each event, we calculate the residual along the x-axis, defined as RES = x$\mathrm{_{tr}}$-x$\mathrm{_{cl}}$ , where x$\mathrm{_{tr}}$ is the reconstructed track intersection with the detector. We define the detector position resolution as the RMS of the Gaussian fit of the residuals histogram. The contribution of the position resolution of the telescope ($\sim$0.05~mm RMS) has a negligible effect on that of the RPWELL, and is not taken into account in this work. A matching parameter, W~[mm], is defined as the maximal distance allowed between the reconstructed cluster position and the intersection point of the track with the RPWELL. W-values of the order of 3~mm, significantly larger than the RPWELL resolution, were used to avoid biasing the measured position resolution (see below).
The effective gain of the detector is estimated as follows: a MIP traversing through the 5~mm drift gap of \nech produces on average 20 electron-ion pairs~\cite{sauli2014gaseous}. About half reach the THGEM within 100~ns~\cite{peisert1984drift}, which is approximately the shaping time of the APV25 chip~\cite{french2001design}. Hence, the detector gain is estimated by the total measured charge divided by the charge of 10 primary electrons. We emphasize that the shaping time of the APV25 is much shorter than the rise time of the RPWELL ($\sim$2~$\upmu$s). Hence, the estimated effective gain is a factor of $\sim$3 lower than the total gas gain~\cite{rubin2013first}. 
The detector performance is measured for different ZSF values in the range between 0.7 and 5. Over this range, the strip multiplicity measured with the 1~mm strips varied between 11 to 7 respectively and the position resolution is affected by $\sim$2$\%$. We conclude that within this range of ZSF values, the strip multiplicity is sufficiently high. Over this work we chose ZSF= 1, which we found appropriate for excluding most of the noise hits. In these conditions, the strip multiplicity with the 1.5~mm strips is 6, also not affecting the position resolution. The 2~mm strips were not used in this study. Varying the matching parameter W from 2 to 5 mm did not affect the results; its value is fixed to 3~mm. 

\begin{figure}[h]
\centering
\includegraphics[scale=0.3]{figures/signal_ex.pdf}
\caption{The typical induced-charge distribution, at normal incidence, on the 1~mm readout strips measured by the SRS/APV25 electronics. Fit to a double-Gaussian yields a RMS-value of 2.38~mm.}\label{fig: induced signal}
\end{figure}
