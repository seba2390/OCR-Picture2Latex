%%%%%%%%%%%%%%%%%%%%%%%%%%%%%%%%%%%%%%%%%%%%%%%%%%%%%%%%%%%%%%%%%%%%%%%%%%%%%%
\section{Introduction}
\label{sec: Introduction}
%%%%%%%%%%%%%%%%%%%%%%%%%%%%%%%%%%%%%%%%%%%%%%%%%%%%%%%%%%%%%%%%%%%%%%%%%%%%%%
The Resistive Plate WELL (RPWELL)~\cite{rubin2013first} is a single-element gas-avalanche detector, comprising of a single-sided THick Gaseous Electron Multiplier (THGEM)~\cite{chechik2004thick,breskin2009concise} coupled to a segmented readout anode through a highly-resistive plate. Similarly to the Resistive-Plate Chamber (RPC)~\cite{santonico1981development}, RWELL~\cite{arazi2014laboratory}, Micromegas~\cite{alexopoulos2011spark}, and others, the resistive material was introduced to protect the detector from the occurrence of occasional discharges. With its mm-scale holes pattern, the RPWELL concept is suitable for applications requiring large area-coverage and moderate localization capabilities. Recent experimental results obtained with an RPWELL detector, in the context of digital hadron-calorimetry (DHCAL), are detailed in~\cite{bressler2016first,moleri2016resistive,moleri2016beam}; in these works, the RPWELL detector operated in a discharge-free mode at high gain over a broad dynamic range.  
The present work focuses on the study of the position resolution of a glass-RPWELL detector, with a THGEM coupled to a one-dimensional strips array anode, through a doped silicate-glass resistive plate of 10$^{10}$~$\Omega\cdot$cm bulk resistivity~\cite{wang2010development}.  The goal was to measure the position resolution and to understand the physics phenomena governing it - as to allow for designing detectors with optimized localization capabilities. Previous works investigating position resolution of THGEM-based detectors are reported in~\cite{cortesi2007investigations,cortesi2009thgem,silva2013x,lopes2013position}; they dealt with localization studies of soft x-rays and UV photons, using standard image-analysis techniques and demonstrated a position resolution ranging from 0.3~mm to 2.3~mm FWHM depending on the specific detector configuration, operation gas and type of radiation.  In this work, we investigated the detector with relativistic muons; each muon track was referenced to a high-resolution tracker, and the detector hits were analyzed on an event-by-event basis using a procedure similar to the one presented in~\cite{abusleme2016performance} for a Thin-Gap Chamber detector. This method allows studying the local differences in position resolution due to the detector geometry. 
Similar studies conducted on different detectors, like resistive micro-WELL~\cite{lener2016mu}, with its $\sim$7-fold smaller pitch in respect to the RPWELL, or Resistive-Plate Chamber with its continuous sensitive area~\cite{aielli2014rpc}, yielded RMS resolutions down to 52~$\upmu$m and 70~$\upmu$m (estimated) respectively. The present results show that in the RPWELL the localization capability is limited to significantly higher values mainly by the THGEM-holes pattern; possible improvements are suggested.
In section 2 we present the experimental setup and methodologies, followed by test-beam results in section 3. A detailed comparison with Monte Carlo simulations is described in section 4, followed by a discussion in section 5 and an appendix on the simulation method.


