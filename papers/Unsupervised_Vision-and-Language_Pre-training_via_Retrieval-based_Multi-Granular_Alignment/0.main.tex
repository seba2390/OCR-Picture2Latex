% CVPR 2022 Paper Template
% based on the CVPR template provided by Ming-Ming Cheng (https://github.com/MCG-NKU/CVPR_Template)
% modified and extended by Stefan Roth (stefan.roth@NOSPAMtu-darmstadt.de)

\documentclass[10pt,twocolumn,letterpaper]{article}

%%%%%%%%% PAPER TYPE  - PLEASE UPDATE FOR FINAL VERSION
%\usepackage[review]{cvpr}      % To produce the REVIEW version
% \usepackage{cvpr}              % To produce the CAMERA-READY version
\usepackage[pagenumbers]{cvpr} % To force page numbers, e.g. for an arXiv version

% Include other packages here, before hyperref.
\usepackage{graphicx}
\usepackage{amsmath}
\usepackage{amssymb}
\usepackage{booktabs}
\usepackage{multirow}

\usepackage{booktabs, multirow} % for borders and merged ranges
\usepackage{soul}% for underlines
\usepackage[table]{xcolor} % for cell colors
\usepackage{changepage,threeparttable} % for wide tables
% It is strongly recommended to use hyperref, especially for the review version.
% hyperref with option pagebackref eases the reviewers' job.
% Please disable hyperref *only* if you encounter grave issues, e.g. with the
% file validation for the camera-ready version.
%
% If you comment hyperref and then uncomment it, you should delete
% ReviewTempalte.aux before re-running LaTeX.
% (Or just hit 'q' on the first LaTeX run, let it finish, and you
%  should be clear).
\usepackage[pagebackref=true,breaklinks=true,colorlinks,bookmarks=false]{hyperref}

% Authors
\usepackage{xcolor}
% define colors from here: http://latexcolor.com/
\definecolor{bittersweet}{rgb}{1.0, 0.44, 0.37}
\definecolor{mygreen}{rgb}{0.29, 0.7, 0.48}
% colors: red, orange, cyan, 
\newcommand{\mingyang}[1]{\textcolor{red}{\small{\bf [ #1 -- Mingyang ]}}}
\newcommand{\licheng}[1]{\textcolor{mygreen}{\small{\bf [ #1 -- Licheng ]}}}
\newcommand{\ning}[1]{\textcolor{orange}{\small{\bf [ #1 -- Ning ]}}}
\newcommand{\aman}[1]{\textcolor{cyan}{\small{\bf [ #1 -- Aman ]}}}
\newcommand{\mj}[1]{\textcolor{bittersweet}{\small{\bf [ #1 -- MJ ]}}}


% Support for easy cross-referencing
\usepackage[capitalize]{cleveref}
\crefname{section}{Sec.}{Secs.}
\Crefname{section}{Section}{Sections}
\Crefname{table}{Table}{Tables}
\crefname{table}{Tab.}{Tabs.}

\newcommand{\myparagraph}[1]{\vspace{0.25em}\noindent\textbf{#1}}
\newcommand{\tablestyle}[2]{\setlength{\tabcolsep}{#1}\renewcommand{\arraystretch}{#2}\centering\footnotesize}
%%%%%%%%% PAPER ID  - PLEASE UPDATE
\def\cvprPaperID{5426} % *** Enter the CVPR Paper ID here
\def\confName{CVPR}
\def\confYear{2022}
\def\TaskName{UVLP}
\def\TaskFullName{unsupervised V+L pre-training without parallel data}
\def\ModelName{$\mu$-VLA}
\def\ModelFullName{Unsupervised Vision-and-Language Pre-training via Retrieval-based Multi-Granular Alignment}
\def\uvisualbert{U-VisualBERT}
\newcommand{\head}[1]{\noindent\textbf{#1}}


\begin{document}

%%%%%%%%% TITLE - PLEASE UPDATE
\title{Unsupervised Vision-and-Language Pre-training via Retrieval-based Multi-Granular Alignment} 

\author{
Mingyang Zhou$^1$\thanks{The two authors contribute equally.} \quad Licheng Yu$^3$\footnotemark[1]  \quad  Amanpreet Singh$^3$  \quad  Mengjiao Wang$^3$  \quad  Zhou Yu$^2$  \quad  Ning Zhang$^3$
\\ 
$^1$Uiversity of California, Davis   \quad  $^2$ Columbia University  \\  $^3$Meta AI\\
\tt{\small minzhou@ucdavis.edu}, \tt{\small zy2461@columbia.edu}, \tt{\small\{lichengyu, asg, mengjiaow, ningzhang\}@fb.com} \\
}

\maketitle

%%%%%%%%% ABSTRACT
\begin{abstract}
Vision-and-Language (V+L) pre-training models have achieved tremendous success in recent years on various multi-modal benchmarks.
However, the majority of existing models require pre-training on a large set of parallel image-text data, which is costly to collect, compared to image-only or text-only data.
In this paper, we explore unsupervised Vision-and-Language pre-training (\TaskName) to learn the cross-modal representation from non-parallel image and text datasets. 
We found two key factors that lead to good unsupervised V+L pre-training without parallel data:  $(i)$ \textit{joint image-and-text input} $(ii)$ \textit{overall image-text alignment (even for non-parallel data)}. 
Accordingly, we propose a novel unsupervised V+L pre-training curriculum for non-parallel texts and images. 
We first construct a weakly aligned image-text corpus via a retrieval-based approach, then apply a set of multi-granular alignment pre-training tasks, including region-to-tag, region-to-phrase, and image-to-sentence alignment, to bridge the gap between the two modalities.
A comprehensive ablation study shows each granularity is helpful to learn a stronger pre-trained model.
We adapt our pre-trained model to a set of V+L downstream tasks, including VQA, NLVR2, Visual Entailment, and RefCOCO+. 
Our model achieves the state-of-art performance in all these tasks under the unsupervised setting.
\end{abstract}

%%%%%%%%% BODY TEXT
\section{Introduction}

Defect prediction for software projects has been well studied 
over the past decade and lots of researchers focus on proposing new techniques to improve predictors' performance
or make predictors more feasible for practical situation~\cite{lessmann2008benchmarking,hassan2009predicting,fu2016tuning,graves2000predicting,menzies2007data,hall2012systematic,nagappan2005use,jiang2013personalized,kim2007predicting,kim2008classifying,jing2014dictionary,lee2011micro,moser2008comparative,nam2013transfer,wang2016automatically,yang2016effort,kamei2013large}. 
Traditional defect prediction studies build learners at the granularity of  modules or files level~\cite{lessmann2008benchmarking,menzies2007data,nam2013transfer,wang2016automatically,fu2016tuning,hall2012systematic}, which can be
used to identify whether the file or module has defects after applying prediction models. 
Even though learners
may have very high accuracy, it still requires a lot of efforts from quality assurance team 
to inspect each predicted file~\cite{kamei2013large} or
module and it also adds much difficulty to fix the defects since 
the owner of the related files might not be available ~\cite{anvik2006should} or the owner forgets 
details of the implementation~\cite{weiss2007long}. 

To address these problems, researchers proposed to build defect prediction learners on change level~\cite{kim2008classifying,mockus2000predicting,sliwerski2005changes,shihab2012industrial} and predict the defect-inducing changes at check-in time~\cite{kamei2013large,fukushima2014empirical,yang2016effort}. This type of just-in-time(JIT) defect prediction
has the following benefits over traditional counterpart~\cite{kim2008classifying,kamei2013large}:
\bi
\item Predict defects at a fine granularity: Compared to the whole file or module, a committed change includes small amount of codes, which will dramatically reduce the efforts for inspection if it is predicted as defective.
\item Predict defects in a timely manner: Traditional defect prediction usually happens after the software or the package is completed. However, JIT defect prediction could happen as soon as the change is committed, which
will make the defect fixed easily as the developer would still remember the details about the change.
\ei

Mockus et al.~\cite{mockus2000predicting} conduct the first study to predict defects on change metrics from a telecommunication software project, whereas Kim et al.~\cite{kim2008classifying} further evaluate the effectiveness of change metrics on 12 open source projects.
Shihab et al.~\cite{shihab2012industrial} designed a logistic regression model to predict the risk change on
industrial software. They find that  features, like lines of code added
by the change and the developer experience, are strongly correlated to risk change. To take the effort into account, Kamei et al.~\cite{kamei2013large}
used a modified linear regression to build an effort-ware JIT defect predictor.
They reported that with $20\%$ of the total effort required to inspect all the changes,
their method can identify $35\%$ of all the defect-inducing changes. Yang et al.~\cite{yang2016effort} argued
that data collection for training models is time consuming and 
building supervised learners is expensive. Therefore, they
proposed to build unsupervised learners directly on testing data~\cite{yang2016effort}.
In their study, they identify 12 different change metrics and build
12 unsupervised learners accordingly. Based on their evaluation results, they claimed that many unsupervised learners perform better than Kamei et al.'s supervised learner~\cite{yang2016effort}. However, since the unsupervised learners
are built on the testing data, to deploy unsupervised learners on actual software
project, we still have the following questions:
\bi 
\item Do all the $12$ unsupervised learners perform equally well so that we can randomly pick one and apply to future projects?
\item If not, without any prior knowledge on the future project,  which unsupervised learner we have to use?
\ei

According to our best knowledge, Yang et al. did not shed lights on these issues in ~\cite{yang2016effort}. Therefore, in this paper, we empirically study how to make such
simple unsupervised learners work on actual software projects. Specifically, we set the following research questions:

\bi 
\item \textbf{RQ1:Do all simple unsupervised learners perform better than the best supervised learner?}\\
Experimental results show that not all, actually only two unsupervised learners,  outperform than the best supervised learner for only two evaluation measures. This indicates that without any prior knowledge on the performance of unsupervised learners, we can't find randomly pick an supervised learner to apply. Based on this observation, we proposed a new supervised learner, {\it OneWay}.

\item \textbf{RQ2:Does {\it OneWay} perform  better than unsupervised learners?}\\
Experimental results show that {\it OneWay} has pretty stable and good performance on all data sets for all evaluation measures. When deploying to the actual software projects, it performs better than unsupervised learners on average for effort-aware JIT defect prediction. 

\item \textbf{RQ3:Does {\it OneWay} perform  better than supervised learners?} \\
Experimental results {\it OneWay} performs significantly better than all 4 supervised learners in terms of {\it Recall} and $P_{opt}$, but not for {\it Precision}.
\ei

The rest of this paper is organized as follows. Section 2 describes the background and related work on defect prediction.
Section 3 explains the effort-ware JIT defect prediction methods investigated in this study, while Section 4 describes the
experimental settings of our study, including research questions that motivate our study, data sets and experimental design.
Section 5 presents the results. Section 6 discusses the threats to the validity of our study. Section 7 concludes the paper.



% \vspace{-0.15in}

\section{Related Work}\label{sec:rel_work}

Alternating Direction Method of Multipliers (ADMM) has shown promise in solving optimization problems, especially in large-scale and data-distributed machine learning applications. The power of ADMM comes from its decomposition of the augmented Lagrangian into simpler loosely-coupled sub-problems which enables it to solve each sub-problem in an efficient and potentially parallel manner.
%The standard ADMM generally needs to compute the gradient of all the samples in each update step which makes it impractical for many large scale problems. 
%To alleviate this problem, \textit{stochastic} and \textit{online} versions of the ADMM have been proposed \cite{ouyang2013stochastic,wang2013online,zhong2014fast} and their convergence proofs presented. Further, \cite{kadkhodaie2015accelerated} proposed an accelerated ADMM method which improves the convergence rate from $\mathcal{O}(\frac{1}{k})$ to $\mathcal{O}(\frac{1}{k^2})$.
%Moreover, 
ADMM extensions for non-convex problems have been recently proposed which are more suitable for large data sets and more complicated problems \cite{wang2019global,huang2018mini}.

%Since DNNs are considered to be highly non-convex functions, ADMM has been proposed for solving different optimization problems related to deep learning. \cite{kiaee2016alternating,ye2018progressive} used ADMM for model compression and parameter pruning in deep networks. \cite{murdock2018deep} developed an iterative algorithm using ADMM to enforce constraints on the latent space after a DNN has been trained. \cite{sun2016deep} proposed a deep network for MRI image reconstruction using an ADMM scheme of optimization. Those methods usually do not directly train DNNs.


A recent line of research has focused on training DNNs using optimization techniques that decompose the training into smaller subproblems, including Block Coordinate Descent (BCD) and ADMM. On the BCD algorithms, \cite{carreira2014distributed} was the earliest to propose training a DNN in a distributed setting by formulating it as a constrained optimization problem. Further, \cite{zeng2018global,zhang2017convergent,askari2018lifted,gu2018fenchel} lifted the non-convex activations (e.g. ReLU) and formulating the DNN training as a multi-convex problem and solved it using BCD and \cite{choromanska2018beyond} proposed an online method for training DNNs. 


On the other hand, \cite{taylor2016training} proposed a batch gradient-free algorithm for training neural networks using a variant of ADMM. However, due to the closed-form update of all the parameters, the proposed method has limitations (\textit{e.g.} only capable of using simple losses such as Hinge loss and MSE), and cannot be further extended into more complex problems and larger datasets. However, the scope of
\cite{zhang2016efficient} is limited to a specific application and no convergence proof is presented.


\cite{gotmare2018decoupling} splits DNNs into blocks and trained them separately by introducing gluing variables. This is very close to ADMM, but it did not use the dual variables common in ADMM and did not present a convergence proof for their method.
Recently, \cite{wang2019admm,zeng2019convergence} have provided convergence analysis of ADMM (to a stationary point) in deep learning by linearly approximating the non-linear constraints in the DNN training problem. However, their work did not address stochastic gradients as in our work.


Non-negative Matrix Factorization (NMF) imposes  non-negativity constraints over the factors, hence can lead to more interpretable decompositions than methods such as Principle Component Analysis (PCA)~\cite{lee1999learning,liu2011constrained}.  
\cite{collins2018deep} applied NMF over convolutional activations which has shown interpretable and coherent behavior over image parts. However, in their work, NMF was applied post-hoc over pre-trained CNN activations. There is no guarantee that the disentanglement is faithful to the underlying mechanism of the DNN. 
To the best of our knowledge, NMF layers jointly trained with a deep neural network have not been studied in the past.

%\section{\ModelFullName}
\section{Method}
In this section, we introduce the two core components of our \ModelName's architecture for~\TaskFullName: 
(1) construct a weakly aligned image-text corpus from independent vision and language data sources; 
(2) our novel pre-training curriculum to enable the model to capture the cross-modal alignment on three granularity including region-to-tag level alignment (RT),  region-to-noun phrase level alignment (RN), and image-to-sentence level alignment (IS). 

\subsection{Model Overview}
We use the well-known single-stream model architecture for our experiments as~\cite{li2019visualbert,li2020unicoder,su2019vl,chen2020uniter,li2020unimo}.
As shown in Fig.~\ref{fig:model}, our main backbone is a single transformer, where we feed the concatenation of visual embeddings of an image and the tokens of a caption as its input.
Given an image $\mathbf i$, we first use an off-the-shelf Faster R-CNN (VinVL~\cite{Zhang_2021_CVPR}) to detect the objects $\mathbf v = \{ v_1, ..., v_{k^v} \}$.
The visual embedding of each region is then encoded as the sum of its regional feature, its location embedding\footnote{The 5-dimensional vector [$\frac{x_1}{W}$,$\frac{y_1}{H}$,$\frac{x_2}{W}$,$\frac{y_2}{H}$,$\frac{(y_2-y_1)(x_2-x_1)}{W.H}$] is projected to the visual embedding space. $(x_1,y_1), (x_2,y_2)$ are the coordinates of the top left and bottom right point of the detected region, and $W,H$ are the image width and height.}, and the modality embedding.
For a given caption $\mathbf t$, we denote its tokenized sequence as $\mathbf t = \{ t_1, ..., t_{k^t} \} $.
After multiple layers of self-attention, the two modalities are fused together and the output hidden vectors can be used for various pre-training tasks.

\subsection{Weakly-aligned Image-Text Corpus}
\label{section:data_aug}
As in the analysis of Sec~\ref{section:intro}, we observe a strong correlation between the degree of image-text alignment in the training data and the performance of the pre-trained model.
% Given the unpaired image collection $I =\{I_1, I_2, \dots, I_{n_I}\}$ and text corpus $T = \{T_1, T_2, \dots, T_{n_T}\}$, 
Inspired by this finding, we believe it important to initialize some weak semantic alignment between the two modalities as the input source.
Specifically, we retrieve $\mathrm{k}$ sentences that are semantically closed to a given $I_i$. 
Previous work~\cite{tan2020vokenization} shows the visually grounded caption covers a good ratio of words that are naturally related to specific visual contents, \eg concrete nouns. 
Thus, we utilize the semantic association between the objects that appear in the image and a candidate sentence as the indicator to measure the alignment degree.

Specifically, we take the object tags $\mathbf o = \{ o_1, ..., o_{k^o} \}$ from the above detected $\mathbf v$ and feed the sequence into an off-the-shelf sentence BERT embedding model~\cite{reimers-gurevych-2019-sentence} to obtain the query embedding $\mathbf e_{\mathbf o}$.
Similarly, we feed each candidate sentence into the same model getting the candidate embedding $\mathbf e_{\mathbf t}$.
We retrieve the top $\mathrm{K}$ candidates with the highest cosine similarity score to form an initial weakly-aligned image-text pairs for a given image $\mathbf i$.
We denote the retrieved captions as $\{ \mathbf t^r (\mathbf i) \}_{r=1}^K$ and the overall weakly aligned corpus as $\mathbf R$.

% Specifically, we filter objects that are detected with confidence score lower than 0.2 and with small bounding box region size that is less than 0.05 of the image size  to focus on critical objects shown in the image. \mingyang{Should appear in the training set up.}
% Specifically, we feed the object-tag sentence and a candidate sentence into the pre-trained off-the-shelf Sentence Bert Embedding model \cite{reimers-gurevych-2019-sentence} to obtain their corresponding embeddings $\mathbf e_{\mathbf{o}}$ and $\mathbf e_{\mathbf{t}}$. 
% We then select the top $\mathrm{k}$ sentence candidates $T_{\text{ret}_i} = {T_{\text{ret}_i}^1, \dots,T_{\text{ret}_i}^k}$ with the highest cosine similarity score with the object list embedding vector $e_{O_i}$
% Eventually, we would have k retrieved sentences $T_{ret} = {T_{ret}^{1}, \dots,T_{ret}^{k}}$ for each image 
% to form the weakly-aligned image-text pairs with the query image $I_i$. 

\subsection{Pre-training Tasks}
% We consider that the success of vision and language pre-training relies on the capability to understand the cross-modal alignment on various granularity including: region-to-object tag level, region-to-noun phrase level, and image-to-sentence level. 
In this subsection, we introduce a set of pre-training objectives that we designed to facilitate the model to capture the different levels of vision and language alignment.
Fig.~\ref{fig:model} shows the overview of our model and its pre-training tasks.

% Unlike \uvisualbert \cite{li2020unsupervised} that focus on just capturing the region to object-tag alignment, we consider the successful unsupervised V+L pre-training should understand the cross-modal alignment on various granularity. Thus, we propose pre-trianing cu

\subsubsection{Region-Tag Alignment Learning}
We first propose to align the object tags onto the image regions.
As shown in Fig.~\ref{fig:model}(a), We concatenate the object tags detected from each image with its source image to form an input pair $[\mathbf o, \mathbf v]$ fed into the model. 
% The detected tags are processed similarly as the normal caption as a sequence of tag tokens $o= [o_{1:l}]$. 
%Following \cite{li2020unsupervised}, the position embedding for each tag token is the spatial box coordinate embedding of its corresponding region. 
% The position embeddings allow the model to distinguish the tags from different regions. 
% Given the pair of object list and image [$o$, $v$] from the training dataset $D$, 
% We randomly mask some tag tokens $o_k$, and some regions $v_j$, and train our model to predict the masked tag tokens and the properties of the masked regions. 
We denote the mask indices as $\mathbf{m}\in \mathbb{N}^M$\footnote{$\mathbb{N}$ is the natural numbers, $M$ is the vocabulary size, and $\mathbf{m}$ is the set of masked indices.}. 
We randomly mask out the object tags and regions, and apply masked language modeling (MLM) and masked region modeling (MRM) for the pre-training.

Specifically, MLM on the object tags is formulated as
\begin{equation*}
    \mathcal{L}_{\text{MLM}}^{\text{R-T}} = - \mathbb{E}_{(\mathbf o, \mathbf v)\sim \mathbf I} \log{P(\mathbf o_{\mathbf m} |\mathbf o_{\backslash \mathbf m}, \mathbf v)},
\end{equation*}
where the goal is to predict the masked object tags based on the observation of their surrounding tags $\mathbf o_{\backslash \mathbf m}$ and image regions $\mathbf v$.
On the vision side, MRM includes both masked region classification loss (MRC) and masked region feature regression loss (MRFR):
\begin{equation*}
    \begin{split}
    \mathcal{L}_{\text{MRM}}^{\text{R-T}} = \mathbb{E}_{(\mathbf o, \mathbf v)\sim \mathbf I}  [f_{\text{MRC}}(\mathbf v_{\mathbf m} | \mathbf v_{\backslash \mathbf m}, \mathbf o) + f_{\text{MRFR}}(\mathbf v_{\mathbf m} | \mathbf v_{\backslash \mathbf m}, \mathbf o) ].
    \end{split}
\end{equation*}
% To calculate the $\mathcal{L}_{\text{MRC}}$ and $\mathcal{L}_{\text{MRFR}}$, we first obtain the transformer output $h_j$ of the masked region $i=v_j$ at the final layer. For $\text{MRC}$, a fully connected (FC) layer $\phi_{\text{MRC}}$ is applied to predict the object category as a normalized distribution over the total number of $K$ classes of the object categories. 
% Thus, $\mathcal{f}_{\text{MRC}}=CE(\phi_{\text{MRC}(h_j)}, c_j)$ is the standard cross-entropy loss. 
% Additionally, for $\text{MRFR}$ we have another FC layer $\phi_{\text{MRFR}}$ to project $h_j$ into the same dimension space of the ROI feature of the masked region $f_j$. Then we apply L2 regression to compute the loss: $f_{\text{MRFR}}=||\phi_{\text{MRFR}(h_j)}- f_j||_2^2$.
Between the two, MRC learns to predict the object semantic class for each masked region  $c(\mathbf v_{\mathbf m})$.
We feed the last hidden output of the masked region $\mathbf v_{\mathbf m}$ into a FC layer and softmax function to predict the classification probabilities $g_{\theta} ( \mathbf v_{\mathbf m} )$.
The objective is to minimize the cross-entropy of
$ f_{\text{MRC}}(\mathbf v_{\mathbf m} | \mathbf v_{\backslash \mathbf m}, \mathbf o) = \mbox{CE}( c(\mathbf v_{\mathbf m}) , g_{\theta} ( \mathbf v_{\mathbf m} ) ) $.
MRFR learns to regress the transformer output of each masked region $\mathbf v_{\mathbf m}$ to its visual features. 
We apply a FC layer to convert its hidden output to a vector $h_\theta (\mathbf v_{\mathbf m})$ of the same dimension as the input regional feature $r(\mathbf v_{\mathbf m})$.
We apply L2 regression:
$f_{\text{MRFR}}(\mathbf v_{\mathbf m} | \mathbf v_{\backslash \mathbf m}, \mathbf o)  = || h_\theta (\mathbf v_{\mathbf m}) - r(\mathbf v_{\mathbf m}) ||^2_2$.

For region-tag alignment learning, we have our pretraining objective function as 
\begin{equation}\nonumber
\mathcal{L}^{\text{R-T}} =  \mathcal{L}_{\text{MLM}}^{\text{R-T}} + \mathcal{L}_{\text{MRM}}^{\text{R-T}} 
\end{equation}

\subsubsection{Region-Noun Phrase Alignment Learning}

Due to the small vocabulary size of object tags, the region-tag alignment learning can only capture a limited amount of localized concepts.
To increase the diversity of concepts, we propose to align the noun phrases from the retrieved sentences to the corresponding regions as well.
As in Fig.~\ref{fig:model}(b), given an image $\mathbf i$ and its retrieved weakly aligned caption $\mathbf t^r (\mathbf i)$, we first detect the noun phrases from the caption using spacy~\cite{spacy2}.
Note the detected noun phrases sometimes contain the attribute words, which further benefits this pre-training task.
We link the noun phrase to its closest visual region by computing the word2vec similarity between the phrase and object tag (associated to each region).
The pre-training still consists of MLM and MRM but are performed with different masking strategy and supervision signal.

Specifically, for both MRM and MLM, we only mask the linked noun phrases from the caption or the linked object regions.
We make the masking probability proportional to the linked similarity score.
Each time we only mask out one modality (phrase or region) to encourage it to be recovered by its linked content.
The region-to-phrase MLM is then formulated as
$\mathcal{L}_{\text{MLM}}^{\text{R-P}} = - \mathbb{E}_{(\mathbf v, \mathbf t^r)\sim \mathbf R} \log{P(\mathbf t^r_{\mathbf m} |\mathbf t^r_{\backslash \mathbf m}, \mathbf v)}$.

On the vision side, we propose using the phrase-guided masked region-to-token classification (p-MRTC) on the masked regions:
\begin{equation*}
    \begin{split}
    \mathcal{L}_{\text{MRM}}^{\text{R-P}} = \mathbb{E}_{(\mathbf v, \mathbf t^r)\sim \mathbf R}  f_{\text{p-MRTC}}(\mathbf v_{\mathbf m} | \mathbf v_{\backslash \mathbf m}, \mathbf t^r),
    \end{split}
\end{equation*}
where we directly classify the masked region to its linked noun phrase (sub-word tokens) in BERT vocabulary.
Enlarging the vocabulary has shown to be beneficial to MRM~\cite{Zhou_2021_CVPR}.
Our proposed p-MRTC leverages the additional noun-phrase to encourage more diverse local region to language alignment.

For region-noun phrase alignment learning, we have our pretraining objective function as 
\begin{equation}\nonumber
\mathcal{L}^{\text{R-P}} =  \mathcal{L}_{\text{MLM}}^{\text{R-P}} + \mathcal{L}_{\text{MRM}}^{\text{R-P}} 
\end{equation}


\subsubsection{Image-Sentence Alignment Learning}
\label{section:itm}
We apply image-text matching (ITM) objective as the previous supervised V+L pre-training research \cite{chen2020uniter,li2020unicoder} to learn the cross-modal sentence-level alignment. 
As in Fig.~\ref{fig:model}(c), given an input pair [$\mathbf v$, $\mathbf t^r$], the final hidden vector of the special token $\text{[CLS]}$ is fed through a FC layer to output a single score $\mathbf s_{\theta}(\mathbf v, \mathbf t^r)$, which predicts if the given image-text input is a semantically matched pair or not. 
We use the label $y\in \{0,1\}$ to indicate if a retrieved pair is a match.
The training objective for the ITM task is to minimize the binary cross-entropy loss:
$
\mathcal{L}_{\text{ITM}} =\mbox{CE}( y ,s_{\theta}(\mathbf{v}, \mathbf{t}^r) ) 
$.
% \begin{equation*}
%     \mathcal{L}_{\text{ITM}} = -\mathbb{E}_{(\mathbf{v},\mathbf{t^r})\sim \mathbf{R}} [y \log s_{\theta}(\mathbf{v}, \mathbf{t}^r) + (1-y) \log (1-s_{\theta}(\mathbf{v}, \mathbf{t}^r))]
% \end{equation*}
On the language side, we also apply standard MLM to help the model learn to align other language tokens besides noun phrases and object tags to the visual context.
The objective function is then formulated as $\mathcal{L}_{\text{MLM}}^{\text{I-S}} = - \mathbb{E}_{(\mathbf v, \mathbf t^r)\sim \mathbf R} \log{P(\mathbf t^r_{\mathbf m} |\mathbf t^r_{\backslash \mathbf m}, \mathbf v)}$.
The image-sentence level alignment pretraining objective function is
\begin{equation}\nonumber
\mathcal{L}^{\text{I-S}} =  \mathcal{L}_{\text{MLM}}^{\text{I-S}} + \mathcal{L}_{\text{ITM}}
\end{equation}


\subsection{Multi-Granular Pre-training Curriculum}
We propose a multi-granular curriculum to iteratively pre-train the model on the region-to-tag, region-to-noun phrase, and image-to-sentence level. 
According to our findings in Sec.~\ref{section:intro}, learning from image-text pairs with higher degree of cross-modal alignment is beneficial to the performance of unsupervised V+L pre-trained model. 
Therefore, we propose using an estimated image-text alignment score to guide our multi-granular pre-training. 
Specifically, we have an ITM header defined in Sec.~\ref{section:itm} to learn the image-text alignment. 
We also use it to predict matching score as a weight to modulate the input data for each of our retrieval-based pre-training tasks. 
This allows us to provide more importance to relatively more aligned image-text pairs over time to help our model to learn better cross-modal alignment on multiple granularities. 

To train the alignment model's ITM classifier, we use our retrieved corpus $\mathbf{R}$ as positive samples and randomly shuffled pairs as negative samples in the first $m$ epochs. This warms up the models to make reasonable estimations on the alignment of image-text input pairs.
After $m$ epochs, we start to incorporate the alignment prediction score $w_{\text{ITM}}$ in our training objective. 
To summarize, our multi-granular pre-training loss is 
\begin{equation}\nonumber
\mathcal{L}=  
\begin{cases}
      \mathcal{L}^{\text{R-T}} + \mathcal{L}^{\text{R-P}} +\mathcal{L}^{\text{I-S}}   &\text{if epoch} < m\\ 
      \mathcal{L}^{\text{R-T}} + w_{\text{ITM}} (\mathcal{L}^{\text{R-P}}  +  \mathcal{L}^{\text{I-S}}) &\text{if epoch} \geq m, \\
    \end{cases}
\end{equation}
where $\mathcal{L}^{\text{R-T}}$, $\mathcal{L}^{\text{R-P}}$, and $\mathcal{L}^{\text{I-S}}$ are the loss functions for region-tag alignment pre-training, region-noun phrase alignment pre-training, and image-sentense alignment pre-training. We set m as 1 in our final implementation.

% \vspace{-0.1in}
\section{Experiments} \label{sec:exp}


%Our implementation uses the PyTorch framework \cite{paszke2017automatic}. 
All the experiments are run on a machine with a single NVIDIA GeForce RTX 2080 Ti GPU. The results presented for each of the following experiments are selected from their best performance after grid search over the hyper-parameters, both for our method and the baselines. %Note, in the following Figures (\ref{fig:mnist_acc}, \ref{fig:mnist_deep}, \ref{fig:cifar}), 
Each algorithm is ran five times with different initialization and the average test set accuracy is reported. The shaded area corresponds to $\pm1$ standard deviation. We will make our code available online. %for reproducibility. %and a more precise access to the set of parameters used in the experiments.

% \vspace{-0.05in}
\subsection{Supervised Deep Network Training}\label{exp:conv}
In this section, we present the experiment results from training conventional neural networks in a supervised setting on the MNIST, Fashion-MNIST, and CIFAR-10 datasets. For experiments results on Fashion-MNIST and CIFAR-10, see supplementary materials \ref{sec:sup_train}. %The results from the proposed methods in section \ref{sec:method} are compared with baselines including training a conventional neural network in an end-to-end setting using SGD. 
% For the CIFAR-10 dataset, our formulation enables training complicated networks such as ResNets \cite{he2016deep} using ADMM which has never been done before.


\subsubsection{MNIST}\label{exp:mnist}
For the first supervised learning experiment, the MNIST dataset of handwritten digits \cite{mnist}, is used for the evaluation of ADMM/BCD methods for training DNNs. We use the standard train/test split. %Throughout the experiments, 60,000 samples are used during training and %
The performance on the testing set of 10,000 samples is reported in Figure \ref{fig:mnist_acc}. The architecture of the \emph{shallow} network used for the experiments incorporates three fully-connected layers with 128-neuron hidden layers $(784-128-128-10)$ and \emph{ReLU} nonlinearity. In order to make a fair comparison with ~\cite{taylor2016training} which can only work with Mean Squared Error (MSE), we utilize MSE as the training objective ($\mathcal{J}$) while the more common Cross-Entropy (CE) is applicable in our block-ADMM formulation and utilized in the experiments in the supplementary materials. 


In training standard ADMM and \cite{taylor2016training} as baselines, all the parameters are initialized by sampling from the uniform distribution $x \sim {U}(0, 10^{-4})$.% and are down scaled by a factor of $1e^{-4}$. 
We set $\beta_l = \gamma_l = 10$ for all of the layers. 
%Note that Algorithm \ref{alg:admm} can be converted to the formulation in \cite{taylor2016training} by setting dual variables $\forall \ell \neq L \; \mU_\ell = \bm 0;$ and discarding their updates. 
%To regularize the weights, $\mW_l$ during the training, \emph{$L_2$} norm 
Weight decay is used with $\lambda_l = 5 \times 10^{-5}$. %We observed that the regularization term significantly improves the optimization behavior in standard ADMM and without it, the training is not stable. 
For baselines with backpropagation in Fig. \ref{fig:mnist_acc}, a  learning rate of $5 \times 10^{-3}$ is used. 
 
 
Further, for the training of the batch and online Stochastic Block-ADMM algorithms presented in Algorithm \ref{alg:blockadmm} and \ref{alg:online_admm}, the aforementioned three-layer architecture is split into 3 one-layer blocks. $\beta_t$ is set to 1 for all layers, the weights are initialized using the normal distribution, dual variables $\mU_t$ are initialized using a uniform distribution, and auxiliary variables $\mZ_t$ are initialized in a forward pass. During training, the block parameters ($\Theta_t$) are updated stochastically, and both of sub-problem updates for the $\text{block}_{\Theta_t}$ and $\mZ_t$ are performed using \textit{Adam}. In our experiments in the batch mode, we performed the primal updates for $3$ steps during each iteration. For the online version, we set the batch size to 64 and auxiliary variables are re-initialized at each iteration (see Algorithm \ref{alg:online_admm}). 

Figure \ref{fig:mnist_acc} shows that Stochastic Block-ADMM outperforms the baselines by reaching $97.61 \%$ average test accuracy. Note the accuracy for all methods is lower than normal because of the MSE loss function that is used --- which is not the best choice for classification yet chosen for fair comparison with previous ADMM methods. The online version performs slightly worse with a $93.88 \%$ test accuracy. However, this comes with enormous advantage in terms of memory utilization, e.g. given the configuration for training on MNIST, the online version uses \~ 10$\times$ less memory to store training variables compared to the batch version.


%---------------------------- fig mnist acc ------------------------------
\begin{figure}[ht]
%  \vskip -0.05in
\begin{center}
\centerline{
\includegraphics[width=\columnwidth]{imgs/mnist_acc_new.pdf}
}
%  \vskip -0.05in
\caption{Test set accuracy on MNIST using network with 3 fully-connected layers: $784-128-128-10$. 
Final test accuracy: ``Stochastic Block-ADMM'': {\bf 97.61\%}, 
``Online Stochastic Block-ADMM'': 93.88\%, 
``Standard ADMM'': 95.02\%, 
%``Taylor et al.,  
\protect \cite{taylor2016training}
: 87.52\%, 
%``Wang et al. 
\protect \cite{wang2019admm}: 83.89\% ,
%``Zeng et al. 
\protect \cite{zeng2018global}: 83.28\% , 
``SGD'': 95.29\% 
(Best viewed in color)}
\label{fig:mnist_acc}
\end{center}
%  \vskip -0.4in
\end{figure}


\subsubsection{Vanishing Gradient}\label{exp:vanish}

Since no gradient is backpropagated through the entire network in our proposed algorithm, stochastic block-ADMM is robust against vanishing gradients. We run the previous experiments on an unconventional architecture with 10 fully-connected layers --- this is to make the vanishing gradient problem obvious. Note that normally this will not be adopted because of the severe overfitting and gradient vanishing problems, but here we utilized this setting to test our resistance to these problems. Figure \ref{fig:mnist_deep} illustrates the experiment results. Stochastic Block-ADMM reaches final test accuracy of $94.43\%$ while SGD and ADAM only reach to $10.28\%$ and $58\%$, respectively. As it can be seen in Figure \ref{fig:mnist_deep}, we also compared our method with the recent work of \cite{zeng2018global}. We observed the BCD in \cite{zeng2018global} %\footnote{code taken from \url{https://github.com/timlautk/BCD-for-DNNs-PyTorch}} 
to be unstable, sensitive to network architectures, and eventually, not converging after 300 epochs. Although we still exhibited some overfitting, we can see our approach is significantly better in handling of the vanishing gradient problem, and performs reasonably well. We further tested our performance with 20 fully-connected layers. Results show that although there is slightly more overfitting, our algorithm can still find a reasonable solution (Fig.~\ref{fig:mnist_deep}), showing its potential in helping with training scenarios with vanishing gradients.



%---------------------------- fig deep mnist ------------------------------
\begin{figure}[ht]
% \vskip -0.05in
\begin{center}
\centerline{
\includegraphics[width=\columnwidth]{imgs/mnist_deep.pdf}
}
% \vskip -0.05in
\caption{Test accuracies from deep architectures on MNIST. Block-ADMM demonstrates stable convergence and obtains final test accuracy of $\bf 94.43\%$ (10 layers), and $91.75\%$ (20 layers) respectively, while SGD and Adam (10 layers) fail due to vanishing gradients (Best viewed in color)}
\label{fig:mnist_deep}
\end{center}
% \vskip -0.35in
\end{figure}



%----------------------------
\subsubsection{Wall Clock Time Comparison}\label{time_cmp}

In this section, we analyze the batch and online versions of stochastic block-ADMM in training wall clock time and compare them against other baselines as illustrated in Figure \ref{fig:time}.  %The methods are implemented in PyTorch framework -- except for \cite{wang2019admm} that is implemented\footnote{code taken from \url{https://github.com/xianggebenben/dlADMM}} in "cupy", a NumPy-compatible matrix library accelerated by CUDA. 
Note Gotmare \etal and SGD are trained with a mini-batch size of 64 and \cite{zeng2018global,wang2019admm} are trained in a batch setting. Only the time taken for the \emph{training} was plotted in Fig.~\ref{fig:time} and stages such as initialization, data loading, etc were excluded. The online version shows faster convergence than \cite{gotmare2018decoupling} and simple SGD. Although \cite{zeng2018global} and \cite{wang2019global} have been convergence rates due to being batch methods, our approach achieves higher performance later on.% It can be also observed that our stochastic block-ADMM approach has comparable convergence speed with \cite{zeng2018global} while having noticeably superior performance over other baselines. We speculate that enforcing all the constraints by dual variables along with the efficient and cheap mini-batch updates in our method highly contributes to the convergence speed as well as its performance superiority over the other methods, including \cite{zeng2018global}. 


%---------------------------- fig time cmp mnist ------------------------------

\begin{figure}[ht]
\begin{center}
\centerline{
\includegraphics[width=\columnwidth]{imgs/time_comparison_new.pdf}
}
% \vskip -0.1in
\caption{Test set accuracy v.s. training wall clock time comparison of different alternating optimization methods for training DNNs on the MNIST dataset. Our methods (blue and orange) show superior performance vs. \protect\cite{zeng2018global} and \protect\cite{wang2019global} while converge faster than all other methods}
%  \vskip - 0.15in
\label{fig:time}
\end{center}
% \vskip -0.15in
\end{figure}



% %----------------------------
% \subsubsection{CIFAR-10}\label{exp:cifar}

% The previous works on training deep netowrks using ADMM have been limited to trivial networks and datasets (e.g. MNIST) \cite{taylor2016training,wang2019admm}. However, our proposed method does not have many of the existing restrictions and assumptions in the network architecture, as in previous works do, and can easily be extended to train non-trivial applications. It is critical to validate stochastic block-ADMM in settings where deep and modern architectures such as deep residual networks, convolutional layers, cross-entropy loss function, etc., are used. To that end, we validate the ability of our method is a supervised setting (image classification) on the CIFAR-10 dataset \cite{cifar} using ResNet-18 \cite{he2016deep}. To best of our knowledge, this is the first attempt of using ADMM for training complex networks such as ResNets. 


% For this purpose, we used 50,000 samples for training and the remaining 10,000 for evaluation. 
% To have a fair comparison, we followed the configuration suggested in \cite{gotmare2018decoupling} by converting Resnet-18 network into two blocks $(T=2)$, with the splitting point located at the end of {\sc conv3\_x} layer. We used the Adam optimizer to update both the blocks and the decoupling variables with the learning rates of $\eta_t = 5e^{-3}$ and $\zeta_t = 0.5$. We noted since the auxiliary variables $\mZ_t$ are not "shared parameters" across data samples, they usually require a higher learning rate in Algorithm \ref{alg:blockadmm}. Also, we found the ADMM step size $\beta_t = 1$ to be sufficient for enforcing the block's coupling. 


% Figure. \ref{fig:cifar} shows the results from our method compared with two baselines: \cite{gotmare2018decoupling}, and conventional end-to-end neural network training using back-propagation and SGD. Our algorithm consistently outperformed ~\cite{gotmare2018decoupling} however cannot match the conventional SGD results. There are several factors that we hypothesize that might have contributed to the performance difference: 1) in a ResNet the residual structure already partially solved the vanishing gradient problem, hence SGD/Adam performs significantly better than a fully-connected version; 
% % 2) The common data augmentation in CIFAR will end up sending a different training example to the optimization algorithm at each iteration, which does not seem to affect SGD but seem to affect ADMM convergence somewhat; 
% 2) we noticed decreasing the learning rate for $\Theta_t$ updates does not impact the performance as it does for an end-to-end back-propagation using SGD. Still, we obtained the best performance of ADMM-type methods on both MNIST and CIFAR datasets, showing the promise of our approach.
% % As illustrated, ADMM gets to a good performance fast and then slowly progress to higher accuracy..


% %---------------------------- fig cifar  ------------------------------

% \begin{figure}[htb]
% % \vskip 0.15in
% \begin{center}
% \centerline{
% \includesvg[width=\columnwidth]{imgs/cifar.svg}
% }
% % \vskip -0.05in
% \caption{Test set accuracy on CIFAR-10 dataset. Final accuracy "Block ADMM": $89.66\%$, "Gotmare \etal":$87.12 \%$, "SGD": $\bf 92.70\%$. (Best viewed in color.)}
% \label{fig:cifar}
% \end{center}
% % \vskip -0.2in
% \end{figure}

%----------------------------
\subsection{Supervised Disentangling on LFWA}\label{exp:hetero}


%----------------------------
In this section, we showcase the flexibility of stochstic block-ADMM in trainig deep networks with non-differentiable layers where conventional backpropagation cannot be used. For that purpose, we evaluate our proposed method in a supervised disentanglement problem where we used DeepFacto \ref{sec:deepfacto} to learn a nonnegative factorized representation of the DNN activations while training end-to-end on the LFWA dataset \cite{LFWTech}. Next, similar to \cite{liu2018exploring}, linear SVMs are used over the factorized space to predict face attributes. This setup examines the capability of the network to extract a disentangled representation that linearly corresponds to human-marked attributes that the network does not have prior knowledge of.

%This is to show the discriminative power of a disentangled representation. Note that there is no supervision over the attributes during the training of DeepFacto. 
% LFWA \cite{LFWTech} is a face verification dataset that contains 13,233 images with 72 attribute tags from 5,749 distinct people. 
We used the Inception-Resnet architecture from \cite{schroff2015facenet}, pre-trained on the VGGFace-2 \cite{Cao18} dataset as the back-bone. To incorporate an NMF, we follow the same approach as in Fig.~\ref{fig:deepfacto} where the pretrained DNN is the first block, and we add a simple fully-connected layer over the score matrix $\mS_t$ to train a face-verification network with a triplet loss~\cite{hoffer2015deep}.
%The choice of a simple fully-connected layer is two-folded. First, to lift the dimensions of the embedding needed for training, particularly when $\mS_t$ is low rank. Second, the embedding would be only a linear combination of the score matrix $\mS_t$, directly guiding it using the supervised signal coming from the Triplet Loss . 
We conjecture the score matrix $\mS_t$ will be guided to learn an disentangled factorization due to the nonnegativity constraint \cite{collins2018deep}. 
%Note that the latest activation in the network that is followed by a ReLU is selected from the Resnet-Inception network. This is due to the nonnegative constraint in the NMF, i.e. the input to the NMF should be also positive.
To have a warm start for an end-to-end training of DeepFacto, we first pre-train the NMF module having the Inception-Resnet block freezed. Then, we fine-tune the block parameters as well as the NMF module in an alternating fashion, similar to Algorithm \ref{alg:blockadmm}. Note, the rank of the NMF in DeepFacto is a hyperparameter and we selected three different values ($r=4, 32, 256$) in the experiments. The final $r=256$ is also the latent space dimensionality in \cite{liu2018exploring}.
Table. \ref{table:lfw} illustrates average prediction accuracy over LFWA attributes
% \footnote{The common 40 attributes with Celeb-A dataset \cite{liu2015faceattributes}} 
from DeepFacto and other supervised and weakly supervised baselines. This validates that DeepFacto has learned a meaningful representation of the attributes by disentangling the activations. To see visualization for individual dimensions learned by DeepFacto see supplementary materials \ref{sec:weakly_sup}.%, and the methodolgy to reshape the activations tensors into a matrix, 


% More details about the experiments are presented in the supplementary material.


\begin{table}[t]
\caption{Average prediction accuracy on 40 attributes from LFWA dataset. Weakly-supervised methods train the network without access to attribute labels. Final classification then comes from a linear SVM on their latent representations.}
\label{table:lfw}
\begin{center}
\begin{small}
\begin{sc}
% \vskip -0.15in
\begin{tabular}{lcccr}
\toprule
LFWA & Accuracy \\
\midrule
\cite{zhang2014panda} {\tiny (supervised)}                      &  81.00\%\\
\cite{liu2015deep} {\tiny (supervised)}                         &  84.00\%\\
\cite{liu2018exploring}  {\tiny (weakly-supervised)}             &  83.16\%\\
Deepfacto - rank 4 {\tiny (weakly-supervised)}                   & 74.80\%\\
Deepfacto - rank 32 {\tiny (weakly-supervised)}                  & 81.39\%\\
Deepfacto - rank 256 {\tiny (weakly-supervised)}                 & \textbf{87.03}\%\\
\bottomrule
\end{tabular}
\end{sc}
\end{small}
\end{center}
% \vskip -0.25in
\end{table}




% \textsc{Factorize the latent space\;} 
% {\textsc{Heatmaps.}} \cite{collins2018deep}
% {\subsection{all positive network.}}




\section{Conclusion}
We propose an unsupervised vision-and-language pre-training approach via retrieval-based multi-granular alignment to learn strong vision and language joint representations with un-aligned text and image sources. We introduce two core designs of our proposed approach: (1) construct a retrieval-based weakly-aligned image-text corpus. (2) multi-granular pre-training objectives to enable the model to capture the cross-modal alignment at different granularity levels. 
Our experiments show that our model can consistently outperform the previous state-of-the-art unsupervised pre-trained models and can achieve similar performance as the fully-aligned pre-trained models. 
\vspace{-0.4cm}
\paragraph{Limitations:} 
As we only consider the detected object list to retrieve the candidate sentences, the retrieved sentences often do not cover other visually grounded information compared to the ground truth captions.
Besides, the detected object tags are often those general concepts lacking diversity. 
% For example, in row 3 in Fig.~\ref{fig:visualization}, the stadium in the image is detected as ``building", which leads to the failure of retrieving any candidate that covers this object. Another limitation is that each object in the object list contributes equally during the retrieval. 
% This is sub-optimal as different objects should be weighted differently based on their detection confidence and visual importance. 
% Otherwise, the retrieved candidate sentences might focus on less important objects. For example in row 3, the retrieved sentences only cover ``tree" or ``water" from the object list, which leads to semantic discrepancy between the retrieved sentences and the image. 
Our retrieval results and in turn our pre-trained models could be affected by such limitations.
We hope to address the issue by learning a Siamese network between visual concepts and sentence for better retrieval and exploiting even larger uni-modal datasets to increase the diversity in the future research.
\vspace{-0.5cm}
\paragraph{Societal Impact:} 
The models are trained on the public datasets widely used in the community.
However, these datasets are known with biases, which may in turn affect our model predictions.
We do not recommend relying on the models to make real-world decisions.
%%%%%%%%% REFERENCES
{\small
\bibliographystyle{ieee_fullname}
\bibliography{egbib}
}

%%%%%%%%% Appendix
\clearpage
\appendix


\section{Details of Motivation Study}
As introduced in Section~\ref{section:intro}, we try to answer two questions: $(i)$ whether presenting a joint image-text data from non-parallel sources would improve the learned joint embedding space than alternatively presenting uni-modal data during pre-training. $(ii)$ If we fed joint image-text data to the model, how does its existing latent alignment affect the cross-modal representation learning. 

We conduct the unsupervised vision and language pre-training on Conceptual Captions (CC) by shuffling the image-text pairs. 
For pre-training objectives, we apply standard MLM + MRM. 
All other pre-training setup is the same as introduced in Section~\ref{sec:training_setup}. 
We first compare the round-robin and joint MLM + MRM pre-training, whose results are shown in Table~\ref{tab:data-fedding}.
We then evaluate how the alignment degree of the pre-training dataset affects the model performance, where the degree is controlled by the ratio of originally aligned image-text data in Conceptual Captions.
Table~\ref{tab:paired-ratio} shows the detailed results of each downstream task.
Their Meta-Ave scores are also plotted in Fig.~\ref{fig:intro}.
From these results, we obtained two important messages: 
$(i)$ joint image-and-text input is more optimal for UVLP than alternatively presenting uni-modal data from unparallel image and text corpus. 
$(ii)$ The more the latent semantic alignment exists in the image-text data the better the pre-trained model performs. 

We further explore the realistic unsupervised V+L pre-training, where the images and texts are from two different sources.
Specifically, we sample the images from Conceptual Captions and the texts from Book Corpus respectively.
Table~\ref{tab:bc_alignment} shows that the pre-trained model on our weakly aligned CC image and BC sentence corpus far outperforms that on random pairs, indicating it also holds that better latent image-text alignment leads to better pre-trained model's performance under realistic setting.
\begin{table}[!h]
\centering
\small
\tablestyle{5pt}{0.80}
\begin{tabular}{l|ccccc}\toprule
\multirow{2}{*}{} &VQA2 &NLVR2 &VE & RefCOCO+ & \multirow{2}{*}{ Meta-Ave } \\
&Test-Dev &Test-P &Test &Devs & \\\cmidrule{1-6}
random &70.3 &51.2 &75.3 &76.5 & 68.3 \\
proposed & \bf 71.2 & \bf 67.1 & \bf 77.1 & \bf 79.7 & \bf 73.8 \\
\bottomrule
\end{tabular}
\vspace{-0.3cm}
\caption{Pre-training on realistic CC + BC data}
\label{tab:bc_alignment}
\end{table}

\section{Effectiveness of Weighted ITM}
We compared the performance of pre-training our model with or without weighted ITM. 
The models are pre-trained on CC images and texts. 
As shown in Table~\ref{tab:WITM}, weighted ITM are consistently better than treating all the retrieved pairs with the same weight. 


\begin{table}[!htp]\centering
\footnotesize
\tablestyle{3pt}{0.80}
\begin{tabular}{l|ccccc}\toprule
\multirow{2}{*}{} &VQA2 &NLVR2 &VE & RefCOCO+ & \multirow{2}{*}{ Meta-Ave } \\
&Test-Dev &Test-P &Test &Devs & \\\cmidrule{1-6}
w/o $w_{\text{ITM}}$ &71.9 &72.6 &77.0 &79.7 & 75.3 \\
$w_{\text{ITM}}$ & \bf 72.1 & \bf 73.4 & \bf 77.3 & \bf 80.3 & \bf 75.8 \\
\bottomrule
\end{tabular}
\vspace{-0.3cm}
\caption{Ablation Study on weighted ITM}
\label{tab:WITM}
\end{table}


\begin{table*}[!ht]\centering
\small
\begin{tabular}{l|c|c|c|ccc|c}\toprule
\multirow{2}{*}{ Pre-training } &VQA2 &NLVR2 &VE & \multicolumn{3}{c|}{RefCOCO+} & \multirow{2}{*}{ Meta-Ave } \\
&Test-Dev &Test-P &Test &Dev &TestA &TestB & \\\cmidrule{1-8}
Round-Robin MLM+MRM &70.4 &51.1 &74.8 &73.3 &78.3 &\textbf{67.4} & 67.4 \\
Joint MLM+MRM &\textbf{70.6} &\textbf{52.4} &\textbf{74.9} &\textbf{74.5} &\textbf{79.4} &66.8 & \textbf{68.1} \\
\bottomrule
\end{tabular}
\vspace{-0.3cm}
\caption{Detailed evaluation results on four V+L downstream tasks with two different data feeding strategy for UVLP: (1) joint image-text data (joint MLM+MRM); (2) alternative uni-modal data (round-robin MLM+MRM).}
\label{tab:data-fedding}
\end{table*}

\begin{table*}[!ht]\centering
\small
\begin{tabular}{l|c|c|c|ccc|c}\toprule
\multirow{2}{*}{ Paired Ratio } &VQA2 &NLVR2 &VE & \multicolumn{3}{c|}{RefCOCO+} & \multirow{2}{*}{ Meta-Ave } \\
&Test-Dev &Test-P &Test &Dev &TestA &TestB & \\\cmidrule{1-8}
0\% &70.6 &52.4 &74.9 &74.5 &79.4 &66.8 & 68.1 \\
20\% &71.1 &70.0 &76.4 & 76.3 &80.3 &67.5 & 73.5 \\
40\% &71.4 &71.6 &77.2 &77.9 &82.4 &68.8 & 74.5 \\
60\% &71.9 &74.5 &77.8 &79.9 &84.4 &69.9 & 76.0 \\
80\% &72.2 &75.7 &78.4 &80.9 &85.7 &71.8 & 76.8 \\
100\% &72.5 &75.9 &78.7 &82.1 &86.6 &75.0 & 77.3 \\
\bottomrule
\end{tabular}
\vspace{-0.3cm}
\caption{Detailed evaluation results on four V+L downstream tasks with 6 sets of image and text corpus of different latent cross-modal alignment degree. The alignment degree is controlled by changing the ratio of original aligned image-text data from 0\% to 100\%.}
\label{tab:paired-ratio}
\end{table*}


\begin{figure*}[h!]
\centering
\includegraphics[width=14cm]{figures/Pos_Retrieve.png}
\vspace{-0.3cm}
\caption{Examples of retrieved text from both CC and BC. The covered grounded noun phrases in retrieved sentences are highlighted in green bar for positive examples.}
\label{fig:pos-ret}
\end{figure*}

\section{Downstream Task Details}
We describe the details of fine-tuning on the four different downstream tasks: Visual Question Answering (VQA2), Natural Language for Visual Reasoning (NLVR2), Visual Entailment (VE), and Referring Expression (RefCOCO+). We mainly follow the setup of UNITER\cite{chen2020uniter} for each downstream task with minor adjustments.  

\noindent\textbf{VQA2}
Given a question about an image, the task is to predict the answer to the question. Following \cite{yu2019mcan}, we take 3,129 most frequent answers as answer candidates. We use both training and validation sets from VQA 2.0 for fine-tuning. Following UNITER, we also leverage data from Visual Genome\cite{krishna2017visualgenome} to augment the best performance on the test-dev split. We fine-tune the model with a binary cross-entropy loss with a peak learning rate of $6\times10^{-5}$ for 20 epochs. The training batch size is set as 480. 

\noindent\textbf{NLVR2}
NLVR2 is a task for visual reasoning. The objective is to determine whether a natural language statement is true or not given a pair of input images. 
We follow UNITER's setup treating each data point as two text-image pairs with repeated text. 
The two [CLS] outputs from the model are then concatenated as the joint embedding for the example. We apply a multi-layer perceptron (MLP) classifier on top of this joint embedding for the final classification. Unlike~\cite{li2020unsupervised} that conducts additional ``pre-training" on NLVR2 datasets, we only fine-tune the model with the task-specific objective to maintain the same setting as all other downstream tasks. We train the model for 8 epochs with a batch size of 60 and a peak learning rate of $3\times10^{-5}$. 

\noindent\textbf{VE}
Visual Entailment is a task built on Flickr30k Images\cite{young-etal-2014-image}, where the goal is to determine the logical relationship between a natural language statement and an image. Similar to the Natural Language Inference problem in NLP, this task is formatted as a 3-way classification problem to predict if the language statement entails, contradicts, or is undetermined with respect to the given image. An MLP transformer classifier is applied to the output of the $\text{[CLS]}$ token to make the final prediction. The model is fine-tuned using cross-entropy loss. We set the batch size as 480 and the peak learning rate as $8\times10^{-5}$. The model is fine-tuned for 4 epochs for this downstream task. 

\noindent\textbf{RefCOCO+}
The referring expression task involves locating an image region given a natural language phrase. We use RefCOCO+ \cite{yu2016modeling} as the evaluation dataset. Bounding box proposals from VinVL object detectors are used for fine-tuning. A proposal box is considered correct if it has an IoU with a gold box larger than 0.5. We add an MLP layer on top of the region outputs from the Transformer to compute the alignment score between the language phrase and each proposed region. We fine-tune our model for 20 epochs with a peak-learning rate of $2\times10^{-4}$.


\begin{figure*}[h!]
\centering
\includegraphics[width=14cm]{figures/Neg_Retrieve.png}
\vspace{-0.1cm}
\caption{Examples of retrieved text from both CC and BC. The mistakenly covered grounded noun phrases in retrieved sentences are highlighted in red bar for negative examples.}
\label{fig:neg-ret}
\end{figure*}

\begin{figure}[h!]
\centering
\includegraphics[width=0.7\linewidth]{figures/attention_viz_1.jpg}
\vspace{-0.2cm}
\caption{Text-to-image attention given the aligned pair whose caption is ``person in a leather jacket riding a motorcycle on the road".}
\label{fig:attn_viz_1}
\end{figure}

\begin{figure}[h!]
\centering
\includegraphics[width=0.7\linewidth]{figures/attention_viz_2.jpg}
\vspace{-0.2cm}
\caption{Text-to-image attention given the aligned pair whose caption is ``girl in a blue kayak floating on the picturesque river at sunset".}
\label{fig:attn_viz_2}
\end{figure}

\begin{figure}[h!]
\centering
\includegraphics[width=0.7\linewidth]{figures/attention_viz_3.jpg}
\vspace{-0.2cm}
\caption{Text-to-image attention given the aligned pair whose caption is ``people walking by the christmas tree and stage area".}
\label{fig:attn_viz_3}
\end{figure}

\begin{figure}[h!]
\centering
\includegraphics[width=0.7\linewidth]{figures/attention_viz_4.jpg}
\vspace{-0.2cm}
\caption{Text-to-image attention given the aligned pair whose caption is ``single cowboy guiding a line of horses through the desert".}
\label{fig:attn_viz_4}
\end{figure}
\section{Additional Visualization}
We present additional examples of retrieved text from both CC and BookCorpus. Specifically, we demonstrate more positive examples in Fig \ref{fig:pos-ret} that covers the appropriate grounded noun phrases. We also share some negative examples in Fig \ref{fig:pos-ret}. As analyzed in the limitation section, the current language embedding model weighs all the object tags equally to generate the joint embedding representation. This can lead to mistakenly focused object tags when retrieving the text. In row 1 of Fig \ref{fig:neg-ret}, texts retrieved cover less important noun phrases such as ``finger" and ``hair" instead of the more important noun phrase "baby". Row 2 of Fig \ref{fig:neg-ret} demonstrate mistakenly retrieved texts due to the limitation of the pre-defined object categories in the object detector. In this example,  the students in the image are detected as ``person" or ``man", which leads to the failure of retrieving any valid text.    

We also demonstrate more examples on text-to-image attention between the pre-trained U-VisualBert and {\ModelName } on the Conceptual Captions Validation set in Fig \ref{fig:attn_viz_1}, \ref{fig:attn_viz_2}, \ref{fig:attn_viz_3}, \ref{fig:attn_viz_4}. These examples provide additional evidence on the better local alignment captured by \ModelName. 

\end{document}
