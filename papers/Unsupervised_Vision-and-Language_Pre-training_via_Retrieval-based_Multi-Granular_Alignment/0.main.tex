% CVPR 2022 Paper Template
% based on the CVPR template provided by Ming-Ming Cheng (https://github.com/MCG-NKU/CVPR_Template)
% modified and extended by Stefan Roth (stefan.roth@NOSPAMtu-darmstadt.de)

\documentclass[10pt,twocolumn,letterpaper]{article}

%%%%%%%%% PAPER TYPE  - PLEASE UPDATE FOR FINAL VERSION
%\usepackage[review]{cvpr}      % To produce the REVIEW version
% \usepackage{cvpr}              % To produce the CAMERA-READY version
\usepackage[pagenumbers]{cvpr} % To force page numbers, e.g. for an arXiv version

% Include other packages here, before hyperref.
\usepackage{graphicx}
\usepackage{amsmath}
\usepackage{amssymb}
\usepackage{booktabs}
\usepackage{multirow}

\usepackage{booktabs, multirow} % for borders and merged ranges
\usepackage{soul}% for underlines
\usepackage[table]{xcolor} % for cell colors
\usepackage{changepage,threeparttable} % for wide tables
% It is strongly recommended to use hyperref, especially for the review version.
% hyperref with option pagebackref eases the reviewers' job.
% Please disable hyperref *only* if you encounter grave issues, e.g. with the
% file validation for the camera-ready version.
%
% If you comment hyperref and then uncomment it, you should delete
% ReviewTempalte.aux before re-running LaTeX.
% (Or just hit 'q' on the first LaTeX run, let it finish, and you
%  should be clear).
\usepackage[pagebackref=true,breaklinks=true,colorlinks,bookmarks=false]{hyperref}

% Authors
\usepackage{xcolor}
% define colors from here: http://latexcolor.com/
\definecolor{bittersweet}{rgb}{1.0, 0.44, 0.37}
\definecolor{mygreen}{rgb}{0.29, 0.7, 0.48}
% colors: red, orange, cyan, 
\newcommand{\mingyang}[1]{\textcolor{red}{\small{\bf [ #1 -- Mingyang ]}}}
\newcommand{\licheng}[1]{\textcolor{mygreen}{\small{\bf [ #1 -- Licheng ]}}}
\newcommand{\ning}[1]{\textcolor{orange}{\small{\bf [ #1 -- Ning ]}}}
\newcommand{\aman}[1]{\textcolor{cyan}{\small{\bf [ #1 -- Aman ]}}}
\newcommand{\mj}[1]{\textcolor{bittersweet}{\small{\bf [ #1 -- MJ ]}}}


% Support for easy cross-referencing
\usepackage[capitalize]{cleveref}
\crefname{section}{Sec.}{Secs.}
\Crefname{section}{Section}{Sections}
\Crefname{table}{Table}{Tables}
\crefname{table}{Tab.}{Tabs.}

\newcommand{\myparagraph}[1]{\vspace{0.25em}\noindent\textbf{#1}}
\newcommand{\tablestyle}[2]{\setlength{\tabcolsep}{#1}\renewcommand{\arraystretch}{#2}\centering\footnotesize}
%%%%%%%%% PAPER ID  - PLEASE UPDATE
\def\cvprPaperID{5426} % *** Enter the CVPR Paper ID here
\def\confName{CVPR}
\def\confYear{2022}
\def\TaskName{UVLP}
\def\TaskFullName{unsupervised V+L pre-training without parallel data}
\def\ModelName{$\mu$-VLA}
\def\ModelFullName{Unsupervised Vision-and-Language Pre-training via Retrieval-based Multi-Granular Alignment}
\def\uvisualbert{U-VisualBERT}
\newcommand{\head}[1]{\noindent\textbf{#1}}


\begin{document}

%%%%%%%%% TITLE - PLEASE UPDATE
\title{Unsupervised Vision-and-Language Pre-training via Retrieval-based Multi-Granular Alignment} 

\author{
Mingyang Zhou$^1$\thanks{The two authors contribute equally.} \quad Licheng Yu$^3$\footnotemark[1]  \quad  Amanpreet Singh$^3$  \quad  Mengjiao Wang$^3$  \quad  Zhou Yu$^2$  \quad  Ning Zhang$^3$
\\ 
$^1$Uiversity of California, Davis   \quad  $^2$ Columbia University  \\  $^3$Meta AI\\
\tt{\small minzhou@ucdavis.edu}, \tt{\small zy2461@columbia.edu}, \tt{\small\{lichengyu, asg, mengjiaow, ningzhang\}@fb.com} \\
}

\maketitle

%%%%%%%%% ABSTRACT
\begin{abstract}
Vision-and-Language (V+L) pre-training models have achieved tremendous success in recent years on various multi-modal benchmarks.
However, the majority of existing models require pre-training on a large set of parallel image-text data, which is costly to collect, compared to image-only or text-only data.
In this paper, we explore unsupervised Vision-and-Language pre-training (\TaskName) to learn the cross-modal representation from non-parallel image and text datasets. 
We found two key factors that lead to good unsupervised V+L pre-training without parallel data:  $(i)$ \textit{joint image-and-text input} $(ii)$ \textit{overall image-text alignment (even for non-parallel data)}. 
Accordingly, we propose a novel unsupervised V+L pre-training curriculum for non-parallel texts and images. 
We first construct a weakly aligned image-text corpus via a retrieval-based approach, then apply a set of multi-granular alignment pre-training tasks, including region-to-tag, region-to-phrase, and image-to-sentence alignment, to bridge the gap between the two modalities.
A comprehensive ablation study shows each granularity is helpful to learn a stronger pre-trained model.
We adapt our pre-trained model to a set of V+L downstream tasks, including VQA, NLVR2, Visual Entailment, and RefCOCO+. 
Our model achieves the state-of-art performance in all these tasks under the unsupervised setting.
\end{abstract}

%%%%%%%%% BODY TEXT
% \vspace{-0.15in}
\section{Introduction} \label{sec:intro}

Deep Neural Networks (DNNs) are highly non-convex functions with ill-conditioned Hessians and are believed to have multiple local minima and saddle points. Most networks are trained with %This makes training them very challenging~\cite{dauphin2014identifying}. Optimization methods for training DNNs have been an active line of research; commonly, 
Stochastic Gradient Descent (\textit{SGD}) and its adaptive learning rate variants \textit{e.g.,} \textit{Adam} \cite{kingma2014adam} are used to optimize the DNNs with backpropagation. Although these approaches have been the most successful, they suffer from issues such as vanishing gradients in deep layers, a significant memory footprint for storing the gradients, and difficulty to parallelize across layers because backpropagation has to be done sequentially\cite{taylor2016training}. In addition, in the presence of non-differentiable layers, conventional backpropagation training cannot be applied.% as error signals can not pass through.


Alternating Direction Method of Multipliers (ADMM) is a simple yet powerful approach that decouples optimization variables and optimizes the augmented Lagrangian in a primal-dual scheme. %It is known to be a simple yet powerful algorithm to solve convex problems with linear convergence rate in a distributed and parallel manner \cite{boyd2011distributed}. 
It has shown promise in solving certain families of non-convex problems \cite{wang2019global,huang2018mini}.
Recently, optimization of the neural networks with such alternating direction techniques has gained rising attention \cite{zeng2018global,zeng2019convergence,zhang2017convergent,gu2018fenchel,askari2018lifted} which would potentially avoid the disadvantages of the SGD and introduce beneficial properties such as \textit{fast(er)} convergence, ease of \textit{parallelization} and \textit{distributed} training, and being able to enforce additional (non-differentiable) constraints on the DNN tensors.


Despite their advantages, there are several reasons ADMM-like methods are not widely used in DNN training. The performance of these methods is usually not as good as conventional backpropagation with SGD variants, the algorithms are usually batch mode which directly restricts the number of trainable parameters and training data as well, updates are in closed-from which prohibits the use of complicated architectures while being memory intensive, \textit{etc.} Further, existing ADMM-like methods have restrictive assumptions in the architecture of the network which prohibits the extension to non-trivial networks such as ResNets \cite{he2016deep}. Work of \cite{taylor2016training} is of this kind which, despite the parallelization capabilities introduced by ADMM, the size of the training data is linearly limited by the number of cores. 
% \cite{gotmare2018decoupling} proposed to split DNN into blocks using gluing variables similar to ADMM, but they did not utilize dual variables as in ADMM.


In this paper, we propose Stochastic Block-ADMM which addresses the aforementioned issues. Stochastic Block-ADMM separates DNN parameters into an arbitrary number of blocks and uses stochastic gradients to update each block. The error signals are passed between the blocks by introducing auxiliary variables at the splitting points. We present both batch and online versions of the Stochastic Block-ADMM which can be extended to settings where computational resources are limited, data is constantly changing such as in reinforcement learning or training with data augmentation techniques. We provide a convergence proof for the proposed approach and verify its performance on several deep learning benchmarks.


%Unlike in conventional end-to-end backpropagation training scheme, a
An ADMM formulation of deep networks also allows us to add additional \textit{non-differentiable} constraints to the learning problem. In this paper, %as an illustration of such potential application using ADMM, 
we explore the problem of supervised feature disentanglement by inserting non-negative factorization layers into the network. Nonnegative Matrix Factorization (NMF) has been shown to generate sparse and interpretable representations due to the non-negative constraints over the factorization matrices \cite{lee1999learning}. Jointly training an NMF decomposition with deep learning adds non-differentiable non-linearity and \textit{cannot} be addressed by the conventional backpropagation with SGD algorithms. We show results training these networks via ADMM and their performance on a supervised feature disentanglement benchmark.
% \cite{collins2018deep} proposed applying NMF over different activations of CNN for object localization. 
% This would support the belief that the CNNs would learn semantic part-of-object filters during training \cite{gonzalez2018semantic, bau2017network}. 


In summary, our paper makes the following contributions:
\begin{itemize}
% \vspace{-0.05in}
    \item We propose Stochastic Block-ADMM for training deep networks. This improves over previous ADMM approaches (in training deep networks) which only work in batch setting.
    % Experiments show that our algorithm outperforms previous attempts of using ADMM in deep network training.
    \item We propose an online variant of the Stochastic Block-ADMM for further efficiency in computations. 
    \item We prove the convergence of the proposed Stochastic Block-ADMM algorithm.
    \item We propose DeepFacto, which jointly trains a non-negative matrix factorization layer with a deep network using ADMM, and show its capability in supervised feature disentanglement.
    % \vspace{-0.05in}
\end{itemize}


\section{Related Work}
\begin{figure*}[ht!]
\centering
\includegraphics[width=15.4cm]{figures/upvlp_model_4.pdf}
\caption{Overview of our method. On the left we form three types of image-text pairs as input data to learn cross-modal alignment on three different granularities: region-tag alignment, region-phrase alignment, and image-text alignment. The models is iteratively pre-trained on each granularity and the model parameters are shared. On the right-hand side, we demonstrate the details of the pre-training objectives for each granularity.  
}
\label{fig:model}
\end{figure*}

\paragraph{Vision-and-Language Pre-training}  
Inspired by the success of natural language processing~\cite{devlin2018bert, brown2020language}, there is a recent surge of interest in pre-training for vision and language.
For example, there are different architectures (\eg two-stream models~\cite{lu2019vilbert,tan2019lxmert,lu202012,yu2020ernie,li2021albef,kamath2021mdetr} vs. single-stream models~\cite{li2019visualbert,li2020unicoder,su2019vl,chen2020uniter,li2020unimo}), features (\eg regions~\cite{anderson2018bottom} \vs grids~\cite{huang2020pixel}), backbones (\eg ConvNets~\cite{huang2020pixel} \vs Transformers~\cite{kim2021vilt}) \etc.
All these works aim to exploit the large-scale aligned image-text corpora~\cite{lin2014microsoft,krishna2017visualgenome,sharma2018conceptual,ordonez2011sbu,kamath2021mdetr} to pre-train a powerful multi-modal model, which is then adapted to various downstream V+L tasks, such as VQA~\cite{antol2015vqa}, NLVR2~\cite{suhr2018corpus}, Visual Entailment (VE)~\cite{xie2019visual}, Referring Expression Comprehension~\cite{yu2016modeling}, and Image-Text Retrieval.

Various pre-training tasks have been introduced to achieve this, including the most notable Masked Language Modeling (MLM), Masked Region Modeling (MRM), and Image-Text Matching (ITM).
Several other variants have also been explored, such as predicting the object tags~\cite{li2020oscar, hu2020vivo}, sequence generation~\cite{zhou2020unified, wang2021simvlm}, word-region alignment~\cite{chen2020uniter}.
In this paper, we propose learning a multi-granular alignment between word and region, phrase and region, and image and sentence to better bridge the gap between vision and language.

\paragraph{Unsupervised Vision-and-Language Pre-training without Parallel Data}
Inspired by the works on multi-lingual contextual language modeling~\cite{conneau2019unsupervised,lample2018phrase,lample2017unsupervised, conneau2017word}, \uvisualbert~\cite{li2020unsupervised} first propose the \textit{unsupervised} vision-and-language pre-training without parallel data (\TaskName). 
\uvisualbert~\cite{li2020unsupervised} conducts the masked prediction on the text-only and image-only corpora and introduce the object tags as anchor points to bridge the two modalities.
The authors treat the tags as a sentence when performing MLM, where tags provide alignment with the regions in a picture and implicitly learn a tag-region-level alignment.
However, the anchor tags are still quite different from the text input, missing the sentence completeness and naturalness.
Besides, the latent cross-modal alignment is shown to be important in our analysis (from Fig.~\ref{fig:intro}).
As comparison, our pre-training involves a retrieval-based weakly aligned V+L data construction and learns a more comprehensive multi-granular cross-modal alignment.
With same data as \uvisualbert, our approach achieves a clear and consistent gain across all the downstream tasks in our experiments.



%\section{\ModelFullName}
\section{Method}
In this section, we introduce the two core components of our \ModelName's architecture for~\TaskFullName: 
(1) construct a weakly aligned image-text corpus from independent vision and language data sources; 
(2) our novel pre-training curriculum to enable the model to capture the cross-modal alignment on three granularity including region-to-tag level alignment (RT),  region-to-noun phrase level alignment (RN), and image-to-sentence level alignment (IS). 

\subsection{Model Overview}
We use the well-known single-stream model architecture for our experiments as~\cite{li2019visualbert,li2020unicoder,su2019vl,chen2020uniter,li2020unimo}.
As shown in Fig.~\ref{fig:model}, our main backbone is a single transformer, where we feed the concatenation of visual embeddings of an image and the tokens of a caption as its input.
Given an image $\mathbf i$, we first use an off-the-shelf Faster R-CNN (VinVL~\cite{Zhang_2021_CVPR}) to detect the objects $\mathbf v = \{ v_1, ..., v_{k^v} \}$.
The visual embedding of each region is then encoded as the sum of its regional feature, its location embedding\footnote{The 5-dimensional vector [$\frac{x_1}{W}$,$\frac{y_1}{H}$,$\frac{x_2}{W}$,$\frac{y_2}{H}$,$\frac{(y_2-y_1)(x_2-x_1)}{W.H}$] is projected to the visual embedding space. $(x_1,y_1), (x_2,y_2)$ are the coordinates of the top left and bottom right point of the detected region, and $W,H$ are the image width and height.}, and the modality embedding.
For a given caption $\mathbf t$, we denote its tokenized sequence as $\mathbf t = \{ t_1, ..., t_{k^t} \} $.
After multiple layers of self-attention, the two modalities are fused together and the output hidden vectors can be used for various pre-training tasks.

\subsection{Weakly-aligned Image-Text Corpus}
\label{section:data_aug}
As in the analysis of Sec~\ref{section:intro}, we observe a strong correlation between the degree of image-text alignment in the training data and the performance of the pre-trained model.
% Given the unpaired image collection $I =\{I_1, I_2, \dots, I_{n_I}\}$ and text corpus $T = \{T_1, T_2, \dots, T_{n_T}\}$, 
Inspired by this finding, we believe it important to initialize some weak semantic alignment between the two modalities as the input source.
Specifically, we retrieve $\mathrm{k}$ sentences that are semantically closed to a given $I_i$. 
Previous work~\cite{tan2020vokenization} shows the visually grounded caption covers a good ratio of words that are naturally related to specific visual contents, \eg concrete nouns. 
Thus, we utilize the semantic association between the objects that appear in the image and a candidate sentence as the indicator to measure the alignment degree.

Specifically, we take the object tags $\mathbf o = \{ o_1, ..., o_{k^o} \}$ from the above detected $\mathbf v$ and feed the sequence into an off-the-shelf sentence BERT embedding model~\cite{reimers-gurevych-2019-sentence} to obtain the query embedding $\mathbf e_{\mathbf o}$.
Similarly, we feed each candidate sentence into the same model getting the candidate embedding $\mathbf e_{\mathbf t}$.
We retrieve the top $\mathrm{K}$ candidates with the highest cosine similarity score to form an initial weakly-aligned image-text pairs for a given image $\mathbf i$.
We denote the retrieved captions as $\{ \mathbf t^r (\mathbf i) \}_{r=1}^K$ and the overall weakly aligned corpus as $\mathbf R$.

% Specifically, we filter objects that are detected with confidence score lower than 0.2 and with small bounding box region size that is less than 0.05 of the image size  to focus on critical objects shown in the image. \mingyang{Should appear in the training set up.}
% Specifically, we feed the object-tag sentence and a candidate sentence into the pre-trained off-the-shelf Sentence Bert Embedding model \cite{reimers-gurevych-2019-sentence} to obtain their corresponding embeddings $\mathbf e_{\mathbf{o}}$ and $\mathbf e_{\mathbf{t}}$. 
% We then select the top $\mathrm{k}$ sentence candidates $T_{\text{ret}_i} = {T_{\text{ret}_i}^1, \dots,T_{\text{ret}_i}^k}$ with the highest cosine similarity score with the object list embedding vector $e_{O_i}$
% Eventually, we would have k retrieved sentences $T_{ret} = {T_{ret}^{1}, \dots,T_{ret}^{k}}$ for each image 
% to form the weakly-aligned image-text pairs with the query image $I_i$. 

\subsection{Pre-training Tasks}
% We consider that the success of vision and language pre-training relies on the capability to understand the cross-modal alignment on various granularity including: region-to-object tag level, region-to-noun phrase level, and image-to-sentence level. 
In this subsection, we introduce a set of pre-training objectives that we designed to facilitate the model to capture the different levels of vision and language alignment.
Fig.~\ref{fig:model} shows the overview of our model and its pre-training tasks.

% Unlike \uvisualbert \cite{li2020unsupervised} that focus on just capturing the region to object-tag alignment, we consider the successful unsupervised V+L pre-training should understand the cross-modal alignment on various granularity. Thus, we propose pre-trianing cu

\subsubsection{Region-Tag Alignment Learning}
We first propose to align the object tags onto the image regions.
As shown in Fig.~\ref{fig:model}(a), We concatenate the object tags detected from each image with its source image to form an input pair $[\mathbf o, \mathbf v]$ fed into the model. 
% The detected tags are processed similarly as the normal caption as a sequence of tag tokens $o= [o_{1:l}]$. 
%Following \cite{li2020unsupervised}, the position embedding for each tag token is the spatial box coordinate embedding of its corresponding region. 
% The position embeddings allow the model to distinguish the tags from different regions. 
% Given the pair of object list and image [$o$, $v$] from the training dataset $D$, 
% We randomly mask some tag tokens $o_k$, and some regions $v_j$, and train our model to predict the masked tag tokens and the properties of the masked regions. 
We denote the mask indices as $\mathbf{m}\in \mathbb{N}^M$\footnote{$\mathbb{N}$ is the natural numbers, $M$ is the vocabulary size, and $\mathbf{m}$ is the set of masked indices.}. 
We randomly mask out the object tags and regions, and apply masked language modeling (MLM) and masked region modeling (MRM) for the pre-training.

Specifically, MLM on the object tags is formulated as
\begin{equation*}
    \mathcal{L}_{\text{MLM}}^{\text{R-T}} = - \mathbb{E}_{(\mathbf o, \mathbf v)\sim \mathbf I} \log{P(\mathbf o_{\mathbf m} |\mathbf o_{\backslash \mathbf m}, \mathbf v)},
\end{equation*}
where the goal is to predict the masked object tags based on the observation of their surrounding tags $\mathbf o_{\backslash \mathbf m}$ and image regions $\mathbf v$.
On the vision side, MRM includes both masked region classification loss (MRC) and masked region feature regression loss (MRFR):
\begin{equation*}
    \begin{split}
    \mathcal{L}_{\text{MRM}}^{\text{R-T}} = \mathbb{E}_{(\mathbf o, \mathbf v)\sim \mathbf I}  [f_{\text{MRC}}(\mathbf v_{\mathbf m} | \mathbf v_{\backslash \mathbf m}, \mathbf o) + f_{\text{MRFR}}(\mathbf v_{\mathbf m} | \mathbf v_{\backslash \mathbf m}, \mathbf o) ].
    \end{split}
\end{equation*}
% To calculate the $\mathcal{L}_{\text{MRC}}$ and $\mathcal{L}_{\text{MRFR}}$, we first obtain the transformer output $h_j$ of the masked region $i=v_j$ at the final layer. For $\text{MRC}$, a fully connected (FC) layer $\phi_{\text{MRC}}$ is applied to predict the object category as a normalized distribution over the total number of $K$ classes of the object categories. 
% Thus, $\mathcal{f}_{\text{MRC}}=CE(\phi_{\text{MRC}(h_j)}, c_j)$ is the standard cross-entropy loss. 
% Additionally, for $\text{MRFR}$ we have another FC layer $\phi_{\text{MRFR}}$ to project $h_j$ into the same dimension space of the ROI feature of the masked region $f_j$. Then we apply L2 regression to compute the loss: $f_{\text{MRFR}}=||\phi_{\text{MRFR}(h_j)}- f_j||_2^2$.
Between the two, MRC learns to predict the object semantic class for each masked region  $c(\mathbf v_{\mathbf m})$.
We feed the last hidden output of the masked region $\mathbf v_{\mathbf m}$ into a FC layer and softmax function to predict the classification probabilities $g_{\theta} ( \mathbf v_{\mathbf m} )$.
The objective is to minimize the cross-entropy of
$ f_{\text{MRC}}(\mathbf v_{\mathbf m} | \mathbf v_{\backslash \mathbf m}, \mathbf o) = \mbox{CE}( c(\mathbf v_{\mathbf m}) , g_{\theta} ( \mathbf v_{\mathbf m} ) ) $.
MRFR learns to regress the transformer output of each masked region $\mathbf v_{\mathbf m}$ to its visual features. 
We apply a FC layer to convert its hidden output to a vector $h_\theta (\mathbf v_{\mathbf m})$ of the same dimension as the input regional feature $r(\mathbf v_{\mathbf m})$.
We apply L2 regression:
$f_{\text{MRFR}}(\mathbf v_{\mathbf m} | \mathbf v_{\backslash \mathbf m}, \mathbf o)  = || h_\theta (\mathbf v_{\mathbf m}) - r(\mathbf v_{\mathbf m}) ||^2_2$.

For region-tag alignment learning, we have our pretraining objective function as 
\begin{equation}\nonumber
\mathcal{L}^{\text{R-T}} =  \mathcal{L}_{\text{MLM}}^{\text{R-T}} + \mathcal{L}_{\text{MRM}}^{\text{R-T}} 
\end{equation}

\subsubsection{Region-Noun Phrase Alignment Learning}

Due to the small vocabulary size of object tags, the region-tag alignment learning can only capture a limited amount of localized concepts.
To increase the diversity of concepts, we propose to align the noun phrases from the retrieved sentences to the corresponding regions as well.
As in Fig.~\ref{fig:model}(b), given an image $\mathbf i$ and its retrieved weakly aligned caption $\mathbf t^r (\mathbf i)$, we first detect the noun phrases from the caption using spacy~\cite{spacy2}.
Note the detected noun phrases sometimes contain the attribute words, which further benefits this pre-training task.
We link the noun phrase to its closest visual region by computing the word2vec similarity between the phrase and object tag (associated to each region).
The pre-training still consists of MLM and MRM but are performed with different masking strategy and supervision signal.

Specifically, for both MRM and MLM, we only mask the linked noun phrases from the caption or the linked object regions.
We make the masking probability proportional to the linked similarity score.
Each time we only mask out one modality (phrase or region) to encourage it to be recovered by its linked content.
The region-to-phrase MLM is then formulated as
$\mathcal{L}_{\text{MLM}}^{\text{R-P}} = - \mathbb{E}_{(\mathbf v, \mathbf t^r)\sim \mathbf R} \log{P(\mathbf t^r_{\mathbf m} |\mathbf t^r_{\backslash \mathbf m}, \mathbf v)}$.

On the vision side, we propose using the phrase-guided masked region-to-token classification (p-MRTC) on the masked regions:
\begin{equation*}
    \begin{split}
    \mathcal{L}_{\text{MRM}}^{\text{R-P}} = \mathbb{E}_{(\mathbf v, \mathbf t^r)\sim \mathbf R}  f_{\text{p-MRTC}}(\mathbf v_{\mathbf m} | \mathbf v_{\backslash \mathbf m}, \mathbf t^r),
    \end{split}
\end{equation*}
where we directly classify the masked region to its linked noun phrase (sub-word tokens) in BERT vocabulary.
Enlarging the vocabulary has shown to be beneficial to MRM~\cite{Zhou_2021_CVPR}.
Our proposed p-MRTC leverages the additional noun-phrase to encourage more diverse local region to language alignment.

For region-noun phrase alignment learning, we have our pretraining objective function as 
\begin{equation}\nonumber
\mathcal{L}^{\text{R-P}} =  \mathcal{L}_{\text{MLM}}^{\text{R-P}} + \mathcal{L}_{\text{MRM}}^{\text{R-P}} 
\end{equation}


\subsubsection{Image-Sentence Alignment Learning}
\label{section:itm}
We apply image-text matching (ITM) objective as the previous supervised V+L pre-training research \cite{chen2020uniter,li2020unicoder} to learn the cross-modal sentence-level alignment. 
As in Fig.~\ref{fig:model}(c), given an input pair [$\mathbf v$, $\mathbf t^r$], the final hidden vector of the special token $\text{[CLS]}$ is fed through a FC layer to output a single score $\mathbf s_{\theta}(\mathbf v, \mathbf t^r)$, which predicts if the given image-text input is a semantically matched pair or not. 
We use the label $y\in \{0,1\}$ to indicate if a retrieved pair is a match.
The training objective for the ITM task is to minimize the binary cross-entropy loss:
$
\mathcal{L}_{\text{ITM}} =\mbox{CE}( y ,s_{\theta}(\mathbf{v}, \mathbf{t}^r) ) 
$.
% \begin{equation*}
%     \mathcal{L}_{\text{ITM}} = -\mathbb{E}_{(\mathbf{v},\mathbf{t^r})\sim \mathbf{R}} [y \log s_{\theta}(\mathbf{v}, \mathbf{t}^r) + (1-y) \log (1-s_{\theta}(\mathbf{v}, \mathbf{t}^r))]
% \end{equation*}
On the language side, we also apply standard MLM to help the model learn to align other language tokens besides noun phrases and object tags to the visual context.
The objective function is then formulated as $\mathcal{L}_{\text{MLM}}^{\text{I-S}} = - \mathbb{E}_{(\mathbf v, \mathbf t^r)\sim \mathbf R} \log{P(\mathbf t^r_{\mathbf m} |\mathbf t^r_{\backslash \mathbf m}, \mathbf v)}$.
The image-sentence level alignment pretraining objective function is
\begin{equation}\nonumber
\mathcal{L}^{\text{I-S}} =  \mathcal{L}_{\text{MLM}}^{\text{I-S}} + \mathcal{L}_{\text{ITM}}
\end{equation}


\subsection{Multi-Granular Pre-training Curriculum}
We propose a multi-granular curriculum to iteratively pre-train the model on the region-to-tag, region-to-noun phrase, and image-to-sentence level. 
According to our findings in Sec.~\ref{section:intro}, learning from image-text pairs with higher degree of cross-modal alignment is beneficial to the performance of unsupervised V+L pre-trained model. 
Therefore, we propose using an estimated image-text alignment score to guide our multi-granular pre-training. 
Specifically, we have an ITM header defined in Sec.~\ref{section:itm} to learn the image-text alignment. 
We also use it to predict matching score as a weight to modulate the input data for each of our retrieval-based pre-training tasks. 
This allows us to provide more importance to relatively more aligned image-text pairs over time to help our model to learn better cross-modal alignment on multiple granularities. 

To train the alignment model's ITM classifier, we use our retrieved corpus $\mathbf{R}$ as positive samples and randomly shuffled pairs as negative samples in the first $m$ epochs. This warms up the models to make reasonable estimations on the alignment of image-text input pairs.
After $m$ epochs, we start to incorporate the alignment prediction score $w_{\text{ITM}}$ in our training objective. 
To summarize, our multi-granular pre-training loss is 
\begin{equation}\nonumber
\mathcal{L}=  
\begin{cases}
      \mathcal{L}^{\text{R-T}} + \mathcal{L}^{\text{R-P}} +\mathcal{L}^{\text{I-S}}   &\text{if epoch} < m\\ 
      \mathcal{L}^{\text{R-T}} + w_{\text{ITM}} (\mathcal{L}^{\text{R-P}}  +  \mathcal{L}^{\text{I-S}}) &\text{if epoch} \geq m, \\
    \end{cases}
\end{equation}
where $\mathcal{L}^{\text{R-T}}$, $\mathcal{L}^{\text{R-P}}$, and $\mathcal{L}^{\text{I-S}}$ are the loss functions for region-tag alignment pre-training, region-noun phrase alignment pre-training, and image-sentense alignment pre-training. We set m as 1 in our final implementation.

\begin{table*}[!ht]\centering
\small
\begin{tabular}{l|c|c|c|ccc|c}\toprule
\multirow{2}{*}{ Model } &VQA2 &NLVR2 &VE & \multicolumn{3}{c|}{RefCOCO+} & \multirow{2}{*}{ Meta-Ave } \\
&Test-Dev &Test-P &Test &Dev &TestA &TestB & \\\cmidrule{1-8}
ViLBERT\cite{lu2019vilbert} &70.6 &- &- &72.3 &78.5 &62.6 & - \\
VL-BERT\cite{su2019vl} &71.2 &- &- &71.6 &77.7 &61.0 & - \\
$\text{UNITER}_{\text{CC}}$\cite{chen2020uniter} &71.2 &- &- &72.5 &79.4 &63.7 & - \\
VisualBERT \cite{li2019visualbert,li2020unsupervised} &70.9 &73.9 &- &73.7 &79.5 &64.5 & - \\
Aligned VLP &\textbf{72.5} &\textbf{75.9} &\textbf{78.7} &\textbf{82.1} &\textbf{86.6} &\textbf{75.0} & \textbf{77.3} \\
\midrule
Base &70.1 &51.2 &73.2 &69.4 &74.8 &60.3 & 65.9 \\
\uvisualbert \cite{li2020unsupervised} &71.8 &53.2 &76.8 &78.2 &83.6 &69.9 & 70.0\\
$\text{\ModelName}_{\text{CC}}$ &\textbf{72.1} &\textbf{73.4} &\textbf{77.3} &\textbf{80.3} &\textbf{85.5} & \textbf{73.7} & \textbf{75.8} \\
$\text{\ModelName}_{\text{BC}}$ &71.2 &67.1 &77.1 &79.7 &85.0 &72.7 & 73.8 \\
\bottomrule
\end{tabular}
\caption{Evaluation results on four V+L downstream tasks. Our model trained with un-aligned data ($\text{\ModelName}_{\text{CC}}$, $\text{\ModelName}_{\text{BC}}$) achieves comparable performance with the supervised model trained with aligned data (Aligned VLP). $\text{\ModelName}_{\text{CC}}$ and $\text{\ModelName}_{\text{BC}}$ also outperform {\uvisualbert } on nearly all tasks.}
\label{tab:main}
\end{table*}

\begin{table*}[!ht]\centering
\small
\begin{tabular}{l|c|c|c|ccc|c}\toprule
\multirow{2}{*}{V+L Alignment} &VQA &NLVR2 &VE &\multicolumn{3}{c|}{RefCOCO+} & \multirow{2}{*}{Meta-Ave}\\
&Test-Dev &Test-P &Test &Dev &TestA &TestB & \\\cmidrule{1-8}
$\text{\ModelName}_{\text{CC}}$ (R-T)  &71.7 &52.0 &75.6 &78.7 &83.3 &70.0 & 69.5\\
$\text{\ModelName}_{\text{CC}}$ (R-N)  &71.4 &69.4 &76.5 &77.4 &81.5 & 68.7 & 73.7 \\
$\text{\ModelName}_{\text{CC}}$ (I-S)  &71.6 &71.5 &76.8 &75.7 &80.3 &67.9 & 73.9\\
$\text{\ModelName}_{\text{CC}}$ (R-T + R-N) &71.9 &72.4 &76.4 &79.3 & 84.5 & 71.7 & 75.0 \\
$\text{\ModelName}_{\text{CC}}$ (R-T + R-N + I-S) &\textbf{72.1} & \textbf{73.4} &\textbf{77.3} &\textbf{80.3} & \textbf{85.0} &\textbf{73.7} & \textbf{75.8}\\
\bottomrule
\end{tabular}
\caption{Effect of cross-modal alignment on the three types of granularities: region-tag alignment(R-T), region-noun phrase alignment(R-N), and image-sentence alignment(I-S)}\label{tab:ablation_align}
\end{table*}


\section{Experiments}
In this section, we provide the detailed experimental set up to evaluate our proposed {\ModelName } against previous supervised and unsupervised VLP models. More specifically, we introduce our pre-training dataset, baselines, and our pre-training setting.

\subsection{Pre-training Datasets}
We prepare the un-aligned data under two different settings: (1) We use images and text separately from Conceptual Captions (CC) \cite{sharma2018conceptual} ignoring the alignment information; (2) We use images from Conceptual Captions (CC) \cite{sharma2018conceptual} and text from BookCorpus (BC) \cite{Zhu_2015_ICCV}. 
Setting (1) sets up a fair comparison with previous supervised methods by keeping the domain and the quality of training data consistent. Our proposed model trained in this setting is called \ModelName$_{CC}$. 
Setting (2) mimics a more realistic challenge where we have large-scale images and text data from different domains, in particular the text sources are not similar to captions of the images. \ModelName$_{BC}$ has been trained in this setting.

As introduced in section \ref{section:data_aug}, for each image we retrieve 5 text data points (captions from CC or sentences from BC) from the text corpus that are semantically similar to the detected objects in the image. 
This creates weakly-aligned image-text pairs for our pre-training models. 

\subsection{Baselines}
We compare the performance of our proposed {\ModelName } to the following baselines: 

\head{Base Model} VisualBERT that is initialized from BERT. It does not undergo any pre-training but is directly fine-tuned on the downstream tasks. 

\head{Supervised Pre-trained Models} Supervised pre-trained VLP models that are trained only on CC, including VILBERT\cite{lu2019vilbert}, VL-BERT\cite{su2019vl}, and UNITER\cite{chen2020uniter}. We also report the numbers on the Supervised VisualBERT implemented in \uvisualbert\cite{li2020unsupervised} that is trained on CC and an additional 2.5 Million text segments from BC. 
For fair comparison with our proposed method, we also introduce the aligned vision-language pre-training model (Aligned VLP) that is pre-trained on the 3M (image, caption) pairs from CC and 3M (image, object tag) pairs. 

\head{Unsupervised Pre-trained Models} 
{\uvisualbert } is pre-trained on individual image or text corpus in a round-robin fashion and captures the cross-modal alignment by using detected object tags as the anchor point. 
For fair comparison, we re-implemented this method to pre-train with the VinVL object features\cite{zhang2021vinvl} and BC. 
\subsection {Training Setup}\label{sec:training_setup}
Our transformer architecture consists of 12 layers of transformer blocks, where each block has 768 hidden units and 12 self-attention heads. 
We initialize the model from $\text{BERT}_{base}$ and pre-train for 20 epochs on their respective pre-training datasets with a batch size of 480. The region features for images are obtained from the pre-trained VinVL object detectors \cite{zhang2021vinvl}. We use Adam optimizer \cite{ADAM} with a linear warm-up for the first 10\% of training steps, and set the peak learning rate as 6e-5. After warm up, a linear-decayed learning-rate scheduler gradually drops the learning rate for the rest of training steps. 
All models were trained on 4 NVIDIA A100 GPUs, with 40GB of memory per GPU using MMF\cite{singh2020mmf}.
The pre-training takes 3 days.
We evaluate our pre-trained models on four downstream tasks: Visual Question Answering (VQA 2.0)\cite{anderson2018bottom}, Natural Language for Visual reasoning\cite{suhr2018corpus} ($\text{NLVR}^2$), Visual Entailment\cite{xie2019visual} (VE), and Referring Expression\cite{yu2016modeling} (RefCOCO+). 
% To validate our proposed sentence-image alignment pre-training, we also conduct a zero-shot evaluation with image-text retrieval task on Flickr30K\cite{young-etal-2014-image}. 
Detailed training settings for each task can be found in our supplementary material. 
\subsection{Experimental Results}
We first compare {\ModelName } to various supervised models that are pre-trained on CC and to the state-of-the-art unsupervised V+L pre-training method, {\uvisualbert } on the four downstream tasks. Besides reporting scores for each individual task, we also compute the meta-average score to reflect the overall performance across all tasks. 
The results are summarized in Table \ref{tab:main}.

\myparagraph{Compared to Base.} It is clear from Table~\ref{tab:main} that both \ModelName$_{CC}$ and \ModelName$_{BC}$ outperform the Base model by a large margin on all benchmarks. 


\myparagraph{Compared to Aligned VLP.} It also achieves better performance than existing supervised models like VilBERT\cite{lu2019vilbert}, which is potentially due to the usage of better visual regional features of VinVL~\cite{Zhang_2021_CVPR}. When compared to Aligned VLP, which is trained with the same architecture and visual features, our model is only slightly worse. 
This shows the effectiveness of our proposed pre-training curriculum which can learn comparable universal representation across vision and language as the supervised models without any parallel image-text corpus. 

\myparagraph{Compared to UVLP. }Our {\ModelName } also achieves consistently better performance than the previous UVLP method: \uvisualbert. 
This improvement shows how our proposed cross-modal alignment pre-training curriculum effectively bridges the gap across the two modalities.
In particular, our model outperforms {\uvisualbert } in the task of NLVR2 by more than 20\%. 
As NLVR2 is known to benefit more from image-sentence cross-modal alignment from previous supervised V+L pre-training research \cite{chen2020uniter}, this observation indicates that our model is able to capture the instance-level cross-modal alignment without parallel data. 
When {\ModelName } is trained on BC text and CC images \ie \ModelName$_{BC}$, it still achieves comparable or better performance than {\uvisualbert } except for VQA.  
The slight advantage  {\uvisualbert } has over \ModelName$_{BC}$ in VQA is potentially due the similar style between the VQA text and the pre-trained CC captions. 
However, this does not overshadow the overall better performance of {\ModelName}. 
It shows that our proposed method is more robust than {\uvisualbert } training on the uni-modal datasets collected from separate domains, which makes it more useful in practical settings.    
\begin{figure}[h!]
\centering
\includegraphics[width=0.8\linewidth]{figures/number_candidate_plot.pdf}
\caption{Meta average scores of non-parallel V+L pre-training with different number of retrieved candidate sentences.}
\label{fig:ablation_ncandidate}
\end{figure}


\begin{figure*}[h!]
\centering
\includegraphics[width=16cm]{figures/retrieved_positives.pdf}
\caption{Examples of retrieved text from both CC and BC. The covered grounded noun phrases in retrieved sentences are highlighted in green bar for positive examples.}
\label{fig:visualization}
\end{figure*}

\subsubsection{Ablation Study on Multi-Granular Alignment}
We conduct ablation study to verify the effectiveness of the three types of visual-language alignment for unsupervised V+L pre-training, namely region-tag alignment (R-T), region-noun Phrase alignment (R-N), and image-sentence alignment (I-S). We first evaluate each individual type of alignment to measure its usefulness for different downstream tasks. Then, we gradually add each type of alignment into the UVLP.  For this ablation study we pre-train {\ModelName } on CC images and text, and the results are summarized in Table \ref{tab:ablation_align}. 

From Table~\ref{tab:ablation_align}, we can see that aligning local regions to object tags (R-T) and noun phrases (R-N) are especially helpful for the task of RefCOCO+, which requires the model to understand specific objects that natural expressions describe. Meanwhile, aligning the image and sentence at instance-level (I-S) benefits NLVR2 and VE. Especially on NLVR2, the model that captures the global vision and language alignment \ModelName$_{CC}$ (I-S) obtains 19.5\% gain over the model that only learns the local alignments between regions and object tags \ModelName$_{CC}$ (R-T). This observation is consistent with previous research \cite{chen2020uniter}, where the performance of model on NLVR2 is boosted after introducing pre-training objectives that capture the cross-modal alignment in the image-text pairs. Our results demonstrate that even with just weakly-aligned sentences, we can still effectively learn the instance-level cross-modal alignment. 
% \mingyang{Maybe also show zero-shot Image-Sentence Alignment performance to further proves the learned cross-modal alignment on vision and language;} 
Combining the region-tag and region-noun phrase alignment (R-T+R-N) for UVLP, we observe that these two types of grounding and matching compensate each other. \ModelName$_{CC}$ (R-T+R-N) shows a marginal but consistent improvement over models that only learn a single type of local region-to-language alignment (R-T, R-N). After adding object-phrase level alignment we can further improve the performance on NLVR2 and VE, which gives us our best performing model \ModelName$_{CC}$ (R-T + R-N + I-S). 

\subsubsection{Ablation Study on Number of Retrieved Candidates}
We conduct experiments to verify the impact the number of retrieved candidate text for each image has on the performance. We create three variants of pre-training corpus, where the number of retrieved candidate are 1, 5, and 10 based on the rank of the similarity of each candidate text to the query image's detected object tags. The candidate text is sampled from CC. We pre-train our {\ModelName } model with only the pre-training objectives to capture the sentence-image alignment (I-S). For each variant of pre-training corpus, we train the model for the same number of steps. We compute the meta average score for the three resulting pre-trained models and visualize them in Fig.~\ref{fig:ablation_ncandidate}. 

Fig.~\ref{fig:ablation_ncandidate} shows that retrieving more than one candidate text for an image greatly benefits the pre-trained model to learn a better joint representation between vision and language, demonstrated by stronger performance in the downstream tasks. 
We suspect this is because the closeness between the candidate caption and the detected object tags in language embedding space does not always mean a better alignment between the candidate caption and the image. A better and more semantically similar caption candidate for the image could be found in the other caption candidates. However, when we increase the number of candidate captions to 10, we observe a slight drop on the overall performance compared to the model that is pre-trained on corpus with 5 candidate captions. This indicates that having too many candidate captions to form the weakly-aligned pairs with the query image for V+L pre-training may also introduce unnecessary noise. Hence, we set the number of retrieved captions in our experiments to 5.
 

\subsubsection{Visualization}
To get a sense of the quality of the retrieved sentences, we show some examples of retrieved text from both CC and BC in Fig.~\ref{fig:visualization}. The first row demonstrates a positive case of retrieved captions from CC, where we observe a good coverage of the objects in the image such as ``young woman", ``sofa", and ``couch" in the top retrieved sentences. Similarly, our retrieval method can also retrieve good candidates from BC that describe many visual objects from the image as depicted in row 2. This observation demonstrate the effectiveness of picking candidates based on their closeness to the object list in the language embedding space. 

We also compare the text-to-image attention between the pre-trained \uvisualbert~and \ModelName~without task-specific fine-tuning as~\cite{chen2020uniter,Zhou_2021_CVPR}.
As shown in Fig.~\ref{fig:attention}, we feed into the models an aligned pair whose caption is ``young woman seated on the beach", we visualize the local cross-modality alignment between regions and tokens.
we found our full model \ModelName~can better attend on the described regions, showing higher-quality alignment is learned through the proposed pre-training.
More visualizations are in the supplementary file.



\begin{figure}[h!]
\centering
\includegraphics[width=0.8\linewidth]{figures/attention.pdf}
\vspace{-0.5cm}
\caption{Text-to-image attention given the aligned pair whose caption is ``young woman seated on the beach".}
\label{fig:attention}
\end{figure}






\section{Conclusion}
We propose an unsupervised vision-and-language pre-training approach via retrieval-based multi-granular alignment to learn strong vision and language joint representations with un-aligned text and image sources. We introduce two core designs of our proposed approach: (1) construct a retrieval-based weakly-aligned image-text corpus. (2) multi-granular pre-training objectives to enable the model to capture the cross-modal alignment at different granularity levels. 
Our experiments show that our model can consistently outperform the previous state-of-the-art unsupervised pre-trained models and can achieve similar performance as the fully-aligned pre-trained models. 
\vspace{-0.4cm}
\paragraph{Limitations:} 
As we only consider the detected object list to retrieve the candidate sentences, the retrieved sentences often do not cover other visually grounded information compared to the ground truth captions.
Besides, the detected object tags are often those general concepts lacking diversity. 
% For example, in row 3 in Fig.~\ref{fig:visualization}, the stadium in the image is detected as ``building", which leads to the failure of retrieving any candidate that covers this object. Another limitation is that each object in the object list contributes equally during the retrieval. 
% This is sub-optimal as different objects should be weighted differently based on their detection confidence and visual importance. 
% Otherwise, the retrieved candidate sentences might focus on less important objects. For example in row 3, the retrieved sentences only cover ``tree" or ``water" from the object list, which leads to semantic discrepancy between the retrieved sentences and the image. 
Our retrieval results and in turn our pre-trained models could be affected by such limitations.
We hope to address the issue by learning a Siamese network between visual concepts and sentence for better retrieval and exploiting even larger uni-modal datasets to increase the diversity in the future research.
\vspace{-0.5cm}
\paragraph{Societal Impact:} 
The models are trained on the public datasets widely used in the community.
However, these datasets are known with biases, which may in turn affect our model predictions.
We do not recommend relying on the models to make real-world decisions.
%%%%%%%%% REFERENCES
{\small
\bibliographystyle{ieee_fullname}
\bibliography{egbib}
}

%%%%%%%%% Appendix
\clearpage
\appendix

\appendix

\section{Appendix}

\paragraph{Data Analysis.}
In Table~\ref{tab:dataset_comparison} we show a comparison of \DsetName~with existing moment retrieval datasets and related video and language datasets. 
Compared to other moment retrieval datasets, \DsetName~is significantly larger in scale, and comes with query type annotations that allows in-depth analyses for the models trained on it.
Besides, it is also the only moment retrieval dataset with multilingual annotations, which is vital in studying the moment retrieval problem under the multilingual context. 
Compared to the existing multilingual video and language datasets, \DsetName~is unique as it has a more diverse set of context and annotations, i.e., dialogue, query type, and timestamps.


\paragraph{Training and Inference Details.}
In Figure~\ref{fig:mxml_overview} we show an overview of the \ModelName~model.
We compute video retrieval score as:
\begin{align}
    s^{vr} = \frac{1}{2}\sum_{m \in \{v, s\}} \mathrm{max}(\frac{H^{m}_{vr}}{\left\Vert H^{m}_{vr}\right\Vert} \frac{\boldsymbol{q}^{m}}{\left\Vert \boldsymbol{q}^{m}\right\Vert}).
\end{align}
The subscript $lang \in \{en, zh\}$ is omitted for simplicity.
It is optimized using a triplet loss similar to main text Equation (1).
For moment retrieval, we first compute the query-clip similarity scores $S^{q,c} \in \mathbb{R}^{l}$ as:
\begin{align}
    S^{q,c} = \frac{1}{2}(H^{s}_{mr}\boldsymbol{q}^{s} + H^{v}_{mr}\boldsymbol{q}^{v}).
\end{align}
Next, we apply Convolutional Start-End Detector (ConvSE module)~\cite{lei2020tvr} to obtain start, end probabilities $P_{st}, P_{ed} \in \mathbb{R}^{l}$. These scores are optimized using a cross-entropy loss. The single video moment retrieval score for moment $[t_{st}, t_{ed}]$ is computed as:
\begin{align}
    s^{mr}(t_{st}, t_{ed}) = P_{st}(t_{st}) P_{ed}(t_{ed}), \, t_{st} \leq t_{ed}.
\end{align}

\noindent
Given a query $q_i$, the retrieval score for moment [$t_{st}$:$t_{ed}$] in video $v_j$ is computed following the aggregation function as in~\cite{lei2020tvr}:
\begin{align}
    s^{vcmr}&(v_j,t_{st}, t_{ed}|q_i) = \nonumber \\ &s^{mr}(t_{st}, t_{ed})\mathrm{exp}(\alpha s^{vr}(v_j|q_i)),
\end{align}


\noindent
where $\alpha{=}20$ is used to assign higher weight to the video retrieval scores.
The overall loss is a simple summation of video and moment retrieval loss across the two languages, and the language neighborhood constraint loss. 








\paragraph{Implementation Details.}
\ModelName~is implemented in PyTorch~\cite{paszke2017automatic}.
We use Adam~\cite{kingma2014adam} with initial learning rate 1e-4, $\beta_1{=}0.9$, $\beta_2{=}0.999$, L2 weight decay 0.01, learning rate warm-up over the first 5 epochs. 
We train \ModelName~for at most 100 epochs at batch size 128, with early stop based on the sum of R@1 (IoU=0.7) scores for English and Chinese.
The experiments are conducted on a NVIDIA RTX 2080Ti GPU. 
Each run takes around 7 hours.



\begin{table}[!t]
\centering
\small
\setlength{\tabcolsep}{3.5pt}
\renewcommand{\arraystretch}{1.05}
\scalebox{1.0}{
\begin{tabular}{lcccc}
\toprule
& \multicolumn{2}{c}{English R@1} & \multicolumn{2}{c}{Chinese R@1} \\  \cmidrule(l){2-3} \cmidrule(l){4-5}
Setting & IoU=0.5 & IoU=0.7 & IoU=0.5 & IoU=0.7 \\
\midrule
unseen & 1.68 & 0.79 & 1 & 0.54 \\
seen & 4.82 & 2.79 & 4.18 & 2.32 \\
\bottomrule
\end{tabular}
}
\caption{\ModelName~performance on the \DsetName~val split \textit{Friends} examples, in both \textit{unseen} and \textit{seen} settings. 
}
\label{tab:ablation_unseen}
\end{table}



\begin{figure*}[!t]
  \includegraphics[width=\linewidth]{res/mXML_overview.pdf}
  \caption{
  \ModelName~overview. For brevity, we only show the modeling process for a single language (Chinese). The cross-language modifications, i.e., parameter sharing and neighborhood constraint are illustrated in Figure~\ref{fig:tvrm_encoding}. This figure is edited from the Figure 4 in~\citep{lei2020tvr}. 
  }
  \label{fig:mxml_overview}
\end{figure*}



\begin{table*}[ht]
\centering
\small
\setlength{\tabcolsep}{5pt}
\scalebox{0.96}{
\begin{tabular}{lcccccc}
\toprule
Dataset & Domain & \#Q/\#videos & Multilingual & Dialogue & QType & Timestamp \\
\midrule
\bf QA datasets with temporal annotation &  &  &  &  &  &  \\
TVQA~\cite{Lei2018TVQALC} & TV show & 152.5K/21.8K & - & \checkmark & - & \checkmark \\
How2QA~\cite{li2020hero} & Instructional & 44K/22K & - & \checkmark & - & \checkmark \\
\bf Multilingual video description datasets &  &  &  &  &  &  \\
MSVD~\cite{chen2011collecting} & Open & 70K/2K & \checkmark & - & - & - \\
VATEX~\cite{wang2019vatex} & Activity & 826K/41.3K & \checkmark & - & - & - \\
\bf Moment retrieval datasets &  &  &  &  &  &  \\
TACoS~\cite{regneri2013grounding} & Cooking & 16.2K/0.1K & - & - & - & \checkmark \\
DiDeMo~\cite{anne2017localizing} & Flickr & 41.2K/10.6K & - & - & - & \checkmark \\
ActivityNet Captions~\cite{Krishna2017DenseCaptioningEI} & Activity & 72K/15K & - & - & - & \checkmark \\
CharadesSTA~\cite{gao2017tall} & Activity & 16.1K/6.7K & - & - & - & \checkmark \\
How2R~\cite{li2020hero} & Instructional & 51K/24K & - & \checkmark & - & \checkmark \\
TVR~\cite{lei2020tvr} & TV show & 109K/21.8K & - & \checkmark & \checkmark & \checkmark \\
\midrule
\DsetName & TV show & 218K/21.8K & \checkmark & \checkmark & \checkmark & \checkmark \\ 
\bottomrule
\end{tabular}
}
\caption{
Comparison of~\DsetName~with related video and language datasets.   
}
\label{tab:dataset_comparison}
\end{table*}

 

\paragraph{Generalization to Unseen TV shows.} 
To investigate whether the learned model can be transferred to other TV shows, we conduct an experiment by using the TV show `\textit{Friends}' as an `\textit{unseen}' TV show for testing, and train the model on all the other 5 TV shows. 
For comparison, we also include a model trained on `\textit{seen}' setting, where we use all the 6 TV shows including \textit{Friends} for training. 
To ensure the models on these two settings are trained on the same number of examples, we downsample the examples in the \textit{seen} setting to match the \textit{unseen} setting.
The results are shown in Table~\ref{tab:ablation_unseen}.
We notice our \ModelName~achieves a reasonable performance even though it does see a single example from the TV show \textit{Friends}.
Meanwhile, the gap between \textit{unseen} and \textit{seen} settings are still large, we encourage future work to further explore this direction.


\paragraph{Prediction Examples}
We show \ModelName~prediction examples in Figure~\ref{fig:pred_examples}. 
We show both Chinese (\textit{top}) and English (\textit{bottom}) prediction examples, and correct (\textit{left}) and incorrect (\textit{right}) examples.


\begin{figure*}[!t]
  \includegraphics[width=\linewidth]{res/pred_examples.pdf}
  \caption{
  Qualitative examples of \ModelName. \textit{Top:} examples in Chinese. \textit{Bottom:} examples in English. \textit{Left:} correct predictions. \textit{Right:} incorrect predictions.
  We show top-3 retrieved moments for each query. \textcolor{salmon}{salmon bar} shows the predictions, \textcolor{ForestGreen}{green box} indicates the ground truth.
  }
  \label{fig:pred_examples}
\end{figure*}


\end{document}
