\newif\ifDRAFT
\DRAFTtrue

\listfiles
\documentclass[prd ,twocolumn ,secnumarabic,dvips
,amssymb, amsmath,nobibnotes, aps, prd,superscriptaddress]{revtex4-1}
\usepackage{graphicx}

\ifDRAFT
	\usepackage[pagewise]{lineno}
	\usepackage[printwatermark]{xwatermark}
	\usepackage{xcolor}
	\usepackage{tikz}
\fi

\expandafter\ifx\csname package@font\endcsname\relax\else
 \expandafter\expandafter
 \expandafter\usepackage
 \expandafter\expandafter
 \expandafter{\csname package@font\endcsname}%
\fi

\if0
\documentclass[aps,prd,twocolumn,amsfonts,amssymb,amsmath,showpacs,letterpaper,superscriptaddress,nofootinbib,altaffilletter]{revtex4-1}

\usepackage{hyperref}

\usepackage{color}
\usepackage{graphicx}
\usepackage{latexsym}
\usepackage{float}
\usepackage{amsmath}
\usepackage{amssymb}
\usepackage{multirow}
\usepackage{ifthen}
\usepackage{slashed}
\fi



\begin{document}

\title{Construction of KAGRA: an Underground Gravitational Wave Observatory}
%%%%%%%%%%%%%%%%%%%%% author.tex %%%%%%%%%%%%%%%%%%%%%%%%%%%%%%%%%%%
%
% sample root file for your "contribution" to a contributed volume
%
% Use this file as a template for your own input.
%
%%%%%%%%%%%%%%%% Springer %%%%%%%%%%%%%%%%%%%%%%%%%%%%%%%%%%


% RECOMMENDED %%%%%%%%%%%%%%%%%%%%%%%%%%%%%%%%%%%%%%%%%%%%%%%%%%%
\documentclass[graybox]{svmult}

% choose options for [] as required from the list
% in the Reference Guide

\usepackage{mathptmx}       % selects Times Roman as basic font
\usepackage{helvet}         % selects Helvetica as sans-serif font
\usepackage{courier}        % selects Courier as typewriter font
\usepackage{type1cm}        % activate if the above 3 fonts are
                            % not available on your system

\usepackage{makeidx}         % allows index generation
\usepackage{graphicx}        % standard LaTeX graphics tool
                             % when including figure files
\usepackage{multicol}        % used for the two-column index
\usepackage[bottom]{footmisc}% places footnotes at page bottom

\usepackage{booktabs}

\usepackage[caption=false]{subfig}

\usepackage{geometry} \geometry{margin=3.5cm}

%\usepackage[table]{xcolor}

% see the list of further useful packages
% in the Reference Guide

\makeindex             % used for the subject index
                       % please use the style svind.ist with
                       % your makeindex program

%%%%%%%%%%%%%%%%%%%%%%%%%%%%%%%%%%%%%%%%%%%%%%%%%%%%%%%%%%%%%%%%%%%%%%%%%%%%%%%%%%%%%%%%%

\begin{document}

\title*{Epidemic Contact Tracing with Smartphone Sensors}
% Use \titlerunning{Short Title} for an abbreviated version of
% your contribution title if the original one is too long
\author{Khuong An Nguyen, Zhiyuan Luo, Chris Watkins \\ \hfil \\ }
% Use \authorrunning{Short Title} for an abbreviated version of
% your contribution title if the original one is too long
\institute{Khuong An Nguyen \at School of Computing, Engineering \& Mathematics, University of Brighton, UK \\ \email{K.A.Nguyen@bton.ac.uk}
\and Zhiyuan Luo, Chris Watkins \at Computer Science Department, Royal Holloway University of London, UK \\ \email{Zhiyuan.Luo@rhul.ac.uk, C.J.Watkins@rhul.ac.uk}}


%
% Use the package "url.sty" to avoid
% problems with special characters
% used in your e-mail or web address
%
\maketitle

% Instead the app uses three metrics to work out a risk score: the proximity of the devices; the length of time the phones continued to detect each other; how infectious the person with the coronavirus was judged to be, based on how close the meeting happened to when they noticed their symptoms

% maximise the utility of contact tracing information while minimising the more general risk to privacy.
% Alice should know that she has been in contact with somebody (in fact Bob) without knowing that her contact is indeed Bob or indeed knowing anything else at all about Bob. Her ignorance of the latter two facts is as important on privacy grounds as her knowledge of the first on grounds of safety. Contact information, however, can connect with personal information in two ways.

% authority does not reveal the entire trajectory of an infected user to all non-infected users.To avoid such undesirable disclosure, the authority has to know the trajectories of all users, whether infected or not

% Contact tracing provides a mechanism of identifying individuals with a high likelihood of previous exposure to a contagious disease, allowing additional precautions to be put in place to prevent continued transmission.

% the problem of comparing a set of location histories held by two parties to determine whether they have come within some threshold distance while at the same time maintaining the privacy of the location histories.
% With the availability of users' location, there is a growing concern that private data can be used by an unauthorized entity in ways that infringe the individual right to privacy, for example by proling user's behavior, targeting the user by stalking and spamming
% any scheme that is used has to be both scalable and effective. For example, schemes that rely on a smartphone application that has to be installed by users on their smartphones can only be effective if a significant proportion of the population has a smartphone, takes an active step to install the app, and allows it to run continuously.

% This proximity and duration information is stored in an encrypted form on a person’s phone for 21 days on a rolling basis. No location data is collected.

% The system relies on short-range Bluetooth signals emitted from people’s smartphones. These signals represent random strings of numbers, likened to “chirps” that other nearby smartphones can remember hearing. If a person tests positive, he/she can upload the list of chirps the person’s phone has put out in the past 14 days to a database. Other people can then scan the database to see if any of those chirps match the ones picked up by their phones. If there’s a match, a notification will inform that person that they may have been exposed to the virus, and will include information from public health authorities on next steps to take.

% Alice and Bob meet each other for the first time and have a 10-minute conversation. Their phones exchange anonymous identifier beacons (which change frequently).
% Few days later, Bob is positively diagnosed for COVID-19 and enters the test result in an app from a public health authority. With Bob’s consent, his phone uploads the last 14 days of keys for his broadcast beacons to the cloud.

% when there were only a few cases, contact tracing could be done manually. With hundreds to thousands of cases surfacing in some cities, contact tracing has become much more difficult

% Countries have been employing a variety of means to enable contact tracing. In Israel, legislation was passed to allow the government to track the mobile-phone data of people with suspected infection. In South Korea, the government has maintained a public database of known patients, including information about their age, gender, occupation, and travel routes. In Taiwan, medical institutions were given access to patients travel histories. and authorities track phone location data for anyone under quarantine

% By decentralising the service, the group expects to increase the security of the app and the protection of personal data, avoiding any data breach that would lead to reidentification of end-users. At the same time, they seek for data minimisation by collecting only anonymous information about infected people and discarding any tracing information from non-infectious individuals.

% COVID-19 Contact Tracing and Data Protection Can Go Together paper :
% Why would such an app be useful at all ? The data so far suggest, however, that about half of all infections occur before the dreaded symptoms of fever or a persistent cough appear. It is therefore not enough to quarantine people only after they show symptoms. To reduce infections, one would need to act quickly when a person is diagnosed with COVID-19 to find all people this person was in close proximity with. The risk of infection is highest if one has been within 1.5-2 m of an infected person for at least 10-15 minutes. If it could be determined who had been in such close proximity, then one could ask freshly infected, presymptomatic people to self-isolate and thus stop them from infecting more people. However, such fast contact tracing is not possible manually. Only a digital, largely automatic solution would help.

% many countries have started tracking their citizens’ phones and using location data to monitor the spread of the virus to enforce both lockdown and early isolation restrictions as well as most prominent example of this is China. However, countries like Israel [8] or South Korea [9] use location data as well

% A contact tracing system can be set up in a way that would allow for most data processing to happen locally on users’ mobile phones rather than on a central server. Only the notification of users who have been in contact with an infected person would need to be coordinated centrally. Even in this case, the necessary data could be processed in a way that would effectively preclude the central server from identifying users. The system would also not require collecting any location data.

% The fundamental idea is simple: it does not matter where people get in contact with an infected person. Be it on the bus or at work—what matters is proximity to a contagious person. This means that particularly sensitive location data, such as GPS or radio cell data, is actually neither necessary nor useful.

% One example of this would be contact tracing based on “contact points” as suggested by Yasaka et al [10]. QR Code.

% For example, every app installation in Singapore is linked with the user’s telephone number, making the user identifiable—something that is not strictly necessary

% Moreover, even digital contact-tracing systems need users to cooperate (by installing the app and carrying their phones with them) for any chance of success. Consequently, the effectiveness of any contact-tracing system depends on public support A representative survey [11] across the United States, United Kingdom, Germany, Italy, and France shows that about 70% of respondents would install an app like the one described above on their phones. Only if people trust a system—because it does not spy on them—will the system find broad support in the population


% For instance, in the recently released BlueTrace protocol used by the Singapore Government [34], users are guaranteed privacy from each other, but this model places complete trust in certain operating entities for protecting user information.

% In our model, the health authorities maintain a database of tokens corresponding to users which have been diagnosed with the disease. The user’s tracing app periodically checks an untrusted server to check if the user is potentially at risk in such a way that the server cannot deduce any information about the user which is not implied by the desired functionality and the user learns no information beyond whether they may have been exposed to the disease.

% TraceSecure paper:
% While the emphasis during the pandemic should be on stopping the spread, if users are reluctant to use contact tracing apps due to privacy concerns they will be rendered useless. In particular, adoption of a contact tracing apps must have a higher \transmission rate" than the virus itself in order to be effective
% At present the main case study is the TraceTogether app developed and deployed in Singapore. TraceTogether provides security between users, but relies on the assumption that the government is honest and trusted. The government learns everyone who is infected and is able to directly contact (by phone) those who may be at risk due to exposure.

% Some work has proposed solutions that do not require users to put complete trust in a third party. Our rst solution, presented in Section 3, is based on sending messages between stakeholders in the system. We have two versions of this method requiring two or three separate non-colluding parties who administer the system. One of these parties is the health care provider (for example the NHS in the UK) and the others we envision as separate government run servers.
% Instead we assume that there exists a mechanism for nearby phones to exchange short tokens if the devices come within 2 meters of each other, the estimated radius within which viral transmission is a signicant risk. Bluetooth Low Energy (BLE) features a \broadcast mode", often used for advertising, which can be used to send unsolicited messages to all BLE-enabled devices within range. Within the typical operational range of 10 meters, BLE can be used to transmit a 21-byte payload and allows proximity estimation based on the Received Signal Strength (RSS). Within a metre the positioning uncertainty can be as little as a few centimetres.
% The TraceTogether app [3] was developed by the Singapore Ministry of Health. It uses the idea of Bluetooth contact tracing outlined in Section 1.1. When devices are \in contact" the devices exchange random strings as tokens, these tokens act as a UID for the device for a set period of time. When a user is diagnosed with COVID-19 the government requires them to release their list of collected tokens | it is against the law not to cooperate [1]. The government also has a database that links the tokens to phone numbers and identities, allowing them to directly contact those who have been in contact with the infected user. In particular, there is no privacy for users who become infected. The government learns all contacts an infected user has had which is not desirable. The app does however oer privacy for users with respect to passive snoopers and other users. This is realised by allowing the user to generate a fresh token for each new time interval. To the best of our knowledge this is every 30 minutes.

%Snoopers are passive adversaries that capture communications between users and aim to learn a detailed view of their activities. This is primarily addressed with techniques that obfuscate the relationship between each user and the pseudonyms that they use when interacting with a service, The Singaporean TraceTogether contact tracing app provides privacy from snoopers in the form of time-limited pseudonyms which obfuscate each users interactions over any period in which more than a single pseudonym is used.

%Ideally, contacts would learn only whether they have a risk of being infected and would not learn which of their contacts had exposed them. Users are directly contacted by the authorities to inform them of their infection status and so learn nothing more than the single bit which is necessary.
% The two main techniques for privacy from the government used in TraceTogether are the decentralisation of data storage and the principle of data minimisation. Firstly, user contact is recorded by each mobile device running the app and is not routinely transmitted to the government. Secondly TraceTogether does not collect any user location data. Despite these steps, the government does learn the personal identity of any user that may have risked infection.

% including high death rates, if the virus is permitted to run its course without any intervention or response.

% Contact tracing is a well-studied method for curbing the spread of infectious diseases. It requires modeling the potential transmission routes through which an individual could infect the others [10]. This enables authorities in isolating the suspected and confirmed cases for further observation or treatment. This eases the load on already overstretched infrastructure. At some point nations are expected to ease in social distancing as a prolonged lockdown has adverse effects on economy and society. Further there is no clarity about availability of a vaccine for COVID-19 for citizens as development, trials and approvals from concerned authorities usually take time. This relaxation in lockdown can be facilitated by an effective contact tracing solution.

% Absolute location of the individual based on device GPS data, cell phone towers, Wi-Fi routers, historical location data from a third-party service provider, like Google.
% Relative location with respect to other individuals in proximity by utilizing device Bluetooth data.
% Related work in Privacy Guidelines for Contact Tracing Applications paper

% We conclude that viral spread is too fast to be contained by manual contact tracing, but could be controlled if this process was faster, more efficient and happened at scale.

% Several communities and nations seeking to minimize death tolls from COVID-19, are resorting to mobile-based, contact tracing technologies as a key tool in mitigating the pandemic. Harnessing mobile computing technologies is an obvious means to dramatically scale-up conventional epidemic response strategies to do tracking at population scale. However, straightforward and well-intentioned contact-tracing applications can invade personal privacy

% Positively tested citizens are asked to reveal (voluntarily, or enforced via public health policy or by law depending on region) their contact history to public health officers. The public health ocers then inform other citizens who have been at risk to the infectious agent based on co-location,via some denition of co-location, supported by look-up or inference about locations. The citizens deemed to be at risk are then asked to take appropriate action (often to either seek tests or to quarantine themselves and to be vigilant about symptoms).

% suppose a malicious entity observes all Bluetooth signals sent from your phone. Then, you would not want this entity to report you as positive. This attack is not possible as the seed uniquely identies your broadcasts and remains unknown to the attacker, unless the attacker is able to successfully break the underlying cryptographic mechanism, which is unlikely to be possible.
% It is possible for a malicious party to simultaneously record broadcasts at multiple dierent locations, including those that the positive citizen visited. Using these recordings, the malicious party could infer where the positive citizen was. The times at which the citizen visited these locations can also be inferred.


\abstract{Contact tracing is widely considered as an effective procedure in the fight against epidemic diseases. However, one of the challenges for technology based contact tracing is the high number of false positives, questioning its trust-worthiness and efficiency amongst the wider population for mass adoption. To this end, this paper proposes a novel, yet practical smartphone-based contact tracing approach, employing WiFi and acoustic sound for relative distance estimate, in addition to the air pressure and the magnetic field for ambient environment matching. We present a model combining 6 smartphone sensors, prioritising some of them when certain conditions are met. We empirically verified our approach in various realistic environments to demonstrate an achievement of up to 95\% fewer false positives, and 62\% more accurate than Bluetooth-only system. To the best of our knowledge, this paper was one of the first work to propose a combination of smartphone sensors for contact tracing.}

\vspace{1em}
\textbf{\textit{Keywords---} contact tracing, Covid-19, smartphone sensors.}

% Idea of the paper is implementing contact tracing without explicit location related data (e.g. GPS), reducing the number of false positives.

% The focal point of the paper is probabilistically determine if two users are in infectious range. Although there are other important parameter such as day of incubations, health condition, is wearing protective gear, we leave them as future work.

% Test feasibility of proposed approach

% Weighted average sensor fusion approach:
% Each sensor is given a weight, distributed evenly at the beginning. A binary threshold decides if the sensor is reliable in current condition (e.g. if too noisy), if not reliable, its weight is halved. 

% talk about off-line tracing works


% Motivation of work
\section{Introduction}
During any viral outbreak, people who have been in close contact with a contagious victim, are at risk of being infected themselves. Therefore, being able to detect such `contacts' early, correctly, and effectively, is critical to manage and suppress the disease. In the past outbreaks (e.g. SARS, Ebola, Swine flu, etc.), contact tracing has proven to be one of the most vital instruments for public health experts. However, as modern viruses (e.g. Covid-19) have evolved to become far deadlier and more infectious, conventional contact tracing approaches are urgently in need to be revamped by modern technology.

In the past decade, the continual proliferation of smartphones has changed the consumers' behaviour. Globally, more than 3.5 billion people own a smartphone, and in the UK alone, more than 94\% of adults have one\footnote{https://www.statista.com/statistics/300378/mobile-phone-usage-in-the-uk - last accessed in 5/2020.}. The smartphone may now be considered as an indispensable necessity to serve most people's daily routines, from essential communications (e.g. family and friend chatting, e-mail and text messaging), to information seeking (e.g. news reading, web surfing, route navigation), to entertainment purposes (e.g. music listening, photo taking, game playing), to health management (e.g. fitness tracking). Coupling with the fact that smartphones are powerful mini-computers equipped with a variety of sensors, this may well be the leverage for technology based contact tracing that we have been searching for.

Recently, Bluetooth Low Energy (BLE) technology has been viewed as the future prospect for automated contact tracing, thanks to its low power consumption and its relatively short communicating distance. However, from the application viewpoint, BLE bears two major issues, which were revealed by recording the raw BLE received signal strength (RSS) between two smartphones at fixed positions, with increasing distance away from each other in a straight line (see Figure~\ref{BLEproblem}). Visibility wise, two smartphones may still be reached at up to 20 metres indoors and 30 metres outdoors, because of the wireless signal multi-path. Distance wise, it is challenging to determine when two smartphones are 2 metres or 6 metres apart, based on the BLE RSS which varies strongly (much more indoors) because of its frequency hopping technique (to be discussed in Section~\ref{bluetooth}).
\begin{figure}[h]
    \centering
    \sidecaption
    \includegraphics[scale=.18]{Figures/BLEproblem.png}
    \caption{The two problems of BLE based contact tracing system. Firstly, two smartphones may still be visible at 30 metres apart outdoors. Secondly, it is hard to tell the difference between 2 to 6 metres as the RSS standard deviation (demonstrated by the shading areas) is noticeably high. This experiment was performed by measuring the BLE RSS between two phones at pre-determined fixed positions with increasing distance away from each other in a straight line.}
    % At 20 metres indoors and 30 metres outdoors, we started losing signals hence the cut-off point
    \label{BLEproblem}       
\end{figure}

To this end, this paper proposes a contact tracing system based on non-location non-intrusive smartphones sensors (no GPS or Cellular). Our approach combines 6 such sensors (i.e. barometer, Bluetooth, magnetometer, microphone, proximity, and WiFi), in one uniform model to detect the contacts of nearby phones. We present extensive empirical experiments to assess the feasibility and performance of such system.

\subsection{The contributions of this paper}
Overall, the paper makes the following novel contributions.
\begin{itemize}
    \item We propose a new concept of multi-smartphone-sensor system for contact tracing.
    \item We assess the feasibility of individual sensor with respect to contact tracing.
    \item We analyse the performance of the proposed system in three real-world testbeds.
\end{itemize}

To the best of our knowledge, this paper was one of the first work to propose a combination of smartphone sensors for contact tracing, and the first one to assess the feasibility and the performance of such approach.

The remaining of the paper is organised into six sections. Section 2 overviews other related work. Section 3 explains the concept of contact tracing, and its significance. So that, Section 4 will build on, to introduce the idea of contact tracing with multiple smartphone sensors. Then, Section 5 details the empirical experiments. Finally, Section 6 concludes our work and outlines future research.


\section{Related work}
As our work is dedicated to co-location smartphone-based contact tracing system, which does not involve any type of location database, we will mainly concentrate on similar approaches in the literature. At the end of the section, Table~\ref{literature} will briefly summarise these existing works.

BLE based contact tracing is perhaps the most popular approach, which was first proposed by the FluPhone project~\cite{yoneki2011fluphone} (note that they also mixed in GPS coordination data). Due to the Covid-19 pandemic, several independent BLE solutions were implemented around the world, often contracted by national agencies (see the MIT list of Covid tracker apps by nations\footnote{https://public.flourish.studio/visualisation/2241702/ - last accessed in 7/2020.}). One of the very first was BlueTrace by the Singaporean government, which senses nearby smartphones via BLE scanning and approximates the distance between them via the BLE RSS~\cite{bay2020bluetrace}, which was also the standard blueprint for most BLE based apps. Most recently, Google and Apple joined force to create the Google/Apple Exposure Notification (GAEN) API\footnote{https://www.apple.com/covid19/contacttracing - last accessed in 7/2020.}, aiming to streamline the BLE background scanning and recording process, as well as addressing the user privacy concern for contact tracing apps, set by the Decentralised Privacy-Preserving Proximity Tracing (DP-3T) protocol~\cite{hubaux2020decentralized}. Unfortunately, it is currently only available for government agencies to adopt, and there have been some concerns of its effectiveness due to sporadic scanning interval and over-simplified BLE scanning report~\cite{dehaye2020swisscovid,leith2020gaen}.

Camera based solution, which is arguably intrusive due to its mechanism of recording of the user's physical appearance (e.g. facial information), has also been proposed in the literature, although very few made it to real-world contact tracing. Some notable examples included the government approaches in India, South Korea, and China~\cite{preethika2020artificial,seetharaman74867177countries,tabari2020nations,vaughan2020tracking}.

GPS based solution, which records the user's precise position in latitude and longitude, has also been attempted, with the most recent work by Raskar et al.~\cite{raskar2020apps}, Wang et al.~\cite{wang2020new}, and the HaMagen app by the Israeli government\footnote{https://govextra.gov.il/ministry-of-health/hamagen-app - last accessed in 7/2020.}~\cite{sonmezdigital}, which build the GPS location trails of the registered users. Having access to this type of detailed location coordination would make contact tracing much easier and more accurate, should the privacy concern be properly addressed.

Magnetometer based approach was proposed in our previous work to track passengers on the public transports~\cite{nguyen2017co}. The two vital observations of this work were that the electric currents powering the rail lines would alter the on-board magnetic field in such a way that people in different carriages experience various non-deterministic measures; and the fact that passengers must share the same journey between at least two consecutive stations. Similar works utilising the magnetic field to detect colocation of the users were reported in other environments, especially indoors with a high degree of magnetic anomalies due to the building infrastructure~\cite{jeon2017judging,jeong2019smartphone,nguyen2019location}. 

WiFi based solution, which features dominantly in the indoor positioning research, has become more attractive for epidemic tracking, thanks to the increasing number of indoor and outdoor public WiFi Access Points (APs). Our previous work demonstrated that contact detection using pure WiFi RSS could closely match the accuracy of GPS (used as a reference) in the city centre, when at least 10 WiFi APs were around~\cite{nguyen2015feasibility}. Similar earlier works focused on the indoor WiFi APs to identify the proximity of the users and the devices~\cite{carlotto2008proximity,krumm2004nearme}.
\begin{table}[h]
	\caption{Overview of existing smartphone-based contact tracing approaches. Most of them are single technology based.}
	\centering
	\begin{tabular}{c c p{10cm}}
		\toprule
		\textbf{Sensor employed} & \textbf{References} & \textbf{Notes} \\
		\midrule
		BLE & \cite{bay2020bluetrace,dehaye2020swisscovid,leith2020gaen,yoneki2011fluphone} & This is the most popular approach for contact tracing right now. It employs BLE scanning for visibility discovery, and the signal strength to estimate the relative distance between devices. \\ \addlinespace[0.2cm]
		
		Camera & \cite{preethika2020artificial,seetharaman74867177countries,tabari2020nations,vaughan2020tracking} & Employing machine learning techniques such as facial recognition to track civilians. \\ \addlinespace[0.2cm]
		
		GPS & \cite{raskar2020apps,sonmezdigital,wang2020new} & Cross-checking the location trails of registered users to discover contacts. \\ \addlinespace[0.2cm]
		
		Magnetism & \cite{jeon2017judging,jeong2019smartphone,nguyen2019location,nguyen2017co} & Relying on the magnetic anomalies caused by ferrous metals in most building infrastructure to record the users' magnetic fingerprints. \\ \addlinespace[0.2cm]
		
		WiFi & \cite{carlotto2008proximity,krumm2004nearme,nguyen2015feasibility} & Utilising the public WiFi APs to detect contacts, where two co-located smartphones share the same set of APs at a particular moment. \\ \addlinespace[0.2cm]
		
		\bottomrule
	\end{tabular}
	\label{literature}
\end{table}



% Introduce the concept of general contact tracing, challenges
\section{What is contact tracing, and why is it essential ?}
This section introduces the core idea behind contact tracing and its roles in fighting epidemic diseases.

\subsection{What is contact tracing ?}
In essence, contact tracing in an epidemic is the process of identifying all potential victims, who have been in \textit{`close'} contact with an infected person, and iteratively tracing the victims' subsequent contacts in turn.

However, being in close proximity with a contagious individual does not strictly guarantee in getting the virus, as there are other factors such as the person's health condition, the protective equipment (e.g. facial mask) being used, the duration of exposure, the viral load, and many more. The overall hypothesis, adopted by the World Health Organisation (WHO), is that a `contact' is registered when two persons, one of whom is positively tested for Covid-19, are within 1 metre of each other, for at least 15 minutes\footnote{https://www.who.int/docs/default-source/coronaviruse/situation-reports/20200326-sitrep-66-covid-19.pdf - last accessed in 5/2020}.


\subsection{Why we need contact tracing ?}
Most viral infection diseases (e.g. Covid-19, SARS, Swine-flu etc.) share the common trait of being contagious during the incubation period (i.e. the time elapsed between being exposed to the virus, and when the first symptoms are shown) which may last for weeks without any apparent signs, during which the infectious patients unknowingly spread the virus to other victims~\cite{jiang2020does,leung2020difference}. Therefore, it is essential to pro-actively quarantine all potential patients who have been in prolonged contact with the confirmed virus host.

Contact tracing plays three critical roles in the fight against an epidemic. The first role is early treatment, that is, helping exposed patients to seek timely medical treatments, hence boosting successful recovery chance. The second role is transmission control, that is, informing potential victims to self-isolate, hence stopping the chain of onward transmission. The third role is epidemiology study, that is, gaining more insights (e.g. infectious origins, region, route, gender, etc.) of the epidemic, enabling a better strategy to fight the disease in the long term.
% condition for lifting lock-downs

\subsection{Three forms of contact tracing}
Conventionally, there are three mainstream forms of contact tracing.
\begin{itemize}
    \item \textbf{Interviews} are perhaps the oldest, but are still widely used in present days, where sick patients are requested to recall as many past contacts as they can~\cite{swanson2018contact}. Such interviews may be completed in person, via a form in the post, or on the internet. Nevertheless, human memory may prove too imprecise for such critical task, not to mention that such interviews are time-consuming and struggle to reach the wider population.
    
    \item \textbf{Narrowcasting} aims at a small-scale, concentrated part of the population (e.g. a town, a building), where the health authorities broadcast an announcement (e.g. on local radio, newspapers, bulletin boards, etc.) asking past visitors to carry out viral tests, and informing others to avoid such areas~\cite{kaligotla2016impact}. This approach allows the officials to quickly control an epidemic hotspot. However, since narrowcasting was meant for targeting a concentrated demographic region, it struggles to stay effective when the epidemic spreads across different regions at large-scale.
    
    \item \textbf{Real-time detection} addresses the issues of both approaches above, by recording the disease contact between citizens and managing the epidemic's progress at it happens. Such detection has mostly been attempted via ambient technology such as facial recognition (via Camera CCTV~\cite{hou2017human,wang2017tracking}), signal tracking (via the phone's cellular signal~\cite{aziz2016smart}), and location monitoring (via GPS data~\cite{chaix2018mobile,olu2016contact,stanley2020limits}).
    
    The common deterrent for all of these technologies is being intrusive, which may lead to the reluctance to comply and adopt by the citizens.
\end{itemize}

In the next section, we will introduce our approach using smartphone sensors that improves on the above real-time detection approach.


% Introducing the role of each sensor for contact tracing
\section{Contact tracing with smartphones sensors}
Having instigated the general concept of contact tracing, we may now introduce our approach in smartphone sensors based contact tracing. In doing so, we will outline the idea, process, assumptions, and challenge facing our approach.

% https://www.bbc.co.uk/news/technology-52355028
\subsection{The smartphone-based contact tracing model}
\label{sec:model}
In principle, our approach fits in the `real-time detection' category, as detailed above. The complete process involves three steps (see Figure~\ref{contacttracingprocedure}).
\begin{itemize}
    \item \textbf{Registration.} All participants download and start the app on their smartphones. It will register with the central server, then generate a unique, temporary ID representing the phone, which will be constantly refreshed after some period of time. The rationale for not using a permanent ID is to make it challenging for snoopers to identify the participants.
    
    \item \textbf{Contact detection.} The app detects nearby phones running the same app, and records such contacts locally on the phone. Both phones will exchange their temporary, current IDs. Although not being the main focus of this paper, the scanning interval should be configured to happen not too frequently (e.g. at 30 seconds or 1 minute interval) to reduce the power consumption and to avoid flooding the BLE and WiFi channels. The rationale for maintaining the contact list locally on each phone, and not centrally on a server is to avoid constant data transmission, preventing potential future data breach, and allowing the non-infectious participants to remain anonymous.
    
    \item \textbf{Infection report.} There are two models for infection report, namely the centralised model, and the decentralised model.
    
    In the centralised model, when a participant is diagnosed positive, his/her app reports the news to a central server, and releasing the locally stored contact list so far. It is clear that such infectious patients must reveal their identity to the server at this stage. The server subsequently informs all contacts from the infected user's list. This model is pioneered by the Pan-European Privacy-Preserving Proximity Tracing (PEPP-PT) group\footnote{https://www.pepp-pt.org - last accessed in 7/2020.}, which promotes standardised approaches (e.g. ROBust and privacy-presERving proximity Tracing protocol - ROBERT~\cite{castelluccia2020robert}) for strong European data privacy in accessing the user smartphone data. Its early adopters included the German and Italian governments~\cite{oxfordeuropean}.
    
    In the decentralised model, the diagnosed participant reports only his/her positive status to the server. S/he still needs to reveal the identity at this point. The server then updates the anonymised public list of infected users, which all participants should frequently check to verify their own status. This model is championed by Google and Apple (i.e. the Google/Apple Exposure Notification (GAEN) API\footnote{https://www.apple.com/covid19/contacttracing - last accessed in 7/2020.}), and the Decentralised Privacy-Preserving Proximity Tracing (DP-3T) initiative~\cite{hubaux2020decentralized}.
    
    %or frequently publish the anonymised list of infected users for all participants to verify by themselves in a decentralised fashion. 
    The benefit of the centralised setting is that the server's owner (e.g. government or health entities) may have a good overview picture of the epidemic's state, and all infected users may remain anonymous amongst other users (except to the server, and in the rare event of a participant being in contact with just a single user, who is later diagnosed positive). In the opposite, the benefit of the decentralised setting is that an infected user needs not expose his/her entire contact list to the server, giving the participants more control of their own data.
\end{itemize}
 
\begin{figure}[h]
    \centering
    \sidecaption
    \includegraphics[scale=.55]{Figures/contacttracingprocedure.png}
    \caption{The general steps of our proposed smartphone-based contact tracing procedure.}
    \label{contacttracingprocedure}       
\end{figure}

It is worth noting that we will focus mostly on detecting a contact between two nearby smartphones (i.e. Step 2 in Figure~\ref{contacttracingprocedure}). As such, other important properties such as secured protocol to exchange information, third party trusted server, user privacy, etc. are beyond the scope of this paper.

%there are other variations of the above contact tracing models

% Bluetooth low energy technology: The tracking itself would work as follows: as many people as possible voluntarily install the app on their phone. The app cryptographically generates a new temporary ID every half hour. As soon as another phone with the same app is in close proximity, both phones receive the temporary ID of the respective other app and record it. This list of logged IDs is encrypted and stored locally on the users’ phones. As soon as an app user is diagnosed with COVID-19, the doctor making the diagnosis asks the user to share their locally stored data with the central server. The server is not able to decrypt this information in a way that allows for the identification of individuals. However, it is able to notify all affected phones. This is because the server does not need any personal data to send a message to someone’s phone. The server only needs a so-called PushToken, a kind of digital address of an app installation on a particular phone. This PushToken is generated when the app is installed on the user’s phone. At the same time, the app will send a copy of the PushToken, as well as the temporary IDs it sends out over time, to a central server.


\subsection{Our assumptions}
In order for smartphones based contact tracing to be effective and reliable, the following assumptions are made:
\begin{itemize}
    \item \textbf{Substantial number of participants.} In common with any other technologies, the success of this approach relies first and foremost to the willingness of the wider population to engage. The first prerequisite is most citizens download the app to their smartphones. This assumption could be satisfied as many countries start to raise their citizen's awareness of the disease's seriousness. It may soon be the pre-condition to relax lock-downs and allow people to better protect themselves.
    
    \item \textbf{Carrying smartphones.} As this approach registers the human contact via the smartphones, it is vital that the devices are present along with the users when such contact happens. This assumption is mostly satisfied as most (if not all) smartphone-owners carry it with them whenever they are outdoors.
    
    \item \textbf{Accurate feedback.} When a user is confirmed positive, s/he must inform the app. Subsequently, ordinary users must not falsify such result. This assumption could be satisfied by official confirmation from the medical test results.
\end{itemize}

It goes without saying that the above assumptions would be made easier, should the proposed technology is shown to be effective with few false positives, and thus gaining trust amongst the users, in which this paper aims to address.


\subsection{Detecting a contact between two smartphones}
Correctly detecting a contact between two nearby smartphones is the key mechanism for this approach. We emphasise that the contact is detected in the proximity based \textit{relative position} (i.e. whether the phones are close or not?), and not the absolute coordination of the phones, which may be too sensitive to privacy issues. The detection happens in real-time, and only a binary (yes/no) decision, along with the nearby phones' temporary ID (as explained in Section~\ref{sec:model}) are recorded locally on the phone.

There are two approaches to register such contacts with smartphone sensors (see Figure~\ref{approaches}).
\begin{itemize}
    \item \textbf{Shared environment comparison.} When two phones are nearby, their respective sensor measures of the current environment should be similar.
    
    %\item \textbf{Distance related measure.}
    \item \textbf{Appearance sensing.} Some sensors have the ability to tell the existence of the same sensor type in other nearby phones.
\end{itemize}
\begin{figure}[h]
    \centering
    \sidecaption
    \includegraphics[scale=.60]{Figures/approaches.png}
    \caption{The two approaches for contact tracing with smartphone sensors. Two nearby phones can either sense their appearance and work out the relative distance; or compare the ambient environment they are sharing.}
    \label{approaches}       
\end{figure}


\subsection{Assessing the feasibility of employing smartphone sensors}
Currently, there are about 14 sensors in modern smartphones, with different functionalities. Table~\ref{comparisonsensors} compares some of their well-documented properties which are useful for contact tracing, namely the permission, power usage, and sampling rate.
\begin{itemize}
    % https://developer.android.com/guide/topics/permissions/overview#normal-dangerous
    \item \textbf{Sensor permission.}
    Permission wise, each sensor has different permission levels, based on how Android deems their threat to the user's privacy~\cite{mehrnezhad2019sensor}. In short, there are three relevant types of Android sensor permission for our purpose.
    \begin{itemize}
        \item \textbf{No permission.} No permission is needed to declare anywhere within the app or during run-time. Sensors in this group can silently access the sensors as they wish, which includes the accelerometer, gyroscope, magnetometer, ambient light, and proximity.
        
        \item \textbf{Normal permission.} These sensors pose little risk to the user's privacy. As such, they only need to be declared in the manifest file, and Android will ask the user just once during installation. Once agreed, the user has no way to refuse access in future runs.
        
        \item \textbf{Dangerous permission.} These sensors access high privacy user data (e.g. contact information, location data) or may affect the operation of other apps. For this group of sensors, the app must display a pop-up window explicitly asking the user for permission to access such information, when the app is first launched. Even after accepted, the user has total control to revoke such permission in future runs.
        
        %\item \textbf{Special permission.}
    \end{itemize}

    Generally speaking, for the contact tracing purpose, there is a trade-off between usability and privacy. Ideally, we would prefer sensors in the `no permission' or `normal permission' groups as they deliver seamless user experience (e.g. the app works in a simple click without convoluted pop-up messages). Nevertheless, the user should be made clear that such sensors (e.g. Bluetooth, WiFi) have the potential to infer their locations.

    It is worth noting that Android do not simply associate each sensor to a permission type, that is, one sensor may need several permissions, and one permission type may be shared amongst different sensors. For example, when an app needs Bluetooth access, it must specify \textit{BLUETOOTH} permission (for Bluetooth communications), and \textit{BLUETOOTH\_ADMIN} permission (to modify Bluetooth settings) in the `Normal permission' group; as well as the \textit{ACCESS\_FINE\_LOCATION} permission (to initiate a scan) in the `Dangerous permission' group, because Android consider that nearby Bluetooth devices information including the signal strength may be used to indirectly infer the user location.
    
    \item \textbf{Power consumption.} This metric reports the overall energy consumed by the app. In the context of contact tracing, this is an important factor to consider, since it prolongs the battery life between charging cycles, thus allowing more opportunities to detect potential disease contacts, as well as allowing the user to continue with other routines on the phone. An interesting correlation we spotted is that most no permission and normal permission sensors consume little power (see Table~\ref{comparisonsensors}).
    
    \item \textbf{Sampling rate.} This metric is the sensor's frequency in providing the latest measure. High sampling rate sensors are desirable to discover the smallest changes in the environment. Generally speaking, motion related sensors (i.e. accelerometer, gyroscope) have higher frequency than others to detect the quick changes in phone motions. At the other end of the spectrum are Bluetooth, GPS, and Cellular, which will keep scanning and reporting the results as they become available, until the process is interrupted.
\end{itemize}

\begin{table}[!ht]
	\caption{Summary of the relevant properties of 14 common sensors in most smartphones, sorted in alphabetical order. The data are surveyed from the LG G7 ThinQ phone.}
	\centering
	%% \tablesize{} %% You can specify the fontsize here, e.g.  \tablesize{\footnotesize}. If commented out \small will be used.
	\begin{tabular}{p{2cm}p{1.3cm}p{1.5cm}p{1.3cm}p{1.6cm}p{6cm}}%{ccccccC{6.5cm}}
		\toprule
		\textbf{Sensor} & \textbf{Measure}	& \textbf{Sampling} & \textbf{Power} & \textbf{Permission} & \textbf{Notes}\\
		& \textbf{unit}	& \textbf{rate (max)} & \textbf{usage} & \textbf{type} & \\
		\midrule
		Accelerometer & $m/s^2$ & 500 Hz & low & none & reporting the changing rate of the phone's velocity. \\ \addlinespace[0.2cm]
		
		Ambient light & $lx$ & 4 Hz & low & none & reporting the magnitude of the surrounding light. \\ \addlinespace[0.2cm]

		Barometer & $hPa$ & 120 Hz & low & none & reporting the surrounding atmosphere's pressure. \\ \addlinespace[0.2cm]	
		
		Bluetooth & $dBm$ & various* & medium & dangerous & exchanging information with nearby Bluetooth-enabled devices. \\ \addlinespace[0.2cm]
		
		Camera & image-form & 60 Hz & medium & dangerous & generating an image of the surroundings. \\ \addlinespace[0.2cm]
		
		Cellular & $dBm$ & various* & medium & dangerous & exchanging information with nearby phone towers. \\ \addlinespace[0.2cm]
		
		Fingerprint & image-form & 0.5 Hz & low & normal & generating an image of the human finger. \\ \addlinespace[0.2cm]
		
		GPS	& $s$, $m$ & various* & high & dangerous & reporting the satellite signals, clock timestamp, and status.  \\ \addlinespace[0.2cm]
		
		Gyroscope & $rad/s$ & 500 Hz & low & none & reporting the changing rate of the phone's rotational motion. \\ \addlinespace[0.2cm]
		
		Magnetometer & $\mu T$ & 200 Hz & low & none & reporting the magnitude of the surrounding magnetic field. \\ \addlinespace[0.2cm]
		
		Microphone & $dB$ & 48 kHz & medium & dangerous & reporting the magnitude and the raw surrounding acoustic noise. \\ \addlinespace[0.2cm]
		
		NFC & N/A & 1 Hz & low & normal & exchanging information with nearby RFID tags within 10 cm. \\ \addlinespace[0.2cm]
		
		Proximity & $cm$ & 4 Hz & low & none & reporting the distance to the nearest object within 10 cm. Some proximity sensors only report a binary near/far result. \\ \addlinespace[0.2cm]
		
		%Speaker & $dB$ & 48 kHz & medium & none & reporting the magnitude and the raw surrounding acoustic noise. \\ \addlinespace[0.2cm]
		
		%Thermometer & $^{\circ}C$ & 10 Hz & low & none & reporting the temperature of the internal phone's components. \\ \addlinespace[0.2cm]
		
		%Time-of-flight & $cm$ & 5 Hz & medium & dangerous & reporting the distance to the nearest objects within 3 m. \\ \addlinespace[0.2cm]
		
		WiFi & $dBm$ & 0.03 Hz & high & dangerous & exchanging information with nearby WiFi-enabled devices. \\ \addlinespace[0.2cm]
		\bottomrule
	\end{tabular}
	
	$*$ The phone will continue scanning for nearby Bluetooth devices, GPS satellites and Cellular towers, updating the results as they come in, until the process is interrupted.
	%$^{\dagger}$ these sensors may not be available in all devices.
	\label{comparisonsensors}
\end{table}

When it comes to picking which sensors to use, we impose a strong criterion that any selected sensor must not directly reveal the smartphone's position, for our sole purpose of proximity tracking, and to avoid potential information misuse. This rules out GPS (which directly yields the longitude and latitude), and Cellular (whose cell tower location database is publicly available\footnote{https://opencellid.org - last accessed in 5/2020.}). The ambient light sensor which measures the lighting intensity, is an interesting source of information. Yet, its readings are too volatile under different phone's angles. On the same note, the camera, fingerprint, NFC, and time-of-flight's usage were rather specific, and do not appear to be useful for our contact tracing purpose at this stage.
%There are concerns regarding WiFi and Bluetooth, which dominantly featured in the indoor location tracking research.

Last but not least, as mentioned in the previous section, only the nearby phones' IDs are stored locally on the smartphones, and not the detailed sensor data.


%\subsubsection{Accelerometer for mobility detection}
%The accelerometer measures the phone's acceleration in 3-dimensional space. For our purpose, we will use the accelerometer to detect whether the user is moving or not. The rationale is that a person is more likely to be infected if s/he stays in one place with the virus host, rather than just walking by.



\subsubsection{Bluetooth for proximity detection}
\label{bluetooth}
On the smartphones, Bluetooth technology, with its latest iteration called Bluetooth Low Energy (BLE), was intended to connect the phone to small peripherals (e.g. headphone, gamepad, etc.) in short distance (typically no more than 10 metres, and ideally within 2 to 3 metres), which is a great fit for close contact detection.

The detection process using BLE works as follows. First and foremost, one phone must act in the `Central' role, and the other phone plays the `Peripheral' role. The peripheral phone will constantly sending out unsolicited messages (including its name, MAC address, etc.) on the 40 BLE channels to inform its existence~\cite{contreras2017performance,de2017study,faragher2015location,kalbandhe2016indoor}. The central phone, at any time of preference, will initiate a scan to look for those peripheral phones. Secondly, both phones may establish a BLE connection to exchange information. Ideally, the smartphones will alternate between these two roles to avoid the situation where everyone is listening while no-one is broadcasting, and vice-versa.

It is worth noting that, as a BLE scan will also reveal the received signal strength (RSS), which roughly indicates how far away the nearby peripheral phones are, we could estimate the distance using the well-known inverse square law of Physics, that is, the signal intensity is inversely proportional to the square of the distance from a signal transmitter~\cite{goldsmith2005wireless}.
\begin{equation}
    d = 10^{(\frac{Power - BLE\_RSS} {10 * n})}
\end{equation}
where d is the estimated distance, Power is the RSS measured at 1 metre, $n$ is the signal propagation constant (e.g. $n = 2$ for free-space path loss), BLE\_RSS is the RSS received by the central phone. Although there are other distance models, without knowing how the BLE signal propagates (especially challenging indoors), we opted for this simple form of free-space path loss model.

Nevertheless, there are two challenges with BLE. Firstly, BLE wireless signal does easily penetrate walls and furniture, which contributes to the false positive contact detection (i.e. two neighbours separated by a thick wall may be registered as a contact). Secondly, the correlation between the RSS and the distance is not strictly linear, because of the signal attenuation (i.e. without an unobstructed line-of-sight between the two phones, the signal waves are strengthened or weakened as they travel in different directions in the air), and the BLE frequency hopping technique (i.e. the phone frequently switches between the BLE channels to avoid signal collision, which inadvertently impacts the receiving signal at the other end).

In short, we will only employ BLE as a rough indicator to discover nearby phones (which could be anywhere up to 30 metres in the vicinity), and to kick-start the upcoming procedures involving other sensors. We will later demonstrate empirically in detail how frequency hopping may affect the accuracy of BLE in estimating the relative distance between the phones in Section~\ref{experiments}.


\subsubsection{WiFi for distance measuring}
\label{WiFi}
On the smartphones, WiFi technology was originally intended to connect the phone to the Access Points (APs) for internet access. Recently, the WiFi Direct peer-to-peer protocol enables two WiFi-enabled devices to communicate directly without an AP, over much longer distance (up to 50 metres indoors) than Bluetooth~\cite{khan2017wi}, in which the smartphones will negotiate directly between themselves to automatically assign the central and peripheral role.

From the contact detection's viewpoint, WiFi technology may be employed for environment comparison, appearance sensing, as well as distance measuring purposes.
\begin{itemize}
    \item \textbf{For environment comparison.} The observation is that most modern buildings and public venues have plenty of WiFi APs to provide internet access to residents and customers. As such, two smartphones with a similar set of observed APs are potentially nearby.
    
    \item \textbf{For appearance sensing and distance measuring.} This is performed in a similar fashion as with BLE.
\end{itemize}

However, there are five major challenges for employing WiFi. Firstly, since Android 9, all apps may only initiate a scan at most 4 times every 2 minutes (about once every 30 seconds)\footnote{https://developer.android.com/guide/topics/connectivity/wifi-scan - last accessed in 5/2020}, whilst there is no scanning restriction for Bluetooth\footnote{Strictly speaking, since Android 8, the \textit{startScan()} method does impose four predefined scanning configurations, where the fastest $SCAN\_MODE\_LOW\_LATENCY$ option only scans for about 300 ms. However, it is still possible (even on the latest Android 10) to call the deprecated $startLeScan() / stopLeScan()$ methods introduced since Android 5 to fully control the scanning cycle.}. Secondly, the current implementation of WiFi Direct on Android does not expose the RSS of the peer devices, which means the phone needs to be set up as a WiFi hotspot for its RSS to be harvested via the usual WiFi scan. Thirdly, the glaring weakness of WiFi technology is its long broadcasting distance of up to 50 metres. As such, smartphones on different floors, or even separate buildings may still see each others, or observe the same set of WiFi APs, which invalidates the environment comparison approach. Fourthly, WiFi consumes much higher battery than BLE and other sensors. Lastly, WiFi signals do suffer from the same signal multi-path problem as BLE.

Taking into account these concerns, WiFi should only be employed to complement BLE. In particular, it should only be called into action when a potential contact has been confirmed by BLE. Given the WiFi RSS is considerably more stable (i.e. no frequency hopping as in BLE), its distance conversion may be more accurately represented (to be empirically assessed in Section~\ref{experiments}).

Without loss of generality, given the WiFi RSS sequence (which reflects the distance) between the two smartphones $W=(w_1, \dots, w_N)$, recorded over a time window (e.g. 15 minutes based on the WHO recommended infectious duration\footnote{https://www.who.int/docs/default-source/coronaviruse/situation-reports/20200326-sitrep-66-covid-19.pdf - last accessed in 7/2020}), where $w_i$ $(1 \leq i \leq N)$ is the WiFi RSS at time point $i^{th}$, we employed the free-space path loss model, described in Section~\ref{bluetooth}, to covert the RSS into a distance estimate. The mean value of all estimates within the entire time window will decide if the two phones were within the infection range (e.g. 1 metre based on the WHO guideline).




\subsubsection{Microphone for proximity detection}
\label{sound}
%On the smartphones, the microphone was included for low frequency human voice recording around 20 Hz to 20 kHz, although some modern smartphones are capable of recording much higher frequency ambient acoustic noises.

Sound is generated by the vibrations of air particles, which are then picked up by the human ear and the smartphone's microphone~\cite{murakami2018smartphone}. For contact detection, sound may be leveraged for both the appearance sensing and the distance measuring purposes as follows (although it is possible to encode information within those sounds, yet, for our contact tracing purpose, we only need to detect the appearance and the rough distance measure).

For appearance sensing, one smartphone will play the peripheral role by emitting a chirp via its built-in loud speaker. The other phones which act in the central role will pick up those sounds by their built-in microphone, and response with their own chirps to make contact, which indicates that they are in close proximity.

For distance measuring, there are two options. The first one is based on the concept of time difference of arrival. The second one is based on sound amplitude.
\begin{itemize}
    \item \textbf{Time difference of arrival (TDoA).} With this approach, the distance between the two phones is inferred by timing the moment the chirp was sent by one phone, until it is received by the other end. If the clock of both phones are perfectly synchronised, the distance is simply calculated as $distance\;=\;speed\;of\;sound\;*\;elapsed\;time$, with speed of sound is a constant (e.g. 343 $m/s$ at 23 $^{\circ}$C room temperature)~\cite{yavuz2015measuring}.
    
    The rationale of this approach is that acoustic signal travels at a much slower velocity (i.e. 343 $m/s$), compared to WiFi and Bluetooth signals which travel at the speed of light (i.e. 300,000 $km/s$). Therefore, it is more feasible to time and compute such distance.
    
    % https://stackoverflow.com/questions/15212361/android-getting-distance-using-sound-between-two-devices
    % https://stackoverflow.com/questions/7266298/android-sound-api-deterministic-low-latency/12309643#12309643
    However, there are several challenges. Firstly, it is unlikely various smartphones would share the same clock timing. Secondly, all Android sound packages (i.e. SoundPool, AudioTrack and OpenSL ES) have considerably non-deterministic latency (i.e. there are unpredictable delay from the moment the audio play command was issued, until the actual sound was sent out by the built-in speaker), in the region of [180 $ms$ to 300 $ms$]. Given that sound travels at 343 $m/s$, an average of 200 $ms$ error in TDoA estimation will lead to more than 70 metres error in ranging estimation, which is simply not usable for our purpose.
    
    \item \textbf{Sound amplitude.} This approach relies on the same concept as in BLE and WiFi RSS ranging, that is, using sound amplitude as distance indicator. The central phone plays a chirp. As this chirp travels in the air, it loses pressure for which the receiving phone may use to work out the distance to the central phone.
\end{itemize}

%However, as various smartphones may not share the same clock timing, we exploit the fact that a chirp can be self-recorded by the microphone on the same phone, whose location and distance to the loud speaker are known (see Figure~\ref{sounddistance}). By employing the orthogonality of the overlapped sounds, we can measure the distance between the two phones using just the local time on each phone as follows :~\cite{lee2016asynchronous}.
%\begin{equation}
    %distance\;=\;\frac{speed\;of\;sound}{2} ((t_{A2} - t_{A1}) - (t_{B1} - t_{B2})) + K
%\end{equation}
%where $K$ is the sum of distance between the loud speaker and the top microphone of each phone, $t_{A1}$ and $t_{B1}$ are the timestamp when the chirps are received by the own microphone of each phone, $t_{A2}$ is the timestamp when phone A receives the chirp from phone B, and $t_{B2}$ is the timestamp when phone B receives the chirp from phone A.
% \begin{figure}[h]
%     \centering
%     \sidecaption
%     \includegraphics[scale=.60]{Figures/sounddistance.png}
%     \caption{The process of chirp communicating between two phones. Most smartphones come with dual-microphones set up to capture both human voice and the ambient sound, for noise reduction and improving sound quality.}
%     \label{sounddistance}       
% \end{figure}

% https://www.siyavula.com/read/science/grade-10/sound/10-sound-03
The final task is designing a chirp signal to be received reliably within short distance by other nearby smartphones, for which there are three criteria to consider.
\begin{itemize}
    \item \textbf{Frequency}. The frequency (in $Hz$) is the speed of vibration which determines the sound pitch (e.g. female voice is perceived to have higher pitch than male one). Distance wise, low frequency sound travels further than high frequency one, as there is less energy being lost in the process. As such, we would prefer high frequency chirp for short distance contact detection. Nevertheless, high frequency chirp tends to attenuate more heavily than low frequency one, because of higher viscosity caused by many peaks pressuring against the air~\cite{hoppe2012acoustic}. The major constraint is that, although our test phones are capable of recording very high frequency acoustic sound, streamlined smartphone microphones may be limited to lower frequencies in the human audible range (i.e. 20 $Hz$ to 20 $kHz$), which was primarily what they were designed for. As such, our chirp signal should be within this range.
    
    \item \textbf{Amplitude.} The amplitude (in $dB$) is the size of the vibration which determines the loudness of the chirp. A high amplitude means louder sound, which in turns, translates to longer travelling distance. For contact detection, we would prefer low amplitude sound for quiet operation as well as limiting the distance to within the short contagion range.
    
    \item \textbf{Duration.} The duration (in $ms$) is the length of the vibration. For contact detection, the chirp duration should be short, as multiple chirps may be sent simultaneously by other nearby phones, and potentially confuse the receiving ends.
\end{itemize}

% CP: Interestingly, as the length of the chirp increased, users began to notice a "swooshing" sound. For this reason, we sized our chirps to be at least 20ms and no longer than 200ms. 

Taking all of these constraints into consideration, an ideal chirp for contact tracing purpose should have a frequency within 2 $kHz$ to 6 $kHz$ range (as most acoustic sounds start to attenuate greatly above 8 $kHz$~\cite{peng2007beepbeep}), with an amplitude of about 20 $dB$ (still within the recording range of all smartphone's microphones, yet is perceived as just a small whisper for the human ear), and a 50 $ms$ duration (the ideal length to avoid collision with other chirps)~\cite{lazik2012indoor}. Lastly, each smartphone should use a different frequency when communicating to better differentiate with other nearby phones\footnote{Although it is possible to construct a new chirp on the fly with new properties to quickly adjust to the real-time environment, we reckon this process would add unnecessary complexity and power usage to the system. Hence, we leave this for future investigations.}.

%The slight inconvenience is that some audible `beeps' may be heard when two smartphones are exchanging information.

There are two advantages of sound based approach. Firstly, while radio waves propagate seamlessly through space, sound waves require a material medium (e.g. water, air) to convey from one place to another, which does not easily penetrate thick walls, furniture. Secondly, we may manipulate the frequency and loudness properties of the a chirp, which indirectly control the travelling distance, whereas although theoretically possible to do the same with BLE, there is no Android API to modify the BLE sensor's sensitivity at the time of writing. Hence, sound is a great fit for contact tracing.

Nevertheless, there are three challenges for employing sound. Firstly, it does suffer from the same multi-path issue as in BLE and WiFi, in which the acoustic signals reach the receiving microphone in different paths, due to reverberation. Secondly, the sound propagation speed does vary according to different temperatures and humidity (this issue may be negligible in real practice, as the distance between the phones is rather short for our purpose). Lastly, while BLE and WiFi encode their unique sender ID within the signal, we need to design such handshaking process from scratch for sound. Our solution is leveraging BLE signal to make contact between the two smartphones first, before transmitting the chirp.


\subsubsection{Magnetometer for ambient magnetic field comparison}
\label{sec:magnetometer}
Magnetism exists everywhere on Earth, which is caused by the movements of molten metal at the Earth's core. This natural magnetic field is strongly perturbed by ferrous metal from most building's materials~\cite{kim2016novel,kim2017indoor,storms2019magnetic}. Therefore, small indoor areas within the building may report different magnetic signatures, which is an opportunity for matching smartphones contact.

The magnetometer measures the ambient magnetic field strength around the phone in 3-dimensional space. However, as the sensor is fixed within the phone body, its coordinates align with the phone's body frame. As such, the orientation of the phone varies the magnetometer reading, even in the same spot. This challenge may be addressed by ignoring the direction of the magnetic field vector, and computing the total scalar magnitude $m$ as follows~\cite{nguyen2017co}.
\begin{equation}
    m = \sqrt{m_x^2 + m_y^2 + m_z^2}
\end{equation}
where $m_x$, $m_y$ and $m_z$ in $(\mu T)$ are the magnetic field strength along the $x$, $y$ and $z$ axis respectively.

Without loss of generality, given Alice's magnetometer measures $A=(m_1, \dots, m_N)$ and Bob's magnetometer measures $B=(m_1', \dots, m_M')$, where $N$ and $M$ may be different due to various sensor's sampling rates. We will apply the method described in~\cite{nguyen2017co} using Dynamic Time Warping (DTW) which was designed to compare the magnetic sequences of different length, and to handle the various sensitivities from different sensor vendors (i.e. a more sensitive sensor may return higher reading values). In essence, DTW finds the optimal warped path between the two sequences by building an N-by-M matrix, in which $[i^{th}, j^{th}]$ is the distance between measures $m_i$ and $m_j'$ calculated as follows~\cite{nguyen2019realtime}.
\begin{equation}
    d(m_i, m_j') = (m_i - m_j')^2
\end{equation}

The optimal warped path of length k : $w(1,1), \dots, w(n,m)$ between the two sequences, that minimises the warping cost is computed as follows~\cite{nguyen2019realtime}.
\begin{equation}
    %\vspace{15pt}
    \hspace{130pt}
	w(n, m) = d(m_i, m'_j) + min\left\{\hspace{5pt}
	\begin{tabular}{@{}l@{}}
	w(i-1, j) \\
	w(i-1, j-1) \\
	w(i, j-1)
	\end{tabular}\hspace{2pt}
	\right\}\hspace{130pt}
\end{equation}

with $(1 < i < N, 1 < j < M)$.

At the end, the DTW score is computed as $w(n,m)$ divided by the length of the warped path. If the computed score of the two magnetic sequences is below the threshold, the two phones are deemed to be in close proximity (see Section~\ref{sec:environment} for the empirical experiments with the magnetic field threshold).



\subsubsection{Barometer for air comparison}
As most viral diseases are airborne, having the ability to detect if two persons are breathing the same air is profoundly valuable. Regrettably, all sensors discussed so far inherit the same weakness from the wireless signal property, that is, electromagnetic and sound wave can penetrate walls. As such, two persons fully segregated by a thick, concrete wall may still be detected by BLE, WiFi, or sound signal, which result in a false positive being registered.

The barometer which measures the ambient air pressure around the phone, may offer a solution for this challenge. The observation is that, the readings are varied by the air weight in the atmosphere, caused by different altitudes (e.g. on different building floors), or by the winds (e.g. a closed indoor space has different measures from an open outdoor one)~\cite{kim2017floor,li2013using,nii2017bar}.

Nevertheless, if the phone is kept in a tight pocket, handbag, etc., the air measures are potentially different, although the two persons are in the same place. This challenge may be addressed with the help of the proximity sensor, which was originally designed to measure the distance to the nearest object facing the phone screen (i.e. it is intended to detect the side of the human face to switch off the screen while answering a call). By using the proximity sensor, the system may detect whether the phone is left in tight space or in the open to trigger the barometer reading.

Without loss of generality, given the barometer measure sequence from two smartphones, we employed the same DTW approach as in the above Section~\ref{sec:magnetometer}, to work out a score reflecting the difference between the two barometer sequences. If the score is below the threshold, the two phones are deemed to be in close proximity, from the barometer's perspective (see Section~\ref{sec:environment} for the empirical experiments with the air pressure threshold).


%\subsubsection{Proximity sensor for open space detection}
%The proximity sensor measures the distance to the nearest object facing the phone screen. It is intended to detect the side of the human face to switch off the screen while answering a call. This sensor may help the challenge facing the barometer by detecting whether the phone is left in tight space or in the open.


\subsection{Fusion of sensors information}
Having studied the individual role of each sensor, we may now present our strategy to combine the above six sensors together. The system prioritises the appearance sensing of other nearby phones first, then moving on to measure the relative distance between them, and finally comparing the shared environment to reduce the number of false positives (see Figure~\ref{decision} and Table~\ref{sensorsemployed}). The detailed steps are as follows.
\begin{figure}[h]
    \centering
    \sidecaption
    \includegraphics[scale=.55]{Figures/decision.png}
    \caption{The flowchart of our contact tracing process which involves different smartphone sensors at each stage. At the end of each of the three main stages, an output regarding the phone's visibility, the relative distance estimate, and the environment difference are produced respectively.}
    \label{decision}
\end{figure}

\begin{table}[h]
	\caption{Overview of the role of the smartphones sensors employed by our system.}
	\centering
	\begin{tabular}{c c c c}
		\toprule
		\textbf{Sensor} & \textbf{Appearance sensing} & \textbf{Distance measuring} & \textbf{Environment comparison} \\
		\midrule
		Barometer* & N/A & N/A & Shared air \\ \addlinespace[0.2cm]
		Bluetooth & Coarse-grained & N/A & N/A \\ \addlinespace[0.2cm]
		Magnetometer & N/A & N/A & Ambient magnetism \\ \addlinespace[0.2cm]
		Microphone & Fine-grained & Fine-grained & N/A \\ \addlinespace[0.2cm]
		%Proximity & N/A & N/A & N/A \\ \addlinespace[0.2cm]
		WiFi & N/A & Medium-grained & N/A \\ \addlinespace[0.2cm]
		\bottomrule
	\end{tabular}
	
	* The proximity sensor is used as a trigger for barometer.
	\label{sensorsemployed}
\end{table}

\begin{description}[Type 1]
    \item[\textbf{Appearance sensing}]{\hfill \\
    \textit{Step 1:} The app periodically scans for nearby smartphones using the BLE signal. The rationale for using BLE as the primary sensor for appearance sensing, although other sensors (i.e. WiFi, microphone) are capable of the same function, is two-folds. Broadcasting distance wise, WiFi's range would be too long for consideration (with visibility of up to 50 metres away); acoustic sound, despite capable of adjusting its range, may not be reliable in noisy environment; whereas BLE offers the most compromised option for this matter. Power consumption wise, BLE consumes the least amount of energy amongst these sensors. This step is repeated until a smartphone is discovered.\\
    
    %The assumption is that if BLE fails to co-locate, the suspected smartphone is either outside the infected zone.
    
    \textit{Step 2:} When a nearby phone is found by BLE, its range could be anywhere within 30 metres, including being on the other side of a thick wall or furniture. As such, this step attempts to further reduce these false positives with sound sensing. Firstly, the app does a preliminary check of the ambient noise. Since our chirps have an amplitude of around 20 $dB$, if the background noise is above this threshold, the system skips to Step 3, otherwise it performs a chirp scanning to see if the other phone can be heard (sound does not penetrate walls as easily as WiFi and BLE, hence there is a chance the phones are truly close).
    }
    
    \item[\textbf{Distance measuring}]{\hfill \\
    \textit{Step 3:} If the environment is too noisy, the system computes the WiFi based distance between the phones (although WiFi has a theoretical longer range than BLE, it does not use frequency hopping, hence resulting in much stable RSS). Otherwise, if the nearby phone can be reached by the chirp, the system computes the sound based distance, then proceeds to compute the WiFi based distance as well. In the end, those two distances are averaged for a better distance estimate.
    }
    
    \item[\textbf{Environment comparison}]{\hfill \\
    \textit{Step 4:} If both smartphones are in open space (as verified by the proximity sensor), the system compares their ambient air pressure, to assess whether the respective phone owners are breathing the same air (which is critical for airborne diseases). Otherwise, the system just compares the ambient magnetic field around the phones.
    }
\end{description}

The above processes may be viewed as in an on-line setting, where sensor measures arrive in real-time, and the system should decide on 3 output metrics, that are (in decreasing priority order), whether the two smartphones are close, what is the relative distance between them, and if they are sharing the same environment. Ideally, for a contact between the two phones to be registered, the app should detect that they are close (via BLE and/or sound), and their estimated distance is less than 1 metre according to WHO's guidance on infection (via WiFi and/or sound), and their shared environment is similar (via barometer and/or magnetometer). The final decision for each metric will be made over a 15 minute window (which is the duration to be infected according to the WHO's guidance), where the appearance can be a simple majority vote (e.g. if there are 10 samples in this period, in which over 50\% of them indicate a match, then the final decision is that they are close), the distance estimate and the environment sharing are an average of individual measures over this period window.



\subsection{Challenges}
Although the aforementioned sensors are undoubtedly useful for contact tracing, it is worth remembering that all of them were primarily designed for other purposes in mind. As such, the following challenges should be taken into account.
\begin{itemize}
    \item \textbf{Heterogeneous sensors.} Smartphones sensors come in different shapes and forms according to their makers, which may impact their sensitivities to the environment (e.g. some phones may receive a strong WiFi or BLE signal from far away thanks to a bigger antenna, whereas others need to be much closer). This challenge specifically impacts the environment observation approach, as one smartphone may view the world around differently from another phone. %We alleviate this concern by looking at the appearance of the objects in the world, and combining the signal strengths from different sensors. 
    
    \item \textbf{Noisy measures.} Smartphones sensors are miniaturised devices being packed tightly in a small phone body, whose measures are particularly noisy, not to mention the interference from other sources (e.g. the 2.4 GHz band is overcrowded with devices such as PC, laptop, microwave, radio, etc.). This challenge impacts the reliability of the sensors, as a sudden electronic noise may be misclassified as a true measure. %We address this problem by applying some sensor filters in our algorithm.
    
    \item \textbf{Signal multi-path.} In a convoluted environment, there is rarely an unobstructed line-of-sight between the two smartphones. As such, the radio and sound waves travel in unexpected fashions in the air, which results in various receiving signals at the other end. This challenge strongly impacts the true distance estimate which is heavily based on the RSS. We address this concern by measuring several signals in the same place (i.e. increasing the probability of observing the good signal), and combining signals from different sensors. It is worth noting that as people approach closer to each other, there is less likely an obstacle between them, which lessens the impact of this challenge. 
\end{itemize}




\section{Empirical experiments}
\label{experiments}
Having presented our contact tracing proposal, we will now assess its feasibility and performance in various experiments. In doing so, we aim to address the following research questions.
\begin{itemize}
    \item Appearance sensing wise, what is the typical indoor and outdoor detection range of BLE and sound ?
    
    \item Relative distance wise, what is the offset between the estimated sound based, and WiFi based distance measure and the true distance ?
    
    \item Environment comparison wise, what is the feasibility of using air pressure and magnetic field ?
    
    \item Overall contact detection wise, what is the accuracy (in terms of the number of false positives) of our approach, compared to pure BLE based system ?
\end{itemize}


\subsection{Testbeds}
\label{sec:testbeds}
Three realistic testbeds, representing both indoor and outdoor environment, were purposely selected to examine the feasibility and performance of our approach (see Figure~\ref{floorplans}). The first testbed contains a five-room office. The second testbed contains the communal areas of a 14-storey building. The third testbed is an open air parking garage.
% Parking garage : 35 lots width, 2.5 m each. 15 lots height, 2.5 m each (87.5 m x 37.5 m)
\begin{figure}[h]
	\centering
	
	\subfloat[The five-room office. There is plenty of electric appliances lying around (not shown here).]{\includegraphics[width=2.9in]{Figures/officefloorplan.png}
		\label{officefloorplan}}
	\hspace{0.5pt}
	\subfloat[The ground floor of the 14-storey building. Only the communal areas (the shaded areas) were used.]{\includegraphics[width=2.6in]{Figures/groundfloorplan.png}
		\label{groundfloorplan}}
	\hfil
	\subfloat[The parking garage. The shaded areas are the parking lots, which were mostly empty on the day. ]{\includegraphics[width=3.6in]{Figures/parkingfloorplan.png}
		\label{parkingfloorplan}}
	
	\caption{The floor plans of the three testbeds.}
	\label{floorplans}
\end{figure}
%Most of the tests will be carried out in the communal spaces (e.g. corridors, gym) on different floor levels.
% \begin{table}[!ht]
% 	\caption{Overview of the indoor and outdoor test beds.}
% 	\centering
% 	\begin{tabular}{c c c c}
% 		\toprule
% 		\textbf{Test bed} & \textbf{Dimension} & \textbf{Territory} & \textbf{Floor} \\
% 		\midrule
% 		Five-room office & 11 m x 6.3 m & Indoor & 1 \\ \addlinespace[0.2cm]
% 		Fourteen-storey building & 68 m x 42 m & Indoor & 13 \\ \addlinespace[0.2cm]
% 		Parking garage & 87.5 m x 37.5 m & Outdoor & 1 \\ \addlinespace[0.2cm]
% 		\bottomrule
% 	\end{tabular}
% 	\label{testbeds}
% \end{table}

Sample wise, for each testbed, we record 80 test instances, where each instance contains a pair of samples from two test phones at two different locations. In total, there are 240 test instances across the 3 testbeds. Distance wise, 60 of these instances are within 1 metre (i.e. the contagion range advised by WHO), another 60 of them are within 1 metre and 2 metres (i.e. the contagion range adopted by most European countries), 40 of them are within 2 metres and 3 metres (i.e. the false positive buffer zone), and the remaining 80 are somewhere between 3 metres and 30 metres (see Table~\ref{sampledistribution}).
% 60 within [0 - 1] m (40 indoors, 20 outdoors)
% 60 within [1 - 2] m (40 indoors, 20 outdoors)
% 40 within [2 - 3] m (20 indoors, 20 outdoors)
% 80 within  [3 - 30] m (20 indoors, 60 outdoors)
\begin{table}[h]
	\caption{Overview of the distribution of test samples. More emphasis was given to samples within 3 metres as they are within the infection range of our interest.}
	\centering
	\begin{tabular}{c c c}
		\toprule
		\textbf{Distance between phones} & \textbf{Indoor samples} & \textbf{Outdoor samples} \\
		\midrule
		(0 - 1) metre & 40 & 20 \\ \addlinespace[0.2cm]
		(1 - 2) metres & 40 & 20 \\ \addlinespace[0.2cm]
		(2 - 3) metres & 20 & 20 \\ \addlinespace[0.2cm]
		(3 - 30) metres & 20 & 60 \\ \addlinespace[0.2cm]
		\bottomrule
	\end{tabular}
	\label{sampledistribution}
\end{table}

Ambient environment wise, the five-room office represents a typical indoor condition, with plenty of electric appliances (e.g. laptops, PCs, cameras) operating in the background, which may impact the wireless signal between the phones. The 14-storey building allowed our approach to be verified on different floor levels. The parking garage represents an ideal environment with wide open space containing unobstructed line-of-sight between the devices for minimal signal path loss. 

% physically separated instances by walls, floors.


\subsection{Test devices and subjects}
\label{sec:devices}
Three smartphones were selected for testing, namely the LG G7 ThinQ, Samsung Galaxy S8, and Lenovo Phab 2 Pro (from now on, they will be simply referred to as LG, Samsung, and Lenovo phones). They were chosen to represent a variety of smartphones in the past 5 years, as well as covering different sensors manufacturers and Android operating systems (see Table~\ref{phones}). The LG phone will play the role as the central phone, whereas the Samsung and Lenovo phones will play the peripheral role.
\begin{table}[h]
	\caption{Overview of the 3 smartphones used to verify our approach.}
	\centering
	%% \tablesize{} %% You can specify the fontsize here, e.g.  \tablesize{\footnotesize}. If commented out \small will be used.
	\begin{tabular}{c c c c c c c c}
		\toprule
		\textbf{Phone model} & \textbf{Year} & \textbf{Android} & \textbf{WiFi/BLE} & \textbf{Magnetometer} & \textbf{Microphone} & \textbf{Barometer} & \textbf{Proximity} \\
		\textbf{} & \textbf{released} & \textbf{OS} & \textbf{vendor} & \textbf{vendor} & \textbf{vendor} & \textbf{vendor} & \textbf{vendor} \\
		\midrule
		LG G7 ThinQ & 2018 & 9.0 & Qualcomm & Asahi Kasei & LG & LG & LG \\ \addlinespace[0.2cm]
		Samsung Galaxy S8 & 2017 & 7.0 & Murata & Asahi Kasei & Knowles & Samsung & Samsung \\ \addlinespace[0.2cm]
		Lenovo Phab 2 & 2016 & 6.0.1 & Qualcomm & Bosch & Lenovo & Lenovo & Liteon \\ \addlinespace[0.2cm]
		\bottomrule
	\end{tabular}
	\label{phones}
\end{table}

Using the above test devices, the sensor data were collected by two people at 240 test locations (as described in Section~\ref{sec:testbeds}), during daytime where there were other people around in the same premise, over the period of two months. During the experiments, the phones were either held in the user's hand or left in the pocket. For each test instance, we collected the sensor measures over 5 minutes (although a true positive contact is registered when two phones are within 1 metre for 15 minutes, according to the WHO's Covid-19 guidelines, yet, we relax the measuring time to speed up the experiments).

% Standard free space path loss plot in open parking garage
% highest WiFi RSS -41 right next to router; furthest 30m starts to lose visibility, but less than BLE; maximum 8 dBm variation at any range.
% highest BLE RSS -34 right next to beacon; stronger than WiFi, but drops rapidly at around 2m, furthest 30m starts to lose visibility; maximum 18 dBm variation at middle range.
% stronger RSSI has less variation (smaller error bars), then increase for medium RSSI, weakest RSSI also has less variation (as it becomes highly saturated).

% Surprises: Results tends to be more robust with less than 1 metre.

% Experiment 1: number of BLE scans VS discoverability, 3 testbeds.
    % The bigger the distance, the fewer the discoverabilities.
    % Much fewer visibility indoors.
% Experiment 2: number of sound scans VS discoverability, 3 testbeds.
    % Slightly noisy (still below big noise).
    % Much fewer visibility indoors with wall, furnitures.

% Experiment 3: BLE distance VS true distance, 3 testbeds.
    % very bad
    % Office with many devices on the same frequency has unpredictable RSSI
% Experiment 4: sound distance VS true distance, 3 testbeds.
    % much better with less than 2m.
% Experiment 5: WiFi distance VS true distance, 3 testbeds.
    % in the middle

% Experiment 6: A walk through most positions in all 3 testbeds with 1 phone in hand, showing the changes in air pressure and magnetic field. Outdoor test bed has different air pressure than indoors. Office testbed has higher magnetism disturbance than outdoor.
    % Another figure for air pressure on each of the floor levels.
% Experiment 7: Comparing false positives between BLE only, BLE + 2m distance, our approach, our approach with environment. 
    % 150 test locations.
    % TRUE positive = seen + within 2m.
    % Table result: Rows are 3 test beds. Columns are the 4 approaches.
    % BLE only will have loads of false positives > followed by BLE + 2m > our approach > our approach + environment.


%CP: false positives (where people are flagged as having been in contact with an infected person but in fact have not been) and false negatives (where people are not flagged as having been in contact with an infected person, but in fact were)

\subsection{Appearance sensing analysis}
\label{sec:appearance}
The purpose of the first experiment is assessing the visibility of BLE and sound, with varying distance between the two smartphones and the environment settings. For each of the 240 test locations, one smartphone constantly looks for the other during a fixed 1 minute period, keeping a record of each scan. We used the same pair of LG, Samsung phones in this experiment for consistency.

For BLE, the result revealed three interesting trends (see Figure~\ref{BLEvisibility}). Firstly, the further the distance between of the two phones was, the lower the discoverability rate was. At 20 metres indoors, the detection number was just 16, compared to 124 of that when two phones were 1 metre apart. Secondly, the indoor visibility was noticeably poorer than outdoors (which is understandable with almost zero line-of-sight between the phones). Thirdly, they may still see each other at 20 metres apart indoors, and 30 metres outdoors. The third result has a strong implication for contact tracing, as these detections are clearly not in the contagion range, and may trigger a false positive.
% 2-3 BLE signals per second
\begin{figure}[h]
    \centering
    \sidecaption
    \includegraphics[scale=.22]{Figures/BLEvisibility.png}
    \caption{The scanning frequency of BLE, reported at 240 test locations. At each location, the central LG phone scans for 60 seconds and reports the result. In general, more measures were available outdoors than indoors, and at shorter distances.}
    \label{BLEvisibility}       
\end{figure}

For sound, the appearance characteristics were similar to that of BLE, in which the further away the distance was, the less likely it could be heard. Yet, there was one clear distinction, that was, far fewer test instances beyond 2 metres can be heard with our chirp than with BLE (e.g. out of 20 test instances with more than 3 metres distance, only 8 of them could be reached with sound) (see Figure~\ref{indooroutdoorsound}).
\begin{figure}[h]
    \centering
    \sidecaption
    \includegraphics[scale=.17]{Figures/indooroutdoorsound.png}
    \caption{The hearing sensitivity of sound indoors and outdoors with respect to different distances between the two smartphones. The shaded areas represent test locations without any BLE or sound signal. Overall, the sensitivity decreases as the distance increases.}
    \label{indooroutdoorsound}       
\end{figure}

Since both BLE and sound may penetrate walls and furniture, their visibility was tested on different floor levels, with about 3 metre ceiling height. Interestingly, we got no sound feedback beyond the first floor, whereas BLE signal could penetrate up to the 6$^{th}$ floor (see Figure~\ref{BLESoundFloor}).
%where there is no shared environment between them (i.e. the 2 phones were in different rooms or floors, segregated by walls, doors), 37 of them were flagged by BLE, yet only 14 were detected by sound. These detections are false positives.
% Put BLE figure here
% 60 within [0 - 1] m (40 indoors, 20 outdoors)
% 60 within [1 - 2] m (40 indoors, 20 outdoors)
% 40 within [2 - 3] m (20 indoors, 20 outdoors)
% 80 within  [3 - 30] m (20 indoors, 60 outdoors)
\begin{figure}[h]
    \centering
    \sidecaption
    \includegraphics[scale=.17]{Figures/BLESoundFloor.png}
    \caption{The visibility of BLE and sound on 13 floor levels, with 10 test locations per floor. The central LG phone was placed on the ground floor, while the Samsung phone keeps climbing up. No sound contact could be made beyond the 1$^{st}$ floor, where BLE can still be heard on the 6$^{th}$ floor. The shaded areas represent test locations without any BLE or sound signal.}
    \label{BLESoundFloor}       
\end{figure}

We do not report the scanning frequency for WiFi and sound, because of the WiFi scanning restriction on Android (at most 4 scans every 2 minutes), and the design of our chirp broadcasting interval (to avoid sound collision), as discussed in detail in Sections~\ref{WiFi} and~\ref{sound}.

In summary, the results in this section reveal that BLE-only based systems may score plenty more false positives due to its high visibility of up to 20 metres indoors and 30 metres outdoors. Sound, on the other hand, with much shorter travelling range, may offer an extra layer of appearance sensing information on top of BLE.



\subsection{Relative distance analysis}
\label{sec:distance}
This section assesses the feasibility and accuracy of the WiFi based and sound based distance estimate.

% ONLY this subsection uses the parking garage for ideal distance experiment.
Since the only output we have, is just a simple number in the form of the WiFi RSS and sound pressure, we will assess how reliable this measure is, in the most ideal condition. As such, we performed the experiment in the open air parking garage, where the LG phone was fixed in one place, while the Lenovo phone moved away from it in a straight line. At every one metre along the way, 10 WiFi RSS and sound signal were taken. Theoretically, we would expect a steady decrease of such measures following the free-space path loss model, with respect to increasing distance.

For WiFi, the results revealed three surprising characteristics (see Figure~\ref{WiFivsBLEvsSound}). Firstly, even when both phones are static, with a clear line-of-sight between them and almost no interference from the environment (i.e. the signal may still bounce off the floor, nevertheless), the WiFi RSS still varies in the same spot. Secondly, strong signals (within 3 metres) and weak signals (beyond 20 metres) tends to be more stable than medium signals (from 3 metres to 20 metres). A plausible explanation is that at short distance the signal does not take long to travel, whereas by the time it travels 20 metres, it has lost most of its energy. Medium distance is the central spot for signal attenuation. Thirdly, at the same position, BLE RSS fluctuates significantly more than WiFi's, due to frequency hopping.
% A plot showing RSS vs distance, where x-axis is the distance, y-axis is the RSS. A dot maps the RSS to the corresponding distance, we will see that most fluctuation is in the middle. Less fluctuation at the beginning and in the end.
\begin{figure}[h]
    \centering
    \sidecaption
    \includegraphics[scale=.17]{Figures/WiFivsBLEvsSound.png}
    \caption{Comparison of the WiFi, BLE, and sound signal propagation in an ideal open air setting with a clear line-of-sight between the two smartphones. The sound signals are scaled by the right side y-axis. Acoustic sound and WiFi signals have the least amount of variations, compared to BLE, at the same location.}
    \label{WiFivsBLEvsSound}       
\end{figure}

For sound, the result revealed a remarkably consistent measure with very little variation in this ideal outdoor setting. This ascertains its usefulness in estimating the relative distance between the smartphones.
%the variation happens much stronger between 5 metres to 15 metres, with up to 23 $dBm$ difference in the same spot, in which BLE has the highest number of variation of up to 18 $dBm$

Having understood the baseline expectation of BLE, WiFi and sound propagation characteristic, we may now assess their estimated distance's accuracy with our 240 test instances, both indoors and outdoors. For each test location, the true distance is the shortest straight line connecting the two smartphones discounting any walls and furniture in-between, the estimated WiFi and sound based distances are obtained from the free-space path loss model (as discussed in Sections~\ref{WiFi} and~\ref{sound}), the result is an average of 10 sound measures and 4 WiFi measures (the Android cap for 2 minute scan).

The result revealed the challenging impact of signal multi-path, where both WiFi based and sound based approaches produced high distance estimate error (see Figure~\ref{CDFWiFiSoundOverall}). In particular, sound based approach achieved less than 2 metre distance error 90\% of the time, whereas WiFi based approach could only manage 6.5 metre distance error 90\% chance, demonstrated by the Cumulative Distribution Function (CDF) plot. The confidence bound of sound based result, computed by the Kaplan-Meier estimator~\cite{goel2010understanding}, was also noticeably tighter than WiFi based one.
% CDF plot to report the estimated distance accuracy
\begin{figure}[h]
    \centering
    \sidecaption
    \includegraphics[scale=.17]{Figures/CDFWiFiSoundOverall.png}
    \caption{Comparison of the overall estimated distance based on WiFi and sound signal. The 95\% confidence bound of sound based result, calculated by the Kaplan-Meier estimator~\cite{goel2010understanding}, is noticeably tighter than WiFi one.}
    \label{CDFWiFiSoundOverall}       
\end{figure}

However, it was surprising that, when focusing on just indoor test locations within 3 metres, where there are furniture, walls between the smartphones, WiFi based estimated distances were slightly more accurate than sound one (see Figure~\ref{CDFWiFiSoundIndoor}). This result suggests that a combination of WiFi and sound may produce a better distance estimate.
\begin{figure}[h]
    \centering
    \sidecaption
    \includegraphics[scale=.17]{Figures/CDFWiFiSoundIndoor.png}
    \caption{Comparison of the indoor estimated distance based on WiFi and sound signal for test locations less than 3 metres apart. WiFi based distance estimates were slightly more accurate than sound one.}
    \label{CDFWiFiSoundIndoor}       
\end{figure}

% 60 within [0 - 1] m (40 indoors, 20 outdoors)
% 60 within [1 - 2] m (40 indoors, 20 outdoors)
% 40 within [2 - 3] m (20 indoors, 20 outdoors)
% 80 within  [3 - 30] m (20 indoors, 60 outdoors)

% In an ideal world, 343 metres per second = 343 m for 1,000 ms => 1 m per 2.9 ms <=> 1 ms error in TOA estimation will lead to about 35 cm error in ranging estimation error.

%The results reveal three interesting characteristics. Firstly, with BLE, many test locations further apart possess similar RSS, which ultimately results in a similar distance estimate. For example, two smartphones 7 metres apart are wrongly perceived to be 2 metres apart with BLE. Secondly, the office environment has the worst signal multi-path, caused by other electric applicants operating on the same wireless frequency.



\subsection{Environment comparison analysis}
\label{sec:environment}
This section assesses the feasibility and accuracy of using the barometer and magnetometer to determine whether the two smartphones are sharing the same environment, in particular, `breathing' the same air, which is critical for airborne infection.

Regarding air pressure, we assess its feasibility in close contact detection by experimenting its measure across the vertical and horizontal spaces.

Vertical space wise, we collect 10 barometer readings with the LG and Samsung phones in the 14 floor building, with about 3 metre ceiling height separating each floor. The result reveals three useful information (see Figure~\ref{pressurefloor}). Firstly, the average air pressure strictly decreases as the altitude increases. Secondly, there was a clear measure gap of around 0.43 $hPa$ between each floor. Thirdly, the reading variations are particularly low (i.e. rather stable) per floor, of around 0.13 $hPa$. 
% 1012 hPa on ground floor, variation very low on each floor about 0.13 hPa
% Formula in ideal condition : 0.12 hPa = 1 m 
% hPa decrease when going down
% 0.43 hPa gap between each floor (~3.6 m).
% 3 m actual physical ceiling height

% Box plot of different air pressure variation on different floors; indoor vs outdoor
\begin{figure}[h]
	\centering
	
	\subfloat[The air pressure on different floor levels exhibits a clear gap of around 0.43 $hPa$. The variation per floor is low at around 0.13 $hPa$.]{\includegraphics[width=3.3in]{Figures/pressurefloor.png}
		\label{pressurefloor}}
	\hspace{0.5pt}
	\subfloat[The distribution of the air pressure indoors and outdoors.]{\includegraphics[width=2.3in]{Figures/pressureindooroutdoor.png}
		\label{pressureindooroutdoor}}
	
	\caption{The air pressure measures in different environments.}
	\label{airpressure}
\end{figure}

% CP: This means that the air pressure inside your cleanroom is greater than the pressure outside of it. This is achieved by pumping clean, filtered air into the cleanroom, generally through the ceiling. Positive pressure is used in cleanrooms where the priority is keeping any possible germs or contaminants out of the cleanroom.
Horizontal space wise, we averaged the air pressure measures at 120 indoor test locations and 120 outdoor test locations, which reported 1,012.4 $hPa$ and 1,012.59 $hPa$ respectively (see Figure~\ref{pressureindooroutdoor}). The 0.19 $hPa$ average difference may provide a good indication to separate the indoor and outdoor users. Note that both the indoor and outdoor test environments are on pretty much the same ground level, with the parking garage locating next to the office building. The physical separation between them are walls, doors which may not be kept closed at all times. Therefore, a plausible explanation of such difference measure is that the building's exhaust ventilation system must have made an impact on the air pressure. Much more importantly, the variation of air pressures within the same building floor, or within the parking garage, is negligible and being lost within the sensor's mechanical noise, which turns out to be a great benefactor for our purpose, since smartphones within those areas are `breathing' the same air space.

Regarding the magnetic field, the magnitude plot of 240 test locations across the 3 testbeds revealed two interesting observations (see Figure~\ref{heatmaps}). Firstly, the magnetic variation oscillates much stronger indoors than outdoors, with a maximum magnitude of nearly 120 $\mu T$ in the office testbed, compared to just 67 $\mu T$ in the parking garage testbed. The high number of electric appliances and the building materials may have been accounted for this variation. Secondly, the magnetic anomalies happened in small areas (i.e. only locations within a small room will have similar high magnetic disturbance).
% 3 magnetic heatmap for 3 datasets - very poor outdoors, much better indoors
\begin{figure}[!ht]
	\centering
	
	\subfloat[The five-room office exhibits high magnetic anomalies at various small areas.]{\includegraphics[width=2.9in]{Figures/heatmapoffice.png}
		\label{heatmapoffice}}
	\hspace{0.5pt}
	\subfloat[Some communal spaces (e.g. by the elevator) on the ground floor of the 14-storey building contains some magnetic disturbances.]{\includegraphics[width=2.6in]{Figures/heatmapgroundfloor.png}
		\label{heatmapgroundfloor}}
	\hfil
	\subfloat[The open air parking garage have low magnetic differences. There were not many cars around on the testing day.]{\includegraphics[width=4.1in]{Figures/heatmapgarage.png}
		\label{heatmapgarage}}
	
	\caption{The magnetic heat map visualisation of the three testbeds. The colour maps have been put to the same scale.}
	\label{heatmaps}
\end{figure}

To assess the efficiency of the magnetic readings with respect to the distance between the two smartphones, we separate the 240 test locations into 4 categories, that are, tests within [0-1] metres, [1-2] metres, [2-3] metres, and [3-30] metres (more emphasis was given for those under 3 metres as they are of our particular interest), then computing the magnetic magnitude difference between the two smartphones in each category. The result demonstrated a much smaller Euclidean distance for close ranges, which implies that smartphones in increasingly long distance may observe much different magnetic readings (see Figure~\ref{magneticdistance}).
% magnetic separation plot, showing a clear separation between 1-3, beyond 5 metres.
\begin{figure}[h]
    \centering
    \sidecaption
    \includegraphics[scale=.17]{Figures/magneticdistance.png}
    \caption{The Euclidean distance between the magnetic measures of the two smartphones at various distances. Close range samples have much smaller magnetic difference than long range ones.}
    \label{magneticdistance}       
\end{figure}

Ultimately, the question is, how would the air pressure and the magnetic field be employed to determine if the two smartphones are in close proximity ? Based on the above empirical experiments, air pressure wise, an empirical constant of 0.15 $hPa$ difference threshold (which is above the sensor noise, but is still below the air pressure separation) may be used to determine if two smartphones are in close proximity. Magnetic field wise, an empirical constant of 20 $\mu T$ difference threshold was chosen. In the next section, we will assess how those measures affect the overall system performance.

%In summary, these results ascertain the usefulness of using the air pressure and the magnetic field for detecting contact between smartphones on different floors or being indoor and outdoor.


\subsection{Contact detection accuracy analysis}
\label{sec:accuracy}
Having assessed each of the three processes separately, we may now present the general system accuracy in detecting smartphones contact. The performance metrics are summarised in Table~\ref{metrics}.

For a true positive contact to be registered, both smartphones need to be within 1 metre of each other for at least 15 minutes (according to the WHO's Covid-19 infection guideline). In our tests, we will slightly neglect the time duration constraint, and will only observe 5 minutes per test location to speed up the experiment procedure. In contrast, a contact is considered as a false positive, if the phones were more than 1 metre apart (i.e. the contact should not have been registered by the system). Along the same line, a true negative contact is one which was not recorded by the system, and the phones were indeed more than 1 metre apart. Lastly, a false negative contact is one which was not recorded by the system, although the phones were within 1 metre in reality.
\begin{table}[h]
	\caption{Performance metrics in the contact tracing context.}
	\centering
	\begin{tabular}{c c}
		\toprule
		\textbf{Metrics} & \textbf{Note} \\
		\midrule
		True positive (TP) & True infection contact registered by the system \\ \addlinespace[0.2cm]
		False positive (FP) & Harmless non-infection contact registered by the system \\ \addlinespace[0.2cm]
		True negative (TN) & Harmless non-infection contact ignored by the system \\ \addlinespace[0.2cm]
		False negative (FN) & Actual contact but ignored by the system \\ \addlinespace[0.2cm]
		Overall accuracy & $\frac{TP + TN}{TP + TN + FP + FN}$ \\ \addlinespace[0.2cm]
		\bottomrule
	\end{tabular}
	\label{metrics}
\end{table}

Overall, out of 240 test instances, when only BLE appearance sensing was employed, the number of false positives was rather high at 180 (with most of them indoors) which results in a system accuracy of just 25\%. These figures are unsurprising given the high visibility of BLE. By incorporating distance measuring, the number of false positives dropped significantly to just 61, and more than doubled the system accuracy. However, the trade-off at this point was the increasing number of false negatives, which was unfortunately wrongly discarded by inaccurate distance estimates. Lastly, by applying the environment comparison on top, the number of false positives was further reduced to just 9, boosting the overall system accuracy to 87.08\% (see Table~\ref{overallperformance}).

% Fluctuate strongly + different air/magnetic field => false positive
% in fact all of our test instances MUST be seen by BLE as a pre-condition for being chosen as test instance, hence very high false negative for BLE only.
% the number in                         () are false negatives - false negatives
% 60 within [0 - 1] m (40 indoors, 20 outdoors) - BLE (0, 0) - Distance (13, 9)
% 60 within [1 - 2] m (40 indoors, 20 outdoors) - BLE (40, 20) - Distance (22, 9)
% 40 within [2 - 3] m (20 indoors, 20 outdoors) - BLE (20, 20) - Distance (14, 13)
% 80 within  [3 - 30] m (20 indoors, 60 outdoors) - BLE (20, 60) - Distance (12, 49)

% Appearance + Distance : 13 + 9 cases within [0-1] were discarded for the wrong reason => false negatives
% Appearance + Distance : only 22 + 9 cases within [1-2] have perceived distance > 1, but in fact all of them 60 cases are > 1. Thus 29 cases are wrongly believed to be within [0-1] => false positives
% Appearance + Distance : only 14 + 13 cases within [2-3] have perceived distance > 1, but in fact all of them 40 cases are > 1. Thus 13 cases are wrongly believed to be within [0-1] => false positives
% Appearance + Distance : only 12 + 49 cases within [3-30] have perceived distance > 1, but in fact all of them 80 cases are > 1. Thus 19 cases are wrongly believed to be within [0-1] => false positives

% Environment check is applied on top of the result of appearance + distance. It is mostly used to reject (reduce) false positives
\begin{table}[h]
	\caption{Performance comparison of different system mechanics. A true positive contact means the real distance between the smartphones is within 1 metre.}
	\centering
	\begin{tabular}{c c c c c c c}
		\toprule
		\textbf{System mechanics} & \textbf{Sensors employed} & \textbf{FP} & \textbf{FN} & \textbf{TP} & \textbf{TN} & \textbf{Accuracy} \\
		\midrule
		Appearance & (BLE only) & 180 & 0 & 60 & 0 & 25\% \\ \addlinespace[0.2cm]
		Appearance, distance & (BLE, WiFi, Microphone) & 61 & 22 & 38 & 119 & 65.42\% \\ \addlinespace[0.2cm]
		Appearance, distance, environment & (BLE, WiFi, Microphone, Barometer, Magnetometer) & 9 & 22 & 38 & 171 & 87.08\% \\ \addlinespace[0.2cm]
		\bottomrule
	\end{tabular}
	\label{overallperformance}
\end{table}



\subsection{Summary of results}
Having presented the above experimental results, we may now reflect on the research questions posed earlier.

With regard to sensing based on the device visibility only, two smartphones employing just BLE may still see each other well beyond the 2 metre contagious threshold (as employed by most European countries), in both indoor and outdoor environments with and without obstacles in-between. In some test instances, the BLE visibility could be up to 20 metres indoors and 30 metres outdoors. On the other hand, two smartphones more than 2 metres apart were barely communicable using sound (with a chirp amplitude of 20 dB). In particular, out of 20 indoor test instances beyond 3 metres, while all of them are visible with BLE, only 8 were reachable with sound. In addition, the further the distance between the phones was, the lower the discoverability rate was. This result suggests that BLE-only based system may return more false positives due to its high visibility, and several sensor measures should be taken at each location to ensure good coverage (see Section~\ref{sec:appearance} for more detailed results).

With regard to the relative distance estimate, it was highlighted that all technologies (i.e. WiFi, BLE, sound) struggled to convert their signal measures to precise distance estimation, especially indoors with a large gap between the devices, due to the signal multi-path phenomenon. However, when focusing on test locations within 3 metres, both WiFi and sound based distance estimates were around 2 metre error, 90\% of the time. This result suggests that WiFi and sound could be combined (e.g. averaged result) to produce a more accurate relative distance estimate (see Section~\ref{sec:distance} for more detailed results).

With regard to the feasibility of environment comparison, the experiments across 14 building floors revealed that the air pressure produced by the smartphone barometer strictly decreases as the altitude increases, with a clear measuring gap between each floor, and a low signal variation on the same floor. In addition, there was a clear difference between the indoor and outdoor air pressures. Lastly, the magnetic field measures vary significantly amongst small indoor areas. This result suggests that the air pressure and the magnetic field could provide an extra piece of information in helping to decide whether two smartphones are sharing the same airspace (see Section~\ref{sec:environment} for more detailed results).

With regard to the overall contact detection accuracy, for BLE-only system, the number of false positives was rather high at 180 (out of 240 test instances), with just 25\% accuracy. However, by incorporating WiFi and sound distance estimate, the system scored just 61 false positives, with more than 65\% accuracy. Finally, by applying air pressure and magnetic field comparison on top, the system only allowed just 9 false positives, with 87\% accuracy. This result demonstrated the usefulness of combining various smartphone sensors, over single BLE technology that most existing contact tracing works are currently relying on (see Section~\ref{sec:accuracy} for more detailed results).





\section{Conclusion and further work}
We have presented our novel, yet practical smartphone-based contact tracing approach. Through-out the six sections, we have thoroughly assessed the feasibility and the accuracy of our system in three realistic indoor and outdoor testbeds.

From our empirical results, we come to the conclusion that BLE only approach would suffer from a high number of false positives, due to its high visibility of up to 30 metres outdoors and 20 metres indoors, as well as high RSS variation because of its frequency hopping technique. WiFi signal, despite its long range distance, possesses much more stable RSS, and combining with sound signal would further enhance the distance estimate. More importantly, we recognise the natural property of WiFi, and BLE signals which may penetrate walls easily. To this end, air pressure measure provided by the barometer and the magnetic field magnitude provided by the magnetometer were employed to determine if the users are breathing the same air.

Ultimately, a system making vital decisions about the people's health such as contact tracing should be reliable and informative. Because of several number of parameters involved in the process (e.g. visibility, distance, time), it is more helpful to attach other information into all contact registrations (i.e. a contact estimated to be within 50 cm in more than 15 minutes should have a higher infection chance, than one estimated to be within 2 metres), rather than wholly relying on a binary (yes/no) contact detection. Additionally, exchanged information between the phones could be compressed to reduce the transferring time. This aspect is particularly relevant in the topic of information exchange and could further improve this work. Looking further ahead, it is critical to recognise that viral infections (e.g. Covid-19, SARS) are spread through the air. As such, contact tracing technologies should be able to make distinction between a true contact in shared air space, or a false one segregated by thick walls, to which, this paper is hoping to inspire future similar contact tracing approaches.


%During our experiments, we discovered that BLE RSS tends to fluctuate if there are obstacles (e.g. walls, furniture) within the smartphones. Perhaps this may be used to d

% Most BLE based solution only wakes up intermittently to save battery. For BLE, it's advised to scan regularly within a short time window to make sure not missing any potential match.

\begin{acknowledgement}
The authors are grateful to the three reviewers for their insightful comments that greatly improve this work. This research is partially supported by AstraZeneca.
\end{acknowledgement}

%\section*{Appendix}
%\addcontentsline{toc}{section}{Appendix}
%
%


\bibliographystyle{spmpsci}%{spphys}%{spbasic}
\bibliography{references}


%%%%%%%%%%%%%%%%%%%%%%%%% referenc.tex %%%%%%%%%%%%%%%%%%%%%%%%%%%%%%
% sample references
% %
% Use this file as a template for your own input.
%
%%%%%%%%%%%%%%%%%%%%%%%% Springer-Verlag %%%%%%%%%%%%%%%%%%%%%%%%%%
%
% BibTeX users please use
% \bibliographystyle{}
% \bibliography{}
%


\begin{thebibliography}{99.}%
%% and use \bibitem to create references.
%%
%% Use the following syntax and markup for your references if
%% the subject of your book is from the field
%% "Mathematics, Physics, Statistics, Computer Science"
%%
%% Contribution
\bibitem{Gerritsma} Gerritsma, M.I.: Edge functions for spectral element methods. In Spectral and High Order Methods for Partial Differential Equations, Eds Jan S. Hesthaven and Einar M. R{\o}nquist, \textbf{76}, 494--522, Springer, (2016)
%\bibitem{science-contrib} Broy, M.: Software engineering --- from auxiliary to key technologies. In: Broy, M., Dener, E. (eds.) Software Pioneers, pp. 10-13. Springer, Heidelberg (2002)
%%
%% Online Document
%\bibitem{science-online} Dod, J.: Effective substances. In: The Dictionary of Substances and Their Effects. Royal Society of Chemistry (1999) Available via DIALOG. \\
%\url{http://www.rsc.org/dose/title of subordinate document. Cited 15 Jan 1999}
%%
%% Monograph
\bibitem{Kreyszig} Kreyszig, E.: Introductory functional analysis with applications. Second Edition John Wiley \& Sons (1978)

\bibitem{BuffaCiarlet} Buffa, A., Ciarlet Jr, P.: On traces for functional spaces related to Maxwell's equations Part I: An integration by parts formula in Lipschitz polyhedra. Math. Meth. Appl Sci., \textbf{24}, 9--30, (2001)

\bibitem{OdenDemkowicz} Oden, J.T., Demkowicz, L.: Applied Functional Analysis. Second Edition Chapman \& Hall/CRC (2010)

%%
% Journal article
\bibitem{Carstensen} Carstensen, C., Demkowicz, L., Gopalakrishnan, J.: Breaking spaces and form for the DPG method and applications including the Maxwell equations. Computers and Mathematics with Applications, \textbf{72}, 494--522, (2016)

\bibitem{Jain} Jain,V., Zhang, Y., Palha, A., Gerritsma, M.I.: Construction and application of algebraic dual polynomial representations for finite element methods. arXiv:1712.09472v1, submitted to Computational Methods in Applied Mathematics, (2018)

\bibitem{MEEVC} Palha, A., Gerritsma, M.I.: A mass, energy, enstrophy and vorticity conserving (MEEVC) mimetic spectral element discretization for the 2D incompressible Navier-Stokes equations. Journal of Computational Physics, \textbf{328}, 200--220, (2017)

\bibitem{Yi} Zhang, Y., Jain,V., Palha, A., Gerritsma, M.I.: The discrete Steklov-Poincar\'{e} operator using algebraic dual polynomials. Submitted to Computational Methods in Applied Mathematics, (2018)

%\bibitem{science-journal} Hamburger, C.: Quasimonotonicity, regularity and duality for nonlinear systems of partial differential equations. Ann. Mat. Pura. Appl. \textbf{169}, 321--354 (1995)
%%
%% Journal article by DOI
%\bibitem{science-DOI} Slifka, M.K., Whitton, J.L.: Clinical implications of dysregulated cytokine production. J. Mol. Med. (2000) doi: 10.1007/s001090000086
%%
%\bigskip
%
%% Use the following (APS) syntax and markup for your references if
%% the subject of your book is from the field
%% "Mathematics, Physics, Statistics, Computer Science"
%%
%% Online Document
%\bibitem{phys-online} J. Dod, in \textit{The Dictionary of Substances and Their Effects}, Royal Society of Chemistry. (Available via DIALOG, 1999),
%\url{http://www.rsc.org/dose/title of subordinate document. Cited 15 Jan 1999}
%%
%% Monograph
%\bibitem{phys-mono} H. Ibach, H. L\"uth, \textit{Solid-State Physics}, 2nd edn. (Springer, New York, 1996), pp. 45-56
%%
%% Journal article
%\bibitem{phys-journal} S. Preuss, A. Demchuk Jr., M. Stuke, Appl. Phys. A \textbf{61}
%%
%% Journal article by DOI
%\bibitem{phys-DOI} M.K. Slifka, J.L. Whitton, J. Mol. Med., doi: 10.1007/s001090000086
%%
%% Contribution
%\bibitem{phys-contrib} S.E. Smith, in \textit{Neuromuscular Junction}, ed. by E. Zaimis. Handbook of Experimental Pharmacology, vol 42 (Springer, Heidelberg, 1976), p. 593
%%
%\bigskip
%%
%% Use the following syntax and markup for your references if
%% the subject of your book is from the field
%% "Psychology, Social Sciences"
%%
%%
%% Monograph
%\bibitem{psysoc-mono} Calfee, R.~C., \& Valencia, R.~R. (1991). \textit{APA guide to preparing manuscripts for journal publication.} Washington, DC: American Psychological Association.
%%
%% Online Document
%\bibitem{psysoc-online} Dod, J. (1999). Effective substances. In: The dictionary of substances and their effects. Royal Society of Chemistry. Available via DIALOG. \\
%\url{http://www.rsc.org/dose/Effective substances.} Cited 15 Jan 1999.
%%
%% Journal article
%\bibitem{psysoc-journal} Harris, M., Karper, E., Stacks, G., Hoffman, D., DeNiro, R., Cruz, P., et al. (2001). Writing labs and the Hollywood connection. \textit{J Film} Writing, 44(3), 213--245.
%%
%% Contribution
%\bibitem{psysoc-contrib} O'Neil, J.~M., \& Egan, J. (1992). Men's and women's gender role journeys: Metaphor for healing, transition, and transformation. In B.~R. Wainrig (Ed.), \textit{Gender issues across the life cycle} (pp. 107--123). New York: Springer.
%%
%% Journal article by DOI
%\bibitem{psysoc-DOI}Kreger, M., Brindis, C.D., Manuel, D.M., Sassoubre, L. (2007). Lessons learned in systems change initiatives: benchmarks and indicators. \textit{American Journal of Community Psychology}, doi: 10.1007/s10464-007-9108-14.
%%
%%
%% Use the following syntax and markup for your references if
%% the subject of your book is from the field
%% "Humanities, Linguistics, Philosophy"
%%
%\bigskip
%%
%% Journal article
%\bibitem{humlinphil-journal} Alber John, Daniel C. O'Connell, and Sabine Kowal. 2002. Personal perspective in TV interviews. \textit{Pragmatics} 12:257--271
%%
%% Contribution
%\bibitem{humlinphil-contrib} Cameron, Deborah. 1997. Theoretical debates in feminist linguistics: Questions of sex and gender. In \textit{Gender and discourse}, ed. Ruth Wodak, 99--119. London: Sage Publications.
%%
%% Monograph
%\bibitem{humlinphil-mono} Cameron, Deborah. 1985. \textit{Feminism and linguistic theory.} New York: St. Martin's Press.
%%
%% Online Document
%\bibitem{humlinphil-online} Dod, Jake. 1999. Effective substances. In: The dictionary of substances and their effects. Royal Society of Chemistry. Available via DIALOG. \\
%http://www.rsc.org/dose/title of subordinate document. Cited 15 Jan 1999
%%
%% Journal article by DOI
%\bibitem{humlinphil-DOI} Suleiman, Camelia, Daniel C. O�Connell, and Sabine Kowal. 2002. `If you and I, if we, in this later day, lose that sacred fire...�': Perspective in political interviews. \textit{Journal of Psycholinguistic Research}. doi: 10.1023/A:1015592129296.
%%
%%
%%
%\bigskip
%%
%%
%% Use the following syntax and markup for your references if
%% the subject of your book is from the field
%% "Computer Science, Economics, Engineering, Geosciences, Life Sciences"
%%
%%
%% Contribution
%\bibitem{basic-contrib} Brown B, Aaron M (2001) The politics of nature. In: Smith J (ed) The rise of modern genomics, 3rd edn. Wiley, New York
%%
%% Online Document
%\bibitem{basic-online} Dod J (1999) Effective Substances. In: The dictionary of substances and their effects. Royal Society of Chemistry. Available via DIALOG. \\
%\url{http://www.rsc.org/dose/title of subordinate document. Cited 15 Jan 1999}
%%
%% Journal article by DOI
%\bibitem{basic-DOI} Slifka MK, Whitton JL (2000) Clinical implications of dysregulated cytokine production. J Mol Med, doi: 10.1007/s001090000086
%%
%% Journal article
%\bibitem{basic-journal} Smith J, Jones M Jr, Houghton L et al (1999) Future of health insurance. N Engl J Med 965:325--329
%%
%% Monograph
%\bibitem{basic-mono} South J, Blass B (2001) The future of modern genomics. Blackwell, London
%
\end{thebibliography}

\end{document}
\author{T.~Akutsu}% Tomotada Akutsu
\affiliation{National Astronomical Observatory of Japan, Tokyo 181-8588, Japan}
\author{M.~Ando}% Masaki Ando
\affiliation{National Astronomical Observatory of Japan, Tokyo 181-8588, Japan}
\affiliation{Research Center for the Early Universe (RESCEU), Graduate School of Science, The University of Tokyo, Tokyo 113-0033, Japan}
\affiliation{Department of Physics, Graduate School of Science, The University of Tokyo, Tokyo 113-0033, Japan}
\author{S.~Araki}% Sakae Araki
\affiliation{High Energy Accelerator Research Organization (KEK), Ibaraki 305-0801, Japan}
\author{A.~Araya}% Akito Araya
\affiliation{Earthquake Research Institute, The University of Tokyo, Tokyo 113-0032, Japan}
\author{T.~Arima}% Tsukasa Arima
\affiliation{Department of Physics, Graduate School of Science, Osaka City University, Osaka 558-8585, Japan}
\author{N.~Aritomi}% Naoki Aritomi
\affiliation{Department of Physics, Graduate School of Science, The University of Tokyo, Tokyo 113-0033, Japan}
\author{H.~Asada}% Hideki Asada
\affiliation{Graduate School of Science and Technology, Hirosaki University, Aomori 036-8561, Japan}
\author{Y.~Aso}% Yoichi Aso
\affiliation{National Astronomical Observatory of Japan, Tokyo 181-8588, Japan}
\author{S.~Atsuta}% Sho Atsuta
\affiliation{Department of Physics, Graduate School of Science and Engineering, Tokyo Institute of Technology, Tokyo 152-8551, Japan}
\author{K.~Awai}% Kyosuke Awai
\affiliation{Institute for Cosmic Ray Research, The University of Tokyo, Chiba 277-8582, Japan}
\affiliation{Institute for Cosmic Ray Research, The University of Tokyo, Gifu 506-1205, Japan}
\author{L.~Baiotti}% Luca Baiotti
\affiliation{Graduate School of Science, Osaka University, Osaka 560-0043, Japan}
\author{M.~A.~Barton}% Mark Barton
\affiliation{National Astronomical Observatory of Japan, Tokyo 181-8588, Japan}
\author{D.~Chen}% Dan Chen
\affiliation{Institute for Cosmic Ray Research, The University of Tokyo, Chiba 277-8582, Japan}
\author{K.~Cho}% Kyuman Cho
\affiliation{Department of Physics, Sogang University, Seoul 121-742, Korea}
\author{K.~Craig}% Kieran Craig
\affiliation{Institute for Cosmic Ray Research, The University of Tokyo, Chiba 277-8582, Japan}
\author{R.~DeSalvo}% Riccardo DeSalvo
\affiliation{The University of Sannio at Benevento, Benevento I-82100, Italy}
\affiliation{The National Institute for Nuclear Physics (INFN), Sezione di Napoli, Napoli I-80126, Italy}
\author{K.~Doi}% Kohei Doi
\affiliation{Institute for Cosmic Ray Research, The University of Tokyo, Chiba 277-8582, Japan}
\affiliation{Institute for Cosmic Ray Research, The University of Tokyo, Gifu 506-1205, Japan}
\affiliation{Grarduate School of Science and Engineering for Education (Science), University of Toyama, Toyama 930-8555, Japan}
\author{K.~Eda}% Kazunari Eda
\affiliation{Research Center for the Early Universe (RESCEU), Graduate School of Science, The University of Tokyo, Tokyo 113-0033, Japan}
\affiliation{Department of Physics, Graduate School of Science, The University of Tokyo, Tokyo 113-0033, Japan}
\author{Y.~Enomoto}% Yutaro Enomoto
\affiliation{Institute for Cosmic Ray Research, The University of Tokyo, Chiba 277-8582, Japan}
\author{R.~Flaminio}% Raffaele Flaminio
\affiliation{National Astronomical Observatory of Japan, Tokyo 181-8588, Japan}
\author{S.~Fujibayashi}% Sho Fujibayashi
\affiliation{Department of Physics, Graduate School of Science, Kyoto University, Kyoto 606-8502, Japan}
\author{Y.~Fujii}% Yoshinori Fujii
\affiliation{National Astronomical Observatory of Japan, Tokyo 181-8588, Japan}
\author{M.-K.~Fujimoto}% Masa-Katsu Fujimoto
\affiliation{National Astronomical Observatory of Japan, Tokyo 181-8588, Japan}
\author{M.~Fukushima}% Mitsuhiro Fukushima
\affiliation{National Astronomical Observatory of Japan, Tokyo 181-8588, Japan}
\author{T.~Furuhata}% Takayuki Furuhata
\affiliation{Grarduate School of Science and Engineering for Education (Science), University of Toyama, Toyama 930-8555, Japan}
\author{A.~Hagiwara}% Ayako Hagiwara
\affiliation{High Energy Accelerator Research Organization (KEK), Ibaraki 305-0801, Japan}
\author{S.~Haino}% Sadakazu Haino
\affiliation{Institute of Physics, Academia Sinica, Taipei 11529, Taiwan}
\author{S.~Harita}% Shohei Harita
\affiliation{Department of Physics, Graduate School of Science and Engineering, Tokyo Institute of Technology, Tokyo 152-8551, Japan}
\author{K.~Hasegawa}% Kunihiko Hasegawa
\affiliation{Institute for Cosmic Ray Research, The University of Tokyo, Chiba 277-8582, Japan}
\author{M.~Hasegawa}% Masaya Hasegawa
\affiliation{Graduate School of Science and Engineering for Education (Engineering), University of Toyama, Toyama 930-8555, Japan}
\author{K.~Hashino}% Katsuya Hashino
\affiliation{Grarduate School of Science and Engineering for Education (Science), University of Toyama, Toyama 930-8555, Japan}
\author{K.~Hayama}% Kazuhiro Hayama
\affiliation{Institute for Cosmic Ray Research, The University of Tokyo, Chiba 277-8582, Japan}
\affiliation{Institute for Cosmic Ray Research, The University of Tokyo, Gifu 506-1205, Japan}
\author{N.~Hirata}% Naoatsu Hirata
\affiliation{National Astronomical Observatory of Japan, Tokyo 181-8588, Japan}
\author{E.~Hirose}% Eiichi Hirose
\affiliation{Institute for Cosmic Ray Research, The University of Tokyo, Chiba 277-8582, Japan}
\affiliation{Institute for Cosmic Ray Research, The University of Tokyo, Gifu 506-1205, Japan}
\author{B.~Ikenoue}% Bungo Ikenoue
\affiliation{National Astronomical Observatory of Japan, Tokyo 181-8588, Japan}
\author{Y.~Inoue}% Yuki Inoue
\affiliation{Institute of Physics, Academia Sinica, Taipei 11529, Taiwan}
\author{K.~Ioka}% Kunihito Ioka
\affiliation{Yukawa Institute for Theoretical Physics, Kyoto University, Kyoto 606-8502, Japan}
\author{H.~Ishizaki}% Hideharu Ishizaki
\affiliation{National Astronomical Observatory of Japan, Tokyo 181-8588, Japan}
\author{Y.~Itoh}% Yousuke Itoh
\email{yousuke\_itoh@resceu.s.u-tokyo.ac.jp}
\affiliation{Research Center for the Early Universe (RESCEU), Graduate School of Science, The University of Tokyo, Tokyo 113-0033, Japan}
\author{D.~Jia}% Dongbao Jia
\affiliation{Graduate School of Science and Engineering for Education (Engineering), University of Toyama, Toyama 930-8555, Japan}
\author{T.~Kagawa}% Tomohiro Kagawa
\affiliation{Grarduate School of Science and Engineering for Education (Science), University of Toyama, Toyama 930-8555, Japan}
\author{T.~Kaji}% Tsuyoshi Kaji
\affiliation{Department of Physics, Graduate School of Science, Osaka City University, Osaka 558-8585, Japan}
\author{T.~Kajita}% Takaaki Kajita
\affiliation{Institute for Cosmic Ray Research, The University of Tokyo, Chiba 277-8582, Japan}
\affiliation{Institute for Cosmic Ray Research, The University of Tokyo, Gifu 506-1205, Japan}
\author{M.~Kakizaki}% Mitsuru Kakizaki
\affiliation{Grarduate School of Science and Engineering for Education (Science), University of Toyama, Toyama 930-8555, Japan}
\author{H.~Kakuhata}% Hiroshi Kakuhata
\affiliation{Graduate School of Science and Engineering for Education (Engineering), University of Toyama, Toyama 930-8555, Japan}
\author{M.~Kamiizumi}% Masahiro Kamiizumi
\affiliation{Institute for Cosmic Ray Research, The University of Tokyo, Chiba 277-8582, Japan}
\affiliation{Institute for Cosmic Ray Research, The University of Tokyo, Gifu 506-1205, Japan}
\author{S.~Kanbara}% Shogo Kanbara
\affiliation{Grarduate School of Science and Engineering for Education (Science), University of Toyama, Toyama 930-8555, Japan}
\author{N.~Kanda}% Nobuyuki Kanda
\affiliation{Department of Physics, Graduate School of Science, Osaka City University, Osaka 558-8585, Japan}
\author{S.~Kanemura}% Shinya Kanemura
\affiliation{Grarduate School of Science and Engineering for Education (Science), University of Toyama, Toyama 930-8555, Japan}
\author{M.~Kaneyama}% Masato Kaneyama
\affiliation{Department of Physics, Graduate School of Science, Osaka City University, Osaka 558-8585, Japan}
\author{J.~Kasuya}% Junko Kasuya
\affiliation{Department of Physics, Graduate School of Science and Engineering, Tokyo Institute of Technology, Tokyo 152-8551, Japan}
\author{Y.~Kataoka}% Yuu Kataoka
\affiliation{Department of Physics, Graduate School of Science and Engineering, Tokyo Institute of Technology, Tokyo 152-8551, Japan}
\author{K.~Kawaguchi}% Kyohei Kawaguchi
\affiliation{Yukawa Institute for Theoretical Physics, Kyoto University, Kyoto 606-8502, Japan}
\author{N.~Kawai}% Nobuyuki Kawai
\affiliation{Department of Physics, Graduate School of Science and Engineering, Tokyo Institute of Technology, Tokyo 152-8551, Japan}
\author{S.~Kawamura}% Seiji Kawamura
\affiliation{Institute for Cosmic Ray Research, The University of Tokyo, Chiba 277-8582, Japan}
\affiliation{Institute for Cosmic Ray Research, The University of Tokyo, Gifu 506-1205, Japan}
\author{F.~Kawazoe}% Fumiko Kawazoe
\affiliation{Albert-Einstein-Institut, Max-Planck-Institut f\"ur Gravitationsphysik, Hannover D-30167, Germany}
\author{C.~Kim}% Chunglee Kim
\affiliation{Department of Physics and Astronomy, Seoul National University, Seoul 151-742, Korea}
\affiliation{Korea Astronomy and Space Science Institute (KASI), Daejeon 34055, Korea}
\author{J.~Kim}% Jaewan Kim
\affiliation{Department of Physics, Myongji University, Yongin 449-728, Korea}
\author{J.~C.~Kim}% Jeongcho Kim
\affiliation{Department of Computer Simulation, Inje University, Gimhae-shi 50834, Korea}
\author{W.~Kim}% Whansun Kim
\affiliation{National Institute for Mathematical Sciences (NIMS), Daejeon 34047, Korea}
\author{N.~Kimura}% Nobuhiro Kimura
\affiliation{High Energy Accelerator Research Organization (KEK), Ibaraki 305-0801, Japan}
\affiliation{Institute for Cosmic Ray Research, The University of Tokyo, Chiba 277-8582, Japan}
\author{Y.~Kitaoka}% Yuichi Kitaoka
\affiliation{Department of Physics, Graduate School of Science, Osaka City University, Osaka 558-8585, Japan}
\author{K.~Kobayashi}% Kaori Kobayashi
\affiliation{Grarduate School of Science and Engineering for Education (Science), University of Toyama, Toyama 930-8555, Japan}
\author{Y.~Kojima}% Yasufumi Kojima
\affiliation{Department of Physical Science, Graduate School of Science, Hiroshima University, Hiroshima 903-0213, Japan}
\author{K.~Kokeyama}% Keiko Kokeyama
\affiliation{Institute for Cosmic Ray Research, The University of Tokyo, Chiba 277-8582, Japan}
\affiliation{Institute for Cosmic Ray Research, The University of Tokyo, Gifu 506-1205, Japan}
\author{K.~Komori}% Kentaro Komori
\affiliation{Department of Physics, Graduate School of Science, The University of Tokyo, Tokyo 113-0033, Japan}
\author{K.~Kotake}% Kei Kotake
\affiliation{Department of Applied Physics, Graduate School of Science, Fukuoka University, Fukuoka 814-0180, Japan}
\author{K.~Kubo}% Kai Kubo
\affiliation{Graduate School of Science and Engineering, Hosei University, Tokyo 184-8584, Japan}
\author{R.~Kumar}% Rahul Kumar
\affiliation{High Energy Accelerator Research Organization (KEK), Ibaraki 305-0801, Japan}
\author{T.~Kume}% Tatsuya Kume
\affiliation{High Energy Accelerator Research Organization (KEK), Ibaraki 305-0801, Japan}
\author{K.~Kuroda}% Kazuaki Kuroda
\affiliation{Institute for Cosmic Ray Research, The University of Tokyo, Chiba 277-8582, Japan}
\author{Y.~Kuwahara}% Yuya Kuwahara
\affiliation{Department of Physics, Graduate School of Science, The University of Tokyo, Tokyo 113-0033, Japan}
\author{H.-K.~Lee}% Hyun-Kyu Lee
\affiliation{Hanyang University, Seoul 133-791, Korea}
\author{H.-W.~Lee}% Hyung-Won Lee
\affiliation{Department of Computer Simulation, Inje University, Gimhae-shi 50834, Korea}
\author{C.-Y.~Lin}% Chun-Yu Lin
\affiliation{National Center for High-performance computing, National Applied Research Laboratories, Hsinchu 30076, Taiwan}
\author{Y.~Liu}% Yingtao Liu
\affiliation{Institute for Cosmic Ray Research, The University of Tokyo, Chiba 277-8582, Japan}
\author{E.~Majorana}% Ettore Majorana
\affiliation{The National Institute for Nuclear Physics (INFN), Sezione di Roma, Rome I-00185, Italy}
\author{S.~Mano}% Shuhei Mano
\affiliation{Department of Mathematical Analysis and Statistical Inference, The Institute of Statistical Mathematics, Tokyo 190-8562, Japan}
\author{M.~Marchio}% Manuel Marchio
\affiliation{National Astronomical Observatory of Japan, Tokyo 181-8588, Japan}
\author{T.~Matsui}% Toshinori Matsui
\affiliation{Grarduate School of Science and Engineering for Education (Science), University of Toyama, Toyama 930-8555, Japan}
\author{N.~Matsumoto}% Nobuyuki Matsumoto
\affiliation{Research Institute of Electrical Communication, Tohoku University, Miyagi 980-8577, Japan}
\affiliation{Frontier Research Institute for Interdisciplinary Sciences, Tohoku University, Miyagi 980-8577, Japan}
\author{F.~Matsushima}% Fusakazu Matsushima
\affiliation{Grarduate School of Science and Engineering for Education (Science), University of Toyama, Toyama 930-8555, Japan}
\author{Y.~Michimura}% Yuta Michimura
\affiliation{Department of Physics, Graduate School of Science, The University of Tokyo, Tokyo 113-0033, Japan}
\author{N.~Mio}% Norikatsu Mio
\affiliation{Photon Science Center, Graduate School of Engineering, The University of Tokyo, Tokyo 113-8656, Japan}
\author{O.~Miyakawa}% Osamu Miyakawa
\affiliation{Institute for Cosmic Ray Research, The University of Tokyo, Chiba 277-8582, Japan}
\affiliation{Institute for Cosmic Ray Research, The University of Tokyo, Gifu 506-1205, Japan}
\author{K.~Miyake}% Kyohei Miyake
\affiliation{Graduate School of Science and Engineering for Education (Engineering), University of Toyama, Toyama 930-8555, Japan}
\author{A.~Miyamoto}% Akinobu Miyamoto
\affiliation{Department of Physics, Graduate School of Science, Osaka City University, Osaka 558-8585, Japan}
\author{T.~Miyamoto}% Takahiro Miyamoto
\affiliation{Institute for Cosmic Ray Research, The University of Tokyo, Chiba 277-8582, Japan}
\affiliation{Institute for Cosmic Ray Research, The University of Tokyo, Gifu 506-1205, Japan}
\author{K.~Miyo}% Kouseki Miyo
\affiliation{Institute for Cosmic Ray Research, The University of Tokyo, Chiba 277-8582, Japan}
\author{S.~Miyoki}% Shinji Miyoki
\affiliation{Institute for Cosmic Ray Research, The University of Tokyo, Chiba 277-8582, Japan}
\affiliation{Institute for Cosmic Ray Research, The University of Tokyo, Gifu 506-1205, Japan}
\author{W.~Morii}% Wataru Morii
\affiliation{Disaster Prevention Research Institute, Kyoto University, Kyoto 611-0011,Japan}
\author{S.~Morisaki}% Souichiro Morisaki
\affiliation{Research Center for the Early Universe (RESCEU), Graduate School of Science, The University of Tokyo, Tokyo 113-0033, Japan}
\affiliation{Department of Physics, Graduate School of Science, The University of Tokyo, Tokyo 113-0033, Japan}
\author{Y.~Moriwaki}% Yoshiki Moriwaki
\affiliation{Grarduate School of Science and Engineering for Education (Science), University of Toyama, Toyama 930-8555, Japan}
\author{Y.~Muraki}% Yutaro Muraki
\affiliation{Department of Physics, Graduate School of Science and Engineering, Tokyo Institute of Technology, Tokyo 152-8551, Japan}
\author{M.~Murakoshi}% Moe Murakoshi
\affiliation{Graduate School of Science and Engineering, Hosei University, Tokyo 184-8584, Japan}
\author{M.~Musha}% Mitsuru Musha
\affiliation{Institute for Laser Science, The University of Electro-Communications, Tokyo 182-8585, Japan}
\author{K.~Nagano}% Koji Nagano
\affiliation{Institute for Cosmic Ray Research, The University of Tokyo, Chiba 277-8582, Japan}
\author{S.~Nagano}% Shigeo Nagano
\affiliation{The Applied Electromagnetic Research Institute, National Institute of Information and Communications Technology, Tokyo 184-8795, Japan}
\author{K.~Nakamura}% Kouji Nakamura
\affiliation{National Astronomical Observatory of Japan, Tokyo 181-8588, Japan}
\author{T.~Nakamura}% Takashi Nakamura
\affiliation{Department of Physics, Graduate School of Science, Kyoto University, Kyoto 606-8502, Japan}
\author{H.~Nakano}% Hiroyuki Nakano
\affiliation{Department of Physics, Graduate School of Science, Kyoto University, Kyoto 606-8502, Japan}
\author{M.~Nakano}% Masaya Nakano
\affiliation{Graduate School of Science and Engineering for Education (Engineering), University of Toyama, Toyama 930-8555, Japan}
\author{M.~Nakano}% Masayuki Nakano
\affiliation{Institute for Cosmic Ray Research, The University of Tokyo, Chiba 277-8582, Japan}
\affiliation{Institute for Cosmic Ray Research, The University of Tokyo, Gifu 506-1205, Japan}
\author{H.~Nakao}% Hayato Nakao
\affiliation{Department of Physics, Graduate School of Science, Osaka City University, Osaka 558-8585, Japan}
\author{K.~Nakao}% Ken-ichi Nakao
\affiliation{Department of Physics, Graduate School of Science, Osaka City University, Osaka 558-8585, Japan}
\author{T.~Narikawa}% Tatsuya Narikawa
\affiliation{Department of Physics, Graduate School of Science, Osaka City University, Osaka 558-8585, Japan}
\author{W.-T.~Ni}% Wei-Tou Ni
\affiliation{Department of Physics, National Tsing Hua University, Hsinchu 30013, Taiwan}
\affiliation{School of Optical Electrical and Computer Engineering, The University of Shanghai for Science and Technology, Shanghai 200093, China}
\author{T.~Nonomura}% Takuya Nonomura
\affiliation{Graduate School of Science and Engineering, Hosei University, Tokyo 184-8584, Japan}
\author{Y.~Obuchi}% Yoshiyuki Obuchi
\affiliation{National Astronomical Observatory of Japan, Tokyo 181-8588, Japan}
\author{J.~J.~Oh}% John Oh
\affiliation{National Institute for Mathematical Sciences (NIMS), Daejeon 34047, Korea}
\author{S.-H.~Oh}% Sang-Hoon Oh
\affiliation{National Institute for Mathematical Sciences (NIMS), Daejeon 34047, Korea}
\author{M.~Ohashi}% Masatake Ohashi
\affiliation{Institute for Cosmic Ray Research, The University of Tokyo, Chiba 277-8582, Japan}
\affiliation{Institute for Cosmic Ray Research, The University of Tokyo, Gifu 506-1205, Japan}
\author{N.~Ohishi}% Naoko Ohishi
\affiliation{National Astronomical Observatory of Japan, Tokyo 181-8588, Japan}
\affiliation{Institute for Cosmic Ray Research, The University of Tokyo, Gifu 506-1205, Japan}
\author{M.~Ohkawa}% Masashi Ohkawa
\affiliation{Graduate School of Science and Technology, Niigata University, Niigata 950-2181, Japan}
\author{N.~Ohmae}% Noriaki Ohmae
\affiliation{Photon Science Center, Graduate School of Engineering, The University of Tokyo, Tokyo 113-8656, Japan}
\author{K.~Okino}% Koji Okino
\affiliation{Information Technology Center, University of Toyama, Toyama 930-8555, Japan}
\author{K.~Okutomi}% Koki Okutomi
\affiliation{The Graduate University for Advanced Studies (SOKENDAI), Tokyo 181-8588, Japan}
\author{K.~Ono}% Kenji Ono
\affiliation{Institute for Cosmic Ray Research, The University of Tokyo, Chiba 277-8582, Japan}
\author{Y.~Ono}% Yukinori Ono
\affiliation{Research Institute of Electronics, Shizuoka University, Shizuoka 432-8011, Japan}
\author{K.~Oohara}% Ken-ichi Oohara
\affiliation{Graduate School of Science and Technology, Niigata University, Niigata 950-2181, Japan}
\author{S.~Ota}% Syoji Ota
\affiliation{Graduate School of Science and Engineering, Hosei University, Tokyo 184-8584, Japan}
\author{J.~Park}% Junegyu Park
\affiliation{Department of Physics, Sogang University, Seoul 121-742, Korea}
\author{F.~E.~Pe\~na Arellano}% Fabi\'an Erasmo Pe\~na Arellano
\affiliation{National Astronomical Observatory of Japan, Tokyo 181-8588, Japan}
\author{I.~M.~Pinto}% Innocenzo Pinto
\affiliation{The University of Sannio at Benevento, Benevento I-82100, Italy}
\affiliation{The National Institute for Nuclear Physics (INFN), Sezione di Napoli, Napoli I-80126, Italy}
\author{M.~Principe}% Maria Principe
\affiliation{The University of Sannio at Benevento, Benevento I-82100, Italy}
\affiliation{The National Institute for Nuclear Physics (INFN), Sezione di Napoli, Napoli I-80126, Italy}
\author{N.~Sago}% Norichika Sago
\affiliation{Faculty of Arts and Science, Kyushu University, Fukuoka 819-0395, Japan}
\author{M.~Saijo}% Motoyuki Saijo
\affiliation{Department of Pure and Applied Physics, Graduate School of Advanced Science and Engineering, Waseda University, Tokyo 169-8555, Japan}
\author{T.~Saito}% Takahiro Saito
\affiliation{Graduate School of Science and Technology, Niigata University, Niigata 950-2181, Japan}
\author{Y.~Saito}% Yoshio Saito
\affiliation{Institute for Cosmic Ray Research, The University of Tokyo, Chiba 277-8582, Japan}
\affiliation{Institute for Cosmic Ray Research, The University of Tokyo, Gifu 506-1205, Japan}
\author{S.~Saitou}% Sakae Saitou
\affiliation{National Astronomical Observatory of Japan, Tokyo 181-8588, Japan}
\author{K.~Sakai}% Kazuki Sakai
\affiliation{Department of Information Science and Control Engineering, Graduate School of Engineering, Nagaoka University of Technology, Niigata 940-2188, Japan}
\author{Y.~Sakakibara}% Yusuke Sakakibara
\affiliation{Institute for Cosmic Ray Research, The University of Tokyo, Chiba 277-8582, Japan}
\author{Y.~Sasaki}% Yukitsugu Sasaki
\affiliation{Department of Information \& Manegement Systems Engineering, Graduate School of Engineering, Nagaoka University of Technology, Niigata 940-2188, Japan}
\author{S.~Sato}% Shuichi Sato
\email{sato.shuichi@hosei.ac.jp}
\affiliation{Graduate School of Science and Engineering, Hosei University, Tokyo 184-8584, Japan}
\author{T.~Sato}% Takashi Sato
\affiliation{Graduate School of Science and Technology, Niigata University, Niigata 950-2181, Japan}
\author{Y.~Sato}% Yoshihiro Sato
\affiliation{High Energy Accelerator Research Organization (KEK), Ibaraki 305-0801, Japan}
\author{T.~Sekiguchi}% Takanori Sekiguchi
\affiliation{Institute for Cosmic Ray Research, The University of Tokyo, Chiba 277-8582, Japan}
\affiliation{Institute for Cosmic Ray Research, The University of Tokyo, Gifu 506-1205, Japan}
\author{Y.~Sekiguchi}% Yuichiro Sekiguchi
\affiliation{Graduate School of Science, Toho University, Chiba, Japan}
\author{M.~Shibata}% Masaru Shibata
\affiliation{Yukawa Institute for Theoretical Physics, Kyoto University, Kyoto 606-8502, Japan}
\author{K.~Shiga}% Kazunari Shiga
\affiliation{Graduate School of Science and Technology, Niigata University, Niigata 950-2181, Japan}
\author{Y.~Shikano}% Yutaka Shikano
\affiliation{Research Center of Integrative Molecular Systems (CIMoS), Institute for Molecular Science, Aichi 444-8585, Japan}
\affiliation{Institute for Quantum Studies, Chapman University, CA 92866, USA}
\author{T.~Shimoda}% Tomofumi Shimoda
\affiliation{Department of Physics, Graduate School of Science, The University of Tokyo, Tokyo 113-0033, Japan}
\author{H.~Shinkai}% Hisaaki Shinkai
\affiliation{Graduate School of Information Science and Technology, Osaka Institute of Technology, Osaka 573-0196, Japan}
\author{A.~Shoda}% Ayaka Shoda
\affiliation{National Astronomical Observatory of Japan, Tokyo 181-8588, Japan}
\author{N.~Someya}% Nozomi Someya
\affiliation{Graduate School of Science and Engineering, Hosei University, Tokyo 184-8584, Japan}
\author{K.~Somiya}% Kentaro Somiya
\email{somiya@phys.titech.ac.jp}
\affiliation{Department of Physics, Graduate School of Science and Engineering, Tokyo Institute of Technology, Tokyo 152-8551, Japan}
\author{E.~J.~Son}% Edwin Son
\affiliation{National Institute for Mathematical Sciences (NIMS), Daejeon 34047, Korea}
\author{T.~Starecki}% Tomasz Starecki
\affiliation{Institute of Electronic Systems, Warsaw University of Technology, Warsaw 00-665 , Poland}
\author{A.~Suemasa}% Aru Suemasa
\affiliation{Institute for Laser Science, The University of Electro-Communications, Tokyo 182-8585, Japan}
\author{Y.~Sugimoto}% Yusuke Sugimoto
\affiliation{Grarduate School of Science and Engineering for Education (Science), University of Toyama, Toyama 930-8555, Japan}
\author{Y.~Susa}% Yuki Susa
\affiliation{Department of Physics, Graduate School of Science and Engineering, Tokyo Institute of Technology, Tokyo 152-8551, Japan}
\author{H.~Suwabe}% Hiroto Suwabe
\affiliation{Graduate School of Science and Technology, Niigata University, Niigata 950-2181, Japan}
\author{T.~Suzuki}% Toshikazu Suzuki
\affiliation{High Energy Accelerator Research Organization (KEK), Ibaraki 305-0801, Japan}
\affiliation{Institute for Cosmic Ray Research, The University of Tokyo, Chiba 277-8582, Japan}
\author{Y.~Tachibana}% Yutaro Tachibana
\affiliation{Department of Physics, Graduate School of Science and Engineering, Tokyo Institute of Technology, Tokyo 152-8551, Japan}
\author{H.~Tagoshi}% Hideyuki Tagoshi
\affiliation{Department of Physics, Graduate School of Science, Osaka City University, Osaka 558-8585, Japan}
\author{S.~Takada}% Suguru Takada
\affiliation{The Device Engineering and Applied Physics Research Division, National Institute for Fusion Science, Gifu 509-5292, Japan}
\author{H.~Takahashi}% Hirotaka Takahashi
\affiliation{Department of Information \& Manegement Systems Engineering, Graduate School of Engineering, Nagaoka University of Technology, Niigata 940-2188, Japan}
\author{R.~Takahashi}% Ryutaro Takahashi
\affiliation{National Astronomical Observatory of Japan, Tokyo 181-8588, Japan}
\author{A.~Takamori}% Akiteru Takamori
\affiliation{Earthquake Research Institute, The University of Tokyo, Tokyo 113-0032, Japan}
\author{H.~Takeda}% Hiroki Takeda
\affiliation{Department of Physics, Graduate School of Science, The University of Tokyo, Tokyo 113-0033, Japan}
\author{H.~Tanaka}% Hiroki Tanaka
\affiliation{Institute for Cosmic Ray Research, The University of Tokyo, Chiba 277-8582, Japan}
\affiliation{Institute for Cosmic Ray Research, The University of Tokyo, Gifu 506-1205, Japan}
\author{K.~Tanaka}% Kazuyuki Tanaka
\affiliation{Department of Physics, Graduate School of Science, Osaka City University, Osaka 558-8585, Japan}
\author{T.~Tanaka}% Takahiro Tanaka
\affiliation{Department of Physics, Graduate School of Science, Kyoto University, Kyoto 606-8502, Japan}
\author{D.~Tatsumi}% Daisuke Tatsumi
\affiliation{National Astronomical Observatory of Japan, Tokyo 181-8588, Japan}
\author{S.~Telada}% Souichi Telada
\affiliation{Metrology Institute of Japan, National Institute of Advanced Industrial Science and Technology, Ibaraki 305-8568, Japan}
\author{T.~Tomaru}% Takayuki Tomaru
\affiliation{High Energy Accelerator Research Organization (KEK), Ibaraki 305-0801, Japan}
\affiliation{Institute for Cosmic Ray Research, The University of Tokyo, Chiba 277-8582, Japan}
\author{K.~Tsubono}% Kimio Tsubono
\affiliation{Department of Physics, Graduate School of Science, The University of Tokyo, Tokyo 113-0033, Japan}
\author{S.~Tsuchida}% Satoshi Tsuchida
\affiliation{Department of Physics, Graduate School of Science, Osaka City University, Osaka 558-8585, Japan}
\author{L.~Tsukada}% Leo Tsukada
\affiliation{Research Center for the Early Universe (RESCEU), Graduate School of Science, The University of Tokyo, Tokyo 113-0033, Japan}
\affiliation{Department of Physics, Graduate School of Science, The University of Tokyo, Tokyo 113-0033, Japan}
\author{T.~Tsuzuki}% Toshihiro Tsuzuki
\affiliation{National Astronomical Observatory of Japan, Tokyo 181-8588, Japan}
\author{N.~Uchikata}% Nami Uchikata
\affiliation{Department of Physics, Graduate School of Science, Osaka City University, Osaka 558-8585, Japan}
\author{T.~Uchiyama}% Takashi Uchiyama
\affiliation{Institute for Cosmic Ray Research, The University of Tokyo, Chiba 277-8582, Japan}
\affiliation{Institute for Cosmic Ray Research, The University of Tokyo, Gifu 506-1205, Japan}
\author{T.~Uehara}% Tomoyuki Uehara
\affiliation{Department of Communications Engineering, National Defense Academy of Japan, Kanagawa 239-8686, Japan}
\affiliation{Department of Physics, The University of Florida, FL 32611, USA}
\author{S.~Ueki}% Satoshi Ueki
\affiliation{Department of Information \& Manegement Systems Engineering, Graduate School of Engineering, Nagaoka University of Technology, Niigata 940-2188, Japan}
\author{K.~Ueno}% Koh Ueno
\affiliation{Department of Physics, The University of Wisconsin Milwaukee, WI 53201, USA}
\author{F.~Uraguchi}% Fumihiro Uraguchi
\affiliation{National Astronomical Observatory of Japan, Tokyo 181-8588, Japan}
\author{T.~Ushiba}% Takafumi Ushiba
\affiliation{Department of Physics, Graduate School of Science, The University of Tokyo, Tokyo 113-0033, Japan}
\author{M.~H.~P.~M.~van Putten}% Maurice van Putten
\affiliation{Sejong University, Seoul 143-747, Korea}
\affiliation{Kavli Institute for Theoretical Physics, The University of California, Santa Barbara, CA 93106-4030, USA}
\author{S.~Wada}% Shotaro Wada
\affiliation{Department of Physics, Graduate School of Science, The University of Tokyo, Tokyo 113-0033, Japan}
\author{T.~Wakamatsu}% Takashi Wakamatsu
\affiliation{Graduate School of Science and Technology, Niigata University, Niigata 950-2181, Japan}
\author{T.~Yaginuma}% Takuya Yaginuma
\affiliation{Department of Physics, Graduate School of Science and Engineering, Tokyo Institute of Technology, Tokyo 152-8551, Japan}
\author{K.~Yamamoto}% Kazuhiro Yamamoto
\affiliation{Institute for Cosmic Ray Research, The University of Tokyo, Chiba 277-8582, Japan}
\affiliation{Institute for Cosmic Ray Research, The University of Tokyo, Gifu 506-1205, Japan}
\author{S.~Yamamoto}% Shun Yamamoto
\affiliation{Graduate School of Information Science and Technology, Osaka Institute of Technology, Osaka 573-0196, Japan}
\author{T.~Yamamoto}% Takahiro Yamamoto
\affiliation{Institute for Cosmic Ray Research, The University of Tokyo, Chiba 277-8582, Japan}
\affiliation{Institute for Cosmic Ray Research, The University of Tokyo, Gifu 506-1205, Japan}
\author{K.~Yano}% Kazushiro Yano
\affiliation{Department of Physics, Graduate School of Science and Engineering, Tokyo Institute of Technology, Tokyo 152-8551, Japan}
\author{J.~Yokoyama}% Jun'ichi Yokoyama
\affiliation{Research Center for the Early Universe (RESCEU), Graduate School of Science, The University of Tokyo, Tokyo 113-0033, Japan}
\affiliation{Department of Physics, Graduate School of Science, The University of Tokyo, Tokyo 113-0033, Japan}
\affiliation{Kavli Institute for the Physics and Mathematics of the Universe (Kavli IPMU), The University of Tokyo, Chiba 277-8568, Japan}
\author{T.~Yokozawa}% Takaaki Yokozawa
\affiliation{Department of Physics, Graduate School of Science, Osaka City University, Osaka 558-8585, Japan}
\author{T.~H.~Yoon}% Tai Hyun Yoon
\affiliation{Department of Physics, Korea University, Seoul 136-701, Korea}
\author{H.~Yuzurihara}% Hirotaka Yuzurihara
\affiliation{Department of Physics, Graduate School of Science, Osaka City University, Osaka 558-8585, Japan}
\author{S.~Zeidler}% Simon Zeidler
\affiliation{National Astronomical Observatory of Japan, Tokyo 181-8588, Japan}
\author{Y.~Zhao}% Yuhang Zhao
\affiliation{Department of Astronomy, Beijing Normal University, Beijing 100875, China}
\author{L.~Zheng}% Lihe Zheng
\affiliation{Shanghai Institute of Ceramics, Chinese Academy of Sciences, Shanghai 200050, China}
%
%
\collaboration{KAGRA Collaboration}
\noaffiliation
%
%
\author{K.~Agatsuma}% Kazuhiro Agatsuma
\affiliation{Nikhef, Amsterdam 1098 XG, The Netherlands}
\author{Y.~Akiyama}% Yuta Akiyama
\affiliation{Faculty of Science and Engineering, Hosei University, Tokyo 184-8584, Japan}
\author{N.~Arai}% Nanako Arai
\affiliation{Faculty of Science and Engineering, Hosei University, Tokyo 184-8584, Japan}
\author{M.~Asano}% Mitsuhiro Asano
\affiliation{Department of Physics, Graduate School of Science, Osaka City University, Osaka 558-8585, Japan}
\author{A.~Bertolini}% Alessandro Bertolini
\affiliation{Nikhef, Amsterdam 1098 XG, The Netherlands}
\author{M.~Fujisawa}% Marie Fujisawa
\affiliation{Grarduate School of Science and Engineering for Education (Science), University of Toyama, Toyama 930-8555, Japan}
\author{R.~Goetz}% Ryan Goetz
\affiliation{Department of Physics, The University of Florida, FL 32611, USA}
\author{J.~Guscott}% Jake Guscott
\affiliation{Department of Physics, Faculty of Science, The University of Tokyo, Tokyo 113-0033, Japan}
\author{Y.~Hashimoto}% Yasuka Hashimoto
\affiliation{Faculty of Science and Engineering, Hosei University, Tokyo 184-8584, Japan}
\author{Y.~Hayashida}% Yuri Hayashida
\affiliation{Astronomical Institute, Graduate School of Science, Tohoku University, Miyagi 980-8578, Japan}
\author{E.~Hennes}% Eric Hennes
\affiliation{Nikhef, Amsterdam 1098 XG, The Netherlands}
\author{K.~Hirai}% Kyohei Hirai
\affiliation{Graduate School of Science and Technology, Niigata University, Niigata 950-2181, Japan}
\author{T.~Hirayama}% Tetsu Hirayama
\affiliation{Faculty of Science and Engineering, Hosei University, Tokyo 184-8584, Japan}
\author{H.~Ishitsuka}% Hideki Ishitsuka
\affiliation{Institute for Cosmic Ray Research, The University of Tokyo, Chiba 277-8582, Japan}
\affiliation{Institute for Cosmic Ray Research, The University of Tokyo, Gifu 506-1205, Japan}
\author{J.~Kato}% Jumpei Kato
\affiliation{Department of Physics, Graduate School of Science and Engineering, Tokyo Institute of Technology, Tokyo 152-8551, Japan}
\author{A.~Khalaidovski}% Alexander Khalaidovski
\affiliation{Institute for Cosmic Ray Research, The University of Tokyo, Chiba 277-8582, Japan}
\affiliation{Albert-Einstein-Institut, Max-Planck-Institut f\"ur Gravitationsphysik, Hannover D-30167, Germany}
\author{S.~Koike}% Shigeaki Koike
\affiliation{High Energy Accelerator Research Organization (KEK), Ibaraki 305-0801, Japan}
\author{A.~Kumeta}% Ayaka Kumeta
\affiliation{Department of Physics, Graduate School of Science and Engineering, Tokyo Institute of Technology, Tokyo 152-8551, Japan}
\author{T.~Miener}% Tjark Miener
\affiliation{Albert-Einstein-Institut, Max-Planck-Institut f\"ur Gravitationsphysik, Hannover D-30167, Germany}
\author{M.~Morioka}% Masayo Morioka
\affiliation{Astronomical Institute, Graduate School of Science, Tohoku University, Miyagi 980-8578, Japan}
\author{C.~L.~Mueller}% Chris Mueller
\affiliation{Department of Physics, The University of Florida, FL 32611, USA}
\author{T.~Narita}% Takashi Narita
\affiliation{Graduate School of Science and Technology, Niigata University, Niigata 950-2181, Japan}
\author{Y.~Oda}% Yusuke Oda
\affiliation{Division of Biology and Geosciences, Graduate School of Science, Osaka City University, Osaka 558-8585, Japan}
\author{T.~Ogawa}% Tsutomu Ogawa
\affiliation{Earthquake Research Institute, The University of Tokyo, Tokyo 113-0032, Japan}
\author{T.~Sekiguchi}% Tetsuya Sekiguchi
\affiliation{Faculty of Science and Engineering, Hosei University, Tokyo 184-8584, Japan}
\author{H.~Tamura}% Hiroki Tamura
\affiliation{Grarduate School of Science and Engineering for Education (Science), University of Toyama, Toyama 930-8555, Japan}
\author{D.~B.~Tanner}% David Tanner
\affiliation{Department of Physics, The University of Florida, FL 32611, USA}
\author{C.~Tokoku}% Chihiro Tokoku
\affiliation{Institute of Space and Astronautical Science, Japan Aerospace Exploration Agency, Kanagawa 252-5210, Japan}
\author{M.~Toritani}% Masato Toritani
\affiliation{Department of Physics, Graduate School of Science, Osaka City University, Osaka 558-8585, Japan}
\author{T.~Utsuki}% Tomomi Utsuki
\affiliation{Faculty of Science and Engineering, Hosei University, Tokyo 184-8584, Japan}
\author{M.~Uyeshima}% Makoto Uyeshima
\affiliation{Earthquake Research Institute, The University of Tokyo, Tokyo 113-0032, Japan}
\author{J.~van den Brand}% Jo van den Brand
\affiliation{Nikhef, Amsterdam 1098 XG, The Netherlands}
\author{J.~van Heijningen}% Joris van Heijningen
\affiliation{Nikhef, Amsterdam 1098 XG, The Netherlands}
\author{S.~Yamaguchi}% Satoru Yamaguchi
\affiliation{Division of Biology and Geosciences, Graduate School of Science, Osaka City University, Osaka 558-8585, Japan}
\author{A.~Yanagida}% Akihiro Yanagida
\affiliation{Faculty of Science and Engineering, Hosei University, Tokyo 184-8584, Japan}



\begin{abstract}
Major construction and initial-phase operation of a second-generation gravitational-wave detector KAGRA has been completed. The entire 3-km detector is installed underground in a mine in order to be isolated from background seismic vibrations on the surface. This allows us to achieve a good sensitivity at low frequencies and high stability of the detector. Bare-bones equipment for the interferometer operation has been installed and the first test run was accomplished in March and April of 2016 with a rather simple configuration. The initial configuration of KAGRA is named {\it iKAGRA}. In this paper, we summarize the construction of KAGRA, including the study of the advantages and challenges of building an underground detector and the operation of the iKAGRA interferometer together with the geophysics interferometer that has been constructed in the same tunnel.
\end{abstract}

\maketitle


\section{Introduction}
%\section{Introduction}

Capturing the motion of the human body pose has great values in widespread applications, such as movement analysis, human-computer interaction, films making, digital avatar animation, and virtual reality. 
Traditional marker-based motion capture system can acquire accurate movement information of humans, but is only applicable to limited scenes due to the time-consuming fitting process and prohibitively expensive costs. 
In contrast, markerless motion capture based on RGB image and video processing algorithms is a promising alternative that has attracted numerous research in the fields of deep learning and computer vision. 
%encoded with prior knowledge of realistic human pose and shape deforming
Especially, thanks to the parameteric SMPL model~\citep{smpl:loper2015smpl} and various diverse datasets with 3D annotations~\citep{h36m:ionescu2013human3, mpii3d:mehta2017monocular, 3dpw:von2018recovering}, remarkable progress has been made on monocular 3D human pose and shape estimation and motion capture. 
%such as~\citep{hmr:kanazawa2018end, vibe:kocabas2020vibe,  maed:wan2021encoder, romp:sun2021monocular, spin:kolotouros2019learning, pare:kocabas2021pare}.




Existing regression-based human mesh recovery methods are actually implicitly based on an assumption that predicting 3d body joint rotations and human shape strongly depends on the given image features. 
The pose and shape parameters are directly estimated from the image feature using MLP regressors. 
Nevertheless, due to the inherent ambiguity, the mapping from the 2D image feature to 3D pose and shape is an ill-posed problem.
To achieve accurate pose and shape estimation, it is necessary to initialize the mean pose and shape parameters and use an \textit{iterative residual regression} manner to reduce error. 
Such an end-to-end learning and inference scheme~\citep{hmr:kanazawa2018end} has been proven to be effective in practice, but ignores the temporal information and produces implausible human motions and unsatisfactory pose jitters for video streaming data. 
Video-based methods such as~\citep{hmmr:kanazawa2019learning,vibe:kocabas2020vibe, tcmr:choi2021beyond, mps-net:wei2022capturing} may leverage large-scale motion capture data as priors and exploit temporal information among different frames to penalize implausible motions. 
 They usually enhance singe-frame feature using a temporal encoder and then still use a deep regressor to predict SMPL parameters based on the temporally enhanced image feature,  as shown in the left subfigure of Fig.\ref{fig:temporal_contrast}. 
This scheme, however, is unable to focus on joint-level rotational motion specific to each joint, failing to ensure the temporal consistency of local joints. 
To address these problems, we attempt to understand the human 3D reconstruction from a causal perspective. 
We argue that assuming a still background, the primary causes behind the image pixel changes and human body appearance changes are 1) the motions of 3D joint rotations in human skeletal dynamics and 2) the viewpoint changes of the observer (camera). 
In fact, a prior human body model exists independently of a given specific image. And the 3D relative rotations of all joints (relative to the parent joint) and body shape can be abstracted beyond image pixels and independent of the image contents and observer views. In other words, the joint rotations cannot be ``seen'' and they are image-independent and viewpoint-independent concepts.  %Additionally, the motion of the combined pose is huge and highly complicated, while the rotation motion of each separate joint is more tracable and periodic.

\begin{figure}
	\centering
	\includegraphics[width=0.95\linewidth]{figures/simple_contrast.pdf}
	\caption{\textbf{Left:} Mainstream temporal-based human mesh methods, e.g.~\citep{hmmr:kanazawa2019learning,vibe:kocabas2020vibe, tcmr:choi2021beyond}, adopt a temporal encoder to mix temporal information from past and future frames and then regress the SMPL parameters from the \textit{temporally enhanced feature} for each frame. \textbf{Right:} Our method first acquires tokens of each joint in the time dimension and then separately capture the motion of each joint using a shared temporal encoder.
	}
	\label{fig:temporal_contrast}\vspace{-0.2in}
\end{figure}

Based on the considerations above, we propose a novel 3D human pose and shape estimation model based on \textbf{in}dependent \textbf{t}okens (INT). 
The core idea of the model is to introduce three types of independent tokens that specifically encode the 3D rotation information of every joint, the shape of human body and the information about camera. 
These initialized tokens learn prior knowledge and mutual relationships from large-scale training data, requiring neither an iterative regressor to take mean shape and pose as initial estimate~\citep{hmr:kanazawa2018end, spin:kolotouros2019learning, vibe:kocabas2020vibe, tcmr:choi2021beyond}, nor a kinematic topology decoder defined by human prior knowledge~\citep{maed:wan2021encoder} . 
Given an image as a conditional observation, these tokens are repeatedly updated by interacting with 2D image evidence using a Transformer~\citep{transformer:vaswani2017attention}. 
Finally, they are transformed into the posterior estimates of pose, shape and camera parameters. 
As a consequence, this method of abstracting joint rotation tokens from image pixels can represent the motion state of each joint and establish correlations in time dimension.
 Benefiting from this, we can separately capture the temporal rotational motion of every joint by sending the tokens of each joint at different timestamps to a temporal model.
In comparison to capturing the overall temporal changes in image features and the whole pose, this modeling scheme focuses on capturing separate rotational motions of all joints, which is conducive to maintaining the temporal coherence and rationality of each joint rotation.

We evaluate our model on the challenging 3DPW~\citep{3dpw:von2018recovering} benchmark and Human3.6m~\citep{h36m:ionescu2013human3}. 
Using vanilla ResNet-50 and Transformer architectures, our model obtains 42.0 mm error in PA-MPJPE metric for 3DPW, outperforming all state-of-the-art counterparts with a large margin. The same model obtains 38.4 mm error in PA-MPJPE metric for Human3.6m, which is on par with the state-of-the-art methods. 
Also, the qualitative results show that our model produces accurate pixel-alignment human mesh reconstructions for indoor or in-the-wild images, and shows fewer motion jitters in local joints when processing video data. 
We strongly encourage the readers to see the video results in the supplementary materials for reference and comparison.

 %simple and straightforward, We first and then extend it to capture separate rotational motion of each joint.

% 可视化prior learned pose and shape



Gravitational waves are ripples of spacetime radiated from dynamic motions of massive and compact 
objects, such as black holes and neutron stars, or from spacetime fluctuations in the early universe.
As gravitational waves pass though matter almost unobstructed, their observation can tell us many 
details of the nature of their sources that cannot be obtained by electromagnetic waves or other cosmic radiation.
The existence of gravitational waves was theoretically predicted by Albert Einstein in 1916. They were
not directly detected for about 100 years until the first discovery by Advanced LIGO, 
the gravitational wave observatory in the US, in 2015\,\cite{aLIGO}. 
Recently, a European gravitational
wave antenna (Advanced Virgo\,\cite{Virgo}) has begun observing and in the next few years, a Japanese
gravitational wave antenna KAGRA\,\cite{LCGT,KAGRA} will be in operation.
The new era of gravitational wave astronomy will begin with network observations by these antennas;
we can expect new astronomical knowledge with an increasing number of detected events, better parameter 
estimation accuracy for the sources (sky localization, mass, spin, distance, and orbital parameters),
and detection of gravitational waves from completely new types of sources.
These antennas (Advanced LIGO,  Advanced Virgo, and KAGRA) are called ``second-generation'' detectors,
having been upgraded from or constructed after the ``first-generation'' gravitational wave detectors
\,\cite{ref-LIGO, ref-VIRGO, ref-GEO, ref-TAMA}.

While scientists are making intense efforts towards the development of second-generation detectors, 
design studies for future antennas with better sensitivities, called  third-generation gravitational wave detectors, have begun. 
One proposal is the European antenna called the Einstein Telescope (ET)\,\cite{ET}. 
Another proposal in the US for a third-generation antenna is called Cosmic Explorer (CE)\,\cite{CE}. 
Though both are designed based on a Michelson-type laser interferometer like the second-generation detectors,
there are some differences in the design concepts.
ET will be a triangular array of three interferometers with a 10-km baseline length, and the entire system 
will be underground. 
CE will be a conventional L-shape interferometer, built either on the Earth's surface or underground, with a baseline 10 times longer than the second-generation detectors.
One of the largest differences in these design concepts is the site selection: underground or surface. 
An underground site is advantageous for gravitational wave detectors; seismic motions are smaller than a surface
site by a few orders of magnitude in typical cases. This fact is critical for the sensitivity to low-frequency gravitational wave sources, 
and long-term stable operation of a sensitive interferometer.
On the other hand, an underground site is disadvantageous from the point of view of construction cost,
requiring excavation of long tunnels and caverns for an interferometer and the vacuum system to house it.
In addition, operating a large-scale interferometer at an underground site creates new challenges 
from several practical points of view.

KAGRA~\cite{KAGRAname}, one of the second-generation detectors, is the world's first large-scale gravitational wave antenna constructed underground. 
It is located in the Kamioka underground site at Gifu Prefecture in Japan. 
At this site,  two prototype detectors, LISM (a 20-m scale interferometer, started in 1999\,\cite{LISM}) 
and  CLIO (a 100-m scale interferometer, started in 2002\,{\cite{CLIO}\cite{Agatsuma}), were constructed and operated to show 
the advantages of an underground site.
With these achievements and experiences, KAGRA was funded in 2010 and the excavation of the tunnel was started in May 2012.
In October 2015, most of the installation activities for the initial KAGRA interferometer had been completed, and
after commissioning work, the interferometer was operated  for the first time in March 2016.
Though the interferometer configuration at that time was  simplified from the final KAGRA design,
KAGRA was operated for the first time as a full 3-km-scale interferometer connected to  the data acquisition, 
transfer, and storage system.
With that configuration, we carried out a three-week test run to check the overall performance 
as an interferometric gravitational wave antenna system.

In this paper, we describe the construction and the initial-phase operation of the world's first kilometer-scale underground gravitational-wave detector, and discuss the advantages and the challenges of going underground.
In Sec.~\ref{sec:2}, we briefly explain the history of our site search and review the construction process. In Sec.~\ref{sec:3} and Sec.~\ref{sec:4}, we list advantages and challenges of building a gravitational-wave detector under the ground, which will serve as a useful reference in the planning of next-generation gravitational-wave detectors. In Sec.~\ref{sec:5}, we show the results from operation of our two underground detectors, iKAGRA itself and a geophysics interferometer built next to it in the same tunnel. In Sec.~\ref{sec:6}, we summarize the work.


\section{Construction of KAGRA}\label{sec:2}
%\input{2construction}
\begin{figure}
\begin{minipage}{8.6cm}
\includegraphics[width=8.6cm]{minemap2}
\end{minipage} 
\caption{\label{minemap2}A map of the Kamioka mine. The blue thick lines are baselines of the KAGRA interferometer. The red points are the measurement points of the seismic motion. (See Sec.~\ref{sec:seismic}.)}
\end{figure}

The site search for KAGRA was performed in the late 90's. We consulted a company (Sumitomo Corporation, Japan) to compare the geological and environmental conditions of several candidates in Japan. One of the candidates was Mt. Tsukuba in Ibaraki Prefecture, but the cost estimate was twice as high as the Kamioka mine. Another candidate was the Kamaishi mine in Iwate Prefecture, but there was a railroad tunnel near the site. The bedrock in Kamioka, Hida gneiss, was no doubt one of the best, the estimated cost was comparatively low, and we had already started an optical experiment near the famous Super-Kamiokande that showed very low influence from seismic motion. It was a natural choice to build KAGRA in the Kamioka mine.

The excavation of the KAGRA tunnels started in 2012. Figure~\ref{minemap2} shows a map around the Kamioka mine. The geographical coordinates of the beamsplitter are 36.41 degrees North and 137.31 degrees East. The Y-arm is in the direction 28.31 degrees west of north. Though the central station and the end stations are rather close to the foot of the 1300-m high mountain, they are still at least 200\,m below the ground surface. The X and Y arms are 3-km long. Access tunnels lead to the central station and to the Y-arm end station. See Reference~\cite{UchiyamaCQG} for more details about the tunnel excavation of the KAGRA site.

A Japanese construction company, Kajima Corporation, completed the two 3-km tunnels and the central/end stations within two years (May 2012 - May 2014), making it the fastest-ever excavation work. 
Installation of the facilities proceeded in parallel with the construction work to meet timelines. This included electricity, air conditioning, water supply, anti-dust wall painting, floor treatment, crane setup, anchors, spiral steps, networks, telecommunication, laser clean room, etc. See Fig.~\ref{fig:photo1} for some photos. Coping with leaking water in the mine took time but the water issues were finally settled.
Soon after the tunnel and facility neared completion in March 2014, installation of the vacuum system and interferometer components (the vibration-isolation systems, the laser source, the optics for the interferometer, the digital control system, and so on) were started. 
\begin{figure}[t]
  \begin{center}
   \includegraphics[width=85mm]{kagraphoto.eps}
  \end{center}
\caption{\label{fig:photo1}Some photos in the KAGRA facility. {\it Top Left}: installation of cryostat underneath the vertical borehole for the 2-story suspension tower. {\it Top Right}: installation of the granite stone for the geophysics interferometer. {\it Bottom}: panorama of the KAGRA tunnels from the central station right after paving the floor.}
\end{figure}

The KAGRA project has been split into two phases. In the first phase, called the {\it initial-phase KAGRA}, or {\it iKAGRA}, the interferometer configuration is a simple Michelson interferometer that consists of two end test masses and a beamsplitter. In the second phase, the interferometer configuration is a Michelson interferometer with cryogenic Fabry-Perot optical resonators in the two arms and also with power- and signal-recycling cavities. The detector in this stage is called the {\it baseline-design KAGRA}, or {\it bKAGRA}~\cite{KAGRA}. 

While the goal of bKAGRA is frequent observations of gravitational waves from various sources, the main goal of iKAGRA was to integrate basic subsystems and to operate the detector. It is one of the most important milestones in the KAGRA project to realize the world's first kilometer-scale underground gravitational-wave detector.

In addition to the 3-km interferometer for detecting gravitational waves, we have also built a 1.5-km interferometer in one of the KAGRA tunnels. The purpose of this interferometer is to measure Earth's strain for seismic and geodetic observations in geoscience and to feed the information back to KAGRA to characterize the behavior of the detector and to correct arm length variation for control and data analysis. It is planned to build another 1.5\,km interferometer in the other tunnel, this pair of Michelson interferometers is called a geophysics interferometer (GIF). More details about GIF are shown in Sec.~\ref{sec:GIF}.

\section{Advantages of the underground detector}\label{sec:3}
There are two main advantages to building a gravitational-wave detector in an underground site. One is its low seismic noise, which not only improves the sensitivity in the observation band but also increases the stability of the detector so that the requirements on the control system are eased. The second advantage is its low gravity gradient noise, which is one of the largest obstacles to improving the low-frequency sensitivity of next-generation detectors. In this section, we first discuss seismic noise and gravity gradient noise in the observation band. We then discuss seismic noise outside the observation band, which includes a microseismic motion at around 0.2\,Hz (caused by ocean waves) and the stability of the detector.

\subsection{Low seismic noise}\label{sec:seismic}
%\input{3seismic}
Prior to the excavation of the KAGRA tunnels, we investigated the seismic motion in the Kamioka mine to make a final decision of the location of the interferometer~\cite{Nishitani}. Araya {\it et al.} developed a low noise accelerometer and measured the seismic motion nearly at the center of the Kamioka mine~\cite{Araya}. The measurement revealed that the seismic motion was two orders of magnitude smaller than the typical seismic motion in a suburban area. Specifically, the power spectrum density was about $10^{-9}$m/$\sqrt{\rm Hz}$ at 1\,Hz. The measurements were repeated at the LISM site~\cite{Sato} and the CLIO site~\cite{Tomaru}, both of which were also near the center of the mine, and the results were similar. 

The locations of the KAGRA test masses need to be rather close to the foot of the mountain because the interferometer size is comparable to the mountain itself. We performed a measurement to investigate the dependence of the seismic motion as a function of the distance from the mountain surface in May 2005~{\cite{Yamamoto}\cite{Nishitani}}. The seismic motions at the various points marked red in Fig.~\ref{minemap2} including two mine entrances; Atotsu and Mozumi (south side and northwest side of the mine, respectively) were measured. The Mozumi tunnel runs straight from the Mozumi entrance to the center of the mine. We measured the seismic motion in this tunnel to investigate the position dependence of the seismic motion (50~m and 100~m from the entrance), and also measured the seismic motion at the CLIO site as a reference.

The top panel of Fig.~\ref{insideoutside} shows a typical horizontal seismic motion at each entrance of the mine. The vertical seismic motion is similar to the horizontal. Below 1~Hz, the seismic motion is comparable to that at the center of the mine (CLIO). Above 10~Hz, however, the seismic motion becomes comparable to that at Kashiwa, a suburban area northeast of Tokyo. The bottom panel of Fig.~\ref{insideoutside} shows the horizontal seismic motion in the Mozumi tunnel. The seismic motion is sufficiently small, even when the distance from the entrance is only 50~m. This means that being far from an urban area does not help reducing seismic noise in the observation band of the gravitational wave detector, but being  {\it underground} is essential for a low seismic noise level. Taking into account a safety factor, we locate all the four KAGRA test masses at least 200\,m from the surface of the mountain.

\begin{figure}
\begin{minipage}{7cm}
\includegraphics[width=7cm]{outside}
\includegraphics[width=7cm]{inside}
\end{minipage} 
\caption{\label{insideoutside}{\it Top}: Typical horizontal seismic motion measured outside of Kamioka mine (Atotsu (red) and Mozumi (green) entrances). {\it Bottom}: Typical horizontal seismic motion measured in the Mozumi tunnel (the distances from the entrance are shown). The seismic motion at the CLIO site (the center of the mine, black solid) and outside of the Mozumi entrance office (black dashed) are also shown as references. It must be noted that the spectrum density at the CLIO site is mainly limited by sensor noise. In other words, the black solid lines show upper limits of seismic noise.}
\end{figure}

\subsection{Low gravity gradient noise}
%\input{3GGN}
\begin{figure}[htbp]
	\begin{center}
		\includegraphics[scale=0.6]{NN.eps}
		\caption{\label{fig:NN}Calculated gravity gradient noise in KAGRA, estimated from seismic motion measured at the CLIO site. It is smaller than that in Advanced LIGO~\cite{JanLIGO} and it does not limit the sensitivity of KAGRA in the observation band.}
	\end{center}
\end{figure}

Seismic activities change the gravitational potential around the test masses, which causes them to move. This noise is called gravity gradient noise. The source of this noise can be vibration of the ground surface, mainly due to human activity, or motion of the ground as a bulk. For a ground-based gravitational-wave detector, gravity gradient noise from the ground surface tends to be larger. Unlike the direct coupling of vibration, gravity gradient noise cannot be attenuated directly using any shield or attenuation systems. Gravity gradient noise can be a limiting noise source of ground-based gravitational wave detectors at the lower end of the observation band.
In order to reduce gravity gradient noise, it is effective to go underground since the gravity gradient noise caused by fluctuations of the ground surface is reduced \cite{Beker2011}.
Figure~\ref{fig:NN} shows the calculated gravity gradient noise spectra of the KAGRA detector. The calculation, provided by J.~Harms, is based on the seismic displacement of the CLIO site, which is very similar to that at the KAGRA site below 20 Hz, as measured by a group from NIKHEF~\cite{NIKHEFmeasurement}. The noise level is about 1 order of magnitude smaller than other sites such as LIGO and Virgo. 

There exist other kinds of gravity gradient noise in addition to that from the seismic motion. One is atmospheric gravity gradient noise~\cite{airGGN}. 
This should be decreased underground since the gravitational potential perturbation from air fluctuations  outside the building will be small. 
Another is gravity gradient noise from the water flow near the detector. This can be an issue for KAGRA as the amount of underground water in the Kamioka mine is enormous. The effect of gravity gradient noise from the water flow is currently under investigation (see App.~\ref{sec:waterGGN} for more detail).

\subsection{Stability}
%\input{3stability}
While we mainly discussed seismic noise in the observation band of the detector in Sec.~\ref{sec:seismic}, the low seismic motion in Kamioka at lower frequencies also helps with stable operation of the detector. One advantage will be a high duty factor. 
Another advantage is that the loop gains of the control circuits can be kept small. In this section we discuss these two advantages and we also introduce the stability of temperature, humidity, and air pressure in the mine.

To accomplish multiple detections of gravitational waves, the duty factor of KAGRA, namely the ratio of the observation time to the entire time of the observation period, has to be sufficiently high. In KAGRA, we will have a longer startup time than other detectors because of the need to cool down and to heat up the cryogenic test masses. On the other hand, we can expect less frequent interruption of the operation since the detector is isolated from background seismic vibrations, which are responsible for most of the interruptions. The advantage of being underground is significant above $\sim1\,\mathrm{Hz}$, which will certainly help the stable operation of the interferometer. LISM, a 20-m prototype locked Fabry-Perot detector in the same mine, demonstrated 270 hours of continuous operation~\cite{LISM}. We can expect the same for KAGRA.

One of the largest technical noise sources in a gravitational-wave detector is control loop noise from auxiliary degrees of freedom (e.g. power- and signal-recycling cavity lengths). The loop noise in the observation band mainly comes from sensing noise of the controller that drives the recycling mirrors and beam splitter, and then couples to the gravitational-wave channel~\cite{loopnoise}. With the low seismic motion in the underground site, the control gain for the auxiliary degrees of freedom can be reduced. The control band of the auxiliary degrees of freedom of KAGRA is required to be lower than 20--50\,Hz in order to avoid loop noise at higher frequencies. This could be challenging but is possible in KAGRA, thanks to the low seismic motion at low frequencies in the underground site.

\begin{figure}[htbp]
	\begin{center}
		\includegraphics[scale=0.6]{microseis.eps}
		\caption{\label{fig:microseis}Low-frequency seismic noise in the Kamioka mine in the horizontal direction.}
		\label{microseis}
	\end{center}
\end{figure}
It should be noted, however, that the microseismic motion at around 0.2\,Hz is not low at the KAGRA site since it is not far from the Sea of Japan (about 40\,km to Toyama Bay), but is still lower than that in LIGO~\cite{LIGOseis}. Figure~\ref{microseis} shows the low-frequency seismic noise level in the horizontal direction. (The noise level in the vertical direction is almost identical.) The measurement was done by Beker et al. (NIKHEF) in 2010 using a Trillium 240 seismometer~\cite{NIKHEFmeasurement}. The same measurement was performed by Sekiguchi (ICRR) in 2014~\cite{Sekiguchi}. The blue curve shows the median of the measured seismic noise level. The red curve and green curve show the 90\,\% and 10\,\% deciles. 
%It was found in the CLIO experiment that the increase of microseismic noise below 1\,Hz due to high ocean waves made it difficult to lock the interferometer in bad weather~\cite{YamamotoAmaldi}.

CLIO performed a one-week observation run from Feb. 12--Feb. 18, 2007. Figure~\ref{fig:weatherCLIO} shows the duty factor of CLIO together with microseismic level and meteorological parameters (wind speed and air pressure outside the mine) during that week. The microseismic level was obtained from the maximum amplitude of horizontal ground velocity at the site within a 10-minute interval after a band-pass filter (0.11 - 0.5 Hz) was applied and the interval was shifted by 5 minutes. The meteorological parameters were shifted and normalized by peak-to-peak amplitudes of the period. We can see a clear correlation between the duty factor and the microseismic level while not much between the duty factor and meteorological parameters. This result implies that the underground detector is well isolated from the weather on the surface and affected mainly by ground motion originating from microseisms. The detail of the observation is shown in the CLIO elog~\cite{CLIOelog}.

\begin{figure}[htbp]
\centering
		\includegraphics[width=8.5cm]{weatherCLIO.eps}
		\caption{\label{fig:weatherCLIO}CLIO duty factor compared to microseismic level and various meteorological parameters. The meteorological parameters are obtained from the Japan Meteorological Agency website and are shifted and normalized by the peak-to-peak value in the week. The duty factor of each day is shown in the middle of the day (noon).}
\end{figure}


\begin{figure}[htbp]
\centering
		\includegraphics[width=7cm]{Tinside.eps}\\
\hspace{-0.7cm}
		\includegraphics[width=6.4cm]{Toutside.eps}
		\caption{\label{fig:ArayaData}{\it Top}: the variation of the temperature, humidity, and air pressure in the KAGRA mine, measured in February 2016. {\it Bottom}: typical variation of the temperature in the experimental room located on surface.}
\end{figure}

The low variation of the room temperature will also help to simplify the temperature control of the whole instruments. The top panel of Fig.~\ref{fig:ArayaData} shows the variation of the temperature, humidity, and air pressure at the central area of KAGRA. The daily temperature variation over one week is as low as 0.1 degrees Celsius, while in the surface experimental room typical temperature variation (in the bottom panel) is ten times and needs to be conditioned if instruments are maintained in the constant temperature.


Small temperature variation will also help stable operation. Figure~\ref{fig:ArayaData} shows the temperature, humidity, and air pressure at the central area of KAGRA over a 1 week period. The daily temperature variation is as low as 0.1 degrees Celsius. The variation of the air pressure inside the tunnel is at the same level as the outside of the tunnel.


\section{Challenges to build the underground detector}\label{sec:4}
Offsetting the advantages we can expect by going underground, we have found various challenges in building a detector in an underground facility. In this section, we report those challenges, which we hope will be of benefit to the next-generation gravitational-wave detectors.

\subsection{GPS}
%\input{4timing}
KAGRA is operated using a digital real-time control system with high-speed 
front-end computers (operating at a sampling frequency of 64\,kHz that is down-converted to 16\,kHz) and a reflective memory system that enables fast data cloning between computers that are 3\,km apart in the tunnel. In order to operate the digital system, it is important to synchronize the clocks in the computers. We use the Global Positioning System (GPS) for the synchronization but since KAGRA is inside the mine, the GPS signal does not reach the KAGRA site directly.

The GPS antenna is placed on the roof of the Hokubu-kaikan building that is 
located next to the KAGRA office outside the mine. The GPS signal is transmitted to a receiver in the Hokubu-kaikan building through an approximately 10-m electrical cable. The receiver is called the timing master and has signal receiver, splitter and transmitter functions with a delay compensation system for consistent timing across multiple items of equipment. Some of the split signals are used at the Hokubu-kaikan building to synchronize servers or other timing clients. One of the signals from the timing master is transmitted into the KAGRA mine through an approximately 7-km optical fiber cable. Inside the mine, another receiver, called the timing fanout and located in the computer room, receives the signal from the timing master. 
However, the distance of 7\,km existing between the outside and inside leads to a big time delay. The timing master and fanout establish the connection by transmitting or receiving sample signals during the first cycle of booting to determine the delay between the master and fanout. Compensation is applied between the master/fanout and the attached timing receivers. Timing receivers are implemented in the IO chassis of computers to synchronize the timing of servers and the analog-to-digital/digital-to-analog converters (ADC/DAC). The central area and both end areas are connected by a timing network with 3 timing fanouts through 3\,km optical fiber cables, each with the same compensation for the delay. Therefore all the equipment in the mine is synchronized to the GPS antenna with negligible delay.

We extract 1\,PPS signals, 65536\,Hz signals and IRIG-B signals from the GPS signal (PPS=Pulse-Per-Second, IRIG=Inter-Range-Instrumentation-Group time code). The 1\,PPS signals and 65536\,Hz signals are used to synchronize all the ADCs and DACs. The ADCs and DACs know the beginning of every 1 second interval accurately but they do not know the absolute time. The IRIG-B signal is used to synchronize all the front-end computers to absolute time. Using both 1\,PPS and IRIG-B signals tells the ADCs and DACs both the synchronization and absolute time.

\subsection{Underground water}\label{sec:water}
%\input{4water}

Water flow from the aquifer is an unavoidable issue in such a vast underground environment. 
The total amount of water flowing along the Y-arm of the KAGRA
tunnel exceeds 1,200 tons per hour in early springtime due
to melting of the heavy wintertime snow, while the amount is as little as 200\,tons per hour in summer and autumn.

It takes 2 to 7 days for rain water to start increasing the amount of underground water in the tunnel. 
The floor of the KAGRA tunnel is tilted by 1/300 in order to drain the underground water without active pumping.
The highest point is the X-end station and the lowest point is the Y-end station, so the water flows naturally from the X-end to the central station and then from the central station to the Y-end.
There are two fault crossings in the KAGRA tunnel.
One is the Ikenoyama fault crossing located in the Y-arm at around the 1800\,m point from the central station and the other is the North-20th fault crossing in the X-arm at around the 2500\,m point from the central station. (See Fig.~\ref{fig:faultcrossings}.)
While the underground water wells up throughout the tunnel, it wells up the most from these fault crossings.
In the Y-arm tunnel, we have a drainage pipe 40\,cm in diameter under the concrete floor, but this passive system alone cannot drain away all the underground water, so we have constructed a pair of additional pipes with drain pumps. 
The compulsory drainage pipes run from the 1-km point and the 2-km point toward the end of the Y-arm tunnel. 
The drain water is sent to a water reservoir located 200\,m before the Y-arm end station and then to another tunnel (owned by a local company) running 11\,m underneath the KAGRA tunnel.
This compulsory drainage system decreases the amount of water flow in the 40\,cm drainage pipe.
In the X-arm tunnel, we have increased the diameter of sections of the drainage pipe to 60\,cm so that all the underground water from the X-arm can be drained to the central station.
The drainage pipe bypasses the central station and leads to the Atotsu exit of the mine.
After all the handlings, no flood will happen in the KAGRA tunnel. Raindrops could be seen but will be handled one by one.

\begin{figure}[htbp]
	\begin{center}
		\includegraphics[scale=0.3, angle=90]{tunnel_map.eps}
		\caption{\label{fig:faultcrossings}KAGRA tunnel schematics. The Ikenoyama fault crossing is in the Y-arm about 1800\,m away from the center and the North-20th fault crossing is in the X-arm about 2500\,m away from the center (the red and blue arrows indicate the directions of the tunnel excavation).}
	\end{center}
\end{figure}

\subsection{Gravel layers}\label{sec:gravel}
%\input{4gravel}
The KAGRA tunnel was excavated by NATM (New Austrian Tunneling Method), which involves blasting the tunnel and cementing the surface wall. To make the flat tunnel floor, a layer of concrete about 200\,mm thick was cast above the gravel layer on the bedrock. Because the gravel layer tends to be deformed by heavy load, a concrete layer without gravel was placed directly on the bedrock below the KAGRA vacuum chambers and in addition the chambers were anchored to the bedrock. Even with such a structure, a long-term drift of the concrete layer causes a tilt of the chambers. 
In the iKAGRA operation, the upper limit of the drift rate of the chambers is estimated to be $<4\times10^{-6}$rad/day from the arm length data. This is manageable since we have an alignment control system of the mirrors which typically has a $\sim$mrad range. It should be noted that the drift rate will decrease in time especially if this drift is caused by the excavation of the tunnel.
GIF requires lower instrumental drift than KAGRA because long-term strain changes including a strain accumulation of a tectonic deformation need to be measured precisely. For this reason, GIF chambers are mounted on granite blocks that are directly attached to the bedrock.  Gravel was removed in part and the exposed surfaces of the bedrock were smoothed to settle the granite blocks; thicknesses of the granite blocks were 568--1100mm, depending on the depth of bedrock below the floor.

\subsection{Cleanliness and high humidity}
%\input{4cleanliness}

During the construction phase, the underground facility was pervaded with the exhaust gas from heavy machinery. The current cleanliness in the facility, however, is kept at ISO Class 6 even outside a clean booth, thanks to anti-dust paint on the wall.
As for the clean booths, KAGRA has defined two clean categories. One is an ISO class 1 category and the other is an ISO class 4 category. The former is applied only for the laser source area and the four cryostats that house the sapphire test masses. The latter is for the vacuum chambers that house silica mirrors and other optics. The system is designed to replace all of the air inside the clean booths within 2 minutes. KOACH filters manufactured by KOKEN Ltd. in Japan achieve the ISO Class 1. The KOACH filters use two kinds of filters sequentially. One, a normal HEPA filter, is a pre-filter and the other is an innovative {\it Ferina} filter that has smaller gaps made by entwining thinner fibers than in ULPA filters. Consequently, the KOACH filters are able to catch 0.1\,$\mu\mathrm{m}$ and larger-diameter particles without much constriction of the air flow. Another innovative technology in the KOACH filter is to generate parallel air flow from any position on the output surface of the filter. This flow can minimize random air movements such as vortexes that would allow particles to float for longer inside the clean area.

Another concern inside the underground facility is its high humidity. We send a total of $900\,\mathrm{m}^3/\mathrm{h}$ of dry air into the center area and the two end stations. The current relative humidity in the arm tunnel is $97\sim99\,\%$ with the temperature $17\sim18^\circ$C. During the installation of the vacuum ducts, we set up a small clean booth to house the connecting part and provided HEPA-filtered dry air using 4 heatless adsorption air driers to lower the humidity inside the booth to $70\,\%$ or less. During the assembly work on the cryostat, we provided ULPA-filtered dry air to lower the humidity inside the cryostat to $30\,\%$. (See Fig.~\ref{fig:cryohumidity}.)

\begin{figure}[htbp]
	\begin{center}
		\includegraphics[scale=0.5]{cryohumidity.eps}
		\caption{\label{fig:cryohumidity}Humidity inside and outside the cryostat during the assembly work. While the temperature increased, the humidity inside the cryostat was kept at 30\,\%. On the first day there was a power outage and the humidity increased for 1.5 days. Note that the humidity is different from that in Fig.~\ref{fig:ArayaData} since the temperature in the central area increased after we enclosed the facility and turned on some machines in the area.}
	\end{center}
\end{figure}


\subsection{Safety and workability}
%\input{4workability}
The working environment in KAGRA has limitations due to the nature of the underground environment. The highest priority is safety. Working time is limited to the hours of 9:00--17:00, with a morning meeting held every weekday at 9:00 to discuss the schedule and to remind the on-site researchers of working precautions. For work in any area of the underground site, we mandate a buddy system, so that we can monitor the health of coworkers and help each other in case of an emergency. To monitor who is in the mine, every time we enter the tunnel, we put our nametags on a white board both in the main entrance and in the KAGRA office. (See Fig.~\ref{minemap2}.) We remove the tags when we exit the tunnel. We are planning to start using an automatic entrance monitoring system with our ID cards instead of manually putting up our nametags. Furthermore, in each underground area (the central area, X-end station and Y-end station), one safety leader is assigned. The safety leader has to carry an oxygen and carbon monoxide (CO) sensor to monitor the atmospheric conditions in the working area. In the past, during the construction work, we encountered a high-CO situation when we had to stop work. The safety leader also has to make sure no one is left in the mine alone when work is done and workers exit the tunnel.

% working style
It is mandatory to wear personal protective equipment such as helmets, work clothes with long sleeves, reflective vests and headlights (Fig.~\ref{fig:workstyle}). One of the disadvantages of the underground site is that this heavy equipment can hamper sensitive and technical tasks. Organic solvents are not allowed to be brought into the tunnel, which can also limit the efficiency of the work. When ultrasonic cleaning is required, one needs to go to University of Toyama, one of the KAGRA collaborative institutions where ultrasonic cleaning facilities are located. It takes one hour to drive from the KAGRA site to the university. In the near future a small amount of fluorinated solvents will be available in the mine, as we have recently installed a ventilation system that will exhaust solvent gas from the central area via a drafter.

\begin{figure}[htbp]
\begin{center}
%\includegraphics[width=8cm]{workstyle.JPG}
\includegraphics[width=8cm]{workstyle.eps}
\caption{Each worker is equipped with a helmet, working cloth, reflective vest and headlight for work inside the tunnel. Small electric carts are used for the 3-km transport from the central area to the end station.}
\label{fig:workstyle}
\end{center}
\end{figure}

% office-mine transport
The other limitation is the transportation to the mine from the KAGRA office. It takes about 10--15 minutes to drive from the KAGRA office to the tunnel entrance, via very narrow mountain roads. Only electric or diesel vehicles are allowed to enter the tunnel.  
Since a limited number of vehicles are available at once, we need to schedule the vehicles when we carry heavy loads into or out of the tunnel. 
Typically from December to early April, the roads to the mine entrance are covered by deep snow.
Snow-plows are operated by Hida city to clear away snow every morning, however, sometimes when it snows heavily, our cars parked in front of the entrance are buried in the snow, and we need to shovel out the cars to drive back. We are planning to install a snow melting system with water flow on the roads and parking area in front of the tunnel entrance.
In any case, we need to be extra cautious about transportation to and from the mine in the snow seasons.

% mine transport
For transportation inside the mine, we use bicycles with electric assistance and also small electric carts (Mitsuoka Motor, Like-T3, shown in Fig.~\ref{fig:workstyle}). It takes about 15 minutes to bike or drive from the central area to each end station. 
%Each end station has temporary restrooms, but practically, many people go back to the central area for better restrooms taking 30 minutes for return because otherwise they need to bring their own waste to the central area for disposal. It is another inconvenient side of the underground environment.

% emergency
While there is an old mine tunnel connecting between the outside and the Y-end station as an emergency exit, there is no emergency exit for the X-end station. In case of an emergency, emergency rooms on the second floor of the Xend station contain oxygen, tents, temporary restrooms, water, food, and bicycles are located.  These supplies will allow five people to survive at least three days while awaiting rescue. We are considering excavating an emergency tunnel from the X-end station, or, if that is not possible, at least a ventilation pipe. The Personal Handy-phone System (PHS), fire alarm and monitoring system are available anywhere inside the tunnel, including the X- and Y-arms.

Another issue in the Kamioka area is Black bears. These bears inhabit the mountains and can attack people when they are encountered especially in autumn before they hibernate. We have set up a music speakerphone at the entrance of the KAGRA mine and remind researchers not to walk around outside a building in autumn unless needed.


\section{Operation of the underground detector}\label{sec:5}
We carried out a test run of iKAGRA to verify its overall performance. Success of this operation means that we have overcome the challenges to build the underground detector. It is the most important milestone toward the completion of bKAGRA, and to observing gravitational waves. In addition to iKAGRA, we performed a test run of GIF; though the GIF test run was performed a few months after the test operation of iKAGRA, we discuss some results of the GIF operation here as well.
\subsection{iKAGRA}\label{sec:iKAGRA}
%\input{2iKAGRA}
By February 2016, we had completed the installation of the vacuum systems, clean booths, seismic isolation systems, and digital control systems to control the interferometer. In March 2016, we aligned the optics to make a 3-km Michelson interferometer, and started the test run operation. The test run was split into two periods, from March 25 to March 31 and from April 11 to April 25. Between the first and the second half of the test run, we had a little interval to improve the sensitivity and the stability of the interferometer.

\begin{figure}[htbp]
	\begin{center}
		\includegraphics[width=80mm]{iKAGRAConfig.eps}
		\caption{\label{fig:iKAGRAConfig}Interferometer configuration of iKAGRA.}
	\end{center}
\end{figure}

The interferometer configuration of iKAGRA is shown in Fig.~\ref{fig:iKAGRAConfig}. We used a single-frequency laser source with a wavelength of 1064\,nm. The laser beam was first injected into a triangular cavity called the pre-mode cleaner (PMC) and then to the suspended input mode cleaner (IMC), to suppress unwanted spatial higher order modes. The PMC was constructed from three mirrors rigidly attached to an Invar spacer. It has a round-trip length of 0.4\,m and a finesse of 197. The IMC was constructed from three mirrors suspended by a double pendulum on a vacuum-compatible vibration isolation stack table. It has a round-trip length of 53.3\,m and finesse of 540. The frequency of the laser was stabilized to the PMC length below 0.3\,mHz and the IMC length above 30\,Hz. 
The beam at IMC and the main interferometer is s-polarized. The incident power before the IMC was 270mW.
%%%polarization was mentioned (YM on June 20, 2017)%%%

The beam transmitted by the IMC is then sent to an input faraday isolator (IFI)~\cite{IFI}, after which the beam radius is expanded by the mirrors called PR2 and PR3. PR2 and PR3 stand for power recycling 2 and 3, and will be used for the folding mirrors of the power recycling cavity in the bKAGRA phase. The main 3-km Michelson interferometer consists of the beam splitter (BS) and two end test masses (ETMX and ETMY). The designed distance between BS and ETMX is 2991.6\,m, and that between BS and ETMY is 2988.3\,m. The IFI was put on a vibration isolation stack table, and PR2 was fixed on a non-isolated table. PR3, BS, ETMX and ETMY are suspended by double pendula. Of all these suspended mirrors, PR3 was the largest (250\,mm diameter and 100\,mm thick) mirror, and was the only one with a full-sized suspension, which will also be used in the bKAGRA phase with minor modifications.
%%%vacuum level was mentioned (YM on June 20, 2017), explanations on PR (YM on July 3, 2017)%%%

All of the optics downstream of the IMC, including the IMC, were placed inside vacuum chambers, but most of the parts were left at atmospheric pressure to allow for easier maintenance of the optics during the test run. Only the IMC and the two 3-km arm ducts are evacuated, to a pressure of $\sim10^2$\,Pa. The differential length signal of the Michelson interferometer was obtained from the reflected beam picked off by the IFI with a photo-diode in air.
%%%at "normal" pressure (YM on July 3, 2017)%%%

The differential length signal was fed back to the ETMs via 3-km reflective memory network to lock the interferometer fringe. During the first half of the test run, we used the DC signal from the photo-diode and locked the interferometer at the mid-fringe. During the second half, we used the pre-modulation technique to lock the interferometer at the dark fringe. The dark-fringe locking resulted in a lower coupling of laser intensity noise. We also increased the bandwidth of the servo from 8\,Hz to 94\,Hz by avoiding parasitic resonances of the ETMs. In the first half of the test run, we did not have enough balancing of the mirror actuators, which resulted in the excitation of the suspension pitch modes at around 15\,Hz. 
Moreover, the servo bandwidth was automatically adjusted by monitoring the loop gain using calibration lines in the second test run. These improvements contributed to the high stability of the interferometer. The duty factor during the first half was 85.2\,\%, whereas that during the second half was 90.4\,\%. The longest lock stretch was 3.6 hours for the first half, and that for the second half was 21.3 hours. The high duty factor compared to other gravitational wave telescopes is mainly due to the simple configuration of the interferometer, but the state machine automaton {\it Guardian}~\cite{guardian_iKAGRA} also helped recovering lock quickly after lock losses.

%%%more explanation on UGF servo added, removed RFM acronym (See Sec IV-1), removed duty factor explanation since already done in Sec III-3 (YM on June 20, 2017)%%%

\begin{figure}[htbp]
	\begin{center}
		\includegraphics[width=80mm]{iKAGRASensitivity.eps}
		\caption{\label{fig:iKAGRASensitivity}The strain sensitivity of iKAGRA during the second run. An expected seismic noise level in bKAGRA is also plotted as a reference. (The noise level is about one order of magnitude smaller than other sites such as LIGO and Virgo~\cite{Rana}.) The peaks at 80\,Hz and 113\,Hz for the measured spectrum are from the calibration lines. The acoustic noise plotted here only shows the acoustic noise coupled via the BS chamber, but it is likely that the acoustic noise is the sensitivity limiting noise source also at neighboring frequencies. Actuator noise is the sum of the displacements of the mirrors from electronics noise for the actuation. Frequency noise is the estimated laser frequency noise suppressed through laser frequency stabilization using PMC and IMC. Seismic noise is the ground displacement attenuated by the mirror suspensions. Sensor noise is the sum of the ADC noise, the dark noise of the photo-diode, and the shot noise.}
	\end{center}
\end{figure}

A typical sensitivity curve for iKAGRA during the second run is shown in Fig.~\ref{fig:iKAGRASensitivity}. Below 3\,Hz, the sensitivity is limited by seismic noise. Over 100\,Hz to 3\,kHz, the sensitivity is likely to be limited by acoustic noise, mainly from the fans of the clean booths. The acoustic noise in iKAGRA was high because main mirrors were inside the vacuum chambers but at atmospheric pressure. Note that the acoustic noise curve in Fig.~\ref{fig:iKAGRASensitivity} only shows the acoustic noise coupled via the BS chamber. We turned off all the fans in the clean booths to confirm that the sensitivity of the frequency regions where the measured spectrum and the sum of all the known noise sources do not match is also limited by acoustic noise.
% by turning off all the fans of the clean booths. 
However, the detailed coupling mechanism still needs further investigation. At 3-5\,kHz, the sensitivity was limited by sensor noise, which mainly comes from the analog-to-digital conversion of the photo-diode output. These noise contributions will be further reduced in the bKAGRA phase by using better vibration isolation systems, a high vacuum system, and better whitening filters.

%% added by somiya start
\begin{figure}[htbp]
	\begin{center}
		\includegraphics[width=85mm]{seis_spectrogram_all.eps}
		\caption{\label{fig:spectrogram}Spectrogram of seismic noise in the 0.1\,Hz$\sim$2\,Hz band during the second run. One can see large spikes on April 14th and 16th due to a series of large earthquakes in Kumamoto. Some data on April 17th and 21st is missing due to an issue in our data taking system.}
	\end{center}
\end{figure}

Figure~\ref{fig:spectrogram} shows a spectrogram of seismic noise in the 0.1\,Hz$\sim$2\,Hz band during the second run. In this period, a series of large earthquakes hit Kumamoto Prefecture that is about 700\,km from KAGRA. According to the United States Geological Survey, the foreshock with a magnitude 6.5 hit Kumamoto at 12:26 UTC on April 14 (2016) and the mainshock with a magnitude 7.3 hit the same area at 16:25 UTC on April 16 (2016). The peak ground acceleration (PGA) levels in the horizontal direction measured at various points in KAGRA are as follows. For the foreshock, PGAs at BS, ETMX, and ETMY were 20\,mGal, 30\,mGal, and 25\,mGal, respectively. For the mainshock, the PGA at BS was not measured as the seismometer was saturated (i.e. over 95\,mGal) as the sensor gain had been set higher than the other two. PGAs at ETMX and ETMY were 306\,mGal and 298\,mGal, respectively. After the mainshock, the suspension system of BS was in trouble and it took half a day to fix it. The earthquake shook the BS suspension and a screw to support a mirror before releasing touched one of the suspension fibers of the BS. The interferometer was not in operation during the time when several aftershock earthquakes hit Kumamoto.

%% added by somiya end

During the test run, 65 people in total participated in shifts to monitor the interferometer conditions. The interferometer was monitored by at least three people on eight-hour shifts. A cumulative total of 186 people from 35 institutes contributed. The members of the shift monitored the status of the interferometer lock, mirror suspensions, data acquisition system and data transfer system to ensure that the interferometer data was properly transferred.

The data including the gravitational wave signal and environmental monitor signals were transferred to the data center at ICRR in Kashiwa and Osaka City University with latency of about 3 seconds. The amount of data was 7.5\,TB in total. We also did a hardware injection test right after the test run. The data management system and the results of the data analysis and the detector characterization will be published in separate papers.

\subsection{Geophysics interferometer}\label{sec:GIF}
%\input{2GIF}
\begin{figure}[htbp]
\centering
		\includegraphics[width=9cm]{GIF.eps}
		\caption{\label{fig:GIF}Schematic view of one of the two asymmetric Michelson interferometers of GIF.}
\end{figure}
GIF is a pair of 1.5\,km asymmetric Michelson interferometers that are to be installed in parallel with the arms of the 3-km KAGRA interferometer (Fig.~\ref{fig:GIF}). Installation of one of the GIF arms, GIF-X along the X-arm of 
KAGRA, was finished in August 2016. Two vacuum chambers, which are fixed to granite 
blocks stably installed on the bedrock as described in Sec.~\ref{sec:gravel}, house retro reflectors to 
form an interferometer. A frequency-doubled Nd:YAG laser ($\lambda$=532\,nm) is used 
as a light source. Its frequency is stabilized to an iodine-saturated absorption spectrum 
to obtain relative frequency stability $\delta\nu/\nu\simeq10^{-13}$ in Allan variance for time intervals of 10 to 1000\,s~\cite{araya1}, which corresponds 
to strain detectability $\varepsilon=\delta L/L\simeq 10^{-13}$, where $L$ is the baseline length.

The test run of GIF-X started in September 2016~\cite{GIF2017}. Fringe intensities of the quadrature interferometer 
were sampled at 50 kHz and the optical phase was calculated to obtain the variation of distance between 
two retro-reflectors separated by 1.5\,km. Typical strain variation of earth tides was clearly 
observed with GIF-X; however a slight reduction in amplitude, about 90\% of the theoretical tidal strain 
calculated by GOTIC2 including solid tides and ocean loads \cite{araya2}, was apparent. It is thought to be a 
topographic effect of the underground site judging from the similar reduction in the CLIO site \cite{araya3, araya4}. 
Strain resolution of GIF-X was estimated to be $10^{-12}- 10^{-10}$ depending on the frequency. 
Abrupt strain changes, such as might occur due to surface weather, could rarely be seen, with the exception of distant earthquakes. GIF-X mounted on the bedrock measures precise ground strain which cannot be fully predicted, for example, by standard tidal models and typical barometric response of the ground; the data are expected to reflect actual baseline of KAGRA and are necessary for the accurate characterization and correction of the arm length variation.

\section{Summary}\label{sec:6}
%\input{6summary}
In this paper, we described the construction of the world's first kilometer-scale underground gravitational-wave detector, KAGRA, together with a long-baseline geophysics interferometer built in the same tunnel. We learned many lessons during the construction and we explained the advantages and challenges of going underground. We performed a first test operation of KAGRA in a simple configuration and the operation was successful. The detector will be upgraded to the cryogenic interferometer and start scientific observations in the next few years, in which the advantages of building the detector in such a quiet place under the ground will be realized.

\appendix*
\section{Possible additional issues of an underground detector}
There are also some possible additional issues in an underground site. Here we briefly discuss two of these issues.
\subsection{Gravity gradient noise of the water flow}
\label{sec:waterGGN}
%\input{5waterGGN}
While the gravity gradient noise from the seismic motion is small in the underground site, a very large amount of the underground water is flowing around the interferometer and the gravity gradient noise from the underground water can be an issue in KAGRA. As explained in Sec.~\ref{sec:water}, the underground water goes through drain pipes under the inclined tunnels, starting from the X-end station. Most of the test mass chambers are far away from the water pipes. Our largest concern is the Y-end chamber, under which we have a drainage pipe. Figure~\ref{fig:waterseismic} shows that the seismic noise level increases near the drain pipe, which was measured before we added the compulsory drainage pipe.
\begin{figure}[htbp]
	\begin{center}
		\includegraphics[width=9cm]{waterseismic.eps}
		\caption{\label{fig:waterseismic}Seismic noise measured at two locations near the Y-end chamber; one is close to the drain pipe and the other far away from the drain pipe.}
	\end{center}
\end{figure}

Chen and Somiya made very rough models of the water flow~{\cite{YanbeiWaterGGN,SomiyaWaterGGN}}, and from these models they were not able to rule out the influence of water gravity-gradient noise on KAGRA. Further refinements of these models based on careful observations of the water flow are necessary. 

\subsection{Environmental magnetic fields}\label{sec:app2}
%\input{5magnetic}
Environmental magnetic fields produce noise in KAGRA through the coil-magnet actuators that drive the suspended mirrors in the control system. While the magnetic noise level is known to be lower than other noise curves in the sensitivity spectrum of KAGRA, the global electromagnetic resonance in the cavity between Earth's surface and the ionosphere produces globally correlated noise components, which hamper a search of stochastic background of gravitational waves. This globally correlated magnetic noise, called Schumann resonance, is excited by lightning discharges. The Schumann resonance will be accumulated in a correlation search for the stochastic gravitational-wave background and can limit the sensitivity of global gravitational-wave detector network below $100\,\mathrm{Hz}$.

We have investigated if this kind of environmental magnetic field could be attenuated in the underground site. First we calculated the skin effect with the bedrock properties of the Kamioka mine. The electrical conductivity is $2.5\times10^{-3}\,\mathrm{\Omega}^{-1}\mathrm{m}^{-1}$ and the magnetic permeability is $1.2\times10^{-6}\,\mathrm{H/m}$. The reduction rate of the magnetic field amplitude due to the skin effect was calculated to be 0.9 at 10\,Hz and 0.7 at 100\,Hz.
We then measured the magnetic field inside and outside of the Kamioka mine on July 21-22 2016.
We arranged a set of magnetometers in the parking area in front of the KAGRA entrance and another set at the second floor of the central room inside the mine.
The magnetometers used were Metronix MFS\mbox{-}06 for the North\mbox{-}South (N\mbox{-}S) and East\mbox{-}West (E\mbox{-}W) directions and Metronix MFS\mbox{-}07e for the vertical (Z) direction.
The data were logged by an Metronix ADU\mbox{-}07e. The data were taken from on 5:00, 21 \mbox{-} July, 2016 UTC to 7:00 on the 22nd., and 7:30 \mbox{-} 8:30 22 July, 2016 UTC.
The left and right panels in Fig.~\ref{fig:mag_out} show the spectrum of the magnetic fields inside and outside the mine, respectively. The x\mbox{-}axis represents the frequency in units of Hz and the y\mbox{-}axis represents the amplitude spectrum density in units of $\mathrm{pT}/\sqrt{\mathrm{Hz}}$. The Schumann resonance can be clearly seen from its 1st mode at $7.8\,\mathrm{Hz}$ up to the 7th mode at around $50\,\mathrm{Hz}$. The coherence between the magnetic field inside and outside of the mine is known to be as high as 0.9 at the Schumann resonances~\cite{atsuta}.

\begin{figure}
\begin{center}
\includegraphics[width=0.8\linewidth]{spectrum_inside.eps}
\includegraphics[width=0.8\linewidth]{spectrum_outside.eps}
\vspace{-0.2cm}
\caption{Spectrum of the magnetic field. The left and right panel show the spectrum measured inside and outside the KAGRA mine, respectively.}
\label{fig:mag_out}
\end{center}
\end{figure}

Figure \ref{fig:mag_in} shows the ratio of the spectrum of the magnetic fields inside and outside of the mine. One can see that the magnetic field inside the mine is larger than that outside the mine across a broad frequency band. In regard to the significant amplification of the magnitude along the Z direction, we can not deny the possibility that the amplification comes 
from the effect of artifacts such as electric instruments or metallic 
facilities around the center room. In order to understand this 
phenomenon, it will be required to make careful measurements of the magnetic 
fields at several locations in the mine, and to build and confirm a model 
to find a reason for the amplification.


\begin{figure}
\begin{center}
\includegraphics[width=0.9\linewidth]{magnetic-spectrum-ratio.eps}
\vspace{-0.2cm}
\caption{Ratio of the magnetic field spectra inside and outside the KAGRA mine.}
\label{fig:mag_in}
\end{center}
\end{figure}

\section*{Acknowledgement}
%This document was prepared by the LArIAT collaboration using the resources of the Fermi National Accelerator Laboratory (Fermilab), a U.S. Department of Energy, Office of Science, HEP User Facility. Fermilab is managed by Fermi Research Alliance, LLC (FRA), acting under Contract No. DE-AC02-07CH11359. We also gratefully acknowledge the support of the National Science Foundation; Brazil CNPq grant number 233511/2014-8, Coordena\c{c}\~ao de Aperfei\c{c}oamento de Pessoal de N\'ivel Superior - Brazil (CAPES) - Finance Code 001, S\~ao Paulo Research Foundation - FAPESP (BR) grant number 16/22738-0; Polish National Science Centre grant Dec-2013/09/N/ST2/02793; the Science and Technology Facilities Council (STFC), part of the United Kingdom Research and Innovation; The Royal Society (United Kingdom); and the JSPS grant-in-aid (Grant Number 25105008), Japan. The collaboration extends a special thank you to the coordinators and technicians of the Fermilab Test Beam Facility, without whom none of this work would have been possible.

%The LArIAT collaboration uses the resources of the Fermi National Accelerator Laboratory (Fermilab), a U.S. Department of Energy, Office of Science, HEP User Facility. Fermilab is managed by Fermi Research Alliance, LLC (FRA).%, acting under Contract No. DEAC02-07CH11359.This research was supported by the U.S. Department of Energy; the U.S. National Science Foundation.%; the Department of Science and Technology, India; the European Research Council; the MSMT CR, GA UK, Czech Republic; the RAS, RMES, and RFBR, Russia; CNPq and FAPEG, Brazil. 



This work was supported by MEXT, JSPS Leading-edge Research Infrastructure Program, JSPS Grant-in-Aid for Specially Promoted Research 26000005, MEXT Grant-in-Aid for Scientific Research on Innovative Areas 24103005, JSPS Core-to-Core Program, A. Advanced Research Networks, and the joint research program of the Institute for Cosmic Ray Research. The magnetic field measurement shown in App.~\ref{sec:app2} was supported by the Joint Usage/ Research Center program of Earthquake Research Institute, the University of Tokyo. A part of the work was supported by the Advanced Technology Center (ATC) in the National Astronomical Observatory of Japan. A part of the work was supported by National Research Foundation (NRF) and Computing Infrastructure Project of KISTI-GSDC in Korea. This research activity has been also supported by the European Commission within the Seventh Framework Programme (FP7) - Project ELiTES (GA 295153). The authors would like to express our sincere appreciation to our colleagues in the LIGO and Virgo groups for their varied and continuing support. The authors would like to thank Albrecht R\"{u}diger for editorial support of this paper. Additionally, the authors also appreciate the intense efforts of the secretaries in the ICRR GW project office. 



\section*{References}
\begin{thebibliography}{10}
\bibitem{aLIGO} B.~P.~Abbott, {\it et al.}, Phys.~Rev.~Lett., \textbf{116}, 061102 (2016).
\bibitem{Virgo} The Virgo Collaboration, ``Advanced Virgo Baseline Design'',  
VIR-0027A-09 (2009); available from https://pub3.ego-gw.it/itf/tds/.
%\bibitem{GEO} B.~Willke {\it et al.}, Class.~Quantum~Grav., \textbf{23}, S207-S214 (2006).
\bibitem{LCGT} K.~Kuroda {\it et al.},
%Large-scale cryogenic gravitational wave telescope,
Int. J. Modern Phys D \textbf{8},  557-579  (1999).
\bibitem{KAGRA} K.~Somiya for KAGRA Collaboration, Class.~Quantum~Grav., \textbf{29}, 124007 (2012).

\bibitem{ref-LIGO} A.~Abramovici {\it et al.},  Science \textbf{256} 325 (1992).
\bibitem{ref-VIRGO} The VIRGO collaboration, ``VIRGO Final Design Report''  (1997).
\bibitem{ref-GEO} K.~Danzmann {\it et al.}, ``Proposal for a 600m Laser-Interferometric Gravitational Wave Antenna'', Max-Planck-Institut f\"ur Quantenoptik Report \textbf{190}, Garching (Germany) (1994).
\bibitem{ref-TAMA} K.~Tsubono,   
%``300-m Laser Interferometer Gravitational Wave Detector (TAMA300) in Japan'' 
in \textit{Gravitational Wave Experiments}, pp.~112-114, eds. E. Coccia, G. Pizzella, and F. Ronga, World Scientific (1995).

\bibitem{ET} S.~Hild, {\it et al.}, Class.~Quantum~Grav., \textbf{27}, 015003 (2010).
\bibitem{CE} B.~P.~Abbott, {\it et al.}, Class.~Quantum~Grav., \textbf{34}, 044001 (2017).
\bibitem{KAGRAname} KAGRA was originally called LCGT (Large-scale cryogenic gravitational wave telescope). After official approval of the project, we have added a nickname of KAGRA. KAGRA is named after the Japanese word for traditional sacred music and dance for the gods.
\bibitem{LISM} S.~Sato, {\it et al.}, Phys.~Rev.~D, \textbf{69}, 102005 (2004).
\bibitem{CLIO} T.~Uchiyama, {\it et al.}, Phys.~Rev.~Lett., \textbf{108}, 141101 (2012).
\bibitem{Agatsuma} K.~Agatsuma, {\it et al.}, Phys.~Rev.~Lett., \textbf{104}, 040602 (2010).

\bibitem{UchiyamaCQG} T.~Uchiyama, K.~Furuta, M.~ Ohashi, S.~ Miyoki, O.~ Miyakawa, and Y.~Saito, Class. Quantum Grav. {\bf 31}, 224005 (2014).

\bibitem{Araya}
A. Araya {\it et al.}, Rev. Sci. Instrum. {\bf 64}  1337 (1993).

\bibitem{Sato}
S. Sato {\it et al.}, Phys. Rev. D {\bf 69}  102005 (2004).

\bibitem{Tomaru}
T. Tomaru {\it et al.}, Cryocoolers {\bf 13}  695 (2005).

\bibitem{Yamamoto}
K. Yamamoto {\it et al.}, ``Measurement of seismic motion at Large-scale Cryogenic Gravitational wave Telescope project site'', JGW-G1000218-v1 (2010); See the JGW document server here http://gwdoc.icrr.u-tokyo.ac.jp/JGWDoc/.

\bibitem{Nishitani} For this measurement, we appreciate a cooperation of the staff of Kamioka Mining and Smelting Co., Ltd. (especially, Mr. Nishitani).

\bibitem{JanLIGO} The data is provided by courtesy of J.~Harms.
\bibitem{Beker2011} M.~G.~Beker {\it et al.} Gen.~Relativ.~Gravit., \textbf{43}, 623 (2011).
\bibitem{NIKHEFmeasurement} M.~Abernathy {\it et al.},``Einstein gravitational wave Telescope conceptual design study,'' ET-0106C-10 (2011); http://tds.ego-gw.it/ql/?c=7954
\bibitem{airGGN} T.~Creighton, Class.~Quantum~Grav., \textbf{25}, 125011 (2008)

\bibitem{loopnoise} K.~Somiya and O.~Miyakawa, App.~Opt., \textbf{49}, 23, 4335 (2010).
\bibitem{LIGOseis} A.~Effler {\it et al.}, Class. Quantum Grav. {\bf 32}, 035017 (2015).
\bibitem{Sekiguchi} T.~Sekiguchi, ``Seismic noise in Kamioka Mine'', JGW-T1402971-v1 (2014).
\bibitem{CLIOelog} \url{http://gw.icrr.u-tokyo.ac.jp/clio_blog/2007/02}

\bibitem{IFI} For further information about the isolator, see: V.~Oleg, {\it et al.}, J.~Opt.~Soc.~Am. {\bf 27}, 1784-1792 (2012).
\bibitem{guardian_iKAGRA}  J.~G.~Rollins, Rev.~Sci.~Instrum. {\bf 87}, 094502 (2016).
\bibitem{Rana} R.~X.~Adhikari, Rev.~Mod.~Phys. {\bf 86}, 121 (2014).

\bibitem{araya1} A.~Araya, T.~Kunugi, Y.~Fukao, I.~Yamada, N.~Suda, S.~Maruyama, N.~Mio, and S.~Moriwaki, Rev.~Sci.~Instrum., \textbf{73}, 2434-2439 (2002).

\bibitem{GIF2017} A.~Araya, A.~Takamori, W.~Morii, K.~Miyo, M.~Ohashi, K.~Hayama, T.~Uchiyama, S.~Miyoki, and Y.~Saito, Earth, Planets and Space, 69:77 (2017).

\bibitem{araya2} K.~Matsumoto, T.~Sato, T.~Takanezawa, and M.~Ooe, 
J.~Geod.~Soc.~Japan, \textbf{47}, 243-248 (2001).

\bibitem{araya3} S.~Takemoto, H.~Momose, A.~Araya, W.~Morii, J.~Akamatsu, M.~Ohashi, A.~Takamori, S.~Miyoki, T.~Uchiyama, D.~Tatsumi, T.~Higashi, S.~Telada, and Y.~Fukuda, Journal of Geodynamics, \textbf{41}, 23-29 (2006).

\bibitem{araya4}  A.~Araya, A.~Takamori, W.~Morii, H.~Hayakawa, T.~Uchiyama, M.~Ohashi, S.~Telada and S.~Takemoto, Geophys.~J.~Int., \textbf{181}, 127-140 (2010).

\bibitem{YanbeiWaterGGN} Y.~Chen, ``Gravity Gradient Noise from Water'', internal document (2016).
\bibitem{SomiyaWaterGGN} K.~Somiya, ``Water Gravity Gradient Noise'', JGW-G1605565-v1 (2016).

\bibitem{atsuta} S.~Atsuta {\it et al.}, J. Phys. Conf. Ser., {\bf 716}, 012032 (2016).

\end{thebibliography}

\end{document}

