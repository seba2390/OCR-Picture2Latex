

\documentclass[pra,twocolumn,showpacs,preprintnumbers,amsmath,amssymb]{revtex4-1}
\usepackage{multirow}
\usepackage{graphicx}% Include figure files
\usepackage{dcolumn}% Align table columns on decimal point
\usepackage{bm}% bold math
\usepackage{psfrag}
\usepackage{epsfig}
\usepackage{amsmath}
\usepackage{amssymb}
\usepackage{MnSymbol}
\usepackage{color}
\usepackage{bbm}
\usepackage[FIGTOPCAP,raggedright,nooneline]{subfigure}
\usepackage{verbatim}
 \usepackage[applemac]{inputenc}
\usepackage{textcmds} 
%circled numbers
\usepackage{tikz}
\usepackage{natbib}
\newcommand*\circled[1]{\tikz[baseline=(char.base)]{
            \node[shape=circle,draw,inner sep=1pt] (char) {#1};}}
            \newcommand{\sign}{\text{sign}}

\newcommand{\bra}[1]{\langle #1 |}
\newcommand{\ket}[1]{| #1 \rangle}
\newcommand{\scpr}[2]{\langle #1 | #2 \rangle}
\newcommand{\proj}[1]{\ket{#1}\bra{#1}}
\newcommand{\nn}{\nonumber}

\newcommand{\Bra}{\langle}
\newcommand{\Ket}{\rangle}
\newcommand{\ve}{\vert}
\newcommand{\iu}{\text{i}}
\newcommand{\ee}{\text{e}}
\newcommand{\lk}{\left}
\newcommand{\rk}{\right}
\newcommand{\id}{{\sf 1 \hspace{-0.3ex} \rule{0.1ex}{1.52ex}\rule[-.01ex]{0.3ex}{0.1ex}}}
\newcommand{\ignore}[1]{}
\newcommand{\pr}[1]{{\color{red} #1}}
\newcommand{\kvk}[1]{{\color{blue} #1}}
\newcommand{\sdb}[1]{{\color{cyan} #1}}
\newcommand{\sgn}{\text{sgn}}

\begin{document}

\title{Emergence of solitons from many-body photon bound states in quantum nonlinear media}


\author{G. Calaj\'o$^1$ and   D. E. Chang$^{1,2}$}



\affiliation{$^1$ICFO-Institut de Ciencies Fotoniques, The Barcelona Institute of
Science and Technology, 08860 Castelldefels (Barcelona), Spain}
\affiliation{$^2$ICREA-Instituci\'{o}  Catalana  de  Recerca  i  Estudis  Avan\c{c}ats,  08015  Barcelona,  Spain}


\date{\today}
 
\begin{abstract}
Solitons are  known to occur in the context of atom-light interaction via the well-known semi-classical  phenomenon of self-induced transparency (SIT). 
Separately, in the regime where both light and atoms are fully treated quantum mechanically, quantum few-photon bound states are known to be a ubiquitous phenomenon that arises in different systems such as atoms coupled to chiral or bidirectional waveguides, and in Rydberg atomic media. In the specific case of two-level atoms coupled to a chiral waveguide, a recent analysis based on Bethe ansatz has established that SIT emerges from the quantum realm as a superposition of quantum many-photon bound states.
Beyond this case, however, the nature of any connection between the full quantum many-body regime and semi-classical behavior has not been established. Here, we employ a general spin-model formulation of quantum atom-light interfaces to numerically investigate this problem, taking advantage of the fact that this approach readily allows for powerful many-body simulations based on matrix product states (MPS). % capable of capturing the involved multi-excitation  dynamics. 
 We first analytically derive the two-photon bound state dispersion relation for a variety of atom-light interfaces, and then proceed to  numerically investigate the  multi-excitation bound states dynamics. 
%how the properties of higher multi-photon bound states manifest themselves in the propagation of large photon number pulses and spatio-temporal correlations at the output. 
Interestingly, for all the specific systems studied, we find that the large-photon number limit always coincides with  the  soliton phenomenon of self-induced transparency or immediate generalizations thereof.

%The emergence of multi-photon bound states in quantum nonlinear media is an intriguing phenomenon that has attracted significant theoretical and experimental interest. Such bound states can emerge in quite different settings,  such as arrays of quantum emitters coupled to photonic waveguides and Rydberg nonlinear media. The theoretical tools to characterize the bound states are equally diverse, ranging from Bethe ansatz to effective field theories. Here, we propose and utilize a spin-model formulation of quantum atom-light interactions as a general, unified framework to study bound states, and apply this formalism to several distinct systems. 

%We show how to obtain the bound-state dispersion relation within the two-excitation subspace, and also numerically investigate how the properties of higher multi-photon bound states manifest themselves in the propagation of large photon number pulses and spatio-temporal correlations at the output. 

%Interestingly, for all the specific systems studied (chiral and bi-directional waveguides, and Rydberg media), we find that the large-photon number limit always coincides with the well-known semi-classical soliton phenomenon of self-induced transparency or immediate generalizations thereof.
\end{abstract}



\maketitle

\section{Introduction}

One of the most distinct predictions within classical nonlinear optics is the emergence of solitons, whose shape does not change during propagation \cite{Agrawal}. From early on, there were also efforts to understand how these solitons might emerge from a fully quantum theory, perhaps most prominently in continuous Kerr nonlinear media~\cite{lai1,lai2,Kurizki1,Kurizki2,Kurizki3}. While the weak nonlinearities of conventional media historically made quantum properties largely inaccessible, in recent years there have been diverse systems, ranging from ensembles of Rydberg atoms~\cite{firstrev,callum,first,LiangBS}  to waveguide QED systems coupled to quantum emitters~\cite{roy_rev_mod,review_lodahl,Darrick_rev_mod,shen_fan_prl,shen_fan,
Harold_strcorr,mahmo_calajo,Harold_bs,Prasadnat}, where strong interactions at the level of individual photons~\cite{changnl} can be achieved. 
An intriguing effect  occurring in such quantum nonlinear media is the existence of  quantum photon bound states, which have been 
predicted and studied in these different systems with equally diverse theoretical techniques, varying from Bethe ansatz~\cite{shen_fan_prl,shen_fan,Bethe,sahandprl} (when the interactions between photons are local in character) to effective field theories \cite{bienas,Efimov,magrebi} (particularly successful for Rydberg nonlinear media).
Most of these techniques have been exclusively applied to few excitations and/or moderate size systems and, generally, the many-excitation dynamics remains largely unexplored. Recently an important step toward this direction 
was made in the specific case of an atomic array chirally (unidirectionally) coupled to a photonic waveguide. There, it has been recently demonstrated~\cite{mahmo_calajo} that a linear combination of quantum 
many-body photon bound states leads to the emergence  of the well-known self induced transparency (SIT) soliton~\cite{McCall1,McCall2,Bullough}. However, besides this specific case, a general connection  between the  quantum many-body dynamics and the emergent semi-classical behavior in other  atom-light interfaces is still generally unknown.

 


%quantum photon bound states are known to occur in several quantum nonlinear media and the theoretical techniques by which can be predicted and studied in these different systems is equally diverse, varying from Bethe ansatz~\cite{shen_fan_prl,shen_fan,Bethe,sahandprl} (when the interactions between photons are local in character) to effective field theories \cite{bienas,Efimov,magrebi} (particularly successful for Rydberg nonlinear media).




In recent years, an alternative formalism to capture quantum atom-light interactions has gained attention. This formalism is based on the insight that the light-matter polaritons that form in near-resonant propagation are in fact almost entirely atomic in character in typical settings. Thus, the photonic degrees of freedom can be effectively integrated out, to yield a quantum spin model describing photon-mediated interactions between the atoms \cite{anapra,anaprx,ana_masson,rev_wqedarray}. In principle, if the dynamics of the spin model can be solved, the correlations of the outgoing quantum fields can be readily obtained using an input-output equation \cite{Caneva,MPSJames,Caneva, MPSJames,glauber,Blais_in}. Within this framework, particularly in the context of atomic arrays and in waveguide QED, the spin model has led to the prediction of many interesting effects, including strongly subradiant states~\cite{Albrecht,Loic,ritch,dimermolmer1,zhang,buchler,dimer_shermet,Pod_scat,Pod_3,Lesanosky,Yudson}, photon mediated localized states~\cite{Pod_photon_loc}, topological states~\cite{Pod_topo}, chaotic states~\cite{caos} and subradiant dimers~\cite{dimermolmer2,dimer_poddu}. Techniques to effectively map continuous, macroscopic nonlinear media such as Rydberg ensembles to the spin model have also been proposed and investigated \cite{james_ryd}. 

%Light propagation in nonlinear optical media is a fascinating topic and its understanding is at the basis of many applications in modern science and engineering, where an effective interaction between electromagnetic signals inside the medium is often required~\cite{Agrawal}. These processes usually occur at high light intensities, but over the years, many efforts have been invested to progressively reduce the minimum required input power and to access the realm of Quantum Nonlinear Optics (QNLO)~\cite{changnl} where nonlinear effects occur at the level of individual photons. Reaching this regime open new intriguing possibilities to study correlated photon transport~\cite{shen_fan_prl,Gorprl11,Harold_strcorr,zeuthen,sahandprl,Prasadnat,PohlEIT} and to generate and manipulate  non-classical states of light~\cite{Birnba,darrick_crsytal,shen_fan}.
%Among such exotic states of light, photon bound states are particularly intriguing for the fact that they are made by \qq{interacting photons} and that their classical counter-part has been associated in certain context to  optical solitons~\cite{lai1,lai2,mahmo_calajo}. These bound states  are ubiquitous in several quantum nonlinear media ranging from nonlinear fibers~\cite{lai1,lai2}, ensembles of Rydberg atoms \cite{bienas,Efimov,magrebi} and waveguides coupled to quantum emitters~\cite{shen_fan,Harold_bs}. In particular, bound states of two and three photons have been observed~\cite{first,LiangBS} in the dispersive regime of Rydberg electromagnetic induced transparency~\cite{petro,Peyr,firstrev,callum,Hofferberth}  where a propagating excitation inside the medium can create an effective potential for the other incoming photons, due to the nonlinear Rydberg interactions.
%In the uprising field of waveguide QED~\cite{ReitzPRL2013,Arcari2014,review_lodahl,Goban2014,Pablo2017,Darrick_rev_mod,roy_rev_mod}, photon bound states  instead arise due to the intrinsic nonlinearity of two-level atoms (TLA) enhanced by the light confinement in one dimension.
%Even if the intrinsic nature of these states is the same,  their  features and the theoretical formalism to describe them vary depending on the specific media. In all these cases, typical approaches focus mainly on the photonic wave function at the end of the scattering process and do not bother of the atomic component that is actually predominant while the pulse is in the medium.

%Recently,  has received great attention an alternative approach to investigate light-matter interactions in generic quantum nonlinear media that relies on an adiabatic elimination of the photonic degree of freedom (valid within Markov approximation)~\cite{anapra,anaprx,MPSJames,ana_masson,rev_wqedarray}. This approach allows to  develop a spin-model description which encodes the photon-mediated interactions of the quantum emitters. 
%Within this framework, in the context of atomic arrays, many intriguing collective states have been predicted such as subradiant states~\cite{Albrecht,Loic,ritch,dimermolmer1,zhang,buchler,korno,dimer_shermet,Pod_scat,Lesanosky,Yudson}, photon mediated localized states~\cite{Pod_photon_loc}, topological states~\cite{Pod_topo}, chaotic states~\cite{caos} and dimers~\cite{,dimermolmer2,dimer_poddu}.
%In particular the latter, characterized by bound spin excitations are mapped out in the photon-bound states when they leave the medium and this process can be described by making use of input output equations that provides the correlated outcoming field~\cite{Caneva}.

%Here, we show that the spin model formalism can be used to achieve a unified description of photon bound states in different atomic nonlinear media and to numerically  investigate their many-body dynamics. In particular, we exploit this formalism to describe both the quantum nonlinear propagation of an optical field in an interacting Rydberg atom ensemble, and in an array of two-level atoms coupled to a waveguide, treating both paradigmatic cases of bidirectional and chiral~\cite{Stannigel2012,Pichler2015,ChiralRev}  (unidirectional) emission. Within the spin model, the procedure to distinguish bound states  is analogous to that of identifying bound magnon excitations in condensed matter spin  models~\cite{Bethe,wortis,Schneider,Mattis,bloch}, a well-known problem in the context of quantum magnetism that has been recently reconsidered in open systems settings~\cite{petroBS,cooper,ramos}. An important physical distinction, however, is the long-range nature of the photon-mediated spin interactions in our model, and also the driven open nature of light-matter interacting systems. In particular, using the input-output equations, we are able to show how these bound states within the atomic medium are excited by non-interacting optical pulses incident from the outside, and how these bound states manifest themselves in the spatio-temporal correlations of the optical field once it leaves the medium. These bound states are also closely related to the dimers previously identified within waveguide QED~\cite{dimermolmer2,dimer_poddu,rev_wqedarray}. However, while the emphasis of those previous works was on the highly subradiant nature of discrete dimer states
%here, we focus on the continuous dispersion relations of such states, which describe the distinct propagation of these states from individual, non-interacting photons, and which are responsible for the interesting spatio-temporal correlations that arise in the output.
Here, we utilize the spin model formalism to investigate photon bound states and their manifestation in photon propagation dynamics across different systems. In particular, we examine an interacting Rydberg atom ensemble, and also an array of two-level atoms coupled to a waveguide, treating both paradigmatic cases of bidirectional and chiral~\cite{Stannigel2012,Pichler2015,ChiralRev}  (unidirectional) emission. Within the spin model, the procedure to distinguish bound states  is analogous to that of identifying bound magnon excitations in condensed matter spin  models~\cite{Bethe,wortis,Schneider,Mattis,bloch,petroBS,cooper,ramos} and 
closely related to previous studies of dimers within waveguide QED~\cite{dimermolmer2,dimer_poddu,rev_wqedarray}. Here we first describe how the dispersion relation of two-photon bound states can be identified and then we show how these bound states manifest themselves in the spatio-temporal correlations of the output field, given classical weak incident pulses.
While the analytical techniques applied to two-photon bound states do not readily scale to investigate the properties of higher photon number, the spin model nonetheless admits an efficient way to numerically investigate the effect of many-photon bound states in the correlation of the output field given large incident pulses. This large photon number limit is made tractable through the possibility to naturally encode the spin model in matrix product state (MPS) representations. In this many-photon limit, we generally identify signatures of a transition  from true quantum many-body behavior, as characterized by non-trivial spatio-temporal correlations, to a semi-classical soliton wave reminiscent of self-induced transparency in atomic media, irrespective of the underlying details of the model. 
%First, we describe how the dispersion relation of two-photon bound states can be identified, in a manner closely related to previous studies of dimers within waveguide QED~\cite{dimermolmer2,dimer_poddu,rev_wqedarray}. We also show how these bound states manifest themselves in the spatio-temporal correlations of the output field, given classical weak incident pulses.


%Importantly, we find that the spin model is versatile, both able to capture bound state properties in regimes where Bethe ansatz is not valid and without employing approximations typical of effective theories, and readily suitable to treat the many-photon regime numerically via matrix product state (MPS) representations~\cite{Verstraete,Schollw}. In this many-photon limit, we generally identify  signatures of a transition  from true quantum many-body behavior, as characterized by non-trivial spatio-temporal correlations, to a semi-classical soliton wave reminiscent of self-induced transparency in atomic media, irrespective of the underlying details of the model. 

%After having introduced the formalism and having recovered known results in the linear optic regime, we derive the full two excitation spectrum in the relative coordinate frame. This procedure immediately reveals the occurrence of energetic gaps and the arising of bound states with well defined dispersion relations. Here relies one of the advantage of our formalism which permits  to derive the bound state dispersion for every parameter regime without making use of effective theories and also it allows to explicitly keep track of the source of non linearities (e.g. TLA, Rydberg interaction) at the base of the effect. In last instance, the spin model formulation it is well suitable for numerical simulations that we perform for large media extending previously developed algorithms~\cite{MPSJames,james_ryd} based on Matrix Product States (MPS) representation~\cite{Verstraete,Schollw}. With this tool we explore the multi-excitation regime showing that, correlated pulse separation and the transition to a solitonic wave  in the many body limit, are not effects restricted to a specific chiral scenario~\cite{mahmo_calajo},  but they are intrinsic properties of bound states propagation in quantum non linear media.

This paper is structured as follows. In Sec.~\ref{Sec. model}  we introduce a generic model for an ensemble of emitters coupled to a quasi-one-dimensional~(1D) quantum optical field. From here, we formulate the spin model, and present general criteria to identify the photon bound states. In Sec.~\ref{Sec.TLAarray} we consider as a first case an array of  two-level atoms~(TLA) coupled to either a  chiral or bidirectional  waveguide. We analyze the single- and two-excitation spectrum, followed by multi-photon propagation. In particular, in the many-photon limit, we recover the semi-classical phenomenon of self-induced transparency~(SIT). In Sec.~\ref{Sec.Rydmedia}, we apply the spin model to an ensemble of Rydberg atoms. After analyzing the two-excitation bound state dispersion relation, we make use of an effective two-level atom description to recover, in the many-body regime, a solitonic behavior similar to SIT, i.e. \emph{Rydberg SIT}. We conclude and provide an outlook of possible future directions in Sec.~\ref{sec:conclusions}.

\section{Model}\label{Sec. model}

\begin{figure}
\centering
\includegraphics[width=0.5\textwidth]{setup_BSQNLM_rev.pdf}%
\caption{Generic quantum nonlinear media. A free-space ensemble of atoms with an unspecified level structure is coupled to photons in a focused propagating beam via a particular two-level transition. The system is characterized by a single-atom emission rate $\Gamma$ into the beam, and $\Gamma_0$ into other modes~(free space). Alternatively, we consider an atom array coupled to a one-dimensional optical waveguide. An individual atom emits into the left-propagating modes, right-propagating modes, and free space at rates $\Gamma_L, \Gamma_R$, and $\Gamma_0$, respectively. The free-space ensemble can be mapped to an effective waveguide model, as we discuss in the main text. In both scenarios, multi-photon bound states (red shaded area) can exist due to the nonlinear interactions induced by the atomic medium.}
\label{fig:setup}
\end{figure}


We consider a setup as illustrated in Fig.~\ref{fig:setup} where an ensemble of  $N$ atoms is coupled to a quasi-1D propagating field. We assume that the field couples to a transition of frequency $\omega_{eg}$ involving ground and excited states~$|g\rangle$ and $|e\rangle$, respectively, while the atoms can have other levels~(\textit{e.g.}, driven by classical control fields) that need not be specified now. The spontaneous emission rate of a single, excited atom into the 1D modes occurs at a rate $\Gamma$, while it can also emit at a rate $\Gamma_0$ into other modes besides the 1D continuum. This generic setup describes well a number of systems consisting of atoms coupled to a physical waveguide, where one might have $\Gamma_0\gg \Gamma$~(such as for atoms coupled to optical nanofibers)~\cite{Prasadnat,ReitzPRL2013,Mitsch2014,Pablo2017}, $\Gamma_0\sim \Gamma$~(for atoms coupled to photonic crystal waveguides)~\cite{Arcari2014,Sollner2015,Goban2014,Hood2016}, or $\Gamma_0\ll \Gamma$~(superconducting qubits coupled to unstructured or photonic crystal transmission lines)~\cite{Sundaresan,painter1,painter2,ustinov,ustinov_topo,marcoBS}. As pointed out in Ref.~\cite{MPSJames}, free space ensembles interacting with quasi-1D optical beams can also be mapped to this waveguide model, by taking $\Gamma_0\gg\Gamma$~(to account for the weak free-space light coupling) and finding suitable maps between the microscopic parameters of this model and the macroscopic parameters of the physical system such as optical depth~(discussed in detail in Sec.~\ref{Sec.Rydmedia}). For our situations of interest, we can consider the waveguide-coupled atoms to be in an array of lattice constant $d$, as shown in Fig.~\ref{fig:setup}. For bidirectional waveguides, this avoids Anderson or many-body localization of light associated with disorder and multiple scattering~\cite{Anderson,ReviewManybodyloc,NikosManybodyloc}. For chiral waveguides (and in free space, where the weak atom-light coupling is essentially equivalent to chiral coupling), the physics is in fact independent of the specific atomic positions, as we will describe below.




\subsection{Full Hamiltonian}
 Within the 1D model the full waveguide QED Hamiltonian of the system is given by the following contributions 
\begin{equation}\label{H_tot}
\hat H=\hat H_{ph}+ \hat H_a+\hat H_{in}.
\end{equation}
The fist term $\hat H_{ph}$ is the photonic Hamiltonian describing right- and left-propagating modes traveling in the channel with group velocity $c$.
It explicitly reads  ($\hbar=1$) 
%\begin{equation}\label{H_ph}
%H_{ph}=-ic\int dx\left[\hat a_R^{\dagger}(x)\frac{\partial}{\partial x}\hat a_R(x)-\hat a_L^{\dagger}(x)\frac{\partial}{\partial x}\hat a_L(x)\right]
%\end{equation}
\begin{equation}\label{H_ph}
\hat H_{ph}=-i\int dx\left[\hat E_R^{\dagger}(x)\frac{\partial}{\partial x}\hat E_R(x)-\hat E_L^{\dagger}(x)\frac{\partial}{\partial x}\hat E_L(x)\right]
\end{equation}
where $\hat E_{R(L)}(x)$ is the bosonic field operator annihilating a right-going (left-going) photon at position $x$ and fulfilling the commutation rule  $[\hat E_{R(L)}(x),\hat E^{\dagger}_{R(L)}(x')]=c\delta(x-x')$. %Note that with this definition of the field operator $\langle \hat E^{\dagger}_{R(L)}(x) \hat E_{R(L)}(x)\rangle$ gives us directly the photon number density of the field.
$\hat H_a$ is the atomic Hamiltonian which at this stage is not fully specified. In particular, we will assume that the atom has ground and excited states $|g,e\rangle$ whose transition of frequency $\omega_{eg}$ couples to the quantum propagating field. Beyond that, the atom could contain additional levels, dissipation from other photonic channels (encoded in the rate $\Gamma_0$),  auxiliary control fields acting on different transitions, or other terms independent from the interaction of atoms with the quantum propagating field. The atom-light interaction allows for the creation or annihilation of excitations on the $n$-th atom through the spin operators $\hat\sigma^n_{ge}=|g_n\rangle\langle e_n|$ and $\hat\sigma^n_{eg}=|e_n\rangle\langle g_n|$. The corresponding interaction Hamiltonian reads
%\begin{equation}\label{H_in}
%H_{in}=\sum_n\left[\hat\sigma_{ge}^n\left(\sqrt{c\Gamma_R}\hat a_R^{\dagger}(x_n)+\sqrt{c\Gamma_L}\hat a_L^{\dagger}(x_n)\right)+ \rm H.c.\right]
%\end{equation}
\begin{equation}\label{H_in}
\hat H_{in}=\sum_n\left[\hat\sigma_{ge}^n\left(\sqrt{\Gamma_R}\hat E_R^{\dagger}(x_n)+\sqrt{\Gamma_L}\hat E_L^{\dagger}(x_n)\right)+ \rm H.c.\right]
\end{equation}
%
where $\Gamma_R$ and $\Gamma_L$, with $\Gamma= \Gamma_R+\Gamma_L$, are the single-atom decay rates associated
respectively to the emission of right- and left-propagating photons. 
Here we explicitly distinguish the emission into the two directions to cover both the paradigmatic cases of bi-directional emission and chiral (uni-directional) emission that will be discussed in the rest of the paper.\\

\subsection{Spin model}\label{Secspin}

Standard treatments to tackle Hamiltonian~\eqref{H_tot} in the multi-excitation sector~(multiple photons and/or excited atoms) usually follow two different strategies depending on the regime considered. First, in the limit of strong dissipation, $\Gamma_0\gg\Gamma$, individual atoms have negligible interaction with light and interesting phenomena instead arise by collective coupling. Then, the ensemble of atoms can be treated as a continuous bosonic field. Interesting nonlinearities, such as arising from Rydberg interactions, can be added and be treated by effective field theories~(so far, limited to a few excitations)~\cite{bienas,Efimov,magrebi}, for example. Conversely, in the waveguide QED regime with negligible loss, $\Gamma_0\sim 0$, the eigenstates can be exactly computed using scattering theory formalism (e.g. S-matrix, Bethe ansatz, etc.)~\cite{shen_fan_prl,shen_fan,Bethe,sahandprl} for a few excitations, either for small atom number in general or for large atom number and chiral waveguides.

To go beyond specific limitations on excitation number or system details, we consider an alternative approach, where the photons are integrated out to arrive at an effective spin model for the atoms. This approximation takes advantage of the fact that in most physical systems of interest, the atom-photon dynamics occur on a time scale longer than the photon propagation time through the system, i.e. $\Gamma\ll c/Nd$, so that the light-mediated interactions can be considered instantaneous. Equivalently, the highly dispersive nature of the atoms due to the small linewidth causes the light-matter polaritons to be almost entirely atomic in character. The full system evolution is then given by the master equation (ME) for the reduced atomic density operator~\cite{Caneva, MPSJames,glauber} 
\begin{equation}\label{eq:MasterEq}
\dot{\hat \rho}= -i\left[ (\hat H_{\rm eff}+\hat H_{\rm drive})\hat \rho-\hat\rho (\hat H_{\rm eff}+\hat H_{\rm drive})^{\dagger}\right]+\mathcal{J}[\hat\rho].
\end{equation}
Here the non-Hermitian collective evolution
of the system is given by the effective Hamiltonian
\begin{equation}\label{Heff}
\hat H_{\rm eff}=\hat H_a-i\sum_{nm}A_{nm}\hat\sigma^n_{eg}\hat\sigma^m_{ge},
%-i\sum_{n>m}\left[\Gamma_L e^{ik_0|x_n-x_m|}\hat\sigma^n_{eg}\hat\sigma^m_{ge}+\Gamma_R e^{ik_0|x_n-x_m|}\hat\sigma^m_{eg}\hat\sigma^n_{ge}\right],
\end{equation}
where the matrix 
\begin{equation}
A_{nm}=\left[\Gamma_L e^{ik_0|x_n-x_m|}\theta_{nm}+\Gamma_R e^{ik_0|x_n-x_m|}\theta_{mn}\right]
\end{equation}
encodes the photon mediated atom-atom interactions and $\theta_{nm}=\theta(x_n-x_m)$, with $\theta_{nn}=1/2$, is the Heaviside function.  The Hamiltonian $\hat H_{\rm eff}$ is invariant 
under discrete translations of the product of the resonant wavevector $k_0=\omega_{eg}/c$ and the lattice constant $d$, $k_0 d \rightarrow k_0 d + 2\pi$, so a full description is obtained by considering $k_0d\in[-\pi,\pi]$.
The population recycling contribution to the evolution in Eq.\eqref{eq:MasterEq} is given by
 \begin{equation}
\mathcal{J}[\hat\rho]=\Gamma_0\sum_n\hat\sigma^{n}_{ge}\hat\rho \hat\sigma_{eg}^n+\sum_{nm}\left[\left(A_{nm}+A^*_{nm}\right)\hat\sigma^{n}_{ge}\hat\rho\hat\sigma_{eg}^m+\rm H.c.\right].
\end{equation}
Finally in Eq.\eqref{eq:MasterEq} we also explicitly add a driving term
\begin{equation}
\hat H_{\rm drive}=\sum_n\sqrt{\Gamma_R}\left[E_{in}(t,x_n)\hat\sigma_{eg}^n+ \rm H.c.\right],
\end{equation}
which couples the emitters to a right-propagating coherent state input field $E_{in}(t,x)=\mathcal{E}_{in}(t)e^{ik_{0} x-i\omega_{\rm in}t}$ with $\omega_{\rm in}$ being the central frequency of the driving field.\\

While such master equations describing photon-mediated dipole-dipole interactions have long existed, in recent years it has been realized that one can also use their solutions to re-construct the quantum field that has been previously integrated out, thus constituting a complete model of atom-light interactions. This field takes the form of an input-output relation~\cite{Caneva, MPSJames,glauber,Blais_in} 
 \begin{equation}\label{in_outR}
 \hat E_R(t)=\mathcal{E}_{\rm in}(t)+i\sum_n\sqrt{\Gamma_R}e^{-ik_0x_n}\hat\sigma_{ge}^n(t),
\end{equation}  
 \begin{equation}\label{in_outL}
 \hat E_L(t)=i\sum_n\sqrt{\Gamma_L}e^{ik_0x_n}\hat\sigma_{ge}^n(t).
\end{equation}  

In this work, we will on one hand consider the microscopic properties of the photon bound states themselves, which can be derived from the eigenstates of~\eqref{Heff} alone~(practically in the few-excitation limit). Separately, we will solve the full master equation dynamics of Eq.~(\ref{eq:MasterEq}) in the presence of the input field, to see how the bound states manifest themselves in the outgoing field properties of Eq.~(\ref{in_outR}). This dynamics can be computed numerically even in the many-excitation limit, with an MPS-based quantum trajectories algorithm discussed in Appendix~\ref{AppMPS}, and, as we are going to discuss in the following, is able to capture the progressive transition from a correlated output field to  semi-classical solitons. 
While we will specifically focus on coherent state (photon number uncertain) input fields here, these calculations can also be used to study Fock state inputs. 
 In particular, by simulating the dynamics with a quantum jump algorithm, one can post-select on the total number of jumps that occurred in the output fields to obtain a fixed photon number~\cite{MPSJames}.
 
  

 
 
 
 \subsection{Photon bound states}
%\textcolor{red}{I like the changes. The only thing is that in the definition of the bound states I would say that they experience a reduced change in their shape. From Sahand calculation the distortion, even if  highly suppressed, becomes 0 only in the many-body limit}. 
Photon bound states are states with spatially correlated positions, which propagate through the bulk of a nonlinear medium experiencing low distortion~\cite{mahmo_calajo}. %without a change in shape. 
%An alternative definition could be a multi-photon state that remains the same after transmission through an ensemble, apart from possibly an overall reduced transmission factor. While these two definitions coincide for a chiral system, we prefer the former definition in general, as it is associated solely with the bulk properties of a medium, while the latter could also involve complicating factors such as partial reflection from the boundaries of the atomic medium.
%Many-body photon bound states are elementary propagating bunched excitations that occur in several quantum nonlinear media such as waveguide QED systems~\cite{shen_fan,Harold_bs} and  Rydberg atomic ensembles~\cite{first,LiangBS}. They are scattering eigenstates, i.e. states that propagate without change through the system apart from global phase, of the  full Hamiltonian~\eqref{H_tot} in the non dissipative limit ($\Gamma_0=0$), which arise due to the effective photon-photon interaction induced by the nonlinearity of the medium. 
Despite the name, these bound states are actually almost entirely atomic~(spin-like) in nature when they propagate through the medium~\cite{chang_mirror,chang_tao}. The problem of identifying these states then becomes similar to that of magnon bound excitations in spin chains~\cite{Bethe,wortis}. For example, within the two-excitation subspace we can follow standard procedures to diagonalize the Hamiltonian~\eqref{Heff} in the relative and center-of-mass coordinate frame, which is convenient because the latter is characterized simply by a plane wave of wavevector $K$. On the other hand, the problem in the relative coordinate reduces to finding single-particle bound states in an effective (and possibly non-trivial) impurity model, whose energies depend parametrically on $K$. Due to the nonlinear interaction, this bound state will have a different energy $E_{\rm BS}$ than the sum of individual photon energies, which allows for their spectral identification:
 \begin{equation}\label{BS condition}
E_{\rm BS}(K)\ne J(q)+J(K-q).
\end{equation}
Here, $J(k)$ denotes the single-excitation dispersion relation and $q$ the relative momentum.

A crucial difference compared to canonical condensed matter spin models lies in the dissipative, long-range nature of the interactions encoded in the non-Hermitian Hamiltonian~\eqref{Heff}. A number of studies~\cite{dimermolmer1,dimermolmer2,dimer_shermet,dimer_poddu} have already pointed out that in a finite system, this Hamiltonian can give rise to two-excitation dimers as eigenstates that are generally lossy, physically due to the non-zero support of the wave functions with the system boundaries and their subsequent radiation into the empty waveguide. While these studies largely focused on how long-lived or subradiant these states could be, here, we more generally will study the dispersion relation of this continuum of bound states, and show that this dispersion relation and the spatial properties of the bound states indeed manifest themselves in the quantum nonlinear optics problem, in terms of correlations of the output field given a multi-photon input field.  

 
 
 \subsection{Self-induced transparency}\label{Sec.SIT}

In~\cite{mahmo_calajo} it has been rigorously proven that in the many-body limit with finite size systems a linear combination of multi-photon bound states gives rise to the formation of self-induced transparency~(SIT) solitons, when the system consists of two-level atoms coupled to a chiral waveguide. As one of our main goals is to investigate whether SIT emerges in other quantum nonlinear systems from a full quantum picture, we first briefly review the effect of SIT, in the previously introduced language of waveguide QED.

SIT is a semi-classical phenomenon, involving the emergence of a soliton when the atoms are treated as two-level systems and the field classically. 
In order to write down the SIT equations, we move to a continuum description, mapping the spin operators to a continuous density, i.e. $\hat \sigma^n\rightarrow \hat \sigma(x)/\nu$, where $\nu$ is the linear density of the medium  and the spin operators fulfill the commutation relation $[\hat \sigma_{ge}(x),\hat \sigma_{eg}(x')]=-\hat \sigma_{z}(x)\delta(x-x')$ with $\hat \sigma_{z}(x)$ being the Z Pauli matrix. With this mapping, and assuming that the atoms resonantly interact with a right propagating field (we omit the $R$ label),
the total Hamiltonian~\eqref{H_tot} becomes
\begin{equation}\label{H_tot_cont}
\hat H=\!-i\!\!\int\!\! dx\hat E^{\dagger}(x)\frac{\partial}{\partial x}\hat E(x)+\sqrt{\Gamma}\!\!\int\!\! dx\left[\hat\sigma_{ge}(x)\hat E^{\dagger}(x)+ \!\rm H.c.\right].
\end{equation}

Using the Heisenberg equations with respect to this Hamiltonian and taking the expectation values $E(x,t)=\langle \hat E(x,t)\rangle$ and $\sigma(x,t)=\langle \hat \sigma(x,t)\rangle$, we get the mean field equations
\begin{equation}\label{eqMF}
\begin{split}
&\left[\frac{\partial}{\partial t}+c \frac{\partial}{\partial x}\right ]E(x,t)=-ic\sqrt{\Gamma}\sigma_-(x,t)\\
&\frac{\partial}{\partial t}\sigma_-(x,t)=i\sqrt{\Gamma}\sigma_z(x,t)E(x,t)\\
&\frac{\partial}{\partial t}\sigma_z(x,t)=4\sqrt{\Gamma }{\rm Im}[E(x,t)\sigma^*_-(x,t)],
\end{split}
\end{equation}
where  the external dissipation has been set to zero, $\Gamma_0=0$.
This set of equations~\eqref{eqMF} admits a solitonic solution for the photonic field (see Refs.~\cite{Bullough,mahmo_calajo} for a detailed derivation):
 \begin{equation}\label{eqSIT_solution}
  E(x,t)=\frac{ n_{\rm ph}\sqrt{\Gamma}}{2}\operatorname{sech}\left[\frac{ n_{\rm ph}\Gamma}{2}\left(\frac{x}{v_g}-t\right)\right],
 \end{equation}
with $v_g=(n^2_{\rm ph}\Gamma c)/(n^2_{\rm ph}\Gamma+4c\nu)$ being the group velocity inside the medium and $n_{\rm ph}$ being the average photon number in the pulse. Translated to a 1D   setting for a chain of $N$ atoms with linear density $\nu=1/d$, this leads to an overall pulse delay of~\cite{mahmo_calajo}
\begin{equation}\label{delaySIT}
\tau_{n_{\rm ph}}=4N/( n^2_{\rm ph}\Gamma).
\end{equation}
This solitonic solution has the property that its integrated Rabi frequency gives a 2$\pi$ pulse (area law)~\cite{McCall1,McCall2,Bullough,mahmo_calajo}
 \begin{equation}\label{eqSITarea}
2 \sqrt{\Gamma}\int dt E(x,t) = 2\pi,
 \end{equation}
letting each individual atom undergoing a full Rabi oscillation from the ground to excited state and back. This condition implies that photons are not taken away
from the original pulse, if the Rabi oscillation occurs on a time scale faster than the spontaneous emission rate $\Gamma_0$, allowing the soliton to propagate in a transparent fashion through the medium.%Generic incoming pulses $E_{\rm in}(x,t)$ satisfying the given area law will propagate through the atoms  acquiring this  solitonic behavior. \\

%{\color{red} Note that here I kept the notation $a(x,t)=\langle \hat a(x,t)\rangle$ because 
%$\mathcal{E}(x,t)\ne\langle \hat a(x,t)\rangle$. Indeed while the bosonic operator $a(x,t)$ has dimension $\sqrt{m^{-1}}$ the incoming pulse has dimension $\sqrt{Hz}$. In principle they could be converted using  $\mathcal{E}(x,t)=\langle \hat a(x,t)\rangle v_g/\sqrt{c}$.}




%We will shortly discuss bound states in the many-photon limit, but given the close relation of those results to self-induced transparency~(SIT), we first briefly review SIT. SIT is a semi-classical phenomenon, involving the emergence of a soliton when the atoms are treated as two-level and the field classically. In SIT, the pulse shape is chosen so that each individual atom undergoes a $2\pi$ from the ground to excited state and back, so as to not take away photons from the original pulse, and to minimize spontaneous emission in the case of a non-zero $\Gamma_0$. Simultaneously, one can find that certain pulse shapes are preserved as they interact with and propagate through the atoms, thus acquiring the soliton behavior.
 

 
 
\section{Array of two-level atoms}\label{Sec.TLAarray}
As a first example of a quantum nonlinear medium, we consider an array of two-level atoms~(TLA) where the atomic Hamiltonian is simply given by $\hat H_a=(\omega_{eg}-i\Gamma_0/2)\sum_n\hat\sigma_{eg}^n\hat\sigma_{ge}^n$. In particular, we focus on the ideal waveguide QED scenario of low dissipation, $\Gamma_0\rightarrow 0$. In this regime there is a non-negligible probability that two propagating photons can interact simultaneously with the same atom  inducing a nonlinear optical response. Motivated by the previous discussion, we will first calculate the single-excitation dispersion relation, in order to identify energy gaps in which two-excitation bound states might exist, and  proceed to calculate their spectrum. We then consider the many-excitation limit observing numerically the emergence of a propagating soliton.  We will consider both cases of a chiral waveguide coupling, $\Gamma_L=0$ and $\Gamma_R=\Gamma$, where the results for multiple excitations are exactly known from the Bethe ansatz for photons~\cite{mahmo_calajo}, and a bidirectional waveguide, $\Gamma_L=\Gamma_R=\Gamma/2$.





\subsection{Single excitation sector}


\subsubsection{Chiral array}


We work in a rotating frame in order to discard the atomic frequency $\omega_{eg}$ and we assume  no additional dissipation~($\Gamma_0=0$). 
For chiral coupling, one can readily observe that the specific positions of the atoms do not affect the physics, provided that no two atoms are at the same position. In particular, in the master equation \eqref{eq:MasterEq} and in the input-output equations \eqref{in_outR} and  \eqref{in_outL}, one can make the transformation $\hat \sigma_{ge}^ne^{ik_0x_n}\rightarrow\hat \sigma_{ge}^n$~\cite{Pichler2015,ChiralRev} that eliminates all of the position-dependent phases $e^{ikx}$. For example, the effective Hamiltonian~\eqref{Heff} after this transformation reads:
\begin{equation}\label{Heffchi}
\hat H_{\rm eff}=-\frac{i}{2}\Gamma\sum_{n}\hat\sigma^n_{eg}\hat\sigma^m_{ge}-i\Gamma\sum_{n>m}\hat\sigma^n_{eg}\hat\sigma^m_{ge}.
\end{equation}
%
The second term~(involving the sum $n>m$) enforces that a given atom does not couple to atoms to the right, and thus encodes the unidirectional propagation of excitations along the array. 
For an infinite lattice of atoms ($N\rightarrow \infty$), the single-excitation sector is diagonalized by Bloch waves of wavevector $k$, $\hat H_{\rm eff}|\psi_k\rangle=J_k|\psi_k\rangle$ and the dispersion relation $J_k$ is given by 
\begin{equation}\label{dis_chi}
J_k=-\frac{\Gamma}{2}\rm cot\left[\frac{kd}{2}\right].
\end{equation}
The dispersion relation is plotted in Fig.~\ref{fig:setup_dis1ex}(a) and is characterized by a positive group velocity, $v^{(1)}_g(k)=\Gamma d/(4\sin^2{\left(\frac{kd}{2}\right)})$, for every value of $k$.
Interestingly, Eq.~(\ref{dis_chi}) cannot be derived by directly diagonalizing the Hamiltonian~\eqref{Heffchi} for a finite system, as $H_{\rm eff}$ is a triangular matrix with trivial eigenvalues. Instead, one should realize that the dispersion relation in principle should be real, and thus diagonalize the Hermitian part of the effective Hamiltonian:
\begin{equation}\label{Heff_herm}
\hat H_{\rm h}=-\frac{i\Gamma}{2}\left(\sum_{n>m}\hat\sigma^n_{eg}\hat\sigma^m_{ge}-\sum_{n>m}\hat\sigma^m_{eg}\hat\sigma^n_{ge}\right).
\end{equation}
This observation will be useful to numerically find the dispersion relation of multi-excitation bound states.


\subsubsection{Bidirectional ordered array}

\begin{figure}
\centering
\includegraphics[width=0.48\textwidth]{dis_bidi_waveguide_1ex.pdf}%
\caption{Single excitation dispersion relation $J_k$~(in units of the emission rate $\Gamma$) versus dimensionless wavevector $kd$ for the (a) chiral waveguide case, and  (b) the bidirectional case with $k_0d=0.2$. The vertical dashed lines indicate the values of the wavevector where the dispersion relation diverges.}
\label{fig:setup_dis1ex}
\end{figure}

For a bidirectional ordered array  Eq.~\eqref{Heff} reads
\begin{equation}\label{Heffbidi}
\hat H_{\rm eff}=-\frac{i}{2}\Gamma\sum_{mn}e^{ik_0|x_m-x_n|}\hat\sigma^n_{eg}\hat\sigma^m_{ge}.
\end{equation}
An infinite lattice is again diagonalized by Bloch waves and the  dispersion relation
 has previously been calculated to be~\cite{Albrecht}
\begin{equation}
J_k=\frac{\Gamma}{4}\left(\rm cot\left[\frac{(k_0+k)d}{2}\right]+\rm cot\left[\frac{(k_0-k)d}{2}\right]\right),\label{eq:Jk_bidirectional}
\end{equation}
as plotted in Fig.~\ref{fig:setup_dis1ex}(a). 
The dispersion relation~\eqref{eq:Jk_bidirectional} exhibits two distinct branches, along with a band gap near the atomic resonance frequency where no excitations are allowed. The dispersion relation reveals the polaritonic nature of the excitations, being close to the atomic resonance frequency~($J_k\sim 0$) for wavevectors significantly different than the resonant wavevector of light~($|k|\neq k_0$), while strongly hybridizing with light around $|k|\approx k_0$. Due to the Markov approximation, the slope of the polariton bands approaches the dashed vertical lines $|k|=k_0$ (see Fig.~\ref{fig:setup_dis1ex})(b) rather than a line of slope $c$. Note that the same behavior was also occurring in the chiral case with the dispersion diverging at $k=0$.

Notably, the dispersion relations of Eq.~\eqref{dis_chi} and Eq.~\eqref{eq:Jk_bidirectional} are purely real. This reflects the fact that in an infinite atom-waveguide system with no additional dissipation, the full system Hamiltonian \eqref{H_tot} (absent the input field) is Hermitian and thus the system forms lossless polaritons. Separately, we note that a large body of work has been devoted to the study of super- and sub-radiant eigenstates of finite atomic chains coupled to waveguides\cite{Albrecht,Loic,buchler}. In this case, the superradiant eigenstates are quasi-spin wave excitations with wavevectors close to the resonant wavevector of the waveguide ($k=k_0$ for bi-directional, and $k=0$ for chiral). As $N$ increases, the distribution of wavevectors that are superradiant becomes increasingly narrow, thus providing consistency with the lossless dispersion in the infinite system limit.




%In both scenarios and for  finite atomic arrays,  these regions, where the polariton bands become strongly hybridized with light, are the ones corresponding to the most radiant eigenstates of the Hamiltonian. The radiant and subradiant behavior of the eigenstates of \eqref{Heffbidi} has been  discussed in previous works \cite{Albrecht,Loic,buchler}. Here, being mainly interested in deriving  the dispersion properties, we consider an infinite waveguide with no additional dissipation, where  atoms and light must form completely lossless polaritons. In this case,  although the effective Hamiltonian is in principle non-Hermitian, the dispersion relation is purely real.


%A special case in the dispersion occurs for $k_0d=\pm\pi$, where $J_k=0$. In this scenario, known as mirror configuration~\cite{chang_mirror},  Eq.~\eqref{Heffbidi} becomes equivalent to a Dicke model with a set of $N-1$ sub-radiant  modes and a single mode with super-radiant decay rate $\Gamma_{\rm sr}=N\Gamma$ \cite{Albrecht}. In the following discussion we will not consider this scenario restricting our discussion to the interval $k_0d=]-\pi,\pi[$.




\subsection{Two excitation sector and bound states}\label{Secarray2ex}

Similar to the single-excitation sector, we can look for solutions of $\hat H_{\rm eff}|\psi^{(2)}\rangle=E|\psi^{(2)}\rangle$, where the two-excitation eigenstate generically takes the form $|\psi^{(2)}\rangle=\sum_{m<n}c^{(2)}_{mn}|e_m,e_n\rangle$. It is convenient to re-parametrize the coordinates in terms of center-of-mass~($x_{\rm cm}=(x_m+x_n)/2$) and relative coordinates~($x_r=|x_n-x_m|$). 
We will again consider the infinite system limit, where the dispersion relation should be purely real, and utilize Bloch's theorem to write the center-of-mass wave function in terms of a wavevector $K$,
\begin{equation}\label{2 ans}
 |\psi^{(2)}\rangle=\sum_{x_{\rm cm}}e^{iKx_{\rm cm}}f(x_r)|x_{\rm cm}-\frac{x_r}{2},x_{\rm cm}+\frac{x_r}{2}\rangle,
\end{equation}
%
where $f(x_r)$ is a generic function of the relative coordinate. This form allows one to reduce the two-excitation problem to a single-excitation one involving just the relative coordinate, and derive its spectrum as a function of $K$.

\subsubsection{Chiral array}\label{Sec_chiral2ex}

Let us first consider the chiral waveguide case. This scenario is solvable by Bethe ansatz incorporating both photons and atoms~\cite{mahmo_calajo}, so we just briefly review how the same results emerge from the spin model. In the relative coordinate frame  Hamiltonian~\eqref{Heff_herm} can be rewritten as:
\begin{equation}\label{HK_chiral}
\hat H^K=-i\frac{\Gamma}{2}\sum_{r,r'>0}\sum_{\epsilon=\pm1}\left[e^{-i\frac{K}{2}|x_r+\epsilon x_{r'}|}-e^{i\frac{K}{2}|x_r+\epsilon x_{r'}|}\right]\hat\sigma^+_{r}\hat\sigma^-_{r'},
\end{equation}
which depends parametrically on the center of mass momentum $K$.
The full two-excitation spectrum can be easily obtained 
by numerically diagonalizing~\eqref{HK_chiral}. The real part of the eigenvalues, ${\rm Re}(E_K)$, as a function of the center of mass momentum $K$ is plotted in  Fig.~\ref{fig:chiral_varie}(a).  We implement this by truncating the single-particle problem of Eq.~\eqref{HK_chiral} to a large, but finite set of sites, with $1\leq r,r' \leq N-1$ and diagonalizing the resulting $(N-1)\times (N-1)$ matrix. The center of mass momenta are sampled at discrete points $K=2\pi m_K/N$, with $m_K=-N/2,...N/2-1$ for even $N$.

As anticipated in Sec.~\ref{Sec. model}, we observe in Fig.~\ref{fig:chiral_varie}(a) an energy continuum (blue eigenvalues) that corresponds to unbound states whose  energies can be obtained analytically by adding up the single-particle dispersion relation $\omega(K)=J(q)+J(K-q)$. 
 This continuum exhibits a gap where energy and momentum conservation are not simultaneously satisfied, whose boundaries are indicated by black curves.
Within this gap, we clearly observe a discrete dispersion branch~(red curve),  which  satisfies condition~\eqref{BS condition} and it can be identified as a two-excitation bound state.
\begin{figure}
\centering
\includegraphics[width=0.48\textwidth]{chiral_varie.pdf}%
\caption{(a) Spectrum $E_K$ of two-excitation eigenstates, as obtained from Eq.~\eqref{HK_chiral}, as a function of the center of mass momentum $K$. The red curves indicate the bound state dispersion relation, while the blue dots indicate continuum states whose boundaries are given by the black curves. (b) Population distribution of the bound state as a function of the dimensionless relative coordinate $r=x_r/d$, for different values of total momentum $Kd$. These numerical calculations were performed for $N=150$.}
\label{fig:chiral_varie}
\end{figure}
This bound state exists for any value of $K$ and its dispersion relation can be 
exactly computed,  by using a Fourier transform ansatz, $|q\rangle=\sum_{r>0}e^{iqx_r}\hat\sigma^{r}_{eg}|0\rangle$,
\begin{equation}\label{dis_2ex}
E_{\rm BS}(K)=-2\Gamma\rm cot\left[\frac{Kd}{2}\right].
\end{equation}
The wave function itself is given by $p(r)\propto e^{-x_r \kappa}$ with $\kappa d=-i\log{\cos{\left(\frac{Kd}{2}\right)}}$, which is plotted in Fig.~\ref{fig_bandK}(b) for different values  of $Kd$.
The exact dispersion also allows to derive the bound state group velocity, 
\begin{equation}\label{vgchiral}
v^{(2)}_g(K)=\frac{\partial E_{\rm BS}(K)}{\partial K}=\frac{\Gamma d}{\sin^2{\left(\frac{Kd}{2}\right)}}=4v^{(1)}_g(k).
\end{equation}
In other words, the bound state travels faster than a single excitation. This result coincides with the Bethe ansatz calculation of Ref.~\cite{mahmo_calajo}, which is not surprising, since the polariton propagation speed is almost entirely dictated by the atomic dispersion, rather than the speed of light itself.




\begin{figure*}[!t]
\includegraphics[width=1.8\columnwidth]{band_bidi.pdf}%
\caption{(a) Spectrum $E_K$ of two-excitation eigenstates of a bi-directional waveguide, as obtained from Eq.~\eqref{HK_bidi}, as a function of the center of mass momentum $K$ and for different values of the lattice constant $k_0 d$. The blue dots indicate continuum states, whose boundaries are denoted by the black curves. The solid red curves indicate the bound state dispersion relation, while the red dots indicate the special values $Kd=0$ and $Kd=\pm\pi$ where the bound-state properties can be computed analytically. The red dashed lines indicate the values of $K=\pm2k_0$ where the bound state dispersion branch emerges from the continuum. The horizontal dashed black lines in the first two panels correspond to twice the frequency of the input field used in Fig.~\ref{fig:pulse_bidi}. (b) Population distribution of the most localized state as a function of the dimensionless relative coordinate $r=x_r/d$, for different values of total momentum $Kd$ and lattice constant $k_0 d$. These numerical calculations were performed for $N=150$.}
\label{fig_bandK}
\end{figure*}




\subsubsection{Bidirectional ordered array}


Using the same methodology as in the chiral case, we rewrite Hamiltonian ~\eqref{Heffbidi}  in the relative coordinate frame:
\begin{equation}\label{HK_bidi}
\hat H^K=-i\frac{\Gamma}{2}\sum_{r,r'>0}\sum_{\epsilon,\epsilon'=\pm1}e^{i(k_0+\epsilon\frac{K}{2})|x_r+\epsilon'x_{r'}|}\hat\sigma^{r}_{eg}\hat\sigma^{r'}_{ge},
\end{equation}
and numerically compute the full spectrum, which is plotted in Fig.~\ref{fig_bandK}(a) for different values of the atomic distance $k_0d$. 

Similarly as in the chiral case, we observe a continuum of unbound states~(blue points) with energies given by $\omega(K)=J(k)+J(K-q)$ (black curve).
 On the other hand, a gap of forbidden energies occurs in the intervals $Kd\in[2k_0d,\pi]$ and $Kd\in[-\pi,-2k_0d]$ (red dashed lines), and at the singular point $Kd=0$. Note that the eigenvalues surrounding $Kd=0$ are characterized by a vanishing density of states but do not exhibit a gap. 
Within the gaps, we find some discrete two-excitation bound state energies that satisfy condition~\eqref{BS condition} and, besides the singular point at $Kd=0$, 
present a well-defined dispersion branches (red lines).
For the special values $Kd=0$ and $Kd=\pm\pi$ the two-excitation bound states have a simple analytical form~\cite{dimermolmer2},  which can be obtained  by using a Fourier transform ansatz in the relative coordinate.
These three dimers have energies $\omega_0=2\Gamma \rm cot(k_0d)$ and $\omega_{\pi}=\omega_{-\pi}=2\Gamma \rm cot(2k_0d)$, and
are characterized  by a complex relative  momentum, $q_0 d=\log{\cos{(k_0d)}}$ and $q_{\pm\pi}d=-\log{\cos{(2k_0d)}}$. These complex momenta  lead to an exponential localization of the wave function along the relative coordinate, which explicitly reads $p_0(r)\propto e^{-2x_r q_0}$ and $p_{\pm\pi}(r)\propto \cos^2( \frac{Kx_r}{2})e^{-x_r q_{\pm\pi}}$. Note that while for the dimer at $Kd=0$ the two excitations have a relative distance of $d$ given by the TLA saturation, for $Kd=\pm\pi$ they are separated by $2d$ due to the oscillating term in the population. An analytical expression for the bound states with generic center of mass momentum is reported in Ref.~\cite{Bakkensen}.
The population distribution of these states is shown in Fig.~\ref{fig_bandK}(b) as a function of the relative coordinate, for different representative values of the total momentum $Kd$ and of the atomic separation $k_0d$. A full localization is indeed observed when the parameter choices coincide with a gap. However, the neighborhood around $Kd=0$ does not exhibit a true bound state solution, as evidenced by the large-$r$ tail in the population of the most localized state plotted in Fig.~\ref{fig_bandK}(b) for $Kd=0.1$. These quasi-localized states in the continuum arise from scattering resonances and are extensively discussed in Ref.~\cite{Bakkensen}.


Such a bound state dispersion relation, as far as we know, has not been previously derived when the photonic degree of freedom is explicitly kept. The reduced complexity is one of the strengths of the spin model approach, which we will soon see also extends to numerically exploring the many-body limit.



\subsection{Many-body pulse propagation}\label{Sec_TLA_many}


%Especially if you make the previous revision, I would make more of an emphasis on this point: in the chiral plots, we never assume mean field. Rather, we perform full many-body calculations, and see numerically (and analytically) that there is indeed a transition to a regime where the SIT description is valid.
%While here, we have simply summarized the results already derived for two-level atoms coupled to a chiral waveguide, in the following, we would like to explore the nature of the transition to large-photon number for other systems, and look for the emergence of soliton behavior.


While analytically deriving the properties of few-excitation bound states (beyond two) seems generally challenging within the spin model, this framework still offers a route towards numerical investigations, even in the many-body limit. 
In this section we utilize such numerics to explore the dynamics of these many-body bound states.  This problem has been recently studied in Ref.~\cite{mahmo_calajo} for an array of atoms coupled to a chiral waveguide, which established the connection between quantum photon bound states
 and the emergence of the SIT soliton discussed in Sec.\ref{Sec.SIT}. In the following, we first briefly summarize the general methodology and the results obtained for the chiral case and then we explore the nature of this transition in a different setting: a bidirectional array.




\subsubsection{Methodology}\label{Secmet}

In order to investigate the change in behavior from weak to strong pulses, we utilize a previously developed matrix product state (MPS) algorithm for this problem (see App.~\ref{AppMPS} and Ref.~\cite{MPSJames,james_ryd,mahmo_calajo} for further details). In this representation, the maximum bond dimension needed to obtain convergence in the simulations, $D_{\rm max}$, is directly connected to the entanglement entropy of the system and it acts as an important figure of merit to understand if the dynamics is highly correlated, $D_{\rm max}\gg1$, or if a mean field approximation for the atoms suffices,  $D_{\rm max}\sim 1$.

With this formalism we solve the full emitter dynamics, as governed by the master equation~\eqref{eq:MasterEq}. 
Considering that the finite spatial extent of the bound states generally allows for their excitation given a coherent state input pulse,
we specifically start with the atoms in the ground state $|g\rangle^{\otimes N}$ at $t=0$, and we consider two different  pulse shapes. The first consists of a Gaussian pulse, peaked at time $t_0$, with amplitude $\mathcal{E}_{\rm in}(t)=\sqrt{n_{\rm ph}}e^{(t- t_0)^2/(2 \sigma^2)}/(\sqrt{\sigma}\pi^{1/4})$, where $n_{\rm ph}$ is the average number of photons in the pulse and $\sigma$ the pulse width.  
The second, useful  to capture the solitonic transition, has an SIT-like solitonic shape~\cite{McCall1,McCall2,Bullough,mahmo_calajo},  $\mathcal{E}_{\rm in}(t)=(\tilde n_{\rm ph}\sqrt{\Gamma_R}/2)\operatorname{sech}(\tilde n_{\rm ph}\Gamma_R (t-t_0)/2)$,  that matches the 
SIT solution given in Eq.~\eqref{eqSIT_solution}. Here  $\tilde n_{\rm ph}=\sqrt{n_{\rm ph}^2+(\omega_{\rm in}/\Gamma)^2}$ is the generalized Rabi amplitude and we remind that for the chiral case $\Gamma_R=\Gamma$.

Once sent an incoming pulse trough the system, the transmitted intensity $I_R=\langle \hat E_R^{\dagger}(t) \hat E_R(t)\rangle$ and the equal-time higher-order correlation functions $G^{(m)}(t)=\langle \left[ \hat{E_R}^\dagger(t)\right]^m\left[\hat{E_R}(t)\right]^m \rangle$~($m\geq 2$) are
 computed by using the expression for the output field operator given in Eq.~\eqref{in_outR} in terms of the input field amplitude and emitter correlations.
 


\subsubsection{Many-body bound states-SIT transition in a chiral array}


\begin{figure}
\includegraphics[width=0.48\textwidth]{Fig_chiral_pulse.pdf}%
\caption{Few to many-body pulse propagation in a chiral  array of $N=30$ atoms. In (a) we plot the time-dependent output intensity $I_R(t)$ and correlation functions $G^{(n)}(t)$, given a coherent Gaussian input pulse with width $\sigma\Gamma=3$ and average photon number $n_{ph}=2.0$. The input frequencies are set to resonance with the atomic transition, $\omega_{\rm in}=0$.  In (b)-(c) we plot the output intensity $I_R(t)$ for a solitonic input pulse (see main text) with different strengths $ n_{\rm ph}$ as indicated in the panels.  In (c) the black dashed line corresponds to  the mean field SIT soliton solution. In all plots the peak of the input pulse is indicated by the grey vertical line and is set to $\Gamma t_0=10$ for (a)-(b) and  $\Gamma t_0=0.3$ for (c). The vertical dashed lines  indicate  the expected delay of the many-body photon bound states. In the inste of (c) we also plot the atomic population of an atom in the middle of the array  as a function of time for the same average photon number as in (c). The simulation has been performed with an MPS based quantum trajectories algorithm involving $N_t=5000$ trajectories and maximum bond dimension $D_{\rm max}=40$.}
\label{fig:pulse_chiral}
\end{figure}

For a chiral array, it has been proven that a coherent pulse resonant with the atomic transition can efficiently excite $n_{\rm ph}$-excitation bound states that experience a photon number-dependent time delay, $\tau_{n_{\rm ph}} = 4N/(n_{\rm ph}^2\Gamma)$. Note that for $n_{\rm ph}=2$, this result coincides with the bound state group velocity derived in Eq.~\eqref{vgchiral}. In sufficiently long systems, the combination of photon number uncertainty of the coherent state and the number-dependent delay results in a train of correlated photons ordered by photon number at the output~\cite{mahmo_calajo}, as shown in Fig.~\ref{fig:pulse_chiral}(a). Here we used an input Gaussian pulse, on resonance with the atomic transition and with $n_{\rm ph}=2$  and $\sigma\Gamma=3$, and we plotted   the output intensity and the photon correlation functions as a function of time, indicating with the vertical dashed line the expected delay.  Increasing the average photon number of the input pulse, large photon bound states with small delay are excited and the number components become progressively less separated. This is shown in Fig.~\ref{fig:pulse_chiral}(b)-(c) where we used the solitonic input pulse and we compared the ouptut intensities for different average photon numbers $n_{\rm ph}$. Note that here  the incoming pulse is on resonance with the atomic transition thus $\tilde n_{\rm ph}=n_{\rm ph}$. For sufficiently large photon number, the individual bound states cannot be resolved anymore and a single soliton-like peak emerges (see Fig.~\ref{fig:pulse_chiral}(c)). This behavior is also associated to a full $2\pi$ rotation of the atoms, as shown in the inset of Fig.~\ref{fig:pulse_chiral}(c), where we plot the  time-dependent excited-state population $p_s(t)$ of a single atom in the middle of the array.
In this case, it can be explicitly shown that a coherent state distribution of bound states of different photon number in fact coincides with the classical SIT soliton~\cite{McCall1,McCall2,Bullough,mahmo_calajo}, given in Eq. \eqref{eqSIT_solution} and plotted in Fig.~\ref{fig:pulse_chiral}(c) with the black dashed line, thus establishing a crossover from quantum to classical nonlinear optics. Indeed, one can note that the predicted time delay coincides with the one given in Eq.~\eqref{delaySIT} obtained by the solution of the SIT mean-field equations. 
While here, we have simply summarized the results already derived for two-level atoms coupled to a chiral waveguide, in the following, we would like to explore the nature of the transition to large photon number for other systems, and look for the emergence of soliton behavior.





\subsubsection{Exciting many-body bound states in a bidirectional array}
 
\begin{figure}
\includegraphics[width=0.48\textwidth]{Ir_N30_bidi.pdf}%
\caption{Few to many-body pulse propagation in a bi-directional array of $N=30$ atoms. In (a)-(b) we plot the time-dependent output intensity $I_R(t)$ and correlation functions $G^{(n)}(t)$, given a coherent Gaussian input pulse with width $\sigma\Gamma=3$ and average photon number $n_{ph}=2.0$, for two different values of the atomic distance $k_0d$. The input frequencies are respectively $\omega_{\rm in}=0.33$ in (a) and $\omega_{\rm in}=-0.4\Gamma$ in (b) and are indicated by the black horizontal dashed lines of Fig.~\ref{fig_bandK}(a). In (c)-(d) we plot the output intensity $I_R(t)$ for a solitonic input pulse (see main text) with different strengths $\tilde n_{ph}$ as indicated in the panels. Here we fix $k_0d=0.35\pi$ and $\omega_{\rm in}=-0.4\Gamma$. In (d) the black dashed line corresponds to  the mean field SIT soliton solution. In all plots the peak of the input pulse is indicated by the grey vertical line and is set to $\Gamma t_0=10$ for (a)-(c) and  $\Gamma t_0=0.3$ for (d). The vertical dashed lines in  (b) and (c) indicate  the expected delay of the two-photon bound states. In the inste of (d) we also plot the atomic population of an atom in the middle of the array  as a function of time for same average photon number as in (d).  The simulation has been performed with an MPS based quantum trajectories algorithm involving $N_t=5000$ trajectories and maximum bond dimension $D_{\rm max}=40$.}
\label{fig:pulse_bidi}
\end{figure}

We now explore the analogue of these effects in a bi-directional array. One can already see that a major difference compared to the chiral case is the emergence of a single-excitation band gap around the atomic resonance, as shown in Fig.~\ref{fig:setup_dis1ex}(a), which causes strong reflectance of weak resonant pulses. On the other hand, the experimental observation of SIT does not rely on chirality, and so one might expect that SIT will arise for strong enough input pulses in the bi-directional case. In the following, we investigate the how the behavior of the system changes going from weak to strong pulses.


We start by considering an input Gaussian pulse as the one used for the chiral case with $n_{\rm ph}=2$  and $\sigma\Gamma=3$.
Representative output intensities and correlation functions for such a setup are plotted in Fig.~\ref{fig:pulse_bidi}(a)-(b), for an array of $N=30$ atoms with lattice constants $k_0d=0.2\pi$ and $k_0d=0.35\pi$. Considering a large array is crucial because the propagating bound states experience a time delay at the output, which is proportional to the number of emitters and the inverse of their group velocity evaluated at the frequency of the input pulse $\omega_{\textrm{in}}$, i.e. $\tau=Nd/v_g(\omega_{\rm in})$. 
Note that the bound state energies occur at frequencies away from the atomic resonance, so we consider input fields with a slightly detuned central frequency, i.e. $\omega_{\textrm{in}}\neq \omega_{eg}$. For the lattice constant $k_0d=0.2\pi$, we set $\omega_{\textrm{in}}\approx 0.33\Gamma$, which is situated within the single-excitation band gap of Fig.~\ref{fig:setup_dis1ex}(a), which suppresses transmission of the single-photon component of the pulse. For the case of $k_0d=0.35\pi$, the chosen input frequency $\omega_{\textrm{in}}=-0.4\Gamma$ instead coincides with the flat, uppermost region of the lower dispersion branch, leading to a large delay of the transmitted single-excitation component.
%there is a non-trivial frequency range in which two-photon bound states exist for each lattice constant, and thus to efficiently excite them we choose the central frequency of the pulse to be $\omega_{\textrm{in}}=E_{BS}(K=2k_0)/2$, such that two-photon energy and phase matching are satisfied. 


In the case of inter-atomic distance $k_0d=0.2\pi$, the two-excitation bound state dispersion (Fig.~\ref{fig_bandK}(a)) is extremely flat and significantly deviates from a simple linear slope over the bandwidth of the incoming pulse. Thus, the excited bound states exhibit both a large delay and significant distortion at the output, which suppresses the appearance of a clear peak associated to the bound states in the output intensity of Fig.~\ref{fig:pulse_bidi}(a). One can also observe a faster exiting peak in the higher-order correlation $G^{(3)}(t)$, which could correspond to the excitation of higher photon number bound states. A similar peak immediately following the input is also observed in Fig.~\ref{fig:pulse_bidi}(b) for the lattice constant $k_0d=0.35\pi$. Importantly though, here we also resolve another intensity peak with a significant delay $\Gamma(\tau-t_0)\sim 30$, and an associated peak in $G^{(2)}(t)$. This peak can be clearly associated to the excitation of a two-photon bound state resonant with the input field, as indicated by the dashed black line in the second panel of Fig.~\ref{fig_bandK}(a). The numerically calculated group velocity $v_g(\omega_{\textrm{in}})$ at this frequency can be used to predict the expected delay (vertical dashed line in Fig.~\ref{fig:pulse_bidi}(b)), and coincides closely with the peaks in the intensity and $G^{(2)}$. %We also note that unlike the case of $k_0d=0.2\pi$, the input field frequency does not sit within the single-excitation gap, but instead excites the flat, uppermost region of the lower dispersion branch, leading to a large delay of the transmitted single-excitation component.

In summary, as with the chiral case, it is possible to observe peaks in the output field and its correlations, which are associated with the excitation of few-photon bound states. However, unlike the chiral case, this observation requires more fine tuning in the bi-directional array, due to the finite bandwidth of the bound state band and its possibly strong group velocity dispersion.


\subsubsection{Towards the SIT limit}
We now investigate progressively higher photon number inputs, and the robust transition toward the mean field solution of SIT. Anticipating such a transition, we now drive the system with the SIT-like solitonic input field introduced in Sec.~\ref{Secmet}. This creates a large overlap between the field and the many-photon bound state, assuming the latter approaches the SIT solution. 

In Fig.~\ref{fig:pulse_bidi}(c)-(d) we plot the resulting output intensity for photon numbers ranging from small~($\tilde{n}_{\rm ph}=2$) to large. In  Fig.~\ref{fig:pulse_bidi}(c), for small $\tilde{n}_{\rm ph}$, we clearly observe the appearance of the two bound state peaks previously discussed. Not surprisingly, the second, more delayed peak associated with the two-photon bound state gets rapidly suppressed as the photon number is increased. For even larger photon numbers as plotted in Fig.~\ref{fig:pulse_bidi}(d), we see that the output intensity gradually approaches the mean field solitonic solution
\begin{equation}
\bar I_R(t)=\frac{\tilde n^2_{\rm ph}\Gamma_R}{4}\operatorname{sech}^2\left[\frac{\tilde n_{\rm ph}\Gamma_R(t-t_0-\tau_{\tilde n_{\rm ph}})}{2}\right],
\end{equation}
which is delayed respect to the input by an amount  $\tau_{\tilde n_{\rm ph}}=4N/(\tilde n^2_{\rm ph}\Gamma_R)$~\cite{mahmo_calajo}. The convergence toward this mean field solution also indicates that the medium becomes more transparent to the large photon number pulse, and that the impedance matching of the pulse to the atomic array becomes irrelevant despite the bidirectional coupling to the waveguide. 
As in the chiral case, the emergence of a transparent soliton is 
also signaled by  a full $2\pi$ rotation of the excited-state population $p_s(t)$, as shown in the inset of Fig.~\ref{fig:pulse_bidi}(d).
%This behavior is also associated to a full $2\pi$ rotation of the atoms, as shown in the inset of Fig.~\ref{fig:pulse_chiral}(c), where we plot the  time-dependent excited-state population $p_s(t)$ of a single atom in the middle of the array.
Finally, we comment that, similarly as  studied in Ref.~\cite{mahmo_calajo}, the transition toward classical behavior is evidenced by the decreasing maximum bond dimension required for convergence in the MPS simulations, from $D_{\rm max}\sim 40$, for $\tilde n_{\rm ph}\sim1$, to $D_{\rm max}\sim 1$, for $\tilde n_{\rm ph}\gg1$. %\textcolor{red}{How you do the simulations needs to be more clearly stated starting from the first simulation. I guess you use MPS even for low photon numbers (such as 5a), but it's not really clear from the main text.}

%This analysis shows how, also for a generic bidirectional array, the photon bound states can be interpreted as the quantum few-body counterpart of classical optical solitons. In Sec.~\ref{Sec.Rydmedia} we will discuss how such a connection arises in another quantum nonlinear medium, a Rydberg atomic ensemble.

\subsubsection{Experimental considerations}\label{experimentTLA}

The bound states studied here have the advantage of not relying on a chiral waveguide setup, which was the focus of Ref.~\cite{mahmo_calajo}.
%This makes our predictions  observable in different waveguide QED system. Efficient coupling between quantum emitters and nanophotonics waveguide has been demonstrated in several systems based on  nanofibers~\cite{Mitsch2014}, photonic crystal waveguides~\cite{Sollner2015,Goban2014,Hood2016} or diamond nanostructures~\cite{Sipahigil2016}. These structures, even if extremely promising, could still not be optimal to probe the effect for their indeterminacy on building ordered atomic arrays and for the high external loss $\Gamma_0\sim\Gamma$ that could affect the observation of the pulse separation.
Although schemes to realize chiral coupling with superconducting qubits coupled to microwave transmission lines has been proposed~\cite{Guimond,Gheeraert}, it is more routine to realize bi-directional coupling in circuit QED setups either to microwave transmission lines or coupled resonator arrays~\cite{Sundaresan,painter1,painter2,ustinov,ustinov_topo,marcoBS}. Such systems are ideal to investigate the physics predicted above, as the artificial qubits can experience extremely low loss into channels other than waveguide emission, $\Gamma_0/\Gamma<10^{-2}$, and can be located at precise positions along the waveguide.


\section{Photon bound states in Rydberg media}\label{Sec.Rydmedia}


We now consider a different quantum nonlinear medium consisting of a free space ensemble of Rydberg atoms. The level scheme of the atoms is pictorially shown in Fig.~\ref{Fig.Rtran}(a) where, besides the $|g\rangle$ and $|e\rangle$ states, there is an additional long-lived Rydberg level $|s\rangle$, which couples to $|e\rangle$ via a uniform, classical control field of Rabi frequency $\Omega$ and detuning $\delta$. The propagation of photons interacting with atoms on the $|g\rangle$-$|e\rangle$ transition can become highly nonlinear, due to strong van der Waals interactions between Rydberg excitations. In particular, the van der Waals interaction is responsible for shifting the resonant energy of a Rydberg level by an amount $V(x_r)=C_6/x_r^6$ given the presence of another Rydberg excitation nearby, with $x_r$ being the relative distance between the two atoms. This can significantly modify the propagation of two photons~(or more precisely, Rydberg polaritons) that are closer than a Rydberg blockade radius $r_b$, to be defined later. We first introduce the effective model used to describe the system, and then we numerically investigate the many-body photon dynamics. The results obtained by the full simulation are then interpreted in terms of an effective theory for a generalized version of SIT, which we show as being able to capture the main emergent features.

%Here  showing that in this case a generalized version of SIT can emerge. 

\subsection{Model and spectrum}\label{Sec.modelRyd}
The three-dimensional Rydberg ensemble can be approximately treated as one dimensional, provided that the blockade radius is larger than the beam waist of the photons exciting the $|g\rangle$-$|e\rangle$ transition, such that the photon dynamics occurs within a single transverse mode. As anticipated in Sec.~\ref{Sec. model}, this situation can then be mapped to a 1D array of atoms coupled to a chiral waveguide~\cite{bienas,MPSJames}, with a large additional and independent excited-state dissipation rate $\Gamma_0\gg \Gamma$ to capture the emission of photons into $4\pi$ modes other than the mode of interest. The chirality suppresses reflection from the array~(reproducing the negligible reflection of a free-space ensemble). We will later see how the artificial parameters of our microscopic model~(such as $N,\Gamma,\Gamma_0$) can be matched to physical, macroscopic quantities in an actual ensemble, and in particular, how a numerically tractable atom number $N$ and moderate $\Gamma_0$ for our system can be used to deduce the physics in an ensemble that exhibits much larger $N$ but simultaneously larger $\Gamma_0$.

Within the waveguide mapping, the corresponding spin model~(see  Sec.~\ref{Secspin}) in a rotating frame is given by
\begin{equation}\label{Haryd}
\begin{split}
\hat H_a=&-\frac{i\Gamma_0}{2}\sum_n\hat\sigma_{eg}^n\hat\sigma_{ge}^n -\delta\sum_n \hat\sigma^n_{sg}\hat\sigma^n_{gs}+\Omega\sum_n\left(\hat\sigma_{es}^n+\hat\sigma_{se}^n\right)\\
&+\sum_{n<m}V(|x_n-x_m|)\hat\sigma^n_{sg}\hat\sigma^n_{gs}\hat\sigma^m_{sg}\hat\sigma^m_{gs},
\end{split}
\end{equation}
where $\hat\sigma^n_{gs}=|g\rangle\langle s|$ ($\hat\sigma^n_{sg}=(\hat\sigma^n_{gs})^\dagger$) and $\hat\sigma^n_{es}=|e\rangle\langle s|$ ($\hat\sigma^n_{se}=(\hat\sigma^n_{es})^\dagger$)  are the operators associated with the $|g\rangle$-$|s \rangle$ and $|e\rangle$-$|s \rangle$ transitions of atom $n$, respectively, and $\delta=\omega_c-\omega_{se}$ is the detuning of the control driving field $\Omega$ with respect to the $e$-$s$ transition. Here, we assume that dissipation on the Rydberg level can be neglected owing to its long lifetime.
\begin{figure}
\includegraphics[width=0.48\textwidth]{dispersion_ryd2.pdf}%
\caption{(a) Sketch of the three levels of a Rydberg atom involved in the scheme. (b) Single excitation dispersion relation associated to the Hamiltonian~\eqref{Haryd} for  $\Omega=2\Gamma$ and  $\delta=3\Gamma$. The color scale is associated to the fraction of population in the Rydberg state $|s\rangle$. The horizontal black dashed line indicates the EIT ($E^{\rm EIT}$) resonance while the red dashed line the Stark shifted one $E_{\rm st}$. (c) Two-excitation spectrum $E_K$~(relative to the Stark shifted resonance $E_{\rm st}$) of the effective spin model \eqref{heffr_off}, as a function of the center of mass momentum $K$ for $C_6/(d)^6=600\bar\Gamma$. In grey we have superimposed the equivalent spectrum (continuum and bound states) obtained by solving the full model with $\delta=-10\Gamma$ and $\Omega=\Gamma$ (see App.~\ref{App.A}). The black lines indicate the boundaries of the continuum states, obtained from the single excitation dispersion relation.}
\label{Fig.Rtran}
\end{figure}


It is instructive to first investigate the single excitation spectrum, in the case of a far off-resonant control field $|\delta|\gg \Omega$. The dispersion relation is derived in App.\ref{App.A} and plotted in Fig.~\ref{Fig.Rtran}(b) for a representative set of parameters.  It presents the usual three polariton branches \cite{lukinEIT,bienas}  and it  is characterized by the occurrence of a band that exhibits electromagnetically induced transparency~(EIT), which is centered around the EIT resonance condition $E^{\rm EIT}=-\delta$~\cite{lukinEIT} (black dashed line). 
Around this resonance, the polariton consists mostly of a Rydberg excitation, as indicated by the color scale of Fig.~\ref{Fig.Rtran}(b) associated to the amount of population on the $|s\rangle$ state. The excited state population $|e\rangle$ and its corresponding emission is suppressed via interference in excitation pathways, allowing this ``dark state'' polariton to propagate without loss~\cite{lukinEIT}.
%These dark modes propagate in the medium with EIT group velocity $v_{\rm EIT}=\Omega^2d/\Gamma$ (indicated by the dotted line in Fig.~\ref{Fig.Rtran}(b)) 
%and effective mass $m_{\rm eff}=\hbar \Omega^2/(2 v_g^2\delta)$. 
%$v_{\rm EIT}=\partial E_k/\partial k|_{k=0}=\Omega^2d/\Gamma$ (indicated by the dotted line in Fig.~\ref{Fig.Rtran}(b)) and effective mass $m_{\rm eff}=\hbar/(2\partial E^2_k/\partial k^2|_{k=0})=\hbar \Omega^2/(2 v_g^2\delta)$. 
%Note that  by solving the  full model~\eqref{H_tot} the EIT group velocity reads $v^{\rm full}_g=\frac{\Omega^2cd}{\Omega^2d+c\Gamma}$~\cite{lukinEIT}. This result reduces to the one previously derived in the \qq{slow-light} regime, i.e. $\Omega^2/\Gamma\ll c/d$, which is equivalent to the Markov approximation at the base of the spin model.

Two or more dark state polaritons in a close vicinity of each other then strongly interact via the Rydberg interaction $V$ in~\eqref{Haryd}. This significantly affects the transparency condition and results in a strong photon-photon interaction, which can be either dissipative~\cite{Peyr}, for $|\delta |\ll\Gamma_0$, or dispersive in nature~\cite{first}, for $|\delta |\gg\Gamma_0$, depending on the detuning of the control field.
Quantum nonlinear optics has been extensively explored around the EIT transition. Notably, in the dispersive regime, this includes the observation of signatures of bound states, via photon bunching correlations in the outgoing field given a cw input~\cite{first,LiangBS}. Theoretically, these states, which can be of rich and varying nature, have been characterized via effective continuum theories~\cite{bienas,Efimov,magrebi}.\\
 
%In this regime, the electric field inside the medium acquires an effective mass and the evolution of the two-photon wavefunction, $f_p(x_r)$,  along the relative coordinate $x_r$, it is described by an effective 1D nonlinear Schr\"oedinger  equation:
%\begin{equation}\label{eff_shro}
%Ef_p(x_r)=\left[-\frac{\hbar}{2m_{\rm eff}}\partial_{x_r}+V_{\rm eff}(x_r)\right]f_p(x_r).
%\end{equation}
%Here the effective potential $V_{\rm eff}(x_r)=\sign{(\delta)}C_6/(x_r^6-\sign{(\delta)}R_b^6)$  depends on the  Rydberg radius $R_b=|C_6\delta/2\Omega^2|^{1/6}$ and on the sign of the control field detuning $\delta$. This potential reduces to a potential well~\cite{bienas}, for $\delta<0$, or to a Coulomb-like potential, for $\delta>0$~\cite{magrebi} owning bound or quasi-bound states solution respectively.
%As mentioned, this effective description holds at the EIT resonance  and accordingly it only reproduces a specific portion of the energy spectrum. Here relies one of the advantage of our spin-model base approach which, as we are going to see, can easily reproduce the full spectrum covering the entire parameters range.




 %\textcolor{red}{Show representative plot, which both shows the perfect transparency corresponding to the EIT condition with two-photon resonance, and a narrow transmission dip associated with the renormalized two-level atom. Qualitatively discuss how nonlinear optics has been extensively explored around the EIT transition, where the presence of Rydberg polaritons and the Rydberg interaction significantly modifies the single-photon transparency condition and results in a strong photon-photon interaction. This includes the investigation of bound states based on continuum theories, which can be of rich and varying nature.}






%\subsection{Single excitation sector and EIT}




%For a single excitation the Rydberg interaction term given in~\eqref{Haryd} does not play a role and it is convenient to rewrite the spin Hamiltonian~\eqref{Heff} in $k$ where it reads:
%\begin{equation}\label{Hrydk}
%H_{\rm eff}=\tilde J_k\sum_k\hat\sigma_{eg}^k\hat\sigma_{ge}^k -\delta\sum_k \hat\sigma^k_{se}\hat\sigma^k_{es}+\Omega\sum_k\left(\hat\sigma_{es}^k+\hat\sigma_{se}^k\right),
%\end{equation}
%where $\tilde J_k=-\frac{\Gamma}{2}\cot{(kd/2)}-i\Gamma_0/2$ is the dispersion of an atomic array coupled to a chiral waveguide derived in~\eqref{dis_chi}. 
%In this space \eqref{Hrydk} can be exactly diagonalized  by considering a linear combination of Rydberg and exited state excitations: $|\psi_k\rangle= \alpha_k|e_k\rangle+\beta_k|s_k\rangle$. This leads the following coupled equations:
%\begin{equation}
%\begin{split}
%&\left(E_k-\tilde J_k\right)\alpha_k=\Omega \beta_k\\
%&\left( \delta+E_k\right)\beta_k=\Omega \alpha_k,
%\end{split}
%\end{equation}
%These equations  provide the upper(U) and lower(L)  dispersion branches
%\begin{equation}\label{disEITex}
%E^{(U/L)}_k=\frac{\tilde J_k-\delta}{2}\pm\frac{1}{2}\sqrt{(\tilde J_k-\delta)^2+4(\Omega^2+\delta\tilde J_k)},
%\end{equation}
%associated to the  states: 
%\begin{equation}\label{polariton}
%|\psi^{(U/L)}_k\rangle=\Psi^{\dagger(U/L)}_k|0\rangle=\frac{(\delta+E^{(U/L)}_k)\hat \sigma^k_{ge} +\Omega \hat \sigma^k_{gr}}{\sqrt{|\delta+E^{(U/L)}_k|^2+\Omega^2}}|0\rangle.
%\end{equation}
%The combination of these two branches  gives the dispersion relation  shown in Fig.~\ref{Fig.Rtran}(b), which  is characterized by the occurrence of an EIT transmission band  centered at the EIT resonance $E^{\rm EIT}=-\delta$~\cite{lukinEIT} (dashed horizontal line). 
%Around this resonance the dispersion~\eqref{disEITex}  reduces to ~\cite{Sorensen_EIT_dis}
% \begin{equation}\label{disEIT}
%E_k\simeq-\delta+\frac{\Omega^2}{\Gamma}kd+\frac{\Omega^2}{\Gamma^2}(\delta-i\Gamma_0/2) (kd)^2+...,
%\end{equation}
%and the state~\eqref{polariton} becomes mainly a Rydberg-like excitation with $\alpha_k\sim 0$, as indicated by the color scale of Fig.~\ref{Fig.Rtran}(b) associated to the amount of population on the $|s\rangle$ state.
%The modes occurring around this resonance, being only weakly populated on the excited state,  do not experience spontaneous emission in free space, i.e.  dark states, and make an otherwise absorbing medium effectively transparent~\cite{lukinEIT}.
%These dark modes propagate in the medium with EIT group velocity $v_g=\partial E_k/\partial k|_{k=0}=\Omega^2d/\Gamma$ (indicated by the dotted line in Fig.~\ref{Fig.Rtran}(b)) and effective mass $m_{\rm eff}=\hbar/(2\partial E^2_k/\partial k^2|_{k=0})=\hbar \Omega^2/(2 v_g^2\delta)$. 
%Note that  by solving the  full model~\eqref{H_tot} the EIT group velocity reads $v^{\rm full}_g=\frac{\Omega^2cd}{\Omega^2d+c\Gamma}$~\cite{lukinEIT}. This result reduces to the one previously derived in the \qq{slow-light} regime, i.e. $\Omega^2/\Gamma\ll c/d$, which is equivalent to the Markov approximation at the base of the spin model.

%\subsection{Rydberg bound states and effective TLA description}\label{Sec_ryd_2ex}

%\textcolor{red}{I think that referees might be disinterested or critical of this section, given that the two-photon bound states have no physical realization. I would suggest to reduce this section significantly, and re-structure along the following lines. First, we can certainly show a plot like Fig. 7, and both the EIT-type bound states and the two-level bound states. We can mention that the EIT-type bound states have been studied before, and we thus focus on the two-level bound states instead. State that one can in principle investigate the dispersion relation of two-photon bound states (perhaps in an Appendix), but explain why they are difficult to observe. However, despite losses, one might wonder whether there is a generalization of SIT as one approaches large photon numbers, and the nature of that transition. (Such a structure motivates the many-body calculation in a more proactive way, rather than just going to the two-level regime with no apparent reason.}

%\textcolor{red}{Separately, I think there might still exist a conceptual hole in Fig. 10. When we mention ``soliton'', people would probably expect to see some evidence that the pulse is shape-preserving, which is not illustrated here. My guess is that you could actually take any generic pulse shape, and with sufficient photon number, see something like Fig. 10b. Fig. 10c, on the other hand, suggests that we are close to a soliton, but not quite, since population has a long tail. To address this, would it be possible either in mean field or with a small bond dimension, to go to very large systems $N\gg 1$, and try to identify some shape preservation at the output? If the system is long enough, perhaps the parts of the input pulse that do not overlap with the soliton get separated out to sufficient extent that we can better identify the soliton shape, and plug that back in as a better ansatz for the input.}

%When more than one excitation gets involved the Rydberg interaction term, given in~\eqref{Haryd}, starts to play a crucial role in the dynamics of the propagating excitations leading to strong nonlinear interactions which can be dissipative~\cite{Peyr}  , for $|\delta |\ll\Gamma_0$, or dispersive in nature~\cite{first}, for $|\delta |\gg\Gamma_0$, depending on the detuning of the control field.
%Within EIT conditions dispersive Rydberg interactions are known to induce propagating bound  excitations that have been observed in several  experiments as photon bunching in the continuous wave outgoing field~\cite{first,LiangBS}. 
%In this regime, the electric field inside the medium acquires an effective mass and the evolution of the two-photon wavefunction, $f_p(x_r)$,  along the relative coordinate $x_r$, it is described by an effective 1D nonlinear Schr\"oedinger  equation:
%\begin{equation}\label{eff_shro}
%Ef_p(x_r)=\left[-\frac{\hbar}{2m_{\rm eff}}\partial_{x_r}+V_{\rm eff}(x_r)\right]f_p(x_r).
%\end{equation}
%Here the effective potential $V_{\rm eff}(x_r)=\sign{(\delta)}C_6/(x_r^6-\sign{(\delta)}R_b^6)$  depends on the  Rydberg radius $R_b=|C_6\delta/2\Omega^2|^{1/6}$ and on the sign of the control field detuning $\delta$. This potential reduces to a potential well~\cite{bienas}, for $\delta<0$, or to a Coulomb-like potential, for $\delta>0$~\cite{magrebi} owning bound or quasi-bound states solution respectively.
%As mentioned, this effective description holds at the EIT resonance  and accordingly it only reproduces a specific portion of the energy spectrum. Here relies one of the advantage of our spin-model base approach which, as we are going to see, can easily reproduce the full spectrum covering the entire parameters range.






%\begin{figure}
%\includegraphics[width=0.48\textwidth]{band_reit.pdf}%
%\caption{(a)-(b) Eigenvalues spectrum as function of the center of mass momentum $K$  for $\delta=-4\Gamma$ (a) and $\delta=4\Gamma$ (b). Panels (c)-(d) correspond to the zoomed aerea inside the black dashed boxes of panels (a)-(b). The  black dashed lines indicate the Stark shifted and the  EIT resonance while  bound states are highlighted by the red color. In all plot the black tick line indicates the boundary of the continuum. In all plot  $C_6/(d)^6=5000\Gamma$, $\Omega=\Gamma$.}
%\label{Fig.Rybderg_band}
%\end{figure}

%\subsubsection{Full  two excitation spectrum}\label{Sec.fullsp_Rydberg}

%Within the two excitation sector the Hilbert space of the Rydberg spin Hamiltonian~\eqref{Heff}  is entirely spanned by the basis set $\{|e_ne_m\rangle,|s_ns_m\rangle,|e_n s_m\rangle,|s_n e_m\rangle\}$ with $n>m$.  Similarly as done for the TLA arrays, we   diagonalize  in the center of mass and relative coordinate frame (see App.~\ref{App.A} for more details) and we consider the dispersive regime where bound states are expected to occur.
%The full spectrum, given by the real part of the eigenvalues as function of the center of mass momentum, is shown in  Fig.~\ref{Fig.Rybderg_band}(a)-(b)
%for both positive and negative  control field detuning.
%We observe  the arising of distinct gaps opening within the continuum of the unbound states (blue region),  whose boundaries (black tick lines) are identified by the relation $\omega(K)=E(q)+E(K-q)$.
%Interesting for our discussion is the region around the  EIT Stark shifted resonance, $E_{\rm st}\sim -2\delta-2\frac{\Omega^2}{\delta}$, highlighted  in the zoomed figures~\ref{Fig.Rybderg_band}(c)-(d). Here we can distinguish two different kind of bound states dispersion branches (red dots) that arise in the gap above and below it. 

%\begin{itemize}
%\item   On the right-down corner  the bound states dispersion progressively emerge from the right-down continuum lobe getting steeper for low momenta $K\sim 0$. For negative detuning  (Fig.~\ref{Fig.Rybderg_band}(c)), they hit the EIT resonance $E^{\rm EIT}_{\rm 2ex}$ and their dispersion is equivalent to the one obtained in~\cite{bienas} by solving the effective Schr\"oedinger equation~\eqref{eff_shro} for a  potential well. 

%\item On the left-upper corner the bound states dispersion branches  instead emerge from the lower continuum with a slope that gets flatter for low momenta and for positive detuning they can intersect the  EIT resonance. 

%\end{itemize}

%It is important to make a distinction with usual EIT effective theories  based on a bosonic description of the atomic medium. Indeed  in our case, as we are going to show in the next section, the bound states  can be  described around the EIT Stark shifted resonance in terms of an effective interacting TLA-model.
%This distinction is evident  for positive detuning where effective theories predict counterpropagating Coulomb-like metastable-bound states ~\cite{magrebi}, while we find exact eigenstates with  positive group velocity (being the directionality implicit in the model). Finally, besides the bound states, we also observe some  eigenvalues (blue dots) in the gap which  correspond to physical unbound states that are the remains of a former existing continuum. These  eigenvalues get  progressively  washed out by going to larger detuning (see App.~\ref{App.RydEIT} for more details).






%\subsubsection{Effective TLA description}\label{Sec.EffectiveTLA}
 
%An effective description of the spectrum shown in Fig.~\ref{Fig.Rybderg_band} can be obtained in the off-resonance regime,  $|\delta|\gg \Gamma,\Gamma_0,\Omega$, where  the $|e \rangle$ degree of freedom can be adiabatically eliminated. In this way  we get  the following effective Hamiltonian:

%\textcolor{red}{The text above doesn't seem ideal, because it focuses so much on EIT, an effect that we don't really explore at all. After all of the careful explanation on EIT, we then just state that there is some stark shifted resonance, without any justification or connection to Fig. 6b~(how can one see that there is a narrow resonance?)}\textcolor{blue}{That s true. We could keep Fig.6 (b) and remove Eq.28 and the expression for effective mass and group velocity. Then we could mention EIT to just cite previous work and use Fig.6 (b) to introduce the Stark Shifted resonance (now highlighted in the new figure) where the dispersion looks like the chiral one and it is still Rydberg dressed.}

Here, in order to distinguish from previous work and to also explore connections with SIT, we will consider the nonlinear effects that arise around the narrow, effective two-level (Stark shifted) resonance, $E_{\rm st}\sim -\delta-\frac{\Omega^2}{\delta}$, indicated by the red dashed line of Fig.~\ref{Fig.Rtran}(b). Around this resonance, the excited state $|e\rangle$ can be effectively eliminated, and $|s\rangle$ can be considered the new, effective ``excited state'' of a two-level atom, with renormalized properties. This is evidenced by the similarity between the single-excitation dispersion relation in the vicinity of $E_{\rm st}$~(Fig.~\ref{Fig.Rtran}(b)), and that of Fig.~\ref{fig:setup_dis1ex}(b) for a chain of TLA coupled to a chiral waveguide. This intuition can be made more concrete by formally eliminating the far-detuned excited state, obtaining the effective Hamiltonian
\begin{equation}\label{heffr_off}
\begin{split}
\hat H_{\rm eff}=&\left(E_{\rm st}-i\frac{\bar\Gamma_0+\bar\Gamma}{2}\right)\sum_n\hat\sigma^n_{sg}\hat\sigma^n_{gs}-i\bar\Gamma\sum_{n>m}\hat\sigma^n_{sg}\hat\sigma^m_{gs}\\
&+\sum_{n<m}V(|x_n-x_m|)\hat\sigma^n_{sg}\hat\sigma^n_{gs}\hat\sigma^m_{sg}\hat\sigma^m_{gs}.
\end{split}
\end{equation}
This is indeed equivalent to the one given in Eq.~(\ref{Heffchi}) with renormalized emission rates $\bar\Gamma=\Gamma\Omega^2/\delta^2$ and $\bar\Gamma_0=\Gamma_0\Omega^2/\delta^2$.
The main differences with respect to Eq.~(\ref{Heffchi}) are the additional spin-spin Rydberg interaction term, and the~(large) additional independent emission $\bar\Gamma_0$. 
%In the single excitation sector~(where no Rydberg interactions can take place), the dispersion relation $J_k$ is thus identical in form to that obtained in Eq.~\eqref{dis_chi}, with the addition of a~(large) imaginary component $\sim\Gamma_0$ owing to the additional independent emission now added. 

For more than one excitation, the Rydberg interaction allows for multiple bound state modes. This is illustrated in Fig.~\ref{Fig.Rtran}(c), where we plot the real part of the energy $E_K$ vs $K$ around the Stark shifted resonance $E_{\rm st}$.
%The interaction strength $C_6$ affects number and slope of the bound states and it also rules their population distribution  (see App.~\ref{App.A}) characterized by  the Rydberg radius, $r_b=(C_6/\bar \Gamma)^{1/6}$, that we define as the distance at which the resonance frequency shifts by more than the effective linewidth, i.e. $V(r_b)=\bar{\Gamma}$.
A more in-depth discussion of these Rydberg bound states is provided in  App.~\ref{App.A} and App.~\ref{App.manybodyBSRyd}. However, the large additional dissipation $\Gamma_0$ acting on the bound states would make their observation challenging in free-space ensembles~(although perhaps Rydberg-like interactions could be emulated within waveguide QED~\cite{Efioptica,PRXcaneva} where the ratio of $\Gamma/\Gamma_0$ can be extremely high). Instead, in the following we will investigate the many-body limit, where it is known that SIT is robust in normal TLA ensembles with large dissipation, and examine how SIT changes in the presence of the long-range Rydberg interactions. 

%The  spectrum obtained by the effective model, expressed in unit of $\bar\Gamma$, is shown in Fig.~\ref{Fig.pulse_stark}(a) and it clearly resembles the one obtained by the full model in the dispersive regime (plotted in grey color) and its continuum edges coincides with the ones derived in Sec.~\ref{Sec_chiral2ex} for a chiral array (black tick lines in Fig.~\ref{Fig.pulse_stark}(a)).
%Compared to the non interacting chiral array the Rydberg interaction term induces the formation of  multiple bound states whose slope and number is ruled by the interaction strength $C_6$. The interaction strength also affects the population distribution that experiences a progressive growth of the localization distance between the two excitations (see App.\ref{App.RydEIT_loc}). This localization distance is approximately given by the Rydberg radius, $r_b=(C_6/\bar \Gamma)^{1/6}$, that we define, at the Stark shift resonance, as the distance at which the resonance frequency shifts by more than the effective linewidth, i.e. $V(r_b)=\bar{\Gamma}$. Note that this definition of the Rydberg radius  differs from the one  used in effective theories, (that we previously introduced as $R_b$) which is strictly defined at the EIT resonance.



%Similarly as discussed  in Sec.~\ref{Sec_TLA_many}, the spin bound  excitations can manifest themselves in the output field and  photon correlations when externally excited by a localized pulse. In order to study their dynamics, we operate around the Stark shifted EIT resonance $E_{\rm st}$ making use of the input output relations~\eqref{in_outR}) and master equation~\eqref{eq:MasterEq} adapted for the effective model \eqref{heffr_off}. As  we are going to discuss, these states can be efficiently generated only in the regime of low external dissipation, $\Gamma_0\ll \Gamma$, unrealistic in free space atomic ensembles.  However, for large photon numbers, the enhanced Rabi frequency can overcome the external loss making the system dynamics more robust against it.  In this regime, we find that  the pulse propagation  progressively approaches a solitonic behavior, a regime recently addressed as Rydberg SIT \cite{Rydberg-SIT}.





%Following the reasoning of Sec.~\ref{Sec_TLA_many}, we  expect that the Rydberg bound states  have a similar dynamical behaviour as the waveguide bound states exhibiting a photon-number dependence on the group velocity and a solitonic beahviour in the many-body limit. Both effects have  been hypothesized  in~\cite{bienas,first,LiangBS} but never explicitly  shown as far we know.
%As before, in order to extract the atomic correlations in the output field, we make use of the input output relation given in Eq.~\eqref{in_outR} and 
%we study the full time evolution of the driven ensemble of Rydberg atoms, which evolves following the full master equation given in~\eqref{eq:MasterEq}. In particular, we assume to keep on the control field $\Omega$ and to send a localized pulse trough  the system   as input field.  This choice, as discussed in Sec.~\ref{Sec_TLA_many},  allows to efficiently excite the bound states and eventually to resolve them exploiting the distinct dispersion of the bound excitations compared to the unbound ones. 

%As discussed in  Sec.~\ref{Sec_ryd_2ex}, a convenient regime where multiple bound states can be simultaneously excited is the one occurring  around the Stark shifted resonance $E_{\rm st}$. To simulate this regime for large arrays we use the effective model



%\subsubsection{Exciting multiple bound states}\label{Sec.exRydbs}

%\begin{figure}
%\includegraphics[width=0.48\textwidth]{pulse_Ryd.pdf}%
%\caption{(a) Eigenvalues spectrum of the effective spin model \eqref{heffr_off}  as function of the center of mass momentum $K$  for $C_6/(d)^6=600\bar\Gamma$. In grey we superimposed the equivalent spectrum (continuum and bound states) obtained by solving the full model   with $\delta=-10\Gamma$ and $\Omega=\Gamma$. The tick black lines indicate the edge of the continuum obtained by combining he single excitation dispersion. b) Output intensity  $I$ and two photon correlation functions $G^{(2)}$ for the propagation of a coherent pulse trough an array of $40$ atoms with $\sigma\bar\Gamma=2$ and $n_{ph}=|\mathcal{E}|^2/\bar\Gamma=1.0$
%The input pulse is centered at $\bar\Gamma t_0=10$ (indicated by the grey vertical line) while the expected bound state time delays $\tau_l$ are indicated by vertical dashed lines. The continuous curve refer to  the not dissipative case $\Gamma_0=0$ while the grey dashed line is  the output intensity in the dissipative case with $\Gamma_0/ \Gamma\sim 0.1$.
%For the third bound state we estimated an area (blue shadow region) where to expect to detect the bunching.
%The simulation has been performed with an MPS based quantum trajectories algorithm involving $N_t=5000$ trajectories and maximum bond dimension $D_{\rm max}=40$.}
%\label{Fig.pulse_stark}
%\end{figure}

%The bound states shown in Fig.~\ref{Fig.pulse_stark}(a) can be excited by sending a coherent Gaussian input pulse at the frequency $E_{\rm st}$ trough the atomic array. In the ideal lossless regime, $\Gamma_0=0$, the intensity  $I_R$ and the equal time two-photon correlation function $G^{(2)}(t)$ exhibit travelling isolated peaks that we associate to the different  bound states of Fig.~\ref{Fig.pulse_stark}(a) trough their acquired time delay $\tau_l=Nd/v^l_g(\omega_{in})$, where $l$ is the label of the bound states (circled numbers in Fig.~\ref{Fig.pulse_stark}).
%In order to resolve the  multiple bound states in the output field, we considered in Fig.~\ref{Fig.pulse_stark}(b)  a large array of $N=40$ atoms. This requirement relies on asking that the delay between two different peaks should be bigger than the sum of their widths, a  condition that can be  recast in the form $Nd\gg r_b(v_g^l+v_g^m)/|v_g^l-v_g^m|$. 
 
%Considering  large arrays becomes particularly limiting in presence of external dissipation  where an already  moderate value of  $\Gamma_0/ \Gamma\sim 0.1$ can damp and smear out the output field, as shown by the grey dashed line of Fig.~\ref{Fig.pulse_stark}(b). The effect of external dissipation on the bound states propagation decreases by going toward the many-body regime where the time scale becomes ruled by the stimulated emission decay rate, $\sim \Gamma n$ that can overcome the free space emission for large input pulses.
%Before to proceed toward the many-body regime   we mention that the  few-excitation physics could be effectively capture in circuit QED implementing direct Rydberg-like interactions among arrays of artificial atoms coupled to a transmission line.  Along this direction, it is intriguing to envision the possibility to generate different type of correlated light by engineering direct qubit-qubit interactions.
 
  
  
  %On the other hand,  as discussed in Sec.~\ref{Sec.SIT} and previously studied in literature ~\cite{Pohl_reit_soliton}, we could expect a soliton-like propagation to emerge in the mean field, which could make some of these effects more robust. 

 
 

 %In section~\ref{Sec.exRydbs} we set the external dissipation $\Gamma_0$ to zero but in realistic implementations it can be either not negligible, as in waveguide QED systems, or even dominant, as in free space atomic ensembles.
%To take into account of this issue we plotted in Fig.~\ref{Fig.tow_mf}(a)  the output intensity for different (moderate) values of $\Gamma_0$ and we compare it to the lossless case previously discussed. We observe that, up to values of $\bar \Gamma_0/\bar \Gamma\sim 0.1$, the bound state peaks should be observable in the output intensity. This value of loss  is clearly unrealistic 
 %in current experiments with free space atomic ensembles, where  the observation of these correlated pulse separation at the level of few excitations it will probably be extremely challenging. 
 
 
 
 
 
%A criteria that provides the requirements needed to resolve these multiple bound states in the output field can be established by asking that the difference in the delay between two different bound states should be bigger than the sum of their widths. More concretely we ask that $|\tau_{l}-\tau_{m}|/(\Delta t_l+\Delta t_m)\gg 1$ with $\Delta t_l\sim r_b/v_g^l$ being the width of the $l-th$ bound state. This condition can be recast in the form
 %\begin{equation}\label{Bs_resolve}
% \frac{Nd|v_g^l-v_g^m|}{r_b(v_g^l+v_g^m)}\gg 1
 %\end{equation}
%that for our case required tens of atoms as previously considered.

 % In light of this estimation, we expect that by considering even larger ensemble of atoms it should be possible to resolve the other remaining bound states of the spectrum which own even flatter dispersions.


 %The expected time delays are indicated in  Fig.~\ref{Fig.pulse_stark}(b)  by vertical dashed lines (blue shaded  area for the third bound state) and are computed by estimating the group velocities of the bound states.
  %In particular, by fitting the bound states dispersion around the $E_{\rm st}$ resonance, we estimated $v^1_g(E_{\rm st})\approx  2.8\bar \Gamma d$ and $v^2_g(E_{\rm st})\approx 7.5 \bar \Gamma d$ for the first and second bound states velocity. The third bound state instead experiences a larger variation of the slope within the frequency range $\sigma_\omega=1/(2\sigma)$, given by the pulse width, with an associated variation of the group velocity within the values $\bar \Gamma d\lesssim  v^3_g(E_{\rm st}) \lesssim 2\bar \Gamma d$. This indeterminacy it is also visible in the smearing of the $G^{(2)}$ peak within  the estimated delay time region. 
  
  


%Note that the possibility of exciting multiple bound states with the same amount  of excitations is a peculiarity of this setup compare to the simple waveguide  scenario and it allows to generate highly correlated states of light. By looking at higher order correlations, e.g. $G^{(3)}$,  we observe also a semblance of a photon number dependent velocity for the bound states even if they are not completely resolved because probably composed by the contribution of multiple bound states involving different excitation subspaces.


\subsection{Many-body dynamics}\label{Sec_many_rydberg}

%\subsubsection{Many-body dynamics}

\begin{figure}[h]
\includegraphics[width=0.48\textwidth]{tow_mf.pdf}%
\caption{(a) Sketch of the effective array of superatoms arising in the Rydberg medium. (b)-(c) Output intensity (b) and  atomic population of an atom in the middle of the ensemble (c) as a function of time, for pulses of different average photon number. The form of the input field is given in Eq.~\eqref{ansatz_Rydberg},  with the soliton parameter $\alpha$ determined by a variational approach described in the main text. The input pulse peak is centered at $\bar\Gamma t_0=0.05$. In both plots we have added an external spontaneous emission rate of $\bar\Gamma_0=5\bar\Gamma$ and we set $C_6/d^6=10^5\bar\Gamma$. The continuous lines represents the results obtained numerically while the black dashed lines represent instead the delayed solitonic ansatz. In the inset of (b) we plot the overlap between the numerical obtained output and the solitonic ansatz as a function of  number of photons in the pulse.  (d)-(e) Estimated number of blockade atoms (d)  and  group velocity of the pulse (e) as a function of photon number for different Rydberg interaction strengths. The points are inferred from numerics, while the continuous lines in (d) and~(e) are obtained from our effective model described in the main text. In (e) the black line represents the standard SIT prediction for two-level atoms without Rydberg interactions. 
In all plots we take an array of $N=30$ atoms and we performed an MPS simulation of the full master equation by setting $D_{\rm max}=70$. 
In all plots regarding the soliton characterization, i.e. the inset of (b), and subfigures (d) and (e), we set $\bar\Gamma_0=0$ consistent with the standard SIT analysis.}
\label{Fig.tow_mf}
\end{figure}

%Finding an analytic solution of the mean field equations~\eqref{eqMF} generalized to Rydberg interactions is more involved than in standard SIT due to the interactions. An approximate solution for these equations has been derived in Ref.~\cite{Rydberg-SIT} using a local field approximation that neglects two-body correlations. However, such a mean-field approach fundamentally breaks down in the case where $V$ is sufficiently large that a blockade radius emerges, a correlation effect in which two $|s\rangle$ excitations cannot be excited within a distance $r_b$ of one another due to the prohibitive energy cost. Here, we carry out an MPS simulation with bond dimension $D\gg 1$ to capture the full correlated dynamics, and we show that key aspects of the solitons that appear can be well-described by an SIT theory with renormalized parameters.
%
%
%
%
%
%Numerically, we drive the system with the input field
%\begin{equation}\label{ansatz_Rydberg}
%\mathcal{E}_{\rm in}(t)=\frac{n_{\rm ph}\alpha\sqrt{\bar\Gamma}}{2}\operatorname{sech}\left(\frac{\bar\Gamma}{2}\alpha^2n_{\rm ph}(t-t_0)\right),
%\end{equation}
%which represents a generalization of the SIT solution~(recovered for $\alpha=1$) and our ansatz of what a soliton in the presence of Rydberg interactions might look like. Note that when acting on a single atom, this pulse violates the usual area law for the integrated Rabi frequency:
%\begin{equation}
%2\sqrt{\bar \Gamma}\int dt \mathcal{E}_{\rm in}(t)=\frac{2\pi}{\alpha}.\label{eq:SITarealaw}
%\end{equation}
%The ``best fit'' parameter $\alpha$ will in general be a function of the number of photons and the Rydberg interaction, i.e. $\alpha:=\alpha(n_{\rm ph},C_6)$. We determine the optimal value $\alpha_{\textrm{opt}}$  with a variational approach   that maximizes the overlap  $O=\int dt \,{\rm min}\{I_R(t),I_{\rm sol}(t)\}/\sqrt{\int dt I_R(t)\cdot\int dt I_{\rm sol}(t)}$ between the numeric and the expected output field intensity, $I_R(t)$ and  $I_{\rm sol}(t)=|\mathcal{E}_{\rm in}(t-\tau)|^2$ respectively, where the pulse delay $\tau=N/v_g$ and corresponding group velocity are numerically determined based on the position of the output intensity peak.

%The inset of Fig.~\ref{Fig.tow_mf}(a) shows, for a chain of $N=30$ atoms, a progressively improving agreement between the output intensity and the expected soliton solution, when the number of photons is increased, with the optimal overlap reaching $O(\alpha_{\rm opt})\sim 99\%$ for $n_{\rm ph}=200$.
%For the determined optimal values  $\alpha_{\textrm{opt}}$, we plot in Fig.~\ref{Fig.tow_mf}(a)  the corresponding output intensity at different pulse strengths. We also compare the numerically obtained output intensity with the expected damped solitonic solution $I_{\rm sol}(t)e^{-\bar \Gamma_0 \tau}$~(black dashed line). This approximately accounts for the external dissipation $\bar{\Gamma}_0=5\bar{\Gamma}$ introduced in the numerical simulations, and the two sets of curves are observed to agree well with one another. 
%
%
%
%The difference compared to usual SIT is made evident in Fig.~\ref{Fig.tow_mf}(b), where we plot the time-dependent atomic excited population, $p_s(t)$, of an atom in the middle of the chain. We observe that, even in regimes of strong pulses, the excited population does not go to unity and back to zero as expected of a $2\pi$ pulse, confirming the violation of the area law. 
%
%
%
%We now present an effective theory for the observed phenomena. The procedure consists in postulating a renormalized SIT behavior for the system and in comparing  the results obtained by this effective model with the full numerics. 
%We begin by noting that, if $V$ is strong enough and covers a sufficiently large range, some portion of the atomic medium can be blockaded. In particular, given a Rydberg excitation at the origin and the peak Rabi frequency $\Omega_r=n_{\rm ph}\alpha_{\textrm{opt}}\bar\Gamma$ associated with the pulse \eqref{ansatz_Rydberg}, the condition $V(r_b)=\Omega_r$ sets the distance at which the pulse can overcome the dispersive Rydberg energy shift and efficiently drive a second Rydberg excitation. The resulting ``blockade radius'' is then  $r_b=\left(C_6/\Omega_r\right)^{1/6}$~\cite{Lukin_Ryd}.
%
%The blockade region consisting of $N_b=2r_b$ atoms acts like an effective, single two-level ``super-atom''~\cite{Hofferberth,Hofferberth2}, within which only a single collective Rydberg excitation can be generated. Furthermore, this excitation experiences a collectively enhanced emission rate $\Gamma_{\rm s}=N_b\bar{\Gamma}$ into the probe mode, while the dissipation $\bar{\Gamma}_0$ into other modes remains fixed. This super-atom picture suggests that a generalized version of SIT should emerge, where the parameter in Eq.~\eqref{ansatz_Rydberg} is set to $\alpha_{\textrm{opt}}=\sqrt{N_b}$ to account for the enhanced coupling. This choice of $\alpha$ combined with  Eq.~(\ref{eq:SITarealaw}) implies that each super-atom experiences a full $2\pi$-pulse.
%The whole Rydberg medium can then be described as an array of $N/N_b$ super-atoms, pictorially shown in Fig.~\ref{Fig.tow_mf}(c), and the standard SIT time delay in Eq.~\eqref{delaySIT} is modified to become $\tau=\frac{N}{N_b}\frac{4}{\bar\Gamma n^2_{\rm ph}N_b}$. This equation can be recast in terms of a more experimentally relevant quantity, the optical depth per blockade radius $D_b=2N_b\bar{\Gamma}/\bar{\Gamma}_0$~\cite{MPSJames,james_ryd}, such that
%\begin{equation}\label{delayRydbergSIT}
% \bar\Gamma_0\tau=\frac{N}{N_b}\frac{8}{D_b n^2_{\rm ph}}.
%\end{equation}
%Note that $D_b$, $N/N_b$, the photon number in the pulse, and the free-space spontaneous emission rate $\bar{\Gamma}_0$ of the effective two-level transition all have well-defined meaning in an experimental setup of a Rydberg ensemble, which thus provides a connection between our microscopic spin model and a physical system.
%
%In order to test the accuracy of this effective description, in Fig.~\ref{Fig.tow_mf}(d) we plot in solid lines the expected number of atoms within a blockade radius $N_b$ versus photon number $n_{\rm ph}$ assuming our effective theory is correct, by solving the equation $r_b=N_b/2=\left(C_6/n_{\rm ph}\bar{\Gamma}\sqrt{N_b}\right)^{1/6}$. This is plotted for several different values of $C_6$, as indicated by different colors. On the other hand, from the numerically determined value for $\alpha_{\textrm{opt}}$, we can use $N_b=\alpha_{\textrm{opt}}^2$ to infer $N_b$~(points). It is seen that the two approaches agree well with each other. Likewise, in Fig.~\ref{Fig.tow_mf}(e), we plot with points the group velocity, as given from the numerically determined delay by $v_g=N/\tau$. We then plot the same quantity $v_g=N/\tau$ in solid lines, where $\tau$ is given by the effective description~(\ref{delayRydbergSIT}), and where the expected $N_b$ is taken from the solid lines of Fig.~\ref{Fig.tow_mf}(d). Again, good agreement is seen. Finally, for comparison, we plot the the expected group velocity for normal SIT~(without Rydberg interactions) in black, which is seen to be slower than in the blockaded case. These results shown that an effective description based on an array of Rydberg superatoms is able to capture the main emerging solitonic features of the propagating pulse.







Finding an analytic solution of the mean field equations~\eqref{eqMF} generalized to Rydberg interactions is more involved than in standard SIT due to the interactions. An approximate solution for these equations has been derived in Ref.~\cite{Rydberg-SIT} using a local field approximation that neglects two-body correlations. However, such a mean-field approach fundamentally breaks down in the case where $V$ is sufficiently large that a blockade radius emerges, a correlation effect in which two $|s\rangle$ excitations cannot be excited within a distance $r_b$ of one another due to the prohibitive energy cost. In particular, given a Rydberg excitation in the middle of the ensemble and a peak Rabi frequency $\Omega_r$, associated with the incoming pulse, the condition $V(r_b)=\Omega_r$ sets the distance at which the pulse can overcome the dispersive Rydberg energy shift and efficiently drive a second Rydberg excitation. The resulting ``blockade radius'' is then  $r_b=\left(C_6/\Omega_r\right)^{1/6}$~\cite{Lukin_Ryd}.
The blockade region consisting of $N_b=2r_b$ atoms acts like an effective, single two-level ``super-atom''~\cite{Hofferberth,Hofferberth2}, within which only a single collective Rydberg excitation can be generated. Furthermore, this excitation experiences a collectively enhanced emission rate $\Gamma_{\rm s}=N_b\bar{\Gamma}$ into the probe mode, while the dissipation $\bar{\Gamma}_0$ into other modes remains fixed.  For an extended ensemble the whole Rydberg medium can then be expected to consist of an array of $N/N_b$ effective super-atoms, as pictorially shown in Fig.~\ref{Fig.tow_mf}(a).
If this picture is correct, we might expect that a generalized version of SIT should emerge by simply re-scaling the decay rate in the  SIT solution of Eq.\eqref{eqSIT_solution}, %In this way the area of the pulse $2\pi$, but $2\pi/\sqrt{N_b}$ 
where the standard SIT time delay in Eq.~\eqref{delaySIT} is modified to become $\tau=\frac{N}{N_b}\frac{4}{\bar\Gamma n^2_{\rm ph}N_b}$. This equation can be recast in terms of a more experimentally relevant quantity, the optical depth per blockade radius $D_b=2N_b\bar{\Gamma}/\bar{\Gamma}_0$~\cite{MPSJames,james_ryd}, such that
\begin{equation}\label{delayRydbergSIT}
 \bar\Gamma_0\tau=\frac{N}{N_b}\frac{8}{D_b n^2_{\rm ph}}.
\end{equation}
Note that $D_b$, $N/N_b$, the photon number in the pulse, and the free-space spontaneous emission rate $\bar{\Gamma}_0$ of the effective two-level transition all have well-defined meaning in an experimental setup of a Rydberg ensemble, which thus provides a connection between our microscopic spin model and a physical system.

In spite of the simplicity and semi-classical appearance of this guess, the blockade generates entanglement, and verifying this behavior requires a true many-body calculation, which we carry out using an MPS simulation with bond dimension $D\gg 1$ to capture the full correlated dynamics. The main steps of this procedure are presented in the following.



First we observe that, if the system supports a soliton-like solution, there should exist some input field, $\mathcal{E}_{\rm in}(t)$ (to be determined), which would result in an undistorted output, $I_{\rm sol}(t)=|\mathcal{E}_{\rm in}(t-\tau)|^2$, simply delayed by a time $\tau$ related to the group velocity of the pulse, $\tau=N/v_g$. As is it not feasible to numerically explore the infinite space of all input functions, here, we will restrict ourselves to a variational class, characterized by the parameter $\alpha$, which represents a generalization of the SIT solution
 \begin{equation}\label{ansatz_Rydberg}
\mathcal{E}_{\rm in}(t)=\frac{n_{\rm ph}\alpha\sqrt{\bar\Gamma}}{2}\operatorname{sech}\left(\frac{\bar\Gamma}{2}\alpha^2n_{\rm ph}(t-t_0)\right).
\end{equation}
This ansatz  recovers for $\alpha=1$ the usual SIT solution of Eq. \eqref{eqSIT_solution}, while for $\alpha=\sqrt{N_b}$ corresponds to our super-atom based hypothesis presented above.
Note that when acting on a single atom, this pulse violates the usual area law for the integrated Rabi frequency:
\begin{equation}
2\sqrt{\bar \Gamma}\int dt \mathcal{E}_{\rm in}(t)=\frac{2\pi}{\alpha}.\label{eq:SITarealaw}
\end{equation}
The ``best fit'' parameter $\alpha$ will in general be a function of the number of photons and the Rydberg interaction, i.e. $\alpha:=\alpha(n_{\rm ph},C_6)$. We determine the optimal value $\alpha_{\textrm{opt}}$  with a variational approach   that maximizes the overlap  $O=\int dt \,{\rm min}\{I_R(t),I_{\rm sol}(t)\}/\sqrt{\int dt I_R(t)\cdot\int dt I_{\rm sol}(t)}$ between the numeric and the expected output field intensity, $I_R(t)$ and  $I_{\rm sol}(t)$ respectively, where the pulse delay $\tau$ is numerically determined based on the position of the output intensity peak.
The inset of Fig.~\ref{Fig.tow_mf}(b) shows, for a chain of $N=30$ atoms, a progressively improving agreement between the output intensity and the expected soliton solution, when the number of photons is increased, with the optimal overlap reaching $O(\alpha_{\rm opt})\sim 99\%$ for $n_{\rm ph}=200$.
For the determined optimal values  $\alpha_{\textrm{opt}}$, we plot in Fig.~\ref{Fig.tow_mf}(b)  the corresponding output intensity at different pulse strengths. We also compare the numerically obtained output intensity with the expected damped solitonic solution $I_{\rm sol}(t)e^{-\bar \Gamma_0 \tau}$~(black dashed line). This approximately accounts for the external dissipation $\bar{\Gamma}_0=5\bar{\Gamma}$ introduced in the numerical simulations, and the two sets of curves are observed to agree well with one another. 
The difference compared to usual SIT is made evident in Fig.~\ref{Fig.tow_mf}(c), where we plot the time-dependent atomic excited population, $p_s(t)$, of an atom in the middle of the chain. We observe that, even in regimes of strong pulses, the excited population does not go to unity and back to zero as expected of a $2\pi$ pulse and discussed in Sec.\ref{Sec_TLA_many}, confirming the violation of the area law. 

In order to test the accuracy of the effective description based on the super-atom array, we compare its predictions with the numerical results describing the full correlated dynamics.
For a peak Rabi frequency $\Omega_r=n_{\rm ph}\alpha_{\textrm{opt}}\bar\Gamma$ associated with the pulse \eqref{ansatz_Rydberg}, the effective theory predicts an expected number of atoms within a blockade radius given by the solution of $r_b=N_b/2=\left(C_6/n_{\rm ph}\bar{\Gamma}\sqrt{N_b}\right)^{1/6}$. This is plotted  in Fig.~\ref{Fig.tow_mf}(d) (solid lines) as a function of the photon number $n_{\rm ph}$ for several different values of $C_6$, as indicated by different colors. The numerical estimate for $N_b$ can instead be inferred by 
the variationally determined value for $\alpha_{\textrm{opt}}$, using $N_b=\alpha_{\textrm{opt}}^2$, and is plotted as dots in  Fig.~\ref{Fig.tow_mf}(d).
 It is seen that the two approaches agree well with each other. Likewise, in Fig.~\ref{Fig.tow_mf}(e), we plot with points the group velocity, as given from the numerically determined delay by $v_g=N/\tau$. We then plot the same quantity $v_g=N/\tau$ in solid lines, where $\tau$ is given by the effective description~(\ref{delayRydbergSIT}), and where the expected $N_b$ is taken from the solid lines of Fig.~\ref{Fig.tow_mf}(d). Again, good agreement is seen. Finally, for comparison, we plot the expected group velocity for normal SIT~(without Rydberg interactions) in black, which is seen to be slower than in the blockaded case. These results shown that an effective description based on an array of Rydberg superatoms is able to capture the main emerging solitonic features of the propagating pulse.




We believe that the discrepancies between the numerical results and the effective description in Fig.~\ref{Fig.tow_mf}(d)-(e) can be mainly attributed to three sources of errors. The first is the intrinsic discreteness of our numerical model. For example, the moderate number of atoms $N_b\lesssim 10$ within a blockade region in our simulations suggests that moving to a continuum description should not be entirely accurate. A second error comes from the numerical imperfections in the evaluation of the optimal value $\alpha_{\textrm{opt}}$. Specifically, due to the large simulation complexity, the optimal value was obtained using a limited number of sampling points ($\sim 20$) for $\alpha$.  An additional~(non-numerical) discrepancy is that, similar to the case of SIT with normal TLA, the approach from quantum many-photon bound states to semi-classical SIT is a gradual one as a function of increasing photon number. For the lower range of photon numbers used in our simulations, it is possible that additional quantum many-body features are present, which would be interesting to pinpoint and explore further in future work.


We now consider the experimental feasibility of observing this many-body dynamics in a Rydberg ensemble. State-of-the-art experiments~\cite{LiangBS,Hofferberth,Hofferberth2} allow for a large optical depth per blockade radius, $D_b\gtrsim 1$, and multiple blockade regions in the ensemble, $N/N_b\sim 10$. For moderate photon numbers $n_{\rm ph}\sim 10^2$ in the pulse, it is then possible to acquire the time delay before full absorption takes place, $\bar\Gamma_0\tau\ll 1$. Separately, from Eqs.~(\ref{ansatz_Rydberg}) and~(\ref{delayRydbergSIT}), the ratio of the delay to the temporal width of the input pulse is given by $\tau/t_{\rm in}\sim (2/n_{\rm ph})(N/N_b)$. For parameters like above, this allows for the delay to be a reasonable, detectable fraction of the input pulse width.
Finally, as pointed out in Ref.~\cite{Rydberg-SIT}, this Rydberg SIT dynamics could also be probed without making use of an intermediate $|e\rangle$ state, by directly driving a ground-Rydberg state transition.

 
 
 
 







\section{Conclusions}\label{sec:conclusions}

In summary, we have presented a unified method to treat photon bound states within an atomic nonlinear medium, based upon a spin model formulation. This description  allows one to promptly identify the emerging correlated states in different  scenarios, such as an array of quantum emitters coupled to an optical waveguide and an ensemble of Rydberg atoms. In both cases, the formalism allows one to obtain the two-excitation bound state dispersion relation exploiting a convenient description in the relative coordinate frame, understand the effect of this dispersion relation on the correlation functions of outgoing fields, given an input pulse, and investigate the transition from few- to many-photon behavior. In the many-body case, we show how SIT or generalizations thereof emerge in all the systems studied. 

While the two-excitation limit and many-excitation, semi-classical SIT limit are possible to treat in semi-analytic fashion, in future work it would be interesting to develop techniques to better understand the intermediate case, where quantum effects in the many-photon pulse might still persist in the output. For example, it might be interesting to see whether the spin model might be more amenable to field theoretical techniques. Alternatively, it might be feasible to use time-independent MPS techniques to target and investigate the multi-excitation bound eigenstates themselves, in order to better understand their nature within the atomic medium.

\emph{Note added}. After the initial submission of this work a related preprint on photon bound states in waveguide QED systems  \cite{SorenwqedBS} reported similar results.



\section*{Acknowledgments}
The authors thank Sahand Mahmoodian, Anders S. S{\o}rensen, Angelo Carollo, Francesco Ciccarello  and Cosimo Rusconi for valuable discussions. G.C. acknowledge that results incorporated in this standard have received funding from the European Union Horizon 2020 research and innovation programme under the Marie Sklodowska-Curie grant agreement No. 882536 for the project QUANLUX.
D.E.C. acknowledges support from the European Union's Horizon 2020 research and innovation program, under FET-Open grant agreement No. 899275 (DAALI) and  European Research Council grant agreement No. 101002107 (NEWSPIN); the Government of Spain (Europa Excelencia program EUR2020-112155, Severo Ochoa program CEX2019-000910-S [MCIN/AEI/10.13039/501100011033], and MCIN Plan Nacional Grant PGC2018-096844-B-I00), Generalitat de Catalunya (CERCA program and AGAUR Project No. 2017-SGR-1334), Fundaci\'o Privada Cellex, Fundaci\'o Mir-Puig, and Secretaria d'Universitats i Recerca del Departament d'Empresa i Coneixement de la Generalitat de Catalunya, co-funded by the European Union Regional Development Fund within the ERDF Operational Program of Catalunya (project QuantumCat, ref. 001-P-001644).

\section*{Data  availability} The data presented in the figures of this manuscript are available on the link 
 DOI http://dx.doi.org/10.5281/zenodo.5771926.

\appendix

\section{Matrix product state simulation}\label{AppMPS}

\begin{figure*}[!t]
\includegraphics[width=0.9\textwidth]{band_EIT.pdf}%
\caption{Spectrum $E_K$ of two-excitation eigenstates for the three-level atomic scheme presented in Sec.~\ref{Sec.modelRyd}, as a function of the center of mass momentum $K$ and for different detunings $\delta$ of the control field.  The continuous yellow line in (a)-(d) indicates the bound state obtained for the three level atom, which differs from the expected TLA bound state shown by the yellow dashed line in (a). In (e)-(f) the red continuous line indicates the bound state that emerges around the effective Stark-shifted TLA transition, as discussed in the main text. Its dispersion indeed approaches the one obtained with the rescaled effective model given in Eq.~\eqref{dis_EIT_BS} (red dashed line in (f)).
The dashed red lines in~(a)-(d) indicate the region where the bound states associated with the effective Stark-shifted TLA transition eventually arise for large $|\delta|$. In all plots we choose $C_6/(d)^6=0$ and $\Omega=\Gamma$.}
\label{Fig.Eit_band}
\end{figure*}


Solving the full system dynamics described by the master equation~\eqref{eq:MasterEq} for large atomic arrays, $N\gtrsim 20$, is extremely challenging with brute force numerical techniques.
To overcome this problem we adopt an MPS representation for the atomic array, which allows us to highly reduce the degrees of freedom needed to efficiently simulate the system. In particular, we make use of a quantum trajectories algorithm where the state of the system evolves under the effective Hamiltonian~\eqref{Heff}, along with stochastic quantum jumps as described in the details of Ref.~\cite{MPSJames}.
The main idea of the MPS representation consists in 
reshaping a generic quantum state $|\phi\rangle=\sum_{i_1,..i_N}\psi_{i_1,i_2,..i_N}|i_1,i_2,..i_N\rangle$ into a matrix product state of the form:
%
\begin{equation}
|\phi\rangle=\sum_{i_1,..i_N}A_{i_1}A_{i_2}...A_{i_N}|i_1,i_2,..i_N\rangle
\end{equation}
%
where, for each specific set of physical indices $\{i_1,i_2,..i_N\}$, the product of the $A_{ i_j}$ matrices gives back the state coefficient $\psi_{ i_1, i_2,.. i_N}$. Each matrix $A_{ i_j}$ has dimension $D_{j-1}\times D_{j}$ and finite-edge boundary conditions are assumed by imposing $D_{1}=1$ and $D_{N}=1$. 
The bond dimension $D_j$ is a crucial parameter because is directly connected to the entanglement entropy of the system. This implies that problems characterized  
by a limited entanglement entropy, as the ones considered in this paper, 
can be efficiently described by MPS ansatz with small bond dimension~\cite{Verstraete,Schollw}.
To compute the time evolution of the system we derive a matrix product operator (MPO) representation for the effective Hamiltonian and jump operators as described in Ref.~\cite{MPSJames}. The evolution of the initial ground state is then 
computed by using a Runge-Kutta method.

One difficulty for the simulation arises in the case of Rydberg atoms because the Rydberg interaction term, $\sum_{n<m}V(|x_n-x_m|)\hat\sigma^n_{sg}\hat\sigma^n_{gs}\hat\sigma^m_{sg}\hat\sigma^m_{gs}$ with  $V(|x_n-x_m|)=C_6/|x_n-x_m|$, does not have an exact MPO representation due to its power law nature. To overcome this issue we approximate the power law potential as a series of exponentials $V(x_r)\approx \sum_n\alpha_n\lambda_n^{x_r}$, as described in Refs.~\cite{Pirvu,Crosswhite,fro} and explicitly developed for atom-waveguide interactions in Ref.~\cite{james_ryd}. It is sufficient to truncate the maximum strength of the Rydberg interaction at the radius where the potential assumes a value an order of magnitude bigger than other characteristic scales, given by $\bar \Gamma$ and $\sqrt{N_b} n_{\rm ph}\bar \Gamma$ for the few and many body regimes respectively.

In addition to the Hamiltonian term, an MPO representation can be derived also for the output field and its associated correlation operators.
Both the time evolution and the computation of the observables at each time step are evaluated by applying an MPO to an MPS. This operation progressively increases the MPS bond dimension 
but, in order to keep the computation efficient, the bond dimension $D_j\leq D_{\rm max}$ can be truncated after each step.

In section~\ref{Sec_many_rydberg}, we employ a slightly different technique, by directly representing the density matrix in MPS form and solving the master equation without making use of the quantum trajectories algorithm, as explained in Ref.~\cite{james_ryd}. This choice, exploited for the plots of Fig.~\ref{Fig.tow_mf}, is well motivated by the relatively low maximum bond dimension required for the density matrix.

%\section{Mean field equations for a TLA  media}\label{App.MF}

%\textcolor{red}{There is a bit of disconnect between the notation used here $a(x,t)$ and in the main text ($E$), which I would try to smooth out.}
%Let us consider the full system Hamiltonian \eqref{H_tot} for a two level atomic medium with Rydberg interactions. This model covers both the standard SIT media described in Sec.~\ref{Sec.TLAarray} and the Rydberg media in the effective off-resonance regime  derived in Sec.\ref{Sec.EffectiveTLA}.  %In this regime the purely atomic Hamiltonian is given by 
%\begin{equation} 
% H_a=\omega_{eg}\sum_n\hat\sigma_{sg}^n\hat\sigma_{gs}^n+\sum_{n<m}V(|x_n-x_m|)\hat\sigma^n_{sg}\hat\sigma^n_{gs}\hat\sigma^m_{sg}\hat\sigma^m_{gs}
 %\end{equation}
% To write down the mean field equations, we first treat the atomic medium as a continuum via the mapping of the spin operators to a continuum density $\hat \sigma^n\rightarrow \hat \sigma(x)/\rho$, where $\rho$ is the linear density of the medium. With this definition, the spin operators fulfill the commutation relation $[\hat \sigma_{-}(x),\hat \sigma_{+}(x')]=-\hat \sigma_{z}(x)\delta(x-x')$, with $\hat \sigma_{-}$, $\hat \sigma_{+}$ and 
 %$\hat \sigma_{z}(x)$ being respectively the raising, lowering and Z spin operators associated to a generic TLA transition (either the $|g\rangle-|e\rangle$ or the $|g\rangle-|s\rangle$ of the main text).
% By using the Heisenberg equations and taking the expectation values (we take out the hat from the operators) we get the mean field equations
%\begin{equation}\label{eqMF}
%\begin{split}
%&\left[\frac{\partial}{\partial t}+c \frac{\partial}{\partial x}\right ]a(x,t)=-i\sqrt{\Gamma c }\sigma_-(x,t)\\
%&\frac{\partial}{\partial t}\sigma_-(x,t)=i\sqrt{\Gamma c }\sigma_z(x,t)a(x,t)-iU(x,t)\sigma_-(x,t)\\
%&\frac{\partial}{\partial t}\sigma_z(x,t)=4\sqrt{\Gamma c }{\rm Im}[a(x,t)\sigma^*_-(x,t)].
%\end{split}
%\end{equation}
%where $U(x,t)=(2/\rho^3)\int dx'V(x-x')\sigma_+(x',t)\sigma_-(x',t)$ and
%  in this discussion we neglected the external dissipation fixing $\Gamma_0=0$.
% For $U(x,t)=0$, the equations reduce to the ones of standard SIT where it exists a  solitonic solution 
% \begin{equation}
%  a(x,t)=\sqrt{\frac{ n^2_{\rm ph}\Gamma}{4c}}\operatorname{sech}\left[\frac{ n_{\rm ph}\Gamma(t-t_0-\tau_{ n_{\rm ph}})}{2}\right],
% \end{equation}
%with $\tau_{n_{\rm ph}}=4Nd/v_g=4Nd/( n^2_{\rm ph}\Gamma)$, whose
% integrated Rabi frequency satisfies the area law giving a  2$\pi$ pulse~\cite{McCall1,McCall2,Bullough,mahmo_calajo}
% \begin{equation}
%2 \sqrt{\Gamma c}\int dx a(x,t)/v_g = 2\pi.
% \end{equation}
% In presence of Rydberg interaction the concept of SIT can be extended to a so called  Rydberg-SIT~\cite{Rydberg-SIT}, where a soliton still emerges in the dynamics with the Rydberg interaction compensating for the violation of  the area law. In this case 
% the solution of equations \eqref{eqMF} becomes more involved and  we make use of the variational approach illustrated in the main text.



\section{Spin model approach for the Rydberg media}\label{App.A}
In this section we use the spin model to derive the single- and two-excitation spectrum for the three-level Rydberg atom scheme presented in Sec.~\ref{Sec.modelRyd}.

\subsection{Single excitation and EIT}
For a single excitation the Rydberg interaction term given in~\eqref{Haryd} does not play a role and it is convenient to rewrite the spin Hamiltonian~\eqref{Heff} 
in wavevector space $k$ where it reads:
\begin{equation}\label{Hrydk}
\hat H_{\rm eff}=\tilde J_k\sum_k\hat\sigma_{eg}^k\hat\sigma_{ge}^k -\delta\sum_k \hat\sigma^k_{se}\hat\sigma^k_{es}+\Omega\sum_k\left(\hat\sigma_{es}^k+\hat\sigma_{se}^k\right),
\end{equation}
where $\tilde J_k=-\frac{\Gamma}{2}\cot{(kd/2)}-i\Gamma_0/2$ coincides, except for the spontaneous emission term $\sim \Gamma_0$, with the dispersion for a chiral atomic array derived in Eq.~\eqref{dis_chi}. 
Hamiltonian \eqref{Hrydk} can be exactly diagonalized  by considering a linear combination of Rydberg and excited state excitations: $|\psi_k\rangle= \alpha_k|e_k\rangle+\beta_k|s_k\rangle$. This leads to the following coupled equations:
\begin{equation}
\begin{split}
&\left(E_k-\tilde J_k\right)\alpha_k=\Omega \beta_k\\
&\left( \delta+E_k\right)\beta_k=\Omega \alpha_k.
\end{split}
\end{equation}
These equations provide the  full dispersion divided in two contributions, which  we name upper~(U) and lower~(L):
\begin{equation}\label{disEITex}
E^{(U/L)}_k=\frac{\tilde J_k-\delta}{2}\pm\frac{1}{2}\sqrt{(\tilde J_k-\delta)^2+4(\Omega^2+\delta\tilde J_k)},
\end{equation}
associated to the  states: 
\begin{equation}\label{polariton}
|\psi^{(U/L)}_k\rangle=\Psi^{\dagger(U/L)}_k|0\rangle=\frac{(\delta+E^{(U/L)}_k)\hat \sigma^k_{ge} +\Omega \hat \sigma^k_{gr}}{\sqrt{|\delta+E^{(U/L)}_k|^2+\Omega^2}}|0\rangle.
\end{equation}
The combination of these two contributions  gives the dispersion relation  shown in Fig.~\ref{Fig.Rtran}(b), which  is characterized by three polariton branches, as usually obtained in literature \cite{lukinEIT,bienas} and by the occurrence of an EIT transmission band  centered at the EIT resonance $E^{\rm EIT}=-\delta$~ (dashed horizontal line). Note that similarly as discussed for the waveguide case, the polariton branches diverge at $k=0$, due to the Markov approximation, instead of following the light line with slope $c$. 
Around the EIT resonance the dispersion~\eqref{disEITex}  reduces to 
 \begin{equation}\label{disEIT}
E_k\simeq-\delta+\frac{\Omega^2}{\Gamma}kd+\frac{\Omega^2}{\Gamma^2}(\delta-i\Gamma_0/2) (kd)^2+...,
\end{equation}
and the state~\eqref{polariton} becomes mainly a Rydberg-like excitation 
with $\alpha_k\sim 0$.
The modes occurring around this resonance do not suffer from spontaneous emission into free space~($\Gamma_0$) due to the negligible excited state population, and these ``dark state'' polaritons allow the medium to be transparent to light in this frequency range~\cite{lukinEIT}. These dark state polaritons propagate in the medium with EIT group velocity $v_g=\partial E_k/\partial k|_{k=0}=\Omega^2d/\Gamma$ and effective mass $m_{\rm eff}=\hbar/(2\partial E^2_k/\partial k^2|_{k=0})=\hbar \Omega^2/(2 v_g^2\delta)$.
Note that  by solving the  full polaritonic model the EIT group velocity is known to be given by $v^{\rm full}_g=\frac{\Omega^2cd}{\Omega^2d+c\Gamma}$~\cite{lukinEIT}. This result reduces to the one  derived  with the spin model in the \qq{slow-light} limit, i.e. $\Omega^2/\Gamma\ll c/d$, which is equivalent to the Markov approximation.

\subsection{Two excitation spectrum}
The two-excitation subspace is spanned by the basis set $\{|e_ne_m\rangle,|s_ns_m\rangle,|e_n s_m\rangle,|e_m s_n\rangle\}$ with $n>m$. Similarly as done for the two-level atom array in Sec.~\ref{Secarray2ex},  we can re-parametrize the  eigenstates in the center of mass, $x_{\rm cm}=(x_m+x_n)/2$, and  relative coordinate, $x_r=|x_n-x_m|$, assuming a plane wave ansatz along $x_{\rm cm}$ of the form
\begin{equation}\label{ans_ryd}
\begin{split}
 &|\psi^{(2)}\rangle =\sum_{x_{\rm cm}}e^{iKx_{\rm cm}}\left(f_1(x_r)\hat\sigma^{(x_{\rm cm}-x_r/2)}_{eg}\hat\sigma^{(x_{\rm cm}+x_r/2)}_{eg}\right.\\
& \left. +f_2(x_r)\hat\sigma^{(x_{\rm cm}-x_r/2)}_{sg}\hat\sigma^{(x_{\rm cm}+x_r/2)}_{sg}\right.\\
&\left.+f_3(x_r)\hat\sigma^{(x_{\rm cm}-x_r/2)}_{eg}\hat\sigma^{(x_{\rm cm}+x_r/2)}_{sg}\right.\\
& \left. f_4(x_r)\hat\sigma^{(x_{\rm cm}-x_r/2)}_{sg}\hat\sigma^{(x_{\rm cm}+x_r/2)}_{eg} \right)|0\rangle,
\end{split}
\end{equation}
where $f_l(x_r)$ are generic functions of the relative coordinate. 
The effective single-excitation problem in the relative coordinate is more complicated than the case of TLA, as the excitation can occupy one of four distinct sectors characterized by the creation operators $\hat S^{\dagger}_{ee}$, $\hat S^{\dagger}_{ss}$, $\hat S^{\dagger}_{se}$ and $\hat S^{\dagger}_{es}$. With this notation the Hamiltonian for the relative coordinate reduces to
\begin{equation}\label{HKeff_ryd}
\begin{split}
&\hat H^K=-\sum_{r>0}\left[ \left(2\delta -V(x_r)\right) \hat S^{\dagger r}_{ss}\hat S^{r}_{ss}+ \delta \hat S^{\dagger r}_{se}\hat S^{r}_{se}+\delta \hat S^{\dagger r}_{es}\hat S^{r}_{es}\right]\\
&-i\frac{\Gamma}{2}\sum_{r,r'>0}\sum_{\epsilon=\pm1}\left[e^{-i\frac{K}{2}|x_r+\epsilon x_{r'}|}-e^{i\frac{K}{2}|x_r+\epsilon x_{r'}|}\right]\hat S^{\dagger r}_{ee}\hat S^{r'}_{ee}\\
&-i\frac{\Gamma}{2}\sum_{r>r'}\left[e^{-i\frac{K}{2}|x_r-x_{r'}|}\hat S^{\dagger r}_{es}\hat S^{r'}_{es}-e^{i\frac{K}{2}|x_r-x_{r'}|}\hat S^{\dagger r}_{se}\hat S^{r'}_{se}\right]\\
&-i\frac{\Gamma}{2}\sum_{r<r'}\left[e^{-i\frac{K}{2}|x_r-x_{r'}|}\hat S^{\dagger r}_{se}\hat S^{r'}_{se}-e^{i\frac{K}{2}|x_r-x_{r'}|}\hat S^{\dagger r}_{es}\hat S^{r'}_{es}\right]\\
&-i\frac{\Gamma}{2}\sum_{r,r'>0}\left[e^{-i\frac{K}{2}|x_r-x_{r'}|}\hat S^{\dagger r}_{es}\hat S^{r'}_{se}-e^{i\frac{K}{2}|x_r-x_{r'}|}\hat S^{\dagger r}_{se}\hat S^{r'}_{es}\right]\\
&+\Omega\sum_{r>0}\left[\hat S^{\dagger r}_{es}\hat S^{r'}_{ee}+\hat S^{\dagger r}_{se}\hat S^{r'}_{ee}+\hat S^{\dagger r}_{es}\hat S^{r'}_{ss}+\hat S^{\dagger r}_{se}\hat S^{r'}_{ss}+\rm H.c. \right],
\end{split}
\end{equation}
which depends parametrically on the center-of-mass momentum $K$. Eq.~\eqref{HKeff_ryd} can be diagonalized numerically.

\subsubsection{Two excitation spectrum for $C_6=0$}\label{App.RydEIT}

In Fig.~\ref{Fig.Rtran}(c) of the main text we presented the two-excitation spectrum of the Rydberg medium spectrum, obtained by diagonalizing the effective two-level model~\eqref{heffr_off}. Here we want to show how this spectrum arises, within the full three-level model, starting from the resonant case, $\delta=0$ and  progressively moving off resonance $\delta\gg \Gamma$. In Fig.~\ref{Fig.Eit_band} we plot the spectrum obtained by the diagonalization of Eq.~\eqref{HKeff_ryd} for the paradigmatic case where there is no Rydberg interaction, $C_6=0$.  In Fig.~\ref{Fig.Eit_band} (a) for $\delta=0$ we observe a bound state in the gap (continuous yellow line) with a dispersion that differs from the one of the TLA case (dashed yellow line) due to the three level nature of the medium. 
In the main text we discussed the off resonance regime where an effective (rescaled) TLA atom description can be used. The transition towards this regime is shown in  Fig.~\ref{Fig.Eit_band}(b)-(d), where the density of states inside the region highlighted by the red dashed lines decreases for increasing $\delta$.
Moving further off-resonance (panels (e)-(f)), we observe that a new gap arises in this region, which is captured by the effective TLA Hamiltonian~\eqref{heffr_off}.
This effective model, in absence of Rydberg interactions, owns a bound state solution (red dashed line in  Fig.~\ref{Fig.Eit_band} (f)) that follows the dispersion:
\begin{equation}\label{dis_EIT_BS}
\omega(K)=2\frac{\Omega^2}{\Delta_p}-2\Gamma\left(\frac{\Omega}{\Delta_p}\right)^2\cot(Kd/2)
\end{equation}
and well approximates the one obtained by the full diagonalization of  Eq.~\eqref{HKeff_ryd} (continuous red line).
 


%Once we start to move off resonance  the original bound state is still present but a new gap starts to arise out of the continuum, as indicate by the red dashed squares in Fig.~\ref{Fig.Eit_band}(a)-(d). 
%This gap is the one where the usual Rydberg bound states, obtained by effective theories~\cite{bienas,Efimov,magrebi}, arise if the Rydberg interaction is included. 
%In this region we observe some spurious eigenvalues \textcolor{red}{should explain why you know they are spurious} that represent the leftovers of the previous existing continuum and explain why the off resonance regime is needed to observe the Rydberg bound state formation.\textcolor{red}{That doesn't seem right. ``Spurious'' to me implies some mathematical, non-physical artefact. While you might have to go off resonance to get rid of them mathematically, that should not have anything to do with whether bound states physically exist or whether they can be observed.} %The origin of these states can be explained by thinking in terms of the effective Schr\"oedinger  equation used in continuum theories\cite{bienas,Efimov,magrebi}.  Moving the detuning off resonance, it is like going from a finite potential, that still owns scattering solutions, to a infinite well potential that hosts only localized bound states.

 
 
 
 \begin{figure}
\includegraphics[width=0.48\textwidth]{reit_band_different_V2.pdf}%
\caption{(a)-(c) Spectrum $E_K$ of two-excitation eigenstates versus center-of-mass momentum $K$ for $\delta=-8\Gamma$, $\Omega=\Gamma$ and different strengths of the Rydberg interaction: $C_6/d^6=0.1\Gamma$ in (a), $C_6/d^6=10\Gamma$ in (b) and $C_6/d^6=1000\Gamma$ in (c). Here, we focus on the region of the spectrum near the effective, Stark-shifted TLA transition. The grey line in (a) indicates the bound state dispersion relation in absence of Rydberg interaction due to the TLA nonlinearity, while the Rydberg interaction induced bound states are given in red. The black lines indicate the boundaries of the continuum states. (d)-(f) Rydberg population distribution as a function of the relative distance index $r=x_r/d$, for the corresponding strengths of the Rydberg interaction shown in (a)-(c). The bound states populations plotted in the panels correspond to specific values of $K$ and are highlighted in (a)-(c) by the black dots.}
\label{Fig.Rybderg_band_differentV}
\end{figure}




 

\subsubsection{Two excitation spectrum for $C_6\neq 0$}\label{App.RydEIT_loc}

We now investigate the spectrum of the two-excitation subspace in presence of Rydberg interactions. Different strengths of the Rydberg interaction affect the bound state dispersion relation as illustrated in Fig.~\ref{Fig.Rybderg_band_differentV}. In particular, in Fig.~\ref{Fig.Rybderg_band_differentV}(a), we see how a weak interaction is 
sufficient to drastically change the bound state dispersion~(red) compared to the one previously obtained for an array of two-level atoms~(grey). Increasing the interaction strength $C_6$, multiple bound states progressively arise with a dispersion that becomes steeper as they move further away from the continuum of states.
%Note that even for weak  potential strength we observe a good agreement  between the spectrum obtained by spin-model and by using a  bosonic description (see App. for more details).
The increasing interaction strength also affects the population distribution in the relative coordinate. In particular, in Figs.~\ref{Fig.Rybderg_band_differentV}(d)-(f) one sees that the distribution $P_{rr}(r)$ of two Rydberg excitations exhibits a progressive growth in the separation between them. This localization distance is approximately given by the Rydberg blockade radius, $r_b=(C_6/\bar \Gamma)^{1/6}$, defined in the main text for the frequency regime close to the Stark-shifted resonance.

\section{Exciting multiple bound states in the Rydberg media}\label{App.manybodyBSRyd}
\begin{figure}
\includegraphics[width=0.48\textwidth]{pulse_Ryd2.pdf}%
\caption{(a) Same eigenvalue spectrum as the one shown in Fig.~\ref{Fig.Rtran}(c) of the main text. (b) Output intensity  $I_R(t)$ and two-photon correlation function $G^{(2)}(t)$, due to a coherent Gaussian input pulse propagating through an array of $N=40$ atoms with $\sigma\bar\Gamma=2$ and $n_{ph}=1.0$. The expected time delays arising from excitation and propagation of bound states, $\tau_l=Nd/v^l_g(\omega_{in})$, are indicated by vertical dashed lines and are computed numerically from the group velocity of the bound state branches $l$ plotted in (a). The grey vertical line instead indicates the initial time of the input pulse set to  $\bar\Gamma t_0=10$. The simulation has been performed with an MPS based quantum trajectories algorithm involving $N_t=5000$ trajectories and maximum bond dimension $D_{\rm max}=40$.  }
\label{Fig.pulse_stark}
\end{figure}

Here, we show how the multiple two-excitation bound states occurring in the Rydberg system can be excited given an input pulse. To simulate the dynamics we make use of the effective TLA model~\eqref{heffr_off} and employ the MPS simulation as previously discussed.
The bound states shown in the spectrum of Fig.~\ref{Fig.Rtran}(c) (repeated in 
Fig.~\ref{Fig.pulse_stark}(a) for convenience) can be excited by sending a coherent Gaussian input pulse at the frequency $E_{\rm st}$ through the atomic array. In the ideal lossless regime, $\Gamma_0=0$, the time-dependent output intensity $I_R(t)$ and the equal time two-photon correlation function $G^{(2)}(t)$ exhibit isolated peaks that can be associated to the different bound states of Fig.~\ref{Fig.pulse_stark}(a) through their acquired time delay $\tau_l=Nd/v^l_g(\omega_{in})$, where $l$ is the label of the bound state branches (circled numbers in Fig.~\ref{Fig.pulse_stark}). In order to resolve the multiple bound states in the output field, we have considered in Fig.~\ref{Fig.pulse_stark}(b) a large array of $N=40$ atoms. This allows that the delay between two different peaks is bigger than the sum of their widths, a condition that can be recast in the form $Nd\gg r_b(v_g^l+v_g^m)/|v_g^l-v_g^m|$. 
 
In a large array, moderate values of external spontaneous emission $\Gamma_0/ \Gamma\sim 0.1$ can already damp and smear out the features in the output field. As discussed in Sec.~\ref{Sec_many_rydberg} in the main text, the effect of external dissipation decreases by going toward the many-body regime, where the stimulated emission rate into the waveguide, $\sim \Gamma n_{\rm ph}$, can overcome the free space emission for large photon number input pulses. On the other hand, this  few-excitation dynamics might be observable in circuit QED platforms where Rydberg-like interactions are implemented in arrays of artificial atoms coupled to a transmission line. It would be interesting to further investigate how different types of correlated light might be realized through the engineering of direct qubit-qubit interactions.
 




\begin{thebibliography}{99}

\bibitem{Agrawal} G. Agrawal,  \textit{Nonlinear Fiber Optics},   5th edn Academic (2012).  

\bibitem{lai1} Y. Lai and H. A. Haus, \textit{Quantum Theory of Solitons in Optical Fibers. I. Time-Dependent Hartree Approximation}, Phys. Rev. A {\bf 40}, 844 (1989).


\bibitem{lai2} Y. Lai and H. A. Haus, \textit{Quantum Theory of Solitons in Optical Fibers. II. Exact Solution}, Phys. Rev. A {\bf 40}, 854 (1989).

%Kurizki gap soliton
\bibitem{Kurizki1} A. Kozhekin and G. Kurizki, \textit{Self-Induced Transparency in Bragg Reilectors: Gap Solitons near Absorption Resonances}, Phys. Rev. Lett. {\bf 74}, 5020 (1995).

%Kurizki gap soliton
\bibitem{Kurizki2} Z. Cheng and G. Kurizki, \textit{Optical "Multiexcitons": Quantum Gap Solitons in Nonlinear Bragg ReAectors}, Phys. Rev. Lett. {\bf 75}, 3430 (1995).

%Kurizki gap soliton
\bibitem{Kurizki3} A. Kozhekin, G. Kurizki, and B. Malomed \textit{Standing and Moving Gap Solitons in Resonantly Absorbing Gratings}, Phys. Rev. Lett. {\bf 81}, 3647  (1998).


%reit bound state

\bibitem{firstrev} O. Firstenberg, C. S. Adams, and S. Hofferberth,   \textit{Nonlinear quantum optics mediated by Rydberg interactions},  Phys. B: At. Mol. Opt. Phys.  {\bf 49}, 152003 (2016).

\bibitem{callum} C. Murray and T. Pohl, \textit{Quantum and Nonlinear Optics in Strongly Interacting Atomic Ensembles},  Adv.  in Atomic, Molecular, and Optical Physics   {\bf 65}, 321-372 (2016).





%exp reit

\bibitem{first} O. Firstenberg, T. Peyronel, Q-Y. Liang, A. V. Gorshkov, M. D. Lukin, and V. Vuletic, \textit{Attractive photons in a quantum nonlinear medium},   Nature    {\bf 502}, 71-75 (2013).



\bibitem{LiangBS} Q-Y. Liang, A. V. Venkatramani, S. H. Cantu, T. L. Nicholson, M. J. Gullans, A. V. Gorshkov, J. D. Thompson, C. Chin, M. D. Lukin, and V. Vuletic, \textit{Observation of three-photon bound states in a quantum nonlinear medium},   Science    {\bf 359}, 783 (2018).




% reit generic

%\bibitem{petro}D. Petrosyan, J. Otterbach, and M. Fleischhauer,\textit{Electromagnetically Induced Transparency with Rydberg Atoms},  Phys. Rev. Lett. {\bf 107}, 213601  (2011).





%% review quantum emitter coupled to photonic nano structures
\bibitem{roy_rev_mod} D. Roy, C. M. Wilson, and O. Firstenberg, \textit{Colloquium: Strongly interacting photons in one-dimensional continuum}, Rev. Mod. Phys. {\bf 89},  021001 (2017).



\bibitem{review_lodahl} P. Lodahl, S. Mahmoodian, and S. Stobbe, \textit{Interfacing single photons and single quantum dots with photonic nanostructures}, Rev. Mod. Phys. {\bf 87}, 347 (2015).







\bibitem{Darrick_rev_mod} D. E. Chang, J. S. Douglas, A. Gonz\'alez-Tudela, C. L. Hung, and H. J. Kimble, \textit{Colloquium: Quantum matter built from nanoscopic lattices of atoms and photons}, Rev. Mod. Phys.  {\bf 90}, 031002 (2018).


%BD waveguide

\bibitem{shen_fan_prl} J.-T. Shen and S. Fan,  \textit{Strongly Correlated Two-Photon Transport in One-Dimensional Waveguide Coupled to A Two-Level System},   
Phys. Rev. Lett.  {\bf 98}, 153003  (2007).  

\bibitem{shen_fan} J.-T. Shen and S. Fan,  \textit{Strongly correlated multiparticle
transport in one dimension through a quantum impurity},   
Phys. Rev. A  {\bf 76}, 062709 (2007).  


%\bibitem{Gorprl11} A. V. Gorshkov, J. Otterbach, M. Fleischhauer, T. Pohl, and M. D. Lukin \textit{Photon-Photon Interactions via Rydberg Blockade}, Phys. Rev. Lett. {\bf 107}, 133602 (2011).

\bibitem{Harold_strcorr} H. Zheng, D. Gauthier, and H.U. Baranger, \textit{Strongly-Correlated Photons Generated by Coupling a Three-or Four-level System to a Waveguide}, Phys. Rev. A {\bf 85}, 043832 (2012).


%\bibitem{zeuthen} E. Zeuthen, M. J. Gullans, M. F. Maghrebi, and A. V. Gorshkov \textit{Correlated Photon Dynamics in Dissipative Rydberg Media}, Phys. Rev. Lett. {\bf 119}, 043602 (2017) .



\bibitem{Prasadnat} A. S. Prasad, J. Hinney, S. Mahmoodian, K. Hammerer, S. Rind, P. Schneeweiss, A. S. S\"orensen, J. Volz, and  A. Rauschenbeutel, \textit{Correlating photons using the collective nonlinear response of atoms weakly coupled to an optical mode}, Nat. Photonics {\bf 14}, 719 (2020).


%\bibitem{PohlEIT} O. A. Iversen and T. Pohl,  \textit{Strongly Correlated States of Light and Repulsive Photons in Chiral Chains of Three-Level Quantum Emitters},Phys. Rev. Lett. 
%{\bf 126}, 083605 (2021).



%\bibitem{Birnba} K.M. Birnbaum, A. Boca,A R. Miller, A. D. Boozer, T. E. Northup, and H. J. Kimble, \textit{Theory of Photon Blockade by an Optical Cavity with One Trapped Atom},   Nature {\bf 43}, 87 (2005).

%\bibitem{darrick_crsytal} D. E. Chang, V. Gritsev, G. Morigi, V. Vuletic, M. D. Lukin, and E. A. Demler, \textit{Crystallization of strongly interacting photons in a nonlinear optical fibre},   Nature {\bf 4}, 884 (2008).



\bibitem{mahmo_calajo} S. Mahmoodian, G. Calaj\'o, D. E. Chang, K. Hammerer, and A. S. S\"orensen, \textit{Dynamics of many-body photon bound states in chiral waveguide QED},   Phys. Rev. X {\bf 10}, 031011 (2020).




%waveguide bound state

\bibitem{Harold_bs} H. Zheng, D. Gauthier, and H.U. Baranger, \textit{Cavity-Free Photon Blockade Induced by Many-Body Bound States}, Phys. Rev. Lett. {\bf 107}, 223601 (2011).








\bibitem{changnl} D. E. Chang, V. Vuletic, and M. D. Lukin,  \textit{Quantum nonlinear optics - photon by photon},   Nat. Photonics  {\bf 8}, 685 (2014).

\bibitem{McCall1} S. L. McCall and E. L. Hahn, \textit{Self-Induced Transparency by Pulsed Coherent Light},   Phys. Rev. Lett. {\bf 18}, 908 (1967). 

\bibitem{McCall2} S. L. McCall and E. L. Hahn, \textit{Self-Induced Transparency},   
Phys. Rev.  {\bf 183}, 457 (1969). 

\bibitem{Bullough} R. K. Bullough, P. J. Caudrey, J. C. Eilbeck, and J. D. Gibbon, \textit{A General Theory of Self-Induced Transparency},   
Opto-electronics  {\bf 6}, 121 (1974).





\bibitem{Bethe} H. A. Bethe,  \textit{Strongly correlated multiparticle
transport in one dimension through a quantum impurity},   
Z. Phys.  {\bf 71}, 205 (1931).  

\bibitem{sahandprl} S. Mahmoodian, M. Cepulkovskis, S. Das, P.Lodahl, K. Hammerer, and A. S. S\"orensen, \textit{Strongly Correlated Photon Transport in Waveguide Quantum Electrodynamics with Weakly Coupled Emitters}, Phys. Rev. Lett. {\bf 121}, 143601 (2018).


\bibitem{bienas} P. Bienias, S. Choi, O. Firstenberg, M. F. Maghrebi, M. Gullans, M. D. Lukin, A. V. Gorshkov, and H. P. B\"uchler \textit{Scattering resonances and bound states for strongly interacting Rydberg polaritons},   Phys. Rev. A {\bf 90}, 053804 (2014).

\bibitem{Efimov} M. J. Gullans, S. Diehl, S. T. Rittenhouse, B. P. Ruzic, J. P. D Incao, P. Julienne, A. V. Gorshkov, and J. M. Taylor,  \textit{Efimov States of Strongly Interacting Photons}, Phys. Rev. Lett.  {\bf 119}, 233601 (2017).

\bibitem{magrebi} M. F. Maghrebi, M. J. Gullans, P. Bienias, S. Choi, I. Martin, O. Firstenberg, and
M. D. Lukin, H. P. B\"uchler,  and A. V. Gorshkov,  \textit{Coulomb Bound States of Strongly Interacting Photons},   Phys. Rev. Lett.  {\bf 115}, 123601 (2015).


%spin model



\bibitem{anapra} A. Asenjo-Garcia, J. D. Hood,  D. E. Chang, and H. J. Kimble, \textit{Atom-light interactions in quasi-1D nanostructures: a Green s function perspective}, Phys. Rev. A  {\bf95}, 033818 (2017).

\bibitem{anaprx} A. Asenjo-Garcia, M. Moreno-Cardoner, A. Albrecht, H. J. Kimble, D. E. Chang,   \textit{Exponential Improvement in Photon Storage Fidelities Using Subradiance and Selective Radiance in Atomic Arrays}, Phys. Rev. X    {\bf 7}, 031024 (2017).



\bibitem{ana_masson} S. J. Masson and A. Asenjo-Garcia, \textit{Atomic-waveguide quantum electrodynamics}, Phys. Rev. Research  {\bf 2}, 043213 (2020).

\bibitem{rev_wqedarray} A. S.  Sheremet, M. I. Petrov, I. V. Iorsh, A. V. Poshakinskiy, and A. N. Poddubny, \textit{Waveguide quantum electrodynamics: collective radiance and photon-photon correlations}, arXiv:2103.06824 (2021).

\bibitem{MPSJames} M. T. Manzoni, D. E. Chang, and J. S. Douglas, \textit{Simulating quantum light propagation through atomic ensembles using matrix product states}, Nat. Commun.  {\bf 8}, 1743 (2017).

\bibitem{Caneva} T. Caneva, M. T. Manzoni, T. Shi, J. S. Douglas, J. I. Cirac, and D. E. Chang, \textit{Quantum dynamics of propagating photons with strong interactions: a generalized input-output formalism}, New J. Phys. {\bf 17}, 113001 (2015).

\bibitem{glauber} S. Prasad and R. J. Glauber, \textit{Coherent scattering by a spherical medium of resonant atoms}, Phys. Rev. A {\bf 83}, 063821  (2011).

\bibitem{Blais_in} K. Lalumiere, B. C. Sanders, A. F. Van Loo, A. Fedorov, A. Wallraff, and A. Blais, \textit{Input-output theory for waveguide QED with an ensemble of inhomogeneous atoms}, Phys. Rev. A {\bf 88}, 043806 (2013).






%suberadiance




\bibitem{Albrecht} A. Albrecht, L. Henriet, A. Asenjo-Garcia, P. B. Dieterle, O. Painter, and D. E. Chang, \textit{Subradiant states of quantum bits coupled to a one-dimensional waveguide}, New J. Phys.  {\bf 21}, 025003 (2019).

\bibitem{Loic} L. Henriet, J. S. Douglas, D. E. Chang, and A. Albrecht  \textit{Critical open-system dynamics in a one-dimensional optical lattice clock}, Phys. Rev. A {\bf 99} (2), 023802 (2019).

\bibitem{ritch} L. Ostermann, C. Meignant, C. Genes, and H. Ritsch, \textit{Super-and subradiance of clock atoms in multimode optical waveguides}, New J. Phys.  {\bf 21} (2), 025004 (2019).


\bibitem{dimermolmer1}  Y. Zhang, C. Yu, and K. M\o lmer, \textit{Theory of subradiant states of a one-dimensional two-level atom chain}, Phys. Rev. Lett. {\bf 122}( 20), 203605 (2019).

\bibitem{zhang} Y. Zhang  and K. M\o lmer, \textit{Subradiant Emission from Regular Atomic Arrays: Universal Scaling of Decay Rates from the Generalized Bloch Theorem}, Phys. Rev. Lett. {\bf 125}, 253601 (2020).

\bibitem{buchler} J. Kumlin, K. Kleinbeck, N. Stiesdal, H. Busche, S. Hofferberth, and H. P. B\"uchler, \textit{Non-exponential decay of a collective excitation in an atomic ensemble coupled to a one-dimensional waveguide}, Phys. Rev. A {\bf 102}, 063703 (2020).

%\bibitem{korno} G. Fedorovich, D. Kornovan, and M. Petrov, \textit{Disorder in two-level atom array chirally coupled via waveguiding mode}, arXIv:2012:06886v1 (2020).


\bibitem{dimer_shermet}  D. F. Kornovan, N. V. Corzo, J. Laurat,  A. S. Sheremet,  \textit{Extremely subradiant states in a periodic one-dimensional atomic array}, Phys. Rev. X  {\bf 100}, 063832 (2019).

\bibitem{Pod_scat} Y. Ke, A. V. Poshakinskiy, C. Lee, Y. S. Kivshar, and A. N. Poddubny, \textit{Inelastic scattering of photon pairs in qubit arrays with subradiant states}, Phys. Rev. Lett. {\bf 123}, 253601 (2019).

\bibitem{Pod_3} J. Zhong and A. N. Poddubny, \textit{Classification of three-photon states in waveguide quantum electrodynamics}, Phys. Rev. A  {\bf 103}, 023720 (2021).

\bibitem{Lesanosky} J. A. Needham,  I. Lesanovsky, and B. Olmos,
\textit{Subradiance-protected excitation transport},  New J. Phys. {\bf 21}, 073061 (2019).


\bibitem{Yudson} A. A. Makarov  and V. I. Yudson, \textit{Quantum Engineering of Superdark Excited States in Arrays of Atoms}, Phys. Rev. A {\bf 102}, 053712 (2020).





%photon localized
\bibitem{Pod_photon_loc} J. Zhong, N. A. Olekhno, Y. Ke, A. V. Poshakinskiy, C. Lee, Y. S. Kivshar, and A. N. Poddubny, \textit{Photon-mediated localization in two-level qubit arrays}, Phys. Rev. Lett. {\bf 124}, 093604 (2020).



%topology
\bibitem{Pod_topo}  A. V. Poshakinskiy, J. Zhong, Y. Ke, N. A. Olekhno,  C. Lee, Y. S. Kivshar, and A. N. Poddubny, \textit{Quantum Hall phase emerging in an array of atoms interacting with photons}, arXiv:2003.08257  (2020).

%chaos
\bibitem{caos}  A. V. Poshakinskiy, J. Zhong, and A. N. Poddubny, \textit{Quantum chaos driven by long-range waveguide-mediated interactions}, Phys. Rev. Lett. {\bf 126}, 203602 (2021). 

%dimers
\bibitem{dimermolmer2}  Y. Zhang, C. Yu, K. M\o lmer, \textit{Subradiant Dimer Excitations of Emitter Chains Coupled to a 1D Waveguide}, Phys. Rev. Research {\bf 2}, 013173 (2020).


\bibitem{dimer_poddu}  A.N. Poddubny, \textit{Quasi-flat band enables subradiant two-photon bound states}, Phys. Rev. A  {\bf 101}, 043845 (2020).







\bibitem{james_ryd} P. Bienias, J. Douglas, A. Paris-Mandoki, P. Titum,
I. Mirgorodskiy, C. Tresp, E. Zeuthen, M. J. Gullans, M. Manzoni, S. Hoerberth, D. Chang, and A. Gorshkov,  \textit{Photon propagation through dissipative rydberg media at large input rates}, Phys. Rev. Research {\bf 2}, 033049 (2020).



%chiral
\bibitem{Stannigel2012} K. Stannigel, P. Rabl, and P. Zoller, \textit{Driven-dissipative preparation of entangled states in cascaded quantum-optical networks}, New J. Phys. {\bf 14}, 063014 (2012). 

\bibitem{Pichler2015} H. Pichler,  T. Ramos, A. J. Daley, and P. Zoller, \textit{Quantum Optics of chiral spin networks},  Phys. Rev. A   {\bfseries 91}, 042116 (2015).

\bibitem{ChiralRev} P. Lodahl, S. Mahmoodian, S. Stobbe, P. Schneeweiss, J. Volz, A. Rauschenbeutel, H. Pichler, and P. Zoller, \textit{Chiral Quantum Optics}, Nature {\bf 541} 473 (2017).




%spin chain magnon


\bibitem{wortis} M. Wortis,  \textit{Solitons and magnon bound states in ferromagnetic Heisenberg chains},   
Phys. Rev.  {\bf 132}, 85 (1963). 

\bibitem{Schneider} T. Schneider,  \textit{Solitons and magnon bound states in ferromagnetic Heisenberg chains},   
Phys. Rev. B  {\bf 24}, 5327 (1981).  

\bibitem{Mattis} D. C. Mattis, \textit{The few-body problem on a lattice},  Rev. Mod. Phys {\bfseries 58}, 361 (1986).

\bibitem{bloch} Fukuhara, T., Schaus, P., Endres, S. Hild, M. Cheneau, I. Bloch, and C. Gross, \textit{Solitons and magnon bound states in ferromagnetic Heisenberg chains},   Nature  {\bf 502}, 76-79  (2013). 


%%spin chain rydberg



\bibitem{petroBS} F. Letscher and D. Petrosyan, \textit{Mobile bound states of Rydberg excitations in a lattice},  Phys. Rev. A   {\bfseries 97}, 043415 (2018).

\bibitem{cooper} C. D. Parmee  and N. R. Cooper, \textit{Decay rates and energies of free magnons and bound states in dissipative XXZ chains},  Phys. Rev. A   {\bfseries 99}, 063615 (2019).

\bibitem{ramos} Y. Chougale, J. Talukdar,  T. Ramos, and R. Nath, \textit{Dynamics of Rydberg excitations and quantum correlations in an atomic array coupled to a photonic crystal waveguide},  Phys. Rev. A {\bfseries 102}, 022816 (2020).







%nanofiber
\bibitem{ReitzPRL2013} D. Reitz, C. Sayrin, R. Mitsch, P. Schneeweiss, and A. Rauschenbeutel, \textit{Coherence Properties of Nanofiber-Trapped Cesium Atoms}, Phys. Rev. Lett. {\bf 110}, 243603 (2013). %nanofiber

\bibitem{Mitsch2014} R. Mitsch,  C. Sayrin,  B. Albrecht,  P.  Schneeweiss, and A. 
Rauschenbeutel, {\it Quantum state-controlled directional spontaneous  emission  of  photons  into  a  nanophotonic waveguide}, Nat. Commun. {\bf 5}, 5713 (2014).


\bibitem{Pablo2017} P. Solano, P. Barberis-Blostein, F. K. Fatemi, L. A. Orozco, and S. L. Rolston, \textit{Super-radiance reveals infinite-range dipole interactions through a nanofiber}, Nat. Commun. {\bf 8},  1857 (2017).



%photonic crystal

\bibitem{Arcari2014} M. Arcari, I. S\"ollner, A. Javadi, S. Lindskov Hansen, S. Mahmoodian, J. Liu, H. Thyrrestrup, E. H. Lee, J. D. Song, S. Stobbe, and P. Lodahl,  \textit{Near-Unity Coupling Efficiency of a Quantum Emitter to a Photonic Crystal Waveguide},  Phys. Rev. Lett. {\bf 113}, 093603 (2014).

\bibitem{Sollner2015} I. S\"ollner, S. Mahmoodian, S. L. Hansen, L. Midolo, A. Javadi, G. Kir\v{s}anske, T. Pregnolato, H.El-Ella, E. H. Lee, J. D. Song, S.Stobbe, and P. Lodahl,
\textit{Deterministic photon-emitter coupling in chiral photonic circuits},  Nat.  Nanotechnol. {\bf 10},  775 (2015).

\bibitem{Goban2014} A. Goban, C.-L. Hung, S. P. Yu, J. D. Hood, J. A. Muniz, J. H. Lee, M. J. Martin, A. C. McClung, K. S. Choi, D. E. Chang, O. Painter, and H. J. Kimble, \textit{Atom-light interactions in photonic crystals}, Nat. Commun. {\bf 5}, 3808 (2014).  

\bibitem{Hood2016} J. D. Hood, A. Goban, A. Asenjo-Garcia, M. Lu, S. P. Yu, D. E. Chang, and H. J. Kimble,  \textit{Atom-atom interactions around the band edge of a photonic crystal waveguide},  PNAS {\bf 113},  10507 (2016).





%circuit QED




\bibitem{Sundaresan} N. M. Sundaresan, R. Lundgren, G. Zhu, A. V. Gorshkov, and A. A. Houck, \textit{Interacting qubit-photon bound states with superconducting circuits}, Phys. Rev. X  {\bf 9}, 011021 (2019).

\bibitem{painter1} M. Mirhosseini, E. Kim, V. S. Ferreira, M. Kalaee, A. Sipahigil, A. J. Keller, and O.Painter, \textit{Superconducting metamaterials for waveguide quantum electrodynamics},   Nat. Commun.  {\bf 9}, 2706 (2018). 

\bibitem{painter2} M. Mirhosseini, E. Kim, X. Zhang, A. Sipahigil, P. B. Dieterle, A. J. Keller, A. Asenjo-Garcia, D. E. Chang, and O. Painter, \textit{Cavity quantum electrodynamics with atom-like mirrors},   Nature  {\bf 569}, 692-697 (2019). 

\bibitem{ustinov} J. D. Brehm, A. N. Poddubny, A. Stehli, T. Wolz, H. Rotzinger,  and A. V. Ustinov,  \textit{Waveguide Bandgap Engineering with an Array of Superconducting Qubits}, npj Quantum Mater., {\bf 6}, 10 (2021).


\bibitem{ustinov_topo} I. S. Besedin, M. A. Gorlach, N. N. Abramov, I. Tsitsilin, I. N. Moskalenko, A. A. Dobronosova, D. O. Moskalev, A. R. Matanin, N. S. Smirnov,
I. A. Rodionov, A. N. Poddubny, and A. V. Ustinov, \textit{Topological photon pairs in a superconducting quantum metamaterial},   Phys. Rev. B {\bf 103}, 224520 (2021).

\bibitem{marcoBS} M. Scigliuzzo, G. Calajo, F. Ciccarello, D.Perez Lozano, A. Bengtsson, P. Scarlino, A. Wallraff, D. Chang, P. Delsing, and S. Gasparinetti, \textit{Extensible quantum simulation architecture based on atom-photon bound states in an array of high-impedance resonators},  arXiv:2107.06852 (2021).

%Many-body loc


\bibitem{Anderson} P. W. Anderson, \textit{Absence of Diffusion in Certain Random Lattices}, Phys. Rev.   {\bf 109}, 1492 (1958).  

\bibitem{ReviewManybodyloc} D. A. Abanin, E. Altman, I. Bloch, and M. Serbyn, \textit{Colloquium: Many-body localization, thermalization, and entangle-
ment},  Rev.  Mod. Phys.  {\bf 91}, 021001 (2019).  

\bibitem{NikosManybodyloc} N. Fayard, L. Henriet, A. Asenjo-Garcia, and D. Chang, \textit{Many-body localization in waveguide QED},  Phys. Rev. Research {\bf 3}, 033233 (2021).  




\bibitem{chang_mirror} D. E. Chang, L. Jiang, A. V. Gorshkov, and H. J. Kimble,  \textit{Cavity QED with atomic mirrors}, New J. Phys. {\bf 14}, 063003 (2012). 

\bibitem{chang_tao} T.Shi, D. E. Chang, and J. I. Cirac, \textit{Multiphoton-scattering theory and generalized master equations},  Phys. Rev. A {\bf 92}, 053834 (2015).

\bibitem{Bakkensen} B. Bakkensen, Y.-X. Zhang, J. Bjerlin, and A. S. S\"orensen, \textit{Photonic Bound States and Scattering Resonances in Waveguide QED},  arXiv:2110.06093 (2021).





%\bibitem{Sipahigil2016}  A. Sipahigil,  R. E. Evans, D. D. Sukachev, M. J. Burek, J. Borregaard, M. K. Bhaskar, C. T. Nguyen, J. L. Pacheco, H. A. Atikian, C. Meuwly, R. M. Camacho, F. Jelezko, E. Bielejec, H. Park, M. Lon\v{c}ar, and M. D. Lukin, \textit{An integrated diamond nanophotonics platform for quantum-optical networks}, Science {\bf 354}, 847 (2016).

%Chiral circuit

\bibitem{Guimond} P. O. Guimond, B. Vermersch, M. L. Juan, A. Sharafiev, G. Kirchmair, and P. Zoller, \textit{A Unidirectional On-Chip Photonic Interface for Superconducting Circuits}, npj Quantum Inf.  {\bf 6}, 32 (2020).

\bibitem{Gheeraert} N. Gheeraert, S. Kono, and Y. Nakamura, \textit{Programmable directional emitter and receiver of itinerant microwave photons in a waveguide}, Phys. Rev. A  {\bf 102}, 053720 (2020).


\bibitem{lukinEIT} M. Fleischhauer and M. D. Lukin, \textit{Quantum memory for photons: Dark-state polaritons},   Phys. Rev. A    {\bf 65}, 022314  (2002).  





\bibitem{Peyr} T. Peyronel, O. Firstenberg, Q.Y. Liang, S. Hofferberth, A. V. Gorshkov, T. Pohl, M. D. Lukin, and V. Vuletic, \textit{Quantum nonlinear optics with single photons
enabled by strongly interacting atoms},   Nature    {\bf 488}, 57-60 (2012).





\bibitem{Efioptica} E. Shahmoon, P. Grisins, H. P. Stimming, I. Mazets, and G. Kurizki \textit{Highly nonlocal optical nonlinearities in atoms trapped near a waveguide}, Optica {\bf 3},  725 (2016).


\bibitem{PRXcaneva} T. Caneva,  J. S. Douglas,  and D. E. Chang, \textit{Photon Molecules in Atomic Gases Trapped Near Photonic Crystal Waveguides}, Phys. Rev.X {\bf 6}, 031017 (2017).

\bibitem{Rydberg-SIT} Z. Bai, C. S. Adams, G. Huang, and W. Li, \textit{Self-Induced Transparency in Warm and Strongly Interacting Rydberg Gases},  Phys. Rev. Lett. {\bf 125}, 263605 (2020).

\bibitem{Lukin_Ryd} M. D. Lukin, M. Fleischhauer, R. Cote, L. M. Duan, D. Jaksch, J. I. Cirac, and P. Zoller, \textit{Dipole Blockade and Quantum Information Processing in Mesoscopic Atomic Ensembles}, Phys. Rev. Lett. {\bf 87}, 037901 (2001).


\bibitem{Hofferberth} N. Stiesdal, J. Kumlin, K. Kleinbeck, P. Lunt, C. Braun, A. Paris-Mandoki, C. Tresp, H. P. B\"uchler, and S. Hofferberth, \textit{Observation of Three-Body Correlations for Photons Coupled to a Rydberg Superatom},  Phys. Rev. Lett.  {\bf 121}, 103601 (2018).

\bibitem{Hofferberth2} N. Stiesdal, H. Busche, J. Kumlin, K. Kleinbeck, H. P. B\"uchler, and S. Hofferberth, \textit{Observation of collective decay dynamics of a single Rydberg superatom},  Phys. Rev. Research  {\bf 2},  043339 (2020).



\bibitem{SorenwqedBS} B. Bakkensen, Y. X. Zhang, J. Bjerlin, A. S. S\"orensen \textit{Photonic bound states and scattering resonances in waveguide qed}, 
arXiv:2110.06093 (2021).


\bibitem{Verstraete} F. Verstraete, V. Murg, and J.I. Cirac,   \textit{Matrix Product States, Projected Entangled Pair States, and Variational Renormalization Group Methods for Quantum Spin Systems}, Adv. Phys. {\bf 57}, 143 (2008).

\bibitem{Schollw} U. Schollw\"ock,   \textit{The Density-Matrix Renormalization Group in the Age of Matrix Product States}, Ann. Phys. (Amsterdam) {\bf 326}, 96 (2011).



\bibitem{Pirvu} B. Pirvu, V. Murg, J. I. Cirac, and F. Verstraete, \textit{Matrix product operator representations}, New
J. Phys. {\bf 12}, 025012 (2010).


\bibitem{Crosswhite} G. M. Crosswhite, A. C. Doherty, and G. Vidal,  \textit{Applying matrix product operators to model systems with long-range interactions}, Phys. Rev. B {\bf 78}, 035116 (2008).

\bibitem{fro} F. Fr\"owis, V. Nebendahl, and W. D\"ur, \textit{Tensor operators: Constructions and applications for long-range interaction systems}, Phys.
Rev. B {\bf 81}, 062337 (2010).

%\bibitem{Pohl_reit_soliton} S. Sevincli, N. Henkel, C. Ates, and T. Pohl, \textit{Nonlocal Nonlinear Optics in Cold Rydberg Gases},  Phys. Rev. Lett. {\bf 107}, 153001 (2011).





\end{thebibliography}

\end{document}