\documentclass[12pt]{article}
\usepackage{anyfontsize}
\usepackage{amsfonts}
\usepackage{amsmath,amsthm,amssymb}
\usepackage{mathptmx}
\usepackage{geometry}
\geometry{top=1in, bottom=1in, left=1.25in, right=0.36in}
\usepackage{graphicx}
\usepackage{subfigure}
\usepackage{hyperref}
\usepackage{cancel}
\usepackage{textcomp}
\usepackage{tikz}
\usepackage{bm}
\usepackage{mathrsfs}
\usepackage{filecontents}
\usepackage{times}
\usepackage{epsfig}
\usepackage{xcolor}
\usepackage{slashed}
\usepackage{booktabs}% build a table
\usepackage{latexsym}
\usepackage{verbatim}
\usepackage{extarrows}
\usepackage{multirow}
\usepackage{rotating}
\usepackage{colortbl}
\usepackage{indentfirst}
\usepackage[numbers,sort&compress]{natbib}
\usepackage{soul}
\definecolor{mygray}{gray}{.9}
\definecolor{intnull}{RGB}{213,229,255}
\usepackage{arydshln}
\usepackage{diagbox}
\usepackage{appendix}
%My new command
\newcommand{\calO}{\rm \mathcal{O}}
\newcommand\mi{\mathrm{i}}
\newcommand\me{\mathrm{e}}
\newcommand\const{\text{const}}
\newcommand*\res[1]{\underset{#1}{\text{res}}}
\newcommand\pp{\uppi}
\newcommand{\dif}{\mathrm{d}}
\newcommand{\ff}{\mathcal{F}}
\newcommand{\ml}{\mathcal{L}}
\newcommand{\Hor}{\mathrm{H}}
\newcommand{\Min}{\mathrm{min}}
\DeclareMathOperator{\diag}{diag}

%\DeclareMathOperator{\sech}{sech}
%\DeclareMathOperator{\csch}{csch}
\DeclareMathOperator{\arcsinh}{arsinh}
\DeclareMathOperator{\arccosh}{arcosh}
\DeclareMathOperator{\arctanh}{artanh}
\DeclareMathOperator{\arccoth}{arcoth}
\DeclareMathOperator{\arcctgh}{arctgh}
\DeclareMathOperator{\arcsech}{arsech}
\DeclareMathOperator{\arccsch}{arcsch}
\renewcommand{\indent}{\hspace{4mm}}
\addtolength{\oddsidemargin}{-1.cm}
\usepackage{caption}
\usepackage{tikz}
\usetikzlibrary{arrows,shapes,chains}
%\usepackage{graphicx, subfig}
\setcounter{MaxMatrixCols}{20}
\begin{document}
\renewcommand{\thefootnote}{\fnsymbol{footnote}}
\baselineskip=16pt
\pagenumbering{arabic}
\vspace{1.0cm}
\begin{center}
{\Large\sf Scalar fields around a rotating loop quantum gravity black hole: Waveform, quasi-normal modes and superradiance}
\\[10pt]
\vspace{.5 cm}
{{Zhong-Wu Xia${}^{1,}$\footnote{E-mail address: xiazw@mail.nankai.edu.cn}}, 
{Hao Yang${}^{1,}$\footnote{E-mail address: hyang@mail.nankai.edu.cn}},
and
{Yan-Gang Miao${}^{1,2,}$\footnote{Corresponding author. E-mail address: miaoyg@nankai.edu.cn}}
	
	\vspace{6mm}
	${}^{1}${\normalsize \em School of Physics, Nankai University, Weijin Road 94, Tianjin 300071, China}
	
	\vspace{3mm}
	${}^{2}${\normalsize \em Faculty of Physics, University of Vienna, Boltzmanngasse 5, A-1090 Vienna, Austria}
}


\vspace{4.0ex}
\end{center}
\begin{center}
{\bf Abstract}
\end{center}

The dynamical behavior of a scalar field near a rotating loop quantum gravity black hole is investigated.
By analyzing the waveform of scalar fields, we find that the loop quantum correction only affects the decaying oscillation of waveforms, which is mainly described by quasi-normal modes.
Moreover, we calculate the quasi-normal modes of scalar field perturbations by using three numerical methods, which are the Prony, WKB, and shooting methods, respectively, and compare the accuracy of results among these methods.
Over the entire parameter space of a rotating loop quantum gravity black hole, we analyze the stability of the spacetime and the influence of loop quantum corrections on the quasi-normal modes of scalar field perturbations, and find that the influence varies with the change of  black hole angular momenta.
Finally, we study the energy amplification effect of black holes on free scalar fields, and analyze the influence of loop quantum corrections on the amplification factor.
Our result shows the diverse influences of loop quantum corrections on the dynamics of scalar fields and superradiance effect of a rotating loop quantum gravity black hole.






\renewcommand{\thefootnote}{\arabic{footnote}}
\newpage
\tableofcontents



%%%%%%%%%%%%%%%%%%%%%%%%%%%%%%%%%%%%%%%%%%%%%%%%%%%%%%%%%%%%%%%%%%%%%%
\newpage
\section{Introduction}
\label{sec:intr}
Although general relativity (GR) is the most widely accepted theory of gravity,  it suffers~\cite{Hawking:1974rv,penrose1969gravitational} from several challenges and unresolved issues, such as singularity, information loss paradox, and breakdown of predictability, etc.
One effective approach to resolve~\cite{Lan:2023cvz} the conundrum of black hole singularities is to construct regular black hole models.
In a regular black hole spacetime,  there are no intrinsic singularities, thus naturally avoiding the issues associated with intrinsic singularities. 
Although the first regular black hole was constructed~\cite{bardeen1968non} within the scope of GR, many regular black holes have been proposed~\cite{Bonanno:2000ep,Koch:2014cqa,Bouhmadi-Lopez:2020wve} in the framework of modified gravity theories.
%Due to the singularity theorem of Hawking and Penrose \cite{Penrose:1964wq}, it is impossible to construct a regular black hole model satisfying all energy conditions under general relativity.
%The construction of a regular black hole is usually realized in the following ways: 
%\begin{itemize}
    %\item Solving the Einstein field equation under a special symmetry or matter source;
   % \item Directly modifying the metric so that the corresponding space-time has no singularity, and then invert the effective matter term;
    %\item Constructing under the framework of modifying gravity.
%\end{itemize}
Among the various theories, the loop quantum gravity (LQG) aims to construct a unified theory of quantum gravity and address the issue of spacetime singularities. 
Within the scope of LQG theory, some static and spherically symmetric models of regular black holes have been given~\cite{Modesto:2005zm,Gambini:2013ooa,Bodendorfer:2019nvy,Bodendorfer:2019jay}, where a quantum parameter was introduced to describe the spacetime of regular black holes.%,BenAchour:2018khr,Ashtekar:2018lag,Blanchette:2020kkk,Alesci:2019pbs,Zhang:2020qxw,Sartini:2020ycs,Assanioussi:2019twp,DeLorenzo:2015taa
%In order to better understand regular black holes, it is necessary to study their observables, such as shadows\cite{Perlick:2021aok}, quasi-normal modes (QNMs)\cite{Konoplya:2011qq,Li:2022kch,Franzin:2022iai}, superradiance\cite{Brito:2015oca,Li:2022kch,Franzin:2022iai} etc.


 
It is known that astrophysical black holes are rotating, which means that the research on static black holes alone has only limited effects on observations.
Recently, a rotating model of regular black holes was suggested~\cite{Brahma:2020eos} in the LQG theory, where its shadow was analyzed and connected to possible future observational data~\cite{Afrin:2022ztr}.
Considering that shadows are only one phenomenon to show the connection between intrinsic properties of black holes and observations, we  
explore other possible phenomena beyond shadows and investigate their potential observable effects  in the scope of LQG theory. 
In the present work we focus on two phenomena: Quasi-normal modes (QNMs) \cite{Konoplya:2011qq,Kokkotas-Schmidt-1999,Li:2022kch,Franzin:2022iai} and superradiance \cite{Brito:2015oca,Li:2022kch,Franzin:2022iai}.

%The ensuing problem is how to detect such regular black holes.
%black holes in nature usually have angular momentum, and the research on the static spherical symmetry model cannot fully meet the detection requirements.
%Recently, a rotating regular black hole model from LQG was proposed, and this study gave a method to judge such black holes through shadows.
%But the parameters of the black hole can only be limited within a certain range through the shadow judgment, and more precise restrictions need to be given by more messenger channels.
%Therefore, this study will start with other important observables different from the shadow, and study the possibility of judging the regular black hole parameters under the quantum gravity of the rotating circle through other messenger signals.
%Here, we focus on two observables, quasi-normal modes and superradiance effects.
In GR, the QNM frequencies of scalar field perturbations are determined by the mass of scalar fields, the mass and angular momentum of black holes. QNMs are complex due to the existence of event horizons,  so we can divide a QNM frequency into a real part and an imaginary part,
\begin{equation}
    \omega=\omega_{\rm R}+i\omega_{\rm I},
\end{equation} 
where the real part $\omega_{\rm R}$ represents the oscillation frequency and the imaginary part $\omega_{\rm I}$ denotes the decay rate.  
In previous works~\cite{Konoplya:2011qq}, it has been noted that QNMs are highly sensitive to boundary conditions, particularly the asymptotic behaviors of scalar fields near event horizons.  The difference between a LQG metric and a Kerr metric will  lead to differences of boundary conditions and then affect QNMs. In astrophysical observations  the ringdown phase after the merger of two black holes is described by perturbation theory and gravitational waves are a linear superposition of QNMs~\cite{cardoso2016gravitational,gerosa2021hierarchical,Weih:2019xvw}. Through computing the QNMs of scalar field perturbations around a rotating LQG black hole (rLQGBH), we can provide some hints of the underlying gravity theory.

 %QNMs of the scalar field perturbation describe the characteristic oscillations and decay of scalar fields in the vicinity of a black hole. 
 %And these QNMs provide\cite{Cardoso-Pani-2019}  a valuable description of the behavior of scalar perturbations, offering crucial information about the stability, dynamics, and information propagation within black holes. 
%What's more,  the properties of the black hole, such as its mass and angular momentum, as well as  the scalar field itself, determine the QNMs of the scalar field perturbation. 
%By analyzing the frequencies, damping rates, and waveform of these QNMs, we can extract essential details about the black hole's fundamental parameters and investigate\cite{Hod-1998} the underlying gravitational theories. 

%QNM is an observable closely related to the structure of spacetime, which characterizes the relationship of perturbations in spacetime with time. 
%And its specific form is
%\begin{equation}\label{QNM}
%    \omega_{QNM}=\omega_R+i\omega_I
%\end{equation}
%where $\omega_R$ represents the oscillation frequency of the perturbation in spacetime, and $\omega_I$ the decay rate of the perturbation, both of which are determined by the parameters of the spacetime metric.
%Therefore, when the structure of spacetime changes, QNM will change accordingly.


The superradiance effect is  a radiation enhancement process in a dissipative system. In black hole theory, the superradiance is closely associated~\cite{PressTeukolsky1972, Bekenstein1973, Zeldovich1971, StarobinskyChurilov1973} with the ergoregion of rotating black holes.   Especially, the superradiance is a powerful tool to detect \cite{Brito:2015oca,East:2018glu} ultralight scalar fields which are a promising candidate of dark matter. Similarly, the superradiance is also sensitive to boundary conditions, thus the quantum parameter that plays a crucial role in a LQG metric, see Sec.~\ref{sec: Rotating_metric} for the details, will leave imprints on the superradiance effects in rLQGBHs, too. We expect to shed some light on the existence of ultralight scalar particles in the LQG theory through the investigation of superradiance. 
%It refers to the situation in which a particle, under specific conditions, interacts with a black hole resulting in an energy flow reflected greater than the incident energy flow.
%And this effect exhibits\cite{BertiCardosoStarinets2009} the potential of black holes to act as particle accelerators due to the amplification of energy.
%Generally, the superradiance effect can occur\cite{Brito:2015oca} in charged black holes and rotating black holes.
%But black holes are assumed to lack electric charge according to astronomical observations. 
%Therefore superradiance is limited to rotating black holes. Understanding the signatures and characteristics of superradiance can help us in the search for observational evidence of this phenomenon, thereby enhancing our overall understanding of black hole properties.


%On the other hand, superradiance effect is an observable effect closely related to rotating black hole.
%When a particle meeting certain conditions incident on a black hole under certain conditions, the effect that the reflected energy flow of the particle is greater than the incident energy flow is called "superradiance effect".
%Because of this energy amplification effect, black holes are thought to have the potential to become particle accelerators.
%In general, astronomical observations generally believe that black holes have no electromagnetic field.
%In this case, the superradiance effect occurs only in the rotating black hole spacetime.
%Therefore, the study of amplification and instability of the superradiation effect can further strengthen our understanding of the rotation parameters of spacetime, which is expected to better judge the rotation parameters in the detection.

Our research focuses on QNMs and superradiance effects, both of which need to deal with Klein-Gordon equations with boundary conditions.
The difficulty of calculations lies in the complicacy of rLQGBHs, where  
one aspect comes from the angular equation due to rotations,  and the other comes from the complicated radial equation.
For example, the  Leaver method~\cite{leaver1985analytic}, previously applied to static and spherically symmetric black holes or some simple rotating black holes, is unable to deal with rLQGBHs.
Owing to this reason, we employ three other numerical methods, the Prony, WKB, and shooting methods, to calculate the QNMs of scalar field perturbations around rLQGBHs and compare the results among the three methods in order to obtain the most precise QNMs. Moreover, we analyze the superradiance effect in rLQGBHs by adopting the shooting method which is the most efficient one among the three.
%And the method corresponding to the most accurate results is then applied to calculate the superradiance effect.

The paper is organized as follows.
In Sec.~\ref{sec: Rotating_metric}, we briefly introduce the rotating black holes in loop quantum gravity.
In Sec.~\ref{sec:TD}, we analyze the time domain waveform under scalar field perturbations.
In Sec.~\ref{sec:QNM}, we compute the QNMs of scalar field perturbations around rLQGBHs by using the three numerical methods. 
In Sec.~\ref{sec:superradiance}, we apply the shooting method to calculate the amplification factor. 
Finally, we give our conclusion in Sec.~\ref{sec:con}. The natural units $(G = c = \hbar =1)$ are adopted in our paper.








%%%%%%%%%%%%%%%%%%%%%%%%%%%%%%%%%%%%%%%%%%%%%%%%%%%%%%%%%%%%%%%%%%%%%%
%%%%%%%%%%%%%%%%%%%%%%%%%%%%%%%%%%%%%%%%



\section{Rotating black holes in loop quantum gravity }
\label{sec: Rotating_metric}
In this section we briefly describe the geometry of rLQGBHs.
From a static and spherically symmetric LQGBH, its rotating counterpart was constructed~\cite{Brahma:2020eos,Azreg-Ainou:2014pra} in terms of the modified Newman-Janis algorithm.  
In the Boyer-Lindquist coordinates $(t,r,\theta,\varphi)$, the line element of rLQGBHs reads 
\begin{equation}
\label{1rotating_LQGBH_metric}
{\mathrm d}s^2=-\left(1-\frac{2Mb}{\rho^2}\right) {\mathrm d}t^2
-\frac{4aMb\mathrm{sin}^2\theta}{\rho^2}{\mathrm d}t{\mathrm d}\varphi
+\rho^2{\mathrm d}\theta^2+\frac{\rho^2}\Delta {\mathrm d}r^2
+\frac{\Sigma \mathrm{sin}^2\theta}{\rho^2}{\mathrm d}\varphi^2,
\end{equation}
where
%\begin{subequations}
\begin{eqnarray}
\rho^2&=&a^2\mathrm{cos}^2\theta+b^2,\label{eq:rho}\\
%\end{equation}
%\begin{equation}
\Delta&=&b^2+a^2-2Mb, \label{eq:Delta}\\
%\end{equation}
%\begin{equation}
\Sigma&=&\left( b^2+a^2\right) ^2-a^2\Delta \mathrm{sin}^2\theta,\label{eq:Sigma}\\
%\end{equation}
%\begin{equation}
b^2&=&\frac{A_\lambda }{\sqrt{1+x^2}}
\frac{M_{\rm B}^2\left( x+\sqrt{1+x^2}\right) ^6+M_{\rm B}^2}{\left( x+\sqrt{1+x^2}\right) ^3},\label{areal}\\
%\end{equation}
%\begin{equation}
M&=&\frac b 2\left[ 1-\frac{8A_\lambda M_{\rm B}^2}{b^2} \left( 1-\sqrt{\frac 1{2A_\lambda}}\frac {1} {\sqrt{1+x^2}}\right)\left(1+x^2\right) \right], \label{Mb}\\
%\end{equation}
%\begin{eqnarray}
x&=&\frac r{\sqrt{8A_\lambda}M_{\rm B}}.\label{eq:x}
\end{eqnarray}
%\end{subequations}
Here the angular momentum $a$ and Arnowitt-Deser-Misner (ADM) mass $M_{\rm B}$ are assumed to be positive, and $A_\lambda$ is a positive dimensionless quantum parameter originated~\cite{Bodendorfer:2019nvy} from holonomy modifications.
The above rLQGBH is regular~\cite{Brahma:2020eos} everywhere when $A_\lambda>0$ and reduces to a Kerr black hole when $A_\lambda=0$.
By introducing the  transformation, 
\begin{equation}\label{eq:h}
h=\sqrt{r^2+8A_\lambda M_{\rm B}^2},
\end{equation}
we simplify  $b^2$,  $M$, and $\Delta$ to be 
%\begin{subequations}
\begin{eqnarray}
%\end{eqnarray}
%\begin{equation}
b^2&=&h^2-6A_\lambda M_{\rm B}^2,\label{eq:b2}\\
%\end{equation}
%\begin{eqnarray}
Mb&=&M_{\rm B}h-3A_\lambda M_{\rm B}^2.\label{eq:Mb2}\\
\Delta&=&h^2+a^2 -2M_{\rm B}h,\label{eq:Delta2}
\end{eqnarray}
%\end{subequations}
   
The location of horizons can be determined by the algebraic equation,
\begin{equation}\label{eq:horizon}
		\Delta=h^2+a^2-2M_{\rm B}h=0,
\end{equation}
and its solutions are
\begin{equation}\label{eq:horizon2}
    	h_\pm=M_{\rm B}\pm \sqrt{M_{\rm B}^2-a^2},
\end{equation}
where a plus or minus sign represents an outer or inner horizon.
According to the existence of an outer  horizon, we restrict the two dimensionless parameters, $A_\lambda$ and $a/M_{\rm B}$ in the shadow region of  Fig.~\ref{fig:rLQG_parameter}, where the blue curve corresponds to the extreme configuration. From the parameter space $(A_\lambda, a/M_{\rm B})$, we obtain that the maximum value of  angular momenta equals one, $a_{\rm Max}/{M_{\rm B}}=1$, which will be adopted in the numerical calculations below. For a given angular momentum $a$, the entire range of quantum parameter $A_\lambda$ is $0\le A_\lambda\le A_{\lambda \mathrm{Max}}$, where
the extreme configuration of rLQGBHs takes the maximum value of quantum parameter $A_{\lambda}$,
\begin{equation}
    A_{\lambda \mathrm{Max}}=\frac{1}{8}\left(1+\sqrt{1-\frac{a^2}{M_{\rm B}^2}}\right)^2.\label{max}
\end{equation}
%When $A_\lambda=A_{\lambda \mathrm{Max}}$, the metric Eq.~(\ref{1rotating_LQGBH_metric}) describes an extreme rLQGBH.

\begin{figure}[htbp]
		\centering
		\begin{minipage}[t]{0.5\linewidth}
			\centering
			\includegraphics[width=1\linewidth]{figure/parameter}
		\end{minipage}
		\caption{The parameter space $(A_\lambda, a/M_{\rm B})$ of rLQGBH spacetime, where the shadow region is allowed.}
\label{fig:rLQG_parameter}
	\end{figure}



%%%%%%%%%%%%%%%%%%%%%%%%%%%%%%%%%%%%%%%%%%%%%%%%%%%%%%%%%%%%%%%%%%%%%    
\section{Time domain waveform}\label{sec:TD}
%%%%%%%%%%%%%%%%%%%%%%%%%%%%%%%%%%%%%%%%%%%%%%%%%%%%%%%%%%%%%%%%%%%%%
The evolution of perturbations can be divided~\cite{Konoplya:2011qq} into three distinct stages. At first, the perturbation field undergoes an initial outburst, and then a relatively long period of decaying oscillation, and finally an exponential late-time tail. In this section we focus on the impact of quantum parameter $A_\lambda$ on the whole time domain evolution of scalar fields.
%%%%%%%%%%%%%%%%%%%%%%%%%%%%%%%%%%%%%%%%%%%%%%%%%%%%%
\subsection{Numerical method}\label{subsec:setup_TD}
%%%%%%%%%%%%%%%%%%%%%%%%%%%%%%%%%%%%%%%%%%%%%%%%%%%%%
When a massive scalar field $\Phi$ acts as perturbation around an rLQGBH, its equation of motion is the Klein-Gordon equation,
    \begin{equation}\label{eq_KG}
        \nabla^\nu\nabla_\nu\Phi=\mu^2\Phi,
    \end{equation}
where $\mu$ is the mass of scalar fields. In this section we merely consider massless scalar fileds, so we set $\mu=0$.
Generally, we can simulate the waveform $\Phi$ of scalar fields by solving the above equation in $(t,r,\theta,\varphi)$ coordinates.
However, the traditional (2+1)-dimensional simulations around rotating black holes often suffer~\cite{ThuestadKhannaPrice2017,zhang2020object} from a serious boundary problem because of the Cauchy foliation, which destroys the precision of numerical calculations. 
To address this issue, we employ the hyperbolic foliation-dependent strategy~\cite{Zenginoglu2008a,Zenginoglu2008b,Zenginoglu2008c,Zenginoglu2009,Zenginoglu2009b,Zenginoglu2010,Zenginoglu2011a,Zenginoglu2011b} to compute the evolution waveform in time domain in order to avoid the subtle boundary problem through the following two coordinate transformations.
    
In the first transformation, we construct the horizon-penetrating coordinates $\{\tilde t, h,\theta, \tilde \varphi\} $ through 
    \begin{equation}\label{eq:trans_hp}
    	\mathrm d \tilde t=\mathrm d t+ \frac{2Mb}{\epsilon\Delta}\mathrm dh,\qquad
    	\mathrm d \tilde \varphi=\mathrm d \varphi+\frac a {\epsilon\Delta}\mathrm dh,
    \end{equation}
where
\begin{equation}\label{eq:defepsilon}
\epsilon=\sqrt{1-\frac{8A_\lambda M_{\rm B}^2}{h^2}}.
\end{equation}
The rLQGBH metric Eq.~\eqref{1rotating_LQGBH_metric} then becomes 
    \begin{equation}
    	\begin{split}
    		\mathrm ds^2=&-\left(1-\frac{2Mb}{\rho^2}\right)\mathrm d \tilde t^2
    		-\frac{4aMb}{\rho^2}\mathrm{sin}^2\theta\mathrm d\tilde t \mathrm d \tilde \varphi+\frac{4Mb}{\epsilon \rho^2}\mathrm d\tilde t\mathrm d h\\
    		&
    		+\frac 1{\epsilon^2}\left(1+ \frac{2Mb}{\rho^2}\right) \mathrm d h^2
    		-\frac 2\epsilon a\mathrm{sin}^2\theta\left( 1+\frac{2Mb}{\rho^2}\right) \mathrm d h \mathrm d \tilde \varphi\\
    		&+\rho^2\mathrm d\theta^2
    		+\left(b^2+a^2+\frac{2Mba^2\mathrm{sin}^2\theta}{\rho^2} \right)\mathrm{sin}^2\theta\mathrm d\tilde\varphi^2.
    	\end{split}
    \end{equation}
Since the metric does not contain $\tilde \varphi$ explicitly,  $\partial_{\tilde \varphi}$ is Killing vector. 
In order to preserve this feature under the time domain evolution, we only separate variable $\tilde{\varphi}$ in the waveform of scalar field perturbations, 
    \begin{equation}
    	\Phi(\tilde t, h, \theta, \tilde \phi)=\frac 1h \sum_{m}\psi(\tilde t,h ,\theta)\mathrm e^{im\tilde\varphi},
    \end{equation}
where $m$ is azimuthal number. After substituting the above expression into Eq.~\eqref{eq_KG} and considering a massless scalar field, we express the Klein-Gordon equation as
\begin{equation}\label{eq:KG-hp}
    	A^{\tilde t\tilde t}\partial^2_{\tilde t}\psi
    	+A^{\tilde t h}\partial_{\tilde t}\partial_h\psi
    	+A^{hh}\partial^2_h\psi
    	+A^{\theta\theta}\partial^2_\theta\psi
    	+B^{\tilde t}\partial_{\tilde t}\psi
    	+B^h\partial_h\psi
    	+B^\theta\partial_\theta\psi
    	+C\psi
    	=0,
\end{equation}
where
\begin{eqnarray}
%\begin{split}
A^{\tilde t\tilde t}&=&\rho^2+2Mb,\nonumber\\
A^{\tilde t h}&=&-4\epsilon Mb,\nonumber\\
A^{hh}&=&-\epsilon^2 \Delta,\nonumber\\
A^{\theta\theta}&=&-1,\nonumber\\
B^{\tilde t}&=&2M_B\epsilon-\epsilon\frac{12A_\lambda M_B^2}{h},\nonumber \\
B^h&=&\epsilon^2  \frac 2h(a^2-M_Bh) -2ima\epsilon-\epsilon\Delta \frac{\mathrm d} {\mathrm d h}\epsilon,\nonumber\\
B^\theta&=&-\mathrm{cot}\theta,\nonumber\\
C&=&\frac{m^2}{\mathrm{sin}^2\theta}-\epsilon^2\frac{2(a^2-M_Bh)}{h^2}+\epsilon\frac{2ima}h+\epsilon\frac\Delta h  \frac{\mathrm d} {\mathrm d h}\epsilon.
%\end{split}
\end{eqnarray}

In the second transformation, we introduce the hyperbolic foliation~\cite{Harms:2014dqa} to define the compact horizon-penetrating and hyperboloidal coordinates (HH coordinates), $\{\tau, \tilde r,\theta, \tilde \varphi\} $,
    \begin{equation}\label{hyper}
    	\tilde t=\tau+f(\tilde r),\qquad h=\frac {\tilde r}{\Omega(\tilde r)},
    \end{equation}
where 
    \begin{equation}
    	f(\tilde r)=\frac {\tilde r} {\Omega({\tilde r})}-{\tilde r}-4M_{\rm B}\, \mathrm{ln}\,\Omega({\tilde r}),\qquad \Omega({\tilde r})=1-\frac {\tilde r}S.
    \end{equation}
Here $S$ is a constant associated with the hyperbolic foliation. 
According to this coordinate transformation and Eq.~\eqref{eq:horizon2}, we obtain the location of outer horizon,
    \begin{equation}
    	{\tilde r}_+=\frac{a^2 S+\left(M_{\rm B}+\sqrt{M_{\rm B}^2-a^2}\right)S^2}{a^2+2M_{\rm B}S+S^2},
    \end{equation}
and the relations between the HH coordinates and the horizon-penetrating coordinates,
    \begin{equation}\label{eq:HH-hp}
    	\partial_{\tilde t}=\partial_\tau,\qquad \partial_h=-H\partial_\tau+K\partial_{\tilde r},
    \end{equation}
where 
\begin{equation}\label{eq:defHandP}
H=\frac{\mathrm df}{\mathrm dh}({\tilde r}), \qquad K=\frac{\mathrm d{\tilde r}}{\mathrm dh}({\tilde r}).
\end{equation} 
Therefore, the Klein-Gordon equation Eq.~\eqref{eq:KG-hp} can be expressed in the HH coordinate as follows:
   \begin{equation}\label{eq:KG-HH}
   	\partial^2_{\tau}\psi
   	={\tilde A}^{\tau {\tilde r}}\partial_{\tau}\partial_{\tilde r}\psi
   	+{\tilde A}^{{\tilde r}{\tilde r}}\partial^2_{\tilde r}\psi
   	+{\tilde A}^{\theta\theta}\partial^2_\theta\psi
   	+{\tilde B}^{\tau}\partial_{\tau}\psi
   	+{\tilde B}^{\tilde r}\partial_{\tilde r}\psi
   	+{\tilde B}^\theta\partial_\theta\psi
   	+{\tilde C}\psi,
   \end{equation}
where
\begin{eqnarray}
%\begin{split}
\{{\tilde A}^{\tau {\tilde r}},
{\tilde A}^{{\tilde r}{\tilde r}},
{\tilde A}^{\theta\theta},
{\tilde B}^{\tau},
{\tilde B}^{\tilde r},
{\tilde B}^\theta,
{\tilde C}\}
&=&-\frac 1{A^{\tau\tau}}
\{{A}^{\tau {\tilde r}},
{A}^{{\tilde r}{\tilde r}},
{A}^{\theta\theta},
{B}^{\tau},
{B}^{\tilde r},
{B}^\theta,
{C}\},\nonumber\\
A^{\tau\tau}&=&A^{\tilde t\tilde t}-HA^{\tilde t h}+H^2A^{hh},\nonumber\\
A^{\tau {\tilde r}}&=&KA^{\tilde t h}-2KHA^{hh},\nonumber\\
A^{{\tilde r}{\tilde r}}&=&K^2A^{hh},\nonumber\\
B^\tau&=&B^{\tilde t}-HB^h-\frac{\mathrm dH}{\mathrm d {\tilde r}}KA^{hh}, \nonumber\\
B^{\tilde r}&=&K\left(B^h+ \frac{\mathrm dK}{\mathrm d {\tilde r}}A^{hh}\right).\label{HHH}
%\end{split}
\end{eqnarray}
In order to numerically solve Eq.~\eqref{eq:KG-HH} , we introduce an auxiliary function $\Pi$, so that we can reduce this equation to two first-order equations,
%\begin{subequations}
\begin{eqnarray}
\partial_\tau \psi&=&\Pi,\label{RK1}\\
%\end{equation}
%\begin{eqnarray}
%\begin{split}
\partial_\tau \Pi&=&{\tilde B}^{\tau}\Pi
+{\tilde A}^{\tau {\tilde r}}\partial_{\tilde r}\Pi
+{\tilde A}^{{\tilde r}{\tilde r}}\partial^2_{\tilde r}\psi
+{\tilde A}^{\theta\theta}\partial^2_\theta\psi
+{\tilde B}^{\tilde r}\partial_{\tilde r}\psi
+{\tilde B}^\theta\partial_\theta\psi
+{\tilde C}\psi,\label{RK2}
%\end{split}
\end{eqnarray}
%\end{subequations}
which can be solved by the fourth-order Runge-Kutta method. 
Here we take a Gaussian distribution as the initial condition,
\begin{eqnarray}
%\begin{split}
\psi(\tau=0,{\tilde r},\theta)&=&Y_{lm}\,{\exp}\left[-\frac{({\tilde r}-{\tilde r}_c)^2}{2\sigma^2}\right],\nonumber \\
\Pi(\tau=0,{\tilde r},\theta)&=&0,
%\end{split}
\end{eqnarray}
where $Y_{lm}$ is the $\theta$-dependent part of spherical harmonics, and $\tilde r_c$ and $\sigma$ are the center and the width of Gaussian packets, respectively. When we choose the location of our observer $\tilde r_c = 6M_{\rm B}$, $\theta=\frac{\pi}{4}$,
 the width $\sigma=0.2$, and the free parameter $S=10$ as suggested by Refs.~\cite{Harms:2014dqa,zhang2020object},  we solve Eqs.~\eqref{RK1} and \eqref{RK2} and obtain the time domain evolution profiles as shown in Fig.~\ref{instab}.

 \begin{figure}[htbp]
 	\centering
 	\begin{minipage}[t]{0.5\linewidth}
 		\centering
 		\includegraphics[width=1\linewidth]{figure/instability.jpg}
 	\end{minipage}
 	\caption{
 		The waveform of massless scalar field perturbations with a varying $A_\lambda$, where the initial mode is chosen to be $l=1=m$, and $M_{\rm B}=1$ and $a=0.1$ are set.}
 	\label{instab}
 \end{figure}

%%%%%%%%%%%%%%%%%%%%%%%%%%%%%%%%%%%%%%%%%%%%%%%%%%%%%%%%%%%%%%%%%%%%%%
\subsection{Results}\label{sec:TD-result}
%%%%%%%%%%%%%%%%%%%%%%%%%%%%%%%%%%%%%%%%%%%%%%%%%%%%%%%%%%%%%%%%%%%%%%
 %Generally, a rotating black hole exhibits~\cite{zhang2020object} mode-mixing phenomena, that is, an initial even (odd) mode with multipole number $l$ can be excited to other even (odd) modes with the same azimuthal number $m$.
 As an exploratory attempt we choose $l=1=m$ %and $l=2$ respectively 
 as  our initial  mode and draw the waveform of a massless scalar field perturbation around an rLQGBH in Fig.~\ref{instab}.  
 Analyzing the evolution profiles, we observe that the introduction of quantum parameter $A_\lambda$ does not significantly affect the evolution of outbursts and late-time tails. 
 However, it exerts a notable influence on the damping oscillation phase, that is, an increase of $A_\lambda$ leads to an accelerated oscillation and a  rapider decay of scalar fields. Therefore, in order to distinguish LQG from GR, i.e., an rLQGBH from a Kerr black hole, we need to study the QNM frequencies of the damping oscillation stage.
 In the subsequent section, we provide a detailed investigation on the influence of $A_\lambda$ on the  QNMs under scalar field perturbations.
 


    
    
%%%%%%%%%%%%%%%%%%%%%%%%%%%%%%%%%%%%%%%%%%%%%%%%%%%%%%    
\section{Quasi-normal modes}\label{sec:QNM}
%%%%%%%%%%%%%%%%%%%%%%%%%%%%%%%%%%%%%%%%%%%%%%%%%%%%%%
 If a scalar field meets the specific boundary conditions: Pure ingoing waves exist at the outer horizon, while pure outgoing waves exist at the spatial infinity, the characteristic complex frequencies of damping oscillations are just QNMs.
Owing to the  complicated boundary conditions around rLQGBHs, it is not feasible to obtain QNMs precisely through analytical methods. 
As a result, various numerical and semi-analytical methods have been adopted~\cite{Konoplya:2011qq, Kokkotas-Schmidt-1999, Li:2022kch, Berti:2007dg, Chandrasekhar:1975zza, Molina:2010fb, Franzin:2022iai, Iyer:1986nq, Seidel:1989bp}. 
Each method possesses its own advantages and disadvantages, so that the accuracy of results cannot be guaranteed if one relies solely on a single method. 
In this section, we employ three methods to compute QNMs and make comparisons among them in order to ensure the accuracy and reliability of our findings.

%%%%%%%%%%%%%%%%%%%%%%%%%%%%%%%%%%%%%%%%%%%%%%%%%%%%%%%%%%%%%%%%%%%%%%    
\subsection{Prony method}\label{sec:prony}
%%%%%%%%%%%%%%%%%%%%%%%%%%%%%%%%%%%%%%%%%%%%%%%%%%%%%%%%%%%%%%%%%%%%%%
In Sec.~\ref{sec:TD} we have presented the approach for obtaining time domain profiles of massless scalar field perturbations in rLQGBHs. Now we can extract QNMs from these profiles by the Prony method~\cite{Konoplya:2011qq,Berti:2007dg}.     
The main idea of this method is to fit the profile data with a superposition of damped exponents,
\begin{equation}\label{eq:prony}
    \Phi(t)\sim \sum_{j=1}^p C_j e^{-i\omega_jt},
\end{equation}
where $C_j$ is the amplitude coefficient of QNMs. In computations, we extract $(2p-1)$ equidistant points from the ringdown phase of a profile. 
These points are used to form one $p\times p$ matrix in order to calculate $\omega_j$. 
Here rows and columns of this matrix are composed of equidistant points with decreasing sequence numbers.
Among the $p$ $\omega_j's$, the one with the largest $|C_j|$ is dominant, which is our demanding QNM frequency.
Nevertheless, the Prony method has two sides: Advantage and disadvantage. The former is its high precision \cite{zhang2020object}, while the latter is its non-suitability for massive scalar fields. %restriction: Since our analysis of time domain waveforms in Sec. \ref{sec:TD} cannot  handle massive scalar fields, we cannot extract QNMs  from corresponding evolution profiles by the Prony method.
%the waveform data in a certain situation is missing, the corresponding QNMs cannot be obtained.For example, the (2+1)-dimensional simulation used  is not capable of handling the case involving a massive scalar field. Therefore we can not obtain the QNMs of a massive scalar field through .
%However, one of our main goals is investigating the ultralight scalar field.
Therefore, we have to ask for alternative methods to compute the QNMs of massive scalar field perturbations in the following sections.

 

    
\subsection{WKB method}
The WKB method is a semi-analytic technique for determining low-lying QNMs~\cite{Seidel:1989bp}. 
Assuming that a scalar field has the same symmetry as that of its background spacetime, we can make the following ansatz,
\begin{equation}\label{eq:Phi-SR}
    	\Phi(t,h,\theta,\varphi)=\int{e^{-i\omega t}\sum_{l,m}S_{lm}(\theta)R_{lm}(h)e^{im\varphi}}\dif\omega,
\end{equation}
where $h$ is defined by Eq.~(\ref{eq:h}) and $\omega$ denotes the characteristic frequency of scalar fields.
After separating variables, we obtain the angular equation,   
\begin{equation}\label{azimuthal}
    	\frac 1 {\mathrm{sin}\theta} \frac {\mathrm d} {{\mathrm d}\theta}\left[ \mathrm{sin}\theta\frac {\mathrm d} {{\mathrm d}\theta}S_{lm}(\theta)\right] 
    	+\left[ a^2\left( \omega^2-\mu^2\right) \mathrm{cos}^2\theta-\frac{m^2}{\mathrm{sin}^2\theta}+A_{lm}\right] S_{lm}(\theta)=0,
    \end{equation}
where $S_{lm}(\theta)$'s are spherical harmonics~\cite{Franzin:2022iai}, which reduce to $Y_{lm}$'s when $a^2(\omega^2-\mu^2)\cos^2\theta=0$, and $A_{lm}$ is angular eigenvalue, and also derive the radial equation, 
    \begin{equation}\label{radialh}
    	\epsilon \frac {\mathrm d}{{\mathrm d}h}\left[\epsilon \Delta\frac {\mathrm d}{{\mathrm d}h}R_{lm}(h)\right] +\left[ \frac{\tilde{K}^2}{\Delta}-\mu^2b^2-(A_{lm}-2am\omega+a^2\omega^2) \right] R_{lm}(h)=0,
    \end{equation}
  where $\tilde{K}=am-(b^2+a^2)\omega$.
If we introduce the following transformation,
    \begin{equation}\label{eq:transform}
        \Psi(y)=(b^2+a^2)^{1/2}R_{lm}(h),
    \end{equation}
we can change the radial equation of motion  to a Schr\"odinger-like one,
    \begin{equation}\label{WKB}
    	  \frac {\mathrm d^2}{{\mathrm d}y^2}\Psi(y) +\left[ \omega^2-V(y)\right]  \Psi(y)=0,
    \end{equation}
where $y$ is the tortoise coordinate determined by 
    \begin{equation}
        \frac{\mathrm d h}{{\mathrm d}y}=\frac{\epsilon\Delta}{b^2+a^2},
    \end{equation}
and $V(y)$ is the effective potential,
\begin{eqnarray}
%\begin{split}
V(y)&=&V_1+V_2,\nonumber \\
V_1&=&\frac{\Delta }{h(b^2+a^2)^4}\Big\{-96 A_\lambda^2 M_{\rm B}^4(M_{\rm B} - h) + a^4 h + 2 M_{\rm B} h^4 \nonumber \\
& &- 2 A_\lambda M_{\rm B}^2 h^2 (4 M_{\rm B} + 5 h) + 
  a^2 \left[h^2 (-4 M_{\rm B} + h) + 2 A_\lambda M_{\rm B}^2(8 M_{\rm B} + h)\right]\Big\},\nonumber \\
V_2&=&\frac{\Delta }{(b^2+a^2)^2}\left[\frac{\tilde{K}^2}{\Delta}-\mu^2b^2-(A_{lm}-2am\omega+a^2\omega^2)\right]. 
%\end{split}
\end{eqnarray}

Following Refs.~\cite{Franzin:2022iai,Konoplya:2019hlu} we choose the fourth-order WKB  approximation to numerically solve the radial equation because its relative error is very small. The QNM frequency $\omega$ can be obtained by the WKB formula~\cite{Iyer:1986nq, Konoplya:2003ii},
    \begin{equation}\label{WKB_formula}
    	\frac{i\left[ \omega^2-V(y_0)\right] }{\sqrt{-2V''(y_0)}}-\sum_{j=2}^{4}\Lambda_j=n+\frac 1 2,\qquad n=0,1,2,\cdots,
    \end{equation}
where the prime means the derivative with respect to the tortoise coordinate, $y_0$ is fixed by the condition, 
\begin{equation}\label{eq:V0}
    \frac{\mathrm d}{\mathrm dy}V(y)\Big|_{y=y_0}=0,
\end{equation}
and $\Lambda_j$ denotes higher (than one) order corrections dependent on $V(y_0)$ and its higher order derivatives.


A rotating black hole is more complicated than a static one because both the scalar potential $V(y)$ and the angular eigenvalue  $A_{lm}$ depend on frequency $\omega$ in the former case. 
For massless scalar field perturbations, we adopt the series expansion method  \cite{Seidel:1989bp} to solve Eq.~\eqref{WKB_formula} up to the order of $(a\omega)^6$. 
However, for massive scalar field perturbations, the angular eigenvalue $A_{lm}$ should be expanded~\cite{seidel1989comment} as a series of $(a\sqrt{\omega^2-\mu^2})$, where it contains only even-order terms,
\begin{equation}
    A_{lm}=\sideset{_{0}}{^{lm}_0}{\mathop f}
    +\sideset{_{0}}{^{lm}_2}{\mathop f}a^2(\omega^2-\mu^2)
    +\sideset{_{0}}{^{lm}_4}{\mathop f}a^4(\omega^2-\mu^2)^2
    +\cdots,
\end{equation}
and $\sideset{_{0}}{^{lm}_j}{\mathop f}$ is the $j$th order expansion coefficient.
By employing the expressions of $A_{lm}$ and $y_0$, we can also express both $V_0$ and its higher order derivatives as series of $(a\omega)$ up to order $(a\omega)^6$. 
Finally, we solve Eq.~\eqref{WKB_formula} numerically and determine the QNMs with given $a$, $\mu$, $n$, $l$, and $m$.    

The advantage of the WKB method lies in its polynomial formula with which we can obtain QNMs with high accuracy through simple numerical calculations. 
As discussed above, this method depends only on the first six terms of series expansions. However, these terms diverge in certain cases, which leads to a breakdown of the method. 
When we apply the WKB method in rLQGBHs, we shall discuss its scope of applicability in Sec.~\ref{sec:num-result}.
    
    
    
    
    
\subsection{Shooting method}\label{shooting}
The shooting method is a numerical technique \cite{Chandrasekhar:1975zza,Franzin:2022iai} for the calculation of QNMs in black hole physics. 
Its main idea lies in numerically integrating radial equation of motion from one  point near an event horizon to an intermediate point and also from the other point near the spatial infinity to this intermediate point,  and then we require that both the wave functions and their first derivatives obtained from the two sides are equal at the intermediate point. 

To achieve the goal, we analyze the asymptotic behaviors of the radial equation, Eq.~\eqref{radialh}. 
The outer event horizon $h=h_+$ and spatial infinity $h=h_\infty$ are two regular singularities, thus we can give the radial wave function $R(h)$ by two  series that are convergent in the range of $h_+<h<h_\infty$.
Near the outer horizon, the asymptotic formulation of radial wave function $R(h)$ reads
\begin{equation}\label{asym_horizon}
    R(h)\sim (h-h_+)^{\pm
    i\alpha},
\end{equation}
where 
\begin{equation}
    \alpha=\frac{a m-2M_{\rm B}\omega( h_+-3A_\lambda M_{\rm B})}{\epsilon_+ (h_+-h_-)}, \qquad \epsilon_+=\sqrt{1-\frac{8A_\lambda M_{\rm B}^2}{h_+^2}},
\end{equation}
$h_+$ and $h_-$ are determined by Eq.~\eqref{eq:horizon2} for rLQGBHs. Near the spatial infinity, it takes the form,
\begin{equation}
    	R(h)\sim \frac{1}{h}\mathrm e^{\pm qh}h^{\pm M_{\rm B}(\mu^2-2\omega^2)/q},\label{asym_infinity}
    \end{equation}
where
\begin{equation}
q=\sqrt{\mu^2-\omega^2}.
\end{equation} 
In Eq.~\eqref{asym_horizon} the plus and minus signs correspond to ingoing and outgoing waves, respectively. In contrast, the plus and minus signs correspond to outgoing and ingoing waves in Eq.~\eqref{asym_infinity}, respectively.
    
The QNMs are eigenvalues of wave equations satisfying specific boundary conditions, where only ingoing waves exist near event horizons and only outgoing waves at the spatial infinity. 
Therefore, we determine the asymptotic formulations of radial wave functions,
    \begin{equation}\label{eq:as_horizon}
    	R(h)\sim (h-h_+)^{ i\alpha},
    \end{equation}
 and 
    \begin{equation}\label{eq:as_spinfty}
    	R(h)\sim \frac{1}{h}\mathrm e^{ qh}h^{M_{\rm B}(\mu^2-2\omega^2)/q},
    \end{equation}
near the outer horizon and near the spatial infinity, respectively.
    
Now we can determine the QNMs with the shooting method in the range of  $h_+<h<h_\infty$. In the first step, we choose a QNM frequency  as our initial value, with which we can determine the angular eigenvalue $A_{lm}$ by using the Leaver method~\cite{leaver1985analytic}. In the second step, considering the asymptotic behavior Eq.~\eqref{eq:as_horizon} near the outer horizon, we integrate Eq.~\eqref{radialh} from the outer event horizon to an intermediate point, where this point is usually chosen with the maximum value of the potential.  
In the third step, considering the asymptotic behavior Eq.~\eqref{eq:as_spinfty} near the infinity, we integrate Eq.~\eqref{radialh} from the infinity to this intermediate point. 
In the final step, we require that the radial wave function $R(h)$ and its first derivative $R'(h)$ are continuous\footnotemark[1]\footnotetext[1]{Here ``continuous" means that the radial wave function $R(h)$ obtained in the second step equals that in the third step, and so does the first derivative of $R(h)$.} at the intermediate point and thus obtain a QNM frequency. Regarding this QNM frequency as the initial value for the next calculation, i.e. after an iterative process we at last get a stable QNM frequency.

The shooting method is very stable so that it can handle some situations in which the WKB method fails.
However, it is not easy in the shooting method to give an initial value of QNMs and to make sure that it does not deviate from the expected value of QNMs too large. Otherwise, the shooting method does not work well.
In practical applications, the first initial value will be fixed with the help of other numerical methods, and then the result from the previous iteration is regarded as the initial value for the next iteration.

%%%%%%%%%%%%%%%%%%%%%%%%%%%%%%%%%%%%%%%%%%%%%%%%%%%%%
\subsection{Numerical results}\label{sec:num-result}
%%%%%%%%%%%%%%%%%%%%%%%%%%%%%%%%%%%%%%%%%%%%%%%%%%%%%
 
%%%%%%%%%%%%%%%%%%%%%%%%%%%%%%%%%%%%%
\subsubsection{Comparison of accuracy among three methods}\label{sec:MPC}
%%%%%%%%%%%%%%%%%%%%%%%%%%%%%%%%%%%%%
With the above three methods, we are able to obtain the QNMs of scalar field perturbations around rLQGBHs. Here we start by presenting the QNMs of massless scalar field perturbations with a varying $A_\lambda$ in Tab.~\ref{tab:QNM1}, where two modes with $l=1=m$ and $l=2=m$ are chosen.
In the special case of $A_\lambda=0$, an rLQGBH reduces to a Kerr black hole, and our results are consistent with those computed by the Leaver method~\cite{konoplya2006stability}. 

Since the Prony method is completely a numerical calculation of waveform simulations, its accuracy and reliability are the highest among the three methods, so we take its results as the standard to measure the accuracy of the results from the other two methods.
Here, we define the relative error between data $D$ and standard data $SD$ as follows:
\begin{equation}
     {\rm Err}=\left|\frac{D-SD}{D}\right|,
\end{equation}
and display the relative errors between the QNMs obtained by the Prony method and those by the other two methods in brackets
of Tab.~\ref{tab:QNM1}, where the left shows the relative error of real parts, while the right that of imaginary parts. We can see that
the relative errors are always less than $2\%$, indicating that both the WKB method and shooting method exhibit high accuracy within the range of quantum parameter, $0\le A_\lambda \le 0.45$.
However, in the mode of $l=1=m$, the relative errors associated with the WKB method grow with an increase of quantum parameter $A_\lambda$, which implies that this method may not work well when other parameters, such as angular momenta, take a certain range.

In order to further study the scope of application of the WKB method and the impact of angular momentum $a$ on the QNMs of massless ({\em massive}) scalar field perturbations, we display in Tab.~\ref{tab:QNM2} the relationship between the QNMs and $a$, where $A_\lambda=0.1$ and $l=1=m$ are set. When $a\lesssim0.4$, the QNMs obtained from the three ({\em two}) methods are consistent. 
However, when $a\gtrsim0.4$, the discrepancy between the WKB method and the other two methods ({\em shooting method}) rapidly increases. 
The reason is that the series of $(a\omega)$ no longer converges in the WKB method  if $a\gtrsim0.4$, rendering the WKB method inapplicable. 
On the other hand, as shown in Tab. \ref{tab:QNM1}, the shooting method yields less relative errors than the 4th-order WKB, showing that the former exhibits higher precision than the latter.
Again considering the shooting method is much more efficient than the Prony method because the latter relies on complicated waveform simulations, we 
therefore prefer to apply the shooting method to the calculation of amplification factors in Sec.~\ref{sec:superradiance}.

\begin{table}[t]
\centering
\begin{tabular}{|l|c|c|c|}
\hline
\multicolumn{4}{|c|}{$l=1=m$}\\
\hline
$A_\lambda$ & Prony method&4th-order WKB&Shooting method\\ \hline
%精确值为0.301 045-0.097 547I
0&$0.3010-0.0975i $&$0.3011 - 0.0973 i (0.03\%, 0.20\%)$&$0.3010 - 0.0973 i (0.00\%, 0.20\%)$\\\hline
0.05&$ 0.3119-0.0998i$&$0.3119 - 0.0996 i (0.00\%, 0.20\%)$&$0.3118 - 0.0996 i (0.03\%, 0.20\%)$\\ \hline
0.10&$0.3242-0.1023i $&$0.3242 - 0.1022 i (0.00\%, 0.10\%)$&$0.3240 - 0.1023 i (0.06\%, 0.00\%)$\\\hline
0.15&$ 0.3382-0.1050i$&$0.3382 - 0.1049 i (0.00\%, 0.10\%)$&$0.3381 - 0.1053 i (0.03\%, 0.29\%)$\\\hline
0.20&$0.3544-0.1080i $&$0.3545 - 0.1077 i (0.03\%, 0.28\%)$&$0.3546 - 0.1082 i (0.06\%, 0.18\%)$\\\hline
0.25&$0.3735-0.1114i $&$0.3738 - 0.1108 i (0.08\%, 0.54\%)$&$0.3738 - 0.1112 i (0.08\%, 0.18\%)$\\\hline
0.30&$0.3964-0.1150i $&$0.3968 - 0.1140 i (0.10\%, 0.87\%)$&$0.3962 - 0.1147 i(0.05\%, 0.26\%)$\\\hline
0.35&$0.4245-0.1190i $&$0.4252 - 0.1174 i (0.16\%, 1.34\%)$&$0.4243 - 0.1193 i(0.05\%, 0.25\%)$\\\hline
0.40&$0.4604-0.1229i $&$0.4612 - 0.1207 i (0.17\%, 1.79\%)$&$0.4608 - 0.1227 i(0.09\%, 0.16\%)$\\\hline
0.45&$0.5082-0.1256i $&$0.5093 - 0.1232 i (0.22\%, 1.91\%)$&$0.5079 - 0.1258 i(0.06\%, 0.16\%)$\\\hline
\end{tabular}
\begin{tabular}{|l|c|c|c|}
\hline
\multicolumn{4}{|c|}{$l=2=m$}\\
\hline
$A_\lambda$ & Prony method&4th-order WKB&Shooting method\\ \hline
0   &$0.4995-0.0967i $&$0.4995 - 0.0966  i(0.00\%, 0.10\%)$&$0.4994 - 0.0967 i(0.02\%, 0.00\%)$\\\hline
0.05&$0.5179-0.0989i $&$0.5179 - 0.0989 i (0.00\%, 0.00\%)$&$0.5180 - 0.0990 i(0.02\%, 0.10\%)$\\ \hline
0.10&$0.5388-0.1015i $&$0.5388 - 0.1014  i(0.00\%, 0.10\%)$&$0.5389 - 0.1014 i(0.02\%, 0.10\%)$\\\hline
0.15&$0.5627-0.1042i $&$0.5627 - 0.1042  i(0.00\%, 0.00\%)$&$0.5626 - 0.1042 i(0.02\%, 0.00\%)$\\\hline
0.20&$0.5904-0.1073i $&$0.5904 - 0.1073 i(0.00\%, 0.00\%)$&$0.5903 - 0.1074 i(0.02\%, 0.09\%)$\\\hline
0.25&$0.6231-0.1108i $&$0.6231 - 0.1107 i(0.00\%, 0.09\%)$&$0.6232 - 0.1107 i(0.02\%, 0.09\%)$\\\hline
0.30&$0.6626-0.1146i $&$0.6626 - 0.1145 i(0.00\%, 0.09\%)$&$0.6624 - 0.1146 i(0.03\%, 0.00\%)$\\\hline
0.35&$0.7114-0.1187i$&$0.7115 - 0.1185 i(0.01\%, 0.17\%)$&$0.7116 - 0.1187 i(0.03\%, 0.00\%)$\\\hline
0.40&$0.7743-0.1229i$&$0.7743 - 0.1225 i (0.00\%, 0.32\%)$&$0.7743 - 0.1229 i(0.00\%, 0.00\%)$\\\hline
0.45&$0.8588-0.1255i$&$0.8587 - 0.1250 i(0.01\%, 0.40\%)$&$0.8588 - 0.1257 i(0.00\%, 0.16\%)$\\\hline
\end{tabular}
\caption{QNMs for the modes of $l=1=m$ and $l=2=m$, respectively, where  $a=0.1$, $M_{\rm B}=1$, and $\mu=0$ are set.}
\label{tab:QNM1}
\end{table}


\begin{table}
\centering
\fontsize{11pt}{12pt}\selectfont
\begin{tabular}{|l|c|c|c|c|c|}
\hline
&\multicolumn{3}{|c|}{$\mu=0$}
&\multicolumn{2}{|c|}{$\mu=0.1$}
\\
\hline
$a$&Prony method &4th-order WKB&Shooting method&4th-order WKB&Shooting method\\ \hline
0&$ 0.3145-0.1024i$&$0.3146 - 0.1022 i $&$0.3144 - 0.1022 i$&$0.3187 - 0.0998 i$&$0.3186 - 0.0996 i$\\\hline
0.1&$0.3242-0.1023i$&$ 0.3242 - 0.1022 i$&$0.3240 - 0.1023 i$&$0.3281 - 0.0999 i$&$0.3278 - 0.0999 i$\\ \hline
0.2&$0.3351-0.1020i$&$0.3351 - 0.1019 i $&$0.3349 - 0.1022 i$&$0.3388 - 0.0998 i$&$0.3385 - 0.1000 i$\\\hline
0.3&$0.3474-0.1014i$&$ 0.3474 - 0.1012 i$&$ 0.3475-0.1016i$&$0.3509 - 0.0993 i $&$0.3509 - 0.0998 i$\\\hline
0.4&$0.3617-0.1004i$&$0.3617 - 0.0995 i $&$0.3619 - 0.1004 i$&$0.3649 - 0.0978 i$&$0.3651 - 0.0989 i$\\\hline
0.5&$0.3785-0.0987i$& $0.3782 - 0.0939 i$&$0.3785 - 0.0986 i$&$0.3808 - 0.0923 i$&$0.3816 - 0.0972 i$\\\hline
0.6&$0.3987-0.0962i$&$ 0.3909 - 0.0761 i$&$0.3986 - 0.0961 i$&$0.3924 - 0.0752 i$&$0.4013 - 0.0949 i$\\\hline
0.7&$0.4242-0.0921i$&$0.3849 - 0.0530 i$&$0.4242 - 0.0921 i$&$0.3860 - 0.0530 i$&$0.4265 - 0.0912 i$\\\hline
0.8&$0.4585-0.0850i$&$0.3693 - 0.0380 i$&$0.4585 - 0.0849 i$&$0.3697 - 0.0381 i$&$0.4605 - 0.0843 i$\\\hline
0.9&$0.5116-0.0706i$&$0.3535 - 0.0292 i$&$0.5116-0.0706 i$&$0.3529 - 0.0292 i$&$0.5129 - 0.0703 i$\\\hline
\end{tabular}
\caption{QNMs for the mode of $l=1=m$ with $A_\lambda=0.1$, $a=0.1$, $M_{\rm B}=1$, $\mu=0$ or $\mu=0.1$. Note that the Prony method is unsuitable for massive scalar field perturbations.}
\label{tab:QNM2}
\end{table}

%%%%%%%%%%%%%%%%%%%%%%%%%%%%%%%%%%%%%%%%%%%%%%%%%%%%%%%%%%%%%%%%%%%%%%%%%
\subsubsection{Data of quasi-normal modes}
%%%%%%%%%%%%%%%%%%%%%%%%%%%%%%%%%%%%%%%%%%%%%%%%%%%%%%%%%%%%%%%%%%%%%%%%%
\begin{figure}[t]
\centering
    	\subfigure[]{
    		\begin{minipage}[t]{0.4\linewidth}
    			\centering\label{fig:QNMR-A-a-2D}
    			\includegraphics[width=1\linewidth]{figure/l1m1mu00AOmR.jpg}
    		\end{minipage}
    	}
\subfigure[]{
\begin{minipage}[t]{0.4\linewidth}
\centering\label{fig:QNMR-A-a-3D}
\includegraphics[width=1\linewidth]{figure/l1m1mu00AOmR3D.jpg}
\end{minipage}
    	}
\subfigure[]{
\begin{minipage}[t]{0.4\linewidth}
\centering\label{fig:QNMI-A-a-2D}
\includegraphics[width=1\linewidth]{figure/l1m1mu00AOmI.jpg}
\end{minipage}
    	}
\subfigure[]{
\begin{minipage}[t]{0.4\linewidth}
\centering\label{fig:QNMI-A-a-3D}
\includegraphics[width=1\linewidth]{figure/l1m1mu00AOmI3D.jpg}
\end{minipage}
    	}
\caption{QNM frequencies $\omega$ as a function of regularization parameter $A_\lambda$ with a varying angular momentum, $a=0$, $0.3$, $0.6$, $0.9$, and $0.99$, where $M_{\rm B}=1$ and $\mu=0$ are set, and the mode of  $l=1=m$ is chosen. In diagram (c), if we take $a=0.60$ as an example, we can see that  $|\omega_{\rm I}|$ is smaller than that of the Kerr case ($A_\lambda=0$) when $A_\lambda>0.38$, and that it is larger than that of the Kerr case ($A_\lambda=0$) when $A_\lambda<0.38$.}
\label{fig:QNM-A-a}
    \end{figure}

Our analyses reveal that the quantum parameter $A_\lambda$ has a significant impact on the QNMs of massless scalar field perturbations
in Tab.~\ref{tab:QNM1}, where we present data on how the QNMs change with respect to the quantum parameter $A_\lambda$ for both $l=1=m$ and $l=2=m$ modes when $a=0.1$ and $M_{\rm B}=1$ are set.
In the two modes, the real parts $\omega_{\rm R}$ grow with an increase of the quantum parameter $A_\lambda$. 
This indicates that an increase of $A_\lambda$ amplifies the oscillation frequencies of scalar fields. For example, 
in the mode of $l=1=m$, the real part corresponding to $A_\lambda=0.45$ is 1.69 times larger than that corresponding to $A_\lambda=0$.
% which shows that the time interval between adjacent peaks on the damping oscillation waveform under $A_\lambda=0.45$ is only $59\%$ of that under $A_\lambda=0$.
Furthermore, as $A_\lambda$ increases, the absolute value of imaginary parts $|\omega_{\rm I}|$ also increases, indicating a faster dissipation of waveforms.
For instance, in the mode of $l=1=m$, $|\omega_{\rm I}|$ corresponding to $A_\lambda=0.45$ is 1.29 times larger than that corresponding to $A_\lambda=0$.
%which shows that the slope between adjacent peaks on the damping oscillation waveform under $A_\lambda=0.45$ is 1.29 times that under $A_\lambda=0$.
These results highlight the significance of quantum parameter $A_\lambda$ in influencing the behaviors of QNMs of massless scalar field perturbations. 

Now we present the influence of $A_\lambda$ on the QNMs of massless scalar field perturbations under different values of $a$ in Fig.~\ref{fig:QNM-A-a}, where the maximum value of $A_\lambda$ is determined by Eq.~\eqref{max}.
Here we take the mode of  $l=1=m$ as an example and set $M_{\rm B}=1$.
As shown in Figs.~\ref{fig:QNMR-A-a-2D} and \ref{fig:QNMR-A-a-3D}, the real parts of QNMs always increase with an increase of $A_\lambda$ under a fixed $a$, where
the largest real part is $\omega_{\rm R}=0.7828$, located at $a=1$ and $A_\lambda=0.125$, and it corresponds to the extreme configuration of rLQGBHs with the greatest angular momentum, $a_{\rm Max}=1$.
Moreover, as shown in Figs.~\ref{fig:QNMI-A-a-2D} and \ref{fig:QNMI-A-a-3D}, $A_\lambda$ has diverse effects on $|\omega_{\rm I}|$ under different values of $a$, which can be divided into three categories:
\begin{itemize}
    \item When $0<a\leq0.48$, $|\omega_{\rm I}|$ increases at first and then decreases when $A_\lambda$ grows. 
    Within the entire range of $A_\lambda$, $|\omega_{\rm I}|$ is always larger than that corresponding to $A_\lambda=0$. 
    From a physical perspective, a non-vanishing $A_\lambda$ gives rise to a faster decay of massless scalar fields than the case of  vanishing $A_\lambda$, thereby making the spacetime more unstable.
    Note that an rLQGBH reaches its  extreme configuration when $A_\lambda$ takes its maximum value, see Eq.~\eqref{max}, and it reduces to a Kerr black hole when $A_\lambda$ equals zero. 
    When $a$ increases, the gap between $|\omega_{\rm I}|$ of an extreme rLQGBH and that of a Kerr black hole is getting smaller and smaller, and finally it disappears when $a=0.48$.
    \item When $0.48<a\leq0.82$, $|\omega_{\rm I}|$ also initially increases and then  decreases when $A_\lambda$ grows. 
    However, when $A_\lambda$ exceeds some specific value that depends on $a$, $|\omega_{\rm I}|$ is smaller than that of the Kerr case. For instance,
    we show that this specific value equals $0.38$ when $a=0.60$ or $0.23$ when $a=0.80$ in Fig.~\ref{fig:QNMI-A-a-2D}.
    Therefore, $A_\lambda$ plays the role in promoting the stability of spacetime when $A_\lambda$ is larger than this specific value,
    but it plays the role in diminishing the stability of spacetime when $A_\lambda$ is smaller than this specific value.
    Additionally, this specific value becomes small when $a$ increases, and it reaches zero when $a=0.82$.
    
    \item When $0.82<a\leq1.00$, $|\omega_{\rm I}|$ decreases monotonically when $A_\lambda$ grows, implying that
    $A_\lambda$ plays the role in promoting the stability of spacetime.
     When $a=1$ and $A_\lambda=0.125$, $|\omega_{\rm I}|$ reaches the minimum value, $|\omega_{\rm I}|_{\rm Min}=0.0202$.
\end{itemize}

In addition, we present the influence of $A_\lambda$ on the QNMs of massless scalar field perturbations under different values of $a$ for the mode of $l=2=m$ in Fig.~\ref{fig:QNM-A-a-2}, where the maximum value of $A_\lambda$ is also determined by Eq.~\eqref{max}.
In general, the influence of $A_\lambda$ on QNMs in the mode of $l=2=m$ is similar to that in the mode of $l=1=m$, i.e., the former differs from the latter just in  numerical differences.
For the real parts of QNMs, the former is significantly larger than the latter.
But for the absolute value of imaginary parts, the difference between the two modes is very small.
Similarly, for the mode of $l=2=m$, the influence of $A_\lambda$ on $|\omega_{\rm I}|$ can also be divided into three categories for a varying $a$: $0<a\leq0.49$, $0.49<a\leq0.84$, and $0.84<a\leq1.00$.
From Fig.~\ref{fig:QNM-A-a-2} it is clear that the relations between $\omega_{\rm R}$ ($|\omega_{\rm I}|$) and $A_\lambda$ are quite similar in the two modes.

\begin{figure}[t]
\centering
    	\subfigure[]{
    		\begin{minipage}[t]{0.4\linewidth}
    			\centering\label{fig:QNMR-A-a-2-2D}
    			\includegraphics[width=1\linewidth]{figure/l2m2mu00AOmR.jpg}
    		\end{minipage}
    	}
\subfigure[]{
\begin{minipage}[t]{0.4\linewidth}
\centering\label{fig:QNMR-A-a-2-3D}
\includegraphics[width=1\linewidth]{figure/l2m2mu00AOmR3D.jpg}
\end{minipage}
    	}
\subfigure[]{
\begin{minipage}[t]{0.4\linewidth}
\centering\label{fig:QNMI-A-a-2-2D}
\includegraphics[width=1\linewidth]{figure/l2m2mu00AOmI.jpg}
\end{minipage}
    	}
\subfigure[]{
\begin{minipage}[t]{0.4\linewidth}
\centering\label{fig:QNMI-A-a-2-3D}
\includegraphics[width=1\linewidth]{figure/l2m2mu00AOmI3D.jpg}
\end{minipage}
    	}
\caption{QNM frequencies $\omega$ as a function of quantum parameter $A_\lambda$ with a varying angular momentum, $a=0$, $0.3$, $0.6$, $0.9$, and $0.99$, where $M_{\rm B}=1$ and $\mu=0$ are set, and the mode of $l=2=m$ is chosen. In diagram (c), if we take $a=0.60$ as an example, we can see that  $|\omega_{\rm I}|$ is smaller than that of the Kerr case ($A_\lambda=0$) when $A_\lambda>0.37$, and that it is larger than that of the Kerr case ($A_\lambda=0$) when $A_\lambda<0.37$.}
\label{fig:QNM-A-a-2}
    \end{figure}


In Fig.\ref{mu_omega}, we demonstrate the relationship between the QNMs and the quantum parameter $A_\lambda$ under a varying scalar field mass $\mu$. 
It can be observed for a fixed $A_\lambda$ that the presence of $\mu$ leads to an increase of the real parts of $\omega$ but a decrease of the absolute value of imaginary parts. 
This implies that the massive scalar field perturbations around rLQGBHs oscillate  faster but decay more slowly compared to the massless case, and that the mass $\mu$ plays the role in promoting the stability of spacetime.
When $A_\lambda=0$, i.e., in Kerr black holes $\mu$ has the greatest influence on the QNMs. 
When $A_\lambda$ gradually increases, the influence of $\mu$ gradually decreases. 
Therefore, a non-vanishing $A_\lambda$ makes $\mu$ have a weaker effect on promoting the spacetime stability than a vanishing $A_\lambda$  does.
   
\begin{figure}[t]
\centering
\subfigure[]{\begin{minipage}[t]{0.4\linewidth}
\centering
\includegraphics[width=1\linewidth]{figure/Remass}
\end{minipage}
    	}
\subfigure[]{
\begin{minipage}[t]{0.4\linewidth}
\centering
\includegraphics[width=1\linewidth]{figure/Immass}
\end{minipage}
    	}
\subfigure[]{
\begin{minipage}[t]{0.4\linewidth}
\centering
\includegraphics[width=1\linewidth]{figure/Remass2}
\end{minipage}
    	}
\subfigure[]{
\begin{minipage}[t]{0.4\linewidth}
\centering
\includegraphics[width=1\linewidth]{figure/Immass2}
\end{minipage}
    	}
\caption{QNM frequencies $\omega$ as a function of quantum parameter $A_\lambda$ with different masses of scalar fields, $\mu=0$, $0.1$, and $0.2$, where $a=0.1$, $M_{\rm B}=1$, and $n=0$ are set, the upper two diagrams correspond to the mode of $l=1=m$, and the lower two diagrams the mode of $l=2=m$.}
\label{mu_omega}
\end{figure}
    
    
%%%%%%%%%%%%%%%%%%%%%%%%%%%%%%%%%%%%%%%%%%%%%%%%%%%%%%%%%%%%%%%%%%%%%%%%    
 \section{Superradiance}\label{sec:superradiance}
 %%%%%%%%%%%%%%%%%%%%%%%%%%%%%%%%%%%%%%%%%%%%%%%%%%%%%%%%%%%%%%%%%%%%%%%
When a free scalar field is incident on a black hole, the incident wave will be decomposed into two parts, a reflected wave and a transmitted wave, due to the scattering effect of potential barriers near the black hole.
The boundary conditions of this process include pure ingoing wave at the event horizon and both ingoing wave and outgoing wave at the spatial infinity, where the ingoing wave at the event horizon  corresponds to the transmitted wave, while the ingoing wave and outgoing wave at the spatial infinity correspond to the incident wave and reflected wave, respectively.
The energy of reflected waves is greater than the energy of incident waves when the frequency $\omega$ of incident scalar fields satisfies~\cite{Yang:2022yvq} the following conditions:
\begin{equation}
     m\Omega_{\rm H}>\omega>\mu, \label{instability}
\end{equation}
where 
\begin{equation}\label{eq:angular-vec}
    \Omega_{\rm H}=\frac{a}{h_+^2-6A_\lambda M_{\rm B}^2+a^2}
\end{equation}
is the angular velocity of scalar particles at the outer event horizon $h_+$ determined by Eq.~\eqref{eq:horizon2} for rLQGBHs.
It is worth noting that the frequency $\omega$ of scalar fields is always real owing to the special boundary conditions of scattering processes.
This phenomenon of energy amplification is called~\cite{Brito:2015oca} superradiance.
Next we study how the energy amplification factor is affected by the quantum parameter $A_\lambda$.
 
%When a wave with a specific frequency and angular momentum approaches a rotating black hole, it can undergo a scattering process called superradiant scattering. In this process, the wave can extract energy from the black hole's rotational energy and be amplified, resulting in an exponential growth in amplitude. This is superradiance. It occurs due to the existence of a ergoregion near the black hole's event horizon, which allows for energy extraction.
    
    
    
    
%%%%%%%%%%%%%%%%%%%%%%%%%%%%%%%%%%%%%%%%%%%%%%%%%%%%%%%%%%%%%%%%
   \subsection{Numerical method} 
%%%%%%%%%%%%%%%%%%%%%%%%%%%%%%%%%%%%%%%%%%%%%%%%%%%%%%%%%%%%%%%%
In terms of the boundary conditions  mentioned above, the radial equation of motion  Eq.~\eqref{radialh} takes the asymptotic solution at the spatial infinity as follows:
\begin{equation}
   	R(h)\sim \mathscr{I} \frac{1}{h}\mathrm e^{- qh}h^{ -M_{\rm B}(\mu^2-2\omega^2)/q}+ \mathscr{R}\frac{1}{h} \mathrm e^{ qh}h^{ M_{\rm B}(\mu^2-2\omega^2)/q},\label{super_infinity}
   \end{equation}
where $\mathscr{I}$ and $\mathscr{R}$ stand for the incident and  reflection amplitudes, respectively. Moreover,
the asymptotic wave function near the outer horizon $h_+$ reads 
   \begin{equation}
   	R(h)\sim \mathscr{T}(h-h_+)^{i\alpha},
   \end{equation}
where $\mathscr{T}$ is the transmission amplitude. 
Then the amplification factor is given~\cite{Brito:2015oca} by 
\begin{equation}\label{eq:Z}
   	Z_{lm}=\frac{\mathrm dE_{\rm out}}{\mathrm dE_{\rm in}}=\left| \frac{\mathscr R}{\mathscr I}\right| ^2-1.
\end{equation} 
Following the shooting method outlined in Sec.~\ref{shooting}, we establish the relationship among $\mathscr{I}$, $\mathscr{R}$, and $\mathscr{T}$, and thus give the amplification factor of superradiance. 
%%%%%%%%%%%%%%%%%%%%%%%%%%%%%%%%%%%%%%%%%%%%%%%%%%%%%%%%%%%%%%%%%%
\subsection{Results}
%%%%%%%%%%%%%%%%%%%%%%%%%%%%%%%%%%%%%%%%%%%%%%%%%%%%%%%%%%%%%%%%%%

In Fig.~\ref{fig:rLQGBH_shooting_superradiance} we present the relations between the amplification factor and $\omega$ under a varying $A_\lambda$ for a massless incident particle and a massive one with mass $\mu=0.2$, respectively.
In the region where the superradiance effect appears, the energy amplification factor always increases at first and then decreases as the particle frequency $\omega$ grows.
On the boundaries of the region, $\omega=\mu$ and $\omega=m\Omega_\Hor$, see Eq.~(\ref{instability}) with $M_{\rm B}=1$, the amplification factor is always zero.
Moreover, the amplification factor decreases as the particle mass $\mu$ increases, indicating that the superradiance effect is inhibited by particle mass.

The influence of quantum parameter $A_\lambda$ on the amplification factor varies with the particle frequency $\omega$. 
When the particle frequency $\omega$ is low, a small $A_\lambda$ leads to an amplification factor that is larger than that led by a big $A_\lambda$.
The specific situation can be observed in the small picture of  Fig.~\ref{fig:rss-a}, where 
an increase of $A_\lambda$ restrains the superradiance effect in the region of small $\omega$. 
However, as $\omega$ gradually increases, the amplification factor associated with a small $A_\lambda$ is gradually surpassed by the amplification factor associated with a big $A_\lambda$. 
Ultimately, a larger $A_\lambda$ results in a greater peak of amplification factors.
On the whole, an increase of $A_\lambda$ plays the role in promoting the superradiance effect.

\begin{figure}[htbp]
   	\centering
   	\subfigure[]{
   		\begin{minipage}[t]{0.4\linewidth}
   			\centering \label{fig:rss-a}
   			\includegraphics[width=1\linewidth]{figure/superradiancemu0.jpg}
   		\end{minipage}
   	}
   	\subfigure[]{
   		\begin{minipage}[t]{0.4\linewidth}
   			\centering \label{fig:rss-b}
   			\includegraphics[width=1\linewidth]{figure/superradiancemu02.jpg}
   		\end{minipage}
   	}
\caption{Amplification factor with respect to the frequency of scalar fields for the mode of $l=1=m$ in rLQGBHs,   where $n=0$, $M_{\rm B}=1$, and $a=0.9$ are set, and the left diagram corresponds to the case of massless scalar fields, $\mu=0$, and the right one to the case of massive scalar fields with mass $\mu=0.2$.}
\label{fig:rLQGBH_shooting_superradiance}
\end{figure}


%%%%%%%%%%%%%%%%%%%%%%%%%%%%%%%%%%%%%%%%%%%%%%%%%%%%%%%%%%%%%%%%%%
\section{Conclusion}\label{sec:con}
%%%%%%%%%%%%%%%%%%%%%%%%%%%%%%%%%%%%%%%%%%%%%%%%%%%%%%%%%%%%%%%%%%
In the present work, we investigate the scalar field perturbations in the background of rLQGBHs. 
In order to analyze the time domain evolution of scalar fields around rLQGBHs, we employ the hyperbolic foliation Eq.~\eqref{hyper} to overcome numerical issues at the boundaries and obtain the time domain evolution profiles. By comparing the evolution profiles of scalar fields under different quantum parameter $A_\lambda$, we observe that the introduction of $A_\lambda$ has no impact on the outburst and late-time tail stages, but affects the damping oscillation stage significantly.

In order to gain a deeper understanding of the influence of $A_\lambda$ on scalar field evolution, we extract the QNMs of scalar field perturbations from damping oscillations by  the Prony method, the 4th-order WKB method and the shooting method. 
We point out that the Prony method has the highest precision among the three methods, but the shooting method is the most efficient in numerical simulations. 
%To validate the accuracy of the Prony method and compute the QNMs of the massive scalar field, we introduce  the WKB method and the shooting method. 
Moreover, we find that  the influence of scalar field mass $\mu$ on the QNMs becomes more pronounced for a larger $A_\lambda$. Further, we investigate the impact of $A_\lambda$ on the QNMs for different angular momentum $a$. Interestingly, we observe that the effect of $A_\lambda$ on the QNMs varies with different values of angular momenta. %We demonstrate that the regular parameter indeed significantly affects the QNMs of the black hole.

We also analyze the impact of $A_\lambda$ on the superradiance effect in rLQGBHs. We find that $A_\lambda$ expands the frequency range of superradiance effects  and enhances the efficiency of superradiant amplification  significantly  when $\omega\sim m\Omega_{\rm H}$.

Based on our investigations on QNMs and superradiance, we conclude that the quantum parameter plays an important role in determining the dynamics of ultralight massive scalar fields.  Our results provide some new understanding of rLQGBHs. Considering that the metric of rLQGBHs can also describe  rotating LQG wormholes, we plan to analyze the dynamical behaviors of rotating LQG wormholes. The issue probably lies in solving the radial equation of motion under the special boundary conditions  of rotating LQG wormholes. We leave it in our future research.

\section*{Acknowledgments}
Y-GM would like to thank Emmanuele Battista, Stenfan Fredenhagen, and Harold Steinacker for the warm hospitality during his stay at University of Vienna. We would also like to thank Shao-Jun Zhang for his valuable advice in our calculation of waveforms. This work was supported in part by the National Natural Science Foundation of China under Grant No.\ 12175108.
%The authors would like to thank  C. Lan for useful discussions.
%This work was supported in part by the National Natural Science Foundation of China under grant Nos. 11675081 and 12175108.
%{\color{red}The authors would like to thank the anonymous referees for the helpful comments that improve this work greatly.}













%\newpage
%\section*{Appendix}
%\appendix














%%%%%%%%%%%%%%%%%%%%%%%%%%%%


%\bibliographystyle{utphys}
\bibliographystyle{unsrturl}
\bibliography{references}
%%%%%%%%%%%%%%%%%%%%%%%%%%%%

\end{document}
