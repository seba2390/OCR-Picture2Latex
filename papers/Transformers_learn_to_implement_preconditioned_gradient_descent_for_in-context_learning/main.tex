\documentclass{article}
\usepackage[utf8]{inputenc} % allow utf-8 input
\usepackage[T1]{fontenc}    % use 8-bit T1 fonts


% if you need to pass options to natbib, use, e.g.:
%     
\PassOptionsToPackage{round}{natbib}
% before loading neurips_2023


% ready for submission
\usepackage[final]{neurips_2023}




\usepackage{titletoc}
\usepackage[page, header, toc, page]{appendix} % MAKE SURE THIS IS LOADED BEFORE hyperref PACKAGE!

\usepackage[usenames,dvipsnames]{xcolor}   % NEEDS to be first!
\usepackage{hyperref}       % hyperlinks

\definecolor{darkblue}{rgb}{0.0,0.0,0.65}
\definecolor{darkred}{rgb}{0.68,0.05,0.0}
\definecolor{darkgreen}{rgb}{0.0,0.29,0.29}
\definecolor{darkpurple}{rgb}{0.47,0.09,0.29}
\hypersetup{
colorlinks = true,
citecolor  = darkblue,
linkcolor  = darkred,
filecolor  = darkblue,
urlcolor   = darkblue,
}


\usepackage{url}            % simple URL typesetting
\usepackage{booktabs}       % professional-quality tables
\usepackage{amsfonts}       % blackboard math symbols
\usepackage{nicefrac}       % compact symbols for 1/2, etc.
\usepackage{microtype}      % microtypography
%\usepackage{xcolor}         % colors
\usepackage{csquotes}

\usepackage{subcaption}
\usepackage{graphicx}
\usepackage{enumitem}
\usepackage{amsmath} 


\DeclareMathOperator{\E}{\mathbb{E}}

\usepackage{amsmath,amsthm, amssymb, bbm,  color, natbib,mathtools}

\newcommand{\F}{\mathcal{F}}
\newcommand{\N}{\mathcal{N}}
\renewcommand{\P}{\mathcal{P}}
\renewcommand{\L}{\mathcal{L}}

\newcommand{\R}{\mathbb{R}}

\renewcommand{\S}{{\mathcal S}}

\newtheorem{theorem}{Theorem} 
\newtheorem{definition}{Definition}
\newtheorem{lemma}{Lemma}
\newtheorem{corollary}[theorem]{Corollary}
\newtheorem{proposition}{Proposition}
\newtheorem{remark}{Remark}
\newtheorem{conjecture}{Conjecture}
\newtheorem{assumption}{Assumption}
\newtheorem{fact}[theorem]{Fact}
\newtheorem{condition}{Condition}
\newtheorem{setting}{Setting}
\newtheorem{example}{Example}

\usepackage{mdframed}
\usepackage{algorithmicx}


\usepackage{tikz}
\usetikzlibrary{positioning}

\usepackage{float}
\usepackage{wrapfig}

\def\lemmaautorefname{Lemma} 
\def\sectionautorefname{Section}
\def\subsectionautorefname{Subsection}
\def\appendixautorefname{Appendix}




\DeclareMathOperator{\Dist}{Dist} 



\newcommand{\softatt}{ \mathrm{Attn}^{\sf smax}}
\newcommand{\att}{\mathrm{Attn}}
\newcommand{\atth}{\mathrm{Attn}^\sigma} 
\newcommand{\relu}{\sigma}

\renewcommand{\aa}{M}
\newcommand{\bb}{A}
\newcommand{\bbb}{a} 
\newcommand{\cc}{b}
\newcommand{\ttbb}{\widetilde{\bb}}

\newcommand{\ttcc}{\widetilde{\cc}}

\newcommand{\wstar}{w_\star}
\newcommand{\twstar}{\widetilde{w}_\star}
\newcommand{\tz}[1]{{z^{(#1)}}}
\newcommand{\tx}[1]{x^{(#1)}}
\newcommand{\ttx}[1]{\widetilde{x}^{(#1)}}
\newcommand{\ty}[1]{y^{(#1)}}
\newcommand{\NF}{{\sf TF}} 
\newcommand{\ttf}{f_{\sf lower}} 

\newcommand{\MM}{\mathsf{G}}
\newcommand{\MMR}{\mathsf{G}^{\sf relu}}

\newcommand{\ind}[1]{\mathbbm{1}\lrbb{#1}}
\newcommand{\indi}{\mathbb{I}} 

\newcommand{\emphh}[1]{\textbf{\emph{#1}}}
\DeclareMathOperator{\tr}{Tr} 
\newcommand{\smax}{\mathrm{softmax}}
\newcommand{\inp}[2]{\left \langle #1,#2\right\rangle}
\newcommand{\inpp}[2]{ \langle #1,#2\rangle}
\newcommand{\wgd}{w^{\sf gd}} 
\newcommand{\pp}{P}
\newcommand{\qq}{Q}
\newcommand{\setSI}{\mathcal{S}_{\text{Id}}}
\newcommand{\setSD}{\mathcal{S}_{\text{Diag}}}
\newcommand{\setW}{\mathcal{W}}

\newcommand{\setWQ}{\mathcal{W}^Q}
\newcommand{\setSIQ}{\mathcal{S}^Q_{\text{Id}}}
\newcommand{\setSDQ}{\mathcal{S}^Q_{\text{Diag}}}
\newcommand{\dist}{\mathcal{D}}

\newcommand*\lin[1]{\left\langle #1 \right\rangle}

\newcommand*\lrb[1]{\left[ #1 \right]}
\newcommand*\lrn[1]{\left\| #1 \right\|}
\newcommand*\lrp[1]{\left( #1 \right)}
\newcommand*\lrbb[1]{\left\{ #1 \right\}}

\renewcommand\th{{\tilde{h}}}
\newcommand\jA{{\mathcal{J}_{\th}}}
\newcommand\tf{{\tilde{f}_\delta}}
\newcommand\cd{{c_\delta}}
\newcommand{\trelu}{\widetilde{\mathsf{relu}}_\delta}


\newcommand{\vectornorm}[1]{\left\|#1\right\|}
\newcommand{\norm}[1]{\left\| #1 \right\|}

\newcommand{\e}{\mathbf{e}}

\newcommand{\model}{C}

\newcommand{\diag}{\mathrm{diag}} 


\newcommand\numberthis{\addtocounter{equation}{1}\tag{\theequation}}


\newcommand*\at[2]{\left.#1\right|_{#2}}
\newcommand{\US}{U_\Sigma}

\makeatletter
\newcommand{\ostar}{\mathbin{\mathpalette\make@circled\star}}
\newcommand{\make@circled}[2]{%
\ooalign{$\m@th#1\smallbigcirc{#1}$\cr\hidewidth$\m@th#1#2$\hidewidth\cr}%
}
\newcommand{\smallbigcirc}[1]{%
\vcenter{\hbox{\scalebox{0.77778}{$\m@th#1\bigcirc$}}}%
}
\makeatother





\newcommand{\suv}[1]{\textcolor{green!70!blue!80}{[SS: \it #1}]}
\newcommand{\xc}[1]{{\color{black!2!green} [Xiang: #1]}}
\newcommand{\hlt}[1]{{\color{black!2!green} {\boldsymbol{#1}}}}

\mathtoolsset{showonlyrefs}



\title{Transformers learn to implement preconditioned gradient descent for in-context learning}


\author{ Kwangjun Ahn\thanks{Equal contribution, alphabetical order.}\\
MIT EECS/LIDS\\  
\texttt{kjahn@mit.edu} 
\And
Xiang Cheng$^*$\\
MIT LIDS\\  
\texttt{chengx@mit.edu} 
\And
Hadi Daneshmand$^*$\\
MIT LIDS/FODSI\\  
\texttt{hdanesh@mit.edu} 
\And
Suvrit Sra\\
TU Munich / MIT\\  
\texttt{suvrit@mit.edu} 
}


\begin{document}


\maketitle

\begin{abstract}   
Several recent works demonstrate that transformers can implement algorithms like gradient descent. By a careful construction of weights, these works show that multiple layers of transformers are expressive enough to simulate iterations of gradient descent. Going beyond the question of expressivity, we ask: \emph{Can transformers learn to implement such algorithms by training over random problem instances?} To our knowledge, we make the first theoretical progress on this question via an analysis of the loss landscape for linear transformers trained over random instances of linear regression. For a single attention layer, we prove the global minimum of the training objective implements a single iteration of preconditioned gradient descent. Notably, the preconditioning matrix not only adapts to the input distribution but also to the variance induced by data inadequacy.  For a transformer with $L$ attention layers, we prove certain critical points of the training objective implement $L$ iterations of preconditioned gradient descent. Our results call for future theoretical studies on learning algorithms by training transformers. 
\end{abstract} 

\section{Introduction} 
In-context learning (ICL) is the striking capability of large language models: Given a prompt containing examples and a query, the transformer produces the correct output based on the context provided by the examples, \emph{without adapting its parameters}~\citep{brown2020language,lieber2021jurassic, rae2021scaling, black2022gpt}. This property has become the focus of body of recent research that aims to shed light on the underlying mechanism of large language models~\citep{garg2022can,akyurek2022learning,von2022transformers,li2016learning,min2021metaicl,xie2021explanation,elhage2021mathematical,olsson2022context}.

A line of research studies ICL via the expressive power of transformers. Transformer architectures are powerful  Turing machines,  capable of implementing various algorithms~\citep{perez2021attention,wei2022statistically}. Given an in-context prompt,    
\cite{edelman2022inductive,olsson2022context} argue that transformers are able to implement algorithms through the recurrence of multi-head attentions to extract coarse information from raw input prompts. \citet{akyurek2022learning,von2022transformers} assert that transformers can implement gradient descent on linear regression encoded in a given input prompt. It is thought provoking that transformers can implement such algorithms.

Although transformers are universal machines to implement algorithms, they need specific parameter configurations for achieving these implementations. In practice, their parameters are adjusted via training using non-convex optimization over random problem instances. Hence, it remains unclear whether this non-convex optimization can be used to learn algorithms. The present paper investigates \emph{the possibility of learning algorithms via training over random problem instances.}

More specifically, we investigate the learning of gradient-based methods.  It is hard to mathematically formulate what it means to learn gradient descent for general functions with transformers. Yet, \citet{garg2022can} elegantly examine it in the specific setting of ICL for learning functions. Empirical evidence suggests that transformers indeed learn to implement gradient descent, after training on random instances of linear regression~\citep{garg2022can,akyurek2022learning,von2022transformers}. % This proclivity for learning gradient descent is somewhat surprising,  as the transformer is over-parameterized and thus expressive enough to implement various complicated optimization methods that may work just as well.  
Motivated by these observations, we theoretically investigate the loss landscape of a simple transformer architecture based on \emphh{ attention without softmax}~\citep{schlag2021linear,von2022transformers} (see \autoref{sec:problem} for details).

\emphh{Summary of our main results.} 
Our main contributions are the following:
\begin{list}{$\blacktriangleright$}{\leftmargin=1.5em}
\vspace*{-6pt}
\setlength{\itemsep}{1pt}
\item We provide a complete characterization of the global optimum of a single-layer linear transformer. In particular, we observe that, with the optimal parameters, the transformer implements a single step of preconditioned gradient descent. Notably, the preconditioning matrix not only adapts to the distribution of input data but also to the variance caused by data inadequacy. We present this result in \autoref{thm:main_single} in \autoref{sec:single}.

\item Next, we focus on a subset of the transformer parameter space, defined by a special sparsity condition \eqref{eq:sparse_attention}. Such a parameter configuration allows us to formulate training transformers as a search over \emph{$k$-step adaptive gradient-based algorithms}. \autoref{t:two_layer} characterizes the global minimizers of the training objective of a two-layer linear transformer over isotropic regression instances, and shows that the optima correspond to gradient descent with adaptive stepsizes. For multilayer transformers, \autoref{t:L_layer_P_0} demonstrates that gradient descent, with a data-dependent preconditioning, can be derived from a critical point of the training objective. 

\item Finally, we study the loss landscape in the absence of the sparsity condition \eqref{eq:sparse_attention}, which goes beyond searching over conventional gradient-based optimization methods. In this case, we prove and interpret the structure of a critical point of the training objective. We show that a certian critical point in parameter space leads to an intriguing gradient-based algorithm that simultaneously takes gradient steps preconditioned by data covariance, and applies a linear transformation to further improve the conditioning. In the specific case when data covariance is isotropic, this algorithm corresponds to the GD++ algorithm of \citet{von2022transformers} which is  experimentally observed to be the outcome of training. 

\end{list}
We empirically validate the critical points analyzed in \autoref{t:L_layer_P_0} and \autoref{t:L_layer_P_identity}.
For a transformer with three layers, our experimental results confirm the structural of critical points. Furthermore, we observed the objective value associated with these critical points is close to $0$, suggesting that the critical points might be global optima. These experiments substantiate our theoretical analysis and suggests that our theory indeed \emph{aligns with practice}.  Code for our experiments is available at \url{https://github.com/chengxiang/LinearTransformer}.

% We acknowledge that the \emph{linear attention} model appears to be a significant departure from the standard (softmax) attention model. However, we present evidence in  \autoref{sec:problem} that linear attention is \emph{more suited} for the linear ICL problem, and may in fact be a good proxy for understanding Transformer learning in broader generality.


\subsection{Related works}  


The ability of neural network architectures to implement algorithms has been investigated in various context. The seminal work by \cite{siegelmann1992computational} investigate the Turing completeness of recurrent neural networks. Despite this computational power, training recurrent networks remains a challenge.  \cite{graves2014neural} design an alternative neural architecture known as the \emph{neural Turing machine}, building on \emph{attention layers} introduced by \cite{hochreiter1997long}. Leveraging attention, \cite{vaswani2017attention} propose transformers as powerful neural architectures, capable of solving various tasks in natural language processing~\citep{bert}. This capability inspired a line of research that examines the algorithmic power of transformers \citep{perez2021attention,wei2022statistically,giannou2023looped,akyurek2022learning,olsson2022context}.    What sets transformers apart from conventional neural networks is their impressive performance after training. In this work, we focus on understanding \emph{how transformers learn to implement algorithms} by training over problem instances.


A line of research investigates how deep neural networks process data across their layers. The seminal work by \cite{jastrzebski2018residual} observes that hidden representations across the layers of deep neural  networks approximately implement gradient descent.  
Recent observations provide novel insights into the working mechanism of ICL for large language models, showing they can implement optimization algorithms across their layers~\citep{garg2022can,akyurek2022learning,von2022transformers}. Moreover, \citet{zhao2023transformers,allen2023physics} observe transformer perform dynamic programming to generate text.  
In this work, we theoretically study how transformer learns gradient-based algorithms for ICL.


 We discuss here two related works \citep{zhang2023trained,mahankali2023one} that appeared shortly after publication of our original draft. Both of these studies focus on a single layer attention network (see  \autoref{sec:single}). \cite{zhang2023trained} prove the global convergence of gradient descent to the global optimum whose structure is analyzed independently from this study and it the same as that in~\autoref{thm:main_single}. \cite{mahankali2023one} also characterize the global minimizer of a single layer attention without softmax for a different data distribution. In addition to results for a single-layer attention, we analyze the landscape of two and multi-layer transformers. 
 
 


\section{Setting: training linear transformers over random linear regression}  

\label{sec:problem} 


In order to understand the mechanism of ICL, we consider the setting of training transformers over the random instances of linear regression, following \citep{garg2022can,akyurek2022learning,von2022transformers}.
In particular, the random instances of linear regression are formalized as follows.

\underline{\emphh{Data distribution: random linear regression instances.}}
Let  ${\tx{i}} \in \R^d$  be the covariates drawn i.i.d.\ from a distribution $D_{\mathcal{X}}$, and $\wstar\in \R^d$ be drawn from  $D_{\mathcal{W}}$.
Let $X \in \R^{(n+1)\times d}$ be the matrix of covariates  whose row $i$ contains tokens ${\tx{i}}$.  
Given $\tx{i}$'s and $\wstar$, the responses are defined as $y = [ \langle\tx{1}, \wstar \rangle,\dots,  \langle\tx{n},\wstar \rangle] \in \R^n$. Define the \emphh{input matrix} $Z_0$ as
\begin{align}
\label{d:Z_0}
Z_0 = \begin{bmatrix}
\tz{1} \ \tz{2} \ \cdots \ \tz{n}  \ \tz{n+1}
\end{bmatrix} = \begin{bmatrix}
\tx{1} & \tx{2} & \cdots & \tx{n} &\tx{n+1} \\ 
\ty{1} & \ty{2} & \cdots &\ty{n}& 0
\end{bmatrix} \in \R^{(d+1) \times (n+1)},
\end{align}
where zero in the above matrix is used to replace the unknown response variable corresponding to $\tx{n+1}$.  
Then, our goal is to predict  $\wstar^\top \tx{n+1}$ given $Z_0$. 
In other words, the training data consists of pairs $(Z_0, \wstar^\top \tx{n+1})$ for $\tx{i}\sim D_{\mathcal{X}}$ and 
$\wstar \sim D_{\mathcal{W}}$.
We then consider training transformers over this data distribution.



\underline{\emphh{Self-attention layer without softmax.}}
Following \citep{schlag2021linear,von2022transformers}, we consider the linear self-attention layer.
To motivate, we first briefly review the standard self-attention layer \citep{vaswani2017attention}. Letting  $Z\in \R^{(d+1) \times (n+1)}$ be the input matrix with $n+1$ tokens in $\R^{d+1}$, a single-head self-attention layer denoted by $\softatt$ is a parametric map defined as
\begin{align} \label{eq:softmax}
\softatt_{W_{k,q,v}}(Z) =  W_v Z \aa  \cdot {\sf smax}(Z^\top W_k^\top W_q Z)\,, \quad \aa \coloneqq \begin{bmatrix} I_n & 0 \\0 & 0 \end{bmatrix} \in \R^{(n+1) \times (n+1)},
\end{align}

where $W_v, W_k,W_q \in \R^{(d+1)\times (d+1)}$ are  the  (value, key and query) weight matrices, and $\mathrm{smax}(\cdot)$ is the softmax operator which applies softmax operation to each column of the input matrix. Note that the prompt is asymmetric since the label for $\tx{n+1}$ is excluded from the input. To reflect this asymmetric structure, the mask matrix $M$ is included in the attention.
In our setting, we consider the self-attention layer that omits the softmax operation in \eqref{eq:softmax}. In particular, we reparameterize weights as $P\coloneqq W_v\in \R^{(d+1)\times (d+1)}$ and  $Q  \coloneqq {W_k}^\top W_q \in \R^{(d+1)\times (d+1)}$ and consider  
\begin{align} \label{eq:linear}
\att_{P,Q}(Z) = P Z \aa (Z^\top Q Z) \,.
\end{align}  
At first glance, the omission of the softmax operation \eqref{eq:linear} might seem over-simplified. But, \citep{von2022transformers} proves such attention can implement gradient descent, and we will prove in \autoref{lem:express} that it can also implement various algorithms to solve linear regression in-context.

\underline{\emphh{Architecture for prediction.}}
We now present the neural network architecture that will be used throughout this paper.  For the number of layers $L$, we define an \emphh{$L$-layer transformer} as a stack of $L$ linear self-attention blocks. Formally, denoting by $Z_\ell$ the output of the $\ell^{\text{th}}$ layer attention, we define
\begin{align} \label{eq:recursion}
Z_{\ell+1} = Z_{\ell} +\frac{1}{n}  \att_{P_\ell,Q_\ell}(Z_\ell)\quad \text{for $\ell=0,1,\dots,L-1$},
\end{align}  
The scaling factor $\nicefrac{1}{n}$ is used only for ease of notation and does not influence the expressive power of the transformer.
Given $Z_L$, we define $\NF_L (Z_0; \{P_\ell,Q_\ell\}_{\ell=0,1,\dots L-1})  = -[Z_{L}]_{(d+1),(n+1)}$, i.e., the $(d+1,n+1)$-th entry of $Z_{L}$. 
The reason for the minus sign is to be consistent with \citep{von2022transformers}, and we will  clarify such a choice in \autoref{lem:express}. For training, the parameters are optimized to minimize in-context loss as 
\begin{align} \label{def:ICL linear}
f\left(\{P_\ell, Q_\ell\}^{L}_{\ell=0}\right) = \E_{(Z_0,\wstar)} \Bigl[ \left( \NF_L(Z_0, \{ P_\ell, Q_\ell \}_{\ell=0}^L)+ \wstar^\top \tx{n+1}  \right)^2\Bigr].
\end{align}  

\emphh{\underline{Goal: the landscape analysis of the training objective functions.}}
We are interested in understanding how the optimization of $f$ leads to in-context learning. We investigate this question by analyzing its loss landscape.
Such analysis is challenging due to two major reasons: \emph{(i) $f$ is non-convex in parameters $\{ P_i, Q_i\}$ even for a single layer transformer.
(ii) The cross-product structures in attention makes $f$ a highly nonlinear function in its parameters. }
Hence, we analyze a spectrum of settings from single-layer transformers to multi-layer transformers. For simpler settings such as single-layer transformers, we prove stronger results such as the full characterization of the global minimizers. For networks with more layers, we characterize the structure of critical points. Furthermore, we provide algorithmic interpretations of the critical points. Table~\ref{tab:summary} summarizes our results for various parameteric models. 

\begin{table}[h!]
\centering
\begin{tabular}{c|l l l l}
Results  & $\tx{i}$& $\wstar$ &  Setting & Guarantees \\
\hline
\hline
\autoref{thm:main_single} & $\N(0,\Sigma)$ & $\N(0,I)$ & single-layer & global minimizers \\
\autoref{t:two_layer} & $\N(0,I)$ & $\N(0,I)$ & two-layer +  symmetric \eqref{eq:sparse_attention}   & global minimizers \\
\autoref{t:L_layer_P_0} & $\N(0,\Sigma)$ & $\N(0,\Sigma^{-1})$ & multi-layer + \eqref{eq:sparse_attention}   & critical points \\
\autoref{t:L_layer_P_identity} & $\N(0,\Sigma)$ & $\N(0,\Sigma^{-1})$ & multi-layer +    \eqref{eq:full_attention} & critical points \\
\autoref{thm:nonlinear} & $\N(0,I)$ & $\N(0,I)$ & single-layer +  ReLU activation & global minimizers  \\
\hline
\hline
\end{tabular}
\vspace{5pt}
\caption{Summary of our analyses for various models and input distributions. The additional conditions \eqref{eq:sparse_attention} and  \eqref{eq:full_attention} are about the sparsity structure of parameters. In addition, ``symmetric \eqref{eq:sparse_attention}'' means we additionally impose the weights to be symmetric.}
\label{tab:summary}
\end{table}


\begin{remark}[\emphh{Optimizing \eqref{def:ICL linear} vs. practical transformer optimization}]
Interestingly, a recent work by \cite{ahn2023linear} reports that common optimization algorithms such as SGD/ADAM behave remarkably similarly on the (linear Transformers + linear regression) problem as they do on (practical transformers + real language modeling tasks).
In particular, they reproduce several distinctive features of transformer optimization under a simple shallow linear transformer. This work suggests that (linear transformer + linear regression) may serve as a good proxy for understanding practical transformer optimization. 
\end{remark}

 

\section{The global optimum for a single-layer transformer}
\label{sec:single}
For the single layer case of $L=1$, the following result characterizes the optimal parameters $P_0$ and $Q_0$ for the in-context loss \eqref{def:ICL linear}.  
 
\begin{theorem}
[\textbf{Single-layer; non-isotropic data}] \label{thm:main_single}
Assume that vector $\tx{i}$ is sampled from $\mathcal{N}(0, \Sigma)$, i.e., a Gaussian with covariance $\Sigma = U\Lambda U^\top$ where $\Lambda = \mathrm{diag}(\lambda_1,\dots, \lambda_d)$. 
Moreover, assume that  $\wstar$ is sampled  from $\mathcal{N}(0, I_d)$.  
Then, the following choice of parameters
\begin{align} \label{eq:single_layer}
P_0 = \begin{bmatrix}
0_{d\times d} & 0 \\ 
0 & 1 
\end{bmatrix} ,\quad Q_0 =  - \begin{bmatrix}
U \mathrm{diag}\left(\left\{\frac{1}{  \frac{n+1}{n}  \lambda_i +   \frac{1}{n} \cdot \left( \sum_k \lambda_k  \right)   } \right\}_{i=1,\dots,d}\right) U^\top & 0\\
0  & 0
\end{bmatrix} .
\end{align} 
is a  global minimizer of $f(P,Q)$ up to re-scaling, i.e., $P_0 \leftarrow \gamma P_0$ and $Q_0 \leftarrow \gamma^{-1} Q_0$ for a scalar $\gamma$.
\end{theorem}  

See \autoref{sec:single_proofs} for the proof of \autoref{thm:main_single}. In the specific case when the Gaussian is isotropic, i.e., $\Sigma = I_d$, the optimal $Q_0$ has the following simple form 
\begin{align} \label{minimum:linear}
Q_0 = - \frac{1}{\left( \frac{n-1}{n} +  (d+2) \frac{1}{n} \right)}  \begin{bmatrix}I_d &0   \\ 0 &0\end{bmatrix}.
\end{align}
Up to scaling, the above parameter configuration is equivalent to the parameters used by \citet{von2022transformers} to perform one step of gradient descent. Thus, in the single-layer setting, the in-context loss is indeed minimized by a transformer that implements the gradient descent algorithm.

More generally, when the in-context samples are non-isotropic, the transformer learns to implement one step of a \emph{preconditioned} gradient descent as we shall detail in \autoref{lem:express}. Here the ``preconditioning matrix''  given in \eqref{eq:single_layer} has interesting properties:
\begin{list}{$\bullet$}{\leftmargin=1.5em}
\setlength{\itemsep}{1pt}
\item When the number of samples $n$ is large, the first $d\times d$ submatrix of $Q_0$ approximates $\Sigma^{-1}$, the inverse of the data covariance matrix, which is also close to the Gram matrix formed from $\tx{1},\ldots,\tx{n}$. Hence the preconditioning can lead to considerably faster convergence rate when $\Sigma$ is ill-conditioned.
\item Moreover, $\frac{1}{n} \sum_k \lambda_k$  in \eqref{eq:single_layer}  acts as a regularizer. It becomes more significant when $n$ is small and variance of the $\tx{i}$'s is high. Such an adjustment resembles structural risk minimization \citep{vapnik1999nature} where the regularization strength is adapted to the sample size.   
\end{list}








\section{Multi-layer transformers with sparse parameters}
\label{s:k_layer_Q}

\autoref{thm:main_single} proves a single layer of linear attention can implement a single step of preconditioned gradient descent. Inspired by this result, we investigate the algorithmic power of the linear transformer architecture. We show that the model can implement various optimization methods even under sparsity constraints. In particular, we impose the following restrictions on the parameters:
\begin{align}\label{eq:sparse_attention}
P_i = \begin{bmatrix}
0_{d\times d} & 0 \\ 
0 & 1 
\end{bmatrix}, \quad Q_i = -  \begin{bmatrix}
A_i & 0 \\ 
0 & 0
\end{bmatrix} \quad \text{where $A_i \in \R^{d\times d}$.}
\end{align}
The next lemma proves that a forward-pass of a $L$-layer transformer, with the parameter configuration \eqref{eq:sparse_attention} is the same as taking $L$ steps of gradient descent, preconditioned by $A_\ell$.
 
  



\begin{lemma}[\emphh{Forward pass as a preconditioned gradient descent}] 
\label{lem:express}
Consider the $L$-layer linear transfomer parameterized by $A_0,\dots,A_{L-1}$ as in \eqref{eq:sparse_attention}.  
Let $\ty{n+1}_\ell$ be the $(d+1,n+1)$-th entry of the $\ell$-th layer output, i.e., $\ty{n+1}_\ell = [Z_{\ell}]_{(d+1),(n+1)}$ for $\ell=1,\dots, L$.
Then, it holds that $\ty{n+1}_\ell = - \langle\tx{n+1}, \wgd_\ell \rangle$ where $\{\wgd_\ell\}$ is defined as $\wgd_0=0$ and as follows for $\ell=1,\dots, L-1$:
\begin{align} \label{def:wgd}
\wgd_{\ell+1} = \wgd_{\ell} - A_\ell \nabla R_{\wstar}\lrp{\wgd_{\ell}}\quad \text{where} \quad R_{\wstar}(w) \coloneqq \frac{1}{2n}\sum_{i=1}^{n}(w^\top x_{i}- {\wstar}^\top x_{i})^2.
\end{align} 
\end{lemma}

See \autoref{pf:express} for a proof. The iterative scheme \eqref{def:wgd} includes various optimization methods including  gradient descent with $A_\ell = \gamma_\ell I_d$, and (adaptive) preconditioned gradient descent, where the preconditioner $A_\ell$ depends on the time step. In the upcoming sections, we characterize how the optimal $\{A_\ell\}$ are linked to the input distribution.


\subsection{Warm-up: optimal two-layer transformer with symmetric weights}
\label{s:two_layer}

 For the rest of this section, we will study the optimal parameters for the in-context loss under the constraint of Eq.~\eqref{eq:sparse_attention}. Later in \autoref{s:k_layer_PQ}, we analyze the optimal model for a more general parameters.
For a two-layer transformer, the next Theorem proves the optimal in-context loss obtains the simple gradient descent with adaptive coordinate-wise stepsizes. 


\begin{theorem}[Global optimality for the two-layer (symmetric) transformer]
\label{t:two_layer}
Consider the optimization of in-context loss for a two-layer transformer with the parameter configuration in Eq.~\eqref{eq:sparse_attention}, and additionally assume that $A_1,A_2$ are symmetric matrices. 
More formally, consider
\begin{align}
\min_{A_1,A_2 \text{ are symmetric}} f  \left\{  P_\ell = \begin{bmatrix}
0_{d\times d} & 0 \\ 
0 & 1 
\end{bmatrix},~Q_\ell = \begin{bmatrix}
-A_\ell & 0 \\ 
0 & 0
\end{bmatrix}\right\}_{\ell=1,2}\,.
\end{align}
Assume $\tx{i} \stackrel{\text{i.i.d.}}{\sim}N(0,I_d)$  and $\wstar \sim N(0,I_d)$; then, there are diagonal matrices $A_1$ and $A_2$ that are a global minimizer of $f$.
\end{theorem} 
Combining the above result with \autoref{lem:express} concludes that the two iterations of gradient descent with \emph{coordinate-wise adaptive stepsizes} achieve the minimal in-context loss for isotropic Gaussian inputs. Gradient descent with adaptive stepsizes such as Adagrad \citep{duchi2011adaptive} are widely used in machine learning. While Adagrad adjusts its stepsize based on the individual problem instance, the algorithm learned adjusts its stepsize to the underlying data distribution. 



\subsection{Multi-layer transformers}
\label{s:L_layer_P_0}
We now turn to the setting of general $L$-layer transformers, for any positive integer $L$. The next theorem proves that certain critical points of the in-context loss effectively implement a specific preconditioned gradient algorithm, where the preconditioning matrix is the inverse covariance of the input distribution. Before stating this result, let us first consider a motivating scenario in which the data-covariance matrix is non-identity:

\emphh{Linear regression with distorted view of the data:} Suppose that $\overline{w}_\star \sim \mathcal{N}(0,I)$ and the \emph{latent} covariates are $\overline{x}^{(1)},\dots,\overline{x}^{(n+1)}$, drawn i.i.d from $\N(0,I)$. We are given $\ty{1},\dots,\ty{n}$, with $\ty{i} = \lin{\overline{x}^{(i)}, \overline{w}_\star}$. However, we \emph{do not observe} the latent covariates $\overline{x}^{(i)}$. Instead, we observe the \emph{distorted} covariates $\tx{i} = W \overline{x}^{(i)}$, where $W\in \R^{d\times d}$ is a distortion matrix. Thus the prompt consists of $(\tx{1},\ty{1}),\dots,(\tx{n},\ty{n})$, as well as $\tx{n+1}$. The goal is still to predict $\ty{n+1}$. Note that this setting is quite common in practice, when covariates are often represented in an arbitrary basis. 

Assume that $\Sigma := W W^\top \succ 0$. We verify from our definitions that for $\wstar := \Sigma^{-1/2} \overline{w}_\star$, $\ty{i} = \lin{\tx{i},\wstar}$. Furthermore, $\tx{i} \sim \N(0, \Sigma)$ and $\wstar \sim \N(0, \Sigma^{-1})$. From \autoref{lem:express}, the transformer with weight matrices $\lrbb{A_0,\dots,A_{L-1}}$ implements preconditioned gradient descent with respect to $R_{\wstar}(w) = \frac{1}{2n} (w - \wstar)^T X X^\top (w-\wstar)$, with $X = \lrb{\tx{1}, \dots, \tx{n}}$. Under this loss, the Hessian matrix $\nabla^2 R_{\wstar}(w) = \frac{1}{2n} X X^\top$ (at least in the case of large $n$). For any fixed prompt, Newton's method corresponds to $A_i \propto \lrp{X X^\top}^{-1}$, which makes the problem well-conditioned even if $\Sigma$ is very degenerate. As we will see in \autoref{t:L_layer_P_0} below, the choice of $A_i \propto \Sigma^{-1} = \E\lrb{X X^\top}^{-1}$ appears to be a \emph{stationary point} of the loss landscape, in expectation over prompts.


Before stating the theorem, we introduce the following simplified notation: let $A := \lrbb{A_i}_{i=0}^{L-1} \in \R^{L \times d \times d}$. We use $f(A)$ to denote the in-context loss of $f\left(\{P_i, Q_i\}^{L-1}_{i=0}\right)$ as defined in \eqref{def:ICL linear}, when $Q_i$ depends on $A_i$, and $P_i$ is a constant matrix, as described in \eqref{eq:sparse_attention}.


\begin{theorem}\label{t:L_layer_P_0}  
Assume that $\tx{i} \overset{iid}{\sim} \mathcal{N}(0,\Sigma)$ and $\wstar \sim \mathcal{N}(0,\Sigma^{-1})$, for $i=1,\dots, n$, and for some $\Sigma \succ 0$. Consider the optimization of in-context loss for a $k$-layer transformer with the the parameter configuration in Eq.~\eqref{eq:sparse_attention} given by:
\begin{align}
\min_{\lrbb{A_i}_{i=0}^{L-1}} f \lrp{A}.
\end{align}
Let $\S \subset \R^{L \times d \times d}$ be defined as follows: $A \in \S$ if and only if for all $i = 0,\dots,L-1$, there exists scalars $a_i\in \R$ such that $A_i = a_i \Sigma^{-1}$. Then
\begin{align*}
\inf_{(A,B) \in \S} \sum_{i=0}^{L-1} \lrn{\nabla_{A_i} f(A,B)}_F^2 = 0,
\numberthis \label{e:T:near-stationarity_P0}
\end{align*}
where $\nabla_{A_i} f$ denotes derivative wrt the Frobenius norm $\lrn{A_i}_F$.

\end{theorem}


As discussed in the motivation above, under the setting of $A_i = a_i \Sigma^{-1}$, the linear transformer implements an algorithm that is reminiscent of Newton's method (as well as a number of other adaptive algorithms such as the full-matrix variant of Adagrad); these can converge significantly faster than vanilla gradient descent when the problem is ill-conditioned. The proposed parameters $A_i$ in \autoref{t:L_layer_P_0} are also similar to $\bb_i$'s in \autoref{thm:main_single} when $n$ is large. However, in contrast to \autoref{thm:main_single}, there is no trade-off with statistical robustness; this is because $\wstar$ has covariance matrix $\Sigma^{-1}$ in the \autoref{t:L_layer_P_0}, while \autoref{thm:main_single} has isotropic $\wstar$.

Unlike our prior results, \autoref{t:L_layer_P_0} only guarantees that the set $\S$ of transformer prameters satisfying $\lrbb{A_i \propto \Sigma^{-1}}_{i=0}^{L-1}$ \emph{essentially}\footnote{A subtle issue is that the infimum may not be attained, so it is possible that $\S$ contains points with arbitrarily small gradient, but does not contain a point with exactly $0$ gradient.} contains critical points of the in-context loss. However, in the next section, we show experimentally that this choice of $A_i$'s does indeed seem to be recovered by training. 


We defer the proof of \autoref{t:L_layer_P_0} to \autoref{sec:pf:t:L_layer_P_0}. Due to the complexity of the transformer function, even verifying critical points can be challenging. We show that the in-context loss can be equivalently written as (roughly) a matrix polynomial involving the weights at each layer. By exploiting invariances in the underlying distribution of prompts, we construct a flow, contained entirely in $\S$, whose objective value decreases as fast as gradient flow. Since $f$ is lower bounded, we conclude that there must be points in $\S$ whos gradient is arbitrarily small.


\subsection{Experimental validations for \autoref{t:L_layer_P_0}}
\label{s:experiment_pnull}
We present here an empirical verification of our results in \autoref{t:L_layer_P_0}. We consider the ICL loss for linear regression. The dimension is $d=5$, and the number of training samples in the prompt is $n=20$. Both $\tx{i}\sim \mathcal{N}(0,\Sigma)$ and $\wstar \sim \mathcal{N}(0,\Sigma^{-1})$, where $\Sigma = U^T D U$, where $U$ is a uniformly random orthogonal matrix, and $D$ is a fixed diagonal matrix with entries $(1,1,0.25,0.0625,1)$. 

We optimizes $f$ for a three-layer linear transformer using ADAM, where the matrices $A_0,A_1,$ and $A_2$ are initialized by i.i.d. Gaussian matrices. Each gradient step is computed from a minibatch of size 20000, and we resample the minibatch every 100 steps. We clip the gradient of each matrix to 0.01. All plots are averaged over $5$ runs with different $U$ (i.e. $\Sigma$) sampled each time.


\autoref{fig:loss_pnull} plots the average loss. We observe that the training converges to an almost $0$ value, suggesting the convergence to global minimum. The parameters at convergence match the stationary point introduced in \autoref{t:L_layer_P_0}, and indeed appear to be globally optimal. 

To quantify the similarity between $A_0,A_1,A_2$ and $\Sigma^{-1}$ (up to scaling), we use the \emph{normalized Frobenius norm distance}: $\Dist(M,I) := \min_{\alpha}  \frac{\lrn{M - \alpha  \cdot I}}{\lrn{M}_F}$, (equivalent to choosing $\alpha := \frac{1}{d} \sum_{i=1}^d M[i,i]$). This is essentially the projection distance of $\nicefrac{M}{\lrn{M}}_F$ onto the space of scaled identity matrices. 

We plot $\Dist\lrp{A_i, I}$, averaged over $5$ runs, against iteration in Figures \ref{fig:A0_pnull_trend},\ref{fig:A1_pnull_trend},\ref{fig:A2_pnull_trend}. In each plot, the blue line represents $\Dist(\Sigma^{1/2} A_i \Sigma^{1/2},I)$, and we verify that the optimal parameters are converging to the critical point introduced in \autoref{t:L_layer_P_0}, which implements preconditioned gradient descent. The red line  represents $\Dist(A_i,I)$; it remains constant indicating that the trained transformer is not implementing plain gradient descent.  Figures~\ref{fig:A0_imshow_pnull}--\ref{fig:A2_imshow_pnull} visualize each $\Sigma^{1/2} A_i \Sigma^{1/2}$ matrix at the end of training to further validate that the learned parameter is as described in \autoref{t:L_layer_P_0}.

\iffalse
\begin{figure}[h]
\centering
\begin{subfigure}{0.24\textwidth}
\centering
\includegraphics[width=\textwidth]{random_init_20_pnull/Q0_err_20_pnull.pdf} % first figure
\caption{Error for $A_0$}
\label{fig:A0_pnull_trend}
\end{subfigure}\hfill
\begin{subfigure}{0.24\textwidth}
\centering
\includegraphics[width=\textwidth]{random_init_20_pnull/Q1_err_20_pnull.pdf} % second figure
\caption{Error for $A_1$}
\label{fig:A1_pnull_trend}
\end{subfigure}
\begin{subfigure}{0.24\textwidth}
\centering
\includegraphics[width=\textwidth]{random_init_20_pnull/Q2_err_20_pnull.pdf} % second figure
\caption{Error for $A_2$}
\label{fig:A2_pnull_trend}
\end{subfigure}
\begin{subfigure}{0.24\textwidth}
\centering
\includegraphics[width=\textwidth]{random_init_20_pnull/Loss_loss_20_pnull.pdf} % second figure
\caption{log(Loss)}
\label{fig:loss_pnull}
\end{subfigure}
\begin{subfigure}[b]{0.32\textwidth}
\centering
\includegraphics[width=\textwidth]{random_init_20_pnull/Q0imshow.pdf}
\caption{Visualization of $A_0$} 
\label{fig:A0_imshow_pnull}
\end{subfigure}
\begin{subfigure}[b]{0.32\textwidth}
\centering
\includegraphics[width=\textwidth]{random_init_20_pnull/Q1imshow.pdf}
\caption{Visualization of $A_1$} 
\label{fig:A1_imshow_pnull}
\end{subfigure}
\begin{subfigure}[b]{0.32\textwidth}
\centering
\includegraphics[width=\textwidth]{random_init_20_pnull/Q2imshow.pdf}
\caption{Visualization of $A_2$} 
\label{fig:A2_imshow_pnull}
\end{subfigure} 
\end{figure}
\fi





\begin{figure}[h]
\centering
\begin{subfigure}{0.24\textwidth}
\centering
\includegraphics[width=\textwidth]{camera-ready-figs/rotation_demonstration_dist_to_id_adam_pnull_A0.pdf} % first figure
\caption{$\Dist(\Sigma^{1/2} A_0 \Sigma^{1/2},I)$}
\label{fig:A0_pnull_trend}
\end{subfigure}\hfill
\begin{subfigure}{0.24\textwidth}
\centering
\includegraphics[width=\textwidth]{camera-ready-figs/rotation_demonstration_dist_to_id_adam_pnull_A1.pdf}  % second figure
\caption{$\Dist(\Sigma^{1/2} A_1 \Sigma^{1/2},I)$}
\label{fig:A1_pnull_trend}
\end{subfigure}
\begin{subfigure}{0.24\textwidth}
\centering
\includegraphics[width=\textwidth]{camera-ready-figs/rotation_demonstration_dist_to_id_adam_pnull_A2.pdf}  % second figure
\caption{$\Dist(\Sigma^{1/2} A_2 \Sigma^{1/2},I)$}
\label{fig:A2_pnull_trend}
\end{subfigure}
\begin{subfigure}{0.24\textwidth}
\centering
\includegraphics[height=0.71\textwidth]{camera-ready-figs/rotation_demonstration_adam_pnull_loss_plot.pdf}  % second figure
\caption{log(Loss)}
\label{fig:loss_pnull}
\end{subfigure}
\caption{Plots for verifying convergence of general linear transformer, defined in \autoref{t:L_layer_P_0}. Figure (d) shows convergence of loss to $0$. Figures (a),(b),(c) illustrate convergence of $A_i$'s to identity. More specifically,  the blue line represents $\Dist(\Sigma^{1/2} A_i \Sigma^{1/2},I)$,  which measures the convergence to the critical point introduced in \autoref{t:L_layer_P_0} (corresponding to $\Sigma^{-1}$-preconditioned gradient descent). The red line  represents $\Dist(A_i,I)$; it remains constant indicating that the trained transformer is not implementing plain gradient descent. }

\label{fig:pnull_trend}
\end{figure}
\begin{figure}
\begin{subfigure}[b]{0.32\textwidth}
\centering
\includegraphics[width=\textwidth]{camera-ready-figs/rotation_demonstration_pnull_A0.pdf}
\caption{Visualization of $\Sigma^{1/2} A_0 \Sigma^{1/2}$} 
\label{fig:A0_imshow_pnull}
\end{subfigure}
\begin{subfigure}[b]{0.32\textwidth}
\centering
\includegraphics[width=\textwidth]{camera-ready-figs/rotation_demonstration_pnull_A1.pdf}
\caption{Visualization of $\Sigma^{1/2} A_1 \Sigma^{1/2}$} 
\label{fig:A1_imshow_pnull}
\end{subfigure}
\begin{subfigure}[b]{0.32\textwidth}
\centering
\includegraphics[width=\textwidth]{camera-ready-figs/rotation_demonstration_pnull_A2.pdf}
\caption{Visualization of $\Sigma^{1/2} A_2 \Sigma^{1/2}$} 
\label{fig:A2_imshow_pnull}
\end{subfigure} 
\caption{Visualization of learned weights  for the setting of \autoref{t:L_layer_P_0}. We visualize each $\Sigma^{1/2} A_i \Sigma^{1/2}$ matrix at the end of training.
Note that the optimized weights match the stationary point discussed in \autoref{t:L_layer_P_0}.}
\label{fig:pnull_imshow}
\end{figure}




\section{Multi-layer transformers beyond standard optimization methods}
\label{s:k_layer_PQ}
In this section, we study the more general setting of
\begin{align}\label{eq:full_attention}
P_i = \begin{bmatrix}
B_i & 0 \\ 
0 & 1
\end{bmatrix}, \quad Q_i = \begin{bmatrix}
A_i & 0 \\ 
0 & 0
\end{bmatrix}   \quad \text{where $A_i, B_i \in \R^{d\times d}$.}
\end{align}
Note that $A_i, B_i$ are not constrained to be symmetric.
Similar to \autoref{s:k_layer_Q}, we introduce the following simplified notation: let $A := \lrbb{A_i}_{i=0}^{L-1} \in \R^{L \times d \times d}$ and $B := \lrbb{B_i}_{i=0}^{L-1} \in \R^{L \times d \times d}$. We use $f(A,B)$ to denote the in-context loss of $f\left(\{P_i, Q_i\}^{L-1}_{i=0}\right)$ as defined in \eqref{def:ICL linear}, when $P_i$ and $Q_i$ depend on $B_i$ and $A_i$ as described in \eqref{eq:full_attention}.



With this relaxed parameter configuration, it turns out transformers can learn algorithms beyond the conventional preconditioned gradient descent. The next theorem asserts the possibility of learning a novel preconditioned gradient method. Let $L$ be a fixed but arbitrary number of layers. 

\begin{theorem}\label{t:L_layer_P_identity}
Let $\Sigma$ denote any PSD matrix. Assume that $\tx{i} \overset{iid}{\sim} \mathcal{N}(0,\Sigma)$ and $\wstar \sim \mathcal{N}(0,\Sigma^{-1})$, for $i=1, \dots, n$, and for some $\Sigma \succ 0$. Consider the optimization of in-context loss for a $L$-layer linear transformer with the the parameter configuration in Eq.~\eqref{eq:full_attention} given by:
\begin{align}
\min_{\lrbb{A_i,B_i}_{i=0}^{L-1}} f \lrp{A,B}.
\end{align}

Let $\S \subset \R^{2\times L \times d \times d}$ be defined as follows: $(A,B) \in \S$ if and only if for all $i\in \lrbb{0,\dots,k}$, there exists scalars $a_i,b_i \in \R$ such that $A_i = a_i \Sigma^{-1}$ and $B_i = b_i I$. Then
\begin{align*}
\inf_{(A,B) \in \S} \sum_{i=0}^{L-1} \lrn{\nabla_{A_i} f(A,B)}_F^2 + \lrn{\nabla_{B_i} f(A,B)}_F^2 = 0,
\numberthis \label{e:T:near-stationarity}
\end{align*}
where $\nabla_{A_i} f$ denotes derivative wrt the Frobenius norm $\lrn{A_i}_F$.
\end{theorem}
In words, parameter matrices in $\S$ implement the following  algorithm: $\lrbb{A_i = a_i \Sigma^{-1}}_{i=0}^{L-1}$ plays the role of a distribution-dependent preconditioner for the gradient steps. At the same time, $B_i = b_i I$ transforms the covariates themselves to make the Gram matrix have better condition number with each iteration. When the $\Sigma = I$, the algorithm implemented by $A_i \propto I, b_i \propto I$ is exactly the GD++ algorithm proposed in \citep{von2022transformers} (up to stepsize).

The result in \eqref{e:T:near-stationarity} says that the set $\S$ \emph{essentially}\footnote{Once again, similar to the case of \autoref{t:L_layer_P_0}, the infimum may not be attained, so it is possible that $\S$ contains points with arbitrarily small gradient, but does not contain a point with exactly $0$ gradient.} contains critical points of the in-context loss $f(A,B)$. In the next section, we provide empirical evidence that the trained transformer parameters do in fact converge to a point in $\S$.

%\eqref{e:L_layer_P_identity} even for non-isotropic data.

\subsection{Experimental validations for \autoref{t:L_layer_P_identity}}
\label{s:experiment_PQ}
The experimental setup is similar to \autoref{s:experiment_pnull}: we consider ICL for linear regression with $n=10,d=5$, with $\tx{i}\sim \mathcal{N}(0,\Sigma)$ and $\wstar \sim \mathcal{N}(0,\Sigma^{-1})$, where $\Sigma = U^T D U$, where $U$ is a uniformly random orthogonal matrix, and $D$ is a fixed diagonal matrix with entries $(1,1,0.25,0.0625,1)$. We train a three-layer linear transformer, under the constraints in \eqref{eq:full_attention} which is less restrictive than \eqref{eq:sparse_attention} in \autoref{s:experiment_pnull}. We train the matrices $A_0,A_1,A_2, B_0, B_1$~\footnote{Note that the objective function does not depend on $B_2$.} using ADAM with the same setup as in Section \autoref{s:experiment_pnull}. We repeat this experiment 5 times with different random seeds, each time we sample a different $U$ (i.e. $\Sigma$).

In \autoref{fig:log_loss_PQ}, we plot the in-context loss through the iterations of L-BFGS; the loss appears to be converging to 0, suggesting that parameters are converging to the global minimum. 

\begin{figure}[htbp]
\centering 
\centering
\begin{subfigure}[b]{0.32\textwidth}
\centering
\includegraphics[width=\textwidth]{camera-ready-figs/rotation_demonstration_dist_to_id_adam_B0.pdf}
\caption{$\Dist(B_0,I)$} 
\label{fig:B0_PQ_trend}
\end{subfigure}
\begin{subfigure}[b]{0.32\textwidth}
\centering
\includegraphics[width=\textwidth]{camera-ready-figs/rotation_demonstration_dist_to_id_adam_B1.pdf}
\caption{$\Dist(B_1,I)$} 
\label{fig:B1_PQ_trend}
\end{subfigure}
\begin{subfigure}[b]{0.32\textwidth}
\centering
\includegraphics[height=0.71\textwidth]{camera-ready-figs/rotation_demonstration_adam_loss_plot.pdf}
\caption{log(Loss)} 
\label{fig:log_loss_PQ}
\end{subfigure}
\begin{subfigure}[b]{0.32\textwidth}
\centering
\includegraphics[width=\textwidth]{camera-ready-figs/rotation_demonstration_dist_to_id_adam_A0.pdf}
\caption{Distances for $A_0$ } 
\label{fig:A0_PQ_trend}
\end{subfigure}
\begin{subfigure}[b]{0.32\textwidth}
\centering
\includegraphics[width=\textwidth]{camera-ready-figs/rotation_demonstration_dist_to_id_adam_A1.pdf}
\caption{Distances for $A_1$ } 
\label{fig:A1_PQ_trend}
\end{subfigure}
\begin{subfigure}[b]{0.32\textwidth}
\centering
\includegraphics[width=\textwidth]{camera-ready-figs/rotation_demonstration_dist_to_id_adam_A2.pdf}
\caption{Distances for $A_2$ } 
\label{fig:A2_PQ_trend}
\end{subfigure}  
\caption{Plots for verifying convergence of general linear transformer, defined in \autoref{t:L_layer_P_identity}. Figure (c) shows convergence of loss to $0$. Figures (a),(b) illustrate convergence of $B_0,B_1$ to identity. Figures (d),(e),(f) illustrate convergence of $A_i$'s to $\Sigma^{-1}$.}
\end{figure}

We next verify that the parameters at convergence are consistent with \autoref{t:L_layer_P_identity}. We will once again use $\Dist(M,I)$ to measure the distance from $M$ to the identity matrix, up to scaling (see \autoref{s:experiment_pnull} for definition of $Dist$). Figures \ref{fig:B0_PQ_trend} and \ref{fig:B1_PQ_trend} show that $B_0$ and $B_1$ are close to identity, as $\Dist(B_i, I)$ appears to be decreasing to 0. Figures \ref{fig:A0_PQ_trend}, \ref{fig:A1_PQ_trend} and \ref{fig:A2_PQ_trend} plot $\Dist(A_i, I)$ (red line) and $\Dist(\Sigma^{1/2} A_i \Sigma^{1/2}, I)$ (blue line); the results here suggest that $A_i$ is converging to $\Sigma^{-1}$, up to scaling. In Figures \ref{fig:B0_PQ_trend} and \ref{fig:B1_PQ_trend}, we observe that $B_0$ and $B_1$ also converge to the identity matrix (\emph{without} left and right multiplication by $\Sigma^{1/2}$), consistent with \autoref{t:L_layer_P_identity}. 
% Nonetheless, it is somewhat surprising that $A_i$'s adapt to the data covariance, but $B_i$'s remain identity.

%Finally, in Figures \ref{fig:A0_PQ},\ref{fig:A1_PQ},\ref{fig:A2_PQ}, we provide a direct visualization of of each $\Sigma^{1/2} A_i \Sigma^{1/2}$ matrix at the end of training. One can visually verify that these matrices are indeed close to identity, up to scaling.

We visualize each of $B_0,B_1$ in \autoref{fig:B0_B1} and $A_0,A_1,A_2$ in Figure~\ref{fig:A0_PQ_app}-\ref{fig:A2_PQ_app} at the end of training. We highlight two noteworthy observations:
\begin{enumerate}
\item Let $X_k\in \R^{d\times n}$ denote the first $d$ rows of $Z_k$, which are the output at layer $k-1$ defined in \eqref{eq:recursion}. Then the update to $X_k$ is $X_{k+1} = X_k + B_k X_k \aa X_k^T A_k X_k \approx X_{k+1} = X_k \lrp{I - |a_kb_k| \aa X_k^T X_k}$, where $\aa$ is a mask defined in \eqref{eq:softmax}. As noted by \cite{von2022transformers}, this may be motivated by curvature correction. 
\item As seen in Figures \ref{fig:A0_PQ_app}-\ref{fig:A2_PQ_app} in the Appendix, $\| A_0\| \leq \| A_1\| \leq \| A_2\|$ that implies the transformer implements gradient descent with a small stepsize at the beginning and a large stepsize at the end. This makes intuitive sense as $X_2$ is better-conditioned compared to $X_1$, due to the choice of $B_0,B_1$. This can be contrasted with the plots in Figures \eqref{fig:A0_imshow_pnull}-\eqref{fig:A2_imshow_pnull}, where similar trends are not as pronounced because $B_i$'s are constrained to be $0$.
\end{enumerate}
 

 

\begin{figure}
\centering  
\includegraphics[width=0.35\textwidth]{camera-ready-figs/rotation_demonstration_adam_B0.pdf}
\includegraphics[width=0.35\textwidth]{camera-ready-figs/rotation_demonstration_adam_B1.pdf} 
\vspace{-10pt}
\caption{Visualization of optimized weight matrices $B_0$ (left) and $B_1$ (right). One can see that the weight pattern matches the stationary point analyzed in \autoref{t:L_layer_P_identity}. Matrices $A_0$, $A_1$ and $A_2$ are similar to \autoref{fig:pnull_imshow}, and are visualized in \autoref{fig:appendix_full} in \autoref{s:additional_plots}.}
\label{fig:B0_B1}
\end{figure}




\section{Discussion} 

We take a first step toward proving that transformers can learn algorithms when trained over a set of random problem instances. Specifically, we investigate the possibility of learning gradient based methods when training on the in-context loss for linear regression. 
For a single layer transformer, we prove that the global minimum corresponds to a single iteration of preconditioned gradient descent.  
For multiple layers, we show that certain parameters that correspond to the critical points of the in-context loss can be interpreted as a broad family of adaptive gradient-based algorithms.

 
We discuss below two interesting future directions.

\textbf{Beyond linear attention.} The standard transformer architecture comes with nonlinear activations in attention. Hence, the natural question here is to ask the effect of nonlinear activations for our main results. Empirically, \cite{von2022transformers} have observed that for linear regression task, softmax activations  generally degrade the prediction performance, and in particular, softmax transformers typically need more attention heads to match their performance with that of linear transformers. 

As a first step analysis, we consider the nonlinear attention defined as 
\begin{align}
    \atth_{P,Q}(Z) \coloneqq  P ZM \  \sigma (Z^\top Q Z)\quad \text{where }\sigma :\R\to \R \text{ is applied entry-wise.}
\end{align}  
The following result is an analog of \autoref{thm:main_single} for single-layer nonlinear attention. It characterizes a global minimizer for this setting with ReLU activation.
Here, our choice of ReLU activation was motivated by \cite{wortsman2023replacing} who observed that ReLU attention matches the performance of softmax attention for vision transformers. 

\begin{theorem} \label{thm:nonlinear} 
Consider the single layer nonlinear attention setting with  $\sigma = \mathrm{ReLU}$.
Assume that vector $\tx{i}$ is sampled from $\mathcal{N}(0, I_{d})$. 
Moreover, assume that  $\wstar$ is sampled  from $\mathcal{N}(0, I_d)$.  
Consider the parameter configuration $P_0,Q_0$  where we additionally assume that the last row of $Q_0$ is zero. 
Then, the following parameters form a global minimizer of the corresponding in-context loss:
    \begin{align}  
P_0 = \begin{bmatrix}
0_{d\times d} & 0 \\ 
0 & 1 
\end{bmatrix} ,\quad Q_0 = -\frac{1}{\frac{1}{2}\frac{n-1}{n}+(d+2)\frac{1}{n}} \cdot \begin{bmatrix}
 I_d & 0\\
0  & 0
\end{bmatrix} .
\end{align} 
\end{theorem}
The proof of \autoref{thm:nonlinear}  involves an instructive argument and leverages tools from \citep{erdogdu2016scaled}; we defer it to \autoref{pf:nonlinear}. Thus, for isotropic Gaussian data, the structure of global minimum under ReLU attention is similar to the global minimum with linear attention, established in \autoref{thm:main_single} (specifically the minimizer for the isotropic date given in \eqref{minimum:linear}). 

 

\textbf{Refined landscape analysis for multilayer transformer.} \autoref{t:L_layer_P_identity}   proves that a stationary point of the in-context loss corresponds to  implementing a preconditioned gradient method. However, we do not prove that all critical points of the non-convex objective lead to similar optimization methods. In fact, in \autoref{l:bar} in   \autoref{a:foo}, we prove that the in-context loss can have multiple critical points. It will be interesting to analyze the set of all critical points and try to understand their algorithmic interpretations, as well as quantify their (sub)optimality.

\iffalse
\begin{list}{$\bullet$}{\leftmargin=1.3em}
\item \emphh{Refined landscape analysis for multilayers.} \autoref{t:L_layer_P_identity}   proves that a multi-layer transformer implementing a gradient method is a critical point of the training objective. However, we do not prove that all critical points of the non-convex objective lead to similar optimization methods. In fact, in \autoref{l:bar} in Appendix \ref{a:foo}, we prove that the in-context loss can have multiple critical points. It will be interesting to analyze the set of all critical points and try to understand their algorithmic interpretations, as well as quantify their (sub)optimality.

% \item \emphh{Convergence to the global optimum.}  Although \autoref{thm:main_single} establishes a closed form for the global minimizer of the training objective, gradient-based optimization used for training may nonetheless converge to a different stationary point. Analyzing the convergence of gradient descent on $f$ will provide insights into how training of transformer architecture works.



\item \emphh{Attention with softmax.}
We omit the softmax in attention to simplify our analysis. For the specific example of learning linear functions,  \cite{von2022transformers} experimentally show that training a two-head attention with softmax can implement gradient descent. It remains open whether training indeed still learns gradient descent  with softmax in attention.

%\item \emphh{Applications for domain adaptation.}
%The algorithmic view can potentially explain various properties of ICL. For example, one could explain the robustness to distribution shift if the minimizer of in-context loss was implementing gradient descent -- gradient descent is inherently robust against the shift in the prompt distribution. This motivates future studies on  the algorithmic power of  trained transformers.  
\end{list}
\fi
%\paragraph{Beyond linear self-attention.}
%Following (\cite{wortsman2023replacing,zhao2023transformers}), we consider a specific attention obtained by replacing softmax with linear unit (ReLU). In fact, \cite{wortsman2023replacing} observe that ReLU attention matches the performance of softmax attention for vision transformers. 
%Let $\atth_{P,Q}(Z) \coloneqq  P ZM \  \relu (
%Z^\top Q Z)$, 
%where $\relu$ is the ReLU activation function.
%The following result characterizes a global minimizer for this setting, and is the analog of \autoref{thm:main_single} for ReLU attention.

%See \autoref{pf:nonlinear} for the proof. Thus, for isotropic Gaussian data, the structure of global minimum under ReLU attention is similar to the global minimum with linear attention, established in \autoref{thm:main_single}. This result is a step to bridge the gap between our theoretical model and practical models used in practice.

\begin{ack}
We thank Ekin Akyürek,  Johannes von Oswald, Alex Gu and Joshua Robinson for helpful discussions.
Kwangjun Ahn was supported by the ONR grant (N00014-20-1-2394) and MIT-IBM Watson as
well as a Vannevar Bush fellowship from Office of the Secretary of Defense. Kwangjun Ahn also
acknowledges support from the Kwanjeong Educational Foundation. Xiang Cheng acknowledges
support from NSF CCF-2112665 (TILOS AI Research Institute). Hadi Daneshmand acknowledges support from NSF TRIPODS program (award DMS-2022448). Suvrit Sra acknowledges support
from an NSF CAREER grant (1846088), and NSF CCF-2112665 (TILOS AI Research Institute).


\end{ack}
\bibliographystyle{plainnat}
\bibliography{ref}
\appendix
%\onecolumn


% \tableofcontents{}

% \newpage

\section*{Supplementary Material}
\addcontentsline{toc}{section}{Supplementary Material}


Throughout this discussion, 
we will make frequently use 
of the following standard results
concerning the exponential concentration 
of random variables:

\begin{lemma}[Hoeffding's inequality for independent RVs~\citep{hoeffding1994probability}] Let $Z_1, Z_2, \ldots, Z_n$ be independent bounded random variables with $Z_i \in [a,b]$ for all $i$, then 
    \begin{align*}
        \prob\left( \frac{1}{n} \sum_{i=1}^n (Z_i - \Expo{Z_i}) \ge t \right) \le \exp{\left( -\frac{2nt^2}{(b-a)^2} \right) }
    \end{align*} 
    and 
    \begin{align*}
        \prob\left( \frac{1}{n} \sum_{i=1}^n (Z_i - \Expo{Z_i}) \le -t \right) \le \exp{\left( -\frac{2nt^2}{(b-a)^2} \right) }
    \end{align*} 
    for all $t \ge 0$. 
\end{lemma}

\begin{lemma}[Hoeffding's inequality for sampling with replacement~\citep{hoeffding1994probability}] \label{lem:hoeffding_sampling} Let $\calZ = (Z_1, Z_2, \ldots, Z_N)$ be a finite population of $N$ points with $Z_i \in [a.b]$ for all $i$. Let $X_1, X_2, \ldots X_n$ be a random sample drawn without replacement from $\calZ$. Then for all $t \ge 0$, we have 
    \begin{align*}
        \prob\left( \frac{1}{n} \sum_{i=1}^n (X_i - \mu ) \ge t \right) \le \exp{\left( -\frac{2nt^2}{(b-a)^2} \right) }
    \end{align*} 
    and 
    \begin{align*}
        \prob\left( \frac{1}{n} \sum_{i=1}^n (X_i - \mu ) \le -t \right) \le \exp{\left( -\frac{2nt^2}{(b-a)^2} \right) } \,,
    \end{align*} 
    where $\mu = \frac{1}{N} \sum_{i=1}^{N} Z_i$. 
\end{lemma}

We now discuss one condition that generalizes the exponential concentration to dependent random variables.
\begin{condition}[Bounded difference inequality] \label{cond:BDC} Let $\calZ$ be some set and $\phi: \calZ^n \to \Real$. We say that $\phi$ satisfies the bounded difference assumption if 
there exists $c_1, c_2, \ldots c_n \ge 0$ s.t. for all $i$, we have 
\begin{align*}
    \sup_{Z_1,Z_2, \ldots,Z_n, Z_i^\prime \in \calZ^{n+1} } \abs{\phi (Z_1, \ldots, Z_i, \ldots, Z_n ) - \phi (Z_1, \ldots, Z_i^\prime, \ldots, Z_n ) } \le c_i \,.
\end{align*} 
\end{condition}

\begin{lemma}[McDiarmid’s inequality~\citep{mcdiarmid1989}] \label{lem:McDiarmid} Let $Z_1, Z_2, \ldots, Z_n$ be independent random variables on set $\calZ$ and $\phi : \calZ^n \to \Real$ satisfy bounded difference inequality (\codref{cond:BDC}). Then for all $t>0$, we have 
    \begin{align*}
        \prob\left( \phi(Z_1, Z_2, \ldots, Z_n) - \Expo{\phi(Z_1, Z_2, \ldots, Z_n)} \ge t \right) \le \exp{\left( -\frac{2t^2}{\sum_{i=1}^n c_i^2} \right) } 
    \end{align*} 
    and 
    \begin{align*}
        \prob\left( \phi(Z_1, Z_2, \ldots, Z_n) - \Expo{\phi(Z_1, Z_2, \ldots, Z_n)} \le -t \right) \le \exp{\left( -\frac{2t^2}{\sum_{i=1}^n c_i^2} \right) } \,.
    \end{align*} 
\end{lemma}


\section{Proofs from \secref{sec:ERM_training}}\label{app:proof_erm}

\textbf{Additional notation {} {}} Let $m_1$ be the number of mislabeled points ($\wt S_M$) and $m_2$ be the number of correctly labeled points ($\wt S_C$). Note $m_1 + m_2 = m$. 


\subsection{Proof of \thmref{thm:error_ERM}}


\begin{proof}[Proof of \lemref{lem:fit_mislabeled}] 
    The main idea of our proof is to regard 
    the clean portion of the data 
    ($S \cup \wt S_C$) as fixed.   
    Then, there exists an (unknown) classifier $f^*$ 
    that minimizes the expected risk
    calculated on the (fixed) clean data
    and (random draws of) the mislabeled data $\wt S_M$. 
    % 
    % 
    Formally, 
    \begin{align}
    f^* \defeq \argmin_{f \in \calF} \error_{\widecheck {\calD}} (f) \,, \label{eq:modified_ERM}
    \end{align}
    where $$\widecheck \calD = \frac{n}{m+n} \calS + \frac{m_2}{m+n} \wt \calS_C  + \frac{m_1}{m+n}\calDm \,.$$ 
    Note here that $\widecheck \calD$ is a combination 
    of the \emph{empirical distribution} 
    over correctly labeled data $S \cup \wt S_C$
    and the (population) distribution 
    over mislabeled data $\calDm$.
    Recall that 
    \begin{align}
    \wh f \defeq \argmin_{f \in \calF} \error_{\calS \cup \wt S} (f) \,. \label{eq:orig_ERM}
    \end{align}
    % 
    % 
    Since, $\widehat f$ minimizes 0-1 error 
    on $S \cup \wt S$, using ERM optimality on \eqref{eq:orig_ERM},  
    we have 
    \begin{align}
        \error_{\calS \cup \wt \calS}(\widehat f) \le \error_{
            \calS \cup \wt \calS}(f^*) \,.    \label{eq:step1}
    \end{align}
    Moreover, since $f^*$ is independent of $\wt S_M$, using Hoeffding's bound,
    % \footnote{For a fully rigorous argument,
    % refer to the complete proof in App.~\ref{app:proof_erm}.} 
    we have with probability at least $1-\delta$ that
    \begin{align}
      \error_{\wt \calS_M}(f^*) \le \error_{ \calDm}(f^*) +  \sqrt{\frac{\log(1/\delta)}{2 m_1}} \,. \label{eq:step2} 
    \end{align}
    %$ 
    %for some constant $c_1\le 1/2$. 
    Finally, since $f^*$ is the optimal classifier on $\widecheck \calD$, 
    we have 
    \begin{align}
        \error_{\widecheck \calD}(f^*) \le \error_{\widecheck \calD}(\widehat f) \,. \label{eq:step3}
    \end{align}
    Now to relate \eqref{eq:step1} and \eqref{eq:step3}, we multiply \eqref{eq:step2} by $\frac{m_1}{m+n}$ and add $\frac{n}{m+n} \error_{\calS} (f)  + \frac{m_2}{m+n} \error_{\wt \calS_C} (f)$ both the sides. Hence, 
    we can rewrite \eqref{eq:step2} as follows: 
    \begin{align}
        \error_{\calS \cup \wt\calS}(f^*) \le \error_{ \widecheck \calD}(f^*) +  \frac{m_1}{m+n}\sqrt{\frac{\log(1/\delta)}{2 m_1}} \,. \label{eq:step4} 
    \end{align}
    Now we combine equations \eqref{eq:step1}, \eqref{eq:step4}, and \eqref{eq:step3}, to get 
    \begin{align}
        \error_{\calS \cup \wt \calS}(\wh f) \le \error_{\widecheck \calD}(\wh f) +  \frac{m_1}{m+n}\sqrt{\frac{\log(1/\delta)}{2 m_1}} \,, 
    \end{align}
    which implies 
    \begin{align}
        \error_{ \wt \calS_M}(\wh f) \le \error_{\calDm}(\wh f) + \sqrt{\frac{\log(1/\delta)}{2 m_1}} \,. \label{eq:lemma1_final}
    \end{align}
    Since $\wt S$ is obtained by randomly labeling an unlabeled dataset, we assume $2m_1 \approx m$ \footnote{Formally, with probability at least $1-\delta$, we have  $(m - 2m_1)\le \sqrt{m\log(1/\delta)/2}$.}. Moreover, using $\error_{\calDm} = 1 - \error_{\calD}$ we obtain the desired result.   
    % Combining the above steps and using the fact 
    % that $\error_\calD = 1- \error_{\calDm} $, 
    % we obtain the desired result.
\end{proof}

\begin{proof}[Proof of \lemref{lem:mislabeled_error}]
    Recall $\error_{\wt S} (f) = \frac{m_1}{m} \error_{\wt S_M}(f) + \frac{m_2}{m} \error_{\wt S_C}(f)$. Hence, we have 
    \begin{align}
        2\error_{\wt S}(f) - \error_{\wt S_M}(f) - \error_{\wt S_C}(f) &= \left(\frac{2m_1}{m} \error_{\wt S_M}(f) - \error_{\wt S_M}(f)\right) + \left(\frac{2m_2}{m} \error_{\wt S_C}(f) - \error_{\wt S_C}(f)\right) \\ &= \left(\frac{2m_1}{m} - 1\right) \error_{\wt S_M}(f) + \left(\frac{2m_2}{m} - 1 \right)\error_{\wt S_C} (f) \,.
    \end{align} 
    Since the dataset is labeled uniformly at random, with probability at least $1-\delta$, we have  $\left(\frac{2m_1}{m} - 1\right) \le \sqrt{\frac{\log(1/\delta)}{2m}}$. Similarly, we have with probability at least $1-\delta$, $\left(\frac{2m_2}{m} - 1\right) \le \sqrt{\frac{\log(1/\delta)}{2m}}$. Using union bound, with probability at least $1-\delta$, we have
    % \begin{align}
    %     2\error_{\wt S} - \error_{\wt S_M}(f) - \error_{\wt S_C}(f) \le \sqrt{\frac{\log(2/\delta)}{2m}} \left(\error_{\wt S_M}(f) + \error_{\wt S_C}(f) \right) \le 2\sqrt{\frac{\log(2/\delta)}{2m}} \,. \label{eq:lemma2_final}
    % \end{align}
    \begin{align}
        2\error_{\wt S} - \error_{\wt S_M}(f) - \error_{\wt S_C}(f) \le \sqrt{\frac{\log(2/\delta)}{2m}} \left(\error_{\wt S_M}(f) + \error_{\wt S_C}(f) \right) \,. \label{eq:lemma2_prefinal}
    \end{align}
    With re-arranging $\error_{\wt S_M}(f) + \error_{\wt S_C}(f)$ and using the inequality $ 1- a\le \frac{1}{1+a} $, we have  
    \begin{align}
        2\error_{\wt S} - \error_{\wt S_M}(f) - \error_{\wt S_C}(f) \le 2\error_{\wt \calS} \sqrt{\frac{\log(2/\delta)}{2m}}  \,. \label{eq:lemma2_final}
    \end{align}

    % We obtain the desired result by using 
\end{proof}

\begin{proof}[Proof of \lemref{lem:clear_error}]
% Recall 0-1 error on each point  $(x,y) \in S \cup \wt S$ is given by $\I{ f(x)\ne y}$.
In the set of correctly labeled points $S \cup \wt S_C$, we have $S$ as a random subset of $S \cup \wt S_C$. Hence, using Hoeffding's inequality for sampling without replacement (\lemref{lem:hoeffding_sampling}), we have with probability at least $1-\delta$
\begin{align}
    \error_{\wt \calS_C} (\wh f)- \error_{\calS \cup \wt \calS_C}( \wh f) \le  \sqrt{\frac{\log(1/\delta)}{2m_2}} \,.
\end{align}
Re-writing $\error_{\calS \cup \wt \calS_C}( \wh f)$ as $\frac{m_2}{m_2 + n} \error_{\wt \calS_C }(\wh f) + \frac{n}{m_2 + n} \error_{\calS }(\wh f)$, we have with probability at least $1-\delta$
\begin{align}
   \left(\frac{n}{n+m_2}\right) \left(\error_{\wt \calS_C} (\wh f)- \error_{\calS}( \wh f) \right) \le  \sqrt{\frac{\log(1/\delta)}{2m_2}} \,.
\end{align}
As before, assuming $2m_2 \approx m$, we have with probability at least $1-\delta$ 
\begin{align}
    \error_{\wt \calS_C} (\wh f)- \error_{\calS}( \wh f) \le \left(1+\frac{m_2}{n}\right)  \sqrt{\frac{\log(1/\delta)}{m}} \le \left(1 + \frac{m}{2n}\right) \sqrt{\frac{\log(1/\delta)}{m}} \,. \label{eq:lemma3_final}
\end{align} 
\end{proof}

\begin{proof}[Proof of \thmref{thm:error_ERM}] 
    Having established these core intermediate results, we can now combine above three lemmas to prove the main result. 
    In particular, we bound the population error on clean data ($\error_\calD(\wh f)$) as follows:  
    \begin{enumerate}[(i)]
        \item First, use \eqref{eq:lemma1_final}, to obtain an upper bound on the population error on clean data, i.e., with probability at least $1-\delta/4$, we have
        \begin{align}
            \error_{ \calD} (\wh f) \le 1 - \error_{ \wt \calS_M}(\wh f) + \sqrt{\frac{\log(4/\delta)}{m}} \,. 
        \end{align}
        \item  Second, use \eqref{eq:lemma2_final}, to relate the error on the mislabeled fraction with error on clean portion of randomly labeled data and error on whole randomly labeled dataset, i.e., with probability at least $1-\delta/2$, we have 
        \begin{align}
            - \error_{\wt S_M}(f) \le \error_{\wt S_C}(f) - 2\error_{\wt S}  + 2\error_{\wt S} \sqrt{\frac{\log(4/\delta)}{2m}}  \,. 
        \end{align} 
        \item Finally, use \eqref{eq:lemma3_final} to relate the error on the clean portion of randomly labeled data and error on clean training data, i.e., with probability $1-\delta/4$, we have 
        \begin{align}
            \error_{\wt \calS_C} (\wh f)\le - \error_{\calS}( \wh f) + \left(1 + \frac{m}{2n} \right) \sqrt{\frac{\log(4/\delta)}{m}} \,. 
        \end{align} 
    \end{enumerate}

    Using union bound on the above three steps, we have with probability at least $1-\delta$: 
    \begin{align}
        \error_\calD (\wh f) \le \error_{\calS}(\wh f)   + 1 - 2\error_{\wt \calS}(\wh f)   + \left(\sqrt{2} \error_{\wt S} + 2 + \frac{m}{2n}\right)  \sqrt{\frac{\log(4/\delta)}{m}} \,.
    \end{align}
    % Note that $(1/\sqrt{2} + 2.5)$ is a loose constant. In experiments, we use the ratio $\frac{m}{n}$
    %  the exact error $\error_{\wt \calS}(\wh f)$ 
    % to evaluate R.H.S.    
\end{proof}

\subsection{Proof of \propref{prop:rademacher}}

\begin{proof}[Proof of \propref{prop:rademacher}]
    For a classifier $ f: \calX \to \{-1, 1\}$, we have $1 - 2\,\indict{ f(x) \ne y} = y \cdot f(x)$. Hence, by definition of $\error$, we have 
    \begin{align}
        1 -2\error_{\wt \calS}(f) = \frac{1}{m}\sum_{i=1}^m y_i \cdot f(x_i) \le \sup_{f \in \calF} \, \frac{1}{m} \sum_{i=1}^m y_i \cdot f(x_i)  \,. \label{eq:error_rademacher}
    \end{align}
    Note that for fixed inputs $(x_1, x_2, \ldots, x_m)$ in $\wt S$, $(y_1, y_2, \ldots y_m)$ are random labels. Define $\phi_1 (y_1, y_2, \ldots, y_m) \defeq \sup_{f \in \calF} \, \frac{1}{m} \sum_{i=1}^m y_i \cdot f(x_i)$. We have the following bounded difference condition on $\phi_1$. For all i, 
    \begin{align}
        \sup_{y_1, \ldots y_m, y_i^\prime \in \{-1, 1\}^{m+1} } \abs{ \phi_1 (y_1,\ldots, y_i, \ldots, y_m) - \phi_1 (y_1,\ldots, y_i^\prime, \ldots, y_m)  } \le 1/m \,. \label{cond1_rademacher}
    \end{align} 
    
    Similarly, we define $\phi_2 (x_1, x_2, \ldots, x_m) \defeq \Expt{ y_i \sim_U \{-1, 1\}  }{ \sup_{f \in \calF} \, \frac{1}{m}  \sum_{i=1}^m y_i \cdot f(x_i)}$. We have the following bounded difference condition on $\phi_2$. 
    For all i,
    \begin{align}
        \sup_{x_1, \ldots x_m, x_i^\prime \in \calX^{m+1} } \abs{ \phi_2 (x_1,\ldots, x_i, \ldots, x_m) - \phi_1 (x_1,\ldots, x_i^\prime, \ldots, x_m)  } \le 1/m \,. \label{cond2_rademacher}
    \end{align}
    Using McDiarmid’s inequality (\lemref{lem:McDiarmid}) twice 
    with Condition \eqref{cond1_rademacher} and \eqref{cond2_rademacher}, 
    with probability at least $1-\delta$, we have
    \begin{align}
        \sup_{f \in \calF} \, \frac{1}{m} \sum_{i=1}^m y_i \cdot f(x_i)  - \Expt{x,y}{\sup_{f \in \calF} \, \frac{1}{m} \sum_{i=1}^m y_i \cdot f(x_i) } \le \sqrt{\frac{2\log(2/\delta)}{m}} \,. \label{eq:final_rademacher}
    \end{align} 
    Combining \eqref{eq:error_rademacher} and \eqref{eq:final_rademacher}, we obtain the desired result. 
\end{proof}


\subsection{Proof of \thmref{thm:error_regularized_ERM}}

Proof of \thmref{thm:error_regularized_ERM} follows similar to the proof of \thmref{thm:error_ERM}. Note that the same results in \lemref{lem:fit_mislabeled}, \lemref{lem:mislabeled_error}, and \lemref{lem:clear_error} hold in the regularized ERM case. However, the arguments in the proof of \lemref{lem:fit_mislabeled} change slightly. Hence, we state the lemma for regularized ERM and prove it here for completeness. 

\begin{lemma} \label{lem:lemma1_reg}
    Assume the same setup as \thmref{thm:error_regularized_ERM}. 
    Then for any $\delta >0$, with probability at least  $1-\delta$ 
    over the random draws of mislabeled data $\wt S_M$, we have 
    \begin{align}
        \error_\calD(\widehat f)  \le 1 -\error_{\wt \calS_M}(\widehat f) + \sqrt{\frac{\log(1/\delta)}{m}}\,. 
    \end{align} 
\end{lemma}
\begin{proof}
    The main idea of the proof remains the same, i.e. regard 
    the clean portion of the data 
    ($S \cup \wt S_C$) as fixed.   
    Then, there exists a classifier $f^*$ 
    that is optimal over draws 
    of the mislabeled data $\wt S_M$. 

    
    Formally, 
    \begin{align}
    f^* \defeq \argmin_{f \in \calF} \error_{\widecheck {\calD}} (f)  + \lambda R(f) \,, \label{eq:modified_ERM_reg}
    \end{align}
    where $$\widecheck \calD = \frac{n}{m+n} \calS + \frac{m_1}{m+n} \wt \calS_C  + \frac{m_2}{m+n}\calDm \,.$$ That is, $\widecheck \calD$ a combination of 
    the \emph{empirical distribution} 
    over correctly labeled data $S \cup \wt S_C$
    % in $S\cup \wt S$ 
    and the (population) distribution 
    over mislabeled data $\calDm$.
    Recall that 
    \begin{align}
    \wh f \defeq \argmin_{f \in \calF} \error_{\calS \cup \wt S} (f) + \lambda R(f) \,. \label{eq:orig_ERM_reg}
    \end{align}
    % 
    % 
    Since, $\widehat f$ minimizes 0-1 error 
    on $S \cup \wt S$, using ERM optimality on \eqref{eq:orig_ERM},  
    we have 
    \begin{align}
        \error_{\calS \cup \wt \calS}(\widehat f) + \lambda R(\wh f) \le \error_{
            \calS \cup \wt \calS}(f^*) + \lambda R(f^*) \,.    \label{eq:step1_reg}
    \end{align}
    Moreover, since $f^*$ is independent of $\wt S_M$, using Hoeffding's bound,
    % \footnote{For a fully rigorous argument,
    % refer to the complete proof in App.~\ref{app:proof_erm}.} 
    we have with probability at least $1-\delta$ that
    \begin{align}
      \error_{\wt \calS_M}(f^*) \le \error_{ \calDm}(f^*) +  \sqrt{\frac{\log(1/\delta)}{2 m_1}} \,. \label{eq:step2_reg} 
    \end{align}
    %$ 
    %for some constant $c_1\le 1/2$. 
    Finally, since $f^*$ is the optimal classifier on $\widecheck \calD$, 
    we have 
    \begin{align}
        \error_{\widecheck \calD}(f^*) + \lambda R(f^*) \le \error_{\widecheck \calD}(\widehat f) + \lambda R(\wh f) \,. \label{eq:step3_reg}
    \end{align}
     Now to relate \eqref{eq:step1_reg} and \eqref{eq:step3_reg}, we can re-write the \eqref{eq:step2_reg} as follows: 
    \begin{align}
        \error_{\calS \cup \wt\calS}(f^*) \le \error_{ \widecheck \calD}(f^*) +  \frac{m_1}{m+n}\sqrt{\frac{\log(1/\delta)}{2 m_1}} \,. \label{eq:step4_reg} 
    \end{align}
    After adding $\lambda R(f^*)$ on both sides in \eqref{eq:step4_reg}, we combine equations \eqref{eq:step1_reg}, \eqref{eq:step4_reg}, and \eqref{eq:step3_reg}, to get 
    \begin{align}
        \error_{\calS \cup \wt \calS}(\wh f) \le \error_{\widecheck \calD}(\wh f) +  \frac{m_1}{m+n}\sqrt{\frac{\log(1/\delta)}{2 m_1}} \,, 
    \end{align}
    which implies 
    \begin{align}
        \error_{ \wt \calS_M}(\wh f) \le \error_{\calDm}(\wh f) + \sqrt{\frac{\log(1/\delta)}{2 m_1}} \,. \label{eq:lemma_reg_final}
    \end{align}
    Similar as before, since $\wt S$ is obtained by randomly labeling an unlabeled dataset, we assume 
    $2m_1 \approx m$. Moreover, using $\error_{\calDm} = 1 - \error_{\calD}$ we obtain the desired result. 
\end{proof}
% \begin{proof}[Proof of ]
    
% \end{proof}

\subsection{Proof of \thmref{thm:multiclass_ERM}}

To prove our results in the multiclass case,
we first state and prove lemmas
parallel to those
% We first state and prove lemmas 
% parallel 
% to the three lemmas 
used in the proof of balanced binary case. 
We then combine these results 
% in the three lemmas 
to obtain the result in \thmref{thm:multiclass_ERM}. 

Before stating the result, 
we define mislabeled distribution $\calDm$ for any $\calD$.
While $\calDm$ and $\calD$ share 
the same marginal distribution over inputs $\calX$,
the conditional distribution over labels $y$ 
given an input $x\sim \calD_\calX$ is changed as follows:
For any $x$, the Probability Mass Function (PMF) over $y$ is defined as:  
$p_{\calDm} (\cdot \vert x) \defeq \frac{1 - p_{\calD}(\cdot \vert x)}{k - 1}$, where $ p_{\calD}(\cdot \vert x)$ is the PMF over $y$ for the distribution $\calD$. 

\begin{lemma} \label{lem:fit_mislabeled_multi}
    Assume the same setup as \thmref{thm:multiclass_ERM}. 
    Then for any $\delta >0$, with probability at least  $1-\delta$ 
    over the random draws of mislabeled data $\wt S_M$, we have 
    \begin{align}
        \error_\calD(\widehat f)  \le (k-1)\left(1 -\error_{\wt \calS_M}(\widehat f)\right) + (k-1)\sqrt{\frac{\log(1/\delta)}{m}}\,. \label{eq:lemma1_multi}
    \end{align}   
\end{lemma} 

\begin{proof}
   
    The main idea of the proof remains the same.
    We begin by regarding the clean portion of the data 
    ($S \cup \wt S_C$) as fixed. 
    Then, there exists a classifier $f^*$ 
    that is optimal over draws 
    of the mislabeled data $\wt S_M$. 
    
    However, in the multiclass case,
    we cannot as easily relate the population error on mislabeled data 
    to the population accuracy on clean data.   
    While for binary classification, 
    % we could upper bound $\error_{\wt \calS_M}$ 
    % with $1-\error_\calD$ 
    we could lower bound the population accuracy $1-\error_\calD$
    with the empirical error on mislabeled data $\error_{\wt \calS_M}$ 
    (in the proof of \lemref{lem:fit_mislabeled}), 
    for multiclass classification, 
    error on the mislabeled data 
    and accuracy on the clean data 
    in the population 
    are not so directly related.  
    To establish \eqref{eq:lemma1_multi},
    we break the error on the 
    (unknown) mislabeled data 
    into two parts: one term corresponds 
    to predicting the true label on mislabeled data, 
    and the other corresponds to predicting 
    neither the true label 
    nor the assigned (mis-)label.  
    Finally, we relate these errors to their
    population counterparts to establish \eqref{eq:lemma1_multi}. 
    
    Formally, 
    \begin{align}
    f^* \defeq \argmin_{f \in \calF} \error_{\widecheck {\calD}} (f)  + \lambda R(f) \,, \label{eq:modified_ERM_reg2}
    \end{align}
    where $$\widecheck \calD = \frac{n}{m+n} \calS + \frac{m_1}{m+n} \wt \calS_C  + \frac{m_2}{m+n}\calDm \,.$$ 
    That is, $\widecheck \calD$ is a combination 
    of the \emph{empirical distribution} 
    over correctly labeled data $S \cup \wt S_C$
    % in $S\cup \wt S$ 
    and the (population) distribution 
    over mislabeled data $\calDm$.
    Recall that 
    \begin{align}
    \wh f \defeq \argmin_{f \in \calF} \error_{\calS \cup \wt S} (f) + \lambda R(f) \,. \label{eq:orig_ERM_reg2}
    \end{align}
    % 
    % 
    Following the exact steps from the proof of \lemref{lem:lemma1_reg}, 
    with probability at least $1-\delta$, we have  
    \begin{align}
        \error_{ \wt \calS_M}(\wh f) \le \error_{\calDm}(\wh f) + \sqrt{\frac{\log(1/\delta)}{2 m_1}} \,. \label{eq:lemma1_final_multi_prev}
    \end{align}
    Similar to before, since $\wt S$ is obtained 
    by randomly labeling an unlabeled dataset, 
    we assume 
    $\frac{k}{k-1} m_1 \approx m$. 
    
    Now we will relate $\error_{\calDm} (\wh f)$ with $\error_{\calD}(\wh f)$. 
    Let $y^T$ denote the (unknown) true label 
    for a mislabeled point $(x, y)$ 
    (i.e., label before replacing it with a mislabel). 
    \begin{align*}    
         \Expt{(x, y) \in \sim \calDm}{\indict{ \wh f(x) \ne y }}  &= \underbrace{\Expt{(x, y) \in \sim \calDm}{\indict{ \wh f(x) \ne y \land \wh f(x) \ne y^T}}}_{\RN{1}} \\ &\qquad \qquad + \underbrace{\Expt{(x, y) \in \sim \calDm}{\indict{ \wh f(x) \ne y \land \wh f(x) = y^T}}}_{\RN{2}} \,. \numberthis \label{eq:excess_term}
    \end{align*}
    Clearly, term 2 is one minus the accuracy 
    on the clean unseen data, i.e.,
    \begin{align}
        \RN{2} = 1 - \Expt{{x,y} \sim \calD}{ \indict{ \wh f(x) \ne y}} = 1- \error_{\calD}(\wh f) \,. \label{eq:term1}    
    \end{align}
    Next, we relate term 1 with the error on the unseen clean data. 
    We show that term 1 is equal to the error on the unseen clean data 
    scaled by $\frac{k-2}{k-1}$,
    where $k$ is the number of labels.
    Using the definition of mislabeled distribution $\calDm$,  
    we have 
    \begin{align}
        \RN{1} = \frac{1}{k-1} \left( \Expt{(x, y) \in \sim \calD}{ \sum_{i \in \calY \land i\ne y}  \indict{ \wh f(x) \ne i \land \wh f(x) \ne y}} \right) = \frac{k-2}{k-1} \error_{\calD}(\wh f) \,.\label{eq:term2}
    \end{align}    

    Combining the result in \eqref{eq:term1}, \eqref{eq:term2} and \eqref{eq:excess_term}, we have 
    \begin{align}
        \error_{\calDm}(\wh f) = 1- \frac{1}{k-1} \error_{\calD}(\wh f) \,.\label{eq:combine_terms}
    \end{align}
    Finally, combining the result in \eqref{eq:combine_terms} 
    with equation \eqref{eq:lemma1_final_multi_prev}, 
    we have with probability $1-\delta$, 
    \begin{align}
      \error_{\calD}(\wh f) \le  (k-1) \left( 1- \error_{ \wt \calS_M}(\wh f) \right)  + (k-1) \sqrt{\frac{k \log(1/\delta)}{ 2(k-1)m}} \,. \label{eq:lemma1_final_multi}
    \end{align}
\end{proof}

\begin{lemma} \label{lem:mislabeled_error_multi}
    Assume the same setup as \thmref{thm:multiclass_ERM}. 
    Then for any $\delta >0$, 
    with probability at least $1-\delta$ 
    over the random draws of $\wt S$, we have  
    % \begin{align}
        $$\abs{k\error_{\wt \calS}(\widehat f) - \error_{\wt \calS_C}(\widehat f) -  (k-1)\error_{\wt \calS_M}(\widehat f) } \le  2k\sqrt{\frac{\log(4/\delta)}{2m}}\,. $$ % \label{eq:lemma2}
    % \end{align}   
    %  for some constant $c_3 \le 1.0\,$.
\end{lemma} 


\begin{proof}
    Recall $\error_{\wt S} (f) = \frac{m_1}{m} \error_{\wt S_M}(f) + \frac{m_2}{m} \error_{\wt S_C}(f)$. Hence, we have 
    \begin{align*}
        k\error_{\wt S}(f) - (k-1)\error_{\wt S_M}(f) - \error_{\wt S_C}(f) &= (k-1)\left(\frac{k m_1}{(k-1) m} \error_{\wt S_M}(f) - \error_{\wt S_M}(f)\right) \\ & \qquad \qquad + \left(\frac{km_2}{m} \error_{\wt S_C}(f) - \error_{\wt S_C}(f)\right) \\ &= k \left[ \left(\frac{m_1}{m} - \frac{k-1}{k}\right) \error_{\wt S_M}(f) + \left(\frac{m_2}{m} - \frac{1}{k} \right) \error_{\wt S_C} (f) \right] \,.
    \end{align*} 
    Since the dataset is randomly labeled, 
    we have with probability at least $1-\delta$, 
    $\left(\frac{m_1}{m} - \frac{k-1}{k}\right) \le \sqrt{\frac{\log(1/\delta)}{2m}}$. 
    Similarly, we have with probability at least $1-\delta$, 
    $\left(\frac{m_2}{m} - \frac{1}{k}\right) \le \sqrt{\frac{\log(1/\delta)}{2m}}$. 
    Using union bound, we have with probability at least $1-\delta$
    % \begin{align}
    %     2\error_{\wt S} - \error_{\wt S_M}(f) - \error_{\wt S_C}(f) \le \sqrt{\frac{\log(2/\delta)}{2m}} \left(\error_{\wt S_M}(f) + \error_{\wt S_C}(f) \right) \le 2\sqrt{\frac{\log(2/\delta)}{2m}} \,. \label{eq:lemma2_final}
    % \end{align}
    \begin{align}
        k\error_{\wt S}(f) - (k-1)\error_{\wt S_M}(f) - \error_{\wt S_C}(f)  \le k \sqrt{\frac{\log(2/\delta)}{2m}} \left(\error_{\wt S_M}(f) + \error_{\wt S_C}(f) \right) \,. \label{eq:lemma2_final_multi}
    \end{align}

    % We obtain the desired result by using 
\end{proof}

\begin{lemma} \label{lem:clear_error_multi}
    Assume the same setup as \thmref{thm:multiclass_ERM}. 
    Then for any $\delta >0$, with probability at least $1-\delta$ 
    over the random draws of $\wt S_C$ and $S$, we have 
    % \begin{align}
        $$\abs{\error_{\wt \calS_C}(\widehat f) - \error_{\calS}(\widehat f) } \le 1.5 \sqrt{\frac{k\log(2/\delta)}{2m}}\,.$$ %\label{eq:lemma3}
    % \end{align}   
    % for some constant $c_2 \le 1.2\,$.
\end{lemma} 
\begin{proof}
    % Recall 0-1 error on each point  $(x,y) \in S \cup \wt S$ is given by $\I{ f(x)\ne y}$.
    In the set of correctly labeled points $S \cup \wt S_C$,
    we have $S$ as a random subset of $S \cup \wt S_C$. 
    Hence, using Hoeffding's inequality 
    for sampling without replacement 
    (\lemref{lem:hoeffding_sampling}), 
    we have with probability at least $1-\delta$
    \begin{align}
        \error_{\wt \calS_c} (\wh f)- \error_{\calS \cup \wt \calS_C}( \wh f) \le  \sqrt{\frac{\log(1/\delta)}{2m_2}} \,.
    \end{align}
    Re-writing $\error_{\calS \cup \wt \calS_C}( \wh f)$ 
    as $\frac{m_2}{m_2 + n} \error_{\wt \calS_C }(\wh f) + \frac{n}{m_2 + n} \error_{\calS }(\wh f)$, 
    we have with probability at least $1-\delta$
    \begin{align}
       \left(\frac{n}{n+m_2}\right) \left(\error_{\wt \calS_c} (\wh f)- \error_{\calS}( \wh f) \right) \le  \sqrt{\frac{\log(1/\delta)}{2m_2}} \,.
    \end{align}
    As before, assuming $km_2 \approx m$, 
    we have with probability at least $1-\delta$ 
    \begin{align}
        \error_{\wt \calS_c} (\wh f)- \error_{\calS}( \wh f) \le \left(1+\frac{m_2}{n}\right)  \sqrt{\frac{k\log(1/\delta)}{2m}} \le \left( 1 + \frac{1}{k}\right) \sqrt{\frac{k\log(1/\delta)}{2m}} \,. \label{eq:lemma3_final_multi}
    \end{align} 
\end{proof}

\begin{proof}[Proof of \thmref{thm:multiclass_ERM}] 
    Having established these core intermediate results, 
    we can now combine above three lemmas. 
    In particular, we bound the population error 
    on clean data ($\error_\calD(\wh f)$) as follows:  
    \begin{enumerate}[(i)]
        \item First, use \eqref{eq:lemma1_final_multi}, 
        to obtain an upper bound on the population error on clean data, 
        i.e., with probability at least $1-\delta/4$, we have
        \begin{align}
            \error_{ \calD} (\wh f) \le (k-1)\left(1 - \error_{ \wt \calS_M}(\wh f) \right) + (k-1) \sqrt{\frac{k\log(4/\delta)}{2(k-1)m}} \,. 
        \end{align}
        \item  Second, use \eqref{eq:lemma2_final_multi}
        to relate the error on the mislabeled fraction 
        with error on clean portion of randomly labeled data 
        and error on whole randomly labeled dataset, 
        i.e., with probability at least $1-\delta/2$, we have 
        \begin{align}
            - (k-1)\error_{\wt S_M}(f) \le \error_{\wt S_C}(f) - k\error_{\wt S}  + k\sqrt{\frac{\log(4/\delta)}{2m}}  \,. 
        \end{align} 
        \item Finally, use \eqref{eq:lemma3_final_multi} 
        to relate the error on the clean portion of randomly labeled data 
        and error on clean training data, 
        i.e., with probability $1-\delta/4$, we have 
        \begin{align}
            \error_{\wt \calS_C} (\wh f)\le - \error_{\calS}( \wh f) + \left(1 + \frac{m}{kn} \right) \sqrt{\frac{k\log(4/\delta)}{2m}} \,. 
        \end{align} 
    \end{enumerate}

    Using union bound on the above three steps, 
    we have with probability at least $1-\delta$: 
    \begin{align}
        \error_\calD (\wh f) \le \error_{\calS}(\wh f) + (k-1) - k\error_{\wt \calS}(\wh f)   + (\sqrt{k(k-1)} + k + \sqrt{k} + \frac{m}{n\sqrt{k}})  \sqrt{\frac{\log(4/\delta)}{2m}} \,.\label{eq:multiclass_ERM_final}
    \end{align}
    Simplifying the term in RHS of \eqref{eq:multiclass_ERM_final}, 
    we get the desired result. 
    % Note that since $\frac{m}{n\sqrt{k}}$ 
    % is much smaller than the sum of the other terms
    % the other terms in summation, 
    % we ignore $\frac{m}{n\sqrt{k}}$  
    % Z: ??? --- great
    % that 
    % them
    in the final bound. 
    % we ignore that in the final bound. 
    % Note that $(1/\sqrt{2} + 2.5)$ is a loose constant. In experiments, we use the ratio $\frac{m}{n}$
    %  the exact error $\error_{\wt \calS}(\wh f)$ 
    % to evaluate R.H.S.    
\end{proof}

\newpage
\section{Proofs from \secref{sec:linear_models}}\label{app:proof_gd}
We suppose that the parameters of the linear function 
are obtained via gradient descent on 
the following $L_2$ regularized problem: 
\begin{align}
    % n in denominator is avoided deliberately
    \calL_S(w; \lambda) \defeq \sum_{i=1}^n{(w^Tx_i - y_i)^2} + \lambda \norm{w}{2}^2 \,, \label{eq:l2_MSE_app}   
\end{align}
where $\lambda\ge0$ is a regularization parameter. 
We assume access to a clean dataset 
$S = \{(x_i, y_i)\}_{i=1}^n \sim \calD^n$ 
and randomly labeled dataset 
$\wt S = \{(x_i, y_i)\}_{i=n+1}^{n+m} \sim \wt \calD^m$. 
Let $\bX = [x_1, x_2, \cdots, x_{m+n}]$ 
and $\by = [y_1, y_2, \cdots, y_{m+n}]$. 
Fix a positive learning rate $\eta$ such that 
$\eta \le 1/\left(\norm{\bX^T\bX}{\text{op}} + \lambda^2\right)$ 
and an initialization $w_0 = 0$. 
% \todos{Assumption made for simplicty}. 
Consider the following gradient descent iterates 
to minimize objective \eqref{eq:l2_MSE_app} on $S \cup \wt S$:
\begin{align}
w_t = w_{t-1} - \eta \grad_w \calL_{S \cup \wt S} (w_{t-1}; \lambda) \quad \forall t=1,2,\ldots \label{eq:GD_iterates_app}
\end{align} 
Then we have $\{ w_t\}$ converge to the limiting solution 
$\wh w = \left( \bX^T\bX+\lambda \boldsymbol{I}\right)^{-1}\bX^T\by$. Define $\widehat f (x) \defeq f(x ; \wh w) $.  

% \subsection{\textcolor{red}{Errata}}

% We wish to correct the following error in the body:
% \codref{cond:error_stability} is not enough 
% to guarantee the result in \thmref{thm:linear}. 
% We now present a slightly stronger condition 
% called \emph{hypothesis stability} 
% under which we obtain a result 
% similar to \thmref{thm:linear}. 

% This error doesn't change the main arguments of the proof,
% where we show that the empirical train error 
% is less than or equal to the leave-one-out error.
% We need a stronger condition to relate leave-one-out error 
% with the population error of the original classifier. 
% Specifically, while \codref{cond:error_stability} 
% relates the average population error of leave-one-out classifiers 
% with the population error of the original classifier, 
% we need the new condition to show the concentration 
% of the empirical leave-one-out error 
% and average population error of leave-one-out classifiers. 
% main takeaway 

% Note that the new condition, 
% while being stronger than the previous one, 
% still doesn't imply generalization \citep{bousquet2002stability,elisseeff2003leave,abou2019exponential}. 
% Overall, the main results in \secref{sec:ERM_training} 
% and takeaways of the paper remain unaffected by the error.  

% We now present the new condition 
% and a corrected statement of \thmref{thm:linear}. 
% Recall, for a given training set $S \sim \calD^n $, 
% we use $S_{(i)}$ to denote the training set $S$ 
% with the $i^{\text{th}}$ point removed.

% \begin{condition}[Hypothesis Stability] 
%     \label{cond:hypothesis_stability}
%     We have $\beta$ hypothesis stability 
%     if our training algorithm $\calA$ satisfies the following: 
%     \begin{align*}
%     % ${\sum_{i=1}^n \frac{\error_{\calD}( f(\calA, S_{(i)}))}{n} - \error_\calD(f(\calA, S))} \le \beta\,$.
%     \forall i \in \{1,2,\ldots, n\}, \quad  \Expt{\calS, (x,y) \in \calD}{ \abs{\error\left( f(x) ,y  \right) - \error\left( f_{(i)}(x), y \right) }} \le \frac{\beta}{n} \,,
%     \end{align*}
%     where $f_{(i)} \defeq f(\calA, S_{(i)})$ and $ f \defeq f(\calA, S)$.
% \end{condition}

% \begin{theorem}[Correct statement of \thmref{thm:linear}] \label{thm:new_linear}
%     Assume that this gradient descent algorithm satisfies \codref{cond:hypothesis_stability}
%     with $\beta=\calO(1)$.  
%     Then for any $\delta >0$, with probability at least $1-\delta$ 
%     over the random draws of datasets $\wt S$ and $S$, we have:
%     \begin{align}
%         \error_\calD(\widehat f) \le \error_\calS(\widehat f) + 1 - 2 \error_{\wt\calS}(\widehat f) + \left(\frac{1}{\sqrt{2}} + 1.5 \right) \sqrt{\frac{\log(4/\delta)}{m}} + \sqrt{\frac{4}{\delta}\left(\frac{1}{m} +\frac{3\beta}{m+n} \right)}  \,. \label{eq:gd_error}
%     \end{align} 
%     % for some constant $c\le 3.2$.
% \end{theorem}

\subsection{Proof of \thmref{thm:linear}}
We use a standard result from linear algebra, 
namely the Shermann-Morrison formula 
\citep{sherman1950adjustment} for matrix inversion:  

\begin{lemma}[\citet{sherman1950adjustment}] \label{lem:sherman}
    Suppose $\bA \in \Real^{n \times n}$ 
    is an invertible square matrix 
    and $u,v \in \Real^n$ are column vectors. 
    Then $\bA + uv^T$ is invertible iff $1 + v^T \bA u \ne 0$ 
    and in particular
    \begin{align}
        (\bA + u v^T)^{-1} = \bA^{-1}  - \frac{\bA^{-1} uv^T \bA^{-1} }{ 1 + v^T \bA^{-1} u} \,.
    \end{align}   
\end{lemma}
\newcommand\byy[1]{\by_{\left(#1\right)}}
\newcommand\bXX[1]{\bX_{\left(#1\right)}}
\newcommand\ff[1]{\wh f_{\left(#1\right)}}

For a given training set $S \cup \wt S_C$, 
define leave-one-out error 
on mislabeled points in the training data 
as $$\error_{\text{LOO}(\wt S_M) } = \frac{\sum_{(x_i, y_i) \in \wt S_M} \error( f_{(i)}( x_i), y_i)}{ \abs{\wt S_M }} \,, $$
where $f_{(i)} \defeq f(\calA, (S \cup \wt S)_{(i)})$. 
To relate empirical leave-one-out error and population error 
with hypothesis stability condition, 
we use the following lemma:   

\begin{lemma}[\citet{bousquet2002stability}] \label{lem:stability_error}
    For the leave-one-out error, we have
    \begin{align}
        \Expo{ \left( \error_{\calDm}(\wh f) -\error_{\text{LOO}(\wt S_M) } \right)^2 } \le \frac{1}{2m_1}+  \frac{3\beta}{n + m}\,.
    \end{align}   
    % where $ f \defeq f(\calA, S \cup \wt S) $.
\end{lemma}

Proof of the above lemma is similar 
to the proof of Lemma 9 in \citet{bousquet2002stability} 
and can be found in \appref{app:proof_lem_error}. 
% 
% Before presenting the result, we introduce some notation. 
Before presenting the proof of \thmref{thm:linear}, 
we introduce some more notation. 
Let $\bX_{(i)}$ denote the matrix of covariates 
with the $i^{\text{th}}$ point removed. 
Similarly, let $\by_{(i)}$ be the array of responses 
with the $i^{\text{th}}$ point removed. 
Define the corresponding regularized GD solution 
as $\wh w_{(i)} = \left( \bXX{i}^T\bXX{i}+\lambda \boldsymbol{I}\right)^{-1}\bXX{i}^T\byy{i}$. 
Define $\ff{i}(x) \defeq f(x ; \wh w_{(i)}) $.

\begin{proof}[Proof of \thmref{thm:linear}]
    Because squared loss minimization does not imply 0-1 error minimization, 
    we cannot use arguments from \lemref{lem:fit_mislabeled}. 
    This is the main technical difficulty. 
    To compare the 0-1 error at a train point with an unseen point, 
    we use the closed-form expression for $\widehat{w}$ 
    and Shermann-Morrison formula 
    to upper bound training error 
    with leave-one-out cross validation error. 
    
    The proof is divided into three parts: 
    In part one, we show that 0-1 error 
    on mislabeled points in the training set 
    is lower than the error obtained 
    by leave-one-out error at those points. 
    In part two, we relate this leave-one-out error 
    with the population error on mislabeled distribution
    using \codref{cond:hypothesis_stability}.
    While the empirical leave-one-out error is an unbiased estimator 
    of the average population error of leave-one-out classifiers, 
    we need hypothesis stability 
    to control the variance 
    of empirical leave-one-out error. 
    Finally, in part three, we show 
    that the error on the mislabeled training points 
    can be estimated with just the randomly labeled 
    and clean training data (as in proof of \thmref{thm:error_ERM}).  

    \textbf{Part 1 {} {}} First we relate training error with leave-one-out error.        
    For any training point $(x_i, y_i)$ in $\wt S \cup S$, we have 
    \begin{align}
        \error(\wh f(x_i), y_i ) &= \indict{ y_i \cdot x_i^T \wh w < 0 } = \indict{ y_i \cdot x_i^T \left( \bX^T\bX+\lambda \boldsymbol{I}\right)^{-1}\bX^T\by < 0 } \\
        &= \indict{ y_i \cdot x_i^T \underbrace{\left( \bXX{i}^T\bXX{i} + x_i ^T x_i +\lambda \boldsymbol{I}\right)^{-1}}_{\RN{1}} (\bXX{i}^T\byy{i} + y_i \cdot x_i) < 0 } \,.
    \end{align}
    Letting $\bA = \left(\bXX{i}^T\bXX{i} +\lambda \boldsymbol{I}\right)$ 
    and using \lemref{lem:sherman} on term 1, we have 
    \begin{align}
        \error(\wh f(x_i), y_i ) &= \indict{ y_i \cdot x_i^T \left[\bA^{-1} -  \frac{\bA^{-1} x_i x_i^T \bA^{-1}}{ 1 + x_i ^T \bA^{-1} x_i } \right] (\bXX{i}^T\byy{i} + y_i \cdot x_i) < 0 } \\
        &= \indict{ y_i \cdot\left[ \frac{ x_i^T \bA^{-1} ( 1 + x_i ^T \bA^{-1} x_i ) -  x_i^T \bA^{-1} x_i x_i^T \bA^{-1}}{ 1 + x_i ^T \bA ^{-1}x_i } \right] (\bXX{i}^T\byy{i} + y_i \cdot x_i) < 0 } \\
        &= \indict{ y_i \cdot\left[ \frac{ x_i^T \bA^{-1}}{ 1 + x_i ^T \bA ^{-1}x_i } \right] (\bXX{i}^T\byy{i} + y_i \cdot x_i) < 0 } \,.
    \end{align}

    Since $1 + x_i^T \bA^{-1} x_i > 0$, we have 
    \begin{align}
        \error(\wh f(x_i), y_i ) &= \indict{ y_i \cdot x_i^T \bA^{-1} (\bXX{i}^T\byy{i} + y_i \cdot x_i) < 0 } \\
        &= \indict{ x_i^T \bA^{-1} x_i +  y_i \cdot x_i^T \bA^{-1} (\bXX{i}^T\byy{i}) < 0 } \\
        &\le \indict{ y_i \cdot x_i^T \bA^{-1} (\bXX{i}^T\byy{i}) < 0 } = \error(\ff{i}(x_i), y_i ) \,.\label{eq:LOO_error}
    \end{align}

    Using \eqref{eq:LOO_error}, we have 
    \begin{align}
        \error_{\wt \calS_M } (\wh f) \le \error_{\text{LOO} (\wt S_M)} \defeq \frac{\sum_{(x_i, y_i) \in \wt S_M} \error(\ff{i}(x_i), y_i ) }{\abs{\wt \calS_M}}\label{eq:LOO_error_final} \,.
    \end{align}
    \textbf{Part 2 {}{}} We now relate RHS in \eqref{eq:LOO_error_final} 
    with the population error on mislabeled distribution. 
    To do this, we leverage \codref{cond:hypothesis_stability} 
    and \lemref{lem:stability_error}. 
    In particular, we have 

    \begin{align}
        \Expt{\calS \cup \wt \calS_M }{ \left(\error_{\calDm}(\wh f) - \error_{\text{LOO} (\wt S_M)}\right)^2 } \le \frac{1}{2m_1} + \frac{3\beta}{m+n} \,.
    \end{align}

    Using Chebyshev's inequality, with probability at least $1-\delta$, we have 
    \begin{align}
        \error_{\text{LOO} (\wt S_M)} \le  \error_{\calDm}(\wh f)   + \sqrt{\frac{1}{\delta}\left(\frac{1}{2m_1} +\frac{3\beta}{m+n} \right)} \,. \label{eq:final_mislabeled_linear}
    \end{align}
    

    \textbf{Part 3 {}{}} Combining \eqref{eq:final_mislabeled_linear} and \eqref{eq:LOO_error_final}, we have 

    \begin{align}
        \error_{\wt \calS_M } (\wh f) \le \error_{\calDm}(\wh f)   + \sqrt{\frac{1}{\delta}\left(\frac{1}{2m_1} +\frac{3\beta}{m+n} \right)} \,. \label{eq:linear_parallel_lem1}
    \end{align}

    Compare \eqref{eq:linear_parallel_lem1} with \eqref{eq:lemma1_final} 
    in the proof of \lemref{lem:fit_mislabeled}. 
    We obtain a similar relationship 
    between $\error_{\wt \calS_M }$ and $\error_{\calDm}$ 
    but with a polynomial concentration 
    instead of exponential concentration. 
    In addition, since we just use concentration arguments 
    to relate mislabeled error to the errors
    on the clean and unlabeled portions 
    of the randomly labeled data, 
    we can directly use the results 
    in \lemref{lem:mislabeled_error} and \lemref{lem:clear_error}. 
    Therefore, combining results in \lemref{lem:mislabeled_error}, \lemref{lem:clear_error}, and \eqref{eq:linear_parallel_lem1} with union bound, 
    we have with probability at least $1-\delta$
    \begin{align}
        \error_\calD(\widehat f) \le \error_\calS(\widehat f) + 1 - 2 \error_{\wt\calS}(\widehat f) + \left(\sqrt{2}\error_{\wt\calS}(\widehat f) + 1 + \frac{m}{2n} \right) \sqrt{\frac{\log(4/\delta)}{m}} + \sqrt{\frac{4}{\delta}\left(\frac{1}{m} +\frac{3\beta}{m+n} \right)}  \,.
    \end{align}
    

       
\end{proof}

\subsection{Extension to multiclass classification} \label{app:multiclass_linear}
For multiclass problems with squared loss minimization, as standard practice, we consider one-hot encoding for the underlying label, i.e., a class label $c \in [k]$ is treated as $(0, \cdot, 0,1,0, \cdot, 0) \in \Real^k$ (with $c$-th coordinate being 1).  As before, we suppose that the parameters of the linear function 
are obtained via gradient descent on the following $L_2$ regularized problem: 
\begin{align}
    % n in denominator is avoided deliberately
    \calL_S(w; \lambda) \defeq \sum_{i=1}^n\norm{w^Tx_i - y_i}{2}^2 + \lambda \sum_{j=1}^k \norm{w_j}{2}^2 \,, \label{eq:l2_multiclass_MSE_app}   
\end{align}
where $\lambda\ge0$ is a regularization parameter. 
We assume access to a clean dataset 
$S = \{(x_i, y_i)\}_{i=1}^n \sim \calD^n$ 
and randomly labeled dataset 
$\wt S = \{(x_i, y_i)\}_{i=n+1}^{n+m} \sim \wt \calD^m$. 
Let $\bX = [x_1, x_2, \cdots, x_{m+n}]$ 
and $\by = [e_{y_1}, e_{y_2}, \cdots, e_{y_{m+n}}]$. 
Fix a positive learning rate $\eta$ such that 
$\eta \le 1/\left(\norm{\bX^T\bX}{\text{op}} + \lambda^2\right)$ 
and an initialization $w_0 = 0$. 
% \todos{Assumption made for simplicty}. 
Consider the following gradient descent iterates 
to minimize objective \eqref{eq:l2_MSE_app} on $S \cup \wt S$:
\begin{align}
{w_j}^t = {w_j}^{t-1} - \eta \grad_{w_j} \calL_{S \cup \wt S} (w^{t-1}; \lambda) \quad \forall t=1,2,\ldots \text{ and } j=1,2,\ldots,k  \,. \label{eq:GD_multi_iterates_app}
\end{align} 
Then we have $\{ {w_j}^t\}$ for all $j =1,2,\cdots, k$ converge to the limiting solution 
$\wh w_j = \left( \bX^T\bX+\lambda \boldsymbol{I}\right)^{-1}\bX^T\by_j$. Define $\widehat f (x) \defeq f(x ; \wh w) $.  

\begin{theorem}\label{thm:multi_linear}
    Assume that this gradient descent algorithm satisfies \codref{cond:hypothesis_stability}
    with $\beta=\calO(1)$.  
    Then for a multiclass classification problem wth $k$ classes, for any $\delta >0$, with probability at least $1-\delta$, we have:
    \begin{align*}
        \error_\calD(\widehat f) \le \error_\calS(\widehat f) &+ (k-1)\left(1 - \frac{k}{k-1} \error_{\wt\calS}(\widehat f) \right) \\ &+ \left(k + \sqrt{k} + \frac{m}{n\sqrt{k}} \right) \sqrt{\frac{\log(4/\delta)}{2m}} + \sqrt{k(k-1)} \sqrt{\frac{4}{\delta}\left(\frac{1}{m} +\frac{3\beta}{m+n} \right)}  \,. \numberthis \label{eq:gd_multi_error}
    \end{align*} 
    % for some constant $c\le 3.2$.
\end{theorem}
\begin{proof}
    The proof of this theorem is divided into two parts. In the first part, we relate the error on the mislabeled samples with the population error on the mislabeled data. Similar to the proof of \thmref{thm:linear}, we use Shermann-Morrison formula to upper bound training error with leave-one-out error on each $\wh w^j$. Second part of the proof follows entirely from the proof of \thmref{thm:multiclass_ERM}. In essence, the first part derives an equivalent of \eqref{eq:lemma1_final_multi_prev} for GD training with squared loss and then the second part follows from the proof  of \thmref{thm:multiclass_ERM}. 
    
    \textbf{Part-1:} Consider a training point $(x_i,y_i)$ in $\wt S \cup S $. For simplicity, we use $c_i$ to denote the class of $i$-th point and use $y_i$ as the corresponding one-hot embedding. Recall error in multiclass point is given by $\error(\wh f(x_i), y_i ) = \indict{ c_i \not \in \argmax x_i^T \wh w }$. Thus, there exists a $j \ne c_i \in [k]$, such that we have
     \begin{align}
        \error(\wh f(x_i), y_i ) &= \indict{ c_i \not \in \argmax x_i^T \wh w } = \indict{ x_i^T \wh w_{c_i} < x_i^T \wh w_{j}  } \\ &= \indict{ x_i^T \left( \bX^T\bX+\lambda \boldsymbol{I}\right)^{-1}\bX^T\by_{c_i} < x_i^T \left( \bX^T\bX+\lambda \boldsymbol{I}\right)^{-1}\bX^T\by_{j} } \\
        &= \indict{ x_i^T \underbrace{\left( \bXX{i}^T\bXX{i} + x_i ^T x_i +\lambda \boldsymbol{I}\right)^{-1}}_{\RN{1}} \left(\bXX{i}^T{\by_{c_i}}_{(i)} + x_i - \bXX{i}^T{\by_{j}}_{(i)}\right) < 0 } \,.
    \end{align}
    Letting $\bA = \left(\bXX{i}^T\bXX{i} +\lambda \boldsymbol{I}\right)$ 
    and using \lemref{lem:sherman} on term 1, we have 
    \begin{align}
        \error(\wh f(x_i), y_i ) &= \indict{ x_i^T \left[\bA^{-1} -  \frac{\bA^{-1} x_i x_i^T \bA^{-1}}{ 1 + x_i ^T \bA^{-1} x_i } \right]  \left(\bXX{i}^T{\by_{c_i}}_{(i)} + x_i - \bXX{i}^T{\by_{j}}_{(i)}\right) < 0 } \\
        &= \indict{ \left[ \frac{ x_i^T \bA^{-1} ( 1 + x_i ^T \bA^{-1} x_i ) -  x_i^T \bA^{-1} x_i x_i^T \bA^{-1}}{ 1 + x_i ^T \bA ^{-1}x_i } \right]  \left(\bXX{i}^T{\by_{c_i}}_{(i)} + x_i - \bXX{i}^T{\by_{j}}_{(i)}\right) < 0 } \\
        &= \indict{ \left[ \frac{ x_i^T \bA^{-1}}{ 1 + x_i ^T \bA ^{-1}x_i } \right]  \left(\bXX{i}^T{\by_{c_i}}_{(i)} + x_i - \bXX{i}^T{\by_{j}}_{(i)}\right) < 0} \,.
    \end{align}
    Since $1 + x_i^T \bA^{-1} x_i > 0$, we have 
    \begin{align}
        \error(\wh f(x_i), y_i ) &= \indict{ x_i^T \bA^{-1}  \left(\bXX{i}^T{\by_{c_i}}_{(i)} + x_i - \bXX{i}^T{\by_{j}}_{(i)}\right) < 0 } \\
        &= \indict{ x_i^T \bA^{-1} x_i +  x_i^T \bA^{-1}  \bXX{i}^T{\by_{c_i}}_{(i)}  - x_i^T\bA^{-1}  \bXX{i}^T{\by_{j}}_{(i)} < 0 } \\
        &\le \indict{  x_i^T \bA^{-1}  \bXX{i}^T{\by_{c_i}}_{(i)}  - x_i^T\bA^{-1}  \bXX{i}^T{\by_{j}}_{(i)} < 0  } = \error(\ff{i}(x_i), y_i ) \,.\label{eq:LOO_error_multi}
    \end{align}
    Using \eqref{eq:LOO_error_multi}, we have 
    \begin{align}
        \error_{\wt \calS_M } (\wh f) \le \error_{\text{LOO} (\wt S_M)} \defeq \frac{\sum_{(x_i, y_i) \in \wt S_M} \error(\ff{i}(x_i), y_i ) }{\abs{\wt \calS_M}}\label{eq:LOO_error_multi_final} \,.
    \end{align}
    
    We now relate RHS in \eqref{eq:LOO_error_final} 
    with the population error on mislabeled distribution. 
    Similar as before, to do this, we leverage \codref{cond:hypothesis_stability} 
    and \lemref{lem:stability_error}. Using  \eqref{eq:final_mislabeled_linear} and \eqref{eq:LOO_error_multi_final}, we have 
    \begin{align}
        \error_{\wt \calS_M } (\wh f) \le \error_{\calDm}(\wh f)   + \sqrt{\frac{1}{\delta}\left(\frac{1}{2m_1} +\frac{3\beta}{m+n} \right)} \,. \label{eq:linear_multi_parallel_lem1}
    \end{align}
    
    We have now derived a parallel to \eqref{eq:lemma1_final_multi_prev}. Using the same arguments in the proof of \lemref{lem:fit_mislabeled_multi}, we have 
    \begin{align}
      \error_{\calD}(\wh f) \le  (k-1) \left( 1- \error_{ \wt \calS_M}(\wh f) \right)  + (k-1)\sqrt{\frac{k}{\delta(k-1)}\left(\frac{1}{2m_1} +\frac{3\beta}{m+n} \right)}  \,. \label{eq:lemma1_linear_final_multi}
    \end{align}
    
    \textbf{Part-2:} We now combine the results in \lemref{lem:mislabeled_error_multi} and \lemref{lem:clear_error_multi} to obtain the final inequality in terms of quantities that can be computed from just the randomly labeled and clean data. Similar to the binary case, we obtained a polynomial concentration instead of exponential concentration. Combining \eqref{eq:lemma1_linear_final_multi} with \lemref{lem:mislabeled_error_multi} and \lemref{lem:clear_error_multi}, we have with probability at least $1-\delta$
    \begin{align*}
        \error_\calD(\widehat f) \le \error_\calS(\widehat f) &+ (k-1)\left(1 - \frac{k}{k-1} \error_{\wt\calS}(\widehat f) \right) \\ &+ \left(k + \sqrt{k} + \frac{m}{n\sqrt{k}} \right) \sqrt{\frac{\log(4/\delta)}{2m}} + \sqrt{k(k-1)} \sqrt{\frac{4}{\delta}\left(\frac{1}{m} +\frac{3\beta}{m+n} \right)}  \,. \numberthis \label{eq:gd_multi_error_proof}
    \end{align*} 
\end{proof}

\subsection{Discussion on \codref{cond:hypothesis_stability}} \label{app:discuss_cond1}
The quantity in LHS of \codref{cond:hypothesis_stability} 
measures how much the function learned by the algorithm 
(in terms of error on unseen point) will change 
when one point in the training set is removed. 
% Discussion on exponential concentration and stronger condition. 
% Notice that hypothesis stability implies error stability, i.e., \codref{cond:error_stability} \citep{bousquet2002stability}.  
% In summary, while error stability allowed us 
% to relate the average population error 
% of the leave-one-out classifiers 
% with the population error of the original classifier, 
We need hypothesis stability condition 
to control the variance of the empirical leave-one-out error to show concentration of average leave-one-error with the population error. 

Additionally, we note that while the dominating term in the RHS of \thmref{thm:linear} matches with the dominating term in ERM bound in \thmref{thm:error_ERM}, there is a polynomial concentration term 
(dependence on $1/\delta$ instead of $\log(\sqrt{1/\delta})$) 
in \thmref{thm:linear}. 
Since with hypothesis stability, 
we just bound the variance, 
the polynomial concentration is due 
to the use of Chebyshev's inequality 
instead of an exponential tail inequality
(as in \lemref{lem:fit_mislabeled}).
Recent works have highlighted that 
a slightly stronger condition than hypothesis stability 
can be used to obtain an exponential concentration 
for leave-one-out error \citep{abou2019exponential},
but we leave this for future work for now. 
% We leave 
% However, the constants 

% we also want to highlight  

\subsection{Formal statement and proof of \propref{prop:early_stop}} \label{app:formal_early_stop}

Before formally presenting the result, 
we will introduce some notation.  
By $\calL_{S}(w)$, we denote 
the objective in \eqref{eq:l2_MSE_app} with $\lambda=0$. 
Assume Singular Value Decomposition (SVD) of $\bX$
as $\sqrt{n} \bU \bS^{1/2} \bV^T$. 
Hence $\bX^T \bX = \bV \bS \bV^T$.
Consider the GD iterates defined in \eqref{eq:GD_iterates_app}. 
% 
We now derive closed form expression 
for the $t^\text{th}$ iterate of gradient descent:  
% 
\begin{align}
    w_t = w_{t-1} + \eta \cdot \bX^T (\by - \bX w_{t-1}) = (\bI - \eta \bV \bS \bV^T )w_{k-1} + \eta \bX^T \by \,.
\end{align}
Rotating by $\bV^T$, we get 
\begin{align}
    \wt w_t = (\bI - \eta\bS )\wt w_{k-1} + \eta \wt \by \label{eq:GD_recur},
\end{align}
where $\wt w_t = \bV^T w_t $ and $\wt \by = \bV^T \bX^T \by$. 
Assuming the initial point $w_0 = 0$ 
and applying the recursion in \eqref{eq:GD_recur}, we get
\begin{align}
    \wt w_t = \bS ^{-1} ( \bI - (\bI - \eta \bS)^k ) \wt \by \,, 
\end{align} 
Projecting solution back to the original space, we have 
\begin{align}
     w_t = \bV \bS ^{-1} ( \bI - (\bI - \eta \bS)^k ) \bV^T \bX^T \by \,. 
\end{align} 
% We will work with this GD solution at any iterate $t$ in the next proposition. 
Define $f_t(x) \defeq f(x;w_t)$ 
as the solution at the $t^{\text{th}}$ iterate. 
Let $\wt w_{\lambda} = \argmin_{w} \calL_\calS (w;\lambda) = (\bX^T \bX + \lambda \bI)^{-1} \bX^T \by = \bV (\bS + \lambda \bI )^{-1} \bV^T \bX^T \by $. 
% ) \,,$ for all $t=1,2,\ldots\,.$ 
and define $\wt f_\lambda(x) \defeq f(x;\wt w_\lambda)$ as the regularized solution. 
Assume $\kappa$ be the condition number 
of the population covariance matrix 
and let $s_\text{min}$ be the minimum positive 
singular value of the empirical covariance matrix. 
Our proof idea is inspired from recent work 
on relating gradient flow solution 
and regularized solution 
for regression problems \citep{ali2018continuous}. 
We will use the following lemma in the proof: 
\begin{lemma} \label{lem:ineq_soln}
    For all $x \in [0,1]$ and for all $ k \in \mathbb{N}$, 
    we have (a) $ \frac{kx}{1+kx} \le 1- (1-x)^k$ 
    and (b) $ 1- (1-x)^k \le 2 \cdot \frac{kx}{kx+1} $.
    %  where $g(c)$ is a constant dependent on $c$. For $c = 1$, $g(c) = 2.0$.   
\end{lemma}
\begin{proof}
    % [Proof of \lemref{lem:ineq_soln}]
    % Part (a) is easy. 
    Using $ (1-x)^k \le \frac{1}{1+kx}$, we have part (a). 
    For part (b), we numerically maximize 
    $\frac{ (1+kx ) (1 - (1-x)^k) }{kx}$ 
    for all $k\ge 1$ and for all $x \in [0, 1]$.  
\end{proof}

% 
% Next, 

\begin{prop}[Formal statement of \propref{prop:early_stop}] \label{prop:formal_early_stop}
Let $\lambda = \frac{1}{t\eta}$. 
For a training point $x$, we have 
\begin{align*}
    \Expt{x \sim \calS}{(f_t(x) - \wt f_\lambda(x))^2} &\le c(t,\eta) \cdot \Expt{x \sim \calS}{f_t(x)^2} \,, %\label{eq:early_stop}
\end{align*}
where $c(t, \eta) \defeq \min( 0.25, \frac{1}{s_\text{min}^2 t^2 \eta^2})$. 
Similarly for a test point, we have 
\begin{align*}
    \Expt{x \sim \calD_\calX}{(f_t(x) - \wt f_\lambda(x))^2} &\le \kappa \cdot c(t,\eta) \cdot \Expt{x \sim \calD_\calX}{f_t(x)^2} \,. %\label{eq:early_stop}
\end{align*}
\end{prop} 

\begin{proof}
    %%%%%%%%%%%%% 
    We want to analyze the expected squared difference output 
    of regularized linear regression 
    with regularization constant $\lambda = \frac{1}{\eta t}$ 
    and the gradient descent solution at the $t^\text{th}$ iterate. 
    We separately expand the algebraic expression 
    for squared difference at a training point and a test point. 
    % We start by considering the difference  
    Then the main step is to show that 
    $\left[ \bS ^{-1} ( \bI - (\bI - \eta \bS)^k )  - (\bS + \lambda \bI )^{-1}\right] \preceq c(\eta, t) \cdot \bS ^{-1} ( \bI - (\bI - \eta \bS)^k ) $.

    %%%%%%%%%%%%%
    
   \textbf{Part 1 {} {}} 
    First, we will analyze the squared difference 
    of the output at a training point 
    (for simplicity, we refer to $S \cup \wt S$ as $S$), i.e., 
    \begin{align}
        \Expt{ x \sim \calS }{\left(f_t(x) - \wt f_\lambda (x)\right)^2} &= \norm{\bX w_t - \bX \wt w_\lambda}{2}^2\\ &=   \norm{\bX \bV \bS ^{-1} ( \bI - (\bI - \eta \bS)^t ) \bV^T \bX^T \by - \bX \bV (\bS + \lambda \bI )^{-1} \bV^T \bX^T \by }{2}^2 \\
        &= \norm{\bX \bV \left(\bS ^{-1} ( \bI - (\bI - \eta \bS)^t ) - (\bS + \lambda \bI )^{-1} \right) \bV^T \bX^T \by  }{2} \\
        &=  \by^T \bV \bX \left( \underbrace{\bS ^{-1} ( \bI - (\bI - \eta \bS)^t ) - (\bS + \lambda \bI )^{-1}}_{\RN{1}} \right)^2 \bS \bV^T \bX^T \by \label{eq:train_GD_rel} \,.
        %  (\bX \bV \bS ^{-1} ( \bI - (\bI - \eta \bS)^k ) \bV^T \bX^T \by)^T \bX \bV \bS ^{-1} ( \bI - (\bI - \eta \bS)^k ) \bV^T \bX^T \by
    \end{align}
    We now separately consider term 1. 
    Substituting $\lambda = \frac{1}{t \eta}$, 
    we get
    \begin{align}
        \bS ^{-1} ( \bI - (\bI - \eta \bS)^t ) - (\bS + \lambda \bI )^{-1} &= \bS^{-1} \left( ( \bI - (\bI - \eta \bS)^t ) - (\bI + \bS^{-1} \lambda )^{-1}\right) \\
        &= \underbrace{\bS^{-1} \left( ( \bI - (\bI - \eta \bS)^t ) - (\bI + ( \bS t \eta)^{-1}  )^{-1}\right)}_{\bA} \,.
    \end{align}

    We now separately bound the diagonal entries in matrix $\bA$. 
    With $s_i$, we denote $i^{\text{th}}$ diagonal entry of $\bS$.
    Note that since $ \eta\le 1/\norm{S}{\text{op}}$, 
    for all $i$, $\eta s_i  \le 1$.  
    Consider $i^{\text{th}}$ diagonal term (which is non-zero) 
    of the diagonal matrix $\bA$, we have 
    \begin{align}
        \bA_{ii} = \frac{1}{s_i} \left(  1 - (1 - s_i \eta)^t - \frac{t \eta s_i}{1 + t \eta s_i } \right) &=  \frac{1 - (1 - s_i \eta)^t}{s_i} \left( \underbrace{ 1 - \frac{t \eta s_i}{(1 + t \eta s_i)(1 - (1 - s_i \eta)^t)}}_{\RN{2}} \right) \\ 
         &\le \frac{1}{2}\left[ \frac{1 - (1 - s_i \eta)^t}{ s_i} \right] \tag*{(Using \lemref{lem:ineq_soln} (b))} \,.
    \end{align} 
    Additionally, we can also show the following upper bound on term 2: 
    \begin{align}
         1 - \frac{t \eta s_i}{(1 + t \eta s_i)(1 - (1 - s_i \eta)^t)} &= \frac{(1 + t \eta s_i)(1 - (1 - s_i \eta)^t) - t \eta s_i }{(1 + t \eta s_i)(1 - (1 - s_i \eta)^t)} \\
         & \le  \frac{ 1 -  (1 - s_i \eta)^t - t \eta s_i (1 - s_i \eta)^t}{(1 + t \eta s_i)(1 - (1 - s_i \eta)^t)} \\
         & \le \frac{1}{t\eta s_i} \,. \tag{Using \lemref{lem:ineq_soln} (a)}
        %  &\le \frac{1}{2}\left[ \frac{1 - (1 - s_i \eta)^t}{ s_i} \right] \tag*{(Using \lemref{lem:ineq_soln})} \,.
    \end{align} 

    Combining both the upper bounds 
    on each diagonal entry $\bA_{ii}$, we have 
    \begin{align}
    \bA \preceq c_1(\eta, t) \cdot \bS^{-1} ( \bI - (\bI - \eta \bS)^t ) \,, \label{eq:upperbound_diagonal}
    \end{align}
    where $c_1(\eta, t ) = \min(0.5, \frac{1}{t s_i \eta })$. Plugging this into \eqref{eq:train_GD_rel}, we have 
    \begin{align}
        \Expt{ x \sim \calS }{\left(f_t(x) - \wt f_\lambda (x)\right)^2} &\le c(\eta, t) \cdot \by^T \bV \bX  \left( \bS^{-1} ( \bI - (\bI - \eta \bS)^t ) \right)^2 \bS \bV^T \bX^T \by \\
        &=   c(\eta, t) \cdot \by^T \bV \bX  \left( \bS^{-1} ( \bI - (\bI - \eta \bS)^t ) \right) \bS \left( \bS^{-1} ( \bI - (\bI - \eta \bS)^t ) \right) \bV^T \bX^T \by \\
        & =  c(\eta, t) \cdot \norm{\bX w_t}{2}^2 \\
        &= c(\eta, t) \cdot  \Expt{ x \sim \calS }{\left(f_t(x) \right)^2} \,,
    \end{align}
    where $c(\eta, t ) = \min(0.25, \frac{1}{t^2 s^2_i \eta^2 })$.

    \textbf{Part 2 {} {}} With $\bSigma$, 
    we denote the underlying true covariance matrix. 
    We now consider the squared difference of output at an unseen point: 
    \begin{align}
        \Expt{ x \sim \calD_{\calX} }{\left(f_t(x) - \wt f_\lambda (x)\right)^2} &= \Expt{x \sim \calD_{\calX}}{\norm{x^T w_t - x^T \wt w_\lambda}{2}} \\
        &=   \norm{x^T \bV \bS ^{-1} ( \bI - (\bI - \eta \bS)^t ) \bV^T \bX^T \by - x^T \bV (\bS + \lambda \bI )^{-1} \bV^T \bX^T \by }{2} \\
        &= \norm{x^T \bV \left(\bS ^{-1} ( \bI - (\bI - \eta \bS)^t ) - (\bS + \lambda \bI )^{-1} \right) \bV^T \bX^T \by  }{2} \\
        &= \by^T \bV \bX \left( \bS ^{-1} ( \bI - (\bI - \eta \bS)^t ) - (\bS + \lambda \bI )^{-1} \right) \bV^T \bSigma \bV \\ &\qquad \qquad \qquad \qquad \qquad \left( (\bI - (\bI - \eta \bS)^t ) - (\bS + \lambda \bI )^{-1} \right) \bV^T \bX^T \by \\
        &\le \sigma_{\text{max}} \cdot \by^T \bV \bX \left( \underbrace{\bS ^{-1} ( \bI - (\bI - \eta \bS)^t ) - (\bS + \lambda \bI )^{-1}}_{\RN{1}} \right)^2 \bV^T \bX^T \by \,, \label{eq:test_GD_rel}
        %  (\bX \bV \bS ^{-1} ( \bI - (\bI - \eta \bS)^k ) \bV^T \bX^T \by)^T \bX \bV \bS ^{-1} ( \bI - (\bI - \eta \bS)^k ) \bV^T \bX^T \by
    \end{align}
    where $\sigma_{\text{max}}$ is the maximum eigenvalue 
    of the underlying covariance matrix $\bSigma$. 
    Using the upper bound on term 1 in \eqref{eq:upperbound_diagonal}, 
    we have 
    \begin{align}
        \Expt{ x \sim \calD_{\calX} }{\left(f_t(x) - \wt f_\lambda (x)\right)^2} &\le \sigma_{\text{max}} \cdot c(\eta, t) \cdot \by^T \bV \bX  \left( \bS^{-1} ( \bI - (\bI - \eta \bS)^t ) \right)^2 \bV^T \bX^T \by \\
        &=   \kappa \cdot c(\eta, t) \cdot \sigma_{\text{min}}\cdot \norm{\bV \left( \bS^{-1} ( \bI - (\bI - \eta \bS)^t ) \right) \bV^T \bX^T \by}{2}^2 \\
        &\le \kappa \cdot c(\eta, t) \cdot \left[ \bV \left( \bS^{-1} ( \bI - (\bI - \eta \bS)^t ) \right) \bV^T \bX^T \right]^T \bSigma \\
        &\qquad \qquad \qquad \qquad \qquad \left[ \bV \left( \bS^{-1} ( \bI - (\bI - \eta \bS)^t ) \right) \bV^T \bX^T \right] \by \\
        & = \kappa \cdot c(\eta, t) \cdot \Expt{x \sim \calD_{\calX}}{\norm{x^T w_t}{2}} \,.
    \end{align}
% 
% 
    % Since $ \eta\le 1/\norm{S}{\text{op}}$, invoking \lemref{lem:ineq_soln} to upper bound term 1 with
\end{proof}

\subsection{Extension to deep learning} \label{appsubsec:ext_DL}
Under \asmpref{appsubsec:justifying_assumption1}, we present the formal result parallel to \thmref{thm:multiclass_ERM}. 
\begin{theorem} \label{thm:multiclass_ERM_algoA}
    Consider a multiclass classification problem 
    with $k$ classes. Under \asmpref{asmp:deep_models}, 
    for any $\delta >0$, with probability at least $1-\delta$,
    we have
    \vspace{-10pt}
    \begin{align*}
        \error_\calD(\widehat f)  \le \error_\calS(\widehat f) + (k-1) \left(1 - \tfrac{k}{k-1} \error_{\wt\calS}(\widehat f)\right) + c\sqrt{\frac{\log(\frac{4}{\delta})}{2m}} \,,\numberthis \label{eq:multiclass_ERM_deep}
    % \vspace{-20pt}
    \end{align*}
    for some constant $c \le ((c+1) k+\sqrt{k} + \frac{m}{n\sqrt{k}})$.
\end{theorem}

The proof follows exactly as in step (i) to (iii) in \thmref{thm:multiclass_ERM}.  

\subsection{Justifying~\asmpref{asmp:deep_models}} \label{appsubsec:justifying_assumption1}

Motivated by the analysis on linear models, we now discuss alternate (and weaker) conditions that imply \asmpref{asmp:deep_models}. 
We need hypothesis stability (\codref{cond:hypothesis_stability}) and the following assumption relating training error and leave-one-error: 

\begin{assumption} \label{asmp:loo_error}
Let $\wh f$ be a model obtained by training with algorithm $\calA$ on a mixture of clean $S$ and randomly labeled data $\wt S$. Then we assume we have 
\begin{align*}
    \error_{\wt \calS_M} (\wh f) \le  \error_{\text{LOO} (\wt S_M)} \,, 
\end{align*}
for all $(x_i, y_i) \in  \wt S_M$ where $\wh f_{(i)} \defeq f(\calA, S \cup {{}\wt S_M}_{(i)})$ and  $\error_{\text{LOO} (\wt S_M)} \defeq  \frac{\sum_{(x_i, y_i) \in \wt S_M} \error(\ff{i}(x_i), y_i ) }{\abs{\wt \calS_M}}$.  
\end{assumption}

% we assume this to extend our result (parallel to \thmref{thm:multi_linear}) for deep models. 
Intuitively, this assumption states that the error on a (mislabeled) datum $(x,y)$ included in the training set is less than the error on that datum $(x,y)$ obtained by a model trained on the training set $S - \{(x,y)\}$. We proved this for linear models trained with GD in the proof of \thmref{thm:multi_linear}. 
% 
\codref{cond:hypothesis_stability} with $\beta = \calO(1)$ and \asmpref{asmp:loo_error} together with \lemref{lem:stability_error} implies \asmpref{asmp:deep_models} with a polynomial residual term (instead of logarithmic in $1/\delta$): 
\begin{align}
     \error_{\calS_M} (\wh f) \le  \error_{\calDm}(\wh f)   + \sqrt{\frac{1}{\delta}\left(\frac{1}{m} +\frac{3\beta}{m+n} \right)} \,.
\end{align}
% Note that this  

\newpage 
\section{Additional experiments and details}\label{app:exp}
\newcommand\tab[1][1cm]{\hspace*{#1}}

\subsection{Datasets} \label{sec:app_dataset}

\textbf{Toy Dataset {} {}} Assume fixed constants $\mu$ and $\sigma$. For a given label $y$, we simulate features $x$ in our toy classification setup as follows: 
\begin{align*}
    x \defeq \texttt{concat} \left[ x_1, x_2\right] \quad \text{where} \quad  x_1 \sim  \calN( y \cdot \mu, \sigma^2 I_{d \times d}) \ \  \text{and} \ \  x_1 \sim  \calN( 0, \sigma^2 I_{d \times d}) \,.
\end{align*}  
% where $y$ is the true label and $x$ is the corresponding feature vector. 
In experiements throughout the paper, we fix dimention $d=100$, $\mu = 1.0 $, and $\sigma = \sqrt{d}$. Intuitively, $x_1$ carries the information about the underlying label and $x_2$ is additional noise independent of the underlying label. 

\textbf{CV datasets {} {}} We use MNIST~\citep{lecun1998mnist} and CIFAR10~\cite{krizhevsky2009learning}. 
% For binary tasks, 
We produce a binary variant from the multiclass classification problem by mapping classes $\{0,1,2,3,4\}$ to label $1$ and $\{ 5,6,7,8,9\}$ to label $-1$. For CIFAR dataset, we also use the standard data augementation of random crop and horizontal flip. PyTorch code is as follows: 

\texttt{(transforms.RandomCrop(32, padding=4),\\
\tab transforms.RandomHorizontalFlip())}

\textbf{NLP dataset {} {}} We use IMDb Sentiment analysis~\citep{maas2011learning} corpus.  

\subsection{Architecture Details} 

All experiments were run on NVIDIA GeForce RTX 2080 Ti GPUs. We used PyTorch~\citep{NEURIPS2019a9015} and Keras with Tensorflow~\citep{abadi2016tensorflow} backend for experiments. 
% , ELMo embeddings~\citep{Peters:2018}, and Hugging Face Transformers~\citep{wolf-etal-2020-transformers}. 

\textbf{Linear model {} {}} For the toy dataset, we simulate a linear model with scalar output and the same number of parameters as the number of dimensions.   

\textbf{Wide nets {} {}} To simulate the NTK regime, we experiment with $2-$layered wide nets. The PyTorch code for 2-layer wide MLP is as follows: 


\texttt{ nn.Sequential( \\
\tab     nn.Flatten(),\\
\tab    nn.Linear(input\_dims, 200000, bias=True),\\
\tab    nn.ReLU(),\\
\tab    nn.Linear(200000, 1, bias=True)\\
\tab     )}


We experiment both (i) with the second layer fixed at random initialization; (ii)  and updating both layers' weights.     

\textbf{Deep nets for CV tasks {} {}} We consider a 4-layered MLP. The PyTorch code for 4-layer MLP is as follows: 

\texttt{ nn.Sequential(nn.Flatten(), \\
\tab        nn.Linear(input\_dim, 5000, bias=True),\\
\tab        nn.ReLU(),\\
\tab        nn.Linear(5000, 5000, bias=True),\\
\tab        nn.ReLU(),\\
\tab        nn.Linear(5000, 5000, bias=True),\\
\tab        nn.ReLU(),\\
% \tab        nn.Linear(5000, 5000, bias=True),\\
% \tab        nn.ReLU(),\\
\tab        nn.Linear(1024, num\_label, bias=True)\\
\tab        )}

For MNIST, we use $1000$ nodes instead of $5000$ nodes in the hidden layer. 
% 
We also experiment with convolutional nets. In particular, we use ResNet18 \citep{he2016deep}. Implementation adapted from:  \url{https://github.com/kuangliu/pytorch-cifar.git}. 

\textbf{Deep nets for NLP {} {}} We use a simple LSTM model with embeddings intialized with ELMo embeddings~\citep{Peters:2018}. Code adapted from: \url{https://github.com/kamujun/elmo_experiments/blob/master/elmo_experiment/notebooks/elmo_text_classification_on_imdb.ipynb} 

We also evaluate our bounds with a BERT model. In particular, we fine-tune an off-the-shelf uncased BERT model~\citep{devlin2018bert}. Code adapted from Hugging Face Transformers~\citep{wolf-etal-2020-transformers}: \url{https://huggingface.co/transformers/v3.1.0/custom_datasets.html}. 


\subsection{Additonal experiments}

\textbf{Results with SGD on underparameterized linear models {} {}} 

\begin{figure*}[h]
    \centering 
    % \vspace{-15pt}
    % \includegraphics[width=0.9\linewidth]{example-image-a}
    \includegraphics[width=0.3\linewidth]{figures/lowdim-Gaussian-SGD.pdf}
    % \includegraphics[width=0.9\linewidth]{figures/{CIFAR10_rn=0.1_lr=0.2_wd=0.005}.png}
    \vspace{-5pt}
    \caption{ 
    % Predicted lower bound 
    % on different
    We plot the accuracy and corresponding bound 
    (RHS in \eqref{eq:erm}) at $\delta = 0.1$
    for toy binary classification task. 
    Results aggregated over $3$ seeds. 
    % i.e., $1-\error$ where $\error$ is the term in the RHS of \eqref{eq:erm}
    Accuracy vs fraction of unlabeled data (w.r.t clean data) 
    in the toy setup with a linear model trained with SGD. Results parallel to \figref{fig:error_binary}(a) with SGD.  }
    \label{fig:error_binary_linear}
    \vspace{-5pt}
\end{figure*}

\textbf{Results with wide nets on binary MNIST {} {}}

\begin{figure*}[h]
    \centering 
    % \vspace{-15pt}
    % \includegraphics[width=0.9\linewidth]{example-image-a}
    \subfigure[GD with MSE loss]{\includegraphics[width=0.3\linewidth]{figures/MNIST-GD_MSE.pdf}} \hfil
    \subfigure[SGD with CE loss]{\includegraphics[width=0.3\linewidth]{figures/MNIST-SGD_CE.pdf}}
    \subfigure[SGD with MSE loss]{\includegraphics[width=0.3\linewidth]{figures/MNIST-SGD_MSE-first-layer.pdf}}
    % \includegraphics[width=0.9\linewidth]{figures/{CIFAR10_rn=0.1_lr=0.2_wd=0.005}.png}
    \vspace{-5pt}
    \caption{ 
    % Predicted lower bound 
    % on different
    We plot the accuracy and corresponding bound 
    (RHS in \eqref{eq:erm}) at $\delta = 0.1$ 
    for binary MNIST classification. 
    Results aggregated over $3$ seeds. 
    % i.e., $1-\error$ where $\error$ is the term in the RHS of \eqref{eq:erm}
    Accuracy vs fraction of unlabeled data 
    for a 2-layer wide network on binary MNIST with both the layers training in (a,b) and only first layer training in (c). 
    Results parallel to \figref{fig:error_binary}(b) .  }
    \label{fig:error_binary_MNIST}
    \vspace{-5pt}
\end{figure*}

% \begin{figure*}[h]
%     \centering 
%     % \vspace{-15pt}
%     % \includegraphics[width=0.9\linewidth]{example-image-a}
%     \subfigure[GD with MSE loss]{\includegraphics[width=0.3\linewidth]{figures/MNIST.pdf}} \hfil
    
%     \subfigure[SGD with CE loss]{\includegraphics[width=0.3\linewidth]{figures/MNIST.pdf}}
%     % \includegraphics[width=0.9\linewidth]{figures/{CIFAR10_rn=0.1_lr=0.2_wd=0.005}.png}
%     \vspace{-5pt}
%     \caption{ 
%     % Predicted lower bound 
%     % on different
%     We plot the accuracy and corresponding bound 
%     (RHS in \eqref{eq:erm}) at $\delta = 0.1$
%     for binary MNIST classification. 
%     Results aggregated over $3$ seeds. 
%     % i.e., $1-\error$ where $\error$ is the term in the RHS of \eqref{eq:erm}
%     Accuracy vs fraction of unlabeled data 
%     for a 2-layer wide network on binary MNIST with just the first layer training. 
%     Results parallel to \figref{fig:error_binary}(b) with only the first layer training.  }
%     \label{fig:error_binary_MNIST}
%     \vspace{-5pt}
% \end{figure*}

\textbf{Results on CIFAR 10 and MNIST {} {}} 
% 
We plot epoch wise error curve for results in \tabref{table:multiclass}(\figref{fig:error_epoch_CIFAR10} and \figref{fig:error_epoch_MNIST}). We observe the same trend as in \figref{fig:error_CIFAR10}. Additionally, we plot an \emph{oracle bound} obtained by tracking the error on mislabeled data which nevertheless were predicted as true label. To obtain an exact emprical value of the oracle bound, we need underlying true labels for the randomly labeled data. 
% Note that our bound in \thmref{thm:multiclass_ERM}, lower bounds the accuracy as predicted by the oracle bound. 
While with just access to extra unlabeled data we cannot calculate oracle bound, we note that the oracle bound is very tight and never violated in practice underscoring an importamt aspect of generalization in multiclass problems. This highlight that even a stronger conjecture may hold in multiclass classification, i.e., error on mislabeled data (where nevertheless true label was predicted) lower bounds the population error on the distribution of mislabeled data and hence, the error on (a specific) mislabeled portion predicts the population accuracy on clean data. 
% 
On the other hand, the dominating term of in \thmref{thm:multiclass_ERM} is loose when compared with the oracle bound. The main reason, we believe is the pessimistic upper bound in \eqref{eq:lemma1_final_multi_prev} in the proof of \lemref{lem:fit_mislabeled_multi}. We leave an investigation on this gap for future. 
% of fit 

% However, oracle bound highlights two . One,  



\begin{figure}[h]
    \centering 
    % \vspace{-15pt}
    % \includegraphics[width=0.9\linewidth]{example-image-a}
    \subfigure[MLP]{\includegraphics[width=0.3\linewidth]{figures/CIFAR10-FNN.pdf}} \hfil
    \subfigure[ResNet]{\includegraphics[width=0.3\linewidth]{figures/CIFAR10-Resnet.pdf}}
    % \includegraphics[width=0.9\linewidth]{figures/{CIFAR10_rn=0.1_lr=0.2_wd=0.005}.png}
    % \vspace{-10pt}
    \caption{ Per epoch curves for CIFAR10 corresponding results in \tabref{table:multiclass}. As before, we just plot the dominating term in the RHS of \eqref{eq:multiclass_ERM} as predicted bound. Additionally, we also plot the predicted lower bound by the error on mislabeled data which nevertheless were predicted as true label. We refer to this as ``Oracle bound''. See text for more details. 
    % 
    % except for the stopping point. 
    % The bound predicted by RATT (RHS in \eqref{eq:multiclass_ERM}) is vacuous. 
    }\label{fig:error_epoch_CIFAR10}
    % \vspace{-15pt}
\end{figure}


\begin{figure}[h]
    \centering 
    % \vspace{-15pt}
    % \includegraphics[width=0.9\linewidth]{example-image-a}
    \subfigure[MLP]{\includegraphics[width=0.3\linewidth]{figures/MNIST-FNN.pdf}} \hfil
    \subfigure[ResNet]{\includegraphics[width=0.3\linewidth]{figures/MNIST-Resnet.pdf}}
    % \includegraphics[width=0.9\linewidth]{figures/{CIFAR10_rn=0.1_lr=0.2_wd=0.005}.png}
    % \vspace{-10pt}
    \caption{ Per epoch curves for MNIST corresponding results in \tabref{table:multiclass}. As before, we just plot the dominating term in the RHS of \eqref{eq:multiclass_ERM} as predicted bound. Additionally, we also plot the predicted lower bound by the error on mislabeled data which nevertheless were predicted as true label. We refer to this as ``Oracle bound''. See text for more details. 
    % 
    % except for the stopping point. 
    % The bound predicted by RATT (RHS in \eqref{eq:multiclass_ERM}) is vacuous. 
    }\label{fig:error_epoch_MNIST}
    % \vspace{-15pt}
\end{figure}

\textbf{Results on CIFAR 100 {} {}} 
% 
On CIFAR100, our bound in \eqref{eq:multiclass_ERM} yields vacous bounds. However, the oracle bound as explained above yields tight guarantees in the initial phase of the learning (i.e., when learning rate is less than $0.1$) (\figref{fig:error_CIFAR100}).  

\begin{figure}[h]
    \centering 
    % \vspace{-15pt}
    % \includegraphics[width=0.9\linewidth]{example-image-a}
    \includegraphics[width=0.3\linewidth]{figures/CIFAR100-Resnet.pdf}
    % \includegraphics[width=0.9\linewidth]{figures/{CIFAR10_rn=0.1_lr=0.2_wd=0.005}.png}
    % \vspace{-10pt}
    \caption{ Predicted lower bound by the error on mislabeled data which nevertheless were predicted as true label with ResNet18 on CIFAR100. We refer to this as ``Oracle bound''. See text for more details. 
    % 
    % except for the stopping point. 
    The bound predicted by RATT (RHS in \eqref{eq:multiclass_ERM}) is vacuous. 
    }\label{fig:error_CIFAR100}
    % \vspace{-15pt}
\end{figure}


% \paragraph{Experiments on CIFAR100} 


% \subsection{Model Selection using RATT}


\subsection{Hyperparameter Details}


\textbf{\figref{fig:error_CIFAR10} {} {}} We use clean training dataset of size $40,000$. We fix the amount of unlabeled data at $20\%$ of the clean size, i.e. we include additional $8,000$ points with randomly assigned labels. We use test set of $10,000$ points. For both MLP and ResNet, we use SGD with an initial learning rate of $0.1$ and momentum $0.9$. We fix the weight decay parameter at $5\times 10^{-4}$. After $100$ epochs, we decay the learning rate to $0.01$. We use SGD batch size of $100$. 

\textbf{\figref{fig:error_binary} (a) {} {}} We obtain a toy dataset according to the process described in \secref{sec:app_dataset}. We fix $d=100$ and create a dataset of $50,000$ points with balanced classes. Moreover, we sample additional covariates with the same procedure to create randomly labeled dataset. For both SGD and GD training, we use a fixed learning rate $0.1$.    

\textbf{\figref{fig:error_binary} (b) {} {}} Similar to binary CIFAR, we use clean training dataset of size $40,000$ and fix the amount of unlabeled data at $20\%$ of the clean dataset size. To train wide nets, we use a fixed learning of $0.001$ with GD and SGD. We decide the weight decay parameter and the early stopping point that maximizes our generalization bound (i.e. without peeking at unseen data ).  We use SGD batch size of $100$. 

\textbf{\figref{fig:error_binary} (c) {} {}} With IMDb dataset, we use a clean dataset of size $20,000$ and as before, fix the amount of unlabeled data at $20\%$ of the clean data. To train ELMo model, we use Adam optimizer with a fixed learning rate $0.01$ and weight decay $10^{-6}$ to minimize cross entropy loss. We train with batch size $32$ for 3 epochs. To fine-tune BERT model, we use Adam optimizer with learning rate $5\times 10^{-5}$ to minimize cross entropy loss. We train with a batch size of $16$ for 1 epoch.    

\textbf{\tabref{table:multiclass} {} {}} For multiclass datasets, we train both MLP and ResNet with the same hyperparameters as described before. We sample a clean training dataset of size $40,000$ and fix the amount of unlabeled data at $20\%$ of the clean size. We use SGD with an initial learning rate of $0.1$ and momentum $0.9$. We fix the weight decay parameter at $5\times 10^{-4}$. After $30$ epochs for ResNet and after $50$ epochs for MLP, we decay the learning rate to $0.01$.  We use SGD with batch size $100$. 
For \figref{fig:error_CIFAR100}, we use the same hyperparameters as 
CIFAR10 training, except we now decay learning rate after $100$ epochs. 


In all experiments, to identify the best possible accuracy on just the clean data, we use the exact same set of hyperparamters except the stopping point. We choose a stopping point that maximizes test performance. 

\subsection{Summary of experiments }

\begin{center}
    \begin{table}[H] 
        \centering
        \begin{tabular}{|c|c|c|c|} 
        \hline
        Classification type & Model category & Model & Dataset  \\ [0.5ex] 
        \hline
        \hline
        \multirow{10}{*}{Binary} & Low dimensional & Linear model & Toy Gaussain dataset  \\
                        \cline{2-4}
                         & Overparameterized 
                        %  & Linear model & Toy Gaussain dataset \\
                        %  \cline{3-4}
                        %  & & 2-layer wide net& Toy Gaussain dataset \\
                        %  \cline{3-4}
                         & \multirow{2}{*}{2-layer wide net} & \multirow{2}{*}{Binary MNIST} \\
                         & linear nets & &  
                         \\
                         \cline{2-4}                 
                         & \multirow{6}{*}{Deep nets} & \multirow{2}{*}{MLP} & Binary MNIST \\
                         \cline{4-4}
                         & &  & Binary CIFAR \\
                         \cline{3-4}
                         &  & \multirow{2}{*}{ResNet} & Binary MNIST \\
                         \cline{4-4}
                         & &  & Binary CIFAR \\
                         \cline{3-4}
                         &  & ELMo-LSTM model & IMDb Sentiment Analysis \\
                         \cline{3-4}
                         & & BERT pre-trained model & IMDb Sentiment Analysis \\
        \hline
        \multirow{5}{*}{Multiclass} & \multirow{5}{*}{Deep nets} & \multirow{2}{*}{MLP} & MNIST \\
                        \cline{4-4} 
                        & & & CIFAR10 \\                   
                        \cline{3-4}
                         &   & \multirow{3}{*}{ResNet} & MNIST \\
                         \cline{4-4}
                         &   & & CIFAR10 \\
                         \cline{4-4}
                         &   & & CIFAR100 \\
        \hline
        \end{tabular}
        % \caption{Summary of experiments performed} \label{table:experiments}
    \end{table}    
    % \footnotetext[6]{We use both MSE loss and cross-entropy loss.}
    % \footnotetext[6]{We try 2 variants: one with a fixed first layer and the other with both layers trainable.}
\end{center}

\newpage
\section{Proof of \lemref{lem:stability_error}} \label{app:proof_lem_error}

\begin{proof}[Proof of \lemref{lem:stability_error}]
    Recall, we have a training set $S \cup \wt S_C$. We defined leave-one-out error on mislabeled points as $$\error_{\text{LOO}(\wt S_M) } = \frac{\sum_{(x_i, y_i) \in \wt S_M} \error( f_{(i)}( x_i), y_i)}{ \abs{\wt S_M }} \,, $$
    where $f_{(i)} \defeq f(\calA, (S \cup \wt S)_{(i)})$. Define $S^\prime \defeq S \cup \wt S$. Assume $(x,y)$ and $(x^\prime,y^\prime)$ as i.i.d. samples from ${\calDm}$. 
    Using Lemma 25 in \citet{bousquet2002stability}, we have
    \begin{align*}
        \Expo{ \left( \error_{\calDm}(\wh f) -\error_{\text{LOO}(\wt S_M) } \right)^2 } \le & \Expt{ S^\prime, (x,y), (x^\prime,y^\prime) }{ \error(\wh f(x), y ) \error(\wh f(x^\prime), y^\prime )} - 2 \Expt{ S^\prime, (x,y) }{ \error(\wh f(x), y ) \error(f_{(i)}(x_i), y_i )} \\
        & + \frac{m_1-1}{m_1}\Expt{ S^\prime }{  \error(f_{(i)}(x_i), y_i )  \error(f_{(j)}(x_j), y_j )} + \frac{1}{m_1} \Expt{ S^\prime }{  \error(f_{(i)}(x_i), y_i ) } \,. \numberthis \label{eq:main_reln}
    \end{align*}
    We can rewrite the equation above as : 
    \begin{align*}
        \Expo{ \left( \error_{\calDm}(\wh f) -\error_{\text{LOO}(\wt S_M) } \right)^2 } \le &  \, \underbrace{\Expt{ S^\prime, (x,y), (x^\prime,y^\prime) }{ \error(\wh f(x), y ) \error(\wh f(x^\prime), y^\prime ) - \error(\wh f(x), y ) \error(f_{(i)}(x_i), y_i )}}_{\RN{1}} \\
        & + \underbrace{\Expt{ S^\prime }{  \error(f_{(i)}(x_i), y_i )  \error(f_{(j)}(x_j), y_j ) -  \error(\wh f(x), y ) \error(f_{(i)}(x_i), y_i )}}_{\RN{2}} \\ &+ \underbrace{\frac{1}{m_1} \Expt{ S^\prime }{  \error(f_{(i)}(x_i), y_i ) - \error(f_{(i)}(x_i), y_i )  \error(f_{(j)}(x_j), y_j ) }}_{\RN{3}} \,. \numberthis \label{eq:main_reln2}
    \end{align*}
    
    We will now bound term $\RN{3}$.  Using Cauchy-Schwarz's inequality, we have
    
    \begin{align}
        \Expt{ S^\prime }{  \error(f_{(i)}(x_i), y_i ) - \error(f_{(i)}(x_i), y_i )  \error(f_{(j)}(x_j), y_j ) }^2 &\le  \Expt{ S^\prime }{  \error(f_{(i)}(x_i), y_i ) }^2 \Expt{S^\prime}{1 -   \error(f_{(j)}(x_j), y_j ) }^2 \\
        &\le \frac{1}{4} \,.\label{eq:term1_lem12}
    \end{align}
    
    Note that since $(x_i,y_i)$, $(x_j ,y_j )$, $(x,y)$, and $(x^\prime, y^\prime)$ are all from same distribution $\calDm$, we directly incorporate the bounds on term $\RN{1}$ and $\RN{2}$ from the proof of Lemma 9 in \citet{bousquet2002stability}. Combining that with \eqref{eq:term1_lem12} and our definition of hypothesis stability in \codref{cond:hypothesis_stability}, we have the required claim. 
    
    
    % We now re-write term $\RN{1}$ as
    % \begin{align*}
    %         &\Expt{S^\prime, (x,y), (x^\prime,y^\prime) }{ \error(\wh f(x), y ) \error(\wh f(x^\prime), y^\prime ) - \error(\wh f(x), y ) \error(f_{(i)}(x_i), y_i )} \\ & \qquad = \Expt{ S^\prime, (x,y), (x^\prime,y^\prime) }{ \error(\wh f(x), y ) \error(\wh f  (x^\prime), y^\prime ) - \error(\wh f ^\prime(x), y ) \error(f_{(i)}(x^\prime), y^\prime )} \tag{Exchanging $(x_i, y_i)$ with $(x^\prime, y^\prime)$ in the second term} \\
    %         & \qquad = \Expt{ S^\prime, (x,y), (x^\prime,y^\prime) }{  \left(\error(\wh f(x), y )-  \error(f_{(i)}(x), y ) \right) \error(\wh f  (x^\prime), y^\prime )  } \\
    %         & \qquad  + \Expt{ S^\prime, (x,y), (x^\prime,y^\prime) }{  \left(\error(f_{(i)}(x), y ) -\error(\wh f ^\prime(x), y ) \right) \error(\wh f  (x^\prime), y^\prime )}  \\
    %         & \qquad +\Expt{ S^\prime, (x,y), (x^\prime,y^\prime) }{  \left( \error(\wh f  (x^\prime), y^\prime ) -  \error(f_{(i)}(x^\prime), y^\prime ) \right) \error(\wh f ^\prime(x), y ) }  \,, \numberthis \label{eq:term1_final}
    % \end{align*}
    % where $\wh f^\prime$ is the classifier obtained by training on $ S^\prime_{(i)} \cup \{ (x^\prime, y^\prime) \} $. Similarly we can re-write term $\RN{2}$ as 
    % \begin{align*}
    %     & \Expt{ S^\prime }{  \error(f_{(i)}(x_i), y_i )  \error(f_{(j)}(x_j), y_j ) -  \error(\wh f(x), y ) \error(f_{(i)}(x_i), y_i )} \\
    %     &\quad  = \Expt{ S^\prime, (x,y), (x^\prime,y^\prime)}{  \error(f^{\prime\prime}_{(i)}(x), y )  \error(f_{(j)}^{\prime}(x^\prime), y^\prime ) -  \error(\wh f(x), y ) \error(f_{(i)}(x_i), y_i )} \tag{Exchanging $(x_i, y_i)$ with $(x, y)$ and $(x_j, y_j)$ with $(x^\prime, y^\prime)$ in the first term}\\
    %     &\quad = \Expt{ S^\prime, (x,y), (x^\prime,y^\prime)}{  \error(f^{\prime\prime}_{(j)}(x), y )  \error(f_{(i)}^{\prime}(x^\prime), y^\prime ) -  \error(\wh f^\prime (x), y ) \error(f^\prime_{(j)}(x^\prime), y^\prime )} \tag{Exchanging $(x_i, y_i)$ and $(x_j, y_j)$ and then replacing $(x_j, y_j)$ with $(x^\prime, y^\prime)$ in the second term} \\
    %     & \quad = \Expt{ S^\prime, (x,y), (x^\prime,y^\prime) }{  \left( \error(f_{(i)}^{\prime}(x^\prime), y^\prime )   -  \error(\wh f^{\prime\prime}  (x^\prime), y^\prime ) \right)  \error(f^{\prime\prime}_{(j)}(x), y )   } \\
    %     & \quad  + \Expt{ S^\prime, (x,y), (x^\prime,y^\prime) }{  \left( \error(f^{\prime\prime}_{(j)}(x), y )  -\error(\wh f ^\prime(x), y ) \right) \error(\wh f^{\prime\prime}  (x^\prime), y^\prime )  }  \\
    %     & \quad+ \Expt{ S^\prime, (x,y), (x^\prime,y^\prime) }{  \left( \error(\wh f^{\prime\prime}  (x^\prime), y^\prime )  -  \error(f^\prime_{(j)}(x^\prime), y^\prime ) \right)  \error(\wh f^\prime (x), y ) }   \\
    %     & \quad = \Expt{ S^\prime, (x,y), (x^\prime,y^\prime) }{  \left( \error(f_{(i)}^{\prime}(x^\prime), y^\prime )   -  \error(\wh f (x^\prime), y^\prime ) \right)  \error(f_{(i)}(x_j), y_j )   } \\
    %     & \quad  + \Expt{ S^\prime, (x,y), (x^\prime,y^\prime) }{  \left( \error(f^{\prime\prime}_{(j)}(x), y )  -\error(\wh f (x), y ) \right) \error(\wh f^{\prime\prime}  (x_j), y_j )  }  \\
    %     & \quad+ \Expt{ S^\prime, (x,y), (x^\prime,y^\prime) }{  \left( \error(\wh f^{\prime\prime}  (x^\prime), y^\prime )  -  \error(f^\prime_{(j)}(x^\prime), y^\prime ) \right)  \error(\wh f^\prime (x^\prime), y^\prime ) }  \,, \numberthis \label{eq:term2_final}
    % \end{align*}
    % where $f^{\prime\prime}_{(j)}$ is trained on $S^\prime_{(j,i)} \cup {(x,y)}$, $f^{\prime}_{(i)}$ is trained on $S^\prime_{(j,i)} \cup {(x^\prime,y^\prime)}$, and $\wh f^{\prime\prime} $ is trained on $S^\prime_{(j)} \cup {(x,y)}$. Note in the last line we replaced $(x,y)$ by $(x_j, y_j)$ in the first term, replaced $(x^\prime,y^\prime)$ by $(x_j, y_j)$ in the second term and exchanged $(x_i,y_i)$ with $(x_j,y_j)$ and also $(x,y)$ and $(x^\prime, y^\prime)$
    
    
\end{proof}


% 
% 16th Century Version Control 
% 

% \onecolumn

% \section*{Supplementary Material}
% We will be using the following standard results
% on exponential concentration of random variables 
% all throughout the discussion:

% \begin{lemma}[Hoeffding's inequality for independent RVs~\citep{hoeffding1994probability}] Let $Z_1, Z_2, \ldots, Z_n$ be independent bounded random variables with $Z_i \in [a,b]$ for all $i$, then 
%     \begin{align*}
%         \prob\left( \frac{1}{n} \sum_{i=1}^n (Z_i - \Expo{Z_i}) \ge t \right) \le \exp{\left( -\frac{2nt^2}{(b-a)^2} \right) }
%     \end{align*} 
%     and 
%     \begin{align*}
%         \prob\left( \frac{1}{n} \sum_{i=1}^n (Z_i - \Expo{Z_i}) \le -t \right) \le \exp{\left( -\frac{2nt^2}{(b-a)^2} \right) }
%     \end{align*} 
%     for all $t \ge 0$. 
% \end{lemma}

% \begin{lemma}[Hoeffding's inequality for sampling with replacement~\citep{hoeffding1994probability}] \label{lem:hoeffding_sampling} Let $\calZ = (Z_1, Z_2, \ldots, Z_N)$ be a finite population of $N$ points with $Z_i \in [a.b]$ for all $i$. Let $X_1, X_2, \ldots X_n$ be a random sample drawn without replacement from $\calZ$. Then for all $t \ge 0$, we have 
%     \begin{align*}
%         \prob\left( \frac{1}{n} \sum_{i=1}^n (X_i - \mu ) \ge t \right) \le \exp{\left( -\frac{2nt^2}{(b-a)^2} \right) }
%     \end{align*} 
%     and 
%     \begin{align*}
%         \prob\left( \frac{1}{n} \sum_{i=1}^n (X_i - \mu ) \le -t \right) \le \exp{\left( -\frac{2nt^2}{(b-a)^2} \right) } \,,
%     \end{align*} 
%     where $\mu = \frac{1}{N} \sum_{i=1}^{N} Z_i$. 
% \end{lemma}

% We now discuss one condition that generalizes the exponential concentration to dependent random variables.
% \begin{condition}[Bounded difference inequality] \label{cond:BDC} Let $\calZ$ be some set and $\phi: \calZ^n \to \Real$. We say that $\phi$ satisfies the bounded difference assumption if 
% there exists $c_1, c_2, \ldots c_n \ge 0$ s.t. for all $i$, we have 
% \begin{align*}
%     \sup_{Z_1,Z_2, \ldots,Z_n, Z_i^\prime in \calZ^{n+1} } \abs{\phi (Z_1, \ldots, Z_i, \ldots, Z_n ) - \phi (Z_1, \ldots, Z_i^\prime, \ldots, Z_n ) } \le c_i \,.
% \end{align*} 
% \end{condition}

% \begin{lemma}[McDiarmid’s inequality~\citep{mcdiarmid1989}] \label{lem:McDiarmid} Let $Z_1, Z_2, \ldots, Z_n$ be independent random variables on set $\calZ$ and $\phi : \calZ^n \to \Real$ satisfy bounded difference assumption (\codref{cond:BDC}). Then for all $t>0$, we have 
%     \begin{align*}
%         \prob\left( \phi(Z_1, Z_2, \ldots, Z_n) - \Expo{\phi(Z_1, Z_2, \ldots, Z_n)} \ge t \right) \le \exp{\left( -\frac{2t^2}{\sum_{i=1}^n c_i^2} \right) } 
%     \end{align*} 
%     and 
%     \begin{align*}
%         \prob\left( \phi(Z_1, Z_2, \ldots, Z_n) - \Expo{\phi(Z_1, Z_2, \ldots, Z_n)} \le -t \right) \le \exp{\left( -\frac{2t^2}{\sum_{i=1}^n c_i^2} \right) } \,
%     \end{align*} 
% \end{lemma}


% \section{Proofs from \secref{sec:ERM_training}}\label{app:proof_erm}

% \textbf{Additional notation {} {}} Let $m_1$ be the number of mislabeled points ($\wt S_M$) and $m_2$ be the number of correctly labeled points ($\wt S_C$). Note $m_1 + m_2 = m$. 


% \subsection{Proof of \thmref{thm:error_ERM}}


% \begin{proof}[Proof of \lemref{lem:fit_mislabeled}] 
%     The main idea of our proof is to regard 
%     the clean portion of the data 
%     ($S \cup \wt S_C$) as fixed.   
%     Then, there exists a classifier $f^*$ 
%     that is optimal over draws 
%     of the mislabeled data $\wt S_M$. 
% % 
%     % 
%     Formally, 
%     \begin{align}
%     f^* \defeq \argmin_{f \in \calF} \error_{\widecheck {\calD}} (f) \,, \label{eq:modified_ERM}
%     \end{align}
%     where $$\widecheck \calD = \frac{n}{m+n} \calS + \frac{m_1}{m+n} \wt \calS_C  + \frac{m_2}{m+n}\calDm \,.$$ That is, $\widecheck \calD$ a combination of 
%     the \emph{empirical distribution} 
%     over correctly labeled data $S \cup \wt S_C$
%     % in $S\cup \wt S$ 
%     and the (population) distribution 
%     over mislabeled data $\calDm$.
%     Recall that 
%     \begin{align}
%     \wh f \defeq \argmin_{f \in \calF} \error_{\calS \cup \wt S} (f) \,. \label{eq:orig_ERM}
%     \end{align}
%     % 
%     % 
%     Since, $\widehat f$ minimizes 0-1 error 
%     on $S \cup \wt S$, using ERM optimality on \eqref{eq:orig_ERM},  
%     we have 
%     \begin{align}
%         \error_{\calS \cup \wt \calS}(\widehat f) \le \error_{
%             \calS \cup \wt \calS}(f^*) \,.    \label{eq:step1}
%     \end{align}
%     Moreover, since $f^*$ is independent of $\wt S_M$, using Hoeffding's bound,
%     % \footnote{For a fully rigorous argument,
%     % refer to the complete proof in App.~\ref{app:proof_erm}.} 
%     we have with probability at least $1-\delta$ that
%     \begin{align}
%       \error_{\wt \calS_M}(f^*) \le \error_{ \calDm}(f^*) +  \sqrt{\frac{\log(1/\delta)}{2 m_1}} \,. \label{eq:step2} 
%     \end{align}
%     %$ 
%     %for some constant $c_1\le 1/2$. 
%     Finally, since $f^*$ is the optimal classifier on $\widecheck \calD$, 
%     we have 
%     \begin{align}
%         \error_{\widecheck \calD}(f^*) \le \error_{\widecheck \calD}(\widehat f) \label{eq:step3}
%     \end{align}
%      Now to relate \eqref{eq:step1} and \eqref{eq:step3}, we can re-write the \eqref{eq:step2} as follows: 
%     \begin{align}
%         \error_{\calS \cup \wt\calS}(f^*) \le \error_{ \widecheck \calD}(f^*) +  \frac{m_1}{m+n}\sqrt{\frac{\log(1/\delta)}{2 m_1}} \,. \label{eq:step4} 
%     \end{align}
%     Now we combine equations \eqref{eq:step1}, \eqref{eq:step4}, and \eqref{eq:step3}, to get 
%     \begin{align}
%         \error_{\calS \cup \wt \calS}(\wh f) \le \error_{\widecheck \calD}(\wh f) +  \frac{m_1}{m+n}\sqrt{\frac{\log(1/\delta)}{2 m_1}} \,, 
%     \end{align}
%     which implies 
%     \begin{align}
%         \error_{ \wt \calS_M}(\wh f) \le \error_{\calDm}(\wh f) + \sqrt{\frac{\log(1/\delta)}{2 m_1}} \,. \label{eq:lemma1_final}
%     \end{align}
%     Since $\wt S$ is obtained by randomly labeling an unlabeled dataset, we assume $2m_1 \approx m$ \footnote{Formally, with probability at least $1-\delta$, we have  $(m - 2m_1)\le \sqrt{m\log(1/\delta)/2}$ }. Moreover, using $\error_{\calDm} = 1 - \error_{\calD}$ we obtain the desired result.   
%     % Combining the above steps and using the fact 
%     % that $\error_\calD = 1- \error_{\calDm} $, 
%     % we obtain the desired result.
% \end{proof}

% \begin{proof}[Proof of \lemref{lem:mislabeled_error}]
%     Recall $\error_{\wt S} (f) = \frac{m_1}{m} \error_{\wt S_M}(f) + \frac{m_2}{m} \error_{\wt S_C}(f)$. Hence, we have 
%     \begin{align}
%         2\error_{\wt S}(f) - \error_{\wt S_M}(f) - \error_{\wt S_C}(f) &= \left(\frac{2m_1}{m} \error_{\wt S_M}(f) - \error_{\wt S_M}(f)\right) + \left(\frac{2m_2}{m} \error_{\wt S_C}(f) - \error_{\wt S_C}(f)\right) \\ &= \left(\frac{2m_1}{m} - 1\right) \error_{\wt S_M}(f) + \left(\frac{2m_2}{m} - 1 \right)\error_{\wt S_C} (f) \,.
%     \end{align} 
%     Since the dataset is randomly labeled, with probability at least $1-\delta$, we have  $\left(\frac{2m_1}{m} - 1\right) \le \sqrt{\frac{\log(1/\delta)}{2m}}$. Similarly, we have with probability at least $1-\delta$, $\left(\frac{2m_2}{m} - 1\right) \le \sqrt{\frac{\log(1/\delta)}{2m}}$. Using union bound, we have with probability at least $1-\delta$
%     % \begin{align}
%     %     2\error_{\wt S} - \error_{\wt S_M}(f) - \error_{\wt S_C}(f) \le \sqrt{\frac{\log(2/\delta)}{2m}} \left(\error_{\wt S_M}(f) + \error_{\wt S_C}(f) \right) \le 2\sqrt{\frac{\log(2/\delta)}{2m}} \,. \label{eq:lemma2_final}
%     % \end{align}
%     \begin{align}
%         2\error_{\wt S} - \error_{\wt S_M}(f) - \error_{\wt S_C}(f) \le \sqrt{\frac{\log(2/\delta)}{2m}} \left(\error_{\wt S_M}(f) + \error_{\wt S_C}(f) \right) \,. \label{eq:lemma2_prefinal}
%     \end{align}
%     With re-arranging $\error_{\wt S_M}(f) + \error_{\wt S_C}(f)$ and using the inequality $ 1- a\le \frac{1}{1+a} $, we have  
%     \begin{align}
%         2\error_{\wt S} - \error_{\wt S_M}(f) - \error_{\wt S_C}(f) \le 2\error_{\wt \calS} \sqrt{\frac{\log(2/\delta)}{2m}}  \,. \label{eq:lemma2_final}
%     \end{align}

%     % We obtain the desired result by using 
% \end{proof}

% \begin{proof}[Proof of \lemref{lem:clear_error}]
% % Recall 0-1 error on each point  $(x,y) \in S \cup \wt S$ is given by $\I{ f(x)\ne y}$.
% In the set of correctly labeled points $S \cup \wt S_C$, we have $S$ as a random subset of $S \cup \wt S_C$. Hence, using Hoeffding's inequality for sampling without replacement (\lemref{lem:hoeffding_sampling}), we have with probability at least $1-\delta$
% \begin{align}
%     \error_{\wt \calS_c} (\wh f)- \error_{\calS \cup \wt \calS_C}( \wh f) \le  \sqrt{\frac{\log(1/\delta)}{2m_2}} \,.
% \end{align}
% Re-writing $\error_{\calS \cup \wt \calS_C}( \wh f)$ as $\frac{m_2}{m_2 + n} \error_{\wt \calS_C }(\wh f) + \frac{n}{m_2 + n} \error_{\calS }(\wh f)$, we have with probability at least $1-\delta$
% \begin{align}
%   \left(\frac{n}{n+m_2}\right) \left(\error_{\wt \calS_c} (\wh f)- \error_{\calS}( \wh f) \right) \le  \sqrt{\frac{\log(1/\delta)}{2m_2}} \,.
% \end{align}
% As before, assuming $2m_2 \approx m$, we have with probability at least $1-\delta$ 
% \begin{align}
%     \error_{\wt \calS_c} (\wh f)- \error_{\calS}( \wh f) \le \left(1+\frac{m_2}{n}\right)  \sqrt{\frac{\log(1/\delta)}{m}} \le 1.5 \sqrt{\frac{\log(1/\delta)}{m}} \,. \label{eq:lemma3_final}
% \end{align} 
% \end{proof}

% \begin{proof}[Proof of \thmref{thm:error_ERM}] 
%     Having established these core intermediate results, we can now combine above three lemmas to prove the main result. 
%     In particular, we bound the population error on clean data ($\error_\calD(\wh f)$) as follows:  
%     \begin{enumerate}[(i)]
%         \item First, use \eqref{eq:lemma1_final}, to obtain an upper bound on the population error on clean data, i.e., with probability at least $1-\delta/4$, we have
%         \begin{align}
%             \error_{ \calD} (\wh f) \le 1 - \error_{ \wt \calS_M}(\wh f) + \sqrt{\frac{\log(4/\delta)}{m}} \,. 
%         \end{align}
%         \item  Second, use \eqref{eq:lemma2_final}, to relate the error on the mislabeled fraction with error on clean portion of randomly labeled data and error on whole randomly labeled dataset, i.e., with probability at least $1-\delta/2$, we have 
%         \begin{align}
%             - \error_{\wt S_M}(f) \le \error_{\wt S_C}(f) - 2\error_{\wt S}  + \sqrt{\frac{\log(4/\delta)}{2m}}  \,. 
%         \end{align} 
%         \item Finally, use \eqref{eq:lemma3_final} to relate the error on the clean portion of randomly labeled data and error on clean training data, i.e., with probability $1-\delta/4$, we have 
%         \begin{align}
%             \error_{\wt \calS_C} (\wh f)\le - \error_{\calS}( \wh f) + \left(1 + \frac{m}{2n} \right) \sqrt{\frac{\log(4/\delta)}{m}} \,. 
%         \end{align} 
%     \end{enumerate}

%     Using union bound on the above three steps, we have with probability at least $1-\delta$: 
%     \begin{align}
%         \error_\calD (\wh f) \le \error_{\calS}(\wh f)   + 1 - 2\error_{\wt \calS}(\wh f)   + (1/\sqrt{2} + 2.5)  \sqrt{\frac{\log(4/\delta)}{m}} \,.
%     \end{align}
%     Note that $(1/\sqrt{2} + 2.5)$ is a loose constant. In experiments, we use the ratio $\frac{m}{n}$
%     %  the exact error $\error_{\wt \calS}(\wh f)$ 
%     to evaluate R.H.S.    
% \end{proof}

% \subsection{Proof of \propref{prop:rademacher}}

% \begin{proof}[Proof of \propref{prop:rademacher}]
%     For a classifier $ f: \calX \to \{-1, 1\}$, we have $1 - 2\,\indict{ f(x) \ne y} = y \cdot f(x)$. Hence, by definition of $\error$, we have 
%     \begin{align}
%         1 -2\error_{\wt \calS}(f) = \frac{1}{m}\sum_{i=1}^m y_i \cdot f(x_i) \le \sup_{f \in \calF} \, \frac{1}{m} \sum_{i=1}^m y_i \cdot f(x_i)  \,. \label{eq:error_rademacher}
%     \end{align}
%     Note that for fixed inputs $(x_1, x_2, \ldots, x_m)$ in $\wt S$, $(y_1, y_2, \ldots y_m)$ are random labels. Define $\phi_1 (y_1, y_2, \ldots, y_m) \defeq \sup_{f \in \calF} \, \frac{1}{m} \sum_{i=1}^m y_i \cdot f(x_i)$. We have the following bounded difference condition on $\phi_1$. For all i, 
%     \begin{align}
%         \sup_{y_1, \ldots y_m, y_i^\prime \in \{-1, 1\}^{m+1} } \abs{ \phi_1 (y_1,\ldots, y_i, \ldots, y_m) - \phi_1 (y_1,\ldots, y_i^\prime, \ldots, y_m)  } \le 1/m \,. \label{cond1_rademacher}
%     \end{align} 
    
%     Similarly define $\phi_2 (x_1, x_2, \ldots, x_m) \defeq \Expt{ y_i \sim_U \{-1, 1\}  }{ \sup_{f \in \calF} \, \frac{1}{m}  \sum_{i=1}^m y_i \cdot f(x_i)}$. We have the following bounded difference condition on $\phi_2$. For all i,
%     \begin{align}
%         \sup_{x_1, \ldots x_m, x_i^\prime \in \calX^{m+1} } \abs{ \phi_2 (x_1,\ldots, x_i, \ldots, x_m) - \phi_1 (x_1,\ldots, x_i^\prime, \ldots, x_m)  } \le 1/m \,. \label{cond2_rademacher}
%     \end{align}
%     Using McDiarmid’s inequality (\lemref{lem:McDiarmid}) twice with Condition \eqref{cond1_rademacher} and \eqref{cond2_rademacher}, with probability at least $1-\delta$, we have
%     \begin{align}
%         \sup_{f \in \calF} \, \frac{1}{m} \sum_{i=1}^m y_i \cdot f(x_i)  - \Expt{x,y}{\sup_{f \in \calF} \, \frac{1}{m} \sum_{i=1}^m y_i \cdot f(x_i) } \le \sqrt{\frac{2\log(2/\delta)}{m}} \label{eq:final_rademacher}
%     \end{align} 
%     Combining \eqref{eq:error_rademacher} and \eqref{eq:final_rademacher}, we obtain the desired result. 
% \end{proof}


% \subsection{Proof of \thmref{thm:error_regularized_ERM}}

% Proof of \thmref{thm:error_regularized_ERM} follows similar to the proof of \thmref{thm:error_ERM}. Note that the same results in \lemref{lem:fit_mislabeled}, \lemref{lem:mislabeled_error}, and \lemref{lem:clear_error} hold in the regularized ERM case. However, the arguments in the proof of \lemref{lem:fit_mislabeled} changes slightly. Hence, we state and prove a lemma parallel to \lemref{lem:fit_mislabeled} for completeness. 

% \begin{lemma} \label{lem:lemma1_reg}
%     Assume the same setup as \thmref{thm:error_regularized_ERM}. 
%     Then for any $\delta >0$, with probability at least  $1-\delta$ 
%     over the random draws of mislabeled data $\wt S_M$, we have 
%     \begin{align}
%         \error_\calD(\widehat f)  \le 1 -\error_{\wt \calS_M}(\widehat f) + \sqrt{\frac{\log(1/\delta)}{m}}\,. 
%     \end{align} 
% \end{lemma}
% \begin{proof}
%     The main idea of the proof remains the same, i.e. regard 
%     the clean portion of the data 
%     ($S \cup \wt S_C$) as fixed.   
%     Then, there exists a classifier $f^*$ 
%     that is optimal over draws 
%     of the mislabeled data $\wt S_M$. 

    
%     Formally, 
%     \begin{align}
%     f^* \defeq \argmin_{f \in \calF} \error_{\widecheck {\calD}} (f)  + \lambda R(f) \,, \label{eq:modified_ERM_reg}
%     \end{align}
%     where $$\widecheck \calD = \frac{n}{m+n} \calS + \frac{m_1}{m+n} \wt \calS_C  + \frac{m_2}{m+n}\calDm \,.$$ That is, $\widecheck \calD$ a combination of 
%     the \emph{empirical distribution} 
%     over correctly labeled data $S \cup \wt S_C$
%     % in $S\cup \wt S$ 
%     and the (population) distribution 
%     over mislabeled data $\calDm$.
%     Recall that 
%     \begin{align}
%     \wh f \defeq \argmin_{f \in \calF} \error_{\calS \cup \wt S} (f) + \lambda R(f) \,. \label{eq:orig_ERM_reg}
%     \end{align}
%     % 
%     % 
%     Since, $\widehat f$ minimizes 0-1 error 
%     on $S \cup \wt S$, using ERM optimality on \eqref{eq:orig_ERM},  
%     we have 
%     \begin{align}
%         \error_{\calS \cup \wt \calS}(\widehat f) + \lambda R(\wh f) \le \error_{
%             \calS \cup \wt \calS}(f^*) + \lambda R(f^*) \,.    \label{eq:step1_reg}
%     \end{align}
%     Moreover, since $f^*$ is independent of $\wt S_M$, using Hoeffding's bound,
%     % \footnote{For a fully rigorous argument,
%     % refer to the complete proof in App.~\ref{app:proof_erm}.} 
%     we have with probability at least $1-\delta$ that
%     \begin{align}
%       \error_{\wt \calS_M}(f^*) \le \error_{ \calDm}(f^*) +  \sqrt{\frac{\log(1/\delta)}{2 m_1}} \,. \label{eq:step2_reg} 
%     \end{align}
%     %$ 
%     %for some constant $c_1\le 1/2$. 
%     Finally, since $f^*$ is the optimal classifier on $\widecheck \calD$, 
%     we have 
%     \begin{align}
%         \error_{\widecheck \calD}(f^*) + \lambda R(f^*) \le \error_{\widecheck \calD}(\widehat f) + \lambda R(\wh f) \label{eq:step3_reg}
%     \end{align}
%      Now to relate \eqref{eq:step1_reg} and \eqref{eq:step3_reg}, we can re-write the \eqref{eq:step2_reg} as follows: 
%     \begin{align}
%         \error_{\calS \cup \wt\calS}(f^*) \le \error_{ \widecheck \calD}(f^*) +  \frac{m_1}{m+n}\sqrt{\frac{\log(1/\delta)}{2 m_1}} \,. \label{eq:step4_reg} 
%     \end{align}
%     After adding $\lambda R(f^*)$ on both sides in \eqref{eq:step4_reg}, we combine equations \eqref{eq:step1_reg}, \eqref{eq:step4_reg}, and \eqref{eq:step3_reg}, to get 
%     \begin{align}
%         \error_{\calS \cup \wt \calS}(\wh f) \le \error_{\widecheck \calD}(\wh f) +  \frac{m_1}{m+n}\sqrt{\frac{\log(1/\delta)}{2 m_1}} \,, 
%     \end{align}
%     which implies 
%     \begin{align}
%         \error_{ \wt \calS_M}(\wh f) \le \error_{\calDm}(\wh f) + \sqrt{\frac{\log(1/\delta)}{2 m_1}} \,. \label{eq:lemma_reg_final}
%     \end{align}
%     Similar as before, since $\wt S$ is obtained by randomly labeling an unlabeled dataset, we assume 
%     $2m_1 \approx m$. Moreover, using $\error_{\calDm} = 1 - \error_{\calD}$ we obtain the desired result. 
% \end{proof}
% % \begin{proof}[Proof of ]
    
% % \end{proof}

% \subsection{Proof of \thmref{thm:multiclass_ERM}}

% We first state and prove lemmas parallel to three lemmas used in the proof of balanced binary case. Then we combine the results in the three lemmas to obtain the result in \thmref{thm:multiclass_ERM}. 

% Before stating the result, we define mislabeled distribution $\calDm$ for any $\calD$. While $\calDm$ and $\calD$ share 
% the same marginal distribution over $\calX$, 
% the distribution over labels $y$ 
% given an input $x\sim \calD_\calX$ is changed.
% In particular, for any $x$, the pdf over $y$ is changed to:  
% $p_{\calDm} (\cdot \vert x) \defeq \frac{1 - p_{\calD}(\cdot \vert x)}{k - 1}$.

% \begin{lemma} \label{lem:fit_mislabeled_multi}
%     Assume the same setup as \thmref{thm:multiclass_ERM}. 
%     Then for any $\delta >0$, with probability at least  $1-\delta$ 
%     over the random draws of mislabeled data $\wt S_M$, we have 
%     \begin{align}
%         \error_\calD(\widehat f)  \le (k-1)\left(1 -\error_{\wt \calS_M}(\widehat f)\right) + (k-1)\sqrt{\frac{\log(1/\delta)}{m}}\,. \label{eq:lemma1_multi}
%     \end{align}   
% \end{lemma} 

% \begin{proof}
%     The main idea of the proof remains the same, i.e. regard 
%     the clean portion of the data 
%     ($S \cup \wt S_C$) as fixed. 
%     Then, there exists a classifier $f^*$ 
%     that is optimal over draws 
%     of the mislabeled data $\wt S_M$. 
    
%     However, we need to be careful while relating population error on mislabeled data with population accuracy on clean data.   
%     While for binary classification,  we could upper bound $\error_{\wt \calS_M}$ 
%     with $1-\error_\calD$  (in the proof of \lemref{lem:fit_mislabeled}), 
%     for multiclass classification, 
%     error on the mislabeled data 
%     and accuracy on the clean data 
%     in the population 
%     are not so directly related.  
%     To establish \eqref{eq:lemma1_multi},
%     we break the error on the 
%     (unknown) mislabeled data 
%     into two parts: one term corresponds 
%     to predicting the true label on mislabeled data, 
%     and the other corresponds to predicting 
%     neither the true label 
%     nor the assigned (mis-)label.  
%     Finally, we relate these errors to their
%     population counterparts to establish \eqref{eq:lemma1_multi}. 
    
%     Formally, 
%     \begin{align}
%     f^* \defeq \argmin_{f \in \calF} \error_{\widecheck {\calD}} (f)  + \lambda R(f) \,, \label{eq:modified_ERM_reg2}
%     \end{align}
%     where $$\widecheck \calD = \frac{n}{m+n} \calS + \frac{m_1}{m+n} \wt \calS_C  + \frac{m_2}{m+n}\calDm \,.$$ That is, $\widecheck \calD$ a combination of 
%     the \emph{empirical distribution} 
%     over correctly labeled data $S \cup \wt S_C$
%     % in $S\cup \wt S$ 
%     and the (population) distribution 
%     over mislabeled data $\calDm$.
%     Recall that 
%     \begin{align}
%     \wh f \defeq \argmin_{f \in \calF} \error_{\calS \cup \wt S} (f) + \lambda R(f) \,. \label{eq:orig_ERM_reg2}
%     \end{align}
%     % 
%     % 
%     Following the exact steps from the proof of \lemref{lem:lemma1_reg}, with probability at least $1-\delta$, we have  
%     \begin{align}
%         \error_{ \wt \calS_M}(\wh f) \le \error_{\calDm}(\wh f) + \sqrt{\frac{\log(1/\delta)}{2 m_1}} \,. \label{eq:lemma1_final_multi_prev}
%     \end{align}
%     Similar to before, since $\wt S$ is obtained by randomly labeling an unlabeled dataset, we assume 
%     $\frac{k}{k-1} m_1 \approx m$. 
    
%     Now we will relate $\error_\calDm (\wh f)$ with $\error_{\calD}(\wh f)$. Let $y^T$ denote the (unknown) true label for a mislabeled point $(x, y)$ (i.e., label before replacing it with a mislabel). 
%     \begin{align}    
%          \Expt{(x, y) \in \sim \calDm}{\indict{ \wh f(x) \ne y }}  &= \underbrace{\Expt{(x, y) \in \sim \calDm}{\indict{ \wh f(x) \ne y \land \wh f(x) \ne y^T}}}_{\RN{1}} + \underbrace{\Expt{(x, y) \in \sim \calDm}{\indict{ \wh f(x) \ne y \land \wh f(x) = y^T}}}_{\RN{2}} \,. \label{eq:excess_term}
%     \end{align}
%     Clearly, term 2 is one minus the accuracy on the clean unseen data, i.e. 
%     \begin{align}
%         \RN{2} = 1 - \Expt{{x,y} \sim \calD}{ \indict{ \wh f(x) \ne y}} = 1- \error_{\calD}(\wh f) \,. \label{eq:term1}    
%     \end{align}
%     Next, we  relate term 1 with the error on the unseen clean data. We show that term 1 is equal to the error on the unseen clean data scaled by $\frac{k-2}{k-1}$ where $k$ is the number of labels. Using the definition of mislabeled distribution $\calDm$,  we have 
%     \begin{align}
%         \RN{1} = \frac{1}{k-1} \left( \Expt{(x, y) \in \sim \calD}{ \sum_{i \in \calY \land i\ne y}  \indict{ \wh f(x) \ne i \land \wh f(x) \ne y}} \right) = \frac{k-2}{k-1} \error_{\calD}(\wh f) \,.\label{eq:term2}
%     \end{align}    

%     Combining the result in \eqref{eq:term1}, \eqref{eq:term2} and \eqref{eq:excess_term}, we have 
%     \begin{align}
%         \error_{\calDm}(\wh f) = 1- \frac{1}{k-1} \error_{\calD}(\wh f) \,.\label{eq:combine_terms}
%     \end{align}
%     Finally, combining the result in \eqref{eq:combine_terms} with equation \eqref{eq:lemma1_final_multi_prev}, we have with probability $1-\delta$, 
%     \begin{align}
%       \error_{\calD}(\wh f) \le  (k-1) \left( 1- \error_{ \wt \calS_M}(\wh f) \right)  + (k-1) \sqrt{\frac{k \log(1/\delta)}{ 2(k-1)m}} \,. \label{eq:lemma1_final_multi}
%     \end{align}
% \end{proof}

% \begin{lemma} \label{lem:mislabeled_error_multi}
%     Assume the same setup as \thmref{thm:multiclass_ERM}.  Then for any $\delta >0$, with probability at least $1-\delta$ over the random draws of $\wt S$, we have  
%     % \begin{align}
%         $$\abs{k\error_{\wt \calS}(\widehat f) - \error_{\wt \calS_C}(\widehat f) -  (k-1)\error_{\wt \calS_M}(\widehat f) } \le  2k\sqrt{\frac{\log(4/\delta)}{2m}}\,. $$ % \label{eq:lemma2}
%     % \end{align}   
%     %  for some constant $c_3 \le 1.0\,$.
% \end{lemma} 


% \begin{proof}
%     Recall $\error_{\wt S} (f) = \frac{m_1}{m} \error_{\wt S_M}(f) + \frac{m_2}{m} \error_{\wt S_C}(f)$. Hence, we have 
%     \begin{align}
%         k\error_{\wt S}(f) - (k-1)\error_{\wt S_M}(f) - \error_{\wt S_C}(f) &= (k-1)\left(\frac{k m_1}{(k-1) m} \error_{\wt S_M}(f) - \error_{\wt S_M}(f)\right) + \left(\frac{km_2}{m} \error_{\wt S_C}(f) - \error_{\wt S_C}(f)\right) \\ &= k \left[ \left(\frac{m_1}{m} - \frac{k-1}{k}\right) \error_{\wt S_M}(f) + \left(\frac{m_2}{m} - \frac{1}{k} \right) \error_{\wt S_C} (f) \right] \,.
%     \end{align} 
%     Since the dataset is randomly labeled, we have with probability at least $1-\delta$, $\left(\frac{m_1}{m} - \frac{k-1}{k}\right) \le \sqrt{\frac{\log(1/\delta)}{2m}}$. Similarly, we have with probability at least $1-\delta$, $\left(\frac{m_2}{m} - \frac{1}{k}\right) \le \sqrt{\frac{\log(1/\delta)}{2m}}$. Using union bound, we have with probability at least $1-\delta$
%     % \begin{align}
%     %     2\error_{\wt S} - \error_{\wt S_M}(f) - \error_{\wt S_C}(f) \le \sqrt{\frac{\log(2/\delta)}{2m}} \left(\error_{\wt S_M}(f) + \error_{\wt S_C}(f) \right) \le 2\sqrt{\frac{\log(2/\delta)}{2m}} \,. \label{eq:lemma2_final}
%     % \end{align}
%     \begin{align}
%         k\error_{\wt S}(f) - (k-1)\error_{\wt S_M}(f) - \error_{\wt S_C}(f)  \le k \sqrt{\frac{\log(2/\delta)}{2m}} \left(\error_{\wt S_M}(f) + \error_{\wt S_C}(f) \right) \,. \label{eq:lemma2_final_multi}
%     \end{align}

%     % We obtain the desired result by using 
% \end{proof}

% \begin{lemma} \label{lem:clear_error_multi}
%     Assume the same setup as \thmref{thm:multiclass_ERM}. 
%     Then for any $\delta >0$, with probability at least $1-\delta$ 
%     over the random draws of $\wt S_C$ and $S$, we have 
%     % \begin{align}
%         $$\abs{\error_{\wt \calS_C}(\widehat f) - \error_{\calS}(\widehat f) } \le 1.5 \sqrt{\frac{k\log(2/\delta)}{2m}}\,.$$ %\label{eq:lemma3}
%     % \end{align}   
%     % for some constant $c_2 \le 1.2\,$.
% \end{lemma} 
% \begin{proof}
%     % Recall 0-1 error on each point  $(x,y) \in S \cup \wt S$ is given by $\I{ f(x)\ne y}$.
%     In the set of correctly labeled points $S \cup \wt S_C$, we have $S$ as a random subset of $S \cup \wt S_C$. Hence, using Hoeffding's inequality for sampling without replacement (\lemref{lem:hoeffding_sampling}), we have with probability at least $1-\delta$
%     \begin{align}
%         \error_{\wt \calS_c} (\wh f)- \error_{\calS \cup \wt \calS_C}( \wh f) \le  \sqrt{\frac{\log(1/\delta)}{2m_2}} \,.
%     \end{align}
%     Re-writing $\error_{\calS \cup \wt \calS_C}( \wh f)$ as $\frac{m_2}{m_2 + n} \error_{\wt \calS_C }(\wh f) + \frac{n}{m_2 + n} \error_{\calS }(\wh f)$, we have with probability at least $1-\delta$
%     \begin{align}
%       \left(\frac{n}{n+m_2}\right) \left(\error_{\wt \calS_c} (\wh f)- \error_{\calS}( \wh f) \right) \le  \sqrt{\frac{\log(1/\delta)}{2m_2}} \,.
%     \end{align}
%     As before, assuming $km_2 \approx m$, we have with probability at least $1-\delta$ 
%     \begin{align}
%         \error_{\wt \calS_c} (\wh f)- \error_{\calS}( \wh f) \le \left(1+\frac{m_2}{n}\right)  \sqrt{\frac{k\log(1/\delta)}{2m}} \le \left( 1 + \frac{1}{k}\right) \sqrt{\frac{k\log(1/\delta)}{2m}} \,. \label{eq:lemma3_final_multi}
%     \end{align} 
% \end{proof}

% \begin{proof}[Proof of \thmref{thm:multiclass_ERM}] 
%     Having established these core intermediate results, we can now combine above three lemmas. 
%     In particular, we bound the population error on clean data ($\error_\calD(\wh f)$) as follows:  
%     \begin{enumerate}[(i)]
%         \item First, use \eqref{eq:lemma1_final_multi}, to obtain an upper bound on the population error on clean data, i.e., with probability at least $1-\delta/4$, we have
%         \begin{align}
%             \error_{ \calD} (\wh f) \le (k-1)\left(1 - \error_{ \wt \calS_M}(\wh f) \right) + (k-1) \sqrt{\frac{k\log(4/\delta)}{2(k-1)m}} \,. 
%         \end{align}
%         \item  Second, use \eqref{eq:lemma2_final_multi}, to relate the error on the mislabeled fraction with error on clean portion of randomly labeled data and error on whole randomly labeled dataset, i.e., with probability at least $1-\delta/2$, we have 
%         \begin{align}
%             - (k-1)\error_{\wt S_M}(f) \le \error_{\wt S_C}(f) - k\error_{\wt S}  + k\sqrt{\frac{\log(4/\delta)}{2m}}  \,. 
%         \end{align} 
%         \item Finally, use \eqref{eq:lemma3_final_multi} to relate the error on the clean portion of randomly labeled data and error on clean training data, i.e., with probability $1-\delta/4$, we have 
%         \begin{align}
%             \error_{\wt \calS_C} (\wh f)\le - \error_{\calS}( \wh f) + \left(1 + \frac{m}{kn} \right) \sqrt{\frac{k\log(4/\delta)}{2m}} \,. 
%         \end{align} 
%     \end{enumerate}

%     Using union bound on the above three steps, we have with probability at least $1-\delta$: 
%     \begin{align}
%         \error_\calD (\wh f) \le \error_{\calS}(\wh f) + (k-1) - k\error_{\wt \calS}(\wh f)   + (\sqrt{k(k-1)} + k + \sqrt{k} + \frac{m}{n\sqrt{k}})  \sqrt{\frac{\log(4/\delta)}{2m}} \,.
%     \end{align}
%     % Note that $\frac{m}{n\sqrt{k}}$ is much smaller than the other terms in addition. Hence, we ignore this in the final bound. 
%     % Note that $(1/\sqrt{2} + 2.5)$ is a loose constant. In experiments, we use the ratio $\frac{m}{n}$
%     %  the exact error $\error_{\wt \calS}(\wh f)$ 
%     % to evaluate R.H.S.    
% \end{proof}

% \newpage
% \section{Proofs from \secref{sec:linear_models}}\label{app:proof_gd}

% We suppose that the parameters of the linear function 
% are obtained via gradient descent on 
% the following $L_2$ regularized problem: 
% \begin{align}
%     % n in denominator is avoided deliberately
%     \calL_S(w; \lambda) \defeq \sum_{i=1}^n{(w^Tx_i - y_i)^2} + \lambda \norm{w}{2}^2 \,, \label{eq:l2_MSE_app}   
% \end{align}
% where $\lambda\ge0$ is a regularization parameter. 
% We assume access to a clean dataset 
% $S = \{(x_i, y_i)\}_{i=1}^n \sim \calD^n$ 
% and randomly labeled dataset 
% $\wt S = \{(x_i, y_i)\}_{i=n+1}^{n+m} \sim \wt \calD^m$. 
% Let $\bX = [x_1, x_2, \cdots, x_{m+n}]$ 
% and $\by = [y_1, y_2, \cdots, y_{m+n}]$. 
% Fix a positive learning rate $\eta$ such that 
% $\eta \le 1/\left(\norm{\bX^T\bX}{\text{op}} + \lambda^2\right)$ 
% and an initialization $w_0 = 0$. 
% % \todos{Assumption made for simplicty}. 
% Consider the following gradient descent iterates 
% to minimize objective \eqref{eq:l2_MSE_app} on $S \cup \wt S$:
% \begin{align}
% w_t = w_{t-1} - \eta \grad_w \calL_{S \cup \wt S} (w_{t-1}; \lambda) \quad \forall t=1,2,\ldots \label{eq:GD_iterates_app}
% \end{align} 
% Then we have $\{ w_t\}$ converge to the limiting solution 
% $\wh w = \left( \bX^T\bX+\lambda \boldsymbol{I}\right)^{-1}\bX^T\by$. Define $\widehat f (x) \defeq f(x ; \wh w) $.  

% \subsection{\textcolor{red}{Errata}}

% We wish to correct the following error in the body: \codref{cond:error_stability} is not enough to guarantee the result in \thmref{thm:linear}. We now present a slightly stronger condition called \emph{hypothesis stability} under which we obtain a result similar to \thmref{thm:linear}. 

% This error doesn't change the main arguments of the proof where we show that the empirical train error is less than or equal to the leave-one-out error. We need a stronger condition to relate leave-one-out error with the population error of the original classifier. Specifically, while \codref{cond:error_stability} relates the average population error of leave-one-out classifiers with the population error of the original classifier, we need the new condition to show the concentration of the empirical leave-one-out error and  average population error of leave-one-out classifiers. 
% % main takeaway 

% Note that the new condition, while being stronger than the previous one, still doesn't imply generalization~\cite{bousquet2002stability,elisseeff2003leave,abou2019exponential}. Overall, the main results in \secref{sec:ERM_training} and takeaways of the paper remain unaffected by the error.  

% We now present the new condition and a corrected statement of \thmref{thm:linear}. Recall, for a given training set $S \sim \calD^n $, 
% we use $S_{(i)}$ to denote the training set $S$ 
% with the $i^{\text{th}}$ point removed.

% \begin{condition}[Hypothesis Stability] 
%     \label{cond:hypothesis_stability}
%     We have $\beta$ hypothesis stability 
%     if our training algorithm $\calA$ satisfies the following: 
%     \begin{align*}
%     % ${\sum_{i=1}^n \frac{\error_{\calD}( f(\calA, S_{(i)}))}{n} - \error_\calD(f(\calA, S))} \le \beta\,$.
%     \forall i \in \{1,2,\ldots, n\}, \quad  \Expt{\calS, (x,y) \in \calD}{ \abs{\error\left( f(x) ,y  \right) - \error\left( f_{(i)}(x), y \right) }} \le \frac{\beta}{n} \,,
%     \end{align*}
%     where $f_{(i)} \defeq f(\calA, S_{(i)})$ and $ f \defeq f(\calA, S)$.
% \end{condition}

% \begin{theorem}[Correct statement of \thmref{thm:linear}] \label{thm:new_linear}
%     Assume that this gradient descent algorithm satisfies \codref{cond:hypothesis_stability}
%     with $\beta=\calO(1)$.  
%     Then for any $\delta >0$, with probability at least $1-\delta$ 
%     over the random draws of datasets $\wt S$ and $S$, we have:
%     \begin{align}
%         \error_\calD(\widehat f) \le \error_\calS(\widehat f) + 1 - 2 \error_{\wt\calS}(\widehat f) + \left(\frac{1}{\sqrt{2}} + 1.5 \right) \sqrt{\frac{\log(4/\delta)}{m}} + \sqrt{\frac{4}{\delta}\left(\frac{1}{m} +\frac{3\beta}{m+n} \right)}  \,. \label{eq:gd_error}
%     \end{align} 
%     % for some constant $c\le 3.2$.
% \end{theorem}

% \subsection{Proof of \thmref{thm:new_linear}}
% We use a standard result from linear algebra, namely Shermann-Morrison formula~\citep{sherman1950adjustment} for matrix inversion:  

% \begin{lemma}[\citet{sherman1950adjustment}] \label{lem:sherman}
%     Suppose $\bA \in \Real^{n \times n}$ is an invertible square matrix and $u,v \in \Real^n$ are column vectors. Then $\bA + uv^T$ is invertible iff $1 + v^T \bA u \ne 0$ and in particular
%     \begin{align}
%         (\bA + u v^T)^{-1} = \bA^{-1}  - \frac{\bA^{-1} uv^T \bA^{-1} }{ 1 + v^T \bA^{-1} u} \,.
%     \end{align}   
% \end{lemma}
% \newcommand\byy[1]{\by_{\left(#1\right)}}
% \newcommand\bXX[1]{\bX_{\left(#1\right)}}
% \newcommand\ff[1]{\wh f_{\left(#1\right)}}

% For a given training set $S \cup \wt S_C$, define leave-one-out error on mislabeled points in the training data as $$\error_{\text{LOO}(\wt S_M) } = \frac{\sum_{(x_i, y_i) \in \wt S_M} \error( f_{(i)}( x_i), y_i)}{ \abs{\wt S_M }} \,, $$
% where $f_{(i)} \defeq f(\calA, (S \cup \wt S)_{(i)})$. To relate empirical leave-one-out error and population error with hypothesis stability condition, we use the following lemma:   

% \begin{lemma}[\citet{bousquet2002stability}] \label{lem:stability_error}
%     For the leave-one-out error, we have
%     \begin{align}
%         \Expo{ \left( \error_{\calDm}(\wh f) -\error_{\text{LOO}(\wt S_M) } \right)^2 } \le \frac{1}{2m_1}+  \frac{3\beta}{n + m}\,.
%     \end{align}   
%     % where $ f \defeq f(\calA, S \cup \wt S) $.
% \end{lemma}

% Proof of the above lemma is similar to the proof of  Lemma 9 in \citet{bousquet2002stability} and can be found in \appref{app:proof_lem_error}. 
% % 
% % Before presenting the result, we introduce some notation. 
% Before presenting the proof of \thmref{thm:new_linear}, we introduce some more notation. Let $\bX_{(i)}$ denote the matrix of covariates with $i^{\text{th}}$ point removed. Similarly let $\by_{(i)}$ be the array of responses with $i^{\text{th}}$ point removed. Define the corresponding regularized GD solution as $\wh w_{(i)} = \left( \bXX{i}^T\bXX{i}+\lambda \boldsymbol{I}\right)^{-1}\bXX{i}^T\byy{i}$. Define $\ff{i}(x) \defeq f(x ; \wh w_{(i)}) $.

% \begin{proof}[Proof of \thmref{thm:new_linear}]
%     Because squared loss minimization does not imply 0-1 error minimization, we cannot use arguments from \lemref{lem:fit_mislabeled}. This is the main technical difficulty. To compare the 0-1 error at a train point with an unseen point, 
%     we use the closed-form expression for $\widehat{w}$ and Shermann-Morrison formula to upper bound training error with leave-one-out cross validation error. 
    
%     The proof is divided into three parts: In part one, we show that 0-1 error on mislabeled points in the training set is lower than the error obtained by leave-one-out error at those points. In part two, we relate this leave-one-out error with the population error on mislabeled distribution using \codref{cond:hypothesis_stability}. While the empirical leave-one-out error is unbiased estimator of the average population error of leave-one-out classifiers, we need hypothesis stability to control the variance of empirical leave-one-out error. Finally in part three, we show that the error on the mislabeled training points can be estimated with just the randomly labeled and  clean training data (as in proof of \thmref{thm:error_ERM}).  

%     \textbf{Part 1 {} {}} First we relate training error with leave-one-out error.        
%     For any 
%     training point $(x_i, y_i)$ in $\wt S \cup S$, we have 
%     \begin{align}
%         \error(\wh f(x_i), y_i ) &= \indict{ y_i \cdot x_i^T \wh w < 0 } = \indict{ y_i \cdot x_i^T \left( \bX^T\bX+\lambda \boldsymbol{I}\right)^{-1}\bX^T\by < 0 } \\
%         &= \indict{ y_i \cdot x_i^T \underbrace{\left( \bXX{i}^T\bXX{i} + x_i ^T x_i +\lambda \boldsymbol{I}\right)^{-1}}_{\RN{1}} (\bXX{i}^T\byy{i} + y \cdot x_i) < 0 }
%     \end{align}
%     Letting $\bA = \left(\bXX{i}^T\bXX{i} +\lambda \boldsymbol{I}\right)$ and using \lemref{lem:sherman} on term 1, we have 
%     \begin{align}
%         \error(\wh f(x_i), y_i ) &= \indict{ y_i \cdot x_i^T \left[\bA^{-1} -  \frac{\bA^{-1} x_i x_i^T \bA^{-1}}{ 1 + x_i ^T \bA^{-1} x_i } \right] (\bXX{i}^T\byy{i} + y \cdot x_i) < 0 } \\
%         &= \indict{ y_i \cdot\left[ \frac{ x_i^T \bA^{-1} ( 1 + x_i ^T \bA^{-1} x_i ) -  x_i^T \bA^{-1} x_i x_i^T \bA^{-1}}{ 1 + x_i ^T \bA ^{-1}x_i } \right] (\bXX{i}^T\byy{i} + y \cdot x_i) < 0 } \\
%         &= \indict{ y_i \cdot\left[ \frac{ x_i^T \bA^{-1}}{ 1 + x_i ^T \bA ^{-1}x_i } \right] (\bXX{i}^T\byy{i} + y \cdot x_i) < 0 } \,.
%     \end{align}

%     Since $1 + x_i^T \bA^{-1} x_i > 0$, we have 
%     \begin{align}
%         \error(\wh f(x_i), y_i ) &= \indict{ y_i \cdot x_i^T \bA^{-1} (\bXX{i}^T\byy{i} + y \cdot x_i) < 0 } \\
%         &= \indict{ x_i^T \bA^{-1} x_i +  y_i \cdot x_i^T \bA^{-1} (\bXX{i}^T\byy{i}) < 0 } \\
%         &\le \indict{ y_i \cdot x_i^T \bA^{-1} (\bXX{i}^T\byy{i}) < 0 } = \error(\ff{i}(x_i), y_i ) \,.\label{eq:LOO_error}
%     \end{align}

%     Using \eqref{eq:LOO_error}, we have 
%     \begin{align}
%         \error_{\wt \calS_M } (\wh f) \le \error_{\text{LOO} (S_M)} \defeq \frac{\sum_{(x_i, y_i) \in \wt S_M} \error(\ff{i}(x_i), y_i ) }{\abs{\wt \calS_M}}\label{eq:LOO_error_final}
%     \end{align}
%     \textbf{Part 2 {}{}} We now relate RHS in \eqref{eq:LOO_error_final} with the population error on mislabeled distribution. To do this, we leverage \codref{cond:hypothesis_stability} and \lemref{lem:stability_error}. In particular, we have 

%     \begin{align}
%         \Expt{\calS \cup \wt \calS_M }{ \left(\error_{\calDm}(\wh f) - \error_{\text{LOO} (S_M)}\right)^2 } \le \frac{1}{2m_1} + \frac{3\beta}{m+n} \,.
%     \end{align}

%     Using Chebyshev's inequality, with probability at least $1-\delta$, we have 
%     \begin{align}
%         \error_{\text{LOO} (S_M)} \le  \error_{\calDm}(\wh f)   + \sqrt{\frac{1}{\delta}\left(\frac{1}{2m_1} +\frac{3\beta}{m+n} \right)} \,. \label{eq:final_mislabeled_linear}
%     \end{align}
    

%     \textbf{Part 3 {}{}} Combining \eqref{eq:final_mislabeled_linear} and \eqref{eq:LOO_error_final}, we have 

%     \begin{align}
%         \error_{\wt \calS_M } (\wh f) \le \error_{\calDm}(\wh f)   + \sqrt{\frac{1}{\delta}\left(\frac{1}{2m_1} +\frac{3\beta}{m+n} \right)} \,. \label{eq:linear_parallel_lem1}
%     \end{align}

%     Compare \eqref{eq:linear_parallel_lem1}, with \eqref{eq:lemma1_final} in the proof of \lemref{lem:fit_mislabeled}. We obtain a similar relationship between $\error_{\wt \calS_M }$ and $\error_{\calDm}$ but with a polynomial concentration instead of exponential concentration. 
%     In addition, since we just use concentration arguments to relate mislabeled error with the error on clean portion and unlabeled portion, we can directly use the results in \lemref{lem:mislabeled_error} and \lemref{lem:clear_error}. Therefore, combining results in \lemref{lem:mislabeled_error}, \lemref{lem:clear_error}, and \eqref{eq:linear_parallel_lem1} with union bound, we have with probability at least $1-\delta$

%     \begin{align}
%         \error_\calD(\widehat f) \le \error_\calS(\widehat f) + 1 - 2 \error_{\wt\calS}(\widehat f) + \left(\frac{1}{\sqrt{2}} + 1.5 \right) \sqrt{\frac{\log(4/\delta)}{m}} + \sqrt{\frac{4}{\delta}\left(\frac{1}{m} +\frac{3\beta}{m+n} \right)}  \,.
%     \end{align}
    

       
% \end{proof}

% \subsection{Discussion on \codref{cond:hypothesis_stability}}

% The quantity in LHS of \codref{cond:hypothesis_stability} measures how much the function learned by the algorithm (in terms of error on unseen point) will change when one point in the training set is removed. 
% % Discussion on exponential concentration and stronger condition. 
% Notice that hypothesis stability implies error stability, i.e., \codref{cond:error_stability} ~\cite{bousquet2002stability}.  In summary, while error stability allowed us to relate the average population error of the leave-one-out classifiers with the population error of the original classifier, we need hypothesis stability condition to control the variance of the empirical leave-one-out error. 

% Additionally, we note that while the dominating term in the RHS of \thmref{thm:new_linear} matches with the dominating term in ERM bound in \thmref{thm:error_ERM}, there is a polynomial concentration term (dependence on $1/\delta$ instead of $\log(\sqrt{1/\delta})$) in  \thmref{thm:new_linear}. 
% Since with hypothesis stability, we just bound the variance,  the polynomial concentration is due to the use of Chebyshev's inequality instead of an exponential tail inequality (as in \lemref{lem:fit_mislabeled}).
% Recent works have highlighted that slightly stronger condition than hypothesis stability can be used to obtained an exponential concentration for leave-one-out error~\citep{abou2019exponential}, but we leave this for future work for now. 
% % We leave 
% % However, the constants 

% % we also want to highlight  

% \subsection{Formal statement and proof of  of \propref{prop:early_stop}}

% Before formally presenting the result, we will introduce some notation.  By $\calL_{S}(w)$, we denote 
% the objective in \eqref{eq:l2_MSE_app} with $\lambda=0$. 
% Assume Singular Value Decomposition (SVD) of $\bX$  as $\sqrt{n} \bU \bS^{1/2} \bV^T$. Hence $\bX^T \bX = \bV \bS \bV^T$.
% Consider the GD iterates defined in \eqref{eq:GD_iterates_app}. 
% % 
% We now derive closed form expression for the $t^\text{th}$ iterate of gradient descent:  
% % 
% \begin{align}
%     w_t = w_{t-1} + \eta \cdot \bX^T (\by - \bX w_{t-1}) = (\bI - \eta \bV \bS \bV^T )w_{k-1} + \eta \bX^T \by \,.
% \end{align}
% Rotating by $\bV^T$, we get 
% \begin{align}
%     \wt w_t = (\bI - \eta\bS )\wt w_{k-1} + \eta \wt \by \,, \label{eq:GD_recur}
% \end{align}
% where $\wt w_t = \bV^T w_t $ and $\wt \by = \bV^T \bX^T \by$. Assuming the initial point $w_0 = 0$ and applying the recursion in \eqref{eq:GD_recur}, we get
% \begin{align}
%     \wt w_t = \bS ^{-1} ( \bI - (\bI - \eta \bS)^k ) \wt \by \,, 
% \end{align} 
% Projecting solution back to the original space, we have 
% \begin{align}
%      w_t = \bV \bS ^{-1} ( \bI - (\bI - \eta \bS)^k ) \bV^T \bX^T \by \,, 
% \end{align} 
% % We will work with this GD solution at any iterate $t$ in the next proposition. 
% Define $f_t(x) \defeq f(x;w_t)$ as the solution at the $t^{\text{th}}$ iterate. 
% Let $\wt w_{\lambda} = \argmin_{w} \calL_\calS (w;\lambda) = (\bX^T \bX + \lambda \bI)^{-1} \bX^T \by = \bV (\bS + \lambda \bI )^{-1} \bV^T \bX^T \by $. 
% % ) \,,$ for all $t=1,2,\ldots\,.$ 
% and define $\wt f_\lambda(x) \defeq f(x;\wt w_\lambda)$ as the regularized solution. 
% Assume $\kappa$ be the condition number of the population covariance matrix 
% and 
% let $s_\text{min}$ be the minimum positive singular value of the empirical covariance matrix. Our proof idea is inspired from recent work on relating gradient flow solution and regularized solution for regression problems \citep{ali2018continuous}. We will use the following lemma in the proof: 
% \begin{lemma} \label{lem:ineq_soln}
%     For all $x \in [0,1]$ and for all $ k \in \mathbb{N}$, we have (a) $ \frac{kx}{1+kx} \le 1- (1-x)^k$ and (b) $ 1- (1-x)^k \le 2 \cdot \frac{kx}{kx+1} $.
%     %  where $g(c)$ is a constant dependent on $c$. For $c = 1$, $g(c) = 2.0$.   
% \end{lemma}
% \begin{proof}
%     % [Proof of \lemref{lem:ineq_soln}]
%     % Part (a) is easy. 
%     Using $ (1-x)^k \le \frac{1}{1+kx}$, we have part (a). For part (b), we numerically maximize $\frac{ (1+kx ) (1 - (1-x)^k) }{kx}$ for all $k\ge 1$ and for all $x \in [0, 1]$.  
% \end{proof}

% % 
% % Next, 

% \begin{prop}[Formal statement of \propref{prop:early_stop}] \label{prop:formal_early_stop}
% Let $\lambda = \frac{1}{t\eta}$. For a training point $x$, we have 
% \begin{align*}
%     \Expt{x \sim \calS}{(f_t(x) - \wt f_\lambda(x))^2} &\le c(t,\eta) \cdot \Expt{x \sim \calS}{f_t(x)^2} \,, %\label{eq:early_stop}
% \end{align*}
% where $c(t, \eta) \defeq \min( 0.25, \frac{1}{s_\text{min}^2 t^2 \eta^2})$. Similarly for a test point, we have 
% \begin{align*}
%     \Expt{x \sim \calD_\calX}{(f_t(x) - \wt f_\lambda(x))^2} &\le \kappa \cdot c(t,\eta) \cdot \Expt{x \sim \calD_\calX}{f_t(x)^2} \,. %\label{eq:early_stop}
% \end{align*}
% \end{prop} 

% \begin{proof}
%     %%%%%%%%%%%%% 
%     We want to analyze the expected squared difference output of regularized linear regression with regularization constant $\lambda = \frac{1}{\eta t}$ and gradient descent solution at $t^\text{th}$ iterate. We separately expand the algebraic expression for squared difference at a training point and a test point. 
%     % We start by considering the difference  
%     Then the main step is to show that  $\left[ \bS ^{-1} ( \bI - (\bI - \eta \bS)^k )  - (\bS + \lambda \bI )^{-1}\right] \preceq c(\eta, t) \cdot \bS ^{-1} ( \bI - (\bI - \eta \bS)^k ) $.

%     %%%%%%%%%%%%%
    
%   \textbf{Part 1 {} {}} 
%     First, we will analyze the squared difference of output at a training point (for simplicity, we refer to $S \cup \wt S$ as $S$), i.e. 
%     \begin{align}
%         \Expt{ x \sim \calS }{\left(f_t(x) - \wt f_\lambda (x)\right)^2} &= \norm{\bX w_t - \bX \wt w_\lambda}{2}^2 =   \norm{\bX \bV \bS ^{-1} ( \bI - (\bI - \eta \bS)^t ) \bV^T \bX^T \by - \bX \bV (\bS + \lambda \bI )^{-1} \bV^T \bX^T \by }{2}^2 \\
%         &= \norm{\bX \bV \left(\bS ^{-1} ( \bI - (\bI - \eta \bS)^t ) - (\bS + \lambda \bI )^{-1} \right) \bV^T \bX^T \by  }{2} \\
%         &=  \by^T \bV \bX \left( \underbrace{\bS ^{-1} ( \bI - (\bI - \eta \bS)^t ) - (\bS + \lambda \bI )^{-1}}_{\RN{1}} \right)^2 \bS \bV^T \bX^T \by \label{eq:train_GD_rel}
%         %  (\bX \bV \bS ^{-1} ( \bI - (\bI - \eta \bS)^k ) \bV^T \bX^T \by)^T \bX \bV \bS ^{-1} ( \bI - (\bI - \eta \bS)^k ) \bV^T \bX^T \by
%     \end{align}
%     We now separately consider term 1. Substituting $\lambda = \frac{1}{t \eta}$, we get
%     \begin{align}
%         \bS ^{-1} ( \bI - (\bI - \eta \bS)^t ) - (\bS + \lambda \bI )^{-1} &= \bS^{-1} \left( ( \bI - (\bI - \eta \bS)^t ) - (\bI + \bS^{-1} \lambda )^{-1}\right) \\
%         &= \underbrace{\bS^{-1} \left( ( \bI - (\bI - \eta \bS)^t ) - (\bI + ( \bS t \eta)^{-1}  )^{-1}\right)}_{\bA}
%     \end{align}

%     We now separately bound the diagonal entries in matrix $\bA$. 
%     With $s_i$, we denote $i^{\text{th}}$ diagonal entry of $\bS$. Note that since $ \eta\le 1/\norm{S}{\text{op}}$, for all $i$, $\eta s_i  \le 1$.  Consider $i^{\text{th}}$ diagonal term (which is non-zero) of the diagonal matrix $\bA$, we have 
%     \begin{align}
%         \bA_{ii} = \frac{1}{s_i} \left(  1 - (1 - s_i \eta)^t - \frac{t \eta s_i}{1 + t \eta s_i } \right) &=  \frac{1 - (1 - s_i \eta)^t}{s_i} \left( \underbrace{ 1 - \frac{t \eta s_i}{(1 + t \eta s_i)(1 - (1 - s_i \eta)^t)}}_{\RN{2}} \right) \\ 
%          &\le \frac{1}{2}\left[ \frac{1 - (1 - s_i \eta)^t}{ s_i} \right] \tag*{(Using \lemref{lem:ineq_soln} (b))} \,.
%     \end{align} 
%     Additionally, we can also show the following upper bound on term 2: 
%     \begin{align}
%          1 - \frac{t \eta s_i}{(1 + t \eta s_i)(1 - (1 - s_i \eta)^t)} &= \frac{(1 + t \eta s_i)(1 - (1 - s_i \eta)^t) - t \eta s_i }{(1 + t \eta s_i)(1 - (1 - s_i \eta)^t)} \\
%          & \le  \frac{ 1 -  (1 - s_i \eta)^t - t \eta s_i (1 - s_i \eta)^t}{(1 + t \eta s_i)(1 - (1 - s_i \eta)^t)} \\
%          & \le \frac{1}{t\eta s_i} \,. \tag{Using \lemref{lem:ineq_soln} (a)}
%         %  &\le \frac{1}{2}\left[ \frac{1 - (1 - s_i \eta)^t}{ s_i} \right] \tag*{(Using \lemref{lem:ineq_soln})} \,.
%     \end{align} 

%     Combining both the upper bounds on each diagonal entry $\bA_{ii}$, we have 
%     \begin{align}
%     \bA \preceq c_1(\eta, t) \cdot \bS^{-1} ( \bI - (\bI - \eta \bS)^t ) \,, \label{eq:upperbound_diagonal}
%     \end{align}
%     where $c_1(\eta, t ) = \min(0.5, \frac{1}{t s_i \eta })$. Plugging this into \eqref{eq:train_GD_rel}, we have 
%     \begin{align}
%         \Expt{ x \sim \calS }{\left(f_t(x) - \wt f_\lambda (x)\right)^2} &\le c(\eta, t) \cdot \by^T \bV \bX  \left( \bS^{-1} ( \bI - (\bI - \eta \bS)^t ) \right)^2 \bS \bV^T \bX^T \by \\
%         &=   c(\eta, t) \cdot \by^T \bV \bX  \left( \bS^{-1} ( \bI - (\bI - \eta \bS)^t ) \right) \bS \left( \bS^{-1} ( \bI - (\bI - \eta \bS)^t ) \right) \bV^T \bX^T \by \\
%         & =  c(\eta, t) \cdot \norm{\bX w_t}{2}^2 \\
%         &= c(\eta, t) \cdot  \Expt{ x \sim \calS }{\left(f_t(x) \right)^2} \,,
%     \end{align}
%     where $c(\eta, t ) = \min(0.25, \frac{1}{t^2 s^2_i \eta^2 })$.

%     \textbf{Part 2 {} {}} With $\bSigma$, we denote the underlying true covariance matrix. We now consider the squared difference of output at an unseen point: 
%     \begin{align}
%         \Expt{ x \sim \calD_{\calX} }{\left(f_t(x) - \wt f_\lambda (x)\right)^2} &= \Expt{x \sim \calD_{\calX}}{\norm{x^T w_t - x^T \wt w_\lambda}{2}} \\
%         &=   \norm{x^T \bV \bS ^{-1} ( \bI - (\bI - \eta \bS)^t ) \bV^T \bX^T \by - x^T \bV (\bS + \lambda \bI )^{-1} \bV^T \bX^T \by }{2} \\
%         &= \norm{x^T \bV \left(\bS ^{-1} ( \bI - (\bI - \eta \bS)^t ) - (\bS + \lambda \bI )^{-1} \right) \bV^T \bX^T \by  }{2} \\
%         &= \by^T \bV \bX \left( \bS ^{-1} ( \bI - (\bI - \eta \bS)^t ) - (\bS + \lambda \bI )^{-1} \right) \bV^T \bSigma \bV \\ &\qquad \qquad \qquad \qquad \qquad \left( (\bI - (\bI - \eta \bS)^t ) - (\bS + \lambda \bI )^{-1} \right) \bV^T \bX^T \by \\
%         &\le \sigma_{\text{max}} \cdot \by^T \bV \bX \left( \underbrace{\bS ^{-1} ( \bI - (\bI - \eta \bS)^t ) - (\bS + \lambda \bI )^{-1}}_{\RN{1}} \right)^2 \bV^T \bX^T \by \,, \label{eq:test_GD_rel}
%         %  (\bX \bV \bS ^{-1} ( \bI - (\bI - \eta \bS)^k ) \bV^T \bX^T \by)^T \bX \bV \bS ^{-1} ( \bI - (\bI - \eta \bS)^k ) \bV^T \bX^T \by
%     \end{align}
%     where $\sigma_{\text{max}}$ is the maximum eigenvalue of the underlying covariance matrix $\bSigma$. Using the upper bound on term 1 in \eqref{eq:upperbound_diagonal}, we have 
%     \begin{align}
%         \Expt{ x \sim \calD_{\calX} }{\left(f_t(x) - \wt f_\lambda (x)\right)^2} &\le \sigma_{\text{max}} \cdot c(\eta, t) \cdot \by^T \bV \bX  \left( \bS^{-1} ( \bI - (\bI - \eta \bS)^t ) \right)^2 \bV^T \bX^T \by \\
%         &=   \kappa \cdot c(\eta, t) \cdot \sigma_{\text{min}}\cdot \norm{\bV \left( \bS^{-1} ( \bI - (\bI - \eta \bS)^t ) \right) \bV^T \bX^T \by}{2}^2 \\
%         &\le \kappa \cdot c(\eta, t) \cdot \left[ \bV \left( \bS^{-1} ( \bI - (\bI - \eta \bS)^t ) \right) \bV^T \bX^T \right]^T \bSigma \\
%         &\qquad \qquad \qquad \qquad \qquad \left[ \bV \left( \bS^{-1} ( \bI - (\bI - \eta \bS)^t ) \right) \bV^T \bX^T \right] \by \\
%         & = \kappa \cdot c(\eta, t) \cdot \Expt{x \sim \calD_{\calX}}{\norm{x^T w_t}{2}} \,.
%     \end{align}
% % 
% % 
%     % Since $ \eta\le 1/\norm{S}{\text{op}}$, invoking \lemref{lem:ineq_soln} to upper bound term 1 with
% \end{proof}


% \newpage
% \section{Additional experiments and details}\label{app:exp}
% \newcommand\tab[1][1cm]{\hspace*{#1}}

% \subsection{Datasets} \label{sec:app_dataset}

% \textbf{Toy Dataset {} {}} Assume fixed constants $\mu$ and $\sigma$. For a given label $y$, we simulate features $x$ in our toy classification setup as follows: 
% \begin{align*}
%     x \defeq \texttt{concat} \left[ x_1, x_2\right] \quad \text{where} \quad  x_1 \sim  \calN( y \cdot \mu, \sigma^2 I_{d \times d}) \ \  \text{and} \ \  x_1 \sim  \calN( 0, \sigma^2 I_{d \times d}) \,.
% \end{align*}  
% % where $y$ is the true label and $x$ is the corresponding feature vector. 
% In experiements throughout the paper, we fix dimention $d=100$, $\mu = 1.0 $, and $\sigma = \sqrt{d}$. Intuitively, $x_1$ carries the information about the underlying label and $x_2$ is additional noise independent of the underlying label. 

% \textbf{CV datasets {} {}} We use MNIST~\citep{lecun1998mnist} and CIFAR10~\cite{krizhevsky2009learning}. 
% % For binary tasks, 
% We produce a binary variant from the multiclass classification problem by mapping classes $\{0,1,2,3,4\}$ to label $1$ and $\{ 5,6,7,8,9\}$ to label $-1$. For CIFAR dataset, we also use the standard data augementation of random crop and horizontal flip. PyTorch code is as follows: 

% \texttt{(transforms.RandomCrop(32, padding=4),\\
% \tab transforms.RandomHorizontalFlip())}

% \textbf{NLP dataset {} {}} We use IMDb Sentiment analysis~\citep{maas2011learning} corpus.  

% \subsection{Architecture Details} 

% All experiments were run on NVIDIA GeForce RTX 2080 Ti GPUs. We used PyTorch~\citep{NEURIPS2019a9015} and Keras with Tensorflow~\citep{abadi2016tensorflow} backend for experiments. 
% % , ELMo embeddings~\citep{Peters:2018}, and Hugging Face Transformers~\citep{wolf-etal-2020-transformers}. 

% \textbf{Linear model {} {}} For the toy dataset, we simulate a linear model with scalar output and the same number of parameters as the number of dimensions.   

% \textbf{Wide nets {} {}} To simulate the NTK regime, we experiment with $2-$layered wide nets. The PyTorch code for 2-layer wide MLP is as follows: 


% \texttt{ nn.Sequential( \\
% \tab     nn.Flatten(),\\
% \tab    nn.Linear(input\_dims, 200000, bias=True),\\
% \tab    nn.ReLU(),\\
% \tab    nn.Linear(200000, 1, bias=True)\\
% \tab     )}


% We experiment both (i) with the first layer fixed at random initialization; (ii)  and updating both layers' weights.     

% \textbf{Deep nets for CV tasks {} {}} We consider a 4-layered MLP. The PyTorch code for 4-layer MLP is as follows: 

% \texttt{ nn.Sequential(nn.Flatten(), \\
% \tab        nn.Linear(input\_dim, 5000, bias=True),\\
% \tab        nn.ReLU(),\\
% \tab        nn.Linear(5000, 5000, bias=True),\\
% \tab        nn.ReLU(),\\
% \tab        nn.Linear(5000, 5000, bias=True),\\
% \tab        nn.ReLU(),\\
% % \tab        nn.Linear(5000, 5000, bias=True),\\
% % \tab        nn.ReLU(),\\
% \tab        nn.Linear(1024, num\_label, bias=True)\\
% \tab        )}

% For MNIST, we use $1000$ nodes instead of $5000$ nodes in the hidden layer. 
% % 
% We also experiment with convolutional nets. In particular, we use ResNet18 \citep{he2016deep}. Implementation adapted from:  \url{https://github.com/kuangliu/pytorch-cifar.git}. 

% \textbf{Deep nets for NLP {} {}} We use a simple LSTM model with embeddings intialized with ELMo embeddings~\citep{Peters:2018}. Code adapted from: \url{https://github.com/kamujun/elmo_experiments/blob/master/elmo_experiment/notebooks/elmo_text_classification_on_imdb.ipynb} 

% We also evaluate our bounds with a BERT model. In particular, we fine-tune an off-the-shelf uncased BERT model~\citep{devlin2018bert}. Code adapted from Hugging Face Transformers~\citep{wolf-etal-2020-transformers}: \url{https://huggingface.co/transformers/v3.1.0/custom_datasets.html}. 


% \subsection{Additonal experiments}

% 1. SGD with linear models on cross entropy and MSE loss. 

% 2. CE loss and SGD. GD with MSE loss 

% 3. Binary MNIST with MLP. multiclass MNIST  

% \textbf{Results on CIFAR 10 {} {}} 
% % 
% We plot epoch wise error curve for results in \tabref{table:multiclass}. We observe the same trend as in \figref{fig:error_CIFAR10}. Additionally, we plot an \emph{oracle bound} obtained by tracking the error on mislabeled data which nevertheless were predicted as true label. To obtain an exact emprical value of the oracle bound, we need underlying true labels for the randomly labeled data. 
% % Note that our bound in \thmref{thm:multiclass_ERM}, lower bounds the accuracy as predicted by the oracle bound. 
% While with just access to extra unlabeled data we cannot calculate oracle bound, we note that the oracle bound is very tight and never violated in practice underscoring an importamt aspect of generalization in multiclass problems. This highlight that even a stronger conjecture may hold in multiclass classification, i.e., error on mislabeled data (where nevertheless true label was predicted) lower bounds the population error on the distribution of mislabeled data and hence, the error on (a specific) mislabeled portion predicts the population accuracy on clean data. 
% % 
% On the other hand, the dominating term of in \thmref{thm:multiclass_ERM} is loose when compared with the oracle bound. The main reason, we believe is the pessimistic upper bound in \eqref{eq:lemma1_final_multi_prev} in the proof of \lemref{lem:fit_mislabeled_multi}. We leave an investigation on this gap for future. 
% % of fit 

% % However, oracle bound highlights two . One,  



% \begin{figure}[h]
%     \centering 
%     % \vspace{-15pt}
%     % \includegraphics[width=0.9\linewidth]{example-image-a}
%     \includegraphics[width=0.4\linewidth]{figures/CIFAR10-FNN.pdf} \hfil
%     \includegraphics[width=0.4\linewidth]{figures/CIFAR10-Resnet.pdf}
%     % \includegraphics[width=0.9\linewidth]{figures/{CIFAR10_rn=0.1_lr=0.2_wd=0.005}.png}
%     % \vspace{-10pt}
%     \caption{ Per epoch curves for CIFAR10 corresponding results in \tabref{table:multiclass}. As before, we just plot the dominating term in the RHS of \eqref{eq:multiclass_ERM} as predicted bound. Additionally, we also plot the predicted lower bound by the error on mislabeled data which nevertheless were predicted as true label. We refer to this as ``Oracle bound''. See text for more details. 
%     % 
%     % except for the stopping point. 
%     % The bound predicted by RATT (RHS in \eqref{eq:multiclass_ERM}) is vacuous. 
%     }\label{fig:error_epoch_CIFAR10}
%     % \vspace{-15pt}
% \end{figure}


% \textbf{Results on CIFAR 100 {} {}} 
% % 
% On CIFAR100, our bound in \eqref{eq:multiclass_ERM} yields vacous bounds. However, the oracle bound as explained above yields tight guarantees in the initial phase of the learning (i.e., when learning rate is less than $0.1$). 

% \begin{figure}[h]
%     \centering 
%     % \vspace{-15pt}
%     % \includegraphics[width=0.9\linewidth]{example-image-a}
%     \includegraphics[width=0.4\linewidth]{figures/CIFAR100-Resnet.pdf}
%     % \includegraphics[width=0.9\linewidth]{figures/{CIFAR10_rn=0.1_lr=0.2_wd=0.005}.png}
%     % \vspace{-10pt}
%     \caption{ Predicted lower bound by the error on mislabeled data which nevertheless were predicted as true label with ResNet18 on CIFAR100. We refer to this as ``Oracle bound''. See text for more details. 
%     % 
%     % except for the stopping point. 
%     The bound predicted by RATT (RHS in \eqref{eq:multiclass_ERM}) is vacuous. 
%     }\label{fig:error_CIFAR100}
%     % \vspace{-15pt}
% \end{figure}


% % \paragraph{Experiments on CIFAR100} 



% \subsection{Hyperparameter Details}


% \textbf{\figref{fig:error_CIFAR10} {} {}} We use clean training dataset of size $40,000$. We fix the amount of unlabeled data at $20\%$ of the clean size, i.e. we include additional $8,000$ points with randomly assigned labels. We use test set of $10,000$ points. For both MLP and ResNet, we use SGD with an initial learning rate of $0.1$ and momentum $0.9$. We fix the weight decay parameter at $5\times 10^{-4}$. After $100$ epochs, we decay the learning rate to $0.01$. We use SGD batch size of $100$. 

% \textbf{\figref{fig:error_binary} (a) {} {}} We obtain a toy dataset according to the process described in \secref{sec:app_dataset}. We fix $d=100$ and create a dataset of $50,000$ points with balanced classes. Moreover, we sample additional covariates with the same procedure to create randomly labeled dataset. For both SGD and GD training, we use a fixed learning rate $0.1$.    

% \textbf{\figref{fig:error_binary} (b) {} {}} Similar to binary CIFAR, we use clean training dataset of size $40,000$ and fix the amount of unlabeled data at $20\%$ of the clean dataset size. To train wide nets, we use a fixed learning of $0.001$ with GD and SGD. We decide the weight decay parameter and the early stopping point that maximizes our generalization bound (i.e. without peeking at unseen data ).  We use SGD batch size of $100$. 

% \textbf{\figref{fig:error_binary} (c) {} {}} With IMDb dataset, we use a clean dataset of size $20,000$ and as before, fix the amount of unlabeled data at $20\%$ of the clean data. To train ELMo model, we use Adam optimizer with a fixed learning rate $0.01$ and weight decay $10^{-6}$ to minimize cross entropy loss. We train with batch size $32$ for 3 epochs. To fine-tune BERT model, we use Adam optimizer with learning rate $5\times 10^{-5}$ to minimize cross entropy loss. We train with a batch size of $16$ for 1 epoch.    

% \textbf{\tabref{table:multiclass} {} {}} For multiclass datasets, we train both MLP and ResNet with the same hyperparameters as described before. We sample a clean training dataset of size $40,000$ and fix the amount of unlabeled data at $20\%$ of the clean size. We use SGD with an initial learning rate of $0.1$ and momentum $0.9$. We fix the weight decay parameter at $5\times 10^{-4}$. After $30$ epochs for ResNet and after $50$ epochs for MLP, we decay the learning rate to $0.01$.  We use SGD with batch size $100$. 
% For \figref{fig:error_CIFAR100}, we use the same hyperparameters as 
% CIFAR10 training, except we now decay learning rate after $100$ epochs. 


% In all experiments, to identify the best possible accuracy on just the clean data, we use the exact same set of hyperparamters except the stopping point. We choose a stopping point that maximizes test performance. 

% \subsection{Summary of experiments }

% \begin{center}
%     \begin{table}[H] 
%         \centering
%         \begin{tabular}{|c|c|c|c|} 
%         \hline
%         Classification type & Model category & Model & Dataset  \\ [0.5ex] 
%         \hline
%         \hline
%         \multirow{9}{*}{Binary} & Low dimensional & Linear model & Toy Gaussain dataset  \\
%                         \cline{2-4}
%                          & \multirow{1}{*}{Overparameterized linear nets} 
%                         %  & Linear model & Toy Gaussain dataset \\
%                         %  \cline{3-4}
%                         %  & & 2-layer wide net& Toy Gaussain dataset \\
%                         %  \cline{3-4}
%                          & 2-layer wide net & Binary MNIST \\
%                          \cline{2-4}                 
%                          & \multirow{6}{*}{Deep nets} & \multirow{2}{*}{MLP} & Binary MNIST \\
%                          \cline{4-4}
%                          & &  & Binary CIFAR \\
%                          \cline{3-4}
%                          &  & \multirow{2}{*}{ResNet} & Binary MNIST \\
%                          \cline{4-4}
%                          & &  & Binary CIFAR \\
%                          \cline{3-4}
%                          &  & ELMo-LSTM model & IMDb Sentiment Analysis \\
%                          \cline{3-4}
%                          & & BERT pre-trained model & IMDb Sentiment Analysis \\
%         \hline
%         \multirow{5}{*}{Multiclass} & \multirow{5}{*}{Deep nets} & \multirow{2}{*}{MLP} & MNIST \\
%                         \cline{4-4} 
%                         & & & CIFAR10 \\                   
%                         \cline{3-4}
%                          &   & \multirow{3}{*}{ResNet} & MNIST \\
%                          \cline{4-4}
%                          &   & & CIFAR10 \\
%                          \cline{4-4}
%                          &   & & CIFAR100 \\
%         \hline
%         \end{tabular}
%         % \caption{Summary of experiments performed} \label{table:experiments}
%     \end{table}    
%     % \footnotetext[6]{We use both MSE loss and cross-entropy loss.}
%     % \footnotetext[6]{We try 2 variants: one with a fixed first layer and the other with both layers trainable.}
% \end{center}

% \newpage
% \section{Proof of \lemref{lem:stability_error}} \label{app:proof_lem_error}

% \begin{proof}[Proof of \lemref{lem:stability_error}]
%     Recall, we have a training set $S \cup \wt S_C$. We defined leave-one-out error on mislabeled points as $$\error_{\text{LOO}(\wt S_M) } = \frac{\sum_{(x_i, y_i) \in \wt S_M} \error( f_{(i)}( x_i), y_i)}{ \abs{\wt S_M }} \,, $$
%     where $f_{(i)} \defeq f(\calA, (S \cup \wt S)_{(i)})$. Define $S^\prime \defeq S \cup \wt S$. Assume $(x,y)$ and $(x^\prime,y^\prime)$ as i.i.d. samples from ${\calDm}$. 
%     Using Lemma 25 in \citet{bousquet2002stability}, we have
%     \begin{align*}
%         \Expo{ \left( \error_{\calDm}(\wh f) -\error_{\text{LOO}(\wt S_M) } \right)^2 } \le & \Expt{ S^\prime, (x,y), (x^\prime,y^\prime) }{ \error(\wh f(x), y ) \error(\wh f(x^\prime), y^\prime )} - 2 \Expt{ S^\prime, (x,y) }{ \error(\wh f(x), y ) \error(f_{(i)}(x_i), y_i )} \\
%         & + \frac{m_1-1}{m_1}\Expt{ S^\prime }{  \error(f_{(i)}(x_i), y_i )  \error(f_{(j)}(x_j), y_j )} + \frac{1}{m_1} \Expt{ S^\prime }{  \error(f_{(i)}(x_i), y_i ) } \,. \numberthis \label{eq:main_reln}
%     \end{align*}
%     We can rewrite the equation above as : 
%     \begin{align*}
%         \Expo{ \left( \error_{\calDm}(\wh f) -\error_{\text{LOO}(\wt S_M) } \right)^2 } \le &  \, \underbrace{\Expt{ S^\prime, (x,y), (x^\prime,y^\prime) }{ \error(\wh f(x), y ) \error(\wh f(x^\prime), y^\prime ) - \error(\wh f(x), y ) \error(f_{(i)}(x_i), y_i )}}_{\RN{1}} \\
%         & + \underbrace{\Expt{ S^\prime }{  \error(f_{(i)}(x_i), y_i )  \error(f_{(j)}(x_j), y_j ) -  \error(\wh f(x), y ) \error(f_{(i)}(x_i), y_i )}}_{\RN{2}} \\ &+ \underbrace{\frac{1}{m_1} \Expt{ S^\prime }{  \error(f_{(i)}(x_i), y_i ) - \error(f_{(i)}(x_i), y_i )  \error(f_{(j)}(x_j), y_j ) }}_{\RN{3}} \,. \numberthis \label{eq:main_reln2}
%     \end{align*}
    
%     We will now bound term $\RN{3}$.  Using Schwarz's inequality, we have
    
%     \begin{align}
%         \Expt{ S^\prime }{  \error(f_{(i)}(x_i), y_i ) - \error(f_{(i)}(x_i), y_i )  \error(f_{(j)}(x_j), y_j ) }^2 &\le  \Expt{ S^\prime }{  \error(f_{(i)}(x_i), y_i ) }^2 \Expt{S^\prime}{1 -   \error(f_{(j)}(x_j), y_j ) }^2 \\
%         &\le \frac{1}{4} \label{eq:term1_lem12}
%     \end{align}
    
%     Note that since $(x_i,y_i)$, $(x_j ,y_j )$, $(x,y)$, and $(x^\prime, y^\prime)$ are all from same distribution $\calDm$, we directly incorporate the bounds on term $\RN{1}$ and $\RN{2}$ from proof of Lemma 9 in \citet{bousquet2002stability}. Combining that with \eqref{eq:term1_lem12} and our definition of hypothesis stability in \codref{cond:hypothesis_stability}, we have the required claim. 
    
    
%     % We now re-write term $\RN{1}$ as
%     % \begin{align*}
%     %         &\Expt{S^\prime, (x,y), (x^\prime,y^\prime) }{ \error(\wh f(x), y ) \error(\wh f(x^\prime), y^\prime ) - \error(\wh f(x), y ) \error(f_{(i)}(x_i), y_i )} \\ & \qquad = \Expt{ S^\prime, (x,y), (x^\prime,y^\prime) }{ \error(\wh f(x), y ) \error(\wh f  (x^\prime), y^\prime ) - \error(\wh f ^\prime(x), y ) \error(f_{(i)}(x^\prime), y^\prime )} \tag{Exchanging $(x_i, y_i)$ with $(x^\prime, y^\prime)$ in the second term} \\
%     %         & \qquad = \Expt{ S^\prime, (x,y), (x^\prime,y^\prime) }{  \left(\error(\wh f(x), y )-  \error(f_{(i)}(x), y ) \right) \error(\wh f  (x^\prime), y^\prime )  } \\
%     %         & \qquad  + \Expt{ S^\prime, (x,y), (x^\prime,y^\prime) }{  \left(\error(f_{(i)}(x), y ) -\error(\wh f ^\prime(x), y ) \right) \error(\wh f  (x^\prime), y^\prime )}  \\
%     %         & \qquad +\Expt{ S^\prime, (x,y), (x^\prime,y^\prime) }{  \left( \error(\wh f  (x^\prime), y^\prime ) -  \error(f_{(i)}(x^\prime), y^\prime ) \right) \error(\wh f ^\prime(x), y ) }  \,, \numberthis \label{eq:term1_final}
%     % \end{align*}
%     % where $\wh f^\prime$ is the classifier obtained by training on $ S^\prime_{(i)} \cup \{ (x^\prime, y^\prime) \} $. Similarly we can re-write term $\RN{2}$ as 
%     % \begin{align*}
%     %     & \Expt{ S^\prime }{  \error(f_{(i)}(x_i), y_i )  \error(f_{(j)}(x_j), y_j ) -  \error(\wh f(x), y ) \error(f_{(i)}(x_i), y_i )} \\
%     %     &\quad  = \Expt{ S^\prime, (x,y), (x^\prime,y^\prime)}{  \error(f^{\prime\prime}_{(i)}(x), y )  \error(f_{(j)}^{\prime}(x^\prime), y^\prime ) -  \error(\wh f(x), y ) \error(f_{(i)}(x_i), y_i )} \tag{Exchanging $(x_i, y_i)$ with $(x, y)$ and $(x_j, y_j)$ with $(x^\prime, y^\prime)$ in the first term}\\
%     %     &\quad = \Expt{ S^\prime, (x,y), (x^\prime,y^\prime)}{  \error(f^{\prime\prime}_{(j)}(x), y )  \error(f_{(i)}^{\prime}(x^\prime), y^\prime ) -  \error(\wh f^\prime (x), y ) \error(f^\prime_{(j)}(x^\prime), y^\prime )} \tag{Exchanging $(x_i, y_i)$ and $(x_j, y_j)$ and then replacing $(x_j, y_j)$ with $(x^\prime, y^\prime)$ in the second term} \\
%     %     & \quad = \Expt{ S^\prime, (x,y), (x^\prime,y^\prime) }{  \left( \error(f_{(i)}^{\prime}(x^\prime), y^\prime )   -  \error(\wh f^{\prime\prime}  (x^\prime), y^\prime ) \right)  \error(f^{\prime\prime}_{(j)}(x), y )   } \\
%     %     & \quad  + \Expt{ S^\prime, (x,y), (x^\prime,y^\prime) }{  \left( \error(f^{\prime\prime}_{(j)}(x), y )  -\error(\wh f ^\prime(x), y ) \right) \error(\wh f^{\prime\prime}  (x^\prime), y^\prime )  }  \\
%     %     & \quad+ \Expt{ S^\prime, (x,y), (x^\prime,y^\prime) }{  \left( \error(\wh f^{\prime\prime}  (x^\prime), y^\prime )  -  \error(f^\prime_{(j)}(x^\prime), y^\prime ) \right)  \error(\wh f^\prime (x), y ) }   \\
%     %     & \quad = \Expt{ S^\prime, (x,y), (x^\prime,y^\prime) }{  \left( \error(f_{(i)}^{\prime}(x^\prime), y^\prime )   -  \error(\wh f (x^\prime), y^\prime ) \right)  \error(f_{(i)}(x_j), y_j )   } \\
%     %     & \quad  + \Expt{ S^\prime, (x,y), (x^\prime,y^\prime) }{  \left( \error(f^{\prime\prime}_{(j)}(x), y )  -\error(\wh f (x), y ) \right) \error(\wh f^{\prime\prime}  (x_j), y_j )  }  \\
%     %     & \quad+ \Expt{ S^\prime, (x,y), (x^\prime,y^\prime) }{  \left( \error(\wh f^{\prime\prime}  (x^\prime), y^\prime )  -  \error(f^\prime_{(j)}(x^\prime), y^\prime ) \right)  \error(\wh f^\prime (x^\prime), y^\prime ) }  \,, \numberthis \label{eq:term2_final}
%     % \end{align*}
%     % where $f^{\prime\prime}_{(j)}$ is trained on $S^\prime_{(j,i)} \cup {(x,y)}$, $f^{\prime}_{(i)}$ is trained on $S^\prime_{(j,i)} \cup {(x^\prime,y^\prime)}$, and $\wh f^{\prime\prime} $ is trained on $S^\prime_{(j)} \cup {(x,y)}$. Note in the last line we replaced $(x,y)$ by $(x_j, y_j)$ in the first term, replaced $(x^\prime,y^\prime)$ by $(x_j, y_j)$ in the second term and exchanged $(x_i,y_i)$ with $(x_j,y_j)$ and also $(x,y)$ and $(x^\prime, y^\prime)$
    
    
% \end{proof}



\newpage


\appendix
\renewcommand{\appendixpagename}{\centering \LARGE Appendix}
\appendixpage

\startcontents[section]
\printcontents[section]{l}{1}{\setcounter{tocdepth}{2}}



\newcommand{\proofstep}[2]{{\large \textbf{Step #1:  #2}}\\}

\section{Proofs for the single layer case}
\label{sec:single_proofs}

In this section, we prove our characterization of global minima for the single layer case (\autoref{thm:main_single}).
We begin by simplifying the loss into a more concrete form.
Throughout the proof, we will  write $P,Q$ instead of $P_0,Q_0$ for brevity.


\subsection{Rewriting the loss function}
\label{sec:rewrite loss}



Recall the in-context loss \eqref{def:ICL linear} for the single layer case $f(P,Q)$  is defined as:
\begin{align} 
f\left(P,Q\right)  
&=\E_{Z_0,\wstar} \left[ \left(Z_{0} +\frac{1}{n} \att_{P,Q}(Z_0) \right)_{(d+1),(n+1)} + \wstar^\top \tx{n+1}\right]^2
\end{align} 
From the definition of attention given in \eqref{eq:linear},
one can further spell out the expression $Z_0 +\frac{1}{n} \att_{P,Q}(Z_0)$ using the notation $Z_0 = [\tz{1} \ \tz{2} \ \cdots \ \tz{n+1}]$ as follows:
\begin{align}
&[\tz{1} \ \cdots \ \tz{n+1}] + \frac{1}{n}  P [\tz{1} \  \cdots \ \tz{n+1}]\aa \left([\tz{1} \ \cdots \ \tz{n+1}]^\top Q [\tz{1} \ \cdots \ 
\tz{n+1}] \right)\\
\quad\quad &= [\tz{1} \ \cdots \ \tz{n+1}] + \frac{1}{n}  P  \left(\sum_{i=1}^n \tz{i} {\tz{i}}^\top\right) Q  [\tz{1} \ \cdots \ 
\tz{n+1}]\,.
\end{align}
Thus, the last column of the above matrix can be expressed as
\begin{align} 
\begin{bmatrix}
\tx{n+1}\\ 0
\end{bmatrix}  + \frac{1}{n}  P\left(\sum_{i=1}^n \tz{i} {\tz{i}}^\top \right) Q  
\begin{bmatrix}
\tx{n+1}\\ 0
\end{bmatrix}\,, 
\end{align}
where note that the summation is for $i=1,2,\dots, n$ due to the mask matrix $\aa$.
Therefore, letting $\cc^\top $ be the last row of $P$, and $\bb\in \R^{d+1,d}$ be the first $d$ columns of $Q$ (as we did in \eqref{exp:simplify}), then  $\left[Z_{0} +\frac{1}{n} \att_{P,Q}(Z_0) \right]_{(d+1),(n+1)}$ can be written as
\begin{align} \label{exp:attention_single}
\frac{1}{n}  \cc^\top \left( \sum_{i=1}^n \tz{i} {\tz{i}}^\top \right) \bb 
\begin{bmatrix}
\tx{n+1}\\ 0
\end{bmatrix}\,,
\end{align}
in other words, $f(P,Q)$ only depends on the parameter $\cc$ and $\bb$.
Henceforth, we will write $f(P,Q)$ as  $f(\cc,\bb)$. Let us summarize our conclusion so far since it's crucial for the analysis to follow.
\begin{mdframed}[linecolor=black!40,linewidth=0.5pt,innertopmargin=3pt,innerleftmargin=3pt,innerrightmargin=3pt,innerbottommargin=3pt]
\textbf{Conclusion so far:} A careful inspection reveals that the in-context loss  only depends on the last row of $P$ and the first $d$ columns of $Q$. 
Thus, consider the following parametrization
\begin{align} \label{exp:simplify}
P = \begin{bmatrix}
0\\
\cc^\top
\end{bmatrix}\quad \text{and} \quad Q=\begin{bmatrix}
\bb & 0 
\end{bmatrix}\,, \text{ where $\cc\in\R^{d+1}$ and $\bb\in \R^{(d+1)\times d}$}.
\end{align}
Now with this parametrization, the in-context loss can be written as $f(\cc,\bb) \coloneqq  f([0 \,\,\ \cc]^\top, [\bb \,\, 0])$.  
\end{mdframed}

Now, let us spell out $f(\cc,\bb)$ based on \eqref{exp:attention_single} as follows:
\begin{align}
f(\cc,\bb) &= \E_{Z_0,\wstar} \left[\cc^\top \underbrace{\frac{1}{n}  \sum_i \tz{i} {\tz{i}}^\top}_{=:\MM} \bb \tx{n+1} + \wstar^\top \tx{n+1} \right]^2 \\
& =: \E_{Z_0,\wstar} \left[\cc^\top \MM \bb \tx{n+1} + \wstar^\top \tx{n+1} \right]^2 = \E_{Z_0,\wstar} \left[(\cc^\top \MM \bb   + \wstar^\top) \tx{n+1} \right]^2 \,, \label{exp:loss_single}
\end{align}
where we used the notation $\MM \coloneqq \frac{1}{n}  \sum_i \tz{i} {\tz{i}}^\top$ to simplify.
We now analyze the global minima of this loss function.
To illustrate the proof idea clearly, we begin with the proof for the simpler case of isotropic data.

\subsection{Warm-up: proof for the isotropic data} \label{e:single_layer_non_isotropic_proof}


As a warm-up, we first prove the result for the special case where  $\tx{i}$ is sampled from $\mathcal{N}(0, I_d)$.  



\underline{\emphh{1. Decomposing the loss function into components.}}
Writing $\bb = [\bbb_1\,\, \bbb_1 \,\, \cdots \,\, \bbb_d ]$, and use the fact that $\E[\tx{n+1}[j]\tx{n+1}[j']] =0$ for $j\neq j'$ and $\E[\tx{n+1}[j]^2] =1$, we get
\begin{align}
f\left(\cc, \bb\right) = \sum_{j=1}^d \E_{Z_0,\wstar}\left[ \cc^\top \MM \,\bbb_j + \wstar[j] \right]^2 \E[\tx{n+1}[j]^2]  = \sum_{j=1}^d \E_{Z_0,\wstar}\left[ \cc^\top \MM \,\bbb_j  + \wstar[j] \right]^2 \,. 
\end{align}
The key idea is to characterize the global minima of each component in the summation separately.
Another key idea is to reparametrize the cost function  given the following identity:
\begin{align}
\E_{Z_0,\wstar}\left[ \cc^\top \MM \,\bbb_j  + \wstar[j] \right]^2 = \E_{Z_0,\wstar}\left[ \tr (\MM \,\bbb_j \cc^\top)  + \wstar[j] \right]^2= \E_{Z_0,\wstar}\left[\inp{\MM}{\cc \bbb_j^\top}  + \wstar[j] \right]^2\,,
\end{align}
where we use the notation $\inp{X}{Y} \coloneqq \tr(XY^\top)$ for two matrices $X$ and $Y$ here and below.
Given the above identity, we define each component in the summation as follows.
\begin{align}
\boxed{f_j(X)\coloneqq   \E_{Z_0,\wstar}\Big[\inp{\MM}{X}  + \wstar[j] \Big]^2\quad \text{for }  X\in \R^{(d+1)\times (d+1)}\,.}
\end{align} 



\underline{\emphh{2. Characterizing global minima of each component.}} 
To characterize the global minima of each objective, we prove the following result.

% \begin{mdframed}[linecolor=black!40,linewidth=0.5pt,innertopmargin=3pt,innerleftmargin=3pt,innerrightmargin=3pt,innerbottommargin=3pt]
\begin{lemma}[\textbf{Global minima of each component}] \label{lem:component opt}
Suppose that $\tx{i}$ is sampled from $\mathcal{N}(0, I_d)$ and  $\wstar$ is sampled  from $\mathcal{N}(0, I_d)$.      Consider the following objective (here, $\inp{X}{Y} \coloneqq \tr(XY^\top)$ for two matrices $X$ and $Y$)
\begin{align}
f_j(X) =  \E_{Z_0,\wstar}\left[ \inp{\MM}{X}  + \wstar[j] \right]^2\,.
\end{align} 
Then a global minimum is given as 
\begin{align} \label{exp:opt_single}
X_j = - \frac{1}{\left( \frac{n-1}{n} +  (d+2) \frac{1}{n} \right)}E_{d+1,j}\,,
\end{align} where $E_{i_1,i_2}$ is the matrix whose $(i_1,i_2)$-th entry is $1$,  and the other entries are zero.
\end{lemma}
% \end{mdframed}
\begin{proof}[{\bf Proof of \autoref{lem:component opt}}] 
Note first that $f_j$ is convex in $X$.
Hence, in order to show that matrix $X_j$ is  the global optimum of $f_j$, it suffices to show that the gradient vanishes at that point, in other words, $\nabla f_j(X_j)  = 0$. 
To verify this,  let us compute the gradient of $f_j$: for a matrix $X$, 
\begin{align}
\nabla f_j(X)  = 2\E \left[ \inp{\MM}{X} \MM \right] + 2 \E \left[\wstar[j] \ 
\MM \right]\,,
\end{align}
where we recall that $\MM$ is defined as  
\begin{align}
\MM = \frac{1}{n}\sum_i\begin{bmatrix}
\tx{i} {\tx{i}}^\top & \ty{i} \tx{i}\\
\ty{i} {\tx{i}}^\top & {\ty{i}}^2
\end{bmatrix}.
\end{align}
To verify that the gradient is equal to zero, let us first compute $\E \left[ \wstar[j] \,
\MM \right]$.  
For each $i=1,\dots,n$, note that $\E[\wstar[j] \ \tx{i} {\tx{i}}^\top] =O$ because $\E[\wstar]=0$.
Moreover, $\E[\wstar[j] \ (\ty{i})^2] =0$ because $\wstar$ is symmetric, i.e., $\wstar \overset{d}{=} -\wstar$, and $\ty{i} =\langle{\wstar},{\tx{i}} \rangle$. 
Lastly, for $k=1,2,\dots, d$, we have 
\begin{align} \label{exp:wj}
\E[\wstar[j]  \ {\ty{i}} \ {\tx{i}}[k]] =\E[\wstar[j] \ \langle{\wstar},{\tx{i}} \rangle \ {\tx{i}}[k]] =   \E\left[ \wstar[j]^2 \  {\tx{i}}[j] \  {\tx{i}}[k] \right]  = \mathbbm{1}_{[j=k]}     
\end{align}
because $\E[\wstar[i] \ \wstar[j]]=0$ for $i\neq j$. Combining the above calculations, it follows that
\begin{align} \label{exp:second}
\boxed{ \E \left[ \wstar[j] \ 
\MM \right]  =  E_{d+1,j} + E_{j,d+1}\,.} 
\end{align}

In order to compute $\E \left[ \inp{\MM}{X} \MM \right]$, let us compute $\E \left[ \inp{\MM}{E_{i,i'}} \MM \right]$ for $i, i' = 1,\dots, d+1$.
Without loss of generality, $i\geq i'$.
First of all 
We now compute   compute $\E \left[ \inp{\MM}{E_{d+1,j}} \MM \right]$.
Note first that 
\begin{align}
\inp{\MM}{E_{d+1,j}} = \sum_i\langle{\wstar}, \tx{i} \rangle \ {\tx{i}}[j]\,.
\end{align}
Hence, it holds that
\begin{align}
\E\left[  \inp{\MM}{E_{d+1,j}} \left(\sum_i {\tx{i}}{\tx{i}}^\top\right)\right]
= \E\left[\left(\sum_i\langle{\wstar}, \tx{i} \rangle  \ {\tx{i}}[j]\right) \left(\sum_i {\tx{i}}{\tx{i}}^\top\right)\right] =O\,.    
\end{align}
because $\E[\wstar]=0$.
Next, we have 
\begin{align}
\E \left[  \inp{\MM}{E_{d+1,j}}\left(\sum_i{\ty{i}}^2\right) \right]    =  \E\left[ \left(\sum_i \langle{\wstar}, \tx{i} \rangle  \ {\tx{i}}[j]\right) \left(\sum_i{\ty{i}}^2\right)\right] =0
\end{align} because $\wstar \overset{d}{=} -\wstar$.
Lastly, we compute 
\begin{align}
\E\left[  \inp{\MM}{E_{d+1,j}} \left(\sum_i {\ty{i}}{\tx{i}}^\top\right)\right]\,.
\end{align}
To that end, note that for $j\neq j'$,
\begin{align}
\E\left[\langle{\wstar}, \tx{i}  \rangle   \ {\tx{i}}[j] \ \langle{\wstar},{\tx{i'}}\rangle \ \tx{i'} [j'] \right] = \begin{cases}
\E[\langle \tx{i},\tx{i'} \rangle \ {\tx{i}}[j] \  \tx{i'}[j']]= 0 &\text{if }i\neq i',\\
\E[\|\tx{i}\|^2 \ {\tx{i}}[j] \  {\tx{i}}[j']]= 0 &\text{if }i=i',
\end{cases}
\end{align}
and
\begin{align} \label{exp:cross}    \E\left[\langle{\wstar}, \tx{i} \rangle \ {\tx{i}}[j] 
\ \langle \wstar,\tx{i'}\rangle \ \tx{i'} [j] \right] = \begin{cases}
\E[{(\tx{i}}[j])^2 \ (\tx{i'}[j])^2]= 1&\text{if }i\neq i',\\
\E\left[\langle{\wstar}, \tx{i}  \rangle^2  \ ({\tx{i}}[j])^2\right]= d+2&\text{if }i=i',
\end{cases}
\end{align}
where the last case follows from the fact that the fourth moment of Gaussian is $3$ and
\begin{align}
\E\left[\langle{\wstar}, \tx{i} \rangle^2 \ ({\tx{i}}[j])^2\right] = \E\left[ \|{\tx{i}}\|^2 \  ({\tx{i}}[j])^2 \right] = 3 +d-1 = d+2.
\end{align} 
Combining the above calculations together, we arrive at
\begin{align}
\E \left[ \inp{\MM}{E_{d+1,j}}  \MM \right]  &=   \frac{1}{n^2} \cdot \left( n(n-1) +  (d+2) n \right)  (E_{d+1,j} + E_{j,d+1})\\
&=    \left( \frac{n-1}{n} +  (d+2) \frac{1}{n} \right)    (E_{d+1,j} + E_{j,d+1})\,. \label{exp:first}
\end{align}
Therefore, combining \eqref{exp:second} and \eqref{exp:first}, the results follows. 
\end{proof}




\underline{\emphh{3. Combining global minima of each component.}} 
From \autoref{lem:component opt}, it follows that 
\begin{align}  
X_j = - \frac{1}{\left( \frac{n-1}{n} +  (d+2) \frac{1}{n} \right)}E_{d+1,j}\,,
\end{align} 
is the unique global minimum of $f_j$.
Hence, $\cc$ and $\bb = [\bbb_1\ \bbb_1 \ \cdots \ \bbb_d ]$ achieve the global minimum of $f(\cc,\bb)=\sum_{j=1}^d f_j(\cc \bbb_j^\top)$ if they satisfy
\begin{align}
\cc\bbb_j^\top =  - \frac{1}{\left( \frac{n-1}{n} +  (d+2) \frac{1}{n} \right)}E_{d+1,j} \quad \text{for all }i=1,2,\dots,d. 
\end{align}
This can be achieve by the following choice:
\begin{align}  
\cc^\top = \e_{d+1}  ,\quad \bbb_j =     -  \frac{1}{\left( \frac{n-1}{n} +  (d+2) \frac{1}{n} \right)} \e_j \quad \text{for }i=1,2,\dots,d\,,
\end{align} 
where $\e_j$ is the $j$-th coordinate vector. This choice precisely corresponds to 
\begin{align}
\cc = \e_{d+1},\quad \bb = - \frac{1}{\left( \frac{n-1}{n} +  (d+2) \frac{1}{n} \right)}  \begin{bmatrix}I_d  \\ 0 \end{bmatrix}\,.
\end{align}

We next move on to the non-isotropic case.


\subsection{Proof for the non-isotropic case}

\underline{\emphh{1. Diagonal covariance case.}}
We first  consider the case where  $\tx{i}$ is sampled from $\mathcal{N}(0, \Lambda)$ where $\Lambda = \mathrm{diag}(\lambda_1,\dots, \lambda_d)$ and  $\wstar$ is sampled  from $\mathcal{N}(0, I_d)$.  
We prove the following generalization of \autoref{lem:component opt}.
 
\begin{lemma} \label{lem:component opt_noniso}
Suppose that $\tx{i}$ is sampled from $\mathcal{N}(0, \Lambda)$ where $\Lambda = \mathrm{diag}(\lambda_1,\dots, \lambda_d)$ and  $\wstar$ is sampled  from $\mathcal{N}(0, I_d)$. 
Consider the following objective
\begin{align}
f_j(X) =  \E_{Z_0,\wstar}\left[ \inp{\MM}{X}  + \wstar[j] \right]^2\,.
\end{align} 
Then a global minimum is given as 
\begin{align} \label{exp:opt_single_non}
X_j = - \frac{1}{  \frac{n+1}{n}  \lambda_j +   \frac{1}{n} \cdot \left( \sum_k \lambda_k  \right)   } E_{d+1,j}\,,
\end{align} where $E_{i_1,i_2}$ is the matrix whose $(i_1,i_2)$-th entry is $1$,  and the other entries are zero.
\end{lemma}
\begin{proof}[{\bf Proof of \autoref{lem:component opt_noniso}}] 
Similarly to the proof of \autoref{lem:component opt}, it suffices to check that  
\begin{align}
2\E \left[ \inp{\MM}{X_0} \MM \right] + 2 \E \left[\wstar[j] \ 
\MM \right]=0\,,
\end{align}
where we recall that $\MM$ is defined as  
\begin{align}
\MM = \frac{1}{n}\sum_i\begin{bmatrix}
\tx{i} {\tx{i}}^\top & \ty{i} \tx{i}\\
\ty{i} {\tx{i}}^\top & {\ty{i}}^2
\end{bmatrix}\,.
\end{align}
A similar calculation as the proof of \autoref{lem:component opt} yields  
\begin{align} \label{exp:second_non}
\E\left[\wstar[j] \ \MM\right] &=  \lambda_j (E_{d+1,j} + E_{j,d+1}).
\end{align}
Here the factor of $\lambda_j$ comes from the following generalization of \eqref{exp:wj}:
\begin{align}
\E[\wstar[j] \ {\ty{i}} \ {\tx{i}}[k]] =\E[\wstar[j]  \ \langle{\wstar}, \tx{i} \rangle  \ {\tx{i}}[k]] =   \E\left[ \wstar[j]^2 \ {\tx{i}}[j] \ {\tx{i}}[k] \right]  = \lambda_j \mathbbm{1}_{[j=k]}\,.
\end{align}

Next, we compute $\E \left[ \inp{\MM}{E_{d+1,j}} \MM \right]$.
Again, we follow a similar calculation to the proof of \autoref{lem:component opt} except that this time we use the following generalization of 
\eqref{exp:cross}:
\begin{align}
\E\left[\langle{\wstar}, \tx{i} \rangle \ {\tx{i}}[j] \ \langle \wstar,{\tx{i'}} \rangle  \ \tx{i'} [j] \right] = \begin{cases}
\E[{\tx{i}}[j]^2 \ \tx{i'}[j]^2]= \lambda_j^2 &\text{if }i\neq i',\\
\E\left[\langle{\wstar}, \tx{i} \rangle^2 \ {\tx{i}}[j]^2\right]= \lambda_j \sum_k \lambda_k + 2 \lambda_j^2&\text{if }i=i',
\end{cases}
\end{align}
where the last line follows since
\begin{align}
\E\left[\langle{\wstar}, \tx{i} \rangle^2 \ {\tx{i}}[j]^2\right] &= \E\left[ \|{\tx{i}}\|^2 \  {\tx{i}}[j]^2 \right] =\E\left[ {\tx{i}}[j]^2 \ \sum_k {\tx{i}}[k]^2 \right]  = \lambda_j \sum_k \lambda_k + 2 \lambda_j^2\,.
\end{align} 
Therefore, we have
\begin{align}
\E \left[ \inp{\MM}{E_{d+1,j}}  \MM \right]  &=   \frac{1}{n^2} \cdot \left( n(n-1) \lambda_j^2 +   n  \lambda_j \sum_k \lambda_k + 2 n\lambda_j^2\right)  (E_{d+1,j} + E_{j,d+1})\\
&=    \left( \frac{n+1}{n} \lambda_j^2 +    \frac{1}{n} (\lambda_j \sum_k \lambda_k  )  \right)    (E_{d+1,j} + E_{j,d+1})\,. \label{exp:first_non}
\end{align}
Therefore, combining \eqref{exp:second_non} and \eqref{exp:first_non},  
the results follows. 
\end{proof}

Now we finish the proof. 
From \autoref{lem:component opt}, it follows that 
\begin{align}  
X_j = - \frac{1}{  \frac{n+1}{n}  \lambda_j +   \frac{1}{n} \cdot \left( \sum_k \lambda_k  \right)   } E_{d+1,j}
\end{align} 
is the unique global minimum of $f_j$.
Hence, $\cc$ and $\bb = [\bbb_1\ \bbb_1 \ \cdots \ \bbb_d ]$ achieve the global minimum of $f(\cc,\bb)=\sum_{j=1}^d f_j(\cc,\bb_j)$ if they satisfy
\begin{align}
\cc\bbb_j^\top =   X_j = - \frac{1}{  \frac{n+1}{n}  \lambda_j +   \frac{1}{n} \cdot \left( \sum_k \lambda_k  \right)   } E_{d+1,j}\quad \text{for all }i=1,2,\dots,d. 
\end{align}
This can be achieve by the following choice:
\begin{align}  
\cc^\top = \e_{d+1}  ,\quad \bbb_j =     -  \frac{1}{  \frac{n+1}{n}  \lambda_j +   \frac{1}{n} \cdot \left( \sum_k \lambda_k  \right)   } \e_j \quad \text{for }i=1,2,\dots,d\,,
\end{align} 
where $\e_j$ is the $j$-th coordinate vector. This choice precisely corresponds to 
\begin{align}
\cc = \e_{d+1},\quad \bb =  - \begin{bmatrix}
\mathrm{diag}\left(\left\{\frac{1}{  \frac{n+1}{n}  \lambda_j +   \frac{1}{n} \cdot \left( \sum_k \lambda_k  \right)   } \right\}_j\right)   \\
0  
\end{bmatrix} .
\end{align}


\underline{\emphh{2. Non-diagonal covariance case (the setting of \autoref{thm:main_single}).}}
We finally prove the general result of \autoref{thm:main_single}, namely  $\tx{i}$ is sampled from a Gaussian with covariance $\Sigma = U\Lambda U^\top$ where $\Lambda = \mathrm{diag}(\lambda_1,\dots, \lambda_d)$ and $\wstar$ is sampled  from $\mathcal{N}(0, I_d)$.  
The proof works by reducing this case to the previous case.
For each $i$, define $\ttx{i} :=  U^T\tx{i}$. Then $\E[\ttx{i}(\ttx{i})^\top ] = \E[U^\top (U\Lambda U^\top) U] = \Lambda$.
Now let us write the loss function \eqref{exp:loss_single} with this new coordinate system: since $\tx{i} = U \ttx{i}$, we have  
\begin{align}
f(\cc,\bb) &=  \E_{Z_0,\wstar} \left[(\cc^\top \MM \bb   + \wstar^\top) U\ttx{n+1} \right]^2 =\sum_{j=1}^d   \lambda_j \E_{Z_0,\wstar} \left[ \left((\cc^\top \MM \bb   + \wstar^\top) U\right) [j]  \right]^2 \,.
\end{align}
Hence, let us consider 
the vector $(\cc^\top \MM \bb   + \wstar^\top) U$. By definition of $\MM$, we have
\begin{align}
(\cc^\top \MM \bb   + \wstar^\top) U &= \frac{1}{n}\sum_i  \cc^\top \begin{bmatrix} \tx{i} \\
\inp{\tx{i}}{\wstar} \end{bmatrix}^{\otimes 2}  \bb U + \wstar^\top U\\
&=  \frac{1}{n}\sum_i  \cc^\top \begin{bmatrix} U\tilde{x}_i \\
\inp{U\tx{i}}{\wstar} \end{bmatrix}^{\otimes 2} \bb U+ \wstar^\top U\\
&=  \frac{1}{n}\sum_i  \cc^\top \begin{bmatrix}U & 0 \\0 & 1\end{bmatrix}\begin{bmatrix} \tilde{x}_i \\
\inp{U\tx{i}}{\wstar} \end{bmatrix}^{\otimes 2} \begin{bmatrix}U^\top & 0 \\0 & 1\end{bmatrix}\bb U+ \wstar^\top U \\
&= \frac{1}{n}\sum_i  \ttcc^\top  \begin{bmatrix} \tilde{x}_i \\
\inp{\tx{i}}{ \twstar} \end{bmatrix}^{\otimes 2} \ttbb + \twstar^\top 
\end{align}
where we define $\ttcc^\top  \coloneqq \cc^\top \begin{bmatrix}U & 0 \\0 & 1\end{bmatrix}$, $\ttbb\coloneqq \begin{bmatrix}U^\top & 0 \\0 & 1\end{bmatrix}\bb U$, and $\twstar \coloneqq U^\top \wstar$.
By the rotational symmetry, $\twstar$ is also distributed as $\mathcal{N}(0, I_d)$.
Hence, this reduces to the previous case, and a global minimum is given as  
\begin{align}
\ttcc = \e_{d+1},\quad \ttbb =  - \begin{bmatrix}
\mathrm{diag}\left(\left\{\frac{1}{  \frac{n+1}{n}  \lambda_j +   \frac{1}{n} \cdot \left( \sum_k \lambda_k  \right)   } \right\}_j\right)   \\
0  
\end{bmatrix} .
\end{align}
From the definition of $\ttcc,\ttbb$, it thus follows that a global minimum is given by
\begin{align}      \cc^\top = \e_{d+1} ,\quad \bb =  - \begin{bmatrix}
U \mathrm{diag}\left(\left\{\frac{1}{  \frac{n+1}{n}  \lambda_i +   \frac{1}{n} \cdot \left( \sum_k \lambda_k  \right)   } \right\}_i\right) U^\top \\
0  
\end{bmatrix}\,,
\end{align}
as desired.
 
 

\subsection{Proof for non-linear attentions (\autoref{thm:nonlinear})}
\label{pf:nonlinear}

As mentioned in  \autoref{thm:nonlinear}, we focus on the setting where the last row of $Q$ is zero, i.e., let
\begin{align}
Q = \begin{bmatrix} A & a  \\
0^\top & 0 \end{bmatrix}\quad \text{for }A\in \R^{d\times d}~\text{and}~a \in \R^d.
\end{align}
We first rewrite the loss function and simplify it 
following \autoref{sec:rewrite loss}.
Moreover, for simple notation we will often write $z,x$ instead of  $\tz{n+1},\tx{n+1}$.

\underline{\emphh{1. Rewriting loss function.}}


Following \autoref{sec:rewrite loss}, let us write down the in-context loss  \eqref{def:ICL linear}.  for the single-layer nonlinear attention denoted by $f(P,Q)$:  
\begin{align} 
f\left(P,Q\right)  
&=\E_{Z_0,\wstar} \left( \left[Z_{0} +\frac{1}{n} \atth_{P,Q}(Z_0) \right]_{d+1,n+1} + \wstar^\top x\right)^2
\end{align} 
Recalling the definition of the ReLU attention $\atth_{P,Q}(Z) \coloneqq  P ZM \  \relu (
Z^\top Q Z)$, the data matrix $Z \coloneqq [\tz{1} \ \cdots \ \tz{n+1}]$, and the mask matrix $\aa$, the term $ \atth_{P,Q}(Z_0)$ can be written as:
\begin{align}
\atth_{P,Q}(Z_0) = \underbrace{PZ_0}_{\R^{(d+1)\times (n+1)} } \cdot  \begin{bmatrix} I_{n\times n} & 0 \\0 & 0 \end{bmatrix}  \cdot   \underbrace{\relu\left(Z_0^\top Q Z_0 \right)\,.}_{\R^{(n+1)\times (n+1)}}
\end{align}
Hence, it follows that the $(d+1, n+1)$-th entry of $\atth_{P,Q}(Z_0)$ is equal to the product of the $(d+1)$-th row of $PZ_0$, the mask matrix $\aa$, and the $(n+1)$-th column of $\relu\left(Z_0^\top Q Z_0 \right)$. Hence, let us write them down explicitly:
\begin{list}{$\bullet$}{\leftmargin=1.5em}
\setlength{\itemsep}{1pt}
\item Letting $\cc^\top$ be the last row of the matrix $P$, it holds that the $(d+1)$-th row of $PZ_0$ is equal to $[\inpp{\cc}{\tz{i}}]_{i=1,\dots,n+1}$.
\item The $(n+1)$-th column of $\relu\left(Z_0^\top Q Z_0 \right)$ is equal to $\left[\relu\left((\tz{i})^\top Q \tz{n+1} \right)  \right]_{i=1,\dots, n+1}$. Letting  $\bb$ the first $d$ columns of $Q$,  this vector is equal to $\left[\relu\left((\tx{i})^\top \bb \tx{n+1} \right) \right]_{i=1,\dots, n+1}$ because the last row of $Q$ is zero and the last row of $\tz{n+1}$ is zero (since $(\tz{n+1})^\top = [(\tx{n+1})^\top \ 0]$). 
\end{list}
Thus, the product of   $[\inpp{\cc}{\tz{i}}]_{i=1,\dots,n+1}$, the mask matrix $\aa$, and $\left[\relu\left((\tx{i})^\top \bb \tx{n+1} \right) \right]_{i=1,\dots, n+1}$ results in the following expression of the attention (writing $z,x$ instead of  $\tz{n+1},\tx{n+1}$):
\begin{align}
    \left[ \atth_{P,Q}(Z_0) \right]_{d+1,n+1}    =  \sum_{i=1}^n \left[ \inpp{\cc}{\tz{i}} \cdot \relu((\tx{i})^\top \bb x ) \right] \,.
\end{align} 
Since $[Z_0]_{d+1,n+1}=0$, we therefore have 
\begin{align}
    \left[Z_{0} +\frac{1}{n} \atth_{P,Q}(Z_0) \right]_{d+1,n+1} = \frac{1}{n}\sum_{i=1}^n \left[ \inpp{\cc}{\tz{i}}\cdot \relu((\tx{i})^\top \bb x ) \right]\,.
\end{align}
Therefore, it follows that the in-context loss $f(P,Q)$ only depends on $\cc$ and $\bb$.
Henceforth, let us write $f(\cc,\bb)$ instead of $f(P,Q)$ following \autoref{sec:rewrite loss}. 
In particular, writing $\cc^\top = [\cc_0^\top ,\cc_1]$ for $\cc_0\in \R^d$ and $\cc_1\in \R$, the loss function can be expressed as 
\begin{align} \label{exp:nonlinear_loss}
    f(\cc,\bb) \coloneqq \E \left( \frac{1}{n}\sum_{i=1}^n \left[(\inpp{\cc_0}{\tx{i}}+ \cc_1 \ty{i})\cdot \relu((\tx{i})^\top \bb x ) \right] + \inpp{\wstar}{x}  \right)^2\,.
\end{align}






 \underline{\emphh{2. Simplifying the loss function with symmetry.}}

 
Now, we use the fact that both $\tx{i}$'s and $\wstar$ are sampled from the isotropic Gaussian, i.e., $\mathcal{N}(0, I_{d})$ in order to further simplify the loss function in \eqref{exp:nonlinear_loss}.
In particular, we use the following facts:
\begin{list}{$\bullet$}{\leftmargin=1.5em}
\setlength{\itemsep}{1pt}
    \item[($a$)] For orthonormal matrices $U,V \in \R^{d\times d}$, it holds that $U\tx{i}$, $Vx$ and $U\wstar$ have the same distributions as $\mathcal{N}(0, I_{d})$. 
    \item[($b$)] Moreover, for a diagonal matrix $\Xi = \diag(\xi_i) \in \R^{d\times d}$ with the diagonal entries being  random signs $\xi_i \sim \{\pm 1\}$, it holds that  $\Xi\tx{i}$, $\Xi x$ and $\Xi\wstar$ have the same distributions as $\mathcal{N}(0, I_{d})$.
\end{list}
Now let us fix a matrix $\bb\in\R^{d\times d}$ and $\cc^\top = [\cc_0^\top ,\cc_1]$ for $\cc_0\in \R^d$ and $\cc_1\in \R$. Letting $\bb= U\Sigma V^\top$ be the SVD of the matrix $A$, it follows that 
\begin{align}  
    f(\cc,\bb) &= \E \left[ \frac{1}{n}\sum_{i=1}^n \left[(\cc_0^\top\tx{i}+ \cc_1 \wstar^\top\tx{i} )\cdot \relu((\tx{i})^\top \bb x ) \right] + \wstar^\top x  \right]^2\\
    &\overset{(a)}{=}  \E \left[ \frac{1}{n}\sum_{i=1}^n \left[(\cc_0^\top U\tx{i}+ \cc_1 \wstar^\top U^\top U\tx{i})\cdot \relu((\tx{i})^\top U^\top \bb V x ) \right] + \wstar^\top U^\top Vx  \right]^2\\
    &\overset{(b)}{=}  \E \left[ \frac{1}{n}\sum_{i=1}^n \left[(\cc_0^\top U\Xi\tx{i}+ \cc_1 \wstar^\top\tx{i})\cdot \relu((\tx{i})^\top \Sigma x ) \right] +  \wstar^\top \Xi U^\top V\Xi x   \right]^2\\
    &\geq  \E \left[ \E_{\Xi}\left\{\frac{1}{n}\sum_{i=1}^n \left[(\cc_0^\top U\Xi\tx{i}+ \cc_1 \wstar^\top\tx{i})\cdot \relu((\tx{i})^\top \Sigma x ) \right] +  \wstar^\top \Xi U^\top V\Xi x  \right\} \right]^2\\
     &=  \E \left[  \frac{1}{n}\sum_{i=1}^n \left[\cc_1 \wstar^\top\tx{i}\cdot \relu((\tx{i})^\top \Sigma x ) \right] +  \wstar^\top \diag( U^\top V) x   \right]^2\\
     &=: \ttf(\cc_1,\Sigma, D \coloneqq \diag( U^\top V))\,. 
\end{align}
where in the third line we use the fact that $\Xi^
\top\Sigma \Xi= \Xi \Sigma \Xi = \Sigma$; and the fourth line follows from the Jensen's inequality.
Hence for the remainder of the proof, we will  characterize the global minimizer of the lower bound, i.e., $\ttf$ and then we will connect it back to the original objective.



 \underline{\emphh{3. Computation of the lower bound $\ttf$.}}


Let us now explicitly compute
$\ttf$. Let us rewrite the definition of $\ttf$.
In fact since, $\relu = \mathrm{ReLU}$ is homogenous, one can further simplify the lower bound by pushing the constant $\cc_1$ inside and write $\cc_1 \Sigma$  as $\Sigma$.
Hence, for two diagonal matrices $\Sigma, D\in \R^{d\times d}$, $\ttf$ is defined as:
\begin{align}
    \ttf(\Sigma, D) \coloneqq  \E \left[  \frac{1}{n}\sum_{i=1}^n \left[ \inpp{\wstar}{\tx{i}}\cdot \relu (x^\top \Sigma \tx{i}  )\right] +  \wstar^\top D x   \right]^2\,.
\end{align}
In particular, $D$ is constrained to be the diagonal part of an orthogonal matrix (since $D=  \diag( U^\top V)$ in the above derivation).
Now we focus on characterizing the global minimizers of $\ttf$.

The main part of the argument is inspired by the elegant observation of \cite{erdogdu2016scaled}, which says that the solution of least squares and generalized linear models are collinear for Gaussian inputs. 
We leverage the same proof technique (\emph{\`{a} la} Stein's Lemma) to prove that the presence of ReLU only changes the scaling of global optimum. 

First, since $\wstar$ is isotropic Gaussian, we can take the expectation over $\wstar$ to obtain
\begin{align}
     \ttf(\Sigma, D) =  \E \left[  \frac{1}{n}\sum_{i=1}^n \left[  \relu (x^\top \Sigma \tx{i}  ) \ \tx{i}\right] +   D x   \right]^2\,,
\end{align} 
which after a careful expansion becomes
\begin{align} \label{exp:nonlinear_expansion}
    \E \left[  \frac{1}{n^2}\sum_{i,j}   \relu (x^\top \Sigma \tx{i}  )  \relu (x^\top \Sigma \tx{j}  )  \inpp{\tx{i}}{\tx{j}} +  \frac{2}{n}\sum_{i}    \relu (x^\top \Sigma \tx{i}  )  \inpp{\tx{i}}{Dx}   \right]  + \text{const.} \quad\quad
\end{align} 
In order to compute \eqref{exp:nonlinear_expansion}, we will rely on the aforementioned argument of \cite{erdogdu2016scaled}.
In particular, from integration by parts, or Stein's lemma~\citep{erdogdu2016scaled} (since $x\sim \mathcal{N}(0,I_d)$), we have 
\begin{align}
\mathbb{E}_{x} [  \sigma(x^\top v) x] = \mathbb{E}_{x} [\sigma'(x^\top v) ] v \quad \text{for a fixed $v\in \R^d$.}    
\end{align} 
We use this to compute all the terms in \eqref{exp:nonlinear_expansion} as follows:
\begin{list}{$\bullet$}{\leftmargin=1.5em}
\setlength{\itemsep}{1pt}
\item 
We first apply Stein's lemma  to the first term of \eqref{exp:nonlinear_expansion} for $i\neq j$. This results in 
\begin{align}
&\E_{\tx{i},\tx{j},x}\relu (x^\top \Sigma \tx{i}  )  \relu (x^\top \Sigma \tx{j}  )  \inpp{\tx{i}}{\tx{j}}  \\
&\quad =   \E_{\tx{j},x} [\E_{\tx{i}} [\relu'(x^\top \Sigma \tx{i})] \ \relu(x^\top \Sigma \tx{j}) \ x^\top \Sigma x^{(j)}] 
\end{align} 
Using  the fact that $\tx{i}$ is a symmetric random variable, one can compute the expectation above as follows: one the one hand, we know  $\E_{\tx{i}} [\relu'(x^\top \Sigma \tx{i})] = \mathbb{E}_{\tx{i}} [\relu'(-x^\top \Sigma \tx{i})]$.  On the other hand, we also know that for any scalar $\alpha$, $\relu'(-\alpha)+ \relu'(\alpha) =1$. Therefore, we conclude that  $\E_{\tx{i}} [\relu'(x^\top \Sigma \tx{i})] = 1/2$. Thus, applying this technique twice, we obtain the following
\begin{align}
\E_{\tx{i},\tx{j},x}\relu (x^\top \Sigma \tx{i}  )  \relu (x^\top \Sigma \tx{j}  )  \inpp{\tx{i}}{\tx{j}}  =  \frac{1}{4}\mathbb{E}_{ x} [ x^\top \Sigma^2 x  ] = \frac{1}{4} \tr(\Sigma^2)\,.
\end{align}

\item 
Similarly, we can use Stein's lemma to the second term of \eqref{exp:nonlinear_expansion} to conclude
\begin{align}
    \E \relu (x^\top \Sigma \tx{i}  )  \inpp{\tx{i}}{Dx} = \frac{1}{2}\E_x x^\top \Sigma D x =  \frac{1}{2}\tr(\Sigma D)\,.
\end{align} 

\item Lastly, the computation of the first term of \eqref{exp:nonlinear_expansion} for $i=j$ is straightforward.
Using the fact that $\forall \alpha\in\R$, $\relu^2(\alpha)+\relu^2(-\alpha) = a^2$, we get 
\begin{align}
    &\E \left[ \relu^2(x^\top \Sigma \tx{i}) \ \|\tx{i} \|^2 \right] = \frac{1}{2} \mathbb{E} [(x^\top \Sigma \tx{i})^2 \ \| \tx{i}\|^2 ]\\
    &\quad = \frac{1}{2}\E \left[ (\tx{i})^\top \Sigma^2 \tx{i} \ \| \tx{i} \|^2 \right] = \frac{d+2}{2} \tr(\Sigma^2)\,.
\end{align}
\end{list}
Putting things all together (and ignoring the constant part in \eqref{exp:nonlinear_expansion}), we have 
\begin{align}\label{exp:lower}
     \ttf(\Sigma, D) = \frac{2(d+2) + (n-1)}{4n} \tr(\Sigma^2) + \tr(\Sigma D)\,.
\end{align}


 \underline{\emphh{4. Connecting back to the original loss function.}}
 
One can in fact write \eqref{exp:lower} solely in terms of $\cc_1$ and $\bb$ as follows:
\begin{align}
    \frac{2(d+2) + (n-1)}{4n} \tr(\Sigma^2) + \tr(\Sigma D) = \frac{2(d+2) + (n-1)}{4n} \norm{\cc_1\bb}_F^2 + \tr(\cc_1\bb)\,.
\end{align}
Since the latter is a convex function in the matrix $\cc_1\bb$, it follows that the minimizer corresponds to 
\begin{align}
    \cc_1\bb = -\frac{2n}{2(d+2)+(n-1)} \cdot I_d
\end{align}
In fact the choice $\cc_1 = 1$, $\cc_0 =0$, and $\bb =  -\frac{2n}{2(d+2)+(n-1)} \cdot I_d$ achieves this, and more crucially, satisfies the property that $\ttf = f$ for the corresponding parameters. Therefore, this shows that such choice is a global minimizer.


 
 




\section{Proofs for the multi-layer case}
\label{a:foo}

\subsection{Proof of \autoref{t:two_layer}}
The proof is based on probabilistic methods \citep{alon2016probabilistic}. 
According to \autoref{l:icl_trace_form}, the objective function can be written as (for more details check the derivations in ~\eqref{e:t:dynamic_Y_only}) 
\begin{align}
f(A_1,A_2) & = \E \tr\left( \E \left[\prod_{i=1}^2 (I -  X_0^\top A_i X_0 M) X_0^\top \wstar \wstar^\top X_0 \prod_{i=1}^2 (I - M X_0^T A_i X_0) \right]\right) \\ 
& = \E \tr\left( \E \left[\prod_{i=2}^1 (I -  X_0^\top A_i X_0 M) X_0^\top X_0 \prod_{j=1}^2 (I - M X_0^T A_j X_0) \right]\right), 
\end{align}
where we use the isotropy of $\wstar$ and the linearity of trace to get the last equation.
Suppose that $A_0^*$ and $A_1^*$ denote the global minimizer of $f$ over symmetric matrices. Since $A_1^*$ is a symmetric matrix, it admits the spectral  decomposition $A_1 = U D_1 U^\top$ where $D_1$ is a diagonal matrix and $U$ is an orthogonal matrix. Remarkably,  the distribution of $X_0$ is invariant to a linear transformation by an orthogonal matrix, i.e, $X_0$ has the same distribution as $X_0 U^\top$. This invariance yields 
\begin{align}
f(U D_1 U^\top, A_2^*) = f(D_1, U^\top A_2^* U).
\end{align}
Thus, we can assume $A_1^*$ is diagonal without loss of generality. To prove $A_2^*$ is also diagonal, we leverage a probabilistic proof technique. Consider the random diagonal matrix $S$ whose diagonal elements are either $1$ or $-1$ with probability $\frac{1}{2}$. Since the input distribution is invariant to orthogonal transformations, we have 
\begin{align}
f( D_1, A_2^*) = f(S D_1 S, S A_2^* S) = f(D_1, S A_2^* S).
\end{align}
Note that we use $S D_1 S = D_1$ in the last equation, which holds due to $D_1$ and $S$ are diagonal matrices and $S$ has diagonal elements in $\{+1, -1 \}$. Since $f$ is convex in $A_2$, a straightforward application of Jensen's inequality yields 
\begin{align}
f(D_1, A_2^*) = \E \left[ f(D_1, S A_2^* S) \right] \geq f(D_1, \E \left[ S A_2^* S \right]) = f(D_1, \diag(A_2^*)).
\end{align}
Thus, there are diagonal $D_1$ and $\diag(A_2^*)$ for which $f(D_1, \diag(A_2^*)) \leq f(A_1^*, A_2^*)$ holds for an optimal $A_1^*$ and $A_2^*$. This concludes the proof.
 
\subsection{Proof of \autoref{t:L_layer_P_0}}
\label{sec:pf:t:L_layer_P_0}

Let us drop the factor of $\nicefrac{1}{n}$ which was present in the original update \eqref{e:dynamics_Z}. This is because the constant $1/n$ can be absorbed into $A_i$'s. Doing so does not change the theorem statement, but reduces notational clutter.

Let us consider the reformulation of the in-context loss $f$ presented in \autoref{l:icl_trace_form}. Specifically, let $\overline{Z}_0$ be defined as 
\begin{align*}
\overline{Z}_0 = \begin{bmatrix}
\tx{1} & \tx{2} & \cdots & \tx{n} &\tx{n+1} \\ 
\ty{1} & \ty{2} & \cdots &\ty{n}& \ty{n+1}
\end{bmatrix} \in \R^{(d+1) \times (n+1)},
\end{align*}
where $\ty{n+1} = \lin{\wstar, \tx{n+1}}$. Let $\overline{Z}_i$ denote the output of the $(i-1)^{th}$ layer of the linear transformer (as defined in \eqref{e:dynamics_Z}, initialized at $\overline{Z}_0$). For the rest of this proof, we will drop the bar, and simply denote $\overline{Z}_i$ by $Z_i$.\footnote{This use of $Z_i$ differs the original definition in \eqref{d:Z_0}. But we will not refer to the original definition anywhere in this proof.} Let $X_i\in \R^{d\times (n+1)}$ denote the first $d$ rows of $Z_i$ and let $Y_i\in \R^{1\times (n+1)}$ denote the $(d+1)^{th}$ row of $Z_k$. Under the sparsity pattern enforced in \eqref{eq:sparse_attention}, we verify that, for any $i \in \lrbb{0,\dots,k}$,
\begin{align*}
& X_i = X_0,\\
& Y_{i+1} =  Y_{i}  + Y_{i} M X_{i}^\top A_i X_{i} = Y_0 \prod_{\ell=0}^{i} \lrp{I + M X_0^\top A_\ell X_0}.
\numberthis \label{e:t:dynamic_Y_only}
\end{align*}
where $M = \begin{bmatrix}I_{n\times n} & 0 \\0 & 0\end{bmatrix}$. We adopt the shorthand $A = \lrbb{A_i}_{i=0}^k$.

We adopt the shorthand $A = \lrbb{A_i}_{i=0}^k$. Let $\S \subset \R^{(k+1) \times d \times d}$, and $A \in \S$ if and only if for all $i\in \lrbb{0,\dots,k}$, there exists scalars $a_i \in \R$ such that $A_i = a_i \Sigma^{-1}$ and $B_i = b_i I$. We use $f(A)$ to refer to the in-context loss of \autoref{t:L_layer_P_0}, that is,
\begin{align*}
f(A) := f \lrp{ \left\{ Q_i = \begin{bmatrix}
A_i & 0 \\ 
0 & 0
\end{bmatrix}, P_i = \begin{bmatrix}
0_{d\times d} & 0 \\ 
0 & 1 
\end{bmatrix}\right\}_{i=0}^k}.
\end{align*}

Throughout this proof, we will work with the following formulation of the \emph{in-context loss} from \autoref{l:icl_trace_form}:
\begin{align*}
f(A)
=& \E_{(X_0,\wstar)} \lrb{\tr\lrp{\lrp{I-M}Y_{k+1}^\top Y_{k+1}\lrp{I-M}}}.
\numberthis \label{e:t:oiemgrkwfdlw:0}
\end{align*}
The theorem statement is equivalent to the following:
\begin{align*}
\inf_{A \in \S} \sum_{i=0}^k \lrn{\nabla_{A_i} f(A)}_F^2 = 0,
\numberthis \label{e:t:unkqdwon:0}
\end{align*}
where $\nabla_{A_i} f$ denotes derivative wrt the Frobenius norm $\lrn{A_i}_F$. Towards this end, we establish the following intermediate result: if $A \in \S$, then for any $R\in \R^{(k+1) \times d \times d}$, there exists $\tilde{R} \in \S$, such that, at $t=0$,
\begin{align*}
\frac{d}{dt} f(A + t\tilde{R}) \leq \frac{d}{dt} f(A + tR).
\numberthis \label{e:t:unkqdwon:1}
\end{align*}
In fact, we show that $\tilde{R}_i := r_i I$, for $r_i = \frac{1}{d} \tr\lrp{\Sigma^{1/2} R_i \Sigma^{1/2}}$. This implies \eqref{e:t:unkqdwon:0} via the following simple argument: Consider the "$\S$-constrained gradient flow": let $A(t): \R^+ \to \R^{(k+1)\times d \times d}$ be defined as 
\begin{align*}
& \frac{d}{dt} A_i(t) = - r_i(t) \Sigma^{-1}, \quad r_i(t) := \tr(\Sigma^{1/2} \nabla_{A_i} f (A(t)) \Sigma^{1/2})
\end{align*}
for $i=0,\dots,k$. By \eqref{e:t:unkqdwon:1}, we verify that
\begin{align*}
\frac{d}{dt} f(A(t)) \leq - \sum_{i=0}^k \lrn{\nabla_{A_i} f(A(t))}_F^2.
\numberthis \label{e:t:unkqdwon:2}
\end{align*}
We verify from its definition that $f(A) \geq 0$; if the infimum in \eqref{e:t:unkqdwon:0} fails to be zero, then inequality \eqref{e:t:unkqdwon:2} will ensure unbounded descent as $t\to \infty$, contradicting the fact that $f(A)$ is lower-bounded. This concludes the proof.

\underline{\emphh{Proof outline.}}
The remainder of the proof will be devoted to showing \eqref{e:t:unkqdwon:1}, which we outline as follows:


\begin{itemize}[leftmargin=*]
\item  In Step 1, we reduce the condition in \eqref{e:t:unkqdwon:2} to a more easily verified \emph{layer-wise} condition. Specifically, we only need to verify \eqref{e:t:unkqdwon:2} when $R_i$ are all zero except for $R_j$ for some fixed $j$ (see \eqref{e:t:unkqdwon:3})

At the end of Step 1, we set up some additional notation, and introduce an important matrix $G$, which is roughly "a product of attention layer matrices". In \eqref{e:t:unkqdwon:4}, we study the evolution of $f(A(t))$ when $A(t)$ moves in the direction of $R$, as $X_0$ is (roughly speaking) randomly transformed. 

\item In Step 2, we use the results of Step 2 to to study $G$ (see \eqref{e:t:UG_pnull}) and $\frac{d}{dt} G(A(t))$ (see \eqref{e:t:unkqdwon:5}) under random transformation of $X_0$. The idea in \eqref{e:t:unkqdwon:5} is that "randomly transforming $X_0$" has the same effect as "randomly transforming $\textcolor{red}{S}$" (recall $\textcolor{red}{S}$ is the perturbation to $\textcolor{red}{B}$). 

\item In Step 3, we apply the result from Step 2 to the expression of $\frac{d}{dt} f(A(t))$ in \eqref{e:t:unkqdwon:4}. We verify that $\tilde{R}$ in \eqref{e:t:unkqdwon:1} is exactly the expected matrix after "randomly transforming $\textcolor{red}{S}$". This concludes our proof.
\end{itemize}

\underline{\emphh{1. Reduction to layer-wise condition.}}
To prove \eqref{e:t:unkqdwon:1}, it suffices to show the following simpler condition: Let $j\in \lrbb{0,\dots,k}$. Let $R_j\in \R^{d\times d}$ be arbitrary matrices. For $C\in \R^{d\times d}$, let $A(t C, j)$ denote the collection of matrices, where $\lrb{A(t C, j)}_j = A_j + t C$, and for $i\neq j$, $A(t C, j)_i = A_i$. We show that for all $j\in \lrbb{0,\dots,k},R_j\in \R^{d\times d}$, there exists $\tilde{R}_j = r_j \Sigma^{-1}$, such that, at $t=0$,
\begin{align*}
& \frac{d}{dt} f(A(t\tilde{R}_j, j)) \leq \frac{d}{dt} f(A(tR_j,j))
\numberthis \label{e:t:unkqdwon:3}
\end{align*}
We can verify that \eqref{e:t:unkqdwon:1} is equivalent to \eqref{e:t:unkqdwon:3} by noticing that for any $R$, at $t=0$, $\frac{d}{dt} f(A + tR) = \sum_{j=0}^k \frac{d}{dt} f(A(tR_j,j))$. We will now work towards proving \eqref{e:t:unkqdwon:3} for some index $j$ that is arbitrarily chosen but fixed throughout.

Let us define, for any $C\in \R^{d\times d}$, $G(X,A_j + C) := X \prod_{i=0}^{k} \lrp{I - M X^\top \lrb{A(C, j)}_i X}$. By \eqref{e:t:dynamic_Y_only} and \eqref{e:t:oiemgrkwfdlw:0},
\begin{align*}
& f({A(tR_j,j)}) \\
=& \E \lrb{\tr\lrp{\lrp{I-M} Y_{k+1}^{\top} Y_{k+1} \lrp{I-M}}}\\
=& \E \lrb{\tr\lrp{\lrp{I-M} G(X_0,A_j + t{R}_j)^\top \wstar^\top \wstar G(X_0,A_j + t{R}_j) \lrp{I-M}}}\\
=& \E \lrb{\tr\lrp{\lrp{I-M} G(X_0,A_j + t{R}_j)^\top \Sigma^{-1} G(X_0,A_j + t{R}_j) \lrp{I-M}}}
\end{align*}
The second equality follows from plugging in \eqref{e:t:dynamic_Y_only}. For the rest of this proof, let $U$ denote a uniformly randomly sampled orthogonal matrix. Let $\US:= \Sigma^{1/2} U \Sigma^{-1/2}$. Using the fact that $X_0 \overset{d}{=} \US X_0$, we can verify
\begin{align*}
& \at{\frac{d}{dt} f(A(tR_j,j))}{t=0}\\
=& \at{\frac{d}{dt}\E \lrb{\tr\lrp{\lrp{I-M} G(X_0,A_j + t{R}_j)^\top \Sigma^{-1} G(X_0,A_j + t{R}_j) \lrp{I-M}}}}{t=0}\\
=& \at{\frac{d}{dt}\E_{X_0, U} \lrb{\tr\lrp{\lrp{I-M} G(\US X_0, A_j + t R_j)^\top \Sigma^{-1} G(\US X_0, A_j + t R_j) \lrp{I-M}}}}{t=0}\\
=& 2\E_{X_0, U} \lrb{\tr\lrp{\lrp{I-M} G(\US X_0, A_j )^\top \Sigma^{-1} \at{\frac{d}{dt} G(\US X_0, A_j + t R_j)}{t=0} \lrp{I-M}}}.
\numberthis \label{e:t:unkqdwon:4}
\end{align*}

\underline{\emphh{2. $G$ and $\frac{d}{dt} G$ under random transformation of $X_0$.}}
We will now verify that $G(\US X_0, A_j) = \US G(X_0, A_j)$:
\begin{align*}
& G(\US X_0, A_j)\\
=& \US X_0 \prod_{i=0}^{k} \lrp{I + M X_0^T \US^{\top} A_i \US X_0}\\
=& \US G(X_0, A_j),
\numberthis \label{e:t:UG_pnull}
\end{align*}
where we use the fact that $\US^\top A_i \US = \US^\top (a_i \Sigma^{-1}) \US = A_i$. Next, we verify that
\begin{align*}
& \at{\frac{d}{dt} G(\US X_0, A + tR_j)}{t=0}\\
=& \US X_0 \lrp{\prod_{i=0}^{j-1} (I + M X_0^T A_i X_0)} M X_0^T \US^{\top} R_j \US X_0 \prod_{i=j+1}^k (I + M X_0^T A_i X_0)\\
=& \US \frac{d}{dt} G(X_0, A_j + t\US^{\top} R_j \US)
\numberthis \label{e:t:unkqdwon:5}
\end{align*}
where the first equality again uses the fact that $\US^{\top} A_i \US = A_i$.

\underline{\emphh{3. Putting everything together.}} Let us continue from \eqref{e:t:unkqdwon:4}. Plugging \eqref{e:t:UG_pnull} and \eqref{e:t:unkqdwon:5} into \eqref{e:t:unkqdwon:4}, 
{\allowdisplaybreaks
\begin{align*}
& \at{\frac{d}{dt} f({A(tR_j,j)})}{t=0}\\
=& 2\E_{X_0, U} \lrb{\tr\lrp{\lrp{I-M} G(\US X_0, A_j )^\top \Sigma^{-1} \at{\frac{d}{dt} G(\US X_0, A_j + t R_j)}{t=0} \lrp{I-M}}}\\
\overset{(i)}{=}& 2\E_{X_0, U} \lrb{\tr\lrp{\lrp{I-M} G( X_0, A_j )^\top \Sigma^{-1} \at{\frac{d}{dt} G(X_0, A_j + t \US^{\top} R_j \US)}{t=0} \lrp{I-M}}}\\
=& 2\E_{X_0} \lrb{\tr\lrp{\lrp{I-M} G( X_0, A_j )^\top \Sigma^{-1} \E_{U}\lrb{\at{\frac{d}{dt} G(X_0, A_j + t \US^{\top} R_j \US)}{t=0}} \lrp{I-M}}}\\
\overset{(ii)}{=}& 2\E_{X_0} \lrb{\tr\lrp{\lrp{I-M} G( X_0, A_j )^\top \Sigma^{-1} \at{\frac{d}{dt} G(X_0, A_j + t \E_U\lrb{\US^{\top} R_j \US} )}{t=0} \lrp{I-M}}}\\
=& 2\E_{X_0} \lrb{\tr\lrp{\lrp{I-M} G( X_0, A_j )^\top \Sigma^{-1} \at{\frac{d}{dt} G(X_0, A_j + t \cdot r_j \Sigma^{-1})}{t=0} \lrp{I-M}}}\\
=& \at{\frac{d}{dt} f(A(t\cdot r_j \Sigma^{-1}, j))}{t=0},
\end{align*}
}
where $r_j := \frac{1}{d} \tr\lrp{\Sigma^{1/2} R_j \Sigma^{1/2}}$. In the above, $(i)$ uses 1. \eqref{e:t:UG_pnull} and \eqref{e:t:unkqdwon:5}, as well as the fact that $\US^\top \Sigma^{-1} \US = \Sigma^{-1}$. $(ii)$ uses the fact that $\at{\frac{d}{dt} G(X_0, A_j + t C)}{t=0}$ is affine in $C$. To see this, one can verify from the definition of $G$, e.g. using similar algebra as \eqref{e:t:unkqdwon:5}, that $\frac{d}{dt} G(X_0, A_j + C)$ is affine in $C$. Thus $\E_U\lrb{G(X_0, A_j + t \US^{\top} R_j \US)} = G(X_0, A_j + t \E_U\lrb{\US^{\top} R_j \US)}$. 


\iffalse
\newpage
\xc{old proof below}
%with $\overline{Y}_0 = \begin{bmatrix}\ty{1} & \ty{2} \cdots & \ty{n+1}\end{bmatrix}$ and 
\begin{align*}
\overline{Y}_{k+1} 
=& \overline{Y}_0 \prod_{i=0}^{k} \lrp{I - M X_0^T A_i X_0}\\
=& \wstar^\top \underbrace{X_0 \prod_{i=0}^{k} \lrp{I - M X_0^T A_i X_0}}_{:= G(X_0,A)},
\numberthis \label{e:t:G(X,A)}
\end{align*}
In the above, we use $G(X_0,A)$  to denote the product of matrices as indicated above (recall that $A$ is shorthand for $\lrbb{A_i}_{i=0}^k$). Plugging into \eqref{e:t:oiemgrkwfdlw:0}, we can further simplify
\begin{align*}
& f(\lrbb{A_i}_{i=0}^k) \\
=& \E \lrb{\Tr\lrp{\lrp{I-M} \lrp{\wstar^\top G(X_0,A)}^{\top}\lrp{\wstar^\top G(X_0,A)} \lrp{I-M}}}\\
=& \E \lrb{\Tr\lrp{\lrp{I-M} G(X_0,A)^\top \wstar \wstar^\top G(X_0,A)\lrp{I-M}}}\\
\overset{(i)}{=}& \E \lrb{\Tr\lrp{\lrp{I-M} G(X_0,A)^\top \Sigma^{-1} G(X_0,A)\lrp{I-M}}}\\
=:& \E\lrb{h(G(X_0,A))}
\numberthis \label{e:t:h(G)}
\end{align*}
where $(i)$ uses the fact that $\wstar$ is sampled independently of $X_0$, and $\E{\wstar \wstar^{\top}} = \Sigma^{-1}$ as assumed in the Theorem statement. We also define $h(G): \R^{d \times d} \to \R$ as $h(G) = \Tr\lrp{(I-M) G(X_0,A)^{\top} \Sigma^{-1} G(X_0,A) (I-M)}$. To recap: so far we show that the in-context loss, as a function of $\left\{ Q_i = \begin{bmatrix}
A_i & 0 \\ 
0 & 0
\end{bmatrix}, P_i = \begin{bmatrix}
0_{d\times d} & 0 \\ 
0 & 1 
\end{bmatrix}\right\}_{i=0}^k$, as defined in \eqref{def:ICL linear}, is equivalent to $\E\lrb{h(G(X_0,A))}$, defined in \eqref{e:t:h(G)}, where $A = \lrbb{A_i}_{i=0}^k$. 

We will now demonstrate the existence of the proposed critical points for \eqref{e:t:h(G)}. For notational convenience, let $\S \subset \R^{k+1 \times d \times d}$ denote a subset of the parameter space, such that $A\in \S$ is equivalent to "exists $a_0,a_i,\dots,a_k \in \R$, such that for all $i=0,\dots,k$, $A_i = a_i \Sigma^{-1}$". 

Next, we observe two crucial facts about the in-context loss \eqref{e:t:h(G)}: First, $h(G)$ is a convex function of $G$. Second, if $U \in \R^d$ is a uniformly-sampled rotation matrix,\footnote{Same argument works if $U$ is simply a uniformly-sampled random permutation matrix.} then $X_0$ has the same distribution as $\Sigma^{1/2} U \Sigma^{-1/2} X_0$. Thus $\E_{X_0}\lrb{h(G(X_0,A))} = \E_{U, X_0}\lrb{h(G(\Sigma^{1/2} U \Sigma^{-1/2} X_0,A))}$. Combining these two facts, and applying Jensen's inequality, we obtain that for any $A$,
\begin{align*}
\E_{U, X_0}\lrb{h(G(\Sigma^{1/2} U \Sigma^{-1/2} X_0,A))} \geq \E_{X_0}\lrb{h\lrp{\E_{U}\lrb{G(\Sigma^{1/2} U \Sigma^{-1/2} X_0,A)}}}.
\numberthis \label{e:t:jensen_pnull}
\end{align*}
\xc{ended up not needing Jensens!}


Let us now consider some fixed but arbitrary layer index $j \in \lrbb{0,\dots,k}$. Let $R_j\in \R^{d\times d}$, and let $A(R_j,j) := \lrbb{A_0, A_1, A_2,\dots, A_{j-1}, A_j + R_j, A_{j+1}, ,\dots, A_k}$. We also define $N(X, A, R_j, j):= \lrp{\prod_{i=0}^{j-1} (I + M X^T A_i X)} M X^T R_j X \prod_{i=j+1}^k (I + M X^T A_i X)$. From these definitions, we verify
\begin{enumerate}
\item $G(X_0, A(R_j,j)) = G(X_0,  A) + N(X_0, A, R_j, j)$.
\item If $A\in \S$, then for any orthogonal matrix $U \in R^{d\times d}$,  
$$G(\Sigma^{1/2} U \Sigma^{-1/2} X_0, A) = \Sigma^{1/2} U \Sigma^{-1/2} G(X_0,A)$$

\item If $A\in \S$, then for any orthogonal matrix $U \in R_j^{d\times d}$, 
$$N(\Sigma^{1/2} U \Sigma^{-1/2} X_0, A, R_j, j) = \Sigma^{1/2} U \Sigma^{-1/2} N(X_0, A, U^T R_j U, j).$$
\end{enumerate}
Items 2 and 3 above can be verified by considering the definition of $G$ and $N$, and noticing that $(\Sigma^{1/2} U \Sigma^{-1/2} X_0)^T A_i (\Sigma^{1/2} U \Sigma^{-1/2} X_0) = (\Sigma^{1/2} U \Sigma^{-1/2} X_0)^T (a_i \Sigma^{-1}) (\Sigma^{1/2} U \Sigma^{-1/2} X_0) = X_0^T (a_i \Sigma^{-1/2} U^\top \Sigma^{1/2} \Sigma^{-1} \Sigma^{1/2} U \Sigma^{-1/2}) X_0 = X_0^T A_i X_0$.

Finally, let us define $A(t):= A(tR_j,j)$. Then for sufficiently small $t\in \R^+$,
\begin{align*}
& \E_{X_0}\lrb{h(G(X_0, A(tR_j, j))}\\
=& \E_{U,X_0}\lrb{h(G(\Sigma^{1/2} U \Sigma^{-1/2}X_0, A(tR_j, j))}\\
\overset{(i)}{=}& \E_{U,X_0}\lrb{h\lrp{\Sigma^{1/2} U \Sigma^{-1/2}G(X_0, A) + \Sigma^{1/2} U \Sigma^{-1/2} N(X_0, A, \Sigma^{-1/2} U^\top \Sigma^{1/2} tR_j \Sigma^{1/2} U \Sigma^{-1/2},j)}}\\
\overset{(ii)}{=}& \E_{X_0}\lrb{h\lrp{G(X_0, A)}} \\
& + 2 \E_{X_0}\lrb{\Tr\lrp{\lrp{I-M} G(X_0,A)^\top \Sigma^{-1} N(X_0, A, \Sigma^{-1/2} \E_{U}\lrb{U^\top \Sigma^{1/2} tR_j \Sigma^{1/2} U} \Sigma^{-1/2},j)}} + o(t^2)\\
\overset{(iii)}{=}& \E_{X_0}\lrb{h\lrp{G(X_0, A)}} + 2 \E_{X_0}\lrb{\Tr\lrp{\lrp{I-M} G(X_0,A)^\top \Sigma^{-1} N(X_0, A, r \Sigma^{-1},j)}} + o(t^2)\\
=& \E_{X_0}\lrb{h(G(X_0, A(t\cdot r \Sigma^{-1}, j))} + o(t^2),
\end{align*}
where $r:= \frac{1}{d} \sum_{j=1}^d \lrb{\Sigma^{1/2} R_j \Sigma^{1/2}}_{j,j}$. In the above, $(i)$ uses items 1-3 listed above. $(ii)$ is by expanding the expression for $h$ in \eqref{e:t:h(G)}, and the fact that $N(X,A,R_j,j)$ is linear in $R_j$. Step $(iii)$ uses the fact that $\E_{U}\lrb{U^\top \Sigma^{1/2} tR_j \Sigma^{1/2} U} = r I$.

Taking derivative with respect to $t$, we verify that
\begin{align*}
\frac{d}{dt} \E_{X_0}\lrb{h(G(X_0, A(tR_j, j))} = \frac{d}{dt} \E_{X_0}\lrb{h(G(X_0, A(t\cdot r \Sigma^{-1}, j))}.
\end{align*}

Now suppose that we have a $R_i$ for each layer $i=0,\dots,k$. Let ${R} = \lrbb{{R}_i}_{i=0}^k$. Let ${A}(t) = \lrbb{A_i + t{R}_i}_{i=0}^k$. Let ${r}_i := \frac{1}{d} \sum_{j=1}^d \lrb{\Sigma^{1/2} R \Sigma^{1/2}}_{j,j}$. Then
\begin{align*}
& \frac{d}{dt} \E_{X_0}\lrb{h(G(X_0, {A}(t))}\\
=& \sum_{j=1}^d \frac{d}{dt} \E_{X_0}\lrb{h(G(X_0, A(t{R}_j, j))}\\
=& \sum_{j=1}^d \frac{d}{dt} \E_{X_0}\lrb{h(G(X_0, A(t\cdot {r}_j \Sigma^{-1}, j))}
\end{align*}

Let $\tilde{R} = \lrbb{r_i \Sigma^{-1}}_{i=0}^k$. 

We have thus verified the following: Let $R= \lrbb{R_i}_{i=0}^k$ denote any descent direction, and let $\tilde{R}= \lrbb{r_i}_{i=0}^k$, where $r_i := \frac{1}{d} \sum_{j=1}^d \lrb{\Sigma^{1/2} R \Sigma^{1/2}}_{j,j}$. Then moving in the direction of $\tilde{R}$ gives the same amount of descent as $R$, i.e.
\begin{align*}
\frac{d}{dt} f\lrp{\lrbb{A_i + t\tilde{R}_i}_{i=0}^k} = \frac{d}{dt} f\lrp{\lrbb{A_i + t R_i}_{i=0}^k}.
\end{align*}
Consequently, given any initial $A(0) \in \S$, we can define a flow $A(t) : [0,\infty) \to \R^{(k+1)\times d \times d}$, such that 1. $A(t) \in \S$, and 2. for all $t$. $\frac{d}{dt} f(A(t)) \leq -\lrn{\nabla_A f(A(t))}$. \xc{quantify the norm more precisely? see the discussion below.}

\xc{How to show convergence to stationary point? The above only shows that gradient is arbitrarily close to $0$, which is a subtle difference. E.g. gradient descent of $y = 1/x$ never finds a stationary point.}


\xc{\textbf{Proof that the proposed flow is in fact steepst descent:}\\
Let $\lrn{M}_{\Sigma, F}$ denote the Frobenius norm wrt $\Sigma$, i.e. $\lrn{M}_{\Sigma, F}^2 = \Tr\lrp{\Sigma M \Sigma M^{\top}}$. Then 
\begin{align*}
& \lrn{M}_{\Sigma, F}^2\\
=& \Tr\lrp{\Sigma M \Sigma M^{\top}}\\
=& \Tr\lrp{\Sigma M \Sigma M^{\top}}\\
=& \E_U\lrb{\Tr\lrp{\Sigma \Sigma^{-1/2} U^\top \Sigma^{1/2} M \Sigma^{1/2} U \Sigma^{-1/2}\Sigma \Sigma^{-1/2} U^\top \Sigma^{1/2} M^{\top} \Sigma^{1/2} U \Sigma^{-1/2}}}\\
=& \E_u \lrb{\lrn{\Sigma^{-1/2} U^\top \Sigma^{1/2} M \Sigma^{1/2} U \Sigma^{1/2}}_{\Sigma, F}^2}
\end{align*}
On the other hand, by Jensen's inequality, 
\begin{align*}
& \E_u \lrb{\lrn{\Sigma^{-1/2} U^\top \Sigma^{1/2} M \Sigma^{1/2} U \Sigma^{1/2}}_{\Sigma, F}^2}\\
\geq& \lrn{\E_u \lrb{\Sigma^{-1/2} U^\top \Sigma^{1/2} M \Sigma^{1/2} U \Sigma^{1/2}}}_{\Sigma, F}^2\\
\geq& \lrp{\frac{1}{d} \sum_{j=1}^d \lrb{\Sigma^{1/2} M \Sigma^{1/2}}_{j,j}}^2.
\end{align*}
Thus $r_i \Sigma^{-1}$ is in fact the \emph{steepest descent} direction, wrt $\lrn{\cdot}_{\Sigma, F}$.}

\fi












\subsection{Proof of \autoref{t:L_layer_P_identity}}
The proof of \autoref{t:L_layer_P_identity} is similar to that of \autoref{t:L_layer_P_0}, and with a similar setup. However to keep the proof self-contained, we will restate the setup. Once again, we drop the factor of $\frac{1}{n}$ which was present in the original update \eqref{e:dynamics_Z}. This is because the constant $1/n$ can be absorbed into $A_i$'s. Doing so does not change the theorem statement, but reduces notational clutter.

Let us consider the reformulation of the in-context loss $f$ presented in \autoref{l:icl_trace_form}. Specifically, let $\overline{Z}_0$ be defined as 
\begin{align*}
\overline{Z}_0 = \begin{bmatrix}
\tx{1} & \tx{2} & \cdots & \tx{n} &\tx{n+1} \\ 
\ty{1} & \ty{2} & \cdots &\ty{n}& \ty{n+1}
\end{bmatrix} \in \R^{(d+1) \times (n+1)},
\end{align*}
where $\ty{n+1} = \lin{\wstar, \tx{n+1}}$. Let $\overline{Z}_i$ denote the output of the $(i-1)^{th}$ layer of the linear transformer (as defined in \eqref{e:dynamics_Z}, initialized at $\overline{Z}_0$). For the rest of this proof, we will drop the bar, and simply denote $\overline{Z}_i$ by $Z_i$.\footnote{This use of $Z_i$ differs the original definition in \eqref{d:Z_0}. But we will not refer to the original definition anywhere in this proof.} Let $X_i\in \R^{d\times n+1}$ denote the first $d$ rows of $Z_i$ and let $Y_i\in \R^{1\times n+1}$ denote the $(d+1)^{th}$ row of $Z_k$. Under the sparsity pattern enforced in \eqref{eq:full_attention}, we verify that, for any $i \in \lrbb{0,\dots,k}$,
\begin{align*}
& X_{i+1} =  X_{i}  + B_i X_i M X_{i}^\top A_i X_{i} \\
& Y_{i+1} =  Y_{i}  + Y_{i} M X_{i}^\top A_i X_{i} = Y_0 \prod_{\ell=0}^{i} \lrp{I + M X_\ell^T A_\ell X_\ell}.
\numberthis \label{e:t:XY_dynamics}
\end{align*}
We adopt the shorthand $A = \lrbb{A_i}_{i=0}^k$ and $B = \lrbb{B_i}_{i=0}^k$. Let $\S \subset \R^{2\times (k+1) \times d \times d}$, and $(A,B) \in \S$ if and only if for all $i\in \lrbb{0,\dots,k}$, there exists scalars $a_i,b_i \in \R$ such that $A_i = a_i \Sigma^{-1}$ and $B_i = b_i I$. Throughout this proof, we will work with the following formulation of the \emph{in-context loss} from \autoref{l:icl_trace_form}:
\begin{align*}
f(A,B) := \E_{(X_0,\wstar)} \lrb{\tr\lrp{\lrp{I-M}Y_{k+1}^\top Y_{k+1}\lrp{I-M}}}.
\numberthis \label{e:t:simpler_f}
\end{align*}
(note that the only randomness in $Z_0$ comes from $X_0$ as $Y_0$ is a deterministic function of $X_0$). The theorem statement is equivalent to the following:

\begin{align*}
\inf_{(A,B) \in \S} \sum_{i=0}^k \lrn{\nabla_{A_i} f(A,B)}_F^2 + \lrn{\nabla_{B_i} f(A,B)}_F^2 = 0
\numberthis \label{e:main_theorem_goal_PQ}
\end{align*}
where $\nabla_{A_i} f$ denotes derivative wrt the Frobenius norm $\lrn{A_i}_F$.

Our goal is to show that, if $(A,B) \in \S$, then for any $(R,S)\in \R^{2\times (k+1) \times d \times d}$, there exists $(\tilde{R},\tilde{S}) \in \S$, such that, at $t=0$,
\begin{align*}
\frac{d}{dt} f(A + t\tilde{R}, B + t \tilde{S}) \leq \frac{d}{dt} f(A + tR, B + tS).
\numberthis \label{e:t:wfoqmdksa:1}
\end{align*}
In fact, we show that $\tilde{R}_i := r_i I$, for $r_i = \frac{1}{d} \tr\lrp{\Sigma^{1/2} R_i \Sigma^{1/2}}$ and $\tilde{S}_i = s_i I$, for $s_i = \frac{1}{d} \tr \lrp{\Sigma^{-1/2} S_i \Sigma^{1/2}}$. This implies \eqref{e:main_theorem_goal_PQ} via the following simple argument: Consider the "$\S$-constrained gradient flow": let $A(t): \R^+ \to \R^{(k+1)\times d \times d}$ and $B(t): \R^+ \to \R^{(k+1)\times d \times d}$ be defined as 
\begin{align*}
& \frac{d}{dt} A_i(t) = - r_i(t) \Sigma^{-1}, \quad r_i(t) := \tr(\Sigma^{1/2} \nabla_{A_i} f (A(t),B(t)) \Sigma^{1/2})\\
& \frac{d}{dt} B_i(t) = - s_i(t) \Sigma^{-1}, \quad s_i(t) := \tr(\Sigma^{-1/2} \nabla_{B_i} f(A(t),B(t)) \Sigma^{1/2}),
\end{align*}
for $i=0,\dots,k$. By \eqref{e:t:wfoqmdksa:1}, we verify that
\begin{align*}
\frac{d}{dt} f(A(t),B(t)) \leq - \lrp{\sum_{i=0}^k \lrn{\nabla_{A_i} f(A(t),B(t))}_F^2 + \lrn{\nabla_{B_i}f(A(t),B(t))}_F^2}.
\numberthis \label{e:t:wfoqmdksa:4}
\end{align*}
We verify from its definition that $f(A,B) \geq 0$; if \eqref{e:main_theorem_goal_PQ} does not hold then \eqref{e:t:wfoqmdksa:4} will ensure unbounded descent as $t\to \infty$, contradicting the fact that $f(A,B)$ is lower-bounded. This concludes the proof.

\underline{\emphh{Proof outline.}}
The remainder of the proof will be devoted to showing \eqref{e:t:wfoqmdksa:1}, which we outline as follows:
\begin{itemize}[leftmargin=*]
\item In Step 1, we reduce the condition in \eqref{e:t:wfoqmdksa:1} to a more easily verified \emph{layer-wise} condition. Specifically, we only need to verify \eqref{e:t:wfoqmdksa:1} in one of the two cases: (I) when $R_i,S_i$ are all zero except for $R_j$ for some fixed $j$ (see \eqref{e:t:wfoqmdksa:3}), or (II) when $R_i,S_i$ are all zero except for $S_j$ for some fixed $j$ (see \eqref{e:t:wfoqmdksa:2}).

We focus on the proof of (II), as the proof of (I) is almost identical. At the end of Step 1, we set up some additional notation, and introduce an important matrix $G$, which is roughly "a product of attention layer matrices". In \eqref{e:t:rmlsdm:0}, we study the evolution of $f(A,B(t))$ when $B(t)$ moves in the direction of $S$, as $X_0$ is (roughly speaking) randomly transformed. This motivates the subsequent analysis in Steps 2 and 3 below.

\item In Step 2, we study how outputs of each layer \eqref{e:t:XY_dynamics} changes when $X_0$ is randomly transformed. There are two main results here: First we provide the expression for $X_i$ in \eqref{e:t:UX}. Second, we provide the expression for $\frac{d}{dt} X_i(B(t))$ in \eqref{e:t:key_induction}.

\item In Step 3, we use the results of Step 2 to to study $G$ (see \eqref{e:t:UG}) and $\frac{d}{dt} G(B(t))$ (see \eqref{e:t:rmlsdm:3}) under random transformation of $X_0$.

The idea in \eqref{e:t:rmlsdm:3} is that "randomly transforming $X_0$" has the same effect as "randomly transforming $S$" (recall $S$ is the perturbation to $B$). 

\item In Step 4, we use the results from Steps 2 and 3 to the expression of $\frac{d}{dt} f(A, B(t))$ in \eqref{e:t:rmlsdm:0}. We verify that $\tilde{S}$ in \eqref{e:t:wfoqmdksa:1} is exactly the expected matrix after "randomly transforming $S$". This concludes our proof of (II).

\item In Step 5, we sketch the proof of (I), which is almost identical to Steps 2-4. 
\end{itemize}


\underline{\emphh{1. Reduction to layer-wise condition.}}
To prove \eqref{e:t:wfoqmdksa:1}, it suffices to show the following simpler condition: Let $j\in \lrbb{0,\dots,k}$. Let $R_j, S_j \in \R^{d\times d}$ be arbitrary matrices. For $C\in \R^{d \times d}$, let $A(t C, j)$ denote the collection of matrices, where $A(t C, j)_j = A_j + t C$, and for $i\neq j$, $A(t C, j)_i = A_i$. Define $B(t C,j)$ analogously. We show that for all $j\in \lrbb{0,\dots,k}$ and all $R_j,S_j\in \R^{d\times d}$, there exists $\tilde{R}_j = r_j \Sigma^{-1}$ and $\tilde{S}_j = s_j \Sigma^{-1}$, such that, at $t=0$,
\begin{align*}
& \frac{d}{dt} f(A(t\tilde{R}_j, j), B) \leq \frac{d}{dt} f(A(tR_j,j), B)
\numberthis \label{e:t:wfoqmdksa:2}\\
\text{and} \qquad 
& \frac{d}{dt} f(A, B(t\tilde{S}_j,j)) \leq \frac{d}{dt} f(A, B(tS_j,j)).
\numberthis \label{e:t:wfoqmdksa:3}
\end{align*}
We can verify that \eqref{e:t:wfoqmdksa:1} is equivalent to \eqref{e:t:wfoqmdksa:2}+\eqref{e:t:wfoqmdksa:3} by noticing that for any $(R,S) \in \R^{2 \times (k+1) \times d\times d}$, at $t=0$, $\frac{d}{dt} f(A + tR, B + tS) = \sum_{j=0}^k \lrp{\frac{d}{dt} f(A(tR_j,j), B) + \frac{d}{dt} f(A, B(tS_j,j))}$. 

We will first focus on proving \eqref{e:t:wfoqmdksa:3} (the proof of \eqref{e:t:wfoqmdksa:2} is similar, and we present it in Step 5 at the end), for some index $j$ that is arbitrarily chosen but fixed throughout. Notice that $X_i$ and $Y_i$ in \eqref{e:t:XY_dynamics} are in fact functions of $A,B$ and $X_0$. For most of our subsequent discussion, $A_i$ (for all $i$) and $B_i$ (for all $i\neq j$) can be treated as constant matrices. We will however make the dependence on $X_0$ and $B_j$ explicit (as we consider the curve $B_j + t S$), i.e. we use $X_i(X, C)$ (resp $Y_i(X,C)$) to denote the value of $X_i$ (resp $Y_i$) from \eqref{e:t:XY_dynamics}, with $X_0 = X$, and $B_j = C$.

By \eqref{e:t:simpler_f} and \eqref{e:t:XY_dynamics},
\begin{align*}
& f(A, B(tS_j,j)) \\
=& \E \lrb{\tr\lrp{\lrp{I-M} Y_{k+1}(X_0, B_j+tS)^{\top} Y_{k+1}(X_0, B_j+tS_j) \lrp{I-M}}}\\
=& \E \lrb{\tr\lrp{\lrp{I-M} G(X_0, B_j + t S_j)^\top \wstar^\top \wstar G(X_0, B_j+tS_j) \lrp{I-M}}}\\
=& \E \lrb{\tr\lrp{\lrp{I-M} G(X_0, B_j + t S_j)^\top \Sigma^{-1} G(X_0, B_j + t S_j) \lrp{I-M}}}
\end{align*}
where $G(X, C) := X \prod_{i=0}^{k} \lrp{I - M X_i(X,C)^T A_i X_i(X,C)}$. The second equality follows from plugging in \eqref{e:t:XY_dynamics}. 

For the rest of this proof, let $U$ denote a uniformly randomly sampled orthogonal matrix. Let $\US:= \Sigma^{1/2} U \Sigma^{-1/2}$. Using the fact that $X_0 \overset{d}{=} \US X_0$, we can verify
\begin{align*}
& \at{\frac{d}{dt} f(A, B(tS_j,j))}{t=0}\\
=& \at{\frac{d}{dt}\E_{X_0} \lrb{\tr\lrp{\lrp{I-M} G(X_0, B_j + t S_j)^\top \Sigma^{-1} G( X_0, B_j + t S_j) \lrp{I-M}}}}{t=0}\\
=& \at{\frac{d}{dt}\E_{X_0, U} \lrb{\tr\lrp{\lrp{I-M} G(\US X_0, B_j + t S_j)^\top \Sigma^{-1} G(\US X_0, B_j + t S_j) \lrp{I-M}}}}{t=0}\\
=& 2\E_{X_0, U} \lrb{\tr\lrp{\lrp{I-M} G(\US X_0, B_j )^\top \Sigma^{-1} \at{\frac{d}{dt} G(\US X_0, B_j + t S_j)}{t=0} \lrp{I-M}}}.
\numberthis \label{e:t:rmlsdm:0}
\end{align*}


\underline{\emphh{2. $X_i$ and $\frac{d}{dt} X_i$ under random transformation of $X_0$.}} In this step, we prove that when $X_0$ is transformed by $\US$, $X_i$ for $i\geq 1$ are likewise transformed in a simple manner. The first goal of this step is to show
\begin{align*}
X_i(\US X_0, B_j) = \US X_i(X_0, B_j).
\numberthis \label{e:t:UX}
\end{align*}
We will prove this by induction. When $i=0$, this clearly holds by definition. Suppose that \eqref{e:t:UX} holds for some $i$. Then
\begin{align*}
& X_{i+1}(\US X_0, B_j)\\
=& X_i(\US X_0, B_j) + B_i X_i(\US X_0, B_j) M X_i(\US X_0, B_j)^T A_i X_i(\US X_0, B_j) \\
=& \US X_i(X_0, B_j) + \US B_i X_i(X_0, B_j) M X_i(X_0, B_j)^T A_i X_i(X_0, B_j)\\
=& \US X_{i+1} (X_0, B_j)
\end{align*}
where the second equality uses the inductive hypothesis, and the fact that $A_i = a_i \Sigma^{-1}$, so that $\US^T A_i \US = A_i$, and the fact that $B_i = b_i I$, from the definition of $\S$ and our assumption that $(A,B)\in \S$. This concludes the proof of \eqref{e:t:UX}.

We now present the second main result of this step. Let $\US^{-1} := \Sigma^{1/2} U^T \Sigma^{-1/2}$, so that it satisfies $\US \US^{-1}= \US^{-1} \US = I$. For all $i$,
\begin{align*}
\at{\US^{-1} \frac{d}{dt} X_i(\US X_0, B_j + t S_j)}{t=0} = \at{\frac{d}{dt} X_i(X_0, B_j + t \US^{-1} S_j \US)}{t=0}.
\numberthis \label{e:t:key_induction}
\end{align*}
To reduce notation, we will not write $\at{\cdot}{t=0}$ explicitly in the subsequent proof. We first write down the dynamics for the right-hand-side term of \eqref{e:t:key_induction}:
From \eqref{e:t:XY_dynamics}, for any $\ell \leq j$, and for any $i \geq j+1$, and for any $C\in \R^{d\times d}$,
\begin{align*}
& \frac{d}{dt} X_{\ell} \lrp{X_0, B_j + t C} = 0\\
& \frac{d}{dt} X_{j+1} \lrp{X_0, B_j + t C} = C {X_{j}\lrp{X_0, B_j}} M {X_{j}\lrp{X_0, B_j}}^{\top} A_j  {X_{j}\lrp{X_0, B_j}}\\
& \frac{d}{dt} X_{i+1} \lrp{X_0, B_j + t C}
= \frac{d}{dt} X_{i} \lrp{X_0, B_j + t C}\\
&\qquad\qquad\qquad\quad + B_i \lrp{\frac{d}{dt} X_{i}\lrp{X_0, B_j + t C}} M X_{i}\lrp{X_0, B_j}^{\top} A_i  X_{i}\lrp{X_0, B_j}\\
&\qquad\qquad\qquad\quad + B_i {X_{i}\lrp{X_0, B_j}} M \lrp{\frac{d}{dt} X_{i}\lrp{X_0, B_j + tC}}^{\top} A_i  X_{i}\lrp{X_0, B_j}\\
&\qquad\qquad\qquad\quad + B_i {X_{i}\lrp{X_0, B_j}} M {X_{i}\lrp{X_0, B_j}}^{\top} A_i  \lrp{\frac{d}{dt}X_{i}\lrp{X_0, B_j+tC}}
\numberthis \label{e:t:rmlsdm:2}
\end{align*}


We are now ready to prove \eqref{e:t:key_induction} using induction. For the base case, we verify that for $\ell \leq j$, $\US^{-1} \frac{d}{dt} X_{\ell} \lrp{\US X_0, B_k + tS_j} = 0 = \frac{d}{dt} X_{\ell} \lrp{X_0, B_j + t\US^{-1} S_j \US}$ (see first equation in \eqref{e:t:rmlsdm:2}). For index $j+1$, we verify that
\begin{align*}
& \US^{-1} \frac{d}{dt} X_{j+1} \lrp{\US X_0, B_j + t S_j} \\
=& \US^{-1} S_j \US X_j( X_0, B_j) M X_j(\US X_0, B_j)^\top A_j X_j(\US X_0, B_j)\\
=& \frac{d}{dt} X_{j+1} \lrp{X_0, B_j + t \US^{-1} S_j \US}
\numberthis \label{e:t:rmlsdm:4}
\end{align*}
where we use two facts:  1. $X_i(\US X_0, B_j) = \US X_i(X_0, B_j)$ from \eqref{e:t:UX}, 2. $A_i = a_i \Sigma^{-1}$, so that $\US^{\top} A_i \US = A_i$. We verify by comparison to the second equation in \eqref{e:t:rmlsdm:2} that $\US^{-1} \frac{d}{dt} X_{j} \lrp{\US X_0, B_j + tS_j} = 0 = \frac{d}{dt} X_{j} \lrp{X_0, B_j + t\US^{-1} S_j \US}$. These conclude the proof of the base case.

Now suppose that \eqref{e:t:key_induction} holds for some $i$. We will now prove \eqref{e:t:key_induction} holds for $i+1$. From \eqref{e:t:XY_dynamics},
{\allowdisplaybreaks
\begin{align*}
& \US^{-1} \frac{d}{dt} X_{i+1} \lrp{\US X_0, B_j + t S_j}\\
=& \US^{-1}\frac{d}{dt} \lrp{X_i\lrp{\US X_0, B_j + t S_j}}\\
&\qquad + \US^{-1} \frac{d}{dt}\lrp{B_i X_i\lrp{\US X_0, B_j + t S_j} M X_i\lrp{\US X_0, B_j + t S_j}^\top A_i X_i\lrp{\US X_0, B_j + t S_j}}\\
=& \US^{-1}\frac{d}{dt} \lrp{X_i\lrp{\US X_0, B_j + t S_j}}\\
&\qquad + \US^{-1} B_i \lrp{\frac{d}{dt}X_i\lrp{\US X_0, B_j + t S_j}} M X_i\lrp{\US X_0, B_j}^{\top} A_i X_i\lrp{\US X_0, B_j}\\
&\qquad + \US^{-1} B_i {X_i\lrp{\US X_0, B_j}} M \lrp{\frac{d}{dt}X_i\lrp{\US X_0, B_j + t S_j}}^{\top} A_i X_i\lrp{\US X_0, B_j}\\
&\qquad + \US^{-1} B_i {X_i\lrp{\US X_0, B_j}} M {X_i\lrp{\US X_0, B_j}}^{\top} A_i \lrp{\frac{d}{dt}X_i\lrp{\US X_0, B_j + t S_j}}\\
\overset{(i)}{=}&\US^{-1}\frac{d}{dt} {X_i\lrp{\US X_0, B_j + t S_j}}\\
&\qquad + B_i \lrp{\US^{-1}\frac{d}{dt}X_i\lrp{\US X_0, B_j + t S_j}} M X_i\lrp{X_0, B_j}^{\top} A_i X_i\lrp{X_0, B_j}\\
&\qquad + B_i {X_i\lrp{X_0, B_j}} M \lrp{\US^{-1} \frac{d}{dt}X_i\lrp{\US X_0, B_j + t S_j}}^{\top} A_i X_i\lrp{X_0, B_j}\\
&\qquad - B_i {X_i\lrp{X_0, B_j}} M {X_i\lrp{X_0, B_j}}^{\top} A_i \lrp{\US^{-1} \frac{d}{dt}X_i\lrp{\US X_0, B_j + t S_j}}\\
\overset{(ii)}{=}&\frac{d}{dt} {X_i\lrp{X_0, B_j + t \US^{-1} S_j \US}}\\
&\qquad + B_i \lrp{\frac{d}{dt}X_i\lrp{ X_0, B_j + t \US^{-1} S_j \US}} M X_i\lrp{X_0, B_j}^{\top} A_i X_i\lrp{X_0, B_j}\\
&\qquad + B_i {X_i\lrp{X_0, B_j}} M \lrp{\frac{d}{dt}X_i\lrp{ X_0, B_j + t \US^{-1} S_j \US}}^{\top} A_i X_i\lrp{X_0, B_j}\\
&\qquad + B_i {X_i\lrp{X_0, B_j}} M {X_i\lrp{X_0, B_j}}^{\top} A_i \lrp{\frac{d}{dt}X_i\lrp{ X_0, B_j + t \US^{-1} S_j \US}}
\numberthis \label{e:t:rmlsdm:1}
\end{align*}
}

In $(i)$ above, we crucially use the following facts: 1. $B_i = b_i I$ so that $\US^{-1} B_i = B_i \US^{-1}$, 2. $X_i(\US X_0, B_j) = \US X_i(X_0, B_j)$ from \eqref{e:t:UX}, 3. $A_i = a_i \Sigma^{-1}$, so that $\US^{\top} A_i \US = A_i$, 4. $\US\US^{-1} = \US^{-1} \US = I$. $(ii)$ follows from our inductive hypothesis. The inductive proof is complete by verifying that \eqref{e:t:rmlsdm:1} exactly matches the third equation of \eqref{e:t:rmlsdm:2} when $C = \US^{-1} S \US$.


\underline{\emphh{3. $G$ and $\frac{d}{dt} G$ under random transformation of $X_0$.}}
We now verify that $G(\US X_0, B_j) = \US G(X_0, B_j)$. This is a straightforward consequence of \eqref{e:t:UX} as
\begin{align*}
& G(\US X_0, B_j)\\
=& \US X_0 \prod_{i=0}^{k} \lrp{I + M X_i(\US X_0,B_j)^T A_i X_i(\US X_0,B_j)}\\
=& \US X_0 \prod_{i=0}^{k} \lrp{I + M X_i(X_0,B_j)^T A_i X_i(X_0,B_j)}\\
=& \US G(X_0, B_j),
\numberthis \label{e:t:UG}
\end{align*}
where the second equality uses \eqref{e:t:UX}, as well as the fact that $\US^\top A_i \US = A_i$. Next, we will show that
\begin{align*}
\US^{-1} \at{\frac{d}{dt} G(\US X_0, B_j + t S_j)}{t=0} = \at{\frac{d}{dt} G(X_0, B_j + t \US^{-1} S_j \US)}{t=0}.
\numberthis \label{e:t:rmlsdm:3}
\end{align*}
To see this, we can expand
{\allowdisplaybreaks
\begin{align*}
& \US^{-1} \frac{d}{dt} G(\US X_0, B_j + t S_j)\\
=& \US^{-1} \frac{d}{dt} \lrp{\US X_0 \prod_{i=0}^{k} \lrp{I + M X_i(\US X_0,B_j+tS_j)^T A_i X_i(\US X_0,B_j+tS_j)}}\\
=& X_0 \sum_{i=0}^k \lrp{\prod_{\ell=0}^{i-1} \lrp{I + M X_\ell(\US X_0,B_j)^T A_\ell X_i(\US X_0,B_\ell)}}\\
\cdot& M \frac{d}{dt} \lrp{X_i(\US X_0,B_j+tS_j)^T A_i X_i(\US X_0,B_j)}\\
\cdot& \lrp{\prod_{\ell=i+1}^{k} \lrp{I + M X_\ell(\US X_0,B_j)^T A_\ell X_i(\US X_0,B_\ell)}}\\
\overset{(i)}{=}& X_0 \sum_{i=0}^k \lrp{\prod_{\ell=0}^{i-1} \lrp{I + M X_\ell(X_0,B_j)^T A_\ell X_\ell(X_0,B_\ell)}}\\
\cdot& M \lrp{\lrp{\US^{-1} \frac{d}{dt}X_i(\US X_0,B_j+tS_j)}^T A_i X_i(X_0,B_j) + M {X_i(X_0,B_j)}^T A_i \lrp{\US^{-1} \frac{d}{dt}X_i(\US X_0,B_j+tS_j)}}\\
\cdot& \lrp{\prod_{\ell=i+1}^{k} \lrp{I + M X_\ell( X_0,B_j)^T A_\ell X_\ell( X_0,B_\ell)}}\\
\overset{(ii)}{=}& X_0 \sum_{i=0}^k \lrp{\prod_{\ell=0}^{i-1} \lrp{I + M X_\ell(X_0,B_j)^T A_\ell X_\ell(X_0,B_\ell)}}\\
\cdot& M \lrp{\lrp{\frac{d}{dt}X_i( X_0,B_j+t \US^{-1} S_j \US)}^T A_i X_i(X_0,B_j) + M {X_i(X_0,B_j)}^T A_i \lrp{\frac{d}{dt}X_i( X_0,B_j+t\US^{-1} S_j \US)}}\\
\cdot& \lrp{\prod_{\ell=i+1}^{k} \lrp{I + M X_\ell( X_0,B_j)^T A_\ell X_\ell( X_0,B_\ell)}}\\
\overset{(iii)}{=}& \frac{d}{dt} G(X_0, B_j + t \US^{-1} S_j \US)
\end{align*}
}
In $(i)$ above, we  the following facts: 1. $X_i(\US X_0, B_j) = \US X_i(X_0, B_j)$ from \eqref{e:t:UX}, 2. $A_i = a_i \Sigma^{-1}$, so that $\US^{\top} A_i \US = A_i$, 3. $\US\US^{-1} = \US^{-1} \US = I$. $(ii)$ follows from \eqref{e:t:key_induction}. $(iii)$ is by definition of $G$.


\underline{\emphh{4. Putting everything together.}}
Let us now continue from \eqref{e:t:rmlsdm:0}. We can now plug \eqref{e:t:UG} and \eqref{e:t:rmlsdm:3} into \eqref{e:t:rmlsdm:0}:
{\allowdisplaybreaks
\begin{align*}
& \at{\frac{d}{dt} f(A, B(tS_j,j))}{t=0}\\
=& 2\E_{X_0, U} \lrb{\tr\lrp{\lrp{I-M} G(\US X_0, B_j )^\top \Sigma^{-1} \at{\frac{d}{dt} G(\US X_0, B_j + t S_j)}{t=0} \lrp{I-M}}}\\
\overset{(i)}{=}& 2\E_{X_0, U} \lrb{\tr\lrp{\lrp{I-M} G( X_0, B_j )^\top \Sigma^{-1} \at{\frac{d}{dt} G(X_0, B_j + t \US^{-1} S_j \US)}{t=0} \lrp{I-M}}}\\
=& 2\E_{X_0} \lrb{\tr\lrp{\lrp{I-M} G( X_0, B_j )^\top \Sigma^{-1} \E_{U}\lrb{\at{\frac{d}{dt} G(X_0, B_j + t \US^{-1} S_j \US)}{t=0}} \lrp{I-M}}}\\
\overset{(ii)}{=}& 2\E_{X_0} \lrb{\tr\lrp{\lrp{I-M} G( X_0, B_j )^\top \Sigma^{-1} \at{\frac{d}{dt} G(X_0, B_j + t \E_U\lrb{\US^{-1} S_j \US} )}{t=0} \lrp{I-M}}}\\
=& 2\E_{X_0} \lrb{\tr\lrp{\lrp{I-M} G( X_0, B_j )^\top \Sigma^{-1} \at{\frac{d}{dt} G(X_0, B_j + t s_j I)}{t=0} \lrp{I-M}}}\\
=& \at{\frac{d}{dt} f(A, B(ts_j I,j))}{t=0}
\end{align*}
}
where $s_j := \frac{1}{d} \tr\lrp{\Sigma^{-1/2} S_j \Sigma^{1/2}}$. In the above, $(i)$ uses 1. \eqref{e:t:UG} and \eqref{e:t:rmlsdm:3}, as well as the fact that $\US^\top \Sigma^{-1} \US = \Sigma^{-1}$. $(ii)$ uses the fact that $\at{\frac{d}{dt} G(X_0, B_j + t C)}{t=0}$ is affine in $C$. To see this, one can verify from \eqref{e:t:rmlsdm:2}, using a simple induction argument, that $\frac{d}{dt} X_i(X_0, B_j + tC)$  is affine in $C$ for all $i$. We can then verify from the definition of $G$, e.g. using similar algebra as the proof of \eqref{e:t:rmlsdm:3}, that $\frac{d}{dt} G(X_0, B_j + C)$ is affine in $\frac{d}{dt} X_i(X_0, B_j + t C)$. Thus $\E_U\lrb{G(X_0, B_j + t \US^{-1} S_j \US)} = G(X_0, B_j + t \E_U\lrb{\US^{-1} S_j \US)}$.

With this, we conclude our proof of \eqref{e:t:wfoqmdksa:3}.

\underline{\emphh{5. Proof of \eqref{e:t:wfoqmdksa:2}.}}
We will now prove \eqref{e:t:wfoqmdksa:2} for fixed but arbitrary $j$, i.e. there is some $r_j$ such that
\begin{align*}
\frac{d}{dt} f(A(t\cdot r_j \Sigma^{-1},j), B) \leq \frac{d}{dt} f(A(tR_j,j), B).
\end{align*}
The proof is very similar to the proof of \eqref{e:t:wfoqmdksa:3} that we just saw, and we will essentially repeat the same steps from Step 2-4 above. 

Since we now consider perturbations to $A$ instead of to $B$, we will need to redefine some notation: let $X_i(X, C)$ (resp $Y_i(X,C)$) to denote the value of $X_i$ (resp $Y_i$) from \eqref{e:t:XY_dynamics}, with $X_0 = X$, and $A_j = C$ (previously it was with $B_j=C$). Let $G(X, A_j + C) := X \prod_{i=0}^{i} \lrp{I + M \lrp{X_i(X, A_j + C)^T A(C, j)_i X_i(X, A_j + C)}}$, where recall that $A(C,j) := A_j + C$, and $A(C,j)_\ell := A_\ell$ for all $\ell \in \lrbb{0...k} \backslash \lrbb{j}$.

We first verify that 
\begin{align*}
&X_i(\US X_0, A_j) = \US X_i(X_0, A_j)\\
&G(\US X_0, A_j) = \US G(X_0, A_j).
\numberthis \label{e:t:UXGA}
\end{align*}
The proofs are identical to the proofs of \eqref{e:t:UX} and \eqref{e:t:UG} so we omit them. Next, we show that for all $i$,
\begin{align*}
\at{\US^{-1} \frac{d}{dt} X_i(\US X_0, A_j + t R_j)}{t=0} = \at{\frac{d}{dt} X_i(X_0, A_j + t \US^{\top} R_j \US)}{t=0}.
\numberthis \label{e:t:reoqlk:0}
\end{align*}
We establish the dynamics for the right-hand-side of \eqref{e:t:reoqlk:0}:
{\allowdisplaybreaks
\begin{align*}
& \frac{d}{dt} X_{\ell} \lrp{X_0, A_j + t C} = 0\\
& \frac{d}{dt} X_{j+1} \lrp{X_0, A_j + t C} = B_j {X_{j}\lrp{X_0, A_j}} M {X_{j}\lrp{X_0, A_j}}^{\top} C  {X_{j}\lrp{X_0, A_j}}\\
& \frac{d}{dt} X_{i+1} \lrp{X_0, A_j + t C}
= \frac{d}{dt} X_{i} \lrp{X_0, A_j + t C}\\
&\qquad\qquad\qquad\quad+ B_i \lrp{\frac{d}{dt} X_{i}\lrp{X_0, A_j + t C}} M X_{i}\lrp{X_0, A_j}^{\top} A_i  X_{i}\lrp{X_0, A_j}\\
&\qquad\qquad\qquad\quad + B_i {X_{i}\lrp{X_0, A_j}} M \lrp{\frac{d}{dt} X_{i}\lrp{X_0, A_j + tC}}^{\top} A_i  X_{i}\lrp{X_0, A_j}\\
&\qquad\qquad\qquad\quad + B_i {X_{i}\lrp{X_0, A_j}} M {X_{i}\lrp{X_0, A_j}}^{\top} A_i  \lrp{\frac{d}{dt}X_{i}\lrp{X_0, A_j+tC}}
\numberthis \label{e:t:reoqlk:1}
\end{align*}
}

Similar to \eqref{e:t:rmlsdm:4}, we show that for $i\leq j$, 
\begin{align*}
\US^{-1} \frac{d}{dt} X_{i} \lrp{\US X_0, A_j + t R_j} 
=& 0 = \US^{-1} \frac{d}{dt} X_{i} \lrp{\US X_0, A_j + t \US R_j \US}
\end{align*}
and
\begin{align*}
&\US^{-1} \frac{d}{dt} X_{j+1} \lrp{\US X_0, A_j + t R_j} \\
=& \US^{-1} B_j \US X_j( X_0, A_j) M X_j(\US X_0, A_j)^\top A_j X_j(\US X_0, A_j)\\
=& \frac{d}{dt} X_{j+1} \lrp{X_0, A_j + t \US^{\top} R_j \US}.
\end{align*}
Finally, for the inductive step, we follow identical steps leading up to \eqref{e:t:rmlsdm:1} to show that
\begin{align*}
& \US^{-1} \frac{d}{dt} X_{i+1} \lrp{\US X_0, A_j + t R_j}\\
=&\frac{d}{dt} {X_i\lrp{X_0, A_j + t \US^{\top} R_j \US}}\\
&\qquad + B_i \lrp{\frac{d}{dt}X_i\lrp{ X_0, A_j + t \US^{\top} R_j \US}} M X_i\lrp{X_0, A_j}^{\top} A_i X_i\lrp{X_0, A_j}\\
&\qquad + B_i {X_i\lrp{X_0, A_j}} M \lrp{\frac{d}{dt}X_i\lrp{ X_0, A_j + t \US^{\top} R_j \US}}^{\top} A_i X_i\lrp{X_0, A_j}\\
&\qquad + B_i {X_i\lrp{X_0, A_j}} M {X_i\lrp{X_0, A_j}}^{\top} A_i \lrp{\frac{d}{dt}X_i\lrp{ X_0, A_j + t \US^{\top} R_j \US}}
\numberthis \label{e:t:reoqlk:2}
\end{align*}
The inductive proof is complete by verifying that \eqref{e:t:reoqlk:2} exactly matches the third equation of \eqref{e:t:reoqlk:1} when $C = \US^{-1} S \US$. This concludes the proof of \eqref{e:t:reoqlk:0}.

Next, we study the time derivative of $G(\US X_0, A_j + t R_j)$ and show that 
\begin{align*}
\US^{-1} \frac{d}{dt} G(\US X_0, A_j + t R_j)=& \frac{d}{dt} G(X_0, A_j + t \US^{\top} R_j \US).
\numberthis \label{e:t:reoqlk:3}
\end{align*}
This proof differs significantly from that of \eqref{e:t:rmlsdm:3} in a few places, so we provide the whole derivation below.  By chain-rule, we can write
\begin{align*}
\US^{-1} \frac{d}{dt} G(\US X_0, A_j + t R_j) = \spadesuit + \heartsuit
\end{align*}
where
\begin{align*}
\spadesuit := \US^{-1} \frac{d}{dt} \lrp{\US X_0 \prod_{i=0}^{k} \lrp{I + M X_i(\US X_0,A_j+tR_j)^T A_i X_i(\US X_0,A_j+tR_j)}}
\end{align*}
and 
\begin{align*}
\heartsuit :=& \US^{-1} \US X_0 \lrp{ \prod_{i=0}^{j-1} \lrp{I + M X_i(\US X_0,A_j)^T A_i X_i(\US X_0,A_j)}} \\
&\qquad \cdot M X_j(\US X_0,A_j)^T R_j X_j(\US X_0,A_j) \\
&\qquad \cdot \lrp{ \prod_{i=j+1}^{k} \lrp{I + M X_i(\US X_0,A_j)^T A_i X_i(\US X_0,A_j)}}.
\end{align*}

We will separately simplify $\spadesuit$ and $\heartsuit$, and verify at the end that summing them recovers the right-hand-side of \eqref{e:t:reoqlk:3}. We begin with $\spadesuit$, and the steps are almost identical to the proof of \eqref{e:t:rmlsdm:3}.

{\allowdisplaybreaks
\begin{align*}
& \spadesuit\\
=& \US^{-1} \frac{d}{dt} \lrp{\US X_0 \prod_{i=0}^{k} \lrp{I + M X_i(\US X_0,A_j+tR_j)^T A_i X_i(\US X_0,A_j+tR_j)}}\\
= & X_0 \sum_{i=0}^k \lrp{\prod_{\ell=0}^{i-1} \lrp{I + M X_\ell(\US X_0,A_j)^T A_\ell X_i(\US X_0,A_\ell)}}\\
\cdot& M \frac{d}{dt} \lrp{X_i(\US X_0,A_j+tR_j)^T A_i X_i(\US X_0,A_j + tR_j)} \\
\cdot& \lrp{\prod_{\ell=i+1}^{k} \lrp{I + M X_\ell(\US X_0,A_j)^T A_\ell X_i(\US X_0,A_\ell)}}\\
\overset{(i)}{=}& X_0 \sum_{i=0}^k \lrp{\prod_{\ell=0}^{i-1} \lrp{I + M X_\ell(X_0,A_j)^T A_\ell X_\ell(X_0,A_\ell)}}\\
\cdot& M \lrp{\lrp{\US^{-1} \frac{d}{dt}X_i(\US X_0,A_j+tR_j)}^T A_i X_i(X_0,A_j) + M {X_i(X_0,A_j)}^T A_i \lrp{\US^{-1} \frac{d}{dt}X_i(\US X_0,A_j+tR_j)}}\\
\cdot& \lrp{\prod_{\ell=i+1}^{k} \lrp{I + M X_\ell( X_0,A_j)^T A_\ell X_\ell( X_0,A_\ell)}}\\
\overset{(ii)}{=}& X_0 \sum_{i=0}^k \lrp{\prod_{\ell=0}^{i-1} \lrp{I + M X_\ell(X_0,A_j)^T A_\ell X_\ell(X_0,A_\ell)}}\\
\cdot& M \lrp{\lrp{\frac{d}{dt}X_i( X_0,A_j+t \US^{\top} R_j \US)}^T A_i X_i(X_0,A_j) + M {X_i(X_0,A_j)}^T A_i \lrp{\frac{d}{dt}X_i( X_0,A_j+t\US^{\top} R_j \US)}}\\
\cdot& \lrp{\prod_{\ell=i+1}^{k} \lrp{I + M X_\ell( X_0,A_j)^T A_\ell X_\ell( X_0,A_\ell)}}\\
=& X_0 \sum_{i=0}^k \lrp{\prod_{\ell=0}^{i-1} \lrp{I + M X_\ell(X_0,A_j)^T A_\ell X_\ell(X_0,A_\ell)}}\\
\cdot& M \frac{d}{dt} \lrp{X_i(X_0,A_j+t\US^{\top}R_j\US )^T A_i X_i(X_0,A_j + t \US^\top R_j \US)}\\
\cdot& \lrp{\prod_{\ell=i+1}^{k} \lrp{I + M X_\ell( X_0,A_j)^T A_\ell X_\ell( X_0,A_\ell)}}
\numberthis \label{e:t:spadesuit_end}
\end{align*}
}
In $(i)$ above, we  the following facts: 1. $X_i(\US X_0, B_j) = \US X_i(X_0, B_j)$ from \eqref{e:t:UXGA}, 2. $A_i = a_i \Sigma^{-1}$, so that $\US^{\top} A_i \US = A_i$, 3. $\US\US^{-1} = \US^{-1} \US = I$. $(ii)$ follows from \eqref{e:t:reoqlk:0}. 
%The main difference between the above and the proof of \eqref {e:t:rmlsdm:3} is in $(ii)$: we replace $\US^{-1} \frac{d}{dt}X_i(\US X_0,A_j+tS_j)$ by $\frac{d}{dt}X_i( X_0,A_j+t \US^{\top} R_j \US)$ using \eqref{e:t:reoqlk:3} (instead of replacing it by $\frac{d}{dt}X_i( X_0,B_j+t \US^{-1} S_j \US)$ as was done in \eqref{e:t:rmlsdm:3} using \eqref{e:t:key_induction}).

We will now simplify $\heartsuit$.
{\allowdisplaybreaks
\begin{align*}
& \heartsuit\\
=& \US^{-1} \US X_0 \lrp{ \prod_{i=0}^{j-1} \lrp{I + M X_i(\US X_0,A_j)^T A_i X_i(\US X_0,A_j)}} \\
&\qquad \cdot M X_j(\US X_0,A_j)^T R_j X_j(\US X_0,A_j) \\
&\qquad \cdot \lrp{ \prod_{i=j+1}^{k} \lrp{I + M X_i(\US X_0,A_j)^T A_i X_i(\US X_0,A_j)}}\\
\overset{(i)}{=}& X_0 \lrp{ \prod_{i=0}^{j-1} \lrp{I + M X_i(X_0,A_j)^T A_i X_i(X_0,A_j)}} M X_j(X_0,A_j)^\top \US^\top R_j \US X_j(X_0,A_j)\\
&\qquad \cdot \lrp{ \prod_{i=j+1}^{k} \lrp{I + M X_i(X_0,A_j)^T A_i X_i(X_0,A_j)}},
\numberthis \label{e:t:heartsuit_end}
\end{align*}
}
where $(i)$ uses the fact that $X_i(\US X_0, B_j) = \US X_i(X_0, B_j)$ from \eqref{e:t:UXGA} and the fact that $A_i = a_i \Sigma^{-1}$.

By expanding $\frac{d}{dt} G(X_0, A_j + t \US^{\top} R_j \US)$, we verify that 
\begin{align*}
\frac{d}{dt} G(X_0, A_j + t \US^{\top} R_j \US) = \eqref{e:t:spadesuit_end} + \eqref{e:t:heartsuit_end}  
= \spadesuit + \heartsuit
= \US^{-1} \frac{d}{dt} G(\US X_0, A_j + t R_j),
\end{align*}
this concludes the proof of \eqref{e:t:reoqlk:3}.

\iffalse
{\allowdisplaybreaks
\begin{align*}
& \US^{-1} \frac{d}{dt} G(\US X_0, A_j + t R_j)\\
=& \US^{-1} \frac{d}{dt} \lrp{\US X_0 \prod_{i=0}^{k} \lrp{I - M X_i(\US X_0,A_j+tR_j)^T A_i X_i(\US X_0,A_j+tR_j)}}\\
&\quad + \US^{-1} \US X_0 \lrp{ \prod_{i=0}^{j-1} \lrp{I - M X_i(\US X_0,A_j)^T A_i X_i(\US X_0,A_j)}} \\
&\qquad \cdot M X_j(\US X_0,A_j)^T R_j X_j(\US X_0,A_j) \lrp{ \prod_{i=j+1}^{k} \lrp{I - M X_i(\US X_0,A_j)^T A_i X_i(\US X_0,A_j)}}\\
\overset{(i)}{=}& X_0 \sum_{i=0}^k \lrp{\prod_{\ell=0}^{i-1} \lrp{I - M X_\ell(\US X_0,A_j)^T A_\ell X_i(\US X_0,B_\ell)}}\\
\cdot& M \frac{d}{dt} \lrp{X_i(\US X_0,A_j+tR_j)^T A_i X_i(\US X_0,A_j)} \lrp{\prod_{\ell=i+1}^{k} \lrp{I - M X_\ell(\US X_0,A_j)^T A_\ell X_i(\US X_0,B_\ell)}}\\
&\qquad + X_0 \lrp{ \prod_{i=0}^{j-1} \lrp{I - M X_i(X_0,A_j)^T A_i X_i(X_0,A_j)}} M X_j(X_0,A_j)^\top \US^\top R_j \US X_j(X_0,A_j)\\
&\qquad \cdot \lrp{ \prod_{i=j+1}^{k} \lrp{I - M X_i(X_0,A_j)^T A_i X_i(X_0,A_j)}}\\
\overset{(ii)}{=}& X_0 \sum_{i=0}^k \lrp{\prod_{\ell=0}^{i-1} \lrp{I - M X_\ell(X_0,A_j)^T A_\ell X_\ell(X_0,B_\ell)}}\\
\cdot& M \lrp{\lrp{\US^{-1} \frac{d}{dt}X_i(\US X_0,A_j+tR_j)}^T A_i X_i(X_0,A_j) + M {X_i(X_0,A_j)}^T A_i \lrp{\US^{-1} \frac{d}{dt}X_i(\US X_0,A_j+tR_j)}}\\
\cdot& \lrp{\prod_{\ell=i+1}^{k} \lrp{I - M X_\ell( X_0,A_j)^T A_\ell X_\ell( X_0,B_\ell)}}\\
&\qquad + X_0 \lrp{ \prod_{i=0}^{j-1} \lrp{I - M X_i(X_0,A_j)^T A_i X_i(X_0,A_j)}} M X_j(X_0,A_j)^\top \US^\top R_j \US X_j(X_0,A_j)\\
&\qquad \cdot \lrp{ \prod_{i=j+1}^{k} \lrp{I - M X_i(X_0,A_j)^T A_i X_i(X_0,A_j)}}\\
\overset{(iii)}{=}& X_0 \sum_{i=0}^k \lrp{\prod_{\ell=0}^{i-1} \lrp{I - M X_\ell(X_0,A_j)^T A_\ell X_\ell(X_0,B_\ell)}}\\
\cdot& M \lrp{\lrp{\frac{d}{dt}X_i( X_0,A_j+t \US^{\top} R_j \US)}^T A_i X_i(X_0,A_j) + M {X_i(X_0,A_j)}^T A_i \lrp{\frac{d}{dt}X_i(\US X_0,A_j+t\US^{\top} R_j \US)}}\\
\cdot& \lrp{\prod_{\ell=i+1}^{k} \lrp{I - M X_\ell( X_0,A_j)^T A_\ell X_\ell( X_0,B_\ell)}}\\
&\qquad + X_0 \lrp{ \prod_{i=0}^{j-1} \lrp{I - M X_i(X_0,A_j)^T A_i X_i(X_0,A_j)}} M X_j(X_0,A_j)^\top \US^\top R_j \US X_j(X_0,A_j)\\
&\qquad \cdot \lrp{ \prod_{i=j+1}^{k} \lrp{I - M X_i(X_0,A_j)^T A_i X_i(X_0,A_j)}}\\
\overset{(iv)}{=}& \frac{d}{dt} G(X_0, A_j + t \US^{\top} R_j \US)
\end{align*}
}
In $(ii)$ above, we  the following facts: 1. $X_i(\US X_0, B_j) = \US X_i(X_0, B_j)$ from \eqref{e:t:UX}, 2. $A_i = a_i \Sigma^{-1}$, so that $\US^{\top} A_i \US = A_i$, 3. $\US\US^{-1} = \US^{-1} \US = I$. $(iii)$ follows from \eqref{e:t:reoqlk:0}. $(iv)$ is because the preceding line is exactly $\frac{d}{dt} G(X_0, A_j + t \US^{\top} R_j \US)$.
\fi

%The key difference from \eqref{e:t:rmlsdm:3} are in steps $(i)$ and $(iii)$. In $(i)$, the chain-rule produces an extra term due to the appearance of $A_j$ directly in $G$ itself (in contrast $G$ only depends on $B_j$ via $X_i$'s). In $(iii)$, we replace $\US^{-1} \frac{d}{dt}X_i(\US X_0,B_j+tS_j)$ by $\frac{d}{dt}X_i( X_0,B_j+t \US^{\top} R_j \US)$ using \eqref{e:t:reoqlk:3} (instead of replacing it by $\frac{d}{dt}X_i( X_0,B_j+t \US^{-1} S_j \US)$ as was done in \eqref{e:t:rmlsdm:3} using \eqref{e:t:key_induction}).

The remainder of the proof is similar to what was done in \eqref{e:t:rmlsdm:0} in Step 4:
\begin{align*}
& \at{\frac{d}{dt} f(A(tR_j, j), B}{t=0}\\
=& 2\E_{X_0, U} \lrb{\tr\lrp{\lrp{I-M} G(\US X_0, A_j )^\top \Sigma^{-1} \at{\frac{d}{dt} G(\US X_0, A_j + t R_j)}{t=0} \lrp{I-M}}}\\
\overset{(i)}{=}& 2\E_{X_0,U} \lrb{\tr\lrp{\lrp{I-M} G( X_0, A_j )^\top \Sigma^{-1} \at{\frac{d}{dt} G(X_0, A_j + t \US^{\top} R_j \US)}{t=0} \lrp{I-M}}}\\
\overset{(ii)}{=}& 2\E_{X_0} \lrb{\tr\lrp{\lrp{I-M} G( X_0, A_j )^\top \Sigma^{-1} \at{\frac{d}{dt} G(X_0, A_j + t \E_U\lrb{\US^{\top} R_j \US} )}{t=0} \lrp{I-M}}}\\
=& 2\E_{X_0} \lrb{\tr\lrp{\lrp{I-M} G( X_0, A_j )^\top \Sigma^{-1} \at{\frac{d}{dt} G(X_0, A_j + t \cdot r_j \Sigma^{-1})}{t=0} \lrp{I-M}}}\\
=& \at{\frac{d}{dt} f(A(t\cdot r_j \Sigma^{-1},j), B)}{t=0},
\end{align*}
where $r_j := \frac{1}{d} \tr\lrp{\Sigma^{1/2} R_j \Sigma^{1/2}}$. In the above, $(i)$ uses 1. \eqref{e:t:UXGA} and \eqref{e:t:reoqlk:3}, as well as the fact that $\US^\top \Sigma^{-1} \US = \Sigma^{-1}$. $(ii)$ uses the fact that $\at{\frac{d}{dt} G(X_0, A_j + t C)}{t=0}$ is affine in $C$. To see this, one can verify using a simple induction argument, that $\frac{d}{dt} X_i(X_0, A_j + tC)$  is affine in $C$ for all $i$. We can then verify from the definition of $G$, e.g. using similar algebra as the proof of \eqref{e:t:reoqlk:3}, that $\frac{d}{dt} G(X_0, A_j + C)$ is affine in $\frac{d}{dt} X_i(X_0, A_j + t C)$ and $C$. Thus $\E_U\lrb{G(X_0, A_j + t \US^{\top} R_j \US)} = G(X_0, A_j + t \E_U\lrb{\US^{\top} R_j \US})$.

This concludes the proof of \eqref{e:t:wfoqmdksa:2}, and hence of the whole theorem.





\subsection{Equivalence under permutation} 

\begin{lemma}
\label{l:bar}
Consider the same setup as \autoref{t:L_layer_P_0}. Let $A = \lrbb{A_i}_{i=0}^k$, with $A_i = a_i \Sigma^{-1}$. Let 
\begin{align}
    f(A) := f \lrp{ \left\{ Q_i = \begin{bmatrix}
A_i & 0 \\ 
0 & 0
\end{bmatrix}, P_i = \begin{bmatrix}
0_{d\times d} & 0 \\ 
0 & 1 
\end{bmatrix}\right\}_{i=0}^k}.
\end{align}
Let $i,j \in \lrbb{0, \dots, k}$ be any two arbitrary indices, and let $\tilde{A}_i = A_j$, $\tilde{A}_j = A_i$, and let $\tilde{A}_{\ell} = A_\ell$ for all $\ell \in \lrbb{0,\dots ,k} \backslash \lrbb{i,j}$. Then $f(A) = f(\tilde{A})$
\end{lemma}
\begin{proof}
Following the same setup leading up to \eqref{e:t:oiemgrkwfdlw:0} in the proof of \autoref{t:L_layer_P_0}, we verify that the in-context loss is 
\begin{align*}
f(A) = \E \lrb{\tr\lrp{\lrp{I-M} G(X_0,A)^\top \Sigma^{-1} G(X_0,A)\lrp{I-M}}}
\end{align*}
where $G(X_0,A) := X_0 \prod_{\ell=0}^{k} \lrp{I + M X_0^T A_\ell X_0}$.

Consider any fixed index $\ell$. We will show that
\begin{align*}
\lrp{I + M X_0^T A_{\ell} X_0} \lrp{I + M X_0^T A_{\ell+1} X_0}
= \lrp{I + M X_0^T A_{\ell+1} X_0} \lrp{I + M X_0^T A_{\ell} X_0}.
\end{align*}
The lemma can then be proven by repeatedly applying the above, so that indices of $A_i$ and $A_j$ are swapped.

To prove the above equality, 
\begin{align*}
& \lrp{I + M X_0^T A_{\ell} X_0} \lrp{I + M X_0^T A_{\ell+1} X_0}\\
=& I + M X_0^T A_{\ell} X_0 + M X_0^T A_{\ell+1} X_0 + M X_0^T A_{\ell} X_0M X_0^T A_{\ell+1} X_0\\
=& I + M X_0^T A_{\ell} X_0 + M X_0^T A_{\ell+1} X_0 + M X_0^T a_{\ell} \Sigma^{-1} X_0M X_0^T a_{\ell+1} \Sigma^{-1} X_0\\
=& I + M X_0^T A_{\ell} X_0 + M X_0^T A_{\ell+1} X_0 + M X_0^T a_{\ell+1} \Sigma^{-1} X_0M X_0^T a_{\ell} \Sigma^{-1} X_0\\
=& \lrp{I + M X_0^T A_{\ell+1} X_0} \lrp{I + M X_0^T A_{\ell} X_0}.
\end{align*}
This concludes the proof. Notice that we crucially used the fact that $A_\ell$ and $A_{\ell+1}$ are the same matrix up to scaling.
\end{proof}

\section{Auxiliary Lemmas}

\subsection{Proof of \autoref{lem:express} (Equivalence to Preconditioned Gradient Descent)}
 \label{pf:express}
Consider fixed samples $\tx{1}, \dots, \tx{n}$, and fixed $\wstar$. Let $P=\lrbb{P_i}_{i=0}^k,Q=\lrbb{Q_i}_{i=0}^k$ denote fixed weights. Let $Z_i$ evolve as described in \eqref{eq:recursion}. Let $X_i$ denote the first $d$ rows of $Z_k$ (under \eqref{eq:sparse_attention}, $X_i=X_0$ for all $I$) and let $Y_i$ denote the $(d+1)^{th}$ row of $Z_i$. Let $g(x,y,k) : \R^d \times \R \times \mathbb{Z} \to \R$ be a function defined as follows: let $x^{n+1} = x$ and let $y^{n+1}_0 = y$, then $g(x,y,k) := y^{n+1}_k$. Note that $y^{n+1}_k = \lrb{Y_k}_{n+1}$.

We verify that, under \eqref{eq:sparse_attention}, the formula for updating $\ty{n+1}_k$ is given by
\begin{align*}
Y_{k+1}  = Y_{k}  - \frac{1}{n} Y_{k} M X_{0}^\top A_k X_{0}.
\end{align*}
where $M$ is a mask given by $\begin{bmatrix}
I & 0 \\0 & 0
\end{bmatrix}$. We can verify the following facts

\begin{enumerate}
\item $g(x,y,k) = g(x,0,k) + y$. To see this, notice first that for all $i \in \lrbb{1,\dots,n}$, 
$$\ty{i}_{k+1} = \ty{i}_{k} - \frac{1}{n} \sum_{j=1}^{n} {\tx{i}}^T A_k \tx{j} \ty{j}_k.$$ 
In other words, $\ty{i}_k$ does not depend on $\ty{n+1}_t$ for any $t$.  Next, for $\ty{n+1}_k$ itself, 
$$\ty{n+1}_{k+1} = \ty{n+1}_{k} - \frac{1}{n} \sum_{j=1}^{n} {\tx{n+1}}^T A_k \tx{j} \ty{j}_k,$$
which depends on $y^{n+1}_k$ only additively. We can verify under a simple induction that $g(x,y,k+1) - y = g(x,y,k) -y$.

\item $g(x,0,k)$ is linear in $x$. To see this, notice first that for $j \neq n+1$, $\ty{j}_k$ is does not depend on $\tx{n+1}_t$ for all $t,j,k$. Consequently, the update formula for $\ty{n+1}_{k+1}$ depends only linearly on $\tx{n+1}$ and $\ty{n+1}_k$. Finally, $\ty{n+1}_0 = 0$ is linear in $x$, so the conclusion follows by induction.
\end{enumerate}

With these two facts in mind, we verify that for each $k$, there exists a $\theta_k\in \R^d$, such that
\begin{align*}
g(x,y,k) = g(x,0,k) + y = \lin{\theta_k, x} + y
\end{align*}
for all $x,y$. It follows from definition that $g(x,y,0) = y$, so that $\lin{\theta_0,x} = g(x,y,0) - y = 0$, so that $\theta_0 = 0$. 

We now turn our attention to the third crucial fact: for all $i$, 
\begin{align*}
g(\tx{i},\ty{i},k) = \ty{i}_k = \lin{\theta_k, \tx{i}} + \ty{i}
\end{align*}
To see this, suppose that we let $\tx{n+1} := \tx{i}$ for some $i\in 1,\dots,n$. Then
\begin{align*}
& \ty{i}_{k+1} = \ty{i}_{k} - \frac{1}{n} \sum_{j=1}^{n} {\tx{i}}^T A_k \tx{j} \ty{j}_k\\
& \ty{n+1}_{k+1} = \ty{n+1}_{k} - \frac{1}{n} \sum_{j=1}^{n} {\tx{n+1}}^T A_k \tx{j} \ty{j}_k,
\end{align*}

thus $\ty{i}_{k+1} = \ty{n+1}_{k+1}$ if $\ty{i}_{k} = \ty{n+1}_{k}$, and the induction proof is completed by noting that $\ty{i}_{0} = \ty{n+1}_{0}$ by definition. Let $\bar{X} \in R^{d\times n}$ be the matrix whose columns are $\tx{1} ,\dots, \tx{n}$, leaving out $\tx{n+1}$. Let $\bar{Y}_k \in \R^{1\times n}$ denote the vector of $\ty{1}_k,\dots,\ty{n}_k$. Then it follows that
\begin{align*}
\bar{Y}_k = \bar{Y}_0 + \theta_k^T \bar{X}.
\end{align*}

Using the above fact, the update formula for $\ty{n+1}_k$ can be written as 
\begin{align*}
\ty{n+1}_{k+1} =& \ty{n+1}_{k} - \frac{1}{n} \lin{A_k X^\top Y_k, \tx{n+1}}\\
\Rightarrow \qquad 
\lin{\theta_{k+1}, \tx{n+1}} 
=& \lin{\theta_k, \tx{n+1}} - \frac{1}{n} \lin{A_k \bar{X} \lrp{\bar{X}^T \theta_k + \bar{Y}_0}, \tx{n+1}}\\
=& \lin{\theta_k, \tx{n+1}} - \frac{1}{n} \lin{A_k \bar{X} \lrp{\bar{X}^T \lrp{\theta_k + \wstar}}, \tx{n+1}}
\end{align*}

Since the choice of $\tx{n+1}$ is arbitrary, we get the more general update formula 
\begin{align*}
\theta_{k+1} = \theta_k - \frac{1}{n} A_k \bar{X} \bar{X}^T 
\lrp{\theta_k + \wstar}.
\end{align*}
We can treat $A_k$ as a preconditioner. Let $f(\theta):== \frac{1}{2n} \lrp{\theta+\wstar}^T \bar{X} \bar{X}^T (\theta+\wstar)$, then
\begin{align*}
\theta_{k+1} = \theta_k - \frac{1}{n} A_k \nabla f(\theta).
\end{align*}
Finally, let $\wgd_k := - \theta_k$. We verify that $f(-w) = R_{\wstar}(w)$, so that 
\begin{align*}
\wgd_{k+1} = \wgd_k - \frac{1}{n} A_k \nabla R_{\wstar}(\wgd_k).
\end{align*}

We also verify that for any $\tx{n+1}$, the prediction of $\ty{n+1}_k$ is 
\begin{align*}
g\lrp{\tx{n+1}, \ty{n+1}, k} = \ty{n+1} -\lin{\theta, \tx{n+1}} = \ty{n+1} + \lin{\wgd_k, \tx{n+1}}.
\end{align*}

This concludes the proof.



 
\subsection{Reformulating the in-context loss}
In this section, we will develop a re-formulation in-context loss, defined in \eqref{def:ICL linear}, in a more convenient form (see \autoref{l:icl_trace_form}).

For the entirety of this section, we assume that the transformer parameters $\lrbb{P_i,Q_i}_{i=0}^k$ are of the form defined in \eqref{eq:full_attention}, which we reproduce below for ease of reference:
\begin{align}
P_i = \begin{bmatrix}
B_i & 0 \\ 
0 & 1
\end{bmatrix}, \quad Q_i = \begin{bmatrix}
A_i & 0 \\ 
0 & 0
\end{bmatrix}.
\end{align}
Recall the update dynamics in \eqref{eq:recursion}, which we reproduce below:
\begin{align*}
Z_{i+1} = Z_{i} + \frac{1}{n}  P Z_i \aa Z_i^\top Q Z_i,
\numberthis \label{e:dynamics_Z}
\end{align*}
where $M$ is a mask matrix given by $M := \begin{bmatrix}I_{n\times n} & 0 \\ 0 & 0\end{bmatrix}$. Let $X_k \in \R^{d\times n+1}$ denote the first $d$ rows of $Z_k$ and let $Y_k \in \R^{1\times n+1}$ denote the $(d+1)^{th}$ (last) row of $Z_k$. Then the dynamics in \eqref{e:dynamics_Z} is equivalent to
\begin{align*}
& X_{i+1} = X_i + \frac{1}{n} B_i X_i M X_i^T A_i X_i\\
& Y_{i+1} = Y_i + \frac{1}{n}Y_i M X_i^T A_i X_i.
\numberthis \label{e:dynamics_XY}
\end{align*}

We present below an equivalent form for the in-context loss from \eqref{def:ICL linear}:

\begin{lemma}
\label{l:icl_trace_form}
Let $p_x$ and $p_w$ denote distributions over $\R^d$. Let $\tx{1},\dots,\tx{n+1} \overset{iid}{\sim} p_x$ and $\wstar \sim p_w$. Let $Z_0\in \R^{d+1\times n+1}$ be as defined in \eqref{d:Z_0}: 
\begin{align*}
Z_0 = \begin{bmatrix}
\tx{1} & \tx{2} & \cdots & \tx{n} &\tx{n+1} \\ 
\ty{1} & \ty{2} & \cdots &\ty{n}& 0
\end{bmatrix} \in \R^{(d+1) \times (n+1)}.
\end{align*}
Let $Z_k$ denote the output of the $(k-1)^{th}$ layer of the linear transformer (as defined in \eqref{e:dynamics_Z}, initialized at $Z_0$). Let $f\left(\{P_i, Q_i\}^{k}_{i=0}\right)$ denote the in-context loss defined in \eqref{def:ICL linear}, i.e.
\begin{align*}
f\left(\{P_i, Q_i\}^{k}_{i=0}\right) = \E_{(Z_0,\wstar)} \Bigl[ \left( [Z_{k}]_{(d+1),(n+1)} + \wstar^\top \tx{n+1}  \right)^2\Bigr].
\numberthis \label{e:old_icl}
\end{align*}

Let $\overline{Z}_0$ be defined as 
\begin{align*}
\overline{Z}_0 = \begin{bmatrix}
\tx{1} & \tx{2} & \cdots & \tx{n} &\tx{n+1} \\ 
\ty{1} & \ty{2} & \cdots &\ty{n}& \ty{n+1}
\end{bmatrix} \in \R^{(d+1) \times (n+1)},
\end{align*}
where $\ty{n+1} = \lin{\wstar, \tx{n+1}}$. Let $\overline{Z}_k$ denote the output of the $(k-1)^{th}$ layer of the linear transformer (as defined in \eqref{e:dynamics_Z}, initialized at $\overline{Z}_0$). Assume $\lrbb{P_i,Q_i}_{i=0}^k$ be of the form in \eqref{eq:full_attention}. Then the loss in \eqref{def:ICL linear} has the equivalent form
\begin{align*}
f\left(\{A_i, B_i\}^{k}_{i=0}\right) := f\left(\{P_i, Q_i\}^{k}_{i=0}\right) = \E_{(\overline{Z}_0,\wstar)} \lrb{\tr\lrp{\lrp{I-M}\overline{Y}_{k+1}^\top \overline{Y}_{k+1}\lrp{I-M}}},
\end{align*}
where $\overline{Y}_{k+1}\in\R^{1\times n+1}$ is the $(d+1)^{th}$ row of $\overline{Z}_k$.

\end{lemma}


Before proving \autoref{l:icl_trace_form}, we first establish an intermediate result (\autoref{l:additive_dependence_yn+1} below). To facilitate discussion, let us define a function $F_{X}\lrp{\lrbb{A_i,B_i}_{i=0}^k, X_0,Y_0}$ and $F_{Y}\lrp{\lrbb{A_i,B_i}_{i=0}^k, X_0,Y_0}$ to be the outputs, after $k$ layers of linear transformers respectively. I.e.
\begin{align*}
& F_{X}\lrp{\lrbb{A_i,B_i}_{i=0}^k, X_0,Y_0} = X_{k+1}\\
& F_{Y}\lrp{\lrbb{A_i,B_i}_{i=0}^k, X_0,Y_0} = Y_{k+1},
\end{align*}
as defined in \eqref{e:dynamics_XY}, given initialization $X_0,Y_0$.

We now prove a useful lemma showing that $\lrb{Y_0}_{n+1}=\ty{n+1}$ influences $X_i,Y_i$ in a very simple manner:
\begin{lemma}
\label{l:additive_dependence_yn+1}
Let $X_i,Y_i$ follow the dynamics in \eqref{e:dynamics_XY}. Then
\begin{enumerate}
\item $\lrb{X_i}$ is are independent of $\lrb{Y_0}_{n+1}$.
\item For $j\neq n+1$, $\lrb{Y_i}_{j}$ is independent of $\lrb{Y_0}_{n+1}$. 
\item $\lrb{Y_i}_{n+1}$ depends additively on $\lrb{Y_0}_{n+1}$.
\end{enumerate}

In other words, for $C := \lrb{0,0,0,\dots,,0,c} \in \R^{1\times (n+1)}$,
\begin{align*}
1:\ & F_{X}\lrp{\lrbb{A_i,B_i}_{i=0}^k, X_0,Y_0 + C} = F_{X}\lrp{\lrbb{A_i,B_i}_{i=0}^k, X_0,Y_0} \\
2 + 3:\ & F_{Y}\lrp{\lrbb{A_i,B_i}_{i=0}^k, X_0,Y_0 + C} = F_{Y}\lrp{\lrbb{A_i,B_i}_{i=0}^k, X_0,Y_0} + C
\end{align*}
\end{lemma}
\begin{proof}[Proof of \autoref{l:additive_dependence_yn+1}]
The first and second items follows directly from observing that the dynamics for $X_i$ and $Y_i$ in \eqref{e:dynamics_XY} do not involve $\lrb{Y_i}_{n+1}$, due to the effect of $M$.

The third item again uses the fact that $\lrb{Y_{i+1} - Y_i}_{n+1}$ does not depend on $\lrb{Y_{i}}_{n+1}$.
\end{proof}

We are now ready to prove \autoref{l:icl_trace_form}

\begin{proof}[Proof of \autoref{l:icl_trace_form}]
Let $Z_0$, $Z_k$, $\overline{Z}_0$, $\overline{Z}_k$ be as defined in the lemma statement. Let $\overline{X}_k$ and $\overline{Y}_k$ denote first $d$ rows and last row of $\overline{Z}_k$. Then by \autoref{l:additive_dependence_yn+1}, $\overline{X}_{k+1} = X_{k+1}$ and $\overline{Y}_{k+1} = Y_{k+1} + \begin{bmatrix}0 & 0 & \cdots & 0 & \lin{\wstar, \tx{n+1}}\end{bmatrix}$. Therefore, \eqref{e:old_icl} is equivalent to
\begin{align*}
& \E_{(\overline{Z}_0,\wstar)} \Bigl[ \left( [\overline{Z}_{k+1}]_{(d+1),(n+1)}  \right)^2\Bigr]\\
=& \E_{(\overline{Z}_0,\wstar)} \Bigl[ \left( [\overline{Y}_{k+1}]_{(n+1)}  \right)^2\Bigr]\\
=& \E_{(\overline{Z}_0,\wstar)} \lrb{\lrn{\lrp{I-M}\overline{Y}_{k+1}^\top}^2}\\
=& \E_{(\overline{Z}_0,\wstar)} \lrb{\tr\lrp{\lrp{I-M}\overline{Y}_{k+1}^\top \overline{Y}_{k+1}\lrp{I-M}}}.
\end{align*}
This concludes the proof. 
\end{proof}



 

\section{Additional experimental results}
\label{s:additional_plots}

In this section, we present a few addition experimental results. We first present in \autoref{fig:appendix_full} a visualization of learned weights $A_0,A_1,A_2$ for the setting of \autoref{t:L_layer_P_identity}.
One can see that the weight pattern matches the stationary point analyzed in \autoref{t:L_layer_P_identity}; hence, combining \autoref{fig:B0_B1} and \autoref{fig:appendix_full}, we corroborate our  results from \autoref{t:L_layer_P_identity}. Interestingly, it appears that the transformer implements a tiny gradient step using $X_0$ (as $A_0$ is small), and a large gradient step using $X_2$ (as $A_2$ is large). We believe that this is due to $X_2$ being better-conditioned than $X_1$, due to the effects of $B_0,B_1$. 
\begin{figure}[H] 
\centering
 \begin{subfigure}{0.3\textwidth}
\centering
\includegraphics[width=\textwidth]{camera-ready-figs/rotation_demonstration_adam_A0.pdf}
\caption{Visualization of $\Sigma^{1/2} A_0 \Sigma^{1/2}$} 
\label{fig:A0_PQ_app}
\end{subfigure}  
\begin{subfigure}{0.3\textwidth}
\centering
\includegraphics[width=\textwidth]{camera-ready-figs/rotation_demonstration_adam_A1.pdf}
\caption{Visualization of $\Sigma^{1/2} A_1 \Sigma^{1/2}$} 
\label{fig:A1_PQ_app}
\end{subfigure}  
\begin{subfigure}{0.3\textwidth}
\centering
\includegraphics[width=\textwidth]{camera-ready-figs/rotation_demonstration_adam_A2.pdf}
\caption{Visualization of $\Sigma^{1/2} A_2 \Sigma^{1/2}$} 
\label{fig:A2_PQ_app}
\end{subfigure}  
\caption{Visualization of learned weights  for the setting of \autoref{t:L_layer_P_identity}. One can see that the weight pattern matches the stationary point analyzed in \autoref{t:L_layer_P_identity}.   }
\label{fig:appendix_full}
\end{figure}

We next present some additional experiments that investigates the properties of the learned predictors of various algorithms.  First, we plot  the \textbf{test losses against the number of examples provided in the prompt} (``the number of ICL examples''). We compare four different algorithms: (i) the predictor learned by a three-layered of linear transformer, (ii) three steps of GD, (iii)  three steps of preconditioned GD, and (iv) the ordinary least-squared solution (OLS). For GD and preconditioned GD, the optimal stepsizes are found by gridsearch. For preconditioned GD, preconditioner is fixed to be $\Sigma^{-1}$ for comparison. In all cases, the dimension $d=5$, and for each $N$, the linear Transformer is trained using Adam. The result is presented in \autoref{fig:variable-N-plot}.
\begin{figure}[H]
\centering 
\includegraphics[width=0.35\textwidth]{camera-ready-figs/3-step-variable-N-plot.pdf}
\caption{Test loss comparison between (i) the predictor learned by a three-layered of linear transformer, (ii) three steps of GD, (iii)  three steps of preconditioned GD, and (iv) the ordinary least-squared solution (OLS).} 
\label{fig:variable-N-plot} 
\end{figure}
Lastly, in \autoref{fig:variable-L-plot},  we plot the \textbf{test losses against the number of layer $L$} (or the number of steps in the case of gradient-based algorithms).
For $L=1,2,3,4$, we compare between (i) the predictor learned by $L$-linear transformer and (i) $L$-steps of GD, (ii) $L$-steps of preconditioned GD. Again, the optimal stepsize is found by gridsearch, and for preconditioned GD, the preconditioner is fixed to be $\Sigma^{-1}$. In all cases, the dimension $d=5$, and context length $N=20$. The linear transformer is trained with Adam.
\begin{figure}[H]
\centering 
\includegraphics[width=0.4\textwidth]{camera-ready-figs/variable-L-plot.pdf}
\caption{Test loss comparison between (i) the predictor learned by a $L$-layered linear transformer and (i) $L$-steps of GD, (ii) $L$-steps of preconditioned GD, for $L=1,2,3,4$. } 
\label{fig:variable-L-plot} 
\end{figure}

\newpage 

\end{document}