\usepackage{color}
\usepackage{graphicx}
\usepackage{epstopdf}
\usepackage{amsmath}
\usepackage{amssymb}
\usepackage{hyperref}
%\usepackage{amsthm}
\usepackage[english]{babel}
\usepackage{cite}
\usepackage{rotfloat}
\usepackage{mathtools}
%\usepackage[justification=centering]{caption}
%\usepackage{breqn}
\usepackage{amsmath}
\usepackage{makecell}
\usepackage{algorithm,algorithmic}
\usepackage{multirow}
\usepackage{subfigure}
\usepackage{booktabs}
\usepackage{colortbl}
\usepackage{multirow}% http://ctan.org/pkg/multirow
\usepackage{hhline}% http://ctan.org/pkg/hhline
\usepackage{stfloats}% <-- added
\usepackage{multicol}
\usepackage[justification=centering,font=small]{caption}


\DeclarePairedDelimiter\ceil{\lceil}{\rceil}
\DeclarePairedDelimiter\floor{\lfloor}{\rfloor}

\newsavebox{\foobox}
\newcommand{\slantbox}[2][.3]
{%
	\mbox
	{%
		\sbox{\foobox}{#2}%
		\hskip\wd\foobox
		\pdfsave
		\pdfsetmatrix{1 0 #1 1}%
		\llap{\usebox{\foobox}}%
		\pdfrestore
	}%
}
\newcommand{\setA}{\mathbb{A}}

\definecolor{kugray5}{RGB}{224,224,224}

\usepackage[normalem]{ulem}
\newcommand\rsout{\bgroup\markoverwith
	{\textcolor{red}{\rule[0.5ex]{2pt}{0.8pt}}}\ULon}


%% ALLOW TO DIVIDE AN ALGORITHM INTO TWO PAGES
\makeatletter
\newenvironment{breakablealgorithm}
{% \begin{breakablealgorithm}
	\begin{center}
		\refstepcounter{algorithm}% New algorithm
		\hrule height.8pt depth0pt \kern2pt% \@fs@pre for \@fs@ruled
		\renewcommand{\caption}[2][\relax]{% Make a new \caption
			{\raggedright\textbf{\ALG@name~\thealgorithm} ##2\par}%
			\ifx\relax##1\relax % #1 is \relax
			\addcontentsline{loa}{algorithm}{\protect\numberline{\thealgorithm}##2}%
			\else % #1 is not \relax
			\addcontentsline{loa}{algorithm}{\protect\numberline{\thealgorithm}##1}%
			\fi
			\kern2pt\hrule\kern2pt
		}
	}{% \end{breakablealgorithm}
		\kern2pt\hrule\relax% \@fs@post for \@fs@ruled
	\end{center}
}
\makeatother

% Color definition
\newcommand{\blue}[1]{\textcolor{blue}{#1}}
\newcommand{\red}[1]{\textcolor{red}{#1}}
\newcommand{\green}[1]{\textcolor{green}{#1}}
\newcommand{\orange}[1]{\textcolor{orange}{#1}}

% vector
\newcommand{\myvec}[1]{\ensuremath{\begin{pmatrix}#1\end{pmatrix}}}

% Algorithmic modifications
\makeatletter
\newcommand{\ALOOP}[1]{\ALC@it\algorithmicloop\ #1%
	\begin{ALC@loop}}
	\newcommand{\ENDALOOP}{\end{ALC@loop}\ALC@it\algorithmicendloop}
\renewcommand{\algorithmicrequire}{\textbf{Input:}}
\renewcommand{\algorithmicensure}{\textbf{Output:}}
\newcommand{\algorithmicbreak}{\textbf{break}}
\newcommand{\BREAK}{\STATE \algorithmicbreak}
\makeatother

% Folder of all figures
\usepackage{etoolbox}
\let\mybibitem\bibitem
\renewcommand{\bibitem}[1]{%
	\ifstrequal{#1}{nature}
	{\color{blue}\mybibitem{#1}}
	{\color{black}\mybibitem{#1}}%
}

\graphicspath{ {Figures/} }

% To define the theorems, definitions, remark, lemma, corollary, proof
\newtheorem{definition}{Definition}
\newtheorem{theorem}{Theorem}
\newtheorem{remark}{Remark}
\newtheorem{lemma}{Lemma}
\newtheorem{corollary}{Corollary}
\newtheorem{proof}{Proof}
\renewcommand\theproof{\unskip}
\newcommand{\epr}{\hfill\(\Box\)}

% for 0.1cm and 0.25cm space
\newcommand{\dis}{\hspace{0.1cm}}
\DeclareCaptionLabelSeparator{periodspace}{.\quad}

% for format of Fig. in IEEE
\renewcommand{\figurename}{Fig.}
% to be able to place 2 figure consecutive at the top of page
\captionsetup{font=footnotesize,labelsep=periodspace,singlelinecheck=false}
\captionsetup[sub]{font=footnotesize,singlelinecheck=true}
\addto\captionsenglish{\renewcommand{\figurename}{Fig.}}
% to be able to palce appendices right after conclusion
\interdisplaylinepenalty=2500 

\newcommand\nbthis{\addtocounter{equation}{1}\tag{\theequation}}

% math notations and notations
\newcommand{\norm}[1]{\left\lVert#1\right\rVert} % ||.||
\newcommand{\eq}[1]{\begin{align*}#1\end{align*}} % equation
\newcommand{\eqn}[1]{\begin{align}#1\end{align}} % equation
\newcommand{\nt}[1]{\left(#1\right)} % ()
\newcommand{\nv}[1]{\left[#1\right]} % []
\newcommand{\nn}[1]{\left\{#1\right\}} % {}
\newcommand{\abs}[1]{\left|#1\right|} % ||
\newcommand{\nb}{\numberthis}
\newcommand{\tr}[1]{\mathrm{trace}\left(#1\right)} % ||
\newcommand{\diag}[1]{\mathrm{diag}\left\{#1\right\}} % ||

% real and imag part
\newcommand{\re}[1]{\mathfrak{R}{\left(#1\right)}}
\newcommand{\im}[1]{\mathfrak{I}{\left(#1\right)}}

% Prob, PDF, CDF
%\newcommand{\prob}{\mathbb{P}} 
\newcommand{\cdf}{\mathbf{\textit{F}}} 
\newcommand{\pdf}{\mathbf{\textit{f}}} 
\newcommand{\mean}[1]{\mathbb{E} \left\{#1\right\}}

% Problem name
\newcommand{\phybrid}{\left(\mathrm{P}_{\mathrm{hybrid}} \right)}
\newcommand{\pnhybrid}{\left(\mathrm{P-n}_{\mathrm{hybrid}} \right)}
\newcommand{\pnbhybrid}{\left(\bar{\mathrm{Pn}}_{\mathrm{hybrid}} \right)}
\newcommand{\ppassive}{\left(\mathrm{P}_{\mathrm{passive}} \right)}
\newcommand{\pactive}{\left(\mathrm{P}_{\mathrm{active}} \right)}
\newcommand{\pnpassive}{\left(\mathrm{P-n}_{\mathrm{passive}} \right)}
\newcommand{\pnuniform}{\left(\mathrm{P-n}_{\mathrm{uniform}} \right)}
\newcommand{\pnhybridLB}{\left(\mathrm{P-n}_{\mathrm{fixed, uni.}} \right)}
\newcommand{\pdyn}{\left(\mathrm{P}_{\mathrm{dyn}} \right)}
\newcommand{\pdyna}{\left(\mathrm{P1}_{\mathrm{dyn}} \right)}
\newcommand{\prob}{\left(\mathrm{P} \right)}
\newcommand{\probn}{\left(\mathrm{Pn} \right)}
\newcommand{\pupdate}{\left(\mathrm{P_{update}}\right)}
% Parameters

% Matrix
\newcommand{\mQ}{\textbf{\textit{Q}}}
\newcommand{\mR}{\textbf{\textit{R}}}
\newcommand{\mH}{\textbf{\textit{H}}} 
\newcommand{\mhsorted}{\underline{\mH}}
\newcommand{\mA}{\textbf{\textit{A}}}
\newcommand{\mW}{\textbf{\textit{W}}}
\newcommand{\mP}{\textbf{\textit{P}}}
\newcommand{\mI}{\textbf{\textit{I}}}
\newcommand{\mT}{\textbf{\textit{T}}}
\newcommand{\mB}{\textbf{\textit{B}}}
\newcommand{\mC}{\textbf{\textit{C}}}
\newcommand{\mD}{\textbf{\textit{D}}}
\newcommand{\mX}{\textbf{\textit{X}}}
\newcommand{\mY}{\textbf{\textit{Y}}}
\newcommand{\mG}{\textbf{\textit{G}}}
\newcommand{\mF}{\textbf{\textit{F}}}
\newcommand{\mU}{\textbf{\textit{U}}}
\newcommand{\mV}{\textbf{\textit{V}}}
\newcommand{\mE}{\textbf{\textit{E}}}
\newcommand{\mHr}{\textbf{\textit{H}}_r} 
\newcommand{\mHt}{\textbf{\textit{H}}_t} 
\newcommand{\mIr}{\textbf{\textit{I}}_{N_r}}

\def\mathbi#1{\textbf{\em #1}}

% max value
\newcommand{\Xm}{X_{max}} 
\newcommand{\Am}{A_{max}} 
\newcommand{\Bm}{B_{min}} 


%% Set
\newcommand{\setC}{\mathbb{C}} 
\newcommand{\setR}{\mathbb{R}}
%\newcommand{\setA}{\mathcal{A}} 
\newcommand{\setD}{\mathcal{D}} 
\newcommand{\setANt}{\setA^{N_t}} 
\newcommand{\setAtld}{\tilde{\setA}^{N_t}}
\newcommand{\setN}{\mathcal{N}(\vc)}

%% List
\newcommand{\ltabu}{\mathcal{L}}
\newcommand{\leta}{\mathcal{L}_{\eta}}
\newcommand{\lphi}{\mathcal{L}_{\phi}}

%% Probability
\newcommand{\Pe}{\mathcal{P}}
\newcommand{\Pne}{\bar{\mathcal{P}}}

%% Vector
\newcommand{\vxb}{\textbf{\textit{x}}^{\star}}
\newcommand{\vc}{\textbf{\textit{c}}}
\newcommand{\vci}{\textbf{\textit{c}}_{\nn{1}}}
\newcommand{\ve}{\textbf{\textit{e}}} 
\newcommand{\vs}{\textbf{\textit{s}}}
\newcommand{\vx}{\textbf{\textit{x}}}
\newcommand{\vy}{\textbf{\textit{y}}}
\newcommand{\vr}{\textbf{\textit{r}}}
\newcommand{\vqn}{\textbf{\textit{q}}^T_n}
\newcommand{\vpnT}{\textbf{\textit{p}}^T_n}
%\newcommand{\vpnT}{\textbf{\textit{r}}^T_n}
\newcommand{\vpd}{\textbf{\textit{p}}_d}
\newcommand{\vrdc}{\check{\textbf{\textit{r}}}_d}
\newcommand{\vhdc}{\check{\textbf{\textit{h}}}_d}
\newcommand{\vhc}{\check{\textbf{\textit{h}}}}
\newcommand{\vrd}{\textbf{\textit{r}}_d}
\newcommand{\vrds}{\textbf{\textit{r}}_{d^{\star}}}
\newcommand{\vpn}{\textbf{\textit{p}}_n}
\newcommand{\vv}{\textbf{\textit{v}}}
\newcommand{\vn}{\textbf{\textit{n}}}
\newcommand{\vu}{\textbf{\textit{u}}}
\newcommand{\vz}{\textbf{\textit{z}}} 
\newcommand{\vh}{\textbf{\textit{h}}} 
\newcommand{\vhd}{\textbf{\textit{h}}_d}
\newcommand{\vq}{\textbf{\textit{q}}}
\newcommand{\vb}{\textbf{\textit{b}}}
\newcommand{\vw}{\textbf{\textit{w}}}
\newcommand{\va}{\textbf{\textit{a}}}
\newcommand{\vd}{\textbf{\textit{d}}}
\newcommand{\vt}{\textbf{\textit{t}}}
\newcommand{\vg}{\textbf{\textit{g}}}

%% Elements of vectors/matrice
\newcommand{\zbn}{\bar{z}_n}
\newcommand{\rni}{r_{n,i}}
\newcommand{\rnj}{r_{n,j}}
\newcommand{\pnj}{p_{n,j}}
\newcommand{\qni}{q_{n,i}}
\newcommand{\qnj}{q_{n,j}}

%% Other notations
\newcommand{\dx}{\Delta\textbf{\textit{x}}}

\newcommand{\mpeta}[1]{\phi_{#1} (\vx)} % Metric
\newcommand{\mpetai}[1]{\phi_{#1} (\vx_{[i]})} % Metric

\newcommand{\smt}{\sigma_t^2} % Power
\newcommand{\smv}{\sigma_v^2} % Power
\newcommand{\smn}{\sigma_n^2} % Power

\newcommand{\pbar}{\bar{\Phi}}
\newcommand{\ptilde}{\tilde{\Phi}}

\newcommand{\bPhi}{\boldsymbol{\Phi}}
\newcommand{\bUpsilon}{\boldsymbol{\Upsilon}}
\newcommand{\bTheta}{\boldsymbol{\Theta}}
\newcommand{\bPsi}{\boldsymbol{\Psi}}

\newcommand{\Pa}{P_{\mathrm{a}}} 
\newcommand{\Pamax}{P_{\mathrm{a}}^{\mathrm{max}}}
\newcommand{\Pbs}{P_{\mathrm{BS}}}
\newcommand{\an}{\alpha_n}
\newcommand{\ans}{\alpha_n^{\star}}