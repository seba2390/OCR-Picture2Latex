%!TEX root = 00-NVGF.tex

%%%%%%%%%%%%%%%%%%%%%%%%%%%%%%%%%%%%%%%%%%%%%%%%%%%%%%%%%%%%%%%%%%%%%%%%%%%%%%%%
%%%%                                                                        %%%%
%%%%                              CONCLUSIONS                               %%%%
%%%%                                                                        %%%%
%%%%%%%%%%%%%%%%%%%%%%%%%%%%%%%%%%%%%%%%%%%%%%%%%%%%%%%%%%%%%%%%%%%%%%%%%%%%%%%%
%%%% sec:conclusions
%%%%%%%%%%%%%%%%%%%%

The objective of this work was to study the role of frequency creation in GSP problems. To do so, nonlinear activation functions (which theoretical findings suggest give rise to frequency creation) are replaced by NVGFs, which are also capable of creating frequencies, but in a linear manner. In this way, frequency creation was decoupled from the nonlinear nature of activation functions. Numerical experiments show that the main driver of improved performance is frequency creation and not necessarily the nonlinear nature of GCNNs. As future work, we are interested in extending this frequency analysis to non-GSP related problems such as semi-supervised node classification or graph classification problems, which require a careful definition of a notion of frequency.
\ifundefined{arXiv}
\else We discussed the caveats of extending this framework to include semi-supervised learning and graph classification problems, relating to the need of defining an appropriate notion of graph frequency. This opens up an interesting area of future research. We also note that it would be possible to use shift-variant filters to leverage this framework when using CNNs.\fi