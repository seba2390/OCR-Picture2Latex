% Auxiliary commands (typically for editing)
%   Fernando Gama fgama@seas.upenn.edu

%%%%%%%%%%%%%%%%%%%%
%%                %%
%%    COLORING    %%
%%                %%
%%%%%%%%%%%%%%%%%%%%

%   Check if the package has already been loaded, if it hasn't load it
\makeatletter
    \@ifpackageloaded{xcolor}{}{\usepackage{xcolor}}
\makeatother
%

\newcommand{\red}   [1] {{\color{red} #1}}
\newcommand{\blue}  [1] {{\color{blue} #1}}
\newcommand{\green} [1] {{\color[rgb]{0.10,0.50,0.10}#1}}
\newcommand{\gray}  [1] {{\color[rgb]{0.80,0.80,0.80}#1}}
\newcommand{\black} [1] {{\color{black} #1}}

%%%%%%%%%%%%%%%%%%%%
%%                %%
%%    STYLING     %%
%%                %%
%%%%%%%%%%%%%%%%%%%%

\makeatletter
    \@ifpackageloaded{needspace}{}{\usepackage{needspace}}
\makeatother

\newcommand{\titledParagraph}[1]{\needspace{1\baselineskip}\noindent {\bf #1}}

%%%%%%%%%%%%%%%%%%%%%
%%                 %%
%%    CITATIONS    %%
%%                 %%
%%%%%%%%%%%%%%%%%%%%%
%% Normally, these acts as placeholders for citations

\newcommand{\PhCiteX}{{\color[cmyk]{0.76, 0.34, 0.21, 0.00}[X]}} % Just a placeholder
\newcommand{\PhCiteText}[1]{{\color[cmyk]{0.76, 0.34, 0.21, 0.00}[#1]}} % Allows to be 
    % more specific on the citation that wants to be added.
    
%%%%%%%%%%%%%%%%
%%            %%
%%    MATH    %%
%%            %%
%%%%%%%%%%%%%%%%

% Uppercase Greek
\newcommand{\uppercaseGreek}[1]{%
    \begingroup
    \ucmathlist\MakeUppercase{#1}%
    \endgroup
}

\newcommand{\ucmathlist}{%
    \def\alpha{\mathrm{A}}%
    \def\beta{\mathrm{B}}%
    \let\gamma=\Gamma
    \let\delta=\Delta
    \def\epsilon{\mathrm{E}}%
    \def\varepsilon{\mathrm{E}}%
    \def\zeta{\mathrm{Z}}%
    \def\eta{\mathrm{H}}%
    \let\theta=\Theta
    \let\vartheta=\Theta
    \def\iota{\mathrm{I}}%
    \def\kappa{\mathrm{K}}%
    \let\lambda=\Lambda
    \def\mu{\mathrm{M}}%
    \def\nu{\mathrm{N}}%
    \let\xi=\Xi
    \let\pi=\Pi
    \let\varpi=\Pi
    \def\rho{\mathrm{P}}%
    \def\varrho{\mathrm{P}}%
    \let\sigma=\Sigma
    \def\tau{\mathrm{T}}%
    \let\upsilon=\Upsilon
    \let\phi=\Phi
    \let\varphi=\Phi
    \def\chi{\mathrm{X}}%
    \let\psi=\Psi
    \let\omega=\Omega
}

%%%% REQUIRES amsthm PACKAGE %%%%
%   Check if the package has already been loaded, if it hasn't load it
\makeatletter
    \@ifpackageloaded{amsthm}{}{
        \let\proof\relax
        \let\endproof\relax
        \usepackage{amsthm}
    }
\makeatother
%
\theoremstyle{plain}
    \newtheorem{theorem}{Theorem}
    \newtheorem{proposition}[theorem]{Proposition}
    \newtheorem{lemma}[theorem]{Lemma}
    \newtheorem{corollary}[theorem]{Corollary}
\theoremstyle{definition}
    \newtheorem{definition}{Definition}
    \newtheorem{remark}{Remark}

%% THIS IS A MACRO TO RENEWTHEOREMS %%
\makeatletter
\def\renewtheorem#1{%
    \expandafter\let\csname#1\endcsname\relax
    \expandafter\let\csname c@#1\endcsname\relax
    \gdef\renewtheorem@envname{#1}
    \renewtheorem@secpar
}
\def\renewtheorem@secpar{\@ifnextchar[{\renewtheorem@numberedlike}{\renewtheorem@nonumberedlike}}
\def\renewtheorem@numberedlike[#1]#2{\newtheorem{\renewtheorem@envname}[#1]{#2}}
\def\renewtheorem@nonumberedlike#1{  
    \def\renewtheorem@caption{#1}
    \edef\renewtheorem@nowithin{\noexpand\newtheorem{\renewtheorem@envname}{\renewtheorem@caption}}
    \renewtheorem@thirdpar
}
\def\renewtheorem@thirdpar{\@ifnextchar[{\renewtheorem@within}{\renewtheorem@nowithin}}
\def\renewtheorem@within[#1]{\renewtheorem@nowithin[#1]}
\makeatother
%% END OF MACRO %%