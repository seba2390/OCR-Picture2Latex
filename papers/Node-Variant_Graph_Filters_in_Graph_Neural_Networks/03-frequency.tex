%!TEX root = 00-NVGF.tex

%%%%%%%%%%%%%%%%%%%%%%%%%%%%%%%%%%%%%%%%%%%%%%%%%%%%%%%%%%%%%%%%%%%%%%%%%%%%%%%%
%%%%                                                                        %%%%
%%%%                           FREQUENCY ANALYSIS                           %%%%
%%%%                                                                        %%%%
%%%%%%%%%%%%%%%%%%%%%%%%%%%%%%%%%%%%%%%%%%%%%%%%%%%%%%%%%%%%%%%%%%%%%%%%%%%%%%%%
%%%% sec:freq
%%%%%%%%%%%%%

The GMD $\mtS \in \fdR^{N \times N}$ can be used to define a spectral representation of the graph signal $\vcx \in \fdR^{N}$ \cite{Sandryhaila2014-DSPGfreq}. Since the graph is undirected, assume that $\mtS$ is symmetric so that it can be diagonalized by an orthonormal basis of eigenvectors $\{\vcv_{i}\}_{i=1}^N$, where $\vcv_{i} \in \fdR^{N}$ and $\mtS \vcv_{i} = \lambda_{i} \vcv_{i}$, with $\lambda_{i} \in \fdR$ being the corresponding eigenvalue. Then, it holds that $\mtS = \mtV \diag(\vclambda) \mtV^{\Tr}$, where the $i^{\text{th}}$ column of $\mtV$ is $\vcv_{i}$ and where $\vclambda \in \fdR^{N}$ is given by $[\vclambda]_{i} = \lambda_{i}$ for $i=1,\ldots,N$. We assume throughout this paper that the eigenvalues are distinct, which is typically the case for random connected graphs.

The spectral representation of a graph signal $\vcx$ with respect to its underlying graph support described by $\mtS$ is given by
% eq:GFT
\begin{equation} \label{eq:GFT}
    \vctx = \vcV^{\Tr} \vcx
\end{equation}
%
where $\vctx \in \fdR^{N}$, see \cite{Sandryhaila2014-DSPGfreq}. The spectral representation $\vctx$ of the graph signal $\vcx$ contains the coordinates of representing $\vcx$ on the eigenbasis $\{\vcv_{i}\}_{i=1}^N$ of the support matrix $\mtS$. The resulting vector $\vctx$ is known as the frequency response of the signal $\vcx$. The $i^{\text{th}}$ entry $[\vctx]_{i} = \vcv_{i}^{\Tr} \vcx = \sctx_{i} \in \fdR$ of the frequency response $\vctx$ measures how much the $i^{\text{th}}$ eigenvector $\vcv_{i}$ contributes to the signal $\vcx$. The operation in \eqref{eq:GFT} is called the GFT, and thus the frequency response $\vctx$ is often referred to as the GFT of the signal $\vcx$.

The GFT offers an alternative representation of the graph signal $\vcx$ that takes into account the graph structure in $\mtS$. The effect of a filter can be characterized in the spectral domain by computing the GFT of the output. For instance, when considering a LSIGF, the spectral representation of the output $\vcy = \mtfnH^{\text{lsi}}(\mtS) \vcx$ is
% eq:GFToutputLSI
\begin{equation} \label{eq:GFToutputLSI}
    \vcty = \mtV^{\Tr} \sum_{k=0}^{K} h_{k} \mtS^{k} \vcx = \sum_{k=0}^{K} h_{k} \diag(\vclambda^{k}) \vctx = \diag(\vcth) \vctx
\end{equation}
%
where the eigendecomposition of $\mtS$, the GFT of $\vcx$, and the fact that $\mtS^{k} = \mtV \diag(\vclambda^{k}) \mtV^{\Tr}$ were all used. Note that $\vclambda^{k}$ is a shorthand notation that means $[\vclambda^{k}]_{i} = \lambda_{i}^{k}$. The vector $\vcth \in \fdR^{N}$ is known as the frequency response of the filter and its $i^{\text{th}}$ entry is given by
% eq:freqResponse
\begin{equation} \label{eq:freqResponse}
    [\vcth]_{i} = \fnth(\lambda_{i}) = \sum_{k=0}^{K} h_{k} \lambda_{i}^{k}
\end{equation}
%
where $\fnth:\fdR \to \fdR$ is a polynomial defined by the set of filter taps $\{h_{k}\}_{k=0}^K$. The function $\fnth(\cdot)$ depends only on the filter coefficients and not on the specific graph on which it is applied, and thus is valid for all graphs. The effect of filtering on a specific graph comes from instantiating $\fnth(\cdot)$ on the specific eigenvalues of that graph. The function $\fnth(\cdot)$ is denoted as the frequency response as well, and it will be clear from context whether we refer to the function $\fnth(\cdot)$ or to the vector $\vcth$ given in \eqref{eq:freqResponse}\ifundefined{arXiv}\else\
that stems from evaluating $\fnth(\cdot)$ at each of the $N$ eigenvalues $\lambda_{i}$\fi
. Note that since $\fnth(\cdot)$ is an analytic function, it can be applied to the square matrix $\mtS$ so that $\fnth(\mtS) = \mtfnH^{\text{lsi}}(\mtS)$.

In the case of the LSIGF, it is observed from \eqref{eq:GFToutputLSI} that the $i^{\text{th}}$ entry of the frequency response of the output $[\vcty]_{i} = \scty_{i}$ is given by the elementwise multiplication
% eq:convolutionTheorem
\begin{equation} \label{eq:convolutionTheorem}
    \scty_{i} = \fnth(\lambda_{i}) \sctx_{i}.
\end{equation}
%
This implies that the frequency response of the output $\scty_{i}$ is the elementwise multiplication of the frequency response of the filter $\fnth(\lambda_{i})$ and the frequency response of the input $\sctx_{i}$.
This makes \eqref{eq:convolutionTheorem} the analogue of the convolution theorem for graph signals. Therefore, oftentimes the LSIGF in \eqref{eq:LSIfilter} is called graph convolution. It is observed from \eqref{eq:convolutionTheorem} that LSIGFs are capable of learning any type of frequency response (low-pass, high-pass, etc.), but that they are not able to create frequencies, i.e., if $\sctx_{i} = 0$, then $\scty_{i} = 0$.

Unlike discrete-time signals, the frequency response of the LSIGF is not computed in the same manner as the frequency response of graph signals. More specifically, the frequency response of the filter $\vcth$ can be directly computed from the filter taps $\vch$ by means of a Vandermonde matrix $\mtLambda \in \fdR^{N \times (K+1)}$ given by $[\mtLambda]_{ik} = \lambda_{i}^{k-1}$ as follows:
% eq:GFTfilter
\begin{equation} \label{eq:GFTfilter}
    \vcth = \mtLambda \vch.
\end{equation}
%
This implies that\ifundefined{arXiv} \else\ when processing graph signals, the graph filter cannot be uniquely represented by a graph signal \cite[Th. 5]{Sandryhaila2013-DSPG}, and thus the concept of impulse response is no longer valid. Furthermore, \fi
the graph convolution is not \ifundefined{arXiv}commutative.\else symmetric, i.e., the filter and the graph signal do not commute. Finally, it is interesting to note that, when $\mtS$ is the adjacency matrix of a directed cycle and $K=N-1$, then $\mtLambda = \mtV^{\Hr}$ and the GFT of the signal and of the filter taps (in time, known as the impulse response) becomes equivalent, as expected.\fi

When considering NVGFs, as in \eqref{eq:NVGF}, the convolution theorem \eqref{eq:convolutionTheorem} no longer holds. Instead, the frequency response of the output is given in the following proposition.

%%%%%%%%%%%%%%%%%%%%%%%%%%%%%%%%%%%%%%%%
%%%%            PROPOSITION         %%%%  prop:GFToutputNVGF
%%%%%%%%%%%%%%%%%%%%%%%%%%%%%%%%%%%%%%%%
%%
\begin{proposition}[Frequency response of NVGF] \label{prop:GFToutputNVGF}
Let $\vcy = \mtfnH^{\text{nv}}(\mtS) \vcx = \sum_{k=0}^{K} \diag(\vch^{(k)}) \mtS^{k} \vcx$ be the output of an arbitrary NVGF characterized by some filter tap matrix $\mtH \in \fdR^{N \times(K+1)}$. Then, the frequency response $\vcty = \mtV^{\Tr}\vcy$ of the output is given by
% eq:GFToutputNVGF
\begin{equation} \label{eq:GFToutputNVGF}
    \vcty = \mtV^{\Tr} \big( \mtV \circ (\mtH \mtLambda^{\Tr}) \big) \vctx
\end{equation}
%
where $\circ$ denotes elementwise product of matrices.
\end{proposition}
%%
%%%%       End of PROPOSITION       %%%%
%%%%%%%%%%%%%%%%%%%%%%%%%%%%%%%%%%%%%%%%

It is immediately observed that NVGFs are capable of generating new frequency content, even though they are linear.

%%%%%%%%%%%%%%%%%%%%%%%%%%%%%%%%%%%%%%%%
%%%%             COROLLARY          %%%%  cor:NVGFnewFreq
%%%%%%%%%%%%%%%%%%%%%%%%%%%%%%%%%%%%%%%%
%%
\begin{corollary} \label{cor:NVGFnewFreq}
If the matrix $\mtV^{\Tr} \big( \mtV \circ (\mtH \mtLambda^{\Tr}) \big)$ is not diagonal, then the output exhibits frequency content that is not present in the input.
\end{corollary}
%%
%%%%        End of COROLLARY        %%%%
%%%%%%%%%%%%%%%%%%%%%%%%%%%%%%%%%%%%%%%%

\ifundefined{arXiv}
\else
As an example, consider the case where the input is given by a single frequency signal, i.e., $\vcx = \vcv_{t}$ for some $t\in\{1,\ldots,N\}$ so that $\vctx = \vce_{t}$ where $[\vce_{t}]_{i} = 1$ if $i=t$ and $[\vce_{t}]_{i} = 0$ otherwise. This is a signal that has a single frequency component. Yet, the $i^{\text{th}}$ entry of the frequency response of the NVGF output is $\scty_{i} = \sum_{j=1}^{N} \fnth_{j}(\lambda_{t}) v_{ji}v_{jt}$, where $v_{ij} = [\mtV]_{ij}$ and $\fnth_{j}$ is the frequency response obtained from using the $K+1$ coefficients at node $\lmv_{j}$. Unless $\fnth_{j}(\lambda_{t})$ is a constant for all $j$, i.e., $\fnth_{j}(\lambda_{t}) = \scth_{t}$, the $i^{\text{th}}$ entry of the GFT of the output is nonzero even when the $i^{\text{th}}$ entry of the GFT of the input is zero. This also shows that by carefully designing the filter taps for each node, the frequencies that are allowed to be created can be chosen, analyzed, and understood.
\fi

To finalize the frequency analysis of NVGFs, we establish their Lipschitz continuity to changes in the underlying graph support, as decribed by the matrix $\mtS$. In what follows, we denote the spectral norm of a matrix $\mtA$ by $\|\mtA\|_2$.

%%%%%%%%%%%%%%%%%%%%%%%%%%%%%%%%%%%%%%%%
%%%%              THEOREM           %%%%  thm:stability
%%%%%%%%%%%%%%%%%%%%%%%%%%%%%%%%%%%%%%%%
%%
\begin{theorem}[Lipschitz continuity of the NVGF with respect to $\mtS$] \label{thm:stability}
Let $\stG$ and $\sthG$ be two graphs with $N$ nodes, described by GMDs $\mtS \in \fdR^{N \times N}$ and $\mthS \in \fdR^{N \times N}$, respectively. Let $\mtH \in \fdR^{N \times(K+1)}$ be the coefficients of any NVGF. Given a constant $\sceps > 0$, if $\|\mthS - \mtS\|_{2} \leq \sceps$, it holds that
% eq:stability
\begin{equation} \label{eq:stability}
    \Big\| \big( \mtfnH^{\text{nv}}(\mthS) - \mtfnH^{\text{nv}}(\mtS) \big) \vcx \Big\|_{2} \leq \sceps C \sqrt{N} ( 1 + 8N) \| \vcx \|_{2} + \bigOh(\sceps^{2})
\end{equation}
%
where $\mtfnH^{\text{nv}}(\mtS)$ and $\mtfnH^{\text{nv}}(\mthS)$ are the NVGF on $\mtS$ and $\mthS$, respectively, and where $C$ is the Lipschitz constant of the frequency responses at each node, i.e., $|\fnth_{t}(\lambda_{j})-\fnth_{t}(\lambda_{i})| \leq C |\lambda_{j} - \lambda_{i}|$ for all $i,j,t \in \{1,\ldots,N\}$.
\end{theorem}
%%
%%%%       End of THEOREM       %%%%
%%%%%%%%%%%%%%%%%%%%%%%%%%%%%%%%%%%%%%%%

Theorem~\ref{thm:stability} establishes the Lipschitz continuity of the NVGF filter with respect to the support matrix $\mtS$ (Lipschitz continuity with respect to the input $\vcx$ is immediately given for bounded filter taps) as long as the graphs $\mtS$ and $\mthS$ are similar, i.e., $\sceps \ll 1$. The bound is proportional to this difference, $\sceps$, and to the shape of the frequency responses at each node through the Lipschitz constant $C$. It also depends on the number of nodes $N$, but it is fixed for given graphs with the same number of nodes. In short, Theorem~\ref{thm:stability} gives mild guarantees on the expected performance of the NVGF across a wide range of graphs $\mthS$ that are close to the graph $\mtS$. %on which the NVGF was trained.