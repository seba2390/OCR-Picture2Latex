\documentclass[a4paper,10pt,twoside]{cpc-hepnp}
\usepackage{CJK,upgreek,fancyhdr}
\usepackage{multicol}
\usepackage{graphicx}
\usepackage{booktabs}
\usepackage{amssymb,bm,mathrsfs,bbm,amscd}
\usepackage[tbtags]{amsmath}
\usepackage{lastpage}
\usepackage{cases}
\usepackage{subfigure}
\usepackage{epstopdf}
\usepackage{bm}

%\usepackage[pdftex,colorlinks,linkcolor=black,anchorcolor=black,citecolor=black,urlcolor=black]{hyperref}
%Revised for arXiv
\usepackage[colorlinks,linkcolor=black,anchorcolor=black,citecolor=black,urlcolor=black]{hyperref}
%\input{My_commands/My_commands}
%frequently-used particles

\newcommand{\jpsi}{$J/\psi$~}
\newcommand{\jpsiwos}{$J/\psi$}
\newcommand{\psip}{$\psi(3686)$~}
\newcommand{\psipwos}{$\psi(3686)$}

%frequently-used channels
\newcommand{\eetoee}{$e^+e^- \to e^+e^-$~}
\newcommand{\eetomumu}{$e^+e^- \to \mu^+\mu^-$~}
\newcommand{\eetohad}{$e^+e^- \to hadrons$~}
\newcommand{\eetodigam}{$e^+e^- \to \gamma\gamma$~}
\newcommand{\eetotwogam}{$e^+e^- \to e^+e^-X$~}
\newcommand{\eetoll}{$e^+e^- \to l^+l^-$~}
\newcommand{\eetomumugam}{$e^+e^- \to \mu^+\mu^-\gamma$~}
\newcommand{\eetoeewos}{$e^+e^- \to e^+e^-$}
\newcommand{\eetomumuwos}{$e^+e^- \to \mu^+\mu^-$}
\newcommand{\eetohadwos}{$e^+e^- \to hadrons$}
\newcommand{\eetodigamwos}{$e^+e^- \to \gamma\gamma$}
\newcommand{\eetotwogamwos}{$e^+e^- \to e^+e^-X$}
\newcommand{\eetollwos}{$e^+e^- \to l^+l^-$}
\newcommand{\eetomumugamwos}{$e^+e^- \to \mu^+\mu^-\gamma$}

%frequently-used branching ratios
\newcommand{\Bjpsitoll}{B(J/\psi \to l^+l^-)}

%frequently-used parameters
\newcommand{\Tw}{\Gamma_{\rm tot}}
\newcommand{\Ew}{\Gamma_{ee}}
\newcommand{\Mw}{\Gamma_{\mu\mu}}
\newcommand{\Lw}{\Gamma_{ll}}
\newcommand{\EwMMw}{\Gamma_{ee}\Gamma_{\mu\mu}}
\newcommand{\EwDMw}{\Gamma_{ee}/\Gamma_{\mu\mu}}
\newcommand{\LwDTw}{\Gamma_{ll}/\Gamma_{\rm tot}}
\newcommand{\EwMEwDTw}{\Gamma_{ee}\Gamma_{ee}/\Gamma_{\rm tot}}
\newcommand{\EwMMwDTw}{\Gamma_{ee}\Gamma_{\mu\mu}/\Gamma_{\rm tot}}
\newcommand{\LwMLwDTw}{\Gamma_{ll}\Gamma_{ll}/\Gamma_{\rm tot}}
\newcommand{\Es}{\sigma}

%frequently-used math expressions
\newcommand{\csth}{\cos\theta}
\newcommand{\snth}{\sin\theta}


\begin{document}
\begin{CJK*}{UTF8}{gbsn}

%\input{Fancyhead_fancyfoot_footnotetext/Fancyhead_fancyfoot_footnotetext}
%\fancyhead[c]{\small Chinese Physics C~~~Vol. xx, No. x (201x) xxxxxx}
%Revised for arXiv
\fancyhead[c]{\small Accepted by Chinese Physics C}
\fancyfoot[C]{\small xxxxxx-\thepage}
%\footnotetext[0]{Received xx Month xxxx}
%Revised for arXiv
\footnotetext[0]{}

%\input{Title_author_address/Title_author_address}
\title{Analytic Forms for Cross Sections of Di-lepton Production \\ from $e^+e^-$ Collisions around the \jpsi Resonance \thanks{Supported by National Natural Science Foundation of China (11275211) and Istituto Nazionale di Fisica Nucleare, Italy}}

\author{Xing-Yu Zhou (周兴玉)$^{1;1)}$ \email{xyzhou@ihep.ac.cn} \quad Ya-Di Wang (王雅迪)$^{2;2)}$ \email{Y.Wang@him.uni-mainz.de} \quad Li-Gang Xia (夏力钢)$^{3;3)}$ \email{xialigang@tsinghua.edu.cn}}

\maketitle

\address
{
	$^1$ Institute of High Energy Physics, Chinese Academy of Sciences, Beijing 100049, China \\
	$^2$ Helmholtz Institute Mainz, Mainz 55128, Germany \\
	$^3$ Department of Physics, Tsinghua University, Beijing 100084, China
}


%\input{Abstract_keyword_pacs_footnotetext/Abstract_keyword_pacs_footnotetext}
\begin{abstract}
	A detailed theoretical derivation of the cross sections of \eetoee and \eetomumu around the \jpsi resonance is reported. The resonance and interference parts of the cross sections, related to \jpsi resonance parameters, are calculated. Higher-order corrections for vacuum polarization and initial-state radiation are considered. An arbitrary upper limit of radiative correction integration is involved. Full and simplified versions of analytic formulae are given with precision at the level of 0.1\% and 0.2\%, respectively. Moreover, the results obtained in the paper can be applied to the case of the \psip resonance.
\end{abstract}

\begin{keyword}
	initial-state radiation, vacuum polarization, $e^+e^-$ collision, di-lepton production, the \jpsi resonance
\end{keyword}

\begin{pacs}
	13.20.Gd, 13.66.De, 13.66.Jn, 14.40.Pq, 13.40.Hq
\end{pacs}

%The last three lines must not be deleted. They are needed by the CPC template.
%\footnotetext[0]{\hspace*{-3mm}\raisebox{0.3ex}{$\scriptstyle\copyright$}2013 Chinese Physical Society and the Institute of High Energy Physics of the Chinese Academy of Sciences and the Institute of Modern Physics of the Chinese Academy of Sciences and IOP Publishing Ltd}
%Revised out for arXiv
\footnotetext[0]{\ \\}

\begin{multicols}{2}

\section{Introduction}
%\IEEEraisesectionheading{\section{Introduction}}

\IEEEPARstart{V}{ision} system is studied in orthogonal disciplines spanning from neurophysiology and psychophysics to computer science all with uniform objective: understand the vision system and develop it into an integrated theory of vision. In general, vision or visual perception is the ability of information acquisition from environment, and it's interpretation. According to Gestalt theory, visual elements are perceived as patterns of wholes rather than the sum of constituent parts~\cite{koffka2013principles}. The Gestalt theory through \textit{emergence}, \textit{invariance}, \textit{multistability}, and \textit{reification} properties (aka Gestalt principles), describes how vision recognizes an object as a \textit{whole} from constituent parts. There is an increasing interested to model the cognitive aptitude of visual perception; however, the process is challenging. In the following, a challenge (as an example) per object and motion perception is discussed. 



\subsection{Why do things look as they do?}
In addition to Gestalt principles, an object is characterized with its spatial parameters and material properties. Despite of the novel approaches proposed for material recognition (e.g.,~\cite{sharan2013recognizing}), objects tend to get the attention. Leveraging on an object's spatial properties, material, illumination, and background; the mapping from real world 3D patterns (distal stimulus) to 2D patterns onto retina (proximal stimulus) is many-to-one non-uniquely-invertible mapping~\cite{dicarlo2007untangling,horn1986robot}. There have been novel biology-driven studies for constructing computational models to emulate anatomy and physiology of the brain for real world object recognition (e.g.,~\cite{lowe2004distinctive,serre2007robust,zhang2006svm}), and some studies lead to impressive accuracy. For instance, testing such computational models on gold standard controlled shape sets such as Caltech101 and Caltech256, some methods resulted $<$60\% true-positives~\cite{zhang2006svm,lazebnik2006beyond,mutch2006multiclass,wang2006using}. However, Pinto et al.~\cite{pinto2008real} raised a caution against the pervasiveness of such shape sets by highlighting the unsystematic variations in objects features such as spatial aspects, both between and within object categories. For instance, using a V1-like model (a neuroscientist's null model) with two categories of systematically variant objects, a rapid derogate of performance to 50\% (chance level) is observed~\cite{zhang2006svm}. This observation accentuates the challenges that the infinite number of 2D shapes casted on retina from 3D objects introduces to object recognition. 

Material recognition of an object requires in-depth features to be determined. A mineralogist may describe the luster (i.e., optical quality of the surface) with a vocabulary like greasy, pearly, vitreous, resinous or submetallic; he may describe rocks and minerals with their typical forms such as acicular, dendritic, porous, nodular, or oolitic. We perceive materials from early age even though many of us lack such a rich visual vocabulary as formalized as the mineralogists~\cite{adelson2001seeing}. However, methodizing material perception can be far from trivial. For instance, consider a chrome sphere with every pixel having a correspondence in the environment; hence, the material of the sphere is hidden and shall be inferred implicitly~\cite{shafer2000color,adelson2001seeing}. Therefore, considering object material, object recognition requires surface reflectance, various light sources, and observer's point-of-view to be taken into consideration.


\subsection{What went where?}
Motion is an important aspect in interpreting the interaction with subjects, making the visual perception of movement a critical cognitive ability that helps us with complex tasks such as discriminating moving objects from background, or depth perception by motion parallax. Cognitive susceptibility enables the inference of 2D/3D motion from a sequence of 2D shapes (e.g., movies~\cite{niyogi1994analyzing,little1998recognizing,hayfron2003automatic}), or from a single image frame (e.g., the pose of an athlete runner~\cite{wang2013learning,ramanan2006learning}). However, its challenging to model the susceptibility because of many-to-one relation between distal and proximal stimulus, which makes the local measurements of proximal stimulus inadequate to reason the proper global interpretation. One of the various challenges is called \textit{motion correspondence problem}~\cite{attneave1974apparent,ullman1979interpretation,ramachandran1986perception,dawson1991and}, which refers to recognition of any individual component of proximal stimulus in frame-1 and another component in frame-2 as constituting different glimpses of the same moving component. If one-to-one mapping is intended, $n!$ correspondence matches between $n$ components of two frames exist, which is increased to $2^n$  for one-to-any mappings. To address the challenge, Ullman~\cite{ullman1979interpretation} proposed a method based on nearest neighbor principle, and Dawson~\cite{dawson1991and} introduced an auto associative network model. Dawson's network model~\cite{dawson1991and} iteratively modifies the activation pattern of local measurements to achieve a stable global interpretation. In general, his model applies three constraints as it follows
\begin{inlinelist}
	\item \textit{nearest neighbor principle} (shorter motion correspondence matches are assigned lower costs)
	\item \textit{relative velocity principle} (differences between two motion correspondence matches)
	\item \textit{element integrity principle} (physical coherence of surfaces)
\end{inlinelist}.
According to experimental evaluations (e.g.,~\cite{ullman1979interpretation,ramachandran1986perception,cutting1982minimum}), these three constraints are the aspects of how human visual system solves the motion correspondence problem. Eom et al.~\cite{eom2012heuristic} tackled the motion correspondence problem by considering the relative velocity and the element integrity principles. They studied one-to-any mapping between elements of corresponding fuzzy clusters of two consecutive frames. They have obtained a ranked list of all possible mappings by performing a state-space search. 



\subsection{How a stimuli is recognized in the environment?}

Human subjects are often able to recognize a 3D object from its 2D projections in different orientations~\cite{bartoshuk1960mental}. A common hypothesis for this \textit{spatial ability} is that, an object is represented in memory in its canonical orientation, and a \textit{mental rotation} transformation is applied on the input image, and the transformed image is compared with the object in its canonical orientation~\cite{bartoshuk1960mental}. The time to determine whether two projections portray the same 3D object
\begin{inlinelist}
	\item increase linearly with respect to the angular disparity~\cite{bartoshuk1960mental,cooperau1973time,cooper1976demonstration}
	\item is independent from the complexity of the 3D object~\cite{cooper1973chronometric}
\end{inlinelist}.
Shepard and Metzler~\cite{shepard1971mental} interpreted this finding as it follows: \textit{human subjects mentally rotate one portray at a constant speed until it is aligned with the other portray.}



\subsection{State of the Art}

The linear mapping transformation determination between two objects is generalized as determining optimal linear transformation matrix for a set of observed vectors, which is first proposed by Grace Wahba in 1965~\cite{wahba1965least} as it follows. 
\textit{Given two sets of $n$ points $\{v_1, v_2, \dots v_n\}$, and $\{v_1^*, v_2^* \dots v_n^*\}$, where $n \geq 2$, find the rotation matrix $M$ (i.e., the orthogonal matrix with determinant +1) which brings the first set into the best least squares coincidence with the second. That is, find $M$ matrix which minimizes}
\begin{equation}
	\sum_{j=1}^{n} \vert v_j^* - Mv_j \vert^2
\end{equation}

Multiple solutions for the \textit{Wahba's problem} have been published, such as Paul Davenport's q-method. Some notable algorithms after Davenport's q-method were published; of that QUaternion ESTimator (QU\-EST)~\cite{shuster2012three}, Fast Optimal Attitude Matrix \-(FOAM)~\cite{markley1993attitude} and Slower Optimal Matrix Algorithm (SOMA)~\cite{markley1993attitude}, and singular value decomposition (SVD) based algorithms, such as Markley’s SVD-based method~\cite{markley1988attitude}. 

In statistical shape analysis, the linear mapping transformation determination challenge is studied as Procrustes problem. Procrustes analysis finds a transformation matrix that maps two input shapes closest possible on each other. Solutions for Procrustes problem are reviewed in~\cite{gower2004procrustes,viklands2006algorithms}. For orthogonal Procrustes problem, Wolfgang Kabsch proposed a SVD-based method~\cite{kabsch1976solution} by minimizing the root mean squared deviation of two input sets when the determinant of rotation matrix is $1$. In addition to Kabsch’s partial Procrustes superimposition (covers translation and rotation), other full Procrustes superimpositions (covers translation, uniform scaling, rotation/reflection) have been proposed~\cite{gower2004procrustes,viklands2006algorithms}. The determination of optimal linear mapping transformation matrix using different approaches of Procrustes analysis has wide range of applications, spanning from forging human hand mimics in anthropomorphic robotic hand~\cite{xu2012design}, to the assessment of two-dimensional perimeter spread models such as fire~\cite{duff2012procrustes}, and the analysis of MRI scans in brain morphology studies~\cite{martin2013correlation}.

\subsection{Our Contribution}

The present study methodizes the aforementioned mentioned cognitive susceptibilities into a cognitive-driven linear mapping transformation determination algorithm. The method leverages on mental rotation cognitive stages~\cite{johnson1990speed} which are defined as it follows
\begin{inlinelist}
	\item a mental image of the object is created
	\item object is mentally rotated until a comparison is made
	\item objects are assessed whether they are the same
	\item the decision is reported
\end{inlinelist}.
Accordingly, the proposed method creates hierarchical abstractions of shapes~\cite{greene2009briefest} with increasing level of details~\cite{konkle2010scene}. The abstractions are presented in a vector space. A graph of linear transformations is created by circular-shift permutations (i.e., rotation superimposition) of vectors. The graph is then hierarchically traversed for closest mapping linear transformation determination. 

Despite of numerous novel algorithms to calculate linear mapping transformation, such as those proposed for Procrustes analysis, the novelty of the presented method is being a cognitive-driven approach. This method augments promising discoveries on motion/object perception into a linear mapping transformation determination algorithm.



The \jpsi resonance is frequently referred to as a hydrogen atom for QCD, and its resonance parameters (mass $M$, total width $\Tw$, leptonic widths $\Ew$ and $\Mw$, and so on) describe the fundamental properties of the strong and electromagnetic interactions. In theory, the decay widths can be predicted by different potential \\ models \cite{potential models 1,potential models 2} and lattice QCD calculations \cite{lattice calculations}. In experiment, with results from BABAR \cite{BABAR}, CLEO \cite{CLEO} and KEDR \cite{KEDR}, determinations of these decay widths have entered a period of precision measurement.

In 2012, data samples were taken at 15 center-of-mass energy points around the \jpsi resonance with the BESIII detector \cite{BESIII detector} operated at the BEPCII collider \cite{BESIII detector}. In this energy region, BEPCII provides high luminosity and BESIII shows excellent performance, which helps us accurately measure  the cross sections of \eetoee and \eetomumuwos. To measure \jpsi decay widths, accurate theoretical formulae taking into account higher-order corrections are also needed. If one wishes to have a high-efficiency optimization procedure, it is better to have analytic expressions for the theoretical cross sections. Because the continuum parts of these cross sections do not involve \jpsi decay widths and can be evaluated precisely by Monte-Carlo generators such as the Babayaga generator \cite{BABAYAGA}, only the analytic forms for the resonance and interference parts are derived in this paper.

We will start with theoretical fundamentals on the structure function method, its applications to the cases of \eetoee and \eetomumuwos, Born cross sections and the vacuum polarization function in Section \ref{Theoretical fundamentals}. Then, we will give the definitions and resulting formulae for the resonance and interference parts of the cross sections of \eetoee and \eetomumu in Section \ref{Calculations of the resonance and interference parts}. Most of the purely mathematical derivation is given in Appendix A to make the text easier to read.



\section{Theoretical fundamentals}
\label{Theoretical fundamentals}
%\input{Theoretical_fundamentals/Theoretical_fundamentals}

\subsection{Structure function method}

Generally, initial-state radiation (ISR), final-state radiation (FSR) and their interference (ISR-FSR relation) must be considered when one makes higher-order corrections to cross sections. Here, the ISR-FSR relation includes interference of diagrams with emission of real and virtual photons between initial- and final-state particles. The suppression level of the ISR-FSR relation between the production and decay stages of heavy unstable particles is discussed in Ref. \cite{ISR-FSR relation}. According to the conclusion in Ref. \cite{ISR-FSR relation}, there is no need to take into account the ISR-FSR relation in the case of \jpsi, because it is suppressed by $\Gamma_{\rm tot}/M$ (about $3\times10^{-5}$). As for FSR, a universal calculation is impossible if one has no explicit knowledge of selection criteria, so it needs to be handled separately with a numerical method, which is outside the scope of this paper. Thus, in this paper the calculation with ISR only is presented.

The structure function method \cite{STRUCTUREFUNCTIONMETHOD} is adopted here to deal with ISR. Its fundamental formula is
\begin{equation}
	\sigma(s) = \int \int_0^X \frac{d\bar{\sigma}}{d\Omega}(s(1-x),\cos\theta) F(s,x) dx d\Omega.
	\label{Equation: Fundamental formula of structure function method}
\end{equation}
Here, $\sigma$ stands for the cross section after correction, $\frac{d\bar{\sigma}}{d\Omega}$ for the differential cross section before correction, $F$ for the radiator, $s$ for the square of the center-of-mass energy and $\theta$ for the polar angle of the positively charged final particle in the center-of-mass frame. The upper limit $X$ of the integration variable $x$ is usually set as $1-s^{\prime}_{min}/s$, where $s^{\prime}_{min}$ is the minimum of the invariant mass squared of the final-state particle system excluding the emitted photons.

The radiator $F$ adopted in this paper was first derived in Ref. \cite{Radiator 1} and slightly revised in Ref. \cite{Radiator 2}. Both documents are in Chinese, although the former has an English-language preprint (Ref. \cite{Radiator 3}). It is different from but a very good approximation of the classical one in Ref. \cite{STRUCTUREFUNCTIONMETHOD}. Its expression is
\begin{align}
	F(s,x) & = x^{v-1}v(1+\delta) \nonumber \\
	& + x^{v}\left(-v-\frac{v^2}{4}\right) + x^{v+1}\left(\frac{v}{2}-\frac{3}{8}v^2\right),
	\label{Equation: Expression of radiator.}
\end{align}
where
\begin{equation}
	\delta(v) = \frac{\alpha}{\pi}(\frac{\pi^2}{3}-\frac{1}{2})+\frac{3}{4}v+\left(\frac{9}{32}-\frac{\pi^2}{12}\right)v^2
\end{equation}
and
\begin{equation}
	v(s) = \frac{2\alpha}{\pi}\left(\ln\frac{s}{m_e^2}-1\right).
\end{equation}
Here, $\alpha$ stands for the fine structure constant and $m_e$ denotes the electron mass.

\subsection{Applications of the structure fuction meth-\\od to \eetoee and \eetomumu}
\label{Subsection: Applications of the structure fuction method to eetoee and eetomumu}

Applying the structure function method to the cases of \eetoee and \eetomumuwos, one can get
\begin{align}
	&\ \ \  \left(\frac{d\sigma}{d\Omega}\right)_{ee|\mu\mu}(s,\cos\theta) \nonumber \\
	& = \int_0^X \left(\frac{d\bar{\sigma}}{d\Omega}\right)_{ee|\mu\mu}(s(1-x),\cos\theta) F(s,x) dx,
	\label{Equation: Formula of structure function method applied to eetoee and eetomumu.}
\end{align}
%\begin{align}
%&\ \ \  \left(\frac{d\sigma}{d\Omega}\right)_{ee}(s,\cos\theta) \nonumber \\
%& = \int_0^X \left(\frac{d\bar{\sigma}}{d\Omega}\right)_{ee}(s(1-x),\cos\theta) F(s,x) dx
%\label{Equation: Formula of structure function method applied to eetoee.}
%\end{align}
%and
%\begin{align}
%&\ \ \  \left(\frac{d\sigma}{d\Omega}\right)_{\mu\mu}(s,\cos\theta) \nonumber \\
%& = \int_0^X \left(\frac{d\bar{\sigma}}{d\Omega}\right)_{\mu\mu}(s(1-x),\cos\theta) F(s,x) dx,
%\label{Equation: Formula of structure function method applied to eetomumu.}
%\end{align}
where the symbol $|$ stands for ``or",
\begin{align}
	\left(\frac{d\bar{\sigma}}{d\Omega}\right)_{ee} & = \left(\frac{d\sigma_0}{d\Omega}\right)_{ee}^{\rm S}\left|\frac{1}{1-\Pi(s)}\right|^2 \nonumber \\
	& + \left(\frac{d\sigma_0}{d\Omega}\right)_{ee}^{\rm T}\left|\frac{1}{1-\Pi(t)}\right|^2 \nonumber \\
	& + \left(\frac{d\sigma_0}{d\Omega}\right)_{ee}^{\rm STI} Re\left(\frac{1}{1-\Pi(s)}\overline{\frac{1}{1-\Pi(t)}}\right) \label{Equation: Cross sections of eetoee with vacuum polarization considered}
\end{align}
and
\begin{equation}
	\left(\frac{d\bar{\sigma}}{d\Omega}\right)_{\mu\mu} = \left(\frac{d\sigma_0}{d\Omega}\right)_{\mu\mu}^{\rm S}\left|\frac{1}{1-\Pi(s)}\right|^2. \label{Equation: Cross sections of eetomumu with vacuum polarization considered}
\end{equation}
Here, $t$ denotes the square of the 4-momentum transferred in the t channel. As for \eetoeewos, the relation between $t$ and $s$ is
\begin{equation}
	t \approx -\frac{s}{2}(1-\csth).
\end{equation}
In addition, $\left(\frac{d\sigma_0}{d\Omega}\right)_{ee}^{\rm S}$, $\left(\frac{d\sigma_0}{d\Omega}\right)_{ee}^{\rm T}$, $\left(\frac{d\sigma_0}{d\Omega}\right)_{ee}^{\rm STI}$ and $\left(\frac{d\sigma_0}{d\Omega}\right)_{\mu\mu}^{\rm S}$ are Born cross sections, and $\frac{1}{1-\Pi}$ is the vacuum polarization function. They will be discussed in the following two subsections.

\subsection{Born cross sections}

The quantities $\left(\frac{d\sigma_0}{d\Omega}\right)_{ee}^{\rm S}$, $\left(\frac{d\sigma_0}{d\Omega}\right)_{ee}^{\rm T}$ and $\left(\frac{d\sigma_0}{d\Omega}\right)_{ee}^{\rm STI}$ are the s channel part, the t channel part and the s-t interference part of the Born cross section of \eetoee $\left(\left(\frac{d\sigma_0}{d\Omega}\right)_{ee}\right)$, respectively, that is
\begin{align}
	& \ \ \  \left(\frac{d\sigma_0}{d\Omega}\right)_{ee} = \left(\frac{d\sigma_0}{d\Omega}\right)_{ee}^{\rm S} + \left(\frac{d\sigma_0}{d\Omega}\right)_{ee}^{\rm T} + \left(\frac{d\sigma_0}{d\Omega}\right)_{ee}^{\rm STI},
\end{align}
where
\begin{subequations}
	\label{Align: Born cross section of eetoee}
	\begin{align}
		\left(\frac{d\sigma_0}{d\Omega}\right)_{ee}^{\rm S} & = \frac{\alpha^2}{4s} (1+\cos^2\theta), \label{Equation: S channel part of Born cross section of eetoee} \\
		\left(\frac{d\sigma_0}{d\Omega}\right)_{ee}^{\rm T} & = \frac{\alpha^2}{2s} \frac{(1+\cos\theta)^2+4}{(1-\cos\theta)^2}, \label{Equation: T channel part of Born cross section of eetoee} \\
		\left(\frac{d\sigma_0}{d\Omega}\right)_{ee}^{\rm STI} & = - \frac{\alpha^2}{2s} \frac{(1+\cos\theta)^2}{1-\cos\theta}. \label{Equation: S-T interference part of Born cross section of eetoee}
	\end{align}
\end{subequations}
The Born cross section of \eetomumu $\left(\left(\frac{d\sigma_0}{d\Omega}\right)_{\mu\mu}\right)$ has only an s channel part $\left(\frac{d\sigma_0}{d\Omega}\right)_{\mu\mu}^{\rm S}$, which equals exactly $\left(\frac{d\sigma_0}{d\Omega}\right)_{ee}^{\rm S}$ given by Eq. (\ref{Equation: S channel part of Born cross section of eetoee}).

\subsection{Vacuum polarization function}

In Section 4 of Ref. \cite{KEDRpsip}, the distinction and relationship between the ``bare" and ``dressed" parameters of $J^{PC}=1^{--}$ resonances (for example \jpsiwos) are discussed in detail. In the discussion there, the vacuum polarization function is written as
\begin{equation}
	\frac{1}{1-\Pi(q^2)} = \frac{1}{1-\Pi_{\rm 0}(q^2)}+\Pi_{\rm R}(q^2),
	\label{Equation: The vacuum polarization function}
\end{equation}
where $\Pi_{\rm R}$ is expressed with the ``dressed" parameters $M$, $\Gamma_{\rm tot}$ and $\Gamma_{ee}$ as
\begin{equation}
	\Pi_{\rm R}(q^2) = \frac{3\Gamma_{ee}}{\alpha} \frac{q^2}{M} \frac{1}{q^2-M^2+iM\Gamma_{\rm tot}}.
	\label{Equation: Resonance part of vacuum polarization function}
\end{equation}
Here, $\Pi_{\rm R}$ stands for the contribution from the resonance itself (in our case, it is \jpsiwos), while $\Pi_{\rm 0}$ denotes contributions from other sources. Based on the lepton universality assumption, $\Gamma_{ee}$ in Eq. (\ref{Equation: Resonance part of vacuum polarization function}) can be substituted by $\sqrt{\Gamma_{ee}\Gamma_{\mu\mu}}$ in the case of \eetomumuwos.

According to Eq. (\ref{Equation: The vacuum polarization function}), $\frac{1}{1-\Pi(s)}$ and $\frac{1}{1-\Pi(t)}$ in Eq. (\ref{Equation: Cross sections of eetoee with vacuum polarization considered}) and (\ref{Equation: Cross sections of eetomumu with vacuum polarization considered}) can be expressed as
\begin{equation}
	\frac{1}{1-\Pi(s)} = \frac{1}{1-\Pi_0(s)} + \Pi_{\rm R}(s) \label{Equation: Vacuum polarization in the timelike region}
\end{equation}
and
\begin{equation}
	\frac{1}{1-\Pi(t)} = \frac{1}{1-\Pi_0(t)}. \label{Equation: Vacuum polarization in the spacelike region}
\end{equation}
No $\Pi_{\rm R}(t)$ term appears in Eq. (\ref{Equation: Vacuum polarization in the spacelike region}) because it can be safely ignored in the spacelike region. Besides, the imaginary parts of $\frac{1}{1-\Pi_0(s)}$ and $\frac{1}{1-\Pi_0(t)}$ can be safely ignored as well. Consequently, $\frac{1}{1-\Pi_0(s)}$ and $\frac{1}{1-\Pi_0(t)}$ will be regarded as real in the following section.


\section{Calculations of the resonance and interference parts}
\label{Calculations of the resonance and interference parts}
%\input{Calculations_of_the_resonance_and_interference_parts/Calculations_of_the_resonance_and_interference_parts}
\subsection{Definitions}

Considering $\left(\frac{d\bar{\sigma}}{d\Omega}\right)_{ee}$ and $\left(\frac{d\bar{\sigma}}{d\Omega}\right)_{\mu\mu}$ given by Eq. (\ref{Equation: Cross sections of eetoee with vacuum polarization considered}) and (\ref{Equation: Cross sections of eetomumu with vacuum polarization considered}) as well as $\frac{1}{1-\Pi(s)}$ and $\frac{1}{1-\Pi(t)}$ given by Eq. (\ref{Equation: Vacuum polarization in the timelike region}) and (\ref{Equation: Vacuum polarization in the spacelike region}), one can expand $\left(\frac{d\sigma}{d\Omega}\right)_{ee}$ and $\left(\frac{d\sigma}{d\Omega}\right)_{\mu\mu}$ via Eq. (\ref{Equation: Formula of structure function method applied to eetoee and eetomumu.}) into \\ many small terms. With these small terms regrouped, the resonance and interference parts of $\left(\frac{d\sigma}{d\Omega}\right)_{ee}$ and $\left(\frac{d\sigma}{d\Omega}\right)_{\mu\mu}$, namely $\left(\frac{d\sigma}{d\Omega}\right)_{ee}^{\rm R}$, $\left(\frac{d\sigma}{d\Omega}\right)_{ee}^{\rm CRI}$, $\left(\frac{d\sigma}{d\Omega}\right)_{\mu\mu}^{\rm R}$ and $\left(\frac{d\sigma}{d\Omega}\right)_{\mu\mu}^{\rm CRI}$, can be defined as

\end{multicols}

\begin{subequations}
	\label{Align: definitions}
	\begin{align}
		\left(\frac{d\sigma}{d\Omega}\right)_{ee}^{\rm R} & = \int_0^X \left(\frac{d\sigma_0}{d\Omega}\right)_{ee}^{\rm S}(s(1-x),\cos\theta)\left|\Pi_{\rm R}(s(1-x))\right|^2 F(s,x)dx, \label{Equation: Resonance part of cross section of eetoee} \\
		\nonumber \\
		\left(\frac{d\sigma}{d\Omega}\right)_{ee}^{\rm CRI} & = \int_0^X \Bigg(\left(\frac{d\sigma_0}{d\Omega}\right)_{ee}^{\rm S}(s(1-x),\cos\theta) 2Re\left(\frac{1}{1-\Pi_0(s(1-x))}\Pi_{\rm R}(s(1-x))\right) + \nonumber \\
		& \ \ \ \ \ \ \ \ \ \ \  \left(\frac{d\sigma_0}{d\Omega}\right)_{ee}^{\rm STI}(s(1-x),\cos\theta) Re\left(\Pi_{\rm R}(s(1-x))\frac{1}{1-\Pi_0(t(1-x))}\right)\Bigg)F(s,x)dx, \label{Equation: Continuous-Resonance interference part of cross section of eetoee} \\
		\nonumber \\
		\left(\frac{d\sigma}{d\Omega}\right)_{\mu\mu}^{\rm R} & = \int_0^X \left(\frac{d\sigma_0}{d\Omega}\right)_{\mu\mu}^{\rm S}(s(1-x),\cos\theta)\left|\Pi_{\rm R}(s(1-x))\right|^2 F(s,x)dx, \label{Equation: Resonance part of cross section of eetomumu} \\
		\nonumber \\
		\left(\frac{d\sigma}{d\Omega}\right)_{\mu\mu}^{\rm CRI} & = \int_0^X \left(\frac{d\sigma_0}{d\Omega}\right)_{\mu\mu}^{\rm S}(s(1-x),\cos\theta) 2Re\left(\frac{1}{1-\Pi_0(s(1-x))}\Pi_{\rm R}(s(1-x))\right) F(s,x)dx. \label{Equation: Continuous-Resonance interference part of cross section of eetomumu}
	\end{align}
\end{subequations}

\centerline{\rule{80mm}{0.1pt}}

\begin{multicols}{2}
	
	With $\left(\frac{d\sigma_0}{d\Omega}\right)_{ee|\mu\mu}^{\rm S}$ and $\left(\frac{d\sigma_0}{d\Omega}\right)_{ee}^{\rm STI}$ expressed in Eq. (\ref{Equation: S channel part of Born cross section of eetoee}) and (\ref{Equation: S-T interference part of Born cross section of eetoee}) as well as $\Pi_{\rm R}$ expressed in Eq. (\ref{Equation: Resonance part of vacuum polarization function}) further employed, one can rewrite $\left(\frac{d\sigma}{d\Omega}\right)_{ee}^{\rm R}$, $\left(\frac{d\sigma}{d\Omega}\right)_{ee}^{\rm CRI}$, $\left(\frac{d\sigma}{d\Omega}\right)_{\mu\mu}^{\rm R}$ and $\left(\frac{d\sigma}{d\Omega}\right)_{\mu\mu}^{\rm CRI}$ more explicitly as
	
\end{multicols}

\begin{subequations}
	\label{Align: semi-finished results}
	\begin{align}
		\left(\frac{d\sigma}{d\Omega}\right)_{ee}^{\rm R} & = \frac{9\Gamma_{ee}^2}{4M^2} \cdot I^{\rm R} \cdot (1+\cos^2\theta), \label{Equation: Resonance part of cross section of eetoee --- semi-finished results} \\
		\nonumber \\
		\left(\frac{d\sigma}{d\Omega}\right)_{ee}^{\rm CRI} & = \frac{3\Gamma_{ee}\alpha}{2M} \cdot I^{\rm CRI} \cdot \left( (1+\cos^2\theta) \frac{1}{1-\Pi_0(s)} - \frac{(1+\cos\theta)^2}{1-\cos\theta} \frac{1}{1-\Pi_0(t)} \right), \label{Equation: Continuous-Resonance interference part of cross section of eetoee --- semi-finished results} \\
		\nonumber \\
		\left(\frac{d\sigma}{d\Omega}\right)_{\mu\mu}^{\rm R} & = \frac{9\Gamma_{ee}\Gamma_{\mu\mu}}{4M^2} \cdot I^{\rm R} \cdot (1+\cos^2\theta), \label{Equation: Resonance part of cross section of eetomumu --- semi-finished results}
	\end{align}
	\begin{align}
		\left(\frac{d\sigma}{d\Omega}\right)_{\mu\mu}^{\rm CRI} & = \frac{3\sqrt{\Gamma_{ee}\Gamma_{\mu\mu}}\alpha}{2M} \cdot I^{\rm CRI} \cdot (1+\cos^2\theta) \frac{1}{1-\Pi_0(s)} , \label{Equation: Continuous-Resonance interference part of cross section of eetomumu --- semi-finished results}
	\end{align}
\end{subequations}
where
\begin{subequations}
	\label{Align: definitions of the two integrals}
	\begin{align}
		I^{\rm R} & = \int_0^X \frac{s(1-x)}{(s(1-x)-M^2)^2+M^2\Gamma_{\rm tot}^2} F(s,x)dx, \\
		\nonumber \\
		I^{\rm CRI} & = \int_0^X \frac{s(1-x)-M^2}{(s(1-x)-M^2)^2+M^2\Gamma_{\rm tot}^2} F(s,x)dx.
	\end{align}
\end{subequations}

\centerline{\rule{80mm}{0.1pt}}

\begin{multicols}{2}
	Here, in the cases of $\left(\frac{d\sigma}{d\Omega}\right)_{ee}^{\rm CRI}$ and $\left(\frac{d\sigma}{d\Omega}\right)_{\mu\mu}^{\rm CRI}$, $\frac{1}{1-\Pi_0(s)}$ and $\frac{1}{1-\Pi_0(t)}$ are used as very good approximations to the equivalents of $\frac{1}{1-\Pi_0(s(1-x))}$ and $\frac{1}{1-\Pi_0(t(1-x))}$ after integration in Eq. (\ref{Align: definitions}). Numerical calculation indicates that the resulting deviations are less than 0.01\%.
	
	As can be seen from  Eq. (\ref{Align: semi-finished results}), to evaluate further, only $I^{\rm R}$ and $I^{\rm CRI}$ have to be calculated. Detailed calculations of the two integrals are put in Appendix A, which includes three parts: A.1, A.2, A.3. Their analytic formulae are fully derived in part A.1. Due to complexity, simplified versions of the analytic formulae are further obtained in part A.2. Finally, both versions of the analytic formulae are compared with numerical computing results in part A.3.
	
	Based on those of $I^{\rm R}$ and $I^{\rm CRI}$, we will list directly the full and simplified version of analytic results of $\left(\frac{d\sigma}{d\Omega}\right)_{ee}^{\rm R}$, $\left(\frac{d\sigma}{d\Omega}\right)_{ee}^{\rm CRI}$, $\left(\frac{d\sigma}{d\Omega}\right)_{\mu\mu}^{\rm R}$ and $\left(\frac{d\sigma}{d\Omega}\right)_{\mu\mu}^{\rm CRI}$ and discuss briefly their comparisons with numerical computing results in the following three subsections.
	
	\subsection{Full version of analytic results}
	
	With $I^{\rm R}$ and $I^{\rm CRI}$ expressed in Eq. (\ref{Equation: the first key integral}) and (\ref{Equation: the second key integral}) adopted, the full versions of the analytic formulae for $\left(\frac{d\sigma}{d\Omega}\right)_{ee}^{\rm R}$, $\left(\frac{d\sigma}{d\Omega}\right)_{ee}^{\rm CRI}$, $\left(\frac{d\sigma}{d\Omega}\right)_{\mu\mu}^{\rm R}$ and $\left(\frac{d\sigma}{d\Omega}\right)_{\mu\mu}^{\rm CRI}$ can be written as
	
\end{multicols}

\begin{subequations}
	\label{Align: full version of the final results}
	\begin{align}
		\left(\frac{d\sigma}{d\Omega}\right)_{ee}^{\rm R} & = \frac{9\Gamma_{ee}^2}{4M^2} \cdot s(P - Q) \cdot (1+\cos^2\theta), \\
		\nonumber \\
		\left(\frac{d\sigma}{d\Omega}\right)_{ee}^{\rm CRI} & = \frac{3\Gamma_{ee}\alpha}{2M} \cdot ((s-M^2) P  - s Q ) \cdot \left( (1+\cos^2\theta) \frac{1}{1-\Pi_0(s)} - \frac{(1+\cos\theta)^2}{1-\cos\theta} \frac{1}{1-\Pi_0(t)} \right), \\
		\nonumber \\
		\left(\frac{d\sigma}{d\Omega}\right)_{\mu\mu}^{\rm R} & = \frac{9\Gamma_{ee}\Gamma_{\mu\mu}}{4M^2} \cdot s(P - Q) \cdot  (1+\cos^2\theta), \\
		\nonumber \\
		\left(\frac{d\sigma}{d\Omega}\right)_{\mu\mu}^{\rm CRI} & = \frac{3\sqrt{\Gamma_{ee}\Gamma_{\mu\mu}}\alpha}{2M} \cdot ((s-M^2) P - s Q) \cdot (1+\cos^2\theta) \frac{1}{1-\Pi_0(s)},
	\end{align}
\end{subequations}
where
\begin{subequations}
	\begin{align}
		P & = \frac{1}{s^2}(A\ G(a,\beta,v,X) + B\ G(a,\beta,v+1,X) + C\ H(a,\beta,v,X)), \\
		Q & = \frac{1}{s^2}(D\ G(a,\beta,v+1,X) + E\ H(a,\beta,v,X) + C\ H(a,\beta,v+1,X))
	\end{align}
\end{subequations}
\centerline{\rule{80mm}{0.1pt}}
\begin{multicols}{2}
	\noindent with
	\begin{subequations}
		\begin{align}
			a & = \sqrt{\left(\frac{M^2}{s}-1\right)^2+\frac{M^2\Gamma_{\rm tot}^2}{s^2}},
		\end{align}
		\begin{align}
			\beta & = \cos^{-1}\left(\frac{\left(\frac{M^2}{s}-1\right)}{\sqrt{\left(\frac{M^2}{s}-1\right)^2+\frac{M^2\Gamma_{\rm tot}^2}{s^2}}}\right),
		\end{align}
		\begin{align}
			A & = 1+\delta, \\
			B & = \frac{1}{v+1}\left(-v-\frac{v^2}{4}\right), \\
			C & = \frac{v}{2}-\frac{3}{8}v^2, \\
			D & = \frac{Av}{v+1}, \\
			E & = B(v+1)
		\end{align}
	\end{subequations}
	and
	\begin{subequations}
		\begin{align}
			&\ \ \  G(a,\beta,v,X) \nonumber \\
			& = a^{v-2} \left(\frac{\pi v}{\sin\pi v}\right) \left(\frac{\sin[(1-v)\beta]}{\sin\beta}\right) + v X^{v-4} \bigg(\frac{X^2}{v-2} \nonumber \\
			& +\frac{2a(\cos\beta) X}{v-3}-\frac{a^2(4\cos^2\beta-1)}{v-4}\bigg) \ \ \ (0<v<2),
		\end{align}
		\begin{align}
			&\ \ \  H(a,\beta,v,X) \nonumber \\
			& = h(a\sin\beta,a\cos\beta,v+1,X+a\cos\beta) \nonumber \\
			& - h(a\sin\beta,a\cos\beta,v+1,a\cos\beta), \\
			&\ \ \  h(a,b,c,x) = -\frac{i}{2ac} \nonumber \\
			& \cdot \Bigg(\left(\frac{1}{-ia+x}\right)^{-c}{}_2\mathbb{F}_1\left(-c,-c,1-c,\frac{a+ib}{a+ix}\right) \nonumber \\
			& - \left(\frac{1}{ia+x}\right)^{-c}{}_2\mathbb{F}_1\left(-c,-c,1-c,\frac{ia+b}{ia+x}\right)\Bigg).
		\end{align}
	\end{subequations}
	Here, ${}_2\mathbb{F}_1$ is the Gauss hypergeometric function.
	
	\subsection{Simplified version of analytic results}
	
	With $I^{\rm R}$ and $I^{\rm CRI}$ given by Eq. (\ref{Equation: The approximate results 1}) and (\ref{Equation: The approximate results 2}), the simplified versions of the analytic formulae for $\left(\frac{d\sigma}{d\Omega}\right)_{ee}^{\rm R}$, $\left(\frac{d\sigma}{d\Omega}\right)_{ee}^{\rm CRI}$, $\left(\frac{d\sigma}{d\Omega}\right)_{\mu\mu}^{\rm R}$ and $\left(\frac{d\sigma}{d\Omega}\right)_{\mu\mu}^{\rm CRI}$ can be written as
	
\end{multicols}

\begin{subequations}
	\label{Align: Simplified version of the final results}
	\begin{align}
		\left(\frac{d\sigma}{d\Omega}\right)_{ee}^{\rm R} & = \frac{9\Gamma_{ee}^2}{4M^3\Gamma_{\rm tot}} (1+\delta) Im\mathcal{F} \cdot (1+\cos^2\theta), \label{Equation: The approximate results 1} \\
		\nonumber \\
		\left(\frac{d\sigma}{d\Omega}\right)_{ee}^{\rm CRI} & = - \frac{3\Gamma_{ee}\alpha}{2 M s} (1+\delta) Re\mathcal{F} \cdot \left( (1+\cos^2\theta) \frac{1}{1-\Pi_0(s)} - \frac{(1+\cos\theta)^2}{1-\cos\theta} \frac{1}{1-\Pi_0(t)} \right), \label{Equation: The approximate results 2} \\
		\nonumber \\
		\left(\frac{d\sigma}{d\Omega}\right)_{\mu\mu}^{\rm R} & = \frac{9\Gamma_{ee}\Gamma_{\mu\mu}}{4M^3\Gamma_{\rm tot}} (1+\delta) Im\mathcal{F} \cdot (1+\cos^2\theta), \label{Equation: The approximate results 3} \\
		\nonumber \\
		\left(\frac{d\sigma}{d\Omega}\right)_{\mu\mu}^{\rm CRI} & = - \frac{3\sqrt{\Gamma_{ee}\Gamma_{\mu\mu}}\alpha}{2 M s} (1+\delta) Re\mathcal{F} \cdot (1+\cos^2\theta) \frac{1}{1-\Pi_0(s)}, \label{Equation: The approximate results 4}
	\end{align}
\end{subequations}

\centerline{\rule{80mm}{0.1pt}}

\begin{multicols}{2}
	\noindent where
	\begin{equation}
		\mathcal{F} = \left( \frac{\pi v}{\sin\pi v} \right) \left(\frac{s}{M^2 - s - i M\Gamma_{\rm tot}}\right)^{1-v}.
		\label{Equation: The expressions of mathcalF.}
	\end{equation}
	
	\subsection{Comparison of analytic and numerical computing results}
	\label{Subsection: Comparisons with numerical computing results}
	
	As one can see from Eq. (\ref{Align: semi-finished results}) and (\ref{Align: definitions of the two integrals}),
	\begin{align*}
		& \ \ \  \left(\frac{\Delta \sigma}{\sigma}\right)_{ee|\mu\mu}^{\rm R|CRI}(\text{F}|\text{S},\text{N}) = \frac{\sigma_{ee|\mu\mu}^{\rm R|CRI}(\text{F}|\text{S})-\sigma_{ee|\mu\mu}^{\rm R|CRI}(\text{N})}{\sigma_{ee|\mu\mu}^{\rm R|CRI}(\text{N})} \\
		& = \frac{I^{\rm R|CRI}(\text{F}|\text{S})-I^{\rm R|CRI}(\text{N})}{I^{\rm R|CRI}(\text{N})} = \left(\frac{\Delta I}{I}\right)^{\rm R|CRI}(\text{F}|\text{S},\text{N}).
	\end{align*}
	%\end{multicols}
	%
	%\begin{equation*}
	%\left(\frac{\Delta \sigma}{\sigma}\right)_{ee|\mu\mu}^{\rm R|CRI}(\text{F}|\text{S},\text{N}) = \frac{\sigma_{ee|\mu\mu}^{\rm R|CRI}(\text{F}|\text{S})-\sigma_{ee|\mu\mu}^{\rm R|CRI}(\text{N})}{\sigma_{ee|\mu\mu}^{\rm R|CRI}(\text{N})} = \frac{I^{\rm R|CRI}(\text{F}|\text{S})-I^{\rm R|CRI}(\text{N})}{I^{\rm R|CRI}(\text{N})} = \left(\frac{\Delta I}{I}\right)^{\rm R|CRI}(\text{F}|\text{S},\text{N}).
	%\end{equation*}
	%
	%\centerline{\rule{80mm}{0.1pt}}
	%
	%\begin{multicols}{2}
	\noindent Here, the symbols F, S and N stand for the full version of the analytic results, the simplified version of the analytic results and the numerical computing results, respectively.
	
	According to part A.3 (the last part of Appendix A), from $\sqrt{s}=M-10\Tw$ to $\sqrt{s}=M+10\Tw$ with $X$ set at 1 as well as $M$ and $\Tw$ at their PDG values \cite{PDG2016}:
	\begin{equation*}
		\left(\frac{\Delta \sigma}{\sigma}\right)_{ee|\mu\mu}^{\rm R|CRI}(\text{F},\text{N}) = \left(\frac{\Delta I}{I}\right)^{\rm R|CRI}(\text{F},\text{N})<0.01\%
	\end{equation*}
	and
	\begin{equation*}
		\left(\frac{\Delta \sigma}{\sigma}\right)_{ee|\mu\mu}^{\rm R|CRI}(\text{S},\text{N}) = \left(\frac{\Delta I}{I}\right)^{\rm R|CRI}(\text{S},\text{N})<0.1\%.
	\end{equation*}
	Taking into account the precision of the structure function method itself is 0.1\% \cite{STRUCTUREFUNCTIONMETHOD}, we regard 0.1\% and 0.2\% as the precision of the full and simplified versions of the analytic formulae for $\left(\frac{d\sigma}{d\Omega}\right)_{ee}^{\rm R}$, $\left(\frac{d\sigma}{d\Omega}\right)_{ee}^{\rm CRI}$, $\left(\frac{d\sigma}{d\Omega}\right)_{\mu\mu}^{\rm R}$ and $\left(\frac{d\sigma}{d\Omega}\right)_{\mu\mu}^{\rm CRI}$, respectively.



\section{Conclusions}
%\section{Experimental Results And Discussion}
\label{sec:results}
The results presented in this section test the performance of the Autoencoder model. We evaluate our model using the performance metrics: accuracy, precision, recall, and F1 score, defined as follow: 

\vspace{-5mm}
\begin{align*}
    Accuracy &= \frac{TP+TN}{TP+TN+FP+FN}
\end{align*}
\vspace{-3mm}
\begin{align*}
    Precision &= \frac{TP}{TP+FP}
\end{align*}
\vspace{-3mm}
\begin{align*}
    Recall &= \frac{TP}{TP+FN}
\end{align*}
\vspace{-3mm}
\begin{align*}
    F1 ~Score &= 2 \times \frac{Precision \times Recall}{Precision + Recall}
\end{align*}

In our experiments, a \textit{positive} outcome means an abnormal activity was detected, whereas a negative outcome means a normal activity was detected.
True Positive (TP) refers to an abnormal activity that was correctly classified as abnormal. 
True Negative (TN) refers to a normal activity that was correctly classified as normal.
False Positive (FP) refers to a normal activity that was misclassified as abnormal.
and False Negative (FN) refers to an abnormal activity that was misclassified as normal.

The success of our model is based on measuring the reconstruction error that is produced by any given data point. Figure \ref{fig:recon} shows an example of reconstructed data overlaid the original data that was inserted into the model. %To be clear, the values shown in this graph are not measurements of the reconstruction loss that are shown in Figure \ref{fig:thresh}. This figure only shows the normalized temperature measurement from each data point, that is why they are not being represented in degrees.
In this figure, extremely severe dips in temperature denoted by the blue line (representing our original data) can be noticed. The data reconstructed by the model, represented by the red line, does not dip as much as the original data. This is because our model was not able to reconstruct these points accurately due to the fact that they are anomalies. The reconstruction loss (i.e. different between the original and the reconstructed data), where the model recognizes normal or abnormal behavior, is shown in Figure \ref{fig:thresh}. The figure shows a visualization of the mean-squared-error (MSE) generated by the model after it was given each data point within the test data set. The dotted red line denotes the threshold determined as mentioned in Section \ref{sec:ml-model}. Each data point's actual label is represented either by blue color to denote a normal behavior or red color to denote an anomaly and every data point that lies above the threshold was classified as anomalous. This figure illustrates our model's capability to detect the majority of anomalies by measuring the MSE produced by each data point.

Overall, as shown in Figure \ref{fig:aeresults}, our model was able to attain high performance with over $90\%$ in all metrics. The precision is lower than the recall metric which shows that the model produced slightly more false positives than false negatives. In a smart farming environment, a higher rate of false positives would not have a dramatic affect on the productivity of day to day operations and would ensure a higher number of anomalous situations are detected. A rather problematic situation would be if there were more false negatives than positives. A user would much prefer receiving an alert when nothing was wrong than not receiving an alert and enabling potential harm to occur to the crops and hardware. In the future, we hope to further decrease the number of false positives and negatives in order to fine-tune an overall more accurate model. This can be done by using more training samples.


% \begin{table}[!t]
%     \caption{Results}
%     \centering
%     \begin{tabular}{| c | c | c | c |}
%     \hline
%     Accuracy & Precision & Recall & F1\\ [0.5ex] % inserts table %heading
%     \hline
    
%     98.98\% & 90\% & 92.95\% & 91.45\% \\
    
%     \hline
%     \end{tabular}
%     \label{table:results}
% \end{table}

\begin{figure}[t!]
    \centering
    \includegraphics[width=8cm]{figures/aeresults-v2.png}
    \caption{Performance metrics for Autoencoder Model}
    \label{fig:aeresults}
\end{figure}


\section{Conclusion and Future Work}
\label{sec:conclusion}
Our approach has shown that smart farming anomaly detection can be done at an extremely accurate level by using an Autoencoder. Our approach would allow vast scalability by only requiring non-anomalous data for training. Greenhouses provide controlled environments that create consistent conditions for crops and data collection. Environments such as this are a perfect use case for our approach since the performance of an Autoencoder can drastically improve when provided with large amounts of non-anomalous data. Our approach shows that it may not be entirely necessary for machine learning professionals that are working on anomaly detection within smart farming to be highly concerned with developing models that are trained using labeled data that contains both normal and anomalous data. 

In the future, we will explore more anomaly detection models in order to optimize the system's performance. Once the best model has been selected, the architecture could be brought online to be used and tested with the added interactions of Internet connectivity. By bringing the system online we will have the ability to alert users of potential threats or anomalous behavior. These alerts could be coupled with actuators such as fertilization, watering, video monitoring, etc. The introduction of cameras can be ``used to calculate biomass development and fertilization status of crops" \cite{Walter6148}. They can also be used to allow the system-user to monitor their property from afar. We plan to introduce photo and video monitoring as one of our next steps to improve security and broaden our scope.

\section{Acknowledgements}
\label{sec:ack}
We thank TTU Shipley Farms for allowing to use greenhouse, and setup smart farm testbed. Dr. Brian Leckie and his group were instrumental in our system and early stages of data collection. We are thankful to Ms. Deepti Gupta to provide helpful guidance on dealing with time-series, correlated data and gave input on our model selection. This research is partially supported by the NSF Grant 2025682 at TTU.
We have derived the detailed formulae for the resonance and interference parts of the cross sections of \eetoee and \eetomumu around the \jpsi resonance with higher-order corrections for vacuum polarization and initial-state radiation considered. In the derivation, the arbitrary upper limit of radiative correction integration $X$ has been involved. Two (full and simplified) versions of the analytic formulae are given with precision at the levels of 0.1\% and 0.2\%, which are accurate enough for the measurement of \jpsi decay widths at present.

In our derivation, only a very few steps rely on the values of \jpsi resonance parameters and they can be easily verified to be workable for the case of the \psip resonance. In the coming round of data-taking at BESIII, there is a plan for an energy scan around the \psip resonance for the measurement of the resonance parameters. By that time, the results obtained in this paper will be good references.



%\section{Acknowledgments}

This work was supported by NSFC programs (61976138, 61977047), the National Key Research and Development Program (2018YFB2100 500), STCSM (2015F0203-000-06), SHMEC (2019-01-07-00-01-E00003) and Shanghai YangFan Program (21YF1429500).
\ \\

%\acknowledgments{The authors would like to thank Prof. Ping Wang, Prof. Hai-Ming Hu and Prof. Chang-Zheng Yuan for their kind help and beneficial discussions as well as Prof. Wei-Guo Li for his suggestion on the contributing.}
\acknowledgments{The authors would like to thank Prof. Wei-Guo Li for his suggestion on the contributing as well as Prof. Ping Wang, Prof. Hai-Ming Hu and Prof. Chang-Zheng Yuan for their kind help and beneficial discussions.}


\end{multicols}

%\section{}
\label{sec: error state dynamics}
The $\mathbf{F}$ and $\mathbf{G}$ in Eq.~\eqref{eq: error state dynamics} are,
\begin{equation*}
\mathbf{F} = 
\begin{pmatrix}
-\lfloor\hat{\bm{\omega}}{}_{\times}\rfloor & -\mathbf{I}_3 & 
\mathbf{0}_{3\times 3} & \mathbf{0}_{3\times 3} & \mathbf{0}_{3\times 3} \\
\mathbf{0}_{3\times 3} & \mathbf{0}_{3\times 3} & \mathbf{0}_{3\times 3} & 
\mathbf{0}_{3\times 3} & \mathbf{0}_{3\times 3} \\
-C\left({}^I_G\hat{\mathbf{q}}\right)^\top\lfloor\hat{\mathbf{a}}{}_{\times}\rfloor & 
\mathbf{0}_{3\times 3} & \mathbf{0}_{3\times 3} & 
-C\left({}^I_G\hat{\mathbf{q}}\right)^\top & \mathbf{0}_{3\times 3} \\
\mathbf{0}_{3\times 3} & \mathbf{0}_{3\times 3} & \mathbf{0}_{3\times 3} & 
\mathbf{0}_{3\times 3} & \mathbf{0}_{3\times 3} \\
\mathbf{0}_{3\times 3} & \mathbf{0}_{3\times 3} & \mathbf{I}_3 & 
\mathbf{0}_{3\times 3} & \mathbf{0}_{3\times 3} \\
\mathbf{0}_{3\times 3} & \mathbf{0}_{3\times 3} & \mathbf{0}_{3\times 3} & 
\mathbf{0}_{3\times 3} & \mathbf{0}_{3\times 3} \\
\mathbf{0}_{3\times 3} & \mathbf{0}_{3\times 3} & \mathbf{0}_{3\times 3} & 
\mathbf{0}_{3\times 3} & \mathbf{0}_{3\times 3}
\end{pmatrix}
\end{equation*}
and, 
\begin{equation*}
\mathbf{G} = 
\begin{pmatrix}
-\mathbf{I}_3 & \mathbf{0}_{3\times 3} & 
\mathbf{0}_{3\times 3} & \mathbf{0}_{3\times 3} \\
\mathbf{0}_{3\times 3} & \mathbf{I}_3 & 
\mathbf{0}_{3\times 3} & \mathbf{0}_{3\times 3} \\
\mathbf{0}_{3\times 3} & \mathbf{0}_{3\times 3} & 
-C\left({}^I_G\hat{\mathbf{q}}\right)^\top & \mathbf{0}_{3\times 3} \\
\mathbf{0}_{3\times 3} & \mathbf{0}_{3\times 3} & 
\mathbf{0}_{3\times 3} & \mathbf{0}_{3\times 3} \\
\mathbf{0}_{3\times 3} & \mathbf{0}_{3\times 3} & 
\mathbf{0}_{3\times 3} & \mathbf{I}_3 \\
\mathbf{0}_{3\times 3} & \mathbf{0}_{3\times 3} & 
\mathbf{0}_{3\times 3} & \mathbf{0}_{3\times 3} \\
\mathbf{0}_{3\times 3} & \mathbf{0}_{3\times 3} & 
\mathbf{0}_{3\times 3} & \mathbf{0}_{3\times 3}
\end{pmatrix}
\end{equation*}

\section{}
\label{sec: state augmentation jacobian}
The state augmentation Jacobian, $\mathbf{J}$, given in Eq.~\eqref{eq: state covariance augmentation}, is of the form,
\begin{equation*}
\mathbf{J} = 
\begin{pmatrix}
\mathbf{J}_I & \mathbf{0}_{6\times 6N}
\end{pmatrix}
\end{equation*}
where $\mathbf{J}_I$ is,
\begin{equation*}
\mathbf{J}_I = 
\begin{pmatrix}
C\left({}^I_G\hat{\mathbf{q}}\right) & \mathbf{0}_{3\times 9} & 
\mathbf{0}_{3\times 3} & \mathbf{I}_3 & \mathbf{0}_{3\times 3} \\
-C\left({}^I_G\hat{\mathbf{q}}\right)^\top \lfloor{}^I\hat{\mathbf{p}}_C {}_{\times}\rfloor & 
\mathbf{0}_{3\times 9} & \mathbf{I}_3 & \mathbf{0}_{3\times 3} & 
\mathbf{I}_{3}
\end{pmatrix}
\end{equation*}
Note that $\mathbf{J}_I$ given above corrects the typo in Eq. (16) of~\cite{mourikis2007multi}. 

\section{}
\label{sec: measurement jacobian}
Following the chain rule, $\mathbf{H}_{C_i}^j$ and $\mathbf{H}_{f_i}^j$, in Eq.~\eqref{eq: error measurement model}, can be computed as,
\begin{equation}
\label{eq: measurement jacobian}
\begin{gathered}
\mathbf{H}_{C_i}^j = 
\frac{\partial \mathbf{z}_i^j}{\partial {}^{C_{i,1}}\mathbf{p}_j} \cdot 
\frac{\partial {}^{C_{i,1}}\mathbf{p}_j}{\partial \mathbf{x}_{C_{i,1}}} + 
\frac{\partial \mathbf{z}_i^j}{\partial {}^{C_{i,2}}\mathbf{p}_j} \cdot 
\frac{\partial {}^{C_{i,2}}\mathbf{p}_j}{\partial \mathbf{x}_{C_{i,1}}} \\
\mathbf{H}_{f_i}^j = 
\frac{\partial \mathbf{z}_i^j}{\partial {}^{C_{i,1}}\mathbf{p}_j} \cdot 
\frac{\partial {}^{C_{i,1}}\mathbf{p}_j}{\partial {}^G\mathbf{p}_j} +
\frac{\partial \mathbf{z}_i^j}{\partial {}^{C_{i,2}}\mathbf{p}_j} \cdot 
\frac{\partial {}^{C_{i,2}}\mathbf{p}_j}{\partial {}^G\mathbf{p}_j} 
\end{gathered}
\end{equation}
where,
\begin{equation}
\label{eq: measurment jacobian expression}
\begin{gathered}
\frac{\partial \mathbf{z}_i^j}{\partial {}^{C_{i,1}}\mathbf{p}_j} = 
\frac{1}{{}^{C_{i, 1}}\hat{Z}_j}
\begin{pmatrix}
1 & 0 & -\frac{{}^{C_{i, 1}}\hat{X}_j}{{}^{C_{i, 1}}\hat{Z}_j} \\
0 & 1 & -\frac{{}^{C_{i, 1}}\hat{Y}_j}{{}^{C_{i, 1}}\hat{Z}_j} \\
0 & 0 & 0 \\
0 & 0 & 0 
\end{pmatrix} \\
\frac{\partial \mathbf{z}_i^j}{\partial {}^{C_{i,2}}\mathbf{p}_j} = 
\frac{1}{{}^{C_{i, 2}}\hat{Z}_j}
\begin{pmatrix}
0 & 0 & 0 \\
0 & 0 & 0 \\
1 & 0 & -\frac{{}^{C_{i, 2}}\hat{X}_j}{{}^{C_{i, 1}}\hat{Z}_j} \\
0 & 1 & -\frac{{}^{C_{i, 2}}\hat{Y}_j}{{}^{C_{i, 1}}\hat{Z}_j} 
\end{pmatrix} \\
\frac{\partial {}^{C_{i,1}}\mathbf{p}_j}{\partial \mathbf{x}_{C_{i,1}}} = 
\begin{pmatrix}
\lfloor{}^{C_{i,1}}\hat{\mathbf{p}}_j{}_{\times}\rfloor & 
-C\left({}^{C_{i,1}}_G\hat{\mathbf{q}}\right)
\end{pmatrix} \\
\frac{\partial {}^{C_{i,1}}\mathbf{p}_j}{\partial {}^G\mathbf{p}_j} = 
C\left({}^{C_{i,1}}_G\hat{\mathbf{q}}\right) \\
\frac{\partial {}^{C_{i,2}}\mathbf{p}_j}{\partial \mathbf{x}_{C_{i,1}}} = 
C\left({}^{C_{i,1}}_{C_{i,2}}\mathbf{q}\right)^\top
\begin{pmatrix}
\lfloor{}^{C_{i,1}}\hat{\mathbf{p}}_j{}_{\times}\rfloor & 
-C\left({}^{C_{i,1}}_G\hat{\mathbf{q}}\right)
\end{pmatrix} \\
\frac{\partial {}^{C_{i,2}}\mathbf{p}_j}{\partial {}^G\mathbf{p}_j} = 
C\left({}^{C_{i,1}}_{C_{i,2}}\mathbf{q}\right)^\top
C\left({}^{C_{i,1}}_G\hat{\mathbf{q}}\right)
\end{gathered}
\end{equation}

\section{}
\label{sec: nullify measurement jacobian}
By defining the following short-hand notation from Eq.~\eqref{eq: measurment jacobian expression}
\begin{equation*}
\begin{gathered}
\frac{\partial \mathbf{z}_i^j}{\partial {}^{C_{i,1}}\mathbf{p}_j} = 
\begin{pmatrix}
\mathbf{J}_1 \\ \mathbf{0}
\end{pmatrix}, \quad
\frac{\partial \mathbf{z}_i^j}{\partial {}^{C_{i,2}}\mathbf{p}_j} = 
\begin{pmatrix}
\mathbf{0} \\ \mathbf{J}_2
\end{pmatrix}\\
\frac{\partial {}^{C_{i,1}}\mathbf{p}_j}{\partial \mathbf{x}_{C_{i,1}}} = 
\mathbf{H}_1, \quad 
\frac{\partial {}^{C_{i,1}}\mathbf{p}_j}{\partial {}^G\mathbf{p}_j} = 
\mathbf{H}_2, \quad
C\left({}^{C_{i,1}}_{C_{i,2}}\mathbf{q}\right) = 
\mathbf{R}\ ,
\end{gathered}
\end{equation*}
the measurement Jacobian in Eq.~\eqref{eq: measurement jacobian} can be compactly written as
\begin{equation*}
\mathbf{H}_{C_i}^j = 
\begin{pmatrix}
\mathbf{J}_1 \mathbf{H}_1 \\
\mathbf{J}_2 \mathbf{R}^\top \mathbf{H}_1
\end{pmatrix},\quad
\mathbf{H}_{f_i}^j =
\begin{pmatrix}
\mathbf{J}_1 \mathbf{H}_2 \\
\mathbf{J}_2 \mathbf{R}^\top \mathbf{H}_2
\end{pmatrix}\ .
\end{equation*}
Assuming $\mathbf{v} = \left(\mathbf{v}_1^\top,\ \mathbf{v}_2^\top\right)^\top\in\mathbb{R}^4$ is the left null space of $\mathbf{H}_{f_i}^j$, then,
\begin{equation*}
\mathbf{v}^\top \mathbf{H}_{f_i}^j  = 
\left(\mathbf{v}_1^\top \mathbf{J}_1 + 
\mathbf{v}_2^\top\mathbf{J}_2\mathbf{R}^\top\right) 
\mathbf{H}_2 = \mathbf{0}
\end{equation*}
Since $\mathbf{H}_2 = C\left({}^{C_{i,1}}_G\hat{\mathbf{q}}\right)$ is a rotation matrix, $\text{rank}\left(\mathbf{H}_2\right) = 3$ which implies that $\mathbf{v}_1^\top \mathbf{J}_1 + \mathbf{v}_2^\top\mathbf{J}_2\mathbf{R}^\top = \mathbf{0}$. With such property, it immediately follows that $\mathbf{v}$ is also the left null space of $\mathbf{H}_{C_i}^j$, 
\begin{equation*}
\mathbf{v}^\top \mathbf{H}_{C_i}^j = 
\left(\mathbf{v}_1^\top \mathbf{J}_1 + 
\mathbf{v}_2^\top\mathbf{J}_2\mathbf{R}^\top\right) 
\mathbf{H}_1 = \mathbf{0}
\end{equation*}
Therefore, a singe stereo measurement cannot be directly used for measurement update.

%The first line must not be deleted. It is needed by the CPC template.
\vspace{15mm}
%\input{Appendices/Two_key_integrals/Two_key_integrals}
\begin{small}
	\renewcommand{\theequation}{A\arabic{equation}}
	\setcounter{equation}{0}
	\begin{multicols}{2}
		\subsection*{Appendix A}
		\label{Section: Appendix A}
		\noindent{\bf Calculations of $I^{\rm R}$ and $I^{\rm CRI}$} \\
		
		\noindent \textbf{A.1 Full versions of analytic formulae} \\
		%\subsubsection*{\textbf{A.1 The full version of analytic formulae} \\}
		
		In the appendix, we evaluate the two integrals $I^{\rm R}$ and $I^{\rm CRI}$ required in Section \ref{Calculations of the resonance and interference parts}. For the convenience of further calculations, it is necessary to make some simple transformations by introducing some new variables. The first transformation is
		\begin{equation}
			\frac{1}{(s(1-x)-M^2)^2+M^2\Gamma_{\rm tot}^2} = \frac{1}{s^2} \frac{1}{x^2+2a(\cos\beta) x+a^2},
		\end{equation}
		where
		\begin{subnumcases}{}
			a = \sqrt{\left(\frac{M^2}{s}-1\right)^2+\frac{M^2\Gamma_{\rm tot}^2}{s^2}}, \\
			\beta = \cos^{-1}\left(\frac{\left(\frac{M^2}{s}-1\right)}{\sqrt{\left(\frac{M^2}{s}-1\right)^2+\frac{M^2\Gamma_{\rm tot}^2}{s^2}}}\right).
		\end{subnumcases}
		
		The second transformation is
		\begin{align}
			F(s,x) & = x^{v-1}v(1+\delta) \nonumber \\
			& + x^{v}\left(-v-\frac{v^2}{4}\right) + x^{v+1}\left(\frac{v}{2}-\frac{3}{8}v^2\right) \nonumber \\
			& = Avx^{v-1} + B(v+1)x^{v} + Cx^{v+1},
		\end{align}
		where
		\begin{subnumcases}{}
			A = 1+\delta, \\
			B = \frac{1}{v+1}\left(-v-\frac{v^2}{4}\right), \\
			C = \frac{v}{2}-\frac{3}{8}v^2.
		\end{subnumcases}
		
		The third transformation is
		\begin{align}
			xF(s,x) & = x^{v}v(1+\delta) \nonumber \\
			& + x^{v+1}\left(-v-\frac{v^2}{4}\right) + x^{v+2}\left(\frac{v}{2}-\frac{3}{8}v^2\right) \nonumber \\
			& = D(v+1)x^{v} + Ex^{v+1} + Cx^{v+2},
		\end{align}
		where
		\begin{subnumcases}{}
			D = \frac{Av}{v+1}, \\
			E = B(v+1).
		\end{subnumcases}
		
		In addition, some integral formulae are crucial for further calculations. From the following two integral formulae
		\begin{align}
			&\ \ \  \int_0^{\infty} \frac{vx^{v-1}}{x^2+2a(\cos\beta) x+a^2} dx \nonumber \\
			& = a^{v-2} \left(\frac{\pi v}{\sin\pi v}\right) \left(\frac{\sin[(1-v)\beta]}{\sin\beta}\right) \ \ \ (0<v<2)
		\end{align}
		and
		\begin{align}
			&\ \ \  \int_{X}^{\infty} \frac{vx^{v-1}}{x^2+2a(\cos\beta) x+a^2} dx \simeq v X^{v-4} \Bigg(-\frac{X^2}{v-2} \nonumber \\
			& - \frac{2a(\cos\beta) X}{v-3}+\frac{a^2(4\cos^2\beta-1)}{v-4}\Bigg) \ \ \ (v<2),
		\end{align}
		one obtains for the first integral formula
		\begin{align}
			&\ \ \  G(a,\beta,v,X) = \int_{0}^{X} \frac{vx^{v-1}}{x^2+2a(\cos\beta) x+a^2} dx \nonumber \\
			& \simeq a^{v-2} \left(\frac{\pi v}{\sin\pi v}\right) \left(\frac{\sin[(1-v)\beta]}{\sin\beta}\right) + v X^{v-4} \bigg(\frac{X^2}{v-2} \nonumber \\
			& +\frac{2a(\cos\beta) X}{v-3}-\frac{a^2(4\cos^2\beta-1)}{v-4}\bigg) \ \ \ (0<v<2).
		\end{align}
		
		The second integral formula is
		\begin{align}
			&\ \ \  H(a,\beta,v,X) = \int_{0}^{X} \frac{x^{v+1}}{x^2+2a(\cos\beta) x+a^2} dx \nonumber \\
			& = \int_{0}^{X} \frac{x^{v+1}}{(x+a\cos\beta)^2+(a\sin\beta)^2} dx \nonumber \\
			& = \int_{a\cos\beta}^{X+a\cos\beta} \frac{(y-a\cos\beta)^{v+1}}{y^2+(a\sin\beta)^2} dy \nonumber \\
			%\end{align}
			%\begin{align}
			& = h(a\sin\beta,a\cos\beta,v+1,X+a\cos\beta) \nonumber \\
			& - h(a\sin\beta,a\cos\beta,v+1,a\cos\beta),
		\end{align}
		where
		\begin{align}
			&\ \ \  h(a,b,c,x) = \int_0^x \frac{(y-b)^c}{y^2+a^2} dy = -\frac{i}{2ac} \nonumber \\
			& \cdot \Bigg(\left(\frac{1}{-ia+x}\right)^{-c}{}_2\mathbb{F}_1\left(-c,-c,1-c,\frac{a+ib}{a+ix}\right) \nonumber \\
			%\end{align}
			%\begin{align}
			& - \left(\frac{1}{ia+x}\right)^{-c}{}_2\mathbb{F}_1\left(-c,-c,1-c,\frac{ia+b}{ia+x}\right)\Bigg).
		\end{align}
		Here, ${}_2\mathbb{F}_1$ is the Gauss hypergeometric function.
		
		Using the newly introduced variables and the important integral formulae, we get
	\end{multicols}
	\begin{align}
		P & = \int_0^X \frac{1}{(s(1-x)-M^2)^2+M^2\Gamma_{\rm tot}^2} F(s,x)dx = \frac{1}{s^2} \int_0^X \frac{1}{x^2+2a(\cos\beta) x+a^2} (Avx^{v-1} + B(v+1)x^{v} + Cx^{v+1})dx \nonumber \\
		& = \frac{1}{s^2}\left(A \int_0^X \frac{vx^{v-1}}{x^2+2a(\cos\beta) x+a^2} dx + B \int_0^X \frac{(v+1)x^{v}}{x^2+2a(\cos\beta) x+a^2}dx + C \int_0^X \frac{x^{v+1}}{x^2+2a(\cos\beta) x+a^2} dx\right) \nonumber \\
		& = \frac{1}{s^2}(A\ G(a,\beta,v,X) + B\ G(a,\beta,v+1,X) + C\ H(a,\beta,v,X))
	\end{align}
	and
	\begin{align}
		Q & = \int_0^X \frac{x}{(s(1-x)-M^2)^2+M^2\Gamma_{\rm tot}^2} F(s,x)dx = \frac{1}{s^2} \int_0^X \frac{x}{x^2+2a(\cos\beta) x+a^2} (Avx^{v-1} + B(v+1)x^{v} + Cx^{v+1})dx \nonumber \\
		& = \frac{1}{s^2} \int_0^X \frac{1}{x^2+2a(\cos\beta) x+a^2} (D(v+1)x^{v} + Ex^{v+1} + Cx^{v+2})dx \nonumber \\
		& = \frac{1}{s^2} \left( D \int_0^X \frac{(v+1)x^{v}}{x^2+2a(\cos\beta) x+a^2} dx + E \int_0^X \frac{x^{v+1}}{x^2+2a(\cos\beta) x+a^2} dx + C \int_0^X \frac{x^{v+2}}{x^2+2a(\cos\beta) x+a^2} dx \right) \nonumber \\
		& = \frac{1}{s^2} (D\ G(a,\beta,v+1,X) + E\ H(a,\beta,v,X) + C\ H(a,\beta,v+1,X)),
	\end{align}
	and then get
	\begin{align}
		I^{\rm R} & = \int_0^X \frac{s(1-x)}{(s(1-x)-M^2)^2+M^2\Gamma_{\rm tot}^2} F(s,x)dx = s \int_0^X \frac{1-x}{(s(1-x)-M^2)^2+M^2\Gamma_{\rm tot}^2} F(s,x)dx \nonumber \\
		& = s \left(\int_0^X \frac{1}{(s(1-x)-M^2)^2+M^2\Gamma_{\rm tot}^2} F(s,x)dx - \int_0^X \frac{x}{(s(1-x)-M^2)^2+M^2\Gamma_{\rm tot}^2} F(s,x)dx \right) \nonumber \\
		& = s (P - Q)
		\label{Equation: the first key integral}
	\end{align}
	and
	\begin{align}
		I^{\rm CRI} & = \int_0^X \frac{s(1-x)-M^2}{(s(1-x)-M^2)^2+M^2\Gamma_{\rm tot}^2} F(s,x)dx = \int_0^X \frac{(s-M^2)-s x}{(s(1-x)-M^2)^2+M^2\Gamma_{\rm tot}^2} F(s,x)dx \nonumber \\
		& = (s-M^2) \int_0^X \frac{1}{(s(1-x)-M^2)^2+M^2\Gamma_{\rm tot}^2} F(s,x)dx - s \int_0^X \frac{x}{(s(1-x)-M^2)^2+M^2\Gamma_{\rm tot}^2} F(s,x)dx \nonumber \\
		& =(s-M^2) P - s Q.
		\label{Equation: the second key integral}
	\end{align}
	\centerline{\rule{80mm}{0.1pt}}
	\begin{multicols}{2}
		
		Equations (\ref{Equation: the first key integral}) and (\ref{Equation: the second key integral}) give the analytic formulae for $I^{\rm R}$ and $I^{\rm CRI}$. Since there are no approximations made in the derivation, we refer to the formulae as the full versions of the analytic formulae. Considering all the quantities involved in $P$ and $Q$ ($A$, $B$, $C$ and so on), the results are actually very complicated. For ease of use, simplified versions of the analytic formulae are needed.
		
		\noindent \textbf{A.2 Simplified versions of analytic formulae} \\
		%\subsubsection*{\textbf{A.2 The simplified version of analytic formulae} \\}
		
		In this part, we will make some approximations to obtain simplified versions of the analytic formulae. The first step is to reduce $F(s,x)$ to $x^{v-1}v(1+\delta)$. Since $0 \le x \le 1$ and $v \approx 0.08$ in the \jpsi region, the parts discarded are negligible. This reduction leads to $B=0$, $C=0$, $E=0$.
		
		The second step is to reduce $G(a,\beta,v,X)$ to $a^{v-2} \left(\frac{\pi v}{\sin\pi v}\right) \left(\frac{\sin[(1-v)\beta]}{\sin\beta}\right)$. This reduction means that $X \to +\infty$, which is unreasonable from the physical point of view. However, since $v \approx 0.08$ and $a \in (3\times 10^{-5},\  3\times 10^{-2})$, the reduction itself is a reasonable mathematical approximation when $X$ is large enough. In addition, in the cases of $\left(\frac{d\sigma}{d\Omega}\right)_{ee}^{\rm R}$ and $\left(\frac{d\sigma}{d\Omega}\right)_{\mu\mu}^{\rm R}$, a reasonable reduction of $\sin[(1-v)\beta] - a\sin[(-v)\beta]$ to $\sin[(1-v)\beta]$ is also carried out at this step. With the two steps of approximation applied, one can get
		\begin{equation}
			I^{\rm R} \approx \frac{1}{s a\sin\beta} (1+\delta) a^{v-1} \left(\frac{\pi v}{\sin\pi v}\right) \sin[(1-v)\beta]
		\end{equation}
		and
		\begin{equation}
			I^{\rm CRI} \approx - \frac{1}{s} (1+\delta) a^{v-1} \left(\frac{\pi v}{\sin\pi v}\right) \cos[(1-v)\beta].
		\end{equation}
		
		At this point, if one introduces a complex variable
		\begin{equation}
			\mathcal{F} = \left( \frac{\pi v}{\sin\pi v} \right) (a\cos\beta - i a\sin\beta)^{v-1},
		\end{equation}
		then
		\begin{equation}
			a^{v-1} \left(\frac{\pi v}{\sin\pi v}\right) \sin[(1-v)\beta] = Im\mathcal{F},
			\label{Equation: The Im part of mathcalF}
		\end{equation}
		\begin{equation}
			a^{v-1} \left(\frac{\pi v}{\sin\pi v}\right) \cos[(1-v)\beta] = Re\mathcal{F}.
			\label{Equation: The Re part of mathcalF}
		\end{equation}
		Getting $a$ and $\beta$ back to $\sqrt{\left(\frac{M^2}{s}-1\right)^2+\frac{M^2\Gamma_{\rm tot}^2}{s^2}}$ and $\cos^{-1}\left(\frac{\left(\frac{M^2}{s}-1\right)}{\sqrt{\left(\frac{M^2}{s}-1\right)^2+\frac{M^2\Gamma_{\rm tot}^2}{s^2}}}\right)$, respectively, one has
		\begin{equation}
			a \sin\beta=\frac{M\Gamma_{\rm tot}}{s}
			\label{Equation: ASinBeta}
		\end{equation}
		and
		\begin{equation}
			\mathcal{F} = \left( \frac{\pi v}{\sin\pi v} \right) \left(\frac{s}{M^2 - s - i M\Gamma_{\rm tot}}\right)^{1-v}.
			\label{Equation: The expressions of mathcalF.}
		\end{equation}
		
		With Eq. (\ref{Equation: The Im part of mathcalF}), (\ref{Equation: The Re part of mathcalF}) and (\ref{Equation: ASinBeta}), $I^{\rm R}$ and $I^{\rm CRI}$ can be expressed further as
		\begin{equation}
			I^{\rm R} \approx \frac{1}{M \Gamma_{\rm tot}} (1+\delta) Im\mathcal{F} \label{Equation: The approximate results 1}
		\end{equation}
		and
		\begin{equation}
			I^{\rm CRI} \approx - \frac{1}{s} (1+\delta) Re\mathcal{F}. \label{Equation: The approximate results 2}
		\end{equation}
		These are the simplified versions of the analytic formulae we need. \\
		
		\noindent \textbf{A.3 Comparisons of analytic formulae with numerical computing results} \\
		%\subsubsection*{\textbf{A.3 Comparisons of analytic formulae with numerical computing results} \\}
		%\label{Subsubsection: A.3 Comparisons of analytic formulae with numerical computing results}
		
		To check the validity of these analytic formulae, we compare them with numerical computing results. In the comparisons, the two integrals $I^{\rm R}$ and $I^{\rm CRI}$ are compared from $\sqrt{s}=M-10\Tw$ to $\sqrt{s}=M+10\Tw$ with $X$ set at 1 as well as $M$ and $\Tw$ at their PDG values \cite{PDG2016}. The results are shown in Fig. \ref{Figure: Comparisons with numerical computing results.}.
	\end{multicols}
	\begin{figure}[!h]
		\centering
		\subfigure{\includegraphics[width=0.4\textwidth]{ecm_rdperIresFN_rdperIresSN.eps}}
		\subfigure{\includegraphics[width=0.4\textwidth]{ecm_rdperIinfFN_rdperIinfSN.eps}}
		\caption{Comparisons of analytic formulae with numerical computing results. In the middle of the right-hand plot, the dotted line has a similar structure to the solid one. It does not show clearly in the plot because of its small scale.}
		\label{Figure: Comparisons with numerical computing results.}
	\end{figure}
	\begin{multicols}{2}
		The variables in the legends are defined as
		\begin{equation*}
			\left(\frac{\Delta I}{I}\right)^{\rm R|CRI}(\text{F}|\text{S},\text{N}) = \frac{I^{\rm R|CRI}(\text{F}|\text{S})-I^{\rm R|CRI}(\text{N})}{I^{\rm R|CRI}(\text{N})}.
		\end{equation*}
		Here, the symbols $|$, F, S and N are same as those used at the beginning of Subsections \ref{Subsection: Applications of the structure fuction method to eetoee and eetomumu} and \ref{Subsection: Comparisons with numerical computing results}.
		
		As can be seen from the dotted lines, the full versions of the analytic formulae agree very well with the numerical computing results. In fact, detailed numbers show that their relative differences are less than 0.01\%. Similarly, from the solid lines, one can see that except for $I^{\rm CRI}$ at energies very close to the \jpsi peak, the simplified versions of the analytic formulae agree with the numerical computing results to better than 0.1\%. The upward and downward peaks of $\left(\frac{\Delta I}{I}\right)^{\rm CRI}(\text{S},\text{N})$ at energies near the \jpsi peak is caused by the smallness of the absolute values (very close to 0) of $I^{\rm CRI}$, which makes $\sigma^{\rm CRI}$ values negligible when compared with their corresponding $\sigma^{\rm R}$ values. Because in the end, only the sum of $\sigma^{\rm R}$ and $\sigma^{\rm CRI}$ will be used in our data analysis, the peaks of $\left(\frac{\Delta I}{I}\right)^{\rm CRI}(\text{S},\text{N})$ are not worrying for us.
		
	\end{multicols}
	
\end{small}


%%\chapter*{}
%\addcontentsline{toc}{chapter}{Bibliography}

\bibliographystyle{apacite}
\bibliography{
biblio_ch_5}
%The first three lines must not be deleted. They are needed by the CPC template.
%\vspace{-1mm}
%\centerline{\rule{80mm}{0.1pt}}
%\vspace{2mm}

\begin{multicols}{2}
	\begin{thebibliography}{90}
		\vspace{3mm}
		\bibitem{potential models 1}A.M. Badalian and I.V. Danilkin, Phys. Atom. Nucl., $\bm{72}$: 1206 (2009)
		\bibitem{potential models 2}O. Lakhina and E.S. Swanson, Phys. Rev. D, $\bm{74}$: 014012 (2006)
		\bibitem{lattice calculations}J.J. Dudek, R.G. Edwards, and D.G. Richards, Phys. Rev. D, $\bm{73}$: 074507 (2006)
		\bibitem{BABAR}B. Aubert et al (BABAR Collaboration), Phys. Rev. D, $\bm{69}$: 011103 (2004)
		\bibitem{CLEO}G.S. Adams et al (CLEO Collaboration), Phys. Rev. D, $\bm{73}$: 051103 (2006)
		\bibitem{KEDR}V.V. Anashin et al (KEDR Collaboration), Phys. Lett. B, $\bm{685}$: 134 (2010)
		\bibitem{BESIII detector}M. Ablikim et al (BESIII Collaboration), Nucl. Instrum. Methods A, $\bm{614}$: 345 (2010)
		\bibitem{BABAYAGA}C.M. Carloni Calame, G. Montagna, O. Nicrosini et al, Nucl. Phys. B (Proc. Suppl.), $\bm{131}$: 48 (2004)
		\bibitem{ISR-FSR relation}V.S. Fadin, V.A. Khoze and A.D. Martin, Phys. Lett. B, $\bm{320}$: 141 (1994)
		\bibitem{STRUCTUREFUNCTIONMETHOD}E.A. Kuraev and V.S. Fadin, Sov. J. Nucl. Phys., $\bm{41}$: 466 (1985)
		\bibitem{Radiator 1}F.Z. Chen, P. Wang, J.M. Wu et al, HEP \& NP, $\bm{14}$ (7): 585 (1990) (in Chinese)
		\bibitem{Radiator 2}X.H. Mo, Measurement of $\psi(2S)$ Resonance Parameters, Ph.D. Thesis (Beijing: Institute of High Energy Physics, CAS, 2001) (in Chinese)
		\bibitem{Radiator 3}F.Z. Chen, P. Wang, C.M. Wu et al, BIHEP-EP-90-01
		\bibitem{KEDRpsip}V.V. Anashin et al (KEDR Collaboration), Phys. Lett. B, $\bm{711}$: 280 (2012)
		\bibitem{PDG2016}C. Patrignani et al (Particle Data Group), Chin. Phys. C, $\bm{40}$ (10): 1 (2016)
	\end{thebibliography}
\end{multicols}



\clearpage
\end{CJK*}
\end{document}
