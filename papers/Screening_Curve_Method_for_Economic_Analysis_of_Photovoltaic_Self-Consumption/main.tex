\documentclass[lettersize,journal]{IEEEtran}
\usepackage{amsmath,amsfonts}
\usepackage{algorithmic}
\usepackage{algorithm}
\usepackage{array}
%\usepackage[caption=false,font=normalsize,labelfont=sf,textfont=sf]{subfig}
\usepackage{textcomp}
\usepackage{stfloats}
\usepackage{url}
\usepackage{verbatim}
\usepackage{graphicx}


\hyphenation{op-tical net-works semi-conduc-tor IEEE-Xplore}
% updated with editorial comments 8/9/2021

\usepackage{siunitx}
\usepackage{subcaption}
\usepackage{enumerate}
%\usepackage{cite}


\newcommand{\eref}[1]{Eq.\,(\ref{#1})}
\newcommand{\eqsref}[1]{Eqs.\,(\ref{#1})}
\newcommand{\figref}[1]{Fig.\,\ref{#1}}
\newcommand{\Figref}[1]{Figure\,\ref{#1}}
\newcommand{\tabref}[1]{Table\,\ref{#1}}
\newcommand{\secref}[1]{Section\,\ref{#1}}


\begin{document}

\title{Screening Curve Method for Economic Analysis \\of Photovoltaic Self-Consumption}

\author{Hikaru Hoshino,~
        Yosuke Irie,~
        Eiko Furutani
        % <-this % stops a space
\thanks{This work was supported in part by JST, ACT-X Grant Number JPMJAX210M, Japan.}% <-this % stops a space
\thanks{The authors are with the Department of Electrical Materials and Engineering, University of Hyogo, 2167 Shosha, Himeji, Hyogo, 671-2280, Japan (email: hoshino@eng.u-hyogo.ac.jp). }%
%\thanks{Manuscript received April 19, 2021; revised August 16, 2021.}
}

% The paper headers
%\markboth{Journal of \LaTeX\ Class Files,~Vol.~14, No.~8, August~2021}%
\markboth{}%
{Shell \MakeLowercase{\textit{et al.}}: A Sample Article Using IEEEtran.cls for IEEE Journals}

%\IEEEpubid{0000--0000/00\$00.00~\copyright~2021 IEEE}
\IEEEpubid{}
% Remember, if you use this you must call \IEEEpubidadjcol in the second
% column for its text to clear the IEEEpubid mark.

\maketitle

\begin{abstract}
The profitability of the household photovoltaic (PV) self-consumption is expected to boost the deployment of PV and battery storage systems. This paper develops a novel method for economic analysis of PV self-consumption using battery storage based on an extension of the Screening Curve Method (SCM). The SCM enables quick and intuitive estimation of the least-cost generation mix for a target load curve and has been used for generation planning of a bulk power system. In this paper, we generalize the framework of existing SCM to take into account the intermittent nature of renewable energy sources and apply it to the problem of optimal sizing of PV and battery storage systems for a household. Numerical studies are provided to verify the estimation accuracy of the proposed SCM and to illustrate its effectiveness in a sensitivity analysis owing to the ability to show intuitive plots of cost curves for researchers or policy makers to understand the reasons behind the optimization results. 
\end{abstract}

\begin{IEEEkeywords}
 Solar PV, battery storage, screening curve method, feed-in tariffs.
\end{IEEEkeywords}

\section{Introduction}

% Self-consumption: policy 
%
\IEEEPARstart{T}{he} installed photovoltaic (PV) capacity in the world has  increased over the past decades owing to the use of support policies such as  feed-in tariffs \cite{timilsina12,poruschi18}.
%that aim to compensate the PV capital cost by grid-injection remunerations. 
Recently, the price of PV has declined dramatically, and the grid parity has been reached in several countries \cite{iea-pvps22}, meaning that PV can produce electricity at a price below the price of electricity purchased from the grid. 
Because of this, %feed-in tariffs are decreasing, and grid-injection remunerations are already or will be lower than the retail electricity price.
deployment of PV based on self-consumption of PV-produced electricity has attracted great interest \cite{luthander15}. 
%
As one of the policies based on the concept of PV self-consumption,
%concept of PV self-consumption %of PV-produced electricity, %\cite{luthander15}, 
many countries introduced net-metering or net-billing mechanism \cite{ordonez22}, where the surplus electricity fed into the grid is rewarded with credits that can be applied to offset consumption.
However, there have been an intense debates on net‑metering policies regarding  recovery of network costs and equity of cost allocation \cite{felder14,schittekatte18}, and several setbacks and cancellations in the application of these policies have occurred \cite{iea-pvps22}. 
%, and appropriate regulatory frameworks %for PV based on the concept of self-consumption
%remain to be discussed.  
Furthermore, various new self-consumption schemes such as collective or distributed self‑consumption \cite{ines20,dadamo22} are proposed or implemented %to be analysed their effectiveness. 
%For example, collective or distributed self‑consumption \cite{ines20,dadamo22} is allowed 
in several countries,  % such that different consumers. % located in the same building or private area (collective self-consumption), or in the same geographical area which requires to use the public grid (distributed or virtual self‑consumption), to share the self-generated electricity, thereby unlocking access to self‑consumption for a wide range of consumers.
and thus appropriate regulatory frameworks based on the concept of self-consumption remain to be discussed.  

% method 
%
To contribute to such discussions, economic analysis for evaluating the profitability of PV self-consumption has been extensively studied. 
Existing methods can be categorized into optimization and simulation
methods, depending on whether the capacity of PV and battery units are optimization variables or simulated as exogenous parameters \cite{han22}.
Since simulation-based methods might underestimate the economic
value of the investment \cite{zhang17:pv}, it is important
to correctly determine the size of the PV and battery units according to the customers’ load
profile, and thus optimization methods are often used  \cite{nyholm16,schopfer18,cervantes18}. 
While these methods can incorporate constraints to describe various situations, it takes hours or days to obtain optimization results.  
Therefore, only a limited number of scenarios (input data and parameters) are likely to be analyzed to provide inadequate information to infer a trend for inputs.
Besides, these methods work as black boxes, and limited intuition can be gained from the obtained results. 
These two issues of computational time and interpretability of the results make it difficult for policy-makers and regulators to learn from
these optimization studies. 


% SCM
%
The Screening Curve Method (SCM) is an intuitive and quick method that estimates the least-cost generation mix, % for a target year, 
and it is a good tool to provide answers to the above issues.
SCM was first proposed in 1960s \cite{phillips69,lee78}.  
It compares the summations of annualized upfront investment costs (fixed costs) and production costs (variable costs) of multiple technologies to determine the optimal mix when no technical constraints are considered. 
In recent years, much effort has been made to develop enhanced versions of SCM to estimate the optimal mix considering technical details. 
The start-up cost, thermal cycling, and other miscellaneous short-term operation of thermal generation are modeled in \cite{batlle13,zhang15}. 
Furthermore, 
ancillary services are considered in \cite{zhang15:ancillary}, hydro scheduling is modeled in \cite{staffell16}, existing capacities are considered in \cite{zhang17,guner18}, and planned outage is considered in \cite{zhang19}. 
All of these SCMs have been developed for estimating the optimal capacity mix in a bulk power system. 

\IEEEpubidadjcol

% profitability analysis
%
The purpose of this paper is to develop an SCM for economic analysis of photovoltaic self-consumption to estimate optimal sizes of PV and battery storage systems at the demand side. 
It has been known that the framework of SCM can be applied to optimal sizing of various sources to satisfy load demands \cite{stoft02,priya14}. 
However, there has been no existing SCM that can take into account intermittent nature of renewable energy sources. 
In this paper, to consider the fact that the amount of electricity generated by PV depends on the solar irradiation at each time, we newly adopt a time-varying scheme to decompose the load curve, whereas time-invariant load slices have been used in existing SCMs to decompose the original optimization problem into sub-problems. 
It is shown that the proposed SCM can be used to estimate the optimal solution of a Linear Programming (LP) problem. 
While an estimation algorithm has been proposed in \cite{ru14} for the upper bound of the battery size, estimation of the optimal value of the battery size as well as the PV size has not been achieved. 

% preliminary work
%
A preliminary version of the proposed SCM was presented in \cite{hoshino21}, but periodicity in the input data was assumed on a time scale of 24 hours. 
The proposed SCM that can handle arbitrary shape and length of data has been partly presented in the conference publication \cite{hoshino22:nolta}. % without verification using long-term data. 
This paper adds to it the following important contributions:
the proposed algorithm is refined and an analytical expression for estimating economic benefit of battery is newly introduced; 
a systematic verification study is presented by comparing the results with LP optimal solutions under various setting of parameters; %, while limited scenarios were examined in the previous conference paper; 
and the effectiveness of the proposed SCM is examined in terms of the computational time.  
%and %a numerical example is reported implying that there is a certain level of feed-in remuneration price that maximizes the economic benefit of battery. 

% structure of this paper
%
This paper is structured as follows. 
\secref{sec:lp} introduces the optimal sizing problem for PV and battery storage systems studied in this paper. 
The proposed SCM is presented in \secref{sec:scm} after a brief review of the existing SCMs. 
\secref{sec:verification} provides a verification study for the estimation accuracy of the proposed SCM under various setting of parameters. 
\secref{sec:discussion} discusses advantages and disadvantages of the proposed SCM, and \secref{sec:conclusion} concludes this paper. 


%%%%%%%%%%%%%%%%%%%%%%%%%
\section{Optimal Sizing Problem} \label{sec:lp}

%
The proposed SCM is intended to estimate the optimal solution of the LP problem introduced in this section. 
The problem formulation follows from \cite{schittekatte18} and is based on the following assumptions: 
\begin{itemize}
    \item The sizes of PV and battery storage systems are continuous decision variables, while several discrete sizes would be the options in practice. \item The electricity generated by a unit capacity of PV is proportional to the solar irradiation, and the effect of inverter clipping due to oversizing of a solar array is neglected. 
    \item The profiles of load demand and solar irradiation are completely known a priori to optimize the operation of the battery storage system. 
    \item The electricity retail price per unit of energy consumed is fixed, and other practical tariff structures such as three-part tariff are not considered. 
\end{itemize}


%
The optimal sizes of PV and battery are determined with the aim of minimizing the annualized electricity-related costs of a household while satisfying the load demand at each time. 
This problem can be viewed as a multi-scale decision-making problem, where the decision variables include both the long-term decisions on the installation capacities $v_\mathrm{pv}$ of PV and $v_\mathrm{bat}$ of battery and the short-term operational decisions on the amounts of electricity purchased from and sold to the grid at each time step $k$, denoted by $u_{\mathrm{buy},k}$ and $u_{\mathrm{sell},k}$ (for $k = 1, \, 2, \,\dots , \, N_\mathrm{t}$), and the amounts of charging and discharging of battery, denoted by $u_{\mathrm{bin},k}$ and $u_{\mathrm{bout},k}$, respectively, where $N_\mathrm{t}$ stands for the number of time steps. 
\Figref{fig:optimal_sizing} shows the overview of the PV and battery sizing problem considered in this paper. 
The meaning of parameters and these typical values are listed in \tabref{tab:parameters}. 


%%% Optimal sizing %%%%%%%%%%
\begin{figure}[!t]
 \centering
 \vspace{2mm}
 \includegraphics[width=\linewidth]{figs/optimal_sizing.pdf} 
 \caption{Optimal PV and battery sizing problem} \label{fig:optimal_sizing}
\end{figure}

%%%%
\begin{table}[t]% 
\footnotesize
\caption{Parameters of the optimization problem}
\label{tab:parameters}
\centering
\begin{minipage}{\linewidth}
\begin{tabular}{c|l|c}
\hline
Symbol & Meaning &  Value \\
\hline
 $C_\mathrm{pv} $ & Annual fixed cost of PV  & $\SI{12000}{yen/(kW \cdot yr)}$ \footnote{It represents annualized cost and corresponds to, e.g., the case where the PV unit price of \SI{200000}{yen/kW} with the lifetime of 25 years, the discount rate of 0\%, and the maintenance and operation cost of \SI{4000}{yen/(kW\cdot yr)}.} \\
 $C_\mathrm{bat} $ &  Annual fixed cost of battery &  $\SI{4400}{yen/(kWh \cdot yr)}$ \footnote{It corresponds to, e.g., the case where the unit price of the battery is \SI{66000}{yen/kWh}, the lifetime of 15 years, and the discount rate of 0\%.} \\
 $P_\mathrm{buy} $ & Price for buying from the grid &  $\SI{26.0}{yen/kWh}$ \\
 $P_\mathrm{sell}$ & Price for selling to the grid &  $\SI{6.0}{yen/kWh}$ \\
 $E_\mathrm{chg} $ & Efficiency of charging  & $0.9$ \\
 $E_\mathrm{dis} $ & Efficiency of discharging & $0.9$ \\
 $E_\mathrm{pv}$   & Efficiency of PV system & $\SI{0.78}{m^2/kW}$ \\
 $S_k$ & Solar irradiation at time $k$ & Given as input data \\
 $D_k$ & Electricity consumption at time $k$ & Given as input data \\ 
 $M_\mathrm{pv}$   & Maximum capacity of PV  & $\SI{10}{kW}$ \\
 $F_\mathrm{anu}$  & Scaling factor for annualization & $365/\text{\# of days analysed}$ \\
\hline
\end{tabular}
\end{minipage}
\end{table}

% regarding short-term operation, including the amounts of electricity purchased from and sold to the grid at each time step $k$, denoted by $u_{\mathrm{buy},k}$ and $u_{\mathrm{sell},k}$ (for $k = 1, \, 2, \,\dots , \, N_\mathrm{t}$), and the amounts of charging and discharging of battery, denoted by $u_{\mathrm{bin},k}$ and $u_{\mathrm{bout},k}$, respectively, are  affected by the long-term decision on the installation capacities $v_\mathrm{pv}$ of PV and $v_\mathrm{bat}$ of battery. 
The objective function $L$  to be minimized is given as follows:
\begin{align} \label{eq:cost_function}
 L = 
   & F_\mathrm{anu}  \sum_{k=1}^{N_\mathrm{t}}  (P_\mathrm{buy} u_{\mathrm{buy},k} - P_\mathrm{sell} u_{\mathrm{sell},k} )   \notag \\  
   &+ C_\mathrm{pv} v_\mathrm{pv} + C_\mathrm{bat} v_\mathrm{bat} 
\end{align}
where the parameters $P_\mathrm{buy}$ and $P_\mathrm{sell}$ stand for the prices of electricity purchased from and sold to the grid, respectively, and the electricity trading costs are annualized using a scaling factor $F_\mathrm{anu}$. 
The parameters $C_\mathrm{pv}$ and  $C_\mathrm{bat}$ stand for the fixed costs (mainly installation costs) of PV and battery per unit installation amount per year, respectively. 


%
The operation of the battery storage system is subject to the following  state equation: 
\begin{align}
& x_{\mathrm{soc}, k+1} =  x_{\mathrm{soc}, k} + E_\mathrm{chg}  u_{\mathrm{bin},k} - \dfrac{1}{E_\mathrm{dis}} u_{\mathrm{bout},k}, \quad \forall k  
\end{align}
where $x_{\mathrm{soc}, k}$ stands for the state of charge (SOC) of the battery storage at the time step $k$, and the parameters $E_\mathrm{chg}$ and $E_\mathrm{dis}$ for the efficiencies of charging and discharging of battery, respectively. 
In addition, the following constraints need to be addressed: 
\begin{align}
  & x_{\mathrm{soc},k} \le v_\mathrm{bat},  \quad \forall k,  \label{eq:soc_max} \\
  &  x_{\mathrm{soc}, N_\mathrm{t}+1 } = x_{\mathrm{soc}, 1} \label{eq:bondary_condition}
\end{align}
where \eqref{eq:soc_max} imposes an upper bound of SOC being subject to the long-term decision $v_\mathrm{bat}$, and \eqref{eq:bondary_condition}  is the boundary condition.  
Besides, the electricity demand should equal supply at each time step, and the following equation holds:
\begin{align} \label{eq:energy_balance}
  & u_{\mathrm{buy},k} - u_{\mathrm{sell},k}  -  u_{\mathrm{bin},k} + u_{\mathrm{bout},k} = D_k - S_k E_\mathrm{pv} v_\mathrm{pv}, \, \forall k  %\\ 
\end{align}
where $D_k$ stands for the electricity demand at the time step $k$, and $S_k$ for the solar irradiation. 
The parameter $E_\mathrm{pv}$ stands for the amount of electricity generated per unit capacity of PV per unit solar irradiation, and it depends on the efficiencies of the PV panel and the power conditioning system.  
An upper bound is imposed on the size of PV: 
\begin{align}
   v_\mathrm{pv} \le M_\mathrm{pv}.
\end{align}
This condition is needed to prevent the LP problem becomes unbounded when the PV price $C_\mathrm{pv}$ is lower than a ceratain value. 
Finally, the following condition forces all the decision variables to be non-negative:
\begin{align}
    v_\mathrm{pv}, \, v_\mathrm{bat}, \, u_{\mathrm{buy},k}, \, u_{\mathrm{sell},k},\, u_{\mathrm{bin},k}, \, u_{\mathrm{bout},k}, \, x_{\mathrm{soc},k}  \ge 0, \, \forall k. 
\end{align}

%
Note that the dimension of the above LP problem increases with the increase in the length of data analysed. 
For an analysis over 1 year with a temporal resolution of 1 hour, the resultant LP problem becomes $2 + 8760 \times 5 = 43,802$ dimensional optimization problem, and may take hours or days to obtain the result by a general-purpose optimization solver.  


% %%%%%%%%%%%%%%%
% \begin{figure}[t]
%  \centering
%  \includegraphics[width=1.0\linewidth]{../simulation/prosumer_optimization_plot}
%  \caption{Optimal operation obtained  by LP solver} \label{fig:time_course}
%  \vspace{-2mm}
% \end{figure}

%
% Here we provide a numerical example of the above optimal sizing problem. 
% Note that the problem given by Eqs.~\eqref{eq:cost_function} to \eqref{eq:energy_balance} is a linear programming (LP) problem, and the optimal solution can be obtained by a standard LP solver. 
% In this study, we used an inter-point method provided by the python package SciPy (Version 1.5.2). 
% The setting of parameters is shown in \tabref{tab:parameters}. 
% For the demand $D_k$ and the PV generation $S_k$, we prepared time-series data with 15 minute intervals ($\Delta T=0.25\,\mathrm{h}$), which will be shown with the results explained below (\figref{fig:time_course}). 
% %
% With this setting, the optimal installation capacity is $\SI{5.43}{kW}$ for PV and $\SI{5.57}{kWh}$ for battery. 
% The other results regarding the short-term operation are shown in \figref{fig:time_course}. 
% The upper part of the figure shows the time courses of the electric power generated by PV ($S_k v_\mathrm{pv}/\Delta T$),  the electric power sold to the grid ($u_{\mathrm{sell},k}/\Delta T$), and the electric power charged to the battery ($u_{\mathrm{bin},k}/\Delta T$). 
% It can be seen that PV is generating electricity in day time, and the generated electricity is partly charged to the battery and sold to the grid (and the rest is consumed at the household).  
% The middle part of the figure shows the time courses of the electric power consumed at the household ($D_k/\Delta T$), the electric power purchased from the grid ($u_{\mathrm{sell},k}/\Delta T$), and the electric power discharged from the battery ($u_{\mathrm{bat},k}/\Delta T$). 
% The demand at nighttime is covered by buying electricity and discharging from the battery. 
% Finally, the lower part of the figure shows the time course of $x_{\mathrm{soc},k}$, i.e., the state of charge (SOC)  of the battery. 
% It can be seen that the battery is fully charged and discharged every day, except for the 5th day.  
% This observation will be used to develop the proposed SCM in \secref{sec:scm}. 


%%%%%%%%%%%%%%%%%%%%%%%%%
\section{Screening Curve Method} \label{sec:scm}

In this section, we introduce the framework of SCM with a brief review of existing SCMs in \secref{sec:scm_thermal}. 
The proposed SCM for economic analysis of PV self-consumption is presented in  \secref{sec:scm_der}. 

%%%%%
\subsection{Review of Existing SCMs} \label{sec:scm_thermal}

%
The classical SCM \cite{phillips69,lee78} requires two sets of input data: costs and a load curve for a target year. 
Cost data includes the annualized upfront investment cost $C_\mathrm{fix}$  per unit capacity (in \si{kW} or \si{MW}) and the variable production costs $C_\mathrm{var}$ per unit production (in \si{kWh} or \si{MWh}). 
The total annualized cost of a unit capacity of generation technology is represented as follows: 
\begin{align}
 c =C_\mathrm{fix}+C_\mathrm{var} T  \label{eq:cost_classical}
\end{align}
where $T$ stands for the firing hours of the considered thermal generation technology. 
The cost representation in \eqref{eq:cost_classical} is used to draw the so-called \emph{screening curve} for each generation technology as shown in the upper part of \figref{fig:classical_scm}, where the horizontal axis is taken as the capacity factor $f$ defined as the ratio of the firing hours $T$ to the total hours in a year ($f = T/\SI{8760}{h}$). 
\figref{fig:classical_scm} illustrates examples of the screening curves for three candidate technologies: base-load (such as coal and nuclear), combined-cycle gas turbine (CCGT) and simple-cycle combustion turbine (GT).
Then, the intersections of these screening curves separate the regions of $f$ in which the different generation technologies become optimal. 
The least-cost generation mix for a given load curve can be obtained by combining them with the load-duration curve as shown in the lower part of  \figref{fig:classical_scm}, where the load-duration curve is obtained by sorting a chronological load curve in descending order of the load level. 
Since the load-duration curve maps a load level on the vertical axis to a capacity factor on the horizontal axis, and the screening curves map the capacity factor to the annual costs of generation alternatives, the optimal generation capacity mix can be read from the vertical axis of the lower graph as shown in \figref{fig:classical_scm}. 
This analysis gives the optimal solution to the least-cost problem with no technical constraints~\cite{stoft02}. 

%%% Classical SCM %%%%%%%%%%
\begin{figure}[!t]
 \centering
 \includegraphics[width=0.78\linewidth]{./figs/scm_traditional}
 \caption{The classical SCM to determined the optimal thermal generation mix when no technical constraints are considered. } \label{fig:classical_scm}
\end{figure}


%
In recent years, enhanced versions of SCM have been developed to estimate the optimal generation mix considering more detailed situations.
For this purpose, it was proposed in \cite{batlle13} to relate the load level directly to the generation cost. 
With this scheme, a chronological load curve is discretized into multiple load slices as shown in \figref{fig:load_slice}. 
For example, if the width of a load slice is set to be $\SI{0.1}{MW}$, then a load curve with a peak of $\SI{80}{MW}$ will be divided into $800$ load slices.
Then, the original least-cost problem will be decomposed into sub-problems corresponding to these slices, and a heuristic optimization is performed to determine which generation technology is used in each slice. 
This is done by calculating the annualized total cost to balance the demand in each load slice with a production by each generation technology to draw screening curves (cost curves) as shown in \figref{fig:enhanced_scm}. 
The estimated optimal generation mix can be derived by simply choosing the least-cost technology for each slice. 
With this approach, in addition to the fixed and variable costs in \eqref{eq:cost_classical}, a simple representation of the start-up costs can be obtained by counting the number of segments in each load slice and summing up the cost for each start up.  
In the example shown in \figref{fig:load_slice}, the costs for three re-starts will be added. 
Furthermore, the start-up can be separated into hot starts and cold starts depending on the shut-down duration before the restart, which is evaluated from the distance between adjacent segments \cite{batlle13,zhang15}. 
With the above framework of enhanced SCM, other technical details such as thermal cycling \cite{zhang15}, ancillary services \cite{zhang15:ancillary}, 
hydro scheduling \cite{staffell16}, existing capacities \cite{zhang17,guner18}, and planned outages \cite{zhang19} have been taken into account.


%%% Enhanced SCM %%%%%%%%%%%%
\begin{figure}[!t]
    \begin{minipage}[b]{\linewidth}
    	  \centering
        \includegraphics[width=0.95\linewidth]{./figs/scm_loadslice}
        \subcaption{Definition of load slice } \label{fig:load_slice}
    \end{minipage}
%%%
   \begin{minipage}[b]{\linewidth}
        \centering
        \includegraphics[width=0.8\linewidth]{./figs/scm_improved}
        \subcaption{Example of screening curves (cost curves)} \label{fig:enhanced_scm} 
   \end{minipage}
   \caption{The framework of enhanced SCMs to estimate the optimal generation mix when technical constraints are considered. }
\end{figure}





%%%%%%%%%%%%
\subsection{Proposed SCM for PV and battery} \label{sec:scm_der}

Based on the framework of enhanced SCM introduced above, here we develop the proposed SCM for economic analysis of PV self-consumption. 

%%%%
\subsubsection{Overview}

The existing SCMs have been developed mainly for thermal generation technologies and are not able to take into account intermittent nature of renewable energy sources. 
To consider the fact that the amount of electricity generated by PV depends on the solar irradiation at each time, we newly adopt a time-varying scheme for load slicing. 
Specifically, as shown in \figref{fig:pv_load_slice}, the load curve is discretized into multiple load slices using the generation profile of a unit capacity of PV, which corresponds to the width of the load slice.  
As shown in the lower part of the figure, the PV generation shaded in red will be directly consumed and the that shaded in blue becomes surplus electricity, which will be sold to the grid or charged to the battery to increase self-consumption. 
Note that the load demand during nighttime does not directly considered in this method, since there is no power generation by PV at night.  

%
With this modified approach of load slicing, the overall procedure of the analysis is followed from the framework of enhanced SCM and is given as follows:
\begin{enumerate}[(i)]
 \setlength{\parskip}{0cm} 
 \setlength{\itemsep}{0.1cm}  
 \item Discretize the chronological load curve into time-varying load slices, by using the generation profile of a unit capacity of PV, denoted by $\Delta P$, corresponding to the width of a load slice.   
 \item Draw screening curves by calculating the annualized cost to satisfy the demand in each slice by each of the following three technology options: a) buying electricity from the grid, b) installing PV without battery, and c) installing PV with battery. 
 \item Estimate the optimal PV size by counting the number of slices where the option b) or c) becomes the least-cost technology. 
 \item Estimate the optimal battery size by summing up the required amount of battery calculated for each slice where the option c) becomes the least-cost technology.  
\end{enumerate}
The detailed calculations of the screening curves and required amount of battery are presented in the rest of this section. 

%%%%%%%%%%%%%%%
\begin{figure}[t]
 \centering
 \includegraphics[width=0.93\linewidth]{./figs/pv_loadslice}
 \caption{Time-varying scheme of load slicing in the proposed SCM} \label{fig:pv_load_slice}
\end{figure}


%%%%
\subsubsection{Calculation of Screening Curves}

For the technology option a), the annualized cost $c_{\mathrm{grid},i}$ of purchasing electricity from the grid for the slice level $i$ can be calculated as follows: 
\begin{align} \label{eq:cost_grid}
 c_{\mathrm{grid}, i} = F_\mathrm{anu} \sum_{k=1}^{N_\mathrm{t}} P_\mathrm{buy}  q_{\mathrm{load}, i,k}
\end{align}
where $q_{\mathrm{load},i,k}$ stands for the amount of load demand of the slice $i$ at the time step $k$. 
The total amount of electricity $\sum_k q_{\mathrm{load}, i,k}$ corresponds to the area shaded in red in \figref{fig:pv_load_slice}. 
The parameters $P_\mathrm{buy}$ and $F_\mathrm{anu}$ are the same with those defined in \secref{sec:lp} and stand for the electricity retail price and the annualization factor, respectively. 


%
For the technology option b), the annualized cost $c_{\mathrm{pv},i}$ of installing PV without battey for the slice $i$ is given as follows: 
\begin{align} 
 c_{\mathrm{pv},i} = C_\mathrm{pv} \Delta P - F_\mathrm{anu}  \sum_{k=1}^{N_\mathrm{t}} P_\mathrm{sell} q_{\mathrm{sur},i,k} \label{eq:cost_pv}
\end{align}
where $q_{\mathrm{sur},i,k}$ stands for the amount of surplus electricity for the slice $i$ at the time step $k$. 
The first term of the right-hand side of \eqref{eq:cost_pv} represents the investment cost for PV, and the second term the revenue by selling surplus electricity to the grid. 
The total amount of electricity $\sum_k q_{\mathrm{sur},i,k}$ corresponds to the area shaded in blue in \figref{fig:pv_load_slice}. 
The parameters $C_\mathrm{pv}$ and $P_\mathrm{sell}$ stand for the annualized fixed cost of PV and the price of selling electricity to the grid, respectively. 


%
For the technology option c), the cost $c_{\mathrm{pv/bat},i}$ of installing PV and using battery for increasing self-consumption is given by 
\begin{align} \label{eq:cost_pv_battery}
 c_{\mathrm{pv/bat},i} = & \,
   C_\mathrm{pv}\Delta P  + C_\mathrm{bat} q_{\mathrm{bat},i}  \notag \\ 
   & - F_\mathrm{anu} \sum_{k=1}^{N_\mathrm{t}} P_{\mathrm{sell}} \left(  q_{\mathrm{sur},i,k} - q_{\mathrm{chg},i,k} \right)  \notag \\
  &  - F_\mathrm{anu} \sum_{k=1}^{N_\mathrm{t}} P_\mathrm{buy}  E_\mathrm{dis} E_\mathrm{chg}  q_{\mathrm{chg},i,k}   
\end{align}
where $q_{\mathrm{bat,}i}$ for the required amount of battery for the slice $i$, and $q_{\mathrm{chg},i,k}$ stands for the amount of charged electricity for the slice $i$ at the time $k$. 
The procedure of estimation of these values will be presented later. 
The second term of the right-hand side of \eqref{eq:cost_pv_battery} represents the investment cost for battery, the third term the revenue by selling surplus electricity, and the forth term  the economic benefit due to the increase in self-consumption by charging surplus electricity and use it in e.g. nighttime. 
While the profile of charging is modeled in the proposed method, the discharging will not be explicitly considered. 

%%%%%%%%%%
\subsubsection{Estimation of Battery Capacity and Charging Profile}

%
To estimate the values of $q_{\mathrm{bat},i}$ and $q_{\mathrm{chg},i,k}$, we will consider the economic benefit of battery in a sequential manner. 
To this end, let us define an index $j$ that indicates the ascending order of the days with respect to the total amount of surplus PV generation in each day. 
That is, as shown in \figref{fig:def_j}, the day with the index $j=1$ has the smallest  amount of surplus electricity, and the day with $j=N_\mathrm{d}$ has the largest amount, where $N_\mathrm{d}$ stands for the number of days analyzed. 
This order may vary depending on the slice $i$, but we do not explicitly show it in the notation. 
Then, the amount of battery required to store all the surplus electricity of the $j$-th day can be given as 
\begin{align}
 Q_{\mathrm{bat}, i,j} := E_\mathrm{chg} \sum_{k \in \mathcal{K}_j}  q_{\mathrm{sur},i, k}     
\end{align}
where the symbol $\mathcal{K}_j$ stands for the set of indices of the time step $k$ belonging to the $j$-th day.
Note that this assumes that there is no discharging of battery during day time, otherwise the required amount of battery could be smaller.  

%
Now the economic benefit of the battery can be sequentially estimated for each slice. 
First, consider to install a battery storage system of the amount of $Q_{\mathrm{bat},i,1}$ for the slice $\# i$. 
Then, the surplus electricity can be charged up to this amount every day (see \figref{fig:j=1}), and  
the economic benefit $B_1$ associated to this use of battery can be written as
\begin{align}
 B_1 =
  &  F_\mathrm{anu} N_\mathrm{d} P_\mathrm{buy} E_\mathrm{dis} Q_{\mathrm{bat},i, 1} \notag \\ 
  & - C_\mathrm{bat} Q_{\mathrm{bat},i, 1}  - F_\mathrm{anu} N_\mathrm{d}  P_\mathrm{sell} Q_{\mathrm{bat},i, 1}/E_\mathrm{chg}
\end{align}
where the first term represents the amount of payment of electricity that can be avoided, the second term the installation cost of the battery, and the third term the lost revenue that could have been earned by selling this electricity. 
Next, for $2  \le j \le N_\mathrm{d}$,  if the amount of battery is increased to $Q_{\mathrm{bat},i,j}$, the marginal economic benefit $B_j$ due to the incremental amount of battery defined by $\Delta Q_{\mathrm{bat},i,j} := Q_{\mathrm{bat},i,j} - Q_{\mathrm{bat},i,j-1}$ can be given by 
\begin{align}
 B_j = 
 &  F_\mathrm{anu} (N_\mathrm{d}-j+ 1)  P_\mathrm{buy} E_\mathrm{dis}\Delta Q_{\mathrm{bat},i, j}  \notag \\
 & - C_\mathrm{bat} \Delta Q_{\mathrm{bat},i, j}  \notag \\  &  - F_\mathrm{anu} (N_\mathrm{d} -j+ 1) P_\mathrm{sell} \Delta Q_{\mathrm{bat},i, j}/E_\mathrm{chg}.   \label{eq:b_j}
\end{align}
\figref{fig:j=2-3} shows examples for the cases of $j = 2$ and $j=3$. 
Note that the economic benefit of the incremental battery capacity $\Delta Q_{\mathrm{bat},i,j}$ can occur only for $N_\mathrm{d} -j + 1$ days because the surplus electricity over the 1st to $(j-1)$-th days is already accounted in $B_1, \dots,\, B_{j-1}$. 
By rewriting \eref{eq:b_j}, we have 
\begin{align}
 B_j = 
 & \{ F_\mathrm{anu} (N_\mathrm{d}-j+ 1) (P_\mathrm{buy} E_\mathrm{dis}-P_\mathrm{sell} /E_\mathrm{chg})   \notag \\
 & - C_\mathrm{bat} \} \Delta Q_{\mathrm{bat},i, j}. 
\end{align}
Thus, the value of $ B_j/\Delta Q_{\mathrm{bat}i,j}$ decreases as the increase in $j$ when $P_\mathrm{buy} > P_\mathrm{sell} /(E_\mathrm{chg}E_\mathrm{dis})$ holds\footnote{Note that this condition holds for the setting in \tabref{tab:parameters} and is necessary for a self-consumption policy to work. Otherwise $B_j < 0$ for all $j = 1, \dots, N_\mathrm{d}$.}.  
%
%Thus, the value of $ B_j/\Delta Q_{\mathrm{bat}i,j}$, which determines the sign of $B_j$, 
%
%%%%%%%%%%%%%
\begin{figure}[!t]
 \centering
 \subcaptionbox{Definition of the index $j$ \label{fig:def_j}}{
 \includegraphics[width=0.8\linewidth]{figs/pv_loadslice_def_j.pdf} 
  }\\[5mm]
 \subcaptionbox{Charged electricity for the case of $j=1$ \label{fig:j=1}}{
   \includegraphics[width=0.8\linewidth]{figs/pv_loadslice_j_1.pdf} }\\[5mm]
 \subcaptionbox{Charged electricity for the cases of $j=2$ and $3$ \label{fig:j=2-3}}{
   \includegraphics[width=0.8\linewidth]{figs/pv_loadslice_j_2-3svg.pdf} }
 \caption{Examples of charging profile of electricity for each value of $j=1, \, 2$, and $3$.} \label{fig:profile}
\end{figure}
%
%
%With this sequential estimation of the economic benefit, 
Since the economic benefit of the incremental amount of battery $\Delta Q_{\mathrm{bat}, i, j}$ occurs when $B_j > 0$, the installation capacity of battery $q_{\mathrm{bat},i}$ in \eqref{eq:cost_pv_battery} can be given by 
\begin{align}
    q_{\mathrm{bat},i} := Q_{\mathrm{bat},i, J} = E_\mathrm{chg} \sum_{k \in \mathcal{K}_J} q_{\mathrm{sur}, i, k} \label{eq:battery_amount}
\end{align}
with
\begin{align}
    J := \max \{ j \in \mathbb{Z}_{[0, N_\mathrm{d}]} ~|~   B_j \ge 0  \} 
\end{align}
where $\mathbb{Z}_{[0, N_\mathrm{d}]}$ stands for the set of integers ranging from $0$ to $N_\mathrm{d}$, and for $j=0$, we define $B_0 := 0$ and $Q_{\mathrm{bat},i, 0} := 0$. 
From the observation that the value of $B_j/\Delta Q_{\mathrm{bat},i,j}$ decreases with $j$, the value of $J$ can be analytically described as 
\begin{align}
    J = \max\left( 0, \, \left\lfloor N_\mathrm{d}+1-\dfrac{C_\mathrm{bat} E_\mathrm{chg} }{F_\mathrm{anu}(P_\mathrm{buy}E_\mathrm{dis} E_\mathrm{chg} -P_\mathrm{sell} )} \right\rfloor \right) \label{eq:J_analytic}
\end{align}
where $\lfloor \cdot \rfloor$ stands for the floor function. 
%Note that the value of $J$ calculated by \eqref{eq:J_analytic} does not exceed $N_\mathrm{d}$. 

Given $J$ and $q_{\mathrm{bat},i}$ above, we can arbitrarily determine  the charging profile $\{  q_{\mathrm{chg}, i, k} \}_{k=1}^{N_\mathrm{t}}$ used in \eqref{eq:cost_pv_battery}  as long as the total charged amount of electricity at each day does not exceed the upper limit given by $q_{\mathrm{bat},i}/E_\mathrm{chg}$. 
Furthermore, by defining the total amount of electricity $Q_{\mathrm{chg}, i}$ charged to the battery as 
\begin{align}
 Q_{\mathrm{chg},i} =  \sum_{j=1}^{J} \sum_{k \in \mathcal{K}_j } q_{\mathrm{sur},i, k} + (N_\mathrm{d} -J) \sum_{k \in \mathcal{K}_J }  q_{\mathrm{sur},i, k},  \label{eq:Q_charge}
\end{align}
the cost \eqref{eq:cost_pv_battery} can be rewritten as 
\begin{align}
 c_{\mathrm{pv/bat},i}  = & \,
   C_\mathrm{pv}\Delta P  + C_\mathrm{bat} E_\mathrm{chg} \sum_{k \in \mathcal{K}_J} q_{\mathrm{sur}, i, k} \notag \\ 
   & - F_\mathrm{anu} \sum_{k=1}^{N_\mathrm{t}} P_{\mathrm{sell}} q_{\mathrm{sur},i,k} \notag \\
  &  - F_\mathrm{anu} (P_\mathrm{buy}  E_\mathrm{dis} E_\mathrm{chg}-P_\mathrm{sell})  Q_{\mathrm{chg},i}. 
  \label{eq:cost_battery_final}
\end{align}
Thus, the screening curve of $c_{\mathrm{bv/bat},i}$ can be directly calculated from the profile $\{q_{\mathrm{sur}, i,k} \}_{k=1}^{N_\mathrm{t}}$ by using \eqref{eq:J_analytic}, \eqref{eq:Q_charge}, and \eqref{eq:cost_battery_final} without calculating the values of $B_j$. 


%%%%
\section{Verification Study} \label{sec:verification}

This section examines the estimation accuracy of the proposed SCM by comparing the results with the LP problem presented in \secref{sec:lp}. 
The python source code for the implementation of the proposed SCM is available upon request. 

%%%
\subsection{Base-Case Analysis}

The base-case analysis is performed with the setting of parameters shown in \tabref{tab:parameters}. 
A synthetic hourly load profile was generated from real measurement data of a household in Japan \cite{database} by concatenating data for February (Winter), June (Rainy season), and August (Summer) to construct a three-month data including different seasons in a year. 
A generation profile for a unit capacity of PV was generated by using solar irradiation data of METPV20 \cite{metpv20} for corresponding days and the location of the load profile. 
The daily profiles of these input data are shown in \figref{fig:input_data}. 
It can be seen that the solar irradiation is generally low in winter, and there are many days with low solar irradiation in the rainy season. 
On the other hand, demand is higher during winter than other seasons. 

%
The screening curves (cost curves) obtained by the proposed SCM is shown in \figref{fig:cost_curves}. 
The \emph{green} and \emph{blue} lines in the figure show the cost curves for using PV with and without battery storage calculated by \eqref{eq:cost_battery_final} and \eqref{eq:cost_pv}, respectively, and the \emph{purple} line the cost curve for buying electricity from the grid calculated by \eqref{eq:cost_grid}. 
The cost curve of ``PV+battery" (green line) takes least value from $l= 0$ to $\SI{5.19}{kW}$ and has an intersection with the cost curve of ``Grid" (purple line) at the point shown by the cross symbol ($\times$) in the figure.  
Thus, the installation capacity of PV can be estimated as $\SI{5.19}{kW}$. 
The amount of battery can be calculated by summing up $q_{\mathrm{bat}, i}$ given by \eqref{eq:battery_amount} and given as $\SI{4.01}{kWh}$. 
The optimal PV and battery capacities obtained by solving the LP problem are $\SI{5.15}{kW}$ and $\SI{4.06}{kWh}$, respectively, and the proposed SCM approximates these optimal values appropriately. 

%%%%%%%%%%%%%
\begin{figure}[!t]
 \centering
 \subcaptionbox{PV data}{
 \includegraphics[width=0.97\linewidth]{data/base_case_input_pv}
  }\\[2mm]
 \subcaptionbox{Demand data}{
   \includegraphics[width=0.97\linewidth]{data/base_case_input_demand} }
 \caption{Daily profiles of the input PV and demand data.} \label{fig:input_data}
\end{figure}

%%%%%%%%%%%%%
\begin{figure}[!t]
 \centering
 \includegraphics[width=0.85\linewidth]{./data/base_case_cost_curve}
 \caption{Cost curves calculated by the proposed SCM} \label{fig:cost_curves}
\end{figure}

%%%%%%%%%%%%%
\begin{figure}[!t]
 \centering
 \subcaptionbox{Electricity sold to the grid}{
 \includegraphics[width=0.97\linewidth]{./data/base_case_sell_profile}
  }\\[2mm]
 \subcaptionbox{Electricity charged to the battery storage}{
   \includegraphics[width=0.97\linewidth]{./data/base_case_chg_profile} }
 \caption{Amounts of electricity soled and charged for each day.} \label{fig:profile}
\end{figure}

% %%%%%%%%%%%%%
% \begin{figure}[!t]
%  \centering
%  \includegraphics[width=0.98\linewidth]{./data/base_case_lp_profiles}
%  \caption{Optimal operation of battery obtained by LP.} \label{fig:battery_lp}
% \end{figure}

% Also, \figref{fig:battery_lp} shows the profiles including discharging under the optimal operation of the battery.  
% The red line shows the 
% %
% Note that these information 

\Figref{fig:profile} shows the daily profiles of (a) electricity sold to the grid and (b) electricity charged to the battery.  
The profiles obtained by the SCM (shown by \emph{blue}) capture the optimal profiles by the LP problem (shown by \emph{orange}). 
It can be seen that, in many days, the amount of charged electricity takes its maximum value ($\SI{4.06}{kWh}/E_\mathrm{chg}=\SI{4.51}{kWh}$) meaning that the battery is fully charged. 
Although the SCM does not consider the profile of discharging, the ratio of the amount of sold electricity and the amount of charged electricity is properly reproduced throughout the analysis period, including the days when the amount of charge is low as typically seen in winter or rainy season. 



%%%
\subsection{Estimation Accuracy under Various Settings}

Next, the estimation accuracy of the proposed SCM is examined under various settings of parameters. 
To reduce the computational time for obtaining the optimal solutions of the LP problem, the load demand and PV generation profiles were shortened by taking the first ten days of each month in the three-month data used above to construct a one-month long data.
Among the parameters shown in \tabref{tab:parameters}, the four parameters $C_\mathrm{pv}$, $C_\mathrm{bat}$, $P_\mathrm{buy}$, and $P_\mathrm{sell}$, which are directly relevant to the regulatory discussion of the PV self-consumption, are changed from the setting of the base-case analysis.  


%%%%%%%%%%%%%
\begin{figure}[!t]
 \centering
 \subcaptionbox{Changes in $C_\mathrm{pv}$   \label{fig:diff_pv}}{
   \includegraphics[width=0.87\linewidth]{./data/change_Cfp} }\\[2mm]
 \subcaptionbox{Changes in $C_\mathrm{bat}$ \label{fig:diff_bat}}{
  \includegraphics[width=0.87\linewidth]{./data/change_Cfbat} }
 \caption{Results with various setting of PV and battery prices. } \label{fig:diff_fixed_costs}
\end{figure}

\Figref{fig:diff_fixed_costs} shows the comparison between the estimated and optimal solutions under varying prices of PV and battery storage systems. 
\figref{fig:diff_pv} shows the results of changing the setting of the parameter $C_\mathrm{pv}$. 
The \emph{blue circles} show the PV capacities estimated by the proposed SCM, and they are close to the results of the LP problem shown by the doted line. 
Similarly, the estimated battery capacities shown by the \emph{orange triangles} are close to the optimal capacities over the entire range of $C_\mathrm{pv}$. 
%
Next, \figref{fig:diff_bat} shows the result of changing the setting of the battery fixed cost $C_\mathrm{bat}$.  
While the estimated capacity of PV is close to the optimal value over the entire range of $C_\mathrm{bat}$, the installation capacity of battery is overestimated for the region where $C_\mathrm{bat}$ is lower than $\SI{4000}{yen/kWh/yr}$.
A plausible reason for this is that, although the proposed SCM does not take into account the profile of discharging, the battery could not be sufficiently discharged each day such that there was no room for charging as much as calculated by the SCM. 
Nevertheless, the proposed SCM capture the overall trend of the changes in the optimal battery size. 

%%%%%%%%%%%%%
\begin{figure}[!t]
 \centering
 \subcaptionbox{Change in $P_\mathrm{buy}$   \label{fig:diff_pbuy}}{
   \includegraphics[width=0.87\linewidth]{./data/change_Pbuy} }\\[2mm]
 \subcaptionbox{Change in $P_\mathrm{sell}$ \label{fig:diff_psell}}{
  \includegraphics[width=0.87\linewidth]{./data/change_Psell} }
 \caption{Results with various setting of electricity prices.} \label{fig:diff_prices}
\end{figure}


%
\Figref{fig:diff_prices} shows the comparison of the estimated and optimal capacities under various setting of electricity prices. 
\figref{fig:diff_pbuy} shows the results where the retail price $P_\mathrm{buy}$ is changed. 
It can be seen that the proposed SCM captures the overall trend of the changes in the optimal values of PV and battery, while the estimation accuracy of the battery capacity deteriorates in the region where the price $P_\mathrm{buy}$ is higher than $\SI{27}{yen/kWh}$. 
As in the case of \figref{fig:diff_bat}, it start to overestimate the economic benefit of the battery when the battery capacity becomes greater than $\SI{15}{kWh}$. 
Finally, \figref{fig:diff_psell} shows the results where the price $P_\mathrm{sell}$ is changed. 
%As for the estimation accuracy, 
The estimated and optimal values are closed to each other for both PV and battery capacities. 
%From this figure, it can be seen that there is a certain level of selling price that maximizes the economic benefit of battery, feed-in remuneration



%%%%%%%%%%%%%%%%
\section{Discussion} \label{sec:discussion}

In this section, we sort out advantages and disadvantages of the proposed SCM and discuss its effectiveness. 


%%%
\subsection{Economic Analysis Using Screening Curves}

%
Here we illustrate how the cost curves obtained by the proposed SCM can be used for understanding and predicting the changes in the optimal capacities of PV and battery storage systems. % for various parameter settings. 
Although similar examples have been presented in \cite{hoshino21}, we provide recalculated results using the  SCM  proposed in \secref{sec:scm} to show its effectiveness in a self-consistent way. 

%
\Figref{fig:cost_curves_diff_pv} shows the screening curves calculated under the three settings of the price of PV: $C_\mathrm{pv} = \SI{18000}{yen/(kW\cdot yr)}$, $\SI{15000}{yen/(kW\cdot yr)}$, and $\SI{12000}{yen/(kW\cdot yr)}$. 
Since the cost curve of $c_{\mathrm{grid},i}$ is independent of $C_\mathrm{pv}$, it is common for all the above settings and shown by the \emph{black} line. 
The cost curves of $c_{\mathrm{pv},i}$ and $c_{\mathrm{pv/bat},i}$ are shown by \emph{broken} and \emph{solid} lines, respectively, and they move down as $C_\mathrm{pv}$ increases without changing their shapes. 
This is because the parameter $C_\mathrm{pv}$ only affects the first term of \eqref{eq:cost_pv} or  \eqref{eq:cost_battery_final}, and they do not depend on the slice level $i$. 
It can be seen that the intersection of the cost curves of $c_{\mathrm{pv},i}$ and $c_{\mathrm{grid},i}$ moves to the right as $C_\mathrm{pv}$ decreases, implying the increase in the installation capacities of PV and battery.   
Especially, when $C_\mathrm{pv}$ takes a value near $\SI{12000}{yen/kW/yr}$, the slope of the cost curve $c_{\mathrm{grid},i}$ at the intersection becomes small, and thus the optimal capacities of PV and battery drastically  changes as also can be seen in \figref{fig:diff_pv}. 


%
Next, \figref{fig:cost_curves_diff_bat} shows the cost curves calculated under the settings of $C_\mathrm{bat}= \SI{5500}{yen/(kWh\cdot yr)}$, $\SI{5000}{yen/(kWh \cdot yr)}$, and $\SI{4500}{yen/(kWh\cdot yr)}$. 
The cost curves of $c_{\mathrm{grid},i}$ and $c_{\mathrm{pv},i}$ are invariant and shown by \emph{solid} and \emph{broken black} lines, respectively. 
The cost curve of $c_{\mathrm{pv/bat},i}$ goes down as $C_\mathrm{bat}$ decreases. 
The difference between $c_{\mathrm{pv},i}$ and $c_{\mathrm{pv/bat},i}$ describes the economic benefit of the self-consumption owing to battery storage, and it can be explained from this figure that the capacity of the battery storage begins to increase from zero at a point around  $\SI{5000}{yen/(kWh\cdot yr)}$ as can be seen in \figref{fig:diff_bat}. 


%%%%%%%%%%%%%
\begin{figure}[!t]
 \centering
   \includegraphics[width=0.85\linewidth]{data/multiple_pvcost}  \caption{Cost curves with different values of $C_\mathrm{pv}$   \label{fig:cost_curves_diff_pv}} 
\end{figure}

%%%%%%%%%%%%%
\begin{figure}[!t]
 \centering
  \includegraphics[width=0.85\linewidth]{data/multiple_batcost} \caption{Cost curves with different values of $C_\mathrm{bat}$ \label{fig:cost_curves_diff_bat}} 
\end{figure}


%
As shown in these examples, the effect of changes in a parameter on each cost curve can be well easily predicted from their definitions in \eqref{eq:cost_grid}, \eqref{eq:cost_pv}, and \eqref{eq:cost_battery_final}, 
and similar implications can also be obtained when other parameters change. 
By considering the behavior of each cost curve separately, the changes in the resultant optimal capacities in PV and battery storage systems can be understood in an intuitive and consistent way. 



%%%
\subsection{Computational Time}


\begin{table}[!t]
 \centering
 \caption{Computational times}
 \label{tab:time}
 \begin{tabular}{ c | c c }
 Data length & Comp. Time (SCM) &  Comp. Time (LP) \\
 \hline 
 1 months & \SI{1.29}{s} & \SI{57.73}{s} \\ 
 2 months & \SI{2.46}{s} & \SI{276.15}{s} \\  
 3 months & \SI{2.93}{s} & \SI{784.18}{s} \\  
 4 months & \SI{3.49}{s} & \SI{1439.31}{s} \\  
 5 months & \SI{3.96}{s} & \SI{3289.80}{s} \\  
 6 months & \SI{4.56}{s} & \SI{6089.13}{s} \\   
\end{tabular}
\end{table}

%%%%%%%%%%%%%
\begin{figure}[!t]
 \centering
 \includegraphics[width=0.75\linewidth]{./data/computational_time}
 \caption{Computational times with SCM and LP} \label{fig:time}
\end{figure}




%
Since the computations of cost curves in the proposed SCM only requires numerical operations on arrays, it is faster than solving a least-cost problem using a general-purpose optimization solver.
\tabref{tab:time} and \figref{fig:time} show the comparison of the  computational times between the proposed SCM and the LP problem. %presented in \secref{sec:lp}. 
These results have been obtained in simulations with a Core i9-10850K based desktop computer. 
%Note that the vertical axes in the figure are different between these methods, and 
The proposed SCM is much faster than solving the LP problem. 
The LP problem has been solved by using an interior-point method provided by \texttt{linprog} in the python package SciPy, and the computational time could be reduced by using much more powerful commercial solvers. 
However, when the length of data is increased, the rapid increase in the computational time can not be avoided due to the increase in the dimension of the problem. 
The proposed SCM enables quick estimation of the optimization results without any special commercial solver and its computational time increases only linearly to the length of the data. 
The short computational time is useful to perform sensitivity analysis or other studies that require multiple simulations with different settings of parameters or input data.



%%%
\subsection{Advantages and Disadvantages}

The proposed SCM estimates the economic benefit of the photovoltaic self-consumption using battery storage.   
It does not consider other types of battery usage, such as energy arbitrage in response to variations in electricity prices \cite{telaretti16,zurfi17}  or provision of ancillary services for voltage regulation or congestion relief of distribution grid \cite{maeyaert20}. 
There are also limitations to the proposed SCM.
For example, it cannot describe discrete installation capacities, existing capacities, or a common three-part tariff, while some of these issues including ancillary services and existing capacities may be resolved by further modifications as performed with the SCM for thermal generation \cite{zhang15,zhang17}. 

One advantage of the proposed SCM is that it can provide intuitive plots of cost curves to understand the reasons behind the results of optimal installation capacities.
In addition, the computational time of the SCM is short (few seconds) compared to optimization models (usually more than hours). 
These facts are useful to explore cost sensitivity or other studies that require multiple simulations.
Thus, the proposed SCM is more useful for researchers or policy makers studying incentive policy for supporting PV or tariff regulations, who want to draw policy implications from an intuitive and quick trend analysis, rather than residential or commercial consumers who want to derive an optimal energy systems design with detailed specifications taken into account. 
Nevertheless, the SCM can be used to derive an initial estimate prior to the detailed design or to screen out cases where there is no economic benefit of installing  PV or battery storage systems. 




%%%%
\section{Conclusion} \label{sec:conclusion}

This paper developed a method of economic analysis of PV self-consumption with and without battery storage.  The development is based on a generalization of the framework of Screening Curve Method (SCM), which has been utilized for generation expansion problem of a bulk power system. 
The proposed SCM enables quick and intuitive estimation of the installation capacities of PV and battery storage systems. 
The estimation accuracy has been validated through numerical simulations, and high accuracy has been achieved when the battery installation capacity is not large. 
The method is useful to explore cost sensitivity or other studies that require multiple simulations.

Future directions of this study include further improvements of the method to consider, e.g., the usage of battery storage for ancillary services or the effect of existing capacities. 
Another one is to perform practical economic analyses by using the proposed SCM. 


\section*{Acknowledgments}

The authors wish to thank K.~Okada and K.~Furusawa (Central Research Institute of Electric Power Industry; CRIEPI) for feedback during the course of this work.


\bibliography{main}
\bibliographystyle{IEEEtran}


\end{document}


