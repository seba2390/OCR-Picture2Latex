Rigid registration of partial observations is a fundamental problem in various applied fields. In computer graphics, special attention has been given to the registration between two partial point clouds generated by scanning devices. State-of-the-art registration techniques still struggle when the overlap region between the two point clouds is small, and completely fail if there is no overlap between the scan pairs. In this paper, we present a learning-based technique that alleviates this problem, and allows registration between point clouds, presented in arbitrary poses, and having little or even no overlap, a setting that has been referred to as \emph{tele-registration}. Our technique is based on a novel neural network design that learns a prior of a class of shapes and can complete a partial shape. The key idea is combining the registration and completion tasks in a way that reinforces each other. In particular, we simultaneously train the registration network and completion network  using two coupled flows, one that \emph{register-and-complete}, and one that \emph{complete-and-register}, and encourage the two flows to produce a consistent result. We show that, compared with each separate flow, this two-flow training leads to robust and reliable tele-registration, and hence to a better point cloud prediction that completes the registered scans. It is also worth mentioning that each of the components in our neural network outperforms state-of-the-art methods in both completion and registration. We further analyze our network with several ablation studies and demonstrate its performance on a large number of partial point clouds, both synthetic and real-world, that have only small or no overlap.


