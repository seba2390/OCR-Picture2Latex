
\section{Overview} \label{sec:overview}

Our tele-registration method consists of a two-flow network, as illustrated in Fig.~\ref{fig:net_overall}. The idea is to simultaneously learn two networks, one for registration (colored in green in the figure) and one for completion (colored in purple in the figure). Taking a pair of partial shapes as input, the \textit{C-R flow} branch first completes each partial shape, separately, and then registers the parts, which now have higher overlap, to produce an aligned shape. In the \textit{R-C flow} branch, the input pair is first registered by the registration network, and then completed by the completion network. %

The key of our method is to connect and couple the \textit{R-C} and \textit{C-R} flows with two losses: one is a \textit{registration consistency} loss, which encourages the registration networks in the two flows to predict the same transformation parameters; the other is the \textit{completion consistency} loss, which encourages the two flows to output similar reconstruction results.
As we shall show, the two flows strengthen each other. It should be noted that the complexity of the completion and registration in two flows are different. For the completion network, the one trained in \textit{R-C flow} is easier, since the input shape is already registered, and contains more overlapping geometric information than the one in \textit{C-R flow}. Similarly, the task for the registration network in the \textit{C-R flow} is easier because the input pair shapes are more complete than the one in \textit{R-C flow}. Thus, our final output is the shape completion results from the \textit{R-C flow}, and the registration parameters from the \textit{C-R flow}.

In the following method section, we first describe the input and output of our method, then introduce the registration and completion networks separately, which form the proposed CTF-Net. Finally, we describe the loss functions that enable the two flows to reinforce each other. 
