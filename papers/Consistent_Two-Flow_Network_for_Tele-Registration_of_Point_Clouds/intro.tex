\section{Introduction} \label{sec:intro}

Shape registration is a long-standing problem with a large variety of methods proposed over the last decades. The registration of partial shapes is significantly more challenging than complete shapes, particularly when the overlap between the parts is small. Popular methods, like RANSAC that match between three or four points, perform well when the overlapping base is large~\cite{fischler1981random, chen1999ransac, rusinkiewicz2001efficient, aiger20084pcs, segal2009generalized}, but completely fail when there is no overlap between the two partial shapes. In the case of partial matching, the information theoretic explanation is that the lack of sufficient information in the scans leads to a family of plausible completions, which in turn results in failure of traditional rigid registration. 

In recent years, neural networks for geometry processing have rapidly emerged and changed the landscape of 3D processing. One notable competence of neural networks is their ability to learn priors of a family of shapes, thus effectively capturing a distribution over possible shapes. Two examples are the reconstruction of a complete shape from a partial input, and the registration of two non-overlapping partial shapes.

We present a rigid registration technique for two partial scans  presented in arbitrary initial poses and having little or no overlap at all. This non-overlapping setting has been referred to as \textit{tele-registration}, and had been attempted in 2D~\cite{Huang2013} and 3D~\cite{Huang2012}, based on a prescribed feature-conforming prior. In our work, we approach the tele-registration problem using learning tools, in particular, deep learning to encode shape priors. The idea is to jointly train two separate networks on the two tasks of shape completion and shape registration of non-overlapping partial shapes.
In training, the networks learn proper priors that allow performing well on these two difficult tasks, rather than treating the tasks independently. 

Our key observation is that registered shapes are easier to be completed than each one alone, and complete shapes are easier to be registered, since their overlap clearly increases. Hence, we combine the registration and completion tasks in a way that reinforces each other. 
In particular, we train the registration and completion networks simultaneously using two coupled flows. One network performs \emph{register-and-complete} and the other \emph{complete-and-register}, such that both registration and completion consistencies are maximized. Fig.~\ref{fig:teaser} illustrates our consistent two-flow network (CTF-Net).
Note that our completion network only generates information for the missing part, and hence the completions along the two branches of the network can be different, and require a dedicated consistency term to produce canonical completion. 
	
Given two partial point cloud inputs with little or no overlap, our method transforms the partial shapes to canonical positions by learning the prior geometry of its class of shapes,
and thus improves the state-of-the-art in terms of completion results on 3D model from the learned class. By comparing our two-flow method to each single-flow and other baselines, we validate that composing two flows together effectively strengthens each component. We evaluate our method on synthetic and real-world examples (e.g., RedWood and Pix3D), and demonstrate the superiority of the approach compared to state-of-the-art methods in both global registration (4PCS, DCP) and completion networks (TopNet, PF-Net).
In summary, we present a method that addresses the problem of non-overlapping point cloud registration, based on a symmetric neural network that is designed to jointly perform registration and completion, in a way that reinforces each other to establish a new state-of-the-art.
