\documentclass{article} %


\usepackage{hyperref}
\usepackage{url}
\usepackage{booktabs}       %
\usepackage{amsfonts}       %
\usepackage{nicefrac}       %
\usepackage{microtype}      %

\usepackage{amsmath,amsfonts,amsthm}       %
\usepackage{subcaption}
\usepackage{bm}
\usepackage{bbm}
\usepackage{multirow}
\usepackage[inline]{enumitem}
\usepackage{diagbox}
\usepackage[capitalise]{cleveref}  %
\usepackage{comment}
\usepackage{etoolbox}
\usepackage{graphicx}
\usepackage{wrapfig}
\usepackage[font=small]{caption}
\usepackage{algorithm,algorithmicx,algpseudocode}
\usepackage{subcaption}
\usepackage{tablefootnote}


\usepackage{pifont}%
\newcommand{\cmark}{\ding{51}}%
\newcommand{\xmark}{\ding{55}}%


\newtheorem{theorem}{Theorem}
\newtheorem{lemma}{Lemma}[section]
\newtheorem{corollary}[lemma]{Corollary}
\newtheorem{observation}[lemma]{Observation}
\newtheorem{proposition}[theorem]{Proposition}
\newtheorem{definition}{Definition}
\newtheorem{remark}[lemma]{Remark}
\newtheorem{claim}{Claim}

\newcommand{\dt}{\Delta}
\newcommand{\dd}{\mathop{}\!d}
\DeclareMathOperator{\hippo}{\mathsf{hippo}}
\DeclareMathOperator*{\diag}{diag}

\newcommand{\methodabbrv}{S4}

\usepackage[square,numbers,sort]{natbib}
\setlength{\textwidth}{6.5in}
\setlength{\textheight}{9in}
\setlength{\oddsidemargin}{0in}
\setlength{\evensidemargin}{0in}
\setlength{\topmargin}{-0.5in}
\newlength{\defbaselineskip}
\setlength{\defbaselineskip}{\baselineskip}
\setlength{\marginparwidth}{0.8in}
\setlength{\parskip}{5pt}%
\setlength{\parindent}{0pt}%
\usepackage[dvipsnames]{xcolor}         %


\title{Efficiently Modeling Long Sequences with Structured State Spaces}
\usepackage{authblk}
\author[]{Albert Gu}
\author[]{Karan Goel}
\author[]{Christopher R{\'e}}
\affil[]{Department of Computer Science, Stanford University}
\affil[]{{\texttt{\{albertgu,krng\}@stanford.edu}, \texttt{chrismre@cs.stanford.edu}}}
\date{}

\bibliographystyle{plainnat}


\begin{document}


\maketitle

\begin{abstract}
  A central goal of sequence modeling is designing a single principled model that can address sequence data across a range of modalities and tasks, particularly on long-range dependencies.
  Although conventional models including RNNs, CNNs, and Transformers have specialized variants for capturing long dependencies, they still struggle to scale to very long sequences of $10000$ or more steps.
  A promising recent approach proposed modeling sequences by simulating the fundamental state space model (SSM) \( x'(t) = Ax(t) + Bu(t), y(t) = Cx(t) + Du(t) \), and showed that for appropriate choices of the state matrix \( A \), this system could handle long-range dependencies mathematically and empirically.
  However, this method has prohibitive computation and memory requirements, rendering it infeasible as a general sequence modeling solution.
  We propose the Structured State Space sequence model (\methodabbrv{}) based on a new parameterization for the SSM, and show that it can be computed much more efficiently than prior approaches while preserving their theoretical strengths.
  Our technique involves conditioning \( A \) with a low-rank correction, allowing it to be diagonalized stably and reducing the SSM to the well-studied computation of a Cauchy kernel.
  \methodabbrv{} achieves strong empirical results across a diverse range of established benchmarks, including (i) 91\% accuracy on sequential CIFAR-10 with no data augmentation or auxiliary losses, on par with a larger 2-D ResNet, (ii) substantially closing the gap to Transformers on image and language modeling tasks, while performing generation $60\times$ faster (iii) SoTA on every task from the Long Range Arena benchmark, including solving the challenging Path-X task of length 16k that all prior work fails on, while being as efficient as all competitors.\footnote{Code is publicly available at \url{https://github.com/HazyResearch/state-spaces}.}
\end{abstract}

% !TEX root = ../arxiv.tex

Unsupervised domain adaptation (UDA) is a variant of semi-supervised learning \cite{blum1998combining}, where the available unlabelled data comes from a different distribution than the annotated dataset \cite{Ben-DavidBCP06}.
A case in point is to exploit synthetic data, where annotation is more accessible compared to the costly labelling of real-world images \cite{RichterVRK16,RosSMVL16}.
Along with some success in addressing UDA for semantic segmentation \cite{TsaiHSS0C18,VuJBCP19,0001S20,ZouYKW18}, the developed methods are growing increasingly sophisticated and often combine style transfer networks, adversarial training or network ensembles \cite{KimB20a,LiYV19,TsaiSSC19,Yang_2020_ECCV}.
This increase in model complexity impedes reproducibility, potentially slowing further progress.

In this work, we propose a UDA framework reaching state-of-the-art segmentation accuracy (measured by the Intersection-over-Union, IoU) without incurring substantial training efforts.
Toward this goal, we adopt a simple semi-supervised approach, \emph{self-training} \cite{ChenWB11,lee2013pseudo,ZouYKW18}, used in recent works only in conjunction with adversarial training or network ensembles \cite{ChoiKK19,KimB20a,Mei_2020_ECCV,Wang_2020_ECCV,0001S20,Zheng_2020_IJCV,ZhengY20}.
By contrast, we use self-training \emph{standalone}.
Compared to previous self-training methods \cite{ChenLCCCZAS20,Li_2020_ECCV,subhani2020learning,ZouYKW18,ZouYLKW19}, our approach also sidesteps the inconvenience of multiple training rounds, as they often require expert intervention between consecutive rounds.
We train our model using co-evolving pseudo labels end-to-end without such need.

\begin{figure}[t]%
    \centering
    \def\svgwidth{\linewidth}
    \input{figures/preview/bars.pdf_tex}
    \caption{\textbf{Results preview.} Unlike much recent work that combines multiple training paradigms, such as adversarial training and style transfer, our approach retains the modest single-round training complexity of self-training, yet improves the state of the art for adapting semantic segmentation by a significant margin.}
    \label{fig:preview}
\end{figure}

Our method leverages the ubiquitous \emph{data augmentation} techniques from fully supervised learning \cite{deeplabv3plus2018,ZhaoSQWJ17}: photometric jitter, flipping and multi-scale cropping.
We enforce \emph{consistency} of the semantic maps produced by the model across these image perturbations.
The following assumption formalises the key premise:

\myparagraph{Assumption 1.}
Let $f: \mathcal{I} \rightarrow \mathcal{M}$ represent a pixelwise mapping from images $\mathcal{I}$ to semantic output $\mathcal{M}$.
Denote $\rho_{\bm{\epsilon}}: \mathcal{I} \rightarrow \mathcal{I}$ a photometric image transform and, similarly, $\tau_{\bm{\epsilon}'}: \mathcal{I} \rightarrow \mathcal{I}$ a spatial similarity transformation, where $\bm{\epsilon},\bm{\epsilon}'\sim p(\cdot)$ are control variables following some pre-defined density (\eg, $p \equiv \mathcal{N}(0, 1)$).
Then, for any image $I \in \mathcal{I}$, $f$ is \emph{invariant} under $\rho_{\bm{\epsilon}}$ and \emph{equivariant} under $\tau_{\bm{\epsilon}'}$, \ie~$f(\rho_{\bm{\epsilon}}(I)) = f(I)$ and $f(\tau_{\bm{\epsilon}'}(I)) = \tau_{\bm{\epsilon}'}(f(I))$.

\smallskip
\noindent Next, we introduce a training framework using a \emph{momentum network} -- a slowly advancing copy of the original model.
The momentum network provides stable, yet recent targets for model updates, as opposed to the fixed supervision in model distillation \cite{Chen0G18,Zheng_2020_IJCV,ZhengY20}.
We also re-visit the problem of long-tail recognition in the context of generating pseudo labels for self-supervision.
In particular, we maintain an \emph{exponentially moving class prior} used to discount the confidence thresholds for those classes with few samples and increase their relative contribution to the training loss.
Our framework is simple to train, adds moderate computational overhead compared to a fully supervised setup, yet sets a new state of the art on established benchmarks (\cf \cref{fig:preview}).


\section{Background and Motivation}

\subsection{IBM Streams}

IBM Streams is a general-purpose, distributed stream processing system. It
allows users to develop, deploy and manage long-running streaming applications
which require high-throughput and low-latency online processing.

The IBM Streams platform grew out of the research work on the Stream Processing
Core~\cite{spc-2006}.  While the platform has changed significantly since then,
that work established the general architecture that Streams still follows today:
job, resource and graph topology management in centralized services; processing
elements (PEs) which contain user code, distributed across all hosts,
communicating over typed input and output ports; brokers publish-subscribe
communication between jobs; and host controllers on each host which
launch PEs on behalf of the platform.

The modern Streams platform approaches general-purpose cluster management, as
shown in Figure~\ref{fig:streams_v4_v6}. The responsibilities of the platform
services include all job and PE life cycle management; domain name resolution
between the PEs; all metrics collection and reporting; host and resource
management; authentication and authorization; and all log collection. The
platform relies on ZooKeeper~\cite{zookeeper} for consistent, durable metadata
storage which it uses for fault tolerance.

Developers write Streams applications in SPL~\cite{spl-2017} which is a
programming language that presents streams, operators and tuples as
abstractions. Operators continuously consume and produce tuples over streams.
SPL allows programmers to write custom logic in their operators, and to invoke
operators from existing toolkits. Compiled SPL applications become archives that
contain: shared libraries for the operators; graph topology metadata which tells
both the platform and the SPL runtime how to connect those operators; and
external dependencies. At runtime, PEs contain one or more operators. Operators
inside of the same PE communicate through function calls or queues. Operators
that run in different PEs communicate over TCP connections that the PEs
establish at startup. PEs learn what operators they contain, and how to connect
to operators in other PEs, at startup from the graph topology metadata provided
by the platform.

We use ``legacy Streams'' to refer to the IBM Streams version 4 family. The
version 5 family is for Kubernetes, but is not cloud native. It uses the
lift-and-shift approach and creates a platform-within-a-platform: it deploys a
containerized version of the legacy Streams platform within Kubernetes.

\subsection{Kubernetes}

Borg~\cite{borg-2015} is a cluster management platform used internally at Google
to schedule, maintain and monitor the applications their internal infrastructure
and external applications depend on. Kubernetes~\cite{kube} is the open-source
successor to Borg that is an industry standard cloud orchestration platform.

From a user's perspective, Kubernetes abstracts running a distributed
application on a cluster of machines. Users package their applications into
containers and deploy those containers to Kubernetes, which runs those
containers in \emph{pods}. Kubernetes handles all life cycle management of pods,
including scheduling, restarting and migration in case of failures.

Internally, Kubernetes tracks all entities as \emph{objects}~\cite{kubeobjects}.
All objects have a name and a specification that describes its desired state.
Kubernetes stores objects in etcd~\cite{etcd}, making them persistent,
highly-available and reliably accessible across the cluster. Objects are exposed
to users through \emph{resources}. All resources can have
\emph{controllers}~\cite{kubecontrollers}, which react to changes in resources.
For example, when a user changes the number of replicas in a
\code{ReplicaSet}, it is the \code{ReplicaSet} controller which makes sure the
desired number of pods are running. Users can extend Kubernetes through
\emph{custom resource definitions} (CRDs)~\cite{kubecrd}. CRDs can contain
arbitrary content, and controllers for a CRD can take any kind of action.

Architecturally, a Kubernetes cluster consists of nodes. Each node runs a
\emph{kubelet} which receives pod creation requests and makes sure that the
requisite containers are running on that node. Nodes also run a
\emph{kube-proxy} which maintains the network rules for that node on behalf of
the pods. The \emph{kube-api-server} is the central point of contact: it
receives API requests, stores objects in etcd, asks the scheduler to schedule
pods, and talks to the kubelets and kube-proxies on each node. Finally,
\emph{namespaces} logically partition the cluster. Objects which should not know
about each other live in separate namespaces, which allows them to share the
same physical infrastructure without interference.

\subsection{Motivation}
\label{sec:motivation}

Systems like Kubernetes are commonly called ``container orchestration''
platforms. We find that characterization reductive to the point of being
misleading; no one would describe operating systems as ``binary executable
orchestration.'' We adopt the idea from Verma et al.~\cite{borg-2015} that
systems like Kubernetes are ``the kernel of a distributed system.'' Through CRDs
and their controllers, Kubernetes provides state-as-a-service in a distributed
system. Architectures like the one we propose are the result of taking that view 
seriously.

The Streams legacy platform has obvious parallels to the Kubernetes
architecture, and that is not a coincidence: they solve similar problems.
Both are designed to abstract running arbitrary user-code across a distributed
system.  We suspect that Streams is not unique, and that there are many
non-trivial platforms which have to provide similar levels of cluster
management.  The benefits to being cloud native and offloading the platform
to an existing cloud management system are: 
\begin{itemize}
    \item Significantly less platform code.
    \item Better scheduling and resource management, as all services on the cluster are 
        scheduled by one platform.
    \item Easier service integration.
    \item Standardized management, logging and metrics.
\end{itemize}
The rest of this paper presents the design of replacing the legacy Streams 
platform with Kubernetes itself.













\section{Proposed Approach} \label{sec:method}

Our goal is to create a unified model that maps task representations (e.g., obtained using task2vec~\cite{achille2019task2vec}) to simulation parameters, which are in turn used to render synthetic pre-training datasets for not only tasks that are seen during training, but also novel tasks.
This is a challenging problem, as the number of possible simulation parameter configurations is combinatorially large, making a brute-force approach infeasible when the number of parameters grows. 

\subsection{Overview} 

\cref{fig:controller-approach} shows an overview of our approach. During training, a batch of ``seen'' tasks is provided as input. Their task2vec vector representations are fed as input to \ours, which is a parametric model (shared across all tasks) mapping these downstream task2vecs to simulation parameters, such as lighting direction, amount of blur, background variability, etc.  These parameters are then used by a data generator (in our implementation, built using the Three-D-World platform~\cite{gan2020threedworld}) to generate a dataset of synthetic images. A classifier model then gets pre-trained on these synthetic images, and the backbone is subsequently used for evaluation on specific downstream task. The classifier's accuracy on this task is used as a reward to update \ours's parameters. 
Once trained, \ours can also be used to efficiently predict simulation parameters in {\em one-shot} for ``unseen'' tasks that it has not encountered during training. 


\subsection{\ours Model} 


Let us denote \ours's parameters with $\theta$. Given the task2vec representation of a downstream task $\bs{x} \in \mc{X}$ as input, \ours outputs simulation parameters $a \in \Omega$. The model consists of $M$ output heads, one for each simulation parameter. In the following discussion, just as in our experiments, each simulation parameter is discretized to a few levels to limit the space of possible outputs. Each head outputs a categorical distribution $\pi_i(\bs{x}, \theta) \in \Delta^{k_i}$, where $k_i$ is the number of discrete values for parameter $i \in [M]$, and $\Delta^{k_i}$, a standard $k_i$-simplex. The set of argmax outputs $\nu(\bs{x}, \theta) = \{\nu_i | \nu_i = \argmax_{j \in [k_i]} \pi_{i, j} ~\forall i \in [M]\}$ is the set of simulation parameter values used for synthetic data generation. Subsequently, we drop annotating the dependence of $\pi$ and $\nu$ on $\theta$ and $\bs{x}$ when clear.

\subsection{\ours Training} 


Since Task2Sim aims to maximize downstream accuracy after pre-training, we use this accuracy as the reward in our training optimization\footnote{Note that our rewards depend only on the task2vec input and the output action and do not involve any states, and thus our problem can be considered similar to a stateless-RL or contextual bandits problem \cite{langford2007epoch}.}.
Note that this downstream accuracy is a non-differentiable function of the output simulation parameters (assuming any simulation engine can be used as a black box) and hence direct gradient-based optimization cannot be used to train \ours. Instead, we use REINFORCE~\cite{williams1992simple}, to approximate gradients of downstream task performance with respect to model parameters $\theta$. 

\ours's outputs represent a distribution over ``actions'' corresponding to different values of the set of $M$ simulation parameters. $P(a) = \prod_{i \in [M]} \pi_i(a_i)$ is the probability of picking action $a = [a_i]_{i \in [M]}$, under policy $\pi = [\pi_i]_{i \in [M]}$. Remember that the output $\pi$ is a function of the parameters $\theta$ and the task representation $\bs{x}$. To train the model, we maximize the expected reward under its policy, defined as
\begin{align}
    R = \E_{a \in \Omega}[R(a)] = \sum_{a \in \Omega} P(a) R(a)
\end{align}
where $\Omega$ is the space of all outputs $a$ and $R(a)$ is the reward when parameter values corresponding to action $a$ are chosen. Since reward is the downstream accuracy, $R(a) \in [0, 100]$.  
Using the REINFORCE rule, we have
\begin{align}
    \nabla_{\theta} R 
    &= \E_{a \in \Omega} \left[ (\nabla_{\theta} \log P(a)) R(a) \right] \\
    &= \E_{a \in \Omega} \left[ \left(\sum_{i \in [M]} \nabla_{\theta} \log \pi_i(a_i) \right) R(a) \right]
\end{align}
where the 2nd step comes from linearity of the derivative. In practice, we use a point estimate of the above expectation at a sample $a \sim (\pi + \epsilon)$ ($\epsilon$ being some exploration noise added to the Task2Sim output distribution) with a self-critical baseline following \cite{rennie2017self}:
\begin{align} \label{eq:grad-pt-est}
    \nabla_{\theta} R \approx \left(\sum_{i \in [M]} \nabla_{\theta} \log \pi_i(a_i) \right) \left( R(a) - R(\nu) \right) 
\end{align}
where, as a reminder $\nu$ is the set of the distribution argmax parameter values from the \name{} model heads.

A pseudo-code of our approach is shown in \cref{alg:train}.  Specifically, we update the model parameters $\theta$ using minibatches of tasks sampled from a set of ``seen'' tasks. Similar to \cite{oh2018self}, we also employ self-imitation learning biased towards actions found to have better rewards. This is done by keeping track of the best action encountered in the learning process and using it for additional updates to the model, besides the ones in \cref{ln:update} of \cref{alg:train}. 
Furthermore, we use the test accuracy of a 5-nearest neighbors classifier operating on features generated by the pretrained backbone as a proxy for downstream task performance since it is computationally much faster than other common evaluation criteria used in transfer learning, e.g., linear probing or full-network finetuning. Our experiments demonstrate that this proxy evaluation measure indeed correlates with, and thus, helps in final downstream performance with linear probing or full-network finetuning. 






\begin{algorithm}
\DontPrintSemicolon
 \textbf{Input:} Set of $N$ ``seen'' downstream tasks represented by task2vecs $\mc{T} = \{\bs{x}_i | i \in [N]\}$. \\
 Given initial Task2Sim parameters $\theta_0$ and initial noise level $\epsilon_0$\\
 Initialize $a_{max}^{(i)} | i \in [N]$ the maximum reward action for each seen task \\
 \For{$t \in [T]$}{
 Set noise level $\epsilon = \frac{\epsilon_0}{t} $ \\
 Sample minibatch $\tau$ of size $n$ from $\mc{T}$  \\
 Get \ours output distributions $\pi^{(i)} | i \in [n]$ \\
 Sample outputs $a^{(i)} \sim \pi^{(i)} + \epsilon$ \\
 Get Rewards $R(a^{(i)})$ by generating a synthetic dataset with parameters $a^{(i)}$, pre-training a backbone on it, and getting the 5-NN downstream accuracy using this backbone \\
 Update $a_{max}^{(i)}$ if $R(a^{(i)}) > R(a_{max}^{(i)})$ \\
 Get point estimates of reward gradients $dr^{(i)}$ for each task in minibatch using \cref{eq:grad-pt-est} \\
 $\theta_{t,0} \leftarrow \theta_{t-1} + \frac{\sum_{i \in [n]} dr^{(i)}}{n}$ \label{ln:update} \\
 \For{$j \in [T_{si}]$}{ 
    \tcp{Self Imitation}
    Get reward gradient estimates $dr_{si}^{(i)}$ from \cref{eq:grad-pt-est} for $a \leftarrow a_{max}^{(i)}$ \\
    $\theta_{t, j}  \leftarrow \theta_{t, j-1} + \frac{\sum_{i \in [n]} dr_{si}^{(i)}}{n}$
 }
 $\theta_{t} \leftarrow \theta_{t, T_{si}}$
 }
 \textbf{Output}: Trained model with parameters $\theta_T$. 
 \caption{Training Task2Sim}
 \label{alg:train}  
\end{algorithm}


In this section we conduct comprehensive experiments to emphasise the effectiveness of DIAL, including evaluations under white-box and black-box settings, robustness to unforeseen adversaries, robustness to unforeseen corruptions, transfer learning, and ablation studies. Finally, we present a new measurement to test the balance between robustness and natural accuracy, which we named $F_1$-robust score. 


\subsection{A case study on SVHN and CIFAR-100}
In the first part of our analysis, we conduct a case study experiment on two benchmark datasets: SVHN \citep{netzer2011reading} and CIFAR-100 \cite{krizhevsky2009learning}. We follow common experiment settings as in \cite{rice2020overfitting, wu2020adversarial}. We used the PreAct ResNet-18 \citep{he2016identity} architecture on which we integrate a domain classification layer. The adversarial training is done using 10-step PGD adversary with perturbation size of 0.031 and a step size of 0.003 for SVHN and 0.007 for CIFAR-100. The batch size is 128, weight decay is $7e^{-4}$ and the model is trained for 100 epochs. For SVHN, the initial learinnig rate is set to 0.01 and decays by a factor of 10 after 55, 75 and 90 iteration. For CIFAR-100, the initial learning rate is set to 0.1 and decays by a factor of 10 after 75 and 90 iterations. 
%We compared DIAL to \cite{madry2017towards} and TRADES \citep{zhang2019theoretically}. 
%The evaluation is done using Auto-Attack~\citep{croce2020reliable}, which is an ensemble of three white-box and one black-box parameter-free attacks, and various $\ell_{\infty}$ adversaries: PGD$^{20}$, PGD$^{100}$, PGD$^{1000}$ and CW$_{\infty}$ with step size of 0.003. 
Results are averaged over 3 restarts while omitting one standard deviation (which is smaller than 0.2\% in all experiments). As can be seen by the results in Tables~\ref{black-and_white-svhn} and \ref{black-and_white-cifar100}, DIAL presents consistent improvement in robustness (e.g., 5.75\% improved robustness on SVHN against AA) compared to the standard AT 
%under variety of attacks 
while also improving the natural accuracy. More results are presented in Appendix \ref{cifar100-svhn-appendix}.


\begin{table}[!ht]
  \caption{Robustness against white-box, black-box attacks and Auto-Attack (AA) on SVHN. Black-box attacks are generated using naturally trained surrogate model. Natural represents the naturally trained (non-adversarial) model.
  %and applied to the best performing robust models.
  }
  \vskip 0.1in
  \label{black-and_white-svhn}
  \centering
  \small
  \begin{tabular}{l@{\hspace{1\tabcolsep}}c@{\hspace{1\tabcolsep}}c@{\hspace{1\tabcolsep}}c@{\hspace{1\tabcolsep}}c@{\hspace{1\tabcolsep}}c@{\hspace{1\tabcolsep}}c@{\hspace{1\tabcolsep}}c@{\hspace{1\tabcolsep}}c@{\hspace{1\tabcolsep}}c@{\hspace{1\tabcolsep}}c}
    \toprule
    & & \multicolumn{4}{c}{White-box} & \multicolumn{4}{c}{Black-Box}  \\
    \cmidrule(r){3-6} 
    \cmidrule(r){7-10}
    Defense Model & Natural & PGD$^{20}$ & PGD$^{100}$  & PGD$^{1000}$  & CW$^{\infty}$ & PGD$^{20}$ & PGD$^{100}$ & PGD$^{1000}$  & CW$^{\infty}$ & AA \\
    \midrule
    NATURAL & 96.85 & 0 & 0 & 0 & 0 & 0 & 0 & 0 & 0 & 0 \\
    \midrule
    AT & 89.90 & 53.23 & 49.45 & 49.23 & 48.25 & 86.44 & 86.28 & 86.18 & 86.42 & 45.25 \\
    % TRADES & 90.35 & 57.10 & 54.13 & 54.08 & 52.19 & 86.89 & 86.73 & 86.57 & 86.70 &  49.50 \\
    $\DIAL_{\kl}$ (Ours) & 90.66 & \textbf{58.91} & \textbf{55.30} & \textbf{55.11} & \textbf{53.67} & 87.62 & 87.52 & 87.41 & 87.63 & \textbf{51.00} \\
    $\DIAL_{\ce}$ (Ours) & \textbf{92.88} & 55.26  & 50.82 & 50.54 & 49.66 & \textbf{89.12} & \textbf{89.01} & \textbf{88.74} & \textbf{89.10} &  46.52  \\
    \bottomrule
  \end{tabular}
\end{table}


\begin{table}[!ht]
  \caption{Robustness against white-box, black-box attacks and Auto-Attack (AA) on CIFAR100. Black-box attacks are generated using naturally trained surrogate model. Natural represents the naturally trained (non-adversarial) model.
  %and applied to the best performing robust models.
  }
  \vskip 0.1in
  \label{black-and_white-cifar100}
  \centering
  \small
  \begin{tabular}{l@{\hspace{1\tabcolsep}}c@{\hspace{1\tabcolsep}}c@{\hspace{1\tabcolsep}}c@{\hspace{1\tabcolsep}}c@{\hspace{1\tabcolsep}}c@{\hspace{1\tabcolsep}}c@{\hspace{1\tabcolsep}}c@{\hspace{1\tabcolsep}}c@{\hspace{1\tabcolsep}}c@{\hspace{1\tabcolsep}}c}
    \toprule
    & & \multicolumn{4}{c}{White-box} & \multicolumn{4}{c}{Black-Box}  \\
    \cmidrule(r){3-6} 
    \cmidrule(r){7-10}
    Defense Model & Natural & PGD$^{20}$ & PGD$^{100}$  & PGD$^{1000}$  & CW$^{\infty}$ & PGD$^{20}$ & PGD$^{100}$ & PGD$^{1000}$  & CW$^{\infty}$ & AA \\
    \midrule
    NATURAL & 79.30 & 0 & 0 & 0 & 0 & 0 & 0 & 0 & 0 & 0 \\
    \midrule
    AT & 56.73 & 29.57 & 28.45 & 28.39 & 26.6 & 55.52 & 55.29 & 55.26 & 55.40 & 24.12 \\
    % TRADES & 58.24 & 30.10 & 29.66 & 29.64 & 25.97 & 57.05 & 56.71 & 56.67 & 56.77 & 24.92 \\
    $\DIAL_{\kl}$ (Ours) & 58.47 & \textbf{31.19} & \textbf{30.50} & \textbf{30.42} & \textbf{26.91} & 57.16 & 56.81 & 56.80 & 57.00 & \textbf{25.87} \\
    $\DIAL_{\ce}$ (Ours) & \textbf{60.77} & 27.87 & 26.66 & 26.61 & 25.98 & \textbf{59.48} & \textbf{59.06} & \textbf{58.96} & \textbf{59.20} & 23.51  \\
    \bottomrule
  \end{tabular}
\end{table}


% \begin{table}[!ht]
%   \caption{Robustness comparison of DIAL to Madry et al. and TRADES defense models on the SVHN dataset under different PGD white-box attacks and the ensemble Auto-Attack (AA).}
%   \label{svhn}
%   \centering
%   \begin{tabular}{llllll|l}
%     \toprule
%     \cmidrule(r){1-5}
%     Defense Model & Natural & PGD$^{20}$ & PGD$^{100}$ & PGD$^{1000}$ & CW$_{\infty}$ & AA\\
%     \midrule
%     $\DIAL_{\kl}$ (Ours) & $\mathbf{90.66}$ & $\mathbf{58.91}$ & $\mathbf{55.30}$ & $\mathbf{55.12}$ & $\mathbf{53.67}$  & $\mathbf{51.00}$  \\
%     Madry et al. & 89.90 & 53.23 & 49.45 & 49.23 & 48.25 & 45.25  \\
%     TRADES & 90.35 & 57.10 & 54.13 & 54.08 & 52.19 & 49.50 \\
%     \bottomrule
%   \end{tabular}
% \end{table}


\subsection{Performance comparison on CIFAR-10} \label{defence-settings}
In this part, we evaluate the performance of DIAL compared to other well-known methods on CIFAR-10. 
%To be consistent with other methods, 
We follow the same experiment setups as in~\cite{madry2017towards, wang2019improving, zhang2019theoretically}. When experiment settings are not identical between tested methods, we choose the most commonly used settings, and apply it to all experiments. This way, we keep the comparison as fair as possible and avoid reporting changes in results which are caused by inconsistent experiment settings \citep{pang2020bag}. To show that our results are not caused because of what is referred to as \textit{obfuscated gradients}~\citep{athalye2018obfuscated}, we evaluate our method with same setup as in our defense model, under strong attacks (e.g., PGD$^{1000}$) in both white-box, black-box settings, Auto-Attack ~\citep{croce2020reliable}, unforeseen "natural" corruptions~\citep{hendrycks2018benchmarking}, and unforeseen adversaries. To make sure that the reported improvements are not caused by \textit{adversarial overfitting}~\citep{rice2020overfitting}, we report best robust results for each method on average of 3 restarts, while omitting one standard deviation (which is smaller than 0.2\% in all experiments). Additional results for CIFAR-10 as well as comprehensive evaluation on MNIST can be found in Appendix \ref{mnist-results} and \ref{additional_res}.
%To further keep the comparison consistent, we followed the same attack settings for all methods.


\begin{table}[ht]
  \caption{Robustness against white-box, black-box attacks and Auto-Attack (AA) on CIFAR-10. Black-box attacks are generated using naturally trained surrogate model. Natural represents the naturally trained (non-adversarial) model.
  %and applied to the best performing robust models.
  }
  \vskip 0.1in
  \label{black-and_white-cifar}
  \centering
  \small
  \begin{tabular}{cccccccc@{\hspace{1\tabcolsep}}c}
    \toprule
    & & \multicolumn{3}{c}{White-box} & \multicolumn{3}{c}{Black-Box} \\
    \cmidrule(r){3-5} 
    \cmidrule(r){6-8}
    Defense Model & Natural & PGD$^{20}$ & PGD$^{100}$ & CW$^{\infty}$ & PGD$^{20}$ & PGD$^{100}$ & CW$^{\infty}$ & AA \\
    \midrule
    NATURAL & 95.43 & 0 & 0 & 0 & 0 & 0 & 0 &  0 \\
    \midrule
    TRADES & 84.92 & 56.60 & 55.56 & 54.20 & 84.08 & 83.89 & 83.91 &  53.08 \\
    MART & 83.62 & 58.12 & 56.48 & 53.09 & 82.82 & 82.52 & 82.80 & 51.10 \\
    AT & 85.10 & 56.28 & 54.46 & 53.99 & 84.22 & 84.14 & 83.92 & 51.52 \\
    ATDA & 76.91 & 43.27 & 41.13 & 41.01 & 75.59 & 75.37 & 75.35 & 40.08\\
    $\DIAL_{\kl}$ (Ours) & 85.25 & $\mathbf{58.43}$ & $\mathbf{56.80}$ & $\mathbf{55.00}$ & 84.30 & 84.18 & 84.05 & \textbf{53.75} \\
    $\DIAL_{\ce}$ (Ours)  & $\mathbf{89.59}$ & 54.31 & 51.67 & 52.04 &$ \mathbf{88.60}$ & $\mathbf{88.39}$ & $\mathbf{88.44}$ & 49.85 \\
    \midrule
    $\DIAL_{\awp}$ (Ours) & $\mathbf{85.91}$ & $\mathbf{61.10}$ & $\mathbf{59.86}$ & $\mathbf{57.67}$ & $\mathbf{85.13}$ & $\mathbf{84.93}$ & $\mathbf{85.03}$  & \textbf{56.78} \\
    $\TRADES_{\awp}$ & 85.36 & 59.27 & 59.12 & 57.07 & 84.58 & 84.58 & 84.59 & 56.17 \\
    \bottomrule
  \end{tabular}
\end{table}



\paragraph{CIFAR-10 setup.} We use the wide residual network (WRN-34-10)~\citep{zagoruyko2016wide} architecture. %used in the experiments of~\cite{madry2017towards, wang2019improving, zhang2019theoretically}. 
Sidelong this architecture, we integrate a domain classification layer. To generate the adversarial domain dataset, we use a perturbation size of $\epsilon=0.031$. We apply 10 of inner maximization iterations with perturbation step size of 0.007. Batch size is set to 128, weight decay is set to $7e^{-4}$, and the model is trained for 100 epochs. Similar to the other methods, the initial learning rate was set to 0.1, and decays by a factor of 10 at iterations 75 and 90. 
%For being consistent with other methods, the natural images are padded with 4-pixel padding with 32-random crop and random horizontal flip. Furthermore, all methods are trained using SGD with momentum 0.9. For $\DIAL_{\kl}$, we balance the robust loss with $\lambda=6$ and the domains loss with $r=4$. For $\DIAL_{\ce}$, we balance the robust loss with $\lambda=1$ and the domains loss with $r=2$. 
%We also introduce a version of our method that incorporates the AWP double-perturbation mechanism, named DIAL-AWP.
%which is trained using the same learning rate schedule used in ~\cite{wu2020adversarial}, where the initial 0.1 learning rate decays by a factor of 10 after 100 and 150 iterations. 
See Appendix \ref{cifar10-additional-setup} for additional details.

\begin{table}[ht]
  \caption{Black-box attack using the adversarially trained surrogate models on CIFAR-10.}
  \vskip 0.1in
  \label{black-box-cifar-adv}
  \centering
  \small
  \begin{tabular}{ll|c}
    \toprule
    \cmidrule(r){1-2}
    Surrogate (source) model & Target model & robustness \% \\
    % \midrule
    \midrule
    TRADES & $\DIAL_{\ce}$ & $\mathbf{67.77}$ \\
    $\DIAL_{\ce}$ & TRADES & 65.75 \\
    \midrule
    MART & $\DIAL_{\ce}$ & $\mathbf{70.30}$ \\
    $\DIAL_{\ce}$ & MART & 64.91 \\
    \midrule
    AT & $\DIAL_{\ce}$ & $\mathbf{65.32}$ \\
    $\DIAL_{\ce}$ & AT  & 63.54 \\
    \midrule
    ATDA & $\DIAL_{\ce}$ & $\mathbf{66.77}$ \\
    $\DIAL_{\ce}$ & ATDA & 52.56 \\
    \bottomrule
  \end{tabular}
\end{table}

\paragraph{White-box/Black-box robustness.} 
%We evaluate all defense models using Auto-Attack, PGD$^{20}$, PGD$^{100}$, PGD$^{1000}$ and CW$_{\infty}$ with step size 0.003. We constrain all attacks by the same perturbation $\epsilon=0.031$. 
As reported in Table~\ref{black-and_white-cifar} and Appendix~\ref{additional_res}, our method achieves better robustness compared to the other methods. Specifically, in the white-box settings, our method improves robustness over~\citet{madry2017towards} and TRADES by 2\% 
%using the common PGD$^{20}$ attack 
while keeping higher natural accuracy. We also observe better natural accuracy of 1.65\% over MART while also achieving better robustness over all attacks. Moreover, our method presents significant improvement of up to 15\% compared to the the domain invariant method suggested by~\citet{song2018improving} (ATDA).
%in both natural and robust accuracy. 
When incorporating 
%the double-perturbation mechanism of 
AWP, our method improves the results of $\TRADES_{\awp}$ by almost 2\%.
%and reaches state-of-the-art results for robust models with no additional data. 
% Additional results are available in Appendix~\ref{additional_res}.
When tested on black-box settings, $\DIAL_{\ce}$ presents a significant improvement of more than 4.4\% over the second-best performing method, and up to 13\%. In Table~\ref{black-box-cifar-adv}, we also present the black-box results when the source model is taken from one of the adversarially trained models. %Then, we compare our model to each one of them both as the source model and target model. 
In addition to the improvement in black-box robustness, $\DIAL_{\ce}$ also manages to achieve better clean accuracy of more than 4.5\% over the second-best performing method.
% Moreover, based on the auto-attack leader-board \footnote{\url{https://github.com/fra31/auto-attack}}, our method achieves the 1st place among models without additional data using the WRN-34-10 architecture.

% \begin{table}
%   \caption{White-box robustness on CIFAR-10 using WRN-34-10}
%   \label{white-box-cifar-10}
%   \centering
%   \begin{tabular}{lllll}
%     \toprule
%     \cmidrule(r){1-2}
%     Defense Model & Natural & PGD$^{20}$ & PGD$^{100}$ & PGD$^{1000}$ \\
%     \midrule
%     TRADES ~\cite{zhang2019theoretically} & 84.92  & 56.6 & 55.56 & 56.43  \\
%     MART ~\cite{wang2019improving} & 83.62  & 58.12 & 56.48 & 56.55  \\
%     Madry et al. ~\cite{madry2017towards} & 85.1  & 56.28 & 54.46 & 54.4  \\
%     Song et al. ~\cite{song2018improving} & 76.91 & 43.27 & 41.13 & 41.02  \\
%     $\DIAL_{\ce}$ (Ours) & $ \mathbf{90}$  & 52.12 & 48.88 & 48.78  \\
%     $\DIAL_{\kl}$ (Ours) & 85.25 & $\mathbf{58.43}$ & $\mathbf{56.8}$ & $\mathbf{56.73}$ \\
%     \midrule
%     $\DIAL_{\kl}$+AWP (Ours) & $\mathbf{85.91}$ & $\mathbf{61.1}$ & - & -  \\
%     TRADES+AWP \cite{wu2020adversarial} & 85.36 & 59.27 & 59.12 & -  \\
%     % MART + AWP & 84.43 & 60.68 & 59.32 & -  \\
%     \bottomrule
%   \end{tabular}
% \end{table}


% \begin{table}
%   \caption{White-box robustness on MNIST}
%   \label{white-box-mnist}
%   \centering
%   \begin{tabular}{llllll}
%     \toprule
%     \cmidrule(r){1-2}
%     Defense Model & Natural & PGD$^{40}$ & PGD$^{100}$ & PGD$^{1000}$ \\
%     \midrule
%     TRADES ~\cite{zhang2019theoretically} & 99.48 & 96.07 & 95.52 & 95.22 \\
%     MART ~\cite{wang2019improving} & 99.38  & 96.99 & 96.11 & 95.74  \\
%     Madry et al. ~\cite{madry2017towards} & 99.41  & 96.01 & 95.49 & 95.36 \\
%     Song et al. ~\cite{song2018improving}  & 98.72 & 96.82 & 96.26 & 96.2  \\
%     $\DIAL_{\kl}$ (Ours) & 99.46 & 97.05 & 96.06 & 95.99  \\
%     $\DIAL_{\ce}$ (Ours) & $\mathbf{99.49}$  & $\mathbf{97.38}$ & $\mathbf{96.45}$ & $\mathbf{96.33}$ \\
%     \bottomrule
%   \end{tabular}
% \end{table}


% \paragraph{Attacking MNIST.} For consistency, we use the same perturbation and step sizes. For MNIST, we use $\epsilon=0.3$ and step size of $0.01$. The natural accuracy of our surrogate (source) model is 99.51\%. Attacks results are reported in Table~\ref{black-and_white-mnist}. It is worth noting that the improvement margin is not conclusive on MNIST as it is on CIFAR-10, which is a more complex task.

% \begin{table}
%   \caption{Black-box robustness on MNIST and CIFAR-10 using naturally trained surrogate model and best performing robust models}
%   \label{black-box-mnist-cifar}
%   \centering
%   \begin{tabular}{lllllll}
%     \toprule
%     & \multicolumn{3}{c}{MNIST} & \multicolumn{3}{c}{CIFAR-10} \\
%     \cmidrule(r){2-4} 
%     \cmidrule(r){5-7}  
%     Defense Model & PGD$^{40}$ & PGD$^{100}$ & PGD$^{1000}$ & PGD$^{20}$ & PGD$^{100}$ & PGD$^{1000}$ \\
%     \midrule
%     TRADES ~\cite{zhang2019theoretically} & 98.12 & 97.86 & 97.81 & 84.08 & 83.89 & 83.8 \\
%     MART ~\cite{wang2019improving} & 98.16 & 97.96 & 97.89  & 82.82 & 82.52 & 82.47 \\
%     Madry et al. ~\cite{madry2017towards}  & 98.05 & 97.73 & 97.78 & 84.22 & 84.14 & 83.96 \\
%     Song et al. ~\cite{song2018improving} & 97.74 & 97.28 & 97.34 & 75.59 & 75.37 & 75.11 \\
%     $\DIAL_{\kl}$ (Ours) & 98.14 & 97.83 & 97.87  & 84.3 & 84.18 & 84.0 \\
%     $\DIAL_{\ce}$ (Ours)  & $\mathbf{98.37}$ & $\mathbf{98.12}$ & $\mathbf{98.05}$  & $\mathbf{89.13}$ & $\mathbf{88.89}$ & $\mathbf{88.78}$ \\
%     \bottomrule
%   \end{tabular}
% \end{table}



% \subsubsection{Ensemble attack} In addition to the white-box and black-box settings, we evaluate our method on the Auto-Attack ~\citep{croce2020reliable} using $\ell_{\infty}$ threat model with perturbation $\epsilon=0.031$. Auto-Attack is an ensemble of parameter-free attacks. It consists of three white-box attacks: APGD-CE which is a step size-free version of PGD on the cross-entropy ~\citep{croce2020reliable}. APGD-DLR which is a step size-free version of PGD on the DLR loss ~\citep{croce2020reliable} and FAB which  minimizes the norm of the adversarial perturbations, and one black-box attack: square attack which is a query-efficient black-box attack ~\citep{andriushchenko2020square}. Results are presented in Table~\ref{auto-attack}. Based on the auto-attack leader-board \footnote{\url{https://github.com/fra31/auto-attack}}, our method achieves the 1st place among models without additional data using the WRN-34-10 architecture.

%Additional results can be found in Appendix ~\ref{additional_res}.

% \begin{table}
%   \caption{Auto-Attack (AA) on CIFAR-10 with perturbation size $\epsilon=0.031$ with $\ell_{\infty}$ threat model}
%   \label{auto-attack}
%   \centering
%   \begin{tabular}{lll}
%     \toprule
%     \cmidrule(r){1-2}
%     Defense Model & AA \\
%     \midrule
%     TRADES ~\cite{zhang2019theoretically} & 53.08  \\
%     MART ~\cite{wang2019improving} & 51.1  \\
%     Madry et al. ~\cite{madry2017towards} & 51.52    \\
%     Song et al.   ~\cite{song2018improving} & 40.18 \\
%     $\DIAL_{\ce}$ (Ours) & 47.33  \\
%     $\DIAL_{\kl}$ (Ours) & $\mathbf{53.75}$ \\
%     \midrule
%     DIAL-AWP (Ours) & $\mathbf{56.78}$ \\
%     TRADES-AWP \cite{wu2020adversarial} & 56.17 \\
%     \bottomrule
%   \end{tabular}
% \end{table}


% \begin{table}[!ht]
%   \caption{Auto-Attack (AA) Robustness (\%) on CIFAR-10 with $\epsilon=0.031$ using an $\ell_{\infty}$ threat model}
%   \label{auto-attack}
%   \centering
%   \begin{tabular}{cccccc|cc}
%     \toprule
%     % \multicolumn{8}{c}{Defence Model}  \\
%     % \cmidrule(r){1-8} 
%     TRADES & MART & Madry & Song & $\DIAL_{\ce}$ & $\DIAL_{\kl}$ & DIAL-AWP  & TRADES-AWP\\
%     \midrule
%     53.08 & 51.10 & 51.52 &  40.08 & 47.33  & $\mathbf{53.75}$ & $\mathbf{56.78}$ & 56.17 \\

%     \bottomrule
%   \end{tabular}
% \end{table}

% \begin{table}[!ht]
% \caption{$F_1$-robust measurement using PGD$^{20}$ attack in white-box and black-box settings on CIFAR-10}
%   \label{f1-robust}
%   \centering
%   \begin{tabular}{ccccccc|cc}
%     \toprule
%     % \multicolumn{8}{c}{Defence Model}  \\
%     % \cmidrule(r){1-8} 
%     Defense Model & TRADES & MART & Madry & Song & $\DIAL_{\kl}$ & $\DIAL_{\ce}$ & DIAL-AWP  & TRADES-AWP\\
%     \midrule
%     White-box & 0.659 & 0.666 & 0.657 & 0.518 & $\mathbf{0.675}$  & 0.643 & $\mathbf{0.698}$ & 0.682 \\
%     Black-box & 0.844 & 0.831 & 0.846 & 0.761 & 0.847 & $\mathbf{0.895}$ & $\mathbf{0.854}$ &  0.849 \\
%     \bottomrule
%   \end{tabular}
% \end{table}

\subsubsection{Robustness to Unforeseen Attacks and Corruptions}
\paragraph{Unforeseen Adversaries.} To further demonstrate the effectiveness of our approach, we test our method against various adversaries that were not used during the training process. We attack the model under the white-box settings with $\ell_{2}$-PGD, $\ell_{1}$-PGD, $\ell_{\infty}$-DeepFool and $\ell_{2}$-DeepFool \citep{moosavi2016deepfool} adversaries using Foolbox \citep{rauber2017foolbox}. We applied commonly used attack budget 
%(perturbation for PGD adversaries and overshot for DeepFool adversaries) 
with 20 and 50 iterations for PGD and DeepFool, respectively.
Results are presented in Table \ref{unseen-attacks}. As can be seen, our approach  gains an improvement of up to 4.73\% over the second best method under the various attack types and an average improvement of 3.7\% over all threat models.


\begin{table}[ht]
  \caption{Robustness on CIFAR-10 against unseen adversaries under white-box settings.}
  \vskip 0.1in
  \label{unseen-attacks}
  \centering
%   \small
  \begin{tabular}{c@{\hspace{1.5\tabcolsep}}c@{\hspace{1.5\tabcolsep}}c@{\hspace{1.5\tabcolsep}}c@{\hspace{1.5\tabcolsep}}c@{\hspace{1.5\tabcolsep}}c@{\hspace{1.5\tabcolsep}}c@{\hspace{1.5\tabcolsep}}c}
    \toprule
    Threat Model & Attack Constraints & $\DIAL_{\kl}$ & $\DIAL_{\ce}$ & AT & TRADES & MART & ATDA \\
    \midrule
    \multirow{2}{*}{$\ell_{2}$-PGD} & $\epsilon=0.5$ & 76.05 & \textbf{80.51} & 76.82 & 76.57 & 75.07 & 66.25 \\
    & $\epsilon=0.25$ & 80.98 & \textbf{85.38} & 81.41 & 81.10 & 80.04 & 71.87 \\\midrule
    \multirow{2}{*}{$\ell_{1}$-PGD} & $\epsilon=12$ & 74.84 & \textbf{80.00} & 76.17 & 75.52 & 75.95 & 65.76 \\
    & $\epsilon=7.84$ & 78.69 & \textbf{83.62} & 79.86 & 79.16 & 78.55 & 69.97 \\
    \midrule
    $\ell_{2}$-DeepFool & overshoot=0.02 & 84.53 & \textbf{88.88} & 84.15 & 84.23 & 82.96 & 76.08 \\\midrule
    $\ell_{\infty}$-DeepFool & overshoot=0.02 & 68.43 & \textbf{69.50} & 67.29 & 67.60 & 66.40 & 57.35 \\
    \bottomrule
  \end{tabular}
\end{table}


%%%%%%%%%%%%%%%%%%%%%%%%% conference version %%%%%%%%%%%%%%%%%%%%%%%%%%%%%%%%%%%%%
\paragraph{Unforeseen Corruptions.}
We further demonstrate that our method consistently holds against unforeseen ``natural'' corruptions, consists of 18 unforeseen diverse corruption types proposed by \citet{hendrycks2018benchmarking} on CIFAR-10, which we refer to as CIFAR10-C. The CIFAR10-C benchmark covers noise, blur, weather, and digital categories. As can be shown in Figure \ref{corruption}, our method gains a significant and consistent improvement over all the other methods. Our method leads to an average improvement of 4.7\% with minimum improvement of 3.5\% and maximum improvement of 5.9\% compared to the second best method over all unforeseen attacks. See Appendix \ref{corruptions-apendix} for the full experiment results.


\begin{figure}[h]
 \centering
  \includegraphics[width=0.4\textwidth]{figures/spider_full.png}
%   \caption{Summary of accuracy over all unforeseen corruptions compared to the second and third best performing methods.}
  \caption{Accuracy comparison over all unforeseen corruptions.}
  \label{corruption}
\end{figure}


%%%%%%%%%%%%%%%%%%%%%%%%% conference version %%%%%%%%%%%%%%%%%%%%%%%%%%%%%%%%%%%%%

%%%%%%%%%%%%%%%%%%%%%%%%% Arxiv version %%%%%%%%%%%%%%%%%%%%%%%%%%%%%%%%%%%%%
% \newpage
% \paragraph{Unforeseen Corruptions.}
% We further demonstrate that our method consistently holds against unforeseen "natural" corruptions, consists of 18 unforeseen diverse corruption types proposed by \cite{hendrycks2018benchmarking} on CIFAR-10, which we refer to as CIFAR10-C. The CIFAR10-C benchmark covers noise, blur, weather, and digital categories. As can be shown in Figure  \ref{spider-full-graph}, our method gains a significant and consistent improvement over all the other methods. Our approach leads to an average improvement of 4.7\% with minimum improvement of 3.5\% and maximum improvement of 5.9\% compared to the second best method over all unforeseen attacks. Full accuracy results against unforeseen corruptions are presented in Tables \ref{corruption-table1} and \ref{corruption-table2}. 

% \begin{table}[!ht]
%   \caption{Accuracy (\%) against unforeseen corruptions.}
%   \label{corruption-table1}
%   \centering
%   \tiny
%   \begin{tabular}{lcccccccccccccccccc}
%     \toprule
%     Defense Model & brightness & defocus blur & fog & glass blur & jpeg compression & motion blur & saturate & snow & speckle noise  \\
%     \midrule
%     TRADES & 82.63 & 80.04 & 60.19 & 78.00 & 82.81 & 76.49 & 81.53 & 80.68 & 80.14 \\
%     MART & 80.76 & 78.62 & 56.78 & 76.60 & 81.26 & 74.58 & 80.74 & 78.22 & 79.42 \\
%     AT &  83.30 & 80.42 & 60.22 & 77.90 & 82.73 & 76.64 & 82.31 & 80.37 & 80.74 \\
%     ATDA & 72.67 & 69.36 & 45.52 & 64.88 & 73.22 & 63.47 & 72.07 & 68.76 & 72.27 \\
%     DIAL (Ours)  & \textbf{87.14} & \textbf{84.84} & \textbf{66.08} & \textbf{81.82} & \textbf{87.07} & \textbf{81.20} & \textbf{86.45} & \textbf{84.18} & \textbf{84.94} \\
%     \bottomrule
%   \end{tabular}
% \end{table}


% \begin{table}[!ht]
%   \caption{Accuracy (\%) against unforeseen corruptions.}
%   \label{corruption-table2}
%   \centering
%   \tiny
%   \begin{tabular}{lcccccccccccccccccc}
%     \toprule
%     Defense Model & contrast & elastic transform & frost & gaussian noise & impulse noise & pixelate & shot noise & spatter & zoom blur \\
%     \midrule
%     TRADES & 43.11 & 79.11 & 76.45 & 79.21 & 73.72 & 82.73 & 80.42 & 80.72 & 78.97 \\
%     MART & 41.22 & 77.77 & 73.07 & 78.30 & 74.97 & 81.31 & 79.53 & 79.28 & 77.8 \\
%     AT & 43.30 & 79.58 & 77.53 & 79.47 & 73.76 & 82.78 & 80.86 & 80.49 & 79.58 \\
%     ATDA & 36.06 & 67.06 & 62.56 & 70.33 & 64.63 & 73.46 & 72.28 & 70.50 & 67.31 \\
%     DIAL (Ours) & \textbf{48.84} & \textbf{84.13} & \textbf{81.76} & \textbf{83.76} & \textbf{78.26} & \textbf{87.24} & \textbf{85.13} & \textbf{84.84} & \textbf{83.93}  \\
%     \bottomrule
%   \end{tabular}
% \end{table}


% \begin{figure}[!ht]
%   \centering
%   \includegraphics[width=9cm]{figures/spider_full.png}
%   \caption{Accuracy comparison with all tested methods over unforeseen corruptions.}
%   \label{spider-full-graph}
% \end{figure}
% %%%%%%%%%%%%%%%%%%%%%%%%% Arxiv version %%%%%%%%%%%%%%%%%%%%%%%%%%%%%%%%%%%%%
%%%%%%%%%%%%%%%%%%%%%%%%% Arxiv version %%%%%%%%%%%%%%%%%%%%%%%%%%%%%%%%%%%%%

\subsubsection{Transfer Learning}
Recent works \citep{salman2020adversarially,utrera2020adversarially} suggested that robust models transfer better on standard downstream classification tasks. In Table \ref{transfer-res} we demonstrate the advantage of our method when applied for transfer learning across CIFAR10 and CIFAR100 using the common linear evaluation protocol. see Appendix \ref{transfer-learning-settings} for detailed settings.

\begin{table}[H]
  \caption{Transfer learning results comparison.}
  \vskip 0.1in
  \label{transfer-res}
  \centering
  \small
\begin{tabular}{c|c|c|c}
\toprule

\multicolumn{2}{l}{} & \multicolumn{2}{c}{Target} \\
\cmidrule(r){3-4}
Source & Defence Model & CIFAR10 & CIFAR100 \\
\midrule
\multirow{3}{*}{CIFAR10} & DIAL & \multirow{3}{*}{-} & \textbf{28.57} \\
 & AT &  & 26.95  \\
 & TRADES &  & 25.40  \\
 \midrule
\multirow{3}{*}{CIFAR100} & DIAL & \textbf{73.68} & \multirow{3}{*}{-} \\
 & AT & 71.41 & \\
 & TRADES & 71.42 &  \\
%  \midrule
% \multirow{3}{}{SVHN} & DIAL &  &  & \multirow{3}{}{-} \\
%  & Madry et al. &  &  &  \\
%  & TRADES &  &  &  \\ 
\bottomrule
\end{tabular}
\end{table}


\subsubsection{Modularity and Ablation Studies}

We note that the domain classifier is a modular component that can be integrated into existing models for further improvements. Removing the domain head and related loss components from the different DIAL formulations results in some common adversarial training techniques. For $\DIAL_{\kl}$, removing the domain and related loss components results in the formulation of TRADES. For $\DIAL_{\ce}$, removing the domain and related loss components results in the original formulation of the standard adversarial training, and for $\DIAL_{\awp}$ the removal results in $\TRADES_{\awp}$. Therefore, the ablation studies will demonstrate the effectiveness of combining DIAL on top of different adversarial training methods. 

We investigate the contribution of the additional domain head component introduced in our method. Experiment configuration are as in \ref{defence-settings}, and robust accuracy is based on white-box PGD$^{20}$ on CIFAR-10 dataset. We remove the domain head from both $\DIAL_{\kl}$, $\DIAL_{\awp}$, and $\DIAL_{\ce}$ (equivalent to $r=0$) and report the natural and robust accuracy. We perform 3 random restarts and omit one standard deviation from the results. Results are presented in Figure \ref{ablation}. All DIAL variants exhibits stable improvements on both natural accuracy and robust accuracy. $\DIAL_{\ce}$, $\DIAL_{\kl}$, and $\DIAL_{\awp}$ present an improvement of 1.82\%, 0.33\%, and 0.55\% on natural accuracy and an improvement of 2.5\%, 1.87\%, and 0.83\% on robust accuracy, respectively. This evaluation empirically demonstrates the benefits of incorporating DIAL on top of different adversarial training techniques.
% \paragraph{semi-supervised extensions.} Since the domain classifier does not require the class labels, we argue that additional unlabeled data can be leveraged in future work.
%for improved results. 

\begin{figure}[ht]
  \centering
  \includegraphics[width=0.35\textwidth]{figures/ablation_graphs3.png}
  \caption{Ablation studies for $\DIAL_{\kl}$, $\DIAL_{\ce}$, and $\DIAL_{\awp}$ on CIFAR-10. Circle represent the robust-natural accuracy without using DIAL, and square represent the robust-natural accuracy when incorporating DIAL.
  %to further investigate the impact of the domain head and loss on natural and robust accuracy.
  }
  \label{ablation}
\end{figure}

\subsubsection{Visualizing DIAL}
To further illustrate the superiority of our method, we visualize the model outputs from the different methods on both natural and adversarial test data.
% adversarial test data generated using PGD$^{20}$ white-box attack with step size 0.003 and $\epsilon=0.031$ on CIFAR-10. 
Figure~\ref{tsne1} shows the embedding received after applying t-SNE ~\citep{van2008visualizing} with two components on the model output for our method and for TRADES. DIAL seems to preserve strong separation between classes on both natural test data and adversarial test data. Additional illustrations for the other methods are attached in Appendix~\ref{additional_viz}. 

\begin{figure}[h]
\centering
  \subfigure[\textbf{DIAL} on natural logits]{\includegraphics[width=0.21\textwidth]{figures/domain_ce_test.png}}
  \hspace{0.03\textwidth}
  \subfigure[\textbf{DIAL} on adversarial logits]{\includegraphics[width=0.21\textwidth]{figures/domain_ce_adversarial.png}}
  \hspace{0.03\textwidth}
    \subfigure[\textbf{TRADES} on natural logits]{\includegraphics[width=0.21\textwidth]{figures/trades_test.png}}
    \hspace{0.03\textwidth}
    \subfigure[\textbf{TRADES} on adversarial logits]{\includegraphics[width=0.21\textwidth]{figures/trades_adversarial.png}}
  \caption{t-SNE embedding of model output (logits) into two-dimensional space for DIAL and TRADES using the CIFAR-10 natural test data and the corresponding PGD$^{20}$ generated adversarial examples.}
  \label{tsne1}
\end{figure}


% \begin{figure}[ht]
% \centering
%   \begin{subfigure}{4cm}
%     \centering\includegraphics[width=3.3cm]{figures/domain_ce_test.png}
%     \caption{\textbf{DIAL} on nat. examples}
%   \end{subfigure}
%   \begin{subfigure}{4cm}
%     \centering\includegraphics[width=3.3cm]{figures/domain_ce_adversarial.png}
%     \caption{\textbf{DIAL} on adv. examples}
%   \end{subfigure}
  
%   \begin{subfigure}{4cm}
%     \centering\includegraphics[width=3.3cm]{figures/trades_test.png}
%     \caption{\textbf{TRADES} on nat. examples}
%   \end{subfigure}
%   \begin{subfigure}{4cm}
%     \centering\includegraphics[width=3.3cm]{figures/trades_adversarial.png}
%     \caption{\textbf{TRADES} on adv. examples}
%   \end{subfigure}
%   \caption{t-SNE embedding of model output (logits) into two-dimensional space for DIAL and TRADES using the CIFAR-10 natural test data and the corresponding adversarial examples.}
%   \label{tsne1}
% \end{figure}



% \begin{figure}[ht]
% \centering
%   \begin{subfigure}{6cm}
%     \centering\includegraphics[width=5cm]{figures/domain_ce_test.png}
%     \caption{\textbf{DIAL} on nat. examples}
%   \end{subfigure}
%   \begin{subfigure}{6cm}
%     \centering\includegraphics[width=5cm]{figures/domain_ce_adversarial.png}
%     \caption{\textbf{DIAL} on adv. examples}
%   \end{subfigure}
  
%   \begin{subfigure}{6cm}
%     \centering\includegraphics[width=5cm]{figures/trades_test.png}
%     \caption{\textbf{TRADES} on nat. examples}
%   \end{subfigure}
%   \begin{subfigure}{6cm}
%     \centering\includegraphics[width=5cm]{figures/trades_adversarial.png}
%     \caption{\textbf{TRADES} on adv. examples}
%   \end{subfigure}
%   \caption{t-SNE embedding of model output (logits) into two-dimensional space for DIAL and TRADES using the CIFAR-10 natural test data and the corresponding adversarial examples.}
%   \label{tsne1}
% \end{figure}



\subsection{Balanced measurement for robust-natural accuracy}
One of the goals of our method is to better balance between robust and natural accuracy under a given model. For a balanced metric, we adopt the idea of $F_1$-score, which is the harmonic mean between the precision and recall. However, rather than using precision and recall, we measure the $F_1$-score between robustness and natural accuracy,
using a measure we call
%We named it
the
\textbf{$\mathbf{F_1}$-robust} score.
\begin{equation}
F_1\text{-robust} = \dfrac{\text{true\_robust}}
{\text{true\_robust}+\frac{1}{2}
%\cdot
(\text{false\_{robust}}+
\text{false\_natural})},
\end{equation}
where $\text{true\_robust}$ are the adversarial examples that were correctly classified, $\text{false\_{robust}}$ are the adversarial examples that were miss-classified, and $\text{false\_natural}$ are the natural examples that were miss-classified.
%We tested the proposed $F_1$-robust score using PGD$^{20}$ on CIFAR-10 dataset in white-box and black-box settings. 
Results are presented in Table~\ref{f1-robust} and demonstrate that our method achieves the best $F_1$-robust score in both settings, which supports our findings from previous sections.

% \begin{table}[!ht]
%   \caption{$F_1$-robust measurement using PGD$^{20}$ attack in white and black box settings on CIFAR-10}
%   \label{f1-robust}
%   \centering
%   \begin{tabular}{lll}
%     \toprule
%     \cmidrule(r){1-2}
%     Defense Model & White-box & Black-box \\
%     \midrule
%     TRADES & 0.65937  & 0.84435 \\
%     MART & 0.66613  & 0.83153  \\
%     Madry et al. & 0.65755 & 0.84574   \\
%     Song et al. & 0.51823 & 0.76092  \\
%     $\DIAL_{\ce}$ (Ours) & 0.65318   & $\mathbf{0.88806}$  \\
%     $\DIAL_{\kl}$ (Ours) & $\mathbf{0.67479}$ & 0.84702 \\
%     \midrule
%     \midrule
%     DIAL-AWP (Ours) & $\mathbf{0.69753}$  & $\mathbf{0.85406}$  \\
%     TRADES-AWP & 0.68162 & 0.84917 \\
%     \bottomrule
%   \end{tabular}
% \end{table}

\begin{table}[ht]
\small
  \caption{$F_1$-robust measurement using PGD$^{20}$ attack in white and black box settings on CIFAR-10.}
  \vskip 0.1in
  \label{f1-robust}
  \centering
%   \small
  \begin{tabular}{c
  @{\hspace{1.5\tabcolsep}}c @{\hspace{1.5\tabcolsep}}c @{\hspace{1.5\tabcolsep}}c @{\hspace{1.5\tabcolsep}}c
  @{\hspace{1.5\tabcolsep}}c @{\hspace{1.5\tabcolsep}}c @{\hspace{1.5\tabcolsep}}|
  @{\hspace{1.5\tabcolsep}}c
  @{\hspace{1.5\tabcolsep}}c}
    \toprule
    % \cmidrule(r){8-9}
     & TRADES & MART & AT & ATDA & $\DIAL_{\ce}$ & $\DIAL_{\kl}$ & $\DIAL_{\awp}$ & $\TRADES_{\awp}$ \\
    \midrule
    White-box & 0.659 & 0.666 & 0.657 & 0.518 & 0.660 & \textbf{0.675} & \textbf{0.698} & 0.682 \\
    Black-box & 0.844 & 0.831 & 0.845 & 0.761 & \textbf{0.890} & 0.847 & \textbf{0.854} & 0.849 \\ 
    \bottomrule
  \end{tabular}
\end{table}


% \vspace{-0.5em}
\section{Conclusion}
% \vspace{-0.5em}
Recent advances in multimodal single-cell technology have enabled the simultaneous profiling of the transcriptome alongside other cellular modalities, leading to an increase in the availability of multimodal single-cell data. In this paper, we present \method{}, a multimodal transformer model for single-cell surface protein abundance from gene expression measurements. We combined the data with prior biological interaction knowledge from the STRING database into a richly connected heterogeneous graph and leveraged the transformer architectures to learn an accurate mapping between gene expression and surface protein abundance. Remarkably, \method{} achieves superior and more stable performance than other baselines on both 2021 and 2022 NeurIPS single-cell datasets.

\noindent\textbf{Future Work.}
% Our work is an extension of the model we implemented in the NeurIPS 2022 competition. 
Our framework of multimodal transformers with the cross-modality heterogeneous graph goes far beyond the specific downstream task of modality prediction, and there are lots of potentials to be further explored. Our graph contains three types of nodes. While the cell embeddings are used for predictions, the remaining protein embeddings and gene embeddings may be further interpreted for other tasks. The similarities between proteins may show data-specific protein-protein relationships, while the attention matrix of the gene transformer may help to identify marker genes of each cell type. Additionally, we may achieve gene interaction prediction using the attention mechanism.
% under adequate regulations. 
% We expect \method{} to be capable of much more than just modality prediction. Note that currently, we fuse information from different transformers with message-passing GNNs. 
To extend more on transformers, a potential next step is implementing cross-attention cross-modalities. Ideally, all three types of nodes, namely genes, proteins, and cells, would be jointly modeled using a large transformer that includes specific regulations for each modality. 

% insight of protein and gene embedding (diff task)

% all in one transformer

% \noindent\textbf{Limitations and future work}
% Despite the noticeable performance improvement by utilizing transformers with the cross-modality heterogeneous graph, there are still bottlenecks in the current settings. To begin with, we noticed that the performance variations of all methods are consistently higher in the ``CITE'' dataset compared to the ``GEX2ADT'' dataset. We hypothesized that the increased variability in ``CITE'' was due to both less number of training samples (43k vs. 66k cells) and a significantly more number of testing samples used (28k vs. 1k cells). One straightforward solution to alleviate the high variation issue is to include more training samples, which is not always possible given the training data availability. Nevertheless, publicly available single-cell datasets have been accumulated over the past decades and are still being collected on an ever-increasing scale. Taking advantage of these large-scale atlases is the key to a more stable and well-performing model, as some of the intra-cell variations could be common across different datasets. For example, reference-based methods are commonly used to identify the cell identity of a single cell, or cell-type compositions of a mixture of cells. (other examples for pretrained, e.g., scbert)


%\noindent\textbf{Future work.}
% Our work is an extension of the model we implemented in the NeurIPS 2022 competition. Now our framework of multimodal transformers with the cross-modality heterogeneous graph goes far beyond the specific downstream task of modality prediction, and there are lots of potentials to be further explored. Our graph contains three types of nodes. while the cell embeddings are used for predictions, the remaining protein embeddings and gene embeddings may be further interpreted for other tasks. The similarities between proteins may show data-specific protein-protein relationships, while the attention matrix of the gene transformer may help to identify marker genes of each cell type. Additionally, we may achieve gene interaction prediction using the attention mechanism under adequate regulations. We expect \method{} to be capable of much more than just modality prediction. Note that currently, we fuse information from different transformers with message-passing GNNs. To extend more on transformers, a potential next step is implementing cross-attention cross-modalities. Ideally, all three types of nodes, namely genes, proteins, and cells, would be jointly modeled using a large transformer that includes specific regulations for each modality. The self-attention within each modality would reconstruct the prior interaction network, while the cross-attention between modalities would be supervised by the data observations. Then, The attention matrix will provide insights into all the internal interactions and cross-relationships. With the linearized transformer, this idea would be both practical and versatile.

% \begin{acks}
% This research is supported by the National Science Foundation (NSF) and Johnson \& Johnson.
% \end{acks}


\subsubsection*{Acknowledgments}
We thank Aditya Grover and Chris Cundy for helpful discussions about earlier versions of the method.
We thank Simran Arora, Sabri Eyuboglu, Bibek Paudel, and Nimit Sohoni for valuable feedback on earlier drafts of this work.
This work was done with the support of Google Cloud credits under HAI proposals 540994170283 and 578192719349.
We gratefully acknowledge the support of NIH under No. U54EB020405 (Mobilize), NSF under Nos. CCF1763315 (Beyond Sparsity), CCF1563078 (Volume to Velocity), and 1937301 (RTML); ONR under No. N000141712266 (Unifying Weak Supervision); ONR N00014-20-1-2480: Understanding and Applying Non-Euclidean Geometry in Machine Learning; N000142012275 (NEPTUNE); the Moore Foundation, NXP, Xilinx, LETI-CEA, Intel, IBM, Microsoft, NEC, Toshiba, TSMC, ARM, Hitachi, BASF, Accenture, Ericsson, Qualcomm, Analog Devices, the Okawa Foundation, American Family Insurance, Google Cloud, Salesforce, Total, the HAI-AWS Cloud Credits for Research program, the Stanford Data Science Initiative (SDSI), and members of the Stanford DAWN project: Facebook, Google, and VMWare. The Mobilize Center is a Biomedical Technology Resource Center, funded by the NIH National Institute of Biomedical Imaging and Bioengineering through Grant P41EB027060. The U.S. Government is authorized to reproduce and distribute reprints for Governmental purposes notwithstanding any copyright notation thereon. Any opinions, findings, and conclusions or recommendations expressed in this material are those of the authors and do not necessarily reflect the views, policies, or endorsements, either expressed or implied, of NIH, ONR, or the U.S. Government.


\bibliography{biblio}

\newpage

\appendix

\mySection{Related Works and Discussion}{}
\label{chap3:sec:discussion}

In this section we briefly discuss the similarities and differences of the model presented in this chapter, comparing it with some related work presented earlier (Chapter \ref{chap1:artifact-centric-bpm}). We will mention a few related studies and discuss directly; a more formal comparative study using qualitative and quantitative metrics should be the subject of future work.

Hull et al. \citeyearpar{hull2009facilitating} provide an interoperation framework in which, data are hosted on central infrastructures named \textit{artifact-centric hubs}. As in the work presented in this chapter, they propose mechanisms (including user views) for controlling access to these data. Compared to choreography-like approach as the one presented in this chapter, their settings has the advantage of providing a conceptual rendezvous point to exchange status information. The same purpose can be replicated in this chapter's approach by introducing a new type of agent called "\textit{monitor}", which will serve as a rendezvous point; the behaviour of the agents will therefore have to be slightly adapted to take into account the monitor and to preserve as much as possible the autonomy of agents.

Lohmann and Wolf \citeyearpar{lohmann2010artifact} abandon the concept of having a single artifact hub \cite{hull2009facilitating} and they introduce the idea of having several agents which operate on artifacts. Some of those artifacts are mobile; thus, the authors provide a systematic approach for modelling artifact location and its impact on the accessibility of actions using a Petri net. Even though we also manipulate mobile artifacts, we do not model artifact location; rather, our agents are equipped with capabilities that allow them to manipulate the artifacts appropriately (taking into account their location). Moreover, our approach considers that artifacts can not be remotely accessed, this increases the autonomy of agents.

The process design approach presented in this chapter, has some conceptual similarities with the concept of \textit{proclets} proposed by Wil M. P. van der Aalst et al. \citeyearpar{van2001proclets, van2009workflow}: they both split the process when designing it. In the model presented in this chapter, the process is split into execution scenarios and its specification consists in the diagramming of each of them. Proclets \cite{van2001proclets, van2009workflow} uses the concept of \textit{proclet-class} to model different levels of granularity and cardinality of processes. Additionally, proclets act like agents and are autonomous enough to decide how to interact with each other.

The model presented in this chapter uses an attributed grammar as its mathematical foundation. This is also the case of the AWGAG model by Badouel et al. \citeyearpar{badouel14, badouel2015active}. However, their model puts stress on modelling process data and users as first class citizens and it is designed for Adaptive Case Management.

To summarise, the proposed approach in this chapter allows the modelling and decentralized execution of administrative processes using autonomous agents. In it, process management is very simply done in two steps. The designer only needs to focus on modelling the artifacts in the form of task trees and the rest is easily deduced. Moreover, we propose a simple but powerful mechanism for securing data based on the notion of accreditation; this mechanism is perfectly composed with that of artifacts. The main strengths of our model are therefore : 
\begin{itemize}
	\item The simplicity of its syntax (process specification language), which moreover (well helped by the accreditation model), is suitable for administrative processes;
	\item The simplicity of its execution model; the latter is very close to the blockchain's execution model \cite{hull2017blockchain, mendling2018blockchains}. On condition of a formal study, the latter could possess the same qualities (fault tolerance, distributivity, security, peer autonomy, etc.) that emanate from the blockchain;
	\item Its formal character, which makes it verifiable using appropriate mathematical tools;
	\item The conformity of its execution model with the agent paradigm and service technology.
\end{itemize}
In view of all these benefits, we can say that the objectives set for this thesis have indeed been achieved. However, the proposed model is perfectible. For example, it can be modified to permit agents to respond incrementally to incoming requests as soon as any prefix of the extension of a bud is produced. This makes it possible to avoid the situation observed on figure \ref{chap3:fig:execution-figure-4} where the associated editor is informed of the evolution of the subtree resulting from $C$ only when this one is closed. All the criticisms we can make of the proposed model in particular, and of this thesis in general, have been introduced in the general conclusion (page \pageref{chap5:general-conclusion}) of this manuscript.




\section{Numerical Instability of LSSL}
\label{sec:lssl-instability}


This section proves the claims made in \cref{sec:s4-motivation} about prior work.
We first derive the explicit diagonalization of the HiPPO matrix, confirming its instability because of exponentially large entries.
We then discuss the proposed theoretically fast algorithm from \citep{gu2021lssl} (Theorem 2) and show that it also involves exponentially large terms and thus cannot be implemented.

\subsection{HiPPO Diagonalization}

\begin{proof}[Proof of \cref{lmm:hippo-diagonalization}]%
  The HiPPO matrix \eqref{eq:hippo} is equal, up to sign and conjugation by a diagonal matrix, to
  \begin{align*}
    \bm{A} &=
    \begin{bmatrix}
      1 \\
      -1 & 2 \\
      1 & -3 & 3 \\
      -1 & 3 & -5 & 4 \\
      1 & -3 & 5 & -7 & 5 \\
      -1 & 3 & -5 & 7 & -9 & 6 \\
      1 & -3 & 5 & -7 & 9 & -11 & 7 \\
      -1 & 3 & -5 & 7 & -9 & 11 & -13 & 8 \\
      \vdots & & & & & & & & \ddots \\
    \end{bmatrix}
    \\
    \bm{A}_{nk} &=
    \begin{cases}%
      (-1)^{n-k} (2k+1) & n > k \\
      k+1 & n=k \\
      0 & n<k
    \end{cases}
    .
  \end{align*}
  Our goal is to show that this \( \bm{A} \) is diagonalized by the matrix
  \begin{align*}
    \bm{V} = \binom{i+j}{i-j}_{ij} =
    \begin{bmatrix}
      1 \\
      1 & 1 \\
      1 & 3 & 1 \\
      1 & 6 & 5 & 1 \\
      1 & 10 & 15 & 7 & 1 \\
      1 & 15 & 35 & 28 & 9 & 1 \\
      \vdots & & & & & & \ddots \\
    \end{bmatrix}
    ,
  \end{align*}
  or in other words that columns of this matrix are eigenvectors of \( \bm{A} \).

  Concretely, we will show that the \( j \)-th column of this matrix \( \bm{v}^{(j)} \) with elements
  \begin{align*}
    \bm{v}^{(j)}_i =
    \begin{cases}%
      0 & i < j \\
      \binom{i+j}{i-j} = \binom{i+j}{2j} & i \ge j
    \end{cases}
  \end{align*}
  is an eigenvector with eigenvalue \( j+1 \).
  In other words we must show that for all indices \( k \in [N] \),
  \begin{equation}
    \label{eq:diagonalization-proof}
    (\bm{A}\bm{v}^{(j)})_k = \sum_i \bm{A}_{ki} \bm{v}^{(j)}_{i} = (j+1) \bm{v}^{(j)}_k.
  \end{equation}

  If \( k < j \), then for all \( i \) inside the sum, either \( k < i \) or \( i < j \).
  In the first case \( \bm{A}_{ki} = 0 \) and in the second case \( \bm{v}^{(j)}_i = 0 \),
  so both sides of equation \eqref{eq:diagonalization-proof} are equal to \( 0 \).

  It remains to show the case \( k \ge j \), which proceeds by induction on \( k \).
  Expanding equation \eqref{eq:diagonalization-proof} using the formula for \( \bm{A} \) yields
  \begin{align*}
    (\bm{A}\bm{v})^{(j)}_k = \sum_i \bm{A}_{ki} \bm{v}^{(j)}_{i}
    = \sum_{i=j}^{k-1} (-1)^{k-i} (2i+1) \binom{i+j}{2j} + (k+1) \binom{k+j}{2j}.
  \end{align*}

  In the base case \( k = j \), the sum disappears and we are left with \( (\bm{A}\bm{v}^{(j)})_j = (j+1) \binom{2j}{2j} = (j+1) \bm{v}^{(j)}_j \), as desired.

  Otherwise, the sum for \( (\bm{A}\bm{v})^{(j)}_{k} \) is the same as the sum for \( (\bm{A}\bm{v})^{(j)}_{k-1} \) but with sign reversed and a few edge terms.
  The result follows from applying the inductive hypothesis and algebraic simplification:
  \begin{align*}
    (\bm{A}\bm{v})^{(j)}_{k}
    &= -(\bm{A}\bm{v})^{(j)}_{k-1} - (2k-1) \binom{k-1+j}{2j} + k\binom{k-1+j}{2j} + (k+1)\binom{k+j}{2j}
    \\&= -(j+1)\binom{k-1+j}{2j} - (k-1)\binom{k-1+j}{2j} + (k+1)\binom{k+j}{2j}
    \\&= -(j+k)\binom{k-1+j}{2j} + (k+1)\binom{k+j}{2j}
    \\&= -(j+k)\frac{(k-1+j)!}{(k-1-j)!(2j)!} + (k+1)\binom{k+j}{2j}
    \\&= -\frac{(k+j)!}{(k-1-j)!(2j)!} + (k+1)\binom{k+j}{2j}
    \\&= -(k-j)\frac{(k+j)!}{(k-j)!(2j)!} + (k+1)\binom{k+j}{2j}
    \\&= (j-k)(k+1)\binom{k+j}{2j} + (k+1)\binom{k+j}{2j}
    \\&= (j+1)\bm{v}^{(j)}_k
    .
  \end{align*}

\end{proof}

\subsection{Fast but Unstable LSSL Algorithm}

Instead of diagonalization,
\citet[Theorem 2]{gu2021lssl} proposed a sophisticated fast algorithm to compute
\begin{align*}
  K_L(\bm{\overline{A}}, \bm{\overline{B}}, \bm{\overline{C}}) = (\bm{\overline{C}}\bm{\overline{B}}, \bm{\overline{C}}\bm{\overline{A}}\bm{\overline{B}}, \dots, \bm{\overline{C}}\bm{\overline{A}}^{L-1}\bm{\overline{B}}).
\end{align*}
This algorithm runs in \( O(N\log^2 N + L\log L) \) operations and \( O(N+L) \) space.
However, we now show that this algorithm is also numerically unstable.

There are several reasons for the instability of this algorithm, but most directly we can pinpoint a particular intermediate quantity that they use.
\begin{definition}%
  The fast LSSL algorithm computes coefficients of \( p(x) \), the characteristic polynomial of \( A \), as an intermediate computation.
  Additionally, it computes the coefficients of its inverse, \( p(x)^{-1} \pmod{x^L} \).
\end{definition}

We now claim that this quantity is numerically unfeasible.
We narrow down to the case when \( \bm{\overline{A}} = \bm{I} \) is the identity matrix.
Note that this case is actually in some sense the most typical case:
when discretizing the continuous-time SSM to discrete-time by a step-size \( \dt \),
the discretized transition matrix \( \bm{\overline{A}} \) is brought closer to the identity.
For example, with the Euler discretization \( \bm{\overline{A}} = \bm{I} + \dt \bm{A} \),
we have \( \bm{\overline{A}} \to \bm{I} \) as the step size \( \dt \to 0 \).

\begin{lemma}%
  When \( \bm{\overline{A}} = \bm{I} \), the fast LSSL algorithm requires computing terms exponentially large in \( N \).
\end{lemma}
\begin{proof}%
  The characteristic polynomial of \( \bm{I} \) is
  \begin{align*}
    p(x) =
    \mathsf{det}\left| \bm{I} - x\bm{I} \right|
    = (1-x)^N.
  \end{align*}
  These coefficients have size up to \( \binom{N}{\frac{N}{2}} \approx \frac{2^N}{\sqrt{\pi N/2}} \).

  The inverse of \( p(x) \) has even larger coefficients.
  It can be calculated in closed form by the generalized binomial formula:
  \begin{align*}
    (1-x)^{-N} = \sum_{k=0}^{\infty} \binom{N+k-1}{k}x^k.
  \end{align*}
  Taking this \( \pmod{x^L} \), the largest coefficient is
  \begin{align*}
    \binom{N+L-2}{L-1} = \binom{N+L-2}{N-1} = \frac{(L-1)(L-2)\dots(L-N+1)}{(N-1)!}.
  \end{align*}
  When \( L=N-1 \) this is
  \begin{align*}
    \binom{2(N-1)}{N-1} \approx \frac{2^{2N}}{\sqrt{\pi N}}
  \end{align*}
  already larger than the coefficients of \( (1-x)^N \), and only increases as \( L \) grows.
\end{proof}

\section{\methodabbrv{} Algorithm Details}
\label{sec:s4-details}

This section proves the results of \cref{sec:s4-efficiency}, providing complete details of our efficient algorithms for \methodabbrv{}.

\cref{sec:s4-nplr-proof,sec:s4-recurrence-proof,sec:s4-convolution-proof}
prove \cref{thm:hippo-nplr,thm:s4-recurrence,thm:s4-convolution}
respectively.

\subsection{NPLR Representations of HiPPO Matrices}
\label{sec:s4-nplr-proof}

We first prove \cref{thm:hippo-nplr},
showing that all HiPPO matrices for continuous-time memory fall under the \methodabbrv{} normal plus low-rank (NPLR) representation.

\begin{proof}[Proof of \cref{thm:hippo-nplr}]%
  We consider each of the three cases HiPPO-LagT, HiPPO-LegT, and HiPPO-LegS separately.
  Note that the primary HiPPO matrix defined in this work (equation \eqref{eq:hippo}) is the HiPPO-LegT matrix.

  \textbf{HiPPO-LagT.}
  The HiPPO-LagT matrix is simply
  \begin{align*}
    \bm{A}_{nk} &=
    \begin{cases}%
      0            & n < k \\
      -\frac{1}{2} & n=k   \\
      -1           & n > k \\
    \end{cases}
    \\
    \bm{A} &=
    -
    \begin{bmatrix}
      \frac{1}{2} &             &             &            & \dots \\
      1           & \frac{1}{2} &             &             \\
      1           & 1           & \frac{1}{2} &             \\
      1           & 1           & 1           & \frac{1}{2} \\
      \vdots      &             &             &              & \ddots \\
    \end{bmatrix}
    .
  \end{align*}
  Adding the matrix of all \( \frac{1}{2} \), which is rank 1, yields
  \begin{align*}
    -
    \begin{bmatrix}
      & -\frac{1}{2} & -\frac{1}{2} & -\frac{1}{2} \\
      \frac{1}{2} &              & -\frac{1}{2} & -\frac{1}{2} \\
      \frac{1}{2} & \frac{1}{2}  &              & -\frac{1}{2} \\
      \frac{1}{2} & \frac{1}{2}  & \frac{1}{2}  &              \\
    \end{bmatrix}
    .
  \end{align*}
  This matrix is now skew-symmetric.
  Skew-symmetric matrices are a particular case of normal matrices
  with pure-imaginary eigenvalues.

  \citet{gu2020hippo} also consider a case of HiPPO corresponding to the generalized Laguerre polynomials that generalizes
  the above HiPPO-LagT case.
  In this case, the matrix \( \bm{A} \) (up to conjugation by a diagonal matrix) ends up being close to the above matrix,
  but with a different element on the diagonal.
  After adding the rank-1 correction, it becomes the above skew-symmetric matrix plus a multiple of the identity.
  Thus after diagonalization by the same matrix as in the LagT case, it is still reduced to diagonal plus low-rank (DPLR) form,
  where the diagonal is now pure imaginary plus a real constant.

  \textbf{HiPPO-LegS.}
  We restate the formula from equation \eqref{eq:hippo} for convenience.
  \begin{align*}
    \bm{A}_{nk}
    =
    -
    \begin{cases}
      (2n+1)^{1/2}(2k+1)^{1/2} & \mbox{if } n > k \\
      n+1                      & \mbox{if } n = k \\
      0                        & \mbox{if } n < k
    \end{cases}
    .
  \end{align*}
  Adding \( \frac{1}{2}(2n+1)^{1/2}(2k+1)^{1/2} \) to the whole matrix gives
  \begin{align*}
    -
    \begin{cases}
      \frac{1}{2} (2n+1)^{1/2}(2k+1)^{1/2}  & \mbox{if } n > k \\
      \frac{1}{2}                           & \mbox{if } n = k \\
      -\frac{1}{2} (2n+1)^{1/2}(2k+1)^{1/2} & \mbox{if } n < k \\
    \end{cases}
  \end{align*}

  Note that this matrix is not skew-symmetric,
  but is \( \frac{1}{2}\bm{I} + \bm{S} \) where \( \bm{S} \) is a skew-symmetric matrix.
  This is diagonalizable by the same unitary matrix that diagonalizes \( \bm{S} \).

  \textbf{HiPPO-LegT.}

  Up to the diagonal scaling,
  the LegT matrix is
  \begin{align*}
    \bm{A} =
    -
    \begin{bmatrix}
      1      & -1 & 1  & -1  & \dots \\
      1      & 1  & -1 & 1  \\
      1      & 1  & 1  & -1 \\
      1      & 1  & 1  & 1  \\
      \vdots &    &    &     & \ddots
    \end{bmatrix}
    .
  \end{align*}
  By adding \( -1 \) to this matrix and then the matrix
  \begin{align*}
    \begin{bmatrix}
      &  &   &  &  \\
      2 &  & 2 &  &  \\
      &  &   &  &  \\
      2 &  & 2 &  &  \\
    \end{bmatrix}
  \end{align*}
  the matrix becomes
  \begin{align*}
    \begin{bmatrix}
      & -2 &   & -2 &  \\
      2 &    &   &    &  \\
      &    &   & -2 &  \\
      2 &    & 2 &    &  \\
    \end{bmatrix}
  \end{align*}
  which is skew-symmetric.
  In fact, this matrix is the inverse of the Chebyshev Jacobi.

  An alternative way to see this is as follows.
  The LegT matrix is the inverse of the matrix
  \begin{align*}
    \begin{bmatrix}
      -1 & 1  &    & 0  \\
      -1 &    & 1  &    \\
      & -1 &    & 1  \\
      &    & -1 & -1 \\
    \end{bmatrix}
  \end{align*}
  This can obviously be converted to a skew-symmetric matrix by adding a rank 2 term.
  The inverses of these matrices are also rank-2 differences from each other by the Woodbury identity.

  A final form is
  \begin{align*}
    \begin{bmatrix}
      -1 & 1 & -1 & 1 \\
      -1 & -1 & 1 & -1 \\
      -1 & -1 & -1 & 1 \\
      -1 & -1 & -1 & -1 \\
    \end{bmatrix}
    +
    \begin{bmatrix}
      1 & 0 & 1 & 0 \\
      0 & 1 & 0 & 1 \\
      1 & 0 & 1 & 0 \\
      0 & 1 & 0 & 1 \\
    \end{bmatrix}
    =
    \begin{bmatrix}
      0 & 1 & 0 & 1 \\
      -1 & 0 & 1 & 0 \\
      0 & -1 & 0 & 1 \\
      -1 & 0 & -1 & 0 \\
    \end{bmatrix}
  \end{align*}
  This has the advantage that the rank-2 correction is symmetric (like the others),
  but the normal skew-symmetric matrix is now \( 2 \)-quasiseparable instead of \( 1 \)-quasiseparable.

\end{proof}

\subsection{Computing the \methodabbrv{} Recurrent View}
\label{sec:s4-recurrence-proof}

We prove \cref{thm:s4-recurrence} showing the efficiency of the \methodabbrv{} parameterization for computing one step of the recurrent representation (\cref{sec:ss-recurrent}).

Recall that without loss of generality, we can assume that the state matrix \( \bm{A} = \bm{\Lambda} - \bm{P}\bm{Q}^* \) is diagonal plus low-rank (DPLR), potentially over \( \mathbbm{C} \).
Our goal in this section is to explicitly write out a closed form for the discretized matrix \( \bm{\overline{A}} \).

Recall from equation \eqref{eq:2} that
\begin{align*}
  \bm{\overline{A}} &= (\bm{I} - \dt/2 \cdot \bm{A})^{-1}(\bm{I} + \dt/2 \cdot \bm{A}) \\
  \bm{\overline{B}} &= (\bm{I} - \dt/2 \cdot \bm{A})^{-1} \dt \bm{B}
  .
\end{align*}



We first simplify both terms in the definition of \( \bm{\overline{A}} \) independently.

\textbf{Forward discretization.}
The first term is essentially the Euler discretization motivated in \cref{sec:ss-recurrent}.
\begin{align*}
  \bm{I} + \frac{\dt}{2} \bm{A}
  &= \bm{I} + \frac{\dt}{2} (\bm{\Lambda} - \bm{P} \bm{Q}^*)
  \\&= \frac{\dt}{2} \left[ \frac{2}{\dt}\bm{I} + (\bm{\Lambda} - \bm{P} \bm{Q}^*) \right]
  \\&= \frac{\dt}{2} \bm{A_0}
\end{align*}
where \( \bm{A_0} \) is defined as the term in the final brackets.

\textbf{Backward discretization.}
The second term is known as the Backward Euler's method.
Although this inverse term is normally difficult to deal with,
in the DPLR case we can simplify it using Woodbury's Identity (\cref{prop:woodbury}).
\begin{align*}
  \left( \bm{I} - \frac{\dt}{2} \bm{A} \right)^{-1}
  &=
  \left( \bm{I} - \frac{\dt}{2} (\bm{\Lambda} - \bm{P} \bm{Q}^*) \right)^{-1}
  \\&=
  \frac{2}{\dt} \left[ \frac{2}{\dt} - \bm{\Lambda} + \bm{P} \bm{Q}^* \right]^{-1}
  \\&=
  \frac{2}{\dt} \left[ \bm{D} - \bm{D} \bm{P} \left( \bm{I} + \bm{Q}^* \bm{D} \bm{P} \right)^{-1} \bm{Q}^* \bm{D} \right]
  \\&= \frac{2}{\dt} \bm{A_1}
\end{align*}
where \( \bm{D} = \left( \frac{2}{\dt}-\bm{\Lambda} \right)^{-1} \)
and \( \bm{A_1} \) is defined as the term in the final brackets.
Note that
\( \left( 1 + \bm{Q}^* \bm{D} \bm{P} \right) \)
is actually a scalar in the case when the low-rank term has rank \( 1 \).


\textbf{\methodabbrv{} Recurrence.}
Finally, the full bilinear discretization can be rewritten in terms of these matrices as
\begin{align*}
  \bm{\overline{A}} &= \bm{A_1} \bm{A_0} \\
  \bm{\overline{B}} &= \frac{2}{\dt} \bm{A_1} \dt \bm{B} = 2 \bm{A_1} \bm{B}
  .
\end{align*}
The discrete-time SSM \eqref{eq:2} becomes
\begin{align*}
  x_{k} &= \bm{\overline{A}} x_{k-1} + \bm{\overline{B}} u_k \\
  &= \bm{A_1} \bm{A_0} x_{k-1} + 2 \bm{A_1} \bm{B} u_k \\
  y_k &= \bm{C} x_k
  .
\end{align*}
Note that \( \bm{A_0}, \bm{A_1} \) are accessed only through matrix-vector multiplications.
Since they are both DPLR, they have \( O(N) \) matrix-vector multiplication,
showing \cref{thm:s4-recurrence}.


\subsection{Computing the Convolutional View}
\label{sec:s4-convolution-proof}

The most involved part of using SSMs efficiently is computing \( \bm{\overline{K}} \).
This algorithm was sketched in \cref{sec:s4-overview} and is the main motivation for the \methodabbrv{} parameterization.
In this section, we define the necessary intermediate quantities and prove the main technical result. %


The algorithm for \cref{thm:s4-convolution} falls in roughly three stages, leading to \cref{alg:s4-convolution}.
Assuming \( \bm{A} \) has been conjugated into diagonal plus low-rank form, we successively simplify the problem of computing \( \bm{\overline{K}} \)
by applying the techniques outlined in \cref{sec:s4-overview}.

\begin{remark}
  \textbf{We note that for the remainder of this section}, we transpose \( \bm{C} \) to be a column vector of shape \( \mathbbm{C}^{N} \) or \( \mathbbm{C}^{N \times 1} \) instead of matrix or row vector \( \mathbbm{C}^{1 \times N} \) as in \eqref{eq:1}.
  In other words the SSM is
  \begin{equation}
    \begin{aligned}
      x'(t) &= \bm{A}x(t) + \bm{B}u(t) \\
      y(t) &= \bm{C}^* x(t) + \bm{D}u(t)
      .
    \end{aligned}
  \end{equation}
  This convention is made so that \( \bm{C} \) has the same shape as \( \bm{B}, \bm{P}, \bm{Q} \) and simplifies the implementation of S4.
\end{remark}

\paragraph{Reduction 0: Diagonalization}
By \cref{lmm:conjugation}, we can switch the representation by conjugating with any unitary matrix.
For the remainder of this section, we can assume that \( \bm{A} \) is (complex) diagonal plus low-rank (DPLR).

Note that unlike diagonal matrices, a DPLR matrix does not lend itself to efficient computation of \( \bm{\overline{K}} \).
The reason is that \( \bm{\overline{K}} \) computes terms \( \bm{\overline{C}}^* \bm{\overline{A}}^i \bm{\overline{B}} \) which involve powers of the matrix \( \bm{\overline{A}} \).
These are trivially computable when \( \bm{\overline{A}} \) is diagonal, but is no longer possible for even simple modifications to diagonal matrices such as DPLR.

\paragraph{Reduction 1: SSM Generating Function}

To address the problem of computing powers of \( \bm{\overline{A}} \), we introduce another technique.
Instead of computing the SSM convolution filter \( \bm{\overline{K}} \) directly,
we introduce a generating function on its coefficients and compute evaluations of it.

\begin{definition}[SSM Generating Function]%
  \label{def:generating-function}
  We define the following quantities:
  \begin{itemize}%
    \item The \emph{SSM convolution function} is \( \mathcal{K}(\bm{\overline{A}}, \bm{\overline{B}}, \bm{\overline{C}}) = (\bm{\overline{C}}^*\bm{\overline{B}}, \bm{\overline{C}}^*\bm{\overline{A}}\bm{\overline{B}}, \dots) \)
      and the (truncated) SSM filter of length \( L \)
      \begin{equation}
        \mathcal{K}_L(\bm{\overline{A}}, \bm{\overline{B}}, \bm{\overline{C}}) = (\bm{\overline{C}}^*\bm{\overline{B}}, \bm{\overline{C}}^*\bm{\overline{A}}\bm{\overline{B}}, \dots, \bm{\overline{C}}^*\bm{\overline{A}}^{L-1}\bm{\overline{B}}) \in \mathbbm{R}^L
      \end{equation}
    \item The \emph{SSM generating function} at node \( z \) is
      \begin{equation}
        \label{eq:generating-function}
        \hat{\mathcal{K}}(z; \bm{\overline{A}}, \bm{\overline{B}}, \bm{\overline{C}}) \in \mathbbm{C} := \sum_{i=0}^\infty \bm{\overline{C}}^* \bm{\overline{A}}^i \bm{\overline{B}} z^i
        = \bm{\overline{C}}^* (\bm{I} - \bm{\overline{A}} z)^{-1} \bm{\overline{B}}
      \end{equation}
      and the \emph{truncated SSM generating function} at node \( z \) is
      \begin{equation}
        \hat{\mathcal{K}}_L(z; \bm{\overline{A}}, \bm{\overline{B}}, \bm{\overline{C}})^* \in \mathbbm{C} := \sum_{i=0}^{L-1} \bm{\overline{C}}^* \bm{\overline{A}}^i \bm{\overline{B}} z^i
        = \bm{\overline{C}}^* (\bm{I} - \bm{\overline{A}}^L z^L) (\bm{I} - \bm{\overline{A}} z)^{-1} \bm{\overline{B}}
      \end{equation}
    \item The truncated SSM generating function at nodes \( \Omega \in \mathbbm{C}^M \) is
      \begin{equation}
        \hat{\mathcal{K}}_L(\Omega; \bm{\overline{A}}, \bm{\overline{B}}, \bm{\overline{C}}) \in \mathbbm{C}^M := \left( \hat{\mathcal{K}}_L(\omega_k; \bm{\overline{A}}, \bm{\overline{B}}, \bm{\overline{C}}) \right)_{k \in [M]}
      \end{equation}

  \end{itemize}
\end{definition}

Intuitively, the generating function essentially converts the SSM convolution filter from the time domain to frequency domain.
Importantly, it preserves the same information, and the desired SSM convolution filter can be recovered from evaluations of its generating function.
\begin{lemma}%
  The SSM function \( \mathcal{K}_L(\bm{\overline{A}}, \bm{\overline{B}}, \bm{\overline{C}}) \) can be computed from the SSM generating function \( \hat{\mathcal{K}}_L(\Omega; \bm{\overline{A}}, \bm{\overline{B}}, \bm{\overline{C}}) \)
  at the roots of unity \( \Omega = \{ \exp(-2\pi i \frac{k}{L} : k \in [L] \} \)
  stably in \( O(L \log L) \) operations.
\end{lemma}
\begin{proof}%
  For convenience define
  \begin{align*}
    \bm{\overline{K}} &= \mathcal{K}_L(\bm{\overline{A}}, \bm{\overline{B}}, \bm{\overline{C}}) \\
    \bm{\hat{K}} &=  \hat{\mathcal{K}}_L(\Omega; \bm{\overline{A}}, \bm{\overline{B}}, \bm{\overline{C}}) \\
    \bm{\hat{K}}(z) &=  \hat{\mathcal{K}}_L(z; \bm{\overline{A}}, \bm{\overline{B}}, \bm{\overline{C}})
    .
  \end{align*}
  Note that
  \begin{align*}
    \bm{\hat{K}}_j = \sum_{k=0}^{L-1} \bm{\overline{K}}_k \exp\left(-2\pi i \frac{jk}{L}\right).
  \end{align*}
  Note that this is exactly the same as the Discrete Fourier Transform (DFT):
  \begin{align*}
    \bm{\hat{K}} = \mathcal{F}_L \bm{K}.
  \end{align*}
  Therefore \( \bm{K} \) can be recovered from \( \bm{\hat{K}} \) with a single inverse DFT,
  which requires \( O(L \log L) \) operations with the Fast Fourier Transform (FFT) algorithm.
\end{proof}


\paragraph{Reduction 2: Woodbury Correction}

The primary motivation of \cref{def:generating-function} is that it turns \emph{powers} of \( \bm{\overline{A}} \) into a single \emph{inverse} of \( \bm{\overline{A}} \) (equation \eqref{eq:generating-function}).
While DPLR matrices cannot be powered efficiently due to the low-rank term, they can be inverted efficiently by the well-known Woodbury identity.

\begin{proposition}[Binomial Inverse Theorem or Woodbury matrix identity~\cite{woodbury1950,golub2013matrix}]
  \label{prop:woodbury}
  Over a commutative ring $\mathcal{R}$, let $\bm{A} \in \mathcal{R}^{N \times N}$ and $\bm{U},\bm{V} \in \mathcal{R}^{N \times p}$. Suppose $\bm{A}$ and $\bm{A}+\bm{U}\bm{V}^*$ are invertible. Then $\bm{I}_p + \bm{V}^*\bm{A}^{-1}\bm{U} \in \mathcal{R}^{p \times p}$ is invertible and
  \begin{align*}
    (\bm{A} + \bm{U}\bm{V}^*)^{-1} = \bm{A}^{-1} - \bm{A}^{-1}\bm{U}(\bm{I}_p + \bm{V}^*\bm{A}^{-1}\bm{U})^{-1}\bm{V}^*\bm{A}^{-1}
  \end{align*}
\end{proposition}

With this identity, we can convert the SSM generating function on a DPLR matrix \( \bm{A} \) into one on just its diagonal component.

\begin{lemma}%
  \label{lmm:resolvent-woodbury}
  Let \( \bm{A} = \bm{\Lambda} - \bm{P} \bm{Q}^* \) be a diagonal plus low-rank representation.
  Then for any root of unity \( z \in \Omega \), the truncated generating function satisfies
  \begin{align*}
    \bm{\hat{K}}(z) &=
    \frac{2}{1+z}\left[ \bm{\tilde{C}}^* \bm{R}(z) \bm{B} - \bm{\tilde{C}}^* \bm{R}(z) \bm{P} \left( 1 + \bm{Q}^* \bm{R}(z) \bm{P} \right)^{-1} \bm{Q}^* \bm{R}(z) \bm{B} \right]
    \\
    \bm{\tilde{C}} &= (\bm{I}-\bm{\overline{A}}^L)^* \bm{C}
    \\
    \bm{R}(z; \bm{\Lambda}) &= \left(\frac{2}{\dt} \frac{1-z}{1+z} - \bm{\Lambda}\right)^{-1}
    .
  \end{align*}
\end{lemma}
%
\begin{proof}%
  Directly expanding \cref{def:generating-function} yields
  \begin{align*}
    \mathcal{K}_L(z; \bm{\overline{A}}, \bm{\overline{B}}, \bm{\overline{C}})
    &=
    \bm{\overline{C}}^* \bm{\overline{B}} + \bm{\overline{C}}^* \bm{\overline{A}} \bm{\overline{B}} z + \dots + \bm{\overline{C}}^* \bm{\overline{A}}^{L-1} \bm{\overline{B}} z^{L-1}
    \\&=
    \bm{\overline{C}}^* \left(\bm{I}-\bm{\overline{A}}^L\right) \left(\bm{I} - \bm{\overline{A}} z\right)^{-1} \bm{\overline{B}}
    \\&=
    \bm{\tilde{C}}^* \left(\bm{I} - \bm{\overline{A}} z\right)^{-1} \bm{\overline{B}}
  \end{align*}
  where \(  \bm{\tilde{C}}^* = \bm{C}^* \left(\bm{I}-\bm{\overline{A}}^L\right) \).

  We can now explicitly expand the discretized SSM matrices \( \bm{\overline{A}} \) and \( \bm{\overline{B}} \) back in terms of the original SSM parameters \( \bm{A} \) and \( \bm{B} \).
  \cref{lmm:bilinear-resolvent} provides an explicit formula, which allows further simplifying
  \begin{align*}
    \bm{\tilde{C}}^* \left(\bm{I} - \bm{\overline{A}} z\right)^{-1} \bm{\overline{B}}
    &= \frac{2}{1+z} \bm{\tilde{C}}^* \left(\frac{2}{\Delta} \frac{1-z}{1+z} - \bm{A}\right)^{-1}  \bm{B}
    \\&=
    \frac{2}{1+z} \bm{\tilde{C}}^* \left(\frac{2}{\Delta} \frac{1-z}{1+z} - \bm{\Lambda} + \bm{P}\bm{Q}^* \right)^{-1} \bm{B}
    \\&=
    \frac{2}{1+z}\left[ \bm{\tilde{C}}^* \bm{R}(z) \bm{B} - \bm{\tilde{C}}^* \bm{R}(z) \bm{P} \left( 1 + \bm{Q}^* \bm{R}(z) \bm{P} \right)^{-1} \bm{Q}^* \bm{R}(z) \bm{B} \right]
    .
  \end{align*}
  The last line applies the Woodbury Identity (\cref{prop:woodbury}) where \( \bm{R}(z) = \left(\frac{2}{\Delta} \frac{1-z}{1+z} - \bm{\Lambda}\right)^{-1} \).
\end{proof}


The previous proof used the following self-contained result to back out the original SSM matrices from the discretization.
\begin{lemma}%
  \label{lmm:bilinear-resolvent}
  Let \( \bm{\overline{A}}, \bm{\overline{B}} \) be the SSM matrices \( \bm{A}, \bm{B} \) discretized by the bilinear discretization with step size \( \dt \). Then
  \begin{align*}
    \bm{C}^*\left(\bm{I} - \bm{\overline{A}z} \right)^{-1} \bm{\overline{B}}
    =
    \frac{2\Delta}{1+z} \bm{C}^* \left[ {2 \frac{1-z}{1+z}} - \Delta \bm{A} \right]^{-1} \bm{B}
  \end{align*}
\end{lemma}
\begin{proof}%
  Recall that the bilinear discretization that we use (equation \eqref{eq:2}) is
  \begin{align*}
    \bm{\overline{A}}
    &=
    \left(\bm{I} - \frac{\Delta}{2} \bm{A}\right)^{-1} \left(\bm{I} + \frac{\Delta}{2} \bm{A}\right)
    \\
    \bm{\overline{B}} &= \left(\bm{I} - \frac{\Delta}{2} \bm{A}\right)^{-1} \Delta \bm{B}
  \end{align*}
  The result is proved algebraic manipulations.
  \begin{align*}
    \bm{C}^*\left(\bm{I} - \bm{\overline{A}} z\right)^{-1} \bm{\overline{B}}
    &=
    \bm{C}^* \left[ \left(\bm{I} - \frac{\Delta}{2} \bm{A}\right)^{-1}\left(\bm{I} - \frac{\Delta}{2} \bm{A}\right)  - \left(\bm{I} - \frac{\Delta}{2} \bm{A}\right)^{-1} \left(\bm{I} + \frac{\Delta}{2} \bm{A}\right) z \right]^{-1} \bm{\overline{B}}
    \\&=
    \bm{C}^* \left[ \left(\bm{I} - \frac{\Delta}{2} \bm{A}\right) - \left(\bm{I} + \frac{\Delta}{2} \bm{A}\right) z \right]^{-1} \left(\bm{I} - \frac{\Delta}{2} \bm{A}\right) \bm{\overline{B}}
    \\&=
    \bm{C}^* \left[ \bm{I}(1 - z) - \frac{\Delta}{2} \bm{A} (1+z) \right]^{-1} \Delta\bm{B}
    \\&=
    \frac{\Delta}{1-z} \bm{C}^* \left[ \bm{I} - \frac{\Delta \bm{A}}{2 \frac{1-z}{1+z}} \right]^{-1} \bm{B}
    \\&=
    \frac{2\Delta}{1+z} \bm{C}^* \left[ {2 \frac{1-z}{1+z}} \bm{I} - \Delta \bm{A} \right]^{-1} \bm{B}
  \end{align*}
\end{proof}

Note that in the \methodabbrv{} parameterization, instead of constantly computing \( \bm{\tilde{C}} = \left(\bm{I} - \bm{\overline{A}}^L\right)^* \bm{C} \),
we can simply reparameterize our parameters to learn \( \bm{\tilde{C}} \) directly instead of \( \bm{C} \),
saving a minor computation cost and simplifying the algorithm.

\paragraph{Reduction 3: Cauchy Kernel}
We have reduced the original problem of computing \( \bm{\overline{K}} \) to the problem of computing the SSM generating function \( \hat{\mathcal{K}}_L(\Omega; \bm{\overline{A}}, \bm{\overline{B}}, \bm{\overline{C}}) \)
in the case that \( \bm{\overline{A}} \) is a diagonal matrix.
We show that this is exactly the same as a Cauchy kernel, which is a well-studied problem with fast and stable numerical algorithms.

\begin{definition}%
  \label{def:cauchy}
  A \textbf{Cauchy matrix} or kernel on nodes \( \Omega = (\omega_i) \in \mathbbm{C}^M \) and \( \Lambda = (\lambda_j) \in \mathbbm{C}^N \) is
  \begin{align*}
    \bm{M} \in \mathbbm{C}^{M \times N} &= \bm{M}(\Omega, \Lambda) = (\bm{M}_{ij})_{i \in [M], j \in [N]} \qquad
    \bm{M}_{ij} = \frac{1}{\omega_i - \lambda_j}
    .
  \end{align*}
  The computation time of a Cauchy matrix-vector product of size \( M \times N \) is denoted by \( \mathcal{C}(M, N) \).
\end{definition}

Computing with Cauchy matrices is an extremely well-studied problem in numerical analysis,
with both fast arithmetic algorithms and fast numerical algorithms based on the famous Fast Multipole Method (FMM)
\citep{pan2001structured,pan2015transformations,pan2017fast}.
\begin{proposition}[Cauchy]%
  \label{prop:cauchy}
  A Cauchy kernel requires \( O(M+N) \) space, and operation count
  \begin{align*}
    \mathcal{C}(M, N) =
    \begin{cases}%
      O\left( MN \right)  & \text{naively} \\
      O\left( (M+N) \log^2(M+N) \right) & \text{in exact arithmetic} \\
      O\left( (M+N) \log(M+N) \log \frac{1}{\varepsilon} \right) & \text{numerically to precision \( \varepsilon \)}
      .
    \end{cases}
  \end{align*}
\end{proposition}

\begin{corollary}%
  Evaluating \( \bm{Q}^* \bm{R}(\Omega; \Lambda) \bm{P} \) (defined in \cref{lmm:resolvent-woodbury}) for any set of nodes \( \Omega \in \mathbbm{C}^L \), diagonal matrix \( \Lambda \), and vectors \( \bm{P}, \bm{Q} \) can be computed in \( \mathcal{C}(L,N) \) operations and \( O(L+N) \) space, where \( \mathcal{C}(L,N) = \tilde{O}(L+N) \) is the cost of a Cauchy matrix-vector multiplication.
\end{corollary}
\begin{proof}%
  For any fixed \( \omega \in \Omega \), we want to compute \( \sum_{j} \frac{q_j^* p_j}{\omega - \lambda_j} \). Computing this over all \( \omega_i \) is therefore exactly a Cauchy matrix-vector multiplication.
\end{proof}

This completes the proof of \cref{thm:s4-convolution}.
In \cref{alg:s4-convolution},
note that the work is dominated by Step \ref{step:cauchy},
which has a constant number of calls to a black-box Cauchy kernel, with complexity given by \cref{prop:cauchy}.



\section{Experiment Details and Full Results}
\label{sec:experiment-details}

This section contains full experimental procedures and extended results and citations for our experimental evaluation in \cref{sec:experiments}.
\cref{sec:experiment-details-benchmarking} corresponds to benchmarking results in \cref{sec:experiments-benchmark},
\cref{sec:experiment-details-lrd} corresponds to LRD experiments (LRA and Speech Commands) in \cref{sec:experiments-lrd},
and \cref{sec:experiment-details-general} corresponds to the general sequence modeling experiments (generation, image classification, forecasting) in \cref{sec:experiments-general}.

\subsection{Benchmarking}
\label{sec:experiment-details-benchmarking}

Benchmarking results from \cref{tab:ssm-benchmark} and \cref{tab:lra-benchmark} were tested on a single A100 GPU.

\paragraph{Benchmarks against LSSL}

For a given dimension \( H \), a single LSSL or \methodabbrv{} layer was constructed with \( H \) hidden features.
For LSSL, the state size \( N \) was set to \( H \) as done in \citep{gu2021lssl}.
For \methodabbrv{}, the state size \( N \) was set to parameter-match the LSSL, which was a state size of \( \frac{N}{4} \) due to differences in the parameterization.
\cref{tab:ssm-benchmark} benchmarks a single forward+backward pass of a single layer.

\paragraph{Benchmarks against Efficient Transformers}
Following \citep{tay2021long}, the Transformer models had 4 layers, hidden dimension \( 256 \) with \( 4 \) heads, query/key/value projection dimension \( 128 \), and batch size \( 32 \), for a total of roughly \( 600k \) parameters.
The \methodabbrv{} model was parameter tied while keeping the depth and hidden dimension constant (leading to a state size of \( N = 256 \)).

We note that the relative orderings of these methods can vary depending on the exact hyperparameter settings.

\subsection{Long-Range Dependencies}
\label{sec:experiment-details-lrd}
This section includes information for reproducing our experiments on the Long-Range Arena and Speech Commands long-range dependency tasks.

\paragraph{Long Range Arena}

\cref{tab:lra-full} contains extended results table with all 11 methods considered in \citep{tay2021long}.

\begin{table}[t]
  \small
  \caption{Full results for the Long Range Arena (LRA) benchmark for long-range dependencies in sequence models. (Top): Original Transformer variants in LRA. (Bottom): Other models reported in the literature.}
    \centering
    \begin{tabular}{@{}llllllll@{}}
        \toprule
        Model                 & \textsc{ListOps}  & \textsc{Text}     & \textsc{Retrieval} & \textsc{Image}    & \textsc{Pathfinder} & \textsc{Path-X} & \textsc{Avg}      \\
        \midrule
        Random                & 10.00             & 50.00             & 50.00              & 10.00             & 50.00               & 50.00           & 36.67             \\
        \midrule
        Transformer           & 36.37             & 64.27             & 57.46              & 42.44             & 71.40               & \xmark          & 53.66             \\
        Local Attention       & 15.82             & 52.98             & 53.39              & 41.46             & 66.63               & \xmark          & 46.71             \\
        Sparse Trans.         & 17.07             & 63.58             & 59.59              & 44.24             & 71.71               & \xmark          & 51.03             \\
        Longformer            & 35.63             & 62.85             & 56.89              & 42.22             & 69.71               & \xmark          & 52.88             \\
        Linformer             & 35.70             & 53.94             & 52.27              & 38.56             & 76.34               & \xmark          & 51.14             \\
        Reformer              & \underline{37.27} & 56.10             & 53.40              & 38.07             & 68.50               & \xmark          & 50.56             \\
        Sinkhorn Trans.       & 33.67             & 61.20             & 53.83              & 41.23             & 67.45               & \xmark          & 51.23             \\
        Synthesizer           & 36.99             & 61.68             & 54.67              & 41.61             & 69.45               & \xmark          & 52.40             \\
        BigBird               & 36.05             & 64.02             & 59.29              & 40.83             & 74.87               & \xmark          & 54.17             \\
        Linear Trans.         & 16.13             & \underline{65.90} & 53.09              & 42.34             & 75.30               & \xmark          & 50.46             \\
        Performer             & 18.01             & 65.40             & 53.82              & 42.77             & 77.05               & \xmark          & 51.18             \\
        \midrule
        FNet                  & 35.33             & 65.11             & 59.61              & 38.67             & \underline{77.80}   & \xmark          & 54.42             \\
        Nystr{\"o}mformer     & 37.15             & 65.52             & \underline{79.56}  & 41.58             & 70.94               & \xmark          & 57.46             \\
        Luna-256              & 37.25             & 64.57             & 79.29              & \underline{47.38} & 77.72               & \xmark          & \underline{59.37} \\
        \textbf{\methodabbrv} (original) & 58.35   & 76.02 & 87.09     & 87.26 & 86.05      & 88.10  & 80.48 \\
        \textbf{\methodabbrv} (updated)  & \textbf{59.60}   & \textbf{86.82} & \textbf{90.90}     & \textbf{88.65} & \textbf{94.20}      & \textbf{96.35}  & \textbf{86.09} \\
        \bottomrule
    \end{tabular}
    \label{tab:lra-full}
\end{table}

For the \methodabbrv{} model, hyperparameters for all datasets are reported in \cref{tab::best-hyperparameters}.
For all datasets, we used the AdamW optimizer with a constant learning rate schedule with decay on validation plateau.
However, the learning rate on HiPPO parameters (in particular \( \bm{\Lambda}, \bm{P}, \bm{Q}, \bm{B}, \bm{C}, \dt \)) were reduced to a maximum starting LR of \( 0.001 \), which improves stability since the HiPPO equation is crucial to performance.

The \methodabbrv{} state size was always fixed to \( N=64 \).

As \methodabbrv{} is a sequence-to-sequence model with output shape (batch, length, dimension) and LRA tasks are classification,
mean pooling along the length dimension was applied after the last layer.

We note that most of these results were trained for far longer than what was necessary to achieve SotA results (e.g., the \texttt{Image} task reaches SotA in 1 epoch).
Results often keep improving with longer training times.

\textbf{Updated results.}
The above hyperparameters describe the results reported in the original paper, shown in \cref{tab:lra-full}, which have since been improved.
See \cref{sec:reproduction}.

\textbf{Hardware.}
All models were run on single GPU.
Some tasks used an A100 GPU (notably, the Path-X experiments), which has a larger max memory of 40Gb.
To reproduce these on smaller GPUs, the batch size can be reduced or gradients can be accumulated for two batches.

\begin{table*}[!t]
  \caption{
    The values of the best hyperparameters found for classification datasets; LRA (Top) and images/speech (Bottom).
    LR is learning rate and WD is weight decay. BN and LN refer to Batch Normalization and Layer Normalization.
  }
  \label{tab::best-hyperparameters}
  \centering
  \resizebox{\textwidth}{!}{%
    \begin{tabular}{@{}llllllllllll@{}}
      \toprule
                                      & \textbf{Depth} & \textbf{Features \( H \)} & \textbf{Norm} & \textbf{Pre-norm} & {\bf Dropout} & {\bf LR} & {\bf Batch Size} & {\bf Epochs} & \textbf{WD} & \textbf{Patience} \\
      \midrule
      \textbf{ListOps}                & 6              & 128                       & BN            & False             & 0             & 0.01     & 100              & 50           & 0.01        & 5                 \\
      \textbf{Text}                   & 4              & 64                        & BN            & True              & 0             & 0.001    & 50               & 20           & 0           & 5                 \\
      \textbf{Retrieval}              & 6              & 256                       & BN            & True              & 0             & 0.002    & 64               & 20           & 0           & 20                \\
      \textbf{Image}                  & 6              & 512                       & LN            & False             & 0.2           & 0.004    & 50               & 200          & 0.01        & 20                \\
      \textbf{Pathfinder}             & 6              & 256                       & BN            & True              & 0.1           & 0.004    & 100              & 200          & 0           & 10                \\
      \textbf{Path-X}                 & 6              & 256                       & BN            & True              & 0.0           & 0.0005   & 32               & 100          & 0           & 20                \\
      \midrule
      \textbf{CIFAR-10}               & 6              & 1024                      & LN            & False             & 0.25          & 0.01     & 50               & 200          & 0.01        & 20                \\
      \midrule
      \textbf{Speech Commands (MFCC)} & 4              & 256                       & LN            & False             & 0.2           & 0.01     & 100              & 50           & 0           & 5                 \\
      \textbf{Speech Commands (Raw)}  & 6              & 128                       & BN            & True              & 0.1           & 0.01     & 20               & 150          & 0           & 10                \\
      \bottomrule
    \end{tabular}%
  }
\end{table*}


\paragraph{Speech Commands}
We provide details of sweeps run for baseline methods run by us---numbers for all others method are taken from \citet{gu2021lssl}. The best hyperparameters used for \methodabbrv{} are included in Table~\ref{tab::best-hyperparameters}.

\textit{Transformer~\citep{vaswani2017attention}} For MFCC, we swept the number of model layers $\{2, 4\}$, dropout $\{0, 0.1\}$ and learning rates $\{0.001, 0.0005\}$. We used $8$ attention heads, model dimension $128$, prenorm, positional encodings, and trained for $150$ epochs with a batch size of $100$. For Raw, the Transformer model's memory usage made training impossible.

\textit{Performer~\citep{choromanski2020rethinking}} For MFCC, we swept the number of model layers $\{2, 4\}$, dropout $\{0, 0.1\}$ and learning rates $\{0.001, 0.0005\}$. We used $8$ attention heads, model dimension $128$, prenorm, positional encodings, and trained for $150$ epochs with a batch size of $100$. For Raw, we used a model dimension of $128$, $4$ attention heads, prenorm, and a batch size of $16$. We reduced the number of model layers to $4$, so the model would fit on the single GPU. We trained for $100$ epochs with a learning rate of $0.001$ and no dropout.

\textit{ExpRNN~\citep{lezcano2019cheap}} For MFCC, we swept hidden sizes $\{256, 512\}$ and learning rates $\{0.001, 0.002, 0.0005\}$. Training was run for $200$ epochs, with a single layer model using a batch size of $100$. For Raw, we swept hidden sizes $\{32, 64\}$ and learning rates $\{0.001, 0.0005\}$ (however, ExpRNN failed to learn).

\textit{LipschitzRNN~\citep{erichson2021lipschitz}} For MFCC, we swept hidden sizes $\{256, 512\}$ and learning rates $\{0.001, 0.002, 0.0005\}$. Training was run for $150$ epochs, with a single layer model using a batch size of $100$. For Raw, we found that LipschitzRNN was too slow to train on a single GPU (requiring a full day for $1$ epoch of training alone).

\textit{WaveGAN Discriminator~\citep{Donahue2019AdversarialAS}}
The WaveGAN-D in \cref{tab:sc} is actually our improved version of the discriminator network from the recent WaveGAN model for speech~\citep{Donahue2019AdversarialAS}.
This CNN actually did not work well out-of-the-box, and we added several features to help it perform better.
The final model is highly specialized compared to our model, and includes:
\begin{itemize}%
  \item Downsampling or pooling between layers, induced by strided convolutions, that decrease the sequence length between layers.
  \item A global fully-connected output layer; thus the model only works for one input sequence length and does not work on MFCC features or the frequency-shift setting in \cref{tab:sc}.
  \item Batch Normalization is essential, whereas \methodabbrv{} works equally well with either Batch Normalization or Layer Normalization.
  \item Almost \( 90\times \) as many parameters as the \methodabbrv{} model ($26.3$M vs. $0.3$M).
\end{itemize}

\subsection{General Sequence Modeling}
\label{sec:experiment-details-general}

This subsection corresponds to the experiments in \cref{sec:experiments-general}.
Because of the number of experiments in this section,
we use subsubsection dividers for different tasks to make it easier to follow:
CIFAR-10 density estimation (\cref{sec:experiment-details-general-cifargen}),
WikiText-103 language modeling (\cref{sec:experiment-details-general-wt103}),
autoregressive generation (\cref{sec:experiment-details-general-speed}),
sequential image classification (\cref{sec:experiment-details-general-image}),
and time-series forecasting (\cref{sec:experiment-details-general-informer}).

\subsubsection{CIFAR Density Estimation}
\label{sec:experiment-details-general-cifargen}

This task used a different backbone than the rest of our experiments.
We used blocks of alternating \methodabbrv{} layers and position-wise feed-forward layers (in the style of Transformer blocks).
Each feed-forward intermediate dimension was set to \( 2\times \) the hidden size of the incoming \methodabbrv{} layer.
Similar to \citet{salimans2017pixelcnn++}, we used a UNet-style backbone consisting of \( B \) identical blocks followed by a downsampling layer.
The downsampling rates were \( 3, 4, 4 \) (the 3 chosen because the sequence consists of RGB pixels).
The base model had \( B=8 \) with starting hidden dimension 128,
while the large model had \( B=16 \) with starting hidden dimension 192.

We experimented with both the mixture of logistics from \citep{salimans2017pixelcnn++} as well as a simpler 256-way categorical loss.
We found they were pretty close and ended up using the simpler softmax loss along with using input embeddings.

We used the LAMB optimizer with learning rate 0.005.
The base model had no dropout, while the large model had dropout 0.1 before the linear layers inside the \methodabbrv{} and FF blocks.


\subsubsection{WikiText-103 Language Modeling}
\label{sec:experiment-details-general-wt103}

The RNN baselines included in \cref{tab:wt103} are the
AWD-QRNN~\citep{merity2018scalable}, an efficient linear gated RNN,
and the LSTM + Cache + Hebbian + MbPA \citep{rae2018fast}, the best performing pure RNN in the literature.
The CNN baselines are
the CNN with GLU activations~\citep{dauphin2017language},
the TrellisNet~\citep{trellisnet},
Dynamic Convolutions~\citep{wu2019pay},
and TaLK Convolutions~\citep{lioutas2020time}.

The Transformer baseline is \citep{baevski2018adaptive},
which uses Adaptive Inputs with a tied Adaptive Softmax.
This model is a standard high-performing Transformer baseline on this benchmark,
used for example by \citet{lioutas2020time} and many more.

Our \methodabbrv{} model uses the same Transformer backbone as in \citep{baevski2018adaptive}.
The model consists of 16 blocks of \methodabbrv{} layers alternated with position-wise feedforward layers, with a feature dimension of 1024.
Because our \methodabbrv{} layer has around 1/4 the number of parameters as a self-attention layer with the same dimension, we made two modifications to match the parameter count better:
(i) we used a GLU activation after the \methodabbrv{} linear layer (\cref{sec:s4-architecture})
(ii) we used two \methodabbrv{} layers per block.
Blocks use Layer Normalization in the pre-norm position.
The embedding and softmax layers were the Adaptive Embedding from \citep{baevski2018adaptive} with standard cutoffs 20000, 40000, 200000.

Evaluation was performed similarly to the basic setting in \citep{baevski2018adaptive}, Table 5,
which uses sliding non-overlapping windows.
Other settings are reported in \citep{baevski2018adaptive} that include more context at training and evaluation time and improves the score.
Because such evaluation protocols are orthogonal to the basic model, we do not consider them and report the base score from \citep{baevski2018adaptive} Table 5.

Instead of SGD+Momentum with multiple cosine learning rate annealing cycles,
our \methodabbrv{} model was trained with the simpler AdamW optimizer with a single cosine learning rate cycle with a maximum of 800000 steps.
The initial learning rate was set to 0.0005.
We used 8 A100 GPUs with a batch size of 1 per gpu and context size 8192.
We used no gradient clipping and a weight decay of 0.1.
Unlike \citep{baevski2018adaptive} which specified different dropout rates for different parameters,
we used a constant dropout rate of 0.25 throughout the network, including before every linear layer and on the residual branches.


\subsubsection{Autoregressive Generation Speed}
\label{sec:experiment-details-general-speed}

\paragraph{Protocol.}
To account for different model sizes and memory requirements for each method,
we benchmark generation speed by throughput,
measured in images per second (\cref{tab:cifar-generation}) or tokens per second (\cref{tab:wt103}).
Each model generates images on a single \( A100 \) GPU,
maximizing batch size to fit in memory.
(For CIFAR-10 generation we limited memory to 16Gb, to be more comparable to the Transformer and Linear Transformer results reported from \citep{katharopoulos2020transformers}.)

\paragraph{Baselines.}
The Transformer and Linear Transformer baselines reported in \cref{tab:cifar-generation} are the results reported directly from \citet{katharopoulos2020transformers}.
Note that the Transformer number is the one in their Appendix, which implements the optimized cached implementation of self-attention.

For all other baseline models, we used open source implementations of the models to benchmark generation speed.
For the PixelCNN++, we used the fast cached version by \citet{ramachandran2017fast},
which sped up generation by orders of magnitude from the naive implementation.
This code was only available in TensorFlow, which may have slight differences compared to the rest of the baselines which were implemented in PyTorch.

We were unable to run the Sparse Transformer~\citep{child2019generating} model due to issues with their custom CUDA implementation of the sparse attention kernel, which we were unable to resolve.

The Transformer baseline from \cref{tab:wt103} was run using a modified GPT-2 backbone from the HuggingFace repository, configured to recreate the architecture reported in \citep{baevski2018adaptive}.
These numbers are actually slightly favorable to the baseline, as we did not include the timing of the embedding or softmax layers, whereas the number reported for \methodabbrv{} is the full model.

\subsubsection{Pixel-Level Sequential Image Classification}
\label{sec:experiment-details-general-image}

Our models were trained with the AdamW optimizer for up to 200 epochs.
Hyperparameters for the CIFAR-10 model is reported in \cref{tab::best-hyperparameters}.

For our comparisons against ResNet-18, the main differences between the base models are that \methodabbrv{} uses LayerNorm by default while ResNet uses BatchNorm.
The last ablation in \cref{sec:experiments-general} swaps the normalization type,
using BatchNorm for \methodabbrv{} and LayerNorm for ResNet,
to ablate this architectural difference.
The experiments with augmentation take the base model and train with mild data augmentation: horizontal flips and random crops (with symmetric padding).

\begin{table}[t]
  \small
  \centering
  \captionsetup{type=table}
  \caption{
    (\textbf{Pixel-level image classification.})
    Citations refer to the original model; additional citation indicates work from which this baseline is reported.
  }
  \begin{tabular}{@{}llll@{}}
    \toprule
    Model                                                      & \textsc{sMNIST}   & \textsc{pMNIST}   & \textsc{sCIFAR}   \\
    \midrule
    Transformer~\citep{vaswani2017attention,trinh2018learning} & 98.9              & 97.9              & 62.2              \\
    \midrule
    CKConv~\citep{romero2021ckconv}                            & 99.32             & \underline{98.54} & 63.74             \\
    TrellisNet~\citep{trellisnet}                              & 99.20             & 98.13             & 73.42             \\
    TCN~\citep{bai2018empirical}                               & 99.0              & 97.2              & -                 \\
    \midrule
    LSTM~\citep{lstm,gu2020improving}                          & 98.9              & 95.11             & 63.01             \\
    r-LSTM ~\citep{trinh2018learning}                          & 98.4              & 95.2              & 72.2              \\
    Dilated GRU~\citep{chang2017dilated}                       & 99.0              & 94.6              & -                 \\
    Dilated RNN~\citep{chang2017dilated}                       & 98.0              & 96.1              & -                 \\
    IndRNN~\citep{indrnn}                                      & 99.0              & 96.0              & -                 \\
    expRNN~\citep{lezcano2019cheap}                            & 98.7              & 96.6              & -                 \\
    UR-LSTM                                                    & 99.28             & 96.96             & 71.00             \\
    UR-GRU~\citep{gu2020improving}                             & 99.27             & 96.51             & \underline{74.4}  \\
    LMU~\citep{voelker2019legendre}                            & -                 & 97.15             & -                 \\
    HiPPO-RNN~\citep{gu2020hippo}                              & 98.9              & 98.3              & 61.1              \\
    UNIcoRNN~\citep{rusch2021unicornn}                         & -                 & 98.4              & -                 \\
    LMUFFT~\citep{chilkuri2021parallelizing}                   & -                 & 98.49             & -                 \\
    LipschitzRNN~\citep{erichson2021lipschitz}                 & \underline{99.4}  & 96.3              & 64.2              \\
    \midrule
    \textbf{\methodabbrv}                                                & \textbf{99.63}    & \textbf{98.70} & \textbf{91.13}    \\
    \bottomrule
  \end{tabular}
  \label{tab:image-full}
\end{table}


\subsubsection{Time Series Forecasting compared to Informer}
\label{sec:experiment-details-general-informer}

We include a simple figure (\cref{fig:s4-architecture}) contrasting the architecture of \methodabbrv{} against that of the Informer \citep{haoyietal-informer-2021}.

In \cref{fig:s4-architecture},
the goal is to forecast a contiguous range of future predictions (Green, length \( F \) )
given a range of past context (Blue, length \( C \) ).
We simply concatenate the entire context with a sequence of masks set to the length of the forecast window.
This input is a single sequence of length \( C+F \) that is run through the same simple deep \methodabbrv{} model used throughout this work,
which maps to an output of length \( C+F \) .
We then use just the last \( F \) outputs as the forecasted predictions.


\begin{figure}[t]
    \centering
    \begin{subfigure}{\linewidth}%
      \centering
      \includegraphics[width=\linewidth]{figs/s4_forecasting.pdf}
    \end{subfigure}
    \caption{Comparison of \methodabbrv{} and specialized time-series models for forecasting tasks. (\textit{Top Left}) The forecasting task involves predicting future values of a time-series given past context. (\textit{Bottom Left}) We perform simple forecasting using a sequence model such as \methodabbrv{} as a black box. (\textit{Right}) Informer uses an encoder-decoder architecture designed specifically for forecasting problems involving a customized attention module (figure taken from~\citet{haoyietal-informer-2021}).}
    \label{fig:s4-architecture}
\end{figure}

\cref{tab:informer-s,tab:informer-m} contain full results on all 50 settings considered by \citet{haoyietal-informer-2021}.
\methodabbrv{} sets the best results on 40 out of 50 of these settings.

\begin{table*}[t]
\centering
\fontsize{9pt}{9pt}\selectfont
\centering
\resizebox{\linewidth}{!}{
\begin{tabular}{c|c|c|c|c|c|c|c|c|c|c|c}
\toprule[1.0pt]
\multicolumn{2}{c|}{Methods}              & \textbf{\methodabbrv} & {Informer}                     & {Informer$^{\dag}$}            & {LogTrans}       & {Reformer}   & {LSTMa}      & {DeepAR}     & {ARIMA}                 & {Prophet}    \\
\midrule[0.5pt]
\multicolumn{2}{c|}{Metric}               & MSE~~MAE              & MSE~~MAE                       & MSE~~MAE                       & MSE~~MAE         & MSE~~MAE     & MSE~~MAE     & MSE~~MAE     & MSE~~MAE                & MSE~~MAE     \\
\midrule[1.0pt]
\multirow{5}{*}{\rotatebox{90}{ETTh$_1$}} & 24                    & \textbf{0.061}~~\textbf{0.191} & 0.098~~0.247                   & {0.092}~~{0.246} & 0.103~~0.259 & 0.222~~0.389 & 0.114~~0.272 & 0.107~~0.280            & 0.108~~0.284  & 0.115~~0.275 \\
                                          & 48                    & \textbf{0.079}~~\textbf{0.220} & {0.158}~~{0.319}               & 0.161~~0.322     & 0.167~~0.328 & 0.284~~0.445 & 0.193~~0.358 & 0.162~~0.327            & 0.175~~0.424  & 0.168~~0.330 \\
                                          & 168                   & \textbf{0.104}~~\textbf{0.258} & {0.183}~~{0.346}               & 0.187~~0.355     & 0.207~~0.375 & 1.522~~1.191 & 0.236~~0.392 & 0.239~~0.422            & 0.396~~0.504  & 1.224~~0.763 \\
                                          & 336                   & \textbf{0.080}~~\textbf{0.229} & 0.222~~0.387                   & {0.215}~~{0.369} & 0.230~~0.398 & 1.860~~1.124 & 0.590~~0.698 & 0.445~~0.552            & 0.468~~0.593  & 1.549~~1.820 \\
                                          & 720                   & \textbf{0.116}~~\textbf{0.271} & 0.269~~0.435                   & {0.257}~~{0.421} & 0.273~~0.463 & 2.112~~1.436 & 0.683~~0.768 & 0.658~~0.707            & 0.659~~0.766  & 2.735~~3.253 \\
\midrule[0.5pt]
\multirow{5}{*}{\rotatebox{90}{ETTh$_2$}} & 24                    & 0.095~~0.234                   & \textbf{0.093}~~\textbf{0.240} & 0.099~~0.241     & 0.102~~0.255 & 0.263~~0.437 & 0.155~~0.307 & 0.098~~0.263            & 3.554~~0.445  & 0.199~~0.381 \\
                                          & 48                    & 0.191~~0.346                   & \textbf{0.155}~~\textbf{0.314} & 0.159~~0.317     & 0.169~~0.348 & 0.458~~0.545 & 0.190~~0.348 & 0.163~~0.341            & 3.190~~0.474  & 0.304~~0.462 \\
                                          & 168                   & \textbf{0.167}~~\textbf{0.333} & {0.232}~~{0.389}               & 0.235~~0.390     & 0.246~~0.422 & 1.029~~0.879 & 0.385~~0.514 & 0.255~~0.414            & 2.800~~0.595  & 2.145~~1.068 \\
                                          & 336                   & \textbf{0.189}~~\textbf{0.361} & 0.263~~{0.417}                 & {0.258}~~0.423   & 0.267~~0.437 & 1.668~~1.228 & 0.558~~0.606 & 0.604~~0.607            & 2.753~~0.738  & 2.096~~2.543 \\
                                          & 720                   & \textbf{0.187}~~\textbf{0.358} & {0.277}~~{ 0.431}              & 0.285~~0.442     & 0.303~~0.493 & 2.030~~1.721 & 0.640~~0.681 & 0.429~~0.580            & 2.878~~1.044  & 3.355~~4.664 \\
\midrule[0.5pt]
\multirow{5}{*}{\rotatebox{90}{ETTm$_1$}} & 24                    & \textbf{0.024}~~\textbf{0.117} & {0.030}~~{0.137}               & 0.034~~0.160     & 0.065~~0.202 & 0.095~~0.228 & 0.121~~0.233 & 0.091~~0.243            & 0.090~~0.206  & 0.120~~0.290 \\
                                          & 48                    & \textbf{0.051}~~\textbf{0.174} & 0.069~~0.203                   & {0.066}~~{0.194} & 0.078~~0.220 & 0.249~~0.390 & 0.305~~0.411 & 0.219~~0.362            & 0.179~~0.306  & 0.133~~0.305 \\
                                          & 96                    & \textbf{0.086}~~\textbf{0.229} & 0.194~~{0.372}                 & {0.187}~~0.384   & 0.199~~0.386 & 0.920~~0.767 & 0.287~~0.420 & 0.364~~0.496            & 0.272~~0.399  & 0.194~~0.396 \\
                                          & 288                   & \textbf{0.160}~~\textbf{0.327} & {0.401}~~0.554                 & 0.409~~{0.548}   & 0.411~~0.572 & 1.108~~1.245 & 0.524~~0.584 & 0.948~~0.795            & 0.462~~0.558  & 0.452~~0.574 \\
                                          & 672                   & \textbf{0.292}~~\textbf{0.466} & {0.512}~~{0.644}               & 0.519~~0.665     & 0.598~~0.702 & 1.793~~1.528 & 1.064~~0.873 & 2.437~~1.352            & 0.639~~0.697  & 2.747~~1.174 \\
\midrule[0.5pt]
\multirow{5}{*}{\rotatebox{90}{Weather}}  & 24                    & 0.125~~0.254                   & \textbf{0.117}~~\textbf{0.251} & 0.119~~0.256     & 0.136~~0.279 & 0.231~~0.401 & 0.131~~0.254 & 0.128~~0.274            & 0.219~~0.355  & 0.302~~0.433 \\
                                          & 48                    & 0.181~~\textbf{0.305}          & \textbf{0.178}~~0.318          & 0.185~~0.316     & 0.206~~0.356 & 0.328~~0.423 & 0.190~~0.334 & 0.203~~0.353            & 0.273~~0.409  & 0.445~~0.536 \\
                                          & 168                   & \textbf{0.198}~~\textbf{0.333} & {0.266}~~{0.398}               & 0.269~~0.404     & 0.309~~0.439 & 0.654~~0.634 & 0.341~~0.448 & 0.293~~0.451            & 0.503~~0.599  & 2.441~~1.142 \\
                                          & 336                   & 0.300~~0.417                   & \textbf{0.297}~~\textbf{0.416} & 0.310~~0.422     & 0.359~~0.484 & 1.792~~1.093 & 0.456~~0.554 & 0.585~~0.644            & 0.728~~0.730  & 1.987~~2.468 \\
                                          & 720                   & \textbf{0.245}~~\textbf{0.375} & {0.359}~~{0.466}               & 0.361~~0.471     & 0.388~~0.499 & 2.087~~1.534 & 0.866~~0.809 & 0.499~~0.596            & 1.062~~0.943  & 3.859~~1.144 \\
\midrule[0.5pt]
\multirow{5}{*}{\rotatebox{90}{ECL}}      & 48                    & 0.222~~\textbf{0.350}          & 0.239~~0.359                   & 0.238~~0.368     & 0.280~~0.429 & 0.971~~0.884 & 0.493~~0.539 & \textbf{0.204}~~{0.357} & 0.879~~0.764  & 0.524~~0.595 \\
                                          & 168                   & 0.331~~\textbf{0.421}          & 0.447~~0.503                   & 0.442~~0.514     & 0.454~~0.529 & 1.671~~1.587 & 0.723~~0.655 & \textbf{0.315}~~{0.436} & 1.032~~0.833  & 2.725~~1.273 \\
                                          & 336                   & \textbf{0.328}~~\textbf{0.422} & 0.489~~0.528                   & 0.501~~0.552     & 0.514~~0.563 & 3.528~~2.196 & 1.212~~0.898 & {0.414}~~{0.519}        & 1.136~~0.876  & 2.246~~3.077 \\
                                          & 720                   & \textbf{0.428}~~\textbf{0.494} & {0.540}~~{0.571}               & 0.543~~0.578     & 0.558~~0.609 & 4.891~~4.047 & 1.511~~0.966 & 0.563~~0.595            & 1.251~~0.933  & 4.243~~1.415 \\
                                          & 960                   & \textbf{0.432}~~\textbf{0.497} & {0.582}~~{0.608}               & 0.594~~0.638     & 0.624~~0.645 & 7.019~~5.105 & 1.545~~1.006 & 0.657~~0.683            & 1.370~~0.982  & 6.901~~4.264 \\

\midrule[1.0pt]
\multicolumn{2}{c|}{Count}                & {22}                  & {5}                            & {0}                            & {0}              & {0}          & {0}          & {2}          & {0}                     & {0}          \\
\bottomrule[1.0pt]

\end{tabular}%
}
\caption{Univariate long sequence time-series forecasting results on four datasets (five cases).}
\label{tab:informer-s}
\end{table*}


\begin{table*}[t]
\centering
\fontsize{9pt}{9pt}\selectfont
\resizebox{\linewidth}{!}{
\begin{tabular}{c|c|cc|cc|cc|cc|cc|cc|cc}
\toprule[1.0pt]
\multicolumn{2}{c}{Methods}               & \multicolumn{2}{|c}{\textbf{\methodabbrv}} & \multicolumn{2}{|c}{Informer} & \multicolumn{2}{|c}{Informer$^{\dag}$} & \multicolumn{2}{|c}{LogTrans} & \multicolumn{2}{|c}{Reformer} & \multicolumn{2}{|c}{LSTMa} & \multicolumn{2}{|c}{LSTnet} \\
\midrule[0.5pt]
\multicolumn{2}{c|}{Metric}               & MSE                                        & MAE                           & MSE                                    & MAE                           & MSE                           & MAE                        & MSE                          & MAE     & MSE   & MAE   & MSE   & MAE   & MSE   & MAE     \\
\midrule[1.0pt]
\multirow{5}{*}{\rotatebox{90}{ETTh$_1$}} & 24                                         & \textbf{0.525}                & \textbf{0.542}                         & {0.577}                       & {0.549}                       & 0.620                      & 0.577                        & 0.686   & 0.604 & 0.991 & 0.754 & 0.650 & 0.624 & 1.293    & 0.901 \\
                                          & 48                                         & \textbf{0.641}                & \textbf{0.615}                         & {0.685}                       & {0.625}                       & 0.692                      & 0.671                        & 0.766   & 0.757 & 1.313 & 0.906 & 0.702 & 0.675 & 1.456    & 0.960 \\
                                          & 168                                        & 0.980                         & 0.779                                  & \textbf{0.931}                & \textbf{0.752}                & 0.947                      & 0.797                        & 1.002   & 0.846 & 1.824 & 1.138 & 1.212 & 0.867 & 1.997    & 1.214 \\
                                          & 336                                        & 1.407                         & 0.910                                  & 1.128                         & 0.873                         & \textbf{1.094}             & \textbf{0.813}               & 1.362   & 0.952 & 2.117 & 1.280 & 1.424 & 0.994 & 2.655    & 1.369 \\
                                          & 720                                        & \textbf{1.162}                & \textbf{0.842}                         & {1.215}                       & {0.896}                       & 1.241                      & 0.917                        & 1.397   & 1.291 & 2.415 & 1.520 & 1.960 & 1.322 & 2.143    & 1.380 \\
\midrule[0.5pt]
\multirow{5}{*}{\rotatebox{90}{ETTh$_2$}} & 24                                         & 0.871                         & 0.736                                  & \textbf{0.720}                & \textbf{0.665}                & 0.753                      & 0.727                        & 0.828   & 0.750 & 1.531 & 1.613 & 1.143 & 0.813 & 2.742    & 1.457 \\
                                          & 48                                         & \textbf{1.240}                & \textbf{0.867}                         & {1.457}                       & {1.001}                       & 1.461                      & 1.077                        & 1.806   & 1.034 & 1.871 & 1.735 & 1.671 & 1.221 & 3.567    & 1.687 \\
                                          & 168                                        & \textbf{2.580}                & \textbf{1.255}                         & 3.489                         & {1.515}                       & 3.485                      & 1.612                        & 4.070   & 1.681 & 4.660 & 1.846 & 4.117 & 1.674 & {3.242}  & 2.513 \\
                                          & 336                                        & \textbf{1.980}                & \textbf{1.128}                         & 2.723                         & 1.340                         & 2.626                      & {1.285}                      & 3.875   & 1.763 & 4.028 & 1.688 & 3.434 & 1.549 & {2.544}  & 2.591 \\
                                          & 720                                        & \textbf{2.650}                & \textbf{1.340}                         & {3.467}                       & {1.473}                       & 3.548                      & 1.495                        & 3.913   & 1.552 & 5.381 & 2.015 & 3.963 & 1.788 & 4.625    & 3.709 \\
\midrule[0.5pt]
\multirow{5}{*}{\rotatebox{90}{ETTm$_1$}} & 24                                         & 0.426                         & 0.487                                  & 0.323                         & \textbf{0.369}                & \textbf{0.306}             & 0.371                        & 0.419   & 0.412 & 0.724 & 0.607 & 0.621 & 0.629 & 1.968    & 1.170 \\
                                          & 48                                         & 0.580                         & 0.565                                  & 0.494                         & 0.503                         & \textbf{0.465}             & \textbf{0.470}               & 0.507   & 0.583 & 1.098 & 0.777 & 1.392 & 0.939 & 1.999    & 1.215 \\
                                          & 96                                         & 0.699                         & 0.649                                  & \textbf{0.678}                & 0.614                         & 0.681                      & \textbf{0.612}               & 0.768   & 0.792 & 1.433 & 0.945 & 1.339 & 0.913 & 2.762    & 1.542 \\
                                          & 288                                        & \textbf{0.824}                & \textbf{0.674}                         & {1.056}                       & {0.786}                       & 1.162                      & 0.879                        & 1.462   & 1.320 & 1.820 & 1.094 & 1.740 & 1.124 & 1.257    & 2.076 \\
                                          & 672                                        & \textbf{0.846}                & \textbf{0.709}                         & {1.192}                       & {0.926}                       & 1.231                      & 1.103                        & 1.669   & 1.461 & 2.187 & 1.232 & 2.736 & 1.555 & 1.917    & 2.941 \\
\midrule[0.5pt]
\multirow{5}{*}{\rotatebox{90}{Weather}}  & 24                                         & \textbf{0.334}                & 0.385                                  & {0.335}                       & \textbf{0.381}                & 0.349                      & 0.397                        & 0.435   & 0.477 & 0.655 & 0.583 & 0.546 & 0.570 & 0.615    & 0.545 \\
                                          & 48                                         & 0.406                         & 0.444                                  & 0.395                         & 0.459                         & \textbf{0.386}             & \textbf{0.433}               & 0.426   & 0.495 & 0.729 & 0.666 & 0.829 & 0.677 & 0.660    & 0.589 \\
                                          & 168                                        & \textbf{0.525}                & \textbf{0.527}                         & {0.608}                       & {0.567}                       & 0.613                      & 0.582                        & 0.727   & 0.671 & 1.318 & 0.855 & 1.038 & 0.835 & 0.748    & 0.647 \\
                                          & 336                                        & \textbf{0.531}                & \textbf{0.539}                         & {0.702}                       & {0.620}                       & 0.707                      & 0.634                        & 0.754   & 0.670 & 1.930 & 1.167 & 1.657 & 1.059 & 0.782    & 0.683 \\
                                          & 720                                        & \textbf{0.578}                & \textbf{0.578}                         & {0.831}                       & {0.731}                       & 0.834                      & 0.741                        & 0.885   & 0.773 & 2.726 & 1.575 & 1.536 & 1.109 & 0.851    & 0.757 \\
\midrule[0.5pt]
\multirow{5}{*}{\rotatebox{90}{ECL}}      & 48                                         & \textbf{0.255}                & \textbf{0.352}                         & 0.344                         & {0.393}                       & 0.334                      & 0.399                        & 0.355   & 0.418 & 1.404 & 0.999 & 0.486 & 0.572 & 0.369    & 0.445 \\
                                          & 168                                        & \textbf{0.283}                & \textbf{0.373}                         & 0.368                         & 0.424                         & {0.353}                    & {0.420}                      & 0.368   & 0.432 & 1.515 & 1.069 & 0.574 & 0.602 & 0.394    & 0.476 \\
                                          & 336                                        & \textbf{0.292}                & \textbf{0.382}                         & 0.381                         & {0.431}                       & 0.381                      & 0.439                        & {0.373} & 0.439 & 1.601 & 1.104 & 0.886 & 0.795 & 0.419    & 0.477 \\
                                          & 720                                        & \textbf{0.289}                & \textbf{0.377}                         & 0.406                         & 0.443                         & {0.391}                    & {0.438}                      & 0.409   & 0.454 & 2.009 & 1.170 & 1.676 & 1.095 & 0.556    & 0.565 \\
                                          & 960                                        & \textbf{0.299}                & \textbf{0.387}                         & {0.460}                       & {0.548}                       & 0.492                      & 0.550                        & 0.477   & 0.589 & 2.141 & 1.387 & 1.591 & 1.128 & 0.605    & 0.599 \\
\midrule[1.0pt]
\multicolumn{2}{c}{Count}                 & \multicolumn{2}{|c}{18}                    & \multicolumn{2}{|c}{5}        & \multicolumn{2}{|c}{6}                 & \multicolumn{2}{|c}{0}        & \multicolumn{2}{|c}{0}        & \multicolumn{2}{|c}{0}     & \multicolumn{2}{|c}{0}      \\
\bottomrule[1.0pt]

\end{tabular}%
}
\caption{Multivariate long sequence time-series forecasting results on four datasets (five cases).}
\label{tab:informer-m}
\end{table*}


\subsection{Visualizations}
We visualize the convolutional filter $\bar{K}$ learned by \methodabbrv{} for the Pathfinder and CIFAR-10 tasks in \cref{fig:pathfinder-all-conv-filters}.

\begin{figure}
    \centering
    \begin{subfigure}{\linewidth}
        \includegraphics[width=\linewidth]{figs/pathfinder_filters_layer_0_trunc.png}
    \end{subfigure}
    \begin{subfigure}{\linewidth}
        \includegraphics[width=\linewidth]{figs/pathfinder_filters_layer_5_trunc.png}
    \end{subfigure}
    \label{fig:pathfinder-all-conv-filters}
    \caption{({\bf Convolutional filters on Pathfinder}) A random selection of filters learned by \methodabbrv{} in the first layer (top 2 rows) and last layer (bottom 2 rows) of the best model.}
\end{figure}

\subsection{Reproduction}
\label{sec:reproduction}

Since the first version of this paper, several experiments have been updated. Please read the corresponding paragraph below before citing LRA or SC results.

\paragraph{Long Range Arena}

Follow-ups to this paper expanded the theoretical understanding of S4 while improving some results.
The results reported in \cref{tab:lra} have been updated to results from the papers \citep{gu2022s4d,gu2022hippo}.
More specifically, the method S4-LegS in those works refers to the \emph{same model} presented in this paper, with the ``-LegS'' suffix referring to the initialization defined in equation \eqref{eq:hippo}. As such, results from the original \cref{tab:lra} have been directly updated.

The updated results have only minor hyperparameter changes compared to the original results. The original results and hyperparameters are shown in \cref{tab:lra-full} (\cref{sec:experiment-details-lrd}).
Appendix B of \citep{gu2022s4d} describes the changes in hyperparameters, which are also documented from the experiment configuration files in the publically available code at \url{https://github.com/HazyResearch/state-spaces}.

\paragraph{Speech Commands}

The Speech Commands (SC) dataset~\citep{Warden2018SpeechCA} is originally a 35-class dataset of spoken English words.
However, this paper was part of a line of work starting with \citet{kidger2020neural} that has used a smaller 10-class subset of SC \citep{kidger2020neural,romero2021ckconv,gu2021lssl,romero2022flexconv}.
\emph{In an effort to avoid dataset fragmentation in the literature, we have since moved to the original dataset.}
We are now calling this 10-class subset \textbf{SC10} to distinguish it from the full 35-class \textbf{SC} dataset.
To cite S4 as a baseline for Speech Commands, please use Table 11 from \citep{gu2022s4d} instead of \cref{tab:sc} from this paper.
In addition to using the full SC dataset, it also provides a number of much stronger baselines than the ones used in this work.


\paragraph{WikiText-103}

The original version of this paper used an S4 model with batch size \( 8 \), context size \( 1024 \) which achieved a validation perplexity of 20.88 and test perplexity of 21.28.
It was later retrained with a batch size of \( 1 \) and context size \( 8192 \) which achieved a validation perplexity of 19.69 and test perplexity of 20.95, and a model checkpoint is available in the public repository.
The rest of the model is essentially identical, so the results from the original table have been updated.


\end{document}
