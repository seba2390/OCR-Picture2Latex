\documentclass{article} %


\usepackage{hyperref}
\usepackage{url}
\usepackage{booktabs}       %
\usepackage{amsfonts}       %
\usepackage{nicefrac}       %
\usepackage{microtype}      %

\usepackage{amsmath,amsfonts,amsthm}       %
\usepackage{subcaption}
\usepackage{bm}
\usepackage{bbm}
\usepackage{multirow}
\usepackage[inline]{enumitem}
\usepackage{diagbox}
\usepackage[capitalise]{cleveref}  %
\usepackage{comment}
\usepackage{etoolbox}
\usepackage{graphicx}
\usepackage{wrapfig}
\usepackage[font=small]{caption}
\usepackage{algorithm,algorithmicx,algpseudocode}
\usepackage{subcaption}
\usepackage{tablefootnote}


\usepackage{pifont}%
\newcommand{\cmark}{\ding{51}}%
\newcommand{\xmark}{\ding{55}}%


\newtheorem{theorem}{Theorem}
\newtheorem{lemma}{Lemma}[section]
\newtheorem{corollary}[lemma]{Corollary}
\newtheorem{observation}[lemma]{Observation}
\newtheorem{proposition}[theorem]{Proposition}
\newtheorem{definition}{Definition}
\newtheorem{remark}[lemma]{Remark}
\newtheorem{claim}{Claim}

\newcommand{\dt}{\Delta}
\newcommand{\dd}{\mathop{}\!d}
\DeclareMathOperator{\hippo}{\mathsf{hippo}}
\DeclareMathOperator*{\diag}{diag}

\newcommand{\methodabbrv}{S4}

\usepackage[square,numbers,sort]{natbib}
\setlength{\textwidth}{6.5in}
\setlength{\textheight}{9in}
\setlength{\oddsidemargin}{0in}
\setlength{\evensidemargin}{0in}
\setlength{\topmargin}{-0.5in}
\newlength{\defbaselineskip}
\setlength{\defbaselineskip}{\baselineskip}
\setlength{\marginparwidth}{0.8in}
\setlength{\parskip}{5pt}%
\setlength{\parindent}{0pt}%
\usepackage[dvipsnames]{xcolor}         %


\title{Efficiently Modeling Long Sequences with Structured State Spaces}
\usepackage{authblk}
\author[]{Albert Gu}
\author[]{Karan Goel}
\author[]{Christopher R{\'e}}
\affil[]{Department of Computer Science, Stanford University}
\affil[]{{\texttt{\{albertgu,krng\}@stanford.edu}, \texttt{chrismre@cs.stanford.edu}}}
\date{}

\bibliographystyle{plainnat}


\begin{document}


\maketitle

\begin{abstract}
  A central goal of sequence modeling is designing a single principled model that can address sequence data across a range of modalities and tasks, particularly on long-range dependencies.
  Although conventional models including RNNs, CNNs, and Transformers have specialized variants for capturing long dependencies, they still struggle to scale to very long sequences of $10000$ or more steps.
  A promising recent approach proposed modeling sequences by simulating the fundamental state space model (SSM) \( x'(t) = Ax(t) + Bu(t), y(t) = Cx(t) + Du(t) \), and showed that for appropriate choices of the state matrix \( A \), this system could handle long-range dependencies mathematically and empirically.
  However, this method has prohibitive computation and memory requirements, rendering it infeasible as a general sequence modeling solution.
  We propose the Structured State Space sequence model (\methodabbrv{}) based on a new parameterization for the SSM, and show that it can be computed much more efficiently than prior approaches while preserving their theoretical strengths.
  Our technique involves conditioning \( A \) with a low-rank correction, allowing it to be diagonalized stably and reducing the SSM to the well-studied computation of a Cauchy kernel.
  \methodabbrv{} achieves strong empirical results across a diverse range of established benchmarks, including (i) 91\% accuracy on sequential CIFAR-10 with no data augmentation or auxiliary losses, on par with a larger 2-D ResNet, (ii) substantially closing the gap to Transformers on image and language modeling tasks, while performing generation $60\times$ faster (iii) SoTA on every task from the Long Range Arena benchmark, including solving the challenging Path-X task of length 16k that all prior work fails on, while being as efficient as all competitors.\footnote{Code is publicly available at \url{https://github.com/HazyResearch/state-spaces}.}
\end{abstract}

Reinforcement learning has achieved great success in areas such as Game-playing \citep{silver2018general,vinyals2019grandmaster}, robotics \cite{kober2013reinforcement}, large language models \citep{ouyang2022training}, etc.
However, due to safety concerns or physical limitations, in some real-world reinforcement learning problems, we must consider additional constraints that may influence the optimal policy and the learning process \citep{garcia2015comprehensive}.
% For example, a robotic arm must not take actions that may cause harm to itself or the environments.
A standard framework to handle such cases is the constrained Markov Decision Process (CMDP) \citep{altman1999constrained}.
Within the CMDP framework, the agent has to maximize
the expected cumulative reward while
obeying a finite number of constraints, which are usually in the form of expected cumulative cost criteria.

However, we are sometimes concerned with the problem with a continuum of constraints.
For example,
the constraints we meet might be time-evolving or subject to uncertain parameters, which
cannot be formulated as an ordinary CMDP
(see Examples \ref{Example_Time_Evolving} and  \ref{Example_Uncertain}).
In this paper we would study a generalized CMDP  
to address the above problem.  Because the constraints are not only infinite-number but also lie
in a continuous set,
the generalization is not trivial. Fortunately, we find that we can borrow the idea behind semi-infinite programming (SIP) \citep{remez1934determination, hettich1993semi} to deal with the semi-infinite constraints.
Accordingly, we propose \emph{semi-infinitely constrained Markov decision processes} (SICMDPs)
as a novel complement to the ordinary CMDP framework.
%More specifically,  an SICMDP model %, we consider 
%contains a continuum of constraints whereas an ordinary CMDP contains a finite number of constraints. 

%This generalization is natural but not trivial. However, we can brows the idea  
%The idea is quite natural and can be backtracked
%to the practice of extending linear programming to linear semi-infinite programming (LSIP) %\cite{remez1934determination, GobernaLSIO1998}.
%In addition, 
%As a complementary approach to the ordinary CMDP framework, 
%SICMDP can be used to model these problems  which cannot be described by a finite number of constraints
%that are not covered by .
%For example,
%the restrictions we consider can be time-evolving or subject to uncertain parameters
%, thus
%cannot be described by a finite number of constraints but a continuum of constraints 
%(see Examples \ref{Example_Time_Evolving} and  \ref{Example_Uncertain}).

We also present two reinforcement learning algorithms to solve SICMDPs called SI-CRL and SI-CPO, respectively.
SI-CRL is a model-based reinforcement learning algorithm designed for tabular cases, and SI-CPO is a policy optimization algorithm for non-tabular cases.
% and analyze its performance both theoretically and empirically.
The main challenge is that we need to deal with a continuum of constraints, thus reinforcement learning algorithms for ordinary CMDPs do not work anymore.
In SI-CRL, we tackle this difficulty by first transforming the reinforcement learning problem to an equivalent LSIP problem, which can then be solved using methods in the LSIP literature like the dual exchange methods \citep{Hu1990,reemtsen1998numerical}.
In SI-CPO, we resort to the idea of cooperative stochastic approximation developed in \cite{lan2020algorithms, wei2020comirror}.
As far as we know, we are the first to introduce tools from semi-infinitely programming (SIP) into the reinforcement learning community for solving constrained reinforcement learning problems.

% To the best of our knowledge, we are the first to apply tools from semi-infinitely programming (SIP) to solve reinforcement learning problems.
Furthermore, we give theoretical analysis for both SI-CRL and SI-CPO.
We decompose the error of SI-CRL into two parts: the statistical error from approximating the true SICMDP with an offline dataset and the optimization error due to the fact that the solution of the LSIP problem obtained by the dual exchange method is inexact.
On the optimization side, we show that the iteration complexity of SI-CRL is $O\left(\left\{\mathrm{diam}(Y)L\sqrt{|\gS|^2|\gA|m}/\left[(1-\gamma)\epsilon\right]\right\}^m\right)$.
On the statistical side, we show that the sample complexity of SI-CRL is $\widetilde O\left(\frac{|S|^2|A|^2}{\epsilon^2(1-\gamma)^3}\right)$ if the offline dataset is generated by a generative model, and $\widetilde O\left(\frac{|S||A|}{\nu_{\min} \epsilon^2(1-\gamma)^3}\right)$ if the dataset is generated by a probability measure $\nu$ as considered in \cite{chen2019information}.
Here $\widetilde O$ means that all logarithm terms are discarded.
For SI-CPO, things become a little more complicated because other than the statistical error and the optimization error, we also need to consider the function approximation error, which comes from imperfect policy parametrizations.
It is shown if the function approximation error can be controlled to $O(\epsilon)$ order, the iteration complexity of SI-CPO is $\widetilde{O}\left(\frac{1}{\epsilon^2(1-\gamma)^6}\right)$ and the sample complexity of SI-CPO is $\widetilde{O}(\frac{1}{\epsilon^4(1-\gamma)^{10}})$.
Here our iteration complexity bound is equivalent to a typical $\widetilde O(1/\sqrt{T})$ global convergence rate.

We perform a set of numerical experiments to illustrate the SICMDP model and validate our proposed algorithms.
Specifically, we examine two numerical examples, namely the discharge of sewage and ship route planning.
Through the discharge of sewage example, we show the advantage of the SICMDP framework over the CMDP baseline obtained by naive discretization in modeling realistic sequential decision-making problems.
Moreover, we demonstrate the effectiveness of the SI-CRL and SI-CPO algorithms in such tabular environments. 
In the ship route planning example, we illustrate the benefits of the SICMDP framework and the ability of the SI-CPO algorithm to address complex continuous control tasks involving continuous state spaces with modern deep reinforcement learning techniques.

% In summary, our contributions are listed as follows.
% First, we present the SICMDP model, which can be viewed as a generalization of the ordinary CMDP model.
% Second, we propose an algorithm to perform reinforcement learning for SICMDPs, which is called SI-CRL, and we believe that we are the first to apply tools from SIP
% to solve reinforcement learning problems.
% Third, we give a theoretical analysis of SI-CRL and identify both its sample complexity and iteration complexity.
% In addition, we perform numerical experiments to illustrate the SICMDP model and validate the SI-CRL algorithm.
% \{This paragraph can be removed!!! \}







% Panoptic segmentation

% 3D segmentation

% Multi-object tracking

% Online 3D panoptic:

% PanopticFusion: (IROS 2019)
% https://arxiv.org/pdf/1903.01177.pdf
%
% - most similar to ours
% - PSPNet + M-RCNN + 2D fusion
% - volumetric mapping, 
% - greedy matching with IoU -> optimal only with 0.5 threshold
% - voxel & class weighting
% - CRF regularisation
%
% - good:
%
% - bad:
%  - CRF post-processing step
%  - greedy data-association
%    - can't be tuned for lower overlap ratios -> has to have high framerate, large changes in viewpoint could break this
%    - IoU: sensitive to 2D labels projecting over object borders (CRF and voxel weighting seem to alleviate this)

% Voxblox++: (Robotics & automation letters 2019)
% https://arxiv.org/pdf/1903.00268.pdf
% https://github.com/ethz-asl/voxblox-plusplus
%
% - M-RCNN + geometric segmentation + fusion 
% - data association of geometric segments with 3D overlap (no. points inside volume), fixed threshold for min number of points
% - instance label is assigned to a segment based on highest overlap
% - only one detected segment per reference label, as in PanopticFusion and Ours
% - TSDF Integration 
%
% good: 
% - because of geometric segmentation objects with no associated semantic class can also be segmented
% bad:
% - two different object segment types -> confusing, overly complicated ?
% - quite inaccurate (fixed below)

% Reconstructing Interactive 3D Scenes by Panoptic Mapping and CAD Model Alignments (ICRA 2021)
% https://arxiv.org/pdf/2103.16095.pdf
% https://github.com/hmz-15/Interactive-Scene-Reconstruction
%
% - based heavily on Voxblox++, much more accurate
% - Scene-graph ("contact graph") for mapping object relations
% - Search & replace voxels with CAD models, with geometrical and physical constraints
% - Object 6D pose
% - Format for robot interaction
%
% - Segmentation: bilateral fusion of geomatric and semantic segments -> reduce segmentation noise compared to Voxblox++
% - Fusion: triplet count improves consistency over Voxblox++ pairwise count strategy (take semantic label into account in addition to instance and geometry)
% - Fusion: instance labels are also combined if there is enough overlap with common geometric label for long enough time
%   - this means multiple detections can match the same reference unlike ours, voxblox++ and PanopticFusion ?
%

% Panoptic-MOPE: (ROBOTICS AND AUTOMATION LETTERS 2020)
% https://ieeexplore.ieee.org/stamp/stamp.jsp?tp=&arnumber=8977356
% https://github.com/hoangcuongbk80/Object-RPE/tree/panoptic-mope
%
% - novel RGB-D semantic segmentation model + M-RCNN
% - camera tracking based on "addaptively weighted optimization of geometric, appearance, and semantic cues"
% - surfel map: 
%   - how does it scale ? authors satate they tested on room-sized environments, but could be applied in larger scale as well ...
%     - could maybe be applied as VO in a SLAM algorithm ...
%   - demo only on a small pallet + surroundings, might not be applicable in large-scale SLAM

% US VS THEM:
%
% - based heavily on PanopticFusion, with modifications:
%   - instead of greedy data-association (which seems to be the case in others as well), we solve LAP (JPDA?)
%     - overlap threshold can be tuned, which renders the algorithm more flexible
%     - could be extended to dynamic tracking ?
%   - multiple options for association likelihood
%   - outlier rejection (either clustering or probabilistic)
%   - test different options for decreasing processing time
%   - no post-processing
%
% - model-agnostic:
%   - completely separated from segmentation
%   - does not care how point clouds are obtained -> applicable for LIDAR segmentation (e.g. EfficientLPS) as well
%
% - also agnostic to localisation method
%   - could, however, be utilised to find landmark locations / poses

% More compact version of this paragraph to introduction to save space?
%Panoptic segmentation -- proposed in \cite{panoptic_segmentation} -- aims to solve the unified task of semantic- and instance segmentation. Semantic classes are separated to \textit{stuff} -- amorphous, unquantifiable regions like sky, road or floor -- and \textit{things} -- quantifiable objects. The distinction between the two can vary depending on the application, but a semantic class can only belong to one or another. The article also proposes a unified panoptic evaluation metric, coined \textbf{Panoptic Quality} (PQ). Many 2D approaches to panoptic segmentation -- \textit{e.g.} \cite{panopticfpn,seamless,panoptic_deeplab,efficientps} -- have since been proposed. Deep neural networks for performing semantic- or instance segmentation directly on the 3D reconstruction -- \textit{e.g.} on \cite{scannet,s3dis,paris_lille_3d} -- have also been proposed, but since they require the reconstructed 3D scene, they are mostly offline approaches and therefore out of scope for this work. Some recent works also apply panoptic segmentation to point clouds -- \textit{e.g.} methods in the SemanticKITTI panoptic segmentation competition \cite{semantic_kitti} -- mostly aimed at segmenting LiDAR output. They are suitable for online processing, but similar to RGB-D images require a method for tracking object instances persistent in both time and space. In fact, our proposed method, as well as some others mentioned in this work, could use segmented LiDAR point clouds as an input similarly to RGB-D images.

PanopticFusion \cite{panopticfusion} is the first work to propose online integration of panoptic image segmentations to a 3D reconstruction. They integrate point clouds generated from segmented images to a TSDF voxel volume \cite{tsdf,voxblox} by greedily matching detected segments with the reconstruction and regulating each voxel's corresponding instance with a weighting function. Semantic labels are inferred in a bayesian manner based on confidence scores provided by the segmentation model. They also apply a Conditional Random Field (CRF) to regularise the reconstruction, improving results significantly. Voxblox++ \cite{voxblox++} -- introduced later the same year -- is a similar approach that also integrates segmented RGB-D images into a TSDF volume. It leverages geometric segmentation of depth images to improve instance segmentation accuracy. Both geometric and semantic segments are used to compute a pair-wise weight, which is used to greedily match them with segments in the reconstruction. Because of the geometric segmentation, the method allows segmentation of objects with no known semantic class in addition to objects recognised by the instance segmentation model. 

Recently, \cite{interactive_3d_scenes} built upon the idea of Voxblox++. They apply Voxblox++ for 3D instance integration, with two small but effective modifications: the pair-wise weight is replaced by a triplet weight that also takes semantic labels into account in the fusion, and -- in addition to geometric segments -- instance segments are fused if they overlap by a significant amount. The article introduces a method for searching and aligning CAD models to reconstructed objects based on geometry and semantic class, as well as geometrical and physical rules. With the CAD models, a contact graph and interactive virtual scene are reconstructed to allow a robot to simulate its interaction with the environment. SceneGraphFusion \cite{scenegraphfusion} is another approach that forms a scene graph online from a stream of RGB-D images, but unlike the above-mentioned approach, it generates the graph with a deep neural network, after which the panoptic labels for geometrically segmented portions of the 3D reconstruction are produced a side product.

Panoptic-MOPE \cite{panoptic_mope} is another recent approach, which integrates sequences of RGB-D images into a surfel reconstruction. Unlike other mentioned approaches -- which assume the camera pose either known or estimated elsewhere -- it also tracks camera movements based on geometric-, appearance- and semantic cues. The method also applies a novel RGB-D panoptic segmentation model. Although it is only tested on room-sized environments, the authors claim it could be scaled to larger environments as well.


The proposed segmentation-by-detection framework, as depicted in Figure \ref{fig:framework}, consists of a detection module and a segmentation module.
In detection stage, 2D slices (layered box) from the input volume are fed to the RPN. Based on the region proposals obtained from RPN, an attention model (block in orange) is formed. The input volume as well as the attention model are further processed in segmentation stage to get the refined anatomical segmentation. 
\vspace{1em} 

\begin{figure}[t]
\centering
\includegraphics[width=0.95\linewidth]{fig/framework.pdf}
\caption{Schematic representation of the segmentation-by-detection framework. The left part is the detection module while the segmentation module is followed on the right. The blue block denotes the input volume which is 3D ultrasound scan of femoral head. The output segmentation is in red.}
\label{fig:framework}
\end{figure}
% dana could you improve the figure. we can try to think together of better ways 

\noindent\textbf{Detection Module:} 
% dana : here you have to make the clarification that you have ground truth on the boxes (in implementation part)
The detection module follows an RPN architecture, a fully convolutional network which takes image slice as input and outputs object region candidates. 
We use the VGG-16 model as the backbone \cite{simonyan2014very} to learn convolutional features and an $3 \times 3$ spatial window to generate region proposals. At each sliding-window location, 9 anchors are predicted associated with different scales and aspect ratios. The last layer consists of a box-regression (reg) layer and a box-classification (cls) layer in parallel. The reg layer outputs 4 regression offsets, $ t = (t_x,t_y,t_w,t_h)$, denoting a scale-invariant translation as well as log-space height and width shift, where $x,y,w$ and $h$ specify two coordinates of the box center, width and height. The cls layer outputs two scores by softmax, related to probabilities of object and background for each proposal. We assign a positive label (of being object) to candidate which has an Intersection-over-Union (IoU) ratio higher than 0.7 with ground truth box. Note that an image slice may contain multiple object regions or none. 

The loss function of RPN follows the multi-task loss \cite{ren2015faster} which is defined as $L = L_{reg} + L_{cls}$. The regression loss, $L_{reg} = -\log p_{obj}$ is log loss and the classification loss,
\begin{equation} \label{eq:loss}
L_{cls} = \sum_{i \in \{x,y,w,h\}} smooth_{L_1} (t_i - t_i^*)
\end{equation}
is smooth $L_1$ loss where $t_i^*$ denotes the ground truth box for the target object. 
\vspace{1em}

\noindent\textbf{Segmentation Module:}
3D U-Net \cite{cciccek20163d} is utilized in the segmentation module as its outstanding performance in medical image segmentation. The u-shaped architecture consists of two paths: a contracting path, where each layer contains two $3\times3\times3$ convolutions followed by a rectified linear unit (ReLU) and then a max pooling, provides high resolution features. While, the symmetric expanding path for semantically richer features replaces max pooling with a upconvolution $2\times2\times2$ with stride of 2 in each dimension, and then two $3\times3\times3$ convolutions each followed by a ReLU. Skip connections between layers of equal resolution in the contracting path and the expanding path enables context information as well as precise localization.

Different from 3D U-Net, to incorporate the attention model detected by the RPN, our architecture takes as input both the volumetric image data and the candidate RoIs proposed by the RPN, concatenated as 3D volume. 
% dana not sure what you like to say below
% densely annotated
The attention model makes the network to focus on the potential RoIs and can reduce the interference of the surrounding noise.
The anatomical segmentation is then generated from a $1\times1\times1$ convolution which reduces the number of feature maps to the number of labels.  The energy function is computed by a pixel-wise softmax combined with the cross entropy loss.
% dana equation ??

\subsection{System and implementation Details}
The segmentation-by-detection approach adopts a cascade structure with two stages: detection and segmentation. The two networks are trained separately in an end-to-end manner. All the new layers are randomly initialized from zero-mean Gaussian distribution with standard deviations 0.01. Biases are initialized to 0. We use Caffe \cite{jia2014caffe} for the implementation and an NVIDIA Titan X GPU for training.

In the detection stage, we initialize the VGG-16 model by the pre-trained model for ImageNet classification \cite{russakovsky2015imagenet} and further fine-tune the model for our detection task. The input fed to the network are image slices with a fixed size of $184\times96$ and the corresponding ground truth boxes are generated from the annotation in the format of tight bounding boxes surrounding the segmentation contour (as illustrated in Figure \ref{fig:hip} (b), the boundary of white area). To optimize the energy function, stochastic gradient descent (SGD) is used. The global learning rate is set to 0.001, while a momentum of 0.9 and a weight decay of 0.0005 are used. The batch size is set to 256 and each mini-batch only contains the positive anchors for training. The region proposals are obtained from the reg path for each image slice. The attention model is then formed by concatenating all the detected regions, as binary masks, into a volume.

In the segmentation stage, we use the Adam optimizer \cite{kingma2014adam} to learn the network parameters. A global learning rate is set to 0.001 while the two momentum coefficients are set to 0.9 and 0.999 respectively. A batch size of 1 is used due to the memory constraints of the GPU. The network takes the volume data as well as the attention model as input. We train the network for a maximum of 30K iterations and reserve the learned weights with the best performance from every 1K iterations. 
\vspace{1em}

\noindent\textbf{Inference:}
At test time, the 2D slices from an input volume are first fed to the detection module. The attention model is obtained based on the output. Then the volume data as well as the attention model are fed to the segmentation module to get the pixel-wise prediction.





\section{Experiments}\label{sec:experiments}
We validate our approach using multiple datasets containing real-life data from the fields of criminal risk assessment, credit, lending, and college admissions. In each of the datasets we select a binary feature and treat it as the protected attribute (e.g., race or gender), which is the feature we require our trained classifier to behave fairly upon. Our proposed method performs well on all of these datasets, succeeding in removing unfairness almost entirely, at a very modest price in terms of accuracy.


\begin{table*}[h]
\centering
\resizebox{\textwidth}{!}{
\def\arraystretch{1.2}

\begin{tabular}{c c c | c | c | c || c | c | c || c | c | c |}

\cline{4-12}
&&&
\multicolumn{9}{ c| }{\textbf{COMPAS Dataset}}
\\ \cline{4-12}
&&&
\multicolumn{3}{ c|| }{\textbf{FPR Considerations}}&
\multicolumn{3}{ c|| }{\textbf{FNR Considerations}}&
\multicolumn{3}{ c| }{\textbf{Both Considerations}}
\\ \cline{4-12}
&&&
 $\mathbf{Acc.}$ &  $\mathbf{D_{FPR}}$ &  $\mathbf{D_{FNR}}$ &  $\mathbf{Acc.}$ &  $\mathbf{D_{FPR}}$ &  $\mathbf{D_{FNR}}$ &  $\mathbf{Acc.}$ &  $\mathbf{D_{FPR}}$ &  $\mathbf{D_{FNR}}$
\\  \cline{4-12}
\vspace*{-0.5ex}
\\ \cline{1-2} \cline{4-12}
\multicolumn{1}{ |c  }{} &
\multicolumn{1}{ c|  }{  \textbf{Our Method (AVD Penalizers)}}  &&
$\mathbf{0.660}$    &  $\mathbf{0.01}$  &  $0.04$ &
$\mathbf{0.653}$    &  $0.02$   &  $\mathbf{0.04}$ &
$\mathbf{0.654}$    &  $\mathbf{0.02}$  &  $\mathbf{0.04}$
\\ \cline{1-2} \cline{4-12}
\multicolumn{1}{ |c  }{} &
\multicolumn{1}{ c|  }{  \textbf{Our Method (SD Penalizers)}}  &&
$\mathbf{0.664}$    &  $\mathbf{0.02}$  &  $0.09$ &
$\mathbf{0.661}$    &  $0.05$   &  $\mathbf{0.03}$ &
$\mathbf{0.661}$    &  $\mathbf{0.02}$  &  $\mathbf{0.03}$
\\ \cline{1-2} \cline{4-12}
\multicolumn{1}{ |c  }{} &
\multicolumn{1}{ c|  }{  Zafar et al.~(\citeyear{disparatemistreatment})}  &&
$0.660$    &   $0.06$    &   $0.14$  &
$0.662$    &   $0.03$    &   $0.10$  &
$0.661$    &   $0.03$    &   $0.11$
\\ \cline{1-2} \cline{4-12}
\multicolumn{1}{ |c  }{} &
\multicolumn{1}{ c|  }{  Zafar et al. Baseline~(\citeyear{disparatemistreatment})}  &&
$0.643$    &   $0.03$    &   $0.11$  &
$0.660$    &   $0.00$    &   $0.07$  &
$0.660$    &   $0.01$    &   $0.09$
\\ \cline{1-2} \cline{4-12}
\multicolumn{1}{ |c  }{} &
\multicolumn{1}{ c|  }{  Hardt et al.~(\citeyear{hardt})}  &&
$0.659$    &  $0.02$    &   $0.08$  &
$0.653$    &  $0.06$   &    $0.01$  &
$0.645$    &  $0.01$   &    $0.01$
\\ \cline{1-2} \cline{4-12}
\multicolumn{1}{ |c  }{} &
\multicolumn{1}{ c|  }{  \textbf{Vanilla Regularized Logistic Regression}}  &&
$\mathbf{0.672}$    &   $\mathbf{0.20}$    &   $\mathbf{0.30}$  &
$\mathbf{0.672}$    &   $\mathbf{0.20}$    &   $\mathbf{0.30}$  &
$\mathbf{0.672}$    &   $\mathbf{0.20}$    &   $\mathbf{0.30}$
\\ \cline{1-2} \cline{4-12}
\end{tabular}
}
\vspace{3mm}
\caption{Performance comparison on the COMPAS dataset. For the approaches in bold -- Accuracy, FPR difference and FNR difference are evaluated on the test set, averaging over five runs and using a 70-30 training/test split. The performance of the remaining three approaches is stated as reported in Zafar et al.~(\citeyear{disparatemistreatment}).} \label{table:comparison_results}
\end{table*}



\begin{figure*}[b]
  \includegraphics[scale=0.6]{compas0-400.png}
  \caption{COMPAS Dataset. Accuracy, FPR difference ($\mathbf{D_{FPR}}$), and FNR difference ($\mathbf{D_{FNR}}$) (all evaluated on the test set) of the learned classifier, as a function of the weight $c=c_1 = c_2 \geq 0$ placed on the fairness penalizer terms. On the left we use the Absolute Value Difference (AVD) penalizer, and the Squared Difference (SD) penalizer on the right, both as presented in Section~\ref{regularization}. ``Relaxed FPR/FNR Diff.'' plots the value of the relevant penalization term.} %In this particular run, parameters chosen for the absolute value relaxation were: $c=80, q_c=60$, and for the squared relaxation: $c=220, q_c=30$.}
  \label{fig:compas}
\end{figure*}


\subsection{Implementation}
\textbf{Our method} 
%We instantiate our method in the following way: Given dataset $Q$, we split it randomly into a training set $S$ (which we will use for learning) and a test set $T$ (which we will only use for reporting performance). 
For the purpose of comparison with  Zafar et al.~(\citeyear{disparatemistreatment}) and Hardt et al.~\cite{hardt} on the COMPAS data, we use a parameter $c$ to induce three possible combinations of weights on the FPR and FNR penalization terms: $c = c_1$ and $c_2 = 0$; $c_1 = 0$ and $c = c_2$; and $c = c_1 = c_2$. For the other three datasets, we consider only $c = c_1 = c_2$.\footnote{The reason for varying the values of $c$ in the training phase is since we shifted to a proxy problem, in which we rely on the distance from the decision boundary rather the actual classifications. 
%Our hope is that there is no need for a worst-case cross validation between all of the combinations of $c_1, c_2, c_3$, and that the training scheme we propose is sufficient. 
It is possible, of course, that even better results are attainable using our scheme with other combinations of $c_1, c_2$, and $q$.} To explore the accuracy/fairness trade-off curve for the relaxed optimization problem~(\ref{eq:2}), we train for different values of $c$, starting at $c=0$ (which is just standard logistic regression), and growing gradually.



Given a dataset $Q$ and fixing a $d_1, d_2 \in \{0, 1\}$ of interest, we use the following training scheme:
\begin{enumerate}
\item Split $Q$ at random into training set $S$ and test set $T$.
\item For each $c$, perform cross-validation on $S$ to select the corresponding best value $q_c$ for the regularization parameter.
\item For each $(c,q_c)$, let $\theta_c = \argmin\limits_{\theta} \text{Proxy}(\theta;S,c,c,q_c)$.
\item Select $\theta^* \in \argmin\limits_{\theta_c} \text{Objective}(\theta_c;S,d_1,d_2)$.
\item Evaluate performance using $\theta^*$ on test set $T$.
\end{enumerate}
We report the average of five such runs, each with a fresh training-test split.




%We instantiate our method by solving the relaxed optimization problem~(\ref{eq:2}), in place of the original, non-convex problem~(\ref{eq:1}).  
%We test our approach with three different combinations of weights on the penalization terms:
%\katrina{What are the $d$, and how are they related to the $c$s?}
%\begin{enumerate}
%\item FPR considerations only: $d_1 = 1, d_2 = 0$.
%\item FNR considerations only: $d_1 = 0, d_2 = 1$.
%\item Both FPR, FNR considerations, assigned similar significance: $d_1 = 1, d_2 = 1$.
%\end{enumerate}
%One could, of course, pick any other combination of the FPR and FNR penalty weights.

%\katrina{I don't understand how the below is distinct from the list above}
%Learning is done by training the parameters of a logistic regressor to solve~\ref{eq:2}, while picking the value of $c_1, %c_2$ as the following:
%\begin{enumerate}
%\item FPR considerations only: $c_1 = c \geq 0$, $c_2 = 0$.
%\item FNR considerations only: $c_1 = 0$, $c_2 = c \geq 0$.
%\item Both FPR, FNR considerations, assigned similar significance: $c_1 = c_2 = c \geq 0$
%\end{enumerate}



% We then cross-validate to pick the best $c_3$ (the weight on the standard $\ell_2$-regularization term) given $c$.\footnote{The reason for varying the values of $c$ in the training phase is since we shifted to a proxy problem, in which we rely on the distance from the decision boundary rather the actual classifications. 
%Our hope is that there is no need for a worst-case cross validation between all of the combinations of $c_1, c_2, c_3$, and that the training scheme we propose is sufficient. 
%It is possible, of course, that even better results are attainable using our scheme with other combinations of $c_1, c_2, c_3$.} For each such combination, we report results as the averages of multiple \katrina{how many?} different runs, each time splitting data randomly into training and test sets.
%\yahav{We need to shorten this description.}

We solve the relaxed convex optimization problem using the CVXPY solver. Due to stability issues with large training sets, we use a train/test split of 30-70 on the larger datasets, rather than 70-30 as on the COMPAS dataset\footnote{The code implementing our method can be found at https://github.com/jjgold012/lab-project-fairness}.

%
%
%We then report the results (as evaluated on the test set) attained by a regressor $\theta \in \mathbb{R}^d$ that minimizes (on the training set $S$) a weighted combination of the $0$-$1$ loss and the differences in FPR and FNR across populations:
%\begin{equation*}
%\begin{aligned}
%&\underset{\theta}{\text{argmin}}
%& & L_{S}^{0\text{-}1}(\theta) \\
%&&& + d_1|FPR_{A=0}(\theta;S)-FPR_{A=1}(\theta;S)| \\
%&&& + d_2|FNR_{A=0}(\theta;S)-FNR_{A=1}(\theta;S)|
%\end{aligned}
%\end{equation*}
%
%\katrina{What is $d_1$ vs. $c_1$ etc.?}



%For classification, we decided use a standard cut-off threshold of $c=0.5$. There are of course, further possible interactions between the FPR, FNR considerations, and picking a certain cut-off level. These are not straightforward, since  these interactions are data-specific. 



%allows for flexibility in picking the values of $c_1, c_2$, which reflect the significance we wish to place on the objectives of achieving accuracy, equal FPR, and equal FNR. As for $c_3$, we will want to find the value of it that achieves the best results, for any combined objective of accuracy and fairness defined by a specific selection of $c_1,c_2$. Therefore, given a specific selection of $c_1, c_2$, we apply cross-validation to select the value of $c_3$. 




We briefly describe the other algorithmic approaches to which we compare:\\
\textbf{Zafar et al.}~(\citeyear{disparatemistreatment}) performs optimization by considering a proxy for the bias: the covariance between the samples' sensitive attributes and the signed distance between the feature vectors of misclassified users and the classifier decision boundary.\\
\textbf{Zafar et al. Baseline}~(\citeyear{disparatemistreatment}) tries to enforce equal FP/FN rates on the different groups by introducing different penalties for misclassified data points with different sensitive attribute values during the training phase.\\
\textbf{Hardt et al.}~(\citeyear{hardt}) performs post-processing on a standard trained (unfair) logistic regressor, picking different decision thresholds for different groups, and possibly adding randomization.


\subsection{Experimental Results}

In what follows, we use the following notation, given a trained classifier $\hat{Y}$:
\begin{align*}
\mathbf{D_{FPR}}&=\left|FPR_{A=0}(\hat{Y})-FPR_{A=1}(\hat{Y})\right| \\ 
\mathbf{D_{FNR}}&=\left|FNR_{A=0}(\hat{Y})-FNR_{A=1}(\hat{Y})\right|
\end{align*}
The values $FPR_{A=0}(\hat{Y})$, $FPR_{A=1}(\hat{Y})$, $FNR_{A=0}(\hat{Y})$, $FNR_{A=1}(\hat{Y})$ are reported as evaluated on the test set.

\paragraph{The COMPAS Dataset\footnote{https://github.com/propublica/compas-analysis}} The Correctional Offender Management Profiling for Alternative Sanctions (COMPAS) records from Broward County, Florida 2013-2014, made available online by ProPublica, are perhaps the best-studied data in the context of fairness.  The goal in this scenario is to successfully predict recidivism within two years, based on features such as age, gender, race, number of prior offenses, and charge degree. The dataset contains 5,278 samples. The protected attribute in this scenario is race, where $A$ indicates black or white. We filtered the dataset using the same features as Zafar et al.~(\citeyear{disparatemistreatment}), to allow for comparison.

%\begin{table}[h]
%\centering
%\begin{tabularx}{\columnwidth}{c|c|c|c}
%\hline
%  &  Recid. ($y = 1$)        & No Recid.  ($y = 0$)       & Total \\ \hline
%Black &  $ 1661   $ & $ 1514 $ &  $ 3175 $ \\ \hline
%White &  $ 822   $  & $1281  $ &  $ 2103 $ \\ \hline
%Total &  $ 2483  $  & $2795 $ &  $ 5278 $ \\\hline
%\end{tabularx}
%\caption{Statistics of the ProPublica COMPAS data.} \label{table:compas-stats}
%\label{tab:stats}
%\end{table}
%\vspace{-1em}

%\begin{table}[h]
%\centering
%\begin{tabularx}{\columnwidth}{c|c}
%\hline
%Feature  &  Description \\ \hline
%Age Category &  $<25$, between $25$ and $45$, $>45$ \\
%Gender &  Male or Female \\
%Race &  White or Black \\
%Priors Count &  0--37 \\
%Charge Degree &  Misconduct or Felony \\
%\hline
%2-year-recid. & Whether or not the  \\
%(target feature)  & defendant recidivated within two years
%\end{tabularx}
%\caption{Description of features used from ProPublica COMPAS data.} \label{table:compas-features}
%\label{tab:features}
%\end{table}




\begin{table*}[t]
\centering
\caption{A description of the datasets used, along with parameters of the training procedure used for each.}
\label{table:datasets_description}
\begin{adjustbox}{max width=\textwidth}
\begin{tabular}{|l|l|l|l|l|l|l|l|}
\hline
\textbf{Dataset} & \textbf{No. Samples} & \textbf{No. Features} & \textbf{Train/Test Split} & \textbf{No. Repetitions} & \textbf{No. Folds in CV} & \textbf{Protected Feature} & \textbf{Target Variable} \\ \hline
COMPAS           & 5,278                     & 5                          & 70-30                     & 5                        & 5                                 & Race                       & 2-Year-Recidivism        \\ \hline
Adult            & 30,162                    & 10                         & 30-70                     & 5                        & 5                                 & Gender                     & Income Over/Under 50K    \\ \hline
Default          & 30,000                    & 23                         & 30-70                     & 5                        & 3                                 & Gender                     & Defaulting On Payments   \\ \hline
Admissions       & 20,839                    & 17                         & 30-70                     & 5                        & 3                                 & Race                       & Passing Bar Exam         \\ \hline
\end{tabular}
\end{adjustbox}
\end{table*}


\begin{table*}[t]
\centering
\resizebox{\textwidth}{!}{
\def\arraystretch{1.2}

\begin{tabular}{c c c | c | c | c || c | c | c || c | c | c |}

\cline{4-12}
&&&
\multicolumn{3}{ c|| }{\textbf{Adult Dataset}}&
\multicolumn{3}{ c|| }{\textbf{Default Dataset}}&
\multicolumn{3}{ c| }{\textbf{Admissions Dataset}}
\\ \cline{4-12}
%&&&
%\multicolumn{3}{ c|| }{\textbf{Both Considerations}}&
%\multicolumn{3}{ c|| }{\textbf{Both Considerations}}&
%\multicolumn{3}{ c| }{\textbf{Both Considerations}}
%\\ \cline{4-12}
&&&
 $\mathbf{Acc.}$ &  $\mathbf{D_{FPR}}$ &  $\mathbf{D_{FNR}}$ &  $\mathbf{Acc.}$ &  $\mathbf{D_{FPR}}$ &  $\mathbf{D_{FNR}}$ &  $\mathbf{Acc.}$ &  $\mathbf{D_{FPR}}$ &  $\mathbf{D_{FNR}}$
\\  \cline{4-12}
\vspace*{-0.5ex}
\\ \cline{1-2} \cline{4-12}
\multicolumn{1}{ |c  }{} &
\multicolumn{1}{ c|  }{  \textbf{Our Method (AVD Penalizers)}}  &&
$\mathbf{0.776}$    &  $\mathbf{0.00}$  &  $\mathbf{0.04}$ &
$\mathbf{0.807}$    &  $\mathbf{0.00}$   &  $\mathbf{0.01}$ &
$\mathbf{0.950}$    &  $\mathbf{0.01}$  &  $\mathbf{0.00}$
\\ \cline{1-2} \cline{4-12}
\multicolumn{1}{ |c  }{} &
\multicolumn{1}{ c|  }{  \textbf{Our Method (SD Penalizers)}}  &&
$\mathbf{0.783}$    &  $\mathbf{0.00}$  &  $\mathbf{0.09}$ &
$\mathbf{0.806}$    &  $\mathbf{0.01}$   &  $\mathbf{0.02}$ &
$\mathbf{0.950}$    &  $\mathbf{0.00}$  &  $\mathbf{0.00}$
\\ \cline{1-2} \cline{4-12}
\multicolumn{1}{ |c  }{} &
\multicolumn{1}{ c|  }{  \textbf{Vanilla Regularized Logistic Regression}}  &&
$\mathbf{0.800}$    &   $\mathbf{0.08}$    &   $\mathbf{0.39}$  &
$\mathbf{0.807}$    &   $\mathbf{0.01}$    &   $\mathbf{0.05}$  &
$\mathbf{0.951}$    &   $\mathbf{0.16}$    &   $\mathbf{0.02}$
\\ \cline{1-2} \cline{4-12}
\end{tabular}
}
\vspace{3mm}
\caption{Performance on the Adult, Loan Default, and Admissions datasets, penalizing for both FPR and FNR difference. Accuracy, FPR difference and FNR difference are evaluated on the test set, averaging over five runs and using a 30-70 training/test split.} \label{table:comparison_results_rest}
\end{table*}


In Table~\ref{table:comparison_results}, we compare the performance of our approach with that of three other techniques from the literature. Each method was trained based on logistic regression.  As a basis for comparison, we also present the performance of vanilla logistic regression, absent fairness considerations, with the regularization parameter selected via cross-validation.\footnote{Zafar et al.~(\citeyear{disparatemistreatment}) do not incorporate regularization in any of the approaches they report.}
%Results are reported as the averages of 5 different runs \katrina{Is that still correct?}, each time splitting data evenly and randomly into training and test sets. 
Results for Zafar et al., Zafar et al. baseline, and Hardt et al. appear here as reported in Zafar et al.~(\citeyear{disparatemistreatment}).\footnote{Our method selects the classifier based on the training set only and reports its performance over the test set. Results for the three other approaches, reported by Zafar et al.~(\citeyear{disparatemistreatment}), are based on tuning parameters after seeing the trade-off curve over the test set, and reporting according to the best selection of these parameters.}
%\katrina{Perhaps here is the right place for a footnote about the discrepancy with the Zafar baseline}

We find that the vanilla logistic regressor (absent fairness considerations) results in significant unfairness, as $\mathbf{D_{FPR}}=0.20$, and $\mathbf{D_{FNR}}=0.30$. The overall accuracy of this classifier measured on the test set was $0.672$.\footnote{Zafar et al.~(\citeyear{disparatemistreatment}) report a slightly different baseline of: Accuracy = 0.668, $\mathbf{D_{FPR}}=0.18$, $\mathbf{D_{FNR}}=0.30$.} Our SD penalization approach empirically achieves approximately the same accuracy as the Zafar et al.~(\citeyear{disparatemistreatment}) approach, with significantly better fairness. It is difficult to compare fairness-accuracy tradeoffs with the Hardt et al.~(\citeyear{hardt}) approach, since their accuracy is significantly lower than ours. A more direct comparison is possible by noting that our learned classifier can be post-processed to improve its fairness at a direct cost to accuracy. Hence, we can achieve accuracy of $0.659$ with $\mathbf{D_{FPR}} = \mathbf{D_{FNR}} = 0.01$, which compares very favorably with the Hardt et al. accuracy rate of 0.645 given the same FPR and FNR rates.\footnote{For completeness, we note that using a 50-50 training-test split (again not using the test set for parameter selection), our method (SD, both considerations) produces a classifier that provides: Accuracy = 0.659, $\mathbf{D_{FPR}} = 0.01, \mathbf{D_{FNR}} = 0.05$. This classifier can be post-processed to achieve rates of: Accuracy = 0.655, $\mathbf{D_{FPR}} = \mathbf{D_{FNR}} = 0.01$.}

Figure \ref{fig:compas} illustrates the accuracy/fairness trade-offs achievable using our scheme. Increasing the weight $c$ on the proxy fairness penalizers results in reducing their magnitude. The figure also illustrates how our relaxed penalizers succeed in tracking the real FPR and FNR differences. 
%
%
%\katrina{Must rewrite the following paragraph}
%We observe that our method succeeds in eliminating unfairness almost completely on the COMPAS dataset, while retaining most of the accuracy, when compared to the vanilla logistic regression. We achieve very low difference rates when penalizing for achieving each of the FPR and FNR criteria individually, and also for both. We achieve preferable results comparing to Zafar et al. and Zafar et al. baseline in all 3 scenarios, and also comparing to Hardt et al. in the settings of false positive/false negative considerations only. In the setting of both considerations - The Hardt et al. method removes a larger portion of the unfairness, however it results in major accuracy loss as it achieves accuracy rate of 0.645 in comparison to our method which results in accuracy of 0.665, retaining most of the original accuracy rate while removing most of the unfairness.




%The Hardt et al.~\cite{hardt} approach as reported removes a smaller portion of the bias in the different scenarios, however for FP/FN constraints alone, it provides higher accuracy rates. The Zafar et al.~(\citeyear{disparatemistreatment}) approach as reported retains significant bias (in most cases), but in some cases  achieves slightly superior accuracy rates to the methods above. 

%These performance comparisons are incomplete in the sense that each of the compared techniques has the potential to trade off between accuracy and fairness, using some degree of parameter tuning; what we report here is only one point on the achievable trade-off frontier for each algorithm. The ``correct'' trade-off, and, in particular, the best manner in which to weigh unfairness in the FPR against unfairness in the FNR, are matters of opinion. We have chosen to report our method's performance under parameters designed to very aggressively mitigate unfairness, at some cost to the accuracy.

%It would certainly be desirable to evaluate these and other approaches to fair learning on other datasets and on different tasks, particularly on larger datasets, which might afford both greater accuracy and better bias-reduction. The present empirical evaluations, however, suggest that our regularization-based approach provides a new tool worthy of consideration---we succeed in almost entirely eliminating bias on the hold-out set, at a modest price in terms of accuracy.

%Due to the fact that our true objective includes the original non-convex penalization terms, our approach does not carry any formal guarantees. However, the ease of implementation, generality, and empirical results are encouraging. Figure~\ref{fig:test1} illustrates the rate of convergence to a fair, accurate classifier on this dataset.
%In terms of computation costs, given that at each iteration we must calculate the gradient according to the FPR and FNR regularizers, we are required to predict the labels for the entire training set at each step. 
%However, this does not pose a computational burden, as it is already required by the (classic) gradient descent algorithm in our logistic regressor fitting scheme. Furthermore, when given a sufficiently large dataset (one or two orders of magnitude larger than the one currently available for the COMPAS scores data), this could be relaxed to sampling only a mini-batch of samples from the training data set at each iteration (much as is done in stochastic gradient descent).






\subsection{Additional Datasets}


Table~\ref{table:datasets_description} provides summary statistics on each of the datasets on which we tested our approach. We also briefly describe the datasets below. 


{\bf The Adult Dataset}\footnote{http://archive.ics.uci.edu/ml/datasets/Adult} is based on 1994 US Census data. The task we consider is to predict whether the income of each individual is over or under 50K dollars per year, based on features such as occupation, marital status, and education. The protected attribute selected in this task is gender. 

{\bf The Loan Default Dataset}\footnote{{\scriptsize https://archive.ics.uci.edu/ml/datasets/default+of+credit+card+clients}}
contains data regrading Taiwanese credit card users. The task we consider is to predict whether an individual will default on payments, based on features such as history of past payments, age, and the amount of given credit. The protected attribute is gender.

{\bf The Admissions Dataset}\footnote{http://www2.law.ucla.edu/sander/Systemic/Data.htm}
contains records of law school students who went on to take the bar exam. The task we consider is to predict whether a student will pass the exam based on features such as LSAT score, undergraduate GPA, and family income. The protected attribute is set to race.

Table~\ref{table:comparison_results_rest} describes the performance of our approach on these datasets, and Figures~\ref{fig:adult},~\ref{fig:default}, and~\ref{fig:lawschool} illustrate the fairness-accuracy trade-offs we achieve in each context. Overall, we see that unfairness is nearly eliminated while accuracy remains quite high. The dataset on which accuracy suffers most under our approach is the Adult dataset, which is also the dataset on which the vanilla regression is the most unfair.


\begin{figure*}[]
  \includegraphics[scale=0.6]{adult0-800.png}
  \caption{Adult Dataset. Fairness-Accuracy tradeoffs, as in Figure~\ref{fig:compas}.}
  \label{fig:adult}  
\end{figure*}



\begin{figure*}[]
  \includegraphics[scale=0.6]{default0-50.png}
  \caption{Loan Default Dataset. Fairness-Accuracy tradeoffs, as in Figure~\ref{fig:compas}.}
  \label{fig:default}
\end{figure*}



\begin{figure*}[]
  \includegraphics[scale=0.6]{admissions0-400.png}
  \caption{Admissions Dataset. Fairness-Accuracy tradeoffs, as in Figure~\ref{fig:compas}.}
  \label{fig:lawschool}
\end{figure*}





\begin{comment}
\begin{figure}
\includegraphics[width=\linewidth]{figs/beyond_tss_lesion.pdf}
\caption[]{End-to-End runtime lesion study of the entire MNIST dataset and the FMA featurized music dataset. Each of DROP's contributions provides a runtime improvement.}
\label{fig:beyond_lesion}
\end{figure}
\end{comment}



\section{Conclusion}
\label{sec:conclusion}

Advanced data analytics techniques must scale to rising data volumes. 
DR techniques offer a powerful toolkit when processing these datasets, with PCA frequently outperforming popular techniques in exchange for high computational cost. 
In response, we propose DROP, a new dimensionality reduction optimizer. 
DROP combines progressive sampling, progress estimation, and online aggregation to identify high quality low dimensional bases via PCA without processing the entire dataset by balancing the runtime of downstream tasks and achieved dimensionality. 
Thus, DROP provides a first step in bridging the gap between quality and efficiency in end-to-end DR for downstream \red{analytics}. 

%We revisit canonical operators for time series dimensionality reduction and the measurement study of~\cite{keogh-study}, and show that PCA is more effective than popular alternatives in the data mining literature often by a margin of over $2\times$ on average on gold-standard time series benchmark data sets with respect to output data dimension. More surprisingly, we empirically demonstrate that a small number of samples are sufficient to accurately characterize directions of maximum variance and obtain a high-quality low-dimensional transformation.





\subsubsection*{Acknowledgments}
We thank Aditya Grover and Chris Cundy for helpful discussions about earlier versions of the method.
We thank Simran Arora, Sabri Eyuboglu, Bibek Paudel, and Nimit Sohoni for valuable feedback on earlier drafts of this work.
This work was done with the support of Google Cloud credits under HAI proposals 540994170283 and 578192719349.
We gratefully acknowledge the support of NIH under No. U54EB020405 (Mobilize), NSF under Nos. CCF1763315 (Beyond Sparsity), CCF1563078 (Volume to Velocity), and 1937301 (RTML); ONR under No. N000141712266 (Unifying Weak Supervision); ONR N00014-20-1-2480: Understanding and Applying Non-Euclidean Geometry in Machine Learning; N000142012275 (NEPTUNE); the Moore Foundation, NXP, Xilinx, LETI-CEA, Intel, IBM, Microsoft, NEC, Toshiba, TSMC, ARM, Hitachi, BASF, Accenture, Ericsson, Qualcomm, Analog Devices, the Okawa Foundation, American Family Insurance, Google Cloud, Salesforce, Total, the HAI-AWS Cloud Credits for Research program, the Stanford Data Science Initiative (SDSI), and members of the Stanford DAWN project: Facebook, Google, and VMWare. The Mobilize Center is a Biomedical Technology Resource Center, funded by the NIH National Institute of Biomedical Imaging and Bioengineering through Grant P41EB027060. The U.S. Government is authorized to reproduce and distribute reprints for Governmental purposes notwithstanding any copyright notation thereon. Any opinions, findings, and conclusions or recommendations expressed in this material are those of the authors and do not necessarily reflect the views, policies, or endorsements, either expressed or implied, of NIH, ONR, or the U.S. Government.


\bibliography{biblio}

\newpage

\appendix

In this paper, 2D and 3D CNN models were used to generate pelvic sCTs from T1-weighted MR images. Our sCT generation methods were fully automated, requiring no deformable registration or manual segmentation of bone tissues. As shown in Figure~\ref{fig3}, the 2D and 3D CNN models generated high quality sCTs. MAE curves shown in Figure~\ref{fig4} indicated that both models could precisely estimate soft-tissue HU values but had difficulty in reproducing air and high-density bone tissues. 

The MAEs within the body contour across all patients were 40.5 $\pm$ 5.4 HU and 37.6 $\pm$ 5.1 HU for the 2D and 3D models, respectively. The time required for generating a pelvic sCT using our CNN models was about 5.5 s. Our MAE results are comparable to previous studies. Kim $et \ al.$\cite{RN41} presented a voxel-based weighted summation method that produced an MAE of 74.3 $\pm$ 3.9 HU. However, manual contouring of bone tissues required for this method can be tedious and time-consuming. An MAE of 40.5 $\pm$ 8.2 HU was achieved by Dowling $et \ al.$\cite{RN11} using an average MRI-CT atlas from 38 patients. Andreasen $et \ al.$\cite{RN42} reported an MAE of 54 $\pm$ 8 HU using an atlas-based method with pattern recognition, and its prediction time was about 20.8 min. Another random forest model proposed by Andreasen $et \ al.$\cite{RN43} generated sCTs with an MAE of 58 $pm$ 9 HU. A hybrid method suggested by Siversson $et \ al.$ \cite{RN45} obtained an MAE of 36.5 $\pm$ 4.1 HU when ignoring errors introduced by gas cavities. This hybrid method was implemented in the cloud-based commercial software MriPlanner (Spectronic Medical AB, Helsingborg, Sweden), which required 50 to 80 min to generate a sCT.\cite{RN45} The patch-based 3D context-aware generative adversarial network presented by Nie $et \ al.$\cite{RN26} achieved an MAE of 39.0 $\pm$ 4.6 HU. 

Our CNN models reproduced low-density bone as shown in Figure ~\ref{fig4}. The bone-region DSCs were 0.81 $\pm$ 0.04 and 0.82 $\pm$ 0.04 from the 2D and 3D models, respectively. These results are comparable to reported DSC results of 0.79 $\pm$ 0.12\cite{RN10} and 0.91$\pm$0.03{\cite{RN11}}, where the authors compared bone contours manually drawn on the sCT and CT.

It was feasible to train the proposed 3D model with 16 image volumes from scratch. Results of the Wilcoxon signed-rank tests shown in Table~\ref{tab1} demonstrated a statistically significant improvement in overall MAE, bone DSC, and bone precision of the 3D model compared to the 2D model. However, as shown in Figure~\ref{fig4}, the 2D model seemed to perform better in estimating the high-density bone HU values. It should be noted that smaller overall MAEs do not guarantee improved sCT dose calculation and patient positioning performance. While the models performed well, we will continue to acquire more patient data to potentially improve model accuracy and further test model differences.

As this was a retrospective study, the MR image voxel sizes were not matched, resulting in different voxel intensities between images. This may have affected the sCT generation accuracy although we applied intensity normalization. A potential study could examine how voxel size variations affects sCT estimation. 

The proposed 3D model can be implemented on a 12 GB GPU to process volumetric images with dimensions of 256 $\times$ 256 $\times$ 30. More GPU memory would be required to process higher resolution 3D images. Considering the limited access to multi-GPU systems, a 3D architecture with fewer convolutional layers could be considered to deal with higher resolutions. However, the performance could be affected by the reduced parameters and smaller receptive fields of the less complex model. Another approach would be to extract 30-slice sub-volumes from CT and MR images for training the 3D model. The sCT could then be generated by averaging 30-slice sCT sub-volumes produced by the model. 

A number of techniques could be investigated for improving model performance.  Nie $et \ al.$\cite{RN26} showed that introducing an additional adversarial discriminator improved overall sCT quality. The same approach could be adapted in our proposed 2D and 3D CNN models.  Non-rigid deformation\cite{RN44} could also be applied to both CT and MR images in the process of the on-the-fly data augmentation to produce more training pairs. Multiple MR images acquired with different sequences could be fed into models to provide more information for distinguishing different tissues. Multi-GPU systems with more memory would enable the exploration of larger batch sizes for training CNN models, which could reduce variances in gradient estimation and accelerate the training. 


\section{Numerical Instability of LSSL}
\label{sec:lssl-instability}


This section proves the claims made in \cref{sec:s4-motivation} about prior work.
We first derive the explicit diagonalization of the HiPPO matrix, confirming its instability because of exponentially large entries.
We then discuss the proposed theoretically fast algorithm from \citep{gu2021lssl} (Theorem 2) and show that it also involves exponentially large terms and thus cannot be implemented.

\subsection{HiPPO Diagonalization}

\begin{proof}[Proof of \cref{lmm:hippo-diagonalization}]%
  The HiPPO matrix \eqref{eq:hippo} is equal, up to sign and conjugation by a diagonal matrix, to
  \begin{align*}
    \bm{A} &=
    \begin{bmatrix}
      1 \\
      -1 & 2 \\
      1 & -3 & 3 \\
      -1 & 3 & -5 & 4 \\
      1 & -3 & 5 & -7 & 5 \\
      -1 & 3 & -5 & 7 & -9 & 6 \\
      1 & -3 & 5 & -7 & 9 & -11 & 7 \\
      -1 & 3 & -5 & 7 & -9 & 11 & -13 & 8 \\
      \vdots & & & & & & & & \ddots \\
    \end{bmatrix}
    \\
    \bm{A}_{nk} &=
    \begin{cases}%
      (-1)^{n-k} (2k+1) & n > k \\
      k+1 & n=k \\
      0 & n<k
    \end{cases}
    .
  \end{align*}
  Our goal is to show that this \( \bm{A} \) is diagonalized by the matrix
  \begin{align*}
    \bm{V} = \binom{i+j}{i-j}_{ij} =
    \begin{bmatrix}
      1 \\
      1 & 1 \\
      1 & 3 & 1 \\
      1 & 6 & 5 & 1 \\
      1 & 10 & 15 & 7 & 1 \\
      1 & 15 & 35 & 28 & 9 & 1 \\
      \vdots & & & & & & \ddots \\
    \end{bmatrix}
    ,
  \end{align*}
  or in other words that columns of this matrix are eigenvectors of \( \bm{A} \).

  Concretely, we will show that the \( j \)-th column of this matrix \( \bm{v}^{(j)} \) with elements
  \begin{align*}
    \bm{v}^{(j)}_i =
    \begin{cases}%
      0 & i < j \\
      \binom{i+j}{i-j} = \binom{i+j}{2j} & i \ge j
    \end{cases}
  \end{align*}
  is an eigenvector with eigenvalue \( j+1 \).
  In other words we must show that for all indices \( k \in [N] \),
  \begin{equation}
    \label{eq:diagonalization-proof}
    (\bm{A}\bm{v}^{(j)})_k = \sum_i \bm{A}_{ki} \bm{v}^{(j)}_{i} = (j+1) \bm{v}^{(j)}_k.
  \end{equation}

  If \( k < j \), then for all \( i \) inside the sum, either \( k < i \) or \( i < j \).
  In the first case \( \bm{A}_{ki} = 0 \) and in the second case \( \bm{v}^{(j)}_i = 0 \),
  so both sides of equation \eqref{eq:diagonalization-proof} are equal to \( 0 \).

  It remains to show the case \( k \ge j \), which proceeds by induction on \( k \).
  Expanding equation \eqref{eq:diagonalization-proof} using the formula for \( \bm{A} \) yields
  \begin{align*}
    (\bm{A}\bm{v})^{(j)}_k = \sum_i \bm{A}_{ki} \bm{v}^{(j)}_{i}
    = \sum_{i=j}^{k-1} (-1)^{k-i} (2i+1) \binom{i+j}{2j} + (k+1) \binom{k+j}{2j}.
  \end{align*}

  In the base case \( k = j \), the sum disappears and we are left with \( (\bm{A}\bm{v}^{(j)})_j = (j+1) \binom{2j}{2j} = (j+1) \bm{v}^{(j)}_j \), as desired.

  Otherwise, the sum for \( (\bm{A}\bm{v})^{(j)}_{k} \) is the same as the sum for \( (\bm{A}\bm{v})^{(j)}_{k-1} \) but with sign reversed and a few edge terms.
  The result follows from applying the inductive hypothesis and algebraic simplification:
  \begin{align*}
    (\bm{A}\bm{v})^{(j)}_{k}
    &= -(\bm{A}\bm{v})^{(j)}_{k-1} - (2k-1) \binom{k-1+j}{2j} + k\binom{k-1+j}{2j} + (k+1)\binom{k+j}{2j}
    \\&= -(j+1)\binom{k-1+j}{2j} - (k-1)\binom{k-1+j}{2j} + (k+1)\binom{k+j}{2j}
    \\&= -(j+k)\binom{k-1+j}{2j} + (k+1)\binom{k+j}{2j}
    \\&= -(j+k)\frac{(k-1+j)!}{(k-1-j)!(2j)!} + (k+1)\binom{k+j}{2j}
    \\&= -\frac{(k+j)!}{(k-1-j)!(2j)!} + (k+1)\binom{k+j}{2j}
    \\&= -(k-j)\frac{(k+j)!}{(k-j)!(2j)!} + (k+1)\binom{k+j}{2j}
    \\&= (j-k)(k+1)\binom{k+j}{2j} + (k+1)\binom{k+j}{2j}
    \\&= (j+1)\bm{v}^{(j)}_k
    .
  \end{align*}

\end{proof}

\subsection{Fast but Unstable LSSL Algorithm}

Instead of diagonalization,
\citet[Theorem 2]{gu2021lssl} proposed a sophisticated fast algorithm to compute
\begin{align*}
  K_L(\bm{\overline{A}}, \bm{\overline{B}}, \bm{\overline{C}}) = (\bm{\overline{C}}\bm{\overline{B}}, \bm{\overline{C}}\bm{\overline{A}}\bm{\overline{B}}, \dots, \bm{\overline{C}}\bm{\overline{A}}^{L-1}\bm{\overline{B}}).
\end{align*}
This algorithm runs in \( O(N\log^2 N + L\log L) \) operations and \( O(N+L) \) space.
However, we now show that this algorithm is also numerically unstable.

There are several reasons for the instability of this algorithm, but most directly we can pinpoint a particular intermediate quantity that they use.
\begin{definition}%
  The fast LSSL algorithm computes coefficients of \( p(x) \), the characteristic polynomial of \( A \), as an intermediate computation.
  Additionally, it computes the coefficients of its inverse, \( p(x)^{-1} \pmod{x^L} \).
\end{definition}

We now claim that this quantity is numerically unfeasible.
We narrow down to the case when \( \bm{\overline{A}} = \bm{I} \) is the identity matrix.
Note that this case is actually in some sense the most typical case:
when discretizing the continuous-time SSM to discrete-time by a step-size \( \dt \),
the discretized transition matrix \( \bm{\overline{A}} \) is brought closer to the identity.
For example, with the Euler discretization \( \bm{\overline{A}} = \bm{I} + \dt \bm{A} \),
we have \( \bm{\overline{A}} \to \bm{I} \) as the step size \( \dt \to 0 \).

\begin{lemma}%
  When \( \bm{\overline{A}} = \bm{I} \), the fast LSSL algorithm requires computing terms exponentially large in \( N \).
\end{lemma}
\begin{proof}%
  The characteristic polynomial of \( \bm{I} \) is
  \begin{align*}
    p(x) =
    \mathsf{det}\left| \bm{I} - x\bm{I} \right|
    = (1-x)^N.
  \end{align*}
  These coefficients have size up to \( \binom{N}{\frac{N}{2}} \approx \frac{2^N}{\sqrt{\pi N/2}} \).

  The inverse of \( p(x) \) has even larger coefficients.
  It can be calculated in closed form by the generalized binomial formula:
  \begin{align*}
    (1-x)^{-N} = \sum_{k=0}^{\infty} \binom{N+k-1}{k}x^k.
  \end{align*}
  Taking this \( \pmod{x^L} \), the largest coefficient is
  \begin{align*}
    \binom{N+L-2}{L-1} = \binom{N+L-2}{N-1} = \frac{(L-1)(L-2)\dots(L-N+1)}{(N-1)!}.
  \end{align*}
  When \( L=N-1 \) this is
  \begin{align*}
    \binom{2(N-1)}{N-1} \approx \frac{2^{2N}}{\sqrt{\pi N}}
  \end{align*}
  already larger than the coefficients of \( (1-x)^N \), and only increases as \( L \) grows.
\end{proof}

\section{\methodabbrv{} Algorithm Details}
\label{sec:s4-details}

This section proves the results of \cref{sec:s4-efficiency}, providing complete details of our efficient algorithms for \methodabbrv{}.

\cref{sec:s4-nplr-proof,sec:s4-recurrence-proof,sec:s4-convolution-proof}
prove \cref{thm:hippo-nplr,thm:s4-recurrence,thm:s4-convolution}
respectively.

\subsection{NPLR Representations of HiPPO Matrices}
\label{sec:s4-nplr-proof}

We first prove \cref{thm:hippo-nplr},
showing that all HiPPO matrices for continuous-time memory fall under the \methodabbrv{} normal plus low-rank (NPLR) representation.

\begin{proof}[Proof of \cref{thm:hippo-nplr}]%
  We consider each of the three cases HiPPO-LagT, HiPPO-LegT, and HiPPO-LegS separately.
  Note that the primary HiPPO matrix defined in this work (equation \eqref{eq:hippo}) is the HiPPO-LegT matrix.

  \textbf{HiPPO-LagT.}
  The HiPPO-LagT matrix is simply
  \begin{align*}
    \bm{A}_{nk} &=
    \begin{cases}%
      0            & n < k \\
      -\frac{1}{2} & n=k   \\
      -1           & n > k \\
    \end{cases}
    \\
    \bm{A} &=
    -
    \begin{bmatrix}
      \frac{1}{2} &             &             &            & \dots \\
      1           & \frac{1}{2} &             &             \\
      1           & 1           & \frac{1}{2} &             \\
      1           & 1           & 1           & \frac{1}{2} \\
      \vdots      &             &             &              & \ddots \\
    \end{bmatrix}
    .
  \end{align*}
  Adding the matrix of all \( \frac{1}{2} \), which is rank 1, yields
  \begin{align*}
    -
    \begin{bmatrix}
      & -\frac{1}{2} & -\frac{1}{2} & -\frac{1}{2} \\
      \frac{1}{2} &              & -\frac{1}{2} & -\frac{1}{2} \\
      \frac{1}{2} & \frac{1}{2}  &              & -\frac{1}{2} \\
      \frac{1}{2} & \frac{1}{2}  & \frac{1}{2}  &              \\
    \end{bmatrix}
    .
  \end{align*}
  This matrix is now skew-symmetric.
  Skew-symmetric matrices are a particular case of normal matrices
  with pure-imaginary eigenvalues.

  \citet{gu2020hippo} also consider a case of HiPPO corresponding to the generalized Laguerre polynomials that generalizes
  the above HiPPO-LagT case.
  In this case, the matrix \( \bm{A} \) (up to conjugation by a diagonal matrix) ends up being close to the above matrix,
  but with a different element on the diagonal.
  After adding the rank-1 correction, it becomes the above skew-symmetric matrix plus a multiple of the identity.
  Thus after diagonalization by the same matrix as in the LagT case, it is still reduced to diagonal plus low-rank (DPLR) form,
  where the diagonal is now pure imaginary plus a real constant.

  \textbf{HiPPO-LegS.}
  We restate the formula from equation \eqref{eq:hippo} for convenience.
  \begin{align*}
    \bm{A}_{nk}
    =
    -
    \begin{cases}
      (2n+1)^{1/2}(2k+1)^{1/2} & \mbox{if } n > k \\
      n+1                      & \mbox{if } n = k \\
      0                        & \mbox{if } n < k
    \end{cases}
    .
  \end{align*}
  Adding \( \frac{1}{2}(2n+1)^{1/2}(2k+1)^{1/2} \) to the whole matrix gives
  \begin{align*}
    -
    \begin{cases}
      \frac{1}{2} (2n+1)^{1/2}(2k+1)^{1/2}  & \mbox{if } n > k \\
      \frac{1}{2}                           & \mbox{if } n = k \\
      -\frac{1}{2} (2n+1)^{1/2}(2k+1)^{1/2} & \mbox{if } n < k \\
    \end{cases}
  \end{align*}

  Note that this matrix is not skew-symmetric,
  but is \( \frac{1}{2}\bm{I} + \bm{S} \) where \( \bm{S} \) is a skew-symmetric matrix.
  This is diagonalizable by the same unitary matrix that diagonalizes \( \bm{S} \).

  \textbf{HiPPO-LegT.}

  Up to the diagonal scaling,
  the LegT matrix is
  \begin{align*}
    \bm{A} =
    -
    \begin{bmatrix}
      1      & -1 & 1  & -1  & \dots \\
      1      & 1  & -1 & 1  \\
      1      & 1  & 1  & -1 \\
      1      & 1  & 1  & 1  \\
      \vdots &    &    &     & \ddots
    \end{bmatrix}
    .
  \end{align*}
  By adding \( -1 \) to this matrix and then the matrix
  \begin{align*}
    \begin{bmatrix}
      &  &   &  &  \\
      2 &  & 2 &  &  \\
      &  &   &  &  \\
      2 &  & 2 &  &  \\
    \end{bmatrix}
  \end{align*}
  the matrix becomes
  \begin{align*}
    \begin{bmatrix}
      & -2 &   & -2 &  \\
      2 &    &   &    &  \\
      &    &   & -2 &  \\
      2 &    & 2 &    &  \\
    \end{bmatrix}
  \end{align*}
  which is skew-symmetric.
  In fact, this matrix is the inverse of the Chebyshev Jacobi.

  An alternative way to see this is as follows.
  The LegT matrix is the inverse of the matrix
  \begin{align*}
    \begin{bmatrix}
      -1 & 1  &    & 0  \\
      -1 &    & 1  &    \\
      & -1 &    & 1  \\
      &    & -1 & -1 \\
    \end{bmatrix}
  \end{align*}
  This can obviously be converted to a skew-symmetric matrix by adding a rank 2 term.
  The inverses of these matrices are also rank-2 differences from each other by the Woodbury identity.

  A final form is
  \begin{align*}
    \begin{bmatrix}
      -1 & 1 & -1 & 1 \\
      -1 & -1 & 1 & -1 \\
      -1 & -1 & -1 & 1 \\
      -1 & -1 & -1 & -1 \\
    \end{bmatrix}
    +
    \begin{bmatrix}
      1 & 0 & 1 & 0 \\
      0 & 1 & 0 & 1 \\
      1 & 0 & 1 & 0 \\
      0 & 1 & 0 & 1 \\
    \end{bmatrix}
    =
    \begin{bmatrix}
      0 & 1 & 0 & 1 \\
      -1 & 0 & 1 & 0 \\
      0 & -1 & 0 & 1 \\
      -1 & 0 & -1 & 0 \\
    \end{bmatrix}
  \end{align*}
  This has the advantage that the rank-2 correction is symmetric (like the others),
  but the normal skew-symmetric matrix is now \( 2 \)-quasiseparable instead of \( 1 \)-quasiseparable.

\end{proof}

\subsection{Computing the \methodabbrv{} Recurrent View}
\label{sec:s4-recurrence-proof}

We prove \cref{thm:s4-recurrence} showing the efficiency of the \methodabbrv{} parameterization for computing one step of the recurrent representation (\cref{sec:ss-recurrent}).

Recall that without loss of generality, we can assume that the state matrix \( \bm{A} = \bm{\Lambda} - \bm{P}\bm{Q}^* \) is diagonal plus low-rank (DPLR), potentially over \( \mathbbm{C} \).
Our goal in this section is to explicitly write out a closed form for the discretized matrix \( \bm{\overline{A}} \).

Recall from equation \eqref{eq:2} that
\begin{align*}
  \bm{\overline{A}} &= (\bm{I} - \dt/2 \cdot \bm{A})^{-1}(\bm{I} + \dt/2 \cdot \bm{A}) \\
  \bm{\overline{B}} &= (\bm{I} - \dt/2 \cdot \bm{A})^{-1} \dt \bm{B}
  .
\end{align*}



We first simplify both terms in the definition of \( \bm{\overline{A}} \) independently.

\textbf{Forward discretization.}
The first term is essentially the Euler discretization motivated in \cref{sec:ss-recurrent}.
\begin{align*}
  \bm{I} + \frac{\dt}{2} \bm{A}
  &= \bm{I} + \frac{\dt}{2} (\bm{\Lambda} - \bm{P} \bm{Q}^*)
  \\&= \frac{\dt}{2} \left[ \frac{2}{\dt}\bm{I} + (\bm{\Lambda} - \bm{P} \bm{Q}^*) \right]
  \\&= \frac{\dt}{2} \bm{A_0}
\end{align*}
where \( \bm{A_0} \) is defined as the term in the final brackets.

\textbf{Backward discretization.}
The second term is known as the Backward Euler's method.
Although this inverse term is normally difficult to deal with,
in the DPLR case we can simplify it using Woodbury's Identity (\cref{prop:woodbury}).
\begin{align*}
  \left( \bm{I} - \frac{\dt}{2} \bm{A} \right)^{-1}
  &=
  \left( \bm{I} - \frac{\dt}{2} (\bm{\Lambda} - \bm{P} \bm{Q}^*) \right)^{-1}
  \\&=
  \frac{2}{\dt} \left[ \frac{2}{\dt} - \bm{\Lambda} + \bm{P} \bm{Q}^* \right]^{-1}
  \\&=
  \frac{2}{\dt} \left[ \bm{D} - \bm{D} \bm{P} \left( \bm{I} + \bm{Q}^* \bm{D} \bm{P} \right)^{-1} \bm{Q}^* \bm{D} \right]
  \\&= \frac{2}{\dt} \bm{A_1}
\end{align*}
where \( \bm{D} = \left( \frac{2}{\dt}-\bm{\Lambda} \right)^{-1} \)
and \( \bm{A_1} \) is defined as the term in the final brackets.
Note that
\( \left( 1 + \bm{Q}^* \bm{D} \bm{P} \right) \)
is actually a scalar in the case when the low-rank term has rank \( 1 \).


\textbf{\methodabbrv{} Recurrence.}
Finally, the full bilinear discretization can be rewritten in terms of these matrices as
\begin{align*}
  \bm{\overline{A}} &= \bm{A_1} \bm{A_0} \\
  \bm{\overline{B}} &= \frac{2}{\dt} \bm{A_1} \dt \bm{B} = 2 \bm{A_1} \bm{B}
  .
\end{align*}
The discrete-time SSM \eqref{eq:2} becomes
\begin{align*}
  x_{k} &= \bm{\overline{A}} x_{k-1} + \bm{\overline{B}} u_k \\
  &= \bm{A_1} \bm{A_0} x_{k-1} + 2 \bm{A_1} \bm{B} u_k \\
  y_k &= \bm{C} x_k
  .
\end{align*}
Note that \( \bm{A_0}, \bm{A_1} \) are accessed only through matrix-vector multiplications.
Since they are both DPLR, they have \( O(N) \) matrix-vector multiplication,
showing \cref{thm:s4-recurrence}.


\subsection{Computing the Convolutional View}
\label{sec:s4-convolution-proof}

The most involved part of using SSMs efficiently is computing \( \bm{\overline{K}} \).
This algorithm was sketched in \cref{sec:s4-overview} and is the main motivation for the \methodabbrv{} parameterization.
In this section, we define the necessary intermediate quantities and prove the main technical result. %


The algorithm for \cref{thm:s4-convolution} falls in roughly three stages, leading to \cref{alg:s4-convolution}.
Assuming \( \bm{A} \) has been conjugated into diagonal plus low-rank form, we successively simplify the problem of computing \( \bm{\overline{K}} \)
by applying the techniques outlined in \cref{sec:s4-overview}.

\begin{remark}
  \textbf{We note that for the remainder of this section}, we transpose \( \bm{C} \) to be a column vector of shape \( \mathbbm{C}^{N} \) or \( \mathbbm{C}^{N \times 1} \) instead of matrix or row vector \( \mathbbm{C}^{1 \times N} \) as in \eqref{eq:1}.
  In other words the SSM is
  \begin{equation}
    \begin{aligned}
      x'(t) &= \bm{A}x(t) + \bm{B}u(t) \\
      y(t) &= \bm{C}^* x(t) + \bm{D}u(t)
      .
    \end{aligned}
  \end{equation}
  This convention is made so that \( \bm{C} \) has the same shape as \( \bm{B}, \bm{P}, \bm{Q} \) and simplifies the implementation of S4.
\end{remark}

\paragraph{Reduction 0: Diagonalization}
By \cref{lmm:conjugation}, we can switch the representation by conjugating with any unitary matrix.
For the remainder of this section, we can assume that \( \bm{A} \) is (complex) diagonal plus low-rank (DPLR).

Note that unlike diagonal matrices, a DPLR matrix does not lend itself to efficient computation of \( \bm{\overline{K}} \).
The reason is that \( \bm{\overline{K}} \) computes terms \( \bm{\overline{C}}^* \bm{\overline{A}}^i \bm{\overline{B}} \) which involve powers of the matrix \( \bm{\overline{A}} \).
These are trivially computable when \( \bm{\overline{A}} \) is diagonal, but is no longer possible for even simple modifications to diagonal matrices such as DPLR.

\paragraph{Reduction 1: SSM Generating Function}

To address the problem of computing powers of \( \bm{\overline{A}} \), we introduce another technique.
Instead of computing the SSM convolution filter \( \bm{\overline{K}} \) directly,
we introduce a generating function on its coefficients and compute evaluations of it.

\begin{definition}[SSM Generating Function]%
  \label{def:generating-function}
  We define the following quantities:
  \begin{itemize}%
    \item The \emph{SSM convolution function} is \( \mathcal{K}(\bm{\overline{A}}, \bm{\overline{B}}, \bm{\overline{C}}) = (\bm{\overline{C}}^*\bm{\overline{B}}, \bm{\overline{C}}^*\bm{\overline{A}}\bm{\overline{B}}, \dots) \)
      and the (truncated) SSM filter of length \( L \)
      \begin{equation}
        \mathcal{K}_L(\bm{\overline{A}}, \bm{\overline{B}}, \bm{\overline{C}}) = (\bm{\overline{C}}^*\bm{\overline{B}}, \bm{\overline{C}}^*\bm{\overline{A}}\bm{\overline{B}}, \dots, \bm{\overline{C}}^*\bm{\overline{A}}^{L-1}\bm{\overline{B}}) \in \mathbbm{R}^L
      \end{equation}
    \item The \emph{SSM generating function} at node \( z \) is
      \begin{equation}
        \label{eq:generating-function}
        \hat{\mathcal{K}}(z; \bm{\overline{A}}, \bm{\overline{B}}, \bm{\overline{C}}) \in \mathbbm{C} := \sum_{i=0}^\infty \bm{\overline{C}}^* \bm{\overline{A}}^i \bm{\overline{B}} z^i
        = \bm{\overline{C}}^* (\bm{I} - \bm{\overline{A}} z)^{-1} \bm{\overline{B}}
      \end{equation}
      and the \emph{truncated SSM generating function} at node \( z \) is
      \begin{equation}
        \hat{\mathcal{K}}_L(z; \bm{\overline{A}}, \bm{\overline{B}}, \bm{\overline{C}})^* \in \mathbbm{C} := \sum_{i=0}^{L-1} \bm{\overline{C}}^* \bm{\overline{A}}^i \bm{\overline{B}} z^i
        = \bm{\overline{C}}^* (\bm{I} - \bm{\overline{A}}^L z^L) (\bm{I} - \bm{\overline{A}} z)^{-1} \bm{\overline{B}}
      \end{equation}
    \item The truncated SSM generating function at nodes \( \Omega \in \mathbbm{C}^M \) is
      \begin{equation}
        \hat{\mathcal{K}}_L(\Omega; \bm{\overline{A}}, \bm{\overline{B}}, \bm{\overline{C}}) \in \mathbbm{C}^M := \left( \hat{\mathcal{K}}_L(\omega_k; \bm{\overline{A}}, \bm{\overline{B}}, \bm{\overline{C}}) \right)_{k \in [M]}
      \end{equation}

  \end{itemize}
\end{definition}

Intuitively, the generating function essentially converts the SSM convolution filter from the time domain to frequency domain.
Importantly, it preserves the same information, and the desired SSM convolution filter can be recovered from evaluations of its generating function.
\begin{lemma}%
  The SSM function \( \mathcal{K}_L(\bm{\overline{A}}, \bm{\overline{B}}, \bm{\overline{C}}) \) can be computed from the SSM generating function \( \hat{\mathcal{K}}_L(\Omega; \bm{\overline{A}}, \bm{\overline{B}}, \bm{\overline{C}}) \)
  at the roots of unity \( \Omega = \{ \exp(-2\pi i \frac{k}{L} : k \in [L] \} \)
  stably in \( O(L \log L) \) operations.
\end{lemma}
\begin{proof}%
  For convenience define
  \begin{align*}
    \bm{\overline{K}} &= \mathcal{K}_L(\bm{\overline{A}}, \bm{\overline{B}}, \bm{\overline{C}}) \\
    \bm{\hat{K}} &=  \hat{\mathcal{K}}_L(\Omega; \bm{\overline{A}}, \bm{\overline{B}}, \bm{\overline{C}}) \\
    \bm{\hat{K}}(z) &=  \hat{\mathcal{K}}_L(z; \bm{\overline{A}}, \bm{\overline{B}}, \bm{\overline{C}})
    .
  \end{align*}
  Note that
  \begin{align*}
    \bm{\hat{K}}_j = \sum_{k=0}^{L-1} \bm{\overline{K}}_k \exp\left(-2\pi i \frac{jk}{L}\right).
  \end{align*}
  Note that this is exactly the same as the Discrete Fourier Transform (DFT):
  \begin{align*}
    \bm{\hat{K}} = \mathcal{F}_L \bm{K}.
  \end{align*}
  Therefore \( \bm{K} \) can be recovered from \( \bm{\hat{K}} \) with a single inverse DFT,
  which requires \( O(L \log L) \) operations with the Fast Fourier Transform (FFT) algorithm.
\end{proof}


\paragraph{Reduction 2: Woodbury Correction}

The primary motivation of \cref{def:generating-function} is that it turns \emph{powers} of \( \bm{\overline{A}} \) into a single \emph{inverse} of \( \bm{\overline{A}} \) (equation \eqref{eq:generating-function}).
While DPLR matrices cannot be powered efficiently due to the low-rank term, they can be inverted efficiently by the well-known Woodbury identity.

\begin{proposition}[Binomial Inverse Theorem or Woodbury matrix identity~\cite{woodbury1950,golub2013matrix}]
  \label{prop:woodbury}
  Over a commutative ring $\mathcal{R}$, let $\bm{A} \in \mathcal{R}^{N \times N}$ and $\bm{U},\bm{V} \in \mathcal{R}^{N \times p}$. Suppose $\bm{A}$ and $\bm{A}+\bm{U}\bm{V}^*$ are invertible. Then $\bm{I}_p + \bm{V}^*\bm{A}^{-1}\bm{U} \in \mathcal{R}^{p \times p}$ is invertible and
  \begin{align*}
    (\bm{A} + \bm{U}\bm{V}^*)^{-1} = \bm{A}^{-1} - \bm{A}^{-1}\bm{U}(\bm{I}_p + \bm{V}^*\bm{A}^{-1}\bm{U})^{-1}\bm{V}^*\bm{A}^{-1}
  \end{align*}
\end{proposition}

With this identity, we can convert the SSM generating function on a DPLR matrix \( \bm{A} \) into one on just its diagonal component.

\begin{lemma}%
  \label{lmm:resolvent-woodbury}
  Let \( \bm{A} = \bm{\Lambda} - \bm{P} \bm{Q}^* \) be a diagonal plus low-rank representation.
  Then for any root of unity \( z \in \Omega \), the truncated generating function satisfies
  \begin{align*}
    \bm{\hat{K}}(z) &=
    \frac{2}{1+z}\left[ \bm{\tilde{C}}^* \bm{R}(z) \bm{B} - \bm{\tilde{C}}^* \bm{R}(z) \bm{P} \left( 1 + \bm{Q}^* \bm{R}(z) \bm{P} \right)^{-1} \bm{Q}^* \bm{R}(z) \bm{B} \right]
    \\
    \bm{\tilde{C}} &= (\bm{I}-\bm{\overline{A}}^L)^* \bm{C}
    \\
    \bm{R}(z; \bm{\Lambda}) &= \left(\frac{2}{\dt} \frac{1-z}{1+z} - \bm{\Lambda}\right)^{-1}
    .
  \end{align*}
\end{lemma}
%
\begin{proof}%
  Directly expanding \cref{def:generating-function} yields
  \begin{align*}
    \mathcal{K}_L(z; \bm{\overline{A}}, \bm{\overline{B}}, \bm{\overline{C}})
    &=
    \bm{\overline{C}}^* \bm{\overline{B}} + \bm{\overline{C}}^* \bm{\overline{A}} \bm{\overline{B}} z + \dots + \bm{\overline{C}}^* \bm{\overline{A}}^{L-1} \bm{\overline{B}} z^{L-1}
    \\&=
    \bm{\overline{C}}^* \left(\bm{I}-\bm{\overline{A}}^L\right) \left(\bm{I} - \bm{\overline{A}} z\right)^{-1} \bm{\overline{B}}
    \\&=
    \bm{\tilde{C}}^* \left(\bm{I} - \bm{\overline{A}} z\right)^{-1} \bm{\overline{B}}
  \end{align*}
  where \(  \bm{\tilde{C}}^* = \bm{C}^* \left(\bm{I}-\bm{\overline{A}}^L\right) \).

  We can now explicitly expand the discretized SSM matrices \( \bm{\overline{A}} \) and \( \bm{\overline{B}} \) back in terms of the original SSM parameters \( \bm{A} \) and \( \bm{B} \).
  \cref{lmm:bilinear-resolvent} provides an explicit formula, which allows further simplifying
  \begin{align*}
    \bm{\tilde{C}}^* \left(\bm{I} - \bm{\overline{A}} z\right)^{-1} \bm{\overline{B}}
    &= \frac{2}{1+z} \bm{\tilde{C}}^* \left(\frac{2}{\Delta} \frac{1-z}{1+z} - \bm{A}\right)^{-1}  \bm{B}
    \\&=
    \frac{2}{1+z} \bm{\tilde{C}}^* \left(\frac{2}{\Delta} \frac{1-z}{1+z} - \bm{\Lambda} + \bm{P}\bm{Q}^* \right)^{-1} \bm{B}
    \\&=
    \frac{2}{1+z}\left[ \bm{\tilde{C}}^* \bm{R}(z) \bm{B} - \bm{\tilde{C}}^* \bm{R}(z) \bm{P} \left( 1 + \bm{Q}^* \bm{R}(z) \bm{P} \right)^{-1} \bm{Q}^* \bm{R}(z) \bm{B} \right]
    .
  \end{align*}
  The last line applies the Woodbury Identity (\cref{prop:woodbury}) where \( \bm{R}(z) = \left(\frac{2}{\Delta} \frac{1-z}{1+z} - \bm{\Lambda}\right)^{-1} \).
\end{proof}


The previous proof used the following self-contained result to back out the original SSM matrices from the discretization.
\begin{lemma}%
  \label{lmm:bilinear-resolvent}
  Let \( \bm{\overline{A}}, \bm{\overline{B}} \) be the SSM matrices \( \bm{A}, \bm{B} \) discretized by the bilinear discretization with step size \( \dt \). Then
  \begin{align*}
    \bm{C}^*\left(\bm{I} - \bm{\overline{A}z} \right)^{-1} \bm{\overline{B}}
    =
    \frac{2\Delta}{1+z} \bm{C}^* \left[ {2 \frac{1-z}{1+z}} - \Delta \bm{A} \right]^{-1} \bm{B}
  \end{align*}
\end{lemma}
\begin{proof}%
  Recall that the bilinear discretization that we use (equation \eqref{eq:2}) is
  \begin{align*}
    \bm{\overline{A}}
    &=
    \left(\bm{I} - \frac{\Delta}{2} \bm{A}\right)^{-1} \left(\bm{I} + \frac{\Delta}{2} \bm{A}\right)
    \\
    \bm{\overline{B}} &= \left(\bm{I} - \frac{\Delta}{2} \bm{A}\right)^{-1} \Delta \bm{B}
  \end{align*}
  The result is proved algebraic manipulations.
  \begin{align*}
    \bm{C}^*\left(\bm{I} - \bm{\overline{A}} z\right)^{-1} \bm{\overline{B}}
    &=
    \bm{C}^* \left[ \left(\bm{I} - \frac{\Delta}{2} \bm{A}\right)^{-1}\left(\bm{I} - \frac{\Delta}{2} \bm{A}\right)  - \left(\bm{I} - \frac{\Delta}{2} \bm{A}\right)^{-1} \left(\bm{I} + \frac{\Delta}{2} \bm{A}\right) z \right]^{-1} \bm{\overline{B}}
    \\&=
    \bm{C}^* \left[ \left(\bm{I} - \frac{\Delta}{2} \bm{A}\right) - \left(\bm{I} + \frac{\Delta}{2} \bm{A}\right) z \right]^{-1} \left(\bm{I} - \frac{\Delta}{2} \bm{A}\right) \bm{\overline{B}}
    \\&=
    \bm{C}^* \left[ \bm{I}(1 - z) - \frac{\Delta}{2} \bm{A} (1+z) \right]^{-1} \Delta\bm{B}
    \\&=
    \frac{\Delta}{1-z} \bm{C}^* \left[ \bm{I} - \frac{\Delta \bm{A}}{2 \frac{1-z}{1+z}} \right]^{-1} \bm{B}
    \\&=
    \frac{2\Delta}{1+z} \bm{C}^* \left[ {2 \frac{1-z}{1+z}} \bm{I} - \Delta \bm{A} \right]^{-1} \bm{B}
  \end{align*}
\end{proof}

Note that in the \methodabbrv{} parameterization, instead of constantly computing \( \bm{\tilde{C}} = \left(\bm{I} - \bm{\overline{A}}^L\right)^* \bm{C} \),
we can simply reparameterize our parameters to learn \( \bm{\tilde{C}} \) directly instead of \( \bm{C} \),
saving a minor computation cost and simplifying the algorithm.

\paragraph{Reduction 3: Cauchy Kernel}
We have reduced the original problem of computing \( \bm{\overline{K}} \) to the problem of computing the SSM generating function \( \hat{\mathcal{K}}_L(\Omega; \bm{\overline{A}}, \bm{\overline{B}}, \bm{\overline{C}}) \)
in the case that \( \bm{\overline{A}} \) is a diagonal matrix.
We show that this is exactly the same as a Cauchy kernel, which is a well-studied problem with fast and stable numerical algorithms.

\begin{definition}%
  \label{def:cauchy}
  A \textbf{Cauchy matrix} or kernel on nodes \( \Omega = (\omega_i) \in \mathbbm{C}^M \) and \( \Lambda = (\lambda_j) \in \mathbbm{C}^N \) is
  \begin{align*}
    \bm{M} \in \mathbbm{C}^{M \times N} &= \bm{M}(\Omega, \Lambda) = (\bm{M}_{ij})_{i \in [M], j \in [N]} \qquad
    \bm{M}_{ij} = \frac{1}{\omega_i - \lambda_j}
    .
  \end{align*}
  The computation time of a Cauchy matrix-vector product of size \( M \times N \) is denoted by \( \mathcal{C}(M, N) \).
\end{definition}

Computing with Cauchy matrices is an extremely well-studied problem in numerical analysis,
with both fast arithmetic algorithms and fast numerical algorithms based on the famous Fast Multipole Method (FMM)
\citep{pan2001structured,pan2015transformations,pan2017fast}.
\begin{proposition}[Cauchy]%
  \label{prop:cauchy}
  A Cauchy kernel requires \( O(M+N) \) space, and operation count
  \begin{align*}
    \mathcal{C}(M, N) =
    \begin{cases}%
      O\left( MN \right)  & \text{naively} \\
      O\left( (M+N) \log^2(M+N) \right) & \text{in exact arithmetic} \\
      O\left( (M+N) \log(M+N) \log \frac{1}{\varepsilon} \right) & \text{numerically to precision \( \varepsilon \)}
      .
    \end{cases}
  \end{align*}
\end{proposition}

\begin{corollary}%
  Evaluating \( \bm{Q}^* \bm{R}(\Omega; \Lambda) \bm{P} \) (defined in \cref{lmm:resolvent-woodbury}) for any set of nodes \( \Omega \in \mathbbm{C}^L \), diagonal matrix \( \Lambda \), and vectors \( \bm{P}, \bm{Q} \) can be computed in \( \mathcal{C}(L,N) \) operations and \( O(L+N) \) space, where \( \mathcal{C}(L,N) = \tilde{O}(L+N) \) is the cost of a Cauchy matrix-vector multiplication.
\end{corollary}
\begin{proof}%
  For any fixed \( \omega \in \Omega \), we want to compute \( \sum_{j} \frac{q_j^* p_j}{\omega - \lambda_j} \). Computing this over all \( \omega_i \) is therefore exactly a Cauchy matrix-vector multiplication.
\end{proof}

This completes the proof of \cref{thm:s4-convolution}.
In \cref{alg:s4-convolution},
note that the work is dominated by Step \ref{step:cauchy},
which has a constant number of calls to a black-box Cauchy kernel, with complexity given by \cref{prop:cauchy}.



\section{Experiment Details and Full Results}
\label{sec:experiment-details}

This section contains full experimental procedures and extended results and citations for our experimental evaluation in \cref{sec:experiments}.
\cref{sec:experiment-details-benchmarking} corresponds to benchmarking results in \cref{sec:experiments-benchmark},
\cref{sec:experiment-details-lrd} corresponds to LRD experiments (LRA and Speech Commands) in \cref{sec:experiments-lrd},
and \cref{sec:experiment-details-general} corresponds to the general sequence modeling experiments (generation, image classification, forecasting) in \cref{sec:experiments-general}.

\subsection{Benchmarking}
\label{sec:experiment-details-benchmarking}

Benchmarking results from \cref{tab:ssm-benchmark} and \cref{tab:lra-benchmark} were tested on a single A100 GPU.

\paragraph{Benchmarks against LSSL}

For a given dimension \( H \), a single LSSL or \methodabbrv{} layer was constructed with \( H \) hidden features.
For LSSL, the state size \( N \) was set to \( H \) as done in \citep{gu2021lssl}.
For \methodabbrv{}, the state size \( N \) was set to parameter-match the LSSL, which was a state size of \( \frac{N}{4} \) due to differences in the parameterization.
\cref{tab:ssm-benchmark} benchmarks a single forward+backward pass of a single layer.

\paragraph{Benchmarks against Efficient Transformers}
Following \citep{tay2021long}, the Transformer models had 4 layers, hidden dimension \( 256 \) with \( 4 \) heads, query/key/value projection dimension \( 128 \), and batch size \( 32 \), for a total of roughly \( 600k \) parameters.
The \methodabbrv{} model was parameter tied while keeping the depth and hidden dimension constant (leading to a state size of \( N = 256 \)).

We note that the relative orderings of these methods can vary depending on the exact hyperparameter settings.

\subsection{Long-Range Dependencies}
\label{sec:experiment-details-lrd}
This section includes information for reproducing our experiments on the Long-Range Arena and Speech Commands long-range dependency tasks.

\paragraph{Long Range Arena}

\cref{tab:lra-full} contains extended results table with all 11 methods considered in \citep{tay2021long}.

\begin{table}[t]
  \small
  \caption{Full results for the Long Range Arena (LRA) benchmark for long-range dependencies in sequence models. (Top): Original Transformer variants in LRA. (Bottom): Other models reported in the literature.}
    \centering
    \begin{tabular}{@{}llllllll@{}}
        \toprule
        Model                 & \textsc{ListOps}  & \textsc{Text}     & \textsc{Retrieval} & \textsc{Image}    & \textsc{Pathfinder} & \textsc{Path-X} & \textsc{Avg}      \\
        \midrule
        Random                & 10.00             & 50.00             & 50.00              & 10.00             & 50.00               & 50.00           & 36.67             \\
        \midrule
        Transformer           & 36.37             & 64.27             & 57.46              & 42.44             & 71.40               & \xmark          & 53.66             \\
        Local Attention       & 15.82             & 52.98             & 53.39              & 41.46             & 66.63               & \xmark          & 46.71             \\
        Sparse Trans.         & 17.07             & 63.58             & 59.59              & 44.24             & 71.71               & \xmark          & 51.03             \\
        Longformer            & 35.63             & 62.85             & 56.89              & 42.22             & 69.71               & \xmark          & 52.88             \\
        Linformer             & 35.70             & 53.94             & 52.27              & 38.56             & 76.34               & \xmark          & 51.14             \\
        Reformer              & \underline{37.27} & 56.10             & 53.40              & 38.07             & 68.50               & \xmark          & 50.56             \\
        Sinkhorn Trans.       & 33.67             & 61.20             & 53.83              & 41.23             & 67.45               & \xmark          & 51.23             \\
        Synthesizer           & 36.99             & 61.68             & 54.67              & 41.61             & 69.45               & \xmark          & 52.40             \\
        BigBird               & 36.05             & 64.02             & 59.29              & 40.83             & 74.87               & \xmark          & 54.17             \\
        Linear Trans.         & 16.13             & \underline{65.90} & 53.09              & 42.34             & 75.30               & \xmark          & 50.46             \\
        Performer             & 18.01             & 65.40             & 53.82              & 42.77             & 77.05               & \xmark          & 51.18             \\
        \midrule
        FNet                  & 35.33             & 65.11             & 59.61              & 38.67             & \underline{77.80}   & \xmark          & 54.42             \\
        Nystr{\"o}mformer     & 37.15             & 65.52             & \underline{79.56}  & 41.58             & 70.94               & \xmark          & 57.46             \\
        Luna-256              & 37.25             & 64.57             & 79.29              & \underline{47.38} & 77.72               & \xmark          & \underline{59.37} \\
        \textbf{\methodabbrv} (original) & 58.35   & 76.02 & 87.09     & 87.26 & 86.05      & 88.10  & 80.48 \\
        \textbf{\methodabbrv} (updated)  & \textbf{59.60}   & \textbf{86.82} & \textbf{90.90}     & \textbf{88.65} & \textbf{94.20}      & \textbf{96.35}  & \textbf{86.09} \\
        \bottomrule
    \end{tabular}
    \label{tab:lra-full}
\end{table}

For the \methodabbrv{} model, hyperparameters for all datasets are reported in \cref{tab::best-hyperparameters}.
For all datasets, we used the AdamW optimizer with a constant learning rate schedule with decay on validation plateau.
However, the learning rate on HiPPO parameters (in particular \( \bm{\Lambda}, \bm{P}, \bm{Q}, \bm{B}, \bm{C}, \dt \)) were reduced to a maximum starting LR of \( 0.001 \), which improves stability since the HiPPO equation is crucial to performance.

The \methodabbrv{} state size was always fixed to \( N=64 \).

As \methodabbrv{} is a sequence-to-sequence model with output shape (batch, length, dimension) and LRA tasks are classification,
mean pooling along the length dimension was applied after the last layer.

We note that most of these results were trained for far longer than what was necessary to achieve SotA results (e.g., the \texttt{Image} task reaches SotA in 1 epoch).
Results often keep improving with longer training times.

\textbf{Updated results.}
The above hyperparameters describe the results reported in the original paper, shown in \cref{tab:lra-full}, which have since been improved.
See \cref{sec:reproduction}.

\textbf{Hardware.}
All models were run on single GPU.
Some tasks used an A100 GPU (notably, the Path-X experiments), which has a larger max memory of 40Gb.
To reproduce these on smaller GPUs, the batch size can be reduced or gradients can be accumulated for two batches.

\begin{table*}[!t]
  \caption{
    The values of the best hyperparameters found for classification datasets; LRA (Top) and images/speech (Bottom).
    LR is learning rate and WD is weight decay. BN and LN refer to Batch Normalization and Layer Normalization.
  }
  \label{tab::best-hyperparameters}
  \centering
  \resizebox{\textwidth}{!}{%
    \begin{tabular}{@{}llllllllllll@{}}
      \toprule
                                      & \textbf{Depth} & \textbf{Features \( H \)} & \textbf{Norm} & \textbf{Pre-norm} & {\bf Dropout} & {\bf LR} & {\bf Batch Size} & {\bf Epochs} & \textbf{WD} & \textbf{Patience} \\
      \midrule
      \textbf{ListOps}                & 6              & 128                       & BN            & False             & 0             & 0.01     & 100              & 50           & 0.01        & 5                 \\
      \textbf{Text}                   & 4              & 64                        & BN            & True              & 0             & 0.001    & 50               & 20           & 0           & 5                 \\
      \textbf{Retrieval}              & 6              & 256                       & BN            & True              & 0             & 0.002    & 64               & 20           & 0           & 20                \\
      \textbf{Image}                  & 6              & 512                       & LN            & False             & 0.2           & 0.004    & 50               & 200          & 0.01        & 20                \\
      \textbf{Pathfinder}             & 6              & 256                       & BN            & True              & 0.1           & 0.004    & 100              & 200          & 0           & 10                \\
      \textbf{Path-X}                 & 6              & 256                       & BN            & True              & 0.0           & 0.0005   & 32               & 100          & 0           & 20                \\
      \midrule
      \textbf{CIFAR-10}               & 6              & 1024                      & LN            & False             & 0.25          & 0.01     & 50               & 200          & 0.01        & 20                \\
      \midrule
      \textbf{Speech Commands (MFCC)} & 4              & 256                       & LN            & False             & 0.2           & 0.01     & 100              & 50           & 0           & 5                 \\
      \textbf{Speech Commands (Raw)}  & 6              & 128                       & BN            & True              & 0.1           & 0.01     & 20               & 150          & 0           & 10                \\
      \bottomrule
    \end{tabular}%
  }
\end{table*}


\paragraph{Speech Commands}
We provide details of sweeps run for baseline methods run by us---numbers for all others method are taken from \citet{gu2021lssl}. The best hyperparameters used for \methodabbrv{} are included in Table~\ref{tab::best-hyperparameters}.

\textit{Transformer~\citep{vaswani2017attention}} For MFCC, we swept the number of model layers $\{2, 4\}$, dropout $\{0, 0.1\}$ and learning rates $\{0.001, 0.0005\}$. We used $8$ attention heads, model dimension $128$, prenorm, positional encodings, and trained for $150$ epochs with a batch size of $100$. For Raw, the Transformer model's memory usage made training impossible.

\textit{Performer~\citep{choromanski2020rethinking}} For MFCC, we swept the number of model layers $\{2, 4\}$, dropout $\{0, 0.1\}$ and learning rates $\{0.001, 0.0005\}$. We used $8$ attention heads, model dimension $128$, prenorm, positional encodings, and trained for $150$ epochs with a batch size of $100$. For Raw, we used a model dimension of $128$, $4$ attention heads, prenorm, and a batch size of $16$. We reduced the number of model layers to $4$, so the model would fit on the single GPU. We trained for $100$ epochs with a learning rate of $0.001$ and no dropout.

\textit{ExpRNN~\citep{lezcano2019cheap}} For MFCC, we swept hidden sizes $\{256, 512\}$ and learning rates $\{0.001, 0.002, 0.0005\}$. Training was run for $200$ epochs, with a single layer model using a batch size of $100$. For Raw, we swept hidden sizes $\{32, 64\}$ and learning rates $\{0.001, 0.0005\}$ (however, ExpRNN failed to learn).

\textit{LipschitzRNN~\citep{erichson2021lipschitz}} For MFCC, we swept hidden sizes $\{256, 512\}$ and learning rates $\{0.001, 0.002, 0.0005\}$. Training was run for $150$ epochs, with a single layer model using a batch size of $100$. For Raw, we found that LipschitzRNN was too slow to train on a single GPU (requiring a full day for $1$ epoch of training alone).

\textit{WaveGAN Discriminator~\citep{Donahue2019AdversarialAS}}
The WaveGAN-D in \cref{tab:sc} is actually our improved version of the discriminator network from the recent WaveGAN model for speech~\citep{Donahue2019AdversarialAS}.
This CNN actually did not work well out-of-the-box, and we added several features to help it perform better.
The final model is highly specialized compared to our model, and includes:
\begin{itemize}%
  \item Downsampling or pooling between layers, induced by strided convolutions, that decrease the sequence length between layers.
  \item A global fully-connected output layer; thus the model only works for one input sequence length and does not work on MFCC features or the frequency-shift setting in \cref{tab:sc}.
  \item Batch Normalization is essential, whereas \methodabbrv{} works equally well with either Batch Normalization or Layer Normalization.
  \item Almost \( 90\times \) as many parameters as the \methodabbrv{} model ($26.3$M vs. $0.3$M).
\end{itemize}

\subsection{General Sequence Modeling}
\label{sec:experiment-details-general}

This subsection corresponds to the experiments in \cref{sec:experiments-general}.
Because of the number of experiments in this section,
we use subsubsection dividers for different tasks to make it easier to follow:
CIFAR-10 density estimation (\cref{sec:experiment-details-general-cifargen}),
WikiText-103 language modeling (\cref{sec:experiment-details-general-wt103}),
autoregressive generation (\cref{sec:experiment-details-general-speed}),
sequential image classification (\cref{sec:experiment-details-general-image}),
and time-series forecasting (\cref{sec:experiment-details-general-informer}).

\subsubsection{CIFAR Density Estimation}
\label{sec:experiment-details-general-cifargen}

This task used a different backbone than the rest of our experiments.
We used blocks of alternating \methodabbrv{} layers and position-wise feed-forward layers (in the style of Transformer blocks).
Each feed-forward intermediate dimension was set to \( 2\times \) the hidden size of the incoming \methodabbrv{} layer.
Similar to \citet{salimans2017pixelcnn++}, we used a UNet-style backbone consisting of \( B \) identical blocks followed by a downsampling layer.
The downsampling rates were \( 3, 4, 4 \) (the 3 chosen because the sequence consists of RGB pixels).
The base model had \( B=8 \) with starting hidden dimension 128,
while the large model had \( B=16 \) with starting hidden dimension 192.

We experimented with both the mixture of logistics from \citep{salimans2017pixelcnn++} as well as a simpler 256-way categorical loss.
We found they were pretty close and ended up using the simpler softmax loss along with using input embeddings.

We used the LAMB optimizer with learning rate 0.005.
The base model had no dropout, while the large model had dropout 0.1 before the linear layers inside the \methodabbrv{} and FF blocks.


\subsubsection{WikiText-103 Language Modeling}
\label{sec:experiment-details-general-wt103}

The RNN baselines included in \cref{tab:wt103} are the
AWD-QRNN~\citep{merity2018scalable}, an efficient linear gated RNN,
and the LSTM + Cache + Hebbian + MbPA \citep{rae2018fast}, the best performing pure RNN in the literature.
The CNN baselines are
the CNN with GLU activations~\citep{dauphin2017language},
the TrellisNet~\citep{trellisnet},
Dynamic Convolutions~\citep{wu2019pay},
and TaLK Convolutions~\citep{lioutas2020time}.

The Transformer baseline is \citep{baevski2018adaptive},
which uses Adaptive Inputs with a tied Adaptive Softmax.
This model is a standard high-performing Transformer baseline on this benchmark,
used for example by \citet{lioutas2020time} and many more.

Our \methodabbrv{} model uses the same Transformer backbone as in \citep{baevski2018adaptive}.
The model consists of 16 blocks of \methodabbrv{} layers alternated with position-wise feedforward layers, with a feature dimension of 1024.
Because our \methodabbrv{} layer has around 1/4 the number of parameters as a self-attention layer with the same dimension, we made two modifications to match the parameter count better:
(i) we used a GLU activation after the \methodabbrv{} linear layer (\cref{sec:s4-architecture})
(ii) we used two \methodabbrv{} layers per block.
Blocks use Layer Normalization in the pre-norm position.
The embedding and softmax layers were the Adaptive Embedding from \citep{baevski2018adaptive} with standard cutoffs 20000, 40000, 200000.

Evaluation was performed similarly to the basic setting in \citep{baevski2018adaptive}, Table 5,
which uses sliding non-overlapping windows.
Other settings are reported in \citep{baevski2018adaptive} that include more context at training and evaluation time and improves the score.
Because such evaluation protocols are orthogonal to the basic model, we do not consider them and report the base score from \citep{baevski2018adaptive} Table 5.

Instead of SGD+Momentum with multiple cosine learning rate annealing cycles,
our \methodabbrv{} model was trained with the simpler AdamW optimizer with a single cosine learning rate cycle with a maximum of 800000 steps.
The initial learning rate was set to 0.0005.
We used 8 A100 GPUs with a batch size of 1 per gpu and context size 8192.
We used no gradient clipping and a weight decay of 0.1.
Unlike \citep{baevski2018adaptive} which specified different dropout rates for different parameters,
we used a constant dropout rate of 0.25 throughout the network, including before every linear layer and on the residual branches.


\subsubsection{Autoregressive Generation Speed}
\label{sec:experiment-details-general-speed}

\paragraph{Protocol.}
To account for different model sizes and memory requirements for each method,
we benchmark generation speed by throughput,
measured in images per second (\cref{tab:cifar-generation}) or tokens per second (\cref{tab:wt103}).
Each model generates images on a single \( A100 \) GPU,
maximizing batch size to fit in memory.
(For CIFAR-10 generation we limited memory to 16Gb, to be more comparable to the Transformer and Linear Transformer results reported from \citep{katharopoulos2020transformers}.)

\paragraph{Baselines.}
The Transformer and Linear Transformer baselines reported in \cref{tab:cifar-generation} are the results reported directly from \citet{katharopoulos2020transformers}.
Note that the Transformer number is the one in their Appendix, which implements the optimized cached implementation of self-attention.

For all other baseline models, we used open source implementations of the models to benchmark generation speed.
For the PixelCNN++, we used the fast cached version by \citet{ramachandran2017fast},
which sped up generation by orders of magnitude from the naive implementation.
This code was only available in TensorFlow, which may have slight differences compared to the rest of the baselines which were implemented in PyTorch.

We were unable to run the Sparse Transformer~\citep{child2019generating} model due to issues with their custom CUDA implementation of the sparse attention kernel, which we were unable to resolve.

The Transformer baseline from \cref{tab:wt103} was run using a modified GPT-2 backbone from the HuggingFace repository, configured to recreate the architecture reported in \citep{baevski2018adaptive}.
These numbers are actually slightly favorable to the baseline, as we did not include the timing of the embedding or softmax layers, whereas the number reported for \methodabbrv{} is the full model.

\subsubsection{Pixel-Level Sequential Image Classification}
\label{sec:experiment-details-general-image}

Our models were trained with the AdamW optimizer for up to 200 epochs.
Hyperparameters for the CIFAR-10 model is reported in \cref{tab::best-hyperparameters}.

For our comparisons against ResNet-18, the main differences between the base models are that \methodabbrv{} uses LayerNorm by default while ResNet uses BatchNorm.
The last ablation in \cref{sec:experiments-general} swaps the normalization type,
using BatchNorm for \methodabbrv{} and LayerNorm for ResNet,
to ablate this architectural difference.
The experiments with augmentation take the base model and train with mild data augmentation: horizontal flips and random crops (with symmetric padding).

\begin{table}[t]
  \small
  \centering
  \captionsetup{type=table}
  \caption{
    (\textbf{Pixel-level image classification.})
    Citations refer to the original model; additional citation indicates work from which this baseline is reported.
  }
  \begin{tabular}{@{}llll@{}}
    \toprule
    Model                                                      & \textsc{sMNIST}   & \textsc{pMNIST}   & \textsc{sCIFAR}   \\
    \midrule
    Transformer~\citep{vaswani2017attention,trinh2018learning} & 98.9              & 97.9              & 62.2              \\
    \midrule
    CKConv~\citep{romero2021ckconv}                            & 99.32             & \underline{98.54} & 63.74             \\
    TrellisNet~\citep{trellisnet}                              & 99.20             & 98.13             & 73.42             \\
    TCN~\citep{bai2018empirical}                               & 99.0              & 97.2              & -                 \\
    \midrule
    LSTM~\citep{lstm,gu2020improving}                          & 98.9              & 95.11             & 63.01             \\
    r-LSTM ~\citep{trinh2018learning}                          & 98.4              & 95.2              & 72.2              \\
    Dilated GRU~\citep{chang2017dilated}                       & 99.0              & 94.6              & -                 \\
    Dilated RNN~\citep{chang2017dilated}                       & 98.0              & 96.1              & -                 \\
    IndRNN~\citep{indrnn}                                      & 99.0              & 96.0              & -                 \\
    expRNN~\citep{lezcano2019cheap}                            & 98.7              & 96.6              & -                 \\
    UR-LSTM                                                    & 99.28             & 96.96             & 71.00             \\
    UR-GRU~\citep{gu2020improving}                             & 99.27             & 96.51             & \underline{74.4}  \\
    LMU~\citep{voelker2019legendre}                            & -                 & 97.15             & -                 \\
    HiPPO-RNN~\citep{gu2020hippo}                              & 98.9              & 98.3              & 61.1              \\
    UNIcoRNN~\citep{rusch2021unicornn}                         & -                 & 98.4              & -                 \\
    LMUFFT~\citep{chilkuri2021parallelizing}                   & -                 & 98.49             & -                 \\
    LipschitzRNN~\citep{erichson2021lipschitz}                 & \underline{99.4}  & 96.3              & 64.2              \\
    \midrule
    \textbf{\methodabbrv}                                                & \textbf{99.63}    & \textbf{98.70} & \textbf{91.13}    \\
    \bottomrule
  \end{tabular}
  \label{tab:image-full}
\end{table}


\subsubsection{Time Series Forecasting compared to Informer}
\label{sec:experiment-details-general-informer}

We include a simple figure (\cref{fig:s4-architecture}) contrasting the architecture of \methodabbrv{} against that of the Informer \citep{haoyietal-informer-2021}.

In \cref{fig:s4-architecture},
the goal is to forecast a contiguous range of future predictions (Green, length \( F \) )
given a range of past context (Blue, length \( C \) ).
We simply concatenate the entire context with a sequence of masks set to the length of the forecast window.
This input is a single sequence of length \( C+F \) that is run through the same simple deep \methodabbrv{} model used throughout this work,
which maps to an output of length \( C+F \) .
We then use just the last \( F \) outputs as the forecasted predictions.


\begin{figure}[t]
    \centering
    \begin{subfigure}{\linewidth}%
      \centering
      \includegraphics[width=\linewidth]{figs/s4_forecasting.pdf}
    \end{subfigure}
    \caption{Comparison of \methodabbrv{} and specialized time-series models for forecasting tasks. (\textit{Top Left}) The forecasting task involves predicting future values of a time-series given past context. (\textit{Bottom Left}) We perform simple forecasting using a sequence model such as \methodabbrv{} as a black box. (\textit{Right}) Informer uses an encoder-decoder architecture designed specifically for forecasting problems involving a customized attention module (figure taken from~\citet{haoyietal-informer-2021}).}
    \label{fig:s4-architecture}
\end{figure}

\cref{tab:informer-s,tab:informer-m} contain full results on all 50 settings considered by \citet{haoyietal-informer-2021}.
\methodabbrv{} sets the best results on 40 out of 50 of these settings.

\begin{table*}[t]
\centering
\fontsize{9pt}{9pt}\selectfont
\centering
\resizebox{\linewidth}{!}{
\begin{tabular}{c|c|c|c|c|c|c|c|c|c|c|c}
\toprule[1.0pt]
\multicolumn{2}{c|}{Methods}              & \textbf{\methodabbrv} & {Informer}                     & {Informer$^{\dag}$}            & {LogTrans}       & {Reformer}   & {LSTMa}      & {DeepAR}     & {ARIMA}                 & {Prophet}    \\
\midrule[0.5pt]
\multicolumn{2}{c|}{Metric}               & MSE~~MAE              & MSE~~MAE                       & MSE~~MAE                       & MSE~~MAE         & MSE~~MAE     & MSE~~MAE     & MSE~~MAE     & MSE~~MAE                & MSE~~MAE     \\
\midrule[1.0pt]
\multirow{5}{*}{\rotatebox{90}{ETTh$_1$}} & 24                    & \textbf{0.061}~~\textbf{0.191} & 0.098~~0.247                   & {0.092}~~{0.246} & 0.103~~0.259 & 0.222~~0.389 & 0.114~~0.272 & 0.107~~0.280            & 0.108~~0.284  & 0.115~~0.275 \\
                                          & 48                    & \textbf{0.079}~~\textbf{0.220} & {0.158}~~{0.319}               & 0.161~~0.322     & 0.167~~0.328 & 0.284~~0.445 & 0.193~~0.358 & 0.162~~0.327            & 0.175~~0.424  & 0.168~~0.330 \\
                                          & 168                   & \textbf{0.104}~~\textbf{0.258} & {0.183}~~{0.346}               & 0.187~~0.355     & 0.207~~0.375 & 1.522~~1.191 & 0.236~~0.392 & 0.239~~0.422            & 0.396~~0.504  & 1.224~~0.763 \\
                                          & 336                   & \textbf{0.080}~~\textbf{0.229} & 0.222~~0.387                   & {0.215}~~{0.369} & 0.230~~0.398 & 1.860~~1.124 & 0.590~~0.698 & 0.445~~0.552            & 0.468~~0.593  & 1.549~~1.820 \\
                                          & 720                   & \textbf{0.116}~~\textbf{0.271} & 0.269~~0.435                   & {0.257}~~{0.421} & 0.273~~0.463 & 2.112~~1.436 & 0.683~~0.768 & 0.658~~0.707            & 0.659~~0.766  & 2.735~~3.253 \\
\midrule[0.5pt]
\multirow{5}{*}{\rotatebox{90}{ETTh$_2$}} & 24                    & 0.095~~0.234                   & \textbf{0.093}~~\textbf{0.240} & 0.099~~0.241     & 0.102~~0.255 & 0.263~~0.437 & 0.155~~0.307 & 0.098~~0.263            & 3.554~~0.445  & 0.199~~0.381 \\
                                          & 48                    & 0.191~~0.346                   & \textbf{0.155}~~\textbf{0.314} & 0.159~~0.317     & 0.169~~0.348 & 0.458~~0.545 & 0.190~~0.348 & 0.163~~0.341            & 3.190~~0.474  & 0.304~~0.462 \\
                                          & 168                   & \textbf{0.167}~~\textbf{0.333} & {0.232}~~{0.389}               & 0.235~~0.390     & 0.246~~0.422 & 1.029~~0.879 & 0.385~~0.514 & 0.255~~0.414            & 2.800~~0.595  & 2.145~~1.068 \\
                                          & 336                   & \textbf{0.189}~~\textbf{0.361} & 0.263~~{0.417}                 & {0.258}~~0.423   & 0.267~~0.437 & 1.668~~1.228 & 0.558~~0.606 & 0.604~~0.607            & 2.753~~0.738  & 2.096~~2.543 \\
                                          & 720                   & \textbf{0.187}~~\textbf{0.358} & {0.277}~~{ 0.431}              & 0.285~~0.442     & 0.303~~0.493 & 2.030~~1.721 & 0.640~~0.681 & 0.429~~0.580            & 2.878~~1.044  & 3.355~~4.664 \\
\midrule[0.5pt]
\multirow{5}{*}{\rotatebox{90}{ETTm$_1$}} & 24                    & \textbf{0.024}~~\textbf{0.117} & {0.030}~~{0.137}               & 0.034~~0.160     & 0.065~~0.202 & 0.095~~0.228 & 0.121~~0.233 & 0.091~~0.243            & 0.090~~0.206  & 0.120~~0.290 \\
                                          & 48                    & \textbf{0.051}~~\textbf{0.174} & 0.069~~0.203                   & {0.066}~~{0.194} & 0.078~~0.220 & 0.249~~0.390 & 0.305~~0.411 & 0.219~~0.362            & 0.179~~0.306  & 0.133~~0.305 \\
                                          & 96                    & \textbf{0.086}~~\textbf{0.229} & 0.194~~{0.372}                 & {0.187}~~0.384   & 0.199~~0.386 & 0.920~~0.767 & 0.287~~0.420 & 0.364~~0.496            & 0.272~~0.399  & 0.194~~0.396 \\
                                          & 288                   & \textbf{0.160}~~\textbf{0.327} & {0.401}~~0.554                 & 0.409~~{0.548}   & 0.411~~0.572 & 1.108~~1.245 & 0.524~~0.584 & 0.948~~0.795            & 0.462~~0.558  & 0.452~~0.574 \\
                                          & 672                   & \textbf{0.292}~~\textbf{0.466} & {0.512}~~{0.644}               & 0.519~~0.665     & 0.598~~0.702 & 1.793~~1.528 & 1.064~~0.873 & 2.437~~1.352            & 0.639~~0.697  & 2.747~~1.174 \\
\midrule[0.5pt]
\multirow{5}{*}{\rotatebox{90}{Weather}}  & 24                    & 0.125~~0.254                   & \textbf{0.117}~~\textbf{0.251} & 0.119~~0.256     & 0.136~~0.279 & 0.231~~0.401 & 0.131~~0.254 & 0.128~~0.274            & 0.219~~0.355  & 0.302~~0.433 \\
                                          & 48                    & 0.181~~\textbf{0.305}          & \textbf{0.178}~~0.318          & 0.185~~0.316     & 0.206~~0.356 & 0.328~~0.423 & 0.190~~0.334 & 0.203~~0.353            & 0.273~~0.409  & 0.445~~0.536 \\
                                          & 168                   & \textbf{0.198}~~\textbf{0.333} & {0.266}~~{0.398}               & 0.269~~0.404     & 0.309~~0.439 & 0.654~~0.634 & 0.341~~0.448 & 0.293~~0.451            & 0.503~~0.599  & 2.441~~1.142 \\
                                          & 336                   & 0.300~~0.417                   & \textbf{0.297}~~\textbf{0.416} & 0.310~~0.422     & 0.359~~0.484 & 1.792~~1.093 & 0.456~~0.554 & 0.585~~0.644            & 0.728~~0.730  & 1.987~~2.468 \\
                                          & 720                   & \textbf{0.245}~~\textbf{0.375} & {0.359}~~{0.466}               & 0.361~~0.471     & 0.388~~0.499 & 2.087~~1.534 & 0.866~~0.809 & 0.499~~0.596            & 1.062~~0.943  & 3.859~~1.144 \\
\midrule[0.5pt]
\multirow{5}{*}{\rotatebox{90}{ECL}}      & 48                    & 0.222~~\textbf{0.350}          & 0.239~~0.359                   & 0.238~~0.368     & 0.280~~0.429 & 0.971~~0.884 & 0.493~~0.539 & \textbf{0.204}~~{0.357} & 0.879~~0.764  & 0.524~~0.595 \\
                                          & 168                   & 0.331~~\textbf{0.421}          & 0.447~~0.503                   & 0.442~~0.514     & 0.454~~0.529 & 1.671~~1.587 & 0.723~~0.655 & \textbf{0.315}~~{0.436} & 1.032~~0.833  & 2.725~~1.273 \\
                                          & 336                   & \textbf{0.328}~~\textbf{0.422} & 0.489~~0.528                   & 0.501~~0.552     & 0.514~~0.563 & 3.528~~2.196 & 1.212~~0.898 & {0.414}~~{0.519}        & 1.136~~0.876  & 2.246~~3.077 \\
                                          & 720                   & \textbf{0.428}~~\textbf{0.494} & {0.540}~~{0.571}               & 0.543~~0.578     & 0.558~~0.609 & 4.891~~4.047 & 1.511~~0.966 & 0.563~~0.595            & 1.251~~0.933  & 4.243~~1.415 \\
                                          & 960                   & \textbf{0.432}~~\textbf{0.497} & {0.582}~~{0.608}               & 0.594~~0.638     & 0.624~~0.645 & 7.019~~5.105 & 1.545~~1.006 & 0.657~~0.683            & 1.370~~0.982  & 6.901~~4.264 \\

\midrule[1.0pt]
\multicolumn{2}{c|}{Count}                & {22}                  & {5}                            & {0}                            & {0}              & {0}          & {0}          & {2}          & {0}                     & {0}          \\
\bottomrule[1.0pt]

\end{tabular}%
}
\caption{Univariate long sequence time-series forecasting results on four datasets (five cases).}
\label{tab:informer-s}
\end{table*}


\begin{table*}[t]
\centering
\fontsize{9pt}{9pt}\selectfont
\resizebox{\linewidth}{!}{
\begin{tabular}{c|c|cc|cc|cc|cc|cc|cc|cc}
\toprule[1.0pt]
\multicolumn{2}{c}{Methods}               & \multicolumn{2}{|c}{\textbf{\methodabbrv}} & \multicolumn{2}{|c}{Informer} & \multicolumn{2}{|c}{Informer$^{\dag}$} & \multicolumn{2}{|c}{LogTrans} & \multicolumn{2}{|c}{Reformer} & \multicolumn{2}{|c}{LSTMa} & \multicolumn{2}{|c}{LSTnet} \\
\midrule[0.5pt]
\multicolumn{2}{c|}{Metric}               & MSE                                        & MAE                           & MSE                                    & MAE                           & MSE                           & MAE                        & MSE                          & MAE     & MSE   & MAE   & MSE   & MAE   & MSE   & MAE     \\
\midrule[1.0pt]
\multirow{5}{*}{\rotatebox{90}{ETTh$_1$}} & 24                                         & \textbf{0.525}                & \textbf{0.542}                         & {0.577}                       & {0.549}                       & 0.620                      & 0.577                        & 0.686   & 0.604 & 0.991 & 0.754 & 0.650 & 0.624 & 1.293    & 0.901 \\
                                          & 48                                         & \textbf{0.641}                & \textbf{0.615}                         & {0.685}                       & {0.625}                       & 0.692                      & 0.671                        & 0.766   & 0.757 & 1.313 & 0.906 & 0.702 & 0.675 & 1.456    & 0.960 \\
                                          & 168                                        & 0.980                         & 0.779                                  & \textbf{0.931}                & \textbf{0.752}                & 0.947                      & 0.797                        & 1.002   & 0.846 & 1.824 & 1.138 & 1.212 & 0.867 & 1.997    & 1.214 \\
                                          & 336                                        & 1.407                         & 0.910                                  & 1.128                         & 0.873                         & \textbf{1.094}             & \textbf{0.813}               & 1.362   & 0.952 & 2.117 & 1.280 & 1.424 & 0.994 & 2.655    & 1.369 \\
                                          & 720                                        & \textbf{1.162}                & \textbf{0.842}                         & {1.215}                       & {0.896}                       & 1.241                      & 0.917                        & 1.397   & 1.291 & 2.415 & 1.520 & 1.960 & 1.322 & 2.143    & 1.380 \\
\midrule[0.5pt]
\multirow{5}{*}{\rotatebox{90}{ETTh$_2$}} & 24                                         & 0.871                         & 0.736                                  & \textbf{0.720}                & \textbf{0.665}                & 0.753                      & 0.727                        & 0.828   & 0.750 & 1.531 & 1.613 & 1.143 & 0.813 & 2.742    & 1.457 \\
                                          & 48                                         & \textbf{1.240}                & \textbf{0.867}                         & {1.457}                       & {1.001}                       & 1.461                      & 1.077                        & 1.806   & 1.034 & 1.871 & 1.735 & 1.671 & 1.221 & 3.567    & 1.687 \\
                                          & 168                                        & \textbf{2.580}                & \textbf{1.255}                         & 3.489                         & {1.515}                       & 3.485                      & 1.612                        & 4.070   & 1.681 & 4.660 & 1.846 & 4.117 & 1.674 & {3.242}  & 2.513 \\
                                          & 336                                        & \textbf{1.980}                & \textbf{1.128}                         & 2.723                         & 1.340                         & 2.626                      & {1.285}                      & 3.875   & 1.763 & 4.028 & 1.688 & 3.434 & 1.549 & {2.544}  & 2.591 \\
                                          & 720                                        & \textbf{2.650}                & \textbf{1.340}                         & {3.467}                       & {1.473}                       & 3.548                      & 1.495                        & 3.913   & 1.552 & 5.381 & 2.015 & 3.963 & 1.788 & 4.625    & 3.709 \\
\midrule[0.5pt]
\multirow{5}{*}{\rotatebox{90}{ETTm$_1$}} & 24                                         & 0.426                         & 0.487                                  & 0.323                         & \textbf{0.369}                & \textbf{0.306}             & 0.371                        & 0.419   & 0.412 & 0.724 & 0.607 & 0.621 & 0.629 & 1.968    & 1.170 \\
                                          & 48                                         & 0.580                         & 0.565                                  & 0.494                         & 0.503                         & \textbf{0.465}             & \textbf{0.470}               & 0.507   & 0.583 & 1.098 & 0.777 & 1.392 & 0.939 & 1.999    & 1.215 \\
                                          & 96                                         & 0.699                         & 0.649                                  & \textbf{0.678}                & 0.614                         & 0.681                      & \textbf{0.612}               & 0.768   & 0.792 & 1.433 & 0.945 & 1.339 & 0.913 & 2.762    & 1.542 \\
                                          & 288                                        & \textbf{0.824}                & \textbf{0.674}                         & {1.056}                       & {0.786}                       & 1.162                      & 0.879                        & 1.462   & 1.320 & 1.820 & 1.094 & 1.740 & 1.124 & 1.257    & 2.076 \\
                                          & 672                                        & \textbf{0.846}                & \textbf{0.709}                         & {1.192}                       & {0.926}                       & 1.231                      & 1.103                        & 1.669   & 1.461 & 2.187 & 1.232 & 2.736 & 1.555 & 1.917    & 2.941 \\
\midrule[0.5pt]
\multirow{5}{*}{\rotatebox{90}{Weather}}  & 24                                         & \textbf{0.334}                & 0.385                                  & {0.335}                       & \textbf{0.381}                & 0.349                      & 0.397                        & 0.435   & 0.477 & 0.655 & 0.583 & 0.546 & 0.570 & 0.615    & 0.545 \\
                                          & 48                                         & 0.406                         & 0.444                                  & 0.395                         & 0.459                         & \textbf{0.386}             & \textbf{0.433}               & 0.426   & 0.495 & 0.729 & 0.666 & 0.829 & 0.677 & 0.660    & 0.589 \\
                                          & 168                                        & \textbf{0.525}                & \textbf{0.527}                         & {0.608}                       & {0.567}                       & 0.613                      & 0.582                        & 0.727   & 0.671 & 1.318 & 0.855 & 1.038 & 0.835 & 0.748    & 0.647 \\
                                          & 336                                        & \textbf{0.531}                & \textbf{0.539}                         & {0.702}                       & {0.620}                       & 0.707                      & 0.634                        & 0.754   & 0.670 & 1.930 & 1.167 & 1.657 & 1.059 & 0.782    & 0.683 \\
                                          & 720                                        & \textbf{0.578}                & \textbf{0.578}                         & {0.831}                       & {0.731}                       & 0.834                      & 0.741                        & 0.885   & 0.773 & 2.726 & 1.575 & 1.536 & 1.109 & 0.851    & 0.757 \\
\midrule[0.5pt]
\multirow{5}{*}{\rotatebox{90}{ECL}}      & 48                                         & \textbf{0.255}                & \textbf{0.352}                         & 0.344                         & {0.393}                       & 0.334                      & 0.399                        & 0.355   & 0.418 & 1.404 & 0.999 & 0.486 & 0.572 & 0.369    & 0.445 \\
                                          & 168                                        & \textbf{0.283}                & \textbf{0.373}                         & 0.368                         & 0.424                         & {0.353}                    & {0.420}                      & 0.368   & 0.432 & 1.515 & 1.069 & 0.574 & 0.602 & 0.394    & 0.476 \\
                                          & 336                                        & \textbf{0.292}                & \textbf{0.382}                         & 0.381                         & {0.431}                       & 0.381                      & 0.439                        & {0.373} & 0.439 & 1.601 & 1.104 & 0.886 & 0.795 & 0.419    & 0.477 \\
                                          & 720                                        & \textbf{0.289}                & \textbf{0.377}                         & 0.406                         & 0.443                         & {0.391}                    & {0.438}                      & 0.409   & 0.454 & 2.009 & 1.170 & 1.676 & 1.095 & 0.556    & 0.565 \\
                                          & 960                                        & \textbf{0.299}                & \textbf{0.387}                         & {0.460}                       & {0.548}                       & 0.492                      & 0.550                        & 0.477   & 0.589 & 2.141 & 1.387 & 1.591 & 1.128 & 0.605    & 0.599 \\
\midrule[1.0pt]
\multicolumn{2}{c}{Count}                 & \multicolumn{2}{|c}{18}                    & \multicolumn{2}{|c}{5}        & \multicolumn{2}{|c}{6}                 & \multicolumn{2}{|c}{0}        & \multicolumn{2}{|c}{0}        & \multicolumn{2}{|c}{0}     & \multicolumn{2}{|c}{0}      \\
\bottomrule[1.0pt]

\end{tabular}%
}
\caption{Multivariate long sequence time-series forecasting results on four datasets (five cases).}
\label{tab:informer-m}
\end{table*}


\subsection{Visualizations}
We visualize the convolutional filter $\bar{K}$ learned by \methodabbrv{} for the Pathfinder and CIFAR-10 tasks in \cref{fig:pathfinder-all-conv-filters}.

\begin{figure}
    \centering
    \begin{subfigure}{\linewidth}
        \includegraphics[width=\linewidth]{figs/pathfinder_filters_layer_0_trunc.png}
    \end{subfigure}
    \begin{subfigure}{\linewidth}
        \includegraphics[width=\linewidth]{figs/pathfinder_filters_layer_5_trunc.png}
    \end{subfigure}
    \label{fig:pathfinder-all-conv-filters}
    \caption{({\bf Convolutional filters on Pathfinder}) A random selection of filters learned by \methodabbrv{} in the first layer (top 2 rows) and last layer (bottom 2 rows) of the best model.}
\end{figure}

\subsection{Reproduction}
\label{sec:reproduction}

Since the first version of this paper, several experiments have been updated. Please read the corresponding paragraph below before citing LRA or SC results.

\paragraph{Long Range Arena}

Follow-ups to this paper expanded the theoretical understanding of S4 while improving some results.
The results reported in \cref{tab:lra} have been updated to results from the papers \citep{gu2022s4d,gu2022hippo}.
More specifically, the method S4-LegS in those works refers to the \emph{same model} presented in this paper, with the ``-LegS'' suffix referring to the initialization defined in equation \eqref{eq:hippo}. As such, results from the original \cref{tab:lra} have been directly updated.

The updated results have only minor hyperparameter changes compared to the original results. The original results and hyperparameters are shown in \cref{tab:lra-full} (\cref{sec:experiment-details-lrd}).
Appendix B of \citep{gu2022s4d} describes the changes in hyperparameters, which are also documented from the experiment configuration files in the publically available code at \url{https://github.com/HazyResearch/state-spaces}.

\paragraph{Speech Commands}

The Speech Commands (SC) dataset~\citep{Warden2018SpeechCA} is originally a 35-class dataset of spoken English words.
However, this paper was part of a line of work starting with \citet{kidger2020neural} that has used a smaller 10-class subset of SC \citep{kidger2020neural,romero2021ckconv,gu2021lssl,romero2022flexconv}.
\emph{In an effort to avoid dataset fragmentation in the literature, we have since moved to the original dataset.}
We are now calling this 10-class subset \textbf{SC10} to distinguish it from the full 35-class \textbf{SC} dataset.
To cite S4 as a baseline for Speech Commands, please use Table 11 from \citep{gu2022s4d} instead of \cref{tab:sc} from this paper.
In addition to using the full SC dataset, it also provides a number of much stronger baselines than the ones used in this work.


\paragraph{WikiText-103}

The original version of this paper used an S4 model with batch size \( 8 \), context size \( 1024 \) which achieved a validation perplexity of 20.88 and test perplexity of 21.28.
It was later retrained with a batch size of \( 1 \) and context size \( 8192 \) which achieved a validation perplexity of 19.69 and test perplexity of 20.95, and a model checkpoint is available in the public repository.
The rest of the model is essentially identical, so the results from the original table have been updated.


\end{document}
