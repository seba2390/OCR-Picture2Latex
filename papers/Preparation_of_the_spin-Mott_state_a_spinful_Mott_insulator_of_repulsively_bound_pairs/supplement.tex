\newcounter{comment}
\newcommand{\comment}[2][]{\todo[color=red!100!green!33, #1]{#2}}
\definecolor{myellow}{rgb}{1., 1., 0.6}
\newcommand{\note}[2][]{\todo[color=myellow, #1]{#2}}
\newcommand{\newmat}[1]{\textcolor{red}{#1}}
\newcommand{\rmtxt}[1]{\st{#1}}


%\begin{CJK*}{UTF8}{gbsn}
%
%
%\author{Julius de Hond}
%\author{Jinggang Xiang (项晶罡)}
%\author{Woo Chang Chung}
%\altaffiliation{Present address: ColdQuanta Inc., Boulder, CO, United States of America}
%\author{Enid Cruz-Col\'{o}n}
%\author{Wenlan Chen}
%\altaffiliation{Present address: Department of Physics and State Key Laboratory of Low Dimensional Quantum Physics, Tsinghua University, and Frontier Science Center for Quantum Information, Beijing, 100084, China}
%\author{William Cody Burton}
%\altaffiliation{Present address: Honeywell Quantum Solutions, Broomfield, CO, United States of America}
%\author{Colin Kennedy}
%\altaffiliation{Present address: Honeywell Quantum Solutions, Broomfield, CO, United States of America}
%\author{Wolfgang Ketterle}
%\affiliation{Research Laboratory of Electronics, MIT-Harvard Center for Ultracold Atoms, Department of Physics, Massachusetts Institute of Technology, Cambridge, Massachusetts 02139, USA}
%
%\title{Supplemental Material for \\ Preparation of the spin-Mott state: a spinful Mott insulator of repulsively bound pairs}
%
%\maketitle
%\end{CJK*}
%
%\setcounter{equation}{0}
%\setcounter{figure}{0}
%\setcounter{table}{0}
%\setcounter{page}{1}
%\makeatletter
%\renewcommand{\theequation}{S\arabic{equation}}
%\renewcommand{\thefigure}{S\arabic{figure}}
%\renewcommand{\bibnumfmt}[1]{[S#1]}
%%\renewcommand{\citenumfont}[1]{S#1}

\section*{Supplemental Material}

\subsection*{Experimental considerations}
We create a spin-dependent lattice using a $810$-nm wavelength laser. The laser has a detuning from the $D_1$ and $D_2$ lines of $^{87}$Rb that is comparable to the fine-structure splitting, and hence there is an appreciable vector AC Stark shift \cite{Grimm00, LeKien13}. The lattice is generated using a polarization gradient, which is created by rotating the linearly polarized input beam before it is reflected back onto the atoms (see Fig.~\ref{fig:cartoon}) 

The 1D lattice is in a lin-$\vartheta$-lin configuration, where both beams are linearly polarized, and there is an angle $\vartheta$ between them. In general, both the intensity and the polarization gradient contribute to the lattice potential via the scalar and vector AC Stark shift, respectively. At the chosen wavelength of 810 nm, the scalar polarizability is larger than the vector polarizability, and therefore $U_{AB} \approx U$ for small polarization angles.  $U_{AB}$ tends to zero only for near orthogonal polarizations.

The optical field in this configuration can be described by a superposition of two circular standing waves.  
For a polarization angle $\vartheta$, the separation between the two circularly polarized sublattices is
\begin{equation}
\label{eq:interwell-separation}
\Delta x / \lambda = \frac{1}{4\pi} \arccos \left( \frac{\cos^2\vartheta - R^2\sin^2\vartheta}{\cos^2\vartheta + R^2\sin^2\vartheta} \right).
\end{equation}
For constant input intensity the lattice depth experienced by $F = 1, |m_F| = 1$ atoms scales with $\vartheta$ as
\begin{equation}
\label{eq:latdepth-scale}
    V_0 \propto \sqrt{\cos^2\vartheta + R^2\sin^2\vartheta}.
\end{equation}
Here $R\approx 1/8$ is the vector-to-scalar AC polarizability ratio for $^{87}$Rb at our wavelength. Using this, $U_{AB}$ can be calculated through the overlap integral of two Wannier functions which have a relative displacement of $\Delta x$. See Fig.~\ref{fig:experimental-calibration} for the numerical result. The scaling of Eq.~(\ref{eq:latdepth-scale}) was confirmed by measuring the lattice depth at various angles through Kapitza--Dirac diffraction \cite{Gould86}, see the right panel in Fig.~\ref{fig:experimental-calibration}.

To calibrate $U_{AB}$, we loaded a small number of atoms into a lattice without any polarization gradient ($\vartheta = 0^\circ$) avoiding any major population of doubly occupied sites.  We then rotated the lattice angle away from zero and lowered the longitudinal lattice depth to $14~E_R$ while the transverse lattices were held at $35~E_R$. Formation of doublons by tunneling was induced by applying a sinusoidal modulation to the lattice depth of $\pm 10\%$. Modulating at a frequency of $U_{AB}/\hbar$ ($U/\hbar$) induces doublon formation between different (the same) spin states \cite{Sias08, Ma11}.

The formed pairs are detected by selectively inducing losses on doubly-occupied sites. Our method is described in detail in Ref.~\cite{Chung21}; briefly, we rotate the lattice back to $\vartheta = 0^\circ$ and transfer the atoms to a pair of states that has a narrow Feshbach resonance at a magnetic field close to $9~\mathrm{G}$ \cite{kaufman2009radio}. By modulating the field around this value, atoms in doubly-occupied sites undergo a loss process. The doublon fraction is obtained by taking the ratio between shots with and without induced losses;  Fig.~\ref{fig:experimental-calibration} shows the spectroscopic measurement of pairing energies.


\subsection*{Fitting the relaxation behavior}
The phenomenological decay rate function of Eq.~(\ref{eq:lifetime}) is inspired by the measurement presented in Fig.~\ref{fig:loading-spin-mott}, where we observe a high-quality spin-Mott state as long as $D$ is above some threshold. Below that threshold thermal effects start to play a role, and we assume that a similar mechanism affects the lifetime:  As the gap shrinks, the system is no longer protected from tunneling from the outer shells of the Mott insulator, and we expect the lifetime to scale with the tunneling rate.

The solid lines shown in Fig.~\ref{fig:lifetime} are based on an overall fit to the lifetime data to Eq.~(\ref{eq:lifetime}) using $c_1 = 0.086$, $c_2 = 0.025U$, and $\Gamma_0 = 2.0~\mathrm{s^{-1}}$.

\begin{figure}
    \centering
    \includegraphics[width=\columnwidth]{cartoon.pdf}
    \caption{Experimental setup for creating the the spin-dependent lattice. The input beam is reflected through standard wave plates ($\lambda/2$ and $\lambda/4$ signify a half- and quarter-wave plate, respectively) and a liquid-crystal rotator (LCR), which has a tunable retardance. The solid lines denote the slow axes of the polarization components. They rotate the polarization by $\vartheta = 4\varphi + \eta$ in total, where $\eta$ is the retardance of the LCR.}
    \label{fig:cartoon}
\end{figure}

\begin{figure}
    \centering
    \includegraphics[width=\columnwidth]{modulation-spectrum.pdf}
    \caption{Characterization of the lattice and interaction energies. Left: $U$ and $U_{AB}$ are observed as two separate energy scales through lattice modulation spectroscopy. The measurements were done at different lattice angles, and we interpolated the data in between. The solid lines are calculations based on the overlap of Wannier functions using the lattice separation $\Delta x$ of Eq.~(\ref{eq:interwell-separation}). Right: Measurement of lattice depth for $m_F=1$ atoms using Kapitza--Dirac scattering.  The ratio of lattice depths at $\vartheta = 0^\circ$ and $90^\circ$ is roughly 1/8, as predicted using the lattice wavelength and the fine structure of rubidium.}
    \label{fig:experimental-calibration}
\end{figure}

\subsection*{Mean-field phase diagram}
We have extended the mean-field treatment of the single-component Bose--Hubbard Hamiltonian (see e.g.\ Refs.~\cite{Freericks94,Fisher89}) to the two-component case of Eq.~(\ref{eq:two-comp-hamiltonian}). There are two types of particles annihilated (created) by the operators acting on site $i$: $a_i$ ($a_i^\dagger$) and $b_i$ ($b_i^{\dagger}$). Under the assumption that the correlations between sites are small, we can replace the operators by their deviation from the mean field (e.g.\ by setting $a_i = \langle a \rangle + \delta a_i$ and neglecting cross terms such as $\delta a_i \delta a_j$). This results in the mean-field Hamiltonian:

\begin{align}
    H_\mathrm{MF} = \sum_i &\Big[ -zt \Big( \alpha a_i^\dagger + \alpha^* a_i + \beta b_i^\dagger + \beta^* b_i \notag \\
    &- |\alpha|^2 - |\beta|^2 \Big) + \frac{U}{2} \left[ n^a_i(n_i^a - 1) + n^b_i (n_i^b - 1) \right] \notag \\
    &+ U_{AB} n_i^a n_i^b - \mu (n_i^a + n_i^b) \Big]
\end{align}

Here we have defined $\alpha \equiv \langle a \rangle$ and $\beta \equiv \langle b \rangle$, and use $z$ to denote the number of nearest neighbors. The averages $\alpha$ and $\beta$ take on the role of variational parameters. In principle, they are independent, but in our experiment we strive to achieve similar populations in both components, so by symmetry we can assume they are equal. The mean-field Hamiltonian is defined on the single-site Fock basis $|n^a\rangle\otimes|n^b\rangle$, where $|n^p\rangle \in \left\{ |0\rangle, |1\rangle \cdots |n_\mathrm{max}\rangle \right\}$, and we can obtain the overall ground state by minimizing the lowest-energy eigenstate as a function of $\alpha$.

To obtain the phase diagram, we carry out this procedure for a range of parameters $\left(t/U, \mu/U, U_{AB}/U \right)$. The Mott insulating phase is characterized the absence of number fluctuations, and hence $\langle a \rangle = 0$.
There are two limiting cases; if $U_{AB}/U = 1$ there is effectively no difference between the two components, and everything maps onto the single-component system. If $U_{AB}/U = 0$, on the other hand, the two components do not interact at all, and they form two separable Mott insulators. This is the spin-Mott phase. See Fig.~\ref{fig:phase-diagram} for the phase diagram evaluated at different values of $U_{AB}/U$. As this parameter is tuned away from zero, shells with an uneven integer number of particles develop between the spin-Mott shells that have an even number of particles.