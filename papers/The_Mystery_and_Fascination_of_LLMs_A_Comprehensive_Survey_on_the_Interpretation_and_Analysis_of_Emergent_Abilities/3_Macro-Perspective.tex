% \subsection{Model Architecture}
In this section, we will look into the transformer~\cite{DBLP:conf/nips/BrownMRSKDNSSAA20} components from the perspective of: 1) what makes transformers pay attention to their context, 2) how they can fit into different downstream tasks, and 3) under which circumstances, they come into play~\cite{von2023transformers}.

\subsection{Preliminary}
Before diving into the related works about how transformer components contribute to the core ability -- in context learning ability, we briefly give the formal notions of the transformer blocks, especially the Self-Attention Network (\textsl{SAN}). For each block, the input word index $t$ is firstly transformed by an Embedding layer $W_{E}$, then the \textsl{SAN} is essentially to incorporate the information from surrounding tokens via multiple heads, $h\in H$ and from itself. The resulting representation $x_1$ is finally transformed back to the output word index $T(t)$.

\begin{eqnarray}
    T(t)& =& W_{U}x_1,\nonumber  \\
    x_1& =& x_0+ \Sigma_{h\in H}h(x_0)\nonumber  \\
    x_0& =& W_E t
\end{eqnarray}

\citet{elhage2021mathematical} developed a mathematical framework for SAN decomposition and found that the OV circuit defined as $W_U W_E$ controls token embedding flows directly down its own path without intervening with its surrounding tokens' information. In contrast, the QK circuit in $h$-head, $A^{h} = \text{softmax}(t^{T}\cdot W_E^{T}W_{QK}^{h}W_{E}\cdot t)$ provides the attention scores between each query and key token, contributes to the so-called induction head, which only occurs in two-layer SAN, knows how the token was previously used and looks out for similar cases in the provided context, making the model adaptive to downstream tasks.

\subsection{Beyond Copy to Function Approximation}
The findings in \cite{elhage2021mathematical,olsson2022context} are limited to two-layer SAN (without MLPs) and copying mechanisms, although the empirical results in ~\citet{olsson2022context} show that induction heads are capable of more sophisticated tasks, for example, machine translation, and a specific type of abstract pattern matching in a 40-layer model with 13 billion parameters. 

Later, researchers give proof on how the SAN (without MLPs) can approximate linear algorithm~\cite{dai2023can,DBLP:conf/nips/0001TLV22,DBLP:conf/iclr/AkyurekSA0Z23} without gradient descent. In their settings, the ICL is framed as meta-learning and fed with a sequence of tasks $D$ and the ICL learner $\text{TF}_{\theta}$ is actually to find optimal parameter $\theta$ to approximate a certain algorithm $f$ (here, $f$ is linear algorithm): 
\begin{align*}
   D &=  \{x_i,f(x_{i}),...,f(x_{n}\}\\
\text{argmin}_{\theta} &= \sum_{i=1}^{n} L(f(x_{i}),\text{TF}_{\theta}(D))
\end{align*}



\citet{irie2022dual} presented that linear
layers optimized by gradient descent (GD) have a dual
form of linear self-attention network(LSN)-without softmax in attention weight calculation. The following works demonstrated that TF’s ability to implicitly execute gradient descent steps during inference could also be central to ICL, supporting their claims with empirical evidence. \citet{dai2023can} gave the dual form of linear model gradient descent and LSN output on in-context samples.~\citet{DBLP:conf/nips/0001TLV22} empirically demonstrated that TFs can learn basic function classes (e.g., linear functions, MLPs, and decision trees) via input
sample sequences.~\citet{DBLP:conf/iclr/AkyurekSA0Z23} consider the complete TF structures by incorporating the softmax activation, MLP with GeLU and empirically demonstrated that in computational level ICL can approximate various regression problems, such as KNN, One-pass stochastic gradient descent, One-step batch gradient descent and Ridge regression. Based on the prior results,~\citet{von2023transformers} further enables solving nonlinear regression tasks, i.e., sinWave fitting, within transformers by showing its equivalence to learning a linear model on deep representations. ~\citet{Xie2021AnEO,swaminathan2023schema,DBLP:journals/corr/abs-2304-09960} formulated the LLM as a latent variable model and demonstrated its ability of Bayesian inference.



% \subsection{Effects of Hyper-Parameters}
% \paragraph{Model Parameter}\citet{DBLP:journals/corr/abs-2212-04458} find that among factors determining the inductive
% bias of the model, \textsl{state-size} ((such as the hidden state size in a recurrent network) is a more crucial parameter than the model size for the emergence of in-context learning ability. Instead, more research results show that \textsl{model size} is the key factor for emergent ability, i.e., the scaling law~\cite{DBLP:journals/corr/abs-2303-03846,DBLP:journals/corr/abs-2001-08361,DBLP:journals/corr/abs-2203-15556}. For example, PaLM-540B are capable of overriding semantic priors in text classification tasks, while smaller counterparts are unable to do so~\cite{DBLP:journals/corr/abs-2303-03846}. 
% \paragraph{Prompt Parameter} In addition to the parameters in the model itself, the performance in ICL also heavily relies on prompt settings~\cite{DBLP:conf/acl/CaoLHL022,DBLP:conf/acl/LuBM0S22,DBLP:journals/csur/LiuYFJHN23,DBLP:journals/corr/abs-2212-04037,DBLP:conf/acl/SorensenRRSRDKF22}: such as \textit{wording}, \textit{perplexity}, \textit{order}, \textit{label distribution} and the \textit{mutual information} between the prompt and the
% language model’s output. More fine-grained, ~\citet{DBLP:journals/corr/abs-2307-03172} studied the how the position of the information affects the model's performance and came to the conclusion that the performance would be the highest if relevant information occurs at the beginning or end of the input context.  Most of the above studies are based on empirical experiments. In order to avoid the effects from confound, \citet{DBLP:conf/acl/CaoLHL022,DBLP:conf/acl/StolfoJSSS23} study the causal effects of prompt variable to model performances by backdoor criteria. On the other hand, some researchers explained the prompt variation from a cognitive perspective~\cite{DBLP:journals/corr/abs-2206-14576}. They designed the experiments to evaluate LLM's cognitive ability, including decision-making, information search, deliberation, and causal reasoning, by applying small perturbations to the prompts. On the other hand, researchers give theory proof of how the prompt parameters affect the performances based on simplified settings. For example, ~\citet{DBLP:journals/corr/abs-2305-19420,Xie2021AnEO} give theory proof that a larger length of the ICL input sequence, i.e., prompt (with possible demonstrations) length can help decrease the regret bound of ICL. 

\subsection{Strengthen by CoT.}
All the above works investigated the power of TFs from an expressively perspective, i.e., the function approximation. However, they pay little attention to the striking contributions from the carefully-designed prompt~\cite{DBLP:journals/tacl/JiangXAN20,DBLP:journals/csur/LiuYFJHN23,Wang2022TowardsUC}. In particular, the so-called Chain-of-Thought prompting (COT) plays an important role in complex reasoning tasks, e.g., mathematical arithmetic. By adding \textit{let's think step by step} to guide the model to generate intermediate output and derive final striking results based on it~\cite{Wei2022EmergentAO}. Empirically, researchers show the factors in CoT affecting the reasoning performances to better understand its working mechanism, such as relevance to the query, the reasoning order, etc~\cite{Wang2022TowardsUC}.~\citet{DBLP:journals/corr/abs-2307-13339} found that CoT would not increase the magnitude of the saliancy score of the important input tokens, but indeed enhance its robustness under different input perturbations. ~\citet{DBLP:journals/corr/abs-2305-15408} takes the first step towards theoretically answering how the complex reasoning tasks can be solved by CoT-assisted LLMs.

\subsection{Challenges and Future Work}
Research methods focusing on the problem of how the model structure and parameters contribute to the emergent abilities can be roughly divided into two directions: (1) ablating the research subject and other factors, and then empirically observing the differences in task performances. (2) Simplifying the real situations by either removing the obscure model components or evaluating with delicately designed synthetic experiments, in order to theoretically deduct the effects from principle elements to output. The findings of the first group of methods are often limited to specific evaluation cases, e.g., the variation of subjects in mathematics can affect LLM calculation
results~\cite{DBLP:conf/acl/StolfoJSSS23,shi2023large}. The conclusions from the second line of research hold true only in restricted situations, e.g., without noise data, nor regularisations~\cite{von2023transformers}. These issues are very common in other research areas, and what we can do to push the limits forward in this direction can be summarized as (a) proposing advanced TF structures with better computation efficiency based on the observation that the self-attention layer actually plays a similar role in optimization~\cite{zucchet2022beyond}. (b) Designing a systematic evaluation schema to evaluate the LLM's capability, one can refer to the cognitive literature for detailed ability measurements and implications. (c) Extending to more learning problems and incorporating more realistic settings in order to better approximate 
the real solutions. (d) Additionally, very limited studies theoretically illustrate the success of Chain-of-Thought, which is the key factor in understanding the LLM's reasoning ability.

\section{Training Data Matters}
Unlike prior work that explores implicit mechanisms behind ICL, some researchers study ICL via investigating how the pretraining data intrigue the emergent abilities.

% \subsection{Task Diversity}
% Although the previous studies~\cite{dai2023can,DBLP:conf/iclr/AkyurekSA0Z23,von2023transformers} have also shown that transformers can do linear regression in ICL. However, their assumptions are based on unlimited task diversity, that is the pretrained tasks are the same as true evaluation tasks.~\citet{wies2023learnability} show that unseen tasks can be efficiently learned via ICL
% in-context learning, when the pretraining distribution is a mixture of latent tasks. ~\cite{DBLP:journals/corr/abs-2212-04458,DBLP:journals/corr/abs-2306-15063} show the emergence of in-context learning with pretraining task diversity.~\cite{DBLP:journals/corr/abs-2212-04458} studies this issue in a more systematic way, they identify a task diversity threshold for the emergence of ICL by deriving the optimal estimator on linear regression problems. While lots of work~\cite{DBLP:journals/corr/abs-2306-15063} empirically observe improvement of MNIST classification performance with pretraining task diversity increased.~\citet{DBLP:journals/corr/abs-2301-07067} framed the ICL as a Multiple Task Learning(MTL) problem and identified the transfer learning risk on unseen tasks is governed by the task complexity and the number of MTL tasks. 

% \subsection{Pretraining Data Distribution}
% In addition to the abundance of pertaining tasks, some other important properties of the pertaining data also matter to the emergence of ICL. \citet{DBLP:conf/naacl/ShinLAKKKCLPHS22} discovered that ICL heavily relies on the \textsl{domain relevance}, but pretraining on a dataset related to a downstream task does not always reflect a competitive ICL performance. 
% ~\citet{DBLP:conf/nips/ChanSLWSRMH22} identify \textsl{training distribution—burstiness} and \textsl{occurrence of rare classes}—that are necessary for the emergence of ICL, although their simulation experiments are based on image data.~\citet{DBLP:conf/acl/HanSMTCW23} obtained the similar conclusions: (1) the supportive pretraining data do not necessarily have a higher domain overlap to downstream tasks, (2) the supportive
% pretraining data with \textsl{relative lower frequency} and 
% \textsl{long-tail tokens} have larger contributions. And they addressed the importance of \textsl{long-range} pertaining data.~\citet{DBLP:journals/corr/abs-2303-03846} study how ICL is affected by semantic priors and input-label mappings. Their empirical results show large improvements in flipped-labels presented in-context, implying that they are more capable of using in-context information to override prior semantic knowledge. ~\citet{gu-etal-2023-pre} enhance the language models’ in-context
% learning ability by pre-training the model on
% of intrinsic tasks in the contrastive training manner.~\citet{DBLP:journals/corr/abs-2305-19420} provide a generalisation bound given the number of token sequences and the length of each sequence in pretraining.

% Some studies focus on exploring the influences of particular training samples on the given test samples~\cite{Zhang2021CounterfactualMI,DBLP:conf/icml/KandpalDRWR23,Akyrek2022TowardsTF}, a.k.a, fact tracing. The key factor in such research line is to detect the influential training samples first, then exclude the identified samples in the training phase, so-called counterfactual memorization. More specifically, ~\citet{DBLP:conf/icml/KandpalDRWR23} observed the positive correlation between model's memorization ability and frequency of the pretraining samples. This conclusion is somewhat aligned with that the long-tail knowledge is essential to the difficult tasks in ICL mentioned before. 


\subsection{Challenges and Future Work}
The challenges mainly lie in how to select the relevant important training samples given the test samples and how to avoid retraining the model in a more effective way~\cite{}. Influence functions~\cite{hampel1974influence,DBLP:conf/icml/KohL17,DBLP:conf/nips/PruthiLKS20} are among the first methods to do this for neural networks, by estimating the marginal effect of a training example on the loss of a test example. Basically, there are divided into gradient-based and embedding-based method according to  their measurement units. However, these methods become less practical due to the in-feasibility to model parameters and intermediate output. Besides, the retrain methods that aim to calculate the prediction differences are time-costing. Therefore, how the evaluate the importance of particle training samples in a effective way can be a promising direction.

% For instance, \textcolor{red}{Forward-INF : Efficient Data Influence Estimation with Duality-based  Counterfactual Analysis.!!!fail to find citation, it is a workshop paper, waiting for author's response} 

% \input{sections/lin}
\section{Interpreting Emergent Abilities from Macro Perspective}
Studies from the macro perspective centred on \textit{mechanistic interpretability}~\cite{Olah2022MI}. It involves delving into the inner mechanism of emergent abilities through different theoretical conceptual lenses such as linear regression formulation, meta-learning, latent space theory, and Bayesian inference.
% with goal of  reverse-engineering frontier LLMs
\subsection{Mechanistic Interpretability}
With the goal of reverse-engineering components of frontier models into more understandable algorithms, \citet{elhage2021mathematical} developed a mathematical framework for decomposing operations within  transformers~\cite{Vaswani2017AttentionIA}. They initially introduced the concept of ``\textit{induction heads}'' in a two-layer attention-only model to explain the functioning of ICL within transformers with Circuits~\cite{cammarata2020thread}.
They found that one-layer attention-only models perform relatively basic ICL in a crude manner, whereas two-layer models perform very general ICL using very different algorithms.
Specifically, they discovered that one-layer models essentially function as an ensemble of bigram and “skip-trigram” models that can be accessed directly from the model weights without running the entire model. Most attention heads in these models allocate significant capacity to copying mechanisms, resulting in very simple ICL.
In contrast, the two-layer models manifest a significantly powerful mechanism that employs more advanced, qualitative algorithms at inference time, referred to as ``\emph{induction heads}''. 
This allows them to perform ICL in a manner that resembles a computer program executing an algorithm, rather than merely referencing skip-trigrams. 
Building on this foundation, \citet{Olsson2022IncontextLA} later investigated the internal structures responsible for ICL by extending the concept of ``\emph{induction head}''~\cite{elhage2021mathematical}. They implemented  circuits consist of two attention heads: the ``\emph{previous token head}'', which copies information from one token to its successor, and the actual ``\emph{induction head}'', which uses this information to target tokens that precede the current one.
Their study revealed a phase change occurring early in the training of LLMs of various sizes. This phase change involves circuits that perform ``fuzzy'' or ``nearest neighbor" pattern completion in a mechanism similar to the two-layer induction heads. These circuits play a crucial role in implementating most ICL in large models.
One pivotal insight from \cite{Olsson2022IncontextLA} presented six arguments supporting their hypothesis that induction heads may serve as the primary mechanistic source of ICL in a significant portion of LLMs, particularly those based on transformer architectures.

While \citet{elhage2021mathematical} and \citet{Olsson2022IncontextLA} contribute to our understanding of ICL by probing the internal architecture of LLMs, it is important to note that their findings represent initial steps towards the comprehensive reverse-engineering of LLMs. It becomes particularly intricate when dealing with LLMs characterized by complex structures comprising hundreds of layers and spanning billions to trillions of parameters. This complexity introduces significant challenges. %, particularly in the context of addressing safety concerns.
Moreover, a substantial portion of their conclusions relies primarily on empirical correlations, which might be susceptible to confounding from various factors, thereby introducing potential vulnerabilities into their findings.

\subsection{Regression Function Learning}
Several research studies posited that the emergence of LLMs' competence in ICL can be attributed to their intrinsic capability to approximate regression functions for a novel query~$Q$ based on the demonstrations $D$.
\citet{Garg2022WhatCT} first formally defined ICL as a problem of learning functions and explored whether LLMs can be trained from scratch to learn simple and well-defined function classes, such as linear regression functions.
To achieve this, they generated examples $D$ using these functions, and trained models to predict the function value for the corresponding query $Q$.
Their empirical findings revealed that trained Transformers exhibited ICL abilities, as they manifested to ``learn'' previously unseen linear functions from examples, achieving an average error comparable to that of the optimal least squares estimator.
Furthermore, \citet{Garg2022WhatCT} demonstrated that ICL can be applied to more complex function classes, including sparse linear functions, decision trees, and two-layer neural networks, and posited that the capability to learn a function class through ICL is an inherent property of the model $M_\Theta$, irrespective of its training methodology.

Later, \citet{Li2023TransformersAA} extended \citet{Garg2022WhatCT} to interpret ICL from a statistical perspective. They derived generalization bounds for ICL, considering two types of input examples: sequences that are independently and identically distributed (i.i.d.) and trajectories originating from a dynamical system.
They established a multitask generalization rate of $1 / \sqrt{n T}$ for both types of examples, addressing temporal dependencies by associating generalization to algorithmic stability, abstracting ICL as an algorithm learning problem. 
They found that transformers can indeed implement near-optimal algorithms on classical regression problems with both types of input example by ICL.
Furthermore, they provided theoretical proof highlighting that self-attention possesses favourable stability properties, established through a rigorous analysis quantifying the influence of one token over another.

At the same time, \citet{Li2023TheCO} took a further step from the work of~\cite{Garg2022WhatCT} to gain a deeper understanding of the role of the softmax unit within the attention mechanism of LLMs. They sought to mathematically interpret ICL based on the softmax regression formulation represented as $\min _x||\left\langle\exp (A x), \mathbf{1}_n\right\rangle^{-1} \exp (A x)-b||_2$.
Their analysis revealed that the upper bounds of data transformations, induced either by a singular self-attention layer or by the application of gradient descent on an $L_2$ regression loss, align with the softmax regression formulation. 
This suggests a noteworthy similarity between models learned through gradient descent and those learned by Transformers, especially when trained solely on fundamental regression tasks using self-attention.

Conversely, ~\citet{Akyrek2023WhatLA} took a different approach by delving into the process through which ICL learns linear functions, rather than analysing the types of functions that ICL can learn.
Through an examination of the inductive biases and algorithmic attributes inherent in transformer-based ICL, they discerned that ICL can be understood in algorithmic terms, and linear learners within the model may essentially rediscover standard estimation algorithms.
More specifically, \citet{Akyrek2023WhatLA} provided a theoretical proof to support the claim that transformers can implement learning algorithms for linear models using gradient descent and closed-form ridge regression. They also empirically demonstrated that trained ICLs closely align with the predictors derived from gradient descent, ridge regression, and precise least-squares regression. 
They also introduced preliminary findings suggesting that ICL exhibits algorithmic characteristics, with both predictors of learners’ late layers encoding weight vectors and moment matrices in a non-linear manner.

Although these studies have either provided theoretical proofs or showcased empirical evidence interpreting the ICL ability of LLMs as a problem of learning regression functions, their conclusions are limited to simplified model architectures and controlled synthetic experimental settings. These findings may not necessarily apply directly to real-world scenarios.

\subsection{Gradient Descent \& Meta-Optimization}
In the realm of gradient descent, \citet{Dai2023WhyCG} adopted a perspective of viewing LLMs as meta-optimizers and interpreting ICL as a form of implicit fine-tuning. 
They first conducted a qualitative analysis of Transformer attention, representing it in a relaxed linear attention form, and identified a dual relationship between it and gradient descent. 
Through a comparative analysis between ICL and explicit fine-tuning, \citet{Dai2023WhyCG} interpreted ICL as a meta-optimization process. They further provided evidence that the transformer attention head possesses a dual nature similar to gradient descent~\cite{Irie2022TheDF}, where the optimizer produces meta-gradients based on the provided examples for ICL through forward computation.
Concurrently, \citet{Oswald2022TransformersLI} also proposed a connection between the training of Transformers on auto-regressive objectives and gradient-based meta-learning formulations. 
They specifically examined how Transformers define a loss function based on the given examples and, subsequently, the mechanisms by which Transformers assimilate knowledge using the gradients of this loss function. 
Their findings suggest that ICL may manifest as an emergent property, approximating gradient-based few-shot learning within the forward pass of the model.

However, it is worth noting that both of these investigations only focused on ICL within Transformer architectures, without considering other architectural variations or emergent capabilities, such as CoT and instruction following.
In addition, their analyses predominantly rely on a simplifed form of linear attention for qualitative assessment. 
This poses a challenge since the operation of standard Transformer attention, without any approximation, may be intricate. Therefore, there is a need for  more nuanced explorations into this mechanism in future studies.

\subsection{Bayesian Inference}
In their work, \citet{Xie2021AnEO} first provided an interpretation of ICL through the lens of Bayesian inference, proposing that LLMs have the capability to perform implicit Bayesian inference via ICL.
Specifically, they synthesized a small-scale dataset to examine how ICL emerges in LSTM and Transformer models during pretraining on text with extended coherence. 
Their findings revealed that both models are capable of inferring latent concepts to generate coherent subsequent tokens during pretraining. Additionally, these models were shown to perform ICL by identifying a shared latent concept among examples during the inference process. 
Their theoretical analysis confirms that this phenomenon persists even when there is a distribution mismatch between the examples and the data used for pretraining,  particularly in settings where the pretraining distribution is derived from a mixture of Hidden Markov Models (HMMs)~\cite{Baum1966StatisticalIF}. 
Furthermore,~\citet{Xie2021AnEO} observed that the ICL error decreases as the length of each example increases, emphasizing the significance of the inherent information within inputs. This goes beyond mere input-label correlations and highlights the roles of intrinsic input characteristics in facilitating ICL.

Following on,~\citet{Wang2023LargeLM} expanded the investigation of ICL by relaxing the assumptions made by~\citet{Xie2021AnEO} and posited that ICL in LLMs essentially operates as a form of topic modeling that implicitly extracts task-relevant information from examples to aid in inference.
\citet{Wang2023LargeLM} grounded their theoretical analysis in a setting with a finite number of demonstrations, and under a more general language generation process. 
Specifically, they characterized the data generation process using a causal graph with three variables and imposed no constraints on the distribution or quantity of samples. Their empirical and theoretical investigations revealed that ICL can approximate the Bayes optimal predictor when a finite number of samples are chosen based on the latent concept variable.
Moreover, \citet{Wang2023LargeLM} devised an effective practical algorithm for demonstration selection tailored to real-world LLMs. 

At the same time, \citet{Jiang2023ALS} also introduced a novel latent space theory extending the idea of~\citet{Xie2021AnEO} to explain emergent abilities in LLMs.
Instead of focusing on specific data distributions generated by HMMs, they delved into general sparse data distributions and employed LLMs as a universal density approximator for the marginal distribution, allowing them to probe these sparse structures more broadly.
\citet{Jiang2023ALS} demonstrated that ICL, CoT, and instruction-following abilities in LLMs can be ascribed to Bayesian inference operating on the broader sparse joint distribution of languages.
To shed light on the significance of the attention mechanism for ICL from a Bayesian view,~\citet{Zhang2023WhatAH} defined ICL as the task of predicting a response that aligns with a given covariate based on examples derived from a latent variable model. 
They established that ICL implicitly implements the Bayesian Model Averaging (BMA) algorithm, which is approximated by the attention mechanism.
Furthermore, they demonstrated that certain attention mechanisms converge towards the conventional softmax attention as the number of examples goes to infinity. 
These attentions, due to their encoding of BMA within their structure, empower the Transformer model to perform ICL.

Although their conclusions are insightful, there is a room for improvement.
Their findings might be influenced by various factors, such as the formats of the  examples, the nature of tasks, and the choice of evaluation metrics. 
Additionally, many of these studies are based on analyses conducted using small synthetic datasets, potentially restricting their relevance and applicability to real-world scenarios.
